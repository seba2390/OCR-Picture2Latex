%%%%%%%%%%%%%%%%%%%%%%%%%%%%%%%%%%%%%%%%%%%%%
%
% Latex file for:
% 
% 
%%%%%%%%%%%%%%%%%%%%%%%%%%%%%%%%%%%%%%%%%%%%%

%\RequirePackage{fix-cm}
\documentclass[11pt,final]{amsart}       % one column (second format)

\usepackage{epsfig, epsf, graphicx, float, color}
\usepackage{pstricks, psfrag}
\usepackage{amssymb,amsthm}
\usepackage[foot]{amsaddr}
\usepackage{verbatim,enumerate,hyperref}
\usepackage{setspace,mathtools,wrapfig}
\usepackage[numbers,sort&compress]{natbib}
\usepackage{algorithm2e}
\usepackage{steinmetz}
\usepackage{tikz-qtree,tikz-qtree-compat}
\usepackage{tikz,pgf}
\usepackage{arydshln}
\usepackage[margin=1in]{geometry}

\usetikzlibrary{decorations,decorations.markings,decorations.text}
\usetikzlibrary{arrows.meta}
\usetikzlibrary{patterns}


\usetikzlibrary{positioning,patterns}
\usetikzlibrary{calc,fadings,decorations.pathreplacing,arrows,datavisualization.formats.functions,shapes.geometric}




%\smartqed % flush right qed marks, e.g. at end of proof

% Macros

\def\dfrac#1#2{{\displaystyle{#1\over#2}}}
\def\VS{{\vskip 3mm\noindent}}
\def\boxit#1{\vbox{\hrule\hbox{\vrule\kern6pt
          \vbox{\kern6pt#1\kern6pt}\kern6pt\vrule}\hrule}}
\def\refhg{\hangindent=20pt\hangafter=1}
\def\refmark{\par\vskip 2mm\noindent\refhg}
\def\naive{\hbox{naive}}
\def\itemitem{\par\indent \hangindent2\pahttprindent \textindent}
\def\var{\hbox{var}}
\def\cov{\hbox{cov}}
\def\corr{\hbox{corr}}
\def\trace{\hbox{trace}}
\def\refhg{\hangindent=20pt\hangafter=1}
\def\refmark{\par\vskip 2mm\noindent\refhg}
\def\Normal{\hbox{Normal}}
\def\povr{\buildrel p\over\longrightarrow}
\def\ccdot{{\bullet}}
\def\bse{\begin{eqnarray*}}
\def\ese{\end{eqnarray*}}
\def\be{\begin{eqnarray}}
\def\ee{\end{eqnarray}}
\def\bq{\begin{equation}}
\def\eq{\end{equation}}
\def\bse{\begin{eqnarray*}}
\def\ese{\end{eqnarray*}}
\def\pr{\hbox{pr}}
\def\CV{\hbox{CV}}
\def\wh{\widehat}
\def\T{^{\rm T}}
\def\myalpha{{\cal A}}
\def\th{^{th}}

% Color corrections in text
\newcommand{\corb}[1]{\textcolor{blue}{#1}}
\newcommand{\corred}[1]{\textcolor{black}{#1}}
\newcommand{\corblue}[1]{\textcolor{black}{#1}}

%\renewcommand{\baselinestretch}{1} % Change this 1.5 or whatever
%\newcommand{\qed}{\hfill\hfill\vbox{\hrule\hbox{\vrule\squarebox
%   {.667em}\vrule}\hrule}\smallskip}


\newcommand{\bR}{\mathbf{R}}
\newcommand{\bD}{\mathbf{D}}
\newcommand{\bI}{\mathbf{I}}
\newcommand{\bL}{\mathbf{L}}
\newcommand{\bG}{\mathbf{G}}
\newcommand{\bW}{\mathbf{W}}
\newcommand{\bP}{\mathbf{P}}
\newcommand{\bU}{\mathbf{U}}
\newcommand{\bC}{\mathbf{C}}
\newcommand{\bA}{\mathbf{A}}
\newcommand{\bE}{\mathbf{E}}
\newcommand{\bF}{\mathbf{F}}
\newcommand{\bK}{\mathbf{K}}
\newcommand{\bM}{\mathbf{M}}
\newcommand{\bJ}{\mathbf{J}}
\newcommand{\bH}{\mathbf{H}}
\newcommand{\bQ}{\mathbf{Q}}
\newcommand{\bS}{\mathbf{S}}
\newcommand{\bV}{\mathbf{V}}
\newcommand{\bX}{\mathbf{X}}
\newcommand{\bY}{\mathbf{Y}}
\newcommand{\bZ}{\mathbf{Z}}
\newcommand{\bh}{\mathbf{h}}
\newcommand{\bx}{\mathbf{x}}
\newcommand{\by}{\mathbf{y}}
\newcommand{\bv}{\mathbf{v}}
\newcommand{\bz}{\mathbf{z}}
\newcommand{\bs}{\mathbf{s}}
\newcommand{\ba}{\mathbf{a}}
\newcommand{\bb}{\mathbf{b}}
\newcommand{\bo}{\mathbf{o}}
\newcommand{\bc}{\mathbf{c}}
\newcommand{\bd}{\mathbf{d}}
\newcommand{\bbe}{\mathbf{e}}
\newcommand{\bff}{\mathbf{f}}
\newcommand{\bqq}{\mathbf{q}}
\newcommand{\bve}{\mathbf{e}}
\newcommand{\bu}{\mathbf{u}}
\newcommand{\bw}{\mathbf{w}}
\newcommand{\bm}{\mathbf{m}}
\newcommand{\bg}{\mathbf{g}}
\newcommand{\bn}{\mathbf{n}}
\newcommand{\bk}{\mathbf{k}}
\newcommand{\bt}{\mathbf{t}}
\newcommand{\bbf}{\mathbf{f}}
\newcommand{\cS}{\cal{S}}
\newcommand{\bmu}{\boldsymbol{\mu}}
\newcommand{\bxi}{\boldsymbol{\xi}}
\newcommand{\bsigma}{\boldsymbol{\sigma}}
\newcommand{\bgamma}{\boldsymbol{\gamma}}
\newcommand{\btau}{\boldsymbol{\tau}}
\newcommand{\brho}{\boldsymbol{\rho}}
\newcommand{\blambda}{\boldsymbol{\lambda}}
\newcommand{\bdelta}{\boldsymbol{\delta}}
\newcommand{\btheta}{\boldsymbol{\theta}}
%\newcommand{\btheta}{\mathbf{\theta}}
\newcommand{\bvartheta}{\boldsymbol{\vartheta}}
\newcommand{\bpsi}{\boldsymbol{\psi}}
\newcommand{\bphi}{\boldsymbol{\phi}}
\newcommand{\bepsilon}{\boldsymbol{\epsilon}}
\newcommand{\balpha}{\boldsymbol{\alpha}}
\newcommand{\bbeta}{\boldsymbol{\beta}}
\newcommand{\bSigma}{\boldsymbol{\Sigma}}
\newcommand{\bLambda}{\boldsymbol{\Lambda}}
\newcommand{\bOmega}{\boldsymbol{\Omega}}
\newcommand{\br}{\boldsymbol{r}}
\newcommand{\0}{\mathbf{0}}
\newcommand{\1}{\mathbf{1}}
\newcommand{\binfty}{\boldsymbol{\infty}}
\newcommand{\E}{\mbox{E}}


\newcommand{\tc}[2]{\textcolor{#1}{#2}}
\def\scrU{\mathscr{U}}
\newcommand{\PP}{\mathbb{P}}
\newcommand{\idxset}{\Lambda}


% caligraphic names
\newcommand{\mcA}{{\mathcal A}}
\newcommand{\mcB}{{\mathcal B}}
\newcommand{\mcC}{{\mathcal C}}
\newcommand{\mcD}{{\mathcal D}}
\newcommand{\mcE}{{\mathcal E}}
\newcommand{\mcF}{\mathcal{F}}
\newcommand{\mcG}{\mathcal{G}}
\newcommand{\mcI}{{\mathcal I}}
\newcommand{\mcJ}{{\mathcal J}}
\newcommand{\mcK}{{\mathcal K}}
\newcommand{\mcL}{{\mathcal L}}
\newcommand{\mcO}{{\mathcal O}}
\newcommand{\mcN}{{\mathcal N}}
\newcommand{\mcM}{{\mathcal M}}
\newcommand{\mcP}{{\mathcal P}}
\newcommand{\mcQ}{{\mathcal Q}}
\newcommand{\mctQ}{\tilde {\mathcal Q}}
\newcommand{\mcR}{{\mathcal R}}
\newcommand{\mcS}{\mathcal{S}}
\newcommand{\mcT}{{\mathcal T}}
\newcommand{\mcU}{\mathcal{U}}
\newcommand{\mcX}{\mathcal{X}}
\newcommand{\mcY}{\mathcal{Y}}



% boldface
\newcommand{\ii}{\mathbf{i}}
\newcommand{\jj}{\mathbf{j}}
\newcommand{\pp}{\mathbf{p}}
\newcommand{\rr}{\mathbf{r}}
\newcommand{\mm}{\mathbf{m}}
\newcommand{\qq}{\mathbf{q}}
\newcommand{\UU}{\mathbf{U}}
\newcommand{\FF}{\mathbf{F}}
\newcommand{\aalpha}{\boldsymbol{\alpha}}
\newcommand{\rrho}{\boldsymbol{\rho}}
\newcommand{\ttheta}{\boldsymbol{y}}
\newcommand{\oone}{\boldsymbol{1}}

% sets
\newcommand{\bbR}{\mathbb{R}}
\newcommand{\bbX}{\mathbb{X}}
\newcommand{\bbN}{\mathbb{N}}
\newcommand{\bbE}{\mathbb{E}}
\newcommand{\bbF}{\mathbb{F}}
\newcommand{\bbS}{\mathbb{S}}
\newcommand{\bbK}{\mathbb{K}}
\newcommand{\bbP}{\mathbb{P}}
\newcommand{\bbC}{\mathbb{C}}
\newcommand{\bbJ}{\mathbb{J}}
\newcommand{\bbI}{\mathbb{I}}
\newcommand{\Nset}{\mathbb{N}_0}
\newcommand{\cset}{{\mathbb C}}
\newcommand{\rset}{{\mathbb R}}
\newcommand{\nset}{{\mathbb N}}
\newcommand{\bbNset}{{\mathbb N}}
\newcommand{\qset}{{\mathbb Q}}
\newcommand{\pset}{{\mathbb P}}
\newcommand{\Pol}{\mathbb{P}}
\newcommand{\eset}[1]{{\mathbb E} \left[ #1 \right] }


% misc.
\newcommand{\Grad}{\nabla}
\newcommand{\ssy}{\scriptscriptstyle}
\newcommand{\dist}{\operatorname{dist}}
\newcommand{\KL}{Karhunen--\Loeve }
\newcommand{\lv}{w}
\def\scrG{\mathscr{G}}
\newcommand{\Real}{\mathop{\text{\rm Re}}}
\newcommand{\Imag}{\mathop{\text{\rm Im}}}
\newcommand{\bno}{n}

\newcommand{\func}{u}



% New Operators
\DeclareMathOperator*{\esssup}{ess\,sup}
\DeclareMathOperator*{\essinf}{ess\,inf}


\newcommand{\sJ}[1]{
\begin{bmatrix*}[r]
  \bJ^{#1}_R   & -\bJ^{#1}_I  \\
  \bJ^{#1}_{I}  & \bJ^{#1}_R   \\
\end{bmatrix*}
}

\newcommand{\szv}[1]{
\begin{bmatrix}
  \bz^{#1}_{R} \\
  \bz^{#1}_{I} \\
\end{bmatrix}
}

\newcommand{\sfv}[1]{
\begin{bmatrix}
  \bbf^{#1}_{R} \\
  \bbf^{#1}_{I} \\
\end{bmatrix}
}

% Flipped

\newcommand{\sJflip}[1]{
\begin{bmatrix*}[r]
  -\bJ^{#1}_R   & \bJ^{#1}_I  \\
  \bJ^{#1}_{I}  & \bJ^{#1}_R   \\
\end{bmatrix*}
}

\newcommand{\szvflip}[1]{
\begin{bmatrix*}[r]
  -\bz^{#1}_{I} \\
  \bz^{#1}_{R} \\
\end{bmatrix*}
}

\newcommand{\sfvflip}[1]{
\begin{bmatrix*}[r]
  -\bbf^{#1}_{I} \\
  \bbf^{#1}_{I} \\
\end{bmatrix*}
}


\newcommand{\BallTaylor}{
\left(\bx_{0},
  \left[
    \begin{array}{c}
  \bqq \\
  \0
  \end{array}
  \right]
+ t
\left[\begin{array}{c}
  \bv_{R} \\
  \bv_{I}
  \end{array}
  \right]
  \right)
}


\def\sD{\mathcal{D}}
\def\sN{\mathcal{N}}
\def\sC{\mathcal{C}}



\def\R{\Bbb{R}}
\newcommand{\verteq}[0]{\begin{turn}{90} $=$\end{turn}}
%\newcommand{\Pr}{\mbox{Pr}}
% \renewcommand{\baselinestretch}{1.25}


\newcommand{\supess}{\mbox{ess} \operatornamewithlimits{sup}}

\newtheorem{asum}{Assumption}
\newtheorem{cond}{Condition}
\newtheorem{exam}{Example}
\newtheorem{prop}{Proposition}
\newtheorem{corollary}{Corollary}
\newtheorem{definition}{Definition}
\newtheorem{remark}{Remark}
\newtheorem{lemma}{Lemma}
\newtheorem{theorem}{Theorem}



\newcommand{\argmax}{\operatornamewithlimits{argmax}}
\newcommand{\argmin}{\operatornamewithlimits{argmin}}


% Colors
\def\boxit#1{%
  \smash{\color{blue}\fboxrule=1pt\relax\fboxsep=2pt\relax%
  \llap{\rlap{\fbox{\vphantom{0}\makebox[#1]{}}}~}}\ignorespaces
}

\def\gboxit#1{%
  \smash{\color{darkgreen}\fboxrule=1pt\relax\fboxsep=2pt\relax%
  \llap{\rlap{\fbox{\vphantom{0}\makebox[#1]{}}}~}}\ignorespaces
}

\definecolor{darkblue}{rgb}{0,0.08,0.4}
\definecolor{brightblue}{rgb}{0.65,0.85,0.85}
\definecolor{darkred}{rgb}{0.8,0.2, 0.2}
\definecolor{darkgreen}{rgb}{0, 0.6, 0}
\definecolor{blueish}{rgb}{0.1176, 0.5647, 1.0000}

\definecolor{darkorange}{RGB}{255,140,0}

\definecolor{colorone}{rgb}{0.1176,0.5647,1.0000}
\definecolor{colortwo}{rgb}{0.5608,0.7373,0.5608}



\begin{document}



\pgfkeys{/pgf/decoration/.cd,
      distance/.initial=10pt
}  

\pgfdeclaredecoration{add dim}{final}{
\state{final}{% 
\pgfmathsetmacro{\dist}{5pt*\pgfkeysvalueof{/pgf/decoration/distance}/abs(\pgfkeysvalueof{/pgf/decoration/distance})}    
          \pgfpathmoveto{\pgfpoint{0pt}{0pt}}             
          \pgfpathlineto{\pgfpoint{0pt}{2*\dist}}   
          \pgfpathmoveto{\pgfpoint{\pgfdecoratedpathlength}{0pt}} 
          \pgfpathlineto{\pgfpoint{(\pgfdecoratedpathlength}{2*\dist}}     
          \pgfsetarrowsstart{latex}
          \pgfsetarrowsend{latex}
          \pgfpathmoveto{\pgfpoint{0pt}{\dist}}
          \pgfpathlineto{\pgfpoint{\pgfdecoratedpathlength}{\dist}} 
          \pgfusepath{stroke} 
          \pgfpathmoveto{\pgfpoint{0pt}{0pt}}
          \pgfpathlineto{\pgfpoint{\pgfdecoratedpathlength}{0pt}}
}}

\tikzset{dim/.style args={#1,#2}{decoration={add dim,distance=#2},
                decorate,
                postaction={decorate,decoration={text along path,
                                                 raise=#2,
                                                 text align={align=center},
                                                 text={#1}}}}}



\title{High dimensional multilevel kriging: A computational
  mathematics approach}

%\title{High dimensional multilevel kriging: A computational
%  mathematics approach \thanks{This material is based upon work
%   supported by the National Science Foundation under Grant
%    No. 1736392.}}



\author{Julio E. Castrill\'on-Cand\'as ${\dagger}$} 
  \email{jcandas@bu.edu}


 \address{
   ${\ddagger}$ Department of Mathematics and Statistics, 
  Boston University, Boston, MA 
  }
   

%%%%%%%%%%%%%%%%%%%%%%%%%%%%%%%%%%%%%%%%%%%%%%%%%%%%%%%%%%%%%%%%%%%%%%%%




\begin{abstract}
With the advent of massive data sets much of the computational science
and engineering communities have been moving toward data-driven
approaches such as regression and classification. However, they
present a significant challenge due to the increasing size, complexity
and dimensionality of the problems.  In this paper a multilevel
Kriging method that scales well with the number of observations and
dimensions is developed.  A multilevel basis is constructed that is
adapted to a kD-tree partitioning of the observations.  Numerically
unstable covariance matrices with large condition numbers are
transformed into well conditioned multilevel matrices without
compromising accuracy. Moreover, it is shown that the multilevel
prediction \emph{exactly} solves the Best Linear Unbiased Predictor
(BLUP), but is numerically stable.  The multilevel method is tested on
numerically unstable problems of up 25 dimensions. Numerical results
show speedups of up to 42,050 for solving the BLUP problem but to the
same accuracy than the traditional iterative approach.
\end{abstract}



\maketitle

\noindent
    {\it Keywords:} Hierarchical Basis, Machine Learning, High
    Performance Computing, Sparsification of Covariance Matrices, Fast
    Multipole Method

    






%%%%%%%%%%%%%%%%%%%%%%%%%%%%%%%%%%%%%%%%%%%%%%%%%%%%%%%%%%%%%%%%%%%%%%%%


%% Introduction  ------------------------------------------------------
%%

\section{Introduction}





%\item {\bf What is the problem?  Why is it hard?}

Massive data sets arise from many fields, including, but not limited
to commerce, astrophysical sky-surveys, environmental data, and tsunami
warning systems.  With the advent of big data sets much of the
computational science and engineering communities have been moving
toward data-driven approaches to regression and classification. These
approaches are effective since the underlying data is incorporated
into the optimization. However, they present a numerical challenge due
to increasing size, complexity and dimensionality.

%\item {\bf How is it done today, and what are the limits of current
%practice?}

Due to the high dimensionality of the underlying data many modern
machine learning methods, such as classification and regression
algorithms, seek a balance between accuracy and computational
complexity. How efficient this balance is depends on the approach.
Linear methods are fast, but only work well when there is linear
separation of the data.

For non-linear description of the data, kernel approaches have been
effective under certain circumstances.  These methods rely on Tikhonov
regularization of the data to obtain a functional representation,
where it is assumed that the noise model of the phenomena is
known. However, this assumption is not necessarily satisfied in
practice and can lead to significant errors as the algorithm cannot
distinguish between noise and the underlying phenomena.


To incorporate the variability of the noise model a class of machine
learning algorithms based on Bayes method have been developed. In this
approach the noise model is assumed to be known up to a class of
probability distributions and an optimal choice is made that fits the
training data and noise. For example, from the Geo-statistics
community a well known approach to identifying the underlying data and
noise model is known as Kriging \cite{Nielsen2002}.  The noise model
parameters are estimated from the Maximum Likelihood Estimation (MLE)
of the likelihood function.

Kriging methods are effective in separating the underlying phenomena
from the noise model. However, in practice the covariance matrices
tend to be ill-conditioned with increasing number of observations
making Kriging methods numerically fragile. Moreover, most
applications are limited to 2 or 3 dimensions. A brief literature
review can be found in \cite{Castrillon2015}.

Kriging methods from the computational mathematics perspective have
been developed using skeletonization factorizations \cite{Minden2016},
low-rank \cite{nowak2013} and Hierarchical Matrices (HM)
\cite{khoromskij2009,Litvinenko2019,Geoga2020} approaches. These
methods are very promising. In particular, for the HM approaches they
have been shown to be near optimal. They work well for low dimensions.
However, they are still subject to ill-conditioning and usually a
nugget is added to change the model to make it more numerically
stable. Moreover, the data is assumed to have zero mean, which many
times will not be the case.

In \cite{Castrillon2015} a novel algorithm to solve Kriging problems
is proposed. The method is fast and robust. In particular, it can
solve Kriging problems that where not tractable with previous
methods. A nugget is not assumed, nor zero mean data.  However, this
approach is limited to 2 or 3 dimensions and the computational cost
scales very fast with spatial dimension, thus making it impractical
for high dimensional problems.

In this paper we extend the Kriging approach in \cite{Castrillon2015}
using binary trees, which are well suited for high dimensional
problems. Ill-conditioned covariance matrices are transformed to
numerically stable multilevel covariance matrices without compromising
accuracy. In addition, a new distance criterion is developed to build
sparse multilevel covariance matrices.  Furthermore, sharper decay
estimates of the coefficients of the multivariate covariance matrix
are derived based on analytic extensions that are well suited for high
dimensional problems. Much of this theory and notation is borrowed
from uncertainty quantification and high dimensional integration for
partial differential equations
\cite{nobile2008a,Castrillon2016,Griebel2016}.

The Kriging estimation is transformed into a multilevel form based on
the numerically stable multilevel covariance matrix. In practice a
sparse version of the multilevel covariance matrix is used. A distance
dependent method is used to build to a sparse version. Sharp decay
estimates (sub-exponential) of the multilevel covariance matrices are
derived using complex analytic extensions of the covariance function
instead of Taylor series expansions, which are infeasible for
relatively large dimensional problems. The numerical results show that
the estimation is solved to good accuracy for a large number of
observations.

The Kriging prediction step is remapped into an equivalent multilevel
formulation that is numerically stable. It is shown that the solution
to the multilevel prediction form \emph{exactly} solves the Best
Linear Unbiased Prediction (BLUP) problem.  To my knowledge, this is a
feature that is unique to the multilevel approach.  If the covariance
matrix is ill-conditioned, then it is not possible to solve the
problem accurately on a computer with a fixed machine
precision. However, the BLUP solution arises from a constrained
optimization problem. By taking advantage of this fact, the multilevel
approach side steps the inversion of the covariance matrix and
directly searches for the solution in a constrained space giving rise
to a stable multilevel covariance matrix.  Moreover, only one matrix
inversion (iterative approach) of the multilevel covariance matrix is
required in contrast to classical BLUP, including the Generalized
Least Squares (GLS) prediction, that requires $p+1$ matrix inversions
(iterative approach), where $p$ is the number of columns of the design
matrix.  Numerical results show speedups of up to 42,050 for solving
the BLUP problem to at least the same accuracy.

The multilevel Kriging method makes previously impractical missing
data problems feasible.  We are currently applying the Kriging
approach to missing data problems for medical data sets and have shown
up to 5-6 times improved accuracy over traditional state of the art
missing data packages.


In Section \ref{Introduction} the problem formulation is introduced.
In section \ref{MultilevelCovarianceMatrix} the construction of the
multilevel covariance matrix is discussed. In section
\ref{multilevelestimator} the multilevel estimator and predictor are
formulated and numerical computational issues are discussed in section
\ref{numericalcomputation}.  In section \ref{errorestimates} a
mathematical analysis of the decay of the entries of the multilevel
covariance matrix is developed. This section can also be skipped for
the less mathematically inclined reader.  In section
\ref{numericalresults} the multilevel Kriging method is tested on
numerically unstable problems of up to 25 dimensions.  In the Appendix
the Multivariate polynomial interpolation based on complex analytic
extensions is discussed. These results are used for to derive the
decay of the entries of the multilevel covariance matrix.  In section
\ref{multilevelapproach} it is shown how to construct the multilevel
basis based on kd-trees.


%% Problem Setup ------------------------------------------------------
%%
\section{Problem setup}
\label{Introduction}

Consider the following model for a Gaussian random field $Z$:
\begin{equation}
Z(\bx) = \bk(\bx)\T \bbeta+\varepsilon(\bx), \qquad \bx \in \R^d,
\label{Introduction:noisemodel}
\end{equation}
where $d$ is the number of spatial dimensions, $\bk:\R^d \rightarrow
\R^p$ is a functional vector of the spatial location $\bx$,
$\bbeta\in\R^p$ is an unknown vector of coefficients, and
$\varepsilon$ is a stationary mean zero Gaussian random field with
parametric covariance function
$C(\bx,\bx';\btheta)=\cov\{\varepsilon(\bx),\varepsilon(\bx')\}$ with
an unknown vector of positive parameters $\btheta\in\R^d$.

Suppose that we obtain $N$ observations and stack them in the data
vector $\bZ=(Z(\bx_1),\ldots,$ $Z(\bx_N))\T$ from locations $\bbS :=\{
\bx_{1},\dots,\bx_{N}\}$, where the elements in $\bbS$ are restricted
such that the design matrix defined below, $\bX$, has full column
rank.  Furthermore, without loss of generality all the locations in
$\bbS$ are contained in the unit hypercube $[-1,1]^{d}$.  Let
$\bC(\btheta)=\cov(\bZ,\bZ\T)\in \R^{N \times N}$ be the covariance
matrix of $\bZ$ and assume it is positive definite for all
$\btheta\in\R^w$.  Define $\bX=\big( \bk(\bx_1) \ldots$ $
\bk(\bx_N)\big)\T\in \R^{n\times p}$ and assume it is of full rank,
$p$. Since the model \eqref{Introduction:noisemodel} is a Gaussian
random field, then from the samples of $\bbS$ the following vectorial
model is obtained
\begin{equation}
{\bf Z} = \bX \bbeta +{\boldsymbol \varepsilon},
\label{Introduction:vectormodel}
\end{equation}
where $\boldsymbol \varepsilon$ is a Gaussian random vector,
${\boldsymbol \varepsilon} \sim \mcN(\0,\bC(\btheta))$. The aim
now is to:

\begin{enumerate}[i)]
\item {\it Estimate} the unknown vectors $\bbeta$ and $\btheta$;

\item {\it Predict} $Z(\bx_0)$, where $\bx_0$ is a new spatial
  location. These two tasks are particularly computationally
  challenging when the sample size $N$ and number of dimensions $d$
  are large.
\end{enumerate}

There is a very large literature on Gaussian process regression that
deal with this problem. Please see \cite{Castrillon2015} for a brief
literature review.  The unknown vectors $\bbeta$ and $\btheta$ are
estimated with the log-likelihood function
\begin{equation}
  \begin{split}
\ell(\bbeta,\btheta)&=-\frac{n}{2}\log(2\pi)-\frac{1}{2}\log
\det\{\bC(\btheta)\} \\ &
-\frac{1}{2}(\bZ-\bX\bbeta)\T\bC(\btheta)^{-1}
(\bZ-\bX\bbeta),
\end{split}
\label{Introduction:loglikelihood}
\end{equation}
which can be profiled by Generalized Least Squares (GLS) with
\begin{equation}
  \hat \bbeta(\btheta)=\{\bX\T \bC(\btheta)^{-1} \bX\}^{-1}\bX\T
  \bC(\btheta)^{-1}\bZ.
  \label{GLSbeta}
\end{equation}
In general this is not a good choice, since profiling with the Maximum
Likelihood Estimator (MLE) of $\btheta$ is prone to be biased
\cite{Castrillon2015}.

%A solution to this problem is to use restricted maximum likelihood
%(REML) estimation which consists in calculating the log-likelihood of
%$n-p$ linearly independent contrasts, that is, linear combinations of
%observations whose joint distribution does not depend on $\bbeta$,
%from the set $\bY=\{\bI_n-\bX(\bX\T\bX)^{-1}\bX\T\}\bZ$.


For the prediction part, consider the Best Linear Unbiased Predictor
(BLUP) $\hat Z(\bx_0)=\lambda_0+\blambda\T\bZ$ where
$\blambda=(\lambda_1,\ldots,\lambda_n)\T$. The unbiased constraint
implies $\lambda_0=0$ and $\bX\T\blambda=\bk(\bx_0)$.  The
minimization of the mean squared prediction error
E$[\{Z(\bx_0)-\blambda\T\bZ\}^2]$ under the constraint
$\bX\T\blambda=\bk(\bx_0)$ yields
\begin{equation}
\hat Z(\bx_0)=\bk(\bx_0)\T\hat \bbeta+\bc(\btheta)\T
\bC(\btheta)^{-1}(\bZ-\bX\hat \bbeta), \label{KrigBLUP}
\end{equation}
where $\bc(\btheta)=\cov\{\bZ,Z(\bx_0)\}\in \R^{n}$ and $\hat \bbeta$ is
defined in (\ref{GLSbeta}).  

Now, let $\alpha:= (\alpha_{1},\dots,\alpha_{d}) \in \mathbb{Z}{^d}$,
$|\alpha| := \alpha_{1}+\dots+\alpha_{d}$, $\bx : =
[x_1,\dots,x_d]$. For any $w \in \bbN_+$ (where $\mathbb{N}_+ :=
\mathbb{N} \cup \{0\}$) let $\mcQ^d_w$ be the set of Total Degree (TD)
monomials $\{x_1^{\alpha_1} \dots x_d^{\alpha_d}\,\,\,|\,\,\, |\alpha|
\leq w\}$. The typical choice for the matrix $\bX$ is to build it from
the monomials of $\mcQ^d_w$ with cardinality
$p(d,w):=\begin{pmatrix} d + w \\ w \end{pmatrix}$.

The challenge is that the covariance matrix $\bC(\btheta)$ in many
practical cases is ill-conditioned, leading to slow and inaccurate
estimates of $\btheta$. Following the approach in
\cite{Castrillon2015} the data vector $\bZ$ is transformed into
decoupled multilevel description of the model
\eqref{Introduction:noisemodel}.  This multilevel representation leads
to significant computational benefits, including numerical stability,
when computing the Kriging predictor $\hat Z(\bx_0)$ in
(\ref{KrigBLUP}) for large sample size $N$ and high dimensions $d$.
Note, that in this paper we shall refer to the \emph{single level}
approach to solving the Kriging problem by applying the estimation and
prediction steps directly to the data $\bZ$ and covariance matrix
$\bC(\btheta)$.





%% \section{Polynomial Interpolation}
%% \label{Polynomial}

%% \corb{In this section we give some background on polynomial
%%   interpolation in high dimensions. This will be critical to estimate
%%   the decay rates of the entries of the multilevel covariance matrix
%%   for high dimensional problems. Note that for the less mathematically
%%   inclided reader this section can be skipped as it is only used for
%%   estimating the decay of the multilevel covariance matrix.}

%% The decay of the coefficients will directly depend on the analytic
%% properties of the covariance function. The traditional error estimates
%% of polynomial interpolation are based on multi-variate $m^{th}$ order
%% derivatives. However, for many cases, such as the Mat\'{e}rn
%% covariance function, the derivatives are too complex or expensive to
%% manipulate for even a moderate number of dimensions. This motivates
%% the study of polynomial numerical approximations based on complex
%% analytic extensions, which are much better suited for high dimensions.
%% Much of the discussion that follows has it roots in the field of
%% uncertainty quantification and high dimensional interpolation
%% \cite{nobile2008a,Castrillon2016,Griebel2016}
%% for partial differential
%% equations.


%% Consider the problem of approximating a function $v: \Gamma^{d}
%% \rightarrow \R$ on the domain $\Gamma^{d}$.  Without loss of
%% generality let $\Gamma : = [-1, 1]$ and $\Gamma^{d} := \prod_{n =
%%   1}^{d} \Gamma$. Suppose that $\mcG \subset \Gamma^{d}$, then define
%% the following spaces
%% \[
%% \begin{split}
%%   &
%% L^q(\mcG) := \{ v(\by)\, | \, \int_{\mcG} v(\by)^q \text{d}
%% \by < \infty  \}
%% \,\,\,
%% \mbox{and} \\
%% &
%% L^{\infty}(\mcG) := \{ v(\by)\, | \, \sup_{\by \in \mcG} |v(\by)|
%% < \infty  \}.
%% \end{split}
%% \]


%% Suppose that $\mcP_{ q}(\Gamma):=\text{\rm span}\{y^k,\,k=0,\dots,q\}$
%% i.e. the space of polynomials of degree at most $q$. Let $\mcI^{m} :
%% C^{0}(\Gamma) \rightarrow \mcP_{m-1}(\Gamma)$ be the univariate
%% Lagrange interpolant
%% \[
%% \mcI_{m}(v(\by)):=
%% \sum_{k=1}^{m}v(y^{(k)})l_{m,k}(y^{(k)}),
%% \]
%% where $y^{(1)}, \dots, y^{(m)}$ is a set of distinct knots on $\Gamma$
%% and $\{ l_{n,k} \}_{k=0}^{m}$ is a Lagrange basis of the space
%% $\mcP_{m-1}(\Gamma)$. The variable $m \in \Nset$
%% %, where $\Nset_{+} := \Nset \cup 0$,
%% corresponds to the order of approximation of the
%% Lagrange interpolant. However, for the case of the zero order
%% interpolation $m = 0$ corresponds to $\mcI_{0} = 0$.


%% \begin{remark}
%% For high dimensional interpolation the particular set of points
%% $y^{(1)}, \dots, y^{(m)}$ that we will use is the Clenshaw-Curtis
%% abscissas.  This is further discussed in this section. However, for
%% now, we assume that the points are only distinct.
%%   \end{remark}


%% For $m \geq 1$ let
%% \[
%% \Delta_{m}
%% := \mcI_{m}-\mcI_{m-1},
%% \]
%% From the difference operator $\Delta_{m}$ we can readily observe that
%% $\mcI_{m} = \sum_{k=1}^{m} \Delta_{k}$, which is reminiscent of multi
%% resolution wavelet decompositions. The idea is to represent
%% multivariate approximation as a summation of the difference operators.

%% Consider the multi-index tupple $\bm = (m_1,\dots,m_d)$, where $\bm
%% \in \Nset^{d}$, and form the tensor product operator
%% $\mcS_{w,d}: \Gamma \rightarrow \R$ as
%% \begin{equation}
%%   \mcS_{w,d}
%%       [v(\by)]
%%       :
%%       =
%%  \sum_{\bm \in \bbNset^{d}: \sum_{i=1}^{d} m_i - 1  \leq w } \;\;
%%  \bigotimes_{n=1}^{d} {\Delta^{n}_{m_n}}(v(\by)).
%% \label{errorestimates:SG}
%% \end{equation}
%% Note that by ${\Delta^{n}_{m_n}}(v(\by))$ we mean that the difference
%% operator ${\Delta_{m_n}}$ is applied along the $n^{th}$ dimension in
%% $\Gamma$.


%% Let $C^{0}(\Gamma_d; \R) : = \{ v: \Gamma_d \rightarrow \R\,\,$ is
%% continuous on $\Gamma_d$ and $\max_{\by\in \Gamma_d} |v(\by)| < \infty
%% \}$.  From Proposition 1 in \cite{Back2011} it is shown that for any
%% $v \in C^0(\Gamma_d;\R)$, we have $\mcS_{w,d}[v]\in \mcQ^{d}_{w}$.
%% Moreover, $\mcS_{w,d}[v] = v$, for all $v \in \mcQ^{d}_{w}$. The key
%% observation to take away is that the operator $\mcS_{w,d}[v]$ is
%% \textit{exact} in the space of polynomials $\mcQ^{d}_{w}$. This will
%% be useful in connecting the Lagrange interpolant with Chebyshev
%% polynomials.


%% Let $T_k:\Gamma \rightarrow \R$, $k = 0, 1, \dots$, be a Chebyshev
%% polynomial over $\Gamma$, which are defined recursively as follows:
%% $T_0(y) = 1$, $T_1(y) = y$, $\dots$, $T_{k+1}(y) = 2yT_{k}(y) -
%% T_{k-1}(y)$, $\dots$, where $y \in \Gamma$. Chebyshev polynomials are
%% well suited for the approximation of functions with analytic
%% extensions on a complex region bounded by a Bernstein ellipse. They
%% bypassing the need of using derivative information and sharp bounds on
%% the error are readily available. Suppose that $\sigma > 0$ and denote
%% by
%% \[
%% \begin{split}
%%   \mcE_{\sigma} := \Big\{
%%   &z \in \bbC, \sigma \geq
%% \delta \geq 0 ;\,\Real{z} = \frac{e^{\delta} + e^{-\delta}
%% }{2}cos(\theta) \\
%% &\Imag{z} = \frac{e^{\delta} 
%%   - e^{-\delta}}{2}sin(\theta),
%% \theta \in [0,2\pi)
%%   \Big\}
%% \end{split}
%%   \]
%% as the region bounded by a Bernstein ellipse (see Figure
%% \ref{erroranalysis:sparsegrid:polyellipse}).

%% The following theorem is based on complex analytic extensions on
%% $\mcE_{\sigma}$ and provides a control for the Chebyshev polynomial
%% approximation.

%% \begin{theorem}
%% Suppose that for $u:\Gamma \rightarrow \R$ there exists an analytic
%% extension on $\mcE_{\sigma}$. If $|u| \leq M < \infty$ on
%% $\mcE_{\sigma}$ then there exists a sequence of coefficients
%% $|\alpha_k| \leq M / e^{k\sigma}$ such that $u \equiv \alpha_0 +
%% 2\sum_{k = 1}^{\infty} \alpha_{k} T_{k}$ on $\mcE_{\sigma}$. Moreover,
%% if $y \in \Gamma$ then
%% \[
%% %\begin{multline*}
%% %\shoveright{|q(y) - \alpha_0  - 2\sum_{k = 1}^{n} \alpha_{k} T_{k}(y)|
%% %\leq 
%% %\frac{2M}{e^{\sigma} - 1} e^{-n \sigma}.}
%% |q(y) - \alpha_0  - 2\sum_{k = 1}^{n} \alpha_{k} T_{k}(y)|
%% \leq 
%% \frac{2M}{e^{\sigma} - 1} e^{-n \sigma}.
%% %\end{multline*}
%% \]
%% \label{errorestimates:theorem}
%% \end{theorem}
%% \begin{proof}
%% See Theorem 2.25 in \cite{Khoromskij2018}
%% \end{proof}
%% \qed

%% \begin{figure}[htb]%[12]{r}{7cm}%[htp]
%% \begin{center}
%% \begin{tikzpicture}
%%     \begin{scope}[font=\scriptsize]

      
%%       \filldraw[fill=blue!20,
%%       semitransparent] (0,0) ellipse (2 and 1);

%%     \draw [->] (-2.5, 0) -- (2.5, 0) node [below left]  {$\Real $};
%%     \draw [->] (0,-1.5) -- (0,1.5) node [below left] {$\Imag$};
%%     \draw (1,-3pt) -- (1,3pt)   node [above] {$1$};
%%     \draw (-1,-3pt) -- (-1,3pt) node [above] {$-1$};
%%     \end{scope}
    
%%     \node [below right] at (-2.5,1.25) {$\mcE_{\sigma}$};

%%     \node [] at (0.75,1.25) {$\frac{e^{
%%           \sigma} - e^{- \sigma}}{2}$};

    
%%     \node [] at (2.75,0.25) {$\frac{e^{
%%       \sigma} + e^{- \sigma}}{2}$}; 
    
%% \end{tikzpicture}
%% \end{center}
%% \caption{Complex region bounded by the Bernstein ellipse.}
%% \label{erroranalysis:sparsegrid:polyellipse}
%% \end{figure}

%% We can now connect the error due to the Lagrange interpolation with
%% Chebyshev expansions. It is known that if $u \in C(\Gamma,\R)$ then
%% \[
%% \|(I - \mcI_{m})u\|_{L^{\infty}(\Gamma)} \leq
%% (1 + \Lambda_{m})
%% \min_{h \in \mcP_{m-1}} \| u - h \|_{L^{\infty}(\Gamma)},
%% \]
%% where $\Lambda_{m}$ is the Lebesgue constant (See Lemma 7 in
%% \cite{babusk_nobile_temp_10}). Note that $I:C^{d}(\Xi;\R) \rightarrow
%% C^{d}(\Xi;\R)$ refers to the identity operator and the domain $\Xi$ is
%% taken from context. For the previous case $\Xi = \Gamma$.  Bounds on
%% $\Lambda_{m}$ are known in the context of the location of the knots
%% $y^{(1)}, \dots, y^{(m)} \in \Gamma$. In this article we restrict our
%% attention to Clenshaw-Curtis abscissas
%% %\[
%% \[
%% y^{(j)} = -\cos \left( \frac{\pi(j-1)}{m - 1} \right),\,\, j =
%% 1,\dots, m
%% \]
%% %\]
%% and $\Lambda_m$ is bounded by $2\pi^{-1}(\log{(m-1)} + 1) \leq 2m - 1$
%% (see \cite{babusk_nobile_temp_10}).  Since the interpolation operator
%% $\mcI_{m}$ is exact on $\mcP_{m - 1}$, then if $u:\Gamma \rightarrow
%% \R$ has an analytic extension in $\mcE_{\sigma}$ we have from Theorem
%% \ref{errorestimates:theorem} (following a similar approach as in
%% \cite{babusk_nobile_temp_10}) that
%% \[
%% \begin{split}
%% \|(I - \mcI_{m})u\|_{L^{\infty}(\Gamma_n)}
%% &\leq
%% (1 + \Lambda_{m})
%% \frac{2M}{e^{\sigma} - 1} e^{-\sigma (m-1)} \\
%% &\leq 
%% 2 C(M,\sigma) m e^{-\sigma (m-1)},
%% \end{split}
%% \]
%% where $C(M,\sigma_n) := \frac{2M}{(e^{ \sigma} - 1)}$. We then
%% conclude that for all $k = 1,\dots, m$
%% \begin{equation}
%% \begin{split}
%% \| \Delta_{k}(u) \|_{L^{\infty}(\Gamma)} 
%% &=
%% \|
%% \mcI^{m}(u) - \mcI^{m-1}(u)
%% \|_{L^{\infty}(\Gamma)} \\
%% &\leq
%% \|(I - \mcI_{m})u\|_{L^{\infty}(\Gamma)} \\
%% &+
%% \|(I - \mcI_{m-1})u\|_{L^{\infty}(\Gamma)} \\
%% &\leq
%% e^{2\sigma}C(M,\sigma) m e^{-\sigma m}.
%% \end{split}
%% \label{interpolation:eqn1}
%% \end{equation}
%% Let $\mcE_{\sigma,n} \subset \bbC^{d}$ a complex region bounded by a
%% Bernstein ellipse such that the restriction on $\Gamma_{d}$ is along
%% the $n^{th}$ dimension and form the polyellipse $\mcE^{d}_{\sigma}:=
%% \prod_{n=1}^{d} \mcE_{\sigma,n}$.  Suppose that $v:\mcE^{d}_{\sigma}
%% \rightarrow \bbC$ is analytic on $\mcE^{d}_{\sigma}$ and let
%% $\tilde{M}(v) := \max_{\bz \in \mcE^{d}_{\sigma}} |v(\bz)|$.

%% Note we refer to $\mcI^{n}_{m}$ as the Lagrange operator of order $m$
%% along the $n^{th}$ dimension and similarly $\mcP^{n}_{m-1}$ is the
%% space of the span of univariate polynomials up to degree $m-1$ along
%% the $n^{th}$ dimension.  Form the tensor product $\bI^{d}_{m} :=
%% \mcI^{1}_{m} \times \dots \times \mcI^{d}_{m}$, thus $\bI:C(\Gamma,\R)
%% \rightarrow \bbP$ where $\bbP := \mcP^{1}_{m-1} \times \dots \times
%% \mcP^{d}_{m-1}$. From Theorem 2.27 in \cite{Khoromskij2018} we can
%% conclude that for a finite dimension $d$, as $m \rightarrow \infty$
%% then $\bI^{d}_{m}[v] \rightarrow v$.

%% Applying equation \eqref{interpolation:eqn1} to equation
%% \eqref{errorestimates:SG} we have that
%% \begin{equation}
%% \begin{split}
%% & \| (I - \mcS_{w,d})
%%  v(\by)
%%  \|_{L^{\infty}(\Gamma^{d})} \\
%%  &\leq
%%  \left\| \sum_{\bm \in \bbNset^{d}: \sum_{i=1}^{d} m_i - 1 > w } \;\;
%%  \bigotimes_{n=1}^{d} {\Delta^{n}_{m_n}}(v(\by))\right\|_{L^{\infty}(\Gamma^d)} \\
%%  &\leq
%%  \sum_{\bm \in \bbNset^{d}: \sum_{i=1}^{d} m_i - 1 > w } \;\;
%%  \bigotimes_{n=1}^{d} \|{\Delta^{n}_{m_n}}(v(\by))\|_{L^{\infty}(\Gamma^d)}  \\
%%  &\leq
%%  \sum_{\bm \in \bbNset^{d}: \sum_{i=1}^{d} m_i - 1 > w }
%%  e^{2d} C(M,\sigma)^{d} \\
%%  &
%%  \left( \prod_{n=1}^{d} m_n\right) \exp{\left( -\sum_{n=1}^{d}
%%    \sigma m_{n} \right)}.
%% % \\
%% %  &\leq
%% % \sum_{\bk \in \bbNset^{d}_{0}: \sum_{i=1}^{d} k_i > w }
%% % e^{2d} C(M,\sigma)^{d} \left( \prod_{n=1}^{d} (k_n + 1)\right)
%% % \exp{\left( -\sum_{n=1}^{d}
%% %   \sigma (k_{n}+1) \right)}.
%% \end{split}
%% \label{interpolation:eqn2}
%% \end{equation}

%% By applying Theorem 2.10 and Corollary 2.11 in \cite{Griebel2016} if
%% $ w \geq  d$ and $p( d, w) \geq
%% \left(\frac{2  d}{\kappa( d)}\right)^{ d}$, where
%% $\kappa( d) := \sqrt[\leftroot{-2}\uproot{2}  d]{
%%   d!} >  d/e$ (Sterling approximation), then for any $\hat
%% \sigma \in \R_{+}$
%% \begin{equation}
%% \begin{split}
%%  & \sum_{\bk \in \bbNset^{ d}_{0}: \sum_{i=1}^{ d} k_i  >  w }
%%  \exp{\left( -\sum_{n=1}^{ d} \hat \sigma
%%    k_{n} \right)} \\
%%  &\leq
%%  \sum_{\bk \in \bbNset^{d}_{0}: \hat \sigma \sum_{i=1}^{ d} k_i  \geq  w \hat \sigma  }
%%  \exp{\left( -\sum_{n=1}^{ d}
%%    \hat \sigma k_{n} \right)} \\
%%  &\leq
%%  \hat \sigma  d e
%%  \left( \frac{e^{\hat \sigma}}{1 - e^{-\hat \sigma}} \right)^{ d}
%%  \exp \left(-\frac{ d}{e} \hat \sigma  p^{\frac{1}{ d}}
%%  \right) p^{\frac{ d-1}{ d}}.
%% \end{split}
%% \label{interpolation:eqn3}
%% \end{equation}
%% where $\bk \in \bbNset^{d}_{0}$ and $\bk:=(k_1,\dots,k_d)$.






%% Following the same approach as in \cite{Griebel2016} observe that for
%% $0 < \delta < 1$ we can obtain a bounded constant $c_{n,\delta} \leq
%% (e\sigma \delta)^{-1}$ such that $m_n \exp(-\sigma m_n) \leq (e\sigma
%% \delta)^{-1}$ $\exp(-\sigma m_n (1 - \delta))$. Set $\hat \sigma :=
%% \sigma (1 - \delta)$ and by combining equations
%% \eqref{interpolation:eqn2} and \eqref{interpolation:eqn3} we have
%% proven the following result.

%% \begin{lemma} Suppose that $0< \delta < 1$, $\hat
%%   \sigma := \sigma (1 - \delta)$, and $p(d,w) \geq \left(\frac{2
%%     d}{\kappa(d)}\right)^{d}$ then
%%   \[
%%   \begin{split}
%%  &\| (I - \mcS_{w,d})
%%  v(\by)
%%  \|_{L^{\infty}(\Gamma^{d})}\\
%%  \leq &
%%  \frac{C(\tilde M,\sigma)^d e^{d - \sigma(1 - \delta) + 1} \hat \sigma d }
%%  {
%% (\sigma \delta)^{d}}
%%  \left( \frac{e^{\hat \sigma}}{1 - e^{-\hat \sigma}} \right)^{d} \\
%%  &
%%  \exp \left(-\frac{d}{e} \hat \sigma  p^{\frac{1}{d}}
%%  \right) p^{\frac{d-1}{d}}.
%%  \end{split}
%%  \]
%%  \label{interpolation:lemma1}
%% \end{lemma}


%% \begin{remark}
%% The restriction $p(d,w) \geq \left(\frac{2
%%   d}{\kappa(d)}\right)^{d}$ is not strict and can be relaxed such that
%% sub-exponential convergence is still obtained.  We refer the reader to
%% the bound of the Gamma function in Lemma 2.5 (\cite{Griebel2016}) and
%% it's application in the proofs of Theorem 2.10 and Corollary 2.11.
%% \label{interpolation:remark1}
%% \end{remark}



\section{Multilevel approach}
\label{multilevelapproach}

The general approach of this paper and multilevel basis construction
are now presented. We mostly follow the exposition laid out in
\cite{Castrillon2015}. The proof of Proposition
\ref{Multilevelapproach:theo1} is repeated, but clarified with
more details.

Let $\mcP^{p}(\bbS)$ be the span of the columns of the design matrix
$\bX$. Suppose that there exists the orthogonal projections $\bL :
\R^n \rightarrow \mcP^{p}(\bbS)$ and $\bW : \R^n \rightarrow
\mcP^{p}(\bbS)^{\perp}$, where $\mcP^{p}(\bbS)^{\perp}$ is the
orthogonal complement of $\mcP^{p}(\bbS)$.  The operator $\left[
\begin{array}{c}
\bW \\
\bL
\end{array}
\right ]$ is assumed to be orthonormal.


The first step is to filter out the effect of the trend by project the
observation onto the orthogonal subspace.  Let $\bZ_{\bW}: = \bW \bZ$,
thus from equation \eqref{Introduction:vectormodel} it follows that
$\bZ_{\bW} = {\bf W} ({\bX \bbeta}+ {\boldsymbol \varepsilon}) = {\bf
  W{\boldsymbol \varepsilon}}$. Notice that the trend component ${\bX}
\bbeta$ is removed from the data ${\bf Z}$. The new log-likelihood
function for $\bZ_{\bW}$ becomes
\begin{equation}
  \begin{split}
\ell_{\bW}(\btheta)
&=-\frac{n}{2}\log(2\pi)-\frac{1}{2}\log
\det\{\bC_{\bW}(\btheta)\} 
-\frac{1}{2}\bZ_{\bW}\T\bC_{\bW}(\btheta)^{-1}\bZ_{\bW},
\end{split}
\label{Introduction:multilevelloglikelihood}
\end{equation}
where $\bC_{\bW}(\btheta) := \bW \bC(\btheta) \bW \T$ and
$\bZ_{\bW}\sim \mcN_{N-p}(\0,$ $\bC_{\bW}(\btheta))$.  A consequence
of the filtering is that we obtain an unbiased estimator
\cite{Castrillon2015}.

The decoupling of the likelihood function is not the only advantage of
using $\bC_{\bW}(\btheta)$. The following theorem also shows that
$\bC_{\bW}(\btheta)$ is more numerically stable than $\bC(\btheta)$.

\begin{prop} 
\label{Multilevelapproach:theo1}
Let $\kappa(A) \rightarrow \R$ be the condition number of the matrix
$A \in \R^{N \times N}$ then
\[
\kappa(\bC_{\bW}(\btheta)) \leq 
\kappa(\bC(\btheta)).
\]
\end{prop}
\noindent 
\begin{proof}
To see this let $\bv := \bW\T \bw$ for all $\bw \in \R^{N-p}$, which
implies that $\bv \in \mathbb{R}^{n} \backslash
\mcP^{p}(\bbS)$. Moreover, this map is bijective.  Now, $\bv \T
\bC(\btheta) \bv = \bw \T \bC_{\bW}(\btheta) \bw$ for all $\bw \in
\R^{N-p}$. From the orthonormal property we have that for all $\bv \in
\mathbb{R}^{n} \backslash \mcP^{p}(\bbS)$
\[
\begin{split}
\min_{\bv \in \mathbb{R}^{n} \backslash \mcP^{p}(\bbS)} \frac{\bv\T\bC(\btheta)\bv}
{\|\bv\|^2} 
= \min_{\bw \in \mathbb{R}^{N-p} } \frac{\bw\T\bC_{\bW}(\btheta)\bw}{\|\bw\|^2} 
\,\,\,
\mbox{and}
\,\,\,
\max_{\bv \in \mathbb{R}^{n} \backslash \mcP^{p}(\bbS) } \frac{\bv\T\bC(\btheta)\bv}
{\|\bv\|^2}  
= \max_{\bw \in \mathbb{R}^{N-p}} \frac{\bw\T\bC_{\bW}(\btheta)\bw}{\|\bw\|^2}.
\end{split}
\]
Now, it is not hard to see that
\[
\begin{split}
  0
  &<
  \min_{\bv \in \mathbb{R}^{n}} \frac{\bv\T\bC(\btheta)\bv}{ \|\bv\|^{2} } 
\leq 
\min_{\bv \in \mathbb{R}^{n} \backslash \mcP^{p}(\bbS)}
 \frac{\bv\T\bC(\btheta)\bv}{ \|\bv\|^{2} }
\leq \max_{\bv \in \mathbb{R}^{n} \backslash \mcP^{p}(\bbS)} \frac{\bv\T\bC(\btheta)\bv}
{ \|\bv\|^{2} } 
\leq
\max_{\bv \in \mathbb{R}^{n}} \frac{\bv\T\bC(\btheta)\bv}{ \|\bv\|^{2} }.
\end{split}
\]
The result follows from the positive definite property of
$\bC(\btheta)$.
\end{proof}


Proposition \ref{Multilevelapproach:theo1} states that the condition
number of $\bC_{\bW}(\btheta)$ is less or equal to the condition
number of $\bC(\btheta)$. Thus computing the inverse of
$\bC_{\bW}(\btheta)$ (using a direct or iterative method) will
generally be more stable.

In practice, computing the inverse of $\bC_{\bW}(\btheta)$ can be
significantly more stable than $\bC(\btheta)$ depending on the choice
of $\mcQ^d_w$. This has many significant implications as it will now
be possible to solve numerically unstable problems. Furthermore, the
following useful result can be proven.

\begin{corollary}
  \label{Multilevelapproach:cor1}
  Let $[\bC_{\bW}(\btheta)]^q$ be the multilevel
  covariance matrix built from a TD basis with cardinality $q \in
  \bbN$.  Suppose that $p \leq q$, then
  \[
  \kappa([\bC_{\bW}(\btheta)]^p)
  \leq \kappa([\bC_{\bW}(\btheta)]^q).
  \]
\end{corollary}
\begin{proof}
  This follows from the fact that $\mcP^{q}(\bbS) \subset
  \mcP^{p}(\bbS)$ and by applying a similar argument as the proof of
  Proposition \ref{Multilevelapproach:theo1}.
\end{proof}


There are other advantages to the structure of the matrix
$\bC_{\bW}(\btheta)$.  In section \ref{errorestimates} we show that
for a good choice of the $\mcP(\bbS)$ the entries of
$\bC_{\bW}(\btheta)$ decay rapidly, and most of the entries can be
safely eliminated. A level dependent criterion approach is shown in
Section \ref{MultilevelCovarianceMatrix} that indicates which entries
are computed and which ones are not. With this approach a sparse
covariance matrix $\tilde{\bC}_{\bW}$ can be constructed such that it
is close to $\bC_{\bW}$ in a matrix norm sense, even if the
observations are highly correlated with distance.
%From the decay estimates of
%Section \ref{MultilevelCovarianceMatrix} 





\subsection{Binary multilevel basis}
\label{MultilevelREML}

In this section the construction of Multilevel Basis (MB) is shown.
The approach followed in this section is a based on the MB
construction in \cite{Castrillon2013}. The MB can then be used to: (i)
form the multilevel likelihood
\eqref{Introduction:multilevelloglikelihood}; (ii) sparsify the
covariance matrix $\bC_{\bW}(\btheta)$; and (iii) improve the numerical
stability of the covariance matrix $\bC(\btheta)$ in it's multilevel
form. But first, let us establish notations and definitions:
\begin{enumerate}

% \item Given $\mcQ^d_w$ and the locations $\bbS$
%  construct the design matrix $\bX$. Furthermore, form a second set of
%  monomials $\mctQ^{a}_{\Lambda^{m,g}(w)} : =
%  \mcQ_{\Lambda^{m,g}(w+a)} $ for $a = 0,1,\dots,$ i.e.
%  $\mctQ_{\Lambda^{m,g}(w)} \subset
%  \mctQ^{a}_{\Lambda^{m,g}(w)}$. Denote the accuracy parameter $\tilde
%  p \in \bbN$ as the cardinality of
%  $\mctQ^{a}_{\Lambda^{m,g}(w)}$. From the set of monomials
%  $\mctQ_{\Lambda^{m,g}(w)}$, for some user given parameter $a \in
%  \bbN_0$, and the set of observations $\bbS$ generate the design
%  matrix $\tilde \bX^{a}$. Denote also the space $\mcP^{\tilde
%    p}(\bbS)$ as the span of the columns of $\tilde \bX^{a}$.

\item For any index $i,j \in \mathbb{N}_{0}$, $1 \leq i \leq N$, $1
  \leq j \leq N$, let $\bve_{i}[j] = \delta[i-j]$, where
  $\delta[\cdot]$ is the discrete Kronecker delta function.

\item Let $\phi(\bx,\by;\btheta):\R^{d} \times \R^{d} \rightarrow \R$
  be the covariance function and assumed to be a positive definite.
  Let $\bC(\btheta)$ be the covariance matrix that is formed from all
  the interactions between the observation locations $\bbS$
  i.e. $\bC(\btheta) := \{ \phi(\bx_i,\by_j) \}$, where $i,j, =
  1,\dots,N$.  Alternatively we refer to $\phi(r; \btheta)$ as the
  covariance function where $r:\R^d \times \R^d \rightarrow \R$ is a
  function of $\bx$, $\by$ and $\btheta$.
\end{enumerate}

\begin{definition} The Mat\'{e}rn covariance function:
\[
\phi(r;\btheta)=\frac{1}{\Gamma(\nu)2^{\nu-1}} \left(
\sqrt{2\nu}\frac{r}{\rho} \right)^{\nu} K_{\nu} \left(
\sqrt{2\nu}\frac{r}{\rho} \right),
\]
where with a slight abuse of notation $\Gamma$ is the gamma function,
$r \in \R_{+}$, $0 < \nu$, $0 < \rho < \infty$, and $K_{\nu}$ is the
modified Bessel function of the second kind. It is understood from
context when $\Gamma$ is the gamma function.
\end{definition}

\begin{remark} The Mat\'{e}rn covariance function is a good choice for
the random field model. The parameter $\rho$ controls the length
correlation and the parameter $\nu$ changes the shape. For example, if
$\nu = 1/2 + n$, where $n \in \bbN_{+}$, then (see
\cite{abramowitz1964})
\[
\begin{split}
\phi(r;\rho) &= \exp   \bigg(-\frac{\sqrt{2\nu}r}{\rho} \bigg)
\frac{\Gamma(n + 1)}{\Gamma(2n + 1)} 
\sum_{k = 1}^{n} \frac{(n+1)!}{k!(n-k)!}
\bigg(
\frac{ \sqrt{8v} r }{ \rho } 
\bigg)^{n-k}
\end{split}
\]
and $\nu \rightarrow \infty \Rightarrow \phi(r;\btheta) \rightarrow
\exp \bigg(-\frac{r^2}{2\rho^2} \bigg)$. Note that even for a moderate
number of derivatives the number of terms will grow exponentially fast
leading to a very complex expression. This motivates the study of
complex analytical extensions of the covariance function. See Section
\ref{errorestimates} for more details.
%bbb

\label{multilevelapproach:remark1}
\end{remark}
The first step is to decompose the domain $\Gamma^{d}$ into a
multilevel domain decomposition. A good choice is based on the a
kD-tree decomposition of the space $\R^{d}$ \cite{Dasgupta2008}.
Other choices include
Projection (RP) tree \cite{Dasgupta2008}.
%This is a good choice for lower dimensions, however, as the number of
%dimensions $d$ becomes larger a better approach is to use a Random
%Projection (RP) tree \cite{Dasgupta2008}.
First start with the root node and cell $B^{0}_{0}$ at level $0$ that
contains all the observation nodes in $\bbS$. Now, split these nodes
into two children cells $B^{1}_{1}$ and $B^{1}_{2}$ at level $1$
according to the following rule:
\begin{enumerate}

\item Choose a unit vector $v$ in $\R^{d}$ along the axis of
  $\R^{d}$. This choice is the direction that leads to the maximum
  variance of the data in the cell along the direction of $v$.

\item Project all the nodes $\bx \in \bbS$ in the cell onto the unit
  vector $v$.

\item Split the cell with respect to the median
of the projections.

\end{enumerate}

For each cell $B^{1}_{1}$ and $B^{1}_{2}$ repeat the procedure until
there is at most $p$ nodes at the leaf nodes. Thus a binary tree is
obtained, which is of the form $B^{0}_{0}$, $B^{1}_{1}$, $B^{1}_{2}$,
$B^{2}_{3}$, $B^{2}_{4}$, $B^{2}_{5}$, $B^{2}_{6}$, $\dots $, where
$t$ is the maximal depth (level) of the tree.  Now, let $\mcB$ be the
set of all the cells in the tree and $\mcB^{n}$ be the set of all the
cells at level $0 \leq n \leq t$.  In addition, for each cell a unique
node number, current tree depth, threshold level and projection vector
are also assigned. This will be useful for searching the tree.
Algorithms \ref{RPMLB:algorithm1} and \ref{RPMLB:algorithm2-kd}
describe in more detail the construction of the kD-tree MB.
 
%\begin{remark}
%A kD-tree can also be constructed with Algorithms
%\ref{RPMLB:algorithm1} and \ref{RPMLB:algorithm2-kd}.
%\end{remark}

\begin{algorithm}[h]
  \KwIn{ $\bbS$, node, currentdepth, $n_0$} \KwOut{Tree, node}

\Begin{

\eIf {Tree = root}{node $\leftarrow$ 0, currentdepth $\leftarrow$ 0
Tree $\leftarrow$ MakeTree($\bbS$, node,
currentdepth + 1, $n_0$)
}
{

Tree.node = node

Tree.currentdepth = currentdepth - 1

node $\leftarrow$ node + 1

\If {$|\bbS| < n_0$}{return (Leaf)}





(Rule, threshold, $v$) $\leftarrow$ ChooseRule($\bbS$)

(Tree.LeftTree, node) 
$\leftarrow$ MakeTree($\bx \in \bbS$: Rule($\bx$) = True, node,
currentdepth + 1, $n_0$)

(Tree.RightTree, node)
$\leftarrow$ MakeTree($\bx \in \bbS$: Rule($\bx$) = false,  node, currentdepth + 1, $n_0$)

Tree.threshold = threshold\\
Tree.$v$ = $v$
}
}
\caption{MakeTree($\bbS$) function}
\label{RPMLB:algorithm1}
\end{algorithm}


%\begin{algorithm}[h]
%  \KwIn{ $\bbS$}
%  \KwOut{Rule, threshold, v}
%\Begin{
%choose a random unit vector $v$ \\
%Rule(x) := $x \cdot v  \leq$ threshold = median 
%$\{z \cdot v : z \in \bbS \}$
%}
%\caption{ChooseRule($\bbS$) function for RP tree}
%\label{RPMLB:algorithm2}
%\end{algorithm}

\begin{algorithm}[h]
  \KwIn{ $\bbS$}
  \KwOut{Rule, threshold, $v$}
\Begin{
    choose a coordinate direction that has maximal variance of the projection
    of the points in $\bbS$. \\
Rule(x) := $x \cdot v  \leq$ threshold = median
}

\caption{ChooseRule($\bbS$) function for kD-tree}
\label{RPMLB:algorithm2-kd}
\end{algorithm}


Now, suppose there is a one-to-one mapping between the set of unit
vectors $\mcE:=\{\bve_{1},\dots,\bve_{N}\}$, which is denoted as
leaf unit vectors, and the set of locations $\{
\bx_{1},\dots,\bx_{N}\}$, i.e. $\bx_{n} \longleftrightarrow \bve_{n}$
for all $n = 1, \dots, N$. It is clear that the span of the vectors
$\{\bve_{1},\dots,\bve_{N}\}$ is $\bbR^{N}$.  The next step is to
construct a new basis of $\R^{n}$ that is multilevel and orthonormal.

\setlength{\tabcolsep}{16pt}
\begin{figure*}
\begin{center}
  \begin{tabular}{c c}
\begin{tikzpicture}[scale=.65] 
  \begin{scope} 
 [place/.style={circle,draw=blue!50,fill=blue!20,thick,
     inner sep=0pt,minimum size=1.5mm}]

 \draw[step=8,gray,very thin] (0, 0) grid (8, 8);
    \draw (4,0) to (4,8);
    
    \draw (0,5) to (4,5);
    \draw (2.2,5) to (2.2,8);
    \draw (0,2) to (4,2);

    \draw (0,5) to (4,5);

    \draw (4,4.15) to (8,4.15);
    \draw (6.75,4.15) to (6.75,8);
    \draw (6,0) to (6,4.15);


    
  
    \node at (0.5,7.5) [place] {};
    \node at (0.3,6.3) [place] {};
    

    \node at (2.5,5.5) [place] {};
    \node at (3.2,5.2) [place] {};

    \node at (5,6) [place] {};
    \node at (3.8,5.5) [place] {}; %
    \node at (3.8,6) [place] {};   %


    \node at (0.5,3.5) [place] {};
    \node at (1.5,2.5) [place] {};
    \node at (2.3,2.2) [place] {};
    
    \node at (1.3,0.3) [place] {};
    \node at (2.7,0.5) [place] {};
    \node at (2.2,1.4) [place] {};
    \node at (2.6,1.4) [place] {};
    \node at (4.2,3.5) [place] {}; %
    \node at (3.7,3.3) [place] {};


    \node at (6.5,4.3) [place] {}; %
    \node at (7.5,5) [place] {};


    \node at (4.3,2.3) [place] {};
    \node at (5.7,3.5) [place] {};
    \node at (6.2,3.4) [place] {};
    \node at (7.3,2.4) [place] {};


    \node at (7,7) [place] {};
    \node at (6,7.5) [place] {};
    \node at (7.5,7.5) [place] {};


    \node at (6.5,2.0) [place] {};
    \node at (0.5,7.0) [place] {};
    \node at (2.0,7.0) [place] {};
    \node at (5,3.75) [place] {}; %
    \node at (6,7.0) [place] {};
    \node at (7,2.0) [place] {}; %
    %\node at (2.0,4.5) [place] {};
    \node at (7.5,4.3) [place] {}; %

    \node at (7.5,8.5) [] {$B^{0}_0$};
    \node at (0.6,5.5) [] {$B^{3}_{7}$};
    \node at (3,7) [] {$B^{3}_{8}$};
  \end{scope}
\end{tikzpicture} 
&
\begin{tikzpicture}[scale=0.85]
    %\node[anchor=center] at (0, -4.5) {$$};
    %\node[anchor=center] at (0,   10) {$$};
\begin{scope}[xshift=5cm, yshift=4cm,
place/.style={circle,draw=blue!50,fill=blue!20,thick,
      inner sep=0pt,minimum size=1.5mm},
placer/.style={circle,draw=blue!50,
  preaction={fill=darkgreen!60,fill opacity=0.5}, thick,inner
  sep=0pt,minimum size=1.5mm}, ]

  %placer/.style={circle,draw=blue!50,
  %preaction={fill=darkgreen!60,fill opacity=0.5}, thick,inner
  %sep=0pt,minimum size=1.5mm}, ]


%\filldraw[fill={rgb:red,143;green,188;blue,143},semitransparent, 
%      thick] (0, 0) rectangle (16, 16);


  
\Tree [.\node[placer]{$B^{0}_{0}$}; 
             [.\node[placer]{$B^{1}_{1}$};
                    [.\node[placer]{$B^{2}_{3}$}; 
                           [.\node[placer]{$B^{3}_{7}$};]
                           [.\node[placer]{$B^{3}_{8}$};] 
                    ]       
                    [.\node[placer]{$B^{2}_{4}$}; 
                           [.\node[placer]{$B^{3}_{9}$};] 
                           [.\node[placer]{$B^{3}_{10}$};] 
                    ] 
             ]                                        
             [.\node[placer]{$B^{1}_{2}$};
                    [      [.\node[placer]{$B^{2}_{5}$};
                                  [.\node[placer]{$B^{3}_{11}$};] 
                                  [.\node[placer]{$B^{3}_{12}$};] 
                           ]
                           [.\node[placer]{$B^{2}_{6}$}; 
                                  [.\node[placer]{$B^{3}_{13}$};] 
                                  [.\node[placer]{$B^{3}_{14}$};] 
                           ] 
                                          ]]
] 

\end{scope}
\end{tikzpicture}
\end{tabular}
\end{center}
\caption{Multilevel domain decomposition of the observations.}
\label{MLRLE:fig1}
\end{figure*}



\begin{enumerate}[(a)]
\item Start at the maximum level of the random projection tree,
  i.e. $q = t$.
\item For each leaf cell $B^{q}_{k} \in \mcB^{q}$ assume without loss
  of generality that there are $s$ observations nodes $\bbS^{q}_{k}:=\{
  \bx_1, \dots, \bx_s \}$ with associated vectors $C_k^{q} := \{
  \bve_1, \dots, \bve_s \}$.
  %Let $\mcE^q_k := \{\bx_1,\dots,\bx_s\}$
  %and
  Denote $\mcC^{q}_{k}$ as the span of the vectors in $C_k^{q}$.
\begin{enumerate}[i)]

\item Let $\bphi^{q,k} _{j} := \sum_{\bve_i \in C^q_k} c^{q,k} _{i,j}
  \bve_i, \hspace{2mm} j=1, \dots, a;
\hspace{2mm} \bpsi^{q,k}_{j} := \sum_{\bve_i \in C^q_k} d^{q,k}_{i,j}
\bve_i, \hspace{2mm} j=a+1, \dots, s$, where $c^{q,k}_{i,j}$,
$d^{q,k}_{i,j} \in \mathbb{R}$ and for some $a \in \mathbb{N}^{+}$. Note
that $a$ is unknown up to this point, but will be computed from the
data.  It is desired that the new discrete MB vector $\bpsi^{q,k}_{j}$
be orthogonal to $\mcP^{p}(\mathbb{S})$, i.e., for all $g \in \mcP^{
  p}(\mathbb{S})$:
\begin{equation}
\sum_{i=1}^{n} g[i] \bpsi^{q,k}_{j}[i] = 0
\label{hbconstruction:eqn1}
\end{equation}

\item Form the matrix $\mcM^{q,k} := \bX \T \bV^{q,k}$, where
  $\mcM^{q,k} \in \R^{p \times s}$, $\bV^{q,k} \in \R^{N \times s}$,
  and $\bV^{q,k}: = [\bve_1, \dots, \bve_i, \dots,\bve_s ]$ for all $\bve_i
  \in C_k^q$. Now, suppose that the matrix $\mcM^{q,k} $ has rank $a$
  and then perform the Singular Value Decomposition (SVD). Denote by
  $\bU \bD \bV $ the SVD of $\mcM^{q,k} $, where $\bU \in \R^{ p \times
    p}$, $\bD \in \R^{p \times s}$, and $\bV \in \R^{s \times s} $.

  \begin{remark} Note that in practice we only keep track of the
    non-zero elements of the vectors $\bve_1, \dots, \bve_s$. Thus the
    computational cost is reduced significantly. This is taken into
    account in the complexity analysis in Lemma
    \ref{MultilevelREML:lemma1} and \ref{MultilevelREML:lemma2}
  \end{remark}
  
\item Following the same argument as in \cite{Castrillon2015} but
  adapted to the kd-tree decomposition equation
  \eqref{hbconstruction:eqn1} is satisfied with the following choice
\[
  \left[ \begin{array}{ccc|ccc}
      c^{q,k}_{0,1} & \dots &c^{q,k}_{a,1} & d^{q,k}_{a+1,1} & \dots &d^{q,k}_{s,1} \\
      c^{q,k}_{0,2} & \dots &c^{q,k}_{a,2} & d^{q,k}_{a+1,2} & \dots &d^{q,k}_{s,2} \\
      \vdots & \vdots & \vdots & \vdots & \vdots & \vdots   \\
      c^{q,k}_{0,s} & \dots &c^{q,k}_{a,s} & d^{q,k}_{a+1,s} & \dots &d^{q,k}_{s,s}
    \end{array}
\right] := \bV\T.
% \label{eqDefVspT*}
  \]
\noindent For this choice the coefficient $a$ is equal to the number
of non-zero singular values. Thus the columns $a+1$, \dots, $s$ form
an orthonormal basis of the nullspace ${N_0}(\mcM^{q,k} )$. Similarly,
the columns $1,\dots, a$ form an orthonormal basis of $\R^s \backslash
{N_0}(\mcM^{q,k})$. Since the vectors in $C^q_k$ are orthonormal then
$\bphi^{q,k}_{1}, \dots, \bphi^{q,k}_a$, $\bpsi^{q,k}_{a+1}, \dots,$
$\bpsi^{q,k}_s$ form an orthonormal basis of $\mcC^{q}_{k}$.  Moreover
$\bpsi^{q,k}_{a+1}, \dots, \bpsi^{q,k}_s$ satisfy equation
\eqref{hbconstruction:eqn1}, i.e., are orthogonal to
$\mcP^{p}(\mathbb{S})$ and are locally adapted to the locations
contained in the cell $B^{q}_{k}$.

\item Denote by $D_k^{q,k}$ the collection of all the vectors
  $\bpsi^{q,k}_{a+1}, \dots, \bpsi^{q,k}_s$. Notice that the vectors
  $\bphi^{q,k}_{1}, \dots,$ $\bphi^{q,k}_a$, which are denoted with a
  slight abuse of notation as the scaling vectors, are {\it not}
  orthogonal to $\mcP^{p}(\mathbb{S})$. They need to be further
  processed.

\item Let $\mcD^{q}$ be the union of the vectors in $D^{q}_k$ for
  all the cells $B^{q}_k \in \mcB^{q}_{k}$. Denote by
  $W_{q}(\mathbb{S})$ as the span of all the vectors in $\mcD^{q}$.

\end{enumerate}



\item The next step is to go to level $q - 1$. For any two sibling
  cells denote $B^{q}_{\tt{left}}$ and $B^{q}_{\tt{right}}$ at level $q$ denote
  $C^{q-1}_{\tilde k}$ as the collection of the scaling functions from
  both cells, for some index $\tilde k$.


\item Let $q: = q - 1$. If $B^{q}_{k} \in \mcB^{q}$ is a leaf cell
  then repeat steps (b) to (d). However, if $B^{q}_{k} \in \mcB^{q}$
    is not a leaf cell, then repeat steps (b) to (d), but replace the
    leaf unit vectors with the scaling vectors contained in $C^{q}_k$
    with $C^{q-1}_{\tilde k}$.


  \item When $q = -1$ is reached stop.
  %repeat steps (b) to (d), but replace
  %$p$ with $p$, e.g. $\mcP^{p}(\bbS)$ with
  %$\mcP^{p}(\bbS)$. The ML basis vectors will span the space
  %$W_{-1}(\bbS) : =\mcP^{p}(\bbS) \backslash \mcP^{p}(\bbS)$.


\end{enumerate}

When the algorithm stops a series orthogonal subspaces
$V_{0}(\bbS), W_{0}(\mathbb{S}),\dots, W_{t}(\mathbb{S})$ (and their corresponding
basis vectors) are obtained. These subspaces are orthogonal to
$V_{0}(\mathbb{S}) : = span \{ \phi_{1}^{0}, \dots, \phi_{p}^{0}
\}$. Note that the orthonormal basis vectors of $V_{0}(\mathbb{S})$
also span the space $\mcP^{p}(\mathbb{S})$.
\begin{remark}
Following Lemma 2 in \cite{Castrillon2013} it can be shown that
\[
\R^{N} = \mcP^{p}(\mathbb{S}) \oplus
%W_{-1}(\mathbb{S}) \oplus
W_{0}(\mathbb{S}) 
\oplus W_{1}(\mathbb{S})
\oplus \dots \oplus W_{t}(\mathbb{S}),
\]
%where $W_{-1}(\bbS) : =\mcP^{\tilde p}(\bbS) \backslash
%\mcP^{p}(\bbS)$.
Also, it can then be shown that at most $\mcO(Nt)$
computational steps are needed to construct the multilevel basis of
$\R^{N}$.
\end{remark}

From the basis vectors of the subspaces $\mcP^{p}(\mathbb{S})^{\perp}
= \cup_{i=0}^{t} W_{i}(\mathbb{S})$ an orthogonal projection matrix
$\bW:\R^{N} \rightarrow (\mcP^{p}(\mathbb{S}))^{\perp}$ can be built.
The dimensions of $\bW$ is $(N - p) \times N$ since the total number
of orthonormal vectors that span $\mcP^{p}(\mathbb{S})$ is
$p$. Conversely, the total number of orthonormal vectors that span
$\mcP^{p}(\mathbb{S})^{\perp}$ is $N-p$.

Let $\bL$ be a matrix where each row is an orthonormal basis vector of
$\mcP^{p}(\mathbb{S})$. For $i = 0,\dots,t$ let $\bW_i$ be a matrix
where each row is a basis vector of the space $W_i(\mathbb{S})$. The
matrix $\bW \in \mathbb{R}^{(N - p) \times N}$ can now be formed,
where $\bW := \left[ \bW_t\T, \dots, \bW_0\T \right] \T$.

Following a similar approach to Lemma 2.11 in \cite{Castrillon2013} it
can be shown that:
\begin{enumerate}[a)]
\item 
The matrix $\bP := \left[
\begin{array}{c}
\bW \\
\bL
\end{array}
\right ]$ is orthonormal, i.e., $\bP\bP\T= \bI$.

\item Any vector $\bv
\in \R^{n}$ can be written as $\bv = \bL\T\bv_{L} + \bW\T\bv_{\bW}$
where $\bv_{L} \in \R^{p} $ and $\bv_{\bW} \in \R^{N-p}$ are unique.

\end{enumerate}





The following useful lemmas are proved:
\begin{lemma} Assuming that $n_0 < 2p$,
for any level $q=0,\dots,t$ there is at most $p2^{q}$ multilevel
basis vectors.
%For level $q = -1$ there is at most $p -
%\tilde p$ multilevel vectors.
\label{MultilevelREML:lemma1}
\end{lemma}
\begin{proof}
Starting at the finest level $t$, for each cell $B^{t}_k \in \mcB^{t}$
there is at most $p$ multilevel vectors.  Since there is at most
$2^t$ cells then there is at most $2^{t} p$ multilevel vectors.

Now, for each pair of left and right (siblings) cells at level $t$ the
parent cell at level $t-1$ will have at most $2 p$ scaling
functions. Thus at most $p$ multilevel vectors and $p$ scaling
vectors are obtained that are to be used for the next level. Now, the
rest of the cells at level $t$ are leafs and will have at most $p$
multilevel vectors and $p$ scaling vectors that are to be used for
the next level. Since there is at most $2^{t-1}$ cells at level $t-1$,
there is at most $2^{t-1} p$ multilevel vectors. Now, follow an
inductive argument until $q = 0$ and the proof is done.
\end{proof}



\begin{lemma} Assuming that $n_0 < 2p$ for any level $q = 0, \dots, t$
  any multilevel vector $\bpsi^{q}_m$ associated with a cell $B^{q}_k
  \in \mcB^{q}$ has at most $2^{t-q+1} p$ non zero entries.
\label{MultilevelREML:lemma2}
\end{lemma}
\begin{proof} For any leaf cell at the bottom of the tree (level $t$)
  there is at most $2 p$ observations.
  %thus the number of non zero entries of level $q$ multilevel
  %vectors is $2 p$.  Combining the left and right cells, the parent
  cell has at most $4 p$ observations, thus the associated multilevel
  vectors has $4p$ non zero entries. By induction at any level $l$ the
  number of nonzero entries is at most $2^{t-q+1} p$.  Now for any
  leaf cell at any other level $l < t$ the number of nonzero entries
  is at most $2 p$. Following an inductive argument the result is
  obtained.
\end{proof}


From Lemma \ref{MultilevelREML:lemma1} and \ref{MultilevelREML:lemma2}
it can be shown that the matrix $\bW$ contains at most $\mcO(Nt)$
non-zero entries and $\bL$ contains at most $\mcO(Np)$ non-zero
entries. Thus for any vector $\bv \in \R^{n}$ the matrix vector
products $\bW \bv$ and $\bL \bv$ are respectively calculated with at
most $\mcO(Nt)$ and $\mcO(Np)$ computational steps.


%% Multilevel Covariance Matrix --------------------------------------
%%
\section{Multilevel covariance matrix}
\label{MultilevelCovarianceMatrix}

The multilevel covariance matrix $\bC_{\bW}(\btheta)$ and sparse
version $\tilde \bC_{\bW}(\btheta)$ can be now constructed.  Recall from
the discussion in Section \ref{multilevelapproach} that
$\bC_{\bW}(\btheta):=\bW \bC(\btheta) \bW \T$. From the multilevel
basis construct in Section \ref{MultilevelREML} the following
operator is built: $\bW := \left[ \bW_t\T, \dots, \bW_0\T
  \right] \T$. Thus the covariance matrix $\bC(\btheta)$ is
transformed into $\bC_{\bW}(\btheta)$, where each of the blocks
$\bC^{i,j}_{\bW}(\btheta) = \bW_i \bC(\btheta) \bW_j \T$ are formed from
all the interactions of the MB vectors between levels $i$ and $j$, for
all $i,j = 0, \dots, t$. The structure of $\bC_{\bW}(\btheta)$ is shown
in Figure \ref{multilevelcov:fig1}.  Thus for any
$\bpsi^{i}_{\tilde{l}}$ and $\bpsi^{j}_{\tilde{k}}$ vectors there is a
unique entry of $\bC^{i,j}_{\bW}$ of the form
$(\bpsi^{i}_{\tilde{k}})\T \bC(\btheta) \bpsi^{j}_{\tilde{l}}$.

%The blocks $\bC_{\bW}^{i,j}$, where
%$i=-1$ or $j=-1$, correspond to the case where the accuracy term
%$\tilde{p} > p$.


In Section \ref{errorestimates} we show that far field entries of
$\bC_{\bW}(\btheta)$, i.e. $(\bpsi^{i}_{\tilde{k}})\T \bC(\btheta)
\bpsi^{j}_{\tilde{l}}$, decay sub-exponentially with respect to
$p(d,w)$ if there exists an analytic extension of the covariance
function on a well defined domain in $\bC^{d}$. Thus it is not
necessary to compute all the entries. We introduce a distance
criterion approach to produce a sparse matrix $\tilde
\bC_{\bW}(\btheta)$.

\subsection{Sparsification of multilevel covariance matrix}

A sparse version of the covariance matrix $\bC_{\bW}(\btheta)$ can be
built by using a level and distance dependent strategy:

\begin{enumerate}[i)]

\item Given a cell $B^{i}_{k}$ at level $i \geq 0$ identify the
  corresponding tree node value Tree.node and the tree depth
  Tree.currentdepth. Note that the

  Tree.currentdepth and the MB level
  $q$ are the same for $q = 0,\dots,t$.
  %However, for $q = -1$ the MB
  %is associated to the Tree.currentdepth = 0.

\item Let $\bbK \subset \bbS$ be all the observations nodes contained
  in the cell $B^{i}_{k}$.

\item Let $\tau_{i,j} \geq 0$ be the distance parameter given by the
  user corresponding to the level $i,j$ from the block
  $\bC^{i,j}_{\bW}(\btheta)$.

\item Let the Targetdepth be equal to the desired level of the tree.
%In the case that it is $-1$ then the Targetdepth is zero.

\end{enumerate}
    The objective now is to find all the cells at the Targetdepth that
    overlap a hyper rectangle which is extended from $B^{i}_{k}$.  For
    all observations $\bx \in B^{i}_{l}$ along each dimension $k = 1,
    \dots, d$ let $x^{min}_k := \min_{ x_k \in B^i_m} x_k$ and
    $x^{max}_k := \max_{ x_k \in B^i_m} x_k$.  Any cell that
    intersects the interval $[x^{min}_{k} - \tau_{i,j} ,x^{max} +
      \tau_{i,j}]$ is included. This is done by searching the tree
    from the root node. At each traversed node check that all the
    nodes $\bx \in \bbK$ satisfy the following rule: If
\[
\bx \cdot \mbox{Tree}.v + \tau_{i,j} \leq Tree.threshold.
\]
then search down the left tree. If 
\[
\bx \cdot \mbox{Tree}.v - \tau_{i,j} > Tree.threshold.
\]
the search down the right tree. Otherwise search both trees.
%If this is true for all $\bx \in \bbK$ then the search continues down
%the left tree.  If this is false for all $\bx \in \bbK$ then the
%search continues down the Right tree, otherwise both the left and
%right tree are searched.
The full search algorithm is described in Algorithms
\ref{MLCM:algorithm3}, \ref{MLCM:algorithm4}, \ref{MLCM:algorithm5}
and \ref{MLCM:algorithm5a}.


\begin{algorithm}[htp]
  \KwIn{Tree, $\bbK$, Targetdepth, $\tau_{i,j}$}
  \KwOut{Targetnodes}
\Begin{
    Targetnodes $\leftarrow \emptyset $
    Targetnodes $\leftarrow$ 
    LocalSearchTree(Tree, $\bbK$, Targetdepth, $\tau_{i,j}$, Targetnodes);
}
\caption{SearchTree function(Tree, $\bbK$, Targetdepth, $\tau_{i,j}$)}
\label{MLCM:algorithm3}
\end{algorithm}






\begin{algorithm}[htp]
  \KwIn{Tree, $\bbK$, Targetdepth, $\tau_{i,j}$, Targetnodes}
  \KwOut{Targetnodes}
\Begin{

\If {Targetdepth = Tree.currentdepth}{return
Targetnodes = Targetnodes $\cup$ Tree.node}

\If {Tree = leaf}{return}

LeftRule  =  ChooseLeftRule($\bbK$, Tree, $\tau_{i,j}$)\\
RightRule =  ChooseRightRule($\bbK$, Tree, $\tau_{i,j}$)\\



\uIf {LeftRule($\bx$)=true $\forall \bx \in \bbK$}
{Targetnodes $\leftarrow$ LocalSearchTree(Tree.LeftTree, 
  $\bbK$, Targetdepth, $\tau_{i,j}$, Targetnodes)}

\uElseIf{RightRule($\bx$)=true $\forall \bx \in \bbK$}
{Targetnodes $\leftarrow$ LocalSearchTree(Tree.RightTree, 
  $\bbK$, Targetdepth, $\tau_{i,j}$, Targetnodes)}

\Else{Targetnodes $\leftarrow$ LocalSearchTree(Tree.LeftTree, 
$\bbK$, Targetdepth, $\tau_{i,j}$, Targetnodes)\\
Targetnodes $\leftarrow$ LocalSearchTree(Tree.RightTree, 
$\bbK$, Targetdepth, $\tau_{i,j}$, Targetnodes)}

%\eIf {Rule($\bx$)=true $\forall \bx \in \bbK$}
%{Targetnodes $\leftarrow$ LocalSearchTree(Tree.LeftTree, 
%$\bbK$, Targetdepth, $\tau$, Targetnodes)}
%{Targetnodes $\leftarrow$ LocalSearchTree(Tree.LeftTree, 
%$\bbK$, Targetdepth, $\tau$, Targetnodes)
%Targetnodes $\leftarrow$ LocalSearchTree(Tree.RightTree, 
%$\bbK$, Targetdepth, $\tau$, Targetnodes)
%}

}
\caption{LocalSearchTree(Tree, $\bbK$, Targetdepth, $\tau_{i,j}$) function}
\label{MLCM:algorithm4}
\end{algorithm}


\begin{algorithm}[htp]

  \KwIn{ $\bbK$, Tree, $\tau_{i,j}$ }
  \KwOut{Rule}
\Begin{
Rule($\bx$) := $\bx \cdot \mbox{Tree}.v + \tau_{i,j} \leq $
        Tree.threshold
 }
\caption{ChooseLeftRule($\bbK$) function}
\label{MLCM:algorithm5}
\end{algorithm}


\begin{algorithm}[htp]

  \KwIn{ $\bbK$, Tree, $\tau_{i,j}$ }
  \KwOut{Rule}
\Begin{
Rule($\bx$) := $\bx \cdot \mbox{Tree}.v - \tau_{i,j} > $
        Tree.threshold
 }
\caption{ChooseRightRule($\bbK$) function}
\label{MLCM:algorithm5a}
\end{algorithm}

In Figure \ref{multilevelcov:fig2}(b) an example for searching local
neighborhood cells of randomly placed observations in $\R^{2}$ is
shown. The orange nodes correspond to the source cell. By choosing a
suitable value for $\tau_{i,j}$ the blue nodes in the immediate cell
neighborhood are found by using Algorithms \ref{MLCM:algorithm3},
\ref{MLCM:algorithm4}, \ref{MLCM:algorithm5} and
\ref{MLCM:algorithm5a}.

The sparse matrix blocks $\bC^{i,j}_{\bW}(\btheta)$ can be built from
all the cells that are obtained from SearchTree function of Algorithm
\ref{MLCM:algorithm5}. Compute all the entries of
$\bC^{i,j}_{\bW}(\btheta)$ that correspond to the interactions between
any two cells $B^{i}_k \in \mcB^{i}$ and $B^{j}_l \in \mcB^{j}$. In
Algorithm \ref{MLCM:algorithm6}) the construction of the sparse matrix
$\tilde \bC^{i,j}_{\bW}(\btheta)$ is shown.

\begin{remark}
Since the matrix $\tilde \bC_{\bW}(\btheta)$ is symmetric it is only
necessary to compute the blocks $\bC^{i,j}_{\bW}(\btheta)$ for $i = 1,
\dots, t$ and $j = i, \dots t$.
\end{remark}

\begin{algorithm}
  \KwIn{Tree, $i$, $j$, $\tau_{i,j}$, $\mcB^i$, $\mcB^j$,
    $\mcD^i$, $\mcD^i$, $\bC(\btheta)$} \KwOut{$\tilde
    \bC^{i,j}_{\bW}(\btheta)$}
  \Begin{
        Targetnodes $\leftarrow \emptyset$\\
        \For{$B^{i}_{m} \in \mcB^{i}$}
            {$\bbK \leftarrow B^{i}_{m}$\\
            \For{$B^{j}_{q} \leftarrow $
            SearchTree(Tree, $\bbK$, Targetdepth $(i)$, $\tau_{i,j}$, 
            Targetnodes)}
            {
              \For{$\psi^i_k \in D^{i}$}{
                \For{$\psi^j_l \in D^{j}$}{
                  Compute $(\bpsi^{i}_{k})\T \bC(\btheta) \bpsi^{j}_{l}$ 
                  in $\tilde \bC^{i,j}_{\bW}(\btheta)$
                }
              }
            }
            }
    }
\caption{Construction of sparse matrix $\tilde \bC^{i,j}_{\bW}(\btheta)$}
\label{MLCM:algorithm6}
\end{algorithm}

\begin{figure}
\begin{center}
\begin{tikzpicture} 
  \begin{scope}[scale = 0.5]
    [place/.style={circle,draw=blue!50,fill=blue!20,thick,
      inner sep=0pt,minimum size=1.5mm}]
    %\draw[fill=red!5, step=16, thick] (0, 0) grid (16, 16);

    \filldraw[fill={rgb:red,143;green,188;blue,143},semitransparent, 
      thick] (0, 0) rectangle (16, 16);

    % \draw[blue, very thick] (0,0)rectangle (3,2);

    \draw[thin] (8,0) to (8,16);
    \draw[thin] (0,8) to (16,8);
    \draw[thin] (14,0) to (14,16);
    \draw[thin] (0,2) to (16,2);
    \draw[thin] (12,0) to (12,16);
    \draw[thin] (0,4) to (16,4);

    \node at (15,1) [] {${\bf G}_{\bW}$ };
    \node at (10,12) [] {$\bC^{t,t-1}_{\bW}(\btheta)$};
    \node at (4.5,6) [] {$\bC^{t-1,t}_{\bW}(\btheta)$};
    \node at (4.5,1) [] {$\bC^{0,t}_{\bW}(\btheta)$};
    \node at (10,6) [] {$\ddots$};
    \node at (13,1) [] {$\dots$};
    \node at (15,3) [] {$\vdots$};
    \node at (4.5,12) [] {$\bC^{t,t}_{\bW}(\btheta)$};

\end{scope}
\end{tikzpicture}
\end{center}
\caption{Multilevel covariance matrix where ${\bf G}_{\bW}
:=\bC^{0,0}_{\bW}(\btheta)$.}
\label{multilevelcov:fig1}
\end{figure}

\begin{figure*}[ht]
\begin{center}

  %\includegraphics[trim = 410 60 370 50, clip, width=4in,
  %  height=4in]{./figures/neighborhoodpattern.pdf}

 \begin{tikzpicture}[scale=0.59, every node/.style={scale=0.59}]
  \begin{scope} 
    [place/.style={circle,draw=blueish,fill=blueish,
        inner sep=0pt,minimum size=1.5mm},
      placegray/.style={circle,draw=gray!50,fill=gray!20,
        inner sep=0pt,minimum size=1.5mm},
        placenew/.style={circle,draw=darkorange!75,fill=darkorange!75,
      inner sep=0pt,minimum size=1.5mm}]
    \draw[step=8,gray,very thin] (0, 0) grid (8, 8);
    
    \node at (15,3.88) [] {\includegraphics[trim = 14.5cm 2cm 10cm 1.75cm,
        clip=true,
        height=8.43cm]{neighborhoodpattern.pdf}};
    
    \draw (2.2,5) to (2.2,8);
    \draw (0,5) to (2.2,5);
    \draw (6,4.15) to (8,4.15);
    \draw (6,0) to (6,4.15);

    \draw (4,0) to (4,8);
    \draw (0,5) to (4,5);
    
    \draw (0,2) to (4,2);
    \draw (0,5) to (4,5);
    \draw (4,4.15) to (8,4.15);
    \draw (6.75,4.15) to (6.75,8);
        
    \draw[dashed,gray] (6,4.15) to (6,8);
    \draw[dashed,gray] (0,2.3) to (6,2.3);
  
    \node at (0.5,7.5) [placenew] {};
    \node at (0.3,6.3) [placenew] {};
    
    \node at (2.5,5.5) [place] {};
    \node at (3.2,5.2) [place] {};

    \node at (5,6) [place] {};
    \node at (3.8,5.5) [place] {}; %
    \node at (3.8,6) [place] {};   %

    \node at (0.5,3.5) [place] {};
    \node at (1.5,2.5) [place] {};
    \node at (2.3,2.2) [place] {};
    
    \node at (1.3,0.3) [placegray] {};
    \node at (2.7,0.5) [placegray] {};
    \node at (2.2,1.4) [placegray] {};
    \node at (2.6,1.4) [placegray] {};
    \node at (4.2,3.5) [place] {}; %
    \node at (3.7,3.3) [place] {};

    \node at (6.5,4.3) [place] {}; %
    \node at (7.5,5) [placegray] {};

    \node at (4.3,2.3) [place] {};
    \node at (5.7,3.5) [place] {};
    \node at (6.2,3.4) [placegray] {};
    \node at (7.3,2.4) [placegray] {};


    \node at (7,7) [placegray] {};
    \node at (6,7.5) [place] {};
    \node at (7.5,7.5) [placegray] {};


    \node at (6.5,2.0) [placegray] {};
    \node at (0.5,7.0) [placenew] {};
    \node at (2.0,7.0) [placenew] {};
    \node at (5,3.75) [place] {}; %
    \node at (6,7.0) [place] {};
    \node at (7,2.0) [placegray] {}; %
    \node at (7.5,4.3) [placegray] {}; %

    \node at (4,8.75) [] {\Large $\tau_{i,j}$};
    \node at (-1,4.25) [] {\Large $\tau_{i,j}$};

    \draw[dashed,gray] (2,6.3) to (2,8);
    \draw[dashed,gray] (0,6.3) to (2,6.3);
    
    \coordinate (A) at (2,8);
    \coordinate (B) at (6,8);
    \coordinate (C) at (0,6.3);
    \coordinate (D) at (0,2.3);

\draw[dim={,10pt}]  (A) --  (B);
\draw[dim={,-15pt}]  (C) --  (D);

\node at (4,-0.65) {\Large (a)};
\node at (14.5,-0.65) {\Large (b)};
  \end{scope}
\end{tikzpicture} 


\end{center}



\caption{Neighborhood identification from source cell on a random
  kD-tree decomposition of observation locations in $\R^{2}$. (a)
  Cartoon example of axis wise distance criterion $\tau_{i,j}$ using
  Algorithms \ref{MLCM:algorithm3}, \ref{MLCM:algorithm4},
  \ref{MLCM:algorithm5} and \ref{MLCM:algorithm5a}. The orange
  observations knots correspond to the source cell. The blue knots
  correspond to all the target nodes. The gray knots are not included
  in the list of target nodes.  (b) Example of local neighborhood
  contained in the axis wise distance $\tau_{i,j}$.  The orange nodes
  are contained in the source cell. The blue nodes are are contained
  in the local neighborhood cells. The grey dots are all the
  observations that are not part of the source or local neighborhood
  cells.}
\label{multilevelcov:fig2}
\end{figure*}

\subsection{Computational cost of the multilevel
  matrix blocks of $\tilde{\bC}_{\bW}$}


The cost of computing the multilevel blocks $\tilde{\bC}^{i,j}_{\bW}$
will in general be $\mcO(N^2)$. However, for the special case that $d
= 2$ and $d = 3$ it is possible to use a fast summation method such as
the Kernel Independent Fast Multipole Method (KIFMM) by
\cite{ying2004} to compute the blocks more efficiently. To my
knowledge, there exists no equivalent fast summation method in higher
dimensions that works satisfactorily.

This KIFMM algorithm is flexible and efficient for computing the
matrix vector products $\bC(\btheta)\bx$ for a large class of kernel
functions, including the Mat\'{e}rn covariance function.  Given
$\tilde N$ sources and $\tilde M$ targets, experimental results show a
computational cost of about $\mcO(\tilde N + \tilde M)$, $\alpha
\approx 1$ with good accuracy ($\varepsilon_{FMM}$ between $10^{-6}$
to $10^{-8}$) with a slight degrade in the accuracy with increased
source nodes.


\begin{asum} Let $\bA(\btheta) \in \R^{\tilde M \times \tilde N}$ be a kernel
matrix formed from $\tilde N$ source observation nodes and $\tilde M$
target nodes in the space $\R^{d}$.  Suppose that there exists a fast
summation method that computes the matrix-vector products
$\bA(\btheta)\bx$ with $\varepsilon_{FMM}>0$ accuracy in $\mcO((\tilde
N + \tilde M)^{\alpha})$ computations, for some $\alpha \geq 1$ and
any $\bx \in \R^{d}$.
\end{asum}

%Given an octree multilevel tree domain decomposition in $\R^{3}$, as
%shown in \cite{Castrillon2013,Castrillon2015}, the authors described
%how to apply a Kernel Independent Fast Multipole Method (KIFMM) by
%\cite{ying2004} to compute all the blocks
%$\tilde{\bC}^{i,i}_{\bW}(\btheta) \in \R^{\tilde N \times \tilde N}$ for
%$i = 0,\dots,t$ in $\mcO(\tilde N(t+1)^2)$ computational steps to a
%fixed accuracy $\varepsilon_{FMM} > 0$.

For the kD-tree it is not possible to determine a-priori the sparsity
of the blocks $\tilde{\bC}^{i,j}_{\bW}(\btheta)$.
%given the level dependent distance parameter $\tau_{i,j}$.
However, for a given a value $\tau_{i,j} \geq 0$ by running Algorithm
\ref{MLCM:algorithm3} on every cell $B^{i}_k \in \mcB^{i}$, at level
$i$, with the Targetdepth corresponding for level $j$ it is possible
to determine the computational cost of constructing the sparse blocks
$\tilde{\bC}^{i,j}_{\bW}(\btheta)$ under the following assumption. Suppose
that maximum number of cells $B^{j}_k \in \mcB^{j}$ given by Algorithm
\ref{MLCM:algorithm3} is bounded by some $\gamma^{i,j} \in \bbN_+$.



\begin{prop} 
  The cost of computing each block $\tilde{\bC}^{i,j}_{\bW}(\btheta)$
for $i,j = 1,\dots,t$ by using a fast summation method with $1 \leq
\alpha \leq 2$ is bounded by
\[
\mcO(\gamma_{i,j} p 2^{i} (2^{t-j+1} p + 2^{t-i+1} p)^{\alpha} + 2p 2^{t}).
\]
\end{prop} 
\begin{proof} Let us look at the cost of computing all the 
interactions between any two cells $B^{i}_k \in \mcB^{i}$ and
$B^{j}_l \in \mcB^{j}$. Without loss of generality assume that $i
\leq j$. For the cell $B^{l}_k$ there is at most $p$
multilevel vectors and from Lemma \ref{MultilevelREML:lemma2}
$2^{t-i+1} p$ non zero entries. Similarly for $B^{j}_l$.  All
the interactions $(\bpsi^{i}_{\tilde{k}})\T \bC(\btheta)
\bpsi^{j}_{\tilde{l}}$ now have to be computed, where
$\bpsi^{i}_{\tilde{k}} \in B^{i}_k$ and $\bpsi^{j}_{\tilde{l}} \in
B^{j}_l$.

The term $\bC(\btheta) \bpsi^{j}_{\tilde{l}}$ is computed using a FMM
with $2^{t-j+1} p$ sources and $2^{t-i+1} p$ targets at a cost of
$\mcO($ $(2^{t-j+1} p + 2^{t-i+1} \tilde p)^{\alpha})$.  Since there
is at most $p$ multilevel vectors in $B^{i}_k$ and $B^{j}_l$ then the
cost for computing all the interactions $(\bpsi^{i}_{\tilde{k}})\T
\bC(\btheta) \bpsi^{j}_{\tilde{l}}$ is $\mcO(p(2^{t-j+1} p + 2^{t-i+1}
p)^{\alpha} + 2^{t-i+1}p)$.

Now, at any level $i$ there is at most $2^{i}$ cells, thus the result
follows.
\end{proof}



%% Multilevel Estimator and Predictor --------------------------------
%%

\section{Multilevel estimator and predictor}
\label{multilevelestimator}
The multilevel random projection tree can be exploited in such a way
to significantly reduce the computational burden and to further
increase the numerical stability of the estimation and prediction
steps. This is an extension of the multilevel estimator and predictor
formulated in \cite{Castrillon2015} to binary tree in higher
dimensions. The former is based on Oct-tree decompositions, thus
making it unsuitable for higher dimensional problems.

\subsection{Estimator}

The multilevel likelihood function, $l_{\bW}(\theta)$ (see equation
\eqref {Introduction:multilevelloglikelihood} ), has the clear
advantage of being decoupled from the vector $\bbeta$. Furthermore,
the multilevel covariance matrix $\bC_{\bW}(\btheta)$ will be more
numerically stable than $\bC(\btheta)$ thus making it easier to invert
and to compute the determinant.

However, it is not necessary to perform the MLE estimation on the full
covariance matrix $\bC_{\bW}(\btheta)$, instead construct a series of
multilevel likelihood functions $\tilde{\ell}^n_{\bW}(\btheta)$, $n =
0,\dots t$, by applying the partial transform $[\bW_t \T, \dots, \bW_n
  \T ]$ to the data $\bZ$ where
\begin{equation}
  \begin{split}
\tilde{\ell}^n_{\bW}(\btheta)
=-\frac{\tilde{N}}{2}\log(2\pi)-\frac{1}{2}\log
\det\{\tilde{\bC}_{\bW}^n(\btheta)\}
-\frac{1}{2}(\bZ^n_{\bW})\T\tilde{\bC}^n_{\bW}(\btheta)^{-1}\bZ^n_{\bW},
\end{split}
\label{Introduction:multilevelloglikelihoodreduced2}
\end{equation}
where $\bZ^n_{\bW} :=[\bW_t \T, \dots, \bW_i \T ] \T \bZ$,
$\tilde{N}$ is the length of $\bZ^n_{\bW}$,
$\tilde{\bC}^n_{\bW}(\btheta)$ is the $\tilde{N} \times \tilde{N}$
upper-left sub-matrix of $\tilde{\bC}_{\bW}(\btheta)$ and
$\bC^n_{\bW}(\btheta)$ is the $\tilde{N} \times \tilde{N}$ upper-left
sub-matrix of $\bC_{\bW}(\btheta)$.

A consequence of this approach is that the matrices
$\bC^n_{\bW}(\btheta)$, $n = -1, \dots, t$ will be increasingly more
numerically stable, thus easier to solve computationally, as shown in
the following theorem.
\begin{prop} 
  \label{Multilevelapproach:theo2}
Let $\kappa(A) \rightarrow \R$ be the condition number of the matrix
$A \in \R^{N \times N}$ then
\[
\kappa(\bC^{t}_{\bW}(\btheta)) \leq \kappa(\bC^{t-1}_{\bW}(\btheta)) \leq 
\dots
\leq
\kappa(\bC_{\bW}(\btheta)) \leq
\kappa(\bC(\btheta)).
\]
\end{prop} 
\begin{proof} A simple extension of the proof in 
Proposition \ref{Multilevelapproach:theo1}.
\end{proof}



\begin{remark} If $\bC(\btheta)$ is symmetric positive definite 
then for $n = 0, \dots, t$ the matrices $\bC^{n}_{\bW}(\btheta)$ are
symmetric positive definite. The proof is immediate.
\end{remark}

\begin{remark} If the matrix $\tilde \bC^{n}_{\bW}(\btheta)$ is close 
to $\bC^{n}_{\bW}(\btheta)$, for $n = 1,\dots,d$, in some matrix norm
sense, the condition number of $\tilde \bC^{n}_{\bW}(\btheta)$ will be
close to $\bC^{n}_{\bW}(\btheta)$. Full error bounds will be derived in
a future publication.
%In section \ref{errorestimates}, for a class of covariance functions,
%it can be shown than for sufficiently large $\tau$ and/or $w,a \in
%\bbN_{+}$ [with the index set $\Lambda(w)^{m,g}$ or $\tilde
%  \Lambda(w)^{m,g}$] $\tilde \bC^{n}_{\bW}(\btheta)$ will be close to
%$\bC^{n}_{\bW}(\btheta)$. Thus $\tilde{\bC}^n_{\bW}(\btheta)$ will be
%symmetric positive definite.
\end{remark}

\subsection{Predictor}
In this section we show how to construct a multilevel BLUP with a
well conditioned multilevel covariance matrix. It will be shown that
the multilevel predictor is \emph{exact}, i.e. the multilevel
predictor and the solution of the constrained predictor problem
(equations \eqref{GLSbeta} and \eqref{KrigBLUP}) are the same.
However, the multilevel form can be significantly easier problem to
solve numerically.

Consider the following system of equations
\begin{eqnarray}
\left( {{\begin{array}{*{20}c}
 \bC(\btheta) \hfill & \bX \hfill \\
 \bX\T \hfill & \0 \hfill \\
\end{array} }} \right)\left( {{\begin{array}{*{20}c}
 \hat \bgamma \hfill \\
 \hat \bbeta \hfill \\
\end{array} }} \right)=\left( {{\begin{array}{*{20}c}
 \bZ \hfill \\
 \0 \hfill \\
\end{array} }} \right).
\label{Kriging:problem}
\end{eqnarray}
From the argument given in \cite{Nielsen2002} it is not hard to show
that the solution of this problem leads to equation \eqref{GLSbeta}
and $\hat \bgamma(\btheta) = \bC^{-1}(\btheta)(\bZ - \bX \hat
\bbeta(\btheta))$. The BLUP can be evaluated as
\begin{equation}
  \hat Z(\bx_0)
  =\bk(\bx_0)\T\hat \bbeta(\btheta)+\bc(\btheta)\T
  \hat \bgamma(\btheta)
\label{Kriging}
\end{equation}
and the Mean Squared Error (MSE) at the target point $\bx_0$ is given by
%\begin{equation}
\[
1 + 
\tilde{\bu}
\T
(\bX \T 
\bC(\btheta)^{-1} \bX )^{-1}
\tilde{\bu}
-\bc(\btheta)\T\bC^{-1}(\btheta)\bc(\btheta)
\]
%\label{Kriging:MSE}
%\end{equation}
where $\tilde{\bu}\T := (\bX \bC^{-1}(\btheta)
\bc(\btheta) - \bk(\bx_0))$.

From \eqref{Kriging:problem} it is observed that $\bX\T \hat
\bgamma(\btheta) = \0$. This implies that $\hat{\bgamma} \in \R^{n}
\backslash \mcP^{p}(\mathbb{S})$ and can be uniquely rewritten as
$\hat{\bgamma} = \bW\T \bgamma_{\bW}$ for some $\bgamma_{\bW} \in
\R^{N-p}$. Now, rewrite $\bC(\btheta) \hat \bgamma + \bX \hat \bbeta
= \bZ$ as
\begin{equation}
\bC(\btheta) \bW\T \bgamma_{\bW} + \bX \hat \bbeta =
       \bZ.
\label{Kriging:eqn1}
\end{equation}
Now apply the matrix $\bW$ to equation \eqref{Kriging:eqn1} and obtain
$\bW \{\bC(\btheta) \bW\T \bgamma_{\bW} + \bX \hat \bbeta\} = \bW
\bZ.$ Since $\bW \bX = \0$ then
\[
\bC_{\bW}(\btheta)
\bgamma_{\bW} = \bZ_{\bW}.
\]
A simple preconditioner $\bP_{\bW}$ can be formed from the diagonal
entries of the matrix $\bC_{\bW}$ i.e. $\bP_{\bW} = diag(\bC_{\bW})$ leading
to the following system of equations
\[
\bP_{\bW}^{-1}\bC_{\bW}(\btheta)
\bgamma_{\bW} = \bP_{\bW}^{-1} \bZ_{\bW}.
\]
Note that in some cases $\bC_{\bW}(\btheta)$ will have very small
condition numbers. For this case we can set $\bP_{\bW}:= I$, i.e. no
preconditioner.

\begin{theorem}
If the covariance function $\phi:\Gamma_d \times \Gamma_d \rightarrow
\R$ is positive definite, then the matrix $\bP_{\bW}(\btheta)$ is always
symmetric positive definite.
\label{multilevelKriging:lemma2}
\end{theorem}
\begin{proof}
Immediate.
\end{proof}


The vector $\hat \bgamma$ can be obtained by applying the inverse
transform $\bW\T$ i.e.  $\hat \bgamma = \bW\T \bgamma_{\bW}$.  From
\eqref{Kriging:problem} the GLS $\hat \bbeta$ can now be computed as a
least squares, i.e.  $\hat{\bbeta} = (\bX\T \bX)^{-1}\bX \T (\bZ -
\bC(\btheta)\hat{\bgamma})$.


\begin{remark}
  Notice that to solve the GLS estimate $\hat \bbeta$ it is not
  necessary to compute the full GLS of equation \eqref{GLSbeta}, but a
  least squares is all that is required. This is in contrast to the
  GLS estimate of equation \eqref{GLSbeta} where if an iterative
  method is used the covariance matrix $\bC(\btheta)$ has to be
  inverted for each of the columns of $\bX$ i.e. $p$ times.
  \end{remark}


\section{Numerical computation of multilevel estimator and predictor}
\label{numericalcomputation}


\subsection{Estimator: Computation of 
$\log{\det\{ \tilde{\bC}^n_{\bW}\}}$ and $(\bZ^n_{\bW})\T
  (\tilde{\bC}^n_{\bW})^{-1}\bZ^n_{\bW}$}

An approach to computing the determinant of
$\tilde{\bC}^n_{\bW}(\btheta)$ is to apply a sparse Cholesky
factorization technique such that $\bG\bG\T =
\tilde{\bC}^n_{\bW}(\btheta)$, where $\bG$ is a lower triangular
matrix. Notice that the eigenvalues of $\bG$ are located on the
diagonal. This leads to $\log \det \{\tilde{\bC}^n_{\bW}(\btheta)\} =
2 \sum_{i = 1}^{\tilde{N}} \log{\bG_{ii}}$.

The direct application of the sparse Cholesky algorithm can leads to
significant fill-in of the factorization matrix $\bG$. To alleviate
this problem it is typical to use matrix reordering techniques. In
particular, the fill-in are reduced by using the sparse Cholesky
factorization \emph{chol} from the Suite Sparse 4.2.1 package
(\cite{Chen2008,Davis2009,Davis2005,Davis2001,Davis1999}) coupled with
Nested Dissection (NESDIS) function package.

In practice, this approach leads to a significant reduction of
fill-in. To my knowledge a theoretical worse case complexity bounded
exists for $d = 2$ or $d = 3$ dimensions (see \cite{Castrillon2015}).

Two choices for the computation of $(\bZ^n_{\bW}) \T \tilde
\bC^n_{\bW}(\btheta)^{-1}$  $\tilde \bZ^n_{\bW}$ are open to us: i) a
Cholesky factorization of $\tilde{\bC}^n_{\bW}(\btheta)$, or ii) a
Preconditioned Conjugate Gradient (PCG).  The PCG choice requires
significantly less memory and allows more control of the error.
However, the sparse Cholesky factorization of
$\tilde{\bC}^n_{\bW}(\btheta)$ has already been used to compute the
determinant. Thus we can use the same factors to compute $ (\tilde
\bZ_{\bW}^n) \T\tilde{\bC}^n_{\bW}(\btheta)^{-1} \tilde \bZ^n_{\bW}$.
The PCG avenue will be explored in more detail in Section
\ref{comppred}.




%% %%%%%%%%%%%%%%%%%%%%%%%%%%%%%%%%%%%%%%%%%%%%%%%%%%%%%%%%%%%%%%%%%%%%%%%%
%\subsection{Predictor: Computation of $\bgamma_{\bW}$ and $\hat{\bbeta}(\btheta)$}

\subsection{Predictor computation}
\label{comppred}

For the predictor stage a different approach is used. Instead of
inverting the sparse matrix $\tilde \bC_{\bW}(\btheta)$ a
Preconditioned Conjugate Gradient (PCG) method is employed to compute
$\hat \bgamma_{\bW} = \bC_{\bW}(\btheta)^{-1} \bZ_{\bW}$.

Recall that $\bC_{\bW} = \bW \bC(\btheta) \bW \T$, $\hat \bgamma_{\bW} =
\bW \hat \bgamma$ and $\bZ_{\bW} = \bW \bZ$. Thus the matrix vector
products $\bC_{\bW}(\btheta) \bgamma_{\bW}^n$ in the PCG iteration are computed
within three steps:
\[
\bgamma_{\bW}^n \xrightarrow[(1)]{\bW \T \bgamma_{\bW}^n} 
\ba_n \xrightarrow[(2)]{\bC(\btheta) \ba_n}
\bb_n \xrightarrow[(3)]{\bW \bb_n}
\bC_{\bW}(\btheta) \bgamma_{\bW}^n 
\]
where $\bgamma_{\bW}^0$ is the initial guess and $\bgamma_{\bW}^n$ is the $n^{th}$
iteration of the PCG.
\begin{enumerate}[(1)]

\item Transformation from multilevel representation to single
  level. This is done in at most $\mcO(Nt)$ steps.

\item Perform matrix vector product using a summation method. For $d =
  2,3$ a KIFMM is used to compute the matrix vector products with
  $\alpha \approx 1$. For $d > 3$ to my knowledge there is no reliable
  fast summation method.

\item Convert back to multilevel representation.

\end{enumerate}

The matrix-vector products $\bC_{\bW}(\btheta) \bgamma_{\bW}^n$, where
$\bgamma_{\bW}^n \in \R^{N-p}$, are computed in $\mcO(N^{\alpha} + 2
Nt)$ computational steps to a fixed accuracy $\varepsilon_{FMM} > 0$.
Note that $\alpha \geq 1$ is dependent on the efficiency of the fast
summation method. The total computational cost is $\mcO(kN^{\alpha} +
2Nt)$, where $k$ is the number of iterations needed to solve
$\bP^{-1}_{\bW} \bC_{\bW}(\btheta) \bar{\bgamma}_{\bW} (\btheta) =
\bP^{-1}_{\bW} \bar{\bZ}_{\bW}$ to a predetermined accuracy
$\varepsilon_{PCG} > 0$.

\begin{remark}
The introduction of a preconditioner can degrade the accuracy for
computing $\hat \bgamma_{\bW} = \bC_{\bW}(\btheta)^{-1} \bZ_{\bW}$
with the PCG method. The residual accuracy $\varepsilon_{PCG}$ of the
PCG iteration has to be set such that the residual of
the\emph{unpreconditioned} system $\|\bC_{\bW}(\btheta) \bgamma_{\bW}
(\btheta) - \bZ_{\bW}\|_{l^2} < \varepsilon$ for a user given
tolerance $\varepsilon > 0$.
\end{remark}

Now compute $\hat \bgamma = \bW\T \hat \bgamma_{\bW}$ and $\hat{\bbeta} =
(\bX\T \bX)^{-1}\bX \T (\bZ - \bC(\btheta)\hat{\bgamma})$ in at most
$\mcO(N^{\alpha} + Np + p^{3})$ computational steps. The matrix vector
product $\bc(\btheta)\T \hat{\bgamma}(\btheta)$ is computed in
$\mcO(N)$ steps.  Finally, the total cost for computing the estimate
$\hat{\bZ}(\bx_0)$ from \eqref{Kriging} is $\mcO(p^{3} + (k +
1)N^{\alpha} + 2Nt)$.

%% Error Estimates ----------------------------------------------------
%%
\section{Multilevel covariance matrix decay}
\label{errorestimates}

We derive decay estimates of the multilevel covariance matrix. It can
be shown that most of the coefficients are small and thus it is not
necessary to compute all of them. The final objective is to build a
posteriori error estimates for $\bx_{\bW} =
\bC^{n}_{\bW}(\btheta)^{-1}\bZ^n_{\bW}$ and $\log
\bC^{n}_{\bW}(\btheta)$ that are needed for solving the multilevel
estimator MLE. However, the full analysis is extensive and will be
completed in a future publication. As a first step we show the decay
of the multievel covariance matrix. Note that this is not trivial and
uses the results derived in the Appendix. We recommend to first read
the appendix since part of the notation used in this section is
defined there.


%Consider the full solution $\bx_{\bW} =
%\bC^{n}_{\bW}(\btheta)^{-1}\bZ^n_{\bW}$ and sparse solution $\tilde
%\bx_{\bW} = \tilde \bC^{n}_{\bW}(\btheta)^{-1}\bZ^n_{\bW}$ for $n = 0,
%\dots, t$, then the error can be bounded as
%\begin{equation}
%\|\bx_{\bW} - \tilde \bx_{\bW}\|_{l^2} 
%\leq 
%\| \bC^{-1}_{\bW}(\btheta) - \tilde \bC^{-1}_{\bW}(\btheta) \|_2 
%\|\bZ_{\bW}\|_{l^2}.
%\label{errorestimates:eqn1}
%\end{equation}
%The ultimate goal is to derive a full a posteriori error estimate of
%$\|\bx_{\bW} - \tilde \bx_{\bW}\|_{l^2}$ with respect to the distance
%criterion $\tau_{i,j}$.  In this section estimates of the decay of the
%multilevel matrix $\bC_{\bW}(\btheta)$ are obtained. These will be
%critical to derive a full a posteriori error scheme.




The decay of the coefficients of the matrix $\bC_{\bW}(\btheta)$ will
depend directly on the choice of the multivariate index set
$\mcQ^{d}_{w}$ and the analytic regularity of the covariance
function. In general, the Mat\'{e}rn covariance function will be
analytic except for a derivative discontinuity at the
origin. However, with the application of the distance criterion a
minimal distance can be guaranteed and the origin will be avoided all
together.

In the following theorem, without loss of generality, it is assumed
that the covariance function domain is $\Gamma^{d} \times \Gamma^{d}$
for any two cells $B^{i}_{m} \in \mcB^{i}$ and $B^{j}_{q} \in
\mcB^{j}$. We will show later that this can be achieved through a
linear pullback.

\begin{theorem} Suppose that $0< \delta < 1$, $\hat
  \sigma := \sigma (1 - \delta)$, and $\phi(\bx,\by;\btheta) \in
  C^{0}(\Gamma^{d} \times \Gamma^{d};\R)$ can be analytically extended
  on $\mcE^{d}_{\sigma} \times \mcE^{d}_{\sigma}$ and is bounded by
  $\tilde M(\phi)$. Let $\mcP^{ p}(\mathbb{S})^{\perp}$ be the
  subspace in $\R^{N}$ generated by the index set $\mcQ^{d}_{w}$ for
  some $w \in \bbN_{+}$. For $i,j = 0,\dots,t$ consider any
  multilevel vector $\bpsi^{i}_m \in \mcP^{p}(\mathbb{S})^{\perp}$,
  with $n_m$ non-zero entries, from the cell $B^{i}_{m} \in \mcB^{i}$
  and any multilevel vector $\bpsi^{j}_{q} \in \mcP^{
    p}(\mathbb{S})^{\perp}$, with $n_q$ non-zero entries, from the
  cell $B^{j}_{q} \in \mcB^{j}$. If $p(d,w) \geq \left(\frac{2
    d}{\kappa(d)}\right)^{d}$ then
\[
\begin{split}
\left|
\sum_{r = 1}^{N} 
\sum_{h = 1}^{N} 
\phi(\bx_r,\by_h; \btheta) 
\bpsi^i_m[h] \bpsi^j_q[r] 
\right|
&\leq
\sqrt{n_mn_q}
\left( \frac{
  C(\tilde M,\sigma)^{d} e^{d - \sigma(1 - \delta) + 1} \hat \sigma d
}
 {
   (\sigma \delta)^{d}} \right)^2
 \\
 &
 \left( \frac{e^{\hat \sigma}}{1 - e^{-\hat \sigma}} \right)^{2d}
 \exp \left(-\frac{2d}{e} \hat \sigma  p^{\frac{1}{d}}
 \right) p^{2\left(\frac{d-1}{d}\right)}.
\end{split}
\]
\label{errorestimates:theorem1}
\end{theorem}
% Proof
\begin{proof} 
We first have that
\[
\begin{split}
\sum_{k = 1}^{N} 
\sum_{l = 1}^{N} 
\phi(\bx_k,\by_l; \btheta) 
\bpsi^i_m[k] \bpsi^j_q[l] 
&=
\sum_{k = 1}^{N} 
\sum_{l = 1}^{N}
\lim_{g \rightarrow \infty}
(\bI^d_{g}
\otimes
\bI^d_{g})[
\phi(\bx_k,\by_l; \btheta)] 
\bpsi^i_m[k] \bpsi^j_q[l] \\
&=
\sum_{k = 1}^{N} 
\sum_{l = 1}^{N}
(I_d - \mcS_{w,d}) 
\otimes
(I_d - \mcS_{w,d}) \\
&[\phi(\bx_k,\by_l; \btheta)] 
\bpsi^i_m[k] \bpsi^j_q[l].
\end{split}
\]
The last equality follows from $\bpsi^{i}_m, \bpsi^{j}_{q} \in \mcP^{
  p}(\mathbb{S})^{\perp}$. We now have that
\[
\begin{split}
& \sum_{k = 1}^{N} 
\sum_{l = 1}^{N}
\|(I_d - \mcS_{w,d}) 
\otimes
(I_d - \mcS_{w,d})
[\phi(\bx_k,\by_l; \btheta)] \|_{L^{\infty}_{\rho}(\Gamma^d)}
|\bpsi^i_m[k]| |\bpsi^j_q[l]|
\\
&\leq 
\|(I_d - \mcS_{w,d}))[\phi(\bx_k,\by_l; \btheta)] \|^{2}_{L^{\infty}_{\rho}(\Gamma^d)}
\sum_{k = 1}^{N} 
\sum_{l = 1}^{N}
|\bpsi^i_m[k]| |\bpsi^j_q[l]|.
\end{split}
\]
Since $\bpsi^i_m$ and $\bpsi^j_q$ are orthonormal then
\[
\begin{split}
\sum_{k = 1}^{N} 
\sum_{l = 1}^{N}
|\bpsi^i_m[k]| |\bpsi^j_q[l]|
&\leq \sqrt{n_m n_q}
\|\bpsi^i_m[k]\|_{l^2} \|\bpsi^j_q[l]\|_{l^2} =
\sqrt{n_m n_q}.
\end{split}
\]
From Lemma \ref{interpolation:lemma1} the result follows.
\end{proof}



\begin{remark} Recall that the restriction $p(d,w) \geq \left(\frac{2
  d}{\kappa(d)}\right)^{d}$ is not strict and can be relaxed such that
  sub-exponential convergence is still obtained. See Remark
  \ref{interpolation:remark1}.
\end{remark}

\begin{remark}
  The decay of the coefficients of $\bC^{i,j}_{\bW}$ is sub-exponential
  with respect to $p$.  Even for a moderate magnitude for $\hat
  \sigma > 0$, $p > 0$ and $d \geq 1$ the entries of the
  multilevel matrix $\bC^{i,j}_{\bW}$ that do not correspond to the
  cells given by the distance criterion parameter $\tau_{i,j} \geq 0$
  will be close to zero.
  %Thus the matrix $\bC^{i,j}_{\bW}$ will be
  %highly sparse as the number of observations $N$ increases.
\end{remark}

Theorem \ref{errorestimates:theorem1} provides a mechanism to control
the decay of the coefficients of the multilevel covariance matrix
$\bC_{\bW}$. To this end, we are interested in the analytic extension
and the uniform bound $\tilde M(\phi) \leq \infty$ of the Mat\'{e}rn
covariance function
\[
\phi(r;\btheta):=\frac{1}{\Gamma(\nu)2^{\nu-1}} \left(
\sqrt{2\nu} r(\btheta) \right)^{\nu} K_{\nu} \left(
\sqrt{2\nu} r(\btheta) \right)
\]
on a subdomain in $\bbC^{d} \times \bbC^{d}$, where $r(\btheta) =
(\bx-\by)^{T}$ $\text{diag}(\btheta) (\bx - \by)
)^{\frac{1}{2}}$, $\btheta=[\theta_1, \dots, \theta_d] \in
\R^{d}_{+}$ are positive constants, $\text{diag}(\theta) \in \R^{d
  \times d}$ is a diagonal matrix with the vector $\btheta$ on the
diagonal, and $\bx, \by \in \R^{d}$.

The polynomial function is an entire function. However, the function
$K_{\nu}(\vartheta)$ and $\vartheta^{\frac{1}{2}}$ are analytic for
all $\vartheta \in \bbC$ except at the branch cut $(-\infty,0]$.  Thus
it is sufficient to check the analytic extension of $r = \|\bx -
\by\|_{l^{2}} = \Big( \sum_{k=1}^{d} \theta_{k} (x_k - y_k)^{2}
\Big)^{\frac{1}{2}}$.  Let $z \in \bbC$ be the complex extension of
$r \in \R$. More precisely, $z = \Big( \sum_{k=1}^{d} \theta_{k}
z_k^{2} \Big)^{\frac{1}{2}}$, where $z_k \in \bbC$ is the complex
extension of $(x_k - y_k)$.

Let $\vartheta = \sum_{k=1}^{d} \theta_{k} z_k^{2}$, then by taking
the appropriate branch $\Real z = r_{\vartheta}$ $\cos{(
  \theta_{\vartheta}/2)}$, where $r^2_{\vartheta} = (\Real
\vartheta)^2 + (\Imag \vartheta)^2$ and $\theta_{\vartheta} =
\tan^{-1} \frac{\Imag \vartheta}{\Real \vartheta}$. Due to the branch
cut at $(-\infty,0]$ we impose the restriction that $\Real \vartheta >
0$ as $x_k$ and $y_k$ are extended in the complex plane.  Consider any
two cells $B^{i}_{m} \in \mcB^i$ and $B^{j}_{q} \in \mcB^j$, at levels
$i$ and $j$ with the associated distance criterion constant
$\tau_{i,j}>0$. From Algorithms \ref{MLCM:algorithm3},
\ref{MLCM:algorithm4}, \ref{MLCM:algorithm5} \ref{MLCM:algorithm6},
for any observations $\bx^{*} \in B^{i}_{m}$ and $\by^{*} \in
B^{j}_{q}$ we have that $(x^*_k - y^*_k)^2 \geq \tau^2_{i,j}$ for $k =
1,\dots,d$.  For the rest of the discussion it is assumed that complex
extension is respect to each component $k = 1,\dots,d$ unless
otherwise specified.


Let $x^{min}_k := \min_{ x^*_k \in B^i_m} x^*_k$, $x^{max}_k := \max_{
  x^*_k \in B^i_m} x^*_k$, $y^{min}_k := \min_{ y^*_k \in B^i_m}
y^*_k$, $y^{max}_k := \max_{ y^*_k \in B^i_m} y^*_k$ and
$\alpha_k,\gamma_k \in [-1,1]$. Define the region $\mcX^{i}_{m} :=
[x^{min}_1,$ $x^{max}_1] \times \dots \times [x^{min}_d,x^{max}_d]$
and $\mcY^{j}_{q} := [y^{min}_1,y^{max}_1] \times \dots \times
[y^{min}_d,y^{max}_d]$.

%rescale the widths of each of the cells $B^{i}_{m}$ and $B^{j}_{q}$
%with respect to the domain $\Gamma$.

The next step is to redefine $\phi(\bx;\by;\btheta):\mcX^{i}_m \times
\mcY^{i}_{q} \rightarrow \R$ as
$\phi(\balpha,\bgamma;\btheta):\Gamma^d \times \Gamma^d \rightarrow
\R$ through a pullback. Let $x_k = \left(\frac{\alpha_k + 1}{2}
\right) a_k + b_k$ and $y_k = \left(\frac{\gamma_k + 1}{2} \right) c_k
+ d_k$, where $a_k = x^{max}_{k} - x^{min}_{k}$, $b_k = x^{min}_{k}$,
$c_k = y^{max}_{k} - y^{min}_{k}$ and $d_k = y^{min}_{k}$.


%The covariance function
%$\phi(\bx;\by;\btheta):\mcX^{i}_m \times \mcY^{i}_{q} \rightarrow \R$
%is reformulated as $\phi(\balpha,\bgamma;\btheta):\Gamma^d \times
%\Gamma^d \rightarrow \R$.

Extend $\alpha_k \rightarrow \alpha_k + v_k$ and $\gamma_k
\rightarrow \gamma_k + w_k$ where $v_k:= v^R_k + iv^I_k$, $w_k:= w^R_k
+ iw^I_k$, and $v^R_k,v^I_k,w^R_k,w^I_k \in \R$. Let $\tilde x_k$ be
the extension of $x_k$ in the complex plane and similarly for
$\tilde y_k$.


It follows that $\tilde x^R_k := \Real \tilde x_k = \frac{1}{2}
(\alpha_k + 1 + v^R_k)a_k + b_k$, $\tilde x^I_k = \Imag \tilde x_k =
\frac{v^I_k}{2} a_k$, $y^R_k := \Real \tilde y_k = \frac{1}{2}
(\gamma_k + 1 + w^R_k)c_k + d_k$, and $y^I_k := \Imag \tilde y_k =
\frac{w^I_k}{2} c_k$.  After some manipulation
\begin{equation}
\begin{split}
\Real z^2_k &= (\tilde x^R_k - \tilde y^R_k)^2
- (\tilde x^I_k - \tilde y^I_k)^2 
=
(x_k - y_k)^2 
+
\frac{1}{4}(v^R_k a_k - w^R_k c_k)
+ (x_k - y_k)(v^R_k a_k - w^R c_k)
\\
&
-\frac{1}{4}(a_kv^I_k - c_k w^I_k)^2.
\end{split}
\label{errorestimates:eqn2}
\end{equation}


Recall that $(x_k - y_k)^2 \geq \tau^2_{i,j}$ and suppose that there
is a positive constant $\delta_{k} > 0 $ such that
\begin{equation}
\delta_{k} \leq 
%-\frac{4\,\tau -\sqrt{32\,\tau ^2+8\,\tau +1}+1}{4\,\tau }
\frac{\sqrt{32\,\tau_{i,j} ^2+8\,\tau_{i,j} +1}-1 - 4\,\tau_{i,j}  }{4\,\tau_{i,j}}.
\label{errorestimates:eqn2a}
\end{equation}

Assume that $|v^R_k|\leq \tau_{i,j} \delta_{k} / a_{k}$, $|v^I_k|\leq
\tau_{i,j} \delta_{k}/a_{k}$, $|w^R_k|\leq \tau_{i,j} \delta_{k} /
c_{k}$, and $|w^I_k|\leq \tau_{i,j} \delta_{k} / c_{k}$. From
equations \eqref{errorestimates:eqn2} and \eqref{errorestimates:eqn2a}
it follows that
\begin{equation}
\Real z^2_k \geq \tau_{i,j}^2 (1 - 4 \delta_{k}^2) - \frac{\tau_{i,j}
  \delta_{k}}{2} > 0.
\label{errorestimates:eqn3}
\end{equation}
%\begin{equation}
%\Real z^2_k \geq \tau_{i,j,m,q}^2 (1 - \frac{9}{16}(a_k + c_k)) > 0
%\label{errorestimates:eqn3}
%\end{equation}
Furthermore,
\begin{equation}
\begin{split}
  &
  |\Real z^2_k| \leq 
(\max\{|y^{max}_k - x^{min}_{k}|,|x^{max}_k - y^{min}_{k}|\})^2
\\
&+ \frac{1}{2}\tau_{i,j} \delta_{k} 
+ 
\max\{|y^{max}_k - x^{min}_{k}|,|x^{max}_k - y^{min}_{k}|\} 
2\tau_{i.j} \delta_{k}
+ \tau^2_{i,j} \delta_{k}^2 
\leq
1 + \frac{5}{2}\tau_{i,j} \delta_{k} + \tau^2_{i,j} \delta_{k}^2. 
\end{split}
\label{errorestimates:eqn6}
\end{equation}
Similarly,
\begin{equation}
  |\Imag z^2_k| = |2(\tilde x^R_k - \tilde y^R_k)(x^I_k - y^I_k)|
  \leq
2 \tau_{i.j} \delta_{k} + 4 \tau^2_{i,j} \delta_{k}^2.
\label{errorestimates:eqn4}
\end{equation}

We now show how $\alpha_k$ and $\gamma_k$ can be extended into the
Bernstein ellipses $\mcE_{\sigma^{\alpha}_k}$ and
$\mcE_{\sigma^{\gamma}_k}$, for some $\sigma^{\alpha}_k > 0$ and
$\sigma^{\gamma}_k > 0$ such that $\Real z^2_k > 0$. Recall that
$|v^R_k|\leq \tau_{i,j} \delta_{k} / a_{k}$, $|v^I_k|\leq \tau_{i,j}
\delta_{k}/a_{k}$, $|w^R_k|\leq \tau_{i,j} \delta_{k} / c_{k}$, and
$|w^I_k|\leq \tau_{i,j} \delta_{k} \ c_{k}$.
%We restrict the length of the extension of $(x^{min}_k, x^{max}_k)$
%and $(y^{min}_k, y^{max}_k)$ by $\tau_{i,j}/2$
These restrictions form a region in $\bbC \times \bbC$ and a Bernstein
ellipse is embedded (See Figure \ref{analyticity:fig1}).  This is done
by solving the following equation: $\frac{e^{\sigma^{\alpha}_k} +
  e^{-\sigma^{\alpha}_k}}{2} = 1 + \frac{\tau_{i,j}
  \delta_{k}}{a_k}$. The unique solution is
\[
\sigma^{\alpha}_k = \cosh^{-1} \left(1 +
\frac{\tau_{i,j} \delta_{k}}{a_k}
\right)
\]
with $\sigma^{\alpha}_k > 0$. Following a
similar argument we have that
\[
\sigma^{\gamma}_k = \cosh^{-1} \left(1
+ \frac{\tau_{i,j} \delta_{k}}{c_k}
\right)
\]
with $\sigma^{\gamma}_k > 0$. Let $\mcE^{d}_{\alpha} :=
\prod_{k=1}^{d} \mcE_{\sigma^{\alpha}_k}$ and $\mcE^{d}_{\gamma} :=
\prod_{k=1}^{d} \mcE_{\sigma^{\gamma}_k}$.  It follows that
\begin{equation}
\begin{split}
  \Real \vartheta &\geq  
  \sum_{k=1}^{d} \theta_k
  \Real z^2_k 
  \geq
\sum_{k=1}^{d} \theta_k 
  \left(
  \tau_{i,j}^2 (1 - 4 \delta_{k}^2) - \frac{\tau_{i,j}
    \delta_{k}}{2} \right)
  > 0.
  \end{split}
\label{errorestimates:eqn8}
\end{equation}
Thus there exist an analytic extension of $\phi(r;\btheta):\Gamma^d
\times \Gamma^d \rightarrow \R$ on $\mcE^{d}_{\alpha} \times
\mcE^{d}_{\gamma}$.



\begin{figure*}[htb!]
\begin{center}
\begin{tikzpicture}
    \begin{scope}[font=\scriptsize]

    \draw [->] (-2.5, 0) -- (2.5, 0) node [above left] {$\Real $};
    \draw [->] (0,-1.5) -- (0,1.5) node [below right] {$\Imag$};
    \draw [-,dashed] (-2,-1.5) -- (-2,1.5);
    \draw [-,very thick] (-1,0) -- (1,0);
    \draw [-,dashed] (2,-1.5) -- (2,1.5);
    \draw (1,-3pt) -- (1,3pt) node [above] {$-1$};
    \draw (-1,-3pt) -- (-1,3pt) node [above] {$1$};


    \draw [-,dashed] (-2,1) -- (2,1);
    \draw [-,dashed] (-2,-1) -- (2,-1);
    
 
    \fill [opacity=0.2, pattern=north west lines, pattern color=red]
    (-2,-1) rectangle (-1.5,1);

    \fill [opacity=0.2, pattern=north west lines, pattern color=red]
    (1.5,-1) rectangle (2,1);
    \end{scope}
    
    \node [below right] at (-1.92,0) {$\frac{\tau_{i,j} \delta_{k}}{a_k}$};
    \node [below right] at ( 1.05,0) {$\frac{\tau_{i,j} \delta_{k}}{a_k}$};

    \node at (0,-1.8) {$(a)$};
    \node at (6,-1.8) {$(b)$};
    
    \begin{scope}[shift={(0,0)},font=\scriptsize]

      \filldraw[fill={rgb:red,143;green,188;blue,143},semitransparent]
      (6,0) ellipse (2 and 1);

    \draw [->] (3, 0) -- (9, 0) node [above left] {$\Real $};
    \draw [->] (6,-1.5) -- (6,1.5) node [below right] {$\Imag$};
    \draw (5,-3pt) -- (5,3pt)   node [above] {$1$};
    \draw (7,-3pt) -- (7,3pt) node [above] {$-1$};


    %\fill [opacity=0.2, pattern=north west lines, pattern color=red]
    %(-2,-1.5) rectangle (-1.5,1.5);
    \end{scope}
    
    \node [below right] at (7.50,1.25) {$\mcE_{\sigma^{\alpha}_k}$};
    \node [below right] at (7.25,0.05) {$\frac{e^{\sigma^{\alpha}_k}
                                       + e^{-\sigma^{\alpha}_k}}{2}$};
        
\end{tikzpicture}
\end{center}
\caption{(a) Region of Complex extension of $\alpha_k$.  (b) Embedding
  of Bernstein ellipse $\mcE_{\sigma^{\alpha}_k}$.}
\label{analyticity:fig1}
\end{figure*}

The next step is to bound the absolute value of the Mat\'{e}rn
covariance function $|\phi(z;\btheta)|$ in $\mcE^{d}_{\alpha} \times
\mcE^{d}_{\gamma}$. If $\nu > \frac{1}{2}$ and $\Real z>0$ then the
modified Bessel function of the second kind satisfies the following
identity
\[
\begin{split}
K_{\nu}(\sqrt{2 \nu}z) &= \frac{\sqrt{\pi} (\sqrt{2 \nu}z)^{\nu}}{2^\nu
  \Gamma(\nu + \frac{1}{2})} \\
&
\int_{1}^{\infty} (t^2 - 1)^{\nu -
  \frac{1}{2}} \exp{(-\sqrt{2 \nu}zt)}\, \text{d}t.
\end{split}
\]
It is not hard to show that for $\nu > \frac{1}{2}$ and $\Real z >0$,
we have that $|K_{\nu}(\sqrt{2 \nu}z)| \leq \frac{|\sqrt{2
    \nu}z|^{\nu}}{(\Real \sqrt{2 \nu}z)^{\nu}} K_{\nu}(\sqrt{2 \nu}
\Real z)$.

Note that $r_{\vartheta} \geq \Real \vartheta > 0$.  From equation
\eqref{errorestimates:eqn4} we have that $\Imag \vartheta =
\sum_{k=1}^{d} \theta_k \Imag z^2_k \leq \sum_{k=1}^{d} 2 \tau
\delta_{k} + 4 \tau^2 \delta_{k}^2$.  From equation
\eqref{errorestimates:eqn8}
\[
\begin{split}
  |\theta_{\vartheta}|
  &\leq
\xi(\btheta,\bdelta,\tau_{i,j}) := \tan^{-1}
\left(
\frac{
  \sum_{k=1}^{d} 2 \tau_{i,j}
\delta_{k} + 4 \tau^2_{i,j} \delta_{k}^2
}
{
\sum_{k=1}^{d} \theta_k 
  \left(
  \tau_{i,j}^2 (1 - 4 \delta_{k}^2) - \frac{\tau_{i,j}
    \delta_{k}}{2} \right)
}
\right)
< \frac{\pi}{2}.
\end{split}
\]
Since $K_{\nu}(\cdot)$ is strictly completely monotonic
\cite{Baricz2011} then
\begin{equation}
\begin{split}
  |K_{\nu}(\sqrt{2 \nu}\Real z)| &=
  |K_{\nu}\ (\sqrt{2 \nu}
  r_{\vartheta} \cos(\theta_{\vartheta}/2))
  | 
  \leq
  \Big| K_{\nu}\Big(\sqrt{\frac{\nu}{2}}
  \cos(\xi(\btheta,\bdelta,\tau)/2) \\
  &
  \sum_{k=1}^{d} \theta_k 
  \Big(
  \tau_{i,j}^2 (1 - 4 \delta_{k}^2) - \frac{\tau_{i,j}
    \delta_{k}}{2} \Big)
  \Big) \Big|.
\end{split}
\label{errorestimates:eqn5}
\end{equation}
From equations \eqref{errorestimates:eqn6}
\eqref{errorestimates:eqn4} 
\[
\begin{split}
|z_k|^{2} &\leq |\Real z^2_k| + |\Imag z^2_k|
\leq \mcR(\delta_k,\tau_{i,j})
:=
1 + \frac{9}{2}\tau_{i,j} \delta_{k} + 5 \tau^2_{i,j} \delta_{k}^2
\end{split}
\]
and therefore
\begin{equation}
\begin{split}
  |z|
  &\leq
\left|
\sum_{k=1}^{d} \theta_{k} z_k^{2} \right|^{\frac{1}{2}}
\leq
\left(
\sum_{k=1}^{d} \theta_{k} |z_k|^{2} \right)^{\frac{1}{2}}
\leq \left( \sum_{k=1}^{d} \theta_k \mcR(\delta_k,\tau_{i,j})
\right)^{\frac{1}{2}}.
\end{split}
\label{errorestimates:eqn7}
\end{equation}
By combining equations \eqref{errorestimates:eqn4},
\eqref{errorestimates:eqn8}, \eqref{errorestimates:eqn5}, and
\eqref{errorestimates:eqn7}, we have now proven the following Theorem.

\begin{theorem} For any two cells $B^{i}_{m}$ and $B^{j}_{q}$
with the associated distance criterion parameter $\tau_{i,j} \geq 0$
let $\phi(\balpha,\bgamma;$ $\btheta):\Gamma^d \times \Gamma^d
\rightarrow \R$ be the pullback of the Mat\'{e}rn covariance function
$\phi(\bx;\by;\btheta):\mcX^{i}_m \times \mcY^{j}_{q} \rightarrow
\R$. Then there exists an analytic extension of
$\phi(\balpha,\bgamma;\btheta):\Gamma^d \times \Gamma^d \rightarrow
\R$ on the polyellipse $ \mcE^{d}_{\alpha} \times \mcE^{d}_{\gamma} $
and
  \[
  |\phi(\cdot,\cdot;\btheta)| \leq
\frac{  \left( 2 \nu \sum_{k=1}^{d} \theta_k \mcR(\delta_k,\tau_{i,j})
\right)^{\frac{\nu}{2}}
|K_{\nu}
(\Xi(\btheta,\bdelta,\tau_{i,j}
)|}{
\Xi(\btheta,\bdelta,\tau_{i,j})^{\nu}
}
\]
on $\mcE^{d}_{\alpha} \times \mcE^{d}_{\gamma}$, where
\[
\begin{split}
  \Xi(\btheta,\bdelta,\tau_{i,j})
  :=
\Big| K_{\nu}\Big(\sqrt{\frac{\nu}{2}}
  \cos(\xi(\btheta,\bdelta,\tau_{i,j})/2)
\sum_{k=1}^{d} \theta_k 
  \Big(
  \tau_{i,j}^2 (1 - 4 \delta_{k}^2) - \frac{\tau_{i,j}
    \delta_{k}}{2} \Big)
  \Big) \Big|.
  \end{split}
\]


\end{theorem}

%% Numerical Results --------------------------------------------------
%%
\section{Numerical results}
\label{numericalresults}

The performance of the multilevel solver for estimation and
prediction formed from random datasets is tested. The results show
that the computational burden is significantly reduced while retaining
good accuracy. In particular, it is possible to now solve
ill-conditioned problems efficiently. The implementation of the code
is as follows:


\begin{enumerate}[i)]

\item {\bf Matlab, C/C++ and MKL:} The binary tree, multilevel basis
  construction, formation of the sparse matrix $\tilde \bC_{\bW}$,
  estimation and prediction components are written and executed on
  Matlab \cite{Matlab2016}. However, the computational bottlenecks are
  executed by C/C++ software packages, Intel MKL \cite{intelmkl}, and
  the highly optimized BLAS and LAPACK packages contained in
  MATLAB. The C/C++ interfaces to matlab are constructed as dynamic
  shared libraries.

  

\item {\bf Direct and fast summation:} The matlab code estimates the
  computational cost between the direct and fast summation methods and
  chooses the most efficient approach.  For the direct method a
  combination of Graphic Processing Unit (GPU) and MKL intel libraries
  are used. For the fast summation method the KIFMM ($d = 3$) c++ code
  is used.  The KIFMM is modified to include a Hermite interpolant
  approximation of the Mat\'{e}rn covariance function, which is
  implemented with the intel MKL package \cite{intelmkl} (see
  \cite{Castrillon2015} for details).


\item {\bf Dynamic shared libraries:} These are produced with the GNU
  gcc/g++ packages. These libraries implement the Hermite interpolant
  with the intel MKL package (about 10 times faster than Matlab
  Mat\'{e}rn interpolant) and link the MATLAB code to the KIFMM.

\item {\bf Cholesky and determinant computation:} The Suite Sparse
  4.2.1 package
  (\cite{Chen2008,Davis2009,Davis2005,Davis2001,Davis1999}) is used
  for the determinant computation of the sparse matrix $\tilde
  \bC_{\bW}(\btheta)$.

\end{enumerate}

The code is tested on a single CPU (4 core Intel i7-3770 CPU @
3.40GHz.), one Nvidia 970 GTX GPU, with Linux Ubuntu 18.04 and 32 GB
memory. In addition, the Boston University Shared Computing Cluster
was used to generate test data.  To test the effectiveness of the
Multilevel solver the following data sets are generated:
\begin{enumerate}[a)]


\item {\bf Random n-sphere data set:} The set of nested random
  observation $\bS_{0}^{d} \subset \dots \subset \bS_{9}^{d}$ vary
  from 1,000, 2000, 4000 to 256,000 knots generated on the n-sphere
  $\bS_{d-1} := \{\bx \in \R^{d}\,\,|\,\,\|\bx\|_{2} = 1 \}$.

\item {\bf Random hypercube data set:} The set of random observation
  locations $\bC_{0}^{d},$ $\dots, \bC_{10}^{d}$ vary from 1,000, 2000,
  4,000 to 512,000 knots generated on the hypercube $[-1,1]^{d}$ for
  $d$ dimensions.  The observations locations are also nested,
  i.e. $\bC_{0}^{d} \subset \dots \subset \bC_{10}^{d}$.
  
\item {\bf Normal test data set} The set of observations values
  $\bZ^d_{0}$, $\bZ^d_{1}$, \dots $\bZ^d_{5}$ are formed from the
  Gaussian random field model \eqref{Introduction:noisemodel} for
  1,000, 2,000, $\dots$ $256,000$ observation locations. The data set
  $\bZ^d_{n}$ is generated from the set of nodes $\bS^{d}_{n}$, with
  the covariance parameters $(\nu,\rho)$ and the corresponding set of
  monomials $\mcQ^d_w$. The Boston University Shared Computing Cluster
  was used to generate the normal test data.
    
%\item All the numerical test are done assuming the $\tilde p = p$.
  
\end{enumerate}


\begin{remark}
 All the timings for the numerical tests are given in wall clock
 times i.e. the actual time is needed to solve a problem. This is to
 distinguish from CPU time, which can be significantly smaller.
  \end{remark}


\subsection{Numerical stability and sparsity of the covariance multilevel
  matrix}

For many practical cases the covariance matrix $\bC(\btheta)$ becomes
increasingly ill-conditioned for the Mat\'{e}rn covariance function as
$\rho$, $\nu$ and the number of observations are increased. This leads
to instability of the numerical solver. It is now shown how effective
Theorem \ref{Multilevelapproach:theo1} becomes in practice.  In
Figure \ref{numericalresults:fig1} the condition number of the
multilevel covariance matrix $\bC_{\bW}(\btheta)$ is plotted with
respect to the cardinality $p(w,d)$ of $\mcQ^d_w$ for different $w$
levels. The multilevel covariance matrix $\bC_{\bW}(\btheta)$ is built
from the random cube $\bC^{d}_{4}$ or n-sphere $\bS^{d}_{4}$
observations. The covariance function is set to Mat\'{e}rn with $\nu =
1$ and $\rho = 1,10$.  As the plots confirm the covariance matrix
condition number significantly improves with increasing level
$w$. This is in contrast with the large condition numbers of the
original covariance matrix $\bC(\btheta)$.  This is consistent with
Theorem \ref{Multilevelapproach:theo1} and Corollary
\ref{Multilevelapproach:cor1}.


We now focus our attention of the sparsity of the covariance matrix
$\tilde \bC_{\bW}(\btheta)$. In Figure \ref{numericalresults:fig2}(a)
the magnitude of the multilevel covariance matrix
$\bC_{\bW}(\btheta)$ is plotted for $N = 8,000$ observations from the
the n-sphere $\bS^{3}_{3}$ with Mat\'{e}rn covariance parameters $\nu
= 0.5$ (exponential) and $\rho = 10$. Due to the large value of $\rho$
the overlap between the covariance function at the different locations
in $\bS^{3}_{3}$ is quite significant, thus leading to a dense
covariance matrix $\bC(\btheta)$ where the coefficients decay
slowly. This is in contrast to the large number of small entries for
$\bC_{\bW}(\btheta)$, as shown in the histogram in Figure
\ref{numericalresults:fig2}(b). Note that the histogram is in terms of
$\log_{10}$ of the absolute value of the entries of
$\bC_{\bW}(\btheta)$. From the histogram it is observed that almost
all the entries are more than 1000 smaller than the largest entries.
This numerical result is consistent with the sub-exponential decay
rates of Theorem \ref{errorestimates:theorem1}.


\begin{figure*}[htpb]
\begin{center}
\begin{tikzpicture}%[thick,scale=1, every node/.style={scale=1}]
  \node[inner sep=0pt] at (0,0)
  {
  \includegraphics[trim = 120 255 120 255,
    clip,width=4.4in,height=4in]{ConditionGraphsReduced.pdf}
  };
  \node[rotate = 90] at (-5.5,2.7) {$\kappa(\bC_{\bW}(\btheta))$};
  \node[rotate = 90] at (0,2.7) {$\kappa(\bC_{\bW}(\btheta))$};

  \node[rotate = 90] at (-5.5,-2.7) {$\kappa(\bC_{\bW}(\btheta))$};
  \node[rotate = 90] at (0,-2.7) {$\kappa(\bC_{\bW}(\btheta))$};

\node at (-2.6,5.3) {
\begin{tabular}{c}
\small $Cube, d = 5, \bC^5_4, \rho=1$ \\
\small $\kappa(\bC(\btheta))= 1.1 \times 10^{7}$
\end{tabular}
};

\node at (2.9,5.3) {
\begin{tabular}{c}
\small $Cube, d = 5, \bC^5_4, \rho=10$ \\
\small $\kappa(\bC(\btheta))= 2.2 \times 10^{9}$
\end{tabular}
};

  
\node at (-2.6,-0.15)
{
\begin{tabular}{c}
\small $Sphere, d = 5,  \bC^5_4, \rho=1$ \\
\small $\kappa(\bC(\btheta))= 2.6 \times 10^{7}$
\end{tabular}
};

\node at (2.9,-0.15)
{
\begin{tabular}{c}
\small $Sphere, d = 5, \bS^5_4, \rho=10$ \\
\small $\kappa(\bC(\btheta))=7.8 \times 10^{9}$
\end{tabular}
};
\node at (2.9,-5.0)
      {$p$
        };
\node at (-2.6,-5.0)
      {$p$
        };
\end{tikzpicture}
\end{center}
\caption{Condition number of the multilevel covariance matrix
  $\bC_{\bW}(\btheta)$ with respect to the size $p$ of the Total
  Degree (TD) polynomial space. The number of observations corresponds
  to 16,000 nodes generated on a hypercube or n-sphere of dimension $d
  = 5$. The covariance function is chosen to be Mat\'{e}rn with $\nu =
  1$ and $\rho=1,10$.  The condition number of the covariance matrix
  $\bC(\btheta)$ is placed on the top of each subplot for
  comparison. The MB is constructed from a kD-tree.  As expected, as
  $p$ increases with $w$ the condition number of $\bC_{\bW}(\btheta)$
  decreases significantly. This is consistent with Theorem
  \ref{Multilevelapproach:theo1} and Corollary
  \ref{Multilevelapproach:cor1}}
\label{numericalresults:fig1}
\end{figure*}

In Table \ref{numericalresults:table1} sparsity and construction wall
clock times of the sparse matrices $\tilde{\bC}^{i}_{\bW}(\btheta)$,
$i = t, t-1, \dots$, for various values of $i$ are shown.  The
polynomial space of the index set $\mcQ^d_w$ is restricted to TD on a
n-Sphere with $d = 10$ dimensions. The domain decomposition is formed
with a kD-tree. The level of the index set is set to $w = 7$, which
corresponds $p = 1001$. The covariance function is Mat\'{e}rn with
$\nu = 3/4$, $\rho = 3/4$. The distance criterion for each $(i,j)$
multilevel covariance matrix block is set to
\[
\tau_{i,j} := 2^{(t - i)/2}2^{(t - j)/2} \tau,
\]
for $i = 1, \dots, t$ and $j = 1 \dots, t$, where $\tau = 3 \times
10^{-6}$.

The first observation to notice is that all the sparse matrices
$\tilde{\bC}^{i}_{\bW}(\btheta)$, $i = t, t-1, \dots$ {\it are very
  well conditioned, thus numerically stable}. This is in contrast to
the original covariance matrices that are in general poorly
conditioned. The sparsity of $\tilde{\bC}^{i}_{\bW}(\btheta)$ and the
Cholesky factor $\bG$ are shown in columns 7 and 9. The construction
time $t_{con}$ of the $\tilde{\bC}^{i}_{\bW}(\btheta)$ is shown in
column 9. In column 5 $t_{ML}$ is the time required to build the
multilevel basis.  We observe that for large matrices the sparse
matrix $\tilde{\bC}^{i}_{\bW}(\btheta)$ are built efficiently.  It is
noted that the sparse matrices in Table \ref{numericalresults:table1}
are built with a direct summation method due to the dimensionality $d
= 10$.


\setlength{\tabcolsep}{6pt}

\begin{figure*}
\begin{center}
\psfrag{A}[c][t]{\small $\log_{10}(abs(\bC_{\bW}(\btheta)))$}
\begin{tabular}{c c}
\includegraphics[width=2in,height=2in]{MatrixDecay.pdf}
&
\hspace*{0mm}
\raisebox{-2.5mm}[0pt][0pt]{
\psfrag{Hist}[c][t]{\small Histogram of $\log_{10} abs(\bC_{\bW}(\btheta))$}
\psfrag{log10}[c][t]{\tiny $\log_{10} abs(\bC_{\bW}(\btheta))$ (100 bins)}
\includegraphics[width=2.1in,height=2.1in]{HistMatrixDecay.pdf}
} \\
    & \\
(a) & (b)
\end{tabular}
\end{center}
\caption{(a) Magnitude pattern and (b) histogram of $\log_{10}
  abs(\bC_{\bW}(\btheta))$ with 100 bins where $abs(\bC_{\bW}(\btheta))
  \in \R^{(N-p) \times (N-p)}$ is the magnitude of the entries of the
  matrix $\bC_{\bW}(\btheta)$.  The matrix $\bC_{\bW}(\btheta)$ is created
  with $d = 3$ dimensions, $N$ = 8,000 random locations on the sphere
  ($\bS^3_3$), and the Mat\'{e}rn covariance function with $\rho =
  10$, $\nu = 0.5$ (exponential), Total Degree index Set $\Lambda(w)$
  with $w = 4$, and $p = 35$. As observed from (a) and (b) most of
  entries of the matrix $\bC_{\bW}(\btheta)$ are very small.}
%and can thus be safely eliminated
%  without comprosing much accuracy.}
\label{numericalresults:fig2}
\end{figure*}




%% \begin{figure}[htpb]
%% \begin{center}
%% \psfrag{A-------------}[c][c]{\raisebox{0mm}[0pt][0pt]{\tiny $w = 3$}}
%% \psfrag{B-------------}[c][c]{\raisebox{0mm}[0pt][0pt]{\tiny $w = 4$}}
%% \psfrag{C-------------}[c][c]{\raisebox{0mm}[0pt][0pt]{\tiny $w = 5$}}
%% \psfrag{Error}[c][t]{\small Total degree log det relative error}
%% \psfrag{x}[c][t]{\tiny Sparsity}
%% \psfrag{y}[c][t]{\tiny $\frac{
%% |\log{det(\bC^n_{\bW})} - \log{det(\tilde \bC^n_{\bW})}|
%% }{
%% \log{det(\bC^n_{\bW})}}$}
%% \begin{tabular}{c c}
%% \includegraphics[width=3in,height=3in]{./figures/sparsitydecay.eps}
%% &
%% \psfrag{Error}[c][t]{\small Smolyak log det relative error}
%% \psfrag{A-------------}[c][c]{\raisebox{0mm}[0pt][0pt]{\tiny $w = 2$}}
%% \psfrag{B-------------}[c][c]{\raisebox{0mm}[0pt][0pt]{\tiny $w = 3$}}
%% \psfrag{Hist}[c][t]{\small Histogram of $\log_{10}|\bC_{\bW}(\btheta)|$}
%% \psfrag{log10}[c][t]{\tiny $\log_{10}|\bC_{\bW}(\btheta)|$ (100 bins)}
%% \includegraphics[width=3in,height=3in]{./figures/sparsitydecay2.eps} 
%% \\
%% (a) TD, $N$ = 16,000, $d = 5$, kD-tree & 
%% (b) SM, $N$ = 16,000, $d = 10$, kD-tree 
%% \end{tabular}
%% \end{center}
%% \caption{Relative log determinant error $\frac{| \log \det \tilde
%%     \bC_{\bW}(\btheta) - \log \det \bC_{\bW}(\btheta)| |}{|\log \det
%%     \bC_{\bW}(\btheta)|}$ with respect to the sparsity of $\tilde
%%   \bC^{n}_{\bW}$ from Random hyper-sphere data $\bS^{d}_{3}$ with $N$
%%   =16,000 observations, Mat\'{e}rn covariance parameters $\nu = 0.5$,
%%   $\rho = 10$ and binary kD-tree. The index sets $\Lambda(\omega)$ are
%%   chosen from (a) Total Degree and (b) Smolyak index sets.}
%% \label{numericalresults:fig3}
%% \end{figure}


\setlength{\tabcolsep}{7pt}
\begin{table*}[htpb]
  \caption{Sparsity test on the matrices $\tilde{\bC}^{i}_{\bW}$, $i =
    t, t-1, \dots$.  The polynomial space of the index set $\mcQ^d_w$
    is restricted to TD on a n-Sphere with $d = 10$ dimensions. The
    domain decomposition is formed from a kD-tree. The level of the
    index set is $w = 7$, which corresponds $p = 1001$. The kernel
    function is Mat\'{e}rn with $\nu = 3/4$, $\rho = 3/4$ and $\tau :=
    3 \times 10^{-6}$. The first column is the number of random
    n-Sphere nodes. The second is the maximum level of the kD tree and
    $i$ is the level of the sparse matrix $\tilde{\bC}^{i}_{\bW}$. The
    fourth column is the condition number of $\tilde{\bC}^{i}_{\bW}$,
    which is excellent.  The fifth column is the size of the matrix
    $\tilde{\bC}^{i}_{\bW}$.  The seventh column, $t_{ML}$, is the
    total time for the construction of the multilevel basis. The
    eighth column is the sparsity of $\tilde{\bC}^{i}_{\bW}$.  The
    nineth column, $t_{con}$ is the total time for the construction of
    the matrix $\tilde{\bC}^{i}_{\bW}$. The tenth column is the
    sparsity of the Cholesky factor $\bG$ (with nested dissection
    reordering) of the sparse matrix $\tilde{\bC}^{i}_{\bW}$. The last
    column is the total time to compute the Cholesky factor $\bG$.}
\begin{center}
\begin{tabular}{ r r r r c c r r c r c r}
\multicolumn{1}{c}{$N$} &
% \multicolumn{1}{c}{$d$} &
\multicolumn{1}{c}{$t$} &
\multicolumn{1}{c}{$i$} &
\multicolumn{1}{c}{$\kappa(\tilde \bC_{\bW}^{i})$} &
\multicolumn{1}{c}{Size} &
%\multicolumn{1}{c}{$\tau$} &  
\multicolumn{1}{c}{$t_{ML}$} &
\multicolumn{1}{c}{$nz$} &
\multicolumn{1}{c}{$t_{con}$} &
\multicolumn{1}{c}{$nz(\bG)$} &
\multicolumn{1}{c}{$t_{sol}$}
 \\ 
 \hline
32,000 & 4 & 4 &   5  & 15,984 &  46 &  6.3\% &   11 &  3.1\% &  1 \\
32,000 & 4 & 3 &   8  & 23,992 &  46 & 10.4\% &   30 &  5.2\% &  3 \\
32,000 & 4 & 2 &  13  & 27,996 &  46 & 15.6\% &   82 &  7.8\% &  7 \\
32,000 & 4 & 1 &  19  & 29,998 &  46 & 20.1\% &  190 & 10.4\% & 16 \\
32,000 & 4 & 0 &  23  & 30,999 &  46 & 25.7\% &  310 & 13.0\% & 17 \\
\hline
64,000 & 5 & 5 &   6  & 31,968 & 104 &  3.5\% &   21 & 1.8\%  &  3 \\
64,000 & 5 & 4 &  11  & 47,984 & 105 &  6.3\% &   90 & 3.1\%  & 12 \\
64,000 & 5 & 3 &  18  & 55,992 & 106 &  9.6\% &  270 & 5.0\%  & 18 \\
64,000 & 5 & 2 & 121  & 59,996 & 121 & 13.4\% & 624 & 6.7\%  & 34 \\
\hline
128,000 & 6 & 6 &  8  &  63,936 &  237 & 4.0 \% & 120  & 2.1 \% & 15 \\
128,000 & 6 & 5 & 17  &  95,968 &  237 & 5.5 \% & 378 & 6.7 \% & 140 \\
\end{tabular}
\vspace{5mm}
\\
\end{center}
\label{numericalresults:table1}
\end{table*}

\subsection{Estimation}

In this section estimation results are presented for the Mat\'{e}rn
covariance matrix on high dimensional n-Sphere random locations by
solving multilevel log-likelihood
\[
\hat{\btheta} : =
\argmax_{\btheta}
\tilde{\ell}^{i}_{\bW}(\bZ^{i,k,d}_{W};\btheta),
\]
where $\bZ^{i,k,d}_{W} := [\bW_t \T, \dots, \bW_i \T ] \T
\bZ^{d}_{k}$, for $i = t, t-1, \dots$. The observation data $\bZ^d_k$
is built from the n-Sphere $\bS^d_k$ for $d = 3,10$, $k = 6$ $(N =
64,000)$ and $k = 7$ $(N = 128,000)$. The covariance function is
Mat\'{e}rn for several values of $\nu$ and $\rho$. To test the
performance of the multilevel estimator, $M = 100$ realizations are
generated for each case.

The optimization problem of the log-likelihood function
\eqref{Introduction:multilevelloglikelihoodreduced2} is solved using
a fmincon iteration search for the estimates $\hat \nu$ and $\hat
\rho$ from the optimization toolbox in MATLAB \cite{Matlab2016}. The
tolerance level is set to $10^{-6}$.

In Table \ref{numericalresults:table2} the mean and standard deviation
of the Mat\'{e}rn covariance parameter estimates $\hat \nu$ and $\hat
\rho$ are presented. The mean estimate $\bbE_M [\hat{\nu}]$ refers to
the mean of $M$ estimates $\hat{\nu}$ for the $M$ realizations of the
stochastic model. Similarly, $std_M [\hat{\nu}]$ refers to the
standard deviation of the $M$ realizations. For case (a) ($d = 3$) the
error mean and std is $\approx 1\%$. For case (b) ($d = 10$) the error
of the mean increase to $\approx 10 \%$. In general, as $i$ is reduced
from $t$ there is a tendency of a drop in the standard deviation
$std_M [\hat{\nu}]$ of the estimator $\hat \nu$. However, there is
also a tendency for the accuracy of the mean to degrade somewhat,
except for (a) $N = 128,000$, $i = 12 \rightarrow 11$.
%This inconsistent
%behavior in accuracy might be due to the standard deviation being
%approximately the same magnitude as the sample mean.


\begin{table*}[htpb]
\caption{Estimation of parameters $\hat \nu$ and $\hat \rho$ with:
  Total Degree polynomial index set $\mcQ^d_w$, kD tree, and n-Sphere
  with for $d = 3$ and $d = 10$.  The observation data $\bZ^d_k$ are
  formed from the Mat\'{e}rn covariance function. The number of
  realizations of the Gaussian random field model is set to $M =
  100$. Several cases are tested and are given by the individual tables
  (a) and (b).  The first to fourth columns are self-explanatory. The
  fifth column is the mean error of $\hat \nu$ with $M$
  realization. The sixth column is the mean error of $\hat \rho$. The
  last two columns are the standard deviation of $M$ realizations of
  the parameters $\hat \nu$ and $\hat \rho$.}
\begin{center}
(a) TD, kD tree, n-Sphere, $d = 3$, $M = 100$, $\nu =
  3/4$, $\rho = 1/6$, $\tau = 5 \times 10^{-2}$
\begin{tabular}{ r r r r r r r r r r}
\multicolumn{1}{c}{$N$} & 
% \multicolumn{1}{c}{$d$} &
\multicolumn{1}{c}{$w$} &
\multicolumn{1}{c}{$t$} &
\multicolumn{1}{c}{$i$} & 
\multicolumn{1}{c}{$\bbE_M [\hat{\nu} - \nu]$} &
\multicolumn{1}{c}{$\bbE_M [\hat{\rho} - \rho]$} &
\multicolumn{1}{c}{$std_M [\hat{\nu}]$} &
\multicolumn{1}{c}{$std_M [\hat{\rho}]$} 
 \\ 
 \hline
64000 & 3 & 11 & 11 & -1.92e-04 &  4.52e-04 & 1.36e-02 & 8.17e-03 \\ 
64000 & 3 & 11 & 10 &  1.17e-03 & -5.90e-04 & 7.08e-03 & 4.04e-03 \\ 
%\hdashline
128000 & 3 & 12 & 12 & -2.51e-03 & 1.81e-03 & 8.54e-03 & 6.11e-03 \\ 
128000 & 3 & 12 & 11 & -6.90e-04 & 5.02e-04 & 4.17e-03 & 2.84e-03 \\ 
\end{tabular}
\\
\bigskip
(b) TD, kD tree, n-Sphere, $d = 10$, $M = 100$, $\nu =
  3/4$, $\rho = 3/4$, $\tau = 1 \times 10^{-5}$
\begin{tabular}{ r r r r r r r r r r}
\multicolumn{1}{c}{$N$} & 
% \multicolumn{1}{c}{$d$} &
\multicolumn{1}{c}{$w$} &
\multicolumn{1}{c}{$t$} &
\multicolumn{1}{c}{$i$} & 
\multicolumn{1}{c}{$\bbE_M [\hat{\nu} - \nu]$} &
\multicolumn{1}{c}{$\bbE_M [\hat{\rho} - \rho]$} &
\multicolumn{1}{c}{$std_M [\hat{\nu}]$} &
\multicolumn{1}{c}{$std_M [\hat{\rho}]$} 
 \\ 
 \hline
64000 & 4 & 5 & 5 &  8.70e-03 & -1.12e-02 & 1.55e-02 & 1.85e-02 \\ 
64000 & 4 & 5 & 4 & -9.31e-02 &  8.02e-02 & 1.67e-02 & 1.97e-02 \\
%\hdashline
128000 & 4 & 6 & 6 & -6.36e-03 & 5.51e-03 & 2.10e-02 & 1.72e-02 \\
128000 & 4 & 6 & 5 & -7.18e-02 & 6.27e-02 & 1.32e-02 & 1.46e-02 \\ 
\end{tabular}
\\
\bigskip
%% (c) TD, kD tree, n-Sphere, $d = 10$, $M = 100$, $\nu =
%%   1.25$, $\rho = 1$, $\tau = 10^{-7}$
%% \begin{tabular}{ r r r r r r r r r r}
%% \multicolumn{1}{c}{$N$} & 
%% % \multicolumn{1}{c}{$d$} &
%% \multicolumn{1}{c}{$w$} &
%% \multicolumn{1}{c}{$t$} &
%% \multicolumn{1}{c}{$i$} & 
%% \multicolumn{1}{c}{$\bbE_M [\hat{\nu} - \nu]$} &
%% \multicolumn{1}{c}{$\bbE_M [\hat{\rho} - \rho]$} &
%% \multicolumn{1}{c}{$std_M [\hat{\nu}]$} &
%% \multicolumn{1}{c}{$std_M [\hat{\rho}]$} 
%%  \\ 
%%  \hline
%%  64000 & 2 &  9 & 9 & -5.86e-03 & 5.21e-03 & 4.85e-02 & 3.15e-02 \\
%%  64000 & 2 &  9 & 8 & -3.37e-02 & 1.93e-02 & 3.50e-02 & 2.46e-02 \\ 
%%  64000 & 2 &  9 & 7 & -1.19e-01 & 6.92e-02 & 2.88e-02 & 2.51e-02 \\ 
%% %\hdashline
%% 128000 & 2 & 10 & 10 & -2.70e-03 & 2.74e-03 & 3.76e-02 & 2.44e-02 \\ 
%% 128000 & 2 & 10 &  9 & -2.00e-02 & 1.20e-02 & 2.47e-02 & 1.80e-02 \\ 
%% 128000 & 2 & 10 &  8 & -8.40e-02 & 5.03e-02 & 2.21e-02 & 1.89e-02 \\ 
%% \end{tabular}
\end{center}
\label{numericalresults:table2}
\end{table*}

%INFINITUMmpfg
%a5c349c2d6


\subsection{Prediction}

In this the computational performance of the multilevel Kriging solver
is analyzed. Given a fixed Mat\'{e}rn parameters $(\nu,\rho)$ the
objective is to compute the BLUP vectors $\hat \bgamma$ and $\hat
\beta$. This involves solving the system of equations $\bP^{-1}_{\bW}
\bC_{\bW}(\btheta) \bgamma_{\bW} = \bP^{-1}_{\bW} \bZ_{\bW}$ and $\hat
\bbeta = (\bX\T \bX)^{-1}$ $\bX\T(\bZ - \bC \hat \bgamma)$.

Numerical results for computing $\hat \bgamma$ and $\hat \bbeta$ for the
hypercube data set with $d = 3$ dimensions, kD tree, and the Total
Degree index set $\mcQ^d_w$ are shown in Table
\ref{numericalresults:table3}. The Mat\'{e}rn covariance coefficients
$\btheta = (\nu,\rho)$ are set to (3/4,1). The relative error of the
residual of PCG method for the unpreconditioned system is set to
$\varepsilon = 10^{-3}$. The KIFMM is set to high accuracy.

For computing the matrix vector products of the PCG iterations, the
computational break even point of the KIFMM solver is reached for $N
\approx 2,500$ compared to using the direct approach (with CPU and
GPU). The increase in computational complexity is linear with respect
to $N$. Thus all the matrix vector products for the PCG iterations are
calculated using the KIFMM.

The preconditioner $\bP_{\bW}$ is built using a combination of the GPU
and CPU. This leads to a quadratic increase in computational cost with
respect to the number of observations $N$. However, due to the high
efficiency of the implementation and $p = 120$, the break even point
for the use of the KIFMM solver is not reached, even for $N = 512,000$
observation points.

From Table \ref{numericalresults:table3} observe that condition number
of the covariance matrix $\bC$ is much larger compared to
$\bC_{\bW}$. This is already a good indication that solving the
Kriging problem will be more efficient using the multilevel approach.

The number of iterations needed to reach the same accuracy for both
approaches are significantly better with the multilevel approach
i.e. $\approx 70$ times less iterations. However, the computation of
$\bbeta$ with the single level method requires solving $p + 1 = 121$
matrix inversions of $\bC$. This is in contrast with a single matrix
inversion of $\bC_{\bW}$ with the multilevel method. In practice, we
did not solve all $p+1$ matrix inversions for the single level
approach, but measure the time required to compute a single matrix
inversion and multiplied it 121 to obtain the estimated time
complexity.  For $N = 64,000$ observations we observe efficiencies of
$\approx 7,000$ compared to the single level iterative approach.

The condition number of the covariance matrices are fairly large,
making this problem somewhat hard to solve numerically.  The results
show that 512,000 size problems with good accuracy are feasible with a
single 4-core processor and GPU. Finally, the total computational cost
varies somewhere between linear and quadratic as the number of
observations $N$ is increased.


\begin{table*}[htbp]
  \caption{Numerical results for computing $\bP^{-1}_{\bW}
    \bC_{\bW}(\btheta) \bgamma_{\bW} = \bP^{-1}_{\bW} \bZ_{\bW}$ and
    $\hat \bbeta = (\bX\T \bX)^{-1} \bX\T(\bZ - \bC \hat \bgamma)$ for
    the hypercube data set with $d = 3$ and the Total Degree index set
    $Q^d_w$. The Mat\'{e}rn covariance coefficients $\btheta =
    (\nu,\rho)$ are set to (3/4,1). The relative error of the residual
    of PCG method for the \emph{unpreconditioned system} is set to
    $\varepsilon = 10^{-3}$. The KIFMM is set to high accuracy. (a)
    The second column of is the condition number of the covariance
    matrix $\bC$, up to $N=64,000$ observations, and is compared with
    the fourth column which corresponds to the condition number of
    $\bC_W$. The third column (itr($\bC_{\bW}$)) is the number of CG
    iterations needed for convergence for $10^{-3}$ residual
    accuracy. The fifth column is the number of iterations need to
    achieve the residual error $10^{-3}$ for the unpreconditioned
    system with the preconditioner $\bP_{\bW}$.  (b) The second column
    corresponds to the wall clock times in seconds for the
    preconditioner computation. The third column is the time for
    construction of the preconditioner $\bP_{\bW}$.  The PCG iteration
    wall clock timings for $\bC_{\bW}$, by using a KIFMM, are given in
    the fourth column. The fifth is the total time to compute
    $\bgamma_{\bW}$, $\bbeta$ and the multilevel basis
    construction. The sixth column is the computational efficiency for
    computing $\bgamma_{\bW}$ vs $\bC^{-1} \bZ$ to same residual
    accuracy with respect to the number of iterations. The last column
    is the estimated efficiency of computing $\hat \bgamma$ and $\hat
    \bbeta$ with the multilevel BLUP compared to the single level
    approach, equation \eqref{Kriging}, to approximately the same
    accuracy using a CG iteration with the KIFMM. We observe the
    significant speed ups ($\approx 7,000$ for $N = 64,000$) for
    calculating the BLUP by using the multilevel approach.
  %We observe that the total computational cost varies between linear
  %and somewhere between linear and quadratic as the number of
  %observations $N$ is increased.  The last column shows the efficiency
  %for computing itr($\bC_{\bW}$) with respect to $\bC$. Notice that as
  %$N$ increases the efficiency of the multilevel method increases
  %significantly.
  %The last column shows the efficiency for computing itr($\bC_{\bW}$
  %with respect to $\bC$. Notice that as $N$ increases the efficiency
  %of the multilevel method increases significantly.
}
\begin{center}
  \bigskip
  (a) $\btheta = (3/4,1)$, $d = 3$, $w = 7$ ($p = 120$) \\
  \bigskip
\begin{tabular} { r c r c r}
  \multicolumn{1}{c}{$N$} &
  \multicolumn{1}{c}{$\kappa(\bC)$} &
    \multicolumn{1}{c}{itr($\bC$)} &
  \multicolumn{1}{c}{$\kappa(\bC_{\bW})$} &
  itr($\bC_{\bW}$) \\
  \hline
  8,000  & $3.2 \times 10^{7}$   &    1,985 & $1.8 \times 10^{4}$ &   52  \\  
  16,000  & $1.1 \times 10^{8}$  &    3,511 & $6.0 \times 10^{4}$ &   67  \\  
  32,000  & $5.6 \times 10^{8}$  &    8,259 & $3.1 \times 10^{5}$ &  116  \\  
  64,000  &  $1.8 \times 10^{9}$ &   12,680 & $9.5 \times 10^{5}$  & 165  \\  
  128,000 &                 -   &      -  &                  -  &  308   \\  
  256,000 &                 -   &      -  &                  -  &  292   \\  
  512,000 &                 -   &      -  &                  -  &  484   \\  
\end{tabular}
\\
\bigskip
(b) $\btheta = (3/4,1)$, $d = 3$, $w = 7$ ($p = 120$) \\
\bigskip
\begin{tabular} { r r r r r c r}    
  \multicolumn{1}{c}{$N$} 
  & itr($\bC_{\bW}$) & $\bP_{\bW}$ (s) & Itr (s) & Total (s) & Eff$_{\bgamma}$ &
  \multicolumn{1}{c}{Eff$_{\bgamma,\bbeta}$} \\
  \hline
  8,000   &   52  &       4   &      29  &     38 &  38 &   3,600  \\
  16,000  &   67  &      13   &      98  &    118 &  52 &   5,000  \\
  32,000  &  116  &      45   &     260  &    317 &  71 &   7,250  \\
  64,000  &  165  &     178   &     798  &    997 &  76 &   7,380  \\
  128,000 &  308  &     713   &   3,934  &  4,687 &   - &    - \\
  256,000 &  292  &   2,837   &   5,745  &  8,663 &   - &    - \\
  512,000 &  484  &   11,392  &  20,637  & 32,202 &   - &    - \\
\end{tabular}
%% \bigskip
%% (b) $\btheta = (3/4,1)$, $d = 3$, $w = 7$ ($p = 120$) \\
%% \bigskip
%% \begin{tabular} { r r r r r c r}    
%%   \multicolumn{1}{c}{$N$} 
%%   & itr($\bC_{\bW}$) & $\bP_{\bW}$ (s) & Itr (s) & Total (s) & Eff$_{\bgamma}$ &
%%   \multicolumn{1}{c}{Eff$_{\bgamma,\bbeta}$} \\
%%   \hline
%%   8,000   &   53  &       3   &      18  &     21 &  48 &  5,808 \\
%%   16,000  &   70  &      11   &      43  &     54 &  89 & 10,769 \\
%%   32,000  &  116  &      45   &     265  &    310 &  58 &  7,018 \\
%%   64,000  &  165  &     178   &     798  &    976 &  76 &  9,196\\
%%   128,000 &  308  &     713   &   3,934  &  4,647 &   - &    - \\
%%   256,000 &  292  &   2,837   &   5,745  &  8,582 &   - &    - \\
%%   512,000 &  484  &   11,392  &  20,319  & 31,711 &   - &    - \\
%% \end{tabular}
\end{center}
\label{numericalresults:table3}
\end{table*}


The multilevel approach is now tested on $d = 20$ and $d = 25$
dimensional problems. Due to the high dimensionality of these
problems, a fast summation approach is not an option. The
matrix-vector products of each iteration are computed with the direct
approach using the GPU and CPU.

In Table \ref{numericalresults:table4}(a) the numerical results for
computing $\bgamma$ and $\bbeta$ for $d = 20$ and $\theta =(5/4,10)$.
Compared to the single level iterative approach the multilevel method
is approximately 42,000 faster for $N = 64,000$ observations. Similar
results are obtained shown in Table \ref{numericalresults:table4}(b). 
for $d = 25$ and $\theta =(5/4,10)$.

\setlength{\tabcolsep}{3pt}

\begin{table*}[htbp]
  \caption{Computing Kriging for the n-sphere data set with $d = 20$
    and $d = 25$ dimensions, TD index set, and Mat\'{e}rn covariance
    function without pre-conditioner. The residual accuracy is set to
    $\varepsilon = 10^{-3}$. Since the dimension is greater than 3,
    the matrix vector products are computed directly with the GPU and
    CPU.  The description of the columns of tables (a) and (b) are the
    same as for Table \ref{numericalresults:table3}. In addition,
    column 6 corresponds to the wall clock time for computing the
    multilevel basis. (a) Computational times for solving the Kriging
    prediction for $d = 20$ and $\theta = (5/4,10)$.  The growth in
    computational cost is slightly faster than quadratic due to the
    lack of fast summation method in higher dimensions. However,
    compared to the single level iterative approach it is
    approximately 42,000 faster for $N = 64,000$ observations. (b)
    Kriging prediction for $d = 25$ and $\theta = (5/4,10)$. The
    growth in computational cost is similar.  The efficiency of this
    method is about 2,840 times faster for $N = 64,000$ observations.}
\begin{center}
(a) $\btheta = (\nu,\rho) = (5/4,10)$, $d = 20$, $w = 3$ ($p =
1771$), No precond., Direct \\
\begin{tabular} { r c c c  c r r r r r}
  \multicolumn{1}{c}{$N$} & $\kappa (\bC)$ & $\kappa (\bC_{\bW})$ &
 itr($\bC$)
  & 
 itr($\bC_{\bW}$) &  MB(s) & Itr(s) & Total(s) &  Eff$_{\bgamma,\bbeta}$ \\
  \hline
 16,000  & $5  \times 10^{7}$ & 7  & 238 & 10  &  52  &      97   &    153 & 26,700   \\
 32,000  & $1 \times 10^{8}$  & 11 & 324 & 13  & 121  &     500   &    628 & 35,160   \\
 64,000  & $2 \times 10^{8}$  & 17 & 444 & 17  & 284  &   2,600   &  2,898 & 42,050 \\
128,000  &  -                & -  &  - & 22  & 628  &  13,494   &  14,153 & -  \\
\end{tabular}\\
\bigskip
(b) $\btheta = (\nu,\rho) = (3/4,10)$, $d = 25$, $w = 2$ ($p = 351$), No precond., Direct \\
\begin{tabular} { r c c c c r r r r}
  \multicolumn{1}{c}{$N$} & $\kappa (\bC)$ & $\kappa (\bC_{\bW})$
  & itr($\bC$)
  & itr($\bC_{\bW}$) & MB(s) & Itr(s) & Total(s) & Eff$_{\bgamma,\bbeta}$ \\
  \hline
 16,000  & $2  \times 10^{6}$  &   7  & 86  & 12 &   5  &         116   &    122 & 2,400   \\
 32,000  &  $4   \times 10^{6}$ &  12 & 109 & 15 &  13  &         582   &    599 & 2,490 \\
 64,000  &  $9  \times 10^{6}$ &   21 & 147  & 18 &  30  &       2,788  &  2,821 & 2,840 \\
 128,000  &  -                  &  - & -  & 25 &  79  &      15,557   & 15,641 &  - \\
 256,000  &  -                  &  - & -  & 33 & 157  &      83,163   & 83,337 &  - \\
\end{tabular}\\
\end{center}
\label{numericalresults:table4}
\end{table*}

\section{Conclusions}

In this paper a multilevel Kriging method is developed that scales
well with high dimensions. A multilevel basis is constructed from a
kD-tree and for the choice of Total Degree polynomial basis
$\mcQ^d_w$.  The approach described in the paper has the following
characteristics and advantages:

\begin{enumerate}[i)]

\item The multilevel method is numerically stable. Hard estimation and
  prediction of large dimensional problems are now feasible.
  
\item The method is efficiently implemented by using a combination of
  MATLAB, c++ software packages and dynamic libraries.

\item Sub-exponential decay of multilevel covariance matrix
  $\bC_{\bW}$ is proven based on complex analytic extensions.
  
\item Numerical results of up to 25 dimensional problems. These
  problems are difficult to solve with traditional methods due to the
  large condition numbers, but feasible with the multilevel method.

\item The multilevel prediction approach is proven to be \emph{exact},
  in the sense that single level and multilevel prediction
  formulations are shown to be equivalent. 
   

\item The efficiency of this approach will be further improved as high
  dimensional fast summation methods are developed.

\item An A-posteriori scheme and estimates for constructing the sparse
  covariance matrix $\tilde \bC$ will be developed in a future
  paper. This will be possible with the error bounds for the entries
  of $\bC$ derived in this paper since all the constants can be
  estimated.

  
\end{enumerate}














\section*{Appendix: Polynomial Interpolation}
\label{PolynomialAppendix}

In this section we provide some background on polynomial interpolation
in high dimensions. This will be critical to estimate the decay rates
of the entries of the multilevel covariance matrix for high
dimensional problems.
%Note that this appendix can be
%  skipped as it is only used for estimating the decay of the
%  multilevel covariance matrix.}

The decay of the coefficients will directly depend on the analytic
properties of the covariance function. The traditional error estimates
of polynomial interpolation are based on multi-variate $m^{th}$ order
derivatives. However, for many cases, such as the Mat\'{e}rn
covariance function, the derivatives are too complex or expensive to
manipulate for even a moderate number of dimensions. This motivates
the study of polynomial numerical approximations based on complex
analytic extensions, which are much better suited for high dimensions.
Much of the discussion that follows has it roots in the field of
uncertainty quantification and high dimensional interpolation
\cite{nobile2008a,Castrillon2016,Griebel2016}
for partial differential
equations.


Consider the problem of approximating a function $v: \Gamma^{d}
\rightarrow \R$ on the domain $\Gamma^{d}$.  Without loss of
generality let $\Gamma : = [-1, 1]$ and $\Gamma^{d} := \prod_{n =
  1}^{d} \Gamma$. Suppose that $\mcG \subset \Gamma^{d}$, then define
the following spaces
\[
\begin{split}
  &
L^q(\mcG) := \{ v(\by)\, | \, \int_{\mcG} v(\by)^q \text{d}
\by < \infty  \}
\,\,\,
\mbox{and} \\
&
L^{\infty}(\mcG) := \{ v(\by)\, | \, \sup_{\by \in \mcG} |v(\by)|
< \infty  \}.
\end{split}
\]


Suppose that $\mcP_{ q}(\Gamma):=\text{\rm span}\{y^k,\,k=0,\dots,q\}$
i.e. the space of polynomials of degree at most $q$. Let $\mcI^{m} :
C^{0}(\Gamma) \rightarrow \mcP_{m-1}(\Gamma)$ be the univariate
Lagrange interpolant
\[
\mcI_{m}(v(\by)):=
\sum_{k=1}^{m}v(y^{(k)})l_{m,k}(y^{(k)}),
\]
where $y^{(1)}, \dots, y^{(m)}$ is a set of distinct knots on $\Gamma$
and $\{ l_{n,k} \}_{k=0}^{m}$ is a Lagrange basis of the space
$\mcP_{m-1}(\Gamma)$. The variable $m \in \Nset$
%, where $\Nset_{+} := \Nset \cup 0$,
corresponds to the order of approximation of the
Lagrange interpolant. However, for the case of the zero order
interpolation $m = 0$ corresponds to $\mcI_{0} = 0$.


\begin{remark}
For high dimensional interpolation the particular set of points
$y^{(1)}, \dots, y^{(m)}$ that we will use is the Clenshaw-Curtis
abscissas.  This is further discussed in this section. However, for
now, we assume that the points are only distinct.
  \end{remark}


For $m \geq 1$ let
\[
\Delta_{m}
:= \mcI_{m}-\mcI_{m-1},
\]
From the difference operator $\Delta_{m}$ we can readily observe that
$\mcI_{m} = \sum_{k=1}^{m} \Delta_{k}$, which is reminiscent of multi
resolution wavelet decompositions. The idea is to represent
multivariate approximation as a summation of the difference operators.

Consider the multi-index tupple $\bm = (m_1,\dots,m_d)$, where $\bm
\in \Nset^{d}$, and form the tensor product operator
$\mcS_{w,d}: \Gamma \rightarrow \R$ as
\begin{equation}
  \mcS_{w,d}
      [v(\by)]
      :
      =
 \sum_{\bm \in \bbNset^{d}: \sum_{i=1}^{d} m_i - 1  \leq w } \;\;
 \bigotimes_{n=1}^{d} {\Delta^{n}_{m_n}}(v(\by)).
\label{errorestimates:SG}
\end{equation}
Note that by ${\Delta^{n}_{m_n}}(v(\by))$ we mean that the difference
operator ${\Delta_{m_n}}$ is applied along the $n^{th}$ dimension in
$\Gamma$.


Let $C^{0}(\Gamma_d; \R) : = \{ v: \Gamma_d \rightarrow \R\,\,$ is
continuous on $\Gamma_d$ and $\max_{\by\in \Gamma_d} |v(\by)| < \infty
\}$.  From Proposition 1 in \cite{Back2011} it is shown that for any
$v \in C^0(\Gamma_d;\R)$, we have $\mcS_{w,d}[v]\in \mcQ^{d}_{w}$.
Moreover, $\mcS_{w,d}[v] = v$, for all $v \in \mcQ^{d}_{w}$. The key
observation to take away is that the operator $\mcS_{w,d}[v]$ is
\textit{exact} in the space of polynomials $\mcQ^{d}_{w}$. This will
be useful in connecting the Lagrange interpolant with Chebyshev
polynomials.


Let $T_k:\Gamma \rightarrow \R$, $k = 0, 1, \dots$, be a Chebyshev
polynomial over $\Gamma$, which are defined recursively as follows:
$T_0(y) = 1$, $T_1(y) = y$, $\dots$, $T_{k+1}(y) = 2yT_{k}(y) -
T_{k-1}(y)$, $\dots$, where $y \in \Gamma$. Chebyshev polynomials are
well suited for the approximation of functions with analytic
extensions on a complex region bounded by a Bernstein ellipse. They
bypassing the need of using derivative information and sharp bounds on
the error are readily available. Suppose that $\sigma > 0$ and denote
by
\[
\begin{split}
  \mcE_{\sigma} := \Big\{
  z \in \bbC, \sigma \geq
\delta \geq 0 ;\,\Real{z} = \frac{e^{\delta} + e^{-\delta}
}{2}cos(\theta) 
\Imag{z} = \frac{e^{\delta} 
  - e^{-\delta}}{2}sin(\theta),
\theta \in [0,2\pi)
  \Big\}
\end{split}
  \]
as the region bounded by a Bernstein ellipse (see Figure
\ref{erroranalysis:sparsegrid:polyellipse}).

The following theorem is based on complex analytic extensions on
$\mcE_{\sigma}$ and provides a control for the Chebyshev polynomial
approximation.

\begin{theorem}
Suppose that for $u:\Gamma \rightarrow \R$ there exists an analytic
extension on $\mcE_{\sigma}$. If $|u| \leq M < \infty$ on
$\mcE_{\sigma}$ then there exists a sequence of coefficients
$|\alpha_k| \leq M / e^{k\sigma}$ such that $u \equiv \alpha_0 +
2\sum_{k = 1}^{\infty} \alpha_{k} T_{k}$ on $\mcE_{\sigma}$. Moreover,
if $y \in \Gamma$ then
\[
%\begin{multline*}
%\shoveright{|q(y) - \alpha_0  - 2\sum_{k = 1}^{n} \alpha_{k} T_{k}(y)|
%\leq 
%\frac{2M}{e^{\sigma} - 1} e^{-n \sigma}.}
|q(y) - \alpha_0  - 2\sum_{k = 1}^{n} \alpha_{k} T_{k}(y)|
\leq 
\frac{2M}{e^{\sigma} - 1} e^{-n \sigma}.
%\end{multline*}
\]
\label{errorestimates:theorem}
\end{theorem}
\begin{proof}
See Theorem 2.25 in \cite{Khoromskij2018}
\end{proof}


\begin{figure}[htb]%[12]{r}{7cm}%[htp]
\begin{center}
\begin{tikzpicture}
    \begin{scope}[font=\scriptsize]

      
      \filldraw[fill=blue!20,
      semitransparent] (0,0) ellipse (2 and 1);

    \draw [->] (-2.5, 0) -- (2.5, 0) node [below left]  {$\Real $};
    \draw [->] (0,-1.5) -- (0,1.5) node [below left] {$\Imag$};
    \draw (1,-3pt) -- (1,3pt)   node [above] {$1$};
    \draw (-1,-3pt) -- (-1,3pt) node [above] {$-1$};
    \end{scope}
    
    \node [below right] at (-2.5,1.25) {$\mcE_{\sigma}$};

    \node [] at (0.75,1.25) {$\frac{e^{
          \sigma} - e^{- \sigma}}{2}$};

    
    \node [] at (2.75,0.25) {$\frac{e^{
      \sigma} + e^{- \sigma}}{2}$}; 
    
\end{tikzpicture}
\end{center}
\caption{Complex region bounded by the Bernstein ellipse.}
\label{erroranalysis:sparsegrid:polyellipse}
\end{figure}

We can now connect the error due to the Lagrange interpolation with
Chebyshev expansions. It is known that if $u \in C(\Gamma,\R)$ then
\[
\|(I - \mcI_{m})u\|_{L^{\infty}(\Gamma)} \leq
(1 + \Lambda_{m})
\min_{h \in \mcP_{m-1}} \| u - h \|_{L^{\infty}(\Gamma)},
\]
where $\Lambda_{m}$ is the Lebesgue constant (See Lemma 7 in
\cite{babusk_nobile_temp_10}). Note that $I:C^{d}(\Xi;\R) \rightarrow
C^{d}(\Xi;\R)$ refers to the identity operator and the domain $\Xi$ is
taken from context. For the previous case $\Xi = \Gamma$.  Bounds on
$\Lambda_{m}$ are known in the context of the location of the knots
$y^{(1)}, \dots, y^{(m)} \in \Gamma$. In this article we restrict our
attention to Clenshaw-Curtis abscissas
%\[
\[
y^{(j)} = -\cos \left( \frac{\pi(j-1)}{m - 1} \right),\,\, j =
1,\dots, m
\]
%\]
and $\Lambda_m$ is bounded by $2\pi^{-1}(\log{(m-1)} + 1) \leq 2m - 1$
(see \cite{babusk_nobile_temp_10}).  Since the interpolation operator
$\mcI_{m}$ is exact on $\mcP_{m - 1}$, then if $u:\Gamma \rightarrow
\R$ has an analytic extension in $\mcE_{\sigma}$ we have from Theorem
\ref{errorestimates:theorem} (following a similar approach as in
\cite{babusk_nobile_temp_10}) that
\[
\begin{split}
\|(I - \mcI_{m})u\|_{L^{\infty}(\Gamma_n)}
\leq
(1 + \Lambda_{m})
\frac{2M}{e^{\sigma} - 1} e^{-\sigma (m-1)}
\leq 
2 C(M,\sigma) m e^{-\sigma (m-1)},
\end{split}
\]
where $C(M,\sigma_n) := \frac{2M}{(e^{ \sigma} - 1)}$. We then
conclude that for all $k = 1,\dots, m$
\begin{equation}
\begin{split}
\| \Delta_{k}(u) \|_{L^{\infty}(\Gamma)} 
&=
\|
\mcI^{m}(u) - \mcI^{m-1}(u)
\|_{L^{\infty}(\Gamma)} 
\leq
\|(I - \mcI_{m})u\|_{L^{\infty}(\Gamma)}
+
\|(I - \mcI_{m-1})u\|_{L^{\infty}(\Gamma)} \\
&\leq
e^{2\sigma}C(M,\sigma) m e^{-\sigma m}.
\end{split}
\label{interpolation:eqn1}
\end{equation}
Let $\mcE_{\sigma,n} \subset \bbC^{d}$ a complex region bounded by a
Bernstein ellipse such that the restriction on $\Gamma_{d}$ is along
the $n^{th}$ dimension and form the polyellipse $\mcE^{d}_{\sigma}:=
\prod_{n=1}^{d} \mcE_{\sigma,n}$.  Suppose that $v:\mcE^{d}_{\sigma}
\rightarrow \bbC$ is analytic on $\mcE^{d}_{\sigma}$ and let
$\tilde{M}(v) := \max_{\bz \in \mcE^{d}_{\sigma}} |v(\bz)|$.

Note we refer to $\mcI^{n}_{m}$ as the Lagrange operator of order $m$
along the $n^{th}$ dimension and similarly $\mcP^{n}_{m-1}$ is the
space of the span of univariate polynomials up to degree $m-1$ along
the $n^{th}$ dimension.  Form the tensor product $\bI^{d}_{m} :=
\mcI^{1}_{m} \times \dots \times \mcI^{d}_{m}$, thus $\bI:C(\Gamma,\R)
\rightarrow \bbP$ where $\bbP := \mcP^{1}_{m-1} \times \dots \times
\mcP^{d}_{m-1}$. From Theorem 2.27 in \cite{Khoromskij2018} we can
conclude that for a finite dimension $d$, as $m \rightarrow \infty$
then $\bI^{d}_{m}[v] \rightarrow v$.

Applying equation \eqref{interpolation:eqn1} to equation
\eqref{errorestimates:SG} we have that
\begin{equation}
\begin{split}
& \| (I - \mcS_{w,d})
 v(\by)
 \|_{L^{\infty}(\Gamma^{d})}
 \leq
 \left\| \sum_{\bm \in \bbNset^{d}: \sum_{i=1}^{d} m_i - 1 > w } \;\;
 \bigotimes_{n=1}^{d} {\Delta^{n}_{m_n}}(v(\by))\right\|_{L^{\infty}(\Gamma^d)} \\
 &\leq
 \sum_{\bm \in \bbNset^{d}: \sum_{i=1}^{d} m_i - 1 > w } \;\;
 \bigotimes_{n=1}^{d} \|{\Delta^{n}_{m_n}}(v(\by))\|_{L^{\infty}(\Gamma^d)} 
 \leq
 \sum_{\bm \in \bbNset^{d}: \sum_{i=1}^{d} m_i - 1 > w }
 e^{2d} C(M,\sigma)^{d} \\
 &
 \left( \prod_{n=1}^{d} m_n\right) \exp{\left( -\sum_{n=1}^{d}
   \sigma m_{n} \right)}.
% \\
%  &\leq
% \sum_{\bk \in \bbNset^{d}_{0}: \sum_{i=1}^{d} k_i > w }
% e^{2d} C(M,\sigma)^{d} \left( \prod_{n=1}^{d} (k_n + 1)\right)
% \exp{\left( -\sum_{n=1}^{d}
%   \sigma (k_{n}+1) \right)}.
\end{split}
\label{interpolation:eqn2}
\end{equation}

By applying Theorem 2.10 and Corollary 2.11 in \cite{Griebel2016} if
$ w \geq  d$ and $p( d, w) \geq
\left(\frac{2  d}{\kappa( d)}\right)^{ d}$, where
$\kappa( d) := \sqrt[\leftroot{-2}\uproot{2}  d]{
  d!} >  d/e$ (Sterling approximation), then for any $\hat
\sigma \in \R_{+}$
\begin{equation}
\begin{split}
 & \sum_{\bk \in \bbNset^{ d}_{0}: \sum_{i=1}^{ d} k_i  >  w }
 \exp{\left( -\sum_{n=1}^{ d} \hat \sigma
   k_{n} \right)}
 \leq
 \sum_{\bk \in \bbNset^{d}_{0}: \hat \sigma \sum_{i=1}^{ d} k_i  \geq  w \hat \sigma  }
 \exp{\left( -\sum_{n=1}^{ d}
   \hat \sigma k_{n} \right)} \\
 &\leq
 \hat \sigma  d e
 \left( \frac{e^{\hat \sigma}}{1 - e^{-\hat \sigma}} \right)^{ d}
 \exp \left(-\frac{ d}{e} \hat \sigma  p^{\frac{1}{ d}}
 \right) p^{\frac{ d-1}{ d}}.
\end{split}
\label{interpolation:eqn3}
\end{equation}
where $\bk \in \bbNset^{d}_{0}$ and $\bk:=(k_1,\dots,k_d)$.






Following the same approach as in \cite{Griebel2016} observe that for
$0 < \delta < 1$ we can obtain a bounded constant $c_{n,\delta} \leq
(e\sigma \delta)^{-1}$ such that $m_n \exp(-\sigma m_n) \leq (e\sigma
\delta)^{-1}$ $\exp(-\sigma m_n (1 - \delta))$. Set $\hat \sigma :=
\sigma (1 - \delta)$ and by combining equations
\eqref{interpolation:eqn2} and \eqref{interpolation:eqn3} we have
proven the following result.

\begin{lemma} Suppose that $0< \delta < 1$, $\hat
  \sigma := \sigma (1 - \delta)$, and $p(d,w) \geq \left(\frac{2
    d}{\kappa(d)}\right)^{d}$ then
  \[
  \begin{split}
 &\| (I - \mcS_{w,d})
 v(\by)
 \|_{L^{\infty}(\Gamma^{d})}
 \leq 
 \frac{C(\tilde M,\sigma)^d e^{d - \sigma(1 - \delta) + 1} \hat \sigma d }
 {
(\sigma \delta)^{d}}
 \left( \frac{e^{\hat \sigma}}{1 - e^{-\hat \sigma}} \right)^{d} 
 \exp \left(-\frac{d}{e} \hat \sigma  p^{\frac{1}{d}}
 \right) p^{\frac{d-1}{d}}.
 \end{split}
 \]
 \label{interpolation:lemma1}
\end{lemma}


\begin{remark}
The restriction $p(d,w) \geq \left(\frac{2
  d}{\kappa(d)}\right)^{d}$ is not strict and can be relaxed such that
sub-exponential convergence is still obtained.  We refer the reader to
the bound of the Gamma function in Lemma 2.5 (\cite{Griebel2016}) and
it's application in the proofs of Theorem 2.10 and Corollary 2.11.
\label{interpolation:remark1}
\end{remark}




\noindent 
\textbf{Acknowledgements:} I appreciate the help and advice from
George Biros and Lexing Ying, for setting up the KIFMM packages.  I
also appreciate the support that King Abdullah University of Science
and Technology has provided to this project.




\bibliographystyle{abrev}
%%%%%%%%%%%%%%%%%%%%%%%%%%%%%%%%%%%%%%%%%%%%%
%
% Latex file for:
% 
% 
%%%%%%%%%%%%%%%%%%%%%%%%%%%%%%%%%%%%%%%%%%%%%

%\RequirePackage{fix-cm}
\documentclass[11pt,final]{amsart}       % one column (second format)

\usepackage{epsfig, epsf, graphicx, float, color}
\usepackage{pstricks, psfrag}
\usepackage{amssymb,amsthm}
\usepackage[foot]{amsaddr}
\usepackage{verbatim,enumerate,hyperref}
\usepackage{setspace,mathtools,wrapfig}
\usepackage[numbers,sort&compress]{natbib}
\usepackage{algorithm2e}
\usepackage{steinmetz}
\usepackage{tikz-qtree,tikz-qtree-compat}
\usepackage{tikz,pgf}
\usepackage{arydshln}
\usepackage[margin=1in]{geometry}

\usetikzlibrary{decorations,decorations.markings,decorations.text}
\usetikzlibrary{arrows.meta}
\usetikzlibrary{patterns}


\usetikzlibrary{positioning,patterns}
\usetikzlibrary{calc,fadings,decorations.pathreplacing,arrows,datavisualization.formats.functions,shapes.geometric}




%\smartqed % flush right qed marks, e.g. at end of proof

% Macros

\def\dfrac#1#2{{\displaystyle{#1\over#2}}}
\def\VS{{\vskip 3mm\noindent}}
\def\boxit#1{\vbox{\hrule\hbox{\vrule\kern6pt
          \vbox{\kern6pt#1\kern6pt}\kern6pt\vrule}\hrule}}
\def\refhg{\hangindent=20pt\hangafter=1}
\def\refmark{\par\vskip 2mm\noindent\refhg}
\def\naive{\hbox{naive}}
\def\itemitem{\par\indent \hangindent2\pahttprindent \textindent}
\def\var{\hbox{var}}
\def\cov{\hbox{cov}}
\def\corr{\hbox{corr}}
\def\trace{\hbox{trace}}
\def\refhg{\hangindent=20pt\hangafter=1}
\def\refmark{\par\vskip 2mm\noindent\refhg}
\def\Normal{\hbox{Normal}}
\def\povr{\buildrel p\over\longrightarrow}
\def\ccdot{{\bullet}}
\def\bse{\begin{eqnarray*}}
\def\ese{\end{eqnarray*}}
\def\be{\begin{eqnarray}}
\def\ee{\end{eqnarray}}
\def\bq{\begin{equation}}
\def\eq{\end{equation}}
\def\bse{\begin{eqnarray*}}
\def\ese{\end{eqnarray*}}
\def\pr{\hbox{pr}}
\def\CV{\hbox{CV}}
\def\wh{\widehat}
\def\T{^{\rm T}}
\def\myalpha{{\cal A}}
\def\th{^{th}}

% Color corrections in text
\newcommand{\corb}[1]{\textcolor{blue}{#1}}
\newcommand{\corred}[1]{\textcolor{black}{#1}}
\newcommand{\corblue}[1]{\textcolor{black}{#1}}

%\renewcommand{\baselinestretch}{1} % Change this 1.5 or whatever
%\newcommand{\qed}{\hfill\hfill\vbox{\hrule\hbox{\vrule\squarebox
%   {.667em}\vrule}\hrule}\smallskip}


\newcommand{\bR}{\mathbf{R}}
\newcommand{\bD}{\mathbf{D}}
\newcommand{\bI}{\mathbf{I}}
\newcommand{\bL}{\mathbf{L}}
\newcommand{\bG}{\mathbf{G}}
\newcommand{\bW}{\mathbf{W}}
\newcommand{\bP}{\mathbf{P}}
\newcommand{\bU}{\mathbf{U}}
\newcommand{\bC}{\mathbf{C}}
\newcommand{\bA}{\mathbf{A}}
\newcommand{\bE}{\mathbf{E}}
\newcommand{\bF}{\mathbf{F}}
\newcommand{\bK}{\mathbf{K}}
\newcommand{\bM}{\mathbf{M}}
\newcommand{\bJ}{\mathbf{J}}
\newcommand{\bH}{\mathbf{H}}
\newcommand{\bQ}{\mathbf{Q}}
\newcommand{\bS}{\mathbf{S}}
\newcommand{\bV}{\mathbf{V}}
\newcommand{\bX}{\mathbf{X}}
\newcommand{\bY}{\mathbf{Y}}
\newcommand{\bZ}{\mathbf{Z}}
\newcommand{\bh}{\mathbf{h}}
\newcommand{\bx}{\mathbf{x}}
\newcommand{\by}{\mathbf{y}}
\newcommand{\bv}{\mathbf{v}}
\newcommand{\bz}{\mathbf{z}}
\newcommand{\bs}{\mathbf{s}}
\newcommand{\ba}{\mathbf{a}}
\newcommand{\bb}{\mathbf{b}}
\newcommand{\bo}{\mathbf{o}}
\newcommand{\bc}{\mathbf{c}}
\newcommand{\bd}{\mathbf{d}}
\newcommand{\bbe}{\mathbf{e}}
\newcommand{\bff}{\mathbf{f}}
\newcommand{\bqq}{\mathbf{q}}
\newcommand{\bve}{\mathbf{e}}
\newcommand{\bu}{\mathbf{u}}
\newcommand{\bw}{\mathbf{w}}
\newcommand{\bm}{\mathbf{m}}
\newcommand{\bg}{\mathbf{g}}
\newcommand{\bn}{\mathbf{n}}
\newcommand{\bk}{\mathbf{k}}
\newcommand{\bt}{\mathbf{t}}
\newcommand{\bbf}{\mathbf{f}}
\newcommand{\cS}{\cal{S}}
\newcommand{\bmu}{\boldsymbol{\mu}}
\newcommand{\bxi}{\boldsymbol{\xi}}
\newcommand{\bsigma}{\boldsymbol{\sigma}}
\newcommand{\bgamma}{\boldsymbol{\gamma}}
\newcommand{\btau}{\boldsymbol{\tau}}
\newcommand{\brho}{\boldsymbol{\rho}}
\newcommand{\blambda}{\boldsymbol{\lambda}}
\newcommand{\bdelta}{\boldsymbol{\delta}}
\newcommand{\btheta}{\boldsymbol{\theta}}
%\newcommand{\btheta}{\mathbf{\theta}}
\newcommand{\bvartheta}{\boldsymbol{\vartheta}}
\newcommand{\bpsi}{\boldsymbol{\psi}}
\newcommand{\bphi}{\boldsymbol{\phi}}
\newcommand{\bepsilon}{\boldsymbol{\epsilon}}
\newcommand{\balpha}{\boldsymbol{\alpha}}
\newcommand{\bbeta}{\boldsymbol{\beta}}
\newcommand{\bSigma}{\boldsymbol{\Sigma}}
\newcommand{\bLambda}{\boldsymbol{\Lambda}}
\newcommand{\bOmega}{\boldsymbol{\Omega}}
\newcommand{\br}{\boldsymbol{r}}
\newcommand{\0}{\mathbf{0}}
\newcommand{\1}{\mathbf{1}}
\newcommand{\binfty}{\boldsymbol{\infty}}
\newcommand{\E}{\mbox{E}}


\newcommand{\tc}[2]{\textcolor{#1}{#2}}
\def\scrU{\mathscr{U}}
\newcommand{\PP}{\mathbb{P}}
\newcommand{\idxset}{\Lambda}


% caligraphic names
\newcommand{\mcA}{{\mathcal A}}
\newcommand{\mcB}{{\mathcal B}}
\newcommand{\mcC}{{\mathcal C}}
\newcommand{\mcD}{{\mathcal D}}
\newcommand{\mcE}{{\mathcal E}}
\newcommand{\mcF}{\mathcal{F}}
\newcommand{\mcG}{\mathcal{G}}
\newcommand{\mcI}{{\mathcal I}}
\newcommand{\mcJ}{{\mathcal J}}
\newcommand{\mcK}{{\mathcal K}}
\newcommand{\mcL}{{\mathcal L}}
\newcommand{\mcO}{{\mathcal O}}
\newcommand{\mcN}{{\mathcal N}}
\newcommand{\mcM}{{\mathcal M}}
\newcommand{\mcP}{{\mathcal P}}
\newcommand{\mcQ}{{\mathcal Q}}
\newcommand{\mctQ}{\tilde {\mathcal Q}}
\newcommand{\mcR}{{\mathcal R}}
\newcommand{\mcS}{\mathcal{S}}
\newcommand{\mcT}{{\mathcal T}}
\newcommand{\mcU}{\mathcal{U}}
\newcommand{\mcX}{\mathcal{X}}
\newcommand{\mcY}{\mathcal{Y}}



% boldface
\newcommand{\ii}{\mathbf{i}}
\newcommand{\jj}{\mathbf{j}}
\newcommand{\pp}{\mathbf{p}}
\newcommand{\rr}{\mathbf{r}}
\newcommand{\mm}{\mathbf{m}}
\newcommand{\qq}{\mathbf{q}}
\newcommand{\UU}{\mathbf{U}}
\newcommand{\FF}{\mathbf{F}}
\newcommand{\aalpha}{\boldsymbol{\alpha}}
\newcommand{\rrho}{\boldsymbol{\rho}}
\newcommand{\ttheta}{\boldsymbol{y}}
\newcommand{\oone}{\boldsymbol{1}}

% sets
\newcommand{\bbR}{\mathbb{R}}
\newcommand{\bbX}{\mathbb{X}}
\newcommand{\bbN}{\mathbb{N}}
\newcommand{\bbE}{\mathbb{E}}
\newcommand{\bbF}{\mathbb{F}}
\newcommand{\bbS}{\mathbb{S}}
\newcommand{\bbK}{\mathbb{K}}
\newcommand{\bbP}{\mathbb{P}}
\newcommand{\bbC}{\mathbb{C}}
\newcommand{\bbJ}{\mathbb{J}}
\newcommand{\bbI}{\mathbb{I}}
\newcommand{\Nset}{\mathbb{N}_0}
\newcommand{\cset}{{\mathbb C}}
\newcommand{\rset}{{\mathbb R}}
\newcommand{\nset}{{\mathbb N}}
\newcommand{\bbNset}{{\mathbb N}}
\newcommand{\qset}{{\mathbb Q}}
\newcommand{\pset}{{\mathbb P}}
\newcommand{\Pol}{\mathbb{P}}
\newcommand{\eset}[1]{{\mathbb E} \left[ #1 \right] }


% misc.
\newcommand{\Grad}{\nabla}
\newcommand{\ssy}{\scriptscriptstyle}
\newcommand{\dist}{\operatorname{dist}}
\newcommand{\KL}{Karhunen--\Loeve }
\newcommand{\lv}{w}
\def\scrG{\mathscr{G}}
\newcommand{\Real}{\mathop{\text{\rm Re}}}
\newcommand{\Imag}{\mathop{\text{\rm Im}}}
\newcommand{\bno}{n}

\newcommand{\func}{u}



% New Operators
\DeclareMathOperator*{\esssup}{ess\,sup}
\DeclareMathOperator*{\essinf}{ess\,inf}


\newcommand{\sJ}[1]{
\begin{bmatrix*}[r]
  \bJ^{#1}_R   & -\bJ^{#1}_I  \\
  \bJ^{#1}_{I}  & \bJ^{#1}_R   \\
\end{bmatrix*}
}

\newcommand{\szv}[1]{
\begin{bmatrix}
  \bz^{#1}_{R} \\
  \bz^{#1}_{I} \\
\end{bmatrix}
}

\newcommand{\sfv}[1]{
\begin{bmatrix}
  \bbf^{#1}_{R} \\
  \bbf^{#1}_{I} \\
\end{bmatrix}
}

% Flipped

\newcommand{\sJflip}[1]{
\begin{bmatrix*}[r]
  -\bJ^{#1}_R   & \bJ^{#1}_I  \\
  \bJ^{#1}_{I}  & \bJ^{#1}_R   \\
\end{bmatrix*}
}

\newcommand{\szvflip}[1]{
\begin{bmatrix*}[r]
  -\bz^{#1}_{I} \\
  \bz^{#1}_{R} \\
\end{bmatrix*}
}

\newcommand{\sfvflip}[1]{
\begin{bmatrix*}[r]
  -\bbf^{#1}_{I} \\
  \bbf^{#1}_{I} \\
\end{bmatrix*}
}


\newcommand{\BallTaylor}{
\left(\bx_{0},
  \left[
    \begin{array}{c}
  \bqq \\
  \0
  \end{array}
  \right]
+ t
\left[\begin{array}{c}
  \bv_{R} \\
  \bv_{I}
  \end{array}
  \right]
  \right)
}


\def\sD{\mathcal{D}}
\def\sN{\mathcal{N}}
\def\sC{\mathcal{C}}



\def\R{\Bbb{R}}
\newcommand{\verteq}[0]{\begin{turn}{90} $=$\end{turn}}
%\newcommand{\Pr}{\mbox{Pr}}
% \renewcommand{\baselinestretch}{1.25}


\newcommand{\supess}{\mbox{ess} \operatornamewithlimits{sup}}

\newtheorem{asum}{Assumption}
\newtheorem{cond}{Condition}
\newtheorem{exam}{Example}
\newtheorem{prop}{Proposition}
\newtheorem{corollary}{Corollary}
\newtheorem{definition}{Definition}
\newtheorem{remark}{Remark}
\newtheorem{lemma}{Lemma}
\newtheorem{theorem}{Theorem}



\newcommand{\argmax}{\operatornamewithlimits{argmax}}
\newcommand{\argmin}{\operatornamewithlimits{argmin}}


% Colors
\def\boxit#1{%
  \smash{\color{blue}\fboxrule=1pt\relax\fboxsep=2pt\relax%
  \llap{\rlap{\fbox{\vphantom{0}\makebox[#1]{}}}~}}\ignorespaces
}

\def\gboxit#1{%
  \smash{\color{darkgreen}\fboxrule=1pt\relax\fboxsep=2pt\relax%
  \llap{\rlap{\fbox{\vphantom{0}\makebox[#1]{}}}~}}\ignorespaces
}

\definecolor{darkblue}{rgb}{0,0.08,0.4}
\definecolor{brightblue}{rgb}{0.65,0.85,0.85}
\definecolor{darkred}{rgb}{0.8,0.2, 0.2}
\definecolor{darkgreen}{rgb}{0, 0.6, 0}
\definecolor{blueish}{rgb}{0.1176, 0.5647, 1.0000}

\definecolor{darkorange}{RGB}{255,140,0}

\definecolor{colorone}{rgb}{0.1176,0.5647,1.0000}
\definecolor{colortwo}{rgb}{0.5608,0.7373,0.5608}



\begin{document}



\pgfkeys{/pgf/decoration/.cd,
      distance/.initial=10pt
}  

\pgfdeclaredecoration{add dim}{final}{
\state{final}{% 
\pgfmathsetmacro{\dist}{5pt*\pgfkeysvalueof{/pgf/decoration/distance}/abs(\pgfkeysvalueof{/pgf/decoration/distance})}    
          \pgfpathmoveto{\pgfpoint{0pt}{0pt}}             
          \pgfpathlineto{\pgfpoint{0pt}{2*\dist}}   
          \pgfpathmoveto{\pgfpoint{\pgfdecoratedpathlength}{0pt}} 
          \pgfpathlineto{\pgfpoint{(\pgfdecoratedpathlength}{2*\dist}}     
          \pgfsetarrowsstart{latex}
          \pgfsetarrowsend{latex}
          \pgfpathmoveto{\pgfpoint{0pt}{\dist}}
          \pgfpathlineto{\pgfpoint{\pgfdecoratedpathlength}{\dist}} 
          \pgfusepath{stroke} 
          \pgfpathmoveto{\pgfpoint{0pt}{0pt}}
          \pgfpathlineto{\pgfpoint{\pgfdecoratedpathlength}{0pt}}
}}

\tikzset{dim/.style args={#1,#2}{decoration={add dim,distance=#2},
                decorate,
                postaction={decorate,decoration={text along path,
                                                 raise=#2,
                                                 text align={align=center},
                                                 text={#1}}}}}



\title{High dimensional multilevel kriging: A computational
  mathematics approach}

%\title{High dimensional multilevel kriging: A computational
%  mathematics approach \thanks{This material is based upon work
%   supported by the National Science Foundation under Grant
%    No. 1736392.}}



\author{Julio E. Castrill\'on-Cand\'as ${\dagger}$} 
  \email{jcandas@bu.edu}


 \address{
   ${\ddagger}$ Department of Mathematics and Statistics, 
  Boston University, Boston, MA 
  }
   

%%%%%%%%%%%%%%%%%%%%%%%%%%%%%%%%%%%%%%%%%%%%%%%%%%%%%%%%%%%%%%%%%%%%%%%%




\begin{abstract}
With the advent of massive data sets much of the computational science
and engineering communities have been moving toward data-driven
approaches such as regression and classification. However, they
present a significant challenge due to the increasing size, complexity
and dimensionality of the problems.  In this paper a multilevel
Kriging method that scales well with the number of observations and
dimensions is developed.  A multilevel basis is constructed that is
adapted to a kD-tree partitioning of the observations.  Numerically
unstable covariance matrices with large condition numbers are
transformed into well conditioned multilevel matrices without
compromising accuracy. Moreover, it is shown that the multilevel
prediction \emph{exactly} solves the Best Linear Unbiased Predictor
(BLUP), but is numerically stable.  The multilevel method is tested on
numerically unstable problems of up 25 dimensions. Numerical results
show speedups of up to 42,050 for solving the BLUP problem but to the
same accuracy than the traditional iterative approach.
\end{abstract}



\maketitle

\noindent
    {\it Keywords:} Hierarchical Basis, Machine Learning, High
    Performance Computing, Sparsification of Covariance Matrices, Fast
    Multipole Method

    






%%%%%%%%%%%%%%%%%%%%%%%%%%%%%%%%%%%%%%%%%%%%%%%%%%%%%%%%%%%%%%%%%%%%%%%%


%% Introduction  ------------------------------------------------------
%%

\section{Introduction}





%\item {\bf What is the problem?  Why is it hard?}

Massive data sets arise from many fields, including, but not limited
to commerce, astrophysical sky-surveys, environmental data, and tsunami
warning systems.  With the advent of big data sets much of the
computational science and engineering communities have been moving
toward data-driven approaches to regression and classification. These
approaches are effective since the underlying data is incorporated
into the optimization. However, they present a numerical challenge due
to increasing size, complexity and dimensionality.

%\item {\bf How is it done today, and what are the limits of current
%practice?}

Due to the high dimensionality of the underlying data many modern
machine learning methods, such as classification and regression
algorithms, seek a balance between accuracy and computational
complexity. How efficient this balance is depends on the approach.
Linear methods are fast, but only work well when there is linear
separation of the data.

For non-linear description of the data, kernel approaches have been
effective under certain circumstances.  These methods rely on Tikhonov
regularization of the data to obtain a functional representation,
where it is assumed that the noise model of the phenomena is
known. However, this assumption is not necessarily satisfied in
practice and can lead to significant errors as the algorithm cannot
distinguish between noise and the underlying phenomena.


To incorporate the variability of the noise model a class of machine
learning algorithms based on Bayes method have been developed. In this
approach the noise model is assumed to be known up to a class of
probability distributions and an optimal choice is made that fits the
training data and noise. For example, from the Geo-statistics
community a well known approach to identifying the underlying data and
noise model is known as Kriging \cite{Nielsen2002}.  The noise model
parameters are estimated from the Maximum Likelihood Estimation (MLE)
of the likelihood function.

Kriging methods are effective in separating the underlying phenomena
from the noise model. However, in practice the covariance matrices
tend to be ill-conditioned with increasing number of observations
making Kriging methods numerically fragile. Moreover, most
applications are limited to 2 or 3 dimensions. A brief literature
review can be found in \cite{Castrillon2015}.

Kriging methods from the computational mathematics perspective have
been developed using skeletonization factorizations \cite{Minden2016},
low-rank \cite{nowak2013} and Hierarchical Matrices (HM)
\cite{khoromskij2009,Litvinenko2019,Geoga2020} approaches. These
methods are very promising. In particular, for the HM approaches they
have been shown to be near optimal. They work well for low dimensions.
However, they are still subject to ill-conditioning and usually a
nugget is added to change the model to make it more numerically
stable. Moreover, the data is assumed to have zero mean, which many
times will not be the case.

In \cite{Castrillon2015} a novel algorithm to solve Kriging problems
is proposed. The method is fast and robust. In particular, it can
solve Kriging problems that where not tractable with previous
methods. A nugget is not assumed, nor zero mean data.  However, this
approach is limited to 2 or 3 dimensions and the computational cost
scales very fast with spatial dimension, thus making it impractical
for high dimensional problems.

In this paper we extend the Kriging approach in \cite{Castrillon2015}
using binary trees, which are well suited for high dimensional
problems. Ill-conditioned covariance matrices are transformed to
numerically stable multilevel covariance matrices without compromising
accuracy. In addition, a new distance criterion is developed to build
sparse multilevel covariance matrices.  Furthermore, sharper decay
estimates of the coefficients of the multivariate covariance matrix
are derived based on analytic extensions that are well suited for high
dimensional problems. Much of this theory and notation is borrowed
from uncertainty quantification and high dimensional integration for
partial differential equations
\cite{nobile2008a,Castrillon2016,Griebel2016}.

The Kriging estimation is transformed into a multilevel form based on
the numerically stable multilevel covariance matrix. In practice a
sparse version of the multilevel covariance matrix is used. A distance
dependent method is used to build to a sparse version. Sharp decay
estimates (sub-exponential) of the multilevel covariance matrices are
derived using complex analytic extensions of the covariance function
instead of Taylor series expansions, which are infeasible for
relatively large dimensional problems. The numerical results show that
the estimation is solved to good accuracy for a large number of
observations.

The Kriging prediction step is remapped into an equivalent multilevel
formulation that is numerically stable. It is shown that the solution
to the multilevel prediction form \emph{exactly} solves the Best
Linear Unbiased Prediction (BLUP) problem.  To my knowledge, this is a
feature that is unique to the multilevel approach.  If the covariance
matrix is ill-conditioned, then it is not possible to solve the
problem accurately on a computer with a fixed machine
precision. However, the BLUP solution arises from a constrained
optimization problem. By taking advantage of this fact, the multilevel
approach side steps the inversion of the covariance matrix and
directly searches for the solution in a constrained space giving rise
to a stable multilevel covariance matrix.  Moreover, only one matrix
inversion (iterative approach) of the multilevel covariance matrix is
required in contrast to classical BLUP, including the Generalized
Least Squares (GLS) prediction, that requires $p+1$ matrix inversions
(iterative approach), where $p$ is the number of columns of the design
matrix.  Numerical results show speedups of up to 42,050 for solving
the BLUP problem to at least the same accuracy.

The multilevel Kriging method makes previously impractical missing
data problems feasible.  We are currently applying the Kriging
approach to missing data problems for medical data sets and have shown
up to 5-6 times improved accuracy over traditional state of the art
missing data packages.


In Section \ref{Introduction} the problem formulation is introduced.
In section \ref{MultilevelCovarianceMatrix} the construction of the
multilevel covariance matrix is discussed. In section
\ref{multilevelestimator} the multilevel estimator and predictor are
formulated and numerical computational issues are discussed in section
\ref{numericalcomputation}.  In section \ref{errorestimates} a
mathematical analysis of the decay of the entries of the multilevel
covariance matrix is developed. This section can also be skipped for
the less mathematically inclined reader.  In section
\ref{numericalresults} the multilevel Kriging method is tested on
numerically unstable problems of up to 25 dimensions.  In the Appendix
the Multivariate polynomial interpolation based on complex analytic
extensions is discussed. These results are used for to derive the
decay of the entries of the multilevel covariance matrix.  In section
\ref{multilevelapproach} it is shown how to construct the multilevel
basis based on kd-trees.


%% Problem Setup ------------------------------------------------------
%%
\section{Problem setup}
\label{Introduction}

Consider the following model for a Gaussian random field $Z$:
\begin{equation}
Z(\bx) = \bk(\bx)\T \bbeta+\varepsilon(\bx), \qquad \bx \in \R^d,
\label{Introduction:noisemodel}
\end{equation}
where $d$ is the number of spatial dimensions, $\bk:\R^d \rightarrow
\R^p$ is a functional vector of the spatial location $\bx$,
$\bbeta\in\R^p$ is an unknown vector of coefficients, and
$\varepsilon$ is a stationary mean zero Gaussian random field with
parametric covariance function
$C(\bx,\bx';\btheta)=\cov\{\varepsilon(\bx),\varepsilon(\bx')\}$ with
an unknown vector of positive parameters $\btheta\in\R^d$.

Suppose that we obtain $N$ observations and stack them in the data
vector $\bZ=(Z(\bx_1),\ldots,$ $Z(\bx_N))\T$ from locations $\bbS :=\{
\bx_{1},\dots,\bx_{N}\}$, where the elements in $\bbS$ are restricted
such that the design matrix defined below, $\bX$, has full column
rank.  Furthermore, without loss of generality all the locations in
$\bbS$ are contained in the unit hypercube $[-1,1]^{d}$.  Let
$\bC(\btheta)=\cov(\bZ,\bZ\T)\in \R^{N \times N}$ be the covariance
matrix of $\bZ$ and assume it is positive definite for all
$\btheta\in\R^w$.  Define $\bX=\big( \bk(\bx_1) \ldots$ $
\bk(\bx_N)\big)\T\in \R^{n\times p}$ and assume it is of full rank,
$p$. Since the model \eqref{Introduction:noisemodel} is a Gaussian
random field, then from the samples of $\bbS$ the following vectorial
model is obtained
\begin{equation}
{\bf Z} = \bX \bbeta +{\boldsymbol \varepsilon},
\label{Introduction:vectormodel}
\end{equation}
where $\boldsymbol \varepsilon$ is a Gaussian random vector,
${\boldsymbol \varepsilon} \sim \mcN(\0,\bC(\btheta))$. The aim
now is to:

\begin{enumerate}[i)]
\item {\it Estimate} the unknown vectors $\bbeta$ and $\btheta$;

\item {\it Predict} $Z(\bx_0)$, where $\bx_0$ is a new spatial
  location. These two tasks are particularly computationally
  challenging when the sample size $N$ and number of dimensions $d$
  are large.
\end{enumerate}

There is a very large literature on Gaussian process regression that
deal with this problem. Please see \cite{Castrillon2015} for a brief
literature review.  The unknown vectors $\bbeta$ and $\btheta$ are
estimated with the log-likelihood function
\begin{equation}
  \begin{split}
\ell(\bbeta,\btheta)&=-\frac{n}{2}\log(2\pi)-\frac{1}{2}\log
\det\{\bC(\btheta)\} \\ &
-\frac{1}{2}(\bZ-\bX\bbeta)\T\bC(\btheta)^{-1}
(\bZ-\bX\bbeta),
\end{split}
\label{Introduction:loglikelihood}
\end{equation}
which can be profiled by Generalized Least Squares (GLS) with
\begin{equation}
  \hat \bbeta(\btheta)=\{\bX\T \bC(\btheta)^{-1} \bX\}^{-1}\bX\T
  \bC(\btheta)^{-1}\bZ.
  \label{GLSbeta}
\end{equation}
In general this is not a good choice, since profiling with the Maximum
Likelihood Estimator (MLE) of $\btheta$ is prone to be biased
\cite{Castrillon2015}.

%A solution to this problem is to use restricted maximum likelihood
%(REML) estimation which consists in calculating the log-likelihood of
%$n-p$ linearly independent contrasts, that is, linear combinations of
%observations whose joint distribution does not depend on $\bbeta$,
%from the set $\bY=\{\bI_n-\bX(\bX\T\bX)^{-1}\bX\T\}\bZ$.


For the prediction part, consider the Best Linear Unbiased Predictor
(BLUP) $\hat Z(\bx_0)=\lambda_0+\blambda\T\bZ$ where
$\blambda=(\lambda_1,\ldots,\lambda_n)\T$. The unbiased constraint
implies $\lambda_0=0$ and $\bX\T\blambda=\bk(\bx_0)$.  The
minimization of the mean squared prediction error
E$[\{Z(\bx_0)-\blambda\T\bZ\}^2]$ under the constraint
$\bX\T\blambda=\bk(\bx_0)$ yields
\begin{equation}
\hat Z(\bx_0)=\bk(\bx_0)\T\hat \bbeta+\bc(\btheta)\T
\bC(\btheta)^{-1}(\bZ-\bX\hat \bbeta), \label{KrigBLUP}
\end{equation}
where $\bc(\btheta)=\cov\{\bZ,Z(\bx_0)\}\in \R^{n}$ and $\hat \bbeta$ is
defined in (\ref{GLSbeta}).  

Now, let $\alpha:= (\alpha_{1},\dots,\alpha_{d}) \in \mathbb{Z}{^d}$,
$|\alpha| := \alpha_{1}+\dots+\alpha_{d}$, $\bx : =
[x_1,\dots,x_d]$. For any $w \in \bbN_+$ (where $\mathbb{N}_+ :=
\mathbb{N} \cup \{0\}$) let $\mcQ^d_w$ be the set of Total Degree (TD)
monomials $\{x_1^{\alpha_1} \dots x_d^{\alpha_d}\,\,\,|\,\,\, |\alpha|
\leq w\}$. The typical choice for the matrix $\bX$ is to build it from
the monomials of $\mcQ^d_w$ with cardinality
$p(d,w):=\begin{pmatrix} d + w \\ w \end{pmatrix}$.

The challenge is that the covariance matrix $\bC(\btheta)$ in many
practical cases is ill-conditioned, leading to slow and inaccurate
estimates of $\btheta$. Following the approach in
\cite{Castrillon2015} the data vector $\bZ$ is transformed into
decoupled multilevel description of the model
\eqref{Introduction:noisemodel}.  This multilevel representation leads
to significant computational benefits, including numerical stability,
when computing the Kriging predictor $\hat Z(\bx_0)$ in
(\ref{KrigBLUP}) for large sample size $N$ and high dimensions $d$.
Note, that in this paper we shall refer to the \emph{single level}
approach to solving the Kriging problem by applying the estimation and
prediction steps directly to the data $\bZ$ and covariance matrix
$\bC(\btheta)$.





%% \section{Polynomial Interpolation}
%% \label{Polynomial}

%% \corb{In this section we give some background on polynomial
%%   interpolation in high dimensions. This will be critical to estimate
%%   the decay rates of the entries of the multilevel covariance matrix
%%   for high dimensional problems. Note that for the less mathematically
%%   inclided reader this section can be skipped as it is only used for
%%   estimating the decay of the multilevel covariance matrix.}

%% The decay of the coefficients will directly depend on the analytic
%% properties of the covariance function. The traditional error estimates
%% of polynomial interpolation are based on multi-variate $m^{th}$ order
%% derivatives. However, for many cases, such as the Mat\'{e}rn
%% covariance function, the derivatives are too complex or expensive to
%% manipulate for even a moderate number of dimensions. This motivates
%% the study of polynomial numerical approximations based on complex
%% analytic extensions, which are much better suited for high dimensions.
%% Much of the discussion that follows has it roots in the field of
%% uncertainty quantification and high dimensional interpolation
%% \cite{nobile2008a,Castrillon2016,Griebel2016}
%% for partial differential
%% equations.


%% Consider the problem of approximating a function $v: \Gamma^{d}
%% \rightarrow \R$ on the domain $\Gamma^{d}$.  Without loss of
%% generality let $\Gamma : = [-1, 1]$ and $\Gamma^{d} := \prod_{n =
%%   1}^{d} \Gamma$. Suppose that $\mcG \subset \Gamma^{d}$, then define
%% the following spaces
%% \[
%% \begin{split}
%%   &
%% L^q(\mcG) := \{ v(\by)\, | \, \int_{\mcG} v(\by)^q \text{d}
%% \by < \infty  \}
%% \,\,\,
%% \mbox{and} \\
%% &
%% L^{\infty}(\mcG) := \{ v(\by)\, | \, \sup_{\by \in \mcG} |v(\by)|
%% < \infty  \}.
%% \end{split}
%% \]


%% Suppose that $\mcP_{ q}(\Gamma):=\text{\rm span}\{y^k,\,k=0,\dots,q\}$
%% i.e. the space of polynomials of degree at most $q$. Let $\mcI^{m} :
%% C^{0}(\Gamma) \rightarrow \mcP_{m-1}(\Gamma)$ be the univariate
%% Lagrange interpolant
%% \[
%% \mcI_{m}(v(\by)):=
%% \sum_{k=1}^{m}v(y^{(k)})l_{m,k}(y^{(k)}),
%% \]
%% where $y^{(1)}, \dots, y^{(m)}$ is a set of distinct knots on $\Gamma$
%% and $\{ l_{n,k} \}_{k=0}^{m}$ is a Lagrange basis of the space
%% $\mcP_{m-1}(\Gamma)$. The variable $m \in \Nset$
%% %, where $\Nset_{+} := \Nset \cup 0$,
%% corresponds to the order of approximation of the
%% Lagrange interpolant. However, for the case of the zero order
%% interpolation $m = 0$ corresponds to $\mcI_{0} = 0$.


%% \begin{remark}
%% For high dimensional interpolation the particular set of points
%% $y^{(1)}, \dots, y^{(m)}$ that we will use is the Clenshaw-Curtis
%% abscissas.  This is further discussed in this section. However, for
%% now, we assume that the points are only distinct.
%%   \end{remark}


%% For $m \geq 1$ let
%% \[
%% \Delta_{m}
%% := \mcI_{m}-\mcI_{m-1},
%% \]
%% From the difference operator $\Delta_{m}$ we can readily observe that
%% $\mcI_{m} = \sum_{k=1}^{m} \Delta_{k}$, which is reminiscent of multi
%% resolution wavelet decompositions. The idea is to represent
%% multivariate approximation as a summation of the difference operators.

%% Consider the multi-index tupple $\bm = (m_1,\dots,m_d)$, where $\bm
%% \in \Nset^{d}$, and form the tensor product operator
%% $\mcS_{w,d}: \Gamma \rightarrow \R$ as
%% \begin{equation}
%%   \mcS_{w,d}
%%       [v(\by)]
%%       :
%%       =
%%  \sum_{\bm \in \bbNset^{d}: \sum_{i=1}^{d} m_i - 1  \leq w } \;\;
%%  \bigotimes_{n=1}^{d} {\Delta^{n}_{m_n}}(v(\by)).
%% \label{errorestimates:SG}
%% \end{equation}
%% Note that by ${\Delta^{n}_{m_n}}(v(\by))$ we mean that the difference
%% operator ${\Delta_{m_n}}$ is applied along the $n^{th}$ dimension in
%% $\Gamma$.


%% Let $C^{0}(\Gamma_d; \R) : = \{ v: \Gamma_d \rightarrow \R\,\,$ is
%% continuous on $\Gamma_d$ and $\max_{\by\in \Gamma_d} |v(\by)| < \infty
%% \}$.  From Proposition 1 in \cite{Back2011} it is shown that for any
%% $v \in C^0(\Gamma_d;\R)$, we have $\mcS_{w,d}[v]\in \mcQ^{d}_{w}$.
%% Moreover, $\mcS_{w,d}[v] = v$, for all $v \in \mcQ^{d}_{w}$. The key
%% observation to take away is that the operator $\mcS_{w,d}[v]$ is
%% \textit{exact} in the space of polynomials $\mcQ^{d}_{w}$. This will
%% be useful in connecting the Lagrange interpolant with Chebyshev
%% polynomials.


%% Let $T_k:\Gamma \rightarrow \R$, $k = 0, 1, \dots$, be a Chebyshev
%% polynomial over $\Gamma$, which are defined recursively as follows:
%% $T_0(y) = 1$, $T_1(y) = y$, $\dots$, $T_{k+1}(y) = 2yT_{k}(y) -
%% T_{k-1}(y)$, $\dots$, where $y \in \Gamma$. Chebyshev polynomials are
%% well suited for the approximation of functions with analytic
%% extensions on a complex region bounded by a Bernstein ellipse. They
%% bypassing the need of using derivative information and sharp bounds on
%% the error are readily available. Suppose that $\sigma > 0$ and denote
%% by
%% \[
%% \begin{split}
%%   \mcE_{\sigma} := \Big\{
%%   &z \in \bbC, \sigma \geq
%% \delta \geq 0 ;\,\Real{z} = \frac{e^{\delta} + e^{-\delta}
%% }{2}cos(\theta) \\
%% &\Imag{z} = \frac{e^{\delta} 
%%   - e^{-\delta}}{2}sin(\theta),
%% \theta \in [0,2\pi)
%%   \Big\}
%% \end{split}
%%   \]
%% as the region bounded by a Bernstein ellipse (see Figure
%% \ref{erroranalysis:sparsegrid:polyellipse}).

%% The following theorem is based on complex analytic extensions on
%% $\mcE_{\sigma}$ and provides a control for the Chebyshev polynomial
%% approximation.

%% \begin{theorem}
%% Suppose that for $u:\Gamma \rightarrow \R$ there exists an analytic
%% extension on $\mcE_{\sigma}$. If $|u| \leq M < \infty$ on
%% $\mcE_{\sigma}$ then there exists a sequence of coefficients
%% $|\alpha_k| \leq M / e^{k\sigma}$ such that $u \equiv \alpha_0 +
%% 2\sum_{k = 1}^{\infty} \alpha_{k} T_{k}$ on $\mcE_{\sigma}$. Moreover,
%% if $y \in \Gamma$ then
%% \[
%% %\begin{multline*}
%% %\shoveright{|q(y) - \alpha_0  - 2\sum_{k = 1}^{n} \alpha_{k} T_{k}(y)|
%% %\leq 
%% %\frac{2M}{e^{\sigma} - 1} e^{-n \sigma}.}
%% |q(y) - \alpha_0  - 2\sum_{k = 1}^{n} \alpha_{k} T_{k}(y)|
%% \leq 
%% \frac{2M}{e^{\sigma} - 1} e^{-n \sigma}.
%% %\end{multline*}
%% \]
%% \label{errorestimates:theorem}
%% \end{theorem}
%% \begin{proof}
%% See Theorem 2.25 in \cite{Khoromskij2018}
%% \end{proof}
%% \qed

%% \begin{figure}[htb]%[12]{r}{7cm}%[htp]
%% \begin{center}
%% \begin{tikzpicture}
%%     \begin{scope}[font=\scriptsize]

      
%%       \filldraw[fill=blue!20,
%%       semitransparent] (0,0) ellipse (2 and 1);

%%     \draw [->] (-2.5, 0) -- (2.5, 0) node [below left]  {$\Real $};
%%     \draw [->] (0,-1.5) -- (0,1.5) node [below left] {$\Imag$};
%%     \draw (1,-3pt) -- (1,3pt)   node [above] {$1$};
%%     \draw (-1,-3pt) -- (-1,3pt) node [above] {$-1$};
%%     \end{scope}
    
%%     \node [below right] at (-2.5,1.25) {$\mcE_{\sigma}$};

%%     \node [] at (0.75,1.25) {$\frac{e^{
%%           \sigma} - e^{- \sigma}}{2}$};

    
%%     \node [] at (2.75,0.25) {$\frac{e^{
%%       \sigma} + e^{- \sigma}}{2}$}; 
    
%% \end{tikzpicture}
%% \end{center}
%% \caption{Complex region bounded by the Bernstein ellipse.}
%% \label{erroranalysis:sparsegrid:polyellipse}
%% \end{figure}

%% We can now connect the error due to the Lagrange interpolation with
%% Chebyshev expansions. It is known that if $u \in C(\Gamma,\R)$ then
%% \[
%% \|(I - \mcI_{m})u\|_{L^{\infty}(\Gamma)} \leq
%% (1 + \Lambda_{m})
%% \min_{h \in \mcP_{m-1}} \| u - h \|_{L^{\infty}(\Gamma)},
%% \]
%% where $\Lambda_{m}$ is the Lebesgue constant (See Lemma 7 in
%% \cite{babusk_nobile_temp_10}). Note that $I:C^{d}(\Xi;\R) \rightarrow
%% C^{d}(\Xi;\R)$ refers to the identity operator and the domain $\Xi$ is
%% taken from context. For the previous case $\Xi = \Gamma$.  Bounds on
%% $\Lambda_{m}$ are known in the context of the location of the knots
%% $y^{(1)}, \dots, y^{(m)} \in \Gamma$. In this article we restrict our
%% attention to Clenshaw-Curtis abscissas
%% %\[
%% \[
%% y^{(j)} = -\cos \left( \frac{\pi(j-1)}{m - 1} \right),\,\, j =
%% 1,\dots, m
%% \]
%% %\]
%% and $\Lambda_m$ is bounded by $2\pi^{-1}(\log{(m-1)} + 1) \leq 2m - 1$
%% (see \cite{babusk_nobile_temp_10}).  Since the interpolation operator
%% $\mcI_{m}$ is exact on $\mcP_{m - 1}$, then if $u:\Gamma \rightarrow
%% \R$ has an analytic extension in $\mcE_{\sigma}$ we have from Theorem
%% \ref{errorestimates:theorem} (following a similar approach as in
%% \cite{babusk_nobile_temp_10}) that
%% \[
%% \begin{split}
%% \|(I - \mcI_{m})u\|_{L^{\infty}(\Gamma_n)}
%% &\leq
%% (1 + \Lambda_{m})
%% \frac{2M}{e^{\sigma} - 1} e^{-\sigma (m-1)} \\
%% &\leq 
%% 2 C(M,\sigma) m e^{-\sigma (m-1)},
%% \end{split}
%% \]
%% where $C(M,\sigma_n) := \frac{2M}{(e^{ \sigma} - 1)}$. We then
%% conclude that for all $k = 1,\dots, m$
%% \begin{equation}
%% \begin{split}
%% \| \Delta_{k}(u) \|_{L^{\infty}(\Gamma)} 
%% &=
%% \|
%% \mcI^{m}(u) - \mcI^{m-1}(u)
%% \|_{L^{\infty}(\Gamma)} \\
%% &\leq
%% \|(I - \mcI_{m})u\|_{L^{\infty}(\Gamma)} \\
%% &+
%% \|(I - \mcI_{m-1})u\|_{L^{\infty}(\Gamma)} \\
%% &\leq
%% e^{2\sigma}C(M,\sigma) m e^{-\sigma m}.
%% \end{split}
%% \label{interpolation:eqn1}
%% \end{equation}
%% Let $\mcE_{\sigma,n} \subset \bbC^{d}$ a complex region bounded by a
%% Bernstein ellipse such that the restriction on $\Gamma_{d}$ is along
%% the $n^{th}$ dimension and form the polyellipse $\mcE^{d}_{\sigma}:=
%% \prod_{n=1}^{d} \mcE_{\sigma,n}$.  Suppose that $v:\mcE^{d}_{\sigma}
%% \rightarrow \bbC$ is analytic on $\mcE^{d}_{\sigma}$ and let
%% $\tilde{M}(v) := \max_{\bz \in \mcE^{d}_{\sigma}} |v(\bz)|$.

%% Note we refer to $\mcI^{n}_{m}$ as the Lagrange operator of order $m$
%% along the $n^{th}$ dimension and similarly $\mcP^{n}_{m-1}$ is the
%% space of the span of univariate polynomials up to degree $m-1$ along
%% the $n^{th}$ dimension.  Form the tensor product $\bI^{d}_{m} :=
%% \mcI^{1}_{m} \times \dots \times \mcI^{d}_{m}$, thus $\bI:C(\Gamma,\R)
%% \rightarrow \bbP$ where $\bbP := \mcP^{1}_{m-1} \times \dots \times
%% \mcP^{d}_{m-1}$. From Theorem 2.27 in \cite{Khoromskij2018} we can
%% conclude that for a finite dimension $d$, as $m \rightarrow \infty$
%% then $\bI^{d}_{m}[v] \rightarrow v$.

%% Applying equation \eqref{interpolation:eqn1} to equation
%% \eqref{errorestimates:SG} we have that
%% \begin{equation}
%% \begin{split}
%% & \| (I - \mcS_{w,d})
%%  v(\by)
%%  \|_{L^{\infty}(\Gamma^{d})} \\
%%  &\leq
%%  \left\| \sum_{\bm \in \bbNset^{d}: \sum_{i=1}^{d} m_i - 1 > w } \;\;
%%  \bigotimes_{n=1}^{d} {\Delta^{n}_{m_n}}(v(\by))\right\|_{L^{\infty}(\Gamma^d)} \\
%%  &\leq
%%  \sum_{\bm \in \bbNset^{d}: \sum_{i=1}^{d} m_i - 1 > w } \;\;
%%  \bigotimes_{n=1}^{d} \|{\Delta^{n}_{m_n}}(v(\by))\|_{L^{\infty}(\Gamma^d)}  \\
%%  &\leq
%%  \sum_{\bm \in \bbNset^{d}: \sum_{i=1}^{d} m_i - 1 > w }
%%  e^{2d} C(M,\sigma)^{d} \\
%%  &
%%  \left( \prod_{n=1}^{d} m_n\right) \exp{\left( -\sum_{n=1}^{d}
%%    \sigma m_{n} \right)}.
%% % \\
%% %  &\leq
%% % \sum_{\bk \in \bbNset^{d}_{0}: \sum_{i=1}^{d} k_i > w }
%% % e^{2d} C(M,\sigma)^{d} \left( \prod_{n=1}^{d} (k_n + 1)\right)
%% % \exp{\left( -\sum_{n=1}^{d}
%% %   \sigma (k_{n}+1) \right)}.
%% \end{split}
%% \label{interpolation:eqn2}
%% \end{equation}

%% By applying Theorem 2.10 and Corollary 2.11 in \cite{Griebel2016} if
%% $ w \geq  d$ and $p( d, w) \geq
%% \left(\frac{2  d}{\kappa( d)}\right)^{ d}$, where
%% $\kappa( d) := \sqrt[\leftroot{-2}\uproot{2}  d]{
%%   d!} >  d/e$ (Sterling approximation), then for any $\hat
%% \sigma \in \R_{+}$
%% \begin{equation}
%% \begin{split}
%%  & \sum_{\bk \in \bbNset^{ d}_{0}: \sum_{i=1}^{ d} k_i  >  w }
%%  \exp{\left( -\sum_{n=1}^{ d} \hat \sigma
%%    k_{n} \right)} \\
%%  &\leq
%%  \sum_{\bk \in \bbNset^{d}_{0}: \hat \sigma \sum_{i=1}^{ d} k_i  \geq  w \hat \sigma  }
%%  \exp{\left( -\sum_{n=1}^{ d}
%%    \hat \sigma k_{n} \right)} \\
%%  &\leq
%%  \hat \sigma  d e
%%  \left( \frac{e^{\hat \sigma}}{1 - e^{-\hat \sigma}} \right)^{ d}
%%  \exp \left(-\frac{ d}{e} \hat \sigma  p^{\frac{1}{ d}}
%%  \right) p^{\frac{ d-1}{ d}}.
%% \end{split}
%% \label{interpolation:eqn3}
%% \end{equation}
%% where $\bk \in \bbNset^{d}_{0}$ and $\bk:=(k_1,\dots,k_d)$.






%% Following the same approach as in \cite{Griebel2016} observe that for
%% $0 < \delta < 1$ we can obtain a bounded constant $c_{n,\delta} \leq
%% (e\sigma \delta)^{-1}$ such that $m_n \exp(-\sigma m_n) \leq (e\sigma
%% \delta)^{-1}$ $\exp(-\sigma m_n (1 - \delta))$. Set $\hat \sigma :=
%% \sigma (1 - \delta)$ and by combining equations
%% \eqref{interpolation:eqn2} and \eqref{interpolation:eqn3} we have
%% proven the following result.

%% \begin{lemma} Suppose that $0< \delta < 1$, $\hat
%%   \sigma := \sigma (1 - \delta)$, and $p(d,w) \geq \left(\frac{2
%%     d}{\kappa(d)}\right)^{d}$ then
%%   \[
%%   \begin{split}
%%  &\| (I - \mcS_{w,d})
%%  v(\by)
%%  \|_{L^{\infty}(\Gamma^{d})}\\
%%  \leq &
%%  \frac{C(\tilde M,\sigma)^d e^{d - \sigma(1 - \delta) + 1} \hat \sigma d }
%%  {
%% (\sigma \delta)^{d}}
%%  \left( \frac{e^{\hat \sigma}}{1 - e^{-\hat \sigma}} \right)^{d} \\
%%  &
%%  \exp \left(-\frac{d}{e} \hat \sigma  p^{\frac{1}{d}}
%%  \right) p^{\frac{d-1}{d}}.
%%  \end{split}
%%  \]
%%  \label{interpolation:lemma1}
%% \end{lemma}


%% \begin{remark}
%% The restriction $p(d,w) \geq \left(\frac{2
%%   d}{\kappa(d)}\right)^{d}$ is not strict and can be relaxed such that
%% sub-exponential convergence is still obtained.  We refer the reader to
%% the bound of the Gamma function in Lemma 2.5 (\cite{Griebel2016}) and
%% it's application in the proofs of Theorem 2.10 and Corollary 2.11.
%% \label{interpolation:remark1}
%% \end{remark}



\section{Multilevel approach}
\label{multilevelapproach}

The general approach of this paper and multilevel basis construction
are now presented. We mostly follow the exposition laid out in
\cite{Castrillon2015}. The proof of Proposition
\ref{Multilevelapproach:theo1} is repeated, but clarified with
more details.

Let $\mcP^{p}(\bbS)$ be the span of the columns of the design matrix
$\bX$. Suppose that there exists the orthogonal projections $\bL :
\R^n \rightarrow \mcP^{p}(\bbS)$ and $\bW : \R^n \rightarrow
\mcP^{p}(\bbS)^{\perp}$, where $\mcP^{p}(\bbS)^{\perp}$ is the
orthogonal complement of $\mcP^{p}(\bbS)$.  The operator $\left[
\begin{array}{c}
\bW \\
\bL
\end{array}
\right ]$ is assumed to be orthonormal.


The first step is to filter out the effect of the trend by project the
observation onto the orthogonal subspace.  Let $\bZ_{\bW}: = \bW \bZ$,
thus from equation \eqref{Introduction:vectormodel} it follows that
$\bZ_{\bW} = {\bf W} ({\bX \bbeta}+ {\boldsymbol \varepsilon}) = {\bf
  W{\boldsymbol \varepsilon}}$. Notice that the trend component ${\bX}
\bbeta$ is removed from the data ${\bf Z}$. The new log-likelihood
function for $\bZ_{\bW}$ becomes
\begin{equation}
  \begin{split}
\ell_{\bW}(\btheta)
&=-\frac{n}{2}\log(2\pi)-\frac{1}{2}\log
\det\{\bC_{\bW}(\btheta)\} 
-\frac{1}{2}\bZ_{\bW}\T\bC_{\bW}(\btheta)^{-1}\bZ_{\bW},
\end{split}
\label{Introduction:multilevelloglikelihood}
\end{equation}
where $\bC_{\bW}(\btheta) := \bW \bC(\btheta) \bW \T$ and
$\bZ_{\bW}\sim \mcN_{N-p}(\0,$ $\bC_{\bW}(\btheta))$.  A consequence
of the filtering is that we obtain an unbiased estimator
\cite{Castrillon2015}.

The decoupling of the likelihood function is not the only advantage of
using $\bC_{\bW}(\btheta)$. The following theorem also shows that
$\bC_{\bW}(\btheta)$ is more numerically stable than $\bC(\btheta)$.

\begin{prop} 
\label{Multilevelapproach:theo1}
Let $\kappa(A) \rightarrow \R$ be the condition number of the matrix
$A \in \R^{N \times N}$ then
\[
\kappa(\bC_{\bW}(\btheta)) \leq 
\kappa(\bC(\btheta)).
\]
\end{prop}
\noindent 
\begin{proof}
To see this let $\bv := \bW\T \bw$ for all $\bw \in \R^{N-p}$, which
implies that $\bv \in \mathbb{R}^{n} \backslash
\mcP^{p}(\bbS)$. Moreover, this map is bijective.  Now, $\bv \T
\bC(\btheta) \bv = \bw \T \bC_{\bW}(\btheta) \bw$ for all $\bw \in
\R^{N-p}$. From the orthonormal property we have that for all $\bv \in
\mathbb{R}^{n} \backslash \mcP^{p}(\bbS)$
\[
\begin{split}
\min_{\bv \in \mathbb{R}^{n} \backslash \mcP^{p}(\bbS)} \frac{\bv\T\bC(\btheta)\bv}
{\|\bv\|^2} 
= \min_{\bw \in \mathbb{R}^{N-p} } \frac{\bw\T\bC_{\bW}(\btheta)\bw}{\|\bw\|^2} 
\,\,\,
\mbox{and}
\,\,\,
\max_{\bv \in \mathbb{R}^{n} \backslash \mcP^{p}(\bbS) } \frac{\bv\T\bC(\btheta)\bv}
{\|\bv\|^2}  
= \max_{\bw \in \mathbb{R}^{N-p}} \frac{\bw\T\bC_{\bW}(\btheta)\bw}{\|\bw\|^2}.
\end{split}
\]
Now, it is not hard to see that
\[
\begin{split}
  0
  &<
  \min_{\bv \in \mathbb{R}^{n}} \frac{\bv\T\bC(\btheta)\bv}{ \|\bv\|^{2} } 
\leq 
\min_{\bv \in \mathbb{R}^{n} \backslash \mcP^{p}(\bbS)}
 \frac{\bv\T\bC(\btheta)\bv}{ \|\bv\|^{2} }
\leq \max_{\bv \in \mathbb{R}^{n} \backslash \mcP^{p}(\bbS)} \frac{\bv\T\bC(\btheta)\bv}
{ \|\bv\|^{2} } 
\leq
\max_{\bv \in \mathbb{R}^{n}} \frac{\bv\T\bC(\btheta)\bv}{ \|\bv\|^{2} }.
\end{split}
\]
The result follows from the positive definite property of
$\bC(\btheta)$.
\end{proof}


Proposition \ref{Multilevelapproach:theo1} states that the condition
number of $\bC_{\bW}(\btheta)$ is less or equal to the condition
number of $\bC(\btheta)$. Thus computing the inverse of
$\bC_{\bW}(\btheta)$ (using a direct or iterative method) will
generally be more stable.

In practice, computing the inverse of $\bC_{\bW}(\btheta)$ can be
significantly more stable than $\bC(\btheta)$ depending on the choice
of $\mcQ^d_w$. This has many significant implications as it will now
be possible to solve numerically unstable problems. Furthermore, the
following useful result can be proven.

\begin{corollary}
  \label{Multilevelapproach:cor1}
  Let $[\bC_{\bW}(\btheta)]^q$ be the multilevel
  covariance matrix built from a TD basis with cardinality $q \in
  \bbN$.  Suppose that $p \leq q$, then
  \[
  \kappa([\bC_{\bW}(\btheta)]^p)
  \leq \kappa([\bC_{\bW}(\btheta)]^q).
  \]
\end{corollary}
\begin{proof}
  This follows from the fact that $\mcP^{q}(\bbS) \subset
  \mcP^{p}(\bbS)$ and by applying a similar argument as the proof of
  Proposition \ref{Multilevelapproach:theo1}.
\end{proof}


There are other advantages to the structure of the matrix
$\bC_{\bW}(\btheta)$.  In section \ref{errorestimates} we show that
for a good choice of the $\mcP(\bbS)$ the entries of
$\bC_{\bW}(\btheta)$ decay rapidly, and most of the entries can be
safely eliminated. A level dependent criterion approach is shown in
Section \ref{MultilevelCovarianceMatrix} that indicates which entries
are computed and which ones are not. With this approach a sparse
covariance matrix $\tilde{\bC}_{\bW}$ can be constructed such that it
is close to $\bC_{\bW}$ in a matrix norm sense, even if the
observations are highly correlated with distance.
%From the decay estimates of
%Section \ref{MultilevelCovarianceMatrix} 





\subsection{Binary multilevel basis}
\label{MultilevelREML}

In this section the construction of Multilevel Basis (MB) is shown.
The approach followed in this section is a based on the MB
construction in \cite{Castrillon2013}. The MB can then be used to: (i)
form the multilevel likelihood
\eqref{Introduction:multilevelloglikelihood}; (ii) sparsify the
covariance matrix $\bC_{\bW}(\btheta)$; and (iii) improve the numerical
stability of the covariance matrix $\bC(\btheta)$ in it's multilevel
form. But first, let us establish notations and definitions:
\begin{enumerate}

% \item Given $\mcQ^d_w$ and the locations $\bbS$
%  construct the design matrix $\bX$. Furthermore, form a second set of
%  monomials $\mctQ^{a}_{\Lambda^{m,g}(w)} : =
%  \mcQ_{\Lambda^{m,g}(w+a)} $ for $a = 0,1,\dots,$ i.e.
%  $\mctQ_{\Lambda^{m,g}(w)} \subset
%  \mctQ^{a}_{\Lambda^{m,g}(w)}$. Denote the accuracy parameter $\tilde
%  p \in \bbN$ as the cardinality of
%  $\mctQ^{a}_{\Lambda^{m,g}(w)}$. From the set of monomials
%  $\mctQ_{\Lambda^{m,g}(w)}$, for some user given parameter $a \in
%  \bbN_0$, and the set of observations $\bbS$ generate the design
%  matrix $\tilde \bX^{a}$. Denote also the space $\mcP^{\tilde
%    p}(\bbS)$ as the span of the columns of $\tilde \bX^{a}$.

\item For any index $i,j \in \mathbb{N}_{0}$, $1 \leq i \leq N$, $1
  \leq j \leq N$, let $\bve_{i}[j] = \delta[i-j]$, where
  $\delta[\cdot]$ is the discrete Kronecker delta function.

\item Let $\phi(\bx,\by;\btheta):\R^{d} \times \R^{d} \rightarrow \R$
  be the covariance function and assumed to be a positive definite.
  Let $\bC(\btheta)$ be the covariance matrix that is formed from all
  the interactions between the observation locations $\bbS$
  i.e. $\bC(\btheta) := \{ \phi(\bx_i,\by_j) \}$, where $i,j, =
  1,\dots,N$.  Alternatively we refer to $\phi(r; \btheta)$ as the
  covariance function where $r:\R^d \times \R^d \rightarrow \R$ is a
  function of $\bx$, $\by$ and $\btheta$.
\end{enumerate}

\begin{definition} The Mat\'{e}rn covariance function:
\[
\phi(r;\btheta)=\frac{1}{\Gamma(\nu)2^{\nu-1}} \left(
\sqrt{2\nu}\frac{r}{\rho} \right)^{\nu} K_{\nu} \left(
\sqrt{2\nu}\frac{r}{\rho} \right),
\]
where with a slight abuse of notation $\Gamma$ is the gamma function,
$r \in \R_{+}$, $0 < \nu$, $0 < \rho < \infty$, and $K_{\nu}$ is the
modified Bessel function of the second kind. It is understood from
context when $\Gamma$ is the gamma function.
\end{definition}

\begin{remark} The Mat\'{e}rn covariance function is a good choice for
the random field model. The parameter $\rho$ controls the length
correlation and the parameter $\nu$ changes the shape. For example, if
$\nu = 1/2 + n$, where $n \in \bbN_{+}$, then (see
\cite{abramowitz1964})
\[
\begin{split}
\phi(r;\rho) &= \exp   \bigg(-\frac{\sqrt{2\nu}r}{\rho} \bigg)
\frac{\Gamma(n + 1)}{\Gamma(2n + 1)} 
\sum_{k = 1}^{n} \frac{(n+1)!}{k!(n-k)!}
\bigg(
\frac{ \sqrt{8v} r }{ \rho } 
\bigg)^{n-k}
\end{split}
\]
and $\nu \rightarrow \infty \Rightarrow \phi(r;\btheta) \rightarrow
\exp \bigg(-\frac{r^2}{2\rho^2} \bigg)$. Note that even for a moderate
number of derivatives the number of terms will grow exponentially fast
leading to a very complex expression. This motivates the study of
complex analytical extensions of the covariance function. See Section
\ref{errorestimates} for more details.
%bbb

\label{multilevelapproach:remark1}
\end{remark}
The first step is to decompose the domain $\Gamma^{d}$ into a
multilevel domain decomposition. A good choice is based on the a
kD-tree decomposition of the space $\R^{d}$ \cite{Dasgupta2008}.
Other choices include
Projection (RP) tree \cite{Dasgupta2008}.
%This is a good choice for lower dimensions, however, as the number of
%dimensions $d$ becomes larger a better approach is to use a Random
%Projection (RP) tree \cite{Dasgupta2008}.
First start with the root node and cell $B^{0}_{0}$ at level $0$ that
contains all the observation nodes in $\bbS$. Now, split these nodes
into two children cells $B^{1}_{1}$ and $B^{1}_{2}$ at level $1$
according to the following rule:
\begin{enumerate}

\item Choose a unit vector $v$ in $\R^{d}$ along the axis of
  $\R^{d}$. This choice is the direction that leads to the maximum
  variance of the data in the cell along the direction of $v$.

\item Project all the nodes $\bx \in \bbS$ in the cell onto the unit
  vector $v$.

\item Split the cell with respect to the median
of the projections.

\end{enumerate}

For each cell $B^{1}_{1}$ and $B^{1}_{2}$ repeat the procedure until
there is at most $p$ nodes at the leaf nodes. Thus a binary tree is
obtained, which is of the form $B^{0}_{0}$, $B^{1}_{1}$, $B^{1}_{2}$,
$B^{2}_{3}$, $B^{2}_{4}$, $B^{2}_{5}$, $B^{2}_{6}$, $\dots $, where
$t$ is the maximal depth (level) of the tree.  Now, let $\mcB$ be the
set of all the cells in the tree and $\mcB^{n}$ be the set of all the
cells at level $0 \leq n \leq t$.  In addition, for each cell a unique
node number, current tree depth, threshold level and projection vector
are also assigned. This will be useful for searching the tree.
Algorithms \ref{RPMLB:algorithm1} and \ref{RPMLB:algorithm2-kd}
describe in more detail the construction of the kD-tree MB.
 
%\begin{remark}
%A kD-tree can also be constructed with Algorithms
%\ref{RPMLB:algorithm1} and \ref{RPMLB:algorithm2-kd}.
%\end{remark}

\begin{algorithm}[h]
  \KwIn{ $\bbS$, node, currentdepth, $n_0$} \KwOut{Tree, node}

\Begin{

\eIf {Tree = root}{node $\leftarrow$ 0, currentdepth $\leftarrow$ 0
Tree $\leftarrow$ MakeTree($\bbS$, node,
currentdepth + 1, $n_0$)
}
{

Tree.node = node

Tree.currentdepth = currentdepth - 1

node $\leftarrow$ node + 1

\If {$|\bbS| < n_0$}{return (Leaf)}





(Rule, threshold, $v$) $\leftarrow$ ChooseRule($\bbS$)

(Tree.LeftTree, node) 
$\leftarrow$ MakeTree($\bx \in \bbS$: Rule($\bx$) = True, node,
currentdepth + 1, $n_0$)

(Tree.RightTree, node)
$\leftarrow$ MakeTree($\bx \in \bbS$: Rule($\bx$) = false,  node, currentdepth + 1, $n_0$)

Tree.threshold = threshold\\
Tree.$v$ = $v$
}
}
\caption{MakeTree($\bbS$) function}
\label{RPMLB:algorithm1}
\end{algorithm}


%\begin{algorithm}[h]
%  \KwIn{ $\bbS$}
%  \KwOut{Rule, threshold, v}
%\Begin{
%choose a random unit vector $v$ \\
%Rule(x) := $x \cdot v  \leq$ threshold = median 
%$\{z \cdot v : z \in \bbS \}$
%}
%\caption{ChooseRule($\bbS$) function for RP tree}
%\label{RPMLB:algorithm2}
%\end{algorithm}

\begin{algorithm}[h]
  \KwIn{ $\bbS$}
  \KwOut{Rule, threshold, $v$}
\Begin{
    choose a coordinate direction that has maximal variance of the projection
    of the points in $\bbS$. \\
Rule(x) := $x \cdot v  \leq$ threshold = median
}

\caption{ChooseRule($\bbS$) function for kD-tree}
\label{RPMLB:algorithm2-kd}
\end{algorithm}


Now, suppose there is a one-to-one mapping between the set of unit
vectors $\mcE:=\{\bve_{1},\dots,\bve_{N}\}$, which is denoted as
leaf unit vectors, and the set of locations $\{
\bx_{1},\dots,\bx_{N}\}$, i.e. $\bx_{n} \longleftrightarrow \bve_{n}$
for all $n = 1, \dots, N$. It is clear that the span of the vectors
$\{\bve_{1},\dots,\bve_{N}\}$ is $\bbR^{N}$.  The next step is to
construct a new basis of $\R^{n}$ that is multilevel and orthonormal.

\setlength{\tabcolsep}{16pt}
\begin{figure*}
\begin{center}
  \begin{tabular}{c c}
\begin{tikzpicture}[scale=.65] 
  \begin{scope} 
 [place/.style={circle,draw=blue!50,fill=blue!20,thick,
     inner sep=0pt,minimum size=1.5mm}]

 \draw[step=8,gray,very thin] (0, 0) grid (8, 8);
    \draw (4,0) to (4,8);
    
    \draw (0,5) to (4,5);
    \draw (2.2,5) to (2.2,8);
    \draw (0,2) to (4,2);

    \draw (0,5) to (4,5);

    \draw (4,4.15) to (8,4.15);
    \draw (6.75,4.15) to (6.75,8);
    \draw (6,0) to (6,4.15);


    
  
    \node at (0.5,7.5) [place] {};
    \node at (0.3,6.3) [place] {};
    

    \node at (2.5,5.5) [place] {};
    \node at (3.2,5.2) [place] {};

    \node at (5,6) [place] {};
    \node at (3.8,5.5) [place] {}; %
    \node at (3.8,6) [place] {};   %


    \node at (0.5,3.5) [place] {};
    \node at (1.5,2.5) [place] {};
    \node at (2.3,2.2) [place] {};
    
    \node at (1.3,0.3) [place] {};
    \node at (2.7,0.5) [place] {};
    \node at (2.2,1.4) [place] {};
    \node at (2.6,1.4) [place] {};
    \node at (4.2,3.5) [place] {}; %
    \node at (3.7,3.3) [place] {};


    \node at (6.5,4.3) [place] {}; %
    \node at (7.5,5) [place] {};


    \node at (4.3,2.3) [place] {};
    \node at (5.7,3.5) [place] {};
    \node at (6.2,3.4) [place] {};
    \node at (7.3,2.4) [place] {};


    \node at (7,7) [place] {};
    \node at (6,7.5) [place] {};
    \node at (7.5,7.5) [place] {};


    \node at (6.5,2.0) [place] {};
    \node at (0.5,7.0) [place] {};
    \node at (2.0,7.0) [place] {};
    \node at (5,3.75) [place] {}; %
    \node at (6,7.0) [place] {};
    \node at (7,2.0) [place] {}; %
    %\node at (2.0,4.5) [place] {};
    \node at (7.5,4.3) [place] {}; %

    \node at (7.5,8.5) [] {$B^{0}_0$};
    \node at (0.6,5.5) [] {$B^{3}_{7}$};
    \node at (3,7) [] {$B^{3}_{8}$};
  \end{scope}
\end{tikzpicture} 
&
\begin{tikzpicture}[scale=0.85]
    %\node[anchor=center] at (0, -4.5) {$$};
    %\node[anchor=center] at (0,   10) {$$};
\begin{scope}[xshift=5cm, yshift=4cm,
place/.style={circle,draw=blue!50,fill=blue!20,thick,
      inner sep=0pt,minimum size=1.5mm},
placer/.style={circle,draw=blue!50,
  preaction={fill=darkgreen!60,fill opacity=0.5}, thick,inner
  sep=0pt,minimum size=1.5mm}, ]

  %placer/.style={circle,draw=blue!50,
  %preaction={fill=darkgreen!60,fill opacity=0.5}, thick,inner
  %sep=0pt,minimum size=1.5mm}, ]


%\filldraw[fill={rgb:red,143;green,188;blue,143},semitransparent, 
%      thick] (0, 0) rectangle (16, 16);


  
\Tree [.\node[placer]{$B^{0}_{0}$}; 
             [.\node[placer]{$B^{1}_{1}$};
                    [.\node[placer]{$B^{2}_{3}$}; 
                           [.\node[placer]{$B^{3}_{7}$};]
                           [.\node[placer]{$B^{3}_{8}$};] 
                    ]       
                    [.\node[placer]{$B^{2}_{4}$}; 
                           [.\node[placer]{$B^{3}_{9}$};] 
                           [.\node[placer]{$B^{3}_{10}$};] 
                    ] 
             ]                                        
             [.\node[placer]{$B^{1}_{2}$};
                    [      [.\node[placer]{$B^{2}_{5}$};
                                  [.\node[placer]{$B^{3}_{11}$};] 
                                  [.\node[placer]{$B^{3}_{12}$};] 
                           ]
                           [.\node[placer]{$B^{2}_{6}$}; 
                                  [.\node[placer]{$B^{3}_{13}$};] 
                                  [.\node[placer]{$B^{3}_{14}$};] 
                           ] 
                                          ]]
] 

\end{scope}
\end{tikzpicture}
\end{tabular}
\end{center}
\caption{Multilevel domain decomposition of the observations.}
\label{MLRLE:fig1}
\end{figure*}



\begin{enumerate}[(a)]
\item Start at the maximum level of the random projection tree,
  i.e. $q = t$.
\item For each leaf cell $B^{q}_{k} \in \mcB^{q}$ assume without loss
  of generality that there are $s$ observations nodes $\bbS^{q}_{k}:=\{
  \bx_1, \dots, \bx_s \}$ with associated vectors $C_k^{q} := \{
  \bve_1, \dots, \bve_s \}$.
  %Let $\mcE^q_k := \{\bx_1,\dots,\bx_s\}$
  %and
  Denote $\mcC^{q}_{k}$ as the span of the vectors in $C_k^{q}$.
\begin{enumerate}[i)]

\item Let $\bphi^{q,k} _{j} := \sum_{\bve_i \in C^q_k} c^{q,k} _{i,j}
  \bve_i, \hspace{2mm} j=1, \dots, a;
\hspace{2mm} \bpsi^{q,k}_{j} := \sum_{\bve_i \in C^q_k} d^{q,k}_{i,j}
\bve_i, \hspace{2mm} j=a+1, \dots, s$, where $c^{q,k}_{i,j}$,
$d^{q,k}_{i,j} \in \mathbb{R}$ and for some $a \in \mathbb{N}^{+}$. Note
that $a$ is unknown up to this point, but will be computed from the
data.  It is desired that the new discrete MB vector $\bpsi^{q,k}_{j}$
be orthogonal to $\mcP^{p}(\mathbb{S})$, i.e., for all $g \in \mcP^{
  p}(\mathbb{S})$:
\begin{equation}
\sum_{i=1}^{n} g[i] \bpsi^{q,k}_{j}[i] = 0
\label{hbconstruction:eqn1}
\end{equation}

\item Form the matrix $\mcM^{q,k} := \bX \T \bV^{q,k}$, where
  $\mcM^{q,k} \in \R^{p \times s}$, $\bV^{q,k} \in \R^{N \times s}$,
  and $\bV^{q,k}: = [\bve_1, \dots, \bve_i, \dots,\bve_s ]$ for all $\bve_i
  \in C_k^q$. Now, suppose that the matrix $\mcM^{q,k} $ has rank $a$
  and then perform the Singular Value Decomposition (SVD). Denote by
  $\bU \bD \bV $ the SVD of $\mcM^{q,k} $, where $\bU \in \R^{ p \times
    p}$, $\bD \in \R^{p \times s}$, and $\bV \in \R^{s \times s} $.

  \begin{remark} Note that in practice we only keep track of the
    non-zero elements of the vectors $\bve_1, \dots, \bve_s$. Thus the
    computational cost is reduced significantly. This is taken into
    account in the complexity analysis in Lemma
    \ref{MultilevelREML:lemma1} and \ref{MultilevelREML:lemma2}
  \end{remark}
  
\item Following the same argument as in \cite{Castrillon2015} but
  adapted to the kd-tree decomposition equation
  \eqref{hbconstruction:eqn1} is satisfied with the following choice
\[
  \left[ \begin{array}{ccc|ccc}
      c^{q,k}_{0,1} & \dots &c^{q,k}_{a,1} & d^{q,k}_{a+1,1} & \dots &d^{q,k}_{s,1} \\
      c^{q,k}_{0,2} & \dots &c^{q,k}_{a,2} & d^{q,k}_{a+1,2} & \dots &d^{q,k}_{s,2} \\
      \vdots & \vdots & \vdots & \vdots & \vdots & \vdots   \\
      c^{q,k}_{0,s} & \dots &c^{q,k}_{a,s} & d^{q,k}_{a+1,s} & \dots &d^{q,k}_{s,s}
    \end{array}
\right] := \bV\T.
% \label{eqDefVspT*}
  \]
\noindent For this choice the coefficient $a$ is equal to the number
of non-zero singular values. Thus the columns $a+1$, \dots, $s$ form
an orthonormal basis of the nullspace ${N_0}(\mcM^{q,k} )$. Similarly,
the columns $1,\dots, a$ form an orthonormal basis of $\R^s \backslash
{N_0}(\mcM^{q,k})$. Since the vectors in $C^q_k$ are orthonormal then
$\bphi^{q,k}_{1}, \dots, \bphi^{q,k}_a$, $\bpsi^{q,k}_{a+1}, \dots,$
$\bpsi^{q,k}_s$ form an orthonormal basis of $\mcC^{q}_{k}$.  Moreover
$\bpsi^{q,k}_{a+1}, \dots, \bpsi^{q,k}_s$ satisfy equation
\eqref{hbconstruction:eqn1}, i.e., are orthogonal to
$\mcP^{p}(\mathbb{S})$ and are locally adapted to the locations
contained in the cell $B^{q}_{k}$.

\item Denote by $D_k^{q,k}$ the collection of all the vectors
  $\bpsi^{q,k}_{a+1}, \dots, \bpsi^{q,k}_s$. Notice that the vectors
  $\bphi^{q,k}_{1}, \dots,$ $\bphi^{q,k}_a$, which are denoted with a
  slight abuse of notation as the scaling vectors, are {\it not}
  orthogonal to $\mcP^{p}(\mathbb{S})$. They need to be further
  processed.

\item Let $\mcD^{q}$ be the union of the vectors in $D^{q}_k$ for
  all the cells $B^{q}_k \in \mcB^{q}_{k}$. Denote by
  $W_{q}(\mathbb{S})$ as the span of all the vectors in $\mcD^{q}$.

\end{enumerate}



\item The next step is to go to level $q - 1$. For any two sibling
  cells denote $B^{q}_{\tt{left}}$ and $B^{q}_{\tt{right}}$ at level $q$ denote
  $C^{q-1}_{\tilde k}$ as the collection of the scaling functions from
  both cells, for some index $\tilde k$.


\item Let $q: = q - 1$. If $B^{q}_{k} \in \mcB^{q}$ is a leaf cell
  then repeat steps (b) to (d). However, if $B^{q}_{k} \in \mcB^{q}$
    is not a leaf cell, then repeat steps (b) to (d), but replace the
    leaf unit vectors with the scaling vectors contained in $C^{q}_k$
    with $C^{q-1}_{\tilde k}$.


  \item When $q = -1$ is reached stop.
  %repeat steps (b) to (d), but replace
  %$p$ with $p$, e.g. $\mcP^{p}(\bbS)$ with
  %$\mcP^{p}(\bbS)$. The ML basis vectors will span the space
  %$W_{-1}(\bbS) : =\mcP^{p}(\bbS) \backslash \mcP^{p}(\bbS)$.


\end{enumerate}

When the algorithm stops a series orthogonal subspaces
$V_{0}(\bbS), W_{0}(\mathbb{S}),\dots, W_{t}(\mathbb{S})$ (and their corresponding
basis vectors) are obtained. These subspaces are orthogonal to
$V_{0}(\mathbb{S}) : = span \{ \phi_{1}^{0}, \dots, \phi_{p}^{0}
\}$. Note that the orthonormal basis vectors of $V_{0}(\mathbb{S})$
also span the space $\mcP^{p}(\mathbb{S})$.
\begin{remark}
Following Lemma 2 in \cite{Castrillon2013} it can be shown that
\[
\R^{N} = \mcP^{p}(\mathbb{S}) \oplus
%W_{-1}(\mathbb{S}) \oplus
W_{0}(\mathbb{S}) 
\oplus W_{1}(\mathbb{S})
\oplus \dots \oplus W_{t}(\mathbb{S}),
\]
%where $W_{-1}(\bbS) : =\mcP^{\tilde p}(\bbS) \backslash
%\mcP^{p}(\bbS)$.
Also, it can then be shown that at most $\mcO(Nt)$
computational steps are needed to construct the multilevel basis of
$\R^{N}$.
\end{remark}

From the basis vectors of the subspaces $\mcP^{p}(\mathbb{S})^{\perp}
= \cup_{i=0}^{t} W_{i}(\mathbb{S})$ an orthogonal projection matrix
$\bW:\R^{N} \rightarrow (\mcP^{p}(\mathbb{S}))^{\perp}$ can be built.
The dimensions of $\bW$ is $(N - p) \times N$ since the total number
of orthonormal vectors that span $\mcP^{p}(\mathbb{S})$ is
$p$. Conversely, the total number of orthonormal vectors that span
$\mcP^{p}(\mathbb{S})^{\perp}$ is $N-p$.

Let $\bL$ be a matrix where each row is an orthonormal basis vector of
$\mcP^{p}(\mathbb{S})$. For $i = 0,\dots,t$ let $\bW_i$ be a matrix
where each row is a basis vector of the space $W_i(\mathbb{S})$. The
matrix $\bW \in \mathbb{R}^{(N - p) \times N}$ can now be formed,
where $\bW := \left[ \bW_t\T, \dots, \bW_0\T \right] \T$.

Following a similar approach to Lemma 2.11 in \cite{Castrillon2013} it
can be shown that:
\begin{enumerate}[a)]
\item 
The matrix $\bP := \left[
\begin{array}{c}
\bW \\
\bL
\end{array}
\right ]$ is orthonormal, i.e., $\bP\bP\T= \bI$.

\item Any vector $\bv
\in \R^{n}$ can be written as $\bv = \bL\T\bv_{L} + \bW\T\bv_{\bW}$
where $\bv_{L} \in \R^{p} $ and $\bv_{\bW} \in \R^{N-p}$ are unique.

\end{enumerate}





The following useful lemmas are proved:
\begin{lemma} Assuming that $n_0 < 2p$,
for any level $q=0,\dots,t$ there is at most $p2^{q}$ multilevel
basis vectors.
%For level $q = -1$ there is at most $p -
%\tilde p$ multilevel vectors.
\label{MultilevelREML:lemma1}
\end{lemma}
\begin{proof}
Starting at the finest level $t$, for each cell $B^{t}_k \in \mcB^{t}$
there is at most $p$ multilevel vectors.  Since there is at most
$2^t$ cells then there is at most $2^{t} p$ multilevel vectors.

Now, for each pair of left and right (siblings) cells at level $t$ the
parent cell at level $t-1$ will have at most $2 p$ scaling
functions. Thus at most $p$ multilevel vectors and $p$ scaling
vectors are obtained that are to be used for the next level. Now, the
rest of the cells at level $t$ are leafs and will have at most $p$
multilevel vectors and $p$ scaling vectors that are to be used for
the next level. Since there is at most $2^{t-1}$ cells at level $t-1$,
there is at most $2^{t-1} p$ multilevel vectors. Now, follow an
inductive argument until $q = 0$ and the proof is done.
\end{proof}



\begin{lemma} Assuming that $n_0 < 2p$ for any level $q = 0, \dots, t$
  any multilevel vector $\bpsi^{q}_m$ associated with a cell $B^{q}_k
  \in \mcB^{q}$ has at most $2^{t-q+1} p$ non zero entries.
\label{MultilevelREML:lemma2}
\end{lemma}
\begin{proof} For any leaf cell at the bottom of the tree (level $t$)
  there is at most $2 p$ observations.
  %thus the number of non zero entries of level $q$ multilevel
  %vectors is $2 p$.  Combining the left and right cells, the parent
  cell has at most $4 p$ observations, thus the associated multilevel
  vectors has $4p$ non zero entries. By induction at any level $l$ the
  number of nonzero entries is at most $2^{t-q+1} p$.  Now for any
  leaf cell at any other level $l < t$ the number of nonzero entries
  is at most $2 p$. Following an inductive argument the result is
  obtained.
\end{proof}


From Lemma \ref{MultilevelREML:lemma1} and \ref{MultilevelREML:lemma2}
it can be shown that the matrix $\bW$ contains at most $\mcO(Nt)$
non-zero entries and $\bL$ contains at most $\mcO(Np)$ non-zero
entries. Thus for any vector $\bv \in \R^{n}$ the matrix vector
products $\bW \bv$ and $\bL \bv$ are respectively calculated with at
most $\mcO(Nt)$ and $\mcO(Np)$ computational steps.


%% Multilevel Covariance Matrix --------------------------------------
%%
\section{Multilevel covariance matrix}
\label{MultilevelCovarianceMatrix}

The multilevel covariance matrix $\bC_{\bW}(\btheta)$ and sparse
version $\tilde \bC_{\bW}(\btheta)$ can be now constructed.  Recall from
the discussion in Section \ref{multilevelapproach} that
$\bC_{\bW}(\btheta):=\bW \bC(\btheta) \bW \T$. From the multilevel
basis construct in Section \ref{MultilevelREML} the following
operator is built: $\bW := \left[ \bW_t\T, \dots, \bW_0\T
  \right] \T$. Thus the covariance matrix $\bC(\btheta)$ is
transformed into $\bC_{\bW}(\btheta)$, where each of the blocks
$\bC^{i,j}_{\bW}(\btheta) = \bW_i \bC(\btheta) \bW_j \T$ are formed from
all the interactions of the MB vectors between levels $i$ and $j$, for
all $i,j = 0, \dots, t$. The structure of $\bC_{\bW}(\btheta)$ is shown
in Figure \ref{multilevelcov:fig1}.  Thus for any
$\bpsi^{i}_{\tilde{l}}$ and $\bpsi^{j}_{\tilde{k}}$ vectors there is a
unique entry of $\bC^{i,j}_{\bW}$ of the form
$(\bpsi^{i}_{\tilde{k}})\T \bC(\btheta) \bpsi^{j}_{\tilde{l}}$.

%The blocks $\bC_{\bW}^{i,j}$, where
%$i=-1$ or $j=-1$, correspond to the case where the accuracy term
%$\tilde{p} > p$.


In Section \ref{errorestimates} we show that far field entries of
$\bC_{\bW}(\btheta)$, i.e. $(\bpsi^{i}_{\tilde{k}})\T \bC(\btheta)
\bpsi^{j}_{\tilde{l}}$, decay sub-exponentially with respect to
$p(d,w)$ if there exists an analytic extension of the covariance
function on a well defined domain in $\bC^{d}$. Thus it is not
necessary to compute all the entries. We introduce a distance
criterion approach to produce a sparse matrix $\tilde
\bC_{\bW}(\btheta)$.

\subsection{Sparsification of multilevel covariance matrix}

A sparse version of the covariance matrix $\bC_{\bW}(\btheta)$ can be
built by using a level and distance dependent strategy:

\begin{enumerate}[i)]

\item Given a cell $B^{i}_{k}$ at level $i \geq 0$ identify the
  corresponding tree node value Tree.node and the tree depth
  Tree.currentdepth. Note that the

  Tree.currentdepth and the MB level
  $q$ are the same for $q = 0,\dots,t$.
  %However, for $q = -1$ the MB
  %is associated to the Tree.currentdepth = 0.

\item Let $\bbK \subset \bbS$ be all the observations nodes contained
  in the cell $B^{i}_{k}$.

\item Let $\tau_{i,j} \geq 0$ be the distance parameter given by the
  user corresponding to the level $i,j$ from the block
  $\bC^{i,j}_{\bW}(\btheta)$.

\item Let the Targetdepth be equal to the desired level of the tree.
%In the case that it is $-1$ then the Targetdepth is zero.

\end{enumerate}
    The objective now is to find all the cells at the Targetdepth that
    overlap a hyper rectangle which is extended from $B^{i}_{k}$.  For
    all observations $\bx \in B^{i}_{l}$ along each dimension $k = 1,
    \dots, d$ let $x^{min}_k := \min_{ x_k \in B^i_m} x_k$ and
    $x^{max}_k := \max_{ x_k \in B^i_m} x_k$.  Any cell that
    intersects the interval $[x^{min}_{k} - \tau_{i,j} ,x^{max} +
      \tau_{i,j}]$ is included. This is done by searching the tree
    from the root node. At each traversed node check that all the
    nodes $\bx \in \bbK$ satisfy the following rule: If
\[
\bx \cdot \mbox{Tree}.v + \tau_{i,j} \leq Tree.threshold.
\]
then search down the left tree. If 
\[
\bx \cdot \mbox{Tree}.v - \tau_{i,j} > Tree.threshold.
\]
the search down the right tree. Otherwise search both trees.
%If this is true for all $\bx \in \bbK$ then the search continues down
%the left tree.  If this is false for all $\bx \in \bbK$ then the
%search continues down the Right tree, otherwise both the left and
%right tree are searched.
The full search algorithm is described in Algorithms
\ref{MLCM:algorithm3}, \ref{MLCM:algorithm4}, \ref{MLCM:algorithm5}
and \ref{MLCM:algorithm5a}.


\begin{algorithm}[htp]
  \KwIn{Tree, $\bbK$, Targetdepth, $\tau_{i,j}$}
  \KwOut{Targetnodes}
\Begin{
    Targetnodes $\leftarrow \emptyset $
    Targetnodes $\leftarrow$ 
    LocalSearchTree(Tree, $\bbK$, Targetdepth, $\tau_{i,j}$, Targetnodes);
}
\caption{SearchTree function(Tree, $\bbK$, Targetdepth, $\tau_{i,j}$)}
\label{MLCM:algorithm3}
\end{algorithm}






\begin{algorithm}[htp]
  \KwIn{Tree, $\bbK$, Targetdepth, $\tau_{i,j}$, Targetnodes}
  \KwOut{Targetnodes}
\Begin{

\If {Targetdepth = Tree.currentdepth}{return
Targetnodes = Targetnodes $\cup$ Tree.node}

\If {Tree = leaf}{return}

LeftRule  =  ChooseLeftRule($\bbK$, Tree, $\tau_{i,j}$)\\
RightRule =  ChooseRightRule($\bbK$, Tree, $\tau_{i,j}$)\\



\uIf {LeftRule($\bx$)=true $\forall \bx \in \bbK$}
{Targetnodes $\leftarrow$ LocalSearchTree(Tree.LeftTree, 
  $\bbK$, Targetdepth, $\tau_{i,j}$, Targetnodes)}

\uElseIf{RightRule($\bx$)=true $\forall \bx \in \bbK$}
{Targetnodes $\leftarrow$ LocalSearchTree(Tree.RightTree, 
  $\bbK$, Targetdepth, $\tau_{i,j}$, Targetnodes)}

\Else{Targetnodes $\leftarrow$ LocalSearchTree(Tree.LeftTree, 
$\bbK$, Targetdepth, $\tau_{i,j}$, Targetnodes)\\
Targetnodes $\leftarrow$ LocalSearchTree(Tree.RightTree, 
$\bbK$, Targetdepth, $\tau_{i,j}$, Targetnodes)}

%\eIf {Rule($\bx$)=true $\forall \bx \in \bbK$}
%{Targetnodes $\leftarrow$ LocalSearchTree(Tree.LeftTree, 
%$\bbK$, Targetdepth, $\tau$, Targetnodes)}
%{Targetnodes $\leftarrow$ LocalSearchTree(Tree.LeftTree, 
%$\bbK$, Targetdepth, $\tau$, Targetnodes)
%Targetnodes $\leftarrow$ LocalSearchTree(Tree.RightTree, 
%$\bbK$, Targetdepth, $\tau$, Targetnodes)
%}

}
\caption{LocalSearchTree(Tree, $\bbK$, Targetdepth, $\tau_{i,j}$) function}
\label{MLCM:algorithm4}
\end{algorithm}


\begin{algorithm}[htp]

  \KwIn{ $\bbK$, Tree, $\tau_{i,j}$ }
  \KwOut{Rule}
\Begin{
Rule($\bx$) := $\bx \cdot \mbox{Tree}.v + \tau_{i,j} \leq $
        Tree.threshold
 }
\caption{ChooseLeftRule($\bbK$) function}
\label{MLCM:algorithm5}
\end{algorithm}


\begin{algorithm}[htp]

  \KwIn{ $\bbK$, Tree, $\tau_{i,j}$ }
  \KwOut{Rule}
\Begin{
Rule($\bx$) := $\bx \cdot \mbox{Tree}.v - \tau_{i,j} > $
        Tree.threshold
 }
\caption{ChooseRightRule($\bbK$) function}
\label{MLCM:algorithm5a}
\end{algorithm}

In Figure \ref{multilevelcov:fig2}(b) an example for searching local
neighborhood cells of randomly placed observations in $\R^{2}$ is
shown. The orange nodes correspond to the source cell. By choosing a
suitable value for $\tau_{i,j}$ the blue nodes in the immediate cell
neighborhood are found by using Algorithms \ref{MLCM:algorithm3},
\ref{MLCM:algorithm4}, \ref{MLCM:algorithm5} and
\ref{MLCM:algorithm5a}.

The sparse matrix blocks $\bC^{i,j}_{\bW}(\btheta)$ can be built from
all the cells that are obtained from SearchTree function of Algorithm
\ref{MLCM:algorithm5}. Compute all the entries of
$\bC^{i,j}_{\bW}(\btheta)$ that correspond to the interactions between
any two cells $B^{i}_k \in \mcB^{i}$ and $B^{j}_l \in \mcB^{j}$. In
Algorithm \ref{MLCM:algorithm6}) the construction of the sparse matrix
$\tilde \bC^{i,j}_{\bW}(\btheta)$ is shown.

\begin{remark}
Since the matrix $\tilde \bC_{\bW}(\btheta)$ is symmetric it is only
necessary to compute the blocks $\bC^{i,j}_{\bW}(\btheta)$ for $i = 1,
\dots, t$ and $j = i, \dots t$.
\end{remark}

\begin{algorithm}
  \KwIn{Tree, $i$, $j$, $\tau_{i,j}$, $\mcB^i$, $\mcB^j$,
    $\mcD^i$, $\mcD^i$, $\bC(\btheta)$} \KwOut{$\tilde
    \bC^{i,j}_{\bW}(\btheta)$}
  \Begin{
        Targetnodes $\leftarrow \emptyset$\\
        \For{$B^{i}_{m} \in \mcB^{i}$}
            {$\bbK \leftarrow B^{i}_{m}$\\
            \For{$B^{j}_{q} \leftarrow $
            SearchTree(Tree, $\bbK$, Targetdepth $(i)$, $\tau_{i,j}$, 
            Targetnodes)}
            {
              \For{$\psi^i_k \in D^{i}$}{
                \For{$\psi^j_l \in D^{j}$}{
                  Compute $(\bpsi^{i}_{k})\T \bC(\btheta) \bpsi^{j}_{l}$ 
                  in $\tilde \bC^{i,j}_{\bW}(\btheta)$
                }
              }
            }
            }
    }
\caption{Construction of sparse matrix $\tilde \bC^{i,j}_{\bW}(\btheta)$}
\label{MLCM:algorithm6}
\end{algorithm}

\begin{figure}
\begin{center}
\begin{tikzpicture} 
  \begin{scope}[scale = 0.5]
    [place/.style={circle,draw=blue!50,fill=blue!20,thick,
      inner sep=0pt,minimum size=1.5mm}]
    %\draw[fill=red!5, step=16, thick] (0, 0) grid (16, 16);

    \filldraw[fill={rgb:red,143;green,188;blue,143},semitransparent, 
      thick] (0, 0) rectangle (16, 16);

    % \draw[blue, very thick] (0,0)rectangle (3,2);

    \draw[thin] (8,0) to (8,16);
    \draw[thin] (0,8) to (16,8);
    \draw[thin] (14,0) to (14,16);
    \draw[thin] (0,2) to (16,2);
    \draw[thin] (12,0) to (12,16);
    \draw[thin] (0,4) to (16,4);

    \node at (15,1) [] {${\bf G}_{\bW}$ };
    \node at (10,12) [] {$\bC^{t,t-1}_{\bW}(\btheta)$};
    \node at (4.5,6) [] {$\bC^{t-1,t}_{\bW}(\btheta)$};
    \node at (4.5,1) [] {$\bC^{0,t}_{\bW}(\btheta)$};
    \node at (10,6) [] {$\ddots$};
    \node at (13,1) [] {$\dots$};
    \node at (15,3) [] {$\vdots$};
    \node at (4.5,12) [] {$\bC^{t,t}_{\bW}(\btheta)$};

\end{scope}
\end{tikzpicture}
\end{center}
\caption{Multilevel covariance matrix where ${\bf G}_{\bW}
:=\bC^{0,0}_{\bW}(\btheta)$.}
\label{multilevelcov:fig1}
\end{figure}

\begin{figure*}[ht]
\begin{center}

  %\includegraphics[trim = 410 60 370 50, clip, width=4in,
  %  height=4in]{./figures/neighborhoodpattern.pdf}

 \begin{tikzpicture}[scale=0.59, every node/.style={scale=0.59}]
  \begin{scope} 
    [place/.style={circle,draw=blueish,fill=blueish,
        inner sep=0pt,minimum size=1.5mm},
      placegray/.style={circle,draw=gray!50,fill=gray!20,
        inner sep=0pt,minimum size=1.5mm},
        placenew/.style={circle,draw=darkorange!75,fill=darkorange!75,
      inner sep=0pt,minimum size=1.5mm}]
    \draw[step=8,gray,very thin] (0, 0) grid (8, 8);
    
    \node at (15,3.88) [] {\includegraphics[trim = 14.5cm 2cm 10cm 1.75cm,
        clip=true,
        height=8.43cm]{neighborhoodpattern.pdf}};
    
    \draw (2.2,5) to (2.2,8);
    \draw (0,5) to (2.2,5);
    \draw (6,4.15) to (8,4.15);
    \draw (6,0) to (6,4.15);

    \draw (4,0) to (4,8);
    \draw (0,5) to (4,5);
    
    \draw (0,2) to (4,2);
    \draw (0,5) to (4,5);
    \draw (4,4.15) to (8,4.15);
    \draw (6.75,4.15) to (6.75,8);
        
    \draw[dashed,gray] (6,4.15) to (6,8);
    \draw[dashed,gray] (0,2.3) to (6,2.3);
  
    \node at (0.5,7.5) [placenew] {};
    \node at (0.3,6.3) [placenew] {};
    
    \node at (2.5,5.5) [place] {};
    \node at (3.2,5.2) [place] {};

    \node at (5,6) [place] {};
    \node at (3.8,5.5) [place] {}; %
    \node at (3.8,6) [place] {};   %

    \node at (0.5,3.5) [place] {};
    \node at (1.5,2.5) [place] {};
    \node at (2.3,2.2) [place] {};
    
    \node at (1.3,0.3) [placegray] {};
    \node at (2.7,0.5) [placegray] {};
    \node at (2.2,1.4) [placegray] {};
    \node at (2.6,1.4) [placegray] {};
    \node at (4.2,3.5) [place] {}; %
    \node at (3.7,3.3) [place] {};

    \node at (6.5,4.3) [place] {}; %
    \node at (7.5,5) [placegray] {};

    \node at (4.3,2.3) [place] {};
    \node at (5.7,3.5) [place] {};
    \node at (6.2,3.4) [placegray] {};
    \node at (7.3,2.4) [placegray] {};


    \node at (7,7) [placegray] {};
    \node at (6,7.5) [place] {};
    \node at (7.5,7.5) [placegray] {};


    \node at (6.5,2.0) [placegray] {};
    \node at (0.5,7.0) [placenew] {};
    \node at (2.0,7.0) [placenew] {};
    \node at (5,3.75) [place] {}; %
    \node at (6,7.0) [place] {};
    \node at (7,2.0) [placegray] {}; %
    \node at (7.5,4.3) [placegray] {}; %

    \node at (4,8.75) [] {\Large $\tau_{i,j}$};
    \node at (-1,4.25) [] {\Large $\tau_{i,j}$};

    \draw[dashed,gray] (2,6.3) to (2,8);
    \draw[dashed,gray] (0,6.3) to (2,6.3);
    
    \coordinate (A) at (2,8);
    \coordinate (B) at (6,8);
    \coordinate (C) at (0,6.3);
    \coordinate (D) at (0,2.3);

\draw[dim={,10pt}]  (A) --  (B);
\draw[dim={,-15pt}]  (C) --  (D);

\node at (4,-0.65) {\Large (a)};
\node at (14.5,-0.65) {\Large (b)};
  \end{scope}
\end{tikzpicture} 


\end{center}



\caption{Neighborhood identification from source cell on a random
  kD-tree decomposition of observation locations in $\R^{2}$. (a)
  Cartoon example of axis wise distance criterion $\tau_{i,j}$ using
  Algorithms \ref{MLCM:algorithm3}, \ref{MLCM:algorithm4},
  \ref{MLCM:algorithm5} and \ref{MLCM:algorithm5a}. The orange
  observations knots correspond to the source cell. The blue knots
  correspond to all the target nodes. The gray knots are not included
  in the list of target nodes.  (b) Example of local neighborhood
  contained in the axis wise distance $\tau_{i,j}$.  The orange nodes
  are contained in the source cell. The blue nodes are are contained
  in the local neighborhood cells. The grey dots are all the
  observations that are not part of the source or local neighborhood
  cells.}
\label{multilevelcov:fig2}
\end{figure*}

\subsection{Computational cost of the multilevel
  matrix blocks of $\tilde{\bC}_{\bW}$}


The cost of computing the multilevel blocks $\tilde{\bC}^{i,j}_{\bW}$
will in general be $\mcO(N^2)$. However, for the special case that $d
= 2$ and $d = 3$ it is possible to use a fast summation method such as
the Kernel Independent Fast Multipole Method (KIFMM) by
\cite{ying2004} to compute the blocks more efficiently. To my
knowledge, there exists no equivalent fast summation method in higher
dimensions that works satisfactorily.

This KIFMM algorithm is flexible and efficient for computing the
matrix vector products $\bC(\btheta)\bx$ for a large class of kernel
functions, including the Mat\'{e}rn covariance function.  Given
$\tilde N$ sources and $\tilde M$ targets, experimental results show a
computational cost of about $\mcO(\tilde N + \tilde M)$, $\alpha
\approx 1$ with good accuracy ($\varepsilon_{FMM}$ between $10^{-6}$
to $10^{-8}$) with a slight degrade in the accuracy with increased
source nodes.


\begin{asum} Let $\bA(\btheta) \in \R^{\tilde M \times \tilde N}$ be a kernel
matrix formed from $\tilde N$ source observation nodes and $\tilde M$
target nodes in the space $\R^{d}$.  Suppose that there exists a fast
summation method that computes the matrix-vector products
$\bA(\btheta)\bx$ with $\varepsilon_{FMM}>0$ accuracy in $\mcO((\tilde
N + \tilde M)^{\alpha})$ computations, for some $\alpha \geq 1$ and
any $\bx \in \R^{d}$.
\end{asum}

%Given an octree multilevel tree domain decomposition in $\R^{3}$, as
%shown in \cite{Castrillon2013,Castrillon2015}, the authors described
%how to apply a Kernel Independent Fast Multipole Method (KIFMM) by
%\cite{ying2004} to compute all the blocks
%$\tilde{\bC}^{i,i}_{\bW}(\btheta) \in \R^{\tilde N \times \tilde N}$ for
%$i = 0,\dots,t$ in $\mcO(\tilde N(t+1)^2)$ computational steps to a
%fixed accuracy $\varepsilon_{FMM} > 0$.

For the kD-tree it is not possible to determine a-priori the sparsity
of the blocks $\tilde{\bC}^{i,j}_{\bW}(\btheta)$.
%given the level dependent distance parameter $\tau_{i,j}$.
However, for a given a value $\tau_{i,j} \geq 0$ by running Algorithm
\ref{MLCM:algorithm3} on every cell $B^{i}_k \in \mcB^{i}$, at level
$i$, with the Targetdepth corresponding for level $j$ it is possible
to determine the computational cost of constructing the sparse blocks
$\tilde{\bC}^{i,j}_{\bW}(\btheta)$ under the following assumption. Suppose
that maximum number of cells $B^{j}_k \in \mcB^{j}$ given by Algorithm
\ref{MLCM:algorithm3} is bounded by some $\gamma^{i,j} \in \bbN_+$.



\begin{prop} 
  The cost of computing each block $\tilde{\bC}^{i,j}_{\bW}(\btheta)$
for $i,j = 1,\dots,t$ by using a fast summation method with $1 \leq
\alpha \leq 2$ is bounded by
\[
\mcO(\gamma_{i,j} p 2^{i} (2^{t-j+1} p + 2^{t-i+1} p)^{\alpha} + 2p 2^{t}).
\]
\end{prop} 
\begin{proof} Let us look at the cost of computing all the 
interactions between any two cells $B^{i}_k \in \mcB^{i}$ and
$B^{j}_l \in \mcB^{j}$. Without loss of generality assume that $i
\leq j$. For the cell $B^{l}_k$ there is at most $p$
multilevel vectors and from Lemma \ref{MultilevelREML:lemma2}
$2^{t-i+1} p$ non zero entries. Similarly for $B^{j}_l$.  All
the interactions $(\bpsi^{i}_{\tilde{k}})\T \bC(\btheta)
\bpsi^{j}_{\tilde{l}}$ now have to be computed, where
$\bpsi^{i}_{\tilde{k}} \in B^{i}_k$ and $\bpsi^{j}_{\tilde{l}} \in
B^{j}_l$.

The term $\bC(\btheta) \bpsi^{j}_{\tilde{l}}$ is computed using a FMM
with $2^{t-j+1} p$ sources and $2^{t-i+1} p$ targets at a cost of
$\mcO($ $(2^{t-j+1} p + 2^{t-i+1} \tilde p)^{\alpha})$.  Since there
is at most $p$ multilevel vectors in $B^{i}_k$ and $B^{j}_l$ then the
cost for computing all the interactions $(\bpsi^{i}_{\tilde{k}})\T
\bC(\btheta) \bpsi^{j}_{\tilde{l}}$ is $\mcO(p(2^{t-j+1} p + 2^{t-i+1}
p)^{\alpha} + 2^{t-i+1}p)$.

Now, at any level $i$ there is at most $2^{i}$ cells, thus the result
follows.
\end{proof}



%% Multilevel Estimator and Predictor --------------------------------
%%

\section{Multilevel estimator and predictor}
\label{multilevelestimator}
The multilevel random projection tree can be exploited in such a way
to significantly reduce the computational burden and to further
increase the numerical stability of the estimation and prediction
steps. This is an extension of the multilevel estimator and predictor
formulated in \cite{Castrillon2015} to binary tree in higher
dimensions. The former is based on Oct-tree decompositions, thus
making it unsuitable for higher dimensional problems.

\subsection{Estimator}

The multilevel likelihood function, $l_{\bW}(\theta)$ (see equation
\eqref {Introduction:multilevelloglikelihood} ), has the clear
advantage of being decoupled from the vector $\bbeta$. Furthermore,
the multilevel covariance matrix $\bC_{\bW}(\btheta)$ will be more
numerically stable than $\bC(\btheta)$ thus making it easier to invert
and to compute the determinant.

However, it is not necessary to perform the MLE estimation on the full
covariance matrix $\bC_{\bW}(\btheta)$, instead construct a series of
multilevel likelihood functions $\tilde{\ell}^n_{\bW}(\btheta)$, $n =
0,\dots t$, by applying the partial transform $[\bW_t \T, \dots, \bW_n
  \T ]$ to the data $\bZ$ where
\begin{equation}
  \begin{split}
\tilde{\ell}^n_{\bW}(\btheta)
=-\frac{\tilde{N}}{2}\log(2\pi)-\frac{1}{2}\log
\det\{\tilde{\bC}_{\bW}^n(\btheta)\}
-\frac{1}{2}(\bZ^n_{\bW})\T\tilde{\bC}^n_{\bW}(\btheta)^{-1}\bZ^n_{\bW},
\end{split}
\label{Introduction:multilevelloglikelihoodreduced2}
\end{equation}
where $\bZ^n_{\bW} :=[\bW_t \T, \dots, \bW_i \T ] \T \bZ$,
$\tilde{N}$ is the length of $\bZ^n_{\bW}$,
$\tilde{\bC}^n_{\bW}(\btheta)$ is the $\tilde{N} \times \tilde{N}$
upper-left sub-matrix of $\tilde{\bC}_{\bW}(\btheta)$ and
$\bC^n_{\bW}(\btheta)$ is the $\tilde{N} \times \tilde{N}$ upper-left
sub-matrix of $\bC_{\bW}(\btheta)$.

A consequence of this approach is that the matrices
$\bC^n_{\bW}(\btheta)$, $n = -1, \dots, t$ will be increasingly more
numerically stable, thus easier to solve computationally, as shown in
the following theorem.
\begin{prop} 
  \label{Multilevelapproach:theo2}
Let $\kappa(A) \rightarrow \R$ be the condition number of the matrix
$A \in \R^{N \times N}$ then
\[
\kappa(\bC^{t}_{\bW}(\btheta)) \leq \kappa(\bC^{t-1}_{\bW}(\btheta)) \leq 
\dots
\leq
\kappa(\bC_{\bW}(\btheta)) \leq
\kappa(\bC(\btheta)).
\]
\end{prop} 
\begin{proof} A simple extension of the proof in 
Proposition \ref{Multilevelapproach:theo1}.
\end{proof}



\begin{remark} If $\bC(\btheta)$ is symmetric positive definite 
then for $n = 0, \dots, t$ the matrices $\bC^{n}_{\bW}(\btheta)$ are
symmetric positive definite. The proof is immediate.
\end{remark}

\begin{remark} If the matrix $\tilde \bC^{n}_{\bW}(\btheta)$ is close 
to $\bC^{n}_{\bW}(\btheta)$, for $n = 1,\dots,d$, in some matrix norm
sense, the condition number of $\tilde \bC^{n}_{\bW}(\btheta)$ will be
close to $\bC^{n}_{\bW}(\btheta)$. Full error bounds will be derived in
a future publication.
%In section \ref{errorestimates}, for a class of covariance functions,
%it can be shown than for sufficiently large $\tau$ and/or $w,a \in
%\bbN_{+}$ [with the index set $\Lambda(w)^{m,g}$ or $\tilde
%  \Lambda(w)^{m,g}$] $\tilde \bC^{n}_{\bW}(\btheta)$ will be close to
%$\bC^{n}_{\bW}(\btheta)$. Thus $\tilde{\bC}^n_{\bW}(\btheta)$ will be
%symmetric positive definite.
\end{remark}

\subsection{Predictor}
In this section we show how to construct a multilevel BLUP with a
well conditioned multilevel covariance matrix. It will be shown that
the multilevel predictor is \emph{exact}, i.e. the multilevel
predictor and the solution of the constrained predictor problem
(equations \eqref{GLSbeta} and \eqref{KrigBLUP}) are the same.
However, the multilevel form can be significantly easier problem to
solve numerically.

Consider the following system of equations
\begin{eqnarray}
\left( {{\begin{array}{*{20}c}
 \bC(\btheta) \hfill & \bX \hfill \\
 \bX\T \hfill & \0 \hfill \\
\end{array} }} \right)\left( {{\begin{array}{*{20}c}
 \hat \bgamma \hfill \\
 \hat \bbeta \hfill \\
\end{array} }} \right)=\left( {{\begin{array}{*{20}c}
 \bZ \hfill \\
 \0 \hfill \\
\end{array} }} \right).
\label{Kriging:problem}
\end{eqnarray}
From the argument given in \cite{Nielsen2002} it is not hard to show
that the solution of this problem leads to equation \eqref{GLSbeta}
and $\hat \bgamma(\btheta) = \bC^{-1}(\btheta)(\bZ - \bX \hat
\bbeta(\btheta))$. The BLUP can be evaluated as
\begin{equation}
  \hat Z(\bx_0)
  =\bk(\bx_0)\T\hat \bbeta(\btheta)+\bc(\btheta)\T
  \hat \bgamma(\btheta)
\label{Kriging}
\end{equation}
and the Mean Squared Error (MSE) at the target point $\bx_0$ is given by
%\begin{equation}
\[
1 + 
\tilde{\bu}
\T
(\bX \T 
\bC(\btheta)^{-1} \bX )^{-1}
\tilde{\bu}
-\bc(\btheta)\T\bC^{-1}(\btheta)\bc(\btheta)
\]
%\label{Kriging:MSE}
%\end{equation}
where $\tilde{\bu}\T := (\bX \bC^{-1}(\btheta)
\bc(\btheta) - \bk(\bx_0))$.

From \eqref{Kriging:problem} it is observed that $\bX\T \hat
\bgamma(\btheta) = \0$. This implies that $\hat{\bgamma} \in \R^{n}
\backslash \mcP^{p}(\mathbb{S})$ and can be uniquely rewritten as
$\hat{\bgamma} = \bW\T \bgamma_{\bW}$ for some $\bgamma_{\bW} \in
\R^{N-p}$. Now, rewrite $\bC(\btheta) \hat \bgamma + \bX \hat \bbeta
= \bZ$ as
\begin{equation}
\bC(\btheta) \bW\T \bgamma_{\bW} + \bX \hat \bbeta =
       \bZ.
\label{Kriging:eqn1}
\end{equation}
Now apply the matrix $\bW$ to equation \eqref{Kriging:eqn1} and obtain
$\bW \{\bC(\btheta) \bW\T \bgamma_{\bW} + \bX \hat \bbeta\} = \bW
\bZ.$ Since $\bW \bX = \0$ then
\[
\bC_{\bW}(\btheta)
\bgamma_{\bW} = \bZ_{\bW}.
\]
A simple preconditioner $\bP_{\bW}$ can be formed from the diagonal
entries of the matrix $\bC_{\bW}$ i.e. $\bP_{\bW} = diag(\bC_{\bW})$ leading
to the following system of equations
\[
\bP_{\bW}^{-1}\bC_{\bW}(\btheta)
\bgamma_{\bW} = \bP_{\bW}^{-1} \bZ_{\bW}.
\]
Note that in some cases $\bC_{\bW}(\btheta)$ will have very small
condition numbers. For this case we can set $\bP_{\bW}:= I$, i.e. no
preconditioner.

\begin{theorem}
If the covariance function $\phi:\Gamma_d \times \Gamma_d \rightarrow
\R$ is positive definite, then the matrix $\bP_{\bW}(\btheta)$ is always
symmetric positive definite.
\label{multilevelKriging:lemma2}
\end{theorem}
\begin{proof}
Immediate.
\end{proof}


The vector $\hat \bgamma$ can be obtained by applying the inverse
transform $\bW\T$ i.e.  $\hat \bgamma = \bW\T \bgamma_{\bW}$.  From
\eqref{Kriging:problem} the GLS $\hat \bbeta$ can now be computed as a
least squares, i.e.  $\hat{\bbeta} = (\bX\T \bX)^{-1}\bX \T (\bZ -
\bC(\btheta)\hat{\bgamma})$.


\begin{remark}
  Notice that to solve the GLS estimate $\hat \bbeta$ it is not
  necessary to compute the full GLS of equation \eqref{GLSbeta}, but a
  least squares is all that is required. This is in contrast to the
  GLS estimate of equation \eqref{GLSbeta} where if an iterative
  method is used the covariance matrix $\bC(\btheta)$ has to be
  inverted for each of the columns of $\bX$ i.e. $p$ times.
  \end{remark}


\section{Numerical computation of multilevel estimator and predictor}
\label{numericalcomputation}


\subsection{Estimator: Computation of 
$\log{\det\{ \tilde{\bC}^n_{\bW}\}}$ and $(\bZ^n_{\bW})\T
  (\tilde{\bC}^n_{\bW})^{-1}\bZ^n_{\bW}$}

An approach to computing the determinant of
$\tilde{\bC}^n_{\bW}(\btheta)$ is to apply a sparse Cholesky
factorization technique such that $\bG\bG\T =
\tilde{\bC}^n_{\bW}(\btheta)$, where $\bG$ is a lower triangular
matrix. Notice that the eigenvalues of $\bG$ are located on the
diagonal. This leads to $\log \det \{\tilde{\bC}^n_{\bW}(\btheta)\} =
2 \sum_{i = 1}^{\tilde{N}} \log{\bG_{ii}}$.

The direct application of the sparse Cholesky algorithm can leads to
significant fill-in of the factorization matrix $\bG$. To alleviate
this problem it is typical to use matrix reordering techniques. In
particular, the fill-in are reduced by using the sparse Cholesky
factorization \emph{chol} from the Suite Sparse 4.2.1 package
(\cite{Chen2008,Davis2009,Davis2005,Davis2001,Davis1999}) coupled with
Nested Dissection (NESDIS) function package.

In practice, this approach leads to a significant reduction of
fill-in. To my knowledge a theoretical worse case complexity bounded
exists for $d = 2$ or $d = 3$ dimensions (see \cite{Castrillon2015}).

Two choices for the computation of $(\bZ^n_{\bW}) \T \tilde
\bC^n_{\bW}(\btheta)^{-1}$  $\tilde \bZ^n_{\bW}$ are open to us: i) a
Cholesky factorization of $\tilde{\bC}^n_{\bW}(\btheta)$, or ii) a
Preconditioned Conjugate Gradient (PCG).  The PCG choice requires
significantly less memory and allows more control of the error.
However, the sparse Cholesky factorization of
$\tilde{\bC}^n_{\bW}(\btheta)$ has already been used to compute the
determinant. Thus we can use the same factors to compute $ (\tilde
\bZ_{\bW}^n) \T\tilde{\bC}^n_{\bW}(\btheta)^{-1} \tilde \bZ^n_{\bW}$.
The PCG avenue will be explored in more detail in Section
\ref{comppred}.




%% %%%%%%%%%%%%%%%%%%%%%%%%%%%%%%%%%%%%%%%%%%%%%%%%%%%%%%%%%%%%%%%%%%%%%%%%
%\subsection{Predictor: Computation of $\bgamma_{\bW}$ and $\hat{\bbeta}(\btheta)$}

\subsection{Predictor computation}
\label{comppred}

For the predictor stage a different approach is used. Instead of
inverting the sparse matrix $\tilde \bC_{\bW}(\btheta)$ a
Preconditioned Conjugate Gradient (PCG) method is employed to compute
$\hat \bgamma_{\bW} = \bC_{\bW}(\btheta)^{-1} \bZ_{\bW}$.

Recall that $\bC_{\bW} = \bW \bC(\btheta) \bW \T$, $\hat \bgamma_{\bW} =
\bW \hat \bgamma$ and $\bZ_{\bW} = \bW \bZ$. Thus the matrix vector
products $\bC_{\bW}(\btheta) \bgamma_{\bW}^n$ in the PCG iteration are computed
within three steps:
\[
\bgamma_{\bW}^n \xrightarrow[(1)]{\bW \T \bgamma_{\bW}^n} 
\ba_n \xrightarrow[(2)]{\bC(\btheta) \ba_n}
\bb_n \xrightarrow[(3)]{\bW \bb_n}
\bC_{\bW}(\btheta) \bgamma_{\bW}^n 
\]
where $\bgamma_{\bW}^0$ is the initial guess and $\bgamma_{\bW}^n$ is the $n^{th}$
iteration of the PCG.
\begin{enumerate}[(1)]

\item Transformation from multilevel representation to single
  level. This is done in at most $\mcO(Nt)$ steps.

\item Perform matrix vector product using a summation method. For $d =
  2,3$ a KIFMM is used to compute the matrix vector products with
  $\alpha \approx 1$. For $d > 3$ to my knowledge there is no reliable
  fast summation method.

\item Convert back to multilevel representation.

\end{enumerate}

The matrix-vector products $\bC_{\bW}(\btheta) \bgamma_{\bW}^n$, where
$\bgamma_{\bW}^n \in \R^{N-p}$, are computed in $\mcO(N^{\alpha} + 2
Nt)$ computational steps to a fixed accuracy $\varepsilon_{FMM} > 0$.
Note that $\alpha \geq 1$ is dependent on the efficiency of the fast
summation method. The total computational cost is $\mcO(kN^{\alpha} +
2Nt)$, where $k$ is the number of iterations needed to solve
$\bP^{-1}_{\bW} \bC_{\bW}(\btheta) \bar{\bgamma}_{\bW} (\btheta) =
\bP^{-1}_{\bW} \bar{\bZ}_{\bW}$ to a predetermined accuracy
$\varepsilon_{PCG} > 0$.

\begin{remark}
The introduction of a preconditioner can degrade the accuracy for
computing $\hat \bgamma_{\bW} = \bC_{\bW}(\btheta)^{-1} \bZ_{\bW}$
with the PCG method. The residual accuracy $\varepsilon_{PCG}$ of the
PCG iteration has to be set such that the residual of
the\emph{unpreconditioned} system $\|\bC_{\bW}(\btheta) \bgamma_{\bW}
(\btheta) - \bZ_{\bW}\|_{l^2} < \varepsilon$ for a user given
tolerance $\varepsilon > 0$.
\end{remark}

Now compute $\hat \bgamma = \bW\T \hat \bgamma_{\bW}$ and $\hat{\bbeta} =
(\bX\T \bX)^{-1}\bX \T (\bZ - \bC(\btheta)\hat{\bgamma})$ in at most
$\mcO(N^{\alpha} + Np + p^{3})$ computational steps. The matrix vector
product $\bc(\btheta)\T \hat{\bgamma}(\btheta)$ is computed in
$\mcO(N)$ steps.  Finally, the total cost for computing the estimate
$\hat{\bZ}(\bx_0)$ from \eqref{Kriging} is $\mcO(p^{3} + (k +
1)N^{\alpha} + 2Nt)$.

%% Error Estimates ----------------------------------------------------
%%
\section{Multilevel covariance matrix decay}
\label{errorestimates}

We derive decay estimates of the multilevel covariance matrix. It can
be shown that most of the coefficients are small and thus it is not
necessary to compute all of them. The final objective is to build a
posteriori error estimates for $\bx_{\bW} =
\bC^{n}_{\bW}(\btheta)^{-1}\bZ^n_{\bW}$ and $\log
\bC^{n}_{\bW}(\btheta)$ that are needed for solving the multilevel
estimator MLE. However, the full analysis is extensive and will be
completed in a future publication. As a first step we show the decay
of the multievel covariance matrix. Note that this is not trivial and
uses the results derived in the Appendix. We recommend to first read
the appendix since part of the notation used in this section is
defined there.


%Consider the full solution $\bx_{\bW} =
%\bC^{n}_{\bW}(\btheta)^{-1}\bZ^n_{\bW}$ and sparse solution $\tilde
%\bx_{\bW} = \tilde \bC^{n}_{\bW}(\btheta)^{-1}\bZ^n_{\bW}$ for $n = 0,
%\dots, t$, then the error can be bounded as
%\begin{equation}
%\|\bx_{\bW} - \tilde \bx_{\bW}\|_{l^2} 
%\leq 
%\| \bC^{-1}_{\bW}(\btheta) - \tilde \bC^{-1}_{\bW}(\btheta) \|_2 
%\|\bZ_{\bW}\|_{l^2}.
%\label{errorestimates:eqn1}
%\end{equation}
%The ultimate goal is to derive a full a posteriori error estimate of
%$\|\bx_{\bW} - \tilde \bx_{\bW}\|_{l^2}$ with respect to the distance
%criterion $\tau_{i,j}$.  In this section estimates of the decay of the
%multilevel matrix $\bC_{\bW}(\btheta)$ are obtained. These will be
%critical to derive a full a posteriori error scheme.




The decay of the coefficients of the matrix $\bC_{\bW}(\btheta)$ will
depend directly on the choice of the multivariate index set
$\mcQ^{d}_{w}$ and the analytic regularity of the covariance
function. In general, the Mat\'{e}rn covariance function will be
analytic except for a derivative discontinuity at the
origin. However, with the application of the distance criterion a
minimal distance can be guaranteed and the origin will be avoided all
together.

In the following theorem, without loss of generality, it is assumed
that the covariance function domain is $\Gamma^{d} \times \Gamma^{d}$
for any two cells $B^{i}_{m} \in \mcB^{i}$ and $B^{j}_{q} \in
\mcB^{j}$. We will show later that this can be achieved through a
linear pullback.

\begin{theorem} Suppose that $0< \delta < 1$, $\hat
  \sigma := \sigma (1 - \delta)$, and $\phi(\bx,\by;\btheta) \in
  C^{0}(\Gamma^{d} \times \Gamma^{d};\R)$ can be analytically extended
  on $\mcE^{d}_{\sigma} \times \mcE^{d}_{\sigma}$ and is bounded by
  $\tilde M(\phi)$. Let $\mcP^{ p}(\mathbb{S})^{\perp}$ be the
  subspace in $\R^{N}$ generated by the index set $\mcQ^{d}_{w}$ for
  some $w \in \bbN_{+}$. For $i,j = 0,\dots,t$ consider any
  multilevel vector $\bpsi^{i}_m \in \mcP^{p}(\mathbb{S})^{\perp}$,
  with $n_m$ non-zero entries, from the cell $B^{i}_{m} \in \mcB^{i}$
  and any multilevel vector $\bpsi^{j}_{q} \in \mcP^{
    p}(\mathbb{S})^{\perp}$, with $n_q$ non-zero entries, from the
  cell $B^{j}_{q} \in \mcB^{j}$. If $p(d,w) \geq \left(\frac{2
    d}{\kappa(d)}\right)^{d}$ then
\[
\begin{split}
\left|
\sum_{r = 1}^{N} 
\sum_{h = 1}^{N} 
\phi(\bx_r,\by_h; \btheta) 
\bpsi^i_m[h] \bpsi^j_q[r] 
\right|
&\leq
\sqrt{n_mn_q}
\left( \frac{
  C(\tilde M,\sigma)^{d} e^{d - \sigma(1 - \delta) + 1} \hat \sigma d
}
 {
   (\sigma \delta)^{d}} \right)^2
 \\
 &
 \left( \frac{e^{\hat \sigma}}{1 - e^{-\hat \sigma}} \right)^{2d}
 \exp \left(-\frac{2d}{e} \hat \sigma  p^{\frac{1}{d}}
 \right) p^{2\left(\frac{d-1}{d}\right)}.
\end{split}
\]
\label{errorestimates:theorem1}
\end{theorem}
% Proof
\begin{proof} 
We first have that
\[
\begin{split}
\sum_{k = 1}^{N} 
\sum_{l = 1}^{N} 
\phi(\bx_k,\by_l; \btheta) 
\bpsi^i_m[k] \bpsi^j_q[l] 
&=
\sum_{k = 1}^{N} 
\sum_{l = 1}^{N}
\lim_{g \rightarrow \infty}
(\bI^d_{g}
\otimes
\bI^d_{g})[
\phi(\bx_k,\by_l; \btheta)] 
\bpsi^i_m[k] \bpsi^j_q[l] \\
&=
\sum_{k = 1}^{N} 
\sum_{l = 1}^{N}
(I_d - \mcS_{w,d}) 
\otimes
(I_d - \mcS_{w,d}) \\
&[\phi(\bx_k,\by_l; \btheta)] 
\bpsi^i_m[k] \bpsi^j_q[l].
\end{split}
\]
The last equality follows from $\bpsi^{i}_m, \bpsi^{j}_{q} \in \mcP^{
  p}(\mathbb{S})^{\perp}$. We now have that
\[
\begin{split}
& \sum_{k = 1}^{N} 
\sum_{l = 1}^{N}
\|(I_d - \mcS_{w,d}) 
\otimes
(I_d - \mcS_{w,d})
[\phi(\bx_k,\by_l; \btheta)] \|_{L^{\infty}_{\rho}(\Gamma^d)}
|\bpsi^i_m[k]| |\bpsi^j_q[l]|
\\
&\leq 
\|(I_d - \mcS_{w,d}))[\phi(\bx_k,\by_l; \btheta)] \|^{2}_{L^{\infty}_{\rho}(\Gamma^d)}
\sum_{k = 1}^{N} 
\sum_{l = 1}^{N}
|\bpsi^i_m[k]| |\bpsi^j_q[l]|.
\end{split}
\]
Since $\bpsi^i_m$ and $\bpsi^j_q$ are orthonormal then
\[
\begin{split}
\sum_{k = 1}^{N} 
\sum_{l = 1}^{N}
|\bpsi^i_m[k]| |\bpsi^j_q[l]|
&\leq \sqrt{n_m n_q}
\|\bpsi^i_m[k]\|_{l^2} \|\bpsi^j_q[l]\|_{l^2} =
\sqrt{n_m n_q}.
\end{split}
\]
From Lemma \ref{interpolation:lemma1} the result follows.
\end{proof}



\begin{remark} Recall that the restriction $p(d,w) \geq \left(\frac{2
  d}{\kappa(d)}\right)^{d}$ is not strict and can be relaxed such that
  sub-exponential convergence is still obtained. See Remark
  \ref{interpolation:remark1}.
\end{remark}

\begin{remark}
  The decay of the coefficients of $\bC^{i,j}_{\bW}$ is sub-exponential
  with respect to $p$.  Even for a moderate magnitude for $\hat
  \sigma > 0$, $p > 0$ and $d \geq 1$ the entries of the
  multilevel matrix $\bC^{i,j}_{\bW}$ that do not correspond to the
  cells given by the distance criterion parameter $\tau_{i,j} \geq 0$
  will be close to zero.
  %Thus the matrix $\bC^{i,j}_{\bW}$ will be
  %highly sparse as the number of observations $N$ increases.
\end{remark}

Theorem \ref{errorestimates:theorem1} provides a mechanism to control
the decay of the coefficients of the multilevel covariance matrix
$\bC_{\bW}$. To this end, we are interested in the analytic extension
and the uniform bound $\tilde M(\phi) \leq \infty$ of the Mat\'{e}rn
covariance function
\[
\phi(r;\btheta):=\frac{1}{\Gamma(\nu)2^{\nu-1}} \left(
\sqrt{2\nu} r(\btheta) \right)^{\nu} K_{\nu} \left(
\sqrt{2\nu} r(\btheta) \right)
\]
on a subdomain in $\bbC^{d} \times \bbC^{d}$, where $r(\btheta) =
(\bx-\by)^{T}$ $\text{diag}(\btheta) (\bx - \by)
)^{\frac{1}{2}}$, $\btheta=[\theta_1, \dots, \theta_d] \in
\R^{d}_{+}$ are positive constants, $\text{diag}(\theta) \in \R^{d
  \times d}$ is a diagonal matrix with the vector $\btheta$ on the
diagonal, and $\bx, \by \in \R^{d}$.

The polynomial function is an entire function. However, the function
$K_{\nu}(\vartheta)$ and $\vartheta^{\frac{1}{2}}$ are analytic for
all $\vartheta \in \bbC$ except at the branch cut $(-\infty,0]$.  Thus
it is sufficient to check the analytic extension of $r = \|\bx -
\by\|_{l^{2}} = \Big( \sum_{k=1}^{d} \theta_{k} (x_k - y_k)^{2}
\Big)^{\frac{1}{2}}$.  Let $z \in \bbC$ be the complex extension of
$r \in \R$. More precisely, $z = \Big( \sum_{k=1}^{d} \theta_{k}
z_k^{2} \Big)^{\frac{1}{2}}$, where $z_k \in \bbC$ is the complex
extension of $(x_k - y_k)$.

Let $\vartheta = \sum_{k=1}^{d} \theta_{k} z_k^{2}$, then by taking
the appropriate branch $\Real z = r_{\vartheta}$ $\cos{(
  \theta_{\vartheta}/2)}$, where $r^2_{\vartheta} = (\Real
\vartheta)^2 + (\Imag \vartheta)^2$ and $\theta_{\vartheta} =
\tan^{-1} \frac{\Imag \vartheta}{\Real \vartheta}$. Due to the branch
cut at $(-\infty,0]$ we impose the restriction that $\Real \vartheta >
0$ as $x_k$ and $y_k$ are extended in the complex plane.  Consider any
two cells $B^{i}_{m} \in \mcB^i$ and $B^{j}_{q} \in \mcB^j$, at levels
$i$ and $j$ with the associated distance criterion constant
$\tau_{i,j}>0$. From Algorithms \ref{MLCM:algorithm3},
\ref{MLCM:algorithm4}, \ref{MLCM:algorithm5} \ref{MLCM:algorithm6},
for any observations $\bx^{*} \in B^{i}_{m}$ and $\by^{*} \in
B^{j}_{q}$ we have that $(x^*_k - y^*_k)^2 \geq \tau^2_{i,j}$ for $k =
1,\dots,d$.  For the rest of the discussion it is assumed that complex
extension is respect to each component $k = 1,\dots,d$ unless
otherwise specified.


Let $x^{min}_k := \min_{ x^*_k \in B^i_m} x^*_k$, $x^{max}_k := \max_{
  x^*_k \in B^i_m} x^*_k$, $y^{min}_k := \min_{ y^*_k \in B^i_m}
y^*_k$, $y^{max}_k := \max_{ y^*_k \in B^i_m} y^*_k$ and
$\alpha_k,\gamma_k \in [-1,1]$. Define the region $\mcX^{i}_{m} :=
[x^{min}_1,$ $x^{max}_1] \times \dots \times [x^{min}_d,x^{max}_d]$
and $\mcY^{j}_{q} := [y^{min}_1,y^{max}_1] \times \dots \times
[y^{min}_d,y^{max}_d]$.

%rescale the widths of each of the cells $B^{i}_{m}$ and $B^{j}_{q}$
%with respect to the domain $\Gamma$.

The next step is to redefine $\phi(\bx;\by;\btheta):\mcX^{i}_m \times
\mcY^{i}_{q} \rightarrow \R$ as
$\phi(\balpha,\bgamma;\btheta):\Gamma^d \times \Gamma^d \rightarrow
\R$ through a pullback. Let $x_k = \left(\frac{\alpha_k + 1}{2}
\right) a_k + b_k$ and $y_k = \left(\frac{\gamma_k + 1}{2} \right) c_k
+ d_k$, where $a_k = x^{max}_{k} - x^{min}_{k}$, $b_k = x^{min}_{k}$,
$c_k = y^{max}_{k} - y^{min}_{k}$ and $d_k = y^{min}_{k}$.


%The covariance function
%$\phi(\bx;\by;\btheta):\mcX^{i}_m \times \mcY^{i}_{q} \rightarrow \R$
%is reformulated as $\phi(\balpha,\bgamma;\btheta):\Gamma^d \times
%\Gamma^d \rightarrow \R$.

Extend $\alpha_k \rightarrow \alpha_k + v_k$ and $\gamma_k
\rightarrow \gamma_k + w_k$ where $v_k:= v^R_k + iv^I_k$, $w_k:= w^R_k
+ iw^I_k$, and $v^R_k,v^I_k,w^R_k,w^I_k \in \R$. Let $\tilde x_k$ be
the extension of $x_k$ in the complex plane and similarly for
$\tilde y_k$.


It follows that $\tilde x^R_k := \Real \tilde x_k = \frac{1}{2}
(\alpha_k + 1 + v^R_k)a_k + b_k$, $\tilde x^I_k = \Imag \tilde x_k =
\frac{v^I_k}{2} a_k$, $y^R_k := \Real \tilde y_k = \frac{1}{2}
(\gamma_k + 1 + w^R_k)c_k + d_k$, and $y^I_k := \Imag \tilde y_k =
\frac{w^I_k}{2} c_k$.  After some manipulation
\begin{equation}
\begin{split}
\Real z^2_k &= (\tilde x^R_k - \tilde y^R_k)^2
- (\tilde x^I_k - \tilde y^I_k)^2 
=
(x_k - y_k)^2 
+
\frac{1}{4}(v^R_k a_k - w^R_k c_k)
+ (x_k - y_k)(v^R_k a_k - w^R c_k)
\\
&
-\frac{1}{4}(a_kv^I_k - c_k w^I_k)^2.
\end{split}
\label{errorestimates:eqn2}
\end{equation}


Recall that $(x_k - y_k)^2 \geq \tau^2_{i,j}$ and suppose that there
is a positive constant $\delta_{k} > 0 $ such that
\begin{equation}
\delta_{k} \leq 
%-\frac{4\,\tau -\sqrt{32\,\tau ^2+8\,\tau +1}+1}{4\,\tau }
\frac{\sqrt{32\,\tau_{i,j} ^2+8\,\tau_{i,j} +1}-1 - 4\,\tau_{i,j}  }{4\,\tau_{i,j}}.
\label{errorestimates:eqn2a}
\end{equation}

Assume that $|v^R_k|\leq \tau_{i,j} \delta_{k} / a_{k}$, $|v^I_k|\leq
\tau_{i,j} \delta_{k}/a_{k}$, $|w^R_k|\leq \tau_{i,j} \delta_{k} /
c_{k}$, and $|w^I_k|\leq \tau_{i,j} \delta_{k} / c_{k}$. From
equations \eqref{errorestimates:eqn2} and \eqref{errorestimates:eqn2a}
it follows that
\begin{equation}
\Real z^2_k \geq \tau_{i,j}^2 (1 - 4 \delta_{k}^2) - \frac{\tau_{i,j}
  \delta_{k}}{2} > 0.
\label{errorestimates:eqn3}
\end{equation}
%\begin{equation}
%\Real z^2_k \geq \tau_{i,j,m,q}^2 (1 - \frac{9}{16}(a_k + c_k)) > 0
%\label{errorestimates:eqn3}
%\end{equation}
Furthermore,
\begin{equation}
\begin{split}
  &
  |\Real z^2_k| \leq 
(\max\{|y^{max}_k - x^{min}_{k}|,|x^{max}_k - y^{min}_{k}|\})^2
\\
&+ \frac{1}{2}\tau_{i,j} \delta_{k} 
+ 
\max\{|y^{max}_k - x^{min}_{k}|,|x^{max}_k - y^{min}_{k}|\} 
2\tau_{i.j} \delta_{k}
+ \tau^2_{i,j} \delta_{k}^2 
\leq
1 + \frac{5}{2}\tau_{i,j} \delta_{k} + \tau^2_{i,j} \delta_{k}^2. 
\end{split}
\label{errorestimates:eqn6}
\end{equation}
Similarly,
\begin{equation}
  |\Imag z^2_k| = |2(\tilde x^R_k - \tilde y^R_k)(x^I_k - y^I_k)|
  \leq
2 \tau_{i.j} \delta_{k} + 4 \tau^2_{i,j} \delta_{k}^2.
\label{errorestimates:eqn4}
\end{equation}

We now show how $\alpha_k$ and $\gamma_k$ can be extended into the
Bernstein ellipses $\mcE_{\sigma^{\alpha}_k}$ and
$\mcE_{\sigma^{\gamma}_k}$, for some $\sigma^{\alpha}_k > 0$ and
$\sigma^{\gamma}_k > 0$ such that $\Real z^2_k > 0$. Recall that
$|v^R_k|\leq \tau_{i,j} \delta_{k} / a_{k}$, $|v^I_k|\leq \tau_{i,j}
\delta_{k}/a_{k}$, $|w^R_k|\leq \tau_{i,j} \delta_{k} / c_{k}$, and
$|w^I_k|\leq \tau_{i,j} \delta_{k} \ c_{k}$.
%We restrict the length of the extension of $(x^{min}_k, x^{max}_k)$
%and $(y^{min}_k, y^{max}_k)$ by $\tau_{i,j}/2$
These restrictions form a region in $\bbC \times \bbC$ and a Bernstein
ellipse is embedded (See Figure \ref{analyticity:fig1}).  This is done
by solving the following equation: $\frac{e^{\sigma^{\alpha}_k} +
  e^{-\sigma^{\alpha}_k}}{2} = 1 + \frac{\tau_{i,j}
  \delta_{k}}{a_k}$. The unique solution is
\[
\sigma^{\alpha}_k = \cosh^{-1} \left(1 +
\frac{\tau_{i,j} \delta_{k}}{a_k}
\right)
\]
with $\sigma^{\alpha}_k > 0$. Following a
similar argument we have that
\[
\sigma^{\gamma}_k = \cosh^{-1} \left(1
+ \frac{\tau_{i,j} \delta_{k}}{c_k}
\right)
\]
with $\sigma^{\gamma}_k > 0$. Let $\mcE^{d}_{\alpha} :=
\prod_{k=1}^{d} \mcE_{\sigma^{\alpha}_k}$ and $\mcE^{d}_{\gamma} :=
\prod_{k=1}^{d} \mcE_{\sigma^{\gamma}_k}$.  It follows that
\begin{equation}
\begin{split}
  \Real \vartheta &\geq  
  \sum_{k=1}^{d} \theta_k
  \Real z^2_k 
  \geq
\sum_{k=1}^{d} \theta_k 
  \left(
  \tau_{i,j}^2 (1 - 4 \delta_{k}^2) - \frac{\tau_{i,j}
    \delta_{k}}{2} \right)
  > 0.
  \end{split}
\label{errorestimates:eqn8}
\end{equation}
Thus there exist an analytic extension of $\phi(r;\btheta):\Gamma^d
\times \Gamma^d \rightarrow \R$ on $\mcE^{d}_{\alpha} \times
\mcE^{d}_{\gamma}$.



\begin{figure*}[htb!]
\begin{center}
\begin{tikzpicture}
    \begin{scope}[font=\scriptsize]

    \draw [->] (-2.5, 0) -- (2.5, 0) node [above left] {$\Real $};
    \draw [->] (0,-1.5) -- (0,1.5) node [below right] {$\Imag$};
    \draw [-,dashed] (-2,-1.5) -- (-2,1.5);
    \draw [-,very thick] (-1,0) -- (1,0);
    \draw [-,dashed] (2,-1.5) -- (2,1.5);
    \draw (1,-3pt) -- (1,3pt) node [above] {$-1$};
    \draw (-1,-3pt) -- (-1,3pt) node [above] {$1$};


    \draw [-,dashed] (-2,1) -- (2,1);
    \draw [-,dashed] (-2,-1) -- (2,-1);
    
 
    \fill [opacity=0.2, pattern=north west lines, pattern color=red]
    (-2,-1) rectangle (-1.5,1);

    \fill [opacity=0.2, pattern=north west lines, pattern color=red]
    (1.5,-1) rectangle (2,1);
    \end{scope}
    
    \node [below right] at (-1.92,0) {$\frac{\tau_{i,j} \delta_{k}}{a_k}$};
    \node [below right] at ( 1.05,0) {$\frac{\tau_{i,j} \delta_{k}}{a_k}$};

    \node at (0,-1.8) {$(a)$};
    \node at (6,-1.8) {$(b)$};
    
    \begin{scope}[shift={(0,0)},font=\scriptsize]

      \filldraw[fill={rgb:red,143;green,188;blue,143},semitransparent]
      (6,0) ellipse (2 and 1);

    \draw [->] (3, 0) -- (9, 0) node [above left] {$\Real $};
    \draw [->] (6,-1.5) -- (6,1.5) node [below right] {$\Imag$};
    \draw (5,-3pt) -- (5,3pt)   node [above] {$1$};
    \draw (7,-3pt) -- (7,3pt) node [above] {$-1$};


    %\fill [opacity=0.2, pattern=north west lines, pattern color=red]
    %(-2,-1.5) rectangle (-1.5,1.5);
    \end{scope}
    
    \node [below right] at (7.50,1.25) {$\mcE_{\sigma^{\alpha}_k}$};
    \node [below right] at (7.25,0.05) {$\frac{e^{\sigma^{\alpha}_k}
                                       + e^{-\sigma^{\alpha}_k}}{2}$};
        
\end{tikzpicture}
\end{center}
\caption{(a) Region of Complex extension of $\alpha_k$.  (b) Embedding
  of Bernstein ellipse $\mcE_{\sigma^{\alpha}_k}$.}
\label{analyticity:fig1}
\end{figure*}

The next step is to bound the absolute value of the Mat\'{e}rn
covariance function $|\phi(z;\btheta)|$ in $\mcE^{d}_{\alpha} \times
\mcE^{d}_{\gamma}$. If $\nu > \frac{1}{2}$ and $\Real z>0$ then the
modified Bessel function of the second kind satisfies the following
identity
\[
\begin{split}
K_{\nu}(\sqrt{2 \nu}z) &= \frac{\sqrt{\pi} (\sqrt{2 \nu}z)^{\nu}}{2^\nu
  \Gamma(\nu + \frac{1}{2})} \\
&
\int_{1}^{\infty} (t^2 - 1)^{\nu -
  \frac{1}{2}} \exp{(-\sqrt{2 \nu}zt)}\, \text{d}t.
\end{split}
\]
It is not hard to show that for $\nu > \frac{1}{2}$ and $\Real z >0$,
we have that $|K_{\nu}(\sqrt{2 \nu}z)| \leq \frac{|\sqrt{2
    \nu}z|^{\nu}}{(\Real \sqrt{2 \nu}z)^{\nu}} K_{\nu}(\sqrt{2 \nu}
\Real z)$.

Note that $r_{\vartheta} \geq \Real \vartheta > 0$.  From equation
\eqref{errorestimates:eqn4} we have that $\Imag \vartheta =
\sum_{k=1}^{d} \theta_k \Imag z^2_k \leq \sum_{k=1}^{d} 2 \tau
\delta_{k} + 4 \tau^2 \delta_{k}^2$.  From equation
\eqref{errorestimates:eqn8}
\[
\begin{split}
  |\theta_{\vartheta}|
  &\leq
\xi(\btheta,\bdelta,\tau_{i,j}) := \tan^{-1}
\left(
\frac{
  \sum_{k=1}^{d} 2 \tau_{i,j}
\delta_{k} + 4 \tau^2_{i,j} \delta_{k}^2
}
{
\sum_{k=1}^{d} \theta_k 
  \left(
  \tau_{i,j}^2 (1 - 4 \delta_{k}^2) - \frac{\tau_{i,j}
    \delta_{k}}{2} \right)
}
\right)
< \frac{\pi}{2}.
\end{split}
\]
Since $K_{\nu}(\cdot)$ is strictly completely monotonic
\cite{Baricz2011} then
\begin{equation}
\begin{split}
  |K_{\nu}(\sqrt{2 \nu}\Real z)| &=
  |K_{\nu}\ (\sqrt{2 \nu}
  r_{\vartheta} \cos(\theta_{\vartheta}/2))
  | 
  \leq
  \Big| K_{\nu}\Big(\sqrt{\frac{\nu}{2}}
  \cos(\xi(\btheta,\bdelta,\tau)/2) \\
  &
  \sum_{k=1}^{d} \theta_k 
  \Big(
  \tau_{i,j}^2 (1 - 4 \delta_{k}^2) - \frac{\tau_{i,j}
    \delta_{k}}{2} \Big)
  \Big) \Big|.
\end{split}
\label{errorestimates:eqn5}
\end{equation}
From equations \eqref{errorestimates:eqn6}
\eqref{errorestimates:eqn4} 
\[
\begin{split}
|z_k|^{2} &\leq |\Real z^2_k| + |\Imag z^2_k|
\leq \mcR(\delta_k,\tau_{i,j})
:=
1 + \frac{9}{2}\tau_{i,j} \delta_{k} + 5 \tau^2_{i,j} \delta_{k}^2
\end{split}
\]
and therefore
\begin{equation}
\begin{split}
  |z|
  &\leq
\left|
\sum_{k=1}^{d} \theta_{k} z_k^{2} \right|^{\frac{1}{2}}
\leq
\left(
\sum_{k=1}^{d} \theta_{k} |z_k|^{2} \right)^{\frac{1}{2}}
\leq \left( \sum_{k=1}^{d} \theta_k \mcR(\delta_k,\tau_{i,j})
\right)^{\frac{1}{2}}.
\end{split}
\label{errorestimates:eqn7}
\end{equation}
By combining equations \eqref{errorestimates:eqn4},
\eqref{errorestimates:eqn8}, \eqref{errorestimates:eqn5}, and
\eqref{errorestimates:eqn7}, we have now proven the following Theorem.

\begin{theorem} For any two cells $B^{i}_{m}$ and $B^{j}_{q}$
with the associated distance criterion parameter $\tau_{i,j} \geq 0$
let $\phi(\balpha,\bgamma;$ $\btheta):\Gamma^d \times \Gamma^d
\rightarrow \R$ be the pullback of the Mat\'{e}rn covariance function
$\phi(\bx;\by;\btheta):\mcX^{i}_m \times \mcY^{j}_{q} \rightarrow
\R$. Then there exists an analytic extension of
$\phi(\balpha,\bgamma;\btheta):\Gamma^d \times \Gamma^d \rightarrow
\R$ on the polyellipse $ \mcE^{d}_{\alpha} \times \mcE^{d}_{\gamma} $
and
  \[
  |\phi(\cdot,\cdot;\btheta)| \leq
\frac{  \left( 2 \nu \sum_{k=1}^{d} \theta_k \mcR(\delta_k,\tau_{i,j})
\right)^{\frac{\nu}{2}}
|K_{\nu}
(\Xi(\btheta,\bdelta,\tau_{i,j}
)|}{
\Xi(\btheta,\bdelta,\tau_{i,j})^{\nu}
}
\]
on $\mcE^{d}_{\alpha} \times \mcE^{d}_{\gamma}$, where
\[
\begin{split}
  \Xi(\btheta,\bdelta,\tau_{i,j})
  :=
\Big| K_{\nu}\Big(\sqrt{\frac{\nu}{2}}
  \cos(\xi(\btheta,\bdelta,\tau_{i,j})/2)
\sum_{k=1}^{d} \theta_k 
  \Big(
  \tau_{i,j}^2 (1 - 4 \delta_{k}^2) - \frac{\tau_{i,j}
    \delta_{k}}{2} \Big)
  \Big) \Big|.
  \end{split}
\]


\end{theorem}

%% Numerical Results --------------------------------------------------
%%
\section{Numerical results}
\label{numericalresults}

The performance of the multilevel solver for estimation and
prediction formed from random datasets is tested. The results show
that the computational burden is significantly reduced while retaining
good accuracy. In particular, it is possible to now solve
ill-conditioned problems efficiently. The implementation of the code
is as follows:


\begin{enumerate}[i)]

\item {\bf Matlab, C/C++ and MKL:} The binary tree, multilevel basis
  construction, formation of the sparse matrix $\tilde \bC_{\bW}$,
  estimation and prediction components are written and executed on
  Matlab \cite{Matlab2016}. However, the computational bottlenecks are
  executed by C/C++ software packages, Intel MKL \cite{intelmkl}, and
  the highly optimized BLAS and LAPACK packages contained in
  MATLAB. The C/C++ interfaces to matlab are constructed as dynamic
  shared libraries.

  

\item {\bf Direct and fast summation:} The matlab code estimates the
  computational cost between the direct and fast summation methods and
  chooses the most efficient approach.  For the direct method a
  combination of Graphic Processing Unit (GPU) and MKL intel libraries
  are used. For the fast summation method the KIFMM ($d = 3$) c++ code
  is used.  The KIFMM is modified to include a Hermite interpolant
  approximation of the Mat\'{e}rn covariance function, which is
  implemented with the intel MKL package \cite{intelmkl} (see
  \cite{Castrillon2015} for details).


\item {\bf Dynamic shared libraries:} These are produced with the GNU
  gcc/g++ packages. These libraries implement the Hermite interpolant
  with the intel MKL package (about 10 times faster than Matlab
  Mat\'{e}rn interpolant) and link the MATLAB code to the KIFMM.

\item {\bf Cholesky and determinant computation:} The Suite Sparse
  4.2.1 package
  (\cite{Chen2008,Davis2009,Davis2005,Davis2001,Davis1999}) is used
  for the determinant computation of the sparse matrix $\tilde
  \bC_{\bW}(\btheta)$.

\end{enumerate}

The code is tested on a single CPU (4 core Intel i7-3770 CPU @
3.40GHz.), one Nvidia 970 GTX GPU, with Linux Ubuntu 18.04 and 32 GB
memory. In addition, the Boston University Shared Computing Cluster
was used to generate test data.  To test the effectiveness of the
Multilevel solver the following data sets are generated:
\begin{enumerate}[a)]


\item {\bf Random n-sphere data set:} The set of nested random
  observation $\bS_{0}^{d} \subset \dots \subset \bS_{9}^{d}$ vary
  from 1,000, 2000, 4000 to 256,000 knots generated on the n-sphere
  $\bS_{d-1} := \{\bx \in \R^{d}\,\,|\,\,\|\bx\|_{2} = 1 \}$.

\item {\bf Random hypercube data set:} The set of random observation
  locations $\bC_{0}^{d},$ $\dots, \bC_{10}^{d}$ vary from 1,000, 2000,
  4,000 to 512,000 knots generated on the hypercube $[-1,1]^{d}$ for
  $d$ dimensions.  The observations locations are also nested,
  i.e. $\bC_{0}^{d} \subset \dots \subset \bC_{10}^{d}$.
  
\item {\bf Normal test data set} The set of observations values
  $\bZ^d_{0}$, $\bZ^d_{1}$, \dots $\bZ^d_{5}$ are formed from the
  Gaussian random field model \eqref{Introduction:noisemodel} for
  1,000, 2,000, $\dots$ $256,000$ observation locations. The data set
  $\bZ^d_{n}$ is generated from the set of nodes $\bS^{d}_{n}$, with
  the covariance parameters $(\nu,\rho)$ and the corresponding set of
  monomials $\mcQ^d_w$. The Boston University Shared Computing Cluster
  was used to generate the normal test data.
    
%\item All the numerical test are done assuming the $\tilde p = p$.
  
\end{enumerate}


\begin{remark}
 All the timings for the numerical tests are given in wall clock
 times i.e. the actual time is needed to solve a problem. This is to
 distinguish from CPU time, which can be significantly smaller.
  \end{remark}


\subsection{Numerical stability and sparsity of the covariance multilevel
  matrix}

For many practical cases the covariance matrix $\bC(\btheta)$ becomes
increasingly ill-conditioned for the Mat\'{e}rn covariance function as
$\rho$, $\nu$ and the number of observations are increased. This leads
to instability of the numerical solver. It is now shown how effective
Theorem \ref{Multilevelapproach:theo1} becomes in practice.  In
Figure \ref{numericalresults:fig1} the condition number of the
multilevel covariance matrix $\bC_{\bW}(\btheta)$ is plotted with
respect to the cardinality $p(w,d)$ of $\mcQ^d_w$ for different $w$
levels. The multilevel covariance matrix $\bC_{\bW}(\btheta)$ is built
from the random cube $\bC^{d}_{4}$ or n-sphere $\bS^{d}_{4}$
observations. The covariance function is set to Mat\'{e}rn with $\nu =
1$ and $\rho = 1,10$.  As the plots confirm the covariance matrix
condition number significantly improves with increasing level
$w$. This is in contrast with the large condition numbers of the
original covariance matrix $\bC(\btheta)$.  This is consistent with
Theorem \ref{Multilevelapproach:theo1} and Corollary
\ref{Multilevelapproach:cor1}.


We now focus our attention of the sparsity of the covariance matrix
$\tilde \bC_{\bW}(\btheta)$. In Figure \ref{numericalresults:fig2}(a)
the magnitude of the multilevel covariance matrix
$\bC_{\bW}(\btheta)$ is plotted for $N = 8,000$ observations from the
the n-sphere $\bS^{3}_{3}$ with Mat\'{e}rn covariance parameters $\nu
= 0.5$ (exponential) and $\rho = 10$. Due to the large value of $\rho$
the overlap between the covariance function at the different locations
in $\bS^{3}_{3}$ is quite significant, thus leading to a dense
covariance matrix $\bC(\btheta)$ where the coefficients decay
slowly. This is in contrast to the large number of small entries for
$\bC_{\bW}(\btheta)$, as shown in the histogram in Figure
\ref{numericalresults:fig2}(b). Note that the histogram is in terms of
$\log_{10}$ of the absolute value of the entries of
$\bC_{\bW}(\btheta)$. From the histogram it is observed that almost
all the entries are more than 1000 smaller than the largest entries.
This numerical result is consistent with the sub-exponential decay
rates of Theorem \ref{errorestimates:theorem1}.


\begin{figure*}[htpb]
\begin{center}
\begin{tikzpicture}%[thick,scale=1, every node/.style={scale=1}]
  \node[inner sep=0pt] at (0,0)
  {
  \includegraphics[trim = 120 255 120 255,
    clip,width=4.4in,height=4in]{ConditionGraphsReduced.pdf}
  };
  \node[rotate = 90] at (-5.5,2.7) {$\kappa(\bC_{\bW}(\btheta))$};
  \node[rotate = 90] at (0,2.7) {$\kappa(\bC_{\bW}(\btheta))$};

  \node[rotate = 90] at (-5.5,-2.7) {$\kappa(\bC_{\bW}(\btheta))$};
  \node[rotate = 90] at (0,-2.7) {$\kappa(\bC_{\bW}(\btheta))$};

\node at (-2.6,5.3) {
\begin{tabular}{c}
\small $Cube, d = 5, \bC^5_4, \rho=1$ \\
\small $\kappa(\bC(\btheta))= 1.1 \times 10^{7}$
\end{tabular}
};

\node at (2.9,5.3) {
\begin{tabular}{c}
\small $Cube, d = 5, \bC^5_4, \rho=10$ \\
\small $\kappa(\bC(\btheta))= 2.2 \times 10^{9}$
\end{tabular}
};

  
\node at (-2.6,-0.15)
{
\begin{tabular}{c}
\small $Sphere, d = 5,  \bC^5_4, \rho=1$ \\
\small $\kappa(\bC(\btheta))= 2.6 \times 10^{7}$
\end{tabular}
};

\node at (2.9,-0.15)
{
\begin{tabular}{c}
\small $Sphere, d = 5, \bS^5_4, \rho=10$ \\
\small $\kappa(\bC(\btheta))=7.8 \times 10^{9}$
\end{tabular}
};
\node at (2.9,-5.0)
      {$p$
        };
\node at (-2.6,-5.0)
      {$p$
        };
\end{tikzpicture}
\end{center}
\caption{Condition number of the multilevel covariance matrix
  $\bC_{\bW}(\btheta)$ with respect to the size $p$ of the Total
  Degree (TD) polynomial space. The number of observations corresponds
  to 16,000 nodes generated on a hypercube or n-sphere of dimension $d
  = 5$. The covariance function is chosen to be Mat\'{e}rn with $\nu =
  1$ and $\rho=1,10$.  The condition number of the covariance matrix
  $\bC(\btheta)$ is placed on the top of each subplot for
  comparison. The MB is constructed from a kD-tree.  As expected, as
  $p$ increases with $w$ the condition number of $\bC_{\bW}(\btheta)$
  decreases significantly. This is consistent with Theorem
  \ref{Multilevelapproach:theo1} and Corollary
  \ref{Multilevelapproach:cor1}}
\label{numericalresults:fig1}
\end{figure*}

In Table \ref{numericalresults:table1} sparsity and construction wall
clock times of the sparse matrices $\tilde{\bC}^{i}_{\bW}(\btheta)$,
$i = t, t-1, \dots$, for various values of $i$ are shown.  The
polynomial space of the index set $\mcQ^d_w$ is restricted to TD on a
n-Sphere with $d = 10$ dimensions. The domain decomposition is formed
with a kD-tree. The level of the index set is set to $w = 7$, which
corresponds $p = 1001$. The covariance function is Mat\'{e}rn with
$\nu = 3/4$, $\rho = 3/4$. The distance criterion for each $(i,j)$
multilevel covariance matrix block is set to
\[
\tau_{i,j} := 2^{(t - i)/2}2^{(t - j)/2} \tau,
\]
for $i = 1, \dots, t$ and $j = 1 \dots, t$, where $\tau = 3 \times
10^{-6}$.

The first observation to notice is that all the sparse matrices
$\tilde{\bC}^{i}_{\bW}(\btheta)$, $i = t, t-1, \dots$ {\it are very
  well conditioned, thus numerically stable}. This is in contrast to
the original covariance matrices that are in general poorly
conditioned. The sparsity of $\tilde{\bC}^{i}_{\bW}(\btheta)$ and the
Cholesky factor $\bG$ are shown in columns 7 and 9. The construction
time $t_{con}$ of the $\tilde{\bC}^{i}_{\bW}(\btheta)$ is shown in
column 9. In column 5 $t_{ML}$ is the time required to build the
multilevel basis.  We observe that for large matrices the sparse
matrix $\tilde{\bC}^{i}_{\bW}(\btheta)$ are built efficiently.  It is
noted that the sparse matrices in Table \ref{numericalresults:table1}
are built with a direct summation method due to the dimensionality $d
= 10$.


\setlength{\tabcolsep}{6pt}

\begin{figure*}
\begin{center}
\psfrag{A}[c][t]{\small $\log_{10}(abs(\bC_{\bW}(\btheta)))$}
\begin{tabular}{c c}
\includegraphics[width=2in,height=2in]{MatrixDecay.pdf}
&
\hspace*{0mm}
\raisebox{-2.5mm}[0pt][0pt]{
\psfrag{Hist}[c][t]{\small Histogram of $\log_{10} abs(\bC_{\bW}(\btheta))$}
\psfrag{log10}[c][t]{\tiny $\log_{10} abs(\bC_{\bW}(\btheta))$ (100 bins)}
\includegraphics[width=2.1in,height=2.1in]{HistMatrixDecay.pdf}
} \\
    & \\
(a) & (b)
\end{tabular}
\end{center}
\caption{(a) Magnitude pattern and (b) histogram of $\log_{10}
  abs(\bC_{\bW}(\btheta))$ with 100 bins where $abs(\bC_{\bW}(\btheta))
  \in \R^{(N-p) \times (N-p)}$ is the magnitude of the entries of the
  matrix $\bC_{\bW}(\btheta)$.  The matrix $\bC_{\bW}(\btheta)$ is created
  with $d = 3$ dimensions, $N$ = 8,000 random locations on the sphere
  ($\bS^3_3$), and the Mat\'{e}rn covariance function with $\rho =
  10$, $\nu = 0.5$ (exponential), Total Degree index Set $\Lambda(w)$
  with $w = 4$, and $p = 35$. As observed from (a) and (b) most of
  entries of the matrix $\bC_{\bW}(\btheta)$ are very small.}
%and can thus be safely eliminated
%  without comprosing much accuracy.}
\label{numericalresults:fig2}
\end{figure*}




%% \begin{figure}[htpb]
%% \begin{center}
%% \psfrag{A-------------}[c][c]{\raisebox{0mm}[0pt][0pt]{\tiny $w = 3$}}
%% \psfrag{B-------------}[c][c]{\raisebox{0mm}[0pt][0pt]{\tiny $w = 4$}}
%% \psfrag{C-------------}[c][c]{\raisebox{0mm}[0pt][0pt]{\tiny $w = 5$}}
%% \psfrag{Error}[c][t]{\small Total degree log det relative error}
%% \psfrag{x}[c][t]{\tiny Sparsity}
%% \psfrag{y}[c][t]{\tiny $\frac{
%% |\log{det(\bC^n_{\bW})} - \log{det(\tilde \bC^n_{\bW})}|
%% }{
%% \log{det(\bC^n_{\bW})}}$}
%% \begin{tabular}{c c}
%% \includegraphics[width=3in,height=3in]{./figures/sparsitydecay.eps}
%% &
%% \psfrag{Error}[c][t]{\small Smolyak log det relative error}
%% \psfrag{A-------------}[c][c]{\raisebox{0mm}[0pt][0pt]{\tiny $w = 2$}}
%% \psfrag{B-------------}[c][c]{\raisebox{0mm}[0pt][0pt]{\tiny $w = 3$}}
%% \psfrag{Hist}[c][t]{\small Histogram of $\log_{10}|\bC_{\bW}(\btheta)|$}
%% \psfrag{log10}[c][t]{\tiny $\log_{10}|\bC_{\bW}(\btheta)|$ (100 bins)}
%% \includegraphics[width=3in,height=3in]{./figures/sparsitydecay2.eps} 
%% \\
%% (a) TD, $N$ = 16,000, $d = 5$, kD-tree & 
%% (b) SM, $N$ = 16,000, $d = 10$, kD-tree 
%% \end{tabular}
%% \end{center}
%% \caption{Relative log determinant error $\frac{| \log \det \tilde
%%     \bC_{\bW}(\btheta) - \log \det \bC_{\bW}(\btheta)| |}{|\log \det
%%     \bC_{\bW}(\btheta)|}$ with respect to the sparsity of $\tilde
%%   \bC^{n}_{\bW}$ from Random hyper-sphere data $\bS^{d}_{3}$ with $N$
%%   =16,000 observations, Mat\'{e}rn covariance parameters $\nu = 0.5$,
%%   $\rho = 10$ and binary kD-tree. The index sets $\Lambda(\omega)$ are
%%   chosen from (a) Total Degree and (b) Smolyak index sets.}
%% \label{numericalresults:fig3}
%% \end{figure}


\setlength{\tabcolsep}{7pt}
\begin{table*}[htpb]
  \caption{Sparsity test on the matrices $\tilde{\bC}^{i}_{\bW}$, $i =
    t, t-1, \dots$.  The polynomial space of the index set $\mcQ^d_w$
    is restricted to TD on a n-Sphere with $d = 10$ dimensions. The
    domain decomposition is formed from a kD-tree. The level of the
    index set is $w = 7$, which corresponds $p = 1001$. The kernel
    function is Mat\'{e}rn with $\nu = 3/4$, $\rho = 3/4$ and $\tau :=
    3 \times 10^{-6}$. The first column is the number of random
    n-Sphere nodes. The second is the maximum level of the kD tree and
    $i$ is the level of the sparse matrix $\tilde{\bC}^{i}_{\bW}$. The
    fourth column is the condition number of $\tilde{\bC}^{i}_{\bW}$,
    which is excellent.  The fifth column is the size of the matrix
    $\tilde{\bC}^{i}_{\bW}$.  The seventh column, $t_{ML}$, is the
    total time for the construction of the multilevel basis. The
    eighth column is the sparsity of $\tilde{\bC}^{i}_{\bW}$.  The
    nineth column, $t_{con}$ is the total time for the construction of
    the matrix $\tilde{\bC}^{i}_{\bW}$. The tenth column is the
    sparsity of the Cholesky factor $\bG$ (with nested dissection
    reordering) of the sparse matrix $\tilde{\bC}^{i}_{\bW}$. The last
    column is the total time to compute the Cholesky factor $\bG$.}
\begin{center}
\begin{tabular}{ r r r r c c r r c r c r}
\multicolumn{1}{c}{$N$} &
% \multicolumn{1}{c}{$d$} &
\multicolumn{1}{c}{$t$} &
\multicolumn{1}{c}{$i$} &
\multicolumn{1}{c}{$\kappa(\tilde \bC_{\bW}^{i})$} &
\multicolumn{1}{c}{Size} &
%\multicolumn{1}{c}{$\tau$} &  
\multicolumn{1}{c}{$t_{ML}$} &
\multicolumn{1}{c}{$nz$} &
\multicolumn{1}{c}{$t_{con}$} &
\multicolumn{1}{c}{$nz(\bG)$} &
\multicolumn{1}{c}{$t_{sol}$}
 \\ 
 \hline
32,000 & 4 & 4 &   5  & 15,984 &  46 &  6.3\% &   11 &  3.1\% &  1 \\
32,000 & 4 & 3 &   8  & 23,992 &  46 & 10.4\% &   30 &  5.2\% &  3 \\
32,000 & 4 & 2 &  13  & 27,996 &  46 & 15.6\% &   82 &  7.8\% &  7 \\
32,000 & 4 & 1 &  19  & 29,998 &  46 & 20.1\% &  190 & 10.4\% & 16 \\
32,000 & 4 & 0 &  23  & 30,999 &  46 & 25.7\% &  310 & 13.0\% & 17 \\
\hline
64,000 & 5 & 5 &   6  & 31,968 & 104 &  3.5\% &   21 & 1.8\%  &  3 \\
64,000 & 5 & 4 &  11  & 47,984 & 105 &  6.3\% &   90 & 3.1\%  & 12 \\
64,000 & 5 & 3 &  18  & 55,992 & 106 &  9.6\% &  270 & 5.0\%  & 18 \\
64,000 & 5 & 2 & 121  & 59,996 & 121 & 13.4\% & 624 & 6.7\%  & 34 \\
\hline
128,000 & 6 & 6 &  8  &  63,936 &  237 & 4.0 \% & 120  & 2.1 \% & 15 \\
128,000 & 6 & 5 & 17  &  95,968 &  237 & 5.5 \% & 378 & 6.7 \% & 140 \\
\end{tabular}
\vspace{5mm}
\\
\end{center}
\label{numericalresults:table1}
\end{table*}

\subsection{Estimation}

In this section estimation results are presented for the Mat\'{e}rn
covariance matrix on high dimensional n-Sphere random locations by
solving multilevel log-likelihood
\[
\hat{\btheta} : =
\argmax_{\btheta}
\tilde{\ell}^{i}_{\bW}(\bZ^{i,k,d}_{W};\btheta),
\]
where $\bZ^{i,k,d}_{W} := [\bW_t \T, \dots, \bW_i \T ] \T
\bZ^{d}_{k}$, for $i = t, t-1, \dots$. The observation data $\bZ^d_k$
is built from the n-Sphere $\bS^d_k$ for $d = 3,10$, $k = 6$ $(N =
64,000)$ and $k = 7$ $(N = 128,000)$. The covariance function is
Mat\'{e}rn for several values of $\nu$ and $\rho$. To test the
performance of the multilevel estimator, $M = 100$ realizations are
generated for each case.

The optimization problem of the log-likelihood function
\eqref{Introduction:multilevelloglikelihoodreduced2} is solved using
a fmincon iteration search for the estimates $\hat \nu$ and $\hat
\rho$ from the optimization toolbox in MATLAB \cite{Matlab2016}. The
tolerance level is set to $10^{-6}$.

In Table \ref{numericalresults:table2} the mean and standard deviation
of the Mat\'{e}rn covariance parameter estimates $\hat \nu$ and $\hat
\rho$ are presented. The mean estimate $\bbE_M [\hat{\nu}]$ refers to
the mean of $M$ estimates $\hat{\nu}$ for the $M$ realizations of the
stochastic model. Similarly, $std_M [\hat{\nu}]$ refers to the
standard deviation of the $M$ realizations. For case (a) ($d = 3$) the
error mean and std is $\approx 1\%$. For case (b) ($d = 10$) the error
of the mean increase to $\approx 10 \%$. In general, as $i$ is reduced
from $t$ there is a tendency of a drop in the standard deviation
$std_M [\hat{\nu}]$ of the estimator $\hat \nu$. However, there is
also a tendency for the accuracy of the mean to degrade somewhat,
except for (a) $N = 128,000$, $i = 12 \rightarrow 11$.
%This inconsistent
%behavior in accuracy might be due to the standard deviation being
%approximately the same magnitude as the sample mean.


\begin{table*}[htpb]
\caption{Estimation of parameters $\hat \nu$ and $\hat \rho$ with:
  Total Degree polynomial index set $\mcQ^d_w$, kD tree, and n-Sphere
  with for $d = 3$ and $d = 10$.  The observation data $\bZ^d_k$ are
  formed from the Mat\'{e}rn covariance function. The number of
  realizations of the Gaussian random field model is set to $M =
  100$. Several cases are tested and are given by the individual tables
  (a) and (b).  The first to fourth columns are self-explanatory. The
  fifth column is the mean error of $\hat \nu$ with $M$
  realization. The sixth column is the mean error of $\hat \rho$. The
  last two columns are the standard deviation of $M$ realizations of
  the parameters $\hat \nu$ and $\hat \rho$.}
\begin{center}
(a) TD, kD tree, n-Sphere, $d = 3$, $M = 100$, $\nu =
  3/4$, $\rho = 1/6$, $\tau = 5 \times 10^{-2}$
\begin{tabular}{ r r r r r r r r r r}
\multicolumn{1}{c}{$N$} & 
% \multicolumn{1}{c}{$d$} &
\multicolumn{1}{c}{$w$} &
\multicolumn{1}{c}{$t$} &
\multicolumn{1}{c}{$i$} & 
\multicolumn{1}{c}{$\bbE_M [\hat{\nu} - \nu]$} &
\multicolumn{1}{c}{$\bbE_M [\hat{\rho} - \rho]$} &
\multicolumn{1}{c}{$std_M [\hat{\nu}]$} &
\multicolumn{1}{c}{$std_M [\hat{\rho}]$} 
 \\ 
 \hline
64000 & 3 & 11 & 11 & -1.92e-04 &  4.52e-04 & 1.36e-02 & 8.17e-03 \\ 
64000 & 3 & 11 & 10 &  1.17e-03 & -5.90e-04 & 7.08e-03 & 4.04e-03 \\ 
%\hdashline
128000 & 3 & 12 & 12 & -2.51e-03 & 1.81e-03 & 8.54e-03 & 6.11e-03 \\ 
128000 & 3 & 12 & 11 & -6.90e-04 & 5.02e-04 & 4.17e-03 & 2.84e-03 \\ 
\end{tabular}
\\
\bigskip
(b) TD, kD tree, n-Sphere, $d = 10$, $M = 100$, $\nu =
  3/4$, $\rho = 3/4$, $\tau = 1 \times 10^{-5}$
\begin{tabular}{ r r r r r r r r r r}
\multicolumn{1}{c}{$N$} & 
% \multicolumn{1}{c}{$d$} &
\multicolumn{1}{c}{$w$} &
\multicolumn{1}{c}{$t$} &
\multicolumn{1}{c}{$i$} & 
\multicolumn{1}{c}{$\bbE_M [\hat{\nu} - \nu]$} &
\multicolumn{1}{c}{$\bbE_M [\hat{\rho} - \rho]$} &
\multicolumn{1}{c}{$std_M [\hat{\nu}]$} &
\multicolumn{1}{c}{$std_M [\hat{\rho}]$} 
 \\ 
 \hline
64000 & 4 & 5 & 5 &  8.70e-03 & -1.12e-02 & 1.55e-02 & 1.85e-02 \\ 
64000 & 4 & 5 & 4 & -9.31e-02 &  8.02e-02 & 1.67e-02 & 1.97e-02 \\
%\hdashline
128000 & 4 & 6 & 6 & -6.36e-03 & 5.51e-03 & 2.10e-02 & 1.72e-02 \\
128000 & 4 & 6 & 5 & -7.18e-02 & 6.27e-02 & 1.32e-02 & 1.46e-02 \\ 
\end{tabular}
\\
\bigskip
%% (c) TD, kD tree, n-Sphere, $d = 10$, $M = 100$, $\nu =
%%   1.25$, $\rho = 1$, $\tau = 10^{-7}$
%% \begin{tabular}{ r r r r r r r r r r}
%% \multicolumn{1}{c}{$N$} & 
%% % \multicolumn{1}{c}{$d$} &
%% \multicolumn{1}{c}{$w$} &
%% \multicolumn{1}{c}{$t$} &
%% \multicolumn{1}{c}{$i$} & 
%% \multicolumn{1}{c}{$\bbE_M [\hat{\nu} - \nu]$} &
%% \multicolumn{1}{c}{$\bbE_M [\hat{\rho} - \rho]$} &
%% \multicolumn{1}{c}{$std_M [\hat{\nu}]$} &
%% \multicolumn{1}{c}{$std_M [\hat{\rho}]$} 
%%  \\ 
%%  \hline
%%  64000 & 2 &  9 & 9 & -5.86e-03 & 5.21e-03 & 4.85e-02 & 3.15e-02 \\
%%  64000 & 2 &  9 & 8 & -3.37e-02 & 1.93e-02 & 3.50e-02 & 2.46e-02 \\ 
%%  64000 & 2 &  9 & 7 & -1.19e-01 & 6.92e-02 & 2.88e-02 & 2.51e-02 \\ 
%% %\hdashline
%% 128000 & 2 & 10 & 10 & -2.70e-03 & 2.74e-03 & 3.76e-02 & 2.44e-02 \\ 
%% 128000 & 2 & 10 &  9 & -2.00e-02 & 1.20e-02 & 2.47e-02 & 1.80e-02 \\ 
%% 128000 & 2 & 10 &  8 & -8.40e-02 & 5.03e-02 & 2.21e-02 & 1.89e-02 \\ 
%% \end{tabular}
\end{center}
\label{numericalresults:table2}
\end{table*}

%INFINITUMmpfg
%a5c349c2d6


\subsection{Prediction}

In this the computational performance of the multilevel Kriging solver
is analyzed. Given a fixed Mat\'{e}rn parameters $(\nu,\rho)$ the
objective is to compute the BLUP vectors $\hat \bgamma$ and $\hat
\beta$. This involves solving the system of equations $\bP^{-1}_{\bW}
\bC_{\bW}(\btheta) \bgamma_{\bW} = \bP^{-1}_{\bW} \bZ_{\bW}$ and $\hat
\bbeta = (\bX\T \bX)^{-1}$ $\bX\T(\bZ - \bC \hat \bgamma)$.

Numerical results for computing $\hat \bgamma$ and $\hat \bbeta$ for the
hypercube data set with $d = 3$ dimensions, kD tree, and the Total
Degree index set $\mcQ^d_w$ are shown in Table
\ref{numericalresults:table3}. The Mat\'{e}rn covariance coefficients
$\btheta = (\nu,\rho)$ are set to (3/4,1). The relative error of the
residual of PCG method for the unpreconditioned system is set to
$\varepsilon = 10^{-3}$. The KIFMM is set to high accuracy.

For computing the matrix vector products of the PCG iterations, the
computational break even point of the KIFMM solver is reached for $N
\approx 2,500$ compared to using the direct approach (with CPU and
GPU). The increase in computational complexity is linear with respect
to $N$. Thus all the matrix vector products for the PCG iterations are
calculated using the KIFMM.

The preconditioner $\bP_{\bW}$ is built using a combination of the GPU
and CPU. This leads to a quadratic increase in computational cost with
respect to the number of observations $N$. However, due to the high
efficiency of the implementation and $p = 120$, the break even point
for the use of the KIFMM solver is not reached, even for $N = 512,000$
observation points.

From Table \ref{numericalresults:table3} observe that condition number
of the covariance matrix $\bC$ is much larger compared to
$\bC_{\bW}$. This is already a good indication that solving the
Kriging problem will be more efficient using the multilevel approach.

The number of iterations needed to reach the same accuracy for both
approaches are significantly better with the multilevel approach
i.e. $\approx 70$ times less iterations. However, the computation of
$\bbeta$ with the single level method requires solving $p + 1 = 121$
matrix inversions of $\bC$. This is in contrast with a single matrix
inversion of $\bC_{\bW}$ with the multilevel method. In practice, we
did not solve all $p+1$ matrix inversions for the single level
approach, but measure the time required to compute a single matrix
inversion and multiplied it 121 to obtain the estimated time
complexity.  For $N = 64,000$ observations we observe efficiencies of
$\approx 7,000$ compared to the single level iterative approach.

The condition number of the covariance matrices are fairly large,
making this problem somewhat hard to solve numerically.  The results
show that 512,000 size problems with good accuracy are feasible with a
single 4-core processor and GPU. Finally, the total computational cost
varies somewhere between linear and quadratic as the number of
observations $N$ is increased.


\begin{table*}[htbp]
  \caption{Numerical results for computing $\bP^{-1}_{\bW}
    \bC_{\bW}(\btheta) \bgamma_{\bW} = \bP^{-1}_{\bW} \bZ_{\bW}$ and
    $\hat \bbeta = (\bX\T \bX)^{-1} \bX\T(\bZ - \bC \hat \bgamma)$ for
    the hypercube data set with $d = 3$ and the Total Degree index set
    $Q^d_w$. The Mat\'{e}rn covariance coefficients $\btheta =
    (\nu,\rho)$ are set to (3/4,1). The relative error of the residual
    of PCG method for the \emph{unpreconditioned system} is set to
    $\varepsilon = 10^{-3}$. The KIFMM is set to high accuracy. (a)
    The second column of is the condition number of the covariance
    matrix $\bC$, up to $N=64,000$ observations, and is compared with
    the fourth column which corresponds to the condition number of
    $\bC_W$. The third column (itr($\bC_{\bW}$)) is the number of CG
    iterations needed for convergence for $10^{-3}$ residual
    accuracy. The fifth column is the number of iterations need to
    achieve the residual error $10^{-3}$ for the unpreconditioned
    system with the preconditioner $\bP_{\bW}$.  (b) The second column
    corresponds to the wall clock times in seconds for the
    preconditioner computation. The third column is the time for
    construction of the preconditioner $\bP_{\bW}$.  The PCG iteration
    wall clock timings for $\bC_{\bW}$, by using a KIFMM, are given in
    the fourth column. The fifth is the total time to compute
    $\bgamma_{\bW}$, $\bbeta$ and the multilevel basis
    construction. The sixth column is the computational efficiency for
    computing $\bgamma_{\bW}$ vs $\bC^{-1} \bZ$ to same residual
    accuracy with respect to the number of iterations. The last column
    is the estimated efficiency of computing $\hat \bgamma$ and $\hat
    \bbeta$ with the multilevel BLUP compared to the single level
    approach, equation \eqref{Kriging}, to approximately the same
    accuracy using a CG iteration with the KIFMM. We observe the
    significant speed ups ($\approx 7,000$ for $N = 64,000$) for
    calculating the BLUP by using the multilevel approach.
  %We observe that the total computational cost varies between linear
  %and somewhere between linear and quadratic as the number of
  %observations $N$ is increased.  The last column shows the efficiency
  %for computing itr($\bC_{\bW}$) with respect to $\bC$. Notice that as
  %$N$ increases the efficiency of the multilevel method increases
  %significantly.
  %The last column shows the efficiency for computing itr($\bC_{\bW}$
  %with respect to $\bC$. Notice that as $N$ increases the efficiency
  %of the multilevel method increases significantly.
}
\begin{center}
  \bigskip
  (a) $\btheta = (3/4,1)$, $d = 3$, $w = 7$ ($p = 120$) \\
  \bigskip
\begin{tabular} { r c r c r}
  \multicolumn{1}{c}{$N$} &
  \multicolumn{1}{c}{$\kappa(\bC)$} &
    \multicolumn{1}{c}{itr($\bC$)} &
  \multicolumn{1}{c}{$\kappa(\bC_{\bW})$} &
  itr($\bC_{\bW}$) \\
  \hline
  8,000  & $3.2 \times 10^{7}$   &    1,985 & $1.8 \times 10^{4}$ &   52  \\  
  16,000  & $1.1 \times 10^{8}$  &    3,511 & $6.0 \times 10^{4}$ &   67  \\  
  32,000  & $5.6 \times 10^{8}$  &    8,259 & $3.1 \times 10^{5}$ &  116  \\  
  64,000  &  $1.8 \times 10^{9}$ &   12,680 & $9.5 \times 10^{5}$  & 165  \\  
  128,000 &                 -   &      -  &                  -  &  308   \\  
  256,000 &                 -   &      -  &                  -  &  292   \\  
  512,000 &                 -   &      -  &                  -  &  484   \\  
\end{tabular}
\\
\bigskip
(b) $\btheta = (3/4,1)$, $d = 3$, $w = 7$ ($p = 120$) \\
\bigskip
\begin{tabular} { r r r r r c r}    
  \multicolumn{1}{c}{$N$} 
  & itr($\bC_{\bW}$) & $\bP_{\bW}$ (s) & Itr (s) & Total (s) & Eff$_{\bgamma}$ &
  \multicolumn{1}{c}{Eff$_{\bgamma,\bbeta}$} \\
  \hline
  8,000   &   52  &       4   &      29  &     38 &  38 &   3,600  \\
  16,000  &   67  &      13   &      98  &    118 &  52 &   5,000  \\
  32,000  &  116  &      45   &     260  &    317 &  71 &   7,250  \\
  64,000  &  165  &     178   &     798  &    997 &  76 &   7,380  \\
  128,000 &  308  &     713   &   3,934  &  4,687 &   - &    - \\
  256,000 &  292  &   2,837   &   5,745  &  8,663 &   - &    - \\
  512,000 &  484  &   11,392  &  20,637  & 32,202 &   - &    - \\
\end{tabular}
%% \bigskip
%% (b) $\btheta = (3/4,1)$, $d = 3$, $w = 7$ ($p = 120$) \\
%% \bigskip
%% \begin{tabular} { r r r r r c r}    
%%   \multicolumn{1}{c}{$N$} 
%%   & itr($\bC_{\bW}$) & $\bP_{\bW}$ (s) & Itr (s) & Total (s) & Eff$_{\bgamma}$ &
%%   \multicolumn{1}{c}{Eff$_{\bgamma,\bbeta}$} \\
%%   \hline
%%   8,000   &   53  &       3   &      18  &     21 &  48 &  5,808 \\
%%   16,000  &   70  &      11   &      43  &     54 &  89 & 10,769 \\
%%   32,000  &  116  &      45   &     265  &    310 &  58 &  7,018 \\
%%   64,000  &  165  &     178   &     798  &    976 &  76 &  9,196\\
%%   128,000 &  308  &     713   &   3,934  &  4,647 &   - &    - \\
%%   256,000 &  292  &   2,837   &   5,745  &  8,582 &   - &    - \\
%%   512,000 &  484  &   11,392  &  20,319  & 31,711 &   - &    - \\
%% \end{tabular}
\end{center}
\label{numericalresults:table3}
\end{table*}


The multilevel approach is now tested on $d = 20$ and $d = 25$
dimensional problems. Due to the high dimensionality of these
problems, a fast summation approach is not an option. The
matrix-vector products of each iteration are computed with the direct
approach using the GPU and CPU.

In Table \ref{numericalresults:table4}(a) the numerical results for
computing $\bgamma$ and $\bbeta$ for $d = 20$ and $\theta =(5/4,10)$.
Compared to the single level iterative approach the multilevel method
is approximately 42,000 faster for $N = 64,000$ observations. Similar
results are obtained shown in Table \ref{numericalresults:table4}(b). 
for $d = 25$ and $\theta =(5/4,10)$.

\setlength{\tabcolsep}{3pt}

\begin{table*}[htbp]
  \caption{Computing Kriging for the n-sphere data set with $d = 20$
    and $d = 25$ dimensions, TD index set, and Mat\'{e}rn covariance
    function without pre-conditioner. The residual accuracy is set to
    $\varepsilon = 10^{-3}$. Since the dimension is greater than 3,
    the matrix vector products are computed directly with the GPU and
    CPU.  The description of the columns of tables (a) and (b) are the
    same as for Table \ref{numericalresults:table3}. In addition,
    column 6 corresponds to the wall clock time for computing the
    multilevel basis. (a) Computational times for solving the Kriging
    prediction for $d = 20$ and $\theta = (5/4,10)$.  The growth in
    computational cost is slightly faster than quadratic due to the
    lack of fast summation method in higher dimensions. However,
    compared to the single level iterative approach it is
    approximately 42,000 faster for $N = 64,000$ observations. (b)
    Kriging prediction for $d = 25$ and $\theta = (5/4,10)$. The
    growth in computational cost is similar.  The efficiency of this
    method is about 2,840 times faster for $N = 64,000$ observations.}
\begin{center}
(a) $\btheta = (\nu,\rho) = (5/4,10)$, $d = 20$, $w = 3$ ($p =
1771$), No precond., Direct \\
\begin{tabular} { r c c c  c r r r r r}
  \multicolumn{1}{c}{$N$} & $\kappa (\bC)$ & $\kappa (\bC_{\bW})$ &
 itr($\bC$)
  & 
 itr($\bC_{\bW}$) &  MB(s) & Itr(s) & Total(s) &  Eff$_{\bgamma,\bbeta}$ \\
  \hline
 16,000  & $5  \times 10^{7}$ & 7  & 238 & 10  &  52  &      97   &    153 & 26,700   \\
 32,000  & $1 \times 10^{8}$  & 11 & 324 & 13  & 121  &     500   &    628 & 35,160   \\
 64,000  & $2 \times 10^{8}$  & 17 & 444 & 17  & 284  &   2,600   &  2,898 & 42,050 \\
128,000  &  -                & -  &  - & 22  & 628  &  13,494   &  14,153 & -  \\
\end{tabular}\\
\bigskip
(b) $\btheta = (\nu,\rho) = (3/4,10)$, $d = 25$, $w = 2$ ($p = 351$), No precond., Direct \\
\begin{tabular} { r c c c c r r r r}
  \multicolumn{1}{c}{$N$} & $\kappa (\bC)$ & $\kappa (\bC_{\bW})$
  & itr($\bC$)
  & itr($\bC_{\bW}$) & MB(s) & Itr(s) & Total(s) & Eff$_{\bgamma,\bbeta}$ \\
  \hline
 16,000  & $2  \times 10^{6}$  &   7  & 86  & 12 &   5  &         116   &    122 & 2,400   \\
 32,000  &  $4   \times 10^{6}$ &  12 & 109 & 15 &  13  &         582   &    599 & 2,490 \\
 64,000  &  $9  \times 10^{6}$ &   21 & 147  & 18 &  30  &       2,788  &  2,821 & 2,840 \\
 128,000  &  -                  &  - & -  & 25 &  79  &      15,557   & 15,641 &  - \\
 256,000  &  -                  &  - & -  & 33 & 157  &      83,163   & 83,337 &  - \\
\end{tabular}\\
\end{center}
\label{numericalresults:table4}
\end{table*}

\section{Conclusions}

In this paper a multilevel Kriging method is developed that scales
well with high dimensions. A multilevel basis is constructed from a
kD-tree and for the choice of Total Degree polynomial basis
$\mcQ^d_w$.  The approach described in the paper has the following
characteristics and advantages:

\begin{enumerate}[i)]

\item The multilevel method is numerically stable. Hard estimation and
  prediction of large dimensional problems are now feasible.
  
\item The method is efficiently implemented by using a combination of
  MATLAB, c++ software packages and dynamic libraries.

\item Sub-exponential decay of multilevel covariance matrix
  $\bC_{\bW}$ is proven based on complex analytic extensions.
  
\item Numerical results of up to 25 dimensional problems. These
  problems are difficult to solve with traditional methods due to the
  large condition numbers, but feasible with the multilevel method.

\item The multilevel prediction approach is proven to be \emph{exact},
  in the sense that single level and multilevel prediction
  formulations are shown to be equivalent. 
   

\item The efficiency of this approach will be further improved as high
  dimensional fast summation methods are developed.

\item An A-posteriori scheme and estimates for constructing the sparse
  covariance matrix $\tilde \bC$ will be developed in a future
  paper. This will be possible with the error bounds for the entries
  of $\bC$ derived in this paper since all the constants can be
  estimated.

  
\end{enumerate}














\section*{Appendix: Polynomial Interpolation}
\label{PolynomialAppendix}

In this section we provide some background on polynomial interpolation
in high dimensions. This will be critical to estimate the decay rates
of the entries of the multilevel covariance matrix for high
dimensional problems.
%Note that this appendix can be
%  skipped as it is only used for estimating the decay of the
%  multilevel covariance matrix.}

The decay of the coefficients will directly depend on the analytic
properties of the covariance function. The traditional error estimates
of polynomial interpolation are based on multi-variate $m^{th}$ order
derivatives. However, for many cases, such as the Mat\'{e}rn
covariance function, the derivatives are too complex or expensive to
manipulate for even a moderate number of dimensions. This motivates
the study of polynomial numerical approximations based on complex
analytic extensions, which are much better suited for high dimensions.
Much of the discussion that follows has it roots in the field of
uncertainty quantification and high dimensional interpolation
\cite{nobile2008a,Castrillon2016,Griebel2016}
for partial differential
equations.


Consider the problem of approximating a function $v: \Gamma^{d}
\rightarrow \R$ on the domain $\Gamma^{d}$.  Without loss of
generality let $\Gamma : = [-1, 1]$ and $\Gamma^{d} := \prod_{n =
  1}^{d} \Gamma$. Suppose that $\mcG \subset \Gamma^{d}$, then define
the following spaces
\[
\begin{split}
  &
L^q(\mcG) := \{ v(\by)\, | \, \int_{\mcG} v(\by)^q \text{d}
\by < \infty  \}
\,\,\,
\mbox{and} \\
&
L^{\infty}(\mcG) := \{ v(\by)\, | \, \sup_{\by \in \mcG} |v(\by)|
< \infty  \}.
\end{split}
\]


Suppose that $\mcP_{ q}(\Gamma):=\text{\rm span}\{y^k,\,k=0,\dots,q\}$
i.e. the space of polynomials of degree at most $q$. Let $\mcI^{m} :
C^{0}(\Gamma) \rightarrow \mcP_{m-1}(\Gamma)$ be the univariate
Lagrange interpolant
\[
\mcI_{m}(v(\by)):=
\sum_{k=1}^{m}v(y^{(k)})l_{m,k}(y^{(k)}),
\]
where $y^{(1)}, \dots, y^{(m)}$ is a set of distinct knots on $\Gamma$
and $\{ l_{n,k} \}_{k=0}^{m}$ is a Lagrange basis of the space
$\mcP_{m-1}(\Gamma)$. The variable $m \in \Nset$
%, where $\Nset_{+} := \Nset \cup 0$,
corresponds to the order of approximation of the
Lagrange interpolant. However, for the case of the zero order
interpolation $m = 0$ corresponds to $\mcI_{0} = 0$.


\begin{remark}
For high dimensional interpolation the particular set of points
$y^{(1)}, \dots, y^{(m)}$ that we will use is the Clenshaw-Curtis
abscissas.  This is further discussed in this section. However, for
now, we assume that the points are only distinct.
  \end{remark}


For $m \geq 1$ let
\[
\Delta_{m}
:= \mcI_{m}-\mcI_{m-1},
\]
From the difference operator $\Delta_{m}$ we can readily observe that
$\mcI_{m} = \sum_{k=1}^{m} \Delta_{k}$, which is reminiscent of multi
resolution wavelet decompositions. The idea is to represent
multivariate approximation as a summation of the difference operators.

Consider the multi-index tupple $\bm = (m_1,\dots,m_d)$, where $\bm
\in \Nset^{d}$, and form the tensor product operator
$\mcS_{w,d}: \Gamma \rightarrow \R$ as
\begin{equation}
  \mcS_{w,d}
      [v(\by)]
      :
      =
 \sum_{\bm \in \bbNset^{d}: \sum_{i=1}^{d} m_i - 1  \leq w } \;\;
 \bigotimes_{n=1}^{d} {\Delta^{n}_{m_n}}(v(\by)).
\label{errorestimates:SG}
\end{equation}
Note that by ${\Delta^{n}_{m_n}}(v(\by))$ we mean that the difference
operator ${\Delta_{m_n}}$ is applied along the $n^{th}$ dimension in
$\Gamma$.


Let $C^{0}(\Gamma_d; \R) : = \{ v: \Gamma_d \rightarrow \R\,\,$ is
continuous on $\Gamma_d$ and $\max_{\by\in \Gamma_d} |v(\by)| < \infty
\}$.  From Proposition 1 in \cite{Back2011} it is shown that for any
$v \in C^0(\Gamma_d;\R)$, we have $\mcS_{w,d}[v]\in \mcQ^{d}_{w}$.
Moreover, $\mcS_{w,d}[v] = v$, for all $v \in \mcQ^{d}_{w}$. The key
observation to take away is that the operator $\mcS_{w,d}[v]$ is
\textit{exact} in the space of polynomials $\mcQ^{d}_{w}$. This will
be useful in connecting the Lagrange interpolant with Chebyshev
polynomials.


Let $T_k:\Gamma \rightarrow \R$, $k = 0, 1, \dots$, be a Chebyshev
polynomial over $\Gamma$, which are defined recursively as follows:
$T_0(y) = 1$, $T_1(y) = y$, $\dots$, $T_{k+1}(y) = 2yT_{k}(y) -
T_{k-1}(y)$, $\dots$, where $y \in \Gamma$. Chebyshev polynomials are
well suited for the approximation of functions with analytic
extensions on a complex region bounded by a Bernstein ellipse. They
bypassing the need of using derivative information and sharp bounds on
the error are readily available. Suppose that $\sigma > 0$ and denote
by
\[
\begin{split}
  \mcE_{\sigma} := \Big\{
  z \in \bbC, \sigma \geq
\delta \geq 0 ;\,\Real{z} = \frac{e^{\delta} + e^{-\delta}
}{2}cos(\theta) 
\Imag{z} = \frac{e^{\delta} 
  - e^{-\delta}}{2}sin(\theta),
\theta \in [0,2\pi)
  \Big\}
\end{split}
  \]
as the region bounded by a Bernstein ellipse (see Figure
\ref{erroranalysis:sparsegrid:polyellipse}).

The following theorem is based on complex analytic extensions on
$\mcE_{\sigma}$ and provides a control for the Chebyshev polynomial
approximation.

\begin{theorem}
Suppose that for $u:\Gamma \rightarrow \R$ there exists an analytic
extension on $\mcE_{\sigma}$. If $|u| \leq M < \infty$ on
$\mcE_{\sigma}$ then there exists a sequence of coefficients
$|\alpha_k| \leq M / e^{k\sigma}$ such that $u \equiv \alpha_0 +
2\sum_{k = 1}^{\infty} \alpha_{k} T_{k}$ on $\mcE_{\sigma}$. Moreover,
if $y \in \Gamma$ then
\[
%\begin{multline*}
%\shoveright{|q(y) - \alpha_0  - 2\sum_{k = 1}^{n} \alpha_{k} T_{k}(y)|
%\leq 
%\frac{2M}{e^{\sigma} - 1} e^{-n \sigma}.}
|q(y) - \alpha_0  - 2\sum_{k = 1}^{n} \alpha_{k} T_{k}(y)|
\leq 
\frac{2M}{e^{\sigma} - 1} e^{-n \sigma}.
%\end{multline*}
\]
\label{errorestimates:theorem}
\end{theorem}
\begin{proof}
See Theorem 2.25 in \cite{Khoromskij2018}
\end{proof}


\begin{figure}[htb]%[12]{r}{7cm}%[htp]
\begin{center}
\begin{tikzpicture}
    \begin{scope}[font=\scriptsize]

      
      \filldraw[fill=blue!20,
      semitransparent] (0,0) ellipse (2 and 1);

    \draw [->] (-2.5, 0) -- (2.5, 0) node [below left]  {$\Real $};
    \draw [->] (0,-1.5) -- (0,1.5) node [below left] {$\Imag$};
    \draw (1,-3pt) -- (1,3pt)   node [above] {$1$};
    \draw (-1,-3pt) -- (-1,3pt) node [above] {$-1$};
    \end{scope}
    
    \node [below right] at (-2.5,1.25) {$\mcE_{\sigma}$};

    \node [] at (0.75,1.25) {$\frac{e^{
          \sigma} - e^{- \sigma}}{2}$};

    
    \node [] at (2.75,0.25) {$\frac{e^{
      \sigma} + e^{- \sigma}}{2}$}; 
    
\end{tikzpicture}
\end{center}
\caption{Complex region bounded by the Bernstein ellipse.}
\label{erroranalysis:sparsegrid:polyellipse}
\end{figure}

We can now connect the error due to the Lagrange interpolation with
Chebyshev expansions. It is known that if $u \in C(\Gamma,\R)$ then
\[
\|(I - \mcI_{m})u\|_{L^{\infty}(\Gamma)} \leq
(1 + \Lambda_{m})
\min_{h \in \mcP_{m-1}} \| u - h \|_{L^{\infty}(\Gamma)},
\]
where $\Lambda_{m}$ is the Lebesgue constant (See Lemma 7 in
\cite{babusk_nobile_temp_10}). Note that $I:C^{d}(\Xi;\R) \rightarrow
C^{d}(\Xi;\R)$ refers to the identity operator and the domain $\Xi$ is
taken from context. For the previous case $\Xi = \Gamma$.  Bounds on
$\Lambda_{m}$ are known in the context of the location of the knots
$y^{(1)}, \dots, y^{(m)} \in \Gamma$. In this article we restrict our
attention to Clenshaw-Curtis abscissas
%\[
\[
y^{(j)} = -\cos \left( \frac{\pi(j-1)}{m - 1} \right),\,\, j =
1,\dots, m
\]
%\]
and $\Lambda_m$ is bounded by $2\pi^{-1}(\log{(m-1)} + 1) \leq 2m - 1$
(see \cite{babusk_nobile_temp_10}).  Since the interpolation operator
$\mcI_{m}$ is exact on $\mcP_{m - 1}$, then if $u:\Gamma \rightarrow
\R$ has an analytic extension in $\mcE_{\sigma}$ we have from Theorem
\ref{errorestimates:theorem} (following a similar approach as in
\cite{babusk_nobile_temp_10}) that
\[
\begin{split}
\|(I - \mcI_{m})u\|_{L^{\infty}(\Gamma_n)}
\leq
(1 + \Lambda_{m})
\frac{2M}{e^{\sigma} - 1} e^{-\sigma (m-1)}
\leq 
2 C(M,\sigma) m e^{-\sigma (m-1)},
\end{split}
\]
where $C(M,\sigma_n) := \frac{2M}{(e^{ \sigma} - 1)}$. We then
conclude that for all $k = 1,\dots, m$
\begin{equation}
\begin{split}
\| \Delta_{k}(u) \|_{L^{\infty}(\Gamma)} 
&=
\|
\mcI^{m}(u) - \mcI^{m-1}(u)
\|_{L^{\infty}(\Gamma)} 
\leq
\|(I - \mcI_{m})u\|_{L^{\infty}(\Gamma)}
+
\|(I - \mcI_{m-1})u\|_{L^{\infty}(\Gamma)} \\
&\leq
e^{2\sigma}C(M,\sigma) m e^{-\sigma m}.
\end{split}
\label{interpolation:eqn1}
\end{equation}
Let $\mcE_{\sigma,n} \subset \bbC^{d}$ a complex region bounded by a
Bernstein ellipse such that the restriction on $\Gamma_{d}$ is along
the $n^{th}$ dimension and form the polyellipse $\mcE^{d}_{\sigma}:=
\prod_{n=1}^{d} \mcE_{\sigma,n}$.  Suppose that $v:\mcE^{d}_{\sigma}
\rightarrow \bbC$ is analytic on $\mcE^{d}_{\sigma}$ and let
$\tilde{M}(v) := \max_{\bz \in \mcE^{d}_{\sigma}} |v(\bz)|$.

Note we refer to $\mcI^{n}_{m}$ as the Lagrange operator of order $m$
along the $n^{th}$ dimension and similarly $\mcP^{n}_{m-1}$ is the
space of the span of univariate polynomials up to degree $m-1$ along
the $n^{th}$ dimension.  Form the tensor product $\bI^{d}_{m} :=
\mcI^{1}_{m} \times \dots \times \mcI^{d}_{m}$, thus $\bI:C(\Gamma,\R)
\rightarrow \bbP$ where $\bbP := \mcP^{1}_{m-1} \times \dots \times
\mcP^{d}_{m-1}$. From Theorem 2.27 in \cite{Khoromskij2018} we can
conclude that for a finite dimension $d$, as $m \rightarrow \infty$
then $\bI^{d}_{m}[v] \rightarrow v$.

Applying equation \eqref{interpolation:eqn1} to equation
\eqref{errorestimates:SG} we have that
\begin{equation}
\begin{split}
& \| (I - \mcS_{w,d})
 v(\by)
 \|_{L^{\infty}(\Gamma^{d})}
 \leq
 \left\| \sum_{\bm \in \bbNset^{d}: \sum_{i=1}^{d} m_i - 1 > w } \;\;
 \bigotimes_{n=1}^{d} {\Delta^{n}_{m_n}}(v(\by))\right\|_{L^{\infty}(\Gamma^d)} \\
 &\leq
 \sum_{\bm \in \bbNset^{d}: \sum_{i=1}^{d} m_i - 1 > w } \;\;
 \bigotimes_{n=1}^{d} \|{\Delta^{n}_{m_n}}(v(\by))\|_{L^{\infty}(\Gamma^d)} 
 \leq
 \sum_{\bm \in \bbNset^{d}: \sum_{i=1}^{d} m_i - 1 > w }
 e^{2d} C(M,\sigma)^{d} \\
 &
 \left( \prod_{n=1}^{d} m_n\right) \exp{\left( -\sum_{n=1}^{d}
   \sigma m_{n} \right)}.
% \\
%  &\leq
% \sum_{\bk \in \bbNset^{d}_{0}: \sum_{i=1}^{d} k_i > w }
% e^{2d} C(M,\sigma)^{d} \left( \prod_{n=1}^{d} (k_n + 1)\right)
% \exp{\left( -\sum_{n=1}^{d}
%   \sigma (k_{n}+1) \right)}.
\end{split}
\label{interpolation:eqn2}
\end{equation}

By applying Theorem 2.10 and Corollary 2.11 in \cite{Griebel2016} if
$ w \geq  d$ and $p( d, w) \geq
\left(\frac{2  d}{\kappa( d)}\right)^{ d}$, where
$\kappa( d) := \sqrt[\leftroot{-2}\uproot{2}  d]{
  d!} >  d/e$ (Sterling approximation), then for any $\hat
\sigma \in \R_{+}$
\begin{equation}
\begin{split}
 & \sum_{\bk \in \bbNset^{ d}_{0}: \sum_{i=1}^{ d} k_i  >  w }
 \exp{\left( -\sum_{n=1}^{ d} \hat \sigma
   k_{n} \right)}
 \leq
 \sum_{\bk \in \bbNset^{d}_{0}: \hat \sigma \sum_{i=1}^{ d} k_i  \geq  w \hat \sigma  }
 \exp{\left( -\sum_{n=1}^{ d}
   \hat \sigma k_{n} \right)} \\
 &\leq
 \hat \sigma  d e
 \left( \frac{e^{\hat \sigma}}{1 - e^{-\hat \sigma}} \right)^{ d}
 \exp \left(-\frac{ d}{e} \hat \sigma  p^{\frac{1}{ d}}
 \right) p^{\frac{ d-1}{ d}}.
\end{split}
\label{interpolation:eqn3}
\end{equation}
where $\bk \in \bbNset^{d}_{0}$ and $\bk:=(k_1,\dots,k_d)$.






Following the same approach as in \cite{Griebel2016} observe that for
$0 < \delta < 1$ we can obtain a bounded constant $c_{n,\delta} \leq
(e\sigma \delta)^{-1}$ such that $m_n \exp(-\sigma m_n) \leq (e\sigma
\delta)^{-1}$ $\exp(-\sigma m_n (1 - \delta))$. Set $\hat \sigma :=
\sigma (1 - \delta)$ and by combining equations
\eqref{interpolation:eqn2} and \eqref{interpolation:eqn3} we have
proven the following result.

\begin{lemma} Suppose that $0< \delta < 1$, $\hat
  \sigma := \sigma (1 - \delta)$, and $p(d,w) \geq \left(\frac{2
    d}{\kappa(d)}\right)^{d}$ then
  \[
  \begin{split}
 &\| (I - \mcS_{w,d})
 v(\by)
 \|_{L^{\infty}(\Gamma^{d})}
 \leq 
 \frac{C(\tilde M,\sigma)^d e^{d - \sigma(1 - \delta) + 1} \hat \sigma d }
 {
(\sigma \delta)^{d}}
 \left( \frac{e^{\hat \sigma}}{1 - e^{-\hat \sigma}} \right)^{d} 
 \exp \left(-\frac{d}{e} \hat \sigma  p^{\frac{1}{d}}
 \right) p^{\frac{d-1}{d}}.
 \end{split}
 \]
 \label{interpolation:lemma1}
\end{lemma}


\begin{remark}
The restriction $p(d,w) \geq \left(\frac{2
  d}{\kappa(d)}\right)^{d}$ is not strict and can be relaxed such that
sub-exponential convergence is still obtained.  We refer the reader to
the bound of the Gamma function in Lemma 2.5 (\cite{Griebel2016}) and
it's application in the proofs of Theorem 2.10 and Corollary 2.11.
\label{interpolation:remark1}
\end{remark}




\noindent 
\textbf{Acknowledgements:} I appreciate the help and advice from
George Biros and Lexing Ying, for setting up the KIFMM packages.  I
also appreciate the support that King Abdullah University of Science
and Technology has provided to this project.




\bibliographystyle{abrev}
%%%%%%%%%%%%%%%%%%%%%%%%%%%%%%%%%%%%%%%%%%%%%
%
% Latex file for:
% 
% 
%%%%%%%%%%%%%%%%%%%%%%%%%%%%%%%%%%%%%%%%%%%%%

%\RequirePackage{fix-cm}
\documentclass[11pt,final]{amsart}       % one column (second format)

\usepackage{epsfig, epsf, graphicx, float, color}
\usepackage{pstricks, psfrag}
\usepackage{amssymb,amsthm}
\usepackage[foot]{amsaddr}
\usepackage{verbatim,enumerate,hyperref}
\usepackage{setspace,mathtools,wrapfig}
\usepackage[numbers,sort&compress]{natbib}
\usepackage{algorithm2e}
\usepackage{steinmetz}
\usepackage{tikz-qtree,tikz-qtree-compat}
\usepackage{tikz,pgf}
\usepackage{arydshln}
\usepackage[margin=1in]{geometry}

\usetikzlibrary{decorations,decorations.markings,decorations.text}
\usetikzlibrary{arrows.meta}
\usetikzlibrary{patterns}


\usetikzlibrary{positioning,patterns}
\usetikzlibrary{calc,fadings,decorations.pathreplacing,arrows,datavisualization.formats.functions,shapes.geometric}




%\smartqed % flush right qed marks, e.g. at end of proof

% Macros

\def\dfrac#1#2{{\displaystyle{#1\over#2}}}
\def\VS{{\vskip 3mm\noindent}}
\def\boxit#1{\vbox{\hrule\hbox{\vrule\kern6pt
          \vbox{\kern6pt#1\kern6pt}\kern6pt\vrule}\hrule}}
\def\refhg{\hangindent=20pt\hangafter=1}
\def\refmark{\par\vskip 2mm\noindent\refhg}
\def\naive{\hbox{naive}}
\def\itemitem{\par\indent \hangindent2\pahttprindent \textindent}
\def\var{\hbox{var}}
\def\cov{\hbox{cov}}
\def\corr{\hbox{corr}}
\def\trace{\hbox{trace}}
\def\refhg{\hangindent=20pt\hangafter=1}
\def\refmark{\par\vskip 2mm\noindent\refhg}
\def\Normal{\hbox{Normal}}
\def\povr{\buildrel p\over\longrightarrow}
\def\ccdot{{\bullet}}
\def\bse{\begin{eqnarray*}}
\def\ese{\end{eqnarray*}}
\def\be{\begin{eqnarray}}
\def\ee{\end{eqnarray}}
\def\bq{\begin{equation}}
\def\eq{\end{equation}}
\def\bse{\begin{eqnarray*}}
\def\ese{\end{eqnarray*}}
\def\pr{\hbox{pr}}
\def\CV{\hbox{CV}}
\def\wh{\widehat}
\def\T{^{\rm T}}
\def\myalpha{{\cal A}}
\def\th{^{th}}

% Color corrections in text
\newcommand{\corb}[1]{\textcolor{blue}{#1}}
\newcommand{\corred}[1]{\textcolor{black}{#1}}
\newcommand{\corblue}[1]{\textcolor{black}{#1}}

%\renewcommand{\baselinestretch}{1} % Change this 1.5 or whatever
%\newcommand{\qed}{\hfill\hfill\vbox{\hrule\hbox{\vrule\squarebox
%   {.667em}\vrule}\hrule}\smallskip}


\newcommand{\bR}{\mathbf{R}}
\newcommand{\bD}{\mathbf{D}}
\newcommand{\bI}{\mathbf{I}}
\newcommand{\bL}{\mathbf{L}}
\newcommand{\bG}{\mathbf{G}}
\newcommand{\bW}{\mathbf{W}}
\newcommand{\bP}{\mathbf{P}}
\newcommand{\bU}{\mathbf{U}}
\newcommand{\bC}{\mathbf{C}}
\newcommand{\bA}{\mathbf{A}}
\newcommand{\bE}{\mathbf{E}}
\newcommand{\bF}{\mathbf{F}}
\newcommand{\bK}{\mathbf{K}}
\newcommand{\bM}{\mathbf{M}}
\newcommand{\bJ}{\mathbf{J}}
\newcommand{\bH}{\mathbf{H}}
\newcommand{\bQ}{\mathbf{Q}}
\newcommand{\bS}{\mathbf{S}}
\newcommand{\bV}{\mathbf{V}}
\newcommand{\bX}{\mathbf{X}}
\newcommand{\bY}{\mathbf{Y}}
\newcommand{\bZ}{\mathbf{Z}}
\newcommand{\bh}{\mathbf{h}}
\newcommand{\bx}{\mathbf{x}}
\newcommand{\by}{\mathbf{y}}
\newcommand{\bv}{\mathbf{v}}
\newcommand{\bz}{\mathbf{z}}
\newcommand{\bs}{\mathbf{s}}
\newcommand{\ba}{\mathbf{a}}
\newcommand{\bb}{\mathbf{b}}
\newcommand{\bo}{\mathbf{o}}
\newcommand{\bc}{\mathbf{c}}
\newcommand{\bd}{\mathbf{d}}
\newcommand{\bbe}{\mathbf{e}}
\newcommand{\bff}{\mathbf{f}}
\newcommand{\bqq}{\mathbf{q}}
\newcommand{\bve}{\mathbf{e}}
\newcommand{\bu}{\mathbf{u}}
\newcommand{\bw}{\mathbf{w}}
\newcommand{\bm}{\mathbf{m}}
\newcommand{\bg}{\mathbf{g}}
\newcommand{\bn}{\mathbf{n}}
\newcommand{\bk}{\mathbf{k}}
\newcommand{\bt}{\mathbf{t}}
\newcommand{\bbf}{\mathbf{f}}
\newcommand{\cS}{\cal{S}}
\newcommand{\bmu}{\boldsymbol{\mu}}
\newcommand{\bxi}{\boldsymbol{\xi}}
\newcommand{\bsigma}{\boldsymbol{\sigma}}
\newcommand{\bgamma}{\boldsymbol{\gamma}}
\newcommand{\btau}{\boldsymbol{\tau}}
\newcommand{\brho}{\boldsymbol{\rho}}
\newcommand{\blambda}{\boldsymbol{\lambda}}
\newcommand{\bdelta}{\boldsymbol{\delta}}
\newcommand{\btheta}{\boldsymbol{\theta}}
%\newcommand{\btheta}{\mathbf{\theta}}
\newcommand{\bvartheta}{\boldsymbol{\vartheta}}
\newcommand{\bpsi}{\boldsymbol{\psi}}
\newcommand{\bphi}{\boldsymbol{\phi}}
\newcommand{\bepsilon}{\boldsymbol{\epsilon}}
\newcommand{\balpha}{\boldsymbol{\alpha}}
\newcommand{\bbeta}{\boldsymbol{\beta}}
\newcommand{\bSigma}{\boldsymbol{\Sigma}}
\newcommand{\bLambda}{\boldsymbol{\Lambda}}
\newcommand{\bOmega}{\boldsymbol{\Omega}}
\newcommand{\br}{\boldsymbol{r}}
\newcommand{\0}{\mathbf{0}}
\newcommand{\1}{\mathbf{1}}
\newcommand{\binfty}{\boldsymbol{\infty}}
\newcommand{\E}{\mbox{E}}


\newcommand{\tc}[2]{\textcolor{#1}{#2}}
\def\scrU{\mathscr{U}}
\newcommand{\PP}{\mathbb{P}}
\newcommand{\idxset}{\Lambda}


% caligraphic names
\newcommand{\mcA}{{\mathcal A}}
\newcommand{\mcB}{{\mathcal B}}
\newcommand{\mcC}{{\mathcal C}}
\newcommand{\mcD}{{\mathcal D}}
\newcommand{\mcE}{{\mathcal E}}
\newcommand{\mcF}{\mathcal{F}}
\newcommand{\mcG}{\mathcal{G}}
\newcommand{\mcI}{{\mathcal I}}
\newcommand{\mcJ}{{\mathcal J}}
\newcommand{\mcK}{{\mathcal K}}
\newcommand{\mcL}{{\mathcal L}}
\newcommand{\mcO}{{\mathcal O}}
\newcommand{\mcN}{{\mathcal N}}
\newcommand{\mcM}{{\mathcal M}}
\newcommand{\mcP}{{\mathcal P}}
\newcommand{\mcQ}{{\mathcal Q}}
\newcommand{\mctQ}{\tilde {\mathcal Q}}
\newcommand{\mcR}{{\mathcal R}}
\newcommand{\mcS}{\mathcal{S}}
\newcommand{\mcT}{{\mathcal T}}
\newcommand{\mcU}{\mathcal{U}}
\newcommand{\mcX}{\mathcal{X}}
\newcommand{\mcY}{\mathcal{Y}}



% boldface
\newcommand{\ii}{\mathbf{i}}
\newcommand{\jj}{\mathbf{j}}
\newcommand{\pp}{\mathbf{p}}
\newcommand{\rr}{\mathbf{r}}
\newcommand{\mm}{\mathbf{m}}
\newcommand{\qq}{\mathbf{q}}
\newcommand{\UU}{\mathbf{U}}
\newcommand{\FF}{\mathbf{F}}
\newcommand{\aalpha}{\boldsymbol{\alpha}}
\newcommand{\rrho}{\boldsymbol{\rho}}
\newcommand{\ttheta}{\boldsymbol{y}}
\newcommand{\oone}{\boldsymbol{1}}

% sets
\newcommand{\bbR}{\mathbb{R}}
\newcommand{\bbX}{\mathbb{X}}
\newcommand{\bbN}{\mathbb{N}}
\newcommand{\bbE}{\mathbb{E}}
\newcommand{\bbF}{\mathbb{F}}
\newcommand{\bbS}{\mathbb{S}}
\newcommand{\bbK}{\mathbb{K}}
\newcommand{\bbP}{\mathbb{P}}
\newcommand{\bbC}{\mathbb{C}}
\newcommand{\bbJ}{\mathbb{J}}
\newcommand{\bbI}{\mathbb{I}}
\newcommand{\Nset}{\mathbb{N}_0}
\newcommand{\cset}{{\mathbb C}}
\newcommand{\rset}{{\mathbb R}}
\newcommand{\nset}{{\mathbb N}}
\newcommand{\bbNset}{{\mathbb N}}
\newcommand{\qset}{{\mathbb Q}}
\newcommand{\pset}{{\mathbb P}}
\newcommand{\Pol}{\mathbb{P}}
\newcommand{\eset}[1]{{\mathbb E} \left[ #1 \right] }


% misc.
\newcommand{\Grad}{\nabla}
\newcommand{\ssy}{\scriptscriptstyle}
\newcommand{\dist}{\operatorname{dist}}
\newcommand{\KL}{Karhunen--\Loeve }
\newcommand{\lv}{w}
\def\scrG{\mathscr{G}}
\newcommand{\Real}{\mathop{\text{\rm Re}}}
\newcommand{\Imag}{\mathop{\text{\rm Im}}}
\newcommand{\bno}{n}

\newcommand{\func}{u}



% New Operators
\DeclareMathOperator*{\esssup}{ess\,sup}
\DeclareMathOperator*{\essinf}{ess\,inf}


\newcommand{\sJ}[1]{
\begin{bmatrix*}[r]
  \bJ^{#1}_R   & -\bJ^{#1}_I  \\
  \bJ^{#1}_{I}  & \bJ^{#1}_R   \\
\end{bmatrix*}
}

\newcommand{\szv}[1]{
\begin{bmatrix}
  \bz^{#1}_{R} \\
  \bz^{#1}_{I} \\
\end{bmatrix}
}

\newcommand{\sfv}[1]{
\begin{bmatrix}
  \bbf^{#1}_{R} \\
  \bbf^{#1}_{I} \\
\end{bmatrix}
}

% Flipped

\newcommand{\sJflip}[1]{
\begin{bmatrix*}[r]
  -\bJ^{#1}_R   & \bJ^{#1}_I  \\
  \bJ^{#1}_{I}  & \bJ^{#1}_R   \\
\end{bmatrix*}
}

\newcommand{\szvflip}[1]{
\begin{bmatrix*}[r]
  -\bz^{#1}_{I} \\
  \bz^{#1}_{R} \\
\end{bmatrix*}
}

\newcommand{\sfvflip}[1]{
\begin{bmatrix*}[r]
  -\bbf^{#1}_{I} \\
  \bbf^{#1}_{I} \\
\end{bmatrix*}
}


\newcommand{\BallTaylor}{
\left(\bx_{0},
  \left[
    \begin{array}{c}
  \bqq \\
  \0
  \end{array}
  \right]
+ t
\left[\begin{array}{c}
  \bv_{R} \\
  \bv_{I}
  \end{array}
  \right]
  \right)
}


\def\sD{\mathcal{D}}
\def\sN{\mathcal{N}}
\def\sC{\mathcal{C}}



\def\R{\Bbb{R}}
\newcommand{\verteq}[0]{\begin{turn}{90} $=$\end{turn}}
%\newcommand{\Pr}{\mbox{Pr}}
% \renewcommand{\baselinestretch}{1.25}


\newcommand{\supess}{\mbox{ess} \operatornamewithlimits{sup}}

\newtheorem{asum}{Assumption}
\newtheorem{cond}{Condition}
\newtheorem{exam}{Example}
\newtheorem{prop}{Proposition}
\newtheorem{corollary}{Corollary}
\newtheorem{definition}{Definition}
\newtheorem{remark}{Remark}
\newtheorem{lemma}{Lemma}
\newtheorem{theorem}{Theorem}



\newcommand{\argmax}{\operatornamewithlimits{argmax}}
\newcommand{\argmin}{\operatornamewithlimits{argmin}}


% Colors
\def\boxit#1{%
  \smash{\color{blue}\fboxrule=1pt\relax\fboxsep=2pt\relax%
  \llap{\rlap{\fbox{\vphantom{0}\makebox[#1]{}}}~}}\ignorespaces
}

\def\gboxit#1{%
  \smash{\color{darkgreen}\fboxrule=1pt\relax\fboxsep=2pt\relax%
  \llap{\rlap{\fbox{\vphantom{0}\makebox[#1]{}}}~}}\ignorespaces
}

\definecolor{darkblue}{rgb}{0,0.08,0.4}
\definecolor{brightblue}{rgb}{0.65,0.85,0.85}
\definecolor{darkred}{rgb}{0.8,0.2, 0.2}
\definecolor{darkgreen}{rgb}{0, 0.6, 0}
\definecolor{blueish}{rgb}{0.1176, 0.5647, 1.0000}

\definecolor{darkorange}{RGB}{255,140,0}

\definecolor{colorone}{rgb}{0.1176,0.5647,1.0000}
\definecolor{colortwo}{rgb}{0.5608,0.7373,0.5608}



\begin{document}



\pgfkeys{/pgf/decoration/.cd,
      distance/.initial=10pt
}  

\pgfdeclaredecoration{add dim}{final}{
\state{final}{% 
\pgfmathsetmacro{\dist}{5pt*\pgfkeysvalueof{/pgf/decoration/distance}/abs(\pgfkeysvalueof{/pgf/decoration/distance})}    
          \pgfpathmoveto{\pgfpoint{0pt}{0pt}}             
          \pgfpathlineto{\pgfpoint{0pt}{2*\dist}}   
          \pgfpathmoveto{\pgfpoint{\pgfdecoratedpathlength}{0pt}} 
          \pgfpathlineto{\pgfpoint{(\pgfdecoratedpathlength}{2*\dist}}     
          \pgfsetarrowsstart{latex}
          \pgfsetarrowsend{latex}
          \pgfpathmoveto{\pgfpoint{0pt}{\dist}}
          \pgfpathlineto{\pgfpoint{\pgfdecoratedpathlength}{\dist}} 
          \pgfusepath{stroke} 
          \pgfpathmoveto{\pgfpoint{0pt}{0pt}}
          \pgfpathlineto{\pgfpoint{\pgfdecoratedpathlength}{0pt}}
}}

\tikzset{dim/.style args={#1,#2}{decoration={add dim,distance=#2},
                decorate,
                postaction={decorate,decoration={text along path,
                                                 raise=#2,
                                                 text align={align=center},
                                                 text={#1}}}}}



\title{High dimensional multilevel kriging: A computational
  mathematics approach}

%\title{High dimensional multilevel kriging: A computational
%  mathematics approach \thanks{This material is based upon work
%   supported by the National Science Foundation under Grant
%    No. 1736392.}}



\author{Julio E. Castrill\'on-Cand\'as ${\dagger}$} 
  \email{jcandas@bu.edu}


 \address{
   ${\ddagger}$ Department of Mathematics and Statistics, 
  Boston University, Boston, MA 
  }
   

%%%%%%%%%%%%%%%%%%%%%%%%%%%%%%%%%%%%%%%%%%%%%%%%%%%%%%%%%%%%%%%%%%%%%%%%




\begin{abstract}
With the advent of massive data sets much of the computational science
and engineering communities have been moving toward data-driven
approaches such as regression and classification. However, they
present a significant challenge due to the increasing size, complexity
and dimensionality of the problems.  In this paper a multilevel
Kriging method that scales well with the number of observations and
dimensions is developed.  A multilevel basis is constructed that is
adapted to a kD-tree partitioning of the observations.  Numerically
unstable covariance matrices with large condition numbers are
transformed into well conditioned multilevel matrices without
compromising accuracy. Moreover, it is shown that the multilevel
prediction \emph{exactly} solves the Best Linear Unbiased Predictor
(BLUP), but is numerically stable.  The multilevel method is tested on
numerically unstable problems of up 25 dimensions. Numerical results
show speedups of up to 42,050 for solving the BLUP problem but to the
same accuracy than the traditional iterative approach.
\end{abstract}



\maketitle

\noindent
    {\it Keywords:} Hierarchical Basis, Machine Learning, High
    Performance Computing, Sparsification of Covariance Matrices, Fast
    Multipole Method

    






%%%%%%%%%%%%%%%%%%%%%%%%%%%%%%%%%%%%%%%%%%%%%%%%%%%%%%%%%%%%%%%%%%%%%%%%


%% Introduction  ------------------------------------------------------
%%

\section{Introduction}





%\item {\bf What is the problem?  Why is it hard?}

Massive data sets arise from many fields, including, but not limited
to commerce, astrophysical sky-surveys, environmental data, and tsunami
warning systems.  With the advent of big data sets much of the
computational science and engineering communities have been moving
toward data-driven approaches to regression and classification. These
approaches are effective since the underlying data is incorporated
into the optimization. However, they present a numerical challenge due
to increasing size, complexity and dimensionality.

%\item {\bf How is it done today, and what are the limits of current
%practice?}

Due to the high dimensionality of the underlying data many modern
machine learning methods, such as classification and regression
algorithms, seek a balance between accuracy and computational
complexity. How efficient this balance is depends on the approach.
Linear methods are fast, but only work well when there is linear
separation of the data.

For non-linear description of the data, kernel approaches have been
effective under certain circumstances.  These methods rely on Tikhonov
regularization of the data to obtain a functional representation,
where it is assumed that the noise model of the phenomena is
known. However, this assumption is not necessarily satisfied in
practice and can lead to significant errors as the algorithm cannot
distinguish between noise and the underlying phenomena.


To incorporate the variability of the noise model a class of machine
learning algorithms based on Bayes method have been developed. In this
approach the noise model is assumed to be known up to a class of
probability distributions and an optimal choice is made that fits the
training data and noise. For example, from the Geo-statistics
community a well known approach to identifying the underlying data and
noise model is known as Kriging \cite{Nielsen2002}.  The noise model
parameters are estimated from the Maximum Likelihood Estimation (MLE)
of the likelihood function.

Kriging methods are effective in separating the underlying phenomena
from the noise model. However, in practice the covariance matrices
tend to be ill-conditioned with increasing number of observations
making Kriging methods numerically fragile. Moreover, most
applications are limited to 2 or 3 dimensions. A brief literature
review can be found in \cite{Castrillon2015}.

Kriging methods from the computational mathematics perspective have
been developed using skeletonization factorizations \cite{Minden2016},
low-rank \cite{nowak2013} and Hierarchical Matrices (HM)
\cite{khoromskij2009,Litvinenko2019,Geoga2020} approaches. These
methods are very promising. In particular, for the HM approaches they
have been shown to be near optimal. They work well for low dimensions.
However, they are still subject to ill-conditioning and usually a
nugget is added to change the model to make it more numerically
stable. Moreover, the data is assumed to have zero mean, which many
times will not be the case.

In \cite{Castrillon2015} a novel algorithm to solve Kriging problems
is proposed. The method is fast and robust. In particular, it can
solve Kriging problems that where not tractable with previous
methods. A nugget is not assumed, nor zero mean data.  However, this
approach is limited to 2 or 3 dimensions and the computational cost
scales very fast with spatial dimension, thus making it impractical
for high dimensional problems.

In this paper we extend the Kriging approach in \cite{Castrillon2015}
using binary trees, which are well suited for high dimensional
problems. Ill-conditioned covariance matrices are transformed to
numerically stable multilevel covariance matrices without compromising
accuracy. In addition, a new distance criterion is developed to build
sparse multilevel covariance matrices.  Furthermore, sharper decay
estimates of the coefficients of the multivariate covariance matrix
are derived based on analytic extensions that are well suited for high
dimensional problems. Much of this theory and notation is borrowed
from uncertainty quantification and high dimensional integration for
partial differential equations
\cite{nobile2008a,Castrillon2016,Griebel2016}.

The Kriging estimation is transformed into a multilevel form based on
the numerically stable multilevel covariance matrix. In practice a
sparse version of the multilevel covariance matrix is used. A distance
dependent method is used to build to a sparse version. Sharp decay
estimates (sub-exponential) of the multilevel covariance matrices are
derived using complex analytic extensions of the covariance function
instead of Taylor series expansions, which are infeasible for
relatively large dimensional problems. The numerical results show that
the estimation is solved to good accuracy for a large number of
observations.

The Kriging prediction step is remapped into an equivalent multilevel
formulation that is numerically stable. It is shown that the solution
to the multilevel prediction form \emph{exactly} solves the Best
Linear Unbiased Prediction (BLUP) problem.  To my knowledge, this is a
feature that is unique to the multilevel approach.  If the covariance
matrix is ill-conditioned, then it is not possible to solve the
problem accurately on a computer with a fixed machine
precision. However, the BLUP solution arises from a constrained
optimization problem. By taking advantage of this fact, the multilevel
approach side steps the inversion of the covariance matrix and
directly searches for the solution in a constrained space giving rise
to a stable multilevel covariance matrix.  Moreover, only one matrix
inversion (iterative approach) of the multilevel covariance matrix is
required in contrast to classical BLUP, including the Generalized
Least Squares (GLS) prediction, that requires $p+1$ matrix inversions
(iterative approach), where $p$ is the number of columns of the design
matrix.  Numerical results show speedups of up to 42,050 for solving
the BLUP problem to at least the same accuracy.

The multilevel Kriging method makes previously impractical missing
data problems feasible.  We are currently applying the Kriging
approach to missing data problems for medical data sets and have shown
up to 5-6 times improved accuracy over traditional state of the art
missing data packages.


In Section \ref{Introduction} the problem formulation is introduced.
In section \ref{MultilevelCovarianceMatrix} the construction of the
multilevel covariance matrix is discussed. In section
\ref{multilevelestimator} the multilevel estimator and predictor are
formulated and numerical computational issues are discussed in section
\ref{numericalcomputation}.  In section \ref{errorestimates} a
mathematical analysis of the decay of the entries of the multilevel
covariance matrix is developed. This section can also be skipped for
the less mathematically inclined reader.  In section
\ref{numericalresults} the multilevel Kriging method is tested on
numerically unstable problems of up to 25 dimensions.  In the Appendix
the Multivariate polynomial interpolation based on complex analytic
extensions is discussed. These results are used for to derive the
decay of the entries of the multilevel covariance matrix.  In section
\ref{multilevelapproach} it is shown how to construct the multilevel
basis based on kd-trees.


%% Problem Setup ------------------------------------------------------
%%
\section{Problem setup}
\label{Introduction}

Consider the following model for a Gaussian random field $Z$:
\begin{equation}
Z(\bx) = \bk(\bx)\T \bbeta+\varepsilon(\bx), \qquad \bx \in \R^d,
\label{Introduction:noisemodel}
\end{equation}
where $d$ is the number of spatial dimensions, $\bk:\R^d \rightarrow
\R^p$ is a functional vector of the spatial location $\bx$,
$\bbeta\in\R^p$ is an unknown vector of coefficients, and
$\varepsilon$ is a stationary mean zero Gaussian random field with
parametric covariance function
$C(\bx,\bx';\btheta)=\cov\{\varepsilon(\bx),\varepsilon(\bx')\}$ with
an unknown vector of positive parameters $\btheta\in\R^d$.

Suppose that we obtain $N$ observations and stack them in the data
vector $\bZ=(Z(\bx_1),\ldots,$ $Z(\bx_N))\T$ from locations $\bbS :=\{
\bx_{1},\dots,\bx_{N}\}$, where the elements in $\bbS$ are restricted
such that the design matrix defined below, $\bX$, has full column
rank.  Furthermore, without loss of generality all the locations in
$\bbS$ are contained in the unit hypercube $[-1,1]^{d}$.  Let
$\bC(\btheta)=\cov(\bZ,\bZ\T)\in \R^{N \times N}$ be the covariance
matrix of $\bZ$ and assume it is positive definite for all
$\btheta\in\R^w$.  Define $\bX=\big( \bk(\bx_1) \ldots$ $
\bk(\bx_N)\big)\T\in \R^{n\times p}$ and assume it is of full rank,
$p$. Since the model \eqref{Introduction:noisemodel} is a Gaussian
random field, then from the samples of $\bbS$ the following vectorial
model is obtained
\begin{equation}
{\bf Z} = \bX \bbeta +{\boldsymbol \varepsilon},
\label{Introduction:vectormodel}
\end{equation}
where $\boldsymbol \varepsilon$ is a Gaussian random vector,
${\boldsymbol \varepsilon} \sim \mcN(\0,\bC(\btheta))$. The aim
now is to:

\begin{enumerate}[i)]
\item {\it Estimate} the unknown vectors $\bbeta$ and $\btheta$;

\item {\it Predict} $Z(\bx_0)$, where $\bx_0$ is a new spatial
  location. These two tasks are particularly computationally
  challenging when the sample size $N$ and number of dimensions $d$
  are large.
\end{enumerate}

There is a very large literature on Gaussian process regression that
deal with this problem. Please see \cite{Castrillon2015} for a brief
literature review.  The unknown vectors $\bbeta$ and $\btheta$ are
estimated with the log-likelihood function
\begin{equation}
  \begin{split}
\ell(\bbeta,\btheta)&=-\frac{n}{2}\log(2\pi)-\frac{1}{2}\log
\det\{\bC(\btheta)\} \\ &
-\frac{1}{2}(\bZ-\bX\bbeta)\T\bC(\btheta)^{-1}
(\bZ-\bX\bbeta),
\end{split}
\label{Introduction:loglikelihood}
\end{equation}
which can be profiled by Generalized Least Squares (GLS) with
\begin{equation}
  \hat \bbeta(\btheta)=\{\bX\T \bC(\btheta)^{-1} \bX\}^{-1}\bX\T
  \bC(\btheta)^{-1}\bZ.
  \label{GLSbeta}
\end{equation}
In general this is not a good choice, since profiling with the Maximum
Likelihood Estimator (MLE) of $\btheta$ is prone to be biased
\cite{Castrillon2015}.

%A solution to this problem is to use restricted maximum likelihood
%(REML) estimation which consists in calculating the log-likelihood of
%$n-p$ linearly independent contrasts, that is, linear combinations of
%observations whose joint distribution does not depend on $\bbeta$,
%from the set $\bY=\{\bI_n-\bX(\bX\T\bX)^{-1}\bX\T\}\bZ$.


For the prediction part, consider the Best Linear Unbiased Predictor
(BLUP) $\hat Z(\bx_0)=\lambda_0+\blambda\T\bZ$ where
$\blambda=(\lambda_1,\ldots,\lambda_n)\T$. The unbiased constraint
implies $\lambda_0=0$ and $\bX\T\blambda=\bk(\bx_0)$.  The
minimization of the mean squared prediction error
E$[\{Z(\bx_0)-\blambda\T\bZ\}^2]$ under the constraint
$\bX\T\blambda=\bk(\bx_0)$ yields
\begin{equation}
\hat Z(\bx_0)=\bk(\bx_0)\T\hat \bbeta+\bc(\btheta)\T
\bC(\btheta)^{-1}(\bZ-\bX\hat \bbeta), \label{KrigBLUP}
\end{equation}
where $\bc(\btheta)=\cov\{\bZ,Z(\bx_0)\}\in \R^{n}$ and $\hat \bbeta$ is
defined in (\ref{GLSbeta}).  

Now, let $\alpha:= (\alpha_{1},\dots,\alpha_{d}) \in \mathbb{Z}{^d}$,
$|\alpha| := \alpha_{1}+\dots+\alpha_{d}$, $\bx : =
[x_1,\dots,x_d]$. For any $w \in \bbN_+$ (where $\mathbb{N}_+ :=
\mathbb{N} \cup \{0\}$) let $\mcQ^d_w$ be the set of Total Degree (TD)
monomials $\{x_1^{\alpha_1} \dots x_d^{\alpha_d}\,\,\,|\,\,\, |\alpha|
\leq w\}$. The typical choice for the matrix $\bX$ is to build it from
the monomials of $\mcQ^d_w$ with cardinality
$p(d,w):=\begin{pmatrix} d + w \\ w \end{pmatrix}$.

The challenge is that the covariance matrix $\bC(\btheta)$ in many
practical cases is ill-conditioned, leading to slow and inaccurate
estimates of $\btheta$. Following the approach in
\cite{Castrillon2015} the data vector $\bZ$ is transformed into
decoupled multilevel description of the model
\eqref{Introduction:noisemodel}.  This multilevel representation leads
to significant computational benefits, including numerical stability,
when computing the Kriging predictor $\hat Z(\bx_0)$ in
(\ref{KrigBLUP}) for large sample size $N$ and high dimensions $d$.
Note, that in this paper we shall refer to the \emph{single level}
approach to solving the Kriging problem by applying the estimation and
prediction steps directly to the data $\bZ$ and covariance matrix
$\bC(\btheta)$.





%% \section{Polynomial Interpolation}
%% \label{Polynomial}

%% \corb{In this section we give some background on polynomial
%%   interpolation in high dimensions. This will be critical to estimate
%%   the decay rates of the entries of the multilevel covariance matrix
%%   for high dimensional problems. Note that for the less mathematically
%%   inclided reader this section can be skipped as it is only used for
%%   estimating the decay of the multilevel covariance matrix.}

%% The decay of the coefficients will directly depend on the analytic
%% properties of the covariance function. The traditional error estimates
%% of polynomial interpolation are based on multi-variate $m^{th}$ order
%% derivatives. However, for many cases, such as the Mat\'{e}rn
%% covariance function, the derivatives are too complex or expensive to
%% manipulate for even a moderate number of dimensions. This motivates
%% the study of polynomial numerical approximations based on complex
%% analytic extensions, which are much better suited for high dimensions.
%% Much of the discussion that follows has it roots in the field of
%% uncertainty quantification and high dimensional interpolation
%% \cite{nobile2008a,Castrillon2016,Griebel2016}
%% for partial differential
%% equations.


%% Consider the problem of approximating a function $v: \Gamma^{d}
%% \rightarrow \R$ on the domain $\Gamma^{d}$.  Without loss of
%% generality let $\Gamma : = [-1, 1]$ and $\Gamma^{d} := \prod_{n =
%%   1}^{d} \Gamma$. Suppose that $\mcG \subset \Gamma^{d}$, then define
%% the following spaces
%% \[
%% \begin{split}
%%   &
%% L^q(\mcG) := \{ v(\by)\, | \, \int_{\mcG} v(\by)^q \text{d}
%% \by < \infty  \}
%% \,\,\,
%% \mbox{and} \\
%% &
%% L^{\infty}(\mcG) := \{ v(\by)\, | \, \sup_{\by \in \mcG} |v(\by)|
%% < \infty  \}.
%% \end{split}
%% \]


%% Suppose that $\mcP_{ q}(\Gamma):=\text{\rm span}\{y^k,\,k=0,\dots,q\}$
%% i.e. the space of polynomials of degree at most $q$. Let $\mcI^{m} :
%% C^{0}(\Gamma) \rightarrow \mcP_{m-1}(\Gamma)$ be the univariate
%% Lagrange interpolant
%% \[
%% \mcI_{m}(v(\by)):=
%% \sum_{k=1}^{m}v(y^{(k)})l_{m,k}(y^{(k)}),
%% \]
%% where $y^{(1)}, \dots, y^{(m)}$ is a set of distinct knots on $\Gamma$
%% and $\{ l_{n,k} \}_{k=0}^{m}$ is a Lagrange basis of the space
%% $\mcP_{m-1}(\Gamma)$. The variable $m \in \Nset$
%% %, where $\Nset_{+} := \Nset \cup 0$,
%% corresponds to the order of approximation of the
%% Lagrange interpolant. However, for the case of the zero order
%% interpolation $m = 0$ corresponds to $\mcI_{0} = 0$.


%% \begin{remark}
%% For high dimensional interpolation the particular set of points
%% $y^{(1)}, \dots, y^{(m)}$ that we will use is the Clenshaw-Curtis
%% abscissas.  This is further discussed in this section. However, for
%% now, we assume that the points are only distinct.
%%   \end{remark}


%% For $m \geq 1$ let
%% \[
%% \Delta_{m}
%% := \mcI_{m}-\mcI_{m-1},
%% \]
%% From the difference operator $\Delta_{m}$ we can readily observe that
%% $\mcI_{m} = \sum_{k=1}^{m} \Delta_{k}$, which is reminiscent of multi
%% resolution wavelet decompositions. The idea is to represent
%% multivariate approximation as a summation of the difference operators.

%% Consider the multi-index tupple $\bm = (m_1,\dots,m_d)$, where $\bm
%% \in \Nset^{d}$, and form the tensor product operator
%% $\mcS_{w,d}: \Gamma \rightarrow \R$ as
%% \begin{equation}
%%   \mcS_{w,d}
%%       [v(\by)]
%%       :
%%       =
%%  \sum_{\bm \in \bbNset^{d}: \sum_{i=1}^{d} m_i - 1  \leq w } \;\;
%%  \bigotimes_{n=1}^{d} {\Delta^{n}_{m_n}}(v(\by)).
%% \label{errorestimates:SG}
%% \end{equation}
%% Note that by ${\Delta^{n}_{m_n}}(v(\by))$ we mean that the difference
%% operator ${\Delta_{m_n}}$ is applied along the $n^{th}$ dimension in
%% $\Gamma$.


%% Let $C^{0}(\Gamma_d; \R) : = \{ v: \Gamma_d \rightarrow \R\,\,$ is
%% continuous on $\Gamma_d$ and $\max_{\by\in \Gamma_d} |v(\by)| < \infty
%% \}$.  From Proposition 1 in \cite{Back2011} it is shown that for any
%% $v \in C^0(\Gamma_d;\R)$, we have $\mcS_{w,d}[v]\in \mcQ^{d}_{w}$.
%% Moreover, $\mcS_{w,d}[v] = v$, for all $v \in \mcQ^{d}_{w}$. The key
%% observation to take away is that the operator $\mcS_{w,d}[v]$ is
%% \textit{exact} in the space of polynomials $\mcQ^{d}_{w}$. This will
%% be useful in connecting the Lagrange interpolant with Chebyshev
%% polynomials.


%% Let $T_k:\Gamma \rightarrow \R$, $k = 0, 1, \dots$, be a Chebyshev
%% polynomial over $\Gamma$, which are defined recursively as follows:
%% $T_0(y) = 1$, $T_1(y) = y$, $\dots$, $T_{k+1}(y) = 2yT_{k}(y) -
%% T_{k-1}(y)$, $\dots$, where $y \in \Gamma$. Chebyshev polynomials are
%% well suited for the approximation of functions with analytic
%% extensions on a complex region bounded by a Bernstein ellipse. They
%% bypassing the need of using derivative information and sharp bounds on
%% the error are readily available. Suppose that $\sigma > 0$ and denote
%% by
%% \[
%% \begin{split}
%%   \mcE_{\sigma} := \Big\{
%%   &z \in \bbC, \sigma \geq
%% \delta \geq 0 ;\,\Real{z} = \frac{e^{\delta} + e^{-\delta}
%% }{2}cos(\theta) \\
%% &\Imag{z} = \frac{e^{\delta} 
%%   - e^{-\delta}}{2}sin(\theta),
%% \theta \in [0,2\pi)
%%   \Big\}
%% \end{split}
%%   \]
%% as the region bounded by a Bernstein ellipse (see Figure
%% \ref{erroranalysis:sparsegrid:polyellipse}).

%% The following theorem is based on complex analytic extensions on
%% $\mcE_{\sigma}$ and provides a control for the Chebyshev polynomial
%% approximation.

%% \begin{theorem}
%% Suppose that for $u:\Gamma \rightarrow \R$ there exists an analytic
%% extension on $\mcE_{\sigma}$. If $|u| \leq M < \infty$ on
%% $\mcE_{\sigma}$ then there exists a sequence of coefficients
%% $|\alpha_k| \leq M / e^{k\sigma}$ such that $u \equiv \alpha_0 +
%% 2\sum_{k = 1}^{\infty} \alpha_{k} T_{k}$ on $\mcE_{\sigma}$. Moreover,
%% if $y \in \Gamma$ then
%% \[
%% %\begin{multline*}
%% %\shoveright{|q(y) - \alpha_0  - 2\sum_{k = 1}^{n} \alpha_{k} T_{k}(y)|
%% %\leq 
%% %\frac{2M}{e^{\sigma} - 1} e^{-n \sigma}.}
%% |q(y) - \alpha_0  - 2\sum_{k = 1}^{n} \alpha_{k} T_{k}(y)|
%% \leq 
%% \frac{2M}{e^{\sigma} - 1} e^{-n \sigma}.
%% %\end{multline*}
%% \]
%% \label{errorestimates:theorem}
%% \end{theorem}
%% \begin{proof}
%% See Theorem 2.25 in \cite{Khoromskij2018}
%% \end{proof}
%% \qed

%% \begin{figure}[htb]%[12]{r}{7cm}%[htp]
%% \begin{center}
%% \begin{tikzpicture}
%%     \begin{scope}[font=\scriptsize]

      
%%       \filldraw[fill=blue!20,
%%       semitransparent] (0,0) ellipse (2 and 1);

%%     \draw [->] (-2.5, 0) -- (2.5, 0) node [below left]  {$\Real $};
%%     \draw [->] (0,-1.5) -- (0,1.5) node [below left] {$\Imag$};
%%     \draw (1,-3pt) -- (1,3pt)   node [above] {$1$};
%%     \draw (-1,-3pt) -- (-1,3pt) node [above] {$-1$};
%%     \end{scope}
    
%%     \node [below right] at (-2.5,1.25) {$\mcE_{\sigma}$};

%%     \node [] at (0.75,1.25) {$\frac{e^{
%%           \sigma} - e^{- \sigma}}{2}$};

    
%%     \node [] at (2.75,0.25) {$\frac{e^{
%%       \sigma} + e^{- \sigma}}{2}$}; 
    
%% \end{tikzpicture}
%% \end{center}
%% \caption{Complex region bounded by the Bernstein ellipse.}
%% \label{erroranalysis:sparsegrid:polyellipse}
%% \end{figure}

%% We can now connect the error due to the Lagrange interpolation with
%% Chebyshev expansions. It is known that if $u \in C(\Gamma,\R)$ then
%% \[
%% \|(I - \mcI_{m})u\|_{L^{\infty}(\Gamma)} \leq
%% (1 + \Lambda_{m})
%% \min_{h \in \mcP_{m-1}} \| u - h \|_{L^{\infty}(\Gamma)},
%% \]
%% where $\Lambda_{m}$ is the Lebesgue constant (See Lemma 7 in
%% \cite{babusk_nobile_temp_10}). Note that $I:C^{d}(\Xi;\R) \rightarrow
%% C^{d}(\Xi;\R)$ refers to the identity operator and the domain $\Xi$ is
%% taken from context. For the previous case $\Xi = \Gamma$.  Bounds on
%% $\Lambda_{m}$ are known in the context of the location of the knots
%% $y^{(1)}, \dots, y^{(m)} \in \Gamma$. In this article we restrict our
%% attention to Clenshaw-Curtis abscissas
%% %\[
%% \[
%% y^{(j)} = -\cos \left( \frac{\pi(j-1)}{m - 1} \right),\,\, j =
%% 1,\dots, m
%% \]
%% %\]
%% and $\Lambda_m$ is bounded by $2\pi^{-1}(\log{(m-1)} + 1) \leq 2m - 1$
%% (see \cite{babusk_nobile_temp_10}).  Since the interpolation operator
%% $\mcI_{m}$ is exact on $\mcP_{m - 1}$, then if $u:\Gamma \rightarrow
%% \R$ has an analytic extension in $\mcE_{\sigma}$ we have from Theorem
%% \ref{errorestimates:theorem} (following a similar approach as in
%% \cite{babusk_nobile_temp_10}) that
%% \[
%% \begin{split}
%% \|(I - \mcI_{m})u\|_{L^{\infty}(\Gamma_n)}
%% &\leq
%% (1 + \Lambda_{m})
%% \frac{2M}{e^{\sigma} - 1} e^{-\sigma (m-1)} \\
%% &\leq 
%% 2 C(M,\sigma) m e^{-\sigma (m-1)},
%% \end{split}
%% \]
%% where $C(M,\sigma_n) := \frac{2M}{(e^{ \sigma} - 1)}$. We then
%% conclude that for all $k = 1,\dots, m$
%% \begin{equation}
%% \begin{split}
%% \| \Delta_{k}(u) \|_{L^{\infty}(\Gamma)} 
%% &=
%% \|
%% \mcI^{m}(u) - \mcI^{m-1}(u)
%% \|_{L^{\infty}(\Gamma)} \\
%% &\leq
%% \|(I - \mcI_{m})u\|_{L^{\infty}(\Gamma)} \\
%% &+
%% \|(I - \mcI_{m-1})u\|_{L^{\infty}(\Gamma)} \\
%% &\leq
%% e^{2\sigma}C(M,\sigma) m e^{-\sigma m}.
%% \end{split}
%% \label{interpolation:eqn1}
%% \end{equation}
%% Let $\mcE_{\sigma,n} \subset \bbC^{d}$ a complex region bounded by a
%% Bernstein ellipse such that the restriction on $\Gamma_{d}$ is along
%% the $n^{th}$ dimension and form the polyellipse $\mcE^{d}_{\sigma}:=
%% \prod_{n=1}^{d} \mcE_{\sigma,n}$.  Suppose that $v:\mcE^{d}_{\sigma}
%% \rightarrow \bbC$ is analytic on $\mcE^{d}_{\sigma}$ and let
%% $\tilde{M}(v) := \max_{\bz \in \mcE^{d}_{\sigma}} |v(\bz)|$.

%% Note we refer to $\mcI^{n}_{m}$ as the Lagrange operator of order $m$
%% along the $n^{th}$ dimension and similarly $\mcP^{n}_{m-1}$ is the
%% space of the span of univariate polynomials up to degree $m-1$ along
%% the $n^{th}$ dimension.  Form the tensor product $\bI^{d}_{m} :=
%% \mcI^{1}_{m} \times \dots \times \mcI^{d}_{m}$, thus $\bI:C(\Gamma,\R)
%% \rightarrow \bbP$ where $\bbP := \mcP^{1}_{m-1} \times \dots \times
%% \mcP^{d}_{m-1}$. From Theorem 2.27 in \cite{Khoromskij2018} we can
%% conclude that for a finite dimension $d$, as $m \rightarrow \infty$
%% then $\bI^{d}_{m}[v] \rightarrow v$.

%% Applying equation \eqref{interpolation:eqn1} to equation
%% \eqref{errorestimates:SG} we have that
%% \begin{equation}
%% \begin{split}
%% & \| (I - \mcS_{w,d})
%%  v(\by)
%%  \|_{L^{\infty}(\Gamma^{d})} \\
%%  &\leq
%%  \left\| \sum_{\bm \in \bbNset^{d}: \sum_{i=1}^{d} m_i - 1 > w } \;\;
%%  \bigotimes_{n=1}^{d} {\Delta^{n}_{m_n}}(v(\by))\right\|_{L^{\infty}(\Gamma^d)} \\
%%  &\leq
%%  \sum_{\bm \in \bbNset^{d}: \sum_{i=1}^{d} m_i - 1 > w } \;\;
%%  \bigotimes_{n=1}^{d} \|{\Delta^{n}_{m_n}}(v(\by))\|_{L^{\infty}(\Gamma^d)}  \\
%%  &\leq
%%  \sum_{\bm \in \bbNset^{d}: \sum_{i=1}^{d} m_i - 1 > w }
%%  e^{2d} C(M,\sigma)^{d} \\
%%  &
%%  \left( \prod_{n=1}^{d} m_n\right) \exp{\left( -\sum_{n=1}^{d}
%%    \sigma m_{n} \right)}.
%% % \\
%% %  &\leq
%% % \sum_{\bk \in \bbNset^{d}_{0}: \sum_{i=1}^{d} k_i > w }
%% % e^{2d} C(M,\sigma)^{d} \left( \prod_{n=1}^{d} (k_n + 1)\right)
%% % \exp{\left( -\sum_{n=1}^{d}
%% %   \sigma (k_{n}+1) \right)}.
%% \end{split}
%% \label{interpolation:eqn2}
%% \end{equation}

%% By applying Theorem 2.10 and Corollary 2.11 in \cite{Griebel2016} if
%% $ w \geq  d$ and $p( d, w) \geq
%% \left(\frac{2  d}{\kappa( d)}\right)^{ d}$, where
%% $\kappa( d) := \sqrt[\leftroot{-2}\uproot{2}  d]{
%%   d!} >  d/e$ (Sterling approximation), then for any $\hat
%% \sigma \in \R_{+}$
%% \begin{equation}
%% \begin{split}
%%  & \sum_{\bk \in \bbNset^{ d}_{0}: \sum_{i=1}^{ d} k_i  >  w }
%%  \exp{\left( -\sum_{n=1}^{ d} \hat \sigma
%%    k_{n} \right)} \\
%%  &\leq
%%  \sum_{\bk \in \bbNset^{d}_{0}: \hat \sigma \sum_{i=1}^{ d} k_i  \geq  w \hat \sigma  }
%%  \exp{\left( -\sum_{n=1}^{ d}
%%    \hat \sigma k_{n} \right)} \\
%%  &\leq
%%  \hat \sigma  d e
%%  \left( \frac{e^{\hat \sigma}}{1 - e^{-\hat \sigma}} \right)^{ d}
%%  \exp \left(-\frac{ d}{e} \hat \sigma  p^{\frac{1}{ d}}
%%  \right) p^{\frac{ d-1}{ d}}.
%% \end{split}
%% \label{interpolation:eqn3}
%% \end{equation}
%% where $\bk \in \bbNset^{d}_{0}$ and $\bk:=(k_1,\dots,k_d)$.






%% Following the same approach as in \cite{Griebel2016} observe that for
%% $0 < \delta < 1$ we can obtain a bounded constant $c_{n,\delta} \leq
%% (e\sigma \delta)^{-1}$ such that $m_n \exp(-\sigma m_n) \leq (e\sigma
%% \delta)^{-1}$ $\exp(-\sigma m_n (1 - \delta))$. Set $\hat \sigma :=
%% \sigma (1 - \delta)$ and by combining equations
%% \eqref{interpolation:eqn2} and \eqref{interpolation:eqn3} we have
%% proven the following result.

%% \begin{lemma} Suppose that $0< \delta < 1$, $\hat
%%   \sigma := \sigma (1 - \delta)$, and $p(d,w) \geq \left(\frac{2
%%     d}{\kappa(d)}\right)^{d}$ then
%%   \[
%%   \begin{split}
%%  &\| (I - \mcS_{w,d})
%%  v(\by)
%%  \|_{L^{\infty}(\Gamma^{d})}\\
%%  \leq &
%%  \frac{C(\tilde M,\sigma)^d e^{d - \sigma(1 - \delta) + 1} \hat \sigma d }
%%  {
%% (\sigma \delta)^{d}}
%%  \left( \frac{e^{\hat \sigma}}{1 - e^{-\hat \sigma}} \right)^{d} \\
%%  &
%%  \exp \left(-\frac{d}{e} \hat \sigma  p^{\frac{1}{d}}
%%  \right) p^{\frac{d-1}{d}}.
%%  \end{split}
%%  \]
%%  \label{interpolation:lemma1}
%% \end{lemma}


%% \begin{remark}
%% The restriction $p(d,w) \geq \left(\frac{2
%%   d}{\kappa(d)}\right)^{d}$ is not strict and can be relaxed such that
%% sub-exponential convergence is still obtained.  We refer the reader to
%% the bound of the Gamma function in Lemma 2.5 (\cite{Griebel2016}) and
%% it's application in the proofs of Theorem 2.10 and Corollary 2.11.
%% \label{interpolation:remark1}
%% \end{remark}



\section{Multilevel approach}
\label{multilevelapproach}

The general approach of this paper and multilevel basis construction
are now presented. We mostly follow the exposition laid out in
\cite{Castrillon2015}. The proof of Proposition
\ref{Multilevelapproach:theo1} is repeated, but clarified with
more details.

Let $\mcP^{p}(\bbS)$ be the span of the columns of the design matrix
$\bX$. Suppose that there exists the orthogonal projections $\bL :
\R^n \rightarrow \mcP^{p}(\bbS)$ and $\bW : \R^n \rightarrow
\mcP^{p}(\bbS)^{\perp}$, where $\mcP^{p}(\bbS)^{\perp}$ is the
orthogonal complement of $\mcP^{p}(\bbS)$.  The operator $\left[
\begin{array}{c}
\bW \\
\bL
\end{array}
\right ]$ is assumed to be orthonormal.


The first step is to filter out the effect of the trend by project the
observation onto the orthogonal subspace.  Let $\bZ_{\bW}: = \bW \bZ$,
thus from equation \eqref{Introduction:vectormodel} it follows that
$\bZ_{\bW} = {\bf W} ({\bX \bbeta}+ {\boldsymbol \varepsilon}) = {\bf
  W{\boldsymbol \varepsilon}}$. Notice that the trend component ${\bX}
\bbeta$ is removed from the data ${\bf Z}$. The new log-likelihood
function for $\bZ_{\bW}$ becomes
\begin{equation}
  \begin{split}
\ell_{\bW}(\btheta)
&=-\frac{n}{2}\log(2\pi)-\frac{1}{2}\log
\det\{\bC_{\bW}(\btheta)\} 
-\frac{1}{2}\bZ_{\bW}\T\bC_{\bW}(\btheta)^{-1}\bZ_{\bW},
\end{split}
\label{Introduction:multilevelloglikelihood}
\end{equation}
where $\bC_{\bW}(\btheta) := \bW \bC(\btheta) \bW \T$ and
$\bZ_{\bW}\sim \mcN_{N-p}(\0,$ $\bC_{\bW}(\btheta))$.  A consequence
of the filtering is that we obtain an unbiased estimator
\cite{Castrillon2015}.

The decoupling of the likelihood function is not the only advantage of
using $\bC_{\bW}(\btheta)$. The following theorem also shows that
$\bC_{\bW}(\btheta)$ is more numerically stable than $\bC(\btheta)$.

\begin{prop} 
\label{Multilevelapproach:theo1}
Let $\kappa(A) \rightarrow \R$ be the condition number of the matrix
$A \in \R^{N \times N}$ then
\[
\kappa(\bC_{\bW}(\btheta)) \leq 
\kappa(\bC(\btheta)).
\]
\end{prop}
\noindent 
\begin{proof}
To see this let $\bv := \bW\T \bw$ for all $\bw \in \R^{N-p}$, which
implies that $\bv \in \mathbb{R}^{n} \backslash
\mcP^{p}(\bbS)$. Moreover, this map is bijective.  Now, $\bv \T
\bC(\btheta) \bv = \bw \T \bC_{\bW}(\btheta) \bw$ for all $\bw \in
\R^{N-p}$. From the orthonormal property we have that for all $\bv \in
\mathbb{R}^{n} \backslash \mcP^{p}(\bbS)$
\[
\begin{split}
\min_{\bv \in \mathbb{R}^{n} \backslash \mcP^{p}(\bbS)} \frac{\bv\T\bC(\btheta)\bv}
{\|\bv\|^2} 
= \min_{\bw \in \mathbb{R}^{N-p} } \frac{\bw\T\bC_{\bW}(\btheta)\bw}{\|\bw\|^2} 
\,\,\,
\mbox{and}
\,\,\,
\max_{\bv \in \mathbb{R}^{n} \backslash \mcP^{p}(\bbS) } \frac{\bv\T\bC(\btheta)\bv}
{\|\bv\|^2}  
= \max_{\bw \in \mathbb{R}^{N-p}} \frac{\bw\T\bC_{\bW}(\btheta)\bw}{\|\bw\|^2}.
\end{split}
\]
Now, it is not hard to see that
\[
\begin{split}
  0
  &<
  \min_{\bv \in \mathbb{R}^{n}} \frac{\bv\T\bC(\btheta)\bv}{ \|\bv\|^{2} } 
\leq 
\min_{\bv \in \mathbb{R}^{n} \backslash \mcP^{p}(\bbS)}
 \frac{\bv\T\bC(\btheta)\bv}{ \|\bv\|^{2} }
\leq \max_{\bv \in \mathbb{R}^{n} \backslash \mcP^{p}(\bbS)} \frac{\bv\T\bC(\btheta)\bv}
{ \|\bv\|^{2} } 
\leq
\max_{\bv \in \mathbb{R}^{n}} \frac{\bv\T\bC(\btheta)\bv}{ \|\bv\|^{2} }.
\end{split}
\]
The result follows from the positive definite property of
$\bC(\btheta)$.
\end{proof}


Proposition \ref{Multilevelapproach:theo1} states that the condition
number of $\bC_{\bW}(\btheta)$ is less or equal to the condition
number of $\bC(\btheta)$. Thus computing the inverse of
$\bC_{\bW}(\btheta)$ (using a direct or iterative method) will
generally be more stable.

In practice, computing the inverse of $\bC_{\bW}(\btheta)$ can be
significantly more stable than $\bC(\btheta)$ depending on the choice
of $\mcQ^d_w$. This has many significant implications as it will now
be possible to solve numerically unstable problems. Furthermore, the
following useful result can be proven.

\begin{corollary}
  \label{Multilevelapproach:cor1}
  Let $[\bC_{\bW}(\btheta)]^q$ be the multilevel
  covariance matrix built from a TD basis with cardinality $q \in
  \bbN$.  Suppose that $p \leq q$, then
  \[
  \kappa([\bC_{\bW}(\btheta)]^p)
  \leq \kappa([\bC_{\bW}(\btheta)]^q).
  \]
\end{corollary}
\begin{proof}
  This follows from the fact that $\mcP^{q}(\bbS) \subset
  \mcP^{p}(\bbS)$ and by applying a similar argument as the proof of
  Proposition \ref{Multilevelapproach:theo1}.
\end{proof}


There are other advantages to the structure of the matrix
$\bC_{\bW}(\btheta)$.  In section \ref{errorestimates} we show that
for a good choice of the $\mcP(\bbS)$ the entries of
$\bC_{\bW}(\btheta)$ decay rapidly, and most of the entries can be
safely eliminated. A level dependent criterion approach is shown in
Section \ref{MultilevelCovarianceMatrix} that indicates which entries
are computed and which ones are not. With this approach a sparse
covariance matrix $\tilde{\bC}_{\bW}$ can be constructed such that it
is close to $\bC_{\bW}$ in a matrix norm sense, even if the
observations are highly correlated with distance.
%From the decay estimates of
%Section \ref{MultilevelCovarianceMatrix} 





\subsection{Binary multilevel basis}
\label{MultilevelREML}

In this section the construction of Multilevel Basis (MB) is shown.
The approach followed in this section is a based on the MB
construction in \cite{Castrillon2013}. The MB can then be used to: (i)
form the multilevel likelihood
\eqref{Introduction:multilevelloglikelihood}; (ii) sparsify the
covariance matrix $\bC_{\bW}(\btheta)$; and (iii) improve the numerical
stability of the covariance matrix $\bC(\btheta)$ in it's multilevel
form. But first, let us establish notations and definitions:
\begin{enumerate}

% \item Given $\mcQ^d_w$ and the locations $\bbS$
%  construct the design matrix $\bX$. Furthermore, form a second set of
%  monomials $\mctQ^{a}_{\Lambda^{m,g}(w)} : =
%  \mcQ_{\Lambda^{m,g}(w+a)} $ for $a = 0,1,\dots,$ i.e.
%  $\mctQ_{\Lambda^{m,g}(w)} \subset
%  \mctQ^{a}_{\Lambda^{m,g}(w)}$. Denote the accuracy parameter $\tilde
%  p \in \bbN$ as the cardinality of
%  $\mctQ^{a}_{\Lambda^{m,g}(w)}$. From the set of monomials
%  $\mctQ_{\Lambda^{m,g}(w)}$, for some user given parameter $a \in
%  \bbN_0$, and the set of observations $\bbS$ generate the design
%  matrix $\tilde \bX^{a}$. Denote also the space $\mcP^{\tilde
%    p}(\bbS)$ as the span of the columns of $\tilde \bX^{a}$.

\item For any index $i,j \in \mathbb{N}_{0}$, $1 \leq i \leq N$, $1
  \leq j \leq N$, let $\bve_{i}[j] = \delta[i-j]$, where
  $\delta[\cdot]$ is the discrete Kronecker delta function.

\item Let $\phi(\bx,\by;\btheta):\R^{d} \times \R^{d} \rightarrow \R$
  be the covariance function and assumed to be a positive definite.
  Let $\bC(\btheta)$ be the covariance matrix that is formed from all
  the interactions between the observation locations $\bbS$
  i.e. $\bC(\btheta) := \{ \phi(\bx_i,\by_j) \}$, where $i,j, =
  1,\dots,N$.  Alternatively we refer to $\phi(r; \btheta)$ as the
  covariance function where $r:\R^d \times \R^d \rightarrow \R$ is a
  function of $\bx$, $\by$ and $\btheta$.
\end{enumerate}

\begin{definition} The Mat\'{e}rn covariance function:
\[
\phi(r;\btheta)=\frac{1}{\Gamma(\nu)2^{\nu-1}} \left(
\sqrt{2\nu}\frac{r}{\rho} \right)^{\nu} K_{\nu} \left(
\sqrt{2\nu}\frac{r}{\rho} \right),
\]
where with a slight abuse of notation $\Gamma$ is the gamma function,
$r \in \R_{+}$, $0 < \nu$, $0 < \rho < \infty$, and $K_{\nu}$ is the
modified Bessel function of the second kind. It is understood from
context when $\Gamma$ is the gamma function.
\end{definition}

\begin{remark} The Mat\'{e}rn covariance function is a good choice for
the random field model. The parameter $\rho$ controls the length
correlation and the parameter $\nu$ changes the shape. For example, if
$\nu = 1/2 + n$, where $n \in \bbN_{+}$, then (see
\cite{abramowitz1964})
\[
\begin{split}
\phi(r;\rho) &= \exp   \bigg(-\frac{\sqrt{2\nu}r}{\rho} \bigg)
\frac{\Gamma(n + 1)}{\Gamma(2n + 1)} 
\sum_{k = 1}^{n} \frac{(n+1)!}{k!(n-k)!}
\bigg(
\frac{ \sqrt{8v} r }{ \rho } 
\bigg)^{n-k}
\end{split}
\]
and $\nu \rightarrow \infty \Rightarrow \phi(r;\btheta) \rightarrow
\exp \bigg(-\frac{r^2}{2\rho^2} \bigg)$. Note that even for a moderate
number of derivatives the number of terms will grow exponentially fast
leading to a very complex expression. This motivates the study of
complex analytical extensions of the covariance function. See Section
\ref{errorestimates} for more details.
%bbb

\label{multilevelapproach:remark1}
\end{remark}
The first step is to decompose the domain $\Gamma^{d}$ into a
multilevel domain decomposition. A good choice is based on the a
kD-tree decomposition of the space $\R^{d}$ \cite{Dasgupta2008}.
Other choices include
Projection (RP) tree \cite{Dasgupta2008}.
%This is a good choice for lower dimensions, however, as the number of
%dimensions $d$ becomes larger a better approach is to use a Random
%Projection (RP) tree \cite{Dasgupta2008}.
First start with the root node and cell $B^{0}_{0}$ at level $0$ that
contains all the observation nodes in $\bbS$. Now, split these nodes
into two children cells $B^{1}_{1}$ and $B^{1}_{2}$ at level $1$
according to the following rule:
\begin{enumerate}

\item Choose a unit vector $v$ in $\R^{d}$ along the axis of
  $\R^{d}$. This choice is the direction that leads to the maximum
  variance of the data in the cell along the direction of $v$.

\item Project all the nodes $\bx \in \bbS$ in the cell onto the unit
  vector $v$.

\item Split the cell with respect to the median
of the projections.

\end{enumerate}

For each cell $B^{1}_{1}$ and $B^{1}_{2}$ repeat the procedure until
there is at most $p$ nodes at the leaf nodes. Thus a binary tree is
obtained, which is of the form $B^{0}_{0}$, $B^{1}_{1}$, $B^{1}_{2}$,
$B^{2}_{3}$, $B^{2}_{4}$, $B^{2}_{5}$, $B^{2}_{6}$, $\dots $, where
$t$ is the maximal depth (level) of the tree.  Now, let $\mcB$ be the
set of all the cells in the tree and $\mcB^{n}$ be the set of all the
cells at level $0 \leq n \leq t$.  In addition, for each cell a unique
node number, current tree depth, threshold level and projection vector
are also assigned. This will be useful for searching the tree.
Algorithms \ref{RPMLB:algorithm1} and \ref{RPMLB:algorithm2-kd}
describe in more detail the construction of the kD-tree MB.
 
%\begin{remark}
%A kD-tree can also be constructed with Algorithms
%\ref{RPMLB:algorithm1} and \ref{RPMLB:algorithm2-kd}.
%\end{remark}

\begin{algorithm}[h]
  \KwIn{ $\bbS$, node, currentdepth, $n_0$} \KwOut{Tree, node}

\Begin{

\eIf {Tree = root}{node $\leftarrow$ 0, currentdepth $\leftarrow$ 0
Tree $\leftarrow$ MakeTree($\bbS$, node,
currentdepth + 1, $n_0$)
}
{

Tree.node = node

Tree.currentdepth = currentdepth - 1

node $\leftarrow$ node + 1

\If {$|\bbS| < n_0$}{return (Leaf)}





(Rule, threshold, $v$) $\leftarrow$ ChooseRule($\bbS$)

(Tree.LeftTree, node) 
$\leftarrow$ MakeTree($\bx \in \bbS$: Rule($\bx$) = True, node,
currentdepth + 1, $n_0$)

(Tree.RightTree, node)
$\leftarrow$ MakeTree($\bx \in \bbS$: Rule($\bx$) = false,  node, currentdepth + 1, $n_0$)

Tree.threshold = threshold\\
Tree.$v$ = $v$
}
}
\caption{MakeTree($\bbS$) function}
\label{RPMLB:algorithm1}
\end{algorithm}


%\begin{algorithm}[h]
%  \KwIn{ $\bbS$}
%  \KwOut{Rule, threshold, v}
%\Begin{
%choose a random unit vector $v$ \\
%Rule(x) := $x \cdot v  \leq$ threshold = median 
%$\{z \cdot v : z \in \bbS \}$
%}
%\caption{ChooseRule($\bbS$) function for RP tree}
%\label{RPMLB:algorithm2}
%\end{algorithm}

\begin{algorithm}[h]
  \KwIn{ $\bbS$}
  \KwOut{Rule, threshold, $v$}
\Begin{
    choose a coordinate direction that has maximal variance of the projection
    of the points in $\bbS$. \\
Rule(x) := $x \cdot v  \leq$ threshold = median
}

\caption{ChooseRule($\bbS$) function for kD-tree}
\label{RPMLB:algorithm2-kd}
\end{algorithm}


Now, suppose there is a one-to-one mapping between the set of unit
vectors $\mcE:=\{\bve_{1},\dots,\bve_{N}\}$, which is denoted as
leaf unit vectors, and the set of locations $\{
\bx_{1},\dots,\bx_{N}\}$, i.e. $\bx_{n} \longleftrightarrow \bve_{n}$
for all $n = 1, \dots, N$. It is clear that the span of the vectors
$\{\bve_{1},\dots,\bve_{N}\}$ is $\bbR^{N}$.  The next step is to
construct a new basis of $\R^{n}$ that is multilevel and orthonormal.

\setlength{\tabcolsep}{16pt}
\begin{figure*}
\begin{center}
  \begin{tabular}{c c}
\begin{tikzpicture}[scale=.65] 
  \begin{scope} 
 [place/.style={circle,draw=blue!50,fill=blue!20,thick,
     inner sep=0pt,minimum size=1.5mm}]

 \draw[step=8,gray,very thin] (0, 0) grid (8, 8);
    \draw (4,0) to (4,8);
    
    \draw (0,5) to (4,5);
    \draw (2.2,5) to (2.2,8);
    \draw (0,2) to (4,2);

    \draw (0,5) to (4,5);

    \draw (4,4.15) to (8,4.15);
    \draw (6.75,4.15) to (6.75,8);
    \draw (6,0) to (6,4.15);


    
  
    \node at (0.5,7.5) [place] {};
    \node at (0.3,6.3) [place] {};
    

    \node at (2.5,5.5) [place] {};
    \node at (3.2,5.2) [place] {};

    \node at (5,6) [place] {};
    \node at (3.8,5.5) [place] {}; %
    \node at (3.8,6) [place] {};   %


    \node at (0.5,3.5) [place] {};
    \node at (1.5,2.5) [place] {};
    \node at (2.3,2.2) [place] {};
    
    \node at (1.3,0.3) [place] {};
    \node at (2.7,0.5) [place] {};
    \node at (2.2,1.4) [place] {};
    \node at (2.6,1.4) [place] {};
    \node at (4.2,3.5) [place] {}; %
    \node at (3.7,3.3) [place] {};


    \node at (6.5,4.3) [place] {}; %
    \node at (7.5,5) [place] {};


    \node at (4.3,2.3) [place] {};
    \node at (5.7,3.5) [place] {};
    \node at (6.2,3.4) [place] {};
    \node at (7.3,2.4) [place] {};


    \node at (7,7) [place] {};
    \node at (6,7.5) [place] {};
    \node at (7.5,7.5) [place] {};


    \node at (6.5,2.0) [place] {};
    \node at (0.5,7.0) [place] {};
    \node at (2.0,7.0) [place] {};
    \node at (5,3.75) [place] {}; %
    \node at (6,7.0) [place] {};
    \node at (7,2.0) [place] {}; %
    %\node at (2.0,4.5) [place] {};
    \node at (7.5,4.3) [place] {}; %

    \node at (7.5,8.5) [] {$B^{0}_0$};
    \node at (0.6,5.5) [] {$B^{3}_{7}$};
    \node at (3,7) [] {$B^{3}_{8}$};
  \end{scope}
\end{tikzpicture} 
&
\begin{tikzpicture}[scale=0.85]
    %\node[anchor=center] at (0, -4.5) {$$};
    %\node[anchor=center] at (0,   10) {$$};
\begin{scope}[xshift=5cm, yshift=4cm,
place/.style={circle,draw=blue!50,fill=blue!20,thick,
      inner sep=0pt,minimum size=1.5mm},
placer/.style={circle,draw=blue!50,
  preaction={fill=darkgreen!60,fill opacity=0.5}, thick,inner
  sep=0pt,minimum size=1.5mm}, ]

  %placer/.style={circle,draw=blue!50,
  %preaction={fill=darkgreen!60,fill opacity=0.5}, thick,inner
  %sep=0pt,minimum size=1.5mm}, ]


%\filldraw[fill={rgb:red,143;green,188;blue,143},semitransparent, 
%      thick] (0, 0) rectangle (16, 16);


  
\Tree [.\node[placer]{$B^{0}_{0}$}; 
             [.\node[placer]{$B^{1}_{1}$};
                    [.\node[placer]{$B^{2}_{3}$}; 
                           [.\node[placer]{$B^{3}_{7}$};]
                           [.\node[placer]{$B^{3}_{8}$};] 
                    ]       
                    [.\node[placer]{$B^{2}_{4}$}; 
                           [.\node[placer]{$B^{3}_{9}$};] 
                           [.\node[placer]{$B^{3}_{10}$};] 
                    ] 
             ]                                        
             [.\node[placer]{$B^{1}_{2}$};
                    [      [.\node[placer]{$B^{2}_{5}$};
                                  [.\node[placer]{$B^{3}_{11}$};] 
                                  [.\node[placer]{$B^{3}_{12}$};] 
                           ]
                           [.\node[placer]{$B^{2}_{6}$}; 
                                  [.\node[placer]{$B^{3}_{13}$};] 
                                  [.\node[placer]{$B^{3}_{14}$};] 
                           ] 
                                          ]]
] 

\end{scope}
\end{tikzpicture}
\end{tabular}
\end{center}
\caption{Multilevel domain decomposition of the observations.}
\label{MLRLE:fig1}
\end{figure*}



\begin{enumerate}[(a)]
\item Start at the maximum level of the random projection tree,
  i.e. $q = t$.
\item For each leaf cell $B^{q}_{k} \in \mcB^{q}$ assume without loss
  of generality that there are $s$ observations nodes $\bbS^{q}_{k}:=\{
  \bx_1, \dots, \bx_s \}$ with associated vectors $C_k^{q} := \{
  \bve_1, \dots, \bve_s \}$.
  %Let $\mcE^q_k := \{\bx_1,\dots,\bx_s\}$
  %and
  Denote $\mcC^{q}_{k}$ as the span of the vectors in $C_k^{q}$.
\begin{enumerate}[i)]

\item Let $\bphi^{q,k} _{j} := \sum_{\bve_i \in C^q_k} c^{q,k} _{i,j}
  \bve_i, \hspace{2mm} j=1, \dots, a;
\hspace{2mm} \bpsi^{q,k}_{j} := \sum_{\bve_i \in C^q_k} d^{q,k}_{i,j}
\bve_i, \hspace{2mm} j=a+1, \dots, s$, where $c^{q,k}_{i,j}$,
$d^{q,k}_{i,j} \in \mathbb{R}$ and for some $a \in \mathbb{N}^{+}$. Note
that $a$ is unknown up to this point, but will be computed from the
data.  It is desired that the new discrete MB vector $\bpsi^{q,k}_{j}$
be orthogonal to $\mcP^{p}(\mathbb{S})$, i.e., for all $g \in \mcP^{
  p}(\mathbb{S})$:
\begin{equation}
\sum_{i=1}^{n} g[i] \bpsi^{q,k}_{j}[i] = 0
\label{hbconstruction:eqn1}
\end{equation}

\item Form the matrix $\mcM^{q,k} := \bX \T \bV^{q,k}$, where
  $\mcM^{q,k} \in \R^{p \times s}$, $\bV^{q,k} \in \R^{N \times s}$,
  and $\bV^{q,k}: = [\bve_1, \dots, \bve_i, \dots,\bve_s ]$ for all $\bve_i
  \in C_k^q$. Now, suppose that the matrix $\mcM^{q,k} $ has rank $a$
  and then perform the Singular Value Decomposition (SVD). Denote by
  $\bU \bD \bV $ the SVD of $\mcM^{q,k} $, where $\bU \in \R^{ p \times
    p}$, $\bD \in \R^{p \times s}$, and $\bV \in \R^{s \times s} $.

  \begin{remark} Note that in practice we only keep track of the
    non-zero elements of the vectors $\bve_1, \dots, \bve_s$. Thus the
    computational cost is reduced significantly. This is taken into
    account in the complexity analysis in Lemma
    \ref{MultilevelREML:lemma1} and \ref{MultilevelREML:lemma2}
  \end{remark}
  
\item Following the same argument as in \cite{Castrillon2015} but
  adapted to the kd-tree decomposition equation
  \eqref{hbconstruction:eqn1} is satisfied with the following choice
\[
  \left[ \begin{array}{ccc|ccc}
      c^{q,k}_{0,1} & \dots &c^{q,k}_{a,1} & d^{q,k}_{a+1,1} & \dots &d^{q,k}_{s,1} \\
      c^{q,k}_{0,2} & \dots &c^{q,k}_{a,2} & d^{q,k}_{a+1,2} & \dots &d^{q,k}_{s,2} \\
      \vdots & \vdots & \vdots & \vdots & \vdots & \vdots   \\
      c^{q,k}_{0,s} & \dots &c^{q,k}_{a,s} & d^{q,k}_{a+1,s} & \dots &d^{q,k}_{s,s}
    \end{array}
\right] := \bV\T.
% \label{eqDefVspT*}
  \]
\noindent For this choice the coefficient $a$ is equal to the number
of non-zero singular values. Thus the columns $a+1$, \dots, $s$ form
an orthonormal basis of the nullspace ${N_0}(\mcM^{q,k} )$. Similarly,
the columns $1,\dots, a$ form an orthonormal basis of $\R^s \backslash
{N_0}(\mcM^{q,k})$. Since the vectors in $C^q_k$ are orthonormal then
$\bphi^{q,k}_{1}, \dots, \bphi^{q,k}_a$, $\bpsi^{q,k}_{a+1}, \dots,$
$\bpsi^{q,k}_s$ form an orthonormal basis of $\mcC^{q}_{k}$.  Moreover
$\bpsi^{q,k}_{a+1}, \dots, \bpsi^{q,k}_s$ satisfy equation
\eqref{hbconstruction:eqn1}, i.e., are orthogonal to
$\mcP^{p}(\mathbb{S})$ and are locally adapted to the locations
contained in the cell $B^{q}_{k}$.

\item Denote by $D_k^{q,k}$ the collection of all the vectors
  $\bpsi^{q,k}_{a+1}, \dots, \bpsi^{q,k}_s$. Notice that the vectors
  $\bphi^{q,k}_{1}, \dots,$ $\bphi^{q,k}_a$, which are denoted with a
  slight abuse of notation as the scaling vectors, are {\it not}
  orthogonal to $\mcP^{p}(\mathbb{S})$. They need to be further
  processed.

\item Let $\mcD^{q}$ be the union of the vectors in $D^{q}_k$ for
  all the cells $B^{q}_k \in \mcB^{q}_{k}$. Denote by
  $W_{q}(\mathbb{S})$ as the span of all the vectors in $\mcD^{q}$.

\end{enumerate}



\item The next step is to go to level $q - 1$. For any two sibling
  cells denote $B^{q}_{\tt{left}}$ and $B^{q}_{\tt{right}}$ at level $q$ denote
  $C^{q-1}_{\tilde k}$ as the collection of the scaling functions from
  both cells, for some index $\tilde k$.


\item Let $q: = q - 1$. If $B^{q}_{k} \in \mcB^{q}$ is a leaf cell
  then repeat steps (b) to (d). However, if $B^{q}_{k} \in \mcB^{q}$
    is not a leaf cell, then repeat steps (b) to (d), but replace the
    leaf unit vectors with the scaling vectors contained in $C^{q}_k$
    with $C^{q-1}_{\tilde k}$.


  \item When $q = -1$ is reached stop.
  %repeat steps (b) to (d), but replace
  %$p$ with $p$, e.g. $\mcP^{p}(\bbS)$ with
  %$\mcP^{p}(\bbS)$. The ML basis vectors will span the space
  %$W_{-1}(\bbS) : =\mcP^{p}(\bbS) \backslash \mcP^{p}(\bbS)$.


\end{enumerate}

When the algorithm stops a series orthogonal subspaces
$V_{0}(\bbS), W_{0}(\mathbb{S}),\dots, W_{t}(\mathbb{S})$ (and their corresponding
basis vectors) are obtained. These subspaces are orthogonal to
$V_{0}(\mathbb{S}) : = span \{ \phi_{1}^{0}, \dots, \phi_{p}^{0}
\}$. Note that the orthonormal basis vectors of $V_{0}(\mathbb{S})$
also span the space $\mcP^{p}(\mathbb{S})$.
\begin{remark}
Following Lemma 2 in \cite{Castrillon2013} it can be shown that
\[
\R^{N} = \mcP^{p}(\mathbb{S}) \oplus
%W_{-1}(\mathbb{S}) \oplus
W_{0}(\mathbb{S}) 
\oplus W_{1}(\mathbb{S})
\oplus \dots \oplus W_{t}(\mathbb{S}),
\]
%where $W_{-1}(\bbS) : =\mcP^{\tilde p}(\bbS) \backslash
%\mcP^{p}(\bbS)$.
Also, it can then be shown that at most $\mcO(Nt)$
computational steps are needed to construct the multilevel basis of
$\R^{N}$.
\end{remark}

From the basis vectors of the subspaces $\mcP^{p}(\mathbb{S})^{\perp}
= \cup_{i=0}^{t} W_{i}(\mathbb{S})$ an orthogonal projection matrix
$\bW:\R^{N} \rightarrow (\mcP^{p}(\mathbb{S}))^{\perp}$ can be built.
The dimensions of $\bW$ is $(N - p) \times N$ since the total number
of orthonormal vectors that span $\mcP^{p}(\mathbb{S})$ is
$p$. Conversely, the total number of orthonormal vectors that span
$\mcP^{p}(\mathbb{S})^{\perp}$ is $N-p$.

Let $\bL$ be a matrix where each row is an orthonormal basis vector of
$\mcP^{p}(\mathbb{S})$. For $i = 0,\dots,t$ let $\bW_i$ be a matrix
where each row is a basis vector of the space $W_i(\mathbb{S})$. The
matrix $\bW \in \mathbb{R}^{(N - p) \times N}$ can now be formed,
where $\bW := \left[ \bW_t\T, \dots, \bW_0\T \right] \T$.

Following a similar approach to Lemma 2.11 in \cite{Castrillon2013} it
can be shown that:
\begin{enumerate}[a)]
\item 
The matrix $\bP := \left[
\begin{array}{c}
\bW \\
\bL
\end{array}
\right ]$ is orthonormal, i.e., $\bP\bP\T= \bI$.

\item Any vector $\bv
\in \R^{n}$ can be written as $\bv = \bL\T\bv_{L} + \bW\T\bv_{\bW}$
where $\bv_{L} \in \R^{p} $ and $\bv_{\bW} \in \R^{N-p}$ are unique.

\end{enumerate}





The following useful lemmas are proved:
\begin{lemma} Assuming that $n_0 < 2p$,
for any level $q=0,\dots,t$ there is at most $p2^{q}$ multilevel
basis vectors.
%For level $q = -1$ there is at most $p -
%\tilde p$ multilevel vectors.
\label{MultilevelREML:lemma1}
\end{lemma}
\begin{proof}
Starting at the finest level $t$, for each cell $B^{t}_k \in \mcB^{t}$
there is at most $p$ multilevel vectors.  Since there is at most
$2^t$ cells then there is at most $2^{t} p$ multilevel vectors.

Now, for each pair of left and right (siblings) cells at level $t$ the
parent cell at level $t-1$ will have at most $2 p$ scaling
functions. Thus at most $p$ multilevel vectors and $p$ scaling
vectors are obtained that are to be used for the next level. Now, the
rest of the cells at level $t$ are leafs and will have at most $p$
multilevel vectors and $p$ scaling vectors that are to be used for
the next level. Since there is at most $2^{t-1}$ cells at level $t-1$,
there is at most $2^{t-1} p$ multilevel vectors. Now, follow an
inductive argument until $q = 0$ and the proof is done.
\end{proof}



\begin{lemma} Assuming that $n_0 < 2p$ for any level $q = 0, \dots, t$
  any multilevel vector $\bpsi^{q}_m$ associated with a cell $B^{q}_k
  \in \mcB^{q}$ has at most $2^{t-q+1} p$ non zero entries.
\label{MultilevelREML:lemma2}
\end{lemma}
\begin{proof} For any leaf cell at the bottom of the tree (level $t$)
  there is at most $2 p$ observations.
  %thus the number of non zero entries of level $q$ multilevel
  %vectors is $2 p$.  Combining the left and right cells, the parent
  cell has at most $4 p$ observations, thus the associated multilevel
  vectors has $4p$ non zero entries. By induction at any level $l$ the
  number of nonzero entries is at most $2^{t-q+1} p$.  Now for any
  leaf cell at any other level $l < t$ the number of nonzero entries
  is at most $2 p$. Following an inductive argument the result is
  obtained.
\end{proof}


From Lemma \ref{MultilevelREML:lemma1} and \ref{MultilevelREML:lemma2}
it can be shown that the matrix $\bW$ contains at most $\mcO(Nt)$
non-zero entries and $\bL$ contains at most $\mcO(Np)$ non-zero
entries. Thus for any vector $\bv \in \R^{n}$ the matrix vector
products $\bW \bv$ and $\bL \bv$ are respectively calculated with at
most $\mcO(Nt)$ and $\mcO(Np)$ computational steps.


%% Multilevel Covariance Matrix --------------------------------------
%%
\section{Multilevel covariance matrix}
\label{MultilevelCovarianceMatrix}

The multilevel covariance matrix $\bC_{\bW}(\btheta)$ and sparse
version $\tilde \bC_{\bW}(\btheta)$ can be now constructed.  Recall from
the discussion in Section \ref{multilevelapproach} that
$\bC_{\bW}(\btheta):=\bW \bC(\btheta) \bW \T$. From the multilevel
basis construct in Section \ref{MultilevelREML} the following
operator is built: $\bW := \left[ \bW_t\T, \dots, \bW_0\T
  \right] \T$. Thus the covariance matrix $\bC(\btheta)$ is
transformed into $\bC_{\bW}(\btheta)$, where each of the blocks
$\bC^{i,j}_{\bW}(\btheta) = \bW_i \bC(\btheta) \bW_j \T$ are formed from
all the interactions of the MB vectors between levels $i$ and $j$, for
all $i,j = 0, \dots, t$. The structure of $\bC_{\bW}(\btheta)$ is shown
in Figure \ref{multilevelcov:fig1}.  Thus for any
$\bpsi^{i}_{\tilde{l}}$ and $\bpsi^{j}_{\tilde{k}}$ vectors there is a
unique entry of $\bC^{i,j}_{\bW}$ of the form
$(\bpsi^{i}_{\tilde{k}})\T \bC(\btheta) \bpsi^{j}_{\tilde{l}}$.

%The blocks $\bC_{\bW}^{i,j}$, where
%$i=-1$ or $j=-1$, correspond to the case where the accuracy term
%$\tilde{p} > p$.


In Section \ref{errorestimates} we show that far field entries of
$\bC_{\bW}(\btheta)$, i.e. $(\bpsi^{i}_{\tilde{k}})\T \bC(\btheta)
\bpsi^{j}_{\tilde{l}}$, decay sub-exponentially with respect to
$p(d,w)$ if there exists an analytic extension of the covariance
function on a well defined domain in $\bC^{d}$. Thus it is not
necessary to compute all the entries. We introduce a distance
criterion approach to produce a sparse matrix $\tilde
\bC_{\bW}(\btheta)$.

\subsection{Sparsification of multilevel covariance matrix}

A sparse version of the covariance matrix $\bC_{\bW}(\btheta)$ can be
built by using a level and distance dependent strategy:

\begin{enumerate}[i)]

\item Given a cell $B^{i}_{k}$ at level $i \geq 0$ identify the
  corresponding tree node value Tree.node and the tree depth
  Tree.currentdepth. Note that the

  Tree.currentdepth and the MB level
  $q$ are the same for $q = 0,\dots,t$.
  %However, for $q = -1$ the MB
  %is associated to the Tree.currentdepth = 0.

\item Let $\bbK \subset \bbS$ be all the observations nodes contained
  in the cell $B^{i}_{k}$.

\item Let $\tau_{i,j} \geq 0$ be the distance parameter given by the
  user corresponding to the level $i,j$ from the block
  $\bC^{i,j}_{\bW}(\btheta)$.

\item Let the Targetdepth be equal to the desired level of the tree.
%In the case that it is $-1$ then the Targetdepth is zero.

\end{enumerate}
    The objective now is to find all the cells at the Targetdepth that
    overlap a hyper rectangle which is extended from $B^{i}_{k}$.  For
    all observations $\bx \in B^{i}_{l}$ along each dimension $k = 1,
    \dots, d$ let $x^{min}_k := \min_{ x_k \in B^i_m} x_k$ and
    $x^{max}_k := \max_{ x_k \in B^i_m} x_k$.  Any cell that
    intersects the interval $[x^{min}_{k} - \tau_{i,j} ,x^{max} +
      \tau_{i,j}]$ is included. This is done by searching the tree
    from the root node. At each traversed node check that all the
    nodes $\bx \in \bbK$ satisfy the following rule: If
\[
\bx \cdot \mbox{Tree}.v + \tau_{i,j} \leq Tree.threshold.
\]
then search down the left tree. If 
\[
\bx \cdot \mbox{Tree}.v - \tau_{i,j} > Tree.threshold.
\]
the search down the right tree. Otherwise search both trees.
%If this is true for all $\bx \in \bbK$ then the search continues down
%the left tree.  If this is false for all $\bx \in \bbK$ then the
%search continues down the Right tree, otherwise both the left and
%right tree are searched.
The full search algorithm is described in Algorithms
\ref{MLCM:algorithm3}, \ref{MLCM:algorithm4}, \ref{MLCM:algorithm5}
and \ref{MLCM:algorithm5a}.


\begin{algorithm}[htp]
  \KwIn{Tree, $\bbK$, Targetdepth, $\tau_{i,j}$}
  \KwOut{Targetnodes}
\Begin{
    Targetnodes $\leftarrow \emptyset $
    Targetnodes $\leftarrow$ 
    LocalSearchTree(Tree, $\bbK$, Targetdepth, $\tau_{i,j}$, Targetnodes);
}
\caption{SearchTree function(Tree, $\bbK$, Targetdepth, $\tau_{i,j}$)}
\label{MLCM:algorithm3}
\end{algorithm}






\begin{algorithm}[htp]
  \KwIn{Tree, $\bbK$, Targetdepth, $\tau_{i,j}$, Targetnodes}
  \KwOut{Targetnodes}
\Begin{

\If {Targetdepth = Tree.currentdepth}{return
Targetnodes = Targetnodes $\cup$ Tree.node}

\If {Tree = leaf}{return}

LeftRule  =  ChooseLeftRule($\bbK$, Tree, $\tau_{i,j}$)\\
RightRule =  ChooseRightRule($\bbK$, Tree, $\tau_{i,j}$)\\



\uIf {LeftRule($\bx$)=true $\forall \bx \in \bbK$}
{Targetnodes $\leftarrow$ LocalSearchTree(Tree.LeftTree, 
  $\bbK$, Targetdepth, $\tau_{i,j}$, Targetnodes)}

\uElseIf{RightRule($\bx$)=true $\forall \bx \in \bbK$}
{Targetnodes $\leftarrow$ LocalSearchTree(Tree.RightTree, 
  $\bbK$, Targetdepth, $\tau_{i,j}$, Targetnodes)}

\Else{Targetnodes $\leftarrow$ LocalSearchTree(Tree.LeftTree, 
$\bbK$, Targetdepth, $\tau_{i,j}$, Targetnodes)\\
Targetnodes $\leftarrow$ LocalSearchTree(Tree.RightTree, 
$\bbK$, Targetdepth, $\tau_{i,j}$, Targetnodes)}

%\eIf {Rule($\bx$)=true $\forall \bx \in \bbK$}
%{Targetnodes $\leftarrow$ LocalSearchTree(Tree.LeftTree, 
%$\bbK$, Targetdepth, $\tau$, Targetnodes)}
%{Targetnodes $\leftarrow$ LocalSearchTree(Tree.LeftTree, 
%$\bbK$, Targetdepth, $\tau$, Targetnodes)
%Targetnodes $\leftarrow$ LocalSearchTree(Tree.RightTree, 
%$\bbK$, Targetdepth, $\tau$, Targetnodes)
%}

}
\caption{LocalSearchTree(Tree, $\bbK$, Targetdepth, $\tau_{i,j}$) function}
\label{MLCM:algorithm4}
\end{algorithm}


\begin{algorithm}[htp]

  \KwIn{ $\bbK$, Tree, $\tau_{i,j}$ }
  \KwOut{Rule}
\Begin{
Rule($\bx$) := $\bx \cdot \mbox{Tree}.v + \tau_{i,j} \leq $
        Tree.threshold
 }
\caption{ChooseLeftRule($\bbK$) function}
\label{MLCM:algorithm5}
\end{algorithm}


\begin{algorithm}[htp]

  \KwIn{ $\bbK$, Tree, $\tau_{i,j}$ }
  \KwOut{Rule}
\Begin{
Rule($\bx$) := $\bx \cdot \mbox{Tree}.v - \tau_{i,j} > $
        Tree.threshold
 }
\caption{ChooseRightRule($\bbK$) function}
\label{MLCM:algorithm5a}
\end{algorithm}

In Figure \ref{multilevelcov:fig2}(b) an example for searching local
neighborhood cells of randomly placed observations in $\R^{2}$ is
shown. The orange nodes correspond to the source cell. By choosing a
suitable value for $\tau_{i,j}$ the blue nodes in the immediate cell
neighborhood are found by using Algorithms \ref{MLCM:algorithm3},
\ref{MLCM:algorithm4}, \ref{MLCM:algorithm5} and
\ref{MLCM:algorithm5a}.

The sparse matrix blocks $\bC^{i,j}_{\bW}(\btheta)$ can be built from
all the cells that are obtained from SearchTree function of Algorithm
\ref{MLCM:algorithm5}. Compute all the entries of
$\bC^{i,j}_{\bW}(\btheta)$ that correspond to the interactions between
any two cells $B^{i}_k \in \mcB^{i}$ and $B^{j}_l \in \mcB^{j}$. In
Algorithm \ref{MLCM:algorithm6}) the construction of the sparse matrix
$\tilde \bC^{i,j}_{\bW}(\btheta)$ is shown.

\begin{remark}
Since the matrix $\tilde \bC_{\bW}(\btheta)$ is symmetric it is only
necessary to compute the blocks $\bC^{i,j}_{\bW}(\btheta)$ for $i = 1,
\dots, t$ and $j = i, \dots t$.
\end{remark}

\begin{algorithm}
  \KwIn{Tree, $i$, $j$, $\tau_{i,j}$, $\mcB^i$, $\mcB^j$,
    $\mcD^i$, $\mcD^i$, $\bC(\btheta)$} \KwOut{$\tilde
    \bC^{i,j}_{\bW}(\btheta)$}
  \Begin{
        Targetnodes $\leftarrow \emptyset$\\
        \For{$B^{i}_{m} \in \mcB^{i}$}
            {$\bbK \leftarrow B^{i}_{m}$\\
            \For{$B^{j}_{q} \leftarrow $
            SearchTree(Tree, $\bbK$, Targetdepth $(i)$, $\tau_{i,j}$, 
            Targetnodes)}
            {
              \For{$\psi^i_k \in D^{i}$}{
                \For{$\psi^j_l \in D^{j}$}{
                  Compute $(\bpsi^{i}_{k})\T \bC(\btheta) \bpsi^{j}_{l}$ 
                  in $\tilde \bC^{i,j}_{\bW}(\btheta)$
                }
              }
            }
            }
    }
\caption{Construction of sparse matrix $\tilde \bC^{i,j}_{\bW}(\btheta)$}
\label{MLCM:algorithm6}
\end{algorithm}

\begin{figure}
\begin{center}
\begin{tikzpicture} 
  \begin{scope}[scale = 0.5]
    [place/.style={circle,draw=blue!50,fill=blue!20,thick,
      inner sep=0pt,minimum size=1.5mm}]
    %\draw[fill=red!5, step=16, thick] (0, 0) grid (16, 16);

    \filldraw[fill={rgb:red,143;green,188;blue,143},semitransparent, 
      thick] (0, 0) rectangle (16, 16);

    % \draw[blue, very thick] (0,0)rectangle (3,2);

    \draw[thin] (8,0) to (8,16);
    \draw[thin] (0,8) to (16,8);
    \draw[thin] (14,0) to (14,16);
    \draw[thin] (0,2) to (16,2);
    \draw[thin] (12,0) to (12,16);
    \draw[thin] (0,4) to (16,4);

    \node at (15,1) [] {${\bf G}_{\bW}$ };
    \node at (10,12) [] {$\bC^{t,t-1}_{\bW}(\btheta)$};
    \node at (4.5,6) [] {$\bC^{t-1,t}_{\bW}(\btheta)$};
    \node at (4.5,1) [] {$\bC^{0,t}_{\bW}(\btheta)$};
    \node at (10,6) [] {$\ddots$};
    \node at (13,1) [] {$\dots$};
    \node at (15,3) [] {$\vdots$};
    \node at (4.5,12) [] {$\bC^{t,t}_{\bW}(\btheta)$};

\end{scope}
\end{tikzpicture}
\end{center}
\caption{Multilevel covariance matrix where ${\bf G}_{\bW}
:=\bC^{0,0}_{\bW}(\btheta)$.}
\label{multilevelcov:fig1}
\end{figure}

\begin{figure*}[ht]
\begin{center}

  %\includegraphics[trim = 410 60 370 50, clip, width=4in,
  %  height=4in]{./figures/neighborhoodpattern.pdf}

 \begin{tikzpicture}[scale=0.59, every node/.style={scale=0.59}]
  \begin{scope} 
    [place/.style={circle,draw=blueish,fill=blueish,
        inner sep=0pt,minimum size=1.5mm},
      placegray/.style={circle,draw=gray!50,fill=gray!20,
        inner sep=0pt,minimum size=1.5mm},
        placenew/.style={circle,draw=darkorange!75,fill=darkorange!75,
      inner sep=0pt,minimum size=1.5mm}]
    \draw[step=8,gray,very thin] (0, 0) grid (8, 8);
    
    \node at (15,3.88) [] {\includegraphics[trim = 14.5cm 2cm 10cm 1.75cm,
        clip=true,
        height=8.43cm]{neighborhoodpattern.pdf}};
    
    \draw (2.2,5) to (2.2,8);
    \draw (0,5) to (2.2,5);
    \draw (6,4.15) to (8,4.15);
    \draw (6,0) to (6,4.15);

    \draw (4,0) to (4,8);
    \draw (0,5) to (4,5);
    
    \draw (0,2) to (4,2);
    \draw (0,5) to (4,5);
    \draw (4,4.15) to (8,4.15);
    \draw (6.75,4.15) to (6.75,8);
        
    \draw[dashed,gray] (6,4.15) to (6,8);
    \draw[dashed,gray] (0,2.3) to (6,2.3);
  
    \node at (0.5,7.5) [placenew] {};
    \node at (0.3,6.3) [placenew] {};
    
    \node at (2.5,5.5) [place] {};
    \node at (3.2,5.2) [place] {};

    \node at (5,6) [place] {};
    \node at (3.8,5.5) [place] {}; %
    \node at (3.8,6) [place] {};   %

    \node at (0.5,3.5) [place] {};
    \node at (1.5,2.5) [place] {};
    \node at (2.3,2.2) [place] {};
    
    \node at (1.3,0.3) [placegray] {};
    \node at (2.7,0.5) [placegray] {};
    \node at (2.2,1.4) [placegray] {};
    \node at (2.6,1.4) [placegray] {};
    \node at (4.2,3.5) [place] {}; %
    \node at (3.7,3.3) [place] {};

    \node at (6.5,4.3) [place] {}; %
    \node at (7.5,5) [placegray] {};

    \node at (4.3,2.3) [place] {};
    \node at (5.7,3.5) [place] {};
    \node at (6.2,3.4) [placegray] {};
    \node at (7.3,2.4) [placegray] {};


    \node at (7,7) [placegray] {};
    \node at (6,7.5) [place] {};
    \node at (7.5,7.5) [placegray] {};


    \node at (6.5,2.0) [placegray] {};
    \node at (0.5,7.0) [placenew] {};
    \node at (2.0,7.0) [placenew] {};
    \node at (5,3.75) [place] {}; %
    \node at (6,7.0) [place] {};
    \node at (7,2.0) [placegray] {}; %
    \node at (7.5,4.3) [placegray] {}; %

    \node at (4,8.75) [] {\Large $\tau_{i,j}$};
    \node at (-1,4.25) [] {\Large $\tau_{i,j}$};

    \draw[dashed,gray] (2,6.3) to (2,8);
    \draw[dashed,gray] (0,6.3) to (2,6.3);
    
    \coordinate (A) at (2,8);
    \coordinate (B) at (6,8);
    \coordinate (C) at (0,6.3);
    \coordinate (D) at (0,2.3);

\draw[dim={,10pt}]  (A) --  (B);
\draw[dim={,-15pt}]  (C) --  (D);

\node at (4,-0.65) {\Large (a)};
\node at (14.5,-0.65) {\Large (b)};
  \end{scope}
\end{tikzpicture} 


\end{center}



\caption{Neighborhood identification from source cell on a random
  kD-tree decomposition of observation locations in $\R^{2}$. (a)
  Cartoon example of axis wise distance criterion $\tau_{i,j}$ using
  Algorithms \ref{MLCM:algorithm3}, \ref{MLCM:algorithm4},
  \ref{MLCM:algorithm5} and \ref{MLCM:algorithm5a}. The orange
  observations knots correspond to the source cell. The blue knots
  correspond to all the target nodes. The gray knots are not included
  in the list of target nodes.  (b) Example of local neighborhood
  contained in the axis wise distance $\tau_{i,j}$.  The orange nodes
  are contained in the source cell. The blue nodes are are contained
  in the local neighborhood cells. The grey dots are all the
  observations that are not part of the source or local neighborhood
  cells.}
\label{multilevelcov:fig2}
\end{figure*}

\subsection{Computational cost of the multilevel
  matrix blocks of $\tilde{\bC}_{\bW}$}


The cost of computing the multilevel blocks $\tilde{\bC}^{i,j}_{\bW}$
will in general be $\mcO(N^2)$. However, for the special case that $d
= 2$ and $d = 3$ it is possible to use a fast summation method such as
the Kernel Independent Fast Multipole Method (KIFMM) by
\cite{ying2004} to compute the blocks more efficiently. To my
knowledge, there exists no equivalent fast summation method in higher
dimensions that works satisfactorily.

This KIFMM algorithm is flexible and efficient for computing the
matrix vector products $\bC(\btheta)\bx$ for a large class of kernel
functions, including the Mat\'{e}rn covariance function.  Given
$\tilde N$ sources and $\tilde M$ targets, experimental results show a
computational cost of about $\mcO(\tilde N + \tilde M)$, $\alpha
\approx 1$ with good accuracy ($\varepsilon_{FMM}$ between $10^{-6}$
to $10^{-8}$) with a slight degrade in the accuracy with increased
source nodes.


\begin{asum} Let $\bA(\btheta) \in \R^{\tilde M \times \tilde N}$ be a kernel
matrix formed from $\tilde N$ source observation nodes and $\tilde M$
target nodes in the space $\R^{d}$.  Suppose that there exists a fast
summation method that computes the matrix-vector products
$\bA(\btheta)\bx$ with $\varepsilon_{FMM}>0$ accuracy in $\mcO((\tilde
N + \tilde M)^{\alpha})$ computations, for some $\alpha \geq 1$ and
any $\bx \in \R^{d}$.
\end{asum}

%Given an octree multilevel tree domain decomposition in $\R^{3}$, as
%shown in \cite{Castrillon2013,Castrillon2015}, the authors described
%how to apply a Kernel Independent Fast Multipole Method (KIFMM) by
%\cite{ying2004} to compute all the blocks
%$\tilde{\bC}^{i,i}_{\bW}(\btheta) \in \R^{\tilde N \times \tilde N}$ for
%$i = 0,\dots,t$ in $\mcO(\tilde N(t+1)^2)$ computational steps to a
%fixed accuracy $\varepsilon_{FMM} > 0$.

For the kD-tree it is not possible to determine a-priori the sparsity
of the blocks $\tilde{\bC}^{i,j}_{\bW}(\btheta)$.
%given the level dependent distance parameter $\tau_{i,j}$.
However, for a given a value $\tau_{i,j} \geq 0$ by running Algorithm
\ref{MLCM:algorithm3} on every cell $B^{i}_k \in \mcB^{i}$, at level
$i$, with the Targetdepth corresponding for level $j$ it is possible
to determine the computational cost of constructing the sparse blocks
$\tilde{\bC}^{i,j}_{\bW}(\btheta)$ under the following assumption. Suppose
that maximum number of cells $B^{j}_k \in \mcB^{j}$ given by Algorithm
\ref{MLCM:algorithm3} is bounded by some $\gamma^{i,j} \in \bbN_+$.



\begin{prop} 
  The cost of computing each block $\tilde{\bC}^{i,j}_{\bW}(\btheta)$
for $i,j = 1,\dots,t$ by using a fast summation method with $1 \leq
\alpha \leq 2$ is bounded by
\[
\mcO(\gamma_{i,j} p 2^{i} (2^{t-j+1} p + 2^{t-i+1} p)^{\alpha} + 2p 2^{t}).
\]
\end{prop} 
\begin{proof} Let us look at the cost of computing all the 
interactions between any two cells $B^{i}_k \in \mcB^{i}$ and
$B^{j}_l \in \mcB^{j}$. Without loss of generality assume that $i
\leq j$. For the cell $B^{l}_k$ there is at most $p$
multilevel vectors and from Lemma \ref{MultilevelREML:lemma2}
$2^{t-i+1} p$ non zero entries. Similarly for $B^{j}_l$.  All
the interactions $(\bpsi^{i}_{\tilde{k}})\T \bC(\btheta)
\bpsi^{j}_{\tilde{l}}$ now have to be computed, where
$\bpsi^{i}_{\tilde{k}} \in B^{i}_k$ and $\bpsi^{j}_{\tilde{l}} \in
B^{j}_l$.

The term $\bC(\btheta) \bpsi^{j}_{\tilde{l}}$ is computed using a FMM
with $2^{t-j+1} p$ sources and $2^{t-i+1} p$ targets at a cost of
$\mcO($ $(2^{t-j+1} p + 2^{t-i+1} \tilde p)^{\alpha})$.  Since there
is at most $p$ multilevel vectors in $B^{i}_k$ and $B^{j}_l$ then the
cost for computing all the interactions $(\bpsi^{i}_{\tilde{k}})\T
\bC(\btheta) \bpsi^{j}_{\tilde{l}}$ is $\mcO(p(2^{t-j+1} p + 2^{t-i+1}
p)^{\alpha} + 2^{t-i+1}p)$.

Now, at any level $i$ there is at most $2^{i}$ cells, thus the result
follows.
\end{proof}



%% Multilevel Estimator and Predictor --------------------------------
%%

\section{Multilevel estimator and predictor}
\label{multilevelestimator}
The multilevel random projection tree can be exploited in such a way
to significantly reduce the computational burden and to further
increase the numerical stability of the estimation and prediction
steps. This is an extension of the multilevel estimator and predictor
formulated in \cite{Castrillon2015} to binary tree in higher
dimensions. The former is based on Oct-tree decompositions, thus
making it unsuitable for higher dimensional problems.

\subsection{Estimator}

The multilevel likelihood function, $l_{\bW}(\theta)$ (see equation
\eqref {Introduction:multilevelloglikelihood} ), has the clear
advantage of being decoupled from the vector $\bbeta$. Furthermore,
the multilevel covariance matrix $\bC_{\bW}(\btheta)$ will be more
numerically stable than $\bC(\btheta)$ thus making it easier to invert
and to compute the determinant.

However, it is not necessary to perform the MLE estimation on the full
covariance matrix $\bC_{\bW}(\btheta)$, instead construct a series of
multilevel likelihood functions $\tilde{\ell}^n_{\bW}(\btheta)$, $n =
0,\dots t$, by applying the partial transform $[\bW_t \T, \dots, \bW_n
  \T ]$ to the data $\bZ$ where
\begin{equation}
  \begin{split}
\tilde{\ell}^n_{\bW}(\btheta)
=-\frac{\tilde{N}}{2}\log(2\pi)-\frac{1}{2}\log
\det\{\tilde{\bC}_{\bW}^n(\btheta)\}
-\frac{1}{2}(\bZ^n_{\bW})\T\tilde{\bC}^n_{\bW}(\btheta)^{-1}\bZ^n_{\bW},
\end{split}
\label{Introduction:multilevelloglikelihoodreduced2}
\end{equation}
where $\bZ^n_{\bW} :=[\bW_t \T, \dots, \bW_i \T ] \T \bZ$,
$\tilde{N}$ is the length of $\bZ^n_{\bW}$,
$\tilde{\bC}^n_{\bW}(\btheta)$ is the $\tilde{N} \times \tilde{N}$
upper-left sub-matrix of $\tilde{\bC}_{\bW}(\btheta)$ and
$\bC^n_{\bW}(\btheta)$ is the $\tilde{N} \times \tilde{N}$ upper-left
sub-matrix of $\bC_{\bW}(\btheta)$.

A consequence of this approach is that the matrices
$\bC^n_{\bW}(\btheta)$, $n = -1, \dots, t$ will be increasingly more
numerically stable, thus easier to solve computationally, as shown in
the following theorem.
\begin{prop} 
  \label{Multilevelapproach:theo2}
Let $\kappa(A) \rightarrow \R$ be the condition number of the matrix
$A \in \R^{N \times N}$ then
\[
\kappa(\bC^{t}_{\bW}(\btheta)) \leq \kappa(\bC^{t-1}_{\bW}(\btheta)) \leq 
\dots
\leq
\kappa(\bC_{\bW}(\btheta)) \leq
\kappa(\bC(\btheta)).
\]
\end{prop} 
\begin{proof} A simple extension of the proof in 
Proposition \ref{Multilevelapproach:theo1}.
\end{proof}



\begin{remark} If $\bC(\btheta)$ is symmetric positive definite 
then for $n = 0, \dots, t$ the matrices $\bC^{n}_{\bW}(\btheta)$ are
symmetric positive definite. The proof is immediate.
\end{remark}

\begin{remark} If the matrix $\tilde \bC^{n}_{\bW}(\btheta)$ is close 
to $\bC^{n}_{\bW}(\btheta)$, for $n = 1,\dots,d$, in some matrix norm
sense, the condition number of $\tilde \bC^{n}_{\bW}(\btheta)$ will be
close to $\bC^{n}_{\bW}(\btheta)$. Full error bounds will be derived in
a future publication.
%In section \ref{errorestimates}, for a class of covariance functions,
%it can be shown than for sufficiently large $\tau$ and/or $w,a \in
%\bbN_{+}$ [with the index set $\Lambda(w)^{m,g}$ or $\tilde
%  \Lambda(w)^{m,g}$] $\tilde \bC^{n}_{\bW}(\btheta)$ will be close to
%$\bC^{n}_{\bW}(\btheta)$. Thus $\tilde{\bC}^n_{\bW}(\btheta)$ will be
%symmetric positive definite.
\end{remark}

\subsection{Predictor}
In this section we show how to construct a multilevel BLUP with a
well conditioned multilevel covariance matrix. It will be shown that
the multilevel predictor is \emph{exact}, i.e. the multilevel
predictor and the solution of the constrained predictor problem
(equations \eqref{GLSbeta} and \eqref{KrigBLUP}) are the same.
However, the multilevel form can be significantly easier problem to
solve numerically.

Consider the following system of equations
\begin{eqnarray}
\left( {{\begin{array}{*{20}c}
 \bC(\btheta) \hfill & \bX \hfill \\
 \bX\T \hfill & \0 \hfill \\
\end{array} }} \right)\left( {{\begin{array}{*{20}c}
 \hat \bgamma \hfill \\
 \hat \bbeta \hfill \\
\end{array} }} \right)=\left( {{\begin{array}{*{20}c}
 \bZ \hfill \\
 \0 \hfill \\
\end{array} }} \right).
\label{Kriging:problem}
\end{eqnarray}
From the argument given in \cite{Nielsen2002} it is not hard to show
that the solution of this problem leads to equation \eqref{GLSbeta}
and $\hat \bgamma(\btheta) = \bC^{-1}(\btheta)(\bZ - \bX \hat
\bbeta(\btheta))$. The BLUP can be evaluated as
\begin{equation}
  \hat Z(\bx_0)
  =\bk(\bx_0)\T\hat \bbeta(\btheta)+\bc(\btheta)\T
  \hat \bgamma(\btheta)
\label{Kriging}
\end{equation}
and the Mean Squared Error (MSE) at the target point $\bx_0$ is given by
%\begin{equation}
\[
1 + 
\tilde{\bu}
\T
(\bX \T 
\bC(\btheta)^{-1} \bX )^{-1}
\tilde{\bu}
-\bc(\btheta)\T\bC^{-1}(\btheta)\bc(\btheta)
\]
%\label{Kriging:MSE}
%\end{equation}
where $\tilde{\bu}\T := (\bX \bC^{-1}(\btheta)
\bc(\btheta) - \bk(\bx_0))$.

From \eqref{Kriging:problem} it is observed that $\bX\T \hat
\bgamma(\btheta) = \0$. This implies that $\hat{\bgamma} \in \R^{n}
\backslash \mcP^{p}(\mathbb{S})$ and can be uniquely rewritten as
$\hat{\bgamma} = \bW\T \bgamma_{\bW}$ for some $\bgamma_{\bW} \in
\R^{N-p}$. Now, rewrite $\bC(\btheta) \hat \bgamma + \bX \hat \bbeta
= \bZ$ as
\begin{equation}
\bC(\btheta) \bW\T \bgamma_{\bW} + \bX \hat \bbeta =
       \bZ.
\label{Kriging:eqn1}
\end{equation}
Now apply the matrix $\bW$ to equation \eqref{Kriging:eqn1} and obtain
$\bW \{\bC(\btheta) \bW\T \bgamma_{\bW} + \bX \hat \bbeta\} = \bW
\bZ.$ Since $\bW \bX = \0$ then
\[
\bC_{\bW}(\btheta)
\bgamma_{\bW} = \bZ_{\bW}.
\]
A simple preconditioner $\bP_{\bW}$ can be formed from the diagonal
entries of the matrix $\bC_{\bW}$ i.e. $\bP_{\bW} = diag(\bC_{\bW})$ leading
to the following system of equations
\[
\bP_{\bW}^{-1}\bC_{\bW}(\btheta)
\bgamma_{\bW} = \bP_{\bW}^{-1} \bZ_{\bW}.
\]
Note that in some cases $\bC_{\bW}(\btheta)$ will have very small
condition numbers. For this case we can set $\bP_{\bW}:= I$, i.e. no
preconditioner.

\begin{theorem}
If the covariance function $\phi:\Gamma_d \times \Gamma_d \rightarrow
\R$ is positive definite, then the matrix $\bP_{\bW}(\btheta)$ is always
symmetric positive definite.
\label{multilevelKriging:lemma2}
\end{theorem}
\begin{proof}
Immediate.
\end{proof}


The vector $\hat \bgamma$ can be obtained by applying the inverse
transform $\bW\T$ i.e.  $\hat \bgamma = \bW\T \bgamma_{\bW}$.  From
\eqref{Kriging:problem} the GLS $\hat \bbeta$ can now be computed as a
least squares, i.e.  $\hat{\bbeta} = (\bX\T \bX)^{-1}\bX \T (\bZ -
\bC(\btheta)\hat{\bgamma})$.


\begin{remark}
  Notice that to solve the GLS estimate $\hat \bbeta$ it is not
  necessary to compute the full GLS of equation \eqref{GLSbeta}, but a
  least squares is all that is required. This is in contrast to the
  GLS estimate of equation \eqref{GLSbeta} where if an iterative
  method is used the covariance matrix $\bC(\btheta)$ has to be
  inverted for each of the columns of $\bX$ i.e. $p$ times.
  \end{remark}


\section{Numerical computation of multilevel estimator and predictor}
\label{numericalcomputation}


\subsection{Estimator: Computation of 
$\log{\det\{ \tilde{\bC}^n_{\bW}\}}$ and $(\bZ^n_{\bW})\T
  (\tilde{\bC}^n_{\bW})^{-1}\bZ^n_{\bW}$}

An approach to computing the determinant of
$\tilde{\bC}^n_{\bW}(\btheta)$ is to apply a sparse Cholesky
factorization technique such that $\bG\bG\T =
\tilde{\bC}^n_{\bW}(\btheta)$, where $\bG$ is a lower triangular
matrix. Notice that the eigenvalues of $\bG$ are located on the
diagonal. This leads to $\log \det \{\tilde{\bC}^n_{\bW}(\btheta)\} =
2 \sum_{i = 1}^{\tilde{N}} \log{\bG_{ii}}$.

The direct application of the sparse Cholesky algorithm can leads to
significant fill-in of the factorization matrix $\bG$. To alleviate
this problem it is typical to use matrix reordering techniques. In
particular, the fill-in are reduced by using the sparse Cholesky
factorization \emph{chol} from the Suite Sparse 4.2.1 package
(\cite{Chen2008,Davis2009,Davis2005,Davis2001,Davis1999}) coupled with
Nested Dissection (NESDIS) function package.

In practice, this approach leads to a significant reduction of
fill-in. To my knowledge a theoretical worse case complexity bounded
exists for $d = 2$ or $d = 3$ dimensions (see \cite{Castrillon2015}).

Two choices for the computation of $(\bZ^n_{\bW}) \T \tilde
\bC^n_{\bW}(\btheta)^{-1}$  $\tilde \bZ^n_{\bW}$ are open to us: i) a
Cholesky factorization of $\tilde{\bC}^n_{\bW}(\btheta)$, or ii) a
Preconditioned Conjugate Gradient (PCG).  The PCG choice requires
significantly less memory and allows more control of the error.
However, the sparse Cholesky factorization of
$\tilde{\bC}^n_{\bW}(\btheta)$ has already been used to compute the
determinant. Thus we can use the same factors to compute $ (\tilde
\bZ_{\bW}^n) \T\tilde{\bC}^n_{\bW}(\btheta)^{-1} \tilde \bZ^n_{\bW}$.
The PCG avenue will be explored in more detail in Section
\ref{comppred}.




%% %%%%%%%%%%%%%%%%%%%%%%%%%%%%%%%%%%%%%%%%%%%%%%%%%%%%%%%%%%%%%%%%%%%%%%%%
%\subsection{Predictor: Computation of $\bgamma_{\bW}$ and $\hat{\bbeta}(\btheta)$}

\subsection{Predictor computation}
\label{comppred}

For the predictor stage a different approach is used. Instead of
inverting the sparse matrix $\tilde \bC_{\bW}(\btheta)$ a
Preconditioned Conjugate Gradient (PCG) method is employed to compute
$\hat \bgamma_{\bW} = \bC_{\bW}(\btheta)^{-1} \bZ_{\bW}$.

Recall that $\bC_{\bW} = \bW \bC(\btheta) \bW \T$, $\hat \bgamma_{\bW} =
\bW \hat \bgamma$ and $\bZ_{\bW} = \bW \bZ$. Thus the matrix vector
products $\bC_{\bW}(\btheta) \bgamma_{\bW}^n$ in the PCG iteration are computed
within three steps:
\[
\bgamma_{\bW}^n \xrightarrow[(1)]{\bW \T \bgamma_{\bW}^n} 
\ba_n \xrightarrow[(2)]{\bC(\btheta) \ba_n}
\bb_n \xrightarrow[(3)]{\bW \bb_n}
\bC_{\bW}(\btheta) \bgamma_{\bW}^n 
\]
where $\bgamma_{\bW}^0$ is the initial guess and $\bgamma_{\bW}^n$ is the $n^{th}$
iteration of the PCG.
\begin{enumerate}[(1)]

\item Transformation from multilevel representation to single
  level. This is done in at most $\mcO(Nt)$ steps.

\item Perform matrix vector product using a summation method. For $d =
  2,3$ a KIFMM is used to compute the matrix vector products with
  $\alpha \approx 1$. For $d > 3$ to my knowledge there is no reliable
  fast summation method.

\item Convert back to multilevel representation.

\end{enumerate}

The matrix-vector products $\bC_{\bW}(\btheta) \bgamma_{\bW}^n$, where
$\bgamma_{\bW}^n \in \R^{N-p}$, are computed in $\mcO(N^{\alpha} + 2
Nt)$ computational steps to a fixed accuracy $\varepsilon_{FMM} > 0$.
Note that $\alpha \geq 1$ is dependent on the efficiency of the fast
summation method. The total computational cost is $\mcO(kN^{\alpha} +
2Nt)$, where $k$ is the number of iterations needed to solve
$\bP^{-1}_{\bW} \bC_{\bW}(\btheta) \bar{\bgamma}_{\bW} (\btheta) =
\bP^{-1}_{\bW} \bar{\bZ}_{\bW}$ to a predetermined accuracy
$\varepsilon_{PCG} > 0$.

\begin{remark}
The introduction of a preconditioner can degrade the accuracy for
computing $\hat \bgamma_{\bW} = \bC_{\bW}(\btheta)^{-1} \bZ_{\bW}$
with the PCG method. The residual accuracy $\varepsilon_{PCG}$ of the
PCG iteration has to be set such that the residual of
the\emph{unpreconditioned} system $\|\bC_{\bW}(\btheta) \bgamma_{\bW}
(\btheta) - \bZ_{\bW}\|_{l^2} < \varepsilon$ for a user given
tolerance $\varepsilon > 0$.
\end{remark}

Now compute $\hat \bgamma = \bW\T \hat \bgamma_{\bW}$ and $\hat{\bbeta} =
(\bX\T \bX)^{-1}\bX \T (\bZ - \bC(\btheta)\hat{\bgamma})$ in at most
$\mcO(N^{\alpha} + Np + p^{3})$ computational steps. The matrix vector
product $\bc(\btheta)\T \hat{\bgamma}(\btheta)$ is computed in
$\mcO(N)$ steps.  Finally, the total cost for computing the estimate
$\hat{\bZ}(\bx_0)$ from \eqref{Kriging} is $\mcO(p^{3} + (k +
1)N^{\alpha} + 2Nt)$.

%% Error Estimates ----------------------------------------------------
%%
\section{Multilevel covariance matrix decay}
\label{errorestimates}

We derive decay estimates of the multilevel covariance matrix. It can
be shown that most of the coefficients are small and thus it is not
necessary to compute all of them. The final objective is to build a
posteriori error estimates for $\bx_{\bW} =
\bC^{n}_{\bW}(\btheta)^{-1}\bZ^n_{\bW}$ and $\log
\bC^{n}_{\bW}(\btheta)$ that are needed for solving the multilevel
estimator MLE. However, the full analysis is extensive and will be
completed in a future publication. As a first step we show the decay
of the multievel covariance matrix. Note that this is not trivial and
uses the results derived in the Appendix. We recommend to first read
the appendix since part of the notation used in this section is
defined there.


%Consider the full solution $\bx_{\bW} =
%\bC^{n}_{\bW}(\btheta)^{-1}\bZ^n_{\bW}$ and sparse solution $\tilde
%\bx_{\bW} = \tilde \bC^{n}_{\bW}(\btheta)^{-1}\bZ^n_{\bW}$ for $n = 0,
%\dots, t$, then the error can be bounded as
%\begin{equation}
%\|\bx_{\bW} - \tilde \bx_{\bW}\|_{l^2} 
%\leq 
%\| \bC^{-1}_{\bW}(\btheta) - \tilde \bC^{-1}_{\bW}(\btheta) \|_2 
%\|\bZ_{\bW}\|_{l^2}.
%\label{errorestimates:eqn1}
%\end{equation}
%The ultimate goal is to derive a full a posteriori error estimate of
%$\|\bx_{\bW} - \tilde \bx_{\bW}\|_{l^2}$ with respect to the distance
%criterion $\tau_{i,j}$.  In this section estimates of the decay of the
%multilevel matrix $\bC_{\bW}(\btheta)$ are obtained. These will be
%critical to derive a full a posteriori error scheme.




The decay of the coefficients of the matrix $\bC_{\bW}(\btheta)$ will
depend directly on the choice of the multivariate index set
$\mcQ^{d}_{w}$ and the analytic regularity of the covariance
function. In general, the Mat\'{e}rn covariance function will be
analytic except for a derivative discontinuity at the
origin. However, with the application of the distance criterion a
minimal distance can be guaranteed and the origin will be avoided all
together.

In the following theorem, without loss of generality, it is assumed
that the covariance function domain is $\Gamma^{d} \times \Gamma^{d}$
for any two cells $B^{i}_{m} \in \mcB^{i}$ and $B^{j}_{q} \in
\mcB^{j}$. We will show later that this can be achieved through a
linear pullback.

\begin{theorem} Suppose that $0< \delta < 1$, $\hat
  \sigma := \sigma (1 - \delta)$, and $\phi(\bx,\by;\btheta) \in
  C^{0}(\Gamma^{d} \times \Gamma^{d};\R)$ can be analytically extended
  on $\mcE^{d}_{\sigma} \times \mcE^{d}_{\sigma}$ and is bounded by
  $\tilde M(\phi)$. Let $\mcP^{ p}(\mathbb{S})^{\perp}$ be the
  subspace in $\R^{N}$ generated by the index set $\mcQ^{d}_{w}$ for
  some $w \in \bbN_{+}$. For $i,j = 0,\dots,t$ consider any
  multilevel vector $\bpsi^{i}_m \in \mcP^{p}(\mathbb{S})^{\perp}$,
  with $n_m$ non-zero entries, from the cell $B^{i}_{m} \in \mcB^{i}$
  and any multilevel vector $\bpsi^{j}_{q} \in \mcP^{
    p}(\mathbb{S})^{\perp}$, with $n_q$ non-zero entries, from the
  cell $B^{j}_{q} \in \mcB^{j}$. If $p(d,w) \geq \left(\frac{2
    d}{\kappa(d)}\right)^{d}$ then
\[
\begin{split}
\left|
\sum_{r = 1}^{N} 
\sum_{h = 1}^{N} 
\phi(\bx_r,\by_h; \btheta) 
\bpsi^i_m[h] \bpsi^j_q[r] 
\right|
&\leq
\sqrt{n_mn_q}
\left( \frac{
  C(\tilde M,\sigma)^{d} e^{d - \sigma(1 - \delta) + 1} \hat \sigma d
}
 {
   (\sigma \delta)^{d}} \right)^2
 \\
 &
 \left( \frac{e^{\hat \sigma}}{1 - e^{-\hat \sigma}} \right)^{2d}
 \exp \left(-\frac{2d}{e} \hat \sigma  p^{\frac{1}{d}}
 \right) p^{2\left(\frac{d-1}{d}\right)}.
\end{split}
\]
\label{errorestimates:theorem1}
\end{theorem}
% Proof
\begin{proof} 
We first have that
\[
\begin{split}
\sum_{k = 1}^{N} 
\sum_{l = 1}^{N} 
\phi(\bx_k,\by_l; \btheta) 
\bpsi^i_m[k] \bpsi^j_q[l] 
&=
\sum_{k = 1}^{N} 
\sum_{l = 1}^{N}
\lim_{g \rightarrow \infty}
(\bI^d_{g}
\otimes
\bI^d_{g})[
\phi(\bx_k,\by_l; \btheta)] 
\bpsi^i_m[k] \bpsi^j_q[l] \\
&=
\sum_{k = 1}^{N} 
\sum_{l = 1}^{N}
(I_d - \mcS_{w,d}) 
\otimes
(I_d - \mcS_{w,d}) \\
&[\phi(\bx_k,\by_l; \btheta)] 
\bpsi^i_m[k] \bpsi^j_q[l].
\end{split}
\]
The last equality follows from $\bpsi^{i}_m, \bpsi^{j}_{q} \in \mcP^{
  p}(\mathbb{S})^{\perp}$. We now have that
\[
\begin{split}
& \sum_{k = 1}^{N} 
\sum_{l = 1}^{N}
\|(I_d - \mcS_{w,d}) 
\otimes
(I_d - \mcS_{w,d})
[\phi(\bx_k,\by_l; \btheta)] \|_{L^{\infty}_{\rho}(\Gamma^d)}
|\bpsi^i_m[k]| |\bpsi^j_q[l]|
\\
&\leq 
\|(I_d - \mcS_{w,d}))[\phi(\bx_k,\by_l; \btheta)] \|^{2}_{L^{\infty}_{\rho}(\Gamma^d)}
\sum_{k = 1}^{N} 
\sum_{l = 1}^{N}
|\bpsi^i_m[k]| |\bpsi^j_q[l]|.
\end{split}
\]
Since $\bpsi^i_m$ and $\bpsi^j_q$ are orthonormal then
\[
\begin{split}
\sum_{k = 1}^{N} 
\sum_{l = 1}^{N}
|\bpsi^i_m[k]| |\bpsi^j_q[l]|
&\leq \sqrt{n_m n_q}
\|\bpsi^i_m[k]\|_{l^2} \|\bpsi^j_q[l]\|_{l^2} =
\sqrt{n_m n_q}.
\end{split}
\]
From Lemma \ref{interpolation:lemma1} the result follows.
\end{proof}



\begin{remark} Recall that the restriction $p(d,w) \geq \left(\frac{2
  d}{\kappa(d)}\right)^{d}$ is not strict and can be relaxed such that
  sub-exponential convergence is still obtained. See Remark
  \ref{interpolation:remark1}.
\end{remark}

\begin{remark}
  The decay of the coefficients of $\bC^{i,j}_{\bW}$ is sub-exponential
  with respect to $p$.  Even for a moderate magnitude for $\hat
  \sigma > 0$, $p > 0$ and $d \geq 1$ the entries of the
  multilevel matrix $\bC^{i,j}_{\bW}$ that do not correspond to the
  cells given by the distance criterion parameter $\tau_{i,j} \geq 0$
  will be close to zero.
  %Thus the matrix $\bC^{i,j}_{\bW}$ will be
  %highly sparse as the number of observations $N$ increases.
\end{remark}

Theorem \ref{errorestimates:theorem1} provides a mechanism to control
the decay of the coefficients of the multilevel covariance matrix
$\bC_{\bW}$. To this end, we are interested in the analytic extension
and the uniform bound $\tilde M(\phi) \leq \infty$ of the Mat\'{e}rn
covariance function
\[
\phi(r;\btheta):=\frac{1}{\Gamma(\nu)2^{\nu-1}} \left(
\sqrt{2\nu} r(\btheta) \right)^{\nu} K_{\nu} \left(
\sqrt{2\nu} r(\btheta) \right)
\]
on a subdomain in $\bbC^{d} \times \bbC^{d}$, where $r(\btheta) =
(\bx-\by)^{T}$ $\text{diag}(\btheta) (\bx - \by)
)^{\frac{1}{2}}$, $\btheta=[\theta_1, \dots, \theta_d] \in
\R^{d}_{+}$ are positive constants, $\text{diag}(\theta) \in \R^{d
  \times d}$ is a diagonal matrix with the vector $\btheta$ on the
diagonal, and $\bx, \by \in \R^{d}$.

The polynomial function is an entire function. However, the function
$K_{\nu}(\vartheta)$ and $\vartheta^{\frac{1}{2}}$ are analytic for
all $\vartheta \in \bbC$ except at the branch cut $(-\infty,0]$.  Thus
it is sufficient to check the analytic extension of $r = \|\bx -
\by\|_{l^{2}} = \Big( \sum_{k=1}^{d} \theta_{k} (x_k - y_k)^{2}
\Big)^{\frac{1}{2}}$.  Let $z \in \bbC$ be the complex extension of
$r \in \R$. More precisely, $z = \Big( \sum_{k=1}^{d} \theta_{k}
z_k^{2} \Big)^{\frac{1}{2}}$, where $z_k \in \bbC$ is the complex
extension of $(x_k - y_k)$.

Let $\vartheta = \sum_{k=1}^{d} \theta_{k} z_k^{2}$, then by taking
the appropriate branch $\Real z = r_{\vartheta}$ $\cos{(
  \theta_{\vartheta}/2)}$, where $r^2_{\vartheta} = (\Real
\vartheta)^2 + (\Imag \vartheta)^2$ and $\theta_{\vartheta} =
\tan^{-1} \frac{\Imag \vartheta}{\Real \vartheta}$. Due to the branch
cut at $(-\infty,0]$ we impose the restriction that $\Real \vartheta >
0$ as $x_k$ and $y_k$ are extended in the complex plane.  Consider any
two cells $B^{i}_{m} \in \mcB^i$ and $B^{j}_{q} \in \mcB^j$, at levels
$i$ and $j$ with the associated distance criterion constant
$\tau_{i,j}>0$. From Algorithms \ref{MLCM:algorithm3},
\ref{MLCM:algorithm4}, \ref{MLCM:algorithm5} \ref{MLCM:algorithm6},
for any observations $\bx^{*} \in B^{i}_{m}$ and $\by^{*} \in
B^{j}_{q}$ we have that $(x^*_k - y^*_k)^2 \geq \tau^2_{i,j}$ for $k =
1,\dots,d$.  For the rest of the discussion it is assumed that complex
extension is respect to each component $k = 1,\dots,d$ unless
otherwise specified.


Let $x^{min}_k := \min_{ x^*_k \in B^i_m} x^*_k$, $x^{max}_k := \max_{
  x^*_k \in B^i_m} x^*_k$, $y^{min}_k := \min_{ y^*_k \in B^i_m}
y^*_k$, $y^{max}_k := \max_{ y^*_k \in B^i_m} y^*_k$ and
$\alpha_k,\gamma_k \in [-1,1]$. Define the region $\mcX^{i}_{m} :=
[x^{min}_1,$ $x^{max}_1] \times \dots \times [x^{min}_d,x^{max}_d]$
and $\mcY^{j}_{q} := [y^{min}_1,y^{max}_1] \times \dots \times
[y^{min}_d,y^{max}_d]$.

%rescale the widths of each of the cells $B^{i}_{m}$ and $B^{j}_{q}$
%with respect to the domain $\Gamma$.

The next step is to redefine $\phi(\bx;\by;\btheta):\mcX^{i}_m \times
\mcY^{i}_{q} \rightarrow \R$ as
$\phi(\balpha,\bgamma;\btheta):\Gamma^d \times \Gamma^d \rightarrow
\R$ through a pullback. Let $x_k = \left(\frac{\alpha_k + 1}{2}
\right) a_k + b_k$ and $y_k = \left(\frac{\gamma_k + 1}{2} \right) c_k
+ d_k$, where $a_k = x^{max}_{k} - x^{min}_{k}$, $b_k = x^{min}_{k}$,
$c_k = y^{max}_{k} - y^{min}_{k}$ and $d_k = y^{min}_{k}$.


%The covariance function
%$\phi(\bx;\by;\btheta):\mcX^{i}_m \times \mcY^{i}_{q} \rightarrow \R$
%is reformulated as $\phi(\balpha,\bgamma;\btheta):\Gamma^d \times
%\Gamma^d \rightarrow \R$.

Extend $\alpha_k \rightarrow \alpha_k + v_k$ and $\gamma_k
\rightarrow \gamma_k + w_k$ where $v_k:= v^R_k + iv^I_k$, $w_k:= w^R_k
+ iw^I_k$, and $v^R_k,v^I_k,w^R_k,w^I_k \in \R$. Let $\tilde x_k$ be
the extension of $x_k$ in the complex plane and similarly for
$\tilde y_k$.


It follows that $\tilde x^R_k := \Real \tilde x_k = \frac{1}{2}
(\alpha_k + 1 + v^R_k)a_k + b_k$, $\tilde x^I_k = \Imag \tilde x_k =
\frac{v^I_k}{2} a_k$, $y^R_k := \Real \tilde y_k = \frac{1}{2}
(\gamma_k + 1 + w^R_k)c_k + d_k$, and $y^I_k := \Imag \tilde y_k =
\frac{w^I_k}{2} c_k$.  After some manipulation
\begin{equation}
\begin{split}
\Real z^2_k &= (\tilde x^R_k - \tilde y^R_k)^2
- (\tilde x^I_k - \tilde y^I_k)^2 
=
(x_k - y_k)^2 
+
\frac{1}{4}(v^R_k a_k - w^R_k c_k)
+ (x_k - y_k)(v^R_k a_k - w^R c_k)
\\
&
-\frac{1}{4}(a_kv^I_k - c_k w^I_k)^2.
\end{split}
\label{errorestimates:eqn2}
\end{equation}


Recall that $(x_k - y_k)^2 \geq \tau^2_{i,j}$ and suppose that there
is a positive constant $\delta_{k} > 0 $ such that
\begin{equation}
\delta_{k} \leq 
%-\frac{4\,\tau -\sqrt{32\,\tau ^2+8\,\tau +1}+1}{4\,\tau }
\frac{\sqrt{32\,\tau_{i,j} ^2+8\,\tau_{i,j} +1}-1 - 4\,\tau_{i,j}  }{4\,\tau_{i,j}}.
\label{errorestimates:eqn2a}
\end{equation}

Assume that $|v^R_k|\leq \tau_{i,j} \delta_{k} / a_{k}$, $|v^I_k|\leq
\tau_{i,j} \delta_{k}/a_{k}$, $|w^R_k|\leq \tau_{i,j} \delta_{k} /
c_{k}$, and $|w^I_k|\leq \tau_{i,j} \delta_{k} / c_{k}$. From
equations \eqref{errorestimates:eqn2} and \eqref{errorestimates:eqn2a}
it follows that
\begin{equation}
\Real z^2_k \geq \tau_{i,j}^2 (1 - 4 \delta_{k}^2) - \frac{\tau_{i,j}
  \delta_{k}}{2} > 0.
\label{errorestimates:eqn3}
\end{equation}
%\begin{equation}
%\Real z^2_k \geq \tau_{i,j,m,q}^2 (1 - \frac{9}{16}(a_k + c_k)) > 0
%\label{errorestimates:eqn3}
%\end{equation}
Furthermore,
\begin{equation}
\begin{split}
  &
  |\Real z^2_k| \leq 
(\max\{|y^{max}_k - x^{min}_{k}|,|x^{max}_k - y^{min}_{k}|\})^2
\\
&+ \frac{1}{2}\tau_{i,j} \delta_{k} 
+ 
\max\{|y^{max}_k - x^{min}_{k}|,|x^{max}_k - y^{min}_{k}|\} 
2\tau_{i.j} \delta_{k}
+ \tau^2_{i,j} \delta_{k}^2 
\leq
1 + \frac{5}{2}\tau_{i,j} \delta_{k} + \tau^2_{i,j} \delta_{k}^2. 
\end{split}
\label{errorestimates:eqn6}
\end{equation}
Similarly,
\begin{equation}
  |\Imag z^2_k| = |2(\tilde x^R_k - \tilde y^R_k)(x^I_k - y^I_k)|
  \leq
2 \tau_{i.j} \delta_{k} + 4 \tau^2_{i,j} \delta_{k}^2.
\label{errorestimates:eqn4}
\end{equation}

We now show how $\alpha_k$ and $\gamma_k$ can be extended into the
Bernstein ellipses $\mcE_{\sigma^{\alpha}_k}$ and
$\mcE_{\sigma^{\gamma}_k}$, for some $\sigma^{\alpha}_k > 0$ and
$\sigma^{\gamma}_k > 0$ such that $\Real z^2_k > 0$. Recall that
$|v^R_k|\leq \tau_{i,j} \delta_{k} / a_{k}$, $|v^I_k|\leq \tau_{i,j}
\delta_{k}/a_{k}$, $|w^R_k|\leq \tau_{i,j} \delta_{k} / c_{k}$, and
$|w^I_k|\leq \tau_{i,j} \delta_{k} \ c_{k}$.
%We restrict the length of the extension of $(x^{min}_k, x^{max}_k)$
%and $(y^{min}_k, y^{max}_k)$ by $\tau_{i,j}/2$
These restrictions form a region in $\bbC \times \bbC$ and a Bernstein
ellipse is embedded (See Figure \ref{analyticity:fig1}).  This is done
by solving the following equation: $\frac{e^{\sigma^{\alpha}_k} +
  e^{-\sigma^{\alpha}_k}}{2} = 1 + \frac{\tau_{i,j}
  \delta_{k}}{a_k}$. The unique solution is
\[
\sigma^{\alpha}_k = \cosh^{-1} \left(1 +
\frac{\tau_{i,j} \delta_{k}}{a_k}
\right)
\]
with $\sigma^{\alpha}_k > 0$. Following a
similar argument we have that
\[
\sigma^{\gamma}_k = \cosh^{-1} \left(1
+ \frac{\tau_{i,j} \delta_{k}}{c_k}
\right)
\]
with $\sigma^{\gamma}_k > 0$. Let $\mcE^{d}_{\alpha} :=
\prod_{k=1}^{d} \mcE_{\sigma^{\alpha}_k}$ and $\mcE^{d}_{\gamma} :=
\prod_{k=1}^{d} \mcE_{\sigma^{\gamma}_k}$.  It follows that
\begin{equation}
\begin{split}
  \Real \vartheta &\geq  
  \sum_{k=1}^{d} \theta_k
  \Real z^2_k 
  \geq
\sum_{k=1}^{d} \theta_k 
  \left(
  \tau_{i,j}^2 (1 - 4 \delta_{k}^2) - \frac{\tau_{i,j}
    \delta_{k}}{2} \right)
  > 0.
  \end{split}
\label{errorestimates:eqn8}
\end{equation}
Thus there exist an analytic extension of $\phi(r;\btheta):\Gamma^d
\times \Gamma^d \rightarrow \R$ on $\mcE^{d}_{\alpha} \times
\mcE^{d}_{\gamma}$.



\begin{figure*}[htb!]
\begin{center}
\begin{tikzpicture}
    \begin{scope}[font=\scriptsize]

    \draw [->] (-2.5, 0) -- (2.5, 0) node [above left] {$\Real $};
    \draw [->] (0,-1.5) -- (0,1.5) node [below right] {$\Imag$};
    \draw [-,dashed] (-2,-1.5) -- (-2,1.5);
    \draw [-,very thick] (-1,0) -- (1,0);
    \draw [-,dashed] (2,-1.5) -- (2,1.5);
    \draw (1,-3pt) -- (1,3pt) node [above] {$-1$};
    \draw (-1,-3pt) -- (-1,3pt) node [above] {$1$};


    \draw [-,dashed] (-2,1) -- (2,1);
    \draw [-,dashed] (-2,-1) -- (2,-1);
    
 
    \fill [opacity=0.2, pattern=north west lines, pattern color=red]
    (-2,-1) rectangle (-1.5,1);

    \fill [opacity=0.2, pattern=north west lines, pattern color=red]
    (1.5,-1) rectangle (2,1);
    \end{scope}
    
    \node [below right] at (-1.92,0) {$\frac{\tau_{i,j} \delta_{k}}{a_k}$};
    \node [below right] at ( 1.05,0) {$\frac{\tau_{i,j} \delta_{k}}{a_k}$};

    \node at (0,-1.8) {$(a)$};
    \node at (6,-1.8) {$(b)$};
    
    \begin{scope}[shift={(0,0)},font=\scriptsize]

      \filldraw[fill={rgb:red,143;green,188;blue,143},semitransparent]
      (6,0) ellipse (2 and 1);

    \draw [->] (3, 0) -- (9, 0) node [above left] {$\Real $};
    \draw [->] (6,-1.5) -- (6,1.5) node [below right] {$\Imag$};
    \draw (5,-3pt) -- (5,3pt)   node [above] {$1$};
    \draw (7,-3pt) -- (7,3pt) node [above] {$-1$};


    %\fill [opacity=0.2, pattern=north west lines, pattern color=red]
    %(-2,-1.5) rectangle (-1.5,1.5);
    \end{scope}
    
    \node [below right] at (7.50,1.25) {$\mcE_{\sigma^{\alpha}_k}$};
    \node [below right] at (7.25,0.05) {$\frac{e^{\sigma^{\alpha}_k}
                                       + e^{-\sigma^{\alpha}_k}}{2}$};
        
\end{tikzpicture}
\end{center}
\caption{(a) Region of Complex extension of $\alpha_k$.  (b) Embedding
  of Bernstein ellipse $\mcE_{\sigma^{\alpha}_k}$.}
\label{analyticity:fig1}
\end{figure*}

The next step is to bound the absolute value of the Mat\'{e}rn
covariance function $|\phi(z;\btheta)|$ in $\mcE^{d}_{\alpha} \times
\mcE^{d}_{\gamma}$. If $\nu > \frac{1}{2}$ and $\Real z>0$ then the
modified Bessel function of the second kind satisfies the following
identity
\[
\begin{split}
K_{\nu}(\sqrt{2 \nu}z) &= \frac{\sqrt{\pi} (\sqrt{2 \nu}z)^{\nu}}{2^\nu
  \Gamma(\nu + \frac{1}{2})} \\
&
\int_{1}^{\infty} (t^2 - 1)^{\nu -
  \frac{1}{2}} \exp{(-\sqrt{2 \nu}zt)}\, \text{d}t.
\end{split}
\]
It is not hard to show that for $\nu > \frac{1}{2}$ and $\Real z >0$,
we have that $|K_{\nu}(\sqrt{2 \nu}z)| \leq \frac{|\sqrt{2
    \nu}z|^{\nu}}{(\Real \sqrt{2 \nu}z)^{\nu}} K_{\nu}(\sqrt{2 \nu}
\Real z)$.

Note that $r_{\vartheta} \geq \Real \vartheta > 0$.  From equation
\eqref{errorestimates:eqn4} we have that $\Imag \vartheta =
\sum_{k=1}^{d} \theta_k \Imag z^2_k \leq \sum_{k=1}^{d} 2 \tau
\delta_{k} + 4 \tau^2 \delta_{k}^2$.  From equation
\eqref{errorestimates:eqn8}
\[
\begin{split}
  |\theta_{\vartheta}|
  &\leq
\xi(\btheta,\bdelta,\tau_{i,j}) := \tan^{-1}
\left(
\frac{
  \sum_{k=1}^{d} 2 \tau_{i,j}
\delta_{k} + 4 \tau^2_{i,j} \delta_{k}^2
}
{
\sum_{k=1}^{d} \theta_k 
  \left(
  \tau_{i,j}^2 (1 - 4 \delta_{k}^2) - \frac{\tau_{i,j}
    \delta_{k}}{2} \right)
}
\right)
< \frac{\pi}{2}.
\end{split}
\]
Since $K_{\nu}(\cdot)$ is strictly completely monotonic
\cite{Baricz2011} then
\begin{equation}
\begin{split}
  |K_{\nu}(\sqrt{2 \nu}\Real z)| &=
  |K_{\nu}\ (\sqrt{2 \nu}
  r_{\vartheta} \cos(\theta_{\vartheta}/2))
  | 
  \leq
  \Big| K_{\nu}\Big(\sqrt{\frac{\nu}{2}}
  \cos(\xi(\btheta,\bdelta,\tau)/2) \\
  &
  \sum_{k=1}^{d} \theta_k 
  \Big(
  \tau_{i,j}^2 (1 - 4 \delta_{k}^2) - \frac{\tau_{i,j}
    \delta_{k}}{2} \Big)
  \Big) \Big|.
\end{split}
\label{errorestimates:eqn5}
\end{equation}
From equations \eqref{errorestimates:eqn6}
\eqref{errorestimates:eqn4} 
\[
\begin{split}
|z_k|^{2} &\leq |\Real z^2_k| + |\Imag z^2_k|
\leq \mcR(\delta_k,\tau_{i,j})
:=
1 + \frac{9}{2}\tau_{i,j} \delta_{k} + 5 \tau^2_{i,j} \delta_{k}^2
\end{split}
\]
and therefore
\begin{equation}
\begin{split}
  |z|
  &\leq
\left|
\sum_{k=1}^{d} \theta_{k} z_k^{2} \right|^{\frac{1}{2}}
\leq
\left(
\sum_{k=1}^{d} \theta_{k} |z_k|^{2} \right)^{\frac{1}{2}}
\leq \left( \sum_{k=1}^{d} \theta_k \mcR(\delta_k,\tau_{i,j})
\right)^{\frac{1}{2}}.
\end{split}
\label{errorestimates:eqn7}
\end{equation}
By combining equations \eqref{errorestimates:eqn4},
\eqref{errorestimates:eqn8}, \eqref{errorestimates:eqn5}, and
\eqref{errorestimates:eqn7}, we have now proven the following Theorem.

\begin{theorem} For any two cells $B^{i}_{m}$ and $B^{j}_{q}$
with the associated distance criterion parameter $\tau_{i,j} \geq 0$
let $\phi(\balpha,\bgamma;$ $\btheta):\Gamma^d \times \Gamma^d
\rightarrow \R$ be the pullback of the Mat\'{e}rn covariance function
$\phi(\bx;\by;\btheta):\mcX^{i}_m \times \mcY^{j}_{q} \rightarrow
\R$. Then there exists an analytic extension of
$\phi(\balpha,\bgamma;\btheta):\Gamma^d \times \Gamma^d \rightarrow
\R$ on the polyellipse $ \mcE^{d}_{\alpha} \times \mcE^{d}_{\gamma} $
and
  \[
  |\phi(\cdot,\cdot;\btheta)| \leq
\frac{  \left( 2 \nu \sum_{k=1}^{d} \theta_k \mcR(\delta_k,\tau_{i,j})
\right)^{\frac{\nu}{2}}
|K_{\nu}
(\Xi(\btheta,\bdelta,\tau_{i,j}
)|}{
\Xi(\btheta,\bdelta,\tau_{i,j})^{\nu}
}
\]
on $\mcE^{d}_{\alpha} \times \mcE^{d}_{\gamma}$, where
\[
\begin{split}
  \Xi(\btheta,\bdelta,\tau_{i,j})
  :=
\Big| K_{\nu}\Big(\sqrt{\frac{\nu}{2}}
  \cos(\xi(\btheta,\bdelta,\tau_{i,j})/2)
\sum_{k=1}^{d} \theta_k 
  \Big(
  \tau_{i,j}^2 (1 - 4 \delta_{k}^2) - \frac{\tau_{i,j}
    \delta_{k}}{2} \Big)
  \Big) \Big|.
  \end{split}
\]


\end{theorem}

%% Numerical Results --------------------------------------------------
%%
\section{Numerical results}
\label{numericalresults}

The performance of the multilevel solver for estimation and
prediction formed from random datasets is tested. The results show
that the computational burden is significantly reduced while retaining
good accuracy. In particular, it is possible to now solve
ill-conditioned problems efficiently. The implementation of the code
is as follows:


\begin{enumerate}[i)]

\item {\bf Matlab, C/C++ and MKL:} The binary tree, multilevel basis
  construction, formation of the sparse matrix $\tilde \bC_{\bW}$,
  estimation and prediction components are written and executed on
  Matlab \cite{Matlab2016}. However, the computational bottlenecks are
  executed by C/C++ software packages, Intel MKL \cite{intelmkl}, and
  the highly optimized BLAS and LAPACK packages contained in
  MATLAB. The C/C++ interfaces to matlab are constructed as dynamic
  shared libraries.

  

\item {\bf Direct and fast summation:} The matlab code estimates the
  computational cost between the direct and fast summation methods and
  chooses the most efficient approach.  For the direct method a
  combination of Graphic Processing Unit (GPU) and MKL intel libraries
  are used. For the fast summation method the KIFMM ($d = 3$) c++ code
  is used.  The KIFMM is modified to include a Hermite interpolant
  approximation of the Mat\'{e}rn covariance function, which is
  implemented with the intel MKL package \cite{intelmkl} (see
  \cite{Castrillon2015} for details).


\item {\bf Dynamic shared libraries:} These are produced with the GNU
  gcc/g++ packages. These libraries implement the Hermite interpolant
  with the intel MKL package (about 10 times faster than Matlab
  Mat\'{e}rn interpolant) and link the MATLAB code to the KIFMM.

\item {\bf Cholesky and determinant computation:} The Suite Sparse
  4.2.1 package
  (\cite{Chen2008,Davis2009,Davis2005,Davis2001,Davis1999}) is used
  for the determinant computation of the sparse matrix $\tilde
  \bC_{\bW}(\btheta)$.

\end{enumerate}

The code is tested on a single CPU (4 core Intel i7-3770 CPU @
3.40GHz.), one Nvidia 970 GTX GPU, with Linux Ubuntu 18.04 and 32 GB
memory. In addition, the Boston University Shared Computing Cluster
was used to generate test data.  To test the effectiveness of the
Multilevel solver the following data sets are generated:
\begin{enumerate}[a)]


\item {\bf Random n-sphere data set:} The set of nested random
  observation $\bS_{0}^{d} \subset \dots \subset \bS_{9}^{d}$ vary
  from 1,000, 2000, 4000 to 256,000 knots generated on the n-sphere
  $\bS_{d-1} := \{\bx \in \R^{d}\,\,|\,\,\|\bx\|_{2} = 1 \}$.

\item {\bf Random hypercube data set:} The set of random observation
  locations $\bC_{0}^{d},$ $\dots, \bC_{10}^{d}$ vary from 1,000, 2000,
  4,000 to 512,000 knots generated on the hypercube $[-1,1]^{d}$ for
  $d$ dimensions.  The observations locations are also nested,
  i.e. $\bC_{0}^{d} \subset \dots \subset \bC_{10}^{d}$.
  
\item {\bf Normal test data set} The set of observations values
  $\bZ^d_{0}$, $\bZ^d_{1}$, \dots $\bZ^d_{5}$ are formed from the
  Gaussian random field model \eqref{Introduction:noisemodel} for
  1,000, 2,000, $\dots$ $256,000$ observation locations. The data set
  $\bZ^d_{n}$ is generated from the set of nodes $\bS^{d}_{n}$, with
  the covariance parameters $(\nu,\rho)$ and the corresponding set of
  monomials $\mcQ^d_w$. The Boston University Shared Computing Cluster
  was used to generate the normal test data.
    
%\item All the numerical test are done assuming the $\tilde p = p$.
  
\end{enumerate}


\begin{remark}
 All the timings for the numerical tests are given in wall clock
 times i.e. the actual time is needed to solve a problem. This is to
 distinguish from CPU time, which can be significantly smaller.
  \end{remark}


\subsection{Numerical stability and sparsity of the covariance multilevel
  matrix}

For many practical cases the covariance matrix $\bC(\btheta)$ becomes
increasingly ill-conditioned for the Mat\'{e}rn covariance function as
$\rho$, $\nu$ and the number of observations are increased. This leads
to instability of the numerical solver. It is now shown how effective
Theorem \ref{Multilevelapproach:theo1} becomes in practice.  In
Figure \ref{numericalresults:fig1} the condition number of the
multilevel covariance matrix $\bC_{\bW}(\btheta)$ is plotted with
respect to the cardinality $p(w,d)$ of $\mcQ^d_w$ for different $w$
levels. The multilevel covariance matrix $\bC_{\bW}(\btheta)$ is built
from the random cube $\bC^{d}_{4}$ or n-sphere $\bS^{d}_{4}$
observations. The covariance function is set to Mat\'{e}rn with $\nu =
1$ and $\rho = 1,10$.  As the plots confirm the covariance matrix
condition number significantly improves with increasing level
$w$. This is in contrast with the large condition numbers of the
original covariance matrix $\bC(\btheta)$.  This is consistent with
Theorem \ref{Multilevelapproach:theo1} and Corollary
\ref{Multilevelapproach:cor1}.


We now focus our attention of the sparsity of the covariance matrix
$\tilde \bC_{\bW}(\btheta)$. In Figure \ref{numericalresults:fig2}(a)
the magnitude of the multilevel covariance matrix
$\bC_{\bW}(\btheta)$ is plotted for $N = 8,000$ observations from the
the n-sphere $\bS^{3}_{3}$ with Mat\'{e}rn covariance parameters $\nu
= 0.5$ (exponential) and $\rho = 10$. Due to the large value of $\rho$
the overlap between the covariance function at the different locations
in $\bS^{3}_{3}$ is quite significant, thus leading to a dense
covariance matrix $\bC(\btheta)$ where the coefficients decay
slowly. This is in contrast to the large number of small entries for
$\bC_{\bW}(\btheta)$, as shown in the histogram in Figure
\ref{numericalresults:fig2}(b). Note that the histogram is in terms of
$\log_{10}$ of the absolute value of the entries of
$\bC_{\bW}(\btheta)$. From the histogram it is observed that almost
all the entries are more than 1000 smaller than the largest entries.
This numerical result is consistent with the sub-exponential decay
rates of Theorem \ref{errorestimates:theorem1}.


\begin{figure*}[htpb]
\begin{center}
\begin{tikzpicture}%[thick,scale=1, every node/.style={scale=1}]
  \node[inner sep=0pt] at (0,0)
  {
  \includegraphics[trim = 120 255 120 255,
    clip,width=4.4in,height=4in]{ConditionGraphsReduced.pdf}
  };
  \node[rotate = 90] at (-5.5,2.7) {$\kappa(\bC_{\bW}(\btheta))$};
  \node[rotate = 90] at (0,2.7) {$\kappa(\bC_{\bW}(\btheta))$};

  \node[rotate = 90] at (-5.5,-2.7) {$\kappa(\bC_{\bW}(\btheta))$};
  \node[rotate = 90] at (0,-2.7) {$\kappa(\bC_{\bW}(\btheta))$};

\node at (-2.6,5.3) {
\begin{tabular}{c}
\small $Cube, d = 5, \bC^5_4, \rho=1$ \\
\small $\kappa(\bC(\btheta))= 1.1 \times 10^{7}$
\end{tabular}
};

\node at (2.9,5.3) {
\begin{tabular}{c}
\small $Cube, d = 5, \bC^5_4, \rho=10$ \\
\small $\kappa(\bC(\btheta))= 2.2 \times 10^{9}$
\end{tabular}
};

  
\node at (-2.6,-0.15)
{
\begin{tabular}{c}
\small $Sphere, d = 5,  \bC^5_4, \rho=1$ \\
\small $\kappa(\bC(\btheta))= 2.6 \times 10^{7}$
\end{tabular}
};

\node at (2.9,-0.15)
{
\begin{tabular}{c}
\small $Sphere, d = 5, \bS^5_4, \rho=10$ \\
\small $\kappa(\bC(\btheta))=7.8 \times 10^{9}$
\end{tabular}
};
\node at (2.9,-5.0)
      {$p$
        };
\node at (-2.6,-5.0)
      {$p$
        };
\end{tikzpicture}
\end{center}
\caption{Condition number of the multilevel covariance matrix
  $\bC_{\bW}(\btheta)$ with respect to the size $p$ of the Total
  Degree (TD) polynomial space. The number of observations corresponds
  to 16,000 nodes generated on a hypercube or n-sphere of dimension $d
  = 5$. The covariance function is chosen to be Mat\'{e}rn with $\nu =
  1$ and $\rho=1,10$.  The condition number of the covariance matrix
  $\bC(\btheta)$ is placed on the top of each subplot for
  comparison. The MB is constructed from a kD-tree.  As expected, as
  $p$ increases with $w$ the condition number of $\bC_{\bW}(\btheta)$
  decreases significantly. This is consistent with Theorem
  \ref{Multilevelapproach:theo1} and Corollary
  \ref{Multilevelapproach:cor1}}
\label{numericalresults:fig1}
\end{figure*}

In Table \ref{numericalresults:table1} sparsity and construction wall
clock times of the sparse matrices $\tilde{\bC}^{i}_{\bW}(\btheta)$,
$i = t, t-1, \dots$, for various values of $i$ are shown.  The
polynomial space of the index set $\mcQ^d_w$ is restricted to TD on a
n-Sphere with $d = 10$ dimensions. The domain decomposition is formed
with a kD-tree. The level of the index set is set to $w = 7$, which
corresponds $p = 1001$. The covariance function is Mat\'{e}rn with
$\nu = 3/4$, $\rho = 3/4$. The distance criterion for each $(i,j)$
multilevel covariance matrix block is set to
\[
\tau_{i,j} := 2^{(t - i)/2}2^{(t - j)/2} \tau,
\]
for $i = 1, \dots, t$ and $j = 1 \dots, t$, where $\tau = 3 \times
10^{-6}$.

The first observation to notice is that all the sparse matrices
$\tilde{\bC}^{i}_{\bW}(\btheta)$, $i = t, t-1, \dots$ {\it are very
  well conditioned, thus numerically stable}. This is in contrast to
the original covariance matrices that are in general poorly
conditioned. The sparsity of $\tilde{\bC}^{i}_{\bW}(\btheta)$ and the
Cholesky factor $\bG$ are shown in columns 7 and 9. The construction
time $t_{con}$ of the $\tilde{\bC}^{i}_{\bW}(\btheta)$ is shown in
column 9. In column 5 $t_{ML}$ is the time required to build the
multilevel basis.  We observe that for large matrices the sparse
matrix $\tilde{\bC}^{i}_{\bW}(\btheta)$ are built efficiently.  It is
noted that the sparse matrices in Table \ref{numericalresults:table1}
are built with a direct summation method due to the dimensionality $d
= 10$.


\setlength{\tabcolsep}{6pt}

\begin{figure*}
\begin{center}
\psfrag{A}[c][t]{\small $\log_{10}(abs(\bC_{\bW}(\btheta)))$}
\begin{tabular}{c c}
\includegraphics[width=2in,height=2in]{MatrixDecay.pdf}
&
\hspace*{0mm}
\raisebox{-2.5mm}[0pt][0pt]{
\psfrag{Hist}[c][t]{\small Histogram of $\log_{10} abs(\bC_{\bW}(\btheta))$}
\psfrag{log10}[c][t]{\tiny $\log_{10} abs(\bC_{\bW}(\btheta))$ (100 bins)}
\includegraphics[width=2.1in,height=2.1in]{HistMatrixDecay.pdf}
} \\
    & \\
(a) & (b)
\end{tabular}
\end{center}
\caption{(a) Magnitude pattern and (b) histogram of $\log_{10}
  abs(\bC_{\bW}(\btheta))$ with 100 bins where $abs(\bC_{\bW}(\btheta))
  \in \R^{(N-p) \times (N-p)}$ is the magnitude of the entries of the
  matrix $\bC_{\bW}(\btheta)$.  The matrix $\bC_{\bW}(\btheta)$ is created
  with $d = 3$ dimensions, $N$ = 8,000 random locations on the sphere
  ($\bS^3_3$), and the Mat\'{e}rn covariance function with $\rho =
  10$, $\nu = 0.5$ (exponential), Total Degree index Set $\Lambda(w)$
  with $w = 4$, and $p = 35$. As observed from (a) and (b) most of
  entries of the matrix $\bC_{\bW}(\btheta)$ are very small.}
%and can thus be safely eliminated
%  without comprosing much accuracy.}
\label{numericalresults:fig2}
\end{figure*}




%% \begin{figure}[htpb]
%% \begin{center}
%% \psfrag{A-------------}[c][c]{\raisebox{0mm}[0pt][0pt]{\tiny $w = 3$}}
%% \psfrag{B-------------}[c][c]{\raisebox{0mm}[0pt][0pt]{\tiny $w = 4$}}
%% \psfrag{C-------------}[c][c]{\raisebox{0mm}[0pt][0pt]{\tiny $w = 5$}}
%% \psfrag{Error}[c][t]{\small Total degree log det relative error}
%% \psfrag{x}[c][t]{\tiny Sparsity}
%% \psfrag{y}[c][t]{\tiny $\frac{
%% |\log{det(\bC^n_{\bW})} - \log{det(\tilde \bC^n_{\bW})}|
%% }{
%% \log{det(\bC^n_{\bW})}}$}
%% \begin{tabular}{c c}
%% \includegraphics[width=3in,height=3in]{./figures/sparsitydecay.eps}
%% &
%% \psfrag{Error}[c][t]{\small Smolyak log det relative error}
%% \psfrag{A-------------}[c][c]{\raisebox{0mm}[0pt][0pt]{\tiny $w = 2$}}
%% \psfrag{B-------------}[c][c]{\raisebox{0mm}[0pt][0pt]{\tiny $w = 3$}}
%% \psfrag{Hist}[c][t]{\small Histogram of $\log_{10}|\bC_{\bW}(\btheta)|$}
%% \psfrag{log10}[c][t]{\tiny $\log_{10}|\bC_{\bW}(\btheta)|$ (100 bins)}
%% \includegraphics[width=3in,height=3in]{./figures/sparsitydecay2.eps} 
%% \\
%% (a) TD, $N$ = 16,000, $d = 5$, kD-tree & 
%% (b) SM, $N$ = 16,000, $d = 10$, kD-tree 
%% \end{tabular}
%% \end{center}
%% \caption{Relative log determinant error $\frac{| \log \det \tilde
%%     \bC_{\bW}(\btheta) - \log \det \bC_{\bW}(\btheta)| |}{|\log \det
%%     \bC_{\bW}(\btheta)|}$ with respect to the sparsity of $\tilde
%%   \bC^{n}_{\bW}$ from Random hyper-sphere data $\bS^{d}_{3}$ with $N$
%%   =16,000 observations, Mat\'{e}rn covariance parameters $\nu = 0.5$,
%%   $\rho = 10$ and binary kD-tree. The index sets $\Lambda(\omega)$ are
%%   chosen from (a) Total Degree and (b) Smolyak index sets.}
%% \label{numericalresults:fig3}
%% \end{figure}


\setlength{\tabcolsep}{7pt}
\begin{table*}[htpb]
  \caption{Sparsity test on the matrices $\tilde{\bC}^{i}_{\bW}$, $i =
    t, t-1, \dots$.  The polynomial space of the index set $\mcQ^d_w$
    is restricted to TD on a n-Sphere with $d = 10$ dimensions. The
    domain decomposition is formed from a kD-tree. The level of the
    index set is $w = 7$, which corresponds $p = 1001$. The kernel
    function is Mat\'{e}rn with $\nu = 3/4$, $\rho = 3/4$ and $\tau :=
    3 \times 10^{-6}$. The first column is the number of random
    n-Sphere nodes. The second is the maximum level of the kD tree and
    $i$ is the level of the sparse matrix $\tilde{\bC}^{i}_{\bW}$. The
    fourth column is the condition number of $\tilde{\bC}^{i}_{\bW}$,
    which is excellent.  The fifth column is the size of the matrix
    $\tilde{\bC}^{i}_{\bW}$.  The seventh column, $t_{ML}$, is the
    total time for the construction of the multilevel basis. The
    eighth column is the sparsity of $\tilde{\bC}^{i}_{\bW}$.  The
    nineth column, $t_{con}$ is the total time for the construction of
    the matrix $\tilde{\bC}^{i}_{\bW}$. The tenth column is the
    sparsity of the Cholesky factor $\bG$ (with nested dissection
    reordering) of the sparse matrix $\tilde{\bC}^{i}_{\bW}$. The last
    column is the total time to compute the Cholesky factor $\bG$.}
\begin{center}
\begin{tabular}{ r r r r c c r r c r c r}
\multicolumn{1}{c}{$N$} &
% \multicolumn{1}{c}{$d$} &
\multicolumn{1}{c}{$t$} &
\multicolumn{1}{c}{$i$} &
\multicolumn{1}{c}{$\kappa(\tilde \bC_{\bW}^{i})$} &
\multicolumn{1}{c}{Size} &
%\multicolumn{1}{c}{$\tau$} &  
\multicolumn{1}{c}{$t_{ML}$} &
\multicolumn{1}{c}{$nz$} &
\multicolumn{1}{c}{$t_{con}$} &
\multicolumn{1}{c}{$nz(\bG)$} &
\multicolumn{1}{c}{$t_{sol}$}
 \\ 
 \hline
32,000 & 4 & 4 &   5  & 15,984 &  46 &  6.3\% &   11 &  3.1\% &  1 \\
32,000 & 4 & 3 &   8  & 23,992 &  46 & 10.4\% &   30 &  5.2\% &  3 \\
32,000 & 4 & 2 &  13  & 27,996 &  46 & 15.6\% &   82 &  7.8\% &  7 \\
32,000 & 4 & 1 &  19  & 29,998 &  46 & 20.1\% &  190 & 10.4\% & 16 \\
32,000 & 4 & 0 &  23  & 30,999 &  46 & 25.7\% &  310 & 13.0\% & 17 \\
\hline
64,000 & 5 & 5 &   6  & 31,968 & 104 &  3.5\% &   21 & 1.8\%  &  3 \\
64,000 & 5 & 4 &  11  & 47,984 & 105 &  6.3\% &   90 & 3.1\%  & 12 \\
64,000 & 5 & 3 &  18  & 55,992 & 106 &  9.6\% &  270 & 5.0\%  & 18 \\
64,000 & 5 & 2 & 121  & 59,996 & 121 & 13.4\% & 624 & 6.7\%  & 34 \\
\hline
128,000 & 6 & 6 &  8  &  63,936 &  237 & 4.0 \% & 120  & 2.1 \% & 15 \\
128,000 & 6 & 5 & 17  &  95,968 &  237 & 5.5 \% & 378 & 6.7 \% & 140 \\
\end{tabular}
\vspace{5mm}
\\
\end{center}
\label{numericalresults:table1}
\end{table*}

\subsection{Estimation}

In this section estimation results are presented for the Mat\'{e}rn
covariance matrix on high dimensional n-Sphere random locations by
solving multilevel log-likelihood
\[
\hat{\btheta} : =
\argmax_{\btheta}
\tilde{\ell}^{i}_{\bW}(\bZ^{i,k,d}_{W};\btheta),
\]
where $\bZ^{i,k,d}_{W} := [\bW_t \T, \dots, \bW_i \T ] \T
\bZ^{d}_{k}$, for $i = t, t-1, \dots$. The observation data $\bZ^d_k$
is built from the n-Sphere $\bS^d_k$ for $d = 3,10$, $k = 6$ $(N =
64,000)$ and $k = 7$ $(N = 128,000)$. The covariance function is
Mat\'{e}rn for several values of $\nu$ and $\rho$. To test the
performance of the multilevel estimator, $M = 100$ realizations are
generated for each case.

The optimization problem of the log-likelihood function
\eqref{Introduction:multilevelloglikelihoodreduced2} is solved using
a fmincon iteration search for the estimates $\hat \nu$ and $\hat
\rho$ from the optimization toolbox in MATLAB \cite{Matlab2016}. The
tolerance level is set to $10^{-6}$.

In Table \ref{numericalresults:table2} the mean and standard deviation
of the Mat\'{e}rn covariance parameter estimates $\hat \nu$ and $\hat
\rho$ are presented. The mean estimate $\bbE_M [\hat{\nu}]$ refers to
the mean of $M$ estimates $\hat{\nu}$ for the $M$ realizations of the
stochastic model. Similarly, $std_M [\hat{\nu}]$ refers to the
standard deviation of the $M$ realizations. For case (a) ($d = 3$) the
error mean and std is $\approx 1\%$. For case (b) ($d = 10$) the error
of the mean increase to $\approx 10 \%$. In general, as $i$ is reduced
from $t$ there is a tendency of a drop in the standard deviation
$std_M [\hat{\nu}]$ of the estimator $\hat \nu$. However, there is
also a tendency for the accuracy of the mean to degrade somewhat,
except for (a) $N = 128,000$, $i = 12 \rightarrow 11$.
%This inconsistent
%behavior in accuracy might be due to the standard deviation being
%approximately the same magnitude as the sample mean.


\begin{table*}[htpb]
\caption{Estimation of parameters $\hat \nu$ and $\hat \rho$ with:
  Total Degree polynomial index set $\mcQ^d_w$, kD tree, and n-Sphere
  with for $d = 3$ and $d = 10$.  The observation data $\bZ^d_k$ are
  formed from the Mat\'{e}rn covariance function. The number of
  realizations of the Gaussian random field model is set to $M =
  100$. Several cases are tested and are given by the individual tables
  (a) and (b).  The first to fourth columns are self-explanatory. The
  fifth column is the mean error of $\hat \nu$ with $M$
  realization. The sixth column is the mean error of $\hat \rho$. The
  last two columns are the standard deviation of $M$ realizations of
  the parameters $\hat \nu$ and $\hat \rho$.}
\begin{center}
(a) TD, kD tree, n-Sphere, $d = 3$, $M = 100$, $\nu =
  3/4$, $\rho = 1/6$, $\tau = 5 \times 10^{-2}$
\begin{tabular}{ r r r r r r r r r r}
\multicolumn{1}{c}{$N$} & 
% \multicolumn{1}{c}{$d$} &
\multicolumn{1}{c}{$w$} &
\multicolumn{1}{c}{$t$} &
\multicolumn{1}{c}{$i$} & 
\multicolumn{1}{c}{$\bbE_M [\hat{\nu} - \nu]$} &
\multicolumn{1}{c}{$\bbE_M [\hat{\rho} - \rho]$} &
\multicolumn{1}{c}{$std_M [\hat{\nu}]$} &
\multicolumn{1}{c}{$std_M [\hat{\rho}]$} 
 \\ 
 \hline
64000 & 3 & 11 & 11 & -1.92e-04 &  4.52e-04 & 1.36e-02 & 8.17e-03 \\ 
64000 & 3 & 11 & 10 &  1.17e-03 & -5.90e-04 & 7.08e-03 & 4.04e-03 \\ 
%\hdashline
128000 & 3 & 12 & 12 & -2.51e-03 & 1.81e-03 & 8.54e-03 & 6.11e-03 \\ 
128000 & 3 & 12 & 11 & -6.90e-04 & 5.02e-04 & 4.17e-03 & 2.84e-03 \\ 
\end{tabular}
\\
\bigskip
(b) TD, kD tree, n-Sphere, $d = 10$, $M = 100$, $\nu =
  3/4$, $\rho = 3/4$, $\tau = 1 \times 10^{-5}$
\begin{tabular}{ r r r r r r r r r r}
\multicolumn{1}{c}{$N$} & 
% \multicolumn{1}{c}{$d$} &
\multicolumn{1}{c}{$w$} &
\multicolumn{1}{c}{$t$} &
\multicolumn{1}{c}{$i$} & 
\multicolumn{1}{c}{$\bbE_M [\hat{\nu} - \nu]$} &
\multicolumn{1}{c}{$\bbE_M [\hat{\rho} - \rho]$} &
\multicolumn{1}{c}{$std_M [\hat{\nu}]$} &
\multicolumn{1}{c}{$std_M [\hat{\rho}]$} 
 \\ 
 \hline
64000 & 4 & 5 & 5 &  8.70e-03 & -1.12e-02 & 1.55e-02 & 1.85e-02 \\ 
64000 & 4 & 5 & 4 & -9.31e-02 &  8.02e-02 & 1.67e-02 & 1.97e-02 \\
%\hdashline
128000 & 4 & 6 & 6 & -6.36e-03 & 5.51e-03 & 2.10e-02 & 1.72e-02 \\
128000 & 4 & 6 & 5 & -7.18e-02 & 6.27e-02 & 1.32e-02 & 1.46e-02 \\ 
\end{tabular}
\\
\bigskip
%% (c) TD, kD tree, n-Sphere, $d = 10$, $M = 100$, $\nu =
%%   1.25$, $\rho = 1$, $\tau = 10^{-7}$
%% \begin{tabular}{ r r r r r r r r r r}
%% \multicolumn{1}{c}{$N$} & 
%% % \multicolumn{1}{c}{$d$} &
%% \multicolumn{1}{c}{$w$} &
%% \multicolumn{1}{c}{$t$} &
%% \multicolumn{1}{c}{$i$} & 
%% \multicolumn{1}{c}{$\bbE_M [\hat{\nu} - \nu]$} &
%% \multicolumn{1}{c}{$\bbE_M [\hat{\rho} - \rho]$} &
%% \multicolumn{1}{c}{$std_M [\hat{\nu}]$} &
%% \multicolumn{1}{c}{$std_M [\hat{\rho}]$} 
%%  \\ 
%%  \hline
%%  64000 & 2 &  9 & 9 & -5.86e-03 & 5.21e-03 & 4.85e-02 & 3.15e-02 \\
%%  64000 & 2 &  9 & 8 & -3.37e-02 & 1.93e-02 & 3.50e-02 & 2.46e-02 \\ 
%%  64000 & 2 &  9 & 7 & -1.19e-01 & 6.92e-02 & 2.88e-02 & 2.51e-02 \\ 
%% %\hdashline
%% 128000 & 2 & 10 & 10 & -2.70e-03 & 2.74e-03 & 3.76e-02 & 2.44e-02 \\ 
%% 128000 & 2 & 10 &  9 & -2.00e-02 & 1.20e-02 & 2.47e-02 & 1.80e-02 \\ 
%% 128000 & 2 & 10 &  8 & -8.40e-02 & 5.03e-02 & 2.21e-02 & 1.89e-02 \\ 
%% \end{tabular}
\end{center}
\label{numericalresults:table2}
\end{table*}

%INFINITUMmpfg
%a5c349c2d6


\subsection{Prediction}

In this the computational performance of the multilevel Kriging solver
is analyzed. Given a fixed Mat\'{e}rn parameters $(\nu,\rho)$ the
objective is to compute the BLUP vectors $\hat \bgamma$ and $\hat
\beta$. This involves solving the system of equations $\bP^{-1}_{\bW}
\bC_{\bW}(\btheta) \bgamma_{\bW} = \bP^{-1}_{\bW} \bZ_{\bW}$ and $\hat
\bbeta = (\bX\T \bX)^{-1}$ $\bX\T(\bZ - \bC \hat \bgamma)$.

Numerical results for computing $\hat \bgamma$ and $\hat \bbeta$ for the
hypercube data set with $d = 3$ dimensions, kD tree, and the Total
Degree index set $\mcQ^d_w$ are shown in Table
\ref{numericalresults:table3}. The Mat\'{e}rn covariance coefficients
$\btheta = (\nu,\rho)$ are set to (3/4,1). The relative error of the
residual of PCG method for the unpreconditioned system is set to
$\varepsilon = 10^{-3}$. The KIFMM is set to high accuracy.

For computing the matrix vector products of the PCG iterations, the
computational break even point of the KIFMM solver is reached for $N
\approx 2,500$ compared to using the direct approach (with CPU and
GPU). The increase in computational complexity is linear with respect
to $N$. Thus all the matrix vector products for the PCG iterations are
calculated using the KIFMM.

The preconditioner $\bP_{\bW}$ is built using a combination of the GPU
and CPU. This leads to a quadratic increase in computational cost with
respect to the number of observations $N$. However, due to the high
efficiency of the implementation and $p = 120$, the break even point
for the use of the KIFMM solver is not reached, even for $N = 512,000$
observation points.

From Table \ref{numericalresults:table3} observe that condition number
of the covariance matrix $\bC$ is much larger compared to
$\bC_{\bW}$. This is already a good indication that solving the
Kriging problem will be more efficient using the multilevel approach.

The number of iterations needed to reach the same accuracy for both
approaches are significantly better with the multilevel approach
i.e. $\approx 70$ times less iterations. However, the computation of
$\bbeta$ with the single level method requires solving $p + 1 = 121$
matrix inversions of $\bC$. This is in contrast with a single matrix
inversion of $\bC_{\bW}$ with the multilevel method. In practice, we
did not solve all $p+1$ matrix inversions for the single level
approach, but measure the time required to compute a single matrix
inversion and multiplied it 121 to obtain the estimated time
complexity.  For $N = 64,000$ observations we observe efficiencies of
$\approx 7,000$ compared to the single level iterative approach.

The condition number of the covariance matrices are fairly large,
making this problem somewhat hard to solve numerically.  The results
show that 512,000 size problems with good accuracy are feasible with a
single 4-core processor and GPU. Finally, the total computational cost
varies somewhere between linear and quadratic as the number of
observations $N$ is increased.


\begin{table*}[htbp]
  \caption{Numerical results for computing $\bP^{-1}_{\bW}
    \bC_{\bW}(\btheta) \bgamma_{\bW} = \bP^{-1}_{\bW} \bZ_{\bW}$ and
    $\hat \bbeta = (\bX\T \bX)^{-1} \bX\T(\bZ - \bC \hat \bgamma)$ for
    the hypercube data set with $d = 3$ and the Total Degree index set
    $Q^d_w$. The Mat\'{e}rn covariance coefficients $\btheta =
    (\nu,\rho)$ are set to (3/4,1). The relative error of the residual
    of PCG method for the \emph{unpreconditioned system} is set to
    $\varepsilon = 10^{-3}$. The KIFMM is set to high accuracy. (a)
    The second column of is the condition number of the covariance
    matrix $\bC$, up to $N=64,000$ observations, and is compared with
    the fourth column which corresponds to the condition number of
    $\bC_W$. The third column (itr($\bC_{\bW}$)) is the number of CG
    iterations needed for convergence for $10^{-3}$ residual
    accuracy. The fifth column is the number of iterations need to
    achieve the residual error $10^{-3}$ for the unpreconditioned
    system with the preconditioner $\bP_{\bW}$.  (b) The second column
    corresponds to the wall clock times in seconds for the
    preconditioner computation. The third column is the time for
    construction of the preconditioner $\bP_{\bW}$.  The PCG iteration
    wall clock timings for $\bC_{\bW}$, by using a KIFMM, are given in
    the fourth column. The fifth is the total time to compute
    $\bgamma_{\bW}$, $\bbeta$ and the multilevel basis
    construction. The sixth column is the computational efficiency for
    computing $\bgamma_{\bW}$ vs $\bC^{-1} \bZ$ to same residual
    accuracy with respect to the number of iterations. The last column
    is the estimated efficiency of computing $\hat \bgamma$ and $\hat
    \bbeta$ with the multilevel BLUP compared to the single level
    approach, equation \eqref{Kriging}, to approximately the same
    accuracy using a CG iteration with the KIFMM. We observe the
    significant speed ups ($\approx 7,000$ for $N = 64,000$) for
    calculating the BLUP by using the multilevel approach.
  %We observe that the total computational cost varies between linear
  %and somewhere between linear and quadratic as the number of
  %observations $N$ is increased.  The last column shows the efficiency
  %for computing itr($\bC_{\bW}$) with respect to $\bC$. Notice that as
  %$N$ increases the efficiency of the multilevel method increases
  %significantly.
  %The last column shows the efficiency for computing itr($\bC_{\bW}$
  %with respect to $\bC$. Notice that as $N$ increases the efficiency
  %of the multilevel method increases significantly.
}
\begin{center}
  \bigskip
  (a) $\btheta = (3/4,1)$, $d = 3$, $w = 7$ ($p = 120$) \\
  \bigskip
\begin{tabular} { r c r c r}
  \multicolumn{1}{c}{$N$} &
  \multicolumn{1}{c}{$\kappa(\bC)$} &
    \multicolumn{1}{c}{itr($\bC$)} &
  \multicolumn{1}{c}{$\kappa(\bC_{\bW})$} &
  itr($\bC_{\bW}$) \\
  \hline
  8,000  & $3.2 \times 10^{7}$   &    1,985 & $1.8 \times 10^{4}$ &   52  \\  
  16,000  & $1.1 \times 10^{8}$  &    3,511 & $6.0 \times 10^{4}$ &   67  \\  
  32,000  & $5.6 \times 10^{8}$  &    8,259 & $3.1 \times 10^{5}$ &  116  \\  
  64,000  &  $1.8 \times 10^{9}$ &   12,680 & $9.5 \times 10^{5}$  & 165  \\  
  128,000 &                 -   &      -  &                  -  &  308   \\  
  256,000 &                 -   &      -  &                  -  &  292   \\  
  512,000 &                 -   &      -  &                  -  &  484   \\  
\end{tabular}
\\
\bigskip
(b) $\btheta = (3/4,1)$, $d = 3$, $w = 7$ ($p = 120$) \\
\bigskip
\begin{tabular} { r r r r r c r}    
  \multicolumn{1}{c}{$N$} 
  & itr($\bC_{\bW}$) & $\bP_{\bW}$ (s) & Itr (s) & Total (s) & Eff$_{\bgamma}$ &
  \multicolumn{1}{c}{Eff$_{\bgamma,\bbeta}$} \\
  \hline
  8,000   &   52  &       4   &      29  &     38 &  38 &   3,600  \\
  16,000  &   67  &      13   &      98  &    118 &  52 &   5,000  \\
  32,000  &  116  &      45   &     260  &    317 &  71 &   7,250  \\
  64,000  &  165  &     178   &     798  &    997 &  76 &   7,380  \\
  128,000 &  308  &     713   &   3,934  &  4,687 &   - &    - \\
  256,000 &  292  &   2,837   &   5,745  &  8,663 &   - &    - \\
  512,000 &  484  &   11,392  &  20,637  & 32,202 &   - &    - \\
\end{tabular}
%% \bigskip
%% (b) $\btheta = (3/4,1)$, $d = 3$, $w = 7$ ($p = 120$) \\
%% \bigskip
%% \begin{tabular} { r r r r r c r}    
%%   \multicolumn{1}{c}{$N$} 
%%   & itr($\bC_{\bW}$) & $\bP_{\bW}$ (s) & Itr (s) & Total (s) & Eff$_{\bgamma}$ &
%%   \multicolumn{1}{c}{Eff$_{\bgamma,\bbeta}$} \\
%%   \hline
%%   8,000   &   53  &       3   &      18  &     21 &  48 &  5,808 \\
%%   16,000  &   70  &      11   &      43  &     54 &  89 & 10,769 \\
%%   32,000  &  116  &      45   &     265  &    310 &  58 &  7,018 \\
%%   64,000  &  165  &     178   &     798  &    976 &  76 &  9,196\\
%%   128,000 &  308  &     713   &   3,934  &  4,647 &   - &    - \\
%%   256,000 &  292  &   2,837   &   5,745  &  8,582 &   - &    - \\
%%   512,000 &  484  &   11,392  &  20,319  & 31,711 &   - &    - \\
%% \end{tabular}
\end{center}
\label{numericalresults:table3}
\end{table*}


The multilevel approach is now tested on $d = 20$ and $d = 25$
dimensional problems. Due to the high dimensionality of these
problems, a fast summation approach is not an option. The
matrix-vector products of each iteration are computed with the direct
approach using the GPU and CPU.

In Table \ref{numericalresults:table4}(a) the numerical results for
computing $\bgamma$ and $\bbeta$ for $d = 20$ and $\theta =(5/4,10)$.
Compared to the single level iterative approach the multilevel method
is approximately 42,000 faster for $N = 64,000$ observations. Similar
results are obtained shown in Table \ref{numericalresults:table4}(b). 
for $d = 25$ and $\theta =(5/4,10)$.

\setlength{\tabcolsep}{3pt}

\begin{table*}[htbp]
  \caption{Computing Kriging for the n-sphere data set with $d = 20$
    and $d = 25$ dimensions, TD index set, and Mat\'{e}rn covariance
    function without pre-conditioner. The residual accuracy is set to
    $\varepsilon = 10^{-3}$. Since the dimension is greater than 3,
    the matrix vector products are computed directly with the GPU and
    CPU.  The description of the columns of tables (a) and (b) are the
    same as for Table \ref{numericalresults:table3}. In addition,
    column 6 corresponds to the wall clock time for computing the
    multilevel basis. (a) Computational times for solving the Kriging
    prediction for $d = 20$ and $\theta = (5/4,10)$.  The growth in
    computational cost is slightly faster than quadratic due to the
    lack of fast summation method in higher dimensions. However,
    compared to the single level iterative approach it is
    approximately 42,000 faster for $N = 64,000$ observations. (b)
    Kriging prediction for $d = 25$ and $\theta = (5/4,10)$. The
    growth in computational cost is similar.  The efficiency of this
    method is about 2,840 times faster for $N = 64,000$ observations.}
\begin{center}
(a) $\btheta = (\nu,\rho) = (5/4,10)$, $d = 20$, $w = 3$ ($p =
1771$), No precond., Direct \\
\begin{tabular} { r c c c  c r r r r r}
  \multicolumn{1}{c}{$N$} & $\kappa (\bC)$ & $\kappa (\bC_{\bW})$ &
 itr($\bC$)
  & 
 itr($\bC_{\bW}$) &  MB(s) & Itr(s) & Total(s) &  Eff$_{\bgamma,\bbeta}$ \\
  \hline
 16,000  & $5  \times 10^{7}$ & 7  & 238 & 10  &  52  &      97   &    153 & 26,700   \\
 32,000  & $1 \times 10^{8}$  & 11 & 324 & 13  & 121  &     500   &    628 & 35,160   \\
 64,000  & $2 \times 10^{8}$  & 17 & 444 & 17  & 284  &   2,600   &  2,898 & 42,050 \\
128,000  &  -                & -  &  - & 22  & 628  &  13,494   &  14,153 & -  \\
\end{tabular}\\
\bigskip
(b) $\btheta = (\nu,\rho) = (3/4,10)$, $d = 25$, $w = 2$ ($p = 351$), No precond., Direct \\
\begin{tabular} { r c c c c r r r r}
  \multicolumn{1}{c}{$N$} & $\kappa (\bC)$ & $\kappa (\bC_{\bW})$
  & itr($\bC$)
  & itr($\bC_{\bW}$) & MB(s) & Itr(s) & Total(s) & Eff$_{\bgamma,\bbeta}$ \\
  \hline
 16,000  & $2  \times 10^{6}$  &   7  & 86  & 12 &   5  &         116   &    122 & 2,400   \\
 32,000  &  $4   \times 10^{6}$ &  12 & 109 & 15 &  13  &         582   &    599 & 2,490 \\
 64,000  &  $9  \times 10^{6}$ &   21 & 147  & 18 &  30  &       2,788  &  2,821 & 2,840 \\
 128,000  &  -                  &  - & -  & 25 &  79  &      15,557   & 15,641 &  - \\
 256,000  &  -                  &  - & -  & 33 & 157  &      83,163   & 83,337 &  - \\
\end{tabular}\\
\end{center}
\label{numericalresults:table4}
\end{table*}

\section{Conclusions}

In this paper a multilevel Kriging method is developed that scales
well with high dimensions. A multilevel basis is constructed from a
kD-tree and for the choice of Total Degree polynomial basis
$\mcQ^d_w$.  The approach described in the paper has the following
characteristics and advantages:

\begin{enumerate}[i)]

\item The multilevel method is numerically stable. Hard estimation and
  prediction of large dimensional problems are now feasible.
  
\item The method is efficiently implemented by using a combination of
  MATLAB, c++ software packages and dynamic libraries.

\item Sub-exponential decay of multilevel covariance matrix
  $\bC_{\bW}$ is proven based on complex analytic extensions.
  
\item Numerical results of up to 25 dimensional problems. These
  problems are difficult to solve with traditional methods due to the
  large condition numbers, but feasible with the multilevel method.

\item The multilevel prediction approach is proven to be \emph{exact},
  in the sense that single level and multilevel prediction
  formulations are shown to be equivalent. 
   

\item The efficiency of this approach will be further improved as high
  dimensional fast summation methods are developed.

\item An A-posteriori scheme and estimates for constructing the sparse
  covariance matrix $\tilde \bC$ will be developed in a future
  paper. This will be possible with the error bounds for the entries
  of $\bC$ derived in this paper since all the constants can be
  estimated.

  
\end{enumerate}














\section*{Appendix: Polynomial Interpolation}
\label{PolynomialAppendix}

In this section we provide some background on polynomial interpolation
in high dimensions. This will be critical to estimate the decay rates
of the entries of the multilevel covariance matrix for high
dimensional problems.
%Note that this appendix can be
%  skipped as it is only used for estimating the decay of the
%  multilevel covariance matrix.}

The decay of the coefficients will directly depend on the analytic
properties of the covariance function. The traditional error estimates
of polynomial interpolation are based on multi-variate $m^{th}$ order
derivatives. However, for many cases, such as the Mat\'{e}rn
covariance function, the derivatives are too complex or expensive to
manipulate for even a moderate number of dimensions. This motivates
the study of polynomial numerical approximations based on complex
analytic extensions, which are much better suited for high dimensions.
Much of the discussion that follows has it roots in the field of
uncertainty quantification and high dimensional interpolation
\cite{nobile2008a,Castrillon2016,Griebel2016}
for partial differential
equations.


Consider the problem of approximating a function $v: \Gamma^{d}
\rightarrow \R$ on the domain $\Gamma^{d}$.  Without loss of
generality let $\Gamma : = [-1, 1]$ and $\Gamma^{d} := \prod_{n =
  1}^{d} \Gamma$. Suppose that $\mcG \subset \Gamma^{d}$, then define
the following spaces
\[
\begin{split}
  &
L^q(\mcG) := \{ v(\by)\, | \, \int_{\mcG} v(\by)^q \text{d}
\by < \infty  \}
\,\,\,
\mbox{and} \\
&
L^{\infty}(\mcG) := \{ v(\by)\, | \, \sup_{\by \in \mcG} |v(\by)|
< \infty  \}.
\end{split}
\]


Suppose that $\mcP_{ q}(\Gamma):=\text{\rm span}\{y^k,\,k=0,\dots,q\}$
i.e. the space of polynomials of degree at most $q$. Let $\mcI^{m} :
C^{0}(\Gamma) \rightarrow \mcP_{m-1}(\Gamma)$ be the univariate
Lagrange interpolant
\[
\mcI_{m}(v(\by)):=
\sum_{k=1}^{m}v(y^{(k)})l_{m,k}(y^{(k)}),
\]
where $y^{(1)}, \dots, y^{(m)}$ is a set of distinct knots on $\Gamma$
and $\{ l_{n,k} \}_{k=0}^{m}$ is a Lagrange basis of the space
$\mcP_{m-1}(\Gamma)$. The variable $m \in \Nset$
%, where $\Nset_{+} := \Nset \cup 0$,
corresponds to the order of approximation of the
Lagrange interpolant. However, for the case of the zero order
interpolation $m = 0$ corresponds to $\mcI_{0} = 0$.


\begin{remark}
For high dimensional interpolation the particular set of points
$y^{(1)}, \dots, y^{(m)}$ that we will use is the Clenshaw-Curtis
abscissas.  This is further discussed in this section. However, for
now, we assume that the points are only distinct.
  \end{remark}


For $m \geq 1$ let
\[
\Delta_{m}
:= \mcI_{m}-\mcI_{m-1},
\]
From the difference operator $\Delta_{m}$ we can readily observe that
$\mcI_{m} = \sum_{k=1}^{m} \Delta_{k}$, which is reminiscent of multi
resolution wavelet decompositions. The idea is to represent
multivariate approximation as a summation of the difference operators.

Consider the multi-index tupple $\bm = (m_1,\dots,m_d)$, where $\bm
\in \Nset^{d}$, and form the tensor product operator
$\mcS_{w,d}: \Gamma \rightarrow \R$ as
\begin{equation}
  \mcS_{w,d}
      [v(\by)]
      :
      =
 \sum_{\bm \in \bbNset^{d}: \sum_{i=1}^{d} m_i - 1  \leq w } \;\;
 \bigotimes_{n=1}^{d} {\Delta^{n}_{m_n}}(v(\by)).
\label{errorestimates:SG}
\end{equation}
Note that by ${\Delta^{n}_{m_n}}(v(\by))$ we mean that the difference
operator ${\Delta_{m_n}}$ is applied along the $n^{th}$ dimension in
$\Gamma$.


Let $C^{0}(\Gamma_d; \R) : = \{ v: \Gamma_d \rightarrow \R\,\,$ is
continuous on $\Gamma_d$ and $\max_{\by\in \Gamma_d} |v(\by)| < \infty
\}$.  From Proposition 1 in \cite{Back2011} it is shown that for any
$v \in C^0(\Gamma_d;\R)$, we have $\mcS_{w,d}[v]\in \mcQ^{d}_{w}$.
Moreover, $\mcS_{w,d}[v] = v$, for all $v \in \mcQ^{d}_{w}$. The key
observation to take away is that the operator $\mcS_{w,d}[v]$ is
\textit{exact} in the space of polynomials $\mcQ^{d}_{w}$. This will
be useful in connecting the Lagrange interpolant with Chebyshev
polynomials.


Let $T_k:\Gamma \rightarrow \R$, $k = 0, 1, \dots$, be a Chebyshev
polynomial over $\Gamma$, which are defined recursively as follows:
$T_0(y) = 1$, $T_1(y) = y$, $\dots$, $T_{k+1}(y) = 2yT_{k}(y) -
T_{k-1}(y)$, $\dots$, where $y \in \Gamma$. Chebyshev polynomials are
well suited for the approximation of functions with analytic
extensions on a complex region bounded by a Bernstein ellipse. They
bypassing the need of using derivative information and sharp bounds on
the error are readily available. Suppose that $\sigma > 0$ and denote
by
\[
\begin{split}
  \mcE_{\sigma} := \Big\{
  z \in \bbC, \sigma \geq
\delta \geq 0 ;\,\Real{z} = \frac{e^{\delta} + e^{-\delta}
}{2}cos(\theta) 
\Imag{z} = \frac{e^{\delta} 
  - e^{-\delta}}{2}sin(\theta),
\theta \in [0,2\pi)
  \Big\}
\end{split}
  \]
as the region bounded by a Bernstein ellipse (see Figure
\ref{erroranalysis:sparsegrid:polyellipse}).

The following theorem is based on complex analytic extensions on
$\mcE_{\sigma}$ and provides a control for the Chebyshev polynomial
approximation.

\begin{theorem}
Suppose that for $u:\Gamma \rightarrow \R$ there exists an analytic
extension on $\mcE_{\sigma}$. If $|u| \leq M < \infty$ on
$\mcE_{\sigma}$ then there exists a sequence of coefficients
$|\alpha_k| \leq M / e^{k\sigma}$ such that $u \equiv \alpha_0 +
2\sum_{k = 1}^{\infty} \alpha_{k} T_{k}$ on $\mcE_{\sigma}$. Moreover,
if $y \in \Gamma$ then
\[
%\begin{multline*}
%\shoveright{|q(y) - \alpha_0  - 2\sum_{k = 1}^{n} \alpha_{k} T_{k}(y)|
%\leq 
%\frac{2M}{e^{\sigma} - 1} e^{-n \sigma}.}
|q(y) - \alpha_0  - 2\sum_{k = 1}^{n} \alpha_{k} T_{k}(y)|
\leq 
\frac{2M}{e^{\sigma} - 1} e^{-n \sigma}.
%\end{multline*}
\]
\label{errorestimates:theorem}
\end{theorem}
\begin{proof}
See Theorem 2.25 in \cite{Khoromskij2018}
\end{proof}


\begin{figure}[htb]%[12]{r}{7cm}%[htp]
\begin{center}
\begin{tikzpicture}
    \begin{scope}[font=\scriptsize]

      
      \filldraw[fill=blue!20,
      semitransparent] (0,0) ellipse (2 and 1);

    \draw [->] (-2.5, 0) -- (2.5, 0) node [below left]  {$\Real $};
    \draw [->] (0,-1.5) -- (0,1.5) node [below left] {$\Imag$};
    \draw (1,-3pt) -- (1,3pt)   node [above] {$1$};
    \draw (-1,-3pt) -- (-1,3pt) node [above] {$-1$};
    \end{scope}
    
    \node [below right] at (-2.5,1.25) {$\mcE_{\sigma}$};

    \node [] at (0.75,1.25) {$\frac{e^{
          \sigma} - e^{- \sigma}}{2}$};

    
    \node [] at (2.75,0.25) {$\frac{e^{
      \sigma} + e^{- \sigma}}{2}$}; 
    
\end{tikzpicture}
\end{center}
\caption{Complex region bounded by the Bernstein ellipse.}
\label{erroranalysis:sparsegrid:polyellipse}
\end{figure}

We can now connect the error due to the Lagrange interpolation with
Chebyshev expansions. It is known that if $u \in C(\Gamma,\R)$ then
\[
\|(I - \mcI_{m})u\|_{L^{\infty}(\Gamma)} \leq
(1 + \Lambda_{m})
\min_{h \in \mcP_{m-1}} \| u - h \|_{L^{\infty}(\Gamma)},
\]
where $\Lambda_{m}$ is the Lebesgue constant (See Lemma 7 in
\cite{babusk_nobile_temp_10}). Note that $I:C^{d}(\Xi;\R) \rightarrow
C^{d}(\Xi;\R)$ refers to the identity operator and the domain $\Xi$ is
taken from context. For the previous case $\Xi = \Gamma$.  Bounds on
$\Lambda_{m}$ are known in the context of the location of the knots
$y^{(1)}, \dots, y^{(m)} \in \Gamma$. In this article we restrict our
attention to Clenshaw-Curtis abscissas
%\[
\[
y^{(j)} = -\cos \left( \frac{\pi(j-1)}{m - 1} \right),\,\, j =
1,\dots, m
\]
%\]
and $\Lambda_m$ is bounded by $2\pi^{-1}(\log{(m-1)} + 1) \leq 2m - 1$
(see \cite{babusk_nobile_temp_10}).  Since the interpolation operator
$\mcI_{m}$ is exact on $\mcP_{m - 1}$, then if $u:\Gamma \rightarrow
\R$ has an analytic extension in $\mcE_{\sigma}$ we have from Theorem
\ref{errorestimates:theorem} (following a similar approach as in
\cite{babusk_nobile_temp_10}) that
\[
\begin{split}
\|(I - \mcI_{m})u\|_{L^{\infty}(\Gamma_n)}
\leq
(1 + \Lambda_{m})
\frac{2M}{e^{\sigma} - 1} e^{-\sigma (m-1)}
\leq 
2 C(M,\sigma) m e^{-\sigma (m-1)},
\end{split}
\]
where $C(M,\sigma_n) := \frac{2M}{(e^{ \sigma} - 1)}$. We then
conclude that for all $k = 1,\dots, m$
\begin{equation}
\begin{split}
\| \Delta_{k}(u) \|_{L^{\infty}(\Gamma)} 
&=
\|
\mcI^{m}(u) - \mcI^{m-1}(u)
\|_{L^{\infty}(\Gamma)} 
\leq
\|(I - \mcI_{m})u\|_{L^{\infty}(\Gamma)}
+
\|(I - \mcI_{m-1})u\|_{L^{\infty}(\Gamma)} \\
&\leq
e^{2\sigma}C(M,\sigma) m e^{-\sigma m}.
\end{split}
\label{interpolation:eqn1}
\end{equation}
Let $\mcE_{\sigma,n} \subset \bbC^{d}$ a complex region bounded by a
Bernstein ellipse such that the restriction on $\Gamma_{d}$ is along
the $n^{th}$ dimension and form the polyellipse $\mcE^{d}_{\sigma}:=
\prod_{n=1}^{d} \mcE_{\sigma,n}$.  Suppose that $v:\mcE^{d}_{\sigma}
\rightarrow \bbC$ is analytic on $\mcE^{d}_{\sigma}$ and let
$\tilde{M}(v) := \max_{\bz \in \mcE^{d}_{\sigma}} |v(\bz)|$.

Note we refer to $\mcI^{n}_{m}$ as the Lagrange operator of order $m$
along the $n^{th}$ dimension and similarly $\mcP^{n}_{m-1}$ is the
space of the span of univariate polynomials up to degree $m-1$ along
the $n^{th}$ dimension.  Form the tensor product $\bI^{d}_{m} :=
\mcI^{1}_{m} \times \dots \times \mcI^{d}_{m}$, thus $\bI:C(\Gamma,\R)
\rightarrow \bbP$ where $\bbP := \mcP^{1}_{m-1} \times \dots \times
\mcP^{d}_{m-1}$. From Theorem 2.27 in \cite{Khoromskij2018} we can
conclude that for a finite dimension $d$, as $m \rightarrow \infty$
then $\bI^{d}_{m}[v] \rightarrow v$.

Applying equation \eqref{interpolation:eqn1} to equation
\eqref{errorestimates:SG} we have that
\begin{equation}
\begin{split}
& \| (I - \mcS_{w,d})
 v(\by)
 \|_{L^{\infty}(\Gamma^{d})}
 \leq
 \left\| \sum_{\bm \in \bbNset^{d}: \sum_{i=1}^{d} m_i - 1 > w } \;\;
 \bigotimes_{n=1}^{d} {\Delta^{n}_{m_n}}(v(\by))\right\|_{L^{\infty}(\Gamma^d)} \\
 &\leq
 \sum_{\bm \in \bbNset^{d}: \sum_{i=1}^{d} m_i - 1 > w } \;\;
 \bigotimes_{n=1}^{d} \|{\Delta^{n}_{m_n}}(v(\by))\|_{L^{\infty}(\Gamma^d)} 
 \leq
 \sum_{\bm \in \bbNset^{d}: \sum_{i=1}^{d} m_i - 1 > w }
 e^{2d} C(M,\sigma)^{d} \\
 &
 \left( \prod_{n=1}^{d} m_n\right) \exp{\left( -\sum_{n=1}^{d}
   \sigma m_{n} \right)}.
% \\
%  &\leq
% \sum_{\bk \in \bbNset^{d}_{0}: \sum_{i=1}^{d} k_i > w }
% e^{2d} C(M,\sigma)^{d} \left( \prod_{n=1}^{d} (k_n + 1)\right)
% \exp{\left( -\sum_{n=1}^{d}
%   \sigma (k_{n}+1) \right)}.
\end{split}
\label{interpolation:eqn2}
\end{equation}

By applying Theorem 2.10 and Corollary 2.11 in \cite{Griebel2016} if
$ w \geq  d$ and $p( d, w) \geq
\left(\frac{2  d}{\kappa( d)}\right)^{ d}$, where
$\kappa( d) := \sqrt[\leftroot{-2}\uproot{2}  d]{
  d!} >  d/e$ (Sterling approximation), then for any $\hat
\sigma \in \R_{+}$
\begin{equation}
\begin{split}
 & \sum_{\bk \in \bbNset^{ d}_{0}: \sum_{i=1}^{ d} k_i  >  w }
 \exp{\left( -\sum_{n=1}^{ d} \hat \sigma
   k_{n} \right)}
 \leq
 \sum_{\bk \in \bbNset^{d}_{0}: \hat \sigma \sum_{i=1}^{ d} k_i  \geq  w \hat \sigma  }
 \exp{\left( -\sum_{n=1}^{ d}
   \hat \sigma k_{n} \right)} \\
 &\leq
 \hat \sigma  d e
 \left( \frac{e^{\hat \sigma}}{1 - e^{-\hat \sigma}} \right)^{ d}
 \exp \left(-\frac{ d}{e} \hat \sigma  p^{\frac{1}{ d}}
 \right) p^{\frac{ d-1}{ d}}.
\end{split}
\label{interpolation:eqn3}
\end{equation}
where $\bk \in \bbNset^{d}_{0}$ and $\bk:=(k_1,\dots,k_d)$.






Following the same approach as in \cite{Griebel2016} observe that for
$0 < \delta < 1$ we can obtain a bounded constant $c_{n,\delta} \leq
(e\sigma \delta)^{-1}$ such that $m_n \exp(-\sigma m_n) \leq (e\sigma
\delta)^{-1}$ $\exp(-\sigma m_n (1 - \delta))$. Set $\hat \sigma :=
\sigma (1 - \delta)$ and by combining equations
\eqref{interpolation:eqn2} and \eqref{interpolation:eqn3} we have
proven the following result.

\begin{lemma} Suppose that $0< \delta < 1$, $\hat
  \sigma := \sigma (1 - \delta)$, and $p(d,w) \geq \left(\frac{2
    d}{\kappa(d)}\right)^{d}$ then
  \[
  \begin{split}
 &\| (I - \mcS_{w,d})
 v(\by)
 \|_{L^{\infty}(\Gamma^{d})}
 \leq 
 \frac{C(\tilde M,\sigma)^d e^{d - \sigma(1 - \delta) + 1} \hat \sigma d }
 {
(\sigma \delta)^{d}}
 \left( \frac{e^{\hat \sigma}}{1 - e^{-\hat \sigma}} \right)^{d} 
 \exp \left(-\frac{d}{e} \hat \sigma  p^{\frac{1}{d}}
 \right) p^{\frac{d-1}{d}}.
 \end{split}
 \]
 \label{interpolation:lemma1}
\end{lemma}


\begin{remark}
The restriction $p(d,w) \geq \left(\frac{2
  d}{\kappa(d)}\right)^{d}$ is not strict and can be relaxed such that
sub-exponential convergence is still obtained.  We refer the reader to
the bound of the Gamma function in Lemma 2.5 (\cite{Griebel2016}) and
it's application in the proofs of Theorem 2.10 and Corollary 2.11.
\label{interpolation:remark1}
\end{remark}




\noindent 
\textbf{Acknowledgements:} I appreciate the help and advice from
George Biros and Lexing Ying, for setting up the KIFMM packages.  I
also appreciate the support that King Abdullah University of Science
and Technology has provided to this project.




\bibliographystyle{abrev}
%%%%%%%%%%%%%%%%%%%%%%%%%%%%%%%%%%%%%%%%%%%%%
%
% Latex file for:
% 
% 
%%%%%%%%%%%%%%%%%%%%%%%%%%%%%%%%%%%%%%%%%%%%%

%\RequirePackage{fix-cm}
\documentclass[11pt,final]{amsart}       % one column (second format)

\usepackage{epsfig, epsf, graphicx, float, color}
\usepackage{pstricks, psfrag}
\usepackage{amssymb,amsthm}
\usepackage[foot]{amsaddr}
\usepackage{verbatim,enumerate,hyperref}
\usepackage{setspace,mathtools,wrapfig}
\usepackage[numbers,sort&compress]{natbib}
\usepackage{algorithm2e}
\usepackage{steinmetz}
\usepackage{tikz-qtree,tikz-qtree-compat}
\usepackage{tikz,pgf}
\usepackage{arydshln}
\usepackage[margin=1in]{geometry}

\usetikzlibrary{decorations,decorations.markings,decorations.text}
\usetikzlibrary{arrows.meta}
\usetikzlibrary{patterns}


\usetikzlibrary{positioning,patterns}
\usetikzlibrary{calc,fadings,decorations.pathreplacing,arrows,datavisualization.formats.functions,shapes.geometric}




%\smartqed % flush right qed marks, e.g. at end of proof

% Macros
\input{trmacros}


\begin{document}



\pgfkeys{/pgf/decoration/.cd,
      distance/.initial=10pt
}  

\pgfdeclaredecoration{add dim}{final}{
\state{final}{% 
\pgfmathsetmacro{\dist}{5pt*\pgfkeysvalueof{/pgf/decoration/distance}/abs(\pgfkeysvalueof{/pgf/decoration/distance})}    
          \pgfpathmoveto{\pgfpoint{0pt}{0pt}}             
          \pgfpathlineto{\pgfpoint{0pt}{2*\dist}}   
          \pgfpathmoveto{\pgfpoint{\pgfdecoratedpathlength}{0pt}} 
          \pgfpathlineto{\pgfpoint{(\pgfdecoratedpathlength}{2*\dist}}     
          \pgfsetarrowsstart{latex}
          \pgfsetarrowsend{latex}
          \pgfpathmoveto{\pgfpoint{0pt}{\dist}}
          \pgfpathlineto{\pgfpoint{\pgfdecoratedpathlength}{\dist}} 
          \pgfusepath{stroke} 
          \pgfpathmoveto{\pgfpoint{0pt}{0pt}}
          \pgfpathlineto{\pgfpoint{\pgfdecoratedpathlength}{0pt}}
}}

\tikzset{dim/.style args={#1,#2}{decoration={add dim,distance=#2},
                decorate,
                postaction={decorate,decoration={text along path,
                                                 raise=#2,
                                                 text align={align=center},
                                                 text={#1}}}}}



\title{High dimensional multilevel kriging: A computational
  mathematics approach}

%\title{High dimensional multilevel kriging: A computational
%  mathematics approach \thanks{This material is based upon work
%   supported by the National Science Foundation under Grant
%    No. 1736392.}}



\author{Julio E. Castrill\'on-Cand\'as ${\dagger}$} 
  \email{jcandas@bu.edu}


 \address{
   ${\ddagger}$ Department of Mathematics and Statistics, 
  Boston University, Boston, MA 
  }
   

%%%%%%%%%%%%%%%%%%%%%%%%%%%%%%%%%%%%%%%%%%%%%%%%%%%%%%%%%%%%%%%%%%%%%%%%




\begin{abstract}
With the advent of massive data sets much of the computational science
and engineering communities have been moving toward data-driven
approaches such as regression and classification. However, they
present a significant challenge due to the increasing size, complexity
and dimensionality of the problems.  In this paper a multilevel
Kriging method that scales well with the number of observations and
dimensions is developed.  A multilevel basis is constructed that is
adapted to a kD-tree partitioning of the observations.  Numerically
unstable covariance matrices with large condition numbers are
transformed into well conditioned multilevel matrices without
compromising accuracy. Moreover, it is shown that the multilevel
prediction \emph{exactly} solves the Best Linear Unbiased Predictor
(BLUP), but is numerically stable.  The multilevel method is tested on
numerically unstable problems of up 25 dimensions. Numerical results
show speedups of up to 42,050 for solving the BLUP problem but to the
same accuracy than the traditional iterative approach.
\end{abstract}



\maketitle

\noindent
    {\it Keywords:} Hierarchical Basis, Machine Learning, High
    Performance Computing, Sparsification of Covariance Matrices, Fast
    Multipole Method

    






%%%%%%%%%%%%%%%%%%%%%%%%%%%%%%%%%%%%%%%%%%%%%%%%%%%%%%%%%%%%%%%%%%%%%%%%


%% Introduction  ------------------------------------------------------
%%

\section{Introduction}





%\item {\bf What is the problem?  Why is it hard?}

Massive data sets arise from many fields, including, but not limited
to commerce, astrophysical sky-surveys, environmental data, and tsunami
warning systems.  With the advent of big data sets much of the
computational science and engineering communities have been moving
toward data-driven approaches to regression and classification. These
approaches are effective since the underlying data is incorporated
into the optimization. However, they present a numerical challenge due
to increasing size, complexity and dimensionality.

%\item {\bf How is it done today, and what are the limits of current
%practice?}

Due to the high dimensionality of the underlying data many modern
machine learning methods, such as classification and regression
algorithms, seek a balance between accuracy and computational
complexity. How efficient this balance is depends on the approach.
Linear methods are fast, but only work well when there is linear
separation of the data.

For non-linear description of the data, kernel approaches have been
effective under certain circumstances.  These methods rely on Tikhonov
regularization of the data to obtain a functional representation,
where it is assumed that the noise model of the phenomena is
known. However, this assumption is not necessarily satisfied in
practice and can lead to significant errors as the algorithm cannot
distinguish between noise and the underlying phenomena.


To incorporate the variability of the noise model a class of machine
learning algorithms based on Bayes method have been developed. In this
approach the noise model is assumed to be known up to a class of
probability distributions and an optimal choice is made that fits the
training data and noise. For example, from the Geo-statistics
community a well known approach to identifying the underlying data and
noise model is known as Kriging \cite{Nielsen2002}.  The noise model
parameters are estimated from the Maximum Likelihood Estimation (MLE)
of the likelihood function.

Kriging methods are effective in separating the underlying phenomena
from the noise model. However, in practice the covariance matrices
tend to be ill-conditioned with increasing number of observations
making Kriging methods numerically fragile. Moreover, most
applications are limited to 2 or 3 dimensions. A brief literature
review can be found in \cite{Castrillon2015}.

Kriging methods from the computational mathematics perspective have
been developed using skeletonization factorizations \cite{Minden2016},
low-rank \cite{nowak2013} and Hierarchical Matrices (HM)
\cite{khoromskij2009,Litvinenko2019,Geoga2020} approaches. These
methods are very promising. In particular, for the HM approaches they
have been shown to be near optimal. They work well for low dimensions.
However, they are still subject to ill-conditioning and usually a
nugget is added to change the model to make it more numerically
stable. Moreover, the data is assumed to have zero mean, which many
times will not be the case.

In \cite{Castrillon2015} a novel algorithm to solve Kriging problems
is proposed. The method is fast and robust. In particular, it can
solve Kriging problems that where not tractable with previous
methods. A nugget is not assumed, nor zero mean data.  However, this
approach is limited to 2 or 3 dimensions and the computational cost
scales very fast with spatial dimension, thus making it impractical
for high dimensional problems.

In this paper we extend the Kriging approach in \cite{Castrillon2015}
using binary trees, which are well suited for high dimensional
problems. Ill-conditioned covariance matrices are transformed to
numerically stable multilevel covariance matrices without compromising
accuracy. In addition, a new distance criterion is developed to build
sparse multilevel covariance matrices.  Furthermore, sharper decay
estimates of the coefficients of the multivariate covariance matrix
are derived based on analytic extensions that are well suited for high
dimensional problems. Much of this theory and notation is borrowed
from uncertainty quantification and high dimensional integration for
partial differential equations
\cite{nobile2008a,Castrillon2016,Griebel2016}.

The Kriging estimation is transformed into a multilevel form based on
the numerically stable multilevel covariance matrix. In practice a
sparse version of the multilevel covariance matrix is used. A distance
dependent method is used to build to a sparse version. Sharp decay
estimates (sub-exponential) of the multilevel covariance matrices are
derived using complex analytic extensions of the covariance function
instead of Taylor series expansions, which are infeasible for
relatively large dimensional problems. The numerical results show that
the estimation is solved to good accuracy for a large number of
observations.

The Kriging prediction step is remapped into an equivalent multilevel
formulation that is numerically stable. It is shown that the solution
to the multilevel prediction form \emph{exactly} solves the Best
Linear Unbiased Prediction (BLUP) problem.  To my knowledge, this is a
feature that is unique to the multilevel approach.  If the covariance
matrix is ill-conditioned, then it is not possible to solve the
problem accurately on a computer with a fixed machine
precision. However, the BLUP solution arises from a constrained
optimization problem. By taking advantage of this fact, the multilevel
approach side steps the inversion of the covariance matrix and
directly searches for the solution in a constrained space giving rise
to a stable multilevel covariance matrix.  Moreover, only one matrix
inversion (iterative approach) of the multilevel covariance matrix is
required in contrast to classical BLUP, including the Generalized
Least Squares (GLS) prediction, that requires $p+1$ matrix inversions
(iterative approach), where $p$ is the number of columns of the design
matrix.  Numerical results show speedups of up to 42,050 for solving
the BLUP problem to at least the same accuracy.

The multilevel Kriging method makes previously impractical missing
data problems feasible.  We are currently applying the Kriging
approach to missing data problems for medical data sets and have shown
up to 5-6 times improved accuracy over traditional state of the art
missing data packages.


In Section \ref{Introduction} the problem formulation is introduced.
In section \ref{MultilevelCovarianceMatrix} the construction of the
multilevel covariance matrix is discussed. In section
\ref{multilevelestimator} the multilevel estimator and predictor are
formulated and numerical computational issues are discussed in section
\ref{numericalcomputation}.  In section \ref{errorestimates} a
mathematical analysis of the decay of the entries of the multilevel
covariance matrix is developed. This section can also be skipped for
the less mathematically inclined reader.  In section
\ref{numericalresults} the multilevel Kriging method is tested on
numerically unstable problems of up to 25 dimensions.  In the Appendix
the Multivariate polynomial interpolation based on complex analytic
extensions is discussed. These results are used for to derive the
decay of the entries of the multilevel covariance matrix.  In section
\ref{multilevelapproach} it is shown how to construct the multilevel
basis based on kd-trees.


%% Problem Setup ------------------------------------------------------
%%
\section{Problem setup}
\label{Introduction}

Consider the following model for a Gaussian random field $Z$:
\begin{equation}
Z(\bx) = \bk(\bx)\T \bbeta+\varepsilon(\bx), \qquad \bx \in \R^d,
\label{Introduction:noisemodel}
\end{equation}
where $d$ is the number of spatial dimensions, $\bk:\R^d \rightarrow
\R^p$ is a functional vector of the spatial location $\bx$,
$\bbeta\in\R^p$ is an unknown vector of coefficients, and
$\varepsilon$ is a stationary mean zero Gaussian random field with
parametric covariance function
$C(\bx,\bx';\btheta)=\cov\{\varepsilon(\bx),\varepsilon(\bx')\}$ with
an unknown vector of positive parameters $\btheta\in\R^d$.

Suppose that we obtain $N$ observations and stack them in the data
vector $\bZ=(Z(\bx_1),\ldots,$ $Z(\bx_N))\T$ from locations $\bbS :=\{
\bx_{1},\dots,\bx_{N}\}$, where the elements in $\bbS$ are restricted
such that the design matrix defined below, $\bX$, has full column
rank.  Furthermore, without loss of generality all the locations in
$\bbS$ are contained in the unit hypercube $[-1,1]^{d}$.  Let
$\bC(\btheta)=\cov(\bZ,\bZ\T)\in \R^{N \times N}$ be the covariance
matrix of $\bZ$ and assume it is positive definite for all
$\btheta\in\R^w$.  Define $\bX=\big( \bk(\bx_1) \ldots$ $
\bk(\bx_N)\big)\T\in \R^{n\times p}$ and assume it is of full rank,
$p$. Since the model \eqref{Introduction:noisemodel} is a Gaussian
random field, then from the samples of $\bbS$ the following vectorial
model is obtained
\begin{equation}
{\bf Z} = \bX \bbeta +{\boldsymbol \varepsilon},
\label{Introduction:vectormodel}
\end{equation}
where $\boldsymbol \varepsilon$ is a Gaussian random vector,
${\boldsymbol \varepsilon} \sim \mcN(\0,\bC(\btheta))$. The aim
now is to:

\begin{enumerate}[i)]
\item {\it Estimate} the unknown vectors $\bbeta$ and $\btheta$;

\item {\it Predict} $Z(\bx_0)$, where $\bx_0$ is a new spatial
  location. These two tasks are particularly computationally
  challenging when the sample size $N$ and number of dimensions $d$
  are large.
\end{enumerate}

There is a very large literature on Gaussian process regression that
deal with this problem. Please see \cite{Castrillon2015} for a brief
literature review.  The unknown vectors $\bbeta$ and $\btheta$ are
estimated with the log-likelihood function
\begin{equation}
  \begin{split}
\ell(\bbeta,\btheta)&=-\frac{n}{2}\log(2\pi)-\frac{1}{2}\log
\det\{\bC(\btheta)\} \\ &
-\frac{1}{2}(\bZ-\bX\bbeta)\T\bC(\btheta)^{-1}
(\bZ-\bX\bbeta),
\end{split}
\label{Introduction:loglikelihood}
\end{equation}
which can be profiled by Generalized Least Squares (GLS) with
\begin{equation}
  \hat \bbeta(\btheta)=\{\bX\T \bC(\btheta)^{-1} \bX\}^{-1}\bX\T
  \bC(\btheta)^{-1}\bZ.
  \label{GLSbeta}
\end{equation}
In general this is not a good choice, since profiling with the Maximum
Likelihood Estimator (MLE) of $\btheta$ is prone to be biased
\cite{Castrillon2015}.

%A solution to this problem is to use restricted maximum likelihood
%(REML) estimation which consists in calculating the log-likelihood of
%$n-p$ linearly independent contrasts, that is, linear combinations of
%observations whose joint distribution does not depend on $\bbeta$,
%from the set $\bY=\{\bI_n-\bX(\bX\T\bX)^{-1}\bX\T\}\bZ$.


For the prediction part, consider the Best Linear Unbiased Predictor
(BLUP) $\hat Z(\bx_0)=\lambda_0+\blambda\T\bZ$ where
$\blambda=(\lambda_1,\ldots,\lambda_n)\T$. The unbiased constraint
implies $\lambda_0=0$ and $\bX\T\blambda=\bk(\bx_0)$.  The
minimization of the mean squared prediction error
E$[\{Z(\bx_0)-\blambda\T\bZ\}^2]$ under the constraint
$\bX\T\blambda=\bk(\bx_0)$ yields
\begin{equation}
\hat Z(\bx_0)=\bk(\bx_0)\T\hat \bbeta+\bc(\btheta)\T
\bC(\btheta)^{-1}(\bZ-\bX\hat \bbeta), \label{KrigBLUP}
\end{equation}
where $\bc(\btheta)=\cov\{\bZ,Z(\bx_0)\}\in \R^{n}$ and $\hat \bbeta$ is
defined in (\ref{GLSbeta}).  

Now, let $\alpha:= (\alpha_{1},\dots,\alpha_{d}) \in \mathbb{Z}{^d}$,
$|\alpha| := \alpha_{1}+\dots+\alpha_{d}$, $\bx : =
[x_1,\dots,x_d]$. For any $w \in \bbN_+$ (where $\mathbb{N}_+ :=
\mathbb{N} \cup \{0\}$) let $\mcQ^d_w$ be the set of Total Degree (TD)
monomials $\{x_1^{\alpha_1} \dots x_d^{\alpha_d}\,\,\,|\,\,\, |\alpha|
\leq w\}$. The typical choice for the matrix $\bX$ is to build it from
the monomials of $\mcQ^d_w$ with cardinality
$p(d,w):=\begin{pmatrix} d + w \\ w \end{pmatrix}$.

The challenge is that the covariance matrix $\bC(\btheta)$ in many
practical cases is ill-conditioned, leading to slow and inaccurate
estimates of $\btheta$. Following the approach in
\cite{Castrillon2015} the data vector $\bZ$ is transformed into
decoupled multilevel description of the model
\eqref{Introduction:noisemodel}.  This multilevel representation leads
to significant computational benefits, including numerical stability,
when computing the Kriging predictor $\hat Z(\bx_0)$ in
(\ref{KrigBLUP}) for large sample size $N$ and high dimensions $d$.
Note, that in this paper we shall refer to the \emph{single level}
approach to solving the Kriging problem by applying the estimation and
prediction steps directly to the data $\bZ$ and covariance matrix
$\bC(\btheta)$.





%% \section{Polynomial Interpolation}
%% \label{Polynomial}

%% \corb{In this section we give some background on polynomial
%%   interpolation in high dimensions. This will be critical to estimate
%%   the decay rates of the entries of the multilevel covariance matrix
%%   for high dimensional problems. Note that for the less mathematically
%%   inclided reader this section can be skipped as it is only used for
%%   estimating the decay of the multilevel covariance matrix.}

%% The decay of the coefficients will directly depend on the analytic
%% properties of the covariance function. The traditional error estimates
%% of polynomial interpolation are based on multi-variate $m^{th}$ order
%% derivatives. However, for many cases, such as the Mat\'{e}rn
%% covariance function, the derivatives are too complex or expensive to
%% manipulate for even a moderate number of dimensions. This motivates
%% the study of polynomial numerical approximations based on complex
%% analytic extensions, which are much better suited for high dimensions.
%% Much of the discussion that follows has it roots in the field of
%% uncertainty quantification and high dimensional interpolation
%% \cite{nobile2008a,Castrillon2016,Griebel2016}
%% for partial differential
%% equations.


%% Consider the problem of approximating a function $v: \Gamma^{d}
%% \rightarrow \R$ on the domain $\Gamma^{d}$.  Without loss of
%% generality let $\Gamma : = [-1, 1]$ and $\Gamma^{d} := \prod_{n =
%%   1}^{d} \Gamma$. Suppose that $\mcG \subset \Gamma^{d}$, then define
%% the following spaces
%% \[
%% \begin{split}
%%   &
%% L^q(\mcG) := \{ v(\by)\, | \, \int_{\mcG} v(\by)^q \text{d}
%% \by < \infty  \}
%% \,\,\,
%% \mbox{and} \\
%% &
%% L^{\infty}(\mcG) := \{ v(\by)\, | \, \sup_{\by \in \mcG} |v(\by)|
%% < \infty  \}.
%% \end{split}
%% \]


%% Suppose that $\mcP_{ q}(\Gamma):=\text{\rm span}\{y^k,\,k=0,\dots,q\}$
%% i.e. the space of polynomials of degree at most $q$. Let $\mcI^{m} :
%% C^{0}(\Gamma) \rightarrow \mcP_{m-1}(\Gamma)$ be the univariate
%% Lagrange interpolant
%% \[
%% \mcI_{m}(v(\by)):=
%% \sum_{k=1}^{m}v(y^{(k)})l_{m,k}(y^{(k)}),
%% \]
%% where $y^{(1)}, \dots, y^{(m)}$ is a set of distinct knots on $\Gamma$
%% and $\{ l_{n,k} \}_{k=0}^{m}$ is a Lagrange basis of the space
%% $\mcP_{m-1}(\Gamma)$. The variable $m \in \Nset$
%% %, where $\Nset_{+} := \Nset \cup 0$,
%% corresponds to the order of approximation of the
%% Lagrange interpolant. However, for the case of the zero order
%% interpolation $m = 0$ corresponds to $\mcI_{0} = 0$.


%% \begin{remark}
%% For high dimensional interpolation the particular set of points
%% $y^{(1)}, \dots, y^{(m)}$ that we will use is the Clenshaw-Curtis
%% abscissas.  This is further discussed in this section. However, for
%% now, we assume that the points are only distinct.
%%   \end{remark}


%% For $m \geq 1$ let
%% \[
%% \Delta_{m}
%% := \mcI_{m}-\mcI_{m-1},
%% \]
%% From the difference operator $\Delta_{m}$ we can readily observe that
%% $\mcI_{m} = \sum_{k=1}^{m} \Delta_{k}$, which is reminiscent of multi
%% resolution wavelet decompositions. The idea is to represent
%% multivariate approximation as a summation of the difference operators.

%% Consider the multi-index tupple $\bm = (m_1,\dots,m_d)$, where $\bm
%% \in \Nset^{d}$, and form the tensor product operator
%% $\mcS_{w,d}: \Gamma \rightarrow \R$ as
%% \begin{equation}
%%   \mcS_{w,d}
%%       [v(\by)]
%%       :
%%       =
%%  \sum_{\bm \in \bbNset^{d}: \sum_{i=1}^{d} m_i - 1  \leq w } \;\;
%%  \bigotimes_{n=1}^{d} {\Delta^{n}_{m_n}}(v(\by)).
%% \label{errorestimates:SG}
%% \end{equation}
%% Note that by ${\Delta^{n}_{m_n}}(v(\by))$ we mean that the difference
%% operator ${\Delta_{m_n}}$ is applied along the $n^{th}$ dimension in
%% $\Gamma$.


%% Let $C^{0}(\Gamma_d; \R) : = \{ v: \Gamma_d \rightarrow \R\,\,$ is
%% continuous on $\Gamma_d$ and $\max_{\by\in \Gamma_d} |v(\by)| < \infty
%% \}$.  From Proposition 1 in \cite{Back2011} it is shown that for any
%% $v \in C^0(\Gamma_d;\R)$, we have $\mcS_{w,d}[v]\in \mcQ^{d}_{w}$.
%% Moreover, $\mcS_{w,d}[v] = v$, for all $v \in \mcQ^{d}_{w}$. The key
%% observation to take away is that the operator $\mcS_{w,d}[v]$ is
%% \textit{exact} in the space of polynomials $\mcQ^{d}_{w}$. This will
%% be useful in connecting the Lagrange interpolant with Chebyshev
%% polynomials.


%% Let $T_k:\Gamma \rightarrow \R$, $k = 0, 1, \dots$, be a Chebyshev
%% polynomial over $\Gamma$, which are defined recursively as follows:
%% $T_0(y) = 1$, $T_1(y) = y$, $\dots$, $T_{k+1}(y) = 2yT_{k}(y) -
%% T_{k-1}(y)$, $\dots$, where $y \in \Gamma$. Chebyshev polynomials are
%% well suited for the approximation of functions with analytic
%% extensions on a complex region bounded by a Bernstein ellipse. They
%% bypassing the need of using derivative information and sharp bounds on
%% the error are readily available. Suppose that $\sigma > 0$ and denote
%% by
%% \[
%% \begin{split}
%%   \mcE_{\sigma} := \Big\{
%%   &z \in \bbC, \sigma \geq
%% \delta \geq 0 ;\,\Real{z} = \frac{e^{\delta} + e^{-\delta}
%% }{2}cos(\theta) \\
%% &\Imag{z} = \frac{e^{\delta} 
%%   - e^{-\delta}}{2}sin(\theta),
%% \theta \in [0,2\pi)
%%   \Big\}
%% \end{split}
%%   \]
%% as the region bounded by a Bernstein ellipse (see Figure
%% \ref{erroranalysis:sparsegrid:polyellipse}).

%% The following theorem is based on complex analytic extensions on
%% $\mcE_{\sigma}$ and provides a control for the Chebyshev polynomial
%% approximation.

%% \begin{theorem}
%% Suppose that for $u:\Gamma \rightarrow \R$ there exists an analytic
%% extension on $\mcE_{\sigma}$. If $|u| \leq M < \infty$ on
%% $\mcE_{\sigma}$ then there exists a sequence of coefficients
%% $|\alpha_k| \leq M / e^{k\sigma}$ such that $u \equiv \alpha_0 +
%% 2\sum_{k = 1}^{\infty} \alpha_{k} T_{k}$ on $\mcE_{\sigma}$. Moreover,
%% if $y \in \Gamma$ then
%% \[
%% %\begin{multline*}
%% %\shoveright{|q(y) - \alpha_0  - 2\sum_{k = 1}^{n} \alpha_{k} T_{k}(y)|
%% %\leq 
%% %\frac{2M}{e^{\sigma} - 1} e^{-n \sigma}.}
%% |q(y) - \alpha_0  - 2\sum_{k = 1}^{n} \alpha_{k} T_{k}(y)|
%% \leq 
%% \frac{2M}{e^{\sigma} - 1} e^{-n \sigma}.
%% %\end{multline*}
%% \]
%% \label{errorestimates:theorem}
%% \end{theorem}
%% \begin{proof}
%% See Theorem 2.25 in \cite{Khoromskij2018}
%% \end{proof}
%% \qed

%% \begin{figure}[htb]%[12]{r}{7cm}%[htp]
%% \begin{center}
%% \begin{tikzpicture}
%%     \begin{scope}[font=\scriptsize]

      
%%       \filldraw[fill=blue!20,
%%       semitransparent] (0,0) ellipse (2 and 1);

%%     \draw [->] (-2.5, 0) -- (2.5, 0) node [below left]  {$\Real $};
%%     \draw [->] (0,-1.5) -- (0,1.5) node [below left] {$\Imag$};
%%     \draw (1,-3pt) -- (1,3pt)   node [above] {$1$};
%%     \draw (-1,-3pt) -- (-1,3pt) node [above] {$-1$};
%%     \end{scope}
    
%%     \node [below right] at (-2.5,1.25) {$\mcE_{\sigma}$};

%%     \node [] at (0.75,1.25) {$\frac{e^{
%%           \sigma} - e^{- \sigma}}{2}$};

    
%%     \node [] at (2.75,0.25) {$\frac{e^{
%%       \sigma} + e^{- \sigma}}{2}$}; 
    
%% \end{tikzpicture}
%% \end{center}
%% \caption{Complex region bounded by the Bernstein ellipse.}
%% \label{erroranalysis:sparsegrid:polyellipse}
%% \end{figure}

%% We can now connect the error due to the Lagrange interpolation with
%% Chebyshev expansions. It is known that if $u \in C(\Gamma,\R)$ then
%% \[
%% \|(I - \mcI_{m})u\|_{L^{\infty}(\Gamma)} \leq
%% (1 + \Lambda_{m})
%% \min_{h \in \mcP_{m-1}} \| u - h \|_{L^{\infty}(\Gamma)},
%% \]
%% where $\Lambda_{m}$ is the Lebesgue constant (See Lemma 7 in
%% \cite{babusk_nobile_temp_10}). Note that $I:C^{d}(\Xi;\R) \rightarrow
%% C^{d}(\Xi;\R)$ refers to the identity operator and the domain $\Xi$ is
%% taken from context. For the previous case $\Xi = \Gamma$.  Bounds on
%% $\Lambda_{m}$ are known in the context of the location of the knots
%% $y^{(1)}, \dots, y^{(m)} \in \Gamma$. In this article we restrict our
%% attention to Clenshaw-Curtis abscissas
%% %\[
%% \[
%% y^{(j)} = -\cos \left( \frac{\pi(j-1)}{m - 1} \right),\,\, j =
%% 1,\dots, m
%% \]
%% %\]
%% and $\Lambda_m$ is bounded by $2\pi^{-1}(\log{(m-1)} + 1) \leq 2m - 1$
%% (see \cite{babusk_nobile_temp_10}).  Since the interpolation operator
%% $\mcI_{m}$ is exact on $\mcP_{m - 1}$, then if $u:\Gamma \rightarrow
%% \R$ has an analytic extension in $\mcE_{\sigma}$ we have from Theorem
%% \ref{errorestimates:theorem} (following a similar approach as in
%% \cite{babusk_nobile_temp_10}) that
%% \[
%% \begin{split}
%% \|(I - \mcI_{m})u\|_{L^{\infty}(\Gamma_n)}
%% &\leq
%% (1 + \Lambda_{m})
%% \frac{2M}{e^{\sigma} - 1} e^{-\sigma (m-1)} \\
%% &\leq 
%% 2 C(M,\sigma) m e^{-\sigma (m-1)},
%% \end{split}
%% \]
%% where $C(M,\sigma_n) := \frac{2M}{(e^{ \sigma} - 1)}$. We then
%% conclude that for all $k = 1,\dots, m$
%% \begin{equation}
%% \begin{split}
%% \| \Delta_{k}(u) \|_{L^{\infty}(\Gamma)} 
%% &=
%% \|
%% \mcI^{m}(u) - \mcI^{m-1}(u)
%% \|_{L^{\infty}(\Gamma)} \\
%% &\leq
%% \|(I - \mcI_{m})u\|_{L^{\infty}(\Gamma)} \\
%% &+
%% \|(I - \mcI_{m-1})u\|_{L^{\infty}(\Gamma)} \\
%% &\leq
%% e^{2\sigma}C(M,\sigma) m e^{-\sigma m}.
%% \end{split}
%% \label{interpolation:eqn1}
%% \end{equation}
%% Let $\mcE_{\sigma,n} \subset \bbC^{d}$ a complex region bounded by a
%% Bernstein ellipse such that the restriction on $\Gamma_{d}$ is along
%% the $n^{th}$ dimension and form the polyellipse $\mcE^{d}_{\sigma}:=
%% \prod_{n=1}^{d} \mcE_{\sigma,n}$.  Suppose that $v:\mcE^{d}_{\sigma}
%% \rightarrow \bbC$ is analytic on $\mcE^{d}_{\sigma}$ and let
%% $\tilde{M}(v) := \max_{\bz \in \mcE^{d}_{\sigma}} |v(\bz)|$.

%% Note we refer to $\mcI^{n}_{m}$ as the Lagrange operator of order $m$
%% along the $n^{th}$ dimension and similarly $\mcP^{n}_{m-1}$ is the
%% space of the span of univariate polynomials up to degree $m-1$ along
%% the $n^{th}$ dimension.  Form the tensor product $\bI^{d}_{m} :=
%% \mcI^{1}_{m} \times \dots \times \mcI^{d}_{m}$, thus $\bI:C(\Gamma,\R)
%% \rightarrow \bbP$ where $\bbP := \mcP^{1}_{m-1} \times \dots \times
%% \mcP^{d}_{m-1}$. From Theorem 2.27 in \cite{Khoromskij2018} we can
%% conclude that for a finite dimension $d$, as $m \rightarrow \infty$
%% then $\bI^{d}_{m}[v] \rightarrow v$.

%% Applying equation \eqref{interpolation:eqn1} to equation
%% \eqref{errorestimates:SG} we have that
%% \begin{equation}
%% \begin{split}
%% & \| (I - \mcS_{w,d})
%%  v(\by)
%%  \|_{L^{\infty}(\Gamma^{d})} \\
%%  &\leq
%%  \left\| \sum_{\bm \in \bbNset^{d}: \sum_{i=1}^{d} m_i - 1 > w } \;\;
%%  \bigotimes_{n=1}^{d} {\Delta^{n}_{m_n}}(v(\by))\right\|_{L^{\infty}(\Gamma^d)} \\
%%  &\leq
%%  \sum_{\bm \in \bbNset^{d}: \sum_{i=1}^{d} m_i - 1 > w } \;\;
%%  \bigotimes_{n=1}^{d} \|{\Delta^{n}_{m_n}}(v(\by))\|_{L^{\infty}(\Gamma^d)}  \\
%%  &\leq
%%  \sum_{\bm \in \bbNset^{d}: \sum_{i=1}^{d} m_i - 1 > w }
%%  e^{2d} C(M,\sigma)^{d} \\
%%  &
%%  \left( \prod_{n=1}^{d} m_n\right) \exp{\left( -\sum_{n=1}^{d}
%%    \sigma m_{n} \right)}.
%% % \\
%% %  &\leq
%% % \sum_{\bk \in \bbNset^{d}_{0}: \sum_{i=1}^{d} k_i > w }
%% % e^{2d} C(M,\sigma)^{d} \left( \prod_{n=1}^{d} (k_n + 1)\right)
%% % \exp{\left( -\sum_{n=1}^{d}
%% %   \sigma (k_{n}+1) \right)}.
%% \end{split}
%% \label{interpolation:eqn2}
%% \end{equation}

%% By applying Theorem 2.10 and Corollary 2.11 in \cite{Griebel2016} if
%% $ w \geq  d$ and $p( d, w) \geq
%% \left(\frac{2  d}{\kappa( d)}\right)^{ d}$, where
%% $\kappa( d) := \sqrt[\leftroot{-2}\uproot{2}  d]{
%%   d!} >  d/e$ (Sterling approximation), then for any $\hat
%% \sigma \in \R_{+}$
%% \begin{equation}
%% \begin{split}
%%  & \sum_{\bk \in \bbNset^{ d}_{0}: \sum_{i=1}^{ d} k_i  >  w }
%%  \exp{\left( -\sum_{n=1}^{ d} \hat \sigma
%%    k_{n} \right)} \\
%%  &\leq
%%  \sum_{\bk \in \bbNset^{d}_{0}: \hat \sigma \sum_{i=1}^{ d} k_i  \geq  w \hat \sigma  }
%%  \exp{\left( -\sum_{n=1}^{ d}
%%    \hat \sigma k_{n} \right)} \\
%%  &\leq
%%  \hat \sigma  d e
%%  \left( \frac{e^{\hat \sigma}}{1 - e^{-\hat \sigma}} \right)^{ d}
%%  \exp \left(-\frac{ d}{e} \hat \sigma  p^{\frac{1}{ d}}
%%  \right) p^{\frac{ d-1}{ d}}.
%% \end{split}
%% \label{interpolation:eqn3}
%% \end{equation}
%% where $\bk \in \bbNset^{d}_{0}$ and $\bk:=(k_1,\dots,k_d)$.






%% Following the same approach as in \cite{Griebel2016} observe that for
%% $0 < \delta < 1$ we can obtain a bounded constant $c_{n,\delta} \leq
%% (e\sigma \delta)^{-1}$ such that $m_n \exp(-\sigma m_n) \leq (e\sigma
%% \delta)^{-1}$ $\exp(-\sigma m_n (1 - \delta))$. Set $\hat \sigma :=
%% \sigma (1 - \delta)$ and by combining equations
%% \eqref{interpolation:eqn2} and \eqref{interpolation:eqn3} we have
%% proven the following result.

%% \begin{lemma} Suppose that $0< \delta < 1$, $\hat
%%   \sigma := \sigma (1 - \delta)$, and $p(d,w) \geq \left(\frac{2
%%     d}{\kappa(d)}\right)^{d}$ then
%%   \[
%%   \begin{split}
%%  &\| (I - \mcS_{w,d})
%%  v(\by)
%%  \|_{L^{\infty}(\Gamma^{d})}\\
%%  \leq &
%%  \frac{C(\tilde M,\sigma)^d e^{d - \sigma(1 - \delta) + 1} \hat \sigma d }
%%  {
%% (\sigma \delta)^{d}}
%%  \left( \frac{e^{\hat \sigma}}{1 - e^{-\hat \sigma}} \right)^{d} \\
%%  &
%%  \exp \left(-\frac{d}{e} \hat \sigma  p^{\frac{1}{d}}
%%  \right) p^{\frac{d-1}{d}}.
%%  \end{split}
%%  \]
%%  \label{interpolation:lemma1}
%% \end{lemma}


%% \begin{remark}
%% The restriction $p(d,w) \geq \left(\frac{2
%%   d}{\kappa(d)}\right)^{d}$ is not strict and can be relaxed such that
%% sub-exponential convergence is still obtained.  We refer the reader to
%% the bound of the Gamma function in Lemma 2.5 (\cite{Griebel2016}) and
%% it's application in the proofs of Theorem 2.10 and Corollary 2.11.
%% \label{interpolation:remark1}
%% \end{remark}



\section{Multilevel approach}
\label{multilevelapproach}

The general approach of this paper and multilevel basis construction
are now presented. We mostly follow the exposition laid out in
\cite{Castrillon2015}. The proof of Proposition
\ref{Multilevelapproach:theo1} is repeated, but clarified with
more details.

Let $\mcP^{p}(\bbS)$ be the span of the columns of the design matrix
$\bX$. Suppose that there exists the orthogonal projections $\bL :
\R^n \rightarrow \mcP^{p}(\bbS)$ and $\bW : \R^n \rightarrow
\mcP^{p}(\bbS)^{\perp}$, where $\mcP^{p}(\bbS)^{\perp}$ is the
orthogonal complement of $\mcP^{p}(\bbS)$.  The operator $\left[
\begin{array}{c}
\bW \\
\bL
\end{array}
\right ]$ is assumed to be orthonormal.


The first step is to filter out the effect of the trend by project the
observation onto the orthogonal subspace.  Let $\bZ_{\bW}: = \bW \bZ$,
thus from equation \eqref{Introduction:vectormodel} it follows that
$\bZ_{\bW} = {\bf W} ({\bX \bbeta}+ {\boldsymbol \varepsilon}) = {\bf
  W{\boldsymbol \varepsilon}}$. Notice that the trend component ${\bX}
\bbeta$ is removed from the data ${\bf Z}$. The new log-likelihood
function for $\bZ_{\bW}$ becomes
\begin{equation}
  \begin{split}
\ell_{\bW}(\btheta)
&=-\frac{n}{2}\log(2\pi)-\frac{1}{2}\log
\det\{\bC_{\bW}(\btheta)\} 
-\frac{1}{2}\bZ_{\bW}\T\bC_{\bW}(\btheta)^{-1}\bZ_{\bW},
\end{split}
\label{Introduction:multilevelloglikelihood}
\end{equation}
where $\bC_{\bW}(\btheta) := \bW \bC(\btheta) \bW \T$ and
$\bZ_{\bW}\sim \mcN_{N-p}(\0,$ $\bC_{\bW}(\btheta))$.  A consequence
of the filtering is that we obtain an unbiased estimator
\cite{Castrillon2015}.

The decoupling of the likelihood function is not the only advantage of
using $\bC_{\bW}(\btheta)$. The following theorem also shows that
$\bC_{\bW}(\btheta)$ is more numerically stable than $\bC(\btheta)$.

\begin{prop} 
\label{Multilevelapproach:theo1}
Let $\kappa(A) \rightarrow \R$ be the condition number of the matrix
$A \in \R^{N \times N}$ then
\[
\kappa(\bC_{\bW}(\btheta)) \leq 
\kappa(\bC(\btheta)).
\]
\end{prop}
\noindent 
\begin{proof}
To see this let $\bv := \bW\T \bw$ for all $\bw \in \R^{N-p}$, which
implies that $\bv \in \mathbb{R}^{n} \backslash
\mcP^{p}(\bbS)$. Moreover, this map is bijective.  Now, $\bv \T
\bC(\btheta) \bv = \bw \T \bC_{\bW}(\btheta) \bw$ for all $\bw \in
\R^{N-p}$. From the orthonormal property we have that for all $\bv \in
\mathbb{R}^{n} \backslash \mcP^{p}(\bbS)$
\[
\begin{split}
\min_{\bv \in \mathbb{R}^{n} \backslash \mcP^{p}(\bbS)} \frac{\bv\T\bC(\btheta)\bv}
{\|\bv\|^2} 
= \min_{\bw \in \mathbb{R}^{N-p} } \frac{\bw\T\bC_{\bW}(\btheta)\bw}{\|\bw\|^2} 
\,\,\,
\mbox{and}
\,\,\,
\max_{\bv \in \mathbb{R}^{n} \backslash \mcP^{p}(\bbS) } \frac{\bv\T\bC(\btheta)\bv}
{\|\bv\|^2}  
= \max_{\bw \in \mathbb{R}^{N-p}} \frac{\bw\T\bC_{\bW}(\btheta)\bw}{\|\bw\|^2}.
\end{split}
\]
Now, it is not hard to see that
\[
\begin{split}
  0
  &<
  \min_{\bv \in \mathbb{R}^{n}} \frac{\bv\T\bC(\btheta)\bv}{ \|\bv\|^{2} } 
\leq 
\min_{\bv \in \mathbb{R}^{n} \backslash \mcP^{p}(\bbS)}
 \frac{\bv\T\bC(\btheta)\bv}{ \|\bv\|^{2} }
\leq \max_{\bv \in \mathbb{R}^{n} \backslash \mcP^{p}(\bbS)} \frac{\bv\T\bC(\btheta)\bv}
{ \|\bv\|^{2} } 
\leq
\max_{\bv \in \mathbb{R}^{n}} \frac{\bv\T\bC(\btheta)\bv}{ \|\bv\|^{2} }.
\end{split}
\]
The result follows from the positive definite property of
$\bC(\btheta)$.
\end{proof}


Proposition \ref{Multilevelapproach:theo1} states that the condition
number of $\bC_{\bW}(\btheta)$ is less or equal to the condition
number of $\bC(\btheta)$. Thus computing the inverse of
$\bC_{\bW}(\btheta)$ (using a direct or iterative method) will
generally be more stable.

In practice, computing the inverse of $\bC_{\bW}(\btheta)$ can be
significantly more stable than $\bC(\btheta)$ depending on the choice
of $\mcQ^d_w$. This has many significant implications as it will now
be possible to solve numerically unstable problems. Furthermore, the
following useful result can be proven.

\begin{corollary}
  \label{Multilevelapproach:cor1}
  Let $[\bC_{\bW}(\btheta)]^q$ be the multilevel
  covariance matrix built from a TD basis with cardinality $q \in
  \bbN$.  Suppose that $p \leq q$, then
  \[
  \kappa([\bC_{\bW}(\btheta)]^p)
  \leq \kappa([\bC_{\bW}(\btheta)]^q).
  \]
\end{corollary}
\begin{proof}
  This follows from the fact that $\mcP^{q}(\bbS) \subset
  \mcP^{p}(\bbS)$ and by applying a similar argument as the proof of
  Proposition \ref{Multilevelapproach:theo1}.
\end{proof}


There are other advantages to the structure of the matrix
$\bC_{\bW}(\btheta)$.  In section \ref{errorestimates} we show that
for a good choice of the $\mcP(\bbS)$ the entries of
$\bC_{\bW}(\btheta)$ decay rapidly, and most of the entries can be
safely eliminated. A level dependent criterion approach is shown in
Section \ref{MultilevelCovarianceMatrix} that indicates which entries
are computed and which ones are not. With this approach a sparse
covariance matrix $\tilde{\bC}_{\bW}$ can be constructed such that it
is close to $\bC_{\bW}$ in a matrix norm sense, even if the
observations are highly correlated with distance.
%From the decay estimates of
%Section \ref{MultilevelCovarianceMatrix} 





\subsection{Binary multilevel basis}
\label{MultilevelREML}

In this section the construction of Multilevel Basis (MB) is shown.
The approach followed in this section is a based on the MB
construction in \cite{Castrillon2013}. The MB can then be used to: (i)
form the multilevel likelihood
\eqref{Introduction:multilevelloglikelihood}; (ii) sparsify the
covariance matrix $\bC_{\bW}(\btheta)$; and (iii) improve the numerical
stability of the covariance matrix $\bC(\btheta)$ in it's multilevel
form. But first, let us establish notations and definitions:
\begin{enumerate}

% \item Given $\mcQ^d_w$ and the locations $\bbS$
%  construct the design matrix $\bX$. Furthermore, form a second set of
%  monomials $\mctQ^{a}_{\Lambda^{m,g}(w)} : =
%  \mcQ_{\Lambda^{m,g}(w+a)} $ for $a = 0,1,\dots,$ i.e.
%  $\mctQ_{\Lambda^{m,g}(w)} \subset
%  \mctQ^{a}_{\Lambda^{m,g}(w)}$. Denote the accuracy parameter $\tilde
%  p \in \bbN$ as the cardinality of
%  $\mctQ^{a}_{\Lambda^{m,g}(w)}$. From the set of monomials
%  $\mctQ_{\Lambda^{m,g}(w)}$, for some user given parameter $a \in
%  \bbN_0$, and the set of observations $\bbS$ generate the design
%  matrix $\tilde \bX^{a}$. Denote also the space $\mcP^{\tilde
%    p}(\bbS)$ as the span of the columns of $\tilde \bX^{a}$.

\item For any index $i,j \in \mathbb{N}_{0}$, $1 \leq i \leq N$, $1
  \leq j \leq N$, let $\bve_{i}[j] = \delta[i-j]$, where
  $\delta[\cdot]$ is the discrete Kronecker delta function.

\item Let $\phi(\bx,\by;\btheta):\R^{d} \times \R^{d} \rightarrow \R$
  be the covariance function and assumed to be a positive definite.
  Let $\bC(\btheta)$ be the covariance matrix that is formed from all
  the interactions between the observation locations $\bbS$
  i.e. $\bC(\btheta) := \{ \phi(\bx_i,\by_j) \}$, where $i,j, =
  1,\dots,N$.  Alternatively we refer to $\phi(r; \btheta)$ as the
  covariance function where $r:\R^d \times \R^d \rightarrow \R$ is a
  function of $\bx$, $\by$ and $\btheta$.
\end{enumerate}

\begin{definition} The Mat\'{e}rn covariance function:
\[
\phi(r;\btheta)=\frac{1}{\Gamma(\nu)2^{\nu-1}} \left(
\sqrt{2\nu}\frac{r}{\rho} \right)^{\nu} K_{\nu} \left(
\sqrt{2\nu}\frac{r}{\rho} \right),
\]
where with a slight abuse of notation $\Gamma$ is the gamma function,
$r \in \R_{+}$, $0 < \nu$, $0 < \rho < \infty$, and $K_{\nu}$ is the
modified Bessel function of the second kind. It is understood from
context when $\Gamma$ is the gamma function.
\end{definition}

\begin{remark} The Mat\'{e}rn covariance function is a good choice for
the random field model. The parameter $\rho$ controls the length
correlation and the parameter $\nu$ changes the shape. For example, if
$\nu = 1/2 + n$, where $n \in \bbN_{+}$, then (see
\cite{abramowitz1964})
\[
\begin{split}
\phi(r;\rho) &= \exp   \bigg(-\frac{\sqrt{2\nu}r}{\rho} \bigg)
\frac{\Gamma(n + 1)}{\Gamma(2n + 1)} 
\sum_{k = 1}^{n} \frac{(n+1)!}{k!(n-k)!}
\bigg(
\frac{ \sqrt{8v} r }{ \rho } 
\bigg)^{n-k}
\end{split}
\]
and $\nu \rightarrow \infty \Rightarrow \phi(r;\btheta) \rightarrow
\exp \bigg(-\frac{r^2}{2\rho^2} \bigg)$. Note that even for a moderate
number of derivatives the number of terms will grow exponentially fast
leading to a very complex expression. This motivates the study of
complex analytical extensions of the covariance function. See Section
\ref{errorestimates} for more details.
%bbb

\label{multilevelapproach:remark1}
\end{remark}
The first step is to decompose the domain $\Gamma^{d}$ into a
multilevel domain decomposition. A good choice is based on the a
kD-tree decomposition of the space $\R^{d}$ \cite{Dasgupta2008}.
Other choices include
Projection (RP) tree \cite{Dasgupta2008}.
%This is a good choice for lower dimensions, however, as the number of
%dimensions $d$ becomes larger a better approach is to use a Random
%Projection (RP) tree \cite{Dasgupta2008}.
First start with the root node and cell $B^{0}_{0}$ at level $0$ that
contains all the observation nodes in $\bbS$. Now, split these nodes
into two children cells $B^{1}_{1}$ and $B^{1}_{2}$ at level $1$
according to the following rule:
\begin{enumerate}

\item Choose a unit vector $v$ in $\R^{d}$ along the axis of
  $\R^{d}$. This choice is the direction that leads to the maximum
  variance of the data in the cell along the direction of $v$.

\item Project all the nodes $\bx \in \bbS$ in the cell onto the unit
  vector $v$.

\item Split the cell with respect to the median
of the projections.

\end{enumerate}

For each cell $B^{1}_{1}$ and $B^{1}_{2}$ repeat the procedure until
there is at most $p$ nodes at the leaf nodes. Thus a binary tree is
obtained, which is of the form $B^{0}_{0}$, $B^{1}_{1}$, $B^{1}_{2}$,
$B^{2}_{3}$, $B^{2}_{4}$, $B^{2}_{5}$, $B^{2}_{6}$, $\dots $, where
$t$ is the maximal depth (level) of the tree.  Now, let $\mcB$ be the
set of all the cells in the tree and $\mcB^{n}$ be the set of all the
cells at level $0 \leq n \leq t$.  In addition, for each cell a unique
node number, current tree depth, threshold level and projection vector
are also assigned. This will be useful for searching the tree.
Algorithms \ref{RPMLB:algorithm1} and \ref{RPMLB:algorithm2-kd}
describe in more detail the construction of the kD-tree MB.
 
%\begin{remark}
%A kD-tree can also be constructed with Algorithms
%\ref{RPMLB:algorithm1} and \ref{RPMLB:algorithm2-kd}.
%\end{remark}

\begin{algorithm}[h]
  \KwIn{ $\bbS$, node, currentdepth, $n_0$} \KwOut{Tree, node}

\Begin{

\eIf {Tree = root}{node $\leftarrow$ 0, currentdepth $\leftarrow$ 0
Tree $\leftarrow$ MakeTree($\bbS$, node,
currentdepth + 1, $n_0$)
}
{

Tree.node = node

Tree.currentdepth = currentdepth - 1

node $\leftarrow$ node + 1

\If {$|\bbS| < n_0$}{return (Leaf)}





(Rule, threshold, $v$) $\leftarrow$ ChooseRule($\bbS$)

(Tree.LeftTree, node) 
$\leftarrow$ MakeTree($\bx \in \bbS$: Rule($\bx$) = True, node,
currentdepth + 1, $n_0$)

(Tree.RightTree, node)
$\leftarrow$ MakeTree($\bx \in \bbS$: Rule($\bx$) = false,  node, currentdepth + 1, $n_0$)

Tree.threshold = threshold\\
Tree.$v$ = $v$
}
}
\caption{MakeTree($\bbS$) function}
\label{RPMLB:algorithm1}
\end{algorithm}


%\begin{algorithm}[h]
%  \KwIn{ $\bbS$}
%  \KwOut{Rule, threshold, v}
%\Begin{
%choose a random unit vector $v$ \\
%Rule(x) := $x \cdot v  \leq$ threshold = median 
%$\{z \cdot v : z \in \bbS \}$
%}
%\caption{ChooseRule($\bbS$) function for RP tree}
%\label{RPMLB:algorithm2}
%\end{algorithm}

\begin{algorithm}[h]
  \KwIn{ $\bbS$}
  \KwOut{Rule, threshold, $v$}
\Begin{
    choose a coordinate direction that has maximal variance of the projection
    of the points in $\bbS$. \\
Rule(x) := $x \cdot v  \leq$ threshold = median
}

\caption{ChooseRule($\bbS$) function for kD-tree}
\label{RPMLB:algorithm2-kd}
\end{algorithm}


Now, suppose there is a one-to-one mapping between the set of unit
vectors $\mcE:=\{\bve_{1},\dots,\bve_{N}\}$, which is denoted as
leaf unit vectors, and the set of locations $\{
\bx_{1},\dots,\bx_{N}\}$, i.e. $\bx_{n} \longleftrightarrow \bve_{n}$
for all $n = 1, \dots, N$. It is clear that the span of the vectors
$\{\bve_{1},\dots,\bve_{N}\}$ is $\bbR^{N}$.  The next step is to
construct a new basis of $\R^{n}$ that is multilevel and orthonormal.

\setlength{\tabcolsep}{16pt}
\begin{figure*}
\begin{center}
  \begin{tabular}{c c}
\begin{tikzpicture}[scale=.65] 
  \begin{scope} 
 [place/.style={circle,draw=blue!50,fill=blue!20,thick,
     inner sep=0pt,minimum size=1.5mm}]

 \draw[step=8,gray,very thin] (0, 0) grid (8, 8);
    \draw (4,0) to (4,8);
    
    \draw (0,5) to (4,5);
    \draw (2.2,5) to (2.2,8);
    \draw (0,2) to (4,2);

    \draw (0,5) to (4,5);

    \draw (4,4.15) to (8,4.15);
    \draw (6.75,4.15) to (6.75,8);
    \draw (6,0) to (6,4.15);


    
  
    \node at (0.5,7.5) [place] {};
    \node at (0.3,6.3) [place] {};
    

    \node at (2.5,5.5) [place] {};
    \node at (3.2,5.2) [place] {};

    \node at (5,6) [place] {};
    \node at (3.8,5.5) [place] {}; %
    \node at (3.8,6) [place] {};   %


    \node at (0.5,3.5) [place] {};
    \node at (1.5,2.5) [place] {};
    \node at (2.3,2.2) [place] {};
    
    \node at (1.3,0.3) [place] {};
    \node at (2.7,0.5) [place] {};
    \node at (2.2,1.4) [place] {};
    \node at (2.6,1.4) [place] {};
    \node at (4.2,3.5) [place] {}; %
    \node at (3.7,3.3) [place] {};


    \node at (6.5,4.3) [place] {}; %
    \node at (7.5,5) [place] {};


    \node at (4.3,2.3) [place] {};
    \node at (5.7,3.5) [place] {};
    \node at (6.2,3.4) [place] {};
    \node at (7.3,2.4) [place] {};


    \node at (7,7) [place] {};
    \node at (6,7.5) [place] {};
    \node at (7.5,7.5) [place] {};


    \node at (6.5,2.0) [place] {};
    \node at (0.5,7.0) [place] {};
    \node at (2.0,7.0) [place] {};
    \node at (5,3.75) [place] {}; %
    \node at (6,7.0) [place] {};
    \node at (7,2.0) [place] {}; %
    %\node at (2.0,4.5) [place] {};
    \node at (7.5,4.3) [place] {}; %

    \node at (7.5,8.5) [] {$B^{0}_0$};
    \node at (0.6,5.5) [] {$B^{3}_{7}$};
    \node at (3,7) [] {$B^{3}_{8}$};
  \end{scope}
\end{tikzpicture} 
&
\begin{tikzpicture}[scale=0.85]
    %\node[anchor=center] at (0, -4.5) {$$};
    %\node[anchor=center] at (0,   10) {$$};
\begin{scope}[xshift=5cm, yshift=4cm,
place/.style={circle,draw=blue!50,fill=blue!20,thick,
      inner sep=0pt,minimum size=1.5mm},
placer/.style={circle,draw=blue!50,
  preaction={fill=darkgreen!60,fill opacity=0.5}, thick,inner
  sep=0pt,minimum size=1.5mm}, ]

  %placer/.style={circle,draw=blue!50,
  %preaction={fill=darkgreen!60,fill opacity=0.5}, thick,inner
  %sep=0pt,minimum size=1.5mm}, ]


%\filldraw[fill={rgb:red,143;green,188;blue,143},semitransparent, 
%      thick] (0, 0) rectangle (16, 16);


  
\Tree [.\node[placer]{$B^{0}_{0}$}; 
             [.\node[placer]{$B^{1}_{1}$};
                    [.\node[placer]{$B^{2}_{3}$}; 
                           [.\node[placer]{$B^{3}_{7}$};]
                           [.\node[placer]{$B^{3}_{8}$};] 
                    ]       
                    [.\node[placer]{$B^{2}_{4}$}; 
                           [.\node[placer]{$B^{3}_{9}$};] 
                           [.\node[placer]{$B^{3}_{10}$};] 
                    ] 
             ]                                        
             [.\node[placer]{$B^{1}_{2}$};
                    [      [.\node[placer]{$B^{2}_{5}$};
                                  [.\node[placer]{$B^{3}_{11}$};] 
                                  [.\node[placer]{$B^{3}_{12}$};] 
                           ]
                           [.\node[placer]{$B^{2}_{6}$}; 
                                  [.\node[placer]{$B^{3}_{13}$};] 
                                  [.\node[placer]{$B^{3}_{14}$};] 
                           ] 
                                          ]]
] 

\end{scope}
\end{tikzpicture}
\end{tabular}
\end{center}
\caption{Multilevel domain decomposition of the observations.}
\label{MLRLE:fig1}
\end{figure*}



\begin{enumerate}[(a)]
\item Start at the maximum level of the random projection tree,
  i.e. $q = t$.
\item For each leaf cell $B^{q}_{k} \in \mcB^{q}$ assume without loss
  of generality that there are $s$ observations nodes $\bbS^{q}_{k}:=\{
  \bx_1, \dots, \bx_s \}$ with associated vectors $C_k^{q} := \{
  \bve_1, \dots, \bve_s \}$.
  %Let $\mcE^q_k := \{\bx_1,\dots,\bx_s\}$
  %and
  Denote $\mcC^{q}_{k}$ as the span of the vectors in $C_k^{q}$.
\begin{enumerate}[i)]

\item Let $\bphi^{q,k} _{j} := \sum_{\bve_i \in C^q_k} c^{q,k} _{i,j}
  \bve_i, \hspace{2mm} j=1, \dots, a;
\hspace{2mm} \bpsi^{q,k}_{j} := \sum_{\bve_i \in C^q_k} d^{q,k}_{i,j}
\bve_i, \hspace{2mm} j=a+1, \dots, s$, where $c^{q,k}_{i,j}$,
$d^{q,k}_{i,j} \in \mathbb{R}$ and for some $a \in \mathbb{N}^{+}$. Note
that $a$ is unknown up to this point, but will be computed from the
data.  It is desired that the new discrete MB vector $\bpsi^{q,k}_{j}$
be orthogonal to $\mcP^{p}(\mathbb{S})$, i.e., for all $g \in \mcP^{
  p}(\mathbb{S})$:
\begin{equation}
\sum_{i=1}^{n} g[i] \bpsi^{q,k}_{j}[i] = 0
\label{hbconstruction:eqn1}
\end{equation}

\item Form the matrix $\mcM^{q,k} := \bX \T \bV^{q,k}$, where
  $\mcM^{q,k} \in \R^{p \times s}$, $\bV^{q,k} \in \R^{N \times s}$,
  and $\bV^{q,k}: = [\bve_1, \dots, \bve_i, \dots,\bve_s ]$ for all $\bve_i
  \in C_k^q$. Now, suppose that the matrix $\mcM^{q,k} $ has rank $a$
  and then perform the Singular Value Decomposition (SVD). Denote by
  $\bU \bD \bV $ the SVD of $\mcM^{q,k} $, where $\bU \in \R^{ p \times
    p}$, $\bD \in \R^{p \times s}$, and $\bV \in \R^{s \times s} $.

  \begin{remark} Note that in practice we only keep track of the
    non-zero elements of the vectors $\bve_1, \dots, \bve_s$. Thus the
    computational cost is reduced significantly. This is taken into
    account in the complexity analysis in Lemma
    \ref{MultilevelREML:lemma1} and \ref{MultilevelREML:lemma2}
  \end{remark}
  
\item Following the same argument as in \cite{Castrillon2015} but
  adapted to the kd-tree decomposition equation
  \eqref{hbconstruction:eqn1} is satisfied with the following choice
\[
  \left[ \begin{array}{ccc|ccc}
      c^{q,k}_{0,1} & \dots &c^{q,k}_{a,1} & d^{q,k}_{a+1,1} & \dots &d^{q,k}_{s,1} \\
      c^{q,k}_{0,2} & \dots &c^{q,k}_{a,2} & d^{q,k}_{a+1,2} & \dots &d^{q,k}_{s,2} \\
      \vdots & \vdots & \vdots & \vdots & \vdots & \vdots   \\
      c^{q,k}_{0,s} & \dots &c^{q,k}_{a,s} & d^{q,k}_{a+1,s} & \dots &d^{q,k}_{s,s}
    \end{array}
\right] := \bV\T.
% \label{eqDefVspT*}
  \]
\noindent For this choice the coefficient $a$ is equal to the number
of non-zero singular values. Thus the columns $a+1$, \dots, $s$ form
an orthonormal basis of the nullspace ${N_0}(\mcM^{q,k} )$. Similarly,
the columns $1,\dots, a$ form an orthonormal basis of $\R^s \backslash
{N_0}(\mcM^{q,k})$. Since the vectors in $C^q_k$ are orthonormal then
$\bphi^{q,k}_{1}, \dots, \bphi^{q,k}_a$, $\bpsi^{q,k}_{a+1}, \dots,$
$\bpsi^{q,k}_s$ form an orthonormal basis of $\mcC^{q}_{k}$.  Moreover
$\bpsi^{q,k}_{a+1}, \dots, \bpsi^{q,k}_s$ satisfy equation
\eqref{hbconstruction:eqn1}, i.e., are orthogonal to
$\mcP^{p}(\mathbb{S})$ and are locally adapted to the locations
contained in the cell $B^{q}_{k}$.

\item Denote by $D_k^{q,k}$ the collection of all the vectors
  $\bpsi^{q,k}_{a+1}, \dots, \bpsi^{q,k}_s$. Notice that the vectors
  $\bphi^{q,k}_{1}, \dots,$ $\bphi^{q,k}_a$, which are denoted with a
  slight abuse of notation as the scaling vectors, are {\it not}
  orthogonal to $\mcP^{p}(\mathbb{S})$. They need to be further
  processed.

\item Let $\mcD^{q}$ be the union of the vectors in $D^{q}_k$ for
  all the cells $B^{q}_k \in \mcB^{q}_{k}$. Denote by
  $W_{q}(\mathbb{S})$ as the span of all the vectors in $\mcD^{q}$.

\end{enumerate}



\item The next step is to go to level $q - 1$. For any two sibling
  cells denote $B^{q}_{\tt{left}}$ and $B^{q}_{\tt{right}}$ at level $q$ denote
  $C^{q-1}_{\tilde k}$ as the collection of the scaling functions from
  both cells, for some index $\tilde k$.


\item Let $q: = q - 1$. If $B^{q}_{k} \in \mcB^{q}$ is a leaf cell
  then repeat steps (b) to (d). However, if $B^{q}_{k} \in \mcB^{q}$
    is not a leaf cell, then repeat steps (b) to (d), but replace the
    leaf unit vectors with the scaling vectors contained in $C^{q}_k$
    with $C^{q-1}_{\tilde k}$.


  \item When $q = -1$ is reached stop.
  %repeat steps (b) to (d), but replace
  %$p$ with $p$, e.g. $\mcP^{p}(\bbS)$ with
  %$\mcP^{p}(\bbS)$. The ML basis vectors will span the space
  %$W_{-1}(\bbS) : =\mcP^{p}(\bbS) \backslash \mcP^{p}(\bbS)$.


\end{enumerate}

When the algorithm stops a series orthogonal subspaces
$V_{0}(\bbS), W_{0}(\mathbb{S}),\dots, W_{t}(\mathbb{S})$ (and their corresponding
basis vectors) are obtained. These subspaces are orthogonal to
$V_{0}(\mathbb{S}) : = span \{ \phi_{1}^{0}, \dots, \phi_{p}^{0}
\}$. Note that the orthonormal basis vectors of $V_{0}(\mathbb{S})$
also span the space $\mcP^{p}(\mathbb{S})$.
\begin{remark}
Following Lemma 2 in \cite{Castrillon2013} it can be shown that
\[
\R^{N} = \mcP^{p}(\mathbb{S}) \oplus
%W_{-1}(\mathbb{S}) \oplus
W_{0}(\mathbb{S}) 
\oplus W_{1}(\mathbb{S})
\oplus \dots \oplus W_{t}(\mathbb{S}),
\]
%where $W_{-1}(\bbS) : =\mcP^{\tilde p}(\bbS) \backslash
%\mcP^{p}(\bbS)$.
Also, it can then be shown that at most $\mcO(Nt)$
computational steps are needed to construct the multilevel basis of
$\R^{N}$.
\end{remark}

From the basis vectors of the subspaces $\mcP^{p}(\mathbb{S})^{\perp}
= \cup_{i=0}^{t} W_{i}(\mathbb{S})$ an orthogonal projection matrix
$\bW:\R^{N} \rightarrow (\mcP^{p}(\mathbb{S}))^{\perp}$ can be built.
The dimensions of $\bW$ is $(N - p) \times N$ since the total number
of orthonormal vectors that span $\mcP^{p}(\mathbb{S})$ is
$p$. Conversely, the total number of orthonormal vectors that span
$\mcP^{p}(\mathbb{S})^{\perp}$ is $N-p$.

Let $\bL$ be a matrix where each row is an orthonormal basis vector of
$\mcP^{p}(\mathbb{S})$. For $i = 0,\dots,t$ let $\bW_i$ be a matrix
where each row is a basis vector of the space $W_i(\mathbb{S})$. The
matrix $\bW \in \mathbb{R}^{(N - p) \times N}$ can now be formed,
where $\bW := \left[ \bW_t\T, \dots, \bW_0\T \right] \T$.

Following a similar approach to Lemma 2.11 in \cite{Castrillon2013} it
can be shown that:
\begin{enumerate}[a)]
\item 
The matrix $\bP := \left[
\begin{array}{c}
\bW \\
\bL
\end{array}
\right ]$ is orthonormal, i.e., $\bP\bP\T= \bI$.

\item Any vector $\bv
\in \R^{n}$ can be written as $\bv = \bL\T\bv_{L} + \bW\T\bv_{\bW}$
where $\bv_{L} \in \R^{p} $ and $\bv_{\bW} \in \R^{N-p}$ are unique.

\end{enumerate}





The following useful lemmas are proved:
\begin{lemma} Assuming that $n_0 < 2p$,
for any level $q=0,\dots,t$ there is at most $p2^{q}$ multilevel
basis vectors.
%For level $q = -1$ there is at most $p -
%\tilde p$ multilevel vectors.
\label{MultilevelREML:lemma1}
\end{lemma}
\begin{proof}
Starting at the finest level $t$, for each cell $B^{t}_k \in \mcB^{t}$
there is at most $p$ multilevel vectors.  Since there is at most
$2^t$ cells then there is at most $2^{t} p$ multilevel vectors.

Now, for each pair of left and right (siblings) cells at level $t$ the
parent cell at level $t-1$ will have at most $2 p$ scaling
functions. Thus at most $p$ multilevel vectors and $p$ scaling
vectors are obtained that are to be used for the next level. Now, the
rest of the cells at level $t$ are leafs and will have at most $p$
multilevel vectors and $p$ scaling vectors that are to be used for
the next level. Since there is at most $2^{t-1}$ cells at level $t-1$,
there is at most $2^{t-1} p$ multilevel vectors. Now, follow an
inductive argument until $q = 0$ and the proof is done.
\end{proof}



\begin{lemma} Assuming that $n_0 < 2p$ for any level $q = 0, \dots, t$
  any multilevel vector $\bpsi^{q}_m$ associated with a cell $B^{q}_k
  \in \mcB^{q}$ has at most $2^{t-q+1} p$ non zero entries.
\label{MultilevelREML:lemma2}
\end{lemma}
\begin{proof} For any leaf cell at the bottom of the tree (level $t$)
  there is at most $2 p$ observations.
  %thus the number of non zero entries of level $q$ multilevel
  %vectors is $2 p$.  Combining the left and right cells, the parent
  cell has at most $4 p$ observations, thus the associated multilevel
  vectors has $4p$ non zero entries. By induction at any level $l$ the
  number of nonzero entries is at most $2^{t-q+1} p$.  Now for any
  leaf cell at any other level $l < t$ the number of nonzero entries
  is at most $2 p$. Following an inductive argument the result is
  obtained.
\end{proof}


From Lemma \ref{MultilevelREML:lemma1} and \ref{MultilevelREML:lemma2}
it can be shown that the matrix $\bW$ contains at most $\mcO(Nt)$
non-zero entries and $\bL$ contains at most $\mcO(Np)$ non-zero
entries. Thus for any vector $\bv \in \R^{n}$ the matrix vector
products $\bW \bv$ and $\bL \bv$ are respectively calculated with at
most $\mcO(Nt)$ and $\mcO(Np)$ computational steps.


%% Multilevel Covariance Matrix --------------------------------------
%%
\section{Multilevel covariance matrix}
\label{MultilevelCovarianceMatrix}

The multilevel covariance matrix $\bC_{\bW}(\btheta)$ and sparse
version $\tilde \bC_{\bW}(\btheta)$ can be now constructed.  Recall from
the discussion in Section \ref{multilevelapproach} that
$\bC_{\bW}(\btheta):=\bW \bC(\btheta) \bW \T$. From the multilevel
basis construct in Section \ref{MultilevelREML} the following
operator is built: $\bW := \left[ \bW_t\T, \dots, \bW_0\T
  \right] \T$. Thus the covariance matrix $\bC(\btheta)$ is
transformed into $\bC_{\bW}(\btheta)$, where each of the blocks
$\bC^{i,j}_{\bW}(\btheta) = \bW_i \bC(\btheta) \bW_j \T$ are formed from
all the interactions of the MB vectors between levels $i$ and $j$, for
all $i,j = 0, \dots, t$. The structure of $\bC_{\bW}(\btheta)$ is shown
in Figure \ref{multilevelcov:fig1}.  Thus for any
$\bpsi^{i}_{\tilde{l}}$ and $\bpsi^{j}_{\tilde{k}}$ vectors there is a
unique entry of $\bC^{i,j}_{\bW}$ of the form
$(\bpsi^{i}_{\tilde{k}})\T \bC(\btheta) \bpsi^{j}_{\tilde{l}}$.

%The blocks $\bC_{\bW}^{i,j}$, where
%$i=-1$ or $j=-1$, correspond to the case where the accuracy term
%$\tilde{p} > p$.


In Section \ref{errorestimates} we show that far field entries of
$\bC_{\bW}(\btheta)$, i.e. $(\bpsi^{i}_{\tilde{k}})\T \bC(\btheta)
\bpsi^{j}_{\tilde{l}}$, decay sub-exponentially with respect to
$p(d,w)$ if there exists an analytic extension of the covariance
function on a well defined domain in $\bC^{d}$. Thus it is not
necessary to compute all the entries. We introduce a distance
criterion approach to produce a sparse matrix $\tilde
\bC_{\bW}(\btheta)$.

\subsection{Sparsification of multilevel covariance matrix}

A sparse version of the covariance matrix $\bC_{\bW}(\btheta)$ can be
built by using a level and distance dependent strategy:

\begin{enumerate}[i)]

\item Given a cell $B^{i}_{k}$ at level $i \geq 0$ identify the
  corresponding tree node value Tree.node and the tree depth
  Tree.currentdepth. Note that the

  Tree.currentdepth and the MB level
  $q$ are the same for $q = 0,\dots,t$.
  %However, for $q = -1$ the MB
  %is associated to the Tree.currentdepth = 0.

\item Let $\bbK \subset \bbS$ be all the observations nodes contained
  in the cell $B^{i}_{k}$.

\item Let $\tau_{i,j} \geq 0$ be the distance parameter given by the
  user corresponding to the level $i,j$ from the block
  $\bC^{i,j}_{\bW}(\btheta)$.

\item Let the Targetdepth be equal to the desired level of the tree.
%In the case that it is $-1$ then the Targetdepth is zero.

\end{enumerate}
    The objective now is to find all the cells at the Targetdepth that
    overlap a hyper rectangle which is extended from $B^{i}_{k}$.  For
    all observations $\bx \in B^{i}_{l}$ along each dimension $k = 1,
    \dots, d$ let $x^{min}_k := \min_{ x_k \in B^i_m} x_k$ and
    $x^{max}_k := \max_{ x_k \in B^i_m} x_k$.  Any cell that
    intersects the interval $[x^{min}_{k} - \tau_{i,j} ,x^{max} +
      \tau_{i,j}]$ is included. This is done by searching the tree
    from the root node. At each traversed node check that all the
    nodes $\bx \in \bbK$ satisfy the following rule: If
\[
\bx \cdot \mbox{Tree}.v + \tau_{i,j} \leq Tree.threshold.
\]
then search down the left tree. If 
\[
\bx \cdot \mbox{Tree}.v - \tau_{i,j} > Tree.threshold.
\]
the search down the right tree. Otherwise search both trees.
%If this is true for all $\bx \in \bbK$ then the search continues down
%the left tree.  If this is false for all $\bx \in \bbK$ then the
%search continues down the Right tree, otherwise both the left and
%right tree are searched.
The full search algorithm is described in Algorithms
\ref{MLCM:algorithm3}, \ref{MLCM:algorithm4}, \ref{MLCM:algorithm5}
and \ref{MLCM:algorithm5a}.


\begin{algorithm}[htp]
  \KwIn{Tree, $\bbK$, Targetdepth, $\tau_{i,j}$}
  \KwOut{Targetnodes}
\Begin{
    Targetnodes $\leftarrow \emptyset $
    Targetnodes $\leftarrow$ 
    LocalSearchTree(Tree, $\bbK$, Targetdepth, $\tau_{i,j}$, Targetnodes);
}
\caption{SearchTree function(Tree, $\bbK$, Targetdepth, $\tau_{i,j}$)}
\label{MLCM:algorithm3}
\end{algorithm}






\begin{algorithm}[htp]
  \KwIn{Tree, $\bbK$, Targetdepth, $\tau_{i,j}$, Targetnodes}
  \KwOut{Targetnodes}
\Begin{

\If {Targetdepth = Tree.currentdepth}{return
Targetnodes = Targetnodes $\cup$ Tree.node}

\If {Tree = leaf}{return}

LeftRule  =  ChooseLeftRule($\bbK$, Tree, $\tau_{i,j}$)\\
RightRule =  ChooseRightRule($\bbK$, Tree, $\tau_{i,j}$)\\



\uIf {LeftRule($\bx$)=true $\forall \bx \in \bbK$}
{Targetnodes $\leftarrow$ LocalSearchTree(Tree.LeftTree, 
  $\bbK$, Targetdepth, $\tau_{i,j}$, Targetnodes)}

\uElseIf{RightRule($\bx$)=true $\forall \bx \in \bbK$}
{Targetnodes $\leftarrow$ LocalSearchTree(Tree.RightTree, 
  $\bbK$, Targetdepth, $\tau_{i,j}$, Targetnodes)}

\Else{Targetnodes $\leftarrow$ LocalSearchTree(Tree.LeftTree, 
$\bbK$, Targetdepth, $\tau_{i,j}$, Targetnodes)\\
Targetnodes $\leftarrow$ LocalSearchTree(Tree.RightTree, 
$\bbK$, Targetdepth, $\tau_{i,j}$, Targetnodes)}

%\eIf {Rule($\bx$)=true $\forall \bx \in \bbK$}
%{Targetnodes $\leftarrow$ LocalSearchTree(Tree.LeftTree, 
%$\bbK$, Targetdepth, $\tau$, Targetnodes)}
%{Targetnodes $\leftarrow$ LocalSearchTree(Tree.LeftTree, 
%$\bbK$, Targetdepth, $\tau$, Targetnodes)
%Targetnodes $\leftarrow$ LocalSearchTree(Tree.RightTree, 
%$\bbK$, Targetdepth, $\tau$, Targetnodes)
%}

}
\caption{LocalSearchTree(Tree, $\bbK$, Targetdepth, $\tau_{i,j}$) function}
\label{MLCM:algorithm4}
\end{algorithm}


\begin{algorithm}[htp]

  \KwIn{ $\bbK$, Tree, $\tau_{i,j}$ }
  \KwOut{Rule}
\Begin{
Rule($\bx$) := $\bx \cdot \mbox{Tree}.v + \tau_{i,j} \leq $
        Tree.threshold
 }
\caption{ChooseLeftRule($\bbK$) function}
\label{MLCM:algorithm5}
\end{algorithm}


\begin{algorithm}[htp]

  \KwIn{ $\bbK$, Tree, $\tau_{i,j}$ }
  \KwOut{Rule}
\Begin{
Rule($\bx$) := $\bx \cdot \mbox{Tree}.v - \tau_{i,j} > $
        Tree.threshold
 }
\caption{ChooseRightRule($\bbK$) function}
\label{MLCM:algorithm5a}
\end{algorithm}

In Figure \ref{multilevelcov:fig2}(b) an example for searching local
neighborhood cells of randomly placed observations in $\R^{2}$ is
shown. The orange nodes correspond to the source cell. By choosing a
suitable value for $\tau_{i,j}$ the blue nodes in the immediate cell
neighborhood are found by using Algorithms \ref{MLCM:algorithm3},
\ref{MLCM:algorithm4}, \ref{MLCM:algorithm5} and
\ref{MLCM:algorithm5a}.

The sparse matrix blocks $\bC^{i,j}_{\bW}(\btheta)$ can be built from
all the cells that are obtained from SearchTree function of Algorithm
\ref{MLCM:algorithm5}. Compute all the entries of
$\bC^{i,j}_{\bW}(\btheta)$ that correspond to the interactions between
any two cells $B^{i}_k \in \mcB^{i}$ and $B^{j}_l \in \mcB^{j}$. In
Algorithm \ref{MLCM:algorithm6}) the construction of the sparse matrix
$\tilde \bC^{i,j}_{\bW}(\btheta)$ is shown.

\begin{remark}
Since the matrix $\tilde \bC_{\bW}(\btheta)$ is symmetric it is only
necessary to compute the blocks $\bC^{i,j}_{\bW}(\btheta)$ for $i = 1,
\dots, t$ and $j = i, \dots t$.
\end{remark}

\begin{algorithm}
  \KwIn{Tree, $i$, $j$, $\tau_{i,j}$, $\mcB^i$, $\mcB^j$,
    $\mcD^i$, $\mcD^i$, $\bC(\btheta)$} \KwOut{$\tilde
    \bC^{i,j}_{\bW}(\btheta)$}
  \Begin{
        Targetnodes $\leftarrow \emptyset$\\
        \For{$B^{i}_{m} \in \mcB^{i}$}
            {$\bbK \leftarrow B^{i}_{m}$\\
            \For{$B^{j}_{q} \leftarrow $
            SearchTree(Tree, $\bbK$, Targetdepth $(i)$, $\tau_{i,j}$, 
            Targetnodes)}
            {
              \For{$\psi^i_k \in D^{i}$}{
                \For{$\psi^j_l \in D^{j}$}{
                  Compute $(\bpsi^{i}_{k})\T \bC(\btheta) \bpsi^{j}_{l}$ 
                  in $\tilde \bC^{i,j}_{\bW}(\btheta)$
                }
              }
            }
            }
    }
\caption{Construction of sparse matrix $\tilde \bC^{i,j}_{\bW}(\btheta)$}
\label{MLCM:algorithm6}
\end{algorithm}

\begin{figure}
\begin{center}
\begin{tikzpicture} 
  \begin{scope}[scale = 0.5]
    [place/.style={circle,draw=blue!50,fill=blue!20,thick,
      inner sep=0pt,minimum size=1.5mm}]
    %\draw[fill=red!5, step=16, thick] (0, 0) grid (16, 16);

    \filldraw[fill={rgb:red,143;green,188;blue,143},semitransparent, 
      thick] (0, 0) rectangle (16, 16);

    % \draw[blue, very thick] (0,0)rectangle (3,2);

    \draw[thin] (8,0) to (8,16);
    \draw[thin] (0,8) to (16,8);
    \draw[thin] (14,0) to (14,16);
    \draw[thin] (0,2) to (16,2);
    \draw[thin] (12,0) to (12,16);
    \draw[thin] (0,4) to (16,4);

    \node at (15,1) [] {${\bf G}_{\bW}$ };
    \node at (10,12) [] {$\bC^{t,t-1}_{\bW}(\btheta)$};
    \node at (4.5,6) [] {$\bC^{t-1,t}_{\bW}(\btheta)$};
    \node at (4.5,1) [] {$\bC^{0,t}_{\bW}(\btheta)$};
    \node at (10,6) [] {$\ddots$};
    \node at (13,1) [] {$\dots$};
    \node at (15,3) [] {$\vdots$};
    \node at (4.5,12) [] {$\bC^{t,t}_{\bW}(\btheta)$};

\end{scope}
\end{tikzpicture}
\end{center}
\caption{Multilevel covariance matrix where ${\bf G}_{\bW}
:=\bC^{0,0}_{\bW}(\btheta)$.}
\label{multilevelcov:fig1}
\end{figure}

\begin{figure*}[ht]
\begin{center}

  %\includegraphics[trim = 410 60 370 50, clip, width=4in,
  %  height=4in]{./figures/neighborhoodpattern.pdf}

 \begin{tikzpicture}[scale=0.59, every node/.style={scale=0.59}]
  \begin{scope} 
    [place/.style={circle,draw=blueish,fill=blueish,
        inner sep=0pt,minimum size=1.5mm},
      placegray/.style={circle,draw=gray!50,fill=gray!20,
        inner sep=0pt,minimum size=1.5mm},
        placenew/.style={circle,draw=darkorange!75,fill=darkorange!75,
      inner sep=0pt,minimum size=1.5mm}]
    \draw[step=8,gray,very thin] (0, 0) grid (8, 8);
    
    \node at (15,3.88) [] {\includegraphics[trim = 14.5cm 2cm 10cm 1.75cm,
        clip=true,
        height=8.43cm]{neighborhoodpattern.pdf}};
    
    \draw (2.2,5) to (2.2,8);
    \draw (0,5) to (2.2,5);
    \draw (6,4.15) to (8,4.15);
    \draw (6,0) to (6,4.15);

    \draw (4,0) to (4,8);
    \draw (0,5) to (4,5);
    
    \draw (0,2) to (4,2);
    \draw (0,5) to (4,5);
    \draw (4,4.15) to (8,4.15);
    \draw (6.75,4.15) to (6.75,8);
        
    \draw[dashed,gray] (6,4.15) to (6,8);
    \draw[dashed,gray] (0,2.3) to (6,2.3);
  
    \node at (0.5,7.5) [placenew] {};
    \node at (0.3,6.3) [placenew] {};
    
    \node at (2.5,5.5) [place] {};
    \node at (3.2,5.2) [place] {};

    \node at (5,6) [place] {};
    \node at (3.8,5.5) [place] {}; %
    \node at (3.8,6) [place] {};   %

    \node at (0.5,3.5) [place] {};
    \node at (1.5,2.5) [place] {};
    \node at (2.3,2.2) [place] {};
    
    \node at (1.3,0.3) [placegray] {};
    \node at (2.7,0.5) [placegray] {};
    \node at (2.2,1.4) [placegray] {};
    \node at (2.6,1.4) [placegray] {};
    \node at (4.2,3.5) [place] {}; %
    \node at (3.7,3.3) [place] {};

    \node at (6.5,4.3) [place] {}; %
    \node at (7.5,5) [placegray] {};

    \node at (4.3,2.3) [place] {};
    \node at (5.7,3.5) [place] {};
    \node at (6.2,3.4) [placegray] {};
    \node at (7.3,2.4) [placegray] {};


    \node at (7,7) [placegray] {};
    \node at (6,7.5) [place] {};
    \node at (7.5,7.5) [placegray] {};


    \node at (6.5,2.0) [placegray] {};
    \node at (0.5,7.0) [placenew] {};
    \node at (2.0,7.0) [placenew] {};
    \node at (5,3.75) [place] {}; %
    \node at (6,7.0) [place] {};
    \node at (7,2.0) [placegray] {}; %
    \node at (7.5,4.3) [placegray] {}; %

    \node at (4,8.75) [] {\Large $\tau_{i,j}$};
    \node at (-1,4.25) [] {\Large $\tau_{i,j}$};

    \draw[dashed,gray] (2,6.3) to (2,8);
    \draw[dashed,gray] (0,6.3) to (2,6.3);
    
    \coordinate (A) at (2,8);
    \coordinate (B) at (6,8);
    \coordinate (C) at (0,6.3);
    \coordinate (D) at (0,2.3);

\draw[dim={,10pt}]  (A) --  (B);
\draw[dim={,-15pt}]  (C) --  (D);

\node at (4,-0.65) {\Large (a)};
\node at (14.5,-0.65) {\Large (b)};
  \end{scope}
\end{tikzpicture} 


\end{center}



\caption{Neighborhood identification from source cell on a random
  kD-tree decomposition of observation locations in $\R^{2}$. (a)
  Cartoon example of axis wise distance criterion $\tau_{i,j}$ using
  Algorithms \ref{MLCM:algorithm3}, \ref{MLCM:algorithm4},
  \ref{MLCM:algorithm5} and \ref{MLCM:algorithm5a}. The orange
  observations knots correspond to the source cell. The blue knots
  correspond to all the target nodes. The gray knots are not included
  in the list of target nodes.  (b) Example of local neighborhood
  contained in the axis wise distance $\tau_{i,j}$.  The orange nodes
  are contained in the source cell. The blue nodes are are contained
  in the local neighborhood cells. The grey dots are all the
  observations that are not part of the source or local neighborhood
  cells.}
\label{multilevelcov:fig2}
\end{figure*}

\subsection{Computational cost of the multilevel
  matrix blocks of $\tilde{\bC}_{\bW}$}


The cost of computing the multilevel blocks $\tilde{\bC}^{i,j}_{\bW}$
will in general be $\mcO(N^2)$. However, for the special case that $d
= 2$ and $d = 3$ it is possible to use a fast summation method such as
the Kernel Independent Fast Multipole Method (KIFMM) by
\cite{ying2004} to compute the blocks more efficiently. To my
knowledge, there exists no equivalent fast summation method in higher
dimensions that works satisfactorily.

This KIFMM algorithm is flexible and efficient for computing the
matrix vector products $\bC(\btheta)\bx$ for a large class of kernel
functions, including the Mat\'{e}rn covariance function.  Given
$\tilde N$ sources and $\tilde M$ targets, experimental results show a
computational cost of about $\mcO(\tilde N + \tilde M)$, $\alpha
\approx 1$ with good accuracy ($\varepsilon_{FMM}$ between $10^{-6}$
to $10^{-8}$) with a slight degrade in the accuracy with increased
source nodes.


\begin{asum} Let $\bA(\btheta) \in \R^{\tilde M \times \tilde N}$ be a kernel
matrix formed from $\tilde N$ source observation nodes and $\tilde M$
target nodes in the space $\R^{d}$.  Suppose that there exists a fast
summation method that computes the matrix-vector products
$\bA(\btheta)\bx$ with $\varepsilon_{FMM}>0$ accuracy in $\mcO((\tilde
N + \tilde M)^{\alpha})$ computations, for some $\alpha \geq 1$ and
any $\bx \in \R^{d}$.
\end{asum}

%Given an octree multilevel tree domain decomposition in $\R^{3}$, as
%shown in \cite{Castrillon2013,Castrillon2015}, the authors described
%how to apply a Kernel Independent Fast Multipole Method (KIFMM) by
%\cite{ying2004} to compute all the blocks
%$\tilde{\bC}^{i,i}_{\bW}(\btheta) \in \R^{\tilde N \times \tilde N}$ for
%$i = 0,\dots,t$ in $\mcO(\tilde N(t+1)^2)$ computational steps to a
%fixed accuracy $\varepsilon_{FMM} > 0$.

For the kD-tree it is not possible to determine a-priori the sparsity
of the blocks $\tilde{\bC}^{i,j}_{\bW}(\btheta)$.
%given the level dependent distance parameter $\tau_{i,j}$.
However, for a given a value $\tau_{i,j} \geq 0$ by running Algorithm
\ref{MLCM:algorithm3} on every cell $B^{i}_k \in \mcB^{i}$, at level
$i$, with the Targetdepth corresponding for level $j$ it is possible
to determine the computational cost of constructing the sparse blocks
$\tilde{\bC}^{i,j}_{\bW}(\btheta)$ under the following assumption. Suppose
that maximum number of cells $B^{j}_k \in \mcB^{j}$ given by Algorithm
\ref{MLCM:algorithm3} is bounded by some $\gamma^{i,j} \in \bbN_+$.



\begin{prop} 
  The cost of computing each block $\tilde{\bC}^{i,j}_{\bW}(\btheta)$
for $i,j = 1,\dots,t$ by using a fast summation method with $1 \leq
\alpha \leq 2$ is bounded by
\[
\mcO(\gamma_{i,j} p 2^{i} (2^{t-j+1} p + 2^{t-i+1} p)^{\alpha} + 2p 2^{t}).
\]
\end{prop} 
\begin{proof} Let us look at the cost of computing all the 
interactions between any two cells $B^{i}_k \in \mcB^{i}$ and
$B^{j}_l \in \mcB^{j}$. Without loss of generality assume that $i
\leq j$. For the cell $B^{l}_k$ there is at most $p$
multilevel vectors and from Lemma \ref{MultilevelREML:lemma2}
$2^{t-i+1} p$ non zero entries. Similarly for $B^{j}_l$.  All
the interactions $(\bpsi^{i}_{\tilde{k}})\T \bC(\btheta)
\bpsi^{j}_{\tilde{l}}$ now have to be computed, where
$\bpsi^{i}_{\tilde{k}} \in B^{i}_k$ and $\bpsi^{j}_{\tilde{l}} \in
B^{j}_l$.

The term $\bC(\btheta) \bpsi^{j}_{\tilde{l}}$ is computed using a FMM
with $2^{t-j+1} p$ sources and $2^{t-i+1} p$ targets at a cost of
$\mcO($ $(2^{t-j+1} p + 2^{t-i+1} \tilde p)^{\alpha})$.  Since there
is at most $p$ multilevel vectors in $B^{i}_k$ and $B^{j}_l$ then the
cost for computing all the interactions $(\bpsi^{i}_{\tilde{k}})\T
\bC(\btheta) \bpsi^{j}_{\tilde{l}}$ is $\mcO(p(2^{t-j+1} p + 2^{t-i+1}
p)^{\alpha} + 2^{t-i+1}p)$.

Now, at any level $i$ there is at most $2^{i}$ cells, thus the result
follows.
\end{proof}



%% Multilevel Estimator and Predictor --------------------------------
%%

\section{Multilevel estimator and predictor}
\label{multilevelestimator}
The multilevel random projection tree can be exploited in such a way
to significantly reduce the computational burden and to further
increase the numerical stability of the estimation and prediction
steps. This is an extension of the multilevel estimator and predictor
formulated in \cite{Castrillon2015} to binary tree in higher
dimensions. The former is based on Oct-tree decompositions, thus
making it unsuitable for higher dimensional problems.

\subsection{Estimator}

The multilevel likelihood function, $l_{\bW}(\theta)$ (see equation
\eqref {Introduction:multilevelloglikelihood} ), has the clear
advantage of being decoupled from the vector $\bbeta$. Furthermore,
the multilevel covariance matrix $\bC_{\bW}(\btheta)$ will be more
numerically stable than $\bC(\btheta)$ thus making it easier to invert
and to compute the determinant.

However, it is not necessary to perform the MLE estimation on the full
covariance matrix $\bC_{\bW}(\btheta)$, instead construct a series of
multilevel likelihood functions $\tilde{\ell}^n_{\bW}(\btheta)$, $n =
0,\dots t$, by applying the partial transform $[\bW_t \T, \dots, \bW_n
  \T ]$ to the data $\bZ$ where
\begin{equation}
  \begin{split}
\tilde{\ell}^n_{\bW}(\btheta)
=-\frac{\tilde{N}}{2}\log(2\pi)-\frac{1}{2}\log
\det\{\tilde{\bC}_{\bW}^n(\btheta)\}
-\frac{1}{2}(\bZ^n_{\bW})\T\tilde{\bC}^n_{\bW}(\btheta)^{-1}\bZ^n_{\bW},
\end{split}
\label{Introduction:multilevelloglikelihoodreduced2}
\end{equation}
where $\bZ^n_{\bW} :=[\bW_t \T, \dots, \bW_i \T ] \T \bZ$,
$\tilde{N}$ is the length of $\bZ^n_{\bW}$,
$\tilde{\bC}^n_{\bW}(\btheta)$ is the $\tilde{N} \times \tilde{N}$
upper-left sub-matrix of $\tilde{\bC}_{\bW}(\btheta)$ and
$\bC^n_{\bW}(\btheta)$ is the $\tilde{N} \times \tilde{N}$ upper-left
sub-matrix of $\bC_{\bW}(\btheta)$.

A consequence of this approach is that the matrices
$\bC^n_{\bW}(\btheta)$, $n = -1, \dots, t$ will be increasingly more
numerically stable, thus easier to solve computationally, as shown in
the following theorem.
\begin{prop} 
  \label{Multilevelapproach:theo2}
Let $\kappa(A) \rightarrow \R$ be the condition number of the matrix
$A \in \R^{N \times N}$ then
\[
\kappa(\bC^{t}_{\bW}(\btheta)) \leq \kappa(\bC^{t-1}_{\bW}(\btheta)) \leq 
\dots
\leq
\kappa(\bC_{\bW}(\btheta)) \leq
\kappa(\bC(\btheta)).
\]
\end{prop} 
\begin{proof} A simple extension of the proof in 
Proposition \ref{Multilevelapproach:theo1}.
\end{proof}



\begin{remark} If $\bC(\btheta)$ is symmetric positive definite 
then for $n = 0, \dots, t$ the matrices $\bC^{n}_{\bW}(\btheta)$ are
symmetric positive definite. The proof is immediate.
\end{remark}

\begin{remark} If the matrix $\tilde \bC^{n}_{\bW}(\btheta)$ is close 
to $\bC^{n}_{\bW}(\btheta)$, for $n = 1,\dots,d$, in some matrix norm
sense, the condition number of $\tilde \bC^{n}_{\bW}(\btheta)$ will be
close to $\bC^{n}_{\bW}(\btheta)$. Full error bounds will be derived in
a future publication.
%In section \ref{errorestimates}, for a class of covariance functions,
%it can be shown than for sufficiently large $\tau$ and/or $w,a \in
%\bbN_{+}$ [with the index set $\Lambda(w)^{m,g}$ or $\tilde
%  \Lambda(w)^{m,g}$] $\tilde \bC^{n}_{\bW}(\btheta)$ will be close to
%$\bC^{n}_{\bW}(\btheta)$. Thus $\tilde{\bC}^n_{\bW}(\btheta)$ will be
%symmetric positive definite.
\end{remark}

\subsection{Predictor}
In this section we show how to construct a multilevel BLUP with a
well conditioned multilevel covariance matrix. It will be shown that
the multilevel predictor is \emph{exact}, i.e. the multilevel
predictor and the solution of the constrained predictor problem
(equations \eqref{GLSbeta} and \eqref{KrigBLUP}) are the same.
However, the multilevel form can be significantly easier problem to
solve numerically.

Consider the following system of equations
\begin{eqnarray}
\left( {{\begin{array}{*{20}c}
 \bC(\btheta) \hfill & \bX \hfill \\
 \bX\T \hfill & \0 \hfill \\
\end{array} }} \right)\left( {{\begin{array}{*{20}c}
 \hat \bgamma \hfill \\
 \hat \bbeta \hfill \\
\end{array} }} \right)=\left( {{\begin{array}{*{20}c}
 \bZ \hfill \\
 \0 \hfill \\
\end{array} }} \right).
\label{Kriging:problem}
\end{eqnarray}
From the argument given in \cite{Nielsen2002} it is not hard to show
that the solution of this problem leads to equation \eqref{GLSbeta}
and $\hat \bgamma(\btheta) = \bC^{-1}(\btheta)(\bZ - \bX \hat
\bbeta(\btheta))$. The BLUP can be evaluated as
\begin{equation}
  \hat Z(\bx_0)
  =\bk(\bx_0)\T\hat \bbeta(\btheta)+\bc(\btheta)\T
  \hat \bgamma(\btheta)
\label{Kriging}
\end{equation}
and the Mean Squared Error (MSE) at the target point $\bx_0$ is given by
%\begin{equation}
\[
1 + 
\tilde{\bu}
\T
(\bX \T 
\bC(\btheta)^{-1} \bX )^{-1}
\tilde{\bu}
-\bc(\btheta)\T\bC^{-1}(\btheta)\bc(\btheta)
\]
%\label{Kriging:MSE}
%\end{equation}
where $\tilde{\bu}\T := (\bX \bC^{-1}(\btheta)
\bc(\btheta) - \bk(\bx_0))$.

From \eqref{Kriging:problem} it is observed that $\bX\T \hat
\bgamma(\btheta) = \0$. This implies that $\hat{\bgamma} \in \R^{n}
\backslash \mcP^{p}(\mathbb{S})$ and can be uniquely rewritten as
$\hat{\bgamma} = \bW\T \bgamma_{\bW}$ for some $\bgamma_{\bW} \in
\R^{N-p}$. Now, rewrite $\bC(\btheta) \hat \bgamma + \bX \hat \bbeta
= \bZ$ as
\begin{equation}
\bC(\btheta) \bW\T \bgamma_{\bW} + \bX \hat \bbeta =
       \bZ.
\label{Kriging:eqn1}
\end{equation}
Now apply the matrix $\bW$ to equation \eqref{Kriging:eqn1} and obtain
$\bW \{\bC(\btheta) \bW\T \bgamma_{\bW} + \bX \hat \bbeta\} = \bW
\bZ.$ Since $\bW \bX = \0$ then
\[
\bC_{\bW}(\btheta)
\bgamma_{\bW} = \bZ_{\bW}.
\]
A simple preconditioner $\bP_{\bW}$ can be formed from the diagonal
entries of the matrix $\bC_{\bW}$ i.e. $\bP_{\bW} = diag(\bC_{\bW})$ leading
to the following system of equations
\[
\bP_{\bW}^{-1}\bC_{\bW}(\btheta)
\bgamma_{\bW} = \bP_{\bW}^{-1} \bZ_{\bW}.
\]
Note that in some cases $\bC_{\bW}(\btheta)$ will have very small
condition numbers. For this case we can set $\bP_{\bW}:= I$, i.e. no
preconditioner.

\begin{theorem}
If the covariance function $\phi:\Gamma_d \times \Gamma_d \rightarrow
\R$ is positive definite, then the matrix $\bP_{\bW}(\btheta)$ is always
symmetric positive definite.
\label{multilevelKriging:lemma2}
\end{theorem}
\begin{proof}
Immediate.
\end{proof}


The vector $\hat \bgamma$ can be obtained by applying the inverse
transform $\bW\T$ i.e.  $\hat \bgamma = \bW\T \bgamma_{\bW}$.  From
\eqref{Kriging:problem} the GLS $\hat \bbeta$ can now be computed as a
least squares, i.e.  $\hat{\bbeta} = (\bX\T \bX)^{-1}\bX \T (\bZ -
\bC(\btheta)\hat{\bgamma})$.


\begin{remark}
  Notice that to solve the GLS estimate $\hat \bbeta$ it is not
  necessary to compute the full GLS of equation \eqref{GLSbeta}, but a
  least squares is all that is required. This is in contrast to the
  GLS estimate of equation \eqref{GLSbeta} where if an iterative
  method is used the covariance matrix $\bC(\btheta)$ has to be
  inverted for each of the columns of $\bX$ i.e. $p$ times.
  \end{remark}


\section{Numerical computation of multilevel estimator and predictor}
\label{numericalcomputation}


\subsection{Estimator: Computation of 
$\log{\det\{ \tilde{\bC}^n_{\bW}\}}$ and $(\bZ^n_{\bW})\T
  (\tilde{\bC}^n_{\bW})^{-1}\bZ^n_{\bW}$}

An approach to computing the determinant of
$\tilde{\bC}^n_{\bW}(\btheta)$ is to apply a sparse Cholesky
factorization technique such that $\bG\bG\T =
\tilde{\bC}^n_{\bW}(\btheta)$, where $\bG$ is a lower triangular
matrix. Notice that the eigenvalues of $\bG$ are located on the
diagonal. This leads to $\log \det \{\tilde{\bC}^n_{\bW}(\btheta)\} =
2 \sum_{i = 1}^{\tilde{N}} \log{\bG_{ii}}$.

The direct application of the sparse Cholesky algorithm can leads to
significant fill-in of the factorization matrix $\bG$. To alleviate
this problem it is typical to use matrix reordering techniques. In
particular, the fill-in are reduced by using the sparse Cholesky
factorization \emph{chol} from the Suite Sparse 4.2.1 package
(\cite{Chen2008,Davis2009,Davis2005,Davis2001,Davis1999}) coupled with
Nested Dissection (NESDIS) function package.

In practice, this approach leads to a significant reduction of
fill-in. To my knowledge a theoretical worse case complexity bounded
exists for $d = 2$ or $d = 3$ dimensions (see \cite{Castrillon2015}).

Two choices for the computation of $(\bZ^n_{\bW}) \T \tilde
\bC^n_{\bW}(\btheta)^{-1}$  $\tilde \bZ^n_{\bW}$ are open to us: i) a
Cholesky factorization of $\tilde{\bC}^n_{\bW}(\btheta)$, or ii) a
Preconditioned Conjugate Gradient (PCG).  The PCG choice requires
significantly less memory and allows more control of the error.
However, the sparse Cholesky factorization of
$\tilde{\bC}^n_{\bW}(\btheta)$ has already been used to compute the
determinant. Thus we can use the same factors to compute $ (\tilde
\bZ_{\bW}^n) \T\tilde{\bC}^n_{\bW}(\btheta)^{-1} \tilde \bZ^n_{\bW}$.
The PCG avenue will be explored in more detail in Section
\ref{comppred}.




%% %%%%%%%%%%%%%%%%%%%%%%%%%%%%%%%%%%%%%%%%%%%%%%%%%%%%%%%%%%%%%%%%%%%%%%%%
%\subsection{Predictor: Computation of $\bgamma_{\bW}$ and $\hat{\bbeta}(\btheta)$}

\subsection{Predictor computation}
\label{comppred}

For the predictor stage a different approach is used. Instead of
inverting the sparse matrix $\tilde \bC_{\bW}(\btheta)$ a
Preconditioned Conjugate Gradient (PCG) method is employed to compute
$\hat \bgamma_{\bW} = \bC_{\bW}(\btheta)^{-1} \bZ_{\bW}$.

Recall that $\bC_{\bW} = \bW \bC(\btheta) \bW \T$, $\hat \bgamma_{\bW} =
\bW \hat \bgamma$ and $\bZ_{\bW} = \bW \bZ$. Thus the matrix vector
products $\bC_{\bW}(\btheta) \bgamma_{\bW}^n$ in the PCG iteration are computed
within three steps:
\[
\bgamma_{\bW}^n \xrightarrow[(1)]{\bW \T \bgamma_{\bW}^n} 
\ba_n \xrightarrow[(2)]{\bC(\btheta) \ba_n}
\bb_n \xrightarrow[(3)]{\bW \bb_n}
\bC_{\bW}(\btheta) \bgamma_{\bW}^n 
\]
where $\bgamma_{\bW}^0$ is the initial guess and $\bgamma_{\bW}^n$ is the $n^{th}$
iteration of the PCG.
\begin{enumerate}[(1)]

\item Transformation from multilevel representation to single
  level. This is done in at most $\mcO(Nt)$ steps.

\item Perform matrix vector product using a summation method. For $d =
  2,3$ a KIFMM is used to compute the matrix vector products with
  $\alpha \approx 1$. For $d > 3$ to my knowledge there is no reliable
  fast summation method.

\item Convert back to multilevel representation.

\end{enumerate}

The matrix-vector products $\bC_{\bW}(\btheta) \bgamma_{\bW}^n$, where
$\bgamma_{\bW}^n \in \R^{N-p}$, are computed in $\mcO(N^{\alpha} + 2
Nt)$ computational steps to a fixed accuracy $\varepsilon_{FMM} > 0$.
Note that $\alpha \geq 1$ is dependent on the efficiency of the fast
summation method. The total computational cost is $\mcO(kN^{\alpha} +
2Nt)$, where $k$ is the number of iterations needed to solve
$\bP^{-1}_{\bW} \bC_{\bW}(\btheta) \bar{\bgamma}_{\bW} (\btheta) =
\bP^{-1}_{\bW} \bar{\bZ}_{\bW}$ to a predetermined accuracy
$\varepsilon_{PCG} > 0$.

\begin{remark}
The introduction of a preconditioner can degrade the accuracy for
computing $\hat \bgamma_{\bW} = \bC_{\bW}(\btheta)^{-1} \bZ_{\bW}$
with the PCG method. The residual accuracy $\varepsilon_{PCG}$ of the
PCG iteration has to be set such that the residual of
the\emph{unpreconditioned} system $\|\bC_{\bW}(\btheta) \bgamma_{\bW}
(\btheta) - \bZ_{\bW}\|_{l^2} < \varepsilon$ for a user given
tolerance $\varepsilon > 0$.
\end{remark}

Now compute $\hat \bgamma = \bW\T \hat \bgamma_{\bW}$ and $\hat{\bbeta} =
(\bX\T \bX)^{-1}\bX \T (\bZ - \bC(\btheta)\hat{\bgamma})$ in at most
$\mcO(N^{\alpha} + Np + p^{3})$ computational steps. The matrix vector
product $\bc(\btheta)\T \hat{\bgamma}(\btheta)$ is computed in
$\mcO(N)$ steps.  Finally, the total cost for computing the estimate
$\hat{\bZ}(\bx_0)$ from \eqref{Kriging} is $\mcO(p^{3} + (k +
1)N^{\alpha} + 2Nt)$.

%% Error Estimates ----------------------------------------------------
%%
\section{Multilevel covariance matrix decay}
\label{errorestimates}

We derive decay estimates of the multilevel covariance matrix. It can
be shown that most of the coefficients are small and thus it is not
necessary to compute all of them. The final objective is to build a
posteriori error estimates for $\bx_{\bW} =
\bC^{n}_{\bW}(\btheta)^{-1}\bZ^n_{\bW}$ and $\log
\bC^{n}_{\bW}(\btheta)$ that are needed for solving the multilevel
estimator MLE. However, the full analysis is extensive and will be
completed in a future publication. As a first step we show the decay
of the multievel covariance matrix. Note that this is not trivial and
uses the results derived in the Appendix. We recommend to first read
the appendix since part of the notation used in this section is
defined there.


%Consider the full solution $\bx_{\bW} =
%\bC^{n}_{\bW}(\btheta)^{-1}\bZ^n_{\bW}$ and sparse solution $\tilde
%\bx_{\bW} = \tilde \bC^{n}_{\bW}(\btheta)^{-1}\bZ^n_{\bW}$ for $n = 0,
%\dots, t$, then the error can be bounded as
%\begin{equation}
%\|\bx_{\bW} - \tilde \bx_{\bW}\|_{l^2} 
%\leq 
%\| \bC^{-1}_{\bW}(\btheta) - \tilde \bC^{-1}_{\bW}(\btheta) \|_2 
%\|\bZ_{\bW}\|_{l^2}.
%\label{errorestimates:eqn1}
%\end{equation}
%The ultimate goal is to derive a full a posteriori error estimate of
%$\|\bx_{\bW} - \tilde \bx_{\bW}\|_{l^2}$ with respect to the distance
%criterion $\tau_{i,j}$.  In this section estimates of the decay of the
%multilevel matrix $\bC_{\bW}(\btheta)$ are obtained. These will be
%critical to derive a full a posteriori error scheme.




The decay of the coefficients of the matrix $\bC_{\bW}(\btheta)$ will
depend directly on the choice of the multivariate index set
$\mcQ^{d}_{w}$ and the analytic regularity of the covariance
function. In general, the Mat\'{e}rn covariance function will be
analytic except for a derivative discontinuity at the
origin. However, with the application of the distance criterion a
minimal distance can be guaranteed and the origin will be avoided all
together.

In the following theorem, without loss of generality, it is assumed
that the covariance function domain is $\Gamma^{d} \times \Gamma^{d}$
for any two cells $B^{i}_{m} \in \mcB^{i}$ and $B^{j}_{q} \in
\mcB^{j}$. We will show later that this can be achieved through a
linear pullback.

\begin{theorem} Suppose that $0< \delta < 1$, $\hat
  \sigma := \sigma (1 - \delta)$, and $\phi(\bx,\by;\btheta) \in
  C^{0}(\Gamma^{d} \times \Gamma^{d};\R)$ can be analytically extended
  on $\mcE^{d}_{\sigma} \times \mcE^{d}_{\sigma}$ and is bounded by
  $\tilde M(\phi)$. Let $\mcP^{ p}(\mathbb{S})^{\perp}$ be the
  subspace in $\R^{N}$ generated by the index set $\mcQ^{d}_{w}$ for
  some $w \in \bbN_{+}$. For $i,j = 0,\dots,t$ consider any
  multilevel vector $\bpsi^{i}_m \in \mcP^{p}(\mathbb{S})^{\perp}$,
  with $n_m$ non-zero entries, from the cell $B^{i}_{m} \in \mcB^{i}$
  and any multilevel vector $\bpsi^{j}_{q} \in \mcP^{
    p}(\mathbb{S})^{\perp}$, with $n_q$ non-zero entries, from the
  cell $B^{j}_{q} \in \mcB^{j}$. If $p(d,w) \geq \left(\frac{2
    d}{\kappa(d)}\right)^{d}$ then
\[
\begin{split}
\left|
\sum_{r = 1}^{N} 
\sum_{h = 1}^{N} 
\phi(\bx_r,\by_h; \btheta) 
\bpsi^i_m[h] \bpsi^j_q[r] 
\right|
&\leq
\sqrt{n_mn_q}
\left( \frac{
  C(\tilde M,\sigma)^{d} e^{d - \sigma(1 - \delta) + 1} \hat \sigma d
}
 {
   (\sigma \delta)^{d}} \right)^2
 \\
 &
 \left( \frac{e^{\hat \sigma}}{1 - e^{-\hat \sigma}} \right)^{2d}
 \exp \left(-\frac{2d}{e} \hat \sigma  p^{\frac{1}{d}}
 \right) p^{2\left(\frac{d-1}{d}\right)}.
\end{split}
\]
\label{errorestimates:theorem1}
\end{theorem}
% Proof
\begin{proof} 
We first have that
\[
\begin{split}
\sum_{k = 1}^{N} 
\sum_{l = 1}^{N} 
\phi(\bx_k,\by_l; \btheta) 
\bpsi^i_m[k] \bpsi^j_q[l] 
&=
\sum_{k = 1}^{N} 
\sum_{l = 1}^{N}
\lim_{g \rightarrow \infty}
(\bI^d_{g}
\otimes
\bI^d_{g})[
\phi(\bx_k,\by_l; \btheta)] 
\bpsi^i_m[k] \bpsi^j_q[l] \\
&=
\sum_{k = 1}^{N} 
\sum_{l = 1}^{N}
(I_d - \mcS_{w,d}) 
\otimes
(I_d - \mcS_{w,d}) \\
&[\phi(\bx_k,\by_l; \btheta)] 
\bpsi^i_m[k] \bpsi^j_q[l].
\end{split}
\]
The last equality follows from $\bpsi^{i}_m, \bpsi^{j}_{q} \in \mcP^{
  p}(\mathbb{S})^{\perp}$. We now have that
\[
\begin{split}
& \sum_{k = 1}^{N} 
\sum_{l = 1}^{N}
\|(I_d - \mcS_{w,d}) 
\otimes
(I_d - \mcS_{w,d})
[\phi(\bx_k,\by_l; \btheta)] \|_{L^{\infty}_{\rho}(\Gamma^d)}
|\bpsi^i_m[k]| |\bpsi^j_q[l]|
\\
&\leq 
\|(I_d - \mcS_{w,d}))[\phi(\bx_k,\by_l; \btheta)] \|^{2}_{L^{\infty}_{\rho}(\Gamma^d)}
\sum_{k = 1}^{N} 
\sum_{l = 1}^{N}
|\bpsi^i_m[k]| |\bpsi^j_q[l]|.
\end{split}
\]
Since $\bpsi^i_m$ and $\bpsi^j_q$ are orthonormal then
\[
\begin{split}
\sum_{k = 1}^{N} 
\sum_{l = 1}^{N}
|\bpsi^i_m[k]| |\bpsi^j_q[l]|
&\leq \sqrt{n_m n_q}
\|\bpsi^i_m[k]\|_{l^2} \|\bpsi^j_q[l]\|_{l^2} =
\sqrt{n_m n_q}.
\end{split}
\]
From Lemma \ref{interpolation:lemma1} the result follows.
\end{proof}



\begin{remark} Recall that the restriction $p(d,w) \geq \left(\frac{2
  d}{\kappa(d)}\right)^{d}$ is not strict and can be relaxed such that
  sub-exponential convergence is still obtained. See Remark
  \ref{interpolation:remark1}.
\end{remark}

\begin{remark}
  The decay of the coefficients of $\bC^{i,j}_{\bW}$ is sub-exponential
  with respect to $p$.  Even for a moderate magnitude for $\hat
  \sigma > 0$, $p > 0$ and $d \geq 1$ the entries of the
  multilevel matrix $\bC^{i,j}_{\bW}$ that do not correspond to the
  cells given by the distance criterion parameter $\tau_{i,j} \geq 0$
  will be close to zero.
  %Thus the matrix $\bC^{i,j}_{\bW}$ will be
  %highly sparse as the number of observations $N$ increases.
\end{remark}

Theorem \ref{errorestimates:theorem1} provides a mechanism to control
the decay of the coefficients of the multilevel covariance matrix
$\bC_{\bW}$. To this end, we are interested in the analytic extension
and the uniform bound $\tilde M(\phi) \leq \infty$ of the Mat\'{e}rn
covariance function
\[
\phi(r;\btheta):=\frac{1}{\Gamma(\nu)2^{\nu-1}} \left(
\sqrt{2\nu} r(\btheta) \right)^{\nu} K_{\nu} \left(
\sqrt{2\nu} r(\btheta) \right)
\]
on a subdomain in $\bbC^{d} \times \bbC^{d}$, where $r(\btheta) =
(\bx-\by)^{T}$ $\text{diag}(\btheta) (\bx - \by)
)^{\frac{1}{2}}$, $\btheta=[\theta_1, \dots, \theta_d] \in
\R^{d}_{+}$ are positive constants, $\text{diag}(\theta) \in \R^{d
  \times d}$ is a diagonal matrix with the vector $\btheta$ on the
diagonal, and $\bx, \by \in \R^{d}$.

The polynomial function is an entire function. However, the function
$K_{\nu}(\vartheta)$ and $\vartheta^{\frac{1}{2}}$ are analytic for
all $\vartheta \in \bbC$ except at the branch cut $(-\infty,0]$.  Thus
it is sufficient to check the analytic extension of $r = \|\bx -
\by\|_{l^{2}} = \Big( \sum_{k=1}^{d} \theta_{k} (x_k - y_k)^{2}
\Big)^{\frac{1}{2}}$.  Let $z \in \bbC$ be the complex extension of
$r \in \R$. More precisely, $z = \Big( \sum_{k=1}^{d} \theta_{k}
z_k^{2} \Big)^{\frac{1}{2}}$, where $z_k \in \bbC$ is the complex
extension of $(x_k - y_k)$.

Let $\vartheta = \sum_{k=1}^{d} \theta_{k} z_k^{2}$, then by taking
the appropriate branch $\Real z = r_{\vartheta}$ $\cos{(
  \theta_{\vartheta}/2)}$, where $r^2_{\vartheta} = (\Real
\vartheta)^2 + (\Imag \vartheta)^2$ and $\theta_{\vartheta} =
\tan^{-1} \frac{\Imag \vartheta}{\Real \vartheta}$. Due to the branch
cut at $(-\infty,0]$ we impose the restriction that $\Real \vartheta >
0$ as $x_k$ and $y_k$ are extended in the complex plane.  Consider any
two cells $B^{i}_{m} \in \mcB^i$ and $B^{j}_{q} \in \mcB^j$, at levels
$i$ and $j$ with the associated distance criterion constant
$\tau_{i,j}>0$. From Algorithms \ref{MLCM:algorithm3},
\ref{MLCM:algorithm4}, \ref{MLCM:algorithm5} \ref{MLCM:algorithm6},
for any observations $\bx^{*} \in B^{i}_{m}$ and $\by^{*} \in
B^{j}_{q}$ we have that $(x^*_k - y^*_k)^2 \geq \tau^2_{i,j}$ for $k =
1,\dots,d$.  For the rest of the discussion it is assumed that complex
extension is respect to each component $k = 1,\dots,d$ unless
otherwise specified.


Let $x^{min}_k := \min_{ x^*_k \in B^i_m} x^*_k$, $x^{max}_k := \max_{
  x^*_k \in B^i_m} x^*_k$, $y^{min}_k := \min_{ y^*_k \in B^i_m}
y^*_k$, $y^{max}_k := \max_{ y^*_k \in B^i_m} y^*_k$ and
$\alpha_k,\gamma_k \in [-1,1]$. Define the region $\mcX^{i}_{m} :=
[x^{min}_1,$ $x^{max}_1] \times \dots \times [x^{min}_d,x^{max}_d]$
and $\mcY^{j}_{q} := [y^{min}_1,y^{max}_1] \times \dots \times
[y^{min}_d,y^{max}_d]$.

%rescale the widths of each of the cells $B^{i}_{m}$ and $B^{j}_{q}$
%with respect to the domain $\Gamma$.

The next step is to redefine $\phi(\bx;\by;\btheta):\mcX^{i}_m \times
\mcY^{i}_{q} \rightarrow \R$ as
$\phi(\balpha,\bgamma;\btheta):\Gamma^d \times \Gamma^d \rightarrow
\R$ through a pullback. Let $x_k = \left(\frac{\alpha_k + 1}{2}
\right) a_k + b_k$ and $y_k = \left(\frac{\gamma_k + 1}{2} \right) c_k
+ d_k$, where $a_k = x^{max}_{k} - x^{min}_{k}$, $b_k = x^{min}_{k}$,
$c_k = y^{max}_{k} - y^{min}_{k}$ and $d_k = y^{min}_{k}$.


%The covariance function
%$\phi(\bx;\by;\btheta):\mcX^{i}_m \times \mcY^{i}_{q} \rightarrow \R$
%is reformulated as $\phi(\balpha,\bgamma;\btheta):\Gamma^d \times
%\Gamma^d \rightarrow \R$.

Extend $\alpha_k \rightarrow \alpha_k + v_k$ and $\gamma_k
\rightarrow \gamma_k + w_k$ where $v_k:= v^R_k + iv^I_k$, $w_k:= w^R_k
+ iw^I_k$, and $v^R_k,v^I_k,w^R_k,w^I_k \in \R$. Let $\tilde x_k$ be
the extension of $x_k$ in the complex plane and similarly for
$\tilde y_k$.


It follows that $\tilde x^R_k := \Real \tilde x_k = \frac{1}{2}
(\alpha_k + 1 + v^R_k)a_k + b_k$, $\tilde x^I_k = \Imag \tilde x_k =
\frac{v^I_k}{2} a_k$, $y^R_k := \Real \tilde y_k = \frac{1}{2}
(\gamma_k + 1 + w^R_k)c_k + d_k$, and $y^I_k := \Imag \tilde y_k =
\frac{w^I_k}{2} c_k$.  After some manipulation
\begin{equation}
\begin{split}
\Real z^2_k &= (\tilde x^R_k - \tilde y^R_k)^2
- (\tilde x^I_k - \tilde y^I_k)^2 
=
(x_k - y_k)^2 
+
\frac{1}{4}(v^R_k a_k - w^R_k c_k)
+ (x_k - y_k)(v^R_k a_k - w^R c_k)
\\
&
-\frac{1}{4}(a_kv^I_k - c_k w^I_k)^2.
\end{split}
\label{errorestimates:eqn2}
\end{equation}


Recall that $(x_k - y_k)^2 \geq \tau^2_{i,j}$ and suppose that there
is a positive constant $\delta_{k} > 0 $ such that
\begin{equation}
\delta_{k} \leq 
%-\frac{4\,\tau -\sqrt{32\,\tau ^2+8\,\tau +1}+1}{4\,\tau }
\frac{\sqrt{32\,\tau_{i,j} ^2+8\,\tau_{i,j} +1}-1 - 4\,\tau_{i,j}  }{4\,\tau_{i,j}}.
\label{errorestimates:eqn2a}
\end{equation}

Assume that $|v^R_k|\leq \tau_{i,j} \delta_{k} / a_{k}$, $|v^I_k|\leq
\tau_{i,j} \delta_{k}/a_{k}$, $|w^R_k|\leq \tau_{i,j} \delta_{k} /
c_{k}$, and $|w^I_k|\leq \tau_{i,j} \delta_{k} / c_{k}$. From
equations \eqref{errorestimates:eqn2} and \eqref{errorestimates:eqn2a}
it follows that
\begin{equation}
\Real z^2_k \geq \tau_{i,j}^2 (1 - 4 \delta_{k}^2) - \frac{\tau_{i,j}
  \delta_{k}}{2} > 0.
\label{errorestimates:eqn3}
\end{equation}
%\begin{equation}
%\Real z^2_k \geq \tau_{i,j,m,q}^2 (1 - \frac{9}{16}(a_k + c_k)) > 0
%\label{errorestimates:eqn3}
%\end{equation}
Furthermore,
\begin{equation}
\begin{split}
  &
  |\Real z^2_k| \leq 
(\max\{|y^{max}_k - x^{min}_{k}|,|x^{max}_k - y^{min}_{k}|\})^2
\\
&+ \frac{1}{2}\tau_{i,j} \delta_{k} 
+ 
\max\{|y^{max}_k - x^{min}_{k}|,|x^{max}_k - y^{min}_{k}|\} 
2\tau_{i.j} \delta_{k}
+ \tau^2_{i,j} \delta_{k}^2 
\leq
1 + \frac{5}{2}\tau_{i,j} \delta_{k} + \tau^2_{i,j} \delta_{k}^2. 
\end{split}
\label{errorestimates:eqn6}
\end{equation}
Similarly,
\begin{equation}
  |\Imag z^2_k| = |2(\tilde x^R_k - \tilde y^R_k)(x^I_k - y^I_k)|
  \leq
2 \tau_{i.j} \delta_{k} + 4 \tau^2_{i,j} \delta_{k}^2.
\label{errorestimates:eqn4}
\end{equation}

We now show how $\alpha_k$ and $\gamma_k$ can be extended into the
Bernstein ellipses $\mcE_{\sigma^{\alpha}_k}$ and
$\mcE_{\sigma^{\gamma}_k}$, for some $\sigma^{\alpha}_k > 0$ and
$\sigma^{\gamma}_k > 0$ such that $\Real z^2_k > 0$. Recall that
$|v^R_k|\leq \tau_{i,j} \delta_{k} / a_{k}$, $|v^I_k|\leq \tau_{i,j}
\delta_{k}/a_{k}$, $|w^R_k|\leq \tau_{i,j} \delta_{k} / c_{k}$, and
$|w^I_k|\leq \tau_{i,j} \delta_{k} \ c_{k}$.
%We restrict the length of the extension of $(x^{min}_k, x^{max}_k)$
%and $(y^{min}_k, y^{max}_k)$ by $\tau_{i,j}/2$
These restrictions form a region in $\bbC \times \bbC$ and a Bernstein
ellipse is embedded (See Figure \ref{analyticity:fig1}).  This is done
by solving the following equation: $\frac{e^{\sigma^{\alpha}_k} +
  e^{-\sigma^{\alpha}_k}}{2} = 1 + \frac{\tau_{i,j}
  \delta_{k}}{a_k}$. The unique solution is
\[
\sigma^{\alpha}_k = \cosh^{-1} \left(1 +
\frac{\tau_{i,j} \delta_{k}}{a_k}
\right)
\]
with $\sigma^{\alpha}_k > 0$. Following a
similar argument we have that
\[
\sigma^{\gamma}_k = \cosh^{-1} \left(1
+ \frac{\tau_{i,j} \delta_{k}}{c_k}
\right)
\]
with $\sigma^{\gamma}_k > 0$. Let $\mcE^{d}_{\alpha} :=
\prod_{k=1}^{d} \mcE_{\sigma^{\alpha}_k}$ and $\mcE^{d}_{\gamma} :=
\prod_{k=1}^{d} \mcE_{\sigma^{\gamma}_k}$.  It follows that
\begin{equation}
\begin{split}
  \Real \vartheta &\geq  
  \sum_{k=1}^{d} \theta_k
  \Real z^2_k 
  \geq
\sum_{k=1}^{d} \theta_k 
  \left(
  \tau_{i,j}^2 (1 - 4 \delta_{k}^2) - \frac{\tau_{i,j}
    \delta_{k}}{2} \right)
  > 0.
  \end{split}
\label{errorestimates:eqn8}
\end{equation}
Thus there exist an analytic extension of $\phi(r;\btheta):\Gamma^d
\times \Gamma^d \rightarrow \R$ on $\mcE^{d}_{\alpha} \times
\mcE^{d}_{\gamma}$.



\begin{figure*}[htb!]
\begin{center}
\begin{tikzpicture}
    \begin{scope}[font=\scriptsize]

    \draw [->] (-2.5, 0) -- (2.5, 0) node [above left] {$\Real $};
    \draw [->] (0,-1.5) -- (0,1.5) node [below right] {$\Imag$};
    \draw [-,dashed] (-2,-1.5) -- (-2,1.5);
    \draw [-,very thick] (-1,0) -- (1,0);
    \draw [-,dashed] (2,-1.5) -- (2,1.5);
    \draw (1,-3pt) -- (1,3pt) node [above] {$-1$};
    \draw (-1,-3pt) -- (-1,3pt) node [above] {$1$};


    \draw [-,dashed] (-2,1) -- (2,1);
    \draw [-,dashed] (-2,-1) -- (2,-1);
    
 
    \fill [opacity=0.2, pattern=north west lines, pattern color=red]
    (-2,-1) rectangle (-1.5,1);

    \fill [opacity=0.2, pattern=north west lines, pattern color=red]
    (1.5,-1) rectangle (2,1);
    \end{scope}
    
    \node [below right] at (-1.92,0) {$\frac{\tau_{i,j} \delta_{k}}{a_k}$};
    \node [below right] at ( 1.05,0) {$\frac{\tau_{i,j} \delta_{k}}{a_k}$};

    \node at (0,-1.8) {$(a)$};
    \node at (6,-1.8) {$(b)$};
    
    \begin{scope}[shift={(0,0)},font=\scriptsize]

      \filldraw[fill={rgb:red,143;green,188;blue,143},semitransparent]
      (6,0) ellipse (2 and 1);

    \draw [->] (3, 0) -- (9, 0) node [above left] {$\Real $};
    \draw [->] (6,-1.5) -- (6,1.5) node [below right] {$\Imag$};
    \draw (5,-3pt) -- (5,3pt)   node [above] {$1$};
    \draw (7,-3pt) -- (7,3pt) node [above] {$-1$};


    %\fill [opacity=0.2, pattern=north west lines, pattern color=red]
    %(-2,-1.5) rectangle (-1.5,1.5);
    \end{scope}
    
    \node [below right] at (7.50,1.25) {$\mcE_{\sigma^{\alpha}_k}$};
    \node [below right] at (7.25,0.05) {$\frac{e^{\sigma^{\alpha}_k}
                                       + e^{-\sigma^{\alpha}_k}}{2}$};
        
\end{tikzpicture}
\end{center}
\caption{(a) Region of Complex extension of $\alpha_k$.  (b) Embedding
  of Bernstein ellipse $\mcE_{\sigma^{\alpha}_k}$.}
\label{analyticity:fig1}
\end{figure*}

The next step is to bound the absolute value of the Mat\'{e}rn
covariance function $|\phi(z;\btheta)|$ in $\mcE^{d}_{\alpha} \times
\mcE^{d}_{\gamma}$. If $\nu > \frac{1}{2}$ and $\Real z>0$ then the
modified Bessel function of the second kind satisfies the following
identity
\[
\begin{split}
K_{\nu}(\sqrt{2 \nu}z) &= \frac{\sqrt{\pi} (\sqrt{2 \nu}z)^{\nu}}{2^\nu
  \Gamma(\nu + \frac{1}{2})} \\
&
\int_{1}^{\infty} (t^2 - 1)^{\nu -
  \frac{1}{2}} \exp{(-\sqrt{2 \nu}zt)}\, \text{d}t.
\end{split}
\]
It is not hard to show that for $\nu > \frac{1}{2}$ and $\Real z >0$,
we have that $|K_{\nu}(\sqrt{2 \nu}z)| \leq \frac{|\sqrt{2
    \nu}z|^{\nu}}{(\Real \sqrt{2 \nu}z)^{\nu}} K_{\nu}(\sqrt{2 \nu}
\Real z)$.

Note that $r_{\vartheta} \geq \Real \vartheta > 0$.  From equation
\eqref{errorestimates:eqn4} we have that $\Imag \vartheta =
\sum_{k=1}^{d} \theta_k \Imag z^2_k \leq \sum_{k=1}^{d} 2 \tau
\delta_{k} + 4 \tau^2 \delta_{k}^2$.  From equation
\eqref{errorestimates:eqn8}
\[
\begin{split}
  |\theta_{\vartheta}|
  &\leq
\xi(\btheta,\bdelta,\tau_{i,j}) := \tan^{-1}
\left(
\frac{
  \sum_{k=1}^{d} 2 \tau_{i,j}
\delta_{k} + 4 \tau^2_{i,j} \delta_{k}^2
}
{
\sum_{k=1}^{d} \theta_k 
  \left(
  \tau_{i,j}^2 (1 - 4 \delta_{k}^2) - \frac{\tau_{i,j}
    \delta_{k}}{2} \right)
}
\right)
< \frac{\pi}{2}.
\end{split}
\]
Since $K_{\nu}(\cdot)$ is strictly completely monotonic
\cite{Baricz2011} then
\begin{equation}
\begin{split}
  |K_{\nu}(\sqrt{2 \nu}\Real z)| &=
  |K_{\nu}\ (\sqrt{2 \nu}
  r_{\vartheta} \cos(\theta_{\vartheta}/2))
  | 
  \leq
  \Big| K_{\nu}\Big(\sqrt{\frac{\nu}{2}}
  \cos(\xi(\btheta,\bdelta,\tau)/2) \\
  &
  \sum_{k=1}^{d} \theta_k 
  \Big(
  \tau_{i,j}^2 (1 - 4 \delta_{k}^2) - \frac{\tau_{i,j}
    \delta_{k}}{2} \Big)
  \Big) \Big|.
\end{split}
\label{errorestimates:eqn5}
\end{equation}
From equations \eqref{errorestimates:eqn6}
\eqref{errorestimates:eqn4} 
\[
\begin{split}
|z_k|^{2} &\leq |\Real z^2_k| + |\Imag z^2_k|
\leq \mcR(\delta_k,\tau_{i,j})
:=
1 + \frac{9}{2}\tau_{i,j} \delta_{k} + 5 \tau^2_{i,j} \delta_{k}^2
\end{split}
\]
and therefore
\begin{equation}
\begin{split}
  |z|
  &\leq
\left|
\sum_{k=1}^{d} \theta_{k} z_k^{2} \right|^{\frac{1}{2}}
\leq
\left(
\sum_{k=1}^{d} \theta_{k} |z_k|^{2} \right)^{\frac{1}{2}}
\leq \left( \sum_{k=1}^{d} \theta_k \mcR(\delta_k,\tau_{i,j})
\right)^{\frac{1}{2}}.
\end{split}
\label{errorestimates:eqn7}
\end{equation}
By combining equations \eqref{errorestimates:eqn4},
\eqref{errorestimates:eqn8}, \eqref{errorestimates:eqn5}, and
\eqref{errorestimates:eqn7}, we have now proven the following Theorem.

\begin{theorem} For any two cells $B^{i}_{m}$ and $B^{j}_{q}$
with the associated distance criterion parameter $\tau_{i,j} \geq 0$
let $\phi(\balpha,\bgamma;$ $\btheta):\Gamma^d \times \Gamma^d
\rightarrow \R$ be the pullback of the Mat\'{e}rn covariance function
$\phi(\bx;\by;\btheta):\mcX^{i}_m \times \mcY^{j}_{q} \rightarrow
\R$. Then there exists an analytic extension of
$\phi(\balpha,\bgamma;\btheta):\Gamma^d \times \Gamma^d \rightarrow
\R$ on the polyellipse $ \mcE^{d}_{\alpha} \times \mcE^{d}_{\gamma} $
and
  \[
  |\phi(\cdot,\cdot;\btheta)| \leq
\frac{  \left( 2 \nu \sum_{k=1}^{d} \theta_k \mcR(\delta_k,\tau_{i,j})
\right)^{\frac{\nu}{2}}
|K_{\nu}
(\Xi(\btheta,\bdelta,\tau_{i,j}
)|}{
\Xi(\btheta,\bdelta,\tau_{i,j})^{\nu}
}
\]
on $\mcE^{d}_{\alpha} \times \mcE^{d}_{\gamma}$, where
\[
\begin{split}
  \Xi(\btheta,\bdelta,\tau_{i,j})
  :=
\Big| K_{\nu}\Big(\sqrt{\frac{\nu}{2}}
  \cos(\xi(\btheta,\bdelta,\tau_{i,j})/2)
\sum_{k=1}^{d} \theta_k 
  \Big(
  \tau_{i,j}^2 (1 - 4 \delta_{k}^2) - \frac{\tau_{i,j}
    \delta_{k}}{2} \Big)
  \Big) \Big|.
  \end{split}
\]


\end{theorem}

%% Numerical Results --------------------------------------------------
%%
\section{Numerical results}
\label{numericalresults}

The performance of the multilevel solver for estimation and
prediction formed from random datasets is tested. The results show
that the computational burden is significantly reduced while retaining
good accuracy. In particular, it is possible to now solve
ill-conditioned problems efficiently. The implementation of the code
is as follows:


\begin{enumerate}[i)]

\item {\bf Matlab, C/C++ and MKL:} The binary tree, multilevel basis
  construction, formation of the sparse matrix $\tilde \bC_{\bW}$,
  estimation and prediction components are written and executed on
  Matlab \cite{Matlab2016}. However, the computational bottlenecks are
  executed by C/C++ software packages, Intel MKL \cite{intelmkl}, and
  the highly optimized BLAS and LAPACK packages contained in
  MATLAB. The C/C++ interfaces to matlab are constructed as dynamic
  shared libraries.

  

\item {\bf Direct and fast summation:} The matlab code estimates the
  computational cost between the direct and fast summation methods and
  chooses the most efficient approach.  For the direct method a
  combination of Graphic Processing Unit (GPU) and MKL intel libraries
  are used. For the fast summation method the KIFMM ($d = 3$) c++ code
  is used.  The KIFMM is modified to include a Hermite interpolant
  approximation of the Mat\'{e}rn covariance function, which is
  implemented with the intel MKL package \cite{intelmkl} (see
  \cite{Castrillon2015} for details).


\item {\bf Dynamic shared libraries:} These are produced with the GNU
  gcc/g++ packages. These libraries implement the Hermite interpolant
  with the intel MKL package (about 10 times faster than Matlab
  Mat\'{e}rn interpolant) and link the MATLAB code to the KIFMM.

\item {\bf Cholesky and determinant computation:} The Suite Sparse
  4.2.1 package
  (\cite{Chen2008,Davis2009,Davis2005,Davis2001,Davis1999}) is used
  for the determinant computation of the sparse matrix $\tilde
  \bC_{\bW}(\btheta)$.

\end{enumerate}

The code is tested on a single CPU (4 core Intel i7-3770 CPU @
3.40GHz.), one Nvidia 970 GTX GPU, with Linux Ubuntu 18.04 and 32 GB
memory. In addition, the Boston University Shared Computing Cluster
was used to generate test data.  To test the effectiveness of the
Multilevel solver the following data sets are generated:
\begin{enumerate}[a)]


\item {\bf Random n-sphere data set:} The set of nested random
  observation $\bS_{0}^{d} \subset \dots \subset \bS_{9}^{d}$ vary
  from 1,000, 2000, 4000 to 256,000 knots generated on the n-sphere
  $\bS_{d-1} := \{\bx \in \R^{d}\,\,|\,\,\|\bx\|_{2} = 1 \}$.

\item {\bf Random hypercube data set:} The set of random observation
  locations $\bC_{0}^{d},$ $\dots, \bC_{10}^{d}$ vary from 1,000, 2000,
  4,000 to 512,000 knots generated on the hypercube $[-1,1]^{d}$ for
  $d$ dimensions.  The observations locations are also nested,
  i.e. $\bC_{0}^{d} \subset \dots \subset \bC_{10}^{d}$.
  
\item {\bf Normal test data set} The set of observations values
  $\bZ^d_{0}$, $\bZ^d_{1}$, \dots $\bZ^d_{5}$ are formed from the
  Gaussian random field model \eqref{Introduction:noisemodel} for
  1,000, 2,000, $\dots$ $256,000$ observation locations. The data set
  $\bZ^d_{n}$ is generated from the set of nodes $\bS^{d}_{n}$, with
  the covariance parameters $(\nu,\rho)$ and the corresponding set of
  monomials $\mcQ^d_w$. The Boston University Shared Computing Cluster
  was used to generate the normal test data.
    
%\item All the numerical test are done assuming the $\tilde p = p$.
  
\end{enumerate}


\begin{remark}
 All the timings for the numerical tests are given in wall clock
 times i.e. the actual time is needed to solve a problem. This is to
 distinguish from CPU time, which can be significantly smaller.
  \end{remark}


\subsection{Numerical stability and sparsity of the covariance multilevel
  matrix}

For many practical cases the covariance matrix $\bC(\btheta)$ becomes
increasingly ill-conditioned for the Mat\'{e}rn covariance function as
$\rho$, $\nu$ and the number of observations are increased. This leads
to instability of the numerical solver. It is now shown how effective
Theorem \ref{Multilevelapproach:theo1} becomes in practice.  In
Figure \ref{numericalresults:fig1} the condition number of the
multilevel covariance matrix $\bC_{\bW}(\btheta)$ is plotted with
respect to the cardinality $p(w,d)$ of $\mcQ^d_w$ for different $w$
levels. The multilevel covariance matrix $\bC_{\bW}(\btheta)$ is built
from the random cube $\bC^{d}_{4}$ or n-sphere $\bS^{d}_{4}$
observations. The covariance function is set to Mat\'{e}rn with $\nu =
1$ and $\rho = 1,10$.  As the plots confirm the covariance matrix
condition number significantly improves with increasing level
$w$. This is in contrast with the large condition numbers of the
original covariance matrix $\bC(\btheta)$.  This is consistent with
Theorem \ref{Multilevelapproach:theo1} and Corollary
\ref{Multilevelapproach:cor1}.


We now focus our attention of the sparsity of the covariance matrix
$\tilde \bC_{\bW}(\btheta)$. In Figure \ref{numericalresults:fig2}(a)
the magnitude of the multilevel covariance matrix
$\bC_{\bW}(\btheta)$ is plotted for $N = 8,000$ observations from the
the n-sphere $\bS^{3}_{3}$ with Mat\'{e}rn covariance parameters $\nu
= 0.5$ (exponential) and $\rho = 10$. Due to the large value of $\rho$
the overlap between the covariance function at the different locations
in $\bS^{3}_{3}$ is quite significant, thus leading to a dense
covariance matrix $\bC(\btheta)$ where the coefficients decay
slowly. This is in contrast to the large number of small entries for
$\bC_{\bW}(\btheta)$, as shown in the histogram in Figure
\ref{numericalresults:fig2}(b). Note that the histogram is in terms of
$\log_{10}$ of the absolute value of the entries of
$\bC_{\bW}(\btheta)$. From the histogram it is observed that almost
all the entries are more than 1000 smaller than the largest entries.
This numerical result is consistent with the sub-exponential decay
rates of Theorem \ref{errorestimates:theorem1}.


\begin{figure*}[htpb]
\begin{center}
\begin{tikzpicture}%[thick,scale=1, every node/.style={scale=1}]
  \node[inner sep=0pt] at (0,0)
  {
  \includegraphics[trim = 120 255 120 255,
    clip,width=4.4in,height=4in]{ConditionGraphsReduced.pdf}
  };
  \node[rotate = 90] at (-5.5,2.7) {$\kappa(\bC_{\bW}(\btheta))$};
  \node[rotate = 90] at (0,2.7) {$\kappa(\bC_{\bW}(\btheta))$};

  \node[rotate = 90] at (-5.5,-2.7) {$\kappa(\bC_{\bW}(\btheta))$};
  \node[rotate = 90] at (0,-2.7) {$\kappa(\bC_{\bW}(\btheta))$};

\node at (-2.6,5.3) {
\begin{tabular}{c}
\small $Cube, d = 5, \bC^5_4, \rho=1$ \\
\small $\kappa(\bC(\btheta))= 1.1 \times 10^{7}$
\end{tabular}
};

\node at (2.9,5.3) {
\begin{tabular}{c}
\small $Cube, d = 5, \bC^5_4, \rho=10$ \\
\small $\kappa(\bC(\btheta))= 2.2 \times 10^{9}$
\end{tabular}
};

  
\node at (-2.6,-0.15)
{
\begin{tabular}{c}
\small $Sphere, d = 5,  \bC^5_4, \rho=1$ \\
\small $\kappa(\bC(\btheta))= 2.6 \times 10^{7}$
\end{tabular}
};

\node at (2.9,-0.15)
{
\begin{tabular}{c}
\small $Sphere, d = 5, \bS^5_4, \rho=10$ \\
\small $\kappa(\bC(\btheta))=7.8 \times 10^{9}$
\end{tabular}
};
\node at (2.9,-5.0)
      {$p$
        };
\node at (-2.6,-5.0)
      {$p$
        };
\end{tikzpicture}
\end{center}
\caption{Condition number of the multilevel covariance matrix
  $\bC_{\bW}(\btheta)$ with respect to the size $p$ of the Total
  Degree (TD) polynomial space. The number of observations corresponds
  to 16,000 nodes generated on a hypercube or n-sphere of dimension $d
  = 5$. The covariance function is chosen to be Mat\'{e}rn with $\nu =
  1$ and $\rho=1,10$.  The condition number of the covariance matrix
  $\bC(\btheta)$ is placed on the top of each subplot for
  comparison. The MB is constructed from a kD-tree.  As expected, as
  $p$ increases with $w$ the condition number of $\bC_{\bW}(\btheta)$
  decreases significantly. This is consistent with Theorem
  \ref{Multilevelapproach:theo1} and Corollary
  \ref{Multilevelapproach:cor1}}
\label{numericalresults:fig1}
\end{figure*}

In Table \ref{numericalresults:table1} sparsity and construction wall
clock times of the sparse matrices $\tilde{\bC}^{i}_{\bW}(\btheta)$,
$i = t, t-1, \dots$, for various values of $i$ are shown.  The
polynomial space of the index set $\mcQ^d_w$ is restricted to TD on a
n-Sphere with $d = 10$ dimensions. The domain decomposition is formed
with a kD-tree. The level of the index set is set to $w = 7$, which
corresponds $p = 1001$. The covariance function is Mat\'{e}rn with
$\nu = 3/4$, $\rho = 3/4$. The distance criterion for each $(i,j)$
multilevel covariance matrix block is set to
\[
\tau_{i,j} := 2^{(t - i)/2}2^{(t - j)/2} \tau,
\]
for $i = 1, \dots, t$ and $j = 1 \dots, t$, where $\tau = 3 \times
10^{-6}$.

The first observation to notice is that all the sparse matrices
$\tilde{\bC}^{i}_{\bW}(\btheta)$, $i = t, t-1, \dots$ {\it are very
  well conditioned, thus numerically stable}. This is in contrast to
the original covariance matrices that are in general poorly
conditioned. The sparsity of $\tilde{\bC}^{i}_{\bW}(\btheta)$ and the
Cholesky factor $\bG$ are shown in columns 7 and 9. The construction
time $t_{con}$ of the $\tilde{\bC}^{i}_{\bW}(\btheta)$ is shown in
column 9. In column 5 $t_{ML}$ is the time required to build the
multilevel basis.  We observe that for large matrices the sparse
matrix $\tilde{\bC}^{i}_{\bW}(\btheta)$ are built efficiently.  It is
noted that the sparse matrices in Table \ref{numericalresults:table1}
are built with a direct summation method due to the dimensionality $d
= 10$.


\setlength{\tabcolsep}{6pt}

\begin{figure*}
\begin{center}
\psfrag{A}[c][t]{\small $\log_{10}(abs(\bC_{\bW}(\btheta)))$}
\begin{tabular}{c c}
\includegraphics[width=2in,height=2in]{MatrixDecay.pdf}
&
\hspace*{0mm}
\raisebox{-2.5mm}[0pt][0pt]{
\psfrag{Hist}[c][t]{\small Histogram of $\log_{10} abs(\bC_{\bW}(\btheta))$}
\psfrag{log10}[c][t]{\tiny $\log_{10} abs(\bC_{\bW}(\btheta))$ (100 bins)}
\includegraphics[width=2.1in,height=2.1in]{HistMatrixDecay.pdf}
} \\
    & \\
(a) & (b)
\end{tabular}
\end{center}
\caption{(a) Magnitude pattern and (b) histogram of $\log_{10}
  abs(\bC_{\bW}(\btheta))$ with 100 bins where $abs(\bC_{\bW}(\btheta))
  \in \R^{(N-p) \times (N-p)}$ is the magnitude of the entries of the
  matrix $\bC_{\bW}(\btheta)$.  The matrix $\bC_{\bW}(\btheta)$ is created
  with $d = 3$ dimensions, $N$ = 8,000 random locations on the sphere
  ($\bS^3_3$), and the Mat\'{e}rn covariance function with $\rho =
  10$, $\nu = 0.5$ (exponential), Total Degree index Set $\Lambda(w)$
  with $w = 4$, and $p = 35$. As observed from (a) and (b) most of
  entries of the matrix $\bC_{\bW}(\btheta)$ are very small.}
%and can thus be safely eliminated
%  without comprosing much accuracy.}
\label{numericalresults:fig2}
\end{figure*}




%% \begin{figure}[htpb]
%% \begin{center}
%% \psfrag{A-------------}[c][c]{\raisebox{0mm}[0pt][0pt]{\tiny $w = 3$}}
%% \psfrag{B-------------}[c][c]{\raisebox{0mm}[0pt][0pt]{\tiny $w = 4$}}
%% \psfrag{C-------------}[c][c]{\raisebox{0mm}[0pt][0pt]{\tiny $w = 5$}}
%% \psfrag{Error}[c][t]{\small Total degree log det relative error}
%% \psfrag{x}[c][t]{\tiny Sparsity}
%% \psfrag{y}[c][t]{\tiny $\frac{
%% |\log{det(\bC^n_{\bW})} - \log{det(\tilde \bC^n_{\bW})}|
%% }{
%% \log{det(\bC^n_{\bW})}}$}
%% \begin{tabular}{c c}
%% \includegraphics[width=3in,height=3in]{./figures/sparsitydecay.eps}
%% &
%% \psfrag{Error}[c][t]{\small Smolyak log det relative error}
%% \psfrag{A-------------}[c][c]{\raisebox{0mm}[0pt][0pt]{\tiny $w = 2$}}
%% \psfrag{B-------------}[c][c]{\raisebox{0mm}[0pt][0pt]{\tiny $w = 3$}}
%% \psfrag{Hist}[c][t]{\small Histogram of $\log_{10}|\bC_{\bW}(\btheta)|$}
%% \psfrag{log10}[c][t]{\tiny $\log_{10}|\bC_{\bW}(\btheta)|$ (100 bins)}
%% \includegraphics[width=3in,height=3in]{./figures/sparsitydecay2.eps} 
%% \\
%% (a) TD, $N$ = 16,000, $d = 5$, kD-tree & 
%% (b) SM, $N$ = 16,000, $d = 10$, kD-tree 
%% \end{tabular}
%% \end{center}
%% \caption{Relative log determinant error $\frac{| \log \det \tilde
%%     \bC_{\bW}(\btheta) - \log \det \bC_{\bW}(\btheta)| |}{|\log \det
%%     \bC_{\bW}(\btheta)|}$ with respect to the sparsity of $\tilde
%%   \bC^{n}_{\bW}$ from Random hyper-sphere data $\bS^{d}_{3}$ with $N$
%%   =16,000 observations, Mat\'{e}rn covariance parameters $\nu = 0.5$,
%%   $\rho = 10$ and binary kD-tree. The index sets $\Lambda(\omega)$ are
%%   chosen from (a) Total Degree and (b) Smolyak index sets.}
%% \label{numericalresults:fig3}
%% \end{figure}


\setlength{\tabcolsep}{7pt}
\begin{table*}[htpb]
  \caption{Sparsity test on the matrices $\tilde{\bC}^{i}_{\bW}$, $i =
    t, t-1, \dots$.  The polynomial space of the index set $\mcQ^d_w$
    is restricted to TD on a n-Sphere with $d = 10$ dimensions. The
    domain decomposition is formed from a kD-tree. The level of the
    index set is $w = 7$, which corresponds $p = 1001$. The kernel
    function is Mat\'{e}rn with $\nu = 3/4$, $\rho = 3/4$ and $\tau :=
    3 \times 10^{-6}$. The first column is the number of random
    n-Sphere nodes. The second is the maximum level of the kD tree and
    $i$ is the level of the sparse matrix $\tilde{\bC}^{i}_{\bW}$. The
    fourth column is the condition number of $\tilde{\bC}^{i}_{\bW}$,
    which is excellent.  The fifth column is the size of the matrix
    $\tilde{\bC}^{i}_{\bW}$.  The seventh column, $t_{ML}$, is the
    total time for the construction of the multilevel basis. The
    eighth column is the sparsity of $\tilde{\bC}^{i}_{\bW}$.  The
    nineth column, $t_{con}$ is the total time for the construction of
    the matrix $\tilde{\bC}^{i}_{\bW}$. The tenth column is the
    sparsity of the Cholesky factor $\bG$ (with nested dissection
    reordering) of the sparse matrix $\tilde{\bC}^{i}_{\bW}$. The last
    column is the total time to compute the Cholesky factor $\bG$.}
\begin{center}
\begin{tabular}{ r r r r c c r r c r c r}
\multicolumn{1}{c}{$N$} &
% \multicolumn{1}{c}{$d$} &
\multicolumn{1}{c}{$t$} &
\multicolumn{1}{c}{$i$} &
\multicolumn{1}{c}{$\kappa(\tilde \bC_{\bW}^{i})$} &
\multicolumn{1}{c}{Size} &
%\multicolumn{1}{c}{$\tau$} &  
\multicolumn{1}{c}{$t_{ML}$} &
\multicolumn{1}{c}{$nz$} &
\multicolumn{1}{c}{$t_{con}$} &
\multicolumn{1}{c}{$nz(\bG)$} &
\multicolumn{1}{c}{$t_{sol}$}
 \\ 
 \hline
32,000 & 4 & 4 &   5  & 15,984 &  46 &  6.3\% &   11 &  3.1\% &  1 \\
32,000 & 4 & 3 &   8  & 23,992 &  46 & 10.4\% &   30 &  5.2\% &  3 \\
32,000 & 4 & 2 &  13  & 27,996 &  46 & 15.6\% &   82 &  7.8\% &  7 \\
32,000 & 4 & 1 &  19  & 29,998 &  46 & 20.1\% &  190 & 10.4\% & 16 \\
32,000 & 4 & 0 &  23  & 30,999 &  46 & 25.7\% &  310 & 13.0\% & 17 \\
\hline
64,000 & 5 & 5 &   6  & 31,968 & 104 &  3.5\% &   21 & 1.8\%  &  3 \\
64,000 & 5 & 4 &  11  & 47,984 & 105 &  6.3\% &   90 & 3.1\%  & 12 \\
64,000 & 5 & 3 &  18  & 55,992 & 106 &  9.6\% &  270 & 5.0\%  & 18 \\
64,000 & 5 & 2 & 121  & 59,996 & 121 & 13.4\% & 624 & 6.7\%  & 34 \\
\hline
128,000 & 6 & 6 &  8  &  63,936 &  237 & 4.0 \% & 120  & 2.1 \% & 15 \\
128,000 & 6 & 5 & 17  &  95,968 &  237 & 5.5 \% & 378 & 6.7 \% & 140 \\
\end{tabular}
\vspace{5mm}
\\
\end{center}
\label{numericalresults:table1}
\end{table*}

\subsection{Estimation}

In this section estimation results are presented for the Mat\'{e}rn
covariance matrix on high dimensional n-Sphere random locations by
solving multilevel log-likelihood
\[
\hat{\btheta} : =
\argmax_{\btheta}
\tilde{\ell}^{i}_{\bW}(\bZ^{i,k,d}_{W};\btheta),
\]
where $\bZ^{i,k,d}_{W} := [\bW_t \T, \dots, \bW_i \T ] \T
\bZ^{d}_{k}$, for $i = t, t-1, \dots$. The observation data $\bZ^d_k$
is built from the n-Sphere $\bS^d_k$ for $d = 3,10$, $k = 6$ $(N =
64,000)$ and $k = 7$ $(N = 128,000)$. The covariance function is
Mat\'{e}rn for several values of $\nu$ and $\rho$. To test the
performance of the multilevel estimator, $M = 100$ realizations are
generated for each case.

The optimization problem of the log-likelihood function
\eqref{Introduction:multilevelloglikelihoodreduced2} is solved using
a fmincon iteration search for the estimates $\hat \nu$ and $\hat
\rho$ from the optimization toolbox in MATLAB \cite{Matlab2016}. The
tolerance level is set to $10^{-6}$.

In Table \ref{numericalresults:table2} the mean and standard deviation
of the Mat\'{e}rn covariance parameter estimates $\hat \nu$ and $\hat
\rho$ are presented. The mean estimate $\bbE_M [\hat{\nu}]$ refers to
the mean of $M$ estimates $\hat{\nu}$ for the $M$ realizations of the
stochastic model. Similarly, $std_M [\hat{\nu}]$ refers to the
standard deviation of the $M$ realizations. For case (a) ($d = 3$) the
error mean and std is $\approx 1\%$. For case (b) ($d = 10$) the error
of the mean increase to $\approx 10 \%$. In general, as $i$ is reduced
from $t$ there is a tendency of a drop in the standard deviation
$std_M [\hat{\nu}]$ of the estimator $\hat \nu$. However, there is
also a tendency for the accuracy of the mean to degrade somewhat,
except for (a) $N = 128,000$, $i = 12 \rightarrow 11$.
%This inconsistent
%behavior in accuracy might be due to the standard deviation being
%approximately the same magnitude as the sample mean.


\begin{table*}[htpb]
\caption{Estimation of parameters $\hat \nu$ and $\hat \rho$ with:
  Total Degree polynomial index set $\mcQ^d_w$, kD tree, and n-Sphere
  with for $d = 3$ and $d = 10$.  The observation data $\bZ^d_k$ are
  formed from the Mat\'{e}rn covariance function. The number of
  realizations of the Gaussian random field model is set to $M =
  100$. Several cases are tested and are given by the individual tables
  (a) and (b).  The first to fourth columns are self-explanatory. The
  fifth column is the mean error of $\hat \nu$ with $M$
  realization. The sixth column is the mean error of $\hat \rho$. The
  last two columns are the standard deviation of $M$ realizations of
  the parameters $\hat \nu$ and $\hat \rho$.}
\begin{center}
(a) TD, kD tree, n-Sphere, $d = 3$, $M = 100$, $\nu =
  3/4$, $\rho = 1/6$, $\tau = 5 \times 10^{-2}$
\begin{tabular}{ r r r r r r r r r r}
\multicolumn{1}{c}{$N$} & 
% \multicolumn{1}{c}{$d$} &
\multicolumn{1}{c}{$w$} &
\multicolumn{1}{c}{$t$} &
\multicolumn{1}{c}{$i$} & 
\multicolumn{1}{c}{$\bbE_M [\hat{\nu} - \nu]$} &
\multicolumn{1}{c}{$\bbE_M [\hat{\rho} - \rho]$} &
\multicolumn{1}{c}{$std_M [\hat{\nu}]$} &
\multicolumn{1}{c}{$std_M [\hat{\rho}]$} 
 \\ 
 \hline
64000 & 3 & 11 & 11 & -1.92e-04 &  4.52e-04 & 1.36e-02 & 8.17e-03 \\ 
64000 & 3 & 11 & 10 &  1.17e-03 & -5.90e-04 & 7.08e-03 & 4.04e-03 \\ 
%\hdashline
128000 & 3 & 12 & 12 & -2.51e-03 & 1.81e-03 & 8.54e-03 & 6.11e-03 \\ 
128000 & 3 & 12 & 11 & -6.90e-04 & 5.02e-04 & 4.17e-03 & 2.84e-03 \\ 
\end{tabular}
\\
\bigskip
(b) TD, kD tree, n-Sphere, $d = 10$, $M = 100$, $\nu =
  3/4$, $\rho = 3/4$, $\tau = 1 \times 10^{-5}$
\begin{tabular}{ r r r r r r r r r r}
\multicolumn{1}{c}{$N$} & 
% \multicolumn{1}{c}{$d$} &
\multicolumn{1}{c}{$w$} &
\multicolumn{1}{c}{$t$} &
\multicolumn{1}{c}{$i$} & 
\multicolumn{1}{c}{$\bbE_M [\hat{\nu} - \nu]$} &
\multicolumn{1}{c}{$\bbE_M [\hat{\rho} - \rho]$} &
\multicolumn{1}{c}{$std_M [\hat{\nu}]$} &
\multicolumn{1}{c}{$std_M [\hat{\rho}]$} 
 \\ 
 \hline
64000 & 4 & 5 & 5 &  8.70e-03 & -1.12e-02 & 1.55e-02 & 1.85e-02 \\ 
64000 & 4 & 5 & 4 & -9.31e-02 &  8.02e-02 & 1.67e-02 & 1.97e-02 \\
%\hdashline
128000 & 4 & 6 & 6 & -6.36e-03 & 5.51e-03 & 2.10e-02 & 1.72e-02 \\
128000 & 4 & 6 & 5 & -7.18e-02 & 6.27e-02 & 1.32e-02 & 1.46e-02 \\ 
\end{tabular}
\\
\bigskip
%% (c) TD, kD tree, n-Sphere, $d = 10$, $M = 100$, $\nu =
%%   1.25$, $\rho = 1$, $\tau = 10^{-7}$
%% \begin{tabular}{ r r r r r r r r r r}
%% \multicolumn{1}{c}{$N$} & 
%% % \multicolumn{1}{c}{$d$} &
%% \multicolumn{1}{c}{$w$} &
%% \multicolumn{1}{c}{$t$} &
%% \multicolumn{1}{c}{$i$} & 
%% \multicolumn{1}{c}{$\bbE_M [\hat{\nu} - \nu]$} &
%% \multicolumn{1}{c}{$\bbE_M [\hat{\rho} - \rho]$} &
%% \multicolumn{1}{c}{$std_M [\hat{\nu}]$} &
%% \multicolumn{1}{c}{$std_M [\hat{\rho}]$} 
%%  \\ 
%%  \hline
%%  64000 & 2 &  9 & 9 & -5.86e-03 & 5.21e-03 & 4.85e-02 & 3.15e-02 \\
%%  64000 & 2 &  9 & 8 & -3.37e-02 & 1.93e-02 & 3.50e-02 & 2.46e-02 \\ 
%%  64000 & 2 &  9 & 7 & -1.19e-01 & 6.92e-02 & 2.88e-02 & 2.51e-02 \\ 
%% %\hdashline
%% 128000 & 2 & 10 & 10 & -2.70e-03 & 2.74e-03 & 3.76e-02 & 2.44e-02 \\ 
%% 128000 & 2 & 10 &  9 & -2.00e-02 & 1.20e-02 & 2.47e-02 & 1.80e-02 \\ 
%% 128000 & 2 & 10 &  8 & -8.40e-02 & 5.03e-02 & 2.21e-02 & 1.89e-02 \\ 
%% \end{tabular}
\end{center}
\label{numericalresults:table2}
\end{table*}

%INFINITUMmpfg
%a5c349c2d6


\subsection{Prediction}

In this the computational performance of the multilevel Kriging solver
is analyzed. Given a fixed Mat\'{e}rn parameters $(\nu,\rho)$ the
objective is to compute the BLUP vectors $\hat \bgamma$ and $\hat
\beta$. This involves solving the system of equations $\bP^{-1}_{\bW}
\bC_{\bW}(\btheta) \bgamma_{\bW} = \bP^{-1}_{\bW} \bZ_{\bW}$ and $\hat
\bbeta = (\bX\T \bX)^{-1}$ $\bX\T(\bZ - \bC \hat \bgamma)$.

Numerical results for computing $\hat \bgamma$ and $\hat \bbeta$ for the
hypercube data set with $d = 3$ dimensions, kD tree, and the Total
Degree index set $\mcQ^d_w$ are shown in Table
\ref{numericalresults:table3}. The Mat\'{e}rn covariance coefficients
$\btheta = (\nu,\rho)$ are set to (3/4,1). The relative error of the
residual of PCG method for the unpreconditioned system is set to
$\varepsilon = 10^{-3}$. The KIFMM is set to high accuracy.

For computing the matrix vector products of the PCG iterations, the
computational break even point of the KIFMM solver is reached for $N
\approx 2,500$ compared to using the direct approach (with CPU and
GPU). The increase in computational complexity is linear with respect
to $N$. Thus all the matrix vector products for the PCG iterations are
calculated using the KIFMM.

The preconditioner $\bP_{\bW}$ is built using a combination of the GPU
and CPU. This leads to a quadratic increase in computational cost with
respect to the number of observations $N$. However, due to the high
efficiency of the implementation and $p = 120$, the break even point
for the use of the KIFMM solver is not reached, even for $N = 512,000$
observation points.

From Table \ref{numericalresults:table3} observe that condition number
of the covariance matrix $\bC$ is much larger compared to
$\bC_{\bW}$. This is already a good indication that solving the
Kriging problem will be more efficient using the multilevel approach.

The number of iterations needed to reach the same accuracy for both
approaches are significantly better with the multilevel approach
i.e. $\approx 70$ times less iterations. However, the computation of
$\bbeta$ with the single level method requires solving $p + 1 = 121$
matrix inversions of $\bC$. This is in contrast with a single matrix
inversion of $\bC_{\bW}$ with the multilevel method. In practice, we
did not solve all $p+1$ matrix inversions for the single level
approach, but measure the time required to compute a single matrix
inversion and multiplied it 121 to obtain the estimated time
complexity.  For $N = 64,000$ observations we observe efficiencies of
$\approx 7,000$ compared to the single level iterative approach.

The condition number of the covariance matrices are fairly large,
making this problem somewhat hard to solve numerically.  The results
show that 512,000 size problems with good accuracy are feasible with a
single 4-core processor and GPU. Finally, the total computational cost
varies somewhere between linear and quadratic as the number of
observations $N$ is increased.


\begin{table*}[htbp]
  \caption{Numerical results for computing $\bP^{-1}_{\bW}
    \bC_{\bW}(\btheta) \bgamma_{\bW} = \bP^{-1}_{\bW} \bZ_{\bW}$ and
    $\hat \bbeta = (\bX\T \bX)^{-1} \bX\T(\bZ - \bC \hat \bgamma)$ for
    the hypercube data set with $d = 3$ and the Total Degree index set
    $Q^d_w$. The Mat\'{e}rn covariance coefficients $\btheta =
    (\nu,\rho)$ are set to (3/4,1). The relative error of the residual
    of PCG method for the \emph{unpreconditioned system} is set to
    $\varepsilon = 10^{-3}$. The KIFMM is set to high accuracy. (a)
    The second column of is the condition number of the covariance
    matrix $\bC$, up to $N=64,000$ observations, and is compared with
    the fourth column which corresponds to the condition number of
    $\bC_W$. The third column (itr($\bC_{\bW}$)) is the number of CG
    iterations needed for convergence for $10^{-3}$ residual
    accuracy. The fifth column is the number of iterations need to
    achieve the residual error $10^{-3}$ for the unpreconditioned
    system with the preconditioner $\bP_{\bW}$.  (b) The second column
    corresponds to the wall clock times in seconds for the
    preconditioner computation. The third column is the time for
    construction of the preconditioner $\bP_{\bW}$.  The PCG iteration
    wall clock timings for $\bC_{\bW}$, by using a KIFMM, are given in
    the fourth column. The fifth is the total time to compute
    $\bgamma_{\bW}$, $\bbeta$ and the multilevel basis
    construction. The sixth column is the computational efficiency for
    computing $\bgamma_{\bW}$ vs $\bC^{-1} \bZ$ to same residual
    accuracy with respect to the number of iterations. The last column
    is the estimated efficiency of computing $\hat \bgamma$ and $\hat
    \bbeta$ with the multilevel BLUP compared to the single level
    approach, equation \eqref{Kriging}, to approximately the same
    accuracy using a CG iteration with the KIFMM. We observe the
    significant speed ups ($\approx 7,000$ for $N = 64,000$) for
    calculating the BLUP by using the multilevel approach.
  %We observe that the total computational cost varies between linear
  %and somewhere between linear and quadratic as the number of
  %observations $N$ is increased.  The last column shows the efficiency
  %for computing itr($\bC_{\bW}$) with respect to $\bC$. Notice that as
  %$N$ increases the efficiency of the multilevel method increases
  %significantly.
  %The last column shows the efficiency for computing itr($\bC_{\bW}$
  %with respect to $\bC$. Notice that as $N$ increases the efficiency
  %of the multilevel method increases significantly.
}
\begin{center}
  \bigskip
  (a) $\btheta = (3/4,1)$, $d = 3$, $w = 7$ ($p = 120$) \\
  \bigskip
\begin{tabular} { r c r c r}
  \multicolumn{1}{c}{$N$} &
  \multicolumn{1}{c}{$\kappa(\bC)$} &
    \multicolumn{1}{c}{itr($\bC$)} &
  \multicolumn{1}{c}{$\kappa(\bC_{\bW})$} &
  itr($\bC_{\bW}$) \\
  \hline
  8,000  & $3.2 \times 10^{7}$   &    1,985 & $1.8 \times 10^{4}$ &   52  \\  
  16,000  & $1.1 \times 10^{8}$  &    3,511 & $6.0 \times 10^{4}$ &   67  \\  
  32,000  & $5.6 \times 10^{8}$  &    8,259 & $3.1 \times 10^{5}$ &  116  \\  
  64,000  &  $1.8 \times 10^{9}$ &   12,680 & $9.5 \times 10^{5}$  & 165  \\  
  128,000 &                 -   &      -  &                  -  &  308   \\  
  256,000 &                 -   &      -  &                  -  &  292   \\  
  512,000 &                 -   &      -  &                  -  &  484   \\  
\end{tabular}
\\
\bigskip
(b) $\btheta = (3/4,1)$, $d = 3$, $w = 7$ ($p = 120$) \\
\bigskip
\begin{tabular} { r r r r r c r}    
  \multicolumn{1}{c}{$N$} 
  & itr($\bC_{\bW}$) & $\bP_{\bW}$ (s) & Itr (s) & Total (s) & Eff$_{\bgamma}$ &
  \multicolumn{1}{c}{Eff$_{\bgamma,\bbeta}$} \\
  \hline
  8,000   &   52  &       4   &      29  &     38 &  38 &   3,600  \\
  16,000  &   67  &      13   &      98  &    118 &  52 &   5,000  \\
  32,000  &  116  &      45   &     260  &    317 &  71 &   7,250  \\
  64,000  &  165  &     178   &     798  &    997 &  76 &   7,380  \\
  128,000 &  308  &     713   &   3,934  &  4,687 &   - &    - \\
  256,000 &  292  &   2,837   &   5,745  &  8,663 &   - &    - \\
  512,000 &  484  &   11,392  &  20,637  & 32,202 &   - &    - \\
\end{tabular}
%% \bigskip
%% (b) $\btheta = (3/4,1)$, $d = 3$, $w = 7$ ($p = 120$) \\
%% \bigskip
%% \begin{tabular} { r r r r r c r}    
%%   \multicolumn{1}{c}{$N$} 
%%   & itr($\bC_{\bW}$) & $\bP_{\bW}$ (s) & Itr (s) & Total (s) & Eff$_{\bgamma}$ &
%%   \multicolumn{1}{c}{Eff$_{\bgamma,\bbeta}$} \\
%%   \hline
%%   8,000   &   53  &       3   &      18  &     21 &  48 &  5,808 \\
%%   16,000  &   70  &      11   &      43  &     54 &  89 & 10,769 \\
%%   32,000  &  116  &      45   &     265  &    310 &  58 &  7,018 \\
%%   64,000  &  165  &     178   &     798  &    976 &  76 &  9,196\\
%%   128,000 &  308  &     713   &   3,934  &  4,647 &   - &    - \\
%%   256,000 &  292  &   2,837   &   5,745  &  8,582 &   - &    - \\
%%   512,000 &  484  &   11,392  &  20,319  & 31,711 &   - &    - \\
%% \end{tabular}
\end{center}
\label{numericalresults:table3}
\end{table*}


The multilevel approach is now tested on $d = 20$ and $d = 25$
dimensional problems. Due to the high dimensionality of these
problems, a fast summation approach is not an option. The
matrix-vector products of each iteration are computed with the direct
approach using the GPU and CPU.

In Table \ref{numericalresults:table4}(a) the numerical results for
computing $\bgamma$ and $\bbeta$ for $d = 20$ and $\theta =(5/4,10)$.
Compared to the single level iterative approach the multilevel method
is approximately 42,000 faster for $N = 64,000$ observations. Similar
results are obtained shown in Table \ref{numericalresults:table4}(b). 
for $d = 25$ and $\theta =(5/4,10)$.

\setlength{\tabcolsep}{3pt}

\begin{table*}[htbp]
  \caption{Computing Kriging for the n-sphere data set with $d = 20$
    and $d = 25$ dimensions, TD index set, and Mat\'{e}rn covariance
    function without pre-conditioner. The residual accuracy is set to
    $\varepsilon = 10^{-3}$. Since the dimension is greater than 3,
    the matrix vector products are computed directly with the GPU and
    CPU.  The description of the columns of tables (a) and (b) are the
    same as for Table \ref{numericalresults:table3}. In addition,
    column 6 corresponds to the wall clock time for computing the
    multilevel basis. (a) Computational times for solving the Kriging
    prediction for $d = 20$ and $\theta = (5/4,10)$.  The growth in
    computational cost is slightly faster than quadratic due to the
    lack of fast summation method in higher dimensions. However,
    compared to the single level iterative approach it is
    approximately 42,000 faster for $N = 64,000$ observations. (b)
    Kriging prediction for $d = 25$ and $\theta = (5/4,10)$. The
    growth in computational cost is similar.  The efficiency of this
    method is about 2,840 times faster for $N = 64,000$ observations.}
\begin{center}
(a) $\btheta = (\nu,\rho) = (5/4,10)$, $d = 20$, $w = 3$ ($p =
1771$), No precond., Direct \\
\begin{tabular} { r c c c  c r r r r r}
  \multicolumn{1}{c}{$N$} & $\kappa (\bC)$ & $\kappa (\bC_{\bW})$ &
 itr($\bC$)
  & 
 itr($\bC_{\bW}$) &  MB(s) & Itr(s) & Total(s) &  Eff$_{\bgamma,\bbeta}$ \\
  \hline
 16,000  & $5  \times 10^{7}$ & 7  & 238 & 10  &  52  &      97   &    153 & 26,700   \\
 32,000  & $1 \times 10^{8}$  & 11 & 324 & 13  & 121  &     500   &    628 & 35,160   \\
 64,000  & $2 \times 10^{8}$  & 17 & 444 & 17  & 284  &   2,600   &  2,898 & 42,050 \\
128,000  &  -                & -  &  - & 22  & 628  &  13,494   &  14,153 & -  \\
\end{tabular}\\
\bigskip
(b) $\btheta = (\nu,\rho) = (3/4,10)$, $d = 25$, $w = 2$ ($p = 351$), No precond., Direct \\
\begin{tabular} { r c c c c r r r r}
  \multicolumn{1}{c}{$N$} & $\kappa (\bC)$ & $\kappa (\bC_{\bW})$
  & itr($\bC$)
  & itr($\bC_{\bW}$) & MB(s) & Itr(s) & Total(s) & Eff$_{\bgamma,\bbeta}$ \\
  \hline
 16,000  & $2  \times 10^{6}$  &   7  & 86  & 12 &   5  &         116   &    122 & 2,400   \\
 32,000  &  $4   \times 10^{6}$ &  12 & 109 & 15 &  13  &         582   &    599 & 2,490 \\
 64,000  &  $9  \times 10^{6}$ &   21 & 147  & 18 &  30  &       2,788  &  2,821 & 2,840 \\
 128,000  &  -                  &  - & -  & 25 &  79  &      15,557   & 15,641 &  - \\
 256,000  &  -                  &  - & -  & 33 & 157  &      83,163   & 83,337 &  - \\
\end{tabular}\\
\end{center}
\label{numericalresults:table4}
\end{table*}

\section{Conclusions}

In this paper a multilevel Kriging method is developed that scales
well with high dimensions. A multilevel basis is constructed from a
kD-tree and for the choice of Total Degree polynomial basis
$\mcQ^d_w$.  The approach described in the paper has the following
characteristics and advantages:

\begin{enumerate}[i)]

\item The multilevel method is numerically stable. Hard estimation and
  prediction of large dimensional problems are now feasible.
  
\item The method is efficiently implemented by using a combination of
  MATLAB, c++ software packages and dynamic libraries.

\item Sub-exponential decay of multilevel covariance matrix
  $\bC_{\bW}$ is proven based on complex analytic extensions.
  
\item Numerical results of up to 25 dimensional problems. These
  problems are difficult to solve with traditional methods due to the
  large condition numbers, but feasible with the multilevel method.

\item The multilevel prediction approach is proven to be \emph{exact},
  in the sense that single level and multilevel prediction
  formulations are shown to be equivalent. 
   

\item The efficiency of this approach will be further improved as high
  dimensional fast summation methods are developed.

\item An A-posteriori scheme and estimates for constructing the sparse
  covariance matrix $\tilde \bC$ will be developed in a future
  paper. This will be possible with the error bounds for the entries
  of $\bC$ derived in this paper since all the constants can be
  estimated.

  
\end{enumerate}














\section*{Appendix: Polynomial Interpolation}
\label{PolynomialAppendix}

In this section we provide some background on polynomial interpolation
in high dimensions. This will be critical to estimate the decay rates
of the entries of the multilevel covariance matrix for high
dimensional problems.
%Note that this appendix can be
%  skipped as it is only used for estimating the decay of the
%  multilevel covariance matrix.}

The decay of the coefficients will directly depend on the analytic
properties of the covariance function. The traditional error estimates
of polynomial interpolation are based on multi-variate $m^{th}$ order
derivatives. However, for many cases, such as the Mat\'{e}rn
covariance function, the derivatives are too complex or expensive to
manipulate for even a moderate number of dimensions. This motivates
the study of polynomial numerical approximations based on complex
analytic extensions, which are much better suited for high dimensions.
Much of the discussion that follows has it roots in the field of
uncertainty quantification and high dimensional interpolation
\cite{nobile2008a,Castrillon2016,Griebel2016}
for partial differential
equations.


Consider the problem of approximating a function $v: \Gamma^{d}
\rightarrow \R$ on the domain $\Gamma^{d}$.  Without loss of
generality let $\Gamma : = [-1, 1]$ and $\Gamma^{d} := \prod_{n =
  1}^{d} \Gamma$. Suppose that $\mcG \subset \Gamma^{d}$, then define
the following spaces
\[
\begin{split}
  &
L^q(\mcG) := \{ v(\by)\, | \, \int_{\mcG} v(\by)^q \text{d}
\by < \infty  \}
\,\,\,
\mbox{and} \\
&
L^{\infty}(\mcG) := \{ v(\by)\, | \, \sup_{\by \in \mcG} |v(\by)|
< \infty  \}.
\end{split}
\]


Suppose that $\mcP_{ q}(\Gamma):=\text{\rm span}\{y^k,\,k=0,\dots,q\}$
i.e. the space of polynomials of degree at most $q$. Let $\mcI^{m} :
C^{0}(\Gamma) \rightarrow \mcP_{m-1}(\Gamma)$ be the univariate
Lagrange interpolant
\[
\mcI_{m}(v(\by)):=
\sum_{k=1}^{m}v(y^{(k)})l_{m,k}(y^{(k)}),
\]
where $y^{(1)}, \dots, y^{(m)}$ is a set of distinct knots on $\Gamma$
and $\{ l_{n,k} \}_{k=0}^{m}$ is a Lagrange basis of the space
$\mcP_{m-1}(\Gamma)$. The variable $m \in \Nset$
%, where $\Nset_{+} := \Nset \cup 0$,
corresponds to the order of approximation of the
Lagrange interpolant. However, for the case of the zero order
interpolation $m = 0$ corresponds to $\mcI_{0} = 0$.


\begin{remark}
For high dimensional interpolation the particular set of points
$y^{(1)}, \dots, y^{(m)}$ that we will use is the Clenshaw-Curtis
abscissas.  This is further discussed in this section. However, for
now, we assume that the points are only distinct.
  \end{remark}


For $m \geq 1$ let
\[
\Delta_{m}
:= \mcI_{m}-\mcI_{m-1},
\]
From the difference operator $\Delta_{m}$ we can readily observe that
$\mcI_{m} = \sum_{k=1}^{m} \Delta_{k}$, which is reminiscent of multi
resolution wavelet decompositions. The idea is to represent
multivariate approximation as a summation of the difference operators.

Consider the multi-index tupple $\bm = (m_1,\dots,m_d)$, where $\bm
\in \Nset^{d}$, and form the tensor product operator
$\mcS_{w,d}: \Gamma \rightarrow \R$ as
\begin{equation}
  \mcS_{w,d}
      [v(\by)]
      :
      =
 \sum_{\bm \in \bbNset^{d}: \sum_{i=1}^{d} m_i - 1  \leq w } \;\;
 \bigotimes_{n=1}^{d} {\Delta^{n}_{m_n}}(v(\by)).
\label{errorestimates:SG}
\end{equation}
Note that by ${\Delta^{n}_{m_n}}(v(\by))$ we mean that the difference
operator ${\Delta_{m_n}}$ is applied along the $n^{th}$ dimension in
$\Gamma$.


Let $C^{0}(\Gamma_d; \R) : = \{ v: \Gamma_d \rightarrow \R\,\,$ is
continuous on $\Gamma_d$ and $\max_{\by\in \Gamma_d} |v(\by)| < \infty
\}$.  From Proposition 1 in \cite{Back2011} it is shown that for any
$v \in C^0(\Gamma_d;\R)$, we have $\mcS_{w,d}[v]\in \mcQ^{d}_{w}$.
Moreover, $\mcS_{w,d}[v] = v$, for all $v \in \mcQ^{d}_{w}$. The key
observation to take away is that the operator $\mcS_{w,d}[v]$ is
\textit{exact} in the space of polynomials $\mcQ^{d}_{w}$. This will
be useful in connecting the Lagrange interpolant with Chebyshev
polynomials.


Let $T_k:\Gamma \rightarrow \R$, $k = 0, 1, \dots$, be a Chebyshev
polynomial over $\Gamma$, which are defined recursively as follows:
$T_0(y) = 1$, $T_1(y) = y$, $\dots$, $T_{k+1}(y) = 2yT_{k}(y) -
T_{k-1}(y)$, $\dots$, where $y \in \Gamma$. Chebyshev polynomials are
well suited for the approximation of functions with analytic
extensions on a complex region bounded by a Bernstein ellipse. They
bypassing the need of using derivative information and sharp bounds on
the error are readily available. Suppose that $\sigma > 0$ and denote
by
\[
\begin{split}
  \mcE_{\sigma} := \Big\{
  z \in \bbC, \sigma \geq
\delta \geq 0 ;\,\Real{z} = \frac{e^{\delta} + e^{-\delta}
}{2}cos(\theta) 
\Imag{z} = \frac{e^{\delta} 
  - e^{-\delta}}{2}sin(\theta),
\theta \in [0,2\pi)
  \Big\}
\end{split}
  \]
as the region bounded by a Bernstein ellipse (see Figure
\ref{erroranalysis:sparsegrid:polyellipse}).

The following theorem is based on complex analytic extensions on
$\mcE_{\sigma}$ and provides a control for the Chebyshev polynomial
approximation.

\begin{theorem}
Suppose that for $u:\Gamma \rightarrow \R$ there exists an analytic
extension on $\mcE_{\sigma}$. If $|u| \leq M < \infty$ on
$\mcE_{\sigma}$ then there exists a sequence of coefficients
$|\alpha_k| \leq M / e^{k\sigma}$ such that $u \equiv \alpha_0 +
2\sum_{k = 1}^{\infty} \alpha_{k} T_{k}$ on $\mcE_{\sigma}$. Moreover,
if $y \in \Gamma$ then
\[
%\begin{multline*}
%\shoveright{|q(y) - \alpha_0  - 2\sum_{k = 1}^{n} \alpha_{k} T_{k}(y)|
%\leq 
%\frac{2M}{e^{\sigma} - 1} e^{-n \sigma}.}
|q(y) - \alpha_0  - 2\sum_{k = 1}^{n} \alpha_{k} T_{k}(y)|
\leq 
\frac{2M}{e^{\sigma} - 1} e^{-n \sigma}.
%\end{multline*}
\]
\label{errorestimates:theorem}
\end{theorem}
\begin{proof}
See Theorem 2.25 in \cite{Khoromskij2018}
\end{proof}


\begin{figure}[htb]%[12]{r}{7cm}%[htp]
\begin{center}
\begin{tikzpicture}
    \begin{scope}[font=\scriptsize]

      
      \filldraw[fill=blue!20,
      semitransparent] (0,0) ellipse (2 and 1);

    \draw [->] (-2.5, 0) -- (2.5, 0) node [below left]  {$\Real $};
    \draw [->] (0,-1.5) -- (0,1.5) node [below left] {$\Imag$};
    \draw (1,-3pt) -- (1,3pt)   node [above] {$1$};
    \draw (-1,-3pt) -- (-1,3pt) node [above] {$-1$};
    \end{scope}
    
    \node [below right] at (-2.5,1.25) {$\mcE_{\sigma}$};

    \node [] at (0.75,1.25) {$\frac{e^{
          \sigma} - e^{- \sigma}}{2}$};

    
    \node [] at (2.75,0.25) {$\frac{e^{
      \sigma} + e^{- \sigma}}{2}$}; 
    
\end{tikzpicture}
\end{center}
\caption{Complex region bounded by the Bernstein ellipse.}
\label{erroranalysis:sparsegrid:polyellipse}
\end{figure}

We can now connect the error due to the Lagrange interpolation with
Chebyshev expansions. It is known that if $u \in C(\Gamma,\R)$ then
\[
\|(I - \mcI_{m})u\|_{L^{\infty}(\Gamma)} \leq
(1 + \Lambda_{m})
\min_{h \in \mcP_{m-1}} \| u - h \|_{L^{\infty}(\Gamma)},
\]
where $\Lambda_{m}$ is the Lebesgue constant (See Lemma 7 in
\cite{babusk_nobile_temp_10}). Note that $I:C^{d}(\Xi;\R) \rightarrow
C^{d}(\Xi;\R)$ refers to the identity operator and the domain $\Xi$ is
taken from context. For the previous case $\Xi = \Gamma$.  Bounds on
$\Lambda_{m}$ are known in the context of the location of the knots
$y^{(1)}, \dots, y^{(m)} \in \Gamma$. In this article we restrict our
attention to Clenshaw-Curtis abscissas
%\[
\[
y^{(j)} = -\cos \left( \frac{\pi(j-1)}{m - 1} \right),\,\, j =
1,\dots, m
\]
%\]
and $\Lambda_m$ is bounded by $2\pi^{-1}(\log{(m-1)} + 1) \leq 2m - 1$
(see \cite{babusk_nobile_temp_10}).  Since the interpolation operator
$\mcI_{m}$ is exact on $\mcP_{m - 1}$, then if $u:\Gamma \rightarrow
\R$ has an analytic extension in $\mcE_{\sigma}$ we have from Theorem
\ref{errorestimates:theorem} (following a similar approach as in
\cite{babusk_nobile_temp_10}) that
\[
\begin{split}
\|(I - \mcI_{m})u\|_{L^{\infty}(\Gamma_n)}
\leq
(1 + \Lambda_{m})
\frac{2M}{e^{\sigma} - 1} e^{-\sigma (m-1)}
\leq 
2 C(M,\sigma) m e^{-\sigma (m-1)},
\end{split}
\]
where $C(M,\sigma_n) := \frac{2M}{(e^{ \sigma} - 1)}$. We then
conclude that for all $k = 1,\dots, m$
\begin{equation}
\begin{split}
\| \Delta_{k}(u) \|_{L^{\infty}(\Gamma)} 
&=
\|
\mcI^{m}(u) - \mcI^{m-1}(u)
\|_{L^{\infty}(\Gamma)} 
\leq
\|(I - \mcI_{m})u\|_{L^{\infty}(\Gamma)}
+
\|(I - \mcI_{m-1})u\|_{L^{\infty}(\Gamma)} \\
&\leq
e^{2\sigma}C(M,\sigma) m e^{-\sigma m}.
\end{split}
\label{interpolation:eqn1}
\end{equation}
Let $\mcE_{\sigma,n} \subset \bbC^{d}$ a complex region bounded by a
Bernstein ellipse such that the restriction on $\Gamma_{d}$ is along
the $n^{th}$ dimension and form the polyellipse $\mcE^{d}_{\sigma}:=
\prod_{n=1}^{d} \mcE_{\sigma,n}$.  Suppose that $v:\mcE^{d}_{\sigma}
\rightarrow \bbC$ is analytic on $\mcE^{d}_{\sigma}$ and let
$\tilde{M}(v) := \max_{\bz \in \mcE^{d}_{\sigma}} |v(\bz)|$.

Note we refer to $\mcI^{n}_{m}$ as the Lagrange operator of order $m$
along the $n^{th}$ dimension and similarly $\mcP^{n}_{m-1}$ is the
space of the span of univariate polynomials up to degree $m-1$ along
the $n^{th}$ dimension.  Form the tensor product $\bI^{d}_{m} :=
\mcI^{1}_{m} \times \dots \times \mcI^{d}_{m}$, thus $\bI:C(\Gamma,\R)
\rightarrow \bbP$ where $\bbP := \mcP^{1}_{m-1} \times \dots \times
\mcP^{d}_{m-1}$. From Theorem 2.27 in \cite{Khoromskij2018} we can
conclude that for a finite dimension $d$, as $m \rightarrow \infty$
then $\bI^{d}_{m}[v] \rightarrow v$.

Applying equation \eqref{interpolation:eqn1} to equation
\eqref{errorestimates:SG} we have that
\begin{equation}
\begin{split}
& \| (I - \mcS_{w,d})
 v(\by)
 \|_{L^{\infty}(\Gamma^{d})}
 \leq
 \left\| \sum_{\bm \in \bbNset^{d}: \sum_{i=1}^{d} m_i - 1 > w } \;\;
 \bigotimes_{n=1}^{d} {\Delta^{n}_{m_n}}(v(\by))\right\|_{L^{\infty}(\Gamma^d)} \\
 &\leq
 \sum_{\bm \in \bbNset^{d}: \sum_{i=1}^{d} m_i - 1 > w } \;\;
 \bigotimes_{n=1}^{d} \|{\Delta^{n}_{m_n}}(v(\by))\|_{L^{\infty}(\Gamma^d)} 
 \leq
 \sum_{\bm \in \bbNset^{d}: \sum_{i=1}^{d} m_i - 1 > w }
 e^{2d} C(M,\sigma)^{d} \\
 &
 \left( \prod_{n=1}^{d} m_n\right) \exp{\left( -\sum_{n=1}^{d}
   \sigma m_{n} \right)}.
% \\
%  &\leq
% \sum_{\bk \in \bbNset^{d}_{0}: \sum_{i=1}^{d} k_i > w }
% e^{2d} C(M,\sigma)^{d} \left( \prod_{n=1}^{d} (k_n + 1)\right)
% \exp{\left( -\sum_{n=1}^{d}
%   \sigma (k_{n}+1) \right)}.
\end{split}
\label{interpolation:eqn2}
\end{equation}

By applying Theorem 2.10 and Corollary 2.11 in \cite{Griebel2016} if
$ w \geq  d$ and $p( d, w) \geq
\left(\frac{2  d}{\kappa( d)}\right)^{ d}$, where
$\kappa( d) := \sqrt[\leftroot{-2}\uproot{2}  d]{
  d!} >  d/e$ (Sterling approximation), then for any $\hat
\sigma \in \R_{+}$
\begin{equation}
\begin{split}
 & \sum_{\bk \in \bbNset^{ d}_{0}: \sum_{i=1}^{ d} k_i  >  w }
 \exp{\left( -\sum_{n=1}^{ d} \hat \sigma
   k_{n} \right)}
 \leq
 \sum_{\bk \in \bbNset^{d}_{0}: \hat \sigma \sum_{i=1}^{ d} k_i  \geq  w \hat \sigma  }
 \exp{\left( -\sum_{n=1}^{ d}
   \hat \sigma k_{n} \right)} \\
 &\leq
 \hat \sigma  d e
 \left( \frac{e^{\hat \sigma}}{1 - e^{-\hat \sigma}} \right)^{ d}
 \exp \left(-\frac{ d}{e} \hat \sigma  p^{\frac{1}{ d}}
 \right) p^{\frac{ d-1}{ d}}.
\end{split}
\label{interpolation:eqn3}
\end{equation}
where $\bk \in \bbNset^{d}_{0}$ and $\bk:=(k_1,\dots,k_d)$.






Following the same approach as in \cite{Griebel2016} observe that for
$0 < \delta < 1$ we can obtain a bounded constant $c_{n,\delta} \leq
(e\sigma \delta)^{-1}$ such that $m_n \exp(-\sigma m_n) \leq (e\sigma
\delta)^{-1}$ $\exp(-\sigma m_n (1 - \delta))$. Set $\hat \sigma :=
\sigma (1 - \delta)$ and by combining equations
\eqref{interpolation:eqn2} and \eqref{interpolation:eqn3} we have
proven the following result.

\begin{lemma} Suppose that $0< \delta < 1$, $\hat
  \sigma := \sigma (1 - \delta)$, and $p(d,w) \geq \left(\frac{2
    d}{\kappa(d)}\right)^{d}$ then
  \[
  \begin{split}
 &\| (I - \mcS_{w,d})
 v(\by)
 \|_{L^{\infty}(\Gamma^{d})}
 \leq 
 \frac{C(\tilde M,\sigma)^d e^{d - \sigma(1 - \delta) + 1} \hat \sigma d }
 {
(\sigma \delta)^{d}}
 \left( \frac{e^{\hat \sigma}}{1 - e^{-\hat \sigma}} \right)^{d} 
 \exp \left(-\frac{d}{e} \hat \sigma  p^{\frac{1}{d}}
 \right) p^{\frac{d-1}{d}}.
 \end{split}
 \]
 \label{interpolation:lemma1}
\end{lemma}


\begin{remark}
The restriction $p(d,w) \geq \left(\frac{2
  d}{\kappa(d)}\right)^{d}$ is not strict and can be relaxed such that
sub-exponential convergence is still obtained.  We refer the reader to
the bound of the Gamma function in Lemma 2.5 (\cite{Griebel2016}) and
it's application in the proofs of Theorem 2.10 and Corollary 2.11.
\label{interpolation:remark1}
\end{remark}




\noindent 
\textbf{Acknowledgements:} I appreciate the help and advice from
George Biros and Lexing Ying, for setting up the KIFMM packages.  I
also appreciate the support that King Abdullah University of Science
and Technology has provided to this project.




\bibliographystyle{abrev}
\input{MLKriging.bbl} 


\end{document}




 


\end{document}




 


\end{document}




 


\end{document}




