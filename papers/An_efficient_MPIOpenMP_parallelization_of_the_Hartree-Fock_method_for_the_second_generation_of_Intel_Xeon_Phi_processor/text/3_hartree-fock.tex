\section{Hartree-Fock method}
\label{sec:hf}
The HF method is used to iteratively solve the electronic Schr\"odinger equation for a many-body system. The resulting electronic energy and electronic wave function can be used to compute equilibrium geometries and a variety of molecular properties. The wave function is constructed of a finite set of basis functions suitable for algebraic representation of the integro-differential HF equations.  Central to HF is an effective one-electron Hamiltonian called the Fock operator which describes electron-electron interactions by mean field theory. In computational practice, the Fock operator is defined in matrix form (Fock matrix). The HF working equations are then represented by a nonlinear eigenvalue problem called the Hartree-Fock equations: 
\begin{equation}\label{eqn:scf}
	\mathbf{FC}={\epsilon}\mathbf{SC}
\end{equation}
where $\epsilon$ is a diagonal matrix corresponding to the electronic orbital energies, $\mathbf{F}$ is a Fock matrix, $\mathbf{C}$ is matrix of molecular orbital (MO) coefficients, and $\mathbf{S}$ is the overlap matrix of the atomic orbital (AO) basis set. The HF equations are solved numerically by self-consistent field (SCF) iterations.

The SCF iterations are preceded by computation of an initial guess density matrix and core Hamiltonian. An initial Fock matrix is constructed from terms of the core Hamiltonian and a symmetric orthogonalization matrix. Next, the Fock matrix is diagonalized to provide the MO coefficients $\mathbf{C}$. These MO coefficients are used to compute an initial guess density matrix. The SCF iterations follow, in which a new Fock matrix is constructed as a function of the guess density matrix. Diagonalization of the updated Fock matrix provides a new set of MO coefficients which are used to update the density matrix. This iterative process continues until convergence is reached, which is defined by the root-mean-squared difference of consecutive densities lying below a chosen convergence threshold.

Contrary to what one might expect, the most time-consuming part of the calculation is not the solution of the Hartree-Fock equations, but rather the construction of the Fock matrix~\cite{janssen2008}. The calculation of the Fock matrix elements can be separated into one-electron and two-electron components. The computational complexity of these two parts is \bigon{2} and \bigon{4}, respectively. In most cases of practical interest, the calculation of the two-electron contribution to the Fock matrix occupies the majority of the overall compute time.