\section{Introduction}
The field of computational chemistry encompasses a wide range of empirical, semi-empirical, and \abinitio\ methods that are used to compute the structure and properties of molecular systems. These methods therefore have a significant impact on not only chemistry, but materials, physics, engineering and the biological sciences as well. \textit{Ab initio} methods are rigorously derived from quantum mechanics. In principle, \abinitio\ methods are more accurate than methods with empirically fitted parameters. Unfortunately, this accuracy comes at significant computational expense. For example, the time to solution for Hartree-Fock (HF) and Density Functional Theory (DFT) methods scale as approximately \bigon{3}, where N is the number of degrees of freedom in the molecular system. The HF solution is commonly used as a starting point for more accurate \abinitio\ methods, such as second order perturbation theory and coupled-cluster theory with single, double, and perturbative triple excitations. These post-HF methods scale as \bigon{5} and \bigon{7}, respectively. These computational demands clearly require efficient utilization of parallel computers to treat increasingly large molecular systems with high accuracy.

Modern high performance computing hardware architecture has substantially changed over the last 10 to 15 years. Nowadays, a ``many-core'' philosophy is common to most platforms. For example, the Intel Xeon Phi processor can have up to 72 cores. For good resource utilization, this necessitates (hybrid) MPI+X parallelism in application software.

The subject of this work is the successful adaptation of the HF method in the General Atomic and Molecular Electronic Structure System (GAMESS) quantum chemistry package to the second-generation Intel Xeon Phi processor platform. GAMESS is a free quantum chemistry software package maintained by the Gordon research group at Iowa State University \cite{gordon2005advances}. GAMESS has been cited more than 10,000 times in the literature, downloaded more than 30,000 times and includes a wide array of quantum chemistry methods. The objective here is to start with the MPI-only version of GAMESS HF and systematically introduce optimizations which improve performance and reduce the memory footprint. Many existing methods in GAMESS are parallelized with MPI. OpenMP is an attractive high-level threading application program interface (API) that is scalable and portable. The OpenMP interface conveniently enables sharing of the two major objects in the HF self-consistent field (SCF) loop: the density matrix and the Fock matrix.

The density and Fock data structures account for the majority of the memory footprint of each MPI process. Indeed, since these two objects are replicated across the MPI processes, memory capacity limits can easily come into play if one tries to improve the time to solution using a large number of cores. By sharing one or both of the aforementioned objects between threads, one can reduce the memory footprint and more easily leverage all of the resources (cores, fast memory etc.) of the Intel Xeon Phi processor. Reducing the memory footprint is also expected to lead to better cache utilization, and, therefore, enhanced performance. Two hybrid OpenMP/MPI implementations of the publicly available version of the GAMESS (MPI-only) code base were constructed for this work. The first version is referred to as the ``shared density private Fock'', or ``private Fock'' version of the code. The second version is referred to as the ``shared density shared Fock'', or ``shared Fock'' version.

In the following section, a brief survey of related work is presented. Next, key algorithmic features of the HF self-consistent field (SCF) method are discussed. Then, a description of the computer hardware test bed that was used for benchmarking purposes is presented. An explanation of the code transformations employed in the hybrid implementation in this work follows. Next, the memory and time-to-solution results of the hybrid approach are shown. Results on up to 3,000 Intel Xeon Phi processors are presented for a range of chemical system sizes. The work ends with concluding remarks and a discussion of directions for future work.
