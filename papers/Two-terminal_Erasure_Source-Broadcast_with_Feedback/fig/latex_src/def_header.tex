% GWW DEFINITIONS AND ABBREVIATIONS

% TeX Defs

                 % subequations environment, etc
%\usepackage{amssymb}  % get, among others, blackboard bold fonts
                      % defines extra symbols like \gtreqless, etc
%\usepackage{verbatim} % get comment environment, + new verbatim
% \usepackage{amsxtra}  % get, eg, \accentedsymbol
%
%\DeclareMathOperator*{\argmax}{arg\,max}
%\DeclareMathOperator*{\argmin}{arg\,min}
%\DeclareMathOperator*{\argsup}{arg\,sup}
%\DeclareMathOperator*{\arginf}{arg\,inf}
%\DeclareMathOperator{\erfc}{erfc}
%\DeclareMathOperator{\diag}{diag}
%\DeclareMathOperator{\cum}{cum}
%\DeclareMathOperator{\sgn}{sgn}
%\DeclareMathOperator{\tr}{tr}
%\DeclareMathOperator{\spn}{span}
%\DeclareMathOperator{\adj}{adj}
%\DeclareMathOperator{\var}{var}
%\DeclareMathOperator{\cov}{cov}
%\DeclareMathOperator{\sech}{sech}
%\DeclareMathOperator{\sinc}{sinc}
%\DeclareMathOperator*{\lms}{l.i.m.\,}
%\newcommand{\varop}[1]{\var\left[{#1}\right]}
%\newcommand{\covop}[2]{\cov\left({#1},{#2}\right)}
%
%\newcommand{\p}{\partial}

% LIST ENVIRONMENTS

\newcounter{actr}
\newenvironment{alist}%
{\begin{list}{(\alph{actr})}{\usecounter{actr}}}{\end{list}}

\newcounter{ictr}
\newenvironment{ilist}%
{\begin{list}{(\roman{ictr})}{\usecounter{ictr}}}{\end{list}}

\iffalse

% SPACING ENVIRONMENTS

\newenvironment{singlespace}%
{\begin{spacing}{1}}{\end{spacing}}

\newenvironment{onehalfspace}% for 11pt font
{\begin{spacing}{1.21}}{\end{spacing}}

\newenvironment{doublespace}% for 11pt font
{\begin{spacing}{1.62}}{\end{spacing}}

\fi

% THEOREM ENVIRONMENTS
\newtheorem{obs}{Observation}
\newtheorem{remark}{Remark}
\newtheorem{thm}{Theorem}
\newtheorem{lemma}{Lemma}
\newtheorem{claim}{Claim}
\newtheorem{corol}{Corollary}
\newtheorem{prop}{Proposition}
\newtheorem{defn}{Definition}
\newtheorem{fact}{Fact}
%\newenvironment{proof}%
%{\noindent{\em Proof: } \begin{singlespace} \small \noindent}%
%{\noindent\qed \end{singlespace}}
\newenvironment{new-proof}[1]
{{\em Proof }:\\}%
{ \noindent\qed }
%
\newcommand{\abs}[1]{\left|#1\right|}
%\newcommand{\comb}[2]{{#1\choose#2}}
\newcommand{\comb}[2]{\binom{#1}{#2}}
\newcommand{\ie}{i.e.}
\newcommand{\eg}{e.g.}
\newcommand{\etc}{etc.}
\newcommand{\viz}{viz.}
\newcommand{\etal}{et al.}
\newcommand{\cf}{cf.}

\newcommand{\vect}[3]{\begin{bmatrix} #1 & #2 & \cdots & #3 \end{bmatrix}^\T}

\newcommand{\dsp}{.5\baselineskip}             % double space amount
\newcommand{\down}{\vspace{\dsp}}              % double space command
\newcommand{\ddown}{\vspace{\baselineskip}}    % quadruple space command
\newcommand{\spec}{\hspace*{1pt}}              % little bit of space
\newcommand{\ds}{\displaystyle}                % abbreviation
\newcommand{\ts}{\textstyle}                % abbreviation
\newcommand{\nin}{\noindent}                   % noindent abbreviation
\newcommand{\cvar}[1]{\mathrm{var_{#1}\,}}
%\newcommand{\qed}{\rule[0.1ex]{1.4ex}{1.6ex}}
\newcommand{\mycap}[2]{\caption{\sl #2 \label{#1}}}
\newcommand{\subcap}[1]{{\begin{center}\sl #1\end{center}}}
\newcommand{\ditem}[1]{\item[#1 \hspace*{\fill}]}
\newcommand{\appfig}{\vspace*{1in}\begin{center} Figure appended to
                       end of manuscript. \end{center} \vspace*{1in}}
\newcommand{\psx}[1]{\centerline{\epsfxsize=6in \epsfbox{#1}}}
\newcommand{\psy}[1]{\centerline{\epsfysize=7in \epsfbox{#1}}}
\newcommand{\psxs}[2]{\centerline{\epsfxsize=#1in \epsfbox{#2}}}
\newcommand{\psxsbb}[3]{\centerline{\epsfxsize=#1in \epsfbox[#3]{#2}}}
\newcommand{\psys}[2]{\centerline{\epsfysize=#1in \epsfbox{#2}}}
\newcommand{\convsamp}[3]{\left.\left\{#1 \ast #2\right\}\right|_{#3}}
\newcommand{\gap}{\qquad}
\newcommand{\order}[1]{\mathcal{O}\left(#1\right)}
\newcommand{\arror}[3]{\begin{cases} #1 & #2 \\
                                     #3 & \text{otherwise} \end{cases}}
\newcommand{\arrorc}[3]{\begin{cases} #1 & #2 \\
                                     #3 & \text{otherwise,} \end{cases}}
\newcommand{\arrorp}[3]{\begin{cases} #1 & #2 \\
                                     #3 & \text{otherwise.} \end{cases}}
\newcommand{\darror}[4]{\begin{cases} #1 & #2 \\ #3 & #4 \end{cases}}
% \newcommand{\defeq}{\stackrel{\triangle}{=}}
\newcommand{\defeq}{\stackrel{\Delta}{=}}
\newcommand{\msconv}{\stackrel{\mathrm{m.s.}}{\longrightarrow}}
\newcommand{\pwaeconv}{\stackrel{\mathrm{p.w.a.e.}}{\longrightarrow}}
\newcommand{\peq}{\stackrel{\mathcal{P}}{=}}
% \newcommand{\glt}{ \begin{array}{c} \Hh=H_1 \\
%  \renewcommand{\arraystretch}{.3}
%  \begin{array}{c} > \\ < \end{array}
%  \renewcommand{\arraystretch}{1} \\ \Hh=H_0 \end{array}}

\hyphenation{or-tho-nor-mal}
\hyphenation{wave-let wave-lets}

\newcommand{\crb}{Cram\'{e}r-Rao}  % obsolete
\newcommand{\CR}{Cram\'{e}r-Rao}
\newcommand{\KL}{Karhunen-Lo\`{e}ve}
\newcommand{\sE}{\sqrt{E_0}}
\newcommand{\pe}{\Pr(\eps)}
\newcommand{\jw}{j\w}
\newcommand{\ejw}{e^{j\w}}
\newcommand{\ejv}{e^{j\nu}}
\newcommand{\wo}{{\w_0}}
\newcommand{\woh}{{\wh_0}}
\newcommand{\sumi}[1]{\sum_{#1=-\infty}^{+\infty}}
\newcommand{\inti}{\int_{-\infty}^{+\infty}}
\newcommand{\intp}{\int_{-\pi}^{\pi}}
\newcommand{\nintp}{\frac{1}{2\pi}\int_{-\pi}^{\pi}}
\newcommand{\inth}{\int_{0}^{\infty}}
\newcommand{\E}[1]{E\left[{#1}\right]}
\newcommand{\bigE}[1]{E\bigl[{#1}\bigr]}
\newcommand{\BigE}[1]{E\Bigl[{#1}\Bigr]}
\newcommand{\biggE}[1]{E\biggl[{#1}\biggr]}
\newcommand{\BiggE}[1]{E\Biggl[{#1}\Biggr]}
\newcommand{\Prob}[1]{\Pr\left[{#1}\right]}
\newcommand{\Pu}[1]{\Pr\left[{#1}\right]} % obsolete; same as \Prob now
\newcommand{\Pc}[2]{\Pr\left[{#1}\mid{#2}\right]}  % obsolete
\newcommand{\Pcb}[2]{\Pr\left[{#1}\Bigm|{#2}\right]} % obsolete
\newcommand{\Q}[1]{\mathcal{Q}\left({#1}\right)}
\newcommand{\FT}[1]{\mathcal{F}\left\{{#1}\right\}}
\newcommand{\LT}[1]{\mathcal{L}\left\{{#1}\right\}}
\newcommand{\ZT}[1]{\mathcal{Z}\left\{{#1}\right\}}
%\newcommand{\reals}{\mathbf{R}}
\newcommand{\reals}{\mathbb{R}}
%\newcommand{\ints}{\mathbf{Z}}
\newcommand{\ints}{\mathbb{Z}}
\newcommand{\compls}{\mathbb{C}}
\newcommand{\nats}{\mathbb{N}}
\newcommand{\rats}{\mathbb{Q}}
\newcommand{\ltwor}{L^2(\reals)}
\newcommand{\ltwoz}{\ell^2(\ints)}
\newcommand{\ltwow}{L^2(\Omega)}
% \newcommand{\ltwo}{\mathbf{L}^2}
% \newcommand{\ltwor}{\mathbf{L}^2 (\reals)}
% \newcommand{\ltwoz}{\mathbf{l}^2 (\ints)}
\newcommand{\sys}[1]{\mathcal{S}\left\{#1\right\}}
\newcommand{\nn}{\nonumber}
\newcommand{\mrm}{\mathrm}
\newcommand{\ip}[2]{\left\langle{#1},{#2}\right\rangle}
\newcommand{\di}[2]{d\left({#1},{#2}\right)}
\newcommand{\ceil}[1]{\lceil{#1}\rceil}
\newcommand{\floor}[1]{\lfloor{#1}\rfloor}
\newcommand{\phase}{\measuredangle}

\newcommand{\Ht}{\mathrm{H}}
\newcommand{\T}{{\mathrm{T}}}
% \newcommand{\R}{\Re\mathit{e}}
% \newcommand{\I}{\Im\mathit{m}}
\DeclareMathOperator{\R}{Re}
\DeclareMathOperator{\I}{Im}
