\section{Source-broadcast with Universal Feedback}
\label{sec:two_users} 
We first show that the case involving only two users can be fully solved.  We do this by demonstrating an algorithm that achieves point-to-point optimality for both users at any time during transmission.  Specifically, for $i \in \{1, 2\}$, let $\wid$ be the point-to-point optimal latency for user~$i$ obtained from the source-channel separation theorem where

\begin{align}
\label{eq:wid_optimal}
	\wid = \frac{1 - d_i}{1 - \epsilon_i}.
\end{align}
%The coding scheme for this case is shown in Algorithm~\ref{alg:two_users}.  
We now present an algorithm to achieve this outer bound.  In the first phase of the algorithm, we successively transmit each source symbol uncoded until at least one user receives it. If $S(t)$ is received only by user~$i$,  then the transmitter places $S(t)$ into queue $Q_j$.
No action is taken if both users receive $S(t)$. By assumption, feedback is universally available, and so user~$i$ is also able to maintain a local version of queue~$Q_j$.

Now, after this first phase, the transmitter has built queues $Q_1$ and $Q_2$, where 
for $i,j \in \{1, 2\}, i\neq j$, 
user~$i$ has knowledge of packets in $Q_j$ and is in need of those in $Q_i$.
% for $i, j \in \{1, 2\}, i \neq j$. 
Thus, the algorithm's next phase involves successively transmitting a linear combination of the packets at the fronts of $Q_1$ and $Q_2$.  Let $q_i$ be the packet at the front of $Q_i$.  Notice that a successfully received channel symbol of the form $q_1 \oplus q_2$ means that user~1 is able to decode $q_1 \in Q_1$,  since he has access to $q_2 \in Q_2$. 
We therefore remove $q_i$ from $Q_i$ whenever a linear combination involving $q_i$ is received by user~$i$.
%
This entire phase continues until the users' distortion constraints are met.  The decoding algorithm for this scheme is also simple.  Given that user~$i$ has decoded source symbol~$S(t)$ from the linear combination he received, user~$i$ sets $\hat{S}_i(t) = S(t)$.

%%% Good
%%%\begin{algorithm}
%%%\caption{Myalgorithm}\label{alg:two_users}
%%%\begin{algorithmic}[1]
%%%\For {$t = 1, 2, \ldots N$ \textbf{AND} $R_i < (1 - d_i)N, i \in \{1, 2\}$}
%%%    \State Send $S(t)$ until at least one user receives it
%%%    \If {$N_i = 1, N_j = 0$ for some $i, j \in \{1, 2\}, i \neq j$}
%%%        \State $Q_i\gets Q_i \cup \{S(t)\}$
%%%    \EndIf
%%%    \If {$N_i = 0$}
%%%        \State $R_i\gets R_i + 1$
%%%    \EndIf
%%%\EndFor
%%%
%%%\While{$R_i < (1 - d_i) N, i \in \{1, 2\}$}
%%%    \If {$R_i < (1 - d_i)N$ \textbf{AND} $Q_i \neq \emptyset$}
%%%        \State Let $q_i \in Q_i$
%%%    \Else
%%%    	\State $q_i = 0$
%%%    \EndIf	
%%%    \State Send $q_1 \oplus q_2$ until at least one user receives it
%%%    \If {$N_i = 0$}
%%%        \State $R_i\gets R_i + 1$
%%%        \State $Q_i \gets Q_i \setminus \{q_i\}$
%%%    \EndIf    
%%%\EndWhile
%%%
%%%
%%%\end{algorithmic}
%%%\end{algorithm}

%\begin{algorithm}
%\caption{Two-user Feedback}\label{lag:two_user}
%\begin{algorithmic}[1]
%	\Procedure{MyProcedure}{}
%	\State $\textit{stringlen} \gets \text{length of }\textit{string}$
%	\State $i \gets \textit{patlen}$
%	\BState \emph{top}:
%	\If {$i > \textit{stringlen}$} \Return false
%	\EndIf
%	\State $j \gets \textit{patlen}$
%	\BState \emph{loop}:
%	\If {$\textit{string}(i) = \textit{path}(j)$}
%	\State $j \gets j-1$.
%	\State $i \gets i-1$.
%	\State \textbf{goto} \emph{loop}.
%	\State \textbf{close};
%	\EndIf
%	\State $i \gets i+\max(\textit{delta}_1(\textit{string}(i)),\textit{delta}_2(j))$.
%	\State \textbf{goto} \emph{top}.
%	\EndProcedure
%\end{algorithmic}
%\end{algorithm}

Our algorithm has two appealing properties. The first is that it involves only transmissions that are \emph{instantly decodable}, which is seldom the case when channel codes are used.  
%Here, at any time $t$, a user can produce the reconstruction $\hat{S}^{N}|_t$.  
Secondly, this coding scheme involves only transmissions that are \emph{distortion-innovative}.  This means that any successfully received channel symbol can be immediately used to reconstruct a single source symbol that was hitherto unknown.  %That is, given that a user successfully received a new channel symbol at time $t$, we have that $d(S^N, \hat{S}^N|_t) < d(S^N, \hat{S}^N|_{(t-1)})$.  
%
In fact, our coding scheme has the property that for any latency $w \in [0, 1/(1 - \epsilon)]$, after $wN$ transmissions have been sent over the channel, an expected value of $\gamma = wN(1 - \epsilon)$ channel symbols were received, which leads to the decoding of precisely $\gamma$ source symbols.  The distortion achieved after $wN$ transmissions is thus seen to be $D = 1 - w(1 - \epsilon)$, which we readily recognize as 
%coinciding with 
the separation-based outer bound in~\eqref{eq:wid_optimal}.  Since the transmission of an instantly-decodable, distortion-innovative symbol does not require a channel encoder, we will sometimes refer to such transmissions as \emph{analog} transmissions.

In the next section, we study
a variation of our problem formulation in which only the \emph{stronger} user has a feedback channel available.  For this problem variation, we show that we can still achieve the optimal minmax latency despite the weaker user not having access to a feedback channel.

%Finally, we mention that while our present scheme uses feedback from both users, in Section~\ref{sec:one_sided_feedback} we study
%a variation of our problem formulation in which only the \emph{stronger} user has a feedback channel available.  In that section, we show that we can still achieve the optimal minmax latency despite the weaker user not having access to a feedback channel.
%can also be studied and will be reported in a forthcoming work. 


%\subsection{Single Feedback Channel}

\iffalse
We demonstrate the importance of sending instantly decodable, distortion-innovative symbols by considering a slightly altered problem setup.  Consider if we again have two users that wish to fractionally reconstruct the source, however, this time there is only a single feedback channel available.  We assume that this channel is available to the user with the better channel quality, i.e., at time $T$, only $\{N_1(t)\}_{t = 1, 2, \ldots, T- 1}$ is universally known.  Our motivation for this problem arises from considering broadcast channels with a large number of users.  Given the large amount of feedback that may be available, it may not be possible to process it all, and so we propose a model in which the coding ignores a subset of the feedback available.  In our current discussion we furthermore assume a \emph{physically} degraded channel.  The case of stochastically degraded channels will be explored in a forthcoming work.

We show that in such a case, the following simple repetition-based scheme is optimal.  We repeatedly transmits each source symbol uncoded until we are alerted  that user~1 has received it, which is clearly optimal for user~1.  To see that it is also optimal for user~2, we note that since the channel is physically degraded, a received symbol by user~2's channel at time $t$ implies that user~1 has also received the same symbol at time $t$. Thus, upon the reception of any source symbol by user~2, the transmitter will begin transmitting a different instantly decodable, distortion-innovative symbol.  It is therefore not hard to see that user~2 can also achieve an optimality.
\fi


