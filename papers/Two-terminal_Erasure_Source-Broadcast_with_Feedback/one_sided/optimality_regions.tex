\subsubsection{Case~\Rmnum{1}: $0 \leq d_1 \leq \donedaggall$}

% -------------------------------------------------------------
% d_1 < 
% -------------------------------------------------------------

\begin{figure}
	\centering
%	\includegraphics[scale=0.8]{3/one_sided/fig/system_model_one_sided}
	\includegraphics[scale=1]{one_sided/fig/regions_d1_small}
%%	\includegraphics[width=0.6\textwidth]{outer_bound.png}
	\caption{The different regions requiring separate coding schemes when $0 \leq d_1 \leq \donedaggall$.}
	\label{fig:regions_d1_small}
\end{figure}

\begin{table}
	\begin{center}
		\begin{tabular}{| c | c |}
%    \caption{The justification for the ordering}
			\hline
			\multicolumn{2}{|c|}{{\bf Ordering of Boundaries for $\depstwo$ when $0 \leq d_1 \leq \donedaggall$}} \\
			\hline
			{\bf Inequality} & {\bf Justification}   \\ \hline
			$\cstar < d_1/\epsilon_1$ & $\epsot < \epsilon_2$ \\ \hline 
			$d_1/\epsilon_1 < \mydstar $ & $d_1 < \epsilon_1$ \\ \hline 
			$\mydstar \leq \astar$ & $d_1 \leq \donedaggall$ \\ \hline 
			$\astar < \bstar$ & $\epsot < 1$ \\ \hline 
			$\bstar < 1$ & $d_1 < \epsilon_1$ \\
			\hline
		\end{tabular}
	\end{center}
	\caption{We justify the ordering of the region boundaries of Figure~\ref{fig:regions_d1_small} when $0 \leq d_1 \leq \donedaggall$.  In the left column, we have the ordering between two boundary points, and in the right column, we show the necessary and sufficient condition that justifies the ordering.}	
	\label{tab:d1_small}
\end{table}

We first mention that the upper bound in the assumption $d_1 \leq \donedaggall$ can be either positive or negative depending on the values of $\epsilon_1$ and $\epsot$.  In the case that the upper bound is positive and $d_1$ is in this region, we now go through the process of finding the values of $\gammaparam$ and $\omegaparam$ such that $\LHSfuncparam = \epsilon_2 - d_2$.  

%We will see that this region is the most difficult in the sense that it has the most distinct coding regions and 

We begin by dividing the number line for $\depstwo$ in Figure~\ref{fig:regions_d1_small}, where we have justified the ordering of each boundary point with Table~\ref{tab:d1_small}.  For example, in the third row of the table, we justify the ordering that $\mydstar \leq \astar$ with the necessary and sufficient condition that $d_1 \leq \donedaggall$, which is the assumption in this region of $d_1$ that we consider.  We enumerate all regions of Figure~\ref{fig:regions_d1_small} and provide the values of $\gammaparam$ and $\omegaparam$.

%\renewcommand{\labelenumi}{[\theenumi]} 
%\begin{description}[before={\renewcommand\makelabel[1]{\bfseries ##1.}}]
%    \setcounter{enumi}{#1}
%    \addtocounter{enumi}{1}

%\begin{description}
\begin{LaTeXdescription}
	\item[Region~\Rmnum{5}] In this region, we set $\gammaparam = \omegaparam = 0$ and recover the unmodified repetition coding scheme discussed in the beginning of Section~\ref{subsubsec:repetition_coding}.  In fact, we do not require that $\LHSfuncparam = \epsilon_2 - d_2$ in this region.  Instead, by repeating all $N(\epsilon_1 - d_1)$ source symbols, by Lemma~\ref{lem:repetition}, we can work out that $N\kk$ uncoded source symbols are received by the weaker user, where $\kk$ is given by~\eqref{eq:kk}.  We can confirm that in Region~\Rmnum{5}, in which $\depstwo \geq \bstar$, the distortion requirement of the weaker user is sufficiently large so that it can be met by the amount of uncoded symbols he receives.
	\item[Region~\Rmnum{4}] In this region, we set $\omegaparam = 0$, and $\gammaparam = \gammatilde$ where 
		\begin{align}
			\gammatilde &= \frac{\epsilon_1(\epsilon_2 - d_2)(1 - \epsot) - \epsot(\epsilon_1 - d_1)(1 - \epsilon_2)}{\epsilon_1(1 - \epsot)(\epsilon_2 - \epsot)}.
			\label{eq:gammatilde}
		\end{align}
		Within this region, the weaker user's distortion constraint is sufficiently low so that additional coded symbols must be transmitted.  We can confirm that this choice of $\gammaparam$ and $\omegaparam$ results in $\LHSfuncparam = \epsilon_2 - d_2$ in~\eqref{eq:LHSfuncparam}.  Furthermore, the conditions $\gammatilde \geq 0$ and $\gammatilde \leq 1$ are equivalent to $\depstwo \leq \bstar$ and $\depstwo \geq \astar$ respectively, which are satisfied in this region.
	\item [Region~\Rmnum{3}]  In this region, we set $\gammaparam = 0$ and $\omegaparam = \omegastar$, where
		\begin{align}
			\omegastar &= \frac{\epsilon_1(\epsilon_2 - d_2)(1 - \epsot) - \epsot(\epsilon_1 - d_1)(1 - \epsilon_2)}{\epsot(\epsilon_1 - d_1)(\epsilon_2 - \epsot)}.
%			\omegastar &= \frac{(\epsilon_2 - d_2)(1 - \epsilon_1\epsilon_2) - \epsilon_2(\epsilon_1 - d_1)(1 - \epsilon_2)}{\epsilon_{2}^{2}(\epsilon_1 - d_1)(1 - \epsilon_1)}.
		\end{align}
		We can confirm that this choice of $\gammaparam$ and $\omegaparam$ results in $\LHSfuncparam = \epsilon_2 - d_2$ in~\eqref{eq:LHSfuncparam}.  Furthermore, the conditions $\omegastar \geq 0$ and $\omegastar \leq 1$ are equivalent to $\depstwo \leq \bstar$ and $\depstwo \geq \mydstar$ respectively, which are satisfied in this region.%		To see how we derive this value of $\omegaparam$, we first consider~\eqref{eq:LHSfuncparam}.
	
	\item[Region~\Rmnum{2}]  In this region, we set $\omegaparam= 1$, $\gammaparam = \gammahat$, where
		\begin{align}
			\gammahat &= \frac{\epsilon_1(\epsilon_2 - d_2) - \epsot(\epsilon_1 - d_1)}{\epsilon_1(\epsilon_2 - \epsot)}.
%			\gammahat &= \frac{(\epsilon_2 - d_2) - (\epsilon_1 - d_1)\epsilon_2}{(1 - \epsilon_1)\epsilon_2}.
			\label{eq:gammahat}
		\end{align}
		We can confirm that this choice of $\gammaparam$ and $\omegaparam$ results in $\LHSfuncparam = \epsilon_2 - d_2$ in~\eqref{eq:LHSfuncparam}.  Furthermore, the conditions $\gammahat \geq 0$ and $\gammahat \leq 1$ are equivalent to $\depstwo \leq \mydstar$ and $\depstwo \geq \cstar$ respectively, which are satisfied in this region.%		To see how we derive this 
	
	\item[Region~\Rmnum{1}]  In this region, we ignore feedback altogether and use the successive segmentation coding scheme of~\cite{TLKS_TIT16}, which showed that both users can be point-to-point optimal if $\depstwo \leq \depsone$.
%\end{description}
\end{LaTeXdescription}

%\begin{enumerate}[label = \uline{\textit{Step \arabic*:}}]
%	\item hello
%\end{enumerate}
%\begin{etaremune}[label = \uline{\textit{Step \arabic*:}}]
%\end{etaremune}
%\begin{description}
%	\item[ Region~\Rmnum{5}: $\gammaparam = \omegaparam = 0$]  In this region 
%\end{description}


\subsubsection{Case~\Rmnum{2}: $\donedaggall < d_1 < \doneddaggall$}

We again divide the number line for $\depstwo$ in Figure~\ref{fig:regions_d1_mid} where we have justified the ordering of each boundary point with Table~\ref{tab:d1_mid}.  Notice however, that in the ranges of $d_1$ we now consider, the boundary points $\astar$ and $\mydstar$ have swapped positions compared to Figure~\ref{fig:regions_d1_small}.  We again enumerate all regions of Figure~\ref{fig:regions_d1_mid}.

% -------------------------------------------------------------
% < d_1 < 
% -------------------------------------------------------------
\begin{figure}
	\centering
%	\includegraphics[scale=0.8]{3/one_sided/fig/system_model_one_sided}
	\includegraphics[scale=1]{one_sided/fig/regions_d1_mid}
%%	\includegraphics[width=0.6\textwidth]{outer_bound.png}
	\caption{The different regions requiring separate coding schemes when $\donedaggall < d_1 < \doneddaggall$.}
	\label{fig:regions_d1_mid}
\end{figure}

\begin{table}
	\begin{center}
		\begin{tabular}{| c | c |}
%    \caption{The justification for the ordering}
			\hline
			\multicolumn{2}{|c|}{{\bf Ordering of Boundaries for $\depstwo$ when $\donedaggall < d_1 < \doneddaggall$}} \\
			\hline
			{\bf Inequality} & {\bf Justification}   \\ \hline
			$\cstar < d_1/\epsilon_1$ & $\epsot < \epsilon_1$ \\ \hline 
			$d_1/\epsilon_1 < \astar $ & $d_1 < \doneddaggall$ \\ \hline 
			$\astar < \mydstar $ & $d_1 > \donedaggall$ \\ \hline 			
			$\mydstar < \bstar$ & $d_1 < \epsilon_1$ \\ \hline 
%			$\astar < \bstar$ & $\epsilon_1\epsilon_2 < 1$ \\ \hline 
			$\bstar < 1$ & $d_1 < \epsilon_1$ \\
			\hline
		\end{tabular}
	\end{center}
	\caption{We justify the ordering of the region boundaries of Figure~\ref{fig:regions_d1_mid} when $\donedaggall < d_1 < \doneddaggall$.  In the left column, we have the ordering between two boundary points, and in the right column, we show the necessary and sufficient condition that justifies the ordering.}	
	\label{tab:d1_mid}	
\end{table}


%\begin{description}
\begin{LaTeXdescription}
	\item[Region~\Rmnum{9}] In this region, we set $\gammaparam = \omegaparam = 0$ and recover the unmodified repetition coding scheme discussed in the beginning of Section~\ref{subsubsec:repetition_coding} (see the description of Region~\Rmnum{5} in the previous section).

	\item[Region~\Rmnum{8}] In this region, we set $\omegaparam = 0$, and $\gammaparam = \gammatilde$ where $\gammatilde$ is given by~\eqref{eq:gammatilde} (see the description of Region~\Rmnum{4} in the previous section).
	
	\item[Region~\Rmnum{7}]  In this region, we set $\omegaparam= 1$, $\gammaparam = \gammahat$, where $\gammahat$ is given by~\eqref{eq:gammahat} (see the description of Region~\Rmnum{2} in the previous section).

	
	\item[Region~\Rmnum{6}]  In this region, we again ignore feedback altogether and use the successive segmentation coding scheme of the previous chapter (see the description of Region~\Rmnum{1} in the previous section).
%\end{description}
\end{LaTeXdescription}

\subsubsection{Case~\Rmnum{3}: $\doneddaggall \leq d_1 < \epsilon_1$}

For the final case of $d_1$ we consider, we again divide the number line for $\depstwo$ in Figure~\ref{fig:regions_d1_big}, where we have justified the ordering of each boundary point with Table~\ref{tab:d1_big}.  Again, the values of $d_1$ we consider have resulted in some of the boundaries in Figure~\ref{fig:regions_d1_big} moving relative to Figure~\ref{fig:regions_d1_mid}, most notably that now $\astar < \depsone$.  We again enumerate all regions of Figure~\ref{fig:regions_d1_mid}.

% -------------------------------------------------------------
% d_1 > 
% -------------------------------------------------------------
\begin{figure}
	\centering
%	\includegraphics[scale=0.8]{3/one_sided/fig/system_model_one_sided}
	\includegraphics[scale=1]{one_sided/fig/regions_d1_big}
%%	\includegraphics[width=0.6\textwidth]{outer_bound.png}
	\caption{The different regions requiring separate coding schemes when $\doneddaggall \leq d_1 < \epsilon_1$.}
	\label{fig:regions_d1_big}
\end{figure}

\begin{table}
	\begin{center}
		\begin{tabular}{| c | c |}
%    \caption{The justification for the ordering}
			\hline
			\multicolumn{2}{|c|}{{\bf Ordering of Boundaries for $\depstwo$ when $\doneddaggall \leq d_1 < \epsilon_1$}} \\
			\hline
			{\bf Inequality} & {\bf Justification}   \\ \hline
			$0 < \cstar$ & $d_1 > 0$ \\ \hline 
			$\cstar < \astar$ & $d_1 < \epsilon_1$ \\ \hline 
			$\astar < d_1/\epsilon_1$ & $d_1 \geq \doneddaggall$ \\ \hline 
			$d_1/\epsilon_1 < \bstar$ &$d_1 < \epsilon_1$ \\ \hline 
			$\bstar < 1$ & $d_1 < \epsilon_1$ \\
			\hline
		\end{tabular}
	\end{center}
	\caption{We justify the ordering of the region boundaries of Figure~\ref{fig:regions_d1_big} when $\doneddaggall \leq d_1 < \epsilon_1$.  In the left column, we have the ordering between two boundary points, and in the right column, we show the necessary and sufficient condition that justifies the ordering.}	
	\label{tab:d1_big}	
\end{table}

%\begin{description}
\begin{LaTeXdescription}
	\item[Region~\Rmnum{12}] In this region, we set $\gammaparam = \omegaparam = 0$ and recover the unmodified repetition coding scheme discussed in the beginning of Section~\ref{subsubsec:repetition_coding} (see the description of Region~\Rmnum{5} in the previous section).

	\item[Region~\Rmnum{11}] In this region, we set $\omegaparam = 0$, and $\gammaparam = \gammatilde$ where $\gammatilde$ is given by~\eqref{eq:gammatilde} (see the description of Region~\Rmnum{4} in the previous section).

	
	\item[Region~\Rmnum{10}]  In this region, we again ignore feedback altogether and use the successive segmentation coding scheme of the previous chapter (see the description of Region~\Rmnum{1} in the previous section).
%\end{description}
\end{LaTeXdescription}
