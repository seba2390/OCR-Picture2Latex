\documentclass{IEEEtran}

\usepackage{graphicx,amssymb,amstext,amsmath,cite}
%\usepackage{amssymb,amstext,amsmath}
%\usepackage[dvips]{graphicx}
%\usepackage{tikz}
\usepackage[pdf]{pstricks}
\usepackage{pst-sigsys}
\usepackage{auto-pst-pdf}
\let\clipbox\relax

\usepackage{pgfplots}

\usepackage{standalone}
%\usepackage[hidelinks]{hyperref}
%\usepackage{hyperref}
%\usepackage{caption}
%%\caption@ifamsclass{%
%%\caption@InfoNoLine{AMS or SMF document class}
%\setlength\belowcaptionskip{0pt}% set to 12pt by AMS class
%%}
%\setlength{\belowcaptionskip}{-12pt}
\setlength{\intextsep}{0pt plus 2pt} 

%\usepackage{pstricks-add}
%\usepackage{pst-sigsys,pstricks-add}
%\usepackage{pstricks-add}
%\usepackage{amsthm}
\usepackage{fullpage}
%\usepackage{nccmath}
%\usepackage{kbordermatrix}

%\usepackage{preview}

\pgfplotsset{compat=newest} 
\pgfplotsset{plot coordinates/math parser=false}

\newtheorem{myTheorem}{Theorem}
\newtheorem{mydef}{Definition}
\newcommand{\bstar}[1]{\ensuremath{b_{#1}^{*}}}
\newcommand{\onePlusPN}[1]{\ensuremath{ \left( 1 + \frac{P}{N_{#1}}\right) }}
\newcommand{\Rmnum}[1]{\MakeUppercase{\romannumeral #1}}
\newcommand{\Dfunc}[1]{\mathcal{D}_{#1}}
%\newcommand{\SINR}[2]{\ensuremath{\left(  1 + \frac{\beta_{#1} P}{N_{#2} + (1 - \beta_{#1})P} \right) }}
\newcommand{\SINR}[2]{\ensuremath{\left(  1 + \frac{(1 - \beta_{#1}) P}{N_{#2} + \beta_{#1}P} \right) }}
\newcommand{\myexp}{\frac{b_{2} - 1}{b_{1} - 1}}
\newcommand{\privateSNR}{\left( 1 + \frac{ \beta P }{N_{1}}\right)}
%\newcommand{\betaN}{\ensuremath{\beta_{0}}}
%\renewcommand{\thefigure}{\thechapter.\arabic{figure}}
\newcounter{MYtempeqncnt}

%\input{def_header}
%% ABBREVIATIONS FOR CHARACTERS IN VARIOUS FONTS

% STANDARD CHARACTERS

\newcommand{\ba}{{\mathbf{a}}}
\newcommand{\bah}{{\hat{\ba}}}
\newcommand{\ah}{{\hat{a}}}
\newcommand{\Ah}{{\hat{A}}}
\newcommand{\cA}{{\mathcal{A}}}
\newcommand{\at}{{\tilde{a}}}
\newcommand{\bat}{{\tilde{\ba}}}
\newcommand{\At}{{\tilde{A}}}
\newcommand{\bA}{{\mathbf{A}}}
\newcommand{\ac}{a^{\ast}}

\newcommand{\bb}{{\mathbf{b}}}
\newcommand{\bbt}{{\tilde{\bb}}}
\newcommand{\cB}{{\mathcal{B}}}
\newcommand{\tb}{{\tilde{b}}}
\newcommand{\tB}{{\tilde{B}}}
\newcommand{\hb}{{\hat{b}}}
\newcommand{\hB}{{\hat{B}}}
\newcommand{\bB}{{\mathbf{B}}}

\newcommand{\bc}{{\mathbf{c}}}
\newcommand{\bch}{{\hat{\mathbf{c}}}}
\newcommand{\bC}{{\mathbf{C}}}
\newcommand{\cC}{{\mathcal{C}}}
\newcommand{\ct}{{\tilde{c}}}
\newcommand{\Ct}{{\tilde{C}}}
\newcommand{\ctc}{\ct^{\ast}}

\newcommand{\bd}{{\mathbf{d}}}
\newcommand{\bD}{{\mathbf{D}}}
\newcommand{\cD}{{\mathcal{D}}}
\newcommand{\hd}{{\hat{d}}}  % old: \dh
\newcommand{\dt}{{\tilde{d}}}
\newcommand{\bdt}{{\tilde{\bd}}}
\newcommand{\Dt}{{\tilde{D}}}
\newcommand{\dtc}{\dt^{\ast}}

\newcommand{\et}{{\tilde{e}}}
\newcommand{\bfe}{{\mathbf{e}}}
\newcommand{\bE}{{\mathbf{E}}}
\newcommand{\cE}{{\mathcal{E}}}
\newcommand{\cEt}{{\tilde{\cE}}}
\newcommand{\cEb}{{\bar{\cE}}}
\newcommand{\bcE}{{\mathbf{\cE}}}  % bf cal E doesn't exist

\newcommand{\bff}{{{\mathbf{f}}}}
\newcommand{\bF}{{\mathbf{F}}}
\newcommand{\cF}{{\mathcal{F}}}
\newcommand{\ft}{{\tilde{f}}}
\newcommand{\Ft}{{\tilde{F}}}
\newcommand{\Fh}{{\hat{F}}}
\newcommand{\ftc}{\ft^{\ast}}
\newcommand{\bft}{{\tilde{\bff}}}
\newcommand{\bFt}{{\tilde{\bF}}}
\newcommand{\fh}{{\hat{f}}}

\newcommand{\bg}{{\mathbf{g}}}
\newcommand{\gt}{{\tilde{g}}}
\newcommand{\bgt}{{\tilde{\bg}}}
\newcommand{\bG}{{\mathbf{G}}}
\newcommand{\cG}{{\mathcal{G}}}
\newcommand{\Gt}{{\tilde{\bG}}}
\newcommand{\Ge}{{G_\mathrm{eff}}}


\newcommand{\hti}{{\tilde{h}}}
\newcommand{\Hti}{{\tilde{H}}}
\newcommand{\bh}{{\mathbf{h}}}
\newcommand{\bht}{{\tilde{\bh}}}
\newcommand{\Hh}{{\hat{H}}}
\newcommand{\bH}{{\mathbf{H}}}
\newcommand{\bHh}{{\hat{\mathbf{H}}}}

\newcommand{\ih}{{\hat{\imath}}}
\newcommand{\bI}{{\mathbf{I}}}
\newcommand{\cI}{{\mathcal{I}}}

\newcommand{\jh}{{\hat{\jmath}}}
\newcommand{\bJ}{{\mathbf{J}}}
\newcommand{\cJ}{{\mathcal{J}}}
\newcommand{\Jt}{{\tilde{J}}}

\newcommand{\bk}{{\mathbf{k}}}
\newcommand{\bK}{{\mathbf{K}}}
\newcommand{\Kt}{{\tilde{K}}}
\newcommand{\Kh}{{\hat{K}}}
\newcommand{\cK}{{\mathcal{K}}}

\newcommand{\cl}{\ell}
\newcommand{\bL}{{\mathbf{L}}}
\newcommand{\cL}{{\mathcal{L}}}

\newcommand{\mb}{{\mathbf{m}}}
\newcommand{\mh}{{\hat{m}}}
\newcommand{\bM}{{\mathbf{M}}}
\newcommand{\bm}{{\mathbf{m}}}
\newcommand{\cM}{{\mathcal{M}}}


\newcommand{\cN}{{\mathcal{N}}}
\newcommand{\CN}{{\mathcal{CN}}}
\newcommand{\Nt}{{\tilde{N}}}
\newcommand{\tN}{{\tilde{N}}}  % backward compatibility

\newcommand{\bo}{{\mathbf{o}}}
\newcommand{\cO}{{\mathcal{O}}}

\newcommand{\bp}{{\mathbf{p}}}
\newcommand{\bP}{{\mathbf{P}}}
\newcommand{\cP}{{\mathcal{P}}}
\newcommand{\ph}{{\hat{p}}}
\newcommand{\Ph}{{\hat{P}}}
\newcommand{\Pt}{{\tilde{P}}}
\newcommand{\Ptt}{{\tilde{\tilde{P}}}}
\newcommand{\bq}{{\mathbf{q}}}
\newcommand{\cQ}{{\mathcal{Q}}}
\newcommand{\bQ}{{\mathbf{Q}}}

\newcommand{\br}{{\mathbf{r}}}
\newcommand{\bR}{{\mathbf{R}}}
\newcommand{\cR}{{\mathcal{R}}}
\newcommand{\Rt}{{\tilde{R}}}

\newcommand{\sh}{{\hat{s}}}
\newcommand{\sck}{{\check{s}}}
\newcommand{\shh}{{\Hat{\Hat{s}}}}
\newcommand{\bs}{{\mathbf{s}}}
\newcommand{\bsh}{{\hat{\mathbf{s}}}}
\newcommand{\bsc}{{\check{\mathbf{s}}}}
\newcommand{\bshh}{{\Hat{\Hat{\mathbf{s}}}}}
\newcommand{\bS}{{\mathbf{S}}}
\newcommand{\cS}{{\mathcal{S}}}
\newcommand{\st}{{\tilde{s}}}

\newcommand{\bT}{{\mathbf{T}}}
\newcommand{\cT}{{\mathcal{T}}}
\newcommand{\bu}{{\mathbf{u}}}
\newcommand{\bU}{{\mathbf{U}}}
\newcommand{\bUt}{{\tilde{\bU}}}
\newcommand{\ut}{{\tilde{u}}}
\newcommand{\cU}{{\mathcal{U}}}

\newcommand{\vh}{{\hat{v}}}
\newcommand{\bv}{{\mathbf{v}}}
\newcommand{\bV}{{\mathbf{V}}}
\newcommand{\cV}{{\mathcal{V}}}

\newcommand{\bw}{{\mathbf{w}}}
\newcommand{\bW}{{\mathbf{W}}}
\newcommand{\cW}{{\mathcal{W}}}
\newcommand{\wt}{{\tilde{w}}}

\newcommand{\bx}{{\mathbf{x}}}
\newcommand{\bxt}{{\tilde{\bx}}}
\newcommand{\xt}{{\tilde{x}}}
\newcommand{\Xt}{{\tilde{X}}}
\newcommand{\bX}{{\mathbf{X}}}
\newcommand{\cX}{{\mathcal{X}}}
\newcommand{\bXt}{{\tilde{\bX}}}
\newcommand{\xh}{{\hat{x}}}
\newcommand{\xc}{{\check{x}}}
\newcommand{\xhh}{{\Hat{\Hat{x}}}}
\newcommand{\bxh}{{\hat{\bx}}}
\newcommand{\bxc}{{\check{\bx}}}
\newcommand{\bxhh}{{\Hat{\hat{\bx}}}}

\newcommand{\cY}{{\mathcal{Y}}}
\newcommand{\by}{{\mathbf{y}}}
\newcommand{\byt}{{\tilde{\by}}}
\newcommand{\bY}{{\mathbf{Y}}}
\newcommand{\Yt}{{\tilde{Y}}}
\newcommand{\yt}{{\tilde{y}}}
\newcommand{\yh}{{\hat{y}}}

\newcommand{\zt}{{\tilde{z}}}
\newcommand{\zh}{{\hat{z}}}
\newcommand{\bz}{{\mathbf{z}}}
\newcommand{\bZ}{{\mathbf{Z}}}
\newcommand{\cZ}{{\mathcal{Z}}}

% GREEK CHARACTERS

\newcommand{\al}{\alpha}
\newcommand{\bal}{{\boldsymbol{\al}}}
\newcommand{\balh}{{\hat{\boldsymbol{\al}}}}
\newcommand{\alh}{{\hat{\al}}}
\newcommand{\aln}{{\bar{\al}}}

\newcommand{\bt}{\boldsymbol{t}}
%\newcommand{\bt}{\beta}
\newcommand{\btt}{{\tilde{\bt}}}
\newcommand{\btht}{{\hat{\bt}}}

\newcommand{\g}{\gamma}
\newcommand{\G}{\Gamma}
\newcommand{\bGa}{{\boldsymbol{\Gamma}}}
\newcommand{\gh}{{\hat{\g}}}

\newcommand{\de}{\delta}
\newcommand{\del}{\delta}
\newcommand{\De}{\Delta}
\newcommand{\Deh}{{\hat{\Delta}}}
\newcommand{\bde}{{\boldsymbol{\de}}}
\newcommand{\bDe}{{\boldsymbol{\De}}}

\newcommand{\e}{\epsilon}
\newcommand{\eps}{\varepsilon}

\newcommand{\etah}{{\hat{\eta}}}
\newcommand{\bpi}{{\boldsymbol{\pi}}}

\newcommand{\pht}{{\tilde{\phi}}}
\newcommand{\Pht}{{\tilde{\Phi}}}

\newcommand{\pst}{{\tilde{\psi}}}
\newcommand{\Pst}{{\tilde{\Psi}}}

\newcommand{\s}{\sigma}
\newcommand{\sih}{\hat{\sigma}}

\newcommand{\z}{\zeta}
\newcommand{\ztt}{{\tilde{\z}}}
\newcommand{\ztb}{{\bar{\z}}}

% \newcommand{\th}{\theta} % symbol name used by other latex package
\newcommand{\thh}{{\hat{\theta}}}
\newcommand{\Thh}{{\hat{\Theta}}}
\newcommand{\Th}{\Theta}
\newcommand{\bth}{{\boldsymbol{\theta}}}
\newcommand{\bTh}{{\boldsymbol{\Theta}}}
\newcommand{\bThh}{{\hat{\bTh}}}
\newcommand{\Tht}{{\tilde{\Theta}}}

\newcommand{\la}{\lambda}
\newcommand{\La}{\Lambda}
\newcommand{\lam}{\lambda}  % backward compatibility
\newcommand{\Lam}{\Lambda}  % backward compatibility
\newcommand{\bLa}{{\boldsymbol{\La}}}
\newcommand{\lah}{{\hat{\lam}}}

\newcommand{\bmu}{{\boldsymbol{\mu}}}

\newcommand{\bXi}{{\boldsymbol{\Xi}}}
%\newcommand{\rvx}{{\mathsf{x}}}
%\newcommand{\rvy}{{\mathrm{y}}}
\newcommand{\bPi}{{\boldsymbol{\Pi}}}

\newcommand{\rht}{{\tilde{\rho}}}
\newcommand{\rhc}{{\check{\rho}}}

\newcommand{\bSi}{{\boldsymbol{\Sigma}}}

\newcommand{\ups}{\upsilon}
\newcommand{\Ups}{\Upsilon}
\newcommand{\bUp}{{\boldsymbol{\Ups}}}

\newcommand{\bPs}{{\boldsymbol{\Psi}}}

\newcommand{\w}{\omega}
\newcommand{\wh}{{\hat{\omega}}}
\newcommand{\W}{\Omega}

\newcommand{\dagg}{\dagger}
\newcommand{\dagbh}{{\mathbf{h}}^\dagger}
\newcommand{\dagbH}{{\mathbf{H}}^\dagger}

%\DeclareMathAlphabet{\mathbsf}{OT1}{cmss}{bx}{n}% bold sans serif
\DeclareMathAlphabet{\mathssf}{OT1}{cmss}{m}{sl}% slanted sans serif

% define some useful uppercase Greek letters in regular and bold sf
\DeclareSymbolFont{bsfletters}{OT1}{cmss}{bx}{n}
\DeclareSymbolFont{ssfletters}{OT1}{cmss}{m}{n}
\DeclareMathSymbol{\bsfGamma}{0}{bsfletters}{'000}
\DeclareMathSymbol{\ssfGamma}{0}{ssfletters}{'000}
\DeclareMathSymbol{\bsfDelta}{0}{bsfletters}{'001}
\DeclareMathSymbol{\ssfDelta}{0}{ssfletters}{'001}
\DeclareMathSymbol{\bsfTheta}{0}{bsfletters}{'002}
\DeclareMathSymbol{\ssfTheta}{0}{ssfletters}{'002}
\DeclareMathSymbol{\bsfLambda}{0}{bsfletters}{'003}
\DeclareMathSymbol{\ssfLambda}{0}{ssfletters}{'003}
\DeclareMathSymbol{\bsfXi}{0}{bsfletters}{'004}
\DeclareMathSymbol{\ssfXi}{0}{ssfletters}{'004}
\DeclareMathSymbol{\bsfPi}{0}{bsfletters}{'005}
\DeclareMathSymbol{\ssfPi}{0}{ssfletters}{'005}
\DeclareMathSymbol{\bsfSigma}{0}{bsfletters}{'006}
\DeclareMathSymbol{\ssfSigma}{0}{ssfletters}{'006}
\DeclareMathSymbol{\bsfUpsilon}{0}{bsfletters}{'007}
\DeclareMathSymbol{\ssfUpsilon}{0}{ssfletters}{'007}
\DeclareMathSymbol{\bsfPhi}{0}{bsfletters}{'010}
\DeclareMathSymbol{\ssfPhi}{0}{ssfletters}{'010}
\DeclareMathSymbol{\bsfPsi}{0}{bsfletters}{'011}
\DeclareMathSymbol{\ssfPsi}{0}{ssfletters}{'011}
\DeclareMathSymbol{\bsfOmega}{0}{bsfletters}{'012}
\DeclareMathSymbol{\ssfOmega}{0}{ssfletters}{'012}

\newcommand{\fxfm}{\stackrel{\mathcal{F}}{\longleftrightarrow}}
\newcommand{\lxfm}{\stackrel{\mathcal{L}}{\longleftrightarrow}}
\newcommand{\zxfm}{\stackrel{\mathcal{Z}}{\longleftrightarrow}}

\DeclareMathOperator*{\gltop}{\gtreqless}
\newcommand{\glt}{\;\gltop^{\Hh=\svH_1}_{\Hh=\svH_0}\;}
\newcommand{\glty}{\;\gltop^{\Hh(\svy)=\svH_1}_{\Hh(\svy)=\svH_0}\;}
\newcommand{\gltby}{\;\gltop^{\Hh(\svby)=\svH_1}_{\Hh(\svby)=\svH_0}\;}
\DeclareMathOperator*{\geltop}{\genfrac{}{}{0pt}{}{\ge}{<}}
\newcommand{\gelty}{\;\geltop^{\Hh(\svy)=\svH_1}_{\Hh(\svy)=\svH_0}\;}
\newcommand{\geltby}{\;\geltop^{\Hh(\svby)=\svH_1}_{\Hh(\svby)=\svH_0}\;}
\renewcommand{\pe}{\Pr(e)}
\renewcommand{\defeq}{\triangleq}
\newcommand{\like}{\svlike}
\newcommand{\rvlike}{\mathssf{L}}
\newcommand{\sst}{\cl}
\newcommand{\svlike}{L}
\newcommand{\llike}{\rvllike}
\newcommand{\rvllike}{\cl}
\newcommand{\svllike}{l}
\newcommand{\bllike}{\rvbllike}
\newcommand{\rvbllike}{\boldsymbol{\cl}}
\newcommand{\svbllike}{\mathbf{l}}
\newcommand{\Qb}{\overline{Q}}
\renewcommand{\comb}[2]{\binom{#1}{#2}}


%% Random/sample variable/vector declarations.  Please add in alphabetical
%% order.  First section is for capitals.  Second for lower case.
% Capitals
\newcommand{\rvA}{{\mathssf{A}}}    % A
\newcommand{\svA}{A}
\newcommand{\rvbA}{{\mathbsf{A}}}
\newcommand{\svbA}{{\mathbf{A}}}
\newcommand{\rvD}{{\mathssf{D}}}    % D
\newcommand{\svD}{D}
\newcommand{\rvbD}{{\mathbsf{D}}}
\newcommand{\svbD}{{\mathbf{D}}}
\newcommand{\rvFh}{{\hat{\mathssf{F}}}} % F
\newcommand{\rvF}{{\mathssf{F}}}
\newcommand{\rvHh}{{\hat{\mathssf{H}}}} % H
\newcommand{\rvH}{{\mathssf{H}}}
\newcommand{\svH}{H}
\newcommand{\svHh}{{\hat{\svH}}}
\newcommand{\rvL}{{\mathssf{L}}}    % L
\newcommand{\rvN}{{\mathssf{N}}}    % N
\newcommand{\rvR}{{\mathssf{R}}}    % R
\newcommand{\rvRh}{{\hat{\rvR}}}
\newcommand{\rvS}{{\mathssf{S}}}    % S
\newcommand{\rvSh}{{\hat{\rvS}}}
\newcommand{\rvW}{{\mathssf{W}}}    % W
\newcommand{\rvX}{{\mathssf{X}}}    % X, random variable
\newcommand{\svX}{X}
\newcommand{\rvXt}{{\tilde{\rvX}}}
\newcommand{\rvY}{{\mathssf{Y}}}    % Y
\newcommand{\rvZ}{{\mathssf{Z}}}    % Z

\newcommand{\rva}{{\mathssf{a}}}    % a
\newcommand{\rvah}{{\hat{\rva}}}
\newcommand{\sva}{a}
\newcommand{\svah}{{\hat{\sva}}}
\newcommand{\rvba}{{\mathbsf{a}}}
\newcommand{\svba}{{\mathbf{a}}}
\newcommand{\rvhba}{\hat{{{\mathbsf{{a}}}}}}


\newcommand{\rvb}{{\mathssf{b}}}    % b
\newcommand{\rvbB}{{\mathbsf{B}}}   % b
\newcommand{\rvbb}{{\mathbsf{b}}}
\newcommand{\rvhbb}{\hat{{{\mathbsf{{b}}}}}}
\newcommand{\svbb}{{\mathbf{b}}}

\newcommand{\rvc}{{\mathssf{c}}}    % c
\newcommand{\rvch}{{\hat{\rvc}}}
\newcommand{\svc}{c}
\newcommand{\svch}{{\hat{\svc}}}
\newcommand{\rvbc}{{\mathbsf{c}}}
\newcommand{\svbc}{{\mathbf{c}}}

\newcommand{\rvd}{{\mathssf{d}}}    % d
\newcommand{\rvdh}{{\hat{\rvd}}}
\newcommand{\svd}{d}
\newcommand{\svdh}{{\hat{\svd}}}
\newcommand{\rvbd}{{\mathbsf{d}}}
\newcommand{\svbd}{{\mathbf{d}}}



\newcommand{\rve}{{\mathssf{e}}}    % e
\newcommand{\sve}{e}
\newcommand{\rvbe}{{\mathbsf{e}}}
\newcommand{\svbe}{{\mathbf{e}}}
\newcommand{\rvf}{{\mathssf{f}}}    % f
\newcommand{\rvhf}{{\hat{{\mathssf{f}}}}}    % f

\newcommand{\svf}{f}
\newcommand{\rvbf}{{\mathbsf{f}}}
\newcommand{\svbf}{{\mathbf{f}}}
\newcommand{\rvg}{{\mathssf{g}}}    % g
\newcommand{\svg}{g}
\newcommand{\rvbg}{{\mathbsf{g}}}
\newcommand{\rvbG}{{\mathbsf{G}}}
\newcommand{\svbg}{{\mathbf{g}}}
\newcommand{\rvh}{{\mathssf{h}}}    % h
\newcommand{\svh}{h}
\newcommand{\rvbh}{{\mathbsf{h}}}
\newcommand{\rvbH}{{\mathbsf{H}}}
\newcommand{\svbh}{{\mathbf{h}}}
\newcommand{\rvk}{{\mathssf{k}}}    % k
\newcommand{\rvhk}{{\hat{\mathssf{k}}}}    % k

\newcommand{\svk}{k}
\newcommand{\rvl}{{\mathssf{l}}}    % l

\newcommand{\rvm}{{\mathssf{m}}}    % m
\newcommand{\svm}{m}
\newcommand{\rvbm}{{\mathbsf{m}}}
\newcommand{\svbm}{{\mathbf{m}}}
\newcommand{\rvn}{{\mathssf{n}}}    % n
\newcommand{\rvbn}{{\mathbsf{n}}}
\newcommand{\rvp}{{\mathssf{p}}}    % p
\newcommand{\svp}{p}
\newcommand{\rvq}{{\mathssf{q}}}    % q
\newcommand{\svq}{q}
\newcommand{\svQ}{Q}
\newcommand{\rvr}{{\mathssf{r}}}    % r
\newcommand{\rvbr}{{\mathbsf{r}}}
\newcommand{\svr}{r}
\newcommand{\rvs}{{\mathssf{s}}}    % s
\newcommand{\rvbs}{{\mathbsf{s}}}
\newcommand{\svs}{s}
\newcommand{\svbs}{{\mathbf{s}}}
\newcommand{\rvt}{{\mathssf{t}}}    % t
\newcommand{\rvbt}{{\mathbsf{t}}}
\newcommand{\rvhbt}{\hat{{\mathbsf{t}}}}
\newcommand{\svt}{t}
\newcommand{\svbt}{{\mathbf{t}}}
\newcommand{\rvu}{{\mathssf{u}}}    % u
\newcommand{\svu}{u}
\newcommand{\svuh}{{\hat{\svu}}}
\newcommand{\rvbu}{{\mathbsf{u}}}
\newcommand{\rvbU}{{\mathbsf{U}}}
\newcommand{\svbu}{{\mathbf{u}}}
\newcommand{\rvv}{{\mathssf{v}}}    % v
\newcommand{\svv}{v}
\newcommand{\svvh}{{\hat{\svv}}}
\newcommand{\rvbv}{{\mathbsf{v}}}
\newcommand{\rvbV}{{\mathbsf{V}}}
\newcommand{\svbv}{{\mathbf{v}}}
\newcommand{\rvvh}{{\hat{\rvv}}}
\newcommand{\rvw}{{\mathssf{w}}}    % w
\newcommand{\svw}{w}
\newcommand{\rvwh}{{\hat{\rvw}}}
\newcommand{\svwh}{{\hat{\svw}}}
\newcommand{\rvbw}{{\mathbsf{w}}}
\newcommand{\svbw}{{\mathbf{w}}}
\newcommand{\rvx}{{\mathssf{x}}}    % x, random variable
\newcommand{\rvxh}{{\hat{\rvx}}}
\newcommand{\rvxt}{{\tilde{\rvx}}}
\newcommand{\svx}{x}            % sample value
\newcommand{\svxh}{{\hat{\svx}}}
\newcommand{\svxt}{{\tilde{\svx}}}
\newcommand{\rvbx}{{\mathbsf{x}}}
\newcommand{\rvbxh}{{\hat{\rvbx}}}
\newcommand{\rvbxt}{{\tilde{\rvbx}}}
\newcommand{\svbx}{{\mathbf{\svx}}}
\newcommand{\svbxt}{{\tilde{\svbx}}}
\newcommand{\svbxh}{{\hat{\mathbf{x}}}}
\newcommand{\rvy}{{\mathssf{y}}}    % y
\newcommand{\rvyh}{{\hat{\mathssf{y}}}}
\newcommand{\svy}{y}
\newcommand{\rvyt}{{\tilde{\rvy}}}
\newcommand{\svyt}{{\tilde{\svy}}}
\newcommand{\svyh}{{\hat{\svy}}}
\newcommand{\rvby}{{\mathbsf{y}}}
\newcommand{\rvbyt}{{\tilde{\rvby}}}
\newcommand{\svby}{{\mathbf{y}}}
\newcommand{\svbyt}{{\tilde{\svby}}}
\newcommand{\rvz}{{\mathssf{z}}}    % z
\newcommand{\rvzh}{{\hat{\rvz}}}
\newcommand{\rvzt}{{\tilde{\rvz}}}
\newcommand{\svz}{z}
\newcommand{\svzh}{{\hat{\svz}}}
\newcommand{\rvbz}{{\mathbsf{z}}}
\newcommand{\svbz}{{\mathbf{z}}}

\newcommand{\rvB}{{\mathssf{B}}}
\newcommand{\rvJ}{{\mathssf{J}}}
\newcommand{\rvK}{{\mathssf{K}}}
\newcommand{\rvT}{{\mathssf{T}}}
\newcommand{\rvU}{{\mathssf{U}}}
\newcommand{\rvV}{{\mathssf{V}}}

% Handle uppercase Greek differently
\newcommand{\rvTh}{\ssfTheta}
\newcommand{\svTh}{\Theta}
\newcommand{\rvbTh}{\bsfTheta}
\newcommand{\svbTh}{\boldsymbol{\Theta}}
\newcommand{\rvPh}{\ssfPhi}
\newcommand{\svPh}{\Phi}
\newcommand{\rvbPh}{\bsfPhi}
\newcommand{\svbPh}{\boldsymbol{\Phi}}

\newcommand{\ddx}{\frac{\p}{\p \svx}}
\newcommand{\ddbx}{\frac{\p}{\p\svbx}}

%  --add new macros below this line--

% \newcommand{\iid}{\emph{i.i.d.}}
\newcommand{\iid}{i.i.d.}

\newcommand{\corolref}[1]{Corollary~\mbox{\ref{#1}}}
\newcommand{\thrmref}[1]{Theorem~\mbox{\ref{#1}}}
\newcommand{\lemref}[1]{Lemma~\mbox{\ref{#1}}}
\newcommand{\figref}[1]{Figure~\mbox{\ref{#1}}}
\newcommand{\secref}[1]{Section~\mbox{\ref{#1}}}
\newcommand{\chapref}[1]{Chapter~\mbox{\ref{#1}}}
\newcommand{\appref}[1]{Appendix~\mbox{\ref{#1}}}


%\definecolor{darkblack}{rgb}{0.0,0.0,0.0}
%\hypersetup{colorlinks,breaklinks,
%            linkcolor=darkblack,urlcolor=darkblack,
%            anchorcolor=darkblack,citecolor=darkblack}



\author{\IEEEauthorblockN{Louis Tan and Ashish Khisti} \IEEEauthorblockA{Dept.\ of Electrical \& Computer Engineering\\ University of Toronto\\
Toronto, ON M5S 3G4 Canada\\ \{ltan, akhisti\}@comm.utoronto.ca} \and \IEEEauthorblockN{Emina Soljanin} \IEEEauthorblockA{Bell Labs, Alcatel-Lucent\\ Murray Hill NJ 07974, USA\\
emina@alcatel-lucent.com}}

\newcommand{\bof}[1]{\textbf{#1}}
\title{Quadratic Gaussian Source Broadcast with Individual Bandwidth Mismatches}

\begin{document}

\maketitle
\thispagestyle{empty}
\pagestyle{empty} 

\begin{abstract}
We study the problem of broadcasting a Gaussian source over a Gaussian broadcast channel to two users with individual source-channel bandwidth mismatches, under a quadratic  distortion measure.  Specifically we study the tradeoff between the achievable distortion pairs between the two users. %as a function of the two bandwidth-expansion factors. % as well as the two signal-to-noise ratios. 
The case when the bandwidth-expansion factors of the two users are identical has been well studied in the literature and to our best knowledge remains an open problem. Surprisingly, when the bandwidth expansion factors are different, we characterize a range of values  where both the users simultaneously attain their point-to-point optimal distortion. Furthermore in the high signal-to-noise ratio regime, this set includes nearly all points where the weaker user has the higher bandwidth expansion factor. In other cases, we propose an achievable tradeoff between the distortion pairs.


\end{abstract}

%\setlength{\skip\footins}{0.25in}
\let\thefootnote\relax\footnotetext{This work was supported by an NSERC
(Natural Sciences Engineering Research Council) Discovery Grant from the
Government of Canada and an Ontario Graduate Scholarship (OGS).}

%\section{Introduction}  \label{sec:introduction}

\newcommand\inexpIntro[3]{#1?(#2,#3).}
\newcommand\rinexpIntro[3]{*#1?(#2,#3).}
\newcommand\outexpIntro[3]{#1!(#2,#3).}
\newcommand\outatomIntro[3]{#1!(#2,#3)}

We propose a fully automated method for proving termination of \(\pi\)-calculus processes.
Although there have been a lot of studies on termination analysis for the \(\pi\)-calculus
and related calculi~\cite{Deng06IC,Demangeon07,SangiorgiTermination,KobayashiHybrid,Yoshida04IC,DBLP:journals/jlp/DemangeonHS10,Venet98SAS}, most of them have been rather theoretical,
and there have been surprisingly little efforts in developing  fully automated termination
verification methods and tools based on them. To our knowledge,
Kobayashi's \typical{}~\cite{TyPiCal,KobayashiHybrid} is the only exception that
can prove termination of \(\pi\)-calculus processes (extended with natural numbers)
fully automatically, but its termination analysis is quite limited (see Section~\ref{sec:relatedwork}).

Our method is based on a reduction to termination analysis for sequential programs:
we translate a \(\pi\)-calculus process \(P\) to a sequential program \(S_P\), so that
if \(S_P\) is terminating, so is \(P\). The reduction allows us to use
powerful, mature methods and tools
for termination analysis of sequential programs~\cite{heizmann2016ultimate,freqterm,DBLP:conf/lics/PodelskiR04,Kuwahara2014Termination,DBLP:journals/cacm/CookPR11}.

The idea of the translation is to convert a chain of communications on replicated input
channels to a chain of recursive function calls of the target sequential program.
Let us consider the following Fibonacci process:
\begin{align*}
    & \rinexpIntro{\fib}{n}{r}
        \ifexp{n<2}{ \soutatom{r}{1} \\ &\quad}
                   { \nuexp{s_1} \nuexp{s_2} (\outatomIntro{\fib}{n-1}{s_1} \PAR \outatomIntro{\fib}{n-2}{s_2} \PAR \sinexp{s_1}{x}\sinexp{s_2}{y}\soutatom{r}{x+y}) \\}
    & \PAR \outatomIntro{\fib}{m}{r}
\end{align*}
Here, the process
$\rinexpIntro{\fib}{n}{r} \ldots$ is a function server that computes the \(n\)-th Fibonacci number
in parallel and returns the result to \(r\),
and $\outatom{\fib}{m}{r}$ sends a request for computing the \(m\)-th Fibonacci number;
those who are not familiar with the syntax of the \(\pi\)-calculus may wish to consult
Section~\ref{sec:targetlanguage} first.
To prove that the process above is terminating for any integer \(m\),
it suffices to show that there is no infinite chain of communications on $\fib$:
\[
    \fib(m,r) \to \fib(m_1,r_1) \to \fib(m_2,r_2) \to \cdots.
\]
We convert the process above to the following program:\footnote{The actual translation
  given later is a little more complex.}
\begin{verbatim}
 let rec fib(n) = if n<2 then () else (fib(n-1) [] fib(n-2)) in
 fib(m)
\end{verbatim}
Here, \texttt{[]} represents the non-deterministic choice.
Note that, although the calculation of Fibonacci numbers is not preserved,
for each chain of communications on \texttt{fib}, there is a corresponding
sequence of recursive calls:
\[
\mathtt{fib}(m) \to \mathtt{fib}(m_1) \to \mathtt{fib}(m_2) \to \cdots.
\]
Thus, the termination of the sequential program above implies the termination of
the original process.
As shown in the example above, (i) each communication on a replicated input channel
is converted to a function call, (ii) each communication on a non-replicated input
channel is just removed (or, in the actual translation, replaced by a call of
a trivial function defined by \(f(\seq{x})=(\,)\)), and (iii) parallel composition
is replaced by a non-deterministic choice.
We formalize the translation outlined above and prove its correctness.

The basic translation sketched above sometimes loses too much information.
For example, consider the following process:
\begin{align*}
    & \rinexpIntro{\pre}{n}{r} \soutatom{r}{n-1} \\
    & \PAR \rinexpIntro{f}{n}{r} \ifexp{n<0}{ \soutatom{r}{1} }
                                       { \nuexp{s} (\outatomIntro{\pre}{n}{s} \PAR \sinexp{s}{x}\outatomIntro{f}{x}{r}) } \\
    & \PAR \outatomIntro{f}{m}{r}
\end{align*}
The translation sketched above would yield:
\begin{verbatim}
  let pred(n) = n-1 in
  let rec f(n) = if n<0 then () else (pred(n) [] f(*)) in
  f(m)
\end{verbatim}
Here, \texttt{*} represents a non-deterministic integer: since we have removed
the input $\sinatom{s}{x}$, we do not have information about the value of \( x \).
As a result, the sequential program above is non-terminating, although the original
process is terminating.
To remedy this problem, we also refine the basic translation above by using a refinement
type system for the \(\pi\)-calculus. Using the refinement type system,
we can infer that the value of \(x\) in the original process is less than \(n\),
so that we can refine the definition of \texttt{f} to:
\begin{verbatim}
 let rec f(n) = ... else (pred(n) [] let x=* in assume(x<n);f(x))
\end{verbatim}
The target program is now terminating, from which
we can deduce that the original process is also terminating.
We have implemented an automated tool based on the refined translation above.

The contributions of this paper are summarized as follows.
\begin{itemize}
\item The formalization of the basic translation from the \(\pi\)-calculus
  (extended with integers) to sequential programs, and a proof of its correctness.
\item The formalization of a refined translation based on a refinement type system.
\item An implementation of the refined translation, including automated refinement type
  inference based on CHC solving, and experiments to evaluate the effectiveness of
  our method.
\end{itemize}

The rest of this paper is structured as follows.
Section~\ref{sec:targetlanguage} introduces the source and target languages
of our translation.
Section~\ref{sec:approach} 
formalizes the basic translation, and proves its correctness.
Section~\ref{sec:refinement} refines the basic translation by using a refinement type system.
Section~\ref{sec:implementation} reports an implementation and experiments.
Section~\ref{sec:relatedwork} discusses related work,
and Section~\ref{sec:conclusion} concludes the paper.

%\section{General Framework}
\label{sec:framework}
Our goal in this work is to demonstrate the utility of natural language descriptions in assisting policy transfer across domains. In this section, we first describe our environment setup and the general framework of our approach. The details of our model and algorithm follow in Section~\ref{sec:model}.

\subsection{Environment Setup} 
We model a single environment as a Markov Decision Process (MDP),  $E = \langle S, A, T, R, O, Z \rangle$. Here, $S$ is the state space, and $A$ is the set of actions available to the agent. In this work, we consider every state $s \in S$ to be a 2-dimensional grid of size $m \times n$, with each cell containing an entity symbol $o \in O$.\footnote{In our experiments, we relax this assumption to allow for multiple entities per cell, but for ease of description, we shall assume a single entity per cell. The assumption of 2-D worlds can also be easily relaxed to generalize our model to other situations.} $T$ is the transition distribution over all possible next states $s'$ conditioned on the agent choosing action $a$ in state $s$. $R$ determines the reward provided to the agent at each time step. The agent does not have access to the true $T$ and $R$ of the environment. Each domain also has a goal state $s_g \in S$ which determines when an episode terminates. Finally, $Z$ is the complete set of text descriptions provided to the agent for this particular environment. 

\subsection{Reinforcement Learning (RL)}
The goal of an autonomous agent is to maximize cumulative reward obtained from the environment. A traditional way to achieve this is by learning an action value function $Q(s,a)$ through reinforcement. The \emph{Q-function} predicts the expected future reward for choosing action~$a$ in state~$s$. A straightforward policy then is to simply choose the action that maximizes the $Q$-value in the current state: 

\begin{dmath*}
\pi(s) = \argmax_a Q(s,a)
\end{dmath*}

If we also make use of the descriptions, we have a text-conditioned policy: 
\begin{dmath}
\pi(s, Z) = \argmax_a Q(s, a, Z)
\end{dmath} 

A successful control policy for an environment will contain both knowledge of  the environment dynamics and the capability to identify goal states. While the latter is task-specific, the former characteristic is more useful for learning a general policy that transfers to different domains. Based on this hypothesis, we employ a model-aware RL approach that can learn the dynamics of the world while estimating the optimal $Q$. Specifically, we make use of \emph{Value Iteration (VI)}~\cite{sutton1998introduction}, an algorithm based on dynamic programming. The update equations for value iteration in our setup are:
\begin{align}
Q^{(n+1)}(s, a, Z) &= \sum_{s' \in S} T(s' | s, a, Z) [ R(s', Z) + \gamma V^{(n)}(s', Z) ]  \nonumber \\
V^{(n+1)}(s, Z) &= \max_a Q^{(n+1)}(s,a, Z) 
\label{eq:vi}
\end{align}
where $\gamma$ is a discount factor and $n$ is the iteration number. The updates require an estimate of $T$ and $R$, which the agent must obtain through exploration of the environment.

% Note that this assumes the agent has knowledge of the true $T$ and $R$ in order to estimate $Q$ and $V$. Since our setup does not provide this information, the agent has to estimate the transition and reward functions from its interactions with the world.

\subsection{Text Descriptions}
Estimating the dynamics of the environment from interactive experience can require a significant number of samples. Our main hypothesis is that if an agent can derive information about the dynamics from text descriptions, it can determine $T$ and $R$ faster and more accurately. 
% Hence, we work with text that provides such relevant particulars of the domain.

For instance, consider the sentence \emph{``Red bat that moves horizontally, left to right''}. This talks about the movement of a third-party entity (\emph{bat}), independent of the agent's goal. Provided the agent can learn to interpret this sentence, it can then infer the direction of movement of a different entity (e.g. \emph{``A tan car moving slowly to the left''}) in a different domain. Further, this inference is useful even if the agent has a completely different goal. On the other hand, instruction-like text such as \emph{``Move towards the wooden door''} is highly context-specific and only relevant to domains that have the mentioned goal.

With this in mind, we provide the agent with text descriptions that collectively portray characteristics of the world. These descriptions are crowdsourced by asking humans to view gameplay videos and describe entities.  A single description talks about one particular entity in the world. The text contains (partial) information about the entity's movement and interaction with the player avatar. Each description is also aligned to its corresponding entity in the environment and not all entities may have a description.
% We make sure that a simple mapping cannot be found between entities in different domains using just their names in text.\todo{clarify this}
Figure~\ref{fig:descriptions} provides some samples; more details on data collection and statistics are in Section~\ref{sec:experiments}. 

\begin{figure}
  \begin{annotationbox}
%     \centering
    \small
      \begin{itemize}
        \item Scorpion2: \emph{Red scorpion that moves up and down} 
        \item Alien3: \emph{This character slowly moves from right to left while having the ability to shoot upwards}
        \item Sword1: \emph{This item is picked up and used by the player for attacking enemies}
      \end{itemize}
%     }
  \end{annotationbox}
  \caption{Example text descriptions of entities in different environments, collected using Amazon Mechanical Turk. Turkers were shown videos of gameplay in the different environments and asked to describe each entity's behavior or role. Note that these sentences are not instructive, since they provide no direct information on how to act in the environment.}
  \label{fig:descriptions}
\end{figure}

\subsection{Transfer for RL}
In order to test our grounding hypothesis, we consider learning across multiple environments. Specifically, an agent can learn to ground language semantics in an environment $E_1$ and then we can test its understanding capability by placing it in a new unseen domain, $E_2$. The agent can obtain unlimited experience in $E_1$, and after convergence of its policy, it is allowed to interact with and learn a policy for $E_2$. We do not provide the agent with any explicit mapping between different entities or goals across domains, either directly or through the text. For instance, even though the boulders in \emph{Boulderchase} are impassable objects just like the walls in \emph{Bomberman}~\ref{fig:example}, the agent does not have access to a mapping between these entities. In this setup, the agent's goal is to re-utilize information obtained through its interactions in $E_1$ to learn more efficiently in $E_2$.


% \paragraph{Environment Setup} 
% We model a single environment as a Markov Decision Process (MDP), represented by $E = \langle S, A, T, R, O, Z \rangle$. Here, $S$ is the state space, and $A$ is the set of actions available to the agent. In this work, we consider every state $s \in S$ to be a 2-dimensional grid of size $m \times n$, with each cell containing an entity symbol $o \in O$.\footnote{In our experiments, we relax this assumption to allow for multiple entities per cell, but for ease of description, we shall assume a single entity. The assumption of 2-D worlds can also be easily relaxed to generalize our model to other situations.} $T$ is the transition distribution over all possible next states $s'$ conditioned on the agent choosing action $a$ in state $s$. $R$ determines the reward provided to the agent at each time step. The agent does not have access to the true $T$ and $R$ of the environment. Each domain also has a goal state $s_g \in S$ which determines when an episode terminates. Finally, $Z$ is the complete set of text descriptions provided to the agent for this particular environment. 

% \paragraph{Reinforcement learning (RL)}
% The goal of an autonomous agent is to maximize cumulative reward obtained from the environment. A traditional way to achieve this is by learning an action value function $Q(s,a)$ through reinforcement. The \emph{Q-function} predicts the expected future reward for choosing action~$a$ in state~$s$. A straightforward policy then is to simply choose the action that maximizes the $Q$-value in the current state: $\pi(s) = \argmax_a Q(s,a)$. If we also make use of the descriptions, we have a text-conditioned policy: $\pi(s, Z) = \argmax_a Q(s, a, Z)$. 

% A successful control policy for an environment will contain both knowledge of  the environment dynamics and the capability to identify goal states. While the latter is task-specific, the former characteristic is more useful for learning a general policy that transfers to different domains. Based on this hypothesis, we employ a model-aware RL approach that can learn the dynamics of the world while estimating the optimal $Q$. Specifically, we make use of \emph{Value Iteration (VI)}~\cite{sutton1998introduction}, an algorithm based on dynamic programming. The update equations are as follows:
% \begin{align}
% Q^{(n+1)}&(s, a, Z) = R(s, a, Z)  \nonumber \\ 
% &+ \gamma \sum_{s' \in S} T(s' | s, a, Z) V^{(n)}(s', Z)  \nonumber \\
% V^{(n+1)}&(s, Z) = \max_a Q^{(n+1)}(s,a, Z) 
% \label{eq:vi}
% \end{align}
% where $\gamma$ is a discount factor and $n$ is the iteration number. The updates require an estimate of $T$ and $R$, which the agent must obtain through exploration of the environment.

% % Note that this assumes the agent has knowledge of the true $T$ and $R$ in order to estimate $Q$ and $V$. Since our setup does not provide this information, the agent has to estimate the transition and reward functions from its interactions with the world.

% \paragraph{Text descriptions}
% Estimating the dynamics of the environment from interactive experience can require a significant number of samples. Our main hypothesis is that if an agent can derive information about the dynamics from text descriptions, it can determine $T$ and $R$ faster and more accurately. 
% % Hence, we work with text that provides such relevant particulars of the domain.

% For instance, consider the sentence \emph{``Red bat that moves horizontally, left to right.''}. This talks about the movement of a third-party entity ('bat'), independent of the agent's goal. Provided the agent can learn to interpret this sentence, it can then infer the direction of movement of a different entity (e.g. \emph{``A tan car moving slowly to the left''} in a different domain. Further, this inference is useful even if the agent has a completely different goal. On the other hand, instruction-like text, such as \emph{``Move towards the wooden door''}, is highly context-specific, only relevant to domains that have the mentioned goal.

% With this in mind, we provide the agent with text descriptions that collectively portray characteristics of the world. A single description talks about one particular entity in the world. The text contains (partial) information about the entity's movement and interaction with the player avatar. Each description is also aligned to its corresponding entity in the environment. 
% % We make sure that a simple mapping cannot be found between entities in different domains using just their names in text.\todo{clarify this}
% Figure~\ref{fig:descriptions} provides some samples; details on data collection and statistics are in Section~\ref{sec:experiments}.

% \begin{figure}
%   \begin{annotationbox}
% %     \centering
%     \small
%       \begin{itemize}[leftmargin=0.45cm]
%         \item Scorpion2: \emph{Red scorpion that moves up and down} 
%         \item Alien3: \emph{This character slowly moves from right to left while having the ability to shoot upwards}
%         \item Sword1: \emph{This item is picked up and used by the player for attacking enemies}
%       \end{itemize}
% %     }
%   \end{annotationbox}
%   \caption{Some example text descriptions of entities in different environments.}
%   \label{fig:descriptions}
% \end{figure}

% % \begin{table}[h]
% % \centering
% % \resizebox{\linewidth}{!}{%
% % \begin{tabular}{  l  } \toprule
% % \textit{Red scorpion that moves up and down} \\
% % \textit{This character slowly moves left to right} \\ 
% % \textit{and has the ability to shoot to the left which can kill the player} \\
% % \textit{this item is picked up and used by the player for attacking enemies}
% % \textit{Ghost1 moves horizontally and is an enemy} \\
% % \textit{Alien3 is an enemy bomber shooting upwards} \\
% % \bottomrule
% % \end{tabular}
% % }
% % \caption{Some example text descriptions of various entities for a game environment.}
% % \label{table:descriptions}
% % \end{table}

% \paragraph{Transfer for RL}
% A natural scenario to test our grounding hypothesis is to consider learning across multiple environments. The agent can learn to ground language semantics in an environment $E_1$ and then we can test its understanding capability by placing it in a new unseen domain, $E_2$. The agent is allowed unlimited experience in $E_1$, and after convergence of its policy, it is then allowed to interact with and learn a policy for $E_2$. We do not provide the agent with any mapping between entities or goals across domains, either directly or through the text. The agent's goal is to re-utilize information obtained in $E_1$ to learn more efficiently in $E_2$.

% % For example, a `bat' in $E_1$ is not given the same symbol as a `bat' in $E_2$.\footnote{Also note that the behavior of a bat in the two environments can be substantially different.} The aim of transfer is to re-utilize information obtained in $E_1$ to learn efficiently in $E_2$.


% % \paragraph{Environment}
% % In our setup, an environment consists of a state space $\mathcal{S}$ and a set of text descriptions $\mathcal{Z} = \{z_i\}$. The state is a $m \times n$ grid world containing entities drawn from a set $\mathcal{O}$ (with unique IDs). Given an input state $s \in \mathcal{S}$, the agent can take a discrete action $a \in \mathcal{A}$, and observe a new state $s'$ of the environment, which changes according to a transition distribution $\mathcal{T}(s' | s,a)$. The environment also provides the agent with a reward $\mathcal{R}(s,a)$ at every time step. Note that the agent does not have access to the true $\mathcal{T}$ and $\mathcal{R}$ of its environment. 

% % Each description $z_i$ is a sentence that provides information about one particular entity type such as its movements or interactions with other entities. We assume access to the mapping between each $z_i$ and its corresponding object $o_i$.

% % \paragraph{Reinforcement Learning (RL)}
% % In the RL framework, the goal of an autonomous agent is to perform actions that maximize the cumulative reward it obtains from the environment. This is done by learning an action value function $Q(s,a)$, which predicts the expected future reward of choosing action~$a$ in state~$s$. Using this, a straightforward policy is to simply choose the action that maximizes the $Q$-value in the current state: $\pi(s) = \argmax_a Q(s,a)$.
% % % \todo{define Q and policy}

% % \paragraph{Transfer setup}
% % We are given a source environment $e_u$ and a target environment $e_v$. In addition, we have access to corresponding sets of text descriptions $\mathcal{Z}_u$ and $\mathcal{Z}_v$, respectively. We first estimate parameters $\Theta$ of a policy $\pi_u(s, \mathcal{Z}_u)$ through several interactions with $e_u$. The policy is optimized to obtain maximum possible reward on the source environment. Now, using $\pi_u$, our goal is to learn an optimal policy for the target environment $e_v$ in as few interactions as possible, by transferring knowledge obtained in $e_u$.


% % % Formally, let us consider an environment $e \in \mathcal{E}$, consisting of entities $\mathcal{O}^e = \{o^e_i\}$ and correspondingly aligned text descriptions $\mathcal{Z}^e = \{z^e_i\}$. Our goal is to learn a mapping from these descriptions to the control dynamics, while simultaneously learning an optimal policy. In this work, we consider two-dimensional state spaces, but the main facets of our model can be extended to other scenarios.
%\input{achievable}

%%%%%%%%%%%%%%%%%%%
\begin{figure}
%\begin{figure*}
%	\input{systematic_encoder}
%	\input{systematic_decoder}
%	\input{line_nT}
%	\input{regions}
%\documentclass[11pt,draftcls]{IEEEtran}{\onecolumn}
% \usepackage[driver]{graphicx}
\usepackage{epsfig}
\usepackage{amsmath,epsfig,dsfont}
\usepackage{subfigure}
\usepackage{algorithm}
\usepackage{algorithmic}
\usepackage{amssymb,bm}
\usepackage{amsfonts}
\usepackage{stfloats}
\usepackage{multirow}
\usepackage{cite}
%\usepackage{pdffig}
%\usepackage[pdftex]{graphicx}

% *** GRAPHICS RELATED PACKAGES ***
%
\ifCLASSINFOpdf
\usepackage{graphicx}
  % declare the path(s) where your graphic files are
  % \graphicspath{{../pdf/}{../jpeg/}}
  % and their extensions so you won't have to specify these with
  % every instance of \includegraphics
  % \DeclareGraphicsExtensions{.pdf,.jpeg,.png}
\else
  % or other class option (dvipsone, dvipdf, if not using dvips). graphicx
\fi
% graphicx was written by David Carlisle and Sebastian Rahtz. It is

% correct bad hyphenation here
\hyphenation{op-tical net-works semi-conduc-tor}


\begin{document}
%
% paper title
% can use linebreaks \\ within to get better formatting as desired
\title{Mobile Data Transactions in Device-to-Device Communication Networks: Pricing and Auction}

\author{Jingjing~Wang,~\IEEEmembership{Student Member,~IEEE,}
        Chunxiao~Jiang,~\IEEEmembership{Senior Member,~IEEE,}
        Zhi~Bie,\\
        Tony~Q.~S.~Quek,~\IEEEmembership{Senior Member,~IEEE,}
        and~Yong~Ren,~\IEEEmembership{Member,~IEEE}
\thanks{J. Wang, C. Jiang, Z. Bie and Y. Ren are with the Department
of Electronic  Engineering, Tsinghua University, Beijing 100084,
P. R. China. E-mail: chinaeephd@gmail.com, \{jchx, reny\}@tsinghua.edu.cn, bz12@mails.tsinghua.edu.cn.}
\thanks{T. Q. S. Quek is with the Singapore University of Technology and Design, 8 Somapah Road, Singapore 487372. Email: tonyquek@sutd.edu.sg.}
}
\maketitle


\begin{abstract}
Device-to-Device (D2D) communication is offering smart phone users a choice to share files with each other without communicating with the cellular network. In this paper, we discuss the behaviors of two characters in the D2D data transaction model from an economic point of view: the data buyers who wish to buy a certain quantity of data, and the data sellers who wish to sell data through D2D network. The optimal price setting and purchasing strategies are desirable, and we give the analysis based on game theory.
\end{abstract}

\begin{IEEEkeywords}
D2D, game theory, multimedia.
\end{IEEEkeywords}

% IEEEtran.cls defaults to using nonbold math in the Abstract.
\IEEEpeerreviewmaketitle

\section{Introduction}
Device-to-Device (D2D) communication has been proposed as a revolutionary paradigm to enhance the capacity of cellular networks~\cite{1}.
According to D2D theory, a cell area under the control of one base station is divided into several clusters, and mobile users in the same cluster are close enough to establish direct connections. This clustering concept can remarkably improve the performance of cellular networks, and  several aspects of the benefits have been investigated~\cite{2}, including higher spectral efficiency, enhanced total throughput, higher energy efficiency and shorter delay.

Since multimedia files, like wireless videos, have been the main driver for the inexorable increase in data transmission, experts have proposed many methods to deal with the problem. In retrospect, researchers proposed methods like: decreasing the cell size to improve the spectral efficiency, using additional spectrum and improving the physical-layer link capacity, but all these traditional methods are not satisfactory and may even bring other problems. Later, it was observed that mobile devices have large storage space. Following this idea, a novel architecture based on D2D communication was thoroughly analyzed in~\cite{11}.

In this paper, we explore the mobile users' behaviors when they buy or sell videos within a D2D cluster in terms of an economic point of view. We model the file distribution as a competition mechanism against the data service of base station, and thus the autonomous D2D mechanism is preferred, where the data transaction is managed by D2D users without manipulation of base station. Furthermore, in order to analyze the optimal selling and purchasing strategies of the users, we use game theory to model different situations. Specifically, we model the one buyer multiple sellers case as a Stackelburg game, and model the one seller multiple buyers case as an auction game. The ultimate goal is to maximize the utility of both sides simultaneously.


%\hfill what is this?
%\hfill September 20, 2014

% \subsection{Subsection Heading Here}
% \subsubsection{Subsubsection Heading Here}

\section{System model}

In the formulation of the channel condition, we assume that in a certain cluster $C_{i}$, a data buyer $B_{i}$ purchases files from a data seller $S_{i}$ with total transmission power $G_{i}$. Moreover, the distance between $B_{i}$ and $S_{i}$ is represented by $d_{i}$, and $D_{ij}$ denotes the distance between two neighbor D2D clusters' centers. Also, we assume that the channel gain is $H_{i}$, as well as $\sigma^2$ represents the additive white Gaussian noise (AWGN) power. Considering $M$ neighbor D2D clusters in a cellular network, the cluster interference power of neighbor cluster $C_{j}$ is denoted as $G_{I,j}$ with the corresponding channel gain $H_{I,j}$, $j=1,\ldots,M$. For $C_{i}$, the Signal to Interference plus Noise Ratio (SINR) is given by:
%\begin{equation}\label{SNRi}
%{\textrm{SNR}}=\frac{GH}{\sqrt{d}\sigma^2}.
%\end{equation}
\begin{equation}\label{SNRi}
\textrm{SINR}=\frac{{{G}_{i}}H_{i}/\sqrt{{{d}_{i}}}}{{{\sigma }^{2}}+\sum\limits_{j=1}^{M}{({G_{I,j}}H_{I,j}/\sqrt{{{D}_{ij}}}})}.
\end{equation}
The total channel bandwidth available for data transactions can be deemed as $W$. Then, relying on Shannon formula, the maximal achievable bit rate between $B_{i}$ and $S_{i}$ follows:
%\begin{equation}\label{Ri}
%R=W\log_2\left(
%1+\frac{GH}{\sqrt{d}\sigma^2}
%\right).
%\end{equation}
\begin{equation}\label{Ri}
{{R}_{i}}=W{{\log }_{2}}\left(1+\frac{{{G}_{i}}H_{i}/\sqrt{{{d}_{i}}}}{{{\sigma }^{2}}+\sum\limits_{j=1}^{M}{({G_{I,j}}H_{I,j}/\sqrt{{{D}_{ij}}})}}\right).
\end{equation}

The total nodal delay is calculated as the sum of processing, queuing, transmission and propagation time. Relying on the delay model in~\cite{103}, which approximatively maps the network architecture into bi-directional graphs, the system delay can be denoted as:
\begin{equation}\label{D}
D(Y,V)= \beta(O) \frac{\sqrt{Y+V}}{\sqrt{\log{(Y+V)}}}\textrm{ ms},
\end{equation}
where the $Y$ represents the number of users (buyers and sellers) in a data transaction model, while $V$ denotes the number of other users in the same cellular network who are simultaneously transmitting data. Furthermore, $\beta(O)$ is a function of the size of transaction data $O$.



\section{Transaction Model}

In this section, we will discuss the data transaction problem in two different situations. The first situation consists of one data buyer purchasing from several data sellers, where the buyer can purchase different coding layers of a video from different sellers and combine these streams during the decoding process~\cite{200}. The second model consists of one seller selling data to multiple buyers. The major difference between the two models is whether the buyer or the seller takes the first move in the game.

\subsection{Initiating with the Buyer}

In this subsection, we consider the transaction model of one data buyer $B$ purchasing multimedia files from $N$ data sellers, namely, $\{S_1,\ldots,S_N\}$ in D2D cluster $C_{i}$. Then, the SINR and the bit rate between $B$ and $S_n$ can be calculated by Eq.~(\ref{SNRi}) and Eq.~(\ref{Ri}). Given the the total available bandwidth $W$, which will be evenly allocated to all users. The maximal achievable bit rate of the buyer is given by:
\begin{equation}
R_B=\frac{W}{N+1}\sum_{n=1}^N\log_2\left(1+\frac{{{G}_{n}}H_{n}/\sqrt{{{d}_{n}}}}{{{\sigma }^{2}}+\sum\limits_{j=1}^{M}{({G_{I,j}}H_{I,j}/\sqrt{{{D}_{ij}}})}}\right).
\end{equation}

Due to transmission delay, the strategies of the buyer and the sellers would be announced sequentially, which is beneficial of constructing a Stackelburg game model~\cite{simaan1976stackelberg}, where a data buyer takes the first step to send a detect signal, in order to inform other mobile users in the same cluster what particular video file he/she wants. Next, the potential sellers send back their prices along with the channel conditions. Then, the data buyer decides how much the transmission power he/she intends to buy from different sellers, respectively.

A data buyer can choose to buy from part of the sellers or all the sellers, and even he/she is capable of selecting service providers, if that way can lead to a higher reward.

\subsubsection{Caching Conditions}

Whether a seller has stored a copy of the file or not can affect his/her reward considerably, because this determines whether the seller needs to download the file from service provider before transmitting it to the buyer.

In~\cite{17}, Zipf distribution has been verified as a good model to measure the popularity of a set of video files. Thus, we assume that there are totally $K$ popular videos in this region within a month, and the popularity of the videos is subject to Zipf distribution with the parameter $\eta$. Then, the required frequency of the $i^{th}$ popular video file can be denoted as:
\begin{equation}
{{\lambda }_{i}}=\frac{1/{{i}^{\eta }}}{\sum\nolimits_{k=1}^{K}{1/{{k}^{\eta }}}}, 1\le i\le K.
\end{equation}
Meanwhile, we assume that the caching of the files are uniformly distributed among the users, which means that for every internal storage unit of a seller's device, the probability that it is used to store one of the $K$ files is $1/K$.

Given that a seller can store $\tau$ different popular video files of this month in his/her internal storage, there are $C_{K}^{\tau }$ possible different storage modes. Furthermore, we denote one of the possible storage mode of the seller as $\Gamma=\{\Gamma_1,\ldots,\Gamma_\tau\}$, where $\Gamma_i$, $i=1,\ldots,\tau$, represents the rank order among $K$ popular video files. Thus, the possibility of one of $C_{K}^{\tau }$ storage modes can be given by:
\begin{equation}
\textrm{Pr}(\Gamma )=\frac{\tau !(K-\tau )!}{K!},
\end{equation}
and the probability of this seller having a cached copy of the wanted file in terms of the $\Gamma$ storage mode follows:
\begin{equation}
\textrm{Pr}(\textrm{wanted}|\Gamma)= \sum_{i=1}^\tau \lambda_{\Gamma_i}.
\end{equation}
Relying on the total probability formula, the probability that a seller has a cached copy of the wanted file can be constructed as:
\begin{equation}\label{PrCache}
\textrm{Pr}(\textrm{Cache})=\sum_{\Gamma} \left[\textrm{Pr}(\textrm{wanted}|\Gamma) \cdot \textrm{Pr}(\Gamma ) \right].
\end{equation}

\subsubsection{Reward Functions}

The data buyer $B$ in cluster $C_{i}$ gains reward by successfully receiving the file with a good quality measured by the maximal achievable transmission rate. Specifically, $B$ has to pay the price $p_n (1 \leq n \leq N)$ for a unit of transmission power of $N$ sellers, respectively. Therefore, the data buyer's reward function can be formulated as:
\begin{equation}\label{PhiB-multiple}
\Phi_B\!=\!\frac{W}{N+1}\sum_{n=1}^N \log_2\left(1\!+\!\frac{{{G}_{n}}H_{n}/\sqrt{{{d}_{n}}}}{{{\sigma }^{2}}+\sum\limits_{j=1}^{M}{({G_{I,j}}H_{I,j}/\sqrt{{{D}_{ij}}})}}\right)\cdot \xi_R-D(Y,V)\cdot\xi_D
-\sum_{n=1}^N p_n\cdot G_n,
\end{equation}
where the coefficient $\xi_R$ represents the unit reward in terms of the maximal achievable bit rate, as well as $\xi_D$ denotes the unit loss measured by the system transmission delayed, which can be deemed as normalizing weight parameters for both $R_B$ and $D(Y,V)$. $G_n$ represents the traded transmission power that the seller $S_n$ transmits to $B$. The system delay $D(Y,V)$ is defined in Eq.~(\ref{D}).

In the following, we formulate the sellers' reward function. We introduce parameters $c_n$, $n=1,\ldots,N$, to represent the unit cost for relaying data between $B$ and $S_n$, which is determined by the characteristics of the sellers' devices. Thus, the utility function of $S_n$ can be defined as:
\begin{equation}\label{PhiS}
\begin{aligned}
\Phi_{S_n}=~&\textrm{Pr}(\textrm{Cache})\cdot(p_n-c_n)\cdot G_n+ \\
&\left[1-\textrm{Pr}(\textrm{Cache})\right]\cdot(p_n-c_n-s)\cdot G_n,
\end{aligned}
\end{equation}
where $s$ is the unit cost of downloading files from the nearest base station.
Additionally, we has a maximum cellular users constraint, i.e., $Y+V \leq \Omega$.
\subsubsection{Optimal Strategies}

In a Stackelburg game, the optimal transaction strategies for both sides exist and can be obtained using backward induction. The last move is for the data buyer to determine the optimal transmission power $G_n$ acquired from seller $S_n$ to maximize his/her reward $\Phi_B$. Moreover, we assume that the possible power of a received video is continuous~\cite{15}.

Hence, we can obtain $G_n$ by letting the first-order derivative of $\Phi_B$ with respect to $G_n$ be zero, which leads to
\begin{equation}\label{Qi}
{{G}_{n}}=\frac{W\xi_{R}}{(N+1){{p}_{n}}\ln 2}-\frac{\sqrt{{{d}_{n}}}\left[{{\sigma }^{2}}+\sum\limits_{j=1}^{M}\left({{G_{I,j}}H_{I,j}/\sqrt{{{D}_{ij}}}}\right)\right]}{H_{n}}.
\end{equation}

As the second last move, the sellers need to decide the optimal price $p_n$ in terms of the transmission power $G_n$ purchased by the buyer. We substitute $G_n$ in Eq.~(\ref{PhiS}) with Eq.~(\ref{Qi}), and let the first-order derivative with respect to $p_n$ equal zero, we have the optimal price of the data seller $S_{n}$:
%\begin{equation}
%\frac{W{{\xi }_{R}}[{{c}_{n}}+s(1-\Pr (\textrm{Cache}))]}{(N+1)p_{n}^{2}ln2}\!=\!-\!\frac{\sqrt{{{d}_{n}}}[{{\sigma }^{2}}\!+\!\sum\limits_{j=1}^{M}({{G_{I,j}}H_{I,j}/\sqrt{{{D}_{ij}}}})]}{H_{n}}.
%\end{equation}
%Therefore, the optimal price of the data seller $S_{n}$ should be set as:
\begin{equation}\label{pi}
p_n=\sqrt{\frac{WH_{n}\xi_R[c_n+s(1-\Pr(\textrm{Cache}))]}{(N+1)\sqrt{d_n}\left[\sigma^2+ \sum\limits_{j=1}^{M}\left({{G_{I,j}}H_{I,j}/\sqrt{{{D}_{ij}}}}\right)\right]\ln{2}}}.
\end{equation}
Substituting $p_n$ in Eq.~(\ref{Qi}) with Eq.~(\ref{pi}), we can achieve the optimal transmission power $G_n$.

\subsection{Initiating with the Seller}

In this situation, if a seller wants to sell data to nearby users, he/she takes the first step to send a probe signal to announce the availability of stored data resources. The seller would prefer to maximize his/her reward by transmitting popular video files in terms of a fierce competition among the potential multiple data buyers.

We model this decentralized process as alternative ascending clock auction (ACA-A) game, because it guarantees an efficient and cheat-proof allocation~\cite{chen2010spectrum}. In our model, there are $N$ buyers, namely $B_n$, $n=1,\ldots,N$, competing for high quality data from one seller $S$.

Similarly, the SINR and maximal achievable bit rate are given by Eq.~(\ref{SNRi}) and Eq.~(\ref{Ri}). Then, the reward function of the buyers at clock step $t$ can be deemed as:
\begin{equation}
\Phi_{B_n}^t(p^0,G_n)=R_{B_n}^{t}\cdot \xi_R-D(Y,V)^{t}\cdot\xi_D-p^{t}\cdot G_{n}^{t},
\end{equation}
where, $p^{t}$ represents the price of unit transmission power charged by $S$.
Suppose that the seller wants to sell total power of $G$. In the beginning of the auction, $S$ sets up the clock index $t=0$, regulates time-step size $\delta >0$, and announces the initial price $p^0$ to all the D2D users in the cluster $C_{i}$. Then, each potential buyer $B_n$ offers his/her optimal bid $G_n^0$ by computing
\begin{equation}
G_n^0=\arg \max_{G_n} \Phi_{B_n}\left(p^0,G_n\right).
\end{equation}
Then, the seller sums up all the bids $G_{total}^0=\sum_n {G_n^0}$, and compares $G_{total}^0$ with $G$. If $G_{total}^0 \leq G$, the seller concludes the auction. Otherwise, the seller sets $p^{t+1}=p^t+\delta$, $t=t+1$ and announces the price $p^t$ to all the buyers who bid for the next time. Thus, each buyer offers his/her optimal bid $G_n^{t+1}$ by calculating
\begin{equation}
G_n^{t+1}=\arg \max_{G_n} \Phi_{B_n}\left(p^{t+1},G_n\right).
\end{equation}

After collecting all the bids, the seller computes the total bid $G_{total}^{t+1}$. The seller continues the auction until $G_{total}^{t+1} \leq G$. At every clock $t$, the seller computes the cumulative clinch $\chi_n^t$, which is the amount of data that the buyer $B_{n}$ is guaranteed to win at clock $t$:
\begin{equation}
\chi_n^t=\max \left(0,G-\sum_{j \neq n}{G_j^t}\right),
\end{equation}
and the buyer will purchase this $\chi_n^t$ quantity of power with price $p^t$. This has been proved to be a cheat-proof scheme.

Let the final clock index be $T$. As the price $p$ increases discretely, we may have $G_{total}^{T} < G$, in which case the power resource is not fully utilized. We adopt a proportional rationing scheme in order to make sure that $G_{total}^{T} = G$ ~\cite{ausubel2004efficient}. Therefore, the final allocated transmission power can be denoted by:
\begin{equation}\label{Qi_final}
G_{n,final}=G_n^{T}+\frac{G_n^{T-1}-G_n^{T}}{\sum\limits_{i=1}^{N}{G_i^{T-1}}-\sum\limits_{i=1}^{N}{G_i^{T}}} \left(G-\sum\limits_{i=1}^{N}{G_i^{T}}\right),
\end{equation}
which satisfy $\sum_n{G_{n,final}}=G$.

We set $\chi_n^{T}=G_{n,final}$, and the total payment of buyer $B_n$ is given by:
\begin{equation}
P_n=\chi_n^0p^0+\sum_{t=1}^{T} p^t\left(\chi_n^t-\chi_n^{t-1}\right).
\end{equation}

In this model, the reward of the seller can be formulated as:
\begin{equation}
\Phi_{S}=\sum_{n=1}^N{\left(P_n-c_n G_{n,final}\right)},
\end{equation}
where $c_n$ accounts for the unit transmission cost.

\section{Simulation Results}

In this section, we will show the variation tendency of the sellers' and buyers' reward functions and their optimal choices in different models.

\subsection{Initiating with the Buyer}

In our simulation, we set the data buyer at the origin (0,0) and let other mobile users uniformly distribute in a cell. The radius of a cell is $500$m, as well as the radius of a D2D cluster is set as $100$m. Potential sellers, i.e., $N=10$, are uniformly distributed in a D2D cluster. Besides, the number of clusters, which are randomly scattered in a cell, is $M=10$. We set the noise power spectrum density to be $-174$dBm, and the total available bandwidth for D2D data transaction is $5$MHz under an ideal channel gain $H=1$. Moreover, for all users, the maximal transmit power is $100$mW, i.e., $G_{n}\leq 100$mW, $G_{I,j}\leq 100$mW, and the normalized weighting parameters $\xi_R=3\times10^{-5}$/bps as well as $\xi_D=10^{-1}$/ms in terms of $\beta(O)=20$ under the assumption of the same packet size $O=1$MB. The data seller's unit transmission power cost $c$ is uniformly distributed in $[0.1, 0.5]$/mW. Finally, we set the probability that the seller has stored a copy of the wanted file to be $\textrm{Pr}(\textrm{Cache})=0.3, 0.4, 0.5$, respectively.


\begin{figure}
\begin{center}
\includegraphics[width=0.6\textwidth]{1.eps}
\end{center}
\caption{Variation tendency of the reward functions for the buyer and sellers versus the increasing sellers' number.}\label{MultipleU}
\end{figure}

\begin{figure}
\begin{center}
\includegraphics[width=0.6\textwidth]{2.eps}
\end{center}
\caption{Variation tendency of the optimal choices of the buyer and sellers versus the increasing sellers' number.}\label{MultipleE}
\end{figure}

\begin{figure}
\begin{center}
\includegraphics[width=0.6\textwidth]{3.eps}
\end{center}
\caption{Variation tendency of the seller's reward function and the total rounds of the auction with respect to the number of buyers.}\label{auction1}
\end{figure}
From Fig.~\ref{MultipleU} and Fig.~\ref{MultipleE}, we can conclude that as the number of potential sellers increases, the average price of sellers decreases, while the total traded transmission power improves. Furthermore, the data buyer's reward increases at the beginning, and then keeps constant. That is because when there are more sellers, they may compete with each other and lower the price. However, when the buyer purchase from superfluous buyers, system delay start to play a critical role in reducing the rewards. Also, with the improving of the probability of having a wanted cached video copy, the sellers tend to lower their prices, which contributes to higher rewards.
\subsection{Initiating with the Seller}

The seller's price is set to be $5$ at the first clock step, as well as the total available transmission power of the seller is $G=100$mW. Other simulation parameters are the same as above. The variation tendency of the seller's reward is shown in Fig.~\ref{auction1} in terms of different number of buyers.

With less than $3$ buyers, the transaction is finished in the first round and the seller choose to utilize the data resource him/herself. However, given more buyers, the seller allocates the power resources relying on the ACA-A auction scheme. As the number of buyers increases, the auction takes more rounds, as well as the traded price tends to be higher, which leads to higher rewards of the seller.



\section{Conclusion}

In this paper, we established two data transaction models in D2D communication networks. A Stackelburg game as well as an ACA-A auction model were proposed for one-buyer/multiple-seller situation and one-seller/multiple-buyers situation, respectively. Moreover, theocratical analysis and numerical simulations were conducted in order both to achieve optimal trading strategies and to provide a new research method for D2D users' behavior.



% use section* for acknowledgement
%\section*{Acknowledgment}



%IEEEhowto:kopka


\bibliographystyle{IEEEtran}
\bibliography{IEEEabrv,ref}
%\begin{thebibliography}{1}
%\bibitem{}
%\end{thebibliography}


\end{document}



%	%\documentclass{standalone}
%
%%\usepackage{graphicx}
%\usepackage{tikz}
%%\usepackage[pdf]{pstricks}
%\usepackage{pstricks}
%\usepackage{pst-sigsys}
%
%\begin{document}


\begin{pspicture}[showgrid=false](0,-1)(13, 0)
%\begin{pspicture}[showgrid=true](0,-0.5)(13, 0.5)

	\psaxeslabels{->}(0,0)(0,0)(13,0){$\frac{d_2}{\epsilon_2}$}{}
	\pstick{90}(2, 0){0.25}
	\pstick{90}(4, 0){0.25}	
	\pstick{90}(6, 0){0.25}		
	\pstick{90}(8, 0){0.25}
	\pstick{90}(10, 0){0.25}
	\pstick{90}(12, 0){0.25}
	\pssignal(0,-0.35){zero}{0}
	\pssignal(2,-0.35){one}{$a^{\dagger}$}
	\pssignal(4,-0.35){d1eps1}{$d_1/\epsilon_1$}
	\pssignal(6,-0.35){d1plus}{$b^{\dagger}$}
	\pssignal(8,-0.35){hata}{$c^{\dagger}$}
	\pssignal(10,-0.35){hatb}{$d^{\dagger}$}
	\pssignal(12,-0.35){one}{$1$}
	
	\dotnode(0,0){a}
	\dotnode(2,0){b}
	\dotnode(4,0){c}
	\dotnode(6,0){d}
	\dotnode(8,0){e}		
	\dotnode(10,0){f}
	\dotnode(12,0){g}
	
	\ncline[offset=-1]{|*-|*}{a}{c}
	\ncput*{\Rmnum{1}}
	

	\ncline[offset=-1]{|*-|*}{c}{d}
	\ncput*{\Rmnum{2}}
	
	\ncline[offset=-1]{|*-|*}{d}{e}
	\ncput*{\Rmnum{3}}

	\ncline[offset=-1]{|*-|*}{e}{f}
	\ncput*{\Rmnum{4}}
	
	\ncline[offset=-1]{|*-|*}{f}{g}
	\ncput*{\Rmnum{5}}	

%	\psBraceDown*(a)(b){down}
%	\psBraceDown(0,0)(3,0){Branch}
	
\end{pspicture}

%\end{document}
	\input{regions3}
%	\begin{pspicture}[showgrid=false](0,-4.5)(15, 0)
%\begin{pspicture}[showgrid=true](0,-4.5)(15, 0)
	\psset{gratioWh=2}


%	\psfblock(7,0){S}{$S^N$}
	\pssignal(7,0) {S}{$S^N$}
	
	\pssignal(3,-1.5){A}{$A$}
	\pssignal(11,-1.5){Ab}{$A^{\complement}$}	

	\pssignal(9.5,-3){B}{$B$}
	\pssignal(12.5,-3){Bb}{$B^{\complement}$}	

	\pssignal(4.5,-3){C}{$C$}
	\pssignal(1.5,-3){Cb}{$C^{\complement}$}	
	
	\pssignal(8.5,-4.5){D}{$B_{\theta}$}
	\pssignal(10.5,-4.5){Db}{$B_{\overline{\theta}}$}	

	%%%%%%%%%%%%%%%%%%  Encoders %%%%%%%%%%%%%%%%%%

%%	\newcount\cnt
%%
%%	\psfblock(3,0){enc}{Encoder}
%%	\dotnode(5.5,0){dot1}
%%	\dotnode(5.5,-2.5){dot2}
%%%	\dotnode(7,0){dot2}
%%
%%%	\dotnode(1,-2.5){dot4}
%%
%%	% Test Channel	
%%%	\dotnode(1,0){dot3}
%%%	\pscircleop[operation=plus](1,-2.5){op3} 
%%%	\pssignal(3,-2.5){W}{$W^{k}$}
%%%	\pssignal(1,-3.5) {U}{$U^{k}$}
%%	
%%	%%%%%%%%%%%%%%%%%% Multipliers %%%%%%%%%%%%%%%%%%
%%%	\cnt=0
%%%	\psforeach{\ry}{0,-2.5}{\advance\cnt by 1\relax
%%%		\pscircleop[operation=times](8,\ry){op\the\cnt}
%%%	}
%%
%%	% Channel
%%	\psfblock(8,0){channel1}{$\textrm{BEC}(\epsilon_{1})$}	
%%	\psfblock(8,-2.5){channel2}{$\textrm{BEC}(\epsilon_{2})$}	
%%
%%%	\psfblock(8,0){channel1}{$p(Y_{1}|X)$}	
%%%	\psfblock(8,-2.5){channel2}{$p(Y_{2}|X)$}	
%%	
%%	%%%%%%%%%%%%%%%%%%  Decoders %%%%%%%%%%%%%%%%%%
%%	\psfblock(11,0){dec1}{Decoder 1}
%%	\psfblock(11,-2.5){dec2}{Decoder 2}
%%		
%%	%%%%%%%%%%%%%%%%%%  S hat %%%%%%%%%%%%%%%%%%
%%	\pssignal(14,0){b1}{$\hat{S}_{1}^{k}$}
%%	\pssignal(14,-2.5){b2}{$\hat{S}_2^{k}$}
%%
%%	%%%%%%%%%%%%%%%%%%  Noise symbols %%%%%%%%%%%%%%%%%%
%%	% Draw N1 noise symbols
%%%	\pssignal(8,-1){N1_1}{$N_{1}^{n}$}
%%%	\pssignal(8,-3.5){N2_1}{$N_{2}^{n}$}
%%
%%
%%	% Only one feedback
%%%%	\dotnode(9.5,-2.5){Y2dot_high}
%%%%	\dotnode(9.5,-3.5){Y2dot_low}
%%%%	\pnode(2.5,-0.65){encdot2_high}
%%%%	\pnode(2.5,-3.5){encdot2_low}
%%
%%
%%	\dotnode(9.5,0){Y1dot_high}
%%	\dotnode(9.5,-1){Y1dot_low}
%%	\pnode(5.7,-1){loopdot_r}
%%	\pnode(5.3,-1){loopdot_l}
%%	\pnode(3.5,-0.65){encdot_high}
%%	\pnode(3.5,-1){encdot_low}
		
	%%%%%%%%%%%%%%%%%%  Connecting blocks 	%%%%%%%%%%%%%%%%%%
%	\nclist[style=Arrow]{ncline}[naput]{S,enc,channel1  $X_{i}$,dec1 $Y_{1,i}$,b1}
%	\nclist[style=Arrow]{ncline}[naput]{channel2,dec2 $Y_{2,i}$,b2}
%	\ncangle[style=Arrow,angleA=-90,angleB=180]{dot1}{channel2}
%	\ncline[style=Arrow]{N1_1}{op1}
%	\ncline[style=Arrow]{N2_1}{op2}

	% Test Channel
%	\ncline[style=Arrow]{W}{op3}
%	\ncline[style=Arrow]{dot3}{op3}
%	\ncline[style=Arrow]{op3}{U}

	\ncline[style=Arrow]{S}{A}
	\nbput{$N(1 - \epsilon_1)$}
	\ncline[style=Arrow]{S}{Ab}
	\naput{$N\epsilon_1$}
	\ncline[style=Arrow]{A}{C} 
	\naput{$N\gamma(1 - \epsilon_1)$}
	\ncline[style=Arrow]{A}{Cb}
	\ncline[style=Arrow]{Ab}{B} 
	\nbput{$N(\epsilon_1 - d_1)$}
	\ncline[style=Arrow]{Ab}{Bb}
	\ncline[style=Arrow]{B}{Db} 
	\naput{$N(1 - \theta)(\epsilon_1 - d_1)$}
	\ncline[style=Arrow]{B}{D} 
	\nbput{$N\theta(\epsilon_1 - d_1)$}
	
	\ncbox[nodesep=.05cm,boxsize=.3,linearc=.2,linestyle=dashed]{C}{D}
%	\ncarcbox[nodesep=.05cm,boxsize=.4,linearc=.2,arcangle=10,linestyle=dashed]{C}{D}
	\nbput{$F$}


%	\nclist[style=Arrow]{ncline}[ncput]{S,A $N(1 - \epsilon_1)$}
%	\nclist[style=Arrow]{ncline}[naput]{S,Ab $N\epsilon_1$}
%	\nclist[style=Arrow]{ncline}[naput]{A,C $N\gamma\epsilon_1$}
%	\nclist[style=Arrow]{ncline}[naput]{A,Cb}
%	\nclist[style=Arrow]{ncline}[naput]{Ab,B $N(\epsilon_1 - d_1)$}
%	\nclist[style=Arrow]{ncline}[naput]{Ab,Bb}
%	\nclist[style=Arrow]{ncline}[naput]{B,Db $N\theta(\epsilon_1 - d_1)$}
%	\nclist[style=Arrow]{ncline}[naput]{B,D $N\epsilon_1$}	



%	\nclist{ncline}{S,A}
%	\nclist{ncline}{A, B}
%	\nclist{ncline}{A, Bb}
%%	\nclist{ncline}{S, Ab}
%	\nclist{ncline}{Ab, C}
%	\nclist{ncline}{Ab, Cb}
%	\ncarc[arcangle=-100]{loopdot_l}{loopdot_r}
%	\nclist{ncline}{loopdot_l,encdot_low}
%	\nclist[style=Arrow]{ncline}{encdot_low, encdot_high}
%	\nclist[linecolor=red,style=Arrow]{ncline}{fdbck_dot_enclow,enc}		

	% only one feedback
%%	\nclist{ncline}{Y2dot_high,Y2dot_low,encdot2_low}
%%	\nclist[style=Arrow]{ncline}{encdot2_low, encdot2_high}
	
\end{pspicture}
%\input{schematicE-MD}
%	\input{regions2}
%	\documentclass[11pt,draftcls]{IEEEtran}{\onecolumn}
% \usepackage[driver]{graphicx}
\usepackage{epsfig}
\usepackage{amsmath,epsfig,dsfont}
\usepackage{subfigure}
\usepackage{algorithm}
\usepackage{algorithmic}
\usepackage{amssymb,bm}
\usepackage{amsfonts}
\usepackage{stfloats}
\usepackage{multirow}
\usepackage{cite}
%\usepackage{pdffig}
%\usepackage[pdftex]{graphicx}

% *** GRAPHICS RELATED PACKAGES ***
%
\ifCLASSINFOpdf
\usepackage{graphicx}
  % declare the path(s) where your graphic files are
  % \graphicspath{{../pdf/}{../jpeg/}}
  % and their extensions so you won't have to specify these with
  % every instance of \includegraphics
  % \DeclareGraphicsExtensions{.pdf,.jpeg,.png}
\else
  % or other class option (dvipsone, dvipdf, if not using dvips). graphicx
\fi
% graphicx was written by David Carlisle and Sebastian Rahtz. It is

% correct bad hyphenation here
\hyphenation{op-tical net-works semi-conduc-tor}


\begin{document}
%
% paper title
% can use linebreaks \\ within to get better formatting as desired
\title{Mobile Data Transactions in Device-to-Device Communication Networks: Pricing and Auction}

\author{Jingjing~Wang,~\IEEEmembership{Student Member,~IEEE,}
        Chunxiao~Jiang,~\IEEEmembership{Senior Member,~IEEE,}
        Zhi~Bie,\\
        Tony~Q.~S.~Quek,~\IEEEmembership{Senior Member,~IEEE,}
        and~Yong~Ren,~\IEEEmembership{Member,~IEEE}
\thanks{J. Wang, C. Jiang, Z. Bie and Y. Ren are with the Department
of Electronic  Engineering, Tsinghua University, Beijing 100084,
P. R. China. E-mail: chinaeephd@gmail.com, \{jchx, reny\}@tsinghua.edu.cn, bz12@mails.tsinghua.edu.cn.}
\thanks{T. Q. S. Quek is with the Singapore University of Technology and Design, 8 Somapah Road, Singapore 487372. Email: tonyquek@sutd.edu.sg.}
}
\maketitle


\begin{abstract}
Device-to-Device (D2D) communication is offering smart phone users a choice to share files with each other without communicating with the cellular network. In this paper, we discuss the behaviors of two characters in the D2D data transaction model from an economic point of view: the data buyers who wish to buy a certain quantity of data, and the data sellers who wish to sell data through D2D network. The optimal price setting and purchasing strategies are desirable, and we give the analysis based on game theory.
\end{abstract}

\begin{IEEEkeywords}
D2D, game theory, multimedia.
\end{IEEEkeywords}

% IEEEtran.cls defaults to using nonbold math in the Abstract.
\IEEEpeerreviewmaketitle

\section{Introduction}
Device-to-Device (D2D) communication has been proposed as a revolutionary paradigm to enhance the capacity of cellular networks~\cite{1}.
According to D2D theory, a cell area under the control of one base station is divided into several clusters, and mobile users in the same cluster are close enough to establish direct connections. This clustering concept can remarkably improve the performance of cellular networks, and  several aspects of the benefits have been investigated~\cite{2}, including higher spectral efficiency, enhanced total throughput, higher energy efficiency and shorter delay.

Since multimedia files, like wireless videos, have been the main driver for the inexorable increase in data transmission, experts have proposed many methods to deal with the problem. In retrospect, researchers proposed methods like: decreasing the cell size to improve the spectral efficiency, using additional spectrum and improving the physical-layer link capacity, but all these traditional methods are not satisfactory and may even bring other problems. Later, it was observed that mobile devices have large storage space. Following this idea, a novel architecture based on D2D communication was thoroughly analyzed in~\cite{11}.

In this paper, we explore the mobile users' behaviors when they buy or sell videos within a D2D cluster in terms of an economic point of view. We model the file distribution as a competition mechanism against the data service of base station, and thus the autonomous D2D mechanism is preferred, where the data transaction is managed by D2D users without manipulation of base station. Furthermore, in order to analyze the optimal selling and purchasing strategies of the users, we use game theory to model different situations. Specifically, we model the one buyer multiple sellers case as a Stackelburg game, and model the one seller multiple buyers case as an auction game. The ultimate goal is to maximize the utility of both sides simultaneously.


%\hfill what is this?
%\hfill September 20, 2014

% \subsection{Subsection Heading Here}
% \subsubsection{Subsubsection Heading Here}

\section{System model}

In the formulation of the channel condition, we assume that in a certain cluster $C_{i}$, a data buyer $B_{i}$ purchases files from a data seller $S_{i}$ with total transmission power $G_{i}$. Moreover, the distance between $B_{i}$ and $S_{i}$ is represented by $d_{i}$, and $D_{ij}$ denotes the distance between two neighbor D2D clusters' centers. Also, we assume that the channel gain is $H_{i}$, as well as $\sigma^2$ represents the additive white Gaussian noise (AWGN) power. Considering $M$ neighbor D2D clusters in a cellular network, the cluster interference power of neighbor cluster $C_{j}$ is denoted as $G_{I,j}$ with the corresponding channel gain $H_{I,j}$, $j=1,\ldots,M$. For $C_{i}$, the Signal to Interference plus Noise Ratio (SINR) is given by:
%\begin{equation}\label{SNRi}
%{\textrm{SNR}}=\frac{GH}{\sqrt{d}\sigma^2}.
%\end{equation}
\begin{equation}\label{SNRi}
\textrm{SINR}=\frac{{{G}_{i}}H_{i}/\sqrt{{{d}_{i}}}}{{{\sigma }^{2}}+\sum\limits_{j=1}^{M}{({G_{I,j}}H_{I,j}/\sqrt{{{D}_{ij}}}})}.
\end{equation}
The total channel bandwidth available for data transactions can be deemed as $W$. Then, relying on Shannon formula, the maximal achievable bit rate between $B_{i}$ and $S_{i}$ follows:
%\begin{equation}\label{Ri}
%R=W\log_2\left(
%1+\frac{GH}{\sqrt{d}\sigma^2}
%\right).
%\end{equation}
\begin{equation}\label{Ri}
{{R}_{i}}=W{{\log }_{2}}\left(1+\frac{{{G}_{i}}H_{i}/\sqrt{{{d}_{i}}}}{{{\sigma }^{2}}+\sum\limits_{j=1}^{M}{({G_{I,j}}H_{I,j}/\sqrt{{{D}_{ij}}})}}\right).
\end{equation}

The total nodal delay is calculated as the sum of processing, queuing, transmission and propagation time. Relying on the delay model in~\cite{103}, which approximatively maps the network architecture into bi-directional graphs, the system delay can be denoted as:
\begin{equation}\label{D}
D(Y,V)= \beta(O) \frac{\sqrt{Y+V}}{\sqrt{\log{(Y+V)}}}\textrm{ ms},
\end{equation}
where the $Y$ represents the number of users (buyers and sellers) in a data transaction model, while $V$ denotes the number of other users in the same cellular network who are simultaneously transmitting data. Furthermore, $\beta(O)$ is a function of the size of transaction data $O$.



\section{Transaction Model}

In this section, we will discuss the data transaction problem in two different situations. The first situation consists of one data buyer purchasing from several data sellers, where the buyer can purchase different coding layers of a video from different sellers and combine these streams during the decoding process~\cite{200}. The second model consists of one seller selling data to multiple buyers. The major difference between the two models is whether the buyer or the seller takes the first move in the game.

\subsection{Initiating with the Buyer}

In this subsection, we consider the transaction model of one data buyer $B$ purchasing multimedia files from $N$ data sellers, namely, $\{S_1,\ldots,S_N\}$ in D2D cluster $C_{i}$. Then, the SINR and the bit rate between $B$ and $S_n$ can be calculated by Eq.~(\ref{SNRi}) and Eq.~(\ref{Ri}). Given the the total available bandwidth $W$, which will be evenly allocated to all users. The maximal achievable bit rate of the buyer is given by:
\begin{equation}
R_B=\frac{W}{N+1}\sum_{n=1}^N\log_2\left(1+\frac{{{G}_{n}}H_{n}/\sqrt{{{d}_{n}}}}{{{\sigma }^{2}}+\sum\limits_{j=1}^{M}{({G_{I,j}}H_{I,j}/\sqrt{{{D}_{ij}}})}}\right).
\end{equation}

Due to transmission delay, the strategies of the buyer and the sellers would be announced sequentially, which is beneficial of constructing a Stackelburg game model~\cite{simaan1976stackelberg}, where a data buyer takes the first step to send a detect signal, in order to inform other mobile users in the same cluster what particular video file he/she wants. Next, the potential sellers send back their prices along with the channel conditions. Then, the data buyer decides how much the transmission power he/she intends to buy from different sellers, respectively.

A data buyer can choose to buy from part of the sellers or all the sellers, and even he/she is capable of selecting service providers, if that way can lead to a higher reward.

\subsubsection{Caching Conditions}

Whether a seller has stored a copy of the file or not can affect his/her reward considerably, because this determines whether the seller needs to download the file from service provider before transmitting it to the buyer.

In~\cite{17}, Zipf distribution has been verified as a good model to measure the popularity of a set of video files. Thus, we assume that there are totally $K$ popular videos in this region within a month, and the popularity of the videos is subject to Zipf distribution with the parameter $\eta$. Then, the required frequency of the $i^{th}$ popular video file can be denoted as:
\begin{equation}
{{\lambda }_{i}}=\frac{1/{{i}^{\eta }}}{\sum\nolimits_{k=1}^{K}{1/{{k}^{\eta }}}}, 1\le i\le K.
\end{equation}
Meanwhile, we assume that the caching of the files are uniformly distributed among the users, which means that for every internal storage unit of a seller's device, the probability that it is used to store one of the $K$ files is $1/K$.

Given that a seller can store $\tau$ different popular video files of this month in his/her internal storage, there are $C_{K}^{\tau }$ possible different storage modes. Furthermore, we denote one of the possible storage mode of the seller as $\Gamma=\{\Gamma_1,\ldots,\Gamma_\tau\}$, where $\Gamma_i$, $i=1,\ldots,\tau$, represents the rank order among $K$ popular video files. Thus, the possibility of one of $C_{K}^{\tau }$ storage modes can be given by:
\begin{equation}
\textrm{Pr}(\Gamma )=\frac{\tau !(K-\tau )!}{K!},
\end{equation}
and the probability of this seller having a cached copy of the wanted file in terms of the $\Gamma$ storage mode follows:
\begin{equation}
\textrm{Pr}(\textrm{wanted}|\Gamma)= \sum_{i=1}^\tau \lambda_{\Gamma_i}.
\end{equation}
Relying on the total probability formula, the probability that a seller has a cached copy of the wanted file can be constructed as:
\begin{equation}\label{PrCache}
\textrm{Pr}(\textrm{Cache})=\sum_{\Gamma} \left[\textrm{Pr}(\textrm{wanted}|\Gamma) \cdot \textrm{Pr}(\Gamma ) \right].
\end{equation}

\subsubsection{Reward Functions}

The data buyer $B$ in cluster $C_{i}$ gains reward by successfully receiving the file with a good quality measured by the maximal achievable transmission rate. Specifically, $B$ has to pay the price $p_n (1 \leq n \leq N)$ for a unit of transmission power of $N$ sellers, respectively. Therefore, the data buyer's reward function can be formulated as:
\begin{equation}\label{PhiB-multiple}
\Phi_B\!=\!\frac{W}{N+1}\sum_{n=1}^N \log_2\left(1\!+\!\frac{{{G}_{n}}H_{n}/\sqrt{{{d}_{n}}}}{{{\sigma }^{2}}+\sum\limits_{j=1}^{M}{({G_{I,j}}H_{I,j}/\sqrt{{{D}_{ij}}})}}\right)\cdot \xi_R-D(Y,V)\cdot\xi_D
-\sum_{n=1}^N p_n\cdot G_n,
\end{equation}
where the coefficient $\xi_R$ represents the unit reward in terms of the maximal achievable bit rate, as well as $\xi_D$ denotes the unit loss measured by the system transmission delayed, which can be deemed as normalizing weight parameters for both $R_B$ and $D(Y,V)$. $G_n$ represents the traded transmission power that the seller $S_n$ transmits to $B$. The system delay $D(Y,V)$ is defined in Eq.~(\ref{D}).

In the following, we formulate the sellers' reward function. We introduce parameters $c_n$, $n=1,\ldots,N$, to represent the unit cost for relaying data between $B$ and $S_n$, which is determined by the characteristics of the sellers' devices. Thus, the utility function of $S_n$ can be defined as:
\begin{equation}\label{PhiS}
\begin{aligned}
\Phi_{S_n}=~&\textrm{Pr}(\textrm{Cache})\cdot(p_n-c_n)\cdot G_n+ \\
&\left[1-\textrm{Pr}(\textrm{Cache})\right]\cdot(p_n-c_n-s)\cdot G_n,
\end{aligned}
\end{equation}
where $s$ is the unit cost of downloading files from the nearest base station.
Additionally, we has a maximum cellular users constraint, i.e., $Y+V \leq \Omega$.
\subsubsection{Optimal Strategies}

In a Stackelburg game, the optimal transaction strategies for both sides exist and can be obtained using backward induction. The last move is for the data buyer to determine the optimal transmission power $G_n$ acquired from seller $S_n$ to maximize his/her reward $\Phi_B$. Moreover, we assume that the possible power of a received video is continuous~\cite{15}.

Hence, we can obtain $G_n$ by letting the first-order derivative of $\Phi_B$ with respect to $G_n$ be zero, which leads to
\begin{equation}\label{Qi}
{{G}_{n}}=\frac{W\xi_{R}}{(N+1){{p}_{n}}\ln 2}-\frac{\sqrt{{{d}_{n}}}\left[{{\sigma }^{2}}+\sum\limits_{j=1}^{M}\left({{G_{I,j}}H_{I,j}/\sqrt{{{D}_{ij}}}}\right)\right]}{H_{n}}.
\end{equation}

As the second last move, the sellers need to decide the optimal price $p_n$ in terms of the transmission power $G_n$ purchased by the buyer. We substitute $G_n$ in Eq.~(\ref{PhiS}) with Eq.~(\ref{Qi}), and let the first-order derivative with respect to $p_n$ equal zero, we have the optimal price of the data seller $S_{n}$:
%\begin{equation}
%\frac{W{{\xi }_{R}}[{{c}_{n}}+s(1-\Pr (\textrm{Cache}))]}{(N+1)p_{n}^{2}ln2}\!=\!-\!\frac{\sqrt{{{d}_{n}}}[{{\sigma }^{2}}\!+\!\sum\limits_{j=1}^{M}({{G_{I,j}}H_{I,j}/\sqrt{{{D}_{ij}}}})]}{H_{n}}.
%\end{equation}
%Therefore, the optimal price of the data seller $S_{n}$ should be set as:
\begin{equation}\label{pi}
p_n=\sqrt{\frac{WH_{n}\xi_R[c_n+s(1-\Pr(\textrm{Cache}))]}{(N+1)\sqrt{d_n}\left[\sigma^2+ \sum\limits_{j=1}^{M}\left({{G_{I,j}}H_{I,j}/\sqrt{{{D}_{ij}}}}\right)\right]\ln{2}}}.
\end{equation}
Substituting $p_n$ in Eq.~(\ref{Qi}) with Eq.~(\ref{pi}), we can achieve the optimal transmission power $G_n$.

\subsection{Initiating with the Seller}

In this situation, if a seller wants to sell data to nearby users, he/she takes the first step to send a probe signal to announce the availability of stored data resources. The seller would prefer to maximize his/her reward by transmitting popular video files in terms of a fierce competition among the potential multiple data buyers.

We model this decentralized process as alternative ascending clock auction (ACA-A) game, because it guarantees an efficient and cheat-proof allocation~\cite{chen2010spectrum}. In our model, there are $N$ buyers, namely $B_n$, $n=1,\ldots,N$, competing for high quality data from one seller $S$.

Similarly, the SINR and maximal achievable bit rate are given by Eq.~(\ref{SNRi}) and Eq.~(\ref{Ri}). Then, the reward function of the buyers at clock step $t$ can be deemed as:
\begin{equation}
\Phi_{B_n}^t(p^0,G_n)=R_{B_n}^{t}\cdot \xi_R-D(Y,V)^{t}\cdot\xi_D-p^{t}\cdot G_{n}^{t},
\end{equation}
where, $p^{t}$ represents the price of unit transmission power charged by $S$.
Suppose that the seller wants to sell total power of $G$. In the beginning of the auction, $S$ sets up the clock index $t=0$, regulates time-step size $\delta >0$, and announces the initial price $p^0$ to all the D2D users in the cluster $C_{i}$. Then, each potential buyer $B_n$ offers his/her optimal bid $G_n^0$ by computing
\begin{equation}
G_n^0=\arg \max_{G_n} \Phi_{B_n}\left(p^0,G_n\right).
\end{equation}
Then, the seller sums up all the bids $G_{total}^0=\sum_n {G_n^0}$, and compares $G_{total}^0$ with $G$. If $G_{total}^0 \leq G$, the seller concludes the auction. Otherwise, the seller sets $p^{t+1}=p^t+\delta$, $t=t+1$ and announces the price $p^t$ to all the buyers who bid for the next time. Thus, each buyer offers his/her optimal bid $G_n^{t+1}$ by calculating
\begin{equation}
G_n^{t+1}=\arg \max_{G_n} \Phi_{B_n}\left(p^{t+1},G_n\right).
\end{equation}

After collecting all the bids, the seller computes the total bid $G_{total}^{t+1}$. The seller continues the auction until $G_{total}^{t+1} \leq G$. At every clock $t$, the seller computes the cumulative clinch $\chi_n^t$, which is the amount of data that the buyer $B_{n}$ is guaranteed to win at clock $t$:
\begin{equation}
\chi_n^t=\max \left(0,G-\sum_{j \neq n}{G_j^t}\right),
\end{equation}
and the buyer will purchase this $\chi_n^t$ quantity of power with price $p^t$. This has been proved to be a cheat-proof scheme.

Let the final clock index be $T$. As the price $p$ increases discretely, we may have $G_{total}^{T} < G$, in which case the power resource is not fully utilized. We adopt a proportional rationing scheme in order to make sure that $G_{total}^{T} = G$ ~\cite{ausubel2004efficient}. Therefore, the final allocated transmission power can be denoted by:
\begin{equation}\label{Qi_final}
G_{n,final}=G_n^{T}+\frac{G_n^{T-1}-G_n^{T}}{\sum\limits_{i=1}^{N}{G_i^{T-1}}-\sum\limits_{i=1}^{N}{G_i^{T}}} \left(G-\sum\limits_{i=1}^{N}{G_i^{T}}\right),
\end{equation}
which satisfy $\sum_n{G_{n,final}}=G$.

We set $\chi_n^{T}=G_{n,final}$, and the total payment of buyer $B_n$ is given by:
\begin{equation}
P_n=\chi_n^0p^0+\sum_{t=1}^{T} p^t\left(\chi_n^t-\chi_n^{t-1}\right).
\end{equation}

In this model, the reward of the seller can be formulated as:
\begin{equation}
\Phi_{S}=\sum_{n=1}^N{\left(P_n-c_n G_{n,final}\right)},
\end{equation}
where $c_n$ accounts for the unit transmission cost.

\section{Simulation Results}

In this section, we will show the variation tendency of the sellers' and buyers' reward functions and their optimal choices in different models.

\subsection{Initiating with the Buyer}

In our simulation, we set the data buyer at the origin (0,0) and let other mobile users uniformly distribute in a cell. The radius of a cell is $500$m, as well as the radius of a D2D cluster is set as $100$m. Potential sellers, i.e., $N=10$, are uniformly distributed in a D2D cluster. Besides, the number of clusters, which are randomly scattered in a cell, is $M=10$. We set the noise power spectrum density to be $-174$dBm, and the total available bandwidth for D2D data transaction is $5$MHz under an ideal channel gain $H=1$. Moreover, for all users, the maximal transmit power is $100$mW, i.e., $G_{n}\leq 100$mW, $G_{I,j}\leq 100$mW, and the normalized weighting parameters $\xi_R=3\times10^{-5}$/bps as well as $\xi_D=10^{-1}$/ms in terms of $\beta(O)=20$ under the assumption of the same packet size $O=1$MB. The data seller's unit transmission power cost $c$ is uniformly distributed in $[0.1, 0.5]$/mW. Finally, we set the probability that the seller has stored a copy of the wanted file to be $\textrm{Pr}(\textrm{Cache})=0.3, 0.4, 0.5$, respectively.


\begin{figure}
\begin{center}
\includegraphics[width=0.6\textwidth]{1.eps}
\end{center}
\caption{Variation tendency of the reward functions for the buyer and sellers versus the increasing sellers' number.}\label{MultipleU}
\end{figure}

\begin{figure}
\begin{center}
\includegraphics[width=0.6\textwidth]{2.eps}
\end{center}
\caption{Variation tendency of the optimal choices of the buyer and sellers versus the increasing sellers' number.}\label{MultipleE}
\end{figure}

\begin{figure}
\begin{center}
\includegraphics[width=0.6\textwidth]{3.eps}
\end{center}
\caption{Variation tendency of the seller's reward function and the total rounds of the auction with respect to the number of buyers.}\label{auction1}
\end{figure}
From Fig.~\ref{MultipleU} and Fig.~\ref{MultipleE}, we can conclude that as the number of potential sellers increases, the average price of sellers decreases, while the total traded transmission power improves. Furthermore, the data buyer's reward increases at the beginning, and then keeps constant. That is because when there are more sellers, they may compete with each other and lower the price. However, when the buyer purchase from superfluous buyers, system delay start to play a critical role in reducing the rewards. Also, with the improving of the probability of having a wanted cached video copy, the sellers tend to lower their prices, which contributes to higher rewards.
\subsection{Initiating with the Seller}

The seller's price is set to be $5$ at the first clock step, as well as the total available transmission power of the seller is $G=100$mW. Other simulation parameters are the same as above. The variation tendency of the seller's reward is shown in Fig.~\ref{auction1} in terms of different number of buyers.

With less than $3$ buyers, the transaction is finished in the first round and the seller choose to utilize the data resource him/herself. However, given more buyers, the seller allocates the power resources relying on the ACA-A auction scheme. As the number of buyers increases, the auction takes more rounds, as well as the traded price tends to be higher, which leads to higher rewards of the seller.



\section{Conclusion}

In this paper, we established two data transaction models in D2D communication networks. A Stackelburg game as well as an ACA-A auction model were proposed for one-buyer/multiple-seller situation and one-seller/multiple-buyers situation, respectively. Moreover, theocratical analysis and numerical simulations were conducted in order both to achieve optimal trading strategies and to provide a new research method for D2D users' behavior.



% use section* for acknowledgement
%\section*{Acknowledgment}



%IEEEhowto:kopka


\bibliographystyle{IEEEtran}
\bibliography{IEEEabrv,ref}
%\begin{thebibliography}{1}
%\bibitem{}
%\end{thebibliography}


\end{document}



%	\input{region2}
%	\includegraphics[scale=0.9]{fig/fig1}
	\vspace{-0.5em}
	\caption{Broadcasting a Gaussian source over a Gaussian broadcast channel with individual bandwidth mismatches.  Here, the total number of channel uses, $n = \max (n_{1}, n_{2}) = n_{2}$.}
	\label{fig:schematic}
	\vspace{-1em}
\end{figure}
%\end{figure*}
%%%%%%%%%%%%%%%%%%%
%%%%%%%%%%%%%%%%%%%%

\bibliographystyle{IEEEtran}
%\bibliographystyle{hieeetr}
\bibliography{IEEEabrv,isit2012}

%\begin{thebibliography}{99}
%%\baselineskip 6mm
%
%\bibitem{MP} U.~Mittal and N.~Phamdo, ``Hybrid digital-analog (HDA) joint source-channel codes for broadcasting and robust communications,'' {\it IEEE Trans.~Info.~Theory,} vol.~48, no.~5, pp.~1082--1102, May 2002.
%
%\bibitem{RFZ} Z.~Reznic M.~Feder and R.~Zamir, ``Distortion Bounds for Broadcasting With Bandwidth Expansion,'' {\it IEEE Trans.~Info.~Theory,} vol.~52, no.~8, pp.~3778--3788, Aug.~2006.
%
%\bibitem{PPR} V.~M.~Prabhakaran, R.~Puri and K.~Ramchandran, ``Hybrid Digital-Analog Codes for Source-Channel Broadcast of Gaussian Sources over Gaussian Channels,'' {\it IEEE Trans.~Info.~Theory,} vol.~57, no.~7, pp.~4573--4588, July 2011.
%
%\bibitem{LS} Y.~Li and E.~Soljanin, ``Rateless Codes for Single-Server Streaming to Diverse Users,'' in \it{Proc.~47th Annual Allerton Conference on Communication, Control, and Computing}, Montecello, IL, 2009, pp.~1419--1426.
%
%\bibitem{LT} L.~Tan, MASc.\ thesis, Dept.\ Elect.\ Eng., Univ.\ Toronto, Toronto, ON, 2012.  In preparation.
%
%\bibitem{CT} T.~M.~Cover and J.~A.~Thomas, \emph{Elements of Information Theory}, 2nd Edition.  New Jersey: Wiley, 2006.

%\bibitem{TS} C.~Tian and S.~Shamai, ``A unified coding scheme for hybrid transmission of Gaussian source over Gaussian channel',' in \it{2008 IEEE International Symposium on Information Theory (ISIT)}, Toronto, Canada, Jul. 2008.

%\bibitem{SVZ} S.~Shamai, S.~Verdu and R.~Zamir, ``Systematic lossy source/channel coding,'' {\it IEEE Trans.~Info.~Theory,} vol.~44, no.~2, pp.~564--579, March 1998.  
%
%\bibitem{WZ} A.~D.~Wyner and J.~Ziv, ``The rate-distortion function for source coding with side information at the decoder,'' {\it IEEE Trans.~Info.~Theory,} vol.~IT22, pp.~1--10, Jan.~1976.
%
%\bibitem{RMG} R.~M.~Gray, ``A new class of lower bounds to information rates of stationary sources via conditional rate-distortion functions,'' {\it IEEE Trans.~Info.~Theory,} vol.~IT19, pp.~480--489, July~1973.
%
%\bibitem{TJG} T.~J.~Goblick, ``Theoretical limitations on the transmission of data from analog sources,'' {\it IEEE Trans.~Info.~Theory,} vol.~IT11, pp.~558--567, Oct.~1965.

%\end{thebibliography}
\end{document}
