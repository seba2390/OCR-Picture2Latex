\subsubsection{Inner Bound}
\label{subsubsec:inner_bound}

In this section, we formulate a hybrid coding scheme that incorporates both repetition coding and sending random linear combinations of source symbols.  The code is tuned via two parameters, $\omegaparam, \gammaparam \in [0, 1]$.  We target the point-to-point optimal distortion/latency for the stronger user in this section, and show that the stronger user can achieve point-to-point optimal performance for any values of $\omegaparam$ and $\gammaparam$.  In the next section however, we show how to optimize $\omegaparam$ and $\gammaparam$ to also achieve point-to-point optimal performance for the weaker user.  As in the coding scheme in Section~\ref{sec:two_users}, we again split the coding scheme into phases.  

In Phase~\Rmnum{1}, we begin by sending each source symbol uncoded over the channel.  That is, Phase~\Rmnum{1} consists of $N$ transmissions and at time $t \in \{1, 2, \ldots, N\}$, we transmit $X(t) = S(t)$.  Let $\Aset \subseteq [N]$ be the set of symbols received by the stronger user in Phase~\Rmnum{1}.  Since the stronger user's feedback is available to all receivers and the transmitter, $\Aset$ is known to all parties.  
%We have that $\mathbb{E}|A| = 1 - \epsilon_1$ 

At the conclusion of Phase~\Rmnum{1}, we have that on average, for $i \in \{1, 2\}$, user~$i$ will have received $N(1 - \epsilon_i)$ source symbols and so will require an additional $N(\epsilon_i - d_i)$ symbols in the remaining phases.   Before moving on to Phase~\Rmnum{2}, we first organize the source symbols in $\Aset$ and $\AsetC$ into subsets, where $\AsetC \subseteq [N]$ denotes the complement of set $\Aset$. We first isolate a fraction of $N(\epsilon_1 - d_1)$ source symbols from $\AsetC$ into a set denoted as $\Bset$.  That is, we fix the remaining $N(\epsilon_1 - d_1)$ symbols that the stronger user requires in $\Bset$.  We then partition $\Bset$ into two disjoint sets, one that contains a fraction of $\omegaparam \in [0, 1]$ source symbols from $\Bset$, denoted as $\BsetOmega$, and the other that contains a fraction of $1 - \omegaparam$ symbols, denoted as $\BsetOmegaC$, where $\Bset = \BsetOmega \cup \BsetOmegaC$.  Random linear combinations of the symbols in $\BsetOmega$ will be sent to the stronger user while the symbols in $\BsetOmegaC$ will be sent with repetition coding.  We further take a fraction of $\gammaparam \in [0, 1]$ source symbols from $\Aset$ and denote this set as $\Cset$.  Finally, we define $\Fset$ as the union of sets $\Cset$ and $\BsetOmega$.  Figure~\ref{fig:} illustrates the relationship between all sets and the manner in which they are constructed.

In Phase~\Rmnum{2} of the coding scheme, we send random linear combinations of the source symbols in $\Fset \triangleq \Cset \cup \BsetOmega$.  Let $v(t)$ be a \emph{new} random linear combination of the source symbols in $\Fset$ generated at time $t$.  Since $\Cset \subseteq \Aset$, the stronger user can subtract off all symbols originating from $\Cset$ and so the only symbols providing new information are those originating from $\BsetOmega$.  We use the stronger user's feedback to continue to send random linear combinations until the stronger user has received $|\BsetOmega|$ equations, where $\mathbb{E}|\BsetOmega| = \omegaparam(\epsilon_1 - d_1)$.  At the conclusion of Phase~\Rmnum{2}, by construction, the stronger user has decoded all symbols in $\Fset$.

In Phase~\Rmnum{3}, we use a repetition coding scheme to send each symbol $\bOmegaC \in \BsetOmegaC$.  However, since all symbols in $\Fset$ have been decoded by the stronger user at the beginning of Phase~\Rmnum{3}, notice that we can also combine the symbol $\bOmegaC$ with a random linear combination of the symbols in $\Fset$ to the benefit of the weaker user, but at no extra cost to the stronger user.  That is, let $\bOmegaC$ be a symbol that would be sent with a repetition scheme at time $t$.  Then instead of transmitting $\bOmegaC$ at time $t$, we instead transmit $v'(t) = v(t) + \bOmegaC$ where $v(t)$ is a \emph{new} random linear combination of the source symbols in $\Fset$, which can be subtracted off by the stronger user.
As with a conventional repetition scheme, we also repeat $\bOmegaC$ until it is received by the stronger user for all $\bOmegaC \in \BsetOmegaC$.  However, the difference in Phase~\Rmnum{3} is that we also combine $\bOmegaC$ with $v(t)$, a different linear combination of the source symbols in $\Fset$ at every retransmission.  At the conclusion of Phase~\Rmnum{3} it is not hard to see that the stronger user achieves point-to-point optimal performance, since every channel symbol received has lead to the decoding of a source symbol.  

Finally, in Phase~\Rmnum{4}, if the weaker user has not yet met his distortion constraint, we continue to send random linear combinations of the source symbols in $\Fset$.

%instead of sending the random linear combination $v(t)$ at time $t$ as in Phase~\Rmnum{2}, we instead transmit $v'(t) = v(t) + \bOmegaC$ where $\bOmegaC \in \BsetOmegaC$ is a source symbol that has yet to be received by the stronger user.  Since the stronger user has decoded all symbols in $\Fset$ from the previous phase, he can subtract off $v(t)$ from $v'(t)$ at any time $t$ in Phase~\Rmnum{3}. 

