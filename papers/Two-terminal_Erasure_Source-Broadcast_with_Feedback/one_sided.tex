\section{Source-broadcast with One-sided Feedback}
\label{sec:one_sided_feedback}

In this section we consider a one-sided-feedback variation of the problem in Section~\ref{sec:two_users} whereby in a broadcast network with two receivers, a feedback channel is available to only the stronger user.  In this scenario, we show that given the distortion constraints of both users, there is no overhead in the minmax latency achieved.  Specifically, for $i \in \{1, 2\}$, let $\wid$ be the point-to-point optimal latency for user~$i$ and let $\wplusd$ be the Shannon lower bound for the minmax latency where

\begin{align}
\label{eq:wid_optimal}
	\wid = \frac{1 - d_i}{1 - \epsilon_i},
\end{align}
and
\begin{align}
\label{eq:wplusd}
	\wplusd = \max_{i \in \{1, 2\}} \frac{1 - d_i}{1 - \epsilon_i}.
\end{align}
Section~\ref{sec:two_users} showed that for $i \in \{1, 2\}$, user~$i$ can achieve distortion $d_i$ at the optimal latency $\wid$ when a feedback channel is available to \emph{both} users (c.f.\ Section~\ref{sec:individual_latencies} on individual decoding delays).  Clearly, the optimal minmax system latency $\wplusd$ is also achievable in this case.  In contrast, in this section we show that when a feedback channel is available to \emph{only the stronger user}, while the \emph{individual} optimal latencies may or may not be achievable, the overall system latency $\wplusd$ is still achievable.

% achieve point-to-point optimal performance when a feedback channel is available to \emph{both} receivers, in this section we show that point-to-point optimal performance may also be achieved if \emph{only the stronger user} has a feedback channel available.

% \documentclass[ex_article]{subfiles}
% \begin{document}
\section{Problem Formulation}\label{sec:problem}
%\subsection{Linear system}
We consider the linear time-invariant (LTI) system
\begin{align}
  \begin{aligned}
    \dot x(t) = Ax(t)+Bu(t), \quad
    y(t)     = C x(t),\quad
    x(0) \sim \mathcal D,
  \end{aligned}\label{eq:system}
\end{align}
where $x(t) \in \R^n$ is state, $u(t)\in \R^m$ is input,
$y(t)\in \R^p$ is output, $A\in \R^{n\times n}$,
$B\in \R^{n\times m}$, and $C \in \R^{p\times n}$ are constant matrices, and $\mathcal D$ is a probability distribution over $\R^n$.
In this paper, we assume that $B$ and $C$ are not zero matrices,
and
$(A, B, C, \mathcal {D})$
is unknown unlike the situation in \cite{fatkhullin2021optimizing}.
%\subsection{LQR problem with structured constraints} \label{Sec2-B}
The infinite-horizon continuous-time LQR problem is formulated as
\begin{align}
  \minimize  & E_{x(0)\sim \mathcal{D}}\qty[\int_0^\infty \qty(y^\top(t) Q y(t) + u^\top(t) R u(t))dt ] \label{eq:objectivefunction} \\
  \subjectto & ~\eqref{eq:system}
\end{align}
with constant positive definite matrices $Q \in \R^{p\times p}$ and $R\in \R^{m\times m}$.
The expectation is taken with respect to the initial state $x(0) \sim \mathcal{D}$.
For the static output feedback $u(t) = -Ky(t)$ with $K\in \R^{m\times p}$ to system~\eqref{eq:system},
the objective function~\eqref{eq:objectivefunction} becomes
$
  f(K) := E_{x(0)\sim \mathcal{D}}\qty[\tilde{f}(K;x(0))],
$
where 
\begin{align}
  \tilde f(K;v) & := \int_0^\infty \qty[v^* e^{A_K^\top t}C^\top(Q+ K^\top RK)Ce^{A_K t}v]dt\label{eq:cost}
\end{align}
for $v\in \C^n$.
Then, the closed-loop is given by
\begin{align}
    \dot x(t)  = A_Kx(t),\quad
    y(t)          = Cx(t), \label{eq:closedloop}
\end{align}
where 
\begin{align}
  A_K := A-BKC. \label{eq:AK}
\end{align}



In this paper, we consider the constraints $K\in \Omega$,
where $\Omega \subset \R^{m\times p}$ is a closed convex set
that specifies the structural information of feedback gains.
This is because a structured policy is often used in practical situations.
For example,
%\subsubsection{Decentralized control}
\begin{itemize}
    \item Decentralized control: In decentralized control, some components of $K$ need to be $0$~\cite{jovanovic2016controller}.
This implies that $\Omega$ should be a certain linear subspace of $\R^{m\times p}$.

\item Linear port-Hamiltonian system: For a linear port-Hamiltonian system~\cite{Jacob2012linear}, 
if the feedback gain is positive semi-definite, the closed loop system is also a port-Hamiltonian system and passive. To ensure passivity, $\Omega$ should be defined as the set of positive semi-definite matrices, which is closed and convex.

\end{itemize}
%\subsubsection{Linear port-Hamiltonian system}



%See Section~\ref{sec:examples}.


By using Bellman lemma~\cite{bellman1957notes}, the problem~\eqref{eq:objectivefunction} with structured constraints can be formulated as
\begin{align}
  \begin{aligned}
    \minimize_K & f(K) = \tr(X\Sigma) \\
    \subjectto  & K \in \Omega\,\, \text{and}\,\, A_K\text{ is \textit{Hurwitz}},
  \end{aligned}\label{eq:problem}
\end{align}
where
$
  \Sigma  := E[x(0)x^\top(0)]
$
and $X$ is the solution to
\begin{align}
  A_K^\top X + XA_K + C^\top \qty(K^\top RK + Q)C = 0.
\end{align}
It is difficult to solve \eqref{eq:problem}, since  $f(K)$ is non-convex and saddle points may exist~\cite{fatkhullin2021optimizing}.
Moreover,
 the feasible set may have exponentilally many disconnected components~\cite{feng2019exponential}. 
Although an iterative method was proposed in \cite{zhu2015adaptive} to obtain a suboptimal static output feedback gain in the model free setting,
it cannot be applied directly to problem \eqref{eq:problem} due to the constraint $K\in \Omega$.

% In this paper, we suppose that $(A, B, C, \mathcal {D})$ in \eqref{eq:system}
% is unknown unlike the situation in \cite{fatkhullin2021optimizing}.
% Thus, we develop a model free algorithm in Section \ref{sec:model-free} for solving the problem \eqref{eq:problem}.


To develop a model free algorithm with theoretical guarantees for solving problem \eqref{eq:problem}, we impose the following throughout this paper:
\begin{assumption}\label{assume:sigma-hurwitz}
  \indent
  \begin{enumerate}
    \item $\Sigma \succ 0$.
    \item The pair $(A, C)$ is observable.
    \item There exists $K_0 \in \Omega$
          such that $A_{K_0}$ is \textit{Hurwitz}
          and $K_0$ is known.
  \end{enumerate}
\end{assumption}


Since $A_{K_0}$ is \textit{Hurwitz},
there exist positive definite matrices $G, H$ and a skew-adjoint matrix $J$
such that
$
  A_{K_0} = (J-G)H.
$
The proof is found in~\cite{prajna2002lmi}.
Let $H = L^\top L$ be the Cholesky decomposition.
Using the coordinate transformation $x'(t) = Lx(t)$, the closed-loop system~\eqref{eq:closedloop} becomes
\begin{align}
  \dot{x'}(t)  = A'_{K_0}x'(t), \quad 
  y(t)         = C'x'(t),
\end{align}
where $A'_{K_0}  = LJL^\top-LGL^\top, C' = CL^{-1}$.
Since $LGL^\top\succ 0$ and $LJL^\top = -(LJL^\top)^\top$, we have
  $A'_{K_0}+{A'_{K_0}}^\top = -2LGL^\top \prec 0$.
In the following, we assume system \eqref{eq:closedloop} after the above coordinate transformation, because we consider a static output feedback that is invariant by the coordinate transformation. That is, without loss of generality, we can assume
$A_{K_0}+A_{K_0}^\top \prec 0$.

Under Assumption \ref{assume:sigma-hurwitz},
 $f(K)$ of \eqref{eq:problem} is defined only on the set $S$ of stabilizing controllers, which is defined as
\begin{align}
  S = \{K\in \R^{m\times p}\mid A_K\text{ is \textit{Hurwitz}}\}. \label{eq:S}
\end{align}
If $K\notin S$, there exists an eigenvalue $\mu$ of $A$ such that ${\rm Re}(\mu) \geq 0$ and $f(K)$ goes to infinity.

\begin{remark} \label{remark1}
The objective function of
 problem \eqref{eq:objectivefunction}
is not a
standard LQR cost
%is given by
% \begin{align}
%     \int_0^\infty \qty(x^\top(t) Q x(t) + u^\top(t) R u(t))dt,
% \end{align}
% where $Q \in \R^{n\times n}$ and $R\in \R^{m\times m}$ are positive definite matrices. However, in the model free and output feedback setting, this cost cannot be calculated in practice, because the information of the state $x(t)$ is not available. Therefore, we consider the problem \eqref{eq:objectivefunction} following
as in some previous researches~\cite{modares2016optimal, rizvi2018output}. While similar convergence properties to the standard LQR cost can be obtained for our formulation in the model based setting if $(A, C)$ is observable~\cite{fatkhullin2021optimizing}, more detailed studies of the objective function properties are necessary for model-free version of the convergence analysis.  
\end{remark}



%\subsection{Examples}\label{sec:examples}


% \end{document}

\subsection{Coding Scheme}

We begin by reviewing a repetition coding scheme in the next subsection, whereby we simply ignore the weaker user, and focus on using the stronger user's feedback to retransmit each of his required source symbols until it is received.  A repetition coding scheme is useful insofar as it helps avoid compelling the weaker user to decode additional source symbols that he does not require.  

As an example, consider when the stronger user requires $N$ source symbols.  We could send random linear combinations of the $N$ symbols, which he could decode after receiving $N$ equations through, say, $W$ transmissions.  By the time the stronger user has recovered the $N$ equations, the weaker user, having a weaker channel, would have received less than $N$ equations.  At this point, the weaker user could simply ignore the first $W$ transmissions, and have the transmitter encode another random linear combination of symbols for the weaker user to decode.  However, such a timesharing scheme is inefficient.  
%
On the other hand, the weaker user could prevent the first $W$ transmissions from going to waste by continuing to receive random linear combinations of the group of $N$ source symbols originally intended for the stronger user.  However, if the weaker user requires $M < N$ symbols, he would have had to listen to many more transmissions than necessary to recover $M$ symbols thus introducing delay.

We notice the problem is in the random linear combinations used in our coding scheme.  In such a scheme, we either receive more than $N$ equations and decode the entirety of the $N$ source symbols, or we receive less than $N$ equations and decode none of the source symbols.  This ``threshold effect'' is detrimental when we have heterogeneous users in a network who require $M < N$ source symbols.  

%The weaker user would similarly need to receive $N$ equations to decode the $N$ source symbols, however, if he only requires $M < N$ symbols, he would have had to listen to many more transmissions than necessary to recover $M$ symbols.  That is, 

%Due to the threshold effect of channel coding, if we send 

The repetition coding scheme avoids this pitfall by avoiding random linear combinations altogether and instead transmitting uncoded source symbols over the channel.  While this avoids compelling the weaker user to decode unnecessary source symbols, it can also be inefficient for the weaker user.  Specifically, since the repetition scheme is based solely on the stronger user's feedback, a source symbol can be retransmitted even after it is received by the weaker user.  We show how to circumvent this problem by creating a hybrid coding scheme that consists of both repetitions, and random linear combinations.  The coding scheme is controlled by two variables, $\omegaparam$, and $\gammaparam$.  We show how to choose specific values for these parameters to achieve the optimal minmax latency in Section~\ref{sec:one_sided_optimality}.

%We then show how to incorporate random linear combinations 

\input{one_sided/repetition_coding}
\subsubsection{Inner Bound}
\label{subsubsec:inner_bound}

In this section, we formulate a hybrid coding scheme that incorporates both repetition coding and sending random linear combinations of source symbols.  The code is tuned via two parameters, $\omegaparam, \gammaparam \in [0, 1]$.  We target point-to-point optimal performance for the stronger user in this section, and show that this is possible for any values of $\omegaparam$ and $\gammaparam$.  In the next section however, we show how to optimize $\omegaparam$ and $\gammaparam$ to achieve the optimal minmax latency.  As in the coding scheme in Section~\ref{sec:two_users}, we again split the coding scheme into phases.  

In Phase~\Rmnum{1}, we begin by sending each source symbol uncoded over the channel.  That is, Phase~\Rmnum{1} consists of $N$ transmissions and at time $t \in \{1, 2, \ldots, N\}$, we transmit $X(t) = S(t)$.  Let $\Aset \subseteq \{S(1), S(2), \ldots, S(N)\}$ be the set of symbols received by the stronger user in Phase~\Rmnum{1}.  Since the stronger user's feedback is available to all receivers and the transmitter, $\Aset$ is known to all parties.  
%We have that $\mathbb{E}|A| = 1 - \epsilon_1$ 

At the conclusion of Phase~\Rmnum{1}, we have that on average, for $i \in \{1, 2\}$, user~$i$ will have received $N(1 - \epsilon_i)$ source symbols and so will require an additional $N(\epsilon_i - d_i)$ symbols in the remaining phases.   Before moving on to Phase~\Rmnum{2}, we first organize the source symbols in $\Aset$ and $\AsetC$ into subsets, where $\AsetC \subseteq \{S(1), S(2), \ldots, S(N)\}$ denotes the complement of set $\Aset$. We first isolate a fraction of $N(\epsilon_1 - d_1)$ source symbols from $\AsetC$ into a set denoted as $\Bset$.  That is, we fix the remaining $N(\epsilon_1 - d_1)$ symbols that the stronger user requires in $\Bset$.  We then partition $\Bset$ into two disjoint sets, one that contains a fraction of $\omegaparam \in [0, 1]$ source symbols from $\Bset$, denoted as $\BsetOmega$, and the other that contains the remaining fraction of $1 - \omegaparam$ symbols, denoted as $\BsetOmegaC$, where $\Bset = \BsetOmega \cup \BsetOmegaC$.  Random linear combinations of the symbols in $\BsetOmega$ will be sent to the stronger user while the symbols in $\BsetOmegaC$ will be sent with repetition coding.  We further take a fraction of $\gammaparam \in [0, 1]$ source symbols from $\Aset$ and denote this set as $\Cset$.  Finally, we define $\Fset$ as the union of sets $\Cset$ and $\BsetOmega$.  Figure~\ref{fig:set_construction} illustrates the relationship between all sets and the manner in which they are constructed.

\begin{figure}
	\centering
%	%\documentclass{standalone}
%
%\usepackage{tikz}
%%\usepackage[pdf]{pstricks}
%\usepackage{pstricks}
%\usepackage{pst-sigsys}
%
%\begin{document}

\psset{xunit=0.5cm,yunit=0.5cm}

\begin{pspicture}[showgrid=false](0,-15)(15, 0)
%\begin{pspicture}[showgrid=true](0,-15)(15, 0)
%	\psset{xunit=0.5cm,yunit=0.5cm}
	\scriptsize
	\pssignal(0,-7) {S}{{\normalsize$S^{k}$}}
	
	%%%%%%%%%%%%%%%%%%  Encoder %%%%%%%%%%%%%%%%%%
	\psfblock[framesize=2 7.5](3.5,-7){enc}{{\normalsize Encoder}}
	\newcount\cnt
	
	%%%%%%%%%%%%%%%%%%  Decoder %%%%%%%%%%%%%%%%%%
	\psfblock[framesize=1.5 3](15,-2.5){dec1}{{\small Decoder 1}}
	\psfblock[framesize=1.5 4.5](15,-10.25){dec2}{{\small Decoder 2}}

	%%%%%%%%%%%%%%%%%% Multipliers %%%%%%%%%%%%%%%%%%
	\cnt=0
	\psforeach{\ry}{0,-2,-4,-6,-8,-10,-12,-14}{\advance\cnt by 1\relax
		\pscircleop[oplength=.08](10,\ry){op\the\cnt}
		\pnode(5.5,\ry){enc_dot\the\cnt}
		\pnode(13.5,\ry){dec_dot\the\cnt}
	}
	
	\cnt=0
	\psforeach{\ry}{-1,-3,-5,-7,-9,-11,-13,-15}{\advance\cnt by 1\relax
		\pnode(10,\ry){N_arrow\the\cnt}
		\ncline[style=Arrow]{N_arrow\the\cnt}{op\the\cnt}
	}
	
	\pssignal(6.5,0.5){X1}{$X(1)$}
	\pssignal(6.5,-1.5){X2}{$X(2)$}
	\psldots[angle=90,ldotssep=0.1](6.5,-2.7)
	\pssignal(6.5,-3.5){Xn1}{$X(n_{1})$}
	\psldots[angle=90,ldotssep=0.1](8.5,-9)
	\pssignal(7,-11.5){Xn1p1}{$X(n_{1} + 1)$}	
	\pssignal(6.5,-13.5){Xn2}{$X(n_{2})$}
	\psldots[angle=90,ldotssep=0.1](8.5,-13)	
	
	\pssignal(11,-1){Z1_1}{$Z_{1}(1)$}
	\pssignal(11,-3){Z1_2}{$Z_{1}(2)$}
	\pssignal(11,-5){Z1_n1}{$Z_{1}(n_{1})$}
	\pssignal(11,-7){Z2_1}{$Z_{2}(1)$}
	\pssignal(11,-9){Z2_2}{$Z_{2}(2)$}
	\pssignal(11,-11){Z2_n1}{$Z_{2}(n_{1})$}
	\pssignal(11.5,-13){Z2_n1p1}{$Z_{2}(n_{1} + 1)$}
	\pssignal(11,-15){Z2_n2}{$Z_{2}(n_{2})$}
	
	\pssignal(12.5,0.5){Y1_1}{$Y_{1}(1)$}
	\pssignal(12.5,-1.5){Y1_2}{$Y_{1}(2)$}
	\pssignal(12.5,-3.5){Y1_n1}{$Y_{1}(n_{1})$}
	\pssignal(12.5,-5.5){Y2_1}{$Y_{2}(1)$}
	\pssignal(12.5,-7.5){Y2_2}{$Y_{2}(2)$}
	\pssignal(12.5,-9.5){Y2_n1}{$Y_{2}(n_{1})$}
	\pssignal(12,-11.5){Y2_n1p1}{$Y_{2}(n_{1} + 1)$}
	\pssignal(12.5,-13.5){Y2_n2}{$Y_{2}(n_{2})$}
	
	
	%%%%%%%%%%%%%%%%%%  Looped Input Dots  %%%%%%%%%%%%%%%%%%
	% Draw decoder 2's input dots
	\cnt=0
	\psforeach{\ry}{0,-1.75,-2.25, -3.75, -4.25, -6}{\advance\cnt by 1\relax
		\dotnode[dotsize=.08](9,\ry){dot_x1_\the\cnt}
	}	
	\cnt=0
	\psforeach{\ry}{-2,-3.75,-4.25,-8}{\advance\cnt by 1\relax
		\dotnode[dotsize=.08](8,\ry){dot_x2_\the\cnt}
	}
	\cnt=0
	\psforeach{\ry}{-4,-10}{\advance\cnt by 1\relax
		\dotnode[dotsize=.08](7,\ry){dot_xn1_\the\cnt}
	}
	
	\ncline[style=Arrow]{S}{enc}
	
	% X1
	\ncarc[arcangle=-50]{dot_x1_2}{dot_x1_3}
	\ncarc[arcangle=-50]{dot_x1_4}{dot_x1_5}
	\ncline{dot_x1_1}{dot_x1_2}
	\ncline{dot_x1_3}{dot_x1_4}
	\ncline{dot_x1_5}{dot_x1_6}

	% X2
	\ncarc[arcangle=-50]{dot_x2_2}{dot_x2_3}
	\ncarc[arcangle=-50]{dot_x2_4}{dot_x2_5}
	\ncline{dot_x2_1}{dot_x2_2}	
	\ncline{dot_x2_3}{dot_x2_4}	
	\ncline{dot_x2_5}{dot_x2_6}	
		
	\ncline{dot_xn1_1}{dot_xn1_2}
	
	\nclist[style=Arrow]{ncline}[naput]{enc_dot1,op1,dec_dot1}	
	\nclist[style=Arrow]{ncline}[naput]{enc_dot2,op2,dec_dot2}
	\nclist[style=Arrow]{ncline}[naput]{enc_dot3,op3,dec_dot3}
	\nclist[style=Arrow]{ncline}[naput]{dot_x1_6,op4,dec_dot4}	
	\nclist[style=Arrow]{ncline}[naput]{dot_x2_4,op5,dec_dot5}	
	\nclist[style=Arrow]{ncline}[naput]{dot_xn1_2,op6,dec_dot6}
	\nclist[style=Arrow]{ncline}[naput]{enc_dot7,op7,dec_dot7}
	\nclist[style=Arrow]{ncline}[naput]{enc_dot8,op8,dec_dot8}
	

	
\end{pspicture}

%\end{document}
%	\includegraphics[scale=0.8]{3/one_sided/fig/system_model_one_sided}
	\includegraphics[scale=1]{one_sided/fig/set_construction}
%%	\includegraphics[width=0.6\textwidth]{outer_bound.png}
	\caption{A tree diagram illustrating the relationship between the sets of source symbols.  Each node represents a set of source symbols and a directed edge $(X,Y)$ indicates that set $Y$ is a subset of $X$.  If edge $(X, Y)$ is also \emph{weighted}, then the weight represents the expected cardinality of set $Y$.  Only sets involved in the coding scheme have incoming \emph{weighted} edges.  The direct successors of a node form a partition for the set representing the parent node.  The root of the tree is the entire source sequence, which is subsequently partitioned at each depth of the tree.  We also show set $\Fset$, which is the union of $\Cset$ and $\BsetOmega$.}
	\label{fig:set_construction}
\end{figure}

In Phase~\Rmnum{2} of the coding scheme, we designate the symbols of $\BsetOmegaC$ as the symbols to be transmitted to the stronger user with a repetition scheme.  However, we modify the repetition scheme to incorporate random linear combinations of symbols in $\Fset$.  In a conventional repetition scheme, we would retransmit $\bOmegaC \in \BsetOmegaC$ until it is received by the stronger user.  Upon reception by the stronger user, we move on to the next symbol $\bOmegaC' \in \BsetOmegaC$ and continue in this manner until all symbols in $\BsetOmegaC$ are accounted for.  Let $\bOmegaT \in \BsetOmegaC$ be the source symbol being repeated at time $t$.  Our modified coding scheme is similar to the conventional repetition scheme except that at any time $t$, instead of \emph{only} transmitting $\bOmegaT$, we instead send $v(t) + \bOmegaT$, where $v(t)$ is a \emph{new} random linear combination of the source symbols in $\Fset$ generated for every time $t$.  Let  $\bOmegaC \in \BsetOmegaC$.  If $\bOmegaT = \bOmegaC$ is transmitted and subsequently received by the stronger user at time $t$, the protocol for replacing $\bOmegaC$ at time $t+ 1$ with another source symbol from $\BsetOmegaC$ is identical to the conventional repetition scheme, however the only difference is that we now combine $\bOmegaT$ with a random linear combination of the symbols of $\Fset$ at every transmission.  Phase~\Rmnum{2} concludes when all symbols in $\BsetOmegaC$ have been accounted for by the modified repetition scheme.

\begin{remark}
	When applying a random linear code, we use the maximum distance separable (MDS)-type property that any collection of $N$ channel symbols gives $N$ linearly independent equations.  Although strictly speaking such codes do not exist over the binary field, randomly chosen combinations over long blocks are approximately MDS~\cite{TTAAJ17}.
%	when we use a random linear code, we will use MDS type property that any collection of N symbols gives N linearly independent equations. Although strictly speaking such codes do not exists over the binary field, randomly chosen combinations over long blocks are approximately MDS. You can cite a reference such as the following one: https://link.springer.com/chapter/10.1007/978-3-319-51103-0_14
\end{remark}

At the conclusion of Phase~\Rmnum{2}, since we have transmitted the symbols in $\BsetOmegaC$ as if we were utilizing a repetition scheme, we have that on average, the stronger user will have received $|\BsetOmegaC|$ equations involving $|\BsetOmegaC| + |\Fset|$ variables.  Notice, however, that since $\Fset \triangleq \Cset \cup \BsetOmega$, and $\Cset \subseteq \Aset$, the stronger user can subtract off all symbols originating from $\Cset$.  Therefore, Phase~\Rmnum{1} actually results in the stronger user receiving $|\BsetOmegaC|$ equations involving $|\BsetOmegaC| + |\BsetOmega| = |\Bset|$ \emph{unkown} variables, where $\mathbb{E}|\Bset| = N(\epsilon_1 - d_1)$.  The stronger user therefore requires an additional $|\BsetOmega|$ equations at the conclusion of Phase~\Rmnum{2}.

In Phase~\Rmnum{3}, we send the remaining equations to the stronger user by continuing to send $v(t)$ at any time $t$.  That is, we continue sending random linear combinations of the symbols in $\Fset$.  Phase~\Rmnum{3} concludes when the feedback of the stronger user indicates that he has received the missing $|\BsetOmega|$ equations.

At the conclusion of Phase~\Rmnum{3}, it is not hard to see that the stronger user achieves point-to-point optimal performance, since every channel symbol received has provided an independent equation that can be used to decode a new source symbol.  
%For $i \in \{1, 2\}$, let $w_{i}(d_i) = (1 - d_i)/(1 - \epsilon_i)$ represent the point-to-point optimal latency for user~$i$.  
At this point, if $\wtwo \leq \wone$, we halt any further transmissions, where $\wid$ is given by~\eqref{eq:wid_optimal}.

In Phase~\Rmnum{4}, if $\wtwo > \wone$, we continue to transmit $v(t)$, the random linear combinations of the source symbols in $\Fset$, for an additional $N(\wtwo- \wone)$ transmissions.





%only symbols providing new information are those originating from $\BsetOmega$
%
% involving $|\Bset|$ symbols, where $\mathbb{E}|\BsetOmegaC| = N(1 - \omegaparam)(\epsilon_1 - d_1)$.  We therefore need to send another $N\omegaparam(\epsilon_1 - d_1)$ equations to the stronger user.  
%
%In Phase~\Rmnum{3}, we send the remaining equations necessary for the stronger user to decode by continuing to send $v(t)$, random linear combinations of the source symbols in $\Fset$, until $N\omega(\epsilon_1 - d_1)$ symbols are received.  
%
%
%
%
%
%we send random linear combinations of the source symbols in $\Fset \triangleq \Cset \cup \BsetOmega$.  Let $v(t)$ be a \emph{new} random linear combination of the source symbols in $\Fset$ generated at time $t$.  Since $\Cset \subseteq \Aset$, the stronger user can subtract off all symbols originating from $\Cset$.  Therefore, the only symbols providing new information are those originating from $\BsetOmega$.  We use the stronger user's feedback to continue to send random linear combinations until the stronger user has received $|\BsetOmega|$ equations, where $\mathbb{E}|\BsetOmega| = \omegaparam(\epsilon_1 - d_1)$.  At the conclusion of Phase~\Rmnum{2}, by construction, the stronger user has decoded all symbols in $\Fset$.
%
%In Phase~\Rmnum{3}, we use a repetition coding scheme to send each symbol $\bOmegaC \in \BsetOmegaC$.  However, since all symbols in $\Fset$ have been decoded by the stronger user at the beginning of Phase~\Rmnum{3}, notice that we can also combine the symbol $\bOmegaC$ with a random linear combination of the symbols in $\Fset$ to the benefit of the weaker user, but at no extra cost to the stronger user.  That is, let $\bOmegaC$ be a symbol that would be sent with a repetition scheme at time $t$.  Then instead of transmitting $\bOmegaC$ at time $t$, we instead transmit $v'(t) = v(t) + \bOmegaC$ where $v(t)$ is a \emph{new} random linear combination of the source symbols in $\Fset$, which can be subtracted off by the stronger user.
%As with a conventional repetition scheme, we also repeat $\bOmegaC$ until it is received by the stronger user for all $\bOmegaC \in \BsetOmegaC$.  However, the difference in Phase~\Rmnum{3} is that we also combine $\bOmegaC$ with $v(t)$, a different linear combination of the source symbols in $\Fset$ at every retransmission.  At the conclusion of Phase~\Rmnum{3} it is not hard to see that the stronger user achieves point-to-point optimal performance, since every channel symbol received has lead to the decoding of a source symbol.  
%
%Finally, in Phase~\Rmnum{4}, if the weaker user has not yet met his distortion constraint, we continue to send random linear combinations of the source symbols in $\Fset$.

%instead of sending the random linear combination $v(t)$ at time $t$ as in Phase~\Rmnum{2}, we instead transmit $v'(t) = v(t) + \bOmegaC$ where $\bOmegaC \in \BsetOmegaC$ is a source symbol that has yet to be received by the stronger user.  Since the stronger user has decoded all symbols in $\Fset$ from the previous phase, he can subtract off $v(t)$ from $v'(t)$ at any time $t$ in Phase~\Rmnum{3}. 




\subsection{Minmax Latency Optimality}
\label{sec:one_sided_optimality}

In this section, we show that it is possible to choose values for $\omegaparam, \gammaparam \in [0, 1]$ from Section~\ref{subsubsec:inner_bound} so that the lower bound for the minmax latency in~\eqref{eq:wplusd} is achieved.  We first calculate the expected number of \emph{unknown} variables involved in transmissions to the weaker user from Phase~\Rmnum{2} onwards.  

First, since we send random linear combinations of the symbols in $\Fset$ in Phase~\Rmnum{2}, we initially expect this to contribute $|\Fset|$ variables.  However, some of the symbols in $\Fset$ have already been received by user~2 in Phase~\Rmnum{1}.  Let $\nFUnknown$ be the number of symbols in $\Fset$ \emph{not} received by user~2 in Phase~\Rmnum{1}.  Given a channel noise realization $(Z_{1}^{W}, Z_{2}^{W}) = (z_{1}^{W}, z_{2}^{W})$, we can calculate the expected value of $\nFUnknown$ as

\setcounter{cnt}{1}
\begin{align}
	\mathbb{E}(\nFUnknown | (Z_{1}^{W}, Z_{2}^{W}) = (z_{1}^{W}, z_{2}^{W})) &= \sum_{s \in \Fset} \textrm{Pr}(\textrm{$s$ not received by user~2 in Phase~\Rmnum{1}})\\
	&\stackrel{(\alph{cnt})}{=} \sum_{s \in \Cset} \textrm{Pr}(\textrm{$s$ not received by user~2 in Phase~\Rmnum{1}}) \\ \nonumber
	&\qquad  + \sum_{s' \in \BsetOmega} \textrm{Pr}(\textrm{$s'$ not received by user~2 in Phase~\Rmnum{1}})\\ 
	\addtocounter{cnt}{1}
	&\stackrel{(\alph{cnt})}{=} \sum_{s \in \Cset} \textrm{Pr}(Z_2 = 1 | Z_1 = 0) + \sum_{s' \in \BsetOmega} \textrm{Pr}(Z_2 = 1 | Z_1 = 1) \\ 
	\addtocounter{cnt}{1}
	&\stackrel{(\alph{cnt})}{=} \sum_{s \in \Cset} \left(\frac{\epsilon_2 - \epsot}{1 - \epsilon_1}\right) + \sum_{s' \in \BsetOmega} \left( \frac{\epsot}{\epsilon_1}\right) \\ 
	\addtocounter{cnt}{1}
%	&\stackrel{(\alph{cnt})}{=}  N\left( \omegaparam(\epsilon_1 - d_1) + \gamma(1 - \epsilon_1) \right)\epsilon_2
	\label{eq:last_line_set}
	&= |\Cset| \left(\frac{\epsilon_2 - \epsot}{1 - \epsilon_1} \right) + |\BsetOmega|\left( \frac{\epsot}{\epsilon_1}\right),
\end{align}
where 
\begin{enumerate}[(a)]
	\item follows from the fact that $\Fset = \Cset \cup \BsetOmega$ and $\Cset$ and $\BsetOmega$ are disjoint by construction
	\item follows from the fact that by construction, all symbols in $\Cset$ have been received by user~1 and all symbols in $\BsetOmega$ were not received by user~1
	\item we have calculated the conditional probabilities from~\eqref{eq:pmf_z1z2}.
\end{enumerate}

The cardinality of sets $\Cset$ and $\BsetOmega$ depends on the channel noise variables $(Z_{1}^{W}, Z_{2}^{W})$.  By taking the expectation over the channel noise, we can calculate the unconditional expected value of $\nFUnknown$ as

\setcounter{cnt}{1}
\begin{align}
	\label{eq:nFUnknown}
	\mathbb{E}\nFUnknown &\stackrel{(\alph{cnt})}{=} N \gammaparam (1 - \epsilon_1) \left(\frac{\epsilon_2 - \epsot}{1 - \epsilon_1} \right) + N \omegaparam(\epsilon_1 - d_1) \left( \frac{\epsot}{\epsilon_1}\right),
\end{align}
where 
\begin{enumerate}[(a)]
	\item follows from~\eqref{eq:last_line_set} and by construction of the sets (see Section~\ref{subsubsec:inner_bound} and Figure~\ref{fig:set_construction}).
\end{enumerate}

%the probability that any symbol in $\Fset$ was not already received by the weaker user in Phase~\Rmnum{1} is equal to $\epsilon_2$.  Therefore the expected number of \emph{unknown} variables within $\Fset$ is given by $\nFUnknown$, where
%
%\setcounter{cnt}{1}
%\begin{align}
%	\mathbb{E}\nFUnknown &= \mathbb{E}|\Fset|\epsilon_2 \\
%	&= \mathbb{E}|\BsetOmega \cup \Cset|\epsilon_2 \\
%	&\stackrel{(\alph{cnt})}{=} \left( \mathbb{E}|\BsetOmega| + \mathbb{E} |\Cset| \right) \epsilon_2	 \\
%	\addtocounter{cnt}{1}
%	\label{eq:nFUnknown}	
%	&\stackrel{(\alph{cnt})}{=}  N\left( \omegaparam(\epsilon_1 - d_1) + \gamma(1 - \epsilon_1) \right)\epsilon_2
%\end{align}
%where 
%\begin{enumerate}[(a)]
%	\item follows from the fact that $\BsetOmegaC$ and $\Cset$ are disjoint by construction
%	\item follows by construction (see Section~\ref{subsubsec:inner_bound} and Figure~\ref{fig:set_construction}).
%\end{enumerate}

The use of repetition coding for the symbols in $\BsetOmegaC$ in Phase~\Rmnum{2} further adds additional unknown variables to the coding scheme.  On average, the expected number of symbols repeated is $\mathbb{E}|\BsetOmegaC| = N(1  - \omegaparam)(\epsilon_1 - d_1)$, of which, again, only a fraction of $\textrm{Pr}(Z_2 = 1 | Z_1 = 1) = \epsot/\epsilon_1$ were not already received by the weaker user in Phase~\Rmnum{1}.  By Lemma~\ref{lem:repetition}, the number of additional \emph{unknown} variables introduced to the weaker user as a result of the repetition scheme is therefore given by $\nBUnknown$, where

\setcounter{cnt}{1}
\begin{align}
	\label{eq:nBUnknown}
	\mathbb{E}\nBUnknown &= N(1  - \omegaparam)(\epsilon_1 - d_1)\left(\frac{\epsot}{\epsilon_1}\right) \left(\frac{1 - \epsilon_2}{1 - \epsot} \right).
\end{align}

Let $\LHSfuncparam$ be the expected fraction of all source symbols that are involved in transmissions to the weaker user from Phase~\Rmnum{2} onwards that have not yet been decoded prior to Phase~\Rmnum{2}.  We have that $\LHSfuncparam$ is the normalized sum of~\eqref{eq:nFUnknown} and~\eqref{eq:nBUnknown}, i.e., 

\begin{align}
	\LHSfuncparam &= \frac{\mathbb{E}\nFUnknown + \mathbb{E}\nBUnknown}{N} \\
	&= \kgamma \gammaparam + \komega \omegaparam	+ \kk,
	\label{eq:LHSfuncparam}
\end{align}
where 
\begin{subequations}
\begin{align}
	\kgamma &= \epsilon_2 - \epsot, \label{eq:kgamma} \\	
	\komega &= (\epsilon_1 - d_1) \left(\frac{\epsot}{\epsilon_1}\right) \left(\frac{\epsilon_2 - \epsot}{1 - \epsot} \right), 	\label{eq:komega}\\
	\kk &= (\epsilon_1 - d_1) \left(\frac{\epsot}{\epsilon_1}\right) \left(\frac{1 - \epsilon_2}{1 - \epsot} \right).
	\label{eq:kk}
\end{align}
\end{subequations}

Having calculated the number of unknown variables sent to the weaker user from Phase~\Rmnum{2} onwards, we now consider the number of equations he receives in Phases~\Rmnum{2} and~\Rmnum{3}.  From Section~\ref{subsubsec:inner_bound}, we know that during these phases, the total number of transmissions was simply equal to the number of trials needed to send $N(\epsilon_1 - d_1)$ equations to the stronger user with feedback.  The number of transmissions in Phases~\Rmnum{2} through~\Rmnum{3}  is therefore distributed according to a negative binomial distribution and the mean number of transmissions in this period is $W_{2,3} = N(\epsilon_1 - d_1)/(1 -\epsilon_1)$.  Of these transmissions, the expected number received by the weaker user is equal to $W_{2, 3}(1 - \epsilon_2)$.  We rewrite the expression for $W_{2, 3}(1 - \epsilon_2)$, the expected number of transmissions received by user~2 in Phases~\Rmnum{2} through~\Rmnum{3}, as $NC_{2, 3}$ where

%We therefore define the capacity of the weaker user's channel during Phases~\Rmnum{2} through~\Rmnum{3} as $C_{2,3}$, where
\begin{align}
	C_{2,3} = \frac{(\epsilon_1 - d_1)(1 - \epsilon_2)}{1 - \epsilon_1}.
%	\bstar &= \frac{(1 - \epsilon_1) + d_1(1 - \epsilon_2)}{1 - \epsilon_1\epsilon_2}
\end{align}
%and the expected number of equations received by the weaker user in Phases~\Rmnum{2} through~\Rmnum{3} is given by $NC_{2, 3}$.  
We next compare $N\LHSfuncparam$, the amount of source symbols destined for the weaker user, with $NC_{2, 3}$,  the expected number of equations received over the weaker user's channel during Phases~\Rmnum{2} and~\Rmnum{3}.  

As mentioned in Section~\ref{subsubsec:inner_bound}, the weaker user requires an additional $N(\epsilon_2 - d_2)$ symbols to be sent from Phase~\Rmnum{2} onwards.  Therefore, it is necessary that $\LHSfuncparam \geq \epsilon_2 - d_2$.  However, if $\LHSfuncparam$ is much greater than $\epsilon_2 - d_2$, we encounter the problem explained in the introduction of this section in which the weaker user is forced to decode unnecessary symbols thus introducing delay.  Say that we are able to find values of $\gammaparam', \omegaparam' \in [0, 1]$ such that $\LHSfuncprime = \epsilon_2 - d_2$.  We consider two cases when this is so -- when $\LHSfuncprime \leq C_{2, 3}$ and when $\LHSfuncprime > C_{2, 3}$.  We show that in both cases, we can achieve the optimal minmax  latency so long as $\LHSfuncprime = \epsilon_2 - d_2$.

In the first case, when $\LHSfuncprime \leq C_{2, 3}$, we wish to send less information over the channel than what the channel can support.  Therefore, we expect that the weaker user should be able to decode all source symbols before the conclusion of Phase~\Rmnum{3}.  However, in general, if the weaker user achieves distortion $d_2$ after decoding, it will be at a latency $w$, where $w > \wtwo$.  That is, in general, the weaker user may not achieve an \emph{individual} point-to-point optimal latency.

To see why this is so, recall from Section~\ref{subsubsec:inner_bound} that in Phase~\Rmnum{2} of our coding scheme, we transmit $\bOmegaT + v(t)$ at every time instant $t$, where $v(t)$ is a new random linear combination of the source symbols in $\Fset$ generated at every time $t$.  Since $\LHSfuncprime \leq C_{2, 3}$, there is the possibility that at some point, the weaker user is able to decode all symbols in $\Fset$ even before Phase~\Rmnum{2} has concluded.  Say that this is the case and the stronger user has stalled on receiving a particular symbol $\bOmegaC \in \BsetOmegaC$ being repeated.  Let us further assume that the weaker user has already received $\bOmegaC$.  Then while $\bOmegaC + v(t)$ is being transmitted, all transmissions to the weaker user are redundant.
%and thus the reception of a channel symbol will not lead to the acquisition of an independent equation that can be used to decode an additional source symbol.  
After $\bOmegaC$ is received by the stronger user and the transmitter moves on to  $\bOmegaPrime$, the next symbol in $\BsetOmegaC$ to be sent via the modified repetition scheme, the weaker user can continue to receive innovative information.  However, the set of transmissions received while $\bOmegaC$ is being repeated prevents the weaker user from achieving an optimal \emph{individual} latency.

However, we show that the optimal \emph{minmax} latency can still be achieved.  Notice that the moment all symbols in $\Fset$ can be decoded by the weaker user, the random linear combination $v(t)$ can be subtracted from any transmission $\bOmegaT + v(t)$ in Phase~\Rmnum{2}.  Therefore the remainder of Phase~\Rmnum{2} effectively consists of uncoded transmissions from the weaker user's perspective, and he is eventually able to decode all $\LHSfuncprime$ symbols.  Thus, so long as $\LHSfuncprime = \epsilon_2 - d_2$, the weaker user will decode the necessary amount of symbols before the conclusion of Phase~\Rmnum{3}, while the stronger user decodes at an optimal latency the moment Phase~\Rmnum{3} terminates.  In this case, the stronger user is the bottleneck of the system and in fact, the condition $\LHSfuncprime \leq C_{2, 3}$ is equivalent to $\wone\geq \wtwo$.  

%For example, say that $\bOmegaPrime$ is the last symbol in $\BsetOmegaC$ about to be sent via the modified repetition coding at time $t_2$.  

%Although the weaker user receives $NC_{2, 3}$ equations during Phases~\Rmnum{2} through~\Rmnum{3}, it may not necessarily be the case that all equations are independent.  Recall from Section~\ref{subsubsec:inner_bound} that in Phase~\Rmnum{2} of our coding scheme, we transmit $\bOmegaT + v(t)$ at every time instant $t$, where $v(t)$ is a new random linear combination of the source symbols in $\Fset$ generated at every time $t$.  There is the possibility that at some point, the weaker user is able to decode all symbols in $\Fset$ even before Phase~\Rmnum{2} has concluded.  

%Notice however, that the moment the weaker user can decode all source symbols in $\Fset$, the only remaining symbols missing are those in $\BsetOmegaC$.  Therefore, during Phase~\Rmnum{2} when $\bOmegaT + v(t)$ is being repeated, the weaker user can subtract off $v(t)$, and can thereby 

%may not necessarily lead to an \emph{independent} equation received.  

%Now, although the weaker user receives $NC_{2, 3}$ equations during Phases~\Rmnum{2} through~\Rmnum{3}, it may not necessarily be the case that all equations are independent.  Recall from Section~\ref{subsubsec:inner_bound} that in Phase~\Rmnum{2} of our coding scheme, we transmit $\bOmegaT + v(t)$ at every time instant $t$, where $v(t)$ is a new random linear combination of source symbols in $\Fset$ generated at every time $t$.  There is the possibility that at some point, the weaker user is able to decode all symbols in $\Fset$ even before Phase~\Rmnum{2} has concluded.  If this is the case and the stronger user has stalled on receiving a particular symbol being repeated that the weaker user has already received, then all transmissions to the weaker user will be redundant, and thus the reception of a channel symbol will not lead to the acquisition of an independent equation that can be used to decode an additional source symbol.  

%we will have received more equations than unknown variables and so the weaker user can decode at the conclusion of Phase~\Rmnum{3}.  In this case, the stronger user is the bottleneck of the system and the condition $\LHSfuncprime \leq C_{2, 3}$ is equivalent to $w_{1}(d_1) \geq w_2(d_2)$.  

On the other hand, if $\LHSfuncprime > C_{2, 3}$, the weaker user has more unknown variables than equations and so he cannot yet decode at the conclusion of Phase~\Rmnum{3}.  However, every transmission he has received so far is ``innovative'' in the sense that it provides an independent equation that can be used to decode the $N\LHSfuncprime$ source symbols.  In order to decode, we simply need to send additional equations to the weaker user, and since $\LHSfuncprime = \epsilon_2 - d_2$, there will not be any unnecessary source symbols sent.  Since the stronger user is point-to-point optimal at the conclusion of Phase~\Rmnum{3}, we have therefore sent a total of $N\wone$ transmissions up to that point.  Since, from the weaker user's perspective, we have hitherto been sending random linear combinations of $N\LHSfuncprime$ variables, we simply need to continue doing so for another $N(\wtwo - \wone)$ transmissions in Phase~\Rmnum{4} before he receives the remaining number of equations required and achieves point-to-point optimal performance.

\begin{table}
	\begin{center}
		\begin{tabular}{| c | c |}
%    \caption{The justification for the ordering}
			\hline
			\multicolumn{2}{|c|}{{\bf Ordering of Boundaries for $d_1$}} \\
			\hline
			{\bf Inequality} & {\bf Justification}   \\ \hline
			$\donedaggall < \doneddaggall$ & $(1 - \epsot)^2 > 0$ \\ \hline 
			$\doneddaggall < \epsilon_1$ & $\epsot < 1$, $\epsilon_1 > 0$ \\ \hline 
			\hline
		\end{tabular}
	\end{center}
	\caption{We justify the ordering of the region boundaries for $d_1$.  In the left column, we have the ordering between two boundary points, and in the right column, we show the necessary and sufficient condition that justifies the ordering.}	
	\label{tab:d1_boundaries}	
\end{table}


We therefore see that regardless of whether $\LHSfuncparam$ is greater or less than $C_{2, 3}$, we can achieve an optimal minmax latency so long as we can find $\gammaparam, \omegaparam \in [0, 1]$ such that $\LHSfuncparam = \epsilon_2 - d_2$.   We focus on finding these values of $\gammaparam$ and $\omegaparam$ in the next sections. In doing so, we consider three cases cases for $d_1$.  
%Let 
%\begin{subequations}
%\begin{align}
%	\label{eq:donedagg}
%	\donedagg &= \frac{\epsilon_1}{\epsot}(2 \epsot - 1) \\
%	\label{eq:doneddagg}
%	\doneddagg &= \frac{\epsilon_1}{\epsot}(2 \epsot - 1) \\
%\end{align}
%\end{subequations}
%
The first is when $0 \leq d_1 \leq \donedaggall$, the second when $\donedaggall < d_1 < \doneddaggall$, and third when $\doneddaggall \leq d_1 < \epsilon_1$.  We justify the position of these boundary points with Table~\ref{tab:d1_boundaries}.  For example, in the first row of Table~\ref{tab:d1_boundaries}, we justify that the boundary point $\donedaggall$ is less than the boundary point $\doneddaggall$ with the necessary and sufficient condition that $(1 - \epsot)^2 > 0$.  

After dividing the values of $d_1$ into regions, we then further consider regions of $d_2/\epsilon_2$, where each region requires a distinct choice for the values of $\gammaparam$ and $\omegaparam$.  We note that from Remark~1 in~\cite{TLKS_TIT16} that for $i \in \{1, 2\}$, we assume that $d_i/\epsilon_i < 1$, otherwise an uncoded transmission strategy can achieve~\eqref{eq:wplusd}.  Therefore, we consider only values of $d_2/\epsilon_2 \in [0, 1]$.  In the following sections, the regions of $\depstwo$ will depend on the boundaries $\cstar$, $\mydstar$, $\astar$ and $\bstar$, which we define as

%Within each case that $d_1$ falls into, we further consider regions for $d_2$ that require separate choices of $\gammaparam$ and $\omegaparam$ so that $\LHSfuncparam = \epsilon_2 - d_2$.  These regions depend on the boundaries $\astar$ and $\bstar$, which we define as

\begin{align}
	\cstar &= \left(\frac{\epsot}{\epsilon_2}\right) \left(\frac{d_1}{\epsilon_1}\right), \\
	\mydstar &= \frac{d_1\epsot + \epsilon_1(\epsilon_2 - \epsot)}{\epsilon_1\epsilon_2}, \\	
	\astar &= \frac{\epsot (d_1 (1 - \epsilon_2) + \epsilon_1(\epsilon_2 - \epsot))}{(1 - \epsot)\epsilon_1 \epsilon_2},\\
	\bstar &= \frac{d_1\epsot(1 - \epsilon_2) + \epsilon_1(\epsilon_2 - \epsot)}{(1 - \epsot) \epsilon_1 \epsilon_2}.
\end{align}

%\begin{align}
%	\astar &= \frac{\epsilon_1\epsilon_2(1 - \epsilon_1) + d_1(1 - \epsilon_2)}{1 - \epsilon_1\epsilon_2}, \\
%	\bstar &= \frac{(1 - \epsilon_1) + d_1(1 - \epsilon_2)}{1 - \epsilon_1\epsilon_2}.
%\end{align}

\input{one_sided/optimality_regions}

