\section{Introduction\label{sec:intro}}

The presence of a feedback channel can have many benefits.  In addition to practical issues such as reducing complexity, it can also increase fundamental communication rates in multiuser networks~(see, e.g.,~\cite{OzarowCheong84,Cover_TIT98}).  In particular, feedback has been shown to increase the \emph{channel capacity} of the erasure broadcast channel~\cite{GGT,WangHan14}.  In contrast, we study a \emph{joint source-channel coding} problem of broadcasting an equiprobable binary source over an erasure broadcast channel with feedback.  Each receiver demands a certain fraction of the source, and we look to minimize the overall transmission time needed to satisfy all user demands.  


We are motivated by heterogeneous broadcast networks, where we encounter users with very different channel qualities, processing abilities, mobility, screen resolutions, etc. In such networks, the user diversity can translate to different distortion requirements from the broadcaster since, e.g., a high-quality reconstruction of a video may not be needed by a user with a  limited-capability device.  In addition, if the source we wish to reconstruct is the output of a multiple description code, then the fraction of the source that is recoverable is of interest.

A related problem has been studied for the case without feedback in~\cite{TLKS_TIT16}.  There are also many other variations of the problem~(see~\cite{LTDM} for a thorough literature review).  A channel coding version of the problem was studied in~\cite{GGT}, which proposed a general algorithm for sending a \emph{fixed} group of messages to $n$ users over an erasure broadcast channel with feedback.  In contrast, our formulation allows flexibility in \emph{which} messages are received at a user so long as the total number received exceeds a certain threshold.  The variant of the index coding problem of \cite{BF_ISIT13} is similar in this respect in that given $n$ users, each already possessing a different subset of $m$ messages, the goal is to minimize the number of transmissions over a \emph{noiseless} channel before \emph{each} user receives \emph{any} additional $t$ messages.  

In our work, we utilize uncoded transmissions that are instantly-decodable~\cite{SorourValaee15}, and distortion-innovative.  The zero latency in decoding uncoded packets has benefits in areas in which packets are instantly useful at their destination such as applications in video streaming and disseminating commands to sensors and robots~\cite{SorourValaee15,SorourValaee10}.  We show that when feedback is available from both users, we can always send instantly-decodable, distortion-innovative transmissions.  While this may not necessarily be the case if a feedback channel is available from only the stronger user, we will show that in this case, the optimal minmax latency can still be achieved by using repetition coding  in tandem with the transmission of random linear combinations. 
%The availability of sending such uncoded transmissions is random and based on the channel noise.  
%Consider the case for two users.  
%Typically, we set up queues that track which symbols are required by which group of users and subsequently find opportunities for network coding based on these queues.  

%There have been two primary techniques in deriving a rate region for a queue-based coding scheme.  The first was derived from a channel coding problem involving erasure channels with feedback and memory~\cite{HeindlmaierBidokhti16}.  In~\cite{HeindlmaierBidokhti16}, the authors use a queue stability criterion to set up a system of inequalities involving the achievable rates and flow variables that represent the number of packets that are transferred between queues.  When this technique is applied to the chaining algorithm of Section~\ref{subsubsec:chaining_algorithm} however, we find that our system of inequalities is over-specified.  That is, the large number of queues leaves us with more equations than unknown variables, and we are unable to solve for a rate region.  
%
%The technique of using a Markov rewards process to analyze the rate region of a queue-based algorithm was also used in~\cite{GeorgiadisTassiulas_Netcod09}.  In their work, the transmitter and receivers were modelled with a Markov rewards process.  The authors solved for the steady-state distribution and were able to derive rate region by considering the number of rewards accumulated per timeslot and thus the number of timeslots it would take to send a group of packets.  In our analysis in Section~\ref{subsubsec:chaining_algorithm} however, we find that our transmitter and receiver must be modelled with an \emph{absorbing} Markov rewards process.  Thus, it is the \emph{transient} behaviour that is important and we cannot use a steady-state analysis.
%
%In Section~\ref{subsubsec:chaining_algorithm}, we modify the analysis for that of an \emph{absorbing} Markov rewards process.  We also formulate a linear program in Section~\ref{subsec:instantly_decodable} to solve for the number of instantly-decodable, distortion innovative transmissions that can be sent.


%where which symbols are still required by which users was determined by the channel after sending systematic uncoded transmissions.
%
%Then if we find that we have a network coding opportunity, e.g., symbol $a_1$ is required 



%In contrast, the problem we propose affords different users the ability to recover a different number of messages during transmission.

%In many practical situations, we find the availability of feedback

%\subsection{Motivation}
%Consumers of video and other content in today's networks have very diverse video and computing equipment, ranging from mobile phones and handheld devices to desktops and HDTVs. Downloaded items range from simple stock quotes to full movies.
%When serving multiple diverse users, the most straightforward approach is to establish independent
%unicast sessions. Here, the advantage is in that each unicast session can be precisely tailored to the user it is set up for, thus making the most efficient use of the individual resources.
%
%However, when a large number of users require the same small content, (e.g., video clips at stadiums), or when a small number of users require the same large content (e.g., a large movie),
%the multiple unicast approach clearly results in highly inefficient use of overall network resources.
%As video traffic continues to grow, efficient use of storage and transmission resources continues to gain importance. Whenever there is interest in the same content by multiple users, traditional unicast-based solutions underperform.
%
%In packet-based data networks, large files are usually segmented
%into smaller blocks that are put into transport packets.
%Packet losses occur because of the physical channel
%and other limitations such as processing power and buffer space.
%In point-to-point scenarios, the sender
%can adjust its transmission/coding rate to avoid packet losses, and
%retransmit lost packets according to the feedback from the
%receiver, through very efficient physical-layer schemes such as HARQ.
%
%In broadcast/multicast applications from a single sender to many receivers,
%however, it is costly for the sender to collect and respond to individual receiver
%feedbacks, and thus HARQ schemes are disabled and packet losses are inevitable. 
%With the rapid increase in multicast streaming applications, we see more and more proposals for packet-level rateless erasure coding. A number of these schemes have already been standardized and are currently being implemented and deployed, e.g., Raptor codes for LTE eMBMS.
%
%\subsection{Prior Work and State of the Art}
%Rateless codes enable efficient communications over multiple, unknown erasure channels, at the packet level, by asymptotically and simultaneously
%achieving the channel capacity at all erasure rates. A special class of rateless codes, known as Raptor codes, has in addition, very low encoding and decoding complexity
%\cite{RCmonograph}. Because of these properties, Raptor codes have been standardized for Multimedia Broadcast/Multicast Service (MBMS), and new technology implementing
%these codes is being deployed for LTE eMBMS. Raptor codes are essentially optimal for multicast over erasure channels where all receivers require identical content. This is not the case in scenarios when some of the users adjust their requirements according to e.g., their screen sizes or (un)favorable channel conditions.
%
%
%\iffalse
%These standards are concerned with scenarios like the above described one, but
%assume uniform demands for all users. Standardized coding schemes start with a time-limited broadcast
%{\it delivery phase} by a content server using systematic Raptor codes. This phase is often too short for all broadcast clients
%to collect a sufficient number of Raptor code symbols to be able to decode. Hence, once the delivery (broadcast) session expires,
%a unicast based file repair mechanism becomes available to the users who had experienced bad channel realizations and were not able to
%decode. These users enter the {\it repair phase,} during which Raptor coded symbols are delivered by dedicated repair servers through unicasts.
%
%\fi
%
%Motivated by these considerations, the problem we study in this paper considers serving users with \emph{individual} demand requirements.  This problem has previously been studied for the case in which there are only two receivers, from two different perspectives~\cite{Li2012b,TLKS_ITW13}.  In~\cite{Li2012b}, an optimization problem was formulated to find the degree distribution of a rateless code that minimizes the overall latency before all users can satisfy their individual demands.  In~\cite{TLKS_ITW13}, (see also~\cite{HN}), the problem was formulated as a joint source-channel coding problem for sending an equiprobable binary source over an erasure broadcast channel subject to an erasure distortion criterion.  It was observed that while the joint source-channel coding approach requires a more sophisticated Wyner-Ziv coding scheme, it results in a lower latency compared to the rateless coding scheme in the two-user setup. In this work, we propose a segmentation-based scheme that achieves the same latency as the Wyner-Ziv coding scheme but has certain advantages. First it uses only off-the-shelf systematic erasure codes rather than a joint source-channel code. Second it can naturally be adjusted as users are added or deleted from the system and scales to an arbitrary number of users. 
%%The two approaches were compared in~\cite{}.
%
%
%\iffalse
%%\subsection{Our Contribution and Paper Organization} 
%%In this work, we offer a scalable coding scheme for broadcasting to multiple users of non-uniform distortion demands over non-uniform erasure channels. 
%In the rest of the paper, Sec.~\ref{sec:system_model} defines the multi-user source-broadcast problem. Sec.~\ref{sec:idea} motivates our scalable coding solution to the broadcast problem. We start from the single-user scenario, and intuitively see how a coding solution can be modified to accommodate additional users. In Sec.~\ref{sec:solution}, we present our main contribution: a segmentation-based coding scheme for the multi-user source-broadcast problem. We also describe how to conveniently modify the segment lengths to accommodate the addition or removal of a user. Sec.~\ref{sec:numerical_comparisons} provides numerical comparisons of the proposed scheme with a few baseline schemes. 
%\fi