\pdfoutput=1
%\documentclass[journal,draftclsnofoot,onecolumn,12pt,twoside]{IEEEtranTCOM}
\documentclass[journal,onecolumn,12pt,twoside]{IEEEtranTCOM}
%\documentclass[conference]{IEEEtran}
%\documentclass[peerreview]{IEEEtran}
%\documentclass[journal]{IEEEtran}
\usepackage{etex}
%\usepackage{appendix}

\normalsize

% Some very useful LaTeX packages include:
%
\usepackage{cite}
\usepackage{setspace}
%\usepackage[nomessages]{fp}


% *** GRAPHICS RELATED PACKAGES ***
%
\usepackage{graphicx}
\usepackage{color}
%\usepackage{pstricks}
\usepackage{pst-sigsys}
\ifCLASSINFOpdf
  % \usepackage[pdftex]{graphicx}
  % declare the path(s) where your graphic files are
  % \graphicspath{{../pdf/}{../jpeg/}}
  % and their extensions so you won't have to specify these with
  % every instance of \includegraphics
  % \DeclareGraphicsExtensions{.pdf,.jpeg,.png}
\else
  % or other class option (dvipsone, dvipdf, if not using dvips). graphicx
  % will default to the driver specified in the system graphics.cfg if no
  % driver is specified.
  % \usepackage[dvips]{graphicx}
  % declare the path(s) where your graphic files are
  % \graphicspath{{../eps/}}
  % and their extensions so you won't have to specify these with
  % every instance of \includegraphics
  % \DeclareGraphicsExtensions{.eps}
\fi

% graphicx was written by David Carlisle and Sebastian Rahtz. It is
% required if you want graphics, photos, etc. graphicx.sty is already
% installed on most LaTeX systems. The latest version and documentation can
% be obtained at:
% http://www.ctan.org/tex-archive/macros/latex/required/graphics/
% Another good source of documentation is "Using Imported Graphics in
% LaTeX2e" by Keith Reckdahl which can be found as epslatex.ps or
% epslatex.pdf at: http://www.ctan.org/tex-archive/info/
%
% latex, and pdflatex in dvi mode, support graphics in encapsulated
% postscript (.eps) format. pdflatex in pdf mode supports graphics
% in .pdf, .jpeg, .png and .mps (metapost) formats. Users should ensure
% that all non-photo figures use a vector format (.eps, .pdf, .mps) and
% not a bitmapped formats (.jpeg, .png). IEEE frowns on bitmapped formats
% which can result in "jaggedy"/blurry rendering of lines and letters as
% well as large increases in file sizes.
%
% You can find documentation about the pdfTeX application at:
% http://www.tug.org/applications/pdftex
%\usepackage[textwidth=6.5cm]{geometry}




% *** MATH PACKAGES ***
%
\usepackage[cmex10]{amsmath}
\usepackage{amsmath,dsfont}
\usepackage{verbatim}
%  \usepackage{hyperref}
%\usepackage{nccmath}
%\usepackage[fleqn]{amsmath}
\usepackage{amssymb}
\usepackage{amsthm}
\usepackage{nccmath}
\usepackage{enumerate}
%\usepackage{tikz}
  \usepackage{pgfplots}
%  \usepackage{soul}
  \usepackage[normalem]{ulem}
  \usepackage{cases}
  \usepackage{bbm}  
%\usetikzlibrary{plotmarks}
%  \pgfplotsset{compat=newest} 
\pgfplotsset{plot coordinates/math parser=false}
%\usepackage{verbatim}
%\usepackage{graphicx,amssymb,amstext,amsmath,cite,amsthm}
%\usepackage{amstext}
\newtheorem{mydef}{Definition}
\newtheorem{definition}{Definition}
\newtheorem{theorem}{Theorem}
\newtheorem{lemma}[theorem]{Lemma}
\newtheorem{claim}[theorem]{Claim}
\newtheorem{corollary}[theorem]{Corollary}
\newtheorem{remark}{Remark}
\newcommand{\Rmnum}[1]{\MakeUppercase{\romannumeral #1}}
\newcommand{\mybfs}[1]{\ensuremath{\mathbf{S}_{#1}}}
\newcommand{\mybfx}[1]{\ensuremath{\mathbf{X}_{#1}}}
\newcommand{\mybfp}[1]{\ensuremath{\mathbf{P}_{#1}}}
\DeclareMathOperator*{\argmin}{arg\,min}



\newcommand{\nRepi}[1]{\ensuremath{M_{#1}}}
\newcommand{\vecnRep}{\ensuremath{\underline{M}}}
\newcommand{\indi}[1]{\ensuremath{\mathbbm{1}_{#1}}}

%\newcommand{\Nretransmit}{\ensuremath{\bar{N}_{0}}}
\newcommand{\Nretransmit}{\ensuremath{N_{0}}}
\newcommand{\ENretransmit}{\ensuremath{\mathbb{E}\Nretransmit}}

%\newcommand{\astar}{\ensuremath{\hat{a}}}
%\newcommand{\bstar}{\ensuremath{\hat{b}}}
\newcommand{\astar}{\ensuremath{c^{\dagger}}}
\newcommand{\bstar}{\ensuremath{d^{\dagger}}}
\newcommand{\cstar}{\ensuremath{a^{\dagger}}}
\newcommand{\mydstar}{\ensuremath{b^{\dagger}}}

\newcommand{\omegaparam}{\ensuremath{\theta}}
\newcommand{\gammaparam}{\ensuremath{\gamma}}

\newcommand{\omegastar}{\ensuremath{\omegaparam^{*}}}
\newcommand{\gammahat}{\ensuremath{\hat{\gammaparam}}}
\newcommand{\gammatilde}{\ensuremath{\gammaparam^{*}}}

\newcommand{\Fset}{\ensuremath{F}}
\newcommand{\Cset}{\ensuremath{C}}
\newcommand{\Aset}{\ensuremath{A}}
\newcommand{\AsetC}{\ensuremath{A^{\complement}}}
\newcommand{\Bset}{\ensuremath{B}}
\newcommand{\BsetOmega}{\ensuremath{B_{\omegaparam}}}
\newcommand{\BsetOmegaC}{\ensuremath{B_{\overline{\omegaparam}}}}
%\newcommand{\bOmegaC}{\ensuremath{b_{\overline{\omegaparam}}}}
\newcommand{\bOmegaC}{\ensuremath{b}}
\newcommand{\bOmegaPrime}{\ensuremath{b'}}
\newcommand{\bOmegaT}{\ensuremath{b_{\overline{\omegaparam}}(t)}}

\newcommand{\nFUnknown}{\ensuremath{N_{1}}}
\newcommand{\nBUnknown}{\ensuremath{N_{2}}}

\newcommand{\LHSfunc}{\ensuremath{\mathcal{L}}}
\newcommand{\LHSfuncparam}{\ensuremath{\LHSfunc(\gammaparam, \omegaparam)}}
\newcommand{\LHSfuncprime}{\ensuremath{\LHSfunc(\gammaparam', \omegaparam')}}

\newcommand{\kgamma}{\ensuremath{k_1}}
\newcommand{\komega}{\ensuremath{k_2}}
\newcommand{\kk}{\ensuremath{k_3}}

\newcommand{\wplusd}{\ensuremath{w^{+}(d_1, d_2)}}
\newcommand{\wid}{\ensuremath{w_{i}^{*}(d_i)}}
\newcommand{\wi}{\ensuremath{w_{i}^{*}}}
\newcommand{\wtwo}{\ensuremath{w_{2}^{*}(d_2)}}
\newcommand{\wone}{\ensuremath{w_{1}^{*}(d_1)}}

\newcommand{\depstwo}{\ensuremath{d_2/\epsilon_2}}
\newcommand{\depsone}{\ensuremath{d_1/\epsilon_1}}

\newcommand{\epsot}{\ensuremath{\epsilon_{12}}}

\newcommand{\donedagg}{\ensuremath{d_{1}^{\dagger}}}
\newcommand{\doneddagg}{\ensuremath{d_{1}^{\ddagger}}}

\newcommand{\donedaggall}{\ensuremath{\epsilon_1 (2\epsot - 1)/\epsot}}
\newcommand{\doneddaggall}{\ensuremath{\epsilon_1\epsot}}

\newlength\figureheight 
\newlength\figurewidth 

%% -------------------------------------------------------------
%% *** Converse Chapter Theorems ***
%% -------------------------------------------------------------
\newtheorem{myTheorem}{Theorem}
%\newtheorem{mydef}{Definition}
\newtheorem{myLemma}{Lemma}
\newtheorem{myExample}{Example}
\newtheorem{myprop}{Proposition}
\newtheorem{mycor}{Corollary}

\newcounter{cnt}
\newcommand{\bof}[1]{\textbf{#1}}
\newcommand{\YN}[1]{\ensuremath{\tilde{Y}_{#1}^{n}}}
\newcommand{\yN}[1]{\ensuremath{\tilde{y}_{#1}^{n}}}
\newcommand{\tYN}[1]{\ensuremath{\tilde{Y}_{#1}^{n}}}
\newcommand{\tyN}[1]{\ensuremath{\tilde{y}_{#1}^{n}}}
\newcommand{\nN}[1]{\ensuremath{\tilde{n}_{#1}^{n}}}
\newcommand{\NN}[1]{\ensuremath{\tilde{N}_{#1}^{n}}}
\newcommand{\SHatK}[1]{\ensuremath{\hat{S}_{#1}^{m}}}
%\newcommand{\setI}[1]{\ensuremath{\mathcal{I}(n_{#1}^{n})}}
\newcommand{\setI}[1]{\ensuremath{I_{#1}}}
\newcommand{\setInN}{\ensuremath{\mathcal{I}(\etaN)}}
\newcommand{\tSM}[1]{\ensuremath{\tilde{S}_{#1}^m}}
\newcommand{\setURi}{\ensuremath{\mathcal{A}_i}}
\newcommand{\setUR}{\ensuremath{\mathcal{A}}}
\newcommand{\UN}{\ensuremath{U^n}}
\newcommand{\uN}{\ensuremath{u^n}}
\newcommand{\setIB}[1]{\ensuremath{\mathcal{I}_{#1}(u^n)}}
\newcommand{\setIBH}[1]{\ensuremath{\mathcal{I}_{#1}(\hat{u}^n)}}
\newcommand{\Prob}{\ensuremath{\operatorname{Pr}}}
\newcommand{\Dstar}[1]{\ensuremath{D^{*}(\epsilon_{#1}; b)}}
%\newcommand{\XC}{\ensuremath{X_{\mathcal{\overline{C}(\nN{i})}}}}
\newcommand{\oCn}{\ensuremath{\overline{\mathcal{C}}(\nN{i}, \nN{j})}}
\newcommand{\Cn}{\ensuremath{\mathcal{C}(\nN{i}, \nN{j})}}
\newcommand{\XC}{\ensuremath{X_{\oCn}}}
\newcommand{\tUi}[1]{\ensuremath{\tilde{U}_{#1}}}
\newcommand{\XCn}{\ensuremath{X_{\oCn}}}
\newcommand{\Keps}{\ensuremath{K(\epsilon_1, \epsilon_2; b)}}
\newcommand{\sethV}{\ensuremath{\mathcal{\hat{V}}(p^n)}}

\newcommand{\setV}{\ensuremath{\mathcal{V}}}
\newcommand{\setU}{\ensuremath{\mathcal{U}}}
\newcommand{\setA}{\ensuremath{\mathcal{A}}}
\newcommand{\setB}{\ensuremath{\mathcal{B}}}
\newcommand{\tUNn}{\ensuremath{\tilde{U}^n(\pN)}}
\newcommand{\tUN}{\ensuremath{\tilde{U}^n}}
\newcommand{\etaN}{\ensuremath{\eta^n}}
\newcommand{\tUNPN}{\ensuremath{\hUN}}

\newcommand{\tqN}{\ensuremath{\tilde{q}^n}}
\newcommand{\tQN}{\ensuremath{\tilde{Q}^n}}
\newcommand{\tQi}[1]{\ensuremath{\tilde{Q}_{#1}}}


\newcommand{\setIpN}{\ensuremath{\mathcal{I}(p^n)}}
\newcommand{\setIqN}{\ensuremath{\mathcal{I}(q^n)}}
\newcommand{\hpB}{\ensuremath{p(\setB; \hat{q})}}
\newcommand{\pq}{\ensuremath{p(q^n; \hat{q})}}
\newcommand{\pA}{\ensuremath{p({\setA; \hat{p})}}}
\newcommand{\pI}{\ensuremath{p({\setInN; \hat{p})}}}
\newcommand{\pN}{\ensuremath{p^n}}
\newcommand{\PN}{\ensuremath{P^n}}
\newcommand{\setW}[1]{\ensuremath{\mathcal{W}_{#1}(\uN)}}
\newcommand{\setWU}[1]{\ensuremath{\mathcal{W}_{#1}(\hUN)}}
\newcommand{\hUN}{\ensuremath{\hat{U}^n}}
\newcommand{\hUi}[1]{\ensuremath{\hat{U}_{#1}}}

%\newcommand{\oCn}{\ensuremath{\overline{\mathcal{C}}(\nN{i}, \nN{j})}}
%\newcommand{\Cn}{\ensuremath{\mathcal{C}(\nN{i}, \nN{j})}}
%\newcommand{\XC}{\ensuremath{X_{\oCn}}}

\newcommand{\YS}{\ensuremath{Y_{1}^n}}
\newcommand{\YW}{\ensuremath{Y_{2}^n}}
\newcommand{\NW}{\ensuremath{N_{2}^n}}
\newcommand{\NS}{\ensuremath{N_{1}^n}}
\newcommand{\nS}{\ensuremath{\eta_{1}^n}}


%\input{def_header}
%% ABBREVIATIONS FOR CHARACTERS IN VARIOUS FONTS

% STANDARD CHARACTERS

\newcommand{\ba}{{\mathbf{a}}}
\newcommand{\bah}{{\hat{\ba}}}
\newcommand{\ah}{{\hat{a}}}
\newcommand{\Ah}{{\hat{A}}}
\newcommand{\cA}{{\mathcal{A}}}
\newcommand{\at}{{\tilde{a}}}
\newcommand{\bat}{{\tilde{\ba}}}
\newcommand{\At}{{\tilde{A}}}
\newcommand{\bA}{{\mathbf{A}}}
\newcommand{\ac}{a^{\ast}}

\newcommand{\bb}{{\mathbf{b}}}
\newcommand{\bbt}{{\tilde{\bb}}}
\newcommand{\cB}{{\mathcal{B}}}
\newcommand{\tb}{{\tilde{b}}}
\newcommand{\tB}{{\tilde{B}}}
\newcommand{\hb}{{\hat{b}}}
\newcommand{\hB}{{\hat{B}}}
\newcommand{\bB}{{\mathbf{B}}}

\newcommand{\bc}{{\mathbf{c}}}
\newcommand{\bch}{{\hat{\mathbf{c}}}}
\newcommand{\bC}{{\mathbf{C}}}
\newcommand{\cC}{{\mathcal{C}}}
\newcommand{\ct}{{\tilde{c}}}
\newcommand{\Ct}{{\tilde{C}}}
\newcommand{\ctc}{\ct^{\ast}}

\newcommand{\bd}{{\mathbf{d}}}
\newcommand{\bD}{{\mathbf{D}}}
\newcommand{\cD}{{\mathcal{D}}}
\newcommand{\hd}{{\hat{d}}}  % old: \dh
\newcommand{\dt}{{\tilde{d}}}
\newcommand{\bdt}{{\tilde{\bd}}}
\newcommand{\Dt}{{\tilde{D}}}
\newcommand{\dtc}{\dt^{\ast}}

\newcommand{\et}{{\tilde{e}}}
\newcommand{\bfe}{{\mathbf{e}}}
\newcommand{\bE}{{\mathbf{E}}}
\newcommand{\cE}{{\mathcal{E}}}
\newcommand{\cEt}{{\tilde{\cE}}}
\newcommand{\cEb}{{\bar{\cE}}}
\newcommand{\bcE}{{\mathbf{\cE}}}  % bf cal E doesn't exist

\newcommand{\bff}{{{\mathbf{f}}}}
\newcommand{\bF}{{\mathbf{F}}}
\newcommand{\cF}{{\mathcal{F}}}
\newcommand{\ft}{{\tilde{f}}}
\newcommand{\Ft}{{\tilde{F}}}
\newcommand{\Fh}{{\hat{F}}}
\newcommand{\ftc}{\ft^{\ast}}
\newcommand{\bft}{{\tilde{\bff}}}
\newcommand{\bFt}{{\tilde{\bF}}}
\newcommand{\fh}{{\hat{f}}}

\newcommand{\bg}{{\mathbf{g}}}
\newcommand{\gt}{{\tilde{g}}}
\newcommand{\bgt}{{\tilde{\bg}}}
\newcommand{\bG}{{\mathbf{G}}}
\newcommand{\cG}{{\mathcal{G}}}
\newcommand{\Gt}{{\tilde{\bG}}}
\newcommand{\Ge}{{G_\mathrm{eff}}}


\newcommand{\hti}{{\tilde{h}}}
\newcommand{\Hti}{{\tilde{H}}}
\newcommand{\bh}{{\mathbf{h}}}
\newcommand{\bht}{{\tilde{\bh}}}
\newcommand{\Hh}{{\hat{H}}}
\newcommand{\bH}{{\mathbf{H}}}
\newcommand{\bHh}{{\hat{\mathbf{H}}}}

\newcommand{\ih}{{\hat{\imath}}}
\newcommand{\bI}{{\mathbf{I}}}
\newcommand{\cI}{{\mathcal{I}}}

\newcommand{\jh}{{\hat{\jmath}}}
\newcommand{\bJ}{{\mathbf{J}}}
\newcommand{\cJ}{{\mathcal{J}}}
\newcommand{\Jt}{{\tilde{J}}}

\newcommand{\bk}{{\mathbf{k}}}
\newcommand{\bK}{{\mathbf{K}}}
\newcommand{\Kt}{{\tilde{K}}}
\newcommand{\Kh}{{\hat{K}}}
\newcommand{\cK}{{\mathcal{K}}}

\newcommand{\cl}{\ell}
\newcommand{\bL}{{\mathbf{L}}}
\newcommand{\cL}{{\mathcal{L}}}

\newcommand{\mb}{{\mathbf{m}}}
\newcommand{\mh}{{\hat{m}}}
\newcommand{\bM}{{\mathbf{M}}}
\newcommand{\bm}{{\mathbf{m}}}
\newcommand{\cM}{{\mathcal{M}}}


\newcommand{\cN}{{\mathcal{N}}}
\newcommand{\CN}{{\mathcal{CN}}}
\newcommand{\Nt}{{\tilde{N}}}
\newcommand{\tN}{{\tilde{N}}}  % backward compatibility

\newcommand{\bo}{{\mathbf{o}}}
\newcommand{\cO}{{\mathcal{O}}}

\newcommand{\bp}{{\mathbf{p}}}
\newcommand{\bP}{{\mathbf{P}}}
\newcommand{\cP}{{\mathcal{P}}}
\newcommand{\ph}{{\hat{p}}}
\newcommand{\Ph}{{\hat{P}}}
\newcommand{\Pt}{{\tilde{P}}}
\newcommand{\Ptt}{{\tilde{\tilde{P}}}}
\newcommand{\bq}{{\mathbf{q}}}
\newcommand{\cQ}{{\mathcal{Q}}}
\newcommand{\bQ}{{\mathbf{Q}}}

\newcommand{\br}{{\mathbf{r}}}
\newcommand{\bR}{{\mathbf{R}}}
\newcommand{\cR}{{\mathcal{R}}}
\newcommand{\Rt}{{\tilde{R}}}

\newcommand{\sh}{{\hat{s}}}
\newcommand{\sck}{{\check{s}}}
\newcommand{\shh}{{\Hat{\Hat{s}}}}
\newcommand{\bs}{{\mathbf{s}}}
\newcommand{\bsh}{{\hat{\mathbf{s}}}}
\newcommand{\bsc}{{\check{\mathbf{s}}}}
\newcommand{\bshh}{{\Hat{\Hat{\mathbf{s}}}}}
\newcommand{\bS}{{\mathbf{S}}}
\newcommand{\cS}{{\mathcal{S}}}
\newcommand{\st}{{\tilde{s}}}

\newcommand{\bT}{{\mathbf{T}}}
\newcommand{\cT}{{\mathcal{T}}}
\newcommand{\bu}{{\mathbf{u}}}
\newcommand{\bU}{{\mathbf{U}}}
\newcommand{\bUt}{{\tilde{\bU}}}
\newcommand{\ut}{{\tilde{u}}}
\newcommand{\cU}{{\mathcal{U}}}

\newcommand{\vh}{{\hat{v}}}
\newcommand{\bv}{{\mathbf{v}}}
\newcommand{\bV}{{\mathbf{V}}}
\newcommand{\cV}{{\mathcal{V}}}

\newcommand{\bw}{{\mathbf{w}}}
\newcommand{\bW}{{\mathbf{W}}}
\newcommand{\cW}{{\mathcal{W}}}
\newcommand{\wt}{{\tilde{w}}}

\newcommand{\bx}{{\mathbf{x}}}
\newcommand{\bxt}{{\tilde{\bx}}}
\newcommand{\xt}{{\tilde{x}}}
\newcommand{\Xt}{{\tilde{X}}}
\newcommand{\bX}{{\mathbf{X}}}
\newcommand{\cX}{{\mathcal{X}}}
\newcommand{\bXt}{{\tilde{\bX}}}
\newcommand{\xh}{{\hat{x}}}
\newcommand{\xc}{{\check{x}}}
\newcommand{\xhh}{{\Hat{\Hat{x}}}}
\newcommand{\bxh}{{\hat{\bx}}}
\newcommand{\bxc}{{\check{\bx}}}
\newcommand{\bxhh}{{\Hat{\hat{\bx}}}}

\newcommand{\cY}{{\mathcal{Y}}}
\newcommand{\by}{{\mathbf{y}}}
\newcommand{\byt}{{\tilde{\by}}}
\newcommand{\bY}{{\mathbf{Y}}}
\newcommand{\Yt}{{\tilde{Y}}}
\newcommand{\yt}{{\tilde{y}}}
\newcommand{\yh}{{\hat{y}}}

\newcommand{\zt}{{\tilde{z}}}
\newcommand{\zh}{{\hat{z}}}
\newcommand{\bz}{{\mathbf{z}}}
\newcommand{\bZ}{{\mathbf{Z}}}
\newcommand{\cZ}{{\mathcal{Z}}}

% GREEK CHARACTERS

\newcommand{\al}{\alpha}
\newcommand{\bal}{{\boldsymbol{\al}}}
\newcommand{\balh}{{\hat{\boldsymbol{\al}}}}
\newcommand{\alh}{{\hat{\al}}}
\newcommand{\aln}{{\bar{\al}}}

\newcommand{\bt}{\boldsymbol{t}}
%\newcommand{\bt}{\beta}
\newcommand{\btt}{{\tilde{\bt}}}
\newcommand{\btht}{{\hat{\bt}}}

\newcommand{\g}{\gamma}
\newcommand{\G}{\Gamma}
\newcommand{\bGa}{{\boldsymbol{\Gamma}}}
\newcommand{\gh}{{\hat{\g}}}

\newcommand{\de}{\delta}
\newcommand{\del}{\delta}
\newcommand{\De}{\Delta}
\newcommand{\Deh}{{\hat{\Delta}}}
\newcommand{\bde}{{\boldsymbol{\de}}}
\newcommand{\bDe}{{\boldsymbol{\De}}}

\newcommand{\e}{\epsilon}
\newcommand{\eps}{\varepsilon}

\newcommand{\etah}{{\hat{\eta}}}
\newcommand{\bpi}{{\boldsymbol{\pi}}}

\newcommand{\pht}{{\tilde{\phi}}}
\newcommand{\Pht}{{\tilde{\Phi}}}

\newcommand{\pst}{{\tilde{\psi}}}
\newcommand{\Pst}{{\tilde{\Psi}}}

\newcommand{\s}{\sigma}
\newcommand{\sih}{\hat{\sigma}}

\newcommand{\z}{\zeta}
\newcommand{\ztt}{{\tilde{\z}}}
\newcommand{\ztb}{{\bar{\z}}}

% \newcommand{\th}{\theta} % symbol name used by other latex package
\newcommand{\thh}{{\hat{\theta}}}
\newcommand{\Thh}{{\hat{\Theta}}}
\newcommand{\Th}{\Theta}
\newcommand{\bth}{{\boldsymbol{\theta}}}
\newcommand{\bTh}{{\boldsymbol{\Theta}}}
\newcommand{\bThh}{{\hat{\bTh}}}
\newcommand{\Tht}{{\tilde{\Theta}}}

\newcommand{\la}{\lambda}
\newcommand{\La}{\Lambda}
\newcommand{\lam}{\lambda}  % backward compatibility
\newcommand{\Lam}{\Lambda}  % backward compatibility
\newcommand{\bLa}{{\boldsymbol{\La}}}
\newcommand{\lah}{{\hat{\lam}}}

\newcommand{\bmu}{{\boldsymbol{\mu}}}

\newcommand{\bXi}{{\boldsymbol{\Xi}}}
%\newcommand{\rvx}{{\mathsf{x}}}
%\newcommand{\rvy}{{\mathrm{y}}}
\newcommand{\bPi}{{\boldsymbol{\Pi}}}

\newcommand{\rht}{{\tilde{\rho}}}
\newcommand{\rhc}{{\check{\rho}}}

\newcommand{\bSi}{{\boldsymbol{\Sigma}}}

\newcommand{\ups}{\upsilon}
\newcommand{\Ups}{\Upsilon}
\newcommand{\bUp}{{\boldsymbol{\Ups}}}

\newcommand{\bPs}{{\boldsymbol{\Psi}}}

\newcommand{\w}{\omega}
\newcommand{\wh}{{\hat{\omega}}}
\newcommand{\W}{\Omega}

\newcommand{\dagg}{\dagger}
\newcommand{\dagbh}{{\mathbf{h}}^\dagger}
\newcommand{\dagbH}{{\mathbf{H}}^\dagger}

%\DeclareMathAlphabet{\mathbsf}{OT1}{cmss}{bx}{n}% bold sans serif
\DeclareMathAlphabet{\mathssf}{OT1}{cmss}{m}{sl}% slanted sans serif

% define some useful uppercase Greek letters in regular and bold sf
\DeclareSymbolFont{bsfletters}{OT1}{cmss}{bx}{n}
\DeclareSymbolFont{ssfletters}{OT1}{cmss}{m}{n}
\DeclareMathSymbol{\bsfGamma}{0}{bsfletters}{'000}
\DeclareMathSymbol{\ssfGamma}{0}{ssfletters}{'000}
\DeclareMathSymbol{\bsfDelta}{0}{bsfletters}{'001}
\DeclareMathSymbol{\ssfDelta}{0}{ssfletters}{'001}
\DeclareMathSymbol{\bsfTheta}{0}{bsfletters}{'002}
\DeclareMathSymbol{\ssfTheta}{0}{ssfletters}{'002}
\DeclareMathSymbol{\bsfLambda}{0}{bsfletters}{'003}
\DeclareMathSymbol{\ssfLambda}{0}{ssfletters}{'003}
\DeclareMathSymbol{\bsfXi}{0}{bsfletters}{'004}
\DeclareMathSymbol{\ssfXi}{0}{ssfletters}{'004}
\DeclareMathSymbol{\bsfPi}{0}{bsfletters}{'005}
\DeclareMathSymbol{\ssfPi}{0}{ssfletters}{'005}
\DeclareMathSymbol{\bsfSigma}{0}{bsfletters}{'006}
\DeclareMathSymbol{\ssfSigma}{0}{ssfletters}{'006}
\DeclareMathSymbol{\bsfUpsilon}{0}{bsfletters}{'007}
\DeclareMathSymbol{\ssfUpsilon}{0}{ssfletters}{'007}
\DeclareMathSymbol{\bsfPhi}{0}{bsfletters}{'010}
\DeclareMathSymbol{\ssfPhi}{0}{ssfletters}{'010}
\DeclareMathSymbol{\bsfPsi}{0}{bsfletters}{'011}
\DeclareMathSymbol{\ssfPsi}{0}{ssfletters}{'011}
\DeclareMathSymbol{\bsfOmega}{0}{bsfletters}{'012}
\DeclareMathSymbol{\ssfOmega}{0}{ssfletters}{'012}

\newcommand{\fxfm}{\stackrel{\mathcal{F}}{\longleftrightarrow}}
\newcommand{\lxfm}{\stackrel{\mathcal{L}}{\longleftrightarrow}}
\newcommand{\zxfm}{\stackrel{\mathcal{Z}}{\longleftrightarrow}}

\DeclareMathOperator*{\gltop}{\gtreqless}
\newcommand{\glt}{\;\gltop^{\Hh=\svH_1}_{\Hh=\svH_0}\;}
\newcommand{\glty}{\;\gltop^{\Hh(\svy)=\svH_1}_{\Hh(\svy)=\svH_0}\;}
\newcommand{\gltby}{\;\gltop^{\Hh(\svby)=\svH_1}_{\Hh(\svby)=\svH_0}\;}
\DeclareMathOperator*{\geltop}{\genfrac{}{}{0pt}{}{\ge}{<}}
\newcommand{\gelty}{\;\geltop^{\Hh(\svy)=\svH_1}_{\Hh(\svy)=\svH_0}\;}
\newcommand{\geltby}{\;\geltop^{\Hh(\svby)=\svH_1}_{\Hh(\svby)=\svH_0}\;}
\renewcommand{\pe}{\Pr(e)}
\renewcommand{\defeq}{\triangleq}
\newcommand{\like}{\svlike}
\newcommand{\rvlike}{\mathssf{L}}
\newcommand{\sst}{\cl}
\newcommand{\svlike}{L}
\newcommand{\llike}{\rvllike}
\newcommand{\rvllike}{\cl}
\newcommand{\svllike}{l}
\newcommand{\bllike}{\rvbllike}
\newcommand{\rvbllike}{\boldsymbol{\cl}}
\newcommand{\svbllike}{\mathbf{l}}
\newcommand{\Qb}{\overline{Q}}
\renewcommand{\comb}[2]{\binom{#1}{#2}}


%% Random/sample variable/vector declarations.  Please add in alphabetical
%% order.  First section is for capitals.  Second for lower case.
% Capitals
\newcommand{\rvA}{{\mathssf{A}}}    % A
\newcommand{\svA}{A}
\newcommand{\rvbA}{{\mathbsf{A}}}
\newcommand{\svbA}{{\mathbf{A}}}
\newcommand{\rvD}{{\mathssf{D}}}    % D
\newcommand{\svD}{D}
\newcommand{\rvbD}{{\mathbsf{D}}}
\newcommand{\svbD}{{\mathbf{D}}}
\newcommand{\rvFh}{{\hat{\mathssf{F}}}} % F
\newcommand{\rvF}{{\mathssf{F}}}
\newcommand{\rvHh}{{\hat{\mathssf{H}}}} % H
\newcommand{\rvH}{{\mathssf{H}}}
\newcommand{\svH}{H}
\newcommand{\svHh}{{\hat{\svH}}}
\newcommand{\rvL}{{\mathssf{L}}}    % L
\newcommand{\rvN}{{\mathssf{N}}}    % N
\newcommand{\rvR}{{\mathssf{R}}}    % R
\newcommand{\rvRh}{{\hat{\rvR}}}
\newcommand{\rvS}{{\mathssf{S}}}    % S
\newcommand{\rvSh}{{\hat{\rvS}}}
\newcommand{\rvW}{{\mathssf{W}}}    % W
\newcommand{\rvX}{{\mathssf{X}}}    % X, random variable
\newcommand{\svX}{X}
\newcommand{\rvXt}{{\tilde{\rvX}}}
\newcommand{\rvY}{{\mathssf{Y}}}    % Y
\newcommand{\rvZ}{{\mathssf{Z}}}    % Z

\newcommand{\rva}{{\mathssf{a}}}    % a
\newcommand{\rvah}{{\hat{\rva}}}
\newcommand{\sva}{a}
\newcommand{\svah}{{\hat{\sva}}}
\newcommand{\rvba}{{\mathbsf{a}}}
\newcommand{\svba}{{\mathbf{a}}}
\newcommand{\rvhba}{\hat{{{\mathbsf{{a}}}}}}


\newcommand{\rvb}{{\mathssf{b}}}    % b
\newcommand{\rvbB}{{\mathbsf{B}}}   % b
\newcommand{\rvbb}{{\mathbsf{b}}}
\newcommand{\rvhbb}{\hat{{{\mathbsf{{b}}}}}}
\newcommand{\svbb}{{\mathbf{b}}}

\newcommand{\rvc}{{\mathssf{c}}}    % c
\newcommand{\rvch}{{\hat{\rvc}}}
\newcommand{\svc}{c}
\newcommand{\svch}{{\hat{\svc}}}
\newcommand{\rvbc}{{\mathbsf{c}}}
\newcommand{\svbc}{{\mathbf{c}}}

\newcommand{\rvd}{{\mathssf{d}}}    % d
\newcommand{\rvdh}{{\hat{\rvd}}}
\newcommand{\svd}{d}
\newcommand{\svdh}{{\hat{\svd}}}
\newcommand{\rvbd}{{\mathbsf{d}}}
\newcommand{\svbd}{{\mathbf{d}}}



\newcommand{\rve}{{\mathssf{e}}}    % e
\newcommand{\sve}{e}
\newcommand{\rvbe}{{\mathbsf{e}}}
\newcommand{\svbe}{{\mathbf{e}}}
\newcommand{\rvf}{{\mathssf{f}}}    % f
\newcommand{\rvhf}{{\hat{{\mathssf{f}}}}}    % f

\newcommand{\svf}{f}
\newcommand{\rvbf}{{\mathbsf{f}}}
\newcommand{\svbf}{{\mathbf{f}}}
\newcommand{\rvg}{{\mathssf{g}}}    % g
\newcommand{\svg}{g}
\newcommand{\rvbg}{{\mathbsf{g}}}
\newcommand{\rvbG}{{\mathbsf{G}}}
\newcommand{\svbg}{{\mathbf{g}}}
\newcommand{\rvh}{{\mathssf{h}}}    % h
\newcommand{\svh}{h}
\newcommand{\rvbh}{{\mathbsf{h}}}
\newcommand{\rvbH}{{\mathbsf{H}}}
\newcommand{\svbh}{{\mathbf{h}}}
\newcommand{\rvk}{{\mathssf{k}}}    % k
\newcommand{\rvhk}{{\hat{\mathssf{k}}}}    % k

\newcommand{\svk}{k}
\newcommand{\rvl}{{\mathssf{l}}}    % l

\newcommand{\rvm}{{\mathssf{m}}}    % m
\newcommand{\svm}{m}
\newcommand{\rvbm}{{\mathbsf{m}}}
\newcommand{\svbm}{{\mathbf{m}}}
\newcommand{\rvn}{{\mathssf{n}}}    % n
\newcommand{\rvbn}{{\mathbsf{n}}}
\newcommand{\rvp}{{\mathssf{p}}}    % p
\newcommand{\svp}{p}
\newcommand{\rvq}{{\mathssf{q}}}    % q
\newcommand{\svq}{q}
\newcommand{\svQ}{Q}
\newcommand{\rvr}{{\mathssf{r}}}    % r
\newcommand{\rvbr}{{\mathbsf{r}}}
\newcommand{\svr}{r}
\newcommand{\rvs}{{\mathssf{s}}}    % s
\newcommand{\rvbs}{{\mathbsf{s}}}
\newcommand{\svs}{s}
\newcommand{\svbs}{{\mathbf{s}}}
\newcommand{\rvt}{{\mathssf{t}}}    % t
\newcommand{\rvbt}{{\mathbsf{t}}}
\newcommand{\rvhbt}{\hat{{\mathbsf{t}}}}
\newcommand{\svt}{t}
\newcommand{\svbt}{{\mathbf{t}}}
\newcommand{\rvu}{{\mathssf{u}}}    % u
\newcommand{\svu}{u}
\newcommand{\svuh}{{\hat{\svu}}}
\newcommand{\rvbu}{{\mathbsf{u}}}
\newcommand{\rvbU}{{\mathbsf{U}}}
\newcommand{\svbu}{{\mathbf{u}}}
\newcommand{\rvv}{{\mathssf{v}}}    % v
\newcommand{\svv}{v}
\newcommand{\svvh}{{\hat{\svv}}}
\newcommand{\rvbv}{{\mathbsf{v}}}
\newcommand{\rvbV}{{\mathbsf{V}}}
\newcommand{\svbv}{{\mathbf{v}}}
\newcommand{\rvvh}{{\hat{\rvv}}}
\newcommand{\rvw}{{\mathssf{w}}}    % w
\newcommand{\svw}{w}
\newcommand{\rvwh}{{\hat{\rvw}}}
\newcommand{\svwh}{{\hat{\svw}}}
\newcommand{\rvbw}{{\mathbsf{w}}}
\newcommand{\svbw}{{\mathbf{w}}}
\newcommand{\rvx}{{\mathssf{x}}}    % x, random variable
\newcommand{\rvxh}{{\hat{\rvx}}}
\newcommand{\rvxt}{{\tilde{\rvx}}}
\newcommand{\svx}{x}            % sample value
\newcommand{\svxh}{{\hat{\svx}}}
\newcommand{\svxt}{{\tilde{\svx}}}
\newcommand{\rvbx}{{\mathbsf{x}}}
\newcommand{\rvbxh}{{\hat{\rvbx}}}
\newcommand{\rvbxt}{{\tilde{\rvbx}}}
\newcommand{\svbx}{{\mathbf{\svx}}}
\newcommand{\svbxt}{{\tilde{\svbx}}}
\newcommand{\svbxh}{{\hat{\mathbf{x}}}}
\newcommand{\rvy}{{\mathssf{y}}}    % y
\newcommand{\rvyh}{{\hat{\mathssf{y}}}}
\newcommand{\svy}{y}
\newcommand{\rvyt}{{\tilde{\rvy}}}
\newcommand{\svyt}{{\tilde{\svy}}}
\newcommand{\svyh}{{\hat{\svy}}}
\newcommand{\rvby}{{\mathbsf{y}}}
\newcommand{\rvbyt}{{\tilde{\rvby}}}
\newcommand{\svby}{{\mathbf{y}}}
\newcommand{\svbyt}{{\tilde{\svby}}}
\newcommand{\rvz}{{\mathssf{z}}}    % z
\newcommand{\rvzh}{{\hat{\rvz}}}
\newcommand{\rvzt}{{\tilde{\rvz}}}
\newcommand{\svz}{z}
\newcommand{\svzh}{{\hat{\svz}}}
\newcommand{\rvbz}{{\mathbsf{z}}}
\newcommand{\svbz}{{\mathbf{z}}}

\newcommand{\rvB}{{\mathssf{B}}}
\newcommand{\rvJ}{{\mathssf{J}}}
\newcommand{\rvK}{{\mathssf{K}}}
\newcommand{\rvT}{{\mathssf{T}}}
\newcommand{\rvU}{{\mathssf{U}}}
\newcommand{\rvV}{{\mathssf{V}}}

% Handle uppercase Greek differently
\newcommand{\rvTh}{\ssfTheta}
\newcommand{\svTh}{\Theta}
\newcommand{\rvbTh}{\bsfTheta}
\newcommand{\svbTh}{\boldsymbol{\Theta}}
\newcommand{\rvPh}{\ssfPhi}
\newcommand{\svPh}{\Phi}
\newcommand{\rvbPh}{\bsfPhi}
\newcommand{\svbPh}{\boldsymbol{\Phi}}

\newcommand{\ddx}{\frac{\p}{\p \svx}}
\newcommand{\ddbx}{\frac{\p}{\p\svbx}}

%  --add new macros below this line--

% \newcommand{\iid}{\emph{i.i.d.}}
\newcommand{\iid}{i.i.d.}

\newcommand{\corolref}[1]{Corollary~\mbox{\ref{#1}}}
\newcommand{\thrmref}[1]{Theorem~\mbox{\ref{#1}}}
\newcommand{\lemref}[1]{Lemma~\mbox{\ref{#1}}}
\newcommand{\figref}[1]{Figure~\mbox{\ref{#1}}}
\newcommand{\secref}[1]{Section~\mbox{\ref{#1}}}
\newcommand{\chapref}[1]{Chapter~\mbox{\ref{#1}}}
\newcommand{\appref}[1]{Appendix~\mbox{\ref{#1}}}


% A popular package from the American Mathematical Society that provides
% many useful and powerful commands for dealing with mathematics. If using
% it, be sure to load this package with the cmex10 option to ensure that
% only type 1 fonts will utilized at all point sizes. Without this option,
% it is possible that some math symbols, particularly those within
% footnotes, will be rendered in bitmap form which will result in a
% document that can not be IEEE Xplore compliant!
%
% Also, note that the amsmath package sets \interdisplaylinepenalty to 10000
% thus preventing page breaks from occurring within multiline equations. Use:
%\interdisplaylinepenalty=2500
% after loading amsmath to restore such page breaks as IEEEtran.cls normally
% does. amsmath.sty is already installed on most LaTeX systems. The latest
% version and documentation can be obtained at:
% http://www.ctan.org/tex-archive/macros/latex/required/amslatex/math/





% *** SPECIALIZED LIST PACKAGES ***
%
%\usepackage{algorithmic}
% algorithmic.sty was written by Peter Williams and Rogerio Brito.
% This package provides an algorithmic environment fo describing algorithms.
% You can use the algorithmic environment in-text or within a figure
% environment to provide for a floating algorithm. Do NOT use the algorithm
% floating environment provided by algorithm.sty (by the same authors) or
% algorithm2e.sty (by Christophe Fiorio) as IEEE does not use dedicated
% algorithm float types and packages that provide these will not provide
% correct IEEE style captions. The latest version and documentation of
% algorithmic.sty can be obtained at:
% http://www.ctan.org/tex-archive/macros/latex/contrib/algorithms/
% There is also a support site at:
% http://algorithms.berlios.de/index.html
% Also of interest may be the (relatively newer and more customizable)
% algorithmicx.sty package by Szasz Janos:
% http://www.ctan.org/tex-archive/macros/latex/contrib/algorithmicx/



% *** ALIGNMENT PACKAGES ***
%
%\usepackage{array}
% Frank Mittelbach's and David Carlisle's array.sty patches and improves
% the standard LaTeX2e array and tabular environments to provide better
% appearance and additional user controls. As the default LaTeX2e table
% generation code is lacking to the point of almost being broken with
% respect to the quality of the end results, all users are strongly
% advised to use an enhanced (at the very least that provided by array.sty)
% set of table tools. array.sty is already installed on most systems. The
% latest version and documentation can be obtained at:
% http://www.ctan.org/tex-archive/macros/latex/required/tools/


%\usepackage{mdwmath}
%\usepackage{mdwtab}
% Also highly recommended is Mark Wooding's extremely powerful MDW tools,
% especially mdwmath.sty and mdwtab.sty which are used to format equations
% and tables, respectively. The MDWtools set is already installed on most
% LaTeX systems. The lastest version and documentation is available at:
% http://www.ctan.org/tex-archive/macros/latex/contrib/mdwtools/


% IEEEtran contains the IEEEeqnarray family of commands that can be used to
% generate multiline equations as well as matrices, tables, etc., of high
% quality.


%\usepackage{eqparbox}
% Also of notable interest is Scott Pakin's eqparbox package for creating
% (automatically sized) equal width boxes - aka "natural width parboxes".
% Available at:
% http://www.ctan.org/tex-archive/macros/latex/contrib/eqparbox/





% *** SUBFIGURE PACKAGES ***
%\usepackage[tight,footnotesize]{subfigure}
% subfigure.sty was written by Steven Douglas Cochran. This package makes it
% easy to put subfigures in your figures. e.g., "Figure 1a and 1b". For IEEE
% work, it is a good idea to load it with the tight package option to reduce
% the amount of white space around the subfigures. subfigure.sty is already
% installed on most LaTeX systems. The latest version and documentation can
% be obtained at:
% http://www.ctan.org/tex-archive/obsolete/macros/latex/contrib/subfigure/
% subfigure.sty has been superceeded by subfig.sty.



%\usepackage[caption=false]{caption}
\usepackage{caption}
%\usepackage[font=footnotesize]{subfig}
% subfig.sty, also written by Steven Douglas Cochran, is the modern
% replacement for subfigure.sty. However, subfig.sty requires and
% automatically loads Axel Sommerfeldt's caption.sty which will override
% IEEEtran.cls handling of captions and this will result in nonIEEE style
% figure/table captions. To prevent this problem, be sure and preload
% caption.sty with its "caption=false" package option. This is will preserve
% IEEEtran.cls handing of captions. Version 1.3 (2005/06/28) and later
% (recommended due to many improvements over 1.2) of subfig.sty supports
% the caption=false option directly:
\usepackage[caption=false,font=footnotesize]{subfig}
%
% The latest version and documentation can be obtained at:
% http://www.ctan.org/tex-archive/macros/latex/contrib/subfig/
% The latest version and documentation of caption.sty can be obtained at:
% http://www.ctan.org/tex-archive/macros/latex/contrib/caption/




% *** FLOAT PACKAGES ***
%
%\usepackage{fixltx2e}
% fixltx2e, the successor to the earlier fix2col.sty, was written by
% Frank Mittelbach and David Carlisle. This package corrects a few problems
% in the LaTeX2e kernel, the most notable of which is that in current
% LaTeX2e releases, the ordering of single and double column floats is not
% guaranteed to be preserved. Thus, an unpatched LaTeX2e can allow a
% single column figure to be placed prior to an earlier double column
% figure. The latest version and documentation can be found at:
% http://www.ctan.org/tex-archive/macros/latex/base/



%\usepackage{stfloats}
% stfloats.sty was written by Sigitas Tolusis. This package gives LaTeX2e
% the ability to do double column floats at the bottom of the page as well
% as the top. (e.g., "\begin{figure*}[!b]" is not normally possible in
% LaTeX2e). It also provides a command:
%\fnbelowfloat
% to enable the placement of footnotes below bottom floats (the standard
% LaTeX2e kernel puts them above bottom floats). This is an invasive package
% which rewrites many portions of the LaTeX2e float routines. It may not work
% with other packages that modify the LaTeX2e float routines. The latest
% version and documentation can be obtained at:
% http://www.ctan.org/tex-archive/macros/latex/contrib/sttools/
% Documentation is contained in the stfloats.sty comments as well as in the
% presfull.pdf file. Do not use the stfloats baselinefloat ability as IEEE
% does not allow \baselineskip to stretch. Authors submitting work to the
% IEEE should note that IEEE rarely uses double column equations and
% that authors should try to avoid such use. Do not be tempted to use the
% cuted.sty or midfloat.sty packages (also by Sigitas Tolusis) as IEEE does
% not format its papers in such ways.




% *** PDF, URL AND HYPERLINK PACKAGES ***
%
\usepackage{url}
% url.sty was written by Donald Arseneau. It provides better support for
% handling and breaking URLs. url.sty is already installed on most LaTeX
% systems. The latest version can be obtained at:
% http://www.ctan.org/tex-archive/macros/latex/contrib/misc/
% Read the url.sty source comments for usage information. Basically,
% \url{my_url_here}.





% *** Do not adjust lengths that control margins, column widths, etc. ***
% *** Do not use packages that alter fonts (such as pslatex).         ***
% There should be no need to do such things with IEEEtran.cls V1.6 and later.
% (Unless specifically asked to do so by the journal or conference you plan
% to submit to, of course. )

%\usepackage{enumitem}

% correct bad hyphenation here
\hyphenation{op-tical net-works semi-conduc-tor}


\begin{document}
%
% paper title
% can use linebreaks \\ within to get better formatting as desired
%\title{Lossy Broadcasting to Diverse Users: Rateless Codes and Hybrid Approaches}
%\title{Successive Segmentation-based Coding for Broadcasting over Erasure Channels}
\title{Two-terminal Erasure Source-Broadcast with Feedback}
%
% author names and IEEE memberships
% note positions of commas and nonbreaking spaces ( ~ ) LaTeX will not break
% a structure at a ~ so this keeps an author's name from being broken across
% two lines.
% use \thanks{} to gain access to the first footnote area
% a separate \thanks must be used for each paragraph as LaTeX2e's \thanks
% was not built to handle multiple paragraphs
%

%\author{Michael~Shell,~\IEEEmembership{Member,~IEEE,}
%        John~Doe,~\IEEEmembership{Fellow,~OSA,}
%        and~Jane~Doe,~\IEEEmembership{Life~Fellow,~IEEE}% <-this % stops a space

%\thanks{J. Doe and J. Doe are with Anonymous University.}% <-this % stops a space
%\thanks{TCOM version based on Michael Shell's bare{\textunderscore}jrnl.tex version 1.3.}}
%%%
%\author{\IEEEauthorblockN{Author One\IEEEauthorrefmark{1},
%Author Two\IEEEauthorrefmark{2}, Author Three\IEEEauthorrefmark{3} and
%Author Four\IEEEauthorrefmark{4}}
%\IEEEauthorblockA{Department of Electrical and Computer Engineering,
%University of Toronto\\
%10 King's College\\
%Email: \IEEEauthorrefmark{1}louis.tan@mail.utoronto.ca,
%\IEEEauthorrefmark{2}kaveh.mahdaviani@mail.utoronto.ca,
%\IEEEauthorrefmark{3}akhisti@ece.utoronto.ca,}}

%%\author{\IEEEauthorblockN{Louis Tan\IEEEauthorrefmark{1}, Kaveh Mahdaviani\IEEEauthorrefmark{2} and Ashish Khisti\IEEEauthorrefmark{3}~\IEEEmembership{Member,~IEEE}}
%%
%%\IEEEauthorblockA{Department of Electrical and Computer Engineering,
%%University of Toronto\\
%%10 King's College Road, Toronto, ON, Canada M5S 3G4\\
%%Email: \IEEEauthorrefmark{1}louis.tan@mail.utoronto.ca,
%%\IEEEauthorrefmark{2}kaveh.mahdaviani@mail.utoronto.ca,
%%\IEEEauthorrefmark{3}akhisti@ece.utoronto.ca}


\author{\IEEEauthorblockN{Louis Tan, Kaveh Mahdaviani and Ashish Khisti~\IEEEmembership{Member,~IEEE}}
%\author{\IEEEauthorblockN{Louis Tan, Kaveh Mahdaviani, Ashish Khisti~\IEEEmembership{ Member,~IEEE} and Emina Soljanin~\IEEEmembership{Fellow,~IEEE}.}
%\thanks{L.~Tan and A.~Khisti are with the Dept. of Electrical and Computer Engineering, University of Toronto, Toronto, ON, Canada. Y.~Li is with the Dept.\ of Electrical Engineering, UCLA, Box 951594, Los Angeles, CA 90095. E.~Soljanin is with Bell Labs, Alcatel-Lucent, Murray Hill, NJ 07974, USA. Part of this work was presented at the 2013 Information Theory Workshop in Seville, Spain~\cite{TLKS_ITW13}, and at the 2014 International Symposium on Information Theory in Honolulu, Hawaii~\cite{LTKS_ISIT14}.}%
\thanks{L.~Tan, K.~Mahdaviani and A.~Khisti are with the Dept.\ of Electrical and Computer Engineering, University of Toronto, Toronto, ON, Canada (e-mail: louis.tan@mail.utoronto.ca, mahdaviani@cs.toronto.edu, akhisti@ece.utoronto.ca). 
%Y.~Li is with Akamai Technologies, Inc., 3355 Scott Boulevard, Santa Clara, CA 95054, USA. 
%E.~Soljanin is with the Dept.\ of Electrical and Computer Engineering, Rutgers University, Piscataway, NJ 08854, USA. 
%Part of this work was presented at the 2013 Information Theory Workshop in Seville, Spain~\cite{TLKS_ITW13}, and at the 2014 International Symposium on Information Theory in Honolulu, Hawaii~\cite{LTKS_ISIT14}.
Part of this work was presented at the 2015 International Symposium on Information Theory in Hong Kong~\cite{TMKS_ISIT15}.}%
}%
%%%\\
%%%\IEEEauthorblockA{\IEEEauthorrefmark{1}Dept.\ of Electrical Engineering, UCLA, Box 951594, Los Angeles, CA 90095, USA}%, liyao@ucla.edu}
%%%\IEEEauthorblockA{\IEEEauthorrefmark{2}Dept.\ of Electrical \& Computer Engineering, University of Toronto, Toronto, ON M5S 3G4 Canada} %\{ltan, akhisti\}@comm.utoronto.ca}
%%%\IEEEauthorblockA{\IEEEauthorrefmark{3}Bell Labs, Alcatel-Lucent, Murray Hill, NJ 07974, USA}%, \\emina@alcatel-lucent.com}
%%%\IEEEauthorblockA{Email: liyao@ucla.edu, \{ltan, akhisti\}@comm.utoronto.ca and emina@alcatel-lucent.com}
%%%}

%\IEEEauthorblockA{Dept.\ of Electrical \& Computer Engineering\\ University of Toronto\\
%Toronto, ON M5S 3G4 Canada\\ ltan@comm.utoronto.ca} \and \IEEEauthorblockN{Yao Li} \IEEEauthorblockA{Dept.\ of Electrical Engineering, UCLA\\ Los Angeles CA USA\\
%liyao@ucla.edu} \and \IEEEauthorblockN{Ashish Khisti} \IEEEauthorblockA{Dept.\ of Electrical \& Computer Engineering\\ University of Toronto\\
%Toronto, ON M5S 3G4 Canada\\ akhisti@comm.utoronto.ca} \and \IEEEauthorblockN{Emina Soljanin} \IEEEauthorblockA{Bell Labs, Alcatel-Lucent\\ Murray Hill NJ 07974, USA\\
%emina@alcatel-lucent.com}}


%\author{\IEEEauthorblockN{Louis Tan} \IEEEauthorblockA{Dept.\ of Electrical \& Computer Engineering\\ University of Toronto\\
%Toronto, ON M5S 3G4 Canada\\ ltan@comm.utoronto.ca} \and \IEEEauthorblockN{Yao Li} \IEEEauthorblockA{Dept.\ of Electrical Engineering, UCLA\\ Los Angeles CA USA\\
%liyao@ucla.edu} \and \IEEEauthorblockN{Ashish Khisti} \IEEEauthorblockA{Dept.\ of Electrical \& Computer Engineering\\ University of Toronto\\
%Toronto, ON M5S 3G4 Canada\\ akhisti@comm.utoronto.ca} \and \IEEEauthorblockN{Emina Soljanin} \IEEEauthorblockA{Bell Labs, Alcatel-Lucent\\ Murray Hill NJ 07974, USA\\
%emina@alcatel-lucent.com}}

%\author{\IEEEauthorblockN{Louis Tan and Ashish Khisti} \IEEEauthorblockA{Dept.\ of Electrical \& Computer Engineering\\ University of Toronto\\
%Toronto, ON M5S 3G4 Canada\\ \{ltan, akhisti\}@comm.utoronto.ca} \and \IEEEauthorblockN{Emina Soljanin} \IEEEauthorblockA{Bell Labs, Alcatel-Lucent\\ Murray Hill NJ 07974, USA\\
%emina@alcatel-lucent.com}}

% note the % following the last \IEEEmembership and also \thanks -
% these prevent an unwanted space from occurring between the last author name
% and the end of the author line. i.e., if you had this:
%
% \author{....lastname \thanks{...} \thanks{...} }
%                     ^------------^------------^----Do not want these spaces!
%
% a space would be appended to the last name and could cause every name on that
% line to be shifted left slightly. This is one of those "LaTeX things". For
% instance, "\textbf{A} \textbf{B}" will typeset as "A B" not "AB". To get
% "AB" then you have to do: "\textbf{A}\textbf{B}"
% \thanks is no different in this regard, so shield the last } of each \thanks
% that ends a line with a % and do not let a space in before the next \thanks.
% Spaces after \IEEEmembership other than the last one are OK (and needed) as
% you are supposed to have spaces between the names. For what it is worth,
% this is a minor point as most people would not even notice if the said evil
% space somehow managed to creep in.



% The paper headers
%\markboth{IEEE Journal on Selected Areas in Communications}%
%{Submitted paper}
% The only time the second header will appear is for the odd numbered pages
% after the title page when using the twoside option.
%
% *** Note that you probably will NOT want to include the author's ***
% *** name in the headers of peer review papers.                   ***
% You can use \ifCLASSOPTIONpeerreview for conditional compilation here if
% you desire.




% If you want to put a publisher's ID mark on the page you can do it like
% this:
%\IEEEpubid{0000--0000/00\$00.00~\copyright~2007 IEEE}
% Remember, if you use this you must call \IEEEpubidadjcol in the second
% column for its text to clear the IEEEpubid mark.



% use for special paper notices
%\IEEEspecialpapernotice{(Invited Paper)}



%\vspace{-4em}


% make the title area
\maketitle


\begin{abstract}
%\boldmath
\iffalse

We study a successive segmentation-based coding scheme for broadcasting a binary source over a multi-receiver erasure broadcast channel. Each receiver has a certain demand on the fraction of source symbols to be reconstructed, and its channel is a memoryless erasure channel.  We study the minimum achievable latency at the source required to simultaneously meet all the receiver constraints. We consider a class of schemes that partition the source sequence into multiple segments and apply a systematic erasure code to each segment. We formulate the optimal choice of segment sizes and code-rates in this class of schemes as a linear programming problem and provide an explicit solution. We further show that the optimal solution can be interpreted as a successive segmentation scheme, which naturally adjusts when users are added or deleted from the system. 

Given the solution for the segment sizes, we then consider possible transmission orderings for \emph{individual} user decoding delay considerations.  We provide closed-form expressions for each individual user's excess latency when parity checks are successively transmitted in both increasing and decreasing order of the segment's coded rate.  Finally, we adapt the segmentation-based coding scheme for transmission across the product of two reversely degraded erasure broadcast channels and numerically show that significant gains are achievable compared to baseline separation-based coding schemes.
\fi
%

We study the effects of introducing a feedback channel in the two-receiver erasure source-broadcast problem in which a binary equiprobable source is to be sent over an erasure broadcast channel to two receivers subject to erasure distortion constraints.  The receivers each require a certain fraction of a source sequence, and we are interested in the minimum latency, or transmission time, required to serve them all.  We first show that for a two-user broadcast channel, a point-to-point outer bound can always be achieved.  We further show that the point-to-point outer bound can also be achieved if only one of the users, the \emph{stronger} user, has a feedback channel.  Our coding scheme relies on a hybrid approach that combines transmitting both random linear combinations of source symbols as well as a retransmission strategy.  

%We consider the problem of deriving an outer bound for the source-broadcast problem involving a common equiprobable source that is to be sent to two receivers subject to an erasure distortion constraint.  The system model we study is the same as in Chapter~\ref{chap:no_feedback}, however our derivation makes two additional assumptions.  The first assumption is that we consider the case when $M \triangleq 1/(1 - \epsilon_2)$ is an integer.  The second assumption is that we consider only the class of non-erasure-randomized codes.  As we explain in Section~\ref{sec:virtual_source}, this is the class of codes for which the positions of erasures in the source reconstruction is determined only by the channel noise realization.  

%The outer bound we present is parameterized by the distortion of user~1, the stronger user.  We assume that $D_1$, the distortion achieved by user~1, is given by $D_1 = \Dstar{1} + \delta$, where $\delta \in [0, \epsilon_1 - \Dstar{1})$, and $\Dstar{i}$ is the point-to-point optimal distortion for user~$i$, given by
%
%\begin{equation}
%	\Dstar{i} = 1 - b(1 - \epsilon_i),
%\end{equation}
%%
%where $b = n/m$ is the number of channel uses per source symbol, i.e., the bandwidth expansion factor.  

%Motivated by error correction coding in multimedia applications, we study the problem of broadcasting a single common source to multiple receivers over heterogenous erasure channels. Each receiver is required to partially reconstruct the source sequence by decoding a certain fraction of the source symbols.  We propose a coding scheme that requires only off-the-shelf erasure codes and can be easily adapted as users join and leave the network. Our scheme involves splitting the source sequence into multiple segments and applying a systematic erasure code to each such segment. We formulate the problem of minimizing the transmission latency at the server as a linear programming problem and explicitly characterize an optimal choice for the code-rates and segment sizes. Through numerical comparisons, we demonstrate that our proposed scheme outperforms both separation-based coding schemes, and degree-optimized rateless codes and performs close to a natural outer {(lower)} bound in certain cases.  %\textcolor{brown}{\sout{We also show how our segmentation-based scheme can be naturally extended if the network users instead listened over \emph{parallel} broadcast channels.}}

%We further study \emph{individual} user decoding delays for various orderings of segments in our scheme.  We  provide closed-form expressions for each individual user's excess latency when parity checks are successively transmitted in both increasing and decreasing order of their segment's coded rate and also qualitatively discuss the merits of each order. %\textcolor{red}{Finally, we also present a natural extension of the segmentation-based scheme to parallel broadcast channels.}

\end{abstract}
% IEEEtran.cls defaults to using nonbold math in the Abstract.
% This preserves the distinction between vectors and scalars. However,
% if the journal you are submitting to favors bold math in the abstract,
% then you can use LaTeX's standard command \boldmath at the very start
% of the abstract to achieve this. Many IEEE journals frown on math
% in the abstract anyway.


%\vspace{-2em}

% Note that keywords are not normally used for peerreview papers.
%\begin{IEEEkeywords}
%Rateless Codes, Packet Erasure Channels, Linear Programming, Unequal Error Protection, Multiple Description Coding, eMBMS
%\end{IEEEkeywords}
\begin{IEEEkeywords}
%Application-Layer Error Correction Coding, Broadcast Channels,  Joint Source-Channel Coding, Linear Programming, Multimedia broadcast/multicast services (MBMS), Rateless Codes, Unequal Error Protection.
\end{IEEEkeywords}



% For peer review papers, you can put extra information on the cover
% page as needed:
% \ifCLASSOPTIONpeerreview
% \begin{center} \bfseries EDICS Category: 3-BBND \end{center}
% \fi
%
% For peerreview papers, this IEEEtran command inserts a page break and
% creates the second title. It will be ignored for other modes.
\IEEEpeerreviewmaketitle

Reinforcement learning has achieved great success in areas such as Game-playing \citep{silver2018general,vinyals2019grandmaster}, robotics \cite{kober2013reinforcement}, large language models \citep{ouyang2022training}, etc.
However, due to safety concerns or physical limitations, in some real-world reinforcement learning problems, we must consider additional constraints that may influence the optimal policy and the learning process \citep{garcia2015comprehensive}.
% For example, a robotic arm must not take actions that may cause harm to itself or the environments.
A standard framework to handle such cases is the constrained Markov Decision Process (CMDP) \citep{altman1999constrained}.
Within the CMDP framework, the agent has to maximize
the expected cumulative reward while
obeying a finite number of constraints, which are usually in the form of expected cumulative cost criteria.

However, we are sometimes concerned with the problem with a continuum of constraints.
For example,
the constraints we meet might be time-evolving or subject to uncertain parameters, which
cannot be formulated as an ordinary CMDP
(see Examples \ref{Example_Time_Evolving} and  \ref{Example_Uncertain}).
In this paper we would study a generalized CMDP  
to address the above problem.  Because the constraints are not only infinite-number but also lie
in a continuous set,
the generalization is not trivial. Fortunately, we find that we can borrow the idea behind semi-infinite programming (SIP) \citep{remez1934determination, hettich1993semi} to deal with the semi-infinite constraints.
Accordingly, we propose \emph{semi-infinitely constrained Markov decision processes} (SICMDPs)
as a novel complement to the ordinary CMDP framework.
%More specifically,  an SICMDP model %, we consider 
%contains a continuum of constraints whereas an ordinary CMDP contains a finite number of constraints. 

%This generalization is natural but not trivial. However, we can brows the idea  
%The idea is quite natural and can be backtracked
%to the practice of extending linear programming to linear semi-infinite programming (LSIP) %\cite{remez1934determination, GobernaLSIO1998}.
%In addition, 
%As a complementary approach to the ordinary CMDP framework, 
%SICMDP can be used to model these problems  which cannot be described by a finite number of constraints
%that are not covered by .
%For example,
%the restrictions we consider can be time-evolving or subject to uncertain parameters
%, thus
%cannot be described by a finite number of constraints but a continuum of constraints 
%(see Examples \ref{Example_Time_Evolving} and  \ref{Example_Uncertain}).

We also present two reinforcement learning algorithms to solve SICMDPs called SI-CRL and SI-CPO, respectively.
SI-CRL is a model-based reinforcement learning algorithm designed for tabular cases, and SI-CPO is a policy optimization algorithm for non-tabular cases.
% and analyze its performance both theoretically and empirically.
The main challenge is that we need to deal with a continuum of constraints, thus reinforcement learning algorithms for ordinary CMDPs do not work anymore.
In SI-CRL, we tackle this difficulty by first transforming the reinforcement learning problem to an equivalent LSIP problem, which can then be solved using methods in the LSIP literature like the dual exchange methods \citep{Hu1990,reemtsen1998numerical}.
In SI-CPO, we resort to the idea of cooperative stochastic approximation developed in \cite{lan2020algorithms, wei2020comirror}.
As far as we know, we are the first to introduce tools from semi-infinitely programming (SIP) into the reinforcement learning community for solving constrained reinforcement learning problems.

% To the best of our knowledge, we are the first to apply tools from semi-infinitely programming (SIP) to solve reinforcement learning problems.
Furthermore, we give theoretical analysis for both SI-CRL and SI-CPO.
We decompose the error of SI-CRL into two parts: the statistical error from approximating the true SICMDP with an offline dataset and the optimization error due to the fact that the solution of the LSIP problem obtained by the dual exchange method is inexact.
On the optimization side, we show that the iteration complexity of SI-CRL is $O\left(\left\{\mathrm{diam}(Y)L\sqrt{|\gS|^2|\gA|m}/\left[(1-\gamma)\epsilon\right]\right\}^m\right)$.
On the statistical side, we show that the sample complexity of SI-CRL is $\widetilde O\left(\frac{|S|^2|A|^2}{\epsilon^2(1-\gamma)^3}\right)$ if the offline dataset is generated by a generative model, and $\widetilde O\left(\frac{|S||A|}{\nu_{\min} \epsilon^2(1-\gamma)^3}\right)$ if the dataset is generated by a probability measure $\nu$ as considered in \cite{chen2019information}.
Here $\widetilde O$ means that all logarithm terms are discarded.
For SI-CPO, things become a little more complicated because other than the statistical error and the optimization error, we also need to consider the function approximation error, which comes from imperfect policy parametrizations.
It is shown if the function approximation error can be controlled to $O(\epsilon)$ order, the iteration complexity of SI-CPO is $\widetilde{O}\left(\frac{1}{\epsilon^2(1-\gamma)^6}\right)$ and the sample complexity of SI-CPO is $\widetilde{O}(\frac{1}{\epsilon^4(1-\gamma)^{10}})$.
Here our iteration complexity bound is equivalent to a typical $\widetilde O(1/\sqrt{T})$ global convergence rate.

We perform a set of numerical experiments to illustrate the SICMDP model and validate our proposed algorithms.
Specifically, we examine two numerical examples, namely the discharge of sewage and ship route planning.
Through the discharge of sewage example, we show the advantage of the SICMDP framework over the CMDP baseline obtained by naive discretization in modeling realistic sequential decision-making problems.
Moreover, we demonstrate the effectiveness of the SI-CRL and SI-CPO algorithms in such tabular environments. 
In the ship route planning example, we illustrate the benefits of the SICMDP framework and the ability of the SI-CPO algorithm to address complex continuous control tasks involving continuous state spaces with modern deep reinforcement learning techniques.

% In summary, our contributions are listed as follows.
% First, we present the SICMDP model, which can be viewed as a generalization of the ordinary CMDP model.
% Second, we propose an algorithm to perform reinforcement learning for SICMDPs, which is called SI-CRL, and we believe that we are the first to apply tools from SIP
% to solve reinforcement learning problems.
% Third, we give a theoretical analysis of SI-CRL and identify both its sample complexity and iteration complexity.
% In addition, we perform numerical experiments to illustrate the SICMDP model and validate the SI-CRL algorithm.
% \{This paragraph can be removed!!! \}





\input{SysModel/SysModel2}
\section{Source-broadcast with Universal Feedback}
\label{sec:two_users} 
We first show that the case involving only two users can be fully solved.  We do this by demonstrating an algorithm that achieves point-to-point optimality for both users at any time during transmission.  Specifically, for $i \in \{1, 2\}$, let $\wid$ be the point-to-point optimal latency for user~$i$ obtained from the source-channel separation theorem where

\begin{align}
\label{eq:wid_optimal}
	\wid = \frac{1 - d_i}{1 - \epsilon_i}.
\end{align}
%The coding scheme for this case is shown in Algorithm~\ref{alg:two_users}.  
We now present an algorithm to achieve this outer bound.  In the first phase of the algorithm, we successively transmit each source symbol uncoded until at least one user receives it. If $S(t)$ is received only by user~$i$,  then the transmitter places $S(t)$ into queue $Q_j$.
No action is taken if both users receive $S(t)$. By assumption, feedback is universally available, and so user~$i$ is also able to maintain a local version of queue~$Q_j$.

Now, after this first phase, the transmitter has built queues $Q_1$ and $Q_2$, where 
for $i,j \in \{1, 2\}, i\neq j$, 
user~$i$ has knowledge of packets in $Q_j$ and is in need of those in $Q_i$.
% for $i, j \in \{1, 2\}, i \neq j$. 
Thus, the algorithm's next phase involves successively transmitting a linear combination of the packets at the fronts of $Q_1$ and $Q_2$.  Let $q_i$ be the packet at the front of $Q_i$.  Notice that a successfully received channel symbol of the form $q_1 \oplus q_2$ means that user~1 is able to decode $q_1 \in Q_1$,  since he has access to $q_2 \in Q_2$. 
We therefore remove $q_i$ from $Q_i$ whenever a linear combination involving $q_i$ is received by user~$i$.
%
This entire phase continues until the users' distortion constraints are met.  The decoding algorithm for this scheme is also simple.  Given that user~$i$ has decoded source symbol~$S(t)$ from the linear combination he received, user~$i$ sets $\hat{S}_i(t) = S(t)$.

%%% Good
%%%\begin{algorithm}
%%%\caption{Myalgorithm}\label{alg:two_users}
%%%\begin{algorithmic}[1]
%%%\For {$t = 1, 2, \ldots N$ \textbf{AND} $R_i < (1 - d_i)N, i \in \{1, 2\}$}
%%%    \State Send $S(t)$ until at least one user receives it
%%%    \If {$N_i = 1, N_j = 0$ for some $i, j \in \{1, 2\}, i \neq j$}
%%%        \State $Q_i\gets Q_i \cup \{S(t)\}$
%%%    \EndIf
%%%    \If {$N_i = 0$}
%%%        \State $R_i\gets R_i + 1$
%%%    \EndIf
%%%\EndFor
%%%
%%%\While{$R_i < (1 - d_i) N, i \in \{1, 2\}$}
%%%    \If {$R_i < (1 - d_i)N$ \textbf{AND} $Q_i \neq \emptyset$}
%%%        \State Let $q_i \in Q_i$
%%%    \Else
%%%    	\State $q_i = 0$
%%%    \EndIf	
%%%    \State Send $q_1 \oplus q_2$ until at least one user receives it
%%%    \If {$N_i = 0$}
%%%        \State $R_i\gets R_i + 1$
%%%        \State $Q_i \gets Q_i \setminus \{q_i\}$
%%%    \EndIf    
%%%\EndWhile
%%%
%%%
%%%\end{algorithmic}
%%%\end{algorithm}

%\begin{algorithm}
%\caption{Two-user Feedback}\label{lag:two_user}
%\begin{algorithmic}[1]
%	\Procedure{MyProcedure}{}
%	\State $\textit{stringlen} \gets \text{length of }\textit{string}$
%	\State $i \gets \textit{patlen}$
%	\BState \emph{top}:
%	\If {$i > \textit{stringlen}$} \Return false
%	\EndIf
%	\State $j \gets \textit{patlen}$
%	\BState \emph{loop}:
%	\If {$\textit{string}(i) = \textit{path}(j)$}
%	\State $j \gets j-1$.
%	\State $i \gets i-1$.
%	\State \textbf{goto} \emph{loop}.
%	\State \textbf{close};
%	\EndIf
%	\State $i \gets i+\max(\textit{delta}_1(\textit{string}(i)),\textit{delta}_2(j))$.
%	\State \textbf{goto} \emph{top}.
%	\EndProcedure
%\end{algorithmic}
%\end{algorithm}

Our algorithm has two appealing properties. The first is that it involves only transmissions that are \emph{instantly decodable}, which is seldom the case when channel codes are used.  
%Here, at any time $t$, a user can produce the reconstruction $\hat{S}^{N}|_t$.  
Secondly, this coding scheme involves only transmissions that are \emph{distortion-innovative}.  This means that any successfully received channel symbol can be immediately used to reconstruct a single source symbol that was hitherto unknown.  %That is, given that a user successfully received a new channel symbol at time $t$, we have that $d(S^N, \hat{S}^N|_t) < d(S^N, \hat{S}^N|_{(t-1)})$.  
%
In fact, our coding scheme has the property that for any latency $w \in [0, 1/(1 - \epsilon)]$, after $wN$ transmissions have been sent over the channel, an expected value of $\gamma = wN(1 - \epsilon)$ channel symbols were received, which leads to the decoding of precisely $\gamma$ source symbols.  The distortion achieved after $wN$ transmissions is thus seen to be $D = 1 - w(1 - \epsilon)$, which we readily recognize as 
%coinciding with 
the separation-based outer bound in~\eqref{eq:wid_optimal}.  Since the transmission of an instantly-decodable, distortion-innovative symbol does not require a channel encoder, we will sometimes refer to such transmissions as \emph{analog} transmissions.

In the next section, we study
a variation of our problem formulation in which only the \emph{stronger} user has a feedback channel available.  For this problem variation, we show that we can still achieve the optimal minmax latency despite the weaker user not having access to a feedback channel.

%Finally, we mention that while our present scheme uses feedback from both users, in Section~\ref{sec:one_sided_feedback} we study
%a variation of our problem formulation in which only the \emph{stronger} user has a feedback channel available.  In that section, we show that we can still achieve the optimal minmax latency despite the weaker user not having access to a feedback channel.
%can also be studied and will be reported in a forthcoming work. 


%\subsection{Single Feedback Channel}

\iffalse
We demonstrate the importance of sending instantly decodable, distortion-innovative symbols by considering a slightly altered problem setup.  Consider if we again have two users that wish to fractionally reconstruct the source, however, this time there is only a single feedback channel available.  We assume that this channel is available to the user with the better channel quality, i.e., at time $T$, only $\{N_1(t)\}_{t = 1, 2, \ldots, T- 1}$ is universally known.  Our motivation for this problem arises from considering broadcast channels with a large number of users.  Given the large amount of feedback that may be available, it may not be possible to process it all, and so we propose a model in which the coding ignores a subset of the feedback available.  In our current discussion we furthermore assume a \emph{physically} degraded channel.  The case of stochastically degraded channels will be explored in a forthcoming work.

We show that in such a case, the following simple repetition-based scheme is optimal.  We repeatedly transmits each source symbol uncoded until we are alerted  that user~1 has received it, which is clearly optimal for user~1.  To see that it is also optimal for user~2, we note that since the channel is physically degraded, a received symbol by user~2's channel at time $t$ implies that user~1 has also received the same symbol at time $t$. Thus, upon the reception of any source symbol by user~2, the transmitter will begin transmitting a different instantly decodable, distortion-innovative symbol.  It is therefore not hard to see that user~2 can also achieve an optimality.
\fi



\section{Source-broadcast with One-sided Feedback}
\label{sec:one_sided_feedback}

In this section we consider a one-sided-feedback variation of the problem in Section~\ref{sec:two_users} whereby in a broadcast network with two receivers, a feedback channel is available to only the stronger user.  In this scenario, we show that given the distortion constraints of both users, there is no overhead in the minmax latency achieved.  Specifically, for $i \in \{1, 2\}$, let $\wid$ be the point-to-point optimal latency for user~$i$ and let $\wplusd$ be the Shannon lower bound for the minmax latency where

\begin{align}
\label{eq:wid_optimal}
	\wid = \frac{1 - d_i}{1 - \epsilon_i},
\end{align}
and
\begin{align}
\label{eq:wplusd}
	\wplusd = \max_{i \in \{1, 2\}} \frac{1 - d_i}{1 - \epsilon_i}.
\end{align}
Section~\ref{sec:two_users} showed that for $i \in \{1, 2\}$, user~$i$ can achieve distortion $d_i$ at the optimal latency $\wid$ when a feedback channel is available to \emph{both} users (c.f.\ Section~\ref{sec:individual_latencies} on individual decoding delays).  Clearly, the optimal minmax system latency $\wplusd$ is also achievable in this case.  In contrast, in this section we show that when a feedback channel is available to \emph{only the stronger user}, while the \emph{individual} optimal latencies may or may not be achievable, the overall system latency $\wplusd$ is still achievable.

% achieve point-to-point optimal performance when a feedback channel is available to \emph{both} receivers, in this section we show that point-to-point optimal performance may also be achieved if \emph{only the stronger user} has a feedback channel available.

% \documentclass[ex_article]{subfiles}
% \begin{document}
\section{Problem Formulation}\label{sec:problem}
%\subsection{Linear system}
We consider the linear time-invariant (LTI) system
\begin{align}
  \begin{aligned}
    \dot x(t) = Ax(t)+Bu(t), \quad
    y(t)     = C x(t),\quad
    x(0) \sim \mathcal D,
  \end{aligned}\label{eq:system}
\end{align}
where $x(t) \in \R^n$ is state, $u(t)\in \R^m$ is input,
$y(t)\in \R^p$ is output, $A\in \R^{n\times n}$,
$B\in \R^{n\times m}$, and $C \in \R^{p\times n}$ are constant matrices, and $\mathcal D$ is a probability distribution over $\R^n$.
In this paper, we assume that $B$ and $C$ are not zero matrices,
and
$(A, B, C, \mathcal {D})$
is unknown unlike the situation in \cite{fatkhullin2021optimizing}.
%\subsection{LQR problem with structured constraints} \label{Sec2-B}
The infinite-horizon continuous-time LQR problem is formulated as
\begin{align}
  \minimize  & E_{x(0)\sim \mathcal{D}}\qty[\int_0^\infty \qty(y^\top(t) Q y(t) + u^\top(t) R u(t))dt ] \label{eq:objectivefunction} \\
  \subjectto & ~\eqref{eq:system}
\end{align}
with constant positive definite matrices $Q \in \R^{p\times p}$ and $R\in \R^{m\times m}$.
The expectation is taken with respect to the initial state $x(0) \sim \mathcal{D}$.
For the static output feedback $u(t) = -Ky(t)$ with $K\in \R^{m\times p}$ to system~\eqref{eq:system},
the objective function~\eqref{eq:objectivefunction} becomes
$
  f(K) := E_{x(0)\sim \mathcal{D}}\qty[\tilde{f}(K;x(0))],
$
where 
\begin{align}
  \tilde f(K;v) & := \int_0^\infty \qty[v^* e^{A_K^\top t}C^\top(Q+ K^\top RK)Ce^{A_K t}v]dt\label{eq:cost}
\end{align}
for $v\in \C^n$.
Then, the closed-loop is given by
\begin{align}
    \dot x(t)  = A_Kx(t),\quad
    y(t)          = Cx(t), \label{eq:closedloop}
\end{align}
where 
\begin{align}
  A_K := A-BKC. \label{eq:AK}
\end{align}



In this paper, we consider the constraints $K\in \Omega$,
where $\Omega \subset \R^{m\times p}$ is a closed convex set
that specifies the structural information of feedback gains.
This is because a structured policy is often used in practical situations.
For example,
%\subsubsection{Decentralized control}
\begin{itemize}
    \item Decentralized control: In decentralized control, some components of $K$ need to be $0$~\cite{jovanovic2016controller}.
This implies that $\Omega$ should be a certain linear subspace of $\R^{m\times p}$.

\item Linear port-Hamiltonian system: For a linear port-Hamiltonian system~\cite{Jacob2012linear}, 
if the feedback gain is positive semi-definite, the closed loop system is also a port-Hamiltonian system and passive. To ensure passivity, $\Omega$ should be defined as the set of positive semi-definite matrices, which is closed and convex.

\end{itemize}
%\subsubsection{Linear port-Hamiltonian system}



%See Section~\ref{sec:examples}.


By using Bellman lemma~\cite{bellman1957notes}, the problem~\eqref{eq:objectivefunction} with structured constraints can be formulated as
\begin{align}
  \begin{aligned}
    \minimize_K & f(K) = \tr(X\Sigma) \\
    \subjectto  & K \in \Omega\,\, \text{and}\,\, A_K\text{ is \textit{Hurwitz}},
  \end{aligned}\label{eq:problem}
\end{align}
where
$
  \Sigma  := E[x(0)x^\top(0)]
$
and $X$ is the solution to
\begin{align}
  A_K^\top X + XA_K + C^\top \qty(K^\top RK + Q)C = 0.
\end{align}
It is difficult to solve \eqref{eq:problem}, since  $f(K)$ is non-convex and saddle points may exist~\cite{fatkhullin2021optimizing}.
Moreover,
 the feasible set may have exponentilally many disconnected components~\cite{feng2019exponential}. 
Although an iterative method was proposed in \cite{zhu2015adaptive} to obtain a suboptimal static output feedback gain in the model free setting,
it cannot be applied directly to problem \eqref{eq:problem} due to the constraint $K\in \Omega$.

% In this paper, we suppose that $(A, B, C, \mathcal {D})$ in \eqref{eq:system}
% is unknown unlike the situation in \cite{fatkhullin2021optimizing}.
% Thus, we develop a model free algorithm in Section \ref{sec:model-free} for solving the problem \eqref{eq:problem}.


To develop a model free algorithm with theoretical guarantees for solving problem \eqref{eq:problem}, we impose the following throughout this paper:
\begin{assumption}\label{assume:sigma-hurwitz}
  \indent
  \begin{enumerate}
    \item $\Sigma \succ 0$.
    \item The pair $(A, C)$ is observable.
    \item There exists $K_0 \in \Omega$
          such that $A_{K_0}$ is \textit{Hurwitz}
          and $K_0$ is known.
  \end{enumerate}
\end{assumption}


Since $A_{K_0}$ is \textit{Hurwitz},
there exist positive definite matrices $G, H$ and a skew-adjoint matrix $J$
such that
$
  A_{K_0} = (J-G)H.
$
The proof is found in~\cite{prajna2002lmi}.
Let $H = L^\top L$ be the Cholesky decomposition.
Using the coordinate transformation $x'(t) = Lx(t)$, the closed-loop system~\eqref{eq:closedloop} becomes
\begin{align}
  \dot{x'}(t)  = A'_{K_0}x'(t), \quad 
  y(t)         = C'x'(t),
\end{align}
where $A'_{K_0}  = LJL^\top-LGL^\top, C' = CL^{-1}$.
Since $LGL^\top\succ 0$ and $LJL^\top = -(LJL^\top)^\top$, we have
  $A'_{K_0}+{A'_{K_0}}^\top = -2LGL^\top \prec 0$.
In the following, we assume system \eqref{eq:closedloop} after the above coordinate transformation, because we consider a static output feedback that is invariant by the coordinate transformation. That is, without loss of generality, we can assume
$A_{K_0}+A_{K_0}^\top \prec 0$.

Under Assumption \ref{assume:sigma-hurwitz},
 $f(K)$ of \eqref{eq:problem} is defined only on the set $S$ of stabilizing controllers, which is defined as
\begin{align}
  S = \{K\in \R^{m\times p}\mid A_K\text{ is \textit{Hurwitz}}\}. \label{eq:S}
\end{align}
If $K\notin S$, there exists an eigenvalue $\mu$ of $A$ such that ${\rm Re}(\mu) \geq 0$ and $f(K)$ goes to infinity.

\begin{remark} \label{remark1}
The objective function of
 problem \eqref{eq:objectivefunction}
is not a
standard LQR cost
%is given by
% \begin{align}
%     \int_0^\infty \qty(x^\top(t) Q x(t) + u^\top(t) R u(t))dt,
% \end{align}
% where $Q \in \R^{n\times n}$ and $R\in \R^{m\times m}$ are positive definite matrices. However, in the model free and output feedback setting, this cost cannot be calculated in practice, because the information of the state $x(t)$ is not available. Therefore, we consider the problem \eqref{eq:objectivefunction} following
as in some previous researches~\cite{modares2016optimal, rizvi2018output}. While similar convergence properties to the standard LQR cost can be obtained for our formulation in the model based setting if $(A, C)$ is observable~\cite{fatkhullin2021optimizing}, more detailed studies of the objective function properties are necessary for model-free version of the convergence analysis.  
\end{remark}



%\subsection{Examples}\label{sec:examples}


% \end{document}

\subsection{Coding Scheme}

We begin by reviewing a repetition coding scheme in the next subsection, whereby we simply ignore the weaker user, and focus on using the stronger user's feedback to retransmit each of his required source symbols until it is received.  A repetition coding scheme is useful insofar as it helps avoid compelling the weaker user to decode additional source symbols that he does not require.  

As an example, consider when the stronger user requires $N$ source symbols.  We could send random linear combinations of the $N$ symbols, which he could decode after receiving $N$ equations through, say, $W$ transmissions.  By the time the stronger user has recovered the $N$ equations, the weaker user, having a weaker channel, would have received less than $N$ equations.  At this point, the weaker user could simply ignore the first $W$ transmissions, and have the transmitter encode another random linear combination of symbols for the weaker user to decode.  However, such a timesharing scheme is inefficient.  
%
On the other hand, the weaker user could prevent the first $W$ transmissions from going to waste by continuing to receive random linear combinations of the group of $N$ source symbols originally intended for the stronger user.  However, if the weaker user requires $M < N$ symbols, he would have had to listen to many more transmissions than necessary to recover $M$ symbols thus introducing delay.

We notice the problem is in the random linear combinations used in our coding scheme.  In such a scheme, we either receive more than $N$ equations and decode the entirety of the $N$ source symbols, or we receive less than $N$ equations and decode none of the source symbols.  This ``threshold effect'' is detrimental when we have heterogeneous users in a network who require $M < N$ source symbols.  

%The weaker user would similarly need to receive $N$ equations to decode the $N$ source symbols, however, if he only requires $M < N$ symbols, he would have had to listen to many more transmissions than necessary to recover $M$ symbols.  That is, 

%Due to the threshold effect of channel coding, if we send 

The repetition coding scheme avoids this pitfall by avoiding random linear combinations altogether and instead transmitting uncoded source symbols over the channel.  While this avoids compelling the weaker user to decode unnecessary source symbols, it can also be inefficient for the weaker user.  Specifically, since the repetition scheme is based solely on the stronger user's feedback, a source symbol can be retransmitted even after it is received by the weaker user.  We show how to circumvent this problem by creating a hybrid coding scheme that consists of both repetitions, and random linear combinations.  The coding scheme is controlled by two variables, $\omegaparam$, and $\gammaparam$.  We show how to choose specific values for these parameters to achieve the optimal minmax latency in Section~\ref{sec:one_sided_optimality}.

%We then show how to incorporate random linear combinations 

\input{one_sided/repetition_coding}
\subsubsection{Inner Bound}
\label{subsubsec:inner_bound}

In this section, we formulate a hybrid coding scheme that incorporates both repetition coding and sending random linear combinations of source symbols.  The code is tuned via two parameters, $\omegaparam, \gammaparam \in [0, 1]$.  We target point-to-point optimal performance for the stronger user in this section, and show that this is possible for any values of $\omegaparam$ and $\gammaparam$.  In the next section however, we show how to optimize $\omegaparam$ and $\gammaparam$ to achieve the optimal minmax latency.  As in the coding scheme in Section~\ref{sec:two_users}, we again split the coding scheme into phases.  

In Phase~\Rmnum{1}, we begin by sending each source symbol uncoded over the channel.  That is, Phase~\Rmnum{1} consists of $N$ transmissions and at time $t \in \{1, 2, \ldots, N\}$, we transmit $X(t) = S(t)$.  Let $\Aset \subseteq \{S(1), S(2), \ldots, S(N)\}$ be the set of symbols received by the stronger user in Phase~\Rmnum{1}.  Since the stronger user's feedback is available to all receivers and the transmitter, $\Aset$ is known to all parties.  
%We have that $\mathbb{E}|A| = 1 - \epsilon_1$ 

At the conclusion of Phase~\Rmnum{1}, we have that on average, for $i \in \{1, 2\}$, user~$i$ will have received $N(1 - \epsilon_i)$ source symbols and so will require an additional $N(\epsilon_i - d_i)$ symbols in the remaining phases.   Before moving on to Phase~\Rmnum{2}, we first organize the source symbols in $\Aset$ and $\AsetC$ into subsets, where $\AsetC \subseteq \{S(1), S(2), \ldots, S(N)\}$ denotes the complement of set $\Aset$. We first isolate a fraction of $N(\epsilon_1 - d_1)$ source symbols from $\AsetC$ into a set denoted as $\Bset$.  That is, we fix the remaining $N(\epsilon_1 - d_1)$ symbols that the stronger user requires in $\Bset$.  We then partition $\Bset$ into two disjoint sets, one that contains a fraction of $\omegaparam \in [0, 1]$ source symbols from $\Bset$, denoted as $\BsetOmega$, and the other that contains the remaining fraction of $1 - \omegaparam$ symbols, denoted as $\BsetOmegaC$, where $\Bset = \BsetOmega \cup \BsetOmegaC$.  Random linear combinations of the symbols in $\BsetOmega$ will be sent to the stronger user while the symbols in $\BsetOmegaC$ will be sent with repetition coding.  We further take a fraction of $\gammaparam \in [0, 1]$ source symbols from $\Aset$ and denote this set as $\Cset$.  Finally, we define $\Fset$ as the union of sets $\Cset$ and $\BsetOmega$.  Figure~\ref{fig:set_construction} illustrates the relationship between all sets and the manner in which they are constructed.

\begin{figure}
	\centering
%	\input{3/fig/schematic}
%	\includegraphics[scale=0.8]{3/one_sided/fig/system_model_one_sided}
	\includegraphics[scale=1]{one_sided/fig/set_construction}
%%	\includegraphics[width=0.6\textwidth]{outer_bound.png}
	\caption{A tree diagram illustrating the relationship between the sets of source symbols.  Each node represents a set of source symbols and a directed edge $(X,Y)$ indicates that set $Y$ is a subset of $X$.  If edge $(X, Y)$ is also \emph{weighted}, then the weight represents the expected cardinality of set $Y$.  Only sets involved in the coding scheme have incoming \emph{weighted} edges.  The direct successors of a node form a partition for the set representing the parent node.  The root of the tree is the entire source sequence, which is subsequently partitioned at each depth of the tree.  We also show set $\Fset$, which is the union of $\Cset$ and $\BsetOmega$.}
	\label{fig:set_construction}
\end{figure}

In Phase~\Rmnum{2} of the coding scheme, we designate the symbols of $\BsetOmegaC$ as the symbols to be transmitted to the stronger user with a repetition scheme.  However, we modify the repetition scheme to incorporate random linear combinations of symbols in $\Fset$.  In a conventional repetition scheme, we would retransmit $\bOmegaC \in \BsetOmegaC$ until it is received by the stronger user.  Upon reception by the stronger user, we move on to the next symbol $\bOmegaC' \in \BsetOmegaC$ and continue in this manner until all symbols in $\BsetOmegaC$ are accounted for.  Let $\bOmegaT \in \BsetOmegaC$ be the source symbol being repeated at time $t$.  Our modified coding scheme is similar to the conventional repetition scheme except that at any time $t$, instead of \emph{only} transmitting $\bOmegaT$, we instead send $v(t) + \bOmegaT$, where $v(t)$ is a \emph{new} random linear combination of the source symbols in $\Fset$ generated for every time $t$.  Let  $\bOmegaC \in \BsetOmegaC$.  If $\bOmegaT = \bOmegaC$ is transmitted and subsequently received by the stronger user at time $t$, the protocol for replacing $\bOmegaC$ at time $t+ 1$ with another source symbol from $\BsetOmegaC$ is identical to the conventional repetition scheme, however the only difference is that we now combine $\bOmegaT$ with a random linear combination of the symbols of $\Fset$ at every transmission.  Phase~\Rmnum{2} concludes when all symbols in $\BsetOmegaC$ have been accounted for by the modified repetition scheme.

\begin{remark}
	When applying a random linear code, we use the maximum distance separable (MDS)-type property that any collection of $N$ channel symbols gives $N$ linearly independent equations.  Although strictly speaking such codes do not exist over the binary field, randomly chosen combinations over long blocks are approximately MDS~\cite{TTAAJ17}.
%	when we use a random linear code, we will use MDS type property that any collection of N symbols gives N linearly independent equations. Although strictly speaking such codes do not exists over the binary field, randomly chosen combinations over long blocks are approximately MDS. You can cite a reference such as the following one: https://link.springer.com/chapter/10.1007/978-3-319-51103-0_14
\end{remark}

At the conclusion of Phase~\Rmnum{2}, since we have transmitted the symbols in $\BsetOmegaC$ as if we were utilizing a repetition scheme, we have that on average, the stronger user will have received $|\BsetOmegaC|$ equations involving $|\BsetOmegaC| + |\Fset|$ variables.  Notice, however, that since $\Fset \triangleq \Cset \cup \BsetOmega$, and $\Cset \subseteq \Aset$, the stronger user can subtract off all symbols originating from $\Cset$.  Therefore, Phase~\Rmnum{1} actually results in the stronger user receiving $|\BsetOmegaC|$ equations involving $|\BsetOmegaC| + |\BsetOmega| = |\Bset|$ \emph{unkown} variables, where $\mathbb{E}|\Bset| = N(\epsilon_1 - d_1)$.  The stronger user therefore requires an additional $|\BsetOmega|$ equations at the conclusion of Phase~\Rmnum{2}.

In Phase~\Rmnum{3}, we send the remaining equations to the stronger user by continuing to send $v(t)$ at any time $t$.  That is, we continue sending random linear combinations of the symbols in $\Fset$.  Phase~\Rmnum{3} concludes when the feedback of the stronger user indicates that he has received the missing $|\BsetOmega|$ equations.

At the conclusion of Phase~\Rmnum{3}, it is not hard to see that the stronger user achieves point-to-point optimal performance, since every channel symbol received has provided an independent equation that can be used to decode a new source symbol.  
%For $i \in \{1, 2\}$, let $w_{i}(d_i) = (1 - d_i)/(1 - \epsilon_i)$ represent the point-to-point optimal latency for user~$i$.  
At this point, if $\wtwo \leq \wone$, we halt any further transmissions, where $\wid$ is given by~\eqref{eq:wid_optimal}.

In Phase~\Rmnum{4}, if $\wtwo > \wone$, we continue to transmit $v(t)$, the random linear combinations of the source symbols in $\Fset$, for an additional $N(\wtwo- \wone)$ transmissions.





%only symbols providing new information are those originating from $\BsetOmega$
%
% involving $|\Bset|$ symbols, where $\mathbb{E}|\BsetOmegaC| = N(1 - \omegaparam)(\epsilon_1 - d_1)$.  We therefore need to send another $N\omegaparam(\epsilon_1 - d_1)$ equations to the stronger user.  
%
%In Phase~\Rmnum{3}, we send the remaining equations necessary for the stronger user to decode by continuing to send $v(t)$, random linear combinations of the source symbols in $\Fset$, until $N\omega(\epsilon_1 - d_1)$ symbols are received.  
%
%
%
%
%
%we send random linear combinations of the source symbols in $\Fset \triangleq \Cset \cup \BsetOmega$.  Let $v(t)$ be a \emph{new} random linear combination of the source symbols in $\Fset$ generated at time $t$.  Since $\Cset \subseteq \Aset$, the stronger user can subtract off all symbols originating from $\Cset$.  Therefore, the only symbols providing new information are those originating from $\BsetOmega$.  We use the stronger user's feedback to continue to send random linear combinations until the stronger user has received $|\BsetOmega|$ equations, where $\mathbb{E}|\BsetOmega| = \omegaparam(\epsilon_1 - d_1)$.  At the conclusion of Phase~\Rmnum{2}, by construction, the stronger user has decoded all symbols in $\Fset$.
%
%In Phase~\Rmnum{3}, we use a repetition coding scheme to send each symbol $\bOmegaC \in \BsetOmegaC$.  However, since all symbols in $\Fset$ have been decoded by the stronger user at the beginning of Phase~\Rmnum{3}, notice that we can also combine the symbol $\bOmegaC$ with a random linear combination of the symbols in $\Fset$ to the benefit of the weaker user, but at no extra cost to the stronger user.  That is, let $\bOmegaC$ be a symbol that would be sent with a repetition scheme at time $t$.  Then instead of transmitting $\bOmegaC$ at time $t$, we instead transmit $v'(t) = v(t) + \bOmegaC$ where $v(t)$ is a \emph{new} random linear combination of the source symbols in $\Fset$, which can be subtracted off by the stronger user.
%As with a conventional repetition scheme, we also repeat $\bOmegaC$ until it is received by the stronger user for all $\bOmegaC \in \BsetOmegaC$.  However, the difference in Phase~\Rmnum{3} is that we also combine $\bOmegaC$ with $v(t)$, a different linear combination of the source symbols in $\Fset$ at every retransmission.  At the conclusion of Phase~\Rmnum{3} it is not hard to see that the stronger user achieves point-to-point optimal performance, since every channel symbol received has lead to the decoding of a source symbol.  
%
%Finally, in Phase~\Rmnum{4}, if the weaker user has not yet met his distortion constraint, we continue to send random linear combinations of the source symbols in $\Fset$.

%instead of sending the random linear combination $v(t)$ at time $t$ as in Phase~\Rmnum{2}, we instead transmit $v'(t) = v(t) + \bOmegaC$ where $\bOmegaC \in \BsetOmegaC$ is a source symbol that has yet to be received by the stronger user.  Since the stronger user has decoded all symbols in $\Fset$ from the previous phase, he can subtract off $v(t)$ from $v'(t)$ at any time $t$ in Phase~\Rmnum{3}. 




\subsection{Minmax Latency Optimality}
\label{sec:one_sided_optimality}

In this section, we show that it is possible to choose values for $\omegaparam, \gammaparam \in [0, 1]$ from Section~\ref{subsubsec:inner_bound} so that the lower bound for the minmax latency in~\eqref{eq:wplusd} is achieved.  We first calculate the expected number of \emph{unknown} variables involved in transmissions to the weaker user from Phase~\Rmnum{2} onwards.  

First, since we send random linear combinations of the symbols in $\Fset$ in Phase~\Rmnum{2}, we initially expect this to contribute $|\Fset|$ variables.  However, some of the symbols in $\Fset$ have already been received by user~2 in Phase~\Rmnum{1}.  Let $\nFUnknown$ be the number of symbols in $\Fset$ \emph{not} received by user~2 in Phase~\Rmnum{1}.  Given a channel noise realization $(Z_{1}^{W}, Z_{2}^{W}) = (z_{1}^{W}, z_{2}^{W})$, we can calculate the expected value of $\nFUnknown$ as

\setcounter{cnt}{1}
\begin{align}
	\mathbb{E}(\nFUnknown | (Z_{1}^{W}, Z_{2}^{W}) = (z_{1}^{W}, z_{2}^{W})) &= \sum_{s \in \Fset} \textrm{Pr}(\textrm{$s$ not received by user~2 in Phase~\Rmnum{1}})\\
	&\stackrel{(\alph{cnt})}{=} \sum_{s \in \Cset} \textrm{Pr}(\textrm{$s$ not received by user~2 in Phase~\Rmnum{1}}) \\ \nonumber
	&\qquad  + \sum_{s' \in \BsetOmega} \textrm{Pr}(\textrm{$s'$ not received by user~2 in Phase~\Rmnum{1}})\\ 
	\addtocounter{cnt}{1}
	&\stackrel{(\alph{cnt})}{=} \sum_{s \in \Cset} \textrm{Pr}(Z_2 = 1 | Z_1 = 0) + \sum_{s' \in \BsetOmega} \textrm{Pr}(Z_2 = 1 | Z_1 = 1) \\ 
	\addtocounter{cnt}{1}
	&\stackrel{(\alph{cnt})}{=} \sum_{s \in \Cset} \left(\frac{\epsilon_2 - \epsot}{1 - \epsilon_1}\right) + \sum_{s' \in \BsetOmega} \left( \frac{\epsot}{\epsilon_1}\right) \\ 
	\addtocounter{cnt}{1}
%	&\stackrel{(\alph{cnt})}{=}  N\left( \omegaparam(\epsilon_1 - d_1) + \gamma(1 - \epsilon_1) \right)\epsilon_2
	\label{eq:last_line_set}
	&= |\Cset| \left(\frac{\epsilon_2 - \epsot}{1 - \epsilon_1} \right) + |\BsetOmega|\left( \frac{\epsot}{\epsilon_1}\right),
\end{align}
where 
\begin{enumerate}[(a)]
	\item follows from the fact that $\Fset = \Cset \cup \BsetOmega$ and $\Cset$ and $\BsetOmega$ are disjoint by construction
	\item follows from the fact that by construction, all symbols in $\Cset$ have been received by user~1 and all symbols in $\BsetOmega$ were not received by user~1
	\item we have calculated the conditional probabilities from~\eqref{eq:pmf_z1z2}.
\end{enumerate}

The cardinality of sets $\Cset$ and $\BsetOmega$ depends on the channel noise variables $(Z_{1}^{W}, Z_{2}^{W})$.  By taking the expectation over the channel noise, we can calculate the unconditional expected value of $\nFUnknown$ as

\setcounter{cnt}{1}
\begin{align}
	\label{eq:nFUnknown}
	\mathbb{E}\nFUnknown &\stackrel{(\alph{cnt})}{=} N \gammaparam (1 - \epsilon_1) \left(\frac{\epsilon_2 - \epsot}{1 - \epsilon_1} \right) + N \omegaparam(\epsilon_1 - d_1) \left( \frac{\epsot}{\epsilon_1}\right),
\end{align}
where 
\begin{enumerate}[(a)]
	\item follows from~\eqref{eq:last_line_set} and by construction of the sets (see Section~\ref{subsubsec:inner_bound} and Figure~\ref{fig:set_construction}).
\end{enumerate}

%the probability that any symbol in $\Fset$ was not already received by the weaker user in Phase~\Rmnum{1} is equal to $\epsilon_2$.  Therefore the expected number of \emph{unknown} variables within $\Fset$ is given by $\nFUnknown$, where
%
%\setcounter{cnt}{1}
%\begin{align}
%	\mathbb{E}\nFUnknown &= \mathbb{E}|\Fset|\epsilon_2 \\
%	&= \mathbb{E}|\BsetOmega \cup \Cset|\epsilon_2 \\
%	&\stackrel{(\alph{cnt})}{=} \left( \mathbb{E}|\BsetOmega| + \mathbb{E} |\Cset| \right) \epsilon_2	 \\
%	\addtocounter{cnt}{1}
%	\label{eq:nFUnknown}	
%	&\stackrel{(\alph{cnt})}{=}  N\left( \omegaparam(\epsilon_1 - d_1) + \gamma(1 - \epsilon_1) \right)\epsilon_2
%\end{align}
%where 
%\begin{enumerate}[(a)]
%	\item follows from the fact that $\BsetOmegaC$ and $\Cset$ are disjoint by construction
%	\item follows by construction (see Section~\ref{subsubsec:inner_bound} and Figure~\ref{fig:set_construction}).
%\end{enumerate}

The use of repetition coding for the symbols in $\BsetOmegaC$ in Phase~\Rmnum{2} further adds additional unknown variables to the coding scheme.  On average, the expected number of symbols repeated is $\mathbb{E}|\BsetOmegaC| = N(1  - \omegaparam)(\epsilon_1 - d_1)$, of which, again, only a fraction of $\textrm{Pr}(Z_2 = 1 | Z_1 = 1) = \epsot/\epsilon_1$ were not already received by the weaker user in Phase~\Rmnum{1}.  By Lemma~\ref{lem:repetition}, the number of additional \emph{unknown} variables introduced to the weaker user as a result of the repetition scheme is therefore given by $\nBUnknown$, where

\setcounter{cnt}{1}
\begin{align}
	\label{eq:nBUnknown}
	\mathbb{E}\nBUnknown &= N(1  - \omegaparam)(\epsilon_1 - d_1)\left(\frac{\epsot}{\epsilon_1}\right) \left(\frac{1 - \epsilon_2}{1 - \epsot} \right).
\end{align}

Let $\LHSfuncparam$ be the expected fraction of all source symbols that are involved in transmissions to the weaker user from Phase~\Rmnum{2} onwards that have not yet been decoded prior to Phase~\Rmnum{2}.  We have that $\LHSfuncparam$ is the normalized sum of~\eqref{eq:nFUnknown} and~\eqref{eq:nBUnknown}, i.e., 

\begin{align}
	\LHSfuncparam &= \frac{\mathbb{E}\nFUnknown + \mathbb{E}\nBUnknown}{N} \\
	&= \kgamma \gammaparam + \komega \omegaparam	+ \kk,
	\label{eq:LHSfuncparam}
\end{align}
where 
\begin{subequations}
\begin{align}
	\kgamma &= \epsilon_2 - \epsot, \label{eq:kgamma} \\	
	\komega &= (\epsilon_1 - d_1) \left(\frac{\epsot}{\epsilon_1}\right) \left(\frac{\epsilon_2 - \epsot}{1 - \epsot} \right), 	\label{eq:komega}\\
	\kk &= (\epsilon_1 - d_1) \left(\frac{\epsot}{\epsilon_1}\right) \left(\frac{1 - \epsilon_2}{1 - \epsot} \right).
	\label{eq:kk}
\end{align}
\end{subequations}

Having calculated the number of unknown variables sent to the weaker user from Phase~\Rmnum{2} onwards, we now consider the number of equations he receives in Phases~\Rmnum{2} and~\Rmnum{3}.  From Section~\ref{subsubsec:inner_bound}, we know that during these phases, the total number of transmissions was simply equal to the number of trials needed to send $N(\epsilon_1 - d_1)$ equations to the stronger user with feedback.  The number of transmissions in Phases~\Rmnum{2} through~\Rmnum{3}  is therefore distributed according to a negative binomial distribution and the mean number of transmissions in this period is $W_{2,3} = N(\epsilon_1 - d_1)/(1 -\epsilon_1)$.  Of these transmissions, the expected number received by the weaker user is equal to $W_{2, 3}(1 - \epsilon_2)$.  We rewrite the expression for $W_{2, 3}(1 - \epsilon_2)$, the expected number of transmissions received by user~2 in Phases~\Rmnum{2} through~\Rmnum{3}, as $NC_{2, 3}$ where

%We therefore define the capacity of the weaker user's channel during Phases~\Rmnum{2} through~\Rmnum{3} as $C_{2,3}$, where
\begin{align}
	C_{2,3} = \frac{(\epsilon_1 - d_1)(1 - \epsilon_2)}{1 - \epsilon_1}.
%	\bstar &= \frac{(1 - \epsilon_1) + d_1(1 - \epsilon_2)}{1 - \epsilon_1\epsilon_2}
\end{align}
%and the expected number of equations received by the weaker user in Phases~\Rmnum{2} through~\Rmnum{3} is given by $NC_{2, 3}$.  
We next compare $N\LHSfuncparam$, the amount of source symbols destined for the weaker user, with $NC_{2, 3}$,  the expected number of equations received over the weaker user's channel during Phases~\Rmnum{2} and~\Rmnum{3}.  

As mentioned in Section~\ref{subsubsec:inner_bound}, the weaker user requires an additional $N(\epsilon_2 - d_2)$ symbols to be sent from Phase~\Rmnum{2} onwards.  Therefore, it is necessary that $\LHSfuncparam \geq \epsilon_2 - d_2$.  However, if $\LHSfuncparam$ is much greater than $\epsilon_2 - d_2$, we encounter the problem explained in the introduction of this section in which the weaker user is forced to decode unnecessary symbols thus introducing delay.  Say that we are able to find values of $\gammaparam', \omegaparam' \in [0, 1]$ such that $\LHSfuncprime = \epsilon_2 - d_2$.  We consider two cases when this is so -- when $\LHSfuncprime \leq C_{2, 3}$ and when $\LHSfuncprime > C_{2, 3}$.  We show that in both cases, we can achieve the optimal minmax  latency so long as $\LHSfuncprime = \epsilon_2 - d_2$.

In the first case, when $\LHSfuncprime \leq C_{2, 3}$, we wish to send less information over the channel than what the channel can support.  Therefore, we expect that the weaker user should be able to decode all source symbols before the conclusion of Phase~\Rmnum{3}.  However, in general, if the weaker user achieves distortion $d_2$ after decoding, it will be at a latency $w$, where $w > \wtwo$.  That is, in general, the weaker user may not achieve an \emph{individual} point-to-point optimal latency.

To see why this is so, recall from Section~\ref{subsubsec:inner_bound} that in Phase~\Rmnum{2} of our coding scheme, we transmit $\bOmegaT + v(t)$ at every time instant $t$, where $v(t)$ is a new random linear combination of the source symbols in $\Fset$ generated at every time $t$.  Since $\LHSfuncprime \leq C_{2, 3}$, there is the possibility that at some point, the weaker user is able to decode all symbols in $\Fset$ even before Phase~\Rmnum{2} has concluded.  Say that this is the case and the stronger user has stalled on receiving a particular symbol $\bOmegaC \in \BsetOmegaC$ being repeated.  Let us further assume that the weaker user has already received $\bOmegaC$.  Then while $\bOmegaC + v(t)$ is being transmitted, all transmissions to the weaker user are redundant.
%and thus the reception of a channel symbol will not lead to the acquisition of an independent equation that can be used to decode an additional source symbol.  
After $\bOmegaC$ is received by the stronger user and the transmitter moves on to  $\bOmegaPrime$, the next symbol in $\BsetOmegaC$ to be sent via the modified repetition scheme, the weaker user can continue to receive innovative information.  However, the set of transmissions received while $\bOmegaC$ is being repeated prevents the weaker user from achieving an optimal \emph{individual} latency.

However, we show that the optimal \emph{minmax} latency can still be achieved.  Notice that the moment all symbols in $\Fset$ can be decoded by the weaker user, the random linear combination $v(t)$ can be subtracted from any transmission $\bOmegaT + v(t)$ in Phase~\Rmnum{2}.  Therefore the remainder of Phase~\Rmnum{2} effectively consists of uncoded transmissions from the weaker user's perspective, and he is eventually able to decode all $\LHSfuncprime$ symbols.  Thus, so long as $\LHSfuncprime = \epsilon_2 - d_2$, the weaker user will decode the necessary amount of symbols before the conclusion of Phase~\Rmnum{3}, while the stronger user decodes at an optimal latency the moment Phase~\Rmnum{3} terminates.  In this case, the stronger user is the bottleneck of the system and in fact, the condition $\LHSfuncprime \leq C_{2, 3}$ is equivalent to $\wone\geq \wtwo$.  

%For example, say that $\bOmegaPrime$ is the last symbol in $\BsetOmegaC$ about to be sent via the modified repetition coding at time $t_2$.  

%Although the weaker user receives $NC_{2, 3}$ equations during Phases~\Rmnum{2} through~\Rmnum{3}, it may not necessarily be the case that all equations are independent.  Recall from Section~\ref{subsubsec:inner_bound} that in Phase~\Rmnum{2} of our coding scheme, we transmit $\bOmegaT + v(t)$ at every time instant $t$, where $v(t)$ is a new random linear combination of the source symbols in $\Fset$ generated at every time $t$.  There is the possibility that at some point, the weaker user is able to decode all symbols in $\Fset$ even before Phase~\Rmnum{2} has concluded.  

%Notice however, that the moment the weaker user can decode all source symbols in $\Fset$, the only remaining symbols missing are those in $\BsetOmegaC$.  Therefore, during Phase~\Rmnum{2} when $\bOmegaT + v(t)$ is being repeated, the weaker user can subtract off $v(t)$, and can thereby 

%may not necessarily lead to an \emph{independent} equation received.  

%Now, although the weaker user receives $NC_{2, 3}$ equations during Phases~\Rmnum{2} through~\Rmnum{3}, it may not necessarily be the case that all equations are independent.  Recall from Section~\ref{subsubsec:inner_bound} that in Phase~\Rmnum{2} of our coding scheme, we transmit $\bOmegaT + v(t)$ at every time instant $t$, where $v(t)$ is a new random linear combination of source symbols in $\Fset$ generated at every time $t$.  There is the possibility that at some point, the weaker user is able to decode all symbols in $\Fset$ even before Phase~\Rmnum{2} has concluded.  If this is the case and the stronger user has stalled on receiving a particular symbol being repeated that the weaker user has already received, then all transmissions to the weaker user will be redundant, and thus the reception of a channel symbol will not lead to the acquisition of an independent equation that can be used to decode an additional source symbol.  

%we will have received more equations than unknown variables and so the weaker user can decode at the conclusion of Phase~\Rmnum{3}.  In this case, the stronger user is the bottleneck of the system and the condition $\LHSfuncprime \leq C_{2, 3}$ is equivalent to $w_{1}(d_1) \geq w_2(d_2)$.  

On the other hand, if $\LHSfuncprime > C_{2, 3}$, the weaker user has more unknown variables than equations and so he cannot yet decode at the conclusion of Phase~\Rmnum{3}.  However, every transmission he has received so far is ``innovative'' in the sense that it provides an independent equation that can be used to decode the $N\LHSfuncprime$ source symbols.  In order to decode, we simply need to send additional equations to the weaker user, and since $\LHSfuncprime = \epsilon_2 - d_2$, there will not be any unnecessary source symbols sent.  Since the stronger user is point-to-point optimal at the conclusion of Phase~\Rmnum{3}, we have therefore sent a total of $N\wone$ transmissions up to that point.  Since, from the weaker user's perspective, we have hitherto been sending random linear combinations of $N\LHSfuncprime$ variables, we simply need to continue doing so for another $N(\wtwo - \wone)$ transmissions in Phase~\Rmnum{4} before he receives the remaining number of equations required and achieves point-to-point optimal performance.

\begin{table}
	\begin{center}
		\begin{tabular}{| c | c |}
%    \caption{The justification for the ordering}
			\hline
			\multicolumn{2}{|c|}{{\bf Ordering of Boundaries for $d_1$}} \\
			\hline
			{\bf Inequality} & {\bf Justification}   \\ \hline
			$\donedaggall < \doneddaggall$ & $(1 - \epsot)^2 > 0$ \\ \hline 
			$\doneddaggall < \epsilon_1$ & $\epsot < 1$, $\epsilon_1 > 0$ \\ \hline 
			\hline
		\end{tabular}
	\end{center}
	\caption{We justify the ordering of the region boundaries for $d_1$.  In the left column, we have the ordering between two boundary points, and in the right column, we show the necessary and sufficient condition that justifies the ordering.}	
	\label{tab:d1_boundaries}	
\end{table}


We therefore see that regardless of whether $\LHSfuncparam$ is greater or less than $C_{2, 3}$, we can achieve an optimal minmax latency so long as we can find $\gammaparam, \omegaparam \in [0, 1]$ such that $\LHSfuncparam = \epsilon_2 - d_2$.   We focus on finding these values of $\gammaparam$ and $\omegaparam$ in the next sections. In doing so, we consider three cases cases for $d_1$.  
%Let 
%\begin{subequations}
%\begin{align}
%	\label{eq:donedagg}
%	\donedagg &= \frac{\epsilon_1}{\epsot}(2 \epsot - 1) \\
%	\label{eq:doneddagg}
%	\doneddagg &= \frac{\epsilon_1}{\epsot}(2 \epsot - 1) \\
%\end{align}
%\end{subequations}
%
The first is when $0 \leq d_1 \leq \donedaggall$, the second when $\donedaggall < d_1 < \doneddaggall$, and third when $\doneddaggall \leq d_1 < \epsilon_1$.  We justify the position of these boundary points with Table~\ref{tab:d1_boundaries}.  For example, in the first row of Table~\ref{tab:d1_boundaries}, we justify that the boundary point $\donedaggall$ is less than the boundary point $\doneddaggall$ with the necessary and sufficient condition that $(1 - \epsot)^2 > 0$.  

After dividing the values of $d_1$ into regions, we then further consider regions of $d_2/\epsilon_2$, where each region requires a distinct choice for the values of $\gammaparam$ and $\omegaparam$.  We note that from Remark~1 in~\cite{TLKS_TIT16} that for $i \in \{1, 2\}$, we assume that $d_i/\epsilon_i < 1$, otherwise an uncoded transmission strategy can achieve~\eqref{eq:wplusd}.  Therefore, we consider only values of $d_2/\epsilon_2 \in [0, 1]$.  In the following sections, the regions of $\depstwo$ will depend on the boundaries $\cstar$, $\mydstar$, $\astar$ and $\bstar$, which we define as

%Within each case that $d_1$ falls into, we further consider regions for $d_2$ that require separate choices of $\gammaparam$ and $\omegaparam$ so that $\LHSfuncparam = \epsilon_2 - d_2$.  These regions depend on the boundaries $\astar$ and $\bstar$, which we define as

\begin{align}
	\cstar &= \left(\frac{\epsot}{\epsilon_2}\right) \left(\frac{d_1}{\epsilon_1}\right), \\
	\mydstar &= \frac{d_1\epsot + \epsilon_1(\epsilon_2 - \epsot)}{\epsilon_1\epsilon_2}, \\	
	\astar &= \frac{\epsot (d_1 (1 - \epsilon_2) + \epsilon_1(\epsilon_2 - \epsot))}{(1 - \epsot)\epsilon_1 \epsilon_2},\\
	\bstar &= \frac{d_1\epsot(1 - \epsilon_2) + \epsilon_1(\epsilon_2 - \epsot)}{(1 - \epsot) \epsilon_1 \epsilon_2}.
\end{align}

%\begin{align}
%	\astar &= \frac{\epsilon_1\epsilon_2(1 - \epsilon_1) + d_1(1 - \epsilon_2)}{1 - \epsilon_1\epsilon_2}, \\
%	\bstar &= \frac{(1 - \epsilon_1) + d_1(1 - \epsilon_2)}{1 - \epsilon_1\epsilon_2}.
%\end{align}

\input{one_sided/optimality_regions}



\begin{comment}
\begin{figure}
\includegraphics[width=\linewidth]{figs/beyond_tss_lesion.pdf}
\caption[]{End-to-End runtime lesion study of the entire MNIST dataset and the FMA featurized music dataset. Each of DROP's contributions provides a runtime improvement.}
\label{fig:beyond_lesion}
\end{figure}
\end{comment}



\section{Conclusion}
\label{sec:conclusion}

Advanced data analytics techniques must scale to rising data volumes. 
DR techniques offer a powerful toolkit when processing these datasets, with PCA frequently outperforming popular techniques in exchange for high computational cost. 
In response, we propose DROP, a new dimensionality reduction optimizer. 
DROP combines progressive sampling, progress estimation, and online aggregation to identify high quality low dimensional bases via PCA without processing the entire dataset by balancing the runtime of downstream tasks and achieved dimensionality. 
Thus, DROP provides a first step in bridging the gap between quality and efficiency in end-to-end DR for downstream \red{analytics}. 

%We revisit canonical operators for time series dimensionality reduction and the measurement study of~\cite{keogh-study}, and show that PCA is more effective than popular alternatives in the data mining literature often by a margin of over $2\times$ on average on gold-standard time series benchmark data sets with respect to output data dimension. More surprisingly, we empirically demonstrate that a small number of samples are sufficient to accurately characterize directions of maximum variance and obtain a high-quality low-dimensional transformation.




%%\section{Introduction}  \label{sec:introduction}

\newcommand\inexpIntro[3]{#1?(#2,#3).}
\newcommand\rinexpIntro[3]{*#1?(#2,#3).}
\newcommand\outexpIntro[3]{#1!(#2,#3).}
\newcommand\outatomIntro[3]{#1!(#2,#3)}

We propose a fully automated method for proving termination of \(\pi\)-calculus processes.
Although there have been a lot of studies on termination analysis for the \(\pi\)-calculus
and related calculi~\cite{Deng06IC,Demangeon07,SangiorgiTermination,KobayashiHybrid,Yoshida04IC,DBLP:journals/jlp/DemangeonHS10,Venet98SAS}, most of them have been rather theoretical,
and there have been surprisingly little efforts in developing  fully automated termination
verification methods and tools based on them. To our knowledge,
Kobayashi's \typical{}~\cite{TyPiCal,KobayashiHybrid} is the only exception that
can prove termination of \(\pi\)-calculus processes (extended with natural numbers)
fully automatically, but its termination analysis is quite limited (see Section~\ref{sec:relatedwork}).

Our method is based on a reduction to termination analysis for sequential programs:
we translate a \(\pi\)-calculus process \(P\) to a sequential program \(S_P\), so that
if \(S_P\) is terminating, so is \(P\). The reduction allows us to use
powerful, mature methods and tools
for termination analysis of sequential programs~\cite{heizmann2016ultimate,freqterm,DBLP:conf/lics/PodelskiR04,Kuwahara2014Termination,DBLP:journals/cacm/CookPR11}.

The idea of the translation is to convert a chain of communications on replicated input
channels to a chain of recursive function calls of the target sequential program.
Let us consider the following Fibonacci process:
\begin{align*}
    & \rinexpIntro{\fib}{n}{r}
        \ifexp{n<2}{ \soutatom{r}{1} \\ &\quad}
                   { \nuexp{s_1} \nuexp{s_2} (\outatomIntro{\fib}{n-1}{s_1} \PAR \outatomIntro{\fib}{n-2}{s_2} \PAR \sinexp{s_1}{x}\sinexp{s_2}{y}\soutatom{r}{x+y}) \\}
    & \PAR \outatomIntro{\fib}{m}{r}
\end{align*}
Here, the process
$\rinexpIntro{\fib}{n}{r} \ldots$ is a function server that computes the \(n\)-th Fibonacci number
in parallel and returns the result to \(r\),
and $\outatom{\fib}{m}{r}$ sends a request for computing the \(m\)-th Fibonacci number;
those who are not familiar with the syntax of the \(\pi\)-calculus may wish to consult
Section~\ref{sec:targetlanguage} first.
To prove that the process above is terminating for any integer \(m\),
it suffices to show that there is no infinite chain of communications on $\fib$:
\[
    \fib(m,r) \to \fib(m_1,r_1) \to \fib(m_2,r_2) \to \cdots.
\]
We convert the process above to the following program:\footnote{The actual translation
  given later is a little more complex.}
\begin{verbatim}
 let rec fib(n) = if n<2 then () else (fib(n-1) [] fib(n-2)) in
 fib(m)
\end{verbatim}
Here, \texttt{[]} represents the non-deterministic choice.
Note that, although the calculation of Fibonacci numbers is not preserved,
for each chain of communications on \texttt{fib}, there is a corresponding
sequence of recursive calls:
\[
\mathtt{fib}(m) \to \mathtt{fib}(m_1) \to \mathtt{fib}(m_2) \to \cdots.
\]
Thus, the termination of the sequential program above implies the termination of
the original process.
As shown in the example above, (i) each communication on a replicated input channel
is converted to a function call, (ii) each communication on a non-replicated input
channel is just removed (or, in the actual translation, replaced by a call of
a trivial function defined by \(f(\seq{x})=(\,)\)), and (iii) parallel composition
is replaced by a non-deterministic choice.
We formalize the translation outlined above and prove its correctness.

The basic translation sketched above sometimes loses too much information.
For example, consider the following process:
\begin{align*}
    & \rinexpIntro{\pre}{n}{r} \soutatom{r}{n-1} \\
    & \PAR \rinexpIntro{f}{n}{r} \ifexp{n<0}{ \soutatom{r}{1} }
                                       { \nuexp{s} (\outatomIntro{\pre}{n}{s} \PAR \sinexp{s}{x}\outatomIntro{f}{x}{r}) } \\
    & \PAR \outatomIntro{f}{m}{r}
\end{align*}
The translation sketched above would yield:
\begin{verbatim}
  let pred(n) = n-1 in
  let rec f(n) = if n<0 then () else (pred(n) [] f(*)) in
  f(m)
\end{verbatim}
Here, \texttt{*} represents a non-deterministic integer: since we have removed
the input $\sinatom{s}{x}$, we do not have information about the value of \( x \).
As a result, the sequential program above is non-terminating, although the original
process is terminating.
To remedy this problem, we also refine the basic translation above by using a refinement
type system for the \(\pi\)-calculus. Using the refinement type system,
we can infer that the value of \(x\) in the original process is less than \(n\),
so that we can refine the definition of \texttt{f} to:
\begin{verbatim}
 let rec f(n) = ... else (pred(n) [] let x=* in assume(x<n);f(x))
\end{verbatim}
The target program is now terminating, from which
we can deduce that the original process is also terminating.
We have implemented an automated tool based on the refined translation above.

The contributions of this paper are summarized as follows.
\begin{itemize}
\item The formalization of the basic translation from the \(\pi\)-calculus
  (extended with integers) to sequential programs, and a proof of its correctness.
\item The formalization of a refined translation based on a refinement type system.
\item An implementation of the refined translation, including automated refinement type
  inference based on CHC solving, and experiments to evaluate the effectiveness of
  our method.
\end{itemize}

The rest of this paper is structured as follows.
Section~\ref{sec:targetlanguage} introduces the source and target languages
of our translation.
Section~\ref{sec:approach} 
formalizes the basic translation, and proves its correctness.
Section~\ref{sec:refinement} refines the basic translation by using a refinement type system.
Section~\ref{sec:implementation} reports an implementation and experiments.
Section~\ref{sec:relatedwork} discusses related work,
and Section~\ref{sec:conclusion} concludes the paper.

%%\input{prior_work}
%%% \documentclass[ex_article]{subfiles}
% \begin{document}
\section{Problem Formulation}\label{sec:problem}
%\subsection{Linear system}
We consider the linear time-invariant (LTI) system
\begin{align}
  \begin{aligned}
    \dot x(t) = Ax(t)+Bu(t), \quad
    y(t)     = C x(t),\quad
    x(0) \sim \mathcal D,
  \end{aligned}\label{eq:system}
\end{align}
where $x(t) \in \R^n$ is state, $u(t)\in \R^m$ is input,
$y(t)\in \R^p$ is output, $A\in \R^{n\times n}$,
$B\in \R^{n\times m}$, and $C \in \R^{p\times n}$ are constant matrices, and $\mathcal D$ is a probability distribution over $\R^n$.
In this paper, we assume that $B$ and $C$ are not zero matrices,
and
$(A, B, C, \mathcal {D})$
is unknown unlike the situation in \cite{fatkhullin2021optimizing}.
%\subsection{LQR problem with structured constraints} \label{Sec2-B}
The infinite-horizon continuous-time LQR problem is formulated as
\begin{align}
  \minimize  & E_{x(0)\sim \mathcal{D}}\qty[\int_0^\infty \qty(y^\top(t) Q y(t) + u^\top(t) R u(t))dt ] \label{eq:objectivefunction} \\
  \subjectto & ~\eqref{eq:system}
\end{align}
with constant positive definite matrices $Q \in \R^{p\times p}$ and $R\in \R^{m\times m}$.
The expectation is taken with respect to the initial state $x(0) \sim \mathcal{D}$.
For the static output feedback $u(t) = -Ky(t)$ with $K\in \R^{m\times p}$ to system~\eqref{eq:system},
the objective function~\eqref{eq:objectivefunction} becomes
$
  f(K) := E_{x(0)\sim \mathcal{D}}\qty[\tilde{f}(K;x(0))],
$
where 
\begin{align}
  \tilde f(K;v) & := \int_0^\infty \qty[v^* e^{A_K^\top t}C^\top(Q+ K^\top RK)Ce^{A_K t}v]dt\label{eq:cost}
\end{align}
for $v\in \C^n$.
Then, the closed-loop is given by
\begin{align}
    \dot x(t)  = A_Kx(t),\quad
    y(t)          = Cx(t), \label{eq:closedloop}
\end{align}
where 
\begin{align}
  A_K := A-BKC. \label{eq:AK}
\end{align}



In this paper, we consider the constraints $K\in \Omega$,
where $\Omega \subset \R^{m\times p}$ is a closed convex set
that specifies the structural information of feedback gains.
This is because a structured policy is often used in practical situations.
For example,
%\subsubsection{Decentralized control}
\begin{itemize}
    \item Decentralized control: In decentralized control, some components of $K$ need to be $0$~\cite{jovanovic2016controller}.
This implies that $\Omega$ should be a certain linear subspace of $\R^{m\times p}$.

\item Linear port-Hamiltonian system: For a linear port-Hamiltonian system~\cite{Jacob2012linear}, 
if the feedback gain is positive semi-definite, the closed loop system is also a port-Hamiltonian system and passive. To ensure passivity, $\Omega$ should be defined as the set of positive semi-definite matrices, which is closed and convex.

\end{itemize}
%\subsubsection{Linear port-Hamiltonian system}



%See Section~\ref{sec:examples}.


By using Bellman lemma~\cite{bellman1957notes}, the problem~\eqref{eq:objectivefunction} with structured constraints can be formulated as
\begin{align}
  \begin{aligned}
    \minimize_K & f(K) = \tr(X\Sigma) \\
    \subjectto  & K \in \Omega\,\, \text{and}\,\, A_K\text{ is \textit{Hurwitz}},
  \end{aligned}\label{eq:problem}
\end{align}
where
$
  \Sigma  := E[x(0)x^\top(0)]
$
and $X$ is the solution to
\begin{align}
  A_K^\top X + XA_K + C^\top \qty(K^\top RK + Q)C = 0.
\end{align}
It is difficult to solve \eqref{eq:problem}, since  $f(K)$ is non-convex and saddle points may exist~\cite{fatkhullin2021optimizing}.
Moreover,
 the feasible set may have exponentilally many disconnected components~\cite{feng2019exponential}. 
Although an iterative method was proposed in \cite{zhu2015adaptive} to obtain a suboptimal static output feedback gain in the model free setting,
it cannot be applied directly to problem \eqref{eq:problem} due to the constraint $K\in \Omega$.

% In this paper, we suppose that $(A, B, C, \mathcal {D})$ in \eqref{eq:system}
% is unknown unlike the situation in \cite{fatkhullin2021optimizing}.
% Thus, we develop a model free algorithm in Section \ref{sec:model-free} for solving the problem \eqref{eq:problem}.


To develop a model free algorithm with theoretical guarantees for solving problem \eqref{eq:problem}, we impose the following throughout this paper:
\begin{assumption}\label{assume:sigma-hurwitz}
  \indent
  \begin{enumerate}
    \item $\Sigma \succ 0$.
    \item The pair $(A, C)$ is observable.
    \item There exists $K_0 \in \Omega$
          such that $A_{K_0}$ is \textit{Hurwitz}
          and $K_0$ is known.
  \end{enumerate}
\end{assumption}


Since $A_{K_0}$ is \textit{Hurwitz},
there exist positive definite matrices $G, H$ and a skew-adjoint matrix $J$
such that
$
  A_{K_0} = (J-G)H.
$
The proof is found in~\cite{prajna2002lmi}.
Let $H = L^\top L$ be the Cholesky decomposition.
Using the coordinate transformation $x'(t) = Lx(t)$, the closed-loop system~\eqref{eq:closedloop} becomes
\begin{align}
  \dot{x'}(t)  = A'_{K_0}x'(t), \quad 
  y(t)         = C'x'(t),
\end{align}
where $A'_{K_0}  = LJL^\top-LGL^\top, C' = CL^{-1}$.
Since $LGL^\top\succ 0$ and $LJL^\top = -(LJL^\top)^\top$, we have
  $A'_{K_0}+{A'_{K_0}}^\top = -2LGL^\top \prec 0$.
In the following, we assume system \eqref{eq:closedloop} after the above coordinate transformation, because we consider a static output feedback that is invariant by the coordinate transformation. That is, without loss of generality, we can assume
$A_{K_0}+A_{K_0}^\top \prec 0$.

Under Assumption \ref{assume:sigma-hurwitz},
 $f(K)$ of \eqref{eq:problem} is defined only on the set $S$ of stabilizing controllers, which is defined as
\begin{align}
  S = \{K\in \R^{m\times p}\mid A_K\text{ is \textit{Hurwitz}}\}. \label{eq:S}
\end{align}
If $K\notin S$, there exists an eigenvalue $\mu$ of $A$ such that ${\rm Re}(\mu) \geq 0$ and $f(K)$ goes to infinity.

\begin{remark} \label{remark1}
The objective function of
 problem \eqref{eq:objectivefunction}
is not a
standard LQR cost
%is given by
% \begin{align}
%     \int_0^\infty \qty(x^\top(t) Q x(t) + u^\top(t) R u(t))dt,
% \end{align}
% where $Q \in \R^{n\times n}$ and $R\in \R^{m\times m}$ are positive definite matrices. However, in the model free and output feedback setting, this cost cannot be calculated in practice, because the information of the state $x(t)$ is not available. Therefore, we consider the problem \eqref{eq:objectivefunction} following
as in some previous researches~\cite{modares2016optimal, rizvi2018output}. While similar convergence properties to the standard LQR cost can be obtained for our formulation in the model based setting if $(A, C)$ is observable~\cite{fatkhullin2021optimizing}, more detailed studies of the objective function properties are necessary for model-free version of the convergence analysis.  
\end{remark}



%\subsection{Examples}\label{sec:examples}


% \end{document}
%%\input{intuition}
%%\begin{table*}[t]
\centering
  \caption{Quantitative comparisons with the optimization-based and efficient methods. Encoding time means the time cost to obtain the unique/pseudo embedding. Our method achieves optimal results in terms of text-alignment, face similarity, and encoding time.}
  \label{tab:main_result}
  \begin{tabular}{cccc}
    \toprule
    Methods & Text-alignment $\uparrow$ & Face similarity $\uparrow$ & Encoding Time $\downarrow$ \\
    \midrule
    Textual Inversion \cite{gal2022image} & 0.213 & 0.326 & 20 min \\ 
    Dreambooth \cite{ruiz2022dreambooth} & 0.217 & 0.425 & 4 min  \\ 
    E4T \cite{gal2023designing} & 0.220 & 0.420 & 20 s \\ 
    Elite \cite{wei2023elite} & 0.196 & 0.450 & 0.05 s\\ 
    \midrule
    Ours & \textbf{0.228} & \textbf{0.467} & \textbf{0.04 s}\\
    \bottomrule
  \end{tabular}
\end{table*}
%%\input{construction_y}
%%\input{properties_y}
%%\section{Numerical reconstructions}\label{sec:reconstructions} %bk 2021-11-28

In this section we give some details and show some reconstructions
of the algorithms described in section \ref{sec:recon_meth}.
We assume single mode excitation, set $\beta_i=1$ and
break the reconstructions into three separate paradigms as previously indicated
depending on which time ranges are feasible to be measured.

\input recon_figs.tex

We assume single mode excitation, set $\beta_i=1$ and 
break the reconstructions into three separate paradigms as indicated
in Section~\ref{sec:recon_meth} 
depending on which time ranges are feasible to be measured.

\noindent
\subsection{Full time measurements.}

The leftmost graph in Figure \ref{fig:h_and_hath} shows the actual time trace
of $h(t)$ 
over the range $0\leq t\leq 40$
for a damping model with three terms 
$b_i\partial_t^{\alpha_i}$, 
$i=1,2,3$.
Note the damped oscillatory behaviour evident and the range shown
is the one we are labelling as ``full-time'' despite the fact that we
will later look at the important case of 
the very long term
behaviour of the solution where the time values are substantially outside
of this range.  This time range terminology is highly dependent on the
values of critical constants in the model.
These include the size of the domain (which determines the scaling of the
eigenvalues $\{\lambda_n\}$), the wave speed $c$ and of course the strength
of the damping given by the coefficients $\{b_i\}$.
In our reconstructions we will typically take these constants to be of
approximate order unity but note the dependence on these quantities
and the effect that a substantial change would make.

To obtain this $h(t)$ our damping constants are of order 
$\sim 0.1-0.2$ and the combined term 
$\Lambda=c^2\lambda$ 
is taken to be unity.
Changes to the former would modify the time-decrease in $h$ and changes to the 
latter would alter the frequency of the oscillations.

The rightmost graph in Figure~\ref{fig:h_and_hath} shows the logarithm of the
Laplace transform $\hat{h}(s)$ together with the values of $\hat{h}_i(s)$ after
iteration $i$ of the scheme.
%%%%%%%%%%%%%%%%%%%
\begin{figure}[ht]
\hbox to \hsize{\hfill\copy\figureone\hfill\copy\figuretwo\hfill}
\small
\caption{\small {\bf Profiles of $h(t)$ and $\hat{h}(s)$, actual and after
iterations 1 and 2.}}
\label{fig:h_and_hath}
\end{figure}
%%%%%%%%%%%%

This demonstrates two important facts.
First, the very fast convergence of the scheme in the sense of the
convergence of the target Laplace transform function as the parameters
$\{\alpha_i,b_i\}$ are resolved.
The actual $\hat{h}(s)$ is shown by the solid black line, the initial approximation
by the dotted red curve and the first iteration by the dashed green curve
which, at this scale is already almost indistinguishable from the actual $\hat{h}$.
Second, given this it should be quite feasible to obtain reasonably
accurate values the parameters $\{\alpha_i,\,b_i\}$.
In addition, assuming we excite the system with an initial condition
equal to an eigenfunction corresponding to an actual eigenvalue $\lambda$,
we will be able to reconstruct the value of $c^2$ from 
the composite $\Lambda=c^2\lambda$. %bk 2021-11-28

\smallskip
In this situation, to reconstruct the parameters $\{\alpha_i,\,b_i,\, c^2\}$
we work directly with the representation $\hat{h}(s)$ or more exactly with its
logarithm which is a more convenient form for computing the Jacobian
\begin{equation}\label{eqn:log_F(s)}
\log\bigl(\hat{h}(s)\bigr) =
\log\Bigl( s + \sum_1^n b_i s^{\alpha_i-1}\Bigr) -
\log\Bigl( s^2 + \sum_1^n b_i s^{\alpha_i} + c^2\lambda\Bigr).
\end{equation}
Of course, to obtain $\hat{h}(s)$ we must approximate
the integral $\hat{h}(s) = \int_0^\infty e^{-st}h(t)\,dt$ and this indeed does
require full time measurements.
However, we do not actually need the values of the analytic function
$\hat{h}(s)$ for all $s$ for the Newton scheme and so if we only use $s$
values with $s>s_0$ then this allows us to ignore very large time measurements
and indeed we truncated these to $t\leq 40$.

The reconstruction of the components in the function $\hat{h}(s)$ is shown
graphically in Figure~\ref{fig:b_and_alpha} and in tabular form in 
Table~\ref{Table:Full_time_trace}.
The exact values were $\alpha=\{0.25, \, 0.5, \, 0.75\}$, $b=\{0.2,\,0.25,\,0.1\}$, $\Lambda=4$. %bk 2021-11-28
Note that we have included the value of the residual here and keeping track of
this value allows the iteration scheme to terminate when saturation has
occurred.
%%%%%%%%%%%%%%%%%%%%%%%%%%%%%%%%%%%%%%%%%%%%%%%%
\begin{table}[H]
\centering
\small
\footnotesize
\begin{tabular}{|c|c|c|c|c|c|c|c|c|}
\hline
iter & $\alpha_1$ & $\alpha_2$ & $\alpha_3$ & $b_1$ & $b_2$ & $b_3$ & $\Lambda$ & residual\\
\hline
	0 & 0.3000 & 0.6000 & 0.8000 & 0.3000 & 0.3750 & 0.1500 & 3.500 & {} \\
1 & 0.2448 & 0.5590 & 0.7946 & 0.1907 & 0.2612 & 0.1098 & 4.003 & 0.035032 \\
2 & 0.2506 & 0.5254 & 0.7695 & 0.2057 & 0.2747 & 0.1190 & 4.000 & 0.004371 \\
3 & 0.2492 & 0.5253 & 0.7700 & 0.2060 & 0.2784 & 0.1190 & 4.000 & 0.000075 \\
4 & 0.2491 & 0.5254 & 0.7700 & 0.2060 & 0.2748 & 0.1192 & 4.000 & 0.000000 \\
\hline
\end{tabular}
\small
\footnotesize
\caption{\bf Recovery of  damping terms and unknown $\Lambda$ from full time values.}
 \label{Table:Full_time_trace}
\end{table}
%%%%%%%%%%%%%%%%%%%
\begin{figure}[h]
\small
\hbox to \hsize{\hfill\copy\figurefour\hfill\copy\figurethree\hfill}
\caption{\small {\bf reconstructed values of $\{b_i\}$ and $\{\alpha_i\}$.
%The zero index is the initial approximation
The symbols in red are the exact values.}}
\label{fig:b_and_alpha}
\end{figure}
%%%%%%%%%%%%%%%%%%%%%%%%%%%%%%%%%%%%%%%%%%%%%%%%



\subsection{Large time measurements.}

Here we are trying to simulate the asymptotic values of the constituent
powers of $t$ occurring in the data function $h(t)$.
This was achieved by using a sample of points between $t_{\rm min}$
and $t_{\rm max}$.

\revision{\subsubsection*{A Newton based approach}}
We apply Newton's method to recover the constants $\{p_i,c_{k,\ell}\}$ in 
\[
\begin{aligned}
h(t) &= 
c_{1,1} t^{-p_1}
+c_{1,2} t^{-p_2}
+c_{1,3} t^{-p_3}
+c_{2,1} t^{-2p_1}
+c_{2,2} t^{-2p_2}
+c_{2,3} t^{-2p_3}
+c_{2,4} t^{-p_1-p_2}
\\&\quad+c_{2,5} t^{-p_2-p_3}
+c_{2,6} t^{-p_3-p_1}
+c_{3,1} t^{-3p_1}
+c_{3,2} t^{-3p_2}
+c_{3,3} t^{-2p_1-p_2}
+c_{3,4} t^{-p_1-2p_2}
\\&\quad+c_{3,5} t^{-2p_1-p_3}
+c_{3,6} t^{-p_1-2p_3}
+c_{3,7} t^{-2p_2-p_3}
+c_{3,8} t^{-p_2-2p_3}
+O(t^{-3p_3})
\end{aligned}
\]
which results from the expansion \eqref{eqn:hsing} in case of three terms,
and then recover $\alpha_i=p_i$ and $b_i=c_{1,i}\Gamma(1-p_i)$  
In the above we have neglected the term $t^{-3\alpha_3}$ as this would not arise
from the Tauberian theorem in the case that the largest power
$\alpha_3 \geq \frac{1}{3}$.
We may also have to exclude other terms such as  
$t^{-2\alpha_1-\alpha_2}$
if
$2\alpha_1 + \alpha_2 \geq 1$.
In practice, during the iteration process, terms should be included or excluded
in the code depending on this criterion: we did so by checking
if the argument passed to the $\Gamma$ function would be negative in which
case the term is deleted from use for that iteration step.

Values for the table shown below were
$t_{\rm min}=5\times 10^4$ and $t_{\rm max}=2\times 10^5$.
As a general rule, terms with small $\alpha$ values can be resolved
with a smaller value of $t_{\rm max}$, but for, say the recovery of
a pair of damping terms with $\alpha_i>0.8$, a larger value 
of $t_{\rm max}$ with commensurate accuracy will be needed.

The case of three damping terms 
$\{b_i\partial^{\alpha_i}_t\}_{i=1}^3$ 
with
$\alpha = \{\frac{1}{4},\,\frac{1}{3},\,\frac{2}{3}\}$ and $b_i=0.1$
is shown in Table~\ref{Table:large_tt} below.
The initial starting values were taken to be between ten and thirty percent
of the actual.
These are shown in the line corresponding to iteration $0$.

\begin{table}[ht]
\centering
\footnotesize
\begin{tabular}{|c|c|c|c|c|c|c|c|}
\hline
iter & $\alpha_1$ & $\alpha_2$ & $\alpha_3$ & $b_1$ & $b_2$ & $b_3$ & residual\\
\hline
0 & 0.2000 & 0.3000 & 0.6000 & 0.130 & 0.080 & 0.110 &\\			%bk 2021-11-28
1 & 0.2409 & 0.3572 & 0.6168 & 0.112 & 0.089 & 0.107 & 0.548761\\
2 & 0.2477 & 0.3307 & 0.6531 & 0.102 & 0.091 & 0.104 & 0.104843\\
3 & 0.2499 & 0.3346 & 0.6650 & 0.100 & 0.099 & 0.092 & 0.005124\\
4 & 0.2500 & 0.3332 & 0.6668 & 0.100 & 0.099 & 0.090 & 0.003625\\
5 & 0.2500 & 0.3331 & 0.6658 & 0.100 & 0.100 & 0.090 & 0.001134\\
6 & 0.2500 & 0.3332 & 0.6658 & 0.100 & 0.100 & 0.092 & 0.000578\\
7 & 0.2500 & 0.3332 & 0.6660 & 0.100 & 0.100 & 0.093 & 0.000357\\
8 & 0.2500 & 0.3333 & 0.6665 & 0.100 & 0.100 & 0.094 & 0.000240\\
\hline
\end{tabular}
\small
\caption{Large time values with 3 damping terms}
 \label{Table:large_tt}
\end{table}

There are features here that are typical of such reconstructions.
The reconstruction method resolves the lowest fractional power $\alpha_1$
and its coefficient $b_1$ quickly as this term is the most persistent one
for large times:
essential numerical convergence for $\{\alpha_1,\,b_1\}$ is obtained
by the third iteration.
The next lowest power and coefficient lags behind; here $\{\alpha_2,\,b_2\}$ is already at the stated accuracy by the fifth iteration.
In each case the power is resolved faster and more accurately than its
coefficient.
The third term also illustrates this;  the power is essentially resolved by
iteration 8, but in fact its coefficient $b_3$ is not resolved to the third
decimal place until iteration number 30.
This is seen quite clearly in the singular values of the Jacobian:
the largest singular values correspond to the lowest $\alpha$-values
and the smallest to the coefficients of the largest $\alpha$-powers.

As might be expected, resolving terms whose powers are quite close is
in general more difficult.
This is relatively insignificant for low $\alpha$ values.
For example, with $\alpha_1=0.2$ and $0.22<\alpha_2<0.25$ say, correct
resolution will be obtained although the coefficients will take longer to
resolve than indicated in Table~\ref{Table:large_tt}.
On the other hand if, say $\alpha_1=\frac{1}{4}$ and $\alpha_2=0.85$,
$\alpha_3=0.9$, then with the indicated range of time values used the code
will fail to recover this last pair.
If this is sensed and now only a single second power is requested
this will give a good estimate for $\alpha_2$ but its coefficient will
be overestimated.
Also, as indicated previously, $\alpha$ values close to one require
an extended time measurement range to stay closer to the asymptotic regime of $u(t)$.

\revision{
Note that the starting values were at least ten percent away from the exact values and one cannot expect the convergence radius of Newton's methods to be much larger for a problem exhibiting as high nonlinearity as the one at hand.
Still, let us point to the fact that in the single term case we do not need any starting guesses but obtain very good results directly from the asymptotic formulas \eqref{eqn:alpha-larget}, \eqref{eqn:b1-larget}. This might give the idea of using the asymptotics to construct starting guesses in the multi term case and then apply Newton. However, it is unclear how the asymptotics could yield starting values for terms other than the first one. A (theoretical, as it turns out) possibility for exploiting asympotics in the multi term case is described below. 
}
\Margin{Ref 2 (iv)}

\revision{\subsubsection*{Sucessive sequential use of asymptotic formulas}}
A few words are in order about an approach that from the above discussion
might seem a good or even a better alternative.

Since each damping term 
$b_i \partial^{\alpha_i}_t$ 
contributes a time trace term
with large time behaviour $c_i t^{-\alpha_i}$, it is feasible to take
$T$ sufficiently large so that
$c_i t^{-\alpha_i} <\!\!\!< c_1 t^{-\alpha_1}$ for $t>T$, that is, all but the
smallest damping power is negligible, and this can then be recovered.
In successive steps then we subtract this from the data $h$ to
get $h_1(t) = h(t) - c_1t^{-\alpha_1}$ and now seek to recover the next
lowest $\alpha$ power from the large time values of $h_1(t)$  in a range
$(\delta T,T)$ for $0<\delta<1$.
Then these steps can be repeated until there is no discernible signal remaining
in the sample interval $t_{\rm min},t_{\rm max}$.

This indeed works well under the right circumstances
for recovering two $\alpha$ values but the
coefficients $\{b_i\}$ are less well resolved.
It also requires a delicate splitting of the time interval and gives a
much poorer resolution of the two terms in the case where, say
$\alpha_1=0.2$ and $\alpha_2=0.25$  than that recovered from the Newton scheme.
For the recovery of three damping terms this was in general quite
ineffective.

Every time an $\alpha_i$ has been recovered, the remaining signal is significantly smaller than the previous, leading to an equally significant drop in effective accuracy.
Also, even if we just make a small error in the coefficient $b_i$, the relative error that is caused by this becomes completely dominant for large times. 

In short, this is an elegant and seemingly constructive approach to showing uniqueness for a finite number of damping terms. However, it has limited value from a numerical recovery perspective when used under a wide range of parameter values.

\subsection{Small time measurements.}

In this case we are simulating measurements taken over a very limited initial
time range: in fact we take the measurement interval to be $t\in [0, 0.1)$.
The line of attack is to use the known form of $\hat{h}(s)$ for large values of $s$
and convert the powers of $s$ appearing into powers of $t$ for small times
using the Tauberian theorem.
In case of two damping terms
this gives
\begin{equation*}%\label{eqn:small_t-form}
\begin{aligned}
%f_{\tiny\rm{small}}(t) &= 
%- \frac{\lambda}{\Gamma(4)})t^3 + \frac{\lambda^2}{\Gamma(6)}t^5
%-\frac{\lambda^3}{\Gamma(8)}t^7 \\
%&\quad - c_{1,1}t^{3-p_1} - c_{2,1}t^{3-p_2}) - c_{3,1}t^{3-p_3}  
% + c_{1,4}t^{5-p_1} + c_{2,4}t^{5-p_2} + c_{3,4}t^{5-p_3} \\
%&\quad + c_{1,6}t^{5-2p_1} + c_{2,6}t^{5-2p_2} + c_{3,6}t^{5-2p_3} 
% + c_{4,6}t^{5-p_1-p_2} + c_{5,6}t^{5-p_1-p_3}+ c_{6,6}t^{5-p_2-p_3}\\
-c^2\lambda h_{\tiny\rm{small}}(t)-h_{\tiny\rm{small}}''(t) 
&
=c_{1,1} t^{1-\alpha_1}
+c_{1,2} t^{1-\alpha_2}
+c_{2,1} t^{3-\alpha_1}
+c_{2,2} t^{3-\alpha_2}
\\&\quad
+c_{2,3} t^{3-2\alpha_1}
+c_{2,4} t^{3-\alpha_1-\alpha_2}
+c_{2,5} t^{3-2\alpha_2}
\end{aligned}
\end{equation*}
where each term $c_{k,\ell}$ is computed in terms of
$\{\alpha_i\}$, $\{b_i\}$ and $\lambda$, %bk 2021-11-28
cf. \eqref{eqn:smalltime}. 
							         
The values of $\{\alpha_i,c_{k,\ell}\}$ 
were then computed from the data by a Newton
scheme then finally converted back to the derived values of
$\{\alpha_i\}$ and $\{b_i\}$.

The exact values chosen were
$\alpha = \{0.25,\ 0.2\}$, $b = \{0.1,\ 0.1\}$ and the
initial starting guesses were $\alpha = \{0.3,\ 0.16\}$, $b = \{0.08,\ 0.12\}$.
We show the progression of the iteration in 
Table~\ref{tab:smalltime}. 

\begin{table}[H]
\centering
\small
\footnotesize
\begin{tabular}{|c|c|c|c|c|c|}
\hline
iter & $\alpha_1$ & $\alpha_2$ & $b_1$ & $b_2$ & residual\\
\hline
0 & 0.3000 & 0.1600 & 0.0800 & 0.1200 &  \\
1 & 0.2848 & 0.1673 & 0.0821 & 0.1160 & 0.034308 \\
2 & 0.2632 & 0.1944 & 0.0863 & 0.1142 & 0.024472 \\
3 & 0.2554 & 0.2016 & 0.0862 & 0.1140 & 0.002015 \\
4 & 0.2537 & 0.2033 & 0.0861 & 0.1139 & 0.000444 \\
5 & 0.2534 & 0.2036 & 0.0861 & 0.1139 & 0.000059 \\
6 & 0.2534 & 0.2036 & 0.0861 & 0.1139 & 0.000016 \\
7 & 0.2534 & 0.2036 & 0.0861 & 0.1139 & 0.000010 \\
\hline
\end{tabular}
\small
	\caption{{\small\bf Small time values with 2 damping terms.}
\label{tab:smalltime}}
\end{table}

While theory predicts reconstructibility of an arbitrary number of terms in both cases,
there is a clear difference in ability to reconstruct terms between the small time and the large time asymptotics. 
First of all, the method we described effectively only recovers two terms with small time measurements, as compared to three in the large time. 
The $b_i$ coefficients, which are always harder to obtain than the $\alpha_i$ exponents, are much worse than in the small time than in the large time regime.
This is partly explained by the higher degree of ill-posedness due to the necessity of differentiating the data twice. 




%%\input{main_derivation}
%%
\begin{comment}
\begin{figure}
\includegraphics[width=\linewidth]{figs/beyond_tss_lesion.pdf}
\caption[]{End-to-End runtime lesion study of the entire MNIST dataset and the FMA featurized music dataset. Each of DROP's contributions provides a runtime improvement.}
\label{fig:beyond_lesion}
\end{figure}
\end{comment}



\section{Conclusion}
\label{sec:conclusion}

Advanced data analytics techniques must scale to rising data volumes. 
DR techniques offer a powerful toolkit when processing these datasets, with PCA frequently outperforming popular techniques in exchange for high computational cost. 
In response, we propose DROP, a new dimensionality reduction optimizer. 
DROP combines progressive sampling, progress estimation, and online aggregation to identify high quality low dimensional bases via PCA without processing the entire dataset by balancing the runtime of downstream tasks and achieved dimensionality. 
Thus, DROP provides a first step in bridging the gap between quality and efficiency in end-to-end DR for downstream \red{analytics}. 

%We revisit canonical operators for time series dimensionality reduction and the measurement study of~\cite{keogh-study}, and show that PCA is more effective than popular alternatives in the data mining literature often by a margin of over $2\times$ on average on gold-standard time series benchmark data sets with respect to output data dimension. More surprisingly, we empirically demonstrate that a small number of samples are sufficient to accurately characterize directions of maximum variance and obtain a high-quality low-dimensional transformation.



%\input{numerical_plot}

%Reinforcement learning has achieved great success in areas such as Game-playing \citep{silver2018general,vinyals2019grandmaster}, robotics \cite{kober2013reinforcement}, large language models \citep{ouyang2022training}, etc.
However, due to safety concerns or physical limitations, in some real-world reinforcement learning problems, we must consider additional constraints that may influence the optimal policy and the learning process \citep{garcia2015comprehensive}.
% For example, a robotic arm must not take actions that may cause harm to itself or the environments.
A standard framework to handle such cases is the constrained Markov Decision Process (CMDP) \citep{altman1999constrained}.
Within the CMDP framework, the agent has to maximize
the expected cumulative reward while
obeying a finite number of constraints, which are usually in the form of expected cumulative cost criteria.

However, we are sometimes concerned with the problem with a continuum of constraints.
For example,
the constraints we meet might be time-evolving or subject to uncertain parameters, which
cannot be formulated as an ordinary CMDP
(see Examples \ref{Example_Time_Evolving} and  \ref{Example_Uncertain}).
In this paper we would study a generalized CMDP  
to address the above problem.  Because the constraints are not only infinite-number but also lie
in a continuous set,
the generalization is not trivial. Fortunately, we find that we can borrow the idea behind semi-infinite programming (SIP) \citep{remez1934determination, hettich1993semi} to deal with the semi-infinite constraints.
Accordingly, we propose \emph{semi-infinitely constrained Markov decision processes} (SICMDPs)
as a novel complement to the ordinary CMDP framework.
%More specifically,  an SICMDP model %, we consider 
%contains a continuum of constraints whereas an ordinary CMDP contains a finite number of constraints. 

%This generalization is natural but not trivial. However, we can brows the idea  
%The idea is quite natural and can be backtracked
%to the practice of extending linear programming to linear semi-infinite programming (LSIP) %\cite{remez1934determination, GobernaLSIO1998}.
%In addition, 
%As a complementary approach to the ordinary CMDP framework, 
%SICMDP can be used to model these problems  which cannot be described by a finite number of constraints
%that are not covered by .
%For example,
%the restrictions we consider can be time-evolving or subject to uncertain parameters
%, thus
%cannot be described by a finite number of constraints but a continuum of constraints 
%(see Examples \ref{Example_Time_Evolving} and  \ref{Example_Uncertain}).

We also present two reinforcement learning algorithms to solve SICMDPs called SI-CRL and SI-CPO, respectively.
SI-CRL is a model-based reinforcement learning algorithm designed for tabular cases, and SI-CPO is a policy optimization algorithm for non-tabular cases.
% and analyze its performance both theoretically and empirically.
The main challenge is that we need to deal with a continuum of constraints, thus reinforcement learning algorithms for ordinary CMDPs do not work anymore.
In SI-CRL, we tackle this difficulty by first transforming the reinforcement learning problem to an equivalent LSIP problem, which can then be solved using methods in the LSIP literature like the dual exchange methods \citep{Hu1990,reemtsen1998numerical}.
In SI-CPO, we resort to the idea of cooperative stochastic approximation developed in \cite{lan2020algorithms, wei2020comirror}.
As far as we know, we are the first to introduce tools from semi-infinitely programming (SIP) into the reinforcement learning community for solving constrained reinforcement learning problems.

% To the best of our knowledge, we are the first to apply tools from semi-infinitely programming (SIP) to solve reinforcement learning problems.
Furthermore, we give theoretical analysis for both SI-CRL and SI-CPO.
We decompose the error of SI-CRL into two parts: the statistical error from approximating the true SICMDP with an offline dataset and the optimization error due to the fact that the solution of the LSIP problem obtained by the dual exchange method is inexact.
On the optimization side, we show that the iteration complexity of SI-CRL is $O\left(\left\{\mathrm{diam}(Y)L\sqrt{|\gS|^2|\gA|m}/\left[(1-\gamma)\epsilon\right]\right\}^m\right)$.
On the statistical side, we show that the sample complexity of SI-CRL is $\widetilde O\left(\frac{|S|^2|A|^2}{\epsilon^2(1-\gamma)^3}\right)$ if the offline dataset is generated by a generative model, and $\widetilde O\left(\frac{|S||A|}{\nu_{\min} \epsilon^2(1-\gamma)^3}\right)$ if the dataset is generated by a probability measure $\nu$ as considered in \cite{chen2019information}.
Here $\widetilde O$ means that all logarithm terms are discarded.
For SI-CPO, things become a little more complicated because other than the statistical error and the optimization error, we also need to consider the function approximation error, which comes from imperfect policy parametrizations.
It is shown if the function approximation error can be controlled to $O(\epsilon)$ order, the iteration complexity of SI-CPO is $\widetilde{O}\left(\frac{1}{\epsilon^2(1-\gamma)^6}\right)$ and the sample complexity of SI-CPO is $\widetilde{O}(\frac{1}{\epsilon^4(1-\gamma)^{10}})$.
Here our iteration complexity bound is equivalent to a typical $\widetilde O(1/\sqrt{T})$ global convergence rate.

We perform a set of numerical experiments to illustrate the SICMDP model and validate our proposed algorithms.
Specifically, we examine two numerical examples, namely the discharge of sewage and ship route planning.
Through the discharge of sewage example, we show the advantage of the SICMDP framework over the CMDP baseline obtained by naive discretization in modeling realistic sequential decision-making problems.
Moreover, we demonstrate the effectiveness of the SI-CRL and SI-CPO algorithms in such tabular environments. 
In the ship route planning example, we illustrate the benefits of the SICMDP framework and the ability of the SI-CPO algorithm to address complex continuous control tasks involving continuous state spaces with modern deep reinforcement learning techniques.

% In summary, our contributions are listed as follows.
% First, we present the SICMDP model, which can be viewed as a generalization of the ordinary CMDP model.
% Second, we propose an algorithm to perform reinforcement learning for SICMDPs, which is called SI-CRL, and we believe that we are the first to apply tools from SIP
% to solve reinforcement learning problems.
% Third, we give a theoretical analysis of SI-CRL and identify both its sample complexity and iteration complexity.
% In addition, we perform numerical experiments to illustrate the SICMDP model and validate the SI-CRL algorithm.
% \{This paragraph can be removed!!! \}





%\input{SysModel}


%\input{SysModel/SysModel2}
%
% Panoptic segmentation

% 3D segmentation

% Multi-object tracking

% Online 3D panoptic:

% PanopticFusion: (IROS 2019)
% https://arxiv.org/pdf/1903.01177.pdf
%
% - most similar to ours
% - PSPNet + M-RCNN + 2D fusion
% - volumetric mapping, 
% - greedy matching with IoU -> optimal only with 0.5 threshold
% - voxel & class weighting
% - CRF regularisation
%
% - good:
%
% - bad:
%  - CRF post-processing step
%  - greedy data-association
%    - can't be tuned for lower overlap ratios -> has to have high framerate, large changes in viewpoint could break this
%    - IoU: sensitive to 2D labels projecting over object borders (CRF and voxel weighting seem to alleviate this)

% Voxblox++: (Robotics & automation letters 2019)
% https://arxiv.org/pdf/1903.00268.pdf
% https://github.com/ethz-asl/voxblox-plusplus
%
% - M-RCNN + geometric segmentation + fusion 
% - data association of geometric segments with 3D overlap (no. points inside volume), fixed threshold for min number of points
% - instance label is assigned to a segment based on highest overlap
% - only one detected segment per reference label, as in PanopticFusion and Ours
% - TSDF Integration 
%
% good: 
% - because of geometric segmentation objects with no associated semantic class can also be segmented
% bad:
% - two different object segment types -> confusing, overly complicated ?
% - quite inaccurate (fixed below)

% Reconstructing Interactive 3D Scenes by Panoptic Mapping and CAD Model Alignments (ICRA 2021)
% https://arxiv.org/pdf/2103.16095.pdf
% https://github.com/hmz-15/Interactive-Scene-Reconstruction
%
% - based heavily on Voxblox++, much more accurate
% - Scene-graph ("contact graph") for mapping object relations
% - Search & replace voxels with CAD models, with geometrical and physical constraints
% - Object 6D pose
% - Format for robot interaction
%
% - Segmentation: bilateral fusion of geomatric and semantic segments -> reduce segmentation noise compared to Voxblox++
% - Fusion: triplet count improves consistency over Voxblox++ pairwise count strategy (take semantic label into account in addition to instance and geometry)
% - Fusion: instance labels are also combined if there is enough overlap with common geometric label for long enough time
%   - this means multiple detections can match the same reference unlike ours, voxblox++ and PanopticFusion ?
%

% Panoptic-MOPE: (ROBOTICS AND AUTOMATION LETTERS 2020)
% https://ieeexplore.ieee.org/stamp/stamp.jsp?tp=&arnumber=8977356
% https://github.com/hoangcuongbk80/Object-RPE/tree/panoptic-mope
%
% - novel RGB-D semantic segmentation model + M-RCNN
% - camera tracking based on "addaptively weighted optimization of geometric, appearance, and semantic cues"
% - surfel map: 
%   - how does it scale ? authors satate they tested on room-sized environments, but could be applied in larger scale as well ...
%     - could maybe be applied as VO in a SLAM algorithm ...
%   - demo only on a small pallet + surroundings, might not be applicable in large-scale SLAM

% US VS THEM:
%
% - based heavily on PanopticFusion, with modifications:
%   - instead of greedy data-association (which seems to be the case in others as well), we solve LAP (JPDA?)
%     - overlap threshold can be tuned, which renders the algorithm more flexible
%     - could be extended to dynamic tracking ?
%   - multiple options for association likelihood
%   - outlier rejection (either clustering or probabilistic)
%   - test different options for decreasing processing time
%   - no post-processing
%
% - model-agnostic:
%   - completely separated from segmentation
%   - does not care how point clouds are obtained -> applicable for LIDAR segmentation (e.g. EfficientLPS) as well
%
% - also agnostic to localisation method
%   - could, however, be utilised to find landmark locations / poses

% More compact version of this paragraph to introduction to save space?
%Panoptic segmentation -- proposed in \cite{panoptic_segmentation} -- aims to solve the unified task of semantic- and instance segmentation. Semantic classes are separated to \textit{stuff} -- amorphous, unquantifiable regions like sky, road or floor -- and \textit{things} -- quantifiable objects. The distinction between the two can vary depending on the application, but a semantic class can only belong to one or another. The article also proposes a unified panoptic evaluation metric, coined \textbf{Panoptic Quality} (PQ). Many 2D approaches to panoptic segmentation -- \textit{e.g.} \cite{panopticfpn,seamless,panoptic_deeplab,efficientps} -- have since been proposed. Deep neural networks for performing semantic- or instance segmentation directly on the 3D reconstruction -- \textit{e.g.} on \cite{scannet,s3dis,paris_lille_3d} -- have also been proposed, but since they require the reconstructed 3D scene, they are mostly offline approaches and therefore out of scope for this work. Some recent works also apply panoptic segmentation to point clouds -- \textit{e.g.} methods in the SemanticKITTI panoptic segmentation competition \cite{semantic_kitti} -- mostly aimed at segmenting LiDAR output. They are suitable for online processing, but similar to RGB-D images require a method for tracking object instances persistent in both time and space. In fact, our proposed method, as well as some others mentioned in this work, could use segmented LiDAR point clouds as an input similarly to RGB-D images.

PanopticFusion \cite{panopticfusion} is the first work to propose online integration of panoptic image segmentations to a 3D reconstruction. They integrate point clouds generated from segmented images to a TSDF voxel volume \cite{tsdf,voxblox} by greedily matching detected segments with the reconstruction and regulating each voxel's corresponding instance with a weighting function. Semantic labels are inferred in a bayesian manner based on confidence scores provided by the segmentation model. They also apply a Conditional Random Field (CRF) to regularise the reconstruction, improving results significantly. Voxblox++ \cite{voxblox++} -- introduced later the same year -- is a similar approach that also integrates segmented RGB-D images into a TSDF volume. It leverages geometric segmentation of depth images to improve instance segmentation accuracy. Both geometric and semantic segments are used to compute a pair-wise weight, which is used to greedily match them with segments in the reconstruction. Because of the geometric segmentation, the method allows segmentation of objects with no known semantic class in addition to objects recognised by the instance segmentation model. 

Recently, \cite{interactive_3d_scenes} built upon the idea of Voxblox++. They apply Voxblox++ for 3D instance integration, with two small but effective modifications: the pair-wise weight is replaced by a triplet weight that also takes semantic labels into account in the fusion, and -- in addition to geometric segments -- instance segments are fused if they overlap by a significant amount. The article introduces a method for searching and aligning CAD models to reconstructed objects based on geometry and semantic class, as well as geometrical and physical rules. With the CAD models, a contact graph and interactive virtual scene are reconstructed to allow a robot to simulate its interaction with the environment. SceneGraphFusion \cite{scenegraphfusion} is another approach that forms a scene graph online from a stream of RGB-D images, but unlike the above-mentioned approach, it generates the graph with a deep neural network, after which the panoptic labels for geometrically segmented portions of the 3D reconstruction are produced a side product.

Panoptic-MOPE \cite{panoptic_mope} is another recent approach, which integrates sequences of RGB-D images into a surfel reconstruction. Unlike other mentioned approaches -- which assume the camera pose either known or estimated elsewhere -- it also tracks camera movements based on geometric-, appearance- and semantic cues. The method also applies a novel RGB-D panoptic segmentation model. Although it is only tested on room-sized environments, the authors claim it could be scaled to larger environments as well.
%\input{intuition/intuition-L}


%\input{sbc/sbcE-y-L}

%\input{implication/implication}
%\input{numerical_comparisons/numerical_comparisons}

%\vspace{-1.5em}

%\input{rateless_code_comparison/rateless_code_comparison}
%\input{optimality}
%\input{individual_latencies/individual_latencies}
%\color{red}
%
Parallel composition arises when a trader is faced with multiple AMMs,
but wants to treat them them as if they were a single AMM.
In sequential composition,
the composed AMMs exchange ``hidden'' assets .
In parallel composition,
the composed AMMs compete for overlapping assets.

Suppose Alice wants to trade asset $X$ for asset $Y$.
Bob and Carol both offer AMMs to convert from $X$ to $Y$.
Bob's AMM is $B(x,y):=x^2y=\frac{3}{4}$ in state $(1,\frac{3}{4})$,
while Carol's AMM is $C(x,y):=x y = 1$ in state $(1,1)$.
Alice would like to compose the two AMMs and treat them as one AMM.
Bob provides a better initial rate of exchange for small trades,
but Carol provides less slippage for large trades.
One can check that if Alice converts 1 unit of $X$,
she gets more $Y$ assets from Bob than from Carol,
while if she converts 3 units,
she gets more from Carol.

This process is not the same as order-book clearing,
because order-book offers are typically expressed in terms of a fixed amount and a fixed price,
while parallel AMM offers are expressed in terms of price curves.
This type of composition occurs, for example,
when a trader is faced with multiple pools as in Uniswap v3~\cite{uniswapv3}.

A rational Alice will split her assets between
Bob and Carol to maximize her return.
Suppose $A(x,y)$ is in state $(a,f(a))$
and $B(x,y)$ in $(b,g(b))$.
Alice splits her $d$ assets,
transferring $t d$ to Bob's AMM
and $(1-t)d$ to Carol's,
returning
\begin{equation*}
  f(a) - f(a+tx) + g(b) - g(b+(1-t)x)
\end{equation*}
units of $Y$.
Let $h(x) = f(a+tx) + g(b+(1-t)x)$.
Define the \emph{parallel composition} of $B$ and $C$
with respect to $v=(t,1-t)$ to be.
\begin{equation*}
    (B\|C)(x,y) := y - h(x) = 0.
\end{equation*}
\begin{lemma}
  $(B\|C)(x,y)$ is a 2-dimensional AMM.
\end{lemma}
\begin{proof}
  $(B\|C)(x,y)$ is twice-differentiable because $f$ and $g$ are twice-differentiable.
  To check that $(B\|C)(x,y)$ is strictly increasing,
  let $x' \geq x$ and $y' \geq y$ where at least one inequality is strict.
  For the first case, suppose $x' > x$ and $y' \geq y$.
  \begin{align*}
    f(a+tx') &< f(a+tx) \\
    g(b+(1-t)x') &< g(b+(1-t)x) \\
    f(a+tx') + g(b+(1-t)x') &< f(a+tx) + g(b+(1-t)x) \\
    h(x') &< h(x) \\
    y - h(x') &> y-h(x)\\
    y' - h(x') &> y-h(x)
  \end{align*}
  The case where $x' \geq x$ and $y'> y$ is similar.

  To check that $\upper((B\|C)(x,y))$ is strictly convex,
  we can verify $h$ is strictly convex.
  Pick distinct $x,x'$.
  For $s \in (0,1)$,
  \begin{multline*}
    s f(a+t x) + (1-s) f(a+t x')\\
    > f(s (a+t x) + (1-s) (a+t x')) 
  \end{multline*}
  \begin{multline*}
    s g(b+(1-t)x) + (1-s) g(b+(1-t)x') \\
    >g(s (b+(1-t)x) + (1-s) (b+(1-t)x'))
  \end{multline*}
  \begin{multline*}
    s (f(a+t x) + g(b+(1-t)x))\\ + (1-s)(f(a + t x')+g(b+(1-t)x'))\\
    > f(s (a + t x) + (1-s) (a + t x'))\\ + 
    g(s (b+(1-t)x) + (1-s) (b+(1-t)x'))
  \end{multline*}
  \begin{equation*}
    s h(x) + (1-s)h(x') > h(s x + (1-s)x')  \\
  \end{equation*}
  which establishes the claim.
\end{proof}

\begin{lemma}
  Let $A := (x,f(x))$, $B := (y,g(y))$ be two AMMs trading assets $X$ and $Y$,
  such that $(a,f(a))$ and $(b,g(b))$ are their respective stable points
  for the valuation $(v,1-v)$.
  If $h_t(x) = f(a + tx) + g(b + (1-t)x)$,
  then $(0,h_t(0))$ is stable point on $A \| B$ with respect to $(v,1-v)$ for all $t \in \Reals$.
\end{lemma}

\begin{proof}
By assumption we have
\begin{align*}
    v a + (1-v)f(a) &< v(a + tx) + (1-v)f(a + tx)\\
    (1-v)f(a) &< tvx + (1-v)f(a + tx)
\end{align*}
and
\begin{align*}
  vb + (1-v)g(b) &< v(b + (1-t)x) + (1-v)g(b + (1-t)x)\\
  (1-v)g(b) &< (1-t)vx + (1-v)g(b + (1-t)x),
\end{align*}
yielding
\begin{align*}
  v 0 + (1-v)h_t(0)
  &= (1-v)h_t(0) \\
  &= (1-v)f(a) + (1-v)g(b) \\
  &< tvx + (1-v)f(a + tx) + (1-t)vx \\
  &\quad \quad + (1-v)g(b + (1-t)x) \\
  &= vx + (1-v)(f(a + tx) + g(b + (1-t)x)) \\
  &= vx + (1-v)h(x)
\end{align*}
so $(0,h_t(0))$ is a stable point for $(v,(1-v))$.
\end{proof}


Parallel composition is well-defined for any valuation,
but what valuation should a rational Alice pick?
Differentiating with respect to $t$ yields
\begin{align}
  0 &= -x f'(a+t x) + x g'(b+(1-t)x)\nonumber\\
  x f'(a+t x) &= x g'(b+(1-t)x) \nonumber\\
  f'(a+t x) &= g'(b+(1-t)x).\eqnlabel{best-split}
\end{align}
Alice maximizes her return when she splits her
assets so that Bob and Carol end up offering the same rate.
Informally, if Bob had ended up providing a better rate,
then Alice should have given him a larger share.
If there is no $t \in (0,1)$ that satisfies \eqnref{best-split},
Alice should give all her assets to the AMM with the better rate.

How should parallel composition be defined for AMMs with multiple asset types?
Suppose Alice has some combination $\bd$ of assets in $X_1,\ldots,X_p$
that she wants to convert into some combination of assets in $Y_1,\ldots,Y_q$.
Alice has a choice of two alternative AMMs:
$B(x_1,\ldots,x_p,y_1,\ldots,y_q)$ and $C(x_1,\ldots,x_p,y_1,\ldots,y_q)$.
Perhaps the most sensible way to define parallel composition is through
asset virtualization.
Alice's input asset vector $\bd$ induces a valuation $\bd/ \|\bd\|_1$
which can be used to define a virtual asset $X$ from $X_1,\ldots,X_p$.
Alice chooses a valuation $\bv$ for her $Y_1,\ldots,Y_q$ outputs
(perhaps the market valuation)
which can be used to define a virtual asset $Y$ from $Y_1,\ldots,Y_q$.
After asset virtualization, the alternative AMMs have the form:
$\tilde{B}(\bx,t)$ and $\tilde{C}(\bx,t)$,
and the definition of parallel composition proceeds as before.

The next lemma describes properties of stable points for two 2-dimensional AMMs composed in parallel.

If $(\bx,f(\bx))$ and $(\by,g(\by))$ are two $n$-dimensional AMMs in states $(\ba,f(\ba))$ and $(\bb,g(\bb))$,
we can define parallel composition as follows.
For $\bt \in [0,1]^{n-1}$,
let $h_{\bt}(\bx) = f(\ba + \bt * \bx) + g(\bb + (\bone - \bt )* \bx)$,
where $*$ is component-wise multiplication.

\begin{lemma}
  \lemmalabel{parallel-stable}
  Let $(\bv,1-\|\bv\|_1))$ be a valuation,
  and let $A := (\bx,f(\bx))$ and $B := (\by,g(\by))$.
  If $(\ba,f(\ba))$ and $(\bb,g(\bb))$ are both stable points with respect to $(\bv,1-\|\bv\|_1))$.
  then $(\bzero,h_{\bt}(\bzero))$ is the stable point on $A \| B$  for $(\bv,1-\|\bv\|_1)$
  for all $\bt \in \Reals^{n-1}$.
  \end{lemma}

  \begin{proof}
    For $\bt \in \Reals^{n-1}$,
    \begin{multline*}
      \bv \cdot \ba + (1-\|\bv\|_1)f(\ba)\\
      < \bv \cdot (\ba + \bt * \bx) + (1-\|\bv\|_1)f(\ba + \bt * \bx)
    \end{multline*}
    \begin{equation*}
      (1-\|\bv\|_1)f(\ba)
      < (\bt * \bx) \cdot \bv + (1-\|\bv\|_1)f(\ba + \bt * \bx)
    \end{equation*}
    and
    \begin{multline*}
      \bv \cdot \bb + (1-\|\bv\|_1)g(\bb)\\
      < \bv \cdot (\bb + (\bone - \bt) * \by) + (1-\|\bv\|_1)g(\bb + (\bone - \bt) * \by)
    \end{multline*}
    \begin{multline*}
      (1-\|\bv\|_1)g(\bb) \\
      < ((\bone - \bt) * \by) \cdot \bv + (1-\|\bv\|_1)g(\bb + (\bone - \bt) * \by),
     \end{multline*}
implying
\begin{align*}
  \bv \cdot \bzero + &(1-\|\bv\|_1)h_t(\bzero) \\
  &= (1-\|\bv\|_1)h_t(\bzero) \\
  &= (1-\|\bv\|_1)f(\ba) +(1-\|\bv\|_1)g(\bb) \\
  &< (\bt * \bx) \cdot \bv + (1-\|\bv\|_1)f(\ba + \bt * \bx) \\
  &\quad \quad+ (\bone-\bt) * \bx \cdot \bv + \\
  &\quad \quad (1-\|\bv\|_1)g(\bb + (\bone-\bt) * \bx)\\
    &= \bv \cdot \bx + (1-\|\bv\|_1)(f(\ba + \bt \cdot \bx) \\
    &\quad \quad + g(\bb + (\bone-\bt) \cdot \bx))  \\
    &= \bv \cdot \bx + (1-\|\bv\|_1)h(\bx)
\end{align*}
so $(0,h_t(0))$ is a stable point for $(\bv,1-\|\bv\|_1)$.
\end{proof}
Most generally, if we have two AMMs $A(\bx,\bz)$ and $B(\by,\bz')$, and valuation $\bw$,
we can write $A| \bw$ as $(\bx,f(\bx))$ and $B| \bw$ as $(\by,g(\by))$.
We then can define parallel composition as before.

\begin{theorem}
    Let $(\bv,\bv')$ be a valuation, and let $A(\bx,\bz)$ and $B(\by,\bz')$ two AMMs.
    If $(\ba^{*},\bb^{*})$ is the stable point for $A$,
    and $(\bc^{*},\bd^{*})$ the stable point for $B$,
    both with respect to $(\bv,\bv')$,
    then $(\bv,\|\bv'\|_2^2)/\|(\bv,\|\bv'\|_2^2)\|_1$ is the stable point for $(A | \bv' ) \| (B | \bv')$.
\end{theorem}

\begin{proof}\sloppy
  \lemmaref{virtual-stable} implies that $(\ba^{*},f(\ba^{*}))$ and $(\bc^{*},g(\bc^{*}))$
  are stable on $A | \bv'$ and $B | \bv'$,
  both with respect to $(\bv,\|\bv'\|_2^2)$. 
\lemmaref{parallel-stable} implies that $(\bzero,h_{\bt}(\bzero))$
is the stable point for
$(A | \bv' ) \| (B | \bv')$ with respect to valuation $(\bv,\|\bv'\|_2^2)$.
\end{proof}

%\color{black}
%
Parallel composition arises when a trader is faced with multiple AMMs,
but wants to treat them them as if they were a single AMM.
In sequential composition,
the composed AMMs exchange ``hidden'' assets .
In parallel composition,
the composed AMMs compete for overlapping assets.

Suppose Alice wants to trade asset $X$ for asset $Y$.
Bob and Carol both offer AMMs to convert from $X$ to $Y$.
Bob's AMM is $B(x,y):=x^2y=\frac{3}{4}$ in state $(1,\frac{3}{4})$,
while Carol's AMM is $C(x,y):=x y = 1$ in state $(1,1)$.
Alice would like to compose the two AMMs and treat them as one AMM.
Bob provides a better initial rate of exchange for small trades,
but Carol provides less slippage for large trades.
One can check that if Alice converts 1 unit of $X$,
she gets more $Y$ assets from Bob than from Carol,
while if she converts 3 units,
she gets more from Carol.

This process is not the same as order-book clearing,
because order-book offers are typically expressed in terms of a fixed amount and a fixed price,
while parallel AMM offers are expressed in terms of price curves.
This type of composition occurs, for example,
when a trader is faced with multiple pools as in Uniswap v3~\cite{uniswapv3}.

A rational Alice will split her assets between
Bob and Carol to maximize her return.
Suppose $A(x,y)$ is in state $(a,f(a))$
and $B(x,y)$ in $(b,g(b))$.
Alice splits her $d$ assets,
transferring $t d$ to Bob's AMM
and $(1-t)d$ to Carol's,
returning
\begin{equation*}
  f(a) - f(a+tx) + g(b) - g(b+(1-t)x)
\end{equation*}
units of $Y$.
Let $h(x) = f(a+tx) + g(b+(1-t)x)$.
Define the \emph{parallel composition} of $B$ and $C$
with respect to $v=(t,1-t)$ to be.
\begin{equation*}
    (B\|C)(x,y) := y - h(x) = 0.
\end{equation*}
\begin{lemma}
  $(B\|C)(x,y)$ is a 2-dimensional AMM.
\end{lemma}
\begin{proof}
  $(B\|C)(x,y)$ is twice-differentiable because $f$ and $g$ are twice-differentiable.
  To check that $(B\|C)(x,y)$ is strictly increasing,
  let $x' \geq x$ and $y' \geq y$ where at least one inequality is strict.
  For the first case, suppose $x' > x$ and $y' \geq y$.
  \begin{align*}
    f(a+tx') &< f(a+tx) \\
    g(b+(1-t)x') &< g(b+(1-t)x) \\
    f(a+tx') + g(b+(1-t)x') &< f(a+tx) + g(b+(1-t)x) \\
    h(x') &< h(x) \\
    y - h(x') &> y-h(x)\\
    y' - h(x') &> y-h(x)
  \end{align*}
  The case where $x' \geq x$ and $y'> y$ is similar.

  To check that $\upper((B\|C)(x,y))$ is strictly convex,
  we can verify $h$ is strictly convex.
  Pick distinct $x,x'$.
  For $s \in (0,1)$,
  \begin{multline*}
    s f(a+t x) + (1-s) f(a+t x')\\
    > f(s (a+t x) + (1-s) (a+t x')) 
  \end{multline*}
  \begin{multline*}
    s g(b+(1-t)x) + (1-s) g(b+(1-t)x') \\
    >g(s (b+(1-t)x) + (1-s) (b+(1-t)x'))
  \end{multline*}
  \begin{multline*}
    s (f(a+t x) + g(b+(1-t)x))\\ + (1-s)(f(a + t x')+g(b+(1-t)x'))\\
    > f(s (a + t x) + (1-s) (a + t x'))\\ + 
    g(s (b+(1-t)x) + (1-s) (b+(1-t)x'))
  \end{multline*}
  \begin{equation*}
    s h(x) + (1-s)h(x') > h(s x + (1-s)x')  \\
  \end{equation*}
  which establishes the claim.
\end{proof}

\begin{lemma}
  Let $A := (x,f(x))$, $B := (y,g(y))$ be two AMMs trading assets $X$ and $Y$,
  such that $(a,f(a))$ and $(b,g(b))$ are their respective stable points
  for the valuation $(v,1-v)$.
  If $h_t(x) = f(a + tx) + g(b + (1-t)x)$,
  then $(0,h_t(0))$ is stable point on $A \| B$ with respect to $(v,1-v)$ for all $t \in \Reals$.
\end{lemma}

\begin{proof}
By assumption we have
\begin{align*}
    v a + (1-v)f(a) &< v(a + tx) + (1-v)f(a + tx)\\
    (1-v)f(a) &< tvx + (1-v)f(a + tx)
\end{align*}
and
\begin{align*}
  vb + (1-v)g(b) &< v(b + (1-t)x) + (1-v)g(b + (1-t)x)\\
  (1-v)g(b) &< (1-t)vx + (1-v)g(b + (1-t)x),
\end{align*}
yielding
\begin{align*}
  v 0 + (1-v)h_t(0)
  &= (1-v)h_t(0) \\
  &= (1-v)f(a) + (1-v)g(b) \\
  &< tvx + (1-v)f(a + tx) + (1-t)vx \\
  &\quad \quad + (1-v)g(b + (1-t)x) \\
  &= vx + (1-v)(f(a + tx) + g(b + (1-t)x)) \\
  &= vx + (1-v)h(x)
\end{align*}
so $(0,h_t(0))$ is a stable point for $(v,(1-v))$.
\end{proof}


Parallel composition is well-defined for any valuation,
but what valuation should a rational Alice pick?
Differentiating with respect to $t$ yields
\begin{align}
  0 &= -x f'(a+t x) + x g'(b+(1-t)x)\nonumber\\
  x f'(a+t x) &= x g'(b+(1-t)x) \nonumber\\
  f'(a+t x) &= g'(b+(1-t)x).\eqnlabel{best-split}
\end{align}
Alice maximizes her return when she splits her
assets so that Bob and Carol end up offering the same rate.
Informally, if Bob had ended up providing a better rate,
then Alice should have given him a larger share.
If there is no $t \in (0,1)$ that satisfies \eqnref{best-split},
Alice should give all her assets to the AMM with the better rate.

How should parallel composition be defined for AMMs with multiple asset types?
Suppose Alice has some combination $\bd$ of assets in $X_1,\ldots,X_p$
that she wants to convert into some combination of assets in $Y_1,\ldots,Y_q$.
Alice has a choice of two alternative AMMs:
$B(x_1,\ldots,x_p,y_1,\ldots,y_q)$ and $C(x_1,\ldots,x_p,y_1,\ldots,y_q)$.
Perhaps the most sensible way to define parallel composition is through
asset virtualization.
Alice's input asset vector $\bd$ induces a valuation $\bd/ \|\bd\|_1$
which can be used to define a virtual asset $X$ from $X_1,\ldots,X_p$.
Alice chooses a valuation $\bv$ for her $Y_1,\ldots,Y_q$ outputs
(perhaps the market valuation)
which can be used to define a virtual asset $Y$ from $Y_1,\ldots,Y_q$.
After asset virtualization, the alternative AMMs have the form:
$\tilde{B}(\bx,t)$ and $\tilde{C}(\bx,t)$,
and the definition of parallel composition proceeds as before.

The next lemma describes properties of stable points for two 2-dimensional AMMs composed in parallel.

If $(\bx,f(\bx))$ and $(\by,g(\by))$ are two $n$-dimensional AMMs in states $(\ba,f(\ba))$ and $(\bb,g(\bb))$,
we can define parallel composition as follows.
For $\bt \in [0,1]^{n-1}$,
let $h_{\bt}(\bx) = f(\ba + \bt * \bx) + g(\bb + (\bone - \bt )* \bx)$,
where $*$ is component-wise multiplication.

\begin{lemma}
  \lemmalabel{parallel-stable}
  Let $(\bv,1-\|\bv\|_1))$ be a valuation,
  and let $A := (\bx,f(\bx))$ and $B := (\by,g(\by))$.
  If $(\ba,f(\ba))$ and $(\bb,g(\bb))$ are both stable points with respect to $(\bv,1-\|\bv\|_1))$.
  then $(\bzero,h_{\bt}(\bzero))$ is the stable point on $A \| B$  for $(\bv,1-\|\bv\|_1)$
  for all $\bt \in \Reals^{n-1}$.
  \end{lemma}

  \begin{proof}
    For $\bt \in \Reals^{n-1}$,
    \begin{multline*}
      \bv \cdot \ba + (1-\|\bv\|_1)f(\ba)\\
      < \bv \cdot (\ba + \bt * \bx) + (1-\|\bv\|_1)f(\ba + \bt * \bx)
    \end{multline*}
    \begin{equation*}
      (1-\|\bv\|_1)f(\ba)
      < (\bt * \bx) \cdot \bv + (1-\|\bv\|_1)f(\ba + \bt * \bx)
    \end{equation*}
    and
    \begin{multline*}
      \bv \cdot \bb + (1-\|\bv\|_1)g(\bb)\\
      < \bv \cdot (\bb + (\bone - \bt) * \by) + (1-\|\bv\|_1)g(\bb + (\bone - \bt) * \by)
    \end{multline*}
    \begin{multline*}
      (1-\|\bv\|_1)g(\bb) \\
      < ((\bone - \bt) * \by) \cdot \bv + (1-\|\bv\|_1)g(\bb + (\bone - \bt) * \by),
     \end{multline*}
implying
\begin{align*}
  \bv \cdot \bzero + &(1-\|\bv\|_1)h_t(\bzero) \\
  &= (1-\|\bv\|_1)h_t(\bzero) \\
  &= (1-\|\bv\|_1)f(\ba) +(1-\|\bv\|_1)g(\bb) \\
  &< (\bt * \bx) \cdot \bv + (1-\|\bv\|_1)f(\ba + \bt * \bx) \\
  &\quad \quad+ (\bone-\bt) * \bx \cdot \bv + \\
  &\quad \quad (1-\|\bv\|_1)g(\bb + (\bone-\bt) * \bx)\\
    &= \bv \cdot \bx + (1-\|\bv\|_1)(f(\ba + \bt \cdot \bx) \\
    &\quad \quad + g(\bb + (\bone-\bt) \cdot \bx))  \\
    &= \bv \cdot \bx + (1-\|\bv\|_1)h(\bx)
\end{align*}
so $(0,h_t(0))$ is a stable point for $(\bv,1-\|\bv\|_1)$.
\end{proof}
Most generally, if we have two AMMs $A(\bx,\bz)$ and $B(\by,\bz')$, and valuation $\bw$,
we can write $A| \bw$ as $(\bx,f(\bx))$ and $B| \bw$ as $(\by,g(\by))$.
We then can define parallel composition as before.

\begin{theorem}
    Let $(\bv,\bv')$ be a valuation, and let $A(\bx,\bz)$ and $B(\by,\bz')$ two AMMs.
    If $(\ba^{*},\bb^{*})$ is the stable point for $A$,
    and $(\bc^{*},\bd^{*})$ the stable point for $B$,
    both with respect to $(\bv,\bv')$,
    then $(\bv,\|\bv'\|_2^2)/\|(\bv,\|\bv'\|_2^2)\|_1$ is the stable point for $(A | \bv' ) \| (B | \bv')$.
\end{theorem}

\begin{proof}\sloppy
  \lemmaref{virtual-stable} implies that $(\ba^{*},f(\ba^{*}))$ and $(\bc^{*},g(\bc^{*}))$
  are stable on $A | \bv'$ and $B | \bv'$,
  both with respect to $(\bv,\|\bv'\|_2^2)$. 
\lemmaref{parallel-stable} implies that $(\bzero,h_{\bt}(\bzero))$
is the stable point for
$(A | \bv' ) \| (B | \bv')$ with respect to valuation $(\bv,\|\bv'\|_2^2)$.
\end{proof}

%
\begin{comment}
\begin{figure}
\includegraphics[width=\linewidth]{figs/beyond_tss_lesion.pdf}
\caption[]{End-to-End runtime lesion study of the entire MNIST dataset and the FMA featurized music dataset. Each of DROP's contributions provides a runtime improvement.}
\label{fig:beyond_lesion}
\end{figure}
\end{comment}



\section{Conclusion}
\label{sec:conclusion}

Advanced data analytics techniques must scale to rising data volumes. 
DR techniques offer a powerful toolkit when processing these datasets, with PCA frequently outperforming popular techniques in exchange for high computational cost. 
In response, we propose DROP, a new dimensionality reduction optimizer. 
DROP combines progressive sampling, progress estimation, and online aggregation to identify high quality low dimensional bases via PCA without processing the entire dataset by balancing the runtime of downstream tasks and achieved dimensionality. 
Thus, DROP provides a first step in bridging the gap between quality and efficiency in end-to-end DR for downstream \red{analytics}. 

%We revisit canonical operators for time series dimensionality reduction and the measurement study of~\cite{keogh-study}, and show that PCA is more effective than popular alternatives in the data mining literature often by a margin of over $2\times$ on average on gold-standard time series benchmark data sets with respect to output data dimension. More surprisingly, we empirically demonstrate that a small number of samples are sufficient to accurately characterize directions of maximum variance and obtain a high-quality low-dimensional transformation.




%\vspace{-2em}

%%\appendices
\section{Pseudocode for Algorithm of Section~\ref{subsec:instantly_decodable}}
\label{app:pseudocode}

\algdef{SE}[EVENT]{Event}{EndEvent}[1]{\textbf{upon event}\ #1\ \algorithmicdo}{\algorithmicend\ \textbf{event}}%
\algtext*{EndEvent}

\begin{algorithm}
\caption{Coding for Three Users with Feedback}\label{alg:three_users}
\begin{algorithmic}[1]
\State \textbf{Initialize}: $r_i \gets 0, \forall i \in \mathcal{U}$  % \Comment{Packets received by each user}
\ForAll {$t \in [N]$}
    \While {$\nexists \ i \in \mathcal{U} \ s.t.\ Z_i = 0$}
%    \State Send $S(t)$ until at least one user receives it
    	\State Transmit $S(t)$
    \EndWhile
%    \If {$\exists \ i \in \mathcal{U}, T \in \mathbb{N} \ s.t.\ N_i(T) = 0$}
%    \If {$\exists \ i \in \mathcal{U} \ s.t.\ N_i = 0$}
        \State $Q_{\mathcal{E}} \gets Q_{\mathcal{E}} \cup \{S(t)\}$
        \State $r_j\gets r_j + 1 \qquad \forall \ j \ s.t.\  Z_j = 0$
%    \EndIf
\EndFor

%\While{$\exists \ \textrm{distinct} i, j, k \in \mathcal{U}, i\neq j \neq k \ s.t.\ Q_i \neq \varnothing \ \textbf{and} \ \Q{j,k} \neq \varnothing$}
\While{$\exists \ \textrm{distinct} \ i, j, k \in \mathcal{U}, \ s.t.\ Q_i \neq \varnothing \ \textbf{and} \ \Q{j,k} \neq \varnothing$}
    \While {$\nexists \ l \in \mathcal{U} \ s.t.\ Z_l = 0$}
%	\State Let $q_i \in Q_i$, $q_{j,k} \in \Q{j,k}$ 
%       \State Transmit $q_i \oplus q_{j,k}$ until at least one user receives it  
       \State Transmit $q_i \oplus q_{j,k}$ where $q_i \in Q_i$, $q_{j,k} \in \Q{j,k}$ 
    \EndWhile
    \State $r_l\gets r_l + 1 \qquad \forall \ l \ s.t.\  Z_l = 0$
    \If {$Z_i = 0$}
    	\State $Q_i \gets Q_i \setminus \{q_i\}$
    \EndIf
    \If {$Z_j = 0 \ \textbf{and} \ Z_k = 1$}
    	\State $Q_k \gets Q_k \cup \{q_{j, k}\}$
       \State $\Q{j,k} \gets \Q{j, k} \setminus \{q_{j,k}\}$
    \ElsIf {$Z_j = 1 \ \textbf{and} \ Z_k = 0$}
    	\State $Q_j \gets Q_j \cup \{q_{j, k}\}$
	\State $\Q{j,k} \gets \Q{j, k} \setminus \{q_{j,k}\}$
    \ElsIf {$Z_j = 0 \ \textbf{and} \ Z_k = 0$}
    	\State $\Q{j,k} \gets \Q{j, k} \setminus \{q_{j,k}\}$
    \EndIf    
\EndWhile

\While{$Q_i \neq \varnothing \  \forall \ i \in \mathcal{U}$}
	\While {$\nexists \ l \in \mathcal{U} \ s.t.\ Z_l = 0$}
		\State Transmit $q_1 \oplus q_{2} \oplus q_{3}$, where $q_i \in Q_i \ \forall i \in \mathcal{U}$
	\EndWhile
	\ForAll {$l \in \mathcal{U} \ s.t.\ Z_l = 0$}
		\State $Q_l \gets Q_l \setminus \{q_l\}$
		\State $z_l \gets z_l + 1$
	\EndFor
\EndWhile

\Event{$ \exists \ i \ s.t.\ r_i \geq 1- d_i  $} %\Comment{One user finished}
%	\State Let $j, k \in \mathcal{U} \setminus \{i\}, \ s.t.\ j \neq k$
	\State $Q_j \gets Q_j \cup \Q{i,j} \qquad \forall \ j \in \mathcal{U} \setminus \{i\}$
	\State \textbf{run} two-user algorithm of Section~\ref{sec:two_users}
\EndEvent
\end{algorithmic}
\end{algorithm}


% An example of a floating figure using the graphicx package.
% Note that \label must occur AFTER (or within) \caption.
% For figures, \caption should occur after the \includegraphics.
% Note that IEEEtran v1.7 and later has special internal code that
% is designed to preserve the operation of \label within \caption
% even when the captionsoff option is in effect. However, because
% of issues like this, it may be the safest practice to put all your
% \label just after \caption rather than within \caption{}.
%
% Reminder: the "draftcls" or "draftclsnofoot", not "draft", class
% option should be used if it is desired that the figures are to be
% displayed while in draft mode.
%
%\begin{figure}[!t]
%\centering
%\includegraphics[width=2.5in]{myfigure}
% where an .eps filename suffix will be assumed under latex,
% and a .pdf suffix will be assumed for pdflatex; or what has been declared
% via \DeclareGraphicsExtensions.
%\caption{Simulation Results}
%\label{fig_sim}
%\end{figure}

% Note that IEEE typically puts floats only at the top, even when this
% results in a large percentage of a column being occupied by floats.


% An example of a double column floating figure using two subfigures.
% (The subfig.sty package must be loaded for this to work.)
% The subfigure \label commands are set within each subfloat command, the
% \label for the overall figure must come after \caption.
% \hfil must be used as a separator to get equal spacing.
% The subfigure.sty package works much the same way, except \subfigure is
% used instead of \subfloat.
%
%\begin{figure*}[!t]
%\centerline{\subfloat[Case I]\includegraphics[width=2.5in]{subfigcase1}%
%\label{fig_first_case}}
%\hfil
%\subfloat[Case II]{\includegraphics[width=2.5in]{subfigcase2}%
%\label{fig_second_case}}}
%\caption{Simulation results}
%\label{fig_sim}
%\end{figure*}
%
% Note that often IEEE papers with subfigures do not employ subfigure
% captions (using the optional argument to \subfloat), but instead will
% reference/describe all of them (a), (b), etc., within the main caption.


% An example of a floating table. Note that, for IEEE style tables, the
% \caption command should come BEFORE the table. Table text will default to
% \footnotesize as IEEE normally uses this smaller font for tables.
% The \label must come after \caption as always.
%
%\begin{table}[!t]
%% increase table row spacing, adjust to taste
%\renewcommand{\arraystretch}{1.3}
% if using array.sty, it might be a good idea to tweak the value of
% \extrarowheight as needed to properly center the text within the cells
%\caption{An Example of a Table}
%\label{table_example}
%\centering
%% Some packages, such as MDW tools, offer better commands for making tables
%% than the plain LaTeX2e tabular whiIEEEtranch is used here.
%\begin{tabular}{|c||c|}
%\hline
%One & Two\\
%\hline
%Three & Four\\
%\hline
%\end{tabular}
%\end{table}


% Note that IEEE does not put floats in the very first column - or typically
% anywhere on the first page for that matter. Also, in-text middle ("here")
% positioning is not used. Most IEEE journals use top floats exclusively.
% Note that, LaTeX2e, unlike IEEE journals, places footnotes above bottom
% floats. This can be corrected via the \fnbelowfloat command of the
% stfloats package.



%\section{Conclusion and Future Work}
%One interesting direction is to identify analytical solutions or approximate solutions to the degree distribution optimization problem, and compare with the theoretical results.

%The authors would like to thank...


% Can use something like this to put references on a page
% by themselves when using endfloat and the captionsoff option.
\ifCLASSOPTIONcaptionsoff
  \newpage
\fi



% trigger a \newpage just before the given reference
% number - used to balance the columns on the last page
% adjust value as needed - may need to be readjusted if
% the document is modified later
%\IEEEtriggeratref{8}
% The "triggered" command can be changed if desired:
%\IEEEtriggercmd{\enlargethispage{-5in}}




% references section

% can use a bibliography generated by BibTeX as a .bbl file
% BibTeX documentation can be easily obtained at:
% http://www.ctan.org/tex-archive/biblio/bibtex/contrib/doc/
% The IEEEtran BibTeX style support page is at:
% http://www.michaelshell.org/tex/ieeetran/bibtex/
%\bibliographystyle{IEEEtranTCOM}
% argument is your BibTeX string definitions and bibliography database(s)
%\bibliography{IEEEabrv,../bib/paper}
%
% <OR> manually copy in the resultant .bbl file
% set second argument of \begin to the number of references
% (used to reserve space for the reference number labels box)
%
%\begin{thebibliography}{1}
%
%\bibitem{IEEEhowto:kopka}
%H.~Kopka and P.~W. Daly, \emph{A Guide to \LaTeX}, 3rd~ed.\hskip 1em plus
%  0.5em minus 0.4em\relax Harlow, England: Addison-Wesley, 1999.
%
%\end{thebibliography}


%\vspace{-2em}
\vspace{-1em}

\bibliographystyle{IEEEtran}
%\bibliographystyle{IEEEtranTCOM}
%\bibliographystyle{hieeetr}
%\bibliography{IEEEabrv,itw2013emina}
\bibliography{IEEEabrv,itw2013emina2}

% biography section
%
% If you have an EPS/PDF photo (graphicx package needed) extra braces are
% needed around the contents of the optional argument to biography to prevent
% the LaTeX parser from getting confused when it sees the complicated
% \includegraphics command within an optional argument. (You could create
% your own custom macro containing the \includegraphics command to make things
% simpler here.)
%\begin{biography}[{\includegraphics[width=1in,height=1.25in,clip,keepaspectratio]{mshell}}]{Michael Shell}
% or if you just want to reserve a space for a photo:

%\begin{IEEEbiographynophoto} {Louis Tan}
%	received his BASc degree in Engineering Science with a major in Computer Engineering from the University of Toronto, Toronto, ON, Canada in 2010, and his MASc degree in Electrical Engineering, also from the University of Toronto, in 2012.  He is currently working towards his PhD degree in the Electrical and Computer Engineering Department at the University of Toronto.  His research interests include joint source-channel coding, multimedia systems, machine learning, and network information theory.
%\end{IEEEbiographynophoto}

%\begin{IEEEbiographynophoto} {Yao Li}
%received her B.Eng.\ degree in Information Engineering and Electronics from Tsinghua University, Beijing, China, and her Ph.D degree in Electrical and Computer Engineering from Rutgers, the State University of New Jersey, in 2012. She was a postdoctoral scholar at the Laboratory for Robust Information Systems (LORIS) at University of California, Los Angeles, between 2012 and 2014. She is now with Akamai Technologies, Inc. Her research interests include coding solutions for content distribution, robust inference in sensor networks, and computing on noisy hardware.
%\end{IEEEbiographynophoto}

%\begin{IEEEbiographynophoto} {Ashish Khisti}
%received his BASc Degree (2002) in Engineering Sciences (Electrical Option) from University of Toronto, and his S.M and Ph.D. Degrees in Electrical Engineering from the Massachusetts Institute of Technology. Between 2009-2015, he was an assistant professor in the Electrical and Computer Engineering department at the University of Toronto. He is presently an associate professor, and holds a Canada Research Chair in the same department. He is a recipient of an Ontario Early Researcher Award, the Hewlett-Packard Innovation Research Award and the Harold H. Hazen teaching assistant award from MIT. He presently serves as an associate editor for IEEE Transactions on Information Theory and is also a guest editor for the Proceedings of the IEEE (Special Issue on Secure Communications via Physical-Layer and Information-Theoretic Techniques). 
%\end{IEEEbiographynophoto}

%\begin{IEEEbiographynophoto} {Emina Soljanin}
%is a Professor at Rutgers University. Before joining Rutgers in January 2016, she worked as the Distinguished Member of Technical Staff at Bell Labs, doing research in information, coding, and, more recently, queueing theory. Her interests and expertise are wide. Over the past quarter of the century, she has participated in numerous research and business projects, as diverse as power system optimization, magnetic recording, color space quantization, hybrid ARQ, network coding, data and network security, and quantum information theory and networking. Prof. Soljanin served as the Associate Editor for Coding Techniques, for the IEEE Transactions on Information Theory, on the Information Theory Society Board of Governors, and in various roles on other journal editorial boards and conference program committees. She is a co-organizer of the DIMACS 2001-2005 Special Focus on Computational Information Theory and Coding and 2011-2015 Special Focus on Cybersecurity. Prof. Soljanin is an IEEE Fellow, and serves a distinguished lecturer for the IEEE Information Theory Society in 2015 and 2016. 
%\end{IEEEbiographynophoto}

% You can push biographies down or up by placing
% a \vfill before or after them. The appropriate
% use of \vfill depends on what kind of text is
% on the last page and whether or not the columns
% are being equalized.

\vfill

% Can be used to pull up biographies so that the bottom of the last one
% is flush with the other column.
%\enlargethispage{-5in}



% that's all folks
\end{document}


