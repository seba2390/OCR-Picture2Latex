%
% File acl2020.tex
%
%% Based on the style files for ACL 2020, which were
%% Based on the style files for ACL 2018, NAACL 2018/19, which were
%% Based on the style files for ACL-2015, with some improvements
%%  taken from the NAACL-2016 style
%% Based on the style files for ACL-2014, which were, in turn,
%% based on ACL-2013, ACL-2012, ACL-2011, ACL-2010, ACL-IJCNLP-2009,
%% EACL-2009, IJCNLP-2008...
%% Based on the style files for EACL 2006 by 
%%e.agirre@ehu.es or Sergi.Balari@uab.es
%% and that of ACL 08 by Joakim Nivre and Noah Smith

\documentclass[11pt,a4paper]{article}
\usepackage[hyperref]{acl2020}
\usepackage{times}
\usepackage{latexsym}
\usepackage{bm}
\renewcommand{\UrlFont}{\ttfamily\small}
\usepackage{multirow,multicol} 
\usepackage{booktabs}
%\usepackage{amsmath}
\usepackage{mathtools} % AW: added this to try to fix the ugly equation spacing around \sum. Mathtools also loads amsmath itself.
\usepackage{breqn}
\usepackage{float}
\restylefloat{table}
\usepackage{tabularx}
\usepackage{graphics}
\usepackage{adjustbox} 
\usepackage{subfigure}



% This is not strictly necessary, and may be commented out,
% but it will improve the layout of the manuscript,
% and will typically save some space.
\usepackage{microtype}

\aclfinalcopy % Uncomment this line for the final submission
\def\aclpaperid{725} %  Enter the acl Paper ID here

%\setlength\titlebox{5cm}
% You can expand the titlebox if you need extra space
% to show all the authors. Please do not make the titlebox
% smaller than 5cm (the original size); we will check this
% in the camera-ready version and ask you to change it back.

\newcommand\BibTeX{B\textsc{ib}\TeX}

\newcommand{\PermAcc}{Permutation Acceptance} % a macro
\def\Snospace~{\S{}} % If i don't add this overleaf complains
\renewcommand{\sectionautorefname}{\Snospace}
\renewcommand{\subsectionautorefname}{\Snospace}
\newcommand\boldred[1]{\textcolor{red}{\textbf{#1}}}
\newcommand\boldblue[1]{\textcolor{blue}{\textbf{#1}}}

\DeclareMathOperator*{\argminB}{argmin}

\title{UnNatural Language Inference}

% \author{First Author \\
%   Affiliation / Address line 1 \\
%   Affiliation / Address line 2 \\
%   Affiliation / Address line 3 \\
%   \texttt{email@domain} \\\And
%   Second Author \\
%   Affiliation / Address line 1 \\
%   Affiliation / Address line 2 \\
%   Affiliation / Address line 3 \\
%   \texttt{email@domain} \\}

\author{Koustuv Sinha\textsuperscript{1,2,3},
  Prasanna Parthasarathi\textsuperscript{1,2},
  Joelle Pineau\textsuperscript{1,2,3} and
  Adina Williams\textsuperscript{3} \\
  \textsuperscript{1} School of Computer Science, McGill University, Canada \\
  \textsuperscript{2} Montreal Institute of Learning Algorithms (Mila), Canada \\
  \textsuperscript{3} Facebook AI Research (FAIR)\\
  \{koustuv.sinha, prasanna.parthasarathi, jpineau, adinawilliams\}\\@\{mail.mcgill.ca, mail.mcgill.ca, cs.mcgill.ca, fb.com\}}

\date{}

\begin{document}
\maketitle
\begin{abstract}
Recent investigations into the inner-workings of state-of-the-art large-scale pre-trained Transformer-based Natural Language Understanding (NLU) models indicate that they appear to know humanlike syntax, at least to some extent. We provide novel evidence that complicates this claim: we find that state-of-the-art Natural Language Inference (NLI) models assign the same labels to permuted examples as they do to the original, i.e. they are largely invariant to random word-order permutations. This behavior notably differs from that of humans; we struggle with ungrammatical sentences. To measure the severity of this issue, we propose a suite of metrics and investigate which properties of particular permutations lead models to be word-order invariant. In the MNLI dataset, for example, we find almost all (98.7\%) examples contain at least one permutation which elicits the gold label. Models are sometimes even able to assign gold labels to permutations that they originally failed to predict correctly. We provide a comprehensive empirical evaluation of this phenomenon, and further show that this issue exists for both Transformers and pre-Transformer RNN / ConvNet based encoders, as well as across multiple languages (English and Mandarin Chinese). Our code and data are available at \href{https://github.com/facebookresearch/unlu}{https://github.com/facebookresearch/unlu}. 
    % Natural Language Understanding has witnessed a watershed moment with the introduction of large pre-trained Transformer networks. Despite impressive performance on NLU tasks including Natural Language Inference (NLI), we find that pre-trained Transformers are invariant to random word order permutations---i.e., they assign permuted examples the same labels as their non-permuted counterparts. This issue is present for pre-Transformer encoders as well and holds in both English and Chinese. Furthermore, we find that models are even able to correctly label permutations of examples that they initially failed to predict correctly.  We propose a suite of metrics to measure the severity of the issue and investigate which properties of particular permutations lead models to be word order invariant. %Our findings lead us to question whether our current models have learned anything akin to a human-like understanding of syntax.
\end{abstract}

\section{Introduction}

Of late, large scale pre-trained Transformer-based \citep{vaswani-etal-2017-attention} models---such as RoBERTa \citep{liu-et-al-2019-roberta}, BART \citep{lewis-etal-2020-bart}, and GPT-2 and -3 \citep{radford-etal-2019-language,brown-etal-2020-gpt3}---have exceeded recurrent neural networks' performance on many NLU tasks \citep{wang-etal-2018-glue, wang-etal-2019-superglue}. %The success of these models has prompted serious investigation, leading some to claim that pre-trained transformers can capture a surprising amount of lexical and semantic knowledge \cite{}, on the basis of their performance on either large benchmark suites \citep{wang-etal-2018-glue, wang-etal-2019-superglue} or on specialized test sets designed to evaluate linguistic knowledge \citep{linzen-etal-2016-assessing,  mccoy-etal-2019-right}. 
Several papers have even suggested that Transformers pretrained on a language modeling (LM) objective can capture syntactic information \citep{hewitt-manning-2019-structural,jawahar-etal-2019-bert, warstadt-bowman-2020-can, wu-etal-2020-perturbed}, with their self-attention layers being capable of surprisingly effective learning \cite{rogers2020}. In this work, we question such claims that current models ``know syntax''. 

% TODO: add three examples
\begin{table}[t]
    \centering
    \small
    \begin{adjustbox}{max width=0.95\linewidth}
    \begin{tabular}{p{11em}p{9em}p{3em}} % I hate the vertical line, can we get rid of it... 
    \toprule
     \bf Premise & \bf Hypothesis & \bf Predicted Label \\ \midrule
    Boats in daily use lie within feet of the fashionable bars and restaurants.  & There are boats close to bars and restaurants. & E \\ 
    \addlinespace[0.5em]
    restaurants and use feet of fashionable lie the in Boats within bars daily . & bars restaurants are There and to close boats . & E \\ \midrule
    He and his associates weren't operating at the level of metaphor. & He and his associates were operating at the level of the metaphor. & C\\  \addlinespace[0.5em]
    his at and metaphor the of were He operating associates n't level . & his the and metaphor level the were He at associates operating of . & C\\ 
    \bottomrule
    \end{tabular}
   \end{adjustbox}
    \caption{Examples from the MNLI Matched development set. Both the original example and the permuted one elicit the same classification label (entailment and contradiction respectively) from RoBERTa (large). 
    A simple demo is provided in an associated \href{https://colab.research.google.com/drive/1vv8Xmag1go3dib4vZXUZXAFB4ltDaMH7?usp=sharing}{Google Colab notebook.}}
    \label{tab:example}
\end{table}

Since there are many ways to investigate ``syntax'', we must be clear on what we mean by the term.  %A natural and common perspective from many formal theories of linguistics (e.g., \citealt{chomsky-1995-minimalist}) is that knowing a natural language requires that you know the syntax of that language. Concretely, if a model understands the syntax of a sentence $D= {w_1, w_2, ..., w_n}$, (where $w_i$ is a word), then it should be sensitive to (at least) the \textit{order} of the words in that sentence (potentially among other things).
Knowing the syntax of a sentence means being sensitive to the \textit{order of the words} in that sentence (among other things).  Humans are sensitive to word order, so clearly, ``language is not merely a bag of words'' \citep[p.156]{harris-1954-distributional}.
Moreover, it is easier for us to identify or recall words presented in canonical orders than in disordered, ungrammatical sentences; this phenomenon is called the \textit{``sentence superiority effect''} (\citealt{cattell-1886-time, scheerer1981early, toyota-2001-changes, baddeley-etal-2009-working, snell-grainger-2017-sentence, snell2019word, wen-etal-2019-parallel}, i.a.).   %This effect, i.e.\ the ``sentence superiority effect'', also finds some neurobiological support from work showing ordered text activates portions of the temporal lobe more than unordered word lists \newcite{bemis-pylkkanen-2013-basic, pylkkanen-etal-2014-building}.
In our estimation then, if one wants to claim that  a model ``knows syntax'', then they should minimally show that the model is sensitive to word order (at least for e.g. English or Mandarin Chinese).

Generally, knowing the syntax of a sentence is taken to be a prerequisite for understanding what that sentence means \citep{heim-kratzer-1998-semantics}.
Models should have to know the syntax first then, if performing any particular NLU task that genuinely requires a humanlike understanding of meaning (cf. \citealt{bender-koller-2020-climbing}). %\emph{and} a humanlike understanding sentence meaning requires a knowledge of syntax, \emph{and} a knowledge of syntax requires (at least) 
Thus, if our models are as good at NLU as our current evaluation methods suggest, we should expect them to be sensitive to word order (see \autoref{tab:example}).  We find, based on a suite of permutation metrics, that they are not.

%Synthesizing these findings with the perspective from formal linguistics, we can note that if we want our models to perform an NLU task in a humanlike way, it is not enough to perform well on grammatical sentences, they must also perform poorly on ungrammatical ones.

%Given the ``sentence superiority effect", if NLI models are reasoning through sentences in a humanlike way, they should be unsure of how to classify an ungrammatical, syntax corrupted premise-hypothesis pair . 
 % for which no word is present in its original position , and the relative word ordering is minimized .
We focus here on textual entailment, one of the hallmark tasks used to measure how well models understand language \citep{condoravdi-etal-2003-entailment, dagan-etal-2005-pascal}. This task, often also called Natural Language Inference (NLI; \citealt{bowman-etal-2015-large}, i.a.), typically consists of two sentences: a premise and a hypothesis. The objective is to predict whether the premise entails the hypothesis, contradicts it, or is neutral with respect to it.  We find rampant word order insensitivity in purportedly high performing NLI models. %, somewhat surprisingly, 
For nearly all premise-hypothesis pairs, \textbf{there are many permuted examples that fool the models} into providing the correct prediction. In case of MNLI, for example, the current state-of-the-art of 90.5\% can be increased to \textbf{98.7}\% merely by permuting the word order of test set examples. We even find drastically increased cross-dataset generalization when we reorder words. This is not just a matter of chance---we show that the model output probabilities are significantly different from uniform. 

We verify our findings with three popular English NLI datasets---SNLI \citep{bowman-etal-2015-large}, MultiNLI \citep{williams-etal-2018-broad} and ANLI \citep{nie-etal-2020-adversarial})---%By performing such iterative sampling, we can find a permuted subset of popular NLI datasets which achieves near perfect performance. 
and one Chinese one, OCNLI \cite{hu-etal-2020-ocnli}. It is thus less likely that our findings result from some quirk of English or a particular tokenization strategy. 
We also observe the effect for various transformer architectures pre-trained on language modeling (BERT, RoBERTa, DistilBERT), and non-transformers, including a ConvNet, an InferSent model, and a BiLSTM.
% We further expand our investigation to capture a probabilistic statistics on the acceptance of permutations for each example. We investigate the root cause of the effect, and find high model confidence of over-parameterized models to be one such factor. We also find n-gram overlap of permuted sentences is correlated with model performance on permuted sentences, while the sentences with the least n-gram overlap also reflects high acceptability. We devise a novel Part-of-Speech Minitree hypothesis to further explain the scenario of randomized sentence acceptability, and find our hypothesis correlates with model performance as well. Finally, we devise a simple alternative training regime based on maximizing the entropy of the model on randomized sentences, and find it being effective at reducing the acceptability probability of the language understanding models considerably, without hurting the original model performance.

Our contributions are as follows: 
(i) we propose a suite of metrics (\textit{\PermAcc}) for measuring model insensitivity to word order (\autoref{sec:permutation}), 
(ii) we construct multiple permuted test datasets for measuring NLI model performance at a large scale (\autoref{sec:eval}),
% Since we are moving this section to appendix, should we mention it here?
(iii) we show that NLI models focus on words more than word order, but can partially reconstruct syntactic information from words alone (\autoref{sec:pos_mini_tree}), %, as we explore with metrics for word overlap and Part-of-Speech overlap, and
(iv) we show the problem persists on out-of-domain data,
(v) we show that humans struggle with UnNatural Language Inference, underscoring the non-humanlikeness of SOTA models (\autoref{sec:human_eval}),
(vi) finally, we explore a simple maximum entropy-based method (\autoref{sec:training}) to encourage models not to accept permuted examples.



% \begin{itemize}
%     \item Original Premise: ``But I'll take up my stand somewhere near, and when he comes out of the building I'll drop a handkerchief or something, and off you go!"
%     \item Original hypothesis: ``I want you to follow him, so watch for the signal that I give."
%     \item Sampled premise: ``when 'll my drop I the take But handkerchief , out up and building go somewhere near and he comes of or I off a , something you stand 'll ! ",
%     \item Sampled hypothesis: ``so I that give the to , signal you watch follow him for want I ."
% \end{itemize}

\section{Related Work}

Researchers in NLP have realized the importance of syntactic structure in neural networks going back to \newcite{tabor-1994-syntactic}. An early hand annotation effort on PASCAL RTE \citep{dagan-etal-2006} suggested that ``syntactic information alone was sufficient to make a judgment'' for roughly one third of examples %, whereas almost a half could be solved if annotators were additionally provided a thesaurus
\citep{vanderwende2005syntax}. %  
Anecdotally, large generative language models like GPT-2 or -3 exhibit a seemingly humanlike ability to generate fluent and grammatical text \citep{goldberg2019assessing, wolf2019some}. 
However, the jury is still out as to whether transformers genuinely acquire syntax.

\paragraph{Models appear to have acquired syntax.} When researchers have peeked inside Transformer LM's pretrained representations, familiar syntactic structure \citep{hewitt-manning-2019-structural,jawahar-etal-2019-bert, lin-etal-2019-open, warstadt-bowman-2020-can, wu-etal-2020-perturbed}, or a familiar order of linguistic operations \citep{jawahar-etal-2019-bert,  tenney-etal-2019-bert}, has appeared. There is also evidence, notably from agreement attraction phenomena \citep{linzen-etal-2016-assessing} that transformer-based models pretrained on LM do acquire some knowledge of natural language syntax \citep{gulordava-etal-2018-colorless, chrupala-alishahi-2019-correlating,  jawahar-etal-2019-bert, lin-etal-2019-open, manning-etal-2020-emergent,  hawkins-etal-2020-investigating, linzen-baroni-2021-syntactic}. Results from other phenomena \citep{warstadt-bowman-2020-can} such as NPI licensing  \citep{warstadt-etal-2019-investigating} lend additional support. The claim that LMs acquire some syntactic knowledge has been made not only for transformers, but also for convolutional neural nets \citep{bernardy-lappin-2017-using}, and RNNs \citep{gulordava-etal-2018-colorless, van-schijndel-linzen-2018-neural, wilcox-etal-2018-rnn, zhang-bowman-2018-language, prasad-etal-2019-using, ravfogel-etal-2019-studying}---although there are many caveats (e.g., \citealt{ravfogel-etal-2018-lstm, white-etal-2018-lexicosyntactic,  davis-van-schijndel-2020-recurrent, chaves-2020-dont, da-costa-chaves-2020-assessing, kodner-gupta-2020-overestimation}). 

\paragraph{Models appear to struggle with syntax.} Several works have cast doubt on the extent to which NLI models in particular know syntax (although each work adopts a slightly different idea of what ``knowing syntax'' entails). For example, \newcite{mccoy-etal-2019-right} argued that the knowledge acquired by models trained on NLI (for at least some popular datasets) is actually not as syntactically sophisticated as it might have initially seemed; some transformer models rely mainly on simpler, non-humanlike heuristics. In general, transformer LM performance has been found to be patchy and variable across linguistic phenomena \citep{dasgupta-etal-2018-evaluating, naik-etal-2018-stress, an-etal-2019-representation, ravichander-etal-2019-equate, jeretic-etal-2020-natural}. This is especially true for syntactic phenomena \citep{marvin-linzen-2018-targeted, hu-etal-2020-systematic, gauthier-etal-2020-syntaxgym, mccoy-etal-2020-berts, warstadt-etal-2020-blimp}, where transformers are, for some phenomena and settings, worse than RNNs \citep{van-schijndel-etal-2019-quantity}. From another angle, many have explored architectural approaches for increasing a network's sensitivity to syntactic structure \citep{chen-etal-2017-enhanced, Li-etal-2020-SANLI}. \newcite{williams-etal-2018-latent} showed that learning jointly to perform NLI  and to parse resulted in parse trees that match no popular syntactic formalisms. Furthermore, models trained explicitly to differentiate acceptable sentences from unacceptable ones (i.e., one of the most common syntactic tests used by linguists) have, to date, come nowhere near human performance \citep{warstadt-etal-2019-neural}.

\paragraph{Insensitivity to Perturbation.} Most relatedly, several concurrent works \citep{pham-etal-2020-out, alleman2021syntactic, gupta-etal-2021-bert, sinha2021masked,parthasarathi2021sometimes} investigated the effect of word order permutations on transformer NNs.  \newcite{pham-etal-2020-out} is very nearly a proper subset of our work except for investigating additional tasks (i.e. from the GLUE benchmark of \citealt{wang-etal-2018-glue}) and performing a by-layer-analysis. \newcite{gupta-etal-2021-bert} also relies on the GLUE benchmark, but additionally investigates other types of ``destructive'' perturbations. Our contribution differs from these works %, although our tentative Maximum Entropy solution parallels a similar one in \newcite{} 
in that we additionally include the following: we (i) outline theoretically-informed predictions for how models \emph{should be expected} to react to permuted input (we outline a few options), (ii) show that permuting can ``flip'' an incorrect prediction to a correct one, (iii) show that the problem isn't specific to Transformers, (iv) show that the problem persists on out of domain data, (v) offer a suite of flexible metrics, and (vi) analyze \emph{why} models might be accepting permutations (BLEU and POS-tag neighborhood analysis). Finally, we replicate our findings in another language.
% cite our two permutations papers & Yoon's student's in relation to our translation paper.
While our work (and \citeauthor{pham-etal-2020-out,gupta-etal-2021-bert}) only permutes data during fine-tuning and/or evaluation, 
%raising the natural question: what is the result of permuting during training. \newcite{sinha2021masked} addresses this question. 
recently \citeauthor{sinha2021masked} explored the sensitivity during pre-training, and found that models trained on n-gram permuted sentences perform remarkably close to regular MLM pre-training.
In the context of generation, \newcite{parthasarathi2021sometimes} crafted linguistically relevant perturbations (on the basis of part-of-speech tagging and dependency parsing) to evaluate whether permutation hinders automatic machine translation models. Relatedly, but not for translation, \newcite{alleman2021syntactic} investigated a smaller inventory of perturbations with emphasis on phrasal boundaries and the effects of n-gram perturbations on different layers in the network. % AW: Koustuv, please check this

\paragraph{NLI Models are very sensitive to words.} NLI models often over-attend to particular words to predict the correct answer \citep{gururangan-etal-2018-annotation, clark-etal-2019-bert}. \newcite{wallace-etal-2019-universal} show that some short sequences of non-human-readable text can fool many NLU models, including NLI models trained on SNLI, into predicting a specific label. In fact, \newcite{ettinger-2020-whatbertisnot} observed that for one of three test sets, BERT loses some accuracy in word-perturbed sentences, but that there exists a subset of examples for which BERT’s accuracy remains intact. 
%This led \citeauthor{ettinger-2020-whatbertisnot} to speculate that ``some of BERT's success on these items may be attributable to simpler lexical or n-gram information''. Thus, it is reasonable to wonder how well NLI models will perform on permuted sentences where the model is able to view the same collection of words. 
If performance isn't affected (or if permutation helps, as we find it does in some cases), it suggests that these state-of-the-art models actually perform somewhat similarly to bag-of-words models \cite{blei-etal-2003-latent, mikolov2013efficient}.   

% Todo: cite dasgupta et al
% Todo: cite goodwin et al


\section{Our Approach}\label{sec:permutation}

As we mentioned, linguists generally take syntactic structure to be necessary for humans to know what sentences mean. Many also find the NLI task to a very promising approximation of human natural language understanding, in part because it is rooted in the tradition of logical entailment. In the spirit of propositional logic, sentence meaning is taken to be %the spirit of many propositional logics with
truth-conditional \citep{frege1948sense, montague-1970-universal, chierchia-mcconnell-1990-meaning, heim-kratzer-1998-semantics}. That is to say that understanding a sentence is equivalent to knowing the actual conditions of the world under which the sentences would be (judged) true \citep{wittgenstein-1922-tractatus}. If grammatical sentences are required for sentential inference, as per a truth conditional approach \citep{montague-1970-universal}, then permuted sentences should be meaningless. Put another way, the meanings of highly permuted sentences (if they exist) are not propositions, and thus those sentences don't have truth conditions. Only from their truth conditions of sentences can we tell if a sentence entails another. 
In short, the textual entailment task is technically undefined in our ``unnatural'' setting. %Given our working definitions of textual entailment from the NLI task, it's not reasonable to say, for example, whether ``The the yesterday bit dog hard man'' is entailed by ``The man bit the dog hard yesterday'' or not. 


Since existing definitions don't immediately extend to UnNatural Language Inference (UNLI), we outline several hypothetical \textit{systematic} ways that a model might perform, had it been sensitive to word order. We hypothesize two models that operate on the first principles of NLI, and one that doesn't. In the first case, Model A deems permuted sentences meaningless (devoid of truth values), as formal semantic theories of human language would predict. Thus, it assigns ``neutral" to every permuted example. Next, Model B does not deem permuted sentences meaningless, and attempts to understand them. Humans find understanding permuted sentences difficult (see our human evaluations in \autoref{sec:human_eval}). Model B could also similarly struggle to decipher the meaning, and just equally sample labels for each example (i.e., assigns equal probability mass to the outcome of each label). Finally, we hypothesize a non-systematic model, Model C, which attempts to treat permuted sentences as though they weren't permuted at all. This model could operate similarly as bag-of-words (BOW), and thus always assign the same label to the permuted examples as it would to the un-permuted examples. If the model failed to assign the original gold label to the original unpermuted examples, it will also fail to assign the original gold label to its permutations; it will never get higher accuracy on permuted examples than on unpermuted ones. 

%In the first case, Model A assigns ``neutral" to every permuted ungrammatical example. Since Model A deems the sentences themselves meaningless, it decides the relationship between them is irrelevant or unknowable. (This is not what we observe.) %Second, we have Model B, which assigns any label except ``entailment'' to permuted examples. Model B knows that examples that permuted examples don't have meanings and thus the relationship between them can't be ``entailment''. Thus Model B behaves somewhat like an NLI model performing 2-class textual entailment (like the original RTE models, \citealt{dagan-etal-2005-pascal}). No investigated models behave like Model B.
%Next, 
%we can have a Model C which could present somewhat of an extreme case: Model C never assigns the original example's gold label to any permutation of that example, as it attempts to interpret the meaning of the permuted sentences and falls short of predicting the correct label in each trial. If one of the original example was ``contradiction'', then Model C would assign all the sampled permutations of that example (either ``entailment'' or ``neutral''). We would say then that Model C does not \textbf{accept} any permutations of any example (we don't observe this behavior). Finally, 
%Model B equally samples labels for each example, by assigning equal probability mass to the outcome of each label. Recall the sentence superiority findings which are compatible with humans accepting very few permutations. We show that humans indeed struggle with unnatural language inference data in \autoref{sec:human_eval}. (In fact, humans perform most like the hypothetical Model B.) Model B is systematic and it attempts to understand the meaning of the permuted sentences, but in the end, it just makes a random guess. (We do not observe this behaviour from any investigated models.) 
%Finally, imagine a non-systematic model that doesn't perform at all like Models A or B. This model, Model C operates similarly to a bag-of-words. Since Model C has learned merely to associate labels to a bag-of-words (BOW), it always assigns the same label to permuted examples as it would to the un-permuted examples. If the model failed to assign the original gold label to the original unpermuted examples, it will also fail to assign the original gold label to its permutations; it will never get higher accuracy on permuted examples than on unpermuted ones. 

We find in our experiments that the state-of-the-art Transformer-based NLI models (as well as pre-Transformer class of models) do not perform like any of the above hypothetical models. They perform closest to Model C, but are, in some cases, actually able to achieve \emph{higher} accuracy on permuted examples. To better quantitatively describe this behaviour,
% All investigated models perform like Model D; they accept permutations at high rates, which means they don't really care about word order and don't behave like humans do. % 
%For example, if we permute an example 100 times, and only one of the instances the permuted sentence pairs are assigned same original label by the model, then $PA$ for that example will be 0.01; $PA$ for examples can be averaged across all examples in dataset to arrive at a dataset level score ($\Omega_x$). 
we introduce our suite of \textbf{\PermAcc} metrics that enable us to quantify how accepting models are of permuted sentences. 

\section{Methods}\label{sec:constructpermut}


\paragraph{Constructing the permuted dataset.} For a given dataset $D$ having splits $D_{\text{train}}$ and $D_{\text{test}}$, we first train an NLI model $M$ on $D_{\text{train}}$ to achieve comparable accuracy to what was reported in the original papers. We then construct a randomized version of $D_{\text{test}}$, which we term as $\hat{D}_{\text{test}}$ such that: for each example $(p_i,h_i,y_i) \in D_{\text{test}}$ (where $p_i$ and $h_i$ are the premise and hypothesis sentences of the example respectively and $y_i$ is the gold label), we use a permutation operator $\mathcal{F}$ that returns a list ($\hat{P}_i, \hat{H}_i$) of $q$ permuted sentences ($\hat{p}_i$ and $\hat{h}_i$), where $q$ is a hyperparameter. $\mathcal{F}$ essentially permutes all positions of the words in a given sentence (i.e., either in premise or hypothesis) with the restriction that \textit{no words maintain their original position}.  In our initial setting, we do not explicitly control the placement of the words relative to their original neighbors, but we analyze clumping effects in \autoref{sec:eval}.
$\hat{D}_{\text{test}}$ now consists of $|D_{\text{test}}| \times q$ examples, with $q$ different permutations of hypothesis and premise for each original test example pair. If a sentence $S$ (e.g., $h_i$) contains $w$ words, then the total number of available permutations of $S$ are $(w-1)!$, thus making the output of $\mathcal{F}$ a list of $(w-1)! \choose q$ permutations in this case. For us, the space of possible outputs is larger, since we permute $p_i$ and $h_i$ separately (and ignore examples for which any $|S|\leq5$).

\paragraph{Defining \PermAcc.}
The choice of $q$ naturally allows us to analyze a statistical view of the predictability of a model on the permuted sentences. To that end, we define the following notational conventions. Let $\mathcal{A}$ be the original accuracy of a given model $M$ on a dataset $D$, and \textit{c} be the number of examples in a dataset which are marked as correct according to the standard formulation of accuracy for the original dataset (i.e., they are assigned the ground truth label). Typically $\mathcal{A}$ is given by $\frac{c}{|D_{test}|}$ or $\frac{c}{|D_{dev}|}$. %, where $c$ is the number of examples assigned the ground truth label. 

\begin{figure}[t]
    \centering
    \resizebox{0.5\textwidth}{!}{
        \includegraphics{images/nli_gen_perm_desc.pdf}}
    \caption{Graphical representation of the \PermAcc\ class of metrics. Given a sample test set ${D}_{\text{test}}$ with six examples, three of which originally predicted correctly (model predicts gold label), three incorrectly (model fails to predict gold label), with $n=6$ permutations, $\Omega_{\text{max}}$,$\Omega_{\text{rand}}$, $\Omega_{\text{1.0}}$, $\mathcal{P}^c$ and $\mathcal{P}^f$ are provided. Green boxes indicate permutations accepted by the model. Blue boxes mark examples that crossed each threshold and were used to compute the corresponding metric. }
    \label{fig:def_metrics}
\end{figure}

Let $\Pr_{M}(\hat{P}_i, \hat{H}_i)_{\text{cor}}$ then be the percentage of $q$ permutations of an example $(p_i, h_i)$ assigned the ground truth label $y_i$ by $M$: 
   % \begin{equation}
    %\begin{split}
    \begin{align}
    &&\Pr_{M}(\hat{P}_i, \hat{H}_i)_{\text{cor}} =  
    \frac{1}{q}\sum_{\mathclap{(\hat{p}_j \in \hat{P}_i, \hat{h}_j \in \hat{H}_i)}}((M(\hat{p}_j, \hat{h}_j) = y_i) \rightarrow 1)
    \end{align}
    %\end{split}
    %\end{equation}
To get an overall summary score, we let $\Omega_x$ be the percentage of examples $(p_i, h_i) \in D_{\text{test}}$ for which $\Pr_{M}(\hat{P}_i, \hat{H}_i)_{\text{cor}}$ exceeds a predetermined threshold $0 < x < 1$. %Concretely, $\Omega_x$ is defined as the percentage of examples in $D_{\text{test}}$, for which the probability of model accepting the permutation as the correct answer is greater than $x$ percent.
Concretely, a given example will count as correct according to $\Omega_x$ if more than $x$ percent of its permutations ($\hat{P}_i$ and $\hat{H}_i$) are assigned $y_i$ by the model $M$.
Mathematically, 
    \begin{align} %I don't love this, but it fits...\mathrlap right centers the subscript, if it's centered it overlaps the D_test
    %\begin{split}
    &&\Omega_x = \frac{1}{\mid D_{test}\mid} \sum_{\mathrlap{(p_i, h_i)\in D_{test}}} ((\Pr_{M}(\hat{P}_i, \hat{H}_i)_{\text{cor}} 
    > x) \rightarrow 1).
    %\end{split}
    \end{align}
There are two specific cases of $\Omega_{\text{x}}$ that we are most interested in. First, we define $\Omega_{\text{max}}$ or the \textbf{Maximum Accuracy}, where $x = 1 / |D_{\text{test}}|$. In short, $\Omega_{\text{max}}$ gives the percentage of examples $(p_i, h_i) \in D_{\text{test}}$ for which there is \textit{at least one} permutation $(\hat{p_j}, \hat{h_j})$ that model $M$ assigns the gold label $y_i$
% new: added comment on omega_max tending towards 1
\footnote{Theoretically, $\Omega_{\text{max}} \rightarrow 1$ if the number of permutations $q$ is large. Thus, in our experiments we set $q=100$.}.
Second, we define $\Omega_{\text{rand}}$, or \textbf{Random Baseline Accuracy}, where $x = 1 / m$ or chance probability (for balanced $m$-way classification, where $m=3$ in NLI). This metric is less stringent than $\Omega_{\text{max}}$, as it counts an example if at least \textit{one third} of its permutations are assigned the gold label (hence provides a lower-bound relaxation). See \autoref{fig:def_metrics} for a graphical representation of $\Omega_{\text{x}}$.

We also define %\textit{Flip}
$D^{f}$ to be the list of examples originally marked incorrect according to $\mathcal{A}$, but are now deemed correct according $\Omega_{\text{max}}$. $D^{c}$ is the list of examples originally marked correct according to  $\mathcal{A}$. Thus, we should expect $D^{f}<D^{c}$ for models that have high accuracy. 
Additionally, we define $\mathcal{P}^c$ and $\mathcal{P}^f$, as the dataset average percentage of permutations which predicted the gold label, when the examples were originally correct ($D^{c}$) and when the examples were originally incorrect ($D^{f}$) as per $\mathcal{A}$ (hence, flipped) respectively.
\begin{equation}
\begin{split}
    &\mathcal{P}^{c} = \frac{1}{|D^{c}|} \sum_{i=0}^{|D^{c}|} M(\hat{P}_i, \hat{H}_i)_{\text{cor}}\\
    % &\mathcal{P}^f = \frac{1}{|D^{f}|} \sum_{i=0}^{|D^f|} M(\hat{P}_i, \hat{H}_i)_{\text{cor}}
\end{split}
\end{equation}

\noindent $P^f$ is defined similarly by replacing $D^c$ by $D^f$. Note that for a classic BOW model,  $\mathcal{P}^c=100$ and $\mathcal{P}^f=0$, because it would rely on the words alone (not their order) to make its classification decision. Since permuting removes no words, BOW models should come to the same decisions for permuted examples as for the originals.

%For example, consider 20 examples from MNLI, for which $15$ were originally correct (i.e., $D^{c}=15$ and $5$ were not ($D^{f}=5$). For this example, $P^{c} = (1/15) \sum_p_i$ and $P^f = (1/5) \sum_p_i$

% Since the permutation function $\mathcal{F}$ results in ungrammatical, nonsensical sentences (\autoref{tab:example}), we have two hypotheses given our discussion so far: \textbf{(a)} permutations resulting from permutation operator $\mathcal{F}$ will elicit random outcomes, with $\mathcal{P}^c$ and $\mathcal{P}^f$ being in random uniform probability of $1/n$, \textbf{(b)} permutations from $\mathcal{F}$ will receive incorrect predictions, resulting in $\mathcal{P}^c = 0$ and $\mathcal{P}^f = 0$. %A human-like language learner should behave in a way closer to for scenario \textbf{(b)}, as defined by ``sentence superiority" hypothesis.
% However, we neither observe \textbf{(a)} nor \textbf{(b)} with the state-of-the-art models. %, which we will discuss in \autoref{sec:eval}. %We evaluate the permutations on their ungrammaticality and nonsensical-ness by performing Human Evaluation in \autoref{sec:human_eval}.

%For our initial experiment, we chose various state-of-the-art models trained on MNLI dataset, Roberta-large, BART-large and Distilbert. Here, Distilbert is the smallest model in terms of capacity. Original predictions of MNLI (Matched) are thus 90.5, 90.1 and 80.0\% on these models. Then, for each example in our test set, we permute the words in the sentences such that:

% \begin{itemize}
%     \item No words are in their original position. If a sentence $S$ contains $w$ words, then the total number of available permutations are $(w-1)!$.
%     % \item Minimize relative word positions, such that the probability of words to occur beside their old neighbors are minimized.
% \end{itemize}

% Self Note: maybe punctuation exception is just an ablation study?

%The reason we kept punctuation in their original location is due to recent findings which suggest Transformer based models use punctuation as aggregation mechanism, along with the special tokens [CLS] and [SEP]. We then compute 100 different permuations for each example (with 100 different seeds), and end up with a test dataset which is 100 times larger than the original test set. We compute the statistics and present the results in Table \ref{table:main}.

% TODO: placing this table will be done once all text changes are over
\begin{table}[htbp]
  \centering
  \resizebox{\linewidth}{!}{%
    \begin{tabular}{@{}llrrrrr@{}}
\toprule
           Model &    Eval. Dataset  &  $\mathcal{A}$ &  $\Omega_{\text{max}}$ &  $\mathcal{P}^c$ &  $\mathcal{P}^f$ &  $\Omega_{\text{rand}}$ \\
\midrule
\multirow{7}{*}{\bf RoBERTa-Large} 
 &   MNLI\_m\_dev &              0.906 &         0.987 &                  0.707 &             0.383 &                        0.794 \\
 &  MNLI\_mm\_dev &              0.901 &         0.987 &                  0.707 &             0.387 &                        0.790 \\
 &     SNLI\_dev &              0.879 &         0.988 &                  0.768 &             0.393 &                        0.826 \\
 &    SNLI\_test &              0.883 &         0.988 &                  0.760 &             0.407 &                        0.828 \\
 &  A1* &              0.456 &         0.897 &                  0.392 &             0.286 &                        0.364 \\
 &  A2* &              0.271 &         0.889 &                  0.465 &             0.292 &                        0.359 \\
 &  A3* &              0.268 &         0.902 &                  0.480 &             0.308 &                        0.397 \\ \midrule
 & Mean & 0.652 & 0.948 & 0.611 & \boldred{0.351} & 0.623 \\ 
% & Harmonic Mean & 0.497 & 0.946 & 0.572 & 0.344 & 0.539 \\ 
\midrule
 
\multirow{7}{*}{\bf BART-Large} 
    &   MNLI\_m\_dev &              0.902 &         0.989 &                  0.689 &             0.393 &                        0.784 \\
    &  MNLI\_mm\_dev &              0.900 &         0.986 &                  0.695 &             0.399 &                        0.788 \\
    &     SNLI\_dev &              0.886 &         0.991 &                  0.762 &             0.363 &                        0.834 \\
    &    SNLI\_test &              0.888 &         0.990 &                  0.762 &             0.370 &                        0.836 \\
    &  A1* &              0.455 &         0.894 &                  0.379 &             0.295 &                        0.374 \\
    &  A2* &              0.316 &         0.887 &                  0.428 &             0.303 &                        0.397 \\
    &  A3* &              0.327 &         0.931 &                  0.428 &             0.333 &                        0.424 \\ \midrule
& Mean &  \textbf{0.668} & \boldred{0.953} & 0.592 & \boldred{0.351} & \boldred{0.634} \\
%& Harmonic Mean &  \textbf{0.543} & \boldred{0.951} & 0.546 & \boldred{0.347} & \boldred{0.561} \\ 
\midrule
\multirow{7}{*}{\bf DistilBERT}  &   MNLI\_m\_dev &              0.800 &         0.968 &                  0.775 &             0.343 &                        0.779 \\
      &  MNLI\_mm\_dev &              0.811 &         0.968 &                  0.775 &             0.346 &                        0.786 \\
      &     SNLI\_dev &              0.732 &         0.956 &                  0.767 &             0.307 &                        0.731 \\
      &    SNLI\_test &              0.738 &         0.950 &                  0.770 &             0.312 &                        0.725 \\
      &  A1* &              0.251 &         0.750 &                  0.511 &             0.267 &                        0.300 \\
      &  A2* &              0.300 &         0.760 &                  0.619 &             0.265 &                        0.343 \\
      &  A3* &              0.312 &         0.830 &                  0.559 &             0.259 &                        0.363 \\ \midrule
      & Mean &  0.564 & 0.883 & \boldred{0.682} & 0.300 & 0.575 \\ 
%& Harmonic Mean &  0.445 & 0.873 & \boldred{0.664} & 0.296 & 0.490 \\
% \bottomrule
\midrule\midrule
 \multirow{7}{*}{\bf InferSent} 
 &   MNLI\_m\_dev &              0.658 &         0.904 &                  0.842 &             0.359 &                        0.712 \\
 &  MNLI\_mm\_dev &              0.669 &         0.905 &                  0.844 &             0.368 &                        0.723 \\
 &     SNLI\_dev &              0.556 &         0.820 &                  0.821 &             0.323 &                        0.587 \\
 &    SNLI\_test &              0.560 &         0.826 &                  0.824 &             0.321 &                        0.600 \\
 &  A1* &              0.316 &         0.669 &                  0.425 &             0.395 &                        0.313 \\
 &  A2* &              0.310 &         0.662 &                  0.689 &             0.249 &                        0.330 \\
 &  A3* &              0.300 &         0.677 &                  0.675 &             0.236 &                        0.332 \\ \midrule
 & Mean &  \textbf{0.481} & 0.780 & 0.731 & \boldred{0.322} & 0.514 \\ 
 %& Harmonic Mean &  0.429 & 0.767 & 0.694 & \boldred{0.311} & 0.455 \\ 
 \midrule
 \multirow{7}{*}{\bf ConvNet}
 &   MNLI\_m\_dev &              0.631 &         0.926 &                  0.773 &             0.340 &                        0.684 \\
 &  MNLI\_mm\_dev &              0.640 &         0.926 &                  0.782 &             0.343 &                        0.694 \\
 &     SNLI\_dev &              0.506 &         0.819 &                  0.813 &             0.339 &                        0.597 \\
 &    SNLI\_test &              0.501 &         0.821 &                  0.809 &             0.341 &                        0.596 \\
 &  A1* &              0.271 &         0.708 &                  0.648 &             0.218 &                        0.316 \\
 &  A2* &              0.307 &         0.725 &                  0.703 &             0.224 &                        0.356 \\
 &  A3* &              0.306 &         0.798 &                  0.688 &             0.234 &                        0.388 \\ \midrule
 & Mean &  0.452 & \boldred{0.817} & \boldred{0.745} & 0.291 & 0.519 \\ 
%& Harmonic Mean &  0.404 & \boldred{0.810} & \boldred{0.740} & 0.279 & \boldred{0.473} \\ 
\midrule
 \multirow{7}{*}{\bf BiLSTM} 
 &   MNLI\_m\_dev &              0.662 &         0.925 &                  0.800 &             0.351 &                        0.711 \\
 &  MNLI\_mm\_dev &              0.681 &         0.924 &                  0.809 &             0.344 &                        0.724 \\
 &     SNLI\_dev &              0.547 &         0.860 &                  0.762 &             0.351 &                        0.598 \\
 &    SNLI\_test &              0.552 &         0.862 &                  0.771 &             0.363 &                        0.607 \\
 &  A1* &              0.262 &         0.671 &                  0.648 &             0.271 &                        0.340 \\
 &  A2* &              0.297 &         0.728 &                  0.672 &             0.209 &                        0.328 \\
 &  A3* &              0.304 &         0.731 &                  0.656 &             0.219 &                        0.331 \\ \midrule
 & Mean &  0.472 & 0.814 & 0.731 & 0.301 & \boldred{0.520} \\
%& Harmonic Mean &  0.410 & 0.803 & 0.725 & 0.287 & 0.463 \\
 
\bottomrule
\end{tabular}}
  \caption{Statistics for Transformer-based models trained on MNLI corpus \cite{williams-etal-2018-broad}. 
  %$\Omega_{\text{max}}$ or Max Accuracy is computed if \textit{any} of the $n=100$ permutations per data point yield correct results. The mean number of permutations which were correct, when the original prediction is correct or incorrect  are given by $\mathcal{P}^c$  and $\mathcal{P}^f$ (flipped) respectively. $\Omega_{\text{rand}}$ is the percentage of data points for which models choose the ground truth label over a random uniform baseline (1/3). 
  The highest values are bolded (\boldred{red} indicates the model most insensitive to permutation) per metric and per model class (Transformers and non-Transformers). A1*, A2* and A3* refer to the ANLI dev. sets \citep{nie-etal-2020-adversarial}.}
  \label{table:main}
\end{table}
\begin{table}[htbp]
    \centering
    \footnotesize
    \resizebox{\linewidth}{!}{%
        \begin{tabular}{llrrrrrr}
            \toprule
             Model              & $\mathcal{A}$ & $\Omega_{\text{max}}$ & $\mathcal{P}^c$ & $\mathcal{P}^f$ & $\Omega_{\text{rand}}$ \\ \midrule
             RoBERTa-Large   & \textbf{0.784} &         \boldred{0.988} &                  0.726 &             \boldred{0.339} &                        \boldred{0.773}            \\
             InferSent &             0.573 &         0.931 &                  0.771 &             0.265 &                        0.615 \\
   ConvNet &              0.407 &         0.752 &                  \boldred{0.808} &             0.199 &                        0.426 \\
    BiLSTM &              0.566 &         0.963 &                  0.701 &             0.271 &                        0.611 \\
            \bottomrule
        \end{tabular}}
    \caption{Results on evaluation on OCNLI Dev set. All models are trained on OCNLI corpus \cite{hu-etal-2020-ocnli}. 
    % Max accuracy ($\Omega_{\text{max}}$) is computed based on whether \textit{any} of the $n=100$ permutations per data point yield correct results. $\mathcal{P}^c$  stands for the mean number of permutations which were correct when the original prediction is correct. $\mathcal{P}^f$  stats for the mean number of permutations which are correct when the original prediction is incorrect (flip). 
    Bold marks the highest value per metric (\boldred{red} shows the model is insensitive to permutation).}
    \label{table:ocnli_all}
\end{table}


\section{Results}\label{sec:eval} %Evaluation on Permuted Test Sets

% \begin{table}[ht]
    \centering
    \resizebox{\linewidth}{!}{%
        \begin{tabular}{@{}llrrrrr@{}}
            \toprule
     Model &    Eval Dataset &  $\mathcal{A}$ &  $\Omega_{\text{max}}$ &  $\mathcal{P}^c$ &  $\mathcal{P}^f$ &  $\Omega_{\text{rand}}$ \\
\midrule
 \multirow{7}{*}{\bf InferSent} 
 &   MNLI\_m\_dev &              0.658 &         0.904 &                  0.842 &             0.359 &                        0.712 \\
 &  MNLI\_mm\_dev &              0.669 &         0.905 &                  0.844 &             0.368 &                        0.723 \\
 &     SNLI\_dev &              0.556 &         0.820 &                  0.821 &             0.323 &                        0.587 \\
 &    SNLI\_test &              0.560 &         0.826 &                  0.824 &             0.321 &                        0.600 \\
 &  A1\_dev &              0.316 &         0.669 &                  0.425 &             0.395 &                        0.313 \\
 &  A2\_dev &              0.310 &         0.662 &                  0.689 &             0.249 &                        0.330 \\
 &  A3\_dev &              0.300 &         0.677 &                  0.675 &             0.236 &                        0.332 \\ \midrule
 & Mean &  \textbf{0.481} & 0.780 & 0.731 & \boldred{0.322} & 0.514 \\ 
 & Harmonic Mean &  0.429 & 0.767 & 0.694 & \boldred{0.311} & 0.455 \\ \midrule
 \multirow{7}{*}{\bf ConvNet}
 &   MNLI\_m\_dev &              0.631 &         0.926 &                  0.773 &             0.340 &                        0.684 \\
 &  MNLI\_mm\_dev &              0.640 &         0.926 &                  0.782 &             0.343 &                        0.694 \\
 &     SNLI\_dev &              0.506 &         0.819 &                  0.813 &             0.339 &                        0.597 \\
 &    SNLI\_test &              0.501 &         0.821 &                  0.809 &             0.341 &                        0.596 \\
 &  A1\_dev &              0.271 &         0.708 &                  0.648 &             0.218 &                        0.316 \\
 &  A2\_dev &              0.307 &         0.725 &                  0.703 &             0.224 &                        0.356 \\
 &  A3\_dev &              0.306 &         0.798 &                  0.688 &             0.234 &                        0.388 \\ \midrule
 & Mean &  0.452 & \boldred{0.817} & \boldred{0.745} & 0.291 & 0.519 \\ 
& Harmonic Mean &  0.404 & \boldred{0.810} & \boldred{0.740} & 0.279 & \boldred{0.473} \\ \midrule
 \multirow{7}{*}{\bf BiLSTM} 
 &   MNLI\_m\_dev &              0.662 &         0.925 &                  0.800 &             0.351 &                        0.711 \\
 &  MNLI\_mm\_dev &              0.681 &         0.924 &                  0.809 &             0.344 &                        0.724 \\
 &     SNLI\_dev &              0.547 &         0.860 &                  0.762 &             0.351 &                        0.598 \\
 &    SNLI\_test &              0.552 &         0.862 &                  0.771 &             0.363 &                        0.607 \\
 &  A1\_dev &              0.262 &         0.671 &                  0.648 &             0.271 &                        0.340 \\
 &  A2\_dev &              0.297 &         0.728 &                  0.672 &             0.209 &                        0.328 \\
 &  A3\_dev &              0.304 &         0.731 &                  0.656 &             0.219 &                        0.331 \\ \midrule
 & Mean &  0.472 & 0.814 & 0.731 & 0.301 & \boldred{0.520} \\
& Harmonic Mean &  0.410 & 0.803 & 0.725 & 0.287 & 0.463 \\
 
\bottomrule
        \end{tabular}}
    \caption{Statistics for Non-Transformer Models. All models are trained on MNLI corpus \cite{williams-etal-2018-broad}. $\Omega_{\text{max}}$ or Max accuracy is computed if \textit{any} of the $n=100$ permutations per data point yield correct results. $\mathcal{P}^c$  stands for the mean number of permutations which were correct when the original prediction is correct. $\mathcal{P}^f$  stats for the mean number of permutations which are correct when the original prediction is incorrect (flip). $\Omega_{\text{rand}}$ is computed as the percentage of data points the ground truth label is chosen over a random uniform baseline (1/3). Bold marks the highest value per metric (\boldred{red} shows  the model is insensitive to permutation).}
    \label{table:main_non_t}
\end{table}


We present results for two types of models: \textbf{(a)} Transformer-based models and \textbf{(b)} Non-Transformer Models. In \textbf{(a)}, we investigate the state-of-the-art pre-trained models such as RoBERTa-Large \cite{liu-et-al-2019-roberta}, BART-Large \cite{lewis-etal-2020-bart} and DistilBERT \cite{sanh2020distilbert}. For \textbf{(b)} we consider several recurrent and convolution based neural networks, such as InferSent \cite{conneau-etal-2017-supervised}, Bidirectional LSTM \cite{collobert2008unified} and ConvNet \cite{zhao2015self}. We train all models on MNLI, and evaluate on in-distribution (SNLI and MNLI) and out-of-distribution datasets (ANLI). We independently verify results of \textbf{(a)} using both our fine-tuned model using HuggingFace Transformers \cite{wolf2020transformers} and pre-trained checkpoints from FairSeq \cite{ott2019fairseq} (using PyTorch Model Hub). For \textbf{(b)}, we use the InferSent codebase. We sample $q=100$ permutations for each example in $D_{\text{test}}$, and use 100 seeds for each of those permutations to ensure full reproducibility. We drop examples from test sets where we are unable to compute \textit{all unique} randomizations, typically these are examples with sentences of length of less than 6 tokens. \footnote{Code, data, and model checkpoints will be available at \href{https://github.com/facebookresearch/unlu}{https://github.com/facebookresearch/unlu}.}


\paragraph{Models accept many permuted examples.}

We find $\Omega_{\text{max}}$ is very high for models trained and evaluated on MNLI (in-domain generalization), reaching \textbf{98.7\%} on MNLI dev. and test sets (in RoBERTa, compared to $\mathcal{A}$ of 90.6\% (\autoref{table:main}). Recall, human accuracy is approximately 92\% on MNLI dev.,  \citealt{nangia-bowman-2019-human}). This shows that there exists at least one permutation (usually many more) for almost all examples in $D_{\text{test}}$ such that model $M$ predicts the gold label. We also observe high $\Omega_{\text{rand}}$ at 79.4\%, showing that there are many examples for which the models outperform even a random baseline in accepting permuted sentences (see \autoref{app_sec:threshold} for more $\Omega$ values.)

Evaluating out-of-domain generalization with ANLI dataset splits resulted in an $\Omega_{\text{max}}$ value that is notably higher than $\mathcal{A}$ (89.7\% $\Omega_{\text{max}}$ for RoBERTa compared to 45.6\% $\mathcal{A}$). As a consequence, we encounter many \textit{flips}, i.e., examples where the model is unable to predict the gold label, but at least one permutation of that example is able to. However, recall this analysis expects us to know the gold label upfront, so this test can be thought of as running a word-order probe test on the model until the model predicts the gold label (or give up by exhausting our set of $q$ permutations). For out-of-domain generalization, $\Omega_{\text{rand}}$ decreases considerably (36.4\% $\Omega_{\text{rand}}$ on A1), which means fewer permutations are accepted by the model. Next, recall that a classic bag-of-words model would have $\mathcal{P}^c=100$ and $\mathcal{P}^f=0$. No model performs strictly like a classic bag of words although they do perform somewhat BOW-like ($\mathcal{P}^c >> \mathcal{P}^f$ for all test splits, \autoref{fig:comb_plot}). %Although it is harder to find the correct permutation for already misclassified, non-generalizable sentences, state-of-the-art models still behave somewhat BOW-like. 
We find this BOW-likeness to be higher for certain non-Transformer models, (InferSent) as they exhibit higher $\mathcal{P}^c$ (84.2\% for InferSent compared to 70.7\% for RoBERTa on MNLI).

% Will it be too small?
\begin{figure}[t]
    \centering
    \resizebox{0.48\textwidth}{!}{
        \includegraphics{images/entropy_plot.png}}
    \caption{Average entropy of model confidences on permutations that yielded the correct results for Transformer-based models (top) and Non-Transformer-based models (bottom). Results are shown for $D^c$ (orange) and $D^f$ (blue). The boxes show the quartiles of the entropy distributions.}
    \label{fig:all_entropy}
\end{figure}

\paragraph{Models are very confident.}

The phenomenon we observe would be of less concern if the correct label prediction was just an outcome of chance, which could occur when the entropy of the log probabilities of the model output is high (suggesting uniform probabilities on entailment, neutral and contradiction labels, recall Model B from \autoref{sec:permutation}). We first investigate the model probabilities for the Transformer-based models on the permutations that lead to the correct answer in \autoref{fig:all_entropy}. We find overwhelming evidence that model confidences on in-distribution datasets (MNLI, SNLI) are highly skewed, resulting in low entropy, and it varies among different model types. BART proves to be the most skewed Transformer-based model. This skewness is not a property of model capacity, as we observe DistilBERT log probabilities to have similar skewness as RoBERTa (large) model, while exhibiting lower $\mathcal{A}$, $\Omega_{\text{max}}$, and $\Omega_{\text{rand}}$. 

% KS: I think this paper has the inverse result!
% The high confidence of Transformer-based model can be attributed to their tendency to be mis-calibrated in an out-of-domain setting, which is opposite to \newcite{desai2020calibration}!!!.

% [CHECK] comment on why Transformers are more confident. Find citations
% Do we know of a paper which shows Transformer entropies are low?
% Use the term calibration
% https://arxiv.org/pdf/2003.07892.pdf
% http://proceedings.mlr.press/v119/braverman20a/braverman20a.pdf

For non-Transformers whose accuracy $\mathcal{A}$ is lower, the $\Omega_{\text{max}}$ achieved by these models are also predictably lower. We observe roughly the same relative performance in the terms of $\Omega_{\text{max}}$ (\autoref{fig:comb_plot} and Appendix \autoref{table:main}) and Average entropy (Figure \ref{fig:all_entropy}). However, while comparing the averaged entropy of the model predictions, it is clear that there is some benefit to being a worse model---non-Transformer models are not as overconfident on randomized sentences as Transformers are. 
High confidence of Transformer models can be attributed to the \textit{overthinking} phenomenon commonly observed in deep neural networks \cite{kaya2019shallowdeep} and BERT-based models \cite{zhou2020bert}.
% KS: I think this is a better reasoning? Please check.


\paragraph{Similar artifacts in Chinese NLU.}

We extended the experiments to the Original Chinese NLI dataset \citep[OCNLI]{hu-etal-2020-ocnli}, and re-used the pre-trained RoBERTa-Large and InferSent (non-Transformer) models on OCNLI. Our findings are similar to the English results (\autoref{table:ocnli_all}), thereby suggesting that the phenomenon is not just an artifact of English text or tokenization.



% \begin{figure}
%     \centering
%     \resizebox{0.47\textwidth}{!}{
%         \includegraphics{images/roberta_large_conf.png}}
%     \caption{Model confidences on permuted test set}
%     \label{fig:roberta_conf}
% \end{figure}

\paragraph{Other Results.} We investigated the effect of sentence length (which correlates with number of possible permutations; \autoref{app_sec:length}), and hypothesis-only randomization (models exhibit similar phenomenon even when only hypothesis is permuted; \autoref{app_sec:HO}).

\begin{figure}[t]
    \centering
    \resizebox{0.48\textwidth}{!}{
        \includegraphics{images/bleu_2_all.png}}
    \caption{BLEU-2 score versus acceptability of permuted sentences across all test datasets. RoBERTa and BART performance is similar but differs considerably from the performance of non-Transformer-based models, such as InferSent and ConvNet. }
    \label{fig:bleu_2}
\end{figure}

\section{Analyzing Syntactic Structure Associated with Tokens}\label{sec:pos_mini_tree}

A natural question to ask following our findings: what is it about particular permutations that leads models to accept them? Since the permutation operation is drastic and only rarely preserves local word relations, we first investigate whether there exists a relationship between \PermAcc\ scores and local word order preservation. Concretely, we compare bi-gram word overlap (BLEU-2) with the percentage of permutations that are deemed correct (\autoref{fig:bleu_2}).\footnote{We observe, due to our permutation process, the maximum BLEU-3 and BLEU-4 scores are negligibly low ($< 0.2$  BLEU-3 and $< 0.1$ BLEU-4), already calling into question the hypothesis that n-grams are the sole explanation for our finding. Because of this, we only compare BLEU-2 scores. Detailed experiments on specially constructed permutations that cover the entire range of BLEU-3 and BLEU-4 is provided in \autoref{app_sec:bleu_all}.} Although the probability of a permuted sentence to be predicted correctly does appear to track BLEU-2 score (Figure \ref{fig:bleu_2}), the percentage of examples which were assigned the gold label by the Transformer-based models is still higher than we would expect from permutations with lower BLEU-2 (66\% for the lowest BLEU-2 range of $0-0.15$), suggesting preserved relative word order alone cannot explain the high permutation acceptance rates.

Thus, we find that local order preservation does correlate with \PermAcc, but it doesn't fully explain the high \PermAcc\ scores.
We now further ask whether $\Omega$ is related to a more abstract measure of local word relations, i.e., part-of-speech (POS) neighborhood.

%We perform two initial analyses to shed light on this question. First, we ask whether \PermAcc\ scores are higher when local word order is preserved? 
%We might expect this to be the case if the models memorize words along with their local neighbors. %does this clarify the point?
%We find that local order preservation does correlate with \PermAcc, but it doesn't fully explain the high \PermAcc\ scores. Second, we ask whether $\Omega$ is related to a more abstract measure of local word relations, i.e., part-of-speech (POS) neighborhood. We find that there is little effect of POS-neighbors for non-Transformer models, but that RoBERTa, BART, and DistilBERT show a distinct effect. Taken together these analyses suggest that some local word order information affects models' \PermAcc\ scores, and perhaps incorporating methods of decreasing model reliance on this information could be fruitful. 

%\paragraph{Preserving Local Word Order Leads to Higher \PermAcc.}

%Although we constrained our randomization such that no word appears in its original position, we didn't impose any restrictions on the relative positions of n-grams. To investigate the extent to which relative n-gram order is preserved, we compare 2-, 3- and 4 n-gram BLEU scores \citep{papineni-etal-2002-bleu} to the $\Omega$ values across models. 
%If preserved n-grams are driving the \PermAcc\ effects, we should low $\Omega$ values for examples with low BLEU and high $\Omega$ for examples with high BLEU. % didn't understand this part
%As a result of our permutation process, the maximum BLEU-3 and BLEU-4 scores are negligibly low ($< 0.2$  BLEU-3 and $< 0.1$ BLEU-4), already calling into question the hypothesis that n-grams are the sole explanation for our finding. Because of this, we only compare BLEU-2 scores. (Detailed experiments on specially constructed permutations that cover the entire range of BLEU-3 and BLEU-4 is provided in \autoref{app_sec:bleu_all}). Although the probability of a permuted sentence to be predicted correctly does correlate with BLEU-2 score (Figure \ref{fig:bleu_2}), the percentage of examples predicted correct by Transformer-based models is still higher than we would expect from permutations with lower BLEU-2 (66\% for the lowest BLEU-2 range of $0-0.15$), suggesting preserved relative word order alone cannot explain the high permutation acceptance rates.



% \paragraph{Part-of-speech neighborhood tracks \PermAcc.} 

Many syntactic formalisms, like Lexical Functional Grammar \citep[LFG]{kaplan-bresnan-1995-formal, bresnan-etal-2015-lexical}, Head-drive Phrase Structure Grammar \citep[HPSG]{pollard-sag-1994-head} or Lexicalized Tree Adjoining Grammar \citep[LTAG]{schabes-etal-1988-parsing, abeille-1990-lexical}, are ``lexicalized'', i.e., individual words or morphemes bear syntactic features telling us which other words they can combine with. For example, ``buy'' could be associated with (at least) two lexicalized syntactic structures, one containing two noun phrases (as in \textit{\underline{Kim} bought \underline{cheese}}), and another with three (as in \textit{\underline{Lee} bought \underline{Logan} \underline{cheese}}). %In this way, an average tree family for any particular word, like \textit{buy}, can provide information about neighbors it often appears with. Taking inspiration from formalisms like this, 
We speculate that our NLI models might accept permuted examples at high rates, %even when local n-gram orders aren't preserved, 
because they are (perhaps noisily) reconstructing the original sentence from abstract, word-anchored information about common neighbors. 

To test this, we POS-tagged $D_{\text{train}}$ using 17 Universal Part-of-Speech tags (using spaCy, \citealt{spacy2}). %For each token in $D_{\text{train}}$, we get a summary of that token's syntactic context (at a symmetrical distance set by a hyperparameter we call the \textbf{radius}). Concretely,
For each $w_i \in S_i$, we compute the occurrence probability of POS tags on tokens in the \textit{neighborhood} of $w_i$. The neighborhood is specified by the radius $r$ (a symmetrical window $r$ tokens from $w_i \in S_i$ to the left and right). We denote this sentence level probability of neighbor POS tags for a word $w_i$ as $\psi^r_{\{w_i, S_i\}} \in \mathcal{R}^{17}$ (see an example in \autoref{fig:exPOSsignature} in the Appendix). Sentence-level word POS neighbor scores can be averaged across $ D_{\text{train}}$ to get a type level score $\psi^r_{\{w_i, D_{\text{train}}\}} \in \mathcal{R}^{17}, \forall w_i \in D_{\text{train}}$.  Then, for a sentence $S_i \in D_{\text{test}}$, for each word $w_i \in S_i$, we compute a \textbf{POS mini-tree overlap score}:
%\begin{align}
\begin{equation} % want left align
\begin{split}
\beta^k_{\{w_i,S_i\}} =
\frac{1}{k} \mid \text{argmax}_k &\psi^r_{\{w_i, D_{\text{train}}\}} \cap \\ &\text{argmax}_k\psi^r_{\{w_i, S_i\}} \mid
\end{split}
\end{equation}
%\end{align}

% POS signature?
Concretely, $\beta^k_{\{w_i,S_i\}}$ computes the overlap of top-$k$ POS tags in the neighborhood of a word $w_i$ in $S$ with that of the train statistic. If a word has the same mini-tree in a given sentence as it has in the training set, then the overlap would be 1. For a given sentence $S_i$, the aggregate $\beta^k_{\{S_i\}}$ is defined by the average of the overlap scores of all its words: $\beta^k_{\{S_i\}} = \frac{1}{|S_i|}\sum_{w_i \in S_i} \beta^k_{\{w_i,S_i\}}$, and we call it a POS minitree \textit{signature}. We can also compute the POS minitree signature of a permuted sentence $\hat{S}_i$ to have $\beta^k_{\{\hat{S}_i\}}$.  If the permuted sentence POS signature comes close to that of the true sentence, then their ratio (i.e.,  $\beta^k_{\{\hat{S}_i\}} / \beta^k_{\{S_i\}}$) will be close to 1. Also, since POS signature is computed with respect to the train distribution, a ratio of $>$ 1 indicates that the permuted sentence is closer to the overall train statistic than to the original unpermuted sentence in terms of POS signature. %If models ``know syntax'' then for any samples (from the permuted test set or from the original training) that have high average (?) POS tag minitree overlap scores (i.e., words with POS tag minitree scores similar to their POS tag minitree overlap score on the training set), we should expect those samples to be fairly easy for models to do well on. 
If  high overlap with the training distribution correlates with percentage of permutations deemed correct, then our models treat words as if they project syntactic minitrees. 
%Therefore, for $\hat{D}_{\text{test}}$ where many of the $n$ permutations have high average POS minitree overlap score, we should expect a higher prediction accuracy.

\begin{figure}[t]
    \centering
    \resizebox{\linewidth}{!}{
        \includegraphics{images/min_tree_4.png}}
    \caption{POS Tag Mini Tree overlap score and  percentage of permutations which the models assigned the gold-label.}
    \label{fig:min_tree_4}
\end{figure}

We investigate the relationship with percentage of permuted sentences accepted with $\beta^k_{\{\hat{S}_i\}} / \beta^k_{\{S_i\}}$ in \autoref{fig:min_tree_4}. We observe that the POS Tag Minitree hypothesis holds for Transformer-based models, RoBERTa, BART and DistilBERT, where the percentage of accepted pairs increase as the sentences have higher overlap with the un-permuted sentence in terms of POS signature. For non-Transformer models such as InferSent, ConvNet, and BiLSTM models, the POS signature ratio to percentage of correct permutation remains the same or decreases, suggesting that the reasoning process employed by these models does not preserve local abstract syntax structure (i.e., POS neighbor relations).

%KS: Ideas: should we show POS tag correlation with original un-permuted sentences as well?

\section{Human Evaluation}
\label{sec:human_eval}

%Since our models often accept permuted sentences, we ask how humans perform unnatural language inference on permuted sentences. 
We expect humans to struggle with UNLI, given our intuitions and the sentence superiority findings (but see \citealt{mollica-2020-composition}). To test this, we presented two experts in NLI (one a linguist) with permuted sentence pairs to label.\footnote{Concurrent work by \newcite{gupta-etal-2021-bert} found that untrained crowdworkers accept NLI examples that have been subjected to different kinds of perturbations at roughly most frequent class levels---i.e., only 35\% of the time.} Concretely, we draw equal number of examples from MNLI Matched dev set (100 examples where RoBERTa predicts the gold label, $D^c$ and 100 examples where it fails to do so, $D^f$), and then permute these examples using $\mathcal{F}$. The experts were given no additional information (recall that it is common knowledge that NLI is a roughly balanced 3-way classification task). Unbeknownst to the experts, all permuted sentences in the sample were actually accepted by the RoBERTa (large) model (trained on MNLI dataset). We observe that the experts performed much worse than RoBERTa (\autoref{tab:human_eval}), although their accuracy was a bit higher than random. We also find that for both experts, accuracy on permutations from $D^c$ was higher than on $D^f$, which verifies findings that showed high word overlap can give hints about the ground truth label \citep{dasgupta-etal-2018-evaluating, poliak-etal-2018-hypothesis, gururangan-etal-2018-annotation, naik-etal-2019-exploring}.

% Please add the following required packages to your document preamble:
% \usepackage{booktabs}
% \usepackage{graphicx}
\begin{table}[t]
\centering
\resizebox{\linewidth}{!}{%
\begin{tabular}{@{}lllll@{}}
\toprule
Evaluator & Accuracy & Macro F1 & Acc on $D^{c}$ & Acc on $D^{f}$ \\ \midrule
X & 0.581 $\tiny\pm 0.068$ & 0.454 & 0.649 $\tiny\pm 0.102$ & 0.515 $\tiny\pm 0.089$ \\
Y &  0.378 $\tiny\pm 0.064$ & 0.378 & 0.411 $\tiny\pm 0.098$ & 0.349 $\tiny\pm 0.087$ \\ %\midrule
%crowd^{*} & & 0.35 & \\
\bottomrule
\end{tabular}%
}
\caption{Human (expert) evaluation on 200 permuted examples from the MNLI matched development set. Half of the permuted pairs contained shorter sentences and the other, longer ones. %Experts were provided only the permuted sentences (not the original example or the label) and were disallowed from consulting with one another. 
All permuted examples were assigned the gold label by RoBERTa-Large. %crowd$^*$ was crowdworker accuracy measured by \newcite{gupta-etal-2021-bert} on the GLUE Benchmark. 
}
\label{tab:human_eval}
\end{table}

\section{Training by Maximizing Entropy}
\label{sec:training}
% training done on roberta.large

We propose an initial attempt to mitigate the effect of correct prediction on permuted examples. As we observe in \autoref{sec:eval}, model entropy on permuted examples is significantly lower than expected. 
%This kind of phenomenon has been observed prior in Computer Vision \cite{gandhi2019mutual}, and suggests model struggle to to learn \textit{mutually exclusively}. 
Neural networks tend to output higher confidence than random for even unknown inputs \cite{gandhi2019mutual}, which might be an underlying cause of the high \PermAcc.

% \begin{figure}
%     \centering
%     \resizebox{\linewidth}{!}{
%         \includegraphics{images/me_train_roberta.png}}
%     \caption{Effect of maximizing entropy training on RoBERTa (large)}
%     \label{fig:max_ent_roberta}
% \end{figure}

An ideal model would be ambivalent about randomized ungrammatical sentences. Thus, we train NLI models baking in the principle of mutual exclusivity \citep{gandhi2019mutual} by maximizing model entropy. Concretely, we fine-tune RoBERTa on MNLI while maximizing the entropy ($\bm{\mathrm{H}}$) on a subset of $n$ randomized examples ($(\hat{p}_i, \hat{r}_i$), for each example ($p,h$) in MNLI. We modify the loss function as follows:
% Prasanna
\begin{dmath}
    \mathcal{L}=\argminB_{\theta}\sum_{\left((p, h),y\right)}y\log(p(y|(p,h);\theta)) + \sum_{i=1}^n \bm{\mathrm{H}}\left(y|(\hat{p}_i,\hat{h}_i);\theta\right)
\end{dmath}
Using this maximum entropy method ($n=1$), we find that the model improves considerably with respect to its robustness to randomized sentences, all while taking no hit to accuracy (\autoref{table:ME_train_roberta}). We observe that no model reaches a $\Omega_{\text{max}}$ score close to 0, suggesting further room to explore other methods for decreasing models' \PermAcc. Similar approaches have also proven useful \citep{gupta-etal-2021-bert} for other tasks as well.

\begin{table}[t]
\centering
\resizebox{\linewidth}{!}{%
\begin{tabular}{lrrrr}
\toprule
   Eval Dataset &   $\mathcal{A}$ (V) &  $\mathcal{A}$ (ME) &  $\Omega_{\text{max}}$ (V) &  $\Omega_{\text{max}}$ (ME) \\
\midrule
  MNLI\_m\_dev &  0.905 &     0.908 &    0.984 &        0.328 \\
 MNLI\_mm\_dev &  0.901 &     0.903 &    0.985 &        0.329 \\
   SNLI\_test &  0.882 &     0.888 &    0.983 &        0.329 \\
    SNLI\_dev &  0.879 &     0.887 &    0.984 &        0.333 \\
 ANLI\_r1\_dev &  0.456 &     0.470 &    0.890 &        0.333 \\
 ANLI\_r2\_dev &  0.271 &     0.258 &    0.880 &        0.333 \\
 ANLI\_r3\_dev &  0.268 &     0.243 &    0.892 &        0.334 \\
\bottomrule
\end{tabular}}
\caption{NLI Accuracy ($\mathcal{A}$) and \PermAcc\ metrics ($\Omega_{\text{max}}$) of RoBERTa when trained on MNLI dataset using vanilla (V) and Maximum Random Entropy (ME) method.}
  \label{table:ME_train_roberta}
\end{table}

% add Gupta et al here

\section{Future Work \& Conclusion}

We show that state-of-the-art models do not rely on sentence structure the way we think they should: NLI models (Transformer-based models, RNNs, and ConvNets) are largely insensitive to permutations of word order that corrupt the original syntax. %This raises questions about the extent to which such systems understand ``syntax'', and highlights the unnatural language understanding processes they employ. 
%While we have shown that classification labels can be flipped based solely on a sentence reordering, on interesting future direction to explore might be to additionally consider dropping tokens. Such an approach has been taken in the realm of interpretability to determine the impact of particular tokens on model decisions \citep{serrano-smith-2019-attention}. %% COMMENT OUT ABOVE FOR ARXIV SUBMIT.
We also show that reordering words can cause models to flip classification labels. % future work could additionally explore the relationship between permutation and deletion. 
We do find that models seem to have learned some syntactic information as is evidenced by a correlation between preservation of abstract POS neighborhood information and rate of acceptance by models, but these results do not discount the high rates of \PermAcc, and require further verification. Coupled with the finding that humans cannot perform UNLI at all well, the high rate of permutation acceptance that we observe leads us to conclude that current models do not yet ``know syntax'' in the fully systematic and humanlike way we would like them to. 


A few years ago, \newcite{manning-etal-2015-computational} encouraged NLP to consider ``the details of human language, how it is learned, processed, and how it changes, rather than just chasing state-of-the-art numbers on a benchmark task.'' We expand upon this view, and suggest one particular future direction: we should train models not only to do well on clean test data, but also to not to overgeneralize to corrupted input. %perform natural language understanding in an humanlike way


\section*{Acknowledgments}

Thanks to Omar Agha, Dzmitry Bahdanau, Sam Bowman, Hagen Blix, Ryan Cotterell, Emily Dinan, Michal Drozdal, Charlie Lovering, Nikita Nangia, Alicia Parrish, Grusha Prasad, Roy Schwartz, Shagun Sodhani, Anna Szabolsci, Alex Warstadt, Jackie Chi-kit Cheung, Timothy O'Donnell and members of McGill MCQLL lab for many invaluable comments and feedback on early drafts. 
%This work is dedicated to the countless lives lost due to the Covid-19 pandemic in India and around the world.

\bibliography{acl2020}
\bibliographystyle{acl_natbib}

\clearpage

\appendix

% \section{\PermAcc\ full results}

% KS: do we need this figure now?
\begin{figure}[t]
    \centering
    \resizebox{\linewidth}{!}{
        \includegraphics{images/comb_plot_0.png}}
    \caption{Comparison of $\Omega_{\text{max}}$,$\Omega_{\text{rand}}$,$\mathcal{P}^c$ and $\mathcal{P}^f$ with the model accuracy $\mathcal{A}$ on multiple datasets, where all models are trained on the MNLI corpus \cite{williams-etal-2018-broad}.}
    \label{fig:comb_plot}
\end{figure}

% \begin{table}[htbp]
  \centering
  \resizebox{\linewidth}{!}{%
    \begin{tabular}{@{}llrrrrr@{}}
\toprule
           Model &    Eval. Dataset  &  $\mathcal{A}$ &  $\Omega_{\text{max}}$ &  $\mathcal{P}^c$ &  $\mathcal{P}^f$ &  $\Omega_{\text{rand}}$ \\
\midrule
\multirow{7}{*}{\bf RoBERTa-Large} 
 &   MNLI\_m\_dev &              0.906 &         0.987 &                  0.707 &             0.383 &                        0.794 \\
 &  MNLI\_mm\_dev &              0.901 &         0.987 &                  0.707 &             0.387 &                        0.790 \\
 &     SNLI\_dev &              0.879 &         0.988 &                  0.768 &             0.393 &                        0.826 \\
 &    SNLI\_test &              0.883 &         0.988 &                  0.760 &             0.407 &                        0.828 \\
 &  A1* &              0.456 &         0.897 &                  0.392 &             0.286 &                        0.364 \\
 &  A2* &              0.271 &         0.889 &                  0.465 &             0.292 &                        0.359 \\
 &  A3* &              0.268 &         0.902 &                  0.480 &             0.308 &                        0.397 \\ \midrule
 & Mean & 0.652 & 0.948 & 0.611 & \boldred{0.351} & 0.623 \\ 
% & Harmonic Mean & 0.497 & 0.946 & 0.572 & 0.344 & 0.539 \\ 
\midrule
 
\multirow{7}{*}{\bf BART-Large} 
    &   MNLI\_m\_dev &              0.902 &         0.989 &                  0.689 &             0.393 &                        0.784 \\
    &  MNLI\_mm\_dev &              0.900 &         0.986 &                  0.695 &             0.399 &                        0.788 \\
    &     SNLI\_dev &              0.886 &         0.991 &                  0.762 &             0.363 &                        0.834 \\
    &    SNLI\_test &              0.888 &         0.990 &                  0.762 &             0.370 &                        0.836 \\
    &  A1* &              0.455 &         0.894 &                  0.379 &             0.295 &                        0.374 \\
    &  A2* &              0.316 &         0.887 &                  0.428 &             0.303 &                        0.397 \\
    &  A3* &              0.327 &         0.931 &                  0.428 &             0.333 &                        0.424 \\ \midrule
& Mean &  \textbf{0.668} & \boldred{0.953} & 0.592 & \boldred{0.351} & \boldred{0.634} \\
%& Harmonic Mean &  \textbf{0.543} & \boldred{0.951} & 0.546 & \boldred{0.347} & \boldred{0.561} \\ 
\midrule
\multirow{7}{*}{\bf DistilBERT}  &   MNLI\_m\_dev &              0.800 &         0.968 &                  0.775 &             0.343 &                        0.779 \\
      &  MNLI\_mm\_dev &              0.811 &         0.968 &                  0.775 &             0.346 &                        0.786 \\
      &     SNLI\_dev &              0.732 &         0.956 &                  0.767 &             0.307 &                        0.731 \\
      &    SNLI\_test &              0.738 &         0.950 &                  0.770 &             0.312 &                        0.725 \\
      &  A1* &              0.251 &         0.750 &                  0.511 &             0.267 &                        0.300 \\
      &  A2* &              0.300 &         0.760 &                  0.619 &             0.265 &                        0.343 \\
      &  A3* &              0.312 &         0.830 &                  0.559 &             0.259 &                        0.363 \\ \midrule
      & Mean &  0.564 & 0.883 & \boldred{0.682} & 0.300 & 0.575 \\ 
%& Harmonic Mean &  0.445 & 0.873 & \boldred{0.664} & 0.296 & 0.490 \\
% \bottomrule
\midrule\midrule
 \multirow{7}{*}{\bf InferSent} 
 &   MNLI\_m\_dev &              0.658 &         0.904 &                  0.842 &             0.359 &                        0.712 \\
 &  MNLI\_mm\_dev &              0.669 &         0.905 &                  0.844 &             0.368 &                        0.723 \\
 &     SNLI\_dev &              0.556 &         0.820 &                  0.821 &             0.323 &                        0.587 \\
 &    SNLI\_test &              0.560 &         0.826 &                  0.824 &             0.321 &                        0.600 \\
 &  A1* &              0.316 &         0.669 &                  0.425 &             0.395 &                        0.313 \\
 &  A2* &              0.310 &         0.662 &                  0.689 &             0.249 &                        0.330 \\
 &  A3* &              0.300 &         0.677 &                  0.675 &             0.236 &                        0.332 \\ \midrule
 & Mean &  \textbf{0.481} & 0.780 & 0.731 & \boldred{0.322} & 0.514 \\ 
 %& Harmonic Mean &  0.429 & 0.767 & 0.694 & \boldred{0.311} & 0.455 \\ 
 \midrule
 \multirow{7}{*}{\bf ConvNet}
 &   MNLI\_m\_dev &              0.631 &         0.926 &                  0.773 &             0.340 &                        0.684 \\
 &  MNLI\_mm\_dev &              0.640 &         0.926 &                  0.782 &             0.343 &                        0.694 \\
 &     SNLI\_dev &              0.506 &         0.819 &                  0.813 &             0.339 &                        0.597 \\
 &    SNLI\_test &              0.501 &         0.821 &                  0.809 &             0.341 &                        0.596 \\
 &  A1* &              0.271 &         0.708 &                  0.648 &             0.218 &                        0.316 \\
 &  A2* &              0.307 &         0.725 &                  0.703 &             0.224 &                        0.356 \\
 &  A3* &              0.306 &         0.798 &                  0.688 &             0.234 &                        0.388 \\ \midrule
 & Mean &  0.452 & \boldred{0.817} & \boldred{0.745} & 0.291 & 0.519 \\ 
%& Harmonic Mean &  0.404 & \boldred{0.810} & \boldred{0.740} & 0.279 & \boldred{0.473} \\ 
\midrule
 \multirow{7}{*}{\bf BiLSTM} 
 &   MNLI\_m\_dev &              0.662 &         0.925 &                  0.800 &             0.351 &                        0.711 \\
 &  MNLI\_mm\_dev &              0.681 &         0.924 &                  0.809 &             0.344 &                        0.724 \\
 &     SNLI\_dev &              0.547 &         0.860 &                  0.762 &             0.351 &                        0.598 \\
 &    SNLI\_test &              0.552 &         0.862 &                  0.771 &             0.363 &                        0.607 \\
 &  A1* &              0.262 &         0.671 &                  0.648 &             0.271 &                        0.340 \\
 &  A2* &              0.297 &         0.728 &                  0.672 &             0.209 &                        0.328 \\
 &  A3* &              0.304 &         0.731 &                  0.656 &             0.219 &                        0.331 \\ \midrule
 & Mean &  0.472 & 0.814 & 0.731 & 0.301 & \boldred{0.520} \\
%& Harmonic Mean &  0.410 & 0.803 & 0.725 & 0.287 & 0.463 \\
 
\bottomrule
\end{tabular}}
  \caption{Statistics for Transformer-based models trained on MNLI corpus \cite{williams-etal-2018-broad}. 
  %$\Omega_{\text{max}}$ or Max Accuracy is computed if \textit{any} of the $n=100$ permutations per data point yield correct results. The mean number of permutations which were correct, when the original prediction is correct or incorrect  are given by $\mathcal{P}^c$  and $\mathcal{P}^f$ (flipped) respectively. $\Omega_{\text{rand}}$ is the percentage of data points for which models choose the ground truth label over a random uniform baseline (1/3). 
  The highest values are bolded (\boldred{red} indicates the model most insensitive to permutation) per metric and per model class (Transformers and non-Transformers). A1*, A2* and A3* refer to the ANLI dev. sets \citep{nie-etal-2020-adversarial}.}
  \label{table:main}
\end{table}

% The complete results of our evaluation are provided in \autoref{table:main}. For all experiments, we train the NLI models on MNLI dataset \cite{williams-etal-2018-broad}, and further evaluate on in-distribution (SNLI dev and test) and out-of-distribution (ANLI R1, R2 and R3, denoted as A1, A2 and A3, respectively).

\section{Effect of Length on \PermAcc}
\label{app_sec:length}

We investigate the effect of length on \PermAcc\ in Figure \ref{fig:length}. We observe that shorter sentences in general have a somewhat higher probability of acceptance for examples which was originally predicted correctly---since shorter sentences have fewer unique permutations. However, for the examples which were originally incorrect, the trend is not present.  

\begin{figure}[ht]
    \centering
    \resizebox{0.5\textwidth}{!}{
        \includegraphics{images/transformer_length.png}}
    \caption{Length and \PermAcc by Transformer-based models.}
    \label{fig:length}
\end{figure}

\section{Example of POS Minitree}

In \autoref{sec:pos_mini_tree}, we developed a POS signature for each word in at least one example in a test set, then compare that signature to the distribution of the same word in the training set. \autoref{fig:exPOSsignature} provides a snapshot a word ``river" from the test set and shows how the POS signature distribution of the word in a particular example match with that of aggregated training statistic. In practice, we select the top $k$ POS tags for the word in the test signature as well as the train, and calculate their overlap. When comparing the model performance with permuted sentences, we compute a ratio between the original test overlap score and an overlap score calculated instead from the permuted test. In the \autoref{fig:exPOSsignature}, `river' would have a POS tag minitree score of 0.75.

\begin{figure}[ht]
    \centering
    \resizebox{0.5\linewidth}{!}{
        \includegraphics{images/riverPOSsig}}
    \caption{Example POS signature for the word `river', calculated with a radius of 2. Probability of each neighbor POS tag is provided. Orange examples come from the permuted test set, and blue come from the original training data. }
    \label{fig:exPOSsignature}
\end{figure}

\begin{figure*}
    \centering
    \resizebox{\textwidth}{!}{
        \includegraphics{images/omega_threshold.png}}
    \caption{$\Omega_x$ threshold for all datasets with varying $x$ and computing the percentage of examples that fall within the threshold. The top row consists of in-distribution datasets (MNLI, SNLI) and the bottom row contains out-of-distribution datasets (ANLI)}
    \label{fig:threshold_omega_x}
\end{figure*}

\section{Effect of Hypothesis-Only Randomization}
\label{app_sec:HO}

\begin{figure*}[ht]
    \centering
    \subfigure[]{
    \resizebox{0.48\textwidth}{!}{
        \includegraphics{images/hypothesis_compare.png}}
    \label{subfig:compare}
    }
    \subfigure[]{
    \resizebox{0.48\textwidth}{!}{
        \includegraphics{images/hypothesis_compare_cor_mean.png}}
    \label{subfig:compare_cor}
    }
    \caption{Comparing the effect between randomizing both premise and hypothesis and only hypothesis on two Transformer-based models, RoBERTa and BART (For more comparisons please refer to Appendix). In \ref{subfig:compare}, we observe the difference of $\Omega_{\text{max}}$ is marginal in in-distribution datasets (SNLI, MNLI), while hypothesis-only randomization is worse for out-of-distribution datasets (ANLI). In \ref{subfig:compare_cor}, we compare the mean number of permutations which elicited correct response, and naturally the hypothesis-only randomization causes more percentage of randomizations to be correct.}
    \label{fig:hypothesis_compare}
\end{figure*}


In recent years, the impact of the hypothesis sentence \citep{gururangan-etal-2018-annotation, tsuchiya-2018-performance, poliak-etal-2018-hypothesis} on NLI classification has been a topic of much interest. As we define in \autoref{sec:permutation}, logical entailment can only be defined for pairs of propositions. We investigated one effect where we randomize only the hypothesis sentences while keeping the premise intact. Figure \ref{subfig:compare} shows that the $\Omega_{\text{max}}$ value is almost the same for the two schemes; randomizing the hypothesis alone also leads the model to accept many permutations.

\section{Effect of clumped words in random permutations}
\label{app_sec:bleu_all}

\begin{figure}
    \centering
    \resizebox{0.48\textwidth}{!}{
        \includegraphics{images/bleu_2_3_4.png}}
    \caption{Relation of BLEU-2/3/4 scores against the acceptability of clumped-permuted sentences accross all test datasets on all models. }
    \label{fig:bleu_2_3_4}
\end{figure}

Since our original permuted dataset consists of extremely randomized words, we observe very low BLEU-3 ($<$ 0.2) and BLEU-4 scores ($<$ 0.1). To study the effect of overlap across a wider range of permutations, we devised an experiment where we clump certain words together before performing random permutations. Concretely, we clump 25\%, 50\% and 75\% of the words in a sentence and then permute the remaining words and the clumped word as a whole. This type of clumped-permutation allows us to study the full range of BLEU-2/3/4 scores, which we present in \autoref{fig:bleu_2_3_4}. As expected, the acceptability of permuted sentences increase linearly with BLEU score overlap. 

\section{Effect of the threshold of $\Omega_x$ in various test splits}
\label{app_sec:threshold}

We defined two variations of $\Omega_x$, $\Omega_{\text{max}}$ and $\Omega_{\text{rand}}$, but theoretically it is possible to define any arbitrary threshold percentage $x$ to evaluate the unnatural language inference mechanisms of different models. In \autoref{fig:threshold_omega_x} we show the effect of different thresholds, including $\Omega_{\text{max}}$ where $x = 1/|D_{\text{test}|}$ and $\Omega_{\text{rand}}$ where $x = 0.34$. We observe for in-distribution datasets (top row, MNLI and SNLI splits), in the extreme setting when $x=1.0$, there are more than 10\% of examples available, and more than 25\% in case of InferSent and DistilBERT. For out-of-distribution datasets (bottom row, ANLI splits) we observe a much lower trend, suggesting generalization itself is the bottleneck in permuted sentence understanding.

% \section{Further Explanation of Max Accuracy and Random Accuracy}

% % Tentative figure, will discuss today
% \begin{figure}
%     \centering
%     \resizebox{0.48\textwidth}{!}{
%         \includegraphics{images/comb.png}}
%     \caption{Preparation of Test sets}
%     \label{fig:max_accruacy}
% \end{figure}

% \section{Linear Probe on acceptability}

\section{Training with permuted examples}

In this section, we hypothesize that if the NLU models are mostly insensitive to word order, then training using permuted examples should also yield the same or comparable accuracy as training using grammatically correct data (i.e., the \textit{standard setup}). To test this, we train Transformer-based models on top of $\hat{D}_{\text{train}}$, which is computed by applying $\mathcal{F}$ on each example of $D_{\text{train}}$ for $q=1$ times. This ensures a control case where we keep the same amount of training data as the standard setup (such that models does not benefit from data augmentation). We also ensure that we use the same hyperparameters while training as with the standard setup. Concretely, $\hat{D}_{\text{train}}$ consists of $n$ hypothesis-premise pairs from MNLI training data, where each example is a permuted output of the original pair. 

\begin{table}
\centering
\resizebox{\linewidth}{!}{%
\begin{tabular}{lcccccc}
\toprule
Eval Data & \multicolumn{2}{c}{RoBERTa} & \multicolumn{2}{c}{BART} & \multicolumn{2}{c}{DistilBERT} \\
 & $\mathcal{A}$ & $\hat{\mathcal{A}}$ & $\mathcal{A}$ & $\hat{\mathcal{A}}$ & $\mathcal{A}$ & $\hat{\mathcal{A}}$ \\
\midrule
MNLI Matched & 0.906 & 0.877 & 0.902 & 0.862 & 0.800 & 0.760 \\
MNLI Mismatched & 0.901 & 0.878 & 0.900 & 0.869 & 0.811 & 0.769 \\
SNLI Dev & 0.879 & 0.870 & 0.886 & 0.854 & 0.732 & 0.719 \\
SNLI Test & 0.883 & 0.873 & 0.888 & 0.859 & 0.738 & 0.719  \\
ANLI R1 (Dev) & 0.456 & 0.367 & 0.455 & 0.336 & 0.251 & 0.250 \\
ANLI R2 (Dev) & 0.271 & 0.279 & 0.316 & 0.293 & 0.300 & 0.290  \\
ANLI R3 (Dev) & 0.268 & 0.271 & 0.327 & 0.309 & 0.312 & 0.312 \\
\bottomrule
\end{tabular}}
 \caption{Statistics for Transformer-based models when trained on permuted MNLI corpus. We compare the accuracy for both models trained on unpermuted data ($\mathcal{A}$) and the permuted data ($\hat{\mathcal{A}}$). We use original test sets during inference.}
\label{tab:train_with_rand}
\end{table}

We present the results of such training in \autoref{tab:train_with_rand}, and compare the accuracy ($\hat{\mathcal{A}}$) with that of the standard setup ($\mathcal{A}$). Note, during inference for all the models we use the un-permuted examples.
As we can see, models perform surprisingly close to the original accuracy $\mathcal{A}$ \textit{even when trained with ungrammatical sentences}. This adds further proof to the BOW nature of NLU models.




% Write about 

% Plot of Flip permutations?
% KS: currently training a 2 layer MLP to predict the hit

\begin{table}
\centering
\resizebox{\linewidth}{!}{%
\begin{tabular}{lrr}
\toprule
Dataset & Test Examples & Used Examples\\
\midrule
MNLI Matched & 9815 & 6655\\
MNLI Mismatched & 9832 & 7449\\
SNLI Dev & 9842 & 3697\\
SNLI Test & 9824 & 3671\\
ANLI R1 (Dev) & 1000 & 756\\
ANLI R2 (Dev) & 1000 & 709\\
ANLI R3 (Dev) & 1200 & 754\\
\bottomrule
\end{tabular}}
 \caption{Dataset statistics used in this paper for inference. `Used Examples' provides the number of premise-hypothesis pairs for the dataset which we selected for inference (i.e., examples where at least 100 unique permutations were possible).}
\label{tab:data_stats}
\end{table}

% TOOD: another section on the data generation process

\section{Reproducibility Checklist}

As per the prescribed Reproducibility Checklist, we provide the information of the following:

\begin{itemize}
    \item \textit{A clear description of the mathematical setting, algorithm and/or model}: We provide details of models used in \autoref{sec:eval}
    \item \textit{Description of the computing infrastructure used}: We used 8 NVIDIA V100 32GB GPUs to train the models and perform all necessary inferences. We didn't run hyperparameter tuning for Transformer-based models as we used the published hyperparameters from the original models.
    \item \textit{Average runtime for each approach}: On an average, each model inference experiment consistine of 100 permutations for each example takes roughly 1 hour to complete.
    \item \textit{Relevant statistics of the datasets used}: We provide the statistics of the datasets used in \autoref{tab:data_stats}.
    \item \textit{Explanation of any data that were excluded, and all pre-processing steps}: We exclude examples where either the hypothesis and premise consists of less than 6 tokens. This way, we ensure that we have 100 unique permutations for each example.
    \item \textit{Link to downloadable version of data and code}: We provide downloadable version of our data and code at \href{https://github.com/facebookresearch/unlu}{https://github.com/facebookresearch/unlu}.
\end{itemize}


\end{document}
