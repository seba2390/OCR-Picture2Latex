\documentclass{article}

\usepackage[left=3cm,right=3cm,top=4cm,bottom=4cm]{geometry}

\usepackage{tikz}
\usepackage{amsmath}
%%%%% NEW MATH DEFINITIONS %%%%%

\usepackage{amsmath,amsfonts,bm}

% Mark sections of captions for referring to divisions of figures
\newcommand{\figleft}{{\em (Left)}}
\newcommand{\figcenter}{{\em (Center)}}
\newcommand{\figright}{{\em (Right)}}
\newcommand{\figtop}{{\em (Top)}}
\newcommand{\figbottom}{{\em (Bottom)}}
\newcommand{\captiona}{{\em (a)}}
\newcommand{\captionb}{{\em (b)}}
\newcommand{\captionc}{{\em (c)}}
\newcommand{\captiond}{{\em (d)}}

% Highlight a newly defined term
\newcommand{\newterm}[1]{{\bf #1}}


% Figure reference, lower-case.
\def\figref#1{figure~\ref{#1}}
% Figure reference, capital. For start of sentence
\def\Figref#1{Figure~\ref{#1}}
\def\twofigref#1#2{figures \ref{#1} and \ref{#2}}
\def\quadfigref#1#2#3#4{figures \ref{#1}, \ref{#2}, \ref{#3} and \ref{#4}}
% Section reference, lower-case.
\def\secref#1{section~\ref{#1}}
% Section reference, capital.
\def\Secref#1{Section~\ref{#1}}
% Reference to two sections.
\def\twosecrefs#1#2{sections \ref{#1} and \ref{#2}}
% Reference to three sections.
\def\secrefs#1#2#3{sections \ref{#1}, \ref{#2} and \ref{#3}}
% Reference to an equation, lower-case.
% \def\eqref#1{equation~\ref{#1}}
 \def\eqref#1{(\ref{#1})}
% Reference to an equation, upper case
\def\Eqref#1{Equation~\ref{#1}}
% A raw reference to an equation---avoid using if possible
\def\plaineqref#1{\ref{#1}}
% Reference to a chapter, lower-case.
\def\chapref#1{chapter~\ref{#1}}
% Reference to an equation, upper case.
\def\Chapref#1{Chapter~\ref{#1}}
% Reference to a range of chapters
\def\rangechapref#1#2{chapters\ref{#1}--\ref{#2}}
% Reference to an algorithm, lower-case.
\def\algref#1{algorithm~\ref{#1}}
% Reference to an algorithm, upper case.
\def\Algref#1{Algorithm~\ref{#1}}
\def\twoalgref#1#2{algorithms \ref{#1} and \ref{#2}}
\def\Twoalgref#1#2{Algorithms \ref{#1} and \ref{#2}}
% Reference to a part, lower case
\def\partref#1{part~\ref{#1}}
% Reference to a part, upper case
\def\Partref#1{Part~\ref{#1}}
\def\twopartref#1#2{parts \ref{#1} and \ref{#2}}

\def\ceil#1{\lceil #1 \rceil}
\def\floor#1{\lfloor #1 \rfloor}
\def\1{\bm{1}}
\newcommand{\train}{\mathcal{D}}
\newcommand{\valid}{\mathcal{D_{\mathrm{valid}}}}
\newcommand{\test}{\mathcal{D_{\mathrm{test}}}}

\def\eps{{\epsilon}}


% Random variables
\def\reta{{\textnormal{$\eta$}}}
\def\ra{{\textnormal{a}}}
\def\rb{{\textnormal{b}}}
\def\rc{{\textnormal{c}}}
\def\rd{{\textnormal{d}}}
\def\re{{\textnormal{e}}}
\def\rf{{\textnormal{f}}}
\def\rg{{\textnormal{g}}}
\def\rh{{\textnormal{h}}}
\def\ri{{\textnormal{i}}}
\def\rj{{\textnormal{j}}}
\def\rk{{\textnormal{k}}}
\def\rl{{\textnormal{l}}}
% rm is already a command, just don't name any random variables m
\def\rn{{\textnormal{n}}}
\def\ro{{\textnormal{o}}}
\def\rp{{\textnormal{p}}}
\def\rq{{\textnormal{q}}}
\def\rr{{\textnormal{r}}}
\def\rs{{\textnormal{s}}}
\def\rt{{\textnormal{t}}}
\def\ru{{\textnormal{u}}}
\def\rv{{\textnormal{v}}}
\def\rw{{\textnormal{w}}}
\def\rx{{\textnormal{x}}}
\def\ry{{\textnormal{y}}}
\def\rz{{\textnormal{z}}}

% Random vectors
\def\rvepsilon{{\mathbf{\epsilon}}}
\def\rvtheta{{\mathbf{\theta}}}
\def\rva{{\mathbf{a}}}
\def\rvb{{\mathbf{b}}}
\def\rvc{{\mathbf{c}}}
\def\rvd{{\mathbf{d}}}
\def\rve{{\mathbf{e}}}
\def\rvf{{\mathbf{f}}}
\def\rvg{{\mathbf{g}}}
\def\rvh{{\mathbf{h}}}
\def\rvu{{\mathbf{i}}}
\def\rvj{{\mathbf{j}}}
\def\rvk{{\mathbf{k}}}
\def\rvl{{\mathbf{l}}}
\def\rvm{{\mathbf{m}}}
\def\rvn{{\mathbf{n}}}
\def\rvo{{\mathbf{o}}}
\def\rvp{{\mathbf{p}}}
\def\rvq{{\mathbf{q}}}
\def\rvr{{\mathbf{r}}}
\def\rvs{{\mathbf{s}}}
\def\rvt{{\mathbf{t}}}
\def\rvu{{\mathbf{u}}}
\def\rvv{{\mathbf{v}}}
\def\rvw{{\mathbf{w}}}
\def\rvx{{\mathbf{x}}}
\def\rvy{{\mathbf{y}}}
\def\rvz{{\mathbf{z}}}

% Elements of random vectors
\def\erva{{\textnormal{a}}}
\def\ervb{{\textnormal{b}}}
\def\ervc{{\textnormal{c}}}
\def\ervd{{\textnormal{d}}}
\def\erve{{\textnormal{e}}}
\def\ervf{{\textnormal{f}}}
\def\ervg{{\textnormal{g}}}
\def\ervh{{\textnormal{h}}}
\def\ervi{{\textnormal{i}}}
\def\ervj{{\textnormal{j}}}
\def\ervk{{\textnormal{k}}}
\def\ervl{{\textnormal{l}}}
\def\ervm{{\textnormal{m}}}
\def\ervn{{\textnormal{n}}}
\def\ervo{{\textnormal{o}}}
\def\ervp{{\textnormal{p}}}
\def\ervq{{\textnormal{q}}}
\def\ervr{{\textnormal{r}}}
\def\ervs{{\textnormal{s}}}
\def\ervt{{\textnormal{t}}}
\def\ervu{{\textnormal{u}}}
\def\ervv{{\textnormal{v}}}
\def\ervw{{\textnormal{w}}}
\def\ervx{{\textnormal{x}}}
\def\ervy{{\textnormal{y}}}
\def\ervz{{\textnormal{z}}}

% Random matrices
\def\rmA{{\mathbf{A}}}
\def\rmB{{\mathbf{B}}}
\def\rmC{{\mathbf{C}}}
\def\rmD{{\mathbf{D}}}
\def\rmE{{\mathbf{E}}}
\def\rmF{{\mathbf{F}}}
\def\rmG{{\mathbf{G}}}
\def\rmH{{\mathbf{H}}}
\def\rmI{{\mathbf{I}}}
\def\rmJ{{\mathbf{J}}}
\def\rmK{{\mathbf{K}}}
\def\rmL{{\mathbf{L}}}
\def\rmM{{\mathbf{M}}}
\def\rmN{{\mathbf{N}}}
\def\rmO{{\mathbf{O}}}
\def\rmP{{\mathbf{P}}}
\def\rmQ{{\mathbf{Q}}}
\def\rmR{{\mathbf{R}}}
\def\rmS{{\mathbf{S}}}
\def\rmT{{\mathbf{T}}}
\def\rmU{{\mathbf{U}}}
\def\rmV{{\mathbf{V}}}
\def\rmW{{\mathbf{W}}}
\def\rmX{{\mathbf{X}}}
\def\rmY{{\mathbf{Y}}}
\def\rmZ{{\mathbf{Z}}}

% Elements of random matrices
\def\ermA{{\textnormal{A}}}
\def\ermB{{\textnormal{B}}}
\def\ermC{{\textnormal{C}}}
\def\ermD{{\textnormal{D}}}
\def\ermE{{\textnormal{E}}}
\def\ermF{{\textnormal{F}}}
\def\ermG{{\textnormal{G}}}
\def\ermH{{\textnormal{H}}}
\def\ermI{{\textnormal{I}}}
\def\ermJ{{\textnormal{J}}}
\def\ermK{{\textnormal{K}}}
\def\ermL{{\textnormal{L}}}
\def\ermM{{\textnormal{M}}}
\def\ermN{{\textnormal{N}}}
\def\ermO{{\textnormal{O}}}
\def\ermP{{\textnormal{P}}}
\def\ermQ{{\textnormal{Q}}}
\def\ermR{{\textnormal{R}}}
\def\ermS{{\textnormal{S}}}
\def\ermT{{\textnormal{T}}}
\def\ermU{{\textnormal{U}}}
\def\ermV{{\textnormal{V}}}
\def\ermW{{\textnormal{W}}}
\def\ermX{{\textnormal{X}}}
\def\ermY{{\textnormal{Y}}}
\def\ermZ{{\textnormal{Z}}}

% Vectors
\def\vzero{{\bm{0}}}
\def\vone{{\bm{1}}}
\def\vmu{{\bm{\mu}}}
\def\vtheta{{\bm{\theta}}}
\def\va{{\bm{a}}}
\def\vb{{\bm{b}}}
\def\vc{{\bm{c}}}
\def\vd{{\bm{d}}}
\def\ve{{\bm{e}}}
\def\vf{{\bm{f}}}
\def\vg{{\bm{g}}}
\def\vh{{\bm{h}}}
\def\vi{{\bm{i}}}
\def\vj{{\bm{j}}}
\def\vk{{\bm{k}}}
\def\vl{{\bm{l}}}
\def\vm{{\bm{m}}}
\def\vn{{\bm{n}}}
\def\vo{{\bm{o}}}
\def\vp{{\bm{p}}}
\def\vq{{\bm{q}}}
\def\vr{{\bm{r}}}
\def\vs{{\bm{s}}}
\def\vt{{\bm{t}}}
\def\vu{{\bm{u}}}
\def\vv{{\bm{v}}}
\def\vw{{\bm{w}}}
\def\vx{{\bm{x}}}
\def\vy{{\bm{y}}}
\def\vz{{\bm{z}}}

% Elements of vectors
\def\evalpha{{\alpha}}
\def\evbeta{{\beta}}
\def\evepsilon{{\epsilon}}
\def\evlambda{{\lambda}}
\def\evomega{{\omega}}
\def\evmu{{\mu}}
\def\evpsi{{\psi}}
\def\evsigma{{\sigma}}
\def\evtheta{{\theta}}
\def\eva{{a}}
\def\evb{{b}}
\def\evc{{c}}
\def\evd{{d}}
\def\eve{{e}}
\def\evf{{f}}
\def\evg{{g}}
\def\evh{{h}}
\def\evi{{i}}
\def\evj{{j}}
\def\evk{{k}}
\def\evl{{l}}
\def\evm{{m}}
\def\evn{{n}}
\def\evo{{o}}
\def\evp{{p}}
\def\evq{{q}}
\def\evr{{r}}
\def\evs{{s}}
\def\evt{{t}}
\def\evu{{u}}
\def\evv{{v}}
\def\evw{{w}}
\def\evx{{x}}
\def\evy{{y}}
\def\evz{{z}}

% Matrix
\def\mA{{\bm{A}}}
\def\mB{{\bm{B}}}
\def\mC{{\bm{C}}}
\def\mD{{\bm{D}}}
\def\mE{{\bm{E}}}
\def\mF{{\bm{F}}}
\def\mG{{\bm{G}}}
\def\mH{{\bm{H}}}
\def\mI{{\bm{I}}}
\def\mJ{{\bm{J}}}
\def\mK{{\bm{K}}}
\def\mL{{\bm{L}}}
\def\mM{{\bm{M}}}
\def\mN{{\bm{N}}}
\def\mO{{\bm{O}}}
\def\mP{{\bm{P}}}
\def\mQ{{\bm{Q}}}
\def\mR{{\bm{R}}}
\def\mS{{\bm{S}}}
\def\mT{{\bm{T}}}
\def\mU{{\bm{U}}}
\def\mV{{\bm{V}}}
\def\mW{{\bm{W}}}
\def\mX{{\bm{X}}}
\def\mY{{\bm{Y}}}
\def\mZ{{\bm{Z}}}
\def\mBeta{{\bm{\beta}}}
\def\mPhi{{\bm{\Phi}}}
\def\mLambda{{\bm{\Lambda}}}
\def\mSigma{{\bm{\Sigma}}}

% Tensor
\DeclareMathAlphabet{\mathsfit}{\encodingdefault}{\sfdefault}{m}{sl}
\SetMathAlphabet{\mathsfit}{bold}{\encodingdefault}{\sfdefault}{bx}{n}
\newcommand{\tens}[1]{\bm{\mathsfit{#1}}}
\def\tA{{\tens{A}}}
\def\tB{{\tens{B}}}
\def\tC{{\tens{C}}}
\def\tD{{\tens{D}}}
\def\tE{{\tens{E}}}
\def\tF{{\tens{F}}}
\def\tG{{\tens{G}}}
\def\tH{{\tens{H}}}
\def\tI{{\tens{I}}}
\def\tJ{{\tens{J}}}
\def\tK{{\tens{K}}}
\def\tL{{\tens{L}}}
\def\tM{{\tens{M}}}
\def\tN{{\tens{N}}}
\def\tO{{\tens{O}}}
\def\tP{{\tens{P}}}
\def\tQ{{\tens{Q}}}
\def\tR{{\tens{R}}}
\def\tS{{\tens{S}}}
\def\tT{{\tens{T}}}
\def\tU{{\tens{U}}}
\def\tV{{\tens{V}}}
\def\tW{{\tens{W}}}
\def\tX{{\tens{X}}}
\def\tY{{\tens{Y}}}
\def\tZ{{\tens{Z}}}


% Graph
\def\gA{{\mathcal{A}}}
\def\gB{{\mathcal{B}}}
\def\gC{{\mathcal{C}}}
\def\gD{{\mathcal{D}}}
\def\gE{{\mathcal{E}}}
\def\gF{{\mathcal{F}}}
\def\gG{{\mathcal{G}}}
\def\gH{{\mathcal{H}}}
\def\gI{{\mathcal{I}}}
\def\gJ{{\mathcal{J}}}
\def\gK{{\mathcal{K}}}
\def\gL{{\mathcal{L}}}
\def\gM{{\mathcal{M}}}
\def\gN{{\mathcal{N}}}
\def\gO{{\mathcal{O}}}
\def\gP{{\mathcal{P}}}
\def\gQ{{\mathcal{Q}}}
\def\gR{{\mathcal{R}}}
\def\gS{{\mathcal{S}}}
\def\gT{{\mathcal{T}}}
\def\gU{{\mathcal{U}}}
\def\gV{{\mathcal{V}}}
\def\gW{{\mathcal{W}}}
\def\gX{{\mathcal{X}}}
\def\gY{{\mathcal{Y}}}
\def\gZ{{\mathcal{Z}}}

% Sets
\def\sA{{\mathbb{A}}}
\def\sB{{\mathbb{B}}}
\def\sC{{\mathbb{C}}}
\def\sD{{\mathbb{D}}}
% Don't use a set called E, because this would be the same as our symbol
% for expectation.
\def\sF{{\mathbb{F}}}
\def\sG{{\mathbb{G}}}
\def\sH{{\mathbb{H}}}
\def\sI{{\mathbb{I}}}
\def\sJ{{\mathbb{J}}}
\def\sK{{\mathbb{K}}}
\def\sL{{\mathbb{L}}}
\def\sM{{\mathbb{M}}}
\def\sN{{\mathbb{N}}}
\def\sO{{\mathbb{O}}}
\def\sP{{\mathbb{P}}}
\def\sQ{{\mathbb{Q}}}
\def\sR{{\mathbb{R}}}
\def\sS{{\mathbb{S}}}
\def\sT{{\mathbb{T}}}
\def\sU{{\mathbb{U}}}
\def\sV{{\mathbb{V}}}
\def\sW{{\mathbb{W}}}
\def\sX{{\mathbb{X}}}
\def\sY{{\mathbb{Y}}}
\def\sZ{{\mathbb{Z}}}

% Entries of a matrix
\def\emLambda{{\Lambda}}
\def\emA{{A}}
\def\emB{{B}}
\def\emC{{C}}
\def\emD{{D}}
\def\emE{{E}}
\def\emF{{F}}
\def\emG{{G}}
\def\emH{{H}}
\def\emI{{I}}
\def\emJ{{J}}
\def\emK{{K}}
\def\emL{{L}}
\def\emM{{M}}
\def\emN{{N}}
\def\emO{{O}}
\def\emP{{P}}
\def\emQ{{Q}}
\def\emR{{R}}
\def\emS{{S}}
\def\emT{{T}}
\def\emU{{U}}
\def\emV{{V}}
\def\emW{{W}}
\def\emX{{X}}
\def\emY{{Y}}
\def\emZ{{Z}}
\def\emSigma{{\Sigma}}

% entries of a tensor
% Same font as tensor, without \bm wrapper
\newcommand{\etens}[1]{\mathsfit{#1}}
\def\etLambda{{\etens{\Lambda}}}
\def\etA{{\etens{A}}}
\def\etB{{\etens{B}}}
\def\etC{{\etens{C}}}
\def\etD{{\etens{D}}}
\def\etE{{\etens{E}}}
\def\etF{{\etens{F}}}
\def\etG{{\etens{G}}}
\def\etH{{\etens{H}}}
\def\etI{{\etens{I}}}
\def\etJ{{\etens{J}}}
\def\etK{{\etens{K}}}
\def\etL{{\etens{L}}}
\def\etM{{\etens{M}}}
\def\etN{{\etens{N}}}
\def\etO{{\etens{O}}}
\def\etP{{\etens{P}}}
\def\etQ{{\etens{Q}}}
\def\etR{{\etens{R}}}
\def\etS{{\etens{S}}}
\def\etT{{\etens{T}}}
\def\etU{{\etens{U}}}
\def\etV{{\etens{V}}}
\def\etW{{\etens{W}}}
\def\etX{{\etens{X}}}
\def\etY{{\etens{Y}}}
\def\etZ{{\etens{Z}}}

% The true underlying data generating distribution
\newcommand{\pdata}{p_{\rm{data}}}
% The empirical distribution defined by the training set
\newcommand{\ptrain}{\hat{p}_{\rm{data}}}
\newcommand{\Ptrain}{\hat{P}_{\rm{data}}}
% The model distribution
\newcommand{\pmodel}{p_{\rm{model}}}
\newcommand{\Pmodel}{P_{\rm{model}}}
\newcommand{\ptildemodel}{\tilde{p}_{\rm{model}}}
% Stochastic autoencoder distributions
\newcommand{\pencode}{p_{\rm{encoder}}}
\newcommand{\pdecode}{p_{\rm{decoder}}}
\newcommand{\precons}{p_{\rm{reconstruct}}}

\newcommand{\laplace}{\mathrm{Laplace}} % Laplace distribution

\newcommand{\E}{\mathbb{E}}
\newcommand{\Ls}{\mathcal{L}}
\newcommand{\R}{\mathbb{R}}
\newcommand{\emp}{\tilde{p}}
\newcommand{\lr}{\alpha}
\newcommand{\reg}{\lambda}
\newcommand{\rect}{\mathrm{rectifier}}
\newcommand{\softmax}{\mathrm{softmax}}
\newcommand{\sigmoid}{\sigma}
\newcommand{\softplus}{\zeta}
\newcommand{\KL}{D_{\mathrm{KL}}}
\newcommand{\Var}{\mathrm{Var}}
\newcommand{\standarderror}{\mathrm{SE}}
\newcommand{\Cov}{\mathrm{Cov}}
% Wolfram Mathworld says $L^2$ is for function spaces and $\ell^2$ is for vectors
% But then they seem to use $L^2$ for vectors throughout the site, and so does
% wikipedia.
\newcommand{\normlzero}{L^0}
\newcommand{\normlone}{L^1}
\newcommand{\normltwo}{L^2}
\newcommand{\normlp}{L^p}
\newcommand{\normmax}{L^\infty}

\newcommand{\parents}{Pa} % See usage in notation.tex. Chosen to match Daphne's book.

\DeclareMathOperator*{\argmax}{arg\,max}
\DeclareMathOperator*{\argmin}{arg\,min}

\DeclareMathOperator{\sign}{sign}
\DeclareMathOperator{\Tr}{Tr}
\let\ab\allowbreak

\newcommand{\norm}[2]{\left\| #1 \right\|_{#2}}

\newcommand{\zz}[1]{\textcolor{blue}{ [{\em Zhihui:} #1]}}
\newcommand{\jz}[1]{\textcolor{red}{ [{\em JZ:} #1]}}
% \newcommand{\td}[1]{\textcolor{blue}{ [{\em TD:} #1]}}
\newcommand{\jj}[1]{\textcolor{pink}{ [{\em JJ:} #1]}}

\usepackage{todonotes}
\usepackage{amssymb}
\usepackage{mathtools}
\usepackage{amsthm}

\usepackage[utf8]{inputenc} %
\usepackage[T1]{fontenc}    %
\usepackage{hyperref}       %
\usepackage{url}            %
\usepackage{booktabs}       %
\usepackage{amsfonts}       %
\usepackage{nicefrac}       %
\usepackage{microtype}      %
\usepackage{xcolor}         %


\usepackage{booktabs}
\usepackage{soul}
\usepackage[numbers]{natbib}
\usepackage{comment}

\usepackage[justification=centering]{subfig}
\usepackage{graphicx}
\usepackage{cleveref}
\usepackage{fancyhdr}

\usepackage{enumitem}

\theoremstyle{plain}
\newtheorem{theorem}{Theorem}[section]
\newtheorem{proposition}[theorem]{Proposition}
\newtheorem{lemma}[theorem]{Lemma}
\newtheorem{corollary}[theorem]{Corollary}
\newtheorem{fact}[theorem]{Fact}
\theoremstyle{definition}
\newtheorem{definition}[theorem]{Definition}
\newtheorem{assumption}[theorem]{Assumption}
\theoremstyle{remark}
\newtheorem{remark}[theorem]{Remark}



\newcommand{\fix}{\marginpar{FIX}}
\newcommand{\new}{\marginpar{NEW}}




\newcommand{\anvith}[1]{\textcolor{brown}{2D: #1}}




\begin{document}

\date{}

\title{\Large \bf Gradients Look Alike: Sensitivity is Often Overestimated in DP-SGD}

\author{Anvith Thudi\\
  University of Toronto and Vector Institute\\
  Hengrui Jia \\
  University of Toronto and Vector Institute\\
  Casey Meehan\\
  University of California, San Diego\\
  Ilia Shumailov\\
  University of Oxford\\
  Nicolas Papernot\\
  University of Toronto and Vector Institute\\
}

\maketitle

\begin{abstract}
Differentially private stochastic gradient descent (DP-SGD) is the canonical approach to private deep learning. While the current privacy analysis of DP-SGD is known to be tight in some settings, several empirical results suggest that models trained on common benchmark datasets leak significantly less privacy for many datapoints. Yet, despite past attempts, a rigorous explanation for why this is the case has not been reached. Is it because there exist tighter privacy upper bounds when restricted to these dataset settings, or are our attacks not strong enough for certain datapoints? In this paper, we provide the first per-instance (i.e., ``data-dependent") DP analysis of DP-SGD. Our analysis captures the intuition that points with similar neighbors in the dataset enjoy better data-dependent privacy than outliers. Formally, this is done by modifying the per-step privacy analysis of DP-SGD to introduce a dependence on the distribution of model updates computed from a training dataset. We further develop a new composition theorem to effectively use this new per-step analysis to reason about an entire training run. Put all together, our evaluation shows that this novel DP-SGD analysis allows us to now \emph{formally} show that DP-SGD leaks significantly less privacy for many datapoints (when trained on common benchmarks) than the current data-independent guarantee. This implies privacy attacks will necessarily fail against many datapoints if the adversary does not have sufficient control over the possible training datasets. 
\end{abstract}


\IEEEraisesectionheading{\section{Introduction}}

\IEEEPARstart{V}{ision} system is studied in orthogonal disciplines spanning from neurophysiology and psychophysics to computer science all with uniform objective: understand the vision system and develop it into an integrated theory of vision. In general, vision or visual perception is the ability of information acquisition from environment, and it's interpretation. According to Gestalt theory, visual elements are perceived as patterns of wholes rather than the sum of constituent parts~\cite{koffka2013principles}. The Gestalt theory through \textit{emergence}, \textit{invariance}, \textit{multistability}, and \textit{reification} properties (aka Gestalt principles), describes how vision recognizes an object as a \textit{whole} from constituent parts. There is an increasing interested to model the cognitive aptitude of visual perception; however, the process is challenging. In the following, a challenge (as an example) per object and motion perception is discussed. 



\subsection{Why do things look as they do?}
In addition to Gestalt principles, an object is characterized with its spatial parameters and material properties. Despite of the novel approaches proposed for material recognition (e.g.,~\cite{sharan2013recognizing}), objects tend to get the attention. Leveraging on an object's spatial properties, material, illumination, and background; the mapping from real world 3D patterns (distal stimulus) to 2D patterns onto retina (proximal stimulus) is many-to-one non-uniquely-invertible mapping~\cite{dicarlo2007untangling,horn1986robot}. There have been novel biology-driven studies for constructing computational models to emulate anatomy and physiology of the brain for real world object recognition (e.g.,~\cite{lowe2004distinctive,serre2007robust,zhang2006svm}), and some studies lead to impressive accuracy. For instance, testing such computational models on gold standard controlled shape sets such as Caltech101 and Caltech256, some methods resulted $<$60\% true-positives~\cite{zhang2006svm,lazebnik2006beyond,mutch2006multiclass,wang2006using}. However, Pinto et al.~\cite{pinto2008real} raised a caution against the pervasiveness of such shape sets by highlighting the unsystematic variations in objects features such as spatial aspects, both between and within object categories. For instance, using a V1-like model (a neuroscientist's null model) with two categories of systematically variant objects, a rapid derogate of performance to 50\% (chance level) is observed~\cite{zhang2006svm}. This observation accentuates the challenges that the infinite number of 2D shapes casted on retina from 3D objects introduces to object recognition. 

Material recognition of an object requires in-depth features to be determined. A mineralogist may describe the luster (i.e., optical quality of the surface) with a vocabulary like greasy, pearly, vitreous, resinous or submetallic; he may describe rocks and minerals with their typical forms such as acicular, dendritic, porous, nodular, or oolitic. We perceive materials from early age even though many of us lack such a rich visual vocabulary as formalized as the mineralogists~\cite{adelson2001seeing}. However, methodizing material perception can be far from trivial. For instance, consider a chrome sphere with every pixel having a correspondence in the environment; hence, the material of the sphere is hidden and shall be inferred implicitly~\cite{shafer2000color,adelson2001seeing}. Therefore, considering object material, object recognition requires surface reflectance, various light sources, and observer's point-of-view to be taken into consideration.


\subsection{What went where?}
Motion is an important aspect in interpreting the interaction with subjects, making the visual perception of movement a critical cognitive ability that helps us with complex tasks such as discriminating moving objects from background, or depth perception by motion parallax. Cognitive susceptibility enables the inference of 2D/3D motion from a sequence of 2D shapes (e.g., movies~\cite{niyogi1994analyzing,little1998recognizing,hayfron2003automatic}), or from a single image frame (e.g., the pose of an athlete runner~\cite{wang2013learning,ramanan2006learning}). However, its challenging to model the susceptibility because of many-to-one relation between distal and proximal stimulus, which makes the local measurements of proximal stimulus inadequate to reason the proper global interpretation. One of the various challenges is called \textit{motion correspondence problem}~\cite{attneave1974apparent,ullman1979interpretation,ramachandran1986perception,dawson1991and}, which refers to recognition of any individual component of proximal stimulus in frame-1 and another component in frame-2 as constituting different glimpses of the same moving component. If one-to-one mapping is intended, $n!$ correspondence matches between $n$ components of two frames exist, which is increased to $2^n$  for one-to-any mappings. To address the challenge, Ullman~\cite{ullman1979interpretation} proposed a method based on nearest neighbor principle, and Dawson~\cite{dawson1991and} introduced an auto associative network model. Dawson's network model~\cite{dawson1991and} iteratively modifies the activation pattern of local measurements to achieve a stable global interpretation. In general, his model applies three constraints as it follows
\begin{inlinelist}
	\item \textit{nearest neighbor principle} (shorter motion correspondence matches are assigned lower costs)
	\item \textit{relative velocity principle} (differences between two motion correspondence matches)
	\item \textit{element integrity principle} (physical coherence of surfaces)
\end{inlinelist}.
According to experimental evaluations (e.g.,~\cite{ullman1979interpretation,ramachandran1986perception,cutting1982minimum}), these three constraints are the aspects of how human visual system solves the motion correspondence problem. Eom et al.~\cite{eom2012heuristic} tackled the motion correspondence problem by considering the relative velocity and the element integrity principles. They studied one-to-any mapping between elements of corresponding fuzzy clusters of two consecutive frames. They have obtained a ranked list of all possible mappings by performing a state-space search. 



\subsection{How a stimuli is recognized in the environment?}

Human subjects are often able to recognize a 3D object from its 2D projections in different orientations~\cite{bartoshuk1960mental}. A common hypothesis for this \textit{spatial ability} is that, an object is represented in memory in its canonical orientation, and a \textit{mental rotation} transformation is applied on the input image, and the transformed image is compared with the object in its canonical orientation~\cite{bartoshuk1960mental}. The time to determine whether two projections portray the same 3D object
\begin{inlinelist}
	\item increase linearly with respect to the angular disparity~\cite{bartoshuk1960mental,cooperau1973time,cooper1976demonstration}
	\item is independent from the complexity of the 3D object~\cite{cooper1973chronometric}
\end{inlinelist}.
Shepard and Metzler~\cite{shepard1971mental} interpreted this finding as it follows: \textit{human subjects mentally rotate one portray at a constant speed until it is aligned with the other portray.}



\subsection{State of the Art}

The linear mapping transformation determination between two objects is generalized as determining optimal linear transformation matrix for a set of observed vectors, which is first proposed by Grace Wahba in 1965~\cite{wahba1965least} as it follows. 
\textit{Given two sets of $n$ points $\{v_1, v_2, \dots v_n\}$, and $\{v_1^*, v_2^* \dots v_n^*\}$, where $n \geq 2$, find the rotation matrix $M$ (i.e., the orthogonal matrix with determinant +1) which brings the first set into the best least squares coincidence with the second. That is, find $M$ matrix which minimizes}
\begin{equation}
	\sum_{j=1}^{n} \vert v_j^* - Mv_j \vert^2
\end{equation}

Multiple solutions for the \textit{Wahba's problem} have been published, such as Paul Davenport's q-method. Some notable algorithms after Davenport's q-method were published; of that QUaternion ESTimator (QU\-EST)~\cite{shuster2012three}, Fast Optimal Attitude Matrix \-(FOAM)~\cite{markley1993attitude} and Slower Optimal Matrix Algorithm (SOMA)~\cite{markley1993attitude}, and singular value decomposition (SVD) based algorithms, such as Markley’s SVD-based method~\cite{markley1988attitude}. 

In statistical shape analysis, the linear mapping transformation determination challenge is studied as Procrustes problem. Procrustes analysis finds a transformation matrix that maps two input shapes closest possible on each other. Solutions for Procrustes problem are reviewed in~\cite{gower2004procrustes,viklands2006algorithms}. For orthogonal Procrustes problem, Wolfgang Kabsch proposed a SVD-based method~\cite{kabsch1976solution} by minimizing the root mean squared deviation of two input sets when the determinant of rotation matrix is $1$. In addition to Kabsch’s partial Procrustes superimposition (covers translation and rotation), other full Procrustes superimpositions (covers translation, uniform scaling, rotation/reflection) have been proposed~\cite{gower2004procrustes,viklands2006algorithms}. The determination of optimal linear mapping transformation matrix using different approaches of Procrustes analysis has wide range of applications, spanning from forging human hand mimics in anthropomorphic robotic hand~\cite{xu2012design}, to the assessment of two-dimensional perimeter spread models such as fire~\cite{duff2012procrustes}, and the analysis of MRI scans in brain morphology studies~\cite{martin2013correlation}.

\subsection{Our Contribution}

The present study methodizes the aforementioned mentioned cognitive susceptibilities into a cognitive-driven linear mapping transformation determination algorithm. The method leverages on mental rotation cognitive stages~\cite{johnson1990speed} which are defined as it follows
\begin{inlinelist}
	\item a mental image of the object is created
	\item object is mentally rotated until a comparison is made
	\item objects are assessed whether they are the same
	\item the decision is reported
\end{inlinelist}.
Accordingly, the proposed method creates hierarchical abstractions of shapes~\cite{greene2009briefest} with increasing level of details~\cite{konkle2010scene}. The abstractions are presented in a vector space. A graph of linear transformations is created by circular-shift permutations (i.e., rotation superimposition) of vectors. The graph is then hierarchically traversed for closest mapping linear transformation determination. 

Despite of numerous novel algorithms to calculate linear mapping transformation, such as those proposed for Procrustes analysis, the novelty of the presented method is being a cognitive-driven approach. This method augments promising discoveries on motion/object perception into a linear mapping transformation determination algorithm.



\section{Background and Related Work}~\label{sec:background}
%This section presents the background on MDE, ML, and MDE for systems with ML components. We further present the related work on the existing secondary and relevant studies.
\subsection{Model-driven Engineering}~\label{subsec:MDEBackground}
%The word \textit{model} originates from the Latin word \textit{modulus}, which means a measure, pattern, or example to follow~\cite{ludewig2003models}. 
%While modeling is relatively new to software engineering, it has been successfully applied for a long time in several traditional engineering domains~\cite{selic2012will,bucchiarone2020grand}. 
Model-driven Engineering (MDE) is a software development methodology that relies on models as the primary artifacts that drive the development process~\cite{ciccozzi2019execution, almonte2021recommender,hutchinson2011model}. This differs from traditional software development processes such as waterfall and agile, where the focus is on development phases like requirements engineering, design, and implementation, and models are only used as auxiliary artifacts to support these activities and serve as documentation~\cite{ciccozzi2019execution}. 
%In contrast to traditional software engineering using waterfall or agile methodology, where the focus is on the different phases of development, e.g., requirements engineering, design, implementation, and quality assurance, and the models are used to aid in requirements analysis or design, in MDE models are the primary artifact. 
The focus of MDE is on the continual refinement and transformation of models, beginning with computation-independent models (CIMs), to platform-independent models (PIMs) and then platform-specific models (PSMs)~\cite{brambilla2017model}. Finally, these models are transformed into code, documentation, configurations, and tests for the software system.

MDE relies on two key aspects: abstraction and automation~\cite{mohagheghi2009mde}. Models are abstractions of complex entities; they hide unwanted information so modelers can easily focus on areas of interest~\cite{schmidt2006model, brambilla2017model}. 
%Currently, MDE is the state-of-the-art in software abstraction~\cite{hutchinson2011model} by reducing complexity and offering a more intuitive and natural way to define software compared to programming languages~\cite{ciccozzi2019execution}. 
In MDE, models are automatically transformed into artifacts such as code, documentation, and other models to achieve various goals such as merging, translation, refinement, refactoring, or alignment~\cite{brambilla2017model}. These transformations help reduce developers' manual effort and production time by generating executable artifacts -- leading to improved software quality, reduced complexity, and decreased development time and effort~\cite{kelly2008domain}. There are two types of transformations in MDE: 1) Model-to-Text (M2T) transformations, for a given input model a M2T transformation produces a textual artifact such as code or documentation as output; and ) Model-to-model (M2M) transformations, for a given input model an M2M transformation produces a different kind of model, for example translating a model from one language to another~\cite{brambilla2017model}.

A model is created in a modeling language, conforming to a meta-model that defines the syntax and semantics of that language. There are two types of modeling languages: general-purpose languages (GPL) and domain-specific languages (DSL). GPLs are intended for modeling generic concepts applicable to multiple domains; some examples include the Unified Modeling Language (UML)~\cite{eriksson2003uml}, Petri-nets~\cite{peterson1977petri} and finite state machines~\cite{wagner2006modeling}. On the other hand, a DSL has modeling concepts tailored to a specific domain or context, like SysML for embedded systems, HTML for web page development, and SQL for database queries~\cite{brambilla2017model}.

While exploring the literature, one encounters terms similar to MDE: examples include model-driven architecture (MDA), model-driven development (MDD), and model-based engineering (MBE). MDA is an architectural standard~\cite{mda} developed by the Object Management Group (OMG) \cite{omg} for MDD. MDD refers to automatically generating artifacts from models, whereas MDE has a broader scope and includes analysis, validation~\cite{almonte2021recommender}, interoperability of artifacts and reverse engineering \cite{brambilla2017model}. MBE is a lighter version of MDE, where models are not necessarily the central focus of the engineering process; however, they provide critical support~\cite{brambilla2017model}. This SLR primarily focuses on MDE.

\subsection{Machine Learning}
Machine Learning (ML) is a branch of Artificial Intelligence (AI) that enables machines to learn patterns from data without being explicitly programmed~\cite{samuel1959machine}. ML algorithms are fed with existing data to \textit{train} them and produce an ML model. This trained ML model then has the capability to \textit{infer}, i.e., predict outcomes for new data inputs or also commonly known as \emph{ML model inference}~\cite{mueller2021machine}. For example, an ML model trained on stock prices for a company till September 2023 can predict stock prices in the following months. ML is preferable when solving problems that would require very complex and difficult-to-maintain traditional algorithms~\cite{geron2022hands}. Since ML algorithms can learn autonomously, they reduce complexity and facilitate easier maintenance~\cite{geron2022hands}. This ability of ML to minimise complexity, learn from changing data, and make future predictions is immensely valuable for businesses~\cite{lee2020machine}. According to a recent survey~\cite{rackspace2023report}, organizations report that applying ML increases employee efficiency by 20\%, innovation by 17\%, and lowers costs by 16\% -- leading to increased adoption of ML in practical settings~\cite{rackspace2023report}.

ML can further be divided into three broad categories: supervised learning, unsupervised learning, and reinforcement learning. The most suitable ML approach depends on the specific problem and data.
%
Supervised learning is when an ML algorithm is trained on a labeled dataset that has labels to define the meaning of data~\cite{mueller2021machine}. For example, a dataset with images labeled as ``cat'' or ``not cat'' images. Supervised learning algorithms learn to make classifications or predictions by learning patterns and relationships in labeled data~\cite{lee2020machine,mueller2021machine}. When the labels are discrete, this is known as \textit{classification} and when labels are continuous, this is known as \textit{regression}~\cite{mueller2021machine}. Once the algorithm is trained, the performance is evaluated on unseen or test data. Some popular supervised learning algorithms include linear regression, decision trees, naive Bayes classifier, support vector machines (SVM), random forest, and artificial neural networks (ANNs)~\cite{lee2020machine}. Supervised model applications include fraud detection and recommender systems~\cite{mueller2021machine}. 

Unsupervised learning is when an ML algorithm is trained on an unlabeled dataset with few or no labels to define the meaning of data~\cite{mueller2021machine,lee2020machine}. Unsupervised learning algorithms attempt to understand hidden patterns in data and group similar data together creating a classification of the data~\cite{mueller2021machine}. Unsupervised learning works without any guidance, hence it is most suitable for large volumes of data when classifications are unknown and data cannot be labeled~\cite{mueller2021machine}. Evaluating the performance of such algorithms can be challenging due to the lack of ground truth. Some popular unsupervised techniques include clustering, k-means, principal component analysis, and association rules~\cite{lee2020machine}. Applications of unsupervised models include customer segmentation and clustering user reviews~\cite{mueller2021machine}.

 Reinforcement learning is when an ML algorithm receives feedback on actions to guide the behavior toward an optimal outcome~\cite{mueller2021machine, lee2020machine}. Reinforcement learning algorithms are not trained with datasets; instead, they learn from trial and error in a simulated environment or a real-world environment~\cite{mueller2021machine}. Desired behaviors are rewarded and reinforcement learning algorithms attempt to maximize rewards through successful decisions~\cite{lee2020machine,mueller2021machine}. These algorithms are most suitable when sequential decision-making is required, interaction with an environment is possible and feedback is available. %However, reinforcement learning can be expensive since the algorithms require a large number of interactions with the environment to learn effectively. 
 Some popular reinforcement learning algorithms are Q-learning, temporal difference learning, hierarchal reinforcement learning, and policy gradient~\cite{lee2020machine}. Applications of reinforcement learning include robotics, self-driving cars, and game playing~\cite{lee2020machine}.
 
\subsection{Model-driven Engineering for Machine Learning (MDE4ML)}
%Models are a significant element of both MDE and ML. In MDE, models describe software systems in all phases of their life-cycle: requirements, design, implementation, testing and evolution~\cite{moin2022model}. ML models are mathematical models that learn patterns in data to make predictions~\cite{moin2022model}. 
Developing and managing systems with ML models and components is challenging. 
Some aspects of this complexity are immature requirements specification~\cite{kuwajima2020engineering, ahmad2023requirements}, constantly evolving data~\cite{baumann2022dynamic}, lack of ML domain knowledge~\cite{yohannis2022towards}, integration with traditional software \cite{atouani2021artifact}, responsible use of ML~\cite{yohannis2022towards}, and deployment and maintenance of ML models~\cite{kourouklidis2021model, langford2021modalas}. 

These complexities introduce several challenges. For example, Nils Baumann et al.~\cite{baumann2022dynamic} describe how challenging it is to handle changing datasets; ML engineers have to manually merge new and old datasets and re-train the entire ML model;  Benjamin Jahi et al. \cite{jahic2023semkis} point out how challenging it is to describe the dataset and neural network requirements to satisfy customer expectations;  Benjamin Benni et al. \cite{benni2019devops} state how the development of a correct ML pipeline is a highly demanding task, data scientists must have knowledge and experience to go through numerous data pre-processing and ML models to select the best one; and Kaan Koseler et al. \cite{koseler2019realization} mention the difficulties developers face when attempting to use ML techniques with big data, developers need to acquire knowledge of the problem space, domain and ML concepts. There is a need for solutions to efficiently and effectively address these challenges~\cite{raedler2023model}.
%All these challenges point towards the need for a technique that can efficiently and effectively address them.

A synergy between MDE and ML development exists, where software models are leveraged to drive the development and management of ML components~\cite{safdar2022modlf, yohannis2022towards, kourouklidis2021model}. This should not to be confused with AI or ML for MDE (AI4MDE), where intelligent agents and recommenders support users in modeling and related activities \cite{almonte2021recommender, gil2021artificial, boubekeur2020towards, saini2019teaching}. The application of MDE for ML-based systems (MDE4ML) offers many potential benefits to developers, such as reduced complexity~\cite{kourouklidis2021model, bucchiarone2020grand}, development effort, and time~\cite{yohannis2022towards,gatto2019modeling}. Domain experts, software engineers and ML novices can also take advantage of ML through the abstraction and automation of MDE \cite{shi2022feature,moin2022supporting, bucchiarone2020grand}. Additionally, MDE can also improve the quality of the ML-based system through easier maintainability, scalability~\cite{selic2003pragmatics}, reusability, and interoperability~\cite{brambilla2017model}.

\subsection{Key MDE4ML Related Work}
MDE4ML has received growing attention from researchers in recent years. We found six relevant secondary studies comprising SLRs, scoping reviews, and surveys. In their SLR \cite{raedler2023model}, the authors identify 15 primary studies on MDE for AI and analyze them with respect to MDE practices for the development of AI systems and the stages of AI development aligned with CRoss Industry Standard Process for Data Mining (CRISP-DM) \cite{wirth2000crisp} methodology. However, this study only considers a small subset of studies and performs a shallow analysis with no details about goals, end-users, types of models, implemented tools, and evaluation. A second SLR \cite{zafar2017systematic} reviews 24 papers on MDE for ML in the context of Big data analytics. This study has a narrower scope compared to ours and provides only a brief overview of the models, approaches, tools, and frameworks in the studies. In a third SLR \cite{li2022can}, 31 studies on no/low code platforms for ML applications are reviewed. This study is limited to no/low code approaches and therefore misses out on many other MDE for ML studies. A scoping review is presented in \cite{mardani2023model} on MDE for ML in IoT applications. The study examines 68 studies in depth; however, the review focuses more on MDE for IoT applications and only four of the selected studies apply ML techniques. A preliminary survey on DSLs for ML in Big data is presented in \cite{portugal2016preliminary}, with an extended version in \cite{portugal2016survey}. These surveys do not follow a systematic review process, include studies only for big data, and briefly highlight the DSLs and frameworks in the studies. From the analysis of existing literature, we found that the available secondary studies consist of limited subsets of papers on MDE for ML, lack analysis of key areas like goals, end-users, ML aspects, MDE approach details, evaluation methods, and limitations, and often do not follow a systematic and rigorous review process. Therefore, we aim to address these gaps in this SLR.


\section{A Per-Instance Analysis of DP-SGD}
\label{sec:analysis}





We now present our new analysis of DP-SGD which removes the data-independent nature of the per-step and composition analyses currently used for DP-SGD. The impact of this new analysis is presented in Section~\ref{sec:main_body_emp_results}, where we show that many datapoints have much better privacy than suggested by the current analysis of DP-SGD, explaining the failure of many privacy attacks in practice.

The technical contributions that led to this are two-fold. At the per-step level, we generalize the notion of sensitivity to what we term \emph{sensitivity distributions}; given two datasets, sensitivity distributions capture how similar the updates between mini-batches from either dataset are. At the composition step, we generalize RDP composition to do accounting by the ``expected" intermediate privacy losses during training as opposed to the largest possible intermediate privacy losses. Together, we can now study the data-dependent behaviour of DP-SGD.





\subsection{Sensitivity Distribution Generalize the $(\epsilon,\delta)$-DP Analysis}
\label{ssec:eps_delta_case}




We first turn to $(\epsilon,\delta)$-DP, which is not used to analyze DP-SGD for composition reasons, but allows for simpler expressions to demonstrate the improvements afforded by particular data-dependent random variables we call \textit{sensitivity distributions}. In particular, in this section we will first consider the classical data-independent $(\epsilon,\delta)$-DP analysis of the sampled Gaussian mechanism $M$ and show how one can generalize this analysis and obtain tighter per-instance $(\epsilon,\delta)$-DP guarantees.


Recall that for an update rule $U$, the Gaussian mechanism is defined as $A(X) = U(X) + N(0,\sigma)$. The sampled Gaussian mechanism is then defined as $M(X) = A(\mathbf{X_B})$ where $\mathbf{X_B}$ is a mini-batch constructed from a dataset $X$ by sampling each datapoint $x \in X$ independently with probability $\mathbb{P}_{x}(1)$ (unless otherwise stated we think of $X_B$, not bold-face, as a specific mini-batch). Note, one assumes the sampling probability $\mathbb{P}_{x}(1)$ is only a function of $x$ and not the full dataset $X$, e.g., some fixed constant. The classical data-independent $(\epsilon,\delta)$-DP analysis of the sampled Gaussian mechanism follows two steps. First, we derive the guarantee for just the Gaussian mechanism. To do so, one first assumes a data-independent sensitivity bound $C_U$ on $U$: for all $X,X' = X \cup \{x^*\}$ we have $||U(X) - U(X')||_{2} \leq C_{U}$. This can be achieved by clipping the output values of $U$ to have a small norm. With this constant $C_U$ one has that the Gaussian mechanism $A$ gives the $(\epsilon,\delta)$-DP guarantee $\epsilon = C_{\delta,\sigma} C_U$ for some constant $C_{\delta,\sigma}$ depending on $\delta$ and $\sigma$ where $\sigma$ is the standard deviation of the added Gaussian noise~\footnote{For example, one can take $C_{\delta, \sigma} = \frac{\sqrt{2 \ln (1.25/\delta)}}{\sigma}$~\citep{dwork2014algorithmic}.}. To then analyze the sampled Gaussian mechanism one would incorporate the privacy gain from not sampling $x^*$ sometimes~\citep{beimel2014bounds}\citep{kasiviswanathan2011can} to get the privacy guarantees of $M$ %
as $(\epsilon',\delta')$-DP where $\epsilon' = \ln( \mathbb{P}_{x^*}(1) e^{C_{\delta,\sigma}~C_U} + \mathbb{P}_{x^*}(0))$ and $\delta' = \mathbb{P}_{x^*}(1) \delta$. Here $\mathbb{P}_{x^*}(0)=1-\mathbb{P}_{x^*}(1)$, and this gain in privacy by sometimes not using the datapoint is called privacy amplification by sampling.






Towards tightening this analysis into a per-instance analysis, let $$\Delta_{U,x^*}(X_B) \coloneqq ||U(X_B) - U(X_B \cup \{x^*\})||_2$$
then $\Delta_{U,x^*}(\mathbf{X_B})$ is a data-dependent random variable which we will call a \emph{sensitivity distribution}: it captures the change in the distribution of mini-batches updates caused by adding a point $x^*$ to the mini-batch. The classical data-independent analysis only (implicitly) uses sensitivity distributions via the data-independent bound $|\Delta_{U,x^*}(X_B)| \leq C_{U}~\forall X_B$. Instead, we will show how to directly use the $L_p$ norms $||\Delta_{U,x^*}(\mathbf{X_B})||_{p} = (\mathbb{E}_{X_B} (\Delta_{U,x^*}(X_B)^p))^{1/p}$ (or generally the $L_p$ norm of some monotonic transformation of $\Delta_{U,x^*}(\mathbf{X_B})$) to obtain tighter per-instance privacy guarantees. Furthermore, when using $p < \infty$, this analysis will be able to translate the phenomenon that many mini-batches produce similar updates into better privacy guarantees (as the sensitivity distribution concentrates at smaller values and hence has smaller $p$-norms). To emphasize this ability, past work that studied sampling relied mainly on the intuition that by sampling a datapoint with low probability, we have any given step often does not leak privacy for that point as it was not used. This translates to better privacy guarantes. By using the $L_p$ norms of sensitivity distributions with $p< \infty$ we make an additional observation, which is that if many of the other mini-batches produce the same update, then effectively we have an even lower probability of an attacker observing a noticeable shift due solely to that point.  %


In particular, recall that to prove per-instance $(\epsilon,\delta)$-DP for a pair of datasets $X,X'= X \cup \{x^*\}$ we need to bound $\mathbb{P}(M(X') \in S) \leq e^{\epsilon} \mathbb{P}(M(X) \in S) + \delta$ and $\mathbb{P}(M(X) \in S) \leq e^{\epsilon} \mathbb{P}(M(X') \in S) + \delta$. As a proof-of-concept on the role of sensitivity distributions, we present an analysis for the first inequality in Corollary~\ref{cor:eps_delta_sens} \footnote{We will later turn to R\'enyi-DP which provides both inequalities.}. Inspecting Corollary~\ref{cor:eps_delta_sens}, we see that it approximately follows the formula given by the classical analysis except the role of $C_U$ is replaced with a dependency on how concentrated $\Delta_{U,x^*}(X_B)$ is at small values (the $L_p$ norm of an exponential applied to $\Delta_{U,x^*}(X_B)$). When enough mini-batches provide updates more similar than the upper-bound $C_U$, the per-instance guarantee of Corollary~\ref{cor:eps_delta_sens} will significantly beat the classical data-independent analysis, as demonstrated for MNIST and CIFAR10 in Appendix~\ref{ssec:eps_delta_experiments}.




\begin{corollary}
\label{cor:eps_delta_sens}
For $p \in (1,\infty)$, let $a_p = \mathbb{P}_{x^*}(1) (\mathbb{E}_{x_{B}}(e^{C_{\delta,\sigma} \Delta_{U,x^*}(X_B)p}))^{1/p}$, $\epsilon' = \ln(a_p^{\frac{1}{1-1/p}}\delta'^{\frac{-1}{p-1}} + \mathbb{P}_{x^*}(0)) $ and $\delta'' = \mathbb{P}_{x^*}(1)\delta + \delta'$. Then, for $X' = X \cup \{x^*\}$ $$\mathbb{P}(M(X') \in S) \leq e^{\epsilon'} \mathbb{P}(M(X) \in S) + \delta''$$

\end{corollary}


\emph{Proof Sketch:} The proof of Corollary~\ref{cor:eps_delta_sens} follows two stages. First by expanding mini-batch sampling and applying Holder's inequality, we can show

\begin{multline}
    \mathbb{P}(M(X') \in S) \leq \mathbb{P}_{x^*}(1) \mathbb{E}_{X_B}(e^{C_{\delta,\sigma} \Delta_{U,x^*}(X_B)p})^{1/p} \mathbb{P}(M(X) \in S)^{1-1/p} \\ + \mathbb{P}_{x^*}(1)\delta + \mathbb{P}_{x^*}(0) \mathbb{P}(M(X) \in S)
\end{multline}

This is stated as Lemma~\ref{lem:holder_approach}. One then follows the proof strategy of Proposition 3 in \citet{mironov2017renyi} to convert an inequality bounding $\mathbb{P}(M(X') \in S)$ with a power of $\mathbb{P}(M(X) \in S)$ into an $(\epsilon,\delta)$-DP inequality. The full proof of Corollary~\ref{cor:eps_delta_sens} is in Appendix~\ref{proof:eps_delta_sens}.





\subsection{Per-Instance R\'enyi-DP Analysis for DP-SGD}

With now an understanding of the power of incorporating $L_p$ norms of sensitivity distributions (upto some transformations) into DP analyses, we turn to analyzing the R\'enyi-DP guarantees of DP-SGD. R\'enyi-DP is more suited to compose the guarantees of each step of DP-SGD to obtain the guarantees for an entire training run. We first present per-step analyses for the sampled Gaussian mechanism, and then a new composition theorem to reason about the entire training run. We then discuss how to analyze DP-SGD for general update rules, i.e., not just the sum of gradients.

Our per-step analyses will focus on integer values of $\alpha$ for R\'enyi-DP. This is for simplicity, as R\'enyi divergences $D_{\alpha}(P||Q) \coloneqq \frac{1}{\alpha -1} \ln \mathbb{E}_{x \sim Q} (\frac{P}{Q})^{\alpha}$ are increasing in their order $\alpha$, hence we can bound the guarantee for any $\alpha$ by the guarantee for $\lceil \alpha \rceil$. In terms of notation, we will use ${X_B}^{\tilde \alpha} = ({X_B}^1,\cdots,{X_B}^{\alpha})$ to denote $\alpha$ mini-batches from $X$ (sampled independently if random). Analogously we use ${X'_B}^{\tilde \alpha}$ and $X'_B$ for $X'$.


\subsubsection{Per-Instance R\'enyi DP for the Sum Update Rule}
\label{ssec:sum_update}

In Section~\ref{ssec:eps_delta_case} we introduced the sensitivity distribution $\Delta_{U,x^*}(\mathbf{X_B}) = ||U(\mathbf{X_B}) - U(\mathbf{X_B \cup \{x^*\}})||_2$ and showed how directly leveraging its $L_p$ norms gives better per-instance DP analysis. In particular, how $p < \infty$ allows one to take advantage of expected sensitivity over mini-batches. However, for update rules of the form $U(X_B) = \sum_{x_i \in X_B} g(x_i)$ (i.e., the sum update rule typically used in DP-SGD) we have $\Delta_{U,x^*}(\mathbf{X_B})$ is always a constant: $\Delta_{U,x^*}(\mathbf{X_B}) = ||g(x^*)||_2$. Hence an analysis of the sampled Gaussian mechanism that used $\Delta_{U,x^*} \coloneqq \sup_{X_B \sim X} \Delta_{U,x^*}(X_B)$ would effectively capture all $L_p$ norms of the sensitivity distribution $\Delta_{U,x^*}(X_B)$ for the sum update rule. We state such a per-instance version of the classical RDP analysis of the sampled Gaussian mechanism below.



\begin{theorem}
\label{thm:easy_renyi_dp}
    For integer $\alpha > 1$, the sampled Gaussian mechanism with noise $\sigma$ and sampling probability $\mathbb{P}_{x^*}(1)$ for $x^*$ is $(\alpha,\epsilon)$-R\'enyi DP for:

    
    $$\epsilon = \frac{1}{\alpha -1} \ln(\sum_{k=0}^{\alpha} {\alpha \choose k} (1 - \mathbb{P}_{x^*}(1))^{\alpha -k} \mathbb{P}_{x^*}(1)^k \exp(\frac{\Delta_{U,x^*}^2(k^2 - k)}{2 \sigma^2}))$$

        
\end{theorem}

Note that some key variables in Theorem~\ref{thm:easy_renyi_dp} are the sampling rate $\mathbb{P}_{x^*}(1)$ (increasing it typically increases the bound), the standard deviation of noise $\sigma$ (increasing it typically decreases the bound), and the sensitivity upper-bound over minibatches $\Delta_{U,x^*}$ (increasing it typically increases the bound). The proof strategy is analogous to \citet{mironov2019r} and replaces their sensitivity upper-bound with the per-instance bound $\Delta_{U,x^*}$ on the mini-batches. 


\begin{proof}
    Following the calculation of Theorem 4 in \citet{mironov2019r} we have 

    
    $$D(M(X')| M(X)) \\ \leq D_{\alpha}((1-\mathbb{P}_{x^*}(1))N(0,\sigma^2) + \mathbb{P}_{x^*}(1)N(\Delta_{U,x^*},\sigma^2)|| N(0,\sigma^2))$$
    
    where instead of using $||U(X_B) - U(X_B \cup x^*)||_2 \leq 1$ for $X_B$ batches from $X$ as in the proof of the theorem we used $||U(X_B) - U(X_B \cup x^*)||_2 \leq \Delta_{U,x^*}$ by the definition of $\Delta_{U,x^*}$. Similarly, we have $D(M(X)| M(X')) \leq D_{\alpha}( N(0,\sigma^2) || (1-\mathbb{P}_{x^*}(1))N(0,\sigma^2) + \mathbb{P}_{x^*}(1)N(\Delta_{U,x^*},\sigma^2))$.

    Analogous to Corollary 7 in \citet{mironov2019r} we have

    \begin{multline}
        D_{\alpha}((1-\mathbb{P}_{x^*}(1))N(0,\sigma^2) + \mathbb{P}_{x^*}(1)N(\Delta_{U,x^*},\sigma^2)|| N(0,\sigma^2)) \\ \geq D_{\alpha}( N(0,\sigma^2) || (1-\mathbb{P}_{x^*}(1))N(0,\sigma^2) + \mathbb{P}_{x^*}(1)N(\Delta_{U,x^*},\sigma^2))
    \end{multline}
    
    where instead of using $\nu(x) = 1 -x$ we use $\nu(x) = \Delta_{U,x^*} - x $ which still satisfies $\nu = \nu^{-1}$

    Now one follows the integer $\alpha$ calculations in Section 3.3 of \citet{mironov2019r}, to conclude our theorem statement. The only change is that instead of computing $\mathbb{E}_{z \sim N(0,\sigma^2)}(\frac{N(1,\sigma^2)}{N(0,\sigma^2)})^k$ one computes $\mathbb{E}_{z \sim N(0,\sigma^2)}(\frac{N(\Delta_{U,x^*},\sigma^2)}{N(0,\sigma^2)})^k$ and following analogous calculation get $\leq \exp(\frac{\Delta_{U,x^*}^2(k^2 - k)}{2 \sigma^2})$.

    
\end{proof}






\subsubsection{A Generalized R\'enyi-DP Composition}
\label{ssec:comp}




With now an analysis for the per-step guarantees from DP-SGD (which as currently implemented uses the sum update rule), we now resolve how to obtain a per-instance RDP bound for a full training run with DP-SGD without the limitations of past composition theorem (see Section~\ref{ssec:back_full_comp} for a discussion on past composition bounds). In particular, we provide a composition theorem that bounds the overall per-instance privacy leakage by the ``expected" per-instance privacy guarantee at each step when training on a given dataset. This is presented in Theorem~\ref{thm:better_composition}.


More technically, we once again generalize the classical analysis to look at arbitrary $L_p$ norms, but now for the composition step. The classical R\'enyi DP composition theorem implicity uses the $L_\infty$ norm of the distribution of per-step guarantees at each step (coming from the distribution of possible models at each step as training is random), and Theorem~\ref{thm:better_composition} generalizes this to arbitrary $L_p$ norms of the exponential of the per-step guarantees (with some constants to scale). By using $L_p$ norms with $p < \infty$ we take advantage of cases where many models have better privacy guarantees than the worst model. 




\begin{theorem}
\label{thm:better_composition}

    Let $p \in (1,\infty)$ and consider a sequence of functions $X_1(x_1),$ $X_2(x_1,x_2),\cdots X_n(x_{n-1},x_n)$ where $X_{i}$ is a density function in the second argument for any fixed value of the first argument, except $X_1$ which is a density function in $x_1$. Consider an analogous sequence $Y_1(x_1),\cdots, Y_n(x_{n-1}, x_n)$. Then letting $X = \prod_{j=1}^{n} X_j$ be the density function for a sequence $x_1,\cdots,x_n$ generated according to the Markov chain defined by $X_i$, and similarly $Y$, we have 
    
    
    \begin{multline}
        D_{\alpha}(X || Y)  \leq \frac{1}{\alpha -1} (\sum_{i=0}^{n-2} \frac{(p-1)^i}{p^{i+1}} \ln (\mathbb{E}_{X_1,\cdots X_{n-(i+1)}}  (e^{(g_p^{i}(\alpha) -1)D_{g_p^{i}(\alpha)}(X_{n-i}|| Y_{n-i})p}))) \\ + \frac{1}{\alpha -1} (\frac{p-1}{p})^{n-1} (g_p^{n-1}(\alpha) -1)D_{g_p^{n-1}(\alpha)}(X_{1}|| Y_{1}) 
    \end{multline}


    where $g_p(\alpha) = \frac{p}{p-1} \alpha - \frac{1}{p}$ and $g_p^{i}$ is $g_p$ composed $i$ times, where we defined $g_p^{0}(\alpha) = \alpha$.
\end{theorem}

Note some key variables in Theorem~\ref{thm:better_composition} are a flexible parameter $p$ (which we'll soon describe leads to blow-up as it gets smaller), and the distribution of per-step guarantees $D_{g_p^{i}(\alpha)}(X_{n-i}|| Y_{n-i})$ (the more concentrated at $0$ they are, the smaller the upper-bound). The proof relies on using an induction argument to continually break up the composition and is presented below.

\begin{proof}


The proof follows by repeating a similar reduction as Theorem~\ref{thm:composition}. First note 
    
\begin{multline}
    \int (X_1 \cdots X_n)^{\alpha} (Y_1 \cdots Y_n)^{1 - \alpha} dx_1 \cdots dx_n \\  = \int (X_1 \cdots X_{n-1})^{\alpha - 1/p} (Y_1 \cdots Y_{n-1})^{1 - \alpha}  \\ (\int X_n^{\alpha} Y_n^{1- \alpha} dx_n) (X_1 \cdots X_{n-1})^{1/p} dx_1 \cdots dx_{n-1}
    \\ \leq ( \int (X_1 \cdots X_n)^{\frac{p}{p-1}\alpha - \frac{1}{p-1}} (Y_1 \cdots Y_n)^{ \frac{p}{p-1}(1 - \alpha)}  dx_1 \cdots dx_{n-1})^{\frac{p-1}{p}} \\ (\int (\int X_n^{\alpha} Y_n^{1- \alpha} dx_n)^p (X_1 \cdots X_{n-1}) dx_1 \cdots dx_{n-1})^{1/p} 
\end{multline}

where the first equality was from using the markov property, and the last inequality was from Holder's inequality with Holder constant $p$. Do note that, defining $g_p(\alpha) = \frac{p}{p-1}\alpha - \frac{1}{p-1}$, we have $\frac{p}{p-1}(1 - \alpha) = 1 - g_p(\alpha)$. So now looking at the first term of the upper-bound we got, we are back to the original expression but with $\alpha \rightarrow g_p(\alpha)$ and $n \rightarrow n-1$, and an exponent to $\frac{p-1}{p}$. Note the second term is an expectation over the $n-1$ model state of the Markov chain. Do note $\int X_n^{\alpha} Y_n^{1- \alpha} dx_n$ is $e^{(\alpha -1)D_{\alpha}(X_{n-i}|| Y_{n-i})}$ for a fixed $n-1$ model state (i.e., fixed $x_{n-1}$ ). So repeating this step on the first term until we are left only with an integral over $x_1$ we have

\begin{multline}
    \int (X_1 \cdots X_n)^{\alpha} (Y_1 \cdots Y_n)^{1 - \alpha} dx_1 \cdots dx_n \\  
    \leq (\prod_{i=0}^{n-2} (\mathbb{E}_{X_1,\cdots X_{n-(i+1)}}  ((e^{(g_p^{i}(\alpha) -1)D_{g_p^{i}(\alpha)}(X_{n-i}|| Y_{n-i})})^p))^{\frac{(p-1)^i}{p^{i+1}}}) \\ ( (e^{(g_p^{n-1}(\alpha) -1)D_{g_p^{n-1}(\alpha)}(X_{1}|| Y_{1})})^p)^{\frac{(p-1)^{n-1}}{p^n}}
\end{multline}

So now noting $$D_{\alpha}(X || Y) = \frac{1}{\alpha -1} \ln(\int (X_1 \cdots X_n)^{\alpha} (Y_1 \cdots Y_n)^{1 - \alpha} dx_1 \cdots dx_n)$$

we conclude by the previous expression that 

\begin{multline}
        D_{\alpha}(X || Y) \leq \frac{1}{\alpha -1} (\sum_{i=0}^{n-2} \frac{(p-1)^i}{p^{i+1}}  \ln (\mathbb{E}_{X_1,\cdots X_{n-(i+1)}}  ((e^{(g_p^{i}(\alpha) -1)D_{g_p^{i}(\alpha)}(X_{n-i}|| Y_{n-i})p}))) \\ + \frac{1}{\alpha -1} ((\frac{(p-1)^{n-1}}{p^n}) \ln ((e^{(g_p^{n-1}(\alpha) -1)D_{g_p^{n-1}(\alpha)}(X_{1}|| Y_{1})})^p)) 
    \end{multline}

which completes the proof as the last term simplifies to the term stated in the theorem.
\end{proof}


\paragraph{Applying to DP-SGD.} To interpret Theorem~\ref{thm:better_composition} in the context of DP-SGD, we can let $X_i$ be the distribution of the $i'th$ model update (for a fixed $(i-1)'th$ model) when training on one dataset $D$, and similarly $Y_i$ when training on a neighbouring dataset $D'$. Letting $Train_{DP-SGD}$ denote the Markov chain of the intermediate model updates when using DP-SGD, we have the maximum over the bound given by Theorem~\ref{thm:better_composition} on $D_{\alpha}(Train_{DP-SGD}(D)||Train_{DP-SGD}(D'))$ and $D_{\alpha}(Train_{DP-SGD}(D')||Train_{DP-SGD}(D))$ provides our per-instance RDP guarantee for DP-SGD.




\paragraph{Balancing the value of $p$.}To understand the dependence on $p$ in Theorem~\ref{thm:better_composition}, consider for a moment $p =2$. In this case, we observe that at the $i$'th step, we need to compute a R\'enyi divergence of order $\sim 2^{i} \alpha$. It is known that the R\'enyi divergence $D_{c}(P||Q)$ grows with $c$ \citep{van2014renyi}, and in the case of the Gaussian mechanism, this growth is linear with $c$~\citep{mironov2017renyi}. Hence this exponential growth in the R\'enyi divergence order can prove impractical as a useful tool to analyze DP-SGD. However, as $p \rightarrow \infty$ we see that the growth on the order of the divergence shrinks.

Yet, by taking larger $p$ values we are effectively taking larger $L_{p}$-norms of the per-step guarantees seen in training and so effectively turn to worst-case per-step analysis as $p \rightarrow \infty$. Hence it is desirable to choose $p$ just sufficient for there to not be a significant blow-up in the order of the divergences for a given $n$. This can be done by analyzing how $g_{p}^{i}(\alpha)$ grows.

\begin{fact}\label{fact:p_control}
    If $p = O(n)$ then $g_{p}^i(\alpha) \leq 2 \alpha~\forall i \leq n$. In particular, $p = 3n$ works for sufficiently large $n$.
\end{fact}

The proof follows from direct calculations with the formula for $g_{p}(\alpha)$. 

\begin{proof}

Note that $g_{p}(\alpha) \leq \frac{p}{p-1}\alpha$ hence $g_{p}^{i}(\alpha) \leq (\frac{p}{p-1})^{i}\alpha$. From this we see showing $\frac{p}{p-1}^{n} \leq 2$ for $p = O(n)$ will imply $g_p^{i}(\alpha) \leq 2 \alpha~\forall 1 \leq n$.

Note we can equivalently show $ln(\frac{p}{p-1}) = \ln(p) - \ln(p-1) \leq \frac{\ln (2)}{n}$. But if we take $p = 3n$ note $\ln(3n) - \ln(3n-1) \leq \frac{1}{3n-1}$ by the derivative of $\ln(x) \leq \frac{1}{3n-1}$ for $x \geq 3n-1$. So it suffices to show $\frac{1}{3n-1} \leq \frac{\ln(2)}{n}$, but this is true for sufficiently large $n$.

    
\end{proof}


\paragraph{Estimating Theorem~\ref{thm:better_composition}}


In cases where one does not know the expectations used in Theorem~\ref{thm:better_composition} analytically, as is the case with DP-SGD when it is applied to deep learning, one can resort to empirically estimating the means. Our goal is to understand how much better our data-dependent guarantees are than the data-independent baseline for DP-SGD on common datasets. Hence, we wish to estimate the expression of Theorem~\ref{thm:better_composition} (or specifically the per-step contributions) with an error
$c \epsilon$ for $c < 1$ where $\epsilon$ is the data-independent guarantee (per-step). 


The following fact focuses on estimating the $i'th$ per-step guarantee with an error relative to the worst-case per-step guarantee when $p = 3n$ as is used in our experiments. In particular, we have the $i'th$ per-step guarantee 

\begin{multline*}
    \frac{1}{\alpha-1} \frac{(p-1)^i}{p^{i+1}} \ln (\mathbb{E}_{X_1, \cdots, X_{n-(i+1)}} f) \\ \coloneqq \frac{1}{\alpha-1} \frac{(p-1)^i}{p^{i+1}} \ln (\mathbb{E}_{X_1,\cdots X_{n-(i+1)}}  ((e^{(g_p^{i}(\alpha) -1)D_{g_p^{i}(\alpha)}(X_{n-i}|| Y_{n-i})})^p))
\end{multline*}

is less than the data-independent per-step privacy guarantee $\epsilon/n$ if $\mathbb{E}_{X_1,\cdots X_{n-(i+1)}} f \leq e^{(\alpha-1) 3 \epsilon}$ for $p = 3n$. Hence we describe the number of samples needed to estimate $\mathbb{E} f$ with precision relative to $e^{(\alpha-1) 3 \epsilon}$ (with high probability), which can be done in a constant number of samples relative to the data-independent bound.

\begin{fact}\label{fact:estimating}
    Let $\epsilon/n$ be the classical $\alpha$-R\'enyi DP guarantee for the $i'th$ step, and $\epsilon'/n$ be the analogous $2\alpha$-R\'enyi DP guarantee for the $i'th$ step. Then for $l \geq \frac{- \ln(J)}{c^2} e^{6(\alpha-1)\epsilon - 3(2\alpha -1) \epsilon'}$ and $p = 3n$ with $n$ s.t $g_p^{n-1} \leq 2\alpha$, we have $\mathbb{P}(|\mathbb{E}^{l} f - \mathbb{E}f| \geq c e^{(\alpha-1) 3 \epsilon}) \leq J$. Here $\mathbb{E}^l$ denotes the empirical mean over $l$ samples.
\end{fact}

The proof follows from Hoeffding's inequality.

\begin{proof}
    

For the given choice of $p$ and $\alpha$ we have $g_{p}^{i} \leq 2\alpha$ hence $D_{g_p^{i}(\alpha)}(X_{n-i} || Y_{n-i}) \leq D_{2 \alpha}(X_{n-i} || Y_{n-i}) \leq \epsilon'/n$ where $\epsilon'$ is determined by $\epsilon$ (when accounting for the increase due to the $\alpha$-order). Hence we have that $f \leq e^{3 (2\alpha -1) \epsilon'}$.

By Hoeffding's inequality we can hence conclude $\mathbb{P}(|\mathbb{E}^l f - \mathbb{E}f| \geq c e^{3(\alpha-1)\epsilon}) \leq e^{-\frac{e^{6(\alpha-1)\epsilon}c^2 l}{e^{3(2\alpha - 1) \epsilon'}}}$. Now upper-bounding the right-hand side by $J$ and rearranging to isolate for $l$, we can conclude the stated condition on $l$.

\end{proof}


\subsubsection{Per-Instance R\'enyi DP for General Updates}



The results of Section~\ref{ssec:sum_update} and Section~\ref{ssec:comp} provide a complete per-instance RDP analysis of the current implementation of DP-SGD. In particular, with the per-step update rule being the sum of gradients. In this section we ask, how should we analyze per-step guarantees (and hence DP-SGD given our composition theorem) if the update rule is not the sum? In general, the worst-case sensitivity over mini-batches may be far higher than the expected sensitivity over mini-batches (unlike the sum update rule), meaning the analysis from Theorem~\ref{thm:easy_renyi_dp} may be as bad as a data-independent analysis. For example, the typical update rule used in normal SGD is the mean update rule. However, $\Delta_{U,x^*}(X_B)$ for the mean update rule is the difference between the update for the datapoint $x^*$ and the mean of the updates on $X_B$; this difference is not the same for all minibatches $X_B$ and hence would be overestimated with the analysis of Theorem~\ref{thm:easy_renyi_dp}. One could resolve this issue of overestimating sensitivity by using the $L_p$ norms $||\Delta_{U,x^*}(\mathbf{X_B})||_{p} = (\mathbb{E}_{X_B} (\Delta_{U,x^*}(X_B)^p))^{1/p}$ with $p < \infty$ in the RDP analysis of the sampled Gaussian mechanism, as was done in the $(\epsilon,\delta)$-DP case. %
However, we are not aware of an approach to do this for R\'enyi DP.




Instead, we show how a new sensitivity distribution comparing all mini-batches $X_B$ in $X$ to all mini-batches $X'_B$ in $X' = X \cup \{x^*\}$, as opposed to just a single point $x^*$ as done with $\Delta_{U,x^*}(X_B)$, is amenable to a R\'enyi-DP analysis of the sampled Gaussian mechanism that does not look at the maximum privacy leakage over mini-batches. %
If the distribution of all updates given by $X$ is similar to the distribution of all updates given by $X'$, then analysis with this new sensitivity distribution can be expected to beat the current data-independent analysis.


Specifically, given $\alpha$ minibatches sampled from $X$, ${X_{B}}^{\tilde \alpha} \sim X$, and a particular minibatch sampled from $X'$, $X'_B \sim X'$, we define a new sensitivity distribution for $\alpha$-R\'enyi DP as follows: 

\begin{multline*}
\Delta_{U,\alpha}({X_{B}}^{\tilde \alpha}, X'_B) \coloneqq \sum_{i} ||U({X_B}^i)||_2^2 - (\alpha-1) ||U(X'_B)||_2^2 - ||\Delta_{\alpha}({X_B}^{\tilde \alpha},X'_B)||_2^2
\end{multline*}


where $\Delta_{\alpha}({X_B}^{\tilde \alpha},X'_B) = (\sum_{i} U({X_B}^i)) - (\alpha - 1) U(X'_B)$. When letting ${X_{B}}^{\tilde \alpha}$ and $X'_B$ be random variables, $\Delta_{U,\alpha}$ effectively compares all the mini-batches in $X'$ to all the $\alpha$-tuples of mini-batches in $X$. The $\alpha$-tuples appear here due to their equivalence with an expectation over mini-batches to the power of $\alpha$ which appears when analyzing $\alpha$-R\'enyi divergences. As described earlier, comparing this to the previous sensitivity distribution $\Delta_{U,x^*}(X_B)$, we see that this new sensitivity will compare all mini-batches in $X$ to all mini-batches in $X'$ (and not just to a point $x^*$) and hence captures more ``global" changes in updates due a datapoint $x^*$.




Theorem~\ref{thm:renyi_dp_sens} states the R\'enyi diveregence of the sampled Gaussian mechanism $M$ between two arbitrary datasets using $\Delta_{U, \alpha}$ through applying a transformation on its fixed $X'_B$ marginal values and taking its expectation over $X'_B$. Taking the maximum of the bounds for $D_{\alpha}(M(X)||M(X'))$ and $D_{\alpha}(M(X')||M(X))$ from Theorem~\ref{thm:renyi_dp_sens} where $X' = X \cup \{x^*\}$ gives a per-instance guarantee of $M$ for $X,X'$.




\begin{theorem}
\label{thm:renyi_dp_sens}
Let $\alpha > 1$ be an integer. Given two arbitrary datasets $X,X'$, the sampled Gaussian mechanism $M$ with noise $\sigma$ satisfies: 

$$D_{\alpha}(M(X')||M(X)) \leq \frac{1}{(\alpha-1)} \mathbb{E}_{X_B} (\ln (\mathbb{E}_{{X'_{B}}^{\tilde \alpha}}(e^{\frac{-1}{2\sigma^2}\Delta_{U,\alpha}({X'_{B}}^{\tilde \alpha}, X_B)})))$$



\end{theorem}



Some key variables in Theorem~\ref{thm:renyi_dp_sens} is the standard deviation of noise $\sigma$ (increasing it decreases the upper-bound) and the sensitivity distribution $\Delta_{U,\alpha}({X_{B}}^{\tilde \alpha}, X'_B)$ (the more concentrated at $0$ it is, the smaller the upper-bound). The proof relies on convexity, which is always true for the second argument of the R\'enyi divergence $D_{\alpha}(A||B)$, and then direct calculations involving Gaussians. %

\begin{proof}

For simpler notation, we use $\mu_X = U(X)$. We proceed by taking $\alpha$ to be an integer (to use an expansion similar to Section 3.3 in \citet{mironov2019r}) and utilizing Theorem 12 in~\citet{van2014renyi}. We will let $N_{X_B} = N(\mu_{X_B},\sigma^2)$ where $\mu_{X_B} = U(X_B)$ as stated earlier.


We proceed to bound $D_{\alpha}(M(X') || M(X))$ for arbitrary $X',X$. Hence a completely analogous argument will allow us to also bound $D_{\alpha}(M(X) || M(X'))$ when $X'$ is specifically $X \cup \{x^*\}$. First note


\begin{multline}
    D_{\alpha}(M(X') || M(X)) = D_{\alpha}(\sum_{X'_B} \mathbb{P}(X'_B) N_{X'_B} || \sum_{X_B} \mathbb{P}(X_B) N_{X_B}) \\ \leq \sum_{X_B} \mathbb{P}(X_B) D_{\alpha}(\sum_{X'_B} \mathbb{P}(X'_B) N_{X'_B} || N_{X_B})
\end{multline}


where the last inequality is from the fact the divergence is convex in the second argument (Theorem 12 in~\citet{van2014renyi}). 

Now note 
\begin{multline}
    e^{(\alpha-1)D_{\alpha}(\sum_{X'_B} \mathbb{P}(X'_B) N_{X'_B} || N_{X_B})} \\ = \int (\sum_{X'_B}\mathbb{P}(X'_B) \frac{1}{(\sigma \sqrt{2\pi})^d} e^{\frac{-1}{2\sigma^2} |x - \mu_{X'_B}|^2})^{\alpha} (\frac{1}{(\sigma \sqrt{2\pi})^d} e^{\frac{-1}{2\sigma^2}|x - \mu_{X_B}|^2})^{1- \alpha} dx \\ = \sum_{{X'_B}^{\tilde \alpha}} \mathbb{P}({X'_B}^{\tilde \alpha}) \frac{1}{(\sigma \sqrt{2\pi})^d} \int e^{\frac{-1}{2\sigma^2} ( (\sum_{{X'_B}^i}|x- \mu_{{X'_B}^i}|^2) - (\alpha - 1)|x - \mu_{X_B}|^2)}
\end{multline}


where we expanded $(\sum_{X'_B}\mathbb{P}(X'_B) \frac{1}{(\sigma \sqrt{2\pi})^d} e^{\frac{-1}{2\sigma^2} |x - \mu_{X'_B}|^2})^{\alpha}$ by noting each term in the product is just iterating through all $\alpha$ tuples of mini-batches from $X'$.

Now note we can for now consider the integral in each dimension, as the overall integral is the product of each dimension. Also recall from the theorem statement that we define $$\Delta_{\alpha}({X'_B}^{\tilde \alpha},X_B) = (\sum_{i} \mu_{{X'_B}^i}) - (\alpha - 1) \mu_{X_B}$$ Hence (letting everything be one dimensional for now) we have

\begin{multline}
    (\sum_{{X'_B}^i}|x- \mu_{{X'_B}^i}|^2) - (\alpha - 1)|x - \mu_{X_B}|^2 \\ = x^2 - 2 \Delta_{\alpha}({X'_B}^{\tilde \alpha},X_B)x + \sum_{i} \mu_{{X'_B}^i}^2 - (\alpha-1) \mu_{X_B}^2 \\ = (x - \Delta_{\alpha}({X'_B}^{\tilde \alpha},X_B))^2 + \sum_{i} \mu_{{X'_B}^i}^2 - (\alpha-1) \mu_{X_B}^2 - \Delta_{\alpha}({X'_B}^{\tilde \alpha},X_B)^2
\end{multline}


Hence, we have 

\begin{multline}
    \int e^{\frac{-1}{2\sigma^2} ( (\sum_{{X'_B}^i}|x- \mu_{{X'_B}^i}|^2) - (\alpha - 1)|x - \mu_{X_B}|^2)} \\ = e^{\frac{-1}{2\sigma^2}(\sum_{i} {\mu_{{X'_B}^i}}^2 - (\alpha-1) \mu_{X_B}^2 - \Delta_{\alpha}({X'_B}^{\tilde \alpha},X_B)^2)} \int e^{\frac{-1}{2\sigma^2} (x - \Delta_{\alpha}({X'_B}^{\tilde \alpha},X_B))^2} \\ = \sigma \sqrt{2 \pi} e^{\frac{-1}{2\sigma^2}(\sum_{i} \mu_{{X'_B}^i}^2 - (\alpha-1) \mu_{X_B}^2 - \Delta_{\alpha}({X'_B}^{\tilde \alpha},X_B)^2)}
\end{multline}


Note going back to the integral over all dimensions we get $$= (\sigma \sqrt{2 \pi})^{d} e^{\frac{-1}{2\sigma^2}(\sum_{i} ||{\mu_{{X'_B}^i}||_2}^2 - (\alpha-1) ||\mu_{X_B}||_2^2 - ||\Delta_{\alpha}({X'_B}^{\tilde \alpha},X_B)||_2^2)}$$

Thus to conclude we get 

\begin{multline}
    D_{\alpha}(M(X') || M(X))  \leq \sum_{X_B} \mathbb{P}(X_B) D_{\alpha}(\sum_{X'_B} \mathbb{P}(X'_B) N_{X'_B} || N_{X_B}) \\ = \sum_{X_B} \mathbb{P}(X_B) \frac{1}{(\alpha-1)} \\ \ln (\sum_{{X'_B}^{\tilde \alpha}} \mathbb{P}({X'_B}^{\tilde \alpha}) e^{\frac{-1}{2\sigma^2}(\sum_{i} ||\mu_{{X'_B}^i}||_2^2 - (\alpha-1) ||\mu_{X_B}||_2^2 - ||\Delta_{\alpha}({X'_B}^{\tilde \alpha},X_B)||_2^2)})
\end{multline}

A completely analogous calculation gives the same bound with just $X_B$ replaced with $X_B'$ (and vice-versa) for $D_{\alpha}(M(X)||M(X'))$. Taking the max over both these divergences gives a bound on the per-step per-instance R\'enyi-DP guarantee.

\end{proof}


Hence we now have a per-step RDP analysis for DP-SGD that takes advantage of when expected minibatch sensitivity to $x^*$ is much better than the worst cast minibatch sensitivity. While this phenomenon is not useful for studying the sum update rule (what is currently used for DP-SGD) as every mini-batch has the same sensitivity to $x^*$, in Section~\ref{ssec:exp_hard_renyi} we show this analysis allows us to provide a tighter analysis of the mean update rule. Hence, this opens the possibility of future work deploying DP-SGD with different update rules.

\section{Empirical Results}
\label{sec:main_body_emp_results}


In Section~\ref{sec:analysis} we provided the first framework to analyze DP-SGD's per-instance privacy guarantees. This followed by providing new per-step analyses (Theorem~\ref{thm:easy_renyi_dp} and~\ref{thm:renyi_dp_sens}), and a new composition theorem that relies on summing ``expected" per-step guarantees (Theorem~\ref{thm:better_composition}). 
We now highlight several conclusions our framework allows us to make about per-instance privacy when using DP-SGD. For conciseness, we defer a subset of the experimental results to Appendix~\ref{sec:detailed_emp_res}. %





\paragraph{Experimental Setup.} In the subsequent experiments, we apply our analysis on MNIST~\citep{lecun1998mnist} and CIFAR-10~\citep{krizhevsky2009learning}. Unless otherwise specified, LeNet-5~\citep{lecun1989backpropagation} and ResNet-20~\citep{he2016deep} were trained on the two datasets for 10 and 200 epochs respectively using DP-SGD, with a mini-batch size equal to 128, $\epsilon=10$, $\delta = 10^{-5}$, $\alpha = 8$ (in cases of Renyi DP), and clipping norm $C = 1.0$. All the experiments are repeated 100 times by sampling 100 data points to obtain a distribution/confidence interval if not otherwise stated.
Regarding hardware, we used NVIDIA T4 to accelerate our experiments. 


\begin{figure}[!t]

\centering
\subfloat[Training with the datapoint ($X \cup \{x^*\}$) \label{fig:compo_1_more}]
{
\includegraphics[width=0.33\linewidth]{figures/compo_simple_eps_compo_vary_eps_fraction_curve_MNIST_eps.pdf}
}
\subfloat[Training with the datapoint ($X \cup \{x^*\}$)\\$\text{~~~~~}$($10^{th}$ percentile) \label{fig:compo_1_more_10per}]
{
\includegraphics[width=0.33\linewidth]{figures/compo_simple_eps_compo_vary_eps_fraction_curve_MNIST_epspercentile10.pdf}
}
\subfloat[Training with and without the datapont \\ ($X \cup \{x^*\}$ and $X$) \label{fig:compo_1_less}]
{
\includegraphics[width=0.33\linewidth]{figures/remove_simple_eps_compo_vary_eps_fraction_curve_MNIST_exp.pdf}
}
\caption{Per-step privacy contribution from our composition theorem (Theorem~\ref{thm:better_composition}) using the per-step gurantee for the sum update rule (Theorem~\ref{thm:easy_renyi_dp}) as needed for DP-SGD, plotted as a fraction of the baseline data-independent per-step DP-SGD guarantee (Section 3.3 in~\citet{mironov2019r}). %
The expectations for Theorem~\ref{thm:better_composition} are computed over 10 trials. Figure~\ref{fig:compo_1_more} plots the average relative per-step contributon of 100 random points in MNIST for different strengths of the DP guarantee (i.e., different upper bounds $\varepsilon$) used when training on $X' = X \cup \{x^*\}$. The $10^{th} percentile$ is plotted in Figure~\ref{fig:compo_1_more_10per}. Figure~\ref{fig:compo_1_less} plots expectation when training on $X'$ and $X$ for 10 random points in MNIST. We see from both subfigures our per-step contribution decreases relative to the baseline as training progresses: using Theorem~\ref{thm:better_composition} one can conclude that many datapoints have better overall data-dependent privacy guarantees than expected by classical analysis.
}


\label{fig:composition}
\end{figure}

\subsection{Many Datapoints have Better Privacy}
\label{ssec:exp_better_privacy}




Here we describe how our per-instance RDP analysis of DP-SGD, using Theorem~\ref{thm:easy_renyi_dp} for the per-step analysis (with the update rule being the sum of gradients as is typically used) and Theorem~\ref{thm:better_composition} for the composition analysis, allows us to explain why per-instance privacy attacks will fail for many datapoints: many points have better per-instance privacy than the data-independent analysis. We further investigate the distribution of the per-instance privacy guarantees, and which points exhibit better per-instance privacy with our analysis.

\paragraph{Improved Per-Instance Analysis for Most Points} We compare the guarantees given by Theorem~\ref{thm:easy_renyi_dp} for the per-step guarantee in DP-SGD to the guarantee given by the data-independent analysis (see Section 3.3 in \citet{mironov2019r}), and plot per-step contribution coming from our composition theorem. In particular, we take $X$ to be the full MNIST training set, and randomly sample a data point $x^*$ from the test set to create $X' = X \cup x^*$ (as mentioned earlier, we repeat the sampling of $x^*$ 100 times to obtain a confidence interval). We train 10 different models on $X$ with the same initialization and compute the per-step contribution from Theorem~\ref{thm:better_composition} between $X$ and $X'$ (using Theorem~\ref{thm:easy_renyi_dp} to analyze the per-step guarantee from a given model) over the training run, shown in Figure~\ref{fig:compo_1_more}.
We can see that our analysis of the per-step contribution decreases with respect to the baseline as we progress through training. This persists regardless of the expected mini-batch size, the strength of DP used during training, and model architectures; see Figure~\ref{fig:renyi_simple_composition_mnist_sum} in Appendix~\ref{sec:detailed_emp_res}.
By Theorem~\ref{thm:better_composition} we conclude that $D_{\alpha}(Train_{DP-SGD}(X) || Train_{DP-SGD}(X'))$ is significantly less than the baseline for many data points. %



To see our improvement over the max of $D_{\alpha}(Train_{DP-SGD}(X) || Train_{DP-SGD}(X'))$ and \\ $D_{\alpha}(Train_{DP-SGD}(X') || Train_{DP-SGD}(X))$, i.e., the R\'enyi-DP guarantee, we computed the expectation when training on $X$ and $X'= X \cup \{x^*\}$  for $10$ training points $x^*$ where $X$ is now the training set of MNIST with one point removed and $X'$ is the full training set. Our results are shown in Figure~\ref{fig:compo_1_less} where we see a similar decreasing trend relative to the baseline over training: we conclude by Theorem~\ref{thm:better_composition} that many datapoints have better per-instance R\'enyi DP than the baseline. In other words, we conclude many datapoints have stronger per-instance RDP guarantees than can be demonstrated through the classical data-independent analysis.


\begin{figure}[!t]

\centering
\subfloat[Mini-batch Size = 128 \label{fig:simple_renyi_training_stage}]
{
\includegraphics[width=0.4\linewidth]{figures/renyi_simple_eps_hist_CIFAR10_resnet20_128_8_sum.pdf}
}
\subfloat[Varying Mini-batch Size \label{fig:simple_renyi_vary_bs}]
{
\includegraphics[width=0.4\linewidth]{figures/renyi_simple_eps_hist_vary_bs_CIFAR10_resnet20_sum.pdf}
}

\caption{Distribution plots of the per-step guarantees given by Theorem~\ref{thm:easy_renyi_dp} for $500$ datapoints in CIFAR10 with respect to: (a) different stages of training, and (b) varying mini-batch size. The purple dashed line 
represents the data-independent baseline. We observe a long tail of datapoints with magnitudes better privacy than expected in both plots.
}
\label{fig:simple_renyi}
\end{figure}

\begin{figure}[!t]
\centering
\includegraphics[width=0.4\linewidth]{figures/renyi_simple_eps_resnet20_fraction_curve_CIFAR10_sum.pdf}
\caption{Per-step guarantees given by Theorem~\ref{thm:easy_renyi_dp} for $500$ datapoints in CIFAR10 across training stages with respect to correct or incorrect classifications. It can be seen that correctly classified datapoints are on average more private than incorrectly classified ones.}
\label{fig:correct_incorrect}
\end{figure}



\paragraph{Long-Tail of Better Per-Instance Privacy.} However, the previous figures only show the average effect over datapoints. In Figures~\ref{fig:simple_renyi_training_stage} and~\ref{fig:simple_renyi_vary_bs} we plot the distribution of per-step guarantees over $500$ data points in CIFAR10. The key observations are (1) there exists a long tail of data points with significantly better per-instance privacy than the baseline,  (2) such improvements mostly exist in the middle and end of the training process, and (3) such improvements are mostly independent of mini-batch size.










\paragraph{Correct Points are More Per-Instance Private.} Next, we turn to understanding what datapoints are experiencing better privacy when using DP-SGD. In Figure~\ref{fig:correct_incorrect}, we plot the per-step guarantees given by Theorem~\ref{thm:easy_renyi_dp} for correctly and incorrectly classified data points at the beginning, middle, and end of training
on CIFAR10 (and for MNIST in Figure~\ref{fig:renyi_simple_fraction_curve_vary_arch_mnist} in Appendix~\ref{sec:detailed_emp_res}). We see that, on average, correctly classified data points have better per-step privacy guarantees than incorrectly classified data points across training. This disparity holds most strongly towards the end of training.










\subsection{Higher Sampling Rates can give Better Privacy}
\label{ssec:exp_hard_renyi}

We now highlight how our analysis, if it uses Theorem~\ref{thm:renyi_dp_sens} for the per-step analysis, allows us to better analyze DP-SGD with other update rules (not the sum of gradients which is what the current implementation of DP-SGD uses and Section~\ref{ssec:exp_better_privacy} analyzed). In particular, we will analyze the mean update rule and show how it has a privacy trade-off with sampling rate that is opposite to the trade-off for the sum update rule.

In normal SGD (with gradient clipping), one computes a mean for the per-step update 
$U(X_B) = \frac{1}{|X_B|}\sum_{x \in X_B} \nabla_{\theta}\mathcal{L}(\theta,x)/ \max(1,\frac{||\nabla_{\theta}\mathcal{L}(\theta,x)||_2}{C})$. 
However, DP-SGD computes a weighted sum $U(X_B) = \frac{1}{L} \sum_{x \in X_B} \nabla_{\theta}\mathcal{L}(\theta,x)/ \max(1,\frac{||\nabla_{\theta}\mathcal{L}(\theta,x)||_2}{C})$. Note the subtle difference between dividing by a fixed constant $L$ (typically the expected mini-batch size when Poisson sampling datapoints) and by the mini-batch size $|X_B|$. This means for the sum the upper-bound on sensitivity is $\frac{C}{L}$, while for the mean the upper-bound on sensitivity is only $C$ (consider neighbouring mini-batches of size 1 and 2). Hence using the mean update rule requires far more noise and so is not practical to use. We highlight how our per-instance analysis by sensitivity distributions provides better guarantees for the mean update rule.


\begin{figure}[!t]
\centering
\subfloat[$D_{\alpha}(M(X')||M(X))$\\$\text{~~~~~}$Mini-batch Size = $128$ \label{fig:hard_renyi_training_stage}]
{
\includegraphics[width=0.33\linewidth]{figures/renyi_hard_eps_hist_CIFAR10_resnet20_128.0_8_mean.pdf}
}
\subfloat[$D_{\alpha}(M(X)||M(X'))$ \\$\text{~~~~~}$Mini-batch Size = $128$
\label{fig:hard_renyi_reverse}]
{
\includegraphics[width=0.33\linewidth]{figures/reverse_renyi_hard_eps_hist_CIFAR10_resnet20_128_8_mean.pdf}
}
\subfloat[$D_{\alpha}(M(X')||M(X))$\\$\text{~~~~~}$Varying Mini-batch Size \label{fig:hard_renyi_vary_bs}]
{
\includegraphics[width=0.33\linewidth]{figures/renyi_hard_eps_hist_vary_bs_CIFAR10_resnet20_mean_unnormalized.pdf}
}

\caption{ Distribution plots (log scale) of per-step guarantees from Theorem~\ref{thm:renyi_dp_sens} for $500$ datapoints in CIFAR10 with respect to different training stages and mini-batch sizes. Bounds on both $D_{\alpha}(M(X)||M(X'))$ and $D_{\alpha}(M(X')||M(X))$ are shown for an expected mini-batch size of 128.  From Figures~\ref{fig:hard_renyi_training_stage},\ref{fig:hard_renyi_reverse}, we conclude Theorem~\ref{thm:renyi_dp_sens} gives better data-dependent guarantees for the mean update rule than classicial analysis, and from Figure~\ref{fig:hard_renyi_vary_bs} that increasing the expected mini-batch size decreases our bound for this update rule (counter-intuitive to privacy amplification by subsampling).
}
\label{fig:hard_renyi}
\end{figure}


\paragraph{Better Analysis of the Mean Update Rule. } Letting $M$ now be the sampled Gaussian mechanism with the mean update rule, we compute the bound on $D_{\alpha}(M(X')||M(X))$ and $D_{\alpha}(M(X)||M(X'))$ given by Theorem~\ref{thm:renyi_dp_sens}, where we estimated the inner and outer expectation using $20$ samples, i.e., $20$ random $X_B'^{\alpha}$ (or $X_B^{\alpha}$) for each of the $20$ random $X_B$ (or $X_B'$). We obtain Figure~\ref{fig:hard_renyi_training_stage} and~\ref{fig:hard_renyi_vary_bs} by repeating this for $500$ data points in CIFAR10 while varying the training stage. We observe that for both divergences, we beat the baseline analysis by more than a magnitude at the middle and end of training. We conclude Theorem~\ref{thm:renyi_dp_sens} gives us better per-instance R\'enyi DP guarantees for the mean update rule.








\paragraph{Per-Instance Privacy Improves with Higher Sampling Rate.} Furthermore, counter-intuitively to typical subsample privacy amplification, in Figure~\ref{fig:hard_renyi_vary_bs} we see that our bound decreases with increasing expected mini-batch size: 
we attribute this to the law of large numbers, whereby increasing the expected mini-batch size leads to sampled mini-batches having similar updates more often and hence the sensitivity distribution concentrates at smaller values. An analogous result is shown for MNIST in Figure~\ref{fig:renyi_hard_eps_distrib_bs_mnist_mean} (in Appendix~\ref{sec:detailed_emp_res}).




To put our results in a broader context, let us briefly discuss alternative methods for 
classical simulation of quantum circuits. Vector-based simulators~\cite{de2007massively,smelyanskiy2016qhipster,haner20170}
represent $n$-qubit quantum states by complex vectors of size $2^n$
stored in a classical memory.
The state vector is updated
upon application of each gate by performing sparse
matrix-vector multiplication. The memory footprint limits 
the method to small number of qubits. For example, 
H{\"a}ner and Steiger~\cite{haner20170}
reported a simulation of
quantum circuits with $n=45$ qubits and a few hundred gates 
using a  supercomputer with $0.5$ petabytes of memory.
In certain special cases 
the memory footprint can be reduced 
by recasting the simulation problem as a 
tensor network contraction~\cite{markov2008simulating,boixo2017simulation,aaronson2016complexity}.
Several tensor-based simulators have been developed~\cite{pednault2017breaking,li2018quantum,chen2018classical} 
for geometrically local  shallow quantum  circuits that include only nearest-neighbor
gates on a 2D grid of qubits~\cite{boixo2018characterizing}.
These methods enabled simulations of systems with more than $100$ qubits~\cite{chen2018classical}.
However, it is expected~\cite{alibaba} that for general (geometrically non-local) circuits 
of size $poly(n)$  the runtime of tensor-based simulators scales as $2^{n-o(n)}$.

In contrast, Clifford simulators described in the present paper are applicable to large-scale circuits
without any locality properties as long as the circuit is dominated by Clifford gates. 
This regime may be important for verification of first fault-tolerant quantum circuits
where  logical non-Clifford gates are expected to be scarce due to their high implementation
cost~\cite{fowler2013surface,jones2013low}.
Another advantage of Clifford simulators is their ability to sample the output
distribution of the circuit (as opposed to computing individual output amplitudes).
This is more close to what one would expect from the actual quantum computer. 
For example, a single run of the heuristic sum-over-Cliffords simulator
described in Section~\ref{heuristic} produces thousands of samples from the (approximate) output distribution. 
In contrast, a single run of a tensor-based simulator typically computes a single amplitude of the
output state.  Thus we believe that our techniques 
extend the reach of classical simulation algorithms complementing
the existing vector- or tensor-based simulators.

%PC:
A version of the sum-over-Cliffords simulator using the Metropolis sampling method is also publicly available
as part of \texttt{Qiskit-Aer}, the classical simulation framework of IBM's quantum
programming suite \texttt{Qiskit}~\cite{Qiskit}. This enables classical simulation and verification of quantum
circuits built in Qiskit on system sizes above $30$ qubits, which quickly become inaccessible with the
default vector-based method. This version also supports parallel processing over the stabilizer state decomposition,
which improves the performance of the Metropolis step.

%SBB:
Let us briefly comment on how simulators based on the stabilizer rank compare
with quasi-probability  methods~\cite{pashayan15,Delfosse15rebits,kocia2017discrete}.
The latter use a discrete Wigner function representation of quantum states
and Monte Carlo sampling
to approximate a given output probability of the target circuit with a small
additive error. Negativity of the Wigner function is an important parameter
that quantifies severity of the ``sign problem" associated with the Monte Carlo sampling.
The negativity also controls the runtime of quasi-probability methods. 
For example, the simulator proposed in~\cite{pashayan15}
has runtime $\epsilon^{-2} M^2$, where $M$ is the negativity and $\epsilon$
is the desired approximation error. In contrast to stabilizer rank simulators,
quasi-probability methods do not directly apply to stabilizer
operations on qubits since the latter are not known to have a non-negative Wigner function 
representation~\cite{Delfosse15rebits,karanjai2018contextuality}.
Furthermore, such methods are not well-suited for sampling the output
distribution since this task requires a small {\em multiplicative} error in 
approximating individual output probabilities. 

Our work leaves several  open questions. 
Since the efficiency of Clifford simulators hinges on the ability to find low-rank
stabilizer decompositions of multi-qubit magic states, 
improved techniques for finding such decompositions are of great interest. 
For example, consider a magic state $|\psi\ra=U|+\ra^{\otimes n}$, where 
$U$ is a diagonal circuit composed of $Z,CZ$, and $CCZ$ gates.
We anticipate that a low-rank exact stabilizer decomposition of $\psi$ can be 
found by computing the {\em transversal number}~\cite{alon1990transversal} of a suitable hypergraph describing
the placement of CCZ gates. Such low-rank decompositions may lead to more efficient
simulation algorithms for Clifford+CCZ circuits. We leave as an open question whether
the stabilizer extent $\xi(\psi)$ is multiplicative under tensor products for general states $\psi$.
Finally, it is of great interest to derive lower bounds on the stabilizer rank
of $n$-qubit magic states scaling exponentially with $n$. 






\section{Conclusion}
We have presented a neural performance rendering system to generate high-quality geometry and photo-realistic textures of human-object interaction activities in novel views using sparse RGB cameras only. 
%
Our layer-wise scene decoupling strategy enables explicit disentanglement of human and object for robust reconstruction and photo-realistic rendering under challenging occlusion caused by interactions. 
%
Specifically, the proposed implicit human-object capture scheme with occlusion-aware human implicit regression and human-aware object tracking enables consistent 4D human-object dynamic geometry reconstruction.
%
Additionally, our layer-wise human-object rendering scheme encodes the occlusion information and human motion priors to provide high-resolution and photo-realistic texture results of interaction activities in the novel views.
%
Extensive experimental results demonstrate the effectiveness of our approach for compelling performance capture and rendering in various challenging scenarios with human-object interactions under the sparse setting.
%
We believe that it is a critical step for dynamic reconstruction under human-object interactions and neural human performance analysis, with many potential applications in VR/AR, entertainment,  human behavior analysis and immersive telepresence.





\section*{Acknowledgements}
We would like to acknowledge our sponsors, who support our research with financial and in-kind contributions: Amazon, Apple, CIFAR through the Canada CIFAR AI Chair, DARPA through the GARD project, Intel, Meta, NSERC through the COHESA Strategic Alliance and a Discovery Grant, Ontario through an Early Researcher Award, and the Sloan Foundation. Anvith Thudi is supported by a Vanier Fellowship from the Natural Sciences and Engineering Research Council of Canada.  Resources used in preparing this research were provided, in part, by the Province of Ontario, the Government of Canada through CIFAR, and companies sponsoring the Vector Institute. We would further like to thank Relu Patrascu at the University of Toronto for providing the compute infrastructure needed to perform the experimentation outlined in this work. We would also like to thank members of the CleverHans lab and Mahdi Haghifam for their feedback on drafts of the manuscript.


\bibliographystyle{abbrvnat}
\bibliography{references}

\appendix
\newpage

\newpage
\appendix
\section{Pricing equations}
\subsection{Credit default swap}
\label{CDS_pricing}
A credit default swap (CDS) is a contract designed to exchange credit risk of a Reference Name (RN) between a Protection Buyer (PB) and a Protection Seller (PS). PB makes periodic coupon payments to PS conditional on no default of RN, up to the nearest payment date, in the exchange for receiving from PS the loss given RN's default.

Consider a CDS contract written on the first bank (RN), denote its price $C_1(t, x)$.\footnote{For the CDS contracts written on the second bank, the similar expression could be provided by analogy.} We assume that the coupon is paid continuously and equals to $c$. Then, the value of a standard CDS contract can be given (\cite{BieleckiRutkowski}) by the solution of  (\ref{kolm_1})--(\ref{kolm_2})  with $\chi(t, x) = c$ and terminal condition
\begin{equation*}
	\psi(x) = 
	\begin{cases}
		1 - \min(R_1, \tilde{R}_1(1)), \quad (x_1, x_2) \in D_2, \\
		1 - \min(R_1, \tilde{R}_1(\omega_2)), \quad (x_1, x_2) \in D_{12}, \\		
	\end{cases}
\end{equation*}
where $\omega_2 = \omega_2(x)$ is defined in (\ref{term_cond}) and 
\begin{equation*}
	\tilde{R}_1(\omega_2) = \min \left[1, \frac{A_1(T) +  \omega_2 L_{2 1}(T)}{L_1(T) + \omega_2 L_{12}(T)}\right].
\end{equation*}
Thus, the pricing problem for CDS contract on the first bank is
\begin{equation}
\begin{aligned}
		& \frac{\partial}{\partial t} C_1(t, x) + \mathcal{L} C_1(t, x) = c, \\
		& C_1(t, 0, x_2) = 1 - R_1, \quad C_1(t, \infty, x_2) = -c(T-t), \\
		& C_1(t, x_1, 0) = \Xi(t, x_1) = 
		\begin{cases}
			c_{1,0}(t, x_1), & x_1 \ge \tilde{\mu}_1, \\
			1-R_1, & x_1 < \tilde{\mu}_i,
		\end{cases} \quad C_1(t, x_1, \infty) = c_{1,\infty}(t, x_1),\\
		& C_1(T, x) = \psi(x) = 
	\begin{cases}
		1 - \min(R_1, \tilde{R}_1(1)), \quad (x_1, x_2) \in D_2, \\
		1 - \min(R_1, \tilde{R}_1(\omega_2)), \quad (x_1, x_2) \in D_{12}, \\		
	\end{cases}
\end{aligned}
\end{equation}
where $c_{1,0}(t, x_1)$ is the solution of the following boundary value problem:
\begin{equation}
\begin{aligned}
		& \frac{\partial}{\partial t} c_{1, 0}(t, x_1) + \mathcal{L}_1 c_{1, 0}(t, x_1) = c, \\
		& c_{1, 0}(t, \tilde{\mu}_1^{<}) = 1 - R_1, \quad c_{1, 0}(t, \infty) = -c(T-t), \\
		& c_{1, 0}(T, x_1) = (1 - R_1) \mathbbm{1}_{\{\tilde{\mu}_1^{<} \le x_1 \le \tilde{\mu}_1^{=}\}}, 
\end{aligned}
\end{equation}
and $c_{1,\infty}(t, x_1)$ is the solution of the following boundary value problem
\begin{equation}
\begin{aligned}
		& \frac{\partial}{\partial t} c_{1, \infty}(t, x_1) + \mathcal{L}_1 c_{1, \infty}(t, x_1) = c, \\
		& c_{1, \infty}(t, 0) = 1 - R_1, \quad c_{1, \infty}(t, \infty) = -c(T-t), \\
		& c_{1, \infty}(T, x_1) = (1 - R_1) \mathbbm{1}_{\{x_1 \le \mu_1^{=}\}}.
\end{aligned}
\end{equation}

\subsection{First-to-default swap}
An FTD contract refers to a basket of reference names (RN). Similar to a regular CDS, the Protection Buyer (PB) pays a regular coupon payment $c$ to the Protection Seller (PS) up to the first default of any of the RN in the basket or maturity time $T$. In return, PS compensates PB the loss caused by the first default.

Consider the FTD contract referenced on $2$ banks, and denote its price $F(t, x)$. We assume that the coupon is paid continuously and equals to $c$. Then, the value of FTD contract can be given (\cite{LiptonItkin2015}) by the solution of  (\ref{kolm_1})--(\ref{kolm_2})  with $\chi(t, x) = c$ and terminal condition
\begin{equation*}
	\psi(x) = \beta_0  \mathbbm{1}_{\{x \in D_{12}\}} + \beta_1 \mathbbm{1}_{\{x \in D_{1}\}} + \beta_2 \mathbbm{1}_{\{x \in D_{2}\}},
\end{equation*}
where
\begin{equation*}
	\begin{aligned}
		\beta_0 = 1 - \min[\min(R_1, \tilde{R}_1(\omega_2), \min(R_2, \tilde{R}_2(\omega_1)], \\
		\beta_1 = 1 - \min(R_2, \tilde{R}_2(1)), \quad \beta_2 = 1 - \min(R_1, \tilde{R}_1(1)),
	\end{aligned}
\end{equation*}
and
\begin{equation*}
	\tilde{R}_1(\omega_2) = \min \left[1, \frac{A_1(T) +  \omega_2 L_{2 1}(T)}{L_1(T) + \omega_2 L_{12}(T)}\right], \quad \tilde{R}_2(\omega_1) = \min \left[1, \frac{A_2(T) +  \omega_1 L_{1 2}(T)}{L_2(T) + \omega_1 L_{21}(T)}\right].
\end{equation*}
with $\omega_1 = \omega_1(x)$ and $\omega_2 = \omega_2(x)$ defined in (\ref{term_cond}).

Thus, the pricing problem for a FTD contract is
\begin{equation}
\begin{aligned}
		& \frac{\partial}{\partial t} F(t, x) + \mathcal{L} F(t, x) = c, \\
		& F(t, x_1, 0) = 1 - R_2,  \quad F(t, 0, x_2) = 1 - R_1, \\
		& F(t, x_1, \infty) = f_{2,\infty}(t, x_1), \quad F(t, \infty, x_2) = f_{1,\infty}(t, x_2), \\
		& F(T, x) = \beta_0  \mathbbm{1}_{\{x \in D_{12}\}} + \beta_1 \mathbbm{1}_{\{x \in D_{1}\}} + \beta_2 \mathbbm{1}_{\{x \in D_{2}\}},
\end{aligned}
\end{equation}
where $f_{1,\infty}(t, x_1)$ and $f_{2,\infty}(t, x_2)$ are the solutions of the following boundary value problems
\begin{equation}
\begin{aligned}
		& \frac{\partial}{\partial t} f_{i, \infty}(t, x_i) + \mathcal{L}_i f_{i, \infty}(t, x_i) = c, \\
		& f_{i, \infty}(t, 0) = 1 - R_i, \quad f_{i, \infty}(t, \infty) = -c(T-t), \\
		& f_{1, \infty}(T, x_i) = (1 - R_i) \mathbbm{1}_{\{x_i \le \mu_i^{=}\}}.
\end{aligned}
\end{equation}

\subsection{Credit and Debt Value Adjustments for CDS}

Credit Value Adjustment and Debt Value Adjustment can be considered either unilateral or bilateral. For unilateral counterparty risk, we need to consider only two banks (RN, and PS for CVA and PB for DVA), and a two-dimensional problem can be formulated, while bilateral counterparty risk requires a three-dimensional problem, where Reference Name, Protection Buyer, and Protection Seller are all taken into account. We follow \cite{LiptonSav} for the pricing problem formulation but include jumps and mutual liabilities, which affects the boundary conditions.

\paragraph{Unilateral CVA and DVA}
The Credit Value Adjustment represents the additional price associated with the possibility of a counterparty's default. Then, CVA can be defined as
\begin{equation}
	V^{CVA} = (1- R_{PS}) \mathbb{E}[\mathbbm{1}_{\{\tau^{PS} < \min(T, \tau^{RN}) \}} (V_{\tau^{PS}}^{CDS})^{+} \, | \mathcal{F}_t],
\end{equation}
where $R_{PS}$ is the recovery rate of PS, $\tau^{PS}$ and $\tau^{RN}$ are the default times of PS and RN, and $V_t^{CDS}$ is the price of a CDS without counterparty credit risk.

We associate $x_1$ with the Protection Seller and $x_2$ with the Reference Name, then CVA can be given by the solution of  (\ref{kolm_1})--(\ref{kolm_2})  with $\chi(t, x) = 0$ and $\psi(x) = 0$. Thus,
\begin{equation}
\begin{aligned}
		& \frac{\partial}{\partial t} V^{CVA}+ \mathcal{L} V^{CVA} = 0, \\
		& V^{CVA}(t, 0, x_2) = (1 - R_{PS}) V^{CDS}(t, x_2)^{+}, \quad V^{CVA}(t, x_1, 0) = 0, \\
		& V^{CVA}(T, x_1, x_2) = 0.
\end{aligned}
\end{equation}

Similar, Debt Value Adjustment represents the additional price associated with the default and defined as
\begin{equation}
	V^{DVA} = (1- R_{PB}) \mathbb{E}[\mathbbm{1}_{\{\tau^{PB} < \min(T, \tau^{RN}) \}} (V_{\tau^{PB}}^{CDS})^{-} \, | \mathcal{F}_t],
\end{equation}
where $R_{PB}$ and $\tau^{PB}$ are the recovery rate and default time of the protection buyer.

Here, we associate $x_1$ with the Protection Buyer and $x_2$ with the Reference Name, then, similar to CVA,  DVA can be given by the solution of  (\ref{kolm_1})--(\ref{kolm_2}),
\begin{equation}
\begin{aligned}
		& \frac{\partial}{\partial t} V^{DVA}+ \mathcal{L} V^{DVA} = 0, \\
		& V^{DVA}(t, 0, x_2) = (1 - R_{PB}) V^{CDS}(t, x_2)^{-}, \quad V^{DVA}(t, x_1, 0) = 0, \\
		& V^{DVA}(T, x_1, x_2) = 0.
\end{aligned}
\end{equation}

\paragraph{Bilateral CVA and DVA}

When we defined unilateral CVA and DVA, we assumed that either protection  buyer, or protection seller are risk-free. Here we assume that they are both risky. Then, 
The Credit Value Adjustment represents the additional price associated with the possibility of counterparty's default and defined as
\begin{equation}
	V^{CVA} = (1 - R_{PS}) \mathbb{E}[\mathbbm{1}_{\{\tau^{PS} < \min(\tau^{PB}, \tau^{RN}, T)\}} (V^{CDS}_{\tau^{PS}})^{+} \, | \mathcal{F}_t],
\end{equation} 

Similar, for DVA
\begin{equation}
	V^{DVA} = (1 - R_{PB}) \mathbb{E}[\mathbbm{1}_{\{\tau^{PB} < \min(\tau^{PS}, \tau^{RN}, T)\}} (V^{CDS}_{\tau^{PB}})^{-} \, | \mathcal{F}_t],
\end{equation} 


We associate $x_1$ with protection seller, $x_2$ with protection buyer, and $x_3$ with reference name. Here, we have a three-dimensional process. Applying three-dimensional version of (\ref{kolm_1})--(\ref{kolm_2}) with $\psi(x) = 0, \chi(t, x) = 0$, we get
\begin{equation}
	\label{CVA_pde}
\begin{aligned}
		& \frac{\partial}{\partial t} V^{CVA} + \mathcal{L}_3 V^{CVA} = 0, \\
		& V^{CVA}(t, 0, x_2, x_3) = (1 - R_{PS}) V^{CDS}(t, x_3)^{+}, \\
		& V^{CVA}(t, x_1, 0, x_3 ) = 0, \quad V^{CVA}(t, x_1, x_2, 0)  = 0, \\
		& V^{CVA}(T, x_1, x_2, x_3) = 0,
\end{aligned}
\end{equation}
and
\begin{equation}
\label{DVA_pde}
\begin{aligned}
		& \frac{\partial}{\partial t} V^{DVA} + \mathcal{L}_3 V^{DVA} = 0, \\
		& V^{DVA}(t, 0, x_2, x_3) = (1 - R_{PB}) V^{CDS}(t, x_3)^{-}, \\
		& V^{DVA}(t, x_1, 0, x_3 ) = 0, \quad V^{DVA}(t, x_1, x_2, 0)  = 0, \\
		& V^{DVA}(T, x_1, x_2, x_3) = 0,
\end{aligned}
\end{equation}
where $\mathcal{L}_3 f$ is the three-dimensional infinitesimal generator.




\end{document}