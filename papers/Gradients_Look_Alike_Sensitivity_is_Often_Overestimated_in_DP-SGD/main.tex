\documentclass{article}

\usepackage[left=3cm,right=3cm,top=4cm,bottom=4cm]{geometry}

\usepackage{tikz}
\usepackage{amsmath}
%%%%% NEW MATH DEFINITIONS %%%%%

\usepackage{amsmath,amsfonts,bm}
\usepackage{xifthen}

% Highlight a newly defined term
\newcommand{\newterm}[1]{{\bf #1}}

\def\eps{{\epsilon}}


% Utility for ticks 
\newcommand{\cmark}{\ding{51}}%
\newcommand{\xmark}{\ding{55}}%

% Theorem styles 
\theoremstyle{definition}
\newtheorem{theorem}{Theorem}[section]
\newtheorem{definition}{Definition}[section]
% \newtheorem{remark}{Remark}[theorem] %numbered remark
\newtheorem*{remark}{Remark} %unnumbered remark
\newtheorem{lemma}{Lemma}[section]
\newtheorem{prop}{Proposition}[section]
\newtheorem{corollary}{Corollary}[theorem]
\newtheorem{conjecture}{Conjecture}[section]
\newtheorem{assumption}{Assumption}[section]

\newtheorem{manualtheoreminner}{Theorem}
\newenvironment{manualtheorem}[1]{%
  \renewcommand\themanualtheoreminner{#1}%
  \manualtheoreminner
}{\endmanualtheoreminner}


% Math helper - standard function
\DeclareMathOperator*{\argmax}{arg\,max}
\DeclareMathOperator*{\argmin}{arg\,min}
\DeclareMathOperator{\support}{support}
\DeclareMathOperator{\MAX}{MAX}
\DeclareMathOperator{\term}{\texttt{term}}
\DeclareMathOperator*{\logsumexp}{log-sum-exp}
\DeclareMathOperator*{\TV}{TV}
\newcommand{\norm}[1]{\left\lVert#1\right\rVert}
\DeclarePairedDelimiter\set\{\}
\DeclarePairedDelimiter\abs{\lvert}{\rvert}%
\newcommand*{\mytop}{\mathrel{\scalebox{0.5}{$\top$}}}
\newcommand*{\mybot}{\mathrel{\scalebox{0.5}{$\bot$}}}
\newcommand*{\mydiese}{\mathrel{\scalebox{0.5}{$\#$}}}
\newcommand*{\myplus}{\mathrel{\scalebox{0.5}{$+$}}}
\newcommand*{\myminus}{\mathrel{\scalebox{0.5}{$-$}}}
\newcommand*{\bmg}{\bm{\gamma}}
\newcommand*{\bml}{\bm{\lambda}}

% MDP notation
\renewcommand{\S}{\mathcal{S}}
\newcommand{\X}{\mathcal{X}}
\newcommand{\A}{\mathcal{A}}
\newcommand{\T}{\mathcal{T}}
\newcommand{\M}{\mathcal{M}}
\newcommand{\B}{\mathcal{B}}
\newcommand{\Bset}{\mathfrak{B}}
\newcommand{\Dist}{\mathscr{P}}
\newcommand{\D}{\mathcal{D}}
\newcommand{\Real}{\mathbb{R}}
\renewcommand{\P}{\mathcal{P}}
\newcommand{\E}{\mathop{\mathbb{E}}}
\renewcommand{\H}{\mathcal{H}}
% \newcommand{\R}{\mathcal{R}}
% \newcommand{\C}{\mathcal{C}}

% Extended MDP notation
\newcommand{\Pstar}{p^{\star}}
\newcommand{\Rstar}{\bm{r}^{\star}}
\newcommand{\Cstar}{C^{\star}}
% \newcommand{\rmax}{\textsc{Rmax}}
\newcommand{\rmax}{r_{\mytop}}
\newcommand{\cmax}{\textsc{Cmax}}

\newcommand{\mstar}{m^{\star}}
\newcommand{\mhat}{\hat{m}}
\newcommand{\mopt}{m^{\star}}

\newcommand{\Phat}{\hat{p}}
\newcommand{\Rhat}{\hat{\bm{r}}}
\newcommand{\Chat}{\hat{C}}

% Math helper - custom function
\newcommand{\expwrtpi}[1]{\E_{\pi} [\sum_{t=0}^{\infty} \gamma^t #1(s_t, a_t)]}
\newcommand{\expangle}[1]{\langle #1  \rangle}

% helper function for return and constraints

% for value function, takes arguments:
% #1: policy 
% #2: the function of interest, R or C_i
% #3 (optional): the MDP for which this is estimated
\newcommand{\V}[3]{ %
    \ifthenelse{\isempty{#3}}%
    {V^{#1}(#2)}% #3 is empty 
    {V^{#1}_{#3}(#2)}%
}

\newcommand{\Q}[3]{
    \ifthenelse{\isempty{#3}}
    {Q^{#1}(#2)}% #3 is empty 
    {Q^{#1}_{#3}(#2)}%
}


\newcommand{\Adv}[3]{
    \ifthenelse{\isempty{#3}}
    {A^{#1}(#2)}% #3 is empty 
    {A^{#1}_{#3}(#2)}%
}

% careful diff notation
% 1: pi
% 2: R/C
% 3: M
\newcommand{\J}[3]{
    \ifthenelse{\isempty{#3}}
    {\mathcal{J}^{#1}_{#2}}% #3 is empty -> eg V^{\pi}(x ; R)
    {\mathcal{J}^{#1}_{#3,#2}}% -? eg V^{\pi}_{M}(x ; C)
    % {J_{#2}(#1)}% #3 is empty 
    % {J_{#2}(#1, #3)} %
}



\newcommand{\MRkern}{%
  \mkern-6.5mu
  \mathchoice{}{}{\mkern0.2mu}{\mkern0.5mu}%
}

% for value function, takes arguments:
% #1: policy 
% #2: the function of interest, R or C_i
% #3 (optional): the MDP for which this is estimated
% #4: variables to be given input (x) or (x,a)
\newcommand{\val}[4]{ %
    \ifthenelse{\isempty{#3}}%
    {v^{#1}_{#2}(#4)}% #3 is empty -> eg V^{\pi}(x ; R)
    {v^{#1}_{#3,#2}(#4)}% -? eg V^{\pi}_{M}(x ; C)
    % {V^{#1}_{#3}(#4 ;#2)}% -? eg V^{\pi}_{M}(x ; C)
    % {V_{#2}(#4 ; #1)}% #3 is empty -> eg V_R(x ; \pi)
    % {V_{#2}(#4 ;#1, #3)}% -? eg V_C(x ; \pi, M)
    % {#2 \MRkern V^{#1}_{#3}(#4)}% -? eg V^{\pi}_{M}(x ; C) # combines the letter V and R together
}

\newcommand{\qval}[4]{
    \ifthenelse{\isempty{#3}}
    {q^{#1}_{#2}(#4)}% #3 is empty -> eg V^{\pi}(x ; R)
    {q^{#1}_{#3,#2}(#4)}% -? eg V^{\pi}_{M}(x ; C)
    % {Q^{#1}(#4 ; #2)}% #3 is empty -> eg Q^{\pi}(x,a ; R)
    % {Q^{#1}_{#3}(#4 ;#2)}% -? eg Q^{\pi}_{M}(x,a ; C)
    % {Q_{#2}(#4 ; #1)}% #3 is empty -> eg Q_R(x,a ; \pi)
    % {Q_{#2}(#4 ;#1, #3)}% -? eg Q_C(x,a ; \pi, M)
}
\DeclareMathOperator*{\advantage}{Adv}

\newcommand{\adv}[4]{
    \ifthenelse{\isempty{#3}}
    {\advantage^{#1}_{#2}(#4)}% #3 is empty -> eg V^{\pi}(x ; R)
    {\advantage^{#1}_{#3,#2}(#4)}% -? eg V^{\pi}_{M}(x ; C)
    % {A^{#1}(#4 ; #2)}% #3 is empty -> eg Q^{\pi}(x,a ; R)
    % {A^{#1}_{#3}(#4 ;#2)}% -? eg Q^{\pi}_{M}(x,a ; C)
    % {A_{#2}(#4 ; #1)}% #3 is empty -> eg A_R(x,a ; \pi)
    % {A_{#2}(#4 ;#1, #3)}% -? eg A_C(x,a ; \pi, M)
}




\newcommand{\ci}{C}

\newcommand{\pib}{\pi_{b}}
\newcommand{\piopt}{\pi^{*}}
\newcommand{\pie}{\pi_{t}}

\newcommand{\lR}{\lambda_{R}}
\newcommand{\lC}{\lambda_{C}}
\newcommand{\ephi}{e_{\phi}}

\newcommand{\pr}{\text{Pr}}
\newcommand{\IS}{\text{IS}}
\newcommand{\CI}{\text{CI}}


% SPIBB symbols 
\newcommand{\EpsPib}{(\pi_b, e, \epsilon)}

\usepackage{todonotes}
\usepackage{amssymb}
\usepackage{mathtools}
\usepackage{amsthm}

\usepackage[utf8]{inputenc} %
\usepackage[T1]{fontenc}    %
\usepackage{hyperref}       %
\usepackage{url}            %
\usepackage{booktabs}       %
\usepackage{amsfonts}       %
\usepackage{nicefrac}       %
\usepackage{microtype}      %
\usepackage{xcolor}         %


\usepackage{booktabs}
\usepackage{soul}
\usepackage[numbers]{natbib}
\usepackage{comment}

\usepackage[justification=centering]{subfig}
\usepackage{graphicx}
\usepackage{cleveref}
\usepackage{fancyhdr}

\usepackage{enumitem}

\theoremstyle{plain}
\newtheorem{theorem}{Theorem}[section]
\newtheorem{proposition}[theorem]{Proposition}
\newtheorem{lemma}[theorem]{Lemma}
\newtheorem{corollary}[theorem]{Corollary}
\newtheorem{fact}[theorem]{Fact}
\theoremstyle{definition}
\newtheorem{definition}[theorem]{Definition}
\newtheorem{assumption}[theorem]{Assumption}
\theoremstyle{remark}
\newtheorem{remark}[theorem]{Remark}



\newcommand{\fix}{\marginpar{FIX}}
\newcommand{\new}{\marginpar{NEW}}




\newcommand{\anvith}[1]{\textcolor{brown}{2D: #1}}




\begin{document}

\date{}

\title{\Large \bf Gradients Look Alike: Sensitivity is Often Overestimated in DP-SGD}

\author{Anvith Thudi\\
  University of Toronto and Vector Institute\\
  Hengrui Jia \\
  University of Toronto and Vector Institute\\
  Casey Meehan\\
  University of California, San Diego\\
  Ilia Shumailov\\
  University of Oxford\\
  Nicolas Papernot\\
  University of Toronto and Vector Institute\\
}

\maketitle

\begin{abstract}
Differentially private stochastic gradient descent (DP-SGD) is the canonical approach to private deep learning. While the current privacy analysis of DP-SGD is known to be tight in some settings, several empirical results suggest that models trained on common benchmark datasets leak significantly less privacy for many datapoints. Yet, despite past attempts, a rigorous explanation for why this is the case has not been reached. Is it because there exist tighter privacy upper bounds when restricted to these dataset settings, or are our attacks not strong enough for certain datapoints? In this paper, we provide the first per-instance (i.e., ``data-dependent") DP analysis of DP-SGD. Our analysis captures the intuition that points with similar neighbors in the dataset enjoy better data-dependent privacy than outliers. Formally, this is done by modifying the per-step privacy analysis of DP-SGD to introduce a dependence on the distribution of model updates computed from a training dataset. We further develop a new composition theorem to effectively use this new per-step analysis to reason about an entire training run. Put all together, our evaluation shows that this novel DP-SGD analysis allows us to now \emph{formally} show that DP-SGD leaks significantly less privacy for many datapoints (when trained on common benchmarks) than the current data-independent guarantee. This implies privacy attacks will necessarily fail against many datapoints if the adversary does not have sufficient control over the possible training datasets. 
\end{abstract}


\section{Introduction}
\label{sec:Introduction}


The goal in top-$\size$ recommendation is to recommend to each
consumer a small set of $\size$ items from a large collection of
items~\cite{cremonesi2010performance}.  For example, Netflix may want
to recommend $\size$ appealing movies to each consumer.  Collaborative
Filtering (CF)~\cite{herlocker2002empirical,lee2012comparative} is a
common top-$\size$ recommendation method.  CF infers user interests by
analyzing partially observed user-item interaction data, such as user
ratings on movies or historical purchase
logs~\cite{kanagal2012supercharging}. The main assumption in CF is that
users with similar interaction patterns have similar interests.


Standard CF methods for top-$\size$ recommendation focus on making  suggestions  that accurately reflect the user's preference history. However, as  observed in previous work,  CF recommendations are generally biased toward  popular items, leading to a rich get richer effect~\cite{vargas2014improving,steck2011item}.  The major reasons for this are \textit{popularity bias} and \textit{sparsity} of CF interaction data (detailed in Section~\ref{sec:related-work}). In a nutshell, to maintain  accuracy, recommendations are generated from the dense regions of the data,  where the popular items lie.  

However,  accurately suggesting popular items, may not be satisfactory for the consumers. For example, in Netflix, an accuracy-focused movie recommender may recommend ``Star Wars: The Force Awakens'' to users who have seen ``Star Wars: Rogue One''.  But, those users are probably already aware of ``The Force Awakens''. Considering additional factors, such as novelty of recommendations,  can lead to more effective suggestions~\cite{cremonesi2010performance,Castells2015,zhang2008avoiding,ziegler2005improving,zhang2012auralist}. 
%Second, accuracy-focused models typically achieve a   overall item-space coverage across their recommendations,  whereas high item-space coverage helps providers of the items increase revenue
%, users satisfaction since they are  likely already aware of or can find these items on their own.  

Focusing on popular items also adversely affects the satisfaction of  the providers of the items. This is because  accuracy-focused models typically achieve a  low overall item space coverage across their recommendations, whereas   high item space coverage helps providers of the items increase their revenue~\cite{vargas2014improving,Castells2015,adomavicius2011maximizing,anderson2006thelongtail, yin2012challenging,adomavicius2012improving}.
%accuracy-focused models typically achieve a

In contrast to the relatively small number of popular items, there are copious  {\it long-tail\/} items that have fewer observations (e.g., ratings) available. More precisely,  using the Pareto  principle (i.e.,~the $80/20$ rule),  long-tail items can be defined as items that generate the lower $20\%$ of observations~\cite{yin2012challenging}. Experimentally we found that these items correspond to almost $85\%$ of the items in several datasets (Sections~\ref{sec:Notation} and \ref{sec:Experiments}). %Table~\ref{tab:DatasetStatsticsSmall})


As previously shown, one way to improve the novelty of top-$\size$ sets is to recommend interesting long-tail items~\cite{cremonesi2010performance,ge2010beyond}.  The intuition  is that since they have fewer observations available,  they are more likely to be unseen~\cite{Kaminskas:2016:DSN:3028254.2926720}.  
 %For example, in online commerce,  newly added items are long-tail items that are yet to be discovered.  
Moreover, long-tail item promotion also results in higher overall coverage of the item space%, which increases profits for providers of the items
~\cite{vargas2014improving,Castells2015,zhang2008avoiding,zhang2012auralist,adomavicius2011maximizing,anderson2006thelongtail,yin2012challenging,jambor2010optimizing}. Because long-tail promotion reduces accuracy~\cite{steck2011item}, there are trade-offs to be explored.


%original submitted to ICDE
%This work studies three aspects of top-$\size$ recommendation: accuracy, novelty, and item-space coverage, and examines their trade-offs. In most previous work, predictions of a base recommendation system are re-ranked to handle their trade-offs~\cite{adomavicius2012improving,jambor2010optimizing,zhang2013personalize,wang2009portfolio}. Due to performance considerations, however, these techniques are not customized per user. For example,  parameters that balance the trade-off between novelty and accuracy are cross-validated at a global level.  This can be detrimental since users have varying preferences for  objectives such as long-tail novelty. We explore how to  automatically infer  user  preference for long-tail novelty, and how to leverage  it to correct  the popularity bias in standard recommender models. Our work does not rely on any additional contextual data, although such data, if available, can help promote newly-added long-tail items~\cite{agarwal2009regression,Saveski:2014:ICR:2645710.2645751}.

This work studies three aspects of top-$\size$ recommendation: accuracy, novelty, and item space coverage, and examines their trade-offs. In most previous work, predictions of a base recommendation algorithm are \textit{re-ranked} to handle these trade-offs~\cite{adomavicius2012improving,jambor2010optimizing,zhang2013personalize,wang2009portfolio}. The re-ranking models are computationally efficient but suffer from two drawbacks. First, due to performance considerations,  parameters that balance the trade-off between novelty and accuracy  are not customized per user. Instead they are cross-validated at a global level.  This can be detrimental since users have varying preferences for  objectives such as long-tail novelty. Second,  the re-ranking methods are often limited to a specific base recommender  that may be sensitive to dataset density. 
As a result, the datasets are pruned and the problem is studied in dense settings~\cite{adomavicius2012improving,ho2014likes}; but real world  scenarios are often sparse~\cite{kanagal2012supercharging,liu2017experimental}.   
% Because  dataset density can impact the performance of most base recommenders (like R-SVD), which in turn affects the performance of the re-ranking model, 

\iffalse
We address these limitations by directly inferring  user  preference for long-tail novelty  from interaction data.  This  allows us to customize the re-ranking  per user, and design a \textit{generic} framework, which resolves the second problem. In particular, since the long-tail novelty preferences are estimated independently of any base  recommender model, we can  plug-in an appropriate base recommender w.r.t. the dataset sparsity.% including ones that are more suitable for sparse settings.  

Modelling  user  preference for  long-tail novelty using only item popularity statistics, e.g., the average popularity of rated items as in~\cite{jugovac2017efficient}, disregards additional information like whether the user found the item interesting and the long-tail preferences of other users  of the items. \iffalse To incorporate them, we introduce the notion of  \emph{item long-tail importance}. Both  user long-tail preferences and item long-tail importance are dependent:  a user has high preference for discovering long-tail items if she is interested in important long-tail items, and an item that is associated with many of these kinds of users is likely to be more important.  We propose a joint optimization framework to directly learn,  from interaction data, both the users' long-tail preferences and the  items' long-tail importance. \fi
We propose an optimization approach that  incorporates  this information and  directly learns,  from interaction data, the users' long-tail novelty preferences.

Next, we use these learned preferences  to design a  top-$\size$ recommendation framework thats is generic, and provides customized balance between accuracy, novelty, and coverage. We refer to it as framework as GANC.  Using GANC, we design a novel algorithm, {\it Ordered Sampling-based Locally Greedy (OSLG)\/}, that relies on the learned long-tail novelty preferences  to scalably correct for popularity bias. Our work does not rely on any additional contextual data, although such data, if available, can help promote newly-added long-tail items~\cite{agarwal2009regression,Saveski:2014:ICR:2645710.2645751}. In summary:
\fi

We address the first limitation by directly inferring  user  preference for long-tail novelty  from interaction data.   Estimating these  preferences  using only item popularity statistics, e.g., the average popularity of rated items as in~\cite{jugovac2017efficient}, disregards additional information, like whether the user found the item interesting or the long-tail preferences of other users  of the items. We propose an approach that  incorporates  this information and  learns the users' long-tail novelty preferences from interaction data.

This approach allows us to customize the re-ranking  per user, and  design a \textit{generic} re-ranking framework, which resolves the second limitation of prior work. In particular, since the long-tail novelty preferences are estimated independently of any base recommender, we can  plug-in an appropriate one w.r.t. different factors, such as the dataset sparsity.

Our top-$\size$ recommendation framework, \textbf{GANC}, is \textbf{G}eneric, and provides customized balance between \textbf{A}ccuracy, \textbf{N}ovelty, and \textbf{C}overage. % Moreover, based on the learned long-tail novelty preferences, we also design a novel algorithm, {\it Ordered Sampling-based Locally Greedy (OSLG)\/}, that relies on the learned long-tail novelty preferences  to scalably correct for popularity bias. 
Our work does not rely on any additional contextual data, although such data, if available, can help promote newly-added long-tail items~\cite{agarwal2009regression,Saveski:2014:ICR:2645710.2645751}. In summary:

%Consider  the following toy example:
\vspace{-0.2cm}
\begin{table}[htb]
\centering
\scriptsize
%\small
\begin{tabular}{ccccccc} 
%\toprule
%&\multirow{2}{*}{}&\multicolumn{7}{c}{Ratings}\\
& & \cellcolor{blue!35}$w_1$ &\cellcolor{blue!18} $w_2$ & $\dots$ &\cellcolor{blue!8} $w_{89}$  &\cellcolor{blue!8} $w_{99}$   
\\
&   &$i_1$&$i_2$&$\dots$&$i_{89}$&$i_{90}$\\ 
\cmidrule(r){3-7} 	 
%\midrule
\cellcolor{red!35}$\theta_1$  &$u_1 $   &5 &   & $\dots$ &  &   \\
\cellcolor{red!28}$\theta_2$  &$u_2$     &5 &    & $\dots$ &  &  \\
 $\theta_3=?$  &$\bf u_3$  &5 &  &   $\dots$ &  &  \\
\cellcolor{red!10}$\theta_4$ & $u_4$  &  &5   & $\dots$ & &\\ 
\cellcolor{red!10}$\theta_5$ & $u_5$  &  & 5  & $\dots$ & &\\ 
$\theta_6=?$  & $\bf u_6$ & &5  &      $\dots$& &  \\ 
 & & $\hdots$  &$\hdots$   &$\hdots$   &$\hdots$   &$\hdots$  \\
%\midrule 
\cmidrule(r){3-7} 	 
\multicolumn{2}{c}{item pop.}  & 3  & 3  & $\dots$ &50&60\\  
%\bottomrule
%$ f_i$    &3  &3  &1  &3  &1  &2  \\  \hline
\end{tabular}
%#.
\caption{Simplified user-item interaction data. The user long-tail novelty preference ($\theta_u$), item long-tail importance weight ($w_i$) are highlighted. Darker colors indicate larger values. } \label{tab:example}
\end{table} 
\vspace{-0.2cm}
\begin{example}  
In Table~\ref{tab:example}, we are interested in estimating $\theta_3$ and $\theta_6$,  the long-tail preference of users $u_3$ and $u_6$ who have each rated a single movie. Additional ratings for other users  are not included here.  Considering only rating information, we observe $i_1$ and $i_2$ are  equally popular $|\mathcal{U}_{i_1}^{\trainset}| = |\mathcal{U}_{i_2}^{\trainset}|=3$, and $r_{31}=5$ and $r_{62}=5$. Using Eq.~\ref{eq:tfidf-risk}  we have $\theta_3 = \theta_6$. However, if we were given the long-tail preferences of the each item's user set, specifically that $u_1$ and $u_2$ have high long-tail preference (darker red), while $u_4$ and $u_5$ have lower long-tail preference (lighter red), we could conclude $i_1$ is a more important long-tail item compared to $i_2$ (indicated by a darker blue shade for $w_1$), and we expect  $\theta_3 \geq \theta_6$.

% On the other hand, if we knew that $u_4$ and $u_5$ have lower long-tail preference, we could conclude $i_2$ is a  less significant long-tail item. Therefore, However, if we  consider the long-tail preferences of other users, we may reason differently.    We need another variable $w_i$ which captures this information. 
%we would conclude that $u_3$ has higher long-tail preference compared to $u_6$, since the users $i_1$ is a more prominent long-tail item. 

% Relying only  on item popularity information, we would  conclude   $u_3$ and $u_6$ have equal long-tail preference, since $i_1$ and $i_2$ are  equally popular. However, considering  the second column,  long-tail preference of users,  long-tail importance for each item,  which captures the long-tail preference of its users. Since  that  both users of $i_1$ have high long-tail preference while  the users of $i_2$ have lower preference,  we may conclude $i_1$ is a more important long-tail item compared to $i_2$. Therefore, $u_3$'s long-tail preference should be at least as large as $u_6$'s preference. Specifically, consider two  items $i_1$ and $i_2$, with the following rating data: $i_1=\{u_1:5, u_2:5, u_3:5 \}$, $i_2=\{u_4:5, u_5:5, u_6:5\}$.  

%Table~\ref{tab:example} shows  simplified rating data. We want an estimate of the long-tail preference of $u_3$ and $u_6$, who have each  rated a single movie.  Relying only  on movie popularity information, we would  conclude   $u_3$ and $u_6$ have similar long-tail preference, since $m_1$ and $m_2$ are  equally popular. However, considering the long-tail preferences of other users of those movies, we may reason differently: since $u_1$ and $u_2$ have high long-tail preference, and $u_4$ and $u_5$ have low long-tail preference, $m_1$ is a more prominent long-tail item compared to $m_2$. Therefore, it is likely that $u_3$ has higher long-tail preference compared to $u_6$.considering the long-tail preferences of other users of those movies, we may reason differently.  For example, 
\label{ex:running}
\end{example}



%------------------------------

\iffalse
\begin{example}
Table~\ref{tab:example} shows rating data for a simplified system. %Note the user-item interaction matrix is sparse.
For this example, we define popular movies as those that have received  three or more ratings; $\{m_1, m_2, m_4\}$ are popular and  $\{m_3, m_5, m_6\}$ are niche movies. We observe $u_1$ and $u_3$  have rated relatively popular movies (risk-averse) while $u_2$ and $u_4$ have rated niche movies (risk-loving). 
\label{ex:running}
\end{example}

\begin{table}[htb]
\centering
\scriptsize
\begin{tabular}{ccccccc} 
\toprule
			&$m_1$ &$m_2$   &$m_3$    &$m_4$   &$m_5$ &$m_6$  \\ \hline 
$u_1 $ &5  &4  & - &-  &-  &-   \\
$u_2$  &-  &-  &-  &-  &5  &5   \\
$u_3$  &-  &4  &-  &5  &-  &-   \\
$u_4$  &-  &-  &3  &-  &-  &4   \\ 
$u_5$  &5  &-  &-  &3  &-  &-   \\ 
$u_6$  &4  &2  &-  &4  &-  &-   \\ 
\bottomrule
%$ f_i$    &3  &3  &1  &3  &1  &2  \\  \hline
\end{tabular}
\caption{User-Movie rating data} \label{tab:example}
\end{table}

It is essential to consider consumer characteristics in designing recommender systems so that they promote long-tail items to the right group of users and spread demand evenly between hit and niche items.  

\fi





%------------------------------
\iffalse
\begin{table}[htb]
\centering
\scriptsize
\begin{tabular}{ccccccc} 
\toprule
			&$m_1$ &$m_2$   &$m_3$    &$m_4$   &$m_5$ &$m_6$  \\ \hline 
$u_1 $ &\textbf{5}  & \textbf{4}  &\textcolor{gray}{ 1.2} &-  &-  &-   \\
$u_2$  &-  &-  &-  &-  & \textbf{5}  &\textbf{5}   \\
$u_3$  &-  &\textbf{4}  &-  &\textbf{5}  &-  &-   \\
$u_4$  &-  &-  &\textbf{3}  &-  &-  &\textbf{4}   \\ 
$u_5$  &\textbf{5}  &-  &-  &\textbf{3}  &-  &-   \\ 
$u_6$  &\textbf{4}  &\textbf{2}  &-  &\textbf{4}  &-  &-   \\ 
\bottomrule
%$ f_i$    &3  &3  &1  &3  &1  &2  \\  \hline
\end{tabular}
\caption{User-Movie rating data} \label{tab:example}
\end{table}
% $\mathcal{P}^1= \{ \mathcal{P}_1^1 \{i_1,i_2,i_3\}, \mathcal{P}_2^1:\{i_2,i_3,i_5\}  \}$
 %$\mathcal{P}^2= \{ \mathcal{P}_1^2: \{i_1,i_2,i_3\}, \mathcal{P}_2^2:\{i_2,i_5,i_6\}  \}$
 %$\mathcal{P}^3= \{ \mathcal{P}_1^3: \{i_7,i_8,i_9\}, \mathcal{P}_2^3:\{i_{10},i_{11},i_{12}\}  \}$
\begin{table}[htb]
\centering
\tiny
\begin{tabular}{ccc} 
\toprule
		&$u_1$&$u_2$  \\ \hline 
$\mathcal{P}^1 $ & $\{i_1,i_2,i_3\}$ & $\{i_2,i_3,i_5\} $ \\
$\mathcal{P}^2$ & $\{i_1,i_2,i_3\}$ & $\{i_2,i_5,i_6\} $ \\
$\mathcal{P}^3$ & $\{i_7,i_8,i_9\}$ & $\{i_{10},i_{11},i_{12} \}$ \\
\bottomrule
%$ f_i$    &3  &3  &1  &3  &1  &2  \\  \hline
\end{tabular}
\caption{Top-$\size$ allocations to users.} \label{tab:paretoExamples}
\end{table}
\fi


\iffalse
When considering long-tail items, it is important to consider consumers' willingness  to explore niche or unpopular items and their propensity towards similar items. In particular, they can be characterized by their  {\it risk degree\/} and {\it focusing degree\/}, respectively.  We compute these estimates  based on historical rating information. The following example further describes these notions in the context of movie rating data. 

\begin{example}  
Table~\ref{tab:example} shows rating data for a simplified system with $6$ users, $6$ movies, and $3$ genres. $m_i^{j}$ implies that movie $m_i$ belongs to genre $j$. Note the user-item interaction matrix is sparse. 
  For this setting, we define popular movies as those that have received  three or more ratings; $\{m_1, m_2, m_4\}$ are popular and  $\{m_3, m_5, m_6\}$ are niche movies. We now profile the users according to their risk and focusing degree. E.g., $u_1$ has rated relatively popular movies belonging to the same genre (risk-averse, high focusing degree); $u_2$ has rated niches movies in the same genre (risk-loving, high focusing degree); $u_3$ has rated popular movies in two different genres (risk-averse, low focusing degree), and $u_4$ has rated niches movies in two different genres (risk-loving, low focusing degree). 
\label{ex:running}
\end{example}
\begin{table}[htb]
\centering
\tiny
\begin{tabular}{ccccccc} 
\toprule
			&$m_1^{1}$ &$m_2^{1}$   &$m_3^{2}$    &$m_4^{3}$   &$m_5^{3}$ &$m_6^{3}$  \\ \hline 
$u_1 $ &5  &4  &-  &-  &-  &-   \\
$u_2$  &-  &-  &-  &-  &5  &5   \\
$u_3$  &-  &4  &-  &5  &-  &-   \\
$u_4$  &-  &-  &3  &-  &-  &4   \\ 
$u_5$  &5  &-  &-  &3  &-  &-   \\ 
$u_6$  &4  &2  &-  &4  &-  &-   \\ 
\bottomrule
%$ f_i$    &3  &3  &1  &3  &1  &2  \\  \hline
\end{tabular}
\caption{User-Movie rating data} \label{tab:example}
\end{table}
It is essential to consider these consumer characteristics in designing recommender systems so that they promote long-tail items to the right group of users and spread demand evenly between the hit and niche items.  
\fi
\iffalse
\begin{center}
\begin{figure*}[tp]
%\scalebox{0.5}{%
\resizebox{1\textwidth}{!}{%
%\small%\addtolength{\tabcolsep}{5pt}% below sums to 8
\begin{tabularx}{1.5\textwidth}{>{\hsize=2.5\hsize}X>{\hsize=2.5\hsize}X>{\hsize=0.5\hsize}X>{\hsize=0.5\hsize}X>{\hsize=0.5\hsize}X>{\hsize=0.5\hsize}X>{\hsize=0.5\hsize}X>{\hsize=0.5\hsize}X}
    \multirow{12}{*}{\includegraphics[scale=0.3]{codeForExample/popularity-movie.png}} & \multirow{12}{*}{\includegraphics[scale=0.3]{codeForExample/scatterplot.png}} & & & & & & \\
%   & &               &       &       &       &       &       \\
    & &\multicolumn{1}{l|}{}               &$m_1^{g1}$   	&$m_2^{g1}$    	&$m_3^{g2}$    &$m_4^{g2}$      &$m_5^{g3}$    \\ \cline{3-8}%\hline
    & &\multicolumn{1}{l|}{u1}          &5  &5  &-  &-   &-  \\
    & &\multicolumn{1}{l|}{u2}    		&-  &-  &4  &4  &5  \\
    & &\multicolumn{1}{l|}{u3}   			&1  &2  &1  &-  &-   \\
    & &\multicolumn{1}{l|}{u4}     		&1  &-  &-  &-  &-  \\
    & &               &       &       &       &       &       \\
    & &               &       &       &       &       &       \\
    & &               &       &       &       &       &       \\
    & &               &       &       &       &       &	\\
    \\
\end{tabularx}}
\caption{User-Movie interaction data a) Popularity-Movie histogram b)Movie genres/clusters c) User-Movie rating data} \label{fig:example}
\end{figure*}
\end{center}
\fi



%We propose a novel approach that allows us to  promote long-tail items in a targeted manner, thereby improving the novelty of top-$\size$ sets, the overall item-space coverage across recommendations, while maintaining reasonable levels of accuracy.

%Next, we integrate these learned preferences  in a generic  top-$\size$ recommendation framework to provide customized balance between accuracy and coverage.

%sequentially make recommendations, while adjusting its parameters with regard to the set of top-$\size$ recommendations made so far. However, since  sequential parameter updates  cause  scalability issues, we propose a sampling based algorithm. This variant of our framework, called {\it Ordered Sampling-based Locally Greedy (OSLG)\/},  allows us to  correct for the popularity bias in recommendations with regard to individual user long-tail preferences. 

%ICDE submission
%Our framework differs with  prior work in the following aspects:  unlike~\cite{adomavicius2011maximizing,adomavicius2012improving,zhang2013personalize,ho2014likes},  the long-tail preference personalization in our framework is learned rather than optimized using cross-validation or parameter tuning. In other words, our personalization method is independent of the underlying base  recommendation models.  Moreover, our framework is  generic. This enables us to  plug-in several base recommenders, and evaluate their  effectiveness without requiring  extensive tuning for the accuracy and coverage trade-off. 


%\vspace{-2.8pt}
\begin{itemize}

\item  We examine various measures for estimating user long-tail novelty preference in Section~\ref{sec:lt-pref} and formulate an optimization problem  to directly learn users' preferences for long-tail  items from interaction data in Section~\ref{sec:learning-lt-pref}. %In addition, we introduce several heuristics for measuring the user preference for less common items from historical rating data.% 

\item  We integrate the user preference estimates into GANC %, a generic re-ranking framework that provides customized balance between accuracy, novelty, and coverage 
(Section~\ref{sec:RiskbasedReranking}), and  introduce {\it Ordered Sampling-based Locally Greedy (OSLG)\/}, a scalable algorithm that relies  on user long-tail preferences to correct the popularity bias (Section~\ref{sec:optimizationAlgorithm}).
%We introduce OSLG, a scalable algorithm that relies  on user long-tail preferences to  maximize item space coverage \textcolor{red}{while maintaining acceptable levels of accuracy} (Section~\ref{sec:optimizationAlgorithm}).

\item   We conduct an extensive empirical study and evaluate performance from  accuracy, novelty, and coverage perspectives (Section~\ref{sec:Experiments}).  We use five  datasets with varying density and difficulty levels. %:  Netflix, MovieTweetings, and MovieLens (100K, 1M, 10M). 
  In contrast to most related work,  our evaluation considers realistic settings that include a large number of infrequent  items and users. %This enables us to study the impact of  data density on the performance trade-offs of several  state of the art top-$\size$ recommendation algorithms. %   %,  and use the all-items ranking protocol~\cite{steck2013evaluation,vargas2014improving}, where performance is measured using all items with train data. to evaluate the performance of several  state of the art top-$\size$ recommendation algorithms 
 
\item Our empirical results confirm that the performance of re-ranking models is impacted by the underlying   base recommender and the dataset density. Our generic approach enables us to easily incorporate a suitable base recommender to devise an effective solution for both dense and sparse settings. In dense settings, we use the same base recommender as existing re-ranking approaches, and we outperform them in accuracy and coverage metrics. For sparse settings, we plug-in a more suitable base recommender, and devise an effective solution that is competitive with existing top-$\size$ recommendation methods in accuracy and novelty. 

%Directly estimating the long-tail novelty preferences allows us to customize re-ranking per user, and  devise a generic framework.   
 
\end{itemize}

Section~\ref{sec:related-work} describes related work. Section~\ref{sec:conclusion} concludes.

\section{Background}
\label{sec:background}

%\SHAN{I think the background section is too long, maybe we can remove the SO3, SE3 parts, only keep SIM3, ellipsoid, SDF}

Rigid body orientation, pose, and similarity are represented using the $\text{SO}(3)$, $\text{SE}(3)$, and $\text{SIM}(3)$ Lie groups, respectively, defined as:
%
\begin{gather}
\label{eq:LieGroups}
\scaleMathLine{\begin{aligned}
\text{SO}(3) &\triangleq \crl{ \bfR \in \bbR^{3 \times 3} \mid \bfR^\top\bfR = \bfI, \det(\bfR) = 1},\\
\text{SE}(3) &\triangleq \crl{ \begin{bmatrix} \bfR & \bft\\\mathbf{0}^\top & 1 \end{bmatrix} \in \bbR^{4 \times 4} \,\bigg\vert\, \bfR \in SO(3), \bft \in \bbR^3}, \\
\text{SIM}(3) &\triangleq \crl{ \begin{bmatrix} s\bfR & \bft\\\mathbf{0}^\top & 1 \end{bmatrix} \in \bbR^{4 \times 4} \,\bigg\vert\, \bfR \in SO(3), \bft \in \bbR^3, s \in \bbR}.
\end{aligned}}
\raisetag{10ex}
\end{gather}
%
We overload $\bfxi_\times$ to denote a mapping from a vector in $\bbR^3$ or $\bbR^6$ or $\bbR^7$ to the Lie algebra $\mathfrak{so}(3)$, $\mathfrak{se}(3)$, or $\mathfrak{sim}(3)$, associated with the Lie groups in \eqref{eq:LieGroups}, defined as:
%
\begin{gather}
\label{eq:LieAlgebras}
\begin{aligned}
\mathfrak{so}(3) &\triangleq \crl{\bfxi_\times = \begin{bmatrix}0 & -\xi_3 & \xi_2\\\xi_3 & 0 & -\xi_1\\-\xi_2 & \xi_1 & 0 \end{bmatrix} \,\bigg\vert\, \bfxi \in \bbR^3},\\
\mathfrak{se}(3) &\triangleq \crl{\bfxi_\times = \begin{bmatrix} \bftheta_\times & \bfrho \\ \mathbf{0}^\top & 0 \end{bmatrix}\,\bigg\vert\, \bfxi = \begin{bmatrix} \bfrho \\ \bftheta\end{bmatrix} \in \bbR^6},\\
\mathfrak{sim}(3) &\triangleq \crl{\bfxi_\times = \begin{bmatrix} \sigma \bfI + \bftheta_\times & \bfrho \\ \mathbf{0}^\top & 0 \end{bmatrix} \,\bigg\vert\, \bfxi = \begin{bmatrix} \bfrho \\ \bftheta \\ \sigma \end{bmatrix} \in \bbR^7}.
\end{aligned}
\raisetag{12ex}
\end{gather}
%
We define an infinitesimal change of a Lie group element $\bfT$ via a left perturbation $\exp\prl{\bfxi_\times}\bfT$, using the exponential map $\exp\prl{\bfxi_\times}$ to retract a Lie algebra element $\bfxi_\times$ to the Lie group. Please refer to \cite[Ch.7]{BarfootBook} or \cite{Gao2017SLAM} for details. 

The coarse shape of a rigid body is represented using a \emph{quadric shape} \cite[Ch.3]{MVGBook}, $\crl{ \bfx \in \bbR^3 \mid \underline{\bfx}^\top \bfQ \underline{\bfx} \leq 0}$, where $\underline{\bfx} \triangleq [\bfx^\top, 1]^\top$ denotes the homogeneous coordinates of $\bfx$ and $\bfQ \in \bbR^{4 \times 4}$ is a symmetric matrix. An axis-aligned ellipsoid centered at the origin:
%
\begin{equation}
\label{eq:ellipsoid}
\mathcal{E}_{\bfu} \triangleq \crl{\bfx \in \bbR^3 \mid \bfx^\top \bfU^{-\top}\bfU^{-1}\bfx \leq 1},
\end{equation}
%
where $\bfU \triangleq \diag(\bfu)$ and the elements of the vector $\bfu \in \bbR^3$ specify the lengths of the semi-axes of $\mathcal{E}_{\bfu}$. An ellipsoid $\mathcal{E}_{\bfu}$ is a special case of a quadric shape with $\bfQ = \diag(\bfU^{-2},-1)$. 
% Instead of as the collection of points $\bfx$ contained in it, 
A quadric shape can also be defined in dual form, as the set of planes $\underline{\boldsymbol{\pi}} = \bfQ\underline{\bfx}$ that are tangent to the shape surface at each $\bfx$. This dual quadric surface definition is $\crl{ \bfpi \in \bbR^3 \mid \underline{\bfpi}^\top \bfQ^* \underline{\bfpi} = 0}$, where $\bfQ^* = \adj(\bfQ)$ is the adjugate of $\bfQ$.
%\footnote{If $\bfQ$ is invertible, $\bfQ^* = \adj(\bfQ) \triangleq \det(\bfQ)\bfQ^{-1}$ can be simplified to $\bfQ^* = \bfQ^{-1} = \diag(\bfU^2, -1)$ due to the scale-invariance of the dual quadric definition.}.
A dual quadric defined by $\bfQ^*$ can be scaled, rotated, or translated by a similarity transform $\bfT \in \text{SIM}(3)$ as $\bfT \bfQ^* \bfT^\top$. Similarity, a dual quadric can be projected to a lower-dimensional space by a projection matrix $\bfP = \begin{bmatrix} \bfI & \mathbf{0} \end{bmatrix}$ as $\bfP \bfQ^* \bfP^\top$.

The fine shape of a rigid body is represented as $\crl{\bfx \in \bbR^3 \mid f(\bfx) \leq 0}$ using the \emph{signed distance field} of a set $\calS \subset \bbR^3$:
%
\begin{equation}
f(\bfx) = \begin{cases}
  -d(\bfx,\partial\calS), & \bfx \in \calS,\\
  \phantom{-}d(\bfx,\partial\calS), & \bfx \notin \calS,
\end{cases}
\end{equation}
%
where $d(\bfx,\partial\calS)$ denotes the Euclidean distance from a point $\bfx \in \bbR^3$ to the boundary $\partial\calS$ of $\calS$.























%% \subsection{SE3}

%This section introduces the necessary mathmatical background. 
%The transformation in $\text{SE}(3)$ can be expressed as:
%\begin{equation}
%\bfT \triangleq \begin{bmatrix} \bfR & \bft\\\mathbf{0}^\top & 1 \end{bmatrix} \in \text{SE}(3)
%\end{equation}
%We overload $\bftheta_\times$ to denote the mapping from an axis-angle vector $\bftheta \in \mathbb{R}^3$ to a $3 \times 3$ skew-symmetric matrix $\bftheta_\times \in \mathfrak{so}(3)$ and the mapping from a position-rotation vector $\bfxi \in \mathbb{R}^6$ to a $4 \times 4$ twist matrix $\bfxi_\times \in \mathfrak{se}(3)$. We define an infinitesimal change of pose $\bfT \in SE(3)$ using a left perturbation $\exp\prl{\bfxi_\times}\bfT \in \text{SE}(3)$ (see~\cite[Ch.7]{BarfootBook}).

%% \subsection{SIM(3)}

%We use the space $\text{SIM}(3)$ of similarity transformations to capture scale $s$ in addition to pose: 
%\begin{equation}
%\bfT \triangleq \begin{bmatrix} s\bfR & \bft\\\mathbf{0}^\top & 1 \end{bmatrix} \in \text{SIM}(3).
%\end{equation}
%We also use $\bfxi$ to denote the corresponding Lie algebra $\mathfrak{sim}(3)$, as in \cite{Gao2017SLAM}:
%\begin{equation}
%\label{eq:sim3_to_SIM3}
%\scaleMathLine[0.91]{
%\begin{aligned}
%\mathfrak{sim}(3) \triangleq
%\left\{\bfxi_\times=
%\left[\begin{array}{cc}
%{\sigma \bfI+\bfphi_\times} & {\bfrho} \\
%\mathbf{0}^\top & {0}
%\end{array}\right] \in \mathbb{R}^{4 \times 4} \;\bigg\vert\; \bfxi = \left[\begin{array}{c}
%{\bfrho} \\
%{\bfphi} \\
%{\sigma}
%\end{array}\right] \in \mathbb{R}^{7}\right\}
%\end{aligned}
%}
%\end{equation}
%We define the operator:
%%\NA{what is the operator $\underline{\bfx}^\circledcirc$?}
%%\SHAN{I don't think we will use $\underline{\bfx}^\circledcirc$ anywhere}
%\begin{equation}
%\underline{\bfx}^\odot \triangleq \begin{bmatrix} \bfI_3 & -\bfx_\times & \bfx\\ \mathbf{0}^\top & \mathbf{0}^\top & 0 \end{bmatrix} \in \mathbb{R}^{4 \times 7}
%\end{equation}




%\begin{definition*}
%The \textit{signed distance field} of a set $\calS \subset \mathbb{R}^3$ is
%\begin{equation}
%f(\bfx) = \begin{cases}
%  -d(\bfx,\partial\calS) & \bfx \in \calS\\
%  d(\bfx,\partial\calS) & \bfx \notin \calS
%\end{cases}
%\end{equation}
%where $d(\bfx,\partial\calS)$ is the distance from a point $\bfx \in \mathbb{R}^3$ to the set boundary $\partial\calS$, and we use $d$ as a shorthand notation to denote this distance.
%\end{definition*}


%\begin{definition*}
%\textit{Huber error function} \cite{Huber1964Robust} with parameter $\delta > 0$ is:
%\begin{equation}
%\rho(r) \triangleq 
%\begin{cases}
%\frac{1}{2}r^2 & |r|\leq \delta,\\
%\delta(|r|-\frac{1}{2}\delta) & \text{else}.
%\end{cases}
%\end{equation}
%\end{definition*}
%whose gradient can be computed as 
%\[
%  \frac{\partial \rho(r)}{\partial r}
%  =\left\{\begin{array}{ll}
%    r & |r| \leq \delta \\
%    \text{sign}(r)\delta  & \text{ else}. 
%    \end{array}\right.
%\] 

%An axis-aligned ellipsoid centered at the origin can be described as:
%\begin{equation}
%\mathcal{E}_{\bfu} \triangleq \crl{\bfx \mid \bfx^\top \bfU^{-\top}\bfU^{-1}\bfx \leq 1},
%\end{equation}
%where $\bfU \triangleq \diag(\bfu)$ and the elements of the vector $\bfu = [a,b,c]^{\top}$ are the lengths of the semi-axes of $\mathcal{E}_{\bfu}$. In homogeneous coordinates, $\mathcal{E}_{\bfu}$ can be represented as a quadric surface~\cite[Ch.3]{MVGBook}, $\crl{ \bfx \mid \underline{\bfx}^\top \bfQ_{\bfu} \underline{\bfx} \leq 0}$, where $\bfQ_{\bfu} = \mathbf{diag}(\bfU^{-2},-1)$. This describes the ellipsoid as a collection of points lying on its surface. 

%Alternatively, a quadric can be defined by the set of planes $\underline{\boldsymbol{\pi}} = \bfQ_{\bfu}\underline{\bfx}$ tangent to its surface at $\underline{\bfx}$. This dual quadric surface is defined as $\crl{ \bfpi \mid \underline{\bfpi}^\top \bfQ_{\bfu}^* \underline{\bfpi} = 0}$, where $\bfQ_{\bfu}^* = \mathbf{adj}(\bfQ_{\bfu})$~\footnote{If $\bfQ_{\bfu}$ is invertible, $\bfQ_{\bfu}^* = \mathbf{adj}(\bfQ_{\bfu}) = \det(\bfQ_{\bfu})\bfQ_{\bfu}^{-1}$ can be simplified to $\bfQ_{\bfu}^* = \bfQ_{\bfu}^{-1} = \diag(\bfU^2, -1)$ due to the scale-invariance of the dual quadric definition.}. A dual quadric defined by $\bfQ_{\bfu}^*$ can be transformed by $\bfT \in SE(3)$ to another reference frame as $\bfQ^* = \bfT \bfQ_{\bfu}^* \bfT^\top$, which can be projected to a lower-dimensional space by $\bfP \in \mathbb{R}^{3\times 4}$ as $\bfP \bfQ^* \bfP^\top$. 
%$\bfQ^*$ can be parameterized as:
%\begin{equation}
%\label{eq:ellipsoid}
%\begin{aligned}
%\bfQ^*
%&=
%\mathbf{T} \bfQ_{\bfu}^* \mathbf{T}^{\top}=
%\left[\begin{array}{cc}
%\mathbf{R} & \bft \\
%\mathbf{0}^{\top} & 1
%\end{array}\right]\left[\begin{array}{cc}
%\mathbf{U} \mathbf{U}^{\top} & \mathbf{0} \\
%\mathbf{0} & -1
%\end{array}\right]\left[\begin{array}{ll}
%\mathbf{R}^{\top} & \mathbf{0} \\
%\bft^{\top} & 1
%\end{array}\right] \\ 
%&=
%\begin{pmatrix} 
%\mathbf{R} \mathbf{U} \mathbf{U}^\top \mathbf{R}^\top -  \bft \bft^\top & - \bft \\ -\bft^\top & -1
%\end{pmatrix}
%\end{aligned}
%\end{equation}




\section{A Per-Instance Analysis of DP-SGD}
\label{sec:analysis}





We now present our new analysis of DP-SGD which removes the data-independent nature of the per-step and composition analyses currently used for DP-SGD. The impact of this new analysis is presented in Section~\ref{sec:main_body_emp_results}, where we show that many datapoints have much better privacy than suggested by the current analysis of DP-SGD, explaining the failure of many privacy attacks in practice.

The technical contributions that led to this are two-fold. At the per-step level, we generalize the notion of sensitivity to what we term \emph{sensitivity distributions}; given two datasets, sensitivity distributions capture how similar the updates between mini-batches from either dataset are. At the composition step, we generalize RDP composition to do accounting by the ``expected" intermediate privacy losses during training as opposed to the largest possible intermediate privacy losses. Together, we can now study the data-dependent behaviour of DP-SGD.





\subsection{Sensitivity Distribution Generalize the $(\epsilon,\delta)$-DP Analysis}
\label{ssec:eps_delta_case}




We first turn to $(\epsilon,\delta)$-DP, which is not used to analyze DP-SGD for composition reasons, but allows for simpler expressions to demonstrate the improvements afforded by particular data-dependent random variables we call \textit{sensitivity distributions}. In particular, in this section we will first consider the classical data-independent $(\epsilon,\delta)$-DP analysis of the sampled Gaussian mechanism $M$ and show how one can generalize this analysis and obtain tighter per-instance $(\epsilon,\delta)$-DP guarantees.


Recall that for an update rule $U$, the Gaussian mechanism is defined as $A(X) = U(X) + N(0,\sigma)$. The sampled Gaussian mechanism is then defined as $M(X) = A(\mathbf{X_B})$ where $\mathbf{X_B}$ is a mini-batch constructed from a dataset $X$ by sampling each datapoint $x \in X$ independently with probability $\mathbb{P}_{x}(1)$ (unless otherwise stated we think of $X_B$, not bold-face, as a specific mini-batch). Note, one assumes the sampling probability $\mathbb{P}_{x}(1)$ is only a function of $x$ and not the full dataset $X$, e.g., some fixed constant. The classical data-independent $(\epsilon,\delta)$-DP analysis of the sampled Gaussian mechanism follows two steps. First, we derive the guarantee for just the Gaussian mechanism. To do so, one first assumes a data-independent sensitivity bound $C_U$ on $U$: for all $X,X' = X \cup \{x^*\}$ we have $||U(X) - U(X')||_{2} \leq C_{U}$. This can be achieved by clipping the output values of $U$ to have a small norm. With this constant $C_U$ one has that the Gaussian mechanism $A$ gives the $(\epsilon,\delta)$-DP guarantee $\epsilon = C_{\delta,\sigma} C_U$ for some constant $C_{\delta,\sigma}$ depending on $\delta$ and $\sigma$ where $\sigma$ is the standard deviation of the added Gaussian noise~\footnote{For example, one can take $C_{\delta, \sigma} = \frac{\sqrt{2 \ln (1.25/\delta)}}{\sigma}$~\citep{dwork2014algorithmic}.}. To then analyze the sampled Gaussian mechanism one would incorporate the privacy gain from not sampling $x^*$ sometimes~\citep{beimel2014bounds}\citep{kasiviswanathan2011can} to get the privacy guarantees of $M$ %
as $(\epsilon',\delta')$-DP where $\epsilon' = \ln( \mathbb{P}_{x^*}(1) e^{C_{\delta,\sigma}~C_U} + \mathbb{P}_{x^*}(0))$ and $\delta' = \mathbb{P}_{x^*}(1) \delta$. Here $\mathbb{P}_{x^*}(0)=1-\mathbb{P}_{x^*}(1)$, and this gain in privacy by sometimes not using the datapoint is called privacy amplification by sampling.






Towards tightening this analysis into a per-instance analysis, let $$\Delta_{U,x^*}(X_B) \coloneqq ||U(X_B) - U(X_B \cup \{x^*\})||_2$$
then $\Delta_{U,x^*}(\mathbf{X_B})$ is a data-dependent random variable which we will call a \emph{sensitivity distribution}: it captures the change in the distribution of mini-batches updates caused by adding a point $x^*$ to the mini-batch. The classical data-independent analysis only (implicitly) uses sensitivity distributions via the data-independent bound $|\Delta_{U,x^*}(X_B)| \leq C_{U}~\forall X_B$. Instead, we will show how to directly use the $L_p$ norms $||\Delta_{U,x^*}(\mathbf{X_B})||_{p} = (\mathbb{E}_{X_B} (\Delta_{U,x^*}(X_B)^p))^{1/p}$ (or generally the $L_p$ norm of some monotonic transformation of $\Delta_{U,x^*}(\mathbf{X_B})$) to obtain tighter per-instance privacy guarantees. Furthermore, when using $p < \infty$, this analysis will be able to translate the phenomenon that many mini-batches produce similar updates into better privacy guarantees (as the sensitivity distribution concentrates at smaller values and hence has smaller $p$-norms). To emphasize this ability, past work that studied sampling relied mainly on the intuition that by sampling a datapoint with low probability, we have any given step often does not leak privacy for that point as it was not used. This translates to better privacy guarantes. By using the $L_p$ norms of sensitivity distributions with $p< \infty$ we make an additional observation, which is that if many of the other mini-batches produce the same update, then effectively we have an even lower probability of an attacker observing a noticeable shift due solely to that point.  %


In particular, recall that to prove per-instance $(\epsilon,\delta)$-DP for a pair of datasets $X,X'= X \cup \{x^*\}$ we need to bound $\mathbb{P}(M(X') \in S) \leq e^{\epsilon} \mathbb{P}(M(X) \in S) + \delta$ and $\mathbb{P}(M(X) \in S) \leq e^{\epsilon} \mathbb{P}(M(X') \in S) + \delta$. As a proof-of-concept on the role of sensitivity distributions, we present an analysis for the first inequality in Corollary~\ref{cor:eps_delta_sens} \footnote{We will later turn to R\'enyi-DP which provides both inequalities.}. Inspecting Corollary~\ref{cor:eps_delta_sens}, we see that it approximately follows the formula given by the classical analysis except the role of $C_U$ is replaced with a dependency on how concentrated $\Delta_{U,x^*}(X_B)$ is at small values (the $L_p$ norm of an exponential applied to $\Delta_{U,x^*}(X_B)$). When enough mini-batches provide updates more similar than the upper-bound $C_U$, the per-instance guarantee of Corollary~\ref{cor:eps_delta_sens} will significantly beat the classical data-independent analysis, as demonstrated for MNIST and CIFAR10 in Appendix~\ref{ssec:eps_delta_experiments}.




\begin{corollary}
\label{cor:eps_delta_sens}
For $p \in (1,\infty)$, let $a_p = \mathbb{P}_{x^*}(1) (\mathbb{E}_{x_{B}}(e^{C_{\delta,\sigma} \Delta_{U,x^*}(X_B)p}))^{1/p}$, $\epsilon' = \ln(a_p^{\frac{1}{1-1/p}}\delta'^{\frac{-1}{p-1}} + \mathbb{P}_{x^*}(0)) $ and $\delta'' = \mathbb{P}_{x^*}(1)\delta + \delta'$. Then, for $X' = X \cup \{x^*\}$ $$\mathbb{P}(M(X') \in S) \leq e^{\epsilon'} \mathbb{P}(M(X) \in S) + \delta''$$

\end{corollary}


\emph{Proof Sketch:} The proof of Corollary~\ref{cor:eps_delta_sens} follows two stages. First by expanding mini-batch sampling and applying Holder's inequality, we can show

\begin{multline}
    \mathbb{P}(M(X') \in S) \leq \mathbb{P}_{x^*}(1) \mathbb{E}_{X_B}(e^{C_{\delta,\sigma} \Delta_{U,x^*}(X_B)p})^{1/p} \mathbb{P}(M(X) \in S)^{1-1/p} \\ + \mathbb{P}_{x^*}(1)\delta + \mathbb{P}_{x^*}(0) \mathbb{P}(M(X) \in S)
\end{multline}

This is stated as Lemma~\ref{lem:holder_approach}. One then follows the proof strategy of Proposition 3 in \citet{mironov2017renyi} to convert an inequality bounding $\mathbb{P}(M(X') \in S)$ with a power of $\mathbb{P}(M(X) \in S)$ into an $(\epsilon,\delta)$-DP inequality. The full proof of Corollary~\ref{cor:eps_delta_sens} is in Appendix~\ref{proof:eps_delta_sens}.





\subsection{Per-Instance R\'enyi-DP Analysis for DP-SGD}

With now an understanding of the power of incorporating $L_p$ norms of sensitivity distributions (upto some transformations) into DP analyses, we turn to analyzing the R\'enyi-DP guarantees of DP-SGD. R\'enyi-DP is more suited to compose the guarantees of each step of DP-SGD to obtain the guarantees for an entire training run. We first present per-step analyses for the sampled Gaussian mechanism, and then a new composition theorem to reason about the entire training run. We then discuss how to analyze DP-SGD for general update rules, i.e., not just the sum of gradients.

Our per-step analyses will focus on integer values of $\alpha$ for R\'enyi-DP. This is for simplicity, as R\'enyi divergences $D_{\alpha}(P||Q) \coloneqq \frac{1}{\alpha -1} \ln \mathbb{E}_{x \sim Q} (\frac{P}{Q})^{\alpha}$ are increasing in their order $\alpha$, hence we can bound the guarantee for any $\alpha$ by the guarantee for $\lceil \alpha \rceil$. In terms of notation, we will use ${X_B}^{\tilde \alpha} = ({X_B}^1,\cdots,{X_B}^{\alpha})$ to denote $\alpha$ mini-batches from $X$ (sampled independently if random). Analogously we use ${X'_B}^{\tilde \alpha}$ and $X'_B$ for $X'$.


\subsubsection{Per-Instance R\'enyi DP for the Sum Update Rule}
\label{ssec:sum_update}

In Section~\ref{ssec:eps_delta_case} we introduced the sensitivity distribution $\Delta_{U,x^*}(\mathbf{X_B}) = ||U(\mathbf{X_B}) - U(\mathbf{X_B \cup \{x^*\}})||_2$ and showed how directly leveraging its $L_p$ norms gives better per-instance DP analysis. In particular, how $p < \infty$ allows one to take advantage of expected sensitivity over mini-batches. However, for update rules of the form $U(X_B) = \sum_{x_i \in X_B} g(x_i)$ (i.e., the sum update rule typically used in DP-SGD) we have $\Delta_{U,x^*}(\mathbf{X_B})$ is always a constant: $\Delta_{U,x^*}(\mathbf{X_B}) = ||g(x^*)||_2$. Hence an analysis of the sampled Gaussian mechanism that used $\Delta_{U,x^*} \coloneqq \sup_{X_B \sim X} \Delta_{U,x^*}(X_B)$ would effectively capture all $L_p$ norms of the sensitivity distribution $\Delta_{U,x^*}(X_B)$ for the sum update rule. We state such a per-instance version of the classical RDP analysis of the sampled Gaussian mechanism below.



\begin{theorem}
\label{thm:easy_renyi_dp}
    For integer $\alpha > 1$, the sampled Gaussian mechanism with noise $\sigma$ and sampling probability $\mathbb{P}_{x^*}(1)$ for $x^*$ is $(\alpha,\epsilon)$-R\'enyi DP for:

    
    $$\epsilon = \frac{1}{\alpha -1} \ln(\sum_{k=0}^{\alpha} {\alpha \choose k} (1 - \mathbb{P}_{x^*}(1))^{\alpha -k} \mathbb{P}_{x^*}(1)^k \exp(\frac{\Delta_{U,x^*}^2(k^2 - k)}{2 \sigma^2}))$$

        
\end{theorem}

Note that some key variables in Theorem~\ref{thm:easy_renyi_dp} are the sampling rate $\mathbb{P}_{x^*}(1)$ (increasing it typically increases the bound), the standard deviation of noise $\sigma$ (increasing it typically decreases the bound), and the sensitivity upper-bound over minibatches $\Delta_{U,x^*}$ (increasing it typically increases the bound). The proof strategy is analogous to \citet{mironov2019r} and replaces their sensitivity upper-bound with the per-instance bound $\Delta_{U,x^*}$ on the mini-batches. 


\begin{proof}
    Following the calculation of Theorem 4 in \citet{mironov2019r} we have 

    
    $$D(M(X')| M(X)) \\ \leq D_{\alpha}((1-\mathbb{P}_{x^*}(1))N(0,\sigma^2) + \mathbb{P}_{x^*}(1)N(\Delta_{U,x^*},\sigma^2)|| N(0,\sigma^2))$$
    
    where instead of using $||U(X_B) - U(X_B \cup x^*)||_2 \leq 1$ for $X_B$ batches from $X$ as in the proof of the theorem we used $||U(X_B) - U(X_B \cup x^*)||_2 \leq \Delta_{U,x^*}$ by the definition of $\Delta_{U,x^*}$. Similarly, we have $D(M(X)| M(X')) \leq D_{\alpha}( N(0,\sigma^2) || (1-\mathbb{P}_{x^*}(1))N(0,\sigma^2) + \mathbb{P}_{x^*}(1)N(\Delta_{U,x^*},\sigma^2))$.

    Analogous to Corollary 7 in \citet{mironov2019r} we have

    \begin{multline}
        D_{\alpha}((1-\mathbb{P}_{x^*}(1))N(0,\sigma^2) + \mathbb{P}_{x^*}(1)N(\Delta_{U,x^*},\sigma^2)|| N(0,\sigma^2)) \\ \geq D_{\alpha}( N(0,\sigma^2) || (1-\mathbb{P}_{x^*}(1))N(0,\sigma^2) + \mathbb{P}_{x^*}(1)N(\Delta_{U,x^*},\sigma^2))
    \end{multline}
    
    where instead of using $\nu(x) = 1 -x$ we use $\nu(x) = \Delta_{U,x^*} - x $ which still satisfies $\nu = \nu^{-1}$

    Now one follows the integer $\alpha$ calculations in Section 3.3 of \citet{mironov2019r}, to conclude our theorem statement. The only change is that instead of computing $\mathbb{E}_{z \sim N(0,\sigma^2)}(\frac{N(1,\sigma^2)}{N(0,\sigma^2)})^k$ one computes $\mathbb{E}_{z \sim N(0,\sigma^2)}(\frac{N(\Delta_{U,x^*},\sigma^2)}{N(0,\sigma^2)})^k$ and following analogous calculation get $\leq \exp(\frac{\Delta_{U,x^*}^2(k^2 - k)}{2 \sigma^2})$.

    
\end{proof}






\subsubsection{A Generalized R\'enyi-DP Composition}
\label{ssec:comp}




With now an analysis for the per-step guarantees from DP-SGD (which as currently implemented uses the sum update rule), we now resolve how to obtain a per-instance RDP bound for a full training run with DP-SGD without the limitations of past composition theorem (see Section~\ref{ssec:back_full_comp} for a discussion on past composition bounds). In particular, we provide a composition theorem that bounds the overall per-instance privacy leakage by the ``expected" per-instance privacy guarantee at each step when training on a given dataset. This is presented in Theorem~\ref{thm:better_composition}.


More technically, we once again generalize the classical analysis to look at arbitrary $L_p$ norms, but now for the composition step. The classical R\'enyi DP composition theorem implicity uses the $L_\infty$ norm of the distribution of per-step guarantees at each step (coming from the distribution of possible models at each step as training is random), and Theorem~\ref{thm:better_composition} generalizes this to arbitrary $L_p$ norms of the exponential of the per-step guarantees (with some constants to scale). By using $L_p$ norms with $p < \infty$ we take advantage of cases where many models have better privacy guarantees than the worst model. 




\begin{theorem}
\label{thm:better_composition}

    Let $p \in (1,\infty)$ and consider a sequence of functions $X_1(x_1),$ $X_2(x_1,x_2),\cdots X_n(x_{n-1},x_n)$ where $X_{i}$ is a density function in the second argument for any fixed value of the first argument, except $X_1$ which is a density function in $x_1$. Consider an analogous sequence $Y_1(x_1),\cdots, Y_n(x_{n-1}, x_n)$. Then letting $X = \prod_{j=1}^{n} X_j$ be the density function for a sequence $x_1,\cdots,x_n$ generated according to the Markov chain defined by $X_i$, and similarly $Y$, we have 
    
    
    \begin{multline}
        D_{\alpha}(X || Y)  \leq \frac{1}{\alpha -1} (\sum_{i=0}^{n-2} \frac{(p-1)^i}{p^{i+1}} \ln (\mathbb{E}_{X_1,\cdots X_{n-(i+1)}}  (e^{(g_p^{i}(\alpha) -1)D_{g_p^{i}(\alpha)}(X_{n-i}|| Y_{n-i})p}))) \\ + \frac{1}{\alpha -1} (\frac{p-1}{p})^{n-1} (g_p^{n-1}(\alpha) -1)D_{g_p^{n-1}(\alpha)}(X_{1}|| Y_{1}) 
    \end{multline}


    where $g_p(\alpha) = \frac{p}{p-1} \alpha - \frac{1}{p}$ and $g_p^{i}$ is $g_p$ composed $i$ times, where we defined $g_p^{0}(\alpha) = \alpha$.
\end{theorem}

Note some key variables in Theorem~\ref{thm:better_composition} are a flexible parameter $p$ (which we'll soon describe leads to blow-up as it gets smaller), and the distribution of per-step guarantees $D_{g_p^{i}(\alpha)}(X_{n-i}|| Y_{n-i})$ (the more concentrated at $0$ they are, the smaller the upper-bound). The proof relies on using an induction argument to continually break up the composition and is presented below.

\begin{proof}


The proof follows by repeating a similar reduction as Theorem~\ref{thm:composition}. First note 
    
\begin{multline}
    \int (X_1 \cdots X_n)^{\alpha} (Y_1 \cdots Y_n)^{1 - \alpha} dx_1 \cdots dx_n \\  = \int (X_1 \cdots X_{n-1})^{\alpha - 1/p} (Y_1 \cdots Y_{n-1})^{1 - \alpha}  \\ (\int X_n^{\alpha} Y_n^{1- \alpha} dx_n) (X_1 \cdots X_{n-1})^{1/p} dx_1 \cdots dx_{n-1}
    \\ \leq ( \int (X_1 \cdots X_n)^{\frac{p}{p-1}\alpha - \frac{1}{p-1}} (Y_1 \cdots Y_n)^{ \frac{p}{p-1}(1 - \alpha)}  dx_1 \cdots dx_{n-1})^{\frac{p-1}{p}} \\ (\int (\int X_n^{\alpha} Y_n^{1- \alpha} dx_n)^p (X_1 \cdots X_{n-1}) dx_1 \cdots dx_{n-1})^{1/p} 
\end{multline}

where the first equality was from using the markov property, and the last inequality was from Holder's inequality with Holder constant $p$. Do note that, defining $g_p(\alpha) = \frac{p}{p-1}\alpha - \frac{1}{p-1}$, we have $\frac{p}{p-1}(1 - \alpha) = 1 - g_p(\alpha)$. So now looking at the first term of the upper-bound we got, we are back to the original expression but with $\alpha \rightarrow g_p(\alpha)$ and $n \rightarrow n-1$, and an exponent to $\frac{p-1}{p}$. Note the second term is an expectation over the $n-1$ model state of the Markov chain. Do note $\int X_n^{\alpha} Y_n^{1- \alpha} dx_n$ is $e^{(\alpha -1)D_{\alpha}(X_{n-i}|| Y_{n-i})}$ for a fixed $n-1$ model state (i.e., fixed $x_{n-1}$ ). So repeating this step on the first term until we are left only with an integral over $x_1$ we have

\begin{multline}
    \int (X_1 \cdots X_n)^{\alpha} (Y_1 \cdots Y_n)^{1 - \alpha} dx_1 \cdots dx_n \\  
    \leq (\prod_{i=0}^{n-2} (\mathbb{E}_{X_1,\cdots X_{n-(i+1)}}  ((e^{(g_p^{i}(\alpha) -1)D_{g_p^{i}(\alpha)}(X_{n-i}|| Y_{n-i})})^p))^{\frac{(p-1)^i}{p^{i+1}}}) \\ ( (e^{(g_p^{n-1}(\alpha) -1)D_{g_p^{n-1}(\alpha)}(X_{1}|| Y_{1})})^p)^{\frac{(p-1)^{n-1}}{p^n}}
\end{multline}

So now noting $$D_{\alpha}(X || Y) = \frac{1}{\alpha -1} \ln(\int (X_1 \cdots X_n)^{\alpha} (Y_1 \cdots Y_n)^{1 - \alpha} dx_1 \cdots dx_n)$$

we conclude by the previous expression that 

\begin{multline}
        D_{\alpha}(X || Y) \leq \frac{1}{\alpha -1} (\sum_{i=0}^{n-2} \frac{(p-1)^i}{p^{i+1}}  \ln (\mathbb{E}_{X_1,\cdots X_{n-(i+1)}}  ((e^{(g_p^{i}(\alpha) -1)D_{g_p^{i}(\alpha)}(X_{n-i}|| Y_{n-i})p}))) \\ + \frac{1}{\alpha -1} ((\frac{(p-1)^{n-1}}{p^n}) \ln ((e^{(g_p^{n-1}(\alpha) -1)D_{g_p^{n-1}(\alpha)}(X_{1}|| Y_{1})})^p)) 
    \end{multline}

which completes the proof as the last term simplifies to the term stated in the theorem.
\end{proof}


\paragraph{Applying to DP-SGD.} To interpret Theorem~\ref{thm:better_composition} in the context of DP-SGD, we can let $X_i$ be the distribution of the $i'th$ model update (for a fixed $(i-1)'th$ model) when training on one dataset $D$, and similarly $Y_i$ when training on a neighbouring dataset $D'$. Letting $Train_{DP-SGD}$ denote the Markov chain of the intermediate model updates when using DP-SGD, we have the maximum over the bound given by Theorem~\ref{thm:better_composition} on $D_{\alpha}(Train_{DP-SGD}(D)||Train_{DP-SGD}(D'))$ and $D_{\alpha}(Train_{DP-SGD}(D')||Train_{DP-SGD}(D))$ provides our per-instance RDP guarantee for DP-SGD.




\paragraph{Balancing the value of $p$.}To understand the dependence on $p$ in Theorem~\ref{thm:better_composition}, consider for a moment $p =2$. In this case, we observe that at the $i$'th step, we need to compute a R\'enyi divergence of order $\sim 2^{i} \alpha$. It is known that the R\'enyi divergence $D_{c}(P||Q)$ grows with $c$ \citep{van2014renyi}, and in the case of the Gaussian mechanism, this growth is linear with $c$~\citep{mironov2017renyi}. Hence this exponential growth in the R\'enyi divergence order can prove impractical as a useful tool to analyze DP-SGD. However, as $p \rightarrow \infty$ we see that the growth on the order of the divergence shrinks.

Yet, by taking larger $p$ values we are effectively taking larger $L_{p}$-norms of the per-step guarantees seen in training and so effectively turn to worst-case per-step analysis as $p \rightarrow \infty$. Hence it is desirable to choose $p$ just sufficient for there to not be a significant blow-up in the order of the divergences for a given $n$. This can be done by analyzing how $g_{p}^{i}(\alpha)$ grows.

\begin{fact}\label{fact:p_control}
    If $p = O(n)$ then $g_{p}^i(\alpha) \leq 2 \alpha~\forall i \leq n$. In particular, $p = 3n$ works for sufficiently large $n$.
\end{fact}

The proof follows from direct calculations with the formula for $g_{p}(\alpha)$. 

\begin{proof}

Note that $g_{p}(\alpha) \leq \frac{p}{p-1}\alpha$ hence $g_{p}^{i}(\alpha) \leq (\frac{p}{p-1})^{i}\alpha$. From this we see showing $\frac{p}{p-1}^{n} \leq 2$ for $p = O(n)$ will imply $g_p^{i}(\alpha) \leq 2 \alpha~\forall 1 \leq n$.

Note we can equivalently show $ln(\frac{p}{p-1}) = \ln(p) - \ln(p-1) \leq \frac{\ln (2)}{n}$. But if we take $p = 3n$ note $\ln(3n) - \ln(3n-1) \leq \frac{1}{3n-1}$ by the derivative of $\ln(x) \leq \frac{1}{3n-1}$ for $x \geq 3n-1$. So it suffices to show $\frac{1}{3n-1} \leq \frac{\ln(2)}{n}$, but this is true for sufficiently large $n$.

    
\end{proof}


\paragraph{Estimating Theorem~\ref{thm:better_composition}}


In cases where one does not know the expectations used in Theorem~\ref{thm:better_composition} analytically, as is the case with DP-SGD when it is applied to deep learning, one can resort to empirically estimating the means. Our goal is to understand how much better our data-dependent guarantees are than the data-independent baseline for DP-SGD on common datasets. Hence, we wish to estimate the expression of Theorem~\ref{thm:better_composition} (or specifically the per-step contributions) with an error
$c \epsilon$ for $c < 1$ where $\epsilon$ is the data-independent guarantee (per-step). 


The following fact focuses on estimating the $i'th$ per-step guarantee with an error relative to the worst-case per-step guarantee when $p = 3n$ as is used in our experiments. In particular, we have the $i'th$ per-step guarantee 

\begin{multline*}
    \frac{1}{\alpha-1} \frac{(p-1)^i}{p^{i+1}} \ln (\mathbb{E}_{X_1, \cdots, X_{n-(i+1)}} f) \\ \coloneqq \frac{1}{\alpha-1} \frac{(p-1)^i}{p^{i+1}} \ln (\mathbb{E}_{X_1,\cdots X_{n-(i+1)}}  ((e^{(g_p^{i}(\alpha) -1)D_{g_p^{i}(\alpha)}(X_{n-i}|| Y_{n-i})})^p))
\end{multline*}

is less than the data-independent per-step privacy guarantee $\epsilon/n$ if $\mathbb{E}_{X_1,\cdots X_{n-(i+1)}} f \leq e^{(\alpha-1) 3 \epsilon}$ for $p = 3n$. Hence we describe the number of samples needed to estimate $\mathbb{E} f$ with precision relative to $e^{(\alpha-1) 3 \epsilon}$ (with high probability), which can be done in a constant number of samples relative to the data-independent bound.

\begin{fact}\label{fact:estimating}
    Let $\epsilon/n$ be the classical $\alpha$-R\'enyi DP guarantee for the $i'th$ step, and $\epsilon'/n$ be the analogous $2\alpha$-R\'enyi DP guarantee for the $i'th$ step. Then for $l \geq \frac{- \ln(J)}{c^2} e^{6(\alpha-1)\epsilon - 3(2\alpha -1) \epsilon'}$ and $p = 3n$ with $n$ s.t $g_p^{n-1} \leq 2\alpha$, we have $\mathbb{P}(|\mathbb{E}^{l} f - \mathbb{E}f| \geq c e^{(\alpha-1) 3 \epsilon}) \leq J$. Here $\mathbb{E}^l$ denotes the empirical mean over $l$ samples.
\end{fact}

The proof follows from Hoeffding's inequality.

\begin{proof}
    

For the given choice of $p$ and $\alpha$ we have $g_{p}^{i} \leq 2\alpha$ hence $D_{g_p^{i}(\alpha)}(X_{n-i} || Y_{n-i}) \leq D_{2 \alpha}(X_{n-i} || Y_{n-i}) \leq \epsilon'/n$ where $\epsilon'$ is determined by $\epsilon$ (when accounting for the increase due to the $\alpha$-order). Hence we have that $f \leq e^{3 (2\alpha -1) \epsilon'}$.

By Hoeffding's inequality we can hence conclude $\mathbb{P}(|\mathbb{E}^l f - \mathbb{E}f| \geq c e^{3(\alpha-1)\epsilon}) \leq e^{-\frac{e^{6(\alpha-1)\epsilon}c^2 l}{e^{3(2\alpha - 1) \epsilon'}}}$. Now upper-bounding the right-hand side by $J$ and rearranging to isolate for $l$, we can conclude the stated condition on $l$.

\end{proof}


\subsubsection{Per-Instance R\'enyi DP for General Updates}



The results of Section~\ref{ssec:sum_update} and Section~\ref{ssec:comp} provide a complete per-instance RDP analysis of the current implementation of DP-SGD. In particular, with the per-step update rule being the sum of gradients. In this section we ask, how should we analyze per-step guarantees (and hence DP-SGD given our composition theorem) if the update rule is not the sum? In general, the worst-case sensitivity over mini-batches may be far higher than the expected sensitivity over mini-batches (unlike the sum update rule), meaning the analysis from Theorem~\ref{thm:easy_renyi_dp} may be as bad as a data-independent analysis. For example, the typical update rule used in normal SGD is the mean update rule. However, $\Delta_{U,x^*}(X_B)$ for the mean update rule is the difference between the update for the datapoint $x^*$ and the mean of the updates on $X_B$; this difference is not the same for all minibatches $X_B$ and hence would be overestimated with the analysis of Theorem~\ref{thm:easy_renyi_dp}. One could resolve this issue of overestimating sensitivity by using the $L_p$ norms $||\Delta_{U,x^*}(\mathbf{X_B})||_{p} = (\mathbb{E}_{X_B} (\Delta_{U,x^*}(X_B)^p))^{1/p}$ with $p < \infty$ in the RDP analysis of the sampled Gaussian mechanism, as was done in the $(\epsilon,\delta)$-DP case. %
However, we are not aware of an approach to do this for R\'enyi DP.




Instead, we show how a new sensitivity distribution comparing all mini-batches $X_B$ in $X$ to all mini-batches $X'_B$ in $X' = X \cup \{x^*\}$, as opposed to just a single point $x^*$ as done with $\Delta_{U,x^*}(X_B)$, is amenable to a R\'enyi-DP analysis of the sampled Gaussian mechanism that does not look at the maximum privacy leakage over mini-batches. %
If the distribution of all updates given by $X$ is similar to the distribution of all updates given by $X'$, then analysis with this new sensitivity distribution can be expected to beat the current data-independent analysis.


Specifically, given $\alpha$ minibatches sampled from $X$, ${X_{B}}^{\tilde \alpha} \sim X$, and a particular minibatch sampled from $X'$, $X'_B \sim X'$, we define a new sensitivity distribution for $\alpha$-R\'enyi DP as follows: 

\begin{multline*}
\Delta_{U,\alpha}({X_{B}}^{\tilde \alpha}, X'_B) \coloneqq \sum_{i} ||U({X_B}^i)||_2^2 - (\alpha-1) ||U(X'_B)||_2^2 - ||\Delta_{\alpha}({X_B}^{\tilde \alpha},X'_B)||_2^2
\end{multline*}


where $\Delta_{\alpha}({X_B}^{\tilde \alpha},X'_B) = (\sum_{i} U({X_B}^i)) - (\alpha - 1) U(X'_B)$. When letting ${X_{B}}^{\tilde \alpha}$ and $X'_B$ be random variables, $\Delta_{U,\alpha}$ effectively compares all the mini-batches in $X'$ to all the $\alpha$-tuples of mini-batches in $X$. The $\alpha$-tuples appear here due to their equivalence with an expectation over mini-batches to the power of $\alpha$ which appears when analyzing $\alpha$-R\'enyi divergences. As described earlier, comparing this to the previous sensitivity distribution $\Delta_{U,x^*}(X_B)$, we see that this new sensitivity will compare all mini-batches in $X$ to all mini-batches in $X'$ (and not just to a point $x^*$) and hence captures more ``global" changes in updates due a datapoint $x^*$.




Theorem~\ref{thm:renyi_dp_sens} states the R\'enyi diveregence of the sampled Gaussian mechanism $M$ between two arbitrary datasets using $\Delta_{U, \alpha}$ through applying a transformation on its fixed $X'_B$ marginal values and taking its expectation over $X'_B$. Taking the maximum of the bounds for $D_{\alpha}(M(X)||M(X'))$ and $D_{\alpha}(M(X')||M(X))$ from Theorem~\ref{thm:renyi_dp_sens} where $X' = X \cup \{x^*\}$ gives a per-instance guarantee of $M$ for $X,X'$.




\begin{theorem}
\label{thm:renyi_dp_sens}
Let $\alpha > 1$ be an integer. Given two arbitrary datasets $X,X'$, the sampled Gaussian mechanism $M$ with noise $\sigma$ satisfies: 

$$D_{\alpha}(M(X')||M(X)) \leq \frac{1}{(\alpha-1)} \mathbb{E}_{X_B} (\ln (\mathbb{E}_{{X'_{B}}^{\tilde \alpha}}(e^{\frac{-1}{2\sigma^2}\Delta_{U,\alpha}({X'_{B}}^{\tilde \alpha}, X_B)})))$$



\end{theorem}



Some key variables in Theorem~\ref{thm:renyi_dp_sens} is the standard deviation of noise $\sigma$ (increasing it decreases the upper-bound) and the sensitivity distribution $\Delta_{U,\alpha}({X_{B}}^{\tilde \alpha}, X'_B)$ (the more concentrated at $0$ it is, the smaller the upper-bound). The proof relies on convexity, which is always true for the second argument of the R\'enyi divergence $D_{\alpha}(A||B)$, and then direct calculations involving Gaussians. %

\begin{proof}

For simpler notation, we use $\mu_X = U(X)$. We proceed by taking $\alpha$ to be an integer (to use an expansion similar to Section 3.3 in \citet{mironov2019r}) and utilizing Theorem 12 in~\citet{van2014renyi}. We will let $N_{X_B} = N(\mu_{X_B},\sigma^2)$ where $\mu_{X_B} = U(X_B)$ as stated earlier.


We proceed to bound $D_{\alpha}(M(X') || M(X))$ for arbitrary $X',X$. Hence a completely analogous argument will allow us to also bound $D_{\alpha}(M(X) || M(X'))$ when $X'$ is specifically $X \cup \{x^*\}$. First note


\begin{multline}
    D_{\alpha}(M(X') || M(X)) = D_{\alpha}(\sum_{X'_B} \mathbb{P}(X'_B) N_{X'_B} || \sum_{X_B} \mathbb{P}(X_B) N_{X_B}) \\ \leq \sum_{X_B} \mathbb{P}(X_B) D_{\alpha}(\sum_{X'_B} \mathbb{P}(X'_B) N_{X'_B} || N_{X_B})
\end{multline}


where the last inequality is from the fact the divergence is convex in the second argument (Theorem 12 in~\citet{van2014renyi}). 

Now note 
\begin{multline}
    e^{(\alpha-1)D_{\alpha}(\sum_{X'_B} \mathbb{P}(X'_B) N_{X'_B} || N_{X_B})} \\ = \int (\sum_{X'_B}\mathbb{P}(X'_B) \frac{1}{(\sigma \sqrt{2\pi})^d} e^{\frac{-1}{2\sigma^2} |x - \mu_{X'_B}|^2})^{\alpha} (\frac{1}{(\sigma \sqrt{2\pi})^d} e^{\frac{-1}{2\sigma^2}|x - \mu_{X_B}|^2})^{1- \alpha} dx \\ = \sum_{{X'_B}^{\tilde \alpha}} \mathbb{P}({X'_B}^{\tilde \alpha}) \frac{1}{(\sigma \sqrt{2\pi})^d} \int e^{\frac{-1}{2\sigma^2} ( (\sum_{{X'_B}^i}|x- \mu_{{X'_B}^i}|^2) - (\alpha - 1)|x - \mu_{X_B}|^2)}
\end{multline}


where we expanded $(\sum_{X'_B}\mathbb{P}(X'_B) \frac{1}{(\sigma \sqrt{2\pi})^d} e^{\frac{-1}{2\sigma^2} |x - \mu_{X'_B}|^2})^{\alpha}$ by noting each term in the product is just iterating through all $\alpha$ tuples of mini-batches from $X'$.

Now note we can for now consider the integral in each dimension, as the overall integral is the product of each dimension. Also recall from the theorem statement that we define $$\Delta_{\alpha}({X'_B}^{\tilde \alpha},X_B) = (\sum_{i} \mu_{{X'_B}^i}) - (\alpha - 1) \mu_{X_B}$$ Hence (letting everything be one dimensional for now) we have

\begin{multline}
    (\sum_{{X'_B}^i}|x- \mu_{{X'_B}^i}|^2) - (\alpha - 1)|x - \mu_{X_B}|^2 \\ = x^2 - 2 \Delta_{\alpha}({X'_B}^{\tilde \alpha},X_B)x + \sum_{i} \mu_{{X'_B}^i}^2 - (\alpha-1) \mu_{X_B}^2 \\ = (x - \Delta_{\alpha}({X'_B}^{\tilde \alpha},X_B))^2 + \sum_{i} \mu_{{X'_B}^i}^2 - (\alpha-1) \mu_{X_B}^2 - \Delta_{\alpha}({X'_B}^{\tilde \alpha},X_B)^2
\end{multline}


Hence, we have 

\begin{multline}
    \int e^{\frac{-1}{2\sigma^2} ( (\sum_{{X'_B}^i}|x- \mu_{{X'_B}^i}|^2) - (\alpha - 1)|x - \mu_{X_B}|^2)} \\ = e^{\frac{-1}{2\sigma^2}(\sum_{i} {\mu_{{X'_B}^i}}^2 - (\alpha-1) \mu_{X_B}^2 - \Delta_{\alpha}({X'_B}^{\tilde \alpha},X_B)^2)} \int e^{\frac{-1}{2\sigma^2} (x - \Delta_{\alpha}({X'_B}^{\tilde \alpha},X_B))^2} \\ = \sigma \sqrt{2 \pi} e^{\frac{-1}{2\sigma^2}(\sum_{i} \mu_{{X'_B}^i}^2 - (\alpha-1) \mu_{X_B}^2 - \Delta_{\alpha}({X'_B}^{\tilde \alpha},X_B)^2)}
\end{multline}


Note going back to the integral over all dimensions we get $$= (\sigma \sqrt{2 \pi})^{d} e^{\frac{-1}{2\sigma^2}(\sum_{i} ||{\mu_{{X'_B}^i}||_2}^2 - (\alpha-1) ||\mu_{X_B}||_2^2 - ||\Delta_{\alpha}({X'_B}^{\tilde \alpha},X_B)||_2^2)}$$

Thus to conclude we get 

\begin{multline}
    D_{\alpha}(M(X') || M(X))  \leq \sum_{X_B} \mathbb{P}(X_B) D_{\alpha}(\sum_{X'_B} \mathbb{P}(X'_B) N_{X'_B} || N_{X_B}) \\ = \sum_{X_B} \mathbb{P}(X_B) \frac{1}{(\alpha-1)} \\ \ln (\sum_{{X'_B}^{\tilde \alpha}} \mathbb{P}({X'_B}^{\tilde \alpha}) e^{\frac{-1}{2\sigma^2}(\sum_{i} ||\mu_{{X'_B}^i}||_2^2 - (\alpha-1) ||\mu_{X_B}||_2^2 - ||\Delta_{\alpha}({X'_B}^{\tilde \alpha},X_B)||_2^2)})
\end{multline}

A completely analogous calculation gives the same bound with just $X_B$ replaced with $X_B'$ (and vice-versa) for $D_{\alpha}(M(X)||M(X'))$. Taking the max over both these divergences gives a bound on the per-step per-instance R\'enyi-DP guarantee.

\end{proof}


Hence we now have a per-step RDP analysis for DP-SGD that takes advantage of when expected minibatch sensitivity to $x^*$ is much better than the worst cast minibatch sensitivity. While this phenomenon is not useful for studying the sum update rule (what is currently used for DP-SGD) as every mini-batch has the same sensitivity to $x^*$, in Section~\ref{ssec:exp_hard_renyi} we show this analysis allows us to provide a tighter analysis of the mean update rule. Hence, this opens the possibility of future work deploying DP-SGD with different update rules.

\section{Empirical Results}
\label{sec:main_body_emp_results}


In Section~\ref{sec:analysis} we provided the first framework to analyze DP-SGD's per-instance privacy guarantees. This followed by providing new per-step analyses (Theorem~\ref{thm:easy_renyi_dp} and~\ref{thm:renyi_dp_sens}), and a new composition theorem that relies on summing ``expected" per-step guarantees (Theorem~\ref{thm:better_composition}). 
We now highlight several conclusions our framework allows us to make about per-instance privacy when using DP-SGD. For conciseness, we defer a subset of the experimental results to Appendix~\ref{sec:detailed_emp_res}. %





\paragraph{Experimental Setup.} In the subsequent experiments, we apply our analysis on MNIST~\citep{lecun1998mnist} and CIFAR-10~\citep{krizhevsky2009learning}. Unless otherwise specified, LeNet-5~\citep{lecun1989backpropagation} and ResNet-20~\citep{he2016deep} were trained on the two datasets for 10 and 200 epochs respectively using DP-SGD, with a mini-batch size equal to 128, $\epsilon=10$, $\delta = 10^{-5}$, $\alpha = 8$ (in cases of Renyi DP), and clipping norm $C = 1.0$. All the experiments are repeated 100 times by sampling 100 data points to obtain a distribution/confidence interval if not otherwise stated.
Regarding hardware, we used NVIDIA T4 to accelerate our experiments. 


\begin{figure}[!t]

\centering
\subfloat[Training with the datapoint ($X \cup \{x^*\}$) \label{fig:compo_1_more}]
{
\includegraphics[width=0.33\linewidth]{figures/compo_simple_eps_compo_vary_eps_fraction_curve_MNIST_eps.pdf}
}
\subfloat[Training with the datapoint ($X \cup \{x^*\}$)\\$\text{~~~~~}$($10^{th}$ percentile) \label{fig:compo_1_more_10per}]
{
\includegraphics[width=0.33\linewidth]{figures/compo_simple_eps_compo_vary_eps_fraction_curve_MNIST_epspercentile10.pdf}
}
\subfloat[Training with and without the datapont \\ ($X \cup \{x^*\}$ and $X$) \label{fig:compo_1_less}]
{
\includegraphics[width=0.33\linewidth]{figures/remove_simple_eps_compo_vary_eps_fraction_curve_MNIST_exp.pdf}
}
\caption{Per-step privacy contribution from our composition theorem (Theorem~\ref{thm:better_composition}) using the per-step gurantee for the sum update rule (Theorem~\ref{thm:easy_renyi_dp}) as needed for DP-SGD, plotted as a fraction of the baseline data-independent per-step DP-SGD guarantee (Section 3.3 in~\citet{mironov2019r}). %
The expectations for Theorem~\ref{thm:better_composition} are computed over 10 trials. Figure~\ref{fig:compo_1_more} plots the average relative per-step contributon of 100 random points in MNIST for different strengths of the DP guarantee (i.e., different upper bounds $\varepsilon$) used when training on $X' = X \cup \{x^*\}$. The $10^{th} percentile$ is plotted in Figure~\ref{fig:compo_1_more_10per}. Figure~\ref{fig:compo_1_less} plots expectation when training on $X'$ and $X$ for 10 random points in MNIST. We see from both subfigures our per-step contribution decreases relative to the baseline as training progresses: using Theorem~\ref{thm:better_composition} one can conclude that many datapoints have better overall data-dependent privacy guarantees than expected by classical analysis.
}


\label{fig:composition}
\end{figure}

\subsection{Many Datapoints have Better Privacy}
\label{ssec:exp_better_privacy}




Here we describe how our per-instance RDP analysis of DP-SGD, using Theorem~\ref{thm:easy_renyi_dp} for the per-step analysis (with the update rule being the sum of gradients as is typically used) and Theorem~\ref{thm:better_composition} for the composition analysis, allows us to explain why per-instance privacy attacks will fail for many datapoints: many points have better per-instance privacy than the data-independent analysis. We further investigate the distribution of the per-instance privacy guarantees, and which points exhibit better per-instance privacy with our analysis.

\paragraph{Improved Per-Instance Analysis for Most Points} We compare the guarantees given by Theorem~\ref{thm:easy_renyi_dp} for the per-step guarantee in DP-SGD to the guarantee given by the data-independent analysis (see Section 3.3 in \citet{mironov2019r}), and plot per-step contribution coming from our composition theorem. In particular, we take $X$ to be the full MNIST training set, and randomly sample a data point $x^*$ from the test set to create $X' = X \cup x^*$ (as mentioned earlier, we repeat the sampling of $x^*$ 100 times to obtain a confidence interval). We train 10 different models on $X$ with the same initialization and compute the per-step contribution from Theorem~\ref{thm:better_composition} between $X$ and $X'$ (using Theorem~\ref{thm:easy_renyi_dp} to analyze the per-step guarantee from a given model) over the training run, shown in Figure~\ref{fig:compo_1_more}.
We can see that our analysis of the per-step contribution decreases with respect to the baseline as we progress through training. This persists regardless of the expected mini-batch size, the strength of DP used during training, and model architectures; see Figure~\ref{fig:renyi_simple_composition_mnist_sum} in Appendix~\ref{sec:detailed_emp_res}.
By Theorem~\ref{thm:better_composition} we conclude that $D_{\alpha}(Train_{DP-SGD}(X) || Train_{DP-SGD}(X'))$ is significantly less than the baseline for many data points. %



To see our improvement over the max of $D_{\alpha}(Train_{DP-SGD}(X) || Train_{DP-SGD}(X'))$ and \\ $D_{\alpha}(Train_{DP-SGD}(X') || Train_{DP-SGD}(X))$, i.e., the R\'enyi-DP guarantee, we computed the expectation when training on $X$ and $X'= X \cup \{x^*\}$  for $10$ training points $x^*$ where $X$ is now the training set of MNIST with one point removed and $X'$ is the full training set. Our results are shown in Figure~\ref{fig:compo_1_less} where we see a similar decreasing trend relative to the baseline over training: we conclude by Theorem~\ref{thm:better_composition} that many datapoints have better per-instance R\'enyi DP than the baseline. In other words, we conclude many datapoints have stronger per-instance RDP guarantees than can be demonstrated through the classical data-independent analysis.


\begin{figure}[!t]

\centering
\subfloat[Mini-batch Size = 128 \label{fig:simple_renyi_training_stage}]
{
\includegraphics[width=0.4\linewidth]{figures/renyi_simple_eps_hist_CIFAR10_resnet20_128_8_sum.pdf}
}
\subfloat[Varying Mini-batch Size \label{fig:simple_renyi_vary_bs}]
{
\includegraphics[width=0.4\linewidth]{figures/renyi_simple_eps_hist_vary_bs_CIFAR10_resnet20_sum.pdf}
}

\caption{Distribution plots of the per-step guarantees given by Theorem~\ref{thm:easy_renyi_dp} for $500$ datapoints in CIFAR10 with respect to: (a) different stages of training, and (b) varying mini-batch size. The purple dashed line 
represents the data-independent baseline. We observe a long tail of datapoints with magnitudes better privacy than expected in both plots.
}
\label{fig:simple_renyi}
\end{figure}

\begin{figure}[!t]
\centering
\includegraphics[width=0.4\linewidth]{figures/renyi_simple_eps_resnet20_fraction_curve_CIFAR10_sum.pdf}
\caption{Per-step guarantees given by Theorem~\ref{thm:easy_renyi_dp} for $500$ datapoints in CIFAR10 across training stages with respect to correct or incorrect classifications. It can be seen that correctly classified datapoints are on average more private than incorrectly classified ones.}
\label{fig:correct_incorrect}
\end{figure}



\paragraph{Long-Tail of Better Per-Instance Privacy.} However, the previous figures only show the average effect over datapoints. In Figures~\ref{fig:simple_renyi_training_stage} and~\ref{fig:simple_renyi_vary_bs} we plot the distribution of per-step guarantees over $500$ data points in CIFAR10. The key observations are (1) there exists a long tail of data points with significantly better per-instance privacy than the baseline,  (2) such improvements mostly exist in the middle and end of the training process, and (3) such improvements are mostly independent of mini-batch size.










\paragraph{Correct Points are More Per-Instance Private.} Next, we turn to understanding what datapoints are experiencing better privacy when using DP-SGD. In Figure~\ref{fig:correct_incorrect}, we plot the per-step guarantees given by Theorem~\ref{thm:easy_renyi_dp} for correctly and incorrectly classified data points at the beginning, middle, and end of training
on CIFAR10 (and for MNIST in Figure~\ref{fig:renyi_simple_fraction_curve_vary_arch_mnist} in Appendix~\ref{sec:detailed_emp_res}). We see that, on average, correctly classified data points have better per-step privacy guarantees than incorrectly classified data points across training. This disparity holds most strongly towards the end of training.










\subsection{Higher Sampling Rates can give Better Privacy}
\label{ssec:exp_hard_renyi}

We now highlight how our analysis, if it uses Theorem~\ref{thm:renyi_dp_sens} for the per-step analysis, allows us to better analyze DP-SGD with other update rules (not the sum of gradients which is what the current implementation of DP-SGD uses and Section~\ref{ssec:exp_better_privacy} analyzed). In particular, we will analyze the mean update rule and show how it has a privacy trade-off with sampling rate that is opposite to the trade-off for the sum update rule.

In normal SGD (with gradient clipping), one computes a mean for the per-step update 
$U(X_B) = \frac{1}{|X_B|}\sum_{x \in X_B} \nabla_{\theta}\mathcal{L}(\theta,x)/ \max(1,\frac{||\nabla_{\theta}\mathcal{L}(\theta,x)||_2}{C})$. 
However, DP-SGD computes a weighted sum $U(X_B) = \frac{1}{L} \sum_{x \in X_B} \nabla_{\theta}\mathcal{L}(\theta,x)/ \max(1,\frac{||\nabla_{\theta}\mathcal{L}(\theta,x)||_2}{C})$. Note the subtle difference between dividing by a fixed constant $L$ (typically the expected mini-batch size when Poisson sampling datapoints) and by the mini-batch size $|X_B|$. This means for the sum the upper-bound on sensitivity is $\frac{C}{L}$, while for the mean the upper-bound on sensitivity is only $C$ (consider neighbouring mini-batches of size 1 and 2). Hence using the mean update rule requires far more noise and so is not practical to use. We highlight how our per-instance analysis by sensitivity distributions provides better guarantees for the mean update rule.


\begin{figure}[!t]
\centering
\subfloat[$D_{\alpha}(M(X')||M(X))$\\$\text{~~~~~}$Mini-batch Size = $128$ \label{fig:hard_renyi_training_stage}]
{
\includegraphics[width=0.33\linewidth]{figures/renyi_hard_eps_hist_CIFAR10_resnet20_128.0_8_mean.pdf}
}
\subfloat[$D_{\alpha}(M(X)||M(X'))$ \\$\text{~~~~~}$Mini-batch Size = $128$
\label{fig:hard_renyi_reverse}]
{
\includegraphics[width=0.33\linewidth]{figures/reverse_renyi_hard_eps_hist_CIFAR10_resnet20_128_8_mean.pdf}
}
\subfloat[$D_{\alpha}(M(X')||M(X))$\\$\text{~~~~~}$Varying Mini-batch Size \label{fig:hard_renyi_vary_bs}]
{
\includegraphics[width=0.33\linewidth]{figures/renyi_hard_eps_hist_vary_bs_CIFAR10_resnet20_mean_unnormalized.pdf}
}

\caption{ Distribution plots (log scale) of per-step guarantees from Theorem~\ref{thm:renyi_dp_sens} for $500$ datapoints in CIFAR10 with respect to different training stages and mini-batch sizes. Bounds on both $D_{\alpha}(M(X)||M(X'))$ and $D_{\alpha}(M(X')||M(X))$ are shown for an expected mini-batch size of 128.  From Figures~\ref{fig:hard_renyi_training_stage},\ref{fig:hard_renyi_reverse}, we conclude Theorem~\ref{thm:renyi_dp_sens} gives better data-dependent guarantees for the mean update rule than classicial analysis, and from Figure~\ref{fig:hard_renyi_vary_bs} that increasing the expected mini-batch size decreases our bound for this update rule (counter-intuitive to privacy amplification by subsampling).
}
\label{fig:hard_renyi}
\end{figure}


\paragraph{Better Analysis of the Mean Update Rule. } Letting $M$ now be the sampled Gaussian mechanism with the mean update rule, we compute the bound on $D_{\alpha}(M(X')||M(X))$ and $D_{\alpha}(M(X)||M(X'))$ given by Theorem~\ref{thm:renyi_dp_sens}, where we estimated the inner and outer expectation using $20$ samples, i.e., $20$ random $X_B'^{\alpha}$ (or $X_B^{\alpha}$) for each of the $20$ random $X_B$ (or $X_B'$). We obtain Figure~\ref{fig:hard_renyi_training_stage} and~\ref{fig:hard_renyi_vary_bs} by repeating this for $500$ data points in CIFAR10 while varying the training stage. We observe that for both divergences, we beat the baseline analysis by more than a magnitude at the middle and end of training. We conclude Theorem~\ref{thm:renyi_dp_sens} gives us better per-instance R\'enyi DP guarantees for the mean update rule.








\paragraph{Per-Instance Privacy Improves with Higher Sampling Rate.} Furthermore, counter-intuitively to typical subsample privacy amplification, in Figure~\ref{fig:hard_renyi_vary_bs} we see that our bound decreases with increasing expected mini-batch size: 
we attribute this to the law of large numbers, whereby increasing the expected mini-batch size leads to sampled mini-batches having similar updates more often and hence the sensitivity distribution concentrates at smaller values. An analogous result is shown for MNIST in Figure~\ref{fig:renyi_hard_eps_distrib_bs_mnist_mean} (in Appendix~\ref{sec:detailed_emp_res}).




% !TEX root = Guillon2017_arxiv.tex

%% Short Recap

Graph analysis of brain networks have been largely exploited in the study of AD with the aim to extract new predictive diagnostics of disease progression.
Typical approaches in functional neuroimaging, characterized by oscillatory dynamics, analyze brain networks separately at different frequencies thus neglecting the available multivariate spectral information.
Here, we adopted a method to formally take into account the topological information of multi-frequency connectomes obtained from source-reconstructed MEG signals in a group of AD and healthy subjects during EC resting states.

%% Multiplex Results

Main results showed that, while flattening networks of different frequency bands attenuates differences between AD and HC populations, keeping the multiplex nature of MEG connectomes allow to capture higher-order discriminant information.
AD subjects exhibited an aberrant multiplex brain network structure that significantly reduced the global propensity to facilitate information propagation across frequency bands as compared to HC subjects (\autoref{fig:participation}b, inset). This could be in part explained by the higher variability of the individual node degrees across bands (\autoref{fig:coefficient_of_variation}).

% NOTE: High MPCi does not necessarily mean high oi (autoref fig:mpca) but it also seems that having a high number of connections (high oi) and a low MPC is not possible in the case of the brain.

% NOTE: In general, a ROI with a high MPC but with low oi, will have an even higher MPC if its oi increase (for another subject for instance). In clear: for a given i (i.e. a given ROI), the corrcoeff between oi and MPCi is always positive.

% NOTE: I tested different thresholds and with the ImCoh to check if it was not because of the week noisy connections, but the distribution of MPCi values seems to be always the same. Even with an average degree of 1 meaning that the brain always tends to keep connections in multiples frequency bands in the same time. Could it be explained by the fact that coherence is influenced by harmonics?

Such loss of inter-frequency centrality was mostly localized in association areas as well as in the cingulate cortex (\autoref{fig:participation}b; \autoref{tab:local_participation}), which resulted the most important hub promoting interaction across bands in the HC group (\autoref{fig:mpc}a).
Because all these areas are typically affected by AD atrophy \citep{wenk_neuropathologic_2003} we hypothesize that the anatomical withering might have impacted the neural oscillatory mechanisms supporting large-scale brain functional integration. Notably, the significant alteration of the connectivity across bands observed in the cingulate cortex could be ascribed to typical M/EEG connectivity changes observed in AD, such as reduced $alpha$ coherence \citep{stam_magnetoencephalographic_2006,jeong_eeg_2004,dauwels_diagnosis_2010,wang_power_2015} (\autoref{fig:mpc}b).
We also found a significant decrease in the primary motor cortex (right precentral gyrus). While previous studies have identified this specific region as a connector hub in human brain networks \citep{tijms_alzheimers_2013}, its role in AD still needs to be clarified in terms of node centrality's changes with respect to healthy conditions.
%For these affected ROIs the decreased centrality was reflected by fewer interactions with higher sensory rhythms ($>20$ Hz) \citep{basar_review_2013} and more connections to lower attentional ones ($<13$ Hz) \citep{klimesch_EEG_1999} (\autoref{fig:participation}c).

% Single-Layer Results
While flattening network layers represents in general an oversimplification, analyzing single layers can still be a valid approach that is worth of investigation.
Because the $MPC$ is a pure multiplex quantity, we considered the conceptually akin version for single-layer networks, the standard participation coefficient $PC$, which evaluates the tendency of nodes to integrate information from different modules, rather than from different layers \citep{guimera_cartography_2005, battiston_structural_2014}.
AD patients exhibited lower inter-modular connectivity in the \textit{gamma} band with respect to HC subjects (\autoref{fig:participation}a; \autoref{tab:local_participation}) that was localized in association areas including frontal, temporal, and parietal cortices (\autoref{fig:participation}a; \autoref{tab:local_participation}).
%
Damages to these regions can lead to deficits in attention, recognition and planning \citep{purves_neuroscience_2001}. Our results support the hypothesis that AD could include a disconnection syndrome  \citep{pearson_anatomical_1985,arnold_topographical_1991,catani_rises_2005}.
Furthermore, they are in line with previous findings showing $PC$ decrements in AD, although those declines were more evident in lower frequency bands and therefore ascribed to possible long-range low-frequency connectivity alteration \citep{de_haan_disrupted_2012,tijms_alzheimers_2013}.

%% Conclusion
Put together, our findings indicated that AD alters the global brain network organization through connection disruption in several association regions, which play important roles in sensory processing by integrating information from other cortical regions through high-frequency channels \citep{miltner_coherence_1999-1,buschman_top-down_2007, siegel_neuronal_2008, gregoriou_high-frequency_2009, hipp_oscillatory_2011}.
%
Notably, we showed that the global loss of inter-modular interactions in the \textit{gamma} band is paralleled by a diffused decrease of inter-frequency centrality.
Future studies, involving recordings of limbic structures and/or stimulation-based techniques, should elucidate whether these two distinct reorganizational processes are truly independent or linked through possible cross-frequency mechanisms which are known to be essential for normal memory formation \citep{canolty_high_2006,axmacher_cross-frequency_2010, goutagny_alterations_2013}.


%% Classification Results

As a confirmation of the complementary information carried out by the multi-layer approach, we reported an increased classification accuracy when combining the local $PC$ and $MPC$ features.
The observed diagnostic power is in line with previous accuracy values obtained with standard graph theoretic approaches (around $80\%$) but exhibits slightly higher sensitivity ($>90\%$), which is often desired to avoid false negatives \citep{li_discriminant_2012, wang_disrupted_2013, wee_enriched_2011, wee_identification_2012, horwitz_functional_2011}.
Other approaches should determine if and to what extent the use of more sophisticated machine learning algorithms, or the inclusion of basic connectivity features \citep{hutchison_network-based_2011, shao_prediction_2012, zhou_hierarchical_2011} and different imaging modalities \citep{dai_discriminative_2012}, can lead to higher classification performance and better diagnosis \citep{tijms_alzheimers_2013}.

%% Correlation With MMSE

Previous works have documented relationships between brain network properties and neuropsychological measurements in AD, suggesting a potential impact for monitoring disease progression and for the development of new therapies
\citep{de_haan_functional_2009,lo_diffusion_2010,sanz-arigita_loss_2010-1,shu_disrupted_2012,stam_small-world_2007,wang_disrupted_2013}.
This is especially true for the standard $PC$ which has exhibited stronger correlations and larger between-group differences \citep{tijms_alzheimers_2013}.
In line with this prediction, we also reported significant correlations between the MMSE cognitive scores and the $PC$ values of the AD patients in the \textit{gamma} band (\autoref{fig:correlations}a).
%
An even stronger correlation was found, however, for the global $MPC$ values and the TR scores (\autoref{fig:correlations}b, \autoref{tab:local_correlation}).
Recent studies suggest that TR scores could be more specific for AD \citep{grober_free_2010, velayudhan_review_2014} as compared to MMSE scores which could be biased by differences in years of education, lack of sensitivity to progressive changes occurring with AD, as well as fail in detecting impairment caused by focal lesions \citep{tombaugh_mini-mental_1992}.
Locally, the regions whose $MPC$ correlated with TR were part of the default-mode network (DMN) (\autoref{tab:local_correlation}), which is heavily involved in memory formation and retrieval \citep{buckner_brains_2008,sperling_functional_2010}. According to recent hypothesis, these areas are directly affected by atrophy and metabolism disruption, as well as amyloid-$\beta$ deposition \citep{buckner_molecular_2005, greicius_default-mode_2004}.
Put together, our results suggest that AD symptoms related to episodic memory losses could be determined by the lower capacity of strategic DMN association areas to let information flow across different frequency channels.

\subsection*{Methodological considerations}

We estimated brain networks by means of spectral coherence, a connectivity measure widely used in the electrophysiological literature because of its simplicity and relatively intuitive interpretation \citep{srinivasan_eeg_2007}.
While this measure is known to suffer from possible volume conduction effects, recent evidence showed that source reconstruction techniques, like the one we adopted here, could at least mitigate this bias \citep{schoffelen_source_2009} and generate connectivity patterns consistent within and between subjects \citep{colclough_how_2016}.
In a separate analysis, we used the imaginary coherence as a candidate alternative to eliminate volume conduction effects \citep{nolte_identifying_2004}. We demonstrated that while no significant between-group differences could be obtained in terms of $MPC$ (data not shown here), the spatial distribution of the $MPC$ values was very similar to that observed in the brain networks obtained with the spectral coherence, especially for the internal regions along the longitudinal fissure (\autoref{fig:mpc_imcoh}).

Differently from other multiplex network quantities, such as those based on paths and walks \citep{boccaletti_structure_2014}, the $MPC$ has the advantage to not depend on the weights of the inter-layer links which, in general, are difficult to estimate or to assign from empirically obtained biological data. This is especially true in network neuroscience where, so far, the strength of the inter-layer connections is parametric and subject to arbitrariness \citep{de_domenico_mapping_2016} or estimated through measures of cross-frequency coupling \citep{brookes_multi-layer_2016-1} whose biological interpretation remains still to be completely elucidated \citep{jirsa_cross-frequency_2013}.

\section{Conclusion}
\label{sec:conclusion}
This paper presents a generic top-$\size$ recommendation framework for  trading-off accuracy, novelty, and coverage. To achieve this, we profile the users according to their preference for long-tail novelty. We examine various measures, and formulate an optimization problem to learn these user preferences from interaction data.  We then integrate the user preference estimates in our generic framework, GANC.  Extensive experiments on several datasets confirm that there are trade-offs between accuracy, coverage, and novelty. Almost all re-ranking models increase coverage and novelty at the cost of accuracy. However, existing re-ranking models typically rely on rating prediction models, and are hence more effective in dense settings. Using a generic approach, we can easily incorporate a suitable base accuracy recommender to devise an effective solution for both sparse and dense settings.  %Our results  also indicate there is no single method that outperforms other methods in all metrics. However our techniques obtain a significant improvement in coverage, while  . 
Although we integrated the  long-tail novelty preference estimates into a re-ranking framework, their use-case is not limited to these frameworks. In  the future, we intend to explore the temporal and topical dynamics of long-tail novelty preference, particularly in settings where contextual information is  available.  
%We achieve these objectives without using any additional contextual information.


\iffalse
While we focused on promoting long-tail items across users, we did not consider diversity of individual top-$\size$ recommendations, a factor that has been shown to positively affect consumer satisfaction. This is one direction for future work. Moreover, the sequential setting  in our work, creates a dependency between different batches, where,  the items recommended to a batch of users, depends on those recommended to previous batches. This dependency is created through the parameter $\mathbf{f}$, that is updated every time a top-$\size$ set  is allocated to a batch of users. A future direction for our work is to estimate a distribution over $\mathbf{f}$ that allows us to independently solve the problem for each user, leading to improvements across all performance metrics, including recommendation time. 

We design algorithms that take advantage of the structure in the value functions to obtain both efficient and scalable solutions. 
We design an algorithm that takes advantage of the structure in the value functions to obtain both efficient and scalable solutions. 

\textcolor{red}{Our  sequential  algorithms can be applied for batch recommendation contexts,~e.g., personalized email marketing, where based on prior interaction data between users and items,  a new round of recommendations must be sent to all users in the system.  However, the independent coverage algorithms lift the sequential setting restrictions and allow it be applied for re-ranking the output of base recommender in any setting. }A future direction for our work is to incorporate explicit diversity metrics in the framework. 
\fi


%We have a presented a submodular maximization framework to systematically trade-off relevance and diversity in recommendations to individual users and coverage across the item-space. This ensures both consumer and producer satisfaction. We model users according to their risk and focusing degrees and promote long-tail items to the right group of consumers. Consequently, we obtain a significant improvement in coverage while maintaining reasonable levels of user satisfaction. Furthermore, our methods are able to achieve a more balanced distribution across the set of recommended items. In the future, we plan to investigate the effect of using alternative base recommender systems. 

%Future Work
%However most of these methods assume that the ratings are missing at random (MAR). Since our method of generating recommendations is based on the completed matrix, assuming MAR might introduce additional bias, we will use methods which assume that the ratings at missing not at random (MNAR),explored in~\cite{steck2010training, icml2014c2_hernandez-lobatob14}. 	 
%Long Tail %Recently, authors in~\cite{cremonesi2010performance} conducted extensive experiments to evaluate the performances of various matrix factorization-based algorithms and neighborhood models on the task of recommending long tail items. Their experimental results show that long tail recommendation leads to a decrease in accuracy for all algorithms. They also showed that for this task, SVD outperforms other state-of-the-art algorithms. 


\section*{Acknowledgements}
We would like to acknowledge our sponsors, who support our research with financial and in-kind contributions: Amazon, Apple, CIFAR through the Canada CIFAR AI Chair, DARPA through the GARD project, Intel, Meta, NSERC through the COHESA Strategic Alliance and a Discovery Grant, Ontario through an Early Researcher Award, and the Sloan Foundation. Anvith Thudi is supported by a Vanier Fellowship from the Natural Sciences and Engineering Research Council of Canada.  Resources used in preparing this research were provided, in part, by the Province of Ontario, the Government of Canada through CIFAR, and companies sponsoring the Vector Institute. We would further like to thank Relu Patrascu at the University of Toronto for providing the compute infrastructure needed to perform the experimentation outlined in this work. We would also like to thank members of the CleverHans lab and Mahdi Haghifam for their feedback on drafts of the manuscript.


\bibliographystyle{abbrvnat}
\bibliography{references}

\appendix
\newpage

\newpage
\section{Dataset Visualizations}
\label{sec:app_dataset_visuals}

%%%%%%
%%
%%
\subsection{Examples of each view class}
\newcommand{\BC}{0.33}
\setlength{\tabcolsep}{0.1cm}
\begin{figure}[!h]
\begin{tabular}{c c c c}
    PLAX  & PSAX & OTHER 
    \\
    \includegraphics[width=\BC\textwidth]{figures/small_appendix/Appendix_PLAX1.jpg}
    &
    \includegraphics[width=\BC\textwidth]{figures/small_appendix/Appendix_PSAX1.jpg}
    &
    \includegraphics[width=\BC\textwidth]{figures/small_appendix/Appendix_Other1.jpg}
    &
   
    \\
    
    \includegraphics[width=\BC\textwidth]{figures/small_appendix/Appendix_PLAX2.jpg}
    &
    \includegraphics[width=\BC\textwidth]{figures/small_appendix/Appendix_PSAX2.jpg}
    &
    \includegraphics[width=\BC\textwidth]{figures/small_appendix/Appendix_Other2.jpg}
    &
   
     \\
     
     \includegraphics[width=\BC\textwidth]{figures/small_appendix/Appendix_PLAX3.jpg}
    &
    \includegraphics[width=\BC\textwidth]{figures/small_appendix/Appendix_PSAX3.jpg}
    &
    \includegraphics[width=\BC\textwidth]{figures/small_appendix/Appendix_Other3.jpg}
    &
   
     \\
     
     \includegraphics[width=\BC\textwidth]{figures/small_appendix/Appendix_PLAX4.jpg}
    &
    \includegraphics[width=\BC\textwidth]{figures/small_appendix/Appendix_PSAX4.jpg}
    &
    \includegraphics[width=\BC\textwidth]{figures/small_appendix/Appendix_Other4.jpg}
    &
   
    \end{tabular}	
    \caption{Examples of images for each possible view label in our dataset. \emph{From left to right:} Four examples of peristernal long axis (PLAX) view, four examples of peristernal short axis (PSAX) view, and four examples of other kinds of view in our ``Other'' class. }
    \label{fig:VIEW_SAMPLES_APPENDIX}
\end{figure}

%%%%%%
%%
%%
\newpage
\subsection{Examples of each view for a Severe AS patient}
\newcommand{\BA}{0.33}
\setlength{\tabcolsep}{0.1cm}
\begin{figure}[!h]
\begin{tabular}{c c c c}
    PLAX  & PSAX & OTHER 
    \\
    \includegraphics[width=\BA\textwidth]{figures/small_appendix/SevereAS_11112007_PLAX1.jpg}
    &
    \includegraphics[width=\BA\textwidth]{figures/small_appendix/SevereAS_11112007_PSAX1.jpg}
    &
    \includegraphics[width=\BA\textwidth]{figures/small_appendix/SevereAS_11112007_Other1.jpg}
    &
    
    \\
    
    \includegraphics[width=\BA\textwidth]{figures/small_appendix/SevereAS_11112007_PLAX2.jpg}
    &
    \includegraphics[width=\BA\textwidth]{figures/small_appendix/SevereAS_11112007_PSAX2.jpg}
    &
    \includegraphics[width=\BA\textwidth]{figures/small_appendix/SevereAS_11112007_Other2.jpg}
    &
   
     \\
     
     \includegraphics[width=\BA\textwidth]{figures/small_appendix/SevereAS_11112007_PLAX3.jpg}
    &
    \includegraphics[width=\BA\textwidth]{figures/small_appendix/SevereAS_11112007_PSAX3.jpg}
    &
    \includegraphics[width=\BA\textwidth]{figures/small_appendix/SevereAS_11112007_Other3.jpg}
    &
  
    \end{tabular}	
    \caption{Examples of images from a patient with Severe AS in our dataset. \emph{From left to right:} Three examples of parasternal long axis (PLAX) view, three examples of parasternal short axis (PSAX) view, and three examples of other kinds of view in our ``Other'' class. }
    \label{fig:PatientSevereAS}
\end{figure}


%%%%%%
%%
%%
\newpage
\subsection{Examples of each view for a No AS patient}
\newcommand{\BB}{0.33}
\setlength{\tabcolsep}{0.1cm}
\begin{figure}[!h]
\begin{tabular}{c c c c}
    PLAX  & PSAX & OTHER 
    \\
    \includegraphics[width=\BB\textwidth]{figures/small_appendix/NoAS_1996889_PLAX1.jpg}
    &
    \includegraphics[width=\BB\textwidth]{figures/small_appendix/NoAS_1996889_PSAX1.jpg}
    &
    \includegraphics[width=\BB\textwidth]{figures/small_appendix/NoAS_1996889_Other1.jpg}
    &
    
    \\
    
    \includegraphics[width=\BB\textwidth]{figures/small_appendix/NoAS_1996889_PLAX2.jpg}
    &
    \includegraphics[width=\BB\textwidth]{figures/small_appendix/NoAS_1996889_PSAX2.jpg}
    &
    \includegraphics[width=\BB\textwidth]{figures/small_appendix/NoAS_1996889_Other2.jpg}
    &
   
     \\
     
     \includegraphics[width=\BB\textwidth]{figures/small_appendix/NoAS_1996889_PLAX3.jpg}
    &
    \includegraphics[width=\BB\textwidth]{figures/small_appendix/NoAS_1996889_PSAX3.jpg}
    &
    \includegraphics[width=\BB\textwidth]{figures/small_appendix/NoAS_1996889_Other3.jpg}
    &
  
    \end{tabular}	
    \caption{Examples of images from a patient with No AS in our dataset. \emph{From left to right:} Three examples of parasternal long axis (PLAX) view, three examples of parasternal short axis (PSAX) view, and three examples of other kinds of view in our ``Other'' class. }
    \label{fig:PatientNoAS}
\end{figure}



\newpage 
\section{Further Results}

\subsection{Assessment of ensembling}

Table~\ref{tab:best_single_checkpoint_VS_ensemble_FS_echo260} compares using a single checkpoint (one point estimate of neural network weight vector $\theta$) to using an ensemble of parameters aggregated from the last 25 checkpoints (one per epoch).

\begin{table}[!h]
    \centering
    \begin{tabular}{c|cccc|c}
    \textit{Diagnosis classification} & Split 1  & Split 2 & Split 3 & Split 4 & Average\\
    \hline
    Best single checkpoint  & 61.81 & 59.79 & 56.05 & 64.21 & 60.46\\
    Ensemble  & 62.95 & 61.03 & 56.58 & 63.84 & \textbf{61.13}
	\\ \hline
    \textit{View classification}  &   &  &  &  & 
    \\ \hline
    Best single checkpoint  & 93.03 & 93.24 & 92.39 & 93.79 & 93.11\\
    Ensemble  & 92.37 & 93.24 & 93.72 & 93.87 & \textbf{93.30}\\
    \end{tabular}
    \caption{Comparing best single checkpoint performance with ensemble performance on \textbf{Full-size \datasetName-156-52}}
    \label{tab:best_single_checkpoint_VS_ensemble_FS_echo260}
\end{table}


%%%%%%
%%
%%
\subsection{Patient-level diagnosis performance on bonus heldout set}

Table~\ref{tab:diagnosis classification patient unlabeled_heldout_174} examines the performance of the best labeled-set-only methods and MixMatch methods on the 174 patient studies that have diagnosis but no view labels.
 While the images used here were originally included in the unlabeled training set (which was used to train SSL methods like MixMatch), the diagnosis labels were not provided at all during training time. 
 We thus still believe this is an authentic test of generalization given the scarcity of labeled data available for our task.
 Of course, additional independent evaluation (especially from another institution) is needed.

\begin{table}[!h]
    \centering
    \begin{tabular}{l l l|rrrr|c}
    Pretrain & Method & Voting
    & Split 1  & Split 2 & Split 3 & Split 4 & average\\
    \hline
    & Basic WRN & Simple average & 76.73 & 75.25 & 76.87 & 81.88 & 77.68\\
    & Basic WRN & View-prioritized & 73.63 & 83.21 & 79.70 & 80.08 & 79.18\\
    %SSL & FS & MixMatch & Priority view + confidence & 94.58 & 84.17 & 77.50 & 92.5 & 87.19\\
    \hline
    & MixMatch & Simple average & 85.32 & 76.29 & 74.14 & 79.95 & 78.93\\
    view & MixMatch & Simple average & 83.36 & 77.96 & 75.61 & 81.37 & 79.58\\
    & MixMatch & View-prioritized & 83.27 & 83.76 & 82.34 & 82.83 & \textbf{83.05}\\
    view & MixMatch & View-prioritized & 82.53 & 86.15 & 79.62 & 83.27 & 82.89\\
    %view & MixMatch & LR with view-priority & 80.42 & 84.24 & 76.58 & 80.67 & 80.48\\
    %(MixMatch transfered) + MysteryMethod & NA & NA & NA\\ 
    \end{tabular}
    \caption{Patient-level AS Severity Diagnosis Classification on the \textbf{bonus heldout set} of 174 patients for whom we have diagnosis labels only (no view labels). We show balanced accuracy on models trained on each of the four folds on four \textbf{full-size \datasetName-156-52} dataset.
    }%endcaption
    \label{tab:diagnosis classification patient unlabeled_heldout_174}
\end{table}


%%%%%%
%%
%%
\subsection{Assessment of MixMatch hyperparameter sensitivity}

In Table~\ref{tab:MixMatch hyperparameters ablation study}, we consider four possible strategies for setting the hyperparameters of MixMatch, varying two  key settings for the weight on unlabeled loss $\lambda$. First, we vary whether the final value of $\lambda$ is set to its \emph{best} value among a grid of candidates (based on validation set performance), or \emph{fixed} to a constant.
Second, we vary whether $\lambda$ remains fixed over iterations throughout a training run, or is updated over iterations on a linear ramp schedule from 0 to its final target value. 

From this comparison, we see we consistent gains across splits (average gain across splits of over 1.6\% balanced accuracy) for using a delayed ramp up schedule with target value selected via grid search.

\begin{table}[!h]
    \centering
    \begin{tabular}{l l| rrrr | r}
    Final $\lambda$ value & $\lambda$ update schedule & Split 1  & Split 2 & Split 3 & Split 4 & Average\\
    \hline
    best on val & Delayed ramp-up  & 65.57 & 62.69 & 60.87 & 66.29 & 63.86\\
    best on val & Immediate ramp-up & 65.07 & 61.87 & 60.82 & 65.37 & 63.28\\
    best on val & Constant  & 65.03 & 61.52 & 58.87 & 65.22 & 62.66\\
    100 (fixed) & Constant & 63.94 & 61.79 & 58.87 & 64.35 & 62.24\\
    \end{tabular}
    \caption{Ablation study of different settings of the unlabeled loss weight $\lambda$ for MixMatch. AS severity diagnosis classification for individual images on the \textbf{full-size \datasetName-156-52} dataset. showing balanced accuracy averaged over the test sets from multiple folds (each fold’s test set contains all images from 52 patients). }%endcaption
    \label{tab:MixMatch hyperparameters ablation study}
\end{table}



%%%%%%
%%
%%
\subsection{Assessment of alternative view prioritization strategy using thresholding}


An anonymous reviewer suggested an alternative strategy for prioritizing images of relevant view.
The alternative strategy works as follows: for each image, we compute the predicted probability that the image is a ``relevant view'' (either PLAX and PSAX) by summing the probabilities of each view type.
However, instead of using this raw probability as a weight (as our chosen method does), we use a \emph{cutoff threshold} and simply average the diagnosis predictions of images whose relevant view probability is above the cutoff.
For each patient, we use the majority vote prediction of the diagnosis from the images of relevant views.
The value of the cutoff threshold is selected using the validation set to maximize balanced accuracy.

Table~\ref{tab:Suggested_Aggregation_Ablation} shows the performance of this strategy (``threshold-then-average'') on the full-size dataset.
Using this alternative prioritization strategy together with our suggested methodology for patient-level diagnosis (using MixMatch, pretraining on view), we find the average test set balanced accuracy is around 85.8\%, while the weighted average strategy in the main paper achieves over 90\% balanced accuracy. We take this as reasonably decisive evidence that a weighted average (rather than a simple cutoff) should be preferred.

\begin{table}[!h]
    \centering
    \begin{tabular}{l l l|rrrr|c}
    Pretrain & Method & Aggregation across images
    & Split 1  & Split 2 & Split 3 & Split 4 & average\\
    \hline
    & Basic WRN & Threshold-then-Average & 85.42 & 86.25 & 79.17 & 92.50 & 85.84 \\
    %SSL & FS & MixMatch & Priority view + confidence & 94.58 & 84.17 & 77.50 & 92.5 & 87.19\\
    & MixMatch & Threshold-then-Average & 83.33 & 84.17 & 77.50 & 94.58 & 84.90 \\
    view & MixMatch & Threshold-then-Averagen & 86.67 & 80.00 & 82.50 & 94.17 & 85.84\\
    %view & MixMatch & LR with view-priority & 87.08 & 82.08 & 85.00 & 88.75 & 85.73\\
    %(MixMatch transfered) + MysteryMethod & NA & NA & NA\\ 
    \end{tabular}
    \caption{Alternative view-prioritizing strategy for patient-level AS severity diagnosis classification on the \textbf{full-size \datasetName-156-52} dataset, showing balanced accuracy on the test set across multiple folds (each fold’s test set contains 52 patients).}
    %endcaption
    \label{tab:Suggested_Aggregation_Ablation}
\end{table}



%%%%%%
%%
%%
\subsection{ROC Curve of patient-level diagnosis: no AS vs. mild/moderate/severe AS}

Fig.~\ref{fig: No AS vs Some AS} shows receiver operating curves for several methods for the task of distinguishing no AS vs Some AS (which aggregates both the mild/moderate and severe levels in the 3-level diagnosis task of the main paper).

\begin{figure}[!h]
\begin{tabular}{c c}
	\includegraphics[width=0.43\textwidth]{figures/fold0_multitask_PatientLevel_NoVSSome_NormalizedPriorityStrategyClassProbabilityScore.pdf}
	&
    \includegraphics[width=0.43\textwidth]{figures/fold1_multitask_PatientLevel_NoVSSome_NormalizedPriorityStrategyClassProbabilityScore.pdf}
	\\
	(a) Split 1 & (b) Split 2
	\\
	\includegraphics[width=0.43\textwidth]{figures/fold2_multitask_PatientLevel_NoVSSome_NormalizedPriorityStrategyClassProbabilityScore.pdf}
	&
    \includegraphics[width=0.43\textwidth]{figures/fold3_multitask_PatientLevel_NoVSSome_NormalizedPriorityStrategyClassProbabilityScore.pdf}
	\\
	(c) Split 3 & (d) Split 4
\end{tabular}
    
\caption{ROC curves for binary diagnosis task (no AS vs ``mild/moderate/severe AS'') on \textbf{full-size \datasetName-156-52}.
    }%endcaption
    \label{fig: No AS vs Some AS}
\end{figure}

\section{Methodological Details}

\subsection{Image processing details}
\label{sec:removing_doppler}

\paragraph{Removing doppler images.}
In the raw data of all imagery available for an echocardiogram study, 
we obtained TIFF files that represent both cineloops and Doppler images.

We verified in our labeled set that all Doppler images have one of the following landscape aspect ratio $(831, 323)$, $(901, 384)$, $(901, 390)$, $(704, 305)$, $(831, 421)$, $(901, 469)$ or $(563, 294)$. Only the Dopplers have these aspect ratios. We thus filtered out Doppler completely via these aspect ratios. 

\paragraph{Downsizing}
The original images are provided as high-resolution TIFF format images (hundreds of pixels per side) of varying aspect ratios. Generally, we can expect that both view and diagnosis classifiers would perform better given higher-resolution input (and holding other factors the same). The main trade-off of processing higher-resolution images is increased runtime and memory requirements. In our preliminary experiments, we compared downsizing all images to a standard square aspect ratio at 3 possible sizes: 32x32, 64x64 and 128x128. We found that 64x64 achieves a good balance between model performance and computation cost. 
A prior study by \citet{madaniDeepEchocardiographyDataefficient2018} provides a more extensive study of optimal resolution size. The interested reader can refer to their work for more details. 


\subsection{Architecture Settings and Hyperparameters}
\label{sec:arch_and_hyperparameters}

\paragraph{Weighted cross-entropy for labeled loss}
To counteract the effect of class imbalance in the dataset, we use weighted cross-entropy for the labeled loss. For an input image $x$ whose true label $y$ indicates it belongs to class $c$, the weighted cross-entropy assumes the following form:
\begin{align}
\mathcal{L}^L(\theta, x) = - w_{c} \log \hat{p}_{c}(\theta, x),
\end{align}
where $\hat{p}_{c}$ is the predicted probability of class $c$. The weight $w_{c}$ is calculated using the training set statistics as follow:
\begin{align}
w_{c} = \frac{\prod_{k\neq c}{N_{k}}}{\sum_{j}\prod_{k \neq j}{N_{k}}}
\end{align}
where $N_{k}$ is the number of images of class $k$ in the training set.

\paragraph{Common architecture.}
Following~\citet{oliverRealisticEvaluationDeep2018}, for all considered methods, we use the \emph{same} backbone neural network architecture: a wide residual network~\citep{zagoruykoWideResidualNetworks2017} with 28 layers (WRN-28), which has total of 5,931,683 parameters.
This same network architecture is used in the original MixMatch evaluation~\citep{berthelotMixmatchHolisticApproach2019} with promising results.

\paragraph{Common training protocol.}
All SSL methods we consider follow the loss minimization framework with two primary losses (one for ``labeled'' data and one for ``unlabeled'' data) in Eq.~\eqref{eq:standard-SSL-loss-template}.
We allow every method to train for 32 epochs (where each epoch processes $2^{16}$ images, as in \citet{berthelotMixmatchHolisticApproach2019}).
Our preliminary experiments suggest that after 30 epochs all methods effectively converge in terms of validation balanced accuracy. 

\paragraph{Common regularization.}
For all methods, we expect performance will be vulnerable to overfitting, so we impose an L2-norm penalty on the weights $\theta$, also known as weight decay. Each method selects its preferred value of this penalty strength hyperparameter. We searched values in [0.0002, 0.002, 0.02].

\paragraph{Common optimization.}
We use ADAM \citep{kingma2014adam} to optimize each model.
Each method selects the value of the step size (learning rate) as a hyperparameter. We experimented with 0.002 and 0.0007
%HZ: 'performance being sensitive to learning rate' is very reasonable. But we don't have an ablation to back it. 
%We find performance is sensitive to the step size (learning rate) hyperparameter, so we perform a grid search and select the value that maximizes balanced accuracy on the validation set.

\paragraph{Hyperparameters for Pseudo-Label.}
Beyond the usual hyperparameters for our loss-minimization SSL framework, another important hyperparameter for pseudo-label is the threshold $\tau$. We find that performance is not very sensitive to the chosen $\tau$ value as long as it is within a certain range. We set $\tau$ to 0.95, as done in past literature that evaluates Pseudo-Label as an SSL method ~\citep{oliverRealisticEvaluationDeep2018,berthelotMixmatchHolisticApproach2019, berthelotRemixmatchSemisupervisedLearning2019, sohnFixmatchSimplifyingSemisupervised2020}.


\paragraph{Hyperparameters for VAT.}
Beyond the usual hyperparameters for our SSL framework, for VAT we need to select a value for $\epsilon$.
In \citet{miyatoVirtualAdversarialTraining2019}, the authors claimed that they can achieve superior performance by tuning only $\epsilon$ and fixing $\lambda$ to 1. In our experiment, we used the default $\lambda$ as in \cite{berthelotMixmatchHolisticApproach2019} and searched the value of $\epsilon$ in [2, 6, 18], together with learning rate and weight decay. We select the best hyperparameters using validation set performance. 


\paragraph{Hyperparameters for MixMatch.}
Beyond the usual hyperparameters for our SSL framework, the key hyperparameters for MixMatch include the number of augmentations $K$, the temperature $T>0$ used for sharpening, interpolation hyperparameter $\alpha$ and unlabeled loss coefficient $\lambda$. We set $K=2$, $T=0.5$, and $\alpha=0.75$ as done in \citet{berthelotMixmatchHolisticApproach2019}, and search for $\lambda$ in the range [10, 30, 75, 100, 130] using validation set. 

\paragraph{Hyperparameters for Multitask training.}
We searched $\gamma$, the hyperparameter that control the strength of the auxilliary view loss in Eq.~\eqref{eq:multitask}, in the range [10, 3, 1, 0.3, 0.1]. The best $\alpha$ is selected together with other hyperparameters on validation set. 


\end{document}