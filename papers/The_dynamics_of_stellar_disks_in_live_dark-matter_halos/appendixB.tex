

\section[]{Evolution of spiral arms in a live halo}
\label{Sect:AppB}

For the formation and evolution of bars, the effect of a
live halo has been investigated in previous work, and
it has been shown that the angular-momentum transfer from bars to
live halos helps the growth of bars
\citep{2000ApJ...543..704D,2002ApJ...569L..83A,2003MNRAS.341.1179A}.
Previous work, however, focused on the evolution of bars but
not on spiral arms. 
In most of the previous work rigid halo simulations were used
to study the dynamical evolution of spiral arms. As all
our simulations are performed using a live halo, we made
a comparison to the results in \citet{2011ApJ...730..109F}
to test the effect of a live halo on the spiral structure.


In Fig.~\ref{fig:Q_amp} the relation between $Q$ and
the total amplitude ($\sum _{m=1}^{20}|A_m|^2$) at  $t=0, 0.125, 2.5, 5$, 
and 10\,Gyr is presented for our models at $R=2.2R_{\rm d}$.
The left panel shows the models which did not form a bar until 10\,Gyr. 
In \citet{2011ApJ...730..109F} we found that the spiral amplitudes
grow up to a maximum given by the following equation:
\begin{eqnarray}
A_{m} = 3.5C - 1.0 - 0.75Q^2,
\label{eq:amp}
\end{eqnarray}
where $C$ depends on the shape of the spiral arms. 
Following \citet{2011ApJ...730..109F}, we use $C=3/5$ because we assume that the local 
density enhancement due to spiral formation is described by the collapse 
of a homogeneous sphere.
If we consider that $|A_m|^2$ is similar to
the total amplitude, then the amplitudes of self-gravitating 
spiral arms can be analytically obtained as a function of $Q$. 
\citet{2011ApJ...730..109F} further found that after the spiral arm has 
reached its maximum amplitude, $Q$ increases because of the heating of 
the spiral arms and, following equation~\ref{eq:amp}, the amplitude starts to 
decay (black curve in Fig.~\ref{fig:Q_amp}).
For models without a bar we obtain the same results as for our 
models with live halos, although most of the data points are from before 
the amplitude decrease (see the left panel of Fig.~\ref{fig:Q_amp}).
Results for bar forming models are shown in the right panel of Fig.~\ref{fig:Q_amp}.
Before bar formation (small symbols), the relation between the
amplitude and $Q$ is similar to that in spiral galaxies, i.e., the amplitude
is smaller than the maximum predicted amplitude (black curve).
There are three points which exceed the limit, which was also seen in
\citet{2011ApJ...730..109F}.
After the bar forms, the developed amplitudes exceed that of the 
maximum given by $Q$, this is indicated by the larger symbols in the figures. 
The symbols are clearly above the theoretical line and $Q$ keeps increasing due
to the bar induced heating.

Snapshots of the bar models indicate that in the outer regions of the disk the spiral structure 
seems to disappear after the bar evolves (see Figs.~\ref{fig:snapshots_10Gyr} and
  \ref{fig:snapshots_spin_sp}), but the Fourier amplitudes are still large
compared to the models without bars, as shown in Fig.~\ref{fig:Q_amp}.
In the right panel we see that
the amplitude tends to decrease as $Q$ increases i.e., as the secular evolution
proceeds, except for models md1mb1,
md1mb1Rd1.5, and md0.5mb0, where the length of the bar reaches the reference radius
($2.2R_{\rm d}$)
by 10\,Gyr. In these models, we take the amplitude of the bar as the total amplitude
because the amplitude at $m=2$ is much larger than the others. 
  For these models, the relation between $Q$ and the spiral amplitude is not applicable
  at $2.2R_{\rm d}$ because the bar extends to this distance.
  Therefore, we also see that, even for barred spiral models, the spiral amplitude decreases as 
  $Q$ increases. The measured spiral amplitude, however,
  is much larger compared to the amplitude for spiral only models.

\begin{figure*}
\includegraphics[width=\columnwidth]{figures/Q_amp_spiral2.pdf}\includegraphics[width=\columnwidth]{figures/Q_amp_bar.pdf}
\caption{Relation between $Q$ and the total power of the Fourier amplitudes at the reference radii ($R=6.5\pm0.5$\,kpc, except for models md1mb1Rd1.5, 
md0.5mb1Rd1.5 and md0.5mb1Rd1.5s where $R=9.5\pm0.5$\,kpc) at $t=0, 0.125, 2.5, 5$, and 10\,Gyr. 
Large symbols indicate the data points from after the bar formation.
The total power is averaged over 20 snapshots ($\sim 200$\,Myr).
Black curves indicate
Eq.~\ref{eq:amp} for $|0.1A_{\rm m}|^2$. We adopt this value in order to 
  compare with Fig.~12 in \citet{2011ApJ...730..109F}.
  \label{fig:Q_amp}}
  
\end{figure*}

The distribution of $Q$ as a function of the galactic radius for models md1mb1 (strong bar) and md0.5mb4 (without bar, although 
a bar forms after 10\,Gyr) is presented in Fig.~\ref{fig:Q_ev}. 
The spiral arms heat up the disk moderately (left panel), but the bars
heat up the disk dramatically once they are formed (right panel).

Another effect of the bar is that it seems to prevent the formation of self-gravitating 
spiral arms, which corresponds to the number of spiral arms expected from swing amplification.
In Fig.~\ref{fig:mX_md1mb1}, the expected number of spiral arms ($m_X$)
for model md1mb1 at $t=1.25$, 2.5, 5, and 10\,Gyr is presented.
In the outer regions of the disk, $m_{X}$ decreases as the bar develops,
but still $m_X>2$. 
However, 
when measured using Fourier decomposition (Eq. (\ref{eq:Fourier})) 
we always find $m=2$ after bar formation (see also md1mb1 in Fig.~\ref{fig:snapshots_10Gyr}).
One possible reason for this is that the high $Q$ value prevents the formation of
self-gravitating spiral arms after the bar has formed, even though 
it is known that the bar induces spirals. 
In real galaxies, we often see multiple spiral arms in the bar's outer region.
This discrepancy may be because our simulations ignore the effect of gas.
 

To conclude, using a live halo instead of an analytic halo potential is important when studying the 
formation and evolution of the bar and its properties in disk galaxies. Since the live halo influences the angular 
momentum of the bar, its speed and length will be rather different than when
an analytic halo is adopted. The 
effect on spiral structure without a bar, however, is less pronounced. 


\begin{figure*}
\includegraphics[width=\columnwidth]{figures/Q_md0.5_mb4.pdf}\includegraphics[width=\columnwidth]{figures/Q_md1_mb1.pdf}
\caption{Time evolution of Toomre's $Q$ for models md0.5mb4 (left) and md1mb1 (right). The peak in the cyan curve at $R\sim7$kpc corresponds to the end of the bar.\label{fig:Q_ev}}
\end{figure*}

\begin{figure}
\includegraphics[width=\columnwidth]{figures/mX_disk_md1mb1.pdf}
\caption{The number of spiral arms using Eq.~\ref{eq:mX} at $t=0$, 0.125, 0.25, 5, and 10\,Gyr 
for model md1mb1. The peak in the cyan curve at $R\sim7$\,kpc corresponds to the bar end.
\label{fig:mX_md1mb1}}
\end{figure}
