

\section{Results}

\subsection{The effects of disk and bulge masses}

We study the effect that the disk and bulge mass fractions have on the halo
and on the morphology of spiral arms and bars.  In
Fig.~\ref{fig:rotation_curves}, we summarize the initial 
rotation curves of
several models: models md1mb1, md0.5mb1, md0.3mb1, and md0.1mb1
(varying disk mass) and models md0.5mb0 and md0.5mb3 (varying bulge
mass).  We present the snapshots at $t=5$ and 10\,Gyr in
Figs.~\ref{fig:snapshots_5Gyr} and \ref{fig:snapshots_10Gyr}.  As
reported in previous studies, the number of spiral arms increase as
the disk mass decreases
\citep{1985ApJ...298..486C,2003MNRAS.344..358B,
2011ApJ...730..109F,2015ApJ...808L...8D}
and the formation of the bar is delayed when the bulge mass is increased
\citep{2013MNRAS.434.1287S}. This corresponds to the effect that
centrally concentrated potentials prevent the formation of bars
\citep{2001ApJ...546..176S}.


\begin{figure*}
\includegraphics[width=1.8\columnwidth]{figures/rotation_curves_v2.pdf}
    \caption{Rotation curves of the initial conditions for models 
      md1mb1, md0.5mb1, md0.3mb1, md0.1mb1, md0.5mb0, and md0.5mb3\label{fig:rotation_curves}.
      The gray solid, dotted, dashed, and dot-dashed curves indicate disk, bulge, halo, and total rotation curves of model md1mb1.}
\end{figure*}

\begin{figure*}
\includegraphics[width=1.8\columnwidth]{figures/snapshots_5Gyr.jpg}
%    \includegraphics[width=1.8\columnwidth]{figures/snapshots_5Gyr.jpg}
    \caption{Snapshots (surface densities) at $t=5$ Gyr for models 
      md1mb1, md0.5mb1, md0.3mb1, md0.1mb1, md0.5mb0, and md0.5mb3.
      %\footnote{\dag}{Movies are available in online materials.}
      \label{fig:snapshots_5Gyr}}
\end{figure*}

\begin{figure*}
\includegraphics[width=1.8\columnwidth]{figures/snapshots_10Gyr.jpg}
%   \includegraphics[width=1.8\columnwidth]{figures/snapshots_10Gyr.jpg}
    \caption{Snapshots (surface densities) at $t=10$ Gyr for models 
      md1mb1, md0.5mb1, md0.3mb1, md0.1mb1, md0.5mb0, and md0.5mb3.
      \label{fig:snapshots_10Gyr}}
\end{figure*}






\subsection{Spiral Arms}
\subsubsection{Number of Spiral Arms}
We first focus on the number of spiral arms. As is shown in 
Fig.~\ref{fig:snapshots_10Gyr}, the number of spiral arms increases 
as the disk mass decreases. This relation can be understood by swing 
amplification theory \citep{1981seng.proc..111T}.
In a differentially rotating disk, the epicycle motions of particles 
are amplified and the amplification factor $X$ is written as 
\begin{eqnarray}
X \equiv \frac{k_{\rm crit}R}{m} = \frac{\kappa^2R}{2\pi G\Sigma m}.
\label{eq:X}
\end{eqnarray}
Here $R$ is the distance from the galactic center and $m$ is
the spiral multiplicity (the number of spiral arms). This $m$ is
  usually used for spirals, we adopt $m=2$ for bars, because bars are
  traced as a $m=2$ mode when using a Fourier decomposition. For
typical disk models the amplification is large for $1\lesssim
X\lesssim 2$ and rapidly drops for $2\lesssim X\lesssim
3$~\citep{1965MNRAS.130..125G,1966ApJ...146..810J,1981seng.proc..111T}.
The critical wave number $k_{\rm crit}$ (and also the critical wave
length $\lambda_{\rm crit}$) is obtained from the local stability in a
razor-thin disk using the tight-winding approximation
\citep{1964ApJ...139.1217T}:
\begin{eqnarray}
k_{\rm crit} &=& \frac{\kappa^2}{2\pi G\Sigma},\\
\lambda_{\rm crit} &=& \frac{2\pi}{k_{\rm crit}} = \frac{4\pi^2G\Sigma}{\kappa^2},
\end{eqnarray} 
where $\Sigma$ and $\kappa$ are the surface density and the epicyclic
frequency of the disk, respectively \citep[see also section 6.2.3 of
][]{2008gady.book.....B}.  

Equation (\ref{eq:X}) also predicts the number of spiral arms that
form in a disk.  By inverting equation (\ref{eq:X}) one obtains a
relation for $m$ as a function of the swing amplification factor
$X$:
\begin{eqnarray}
m =\frac{\kappa^2R}{2\pi G\Sigma X}.
\label{eq:mX}
\end{eqnarray}
Because the perturbations grow most efficiently for $X\sim1$--2, we
can relate $m$ as a function of $R$ (here, both $\kappa$ and $\Sigma$
are written as a function of $R$.). The
predicted number of spiral arms from swing amplification theory
has been validated using numerical simulations
\citep{1985ApJ...298..486C,2015ApJ...808L...8D}.

For models with different disk masses, 
we estimate the number of spiral arms using equation (\ref{eq:mX})
and present the results in Fig.~\ref{fig:m_measured}. The dashed curves 
present the estimated number of spiral arms as a function of galactic radii 
where we adopt $X\sim2$ following 
\citet{1985ApJ...298..486C, 2014PASA...31...35D}.
Given the curves we expect fewer arms for the more massive models
and the number of arms increases for larger radii ($R$).

We also determine the number of spiral arms for each of the 
simulated galaxies and overplot the results in Fig.~\ref{fig:m_measured}.
We use a Fourier decomposition of the disks surface density:
\begin{eqnarray}
\frac{\Sigma(R,\phi)}{\Sigma_0(R)}=\sum_{m=0}^{\infty}A_m(R)e^{im[\phi-\phi_m(R)]},
\label{eq:Fourier}
\end{eqnarray}
where $A_m(R)$ and $\phi_m(R)$ are the Fourier amplitude and phase angle for
the $m$-th mode at $R$, respectively. 
We measure the amplitude at each radius up to 20 kpc using radial bins of 
$\Delta R=1$ kpc.
When a bar formed we obtained $m=2$ as strongest amplitude. 

Because the spiral arms are transient structures the dominant number 
of spiral arms,
those with the highest amplitude, changes over time~\citep{2011ApJ...730..109F}.
We therefore use the most frequently appearing 
number of spiral arms (hereafter, principal mode) 
as the number of arms ($m$) of the model.
The principal mode is measured between 2.5 and 14.5\,kpc at 2\,kpc intervals,
and for each the 1000 snapshots
between 0 and 10\,Gyr. The results are presented in Fig.~\ref{fig:m_measured}.
The $m=2$ mode will always become the dominant mode once a bar has formed (see 
red circles in the figure), but spiral arms might have formed before 
the bar formation.
We therefore also show the principal mode before the bar 
formation (triangular symbols). These results are roughly consistent 
with the number of spiral arms predicted by Eq.~\ref{eq:mX}:
the number of spiral arms increases as the galactic radius increases
and as the disk mass decreases.
For model md0.1mb1 we measure a principal mode of 2 at $R=6.5$\,kpc.
However, when we look at Fig.~\ref{fig:snapshots_10Gyr}, we 
see more than 2 faint spiral arms. We therefore also measured the strongest 
modes excluding $m=2$. These modes are indicated by the square symbols.
We perform the same analysis for all the other models and measure the
number of spiral arms (for the details of the individual evolution of these
models, see the following sections and Appendix~\ref{Sect:AppB}). 
The results are summarized in Table~\ref{tb:pitch_angle}.

Numerical results tend to deviate from theoretical predictions
as the number of spiral arms increases. The more spiral arms the 
fainter they become, which makes them harder to measure and identify
in the simulations. Our method has problems tracing these faint spiral arms
due to their lower amplitude. This is in particular the case for model md0.1mb1, for which
the amplitude becomes comparable to the particle noise 
(see Table~\ref{tb:pitch_angle}): it is therefore difficult to 
unambiguously detect spiral arms.

In Fig.~\ref{fig:m_measured} we demonstrate how the number of spiral arms
changes with galactic radius. The number of spiral arms and 
the mass fraction of the disk are measured at $2.2R_{\rm d}$.
The relation between the measured number of spiral arms ($m$) at $2.2R_{\rm d}$
and the disk mass fraction ($f_{\rm d}$) is presented in Fig.~\ref{fig:m_fdisk}
where  ($f_{\rm d}$) is defined as:
\begin{eqnarray}
  f_{\rm d}\equiv \left( \frac{V_{\rm c, d}(R)}{V_{\rm c, tot}(R)} \right)^2_{R=2.2R_{\rm d}},
\label{eq:fd} 
\end{eqnarray}
where $V_{\rm c, d}$ and $V_{\rm c, tot}$ are the circular velocity of the disk
and of the whole galaxy, respectively. We find that $m$, before the bar 
formation, decreases as $f_{\rm d}$ increases. This matches the results of 
\citet{2015ApJ...808L...8D} (their figure 3).




The number of spiral arms is furthermore expected to depend on the shear rate:
\begin{eqnarray}
\Gamma = -\frac{d\ln \Omega}{d \ln R},
\label{Eq:ShearRate}
\end{eqnarray}
where $\Omega$ is the angular velocity. The value of $\Gamma$ indicates the 
shape of the rotation curve: $\Gamma=1$ indicates a flat rotation curve, $\Gamma>1$ indicates
a declining rotation curve, and $\Gamma<1$ indicates an increasing rotation curve.
\citet{1984PhR...114..319A} found that the swing amplification factor also depends on 
$\Gamma$. They computed the swing amplification factor as a function of $X$
for different $\Gamma$ values and found that the peak amplification factor
depends on $\Gamma$; $X\sim 1$ for $\Gamma = 0.5$ and $X\sim 2$ for $\Gamma=1.5$
(see their figure 26). 
Applying these results to Eq. (\ref{eq:mX}), a smaller value of $m$ is expected for 
a larger $\Gamma$.
In Fig.~\ref{fig:m_fdisk}, we show the relation between
$\Gamma$, disk mass fraction ($f_{\rm d}$) and the number of spiral arms ($m$). 
In this figure, we confirm that a larger shear rate 
results in a smaller number of spirals.


\begin{figure}
\includegraphics[width=\columnwidth]{figures/mX_disk_measured.pdf}
    \caption{Theoretically predicted (using Eq.~\ref{eq:mX}, dashed 
    curves) and measured number of spiral arms (symbols) for models
    md0.1mb1 (magenta), md0.3mb1 (cyan), md0.4mb1 (blue), md0.5mb1 (green),
    and md1mb1 (red) from top to bottom. 
    Filled circles indicate the most frequently appearing number of arms (principal modes)
    over a 10\,Gyr period. 
    Triangle symbols indicates the principal mode before the formation of the bar.
    Square symbols indicate the principal mode for model md0.1mb1 excluding $m=2$.
    The symbols for md0.3mb1 and md0.5mb1 are shifted by 0.1\,kpc  
    to avoid overlapping points.\label{fig:m_measured}}
\end{figure}

\begin{figure}
\includegraphics[width=\columnwidth]{figures/fdisk_marm_shear.pdf}
\caption{The number of spiral arms ($m$) before the bar formation epoch.  
  The disk mass fraction ($f_{\rm d}={V_{\rm c, d}(R)^2/V_{\rm c, tot}(R)^2}_{R=2.2R_{\rm d}}$) is for
  the models with $Q_0=1.2$ and without halo spin. 
  We measure $m$ at 6.5\,kpc, which is close to $2.2R_{\rm d}$, for all models except models md1mb1Rd1.5 and md0.5mb1Rd1.5.
  For these models $R_{\rm d}$ is 1.5 times larger and we therefore measure $m$ at 9.5\,kpc.  
  The shear rate ($\Gamma$), measured in the initial conditions at $2.2R_{\rm d}$, is indicated by the different 
  symbols. 
  \label{fig:m_fdisk}  
  }
\end{figure}


\begin{table*}
\begin{center}
\caption{Pitch angle and number of spiral arms\label{tb:pitch_angle}}
\begin{tabular}{lccccc}
%\tabletypesize{\scriptsize}
%\rotate
%\tablewidth{0pt}
%\startdata 
\hline
Model    & Radius ($R$) & Shear rate ($\Gamma$)& Pitch angle ($i$) & Maximum amplitude & Number of arms ($m$)\\
      &      (kpc) &            &  (degree) &                     &  \\ 
\hline  \hline
md1mb1     & 10 & 0.991 & 18 & 0.198 & 2 \\
           & 12 & 1.04 & 18 & 0.164 & 2 \\
           & 14 & 1.07 & 19 & 0.161 & 2 \\
md1mb1s0.8 & 8  & 0.902 & 18 & 0.261 & 2 \\
           & 10 & 0.991 & 19 & 0.169 & 2 \\
           & 12 & 1.04 & 18 & 0.181 & 2 \\
           & 14 & 1.07 & 22 & 0.197 & 2 \\
md0.5mb1   & 6 & 0.804 & 25 & 0.101 & 4 \\
           & 8 & 0.875 & 25 & 0.121 & 6 \\
           & 10 & 0.944 & 27 & 0.0836  & 7 \\
           & 12 & 0.983 & 18 & 0.0561 & 2 \\
md0.4mb1   & 4 & 0.758 & 33 & 0.0372 & 5 \\
           & 6 & 0.796 & 33 & 0.0442 & 6 \\
           & 8 & 0.863 & 27 & 0.0378 & 7 \\
           & 10 & 0.926 & 26 & 0.0249 & 9 \\
md0.3mb1   & 4 & 0.774 & 32 & 0.0483 & 4 \\
           & 6 & 0.792 & 34 & 0.0453 & 7 \\
           & 8 & 0.847 & 26 & 0.0347 & 9 \\
           & 10 & 0.907 & 29 & 0.0189 & 10 \\
md0.1mb1   & 6 & 0.744 & 5 & 0.00965 & 1 \\
           & 8 & 0.777 & 3 & 0.0116 & 1 \\
           & 10 & 0.832 & 3 & 0.0111 & 2 \\
           & 12 & 0.885 & 3 & 0.0107 & 2 \\
md0.5mb0   & 8 & 0.833 & 15 & 0.185 & 2 \\
           & 10 & 0.888 & 11 & 0.169 & 2 \\
           & 12 & 0.905 & 12 & 0.199 & 2 \\
md0.5mb3   & 6 & 0.991 & 25 & 0.131 & 2 \\
           & 8 & 0.955 & 25 & 0.119 & 5 \\
           & 10 & 0.963  & 22 & 0.0959 & 3 \\
md0.5mb4   & 6 & 0.975 & 25 & 0.111 & 4 \\
           & 8 & 0.977 & 23 & 0.109 & 4 \\
           & 10 & 0.996 & 25 & 1.064 & 8 \\
md0.5mb4rb3 & 6 & 0.963 & 26 & 0.0924 & 5 \\
            & 8 & 0.961 & 25 & 0.0923 & 5 \\
            & 10 & 0.965 & 26 & 0.0725 & 5 \\
md1.5mb5   & 8 & 1.01 & 21 & 0.281 & 2 \\
           & 10 & 1.07 & 16 & 0.189 & 2 \\
           & 12 & 1.09 & 14 & 0.212 & 2 \\
           & 14 & 1.13 & 11 & 0.269 & 2 \\
md1mb10    & 6 & 1.10 & 27 & 0.287 & 2 \\
           & 8 & 1.07 & 22 & 0.269 & 2 \\
           & 10 & 1.05 & 18 & 0.211 & 2 \\
md1mb1Rd1.5   & 10 & 0.878 & 24 & 0.152 & 4 \\
           & 12 & 0.963 & 28 & 0.169 & 4 \\
           & 14 & 1.03 & 18 & 0.181 & 4 \\
md0.5mb1Rd1.5 & 10 & 0.850 & 29 & 0.0777 & 7 \\
           & 12 & 0.921 & 27 & 0.0671 & 8 \\
           & 14 & 0.977 & 24 & 0.0572 & 8 \\
md0.5mb1Rd1.5s & 10 & 0.850 & 26 & 0.0719 & 7 \\
           & 12 & 0.921 & 26 & 0.0804 & 7 \\
           & 14 & 0.977 & 24 & 0.0553 & 8 \\
\hline
%\enddata
\end{tabular}
\end{center}
%\medskip
\end{table*}


\subsubsection{Pitch angle}

The pitch angle is an important parameter in the discussion on the morphology 
of spiral galaxies.  We measure the pitch angle of our simulated 
galaxies using the Fourier transform method~\citep[see ][]{2013A&A...553A..77G, 2015MNRAS.454.2954B}.
Using the same Fourier decomposition (Eq.~\ref{eq:Fourier}) as for 
the bar amplitude we compute the phase angle,  $\phi_m(R)$.
Next, the pitch angle at $R$ for $m$ is obtained by using
\begin{eqnarray}
\cot i_m(R) = R\frac{d\phi (R)_m}{dR}.
\end{eqnarray}


In numerical simulations the pitch angle changes over
time~\citep{2013ApJ...763...46B, 2013A&A...553A..77G, 2015MNRAS.454.2954B}.
The pitch angle of spiral arms increases and
decreases repeatedly as the amplitude of transient spiral arms
increases and decreases \mbox{~\citep[see Figures 4 and 5
  in][]{2015MNRAS.454.2954B}}. Furthermore, the number of spiral arms also 
changes as a function of $R$ as we saw in previous sections. We 
therefore measure the most appearing pitch angle for the most 
appearing mode (principal mode) at each galactic radius.  
Following \citet{2015MNRAS.454.2954B},
we define the most frequently appearing pitch angle weighted by the
Fourier amplitude as the pitch angle. 
In Table \ref{tb:pitch_angle}, we report the measured
pitch angle and the number of spiral arms ($m$) for that angle.  
Note that we measure pitch angles at $\gtrsim 2.2R_{\rm d}$, 
except for barred galaxies where we, to avoid the bar's influence,
use an $R$ that is larger than the maximum bar length.

\citet{1966ApJ...146..810J} suggested that the pitch 
angle ($i$) is determined by the shear rate ($\Gamma$, see Eq.~\ref{Eq:ShearRate} and 
Table~\ref{tb:pitch_angle}) of the disk. The relation 
between the shear rate and pitch angle was recently investigated 
using both numerical simulations and analytic models 
\citep{2014ApJ...787..174M,2016ApJ...821...35M}.
The relation is also suggested by galaxy observations 
\citep{2005MNRAS.361L..20S,2006ApJ...645.1012S}.
In Fig.~\ref{fig:pitch_angle}, the relation between $\Gamma$
and $i$ is presented (averaged for each model). 
In order to compare our results with the theory, we also
present the relation between $\Gamma$ and $i$ as derived 
by \citep{2014ApJ...787..174M}:
\begin{eqnarray}
\tan i = \frac{7}{2}\frac{\sqrt{4-2\Gamma}}{\Gamma}.
\label{eq:pitch_shear}
\end{eqnarray}
In the figure this relation is presented with a dashed curve.
Except for models md0.5mb0 and md0.1mb1, the measured relation in our 
simulations is consistent with the theoretical curve.

We now briefly discuss these two outliers.
Model md0.5mb0 has, due to the lack of a bulge component,
a strong bar resulting in a ring structure around the 
bar end (see Fig.~\ref{fig:snapshots_10Gyr}).
Here the ring structure may affect the spiral structure in the 
outer disk region.
The pitch angle as measured before bar formation ($t<2.5$\,Gyr)
is $23^{\circ}$ and $m=$5--10 at $R=8$--12\,kpc which is 
consistent with the curve in Fig.~\ref{fig:pitch_angle}.

The other outlier, md0.1mb1, has a relatively low-mass disk which results
in low-contrast spiral structure (virtually invisible; see Fig.~\ref{fig:snapshots_10Gyr}).
The amplitude, $\sim 0.01$ (see Table~\ref{tb:pitch_angle}), is
comparable to the particle noise level (see Fig.~\ref{fig:A2_max_mdisk}).
This makes it difficult  
to accurately measure the number of spiral arms 
and their pitch angle using the scheme we adopted.



We also present the relation between the shear rate and 
the pitch angles of observed galaxies~\citep{2006ApJ...645.1012S} in 
Fig.~\ref{fig:pitch_angle} (black points). These points are also 
distributed around the theoretical curve with a scatter
larger than the simulated galaxies.   


\begin{figure}
\includegraphics[width=\columnwidth]{figures/shear_angle.pdf}
\caption{The relation between shear rate ($\Gamma$) and pitch angle ($i$)
for the simulated galaxies shown in Table \ref{tb:pitch_angle} (red squares) 
and observed galaxies \citep{2006ApJ...645.1012S} (black circles). 
For simulated galaxies, the error bars on the $x$-axis indicate the range of 
shear rates depending on the radius at which we measured $\Gamma$ and $i$.
The error bars on the $y$-axis indicate the 
standard deviations of the measured pitch angles at each radius. 
The black dashed curve indicates the 
result of \citet{2014ApJ...787..174M} given by equation (\ref{eq:pitch_shear}). \label{fig:pitch_angle}}
\end{figure}




\subsection{Bar Formation}\label{Sect:bf}

We will now investigate the formation of bars.  In order to define the
bar formation, we measure the time evolution of the length (radius) of the bar
and its amplitude which develops in our galaxy simulations. Because
this has to be done for one thousand snapshots for each galaxy
simulation we adopt a relatively simple method for measuring these
properties.  We measure the Fourier amplitude (Eq.\,\ref{eq:Fourier})
in radial bins of 1 kpc for the $m=2$ mode ($A_2(R)$), record the maximum
value and use this as the bar amplitude ($A_{\rm 2, max}$).  In the
left panel of Fig.~\ref{fig:A2_max_mdisk}, we present the time
evolution of $A_{\rm 2, max}$ for models md1mb1 to md0.1mb1.  Once a
bar begins to develop the amplitude increases exponentially and either
reaches a stable maximum (as is the case in model md1mb1) or decreases
slightly to increase again a few Gyr later (see model mb0.5mb1). Model
md0.1mb1 did not form a bar within 10\,Gyr.  

We also measure the bar
length using the method described in~\citet{2012MNRAS.425L..10S}
and \citet{2015PASJ...67...63O}. In this method we compute the phase angle
($\phi_{2}(R)$) and amplitude ($A_{2}(R)$) of the bar at each radius
using the Fourier analysis (Eq.~\ref{eq:Fourier}). 
As $R$ increases, $A_{2}(R)$ increases, reaches its maximum in the middle
of the bar, and then decreases. 
We define the radius at which $A_{2}(R)$ reaches its maximum value
as $R_{\rm max}$ and the phase at $R_{\rm max}$ as the phase 
angle of the bar ($\phi_{\rm 2, max}$). Starting at $R_{\rm max}$,
we compare $\phi_{2}(R)$ with $\phi_{\rm 2, max}$. 
When $\Delta \phi=|\phi_{2}(R)-\phi_{\rm 2, max}|>0.05\pi$, we consider 
that the bar has ended and define the radius as the bar's size, $R_{\rm b}$.
Hereafter, we refer to $R_{\rm b}$ as the length of the bar. The time evolution of
the bar length is presented in the right panel of
Fig.~\ref{fig:A2_max_mdisk}. The length of the bar grows continuously
until the end of simulation ($t=15$ Gyr).

We define the epoch of bar formation ($t_{\rm b}$) as the moment 
when $A_{\rm 2, max} > 0.2$ and $R_{\rm b} > $~1\,kpc. In our models the bar was always longer 
than 1 kpc when  $A_{\rm 2, max}>0.2$. 
In most cases the bar amplitude increases exponentially and therefore the critical 
amplitude has little effect on the moment the bar forms.
For models md0.4mb1 and md0.3mb1, which did not form a bar within 10 Gyr, we continued
the simulations until a bar formed after 13 and 18\,Gyr, respectively
(also see Fig.~\ref{fig:A2_max_mdisk} and Table~\ref{tb:bar_crit}).
We continued the simulations up to 15\,Gyr for md0.5mb4 and md0.5mb4rb3 
to confirm that they form a bar, which they do around  $\sim 10$Gyr.
 
We subsequently investigate the effect of the bulge mass on the bar formation. 
It was suggested that a massive central component, such as a bulge, have a
stabilizing effect 
on the disk and thereby prevents bar formation~\citep{2001ApJ...546..176S,2013MNRAS.434.1287S}.  
To test this we perform a set of simulations in which we vary the bulge mass.
We make the bulge 0 (md0.5mb0), 3 (md0.5mb3) and 4 (mb0.5mb4) times as massive 
as the bulge of model md0.5mb1. We further added model md0.5mb4rb3 with the same mass as md0.5mb4,
but with increased bulge scale length.
The amplitude evolution and bar length of these models is
presented in Fig.~\ref{fig:A2_max_mbulge}. 
When the bulge mass fraction increases the bar formation is delayed 
due to the the decreasing disk mass-fraction ($f_{\rm d}$)
(also see Table~\ref{tb:bar_crit}). 
These results are consistent with observations where the 
fraction of barred galaxies increase when the bulge to disk mass
ratio decreases and where the barred galaxies fraction even increases 
to $\sim 87$\% for the extreme case of bulge-less galaxies.
We further confirm that the bulge scale length does not effect the epoch of bar 
formation, but  the bar length at the end of the 
simulation (at 15\,Gyr). The final bar length for model md0.4mb4rb3 
is longer than that for md0.4mb4 (see Fig.~\ref{fig:A2_max_mbulge}). 
The bar formation epoch for all models is presented in Table~\ref{tb:bar_crit}.




\begin{figure*}
\includegraphics[width=\columnwidth]{figures/mode_A2max_mdisk.pdf}
\includegraphics[width=\columnwidth]{figures/mode_Dbar_mdisk.pdf}
\caption{
Time evolution of the maximum amplitude for $m=2$ (left)
and the bar length (right)
for models md0.1mb1, md0.3mb1, md0.4mb1, md0.5mb1, and md1mb1..
Black curves in the right panel indicate the bar length averaged over every
20 snapshots ($\sim 0.2$Gyr).
\label{fig:A2_max_mdisk}}
\end{figure*}




\begin{figure*}
\includegraphics[width=\columnwidth]{figures/mode_A2max_mbulge.pdf}
\includegraphics[width=\columnwidth]{figures/mode_Dbar_mbulge.pdf}
\caption{Same as Fig.~\ref{fig:A2_max_mdisk} but for models md0.5mb0,
md0.5mb1, md0.5mb3, md0.5mb4, and md0.5mb4rb3.
\label{fig:A2_max_mbulge}}
\end{figure*}

Although increasing the bulge mass sequentially delays the formation of a bar,
the bulge mass fraction is not a critical parameter for the bar formation. We tested
some parameters and found that the disk-mass to the total mass fraction is a more
critical parameter for the bar formation epoch.

In Fig.~\ref{fig:bfe}, we present the relation
between the bar formation epoch and the disk mass fraction, $f_{\rm d}(=1/X'$),
where $X'$ is a parameter adopted by \citet{2008ApJ...679.1239W} as 
a bar formation criterion:
\begin{eqnarray}
 X'\equiv 1/f_{\rm d} = \left( \frac{V_{\rm c, tot}(R)}{V_{\rm c, d}(R)} \right)^2_{R=2.2R_{\rm d}}.
\label{eq:Xprime}
\end{eqnarray}
They argued that $X'\lesssim 3$ (for $f_{\rm d}\gtrsim 0.3$) is the
bar formation criterion in their simulation.  The epoch of bar
formation increases exponentially for decreasing disk mass-fraction,
although the scatter is large.  We fit an exponential function to our
results obtained with $N_{\rm d}=8$M and $Q_0=1.2$ and find that
$t_{\rm b}=0.146\pm 0.079 \exp [(1.38\pm 0.17)/f_{\rm d}]$.  The result
is indicated by the dashed black line in Fig.~\ref{fig:bfe}.

The resolution of the simulation in the number of particles is an important source for the scatter
 \citep{2009ApJ...697..293D}; 
a smaller number of particles for the same model results in faster bar formation.
We confirm this by performing simulations with an order of magnitude lower
resolution (0.8\,M disk particles, open circle symbols), and indeed find that the bar forms
earlier for these models in comparison with the high resolution models (Fig.~\ref{fig:bfe}, Table~\ref{tb:bar_crit}). 
Another parameter which is known to affect the epoch of bar formation is the value of $Q$.
In Fig.~\ref{fig:bfe} we also plot models md0.5mb1Q2.0 and md0.5mb1Q0.5, 
which are identical to model md0.5mb1, with the exception that $Q_0=2.0$ and 0.5, respectively.
As was shown in previous studies \citep[c.f.,][]{1986MNRAS.221..213A}, 
a larger value of $Q_0$ leads to a delay in the formation of the bar 
(see Appendix A2 for details).




\begin{figure}
\includegraphics[width=\columnwidth]{figures/bar_formation_epoch.pdf}
\caption{Bar formation epoch ($t_{\rm b}$) and disk mass fraction 
($f_{\rm d}=1/X'$). 
Filled squares and open circles indicate models with $N_{\rm d}=8$M and 0.8M, respectively. 
Red (blue) indicates models with a bulge-to-disk mass ratio ($B/D$) of 
$>0.5$ ($<0.5$).
The dashed curve indicates a fit to the models with $N_{\rm d}=8$M and $Q_0=1.2$
(squares): $t_{\rm b}=0.146\pm 0.079 \exp [(1.38\pm 0.17)/f_{\rm d}]$.
\label{fig:bfe}}
\end{figure}



The relation between the moment of bar formation ($t_{\rm b}$) and 
the mass fraction of the disk ($f_{\rm d}$) can be
understood from Toomre's $X$ parameter (see Eq.~\ref{eq:X}). 
For a given value of $m$ we can calculate $X$ as a function of the disk radius $R$.
When we adopt $m=2$, i.e. the bar, we obtain $X$ for the bar mode ($X_2$) as a function of $R$.
This distribution is presented in Fig.~\ref{fig:X_disk}. Here, we see that $X_2$
reaches minimum values at $R\sim 2$ kpc. We find that the minimum value of $X_2$ ($X_{\rm min}$)
is roughly correlated with $X'(=1/f_{\rm d})$, and the relation between $X_{\rm min}$ and $X'$ is 
presented in Fig.~\ref{fig:X}. Thus, the disk fraction $f_{\rm d}$ is connected to 
Toomre's $X$.
As shown by \citet{1981seng.proc..111T}, the amplitude grows most efficiently for $1<X<2$
and decreases exponentially when $X$ increases from $\sim 2$ to $\sim 3$.
We find that models in which a bar forms have a minimum value of $X_2\lesssim 2$ 
(see Fig.~\ref{fig:X_disk}).
We conclude, based on these results, that there is no particular rigid criterion for bar
formation, but that the bar formation epoch starts to increase exponentially
when $f_{\rm d}\gtrsim0.3$, or equivalently, if $X'\lesssim0.3$.


\begin{figure}
\includegraphics[width=\columnwidth]{figures/X_disk.pdf}
\caption{$X$ values as a function of radius for $m=2$ mode. \label{fig:X_disk}}
\end{figure}


\begin{figure}
\includegraphics[width=\columnwidth]{figures/X_Xmin.pdf}
\caption{Relation between $X'$ and the minimum value of $X$ ($X_{\rm min}$).\label{fig:X}}
\end{figure}



We also test the bar formation criterion previously suggested by
\citet{1982MNRAS.199.1069E}, who
proposed that bar formation depends on the mass of the
disk ($M_{\rm d}$) within radius $R_{\rm d}$:
\begin{eqnarray}
\epsilon _{\rm m} \equiv \frac{V_{\rm c, max}}{(GM_{\rm d}/R_{\rm d})^{1/2}}<1.1.
\label{eq:epsilon}
\end{eqnarray}
Here $V_{\rm c, max}$ is the maximum circular velocity in the disk.
For this criterion, \citet{2008MNRAS.390L..69A} showed some exceptional
cases using $N$-body simulations of disk models with live halos.
In Table \ref{tb:bar_crit}, we present $\epsilon_{\rm m}$ (Eq.~\ref{eq:epsilon}),
and we confirm that in our simulations Efstathiou's criterion cannot always predict 
the bar formation.






\begin{table*}
\begin{center}
  \caption{Details on the formation of the bar.  The columns give the
    model name, a boolean indicating if a bar formed (Y) or not (N)
    within the simulation time period (0--20\,Gyr), the moment and
    criteria of the bar formation.
%\tablenotetext{\dagger}{Bar formation epoch}
    \label{tb:bar_crit}}
\begin{tabular}{lccccc}
%\tabletypesize{\scriptsize}
%\rotate
%\tablewidth{0pt}
%\startdata 
\hline
Model    & Bar formation & Bar formation epoch &  \multicolumn{3}{c}{Bar formation criteria}\\
      &                                  & $t_{\rm b}$ (Gyr)   &  $\epsilon_{\rm m}$ & $X_{\rm min}$ & $X'(\equiv 1/f_{\rm d})$\\ %& $\eta$\\ 
\hline  \hline
md1mb1     & Y & 0.64 & 0.824 & 0.997 & 1.80 \\%& 1.24 \\
md1mb1s0.65 &  Y & 0.83 & 0.824 & 0.997 & 1.80 \\%& 1.24 \\
md1mb1s0.8 &  Y & 0.73 & 0.824 & 0.997 & 1.80 \\%& 1.24 \\
\\      
md0.5mb1   &  Y & 6.3 & 1.03 & 1.68 & 2.61 \\%& 0.621  \\ 
md0.4mb1   &  Y & 13 & 1.00  & 1.97 & 2.96 \\%& 0.510 \\
md0.3mb1   &  Y & 18 & 1.87 & 2.41 & 3.49 \\%& 0.401 \\
md0.1mb1   &  N & - & 2.12 & 5.78 & 8.34 \\%& 0.136 \\
\\
md0.5mb0   & Y & 3.1 & 1.08  & 0.827 & 2.28 \\%& 0.780 \\
md0.5mb3   & Y & 7.0 & 1.26  & 2.17 & 2.88 \\%& 0.532 \\
md0.5mb4   & Y  & 9.9 & 1.38  & 2.58 & 3.06 \\%& 0.487 \\
md0.5mb4rb3 & Y  & 9.9 & 1.15  & 2.80 & 3.17 \\%& 0.461 \\
\\
md1.5mb5   &  Y & 1.9 & 1.42 & 1.30 & 1.69 \\%& 1.25 \\
md1mb10    & Y  & 7.5 & 1.60 &  2.71 & 2.86 \\%& 0.537 \\
\\
md1mb1Rd1.5   & Y & 2.6 & 0.960  & 1.41 & 2.38 \\%& 0.724 \\
md0.5mb1Rd1.5 & N & - & 1.25 & 2.35 & 3.77 \\%& 0.361 \\
md0.5mb1Rd1.5s & N & - & 1.25 & 2.35 & 3.77 \\%& 0.361 \\
\\
md0.5mb1Q0.5 &  Y & 0.27 & 1.03 & 1.68 & 2.61 \\%& 0.621  \\ 
md0.5mb1Q2   &  Y & 8.8 & 1.03 & 1.68 & 2.61 \\%& 0.621  \\ 
\hline
%\enddata
\end{tabular}
\end{center}
\end{table*}


\subsection{Bulge-to-Disk mass ratio}

From an observational perspective such as in the Hubble sequence \citep{1926ApJ....64..321H}, 
the bulge-to-disk mass ratio ($B/D$) is related to the pitch angle. 
Sa galaxies have a larger bulge-to-disk mass-ratio
compared to Sb and Sc galaxies \citep{1961hag..book.....S}.
To test this hypothesis, we simulate an extra model (md1mb10), with a much larger $B/D$ than the 
the other models. 


The disk-to-halo mass ratio of this model is relatively large  ($B/D=1.0$),
but the disk-to-total mass ratio ($f_{\rm d}=0.35$) is not 
as large as for models which form a bar before forming spiral arms.
The S0--Sa galaxies, for example,
NGC\,1167 \citep{2008ARep...52...79Z} and M\,104
\citep{2006MNRAS.371.1269T}, have such a massive bulge and also many
narrow spiral arms. 

In Fig.~\ref{fig:snapshots_mb10} we present the rotation curves (left
panel) and the surface-density images (middle and right panels) for model
md1mb10.  This model formed multiple spiral arms similar to Sa galaxies 
before it developed a bar. 
The measured pitch angle was $\sim 20^{\circ}$ within 10\,kpc 
but less than 10$^{\circ}$ at $R>10$\,kpc. 

In the previous subsection, we demonstrated the relation between the 
pitch angle and the shear rate ($\Gamma$). Here, in
Fig.~\ref{fig:shear_rate}, we present the relation between $\Gamma$
and the bulge-to-disk mass ratio ($B/D$). 
For our models with $f_{\rm d}>0.3$,
$\Gamma$ appears to correlate with $B/D$, however the models with a small
$f_{\rm d}$ tend to have a small $\Gamma$. 
In the latter sequence, the bulge mass is similar but the disk mass 
decreases as $B/D$ increases and as a result $f_{\rm d}$ decreases.



In order to understand how the 
initial parameters affect the combination of $B/D$ and $\Gamma$,
we create an additional set of models. Here we measure the properties in
the initial realizations without actually running the simulations. 
When we keep the disk mass fraction ($f_{\rm d}$) and the bulge scale-length
($r_{\rm b}$) fixed then $\Gamma$ increases as $B/D$ increases.
Even when we compare models that have similar $B/D$ and $f_{\rm d}$,
we find that models with a larger $r_{\rm b}$ have a larger $\Gamma$.
And when we increase $f_{\rm d}$, while keeping $B/D$ fixed, then $\Gamma$ increases.
We also tested the effect of the halo scale length parameter but found
that this did not affect $B/D$ or $\Gamma$ substantially.
These sequential changes in the relation between $B/D$ and $\Gamma$
are summarized in Fig.~\ref{fig:Gamma_BD} of Appendix C, and the
detailed parameters of the individual models are summarized in
Tables~\ref{tb:models_add} and \ref{tb:models_add2}. Thus, galaxies
with a massive bulge tend to form tightly-wound spirals, but the shear
rate ($\Gamma$) is more essential to the pitch angle than $B/D$.

In addition, we look at the relation between $B/D$ and the bar formation
epoch ($t_{\rm b}$). 
We present this using the red symbols for models with $B/D>0.5$ in Fig.~\ref{fig:bfe}.
Models with a large $B/D$ tend to take a shorter time before the bar formation, 
but compared to the dependence on $f_{\rm d}$, the effect of $B/D$ on $t_{\rm b}$ is 
unclear. 

Summarizing all our simulated results,
we conclude that the disk-to-total mass fraction ($f_{\rm d}$) and the shear rate 
($\Gamma$) are important parameters that decide the disk galaxy morphology,
such as the number of spiral arms, the pitch angle, and the formation of a bar.

\begin{figure*}
\includegraphics[width=55mm]{figures/rotation_curve_md1_mb10.pdf}\includegraphics[width=42mm]{figures/md1_mb10_400M_0128c.pdf}\includegraphics[width=42mm]{figures/md1_mb10_400M_1024c.pdf}\\
\caption{Rotation curve (left) and surface density at 1.25\,Gyr (middle) and 10\,Gyr for model md1mb10. Gray dashed curve in
the left panel indicates the rotation curve of the halo for model md1mb1. \label{fig:snapshots_mb10}}
\end{figure*}



\begin{figure}
\includegraphics[width=\columnwidth]{figures/shear_rate.pdf}
\caption{The relation between the shear rates at $2.2R_{\rm d}$ and the bulge-to-disk mass ratio of our models. 
Circles and squares indicate models with $f_{\rm d}(2.2R_{\rm d})<0.3$ and $f_{\rm d}(2.2R_{\rm d})>0.3$,
respectively. \label{fig:shear_rate}}
\end{figure}


