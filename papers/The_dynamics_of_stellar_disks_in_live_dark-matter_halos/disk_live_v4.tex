% mnras_template.tex
%
% LaTeX template for creating an MNRAS paper
%
% v3.0 released 14 May 2015
% (version numbers match those of mnras.cls)
%
% Copyright (C) Royal Astronomical Society 2015
% Authors:
% Keith T. Smith (Royal Astronomical Society)

% Change log
%
% v3.0 May 2015
%    Renamed to match the new package name
%    Version number matches mnras.cls
%    A few minor tweaks to wording
% v1.0 September 2013
%    Beta testing only - never publicly released
%    First version: a simple (ish) template for creating an MNRAS paper

%%%%%%%%%%%%%%%%%%%%%%%%%%%%%%%%%%%%%%%%%%%%%%%%%%
% Basic setup. Most papers should leave these options alone.
\documentclass[a4paper,fleqn,usenatbib]{mnras}

\usepackage{amsmath,amssymb}    % Advanced maths commands

% MNRAS is set in Times font. If you don't have this installed (most LaTeX
% installations will be fine) or prefer the old Computer Modern fonts, comment
% out the following line
%\usepackage{newtxtext,newtxmath}
% Depending on your LaTeX fonts installation, you might get better results with one of these:
\usepackage{mathptmx}
%\usepackage{txfonts}


% Jeroen: The below is a patch to fix the annoying pdf generation errors related to hyperref
% spreading links on two pages. I have no clue what it does but it works :)
% Source: https://github.com/ho-tex/hyperref/issues/19
\usepackage{etoolbox}
\makeatletter
\patchcmd\@combinedblfloats{\box\@outputbox}{\unvbox\@outputbox}{}{%
   \errmessage{\noexpand\@combinedblfloats could not be patched}%
}%
\makeatother



% Use vector fonts, so it zooms properly in on-screen viewing software
% Don't change these lines unless you know what you are doing
\usepackage[T1]{fontenc}
\usepackage{ae,aecompl}

%%%%% AUTHORS - PLACE YOUR OWN PACKAGES HERE %%%%%

% Only include extra packages if you really need them. Common packages are:
\usepackage{graphicx}   % Including figure files
\usepackage{amssymb}    % Extra maths symbols


\usepackage{natbib, aas_macros}
\usepackage{graphicx, color, url}
\usepackage{grffile} 
%\usepackage[a4paper]{geometry}


% If your system does not have the AMS fonts version 2.0 installed, then
% remove the useAMS option.
%
% useAMS allows you to obtain upright Greek characters.
% e.g. \umu, \upi etc.  See the section on "Upright Greek characters" in
% this guide for further information.
%
% If you are using AMS 2.0 fonts, bold math letters/symbols are available
% at a larger range of sizes for NFSS release 1 and 2 (using \boldmath or
% preferably \bmath).
%
% The usenatbib command allows the use of Patrick Daly's natbib.sty for
% cross-referencing.
%
% If you wish to typeset the paper in Times font (if you do not have the
% PostScript Type 1 Computer Modern fonts you will need to do this to get
% smoother fonts in a PDF file) then uncomment the next line
% \usepackage{Times}

%%%%% AUTHORS - PLACE YOUR OWN MACROS HERE %%%%%
\newcommand{\kms}{\mathrm{km\ s^{-1}}\,}

\def\JB#1{{\bf {\color{red}[#1 -- Jeroen]}}}
\def\MF#1{{\bf {\color{blue}[#1 -- Michiko]}}}
\def\simon#1{{\bf {\color{green}[#1 -- SPZ]}}}
\def\Simon#1{{\bf {\color{green}[#1 -- SPZ]}}}
\def\SPZ#1{{\bf {\color{green}[#1 -- SPZ]}}}
\def\JNB#1{{\bf {\color{orange}[#1 -- Junichi]}}}

%%%%%%%%%%%%%%%%%%%%%%%%%%%%%%%%%%%%%%%%%%%%%%%%

\title[The dynamics of stellar disks in live halos]{The dynamics of stellar disks in live dark-matter halos}
\author[M. S. Fujii et al.]{M. S. Fujii$^{1}$\thanks{E-mail:
fujii@astron.s.u-tokyo.ac.jp (MSF)}, J. B\'edorf$^{2}$, J. Baba$^{3}$, and S. Portegies Zwart$^{2}$\\
$^{1}$Department of Astronomy, Graduate School of Science, The University of Tokyo, 7-3-1 Hongo, Bunkyo-ku, Tokyo, 113-0033, Japan\\
$^{2}$Leiden Observatory, Leiden University, NL-2300RA Leiden, The Netherlands\\
$^{3}$National Astronomical Observatory of Japan, Mitaka-shi, Tokyo 181-8588, Japan}

\date{Accepted . Received ; in original form }

\begin{document}
\label{firstpage}
\pagerange{\pageref{firstpage}--\pageref{lastpage}} \pubyear{2002}
\maketitle



\begin{abstract}

Recent developments in computer hardware and software enable
researchers to simulate the self-gravitating evolution of galaxies at
a resolution comparable to the actual number of stars.  Here we
present the results of a series of such simulations.  We performed
$N$-body simulations of disk galaxies with between 100 and 500 million
particles over a wide range of initial conditions.  Our calculations
include a live bulge, disk, and dark matter halo, each of which is
represented by self-gravitating particles in the $N$-body code.  The
simulations are performed using the gravitational $N$-body tree-code
{\tt Bonsai} running on the Piz Daint supercomputer.
We find that the time scale over which the bar forms increases
exponentially with decreasing disk-mass fraction and that the bar formation
epoch exceeds a Hubble time when the disk-mass fraction is $\sim0.35$.
These results can be explained with the swing-amplification theory.
The condition
for the formation of $m=2$ spirals is consistent with that for the
formation of the bar, which is also an $m=2$ phenomenon.  We further argue
that the non-barred grand-design spiral galaxies
are transitional, and that they evolve to barred
galaxies on a dynamical timescale.
We also confirm that the disk-mass fraction and shear rate
are important parameters for the morphology of disk galaxies.
The former affects the number of spiral arms and the bar formation
epoch, and the latter determines the pitch angle of the spiral arms.


\end{abstract}

\begin{keywords}
galaxies: kinematics and dynamics --- galaxies: spiral --- galaxies: structure ---
galaxies: evolution --- methods: numerical
\end{keywords}


% \leavevmode
% \\
% \\
% \\
% \\
% \\
\section{Introduction}
\label{introduction}

AutoML is the process by which machine learning models are built automatically for a new dataset. Given a dataset, AutoML systems perform a search over valid data transformations and learners, along with hyper-parameter optimization for each learner~\cite{VolcanoML}. Choosing the transformations and learners over which to search is our focus.
A significant number of systems mine from prior runs of pipelines over a set of datasets to choose transformers and learners that are effective with different types of datasets (e.g. \cite{NEURIPS2018_b59a51a3}, \cite{10.14778/3415478.3415542}, \cite{autosklearn}). Thus, they build a database by actually running different pipelines with a diverse set of datasets to estimate the accuracy of potential pipelines. Hence, they can be used to effectively reduce the search space. A new dataset, based on a set of features (meta-features) is then matched to this database to find the most plausible candidates for both learner selection and hyper-parameter tuning. This process of choosing starting points in the search space is called meta-learning for the cold start problem.  

Other meta-learning approaches include mining existing data science code and their associated datasets to learn from human expertise. The AL~\cite{al} system mined existing Kaggle notebooks using dynamic analysis, i.e., actually running the scripts, and showed that such a system has promise.  However, this meta-learning approach does not scale because it is onerous to execute a large number of pipeline scripts on datasets, preprocessing datasets is never trivial, and older scripts cease to run at all as software evolves. It is not surprising that AL therefore performed dynamic analysis on just nine datasets.

Our system, {\sysname}, provides a scalable meta-learning approach to leverage human expertise, using static analysis to mine pipelines from large repositories of scripts. Static analysis has the advantage of scaling to thousands or millions of scripts \cite{graph4code} easily, but lacks the performance data gathered by dynamic analysis. The {\sysname} meta-learning approach guides the learning process by a scalable dataset similarity search, based on dataset embeddings, to find the most similar datasets and the semantics of ML pipelines applied on them.  Many existing systems, such as Auto-Sklearn \cite{autosklearn} and AL \cite{al}, compute a set of meta-features for each dataset. We developed a deep neural network model to generate embeddings at the granularity of a dataset, e.g., a table or CSV file, to capture similarity at the level of an entire dataset rather than relying on a set of meta-features.
 
Because we use static analysis to capture the semantics of the meta-learning process, we have no mechanism to choose the \textbf{best} pipeline from many seen pipelines, unlike the dynamic execution case where one can rely on runtime to choose the best performing pipeline.  Observing that pipelines are basically workflow graphs, we use graph generator neural models to succinctly capture the statically-observed pipelines for a single dataset. In {\sysname}, we formulate learner selection as a graph generation problem to predict optimized pipelines based on pipelines seen in actual notebooks.

%. This formulation enables {\sysname} for effective pruning of the AutoML search space to predict optimized pipelines based on pipelines seen in actual notebooks.}
%We note that increasingly, state-of-the-art performance in AutoML systems is being generated by more complex pipelines such as Directed Acyclic Graphs (DAGs) \cite{piper} rather than the linear pipelines used in earlier systems.  
 
{\sysname} does learner and transformation selection, and hence is a component of an AutoML systems. To evaluate this component, we integrated it into two existing AutoML systems, FLAML \cite{flaml} and Auto-Sklearn \cite{autosklearn}.  
% We evaluate each system with and without {\sysname}.  
We chose FLAML because it does not yet have any meta-learning component for the cold start problem and instead allows user selection of learners and transformers. The authors of FLAML explicitly pointed to the fact that FLAML might benefit from a meta-learning component and pointed to it as a possibility for future work. For FLAML, if mining historical pipelines provides an advantage, we should improve its performance. We also picked Auto-Sklearn as it does have a learner selection component based on meta-features, as described earlier~\cite{autosklearn2}. For Auto-Sklearn, we should at least match performance if our static mining of pipelines can match their extensive database. For context, we also compared {\sysname} with the recent VolcanoML~\cite{VolcanoML}, which provides an efficient decomposition and execution strategy for the AutoML search space. In contrast, {\sysname} prunes the search space using our meta-learning model to perform hyperparameter optimization only for the most promising candidates. 

The contributions of this paper are the following:
\begin{itemize}
    \item Section ~\ref{sec:mining} defines a scalable meta-learning approach based on representation learning of mined ML pipeline semantics and datasets for over 100 datasets and ~11K Python scripts.  
    \newline
    \item Sections~\ref{sec:kgpipGen} formulates AutoML pipeline generation as a graph generation problem. {\sysname} predicts efficiently an optimized ML pipeline for an unseen dataset based on our meta-learning model.  To the best of our knowledge, {\sysname} is the first approach to formulate  AutoML pipeline generation in such a way.
    \newline
    \item Section~\ref{sec:eval} presents a comprehensive evaluation using a large collection of 121 datasets from major AutoML benchmarks and Kaggle. Our experimental results show that {\sysname} outperforms all existing AutoML systems and achieves state-of-the-art results on the majority of these datasets. {\sysname} significantly improves the performance of both FLAML and Auto-Sklearn in classification and regression tasks. We also outperformed AL in 75 out of 77 datasets and VolcanoML in 75  out of 121 datasets, including 44 datasets used only by VolcanoML~\cite{VolcanoML}.  On average, {\sysname} achieves scores that are statistically better than the means of all other systems. 
\end{itemize}


%This approach does not need to apply cleaning or transformation methods to handle different variances among datasets. Moreover, we do not need to deal with complex analysis, such as dynamic code analysis. Thus, our approach proved to be scalable, as discussed in Sections~\ref{sec:mining}.

\section{$N$-body simulations}

We performed a series of $N$-body simulations of galactic stellar
disks embedded in dark matter halos. In this section, we describe our
choice of parameters and the $N$-body code used for these simulations.


\subsection{Model}
Our models are based on those described in \citet{2008ApJ...679.1239W} and 
\citet{2005ApJ...631..838W}. 
We generated the initial conditions using GalactICS \citep{2005ApJ...631..838W}.
The initial conditions for generating the dark mater halo are taken from the
NFW profile \citep{1997ApJ...490..493N}, which has a density profile following:
\begin{eqnarray}
\rho_{\rm NFW}(r) = \frac{\rho_{\rm h}}{(r/a_{\rm h})(1+r/a_{\rm h})^3},
\end{eqnarray}
and the potential is written as
\begin{eqnarray}
\Phi_{\rm NFW} = -\sigma_{\rm h}^2\frac{\log (1+r/a_{\rm h})}{r/a_{\rm h}}.
\end{eqnarray}
Here the gravitational constant, $G$, is unity, 
$a_{\rm h}$ is the scale radius, $\rho_{\rm h}\equiv\sigma^2/4\pi a_{\rm h}^2$
is the characteristic density, and $\sigma_{\rm h}$ is the characteristic 
velocity dispersion. We adopt $\sigma_{\rm h}=340$ (km\,s$^{-1}$), 
$a_{\rm h}=11.5$ (kpc).
Since the NFW profile is infinite in extent and mass, the 
distribution is truncated by a halo tidal radius using
an energy cutoff $E_{\rm h}\equiv\epsilon_{\rm h}\sigma_{\rm h}^2$,
where $\epsilon _{\rm h}$ is the truncation parameter with $0<\epsilon _{\rm h}<1$.
Setting $\epsilon _{\rm h}=0$ yields a full NFW profile
\citep[see][for details]{2005ApJ...631..838W}. 
We choose the parameters of the dark matter halo such that the resulting
rotation curves have a similar shape.
The choice of parameters is summarized in Table~\ref{tb:params}.

For some models we assume the halo to have net angular momentum. This
is realized by changing the sign of the angular momentum about the
symmetry axis ($J_{z}$). The rotation is parameterized using $\alpha
_{\rm h}$.  For $\alpha _{\rm h}=0.5$, the fraction of halo particles
which have positive or negative $J_{z}$ are the same, in which case the disk
has no net angular momentum. If $\alpha _{\rm h}>0.5$, the halo rotates in the same
direction as the disk.

For the disk component, we adopt an exponential disk for which the surface density
distribution is given by
\begin{eqnarray}
\Sigma (R) = \Sigma_{0} {\rm e}^{-R/R_{\rm d}}.
\end{eqnarray}
The vertical structure is given by ${\rm sech}^2(z/z_{\rm d})$,
where $z_{\rm d}$ is the disk scale height.
The radial velocity dispersion is assumed to follow 
$\sigma_{R}^2(R)=\sigma_{R0}^2\exp(-R/R_{\rm d})$, where 
$\sigma_{R0}$ is the radial velocity dispersion at the disk's center.
Toomre's stability parameter $Q$ 
\citep{1964ApJ...139.1217T,2008gady.book.....B} at
a reference radius (we adopt $2.2R_{\rm d}$), $Q_{0}$, is controlled
by the central velocity dispersion of the disk ($\sigma _{R0}$). 
We tune $\sigma_{R0}$ such that for our standard model (md1mb1) $Q_0=1.2$ .

For model md1mb1 we use as disk mass $M_{\rm d}$=$4.9\times 10^{10}$ $M_{\odot}$,
and the scale length $R_{\rm d}$=2.8 kpc. The disk's truncation radius is set to 30 kpc, the 
scale height $z_{\rm d}$=0.36 kpc and the radial velocity
dispersion at the center of the galaxy to $\sigma_{R0}$=$105$ km\,s$^{-1}$.
The disk is truncated at ($R_{\rm out}$) with a 
radial range for disk truncation ($\delta R$). We adopt 
$R_{\rm out}=30.0$ (kpc) and $\delta R=0.8$ (kpc).

For the bulge we use a Hernquist model~\citep{1990ApJ...356..359H},
but the distribution function is extended with an energy cutoff
parameter ($\epsilon_{\rm b}$) to truncate the profile much in the
same way as we did with the halo model.  The density distribution
and potential of the standard Hernquist model is
\begin{eqnarray}
\rho_{\rm H} = \frac{\rho_{\rm b}}{(r/a_{\rm b})(1+r/a_{\rm b})^3}
\end{eqnarray}
and
\begin{eqnarray}
\Phi_{\rm H} = \frac{\sigma_{\rm b}^2}{1+r/a_{\rm b}}.
\end{eqnarray}
Here $a_{\rm b}$, $\rho_{\rm b}=\sigma_{\rm b}^2/(2\pi a_{\rm b}^2)$, and
$\sigma _{\rm b}$ are the scale length, characteristic density, and
the characteristic velocity of the bulge, respectively.
We set $\sigma_{\rm b}$=300 km\,s$^{-1}$, bulge scale length $a_{\rm b}$=0.64 kpc,
and the truncation parameter ($\epsilon _{\rm b}$=0.0).
This results in a bulge mass of $4.6\times 10^9M_{\odot}$,
which is consistent with the Milky Way model proposed by
\citet{2010ApJ...720L..72S}, and reproduces
the bulge velocity distribution obtained by BRAVA observations~\citep{2012AJ....143...57K}.
We do not assume an initial rotational velocity for the bulge.

For the simulation models we vary the disk mass, bulge mass,
scale length, halo spin, and $Q_0$.  Since the adopted generator for
the galaxies is an irreversible process and due to the randomization
of the selection of particle positions and velocities we cannot
guarantee that the eventual velocity profile is identical to the input
profile, but we confirmed by inspection that they are
indistinguishable.  The initial conditions for each of the models are
summarized in Table \ref{tb:params}.  The mass and tidal radius for
the bulge, disk, and halo as created by the initial condition
generator are given in Table \ref{tb:mass_radius}.

In each of the models we fix  the number of particles used for 
the disk component to $8.3 \times 10^6$. For the bulge and halo
particles we adopt the same particle mass as for the disk particles.
As a consequence the mass ratios between the bulge, halo and disk are
set by having a different number of particles 
used per component (Table~\ref{tb:mass_radius}).


\begin{table*}
\begin{center}
%\rotate
  \caption{Model names and their parameters.  The columns represent,
    1: Model name, 2: Halo scale radius, 3: Halo characteristic
    velocity dispersion, 4: Halo truncation parameter, 5: Halo
    rotation parameter, 6: Disk mass, 7: Disk scale radius, 8: Disk
    scale height, 9: Disk radial velocity dispersion at the center of
    the disk, 10: Bulge scale length, 11: Bulge characteristic
    velocity, 12: Bulge truncation parameter.  In the model names,
    `md' and `mb' indicate disk and bulge masses. `Rd' and `rb'
    indicate the disk and halo scale radii. `s' indicates models with
    halo spin. `Q' indicates the initial $Q$ value. For these, the
    values of model md1mb1 are referred to as 1.  For all models,
    movies of the evolution are available in the online materials.
\label{tb:params}
}
\begin{tabular}{lccccccccccc}
%\tabletypesize{\scriptsize}
%\tablewidth{0pt}
%\startdata 
  \hline
  (1)      &   (2)      &   (3)      &   (4)      &   (5)      &   (6)      &   (7)      &   (8)      &   (9)      &   (10)      &   (11)      &   (12)      \\ 
           &  \multicolumn{4}{l}{Halo} &  \multicolumn{4}{l}{Disk} &  \multicolumn{3}{l}{Bulge} \\
Parameters &  $a_{\rm h}$ & $\sigma_{\rm h}$ & $1-\epsilon_{\rm h}$ & $\alpha_{\rm h}$& $M_{\rm d}$ & $R_{\rm d}$ & $z_{\rm d}$ & $\sigma_{R0}$  & $a_{\rm b}$ & $\sigma_{\rm b}$ & $1-\epsilon_{\rm b}$ \\ 
Model   &  (kpc) & ($\kms$) &  &  & $(10^{10}M_{\odot})$ & (kpc) & (kpc) & ($\kms$)  & (kpc) & $(\kms)$\\
\hline \hline
md1mb1        &11.5 &  340 & 0.8 & 0.5 & $4.9$ & 2.8 & 0.36 & 105  & 0.64 & 300 & 1.0 \\
md1mb1s0.65    &11.5 &  340 & 0.8 & 0.65 & $4.9$ & 2.8 & 0.36 & 105  & 0.64 & 300 & 1.0 \\
md1mb1s0.8    &11.5 &  340 & 0.8 & 0.8 & $4.9$ & 2.8 & 0.36 & 105  & 0.64 & 300 & 1.0 \\
\\
md0.5mb1  & 8.2 &  310 & 0.88 & 0.5 & $2.5$ & 2.8 & 0.36 & 59.2  & 0.64 & 300 & 0.86 \\
md0.4mb1  & 7.6 &  300 & 0.91 & 0.5 & $2.0$ & 2.8 & 0.36 & 49.0  & 0.64 & 300 & 0.84 \\
md0.3mb1  & 7.0 &  287 & 0.92 & 0.5 & $1.5$ & 2.8 & 0.36 & 38.5  & 0.64 & 300 & 0.82 \\
md0.1mb1  & 6.0 &  285 & 0.97 & 0.5 &$0.49$ & 2.8 & 0.36 & 13.5  & 0.64 & 300 & 0.79 \\
\\
md0.5mb0  & 22.0 &  450 & 0.7 & 0.5 & $2.3$ & 2.8 & 0.36 & 62.6  & 0.64 & 500 & 0.86 \\
md0.5mb3  & 7.0 &  270 & 0.8 & 0.5 & $2.3$ & 2.8 & 0.36 & 59.0  & 0.64 & 500 & 0.79 \\
md0.5mb4  & 6.6 &  260 & 0.82 & 0.5 & $2.3$ & 2.8 & 0.36 & 58.3  & 0.64 & 545 & 0.80 \\
md0.5mb4rb3  & 13.5 &  360 & 0.8 & 0.5 & $2.3$ & 2.8 & 0.36 & 57.2  & 1.92 & 380 & 0.99 \\
\\
md1mb1Rd1.5      &  9.0 &  290 & 0.95 & 0.5 & $4.9$ & 4.2 & 0.36 & 74.2  & 0.64 & 300 & 0.85 \\
md0.5mb1Rd1.5     &  7.5 &  290 & 0.91 & 0.5 & $2.5$ & 4.2 & 0.36 & 39.8  & 0.64 & 300 & 0.8 \\
md0.5Rmb1d1.5s &  7.5 &  290 & 0.91 & 0.8 & $2.5$ & 4.2 & 0.36 & 39.8  & 0.64 & 300 & 0.8 \\
\\
md1.5mb5      & 13.0 & 280 & 0.9 & 0.5 & 7.3 &  2.8 & 0.36 & 138 & 1.0 & 550 & 0.8 \\
md1mb10       & 18.0 & 500 & 0.9 & 0.5 & 4.9 & 2.8 & 0.36  & 93.2 & 1.5 & 600 & 1.0 \\
\\
md0.5mb0Q0.5  & 22.0 &  450 & 0.7 & 0.5 & $2.3$ & 2.8 & 0.36 & 26.1  & 0.64 & 500 & 0.86 \\
md0.5mb0Q2.0  & 22.0 &  450 & 0.7 & 0.5 & $2.3$ & 2.8 & 0.36 & 105  & 0.64 & 500 & 0.86 \\
%\enddata
\hline
\end{tabular}
\end{center}
\end{table*}


\begin{table*}
\begin{center}
  \caption{Models: mass, radius, and number of particles per component.
    Column 1: Model name, 2: Disk mass, 3: Bulge mass, 4: Halo mass, 5: Disk outer radius, 6: Bulge outer radius, 7: Halo outer radius, 
8: Toomre's $Q$ value at the reference point ($2.2R_{\rm d}$), 
9: Bulge-to-disk mass ratio ($B/D$), 10: Number of particles for the disk, 11:  Number of particles for the bulge, 12: Number of particles for the halo.
\label{tb:mass_radius}}
\begin{tabular}{lccccccccccc}
%\tabletypesize{\scriptsize}
%\rotate
%\tablewidth{0pt}
%\startdata 
\hline
  (1)      &   (2)      &   (3)      &   (4)      &   (5)      &   (6)      &   (7)      &   (8)      &   (9)      &   (10)      &   (11)      &   (12)      \\ 
Model    & $M_{\rm d}$ & $M_{\rm b}$ & $M_{\rm h}$ & $R_{\rm d, t}$ & $r_{\rm b, t}$ & $r_{\rm h, t}$ & $Q_0$   &  $M_{\rm b}/M_{\rm d}$ & $N_{\rm d}$ & $N_{\rm b}$ & $N_{\rm h}$\\ 
   & ($10^{10}M_{\odot}$) & ($10^{10}M_{\odot}$) & ($10^{10}M_{\odot}$) & (kpc) & (kpc) & (kpc) &  &   &  &  & \\ 
\hline  \hline
md1mb1       & 4.97 &  0.462 & 59.7 & 31.6 & 3.17 & 229 & 1.2  & 0.0930 & 8.3M & 0.77M & 100M \\%8330408 (8M) & 773506 & 100000000\\
md1mb1s0.65       & 4.97 &  0.462 & 59.7 & 31.6 & 3.17 & 229 & 1.2   & 0.0930 & 8.3 M & 0.77M & 100M \\%8330408 (8M) & 773506 & 100000000 \\
md1mb1s0.8       & 4.97 &  0.462 & 59.7 & 31.6 & 3.17 & 229 & 1.2  & 0.0930 & 8.3M & 0.77M & 100M \\%8330408 (8M) & 773506 & 100000000 \\
\\
md0.5mb1      & 2.55 &  0.465 & 43.8 & 31.6 & 2.56 & 232 & 1.2  & 0.182  & 8.3M & 1.5M & 140M \\%8341073 (8M) & 1523280 &  143544884 \\
md0.4mb1      & 2.05 &  0.463 & 41.4 & 31.6 & 2.52 & 261 & 1.2  & 0.226 & 8.3M & 1.9M & 170M \\%8341073 (8M) & 1882153 &  168251277 \\
md0.3mb1      & 1.56 &  0.462 & 36.2 & 31.6 & 2.49 & 247 & 1.2 & 0.296 & 8.3M & 2.5M & 190M \\%8341073 (8M) & 2475127 & 194235851 \\ 
md0.1mb1      & 0.546 & 0.466 & 33.3 & 31.6 & 2.44 & 340 & 1.2  & 0.853 & 8.3M & 7.1M & 510M \\%8330000 (8M) & 7100158 & 508319027 \\
\\
md0.5mb0      & 2.53 &  0.0 & 100.0 & 31.6 & - & 295 & 1.2  & 0.00 & 8.3M & - & 330M \\%8341073 (8M) & - & 330789871 \\
md0.5mb3      & 2.61 &  1.37 & 39.7 & 31.6 & 2.81 & 120 & 1.2 & 0.525 & 8.3M & 4.4M & 130M \\%8341073 (8M) & 4444705 & 128685231 \\
md0.5mb4      & 2.62 &  1.69 & 41.4 & 31.6 & 2.96 & 125 & 1.2 & 0.645 & 8.3M & 5.4M & 130M \\%8341073 (8M) & 5447485 &  133496728 \\
md0.5mb4rb3   & 2.60 &  1.76 & 86.7 & 31.6 & 8.55 & 229 & 1.2 & 0.676 & 8.3M & 5.4M & 130M \\%8341073 (8M) & 5447485 &  133496728 \\
\\ 
md1mb1Rd1.5      & 5.06 &  0.464 & 47.1 & 46.6 & 2.61 & 620 & 1.2 &  0.0916 & 8.3M & 0.77M & 78M \\%8329926 (8M) & 7664161 & 77637255 \\ 
md0.5mb1Rd1.5    & 2.59 &  0.457 & 35.2 & 46.6 & 2.47 & 249 & 1.2 & 0.176 & 8.3M & 1.5M & 110M \\%8341073 (8M) & 1472714 & 113312472 \\ 
md0.5mb1Rd1.5s & 2.59 &  0.457 & 35.2 & 46.6 & 2.47 & 249 & 1.2 & 0.176 & 8.3M & 1.5M & 110M \\%8341073 (8M) & 1472714 & 113312472 \\ 
\\
md1.5mb5      & 7.52  &  2.09  & 104.6 & 31.6 & 3.53 & 269 & 1.2 & 0.279 & 8.3M & 2.3M & 120M \\%8330408 (8M) & 2325101 & 116155914 \\
md1mb10       & 5.17  & 5.22  & 2050  & 31.6 & 11.6 & 264 & 1.2 &  1.0 & 8.3M & 8.4M & 400M \\%8330408 (8M) & 8419789 & 399095779 \\
\\
md0.5mb1Q0.5   & 2.55 &  0.465 & 43.8 & 31.6 & 2.56 & 232 & 0.5 & 0.182 & 8.3M & 1.5M & 140M \\%8341073 (8M) & 1523280 &  143544884 \\
md0.5mb1Q2.0   & 2.55 &  0.465 & 43.8 & 31.6 & 2.56 & 232 & 2.0  & 0.182 & 8.3M & 1.5M & 140M \\%8341073 (8M) & 1523280 &  143544884 \\
%\enddata
\\
\hline
\end{tabular}
\end{center}
\end{table*}





\subsection{The {\tt Bonsai} optimized gravitational $N$-body tree-code}

We adopted the {\tt Bonsai} code for all calculations
\citep{2012JCoPh.231.2825B, 2014hpcn.conf...54B}.  {\tt Bonsai}
implements the classical Barnes \& Hut algorithm
\citep{1986Natur.324..446B} but then optimized for Graphics
Processing Units (GPU) and massively parallel operations. In {\tt Bonsai}
all the compute work, including the tree-construction, takes place on
the GPU which frees up the CPU for administrative tasks. By moving all
the compute work to the GPU there is no need for expensive data copies,
and we take full advantage of the large number of compute cores and
high memory bandwidth that is available on the GPU.  The use of GPUs
allows fast simulations, but we are limited by the relatively small
amount of memory on the GPU. To overcome this limitation we
implemented across-GPU and across-node parallelizations which enable
us to use multiple GPUs in parallel for a single simulation
\citep{2014hpcn.conf...54B}.  Combined with the GPU acceleration, this
parallelization method allows {\tt Bonsai} to scale efficiently from
single GPU systems all the way to large GPU clusters and
supercomputers~\citep{2014hpcn.conf...54B}.  We used the version of
{\tt Bonsai} that incorporates quadrupole expansion of the multipole
moments and the improved Barnes \& Hut opening angle criteria
\citep{2013MNRAS.436.1161I}.  We use a shared time-step of $\sim 0.6$
Myr, a gravitational softening length of 10\,pc and the opening angle
$\theta=0.4$.

Our simulations contain hundreds of millions of particles and
therefore it is critical that the post-processing is handled
efficiently.  We therefore implemented the post-processing methods directly in
{\tt Bonsai} and these are executed while the simulation is progressing. 
This eliminates the
need to reload snapshot data (which can be on the order of a few
terabytes) after the simulation.

The simulations in this work have been run on the Piz Daint supercomputer at
the Swiss National Supercomputing Centre. In this machine each compute node
contains an NVIDIA Tesla K20x GPU and an Intel Xeon E5-2670 CPU. Depending on the number
of particles in the simulation we used between 8 and 512 nodes per simulation.

%!TEX ROOT = ../../centralized_vs_distributed.tex

\section{{\titlecap{the centralized-distributed trade-off}}}\label{sec:numerical-results}

\revision{In the previous sections we formulated the optimal control problem for a given controller architecture
(\ie the number of links) parametrized by $ n $
and showed how to compute minimum-variance objective function and the corresponding constraints.
In this section, we present our main result:
%\red{for a ring topology with multiple options for the parameter $ n $},
we solve the optimal control problem for each $ n $ and compare the best achievable closed-loop performance with different control architectures.\footnote{
\revision{Recall that small (large) values of $ n $ mean sparse (dense) architectures.}}
For delays that increase linearly with $n$,
\ie $ f(n) \propto n $, 
we demonstrate that distributed controllers with} {few communication links outperform controllers with larger number of communication links.}

\textcolor{subsectioncolor}{Figure~\ref{fig:cont-time-single-int-opt-var}} shows the steady-state variances
obtained with single-integrator dynamics~\eqref{eq:cont-time-single-int-variance-minimization}
%where we compare the standard multi-parameter design 
%with a simplified version \tcb{that utilizes spatially-constant feedback gains
and the quadratic approximation~\eqref{eq:quadratic-approximation} for \revision{ring topology}
with $ N = 50 $ nodes. % and $ n\in\{1,\dots,10\} $.
%with $ N = 50 $, $ f(n) = n $ and $ \tau_{\textit{min}} = 0.1 $.
%\autoref{fig:cont-time-single-int-err} shows the relative error, defined as
%\begin{equation}\label{eq:relative-error}
%	e \doteq \dfrac{\optvarx-\optvar}{\optvar}
%\end{equation}
%where $ \optvar $ and $ \optvarx $ denote the the optimal and sub-optimal scalar variances, respectively.
%The performance gap is small
%and becomes negligible for large $ n $.
{The best performance is achieved for a sparse architecture with  $ n = 2 $ 
in which each agent communicates with the two closest pairs of neighboring nodes. 
This should be compared and contrasted to nearest-neighbor and all-to-all 
communication topologies which induce higher closed-loop variances. 
Thus, 
the advantage of introducing additional communication links diminishes 
beyond}
{a certain threshold because of communication delays.}

%For a linear increase in the delay,
\textcolor{subsectioncolor}{Figure~\ref{fig:cont-time-double-int-opt-var}} shows that the use of approximation~\eqref{eq:cont-time-double-int-min-var-simplified} with $ \tilde{\gvel}^* = 70 $
identifies nearest-neighbor information exchange as the {near-optimal} architecture for a double-integrator model
with ring topology. 
This can be explained by noting that the variance of the process noise $ n(t) $
in the reduced model~\eqref{eq:x-dynamics-1st-order-approximation}
is proportional to $ \nicefrac{1}{\gvel} $ and thereby to $ \taun $,
according to~\eqref{eq:substitutions-4-normalization},
making the variance scale with the delay.

%\mjmargin{i feel that we need to comment about different results that we obtained for CT and DT double-intergrator dynamics (monotonic deterioration of performance for the former and oscillations for the latter)}
\revision{\textcolor{subsectioncolor}{Figures~\ref{fig:disc-time-single-int-opt-var}--\ref{fig:disc-time-double-int-opt-var}}
show the results obtained by solving the optimal control problem for discrete-time dynamics.
%which exhibit similar trade-offs.
The oscillations about the minimum in~\autoref{fig:disc-time-double-int-opt-var}
are compatible with the investigated \tradeoff~\eqref{eq:trade-off}:
in general, 
the sum of two monotone functions does not have a unique local minimum.
Details about discrete-time systems are deferred to~\autoref{sec:disc-time}.
Interestingly,
double integrators with continuous- (\autoref{fig:cont-time-double-int-opt-var}) ad discrete-time (\autoref{fig:disc-time-double-int-opt-var}) dynamics
exhibits very different trade-off curves,
whereby performance monotonically deteriorates for the former and oscillates for the latter.
While a clear interpretation is difficult because there is no explicit expression of the variance as a function of $ n $,
one possible explanation might be the first-order approximation used to compute gains in the continuous-time case.
%which reinforce our thesis exposed in~\autoref{sec:contribution}.

%\begin{figure}
%	\centering
%	\includegraphics[width=.6\linewidth]{cont-time-double-int-opt-var-n}
%	\caption{Steady-state scalar variance for continuous-time double integrators with $ \taun = 0.1n $.
%		Here, the \tradeoff is optimized by nearest-neighbor interaction.
%	}
%	\label{fig:cont-time-double-int-opt-var-lin}
%\end{figure}
}

\begin{figure}
	\centering
	\begin{minipage}[l]{.5\linewidth}
		\centering
		\includegraphics[width=\linewidth]{random-graph}
	\end{minipage}%
	\begin{minipage}[r]{.5\linewidth}
		\centering
		\includegraphics[width=\linewidth]{disc-time-single-int-random-graph-opt-var}
	\end{minipage}
	\caption{Network topology and its optimal {closed-loop} variance.}
	\label{fig:general-graph}
\end{figure}

Finally,
\autoref{fig:general-graph} shows the optimization results for a random graph topology with discrete-time single integrator agents. % with a linear increase in the delay, $ \taun = n $.
Here, $ n $ denotes the number of communication hops in the ``original" network, shown in~\autoref{fig:general-graph}:
as $ n $ increases, each agent can first communicate with its nearest neighbors,
then with its neighbors' neighbors, and so on. For a control architecture that utilizes different feedback gains for each communication link
	(\ie we only require $ K = K^\top $) we demonstrate that, in this case, two communication hops provide optimal closed-loop performance. % of the system.}

Additional computational experiments performed with different rates $ f(\cdot) $ show that the optimal number of links increases for slower rates: 
for example, 
the optimal number of links is larger for $ f(n) = \sqrt{n} $ than for $ f(n) = n $. 
\revision{These results are not reported because of space limitations.}
\mySection{Related Works and Discussion}{}
\label{chap3:sec:discussion}

In this section we briefly discuss the similarities and differences of the model presented in this chapter, comparing it with some related work presented earlier (Chapter \ref{chap1:artifact-centric-bpm}). We will mention a few related studies and discuss directly; a more formal comparative study using qualitative and quantitative metrics should be the subject of future work.

Hull et al. \citeyearpar{hull2009facilitating} provide an interoperation framework in which, data are hosted on central infrastructures named \textit{artifact-centric hubs}. As in the work presented in this chapter, they propose mechanisms (including user views) for controlling access to these data. Compared to choreography-like approach as the one presented in this chapter, their settings has the advantage of providing a conceptual rendezvous point to exchange status information. The same purpose can be replicated in this chapter's approach by introducing a new type of agent called "\textit{monitor}", which will serve as a rendezvous point; the behaviour of the agents will therefore have to be slightly adapted to take into account the monitor and to preserve as much as possible the autonomy of agents.

Lohmann and Wolf \citeyearpar{lohmann2010artifact} abandon the concept of having a single artifact hub \cite{hull2009facilitating} and they introduce the idea of having several agents which operate on artifacts. Some of those artifacts are mobile; thus, the authors provide a systematic approach for modelling artifact location and its impact on the accessibility of actions using a Petri net. Even though we also manipulate mobile artifacts, we do not model artifact location; rather, our agents are equipped with capabilities that allow them to manipulate the artifacts appropriately (taking into account their location). Moreover, our approach considers that artifacts can not be remotely accessed, this increases the autonomy of agents.

The process design approach presented in this chapter, has some conceptual similarities with the concept of \textit{proclets} proposed by Wil M. P. van der Aalst et al. \citeyearpar{van2001proclets, van2009workflow}: they both split the process when designing it. In the model presented in this chapter, the process is split into execution scenarios and its specification consists in the diagramming of each of them. Proclets \cite{van2001proclets, van2009workflow} uses the concept of \textit{proclet-class} to model different levels of granularity and cardinality of processes. Additionally, proclets act like agents and are autonomous enough to decide how to interact with each other.

The model presented in this chapter uses an attributed grammar as its mathematical foundation. This is also the case of the AWGAG model by Badouel et al. \citeyearpar{badouel14, badouel2015active}. However, their model puts stress on modelling process data and users as first class citizens and it is designed for Adaptive Case Management.

To summarise, the proposed approach in this chapter allows the modelling and decentralized execution of administrative processes using autonomous agents. In it, process management is very simply done in two steps. The designer only needs to focus on modelling the artifacts in the form of task trees and the rest is easily deduced. Moreover, we propose a simple but powerful mechanism for securing data based on the notion of accreditation; this mechanism is perfectly composed with that of artifacts. The main strengths of our model are therefore : 
\begin{itemize}
	\item The simplicity of its syntax (process specification language), which moreover (well helped by the accreditation model), is suitable for administrative processes;
	\item The simplicity of its execution model; the latter is very close to the blockchain's execution model \cite{hull2017blockchain, mendling2018blockchains}. On condition of a formal study, the latter could possess the same qualities (fault tolerance, distributivity, security, peer autonomy, etc.) that emanate from the blockchain;
	\item Its formal character, which makes it verifiable using appropriate mathematical tools;
	\item The conformity of its execution model with the agent paradigm and service technology.
\end{itemize}
In view of all these benefits, we can say that the objectives set for this thesis have indeed been achieved. However, the proposed model is perfectible. For example, it can be modified to permit agents to respond incrementally to incoming requests as soon as any prefix of the extension of a bud is produced. This makes it possible to avoid the situation observed on figure \ref{chap3:fig:execution-figure-4} where the associated editor is informed of the evolution of the subtree resulting from $C$ only when this one is closed. All the criticisms we can make of the proposed model in particular, and of this thesis in general, have been introduced in the general conclusion (page \pageref{chap5:general-conclusion}) of this manuscript.






\section{Summary}

We performed a series of galactic disk $N$-body simulations
to investigate the formation and dynamical evolution of spiral arm 
and bar structures in stellar disks which are embedded in live 
dark matter halos.
We adopted a range of initial conditions where the models have similar halo 
rotation curves, but different masses for the disk and bulge components, 
scale lengths, initial $Q$ values, and halo spin parameters.
The results indicate that the bar formation epoch increases exponentially 
as a function of the disk mass fraction with respect to the total mass at the 
reference radius (2.2 times the disk scale length), $f_{\rm d}$.
This relation is a consequence of swing amplification~\citep{1981seng.proc..111T},
which describes the amplification rate of the spiral arm when it transitions from 
leading arm to trailing arm because of the disk's differential rotation.
Swing amplification depends on the properties that characterize the disk, 
Toomre's $Q$, $X$, and $\Gamma$. The growth rate reaches its maximum
for $1<X<2$,  although the position of the peak slightly depends on $Q$ as well as on
$\Gamma$. We computed $X$ for 
$m=2$ ($X_2$), which corresponds to a bar or two-armed spiral, 
for each of our models and found that this value is related to the bar's
formation epoch.

The bar amplitude grows most efficiently when $1<X_2<2$. For models 
with $1<X_2<2$ the bar develops immediately after the start 
of the simulation. As $X_2$ increases beyond $X_2=2$, the growth rate
decreases exponentially. We find that the bar formation epoch increases
exponentially as $X_2$ increases beyond $X_2=2$, in other words
$f_{\rm d}$ decreases. The bar formation epoch exceeds a Hubble time
for $f_{\rm d}\lesssim 0.35$.

Apart from $X$, the growth rate is also influenced by $Q$ where
a larger $Q$ results in a slower growth. This indicates that the bar formation
occurs later for larger values of $Q$. 
Our simulations confirmed this and showed that for the bar ($m=2$) the growth rate
is predicted by swing amplification and becomes visible when it grows beyond a certain amplitude.

Toomre's swing amplification theory further predicts that
the number of spiral arms is related to the mass of the disk, with
massive disks having fewer spiral arms. In addition, larger $\Gamma$
predicts a smaller number of spiral arms.
We confirmed these relations in our simulations. 
The shear rate ($\Gamma$) also affects the pitch angle of spiral
  arms. We further confirmed that our result is consistent with previous
studies.

We found that the disk-to-total mass fraction ($f_{\rm d}$)
and the shear rate ($\Gamma$) are the most important parameters that determine the
morphology of disk galaxies. 
When juxtaposing our models with the Hubble sequence,
the fundamental subdivisions of (barred-)spiral galaxies with 
massive bulges and tightly wound-up spiral arms from S(B)a to S(B)c is 
also be observed as a sequence in our simulations. Where the models 
with either massive bulges or massive disks have more tightly
wound spiral arms. This is because having both a massive disk and bulge results in 
a larger $\Gamma$, i.e., more tightly wound spiral arms. 


Once the
bar is formed it starts to heat the outer parts of the disk.
From this point onwards, 
the self-gravitating spiral arms disappear.
This may be in part caused by the 
lack of gas in our simulations. 
After the bar grows, we no longer discern  
spiral arms in the outer regions of the disk. This could imply
that gas cooling and star formation are required in order to 
maintain spiral structures in barred spiral galaxies for over 
a Hubble time~\citep{1981ApJ...247...77S,1984ApJ...282...61S}.


Our simulations further indicate that non-barred grand-design spirals are
transient structures which immediately evolve into barred
galaxies. Swing amplification teaches us that a massive disk is
required to form two-armed spiral galaxies. This condition, at the
same time, satisfies the short formation time of the bar structure.
Non-barred grand-design spiral galaxies therefore must evolve into barred
galaxies.  We consider that isolated non-barred grand-design spiral galaxies 
are in the process of developing a bar.






%\bsp
\section*{Acknowledgments}

We thank the anonymous referee for the very helpful comments.
This work was supported by JSPS KAKENHI Grant Number 26800108, 
HPCI Strategic Program Field 5 'The origin of matter and the universe,' 
and the Netherlands Research School for Astronomy (NOVA).
Simulations are performed using GPU clusters, HA-PACS at the
University of Tsukuba, Piz Daint at CSCS and Little Green Machine II
(621.016.701). Initial development has been done using the Titan
computer Oak Ridge National Laboratory.  This work was supported by a
grant from the Swiss National Supercomputing Centre (CSCS) under
project ID s548 and s716.  This research used resources of the Oak Ridge
Leadership Computing Facility at the Oak Ridge National Laboratory,
which is supported by the Office of Science of the U.S. Department of
Energy under Contract No. DE-AC05-00OR22725 and by the European
Union's Horizon 2020 research and innovation programme under grant
agreement No 671564 (COMPAT project).
\appendix


\section[]{The effects of other parameters}

We discussed the effect of the bulge and disk masses on the
development of bars and spiral arms in the main text. Here we briefly
summarize the effects of the other parameters we investigated.


\subsection{The halo spin}

The spin of the halo is known to be an important parameter that 
affects the bar's secular evolution. 
\citet{2014ApJ...783L..18L} showed that a co-rotating disk and halo 
speed up the bar formation, but decrease its final length. This 
is due to the angular momentum transfer between the disk and halo.
If the halo does not spin it absorbs the bar's angular momentum, 
which slows down the bar and increases its length. 
A co-rotating halo, however, returns angular momentum to the disk instead of 
just absorbing it. 
This stabilizes the angular momentum transfer, and the bar evolution ceases.

We setup a few models, based on model md1mb1, but now with a rotating halo. 
In order to give spin to the halo we change the sign of the angular momentum $z$ component, $L_{\rm z}$.
For models md1mb1s0.65 and md1mb1s0.8, 65 and 85\,\% of the halo particles are rotating in the same 
direction as the disk. For models without rotation, this value is 50\,\%. 

To compare our results with previous studies, we measure the spin 
parameter~\citet{1969ApJ...155..393P,1971A&A....11..377P}:
\begin{eqnarray}
\lambda = \frac{J|E|^{1/2}}{GM_{\rm h}^{5/2}},
\end{eqnarray}
where $J$ is the magnitude of the angular momentum vector, and $E$ is the total 
energy.
In our models, $\alpha_{\rm h}=0.65$ (0.8) correspond 
to $\lambda\sim0.03$ (0.06).


In Fig.~\ref{fig:snapshots_spin_b} we present the effect that the halo spin
has on models md1mb1s0.65 and md1mb1s0.8. 
The results indicate that  
the bar is shorter for the models with a stronger halo spin.

In Fig.~\ref{fig:A2_max_spin} we show the length and maximum amplitude of
the resulting bars.
These results are consistent with previous results which show that
the length of the bar and its amplitude decay when the halo spin increases.
However, in contrast to~\citet{2013MNRAS.434.1287S} and \citet{2014ApJ...783L..18L} ,
we find that the epoch of bar formation in our models is similar, 
whereas a faster formation was expected based on the larger halo spin. 
In order to rule out the effect of run-to-run variations~\citep{2009MNRAS.398.1279S},
we performed four additional simulations for each of models md1mb1, md1mb1s0.65 and md1mb1s0.8.
For the bar formation epochs we calculated the average and standard deviation. 

The average bar formation-epoch is $0.674 \pm 0.053$,
$0.691\pm 0.083$, and $0.610\pm 0.069$\,Gyr for models md1mb1, md1mb1s0.65 and md1mb1s0.8, respectively.
This may be caused by the relatively early bar formation (within $\sim0.8$\,Gyr)
compared to the previous
studies; 1--2\,Gyr for \citet{2014ApJ...783L..18L} and 3--4\,Gyr for
\citet{2013MNRAS.434.1287S}.
Indeed, in~\citet{2014ApJ...783L..18L} the bar formation epoch starts slightly earlier when 
a moderate spin parameter ($\lambda=0.045$ and 0.06) is introduced. The
dependence of the bar formation-epoch on the halo spin is even clearer in
\citet{2013MNRAS.434.1287S}, where the formation time is longer
than in~\citet{2014ApJ...783L..18L}.
We therefore argue that the rapid bar formation in our models may hide the sequential delay
of the bar formation as caused by the halo spin.

%
\begin{figure}
\begin{center}
  \includegraphics[width=40mm]{figures/md1_mb1_a0.65_100M_1024c.pdf}
  \includegraphics[width=40mm]{figures/md1_mb1_a0.8_100M_1024c.pdf}\\
    \caption{Snapshots for models md1mb1s0.65 (left) and md1mb1s0.8 (right), which are the same as model md1mb1 (Fig.~\ref{fig:snapshots_10Gyr},far most right panel) but now including halo spin.}\label{fig:snapshots_spin_b}
\end{center}
\end{figure}



\begin{figure*}
\includegraphics[width=\columnwidth]{figures/mode_A2max_spin.pdf}\includegraphics[width=\columnwidth]{figures/mode_Dbar_mb1s.pdf}
\caption{Time evolution of the maximum amplitude for $m=2$ (left) and the bar length (averaged for every $\sim 0.1$\,Gyr) for models md1mb1, md1mb1s0.65, and md1mb1s0.8. For each model, we performed four simulations changing the random seed (varying positions and velocities of the particles) when generating the initial realizations.
\label{fig:A2_max_spin}}
\end{figure*}

In addition to the bar forming models above, we also added halo spin to a model that 
shows no bar formation within 10\, Gyr. This model, md0.5Rd1.5s, is based on md0.5Rd1.5
but now with a halo spin of 0.8. 
In Fig.~\ref{fig:snapshots_spin_sp}, we present the snapshots of the above models at $t=10$\,Gyr. 
In contrast to the barred galaxies, their spiral structures look quite similar. 
To quantitatively compare the spiral amplitudes we use the total amplitude of the spiral arms 
given by $\sum ^{10}_{m=1} |A_m|^2$, where $A_m$ is the Fourier amplitude (Eq.~\ref{eq:Fourier}).
Instead of the bar amplitude, we measured the spirals total amplitude at 
$2.2R_{\rm d}$ and at $4.5R_{\rm d}$ (for this model 9.5 and 19.5\,kpc, respectively), 
the results are shown in Fig.~\ref{fig:mode_spin_sp}. 
The evolution of the spiral amplitudes are quite similar for both models, 
just like the pitch angle  $24^{\circ}$--$29^{\circ}$ (with)
$24^{\circ}$--$26^{\circ}$ (without halo spin) and the number of 
spiral arms $m=7$--8 for $R=10$--14\,kpc (see Table~\ref{tb:pitch_angle}).


In addition, in Fig.~\ref{fig:AM} we investigate the angular-momentum flow for 
the disk and halo as a function of time and cylindrical radius.
Following \citet{2014ApJ...783L..18L} and \citet{2009ApJ...707..218V},
we measure the change in angular momentum of the $z$-component at
every $\sim 10$\,Myr.
For the halos (top panels) there is no continuous angular
momentum transfer from the disk to the halo, but we only discern random variations
in the angular momentum. These fluctuations look stronger at outer
radii, but this is because the angular momentum changes are normalized by
the disk' angular momentum, which is smaller in the outer regions.

The angular momentum of the disks vary with time (see the red and blue
stripes in the bottom panels), but overall the disk loses only 1.9\,\%
of its initial angular momentum for models with spin and 1.7\,\% for
models without.
The amplitude of the stripes for the disks roughly corresponds to the
amplitude of the spiral pattern. In Fig.~\ref{fig:amplitude_ev}, we show the
total power as a function of cylindrical radius and time for models
md0.5Rd1.5 (left) and md0.5Rd1.5s (right).
From this we conclude that for spiral arms the angular momentum transfer between the disk and 
the halo is not efficient.
On the other hand, for barred galaxies the angular momentum flow 
from the disk to the halo is considerably smaller for models with 
a larger halo spin \citep[see Fig. 3 in][]{2014ApJ...783L..18L}.


\begin{figure}
\includegraphics[width=40mm]{figures/md0.5_Rd1.5_110M_1024c.pdf}\includegraphics[width=40mm]{figures/md0.5_Rd1.5_a0.8_110M_1024c.pdf}\\
\caption{Snapshots for models md0.5Rd1.5 (left) and md0.5Rd1.5s (right). \label{fig:snapshots_spin_sp}}
\end{figure}

\begin{figure*}
  \includegraphics[width=\columnwidth]{figures/mode_rot_spiral_R10.pdf}
  \includegraphics[width=\columnwidth]{figures/mode_rot_spiral_R19.pdf}\\
\caption{Total power for models md0.5Rd1.5 and md0.5Rd1.5s at $R=9.5$ kpc (left) and 19.5 kpc (right). \label{fig:mode_spin_sp}}
\end{figure*}

\begin{figure*}
  \includegraphics[width=\columnwidth]{figures/AM_evolution_halo_md0.5mb1Rd1.5.pdf}\includegraphics[width=\columnwidth]{figures/AM_evolution_halo_md0.5mb1Rd1.5s.pdf}\\
  \includegraphics[width=\columnwidth]{figures/AM_evolution_disk_md0.5mb1Rd1.5.pdf}\includegraphics[width=\columnwidth]{figures/AM_evolution_disk_md0.5mb1Rd1.5s.pdf}\\
\caption{Angular momentum flow of the halo (top) and the disk (bottom) as a function of cylindrical radius and time for models md0.5mb1Rd1.5 (left) and md0.5mb1Rd1.5s (right). The angular momentum flow is calculated from the angular momentum's change in the $z$-component for every $\sim 10$\,Myr. The value (color) is scaled to the initial angular momentum of the disk at each radius for both the disks and halos.}
\label{fig:AM}
\end{figure*}


\begin{figure*}
  \includegraphics[width=\columnwidth]{figures/amplitude_evolution_md0.5mb1Rd1.5.pdf}
  \includegraphics[width=\columnwidth]{figures/amplitude_evolution_md0.5mb1Rd1.5s.pdf}\\
\caption{Total power as a function of cylindrical radius and time for models md0.5Rd1.5 (left) and md0.5Rd1.5s (right).}\label{fig:amplitude_ev}
\end{figure*}



\subsection{Initial Q value}

To verify the expectation that the initial value of Toomre's $Q$ parameter 
($Q_0$) influences the bar and spiral structure, we created a set of models in 
which we varied this parameter. 

The models are based on md0.5mb0, with one having an initially unstable disk
(md0.5mb0Q0.5) and 
the other having a large $Q_0$, in which no spiral arms develop (md0.5mb0Q2.0).
The time evolution of the bar's amplitude and length is presented in
Fig.~\ref{fig:A2_max_Q} 
and the surface densities are shown in Fig.~\ref{fig:snapshots_Q}.
For md0.5mb0Q2.0 there is no sign of spiral or bar structure until $\sim 5$\,Gyr, but 
a bar develops shortly after that (left panel of Fig.~\ref{fig:A2_max_Q}).
This matches with the 
expectation that $Q_0$ influences the bar formation epoch,
the smaller the $Q_0$ value the faster the bar forms.
The peak amplitude just after the bar formation is higher for
the larger $Q_0$, but the final amplitude is similar
(see the left panel of Fig.~\ref{fig:A2_max_Q}).
We also confirmed that the final bar length does not depend on $Q_0$
(see the right panel of Fig.~\ref{fig:A2_max_Q}). 
However, the radius that gives the maximum amplitude is different for 
the models with a large or a small value of $Q$.
The radius for $A_{\rm 2, max}$ is 2.6 and 4.9\,kpc for models with $Q_0=0.5$
and $2.0$, respectively. This result is qualitatively consistent with 
\citet{2012PASJ...64....5H} where an initially colder disk forms
a weaker and more compact bar due to the smaller velocity dispersion of the disk
(although they stopped their simulation just after the first amplitude peak).

This further proves (as discussed in Section~3.3) that the growth 
rate of swing amplification governs the bar formation timescale.
The growth rate decreases 
as $Q$ increases \citep{1981seng.proc..111T} which is confirmed by our simulations. 
With $Q_0=2.0$, the disk is initially stable and hence the spiral structure has 
to be induced by the bar. These ring-like spiral arms are sometimes seen in SB0--SBa 
galaxies such as NGC\,5101 \citep{2011ApJS..197...21H}.


\begin{figure*}
\includegraphics[width=\columnwidth]{figures/mode_A2max_Q.pdf}\includegraphics[width=\columnwidth]{figures/mode_Dbar_Q.pdf}
\caption{Same as Fig.~\ref{fig:A2_max_mdisk}, but for models md0.5mb0Q0.5 and md0.5mb0Q2.0.
\label{fig:A2_max_Q}}
\end{figure*}


\begin{figure}
\begin{center}
\includegraphics[width=40mm]{figures/md0.5_mb0_Q0.5_1536c.pdf}\includegraphics[width=40mm]{figures/md0.5_mb0_Q2.0_1536c.pdf}\\
    \caption{Snapshots for models md0.5mb0Q0.5 (left) and md0.5mb0Q2.0 (right).\label{fig:snapshots_Q}}
\end{center}
\end{figure}

\subsection{Disk scale length}

We further examine models md1mb1Rd1.5 and md0.5mb1Rd1.5, which
have a larger disk length scale. For these models the total disk mass is the same 
as that of models md1mb1 and md0.5mb1, but the disk scale length 
is larger. The changed disk scale length results in different rotation 
curves (see Fig.~\ref{fig:snapshots_Rdisk}). Given  Eq.~\ref{eq:mX} we expect 
that this leads to fewer spiral arms. The top views of these models are presented in
Fig.~\ref{fig:snapshots_Rdisk} (right panels) and the evolution of 
the bar's amplitude and length in Fig.~\ref{fig:A2_max_Rd}. 
The bar formation epoch of model md1mb1Rd1.5 (2\,Gyr) is
later than that of model mdmb1 (1\,Gyr). Model md0.5mb1Rd1.5 did not form 
a bar within 10\,Gyr, although model md0.5mb1 formed a bar at $\sim6$\,Gyr.
The difference
between these models is that the disk mass fraction ($f_{\rm d}$) for model
md1mb1R1.5 and md0.5mb1R1.5 is smaller than those for model md1mb1 
and md0.5mb1 (see Table~\ref{tb:bar_crit}).
Although the bar formation starts later for model md1mb1Rd1.5, the bar grows
faster, and 
the final bar length at 10\,Gyr is comparable for these models.
The bar's secular evolution, however, may continue further. 
In order to understand what decides the final bar length further simulations
are required. 


\begin{figure}
\includegraphics[width=50mm]{figures/rotation_curve_md1_Rd1.5.pdf}\includegraphics[width=38mm]{figures/md1_Rd1.5_80M_1024c.pdf}\\
\includegraphics[width=50mm]{figures/rotation_curve_md0.5_Rd1.5.pdf}\includegraphics[width=38mm]{figures/md0.5_Rd1.5_110M_1024c.pdf}\\
    \caption{Rotation curves (left) and snapshots at 10 Gyr (right) for models md1mb1Rd1.5 (top) and md0.5mb1Rd1.5 (bottom). 
    The gray dashed curve is the same as the one in Fig.~\ref{fig:snapshots_mb10}. \label{fig:snapshots_Rdisk}}
\end{figure}


\begin{figure*}
\includegraphics[width=\columnwidth]{figures/mode_A2max_Rd.pdf}\includegraphics[width=\columnwidth]{figures/mode_Dbar_Rd.pdf}
\caption{Same as Fig.~\ref{fig:A2_max_mdisk}, but now for models md1mb1Rd1.5 and md0.5mb1Rd1.5 with md1mb1 shown as reference.
\label{fig:A2_max_Rd}}
\end{figure*}


%!TEX root = hopfwright.tex

%%%%%%%%%%%%%%%%%%
%%% Appendix B %%%
%%%%%%%%%%%%%%%%%%

\section{Appendix: Endomorphism on a Compact Domain}
\label{sec:CompactDomain}



In order to construct the Newton-like map $T$ we defined operators $ A =  DF(\bar{x}_\epsilon) + \cO(\epsilon^2)$ and $A^{\dagger} = A^{-1} + \cO(\epsilon^2)$. 
However, as $(\bar{\alpha}_\epsilon,\bar{\omega}_\epsilon,\bar{c}_\epsilon) = (\pp,\pp,\bar{c}_\epsilon) + \cO(\epsilon^2)$,  the map $A$ can be better thought of as an $\cO(\epsilon^2)$ approximation of $DF(\pp,\pp,\bar{c}_\epsilon)$. 
Thus, when working with the map $T$ and considering points $ x \in  B_\epsilon(r,\rho)$ in its domain, we will often have to measure the distances of $ \alpha$ and $ \omega $ from $ \pp$. 
To that end, we define the following variables which will be used throughout the rest of the appendices. 
\begin{definition}
	\label{def:DeltaDef}
For $ \epsilon \geq 0$, and $r_\alpha,r_\omega,r_c >0$ we define 
\begin{alignat*}{2}
	\da^0 	&:= \tfrac{\epsilon^2}{5} ( 3 \pp -1) & \qquad\qquad
	\da 	&:= \da^0 + r_\alpha \\
	\dw^0 &:=  \tfrac{\epsilon^2}{5} &
	\dw &:=  \dw^0 + r_{\omega} \\ 
	\dc^0 &:=  \tfrac{2 \epsilon}{\sqrt{5}} &
	\dc &:=  \dc^0 + r_c . 
	% \\
	% \dt^0  &:= \dw^0 + \tfrac{1}{2} (\dw^0)^2 &
	% \dt  &:= \dw + \tfrac{1}{2} \dw^2 \\
	% \dtt^0  &:= 2 \dw^0 + \tfrac{1}{2} (2\dw^0)^2 &
	% \dtt  &:= 2 \dw + \tfrac{1}{2} (2\dw)^2  .
\end{alignat*}
\end{definition}


% \note[J]{
% 	I believe that we can replace the bounds $\dt$ by $\dw$  and $\dtt$ by $2 \dw$.In short, this follows from the following estimate.
% 	\[
% 	| e^{-i \omega }+i| \leq \int_{\pp}^\omega |\tfrac{\partial}{\partial \omega}  e^{-i \omega} | d\omega \leq  \int_{\pp}^\omega |1| d\omega = |\omega - \pp| .
% 	\]
% 	I have not gone through and done this yet. }
% \note[JB]{I think you are right. I think it also follows from $|e^{-i(\pp+\dw)}+i|^2=|e^{-i\dw}-1|^2 = (\cos \dw -1)^2+(\sin \dw)^2=2(1-\cos\dw) \leq 2 \cdot \frac{1}{2} \dw^2$.}
%
When considering an element $ ( \alpha , \omega, c)$ for our $\cO(\epsilon^2)$ analysis, we are often concerned with the 
 distances $|\alpha - \pp|$, $|\omega - \pp|$ and $ \| c - \bar{c}_\epsilon\|$, each of which is of order $\epsilon^2$.  
To create some  notational consistency in these definitions, $\da^0$ and $\dw^0$ are of order $\epsilon^2$, whereas $\dc^0$ is not capitalized as it is of order $\epsilon$. 
Using these definitions, it follows that for any $\rho>0$ and all  $(\alpha, \omega, c ) \in B_\epsilon(r,\rho)$ we have: 
\begin{alignat*}{1}
| \alpha - \pp | & \leq  \da       \\ 
	 | \omega - \pp| & \leq  \dw   \\
	\|c \| &\leq  \dc  .
	%  \\
	% | e^{- i \omega} + i| &\leq  \dt \\
	% | e^{-2 i \omega } +1| &\leq \dtt  .
\end{alignat*}
In this interpretation the superscript $0$ simply refers to $r=0$, i.e., the center of the ball $(\alpha,\omega,c) = \bx_\epsilon$.

The following elementary lemma will be used frequently in the estimates. 
\begin{lemma}\label{lem:deltatheta}
For all $x\in \R$ we have $|e^{ix}-1| \leq |x|$.
Furthermore, for all $|\omega - \bomega_\epsilon  | \leq r_\omega$  
%\note[JB]{I think this should be $|\omega - \bomega_\epsilon| \leq r_\omega$, no?} \note[J]{Yes, that is correct } 
we have 
$ |e^{- i \omega} + i| \leq  \dw$ and
$ | e^{-2 i \omega } +1| \leq 2 \dw $ .
\end{lemma}
\begin{proof}
We start with
\[
  |e^{ix}-1|^2 = (\cos x -1)^2+(\sin x)^2=2(1-\cos x) \leq 2 \cdot \tfrac{1}{2} x^2 = x^2.
\]
% Let $w = \omega - \pp$. Then $|w| \leq \dw$ and, using the previous inequality,
% \[
% | e^{- i \omega} + i|^2=
% |e^{-i(\pp+w)}+i|^2=|e^{-i w}-1|^2 \leq  w^2 =  \dw^2.
% \]
% \note[J]{To avoid using $w$ and $\omega$ in the same line, I propose we switch $ w \mapsto \theta$, as below. Also the last equality should be an inequality.}

Let $\theta = \omega - \pp$. Then $|\theta| \leq \dw$ and, using the previous inequality,
\[
| e^{- i \omega} + i|^2=
|e^{-i(\pp+\theta)}+i|^2=|e^{-i\theta}-1|^2 \leq  \theta^2 \leq  \dw^2.
\]
The final asserted inequality follows from an analogous argument.
\end{proof}


While the operators $U_\omega$ and $L_\omega$ are not continuous in $ \omega$ on all of $ \ell^1_0$, they are within the compact set $ B_\epsilon(r,\rho)$. 
To denote the derivative of these operators, we  define
\begin{alignat}{1}
	U_{\omega}' &:=  - i K^{-1} U_{\omega} \nonumber \\
	L_{\omega}' &:= - i \sigma^+( e^{- i \omega} I + K^{-1} U_{\omega}) + i \sigma^-(e^{i \omega} I - K^{-1} U_{\omega})  , \label{e:Lomegaprime}
\end{alignat}
and we derive Lipschitz bounds on $U_\omega$ and $L_\omega$ in the following proposition.
 
\begin{proposition}
	\label{prop:OmegaDerivatives}
	For the definitions above, $ \frac{\partial }{\partial  \omega} U_\omega = U_{\omega}' $ and $ \frac{\partial }{\partial  \omega}  L_\omega= L_{\omega}' $. 
	Furthermore,  for any $ (\alpha, \omega,c) \in B_\epsilon(r,\rho)$, we have the norm estimates
	\begin{alignat}{1}
	\| (U_{\omega} - U_{\omega_0} )c \| &\leq   \dw  \rho \nonumber  \\
	\|( L_{\omega} - L_{\omega_0} )c \| &\leq  2  \dw (  \dc +  \rho) .
	\label{e:LomegaLip}
	\end{alignat}
\end{proposition}
% \note[J]{ There was a mistake in the statement of this proposition. I changed the estimate $\| U_{\omega} - U_{\omega_0}  \| $ to $ \| (U_{\omega} - U_{\omega_0} )c \| $. Likewise for $ L_\omega$. }

\begin{proof}
One easily calculates that $ \frac{\partial U_\omega}{\partial  \omega} =  U_{\omega}'$,  whereby
$
	\| (U_{\omega} - U_{\omega_0} )c \| \leq \int_{\omega_0}^\omega \| \tfrac{\partial}{\partial \omega} U_\omega c \|  \leq    \dw  \rho  
$. 
Calculating $ \frac{\partial }{\partial  \omega}  L_{\omega} $, we obtain the following:
\begin{alignat*}{1}
 \frac{\partial }{\partial  \omega}  L_{\omega} 
&=  \frac{\partial }{\partial  \omega} \left[  \sigma^+( e^{- i \omega} I + U_{\omega}) + \sigma^-(e^{i \omega} I + U_{\omega}) \right] \\
&= - i \sigma^+( e^{- i \omega} I + K^{-1} U_{\omega}) + i \sigma^-(e^{i \omega} I - K^{-1} U_{\omega}) ,
\end{alignat*}
thus proving $ \frac{\partial L_\omega}{\partial  \omega} =  L_{\omega}'$,
and 
$\|( L_{\omega} - L_{\omega_0} )c \| \leq  \int_{\omega_0}^\omega \| \tfrac{\partial}{\partial \omega} L_\omega c \|  \leq   \dw ( 2  \dc + 2 \rho)$.
\end{proof}

\begin{proposition}
	Let $\epsilon\geq 0$ and  $r=(r_\alpha,r_\omega,r_c) \in \R^3_+$. 
	For any $ \rho > 0$ the map 
	 $T:B_{\epsilon}(r,\rho) \to \R^2 \times \ell^K_0 $ is well defined. 	
	We define functions 
% \note[J]{New definitions for $C_0$ and $C_1$ as the old ones did not quite match the estimates proven below. }
	\begin{alignat*}{1}
%	C_0 &:=  \frac{2 \epsilon^2}{\pi} 
%	\left[
%		\frac{8}{5},\frac{8}{5\sqrt{5}} \sqrt{\left(1-3 \pi /4 \right)^2+(2+\pi )^2},\frac{5 \pi }{2} 
%	\right]
%	\cdot \overline{A_0^{-1} A_1} \cdot [ \da , \dw , \dc ]^T ,
%%	\\
	% C_0 &:=  \frac{2 \epsilon^2}{\pi}
	% \left[
	% 	\frac{8}{5},\frac{2}{5} \sqrt{16+ 8\pi + 5 \pi^2},\frac{5 \pi }{2}
	% \right]
	% \cdot \overline{A_0^{-1} A_1} \cdot [ \da , \dw , \dc ]^T ,
	% \\
	C_0 &:=  \frac{2 \epsilon^2}{\pi} 
	\left[
	\frac{8}{5},\frac{2}{5} \sqrt{16+ 8\pi + 5 \pi^2},\frac{5 \pi }{2} 
	\right]
	\cdot \overline{A_0^{-1} A_1} \cdot [0,0 , \dc ]^T ,
	\\
%	C_1 &:= \frac{5 }{2 \pi} \left(1 +   \frac{4 \epsilon  }{5} \left(2+\sqrt{5}\right) \right) , \\
	% C_1 &:= \frac{5 }{2 \pi} + 2  \epsilon   \left(2+\sqrt{5}\right)  , \\
	C_1 &:= \frac{5 }{2 \pi} + \frac{\epsilon \sqrt{10}}{\pi}, \\
	C_2 &:= \dw  \left[  (1 + \pp) + \epsilon \pi  \right] , \\
	C_3 &:=  
	\da (2+ \dc) +	2 \dw (1+\pp) 
		+ \epsilon \left[ \pi + 2\da  + 4 \dc \da + \pi \dw \dc  + (\pp + \da ) \dc^2 \right] ,
	\end{alignat*}
where the expression for $C_0$ should be read as a product of a row vector, a $(3 \times 3)$ matrix and a column vector.
Furthermore we define, for any $\epsilon,r_\omega$ such that $C_1 C_2 <1$,
	\begin{equation}
		C(\epsilon,r_\alpha,r_\omega,r_c) := \frac{C_0+ C_1 C_3}{1 - C_1 C_2}
		 \, .
		\label{eq:RhoConstant}
	\end{equation}
	All of the functions $C_0,C_1,C_2,C_3$ and $C$ are nonnegative and monotonically increasing in their arguments $\epsilon$ and~$r$. 
	Furthermore, if  $C_1 C_2 < 1$ and $	C(\epsilon,r_\alpha,r_\omega,r_c) \leq \rho $
	then $\| K^{-1} \pi_c  T( x) \| \leq \rho $
	for $x \in B_{\epsilon}(r,\rho)$. 
	\label{prop:DerivativeEndo}
\end{proposition}

% \marginpar{This proposition is vague about the actual spaces being used, ie. $\ell_1,\ell^K_0$, etc.}

\begin{proof}
	Given their definitions, it is straightforward to check that the functions $C_i$ and $C$ are monotonically increasing in their arguments.  
	To prove the second half of the proposition, we split 
	$K^{-1} \pi_c  T(x)$ into several pieces. 
%\note[JB]{$\pi_c$ and $\pi_{\ge 2}$ added. Jonathan, could you please go through this and check?}
%\note[JB]{This did not work, since we do not have that $x$ is bounded by $[\da,\dw,\dc]^T$. Jonathan: what you probably meant was what I introduce as $\pi_c^0 x$, but could you please check?} 
	We define the projection $\pi_c^0 x = (0,0,\pi_c x)$.
We then obtain
	\begin{alignat*}{1}
	K^{-1} \pi_c  T(x)  &= K^{-1} \pi_c   [ x - A^{\dagger} F(x) ]   \\
	&= K^{-1} \pi_c  [ I \pi_c^0 x -    A^{\dagger} ( A \pi_c^0 x + F(x) - A \pi_c^0 x)]  \\
	&= \epsilon^2 K^{-1} \pi_c (A_0^{-1}A_{1})^2 \pi_c^0 x + K^{-1} \pi_c A^{\dagger} (F(x) - A \pi_c^0 x) \nonumber \\
	&=  \frac{2 \epsilon^2}{i\pi} \hat{U} \pi_{\ge 2} A_1 A_0^{-1}A_{1} \pi_c^0 x +\frac{2 }{i\pi} \hat{U}  \pi_{\ge 2} (I-\epsilon A_1 A_0^{-1}) (F(x) - A\pi_c^0 x)  ,
\end{alignat*}
where we have used that $K^{-1} \pi_c A_0^{-1} = \frac{2}{i\pi} \hat{U} \pi_{\ge 2}$, with the projection $\pi_{\ge 2}$ defined in~\eqref{e:pige2}.
By using $\| \hat{U} \| \leq \frac{5}{4}$, see Proposition~\ref{p:severalnorms}, we obtain the estimate
%\note[J]{Changed $[\da,\dw,\dc ]^T$ to $[0,0,\dc ]^T$}
\begin{equation}
	\| K^{-1} \pi_c T(x) \| \leq   \frac{2 \epsilon^2}{\pi} \overline{\hat{U}\pi_{\ge 2} A_1} \cdot  \overline{ A_0^{-1}A_{1}}  \cdot
	[0,0,\dc ]^T +\frac{5 }{2 \pi} \left(1 + \epsilon \| A_1 A_0^{-1} \| \right) \|F(x) - A\pi_c^0 x \| .
	\label{eq:DerivativeEndo}
\end{equation}
Here the $(1 \times 3)$ row vector $\overline{\hat{U}\pi_{\ge 2} A_1}$ is an upper bound on $\hat{U}\pi_{\ge 2} A_1$ interpreted as a linear operator from $\R^2 \times \ell^1_0$ to $\ell^1_0$, thus extending in a straightforward manner the definition of upper bounds given in  Definition~\ref{def:upperbound}.
	
	
	We have already calculated  an expression for
	 $ \overline{ A_0^{-1}A_{1}}$ in Proposition~\ref{prop:A0A1},  and  $  \| A_1 A_0^{-1}\| =\frac{2\sqrt{10}}{5}$ by Proposition~\ref{prop:A1A0}.  In order to finish the calculation of the right hand side of Equation \eqref{eq:DerivativeEndo}, we need to  estimate  $\| F(x) - A\pi_c^0 x \|$ and $\overline{\hat{U} \pi_{\ge 2} A_1} $. 
	We first calculate a bound on $\hat{U} \pi_{\ge 2} A_1 $. 
	We note that $ \hat{U} \pi_{\ge 2} A_1  =  \hat{U} \e_2 ( i_\C A_{1,2} \pi_{\alpha,\omega})+ \hat{U} \pi_{\ge 2}A_{1,*} \pi_c$.	
As $\|\hat{U} e_2\| = \| \tfrac{4-2i}{5} \e_2\|$,
it follows from the definition of $A_{1,2}$ 
that 
\[
	 \left| i_\C  A_{1,2}
	 \left( \!\!\begin{array}{c}\alpha \\ \omega \end{array} \!\!\right) \right|  
	 \cdot \| \hat{U} \e_2 \| 
	 \leq 
	 \left(\frac{\sqrt{20}}{5} |\alpha| +  \frac{\sqrt{(2-3 \pi/2)^2 +4(2+\pi)^2}}{5} |\omega| \right)  \cdot \frac{4}{\sqrt{5}}.
\]
	To calculate $ \| \hat{U} \pi_{\ge 2} A_{1,*} \|$ we note that $ \| \hat{U}\| \leq \frac{5}{4}$ and $ \|A_{1,*}\| = \pp \| L_{\omega_0} \| \leq 2 \pi$. 
	Hence $ \| \hat{U} \pi_{\ge 2} A_{1,*} \| \leq \frac{5 \pi}{2}$. 
	Combining these results, we obtain  that
%\note[JB]{I think the second, rearranged version, of the root looks ``nicer''. }
%	\[
%	\overline{\hat{U} \pi_{\ge 2}  A_1 } = \left[\frac{8}{5},\frac{8}{5\sqrt{5}} \sqrt{\left(1-3 \pi /4 \right)^2+(2+\pi )^2},\frac{5 \pi }{2} \right].
%	\] 
	\[
	\overline{\hat{U} \pi_{\ge 2}  A_1 } = \left[\frac{8}{5},\frac{2}{5} \sqrt{16 + 8 \pi + 5 \pi^2},\frac{5 \pi }{2} \right].
	\] 
Thereby, it follows from~\eqref{eq:DerivativeEndo} that 
\begin{equation}\label{e:C0C1}
	\| K^{-1} \pi_c T(x) \| \leq C_0 + C_1 \| F(x) - A \pi_c^0 x\|. 
\end{equation}
We now calculate
	\begin{alignat*}{1}
	F(x) - A \pi_c^0 x &= 
	(i \omega + \alpha e^{-i \omega} ) \e_1 + 
	( i \omega K^{-1} + \alpha U_{\omega}) c + 
	\epsilon \alpha e^{-i \omega} \e_2  +
	\alpha \epsilon L_\omega c + 
	\alpha \epsilon [ U_{\omega} c] * c  
	\\ &\qquad 
	- \pp (i K^{-1} + U_{\omega_0} + \epsilon L_{\omega_0} ) c \\
	&= i ( \omega - \pp) K^{-1} c + ( \alpha - \pp) U_{\omega} c +  \pp ( U_{\omega} - U_{\omega_0})c  \nonumber \\
	&\qquad  + \left[i ( \omega - \pp ) + ( \alpha - \pp) e^{-i \omega} + \pp( e^{- i \omega }+ i)\right] \e_1  \nonumber
	\\ 
	&\qquad  +\epsilon  \alpha   e^{-i \omega}  \e_2  
+  ( \alpha- \pp)  \epsilon L_{\omega} c + \pp \epsilon ( L_{\omega} - L_{\omega_0}) c + \alpha \epsilon [ U_{\omega} c ] * c .
	\end{alignat*}
Taking norms and using~\eqref{e:LomegaLip} and Lemma~\ref{lem:deltatheta}, we obtain 
	\begin{alignat*}{1}
	\| F(x) - A \pi_c^0 x\|& \leq  
	 \dw \rho + \da \dc + \pp \dw \rho
    +	2 (\dw + \da + \pp \dw)  
	   \\
	&\qquad + \epsilon \left[ 2(\pp + \da ) + 4 \dc \da + \pi  \dw (  \dc + \rho) + (\pp + \da ) \dc^2 \right]  \\
		&= \dw [ (1+\pp) +   \epsilon \pi ] \rho \nonumber \\ 
	&\qquad +  \da (2 + \dc)
	+	2 \dw (1+\pp) 
	+ \epsilon \left[ \pi + 2\da  + 4 \dc \da + \pi \dw \dc  + (\pp + \da ) \dc^2 \right].  
	\end{alignat*}


	We have now computed all of the necessary constants. Thus $ \| F(x) - A \pi_c^0 x \| \leq C_2 \rho + C_3$, and from~\eqref{e:C0C1}   we obtain 
	\begin{eqnarray*}
	\| K^{-1} \pi_c T(c) \|
	&\leq & C_0 +  C_1 ( C_2  \rho + C_3),
	\end{eqnarray*}
with the constants defined in the statement of the proposition.
We would like to select values of $\rho$ for which 
	\[
	\| K^{-1} \pi_c T(c) \| \leq \rho
	\]
	This is true if  
	$	C_0 +  C_1 ( C_2  \rho + C_3) \leq \rho$, 
	or equivalently 
	\[
	\frac{C_0 + C_1 C_3 }{1 - C_1 C_2} \leq \rho.
	\]
	This proves the theorem.
\end{proof}

%%%%%%%%%%%%%%%%%%
%%% Appendix C %%%
%%%%%%%%%%%%%%%%%%


\section[]{Relation between bulge-to-disk mass ratio and shear rate}

In Section 3.4 we showed that the bulge-to-disk mass ratio ($B/D$) is not 
always a good indicator for the shear rate ($\Gamma$), because $\Gamma$ 
also depends on other parameters such as the disk-mass fraction ($f_{\rm d}$). 
Here, we construct additional initial conditions by sequentially changing some 
parameters in order to investigate their importance. We do not simulate 
these models, but measure $B/D$ and $\Gamma$ in the generated models at $t=0$.
All parameters of these models are summarized in Tables \ref{tb:models_add} and 
 \ref{tb:models_add2}.

In Fig.~\ref{fig:Gamma_BD}, we present the relation between $B/D$ and $\Gamma$
calculated from the additional initial conditions.
If we keep both $f_{\rm d}$ and the bulge scale length ($r_{\rm b}$) constant,
$\Gamma$ monotonically increases as $B/D$ increases (square symbols).
But if we increase $r_{\rm b}$ while keeping $f_{\rm d}$ constant, then $\Gamma$ 
also increases (triangle symbols). 
If we increase $f_{\rm d}$ and keep $B/D$ and $r_{\rm b}$ constant,
$\Gamma$ increases (diamond symbols). 
The halo scale length ($r_{\rm h}$) and scale velocity 
($\sigma_{\rm h}$), on the other hand, barely affect the relation between $B/D$ and $\Gamma$
(circle symbols).


\begin{figure}
\includegraphics[width=\columnwidth]{figures/shear_rate_BD_fd.pdf}
\caption{Relation between bulge-to-disk mass ratio ($B/D$) and shear rate ($\Gamma$). \label{fig:Gamma_BD}}
\end{figure}


\begin{table*}
\begin{center}
%\rotate
\caption{Parameters for additional initial conditions\label{tb:models_add}}
\begin{tabular}{lccccccccccc}
%\tabletypesize{\scriptsize}
%\tablewidth{0pt}
%\startdata 
\hline
           &  \multicolumn{3}{l}{Halo} &  \multicolumn{4}{l}{Disk} &  \multicolumn{3}{l}{Bulge} \\
Parameters &  $a_{\rm h}$ & $\sigma_{\rm h}$ & $1-\epsilon_{\rm h}$ & $M_{\rm d}$ & $R_{\rm d}$ & $z_{\rm d}$ & $\sigma_{R0}$  & $a_{\rm b}$ & $\sigma_{\rm b}$ & $1-\epsilon_{\rm b}$ \\ 
Model   &  (kpc) & ($\kms$) &  &  $(10^{10}M_{\odot})$ & (kpc) & (kpc) & ($\kms$)  & (kpc) & $(\kms)$\\
\hline \hline
Add1 & 8.2 & 350 & 0.9  & 2.45 & 2.8 & 0.36 & 105 & 0.64 & 300 & 1.0  \\ %t13
Add2 & 11.5 & 443 & 0.9  & 2.45 & 2.8 & 0.36 & 105 & 0.65 & 400 & 1.0  \\ %t5
Add3 & 8.2 & 370 & 0.9  & 2.45 & 2.8 & 0.36 & 105 & 0.64 & 500 & 1.0  \\ %t14
Add4 & 10 & 340 & 0.9  & 2.45 & 2.8 & 0.36 & 105 & 0.64 & 550 & 1.0  \\ %t8
Add5 & 8.2 & 295 & 0.9  & 2.45 & 2.8 & 0.36 & 105 & 0.65 & 600 & 1.0  \\ %t7
Add6 & 8.2 & 284 & 0.9  & 2.45 & 2.8 & 0.36 & 105 & 1.3 & 370 & 1.0  \\ %t11
Add7 & 8.2 & 330 & 0.9  & 2.45 & 2.8 & 0.36 & 105 & 0.8 & 380 & 1.0   \\  %t12
Add8 & 8.2 & 330 & 1.0  & 2.45 & 2.8 & 0.36 & 105 & 0.64 & 550 & 1.0  \\ %t10
Add9 & 12 & 400 & 1.0  & 2.45 & 2.8 & 0.36 & 105 & 0.64 & 540 & 1.0  \\ %t9
Add10 & 8.2 & 370 & 0.9  & 1.47 & 2.8 & 0.36 & 105 & 0.64 & 390 & 1.0  \\ %t15
Add11 & 12 & 330 & 0.9  & 2.45 & 2.8 & 0.36 & 105 & 0.64 & 486 & 1.0  \\ %t17
\hline
\end{tabular}
\end{center}
\end{table*}


\begin{table*}
\begin{center}
\caption{Obtained values for additional initial conditions\label{tb:models_add2}}
\begin{tabular}{lccccccccc}
%\tabletypesize{\scriptsize}
%\rotate
%\tablewidth{0pt}
%\startdata 
\hline
Model    & $M_{\rm d}$ & $M_{\rm b}$ & $M_{\rm h}$ & $M_{\rm b}/M_{\rm d}$& $R_{\rm d, t}$ & $r_{\rm b, t}$ & $r_{\rm h, t}$ & $f_{\rm d}$  &  $\Gamma$\\ 
   & ($10^{10}M_{\odot}$) & ($10^{10}M_{\odot}$) & ($10^{10}M_{\odot}$) & (kpc) & (kpc) & (kpc) &  &   &  \\ 
\hline  \hline
Add1 & 2.57 & 0.514 & 56.0 & 0.20 & 31.6 & 3.57 & 284 &  0.346 & 0.682 \\ %test13
Add2 & 2.58 & 1.21 & 137 & 0.47 & 31.6 & 5.32 & 330 & 0.343 & 0.706 \\ %test5
Add3 & 2.69 & 2.03 & 94.6 & 0.75 & 31.6 & 6.65 & 234 &  0.321 & 0.895 \\ %test14
Add4 & 2.59 & 2.74 & 124 & 1.05 & 31.6 & 8.67 & 288 &   0.340 & 1.04 \\ %test8
Add5 & 2.61 & 3.29 & 93.2 & 1.26 & 31.6 & 9.48 & 270 &  0.307 & 1.10 \\ %test7
Add6 & 2.59 & 1.19 & 45.9 & 0.46 & 31.6 & 6.28& 265 &  0.332 & 0.869 \\ %test11
Add7 & 2.58 & 1.11 & 61.3 & 0.43 & 31.6 & 5.31 & 251 & 0.348  & 0.789 \\ %test12
Add8 & 2.61 & 2.62 & 108 & 1.0 & 31.6 & 7.98 & 494 &  0.322 &  0.996\\ %test10
Add9 & 2.59 & 2.64 & 195 & 1.02 & 31.6 & 8.46 & 687 &  0.341 & 1.00 \\  % test9
Add10 & 1.58 & 1.18 & 74.8 & 0.74 & 31.6 & 5.65 & 234 &  0.251  & 0.675 \\ %test15
Add11 & 2.75 & 2.07 & 130 & 0.75 & 31.6 & 8.00 & 324 &  0.401 & 0.992 \\ %test17
\hline
\end{tabular}
\end{center}
\medskip
\end{table*}





\bibliographystyle{mnras}
\bibliography{reference}


\label{lastpage}
\end{document}
