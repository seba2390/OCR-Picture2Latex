% mnras_template.tex
%
% LaTeX template for creating an MNRAS paper
%
% v3.0 released 14 May 2015
% (version numbers match those of mnras.cls)
%
% Copyright (C) Royal Astronomical Society 2015
% Authors:
% Keith T. Smith (Royal Astronomical Society)

% Change log
%
% v3.0 May 2015
%    Renamed to match the new package name
%    Version number matches mnras.cls
%    A few minor tweaks to wording
% v1.0 September 2013
%    Beta testing only - never publicly released
%    First version: a simple (ish) template for creating an MNRAS paper

%%%%%%%%%%%%%%%%%%%%%%%%%%%%%%%%%%%%%%%%%%%%%%%%%%
% Basic setup. Most papers should leave these options alone.
\documentclass[a4paper,fleqn,usenatbib]{mnras}

\usepackage{amsmath,amssymb}    % Advanced maths commands

% MNRAS is set in Times font. If you don't have this installed (most LaTeX
% installations will be fine) or prefer the old Computer Modern fonts, comment
% out the following line
%\usepackage{newtxtext,newtxmath}
% Depending on your LaTeX fonts installation, you might get better results with one of these:
\usepackage{mathptmx}
%\usepackage{txfonts}


% Jeroen: The below is a patch to fix the annoying pdf generation errors related to hyperref
% spreading links on two pages. I have no clue what it does but it works :)
% Source: https://github.com/ho-tex/hyperref/issues/19
\usepackage{etoolbox}
\makeatletter
\patchcmd\@combinedblfloats{\box\@outputbox}{\unvbox\@outputbox}{}{%
   \errmessage{\noexpand\@combinedblfloats could not be patched}%
}%
\makeatother



% Use vector fonts, so it zooms properly in on-screen viewing software
% Don't change these lines unless you know what you are doing
\usepackage[T1]{fontenc}
\usepackage{ae,aecompl}

%%%%% AUTHORS - PLACE YOUR OWN PACKAGES HERE %%%%%

% Only include extra packages if you really need them. Common packages are:
\usepackage{graphicx}   % Including figure files
\usepackage{amssymb}    % Extra maths symbols


\usepackage{natbib, aas_macros}
\usepackage{graphicx, color, url}
\usepackage{grffile} 
%\usepackage[a4paper]{geometry}


% If your system does not have the AMS fonts version 2.0 installed, then
% remove the useAMS option.
%
% useAMS allows you to obtain upright Greek characters.
% e.g. \umu, \upi etc.  See the section on "Upright Greek characters" in
% this guide for further information.
%
% If you are using AMS 2.0 fonts, bold math letters/symbols are available
% at a larger range of sizes for NFSS release 1 and 2 (using \boldmath or
% preferably \bmath).
%
% The usenatbib command allows the use of Patrick Daly's natbib.sty for
% cross-referencing.
%
% If you wish to typeset the paper in Times font (if you do not have the
% PostScript Type 1 Computer Modern fonts you will need to do this to get
% smoother fonts in a PDF file) then uncomment the next line
% \usepackage{Times}

%%%%% AUTHORS - PLACE YOUR OWN MACROS HERE %%%%%
\newcommand{\kms}{\mathrm{km\ s^{-1}}\,}

\def\JB#1{{\bf {\color{red}[#1 -- Jeroen]}}}
\def\MF#1{{\bf {\color{blue}[#1 -- Michiko]}}}
\def\simon#1{{\bf {\color{green}[#1 -- SPZ]}}}
\def\Simon#1{{\bf {\color{green}[#1 -- SPZ]}}}
\def\SPZ#1{{\bf {\color{green}[#1 -- SPZ]}}}
\def\JNB#1{{\bf {\color{orange}[#1 -- Junichi]}}}

%%%%%%%%%%%%%%%%%%%%%%%%%%%%%%%%%%%%%%%%%%%%%%%%

\title[The dynamics of stellar disks in live halos]{The dynamics of stellar disks in live dark-matter halos}
\author[M. S. Fujii et al.]{M. S. Fujii$^{1}$\thanks{E-mail:
fujii@astron.s.u-tokyo.ac.jp (MSF)}, J. B\'edorf$^{2}$, J. Baba$^{3}$, and S. Portegies Zwart$^{2}$\\
$^{1}$Department of Astronomy, Graduate School of Science, The University of Tokyo, 7-3-1 Hongo, Bunkyo-ku, Tokyo, 113-0033, Japan\\
$^{2}$Leiden Observatory, Leiden University, NL-2300RA Leiden, The Netherlands\\
$^{3}$National Astronomical Observatory of Japan, Mitaka-shi, Tokyo 181-8588, Japan}

\date{Accepted . Received ; in original form }

\begin{document}
\label{firstpage}
\pagerange{\pageref{firstpage}--\pageref{lastpage}} \pubyear{2002}
\maketitle



\begin{abstract}

Recent developments in computer hardware and software enable
researchers to simulate the self-gravitating evolution of galaxies at
a resolution comparable to the actual number of stars.  Here we
present the results of a series of such simulations.  We performed
$N$-body simulations of disk galaxies with between 100 and 500 million
particles over a wide range of initial conditions.  Our calculations
include a live bulge, disk, and dark matter halo, each of which is
represented by self-gravitating particles in the $N$-body code.  The
simulations are performed using the gravitational $N$-body tree-code
{\tt Bonsai} running on the Piz Daint supercomputer.
We find that the time scale over which the bar forms increases
exponentially with decreasing disk-mass fraction and that the bar formation
epoch exceeds a Hubble time when the disk-mass fraction is $\sim0.35$.
These results can be explained with the swing-amplification theory.
The condition
for the formation of $m=2$ spirals is consistent with that for the
formation of the bar, which is also an $m=2$ phenomenon.  We further argue
that the non-barred grand-design spiral galaxies
are transitional, and that they evolve to barred
galaxies on a dynamical timescale.
We also confirm that the disk-mass fraction and shear rate
are important parameters for the morphology of disk galaxies.
The former affects the number of spiral arms and the bar formation
epoch, and the latter determines the pitch angle of the spiral arms.


\end{abstract}

\begin{keywords}
galaxies: kinematics and dynamics --- galaxies: spiral --- galaxies: structure ---
galaxies: evolution --- methods: numerical
\end{keywords}


\section{Introduction}  \label{sec:introduction}

\newcommand\inexpIntro[3]{#1?(#2,#3).}
\newcommand\rinexpIntro[3]{*#1?(#2,#3).}
\newcommand\outexpIntro[3]{#1!(#2,#3).}
\newcommand\outatomIntro[3]{#1!(#2,#3)}

We propose a fully automated method for proving termination of \(\pi\)-calculus processes.
Although there have been a lot of studies on termination analysis for the \(\pi\)-calculus
and related calculi~\cite{Deng06IC,Demangeon07,SangiorgiTermination,KobayashiHybrid,Yoshida04IC,DBLP:journals/jlp/DemangeonHS10,Venet98SAS}, most of them have been rather theoretical,
and there have been surprisingly little efforts in developing  fully automated termination
verification methods and tools based on them. To our knowledge,
Kobayashi's \typical{}~\cite{TyPiCal,KobayashiHybrid} is the only exception that
can prove termination of \(\pi\)-calculus processes (extended with natural numbers)
fully automatically, but its termination analysis is quite limited (see Section~\ref{sec:relatedwork}).

Our method is based on a reduction to termination analysis for sequential programs:
we translate a \(\pi\)-calculus process \(P\) to a sequential program \(S_P\), so that
if \(S_P\) is terminating, so is \(P\). The reduction allows us to use
powerful, mature methods and tools
for termination analysis of sequential programs~\cite{heizmann2016ultimate,freqterm,DBLP:conf/lics/PodelskiR04,Kuwahara2014Termination,DBLP:journals/cacm/CookPR11}.

The idea of the translation is to convert a chain of communications on replicated input
channels to a chain of recursive function calls of the target sequential program.
Let us consider the following Fibonacci process:
\begin{align*}
    & \rinexpIntro{\fib}{n}{r}
        \ifexp{n<2}{ \soutatom{r}{1} \\ &\quad}
                   { \nuexp{s_1} \nuexp{s_2} (\outatomIntro{\fib}{n-1}{s_1} \PAR \outatomIntro{\fib}{n-2}{s_2} \PAR \sinexp{s_1}{x}\sinexp{s_2}{y}\soutatom{r}{x+y}) \\}
    & \PAR \outatomIntro{\fib}{m}{r}
\end{align*}
Here, the process
$\rinexpIntro{\fib}{n}{r} \ldots$ is a function server that computes the \(n\)-th Fibonacci number
in parallel and returns the result to \(r\),
and $\outatom{\fib}{m}{r}$ sends a request for computing the \(m\)-th Fibonacci number;
those who are not familiar with the syntax of the \(\pi\)-calculus may wish to consult
Section~\ref{sec:targetlanguage} first.
To prove that the process above is terminating for any integer \(m\),
it suffices to show that there is no infinite chain of communications on $\fib$:
\[
    \fib(m,r) \to \fib(m_1,r_1) \to \fib(m_2,r_2) \to \cdots.
\]
We convert the process above to the following program:\footnote{The actual translation
  given later is a little more complex.}
\begin{verbatim}
 let rec fib(n) = if n<2 then () else (fib(n-1) [] fib(n-2)) in
 fib(m)
\end{verbatim}
Here, \texttt{[]} represents the non-deterministic choice.
Note that, although the calculation of Fibonacci numbers is not preserved,
for each chain of communications on \texttt{fib}, there is a corresponding
sequence of recursive calls:
\[
\mathtt{fib}(m) \to \mathtt{fib}(m_1) \to \mathtt{fib}(m_2) \to \cdots.
\]
Thus, the termination of the sequential program above implies the termination of
the original process.
As shown in the example above, (i) each communication on a replicated input channel
is converted to a function call, (ii) each communication on a non-replicated input
channel is just removed (or, in the actual translation, replaced by a call of
a trivial function defined by \(f(\seq{x})=(\,)\)), and (iii) parallel composition
is replaced by a non-deterministic choice.
We formalize the translation outlined above and prove its correctness.

The basic translation sketched above sometimes loses too much information.
For example, consider the following process:
\begin{align*}
    & \rinexpIntro{\pre}{n}{r} \soutatom{r}{n-1} \\
    & \PAR \rinexpIntro{f}{n}{r} \ifexp{n<0}{ \soutatom{r}{1} }
                                       { \nuexp{s} (\outatomIntro{\pre}{n}{s} \PAR \sinexp{s}{x}\outatomIntro{f}{x}{r}) } \\
    & \PAR \outatomIntro{f}{m}{r}
\end{align*}
The translation sketched above would yield:
\begin{verbatim}
  let pred(n) = n-1 in
  let rec f(n) = if n<0 then () else (pred(n) [] f(*)) in
  f(m)
\end{verbatim}
Here, \texttt{*} represents a non-deterministic integer: since we have removed
the input $\sinatom{s}{x}$, we do not have information about the value of \( x \).
As a result, the sequential program above is non-terminating, although the original
process is terminating.
To remedy this problem, we also refine the basic translation above by using a refinement
type system for the \(\pi\)-calculus. Using the refinement type system,
we can infer that the value of \(x\) in the original process is less than \(n\),
so that we can refine the definition of \texttt{f} to:
\begin{verbatim}
 let rec f(n) = ... else (pred(n) [] let x=* in assume(x<n);f(x))
\end{verbatim}
The target program is now terminating, from which
we can deduce that the original process is also terminating.
We have implemented an automated tool based on the refined translation above.

The contributions of this paper are summarized as follows.
\begin{itemize}
\item The formalization of the basic translation from the \(\pi\)-calculus
  (extended with integers) to sequential programs, and a proof of its correctness.
\item The formalization of a refined translation based on a refinement type system.
\item An implementation of the refined translation, including automated refinement type
  inference based on CHC solving, and experiments to evaluate the effectiveness of
  our method.
\end{itemize}

The rest of this paper is structured as follows.
Section~\ref{sec:targetlanguage} introduces the source and target languages
of our translation.
Section~\ref{sec:approach} 
formalizes the basic translation, and proves its correctness.
Section~\ref{sec:refinement} refines the basic translation by using a refinement type system.
Section~\ref{sec:implementation} reports an implementation and experiments.
Section~\ref{sec:relatedwork} discusses related work,
and Section~\ref{sec:conclusion} concludes the paper.


\section{$N$-body simulations}

We performed a series of $N$-body simulations of galactic stellar
disks embedded in dark matter halos. In this section, we describe our
choice of parameters and the $N$-body code used for these simulations.


\subsection{Model}
Our models are based on those described in \citet{2008ApJ...679.1239W} and 
\citet{2005ApJ...631..838W}. 
We generated the initial conditions using GalactICS \citep{2005ApJ...631..838W}.
The initial conditions for generating the dark mater halo are taken from the
NFW profile \citep{1997ApJ...490..493N}, which has a density profile following:
\begin{eqnarray}
\rho_{\rm NFW}(r) = \frac{\rho_{\rm h}}{(r/a_{\rm h})(1+r/a_{\rm h})^3},
\end{eqnarray}
and the potential is written as
\begin{eqnarray}
\Phi_{\rm NFW} = -\sigma_{\rm h}^2\frac{\log (1+r/a_{\rm h})}{r/a_{\rm h}}.
\end{eqnarray}
Here the gravitational constant, $G$, is unity, 
$a_{\rm h}$ is the scale radius, $\rho_{\rm h}\equiv\sigma^2/4\pi a_{\rm h}^2$
is the characteristic density, and $\sigma_{\rm h}$ is the characteristic 
velocity dispersion. We adopt $\sigma_{\rm h}=340$ (km\,s$^{-1}$), 
$a_{\rm h}=11.5$ (kpc).
Since the NFW profile is infinite in extent and mass, the 
distribution is truncated by a halo tidal radius using
an energy cutoff $E_{\rm h}\equiv\epsilon_{\rm h}\sigma_{\rm h}^2$,
where $\epsilon _{\rm h}$ is the truncation parameter with $0<\epsilon _{\rm h}<1$.
Setting $\epsilon _{\rm h}=0$ yields a full NFW profile
\citep[see][for details]{2005ApJ...631..838W}. 
We choose the parameters of the dark matter halo such that the resulting
rotation curves have a similar shape.
The choice of parameters is summarized in Table~\ref{tb:params}.

For some models we assume the halo to have net angular momentum. This
is realized by changing the sign of the angular momentum about the
symmetry axis ($J_{z}$). The rotation is parameterized using $\alpha
_{\rm h}$.  For $\alpha _{\rm h}=0.5$, the fraction of halo particles
which have positive or negative $J_{z}$ are the same, in which case the disk
has no net angular momentum. If $\alpha _{\rm h}>0.5$, the halo rotates in the same
direction as the disk.

For the disk component, we adopt an exponential disk for which the surface density
distribution is given by
\begin{eqnarray}
\Sigma (R) = \Sigma_{0} {\rm e}^{-R/R_{\rm d}}.
\end{eqnarray}
The vertical structure is given by ${\rm sech}^2(z/z_{\rm d})$,
where $z_{\rm d}$ is the disk scale height.
The radial velocity dispersion is assumed to follow 
$\sigma_{R}^2(R)=\sigma_{R0}^2\exp(-R/R_{\rm d})$, where 
$\sigma_{R0}$ is the radial velocity dispersion at the disk's center.
Toomre's stability parameter $Q$ 
\citep{1964ApJ...139.1217T,2008gady.book.....B} at
a reference radius (we adopt $2.2R_{\rm d}$), $Q_{0}$, is controlled
by the central velocity dispersion of the disk ($\sigma _{R0}$). 
We tune $\sigma_{R0}$ such that for our standard model (md1mb1) $Q_0=1.2$ .

For model md1mb1 we use as disk mass $M_{\rm d}$=$4.9\times 10^{10}$ $M_{\odot}$,
and the scale length $R_{\rm d}$=2.8 kpc. The disk's truncation radius is set to 30 kpc, the 
scale height $z_{\rm d}$=0.36 kpc and the radial velocity
dispersion at the center of the galaxy to $\sigma_{R0}$=$105$ km\,s$^{-1}$.
The disk is truncated at ($R_{\rm out}$) with a 
radial range for disk truncation ($\delta R$). We adopt 
$R_{\rm out}=30.0$ (kpc) and $\delta R=0.8$ (kpc).

For the bulge we use a Hernquist model~\citep{1990ApJ...356..359H},
but the distribution function is extended with an energy cutoff
parameter ($\epsilon_{\rm b}$) to truncate the profile much in the
same way as we did with the halo model.  The density distribution
and potential of the standard Hernquist model is
\begin{eqnarray}
\rho_{\rm H} = \frac{\rho_{\rm b}}{(r/a_{\rm b})(1+r/a_{\rm b})^3}
\end{eqnarray}
and
\begin{eqnarray}
\Phi_{\rm H} = \frac{\sigma_{\rm b}^2}{1+r/a_{\rm b}}.
\end{eqnarray}
Here $a_{\rm b}$, $\rho_{\rm b}=\sigma_{\rm b}^2/(2\pi a_{\rm b}^2)$, and
$\sigma _{\rm b}$ are the scale length, characteristic density, and
the characteristic velocity of the bulge, respectively.
We set $\sigma_{\rm b}$=300 km\,s$^{-1}$, bulge scale length $a_{\rm b}$=0.64 kpc,
and the truncation parameter ($\epsilon _{\rm b}$=0.0).
This results in a bulge mass of $4.6\times 10^9M_{\odot}$,
which is consistent with the Milky Way model proposed by
\citet{2010ApJ...720L..72S}, and reproduces
the bulge velocity distribution obtained by BRAVA observations~\citep{2012AJ....143...57K}.
We do not assume an initial rotational velocity for the bulge.

For the simulation models we vary the disk mass, bulge mass,
scale length, halo spin, and $Q_0$.  Since the adopted generator for
the galaxies is an irreversible process and due to the randomization
of the selection of particle positions and velocities we cannot
guarantee that the eventual velocity profile is identical to the input
profile, but we confirmed by inspection that they are
indistinguishable.  The initial conditions for each of the models are
summarized in Table \ref{tb:params}.  The mass and tidal radius for
the bulge, disk, and halo as created by the initial condition
generator are given in Table \ref{tb:mass_radius}.

In each of the models we fix  the number of particles used for 
the disk component to $8.3 \times 10^6$. For the bulge and halo
particles we adopt the same particle mass as for the disk particles.
As a consequence the mass ratios between the bulge, halo and disk are
set by having a different number of particles 
used per component (Table~\ref{tb:mass_radius}).


\begin{table*}
\begin{center}
%\rotate
  \caption{Model names and their parameters.  The columns represent,
    1: Model name, 2: Halo scale radius, 3: Halo characteristic
    velocity dispersion, 4: Halo truncation parameter, 5: Halo
    rotation parameter, 6: Disk mass, 7: Disk scale radius, 8: Disk
    scale height, 9: Disk radial velocity dispersion at the center of
    the disk, 10: Bulge scale length, 11: Bulge characteristic
    velocity, 12: Bulge truncation parameter.  In the model names,
    `md' and `mb' indicate disk and bulge masses. `Rd' and `rb'
    indicate the disk and halo scale radii. `s' indicates models with
    halo spin. `Q' indicates the initial $Q$ value. For these, the
    values of model md1mb1 are referred to as 1.  For all models,
    movies of the evolution are available in the online materials.
\label{tb:params}
}
\begin{tabular}{lccccccccccc}
%\tabletypesize{\scriptsize}
%\tablewidth{0pt}
%\startdata 
  \hline
  (1)      &   (2)      &   (3)      &   (4)      &   (5)      &   (6)      &   (7)      &   (8)      &   (9)      &   (10)      &   (11)      &   (12)      \\ 
           &  \multicolumn{4}{l}{Halo} &  \multicolumn{4}{l}{Disk} &  \multicolumn{3}{l}{Bulge} \\
Parameters &  $a_{\rm h}$ & $\sigma_{\rm h}$ & $1-\epsilon_{\rm h}$ & $\alpha_{\rm h}$& $M_{\rm d}$ & $R_{\rm d}$ & $z_{\rm d}$ & $\sigma_{R0}$  & $a_{\rm b}$ & $\sigma_{\rm b}$ & $1-\epsilon_{\rm b}$ \\ 
Model   &  (kpc) & ($\kms$) &  &  & $(10^{10}M_{\odot})$ & (kpc) & (kpc) & ($\kms$)  & (kpc) & $(\kms)$\\
\hline \hline
md1mb1        &11.5 &  340 & 0.8 & 0.5 & $4.9$ & 2.8 & 0.36 & 105  & 0.64 & 300 & 1.0 \\
md1mb1s0.65    &11.5 &  340 & 0.8 & 0.65 & $4.9$ & 2.8 & 0.36 & 105  & 0.64 & 300 & 1.0 \\
md1mb1s0.8    &11.5 &  340 & 0.8 & 0.8 & $4.9$ & 2.8 & 0.36 & 105  & 0.64 & 300 & 1.0 \\
\\
md0.5mb1  & 8.2 &  310 & 0.88 & 0.5 & $2.5$ & 2.8 & 0.36 & 59.2  & 0.64 & 300 & 0.86 \\
md0.4mb1  & 7.6 &  300 & 0.91 & 0.5 & $2.0$ & 2.8 & 0.36 & 49.0  & 0.64 & 300 & 0.84 \\
md0.3mb1  & 7.0 &  287 & 0.92 & 0.5 & $1.5$ & 2.8 & 0.36 & 38.5  & 0.64 & 300 & 0.82 \\
md0.1mb1  & 6.0 &  285 & 0.97 & 0.5 &$0.49$ & 2.8 & 0.36 & 13.5  & 0.64 & 300 & 0.79 \\
\\
md0.5mb0  & 22.0 &  450 & 0.7 & 0.5 & $2.3$ & 2.8 & 0.36 & 62.6  & 0.64 & 500 & 0.86 \\
md0.5mb3  & 7.0 &  270 & 0.8 & 0.5 & $2.3$ & 2.8 & 0.36 & 59.0  & 0.64 & 500 & 0.79 \\
md0.5mb4  & 6.6 &  260 & 0.82 & 0.5 & $2.3$ & 2.8 & 0.36 & 58.3  & 0.64 & 545 & 0.80 \\
md0.5mb4rb3  & 13.5 &  360 & 0.8 & 0.5 & $2.3$ & 2.8 & 0.36 & 57.2  & 1.92 & 380 & 0.99 \\
\\
md1mb1Rd1.5      &  9.0 &  290 & 0.95 & 0.5 & $4.9$ & 4.2 & 0.36 & 74.2  & 0.64 & 300 & 0.85 \\
md0.5mb1Rd1.5     &  7.5 &  290 & 0.91 & 0.5 & $2.5$ & 4.2 & 0.36 & 39.8  & 0.64 & 300 & 0.8 \\
md0.5Rmb1d1.5s &  7.5 &  290 & 0.91 & 0.8 & $2.5$ & 4.2 & 0.36 & 39.8  & 0.64 & 300 & 0.8 \\
\\
md1.5mb5      & 13.0 & 280 & 0.9 & 0.5 & 7.3 &  2.8 & 0.36 & 138 & 1.0 & 550 & 0.8 \\
md1mb10       & 18.0 & 500 & 0.9 & 0.5 & 4.9 & 2.8 & 0.36  & 93.2 & 1.5 & 600 & 1.0 \\
\\
md0.5mb0Q0.5  & 22.0 &  450 & 0.7 & 0.5 & $2.3$ & 2.8 & 0.36 & 26.1  & 0.64 & 500 & 0.86 \\
md0.5mb0Q2.0  & 22.0 &  450 & 0.7 & 0.5 & $2.3$ & 2.8 & 0.36 & 105  & 0.64 & 500 & 0.86 \\
%\enddata
\hline
\end{tabular}
\end{center}
\end{table*}


\begin{table*}
\begin{center}
  \caption{Models: mass, radius, and number of particles per component.
    Column 1: Model name, 2: Disk mass, 3: Bulge mass, 4: Halo mass, 5: Disk outer radius, 6: Bulge outer radius, 7: Halo outer radius, 
8: Toomre's $Q$ value at the reference point ($2.2R_{\rm d}$), 
9: Bulge-to-disk mass ratio ($B/D$), 10: Number of particles for the disk, 11:  Number of particles for the bulge, 12: Number of particles for the halo.
\label{tb:mass_radius}}
\begin{tabular}{lccccccccccc}
%\tabletypesize{\scriptsize}
%\rotate
%\tablewidth{0pt}
%\startdata 
\hline
  (1)      &   (2)      &   (3)      &   (4)      &   (5)      &   (6)      &   (7)      &   (8)      &   (9)      &   (10)      &   (11)      &   (12)      \\ 
Model    & $M_{\rm d}$ & $M_{\rm b}$ & $M_{\rm h}$ & $R_{\rm d, t}$ & $r_{\rm b, t}$ & $r_{\rm h, t}$ & $Q_0$   &  $M_{\rm b}/M_{\rm d}$ & $N_{\rm d}$ & $N_{\rm b}$ & $N_{\rm h}$\\ 
   & ($10^{10}M_{\odot}$) & ($10^{10}M_{\odot}$) & ($10^{10}M_{\odot}$) & (kpc) & (kpc) & (kpc) &  &   &  &  & \\ 
\hline  \hline
md1mb1       & 4.97 &  0.462 & 59.7 & 31.6 & 3.17 & 229 & 1.2  & 0.0930 & 8.3M & 0.77M & 100M \\%8330408 (8M) & 773506 & 100000000\\
md1mb1s0.65       & 4.97 &  0.462 & 59.7 & 31.6 & 3.17 & 229 & 1.2   & 0.0930 & 8.3 M & 0.77M & 100M \\%8330408 (8M) & 773506 & 100000000 \\
md1mb1s0.8       & 4.97 &  0.462 & 59.7 & 31.6 & 3.17 & 229 & 1.2  & 0.0930 & 8.3M & 0.77M & 100M \\%8330408 (8M) & 773506 & 100000000 \\
\\
md0.5mb1      & 2.55 &  0.465 & 43.8 & 31.6 & 2.56 & 232 & 1.2  & 0.182  & 8.3M & 1.5M & 140M \\%8341073 (8M) & 1523280 &  143544884 \\
md0.4mb1      & 2.05 &  0.463 & 41.4 & 31.6 & 2.52 & 261 & 1.2  & 0.226 & 8.3M & 1.9M & 170M \\%8341073 (8M) & 1882153 &  168251277 \\
md0.3mb1      & 1.56 &  0.462 & 36.2 & 31.6 & 2.49 & 247 & 1.2 & 0.296 & 8.3M & 2.5M & 190M \\%8341073 (8M) & 2475127 & 194235851 \\ 
md0.1mb1      & 0.546 & 0.466 & 33.3 & 31.6 & 2.44 & 340 & 1.2  & 0.853 & 8.3M & 7.1M & 510M \\%8330000 (8M) & 7100158 & 508319027 \\
\\
md0.5mb0      & 2.53 &  0.0 & 100.0 & 31.6 & - & 295 & 1.2  & 0.00 & 8.3M & - & 330M \\%8341073 (8M) & - & 330789871 \\
md0.5mb3      & 2.61 &  1.37 & 39.7 & 31.6 & 2.81 & 120 & 1.2 & 0.525 & 8.3M & 4.4M & 130M \\%8341073 (8M) & 4444705 & 128685231 \\
md0.5mb4      & 2.62 &  1.69 & 41.4 & 31.6 & 2.96 & 125 & 1.2 & 0.645 & 8.3M & 5.4M & 130M \\%8341073 (8M) & 5447485 &  133496728 \\
md0.5mb4rb3   & 2.60 &  1.76 & 86.7 & 31.6 & 8.55 & 229 & 1.2 & 0.676 & 8.3M & 5.4M & 130M \\%8341073 (8M) & 5447485 &  133496728 \\
\\ 
md1mb1Rd1.5      & 5.06 &  0.464 & 47.1 & 46.6 & 2.61 & 620 & 1.2 &  0.0916 & 8.3M & 0.77M & 78M \\%8329926 (8M) & 7664161 & 77637255 \\ 
md0.5mb1Rd1.5    & 2.59 &  0.457 & 35.2 & 46.6 & 2.47 & 249 & 1.2 & 0.176 & 8.3M & 1.5M & 110M \\%8341073 (8M) & 1472714 & 113312472 \\ 
md0.5mb1Rd1.5s & 2.59 &  0.457 & 35.2 & 46.6 & 2.47 & 249 & 1.2 & 0.176 & 8.3M & 1.5M & 110M \\%8341073 (8M) & 1472714 & 113312472 \\ 
\\
md1.5mb5      & 7.52  &  2.09  & 104.6 & 31.6 & 3.53 & 269 & 1.2 & 0.279 & 8.3M & 2.3M & 120M \\%8330408 (8M) & 2325101 & 116155914 \\
md1mb10       & 5.17  & 5.22  & 2050  & 31.6 & 11.6 & 264 & 1.2 &  1.0 & 8.3M & 8.4M & 400M \\%8330408 (8M) & 8419789 & 399095779 \\
\\
md0.5mb1Q0.5   & 2.55 &  0.465 & 43.8 & 31.6 & 2.56 & 232 & 0.5 & 0.182 & 8.3M & 1.5M & 140M \\%8341073 (8M) & 1523280 &  143544884 \\
md0.5mb1Q2.0   & 2.55 &  0.465 & 43.8 & 31.6 & 2.56 & 232 & 2.0  & 0.182 & 8.3M & 1.5M & 140M \\%8341073 (8M) & 1523280 &  143544884 \\
%\enddata
\\
\hline
\end{tabular}
\end{center}
\end{table*}





\subsection{The {\tt Bonsai} optimized gravitational $N$-body tree-code}

We adopted the {\tt Bonsai} code for all calculations
\citep{2012JCoPh.231.2825B, 2014hpcn.conf...54B}.  {\tt Bonsai}
implements the classical Barnes \& Hut algorithm
\citep{1986Natur.324..446B} but then optimized for Graphics
Processing Units (GPU) and massively parallel operations. In {\tt Bonsai}
all the compute work, including the tree-construction, takes place on
the GPU which frees up the CPU for administrative tasks. By moving all
the compute work to the GPU there is no need for expensive data copies,
and we take full advantage of the large number of compute cores and
high memory bandwidth that is available on the GPU.  The use of GPUs
allows fast simulations, but we are limited by the relatively small
amount of memory on the GPU. To overcome this limitation we
implemented across-GPU and across-node parallelizations which enable
us to use multiple GPUs in parallel for a single simulation
\citep{2014hpcn.conf...54B}.  Combined with the GPU acceleration, this
parallelization method allows {\tt Bonsai} to scale efficiently from
single GPU systems all the way to large GPU clusters and
supercomputers~\citep{2014hpcn.conf...54B}.  We used the version of
{\tt Bonsai} that incorporates quadrupole expansion of the multipole
moments and the improved Barnes \& Hut opening angle criteria
\citep{2013MNRAS.436.1161I}.  We use a shared time-step of $\sim 0.6$
Myr, a gravitational softening length of 10\,pc and the opening angle
$\theta=0.4$.

Our simulations contain hundreds of millions of particles and
therefore it is critical that the post-processing is handled
efficiently.  We therefore implemented the post-processing methods directly in
{\tt Bonsai} and these are executed while the simulation is progressing. 
This eliminates the
need to reload snapshot data (which can be on the order of a few
terabytes) after the simulation.

The simulations in this work have been run on the Piz Daint supercomputer at
the Swiss National Supercomputing Centre. In this machine each compute node
contains an NVIDIA Tesla K20x GPU and an Intel Xeon E5-2670 CPU. Depending on the number
of particles in the simulation we used between 8 and 512 nodes per simulation.

\begin{table}[t!]
\centering
\caption{Voice conversion \& F0 manipulation results. MOS results are reported with 95\% confidence interval. VDE, and FFE are reported for F0 manipulation while PER, WER, EER, and MOS are reported for voice conversion. Notice, for VDE, and FFE higher is the better since F0 was flattened.}
\label{tab:conv}

\resizebox{1\columnwidth}{!}{
\begin{tabular}{c@{~} | c@{~} | c@{~}c@{~} | c@{~} | c@{~} ||  c@{~}c@{~} }
\toprule
\multirow{2}{*}{Dataset} & \multirow{2}{*}{Method} & \multicolumn{4}{c||}{Voice Conversion} & \multicolumn{2}{c}{F0 Manipulation} \\
\cmidrule{3-8}
& & PER~$\downarrow$ & WER~$\downarrow$ & EER~$\downarrow$ & MOS~$\uparrow$ & VDE~$\uparrow$ & FFE~$\uparrow$ \\
\midrule
VCTK & GT  & 17.16 & 4.32 & 3.25 & 4.11$\pm$0.29 & -- & -- \\
\midrule 
\multirow{3}{*}{LJ}
% & ASR-TTS   & 50.74  & --     & 66.08 & 32.96 & 1.46 \\
& CPC       & 22.22 	& 16.11 		& 0.46 		& 3.57$\pm$0.15 		& \bf 46.68 & \bf 48.71\\
& HuBERT    & \bf 19.09 & \bf 12.23 & \bf 0.31  & \bf 3.71$\pm$0.24 & 39.20 		& 48.42\\
& VQ-VAE    & 40.88 	& 36.96 		& 9.65 		& 2.90$\pm$0.17 		& 10.54 	& 12.08 \\
\midrule 
\multirow{3}{*}{VCTK} 
% & ASR-TTS   & 68.88  & --    & 41.77 & 13.55 & 6.48 \\
& CPC       &  23.58 		& 15.98 		& \bf 4.83  &  3.42 $\pm$ 0.24 		& \bf 25.29 & \bf 26.97 \\
& HuBERT    &  \bf 20.85 	& \bf 12.72 & 6.01  		& \bf  3.58 $\pm$ 0.28 	& 23.46 	& 26.67 \\
& VQ-VAE    & 36.88  		& 29.44 		& 11.56 		& 3.08 $\pm$ 0.34 		& 7.03  	& 7.80  \\
\bottomrule
\end{tabular}}
\vspace{-0.4cm}
\end{table}

\vspace{-0.1cm}
\section{Results}
\vspace{-0.1cm}
Our results cover
% We report results for 
three different settings: (i) speech reconstruction experiments; (ii) speaker conversion and F0 manipulation; (iii) bitrate analysis with subjective tests for speech codec evaluation. We employ two datasets: LJ~\cite{ljspeech17} single speaker dataset and VCTK~\cite{vctk} multi-speaker dataset. All datasets were resampled to a 16kHz sample rate.

% \paragraph*{Implementation Details.}
% \smallskip
\noindent{\bf Implementation Details\quad} 
\label{sec:impl}
We follow the same setup as in~\cite{lakhotia2021generative}. For CPC, we used the model from~\cite{Riviere2020}, which was trained on a ``clean'' 6k hour sub-sample of the LibriLight dataset~\cite{Kahn2020,Riviere2020}. We extract a downsampled representation from an intermediate layer with a 256-dimensional embedding and a hop size of 160 audio samples. For HuBERT we used a \textsc{Base} 12 transformer-layer model trained for two iterations~\cite{hsu2020hubert} on 960 hours of LibriSpeech corpus~\cite{Panayotov2015}. 
% This model encodes every 320 raw audio samples into a 768-dimensional vector. 
This model downsamples the raw audio $\times320$ into a sequence of 768-dimensional vectors. Similarly to~\cite{lakhotia2021generative}, activations were extracted from the sixth layer.

%CPC: We use a dictionary of 100 units, leading to a bitrate of 700bps.
%HuBERT: A dictionary of 100 units is used, leading to a bitrate of 350bps. 
%VQVE: The VQ-VAE discrete code operates at a bitrate of 800bps.
% For both CPC and HuBERT, the k-means algorithm is applied to convert continuous frames to discrete codes, using the LibriSpeech clean-100h~\cite{Panayotov2015} dataset. 
For CPC and HuBERT, the k-means algorithm is trained on LibriSpeech clean-100h~\cite{Panayotov2015} dataset to convert continuous frames to discrete codes. We quantize both learned representations with $K=100$ centroids. Leading to a bitrate of 700bps for CPC and 350bps for HuBERT.

% VQ-VAE
Similarly to CPC models, we trained the VQ-VAE content encoder model on the ``clean'' 6K hours subset from the LibriLight dataset. We use an encoder operating on the raw signal to extract discrete units, similar to~\cite{jukebox}. In addition, ``random restarts'' were performed when the mean usage of a codebook vector fell below a predetermined threshold. Finally, we used HiFiGAN (architecture and objective) as the decoder instead of a simple convolutional decoder, as it improved the overall audio quality. This model encodes the raw audio into a sequence of discrete tokens from 256 possible tokens~\cite{garbacea2019low} with a hop size of 160 raw audio samples. The VQ-VAE discrete code operates at a bitrate of 800bps. We additionally experimented with 100 discrete units for VQ-VAE, however results were the best for 256. This finding is consistent with~\cite{garbacea2019low}.

% verification model
The speaker verification network uses the architecture proposed in~\cite{heigold2016end}. It was trained on the VoxCeleb2~\cite{voxceleb2} dataset, achieving a 7.4\% Equal Error Rate (EER) for speaker verification on the test split of the VoxCeleb1~\cite{Nagrani17} dataset.

% pitch
Only a single F0 representation is considered across all evaluated models, trained on the VCTK dataset.
% The F0 is extracted from the raw audio using YAAPT~\cite{yaapt} algorithm, using a window size of 20ms and a 5ms hop. 
The F0 is extracted from the raw audio using a window size of 20ms and a 5ms hop. 
As a result, the F0 sequence is sampled at 200Hz. 
% We apply the quantization described at Sec.~\ref{sec:method}, using a pitch codebook of $K'=20$ tokens and an encoder that downsamples the pitch by $\times16$. 
The quantization described at Sec.~\ref{sec:method}, is applied using an F0 codebook of $K'=20$ tokens and an encoder that downsamples the signal by $\times16$. Hence, the discrete F0 representation is sampled at 12.5Hz, leading to a bitrate of 65bps. The final bitrate of the evaluated codecs is the sum of the pitch code bitrate with the content code bitrate.

% \paragraph*{Evaluation Metrics}
% \smallskip
\noindent{\bf Evaluation Metrics\quad} 
We consider both subjective and objective evaluation metrics. For subjective tests, we report the Mean Opinion Scores (MOS). In which human evaluators rate the naturalness of audio samples on a scale of 1--5. Each experiment, included 50 randomly selected samples rated by 30 raters. For objective evaluation, we consider: (i) Equal Error Rate~(EER) as an automatic speaker verification metric obtained using a pre-trained speaker verification network. We report EER between test utterances and enrolled speakers; (ii) Voicing Decision Error (VDE)~\cite{nakatani2008method}, which measures the portion of frames with voicing decision error; (iii) F0 Frame Error (FFE)~\cite{chu2009reducing}, measures the percentage of frames that contain a deviation of more than 20\% in pitch value or have a voicing decision error; (iv) Word Error Rate (WER) and Phoneme Error Rate (PER), proxy metrics to the intelligibility of the generated audio. We used a pre-trained ASR network~\cite{baevski2020wav2vec} on both reconstructed and converted samples to calculate both metrics. %To generate target phonemes, the g2p-en~\cite{g2pE2019} Grapheme2Phoneme module was used.

% \vspace{-0.1cm}
% \smallskip
\noindent{\bf Reconstruction \& Conversion}
% \vspace{-0.1cm}
We start by reporting the reconstruction performance. Results are summarized in Table~\ref{tab:recon}. When considering the intelligibility of the reconstructed signal HuBERT reaches the lowest PER and WER scores across all models, where both CPC and HuBERT are superior to VQ-VAE. However, when considering F0 reconstruction VQ-VAE outperforms both HuBERT and CPC by a significant margin. This results are somewhat intuitive, bearing in mind VQ-VAE objective is to fully reconstruct the input signal. In terms of subjective evaluation, all models reach similar MOS scores, with one exception of CPC on LJ. 

%Notice, since the same F0 units are used for each method, this result implies the VQ-VAE units contain some information about the F0 of the signal, enabling better reconstruction. Regarding speaker information, the CPC gets the lowest EER. 

To better evaluate the disentanglement properties of each method with respect to speaker identity and F0, we conducted an additional set of experiments aiming at speaker conversion and F0 manipulation. For voice conversion, we converted each test utterance into five random target speakers. Next, we employed a speaker verification network, which extracts \emph{d-vector} representation to evaluate speaker-converted utterances' similarity to real speaker utterances (low error-rate indicates good conversion), providing measurement to the speaker identity's disentanglement from the evaluated coding method. The error-rate is reported between converted test utterances and enrolled speakers. For the LJ speech single speaker dataset, we converted samples from the VCTK dataset to the single speaker and enrolled all VCTK speakers together with the single speaker. Results are summarized in Table~\ref{tab:conv} (left). Unlike resynthesis results, on voice conversion CPC and HuBERT outperform VQ-VAE on both LJ and VCTK datasets, indicating VQ-VAE contains more information about the speaker in the encoded units, hence producing more artifacts. Notice, this also affects WER, PER, and the overall subjective quality (MOS). 

Next, to evaluate the presence of F0 in the discrete units, we flattened the F0 units before synthesizing the signal and calculated VDE and FFE with respect to the original F0 values. F0 flattening was done by setting the speakers' mean F0 value across all voiced frames. In this experiment, we expected units that contain F0 information to be better at F0 reconstruction over disentangled units. Results are summarized in Table~\ref{tab:conv} (right). Notice VQ-VAE can still reconstruct the F0 almost at the same level as when using the original F0 as conditioning (5.2 vs 7.03, and 5.59 vs 7.8), in contrast to CPC and HuBERT.

\begin{figure}[t!]
\centering
\includegraphics[width=0.65\columnwidth, trim={50 20 70 20}]{figures/codec_2.pdf}
% \caption{MUSHRA subjective listening test results as a function of bitrate per second for various methods. Purple dots denote the baseline methods, and green dots the proposed SSL based method.} 
\caption{MUSHRA subjective quality results as a function of bitrate per second. Purple dots denote the baseline methods, and green dots the proposed SSL based method.} 
\label{fig:codec}
\vspace{-0.5cm}
\end{figure}

% \vspace{-0.1cm}
% \smallskip
\noindent{\bf Speech Codec}
Our final experiment evaluates the obtained speech units as a low bitrate speech codec. 
% Therefore, we evaluate how the performance varies as a function of the number of discrete units. Changing the number of units is equivalent to varying the bitrate of the encoded signal. 
We use a subjective MUSHRA-type listening test~\cite{series2014method} to measure the perceived quality of the proposed speech codec with regard to its bitrate constraints. In MUSHRA evaluations, listeners are presented with a labeled uncompressed signal for reference, a set of test samples to rate, a copy of the uncompressed reference, and a low-quality anchor. Listeners are asked to rate each test utterance and the copy of the uncompressed reference with respect to the labeled reference in a scale of 1-100.

The experiment is performed on the VCTK dataset~\cite{vctk}. For evaluation, we used 20 utterances from 5 speakers. The set of speakers in the test data is disjoint with those in the training data. For this experiment, HuBERT models with 50, 100, and 200 units were trained as described in Sec.~\ref{sec:impl}. For comparison, we included other speech codecs in our evaluation: Opus~\cite{valin2012definition} wideband at 9 kbps VBR, Codec2~\cite{rowe2011codec} at 2.4 kbps and LPCNet~\cite{valin2019real} operating at 1.6 kbps. The LPCNet model was trained from scratch on the VCTK dataset following the experimental setup in~\cite{valin2019real}. The VQ-VAE model employs the HiFiGAN decoder trained on the LibriLight dataset to match the amount of data reported in~\cite{garbacea2019low}. We compressed the anchor sample with Speex~\cite{valin2016speex} at 4 kbps as a low anchor. Fig.~\ref{fig:codec} depicts the results. HuBERT with 50 units reaches the best MUSHRA score while its bitrate is only 365bps, which is significantly lower than the baseline methods.
In this paper, 2D and 3D CNN models were used to generate pelvic sCTs from T1-weighted MR images. Our sCT generation methods were fully automated, requiring no deformable registration or manual segmentation of bone tissues. As shown in Figure~\ref{fig3}, the 2D and 3D CNN models generated high quality sCTs. MAE curves shown in Figure~\ref{fig4} indicated that both models could precisely estimate soft-tissue HU values but had difficulty in reproducing air and high-density bone tissues. 

The MAEs within the body contour across all patients were 40.5 $\pm$ 5.4 HU and 37.6 $\pm$ 5.1 HU for the 2D and 3D models, respectively. The time required for generating a pelvic sCT using our CNN models was about 5.5 s. Our MAE results are comparable to previous studies. Kim $et \ al.$\cite{RN41} presented a voxel-based weighted summation method that produced an MAE of 74.3 $\pm$ 3.9 HU. However, manual contouring of bone tissues required for this method can be tedious and time-consuming. An MAE of 40.5 $\pm$ 8.2 HU was achieved by Dowling $et \ al.$\cite{RN11} using an average MRI-CT atlas from 38 patients. Andreasen $et \ al.$\cite{RN42} reported an MAE of 54 $\pm$ 8 HU using an atlas-based method with pattern recognition, and its prediction time was about 20.8 min. Another random forest model proposed by Andreasen $et \ al.$\cite{RN43} generated sCTs with an MAE of 58 $pm$ 9 HU. A hybrid method suggested by Siversson $et \ al.$ \cite{RN45} obtained an MAE of 36.5 $\pm$ 4.1 HU when ignoring errors introduced by gas cavities. This hybrid method was implemented in the cloud-based commercial software MriPlanner (Spectronic Medical AB, Helsingborg, Sweden), which required 50 to 80 min to generate a sCT.\cite{RN45} The patch-based 3D context-aware generative adversarial network presented by Nie $et \ al.$\cite{RN26} achieved an MAE of 39.0 $\pm$ 4.6 HU. 

Our CNN models reproduced low-density bone as shown in Figure ~\ref{fig4}. The bone-region DSCs were 0.81 $\pm$ 0.04 and 0.82 $\pm$ 0.04 from the 2D and 3D models, respectively. These results are comparable to reported DSC results of 0.79 $\pm$ 0.12\cite{RN10} and 0.91$\pm$0.03{\cite{RN11}}, where the authors compared bone contours manually drawn on the sCT and CT.

It was feasible to train the proposed 3D model with 16 image volumes from scratch. Results of the Wilcoxon signed-rank tests shown in Table~\ref{tab1} demonstrated a statistically significant improvement in overall MAE, bone DSC, and bone precision of the 3D model compared to the 2D model. However, as shown in Figure~\ref{fig4}, the 2D model seemed to perform better in estimating the high-density bone HU values. It should be noted that smaller overall MAEs do not guarantee improved sCT dose calculation and patient positioning performance. While the models performed well, we will continue to acquire more patient data to potentially improve model accuracy and further test model differences.

As this was a retrospective study, the MR image voxel sizes were not matched, resulting in different voxel intensities between images. This may have affected the sCT generation accuracy although we applied intensity normalization. A potential study could examine how voxel size variations affects sCT estimation. 

The proposed 3D model can be implemented on a 12 GB GPU to process volumetric images with dimensions of 256 $\times$ 256 $\times$ 30. More GPU memory would be required to process higher resolution 3D images. Considering the limited access to multi-GPU systems, a 3D architecture with fewer convolutional layers could be considered to deal with higher resolutions. However, the performance could be affected by the reduced parameters and smaller receptive fields of the less complex model. Another approach would be to extract 30-slice sub-volumes from CT and MR images for training the 3D model. The sCT could then be generated by averaging 30-slice sCT sub-volumes produced by the model. 

A number of techniques could be investigated for improving model performance.  Nie $et \ al.$\cite{RN26} showed that introducing an additional adversarial discriminator improved overall sCT quality. The same approach could be adapted in our proposed 2D and 3D CNN models.  Non-rigid deformation\cite{RN44} could also be applied to both CT and MR images in the process of the on-the-fly data augmentation to produce more training pairs. Multiple MR images acquired with different sequences could be fed into models to provide more information for distinguishing different tissues. Multi-GPU systems with more memory would enable the exploration of larger batch sizes for training CNN models, which could reduce variances in gradient estimation and accelerate the training. 




\section{Summary}

We performed a series of galactic disk $N$-body simulations
to investigate the formation and dynamical evolution of spiral arm 
and bar structures in stellar disks which are embedded in live 
dark matter halos.
We adopted a range of initial conditions where the models have similar halo 
rotation curves, but different masses for the disk and bulge components, 
scale lengths, initial $Q$ values, and halo spin parameters.
The results indicate that the bar formation epoch increases exponentially 
as a function of the disk mass fraction with respect to the total mass at the 
reference radius (2.2 times the disk scale length), $f_{\rm d}$.
This relation is a consequence of swing amplification~\citep{1981seng.proc..111T},
which describes the amplification rate of the spiral arm when it transitions from 
leading arm to trailing arm because of the disk's differential rotation.
Swing amplification depends on the properties that characterize the disk, 
Toomre's $Q$, $X$, and $\Gamma$. The growth rate reaches its maximum
for $1<X<2$,  although the position of the peak slightly depends on $Q$ as well as on
$\Gamma$. We computed $X$ for 
$m=2$ ($X_2$), which corresponds to a bar or two-armed spiral, 
for each of our models and found that this value is related to the bar's
formation epoch.

The bar amplitude grows most efficiently when $1<X_2<2$. For models 
with $1<X_2<2$ the bar develops immediately after the start 
of the simulation. As $X_2$ increases beyond $X_2=2$, the growth rate
decreases exponentially. We find that the bar formation epoch increases
exponentially as $X_2$ increases beyond $X_2=2$, in other words
$f_{\rm d}$ decreases. The bar formation epoch exceeds a Hubble time
for $f_{\rm d}\lesssim 0.35$.

Apart from $X$, the growth rate is also influenced by $Q$ where
a larger $Q$ results in a slower growth. This indicates that the bar formation
occurs later for larger values of $Q$. 
Our simulations confirmed this and showed that for the bar ($m=2$) the growth rate
is predicted by swing amplification and becomes visible when it grows beyond a certain amplitude.

Toomre's swing amplification theory further predicts that
the number of spiral arms is related to the mass of the disk, with
massive disks having fewer spiral arms. In addition, larger $\Gamma$
predicts a smaller number of spiral arms.
We confirmed these relations in our simulations. 
The shear rate ($\Gamma$) also affects the pitch angle of spiral
  arms. We further confirmed that our result is consistent with previous
studies.

We found that the disk-to-total mass fraction ($f_{\rm d}$)
and the shear rate ($\Gamma$) are the most important parameters that determine the
morphology of disk galaxies. 
When juxtaposing our models with the Hubble sequence,
the fundamental subdivisions of (barred-)spiral galaxies with 
massive bulges and tightly wound-up spiral arms from S(B)a to S(B)c is 
also be observed as a sequence in our simulations. Where the models 
with either massive bulges or massive disks have more tightly
wound spiral arms. This is because having both a massive disk and bulge results in 
a larger $\Gamma$, i.e., more tightly wound spiral arms. 


Once the
bar is formed it starts to heat the outer parts of the disk.
From this point onwards, 
the self-gravitating spiral arms disappear.
This may be in part caused by the 
lack of gas in our simulations. 
After the bar grows, we no longer discern  
spiral arms in the outer regions of the disk. This could imply
that gas cooling and star formation are required in order to 
maintain spiral structures in barred spiral galaxies for over 
a Hubble time~\citep{1981ApJ...247...77S,1984ApJ...282...61S}.


Our simulations further indicate that non-barred grand-design spirals are
transient structures which immediately evolve into barred
galaxies. Swing amplification teaches us that a massive disk is
required to form two-armed spiral galaxies. This condition, at the
same time, satisfies the short formation time of the bar structure.
Non-barred grand-design spiral galaxies therefore must evolve into barred
galaxies.  We consider that isolated non-barred grand-design spiral galaxies 
are in the process of developing a bar.






%\bsp
\section*{Acknowledgments}

We thank the anonymous referee for the very helpful comments.
This work was supported by JSPS KAKENHI Grant Number 26800108, 
HPCI Strategic Program Field 5 'The origin of matter and the universe,' 
and the Netherlands Research School for Astronomy (NOVA).
Simulations are performed using GPU clusters, HA-PACS at the
University of Tsukuba, Piz Daint at CSCS and Little Green Machine II
(621.016.701). Initial development has been done using the Titan
computer Oak Ridge National Laboratory.  This work was supported by a
grant from the Swiss National Supercomputing Centre (CSCS) under
project ID s548 and s716.  This research used resources of the Oak Ridge
Leadership Computing Facility at the Oak Ridge National Laboratory,
which is supported by the Office of Science of the U.S. Department of
Energy under Contract No. DE-AC05-00OR22725 and by the European
Union's Horizon 2020 research and innovation programme under grant
agreement No 671564 (COMPAT project).
\appendix

%!TEX root = hopfwright.tex

%%%%%%%%%%%%%%%%%%
%%% Appendix A %%%
%%%%%%%%%%%%%%%%%%

\section{Appendix: Operator Norms}
\label{sec:OperatorNorms}
%%
%%\note[JB]{I think we should get rid of this proposition. It is not used anywhere (although there is a vague reference to it in the main text).}
%%
%%\begin{proposition} 
%%	\label{prop:ApproximateSolutionWorks}
%%Define $\bar{x}_{\epsilon}$ as in Definition \ref{def:xepsilon} and let $ x \in \ell^1_0$. 
%%If $ \| x - \bar{x}_\epsilon \| = \cO(\epsilon^2)$ then $F_\epsilon(x) = \cO(\epsilon^2)$. 
%%%%%
%%%%%	 \begin{eqnarray}
%%%%%	 \tilde{\alpha}( \epsilon) &:=& \pi /2 + \tfrac{\epsilon^2}{5} ( \tfrac{3\pi}{2} -1) \nonumber \\
%%%%%	 \tilde{\omega}( \epsilon) &:=& \pi /2 -  \tfrac{\epsilon^2}{5}  \nonumber \\
%%%%%	 \tilde{c}(\epsilon) 	  &:=& \{ \left(\tfrac{2 - i}{5}\right)  \epsilon^2 , 0,0, \dots \} \nonumber
%%%%%	 \end{eqnarray}
%%%%%Then $ \tilde{F}( \tilde{\alpha} (\epsilon) , \tilde{\omega}(\epsilon) , \tilde{c}(\epsilon) ) = \cO(\epsilon^3)$. 
%%\end{proposition}
%%
%%
%%\begin{proof}
%%	It suffices to prove the theorem centering our calculation around $ \{ \pp, \pp, \bar{c}_\epsilon \}$.  
%%	Since $ [ \bar{c}_\epsilon]_{k\geq 3 } =0$ and $ \|\bar{c}_\epsilon \| = \cO(\epsilon)$, then we may expand the function $F$ out  to order $ \cO(\epsilon^2)$ as follows: 
%%	\begin{eqnarray}
%%	\, [F(\alpha,\omega, c)]_1 &=& 
%%			i \omega + \alpha e^{-i \omega} + 
%%	\cO(\epsilon^2)  \\ 
%%	\, [F(\alpha,\omega, c)]_2 &=& 
%%			(  2 i \omega  + \alpha e^{ - 2 i \omega} ) c_2 + 
%%	\epsilon \alpha e^{-i \omega}  +
%%	\cO(\epsilon^2) \\ 
%%	\, [ F(\alpha,\omega, c) ]_{k \geq 3} &=&  
%%			\cO(\epsilon^2)
%%	\end{eqnarray}
%%	When  $ \{\alpha , \omega, c \} = \{ \pp, \pp , \bar{c}_{\epsilon} \}$ then both $0 = 	i \omega + \alpha e^{-i \omega} $ and $ 0 = (  2 i \omega  + \alpha e^{ - 2 i \omega} ) c_2 + 
%%	\epsilon \alpha e^{-i \omega}  $. 
%%	Hence, for any $ \| x - \{\pp,\pp, \bar{c}_{\epsilon}\} \| = \cO(\epsilon^2)$ it follows that $ F(x) = \cO(\epsilon^2)$. 
%%	
%%	
%%\end{proof}

%%%%	THIS IS THE OLD PROOF OF PROPOSITION A.1
%%%%
%%%%\begin{proof}
%%%%	Since $ c_{k\geq 3 } =0$ and $ \|c\| = \cO(\epsilon)$, then we may expand the function $F$ out  to order $ \cO(\epsilon^2)$ as follows: 
%%%%	\begin{eqnarray}
%%%%	\, [F(\alpha,\omega, c)]_1 &=& 
%%%%	i \omega + \alpha e^{-i \omega} + 
%%%%	\cO(\epsilon^2)  \\ 
%%%%	\, [F(\alpha,\omega, c)]_2 &=& 
%%%%	(  2 i \omega  + \alpha e^{ - 2 i \omega} ) c_2 + 
%%%%	\epsilon \alpha e^{-i \omega}  +
%%%%	\cO(\epsilon^2) \\ 
%%%%	\, [ F(\alpha,\omega, c) ]_{k \geq 3} &=&  
%%%%	\cO(\epsilon^2)
%%%%	\end{eqnarray}
%%%%	
%%%%	
%%%%	
%%%%	Hence, if we want to solve $ \tilde{F}(\alpha,\omega, c) = 0 + \cO( \epsilon^3)$, then $c_{k \geq 3} =0$. 
%%%%	To solve for $ \alpha, \omega,$ and $ c_2$, we rescale the first equation by $\epsilon$ and then attempt to solve the following system of equations.
%%%%	\begin{eqnarray}
%%%%	0 &=& 
%%%%	i \omega + \alpha e^{-i \omega} + 
%%%%	\alpha \left(e^{i \omega } + e^{-2 i \omega} \right)  c_2 \\
%%%%	0 &=& 
%%%%	( 2 i \omega  + \alpha e^{ - 2 i \omega} ) c_2 + 
%%%%	\epsilon^2 \alpha e^{-i \omega}  
%%%%	\end{eqnarray}
%%%%	Since we are solving for two real variables ($\alpha$ and $ \omega$) and one complex variable ($c_2$), if the equations are non-degenerate then there should be a unique solution. 
%%%%	We solve this equation by making the change of variables $ \alpha = \pp ( 1 + \alpha_1)$ and $ \omega = \pp ( 1 + \omega_1)$.
%%%%	Dividing through by $\pp$ in both equations and then linearizing in terms of $ \alpha_1, \omega_1 $ and $ c_2$ results in the following system of equations:
%%%%	\begin{eqnarray}
%%%%	0 &=&  -i \alpha_1 +(i - \pp) \omega_1  + (-1 + i) c_2 \\
%%%%	i \epsilon^2  &=&   - i \epsilon^2 \alpha_1 - \pp \epsilon ^2 \omega_1 + ( -1 + 2 i) c_2  
%%%%	\end{eqnarray}
%%%%	Separating this into real and imaginary parts, we obtain a system of four real equations and four real variables. 
%%%%	Solving this matrix equation then results in the desired  approximations. 
%%%%	
%%%%	
%%%%\end{proof}



% THIS PARAGRAPH CAN BE REMOVED TO REDUCE THE LENGTH
%  We evaluate the derivative in $ \tilde{x}_\epsilon = (\pi/2,\pi/2,c_2(\epsilon),0,0,\dots) $. The corrections of order $\epsilon^2$ in the $\alpha$- and $\omega$-component that are incorporated in $x_\epsilon$, see Definition~\ref{def:xepsilon}, are not needed here because we only need ????   then $ DF(x_\epsilon) + \cO(\epsilon^2)$ can be calculated to be:
%  	\begin{eqnarray}
%  	\frac{\partial F}{\partial  \alpha} \left(\tilde{x}_\epsilon \right)
%  	&=&
%  	[- i]_1 + [ -\left(\tfrac{2 - i}{5}\right) \epsilon]_2 + [-i \epsilon]_2 +  cO(\epsilon^2) \\
%  	%
%  	\frac{\partial F}{\partial  \omega}  (\tilde{x}_\epsilon)
%  	&=&
%  	[i - \tfrac{\pi}{2}]_1 +
%  	\epsilon [ (2 + \pi) ( \tfrac{1+ 2 i}{5})- \pp ]_2 + \cO(\epsilon^2) \\
%  	%
%  	\frac{\partial F}{\partial  c}  (\tilde{x}_\epsilon)
%  	&=&
%  	\tfrac{\pi}{2} ( i K^{-1} + U_{\omega_0} + \epsilon L_{\omega_0}) + cO(\epsilon^2)
%  	\end{eqnarray}
%  	Here, $\omega_0 = \omega(0) = \pi/2$.
%  	These equations are used to define $ A = A_0 + \epsilon A_1$, whence $A = F( x_\epsilon) + \cO(\epsilon^2)$.
%  	 By our choice of $\epsilon$ we know that  $ \epsilon\|  A_1 A_0^{-1}\| < $, so we can write the power series expansion of $A^{-1} = A_0^{-1} ( I + epsilon A_1 A_0^{-1})$ as follows
%  	 \[
%  	  A^{-1} = A_0^{-1} \sum_{k=0}^{\infty} \left( - \epsilon A_1 A_0^{-1} \right)^k
%  	 \]
%  	 If we truncate this power series and define $ A^{\dagger } := A_0^{-1} - \epsilon A_{0}^{-1} A_1 A_0^{-1}$, then it follows that $ A^{\dagger} = A^{-1} + \cO(\epsilon^2)$.
%  	 Thus, we have proven that $ A^{\dagger} = [DF(x_\epsilon)]^{-1} + \cO(\epsilon^2)$.
%
% \end{proof}

%
% \begin{proposition}
% 	Fix $ \epsilon \geq 0$ and suppose that  $ | \alpha - \pp| < r_\alpha$ and $| \omega - \pp| < r_\omega$ and  $\frac{1+4 \epsilon}{2} <\frac{\omega }{\alpha }$ and  define
% 	\[
% 	b_* = 2  \frac{\pp - r_{\omega}}{\pp + r_\alpha} -1 - \epsilon  (4/3+\sqrt{2 + 2 r_\omega } )
% 	\]
% 	and
% 	\[
% 	z^{\pm}_* =\frac{b_* \pm \sqrt{(b_*)^2- 4 \epsilon^2 }}{2 } .
% 	\]
% 	If there exists for some $ c \in \ell^1 / \C$ such that $\tilde{F}(\alpha, \omega,c) = 0$, then either $ |c| \leq  z_*^-$ or $ z_*^+ \leq |c| $.
%
% 	\noindent
% 	Additionally, $ \| K^{-1} c \| < 2 (\epsilon^2+ \|c\|^2)/ b_*$.
% 	\label{prop:Cone}
% \end{proposition}
%
%
% \begin{proof}
%
% 	Let us define the linear operator $B: \ell^1 / \C \to \ell^1 / \C$ by
% 	\[
% 	B = i \omega K^{-1} + \alpha U_{\omega} + \alpha \epsilon L_{\omega}.
% 	\]
% 	If  $ \tilde{F}( \alpha, \omega, c) =0$ then it follows that for the equations $\tilde{F}_{k\geq 2}$ we have
% 	\begin{eqnarray}
% 	0 &=& \tilde{F}(\alpha , \omega, c) \\
% 	0 &=& [ \epsilon^2 \alpha e^{- i \omega}]_2 + Bc  + \alpha [ U_{\omega} c ] * c \\
% 	- B c &=& [ \epsilon^2 \alpha e^{- i \omega}]_2 + \alpha [ U_{\omega} c ] * c \\
% 	c &=& - \alpha B^{-1} ( [ \epsilon^2  e^{- i \omega}]_2 + \ [ U_{\omega} c ] * c )
% 	\end{eqnarray}
% 	Taking norms, we obtain the following:
% 	\begin{eqnarray}
% 	|c | & \leq & \alpha \| B^{-1}\| \left( \epsilon^2  + \| [U_\omega c] * c \| \right)  \\
% 	(\alpha \| B^{-1}\|)^{-1} |c | & \leq &  ( \epsilon^2 + |c |^2) \\
% 	0 & \leq & |c|^2 - (\alpha \| B^{-1}\|)^{-1} |c| +  \epsilon^2
% 	\end{eqnarray}
% 	Let us define $ b = ( \alpha \| B^{-1} \|)^{-1}$
% 	The above quadratic has two zeros $z^+$ and $ z^-$ given by
% 	\[
% 	z^{\pm} =\frac{b \pm \sqrt{b^2- 4 \epsilon^2 }}{2 }
% 	\]
% 	These zeros have the property that either $ |c| \leq z^-$ or $ |c| \geq z^+$
% 	We calculate $ \| B^{-1}\|$.
% 	Since $ \frac{1+4 \epsilon}{2} <\frac{\omega }{\alpha }$ then $\| \frac{\alpha}{i \omega} (U_{\omega} + \epsilon L_{\omega})K\| <1$ and we can expand $B^{-1}$ using a geometric series.
% 	First we evaluate $B^{-1}$.
% 	\begin{eqnarray}
% 	B 	&=& i \omega K^{-1} + \alpha U_{\omega} + \alpha \epsilon L_{\omega} \\
% 	&=& i \omega \left[I + \frac{\alpha}{i \omega} (U_{\omega} + \epsilon L_{\omega}) K \right] K^{-1} \\
% 	B^{-1} &=& \frac{1}{i \omega } K \left[I + \frac{\alpha}{i \omega} (U_{\omega} + \epsilon L_{\omega})K \right]^{-1} \\
% 	&=& \frac{1}{i \omega } K  \sum_{n=0}^\infty \left( \frac{ - \alpha}{i \omega} \right)^n [(U_{\omega} + \epsilon L_{\omega})K]^n
% 		\end{eqnarray}
% 		We now calculate $\| B^{-1} \|$.
% 		\begin{eqnarray}
% 	\| B^{-1} \| &\leq &  \frac{\| K\|}{ \omega }   \sum_{n=0}^\infty \left( \frac{ \alpha}{ \omega} \right)^n \|(U_{\omega} + \epsilon L_{\omega}) K\|^n \\
% 	&=& \frac{\| K \|}{\omega  - \alpha \|(U_{\omega} + \epsilon L_{\omega}) K\|} \\
% 	&\leq & \frac{\| K \|}{\omega  - \alpha ( \| K \| + 2 \epsilon \| \sigma^+ K \| + 2 \epsilon \|\sigma^- K\|)} \\
% 	&\leq & \frac{1/2}{\omega  - \alpha (1/2+ 2\epsilon( 1/2 + 1/3) )} \\
% 	&=& \frac{1}{2 \omega  - \alpha (1 + \tfrac{10}{3}\epsilon)}.
% 	\end{eqnarray}
% 	Thereby, we have obtained the inequality $b \geq 2 \tfrac{\omega}{\alpha} - 1 - \tfrac{10}{3} \epsilon $.
%
% 	We can further improve this constant with the following observation.
% 	The norm of $ ( e^{-i \omega } I + U_{\omega}) K$ is concentrated in the first equation.
% 	One calculates that $ | [e^{-i \omega } I + U_{\omega} ]_2 |=  | e^{- i \omega} + e^{-2 i \omega}| = \sqrt{2 - 2 \sin (\omega-\pp) } $. So then $ \|\sigma^+ ( e^{-i \omega } I + U_{\omega}) K\| \leq  \sqrt{2 (1+ r_\omega) } /2$.
% 	Consequently $	\| B^{-1} \| \leq  [ 2 \omega - \alpha ( 1 + \epsilon (4/3+\sqrt{2 + 2 r_\omega } ))]^{-1}$.
% 	We can define a lower approximation to $b$ as follows:
% 	\[
% 	b_* = 2  \frac{\pp - r_{\omega}}{\pp + r_\alpha} -1 - \epsilon  (4/3+\sqrt{2 + 2 r_\omega } )
% 	\]
% 	If we calculate the derivative of $ z^{\pm}$ with respect to $b$, we find that $\frac{\partial }{\partial b} z^+>0 $ and $ \frac{\partial }{\partial b} z^-<0 $.
% 	To minimize $ z^+$ and maximize $z^-$ let us then define approximations to $ z^{\pm}$ as
% 	\[
% 	z^{\pm}_* =\frac{b_* \pm \sqrt{(b_*)^2- 4 \epsilon^2 }}{2 } .
% 	\]
% 	It then follows that either $ |c| \leq  z_*^-$ or $ z_*^+ \leq |c| $.
%
% 	Furthermore, since $\| K^{-1} c \| \leq  \alpha \| K^{-1} B^{-1} \| \, ( \epsilon^2 + \|c \|^2)$ and $ \| K^{-1} B^{-1} \| \leq ( \omega - \alpha \| ( U_\omega + \epsilon L_\omega ) K \|)^{-1}$, then it follows from our calculation of $ b_*$ that $ \| K^{-1} c \| < 2 (\epsilon^2+ \|c\|^2)/b_*$.
% \end{proof}



% Below, we make some definitions and prove some small propositions to assist with future calculations.
We set $\omega_0 = \pp$ and recall that 
\begin{alignat*}{1}
	[U_\omega a]_k & =  e^{-i k\omega} a_k \\
	[U_{\omega_0} a]_k & = (-i)^k a_k \\
	L_{\omega}  & =  \sigma^+( e^{-i\omega}  I + U_{\omega}) + \sigma^-( e^{i\omega} I + U_{\omega})  \\
	L_{\omega_0} & = \sigma^+( -i  I + U_{\omega_0}) + \sigma^-( i I + U_{\omega_0})  .
\end{alignat*}
% For future reference we compute $L_{\omega_0}$ where, as in Definition \ref{def:A}, we take $ \omega_0 = \pp$.
% \begin{eqnarray}
% L_{\omega_0} &=& \sigma^+( e^{- i \pp} I + U_{\omega_0}) + \sigma^-(e^{i \pp} I + U_{\omega_0}) \\
% &=& \sigma^+( -i  I + U_{\omega_0}) + \sigma^-( i I + U_{\omega_0})
% \end{eqnarray}
To more efficiently express the inverse of $ A_{0,*}$ we define an operator $\hat{U}: \ell^1_0 \to \ell^1_0 $ by
%
% \begin{definition}
% Define the map  $ \hat{U} : \ell^1_0 \to \ell^1_0 $ by:
\begin{equation}\label{e:defUhat}
[\hat{U} c]_{k\geq 2} := (1 - i k^{-1}e^{-i k \pi /2} )^{-1} c_k,
\end{equation}
so that  
$ A_{0,*}^{-1}=  \frac{2}{ i \pi } \hat{U} K $.
% \end{definition}
%
%
% \begin{proposition}
% The inverse of $ A_{0,*}$ is given by $A_{0,*}^{-1}=  \frac{2}{ i \pi } \hat{U} K $.
% \end{proposition}

% \begin{proof}
% 	The map $A_{0,*}$ is a diagonal operator and its inverse can be calculated as follows:
% 	\begin{eqnarray*}
% 	A_{0,*}^{-1} &=& \frac{2}{ \pi } (i K^{-1} + U_{\omega_0})^{-1} \\
% 	&=& \frac{2}{  \pi } [  (iK^{-1}) (I -i K U_{\omega_0}) ]^{-1} \\
% 	&=& \frac{2}{ i \pi } (I - i U_{\omega_0}K )^{-1} K   \\
% 	&=& \frac{2}{ i \pi } \hat{U} K
% 	\end{eqnarray*}
% \end{proof}
%

The operator norm of  $Q \in B(\ell^1_0,\ell^1)$ can be expressed using the basis elements $\e_k$ (which have norm $\|\e_k\|=2$):
\begin{equation}\label{e:operatornorm}
  \| Q \| = \frac{1}{2} \sup_{k \geq 2} \|Q \e_k\| .
\end{equation}
Some of the operators in $B(\ell^1_0,\ell^1)$ considered in these appendices restrict naturally to $B(\ell^1_0)$, with the same expression for the norm. For operators in $B(\ell^1)$ a similar expression for the norm holds (the supremum being over $k\geq 1$). We will abuse the notation $\|Q\|$ by not indicating explicitly which of these operator norms is considered; this will always be clear from the context.
%\note[JB]{This is not very clean of course, but I am not inclined to go over the paper and put subscripts everywhere to make all these distinctions.}
\begin{proposition}\label{p:severalnorms}
	The operators $\hat{U}, \hat{U} K, L_{\omega}, A_{0,*}^{-1}   $ and $A_{1,*}$ in $B(\ell^1_0,\ell^1)$  satisfy the bounds
\begin{align*}
\| \hat{U} \| 		=& \tfrac{5}{4} 						&\| A_{0,*}^{-1} \| =& \tfrac{2}{ \pi \sqrt{5}}	\\ 
\| \hat{U} K \| 	=& \tfrac{1}{ \sqrt{5}}	&\| A_{1,*} \| \leq& 2 \pi	 \\
\| L_{\omega} \| \leq& 4
\end{align*}
\end{proposition}
\begin{proof}
		The value $\| \hat{U} \e_k \|$ is maximized when $k=5$, whence  $\| \hat{U} \| = 5/4$. 
		The value $\| \hat{U} K \e_k \|$ is maximized when $k=2$, whence $\| \hat{U} K \| 	= \frac{1}{ \sqrt{5}}$ and $\| A_{0,*}^{-1}\| = \frac{2}{\pi \sqrt{5}} $. 
It follows from the definition of $L_\omega$ and the fact that $U_\omega$ is unitary that $ \| L_{\omega} \| \leq 4$, whereby it follows that  
$ \| A_{1,*} \| = \| \pp L_{\omega_0} \|  \leq 2 \pi$.
\end{proof}

We recall, for any $a\in \ell^1$, the splitting $a=a_1 \e_1 + \tilde{a}$ with $a_1 \in \C$ and $\tilde{a} \in \ell^1_0 $,  and as a tool in the estimates below we  introduce the projections 
\begin{alignat}{1}
	\pi_1 a &= a_1  \in \C \\
	\pi_{\geq 2} a & = \tilde{a}. \label{e:pige2}
\end{alignat}

\begin{proposition}
	\label{prop:A1A0}
	We have for the map $ A_1 A_0^{-1} : \ell^1 \to \ell^1$ that 
%\note[JB]{I don't think we should have a strict inequality here.}
	\begin{equation}
	\label{eq:A1A0}	
	\| A_1 A_0^{-1}\| = \frac{2 \sqrt{10}}{5} \, .
	\end{equation}
%\note[JB]{Perhaps the estimate is now
%$ \| A_1 A_0^{-1}\| = \max\{ \frac{1}{5}\sqrt{\frac{45+5\sqrt{17}}{2}}, \frac{2 \sqrt{10}}{5} \}  = \frac{2 \sqrt{10}}{5}$ ?}
\end{proposition}

\begin{proof}
%%	Expanding $A_1A_{0}^{-1}$  we obtain 
%%\note[JB]{This expansion uses imaginative notation. I propose to remove it.}
%%\begin{eqnarray}
%%A_1 A_0^{-1} &=&  (  A_{1,2} + A_{1,*})(  A_{0,1}^{-1} + A_{0,*}^{-1} ) \\
%%&=&   A_{1,2} A_{0,1}^{-1}  +  A_{1,*}     A_{0,*}^{-1} 
%%\end{eqnarray}
%%\note[JB]{I propose to replace it by the following}
Expanding $A_1A_{0}^{-1}$ we see that it splits into two parts: $A_{1,2} A_{0,1}^{-1}$ and $A_{1,*}     A_{0,*}^{-1}$, which we estimate separately. To be precise
\[
  A_1A_{0}^{-1} a = (i_\C A_{1,2} A_{0,1} i_\C^{-1} \pi_1 a) \e_2 
                    +  A_{1,*} A_{0,*}^{-1} \pi_{\ge 2} a.
\]
First, we calculate the matrix
\[
  A_{1,2} A_{0,1}^{-1}  = 
   \frac{1}{5}
  \left[
  \begin{matrix}
  3 & 2 \\
  -4  & 4 
  \end{matrix} 
  \right] .
\]
Using the identification of $\R^2$ and $\C$, which is an isometry if one uses the $2$-norm on $\R^2$,
this matrix contributes to $A_1 A_0^{-1}$
as an operator mapping the (complex) one-dimensional subspace spanned by~$\e_1$ to the (complex) one-dimensional subspace spanned by~$\e_2$. 
To determine its contribution to the estimate of the norm of $A_1 A_0^{-1}$,
we thus need to determine the $2$-norm of the matrix (as a linear map from $\R^2 \to \R^2$):
\[
  \| A_{1,2} A_{0,1}^{-1} \|  = \frac{1}{5} \sqrt{\frac{45+5\sqrt{17}}{2}}.
\]
%\note[JB]{Explained and improved the bound. Still needs to be reflected in the statement of the Proposition though.}
% We calculate the map $A_{1,2} A_{0,1}^{-1} :\{e_1\}   \to \{ \alpha , \omega \} \to \{e_2\} $
% \begin{eqnarray}
% A_{1,2} A_{0,1}^{-1} 		&=&
% \frac{1}{5}
% \left[
% \begin{matrix}
% -2 & 2-\tfrac{3 \pi}{2} \\
% -4  & 2(2+\pi)
% \end{matrix}
% \right]
% \cdot
% \left[
% \begin{matrix}
% 0 & - \pp \\
% -1  & 1
% \end{matrix}
% \right]^{-1}
% \\
% &=&  \frac{1}{5}
% \left[
% \begin{matrix}
% 3 & 2 \\
% -4  & 4
% \end{matrix}
% \right] \\
% \| A_{1,2} A_{0,1}^{-1} \| &\leq& \frac{8}{5} \label{eq:AppendixCheck}.
% \end{eqnarray}
% \marginpar{JJ: todo - Is the estimate in \ref{eq:AppendixCheck} right/ could it be sharpened?}
Next, we calculate a bound on the map $ A_{1,*}     A_{0,*}^{-1}: \ell^1_0 \to \ell^1$:
\begin{equation}\label{eq:LUK}
  \| A_{1,*}     A_{0,*}^{-1} \| =  \| L_{\omega_0} \hat{U} K \| .
\end{equation}
% \begin{eqnarray}
% A_{1,*}     A_{0,*}^{-1} 	&=&  \frac{\pi}{2} L_{\omega_0} \frac{2}{ i \pi } \hat{U} K  \\
% \| A_{1,*}     A_{0,*}^{-1} \| &= & \| L_{\omega_0} \hat{U} K \|  \label{eq:LUK}
% \end{eqnarray}
To bound \eqref{eq:LUK} we first compute how $L_{\omega_0} K \hat{U}    $ operates on basis elements $\e_k$ for $k\geq 2$: 
\[
L_{\omega_0} K \hat{U}     \e_{k} = \frac{ -i+(-i)^k }{k-i (-i)^{k}}   \e_{k+1}
+
\frac{ i+(-i)^k  }{k-i (-i)^{k}}   \e_{k-1} .
\]
Since the norm of this expression is maximized when $k=2$ and $  \| L_{\omega_0} K \hat{U}    \e_2 \| = \tfrac{4\sqrt{10}}{5}$,
%\note[JB]{Isn't that $\tfrac{4\sqrt{10}}{5}$? (norm below still correct)}
 we have calculated the $B(\ell^1_0,\ell^1)$ operator norm $ \|L_{\omega_0} K \hat{U}      \| = \tfrac{2\sqrt{10}}{5}$. 
%\change[J]{By combining the estimates above and using the triangle inequality, we arrive at the asserted bound.}{ 
As $\|A_1A_{0}^{-1}\|$ is equal to the maximum of  $ \| A_{1,2} A_{0,1}^{-1}\|$ and $\|A_{1,*}     A_{0,*}^{-1}\|$, it follows that 
	$ \| A_1 A_0^{-1}\| = \max\{ \frac{1}{5}\sqrt{\frac{45+5\sqrt{17}}{2}}, \frac{2 \sqrt{10}}{5} \}  = \frac{2 \sqrt{10}}{5}$.
%
% \begin{equation*}
% \| A_1 A_0^{-1}\|  \leq  \|  A_{1,2} A_{0,1}^{-1} 	 \| + \| A_{1,*}     A_{0,*}^{-1} \|  \frac{2}{5} \left(4+\sqrt{10}\right)
% \end{equation}
\end{proof}

\begin{proposition}
		\label{prop:A0A1}
		Define $\overline{A_0^{-1} A_1 } \in \text{\textup{Mat}}((\R^3,\R^3)$ by
	\[
\overline{A_0^{-1} A_1 } :=
\left(
\begin{array}{ccc}
0 & 0 & \tfrac{1}{2}\sqrt{2+\frac{\pi ^2}{2}} \\
0 & 0 & \frac{1}{\sqrt{2}}  \\
\frac{8}{5 \pi } & \frac{2\sqrt{16+8 \pi +5 \pi ^2}}{5 \pi } & \frac{2}{\sqrt{5}} \\
\end{array}
\right)
	\] 
%	\note[J]{Reflected changes in the statement of the proposition.}
	Then $\overline{A_0^{-1} A_1 } $ is an upper bound (as defined in Definition~\ref{def:upperbound}) for $A_0^{-1} A_1 $.
\end{proposition}
\begin{proof}
%%We expand $  A_0^{-1} A_1 $ as follows: 
%%\begin{eqnarray}
%%A_0^{-1} A_1 &=& ( A_{0,1}^{-1} + A_{0,*}^{-1}) ( A_{1,2} + A_{1,*}) \\
%%&=&  A_{0,*}^{-1}  A_{1,2} +  A_{0,1}^{-1} A_{1,*} + A_{0,*}^{-1} A_{1,*}
%%\end{eqnarray}
%%\note[JB]{Although the splitting is morally correct, the notation makes no sense to me. I propose to replace it by what is below.}
We write $x=(\alpha,\omega,c)$.
Let $\pi_{\alpha,\omega}$ be the projection onto $\R^2$, whereas $\pi_c$ is the projection onto $\ell^1_0$. 
Then we can expand $  A_0^{-1} A_1 $ as follows:
\begin{alignat}{1}
  \pi_{\alpha,\omega} A_0^{-1} A_1 x &=   A_{0,1}^{-1}  i_\C^{-1} \pi_{1} A_{1,*} \pi_c x  \label{e:complicated1} \\
  \pi_{c} A_0^{-1} A_1 x  &= 
  A_{0,*}^{-1} (( i_\C A_{1,2} \pi_{\alpha,\omega} x)   \e_2 ) + A_{0,*}^{-1} \pi_{\geq 2} A_{1,*} \pi_c x . \label{e:complicated2}
\end{alignat}
We estimate the three operators that appear separately.

First, we note that the term $A_{0,*}^{-1} (( i_\C A_{1,2} \pi_{\alpha,\omega} x)   \e_2 ) $ in~\eqref{e:complicated2} essentially represents an operator from $\R^2$ to the (complex) one-dimensional subspace spanned by $\e_2$. Using the identification of $\C$ with $\R^2$, this map is represented by the matrix 
\[
  \frac{-2 }{25 \pi }
  \left[
  \begin{matrix}
  1 & -2 \\
  2 & 1
  \end{matrix} 
  \right] 
  \cdot
  \left[
  \begin{matrix}
  -2 & 2-\tfrac{3 \pi}{2} \\
  -4  & 2(2+\pi) 
  \end{matrix} 
  \right] \\
  = \frac{2 }{25 \pi }
  \left[
  \begin{matrix}
  -6 & 6+ 11 \pp \\
  8  & \pi -8
  \end{matrix} 
  \right] .
\]
It then follows that 
%\note[J]{Added factor $2$ and changed $r_\alpha \mapsto |\alpha|$ and $r_\omega \mapsto |\omega|$}
\begin{alignat*}{1}
  \| A_{0,*}^{-1} (( i_\C A_{1,2} \pi_{\alpha,\omega} x)   \e_2 ) \|
&\leq 	\frac{4}{25 \pi} 
	\left(|\alpha| \sqrt{(-6)^2+8^2}   + 
	 |\omega|  \sqrt{( 6+ 11 \pp )^2 + (\pi-8)^2} 
	 \right) \\
	&= \frac{4}{5 \pi} \left( 2 |\alpha| + 
	\frac{\sqrt{16+8 \pi +5 \pi ^2}}{2} |\omega| \right) .
\end{alignat*}

% \begin{eqnarray*}
% A_{0,*}^{-1}  A_{1,2} &=&
% \frac{-2 }{25 \pi }
% \left[
% \begin{matrix}
% 1 & -2 \\
% 2 & 1
% \end{matrix}
% \right]
% \cdot
% \left[
% \begin{matrix}
% -2 & 2-\tfrac{3 \pi}{2} \\
% -4  & 2(2+\pi)
% \end{matrix}
% \right] \\
% &=& \frac{2 }{25 \pi }
% \left[
% \begin{matrix}
% -6 & 6+ 11 \pp \\
% 8  & \pi -8
% \end{matrix}
% \right]
% \end{eqnarray*}
% It then follows that
% \begin{eqnarray*}
% 	A_{0,*}^{-1}  A_{1,2} &\leq&
% 	\frac{2}{25 \pi}
% 	\left(r_\alpha \sqrt{(-6)^2+8^2}   +
% 	 r_\omega  \sqrt{( 6+ 11 \pp )^2 + (\pi-8)^2}
% 	 \right) \\
% 	&=& \frac{2}{5 \pi} \left( 2 r_\alpha +
% 	\frac{\sqrt{16+8 \pi +5 \pi ^2}}{2} r_\omega \right)
% \end{eqnarray*}
%
%
%\[
%A_{0,*}^{-1}  A_{1,2} \leq \frac{2}{5 \pi} \left( 2 r_\alpha + 
%\frac{\sqrt{16+8 \pi +5 \pi ^2}}{2} r_\omega \right) 
%\]
%
%

Next, we note that the term $A_{0,1}^{-1}  i_\C^{-1} \pi_{1} A_{1,*} \pi_c x $
in~\eqref{e:complicated1} essentially represents an operator from the (complex) one-dimensional subspace spanned by $\e_2$ to $\R^2$. Using the identification of $\C$ with $\R^2$, this map is represented by the matrix 
\[
 \pp 
 \left[
 \begin{matrix}
 0 & - \pp \\
 -1  & 1 
 \end{matrix} 
 \right]^{-1}
 \cdot
 \left[
 \begin{matrix}
 -1 & -1 \\
 1 &-1 
 \end{matrix} 
 \right] \\
 =
 \left[
 \begin{matrix}
 1 - \pp & 1 + \pp \\
 1  & 1
 \end{matrix} 
 \right] ,
\]
because $\pi_1 A_{1,*} \e_2 = \pp (i-1)$.
% Noting that $A_{0,1}^{-1}  A_{1,*} :\{ e_2 \} \to \{ \alpha , \omega\} $, we calculate:
% \begin{eqnarray}
% A_{0,1}^{-1}  A_{1,*} &=& A_{0,1}^{-1} [ \pp ( e^{ i \omega_0} +e^{-2 i \omega_0} )] \\
% &=& A_{0,1}^{-1} [ \pp ( i-1 )] \\
% &=& \pp
% \left[
% \begin{matrix}
% 0 & - \pp \\
% -1  & 1
% \end{matrix}
% \right]^{-1}
% \cdot
% \left[
% \begin{matrix}
% -1 & -1 \\
% 1 &-1
% \end{matrix}
% \right] \\
% &=&
% \left[
% \begin{matrix}
% 1 - \pp & 1 + \pp \\
% 1  & 1
% \end{matrix}
% \right]
% \end{eqnarray}
Hence
%\note[J]{Added factor of $\frac{1}{2}$.}
\begin{alignat*}{1}
  | \pi_\alpha  A_0^{-1} A_1 x |  &\leq  \tfrac{1}{2} \sqrt{ 2 + \tfrac{\pi^2}{2}}  \|c\| \\ 
  | \pi_\omega  A_0^{-1} A_1 x |  &\leq  \tfrac{1}{2}  \sqrt{2} \|c\|  . 
\end{alignat*}

% If we convert this into  $ \{ e_2 , \overline{ e_2} \}$ coordinates, then it follows that
% \[
% | \{ \alpha , \omega \} \cdot  A_{0,1}^{-1}  A_{1,*} \cdot c_2 | \leq
% \left\{ \sqrt{ 2 + \tfrac{\pi^2}{2}} , \sqrt{2} \right\}
% \]
%
%
%
 
Finally, note that the term $A_{0,*}^{-1}  \pi_{\geq 2} A_{1,*}$
appearing in~\eqref{e:complicated2} maps $\ell^1_0$ to itself. It can be expressed as
\[
A_{0,*}^{-1}  \pi_{\geq 2} A_{1,*}  = - i K \hat{U} \pi_{\geq 2} L_{\omega_0} .
\]
The operator $K \hat{U} \pi_{\geq 2} L_{\omega_0}$
 acts on basis elements $\{ \e_k\}_{k \geq 2}$ as follows:
\begin{alignat*}{1}
	K \hat{U} \pi_{\geq 2}  L_{\omega_0}  \e_2 &= -\frac{1+i}{4} \e_{3}  \\
	K \hat{U} \pi_{\geq 2}  L_{\omega_0}  \e_{k} &= \frac{-i+(-i)^k}{(k+1)-i (-i)^{k+1}} \e_{k+1} + \frac{i+(-i)^k}{(k-1) - i (-i)^{k-1}} \e_{k-1}
	\qquad\text{for } k \geq 3.
\end{alignat*}
Since $\max_{k\geq 2 } \| K \hat{U} \pi_{\geq 2} L_{\omega_0} \e_k \| = \| K
\hat{U} L_{\omega_0} \e_3 \| = \tfrac{4}{\sqrt{5}}$, 
the operator norm of $A_{0,*}^{-1}  \pi_{\geq 2} A_{1,*}$ is $\tfrac{2}{\sqrt{5}}$.

These three bounds on the three operators appearing in~\eqref{e:complicated1} and~\eqref{e:complicated2} lead to the asserted upper bound.
%
% \begin{eqnarray}
% A_{0,*}^{-1}  A_{1,*}  &=& \left( \frac{2}{i \pi } K \hat{U} \right) \left( \frac{\pi}{2} L_{\omega_0}\right) \\
% &=& - i K \hat{U} L_{\omega_0}
% %&=& - i K \hat{U} \left(  \sigma^+ \left( - i I + U_{\omega_0} \right)  + \sigma^- \left(  i I + U_{\omega_0} \right) \right)
% %
% %\| A_{0,*}^{-1}  A_{1,*} \|  &\leq& \| K \hat{U}   L_{\omega_0} \| \\
% %&\leq& \| K \hat{U} \sigma^+ \left( - i I + U_{\omega_0} \right) \| + \| K \hat{U} \sigma^- \left(  i I + U_{\omega_0} \right) \|\\
% %&\leq & \frac{1}{2 \sqrt{2}}+ \frac{2}{\sqrt{17}} .
% \end{eqnarray}
% Noting that $e_1$ is not in the domain of $\hat{U}$, we compute how $K \hat{U}   L_{\omega_0} $ operates on basis elements $\e_k$.
% \begin{eqnarray}
% 	K \hat{U}   L_{\omega_0}  e_2 &=& -\frac{1+i}{4} e_{3} \\
% 	K \hat{U}   L_{\omega_0}  e_{k\geq 3} &=& \frac{-i+(-i)^k}{(k+1)-i (-i)^{k+1}} e_{k+1} + \frac{i+(-i)^k}{(k-1) - i (-i)^{k-1}} e_{k-1}
% \end{eqnarray}
% Since $\max_{k\geq 2 } \| K \hat{U}   L_{\omega_0}  e_k \|_{\ell^1_0} =  \| K \hat{U}   L_{\omega_0}  e_3 \|_{\ell^1_0} = \tfrac{2}{\sqrt{5}}$, then we have calculated the $\ell^1_0$ operator norm  $ \| K \hat{U}   L_{\omega_0} \| = \tfrac{2}{\sqrt{5}}$.
% Hence, we obtain
% \[
%  A_0^{-1} A_1 \cdot r \leq
% \left(
% \begin{array}{ccc}
% 	0 & 0 & \sqrt{2+\frac{\pi ^2}{2}} \\
% 	0 & 0 & \sqrt{2} \\
% 	\frac{4}{5 \pi } & \frac{\sqrt{16+8 \pi +5 \pi ^2}}{5 \pi } & \frac{2}{
% 		\sqrt{5}}
% \end{array}
% \right) \cdot
% \left(
% \begin{array}{c}
% r_{\alpha } \\
% r_{\omega } \\
% r_c \\
% \end{array}
% \right)
% \]
\end{proof}

%%%%%%%%%%%%%%%%%%%%%%%%%%%%%%%%%%%%%%%%%%%%%%%%%%%%%%%%%%%%%%%%%%%%%%%%


%!TEX root = hopfwright.tex

%%%%%%%%%%%%%%%%%%
%%% Appendix B %%%
%%%%%%%%%%%%%%%%%%

\section{Appendix: Endomorphism on a Compact Domain}
\label{sec:CompactDomain}



In order to construct the Newton-like map $T$ we defined operators $ A =  DF(\bar{x}_\epsilon) + \cO(\epsilon^2)$ and $A^{\dagger} = A^{-1} + \cO(\epsilon^2)$. 
However, as $(\bar{\alpha}_\epsilon,\bar{\omega}_\epsilon,\bar{c}_\epsilon) = (\pp,\pp,\bar{c}_\epsilon) + \cO(\epsilon^2)$,  the map $A$ can be better thought of as an $\cO(\epsilon^2)$ approximation of $DF(\pp,\pp,\bar{c}_\epsilon)$. 
Thus, when working with the map $T$ and considering points $ x \in  B_\epsilon(r,\rho)$ in its domain, we will often have to measure the distances of $ \alpha$ and $ \omega $ from $ \pp$. 
To that end, we define the following variables which will be used throughout the rest of the appendices. 
\begin{definition}
	\label{def:DeltaDef}
For $ \epsilon \geq 0$, and $r_\alpha,r_\omega,r_c >0$ we define 
\begin{alignat*}{2}
	\da^0 	&:= \tfrac{\epsilon^2}{5} ( 3 \pp -1) & \qquad\qquad
	\da 	&:= \da^0 + r_\alpha \\
	\dw^0 &:=  \tfrac{\epsilon^2}{5} &
	\dw &:=  \dw^0 + r_{\omega} \\ 
	\dc^0 &:=  \tfrac{2 \epsilon}{\sqrt{5}} &
	\dc &:=  \dc^0 + r_c . 
	% \\
	% \dt^0  &:= \dw^0 + \tfrac{1}{2} (\dw^0)^2 &
	% \dt  &:= \dw + \tfrac{1}{2} \dw^2 \\
	% \dtt^0  &:= 2 \dw^0 + \tfrac{1}{2} (2\dw^0)^2 &
	% \dtt  &:= 2 \dw + \tfrac{1}{2} (2\dw)^2  .
\end{alignat*}
\end{definition}


% \note[J]{
% 	I believe that we can replace the bounds $\dt$ by $\dw$  and $\dtt$ by $2 \dw$.In short, this follows from the following estimate.
% 	\[
% 	| e^{-i \omega }+i| \leq \int_{\pp}^\omega |\tfrac{\partial}{\partial \omega}  e^{-i \omega} | d\omega \leq  \int_{\pp}^\omega |1| d\omega = |\omega - \pp| .
% 	\]
% 	I have not gone through and done this yet. }
% \note[JB]{I think you are right. I think it also follows from $|e^{-i(\pp+\dw)}+i|^2=|e^{-i\dw}-1|^2 = (\cos \dw -1)^2+(\sin \dw)^2=2(1-\cos\dw) \leq 2 \cdot \frac{1}{2} \dw^2$.}
%
When considering an element $ ( \alpha , \omega, c)$ for our $\cO(\epsilon^2)$ analysis, we are often concerned with the 
 distances $|\alpha - \pp|$, $|\omega - \pp|$ and $ \| c - \bar{c}_\epsilon\|$, each of which is of order $\epsilon^2$.  
To create some  notational consistency in these definitions, $\da^0$ and $\dw^0$ are of order $\epsilon^2$, whereas $\dc^0$ is not capitalized as it is of order $\epsilon$. 
Using these definitions, it follows that for any $\rho>0$ and all  $(\alpha, \omega, c ) \in B_\epsilon(r,\rho)$ we have: 
\begin{alignat*}{1}
| \alpha - \pp | & \leq  \da       \\ 
	 | \omega - \pp| & \leq  \dw   \\
	\|c \| &\leq  \dc  .
	%  \\
	% | e^{- i \omega} + i| &\leq  \dt \\
	% | e^{-2 i \omega } +1| &\leq \dtt  .
\end{alignat*}
In this interpretation the superscript $0$ simply refers to $r=0$, i.e., the center of the ball $(\alpha,\omega,c) = \bx_\epsilon$.

The following elementary lemma will be used frequently in the estimates. 
\begin{lemma}\label{lem:deltatheta}
For all $x\in \R$ we have $|e^{ix}-1| \leq |x|$.
Furthermore, for all $|\omega - \bomega_\epsilon  | \leq r_\omega$  
%\note[JB]{I think this should be $|\omega - \bomega_\epsilon| \leq r_\omega$, no?} \note[J]{Yes, that is correct } 
we have 
$ |e^{- i \omega} + i| \leq  \dw$ and
$ | e^{-2 i \omega } +1| \leq 2 \dw $ .
\end{lemma}
\begin{proof}
We start with
\[
  |e^{ix}-1|^2 = (\cos x -1)^2+(\sin x)^2=2(1-\cos x) \leq 2 \cdot \tfrac{1}{2} x^2 = x^2.
\]
% Let $w = \omega - \pp$. Then $|w| \leq \dw$ and, using the previous inequality,
% \[
% | e^{- i \omega} + i|^2=
% |e^{-i(\pp+w)}+i|^2=|e^{-i w}-1|^2 \leq  w^2 =  \dw^2.
% \]
% \note[J]{To avoid using $w$ and $\omega$ in the same line, I propose we switch $ w \mapsto \theta$, as below. Also the last equality should be an inequality.}

Let $\theta = \omega - \pp$. Then $|\theta| \leq \dw$ and, using the previous inequality,
\[
| e^{- i \omega} + i|^2=
|e^{-i(\pp+\theta)}+i|^2=|e^{-i\theta}-1|^2 \leq  \theta^2 \leq  \dw^2.
\]
The final asserted inequality follows from an analogous argument.
\end{proof}


While the operators $U_\omega$ and $L_\omega$ are not continuous in $ \omega$ on all of $ \ell^1_0$, they are within the compact set $ B_\epsilon(r,\rho)$. 
To denote the derivative of these operators, we  define
\begin{alignat}{1}
	U_{\omega}' &:=  - i K^{-1} U_{\omega} \nonumber \\
	L_{\omega}' &:= - i \sigma^+( e^{- i \omega} I + K^{-1} U_{\omega}) + i \sigma^-(e^{i \omega} I - K^{-1} U_{\omega})  , \label{e:Lomegaprime}
\end{alignat}
and we derive Lipschitz bounds on $U_\omega$ and $L_\omega$ in the following proposition.
 
\begin{proposition}
	\label{prop:OmegaDerivatives}
	For the definitions above, $ \frac{\partial }{\partial  \omega} U_\omega = U_{\omega}' $ and $ \frac{\partial }{\partial  \omega}  L_\omega= L_{\omega}' $. 
	Furthermore,  for any $ (\alpha, \omega,c) \in B_\epsilon(r,\rho)$, we have the norm estimates
	\begin{alignat}{1}
	\| (U_{\omega} - U_{\omega_0} )c \| &\leq   \dw  \rho \nonumber  \\
	\|( L_{\omega} - L_{\omega_0} )c \| &\leq  2  \dw (  \dc +  \rho) .
	\label{e:LomegaLip}
	\end{alignat}
\end{proposition}
% \note[J]{ There was a mistake in the statement of this proposition. I changed the estimate $\| U_{\omega} - U_{\omega_0}  \| $ to $ \| (U_{\omega} - U_{\omega_0} )c \| $. Likewise for $ L_\omega$. }

\begin{proof}
One easily calculates that $ \frac{\partial U_\omega}{\partial  \omega} =  U_{\omega}'$,  whereby
$
	\| (U_{\omega} - U_{\omega_0} )c \| \leq \int_{\omega_0}^\omega \| \tfrac{\partial}{\partial \omega} U_\omega c \|  \leq    \dw  \rho  
$. 
Calculating $ \frac{\partial }{\partial  \omega}  L_{\omega} $, we obtain the following:
\begin{alignat*}{1}
 \frac{\partial }{\partial  \omega}  L_{\omega} 
&=  \frac{\partial }{\partial  \omega} \left[  \sigma^+( e^{- i \omega} I + U_{\omega}) + \sigma^-(e^{i \omega} I + U_{\omega}) \right] \\
&= - i \sigma^+( e^{- i \omega} I + K^{-1} U_{\omega}) + i \sigma^-(e^{i \omega} I - K^{-1} U_{\omega}) ,
\end{alignat*}
thus proving $ \frac{\partial L_\omega}{\partial  \omega} =  L_{\omega}'$,
and 
$\|( L_{\omega} - L_{\omega_0} )c \| \leq  \int_{\omega_0}^\omega \| \tfrac{\partial}{\partial \omega} L_\omega c \|  \leq   \dw ( 2  \dc + 2 \rho)$.
\end{proof}

\begin{proposition}
	Let $\epsilon\geq 0$ and  $r=(r_\alpha,r_\omega,r_c) \in \R^3_+$. 
	For any $ \rho > 0$ the map 
	 $T:B_{\epsilon}(r,\rho) \to \R^2 \times \ell^K_0 $ is well defined. 	
	We define functions 
% \note[J]{New definitions for $C_0$ and $C_1$ as the old ones did not quite match the estimates proven below. }
	\begin{alignat*}{1}
%	C_0 &:=  \frac{2 \epsilon^2}{\pi} 
%	\left[
%		\frac{8}{5},\frac{8}{5\sqrt{5}} \sqrt{\left(1-3 \pi /4 \right)^2+(2+\pi )^2},\frac{5 \pi }{2} 
%	\right]
%	\cdot \overline{A_0^{-1} A_1} \cdot [ \da , \dw , \dc ]^T ,
%%	\\
	% C_0 &:=  \frac{2 \epsilon^2}{\pi}
	% \left[
	% 	\frac{8}{5},\frac{2}{5} \sqrt{16+ 8\pi + 5 \pi^2},\frac{5 \pi }{2}
	% \right]
	% \cdot \overline{A_0^{-1} A_1} \cdot [ \da , \dw , \dc ]^T ,
	% \\
	C_0 &:=  \frac{2 \epsilon^2}{\pi} 
	\left[
	\frac{8}{5},\frac{2}{5} \sqrt{16+ 8\pi + 5 \pi^2},\frac{5 \pi }{2} 
	\right]
	\cdot \overline{A_0^{-1} A_1} \cdot [0,0 , \dc ]^T ,
	\\
%	C_1 &:= \frac{5 }{2 \pi} \left(1 +   \frac{4 \epsilon  }{5} \left(2+\sqrt{5}\right) \right) , \\
	% C_1 &:= \frac{5 }{2 \pi} + 2  \epsilon   \left(2+\sqrt{5}\right)  , \\
	C_1 &:= \frac{5 }{2 \pi} + \frac{\epsilon \sqrt{10}}{\pi}, \\
	C_2 &:= \dw  \left[  (1 + \pp) + \epsilon \pi  \right] , \\
	C_3 &:=  
	\da (2+ \dc) +	2 \dw (1+\pp) 
		+ \epsilon \left[ \pi + 2\da  + 4 \dc \da + \pi \dw \dc  + (\pp + \da ) \dc^2 \right] ,
	\end{alignat*}
where the expression for $C_0$ should be read as a product of a row vector, a $(3 \times 3)$ matrix and a column vector.
Furthermore we define, for any $\epsilon,r_\omega$ such that $C_1 C_2 <1$,
	\begin{equation}
		C(\epsilon,r_\alpha,r_\omega,r_c) := \frac{C_0+ C_1 C_3}{1 - C_1 C_2}
		 \, .
		\label{eq:RhoConstant}
	\end{equation}
	All of the functions $C_0,C_1,C_2,C_3$ and $C$ are nonnegative and monotonically increasing in their arguments $\epsilon$ and~$r$. 
	Furthermore, if  $C_1 C_2 < 1$ and $	C(\epsilon,r_\alpha,r_\omega,r_c) \leq \rho $
	then $\| K^{-1} \pi_c  T( x) \| \leq \rho $
	for $x \in B_{\epsilon}(r,\rho)$. 
	\label{prop:DerivativeEndo}
\end{proposition}

% \marginpar{This proposition is vague about the actual spaces being used, ie. $\ell_1,\ell^K_0$, etc.}

\begin{proof}
	Given their definitions, it is straightforward to check that the functions $C_i$ and $C$ are monotonically increasing in their arguments.  
	To prove the second half of the proposition, we split 
	$K^{-1} \pi_c  T(x)$ into several pieces. 
%\note[JB]{$\pi_c$ and $\pi_{\ge 2}$ added. Jonathan, could you please go through this and check?}
%\note[JB]{This did not work, since we do not have that $x$ is bounded by $[\da,\dw,\dc]^T$. Jonathan: what you probably meant was what I introduce as $\pi_c^0 x$, but could you please check?} 
	We define the projection $\pi_c^0 x = (0,0,\pi_c x)$.
We then obtain
	\begin{alignat*}{1}
	K^{-1} \pi_c  T(x)  &= K^{-1} \pi_c   [ x - A^{\dagger} F(x) ]   \\
	&= K^{-1} \pi_c  [ I \pi_c^0 x -    A^{\dagger} ( A \pi_c^0 x + F(x) - A \pi_c^0 x)]  \\
	&= \epsilon^2 K^{-1} \pi_c (A_0^{-1}A_{1})^2 \pi_c^0 x + K^{-1} \pi_c A^{\dagger} (F(x) - A \pi_c^0 x) \nonumber \\
	&=  \frac{2 \epsilon^2}{i\pi} \hat{U} \pi_{\ge 2} A_1 A_0^{-1}A_{1} \pi_c^0 x +\frac{2 }{i\pi} \hat{U}  \pi_{\ge 2} (I-\epsilon A_1 A_0^{-1}) (F(x) - A\pi_c^0 x)  ,
\end{alignat*}
where we have used that $K^{-1} \pi_c A_0^{-1} = \frac{2}{i\pi} \hat{U} \pi_{\ge 2}$, with the projection $\pi_{\ge 2}$ defined in~\eqref{e:pige2}.
By using $\| \hat{U} \| \leq \frac{5}{4}$, see Proposition~\ref{p:severalnorms}, we obtain the estimate
%\note[J]{Changed $[\da,\dw,\dc ]^T$ to $[0,0,\dc ]^T$}
\begin{equation}
	\| K^{-1} \pi_c T(x) \| \leq   \frac{2 \epsilon^2}{\pi} \overline{\hat{U}\pi_{\ge 2} A_1} \cdot  \overline{ A_0^{-1}A_{1}}  \cdot
	[0,0,\dc ]^T +\frac{5 }{2 \pi} \left(1 + \epsilon \| A_1 A_0^{-1} \| \right) \|F(x) - A\pi_c^0 x \| .
	\label{eq:DerivativeEndo}
\end{equation}
Here the $(1 \times 3)$ row vector $\overline{\hat{U}\pi_{\ge 2} A_1}$ is an upper bound on $\hat{U}\pi_{\ge 2} A_1$ interpreted as a linear operator from $\R^2 \times \ell^1_0$ to $\ell^1_0$, thus extending in a straightforward manner the definition of upper bounds given in  Definition~\ref{def:upperbound}.
	
	
	We have already calculated  an expression for
	 $ \overline{ A_0^{-1}A_{1}}$ in Proposition~\ref{prop:A0A1},  and  $  \| A_1 A_0^{-1}\| =\frac{2\sqrt{10}}{5}$ by Proposition~\ref{prop:A1A0}.  In order to finish the calculation of the right hand side of Equation \eqref{eq:DerivativeEndo}, we need to  estimate  $\| F(x) - A\pi_c^0 x \|$ and $\overline{\hat{U} \pi_{\ge 2} A_1} $. 
	We first calculate a bound on $\hat{U} \pi_{\ge 2} A_1 $. 
	We note that $ \hat{U} \pi_{\ge 2} A_1  =  \hat{U} \e_2 ( i_\C A_{1,2} \pi_{\alpha,\omega})+ \hat{U} \pi_{\ge 2}A_{1,*} \pi_c$.	
As $\|\hat{U} e_2\| = \| \tfrac{4-2i}{5} \e_2\|$,
it follows from the definition of $A_{1,2}$ 
that 
\[
	 \left| i_\C  A_{1,2}
	 \left( \!\!\begin{array}{c}\alpha \\ \omega \end{array} \!\!\right) \right|  
	 \cdot \| \hat{U} \e_2 \| 
	 \leq 
	 \left(\frac{\sqrt{20}}{5} |\alpha| +  \frac{\sqrt{(2-3 \pi/2)^2 +4(2+\pi)^2}}{5} |\omega| \right)  \cdot \frac{4}{\sqrt{5}}.
\]
	To calculate $ \| \hat{U} \pi_{\ge 2} A_{1,*} \|$ we note that $ \| \hat{U}\| \leq \frac{5}{4}$ and $ \|A_{1,*}\| = \pp \| L_{\omega_0} \| \leq 2 \pi$. 
	Hence $ \| \hat{U} \pi_{\ge 2} A_{1,*} \| \leq \frac{5 \pi}{2}$. 
	Combining these results, we obtain  that
%\note[JB]{I think the second, rearranged version, of the root looks ``nicer''. }
%	\[
%	\overline{\hat{U} \pi_{\ge 2}  A_1 } = \left[\frac{8}{5},\frac{8}{5\sqrt{5}} \sqrt{\left(1-3 \pi /4 \right)^2+(2+\pi )^2},\frac{5 \pi }{2} \right].
%	\] 
	\[
	\overline{\hat{U} \pi_{\ge 2}  A_1 } = \left[\frac{8}{5},\frac{2}{5} \sqrt{16 + 8 \pi + 5 \pi^2},\frac{5 \pi }{2} \right].
	\] 
Thereby, it follows from~\eqref{eq:DerivativeEndo} that 
\begin{equation}\label{e:C0C1}
	\| K^{-1} \pi_c T(x) \| \leq C_0 + C_1 \| F(x) - A \pi_c^0 x\|. 
\end{equation}
We now calculate
	\begin{alignat*}{1}
	F(x) - A \pi_c^0 x &= 
	(i \omega + \alpha e^{-i \omega} ) \e_1 + 
	( i \omega K^{-1} + \alpha U_{\omega}) c + 
	\epsilon \alpha e^{-i \omega} \e_2  +
	\alpha \epsilon L_\omega c + 
	\alpha \epsilon [ U_{\omega} c] * c  
	\\ &\qquad 
	- \pp (i K^{-1} + U_{\omega_0} + \epsilon L_{\omega_0} ) c \\
	&= i ( \omega - \pp) K^{-1} c + ( \alpha - \pp) U_{\omega} c +  \pp ( U_{\omega} - U_{\omega_0})c  \nonumber \\
	&\qquad  + \left[i ( \omega - \pp ) + ( \alpha - \pp) e^{-i \omega} + \pp( e^{- i \omega }+ i)\right] \e_1  \nonumber
	\\ 
	&\qquad  +\epsilon  \alpha   e^{-i \omega}  \e_2  
+  ( \alpha- \pp)  \epsilon L_{\omega} c + \pp \epsilon ( L_{\omega} - L_{\omega_0}) c + \alpha \epsilon [ U_{\omega} c ] * c .
	\end{alignat*}
Taking norms and using~\eqref{e:LomegaLip} and Lemma~\ref{lem:deltatheta}, we obtain 
	\begin{alignat*}{1}
	\| F(x) - A \pi_c^0 x\|& \leq  
	 \dw \rho + \da \dc + \pp \dw \rho
    +	2 (\dw + \da + \pp \dw)  
	   \\
	&\qquad + \epsilon \left[ 2(\pp + \da ) + 4 \dc \da + \pi  \dw (  \dc + \rho) + (\pp + \da ) \dc^2 \right]  \\
		&= \dw [ (1+\pp) +   \epsilon \pi ] \rho \nonumber \\ 
	&\qquad +  \da (2 + \dc)
	+	2 \dw (1+\pp) 
	+ \epsilon \left[ \pi + 2\da  + 4 \dc \da + \pi \dw \dc  + (\pp + \da ) \dc^2 \right].  
	\end{alignat*}


	We have now computed all of the necessary constants. Thus $ \| F(x) - A \pi_c^0 x \| \leq C_2 \rho + C_3$, and from~\eqref{e:C0C1}   we obtain 
	\begin{eqnarray*}
	\| K^{-1} \pi_c T(c) \|
	&\leq & C_0 +  C_1 ( C_2  \rho + C_3),
	\end{eqnarray*}
with the constants defined in the statement of the proposition.
We would like to select values of $\rho$ for which 
	\[
	\| K^{-1} \pi_c T(c) \| \leq \rho
	\]
	This is true if  
	$	C_0 +  C_1 ( C_2  \rho + C_3) \leq \rho$, 
	or equivalently 
	\[
	\frac{C_0 + C_1 C_3 }{1 - C_1 C_2} \leq \rho.
	\]
	This proves the theorem.
\end{proof}

%%%%%%%%%%%%%%%%%%
%%% Appendix C %%%
%%%%%%%%%%%%%%%%%%


\section[]{Relation between bulge-to-disk mass ratio and shear rate}

In Section 3.4 we showed that the bulge-to-disk mass ratio ($B/D$) is not 
always a good indicator for the shear rate ($\Gamma$), because $\Gamma$ 
also depends on other parameters such as the disk-mass fraction ($f_{\rm d}$). 
Here, we construct additional initial conditions by sequentially changing some 
parameters in order to investigate their importance. We do not simulate 
these models, but measure $B/D$ and $\Gamma$ in the generated models at $t=0$.
All parameters of these models are summarized in Tables \ref{tb:models_add} and 
 \ref{tb:models_add2}.

In Fig.~\ref{fig:Gamma_BD}, we present the relation between $B/D$ and $\Gamma$
calculated from the additional initial conditions.
If we keep both $f_{\rm d}$ and the bulge scale length ($r_{\rm b}$) constant,
$\Gamma$ monotonically increases as $B/D$ increases (square symbols).
But if we increase $r_{\rm b}$ while keeping $f_{\rm d}$ constant, then $\Gamma$ 
also increases (triangle symbols). 
If we increase $f_{\rm d}$ and keep $B/D$ and $r_{\rm b}$ constant,
$\Gamma$ increases (diamond symbols). 
The halo scale length ($r_{\rm h}$) and scale velocity 
($\sigma_{\rm h}$), on the other hand, barely affect the relation between $B/D$ and $\Gamma$
(circle symbols).


\begin{figure}
\includegraphics[width=\columnwidth]{figures/shear_rate_BD_fd.pdf}
\caption{Relation between bulge-to-disk mass ratio ($B/D$) and shear rate ($\Gamma$). \label{fig:Gamma_BD}}
\end{figure}


\begin{table*}
\begin{center}
%\rotate
\caption{Parameters for additional initial conditions\label{tb:models_add}}
\begin{tabular}{lccccccccccc}
%\tabletypesize{\scriptsize}
%\tablewidth{0pt}
%\startdata 
\hline
           &  \multicolumn{3}{l}{Halo} &  \multicolumn{4}{l}{Disk} &  \multicolumn{3}{l}{Bulge} \\
Parameters &  $a_{\rm h}$ & $\sigma_{\rm h}$ & $1-\epsilon_{\rm h}$ & $M_{\rm d}$ & $R_{\rm d}$ & $z_{\rm d}$ & $\sigma_{R0}$  & $a_{\rm b}$ & $\sigma_{\rm b}$ & $1-\epsilon_{\rm b}$ \\ 
Model   &  (kpc) & ($\kms$) &  &  $(10^{10}M_{\odot})$ & (kpc) & (kpc) & ($\kms$)  & (kpc) & $(\kms)$\\
\hline \hline
Add1 & 8.2 & 350 & 0.9  & 2.45 & 2.8 & 0.36 & 105 & 0.64 & 300 & 1.0  \\ %t13
Add2 & 11.5 & 443 & 0.9  & 2.45 & 2.8 & 0.36 & 105 & 0.65 & 400 & 1.0  \\ %t5
Add3 & 8.2 & 370 & 0.9  & 2.45 & 2.8 & 0.36 & 105 & 0.64 & 500 & 1.0  \\ %t14
Add4 & 10 & 340 & 0.9  & 2.45 & 2.8 & 0.36 & 105 & 0.64 & 550 & 1.0  \\ %t8
Add5 & 8.2 & 295 & 0.9  & 2.45 & 2.8 & 0.36 & 105 & 0.65 & 600 & 1.0  \\ %t7
Add6 & 8.2 & 284 & 0.9  & 2.45 & 2.8 & 0.36 & 105 & 1.3 & 370 & 1.0  \\ %t11
Add7 & 8.2 & 330 & 0.9  & 2.45 & 2.8 & 0.36 & 105 & 0.8 & 380 & 1.0   \\  %t12
Add8 & 8.2 & 330 & 1.0  & 2.45 & 2.8 & 0.36 & 105 & 0.64 & 550 & 1.0  \\ %t10
Add9 & 12 & 400 & 1.0  & 2.45 & 2.8 & 0.36 & 105 & 0.64 & 540 & 1.0  \\ %t9
Add10 & 8.2 & 370 & 0.9  & 1.47 & 2.8 & 0.36 & 105 & 0.64 & 390 & 1.0  \\ %t15
Add11 & 12 & 330 & 0.9  & 2.45 & 2.8 & 0.36 & 105 & 0.64 & 486 & 1.0  \\ %t17
\hline
\end{tabular}
\end{center}
\end{table*}


\begin{table*}
\begin{center}
\caption{Obtained values for additional initial conditions\label{tb:models_add2}}
\begin{tabular}{lccccccccc}
%\tabletypesize{\scriptsize}
%\rotate
%\tablewidth{0pt}
%\startdata 
\hline
Model    & $M_{\rm d}$ & $M_{\rm b}$ & $M_{\rm h}$ & $M_{\rm b}/M_{\rm d}$& $R_{\rm d, t}$ & $r_{\rm b, t}$ & $r_{\rm h, t}$ & $f_{\rm d}$  &  $\Gamma$\\ 
   & ($10^{10}M_{\odot}$) & ($10^{10}M_{\odot}$) & ($10^{10}M_{\odot}$) & (kpc) & (kpc) & (kpc) &  &   &  \\ 
\hline  \hline
Add1 & 2.57 & 0.514 & 56.0 & 0.20 & 31.6 & 3.57 & 284 &  0.346 & 0.682 \\ %test13
Add2 & 2.58 & 1.21 & 137 & 0.47 & 31.6 & 5.32 & 330 & 0.343 & 0.706 \\ %test5
Add3 & 2.69 & 2.03 & 94.6 & 0.75 & 31.6 & 6.65 & 234 &  0.321 & 0.895 \\ %test14
Add4 & 2.59 & 2.74 & 124 & 1.05 & 31.6 & 8.67 & 288 &   0.340 & 1.04 \\ %test8
Add5 & 2.61 & 3.29 & 93.2 & 1.26 & 31.6 & 9.48 & 270 &  0.307 & 1.10 \\ %test7
Add6 & 2.59 & 1.19 & 45.9 & 0.46 & 31.6 & 6.28& 265 &  0.332 & 0.869 \\ %test11
Add7 & 2.58 & 1.11 & 61.3 & 0.43 & 31.6 & 5.31 & 251 & 0.348  & 0.789 \\ %test12
Add8 & 2.61 & 2.62 & 108 & 1.0 & 31.6 & 7.98 & 494 &  0.322 &  0.996\\ %test10
Add9 & 2.59 & 2.64 & 195 & 1.02 & 31.6 & 8.46 & 687 &  0.341 & 1.00 \\  % test9
Add10 & 1.58 & 1.18 & 74.8 & 0.74 & 31.6 & 5.65 & 234 &  0.251  & 0.675 \\ %test15
Add11 & 2.75 & 2.07 & 130 & 0.75 & 31.6 & 8.00 & 324 &  0.401 & 0.992 \\ %test17
\hline
\end{tabular}
\end{center}
\medskip
\end{table*}





\bibliographystyle{mnras}
\bibliography{reference}


\label{lastpage}
\end{document}
