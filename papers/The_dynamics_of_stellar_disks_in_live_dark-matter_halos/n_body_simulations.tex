
\section{$N$-body simulations}

We performed a series of $N$-body simulations of galactic stellar
disks embedded in dark matter halos. In this section, we describe our
choice of parameters and the $N$-body code used for these simulations.


\subsection{Model}
Our models are based on those described in \citet{2008ApJ...679.1239W} and 
\citet{2005ApJ...631..838W}. 
We generated the initial conditions using GalactICS \citep{2005ApJ...631..838W}.
The initial conditions for generating the dark mater halo are taken from the
NFW profile \citep{1997ApJ...490..493N}, which has a density profile following:
\begin{eqnarray}
\rho_{\rm NFW}(r) = \frac{\rho_{\rm h}}{(r/a_{\rm h})(1+r/a_{\rm h})^3},
\end{eqnarray}
and the potential is written as
\begin{eqnarray}
\Phi_{\rm NFW} = -\sigma_{\rm h}^2\frac{\log (1+r/a_{\rm h})}{r/a_{\rm h}}.
\end{eqnarray}
Here the gravitational constant, $G$, is unity, 
$a_{\rm h}$ is the scale radius, $\rho_{\rm h}\equiv\sigma^2/4\pi a_{\rm h}^2$
is the characteristic density, and $\sigma_{\rm h}$ is the characteristic 
velocity dispersion. We adopt $\sigma_{\rm h}=340$ (km\,s$^{-1}$), 
$a_{\rm h}=11.5$ (kpc).
Since the NFW profile is infinite in extent and mass, the 
distribution is truncated by a halo tidal radius using
an energy cutoff $E_{\rm h}\equiv\epsilon_{\rm h}\sigma_{\rm h}^2$,
where $\epsilon _{\rm h}$ is the truncation parameter with $0<\epsilon _{\rm h}<1$.
Setting $\epsilon _{\rm h}=0$ yields a full NFW profile
\citep[see][for details]{2005ApJ...631..838W}. 
We choose the parameters of the dark matter halo such that the resulting
rotation curves have a similar shape.
The choice of parameters is summarized in Table~\ref{tb:params}.

For some models we assume the halo to have net angular momentum. This
is realized by changing the sign of the angular momentum about the
symmetry axis ($J_{z}$). The rotation is parameterized using $\alpha
_{\rm h}$.  For $\alpha _{\rm h}=0.5$, the fraction of halo particles
which have positive or negative $J_{z}$ are the same, in which case the disk
has no net angular momentum. If $\alpha _{\rm h}>0.5$, the halo rotates in the same
direction as the disk.

For the disk component, we adopt an exponential disk for which the surface density
distribution is given by
\begin{eqnarray}
\Sigma (R) = \Sigma_{0} {\rm e}^{-R/R_{\rm d}}.
\end{eqnarray}
The vertical structure is given by ${\rm sech}^2(z/z_{\rm d})$,
where $z_{\rm d}$ is the disk scale height.
The radial velocity dispersion is assumed to follow 
$\sigma_{R}^2(R)=\sigma_{R0}^2\exp(-R/R_{\rm d})$, where 
$\sigma_{R0}$ is the radial velocity dispersion at the disk's center.
Toomre's stability parameter $Q$ 
\citep{1964ApJ...139.1217T,2008gady.book.....B} at
a reference radius (we adopt $2.2R_{\rm d}$), $Q_{0}$, is controlled
by the central velocity dispersion of the disk ($\sigma _{R0}$). 
We tune $\sigma_{R0}$ such that for our standard model (md1mb1) $Q_0=1.2$ .

For model md1mb1 we use as disk mass $M_{\rm d}$=$4.9\times 10^{10}$ $M_{\odot}$,
and the scale length $R_{\rm d}$=2.8 kpc. The disk's truncation radius is set to 30 kpc, the 
scale height $z_{\rm d}$=0.36 kpc and the radial velocity
dispersion at the center of the galaxy to $\sigma_{R0}$=$105$ km\,s$^{-1}$.
The disk is truncated at ($R_{\rm out}$) with a 
radial range for disk truncation ($\delta R$). We adopt 
$R_{\rm out}=30.0$ (kpc) and $\delta R=0.8$ (kpc).

For the bulge we use a Hernquist model~\citep{1990ApJ...356..359H},
but the distribution function is extended with an energy cutoff
parameter ($\epsilon_{\rm b}$) to truncate the profile much in the
same way as we did with the halo model.  The density distribution
and potential of the standard Hernquist model is
\begin{eqnarray}
\rho_{\rm H} = \frac{\rho_{\rm b}}{(r/a_{\rm b})(1+r/a_{\rm b})^3}
\end{eqnarray}
and
\begin{eqnarray}
\Phi_{\rm H} = \frac{\sigma_{\rm b}^2}{1+r/a_{\rm b}}.
\end{eqnarray}
Here $a_{\rm b}$, $\rho_{\rm b}=\sigma_{\rm b}^2/(2\pi a_{\rm b}^2)$, and
$\sigma _{\rm b}$ are the scale length, characteristic density, and
the characteristic velocity of the bulge, respectively.
We set $\sigma_{\rm b}$=300 km\,s$^{-1}$, bulge scale length $a_{\rm b}$=0.64 kpc,
and the truncation parameter ($\epsilon _{\rm b}$=0.0).
This results in a bulge mass of $4.6\times 10^9M_{\odot}$,
which is consistent with the Milky Way model proposed by
\citet{2010ApJ...720L..72S}, and reproduces
the bulge velocity distribution obtained by BRAVA observations~\citep{2012AJ....143...57K}.
We do not assume an initial rotational velocity for the bulge.

For the simulation models we vary the disk mass, bulge mass,
scale length, halo spin, and $Q_0$.  Since the adopted generator for
the galaxies is an irreversible process and due to the randomization
of the selection of particle positions and velocities we cannot
guarantee that the eventual velocity profile is identical to the input
profile, but we confirmed by inspection that they are
indistinguishable.  The initial conditions for each of the models are
summarized in Table \ref{tb:params}.  The mass and tidal radius for
the bulge, disk, and halo as created by the initial condition
generator are given in Table \ref{tb:mass_radius}.

In each of the models we fix  the number of particles used for 
the disk component to $8.3 \times 10^6$. For the bulge and halo
particles we adopt the same particle mass as for the disk particles.
As a consequence the mass ratios between the bulge, halo and disk are
set by having a different number of particles 
used per component (Table~\ref{tb:mass_radius}).


\begin{table*}
\begin{center}
%\rotate
  \caption{Model names and their parameters.  The columns represent,
    1: Model name, 2: Halo scale radius, 3: Halo characteristic
    velocity dispersion, 4: Halo truncation parameter, 5: Halo
    rotation parameter, 6: Disk mass, 7: Disk scale radius, 8: Disk
    scale height, 9: Disk radial velocity dispersion at the center of
    the disk, 10: Bulge scale length, 11: Bulge characteristic
    velocity, 12: Bulge truncation parameter.  In the model names,
    `md' and `mb' indicate disk and bulge masses. `Rd' and `rb'
    indicate the disk and halo scale radii. `s' indicates models with
    halo spin. `Q' indicates the initial $Q$ value. For these, the
    values of model md1mb1 are referred to as 1.  For all models,
    movies of the evolution are available in the online materials.
\label{tb:params}
}
\begin{tabular}{lccccccccccc}
%\tabletypesize{\scriptsize}
%\tablewidth{0pt}
%\startdata 
  \hline
  (1)      &   (2)      &   (3)      &   (4)      &   (5)      &   (6)      &   (7)      &   (8)      &   (9)      &   (10)      &   (11)      &   (12)      \\ 
           &  \multicolumn{4}{l}{Halo} &  \multicolumn{4}{l}{Disk} &  \multicolumn{3}{l}{Bulge} \\
Parameters &  $a_{\rm h}$ & $\sigma_{\rm h}$ & $1-\epsilon_{\rm h}$ & $\alpha_{\rm h}$& $M_{\rm d}$ & $R_{\rm d}$ & $z_{\rm d}$ & $\sigma_{R0}$  & $a_{\rm b}$ & $\sigma_{\rm b}$ & $1-\epsilon_{\rm b}$ \\ 
Model   &  (kpc) & ($\kms$) &  &  & $(10^{10}M_{\odot})$ & (kpc) & (kpc) & ($\kms$)  & (kpc) & $(\kms)$\\
\hline \hline
md1mb1        &11.5 &  340 & 0.8 & 0.5 & $4.9$ & 2.8 & 0.36 & 105  & 0.64 & 300 & 1.0 \\
md1mb1s0.65    &11.5 &  340 & 0.8 & 0.65 & $4.9$ & 2.8 & 0.36 & 105  & 0.64 & 300 & 1.0 \\
md1mb1s0.8    &11.5 &  340 & 0.8 & 0.8 & $4.9$ & 2.8 & 0.36 & 105  & 0.64 & 300 & 1.0 \\
\\
md0.5mb1  & 8.2 &  310 & 0.88 & 0.5 & $2.5$ & 2.8 & 0.36 & 59.2  & 0.64 & 300 & 0.86 \\
md0.4mb1  & 7.6 &  300 & 0.91 & 0.5 & $2.0$ & 2.8 & 0.36 & 49.0  & 0.64 & 300 & 0.84 \\
md0.3mb1  & 7.0 &  287 & 0.92 & 0.5 & $1.5$ & 2.8 & 0.36 & 38.5  & 0.64 & 300 & 0.82 \\
md0.1mb1  & 6.0 &  285 & 0.97 & 0.5 &$0.49$ & 2.8 & 0.36 & 13.5  & 0.64 & 300 & 0.79 \\
\\
md0.5mb0  & 22.0 &  450 & 0.7 & 0.5 & $2.3$ & 2.8 & 0.36 & 62.6  & 0.64 & 500 & 0.86 \\
md0.5mb3  & 7.0 &  270 & 0.8 & 0.5 & $2.3$ & 2.8 & 0.36 & 59.0  & 0.64 & 500 & 0.79 \\
md0.5mb4  & 6.6 &  260 & 0.82 & 0.5 & $2.3$ & 2.8 & 0.36 & 58.3  & 0.64 & 545 & 0.80 \\
md0.5mb4rb3  & 13.5 &  360 & 0.8 & 0.5 & $2.3$ & 2.8 & 0.36 & 57.2  & 1.92 & 380 & 0.99 \\
\\
md1mb1Rd1.5      &  9.0 &  290 & 0.95 & 0.5 & $4.9$ & 4.2 & 0.36 & 74.2  & 0.64 & 300 & 0.85 \\
md0.5mb1Rd1.5     &  7.5 &  290 & 0.91 & 0.5 & $2.5$ & 4.2 & 0.36 & 39.8  & 0.64 & 300 & 0.8 \\
md0.5Rmb1d1.5s &  7.5 &  290 & 0.91 & 0.8 & $2.5$ & 4.2 & 0.36 & 39.8  & 0.64 & 300 & 0.8 \\
\\
md1.5mb5      & 13.0 & 280 & 0.9 & 0.5 & 7.3 &  2.8 & 0.36 & 138 & 1.0 & 550 & 0.8 \\
md1mb10       & 18.0 & 500 & 0.9 & 0.5 & 4.9 & 2.8 & 0.36  & 93.2 & 1.5 & 600 & 1.0 \\
\\
md0.5mb0Q0.5  & 22.0 &  450 & 0.7 & 0.5 & $2.3$ & 2.8 & 0.36 & 26.1  & 0.64 & 500 & 0.86 \\
md0.5mb0Q2.0  & 22.0 &  450 & 0.7 & 0.5 & $2.3$ & 2.8 & 0.36 & 105  & 0.64 & 500 & 0.86 \\
%\enddata
\hline
\end{tabular}
\end{center}
\end{table*}


\begin{table*}
\begin{center}
  \caption{Models: mass, radius, and number of particles per component.
    Column 1: Model name, 2: Disk mass, 3: Bulge mass, 4: Halo mass, 5: Disk outer radius, 6: Bulge outer radius, 7: Halo outer radius, 
8: Toomre's $Q$ value at the reference point ($2.2R_{\rm d}$), 
9: Bulge-to-disk mass ratio ($B/D$), 10: Number of particles for the disk, 11:  Number of particles for the bulge, 12: Number of particles for the halo.
\label{tb:mass_radius}}
\begin{tabular}{lccccccccccc}
%\tabletypesize{\scriptsize}
%\rotate
%\tablewidth{0pt}
%\startdata 
\hline
  (1)      &   (2)      &   (3)      &   (4)      &   (5)      &   (6)      &   (7)      &   (8)      &   (9)      &   (10)      &   (11)      &   (12)      \\ 
Model    & $M_{\rm d}$ & $M_{\rm b}$ & $M_{\rm h}$ & $R_{\rm d, t}$ & $r_{\rm b, t}$ & $r_{\rm h, t}$ & $Q_0$   &  $M_{\rm b}/M_{\rm d}$ & $N_{\rm d}$ & $N_{\rm b}$ & $N_{\rm h}$\\ 
   & ($10^{10}M_{\odot}$) & ($10^{10}M_{\odot}$) & ($10^{10}M_{\odot}$) & (kpc) & (kpc) & (kpc) &  &   &  &  & \\ 
\hline  \hline
md1mb1       & 4.97 &  0.462 & 59.7 & 31.6 & 3.17 & 229 & 1.2  & 0.0930 & 8.3M & 0.77M & 100M \\%8330408 (8M) & 773506 & 100000000\\
md1mb1s0.65       & 4.97 &  0.462 & 59.7 & 31.6 & 3.17 & 229 & 1.2   & 0.0930 & 8.3 M & 0.77M & 100M \\%8330408 (8M) & 773506 & 100000000 \\
md1mb1s0.8       & 4.97 &  0.462 & 59.7 & 31.6 & 3.17 & 229 & 1.2  & 0.0930 & 8.3M & 0.77M & 100M \\%8330408 (8M) & 773506 & 100000000 \\
\\
md0.5mb1      & 2.55 &  0.465 & 43.8 & 31.6 & 2.56 & 232 & 1.2  & 0.182  & 8.3M & 1.5M & 140M \\%8341073 (8M) & 1523280 &  143544884 \\
md0.4mb1      & 2.05 &  0.463 & 41.4 & 31.6 & 2.52 & 261 & 1.2  & 0.226 & 8.3M & 1.9M & 170M \\%8341073 (8M) & 1882153 &  168251277 \\
md0.3mb1      & 1.56 &  0.462 & 36.2 & 31.6 & 2.49 & 247 & 1.2 & 0.296 & 8.3M & 2.5M & 190M \\%8341073 (8M) & 2475127 & 194235851 \\ 
md0.1mb1      & 0.546 & 0.466 & 33.3 & 31.6 & 2.44 & 340 & 1.2  & 0.853 & 8.3M & 7.1M & 510M \\%8330000 (8M) & 7100158 & 508319027 \\
\\
md0.5mb0      & 2.53 &  0.0 & 100.0 & 31.6 & - & 295 & 1.2  & 0.00 & 8.3M & - & 330M \\%8341073 (8M) & - & 330789871 \\
md0.5mb3      & 2.61 &  1.37 & 39.7 & 31.6 & 2.81 & 120 & 1.2 & 0.525 & 8.3M & 4.4M & 130M \\%8341073 (8M) & 4444705 & 128685231 \\
md0.5mb4      & 2.62 &  1.69 & 41.4 & 31.6 & 2.96 & 125 & 1.2 & 0.645 & 8.3M & 5.4M & 130M \\%8341073 (8M) & 5447485 &  133496728 \\
md0.5mb4rb3   & 2.60 &  1.76 & 86.7 & 31.6 & 8.55 & 229 & 1.2 & 0.676 & 8.3M & 5.4M & 130M \\%8341073 (8M) & 5447485 &  133496728 \\
\\ 
md1mb1Rd1.5      & 5.06 &  0.464 & 47.1 & 46.6 & 2.61 & 620 & 1.2 &  0.0916 & 8.3M & 0.77M & 78M \\%8329926 (8M) & 7664161 & 77637255 \\ 
md0.5mb1Rd1.5    & 2.59 &  0.457 & 35.2 & 46.6 & 2.47 & 249 & 1.2 & 0.176 & 8.3M & 1.5M & 110M \\%8341073 (8M) & 1472714 & 113312472 \\ 
md0.5mb1Rd1.5s & 2.59 &  0.457 & 35.2 & 46.6 & 2.47 & 249 & 1.2 & 0.176 & 8.3M & 1.5M & 110M \\%8341073 (8M) & 1472714 & 113312472 \\ 
\\
md1.5mb5      & 7.52  &  2.09  & 104.6 & 31.6 & 3.53 & 269 & 1.2 & 0.279 & 8.3M & 2.3M & 120M \\%8330408 (8M) & 2325101 & 116155914 \\
md1mb10       & 5.17  & 5.22  & 2050  & 31.6 & 11.6 & 264 & 1.2 &  1.0 & 8.3M & 8.4M & 400M \\%8330408 (8M) & 8419789 & 399095779 \\
\\
md0.5mb1Q0.5   & 2.55 &  0.465 & 43.8 & 31.6 & 2.56 & 232 & 0.5 & 0.182 & 8.3M & 1.5M & 140M \\%8341073 (8M) & 1523280 &  143544884 \\
md0.5mb1Q2.0   & 2.55 &  0.465 & 43.8 & 31.6 & 2.56 & 232 & 2.0  & 0.182 & 8.3M & 1.5M & 140M \\%8341073 (8M) & 1523280 &  143544884 \\
%\enddata
\\
\hline
\end{tabular}
\end{center}
\end{table*}





\subsection{The {\tt Bonsai} optimized gravitational $N$-body tree-code}

We adopted the {\tt Bonsai} code for all calculations
\citep{2012JCoPh.231.2825B, 2014hpcn.conf...54B}.  {\tt Bonsai}
implements the classical Barnes \& Hut algorithm
\citep{1986Natur.324..446B} but then optimized for Graphics
Processing Units (GPU) and massively parallel operations. In {\tt Bonsai}
all the compute work, including the tree-construction, takes place on
the GPU which frees up the CPU for administrative tasks. By moving all
the compute work to the GPU there is no need for expensive data copies,
and we take full advantage of the large number of compute cores and
high memory bandwidth that is available on the GPU.  The use of GPUs
allows fast simulations, but we are limited by the relatively small
amount of memory on the GPU. To overcome this limitation we
implemented across-GPU and across-node parallelizations which enable
us to use multiple GPUs in parallel for a single simulation
\citep{2014hpcn.conf...54B}.  Combined with the GPU acceleration, this
parallelization method allows {\tt Bonsai} to scale efficiently from
single GPU systems all the way to large GPU clusters and
supercomputers~\citep{2014hpcn.conf...54B}.  We used the version of
{\tt Bonsai} that incorporates quadrupole expansion of the multipole
moments and the improved Barnes \& Hut opening angle criteria
\citep{2013MNRAS.436.1161I}.  We use a shared time-step of $\sim 0.6$
Myr, a gravitational softening length of 10\,pc and the opening angle
$\theta=0.4$.

Our simulations contain hundreds of millions of particles and
therefore it is critical that the post-processing is handled
efficiently.  We therefore implemented the post-processing methods directly in
{\tt Bonsai} and these are executed while the simulation is progressing. 
This eliminates the
need to reload snapshot data (which can be on the order of a few
terabytes) after the simulation.

The simulations in this work have been run on the Piz Daint supercomputer at
the Swiss National Supercomputing Centre. In this machine each compute node
contains an NVIDIA Tesla K20x GPU and an Intel Xeon E5-2670 CPU. Depending on the number
of particles in the simulation we used between 8 and 512 nodes per simulation.
