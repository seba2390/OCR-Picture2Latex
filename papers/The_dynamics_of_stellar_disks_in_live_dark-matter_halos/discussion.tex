
\section{Discussions}

\subsection{Hubble sequence and galaxy morphology}\label{Sect:Hubble}

We performed $N$-body simulations of disk galaxies
which start in an axisymmetric equilibrium state and which form spiral and 
bar structures. 
Using these simulations we investigated the relation between changes 
in the initial state of the bulge, disk, and halo (initial conditions in the simulations)
and the resulting morphology of the disk galaxies.

Especially the disk-mass fraction ($f_{\rm d}$) and shear rate ($\Gamma$)
are important parameters which influence the morphology of the simulated galaxies.
While $f_{\rm d}$ determines the number of spiral arms and
the bar formation epoch, $\Gamma$ has some influence on the bar formation
epoch and more strongly affects the spiral arms pitch angle.

Observationally, disk galaxies are classified using the
Hubble sequence~\citep{1926ApJ....64..321H}. In the Hubble sequence,
spiral galaxies are classified either as spirals (SA) or barred spirals (SB).
Each of these two branches has sub-types a to d; where the bulge-to-disk 
mass ratio ($B/D$) decreases and the pitch angle increases as we go from a to d~\citep{1961hag..book.....S}. 
In this section,
we discuss how the Hubble sequence can be understood by the disk's initial configuration
and secular evolution.



In Fig.~\ref{fig:Hubble} we present a subset of 
snapshots from our simulation on the Hubble sequence.
In the Hubble sequence, the spirals are more loosely wound and the bulge becomes
fainter when moving from Sa to Sc. This is connected with the results we see in Section 3 where 
galaxies with a massive bulge have more tightly wound spiral arms due to the larger 
shear rate when the disk mass fraction is kept similar. Indeed, selecting
some models, we see that the sequence of spiral galaxies from Sa to Sc 
originates from changes in the initial distribution of the disk, bulge
and dark matter halo. 
However, $B/D$ does not linearly change the shear rate ($\Gamma$) as we
demonstrated in Section 3.4 (see also Appendix C). The shear rate also increases with
the disk mass fraction ($f_{\rm d}$). 
Therefore, $\Gamma$ is a more direct parameter to influence the pitch angle rather than $B/D$.
Of course galaxies with a massive bulge tend to have tightly winding spirals
because they tend to have a larger $\Gamma$.




\begin{figure*}
\includegraphics[width=2.0\columnwidth]{figures/HubbleSequence_v2.pdf}
\caption{The Hubble sequence and de Vaucouleurs classification overlaid with snapshots of our models. 
The red arrows show the models secular evolution.\label{fig:Hubble}}
\end{figure*}

Flocculent galaxies emerge
from models with a small disk to total-mass fraction
(such as in model md0.1mb1). 
Even after 10\,Gyr this model did not form a bar and 
the data in Fig.~\ref{fig:bfe} indicates that
it will take more than a Hubble time before the bar forms.

Once the bar formation criteria is satisfied the spiral galaxies leave the 
spiral sequence and move into the SB sequence. If the barred galaxy started as 
a Sc galaxy then in the early stages it resembles the barred-spiral structures 
as seen in SBc galaxies.
These galaxies continue to evolve towards SBb galaxies. For example, model md0.5mb0 has a strong bar and 
the spirals become more tightly wound in the later phases (see Figs.~\ref{fig:snapshots_5Gyr}
and \ref{fig:snapshots_10Gyr}).


The de Vaucouleurs classification~\citep{1959HDP....53..275D} appears 
when spiral galaxies evolve into barred galaxies. In the models 
where the disk is massive enough to form a bar, but less massive than
the models md0.5mb1 and md1mb1Rd1.5, a ring structure just outside 
of the bar appears after the bar has formed. 
In Fig.~\ref{fig:snapshots_10Gyr} and \ref{fig:snapshots_Rdisk} we see 
that these models still retain some 
spiral structures in the outer parts of the disks.
For models md1mb1 and md1.5mb5, which have a disk mass with
$f_{\rm d}>0.5$, a bar forms directly after the start of the 
simulations (see Fig.~\ref{fig:bfe}). They further form strong s-shaped 
and outer ring structures after the bar fully developed 
(right most of Fig.~\ref{fig:snapshots_10Gyr}). For these models we do not 
observe any clear spiral structure in the outer regions of the disk. 
This may be due to the absence of gas in our simulations \citep{2003MNRAS.344..358B}.


Although we did not take gas into account in our simulations, the fraction of gas is an important factor for the morphology of
disk galaxies.
\citet{1993ApJ...414..474S} suggested that bar formation is prevented if more than 10\% of the disk's mass 
is in the form of gas. \citet{2003MNRAS.344..358B} also 
performed a series of simulations that included gas and concluded that the bar stability
depends on the presence of gas.
We ignore the formation of disks that 
start from gas rich models. Such disk galaxies have been observed at a redshift of $\sim 1$
\citep{2017Natur.543..397G}. The effects of gas, however, must be considered in order to fully understand
the morphology of disk galaxies.
In addition, the galaxies in our study are not positioned in a cosmological framework,
which would be necessary to also take the merging history of disk galaxies into account. 
However, disk galaxies seem to be formed with a relatively quiet merging history 
\citep[e. g., ][]{2002NewA....7..155S, 2003ApJ...591..499A}.
Therefore, our study could be a testbed for studying the origin of disk galaxy morphology. 


\subsection{Non-barred grand-design spirals}

As described in Section~\ref{Sect:bf}, swing amplification theory 
predicts that galaxies with massive disks (large disk-to-halo 
mass fraction) typically develop two spiral arms. This condition at the
same time satisfies the constraints for the rapid formation of a bar.
Both two-armed spirals and bars are structures of $m=2$.
In our models, galaxies with a massive disk often directly
form a bar rather than first forming a two-armed spiral disk. This implies
that $m=2$ structures in galactic disks are mostly bars. 
However, from observations we know that non-barred grand-design spiral
galaxies do actually exist.
One possible cause is that perturbations induced by a companion galaxy leads to 
the formation of such a spiral galaxy. This was tested by \citet{1972ApJ...178..623T} 
who, using simulations, showed that tidal interactions can lead to the 
formation of two spiral arms without a bar. Recently, \citet{2018MNRAS.474.5645P}
also showed this using simulations that included gas.
If accompanying galaxies are indeed the 
driver for the formation of two armed spirals then the 
number of grand-design spirals with companions must exceed the number of 
isolated grand-design galaxies. 
\citet{1979ApJ...233..539K} and \citet{1982MNRAS.201.1021E} observationally 
showed that disk galaxies with companions consist of a larger fraction of grand-design
spirals (0.6--1.0) compared to isolated galaxies (0.2--0.3).
However, not all non-barred grand-design spiral galaxies have companions. 
M\,74, for example, has no apparent companion~\citep{2011MNRAS.414..538K}.
In the following paragraphs we explore the formation of non-barred grand-design 
spirals without a companion. 

In the previous sections we saw that a massive disk leads to $m=2$ structures.
On the other hand, a massive bulge suppresses the formation of a bar, but massive bulges
tend to increase the number of spiral arms (e.g. model md1mb10).
To create a non-barred grand-design spiral we therefore setup a new model, md1.5mb5, which 
has the largest disk mass-fraction of all our models ($f_{\rm d}\sim 0.6$),  and a 
moderately massive bulge, $B/D\sim 0.3$. This model is expected to form a bar.
In Fig.~\ref{fig:snapshots_md1.5} we present the initial rotation curve (left panel) and the
density snapshots (right panels) for this model. 
At an age of $t=1.25$\,Gyr this model shows structure comparable to that observed in non-barred grand-design spirals, 
but only in the short time before the bar is formed. 
We conclude that non-barred grand-design spirals can form without companion, but that the structure is short-lived and disappears as soon as the bar forms. 
In our simulations we ignore the presence of gas, which may change the results.
\citet{2003MNRAS.344..358B}, however, performed a series of simulations 
that included gas dynamics, and they also concluded that a companion is necessary for non-barred grand-design spirals
\citep[see also a review by ][]{2014PASA...31...35D}.

\begin{figure*}
\includegraphics[width=55mm]{figures/rotation_curve_md1.5_mb5.pdf}\includegraphics[width=42mm]{figures/md1.5_mb5_110M_0128c.pdf}\includegraphics[width=42mm]{figures/md1.5_mb5_110M_0256c.pdf}\\
\caption{Initial rotation curves (left) and density snapshots for model md1.5mb5 at $t=1.25$ Gyr (middle) and $t=2.5$ Gyr (right).
    The gray dashed curve in the left panel is the same as the one in Fig.~\ref{fig:snapshots_mb10}.
\label{fig:snapshots_md1.5}}
\end{figure*}




