
\section[]{The effects of other parameters}

We discussed the effect of the bulge and disk masses on the
development of bars and spiral arms in the main text. Here we briefly
summarize the effects of the other parameters we investigated.


\subsection{The halo spin}

The spin of the halo is known to be an important parameter that 
affects the bar's secular evolution. 
\citet{2014ApJ...783L..18L} showed that a co-rotating disk and halo 
speed up the bar formation, but decrease its final length. This 
is due to the angular momentum transfer between the disk and halo.
If the halo does not spin it absorbs the bar's angular momentum, 
which slows down the bar and increases its length. 
A co-rotating halo, however, returns angular momentum to the disk instead of 
just absorbing it. 
This stabilizes the angular momentum transfer, and the bar evolution ceases.

We setup a few models, based on model md1mb1, but now with a rotating halo. 
In order to give spin to the halo we change the sign of the angular momentum $z$ component, $L_{\rm z}$.
For models md1mb1s0.65 and md1mb1s0.8, 65 and 85\,\% of the halo particles are rotating in the same 
direction as the disk. For models without rotation, this value is 50\,\%. 

To compare our results with previous studies, we measure the spin 
parameter~\citet{1969ApJ...155..393P,1971A&A....11..377P}:
\begin{eqnarray}
\lambda = \frac{J|E|^{1/2}}{GM_{\rm h}^{5/2}},
\end{eqnarray}
where $J$ is the magnitude of the angular momentum vector, and $E$ is the total 
energy.
In our models, $\alpha_{\rm h}=0.65$ (0.8) correspond 
to $\lambda\sim0.03$ (0.06).


In Fig.~\ref{fig:snapshots_spin_b} we present the effect that the halo spin
has on models md1mb1s0.65 and md1mb1s0.8. 
The results indicate that  
the bar is shorter for the models with a stronger halo spin.

In Fig.~\ref{fig:A2_max_spin} we show the length and maximum amplitude of
the resulting bars.
These results are consistent with previous results which show that
the length of the bar and its amplitude decay when the halo spin increases.
However, in contrast to~\citet{2013MNRAS.434.1287S} and \citet{2014ApJ...783L..18L} ,
we find that the epoch of bar formation in our models is similar, 
whereas a faster formation was expected based on the larger halo spin. 
In order to rule out the effect of run-to-run variations~\citep{2009MNRAS.398.1279S},
we performed four additional simulations for each of models md1mb1, md1mb1s0.65 and md1mb1s0.8.
For the bar formation epochs we calculated the average and standard deviation. 

The average bar formation-epoch is $0.674 \pm 0.053$,
$0.691\pm 0.083$, and $0.610\pm 0.069$\,Gyr for models md1mb1, md1mb1s0.65 and md1mb1s0.8, respectively.
This may be caused by the relatively early bar formation (within $\sim0.8$\,Gyr)
compared to the previous
studies; 1--2\,Gyr for \citet{2014ApJ...783L..18L} and 3--4\,Gyr for
\citet{2013MNRAS.434.1287S}.
Indeed, in~\citet{2014ApJ...783L..18L} the bar formation epoch starts slightly earlier when 
a moderate spin parameter ($\lambda=0.045$ and 0.06) is introduced. The
dependence of the bar formation-epoch on the halo spin is even clearer in
\citet{2013MNRAS.434.1287S}, where the formation time is longer
than in~\citet{2014ApJ...783L..18L}.
We therefore argue that the rapid bar formation in our models may hide the sequential delay
of the bar formation as caused by the halo spin.

%
\begin{figure}
\begin{center}
  \includegraphics[width=40mm]{figures/md1_mb1_a0.65_100M_1024c.pdf}
  \includegraphics[width=40mm]{figures/md1_mb1_a0.8_100M_1024c.pdf}\\
    \caption{Snapshots for models md1mb1s0.65 (left) and md1mb1s0.8 (right), which are the same as model md1mb1 (Fig.~\ref{fig:snapshots_10Gyr},far most right panel) but now including halo spin.}\label{fig:snapshots_spin_b}
\end{center}
\end{figure}



\begin{figure*}
\includegraphics[width=\columnwidth]{figures/mode_A2max_spin.pdf}\includegraphics[width=\columnwidth]{figures/mode_Dbar_mb1s.pdf}
\caption{Time evolution of the maximum amplitude for $m=2$ (left) and the bar length (averaged for every $\sim 0.1$\,Gyr) for models md1mb1, md1mb1s0.65, and md1mb1s0.8. For each model, we performed four simulations changing the random seed (varying positions and velocities of the particles) when generating the initial realizations.
\label{fig:A2_max_spin}}
\end{figure*}

In addition to the bar forming models above, we also added halo spin to a model that 
shows no bar formation within 10\, Gyr. This model, md0.5Rd1.5s, is based on md0.5Rd1.5
but now with a halo spin of 0.8. 
In Fig.~\ref{fig:snapshots_spin_sp}, we present the snapshots of the above models at $t=10$\,Gyr. 
In contrast to the barred galaxies, their spiral structures look quite similar. 
To quantitatively compare the spiral amplitudes we use the total amplitude of the spiral arms 
given by $\sum ^{10}_{m=1} |A_m|^2$, where $A_m$ is the Fourier amplitude (Eq.~\ref{eq:Fourier}).
Instead of the bar amplitude, we measured the spirals total amplitude at 
$2.2R_{\rm d}$ and at $4.5R_{\rm d}$ (for this model 9.5 and 19.5\,kpc, respectively), 
the results are shown in Fig.~\ref{fig:mode_spin_sp}. 
The evolution of the spiral amplitudes are quite similar for both models, 
just like the pitch angle  $24^{\circ}$--$29^{\circ}$ (with)
$24^{\circ}$--$26^{\circ}$ (without halo spin) and the number of 
spiral arms $m=7$--8 for $R=10$--14\,kpc (see Table~\ref{tb:pitch_angle}).


In addition, in Fig.~\ref{fig:AM} we investigate the angular-momentum flow for 
the disk and halo as a function of time and cylindrical radius.
Following \citet{2014ApJ...783L..18L} and \citet{2009ApJ...707..218V},
we measure the change in angular momentum of the $z$-component at
every $\sim 10$\,Myr.
For the halos (top panels) there is no continuous angular
momentum transfer from the disk to the halo, but we only discern random variations
in the angular momentum. These fluctuations look stronger at outer
radii, but this is because the angular momentum changes are normalized by
the disk' angular momentum, which is smaller in the outer regions.

The angular momentum of the disks vary with time (see the red and blue
stripes in the bottom panels), but overall the disk loses only 1.9\,\%
of its initial angular momentum for models with spin and 1.7\,\% for
models without.
The amplitude of the stripes for the disks roughly corresponds to the
amplitude of the spiral pattern. In Fig.~\ref{fig:amplitude_ev}, we show the
total power as a function of cylindrical radius and time for models
md0.5Rd1.5 (left) and md0.5Rd1.5s (right).
From this we conclude that for spiral arms the angular momentum transfer between the disk and 
the halo is not efficient.
On the other hand, for barred galaxies the angular momentum flow 
from the disk to the halo is considerably smaller for models with 
a larger halo spin \citep[see Fig. 3 in][]{2014ApJ...783L..18L}.


\begin{figure}
\includegraphics[width=40mm]{figures/md0.5_Rd1.5_110M_1024c.pdf}\includegraphics[width=40mm]{figures/md0.5_Rd1.5_a0.8_110M_1024c.pdf}\\
\caption{Snapshots for models md0.5Rd1.5 (left) and md0.5Rd1.5s (right). \label{fig:snapshots_spin_sp}}
\end{figure}

\begin{figure*}
  \includegraphics[width=\columnwidth]{figures/mode_rot_spiral_R10.pdf}
  \includegraphics[width=\columnwidth]{figures/mode_rot_spiral_R19.pdf}\\
\caption{Total power for models md0.5Rd1.5 and md0.5Rd1.5s at $R=9.5$ kpc (left) and 19.5 kpc (right). \label{fig:mode_spin_sp}}
\end{figure*}

\begin{figure*}
  \includegraphics[width=\columnwidth]{figures/AM_evolution_halo_md0.5mb1Rd1.5.pdf}\includegraphics[width=\columnwidth]{figures/AM_evolution_halo_md0.5mb1Rd1.5s.pdf}\\
  \includegraphics[width=\columnwidth]{figures/AM_evolution_disk_md0.5mb1Rd1.5.pdf}\includegraphics[width=\columnwidth]{figures/AM_evolution_disk_md0.5mb1Rd1.5s.pdf}\\
\caption{Angular momentum flow of the halo (top) and the disk (bottom) as a function of cylindrical radius and time for models md0.5mb1Rd1.5 (left) and md0.5mb1Rd1.5s (right). The angular momentum flow is calculated from the angular momentum's change in the $z$-component for every $\sim 10$\,Myr. The value (color) is scaled to the initial angular momentum of the disk at each radius for both the disks and halos.}
\label{fig:AM}
\end{figure*}


\begin{figure*}
  \includegraphics[width=\columnwidth]{figures/amplitude_evolution_md0.5mb1Rd1.5.pdf}
  \includegraphics[width=\columnwidth]{figures/amplitude_evolution_md0.5mb1Rd1.5s.pdf}\\
\caption{Total power as a function of cylindrical radius and time for models md0.5Rd1.5 (left) and md0.5Rd1.5s (right).}\label{fig:amplitude_ev}
\end{figure*}



\subsection{Initial Q value}

To verify the expectation that the initial value of Toomre's $Q$ parameter 
($Q_0$) influences the bar and spiral structure, we created a set of models in 
which we varied this parameter. 

The models are based on md0.5mb0, with one having an initially unstable disk
(md0.5mb0Q0.5) and 
the other having a large $Q_0$, in which no spiral arms develop (md0.5mb0Q2.0).
The time evolution of the bar's amplitude and length is presented in
Fig.~\ref{fig:A2_max_Q} 
and the surface densities are shown in Fig.~\ref{fig:snapshots_Q}.
For md0.5mb0Q2.0 there is no sign of spiral or bar structure until $\sim 5$\,Gyr, but 
a bar develops shortly after that (left panel of Fig.~\ref{fig:A2_max_Q}).
This matches with the 
expectation that $Q_0$ influences the bar formation epoch,
the smaller the $Q_0$ value the faster the bar forms.
The peak amplitude just after the bar formation is higher for
the larger $Q_0$, but the final amplitude is similar
(see the left panel of Fig.~\ref{fig:A2_max_Q}).
We also confirmed that the final bar length does not depend on $Q_0$
(see the right panel of Fig.~\ref{fig:A2_max_Q}). 
However, the radius that gives the maximum amplitude is different for 
the models with a large or a small value of $Q$.
The radius for $A_{\rm 2, max}$ is 2.6 and 4.9\,kpc for models with $Q_0=0.5$
and $2.0$, respectively. This result is qualitatively consistent with 
\citet{2012PASJ...64....5H} where an initially colder disk forms
a weaker and more compact bar due to the smaller velocity dispersion of the disk
(although they stopped their simulation just after the first amplitude peak).

This further proves (as discussed in Section~3.3) that the growth 
rate of swing amplification governs the bar formation timescale.
The growth rate decreases 
as $Q$ increases \citep{1981seng.proc..111T} which is confirmed by our simulations. 
With $Q_0=2.0$, the disk is initially stable and hence the spiral structure has 
to be induced by the bar. These ring-like spiral arms are sometimes seen in SB0--SBa 
galaxies such as NGC\,5101 \citep{2011ApJS..197...21H}.


\begin{figure*}
\includegraphics[width=\columnwidth]{figures/mode_A2max_Q.pdf}\includegraphics[width=\columnwidth]{figures/mode_Dbar_Q.pdf}
\caption{Same as Fig.~\ref{fig:A2_max_mdisk}, but for models md0.5mb0Q0.5 and md0.5mb0Q2.0.
\label{fig:A2_max_Q}}
\end{figure*}


\begin{figure}
\begin{center}
\includegraphics[width=40mm]{figures/md0.5_mb0_Q0.5_1536c.pdf}\includegraphics[width=40mm]{figures/md0.5_mb0_Q2.0_1536c.pdf}\\
    \caption{Snapshots for models md0.5mb0Q0.5 (left) and md0.5mb0Q2.0 (right).\label{fig:snapshots_Q}}
\end{center}
\end{figure}

\subsection{Disk scale length}

We further examine models md1mb1Rd1.5 and md0.5mb1Rd1.5, which
have a larger disk length scale. For these models the total disk mass is the same 
as that of models md1mb1 and md0.5mb1, but the disk scale length 
is larger. The changed disk scale length results in different rotation 
curves (see Fig.~\ref{fig:snapshots_Rdisk}). Given  Eq.~\ref{eq:mX} we expect 
that this leads to fewer spiral arms. The top views of these models are presented in
Fig.~\ref{fig:snapshots_Rdisk} (right panels) and the evolution of 
the bar's amplitude and length in Fig.~\ref{fig:A2_max_Rd}. 
The bar formation epoch of model md1mb1Rd1.5 (2\,Gyr) is
later than that of model mdmb1 (1\,Gyr). Model md0.5mb1Rd1.5 did not form 
a bar within 10\,Gyr, although model md0.5mb1 formed a bar at $\sim6$\,Gyr.
The difference
between these models is that the disk mass fraction ($f_{\rm d}$) for model
md1mb1R1.5 and md0.5mb1R1.5 is smaller than those for model md1mb1 
and md0.5mb1 (see Table~\ref{tb:bar_crit}).
Although the bar formation starts later for model md1mb1Rd1.5, the bar grows
faster, and 
the final bar length at 10\,Gyr is comparable for these models.
The bar's secular evolution, however, may continue further. 
In order to understand what decides the final bar length further simulations
are required. 


\begin{figure}
\includegraphics[width=50mm]{figures/rotation_curve_md1_Rd1.5.pdf}\includegraphics[width=38mm]{figures/md1_Rd1.5_80M_1024c.pdf}\\
\includegraphics[width=50mm]{figures/rotation_curve_md0.5_Rd1.5.pdf}\includegraphics[width=38mm]{figures/md0.5_Rd1.5_110M_1024c.pdf}\\
    \caption{Rotation curves (left) and snapshots at 10 Gyr (right) for models md1mb1Rd1.5 (top) and md0.5mb1Rd1.5 (bottom). 
    The gray dashed curve is the same as the one in Fig.~\ref{fig:snapshots_mb10}. \label{fig:snapshots_Rdisk}}
\end{figure}


\begin{figure*}
\includegraphics[width=\columnwidth]{figures/mode_A2max_Rd.pdf}\includegraphics[width=\columnwidth]{figures/mode_Dbar_Rd.pdf}
\caption{Same as Fig.~\ref{fig:A2_max_mdisk}, but now for models md1mb1Rd1.5 and md0.5mb1Rd1.5 with md1mb1 shown as reference.
\label{fig:A2_max_Rd}}
\end{figure*}

