
\section{Introduction}

Simulations serve as a powerful tool to study the dynamical evolution
of galaxies. Galaxies are extremely complicated, in particular due to
the coupling between a varying environment and their internal
evolution.  Even the relatively simple self-gravity of an isolated
galaxy poses enormous challenges, in particular because of the
non-linear processes that govern the formation of spiral arms and
bar-like structures.  Many of these processes have been attributed to
perturbations from outside, and it is not a priori clear to what
extent internal dynamical processes play a role in the formation of
axis-asymmetric structures in disk galaxies.  Self-gravitating disks
are prone to form spiral arms and/or bars, but the precise conditions
under which these form are not well understood.

In earlier simulations~\citet{1971ApJ...168..343H} demonstrated, using
$\sim 7\times 10^4$ cells with near-neighbour interactions, that a 
stellar disk
without a (dark matter) halo leads to the formation of a bar within a
few galactic rotations.  In a subsequent
study~\citet{1973ApJ...186..467O} concluded that a dark-matter halo is
required to keep the disk stable.  For spiral galaxies,
\citet{1984ApJ...282...61S} performed simulations of two-dimensional 
stellar disks with $\sim 7000$ cells and $2\times 10^4$ particles
that developed multiple spiral arms. 
They suggested that spiral arms
tend to kinematically heat-up the disk, and that in the absence of an
effective coolant, such as ambient gas or star formation, this heating
would cause the spiral structures to disappear within a few galactic
rotations. \citet{1985ApJ...298..486C} also performed a series of 
simulations and found that the number of spiral arms decrease as
the disk-to-halo mass ratio decreases. 
In contradiction to the Lin-Shu quasi-stationary density-wave theory
\citep{1964ApJ...140..646L}, these simulations suggested that 
spiral arms are transient and develop from small perturbations
amplified by the self-gravity in a differentially rotating disk
\citep{1965MNRAS.130..125G, 1966ApJ...146..810J}.
Today this mechanism is known as ``swing amplification''
\citep{1981seng.proc..111T,2016ApJ...823..121M}.

The number of particles in simulations has gradually
increased with time as computers have become more powerful.  
After \citet{1984ApJ...282...61S} and
\citet{1985ApJ...298..486C}, the formation and evolution of bar
structures were often studied using three dimensional $N$-body
simulations. \citet{1990A&A...233...82C} showed that bars induce a
peanut-shaped (boxy) bulge using a three-dimensional Particle-Mesh
method with at most $\sim 8\times 10^6$ cells.  $N$-body simulations
with up to $2\time 10^8$ particles were performed by
\citet{2014ApJ...785..137S}, but they adopted a rigid potential for
the dark matter halo.  Performing such simulations with a `live'
particle halo has been hindered by the sheer computer power needed to
resolve baryonic and non-baryonic material simultaneously.

The importance of resolving the halo in such simulations
using particles was emphasized by \citet{2002ApJ...569L..83A}. They
found that once a bar formed its angular momentum is transferred to
the halo. This angular momentum transport can only be resolved in the
simulations if the halo is represented by particles that are
integrated together with the rest of the galaxy. The dynamics and
back-coupling of such a live halo also affects the evolution of the
bar.  In this study, we adopt a tree method
\citep{1986Natur.324..446B} for solving the equations of motion of all
particles in the simulations.  Several other simulations of barred
galaxies, in which halos were resolved using live particles, also
adopted a tree-code \citep{2005ApJ...631..838W,2008ApJ...679.1239W}.
\citet{2009ApJ...697..293D} performed a series of such $N$-body
simulations with up to $10^8$ particles. They confirmed that such a
large number of particles is required but also sufficient to obtain a
reliable solution for the morphology of the bar.  However, they
also found that bar formation in simulations with a larger number of
particles was systematically delayed.

Another problem of relatively low-resolution is the artificial heating
of the particles by close encounters.  In simulations of spiral
galaxies with multiple arms, \citet{2011ApJ...730..109F} demonstrated that
this effect becomes sufficiently small when the disk is resolved with
at least one million particles.  If a galactic disk needs to be
resolved with at least a million particles, it is understandable that
simulating an entire galaxy, including the dark matter halo, would
require at least 10 times this number in order to properly resolve the
disk and halo in an $N$-body simulation.  To overcome the numerical
limitations researchers tend to adopt a rigid background potential for
the galaxy's dark matter halo.
\citep[e.g.,][]{1984ApJ...282...61S,2013ApJ...763...46B,
  2013A&A...553A..77G}, as was also done in
\citet{2011ApJ...730..109F}.  In these simulations energy cannot be
transported self-consistently between the halo and the
disk, and vice versa.  Taking this coupling into account is
particularly relevant when studying the formation and evolution of
non-axisymmetric structures such as spiral arms and a bar in the disk;
bars tend to slow down due to angular-moment transfer with the halo
and grow faster when compared to models with a rigid
halo~\citep{2002ApJ...569L..83A}.

Up to now it has not been possible to carry out extensive parameter
searches with a sufficient number of particles that include a live
halo, simply because the amount of computer time required for such
studies exceeds the hardware and software capacity
\citep[e.g.][]{2009ApJ...697..293D}.  The current generation of
GPU-based supercomputers and optimized $N$-body algorithms
\cite{7328651} allows us to perform simulations with more than a
hundred billion particles \citep{2014hpcn.conf...54B} over a Hubble
time. The same developments allow us to perform an extensive parameter
over a wide range of initial conditions with a more modest number of
particles.

We developed the gravitational tree-code {\tt Bonsai} to perform such
simulations.  {\tt Bonsai}, uses the GPUs to accelerate the
calculations and achieves excellent efficiency with up to $\sim 19000$
GPUs \citep{2012JCoPh.231.2825B,2014hpcn.conf...54B}.  The efficiency
of {\tt Bonsai} allows us to run simulations with a hundred million
particles for 10\,Gyr in a few hours using 128 GPUs in parallel.  This
development allows us to perform simulations of disk galaxies for the
entire lifetime of the disk, and therefore to study the formation of
structure using a realistic resolution and time scale. The code is
publicly available and part of the {\tt AMUSE} framework
\citep{AMUSE}.

Using {\tt Bonsai} running on the CSCS Piz Daint supercomputer we
performed a large number of disk galaxy simulations with live
dark-matter halos at a sufficiently high resolution to suppress
numerical heating on the growth of the physical instabilities due to
the self-gravity of the disk.  With these simulations, we study the
relation between the initial conditions and the final disk galaxies.





