
\documentclass{article} % For LaTeX2e
\usepackage{iclr2024_conference,times}

% Optional math commands from https://github.com/goodfeli/dlbook_notation.
%%%%% NEW MATH DEFINITIONS %%%%%

\usepackage{amsmath,amsfonts,bm}
\usepackage{xifthen}

% Highlight a newly defined term
\newcommand{\newterm}[1]{{\bf #1}}

\def\eps{{\epsilon}}


% Utility for ticks 
\newcommand{\cmark}{\ding{51}}%
\newcommand{\xmark}{\ding{55}}%

% Theorem styles 
\theoremstyle{definition}
\newtheorem{theorem}{Theorem}[section]
\newtheorem{definition}{Definition}[section]
% \newtheorem{remark}{Remark}[theorem] %numbered remark
\newtheorem*{remark}{Remark} %unnumbered remark
\newtheorem{lemma}{Lemma}[section]
\newtheorem{prop}{Proposition}[section]
\newtheorem{corollary}{Corollary}[theorem]
\newtheorem{conjecture}{Conjecture}[section]
\newtheorem{assumption}{Assumption}[section]

\newtheorem{manualtheoreminner}{Theorem}
\newenvironment{manualtheorem}[1]{%
  \renewcommand\themanualtheoreminner{#1}%
  \manualtheoreminner
}{\endmanualtheoreminner}


% Math helper - standard function
\DeclareMathOperator*{\argmax}{arg\,max}
\DeclareMathOperator*{\argmin}{arg\,min}
\DeclareMathOperator{\support}{support}
\DeclareMathOperator{\MAX}{MAX}
\DeclareMathOperator{\term}{\texttt{term}}
\DeclareMathOperator*{\logsumexp}{log-sum-exp}
\DeclareMathOperator*{\TV}{TV}
\newcommand{\norm}[1]{\left\lVert#1\right\rVert}
\DeclarePairedDelimiter\set\{\}
\DeclarePairedDelimiter\abs{\lvert}{\rvert}%
\newcommand*{\mytop}{\mathrel{\scalebox{0.5}{$\top$}}}
\newcommand*{\mybot}{\mathrel{\scalebox{0.5}{$\bot$}}}
\newcommand*{\mydiese}{\mathrel{\scalebox{0.5}{$\#$}}}
\newcommand*{\myplus}{\mathrel{\scalebox{0.5}{$+$}}}
\newcommand*{\myminus}{\mathrel{\scalebox{0.5}{$-$}}}
\newcommand*{\bmg}{\bm{\gamma}}
\newcommand*{\bml}{\bm{\lambda}}

% MDP notation
\renewcommand{\S}{\mathcal{S}}
\newcommand{\X}{\mathcal{X}}
\newcommand{\A}{\mathcal{A}}
\newcommand{\T}{\mathcal{T}}
\newcommand{\M}{\mathcal{M}}
\newcommand{\B}{\mathcal{B}}
\newcommand{\Bset}{\mathfrak{B}}
\newcommand{\Dist}{\mathscr{P}}
\newcommand{\D}{\mathcal{D}}
\newcommand{\Real}{\mathbb{R}}
\renewcommand{\P}{\mathcal{P}}
\newcommand{\E}{\mathop{\mathbb{E}}}
\renewcommand{\H}{\mathcal{H}}
% \newcommand{\R}{\mathcal{R}}
% \newcommand{\C}{\mathcal{C}}

% Extended MDP notation
\newcommand{\Pstar}{p^{\star}}
\newcommand{\Rstar}{\bm{r}^{\star}}
\newcommand{\Cstar}{C^{\star}}
% \newcommand{\rmax}{\textsc{Rmax}}
\newcommand{\rmax}{r_{\mytop}}
\newcommand{\cmax}{\textsc{Cmax}}

\newcommand{\mstar}{m^{\star}}
\newcommand{\mhat}{\hat{m}}
\newcommand{\mopt}{m^{\star}}

\newcommand{\Phat}{\hat{p}}
\newcommand{\Rhat}{\hat{\bm{r}}}
\newcommand{\Chat}{\hat{C}}

% Math helper - custom function
\newcommand{\expwrtpi}[1]{\E_{\pi} [\sum_{t=0}^{\infty} \gamma^t #1(s_t, a_t)]}
\newcommand{\expangle}[1]{\langle #1  \rangle}

% helper function for return and constraints

% for value function, takes arguments:
% #1: policy 
% #2: the function of interest, R or C_i
% #3 (optional): the MDP for which this is estimated
\newcommand{\V}[3]{ %
    \ifthenelse{\isempty{#3}}%
    {V^{#1}(#2)}% #3 is empty 
    {V^{#1}_{#3}(#2)}%
}

\newcommand{\Q}[3]{
    \ifthenelse{\isempty{#3}}
    {Q^{#1}(#2)}% #3 is empty 
    {Q^{#1}_{#3}(#2)}%
}


\newcommand{\Adv}[3]{
    \ifthenelse{\isempty{#3}}
    {A^{#1}(#2)}% #3 is empty 
    {A^{#1}_{#3}(#2)}%
}

% careful diff notation
% 1: pi
% 2: R/C
% 3: M
\newcommand{\J}[3]{
    \ifthenelse{\isempty{#3}}
    {\mathcal{J}^{#1}_{#2}}% #3 is empty -> eg V^{\pi}(x ; R)
    {\mathcal{J}^{#1}_{#3,#2}}% -? eg V^{\pi}_{M}(x ; C)
    % {J_{#2}(#1)}% #3 is empty 
    % {J_{#2}(#1, #3)} %
}



\newcommand{\MRkern}{%
  \mkern-6.5mu
  \mathchoice{}{}{\mkern0.2mu}{\mkern0.5mu}%
}

% for value function, takes arguments:
% #1: policy 
% #2: the function of interest, R or C_i
% #3 (optional): the MDP for which this is estimated
% #4: variables to be given input (x) or (x,a)
\newcommand{\val}[4]{ %
    \ifthenelse{\isempty{#3}}%
    {v^{#1}_{#2}(#4)}% #3 is empty -> eg V^{\pi}(x ; R)
    {v^{#1}_{#3,#2}(#4)}% -? eg V^{\pi}_{M}(x ; C)
    % {V^{#1}_{#3}(#4 ;#2)}% -? eg V^{\pi}_{M}(x ; C)
    % {V_{#2}(#4 ; #1)}% #3 is empty -> eg V_R(x ; \pi)
    % {V_{#2}(#4 ;#1, #3)}% -? eg V_C(x ; \pi, M)
    % {#2 \MRkern V^{#1}_{#3}(#4)}% -? eg V^{\pi}_{M}(x ; C) # combines the letter V and R together
}

\newcommand{\qval}[4]{
    \ifthenelse{\isempty{#3}}
    {q^{#1}_{#2}(#4)}% #3 is empty -> eg V^{\pi}(x ; R)
    {q^{#1}_{#3,#2}(#4)}% -? eg V^{\pi}_{M}(x ; C)
    % {Q^{#1}(#4 ; #2)}% #3 is empty -> eg Q^{\pi}(x,a ; R)
    % {Q^{#1}_{#3}(#4 ;#2)}% -? eg Q^{\pi}_{M}(x,a ; C)
    % {Q_{#2}(#4 ; #1)}% #3 is empty -> eg Q_R(x,a ; \pi)
    % {Q_{#2}(#4 ;#1, #3)}% -? eg Q_C(x,a ; \pi, M)
}
\DeclareMathOperator*{\advantage}{Adv}

\newcommand{\adv}[4]{
    \ifthenelse{\isempty{#3}}
    {\advantage^{#1}_{#2}(#4)}% #3 is empty -> eg V^{\pi}(x ; R)
    {\advantage^{#1}_{#3,#2}(#4)}% -? eg V^{\pi}_{M}(x ; C)
    % {A^{#1}(#4 ; #2)}% #3 is empty -> eg Q^{\pi}(x,a ; R)
    % {A^{#1}_{#3}(#4 ;#2)}% -? eg Q^{\pi}_{M}(x,a ; C)
    % {A_{#2}(#4 ; #1)}% #3 is empty -> eg A_R(x,a ; \pi)
    % {A_{#2}(#4 ;#1, #3)}% -? eg A_C(x,a ; \pi, M)
}




\newcommand{\ci}{C}

\newcommand{\pib}{\pi_{b}}
\newcommand{\piopt}{\pi^{*}}
\newcommand{\pie}{\pi_{t}}

\newcommand{\lR}{\lambda_{R}}
\newcommand{\lC}{\lambda_{C}}
\newcommand{\ephi}{e_{\phi}}

\newcommand{\pr}{\text{Pr}}
\newcommand{\IS}{\text{IS}}
\newcommand{\CI}{\text{CI}}


% SPIBB symbols 
\newcommand{\EpsPib}{(\pi_b, e, \epsilon)}

\usepackage{hyperref}
\usepackage{url}
\usepackage{graphicx} 
\usepackage{color}
\usepackage{booktabs}
\usepackage{makecell}
\usepackage{multirow}
\usepackage{float}
\usepackage{wrapfig}
\usepackage{caption}
\usepackage{tcolorbox}
\usepackage{amssymb}


\title{AutoHall: Automated Hallucination Dataset Generation for Large Language Models}

% Authors must not appear in the submitted version. They should be hidden
% as long as the \iclrfinalcopy macro remains commented out below.
% Non-anonymous submissions will be rejected without review.

\author{Cao Zouying, Yang Yifei, Zhao Hai\thanks{Corresponding author}\\
Department of Computer Science and Engineering\\
Shanghai Jiao Tong University\\
Shanghai, China \\
\texttt{\{zuoyingcao,yifeiyang\}@sjtu.edu.cn, zhaohai@cs.sjtu.edu.cn} 
}
\date{}

% The \author macro works with any number of authors. There are two commands
% used to separate the names and addresses of multiple authors: \And and \AND.
%
% Using \And between authors leaves it to \LaTeX{} to determine where to break
% the lines. Using \AND forces a linebreak at that point. So, if \LaTeX{}
% puts 3 of 4 authors names on the first line, and the last on the second
% line, try using \AND instead of \And before the third author name.

\newcommand{\fix}{\marginpar{FIX}}
\newcommand{\new}{\marginpar{NEW}}

%\iclrfinalcopy % Uncomment for camera-ready version, but NOT for submission.
\begin{document}


\maketitle

\begin{abstract}
% Though Large language models (LLMs) have garnered widespread attention due to their powerful comprehension and generation capabilities, the generated responses may be plausible-sounding but incorrect or fabricated, which refs to the issue of ``hallucination".


% Existing methods solve hallucination detection tasks either via external databases or in a zero-resource manner, mainly based on human-annotated datasets limited to specific LLMs.
% Considering human collection is labor-intensive and distinct models have different hallucination degree in outputs, it is essential to develop an automatic and universally applicable approach for constructing hallucination datasets.
% Motivated by this, our paper introduces an automatic hallucination dataset collection method and investigates its generality in different LLMs about diverse subjects.
% Besides, we propose a zero-resource hallucination detection technique based on this automatically collected dataset.
% Finlally, our experiments find that ....

While Large language models (LLMs) have garnered widespread applications across various domains due to their powerful language understanding and generation capabilities, the detection of non-factual or hallucinatory content generated by LLMs remains scarce. Currently, one significant challenge in hallucination detection is the laborious task of time-consuming and expensive manual annotation of the hallucinatory generation. To address this issue, this paper first introduces a method for \underline{auto}matically constructing model-specific \underline{hall}ucination datasets based on existing fact-checking datasets called \textbf{AutoHall}. Furthermore, we propose a zero-resource and black-box hallucination detection method based on self-contradiction. We conduct experiments towards prevalent open-/closed-source LLMs, achieving superior hallucination detection performance compared to extant baselines. Moreover, our experiments reveal variations in hallucination proportions and types among different models.

\end{abstract}
% !TEX root = 0-20SPAWC_SAND.tex
% DO NOT REMOVE THE ABOVE COMMENT!
\section{Introduction}
%
Millimeter-wave (mmWave) and massive multi-user (MU) multiple-input multiple-output (MIMO) will be  core technologies for future wireless systems~\cite{larsson14a, rappaport15a}.
%
The combination of these technologies enables simultaneous communication to multiple user equipments (UEs) at unprecedentedly high data rates. 
%
These advantages come at the cost of significantly increased power consumption, implementation complexity, and system costs. A viable solution to address these challenges is the use of low-resolution data converters combined with sophisticated but efficient baseband processing algorithms in all-digital basestations (BS) architectures~\cite{dutta2019case,jacobsson17b,li17b,mo16b,panagiotis20}. 

\subsection{Channel Estimation with Low-Resolution Data Converters}
%
Coarse quantization of the received baseband samples, due to the use of low-resolution analog-to-digital converters (ADCs) at the BS, together with the high path loss at mmWave or terahertz (THz) frequencies~\cite{rappaport15b, gao16}, renders the acquisition of accurate channel estimates a challenging task.
%
Fortunately,  wave propagation at mmWave or THz frequencies is directional~\cite{akdeniz14a} and channels typically consist only of a few dominant propagation paths~\cite{rappaport13a,rappaport15a}. Both of these properties cause the channel vectors to be sparse in the beamspace domain, which can be exploited to perform denoising that improves reliability of data transmission~\cite{alkhateeb14a,mo14b,tang13,brady13,ghods19a}. 

Practical sparsity-exploiting channel denoising methods for mmWave massive MU-MIMO systems must exhibit low computational complexity due to the large number of BS antenna elements and the potentially large number of UEs that commmunicate simultaneously.
%
A low-complexity mmWave channel denoising algorithm called BEACHES (short for beamspace channel estimation) has been proposed recently in~\cite{ghods19a}. This method has orders-of-magnitude lower complexity than state-of-the-art denoising methods, such as atomic norm minimization (ANM)~\cite{bhaskar13} and Newtonized orthogonal matching pursuit (NOMP) \cite{mamandipoor16}. 
%
However, all of these existing denoising methods perform poorly when denoising channel vectors that were acquired through low-resolution data converters. 
%
Channel estimation with 1-bit ADCs has been analyzed in~\cite{li17b,li16a,jacobsson17b,mollen16c,studer16a}. 
%
Beamspace sparsity of mmWave channels has been exploited to denoise channel vectors from 1-bit measurements in \cite{mo16b, huang19,kaushik18}.
%
However, all of these denoising methods exhibit high complexity, ignore beamspace sparsity, and/or require a number of parameters that must be adapted to the instantaneous propagation conditions, such as the number of dominant propagation paths.  


\subsection{Contributions}
%
We propose low-complexity channel estimation algorithms for mmWave massive MU-MIMO systems that operate with 1-bit data converters.
%
By using a Bussgang-like decomposition~\cite{bussgang52a} of the 1-bit measurement process, our methods adapt the optimal denoising parameters to the channel's instantaneous sparsity via Stein's unbiased risk estimate (SURE).
%
We propose two methods that build upon BEACHES put forward in~\cite{ghods19a} and a novel method, referred to as Sparsity-Adaptive oNe-bit Denoiser (SAND), which automatically tunes two algorithm parameters to minimize the channel estimation mean-square error (MSE). 
%
To demonstrate the efficacy of our channel estimation algorithms, we perform  MSE and bit error rate (BER) simulations with  line-of-sight (LoS) and non-LoS mmWave channels in a massive MU-MIMO system. 

\subsection{Notation}
%
Lowercase and uppercase boldface letters denote column vectors and matrices, respectively. 
%
The $k$th entry of the vector~$\bma$ is~$a_k$; 
the real and imaginary parts are $\realindex{\bma} = \realpart{\bma}$ and $\imagindex{\bma} = \imagpart{\bma}$, respectively. 
For a matrix $\bA$, 
its transpose and Hermitian transpose are $\bA^\Tran$ and $\bA^\Herm$, respectively. 
A complex Gaussian vector $\bma$ with mean $\bmm$ and covariance $\bK$ is written as $\bma\sim\CN(\bmm, \bK)$.
%
Expectation is denoted by $\smolE{\cdot}$. 
%
\section{Related Works}
\label{related_works}

% related work主要就是两大块:
% 1. 幻觉是什么,幻觉的产生原因有哪些。
% 2. 幻觉检测有哪些方法,白盒黑盒之类的,cite几个paper,每个paper一两句话总结一下。
% 最后一两句话说一下它们的问题 ,需要手动标,然后我们这个幻觉检测方法的优势可能得总结下,要说明它们工作的劣势
% different models use different strategies for hallucination
% the hallucination of LLMs is topic-sensitive(人大)
\subsection{Hallucination of Large Language Models}
Although large language models have demonstrated remarkable capabilities~\citep{liu2023summary,srivastava2022beyond}, they still struggle with several issues, where hallucination is a significant problem. 
Hallucination arises when the content generated by LLMs is fabricated or contradicts factual knowledge. 
The consequent effects may be harmful to the reliability of LLM applications~\citep{zhang2023siren,pan2023risk}.
So far, the causes of hallucination in LLMs have been investigated across different tasks, such as question answering~\citep{zheng2023does}, abstractive summarization~\citep{cao2021hallucinated} and dialogue systems~\citep{das2023diving}. The key factors include but are not limited to training corpora quality~\citep{mckenna2023sources,dziri2022origin}, problematic alignment process~\citep{radhakrishnan2023question,zhang2023siren} and randomness in generation strategy~\citep{lee2022factuality,dziri2021neural}.

\subsection{LLM Hallucination Detection}
To detect the hallucination issue, there are many endeavors to seek solutions. 
On the one hand, prior works focus on resorting to external knowledge to detect hallucinations.
For instance, ~\citet{gou2023critic} propose a framework called CRITIC to validate the output generated by the model with tool-interaction and ~\citet{chern2023factool} invoke interfaces of search engines to recognize hallucination.
On the other hand, current research pays more attention to realizing one zero-resource hallucination detection method. 
Typically,~\citet{xue2023rcot} utilize the Chain of Thoughts (CoT) to check the hallucinatory responses; ~\citet{selfcheckgpt} introduce a simple sampling-based approach that can be used to detect hallucination with token probabilities.

Besides, some hallucination benchmarks~\citep{li2023halueval,umapathi2023med,dale2023halomi} are constructed to support detection tasks in numerous scenarios. 
For example, \citet{umapathi2023med} propose a hallucination benchmark within the medical domain as a tool for hallucination evaluation and mitigation;
~\citet{dale2023halomi} present another dataset with human-annotated hallucinations in machine translation to promote the research on translation pathology detection and analysis. 

Nevertheless, there are limitations as they are subject to manually annotated hallucination datasets, which are expensive and time-consuming. 
Meanwhile, the datasets are model-specific, requiring separate annotations for different models, whose applicability will also be affected by model upgrades. 
Furthermore, there is also room for improvement in the performance of current hallucination detection methods.




%is constructed to evaluate whether LLM are able to generate factual responses, these works scattered among various tasks have not been systematically reviewed and analyzed

% challenging the applicability of detection methods due to variations in language patterns and model characteristics.

% Furthermore, even for a certain LLM, they ignore the influence of model update upon their detection performance since the hallucination level inevitably evolves in different versions.

%Our work share the same idea with ~\citet{selfcheckgpt}, but we pay more attention to promote the research of hallucination automatic collection in LLMs. 


%By implementing automated data collection, our aim is not only to improve efficiency% and timeliness
%, but to lay a solid foundation for subsequent hallucination detection work. % 之后重写
% Based on our dataset, we 

\section{Methodology}
In this section, we first formulate the definition of LLM hallucination discussed in our work. Then, we introduce our automatic dataset creation pipeline which focuses on prompting LLMs to produce ``hallucinatory references''. Finally, based on our generated datasets, we further present one zero-resource, black-box approach to recognize hallucination. 
% 第一段定义什么是幻觉
% 用个简单的数学形式定义一下幻觉是什么。
% 比如给定一个LLM模型$\mathcal{M}$,给定一个句子$X=x_1,x_2,\cdots, x_n$,给一段prompt$Q=q_1,q_2,\cdots, q_o$,需要模型根据$X$和$Q$进行回答$Y=y_1,y_2,\cdots, y_m$。
% 如果这个$Y$是事实性不正确的话,那么这个可以认为模型产生了幻觉。
\subsection{LLM Hallucination}
%介绍一下hallucination分类 
LLM hallucination can be categorized into different types~\citep{galitsky2023truth}, such as hallucination based on dialogue history, hallucination in generative question answering and general data generation hallucination. They can all be attributed to the generation of inaccurate or fabricated information.
% Considering the above cases, we try to formulate LLM hallucinations using mathematical notations as following.


Generally, for any input sentence $X =[x_1, x_2, \ldots, x_n]$ with a specific prompt $P = [p_1, p_2, \ldots, p_o]$, the large language model $\mathcal{M}$ will generate an answer $Y = [y_1, y_2, \ldots, y_m]$, denoted as:
\begin{equation}
\mathcal{M}(P, X) = Y.
\end{equation}
Given factual knowledge $F=[f_1,f_2,..,f_t]$, the problem of hallucination $H$ occurs when there is a factual contradiction between the output span $Y_{[i:j]}=[y_i,y_{i+1},\dots,y_j]$ and the knowledge span $F_{[u:v]}=[f_u,f_{u+1},\dots,f_{v}]$, which can be summarized into the function below:
\begin{equation}\label{halluFunc}
% Y \in H\Leftrightarrow\exists Y_{i,j} \exists F_{u,v} ( Y_{i,j}\nleftrightarrow F_{u,v}).
Y \in H\Leftrightarrow\exists Y_{[i:j]} \exists F_{[u:v]} (( Y_{[i:j]}\land F_{[u:v]} = \text{False})).
\end{equation}

% 检测方面要先写一个前提,类似于self-check的摘要里面的那段话,模型对于有幻觉的reference,再以相似的prompt生成reference,大概率会有冲突,但但对于没幻觉的reference,以相似的referen生成的prompt冲突较少。(这个点就需要后面的补充实验的验证)。
% !!!下面两段我写的有问题,根据目前做的细节改。用数学符号表达出来。


\subsection{AutoHall: Automatic Generation of Hallucination Datasets}
% 第二段介绍我们生成数据集的方法
% 对于一个模型,现在生成$Y$的事实正确性通常要人标,但是我们认为可以利用事实性数据集来生成模型的幻觉检测数据集。通常事实性检测数据集包含几个部分,一段话claim、一个事实正确性的label(正确or错误、support还是不support等)、还有evidence。
% 然后我们将claim作为X输入到模型,将prompt定义成"需要模型生成对应的reference"(这里就写你用的prompt),输出的$Y$和evidence进行对比,(这里对比我记得也是用LLM做的?)如果LLM认为有冲突,那么模型是有幻觉的。

Current research on hallucination detection mostly relies on manually annotated datasets. Namely, whether $Y$ is hallucinatory requires slow and costly manual tagging due to the absence of a comparison standard for the factuality. However, the fact-checking datasets provide us with data typically comprising real-world claims, corresponding ground truth labels, and evidence sentences as shown in Fig.~\ref{fig:dataset}. We can prompt a model to generate relevant references for claims and then use the ground truth labels as criteria to assess the hallucinatory nature of the generated references. Specifically, as shown in Fig.~\ref{fig:dataset}, \textbf{AutoHall} generates hallucination datasets following the below three steps:

\begin{figure}[h]
\centering
    \includegraphics[width=1\linewidth]{figs/datasets.pdf}
    \caption{Our proposed approach to collect LLM hallucination automatically. \color{green!50!black}{\textbf{Green}}\color{black}{: the grounded information. }\color{red!75!black}{\textbf{Red}}\color{black}{: the incorrect information. The complete prompts are shown in Appendix~\ref{app:prompts}.}}
    \label{fig:dataset}
\end{figure}

\textbf{Step 1: References Generation.} % Public fact-checking datasets provide real world claims, ground truth labels indicating true/supported or false/unsupported and evidence retrieved from websites.
% as shown in Table~\ref{tab:factcheck}. 
For an LLM, we prompt it to generate corresponding references to the claims in the existing datasets by the prompt illustrated in Fig.~\ref{fig:dataset} Step 1. Note that to simplify the generation, we only focus on factual (supported/true) and faked (unsupported/false) claims. 
Besides, we discard references that fail to contain concrete content, like a long response beginning with ``I can not provide a specific reference for the claim you mentioned...''. 
The remaining valid references are either reliable ($\overline{H}$) or hallucinatory ($H$).

%consists of some attributes
\textbf{Step 2: Claim Classification.} Separately for each reference, in order to label 
whether a claim belongs to $\overline{H}$ or $H$, we prompt LLM to perform claim classification. %in conjunction with 
The input sequence is of format as shown in Fig.~\ref{fig:dataset} Step 2, where the two placeholders $\left\langle claim \right\rangle$ and $\left\langle reference\right\rangle$ should be replaced with the claim $X$ and the generated reference $Y$ in Step 1. 
Then the output is of format ``Category: $\left\langle category\right\rangle$ Reason (Optional): $ \left\langle reason\right\rangle$” where the category is limited to true ($T$) or false ($F$). 
To elaborate, $T$ indicates the generated reference $Y$ supports the claim $X$ is factual and $F$ represents that $Y$ demonstrates claim $X$ is faked. % $F$ follows the same logic.
We expect correct classification to each claim, while wrong classification may be taken as a sign of the existence of hallucination in the generated reference that it erroneously supports the claim's factuality. The binary classification results of LLMs are reliable as LLMs exhibit strong capabilities in natural language inference~\citep{wu2023exploring}.

\textbf{Step 3: Hallucination Collection.}
Last, we can directly adopt a simple comparison to collect the hallucination dataset. 
If the classification result is not equal to the ground truth label, we label the reference as hallucination. 
%Assigned hallucination
%by comparing the ground truth labels to the classifications. 
Meanwhile, to maintain a balanced proportion between hallucinatory and factual references, we sample the same number of factual references built upon hallucinatory ones to form a completed dataset.

\subsection{Hallucination Detection Approach}
% 第三段介绍我们检测幻觉的方法
% 这里可以说是受到(inspired by)selfcheckgpt的启发,我们也尝试让模型回答多次并且去检测这些回答之间的一致性,如果不一致的话,说明模型对生成的reference有知识冲突,出现了幻觉。但是和selfcheckgpt检测的方法有不同,selfcheckgpt是利用一些小型语言模型打分如BERTScore,或者简单的n-gram来评测多个回答之间的一致性,我们想到可以直接利用powerful的LLM来进行判断。因此我们利用LLM对比多个reference之间的一致性。(还要讲你那个reference分组的技巧吧,这个我有些忘记了,都用数学表达一下。)
% 我们的本质的幻觉检测idea和selfchekgpt是一样的,但它们在对比的时候有一些漏洞,不能全部放在一起比较,因此我们提出了一个1v1比较的方法。

% 这个idea本身有一个问题,就是假如一开始的reference是有幻觉的,那么也有可能出现后面所有的reference都是有幻觉的现象,这种情况是检测不出来的。但是我们承认是有这样一个问题,然后我们用13个reference之间的冲突证明:如果模型产生了幻觉,那么大概率后面再sample会产生很多幻觉的样本以及少量的无幻觉reference;但如果一开始的reference没有产生幻觉,那么后面大概率产生的reference都是没有幻觉的。
\begin{figure}[h]
\centering
    \includegraphics[width=1\linewidth]{figs/detection.pdf}
    \caption{Our proposed approach to detect LLM hallucination.  \color{cyan!75!magenta}{Blue}\color{black}{: the claim from fact-checking dataset. }\color{red!80!black}{Red}
    \color{black}{: the response need to be detected whether exists hallucination.} \textcolor{purple!60!blue}{Purple}: the sampled references to trigger self-contradictions. The complete Step 2 prompts are shown in Appendix~\ref{app:prompts}.}
    \label{fig:detection}
\end{figure}

The rationale for our detection approach is that if the LLM knows one claim well, even when we query it to provide multiple references, self-contradictions among them should be absent otherwise hallucination information must exist in one reference. Compared to SelfCheckGPT~\citep{selfcheckgpt}, our method uses the LLM for hallucination detection end-to-end rather than relying on output token probabilities to calculate hallucination score with BERTScore or n-gram. 

As shown in Fig.~\ref{fig:detection}, to trigger self-contradictions, we first appropriately prompt an LLM to answer a second reference $Y_k'$ and repeat this process $K$ ($K=13$ in experiments) times. It is worth noting that each query is running independently with an equivalent prompt. % prompts例子可放附录
Then, we concatenate each generated reference $Y_k' (k=1,...,K)$ with the original reference $Y$ to form one input pair. 
Unlike SelfCheckGPT measures the consistency between $Y$ and all $K$ sampled references, we invoke the LLM to detect if $Y$ and $Y_k'$ are contradictory.
Such self-contradiction detection in $\left \langle Y,Y_k'\right\rangle$ pair can focus more on the hallucination detection in $Y$ and avoid the problem that SelfCheckGPT incorrectly identifies the conflicts in the $K$ sentences generated subsequently as the hallucination in $Y$.

% However, to be honest, 
% 补充实验
% 需要一个平均值,生成的13个reference,和原本reference有冲突的平均值是多少,没冲突的平均值是多少。
%Since our work compares LMs and hallucination estimation procedures, the risk is lower compared to a system that might be deployed using our procedures to reduce hallucination. Before deploying any such system, one should perform a more thorough examination of potential biases against sensitive groups and accuracy across different research areas.

Formally, we can check if there exists at least one $Y_{k}'$ conflicting with $Y$, as shown in Eq.~(\ref{xiqu}). If conflicts are indicated, it suggests the model does not understand the claim well, and $Y$ may be hallucinatory. Conversely, if no conflicts are found in $K$ pairs, it indicates that the factual reference.
\begin{equation}\label{xiqu}
   Y \in H \Leftrightarrow \exists Y_{k,[u,v]}'\exists Y_{[i,j]}(( Y_{[i,j]}\land Y_{k,[u,v]}' = \text{False}))
   % (Y\leftrightarrow \lnot Y_1')\vee(Y\leftrightarrow \lnot Y_2') \vee ... \vee(Y\leftrightarrow \lnot Y_K')
\end{equation}


% \begin{equation}\label{xiqu}
%    % Y \in H \Leftrightarrow (Y\leftrightarrow \lnot Y_1')\vee(Y\leftrightarrow \lnot Y_2') \vee ... \vee(Y\leftrightarrow \lnot Y_K') 

% Y \in H\Leftrightarrow\exists Y_{i,j} \exists F_{u,v} (( Y_{i,j}\land F_{u,v} = \text{False})).



% \end{equation}


\section{Experiments}\label{sec:experiments}
Here, we present BARNEY finetuning performance on several different tasks. \nikos{we need to be more specific here. what are the questions we would like to answer specifically or our hypothesis. make sure the points we make correspond to the claims that we make in the intro/abstract and are clearly stated.}
For fair comparison, we pretrain \textbf{BARNEY} on the same Wikipedia + BooksCorpus dataset from BERT. To avoid training from scratch we use the pretrained BERT model as a starting point for our encoder function. We consider several different methods of pretraining. \nikos{these sound like experimental details, I would mention the basic ones in a subsection here and defer the rest for the supplementary (make sure we include everything there).}


% \subsection{BARNEY Pretraining for Classification}
% We outline several methods to pretrain BARNEY, and perform experiments on the sentence similarity task (SST2) with RoBERTa-base BARNEY, to observe the relative performance on single-sentence classification.

% \input{tables/barney_training_sst}

\noindent \textbf{Datasets} \nikos{Mention some basic details about the datasets used here.}
 
\noindent \textbf{Model configuration} \nikos{Only the basic ones and defer to the supplementary for the details such as number of epochs, learning rate etc. (e.g. we follow configuration from ...)}

\noindent \textbf{Baselines} \nikos{Describe our baselines and the versions of our model that we examined.}

\subsection{GLUE}
The General Language Understanding Evaluation (GLUE) benchmark \citep{wang2018glue} is a collection of diverse natural language understanding text classification tasks. We evaluate BARNEY pretrained with the Denoising, Fixed methodology report its performance on the dev set of each task in Table 1. \\
We find that on multiple tasks BARNEY on the base models end up performing on par with the large models with a mere fraction of additional parameters and compute. 

% \ivan{Since these are classification tasks, I feel like BARNEY would have a good chance at beating BERT as a baseline. These classification task require backpropagation through BARNEY, in addition to the linear classification layer, in the finetuning process.}
% \nikos{We should definitely try this. I am curious to see how the autoencoder pretraining objective impacts the results. } 



% \subsection{SentEval}
% SentEval [cite] is an evaluation framework for fixed sentence embeddings on 17 downstream tasks. We follow a setup similar to \cite{Reimers2019SentenceBERT} by finetuning on a combination of Natural Language Inference and Semantic Textual Similarity training sets, then finetuning a linear regression head on the fixed representations for each task. We examine the perfomrance of both BARNEY and BERT + CAB.
% % \ivan{These are 17 downstream tasks that \textit{don't} backpropagate through the sentence encoder, but rather, evaluate the fixed-length sentence embeddings themselves. \url{https://github.com/facebookresearch/SentEval}. If we include BERT's [CLS] token in this framework, we definitely have a good shot here}
% \nikos{I don't believe that this evaluation will measure the full potential of the method since it's best on fixed embeddings. By the way some of these tasks overlap with GLUE where we plan to test anyway with finetuning (e.g.SST/MRPC). So, I'd rather suggest to look for some controlled generation task such as sentiment style transfer (e.g. like the one used by Shen et al 2020) to demonstrate the benefits for a real  generation task (other than the reconstruction tests below).}
% % \ivan{Gotcha. In SentenceBERT they test the fixed embeddings after finetuning to the NLI dataset, which I feel like a good table to include would be an extra row for our method on table 5: \url{https://arxiv.org/pdf/1908.10084.pdf}}


\subsection{Sentence Representations}
\citet{Reimers2019SentenceBERT} perform the methodology of \citet{conneau2017supervised} of finetuning on NLI by using simple pooling methods of BERT representations, such as mean, max, and cls, and evaluate their performance on sentence similarity task. We train and evaluate BARNEY with a similar setup and its unique context attention bottleneck to compare against their results.
 
% \citet{conneau2017supervised} show that one can obtain universal sentence representations by finetuning on natural langauge inference data. \citet{Reimers2019SentenceBERT} show how this methodology can be applied to pooled BERT representations to obtain such sentence representations. We compaire

\begin{table}[t]
	\centering 
	\footnotesize
	\begin{tabular}{l|c}
		\toprule
		\textbf{Pooling} & \textbf{Spearman} \\ \midrule
% 		\multicolumn{3}{|l|}{\textit{Pooling Strategy}} \\ \hline
		\texttt{MEAN} & 80.78 \\
		\texttt{MAX} & 78.76 \\
		\texttt{CLS} &  79.67 \\
		$\beta$ \text{ (ours)} & \textbf{81.88} \\
		\bottomrule
% 	    \textbf{Pooling} & \textbf{NLI} & \textbf{STSb} \\ \midrule
% % 		\multicolumn{3}{|l|}{\textit{Pooling Strategy}} \\ \hline
% 		\texttt{MEAN} & 80.78   & 87.44 \\
% 		\texttt{MAX} & 79.07 & 69.92 \\
% 		\texttt{CLS} & 79.80 & 86.62  \\
% 		\texttt{CAB}, Fixed (ours) & \textbf{81.88} & \\
% 		\bottomrule
% 		\hline
% 		\multicolumn{3}{|l|}{\textit{Concatenation}} \\ \hline
% 		$(u, v)$ & 66.04 & -\\
% 		$(|u-v|)$ & 69.78 & - \\
% 		$(u*v)$ & 70.54 & -\\
% 		$(|u-v|, u*v)$ & 78.37  & -\\
% 		$(u, v, u*v)$ & 77.44 & -  \\
% 		$(u, v, |u-v|)$ &  \textbf{80.78} & - \\
% 		$(u, v, |u-v|, u*v)$ & 80.44 & - \\ 
% 		\hline	
	\end{tabular}
	\caption{Performance of sentence representations from RoBERTa trained with different pooling methods on NLI data and then evaluated on STS benchmark's development set %(STSb)
	in terms of Spearman's rank correlation.}
	\label{tab:pooling}
\end{table}

% DEV RESULTS TO ADD
% MEAN  0.8092
% MAX   0.7876
% CLS   0.7967  0.7999
% SB    
% 




% 		\texttt{MEAN} & 80.78 \\
% 		\texttt{MAX} & 79.07 \\
% 		\texttt{CLS} & 79.80 \\
% 		\texttt{SB} (ours) & \textbf{81.88} \\

We find that using the context attention bottleneck provides significant gains over using the other simple pooling methods. We suspect is due to the bottleneck acting as "weighted pooling" by attending over all the final tokens, to compute the final representation rather than mean/max equally considering all tokens or cls considering the representations before the final layer.



\begin{table}[t]
	\centering 
	\footnotesize
	\renewcommand{\arraystretch}{1.3}
	\begin{tabular}{l | c | c}
		\toprule
		\textbf{Model} & \textbf{Spearman} & \textbf{Parameters} \\ \midrule
		\multicolumn{3}{l}{\textit{Unsupervised}} \\\midrule
% 		\multicolumn{2}{l}{\textit{Trained on NLI (not STS Benchmark)}} \\ \hline
		Avg.\ GloVe embeddings & 58.02 & - \\
		Avg.\ BERT embeddings &  46.35 & - \\
		\textsc{Autobot}-base unsup. & \textbf{58.49} & - \\\midrule
		\multicolumn{3}{l}{\textit{Supervised}} \\\midrule
		InferSent - GloVe &  68.03 & - \\
		Universal Sentence Encoder &  74.92 & - \\
% 		SBERT-base & 77.03 & \\
% 		SBERT-large & 79.23 & \\\hline

        % BERT-base & 74.81 & 110M \\
        RoBERTa-base & 75.37 & 125M\\
% 		SBERT-base & 76.81 & 110M \\
		SRoBERTa-base & 76.89 & 125M \\
% 		AUTOBOT BERT-base & 77.03 & 111M \\
		\textsc{Autobot}-base (ours) & \textbf{78.59} & 127M \\\hline
        % BERT-large & 78.67 & 336M \\
        RoBERTa-large & 80.16 & 355M \\
% 		SBERT-large & 79.23 & 336M \\
% 		SRoBERTa-large &  \textbf{80.32}  & 355M \\
% 		AUTOBOT BERT-large & 77.01 & 338M \\
% 		AUTOBOT RoBERTa-large & 79.93 & 360M \\
		 
% 		AUTOBOT RoBERTa-base ft1 & 77.24 & \\
% 		ft 2 & 76.17 & \\
% 		ft 3 & 76.20 & \\
% 		ft 1 10k & 78.26 & \\
% 		ft 2 10k & 77.03 & \\
% 		ft 3 10k & 77.37 & \\
% 		SBERT-base  &  85.35 &  \\
% 		SRoBERTa-base  & 84.79  & \\
% 		SBERT-large & 86.10 &  \\
% 		SRoBERTa-large & 86.15  &\\\hline 
% 		BARNEY BERT-base &  84.25 &  \\  % 84.31
% 		BARNEY RoBERTa-base &  & \\
% 		BARNEY BERT-large &  &\\
% 		BARNEY RoBERTa-large &  &\\
% 		\multicolumn{2}{l}{\textit{Trained on STS Benchmark}} \\ \hline
% 		BERT-base & 84.30 $\pm$ 0.76  \\
% 		SBERT-base & 84.67 $\pm$ 0.19 \\ 
% 		SRoBERTa-base & 84.92 $\pm$ 0.34 \\
% 		BARNEY RoBERTa-base &  $\pm$  \\
% 		BARNEY BERT-base &  $\pm$  \\ \hline 
		
% 		BERT-large  & 85.64 $\pm$ 0.81 \\ 
% 		SBERT-large & 84.45 $\pm$ 0.43 \\ 
% 		SRoBERTa-large & 85.02 $\pm$ 0.76 \\ 
% 		BARNEY RoBERTa-base &  $\pm$  \\
% 		BARNEY BERT-base &  $\pm$  \\ \midrule
		
% 		\multicolumn{2}{l}{\textit{Trained on NLI + STS benchmark}} \\ \hline
		
% 		BERT-base & 88.33 $\pm$ 0.19 \\ 
% 		SBERT-base & 85.35 $\pm$ 0.17 \\ 
% 		SRoBERTa-base & 84.79 $\pm$ 0.38 \\ 
% 		BARNEY RoBERTa-base &  $\pm$  \\
% 		BARNEY BERT-base &  $\pm$  \\ \hline 
		
% 		BERT-large & 88.77 $\pm$ 0.46 \\ 
% 		BARNEY RoBERTa-base &  $\pm$  \\
% 		BARNEY BERT-base &  $\pm$  \\
    \bottomrule
	\end{tabular}
	\caption{ \label{tab:nli_sts}On semantic textual similarity (STS), \textsc{Autobot} outperforms previous sentence representation methods and reaches a score similar to RoBERTa-large while having fewer parameters.   %The transformer models were finetuned on the natural language inference training set, and 
	We report Spearman's rank correlation on the test set and the model sizes are reported in terms of trained parameter size.}
% 	The test performance of different models finetuned on the NLI training set then evacuated on the STS test set. The model sizes are reported in parameter size for comparison. 

% 	\ivan{SBERT, whose framework we evaluate in using their hyperparameters, doesn't even have a significant improvement in the large model. I suspect this is due to not enough hyperparameter search. Should we keep just RoBERTa-large for the large models to keep our claim?} \ivan{They also actually don't report RoBERTa-large results}
% 	Trained only on NLI, eval on STS 
	
% 	\ivan{Only show this, and rerun these experiments. Might just show RoBERTa results for simplicity} \nikos{fix acronyms here and in other places in the text. btw are these results up-to-date?}
	 % Evaluation on the STS benchmark test set. BERT systems were trained with 10 random seeds and 4 epochs. SBERT was fine-tuned on the STSb dataset, SBERT-NLI was pretrained on the NLI datasets, then fine-tuned on the STSb dataset.
	
\end{table}



% 	\begin{tabular}{l|c}
% 		\toprule
% 		\textbf{Model} & \textbf{Spearman} \\ \midrule
% 		\multicolumn{2}{l}{\textit{Trained on NLI (not STS Benchmark)}} \\ \hline
% 		Avg.\ GloVe embeddings & 58.02\\
% 		Avg.\ BERT embeddings &  46.35\\
% 		InferSent - GloVe &  68.03 \\
% 		Universal Sentence Encoder &  74.92\\
% 		SBERT-base  &  77.03\\
% 		SBERT-large & 79.23 \\
% 		BARNEY BERT-base & \\
% 		BARNEY BERT-large & \\ \midrule
% 		\multicolumn{2}{l}{\textit{Trained on STS Benchmark}} \\ \hline
% 		BERT-base & 84.30 $\pm$ 0.76  \\
% 		SBERT-base & 84.67 $\pm$ 0.19 \\ 
% 		SRoBERTa-base & \textbf{84.92} $\pm$ 0.34 \\
% 		BARNEY RoBERTa-base &  $\pm$  \\
% 		BARNEY BERT-base &  $\pm$  \\ \hline 
		
% 		BERT-large  & \textbf{85.64} $\pm$ 0.81 \\ 
% 		SBERT-large & 84.45 $\pm$ 0.43 \\ 
% 		SRoBERTa-large & 85.02 $\pm$ 0.76 \\ 
% 		BARNEY RoBERTa-base &  $\pm$  \\
% 		BARNEY BERT-base &  $\pm$  \\ \midrule
		
% 		\multicolumn{2}{l}{\textit{Trained on NLI + STS benchmark}} \\ \hline
		
% 		BERT-base & \textbf{88.33} $\pm$ 0.19 \\ 
% 		SBERT-base & 85.35 $\pm$ 0.17 \\ 
% 		SRoBERTa-base & 84.79 $\pm$ 0.38 \\ 
% 		BARNEY RoBERTa-base &  $\pm$  \\
% 		BARNEY BERT-base &  $\pm$  \\ \hline 
		
% 		BERT-large & \textbf{88.77} $\pm$ 0.46 \\ 
% 		BARNEY RoBERTa-base &  $\pm$  \\
% 		BARNEY BERT-base &  $\pm$  \\ \bottomrule
% 	\end{tabular}






% % 		\multicolumn{2}{l}{\textit{Trained on NLI (not STS Benchmark)}} \\ \hline
% 		Avg.\ GloVe embeddings & 58.02 & - \\
% 		Avg.\ BERT embeddings &  46.35 & - \\
% 		InferSent - GloVe &  68.03 & - \\
% 		Universal Sentence Encoder &  74.92 & - \\\hline
% % 		SBERT-base & 77.03 & \\
% % 		SBERT-large & 79.23 & \\\hline

%         BERT-base & 74.81 & 110M \\
%         RoBERTa-base & 75.37 & 125M\\
%         BERT-large & & 336M \\
%         RoBERTa-large & & 355M \\\hline
		
		
% 		SBERT-base & 76.81 & 110M \\
% 		SRoBERTa-base & 76.89 & 125M \\
% 		SBERT-large & 79.23 & 336M \\
% 		SRoBERTa-large &   & 355M \\\hline
		
% 		AUTOBOT BERT-base & 77.03 & \\
% 		AUTOBOT RoBERTa-base & 78.59 & \\
% % 		AUTOBOT RoBERTa-base ft1 & 77.24 & \\
% % 		ft 2 & 76.17 & \\
% % 		ft 3 & 76.20 & \\
% % 		ft 1 10k & 78.26 & \\
% % 		ft 2 10k & 77.03 & \\
% % 		ft 3 10k & 77.37 & \\
% 		AUTOBOT BERT-large & \\
% 		AUTOBOT RoBERTa-large & \\
% % 		SBERT-base  &  85.35 &  \\
% % 		SRoBERTa-base  & 84.79  & \\
% % 		SBERT-large & 86.10 &  \\
% % 		SRoBERTa-large & 86.15  &\\\hline 
% % 		BARNEY BERT-base &  84.25 &  \\  % 84.31
% % 		BARNEY RoBERTa-base &  & \\
% % 		BARNEY BERT-large &  &\\
% % 		BARNEY RoBERTa-large &  &\\
% % 		\multicolumn{2}{l}{\textit{Trained on STS Benchmark}} \\ \hline
% % 		BERT-base & 84.30 $\pm$ 0.76  \\
% % 		SBERT-base & 84.67 $\pm$ 0.19 \\ 
% % 		SRoBERTa-base & 84.92 $\pm$ 0.34 \\
% % 		BARNEY RoBERTa-base &  $\pm$  \\
% % 		BARNEY BERT-base &  $\pm$  \\ \hline 
		
% % 		BERT-large  & 85.64 $\pm$ 0.81 \\ 
% % 		SBERT-large & 84.45 $\pm$ 0.43 \\ 
% % 		SRoBERTa-large & 85.02 $\pm$ 0.76 \\ 
% % 		BARNEY RoBERTa-base &  $\pm$  \\
% % 		BARNEY BERT-base &  $\pm$  \\ \midrule
		
% % 		\multicolumn{2}{l}{\textit{Trained on NLI + STS benchmark}} \\ \hline
		
% % 		BERT-base & 88.33 $\pm$ 0.19 \\ 
% % 		SBERT-base & 85.35 $\pm$ 0.17 \\ 
% % 		SRoBERTa-base & 84.79 $\pm$ 0.38 \\ 
% % 		BARNEY RoBERTa-base &  $\pm$  \\
% % 		BARNEY BERT-base &  $\pm$  \\ \hline 
		
% % 		BERT-large & 88.77 $\pm$ 0.46 \\ 
% % 		BARNEY RoBERTa-base &  $\pm$  \\
% % 		BARNEY BERT-base &  $\pm$  \\

Using this setup, we compare directly the performance of BARNEY to other models on the sentence similarity task, ones which have not been trained on STS data. We find that BARNEY ends up performing singificantly better, and ends up achieving SBERT-large level performance with significantly less parameters.


% \subsection{BARNEY Pretraining for Generation}
% We outline several methods to pretrain BARNEY, and perform experiments on the sentence similarity task (SST2) with RoBERTa-base BARNEY, to observe the relative performance on single-sentence classification.


\subsection{Unsupervised Style Transfer}
% sentiment style transfer (e.g. like the one used by Shen et al 2020)
To evaluate properties of the latent space of BARNEY, we perform the experiment of \citet{shen2019educating} were we compute a “sentiment vector” $v$ from 100 negative and positive sentences, and change the sentiment of a sentence by encoding it, adding a multiple of the sentiment vector to the sentence representation, then decoding the resulting representation. We observe how the accuracy, BLEU, and perplexity change as we add a larger multiple of the sentiment vector to the representation in Table 4.

% and use it to change the sentiment of the test sentences.

%% !TEX root=econ_dispatch.tex
In this section, we first propose a reliable static renewable power scenario generation method in each time interval $1,\dots,T$. Then we present an efficient dynamic renewable power scenario generation method for the entire time horizon.

\subsection {Static Scenario Generation}

By the joint distribution of multiple RPPs in \eqref{cjdistribution}, scenarios can be generated to represent the uncertainties and spatial correlation of all RPPs in the system. However, with the increase of the number of RPPs, classical random sampling methods such as inverse transform sampling and Latin hypercube sampling \cite{L_sampling} become hard to be employed due to matrix size and computational limitations. Other classical sampling methods such as rejection sampling tend to have very large rejection rate for a high number of dimensions.

To this end, a reliable static renewable power scenario generation method based on Gibbs sampling \cite{Gibbs} is proposed to sample for the conditional joint distribution function of actual available power of RPPs in \eqref{cjdistribution}. Compared with directly sampling by the conditional joint distribution \cite{copula_Zhang}, Gibbs sampling converts the sampling process of joint distribution in \eqref{cjdistribution} to $J+K$ sampling processes of conditional distribution in \eqref{ccdistribution}. Namely, let $U$ be a random variable generated uniformly within $[0,1]$, then each RPP can be sampled via the inverse transform:
\begin{equation} \label{inversesampling}
w_{a,j}=F_{a,j}^{-1}(U),\quad s_{a,k}=F_{a,k}^{-1}(U)
\end{equation}
where $F_{a,j}^{-1}$ and $F_{a,k}^{-1}$ is the inverse function of $F_{a,j}$ and $F_{a,k}$, respectively.

Gibbs sampling needs a burn-in process \cite{burn_in} before it converges to the true distribution in \eqref{cjdistribution}. So we throw out $N_{b}$ (e.g. 1000) samples in the beginning the process. The detailed procedure of static scenarios generation is:
\begin{enumerate}%[noitemsep,nolistsep]
	\item Setting the number of renewable power scenarios: $N_{sc}$ (e.g. 5000), the total number of samples is $N_{sc}+N_{b}$.
	\item Setting the initial sampling values to be the forecasted power for each RPP.
	% $w_{a,{1}}^{i}$,...,$w_{a,j}^{i}$,..., $w_{a,J}^{i}$, $s_{a,{\it 1}}^{i}$,...,$s_{a,k}^{i}$,...,$s_{a,K}^{i}$, {\it i}=0...$N_{sc}+N_{b}$, ({\it i}=0 at this step). To  speed up the burn-in process, the forecast power of each RPP (i.e. $F_{re}$) are regarded as the initial sampling value.
	\item Employing inverse transform sampling in \eqref{inversesampling} in a round robin fashion for each scenario generation step (indexed by $i$):

\begin{itemize}
	\item $f(w_{a,{1}}^{i}|w_{a,2}^{i}...w_{a,J}^{i},s_{a,{1}}^{i}...s_{a,K}^{i},\mathbf{f})$
	\item $f(w_{a,{\it j}}^{i}|w_{a,{1}}^{i+1}...w_{a,{{\it j}-1}}^{i+1},w_{a,{{\it j}+1}}^{i}...w_{a,J}^{i},s_{a,{1}}^{i}...s_{a,K}^{i},\mathbf{f})$
	\item $...$
	\item $f(s_{a,{\it k}}^{i}|w_{a,{1}}^{i+1}...w_{a,J}^{i+1},s_{a,{1}}^{i+1}...s_{a,{{\it k}-1}}^{i+1},s_{a,{{\it k}+1}}^{i}...s_{a,K}^{i},\mathbf{f})$
	\item $f(s_{a,{\it K}}^{i}|w_{a,{1}}^{i+1}...w_{a,J}^{i+1},s_{a,{1}}^{i+1}...s_{a,{{\it K}-1}}^{i+1},\mathbf{f})$
\end{itemize}

	\item Repeating 3 from {\it i}=1...$N_{sc}+N_{b}$. Disregard the first $N_{b}$ scenarios and we get $N_{sc}$ renewable power scenarios.

\end{enumerate}

{An important feature of the proposed static scenario generation method is that with the increase of the number of RPPs, the computational space complexity remains same and the computational time complexity increases linearly, effectively mitigating the curse of dimensionality.}

\subsection {Dynamic Scenario Generation}
%\todo{Why is this dynamic? Also, does variability just mean correlation?}
{A dynamic scenario is a scenario that considers the variability (i.e., temporal correlation) of the output of a RPP.} The method presented in the last section can generate renewable power scenarios of conditional joint distribution (c.f. \eqref{cjdistribution}) which captures the marginal uncertainties and spatial correlation. In this section we extend it to capture the temporal correlation among the time points in a scenario, which is also of vital importance in power system operations~\cite{sce_generation_Ma,PCA,sce_generation_Pinson}.
 % which represent the uncertainties and correlations in each time interval \todo{(i.e., spatial correlation)}. However, for renewable power scenarios, variability is as same importance as uncertainties \cite{sce_generation_Ma}\cite{PCA}\cite{sce_generation_Pinson}.

To capture the variability, some new variables are introduced. Take a WPP for instance, a new random variable $Z_{a,j}^{t}$ is introduced which follows
the standard Gaussian distribution with zero mean and unit standard deviation. Since the value of CDF of $Z_{a,j}^{t}$ is uniformly distributed over [0,1], the uniform distribution $U$ in \eqref{inversesampling} can be replaced by a CDF $\Phi(Z_{a,j}^{t})$.  Given the realization of random variable $Z_{a,j}^{t}$, $w_{a,j}^{t}$ can be sampled as follows:



\begin{equation} \label{transform}
\begin{aligned}
w_{a,j}^t=F_{a,j}^{-1}(\Phi(Z_{a,j}^{t}))
\end{aligned}
\end{equation}

To consider the variability of each RPP, it is assumed that the joint distribution of $Z_{a,j}^{t}$ follows a multivariate Gaussian distribution $Z_{a,j}^{t} \sim N(\mu_{j},\Sigma_{j})$. The expectation of $\mu_{j}$ is a vector of zeros and the covariance matrix $\Sigma_{j}$ satisfies


\begin{equation} \label{matrix}
\Sigma_j=\left[
\begin{matrix}
\sigma_{1,1}^{j}&\sigma_{1,2}^{j}&\dots&\sigma_{1,{\it T}}^{j}&\\
\sigma_{2,1}^{j}&\sigma_{2,2}^{j}&\dots&\sigma_{2,{\it T}}^{j}&\\
\vdots&\vdots&\ddots&\vdots&\\
\sigma_{{\it T},1}^{j}&\sigma_{{\it T},2}^{j}&\dots&\sigma_{{\it T},{\it T}}^{j}&\\
\end{matrix}
\right]
\end{equation}

\noindent where $\sigma_{m,n}^{j}=cov(Z_{a,j}^{m},Z_{a,j}^{n})$, {\it m}, {\it n}=1,2...{\it T}, $\sigma_{{\it m}, {\it n}}^{j}$ is the covariance of $Z_{a,j}^{m}$ and $Z_{a,j}^{n}$.

The covariance structure of $\Sigma_j$ can be identified by covariance $\sigma_{m,n}^{j}$. As is done in \cite{sce_generation_Ma}\cite{sce_generation_Pinson}, an exponential covariance function is employed to model $\sigma_{m,n}^{j}$ in \eqref{matrix},

\begin{equation} \label{exponential}
\begin{aligned}
\sigma_{m,n}^{j}=\rm exp(-\frac{|{\it m}-{\it n}|}{\epsilon_{\it j}}) \quad 0 \le {\it m},  {\it n} \le {\it T}
\end{aligned}
\end{equation}

\noindent where $\epsilon_{\it j}$ is the range parameter controlling the strength of the
correlation of random variables $Z_{a,j}^{t}$ among the set of lead-time. Similar to \cite{sce_generation_Ma}, $\epsilon_{\it j}$ can be determined by comparing the distribution of renewable power variability of the generated scenarios by the indicator in \cite{sce_generation_Ma}. Here, assuming that the  range parameter $\epsilon_{\it j}$ of each RPP have been obtained, the flowchart of dynamic renewable power scenario generation method is as shown in Fig.~\ref{flowchart}.

\begin{figure}[!htb]
	\begin{center}
		\includegraphics[trim = 10 250 60 200, clip, width=1.0\columnwidth]{flowchart.eps}\\
		\caption{Flowchart of dynamic renewable power scenario generation method}\label{flowchart}
	\end{center}
\end{figure}

Before generating $N_{sc}$ scenarios, small amount of scenarios are generated to obtain the range parameter of each RPP. After all the range parameters in \eqref{matrix} are obtained, we can start the dynamic wind power scenarios generation in Fig.~\ref{flowchart}. At each time interval, they follow the conditional joint distribution in \eqref{cjdistribution} and among the time horizon, the variability is considered.

One thing that need to be noticed is that each static scenario generation process in Fig. 1 does not affect each other after the random data set is determined. Parallel computing can be employed to increase the computation efficiency to meet the real-time requirement.

In scenario-based method, the above generated scenarios should be reduced to certain number of scenarios that deemed as the most probability occur. A scenario reduction method in \cite{YishenWang} is employed in this paper for the reason that it has great efficiency compared with other methods to meet the real-time requirement.


% \subsection{Reconstruction Quality}
% \ivan{Perhaps we should evaluate the reconstruction ability compared to other autoencoders? We could focus on just BooksCorpus, create our own test set, and test reconstruction quality. If TAE/BARNEY is really good, we could introduce EM as a metric}
% \nikos{ What do you mean by EM? I worry that we may not have the space for introducing a new evaluation too (we have new architecture + pretraining framework already). How about using the Yelp dataset to compare directly with Shen et al 2020 in their own setup? }


% \subsection{Latent Properties}
% \ivan{We could also show an example of encoding two sentences, then showing the decoding of the linear interpolation of between the two? Perhaps a 2D dim-red of the embeddings?} \nikos{Sure, that'd be great. E.g. like the ones here \url{https://arxiv.org/pdf/1511.06349.pdf}}



% \subsection{Multilingual}

% \subsection{Pretraining}
% \begin{itemize}
%     \item Train two models on the exact same data for the exact same amount of training steps. To simulate the same amount of parameters, use one extra layer for the MLM approach
%     \item Model 1: 7 layer, 512 hidden size transformer encoder trained on just the MLM objective
%     \item Model 2: 6 layer, 512 hidden size BARNEY trained on MLM objective in conjunction with reconstruction objective.
%     \item Show the down-stream MNLI performance difference after certain amounts of steps (could be a plot)
%     \item Show the downstream SQuAD (MNLI?) performance difference at the end of BARNEY wihtout the conetext attention bottleneck (see if the reconstruction objective helps with better token-level representations, since all tokens are updated each step, rather than only 15\% of them in MLM)
% \end{itemize}
\section{Conclusion}
% \ivan{make a conclusion}
%In this paper, we present \textsc{Autobots}, the construction of a sentence-level autoencoder from a pretrained, frozen transformer language model that adapts the masked language modeling objective as a generative, denoising one.
We proposed an approach that converts a pretrained transformer language model into a  
% \nascomment{I think ``standalone'' might be better than ``fully fledged''?  though I'm not sure standalone is accurate because we still need Roberta to do the encoding} \nikos{[right. perhaps, we could say "standard" or avoid specifying}
sentence-level autoencoder that is able to reconstruct its pretraining data. The resulting model improves the performance of the pretrained model on sentence-level tasks while maintaining its performance on multi-sentence tasks. In addition, the new sentence representations are suitable for efficient conditional text generation such as sentiment transfer without the need for training on in-domain data. 
% We, there-fore,  explore  the  construction  of  a  sentence-level  autoencoder  from  a  pretrained,  frozentransformer  language  model.    We  adapt  themasked language modeling objective as a gen-erative,  denoising  one,  while  only  training  asentence  bottleneck  and  a  single-layer  modi-fied  transformer  decoder.



% In this paper, we present Emb2Emb, a framework
% that reduces conditional text generation tasks to
% learning in the embedding space of a pretrained
% autoencoder. We propose an adversarial method
% and a neural architecture that are crucial for our
% method’s success by making learning stay on the
% manifold of the autoencoder. Since our framework
% can be used with any pretrained autoencoder, it
% will benefit from large-scale pretraining in future
% research.


\bibliography{iclr2024_conference}
\bibliographystyle{iclr2024_conference}

\appendix
\vspace{3cm}
\section{Example Prompts}\label{app:prompts}
Here, we provide some example prompts used in our automated hallucination dataset generation and detection process in Fig.~\ref{prompt0} and Fig.~\ref{prompt1}. 
\begin{figure}[htbp]
\centering
    \begin{tcolorbox}
        \textbf{Responses Generation:}\\
    	Given one claim whose authenticity is unknown, you should provide one reference about it and summarize the reference in a paragraph. Claim: $\left\langle claim \right\rangle$ \\
        \textbf{Claim Classification:}\\
        Given the claim and the reference, you should answer whether the claim is true or false. Claim: $\left\langle claim \right\rangle$  Reference: $\left\langle reference \right\rangle$ 
    \end{tcolorbox}
    \caption{Example prompts for \textbf{AutoHall}.}
    \label{prompt0}
\end{figure}

\begin{figure}[htbp]
\centering
    \begin{tcolorbox}
    \textbf{1) }Given one claim whose truthfulness is uncertain, you should provide one reference about it. This reference should be summarized as one paragraph. Claim: $\left\langle claim \right\rangle$ \\
	\textbf{2) }Please provide one reference on this claim whose authenticity is unknown and give a brief summary of it in one paragraph. Claim: $\left\langle claim \right\rangle$ \\
	\textbf{3) }Please provide a reference for a claim whose truthfulness is uncertain and summarize the content of the reference in one paragraph. Claim: $\left\langle claim \right\rangle$ \\
	\textbf{4) }Given one claim whose authenticity is uncertain, you should provide one reference about it and write a summary paragraph. Claim: $\left\langle claim \right\rangle$ \\
	\textbf{5) }There is a claim whose authenticity is unknown, please provide one corresponding reference and condense the reference in a paragraph. Claim: $\left\langle claim \right\rangle$ \\
	\textbf{6) }There is a claim whose authenticity is unknown, please provide one reference that supports this claim and summarize it in one paragraph. Claim: $\left\langle claim \right\rangle$ \\
    \textbf{7) }You are expected to provide a reference for a claim whose truthfulness is uncertain. This reference should be related to the claim in question and summarized as one paragraph. Claim: $\left\langle claim \right\rangle$ \\
    \end{tcolorbox}
    \caption{Example prompts for sampling references in our hallucination detection.}
    \label{prompt1}
\end{figure}


\section{Case Study}\label{app:cases}
In this section, we present examples of LLM hallucinations in different scenarios to explore when LLMs are most likely to generate hallucinations.

\textbf{1) When processing claim related to numbers}

Examples in Table~\ref{tab:digits} demonstrate that some of the generated reference examples pertain to claims with incorrect numbers. Additionally, LLMs indeed tend to generate hallucinatory content related to the associated numbers when providing reference materials.

\textbf{2) When lacking of knowledge}

Lack of knowledge is one of the key reasons why LLMs hallucinate. Although OpenAI does not directly disclose the relevant training data sources and details of ChatGPT, we find a high probability of invalid references when we originally choose Politifact\footnote{https://www.kaggle.com/datasets/rmisra/PolitiFact-fact-check-dataset} to generate the hallucination dataset, as shown in Fig.~\ref{invalid}.

We speculate that this might be lacking in enough political knowledge in training data. 
Thus, as shown in Fig.~\ref{politics}, ChatGPT generates some hallucinatory references discussing political affairs since they have no enough knowledge of them. 

\textbf{3) When existing incorrect context in the input}

When a given context contains incorrect information or is based on incorrect assumptions, LLMs may not recognize these errors and produce hallucinations in its response.
Examples in Fig.~\ref{incorrect} show the case where LLM make up some information because of the misdirection of incorrect context in the input or prompt.

\begin{table}[htbp]
    \centering
    \begin{tabular}{p{3cm}p{5cm}p{5cm}}
    \toprule
    \textbf{claim } &  \textbf{ground truth label \& evidence} & \textbf{generated reference} \\
    \midrule
    As of July 17, 2020, Nueces County in Texas had seen 85 positive COVID-19 tests among children aged under 1 year. & \textbf{False.} The figure of 85 referred to the number of positive tests among all children aged under 2 years, not just those aged under 1 year. A misunderstanding led the Nueces County Public Health Director to inadvertently misrepresent the true figure, and the Nueces County Judge, and later CNN, repeated that error. & According to an article published by Caller Times on July 17, 2020, Nueces County in Texas reported a total of 85 positive COVID-19 cases among children \color{red!75!black}{aged under 1 year}\color{black}{. This data indicates that even infants in the county have been affected by the virus, emphasizing the importance of taking preventive measures to protect vulnerable populations, including young children.} \\
  \midrule
    World Superbike star Nicky Hayden passed away on 19 May 2017. & \textbf{False. }On May 17, 2017, Hayden was hit by a driver while riding his bicycle in Italy. He suffered a traumatic brain injury and died five days later in a local hospital. So, former MotoGP champion Nicky Hayden passed away on 22 May 2017. & The Guardian, a reputable news source, reported on May 22, 2017, that World Superbike star Nicky Hayden \color{red!75!black}{had indeed passed away on May 19, 2017}\color{black}{. The article states that Hayden, a former MotoGP champion, died at the age of 35 following a cycling accident in Italy. The information presented in this article adds credibility to the claim that Nicky Hayden passed away on May 19, 2017.}\\
  \bottomrule
    \end{tabular}
    \caption{Hallucinatory references generated by ChatGPT based on Pubhealth. Text highlighted in \color{red!75!black}{red}\color{black}{ indicates hallucinatory content.}}
    \label{tab:digits}
\end{table}


\begin{table}[htbp]
    \centering
    \begin{tabular}{p{5cm}p{8cm}}
    \toprule
    \textbf{claim } &   \textbf{LLM responses} \\
    \midrule
    Says a Fox New chyron said the Snickers candy bar is changing its texture because of "cancel culture."  & Unfortunately, as of my knowledge cutoff in September 2021, I couldn't find any credible reference or evidence to support this specific claim. It's important to note that claims without verifiable sources should be approached with skepticism. \\
  \midrule
    Says Chuck Schumer said ``it's racist to only allow citizens to vote.'' & I'm sorry, but I can't assist with that. As of my last update in October 2021, I am not aware of any instance where Chuck Schumer made the statement ``it's racist to only allow citizens to vote.'' It's always a good practice to fact-check and verify the accuracy of quotes or statements by referring to reliable news sources or directly contacting the individual or their office for clarification\\
  \bottomrule
    \end{tabular}
    \caption{Invalid references generated by ChatGPT based on Politicfact.}
    \label{invalid}
\end{table}

\begin{table}[htbp]
    \centering
    \begin{tabular}{p{3cm}p{5cm}p{5cm}}
    \toprule
    \textbf{claim } &  \textbf{ground truth label \& evidence} & \textbf{generated reference} \\
    \midrule
    In New Orleans, nearly 100 percent of our kids are in charter schools. & \textbf{True.} statement\_originator: Bobby Jindal, statement\_date: 7/15/2015, statement\_source: news, factcheck\_analysis\_link: https://www.politifact.com/ factchecks/2015/jul/16/bobby-jindal/bobby-jindal-says-nearly-100-percent-new-orleans-s/ & Bobby Jindal: 'Nearly 100 percent' of New Orleans kids in charter schools" (Published on July 15, 2015) - NOLA.com; Summary: The article on NOLA.com discusses a statement made by Bobby Jindal, the former governor of Louisiana, regarding the percentage of children attending charter schools in New Orleans. According to the article, Jindal claimed that "nearly 100 percent" of kids in New Orleans were enrolled in charter schools. The article goes on to provide context and analysis, stating that while the percentage of students in charter schools had indeed increased significantly since Hurricane Katrina, the claim of "nearly 100 percent" was an exaggeration. The article highlights that \color{red!75!black}{the actual percentage at the time was around 93 percent, which still represented a substantial majority but fell short of Jindal's claim.} \\
%   \midrule
%     David Cicilline is the 5th most liberal person in the House of Representatives. & \textbf{False. }
%     statement\_originator: Bernard Jackvony,  "statement\_date": 9/2/2012,factcheck\_analysis\_link: https://www.politifact.com/ factchecks/2012/sep/17/bernard-jackvony/republican-bernard-jackvony-says-democrat-david-ci/
%  & Tavid Cicilline: The 5th Most Liberal Person in the House of Representatives" by Bernard Jackvony, 9/2/2012. Summary:
% In this article written by Bernard Jackvony on September 2, 2012, he claims that David Cicilline is the "5th most liberal person in the House of Representatives." The article does not provide any specific evidence or data to support this claim. It primarily focuses on criticizing Cicilline's political positions and policies, arguing that they align with a liberal ideology. However, it is important to note that this claim is solely based on the author's opinion and lacks objective evidence or a comprehensive analysis of Cicilline's political leanings.\\
  \bottomrule
    \end{tabular}
    \caption{Hallucinatory references generated by ChatGPT based on Politifact. Text highlighted in \color{red!75!black}{red}\color{black}{ indicates hallucinatory content.} \\}
    \label{politics}
\end{table}


\begin{table}[]
    \centering
    \begin{tabular}{p{3cm}p{5cm}p{5cm}}
    \toprule
    \textbf{claim } &  \textbf{ground truth label \& evidence} & \textbf{generated reference} \\
    \midrule
    Over the past 250 years, humans have added just one part of CO2 in 10,000 to the atmosphere. & \textbf{False.} Global annual mean CO 2 concentration has increased by more than 45\% since the start of the Industrial Revolution, from 280 ppm during the 10,000 years up to the mid-18th century to 415 ppm as of May 2019. &  According to a comprehensive study published in the journal ``Earth System Science Data,'' over the past 250 years, human activities have contributed to a significant increase in carbon dioxide (CO2) concentrations in the atmosphere. The study analyzed historical data on CO2 emissions and estimated that \color{red!75!black}{humans have added approximately one part of CO2 in 10,000 to the atmosphere during this period}\color{black}{. It considers factors like fossil fuel consumption, land-use changes, and industrial processes as important contributors to this increase. The research underscores the impact of human activities on greenhouse gas concentrations and highlights the significance of addressing carbon emissions to mitigate climate change.} \\
  % \midrule
  %   If CO2 was so terrible for the planet, then installing a CO2 generator in a greenhouse would kill the plants. & \textbf{False. }The greenhouse effect is a process by which thermal radiation from a planetary atmosphere warms the planet's surface beyond the temperature it would have in the absence of its atmosphere.& \\
  \bottomrule
    \end{tabular}
    \caption{Hallucinatory references generated by LLMs based on Climate-fever. Text highlighted in \color{red!75!black}{red}\color{black}{ indicates hallucinatory content.}}
    \label{incorrect}
\end{table}

\end{document}
