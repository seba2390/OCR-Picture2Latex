% !TeX spellcheck = en_US
%  ----------------------------------------------------------------
% AMS-LaTeX Paper ************************************************
% **** -----------------------------------------------------------

\documentclass[12pt,reqno]{article}
%\usepackage[ams]{pdfslide}
%\usepackage[pdftex]{hyperref}
\usepackage{amsmath, amsthm, graphicx,amsfonts, amssymb,color}
\usepackage{bm}
\usepackage{hyperref}
%\usepackage{ulem}

%\usepackage[notref,notcite]{showkeys}
\newcommand{\lj}{\color{blue}}
\newcommand{\wt}{\color{magenta}}


%\pagestyle{myheadings}
\setlength{\topmargin}{-2cm} \setlength{\oddsidemargin}{0cm} \setlength{\evensidemargin}{0cm}
\setlength{\textwidth}{16truecm} \setlength{\textheight}{24truecm}
% ----------------------------------------------------------------
\vfuzz2pt % Don't report over-full v-boxes if over-edge is small
\hfuzz2pt % Don't report over-full h-boxes if over-edge is small
% THEOREMS -------------------------------------------------------
\newtheorem{thm}{Theorem}[section]
\newtheorem{cor}[thm]{Corollary}
\newtheorem{lem}[thm]{Lemma}
\newtheorem{prop}[thm]{Proposition}
\theoremstyle{definition}
\newtheorem{defn}[thm]{Definition}
\theoremstyle{remark}
\newtheorem{rem}[thm]{Remark}
\theoremstyle{example}
\newtheorem{exm}[thm]{Example}
\numberwithin{equation}{section}
% MATH -----------------------------------------------------------
\usepackage{mathrsfs}
\newcommand{\scr}[1]{\mathscr #1}

\newcommand{\norm}[2]{\left\|{#1}\right\|_{#2}}
\newcommand{\abs}[1]{\left\vert#1\right\vert}
\newcommand{\set}[1]{\left\{#1\right\}}
\newcommand{\eps}{\varepsilon}
\newcommand{\nn}{\nonumber}


 \def\d{\mathrm{d}}
\def\D{\scr D}
\def\R{\scr R}
\def\F{\scr F}
\def\G{\scr G}
\def\e{\scr E}
\def\K{\scr K}
%\def\bR{\scr R}
\def\cA{\mathcal A}
\def\cD{\mathcal D}
\def\cL{\mathcal L}
\def\cM{\mathcal M}
\def\cN{\mathcal N}
\def\cB{\mathcal B}
\def\bB{\mathbb B}
\def\bE{\mathbb E}
\def\bP{\mathbb P}
\def\bN{\mathbb N}
\def\bR{\mathbb R}
\def\bZ{\mathbb Z}
\def\d{\rm d}
%newcommand-------------------------------------------------
\def\bg{\begin}
\def\be{\bg{equation}}
\def\de{\end{equation}}
\def\bgar{\bg{eqnarray}}
\def\edar{\end{eqnarray}}
\newcommand{\nnb}{\nonumber}
\def\lb{\label}
\def\ct{\cite}
\def\l{\left}\def\r{\right}
%abbreviation-------------------------------
\def\fr{\frac}
\def\alp{\alpha}
\def\bt{\beta}
\def\gm{\gamma}
\def\Gm{\Gamma}
\def\dlt{\delta}
\def\Dlt{\Delta}
\def\eps{\epsilon}
\def\veps{\varepsilon}
\def\tht{\theta}
\def\Tht{\Theta}
\def\kp{\kappa}
\def\lmd{\lambda}
\def\Lmd{\Lambda}
\def\vro{\varrho}
\def\sgm{\sigma}
\def\Sgm{\Sigma}
\def\vph{\varphi}
\def\omg{\omega}
\def\Omg{\Omega}
\def\fa{\forall}
\def\emp{\emptyset}
\def\ex{\exists}
\def\nbl{\nabla}
\def\pat{\partial}
\def\ift{\infty}
\def\bca{\bigcap}
\def\bcu{\bigcup}
\def\lar{\leftarrow}
\def\Lar{\Leftarrow}
\def\rar{\rightarrow}
\def\bRar{\bRightarrow}
\def\lla{\longleftarrow}
\def\Lla{\Longleftarrow}
%\def\to{\longrightarrow}
\def\To{\Longrightarrow}
\def\lra{\leftrightarrow}
\def\Lra{\Leftrightarrow}
\def\llra{\longleftrightarrow}
\def\Llra{\Longleftrightarrow}

\def\q{\quad}
\def\mpb{\vskip6pt}
\def\gap{\text {\rm gap}}
\def\capp{\text {\rm cap}}
\def\diag{\text {\rm diag}}
\def\var{\text {\rm var}}
\def\ess{{\rm ess}}
\def\hess{{\rm Hess}}
\def\ric{{\rm Ric}}
\def\tr{{\rm tr}}
\def\lan{\langle}\def\ran{\rangle}

\def\[{\l[} \def\]{\r]}
\def\({\l(} \def\){\r)}

\def\hat{\widehat}
\def\bar{\overline}
\def\tld{\widetilde}

\def\dist{{\rm dist}}
\def\supp{{\rm supp\,}}
\def\sep{{\rm sep}}

%%%%%%%%%%%%%%%%%%%%%%%%%%%%%%%%%%%%%%%%%%%%%%%%%%%%%%%%%%%%%%%%%%%%%
\newcommand{\pfthm}[1]{\vskip.5cm \noindent\emph{Proof of Theorem \ref{#1}}}
\newcommand{\pfcor}[1]{\vskip.5cm \noindent\emph{Proof of Corollary \ref{#1}}}
\newcommand{\pfprop}[1]{\vskip.5cm \noindent\emph{Proof of Proposition \ref{#1}}}

%%%colors
\newcommand{\red}[1]{{\color{red} #1}}
\newcommand{\yel}[1]{{\color{yellow} #1}}
\newcommand{\blu}[1]{{\color{blue} #1}}
\newcommand{\grn}[1]{{\color{green} #1}}


 \def\beqlb{\begin{eqnarray}}\def\eeqlb{\end{eqnarray}}
 \def\beqnn{\begin{eqnarray*}}\def\eeqnn{\end{eqnarray*}}
\def\d{{\mbox{\rm d}}}
 \def\supp{{\mbox{\rm supp}}}
\def\dsum{\displaystyle\sum}
\def\dsup{\displaystyle\sup}
\def\dlim{\displaystyle\lim}
\def\dlimsup{\displaystyle\limsup}
\def\prodd{\displaystyle\prod}

\def\dinf{\displaystyle\inf}
 \def\bfE{\mbox{\boldmath $E$}}\def\bfP{\mbox{\boldmath $P$}}
 \def\bfQ{\mbox{\boldmath $Q$}}

% ----------------------------------------------------------------

\title{\bf  {Variational principles for asymptotic variance of general Markov processes}}

\author{Lu-Jing Huang\qquad Yong-Hua Mao \qquad Tao Wang\thanks{Corresponding author:  wang\_tao@mail.bnu.edu.cn}}

%\author{\small
%  Lu-Jing Huang,\thanks{College of Mathematics and Informatics, Fujian Normal University, Fuzhou 350007, P.R. China.}
%  \and
%  \small Kyung-Youn Kim,\thanks{Department of Mathematical Sciences, National Chengchi University, Taipei 116, Taiwan.}
%  \and
%  \small Yong-Hua Mao,\thanks{Laboratory of Mathematics and Complex Systems(Ministry of Education), School of Mathematical Sciences, Beijing Normal University, Beijing 100875, P.R. China.}
%  \and
%   \small and Tao Wang\footnotemark[3]
%}


%\baselineskip{18pt}
\date{}

\begin{document}

%%------------------------------------------------------------
 \maketitle


%%------------------------------------------------------------

\bg{abstract}

A variational formula for the asymptotic variance of general Markov processes is obtained.
As application, we get a upper bound of the mean exit time of reversible Markov processes, and some comparison theorems between the reversible and non-reversible diffusion processes.


\end{abstract}

{\bf Keywords:} Markov process, asymptotic variance, variational formula, the mean exit time, comparison theorem, semi-Dirichlet form

{\bf Mathematics subject classification(2020):} 60J25, 60J46, 60J60

%{\bf Running head} Variational principles for the exit time of Hunt processes


%%%%%%%%%%%%%%%%%%%%%%%%%%%%%%%%%%%%%%%%%%%%%%%%%%%%%%%%%%%%%%%%%%%%%%%%%%
\section{Introduction and main results}

Asymptotic variance is a popular criterion to evaluate the performance of Markov processes, and widely used in Markov chain Monte Carlo(see e.g. \cite{AL19,CCHP12,MDO14,Pe73,RR08}).
%Specifically, if a Markov process $\{X_t\}_{t\geq0}$ has stationary distribution $\pi$ on state space $S$, and  $f:S\rightarrow \bR$ is a $\pi$-integrable function, then $\pi(f):=\int_Sf(x)\pi(\d x)$ can be estimated by $\frac{1}{t}\int_0^tf(X_s)\d s$ for large enough $t$.
%The efficiency of this estimate can be measured by the asymptotic variance
%$$\lim_{t\rightarrow\infty}\frac{1}{t}\text{Var}_\pi\Big[\int_0^tf(X_s)\d s\Big].$$ It is known that the smaller asymptotic variance, the better estimate.

There are numerous studies of the asymptotic variance in the literature. For reversible Markov processes, the asymptotic variance can be presented by a spectral calculation, which brings a lot of applications (see \cite{DL01,KV86,RR97} etc.).
%Moreover,
The comparisons %study
on efficiency of reversible Markov processes, in terms of the asymptotic variance, has been extensively researched(see e.g. \cite{AL19,HM21+,LM08,Pe73,Ti98}). Recently, there are also some comparison results between reversible and non-reversible Markov processes, see e.g.  \cite{Bi16,CH13,Hw05,SGS10} for discrete-time Markov chains, and \cite{DLP16,HNW15,RS15} for diffusions.
However,  the study of the asymptotic variance of non-reversible Markov processes is still a challenge since  the lack of spectral theory of non-symmetric operators.  Very recently, \cite{HM21+} gives some variational formulas for the asymptotic variance of general discrete-time Markov chains by  solving Poisson equation, and obtains some estimates and comparison results of  the asymptotic variance.

In this paper we extend the results in \cite{HM21+} to the general Markov process by constructing the weak solution of Poisson equation with the help of the semi-Dirichlet form.
%For that, we introduce some notations firstly.

Let $X=\{X_t\}_{t\geq0}$ be a positive recurrent (or ergodic) Markov process on a Polish space $(S,\mathcal{S})$, with strongly continuous contraction transition semigroup $\{P_t\}_{t\geq0}$ and stationary distribution $\pi$. Denote $L^2(\pi)$ by the space of square integrable functions with scalar product
$
(u,v):=\int_S u(x)v(x)\pi(dx)
$
and norm $||u||=( u,u)^{1/2}$. Let $L^2_0(\pi)$ be the subspace of functions in $L^2(\pi)$ with mean-zero, i.e.
$$
L^2_0(\pi)=\{u\in L^2(\pi):\ \pi(u):=\int_S u(x)\pi(dx)=0\}.
$$
Denote $(L,\mathscr{D}(L))$ by the infinitesimal generator induced by $\{P_t\}_{t\geq0}$ in $L^2(\pi)$.

 Define a bilinear form associated with $L$ as
$$
\e(u,v)=( -Lu,v),\quad u,v\in\scr{D}(L),
$$
and for  $\alpha\geq0$,
$$
\e_\alpha(u,v):=\e(u,v)+\alpha(u,v),\quad u,v\in\scr{D}(L).
$$
Since $\pi$ is the stationary distribution and $P_{t}$ is $L^{2}$-contractive, for any  $u \in \mathscr{D}(L) $,
$$
\mathscr{E}(u, u)=-\pi(u L u)=\lim_{t\rar0}\frac{\pi(u^2)-\pi\left(\left(P_{t} u\right)^{2}\right)}{t}\ge0,
$$
that is, $\scr{E}$ is non-negative define.

  $(L,\mathscr{D}(L))$ is said to satisfy the sector condition if there exists a constant $K>0$ such that
\begin{equation}\label{weak-sect}
	|\e(u,v)|\leq K\e(u,u)^{1/2}\e(v,v)^{1/2},\quad \text{for  }u,v\in\scr{D}(L).
\end{equation}
Remarkably, if process $X$ is reversible:
	$$
	\pi(\d x)P_t(x,\d y)=\pi(\d y)P_t(y,\d x),\quad \text{for all }t\geq0,\ \pi\text{-a.s. }x,y\in S,
	$$
	%In this case, it is known that $\{P_t\}_{t\geq 0}$ can be viewed as self-adjoint operators in $L^2_0(\pi)$.
	then the sector condition is always true with $K=1$ by Cauchy-Schwartz inequality.

Under the sector condition, we can obtain a unique semi-Dirichlet form $(\e,\scr{F})$, where $\scr{F}$ is the completion of $\scr{D}(L)$ with respect to $\bar{\e}_1^{1/2}$ ($\bar{\e}_1$ is the symmetric part of $\e_1$), see \cite[Chapter 1, Theorem 2.15]{MR92}) for more details.


%For fixed $\alpha\geq0$, define
%$$
%\e_\alpha(u,v)=\e(u,v)+\alpha(u,v),\quad u\in\scr{D}(L), v\in\scr{F}.
%$$
 %We say that {\wt  $(\e,\scr{F})$}
%satisfies the weak sector condition, {\wt if}  there exists constants $K>0$ and $\alpha\geq 0$, such that
%\begin{equation}\lb{weak-sect}
%	|\e_\alpha(u,v)|\leq K\e_\alpha(u,u)^{1/2}\e_\alpha(v,v)^{1/2},\quad \text{for %all }u,v\in\scr{F}.
%\end{equation}
%{\wt If the above weak sector condition holds, then by the denseness (see \cite[Theorem 2.13]{MR92}), we can extend $(\e,\scr{F})$ on $\scr{F}\times\scr{F}$.}

%Remarkably, for symmetric Dirichlet form, i.e.,
%$$
%\e(u,v)=\e(v,u)\quad \text{for all }u,v\in\scr{F},
%$$
%the weak sector condition is always true with $K=1$ and $\alpha=0$ by Cauchy-Schwartz inequality.

We say that  the semigroup $\{P_t\}_{t\geq0}$ is $L^2$-exponentially ergodic, if there exist constants $C, \lambda_1>0$ such that for  $u\in L_0^2(\pi)$,
$$
\|P_tu\|\leqslant C\|u\|\rm{e}^{-\lambda_1 t}.
$$
%We note that if \eqref{exp-const} holds, then process $X$ is $\pi$-a.s. exponentially ergodic, that is, for $\pi$-a.s. $x\in S$, there exists  $C(x)<\infty$ such that \be\label{exp-erg}\|P_t(x,\cdot)-\pi\|_{\mathrm{Var}}\leqslant C(x)\rm{e}^{-\lambda_1 t}, \\de
%where $\|\mu\|_{\text{Var}}=\sup_{|f|\leq1}|\mu(f)|$ is the total variation, see e.g. \cite[Page 158]{cmf05} for more details.
It is well known  that when process $X$ is reversible,
$C$ can be chosen as $1$ and (the optimal) $\lambda_1$ is nothing but  the spectral gap:
\begin{equation}\lb{spec-gap}
\lambda_1=\inf\{\e(u,u):\ u\in\scr{F},\pi(u)=0\ \text{and }\pi(u^2)=1\}.
\end{equation}




Now for $f\in L^2_0(\pi)$, we consider the following asymptotic variance for $X$ and $f$:
\be\lb{av-f}
\sigma^2(X,f)=\limsup_{t\rightarrow\infty}\bE_\pi\Big[\Big(\frac{1}{\sqrt{t}}\int_0^tf(X_s)\d s\Big)^2\Big].
\de
%if the limit exists.

Under the $L^2$-exponential ergodicity and the sector condition, our first main result presents a variational formula for the asymptotic variance as follows.

\begin{thm}\lb{main1}
Suppose that the semigroup $\{P_t\}_{t\geq0}$ associated with process $X$ is
$L^2$-exponentially ergodic. Then the limit in \eqref{av-f} exists and is finite for  $f\in L^2_0(\pi)$. If in addition the associated semi-Dirichlet form $(\e,\scr{F})$ satisfies the sector condition, then for $f\in L^2_0(\pi)$,
\begin{equation}\lb{vf-av}
2/\sigma^2(X,f)=\inf_{u\in\scr{M}_{f,1}}\sup_{v\in\scr{M}_{f,0}}\e(u+v,u-v),
\end{equation}
where $\scr{M}_{f,\delta}=\{u\in\scr{F}:(u,f)=\delta\},\ \delta=0,1$.

Particularly, if process $X$ is reversible, then \eqref{vf-av} is reduced to
\begin{equation}\lb{vf-av-r}
2/\sigma^2(X,f)=\inf_{u\in\scr{M}_{f,1}}\e(u,u).
\end{equation}
\end{thm}

\begin{rem}
\begin{itemize}
\item[(1)] For fixed $f\in L^2_0(\pi)$, from the proof below we will see that functions $Gf:=\int_0^\infty P_tf\d t$ and $G^*f:=\int_0^\infty P^*_t f\d t$ are both in $\scr{F}$, here $P^*_t$ is the dual operator of $P_t$ in $L^2(\pi)$. This is a main reason that we need the semi-Dirichlet form $(\e,\scr{F})$ in \eqref{vf-av}. However, if the generator $L$ is bounded in $L^2(\pi)$, then
 $\scr{D}(L)=L^2(\pi)$, so that $Gf,G^*f\in\scr{D}(L)$. In this case,
$$
2/\sigma^2(X,f)=\inf_{\substack{u\in L^2(\pi),\\ \pi(uf)=1}}\sup_{\substack{v\in L^2(\pi),\\ \pi(vf)=0}}((-L)(u+v),u-v).
$$
The proof of this result can be obtained immediately by replacing $P-I$ by $L$ in the proof of \cite[Theorem 1.1]{HM21+}.

\item[(2)] The assumption of the $L^2$-exponential ergodicity of $\{P_t\}_{t\geq0}$ is not too strong for non-reversible Markov processes, since \cite{Ha05} gives a geometrically ergodic Markov chain such that the asymptotic variance is infinite for some $f\in L_0^2(\pi)$.

\item[(3)] Variational formula for the asymptotic variance has been studied in \cite[Chapter 4]{KLO12}. It is based on a variational formula for positive definite operators in analysis and resolvent equations. Here we obtain a new  variational formula.


\end{itemize}
\end{rem}

As a direct application of Theorem \ref{main1}, bound of the mean exit time of the process is obtained. For that, let $\Omega\subset S$ be an open set, denote by $\tau_\Omega=\inf\{t\geq0:X_t\notin\Omega\}$  the first exit time from $\Omega$ of process $X$.


\begin{cor}\lb{exit-time}
Suppose that process $X$ is reversible with $L^2$-exponentially ergodic semigroup $\{P_t\}_{t\geq0}$ and stationary distribution $\pi$. Let $\Omega\subset S$ be an open set with $\pi(\Omega)\in(0,1)$, then
$$
\bE_\pi\tau_\Omega\leq \frac{\pi(\Omega)}{2\lambda_1\pi(\Omega^c)},
$$
 where $\lambda_1$ is the spectral gap defined in $\eqref{spec-gap}$.
\end{cor}

Note that in \cite[Remark 3.6(1)]{HKMW20}, we gave another upper bound for the mean exit time. Explicitly, $\bE_\pi\tau_\Omega\leq 1/(\lambda_1\pi(\Omega^c))$ for open set $\Omega\subset S$ satisfying $\pi(\Omega^c)>0$. It is obvious that the upper bound in Corollary \ref{exit-time} is more precise than that.












For the reversible case,  similar to \cite[Theorem 1.3]{HM21+}, we could derive variational formula \eqref{vf-av-r} without the assumption of the $L^2$-exponential ergodicity. Since the proof is quite similar, we omit it in this paper.

\begin{thm}\lb{main-r}
Suppose that $X$ is a reversible ergodic Markov process with stationary distribution $\pi$. Then for fixed $f\in L^2_0(\pi)$,
$$
2/\sigma^2(X,f)=\inf_{u\in\scr{M}_{f,1}}\e(u,u).
$$
\end{thm}


Note that in Theorem \ref{main-r}, maybe $\sigma^2(X,f)=\infty$ for some $f\in L^2_0(\pi)$.


The remaining part of this paper is organized as follows. In Section \ref{sect-eg} we apply our main result in two situations.
The first application is extending the comparison result for the asymptotic variance of one dimensional diffusions in \cite[Theorem 1]{RR14} to multi-dimensional reversible diffusions.
We note that \cite[Theorem 1]{RR14} is proved by discrete approximation which is different from our idea, and the less assumptions are requested in our proof. Another application is a comparison result between reversible and non-reversible diffusions on Riemannian Manifolds, which shows the asymptotic variance of a non-reversible diffusion is smaller.
The similar result can be found in \cite{DLP16,HNW15}(for example, \cite{HNW15} proves a similar result on compact manifolds by using a spectral theorem), we provide a complete different proof by the new variational formula.
%The first application (see Section \ref{Lang-rev}) concerns the comparison of the asymptotic variances between different reversible  diffusions having same stationary distribution (see Theorem \ref{compa-r} below). Another application, presented in Section \ref{nonrev-diff}, is a comparison result between reversible and non-reversible diffusions on Riemannian Manifolds, which shows the asymptotic variance of a non-reversible diffusion is smaller.
%That is, the non-reversible diffusion is better than its reversibilisation in the sense of being uniformly more efficient for estimating expectations of functionals.
 Finally, the proofs of Theorem \ref{main1} and Corollary \ref{exit-time} are given in Section 3.



%Indeed, the similar result for discrete-time case can be found in \cite{Bi16,CH13,HM21+,Hw05,SGS10}, for diffusions see e.g. \cite{DLP16,HNW15,RS15}.



%%%%%%%%%%%%%%%%%%%%%%%%%%%%%%%%%%%%%%%%%%%%%%%%%%%%%%%%%%%%%%%%%%%%

\section{Applications}\lb{sect-eg}
\subsection{Reversible  diffusions}\lb{Lang-rev}

%%%%%%%%%%	In particular, for fixed $f\in L^2_0(\pi)$, $\sigma^2(X^{ka},f)$ is non-increasing for $k\in(0,\infty)$.
%%%%%%%%%%%\end{thm}

%%%%%\begin{rem}The result similar to Theorem \ref{compa-1} was established in \cite[Theorem 1]{RR14} by discrete approximation.\end{rem}
 First, we recall the comparison theorem proved in \cite[Theorem 1]{RR14}.
Fix a $C^1$  probability density function $\mu:[I_1,I_2]\rightarrow (0,\infty)$,  where $-\infty\leq  I_1<I_2\leq\infty$.
 	Given a $C^1$ positive function $\eta$ on $[I_1,I_2]$ and consider a one-dimensional Langevin diffusion:
 	$$\d X^\eta_{t}=\eta\left(X^\eta_t\right) \d B_{t}+\left(\frac{1}{2} \eta^{2}\left(X^\eta_{t}\right) \log \mu^{\prime}\left(X^\eta_{t}\right)+\eta\left(X^\eta_{t}\right) \eta^{\prime}\left(X^\eta_{t}\right)\right) \d t.$$
 	Under some additional conditions (see \cite[Page 133]{RR14}), \cite{RR14} proves that for any $f\in L^2_0(\mu)$, and two $C^1$ positive functions $\eta,\eta_1$  on $[I_1,I_2]$ such that   $\eta_1(x)\leq \eta(x)$ for all $x\in[I_1,I_2]$, 	
 $$
 		\sigma^2(X^{\eta_1},f)\geq \sigma^2(X^{\eta},f).
$$
Note that in \cite{RR14}, the above conclusion is proved by discrete approximation. In fact, we can obtain the above result  by a direct calculation as follows. For convenience, we only consider the case on half-line.


Fix a $C^1$ probability density function ${\pi}:[0,\infty)\rightarrow (0,\infty)$.
%%with $\int_{[0,\infty)}\widetilde{\pi}(x)\d x<\infty$.
Given a $C^1$ positive function $a$ on $[0,\infty)$ and consider a one-dimensional diffusion  $X^a$ with reflecting boundary 0 and generator:
\begin{equation}\label{1d-generator}
  L_a=a(x)\frac{\mathrm{d}^2}{\mathrm{d}x^2}+b(x)\frac{\mathrm{d}}{\mathrm{d}x},
\end{equation} %where  $b(x)$ satisfies that $$\int_1^x\frac{b(t)}{a(t)}\d t=\log(a(x)\pi(x)).$$
where $b(x)=a(x)({\pi}'(x)/{\pi}(x))+a'(x)$. Let $\pi(\d x)=\pi(x)\d x$. It is easy to see that $L_a$ is symmetric on $L^2({\pi})$. Choose a point $x_0>0$ and set $$c(x)=\int_{x_0}^{x}\frac{b(y)}{a(y)}\mathrm{d}y\quad\text{and}\quad \varphi(x)=\int_{0}^x\mathrm{e}^{-c(y)}\d y.$$
So we have
\begin{equation}\label{pi}
	{\pi}(x)= {{\mathrm e}^{c(x)}{\pi}(x_0)a(x_0)}/{a(x)}.
\end{equation}
%%%% {\wt Let $$\pi(\d x):=\frac{{\pi}(x)\d x}{\int_0^\infty{\pi}(x)\d x}=\frac{{\rm e}^{c(x)}/a(x)}{\int_0^\infty \mathrm{e}^{c(z)}/a(z)\d z}\d x.$$ and} $\pi(x)=({{\rm e}^{c(x)}/a(x)})/ {\int_0^\infty \mathrm{e}^{c(z)}/a(z)\d z}$.
Assume that  $X^a$ is non-explosive,
%By using Theorem \ref{main-r}, we can prove \eqref{comparison form} directly. In fact, the Dirichlet form associated with the process $X^\sigma$ is \begin{equation}\label{df-1}	\e^\sigma(u,v)=\frac{1}{2}\int_{\bR^1}\sigma^2  u' v'\d \pi,\quad \text{for}\ u,v\in \scr{F}^\sigma:=\{u\in L^2(\pi): \e^\sigma(u,u)<\infty\}.\end{equation} Since $a_1\leq a$, by \eqref{df-1} it is easy to check that $\scr{F}_{\sigma_1}\supseteq \scr{F}_{\sigma}$ and $$ \e_{\sigma_1}(u,u)\leq \e_{\sigma}(u,u)\quad \text{for all }u\in \scr{F}_{\sigma}.$$
%Fix $f\in L^2_0(\pi)$. The inequality \eqref{A12} is trivial when $\sigma^2(X^{\sigma_1},f)=\infty$. Now assume that $\sigma^2(X^{\sigma_1},f)<\infty$. It follows from Theorem \ref{main-r} that
%$$2/\sigma^2(X^{\sigma_1},f)=\inf_{\substack{u\in\scr{F}_{\sigma_1},\\\pi(fu)=1}}\e_{\sigma_1}(u,u)\leq \inf_{\substack{u\in\scr{F}_{\sigma},\\\pi(fu)=1}}\e_{A}(u,u)=2/\sigma^2(X^{\sigma},f).$$
that is,
$$\int_{0}^{\infty}\varphi'(y)\pi([0,y])\mathrm{d}y=\infty,
$$
then  $X$ is ergodic with stationary distribution $\pi(\d x)$(see e.g. \cite[Table 5.1]{cmf05}).


For fixed function $f\in L^2_0(\pi)$, consider Poisson equation $-L_au=f$.  By some direct calculations and \eqref{pi}, the equation has strong solution
$$			u(x)=\int_{0}^{x}\mathrm{e}^{-c(y)}\left(\int_{y}^{\infty}f(z)\frac{\mathrm{e}^{c(z)}}{a(z)}\mathrm{d}z\right)\mathrm{d}y
=\frac{1}{\pi(x_0)a(x_0)}\	\int_{0}^{\infty}f(z)\varphi(x\wedge z)\pi(\d z).
$$

Since $\sigma^2(X,f)=2(u,f)$ by Lemma \ref{Gf-u} and \eqref{representation} below, from $\pi(f)=0$ and the integration by parts we have that 
\be\lb{exp-reps}
\aligned \frac{1}{2}\sigma^2(X^a,f)&=\frac{1}{ a(x_0)\pi(x_0)}\int_{0}^{\infty}\int_{0}^{\infty}f(x)f(y)\varphi(x\wedge y)\pi(\d y)\pi(\d x)\\ &=\frac{2}{ a(x_0)\pi(x_0)}\int_{0}^{\infty}\varphi(x)f(x)\int_{x}^{\infty}f(y)\pi(\d y)\pi(\d x)\\
&=-\frac{2}{a(x_0)\pi(x_0)}\int_{0}^{\infty}\varphi(x)f(x)\int_{0}^{x}f(y)\pi(\d y)\pi(\d x)\\
&=-\frac{1}{ a(x_0)\pi(x_0)}\int_{0}^{\infty}\varphi(x)\Big[\Big(\int_0^x f(y)\pi(\d y)\Big)^2\Big]'\d x\\
&=\int_0^\infty\Big(\int_0^x f(y)\pi(\d y)\Big)^2\frac{1}{a(x)\pi(x)}\d x.
\endaligned
\de

Using the above representation, we obtain the following comparison theorem directly.

\begin{thm}\lb{compa-1}
Let $a,a_1$ be two $C^1$ positive function on $[0,\infty)$. Then Langevin diffusions $X^a$ and $X^{a_1}$, with generators of form \eqref{1d-generator}, possess the same stationary distribution $\pi$.
Moreover, if $a\geq a_1$, then for any $f\in L_0^2(\pi)$,
$$
		 \sigma^2(X^{a},f)\leq \sigma^2(X^{a_1},f).
$$
In particular, for fixed $f\in L^2_0(\pi)$, $\sigma^2(X^{ka},f)$ is non-increasing for $k\in(0,\infty)$.	
\end{thm}	


For multi-dimensional reversible diffusion processes, {explicit representation \eqref{exp-reps} for the asymptotic variance is difficult to obtain. However, we could use Theorem \ref{main-r} to get the similar comparison result as follows.

%although we can use Theorem \ref{main1} to get similar result, explicit representation \eqref{exp-reps} of the asymptotic variance is difficult to obtain.

Let $V\in C^2(\bR^d)$ with $\int_{\bR^d} \mathrm{e}^{V(x)}\mathrm{d} x<\infty$. Consider the reversible diffusion process $X^{A}$ generated by  elliptic operator
$$L_{A}=\sum_{i, j} a_{i j}(x) \frac{\partial^{2}}{\partial x_{i} \partial x_{j}}+\sum_{i} b_{i}(x) \frac{\partial}{\partial x_{i}},$$
where $A(x)=\left(a_{i j}(x)\right)_{1\leq i,j\leq d},x\in\bR^d$ are positive definite matrices with $a_{i j}\in  C^{2}\left(\mathbb{R}^{d}\right)$ and
$$
b_{i}(x)=\sum_{j} a_{i j}(x) \frac{\partial}{\partial x_{j}} V(x)+\sum_{j} \frac{\partial}{\partial x_{j}} a_{i j}(x).
$$
Assume that $X^A$ is non-explosive. By \cite[Theorem 4.2.1]{DZ96}, we see that process $X^{A}$ is ergodic with stationary distribution
$$\pi(\d x):=\frac{\mathrm{e}^{V(x)}}{\int_{\bR^d} \mathrm{e}^{V(y)}\mathrm{d} y}\mathrm{d} x.$$


Denote by $(\e_{A}(\cdot,\cdot),\scr{F}_A)$  the Dirichlet form associated with the process $X^{A}$. Explicitly, we see that
\begin{equation}\lb{df-r}
\e_{A}(u,v)=\int_{\bR^d}\nabla u\cdot A\nabla v\d \pi,\quad \text{for}\ u,v\in \scr{F}_A:=\{u\in L^2(\pi): \e_A(u,u)<\infty\}.
\end{equation}

\begin{thm}\lb{compa-r}
Let $V\in C^2(\bR^d)$ with $\int_{\bR^d} \mathrm{e}^{V(x)}\mathrm{d} x<\infty$, $A(x)=(a_{ij}(x))_{1\leq i,j\leq d}$ and $A_1(x)=(a^1_{ij}(x))_{1\leq i,j\leq d}$, $x\in \bR^d$ be positive definite matrices satisfying $a_{ij}, a^1_{ij}\in C^2(\bR^d)$ for $1\leq i,j\leq d$. Suppose that $A_1\leq A$ in the sense that $A(x)-A_1(x)$ is non-negative definite for all $x\in\bR^d$.
%$$z\cdot A_1(x)z\leq z\cdot A(x)z,\quad \text{for all }x,z\in \bR^d.$$
Then for any $f\in L^2_0(\pi)$,
\begin{equation}\lb{A12}
\sigma^2(X^{A_1},f)\geq \sigma^2(X^{A},f).
\end{equation}
 In particular, for fixed $f\in L^2_0(\pi)$, $\sigma^2(X^{kA},f)$ is non-increasing for $k\in(0,\infty)$.
\end{thm}
\begin{proof}
Since $A_1\leq A$, by \eqref{df-r} it is easy to check that $\scr{F}_{A_1}\supseteq \scr{F}_{A}$ and
$$
\e_{A_1}(u,u)\leq \e_{A}(u,u)\quad \text{for all }u\in \scr{F}_{A}.
$$

Fix $f\in L^2_0(\pi)$. The inequality \eqref{A12} is trivial when $\sigma^2(X^{A_1},f)=\infty$. Now assume that $\sigma^2(X^{A_1},f)<\infty$. It follows from Theorem \ref{main-r} that
$$
2/\sigma^2(X^{A_1},f)=\inf_{\substack{u\in\scr{F}_{A_1},\\\pi(fu)=1}}\e_{A_1}(u,u)
\leq \inf_{\substack{u\in\scr{F}_{A},\\\pi(fu)=1}}\e_{A}(u,u)=2/\sigma^2(X^{A},f).
$$
Hence, the proof is completed.
\end{proof}




%%%%%%%%%%%%%%%%%%%%%%%%%%%%%%%%%%%%%%%%%%%%%%%%%%%%%%%%%%%


\subsection{Non-reversible diffusions on Riemannian Manifolds}\lb{nonrev-diff}
In this section,we turn to non-reversible case. Let $M$ be a connected, complete Riemannian manifold  with empty boundary or convex boundary, and $\lan\cdot,\cdot\ran$ be the inner product  under the Riemannian metric.
Denote  $\d x$ and $\Delta$ by the Riemannian volume and  Laplace operator on $M$, respectively.

Let $\pi(\d x):=e^{-U(x)} \d x$ be a probability measure on $M$ with potential function $U\in C^{2}(M)$.
We consider the following diffusion operator:
\be\label{Diff}
\mathfrak{L} \varphi=\Delta \varphi-\lan\nabla U-Z,\nabla \varphi\ran,
\de
where $Z$ is a $C^1$ vector field on $M$.
% Assume that   the diffusion process $X=(X_t)_{t\geq0}$ associated with $\mathfrak{L}$ is non-explosive,
%	We also can describe \eqref{Diff} succinctly by Hodge-star operator. Let $\nabla$ be an affine connection and $\nabla_{\nu}^{*}$ be the Hodge star dual operator of $\nabla$ on $L^{2}(\nu)$, which is characterized by $$\nabla_{\nu}^{*} \theta=\nabla^{*} \theta+(\nabla U, \theta)$$
%for $\theta \in M,$ where $\nabla^{*}$ is the dual of $\nabla$ on $L^{2}(m)$.	Consider the generator $$\mathfrak{L}=-\frac{1}{2} \nabla_{\nu}^{*} \nabla+Z,$	where $Z$ is a smooth vector field.
Denote by $\mathfrak{L}^{*}$  the dual operator of $\mathfrak{L}$ on  $L^{2}(\pi)$:
$$\mathfrak{L}^{*}\varphi=\Delta\varphi-\lan\nabla U+Z,\nabla\varphi\ran-(\operatorname{div} Z-\lan\nabla U, Z\ran)\varphi,$$
where $\operatorname{div}$ is the divergence operator.
%on $L^{2}(\pi)$ defined by Stokes Theorem:
%	$$	\int_M \lan Z,\nabla \varphi\ran \d \pi=-\int_M \varphi\operatorname{div}( Z\mathrm{e}^{-U})\d x =:-\int_M \varphi \operatorname{div}_{\pi} Z  \d \pi, \quad \varphi \in C_{0}^{\infty}(M).	$$
It is well known that $\pi$ is the invariant measure of $\mathfrak{L}$ if and only if $(\mathfrak{L}^{*}1,\varphi)=0$  for $\varphi\in C_{0}^{\infty}(M),$ i.e.,
$$\int_M(\operatorname{div} Z-\lan\nabla U, Z\ran)\varphi\d \pi=\int_M\operatorname{div}( Z\mathrm{e}^{-U}) \varphi\d x=0.$$
From now on we assume that
\begin{equation}\label{A0}
\operatorname{div}( Z\mathrm{e}^{-U})\equiv 0.
\end{equation}
Then by \cite[Corollary 3.6]{brw01}, the diffusion $X$ with generator $\mathfrak{L}$ is ergdoic with stationary distribution $\pi$.


Denote the symmetric part of  $\mathfrak{L}$ with respect to $\pi$ by  $\bar{\mathfrak{L}}:=\Delta-\lan\nabla U,\nabla\ran.$
and let $\bar{X}$ be the diffusion generated by $\bar{\mathfrak{L}}$.



 Define $(\e,\scr{F})$ as the semi-Dirichlet form generated by $\mathfrak{L}$, and denote its symmetric part and antisymmetric part by $\bar{\e},\ \widehat{\e}$ respectively. So from the integration by parts and \eqref{A0}, we have
$$
\bar{\e}(\varphi, \phi)= \int_M\lan \nabla \varphi,\nabla \phi\ran \d \pi\quad \text{and}\quad \widehat{\e}(\varphi, \phi)=\int_M \phi\lan Z,\nabla\varphi\ran \d\pi  \quad \varphi, \phi \in C_{0}^{\infty}(M).
$$
 Indeed, it is easy to check that $(\bar{\e},\scr{F})$ is the Dirichlet form generated by $\bar{\mathfrak{L}}$.

We suppose that the following {\bf Assumption A} holds:
\begin{itemize}

\item[(\rm A1)] $|\Delta U| \leq \epsilon_*|\nabla U|^{2}+C_U$ for some $\epsilon_*<1$  and $C_U \geq 0;$

\item[(\rm A2)] there is a constant $K$ such that $|Z| \leq K(|\nabla U|+1)$;

\item[(\rm A3)] the symmetric Dirichlet form $(\bar{\e},\scr{F})$ satisfies the Poincar\'{e} inequality, i.e., there exists a constant $\lambda_1>0$ such that
$$\|\varphi\|^2\leq \lambda_1^{-1}\bar{\e}(\varphi,\varphi)\quad \text{for all } \varphi\in\scr{F},$$
where $\|\cdot\|$ is $L^2(\pi)$-norm.
\end{itemize}

We note that (A3) is equivalent to the $L^2$-exponential ergodicity of semigroup of diffusion $\bar{X}$.




\begin{lem}\lb{Lang-sec}
If {\bf Assumption A} and \eqref{A0} hold, then $(\mathscr{E},\scr{F})$ satisfies the sector condition \eqref{weak-sect}. Therefore, Theorem \ref{main1} holds for the diffusion $X$.
\end{lem}
\begin{proof}

	
%Since  $\operatorname{div}_{\nu} Z\equiv 0$,  for any $ f,g \in C_{0}^{\infty}(M)$,
%\begin{equation}
%\begin{split}
%	\int_Mg \lan Z, \nabla f\ran d \nu&=-\int_M f\operatorname{div}( Z\mathrm{e}^{-U}g) d x\\
%	&=-\int_M f\operatorname{div}( Z\mathrm{e}^{-U})g d x-\int_M f\lan \nabla g, Z\ran d \nu\\
%	&=-\int_M f\lan \nabla g, Z\ran d \nu.\\
%\end{split}
%\end{equation}	
Since $(\bar{\e},\scr{F})$ is symmetric, it  satisfies the sector condition, we only need to check the sector condition for the antisymmetric part $\widehat{\e}$.

Fix $\phi,\varphi\in C^\infty_0(M)$. By	Cauchy-Schwarz inequality and (A2) we have
\begin{equation}\lb{c-s}
	\begin{split}
		\int_M\lan\phi Z,\nabla \varphi\ran \d \pi&\leq K\int_M (|\nabla U|+1)|\phi\nabla \varphi|\d \pi\\
		&\leq K\int_M |\phi\nabla \varphi|d \pi+K\bar{\e}(\varphi,\varphi)^{1/2}	\||\nabla U| \phi\|	.
	\end{split}
\end{equation}
For the last term above, the integration by parts on manifold, Cauchy-Schwarz inequality and (A1)  yield that
\begin{equation}\label{i-p and c-s}
\begin{aligned}
	\||\nabla U| \phi\|^2 &=-\int_M \lan\phi^{2} \nabla U , \nabla e^{-U}\ran \d x
	=\int_M \operatorname{div}\left(\phi^{2} \nabla U\right) e^{-U} \d x \\
	&=\int_M \lan2 \phi \nabla \phi , \nabla U\ran \d \pi+\int_M \Delta U \phi^{2} \d \pi\\
&=\int_M \lan2 \phi \nabla \phi , \nabla U\ran \d \pi+\int_M\left(\epsilon_* |\nabla U|^{2}+C_U\right) \phi^{2} \d \pi.
\end{aligned}
\end{equation}

Now fix $\varepsilon>0$ such that $\epsilon_*+\varepsilon<1$. Combining inequality $|x y| \leq\left(x^{2} / \varepsilon+\varepsilon y^{2}\right) / 2$ with \eqref{i-p and c-s} and (A3) we have
%\begin{equation}\label{fe}
$$
\begin{aligned}
		\||\nabla U| \phi\|^2 &\leq 2 \int |\nabla \phi||\phi\nabla U| \mathrm{d} \pi+\int\left(\epsilon_* |\nabla U|^{2}+C_U\right) \phi^{2} \mathrm{~d} \pi \\
	&\leq \frac{1}{\varepsilon}\bar{\e}(\phi,\phi)+(\epsilon_*+\varepsilon)\|\phi|\nabla U|\|^{2}+C_U\|\phi\|^{2}\\
	&\leq \frac{1}{\varepsilon}\bar{\e}(\phi,\phi)+(\epsilon_*+\varepsilon)\|\phi|\nabla U|\|^2+C_U\lambda_1^{-1}\bar{\e}(\phi,\phi),
\end{aligned}
$$
%\end{equation}
which implies that
$$
\||\nabla U| \phi\|^2\leq\frac{\lambda_1+C_U\varepsilon}{(1-\epsilon_*-\varepsilon)\varepsilon\lambda_1}\bar{\e}(\phi,\phi).
$$
Combining this with \eqref{c-s} and (A3), we obtain that $\widehat{\e}$ satisfies the sector condition on $\scr{F}$.% Hence, the proof is completed.
\end{proof}

From Lemma \ref{Lang-sec} and Theorem \ref{main1}, we obtain the following comparison result.

\begin{thm}\lb{compr-diff}
Suppose that {\bf Assumption A} holds. Then for any $f\in L^2_0(\pi)$,
$$
\sigma^2(X,f)\leq \sigma^2(\bar{X},f).
$$
\end{thm}
\begin{proof}
Since the conditions in Theorem \ref{main1} are satisfied by Lemma \ref{Lang-sec}, %for fixed $f\in L_0^2(\pi)$
we obtain by taking $v=0$ that
$$
\aligned
2/\sigma^2(X,f)&=\inf_{u\in\scr{M}_{f,1}}\sup_{v\in\scr{M}_{f,0}}\e(u+v,u-v)\\
&\geq \inf_{u\in\scr{M}_{f,1}}\e(u,u)=\inf_{u\in\scr{M}_{f,1}}\bar{\e}(u,u)=2/\sigma^2(\bar{X},f).
\endaligned
$$
\end{proof}

\begin{rem}
Similar comparison result in Theorem \ref{compr-diff} can be found in \cite{DLP16,HNW15}. For example, \cite{HNW15} proves the comparison theorem by using a spectral theorem (see \cite[Section 3.4.3]{HNW15}). Here we provide a completely different proof by the new variational formula.
\end{rem}



\begin{exm}(\cite[Example 5.2]{KS14})
	Let $M=\mathbb{R}^{2}$, potential function $U(x)=(1 / 2 \pi) e^{-|x|^{2} / 2}$ and vector field
	$$
	Z=-c x_{2} \frac{\partial}{\partial x_{1}}+c x_{1} \frac{\partial}{\partial x_{2}},
	$$ where $c$ is a positive constant. Consider the 2-dimensional Ornstein-Uhlenbeck diffusion with rotation:
	$$
	\mathfrak{L}:=\frac{1}{2}\left(\frac{\partial^{2}}{\partial x_{1}^{2}}+\frac{\partial^{2}}{\partial x_{2}^{2}}\right)-(x_{1}+c x_{2}) \frac{\partial}{\partial x_{1}}-(x_{2}-cx_1) \frac{\partial}{\partial x_{2}}.
	$$
	 Its invariant probability measure is $\pi(d x)=(1 / 2 \pi) \mathrm{e}^{-|x|^{2} / 2} d x$. The symmetric part of $\mathfrak{L}$ with respect to $\pi$ is
	$$\bar{\mathfrak{L}}:=\left(\frac{\partial^{2}}{\partial x_{1}^{2}}+\frac{\partial^{2}}{\partial x_{2}^{2}}\right)-x_{1} \frac{\partial}{\partial x_{1}}-x_{2} \frac{\partial}{\partial x_{2}}. $$%=\Delta-\nabla U

	Since the symmetric Ornstein-Uhlenbeck diffusion generated by $\bar{\mathfrak{L}}$ is exponentially ergodic, (A3) is satisfied.  A direct calculation shows that $ \operatorname{div}( Z\mathrm{e}^{-U})=0$ and (A1), (A2) are satisfied. Hence,  Theorems \ref{main1} and \ref{compr-diff} are valid.  %\footnote{if we let $\nu'(\d x)=\mathrm{e}^{-|x|^{2} / 2-cx_1x} d x$, then $\mathfrak{A}$ is symmetric with respect to $\nu'$, but $\nu'$ is not a probability measure: $$\nu'(\mathbb{R}^2)=\int_{0}^{2\pi}\frac{1}{1+\sin 2r}\d r$$ does not exist.}	
\end{exm}	





%%%%%%%%%%%%%%%%%%%%%%%%%%%%%%%%%%%%%%%%%%%%%%%%%%%%%%%%%%%%%%%%%%%%%%%%%%%%%%%%%%%%%%%%%%

\section{Proofs of Theorem \ref{main1} and Corollary \ref{exit-time}}\lb{proofs}

Recall that $X=\{X_t\}_{t\geq0}$ is a positive recurrent (or ergodic) Markov process on a Polish space $(S,\mathcal{S})$, with strongly continuous contraction transition semigroup $\{P_t\}_{t\geq0}$ and stationary distribution $\pi$.
$(L,\mathscr{D}(L))$, $(\e,\scr{F})$ are its associated infinitesimal generator in $L^2(\pi)$ and semi-Dirichlet form, respectively.
For fixed $f\in L^2_0(\pi)$, we want to study the %following
 asymptotic variance of $X$ and $f$ defined in \eqref{av-f}. Indeed, from \cite[Section 2.5]{KLO12}, we see that the asymptotic variance can be represented by $P_t$ as follows:
\begin{equation}\lb{av-fp}
\sigma^2(X,f)=2\lim_{t\rightarrow\infty}\int_0^t (1-\frac{s}{t})( P_s f,f)\d s.
\end{equation}


To prove Theorem \ref{main1}, first we do some preparations.
For any $\alpha>0$, set $G_\alpha f=\int_0^\infty \text{e}^{-\alpha s}P_sf\d s$ for $f\in L^2(\pi)$. From \cite[Chapter 1, Proposition 1.10]{MR92} we see that $(G_\alpha)_{\alpha>0}$ is the strong continuous contraction resolvent associated to $L$ and $G_\alpha f\in\scr{D}(L)$ for all  $f\in L^2(\pi)$.
If the semigroup $\{P_t\}_{t\geq0}$ is $L^2$-exponentially ergodic, then it is known that $G f:=\int_0^\infty P_sf\d s\in L^2(\pi)$ for $f\in L_0^2(\pi)$.

\begin{lem}\lb{Gf-u}
Suppose that the semigroup $\{P_t\}_{t\geq0}$ is
$L^2$-exponentially ergodic and its corresponding semi-Dirichlet form $(\e,\scr{F})$ satisfies the sector condition \eqref{weak-sect}. Then for all $f\in L^2_0(\pi)$, we have $Gf\in\scr{D}(L)$ and
$$
\e(Gf,u)=(f,u),\quad  u\in \scr{F}.
$$

\end{lem}
\begin{proof}
We first prove that  $Gf\in \scr{D}(L)$ for all $f\in L^2_0(\pi)$.
Note that the generator $L$ is closed and densely defined, that is, $\scr{D}(L)$ is complete with respect to the graph norm $\|Lu\|+\|u\|,u\in\scr{D}(L)$ (see e.g. \cite[Chapter 1, Proposition 1.10]{MR92}). Thus for fixed $f\in L^2_0(\pi)$, we only need to prove that $\|G_{1/n}f-Gf\|\rightarrow 0$ as $n\rightarrow \infty$ and $\{G_{1/n}f\}_{n\geq 1}$ is a Cauchy sequence under $\|L\cdot\|$ by $G_{1/n}f\in \scr{D}(L),n\geq1$. Indeed, it follows from $L^2$-exponential ergodicity and H\"older inequality that
\be\lb{G_1/n}
\aligned
\|G_{1/n} f-Gf\|&= \Big\|\int_0^\infty(1-{\rm e}^{-s/n})P_sf\d s\Big\|
\leq\int_0^\infty (1-{\rm e}^{-s/n})\| P_sf\|\d s\\
&\leq C\|f\|\int_0^\infty(1-{\rm e}^{-s/n}){\rm e}^{-\lambda_1 s}\d s\\
&= C\|f\|\fr{1/n}{\lmd_1(\lmd_1+1/n)}\rightarrow 0,\quad \text{as }n\rightarrow \infty.
\endaligned
\de
On the other hand, since $Lf=(\alpha-G_\alpha^{-1})f$ for all $\alpha>0$ and $f\in L^2_0(\pi)$, we have
$$
\aligned
\|L(G_{1/n}f-G_{1/m}f)\|&=\|(1/n-G_{1/n}^{-1})G_{1/n}f-(1/m-G_{1/m}^{-1})G_{1/m}f\|\\
&=\|\frac{1}{n}G_{1/n}f-\frac{1}{m}G_{1/m}f\|\\
&\leq \frac{1}{n}\|G_{1/n}f-G_{1/m}f\|+|\frac{1}{n}-\frac{1}{m}|\|G_{1/m}f\|\\
&\rightarrow 0,\quad \text{as }n,m\rightarrow\infty.
\endaligned
$$
Therefore $Gf\in \scr{D}(L)$.

Next we prove that $-LGf=f$ for $f\in L^2_0(\pi)$. Arguing similarly as we did in \eqref{G_1/n}, $\lim_{\bt\rar0}\|G_\bt f-Gf\|=0$ for  $f\in L^2_0(\pi)$. Combining this fact with the property $G_\beta-G_\alpha=(\alp-\bt)G_\alpha G_\beta$, we obtain that
$$
Gf-G_\alpha f=\alpha G_\alpha Gf,\quad \text{for }\alpha>0 ,\ f\in L^2_0(\pi).
$$
%from the property $G_\beta-G_\alpha=(\alp-\bt)G_\alpha G_\beta$.
Using this equality and the fact $Gf\in \scr{D}(L)$ shows that for any $\alpha>0$ and $ f\in L^2_0(\pi)$,
$$
G_\alpha(-LGf)=G_\alpha(G_\alpha^{-1}-\alpha)Gf=Gf-\alpha G_\alpha Gf=G_\alpha f.
$$
That is, $-LGf=f$ for all $f\in L_0^2(\pi)$.

From above analysis and \cite[Chapter 1, Corollary 2.10]{MR92} we could obtain that for any $f\in L^2_0(\pi),u\in\scr{F}$,
$$
\e(Gf,u)=((-L)Gf,u)=( f,u).
$$
\end{proof}


We now proceed to prove Theorem \ref{main1}.

\medskip
\noindent{\bf Proof of Theorem \ref{main1}.}
For fixed $f\in L^2_0(\pi)$, we first claim that the limit in \eqref{av-f}, i.e. \eqref{av-fp}, exists and $\sigma^2(X,f)=2(Gf,f)<\infty$. Indeed, for $t>0$,
$$
2\int_0^t (1-\frac{s}{t})( P_s f,f)\d s=2\int_0^t( P_sf,f)\d s-\frac{2}{t}\int_0^ts( P_sf,f)\d s.
$$
Since $\{P_t\}_{t\geq 0}$ is $L^2$-exponentially ergodic, we arrive at

	$$
		\aligned
			\frac{1}{t}\Big|\int_0^ts( P_sf,f)\d s\Big|&\leq \frac{1}{t} \int_0^t s\|P_s f\|\|f\|\d s\leq \frac{C\|f\|^2}{t} \int_0^t s\mathrm{e}^{-\lambda_1 s}\d s\\
			&\leq\frac{1-(1+\lambda_1 t)\mathrm{e}^{-\lambda_1 t}}{t}C\|f\|^2  \rightarrow0, \ \text{as}\ t\rightarrow\infty,
		\endaligned
	$$
and
$$
\Big|\int_t^\infty ( P_sf,f)\d s\Big|\leq C\|f\|^2\int_t^\infty \text{e}^{-\lambda_1 s}\d s\rightarrow 0,\quad \text{as }t\rightarrow\infty.
$$
Therefore, by combining above analysis, we obtain that the limit in \eqref{av-fp} exists and
\begin{equation*}\label{representation}
	\sigma^2(X,f)=2\int_0^\infty( P_sf,f)\d s<\infty.
\end{equation*}

%Since by the $L^2$-exponential ergodicity again
%$$
%\int_0^\infty\int_S |fP_sf|\d\pi\d s\leq \int_0^\infty ||f||||P_sf||ds\leq C\|f\|^2/\lambda_1<\infty.
%$$
By the Fubini-Tonelli’s theorem and $L^2$-exponential ergodicity again we get
$$
\int_0^\infty(P_sf,f)\d s=\int_0^\infty\int_S fP_sf\d\pi\d s=\int_S\int_0^\infty fP_sf\d s\d\pi=( Gf,f).
$$
Thus
\begin{equation}\label{representation}
\sigma^2(X,f)=2(Gf,f)<\infty.
\end{equation}
To prove \eqref{vf-av}, we set $w=Gf/(Gf,f),\ w^*=G^*f/( Gf,f)$ and $u_0=(w+w^*)/2, v_0=(w-w^*)/2$. Then $u_0\in \scr{M}_{f,1}$ and $v_0\in \scr{M}_{f,0}$ by noting
$$(Gf,f)=\int_0^\infty(P_sf,f)\d s=\int_0^\infty(f,P^*_sf)\d s=( G^*f,f).
$$

Now  let $v_1=v-v_0$ for any $v\in\scr{M}_{f,0}$. By the definition of $w,w^*,v_0$ and Lemma \ref{Gf-u}, we have $\pi(v_1f)=0$ and
$$
\e(v_1,w^*)=\e(w,v_1)=\frac{1}{( Gf,f)}\e(Gf,v_1)=\frac{1}{( Gf,f)}( f,v_1)=0.
$$
Therefore, using this fact with $\e(w,w^*)=1/(Gf,f)$ and $\e(u,u)\geq0$ for all $u\in\scr{F}$ gives that
$$
\e(u_0+v,u_0-v)=\e(w-v_1,w^*+v_1)=\e(w,w^*)-\e(v_1,v_1)\leq 1/( Gf,f),
$$
which implies that
\be\lb{geq}
1/( Gf,f)\geq\inf_{u\in\scr{M}_{f,1}}\sup_{v\in\scr{M}_{f,0}}\e(u+v,u-v).
\de

For the converse inequality,  let $u_1=u-u_0$ for any $u\in\scr{M}_{f,1}$. Since $u_0\in\scr{M}_{f,1}$, we also have $\pi(u_1f)=0$. Similar  argument shows that
$$
\e(u+v_0,u-v_0)=\e(w+u_1,w^*+u_1)=\e(w,w^*)+\e(u_1,u_1)\geq 1/( Gf,f).
$$
Therefore,
\be\lb{leq}
1/(Gf,f)\leq\inf_{u\in\scr{M}_{f,1}}\sup_{v\in\scr{M}_{f,0}}\e(u+v,u-v).
\de
So we obtain \eqref{vf-av} by combining \eqref{geq}, \eqref{leq} and the fact $\sigma^2(X,f)=2(Gf,f)$.

When process $X$ is reversible,  $\e(\cdot,\cdot)$ is symmetric, i.e.,
$$
\e(u,v)=\e(v,u),\quad \text{for }u,v\in\scr{F}.
$$
Thus %for any $u,v\in\scr{F}$, we have
$$
\e(u+v,u-v)=\e(u,u)-\e(v,v)\leq \e(u,u).
$$
That is, the supremum in \eqref{vf-av} is attained by $v=0$ for any fixed $u\in\scr{M}_{f,1}$. Hence, we obtain \eqref{vf-av-r}.
\qed

By using Theorem \ref{main1}, we prove Corollary \ref{exit-time} as follows.


\medskip
\noindent{\bf Proof of Corollary \ref{exit-time}.}
Fix an open set $\Omega\subset S$ with $\pi(\Omega)\in (0,1)$. It follows from \cite[Theorem 3.3]{HKMW20} that
\be\label{va-exit}
1/\bE_\pi\tau_\Omega=\inf_{u\in \cN_{\Omega,1}}\e(u,u),
\de
where 	$
\cN_{\Omega,1}:=\{u\in\mathscr{F}:u|_{\Omega^c}=0\ \text{and }\pi(u)=1\}.
$
Take $$f=\frac{\mathbf{1}_\Omega-\pi(\Omega)}{1-\pi(\Omega)}.$$
It is easy to check that
$\pi(f)=0$ and $\|f\|^2=\pi(\Omega)/\pi(\Omega^c).$
Notice that for any $u\in \cN_{\Omega,1},$ by simple calculation we have
$\pi(uf)=1$, thus $u\in \cM_{f,1}$. So we see that $\cN_{\Omega,1}\subset \cM_{f,1}$.
Combining this fact with \eqref{vf-av-r} and \eqref{va-exit}, we obtain that
$$2/\sigma^2(X,f)=\inf_{u\in\scr{M}_{f,1}}\e(u,u)\leq \inf_{u\in \cN_{\Omega,1}}\e(u,u)=1/\bE_\pi\tau_\Omega.$$
That is, $\bE_\pi\tau_\Omega\leq \sigma^2(X,f)/2.$
%Then use this fact and Lemma \ref{av} to derive  that
Moreover, from the reversibility and $L^2$-exponential ergodicity we have
$$
\sigma^2(X,f)/2=\int_0^\infty (P_sf,f)\d s\leq \int_0^\infty \|P_sf\|\|f\|\d s\leq \|f\|^2/\lambda_1.
$$
Hence,
$$
\bE_\pi\tau_\Omega\leq \frac{\|f\|^2}{2\lambda_1}=\frac{\pi(\Omega)}{2\lambda_1\pi(\Omega^c)}.
$$
\qed




%%%%%%%%%%%%%%%%%%%%%%%%%%%%%%%%%%%%%%%%%%%%%%%%%%%%%%%%%%%%%%%%%%%%%%












%----------------------------------------------------------------------------------------------
%----------------------------------------------------------------------------------------------
{\bf Acknowledgement}\
Lu-Jing Huang acknowledges support from NSFC (No. 11901096), NSF-Fujian(No. 2020J05036), the Program for Probability and Statistics: Theory and
Application (No. IRTL1704), and the Program for Innovative Research Team in Science and Technology in Fujian Province University (IRTSTFJ). Yong-Hua Mao and Tao Wang acknowledge support by the National Key R\&D  Program of China (2020YFA0712900) and the National Natural Science
Foundation of China (Grant No.11771047).


%%%%%%%%%%%%%%%%%%%%%%%%%%%%%%%%%%%%%%%%%%%%%%%%%%%%%%%%%%%%%%
%\newpage
%\begin{thebibliography}{00}

%\end{thebibliography}
\bibliographystyle{plain}
\bibliography{jump_drift}

\vskip 0.3truein

{\bf Lu-Jing Huang:}
%\vskip -.1truein
College of Mathematics and Informatics, Fujian Normal University, Fuzhou, 350007, P.R. China. E-mail: \texttt{huanglj@fjnu.edu.cn}

\medskip
{\bf Yong-Hua Mao:}
%\vskip -.1truein
Laboratory of Mathematics and Complex Systems(Ministry of Education), School of Mathematical Sciences, Beijing Normal University, Beijing 100875, P.R. China. E-mail: \texttt{maoyh@bnu.edu.cn}


\medskip
{\bf Tao Wang:}
%\vskip -.1truein
Laboratory of Mathematics and Complex Systems(Ministry of Education), School of Mathematical Sciences, Beijing Normal University, Beijing 100875, P.R. China. E-mail: \texttt{wang\_tao@mail.bnu.edu.cn}

\end{document}


\subsection{Ornstein-Uhlenbeck process driven by symmetric stable process }

 Let $X$ be a  symmetric $\alpha$-stable process on $\mathbb{R}^{d}$ with infinitesimal generator $\Delta^{\alpha / 2}:=-(-\Delta)^{\alpha / 2}$ which enjoys the following expression
$$
\Delta^{\alpha / 2} u(x)=\int_{\mathbb{R}^{d} \backslash\{0\}}\left(u(x+z)-u(x)-\nabla u(x) \cdot z \mathbf{1}_{\{|z| \leq 1\}}\right) \frac{C_{d, \alpha}}{|z|^{d+\alpha}} d z,
$$
where $\alpha \in(0,2)$ and $C_{d, \alpha}=\frac{\alpha 2^{\alpha-1} \Gamma((d+\alpha) / 2)}{\pi^{d / 2} \Gamma(1-\alpha / 2)}$. Consider the Ornstein-Uhlenbeck operator
$$
L_{\alpha} f(x):=-(-\Delta)^{\alpha / 2} f(x)-\langle x, \nabla f(x)\rangle, f \in C_{0}^{\infty}\left(\mathbb{R}^{d}\right).
$$
This is a non-reversible L\'{e}vy process.
According to \cite{ARW00}, the associated Markov semigroup $P_t$ has a unique invariant (but not reversible) probability measure $\mu$. By \cite[(1.9)]{RW03},  on the Hilbert space $L^2(\mu)$,
%which is identified by the Fourier transformation
%$$\hat{\mu}_{\alpha}(\xi):=\int_{\mathbb{R}^{d}} \mathrm{e}^{i\langle x, \xi\rangle} \mu_{\alpha}(\mathrm{d} x)=\mathrm{e}^{-\frac{1}{\alpha}|\xi|^{\alpha}}, \quad \xi \in \mathbb{R}^{d}$
%For any $f \in C_{0}^{\infty}\left(\mathbb{R}^{d}\right),$ the set of all smooth functions on $\mathbb{R}^{d}$ with compact support, we have (see [14, Proposition 4.1] or [18,(1.9)]$)$
$$
\mathscr{E}(f, f):=-\lan f, L_{\alpha} f \ran_\mu=\frac{1}{2} \iint_{\mathbb{R}^{d} \times \mathbb{R}^{d}} \frac{|f(x)-f(y)|^{2}}{|x-y|^{d+\alpha}} \mathrm{d} y \mu_{\alpha}(\mathrm{d} x)
$$
Let $\mathscr{D}\left(\mathscr{E}_{\alpha}\right)=\left\{f \in L^{2}\left(\mu_{\alpha}\right): \mathscr{E}_{\alpha}(f, f)<\infty\right\} .$

\subsection{Non-symmetric diffusions on manifolds}
Let $M$ be a complete connected Riemannian manifold, and  $m$ be the Riemann volume  on $M .$
Define the potential (probability) measure $\nu$ on $M$ by $\nu:=e^{-U} m$, where $U\in C^{2}(M)$ such that $\int_{M} e^{-U} d m=1$. Let $\nabla$ be an affine connection and $\nabla_{\nu}^{*}$ be the Hodge star dual operator of $\nabla$ on $L^{2}(\nu)$, which is characterized by $$\nabla_{\nu}^{*} \theta=\nabla^{*} \theta+(\nabla U, \theta)$$
for $\theta \in M,$ where $\nabla^{*}$ is the dual of $\nabla$ on $L^{2}(m)$.
Consider the generator
$$
\mathfrak{L}=-\frac{1}{2} \nabla_{\nu}^{*} \nabla+Z,
$$
where $Z$ is a smooth vector field.
Then, the dual $\mathfrak{L}_{\nu}^{*}$ of $\mathfrak{L}$ on $L^{2}(\nu)$ satisfies
$$
\mathfrak{L}_{\nu}^{*}=-\frac{1}{2} \nabla_{\nu}^{*} \nabla-Z-\operatorname{div}_{\nu} Z,
$$
where $\operatorname{div}_{\nu}$ is the divergence on $L^{2}(\nu)$ defined by Stokes Theorem:
$$
\int Z f d \nu=-\int f \operatorname{div}_{\nu} Z d \nu, \quad f \in C_{0}^{1}(M).
$$
The symmetric part of  $\mathfrak{L}$ with respect to $\nu$  is  $\mathfrak{B}:=-\frac{1}{2} \nabla_{\nu}^{*} \nabla.$

Assume that $\int_M  Zf \d \nu=0$, then  $\nu$ is the invariant measure of $
\mathfrak{L}$.

Let $\mathscr{E}(f,g)=\lan -\mathfrak{L}f,g\ran_\nu$ and $\mathscr{E}_s=\lan -\mathfrak{B}f,g\ran_\nu$ be the symmetric part  $\mathfrak{B}$. Then
$$
\mathscr{E}_s(f, g)=\frac{1}{2} \int_M(\nabla f, \nabla g) d \nu, \quad f, g \in C_{0}^{\infty}(M).
$$


\begin{exm}\cite[Example 5.2]{ks14}
 Let $M=\mathbb{R}^{2}$, potential function $U(x)=(1 / 2 \pi) e^{-|x|^{2} / 2}$ and vector field
 $$
 Z=-c x_{2} \frac{\partial}{\partial x_{1}}+c x_{1} \frac{\partial}{\partial x_{2}},
 $$ where $c$ is a positive constant.  Then $\nu(d x)=(1 / 2 \pi) \mathrm{e}^{-|x|^{2} / 2} d x$,
 $$-\frac{1}{2} \nabla_{\nu}^{*} \nabla=\frac{1}{2}\left(\frac{\partial^{2}}{\partial x_{1}^{2}}+\frac{\partial^{2}}{\partial x_{2}^{2}}\right)-x_{1} \frac{\partial}{\partial x_{1}}-x_{2} \frac{\partial}{\partial x_{2}}. $$
 Consider the two-dimensional Ornstein-Uhlenbeck diffusion with rotation:
 $$
 \mathfrak{A}=-\frac{1}{2} \nabla_{\nu}^{*} \nabla+Z=\frac{1}{2}\left(\frac{\partial^{2}}{\partial x_{1}^{2}}+\frac{\partial^{2}}{\partial x_{2}^{2}}\right)-(x_{1}+c x_{2}) \frac{\partial}{\partial x_{1}}-(x_{2}-cx_1) \frac{\partial}{\partial x_{2}}.
 $$
In this case, $Z$ is a conservative vector field, i.e.
%$$Zf\d \nu=\left(-c x_{2} \frac{\partial f}{\partial x_{1}}+c x_{1} \frac{\partial f}{\partial x_{2}}\right)\mathrm{e}^U(x)\d x_1\d x$$
$\int_M  Zf \d \nu=0,$ % $Z$ is a conservative vector field, i.e.
therefore, we have that $\nu$ is the invariant measure. %\footnote{if we let $\nu'(\d x)=\mathrm{e}^{-|x|^{2} / 2-cx_1x} d x$, then $\mathfrak{A}$ is symmetric with respect to $\nu'$, but $\nu'$ is not a probability measure: $$\nu'(\mathbb{R}^2)=\int_{0}^{2\pi}\frac{1}{1+\sin 2r}\d r$$ does not exist.}
\end{exm}





%%%%%%%%%%%%%%%%%%%%%%%%%%%%%%%%%%%%%%%%%%%%%%%%%%%%%%%%%%%%
\subsection{Symmetric stable processes}

Similarly, maybe we can  discuss L\'{e}vy process? for example, one-dimensional ergodic time-changed stable process:  Let $X$ be a 1-D symmetric stable process,  $a_i(x)$ ($i=1,2$) be two positive function on $\mathbb{R}^1$ such that $1/a_i$ is $L^1$ locally integrable and  $a_1\leq a$. Define
	$$\tau_{t}^{(i)}=\inf \left\{s \geqslant 0: \int_{0}^{s} a\left(X_{u}\right)^{-1} \mathrm{~d} u>t\right\} \quad \text { and } \quad Y_{t}^{(i)}:=X_{\tau_{t}^{(i)}}, \quad \text { for } t \geq 0.$$
	By \cite[Theorem 5.2.2]{CM12} and \cite[Section 1.2]{CW14}, the generator of  $Y^{(i)}$ is  $L=a_i \Delta^{\alpha / 2}$  which is symmetric with respect to its invariant measure $\pi_i(\mathrm{d} x)=a_i(x)^{-1} \mathrm{d} x$. Assume that $\pi_i(\mathbb{R})<\infty$. The Dirichlet form $(\mathscr{E}_i,\mathscr{D}(\mathscr{E}_i))$ is given by
	$$
	\mathscr{E}_i(f, g)=\frac{1}{2} \int_{\mathbb{R}} \int_{\mathbb{R}}(f(x)-f(y))(g(x)-g(y)) \frac{C_{1, \alpha}\mathrm{d} x \mathrm{d} y}{|x-y|^{1+\alpha}} , \ \text{for any}\ f,g\in \mathscr{F}_i,
	$$
	and $\mathscr{F}_i:=\{f\in L^2(\pi_i): \mathscr{E}(f,f)<\infty \}.$ Then $\scr{F}_{1}\supseteq \scr{F}_{2}$ and $
	\e_{1}(u,u)=\e_{2}(u,u)\quad \text{for all }u\in \scr{F}_{2}.$ Hence
	for any $f\in L^2_0(\pi)$,
$$
2/\sigma^2(Y^{(1)},f)=\inf_{\substack{u\in\scr{F}_{1},\\\pi_1(fu)=1}}\e_{1}(u,u)
\leq \inf_{\substack{u\in\scr{F}_{2},\\\pi(fu)=1}}\e_{2}(u,u)=2/\sigma^2(Y^{(2)},f).
$$





