%% using aastex version 6.3
\documentclass{aastex631}

%%
\newcommand{\vdag}{(v)^\dagger}
\newcommand\aastex{AAS\TeX}
\newcommand\latex{La\TeX}

%% Reintroduced the \received and \accepted commands from AASTeX v5.2
%\received{March 1, 2021}
%\revised{April 1, 2021}
%\accepted{\today}

%% Command to document which AAS Journal the manuscript was submitted to.
%% Adds "Submitted to " the argument.
%\submitjournal{PSJ}

%% For manuscript that include authors in collaborations, AASTeX v6.31
%% builds on the \collaboration command to allow greater freedom to 
%% keep the traditional author+affiliation information but only show
%% subsets. The \collaboration command now must appear AFTER the group
%% of authors in the collaboration and it takes TWO arguments. The last
%% is still the collaboration identifier. The text given in this
%% argument is what will be shown in the manuscript. The first argument
%% is the number of author above the \collaboration command to show with
%% the collaboration text. If there are authors that are not part of any
%% collaboration the \nocollaboration command is used. This command takes
%% one argument which is also the number of authors above to show. A
%% dashed line is shown to indicate no collaboration. This example manuscript
%% shows how these commands work to display specific set of authors 
%% on the front page.
%%
%% For manuscript without any need to use \collaboration the 
%% \AuthorCollaborationLimit command from v6.2 can still be used to 
%% show a subset of authors.
%
%\AuthorCollaborationLimit=2
%
%% will only show Schwarz & Muench on the front page of the manuscript
%% (assuming the \collaboration and \nocollaboration commands are
%% commented out).
%%
%% Note that all of the author will be shown in the published article.
%% This feature is meant to be used prior to acceptance to make the
%% front end of a long author article more manageable. Please do not use
%% this functionality for manuscripts with less than 20 authors. Conversely,
%% please do use this when the number of authors exceeds 40.
%%
%% Use \allauthors at the manuscript end to show the full author list.
%% This command should only be used with \AuthorCollaborationLimit is used.

%% The following command can be used to set the latex table counters.  It
%% is needed in this document because it uses a mix of latex tabular and
%% AASTeX deluxetables.  In general it should not be needed.
%\setcounter{table}{1}

%%%%%%%%%%%%%%%%%%%%%%%%%%%%%%%%%%%%%%%%%%%%%%%%%%%%%%%%%%%%%%%%%%%%%%%%%%%%%%%%
%%
%% The following section outlines numerous optional output that
%% can be displayed in the front matter or as running meta-data.
%%
%% If you wish, you may supply running head information, although
%% this information may be modified by the editorial offices.
\shorttitle{Spectral Identification for White Dwarf from Gaia EDR3}
\shortauthors{Xiao Kong et al.}
%%
%% You can add a light gray and diagonal water-mark to the first page 
%% with this command:
%% \watermark{text}
%% where "text", e.g. DRAFT, is the text to appear.  If the text is 
%% long you can control the water-mark size with:
%% \setwatermarkfontsize{dimension}
%% where dimension is any recognized LaTeX dimension, e.g. pt, in, etc.
%%
%%%%%%%%%%%%%%%%%%%%%%%%%%%%%%%%%%%%%%%%%%%%%%%%%%%%%%%%%%%%%%%%%%%%%%%%%%%%%%%%
\graphicspath{{./}{figures/}}
%% This is the end of the preamble.  Indicate the beginning of the
%% manuscript itself with \begin{document}.

\begin{document}

\title{Identification of White Dwarf from Gaia EDR3 via Spectra from LAMOST DR7}

\correspondingauthor{A-Li Luo}
\email{lal@nao.cas.cn}

\author[0000-0001-8011-8401]{Xiao Kong}
\affil{Key Laboratory of Optical Astronomy, National Astronomical Observatories, Chinese Academy of Sciences, \\
	Beijing 100012, China}

\author[0000-0001-7865-2648]{A-Li Luo}
\affiliation{Key Laboratory of Optical Astronomy, National Astronomical Observatories, Chinese Academy of Sciences, \\
	Beijing 100012, China}
\affiliation{School of Astronomy and Space Science, University of Chinese Academy of Sciences, Beijing 100049, China}

%% Note that the \and command from previous versions of AASTeX is now
%% depreciated in this version as it is no longer necessary. AASTeX 
%% automatically takes care of all commas and "and"s between authors names.

%% AASTeX 6.31 has the new \collaboration and \nocollaboration commands to
%% provide the collaboration status of a group of authors. These commands 
%% can be used either before or after the list of corresponding authors. The
%% argument for \collaboration is the collaboration identifier. Authors are
%% encouraged to surround collaboration identifiers with ()s. The 
%% \nocollaboration command takes no argument and exists to indicate that
%% the nearby authors are not part of surrounding collaborations.

%% Mark off the abstract in the ``abstract'' environment. 
\begin{abstract}

We cross-matched 1.3 million white dwarf (WD) candidates from Gaia EDR3 with spectral data from LAMOST DR7 within $3^{\prime\prime}$. Applying machine learning described in our previous work, we spectroscopically identified 6 190 WD objects after visual inspection, among which 1 496 targets were firstly confirmed. 32 detailed classes were adopted for them, including but not limited to DAB and DB+M. We estimated the atmospheric parameters for the DA and DB type WD using Levenberg- Marquardt least-squares algorithm (LM). Finally, a catalog of WD spectra from LAMOST was provided online.


\end{abstract}

%% Keywords should appear after the \end{abstract} command. 
%% The AAS Journals now uses Unified Astronomy Thesaurus concepts:
%% https://astrothesaurus.org
%% You will be asked to selected these concepts during the submission process
%% but this old "keyword" functionality is maintained in case authors want
%% to include these concepts in their preprints.
\keywords{white dwarfs -- methods: data analysis -- techniques: spectroscopic -- catalogs}

%% From the front matter, we move on to the body of the paper.
%% Sections are demarcated by \section and \subsection, respectively.
%% Observe the use of the LaTeX \label
%% command after the \subsection to give a symbolic KEY to the
%% subsection for cross-referencing in a \ref command.
%% You can use LaTeX's \ref and \label commands to keep track of
%% cross-references to sections, equations, tables, and figures.
%% That way, if you change the order of any elements, LaTeX will
%% automatically renumber them.
%%
%% We recommend that authors also use the natbib \citep
%% and \citet commands to identify citations.  The citations are
%% tied to the reference list via symbolic KEYs. The KEY corresponds
%% to the KEY in the \bibitem in the reference list below. 

\section{Introduction} \label{sec:intro}

WDs are assigned to several subtypes according to the major components of the surface atmosphere.
If normally broad Balmer lines appear in a spectrum, it is DA WD.
Similarly, He {\small \bf I}, He {\small \bf II}, Ca {\small \bf II} H \& K and C line for DB, DO, DZ, and DQ respectively \citep{2013ApJS..204....5K}. 

Approximately 1.3 million targets were selected from {\it Gaia} Early Data Release (EDR) 3 \citep{2021A&A...649A...1G}, considered to be the candidates of WD \citep{2021arXiv210607669G}.
These candidates were derived by several selection criteria in absolute magnitude, color, etc. from {\it Gaia} EDR3 catalog.
The samples of spectroscopically confirmed SDSS WDs were also adopted to calculate probabilities of being a WD ($P_{\rm WD}$).

Up to nearly 7 million objects, LAMOST \citep{2015RAA....15.1095L} released its 7th data product (Data Release 7, DR7) that included more than 10 million low resolution spectral data.
In this research, we aimed to identify WD candidates of {\it Gaia} EDR 3 using spectra from LAMOST DR7.


\section{Data Selection} \label{sec:data}

We performed a cross-match of WD candidates with LAMOST spectral data onto equinox 2000 and epoch 2000 within $3^{\prime\prime}$ utilizing formula \ref{eq:dis}, where $\alpha$ and $\delta$ represent right ascension and declination respectively.

\begin{equation}	\label{eq:dis}
d = \arccos[\cos \delta 1 \cos \delta 2 \cos(\alpha 1-\alpha 2)+\sin \delta 1 \sin \delta 2]
\end{equation}

Considering a LAMOST fiber diameter of $3^{\prime\prime}$ \citep{2015RAA....15.1095L}, the cross radius $d$ was also restricted to $3^{\prime\prime}$.
The accuracy of fiber positioning, on the other hand, had been determined to be no more than $1.5^{\prime\prime}$ on average \citep{2014SPIE..9149....1}.
We start the cross-match procedure based on sky coordinates by using both radii.
The conclusions of this research were based on radius $3^{\prime\prime}$, while a specific field presented in our catalog was used to mark the data if it was more than $1.5^{\prime\prime}$ away from any source of {\it Gaia}.

Moreover, these data that had positive values of offset were dismissed as the inconsistent between observation position and the coordinates from the input catalog.
We then derived 18 232 spectra.
Among all the 1.4 million WD candidates from {\it Gaia}, there existed 12 046 that corresponded with LAMOST spectra.

We also inspected 5.8 million spectra of SDSS DR16 \citep{2020ApJS..249....3A} and calculated the distances between them and our data sample.
Half of the objects of LAMOST's sample were not observed by SDSS yet, they needed to be verified even more.


\section{Identification} \label{sec:identity}

We adopted the same procedure described in \cite{2018PASP..130h4203K} to remove those that had little evidence of spectral lines.
9 496 spectra remained after the machine learning process.
Afterwards, we inspected all the results and rejected those that had little evidence to be a WD and maintained 8 465 spectra of 6 190 objects.
The number of targets would reduce to 6 045 if the radius was fixed at $1.5^{\prime\prime}$.
Meanwhile, these small samples of spectra that located between $1.5^{\prime\prime}$ and $3^{\prime\prime}$ exhibited no characteristics distinct from those within the radius of $1.5^{\prime\prime}$.

During the inspection, all the WD spectra were classified into detailed types, including but not limit of DAB and DBZ.

Considering the $P_{\rm WD}$ from \cite{2021arXiv210607669G}, about 75\% of the spectral confirmed WD objects had values greater than 0.9.
As for the other spectra that may not be WDs, the $P_{\rm WD}$ was relatively lower.
Around 68\% of them were less than 0.5, while 23\% larger than 0.9.
This is partly due to poor quality that we were unable to discover any line features in a spectrum and discard it from our sample.
In the other hand, they might be DC WD that could not be verified by spectra alone.

52 objects with 59 spectra were composed of 2 stellar systems, usually a WD plus a late-type star.
We utilized LAMOST 1D Pipeline to recognize the other components by template fitting and found the majority of them WD plus M.

Interestingly, 64 cataclysmic variables (CVs) \citep{1995cvs..book.....W}, with 49 had obvious double peak structure among their emission Balmer or He lines, were intermixed in our sample.
Most of them had traces of helium in their spectra.


\section{Stellar Parameters} \label{sec:param}

Applying spectral templates for DA and DB \citep{2010MmSAI..81..921K}, we calculated atmosphere parameters of the WD spectra.

In the beginning, all spectral samples were moved to rest-frame relying on the redshift derived from LAMOST 1D Pipeline.
Following the determination of best-fit model, LM  were adopted to estimate $T_{\rm eff}$, $\log g$ ~and their uncertainty.
The depth of lines of several spectra with relatively low signal-to-noise ratio, though, were almost the same intensity as noise.
The parameters of these data went out of bounds of the template scope and were to reset to -9999 manually.


\section{Summary}

We present a catalog of spectral confirmed WDs from LAMOST DR7 based on the candidates from {\it Gaia} EDR3.
The full {\it Gaia}-LAMOST spectroscopic sample catalog contains the main information (see Table \ref{tab:catalog}) of WDs and can be downloaded following link: \url{http://paperdata.china-vo.org/XiaoKong/LAMOST_DR7_WD.fits}

\begin{deluxetable*}{ll}
	\tablenum{1}
	\tablecaption{Format of the LAMOST DR7 Catalog of spectral confirmed WDs.\label{tab:catalog}}
	\tablewidth{0pt}
	\tablehead{
		\colhead{Heading} & \colhead{Description}
	}
	\startdata
	O{\scriptsize BSID}	&	Unique ID of a spectrum	\\
	S{\scriptsize OURCE\_ID}	&	Unique ID for this object in {\it Gaia} EDR3	\\
	W{\scriptsize D\_NAME}	&	LAMOSTJ+J2000 ra (hh mm ss:ss)+dec(dd mm ss.s), equinox and epoch 2000	\\
	RA		&	Right ascension [deg] of object	\\
	DEC		&	Declination [deg] of object		\\
	T{\scriptsize YPE}	&	Detailed classes for WDs from LAMOST	\\
	SNR		&	Signal-to-noise ratio of $u$, $g$, $r$, $i$ and $z$ filter	\\
	RV		&	Radial velocity [km/s] of object from template fitting		\\
	$T_{\rm eff}$	&	Effective temperature [K] from fitting the parameter model \\
	$\log g$	&	Surface gravity from fitting the parameter model	\\
	N\_{\scriptsize BIB}	&	Number of references that confirmed type (identify for the first time if 0)	\\
	R{\scriptsize AD}	&	1 means distance between this spectrum and {\it Gaia} source smaller than $1.5^{\prime\prime}$	\\
	\enddata
	\tablecomments{This catalog involves WD and CV information both. -9999 means the value can not be provided.}
\end{deluxetable*}

Starting from cross-matching spectral data of LAMOST with sources of {\it Gaia}, we identified 8 465 WD spectra, involving DA, DB, DO, DZ, etc.
Some other types, binary or CV for instance, were also noted.
These objects corresponded with 6 190 {\it Gaia} targets.
1 496 stars of our samples were spectral confirmed for the first time.
We then found the best model that fits a spectrum and estimated the atmosphere parameter using LM.


%% IMPORTANT! The old "\acknowledgment" command has be depreciated. It was
%% not robust enough to handle our new dual anonymous review requirements and
%% thus been replaced with the acknowledgment environment. If you try to 
%% compile with \acknowledgment you will get an error print to the screen
%% and in the compiled pdf.
\begin{acknowledgments}
We thank J.K. Zhao and D. Koester for providing the parameter templates for DA and DB WD.
These models were made by D. Koester ranging from 5 000 K to 80 000 K and 7.0 to 9.5 for $T_{\text{eff}}$ and $\log g$ respectively.

This work has made use of data from the European Space Agency (ESA) mission
{\it Gaia} (\url{https://www.cosmos.esa.int/gaia}), processed by the {\it Gaia}
Data Processing and Analysis Consortium (DPAC,
\url{https://www.cosmos.esa.int/web/gaia/dpac/consortium}). Funding for the DPAC
has been provided by national institutions, in particular the institutions
participating in the {\it Gaia} Multilateral Agreement.

Guoshoujing Telescope (the Large Sky Area Multi-Object Fiber Spectroscopic Telescope LAMOST) is a National Major Scientific Project built by the Chinese Academy of Sciences. Funding for the project has been provided by the National Development and Reform Commission. LAMOST is operated and managed by the National Astronomical Observatories, Chinese Academy of Sciences.
\end{acknowledgments}

%% To help institutions obtain information on the effectiveness of their 
%% telescopes the AAS Journals has created a group of keywords for telescope 
%% facilities.
%
%% Following the acknowledgments section, use the following syntax and the
%% \facility{} or \facilities{} macros to list the keywords of facilities used 
%% in the research for the paper.  Each keyword is check against the master 
%% list during copy editing.  Individual instruments can be provided in 
%% parentheses, after the keyword, but they are not verified.

\vspace{5mm}
\facilities{{\it Gaia}, LAMOST}

%% Similar to \facility{}, there is the optional \software command to allow 
%% authors a place to specify which programs were used during the creation of 
%% the manuscript. Authors should list each code and include either a
%% citation or url to the code inside ()s when available.



\bibliography{ref}{}
\bibliographystyle{aasjournal}

%% This command is needed to show the entire author+affiliation list when
%% the collaboration and author truncation commands are used.  It has to
%% go at the end of the manuscript.
%\allauthors

%% Include this line if you are using the \added, \replaced, \deleted
%% commands to see a summary list of all changes at the end of the article.
%\listofchanges

\end{document}
