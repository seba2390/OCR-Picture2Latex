%%%%%%%%%%%%%%%%%%%%%%%
\section{Abstract cell theory}
\label{sec:cells}
%%%%%%%%%%%%%%%%%%%%%%%

In this chapter we discuss cells in algebras and cells in monoidal categories. The eventual application will be to Hecke algebras and Soergel bimodules. In type $A$, the resulting cell theory is much simpler than the general case, having a nice combinatorial description, so the novice reader is encouraged to skip ahead and read \S\ref{sec:typeAcells} concurrently.

In \S\ref{subsec:twists} we introduce the theory of twist-like complexes, and study how they act on cells.



%============================
\subsection{Cells in algebras}
\label{subsec:algCells}
%============================

Let $\kbbm$ be a commutative ring and let $A$ be a $\kbbm$-algebra.  Let $W$ be some indexing set and $\{b_x\}_{x \in W}$ be a $\kbbm$-basis of $A$. For an arbitrary element $a \in A$, we say that $$b_i \babysumset a$$ if, writing $a$ as a linear combination of basis elements, the term $b_i$ appears with nonzero coefficient.   One can place several relations on the elements of this basis (or on the indexing set $I$), by saying that
\[ b_x \le_{L} b_y \quad \textrm{if} \quad b_x \babysumset m \cdot b_y \textrm{ for some } m \in A. \]
\[ b_x \le_{R} b_y \quad \textrm{if} \quad b_x \babysumset b_y \cdot n \textrm{ for some } n \in A. \]
\[ b_x \le_{LR} b_y \quad \textrm{if} \quad b_x \babysumset m \cdot b_y \cdot n \textrm{ for some  } m, n \in A. \]
If the structure coefficients of $A$ in the basis $\{b_x\}$ are non-negative integers, then the above relations are transitive.  If these relations are not transitive, we consider instead their transitive closures, and abusively use the same notation for the transitive closure.

The equivalence classes under these relations are known as \emph{left}, \emph{right}, and \emph{two-sided cells} respectively. We write $b_x \sim_L b_y$ if these two basis elements are in the same left cell, and similarly for $b_x \sim_R b_y$ and $b_x \sim_{LR} b_y$.  The set of left (or right, or two-sided) cells inherits a partial order from $\leq_L$ (resp.~$\leq_R$, resp.~$\leq_{LR}$).  Each two-sided cell is a union of left cells (or right cells).

One of the main reasons for the appearance of cells is that they determine a filtration of the algebra $A$ by ideals.

\begin{prop}\label{prop:cellFiltration}
Let $A$ be an algebra with basis $\{b_x\}_{x\in W}$.  For a given two-sided cell $\l$, let $I_{< \l}\subset A$ denote the subspace spanned by basis elements $b_y$ in (two-sided) cells strictly lower than $\lambda$.  Then $I_{< \l}$ is a two-sided ideal.
\end{prop}

\begin{proof}
Immediate from the definitions.
\end{proof}

Similarly, $I_{\le \l}$ is a two-sided ideal, as is $I_{\ngeq \l}$, and so forth. By considering left or right cells instead, one obtains families of left or right ideals in $A$. By
taking the subquotients of these filtrations, one obtains potentially interesting representations of $A$.

Note that the cell theory of an algebra depends heavily on the chosen basis. For instance, if $A=\HB$ is the Hecke algebra associated to a Coxeter system $(W,S)$, then the cell theory
associated to the standard basis $\{H_w\}$ is boring: since each $H_w$ is invertible, they all generate the unit ideal, so they are left, right, and two-sided equivalent. For the
Kazhdan-Lusztig basis, the resulting cell theory and filtration is far from boring. Let us mention some easy examples of cells.

\begin{ex} \label{ex:threecells} Inside $\HB$, $1=b_1$ is identity, so $b_x \le b_1$ for all $x$ (for all three relations). It turns out that $\{1\}$ is an equivalence class (under all
three relations), called the \emph{identity cell}. At the other extreme, $b_{w_0}$ spans a one-dimensional ideal in $\HB$, so that $\{w_0\}$ is an equivalence class (under all three
relations), called the \emph{longest cell}. For $S_3$, there is only one other 2-sided cell $\{b_s,b_t,b_{st},b_{ts}\}$, which splits into two left (resp. right) cells. In general, as
was shown by Lusztig, all the non-identity elements of $W$ with a unique reduced expression (such as the simple reflections) form a two-sided cell, called the \emph{simple cell}\footnote{If $W$ is not irreducible, then the set of non-identity elements with a unique reduced expression split into a disjoint union of two-sided ``simple cells,'' one for each connected component of the Coxeter graph.}. \end{ex}

In summary, given a choice of basis of $A$, one has an associated cell theory which may or may not be boring.  Now we categorify this notion, replacing the algebra with a monoidal additive category. The (arbitrary) choice of basis becomes the (intrinsic) collection of indecomposable objects.

%============================
\subsection{Categorical cells}
\label{subsec:celldef}
%============================

 % denote the set of indecomposables in $\AC$, and let $\ind^\star\AC\subset \ind \AC$ denote a chosen set of representatives of isomorphism classes of twists of indecomposables.  Here a \emph{twist} of an object $B$ is an object $\Gamma\otimes B$ where $\Gamma\in \AC$ is in invertible (for instance grading shifts $B(k)$ are twists of $B$, whenever this makes sense).

Recall that a full subcategory $\BC\subset \AC$ is determined by the objects of $\AC$ which are contained in $\BC$. Let $\ind \AC$ denote the indecomposable objects of $\AC$.

\begin{defn}
A full subcategory $\BC\subset \AC$ is \emph{essential} if it is closed under isomorphisms, and \emph{thick} if it is closed under taking direct sums and direct summands.
\end{defn}

\begin{remark}
Any thick, essential full subcategory is determined by the indecomposable objects in $\BC \cap \ind \AC$. 
\end{remark}


\begin{remark} To a full subcategory $\BC$, one can associate the ideal $I_{\BC}$ of morphisms in $\AC$ which (are linear combinations of morphisms which) factor through one of the
objects in $\BC$. Note that $I$ contains all morphisms in $\BC$, but also morphisms between objects not in $\BC$, which factor through objects in $\BC$. Adding direct sums, direct
summands, or isomorphic objects to $\BC$ will not change the ideal $I_{\BC}$. Conversely, given an ideal $I$, one can consider the class of objects whose identity maps are in $I$, and
the corresponding full subcategory $\BC_I$ consisting of those objects. Then $\BC_I$ will be essential and thick. This gives a bijection between thick, essential, full subcategories $\BC \subset \AC$ and ideals $I \subset \AC$ which are generated by identity maps.
%\footnote{Not every ideal is generated by identity maps.}.
\end{remark}

Let $\AC$ be an additive monoidal category which has uniqueness of direct sum decompositions. Then $[\AC]$ is an algebra over $\ZM$ with basis given by the set $\ind \AC$ of indecomposables in $\AC$, up to isomorphism. Multiplication is given by $[M][N] = [M \ot N]$. The notion of an ideal in an algebra is lifted to the notion of a tensor ideal.  We warn the reader that tensor ideals are subcategories of $\AC$, and are not ideals in the usual category theoretic sense.% for category $\AC$.

\begin{defn} A \emph{left tensor ideal} $\IC \subset \AC$ is a full subcategory which is essential and thick, such that $M \ot B \in \IC$ whenever $B \in \IC$. The definitions of a \emph{right tensor ideal} and a \emph{two-sided tensor ideal} are similar. \end{defn}

One could of course define the notion of a tensor ideal without the assumption of thickness, but in this paper all tensor ideals will be thick.

\begin{remark}
A tensor ideal is automatically closed under grading shifts, because $\one(k) \ot B \cong B(k)$.
\end{remark}

\begin{remark} An essential, thick full subcategory $\IC$ is a left tensor ideal if and only if the associated ideal $I_{\IC}$ is closed under tensor product with any morphism on the left. Similar statements can be made about right tensor ideals and two-sided tensor ideals.  \end{remark}

If $B\in \AC$, let $\AC B \subset \AC$ denote the full subcategory consisting of objects which are isomorphic to summands of objects of the form $M\otimes B$, with $M \in \AC$. Then $\AC B$ is a left tensor ideal. Equivalently, $\AC B$ is the smallest left tensor ideal containing $B$. We define $B\AC$ and $\AC B \AC$ similarly.

Given indecomposables $B,B'\in \AC$ we write $B \le_L B'$ if $\AC B \subset \AC B'$ or, equivalently, $B\in \AC B'$.  Unlike the previous section, this relation is always transitive.  Similarly, one defines $B \le_R B'$ if $B \in B' \AC$, and $B \le_{LR} B'$ if $B'\in \AC B\AC$.  The corresponding equivalence classes on $\ind \AC$ are called \emph{left cells}, \emph{right cells}, and \emph{two-sided cells} respectively.


\begin{remark}
Let $\IC\subset \AC$ be a left tensor  ideal, and let $B_1,\ldots,B_r \in \IC$ be objects such that any indecomposable $B\in \IC$ satisfies $B\leq_{L} B_i$ for some $i$.  Then $\IC = \AC(B_1\oplus \cdots \oplus B_r)$.  Thus (under appropriate finiteness conditions) every left tensor ideal arises as $\AC B$ for some $B$.  Similar considerations apply to right and two-sided tensor ideals. 
\end{remark}

\begin{remark} If $\AC$ is graded with grading shift functor $(1)$, then the left (or right, or two-sided) cells are closed under the grading shift since $\one(1)\otimes B\cong
B(1)\cong B\otimes \one(1)$.  More generally, scalar functors preserve cells.\end{remark}

Recall from \S\ref{subsec:SoergelCat} that an object in a Krull-Schmidt category is uniquely determined up to isomorphism by its symbol in the Grothendieck group. An implication for Krull-Schmidt monoidal categories is that the decomposition of tensor products into indecomposable objects is controlled by multiplication in the Grothendieck group. For sake of illustration, let $W$ be an indexing set for the isomorphism classes of indecomposable objects, named $\{B_w\}_{w \in W}$, and let $b_w = [B_w]$ denote the corresponding basis of $[\AC]$.


\begin{proposition}\label{prop:structureCoeffs}
If
\begin{equation} \label{eq:multdownstairs} b_w b_x = \sum c^y_{w,x} b_y \end{equation}
for integers $c^y_{w,x}$, then
\begin{equation} \label{eq:multupstairs} B_w \ot B_x \cong \bigoplus B_y^{\oplus c^y_{w,x}},\end{equation}
and in particular the coefficients $c^y_{w,x}$ are non-negative.\qed
\end{proposition}
If $\AC$ is graded with grading shift functor $(1)$ and $\{B_w\}_{w\in W}$ instead denotes a set of representatives of the indecomposables up to grading shift, then a similar statement is true, where now the structure coefficients $c^y_{w,x}$ are elements of $\Z[v,v\inv]$.

For this reason, the cell theory of $\AC$ and the cell theory of $[\AC]$ agree.

%\begin{proof}
% Let $\ind \AC$ denote the set of isomorphism classes of indecomposables in $\AC$.  We give $\ind \AC$ the structure of a semi-ring by
%\begin{subequations}
%\begin{equation}
%[A]+[B]:=[A\oplus B]
%\end{equation}
%\begin{equation}
%[A][B]:=[A\otimes B].
%\end{equation}
%\end{subequations}
%Then clearly the structure coefficients of $b_y$ in $b_wb_x$ agree with the structure coefficients of $B_y$ in $B_w\otimes B_x$.

%To obtain the Grothendieck group $[\AC]$, we formally adjoin additive inverses to $\ind \AC$.  It is easy to see that two elements in $[A],[B]\in \ind \AC$ have the same image in $[\AC]$ if and only if there exists $[C]\in \ind \AC$ such that $[A]+[C]=[B]+[C]$ or, equivalently, $A\oplus C\cong B\oplus C$.  But since $\AC$ has uniqueness of direct sum decompositions, we have the cancellation property, hence
%\[
%A\oplus C\cong B\oplus C \ \ \Rightarrow \ \ A\oplus B.
%\]
%We conclude that two indecomposable objects $A,B\in \AC$ satisfy $[A]-[B]=0$ in $[\AC]$ if and only if $A\cong B$.  The Proposition now follows easily.
%\end{proof}


%============================
\subsection{The example of $\HT_3$ and $\FT_3$}
\label{subsec:HT3}
%============================

We wish to prove some abstract results about the interaction between cells and operators which act like ``twists.'' Eventually, this will be applied to the Hecke algebra and the Soergel
category, and illustrates why cell theory is so important for diagonalizing the full twist. So, to motivate these results, we briefly illustrate this phenomenon for $S_3$.

As mentioned in Example \ref{ex:threecells}, $S_3$ has only three two-sided cells. In order from largest to smallest with respect to $\le_{LR}$, they are: the {\color{brown} identity cell}, the {\color{red} simple cell}, and the {\color{blue} longest cell}. We color-code these cells in the complexes below.

Direct computation shows that
\begin{eqnarray}
\HT_3 \ot {\color{brown}\one} & \simeq & \begin{tikzpicture}[baseline=-.5em]
%\tikzstyle{every node}=[font=\small]
\node  (a) at (0,0) {$\underline{{\color{blue} B_{sts}}}(0)$};
\node at (2,.5) {${\color{red}B_{st}}(1)$};
\node (b) at (2,0) {$\oplus$};
\node at (2,-.5) {${\color{red}B_{ts}}(1)$};
\node at (4,.5) {${\color{red}B_{s}}(2)$};
\node (c) at (4,0) {$\oplus$};
\node at (4,-.5) {${\color{red}B_{t}}(2)$};
\node (d) at (6,0) {${\color{brown}\one}(3)$};
\path[->,>=stealth',shorten >=1pt,auto,node distance=1.8cm,
  thick]
(a) edge node[above] {} (b)
(b) edge node[above] {} (c)
(c) edge node[above] {} (d);
\end{tikzpicture}
\\ \label{eq:HTBs}
\HT_3\otimes {\color{red}B_s} & \simeq & \begin{tikzpicture}[baseline=-.5em]
%\tikzstyle{every node}=[font=\small]
\node (a) at (0,0) {$\underline{{\color{blue}B_{sts}}}(-1)$};
\node (b) at (2,0) {${\color{red}B_{ts}}(0)$};
\path[->,>=stealth',shorten >=1pt,auto,node distance=1.8cm,
  thick]
(a) edge node[above] {} (b);
\end{tikzpicture}
\\ \label{eq:HTBst}
\HT_3\otimes {\color{red}B_{st}} & \simeq & \begin{tikzpicture}[baseline=-.5em]
%\tikzstyle{every node}=[font=\small]
\node (a) at (0,0) {$\underline{{\color{blue}B_{sts}}}(-2)$};
\node (b) at (2,0) {${\color{red}B_{t}}(0)$};
\path[->,>=stealth',shorten >=1pt,auto,node distance=1.8cm,
  thick]
(a) edge node[above] {} (b);
\end{tikzpicture}
\\
\HT_3\otimes {\color{blue}B_{sts}}  & \simeq &  \underline{{\color{blue}B_{sts}}}(-3)
\end{eqnarray}
Equations \eqref{eq:HTBs} and \eqref{eq:HTBst} are also true after swapping $s$ and $t$.  Here is the analogous computation for the full twist:

\[
\FT_3 \ot {\color{brown}\one} \ \ \simeq  \ \ \begin{tikzpicture}[baseline=-.2em]
\tikzstyle{every node}=[font=\scriptsize]
\node (a) at (0,0) {$\underline{{\color{blue}B_{sts}}}(-3)$};
\node at (2,.25) {${\color{blue}B_{sts}}(-1)$};
\node at (2,-.25) {${\color{blue}B_{sts}}(-1)$};
\node (c) at (4,.75) {${\color{blue}B_{sts}}(1)$};
\node at (4,.25) {${\color{blue}B_{sts}}(1)$};
\node at (4,-.25) {${\color{red}B_{s}}(1)$};
\node at (4,-.75) {${\color{red}B_{t}}(1)$};
\node (d) at (6,.5) {${\color{blue}B_{sts}}(3)$};
\node at (6,0) {${\color{red}B_{st}}(2)$};
\node at (6,-.5) {${\color{red}B_{ts}}(2)$};
\node (e) at (8,.25) {${\color{red}B_{st}}(4)$};
\node at (8,-.25) {${\color{red}B_{ts}}(4)$};
\node (f) at (10,.25) {${\color{red}B_{s}}(5)$};
\node at (10,-.25) {${\color{red}B_{t}}(5)$};
\node (g) at (12,0) {${\color{brown}\one}(6)$};
\node (b1) at (1.6,0) {};
\node (b2) at (2.4,0) {};
\node (c1) at (3.6,0) {};
\node (c2) at (4.4,0) {};
\node (d1) at (5.6,0) {};
\node (d2) at (6.4,0) {};
\node (e1) at (7.6,0) {};
\node (e2) at (8.4,0) {};
\node (f1) at (9.6,0) {};
\node (f2) at (10.4,0) {};
\path[->,>=stealth',shorten >=1pt,auto,node distance=1.8cm,
  thick]
(a) edge node[above] {} (b1)
(b2) edge node[above] {} (c1)
(c2) edge node[above] {} (d1)
(d2) edge node[above] {} (e1)
(e2) edge node[above] {} (f1)
(f2) edge node[above] {} (g);
\end{tikzpicture}
\]
\begin{eqnarray}
\label{eq:FTBs}
\FT_3\otimes {\color{red}B_s} & \simeq & \begin{tikzpicture}[baseline=-.2em]
%\tikzstyle{every node}=[font=\small]
\node (a) at (0,0) {$\underline{{\color{blue}B_{sts}}}(-4)$};
\node (b) at (2.5,0) {${\color{blue}B_{sts}}(-2)$};
\node (c) at (5,0) {${\color{red}B_{s}}(0)$};
\path[->,>=stealth',shorten >=1pt,auto,node distance=1.8cm,
  thick]
(a) edge node[above] {} (b)
(b) edge node[above] {} (c);
\end{tikzpicture}
\\  \label{eq:FTBst}
\FT_3\otimes {\color{red}B_{st}} & \simeq & \begin{tikzpicture}[baseline=-.2em]
%\tikzstyle{every node}=[font=\small]
\node (a) at (0,0) {$\underline{{\color{blue}B_{sts}}}(-5)$};
\node (b) at (2.5,0) {${\color{blue}B_{sts}}(-1)$};
\node (c) at (5,0) {${\color{red}B_{ts}}(0)$};
\path[->,>=stealth',shorten >=1pt,auto,node distance=1.8cm,
  thick]
(a) edge node[above] {} (b)
(b) edge node[above] {} (c);
\end{tikzpicture}
\\
\FT_3\otimes {\color{blue}B_{sts}}  & \simeq &  {\color{blue}B_{sts}}(-6)
\end{eqnarray}
Again, equations \eqref{eq:FTBs} and \eqref{eq:FTBst} remain true after swappping $s$ and $t$.

We make the following observation: let $F$ denote either $\HT_3$ or $\FT_3$.  Then for each two-sided cell $\l$ there exists a number $\nb_F(\l)$ such that:
\begin{itemize}
\item if $B$ is in cell $\l$ then every occurence of $B$ in the minimal complex of $F$ is in homological degrees $\geq \nb_F(\l)$.
\item if $B$ is in cell $\l$ then the minimal complex of $F\otimes B$ is supported in homological degrees $\leq \nb_F(\l)$.
\end{itemize}
Furthermore, $\FT_3$ appears to act in a ``block upper triangular'' fashion with respect to the cell filtration.  Before moving on to more advanced cell theory techniques, let us abstract the above phenomena.


%============================
\subsection{Twists in general}
\label{subsec:twists}
%============================

Throughout this section let $\AC$ be an additive monoidal category with the Krull-Schmidt property (so that there is a well behaved notion of minimal complexes).  


\begin{definition}\label{def:twistlike}  We say that an invertible minimal complex $F\in \KC^b(\AC)$ is \emph{twist-like} if for all $B \in \AC$, $F\otimes B\otimes F\inv$ is homotopy equivalent to an object of $\AC$ (that is, a complex supported in degree zero).  
\end{definition}
%Below, $F\in \KC^b(\AC)$ will always denote a twist-like complex.


\begin{remark}\label{rmk:twistsAreTwists}
We expect that the half and full twists are twist-like (hence the name) and, in fact, 
\[
\HT\otimes B_x\otimes \HT\inv \simeq B_{\tau(x)}, \qquad\qquad \FT\otimes B_x\otimes \FT\inv \simeq B_x.
\]
This is well known in type $A$ (and we include a proof in  \S \ref{subsec:commutehalffull}), but we are not aware of any proof in the literature for other finite Coxeter groups.
\end{remark}

Now we define some numerical invariants of twist-like complexes $F$.

\begin{defn}\label{def:nb}
If $F$ is twist-like and $B\in \AC$ is any object, let $\nb_F(B)$ denote the smallest integer such that $F\otimes B$ is homotopy equivalent to a complex supported in degrees $\leq \nb_F(B)$.
\end{defn}


\begin{lemma}\label{lemma:tailProperty}
If $F$ is twist-like and $B'\leq_{LR} B$, then $\nb_F(B')\leq \nb_F(B)$.  In particular, $\nb_F(B)$ depends only on the two-sided cell containing $B$.
\end{lemma}

\begin{proof}
Suppose $B'$ is a direct summand of $M\otimes B\otimes N$.  Then in the homotopy category $F\otimes B'$ is a direct summand of
\[
F\otimes M\otimes B\otimes N \simeq (F\otimes M\otimes F\inv) \otimes  (F\otimes B)\otimes N. 
\]
The first factor $F\otimes M\otimes F\inv$ is homotopy equivalent to an object of $\AC$ by assumption.  The middle factor $F\otimes B$ is homotopy equivalent to complex in degrees $\leq \nb_F(B)$.  From this, the inequality $\nb_F(B')\leq \nb_F(M\otimes B\otimes N)\leq \nb_F(B)$ is clear.
\end{proof}


\begin{definition}\label{def:n}
If $F\in \KC^b(\AC)$ is twist-like and $\l$ is a two-sided cell of $\AC$, let $\nb_F(\l)$ denote $\nb_F(B)$ for any $B\in \AC$ in cell $\l$.  We like to refer to the mapping $\l\mapsto \nb_F(\l)$ has the \emph{homological spectrum} of $F$. 
\end{definition}
Now we relate the invariants $\nb_F(\l)$ to some internally defined numbers.

\begin{definition}\label{def:m}
Let $\pb_F(\l)$ denote the smallest integer $k$ such that $F$ has a summand in cell $\l$ and homological degree $k$.  Let $\mb_F(\l)=\min\{\pb_F(\mu)\:|\: \l\leq \mu\}$.  We say that a twist-like $F$ is \emph{(strictly) increasing} if $\mb_F(\l)<\mb_F(\mu)$ whenever $\l<\mu$.
%We can analogously define $\pb_F^\ast(\l)$ to be the smallest integer $k$ such that $F$ has a summand in cell $\l$ and homological degree $k$.  Let $\mb_F(\l)=\min\{\pb_F(\mu)\:|\: \l\leq \mu\}$.  We say that $F$ is \emph{positive} twist-like if $\mb_F(\l)<\mb_F(\mu)$ whenever $\l<\mu$.$ (resp.~$\mb_F^\ast(\l)>\mb_F^\ast(\mu)$) 
\end{definition}
Note that $\mb_F(\l)\le\mb_F(\mu)$ whenever $\l < \mu$ by definition; a twist-like complex is strictly increasing if each of these inequalities is strict.  We will henceforth omit the adverb `strictly' when refering to increasing twist-like complexes.

\begin{lemma}
If $F$ is twist-like and increasing then $\pb_F(\l)=\mb_F(\l)$ for all two-sided cells $\l$.
\end{lemma}
\begin{proof}
Clear.
\end{proof}




\begin{example}
From the computations in \S \ref{subsec:HT3}, we see that for the half-twist $F=\HT_3$ we have
\begin{itemize}
\item $\nb_F(\l)=\mb_F(\l)=3$ for the maximal cell $\l=(3)$.
\item $\mb_F(\l)=\mb_F(\l)=1$ for the simple cell $\l=(2,1)$.
\item $\mb_F(\l)=\mb_F(\l)=0$ for the minimal cell $\l=(2,1)$.
\end{itemize}
Thus $\HT_3$ is twist-like and increasing according to the definition.
\end{example}

\begin{example} Let $G = \HT_3\inv$, the \emph{negative half twist}. Then $\nb_G(\l)=\mb_G(\l)=0$ for all cells $\l$.  Thus, $\HT_3\inv$ is twist-like but not increasing.
\end{example}

\begin{remark} \label{rmk:frombelow} We could analogously define numbers which bound homological degrees from below. For example, let $\nb^\ast_F(B)$ be the minimal homological degree appearing in the minimal complex of $F \ot B$. Define $\mb^\ast_F(\l)$ and $\pb^\ast_F(\l)$ analogously, as the largest homological degree $k$ where certain summands appear in $F$. A twist-like complex is \emph{(strictly) decreasing} if $\mb_F^\ast(\l)>\mb_F^\ast(\mu)$ when $\l<\mu$. Then $\HT_3\inv$ is strictly decreasing, while $\HT_3$ is not. \end{remark}
	
\begin{remark} We could also define statistics associated to tensor product with $F$ on the right, e.g. the maximal homological degree appearing in the minimal complex of $B \ot F$. We postpone discussion of these generalizations until later in this section. \end{remark}


%Note that, if $\pb_F(\l) \le \pb_F(\mu)$ whenever $\l \le \mu$, then $\pb_F(\l) = \mb_F(\l)$. Alternatively, if $\mb_F(\l) < \mb_F(\mu)$ whenever $\l < \mu$, then $\pb_F(\l) = \mb_F(\l)$ for all $\l$.  This condition will be satisfied (for every $\l$) for every example of interest in this paper, and the notation $\pb_F(\l)$ is not used outside of this section.


\begin{lemma} \label{lem:orderedeasy}
Let $F$ be twist-like.  If $\l$ and $\mu$ are two-sided cells and $\l \le \mu$, then $\nb_F(\l) \le \nb_F(\mu)$ and $\mb_F(\l) \le \mb_F(\mu)$. \end{lemma}

\begin{proof} That $\nb_F(\l)\le \nb_F(\mu)$ follows immediately from Lemma \ref{lemma:tailProperty}. That $\mb_F(\l) \le \mb_F(\mu)$ follows from the definition of $\mb_F$, since any cell $\ge \mu$ is also $\ge \l$. \end{proof}

\begin{lemma}\label{lemma:nmineq}
Let $F,G\in \KC^b(\AC)$ be twist-like complexes, and $\l$ be a fixed two-sided cell.  We have
\begin{subequations}
\begin{equation}\label{eq:nAndm}
\nb_F(\l)\geq \mb_F(\l)
\end{equation}
\begin{equation}\label{eq:nFG}
\nb_{G\otimes F}(\l)\leq \nb_F(\l)+\nb_G(\l)
\end{equation}
\begin{equation}\label{eq:mFG}
\mb_{G\otimes F}(\l)\geq \mb_{F}(\l)+\mb_G(\l).
\end{equation}
\end{subequations}
%Analogous statements hold for negative twist-like complexes.
\end{lemma}

\begin{proof}
Fix $B$ in cell $\l$, and let $X$ denote the minimal complex of $F \ot B$. Clearly $X$ is supported in homological degrees $\le \nb_F(\l)$ and in cells $\le \l$. In homological degrees $< \mb_F(\l)$, every summand of $F$ is in a cell $\ngeq \l$, so every summand of $X$ is in a cell $< \l$.

Suppose $\nb_F(\l)<\mb_F(\l)$. Then $X$ is supported in degrees $< \mb_F(\l)$, so the entire complex is supported in cells $< \l$. But this contradicts the fact that $B\simeq F\inv \ot X$ is in cell $\l$.   The inequality \eqref{eq:nAndm} follows.

To show \eqref{eq:nFG}, consider $G \ot X$. This is the total complex of a bicomplex whose $i$-th column is $G \ot X^i[-i]$, where $X^i$ denotes the $i$-th chain object of $X$. Since
each $X^i$ is in cells $\le \l$, $G \ot X^i$ will be supported in degrees $\le \nb_G(\l)$ (here we have used Lemma \ref{lem:orderedeasy}). Thus $G \ot X^i[-i]$ is supported in degrees
$\le i + \nb_G(\l)$. Since $X$ is supported in degrees $\le \nb_F(\l)$, $G \ot X$ is supported in degrees $\le \nb_F(\l) + \nb_G(\l)$, as desired.

The inequality \eqref{eq:mFG} is relatively straightforward. Every term in $G \ot F$ in homological degree $< \mb_F(\l) + \mb_G(\l)$ is a tensor product of a term in $F$ in degree $< \mb_F(\l)$, or a term in $G$ in degree $< \mb_G(\l)$. Consequently, such a term is in cells $\ngeq \l$.
\end{proof}


This property $\nb_F(\l)=\mb_F(\l)$ seems to be quite special, hence we give it a name.

\begin{definition}\label{def:lsharp}
We say that a twist-like complex $F\in \KC^b(\AC)$ is \emph{$\l$-sharp} if $\nb_F(\l)=\mb_F(\l)$.  We call $F$ \emph{sharp} if it is $\l$-sharp for all $\l$.
\end{definition}



\begin{lemma} \label{lem:sharptensorclosed} If $F$ and $G$ are $\l$-sharp then $\nb_{G \ot F}(\l) = \nb_F(\l) + \nb_G(\l)$. Sharp, $\l$-sharp (for any given $\l$), and sharp increasing complexes are closed under tensor product. %For any positive sharp complex, $\nb_F(\l)=\pb_F(\l) = \mb_F(\l)$.
\end{lemma}

\begin{proof} If $\nb_F(\l)=\mb_F(\l)$ and $\nb_G(\l)=\mb_G(\l)$, then the inequalities from Lemma \ref{lemma:nmineq} give:
\[
\mb_{G\otimes F}(\l) \leq \nb_{G\otimes F}(\l)\leq  \nb_F(\l) +\nb_G(\l) = \mb_F(\l) + \mb_G(\l) \leq \mb_{G\otimes F}(\l),
\]
which forces $\mb_{G\otimes F}(\l) = \nb_{G\otimes F}(\l) = \nb_F(\l) + \nb_G(\l)$. From this it is easy to deduce that $\l$-sharp, sharp, and sharp increasing complexes are closed under tensor product.
\end{proof}



Let us summarize and improve upon some of the ideas of Lemma \ref{lemma:nmineq}, in the special case that $F$ is $\l$-sharp.

\begin{lemma}\label{lemma:tail1}
Suppose $F$ is $\l$-sharp.  If $B$ is an indecomposable object in cell $\l$, then the minimal complex $X\simeq F\otimes B$ satisfies
\[
X = \cdots \rightarrow X^{\nb_F(\l)-2}  \rightarrow X^{\nb_F(\l)-1}  \rightarrow X^{\nb_F(\l)},
\]
where $X^k$ is in cells strictly less than $\l$ for $k< \nb_F(\l)$. Moreover, $X^{\nb_F(\l)}$ has exactly one indecomposable summand in cell $\l$, and the remaining summands are in cells $< \l$.
\end{lemma}

\begin{proof} The first paragraph of the proof of Lemma \ref{lemma:nmineq} implies that $X^k$ is in cells $<\l$ for $k < \mb_F(\l) = \nb_F(\l)$, and that $X^{\nb_F(\l)}$ is in cells $\le
\l$. It remains to prove that $X^{\nb_F(\l)}$ has exactly one indecomposable summand in cell $\l$.

Any invertible additive functor must preserve nonzero indecomposable objects. Since $B$ is indecomposable, so is $F \ot B \simeq X$. The same is true after passage to the Serre quotient
category $\AC/\IC_{< \l}$ or passage to $\KC^b(\AC/\IC_{<\l})$, which is still a monoidal category since $\IC_{< \l}$ is a two-sided tensor ideal. Since $B$ descends to a nonzero indecomposable object in this quotient, so
must $X$. But in the quotient, $X$ is supported in a single homological degree $\nb_F(\l)$. Thus $X^{\nb_F(\l)}$ must be indecomposable in the quotient, so it has exactly one
indecomposable summand in cell $\l$. \end{proof}

\begin{example}
To illustrate the statement of Lemma \ref{lemma:tail1}, take  $F=\HT_3$ and $\l=(2,1)$, so that $\nb_F(\l)=1$.  Note that $\HT_3\otimes B_s \simeq \underline{B_{sts}}(-1)\rightarrow B_{ts}$. The term in homological degree $1$ is in the same cell as $B_s$, and every term in smaller homological degree is in strictly smaller cells.

Moreover, the fact that $\HT_3$ is sharp implies that $\FT_3$ has the same property, by the last statement of Lemma \ref{lemma:nmineq}.
\end{example}

We can improve slightly on Lemma \ref{lemma:tail1} when $F$ is increasing, but for this we need some additional assumptions on $\AC$.

\begin{hypothesis}\label{hyp:rigidetc}
Let $\AC$ be a Krull-Schmidt additive monoidal category. For each indecomposable object $B$ in $\AC$ there is some $B^\vee$ in $\AC$ such that $(-) \ot B$ and $(-) \ot B^\vee$ are both left and right adjoint to each other. Moreover, $B^\vee$ is in the same cell as $B$. Finally, the cells are \emph{nondegenerate} in that, for each indecomposable $B$, there is some indecomposable $B'$ in the same cell as $B$ such that $B$ is isomorphic to a direct summand of $B \ot B'$. \end{hypothesis}


\begin{remark} \label{rmk:rigidmonoidal} The category $\SBim$ is rigid monoidal, where the duality functor $B\mapsto B^\vee$ is defined in the diagrammatic calculus by rotating all diagrams by 180 degrees. Since $B_s^\vee \cong
B_s$, one can deduce that $B_w^\vee \cong B_{w\inv}$. It is not necessarily obvious that $w$ and $w\inv$ are in the same two-sided cell, though it is true (see Proposition \ref{prop:someP}, (P14)).   In fact (see \S\ref{subsec:invact}), $\SBim$ satisfies the assumptions of Hypothesis \ref{hyp:rigidetc}, though this relies on more advanced cell theory of the Hecke algebra. \end{remark}

% Assume that $\AC$ is Krull-Schmidt and rigid. In particular, for each $B \in \AC$ there is some $B^\vee \in \AC$ such that $B \ot (-)$ and $B^\vee \ot (-)$ are biadjoint, and similarly for has the structure of a rigid monoidal category (in addition to its being Krull-Schmidt), hence comes equipped with a
% contravariant duality functor $B\mapsto B^\vee$, and that $B$ and $B^\vee$ are in the same two-sided cell for all indecomposables $B$. Assume that if $B$ is in two-sided cell $\l$ then
% there exists $B'$ in the same two-sided cell $\l$ such that $B$ is isomorphic to a direct summand of $B'\otimes B$.
% \end{hypothesis}

\begin{proposition}\label{prop:tailProperty}
Assume hypothesis \ref{hyp:rigidetc}. Let $F$ twist-like, sharp, and increasing. If $B$ is any indecomposable object in cell $\l$ and $X\simeq F\otimes B$ is the minimal complex, then
\begin{enumerate}
\item if $k<\nb(\l)$ then the chain object $X^k$ is in cells strictly less than $\l$.
\item if $k=\nb(\l)$ then the chain object $X^k$ is an indecomposable object in cell exactly $\l$.
\end{enumerate}
\end{proposition}

%\begin{remark} Note that the assumptions of this proposition imply that $\mb_F(\l) < \mb_F(\mu)$ for $\l < \mu$, and hence $\mb_F(\l) = \pb_F(\l)$. \end{remark}

\begin{proof} The statement (1) is part of Lemma \ref{lemma:tail1}. Now we prove (2). By Lemma \ref{lemma:tail1}, if (2) fails then $X^{\nb_F(\l)}$ has a direct summand of the form $C$, where $C$ is an indecomposable object in cell $\mu < \l$. The inclusion of $C$ induces a chain map $C[-\nb_F(\l)]\rightarrow X$, since $X$ is zero in homological degrees $>\nb_F(\l)$. This will also be thought of as a map $j:C\rightarrow X[\nb_F(\l)]$. We claim that $j$ is null homotopic. If we can prove this, then this would imply that the component of the differential $X^{\nb_F(\l)-1}\rightarrow C$ is projection to a direct summand (with a splitting provided by the null-homotopy for $j$). Then $C$ can be cancelled by a Gaussian
elimination, contradicting the fact that $X$ is a minimal complex.

It remains to prove that $j:C\rightarrow X[\nb_F(\l)]$ is null-homotopic.  By hypothesis \ref{hyp:rigidetc} , there is an object $C'$ in cell $\mu$ such that $C$ is isomorphic to a direct summand of $C \otimes C'$.  Taking hom complexes, it follows that $\Homc_{\AC}(C,X)\simeq \Homc_{\AC}(C, F\otimes B)$ is isomorphic to a direct summand of
\[
\Homc_{\AC}(C\otimes C', F\otimes B)\cong \Homc_{\AC}(C, F\otimes B \ot (C')^\vee)
\]
Meanwhile, each indecomposable summand of $B \ot (C')^\vee$ is in a cell $\le \mu < \l$. Thus $F\otimes B \ot (C')^\vee$ is homotopy equivalent to a complex supported in degrees $\leq \nb_F(\mu)<\nb_F(\l)$. It follows that the homology of $\Homc_{\AC}(C, F\otimes B)$ is supported in homological degrees $<\nb_F(\l)$.  In particular $j$ is null-homotopic.  This completes the proof.
\end{proof}


\begin{definition}\label{def:head} Assume the hypotheses and notation of Proposition \ref{prop:tailProperty}. The maximal chain object $X^{\nb_F(\l)}[-\nb_F(\l)]$ (viewed as a complex
living in its usual homological degree) will be referred to as the \emph{head} of $F\otimes B$. The \emph{tail} of $F\otimes B$ is by definition the truncated complex $\cdots \rightarrow
X^{\nb_F(\l)-2}\rightarrow X^{\nb_F(\l)-1}$. \end{definition}

\begin{remark} When $F$ is sharp but not increasing, one may wish to define the head of $F \ot B$ as the unique indecomposable summand of $X^{\nb_F(\l)}$ living in cell $\l$, guaranteed by Lemma \ref{lemma:tail1}. However, this summand need not be canonically defined (though it is canonically defined in the quotient category $\AC / \IC_{< \l}$).  For simplicity we restrict to increasing  twist-like complexes. %For many purposes it is still possible to discuss the head, but we choose not to.
\end{remark}


Each of the results in this section has an analog where the homological degrees are reversed, as in Remark \ref{rmk:frombelow}. We call $F$ \emph{co-sharp} if $\nb^\ast_F(\l) =
\mb^\ast_F(\l)$ for all $\l$. As noted above $\HT_3$ is sharp and increasing, while $\HT_3\inv$ is co-sharp and decreasing.

\begin{lemma} If $F$ is sharp and $F\inv$ is co-sharp, then $\nb^\ast_{F\inv}(\l) = -\nb_F(\l)$. Thus if $F$ is sharp and increasing and $F\inv$ is co-sharp, then $F\inv$ is also
decreasing. \end{lemma}

\begin{proof} Fix $B$ in cell $\l$. By Lemma \ref{lemma:tail1} we see that, modulo lower cells, $F \ot B$ is a single indecomposable in cell $\l$ and homological degree $\nb_F(\l)$.
Thus, by the same argument, $F\inv \ot (F \ot B)$ is (modulo lower cells) a single indecomposable in cell $\l$ and homological degree $\nb_F(\l) + \nb^\ast_{F\inv}(\l)$. But since this
tensor product is just $B$ in homological degree $0$, we must have $\nb_F(\l) + \nb^\ast_{F\inv}(\l) = 0$. \end{proof}


\begin{remark} Recall that $\nb_F(\l)$ and $\nb_F^\ast(\l)$ were defined in terms of $F\otimes B$ when $B\in\AC$ is in cell $\l$.  We could also have made similar definitions using $B\otimes F$, which we temporarily denote $\nb_{F,R}(\l)$ and $\nb_{F,R}^\ast(\l)$.  If $\AC$ satisfies Hypothesis \ref{hyp:rigidetc}, then taking right adjoints gives a functor $B \mapsto B^\vee$, and this functor extends to a functor on $\KC^(\AC)$ which reverses homological degree. Since the adjoint of an invertible operator is also its inverse, this functor sends $F$ to $F\inv$ and $F \ot B$ to $B^\vee \ot F\inv$. Consequently, $\nb_F(\l) = \nb^\ast_{F\inv,R}(\l)$ and $\nb_F^\ast(\l) = \nb_{F\inv,R}(\l)$.

Moreover, the category of Soergel bimodules has a covariant (!) autoequivalence which swaps the order of tensor products and preserves cells, namely $B_w\mapsto B_{w\inv}$. Applying this equivalence we see that $\nb_F(\l) = \nb_{F,R}(\l)$ for twist-like complexes in $\KC^b(\SBim_n)$.
\end{remark}  


% \begin{lemma}\label{lemma:headindec}
% The head of $F\otimes B$ is indecomposable. (Note: we assume $B$ is indecomposable.)
% \end{lemma}
%
% \begin{proof}
% Let $X\simeq F\otimes B$ be the minimal complex.  Assume that the head of $F\otimes B$ splits as a direct sum $A_1\oplus A_2$.  We must show that $A_1$ or $A_2$ is zero.
%
% Any properties proven for $F$ have analogues for $F\inv$.  In particular, whenever $C$ is in cell $\l$ we have
% \[
% F\inv\otimes C\simeq Y_{C}^{-\mb_F(\l)} \rightarrow Y_C^{1-\mb_F(\l)}\rightarrow \cdots
% \]
% where the left-most nonzero chain object (the \emph{co-head}) is in $\l$ and all other chain objects are in cells $<\l$.  Applying such simplifications termwise to
% \[
% F\inv \otimes X  = \Tot(\cdots \rightarrow  F\inv \otimes X^i\rightarrow F\inv\otimes X^{i+1}\rightarrow \cdots),
% \]
% we see that $B\simeq F\inv\otimes F\otimes B$ is homotopy equivalent to a complex which in degree zero has two terms $A_1'\oplus A_2'$ corresponding to co-heads of $F\inv\otimes A_i$.  All other homological degrees have objects only in cells strictly less than $\l$, hence $A_i'$ both survive in the minimal complex of $F\inv \otimes F\otimes B$.  On the other hand $B$ \emph{is} the  minimal complex; since $B$ is assumed indecomposable this forces $A_1'$ or $A_2'$ to be zero, which in turn forces $A_1$ or $A_2$ to be zero.
% %Let $\IC$ be the ideal of morphisms factoring through objects in cells $<\l$. Then each object in cell $<\l$ becomes zero in the quotient $\AC/\IC$, while each object in cells bigger than or incomparable to $\l$ remain nonzero by definition of cells.  Precisely, if $A'$ is in cells $<\l$ and if the identity morphism of $A$ factors as $A\rightarrow A'\rightarrow A$ then $A$ would be a retract, i.e.~a direct summand,  of $A'$ which implies that $A$ is in cells $<\l$, a contradiction.
% %
% %Now, we claim that if $A$ is in cell $\l$ and is indecomposable, then $A$ remains indecomposable in $\AC/\IC$.
% \end{proof}

%============================
\subsection{Cell triangularity}
\label{subsec:celltri}
%============================

A sharp twist-like complex $F$ seems to act in a block upper-triangular fashion with respect to cells. Namely, an indecomposable object $B$ in cell $\l$ is sent to $B'[-\nb_F(\l)]$ for
some other indecomposable object $B'$ in the same cell $\l$, modulo $\IC_{< \l}$. However, the indecomposables in a given cell can be permuted in an interesting way by $F$. 

\begin{example} Let $F = \HT_3$ and recall the examples in \S\ref{subsec:HT3}. For $B = \one$ we have $B' = \one(3)$; for $B = B_s$ we have $B' = B_{ts}(0)$; for $B = B_{ts}$ we have $B'
= B_s(0)$; for $B = B_{sts}$ we have $B' = B_{sts}(-3)$. \end{example}

\begin{example} Let $F = \FT_3$. Then $B'$ agrees with $B$ up to a grading shift which only depends on $\l$! \end{example}

The situation for $\FT_3$ is much nicer than for $\HT_3$, because the way in which $\FT_3$ acts on cells (in the associated graded) is encapsulated in an invertible scalar functor. Let us make a general definition.

Suppose $\AC$ comes equipped with a notion of scalar functors as in \S \ref{subsec:gradedcats}. A scalar functor acting on $\KC(\AC)$ is any functor isomorphic to a finite direct sum of composites of scalar functors in $\AC$ and homological shift functors. For example, if $\AC$ is graded, one choice is to let the scalar functors be grading shifts, i.e. a functor sending $B\mapsto B(k_1)\oplus \cdots \oplus B(k_r)$ for some $k_1,\ldots,k_r\in \Z$. Then an invertible scalar functor on $\KC(\AC)$ sends $B \mapsto B(a)[b]$ for some $a, b \in \Z$.

\begin{defn} \label{def:twistedunitri}
Let $F \in \KC^b(\AC)$ be twist-like, sharp, and increasing. We say that $F$ is \emph{cell triangular} if there is an invertible scalar functor $\Sigma_\l \co \KC^b(\AC) \to \KC^b(\AC)$ for each cell $\l$, such that if $B$ is in cell $\l$ then the head of $F \ot B$ is isomorphic to $\Sigma_\l(B)$.
\end{defn}

\begin{defn}\label{def:lequiv}
Let $F$ be cell triangular. We say that a chain map $\a_\l \co \Sigma_\l(\one) \to F$ is a \emph{$\l$-equivalence} if $\Cone(\a_\l) \ot B$ is homotopy equivalent to a complex in cells $<\l$.
\end{defn}

The $\l$-equivalence property can be reinterpreted via the following discussion.  If $B$ is in cell $\l$ then $F\otimes B$ is homotopy equivalent to a minimal complex of the form
\[
\cdots \rightarrow X^{\nb_F(\l)-2}\rightarrow X^{\nb_F(\l)-1}\rightarrow X^{\nb_F(\l)}.
\]
where $X^{\nb_F(\l)}[-\nb_F(\l)]=\Sigma_\l(B)$. Let $\iota_B \co \Sigma_\l(B) \to F \ot B$ denote the inclusion of the head of $F \ot B$.

\begin{lemma} \label{lemma:onlyiota}
Suppose that $\AC$ is $\KM$-linear, where $\KM$ is a field.  Let $B\in \AC$ be an indecomposable object in cell $\l$, with $\End(B)\cong \KM$.  Then the space of chain maps from $\Sigma_\l(B)$ to the minimal complex of $F \ot B$ is isomorphic to $\KM$, spanned by $\iota_B$.  If such a chain map is homotopic to zero, then it is equal to zero.
\end{lemma}

%\begin{lemma} \label{lemma:onlyiota}
%Let $B$ be an indecomposable object in cell $\l$, with $\End(B)$ a field. The space of chain maps from $\Sigma_\l(B)$ to the minimal complex of $F \ot B$ is spanned by $\iota_B$. Only the zero chain map is null-homotopic.
%\end{lemma}


\begin{proof} Let $X\simeq F\ot B$ be the minimal complex, and let $\Ch(\AC)$ denote the category of complexes (not considered up to homotopy).  Then $\Sigma_\l(B)$ lives in homological degree $\nb_F(\l)$ and $X$ lives in homological degrees $\leq \nb_F(\l)$, hence any chain map $\psi\in \Hom_{\Ch(\AC)}(\Sigma_\l(B), X)$ is supported in degree $\nb_F(\l)$ and has the same data of an endomorphism of $\Sigma_\l(B)$.  Observe that $\Sigma_\l$ is invertible in $\Ch(\AC)$, not just $\K(\AC)$.  Since $\End_{\AC}(B) \cong \End_{\Ch(\AC)}(\Sigma_\l(B))$ consists only of scalars, this $\psi$ is a $\KM$-multiple of $\iota_B$.  The vector space of chain maps modulo homotopy in this degree is a quotient of the space of chain maps, which is either zero or $\KM\cdot \iota_B$.  But it is not zero, since if $\iota$ were null-homotopic then any homotopy would factor through the tail of $X\simeq F \ot B$, which lives in cells $<\l$. However, the identity of $B$ is not in cells $< \l$, by the definition of cells. \end{proof}


Tensoring the map $\a_\l$ with the identity map of $B$ gives a map
\[
\a_\l \ot \Id_B \co \Sigma_\l(\one)\otimes B\cong \Sigma_\l(B) \rightarrow F\otimes B,
\]
such that $\Cone(\a_\l \ot \Id_B) \cong \Cone(\a_\l)\otimes B$. The chain map $\a_\l \ot \Id_B$ is therefore a multiple of $\iota_B$. If it is a nonzero multiple, then the two copies of $\Sigma_\l(B)$ can be removed via Gaussian Elimination from $\Cone(\a_\l \ot \Id_B)$, leaving only the tail of $F \ot B$ which is supported in lower cells. If not, then $\Cone(\a_\l \ot \Id_B)$ is a minimal complex, and cell $\l$ is there to stay. We summarize in this lemma, which is stated without the assumption that $\End(B)$ is a field.

\begin{lemma} \label{lem:lequivcriterion} A chain map $\a_\l$ is a $\l$-equivalence if and only if the degree $\nb_F(\l)$ component (which is the only nonzero component) of the chain map $\a_\l \ot \Id_B$ is an isomorphism, for all $B$ in cell $\l$. When $\End(B) \cong \End(\Sigma_\l(B))$ is a field, this is equivalent to $\a_\l \ot \Id_B \ne 0$.\qed \end{lemma}

Note that if $\a_\l:\Sigma_\l(\one)\rightarrow F$ is a $\l$-equivalence and $\b$ is any map homotopic to $\a_\l$, then $\b$ is a $\l$-equivalence since $\Cone(\b)\cong \Cone(\a_\l)$.

%The following lemma observes that we can think of $\l$-equivalences as morphism in the homotopy category, rather than as chain maps.

%\begin{lemma} \label{lem:lequivhomotopy} If $\a_\l$ is a $\l$-equivalence and $\b \simeq \a_\l$, then $\b$ is a $\l$-equivalence. \end{lemma}

%\begin{proof} All the chain maps discussed in this proof will be supported in a single homological degree $\nb_F(\l)$, so we will abuse notation slightly: given a chain map $\g$, we also let $\g$ refer to the unique non-zero morphism in this chain map, the part in homological degree $\nb_F(\l)$. For example, we might write $\g \in \IC$ if this unique non-zero morphism is in $\IC$, for an ideal $\IC \subset \AC$.
	
%Any homotopy from $\Sigma_\l(\one)$ to $F$ will factor through the homological degree $\nb_F(\l)-1$ part of $F$. Since $F$ is sharp, only objects in cells $\ngeq \l$ live in homological degree $\nb_F(\l)-1$. Thus for any nulhomotopic chain map $\g \co \Sigma_\l(\one) \to F$, $\g \in \IC_{\ngeq \l}$. Hence $\g \ot \Id_B \in \IC_{< \l}$, for any indecomposable $B$ in cell $\l$.

%Consequently, $\a_\l \ot \Id_B$ and $\b \ot \Id_B$ differ by a morphism in $\IC_{< \l}$. But any endomorphism in $\End(B)$ or $\End(\Sigma_\l(B))$ which factors through lower cells must live in the maximal ideal. Thus the degree $\nb_F(\l)$ part of $\a_\l \ot \Id_B$ is invertible if and only if the same is true for $\b \ot \Id_B$. Now Lemma \ref{lem:lequivcriterion} concludes the proof. \end{proof}

If one has a family of $\l$-equivalences for each cell $\l$, then $F$ is very close to being categorically prediagonalizable \cite[Definition 6.13]{ElHog17a}. 

\begin{prop}\label{prop:lequivmeansdiag} Suppose that $\AC$ is a category with finitely many cells, and $F \in \KC^b(\AC)$ is positive twist-like, sharp, and cell triangular. If there exists a $\l$-equivalence $\a_\l$ for each cell $\l$, then
\[
\bigotimes_{\l}\Cone(\a_\l)\simeq 0
\]
for a particular ordering of the tensor factors (see the proof).
%,
\end{prop}


\begin{remark}\label{rmk:obstructions} Recall that $\Cone(\a) \ot \Cone(\b)$ and $\Cone(\b) \ot \Cone(\a)$ need not be homotopy equivalent, so that the order on the tensor product matters. In \cite[\S 6.2]{ElHog17a}, the authors define certain obstructions, the vanishing of which guarantees that the cones tensor commute up to homotopy.  If the obstructions vanish, then the collection of maps $\{\a_\l\:|\: \text{$\l$ is a 2-sided cell}\}$ is said to be \emph{obstruction-free}.\end{remark}

% living in the graded space of chain maps up to homotopy $\one\rightarrow F^{\otimes 2}$

\begin{cor} \label{cor:lequivmeansdiag} If the $\l$-equivalences of Proposition \ref{prop:lequivmeansdiag} are obstruction-free, then $F$ is categorically prediagonalizable. \end{cor}

\begin{proof}[Proof of Proposition \ref{prop:lequivmeansdiag}] Let $\PC$ denote the finite poset of two-sided cells of $\AC$.  We need only show that \[ T := \bigotimes_{\l \in \PC} \Cone(\a_\l) \simeq 0
\] for some ordering on the tensor product.
	
For any ideal $J$ in the poset $\PC$ (e.g. $J = \{\le \mu\}$ or $J = \{\ngeq \l\}$), let $\IC_J\subset \AC$ denote the two-sided tensor ideal whose indecomposables are those in the cells
of $J$. Let $\ess \KC^b(\IC_J)\subset \KC^b(\AC)$ denote the full subcategory of (complexes which are homotopy equivalent to) complexes whose terms are in $\IC_J$.

Tensoring with $\Cone(\a_\l)$ will preserve $\ess \KC^b(\IC_J)$ for all ideals $J$, and will send $\ess \KC^b(\IC_{\le \l})$ to $\ess \KC^b(\IC_{< \l})$. Thus if $J_1 \subsetneq J_2$ are two ideals with $J_2 = J_1 \cup \{\l\}$, then tensoring with $\Cone(\a_\l)$ will send $\ess\KC^b(\IC_{J_1})$ to $\ess\KC^b(\IC_{J_2})$.

Pick any filtration $\emptyset = J_0 \subset J_1 \subset \cdots \subset J_k = \PC$ by ideals, with $J_n \setminus J_{n-1} = \{\l_n\}$ a singleton, and consider the tensor product $T$
with the order given by this filtration, so that $\l_k$ appears on the far right and $\l_1$ on the far left. Then tensor product with $T$ will send $\KC^b(\AC) = \KC^b(\IC_{J_k})$ to
$\KC^b(\IC_{J_0}) = 0$. Thus $T \ot \one \simeq 0$, and therefore $T \simeq 0$, as desired. \end{proof}

%============================
\subsection{Recap}
\label{subsec:recap}
%============================

We conjecture in this paper that the full twists $\FT$ are categorically diagonalizable, and in particular are categorically prediagonalizable. Our method of proof in type $A$ will begin by proving that $\FT_n$ is positive twist-like, sharp, and cell triangular. This will require a deep dive into the cell theory of Hecke algebras, which we discuss in the following section for arbitrary Coxeter groups. Then we will construct $\l$-equivalences for each cell $\l$. This will be done by computing the space of all maps from $\Sigma(\one)$ to $\FT_n$ for
invertible scalars $\Sigma$, and picking one with satisfactory properties, which is the content of \S\ref{sec:constructing}. An analogous computation shows that a certain obstruction space is trivial, so the $\l$-equivalences are obstruction-free. Then we will apply Corollary \ref{cor:lequivmeansdiag} to prove prediagonalizability.

