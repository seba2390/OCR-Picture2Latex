%%%%%%%%%%%%%%%%%%%%%%
\section{Examples}
\label{sec:examples}
%%%%%%%%%%%%%%%%%%%%%%

In this section we give examples of eigenmaps, finite quasi-idempotents, and projectors related to $\FT_n$ for $n \le 3$. When we choose to display the differentials or chain maps
explicitly (and often we do not, for reasons of space), we do this with the diagrammatic calculus, see \S\ref{subsec:diagram1}.

For shorthand, we write $\CB_\l$ to denote $\Cone(\a_\l)$.  

Throughout this section we will adopt the graphical calculus from \cite{EThick}.  We also introduce the following useful shorthand:
\begin{equation}\label{eq:triangleCalc}
\ig{1}{triang_red} = \frac{1}{2}\left(\ig{1}{leftpoly_red} - \ig{1}{rightpoly_red}\right).
\end{equation}

%========================
\subsection{Two strands}
%========================

Recall that
\begin{equation} \FT_2 = \left(\begin{tikzpicture}[baseline=-.5em]
%\tikzstyle{every node}=[font=\small]
\node  (a) at (0,0) {$\un{B_s}(-1)$};
\node (b) at (2.5,0) {$B_s(1)$};
\node (c) at (5,0) {$R(2)$};
\path[->,>=stealth',shorten >=1pt,auto,node distance=1.8cm,thick]
	(a) edge node[above] {$\ig{.8}{shorttriang_red}$} (b)
	(b) edge node[above] {$\ig{.8}{counit_red}$} (c);
\end{tikzpicture} \right)
\label{eq:FT2} \end{equation}
where the underline indicates homological degree zero. Our two eigenmaps are given as follows.

\begin{equation} \FT_2 = \left(\begin{tikzpicture}[baseline=-.5em]
%\tikzstyle{every node}=[font=\small]
\node  (a) at (0,0) {$\un{B_s}(-1)$};
\node (b) at (2.5,0) {$B_s(1)$};
\node (c) at (5,0) {$R(2)$};
\node (d) at (0,-2) {$R(-2)[0]$};
\node  (e) at (5,-2) {$R(2)[-2]$};
\path[->,>=stealth',shorten >=1pt,auto,node distance=1.8cm,thick]
	(a) edge node[above] {$\ig{.8}{shorttriang_red}$} (b)
	(b) edge node[above] {$\ig{.8}{counit_red}$} (c)
	(d) edge node[right] {$\a_{\yoo}$} (a)
	(d) edge node[left] {$\ig{.8}{unit_red}$} (a)
	(e) edge node[right] {$\a_{\yt}$} (c)
	(e) edge node[left] {$\Id$} (c);
\end{tikzpicture} \right)
\label{eq:FT2_withmaps} \end{equation}


Thus
\begin{equation} \CB_{\yoo} \simeq \left( R(-2) \longrightarrow \un{B_s}(-1) \longrightarrow B_s(1) \longrightarrow R(2) \right), \end{equation}
\begin{equation} \CB_{\yt} \simeq \left( \un{B_s}(-1) \longrightarrow B_s(1) \right). \end{equation}
It is not hard to confirm directly that $\CB_{\yoo} \ot B_s \simeq 0 \simeq B_s \ot \CB_{\yoo}$. Since $\CB_{\yt}$ is built from $B_s$, we see that $\CB_{\yoo} \ot \CB_{\yt} \simeq 0\simeq \CB_{\yt}\otimes \CB_{\yoo}$.   It is also easy to verify that $\CB_{\yoo}$ and $\CB_{\yt}$ are quasi-idempotent.

With only two eigenvalues, we have
\begin{equation} \KB_{\sytoct}\cong \KB_{\yoo} = \CB_{\yt}, \end{equation}
\begin{equation}\KB_{\sytot} \cong  \KB_{\yt} = \CB_{\yoo}. \end{equation}

One obtains $\PB_{\yt}$ by gluing together infinitely many copies of $\KB_{\yoo}$ as in the picture below.
\[
\PB_{\yt} = \left(\begin{diagram}[small]
&& &&&& R(-4) & \rTo^{\ig{.8}{unit_red}}  & B_{s}(-3) &\rTo^{\ig{.8}{shorttriang_red}} & B_{s}(-1)& \rTo^{\ig{.8}{counit_red}} & \underline{R}\\
%
&& &&&& & \rdTo^{-\Id} & &&&&\\
%
&& R(-8) & \rTo^{\ig{.8}{unit_red}}  & B_{s}(-7) &\rTo^{\ig{.8}{shorttriang_red}} & B_{s}(-5)& \rTo^{\ig{.8}{counit_red}}  & R(-4) &&&&\\
%
&& & \rdTo^{-\Id} & &&&& &&&&\\
%
\cdots &\rTo^{\ig{.8}{shorttriang_red}} &  B_{s} &\rTo^{\ig{.8}{counit_red}}  & R(-8) &&&& &&&&
\end{diagram}\right),
\]
which is homotopy equivalent to
\[
\PB_{\yt} \simeq \left(\begin{diagram}[small]
\cdots  & \rTo^{\ig{.8}{twodots_red}}  & B_s(-7)  & \rTo^{\ig{.8}{shorttriang_red}} & B_{s}(-5) & \rTo^{\ig{.8}{twodots_red}}  & B_{s}(-3) &\rTo^{\ig{.8}{shorttriang_red}} & B_{s}(-1)& \rTo^{\ig{.8}{counit_red}} & \underline{R}
\end{diagram}\right).
\]
The complementary idempotent $\PB_{\yt}$ is obtained as below. 
\[
\PB_{\yoo} = \left(\begin{diagram}[small]
\cdots  & \rTo^{\ig{.8}{twodots_red}}  & B_s(-7)  & \rTo^{\ig{.8}{shorttriang_red}} & B_{s}(-5) & \rTo^{\ig{.8}{twodots_red}}  & B_{s}(-3) &\rTo^{\ig{.8}{shorttriang_red}} & \underline{B_{s}(-1)}
\end{diagram}\right).
\]
This is clearly a locally finite convolution constructed from shifted copies of $\KB_{\yt}$.



%========================
\subsection{Three strands}
%========================
We denote $s$ by the color red and $t$ by the color blue in the diagrams below.

\subsubsection{The eigenmaps}
The minimal complex of $\FT_3$, with its eigenmaps, is pictured below.
\begin{equation}
\begin{tikzpicture}[baseline=-.2em]
\tikzstyle{every node}=[font=\scriptsize]
\node (a) at (0,0) {$\underline{{B_{sts}}}(-3)$};
\node at (2,.25) {$B_{sts}(-1)$};
\node at (2,-.25) {$B_{sts}(-1)$};
\node (c) at (4,.75) {$B_{sts}(1)$};
\node at (4,.25) {$B_{sts}(1)$};
\node at (4,-.25) {$B_{s}(1)$};
\node (yt) at (4,-.75) {$B_{t}(1)$};
\node (d) at (6,.5) {$B_{sts}(3)$};
\node at (6,0) {$B_{st}(2)$};
\node at (6,-.5) {$B_{ts}(2)$};
\node (e) at (8,.25) {$B_{st}(4)$};
\node at (8,-.25) {$B_{ts}(4)$};
\node (f) at (10,.25) {$B_{s}(5)$};
\node at (10,-.25) {$B_{t}(5)$};
\node (g) at (12,0) {$\one(6)$};
\node (x) at (0,-4) {$\one(-6)[0]$};
\node (y) at (4,-4) {$\one(0)[-2]$};
\node (z) at (12,-4) {$\one(6)[-6]$};
\node (b1) at (1.6,0) {};
\node (b2) at (2.4,0) {};
\node (c1) at (3.6,0) {};
\node (c2) at (4.4,0) {};
\node (d1) at (5.6,0) {};
\node (d2) at (6.4,0) {};
\node (e1) at (7.6,0) {};
\node (e2) at (8.4,0) {};
\node (f1) at (9.6,0) {};
\node (f2) at (10.4,0) {};
\path[->,>=stealth',shorten >=1pt,auto,node distance=1.8cm,
  thick]
(a) edge node[above] {} (b1)
(b2) edge node[above] {} (c1)
(c2) edge node[above] {} (d1)
(d2) edge node[above] {} (e1)
(e2) edge node[above] {} (f1)
(f2) edge node[above] {} (g)
(x) edge node[right] {$\a_{\yooo}$} (a)
(y) edge node[right] {$\a_{\yto}$} (yt)
(z) edge node[right] {$\a_{\yh}$} (g)
(x) edge node[left] {$\ig{.8}{threedots}$} (a)
(y) edge node[left] {$\sqmatrix{0\\0\\ \ \ \ig{.8}{unit_red} \ \  \\ \ig{.8}{unit_blue}}$} (yt)
(z) edge node[left] {$\Id$} (g);
\end{tikzpicture}
\end{equation}
The components $d^k:C^k\rightarrow C^{k+1}$ of the differential in this compex is given by the following matrices:
\begin{equation}
d^0 =\sqmatrix{-\ig{.8}{midtriang_blue}\\ \ig{.8}{midtriang_red}},\qquad
%
d^1 = \sqmatrix{\ig{.8}{righttriang_red} & \ig{.8}{lefttriang_blue} \\ \ig{.8}{lefttriang_red} & \ig{.8}{righttriang_blue} \\ \ig{.8}{sts_to_s} & 0 \\ 0 & \ig{.8}{sts_to_t}}, \qquad
%
d^2 = \sqmatrix{\ig{.8}{lefttriang_red} & -\ig{.8}{righttriang_red} & 0 & 0 \\ 0 & \ig{.8}{sts_to_st} & -\ig{.8}{rightunit_redblue} & \ig{.8}{leftunit_redblue}\\ \ig{.8}{sts_to_ts} & 0 & \ig{.8}{leftunit_bluered} & -\ig{.8}{rightunit_bluered}},
\end{equation}
\[
d^3 = \sqmatrix{\ig{.8}{sts_to_st} &  -\ig{.8}{lefttriang_redblue} +  \ig{.8}{lefttwodots_redblue} & \ig{.8}{upleft_red} \\ \ig{.8}{sts_to_ts} &  \ig{.8}{upright_red} &  \ig{.8}{righttriang_bluered} +  \ig{.8}{lefttwodots_bluered} },\qquad
%
d^4 = \sqmatrix{-\ig{.8}{rightcounit_redblue} & \ig{.8}{leftcounit_bluered} \\ \ig{.8}{leftcounit_redblue} & -\ig{.8}{rightcounit_bluered}},\qquad
%
d^5 = \sqmatrix{\ig{.8}{counit_red} & \ig{.8}{counit_blue}}.
\]
The purple strand represents the identity of $B_{sts}$, and the diagrams involving it are explained in \cite[\S 4]{EThick}.

Let us indicate why each of the above maps $\a_\l$ is actually a $\l$-equivalence. 

For $\l=\yooo$, recall that the homology of a Rouquier complex $F(\b)$ (in the usual sense for complexes of $R$-bimodules) depends only on the permutation represented by $\b$ and the number of crossings in $\b$ (this is easy to see using the technology of standard bimodules).  In particular the homology $\FT_3$ is isomorphic to $R(-6)$, and in fact the map $\a_{\yooo}$ is a quasi-isomorphism of complexes of $R$-bimodules.  From this there are various approaches (in addition to direct computation) which one can use to quickly deduce that $\CB_{\yooo} \ot B_{w_0} \simeq 0$, and therefore $\a_{\yooo}$ is a $\yooo$-equivalence. One was done in \cite{AbHog17}.  Another
uses standard filtrations and their interaction with the functor $(-) \ot B_{w_0}$, together with work of Libedinsky and Williamson \cite{LibWil}.

It is obvious that $\a_{\yh}$ is a $\yh$-equivalence, since its cone is homotopy equivalent to a complex with no summands of the form $R(a)[b]$.

Now, consider $\Cone(\a_{\yto})$.  Note that the bimodules in cells $\neq \yooo$ form a subcomplex of the form
\begin{equation} \Phi = 
\begin{tikzpicture}[baseline=-.2em]
\tikzstyle{every node}=[font=\scriptsize]
%\node (a) at (0,0) {$\underline{0}$};
\node (b) at (2,0) {$\one(0)$};
\node (c) at (4,.25) {$B_{s}(1)$};
\node at (4,-.25) {$B_{t}(1)$};
\node (d) at (6,.25) {$B_{st}(2)$};
\node at (6,-.25) {$B_{ts}(2)$};
\node (e) at (8,.25) {$B_{st}(4)$};
\node at (8,-.25) {$B_{ts}(4)$};
\node (f) at (10,.25) {$B_{s}(5)$};
\node at (10,-.25) {$B_{t}(5)$};
\node (g) at (12,0) {$\one(6)$};
\node (b1) at (1.6,0) {};
\node (b2) at (2.4,0) {};
\node (c1) at (3.6,0) {};
\node (c2) at (4.4,0) {};
\node (d1) at (5.6,0) {};
\node (d2) at (6.4,0) {};
\node (e1) at (7.6,0) {};
\node (e2) at (8.4,0) {};
\node (f1) at (9.6,0) {};
\node (f2) at (10.4,0) {};
\path[->,>=stealth',shorten >=1pt,auto,node distance=1.8cm,
  thick]
%(a) edge node[above] {} (b1)
(b2) edge node[above] {} (c1)
(c2) edge node[above] {} (d1)
(d2) edge node[above] {} (e1)
(e2) edge node[above] {} (f1)
(f2) edge node[above] {} (g);
\end{tikzpicture}
\end{equation}
(The first term is in homological degree 1.)
	
This lovely complex $\Phi$ has the property that
\begin{equation} \label{eq:PhiBs}\Phi \ot B_s \cong \left( B_{w_0}(2) \longrightarrow B_{w_0}(4) \right), \end{equation}
\begin{equation} \label{eq:PhiBt}\Phi \ot B_t \cong \left( B_{w_0}(2) \longrightarrow B_{w_0}(4) \right). \end{equation} 
(The first term is in homological degree 3.)
From this it follows that $\CB_{\yto}$ sends the simple cell to the longest cell, so $\a_{\yto}$ is a $\yto$-equivalence.  We remark that while \eqref{eq:PhiBs} and \eqref{eq:PhiBt} look similar, the differentials are different and the complexes are non-isomorphic.


\begin{remark} The reader can observe from the differentials above that $\FT_3$ has a filtration $0 \subset F_{\yh} \subset F_{\yto} \subset F_{\yooo} = \FT_3$, where $F_{\l}$ is built
from shifts of objects $B_w$ in cells $\ge \l$. Moreover, the eigenmap $\a_\l$ maps into the subcomplex $F_\l$. The complex $\Phi$ above was the cone of the map $\Sigma_{\yto} \to
F_{\yto}$.

That such a filtration should exist is not obvious. For example, it is not clear that the components of the differential in $\FT_3$ mapping from $B_s(1) \oplus B_t(1)$ in homological
degree $2$ to $B_{sts}(3)$ in homological degree $3$ should be zero.

One can ask whether, for any finite Coxeter group, the full twist $\FT$ has a filtration as above, by subcomplexes associated with two-sided cells. We prove this for dihedral groups in
the sequel. \end{remark}

\begin{remark} Minimal complexes are well-defined up to isomorphism of chain complexes. The minimal complex $\FT_3$ does have non-trivial automorphisms, the most interesting of which come from the non-trivial maps $B_{st}(2) \to B_{sts}(3)$ and $B_{ts}(2) \to B_{sts}(3)$ in homological degree $3$. A subtle point is that an isomorphic minimal complex may give rise to a different filtration $0 \subset F'_{\yh} \subset F'_{\yto} \subset F'_{\yooo} \cong \FT_3$, where $F'_{\yto} \ncong F_{\yto}$ as complexes in $\KC^b(\SBim_n)$ (though they are isomorphic modulo $\SBim_{< \yto}$). \end{remark}

\subsubsection{The quasi-idempotents}

We have four tableaux for $n=3$, hence four complexes $\KB_T$:
\begin{equation}
\KB_{\sytoth} = \CB_{\yoo} \ot \CB_{\yto},\qquad \KB_{\sytotch} = \CB_{\yoo} \ot \CB_{\yh},\qquad \KB_{\sytohct} = \CB_{\yt} \ot \CB_{\yooo},\qquad \KB_{\sytoctch} = \CB_{\yt} \ot \CB_{\yto}.
\end{equation}
Let's record for posterity the involutions corresponding to these tableaux: $1, t,s,sts$, in this order.

We have found explicit forms of their minimal complexes which we include below.%  We do not include the differentials, nor the details of how the computations are accomplished.

The minimal complex of $\KB_{\sytoth}$ is 
\begin{equation}\label{eq:K123}
\begin{tikzpicture}[baseline=-.2em]
\tikzstyle{every node}=[font=\scriptsize]
%
\node (a) at (0,0) {$\one(-2)$};
%
\node at (1.7,.25) {$B_{s}(-1)$};
\node at (1.7,-.25) {$B_{t}(-1)$};
\node (b) at (1.7,0) {\ \ \ \ \ \ \ \ \ \  };
%
\node at (3.4,.25) {$B_{ts}(0)$};
\node at (3.4,-.25) {$B_{st}(0)$};
\node (c) at (3.4,0) {\ \ \ \ \ \ \ \ \ \  };
%
\node at (5.1,.75) {$B_{sts}(1)$};
\node at (5.1,.25) {$B_{st}(2)$};
\node at (5.1,-.25) {$B_{ts}(2)$};
\node at (5.1,-.75) {$\one(2)$};
\node (d) at (5.1,0) {\ \ \ \ \ \ \ \ \ \  };
%
\node at (6.8,1.25) {$B_{sts}(3)$};
\node at (6.8,.75) {$B_{sts}(3)$};
\node at (6.8,.25) {$B_{s}(3)$};
\node at (6.8,-.25) {$B_{s}(3)$};
\node at (6.8,-.75) {$B_{t}(3)$};
\node at (6.8,-1.25) {$B_{t}(3)$};
\node (e) at (6.8,0) {\ \ \ \ \ \ \ \ \ \ };
%
\node at (8.5,.75) {$B_{sts}(5)$};
\node at (8.5,.25) {$B_{st}(4)$};
\node at (8.5,-.25) {$B_{ts}(4)$};
\node at (8.5,-.75) {$\one(4)$};
\node (f) at (8.5,0) {\ \ \ \ \ \ \ \ \ \  };
%
\node at (10.2,.25) {$B_{ts}(6)$};
\node at (10.2,-.25) {$B_{st}(6)$};
\node (g) at (10.2,0) {\ \ \ \ \ \ \ \ \ \ };
%
\node at (11.9,.25) {$B_{s}(7)$};
\node at (11.9,-.25) {$B_{t}(7)$};
\node (h) at (11.9,0) {\ \ \ \ \ \ \ \ \ \  };
%
\node (i) at (13.6,0) {$\one(8)$};
%
\path[->,>=stealth',shorten >=1pt,auto,node distance=1.8cm,
  thick]
(a) edge node{} (b)
(b) edge node{} (c)
(c) edge node{} (d)
(d) edge node{} (e)
(e) edge node{} (f)
(f) edge node{} (g)
(g) edge node{} (h)
(h) edge node{} (i);
\end{tikzpicture}.
\end{equation}
Here, the leftmost term is in homological degree 0.

The minimal complex of $\KB_{\sytotch}$ is
\begin{equation}\label{eq:K12c3}
\begin{tikzpicture}[baseline=-.2em]
\tikzstyle{every node}=[font=\scriptsize]
\node (b) at (1.7,0) {$B_{t}(-1)$};
%
\node at (3.4,.25) {$B_{ts}(0)$};
\node at (3.4,-.25) {$B_{st}(0)$};
\node (c) at (3.4,0) {\ \ \ \ \ \ \ \ \ \  };
%
\node at (5.1,.75) {$B_{sts}(1)$};
\node at (5.1,.25) {$B_{st}(2)$};
\node at (5.1,-.25) {$B_{ts}(2)$};
\node at (5.1,-.75) {$B_s(1)$};
\node (d) at (5.1,0) {\ \ \ \ \ \ \ \ \ \  };
%
\node at (6.8,1.25) {$B_{sts}(3)$};
\node at (6.8,.75) {$B_{sts}(3)$};
\node at (6.8,.25) {$B_{s}(3)$};
\node at (6.8,-.25) {$B_{s}(3)$};
\node at (6.8,-.75) {$B_{t}(3)$};
\node at (6.8,-1.25) {$B_{t}(3)$};
\node (e) at (6.8,0) {\ \ \ \ \ \ \ \ \ \ };
%
\node at (8.5,.75) {$B_{sts}(5)$};
\node at (8.5,.25) {$B_{st}(4)$};
\node at (8.5,-.25) {$B_{ts}(4)$};
\node at (8.5,-.75) {$B_s(5)$};
\node (f) at (8.5,0) {\ \ \ \ \ \ \ \ \ \  };
%
\node at (10.2,.25) {$B_{ts}(6)$};
\node at (10.2,-.25) {$B_{st}(6)$};
\node (g) at (10.2,0) {\ \ \ \ \ \ \ \ \ \ };
%
\node (h) at (11.9,0) {$B_{t}(7)$};
%
\path[->,>=stealth',shorten >=1pt,auto,node distance=1.8cm,
  thick]
(b) edge node{} (c)
(c) edge node{} (d)
(d) edge node{} (e)
(e) edge node{} (f)
(f) edge node{} (g)
(g) edge node{} (h);
\end{tikzpicture}
\end{equation}
Here the left-most term is in homological degree 1.


The minimal complex of $\KB_{\sytohct}$ is
\begin{equation}\label{eq:K13c2}
\begin{tikzpicture}[baseline=-.2em]
\node (b) at (0,0) {$B_{s}(-7)$};
%
\node at (2.5,.25) {$B_{sts}(-5)$};
\node at (2.5,-.25) {$B_{s}(-5)$};
\node (c) at (2.5,0) {\ \ \ \ \ \ \ \ \ \  \ \ \ \  };
%
\node at (5,.25) {$B_{sts}(-3)$};
\node at (5,-.25) {$B_{sts}(-3)$};
\node (d) at (5,0) {\ \ \ \ \ \ \ \ \ \   \ \ \ \ };
%
\node at (7.5,.25) {$B_{sts}(-1)$};
\node at (7.5,-.25) {$B_{s}(-1)$};
\node (e) at (7.5,0) {\ \ \ \ \ \ \ \ \ \  \ \ \ \ };
%
\node (f) at (10,0) {$B_s(1)$};
\path[->,>=stealth',shorten >=1pt,auto,node distance=1.8cm,
  thick]
(b) edge node{} (c)
(c) edge node{} (d)
(d) edge node{} (e)
(e) edge node{} (f);
\end{tikzpicture}.
\end{equation}
The left-most term is in homological degree $-1$.

Finally, the minimal complex of $\KB_{\sytoctch}$ is
\begin{equation}\label{eq:K1c2c3}
\begin{tikzpicture}[baseline=-.2em]
\node (c) at (0,0) {$B_{sts}(-5)$};
%
\node at (2.5,.25) {$B_{sts}(-3)$};
\node at (2.5,-.25) {$B_{st}(-3)$};
\node (d) at (2.5,0) {\ \ \ \ \ \ \ \ \ \  \ \ \ \  };
%
\node (e) at (5,0) {$B_{sts}(-1)$};
%
\path[->,>=stealth',shorten >=1pt,auto,node distance=1.8cm,
  thick]
(c) edge node{} (d)
(d) edge node{} (e);
\end{tikzpicture}.
\end{equation}
Where the left-most nonzero term is in homological degree 0.


Let us illustrate Corollary \ref{cor:KBtoo}, and give some remarks on the computations above.  For $\KB_{\sytoth}$, note that $\CB_{\yoo}\otimes (-)$ annihilates $B_w$ for $w=s, st, sts$.  Thus, to compute $\CB_{\yoo}\otimes \CB_{\yto}$ we may as well tensor $\CB_{\yoo}$ with $\Phi$, contract the contractible terms, and we obtain a convolution of the form (omitting shifts)
\begin{equation}
\KB_{\sytoth}  \ \ \simeq \ \ \left(\CB_{\yoo} \rightarrow \CB_{\yoo}  B_t\rightarrow \CB_{\yoo}B_{ts}\rightarrow \CB_{\yoo} B_{ts}\rightarrow \CB_{\yoo} B_t\rightarrow \CB_{\yoo}\right).
\end{equation}
Expanding $\CB_{\yoo}$, keeping track of differentials, and peforming a pair of Gaussian eliminations yields \eqref{eq:K123}. It is easy to prove that $\KB_{\sytoth} \ot (-)$ and $(-) \ot \KB_{\sytoth}$ annihilates any $B_w$ for $w \ne 1$, as tensoring with $\CB_{\yto}$ takes any such $B_w$ to a complex built from $B_{sts}$, and $\CB_{\yoo}$ kills $B_{sts}$.

We have a similar convolution description
\begin{equation}
\KB_{\sytotch}  \ \ \simeq \ \ \left(\CB_{\yoo}  B_t \rightarrow \CB_{\yoo}B_{ts}\rightarrow \CB_{\yoo} B_{ts}\rightarrow \CB_{\yoo} B_t\right),
\end{equation}
which yields \eqref{eq:K12c3} after expanding and simplifying. In fact, this gives rise to the formula
\begin{equation}
\KB_{\sytotch}  \ \ \simeq \ \ \CB_{\yoo} \ot B_t \ot \CB_{\yoo}.
\end{equation}
Now it is clear that $\KB_{\sytotch} \ot (-)$ will kill $B_w$ whenever $sw<w$, and $(-) \ot \KB_{\sytotch}$ will kill $B_w$ whenever $ws < w$.

Now, recall that $\CB_{\yt}=(B_s(-1)\rightarrow B_s(1))$.  Given that $B_s\otimes \FT_3\simeq (B_{sts}(-4)\rightarrow B_{sts}(-2)\rightarrow B_s(0))$ (see \eqref{eq:FTBs}) it follows that
\[
\CB_{\yt}\otimes \FT_3\simeq \Tot\left(
\begin{tikzpicture}[baseline=-2.5em]
\node (a) at (0,0) {$B_{sts}(-5)$};
\node (b) at (0,-2) {$B_{sts}(-3)$};
\node (c) at (2.5,0) {$B_{sts}(-3)$};
\node (d) at (2.5,-2) {$B_{sts}(-1)$};
\node (e) at (5,0) {$B_{s}(-2)$};
\node (f) at (5,-2) {$B_{s}(0)$};
%
\path[->,>=stealth',shorten >=1pt,auto,node distance=1.8cm,
  thick]
(a) edge node{} (c)
(c) edge node{} (e)
(b) edge node{} (d)
(d) edge node{} (f)
(a) edge node{} (b)
(c) edge node{} (d)
(e) edge node{} (f);
\end{tikzpicture}
\right).
\]
Then $\KB_{\sytohct}$ and $\KB_{\sytoctch}$ can both be expressed as mapping cones of maps
\[
\left(
\begin{tikzpicture}[baseline=-2.5em]
\node (a) at (0,0) {$B_{s}(-1)$};
\node (b) at (0,-2) {$B_{s}(1)$};
%
\path[->,>=stealth',shorten >=1pt,auto,node distance=1.8cm,
  thick]
(a) edge node{} (b);
\end{tikzpicture}
\right)(\text{shift})
\ \ \ \rightarrow \ \ \ \left(
\begin{tikzpicture}[baseline=-2.5em]
\node (a) at (0,0) {$B_{sts}(-5)$};
\node (b) at (0,-2) {$B_{sts}(-3)$};
\node (c) at (2.5,0) {$B_{sts}(-3)$};
\node (d) at (2.5,-2) {$B_{sts}(-1)$};
\node (e) at (5,0) {$B_{s}(-2)$};
\node (f) at (5,-2) {$B_{s}(0)$};
%
\path[->,>=stealth',shorten >=1pt,auto,node distance=1.8cm,
  thick]
(a) edge node{} (c)
(c) edge node{} (e)
(b) edge node{} (d)
(d) edge node{} (f)
(a) edge node{} (b)
(c) edge node{} (d)
(e) edge node{} (f);
\end{tikzpicture}
\right).
\]
From these we obtain \eqref{eq:K13c2} and \eqref{eq:K1c2c3}.

It is clear that $\KB_{\sytohct}$ kills $B_{sts}$ under tensor product on the right and left, since $\CB_{\yooo}$ does. Thus we have seen for all four tableaux that $\KB_T \ot B_{P,Q} \simeq 0$ for $P \ngeq T$, and $B_{P,Q} \ot \KB_T \simeq 0$ for $Q \ngeq T$.

