%%%%%%%%%%%%%%%%%%%%%
\section{The eigenmaps}
\label{sec:constructing}
%%%%%%%%%%%%%%%%%%%%%%


In this chapter we will prove Conjecture \ref{conj:eigenmap} on the existence of special maps $\a_\l:\Sigma_\l(\one)\rightarrow \FT$  in type $A$. % construct some very special elements $\a_\l\in \Homgg(\one,\FT_n)$, indexed by partitions $\l \in \PC(n)$.  It will take some time to develop the most important properties of these maps, but eventually we will prove that $\{\a_\l\}_{\l \in \PC(n)}$ form a saturated collection of eigenmaps for $\FT_n$. 


%============================
\subsection{Reminder on $\l$-equivalences, left versus right}
\label{subsec:lambdaEquiv}
%============================

Recall the result of Theorem \ref{thm:ftcomputation}, which states that
\begin{equation}\label{eq:FTwithtail}
\FT_n\ot B \simeq (\text{tail}\to \shift{\l}{B})
\end{equation}
for each $B\in \SBim_n$ in cell $\l$, where the tail consists of terms in cells $<\l$ and homological degrees $<2\cbb(\l)$.  In particular, the inclusion of the term in maximal homological degree is a chain map \[ \iota_B:\shift{\l}{B} \to \FT_n\ot B\] whose mapping cone is homotopy equivalent to the  tail. Our goal is to show that this inclusion map $\shift{\l}{B}\to \FT_n\ot B$ is induced from a map $\shift{\l}{\one}\to \FT_n$ after tensoring with $B$. Before proving this, we develop the theory of such maps.

We have already discussed such maps in \S\ref{subsec:celltri}, where we called them $\l$-equivalences. We restate the definition here, in order to be more precise about left versus right actions.  Recall that $\FT_n \ot B \simeq B \ot \FT_n$, so that there is also a corresponding inclusion map \[\iota'_B \co \shift{\l}{B} \to B \ot \FT_n.\]

\begin{definition}\label{def:lambdaMap} Choose a partition $\l \in \PC(n)$. A chain map $\a \co \shift{\l}{\one} \to \FT_n$ is said to be a \emph{left $\l$-equivalence} if $\Cone(\a_\l)
\ot B$ is homotopy equivalent to a complex in cells strictly less than $\l$, for each $B\in \SBim_n$ in cell $\l$. It is a \emph{right $\l$-equivalence} if $B \ot \Cone(\a_\l)$  is homotopy equivalent to a complex in cells strictly less than $\l$. \end{definition}

It is easy to see one can check whether $\a$ is a (left or right) $\l$-equivalence by checking its defining condition only when $B$ is indecomposable. Assuming that $\End(B)$ is a field for all indecomposable $B$ in cell $\l$ (which is true by Theorem \ref{thm:SC}), Lemma \ref{lem:lequivcriterion} proved that $\a$ is a left $\l$-equivalence if and only if $\a \ot \Id_B$ is a nonzero scalar multiple of $\iota_B$, if and only if $\a \ot \Id_B$ is not null-homotopic. Similarly, it is a right $\l$-equivalence if and only if $\Id_B \ot \a$ is a nonzero scalar multiple of $\iota'_B$.

\begin{ex}
Consider the one-row partition $\lambda=(n)$. The only indecomposable $B$ in cell $\l$ is $\one$. Then $\colsum(\l)=\binom{n}{2}$, $\rowsum(\l)=0$, and the inclusion of the maximal degree chain bimodule $\a \co \one[-n(n-1)](n(n-1)) \rightarrow \FT$ is a (left or right) $\l$-equivalence. This is because $\Cone(\a)$ is homotopy equivalent to a complex which has no summands isomorphic to shifted copies of $\one$, and hence it lies in lower cells.
\end{ex}

\begin{ex}
If $\lambda$ is the one-column partition, then $\colsum(\l) = 0$ and $\rowsum(\l)=-\binom{n}{2}$.  There is a quasi-isomorphism of complexes of $R$-bimodules $\a:q^{\binom{n}{2}}\one \to \FT$.  In \cite{AbHog17} it was proven that, for the one-column partition, a map is a (left or right) $\l$-equivalence if and only if it is a quasi-isomorphism (not necessarily a homotopy equivalence).  Note that $\SBim_{< \l} = 0$, since $\l$ is the minimal cell.

As a subexample, consider $n=2$. The cone of this quasi-isomorphism $\a$ is the four-term complex
\[
\Big( \one(-2) \to \un{B_s(-1)} \to B_s(1) \to \one(2) \Big).
\]
\end{ex}

We now prove that the notions of left and right $\l$-equivalence agree, so after this section we will simply write $\l$-equivalence.

\begin{lemma}\label{lemma:lequivsymmetry}
Let $B_{T,U}\in \SBim_n$ be in cell $\l$, and let $\a:\shift{\l}{\one}\to \FT_n$ be a chain map.  Then the following are equivalent:
\begin{enumerate}\setlength{\itemsep}{3pt}
\item $\Cone(\a)\ot B_{T,U}$ is in cells $<\l$ up to homotopy.
\item $B_{U,T}\ot \Cone(\a)$ is in cells $<\l$ up to homotopy.
\end{enumerate}
Consequently, $\a$ is a left $\l$-equivalence if and only if it is a right $\l$-equivalence.
\end{lemma}

\begin{proof}
Consider the tensor product
\begin{equation}\label{eq:lequivsym}
X = B_{U,T} \ot \Cone(\a) \ot B_{T,U}.
\end{equation}
If either (1) or (2) holds, then the complex $X$ must live in cells $< \l$ up to homotopy. Suppose that (2) holds but (1) fails. Then, by the previous lemma, we know that $\a \ot \Id_{B_{T,U}} \simeq 0$, and $\Cone(\a) \ot B_{T,U}\cong \shift{\l}{B_{T,U}}[1]\oplus (\FT_n\otimes B_{T,U})$.  Tensoring on the left with $B_{U,T}$, we see that the complex \eqref{eq:lequivsym} has a direct summand of the form $B_{U,T} \ot B_{T,U}$, which in turn has a direct summand of the form $B_{U,U}$ (up to shifts).  This contradicts the fact that $X$ is homotopy equivalent to a complex in cells $< \l$. A similar contradiction arises if we assume (1) holds but (2) fails. \end{proof}

\begin{remark} A similar proof will show that left and right $\l$-equivalence are equivalent, for any two-sided cell in a finite Coxeter group, assuming that $\FT$ is twist-like, sharp and increasing (which would follow from Conjecture \ref{conj:HTaction}. \end{remark}

%========================
\subsection{The Specht module and barbells}
\label{subsec:specht}
%========================

Now we need a slight detour, which will eventually lead to Corollary \ref{cor:sufficientcondition}, a simple criterion for the existence of a $\l$-equivalence.

Fix $n \ge 2$, and let $R = \RM[x_1, \ldots, x_n]$ with its action of $S_n$. All partitions $\l$ will have $n$ boxes. For a standard tableau $T$, recall the definition of the $T$-barbell
$\tbarb{T}$ from \S \ref{subsec:dotsA}.

\begin{defn} \label{def:barbellSpecht} Let $\SM_\l'$ be the $\RM$-span of polynomials $\tbarb{T}\in R$, as $T$ ranges over all $T\in \SYT(\l)$.\end{defn}

Meanwhile, here is a definition dating back to Specht \cite{Specht35}, see also \cite{Peel75}.

\begin{defn}\label{def:specht}
For $T$ any tableau of shape $\l$, with entries $\{1, 2, \ldots, n\}$ but not necessarily standard,  define $$g_T := \prod (x_i - x_j) \in R$$ where the product is over pairs of indices
$1\leq i, j\leq n$ with $i$ above $j$ in the same column of $T$. Let $\SM_\l$ denote the $\RM$-span of the $g_T$, as $T$ ranges over all tableau of shape $\l$. This is the \emph{Specht module}\footnote{The Specht module has various other constructions, but this is the original construction due to Specht.} of $\l$.
\end{defn}

Observe that $T$ is uniquely determined by $g_T$. Also note that $w(g_T) = g_{w(T)}$ for $w \in S_n$, where $w$ acts on $T$ by permuting the box labels. Hence $\SM_\l$ is naturally an $S_n$-representation. The following is a theorem of Specht.

\begin{thm} The $S_n$-representation $\SM_\l$ is irreducible, and it has a basis given by $g_T$ for standard Young tableaux $T \in \SYT(\l)$. \end{thm}

The main result of this section is the following.

\begin{thm}\label{thm:specht}
We have $\SM_\l'=\SM_\l$ as subspaces of $R$.
\end{thm}

The proof follows from a series of Lemmas.  First, we warn the reader that $\tbarb{T}\neq g_T$ in general, as the following example shows.

\begin{ex} \label{ex:fPnotgP} Let $\l = (2,2)$ and let $T$ be the unique standard tableau with $g_T=(x_1-x_3)(x_2-x_4)$. One has $w(T,T) = tsut \in S_4$, and one can compute that $\tbarb{T}$ is $(x_1-x_4)(x_3 - x_2)\neq g_T$. \end{ex}
	
However, there is one case where $\tbarb{T}$ and $g_T$ automatically agree.

\begin{lemma} When $T = P_{\col}$ is the column-reading tableau, one has $\tbarb{T} = g_T$. \end{lemma}
	
\begin{proof} As noted in \eqref{eq:barbellA}, $\tbarb{T}$ is the product of the positive roots in the parabolic subgroup $S_\l$. Clearly, so is $g_T$. \end{proof}

For each $\l$, let $I_{\leq\l}\subset R$ denote the homogeneous ideal generated by polynomials which factor through Soergel bimodules in cells $\le \l$.

\begin{lemma}\label{lemma:specht1}
The ideal $I_{\leq\l}\subset R$ is supported in degrees $\geq 2\rbb(\l)$.  The degree $2\rbb(\l)$ component equals $\SM_\l'$.
\end{lemma}

\begin{proof}
In Proposition \ref{prop:dotspace}, we proved that if $w\in S_n$ is in  cell $\l$, then $\Homg(\one,B_w)$ is supported in degrees $\geq \rbb(\l)$.  Further, the degree $\rbb(\l)$ component is one dimensional if $w$ is an involution and zero otherwise.   Consequently, any endomorphism of $\one$ which factors through an indecomposable $B_w$ in cell $\l$ will have degree at least $2\rbb(\l)$, with equality only if $w$ is an involution.  It follows that the degree $2\rbb(\l)$ component of $I_{\leq\l}$ is spanned by the $w$-barbells $\tbarb{w}$.

Moreover, if $\mu<\l$ then any endomorphism of $\one$ factoring through an indecomposable in cell $\mu$ will have degree at least $2\rbb(\mu)$, which is strictly larger than $2\rbb(\l)$. Thus the ideal $I_{\leq \l}$ is supported in degrees $\geq 2\rb(\l)$.  This completes the proof.
\end{proof}

\begin{lemma}\label{lemma:specht2} The ideal $I_{\le \l}$ is closed under the action of $W$. Consequently, $\SM_{\l}'$ is an $S_n$-representation. \end{lemma}
	
\begin{proof}  Let $\IC_{\leq \l}$ denote the ideal of morphisms in $\SBim_n$ factoring through bimodules in cells $\leq \l$.   The ideal $\IC_{\le \l}$ is closed under tensor products with arbitrary morphisms in $\SBim_n$.  Pick a simple reflection $s$. If $f\in R$ is in $I_{\le \l}$, then so is the morphism $c_s(f)$ given by
\[
R \to B_s \otimes B_s \buildrel m_f \over \longrightarrow B_s \otimes B_s \to R,
\]
where $m_f$ denotes middle multiplication by $f$, and the remaining maps are the units and counits of biadjunction for $B_s$.   Here, middle multiplication by $f$ is the endomorphism of $B\otimes_R B'$ sending
\[
b\otimes b'\mapsto bf\otimes b' = b\otimes fb',
\]
where $B$ and $B'$ are $(R,R)$ bimodules and $\otimes = \otimes_R$, as usual.

Diagrammatically, $c_s(f)$ corresponds to placing $f$ inside a circle colored $s$. A simple computation involving \cite[Equation 5.2]{EWsoergelCalc} will imply that $c_s(f)=f-s(f)$. So $f - s(f) \in I_{\le \l}$, implying that $s(f) \in I_{\le \l}$. This proves that $I_{\le \l}$ is closed under the action of $S_n$. \end{proof}

\begin{proof}[Proof of Theorem \ref{thm:specht}]
We know that $\SM_\l$ is defined as the span of the $S_n$-orbit of $g_{P_{\col}}$, that $g_{P_{\col}} = \tbarb{P_{\col}}$, and that $\SM_\l'$ is closed under the action of $S_n$. Hence $\SM_\l \subset \SM_\l'$. On the other hand, we know that the dimension of $\SM_\l'$ is at most the size of $\SYT(\l)$, which is the dimension of $\SM_\l$. We conclude that $\SM_\l'=\SM_\l$ by a dimension count.
\end{proof}

\begin{remark}
For any two-sided cell $\l$ in any Coxeter group $W$, one can define a vector space $\SM_\l'$ as the span of $\tbarb{d}$ over all distinguished involutions in $\l$. The proofs above work verbatim to show that $\SM_\l'$ is closed under the action of $W$.  As far as we are aware, these analogues $\SM_\l'$ of Specht modules have not appeared elsewhere in the literature! We conjecture that the $d$-barbells form a basis for $\SM_\l'$.
\end{remark}

To avoid primes we write $\SM_\l$ instead of $\SM_\l'$ below, but the crucial point is that it is spanned by barbells.


%============================
\subsection{A sufficient condition for existence of $\l$-equivalences}
\label{subsec:lambdaEquivsufficient}
%============================


For each partition $\l$ recall the grading shift functor $\Sigma_\l=[-2\cbb(\l)](2\xbb(\l))$.

\begin{lemma} If $\a\co\shift{\l}{\one}\to \FT_n$ is a chain map and $\tbarb{P} \cdot \a$ is not null-homotopic, then $\Id_B \ot \a$ and $\a \ot \Id_B$ are not null-homotopic, where $B = B_{P,P}$. \end{lemma}

\begin{proof}  The reader may wish to recall the trivialities of \S\ref{subsec:trivial}, in particular that the left and right actions of $R$ on $\Hom^{\Z\times \Z}(\one,C)$ coincide for all complexes $C$.   Then since $\tbarb{P}$ is the composition of maps
\[
\one\to B_{P,P}(\rbb(\l)) \to \one(2\rbb(\l)),
\]
it follows that $\tbarb{P} \cdot \a=\tbarb{P}\otimes \a = \a\otimes \tbarb{P}$ factors through $\Id_B \ot \a$ and through $\a \ot \Id_B$.   If $\Id_B\otimes \a$ or $\a\otimes \Id_B$ were null-homotopic, then so would be $\tbarb{P}\cdot \a$.\end{proof}
	
\begin{cor} Suppose that a chain map $\a\co\shift{\l}{\one}\to \FT_n$ satisfies the condition that $\tbarb{P} \cdot \a$ is not null-homotopic, for all $P \in \SYT(\l)$. Then $\a$ is a
$\l$-equivalence. \end{cor}

\begin{proof} This follows from the above lemma and Lemma \ref{lem:lequivcriterion}.\end{proof}

Recall the action of the braid group on $\Homgg(\one, \FT_n)$, given by conjugation, as discussed in \S\ref{subsec:conjugate}.  When we refer to the conjugate of a map $\a \co \one \to \FT_n$, we refer to the map $\psi_\b(\a)$ for some braid $\b$.

\begin{lemma}\label{lem:criterion} Fix $\l \in \PC(n)$ and a chain map $\a\co\shift{\l}{\one}\to \FT_n$. Suppose there is a nonzero polynomial $g \in \SM_\l$ such that $g \cdot \a$ is not null-homotopic. Then one can find a linear combination $\a'$ of conjugates of $\a$, such that $\tbarb{P} \cdot \a'$ is not null-homotopic for all $P \in \SYT(\l)$. \end{lemma}
	
\begin{proof} We first claim that for each nonzero polynomial $f \in \SM_\l$ there exists a linear combination $\gamma$ of conjugates of $\a$ such that $f\cdot \g\not\simeq 0$. Indeed, let
\[
X := \{f \in \SM_\l \:|\: f\cdot \psi_\b(\a) \simeq 0 \textrm{ for all } \b\in \Br_n\}.
\]
Clearly $X$ is a subspace of $\SM_\l$, and it is proper because $g \notin X$.  We wish to show that $X$ is preserved by $S_n$, in which case it must be $0$ by the irreducibility of $\SM_\l$. However, we have seen in \eqref{eq:conjugation1} that
\[
\psi_\b(f\cdot \g)\simeq w(f) \cdot \psi_\b(\g)
\]
for any polynomial $f$, map $\g$, and braid $\b$, where $w$ is the element of the symmetric group corresponding to $\b$. Thus, if $f \cdot \psi_\b^{-1} (\gamma) \simeq 0$ then $w(f) \cdot \g \simeq 0$, for a braid lifting $w$. In particular, if $f \in X$ then $w(f) \in X$. Thus we conclude that $X = 0$.

Now let $V\subset \Homgg(\one,\FT_n)$ be the vector space spanned by the classes of the conjugates $\psi_\b(\a')$, as $\b$ ranges over all elements in $\Br_n$. For each $P\in \SYT(\l)$, let
$V_P\subset V$ denote the subspace consisting of those homotopy classes $[\g]$ such that $\tbarb{P}\cdot \g \simeq 0$. If $V_P = V$ then $\tbarb{P} \in X$, a contradiction. Thus each $V_P$ is a proper subspace of $V$. Since there are only finitely many such $P$, the union of the $V_P$ over $P \in \SYT(\l)$ can not be all of $V$. Choosing any element $\a' \in V\setminus
\bigcup_{P\in \SYT(\l)}V_P$, we see that $\tbarb{P} \cdot \a' \not\simeq 0$ for all $P \in \SYT(\l)$. \end{proof}


\begin{cor} \label{cor:sufficientcondition} If there is a map $\a \co \shift{\l}{\one}\to \FT_n$ and a polynomial $g \in \SM_\l$ such that $g \cdot \a$ is not null-homotopic, then some linear combination $\a'$ of braid conjugates of $\a$ is a $\l$-equivalence.\qed \end{cor}

%=============================
\subsection{Addendum: the braid group action on the full twist}
\label{subsec:symgrp}
%=============================

During the unusually long preparation of this work, the second author and Eugene Gorsky posted a preprint \cite{GorHog17} which can be used to strengthen and simplify Corollary \ref{cor:sufficientcondition}. We felt it was worth including this result, even though it comes from a different geological time period than the rest of this paper.

Briefly stated, the result proves that the braid group action on $\Homgg(\one,\FT_n)$ factors through the symmetric group, and that the important part of any $\l$-equivalence lives in a copy of the sign representation. As a consequence, one need not take a linear combination $\a'$ of conjugates in the statement of Corollary \ref{cor:sufficientcondition}; the original map $\a$ will be a $\l$-equivalence. We believe this is worth noting for future work, although it does not make any appreciable difference to the arguments in this paper, so we continue to use Corollary \ref{cor:sufficientcondition} in the rest of the paper.

% Recall by Rouquier canononicity (\S \ref{subsec:RouqCanon}) that for each braid $\b\in \Br_n$ we have a canonical homotopy eqivalence $F(\b)\otimes \FT_n\otimes F(\b\inv)\simeq \FT_n$.  Thus each braid $\b$ determines an automorphism $\psi_\b:\Homgg(\one,\FT_n)\rightarrow \Homgg(\one,\FT_n)$ where for each $\a:\one(a)[b]\rightarrow \FT_n$, $\psi_\b(\a)$ is the composition
% \[
% \one(a)[b] \simeq F(\b)\otimes \one(a)[b]\otimes F(\b\inv) \rightarrow F(\b)\otimes \FT_n\otimes F(\b) \simeq \FT_n.
% \]
% These automorphisms determine an action of $\Br_n$ on $\Homgg(\one,\FT_n)$.  Note also that $\Homgg(\one,\FT_n)$ is a module over $R=\Endgg(\one)$.  The action of polynomials and braids together yields an action of $\Q[\Br_n]\ltimes \Q[x_1,\ldots,x_n]$ on $\Homgg(\one,\FT_n)$ (\S \ref{subsec:conjugate}).

\begin{definition}
Let $\Q[x_1,\ldots,x_n,y_1,\ldots,y_n]$ be a bigraded ring with $\deg(x_i) = (2,0)$ and $\deg(y_i)=(-2,2)$, and let $S_n$ act on $\Q[x_1,\ldots,x_n,y_1,\ldots,y_n]$ by permuting both sets of variables. Let $A\subset \Q[x_1,\ldots,x_n,y_1,\ldots,y_n]$ denote the subspace of polynomials which are anti-symmetric with respect to the $S_n$ action.  Let $I\subset \Q[x_1,\ldots,x_n,y_1,\ldots,y_n]$ denote\footnote{The ideal $I$ is denoted $J_n$ in \cite{GorHog17}.} the ideal generated by $A$.
\end{definition}

Recall again from \S\ref{subsec:conjugate} the conjugation action of the braid group on $\Homgg(\one, \FT_n)$. As proven in Lemma \ref{lem:howtoconj}, the action of
polynomials and braids together yields an action of $\Q[\Br_n]\ltimes \Q[x_1,\ldots,x_n]$ on $\Homgg(\one,\FT_n)$, where the action of the braid group on polynomials factors through the symmetric group. Meanwhile, $\Homgg(\FT_n,\one)$ is also a module over $\Q[\Br_n]\ltimes \Q[x_1,\ldots,x_n]$, and one can prove (say, by \cite{LibWil}) that $\Homgg(\FT_n,\one) \cong R$ as $R$-bimodules. So the action of $\Q[\Br_n] \ltimes \Q[x_1, \ldots, x_n]$ on $\Homgg(\FT_n,\one)$ is determined by the action of $\Br_n$ on the one-dimensional space corresponding to $1 \in R$. Let this one-dimensional representation of $\Br_n$ be temporarily denoted $V$. Being a one-dimensional representation, $V$ is invertible in the sense that there exists $V\inv$ such that $V\otimes_\Q V\inv \cong V\inv\otimes_\Q V\cong \Q$, the trivial representation of $\Q[\Br_n]$.

\begin{theorem}\label{thm:ftsymgrp}
The braid group action on $\Homgg(\one,\FT_n) \ot_\Q V\inv$ factors through the symmetric group, and
 \[
 \Homgg(\one,\FT_n) \ot_\Q V\inv \cong I / (y_1,\ldots,y_n)I
 \]
as bigraded $\Q[S_n]\ltimes \Q[x_1,\ldots,x_n]$-modules.
\end{theorem}

\begin{remark} It is expected that $V$ is the trivial representation, though this was not proven in \cite{GorHog17}. In our proofs below, it is irrelevant what $V$ is.   \end{remark}

%braiding morphisms $\FT_n\otimes F(\b)\simeq F(\b)\otimes \FT_n$

\begin{remark}
In \cite{GorHog17}, Theorem \ref{thm:ftsymgrp} is stated more generally for $\Homgg(\one,\FT^{\otimes k})$ for all $k\geq 0$, and in fact generalizes to the entire triply graded homology (see \S \ref{subsec:FTHHH}).  We will not need the more general statement.
\end{remark}


\begin{example}
In case $n=2$, $A$ is the $\Q[x_1,x_2,y_1,y_2]^{S_2}$-submodule of $\Q[x_1,x_2,y_1,y_2]$ generated by $\a_1 = x_1-x_2$ and $\a_2 = y_1-y_2$.  Then $I/(y_1,y_2)I$ is the $\Q[x_1,x_2]$-module generated by $\a_1,\a_2$ modulo $(x_1-x_2)\a_2 = 0$, since \[ (x_1 - x_2) \a_2 = (y_1 - y_2) \a_1 \in (y_1,y_2)I.\] These two generators have bidegree $(2,0)$ and $(-2,2)$, corresponding to maps $\one(-2)[0] \to \FT_n$ and $\one(2)[-2] \to \FT_n$.
\end{example}

\begin{cor}\label{cor:FTmodMaxl}
Let $\mg\subset \Q[x_1,\ldots,x_n]$ denote the maximal graded ideal (generated by $x_1,\ldots,x_n$).   Then modulo $\mg$ every $\a\in \Homgg(\one,\FT_n)$ spans (the zero subspace or) a copy of the sign representation. \qed
\end{cor}

\begin{lemma}\label{lemma:alphaNotInM}
Let $\b \in \mg \Homgg(\one,\FT_n)$ Then $\a\in \Homgg(\one,\FT_n)$ is a $\l$-equivalence if and only if $\a + \b$ is. In particular, any $\l$-equivalence is not in $\mg \Homgg(\one,\FT_n)$.  
\end{lemma}

\begin{proof}
This is true for degree reasons.   To be precise, recall that $\Sigma_\l(\one)$ is supported in homological degree $2\cbb(\l)$, hence the only nonzero component of $\a$ is $\a^{2\cbb(\l)}$.  In cell $\l$ in this homological degree, $\FT_n$ is a direct sum of $B_d(2\xbb(\l)+\rbb(\l))$, as $d$ ranges over the involutions in cell $\l$.  In fact each component of the composition
\[
\a^{2\cbb(\l)}: \one(2\xbb(\l)) \rightarrow \FT_n^{2\cbb(\l)}\twoheadrightarrow \bigoplus_d B_d(2\xbb(\l)+\rbb(\l))
\]
is a nonzero multiple of the $d$-dot map (\S \ref{subsec:dots}).  The $d$-dots are living in the smallest possible degree, hence any map with smaller bimodule degree must have zero components mapping into $B_d$.  Thus, any morphism $\b \in \mg \Homgg(\one,\FT_n)$ will factor through cells smaller than or incomparable to $\l$. Consequently, if $B$ is any indecomposable in cell $\l$, the coefficient of the identity of $\Sigma_\l(B)$ in $\b \ot \Id_B$ is zero, because this morphism lives in cells $< \l$. In particular, $\a$ and $\a + \b$ have the same coefficient for the identity of $\Sigma_\l(B)$.
\end{proof}

\begin{lemma}\label{lemma:primesnotnecessary}
Let $\a\in \Homgg(\one,\FT_n)$ have the correct degree to be a $\l$-equivalence.  If some element $\a'$ of $\Q[S_n]\cdot \a\subset \Homgg(\one,\FT_n)$ is a $\l$-equivalence, then $\a$ is already a $\l$-equivalence.
\end{lemma}

\begin{proof} By Lemma \ref{lemma:alphaNotInM}, $\a' \notin \mg\Homgg(\one,\FT_n)$, and so $\a\notin \mg\Homgg(\one,\FT_n)$ since $\mg\Homgg(\one,\FT_n)$ is preserved under the action of
$S_n$. Moreover, by Corollary \ref{cor:FTmodMaxl} $\a$ and $\a'$ are colinear (and nonzero) modulo $\mg\Homgg(\one,\FT_n)$. Any invertible scalar multiple of a $\l$-equivalence is a
$\l$-equivalence, so replacing $\a'$ with a scalar multiple, we can assume $\a$ and $\a'$ are equal modulo $\mg\Homgg(\one,\FT_n)$. But then by Lemma \ref{lemma:alphaNotInM}, $\a$ is a
$\l$-equivalence since $\a'$ is. \end{proof}

% Suppose there exists $\a'\in \Q[S_n]\cdot \a$ which is a $\l$-equivalence.  Then $\a'\not\in \mg\Homgg(\one,\FT_n)$ implies that $\a\not\in\mg\Homgg(\one,\FT_n)$.  The symmetric group action satisfies the following:
% \[
% w\cdot \a \in (-1)^{\ell(w)}\a + \mg\Homgg(\one,\FT_n).
% \]
% It follows that $\a' = \kappa\a + \gamma$ for some scalar $\kappa \in \Q$ and some $\gamma\in \mg\Homgg(\one,\FT_n)$.  The proof of Lemma \ref{lemma:alphaNotInM} implies that $\gamma$ factors through cells lower than and incomparable to $\l$, hence $\kappa$ must be nonzero and $\a = \kappa\inv \a  - \kappa\inv\gamma$ is a $\l$-equivalence (the property of being a $\l$-equivalence is preserved by adding maps factoring through cells $\ngeq \l$).
% \end{proof}

Thus we have our improvement upon Corollary \ref{cor:sufficientcondition}.

\begin{cor} \label{cor:sufficientcondition2} If there is a map $\a \co \shift{\l}{\one}\to \FT_n$ and a polynomial $g \in \SM_\l$ such that $g \cdot \a$ is not null-homotopic, then $\a$
is a $\l$-equivalence.\qed \end{cor}

%============================
\subsection{Homology of the full twist}
\label{subsec:FTHHH}
%============================

In previous work \cite{ElHog16a}, the authors studied the entire space of maps $\Homgg(\one,\FT_n)$, with the goal\footnote{Our original goal was to find the map $\a_\l$ explicitly.
Instead, we were only able to find $\a'$.} of finding a map $\a'$ to which Corollary \ref{cor:sufficientcondition} could be applied. Let us recall this work. In this chapter we use some of the notational conventions for convolutions and twisted complexes from \S 4 of \cite{ElHog17a}.

Recall that to each complex $C\in \KC^b(\SBim_n)$ we have a triply-graded vector space $\HHH(C)=\bigoplus_{i,j,k\in \Z}\HHH^{ijk}(C)$ where
\[
\HHH^{ijk}(C)=H^k(\Ext_{(R,R)}^j(\one,C(i)),
\]
and the $\Ext$ groups are taken in the category of graded $(R,R)$ bimodules. That is, one applies the Hochschild cohomology functor $\Ext_{R,R}(\one,-)$ to each homological degree (obtaining a complex of bigraded vector spaces), and then takes the cohomology of this complex (obtaining a triply graded vector space). Here, $k$ is the homological grading, $j$ is the Hochschild grading, and $i$ is the bimodule grading. We are most interested in the bigraded space
\[
\HHH^{\bullet,0,\bullet} =: \Homgg(\one,C),
\]
whose degree $(i,k)$ component consists of chain maps $\one\to C(i)[k]$ modulo homotopy.

For each braid $\b\in \Br_n$, let $\hat{\b}$ denote braid closure, i.e.~ the oriented link in $\R^3$ obtained by connecting the top and bottom boundary points of $\b$ in a planar fashion. 
\begin{remark} When $C=F(\b)$ is the Rouquier complex attached to a braid, Khovanov showed \cite{Kh07} that the triply-graded vector $\HHH(C)$ is isomorphic to the Khovanov-Rozansky homology of $\hat{\b}$ up to overall shift.  In particular, classes in $\Homgg(\one,\FT_n)$ correspond to classes in the Khovanov-Rozansky homology of the $(n,n)$ torus link with Hochschild degree zero.
\end{remark}


The entire triply graded homology $\HHH(\FT_n)$ was computed in the authors' earlier work \cite{ElHog16a}.  Let us recall how this computation is accomplished.  First, as a matter of notation, let us write $B\simeq (C \rightarrow A)$ if there exists a map $\d:C[-1]\rightarrow A$ such that $B\simeq \Cone(\d)$; equivalently there exists a distinguished triangle
\[
A\rightarrow B\rightarrow C\buildrel \d[1]\over \rightarrow A[1].
\]
If $B\simeq (C\rightarrow A)$ then we will also say that $B\simeq C\oplus A$ with \emph{twisted differential}.  More generally, if $I$ is a finite poset, we say that $B\simeq \bigoplus_{i\in I}A_i$ \emph{with twisted differential} there is differential $d$ on $\bigoplus_{i\in I} A_i$ such that
\begin{itemize}
\item the differential $d$ is lower triangular with respect to the partial order on $I$ in the sense that the component $d_{ij}$ from $A_j$ to $A_i$ is zero unless $i\geq j$.
\item the component $d_{ii}$ equals the given differential of $A_i$.
\item the resulting complex $(\bigoplus_i A_i, d)$ is homotopy equivalent to $B$.
\end{itemize}

Let $y_n\in \Br_n$ denote the \emph{Jucys-Murphy braids}, defined by $y_1=\sigma_1^2$ and $y_{n+1} = \sigma_n y_n\sigma_n$.  Let $Y_n$ denote the Rouquier complex associated to $y_n$. To streamline the results below, we abbreviate shift functors by writing $q=(-2)$ and $t=(2)[-2]$.

\begin{proposition}[\cite{Hog15}]\label{prop:KnProps}
There exist complexes $\KB_n\in \KC^b(\SBim_n)$ such that
\begin{subequations}
\begin{equation} \label{eq:K1is}
\KB_1 = (q \RM[x_1]\buildrel x_1\over \longrightarrow \underline{\RM[x_1]})
\end{equation}
\begin{equation}
\KB_n\ot B_w\simeq 0 \simeq B_w\ot \KB_n \ \ \ \ \ \ \text{ for each $w\neq 1\in S_n$}
\end{equation}
\begin{equation}\label{eq:homOneToK}
\Homgg(\one_n,\KB_n) \cong t^{\binom{n}{2}} \RM[x_1,\ldots,x_n] / (x_i-x_j)_{1\leq i<j\leq n}
\end{equation}
\begin{equation} \label{eq:Ktriangle}
(\KB_{n-1}\sqcup \one_1) Y_n \simeq \Big(\KB_{n} \longrightarrow  (\KB_{n-1}\sqcup \one_1) (-2)\Big).
\end{equation}
\begin{equation}\label{eq:RouquierAbsorbing}
\KB_n\ot F(\b)\simeq \KB_n(e)[-e]\simeq \KB_n\ot  F(\b) \ \ \ \ \ \ \ \  \text{ for all $\b\in \Br_n$}
\end{equation}
\end{subequations}
where $e=e(\b)$ is the \emph{braid exponent} or \emph{writhe} (that is, the signed number of crossings in a diagram for $\b$).
\end{proposition}



We make some observations. \begin{itemize} \item Equations \eqref{eq:K1is} and \eqref{eq:Ktriangle} can be used to give an inductive definition of $\KB_n$. \item The RHS of \eqref{eq:Ktriangle} is the cone of an explicit chain map, given in \cite{Hog15}. \item The complex $\KB_n$ categorifies a renormalized Young symmetrizer.  The ``symmetrizer'' property is captured by \eqref{eq:RouquierAbsorbing}. \item Equation \eqref{eq:homOneToK} could be stated more simply as $\Homgg(\one_n,\KB_n) \cong t^{\binom{n}{2}} \RM[x]$, but we have stated it as above to make the $R$-module structure clear. \end{itemize}

Using the complexes $\KB_k$ for $k \le n$, we can give a very useful expression for the Rouquier complex $\FT_n$.

\begin{definition}\label{def:shuffleAndDv}
Each sequence $v \in \{0, 1\}^n$ with $k$ zeroes determines a permutation $\pi_v\in S_n$---which we will call a \emph{shuffle}---that sends $1,\ldots,k$ to the zeroes of $v$ and $k+1,\ldots,n$ to the ones of $v$ in an order-preserving fashion.  Let $\b_v$ denote the positive braid lift of $\pi_v$ and $\omega(\b_v)$ the positive braid lift of $\pi_v\inv$.  Then set $D_v := F(\b_v)\ot(\KB_k\sqcup \FT_{n-k})\ot F(\omega(\b_v))$.
\end{definition}
For example, here is $D_{01001010}$, which occurs (up to shift) in our expression for $\FT_8$.
 \[
 D_{01001010}=
\begin{minipage}{1.1in}
\labellist
\small
\pinlabel $\FT_3$ at 80 40
\pinlabel $\KB_5$ at 30 40
\endlabellist
\begin{center}\includegraphics[scale=1]{fig/Dv}\end{center}
\end{minipage}
 \]
Inside $v$, the zeroes indicate which strands are connected to $\KB_k$, and the ones indicate which are connected to the full-twist $\FT_{n-k}$.  We warn the reader that the conventions for $D_v$ differ by a symmetry from those in \cite{ElHog16a}.  The main result of \cite{ElHog16a} uses the above properties of the complexes $\KB_n$ to prove the following results.

\begin{theorem}\label{thm:ftcomputation}
We have
\begin{equation}\label{eq:ftexpression}
\FT_n \simeq  \bigoplus_{v\in \{0,1\}^n} q^{2|v|}D_v \ \ \ \text{ with twisted differential},
\end{equation}
where $|v|$ is the number of ones in $v$.  The differential respects the the lexicographic order on sequences.\footnote{More precisely, for sequences $w = (0\cdots)$ and $v=(1\cdots)$, we write $w > v$. The differential sends terms associated to $v$ to terms associated to $w$ for $w \ge v$.}
\end{theorem}



\begin{theorem}\label{thm:parity}
For each $v\in \{0,1\}^n$ the bigraded space of homs $\Homgg(\one,D_v)$ is supported in even homological degrees.  The Poincar\'e polynomial of these spaces are given by an explicit recursion.  \qed
\end{theorem}



We do not need the explicit formulae here.  One very important consequence of the parity of these hom spaces is the following isomorphism.

\begin{corollary}\label{cor:FThoms}
There is an isomorphism
\begin{equation}\label{eq:homOneFT}
\Homgg(\one,\FT_n)\cong \bigoplus_v \Homgg(\one,D_v)(-2|v|)
\end{equation}
of bigraded vector spaces, where the sum is over $v\in\{0,1\}^n$ with $v_1=0$.
\end{corollary}

\begin{proof}
Let $\Homc^{\Z}(\one,\FT_n)$ denote the chain complex of bihomogeneous bimodule maps, whose homology is $\Homgg(\one,\FT_n)$.  Theorem \ref{thm:ftcomputation} gives a finite filtration on $\FT_n$, which gives a finite filtration on $\Homgg(\one,\FT_n)$.  The spectral sequence which computes the homology of this complex has $E_2$ page equal to $\bigoplus_v \Homgg(\one,D_v)(-2k)$.  The $E_2$ page is supported in even homological degrees by Theorem \ref{thm:parity}.  Thus, the differentials $d_r$ must be zero for $r\geq 3$, since $d_r$ increases homological degree by 1.  This shows that the spectral sequence satisfies $E_\infty = E_2$, which is as claimed.  Since we are working over a field, there is no issue with the extension problem in passing from $E_\infty$ to $\Homgg(\one,\FT_n)$.
\end{proof}

%\begin{remark}
%Note that $D_{1^n}=\FT_n$ itself appears as one of the summands on the right hand side of \eqref{eq:ftexpression}.  This does not present a problem for the recursive computation of $\Homgg(\one,\FT_n)$ because of the the degree shift $q^n$ appearing on the $D_{1^n}$ on the right hand side.  A more precise argument involves fixing a degree shift $(k)$ and computing the homology of $\Homc(\one,\FT_n)$ using the argument in the proof above, together with an induction on $k$.  Alternately, one may work with the ``reduced'' version $\FT_n\otimes(\KB_1\sqcup \one_{n-1})$ which satisfies
%\begin{equation}\label{eq:ftexpression}
%D_{1^{n-1},0}=\FT_n\otimes(\KB_1\sqcup \one_{n-1}) \simeq  \bigoplus_{v\in \{0,1\}^{n-1}} q^{2|v|}D_{0,v} \ \ \ \text{ with twisted differential},
%\end{equation}
%Now the maximal term on the right hand side is $D_{1,v}\cong \KB_1\sqcup \FT_{n-1}$ which involves the full twist on a smaller number of strands, hence is computed by induction.  To relate back to the homology of $Homc^{\Z}(\one,\FT_n)$ for the unreduced case, one uses
%\[
%Homc^{\Z}(\one,\FT_n) \cong \Homc^{\Z}(\one,FT_n\otimes(\KB_1\sqcup \one_{n-1}))\otimes_\Q \Q[x_1].
%\]
%For more details see \cite{}.
%\end{remark}


The isomorphism of Corollary \ref{cor:FThoms} does not respect the $R$-module structures.  Rather, there is a filtration on $\Homgg(\one,\FT_n)$, as a bigraded $R$-module, whose associated graded is isomorphic $\bigoplus_v q^{|v|} \Homgg(\one,D_v)$ as bigraded $R$-modules.  It is helpful to regard $\bigoplus_v q^{|v|}\Homgg(\one,D_v)$ as having two $R$-module structures: the one which is the direct sum of the actions on $\Homgg(\one,D_v)$ (and preserves this direct sum decomposition), and the one which is induced by a chosen isomorphism with $\Homgg(\one,\FT_n)$.  Given an element $\a\in \Homgg(\one,\FT_n)$, we denote these actions by $f\ot \a$ and $\gr(f)\ot \a$, since the latter is given by taking the associated graded of the filtered endomorphism $f\cdot (-)$ acting on $\Homgg(\one,\FT_n)$.

By definition two actions agree modulo higher terms, in the sense that if $\a\in \Homgg(\one,D_v)(-2|v|)$, then
\begin{equation} \label{eq:fvsgrf}
f\ot \a-  \gr(f)\ot \a\in \bigoplus_{w>v}\Homgg(\one,D_w)(-2|w|).
\end{equation}
In particular if $\gr(f)\ot \a$ is nonzero, then so is $f\ot \a$.


%===========================
\subsection{The eigenmap theorem}
\label{subsec:eigenmaptheorem}
%===========================

Let $\l=(\l_1,\ldots,\l_r)$ be given.  Let $n_1 = n - \l_1 = \l_2 + \cdots + \l_r$.  Note that the right-hand side of \eqref{eq:homOneFT} has a unique summand corresponding to the 01-sequence $(0^{\l_1}1^{n_1}) = (00\cdots 011\cdots 1)$, having the form
\[
\Homgg(\one, q^{n_1}\KB_{\l_1}\sqcup \FT_{n_1}) \cong q^{n_1}\Homgg(\one, \KB_{\l_1})\ot_\RM \Homgg(\one,\FT_{n_1}).
\]
In the second isomorphism we used the fact that for any Soergel bimodules $B_i,B_i'\in \SBim_{m_i}$, $i=1,2$, we have
\[
\Hom_{\SBim_{m_1+m_2}}(B_1\sqcup B_2, B_1'\sqcup B_2')\cong \Hom_{\SBim_{m_1}}(B_1,B_1')\otimes_\RM \Hom_{\SBim_{m_2}}(B_2,B_2').
\]
This is easily proven using the diagrammatic calculus in \cite{EKho}.


Iterating this for $\Homgg(\one,\FT_{n_1})$, we see that $\Homgg(\one,\FT_n)$ has a unique summand of the form
\[
 \Homgg(\one, q^{n_1} \KB_{\l_1}\sqcup \cdots \sqcup  q^{n_r}\KB_{\l_r})  \cong  q^{\rbb(\l)}\Homgg(\one,\KB_{\l_1})\ot_\RM\cdots \ot_\RM\Homgg(\one,\KB_{\l_r}),
 \]
where $n_i:=\l_{i+1}+\l_{i+2}+\cdots+\l_r$, and $n_r := 0$.  Thus $\Homgg(\one,\FT_n)$ has a unique summand of the form
\begin{equation}\label{eq:specialSummand}
\Homgg(\one, q^{n_1} \KB_{\l_1}\sqcup \cdots \sqcup  q^{n_r}\KB_{\l_r})% \ \ \cong \ \ q^{\rbb(\l)}t^{\cbb(\l)} \RM[x_1,\ldots,x_n]/(x_i-x_j\:|\: (i,j)\in Z_\l) 
\end{equation}


Now, for our given partition $\l\in\PC(n)$, let $P_{\row}$ be the row-reading tableau of shape $\l$, and let $Z_\l$ denote the set of pairs of indices $i<j$ which appear in the same row of $P_{\row}$.    Several applications of \eqref{eq:homOneToK} tell us that \eqref{eq:specialSummand} is isomorphic to
\begin{equation}\label{eq:specialSummand2}
 q^{\rbb(\l)}t^{\cbb(\l)} \RM[x_1,\ldots,x_n]/(x_i-x_j\:|\: (i,j)\in Z_\l),
\end{equation}
where we have used the fact that  $n_1+\cdots+n_r=\rbb(\l)$ and $\binom{\l_1}{2}+\cdots+\binom{\l_r}{2}=\cbb(\l)$.  This module has a distinguished generator, namely 1, which gives rise to a distinguished element of \eqref{eq:specialSummand}.

\begin{definition}\label{def:alphaprime}
Let $\a_\l \in \Homgg(\one,\FT_n)$ denote the element corresponding to the distinguished element of \eqref{eq:specialSummand}, with respect to the isomorphism \eqref{eq:homOneFT}.  Note that the degree of $\a_\l$ is $q^{\rbb(\l)} t^{\cbb(\l)}$.%, which equals $\Sigma_\l$.
\end{definition}

Now we make the following crucial observation.

\begin{lemma}\label{lemma:eigenmapNotKilled}   Recall the polynomial $g_{P_{\row}}$ from as in Definition \ref{def:specht}.  That is,
\[
g_{P_{\row}} :=\prod (x_i-x_j)
\]
where the product is over pairs $i<j$ such that $i,j$ are in the same column of $P_{\row}$.  Then $g_{P_{\row}} \cdot \a_\l$ is not null-homotopic. \end{lemma}

\begin{proof}
There is no pair $i < j$ where the numbers $i, j$ appear in both the same row and the same column of $P_{\row}$. Consequently, if $i$ and $j$ are in the same column, then $(i,j) \notin Z_\l$, and $(x_i - x_j)$ does not act by zero on the $R$-module in \eqref{eq:specialSummand2}.  In the notation of \eqref{eq:fvsgrf}, this says that $\gr(g_{P_{\row}}) \ot \a_\l$ is not homotopic to zero.  Consequently, by \eqref{eq:fvsgrf}, $g_{P_{\row}} \ot \a_\l$ is also not homotopic to zero.
%
% By construction, $\a_\l'$ is the image of $1$ under a composition of maps
% \begin{eqnarray*}
% q^{\rbb(\l)}t^{\cbb(\l)}\RM[x_1,\ldots,x_n]/(x_i-x_j\:|\: (i,j)\in Z_\l)  & \cong &  \Homgg(\one,q^{n_1}\KB_{\l_1}\sqcup\cdots \sqcup q^{n_r}\KB_{\l_r}) \\
% & \hookrightarrow & \bigoplus_v q^{|v|} \Homgg(\one, D_v)\\
% & \cong & \gr\Homgg(\one,\FT_n)\\
% & \cong & \Homgg(\one,\FT_n).
% \end{eqnarray*}
% The  first isomorphism is \eqref{eq:specialSummand}, the second is the inclusion of a direct summand, the third is Corollary \ref{cor:FThoms}, and the last isomorphism is an isomorphism of graded vector spaces, coming from the fact that a filtered vector space is always isomorphic to its associated graded.  Our argument above shows that the class of $g_{P_{\row}}\cdot \a_\l'$ is nonzero in $\gr\Homgg(\one,\FT_n)$.  It follows that this class is nonzero in $\Homgg(\one,\FT_n)$ (see Remark \ref{rmk:notModuleDecomp}).
\end{proof}

\begin{remark} Note that $[\gr(f) \ot \a_\l] \simeq 0$ for various polynomials $f \in \SM_\l$, including $\tbarb{P}$ for $P \in \SYT(\l)$. However, this does not imply that $[f \ot \a_\l] \simeq 0$, so it is not clear from this argument whether or not $\a_\l$ is a $\l$-equivalence.  In fact, it is a $\l$-equivalence, see \S\ref{subsec:symgrp}.\end{remark}


\begin{theorem}\label{thm:lambdaMaps}
For each partition $\l \in \PC(n)$ there exists a $\l$-equivalence $\a_\l: \shift{\l}{\one}\to \FT_n$. Furthermore, the collection of maps $\{\a_\l\}$ is obstruction-free in the sense of \cite{ElHog17a}. 
\end{theorem}

Forgiving the abuse of notation, the map $\a_\l$ of Theorem \ref{thm:lambdaMaps} may be assumed to be a linear combination of braid conjugates of the map $\a_\l$ constructed in
Definition \ref{def:alphaprime}.

\begin{proof}
Everything but the last statement follows from Lemma \ref{lemma:eigenmapNotKilled} and Corollary \ref{cor:sufficientcondition}.  The obstructions referred to in the statement are all degree $-1$ classes in $\Homgg(\Sigma_\l\circ \Sigma_\mu(\one), \FT_n^{\otimes 2})$.  Since $\Sigma_\l\circ \Sigma_\mu$ involves an even homological degree shift for each $\l,\mu\vdash n$, the obstructions correspond to classes in $\Homgg(\one,\FT_n^{\otimes 2})$ with odd homological degree.  However, $\Homgg(\one,\FT_n^{\otimes 2})$ is supported in even homological degrees by \cite{Hog17b}.
\end{proof}


\begin{corollary}
Let $\KB_\l:=\bigotimes_{\mu\neq \l}\Cone(\a_\mu)$. We have
\begin{subequations}
\begin{equation} \label{eq:conescommute}
\Cone(\a_\l)\otimes \Cone(\a_\mu)\simeq \Cone(\a_{\mu})\otimes \Cone(\a_\l).
\end{equation}
\begin{equation}\label{eq:conesquared}
\Cone(\a_\l)^{\otimes 2} \simeq (\shift{\l}{\one}[1] \oplus F) \otimes \Cone(\a_\l).
\end{equation}
\begin{equation} \label{eq:itsprediag} \bigotimes_{\l\in\PC(n)} \Cone(\a_\l) \simeq 0 \qquad \textrm{for all orderings of the factors}. \end{equation}
\begin{equation}\label{eq:quasiidemp}
\KB_\l^{\otimes 2} \simeq \left(\bigotimes_{\mu\neq \l}\l\oplus \mu[1]\right)\otimes  \KB_\l.
\end{equation}
\begin{equation}\label{eq:KisEigenobject}
\KB_\l\otimes \Cone(\a_\l)\simeq 0 \simeq \Cone(\a_\l)\otimes \KB_\l.
\end{equation}
\end{subequations}
In particular, $\FT_n$ is categorically prediagonalizable.
\end{corollary}

\begin{proof}
The commuting of cones \eqref{eq:conescommute} follows because the maps $\a_\l$ are obstruction-free, see \cite[\S 6.2]{ElHog17a}. Given this, \eqref{eq:itsprediag} follows from Proposition \ref{prop:lequivmeansdiag}.  Equation \eqref{eq:KisEigenobject} follows from \eqref{eq:itsprediag}. Equation \eqref{eq:conesquared} is proven in \cite[\S A.2]{ElHog17a}, and \eqref{eq:quasiidemp} is an immediate consequence.
\end{proof}
