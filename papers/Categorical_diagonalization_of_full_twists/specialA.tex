

%%%%%%%%%%%%%%%%%%%%%%%
\section{Cell theory in type $A$}
\label{sec:typeAcells}
%%%%%%%%%%%%%%%%%%%%%%%

In this chapter we explain how many of the previous constructions (cell theory, $\rbb$-function, distinguished involutions, Sch\"utzenberger involution) work in the special case of type
$A$. We go into more detail on examples to give more intuition to the novice reader. For a good introduction to this topic in type $A$, we recommend \cite{ArikiCells} or \cite{GeordiePQ}.

%============================
\subsection{Cells in type $A$}
\label{subsec:cellA}
%============================

Let $\PC(n)$ denote the set of partitions of $n$, and $\SYT(\l)$ denote the set of standard Young tableaux of shape $\l$, for $\l \in \PC(n)$. The following theorem is a major tool in
combinatorics; for a good reference we recommend \cite{FultonTab}.

\begin{thm}(Robinson-Schensted correspondence) There is an explicit bijection between elements of $S_n$ and triples $(P,Q,\l)$ where $\l \in \PC(n)$, and $P, Q \in \SYT(\l)$. The
algorithm which takes $w \in S_n$ and returns a triple $(P,Q,\l)$ is called Schensted's \emph{bumping algorithm}. \end{thm}

We will not discuss the bumping algorithm here, though we will state many consequences of it below. Since $\l$ is determined by $P$ and $Q$ we often just write $(P,Q)$ for the pair of
tableaux, with the understanding that $P$ and $Q$ have the same shape. We write $w(P,Q,\l)$ or $w(P,Q)$ for the corresponding element of $S_n$. For shorthand, we write the corresponding
indecomposable Soergel bimodule $B_{w(P,Q)}$ as $B_{P,Q}$, and the KL basis element as $b_{P,Q}$.

The following is a crucial theorem of Kazhdan-Lusztig\footnote{The reader will not find this theorem in this form inside \cite{KazLus79}, though the results of that paper imply it when one understands Knuth equivalence. One can find several expository resources aimed at making this theorem explicit from Kazhdan-Lusztig's work or proving it more simply, see \cite{ArikiCells} or \cite{GeordiePQ}.} \cite{KazLus79}, describing the cells in the Hecke algebra with respect to the KL basis of $\HB(S_n)$.  Since the KL basis corresponds to the indecomposable objects in $\SBim_n$ by Theorem \ref{thm:SC}, this also describes the cells in $\SBim_n$.

\begin{thm}\label{thm:soergelcells}
Let $w = w(P,Q,\l)$ and $w' = w(P',Q',\l')$. Then $w \sim_{LR} w'$ iff $\l = \l'$, $w \sim_L w'$ iff $Q=Q'$, and $w \sim_R w'$ iff $P=P'$. Thus, we associate the two-sided cells of $S_n$ with partitions $\l$, and the left (or right) cells with tableaux. Moreover, we have $\l \le_{LR} \mu$ if and only if $\l \le \mu$ in the dominance order.
\end{thm}

This theorem describes the left and right cells, and the two-sided cells, as well as the partial order on two-sided cells. The partial order on left cells is not as clear-cut, and even the following important result of Lusztig does not have an elementary proof (it follows from (P11) of Proposition \ref{prop:someP}).

\begin{prop} \label{prop:leftincomp} Two non-equal left cells in the same two-sided cell are incomparable. \end{prop}

\begin{remark} It is obvious that comparable left cells live in comparable two-sided cells, because left ideals are contained in two-sided ideals. Conversely, given left cells in
non-equal (but comparable) two-sided cells, it is not obvious when they are comparable in the left partial order. In particular, it is not the case that the ``dominance order on standard
tableaux'' gives the partial order on left cells. See \cite{Taskin} for a comparison of various partial orders on standard tableaux. \end{remark}

\begin{example}
Let $w\in S_n$ be in cell $\l\in \PC(n)$ and $v\in S_m$ be in cell $\mu\in \PC(m)$. With respect to the inclusion $S_n\times S_m\rightarrow S_{n+m}$, $(w,v)$ is in cell $\l+\mu$, which is the partition of $n+m$ given by $(\l+\mu)_i = \l_i+\mu_i$ (extending $\l$ or $\mu$ by zero if necessary).   This is easy to prove using the bumping algorithm.  The tableaux also add in a similar way, though we will not need this.
\end{example}

\begin{example}\label{ex:longesteltcell}
If $w_0\in S_n$ is the longest element, then $w_0$ is in cell $1^n$, the one-column paritition.  More generally, if $w$ is the longest element of a parabolic subgroup $S_{k_1}\times \cdots \times S_{k_r}\subset S_n$, then $w$ is in cell $\l$, where $\l\in\PC(n)$ is the partition whose column lengths coincide with the multiset $\{k_i\}$ (but in decreasing order).
\end{example}

An important special case occurs when the integers $k_i$ in the previous example are non-increasing.

\begin{ex} \label{ex:columnreading} Let $\l \in \PC(n)$, with column sizes $k_1 \ge k_2 \ge \ldots \ge k_r > 0$. Let $P_{\col}$ be the \emph{column-reading tableau}, obtained by placing
the numbers $1$ through $k_1$ in the first column, $k_1 + 1$ through $k_1 + k_2$ in the second column, and so forth. Then $w(P_{\col},P_{\col},\l)$ is the longest element $w_{\l}$ of the
parabolic subgroup $S_\l = S_{k_1} \times S_{k_2} \times \cdots \times S_{k_r} \subset S_n$. \end{ex}

Here is a key property of the Robinson-Schensted correspondence.
	
\begin{prop} \label{prop:involution} If $x = w(P,Q,\l)$, then $x\inv = w(Q,P,\l)$. In particular, $x$ is an involution if and only if $x = w(P,P,\l)$ for some $P \in \SYT(\l)$. \end{prop}


%========================
\subsection{Numerics of the $\rbb$-function in type $A$}
\label{subsec:longestnumerics}
%========================

The statements in this section are all well-known. We go into more detail than necessary, to aid the novice reader.

For any Coxeter group $W$, each left cell contains a unique distinguished involution by Proposition \ref{prop:moreP}. For the symmetric group, every left cell $(-,P,\l)$ contains exactly
one involution $w(P,P,\l)$. Thus in type $A$, $\DC$ is the set of all involutions. By Corollary \ref{cor:computinga}, to compute $\rbb(\l)$ we need only find an involution $d$ in cell $\l$
and compute the largest and smallest powers of $v$ appearing in the expansion of $b_d^2$. The easiest case to analyze will typically be when $d$ is the longest element of a parabolic
subgroup.

Now we recall a property of KL polynomials of longest elements.

\begin{defn} Let $W$ be a finite Coxeter group. Then $\pi(W)$ is its \emph{balanced Poincare polynomial}, which is $v^{-\ell(w_0)} \sum_{w \in W} v^{2 \ell(w)}$. \end{defn}
	
It is clear from the definition that the lowest degree of $v$ which appears in this polynomial is $v^{-\ell(w_0)}$, and the highest degree is $v^{+\ell(w_0)}$..

For example, when $W_I = S_{k_1} \times \cdots \times S_{k_r}$, $\pi(W_I)$ is a product of quantum factorials:
\begin{equation}
\pi(W_I) = [k_1]! [k_2]! \cdots [k_r]!.
\end{equation}

\begin{lemma} \label{lem:longestsquared} When $w_I$ is the longest element of a parabolic subgroup $W_I$, one has \begin{equation} \label{eq:longestsquared} b_{w_I} b_{w_I} = \pi(W_I)
b_{w_I}.\end{equation} More generally, if $x \in W$ and $sx < x$ for all $s \in I$, then \begin{equation} \label{eq:descentfactorial} b_{w_I} b_x = \pi(W_I) b_x. \end{equation}
\end{lemma}

\begin{proof} Let us sketch the proof of this well-known fact. One has 
\begin{equation} \label{eq:swq2}
b_s b_x = (v+v^{-1}) b_x
\end{equation}
whenever $sx < x$. This tells you how $H_s$ acts on $b_x$, and thereby how $H_w$ acts on $b_x$ for all $w \in W_I$. From the fact that $b_{w_I}$ is smooth, it is easy to deduce the lemma. \end{proof}

For the symmetric group $S_4$, every involution is the longest element of a parabolic subgroup except two: $x = tsut$ and $w = sutsu$. Computations for these elements were done in
Example \ref{ex:boundviolated}.

Now we introduce some statistics associated to partitions.

\begin{defn}\label{defn:xcr} We think of a partition $\l \in \PC(n)$ as a Young diagram, drawn in the ``English style'':
\[
\ig{.6}{youngDiagram}
\]
Suppose a box $\square$ is in the $i$-th column and $j$-th row.  Here columns and rows are counted left-to-right and top-to-bottom, starting at zero. We say that the box has \emph{column number} $\cbb(\square) = i$, \emph{row number} $\rbb(\square) = j$, and \emph{content} $\xbb(\square) = i-j$.

For a partition $\l$, we set $\cbb(\l) = \sum_{\square} \cbb(\square)$, $\rbb(\l) = \sum_\square \rbb(\square)$, and $\xbb(\l) = \sum_\square \xbb(\square)$. That is, the column number of a partition is the sum of the column numbers of each box, etcetera.  Equivalently, we have
\[
\rbb(\l) = \sum_i (i-1)\l_i, \qquad \qquad \cbb(\l) = \rbb(\l^t),\qquad\qquad \xbb(\l) = \cbb(\l)-\rbb(\l).
\]
\end{defn}

It is easy to observe that $\rbb(\l)$ is the length of $w_\l$, the longest element of the parabolic subgroup $S_\l$. It is also the length of $w_I$ for any conjugate parabolic subgroup $W_I$.

\begin{example} For the Young diagram $\l=(4,3,1)$ pictured above, one has $\cbb(\l) = 9$, $\rbb(\l) = 5$, and $\xbb(\l)=4$. The number $\rbb(\l)=5$ is the length of $s_1 s_2 s_1 s_4 s_6 \in S_8$.  \end{example}

\begin{example}
The two-row partition $\lambda=(n-1,1)$ has $\rbb(\lambda)=1$ and $\cbb(\lambda)=\binom{n-1}{2}$ for all $n$.
\end{example}

The following proposition follows immediately from the above, by examining $b_{w_\l}^2$.

\begin{proposition}\label{prop:abEqualsrb} For a 2-sided cell $\l \in \PC(n)$, the row number $\rbb(\l)$ agrees with Lusztig's $\rbb$-function for the two-sided cell $\l$.\qed
\end{proposition}

\begin{lemma} \label{lem:rowOrder}
If $\mu < \lambda$ in the dominance order then $\rbb(\mu) > \rbb(\l)$. In particular, if $\rbb(\l)=\rbb(\mu)$, then either $\mu=\l$ or $\mu$ and $\l$ are incomparable in the dominance order. \end{lemma}

\begin{proof} This is an easy exercise.  Alternately, once can use (P4) from Proposition \ref{prop:someP}. \end{proof}

\begin{example} The partitions $\l=(3,1,1,1)$ and $\mu=(2,2,2)$ satisfy $\rbb(\l)=\rbb(\mu)$, and are incomparable. They also satisfy $\cbb(\l) = \cbb(\mu)$ and $\xbb(\l) = \xbb(\mu)$. \end{example}

%========================
\subsection{Multiplication in a cell}
\label{subsec:morestill}
%========================

\begin{notation} \label{not:HBP-} For $\l \in \PC(n)$, let $\HB_\l \subset \HB$ denote the $\Z[v,v\inv]$-span of $b_{P,Q}$, for $P,Q \in \SYT(\l)$. We may refine this span by fixing $Q$
and taking $\HB_{-,Q}$, the $\Z[v,v\inv]$-span of $b_{P,Q}$ for $P \in \SYT(\l)$. We define $\HB_{P,-}$ similarly. Following this analogy, $\HB_{P,Q}$ is just the $\Z[v,v\inv]$-span of
the element $b_{P,Q}$.
	
We define $\HB_\l^+ \subset \HB_\l$ as the $\Z[v]$-span of $v b_{P,Q}$, i.e. as those elements of $\HB_\l$ with strictly positive powers of $v$. For fixed $P \in \SYT(\l)$, we define
$\HB_{P,-}^+$ as $\HB_{P,-} \cap \HB_\l^+$. We define $\HB_{-,Q}^+$ similarly. \end{notation}

Let $P,Q,U,V \in \SYT(\l)$. The Definition \ref{def:a} of the $\rbb$-function gives the following equality:
\[ v^{\rbb(\l)} b_{P,Q} b_{U,V} \equiv \sum_{X,Y \in \SYT(\l)} t_{(P,Q),(U,V)}^{(X,Y)} b_{X,Y} + \HB^+_\l + \HB_{< \l}.\]
Most of the time, this coefficient $t$ is zero. For example, since $\HB_{-,V} + \HB_{< \l}$ forms a left ideal, we must have $Y = V$ to have a nonzero contribution to the sum. By similar arguments, $X = P$. Stronger still, property (P8) from Proposition \ref{prop:someP} implies that $t_{(P,Q),(U,V)}^{(X,Y)} = 0$ unless $X=P$, $Y=V$, and $Q=U$.

The last part of Proposition \ref{prop:moreP} implies that $t_{(P,Q),(Q,P)}^{(P,P)} = 1$, which treats the case when $P = V$. In fact, as we will discuss below,
$t_{(P,Q),(Q,V)}^{(P,V)}=1$ for any $P,Q,V \in \SYT(\l)$. Hence, we have the following result.

\begin{lemma} \label{lem:howPQmult} For any $P, Q, U, V \in \SYT(\l)$, we have
\begin{equation} \label{eq:howPQmult} v^{\rbb(\l)} b_{P,Q} b_{U,V} \equiv \delta_{Q,U} b_{P,V} + \HB^+_\l + \HB_{< \l}.\end{equation} \end{lemma}

The Hecke algebra in type $A$ is a \emph{cellular algebra}\footnote{It is unfortunate that the word ``cell'' was appropriated by the literature for two similar but logically independent concepts: the cells described previously, and the cells in the cellularity theory of Graham and Lehrer. Thankfully, in this case, the Graham-Lehrer cells agree with the two-sided cells.} as defined by Graham and Lehrer \cite{GraLeh}. A major implication of this is that
\begin{equation} \label{eq:howPQmultcloser} b_{P,Q} b_{U,V} = \phi(Q,U) b_{P,V} + \HB_{< \l}. \end{equation}
Here, $\phi(Q,U)$, often called the \emph{cellular form}, is a function which takes $Q, U \in \SYT(\l)$ and returns a coefficient in $\Z[v,v\inv]$. In particular, this coefficient of $b_{P,V}$ does not depend on $P$ and $V$, only on $Q$ and $U$. This reduces the computation of $t_{(P,Q),(Q,V)}^{(P,V)}$ to the special case where $P = V$. For more on the cellular structure in type $A$, see \cite{GeordiePQ}.

Let $P,Q \in \SYT(\l)$, and let $a(P,Q)$ denote the most negative power of $v$ appearing with nonzero coefficient in $\phi(P,Q)$. The results above can be restated as follows: $a(P,Q)
\ge -\rbb(\l)$, with equality if and only if $P = Q$. Moreover, if $P=Q$, then the coefficient of $v^{-\rbb(\l)}$ is one.

\begin{remark} \label{rmk:notametric} A warning for the reader. The numbers $a(P,Q)$ for $P \ne Q$ are quite mysterious. A related number is $\D(P,Q)$, the minimal power of $v$ appearing
in $h_{1,w(P,Q)}$. It is tempting to view these as some measure of the distance between two standard tableaux, but this must be taken with a grain of salt. Set $d(P,Q) = \D(P,Q) -
\rbb(\l)$. It is true that $d(P,Q)$ takes positive values and is zero precisely when $P = Q$, but it fails\footnote{We were unable to find this result in the literature. Counterexamples
for $S_6$ were computed by Benjamin Young in response to my query.} to satisfy the triangle inequality, so it is not a metric. Note that $\D(P,Q)$ controls the ``asympotic'' distribution
of KL basis elements appearing in the half twist (e.g. in which cohomological degree a given bimodule appears for the first time in the half twist complex). The distribution of
indecomposables in the full twist is equally mysterious. \end{remark}


% \begin{remark} One can also provide a sophisticated proof of \eqref{eq:longestsquared} using the categorification: it suffices to show that $B_{w_I} \ot B_{w_I} \cong B_{w_I}^{\oplus
% \pi(W_I)}$, where the polynomial direct sum indicates a direct sum of grading shifts of copies of $B_{w_I}$. But $B_{w_I} \cong R \ot_{R^I} R (\ell(w_I))$, so that $B_{w_I} \ot B_{w_I}
% \cong R \ot_{R^I} R \ot_{R^I} R (2 \ell(w_I))$. Thus, the lemma is equivalent to the fact that $R(\ell(w_I))$ is free over $R^I$ with graded rank $\pi(W_I)$, an immediate consequence of
% the Chevalley theorem.  This proof adapts to a proof of \eqref{eq:descentfactorial} once one knows that $B_w$ is induced from some $(R^I,R)$-bimodule. \end{remark}

%The consequence of this lemma is that the cellular form is known explicitly for those tableaux which give rise to longest elements.

%\begin{cor}  Let $P \in \PYT(\l)$ be such that $w(P,P,\l) = w_I$ for some parabolic subgroup $W_I$. Then $\phi(P,P) = \pi(W_I)$. In particular, the most negative power of $v$ appearing with nonzero coefficient in $\phi(P,P)$ is $v^{- \rbb(\l)}$. \end{cor}

%This negative-most power of $v$ is the behavior which we seek to examine in detail.  So, we interpret Lemma \ref{lem:longestsquared} as a description of the negative-most power of $v$ appearing in the KL basis expansion of $b_{w_I} b_x$ when $sx < x$ for all $s \in I$.

%\begin{remark} \label{rmk:converselongest} A ``converse'' to Lemma \ref{lem:longestsquared} is also true: when $x \in W$ and $sx > x$ for some $s \in I$, then all the powers of $v$ appearing in the KL basis expansion of $b_{w_I} b_x$ will be strictly greater than $v^{-\ell(w_I)}$. The easiest proof is to bound below the power of $v$ which appears in the expansion in the standard basis. Because this converse will be subsumed in Lusztig's results below, we do not bother to sketch the proof in any further detail. \end{remark}




%========================
\subsection{Sch\"utzenberger in type $A$}
\label{subsec:more}
%========================

\begin{prop} \label{prop:Schutz}  There is an involution $P\mapsto P^\vee$ on the set of SYT, called the \emph{Sch\"utzenberger involution}, such that if $x = w(P,Q,\l)$ then
\begin{equation}
w_0 x = w((P^\vee)^t,Q^t,\l^t),
\end{equation}
\begin{equation}
x w_0 = w(P^t,(Q^\vee)^t,\l^t).
\end{equation}
The $w_0$-twisted involutions are therefore precisely the elements in $S_n$ of the form $w(P^\vee,P)$. \end{prop}

We let $\tau$ denote the Dynkin diagram automorphism of $S_n$, which can also be realized as conjugation by the longest element. The above implies that if $x = w(P,Q,\l)$, then $\tau(x) = w(P^\vee,Q^\vee,\l)$.

\begin{ex} In $S_4$, with simple reflections $\{s,t,u\}$, the $w_0$-twisted involutions are: $w_0$ in the longest cell; $tstut$, $tutst$ and $sutsu$ in the subminimal cell; $su$ and
$tsut$ in the middle cell; $stu$, $t$, and $uts$ in the simple cell; and the identity in the identity cell. Comparing this with \eqref{eq:Fstsuts}, we see that the $w_0$-twisted
involutions in cell $\l$ are precisely the terms in cell $\l$ which appear in homological degree $\cbb(\l)$ in the half twist. This was proven in general in Proposition \ref{prop:htm}.
\end{ex}

The Sch\"utzenberger involution has an explicit combinatorial description in terms of jeu-de-taquin.  This explicit construction is not yet illuminating to the authors, so we will not recall it, and the reader is welcome to use Proposition \ref{prop:Schutz} as a definition of Sch\"utzenberger duality.

To relate this construction with the operator $\Schu_L$ defined in \S\ref{subsec:schutz}, we have 
\begin{equation} \Schu_L(w(P,Q)) = w(P^\vee,Q). \end{equation} Consequently, \eqref{eq:schugeneral} becomes
\begin{equation} \label{eq:longesttimescell} H_{w_0} b_{(P,Q,\l)} \equiv (-1)^{\cbb(\l)} v^{\xbb(\l)} b_{(P^\vee, Q, \l)} + \HB_{< \l}. \end{equation}

\begin{ex} In $S_3$, we have already seen the categorification of \eqref{eq:longesttimescell} in \S\ref{subsec:HT3}. \end{ex}
	


%========================
\subsection{Dots in type $A$}
\label{subsec:dotsA}
%========================
Recall that every involution in $S_n$ is distinguished.

\begin{defn} Let $\l \in \PC(n)$ and $T \in \SYT(\l)$.  If $w=w(T,T,\l)$ is an involution, then let $\xi_T = \xi_{w}:R\rightarrow B_w(\rbb(\l))$ and $\tbarb{T} = \tbarb{w}= \xi_T^\ast \circ \xi_T : R\rightarrow R(2\rbb(\l))$ be as in Definition \ref{def:dots}. %for the distinguished involution
$w(T,T)$. \end{defn}

For those $T$ such that $w(T,T,\l)$ is the longest element of a parabolic subgroup, these ``thick dots'' and ``thick barbells'' were already studied in type $A$ in \cite{EThick}.

For the longest element $w_0 \in S_n$, letting $R = \RM[x_1, \ldots, x_n]$ and $\ell = \ell(w_0)$, the map $\xi_{w_0}$ is the map $R \to B_{w_0}(\ell) = R \ot_{R^{S_n}} R(2\ell)$ sending
\begin{equation}
1\mapsto \prod_{1 \le i<j \le n}(x_i\otimes 1 - 1\otimes x_j).
\end{equation}
The dual map $\xi_{w_0}^\ast \co B_{w_0}\rightarrow R(\ell)$ sends $1\otimes 1\mapsto 1$.  Their composition is the polynomial
\begin{equation} \label{eq:barbellA}
\tbarb{w_0}=\prod_{1 \le i<j \le n}(x_i -  x_j) \in R,
\end{equation}
which is the product of the positive roots.












%%%%%%%%%%%%%%%%%%%%%%%
\section{Twists in type $A$}
\label{sec:typeAtwists}
%%%%%%%%%%%%%%%%%%%%%%%

The eventual goal of this chapter is to prove Conjecture \ref{conj:HTaction} in type $A$.

%========================
\subsection{The braid group in type $A$}
\label{subsec:braid}
%========================

The braid group associated to $S_n$ is the usual braid group on $n$ strands. It has invertible generators $\{\s_i\}_{i=1}^{n-1}$ called \emph{(positive) crossings} corresponding to the
simple reflections $s_i \in S$. We may write $\s$ or $\s_s$ to denote one of these generators, the one corresponding to the simple reflection $s$. These generators satisfy the braid
relations, but do not satisfy $\s^2 = 1$.

% If we need to distinguish between an element of the braid group and a word in the braid generators (as we distinguish between permutations and their expressions), we refer to the former as a braid, and the latter as a \emph{braid word}.

Given an element $w \in S_n$, the \emph{positive lift} to the braid group is obtained by taking $\s_{i_1} \s_{i_2} \cdots \s_{i_d}$, whenever $\un{w} = (s_{i_1}, \ldots, s_{i_d})$ is a rex for $w$. This lift is independent of the choice of rex.

Let $\hT_n$ denote the positive lift of the longest element $w_0$ in $S_n$. Let $\fT_n$ denote $\hT_n^2$. These are called the \emph{half twist} and \emph{full twist} respectively. It is known that $\fT_n$ generates the center of $\Br_n$.

Specifically, the half twist is the braid $\hT_n=\s_1(\s_2\s_1)\cdots (\s_{n-1}\cdots\s_2\s_1)$ and the full twist is $\fT_n=\hT_n^2$.  Graphically, these may be pictured as follows:
\[
\hT_4 = \ig{1}{ht4.pdf},\qquad\qquad \fT_4 = \ig{1}{ft4.pdf}.
\]

The braid group admits a homomorphism to the symmetric group sending $\s_s$ to $s$, and admits a homomorphism to the (unit group of the) Hecke algebra sending $\s_s$ to $H_s$.


%========================
\subsection{The external product}
\label{subsec:external}
%========================

We expect the reader to be familiar with string diagrams for permutations (analogous to braid diagrams for braids). We will consistently use $\sqcup$ for horizontal concatenation of
string and braid diagrams in this paper. Thus we will also use $\sqcup$ to denote the \emph{external product} $S_i \times S_j \to S_{i+j}$, which sends simple reflections to simple
reflections in the obvious way.

The external product on symmetric groups induces an external product of Hecke algebras. Letting $\HB_i$ denote $\HB(S_i)$, we have $\sqcup \co \HB_i \times \HB_j \to \HB_{i+j}$. The
following proposition is a straightforward consequence of the definitions.

\begin{proposition}
We have $H_{w\sqcup x} = H_w \sqcup H_x$ and $b_{w\sqcup x} = b_w\sqcup b_x$ for all $w\in S_i$ and all $x\in S_j$.
\end{proposition}


Similarly, we may let $R_i = \RM[x_1,\ldots,x_i]$. We have the familiar isomorphism $R_i\otimes_\RM R_j \cong R_{i+j}$. Thus, tensoring over $\RM$ gives rise to a bilinear functor
$\SBim_i \times \SBim_j \rightarrow \SBim_{i+j}$, which we continue to denote by $\sqcup$, and call the \emph{external product}. In the literature this functor is often denoted by
$\boxtimes$. We use the notation $\one_i$ for the monoidal identity inside $\SBim_i$. If $A$ is an object of $\SBim_i$, we may write $A$ or $(A \sqcup \one_j)$ for the image of $A$ inside $\SBim_{i+j}$.

The external product is monoidal: if $A,A'\in \SBim_i$ and $B,B'\in \SBim_j$ then $(A\sqcup B)\otimes (A'\sqcup B')\cong (A\otimes A')\sqcup (B\otimes B')$. Proving this is an easy
exercise, and holds more generally for the external tensor product of any algebras over a field. An implication is that if $A \in \SBim_i$ and $B \in \SBim_j$, then $A \ot B \cong B \ot
A$.

\begin{remark} When working with Soergel bimodules in finite characteristic, it is better to use the diagrammatically defined category from \cite{EWsoergelCalc}, than the actual category of bimodules. Nonetheless, all the properties of the external product that we use can be proven using easy diagrammatic arguments. \end{remark}

\begin{remark} While $\sqcup$ represents horizontal concatenation of string diagrams for permutations, it is not to be confused with horizontal concatenation of Soergel diagrams in the monoidal category $\SBim$. These operations are taken at different ``categorical levels.''\end{remark}


%========================
\subsection{Bimodule sliding}
\label{subsec:sliding}
%========================

Throughout out the remainder of this chapter we will be illustrating certain statements diagrammatically.  Each of the diagrams is meant to indicate a complex of Soergel bimodules.  For instance, the Rouquier complex associated to a braid $\b$ will simply be drawn using usual pictures for braids.  If $w_0\in S_n$ is the longest element, then $B_{w_0}$ will be drawn by $n$-strands merging into one, and then splitting back into $n$.  For example $B_{w_0}\in \SBim_4$ would be pictured as
\[
B_{w_0} = \eqig{.9}{longestEltWeb}.
\]
The external tensor product $\sqcup$ is drawn by placing diagrams side-by-side, and $\otimes$ is indicated by vertical concatenation.

\begin{lemma} \label{lem:BFcommute} $F(\s_i) \ot B_j \cong B_j \ot F(\s_i)$ whenever $|i-j|>1$. \end{lemma}
This identity is pictured schematically below (in case $i<j$):\vskip4pt
\begin{equation} \label{eq:BFcommute}
\eqig{.9}{slide1} \ \ \ \simeq \ \ \ \eqig{.9}{slide2}.
\end{equation}\vskip4pt

\begin{proof} This follows from the fact that $\sqcup$ is a monoidal functor $\SBim_m\times \SBim_n\rightarrow \SBim_{m+n}$, and hence induces a monoidal functor
$\KC^b(\SBim_m) \times \KC^b(\SBim_n) \to \KC^b(\SBim_{m+n})$. \end{proof}

\begin{lemma}\label{lem:slidingBim}
$F(\sigma_i\sigma_{i+1})\otimes B_i \simeq   B_{i+1}\otimes F(\sigma_i\sigma_{i+1})$.
\end{lemma}
This is pictured schematically below:
\[
\eqig{.8}{slide3} \ \ \ \simeq \ \ \ \eqig{.8}{slide4}.
\]

\begin{proof}
Let $s = s_i$ and $t = s_{i+1}$. A straightforward computation shows that
\[
F(\sigma_s\sigma_t)\otimes B_s \simeq \Big(\underline{B_{sts}}\rightarrow B_{ts}(1)\Big) \simeq B_t\otimes F(\sigma_s\sigma_t).
\]
The differential is the composition $B_{sts}\subset B_sB_tB_s\rightarrow B_s B_t = B_{st}$, where the dot map is applied to the final tensor factor $B_s$.
\end{proof}

\begin{lemma}\label{lem:comparingfunctors}
Let $\FC_1,\FC_2$ be additive graded functors from $\SBim_n$ to $\AC$ for some additive graded category $\AC$ which has unique direct sum decompositions.  If $\FC_1(B)\cong \FC_2(B)$ for every Bott-Samelson bimodule $B\in \SBim_n$, then $\FC_1(B)\cong \FC_2(B)$ for every $B\in \SBim_n$.
\end{lemma}

\begin{proof} It is enough to prove the statement for $B = B_w$ indecomposable, which we prove by induction on the Bruhat order. For $\ell(w) \le 1$, $B_w$ is a Bott-Samelson, so there
is nothing to prove. So fix $w$ with length $\ge 2$, and assume that $\FC_1(B_x) \cong \FC_2(B_x)$ for all $x < w$. Choose a reduced expression $\un{w}$ for $w$. By Theorem
\ref{thm:SCT}, $\BS(\un{w})$ has a unique summand of the form $B_w$, and the remaining summands have the form $B_x(k)$ for $x < w$ and $k \in \Z$. Since $\FC_1(\BS(\un{w})) \cong
\FC_2(\BS(\un{w}))$ and $\FC_1(B_x(k)) \cong \FC_2(B_x(k))$ for all $x < w$, one can cancel summands and deduce that $\FC_1(B_w) \cong \FC_2(B_w)$. \end{proof}

We will apply this lemma shortly to functors from $\SBim_n$ to $\KC^b(\SBim_n)$. We note for this purpose that $\KC^b(\SBim_n)$ is Krull-Schmidt (as is the bounded homotopy category of any Krull-Schmidt additive category).

\begin{remark}
We warn the reader that there is no statement of naturality in Lemma \ref{lem:comparingfunctors}, so that $\FC_1,\FC_2$ may be non-isomorphic as functors, even if $\FC_1(B)\cong \FC_2(B)$ for all objects $B\in \SBim_n$.
\end{remark}

\begin{proposition}\label{prop:slidingBimodules}
For any $B\in \SBim_k$, we have the following isomorphism in $\KC^b(\SBim_{k+1})$:
\begin{equation}\label{eq:slidingBimodules}
F(\sigma_1\cdots \sigma_k) \otimes (B\sqcup \one_1)\simeq (\one_1\sqcup B)\otimes F(\sigma_1\cdots \sigma_k).
\end{equation}
\end{proposition}

This is pictured schematically below:
\[
\begin{minipage}{1.3in}
\labellist
\small
\pinlabel $B$ at 34 19
\endlabellist
\begin{center}\includegraphics[scale=.8]{fig/slide5}\end{center}
\end{minipage}
=
\begin{minipage}{1.3in}
\labellist
\small
\pinlabel $B$ at 58 59
\endlabellist
\begin{center}\includegraphics[scale=.8]{fig/slide6}\end{center}
\end{minipage}
\]

\begin{proof} For each $B\in\SBim_k$, let $\FC_1(B)$ and $\FC_2(B)$ denote the left and right-hand sides of \eqref{eq:slidingBimodules}, respectively. Using Lemma
\ref{lem:comparingfunctors} it suffices to show that $\FC_1(B) \cong \FC_2(B)$ when $B$ is a Bott-Samelson bimodule. The result for Bott-Samelson bimodules follows quickly from the
result for $B_s$. This in turn follows from repeated use of Lemmas \ref{lem:BFcommute} and \ref{lem:slidingBim}. \end{proof}
	
\begin{remark} In fact, the isomorphism $\FC_1(B) \cong \FC_2(B)$ is functorial, for all morphisms in $\SBim_k$. It suffices to show this for the generating morphisms of the diagrammatic
calculus: the dots, trivalent vertices, and $4$- and $6$-valent vertices. Because we do not need this result, we omit the tedious calculation. There are simpler ways to prove this result
as well, see Remark \ref{rmk:provingcentral} below. \end{remark}

%========================
\subsection{Conjugation by twists}
\label{subsec:commutehalffull}
%========================

\begin{defn} As in \S\ref{subsec:Hw0}, we let $\tau$ denote the involutory automorphism of the symmetric group $S_n$ coming from the Dynkin diagram automorphism, defined by
$\tau(s_i)=s_{n-i}$. Equivalently $\tau(x) = w_0xw_0$ for all $x\in S_n$. We let $\tau$ act on $R$ by $\tau(x_i) = x_{n+1-i}$. We also let $\tau$ denote the
corresponding automorphism of $\SBim_n$ and of $\KC^b(\SBim_n)$. This automorphism sends $B_{s_i}$ to $B_{s_{n-i}}$, and acts on morphisms by the \emph{color swap}: given a diagram (a
decorated colored graph), one replaces the color $s_i$ with the color $s_{n-i}$. \end{defn}

Note that $\tau(F(\s_i)) = F(\s_{n-i}) = F(\tau(\s_i))$. Also note that $\hT_n \s_i = \tau(\s_i) \hT_n$ in the braid group. Thus $\hT_n \b = \tau(\b) \hT_n$ for any braid $\b$.
Consequently, Rouquier canonicity gives an isomorphism \begin{equation} \HT_n \ot F(\b) \simeq \tau(F(\b)) \ot \HT_n\end{equation} for any braid $\b$. Similarly, one has an isomorphism
\begin{equation} \FT_n \ot F(\b) \simeq F(\b) \ot \FT_n, \end{equation} as noted in \S\ref{subsec:RouqCanon}.
	
Now we prove the same result, replacing $F(\b)$ by a complex supported in a single degree.

\begin{corollary}\label{cor:HTcommutes}
We have
\begin{equation}\label{eq:HTconjugation}
\HT_n\otimes B\simeq \tau(B)\otimes \HT_n
\end{equation}
for all $B\in \SBim_n$.
\end{corollary}

\begin{proof} By Lemma \ref{lem:comparingfunctors} it suffices to prove \eqref{eq:HTconjugation} for the Bott-Samelson bimodules, and again, just for $B_s$. This follows easily from
Lemma \ref{lem:BabsorbsF} and Proposition \ref{prop:slidingBimodules}. \end{proof}

\begin{corollary}\label{cor:FTcommutes}
We have $\FT_n\otimes B\simeq B\otimes \FT_n$ for all $B\in \SBim_n$.\qed
\end{corollary}

A consequence of these results is that conjugation by $\HT_n$ and $\FT_n$ are both endofunctors of $\SBim_n$; they preserve complexes which are supported in a single degree.  In the language of \S \ref{subsec:twists}, $\HT_n$ and $\FT_n$ are twist-like.

%========================
\subsection{Conjugation by twists, functorially}
\label{subsec:commutehalffulltrue}
%========================

We do not use the results of this section, but include them for the edification of the reader. Surprisingly, we could not find the proofs in the literature; a manuscript with the proofs
is in preparation.

The previous section gives the misleading impression that the functors $B\mapsto \HT_n \ot B \ot \HT_n\inv$ and $B\mapsto \tau(B)$ from $\SBim_n$ to $\SBim_n$ should be isomorphic.  Instead, $\HT_n \ot (-) \ot \HT_n\inv$ is isomorphic to $\tau'$, which is obtained by composing $\tau$ with an automorphism of $\SBim_n$ which fixes objects and multiplies certain morphisms by signs\footnote{Precisely, the end-dot and splitting trivalent vertices should be multiplied by a sign, while the other morphisms are fixed. This leads to the barbell being multiplied by a sign, consistent with the fact that $\tau(x_i - x_{i-1})$ is a negative root, not a positive root.}. The homotopy equivalence in Lemma \ref{lem:BabsorbsF} is not functorial,
but involves twisting morphisms with signs, which leads to this result.

\begin{remark} \label{rmk:provingcentral} Conjugation by $\HT_n$, composed with $\tau$, is an autoequivalence of $\SBim_n$ which sends each object to itself. There are very few
autoequivalences of this form. The generating morphisms (univalent vertices, trivalent vertices, $2m$-valent vertices) all live in one-dimensional morphism spaces, so they must be sent
to non-zero scalar multiples of themselves. The relations of the category imply certain connections between these scalars. By computing directly how conjugation by the half twist affects
the generating univalent vertices, one deduces how it affects every map in the category without any additional work. \end{remark}

Since $\tau'$ is an involution, conjugation by the full twist is isomorphic to the identity functor on $\SBim_n$. Another way of saying this is that $\FT_n$ is an object in the Drinfeld
center of $\KC^b(\SBim_n)$.

\begin{corollary} \label{cor:FTreallycommutes}
We have $\FT_n\otimes C\simeq C\otimes \FT_n$ for all $C\in \KC^b(\SBim_n)$.\qed
\end{corollary}

In this paper we only need this corollary for the cases when $C$ is the Rouquier complex of a braid, or when $C$ is concentrated in a single degree; both of these special cases were
proven in the previous section.

\begin{remark}\label{rmk:HTnotCentral}
Let $W$ be a finite Coxeter group with no diagram automorphisms.  Then the half-twist $H_{w_0}$ is central in the Hecke algebra $\HB(W)$.  However, the comments in this section suggest that one should not expect the Rouquier complex $\HT$ to correspond to an object of the Drinfeld center of $\KC^b(\SBim(W))$, since conjugating by $\HT$ will only preserve the generating morphisms in $\SBim(W)$ up to signs.
\end{remark}

%========================
\subsection{Bounding the action of the half twist}
\label{subsec:bounding}
%========================

\begin{thm} \label{thm:HTactionA} The half twist $\HT_n$ in type $A$ is twist-like, increasing, and sharp. In other words, conjecture \ref{conj:HTaction} holds in type $A$. \end{thm}

\begin{proof} We have shown that $\HT_n$ is twist-like in \S\ref{subsec:commutehalffull}. We have proven that $\mb_{\HT_n}(\l) = \cbb(\l)$ in Proposition \ref{prop:htm}. What remains to be proven is that $\nb_{\HT_n}(\l) = \cbb(\l)$ for all $\l \in \PC(n)$. Note that $\nb_{\HT_n} \ge \mb_{\HT_n}$ by Lemma \ref{lemma:nmineq}, so we need only prove $\nb_{\HT_n} \le \cbb(\l)$. By Lemma \ref{lemma:tailProperty}, $\nb_{\HT_n}(B)$ only depends on the two-sided cell of an indecomposable object $B$, so it suffices to choose a single indecomposable object $B$ in each cell $\l$, and prove that $\HT_n \ot B$ is supported in homological degrees $\le \cbb(\l)$. Thus, the theorem is deduced from Proposition \ref{prop:HTsupport}. \end{proof}

\begin{prop}\label{prop:HTsupport}
For $\l\in \PC(n)$, let $w_\l$ denote the longest element of the parabolic subgroup $S_\l = S_{k_1}\times \cdots \times S_{k_r}$ where $k_1\geq \cdots \geq k_r\geq 1$ are the column lengths of $\l$. Then $\HT_n\otimes B_{w_\l}$ is homotopy equivalent to a complex in homological degrees $\le \cbb(\l)$.
\end{prop}

Before discussing the proof, let us state the implications of Theorem \ref{thm:HTactionA}.

\begin{theorem}\label{thm:halftwistprops}
In case $W=S_n$, we have $\nb_{\HT}(\l)=\mb_{\HT}(\l)=\cbb(\l)$ for every partition $\l\in \PC(n)$.  If $P,Q\in \SYT(\l)$, then the head of $\HT_n\ot B_{P,Q}$ is isomorphic to $B_{P^\vee,Q}[-\cbb(\l)](\xbb(\lambda))$.  
\end{theorem}

\begin{proof} We can apply Proposition \ref{prop:tailProperty} to determine the shape of $\HT_n \ot B_{P,Q}$. Then the statement about its head follows from the decategorified statement, which is \eqref{eq:longesttimescell}. \end{proof}

As an immediate corollary we have the following result on the full-twists.

\begin{corollary}\label{cor:fulltwistprops}
We have $\nb_{\FT}(\l)=\mb_{\FT}(\l)=2\cbb(\l)$.  If $P,Q\in \SYT(\l)$, then the head of $\FT_n\otimes B_{P,Q}$ is isomorphic to $B_{P,Q}[2\cbb(\l)](2\xbb(\lambda))$.
\end{corollary}

Now we get to the proof of Proposition \ref{prop:HTsupport}. In this section we reduce the proposition to a key lemma. In the next section, we prove this lemma. 

Let $w_k$ denote the longest element of $S_k$. Let $T$ and $T^\vee$ be the tableaux of shape $(2,1^{k-1})$ such that
\[
w_k \sqcup 1_1 = w(T,T) \ \ \ \ \ \ \ \ 1_1\sqcup w_k = w(T^\vee,T^\vee).
\]
In $T$, the box in the second column is labeled $k+1$, while in $T^\vee$ this box has label $2$. Note that $T^\vee$ is the Sch\"utzenberger dual of $T$.

\begin{lemma}\label{lem:thickCrossing}
Using the notation of the previous paragraph, we have
\begin{equation}\label{eq:thickcrossing}
F(\sigma_1\cdots\sigma_k)\otimes(B_{w_k}\sqcup \one_1) \ \ \simeq \ \ (\underline{B_{w_{k+1}}}\rightarrow B_{w(T^\vee,T)}(1))
\end{equation}
where $w_{k+1}\in S_{k+1}$ is the longest element. In particular, it is supported in homological degrees $0$ and $1$.
\end{lemma}

The left hand side of \eqref{eq:BFcommute} is the complex pictured below (when $k=4$):
\begin{equation}\label{eq:forkslideish}
\eqig{.9}{Gk1}\ \ \simeq \ \ \eqig{.9}{Gk1_rotated},
\end{equation}
where the equivalence of these two complexes is given by bimodule sliding (Lemma \ref{lem:BFcommute}).


\begin{remark} This lemma would follow from the so-called ``fork-sliding relation'' (not reprinted here) for complexes of singular Soergel bimodules. It is stated without proof in \cite{WW09} and is certainly expected to be true. As there is no proof in the literature, we choose to bypass the fork-lore and prove Lemma \ref{lem:thickCrossing} in the next section.\end{remark}


The rest of this section is a proof of Proposition \ref{prop:HTsupport}, given Lemma \ref{lem:thickCrossing}.

For each pair of integers $k,\ell$, the \emph{cabled crossing} $x(k,\ell)$ is the element of $S_{k+\ell}$ which crosses the first $k$ strands over the last $\ell$, without permuting either block of strands. In other words, $x(k,\ell)$ is the shortest length element in the coset $w_{k+\ell} (S_k \times S_\ell)$. Let $X(k,\ell)$ denote the Rouquier complex associated to the positive braid lift of $x(k,\ell)$. Explicitly,
\begin{equation}
X(k,\ell) =  F\Big((\sigma_k\cdots\sigma_{k+\ell-1})(\sigma_{k-1}\cdots \sigma_{k+\ell-2})\cdots (\sigma_1\cdots \sigma_\ell)\Big).
\end{equation}
We may shorten this to
\begin{equation}
X(k,\ell) =  F_{k, k+1, \ldots, k+\ell-1} \ot \cdots \ot F_{2,3,\ldots,k+1} \otimes  F_{1,2,\ldots,k}.
\end{equation}
For example
\[
X(3,2) = \eqig{.8}{thickcrossing}.
\]

Let $\GB(k,\ell)$ denote the complex
\begin{equation}
\GB(k,\ell):= X(k,\ell)\otimes (B_{w_k}\sqcup \one_{\ell})
\end{equation}
where $w_k\in S_k$ is the longest element.  For example
\[
\GB(3,2)= \eqig{.8}{Gexample}.
\]

\begin{lemma}\label{lem:Gkl}
The complex $\GB(k,\ell)$ is homotopy equivalent to a complex supported in homological degrees between $0$ and $\ell$.
\end{lemma}

\begin{proof}
The main trick in this proof is the fact that, for any $m \ge 1$, $B_{w_k}^{\ot m}$ is just a direct sum of many copies of $B_{w_k}$ (with grading shifts). Thus, if $X(k,\ell) \ot (B_{w_k}^{\ot m} \sqcup \one_\ell)$ can be bounded in homological degree, then so can $\GB(k,\ell)$.

From Lemma \ref{lem:thickCrossing} we know that $\GB(k,1)$ is homotopy equivalent to a complex supported in homological degrees 0,1. This handles the $\ell = 1$ case.

In case $\ell=2$, we claim that
\begin{equation}\label{eq:foobar1}
X(k,2) \ot (B_{w_k}^{\ot 2} \sqcup \one_2) \simeq F_{2,3,\ldots,k+1} \otimes (\one_1\sqcup B_{w_k}\sqcup \one_1)\otimes F_{1,2,\ldots,k}\otimes(B_{w_k}\sqcup \one_2),
\end{equation} where we have applied Lemma \ref{lem:BFcommute} to slide one $B_{w_k}$ past $F_{1, 2, \ldots, k}$. But \eqref{eq:foobar1} can clearly be rewritten as 
\begin{equation}
(\GB(k,1)\sqcup \one_1)\otimes (\one_1\sqcup \GB(k,1)),
\end{equation}
pictured for $k=4$ below:
\[
\eqig{.8}{bubblegum}.
\]
By the $\ell = 1$ case, this is homotopy equivalent to a complex supported in homological degrees between 0 and 2. Since $G(k,2)$ is a direct summand of the complex in \eqref{eq:foobar1}, we have proven the $\ell = 2$ case.

In general $\GB(k,\ell)$ is a direct summand of
\[
\bigotimes_{i=0}^{\ell-1} F_{i+1,\ldots,i+k}\otimes(\one_i\sqcup B_z\sqcup\one_{\ell-i}) = \bigotimes_{i=0}^{\ell-1} \one_{i}\sqcup \GB(k,1)\sqcup \one_{\ell-i-1},
\]
and application of Lemma \ref{lem:thickCrossing} to each occurence of $\GB(k,1)$ completes the proof.
\end{proof}

Now let $x(k_1, k_2, \ldots, k_r)$ denote the \emph{cabled half twist} in $S_n$, where $n = \sum k_i$. Explicitly, the cabled half twist is the shortest length element in the coset $w_n (S_{k_1} \times \cdots \times S_{k_r})$. When $k_1 \ge k_2 \ge \ldots \ge k_r \ge 1$, and $\l$ is the partition with these column sizes, we let $x_\l$ denote this cabled half twist, and $X_\l$ denote the Rouquier complex associated to its positive lift.

For instance if $\l$ is the partition with column lengths $3,2,2$, then
\[
X_{\l} = \eqig{.8}{cabledHalftwist}.
\]

\begin{lemma}\label{lem:thickHT}
We have
\begin{equation}
X_\l \ot B_{w_\l} \simeq (\GB(k_{r},0)\sqcup \one_{k_1+\cdots+k_{r-1}})\otimes (\GB(k_{r-1},k_r)\sqcup \one_{k_1+\cdots+k_{r-2}})\cdots \otimes\GB(k_1,k_2+\cdots+k_r)
\end{equation}
\end{lemma}
\begin{proof}
This is straightforward, and best illustrated by example.  For instance, when $\l=(3,3,2)$ we have $(k_1,k_2,k_3)=(3,2,2)$, and
\begin{equation}
\eqig{.8}{thickHalftwist0} \  \ \ \simeq \ \ \   \eqig{.8}{thickHalftwist}.
\end{equation}
The picture on the left denotes $X_\l \otimes B_{w_\l}$, and the picture on the right is $(\GB(2,0)\sqcup \one_{5})\otimes (\GB(2,2)\sqcup \one{3})\otimes \GB(3,4)$.  These complexes are homotopy equivalent by bimodule sliding (Lemma \ref{lem:slidingBim}).
\end{proof}

Recall that $k_1,\ldots,k_r$ are the column lengths of $\l$, so that
\[
(k_2+\cdots+k_n)+(k_3+\cdots+k_n)+\cdots + k_n = \cbb(\l).
\]
Consequently, the following corollary is immediate from Lemma \ref{lem:thickHT} and Lemma \ref{lem:Gkl}.

\begin{cor} \label{cor:thickHT}
The complex $X_\l \ot B_{w_\l}$ is supported in homological degrees between $0$ and $\cbb(\l)$. \qed\end{cor}

\begin{proof}[Proof of Proposition \ref{prop:HTsupport}]
We observe that $w_0 = x_\l w_\l$ with $\ell(w_0) = \ell(x_\l) + \ell(w_\l)$, so that $\HT_n \simeq X_\l \ot F_{w_\l}$. Using Lemma \ref{lem:BabsorbsF}, $F_{w_\l} \ot B_{w_\l} \simeq B_{w_\l}(-\ell(w_\l))$, and no homological shift is created. Therefore,
\begin{equation} \HT_n \ot B_{w_\l} \simeq X_\l \ot B_{w_\l}(-\ell(w_\l)), \end{equation}
which is supported in homological degrees between $0$ and $\cbb(\l)$ by Corollary \ref{cor:thickHT}. This proves Proposition \ref{prop:HTsupport}, given Lemma \ref{lem:thickCrossing}.
\end{proof}

%========================
\subsection{Proof of the lemma}
\label{subsec:boundinglemma}
%========================

To prove Lemma \ref{lem:thickCrossing} we must prove
\begin{equation} \label{eq:tC}
F(\sigma_1\cdots\sigma_k)\otimes(B_{w_k}\sqcup \one_1) \ \ \simeq \ \ (\underline{B_{w_{k+1}}}\rightarrow B_{w(T^\vee,T)}(1))
\end{equation}
Here, $w_k \in S_k$ is the longest element, and $T, T^\vee$ are tableau describing the involutions $w_k \sqcup 1_1$ and $1_1 \sqcup w_k$ respectively.

We begin by recalling some finer aspects of multiplication in the Hecke algebra, see \cite[Theorem 6.6]{LuszUnequal14}. Given a simple reflection $s$ and an element $x \in W$ such that $sx > x$, define for any $y \in W$ the (non-negative) integer $\mu(x,s;y)$ as follows: \begin{itemize} \item If $sy<y$ and $y < x$, then $\mu(x,s;y)$ is the coefficient of $v^1$ inside $h_{y,x}$. \item Otherwise, $\mu(x,s;y)=0$. \end{itemize} Then one has
\[ b_s b_x = b_{sx} + \sum_{y \in W} \mu(x,s;y) b_y. \]
These lower terms in the product $b_s b_x$, governed by the $\mu$-coefficients, are mysterious in general.

When $b_x$ is smooth (i.e. all the Kazhdan-Lusztig polynomials $h_{y,x}$ are trivial) then things simplify. In this case, $\mu(x,s;y) = 1$ precisely when $\ell(y) = \ell(x)-1$, $y<x$ and
$sy<y$, and it is zero otherwise. To state things more algorithmically: suppose $x$ is smooth. Fix an arbitrary reduced expression of $x$. For each reflection in this expression, try
removing it, and see if what remains is still a reduced expression (for some element $y$), and whether $s$ is in its left descent set. If so, then $b_y$ appears in the expression for
$b_s b_x$; moreover, $b_s b_x = \sum_y b_y$ is the sum of the elements $y$ appearing in this way.

Tensoring Rouquier complexes is difficult, but the computation in \eqref{eq:tC} is drastically simplified by the fact that every $b_x$ in sight will be smooth! Although this can be proven by direct computation, here is an easier justification.

A permutation $x$ in $S_n$ is \emph{$3412$-avoiding} if there does not exist any $1 \le i<j<k<\ell \le n$ such that $x(k) < x(\ell) < x(i) < x(j)$. Similarly, it is
\emph{$4231$-avoiding} if there does not exist any $1 \le i<j<k<\ell \le n$ such that $x(\ell) < x(j) < x(k) < x(i)$. It is a famous result of Lakshmibai-Sandhya \cite{LakSan90} that
whenever $x$ is both $3412$-avoiding and $4231$-avoiding, the corresponding Schubert variety is smooth, and consequently the Kazhdan-Lusztig basis element $b_x$ is also smooth. We state
a lemma to help determine whether a permutation is pattern avoiding.

\begin{lemma} \label{lem:patternavoid} Let us encode a permutation $w \in S_n$ by listing $w(1), w(2), \ldots, w(n)$ in order. For example, $35421$ is the permutation in $S_5$ which
sends $1$ to $3$, $2$ to $5$, etcetera.  The \emph{tail} of $w$ is the sequence of numbers after the largest. For example, the tail of $35421$ is $421$. Let $w^- \in S_{n-1}$ be the permutation on one fewer letter, obtained by removing the largest number. For example, if $w = 35421$ then $w^- =
3421$.

Suppose that $w \in S_n$ is a permutation for which \begin{itemize} \item $w^-$ is $3412$-avoiding and $4231$-avoiding, and \item the tail is a decreasing sequence. \end{itemize} Then $w$ is also $3412$-avoiding and $4231$-avoiding. \end{lemma}

\begin{proof} In order for the lemma to fail, one must find $1 \le i < j < k < \ell \le n$ such that $w$ applied to these indices satisfies a certain forbidden pattern. We can not choose
$\ell = n$, because both the pattern $3412$ and the pattern $4231$ have non-decreasing sequences after the $4$. However, if all the indices are strictly less than $n$, then the same
pattern holds for $w^-$, which can not happen by assumption. \end{proof}

In our proof below, we will be producing elements of $S_{k+1}$ from elements of $S_{k}$ by adding a final strand, and it will be easy for the reader to use Lemma \ref{lem:patternavoid} to
verify that all permutations in sight are $3412$-avoiding and $4231$-avoiding, and therefore smooth. We will also be (tacitly) applying the algorithm discussed above to decompose $b_s
b_x$.

\begin{proof}[Proof of Lemma \ref{lem:thickCrossing}]
Let $z$ be the longest element of $S_k\times S_1\subset S_{k+1}$. Then we are interested in the complex
\begin{equation} F_{s_1} F_{s_2} \cdots F_{s_k} B_z.\end{equation}

Because $z$ is smooth, and no smaller elements in the Bruhat order have $s_k$ in its descent set, we have $b_{s_k} b_z = b_{s_k z}$. Thus $F_{s_k} B_z$ is the two-term
complex \[\Big( \un{B_{s_k z}(0)} \to B_z(1) \Big). \] Under the isomorphism $B_{s_k z} \cong B_{s_k} B_z$, the differential is just the dot map $B_{s_k} \to R(1)$ applied to the first tensor factor.

Because $F_{s_{k-1}} B_z(1) \cong B_z(0)$, we see that $F_{s_{k-1}s_k} B_z = F_{s_{k-1}} F_{s_k} B_z$ has the form
\begin{equation} \label{eq:foobar22}
\Big( \un{B_{s_{k-1}} B_{s_k z} (0)} \to B_z(0) \oplus B_{s_k z}(1) \Big).
\end{equation}
Let us decompose $B_{s_{k-1} B_{s_k z}} = B_{s_{k-1}} B_{s_k} B_z$. The element $b_{s_k z}$ is smooth, so that $b_{s_{k-1}} b_{s_k z} = b_{s_{k-1}s_k z} + b_z$. Correspondingly, $B_{s_{k-1}} B_{s_k z} \cong B_{s_{k-1}s_k z} \oplus B_z$. Let us remark on the
projection map from $B_{s_{k-1}} B_{s_k} B_z$ to $B_z$. This can be obtained as a composition $B_{s_{k-1}} B_{s_k} B_z \to B_{s_{k-1}} B_z(1) \to B_z(0)$. The first
map is the dot $B_{s_k} \to R(1)$ applied to the middle tensor factor. The second map is projection to the first summand under the isomorphism $B_{s_{k-1}} B_z \cong B_z(-1) \oplus
B_z(+1)$ coming from \eqref{eq:sdown}. Verifying that this composition (often called a \emph{pitchfork}) is the projection onto a direct summand is analogous to case of the the
projection map from $B_s B_t B_s$ to $B_s$ which is discussed at length in \cite{EKho}. Keeping careful track of the differential in \eqref{eq:foobar22}, the map from $B_{s_{k-1}} B_{s_k z}(0) \to B_z(0)$ is precisely this pitchfork projection! Thus, one can apply Gaussian elimination to remove the $B_z(0)$ summand from both terms. The result is a two-term complex
\begin{equation}
F_{s_{k-1}} F_{s_k} B_z \simeq \Big( \un{B_{s_{k-1}s_k z}(0)} \to B_{s_k z}(1) \Big).
\end{equation}
Keeping track of the differential is another exercise: it is the inclusion of the direct summand $B_{s_{k-1}s_k z}(0)\rightarrow B_{s_{k-1}}B_{s_k}B_z(0)$ followed by the dot map $B_{s_{k-1}}B_{s_k}B_z(0)\rightarrow B_{s_k}B_z(1)$, followed by the projection onto a direct summand $B_{s_k}B_z(1)\rightarrow B_{s_k z}(1)$.

Assume by (descending) induction that
\begin{equation}\label{eq:twoTermCx}
F_{s_i\cdots s_{k-1}s_k}B_z\simeq \Big(\un{B_{s_is_{i+1}\cdots s_k z}} \rightarrow B_{s_{i+1}\cdots s_k z}(1)\Big).
\end{equation}
where the differential is induced from the dot map $B_{s_i}\rightarrow R(1)$ on the Bott-Samelson bimodules of which these indecomposables are direct summands. We wish to prove the same, replacing $i$ with $i-1$. As above, ${s_is_{i+1}\cdots s_k z}$ is smooth, so $b_{s_{i-1}} b_{s_is_{i+1}\cdots s_k z} = b_{s_{i-1}s_is_{i+1}\cdots s_k z} + b_{s_{i+1}\cdots s_k z}$.  The argument above, involving the pitchfork projection, works almost verbatim to study $F_{s_{i-1}s_is_{i+1}\cdots s_k}B_z$, and prove that it is homotopy equivalent to 
\[ \Big(\un{B_{s_{i-1}s_i\cdots s_k z}} \rightarrow B_{s_{i}\cdots s_k z}(1)\Big).\]
Thus \eqref{eq:twoTermCx} is true for all $i\in\{1,\ldots,k-1\}$ by induction.

When $i=1$ this gives
\[
F_{s_1\cdots s_k}B_z \simeq \Big( \un{B_{s_1s_2\cdots s_k z}(0)} \to B_{s_2\cdots s_k z}(1) \Big).
\]
It is straightforward to verify that $s_1 \cdots s_k z = w_{k+1}$, the longest element of $S_{k+1}$, while $s_2 \cdots s_k z = w(T^\vee,T)$, as desired.
\end{proof}






%========================%========================%========================%========================%========================%========================








% \begin{definition}
% For each partition $\l=(\l_1,\ldots,\l_r)$, set $\rbb(\l):=\sum_i (i-1)\l_i$, $\cbb(\l)=\rbb(\l^t)$, and $\xbb(\l)=\rbb(\l)-\cbb(\l)$.  Also, define the grading shift functor $\shift{\l}{}=t^{\cbb(\l)}q^{\rbb(\l)}$, where $t$ and $q$ are shorthand for the grading shifts
% \[
% q = (-2),\qquad\qquad t=[-2](2),
% \]
% as in \S \ref{}.
% \end{definition}
%
%

%
%
%
%
% Note that in case $n=3$ we recover exactly the observations from \S \ref{}.  We consider some examples in case $n=4$ below.
%
%
% \begin{ex} Consider \eqref{eq:Fstsuts} above, describing $\HT_4$. For the maximal cell, i.e. the partition $\l = (4)$, one has $\colsum(\l)=6$.  In homological degree $6$, $\HT_4$ consists only
% of $R(6)$, and $R = B_{(P,P^\vee)}$ for the unique $P \in \SYT(\l)$.  Similarly, for the minimal cell, i.e. the partition $\l = (1,1,1,1)$, one has $\colsum(\l)=0$. In homological degree $0$,
% $\HT_4$ consists only of $B_{(P,P^\vee)}(0)$ for the unique $P \in \SYT(\l)$.
%
% When $\l = (2,1,1)$, $\colsum(\l)=1$. In homological degree $1$, $\HT_4$ is the sum of $B_{(P,P^\vee)}$ over the three different $P \in \SYT(\l)$.
%
% When $\l = (2,2)$, $\colsum(\l)=2$. In homological degree $2$, every summand of $\HT_4$ is in cell $(2,1,1)$ except for the summands $B_{tsut}(2) \oplus B_{su}(2)$, which are precisely the two elements $(P,P^\vee)$ in cell $\l$. Here, the fact that $tsut$ has non-trivial Kazhdan-Lusztig polynomial allows $w_0(tsut) = su$ to appear in smaller-than-might-be-expected homological degree, which is essential for Proposition \ref{prop:HTdistribution} to hold.
%
% Similarly, when $\l = (3,1)$, $\colsum(\l)=3$. In homological degree $3$, the terms in the relevant cell are $B_{stu}(3) \oplus B_{t}(3) \oplus B_{uts}(3)$. The appearance of $B_t$ relied on the non-trivial Kazhdan-Lusztig polynomial for $sutsu$. \end{ex}



