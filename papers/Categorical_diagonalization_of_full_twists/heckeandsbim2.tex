%%%%%%%%%%%%%%%%%%%%%%%%%
\section{The Hecke algebra and Soergel bimodules}
\label{sec:heckeandsbim}
%%%%%%%%%%%%%%%%%%%%%%%%%

%This chapter is our first background section on the Hecke algebra and its categorification, the category of Soergel bimodules. In this chapter we only recall those definitions and observations which are well-known to anyone familiar with these topics. We draw the expert's attention to \S\ref{subsec:Hw0}, and to the notation in \S\ref{subsec:external}. In later chapters we recall more specialized facts.

%We restrict our attention to type $A$ in the main body of this paper, to put some readers at ease. However, when it does not interfere pedagogically, we use language which adapts to the general case of an arbitrary finite Coxeter system $(W,S)$. In Appendix \ref{sec:arbcoxeter}, we discuss the entire paper in the context of general Coxeter systems.



%Fix a Coxeter system $(W,S)$. Let $\le$ denote the Bruhat order and $\ell(w)$ denote the length of an element $w \in W$. When $W$ is finite we let $w_0$ denote the longest element of $W$.

%\begin{ex} In type $A_{n-1}$, $W = S_n$ is the symmetric group on $n$ letters, with simple reflections $S = \{s_i\}_{1 \le i \le n-1}$ where $s_i$ is the transposition of $i$ and $i+1$.
%The reader unfamiliar with Coxeter groups can work with the symmetric group throughout. \end{ex}

In this chapter, we provide the requisite background information on the Hecke algebra and Hecke category attached to a Coxeter system $(W,S)$. In the next chapter, we provide background information on cell theory. This chapter is largely review, though we call the expert reader's attention to \S\ref{subsec:conjugate} and to the examples in \S\ref{subsec:minimalRouq}.




%========================
\subsection{Notation and the basics}
\label{subsec:notation}
%========================


Throughout this chapter, fix a Coxeter group $W$ and a set of simple reflections $S\subset W$. In this paper we often restrict to \emph{type $A$} or \emph{type $A_{n-1}$}: this means that $W$ is the symmetric group $S_n$, and $S = \{s_i\}_{i=1}^{n-1}$ are the adjacent transpositions $s_i = (i\ i+1)$.

An \emph{expression} for an element $w \in W$ is a sequence $\un{w} = (s_{i_1}, s_{i_2}, \ldots, s_{i_d})$ with $s_{i_k} \in S$, such that $s_{i_1} s_{i_2} \cdots s_{i_d} = w$. When there is no ambiguity we assume that an underlined letter $\un{x}$ represents an expression for the element $x \in W$. A \emph{reduced expression} or \emph{rex} is an expression for $w$ of minimal length $d$, and this number $d$ is the \emph{length} of $w$, denoted $\ell(w)$.

If $W$ is finite, then there is a unique element of longest length, denoted $w_0$. In type $A_{n-1}$, the longest element is the permutation sending $1\leftrightarrow n$,
$2\leftrightarrow n-1$, etcetera. Its length is $\binom{n}{2}$, having a reduced expression \[w_0=s_1(s_2s_1)\cdots (s_{n-1}\cdots s_2s_1).\]

An \emph{involution} in $W$ is an element such that $w^2=1$. Longest elements are always involutions.

The set $W$ has a \emph{Bruhat order}, a partial order respecting the length function, where $x \le w$ if one can obtain a rex for $x$ as a subsequence of a rex for $w$. The identity is the unique minimal element in the Bruhat order, and $w_0$ is the unique maximal element.

A \emph{parabolic subgroup} is a subgroup $W_I$ generated by a subset $I \subset S$. Then $W_I$ is a Coxeter group with simple reflections $I$. Any rex in $S$ for an element $w \in W_I$
will only use simple reflections in $I$, and thus the length function and Bruhat order on $W_I$ are the restrictions of the corresponding notions on $W$. When $W_I$ is finite, its
longest element will be denoted $w_I$. In type $A_{n-1}$, all parabolic subgroups have the form $S_{k_1} \times S_{k_2} \times \cdots \times S_{k_r}$ with $\sum k_i = n$.

Typically the letter $s$ denotes an arbitrary simple reflection. When discussing $S_n$ with $n \le 4$ we let $s = s_1$, $t = s_2$, and $u = s_3$, by convention.



%========================
\subsection{The Hecke algebra}
\label{subsec:Hecke}
%========================

The \emph{Hecke algebra} $\HB = \HB(W)$ is a deformation of the group algebra of $W$ over the base ring $\Z[v,v\inv]$, where $v$ is a formal indeterminate. As a $\Z[v,v\inv]$-module, $\HB$ is free, with a \emph{standard basis} $\{H_w\}_{w \in W}$.  The multiplication is determined by
\begin{itemize}
\item $H_w H_v = H_{wv}$ if $\ell(wv)=\ell(w)+\ell(v)$,
\item $(H_s+v)(H_s-v\inv)=0$ for each simple reflection $s\in S$.
\end{itemize}
Here $H_1 = 1$ is the unit.

A map $f$ of $\Z[v,v\inv]$-modules is \emph{antilinear} if $f(vm) = v^{-1} f(m)$. The Hecke algebra is equipped with an antilinear automorphism called the \emph{bar involution}, uniquely
specified by $\overline{H_s}=H_s\inv$. It is an algebra homomorphism, so that $\overline{ab}=\overline{a}\overline{b}$. In terms of the standard basis, we have $\overline{H_w} =
H_{w\inv}\inv$. The following theorem is due to Kazhdan-Lusztig \cite{KazLus79}.

\begin{theorem}\label{thm:KLbasis}
There are unique elements $b_w\in \HB$ for each $w \in W$, such that $\overline{b_w}=b_w$ and
\[
b_w = H_w + \sum_{y< w} h_{y,w}(v) H_y
\]
where $h_{y,w}(v)\in v\Z[v]$. The elements $\{b_w\}_{w \in W}$ form a basis of $\HB$, which we call the \emph{Kazhdan-Lusztig basis} or \emph{KL basis}. The polynomials $h_{y,w}(v)$ are called \emph{Kazhdan-Lusztig polynomials} or \emph{KL polynomials}.
\end{theorem}

\begin{remark}
We are mostly following the notation of Soergel \cite{Soer97}, and recommend his exposition in \cite[\S 2]{Soer97} for the basics of the Hecke algebra. However, we write $b_w$ where Soergel writes $\un{H}_w$. To match the conventions of Kazhdan-Lusztig \cite{KazLus79}, set $H_w = v^{\ell(w)} T_w$, $q=v^{-2}$, $b_w = C_w'$, and $h_{y,w} = v^{\ell(w) - \ell(y)} P_{y,w}(v^{-2})$.
\end{remark}


Some additional properties of KL polynomials are recorded in this proposition.

\begin{prop} \label{prop:KLpolyprops} Fix $y < w$. The coefficient of $v^{\ell(w) - \ell(y)}$ in $h_{y,w}$ is $1$. Moreover, the coefficient of $v^k$ is zero unless $k \le \ell(w) -
\ell(y)$ and unless $k$ as the same parity as $\ell(w) - \ell(y)$. \end{prop}

By convention, one sets $h_{w,w}(v) = 1$ and $h_{y,w}(v) = 0$ if $y \nleq w$, so that $b_w = \sum_y h_{y,w}(v)H_y$. Thus, the KL polynomials give the change of basis matrix between the standard basis and the KL basis. Note also that $b_1 = H_1 = 1$.
	
\begin{ex} Suppose that $W$ is finite, with longest element $w_0$. Then \[b_{w_0} = \sum_{y \in W} v^{\ell(w_0) - \ell(y)} H_y.\]  \end{ex}
	
When $h_{y,w} = v^{\ell(w)-\ell(y)}$ this KL polynomial is called \emph{trivial}, because it has the simplest possible behavior. We say that $w$ is \emph{smooth} if all Kazhdan-Lusztig
polynomials $h_{y,w}$ are trivial. For instance, the longest word of any parabolic subgroup is smooth. Any element of a dihedral group (e.g. $S_3$) is smooth.

Starting by giving examples of smooth elements is misleading, as smoothness is the exception in larger Coxeter groups. Polo \cite{Polo} proved that KL polynomials become arbitrarily
complicated, in that any polynomial whatsoever satisfying the properties of Proposition \ref{prop:KLpolyprops} is a KL polynomial for some pair of elements in $S_n$, for some $n$. There
are no known closed formulas for KL polynomials in general.

\begin{ex} \label{ex:notsmooth} For symmetric groups, the first non-smooth elements appear in $S_4$: the permutations $tsut$ and $sutsu$. Here are two nontrivial KL polynomials:
$h_{1,tsut} = {\color{red} v^2} + v^4$ and $h_{1,sutsu} = {\color{red} v^3} + v^5$ (the red terms make the KL polynomial nontrivial). \end{ex}

To conclude this section we state the Kazhdan-Lusztig inversion formula, which expresses the standard basis in terms of the KL basis.

\begin{prop}\label{prop:KLinversion} (\cite[Theorem 3.1]{KazLus79}, see also \cite[Remark 3.10]{Soer97})
If $W$ is a finite Coxeter group, then
\begin{equation} \label{eq:KLinversion} H_y = \sum_{x\leq y} (-1)^{\ell(y)-\ell(x)}h_{w_0y,w_0x}(v) b_x, \end{equation}
where $w_0\in W$ denotes the longest element.
\end{prop}


%========================
\subsection{The half twist}
\label{subsec:Hw0}
%========================
% \begin{proof}
% Theorem 3.1 in \cite{KazLus79} states that the polynomials $P_{x,y}(v^{-2})=v^{\ell(x)-\ell(y)}h_{x,y}(v)$ satisfy:
% \[
% \sum_{z} (-1)^{\ell(x)+\ell(z)}P_{x,z} P_{w_0y,w_0z}=\d_{x,y}
% \]
% for all $x,y\in W$.  Note that a summand in the above sum is zero unless $x\leq z\leq y$.  Recall that $\ell(w_0y)=\ell(w_0)-\ell(y)$.  Using this fact, an easy manipulation of the above identity yields:
% \[
% \sum_{z} (-1)^{\ell(z)} h_{x,z}h_{w_0y,w_0z} = (-1)^{\ell(x)}v^{\ell(x)-\ell(y)}\d_{x,y}.
% \]
% Since the right-hand side vanishes for $x\neq y$, the right-hand side of the above equals $(-1)^{\ell(y)}\d_{x,y}$.  It follows that the inverse of the triangular matrix $(h_{x,y})_{x\leq y}$ is $((-1)^{\ell(x)+\ell(y)}h_{w_0y,w_0x})_{x\leq y}$.  Thus,
% \[
% H_y = \sum_{x} \d_{x,y}H_x = \sum_{x,z} (-1)^{\ell(z)+\ell(y)} h_{x,z}h_{w_0y,w_0z}  H_x = \sum_z (-1)^{\ell(z)+\ell(y)} h_{w_0y,w_0z}b_z
% \]
% as claimed.
% \end{proof}

If $W$ is a finite Coxeter group and $w_0\in W$ is the longest element, then we refer to the element $H_{w_0}$ as the \emph{half twist} and $H_{w_0}^2$ as the \emph{full twist}.  These operators will be the major players in this paper.

\begin{definition}
If $W$ is a finite Coxeter group, let $\tau:W\rightarrow W$ denote the group automorphism $\tau(x)=w_0xw_0$.
\end{definition}

Clearly $\tau^2 = \Id$. One can show that $\tau$ preserves the set of simple reflections, so it corresponds to an automorphism of the Coxeter-Dynkin diagram.  Consequently, if there are no nontrivial diagram automorphisms then $\tau=\Id_W$, and $w_0$ is central in $W$. 

%Let $\ell$ denote the length of the longest element.  Then the length of $w_0s$ is $\ell-1$, hence there exists an element $s'$ of length one such that $w_0 = s'w_0s$.  In other words, $s'$ is a simple reflection and $s' = w_0 s w_0=:\tau(s)$.
% I usually use a proof by counting positive roots sent to negative roots, yours is a little nicer.


\begin{lemma}\label{lem:ftIsCentral_decat}
Let $W$ be a finite Coxeter group. Then 
\begin{equation} H_{w_0} H_x H_{w_0}\inv = H_{\tau(x)}.\end{equation}
Thus the full twist $H_{w_0}^2$ is central in $\HB$, and if $W$ has no diagam automorphisms then $H_{w_0}$ is also central.
\end{lemma}

\begin{proof}
It is enough to show the result when $x = s$ is a simple reflection. We have
\[
H_{w_0} = H_{w_0s} H_s  = H_{\tau(s)}H_{\tau(s)w_0}  = H_{\tau(s)} H_{w_0 s}
\]
since $\tau(s)w_0 = w_0s$.  It follows that  $H_{w_0} H_s = H_{\tau(s)} H_{w_0}$.
\end{proof}


Lusztig proved \cite{Lusz81} that, after extending scalars from $\Z[v,v\inv]$ to $\Q(v)$, the Hecke algebra $\HB(W)$ of a finite Coxeter group is isomorphic to the group algebra $\Q(v)[W]$, and hence is semisimple. A central element in a semisimple algebra acts diagonalizably on any representation, so it follows that left multiplication by the full twist is a diagonalizable operator on $\HB$ (after extending scalars).

Let us mention the following consequence of Proposition \ref{prop:KLinversion}.
 
\begin{cor} \label{cor:KLhalftwist}
We have 
\begin{equation} \label{eq:KLhalftwist} H_{w_0} = \sum_{x \in W} (-1)^{\ell(w_0) - \ell(x)} h_{1, w_0 x}(v) b_x.\end{equation}
\end{cor}

Due to the importance of the half and full twists, the KL polynomials $h_{1,w}$, which are as mysterious as any, will appear often throughout this paper.


%========================
\subsection{Soergel's categorification of the Hecke algebra}
\label{subsec:SoergelCat}
%========================


If $\AC$ is an additive category, then the (split) Grothendieck group $[\AC]$ is defined to be the abelian group generated by symbols $[A]$ (called the \emph{class} of $A$) for objects $A$ of $\AC$, modulo $[X]=[A]+[B]$ whenever $X$ is isomorphic to $A \oplus B$. In particular $A\cong A'$ implies $[A]=[A']$.  Let $\AC$ be an idempotent complete additive category such that each object is a direct sum of finitely many indecomposable objects, and these indecomposable summands are unique up to isomorphism and reordering.  For instance, Krull-Schmidt categories have this property.  In this case, $[\AC]$ is a free abelian group with basis given by the isomorphism
classes of indecomposable objects. An object of $\AC$ is thus determined (up to isomorphism) by its class $[A]$ in $[\AC]$, because this symbol determines the indecomposable summands of $A$ with their multiplicities. We say that $\AC$ categorifies $[\AC]$, and that an object $A$ categorifies its class $[A]\in [\AC]$, and so on.

If $\AC$ is graded with grading shift $(1)$, then $[\AC]$ is a $\Z[v,v\inv]$-module via $v[M] = [M(1)]$. If $\AC$ is monoidal with tensor product $\ot$, then $[\AC]$ is a ring via
$[M]\cdot [N] = [M \ot N]$.

\begin{remark} The reader may be more familiar with abelian (or triangulated) categories and their Grothendieck groups.  However, split Grothendieck groups arise naturally in this situation as well.  For instance even if $\AC$ is abelian, the full subcategory $p\AC\subset \AC$ of projectives in $\AC$ is typically additive but not abelian (unless $\AC$ is semi-simple).   We remark that under mild assumptions on $\AC$ the inclusion $p\AC\rightarrow \AC$ induces an isomorphism between the split Grothendieck group $p\AC$ and the usual Grothendieck group of $\AC$.  The reader new to Soergel bimodules should imagine that they are projective objects in some abelian category. \end{remark}

%\begin{remark} The typical example of a Krull-Schmidt additive category would be the full subcategory of projective objects inside an abelian category. For a finite dimensional algebra, the (exact) Grothendieck group of the abelian category of finitely generated modules is isomorphic to the (split) Grothendieck group of the additive category of finitely generated projective modules. One should think of Soergel bimodules, soon to be defined, as being the projective objects in some abelian category. We will only work with additive categories and split Grothendieck groups in this paper. \end{remark}

Now we discuss Soergel's categorification of the Hecke algebra. Let $\hg$ denote the reflection representation\footnote{More generally, $\hg$ is allowed to be any \emph{realization} of $W$ (see \cite{EWsoergelCalc}).} over $\R$. Let $R=\R[\hg^\ast]$ denote its polynomial ring, viewed as a graded ring where the linear functionals $\hg^*$ live in degree $2$. Then $R$ also has an action of $W$. In type $A_{n-1}$, the reader is welcome to assume that $R = \RM[x_1, \ldots, x_n]$, thought of as a graded
ring with $\deg(x_i)=2$, equipped with its natural action of $S_n$.

Let $R\mathrm{-gbimod}$ denote the category of finitely generated graded $(R,R)$ bimodules. Morphisms in $R\mathrm{-gbimod}$ are degree zero maps of graded bimodules. This is a monoidal
category with tensor product $\ot=\otimes_R$ and monoidal identity $\one=R$, the trivial bimodule. We often abuse notation and write $MN$ instead of $M \ot N$ for the tensor product of
two $R$-bimodules.

For each subset $I \subset S$, let $R^I\subset R$ denote the $W_I$-invariant subalgebra, consisting of polynomials $f\in R$ such that $s(f)=f$ for all $s\in I$. If $I=\{s\}$ is a
singleton, then $R^{\{s\}}$ will also be denoted $R^s$. For each simple reflection $s$, let $B_s$ denote the graded $(R,R)$ bimodule defined by
\[
B_s = R\otimes_{R^s} R(1),
\]
where $(1)$ is the grading shift.  Our convention for grading shifts is such that the element $1\otimes 1$ in $B_s$ lies in degree $-1$. In type $A$, we may use the slightly less cumbersome notation $B_i$ for $B_{s_i}$.

For an expression $\un{w}$, i.e. a sequence $(s_{i_1},s_{i_2},\ldots,s_{i_d})$ of simple reflections, let $\BS(\un{w})$ denote the tensor product $B_{s_{i_1}} B_{s_{i_2}} \cdots B_{s_{i_d}}$. A bimodule obtained in this way is called a \emph{Bott-Samelson bimodule}.

\begin{definition}\label{def:SBim_n}
Let $\SBim\subset R\mathrm{-gbimod}$, the category of \emph{Soergel bimodules}, denote the smallest full subcategory containing $R$ and $B_s$ ($s\in S$), and closed under direct sums, direct summands, tensor products, and grading shifts.
\end{definition}

As an idempotent-closed subcategory of a finitely-generated module category over a field, $\SBim$ is Krull-Schmidt.

% One can also think of Soergel bimodules as the idempotent completion of the (additive and graded envelope of the) subcategory of Bott-Samelson bimodules.

For two Soergel bimodules $M$ and $N$, let $\Homg(M,N)$ denote the graded vector space $\bigoplus_{n \in \ZM} \Hom_{\SBim}(M,N(n))$ of bimodule morphisms of all degrees, as discussed in \S \ref{subsec:gradedcats}.  Then $\Homg(M,N)$ is also a graded $R$-bimodule in the obvious way. Note that $\Homg(R,R) \cong R$. For $f \in R$ and $g \in \Homg(M,N)$ we may write $f \ot g$ for the left action of $f$ on $g$, and $g \ot f$ for its right action.

The following is an amalgamation of some major results of Soergel.



\begin{thm}[Soergel Categorification Theorem] \label{thm:SCT} (\cite[Lemma 6.13 and Satz 6.14]{Soer07}) There is an indecomposable object $B_w$ in $\SBim$ for each $w \in W$, which
appears as a summand inside $\BS(\un{w})$ for any reduced expression $\un{w}$ of $w$, and does not appear as a summand in $\BS(\un{x})$ for any shorter expression $\un{x}$. The
isomorphism classes of indecomposable objects in $\SBim$ are parametrized, up to grading shift, by $\{B_w\}_{w \in W}$.

There is a $\ZM[v,v\inv]$-algebra isomorphism from the split Grothendieck group $[\SBim]$ to the Hecke algebra $\HB$, sending \[ [B_s] \mapsto b_s. \] Thus, for any Soergel bimodule
$B$, we let $[B]$ represent the corresponding element in $\HB$.

For any two Soergel bimodules $B$ and $B'$, the space of $R$-bimodule morphisms $\Homg(B,B')$ is a free left (resp. right) $R$-module with graded rank equal to $([B], [B'])$. Here
$(-, -) \co \HB \times \HB \to \ZM[v,v\inv]$ is the standard pairing in the Hecke algebra (see \cite[\S 2.4]{EWsoergelCalc}). This fact is known as the \emph{Soergel Hom Formula}. \end{thm}

\begin{ex} The bimodule $B_1$ is just the monoidal identity $\one$. The bimodule $B_{w_0}$ can be independently described as $R \ot_{R^{W}} R (\ell(w_0))$. \end{ex}

The Soergel categorification theorem can be proven with comparatively elementary techniques, but the following theorem, often known as Soergel's conjecture, is highly nontrivial.  It was proven for Weyl groups and dihedral groups by Soergel \cite{Soer90, Soer07} and for for general Coxeter groups by the first author and Geordie Williamson \cite{EWHodge}.  Furthermore, it relies on the fact that the base field for $\SBim$ is $\R$, and may fail for realizations defined over a field of finite characteristic.

\begin{thm} \label{thm:SC} The isomorphism $[\SBim] \to \HB$ of the Soergel Categorification Theorem sends \[ [B_w] \mapsto b_w. \] \end{thm}


Just as the KL basis $\{b_w\}$ is mysterious, so too are the bimodules $B_w$, and there are no known direct descriptions of $B_w$ in general. However, the bimodules $B_s$ are
straightforward, which is why one often restricts one's attention to Bott-Samelson bimodules.

To give an example of how the Soergel Hom Formula can be used, it implies that the KL polynomial $h_{1w}$ encodes the graded rank of $\Hom(R,B_w)$ as a left $R$-module.

A major implication of Theorem \ref{thm:SC} and the Soergel Hom Formula is that $\SBim$ is \emph{mixed}, which is rephrased in the following proposition. Note that the indecomposable
Soergel bimodules have the form $B_x(k)$ for various $x \in S_n$ and $k \in \Z$, while the particular bimodules $B_x$ where $k=0$ are special (they are ``self-dual").

\begin{prop} \label{prop:soergelmixed} If $w \ne x$ then $\Hom(B_w, B_x(k))$ is zero unless $k \ge 1$. Also, $\Hom(B_w,B_w(k))$ is zero unless $k \ge 0$. Moreover, when $k=0$,
$\Hom(B_w,B_w)$ is one-dimensional, spanned by the identity map. \end{prop}

In other words, if one restricts to degree zero maps between self-dual indecomposable Soergel bimodules then the category looks like it is semisimple: there are no nonzero morphisms in
$\Hom(B_w, B_x)$ unless $w=x$, and $\End(B_w)$ is one-dimensional, spanned by the identity map.

In particular, the graded Jacobson radical of $\Endg(\bigoplus_{w \in W} B_w)$ is precisely the space spanned by positive degree morphisms. We will be using this property to understand
the minimal forms of complexes in $\SBim$.

%========================
\subsection{Diagrammatics}
\label{subsec:diagram1}
%========================

Morphisms in a monoidal category can often be depicted with planar diagrams, where vertical concatenation is composition, and horizontal concatenation is the tensor product. A
diagrammatic calculus for morphisms between Bott-Samelson bimodules is developed in \cite{EKho, ECathedral, EWsoergelCalc}, and a basis for such morphisms is given in
\cite{EWsoergelCalc} based on work of Libedinsky \cite{LibLL}.

Morphisms are expressed as decorated graphs (with boundary) of a certain kind, where each edge is colored by a simple reflection, and only certain ``generating vertices'' are allowed.
One of these generating vertices, the \emph{dot}, is a univalent vertex, with a single edge (colored $s$). Depending on whether the edge runs to the top or bottom boundary, the dot will
represent either a morphism $R \to B_s(1)$ or a morphism $B_s \to R(1)$. See Remark \ref{rmk:dotsdefn} for a precise definition of these dots. The composition of these two dots is a
\emph{barbell}, a map $R \to R(2)$, which happens to be equal to multiplication by the simple root $\a_s$. In type $A$, one has $\a_{s_i} = x_i - x_{i+1}$.

% \begin{remark} Horizontal concatenation of such diagrams is denoted by $\ot$, not by $\sqcup$! These two kinds of horizontal concatenation occur at two different categorical levels: the
% tensor product $\ot$ internal to $\SBim_n$ categorifies the ordinary (vertical) composition in $\HB_n$, while $\sqcup$ relates categories $\SBim_n$ for different values of $n$.
% \end{remark}

A convincing reason to use diagrammatics is that simple diagrams can express rather complicated operations involving polynomials. We do not bother to recall the diagrammatic calculus
here. Without any knowledge of diagrammatics, the reader should still be able to follow all the proofs in this paper (with fairly easy and
believable exceptions), and will only suffer by not being able to follow some examples.

% \begin{remark} There is also a ``thicker'' diagrammatic calculus in some types, describing morphisms between bimodules of the form $B_{w_1} \ot B_{w_2} \ot \cdots \ot B_{w_d}$, where
% each $w_i$ is the longest element of a parabolic subgroup. This is developed for symmetric groups in \cite{EThick}, and for dihedral groups in \cite{ECathedral}. An expansion of the
% thick calculus for dihedral groups will be developed in the sequel to this paper. \end{remark}


%===================
\subsection{Braids and Rouquier complexes}
\label{subsec:Rouq}
%===================


Theorem \ref{thm:SC} says that the indecomposable bimodules $B_w$ categorify the KL basis. It is natural to wonder how to categorify the standard basis using Soergel's theory. Given that expressing the standard basis in terms of the KL basis requires signs (see \eqref{eq:KLinversion}), one should expect that the standard basis can be expressed using complexes of Soergel bimodules. The standard basis is the image of the braid group under its map to the Hecke algebra, so we will instead associate complexes of Soergel bimodules to braids.

Associated to an arbitrary Coxeter group $W$ we have the braid group $\Br(W)$, which is generated by invertible elements $\s_s$ for $s \in S$ called \emph{crossings}.  The Hecke algebra is a quotient of the group algebra $\ZM[v,v\inv] [\Br]$, and the quotient map sends $\s_s$ to $H_{s}$.

The definition below is due to Rouquier \cite{Rou04}.
% \footnote{Rouquier complexes for the symmetric group were well-known in other contexts \cite{}, though it was Rouquier \cite{Rou04} who first defined them using Soergel bimodules, and who generalized them to arbitrary Coxeter groups. Moreover, he first proved that they give a strict braid group action.}


\begin{definition}\label{def:Rouquier}
For each simple reflection $s\in W$, with corresponding braid generator $\s = \s_s$, define the following two complexes of Soergel bimodules, living in the bounded homotopy category $\KC^b(\SBim)$.
\[ F(\s) = \Big(0 \to \un{B_{s}}(0) \longrightarrow R(1) \to 0\Big),\]
\[ F(\s\inv) = \Big(0 \to R(-1) \longrightarrow \un{B_{s}}(0) \to 0 \Big). \]
We have underlined the terms in homological degree zero. If $\bbeta$ is a braid word (a word in the generators $\sigma_s^\pm$), then we define $F(\bbeta)$ to be the corresponding tensor product of complexes $F(\sigma_s^\pm)$. Such a complex $F(\bbeta)$ is called a \emph{Rouquier complex}.
\end{definition}

When $\bbeta$ and $\bbeta'$ are two braid words for the same element $\b$ of the braid group, Rouquier proved that $F(\bbeta)$ and $F(\bbeta')$ are homotopy equivalent, and in fact this equivalence is canonical.

\begin{remark} \label{rmk:dotsdefn} The differential in $F(\s)$ is the multiplication map $f \ot g \mapsto fg$, which is a bimodule map $B_s \to R(1)$. The differential in
$F(\s\inv)$ is the coproduct map $1 \mapsto \frac{1}{2}(\a_s \ot 1 + 1 \ot \a_s)$, which is a bimodule map $R \to B_s(1)$. These are the two ``dot'' maps in the diagrammatic calculus. \end{remark}

If $\AC$ is an idempotent-complete additive category and $\KC^b(\AC)$ is its bounded homotopy category, then the triangulated Grothendieck group of $\KC^b(\AC)$ is isomorphic to the split
Grothendick group of $\AC$, via an isomorphism which sends a complex $C^\bullet$ to its Euler characteristic
$\sum_{i}(-1)^i[C^i]$. Under the isomorphism $[\KC^b(\SBim)] \cong [\SBim] \cong \HB$, it is clear that $[F(\s)] = b_s - v = H_s$ and $[F(\s\inv)] = b_s - v^{-1} = H_s\inv$.  From this it follows that the class of any Rouquier complex $F(\bbeta)$ in the Grothendieck group agrees with the image of its braid $\b$ under the quotient map to the Hecke algebra.  

We conclude this section with the following fact (see \cite[Lemma 6.5]{EWHodge}).

\begin{lemma}\label{lem:BabsorbsF}
Let $s$ be a simple reflection and $x \in W$ an element such that $sx<x$. In the Hecke algebra, we have $H_s b_x = v^{-1} b_x$. In the categorification, we have
\begin{equation} \label{eq:sdown}
F(\s) B_x \simeq B_x(-1).
\end{equation}
\end{lemma}



%========================
\subsection{Canonicity of Rouquier complexes}
\label{subsec:RouqCanon}
%========================

Let us rephrase Rouquier's result from \cite{Rou04} on the canonical homotopy equivalence between Rouquier complexes for the same braid.

\begin{prop}(Rouquier canonicity)  For each pair $\bbeta$, $\bbeta'$ of braid words representing a braid $\b$, then there exists a homotopy equivalence $\phi_{\bbeta,\bbeta'} \co F(\bbeta) \to F(\bbeta')$.  These homotopy equivalences are transitive: $\phi_{\bbeta',\bbeta''} \circ \phi_{\bbeta,\bbeta'} \simeq \phi_{\bbeta,\bbeta''}$.  Moreover, if $\bbeta'$ and $\bbeta'$ represent the same braid and $\s$ is a crossing, then $\phi_{\bbeta \s,\bbeta' \s} \simeq \phi_{\bbeta,\bbeta'} \ot \Id_{F(\s)}$. \end{prop}


% \begin{remark} Rouquier's theorem can be rephrased as follows: assigning the functor $F(\b) \ot (-)$ to a braid word $\b$ defines a strict action of the braid group on $\KC^b(\SBim)$,
% categorifying the braid group action on $\HB$. For further discussion of strict actions of the braid group, see \cite{EWFenn}. \end{remark}

\begin{remark} An explicit description of these homotopy equivalences in type $A$ can be found in \cite{EKra}. There, a homotopy equivalence is defined for each
braid relation (and for the relation $\s \s^{-1} = 1$), and the movie moves (certain coherence relations) are checked between these homotopy equivalences. This gives another proof of
Rouquier's result in type $A$. \end{remark}

\begin{remark} Because $F = F(\bbeta)$ is invertible, $\Homgg(F,F) \cong \Homgg(\one,\one) \cong R$. Hence, the space of homotopy equivalences from $F(\bbeta)$ to $F(\bbeta')$ is one dimensional, and proving Rouquier canonicity amounts to keeping track of certain scalars.

It is a simple consequence of cell theory (see \S\ref{sec:cells}) that no shift of the identity bimodule occurs as a direct summand of $B_x\otimes B_y$ for nontrivial $x,y$, hence each Rouquier complex has a unique summand which is a shift of $R$. A homotopy equivalence $F(\bbeta) \to F(\bbeta')$ is homotopic to $\phi_{\bbeta,\bbeta'}$ if and only if it induces
the identity map on the underlying $R$ summands. \end{remark}

Henceforth, whenever we work in the homotopy category $\KC^b(\SBim_n)$, we write $F(\b)$ to represent $F(\bbeta)$ for any choice of braid word $\bbeta$ representing $\b$, the choice
being irrelevant up to canonical isomorphism.

We will use Rouquier canonicity to prove certain statements about tensor-commutativity. Suppose that the braids $\b$ and $\g$ commute. Then $F(\b) F(\g) F(\b\inv) \simeq F(\g)$ because
they represent the same braid. When identifying these complexes, we always use the canonical homotopy equivalence. If both $\b_1$ and $\b_2$ commute with $\g$, then the composition
\begin{equation} \label{eq:conjblah1} F(\g) \to F(\b_2) F(\g) F(\b_2\inv) \to F(\b_1) F(\b_2) F(\g) F(\b_2\inv) F(\b_1\inv) \to F(\b_1 \b_2) F(\g) F((\b_1 \b_2)\inv)\end{equation} agrees
with the map \begin{equation} \label{eq:conjblah2} F(\g) \to F(\b_1 \b_2) F(\g) F((\b_1\b_2)\inv) \end{equation} up to homotopy. This is because every arrow is the canonical homotopy
equivalence. 

For future reference, we record this in a loosely-worded lemma.

\begin{lemma} \label{lem:centerofbraidgroup} If $\g$ is in the center of the braid group, then $F(\g)$ commutes with $F(\b)$ for any braid, compatibly with the composition of braids. \end{lemma}
	
%========================
\subsection{Trivialities about the polynomial action}
\label{subsec:trivial}
%========================

Let $C \in \KC^b(\SBim_n)$ be an arbitrary complex. Then $\Homgg(\one,C)$ is naturally a bigraded $R$-bimodule, but it is one on which the right and left actions of $R$ agree. More precisely, observe that $R=\Endgg(\one)$.  For each $f\in R$, the following morphisms are all equal for any $\a \in \Homgg(\one,C)$.
\begin{equation} \label{eq:silly} f \cdot \a = \a \cdot f = \a \circ f = \a \ot f = f \ot \a. \end{equation}
The first two terms are the left and right action. The next term is composition. The last two terms implicitly use the isomorphisms \[\Homgg(\one, C) \cong \Homgg(\one \ot \one, \one \ot C) \cong \Homgg(\one \ot \one, C \ot \one).\]

%========================
\subsection{Braid conjugation acting on morphisms}
\label{subsec:conjugate}
%========================

Given a braid $\b$, we let $\Psi_\b \co \KC^b(\SBim) \to \KC^b(\SBim)$ denote the functor which, on objects, sends $M \mapsto F(\b) \ot M \ot F(\b)^{-1}$, and on morphisms, sends $f
\mapsto 1 \ot f \ot 1$. We refer to this functor as \emph{conjugation} by the Rouquier complex of a braid. It is an invertible functor, with inverse $\Psi_{\b^{-1}}$.

Let $\g$ be a braid that commutes with $\b$. Then, as noted in \S\ref{subsec:RouqCanon}, there is a canonical homotopy equivalence $F(\g) \to \Psi_{\b}(F(\g))$, which we temporarily denote $\rho_{\g,\b}$. If $\g$ and $\g'$ are both braids which commute with $\b$, and $f \in \Homg(F(\g),F(\g'))$, we define $\psi_\b(f) \in \Homg(F(\g),F(\g'))$ as the composition
\begin{equation} F(\g) \buildrel {\rho_{\g,\b}} \over \longrightarrow F(\b) \ot F(\g) \ot F(\b\inv) \buildrel {1 \ot f \ot 1} \over \longrightarrow F(\b) \ot F(\g') \ot F(\b\inv) \buildrel {\rho_{\g',\b}\inv} \over \longrightarrow F(\g'). \end{equation}
We refer to $\psi_\b$ acting on $\Homg(F(\g),F(\g'))$ as the \emph{conjugation action of braids on morphisms}.

It is clear that \begin{equation} \label{eq:conjfunctorial} \psi_\b(f \circ g) = \psi_\b(f) \circ \psi_\b(g)\end{equation} for morphisms $f$ and $g$ where this makes sense. It is also straightforward that, for two braids $\b_1$ and $\b_2$ which both commute with $\g$ and $\g'$, one has
one has \begin{equation} \label{eq:conjisaction} \psi_{\b_1 \b_2}(f) \simeq \psi_{\b_1}(\psi_{\b_2}(f)). \end{equation}
This follows from the equality of \eqref{eq:conjblah1} and \eqref{eq:conjblah2}.

The following lemma is not immediately obvious, and describes how multiplication by polynomials interacts with the conjugation action.

\begin{lemma} \label{lem:howtoconj} Let $f \in R$ be a homogeneous polynomial, $\b$ and $\g$ braids with $\g$ in the center of the braid group, and $\a \in \Homg(\one,F(\g))$. Then the following two maps are homotopic:
\begin{equation}\label{eq:conjugation1}
\psi_\b(f \cdot \a) \simeq w(f) \cdot \psi_\b(\a),
\end{equation}
where $w \in W$ is the image of $\b$ under the standard map from the braid group to $W$. \end{lemma}

\begin{proof} Recall from \S\ref{subsec:trivial} that $f \cdot \a$ agrees with $\a \circ f$, viewing $f$ as a chain map in $\Homg(\one,\one)$. Then, by functoriality
\eqref{eq:conjfunctorial}, it is enough to prove the result for $\g = 1$, and $\a$ the identity map. By \eqref{eq:conjisaction}, it is enough to prove this result when $\b$ is either
$\s$ or $\s\inv$ for a simple reflection $s$. Thus it is enough to show that $s(f) \cdot \rho_{1,\s}$ is homotopic to $(1 \ot f \ot 1) \circ \rho_{1,\s}$, and a similar statement
for $\s\inv$.

This is a simple computation, which we leave as an exercise\footnote{Excessive hints for this exercise: The homotopy equivalence $\rho_\s$ can be found in diagrammatic language on
\cite[page 18, move 1a]{EKra}. The homotopy between these two maps is the map $R \to B_s$ obtained by multiplying by $\pa_s(f)$, and then applying the dot. This computation then follows
quickly from \cite[Equation 5.2]{EWsoergelCalc}. As a reminder, $\pa_s(f)$ is the Demazure operator applied to $f$.} for any reader versed in the diagrammatic notation. \end{proof}

We use this lemma in \S\ref{subsec:lambdaEquivsufficient}.

%========================
\subsection{Minimal complexes}
\label{subsec:minimal}
%========================

This section is an aside on homological algebra. We return to the applications to Soergel bimodules in the next section.  Many statements regarding chain complexes are greatly simplified by the existence of minimal complexes, which are nice representatives of complexes up to homotopy equivalence.  In turn, minimal complexes mostly make sense in the setting of Krull-Schmidt categories.


Recall that an idempotent-complete additive category is called \emph{Krull-Schmidt} if each object is a direct sum of finitely many indecomposables, and the endomorphism ring of every indecomposable object is a local  ring. For example, finitely generated abelian groups is a category which has unique decompositions, but is not Krull-Schmidt.


If $\AC$ is an additive category, we let $\JC(\AC)$ denote the set of morphisms which are not isomorphisms.  If $\AC$ is Krull-Schmidt, then $\JC(\AC)$ is an ideal (i.e. it is closed under composition with arbitrary morphisms).

\begin{definition}\label{def:minimalCx}
Let $\AC$ be an additive category.  A complex $D\in \KC(\AC)$ is called \emph{minimal} if the differential $d:D^i\rightarrow D^{i+1}$ lies in $\JC(\AC)$ for all $i$.
\end{definition}

When two indecomposable objects $X$ appear as summands of chain objects in adjacent homological degrees, and the part of the differential between them is an isomorphism, this pair can be
cancelled. There is a process known as Gaussian elimination of complexes \cite{DBNfast} which produces a new complex, whose chain objects agree with the original complex except with the
two copies of $X$ removed (by taking a complementary direct summand), and which is homotopy equivalent to the original complex. We think of Gaussian elimination as a ``deformation
retract'' of complexes. Repeating this process one obtains a complex where no summand of any differential is an isomorphism.

\begin{proposition}\label{prop:minimalCx1}
If $\AC$ is an idempotent complete additive category, then every $D\in \KC^b(\AC)$ deformation retracts onto a minimal complex $D_{\min}$. \qed
\end{proposition}

\begin{remark}
This can be extended to unbounded complexes $D\in \KC(\AC)$, provided that each chain object of $D$ is a finite direct sum of indecomposables.
\end{remark}

Minimal complexes are most nicely behaved when $\AC$ is Krull-Schmidt. Since the differential is in $\JC$, and $\JC$ is an ideal, then any nulhomotopic chain map is also in $\JC$.

\begin{lemma}\label{lem:minimalCxIso}
If $\AC$ is Krull-Schmidt and $D_1,D_2\in \KC(\AC)$ are minimal, then any homotopy equivalence $\phi:D_1\rightarrow D_2$ is an isomorphism of complexes.
\end{lemma}

\begin{corollary}\label{cor:minimalCxUnique}
If $\AC$ is Krull-Schmidt, then the minimal complex $D_{\min}$ from Proposition \ref{prop:minimalCx1} is unique up to isomorphism (not merely homotopy equivalence). \qed
\end{corollary}

This discussion applies, mutatis mutandis, to the graded context. We leave the reader to look up or guess the definitions of the graded Jacobson radical $\JC(\AC)$, a graded Krull-Schmidt category, etcetera.

%========================
\subsection{Minimal complexes, perversity, the half twist}
\label{subsec:minimalRouq}
%========================

\begin{definition}\label{def:rouquierMinCx}
For each $w\in W$, let $F_w$ denote the minimal complex of the Rouquier complex of a positive braid lift of $w$.
\end{definition}

By Rouquier canonicity and Corollary \ref{cor:minimalCxUnique}, $F_w$ depends (up to unique isomorphism) only on $w$, not on the choice of reduced expression giving the positive braid lift.

%\begin{notation}When we refer to a \emph{term} of a complex, we refer to some indecomposable summand in some homological degree. To avoid discussing extraneous summands,  we implicitly assume our complex is in minimal form, unless otherwise stated. \end{notation}
	

\begin{defn}\label{def:perverse}  A complex $D \in \KC(\SBim)$ is \emph{perverse}\footnote{Sometimes also called \emph{linear} or \emph{diagonal}.} if the chain bimodule $D^k$ is a direct sum of bimodules of the form $B_w(k)$.  \end{defn}
There can be no nonzero homotopies between perverse complexes, because $\SBim_n$ is mixed, see Proposition \ref{prop:soergelmixed}.  A perverse complex is necessarily minimal.  

Observe that, for each $s \in S$, the complex $F_s = F(\s)$ is perverse (as is $F_s\inv = F(\s\inv)$). The following crucial result was proven in \cite[Theorem 6.9]{EWHodge}.

\begin{thm}\label{thm:diagonalmiracle}
The minimal complexes $F_w$ are perverse, for all $w \in W$.
\end{thm}


As noted previously, an object $A$ of an additive category $\AC$ is pinned down uniquely up to isomorphism by its class in the Grothendieck group (if $\AC$ is Krull-Schmidt).  However if $C\in \KC^b(\AC)$ is a complex, then the class of $[C]\in K_0(\AC)$ certainly does not determine the chain groups $C^k$ (for a counterexample take $C=B\oplus B[1]$ which always has zero Euler characteristic).  However, knowing the symbol of a bounded perverse complex $C$ in a Krull-Schmidt category does determine the chain objects uniquely (but not the differential)! 

Combining Theorem \ref{thm:diagonalmiracle} and the Kazhdan-Lusztig inversion formula \eqref{eq:KLinversion} gives a description of the chain bimodules of $F_w$ for every $w\in W$.  Namely, each occurence of $(-v)^i b_x$ in the right-hand side of \eqref{eq:KLinversion} contributes a summand $B_x(i)$ in homological degree $i$.   In particular, \eqref{eq:KLhalftwist} implies that the KL polynomials $h_{1,x}$ determine the summands appearing in $F_{w_0}$.



\begin{defn}\label{def:HTandFT} If $W$ is a finite Coxeter group with longest element $w_0$, let $\HT$ denote $F_{w_0}$, and let $\FT=\HT \ot \HT$. We call $\HT$ the \emph{half-twist},
and $\FT$ the \emph{full-twist}. In type $A_{n-1}$, we often write the half-twist as $\HT_n$ and the full-twist as $\FT_n$. We may also use the terms half-twist and full-twist to refer
to the corresponding elements of the braid group, or their images in the Hecke algebra; the meaning will be clear from context. \end{defn}




\begin{example} Let $s$ and $t$ be the simple reflections of $S_3$. Then \begin{equation} \label{eq:Hsts} H_{sts} = b_{sts} - v(b_{st}+b_{ts}) + v^2(b_s+b_t) - v^3.\end{equation} Thus the Rouquier complex $F_{sts} = \HT_3$ has the form
\[ F_{sts} = \Big( \un{B_{sts}(0)} \longrightarrow B_{st}(1) \oplus B_{ts}(1) \longrightarrow B_s(2) \oplus B_t(2) \longrightarrow R(3) \Big). \]
We derive \eqref{eq:Hsts} above. Every element of $S_3$ is smooth, meaning that $h_{x,y} = v^{\ell(y)-\ell(x)}$. Thus by Corollary \ref{cor:KLhalftwist} the coefficient of $b_x$ in $H_{sts}$ is $(-v)^{\ell(w_0) - \ell(x)}$. \end{example}


\begin{ex} Let $\{s,t,u\}$ be the simple reflections of $S_4$. For all but two elements of $S_4$, one has $h_{1,x} = v^{\ell(x)}$. However, one has $h_{1,tsut} = {\color{red} v^2} + v^4$ and $h_{1,sutsu} = {\color{red} v^3} + v^5$. Note that $tsut = w_0(su)$ and $sutsu = w_0(t)$. Thus the Rouquier complex $F_{w_0} = \HT_4$ has the following form.
\begin{equation} \label{eq:Fstsuts} \HT_4 \simeq 
\begin{tikzpicture}[baseline=-.2em]
\tikzstyle{every node}=[font=\scriptsize]
\node at (0,0) {$\underline{B_{stsuts}(0)}$};
\node (y) at (.65,0) {};
%
\node (z) at (1.5,0) {};
\node at (2,.5) {$B_{tstut}(1)$};
\node at (2,0) {$B_{tutst}(1)$};
\node at (2,-.5) {$B_{sutsu}(1)$};
\node (a) at (2.5,0) {};
%
\node (b) at (3.5,0) {};
\node at (4,1.25) {$B_{tstu}(2)$};
\node at (4,.75) {$B_{utst}(2)$};
\node at (4,.25) {$B_{tuts}(2)$};
\node at (4,-.25) {$B_{stut}(2)$};
\node at (4,-.75) {$B_{tsut}(2)$};
\node at (4,-1.25) {${\color{red} B_{su}(2)}$};
\node (c) at (4.5,0) {};
%
\node (d) at (5.5,0) {};
\node at (6,1.5) {${\color{red} B_{t}(3)}$};
\node at (6,1) {$B_{stu}(3)$};
\node at (6,.5) {$B_{uts}(3)$};
\node at (6,0) {$B_{tst}(3)$};
\node at (6,-.5) {$B_{tut}(3)$};
\node at (6,-1) {$B_{tsu}(3)$};
\node at (6,-1.5) {$B_{sut}(3)$};
\node (e) at (6.5,0) {};
%
\node (f) at (7.5,0) {};
\node at (8,1) {$B_{st}(4)$};
\node at (8,.5) {$B_{ts}(4)$};
\node at (8,0) {$B_{ut}(4)$};
\node at (8,-.5) {$B_{tu}(4)$};
\node at (8,-1) {$B_{su}(4)$};
\node (g) at (8.5,0) {};
%
\node (h) at (9.5,0) {};
\node at (10,.5) {$B_{s}(5)$};
\node at (10,0) {$B_{t}(5)$};
\node  at (10,-.5) {$B_{u}(5)$};
\node (i) at (10.5,0) {};
%
\node (j) at (11.5,0) {};
\node at (12,0) {$B_1(6)$};
\path[->,>=stealth',shorten >=1pt,auto,node distance=1.8cm,
  thick]
(y) edge node[above] {} (z)
(a) edge node[above] {} (b)
(c) edge node[above] {} (d)
(e) edge node[above] {} (f)
(g) edge node[above] {} (h)
(i) edge node[above] {} (j);
\end{tikzpicture}
\end{equation}
The nontrivial Kazhdan-Lusztig polynomials mentioned above give rise to the ``additional" terms $B_{su}(2)$ and $B_t(3)$ appearing above. \end{ex}
































% \subsection{Idempotents in $\Hecke_n$}
%
% In this section we give an inductive construction of a complete collection of primitive idempotents $p_T\in \Hecke_n$, indexed by standard Young tableaux.
%
% \begin{construction}
% Assume that we have constructed elements $p_T\in \Hecke_n$, indexed by standard Young tableaux on $n$ boxes, such that
% \begin{enumerate}\setlength{\itemsep}{3pt}
% \item $p_Tp_U =0$ for $T\neq U$, and $p_T^2=p_T$.
% \item $\sum_T p_T = 1_n$.
% \item $\Hecke_np_T$ is isomorphic to the Specht module $\SM^\l$ corresponding to the partition $\l=\sh(T)$, as left $\Hecke_n$-modules.
% \end{enumerate}
% Via the inclusion $\Hecke_n\rightarrow \Hecke_{n+1}$, we obtain a collection of elements $p_T\in \Hecke_{n+1}$ satisfying
% \begin{enumerate}\setlength{\itemsep}{3pt}
% \item $p_Tp_U =0$ for $T\neq U$, and $p_T^2=p_T$.
% \item $\sum_T p_T = 1_{n+1}$.
% \item $\Hecke_{n+1}p_T$ is isomorphic to the induced module $\Ind_{\Hecke_n}^{\Hecke_{n+1}}(\SM^\l)$, as left $\Hecke_{n+1}$-modules.
% \end{enumerate}
%
% Now, the branching rule for $\Hecke_{n+1}$ states that $\Ind_{\Hecke_n}^{\Hecke_{n+1}}(\SM^\l)$ is isomorphic to a direct sum of $\SM^{\l'}$, where $\l'$ ranges over all partitions of $n+1$ obtained from $\l$ by adding a box.  Note that the endomorphism ring of $\Hecke_{n+1}p_T$ is canonically isomorphic to $p_T\Hecke_{n+1}p_T$.  Thus, the idempotents that project $Hecke_{n+1}p_T$ onto its irreducible components correspond to certain idempotent elements $p_{T'}\subset p_T\Hecke_{n+1}p_T$, where $T\in \SYT(n)$ is obtained from $T'\in \SYT(n+1)$ by deleting the box labelled $n+1$.  Since branching rule is multiplicity free, the idempotents $p_T'$ are uniquely characterized by
%
%
%
%
% \end{construction}





