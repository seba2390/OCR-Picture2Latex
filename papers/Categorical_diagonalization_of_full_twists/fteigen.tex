%%%%%%%%%%%%%%%%%%%%%%%
\section{The asymptotic Hecke algebra and the eigenvalues of the full twists}
\label{sec:cellsplus}
%%%%%%%%%%%%%%%%%%%%%%%

A general reference for Hecke algebras and their cell theory is Lusztig's book on Hecke algebras with unequal parameters \cite{LuszUnequal14}, which has been updated on the
arXiv to account for the results of \cite{EWHodge}.

Lusztig has developed a beautiful theory for studying Hecke algebras in terms of their ``asymptotic'' behavior. When the parameter $v$ of the Hecke algebra $\HB$ tends towards $0$, only the smallest power of $v$ in a polynomial should survive. By examining the smallest powers of $v$ which appear in certain key coefficients, Lusztig extracts interesting numerical information. The relationship between this asymptotic behavior and the cells of $\HB$ is encapsulated in Lusztig's famous conjectures P1-P15, see \cite[Section 14]{LuszUnequal14}. Lusztig proved (see \cite[Section 15]{LuszUnequal14}) that when the structure coefficients $c^z_{x,y}$ (to be defined below) and the Kazhdan-Lusztig polynomials $h_{y,w}$ are always positive (that is, in $\NM[v,v^{\inv}]$) then conjectures P1-P15 are true. For Weyl groups (such as the symmetric group), this positivity was known since the proof of the Kazhdan-Lusztig conjectures. For arbitrary Coxeter groups, this positivity is also known thanks to \cite{EWHodge}.\footnote{Conjectures P1-P15 were also made for Hecke algebras with unequal parameters, where they remain open in general.}

Again, the reader new to this theory should read \S\ref{sec:typeAcells} concurrently, to see these results in the more familiar situation of type $A$. In type $A$, the two-sided cells of
$\HB(S_n)$ are in bijection with partitions of $n$.

Let us note a major notational change with the literature. What is commonly called Lusztig's $\ab$-function we will be denoting by $\rbb$, and calling the $\rbb$-function. In type $A$,
$\rbb$ corresponds to the row function of a partition. We will eventually introduce other functions $\cbb$ and $\xbb$ derived from $\rbb$, which correspond in type $A$ to the column and
content functions of partitions. To keep notation consistent with type $A$, we have departed from tradition and used $\rbb$ for the general case.


%========================
\subsection{The $\rbb$-function}
\label{subsec:afunc}
%========================

Henceforth, we will be studying the cell theory (as defined in \S\ref{subsec:algCells}) of the KL basis of $\HB(W)$. Thus, one has three transitive relations $\le_{L}$, $\le_{R}$, and $\le_{LR}$ on
the set $W$, whose equivalence classes are called \emph{left}, \emph{right}, and \emph{two-sided cells} respectively. We write $x \sim_L y$ if $x \le_L y$ and $y \le_L x$, and so forth.
Unless otherwise specified, a \emph{cell} will refer to a two-sided cell.

\begin{lemma} For $x, y \in W$ one has $x \le_L y$ (resp. $x \le_{LR} y$) if and only if $x^{-1} \le_R y^{-1}$ (resp. $x^{-1} \le_{LR} y^{-1}$). \end{lemma}

\begin{proof} This follows immediately from the existence of an algebra antiautomorphism on $\HB$ which sends $H_x\mapsto H_{x\inv}$ and $b_x\mapsto b_{x^{-1}}$. \end{proof}
	
We now recall Lusztig's $\rbb$-function $W \to \NM$, see \cite[Section 13]{LuszUnequal14}.

\begin{defn}\label{def:a} For $x,y,z\in W$, let $c^z_{x,y} \in \Z[v,v^{\inv}]$ be the structure coefficients\footnote{Denoted $h_{x,y,z}$ in \cite[Section 13]{LuszUnequal14}.} of the
Kazhdan-Lusztig basis, that is, $$b_x b_y=\sum_{z}c^z_{x,y} b_z.$$ Note that the polynomials $c^z_{x,y}$ are preserved by swapping $v$ and $v^{\inv}$.

Define $\rbb(z) \in \NM$ such that $v^{-\rbb(z)}$ is the smallest power of $v$ (and $v^{+\rbb(z)}$ the largest power) appearing in $c^z_{x,y}$ for all $x,y \in W$. Let $t^z_{x,y}$\footnote{Denoted $\gamma_{x,y,z^{-1}}$ in \cite[Section 13]{LuszUnequal14}.} denote the coefficient of $v^{-\rbb(z)}$ inside $c^z_{x,y}$.  \end{defn}

The following facts were proven by Lusztig, see \cite[Section 14]{LuszUnequal14}. Note that $t^z_{x,y}$ is zero unless $x$ and $y$ succeed in minimizing the power of $v$ which appears in $c^z_{x,y}$.



\begin{prop} \label{prop:someP} The following properties hold. \begin{itemize}
\item (P14) For all $w$, $w \sim_{LR} w^{-1}$.
\item (P8) If $t^z_{x,y} \ne 0$, then $x \sim_R z$, $y \sim_L z$, and $x \sim_L y\inv$. In particular, they are all in the same two-sided cell.
\item (P7) For any $x, y, z$, $t^z_{x,y} = t^{x\inv}_{y,z\inv}$.
\item (P4) If $x \le_{LR} y$ then $\rbb(x) \ge \rbb(y)$. In particular, $\rbb(x) = \rbb(\l)$ only depends on the two-sided cell $\l$ of $x$.
\item (P11) If $\l \ne \mu$ are two-sided cells with $\l < \mu$, then $\rbb(\l) > \rbb(\mu)$.
\item (P9-10) If $x \le_L y$ and $\rbb(x) = \rbb(y)$ then $x \sim_L y$. Consequently, two comparable left cells in the same two-sided cell are equal. The same holds, replacing left with right.
\end{itemize} \end{prop}

Because of (P4), we will mostly think of $\rbb$ as a function on two-sided cells, rather than a function on $W$. Because of (P8), one can reinterpret the function $\rbb$ as follows: for a
two-sided cell $\l$, $v^{-\rbb(\l)}$ is the minimal power of $v$ appearing in $c^z_{x,y}$ for any $x, y, z$ in cell $\l$.

\begin{example} \label{ex:boundviolated}
Let $\{s,t,u\}$ be the simple reflections of $S_4$. For this example we focus on $x = tsut$ and $w = sutsu$, which happen to be the two involutions in $S_4$ which are not longest elements of parabolic subgroups, and also happen to be the only non-smooth elements.

The element $x$ lies in cell $\mu = (2,2)$, and it turns out that $\rbb(\mu)=2$. One has \begin{equation} \label{eq:x2} b_x b_x = [2]^2 b_x + [2]^2 b_{w_0} + b_{tst} + b_{tut} + b_{tstut}
+ b_{tutst}.\end{equation} Aside from $[2]^2 b_x$, all the other terms lie in lower cells. Thus $t^x_{x,x} = 1$, while $c^y_{x,x} = 0$ for other elements $y$ in this cell.

The element $w$ lies in the cell $\l = (2,1,1)$, and $\rbb(\l) = 3$. One has \begin{equation} \label{eq:w2} b_w b_w = [2]^3 b_w + [2]^4 b_{w_0}.\end{equation} The term $b_{w_0}$ lies in a
lower cell. Thus $t^w_{x,x} = 1$. Note that $c^{w_0}_{w,w}$ has minimal power $v^{-4}$, illustrating that $-\rbb(\l)$ is not the minimal power of $v$ appearing in $c^z_{x,y}$ for $x, y$
in cell $\l$, but also requires $z$ in $\l$. %It turns out that $\rbb(w_0) = 6$
\end{example}

To emphasize a takeaway from this example, it is important to remember that the bound $-\rbb(\l)$ on the $v$-degrees of summands is only valid within the same cell, and that lower cells
may also reach or even exceed this bound.

The \emph{asympotic Hecke algebra} or \emph{J-ring} is morally obtained by multiplying KL basis elements, but only remembering the minimal $v$-degree which ``should" appear.

\begin{prop} \label{prop:Jassoc} Fix a two-sided cell $\l$. Let $J_{\l}$ be the algebra with basis $\{j_x\}_{x \in \l}$ and with multiplication $j_x j_y = \sum_{z \in \l} t^z_{x,y} j_z$. Then $J_\l$ is an associative algebra (possibly without unit when $W$ is infinite, see Remark \ref{rmk:Junit}). \end{prop}

\begin{remark} One could also define the ring $J$ with a basis $\{j_x\}_{x \in W}$ and multiplication as above. Because of (P8), this would just be the product of the rings $J_\l$ for each cell. \end{remark}

\begin{ex} Consider $S_3$ with simple reflections $\{s,t\}$, and its simple cell $\l$ consisting of $\{s,t, st, ts\}$. Then $\rbb(\l)=1$, as can be seen from the products $b_s b_s =
(v+v\inv)b_s$ and $b_s b_t = b_{st}$ (and similarly, swapping $s$ and $t$). The equation $b_sb_s=(v+v\inv)b_s$ implies that in the $J$ ring $j_s j_s = j_s$, while $b_s b_t=b_{st}$
implies $j_s j_t=0$. In particular $j_s$ and $j_t$ are orthogonal idempotents in the $J$-ring. The elements $b_{st}$ and $b_{ts}$ satisfy. \[ b_{st}b_{ts}=b_sb_tb_tb_s =
(v+v\inv)b_sb_tb_s=(v+v\inv)(b_{sts}+b_s). \] This implies that $j_{st} j_{ts} = j_s$. Hence the ring $J_\l$ is a $2\times2$ matrix algebra. In fact, for any cell $\l$ in type $A$, the
$J$-ring is a matrix algebra, though this does not generalize to arbitrary Coxeter groups. \end{ex}

%========================
\subsection{The $\Delta$-function}
\label{subsec:Dfunc}
%========================

Now we recall Lusztig's Delta function, a function $W \to \NM$ which does not only depend on the cell.

\begin{defn}\label{def:D} For $x \in W$, define $\D(x) \in \NM$ such that $v^{\D(x)}$ is the smallest power of $v$ appearing in the Kazhdan-Lusztig polynomial $h_{1,x} \in \ZM[v]$.
\end{defn}	

Note that $\D(x) = \D(x\inv)$, because $h_{1,x} = h_{1\inv,x\inv} = h_{1,x\inv}$.

In the next section, we discuss the connection between $\D$ and the half twist. The first use of $\D(x)$ is to determine, by comparison with $\rbb(x)$, a special set of elements of $W$. Here are more facts proven by Lusztig, see \cite[Section 14]{LuszUnequal14}.

\begin{defn}\label{def:duflo}
Let $\DC\subset W$ denote the set of elements $x$ such that $\Delta(x)=\rb(x)$.  An element of $\DC$ is called a \emph{distinguished involution} or \emph{Duflo involution}.
\end{defn}

The proposition below implies that distinguished involutions are in fact involutions.  We often use the letter `$d$' to denote distinguished involutions.

\begin{prop}\label{prop:moreP} The following properties hold. \begin{itemize}
\item (P1) For all $x \in W$, $\D(x) \ge \rbb(x)$.
\item (P13,P6) If $d\in \DC$ then $d$ is an involution.  Each left cell (resp. right cell) contains a unique element of $\DC$. % $d$ such that $\D(d)=\rbb(d)$; furthermore, this element is an involution. The set of elements $d$ with $\D(d)=\rbb(d)$ is denoted $\DC$ and called the \emph{distinguished involutions} or \emph{Duflo involutions}.
\item (P5 and positivity) If $d \in \DC$ then the coefficient of $v^{\D(d)}$ in $h_{1,d}$ is $1$.
\item (P3,P13,P5) If $d \in \DC$ then $t^d_{x,y} \ne 0$ if and only if $x = y\inv$ and $y \sim_L d$. Moreover, in this case $t^d_{y\inv,y} = 1$.
\end{itemize}
\end{prop}

In type $A$ each left cell contains a unique involution, so it must be distinguished, and thus $\DC$ is the set of all involutions.

\begin{ex} In any dihedral group with simple reflections $\{s,t\}$, there are three cells: the identity, longest, and simple cells, as in Example \ref{ex:threecells}. Within the simple
cell there are two left cells, one containing those elements whose unique reduced expression ends in $s$, and the other those ending in $t$. The simple cell $\l$ satisfies $\rbb(\l)=1$.
All elements are smooth, thus $\D(x) = \ell(x)$. Hence the simple cell contains two distinguished involutions of length $1$, namely $s$ and $t$. Outside of the special case of type
$A_2$, both of these left cells contain multiple involutions, but only $s$ and $t$ are distinguished. \end{ex}

We state a useful corollary which gives a practical way to compute $\rbb(\l)$.

\begin{cor}\label{cor:computinga}
If $\l$ is a two-sided cell in $W$, then $\rbb(\l)$ can be characterized as the minimal power of $v$ occuring in $c^d_{d,d}$, for any distinguished involution in $\DC \cap \l$.
\end{cor}


% the bullshit corollary
% \begin{cor}
% The distinguished involutions can be characterized as those involutions $x\in W$ such that the coefficient of $b_x$ in $b_x^2$ achieves the minimum (equivalently maximum) possible exponent, ranging over all elements in a given two-sided cell.
% \end{cor}
% This is a much more useful characterization than the original definition\footnote{Though we emphasize that this uses the result of Proposition \ref{prop:involutionsAndA}, hence may be false for Hecke algebras with unqual parameters}.


\begin{ex} Note that distinguished involutions are not the only involutions for which $t^d_{d,d} \ne 0$. For example, consider a dihedral group with simple reflections $\{s,t\}$ with $m_{st} \ge 6$. Then $j_{sts} j_{sts} = j_s + j_{sts} + j_{ststs}$. \end{ex}


%========================
\subsection{The action of distinguished involutions}
\label{subsec:invact}
%========================

Combining the results above, Lusztig proves the following.

\begin{prop} \label{prop:dacts} If $d \in \DC$ and $x \in W$ satisfies $x \sim_{LR} d$, then $t^y_{d,x} = 0$ for all $y$ unless $x \sim_R d$, in which case $t^x_{d,x} = 1$ and
$t^y_{d,x} = 0$ for $y \ne x$. In other words, left multiplication by $j_d$ in the $J$-ring is the same as projection to those $j_x$ for which $x$ in the same right cell as $d$.
Similarly, right multiplication by $j_d$ is projection to the left cell of $d$. \end{prop}

\begin{proof} This is proven by combining (P7) and (P8) from Proposition \ref{prop:someP} with the last property from Proposition \ref{prop:moreP}. \end{proof}
	
\begin{remark} \label{rmk:Junit} It follows that if a two-sided cell $\l$ contains finitely many left cells, then the finite sum $\sum_{d \in \DC \cap \l} j_d$ is the identity element of $J_{\l}$. Even if
there are infinitely many left cells, $J_{\l}$ is locally unital for the orthogonal idempotents $\{j_d\}_{d \in \DC}$. \end{remark}	

A restatement of this proposition says that for $x$ and $d$ in cell $\l$, with $d$ a distinguished involution, we have
\begin{equation} \label{eq:dacts} v^{\rbb(d)} b_d b_x \equiv \d_{x \sim_R d} b_x + \HB^+_{\l} + \HB_{< \l}, \end{equation}
where $\HB_{< \l}$ denotes the ideal spanned by $b_y$ for $y$ in a strictly smaller cell than $\l$, and $\HB^+_\l$ denotes those linear combinations of $b_y$ for $y$ in cell $\l$ whose coefficients live in $v\Z[v]$.

A categorical analogue of this proposition is the following.

\begin{cor}\label{cor:weakJringCat}
Let $d\in W$ be a distinguished involution in cell $\l$. If $x \sim_R d$ then $B_x(0)$ is a direct summand of $B_d(\rbb(\l)) \ot B_x$ with multiplicity one, and all other summands in cell $\l$ have strictly positive grading shifts. If $x \sim_{LR} d$ but $x \nsim_R d$ then every summand of $B_d(\rbb(\l)) \ot B_x$ has a strictly positive grading shift. \end{cor}

\begin{proof} This follows directly from the decategorified statement. \end{proof}


%========================
\subsection{First consequences for the half twist}
\label{subsec:halftwistconsequences}
%========================

The following is a result of Lusztig, see \cite[Corollary 11.7]{LuszUnequal14}.

\begin{prop} \label{prop:w0oncells} Let $W$ be a finite Coxeter group with longest element $w_0$. Then $x \le_L y$ if and only if $w_0 x \ge_L w_0 y$ if and only if $x w_0 \ge_L y w_0$.
Similar statements hold replacing $L$ with $R$ or $LR$. Thus, multiplication by $w_0$ induces an order-reversing involution on the set of left cells, right cells, and two-sided cells.
\end{prop}

\begin{lemma} Given a two-sided cell $\l$, the two-sided cells $w_0 \l$ and $\l w_0$ are equal.\footnote{This statement is well-known, and was known to Lusztig in the early days of cell theory. We could not find an early reference in the literature, however. The proof below was told to the first author by Victor Ostrik.} Equivalently, two-sided cells are preserved by conjugation by the longest element. Moreover $w_0 \DC = \DC w_0$ as sets, or equivalently, $w_0 d w_0$ is distinguished whenever $d$ is. \end{lemma}

\begin{proof} A two-sided cell gives rise to a left-module of $\HB$, by taking the associated graded in the cell filtration of $\HB$. Standard results in the representation theory of Hecke algebras show that distinct two-sided cells give rise to non-isomorphic representations of $\HB$. Because of Lemma \ref{lem:ftIsCentral_decat}, twisting under conjugation by $H_{w_0}$ will send the module associated to $w_0 \l$ to the module associated to $\l w_0$. However, conjugation is an inner automorphism, so it preserves the isomorphism class of a left module. Thus $w_0 \l$ and $\l w_0$ are the same two-sided cell.\footnote{Note that distinct left cells may give rise to isomorphic left modules, so this proof does not show that conjugation by $w_0$ fixes left cells. In fact, it does not.}

The automorphism $\tau$ sends $H_x$ to $H_{w_0 x w_0}$, and similarly sends $b_x$ to $b_{w_0 x w_0}$. Thus $\tau$ preserves KL polynomials, structure coefficients, etcetera. It follows that $\Delta(\tau(x)) = \Delta(x)$ and $\rbb(\tau(x)) = \rbb(x)$. Thus $w_0 d w_0$ must be distinguished whenever $d$ is distinguished. \end{proof}

%It follows that $x\mapsto w_0xw_0=:\tau(x)$ is an order preserving involution on the set of left, right, and two-sided cells.

\begin{defn} \label{defn:c} For a two-sided cell $\l$, let $\l^t$ denote the two-sided cell $w_0 \l$. Let $\cbb(\l)$ denote $\rbb(\l^t)$. For any distinguished involution $d \in \DC$, we refer to $w_0 d$ as a \emph{$w_0$-twisted distinguished involution}\footnote{It need not be an involution.}. \end{defn}

\begin{lemma}\label{lemma:cineq}
If $\l<\mu$ then $\cbb(\l)<\cbb(\mu)$.
\end{lemma}
\begin{proof}
Follows from Proposition \ref{prop:w0oncells} and Proposition \ref{prop:someP} (P11).
\end{proof}
	

Here is a crucial consequence of this result for the half twist.

\begin{prop} \label{prop:htm} Let $\l$ be a two-sided cell. In the language of \S\ref{subsec:twists}, one has $\mb_{\HT}(\l) = \cbb(\l)$. Moreover, the summands of $\HT$ in homological degree $\cbb(\l)$ and cell $\l$ are precisely the $w_0$-twisted involutions \begin{equation} \bigoplus_{d \in \DC \cap \l^t} B_{w_0 d}(\cbb(\l)). \end{equation} \end{prop}
	
\begin{proof} By Theorem \ref{thm:diagonalmiracle}, $\HT$ is perverse. By \eqref{eq:KLhalftwist}, $B_x$ appears in $\HT$ with graded multiplicity (for both homological shift and bimodule grading shift) given by $h_{1,w_0 x}$, whose minimum power of $v$ is $\D(w_0 x)$ by definition. As $x$ ranges over the cell $\l$, $w_0 x$ will range over $\l^t$, and by (P1) of Proposition
\ref{prop:moreP} the minimum value of $\D$ in this cell is $\rbb(\l^t) = \cbb(\l)$. This proves that $\mb_{\HT}(\l) = \cbb(\l)$. Moreover (again see Proposition \ref{prop:moreP}), the
only elements of the cell $\l^t$ which minimize $\D$ are the distinguished involutions $d \in \DC \cap \l^t$, for which the coefficient of $v^{\cbb(\l)}$ in $h_{1,d}$ is $1$. \end{proof}

Thus, if the half twist were twist-like, then it would be increasing by Lemma \ref{lemma:cineq}.

%========================
\subsection{The Sch\"utzenberger involution and the eigenvalues of the full twist}
\label{subsec:schutz}
%========================
We now discuss Sch\"utzenberger duality, which is a surprising involutory operation which fixes the left cells. Here is a restatement of a theorem\footnote{This theorem of Mathas was
recently generalized to Hecke algebras with unequal parameters by Lusztig in \cite{LuszLongest15}. In type $A$, Mathas attributes this theorem to J.~J.~Graham's thesis.} of Mathas
\cite[Theorem 3.1]{Mathas96}.

\begin{defn} Let $\xbb(\l) = \cbb(\l) - \rbb(\l)$. We refer to this as the \emph{content} of a two-sided cell. \end{defn}

\begin{thm} \label{thm:schugeneral} If $W$ is any finite Coxeter group, then there is an involution $\Schu_L \co W \to W$ which preserves each left cell, so that if $y$ is in cell $\l$
then \begin{equation} \label{eq:schugeneral} H_{w_0} b_y \equiv (-1)^{\cbb(\l)} v^{\xbb(\l)} b_{\Schu_L(y)} + \HB_{<\l}. \end{equation} Moreover, $\Schu_L(y)$ is in the same right cell
as $w_0 y w_0$.

Similarly, there is an involution $\Schu_R \co W \to W$ and an analogous formula which involves the right action of $H_{w_0}$ instead.
\end{thm}

In type $A$ an element is determined by its left and right cells, so $\Schu_L(y)$ is the unique element with $\Schu_L(y) \sim_L y$ and $\Schu_L(y) \sim_R w_0 y w_0$. In
other types, Mathas provides an additional condition on $\Schu_L$ which pins down the element $\Schu_L(y)$ uniquely; see Lemma \ref{lem:schuvstwist}. 
	
In type $A$, where the elements of a given left cell are parametrized by standard Young tableau with a given shape, $\Schu_L$ corresponds to a standard combinatorial involution on tableau called the \emph{Sch\"utzenberger involution}.  For this reason we call $\Schu_L$ the \emph{(generalized) Sch\"utzenberger involution}.

\begin{remark} If $\HT$ were sharp, then \eqref{eq:schugeneral} would be categorified in the best possible way: the minimal complex of $\HT \ot B_y$ would have most terms in cells $<
\l$, and there would be a unique term in cell $\l$, namely $B_{\Schu_L(y)}$ living in homological degree $\cbb(\l)$ with grading shift $\xbb(\l)$. This idea is pursued in the next
section. \end{remark}

Let us recall some properties of $\Schu_L$ and $\Schu_R$ proven by Mathas, which we do not use.

\begin{prop} Let $W$ be a finite Coxeter group. Then \begin{equation} \Schu_L(y\inv) = w_0 \Schu_L(y)\inv w_0 = \Schu_R(y)\inv. \end{equation} Moreover, $\Schu_L(w_0 y) = w_0 \Schu_L(y)$ and $\Schu_L(y w_0) = \Schu_L(y) w_0$. Thus \begin{equation} \Schu_L(\Schu_R(y)) = \Schu_R(\Schu_L(y)) = w_0 y w_0. \end{equation} \end{prop}

\begin{proof} Most of this can be found in \cite[Proposition 3.9]{Mathas96}. All that remains is to show that $\Schu_L(y\inv) = \Schu_R(y)\inv$. There is an antiautomorphism of $\HB$
(called $\a$ in the paragraph before \cite[Proposition 3.9]{Mathas96}) which sends $b_y \mapsto b_{y\inv}$, and fixes $H_{w_0}$. Applying this antiautomorphism to \eqref{eq:schugeneral},
the equality $\Schu_L(y\inv) = \Schu_R(y)\inv$ follows immediately. \end{proof}

Note the following crucial consequence of Theorem \ref{thm:schugeneral}.

\begin{cor}\label{cor:ftUpperTriangDecat} The operator of left multiplication by $H_{w_0}^2$ is upper-triangular with respect to the cell filtration, and on $\HB_{\le \l}/\HB_{< \l}$ it acts by the scalar\footnote{Of course, $2\cbb(\l)$ is even so the factor $(-1)^{2\cbb(\l)}$ is redundant. Nonetheless, we keep it around in anticipation of the fact that it will be categorified by a nontrivial homological shift.} $(-1)^{2\cbb(\l)} v^{2\xbb(\l)}$. \end{cor}

We have already seen that $H_{w_0}^2$ is central, hence acts diagonalizably on $\HB\otimes_{\Z[v,v\inv]}\Q(v)$.  The above pins down the eigenvalues exactly, and also illustrates the interaction with cell theory.

We conclude this section with another way to think about the Sch\"utzenberger involution. The lemma below was proven by Mathas as well (see \cite[Theorem 3.1]{Mathas96}) but we include a proof to emphasize again the same ideas used in Proposition \ref{prop:htm}.

\begin{lemma} \label{lem:schuvstwist} Fix $y \in W$ in two-sided cell $\l$. Then there is a unique distinguished involution $d$ in the opposite cell $\l^t$, and a unique $z$ in cell
$\l$, such that $t_{w_0 d, y}^z \ne 0$. In particular, $d$ is the distinguished involution in the same right cell as $y w_0$, $z$ is equal to $\Schu_L(y)$, and $t_{w_0 d, y}^z = 1$.
\end{lemma}

\begin{proof} Let $d$ be a distinguished involution in $\l^t$ and $z \in W$. If $t_{w_0 d, y}^z \ne 0$ then by (P8) of Proposition \ref{prop:someP} we have \begin{itemize} \item $w_0 d
\sim_R z$, \item $y \sim_L z$, and \item $w_0 d \sim_L y\inv$. \end{itemize} This last property is equivalent to $d w_0 \sim_R y$ and $d \sim_R y w_0$. Multiplying by $w_0$ on the left,
this is equivalent to $w_0 d \sim_R w_0 y w_0$. Combining this with the first property, we see that $z \sim_R w_0 y w_0$. In particular, $d$ is uniquely specified as above, and $z$ is at
least in the same right and left cell as $\Schu_L(y)$. (In type $A$ this already implies $z = \Schu_L(y)$.)

By the KL inversion formula \eqref{eq:KLhalftwist}, we can write $H_{w_0} b_y$ as
\begin{equation} H_{w_0} b_y = \sum_{x \in W} (-1)^{\ell(w_0) - \ell(w_0 x)} h_{1, x} b_{w_0 x} b_y = \sum_{x, z \in W} (-1)^{\ell(x)} h_{1, x} c_{w_0 x,y}^z b_z. \end{equation}
We simplified this formula by observing that $\ell(w_0) - \ell(w_0 x) = \ell(x)$. Let us consider this formula modulo $\HB_{< \l}$. By \eqref{eq:schugeneral}, we know that the answer must be $(-1)^{\cbb(\l)} v^{\xbb(\l)} b_{\Schu_L(y)}$.

All the terms with $w_0 x$ in cells $< \l$ disappear modulo $\HB_{< \l}$. In fact, if $w_0 x$ is in a cell incomparable to $\l$ then $b_{w_0 x} b_y$ is in cells $< \l$, so these terms
disappear as well. Hence we can assume that $w_0 x$ is in cells $\ge \l$. Therefore, $\D(x) \ge \rbb(x) \ge \rbb(\l^t) = \cbb(\l)$, with equality precisely when $x$ is in cell $\l^t$ and
is a distinguished involution. Consequently, the minimal power of $v$ appearing in $h_{1,x}$ is $v^{\cbb(\l)}$, for all $x$ which are relevant modulo $\HB_{< \l}$.

All terms with $z$ in cells $< \l$ disappear modulo $\HB_{< \l}$. Moreover, every nonzero term has $z$ in cells $\le \l$, because $b_z$ is a summand of $b_{w_0 x} b_y$. So we can assume
$z \sim_{LR} y$. The minimal power of $v$ appearing in $c_{w_0 x,y}^z$ when $z \sim_{LR} y$ is $v^{-\rbb(\l)}$, with equality precisely when $t_{w_0 x,y}^z \ne 0$.

Hence, every coefficient of $b_z$ has power of $v$ greater than $v^{\cbb(\l) - \rbb(\l)} = v^{\xbb(\l)}$, and the terms which contribute to this particular degree are \begin{equation}
\sum_{z \in \l, \\ d \in \DC \cap \l^t} (-1)^{\ell(d)} t_{w_0 d, y}^z b_z. \end{equation} We have already observed in Lemma \ref{lem:schuvstwist} that there is a unique $d \in \DC \cap
\l^t$ for which $t_{w_0 d,y}^z$ can be nonzero, namely the distinguished involution in the same right cell as $y w_0$. Thus, the coefficient of any $b_z$ is simply $\pm t_{w_0 d,y}^z$
for this particular $d$.

Since the answer must be $(-1)^{\cbb(\l)} v^{\xbb(\l)} b_{\Schu_L(y)}$, we see that $t_{w_0 d,y}^{\Schu_L(y)} = \pm 1$, and all other $t_{w_0 d, y}^z$ are zero. We already know by positivity that all structure coefficients live in $\N[v,v\inv]$, so $t_{w_0 d,y}^{\Schu_L(y)} = 1$. \end{proof}

\begin{remark} It is not difficult to show that, for any distinguished involution $d$ in cell $\l^t$, $\ell(d)$ and $\rbb(\l^t)$ have the same parity, which avoids the need for
positivity in this last step. This uses the fact that the KL polynomials $h_{x,y}$ and structure coefficients $c_{x,y}^z$ are all parity (they have only even or odd powers of $v$).
\end{remark}

\begin{cor} \label{cor:schudistinguished} Let $e$ be a distinguished involution in cell $\l$. Then there is a unique distinguished involution $d$ in cell $\l^t$ such that $\Schu_L(e) = d
w_0$. Namely, $d$ is the distinguished involution in the same right cell as $w_0 e$. \end{cor}
	
\begin{proof} By the last property in Proposition \ref{prop:moreP}, if $t_{x,y}^e \ne 0$ then $y = x^{-1}$. Applying this result when $y = \Schu_L(e)$ and $x = w_0 d$ for the
distinguished involution $d$ in the same right cell as $\Schu_L(e) w_0$, Lemma \ref{lem:schuvstwist} says that $t_{x,y}^e \ne 0$, and thus $y = x^{-1}$. Thus $y = d w_0$. Note that
$\Schu_L(e)$ is in the same right cell as $w_0 e w_0$, so $\Schu_L(e) w_0$ is in the same right cell as $w_0 e$. \end{proof}

Thus the set of $w_0$-twisted involutions $w_0 \DC$ is the same as the set of Sch\"utzenberger-twisted involutions $\Schu_L(\DC)$, though this bijection reverses cells, i.e. $w_0 d$ for
$d \in \l^t$ is $\Schu_L(e)$ for $e \in \l$.

%========================
\subsection{Main conjectures}
\label{subsec:sharpconj}
%========================

\begin{conjecture} \label{conj:HTaction} For any finite Coxeter group, the complex $\HT \in \KC^b(\SBim)$ is twist-like and sharp. \end{conjecture}

We prove this conjecture in type $A$ in \S\ref{subsec:bounding}. Let us describe some of the first consequences of this conjecture.

\begin{prop}\label{prop:FTconsequence}
Suppose $\HT$ is twist-like and sharp. Then $\nb_{\HT}(\l)=\mb_{\HT}(\l)=\cbb(\l)$ and $\nb_{\FT}(\l)=\mb_{\FT}(\l)=2\cbb(\l)$ for all two-sided cells $\l$.  In particular both $\HT$ and $\FT$ are sharp and increasing.  Moreover:
\begin{enumerate}
\item If $B_x$ is in cell $\l$ then the head of $\HT\otimes B_x$ equals $B_{\Schu_L(x)}[-\cbb(\l)](\xbb(\l))$.
\item If $B_x$ is in cell $\l$ then the head of $\FT\otimes B_x$ equals $\Sigma_\l(B_{x})$, where
\[
\Sigma_\l=[-2\cbb(\l)](2\xbb(\l)).
\]
\item The terms in $\HT$ in cell $\l$ and minimal homological degree $\cbb(\l)$ are exactly
\[ \bigoplus_d B_{\Schu_L(d)}(\cbb(\l)), \] where $d$ ranges over the distinguished involutions in cell $\l$.
\item The terms in $\FT$ in cell $\l$ and minimal homological degree $2\cbb(\l)$ are exactly are exactly
\[ \bigoplus_d B_d(\cbb(\l) + \xbb(\l)), \] where $d$ ranges over the distinguished involutions in cell $\l$.
\end{enumerate}
\end{prop}

\begin{proof}
Assume that $\HT$ is twist-like and sharp.  In Proposition \ref{prop:htm} we showed that $\mb_{\HT}(\l)=\cbb(\l)$, hence the inequality $\mb_{\HT}(\l)<\mb_{\HT}(\mu)$ when $\l<\mu$ follows from Lemma \ref{lemma:cineq}.  Thus, $\HT$ is increasing.  Aso $\nb_{\HT}(\l)=\mb_{\HT}(\l)=\cbb(\l)$ by sharpness.  Furthermore, by Lemma \ref{lem:sharptensorclosed} we can also deduce that $\FT$ is sharp and increasing, and $\mb_{\FT}(\l) =  2\cbb(\l)$.

By Proposition \ref{prop:tailProperty}, we know a lot about $\HT \ot B_x$ for any $x$ in cell $\l$. In particular, the only indecomposable summand of the minimal complex of $\HT \ot B_x$ in cell $\l$ is its head. By considering its image in the Grothendieck group and comparing with \eqref{eq:schugeneral}, we immediately deduce that its head is as desired. Similar arguments describe the head of $\FT \ot B_x$.  This proves (1) and (2).

We have already proven in Proposition \ref{prop:htm} that the terms in cell $\l$ and minimal homological degree are precisely
\[
\bigoplus_d B_{w_0 d}(\cbb(\l)),
\]
where the sum is over distinguished involutions $d$ in the opposite cell $\l^t$.  This proves (3).  



It remains to study the terms in $\FT$ in homological degree $2\cbb(\l)$ and cell $\l$. For the rest of this proof, let us fix $\l$ and set $\cbb = \cbb(\l)$, $\xbb = \xbb(\l)$, $\rbb = \rbb(\l)$.  If $C$ is a complex and $k$ is an integer, we let $C^k$ denote the $k$-th chain object of $C$.

Let us study $\HT\otimes \HT$, whose minimal complex is $\FT$.   Remark that each indecomposable summand of the chain object $\FT^m$ is a direct summand of $\HT^k\otimes \HT^{\ell}$ for some $k,\ell$ with $k+\ell=m$.   If $k<\cbb$ or $l<\cbb$ then $\HT^k\otimes \HT^l$ is a direct sum of indecomosables in cells less than or incomparable to $\l$.  Thus all terms of $\FT$ in cell $\l$ come from terms of the form $\HT^k\otimes \HT^{\ell}$ with $k,\ell\geq \cbb$.  In particular summands of $\FT^{2\cbb}$ in cell $\l$ are all summands of $\HT^{\cbb}\otimes \HT^{\cbb}$.

Let $Z:=\bigoplus_{d\in \DC\cap \l^t} B_{w_0d}(\cbb)$.  As noted above $\HT^{\cbb}$ is isomorphic to a direct sum of $Z$ and other bimodules in cells smaller than or incomparable to $\l$.  Consider the minimal complex of $\HT\otimes Z$.  First, recall that each $w_0d$ can be written uniquely as $d=\Schu_L(e)$ for some $e\in \DC\cap \l$, by Corollary \ref{cor:schudistinguished}.  The head of $\HT\otimes B_{\Schu_L(e)}$ is isomorphic to $B_e[-\cbb](\xbb)$ since $\Schu_L(\Schu_L(e))=e$.  In particular the head is in homological degree $\cbb$.  It follows that each summand $B_{\Schu_L(e)}(\cbb)\subset Z$ ultimately contributes a summand $B_e(\cbb+\xbb)$ to $\FT$, in homological degree $2\cbb$.

It remains to prove that these terms survive to the minimal complex $\FT$ of $\HT \ot \HT$.  This can be done by proving that no isomorphic terms appear in adjacent homological degrees $2\cbb \pm 1$, so that no Gaussian elimination can cancel these terms.  The constraints $k, \ell \ge \cbb$ already imply that nothing in cell $\l$ appears in homological degree $2\cbb-1$. In homological degree $2\cbb+1$, either $k = \cbb$ and $\ell = \cbb + 1$ or vice versa; without loss of generality assume $k = \cbb$.  Every indecomposable summand of $\HT^{\cbb+1}$ is of the form $B_y(k+1)$ by perversity.   If $B_z(i)$ is a direct summand of $B_y(\cbb+1) \ot B_x(\cbb)$ in cell $\l$, then we must have $i\geq \cbb+1 + \cbb - \rbb = \cbb + \xbb +1$.  Hence nothing with grading shift $\cbb + \xbb$ can appear. This concludes the proof.
\end{proof}


The shift functors $\Sigma_\l=[-2\cbb(\l)](2\xbb(\l))$ therefore play the role of the ``eigenvalues'' of the full twist (modulo Conjecture \ref{conj:HTaction}), and $\FT$ is
cell triangular in the sense of Definition \ref{def:twistedunitri}. Thus it makes sense to look for $\l$-equivalences for each cell $\l$, see Definition \ref{def:lequiv}.

\begin{conjecture}[Eigenmap conjecture]\label{conj:eigenmap}
If $W$ is a finite Coxeter group and $\l$ is a two-sided cell in $W$ then there exists a map $\a_\l:\Sigma_\l(\one)\rightarrow \FT$ which is a $\l$-equivalence.  These maps are obstruction free in the sense of \cite[Definition 6.12]{ElHog16a}.
\end{conjecture}

We prove this for $W=S_n$ in \S \ref{sec:constructing}.








%========================
\subsection{Dot maps}
\label{subsec:dots}
%========================

Previously we used the numerics of the functions $\D$ and $\rbb$ to study complexes of Soergel bimodules, using perversity and plethysm\footnote{From the Greek ``plethysmos,'' meaning ``multiplication.''} rules. Now we discuss another categorical shadow of these numerical invariants, this time using the Soergel Hom formula. We need not assume $W$ is finite in this section.

\begin{proposition}\label{prop:dotspace} Let $x$ be in two-sided cell $\l$. The smallest $k$ such that there exists a nonzero map $R \rightarrow B_x(k)$ equals $\Delta(x)$. Within each
cell, this minimum equals $\rbb(\l)$, and is achieved only for the distinguished involutions. If $x$ is a distinguished involution, then $\Hom(R,B_x(\rbb(\l)))$ is 1-dimensional.
\end{proposition}

\begin{proof}
By the Soergel Hom formula, the graded rank of $\Homg(R,B_x)$ over $R$ is $h_{1,x}$, so its minimal degree is $\D(x)$. The remaining statements follow immediately from Proposition \ref{prop:moreP}.
\end{proof}


%This allows us to define the dot maps and barbells associated to distinguished involutions.

\begin{definition}\label{def:dots}
For each $d\in \DC$ in two-sided cell $\l$, choose $\xi_d: R\rightarrow B_{d}(\rbb(\l))$ which generates the corresponding Hom space.  Let $\xi_d^\ast$ denote the dual map $B_{d}\rightarrow R(\rbb(\l))$. We call these \emph{(generalized) dot maps} or \emph{$d$-dot maps}. Let $\tbarb{d}$, the \emph{(generalized) barbell} denote the composition $\xi_d^\ast \circ \xi_d$.  Then $\tbarb{d}$ is a degree $2\rbb(\l)$ element of $R=\Endg(R)$. 
\end{definition}
%\frac{\a_s}{2} \ot 1 + 1 \ot \frac{\a_s}{2}

\begin{example}\label{ex:usualDot} Fix a simple reflection $s$, which is a distinguished involution in the simple cell $\l$ satisfying $\rbb(\l)=1$. We may choose $\xi_s:R\rightarrow
B_s(1)$ and $\xi_s^\ast:B_s\rightarrow R(1)$ to be the usual \emph{dot maps} in the diagrammatic calculus, see Remark \ref{rmk:dotsdefn}. Composing the two dots gives a morphism $R
\to R(2)$ which is multiplication by $\a_s$, and is pictorially represented by a \emph{barbell}. \end{example}

\begin{example}\label{ex:longestBarbell}
Let $W$ be finite. Then $w_0$ is in the longest cell $\l$ with $\rbb(\l) = \ell(w_0)$. Let $\ell = \ell(w_0)$.  Recall that $B_{w_0}=R\otimes_{R^W}R(\ell)$, and $R$ is a Frobenius extension over $R^W$ with Frobenius trace $\pa_{w_0}$.  We choose $\xi_w:R \rightarrow B_{w_0}(\ell)$ to be the map which sends
\[
1\mapsto \sum a_i \ot b_i,
\]
where the sum is over dual bases for $R$ over $R^W$ with respect to $\pa_{w_0}$. The dual map $B_{w_0}\rightarrow R(\ell)$ sends $f\otimes g\mapsto fg$.  Their composition $\tbarb{w_0}$ is the product of all the positive roots. See \cite[\S 3.6]{EThick} for more discussion of these maps. \end{example}

The importance of $d$-barbells will be highlighted in the next chapter. For now, we focus on $d$-dot maps.

\begin{lemma} \label{lem:notinlower} For $d$ in two-sided cell $\l$, the $d$-dot maps are not in the ideal $\IC_{< \l}$, or the ideal $\IC_{\ngeq \l}$. \end{lemma}

\begin{proof} We need to prove that $\xi_d$ is not a linear combination of maps which factor as $R \to B_z \to B_d$ for $z$ in lower cells. But the minimal degree of a map $R \to B_z$ is
$\D(z) \ge \rbb(z) \ge \rbb(\l)$, and the minimal degree of a map $B_z \to B_d$ is $+1$ (as for any two non-isomorphic indecomposable Soergel bimodules). Thus no map of degree $\rbb(\l)$
can factor through lower cells. Since $\xi_d$ does factor through $B_d$ in cell $\l$, it is in $\IC_{\le \l}$, so not being in $\IC_{< \l}$ is equivalent to not being in $\IC_{\ngeq
\l}$. \end{proof}

Now let us fix a finite Coxeter group $W$ and assume Conjectures \ref{conj:HTaction} and \ref{conj:eigenmap}, so that $\l$-equivalences $\a_\l \co \Sigma_\l (R)\to \FT$ exist. Since $\Sigma_\l(R)$ is just a shift of $R$ concentrated in a single homological degree, $\a_\l$ is effectively a bimodule map from $R(2\xbb(\l))$ to the homological degree $2\cbb(\l)$ chain object of the full twist, which contains summands \[ \bigoplus_d B_d(\cbb(\l) + \xbb(\l)) \] for all $d \in \DC \cap \l$. Thus $\a_\l$ determines and a collection of morphisms $R\rightarrow B_d(\cbb(\l) - \xbb(\l))$, which must be scalar multiples of the dot maps since $\cbb(\l)-\xbb(\l)=\rbb(\l)$.  We call these the \emph{$\xi_d$ components} of $\a_\l$.  Strictly speaking, these components are only well-defined up to unit multiple, since there is a choice in how $B_d(\cbb(\l) + \xbb(\l))$ sits as a direct summand of $\FT^{2\cbb(\l)}$.

The $\xi_d$ components depend only on the homotopy class of $\a_\l$.  Indeed, if $\b-\a_\l = [d,h]$ then actually $\b-\a_\l = d\circ h$ (the differential on $R$ being zero), and $h$ is a bimodule morphism from a shift of $R$ to the homological degree $2\cbb(\l)-1$ part of $\FT$.  It follows that $h$ factors through terms in cells $\ngeq \l$, hence $d\circ h$ cannot affect the $\xi_d$ components by Lemma \ref{lem:notinlower}.

\begin{proposition}\label{prop:dotsFromEigenmaps}
Assume that $W$ is a finite Coxeter group and Conjectures \ref{conj:HTaction} and \ref{conj:eigenmap} hold. Let $x \in W$ be in the same right cell as $d \in \DC$. Then $\xi_d\otimes \Id_{B_x}$ is the inclusion of a direct summand while $\xi_d^\ast\otimes \Id_{B_x}$ is the projection onto a direct summand. The $\xi_d$ component of $\a_\l$ is an invertible\footnote{We work over a field, so we may as well say nonzero. When working over other base rings, it is not hard to prove this slightly stronger result.} scalar multiple of $\xi_d$.
\end{proposition}

\begin{proof}
Fix a $\l$-equivalence $\a_\l$, for the two-sided cell $\l$ containing $x$ and $d$. As before, let $\cbb = \cbb(\l)$, etcetera. By Proposition \ref{prop:FTconsequence} we know that the head of the minimal complex of $\FT \ot B_x$ is $\Sigma_\l(B_x)$ and that $\a_\l \ot \Id_{B_x}$ is an isomorphism from $\Sigma_\l(B_x)$ to this head. We claim that it is only the $\xi_d$ component of $\a_\l$ which could contribute to the isomorphism in $\a_\l \ot \Id_{B_x}$.

In homological degree $2\cbb$ the complex $\FT$ is a direct sum of $B_{d'}(\cbb+\xbb)$, where $d'$ ranges over the distinguished involutions in cell $\l$, together with objects in cells
$\ngeq \l$, by Proposition \ref{prop:FTconsequence}. Consequently, the only direct summands of $\FT \ot B_x$ in this homological degree and in cell $\l$ are summands $B_{d'} \ot B_x(\cbb
+ \xbb)$. The minimal grading shift which could occur in such a summand is $B_z(\cbb + \xbb - \rbb) = B_z(2\xbb)$, and it is realized only when $t_{d',x}^z \ne 0$. But by Proposition
\ref{prop:dacts}, this happens precisely when $d'$ is in the same right cell as $x$ (i.e. $d' = d$), and $z = x$. Hence the summand $B_x(2\xbb)$ which survives to the minimal complex of
$\FT \ot B_x$ must have arisen as a direct summand of $B_d(\cbb + \xbb) \ot B_x$.

If the $\xi_d$ component of $\a_\l$ is zero (or noninvertible), then so is the component of $\a_\l \ot \Id_{B_x} \co B_x[-2\cbb](2\xbb) \to \FT \ot B_x$ which goes to $B_d(\cbb+\xbb) \ot
B_x$ and its summand $B_x(2\xbb)$. Gaussian elimination from $\FT \ot B_x$ to its minimal complex can change the components of a chain map, but only by null-homotopic maps. Any homotopy
would factor through homological degrees $< 2\cbb$ in $\FT \ot B_x$, and therefore would factor through cells $< \l$. Thus it can not make a non-isomorphism into an isomorphism. This
contradicts Lemma \ref{lem:lequivcriterion}. Thus the $\xi_d$-component of $\a_\l$ is invertible, $\xi_d \ot \Id_{B_x}$ must be the inclusion of the $B_x$ summand.

Duality implies that $\xi_D^\ast$ is projection to a summand. \end{proof}

One interprets Proposition \ref{prop:dotsFromEigenmaps} as a strong categorification of the unit axiom in the $J$-ring. Decategorified, if $d$ is a distinguished involution then
$v^{\rbb(d)} b_d$ acts by the identity on its right cell, modulo positive powers of $v$ and lower two-sided cells. Corollary \ref{cor:weakJringCat} was a weak categorification of the
same statement. A strong categorification would give a natural transformation between the identity functor and the functor $B_d(\rbb(d)) \ot (-)$, which realizes the fact that
$B_d(\rbb(d))$ acts by the identity on its right cell (modulo positive shifts and lower cells). By Proposition \ref{prop:dotsFromEigenmaps}, the $d$-dot $\xi_d$ serves this role!

Moreover, in cell $\l$ and homological degree $2\cbb(\l)$, $\FT$ is (up to shift) just a categorification of the element $v^{\rbb(\l)} \sum_{d \in \DC \cap \l} b_d$ in $\HB$ which ``descends'' to the unit of the ring $J_\l$. The $\l$-equivalence $\a_\l$ combines the $d$-dots into a map which realizes the unit axiom of the $J$-ring.

\begin{remark} \label{rmk:syzygy}
We wish to point out an interesting interpretation of the above.  First, let us choose a two-sided cell $\l$, and let $F = \FT[2\cbb(\l)](-2\xbb(\l))$. Let's work modulo lower cells, i.e.~ in the category $\SBim_n/I_{<\l}$.  Modulo Conjecture \ref{conj:HTaction}, Proposition \ref{prop:FTconsequence} tells us that
\[
F = F^0\rightarrow F^1\rightarrow \cdots \rightarrow F^{m},
\]
where $F^0 = \bigoplus_{d\in \DC\cap \l} B_d(\rbb(\l))$ is the Soergel bimodule analogue of the unit in the $J$-ring.  The maximal homological degree happens to be $m=2\ell(w_0)-2\cbb(\l)$, but this won't be relevant.  We know that if $x\in W$ is in cell $\l$ then
\[
F^0\otimes B_x \cong B_x \oplus Y \qquad \text{modulo $I_{<\l}$}
\]
where $Y$ is a direct sum $B_y(k)$ with $y$ in cell $\l$ and $k>0$.  We also know (again modulo Conjecture \ref{conj:HTaction}) that
\[
F\otimes B_x \simeq B_x \qquad \text{modulo $I_{<\l}$}.
\]
That is to say, the terms in $F^0\otimes B_x$ with positive degree shifts are being cancelled by $F^1\otimes B_x$, and the surviving terms in $F^1\otimes B_x$ are cancelled by $F^2\otimes B_x$, and so on.  Thus, one can think of $F^0$ more accurately as a first approximation to the unit in the $J$-ring, $F^0\rightarrow F^1$ as a better approximation, and so on, until $F^\bullet$ itself actually behaves as the unit in the $J$-ring associated to $\l$.

Note that $(F^0\rightarrow \cdots \rightarrow F^i)\otimes B_x\simeq B_x \oplus Y^i[-i]$ where $Y^i\in \SBim$ ($0\leq i\leq m$); these yield potentially interesting new invariants $[Y^i]$ in the Hecke algebra.
\end{remark}

%\begin{remark} We wish to emphasize something magical, which is a restatement of Conjecture \ref{conj:HTaction}: a loose sense in which passage to the $J$-ring is being performed homologically. If $x$ is in cell $\l$ then \begin{equation} \label{eq:Jringnotyet} \bigoplus_{d \in \DC \cap \l} B_d(\rbb(d)) \ot B_x \cong B_x \oplus Y \oplus Z,\end{equation} where $Y$ represents a lot of junk with positive shifts, and $Z$ some junk in lower cells. We think of $Y$ as the ``zero-th $J$-ring syzygy,'' as when you throw it out, all that remains is how $\sum j_d$ would act in $J_{\l}$. In the complex $\FT \ot B_x$, in homological degree $2\cbb(\l)$ and cell $\l$, it is this tensor product \eqref{eq:Jringnotyet} which appears. However, every term in $Y$ disappears because it cancels a term in homological degree $2\cbb(\l)+1$! Conjecture \ref{conj:HTaction} implies this cancellation of $Y$ and more; the ``higher $J$-ring syzygies,'' i.e. the terms in homological degrees $> 2\cbb(\l)$ and cell $\l$ (not counting $Y$ in degree $2\cbb(\l) + 1$), also disappear using Gaussian elimination. One wonders (very vaguely) whether more refined invariants similar to the $J$-ring can be obtained by studying specific homological degrees in the full twist. \end{remark}

Let us conclude this section by motivating and discussing one additional application of Proposition \ref{prop:dotsFromEigenmaps}.

When $d = w_I$ is the longest element of a finite parabolic subgroup $W_I$, the Soergel bimodule $B_{w_I} \cong R \ot_{R^I} R(\ell(w_I))$ is a graded Frobenius algebra object of degree $\ell(w_I) = \rbb(w_I)$. This is because $R^I \subset R$ is a Frobenius extension of commutative graded rings. In this case, the dot maps $\xi$ and $\xi^\ast$ give rise to the unit and counit maps in the Frobenius algebra structure. The multiplication and comultiplication maps arise via the decomposition 
\[ B_{w_I} \ot B_{w_I} \cong B_{w_I}(-\ell(w_I)) \oplus \cdots \oplus B_{w_I}(\ell(w_I)),\]
namely, they are the projection to the minimal degree summand, and the inclusion from the maximal degree summand. All four of these maps are well-defined up to scalar, and some choice of scalars makes the Frobenius algebra axioms hold.

\begin{conj} For any Coxeter group and any distinguished involution $d$, $B_d$ has the structure of a graded Frobenius algebra object in $\SBim$ of degree $\rbb(d)$. The unit and counit are given (up to scalar) by $\xi_d$ and $\xi_d^\ast$ respectively. The multiplication (resp. comultiplication) map is given (up to scalar) by a projection to (resp. inclusion of) the minimal (resp. maximal) degree summand in
\[ B_d \ot B_d \cong B_d(-\rbb(\l)) \oplus \cdots \oplus B_d(\rbb(\l)) \oplus Z. \]
Here $Z$ represents those summands in cells $< \l$, which may in theory have more extreme degree shifts, but which we ignore when we use the words ``minimal'' and ``maximal.''
\end{conj}

In a follow up paper, we will prove this conjecture for finite Coxeter groups, assuming Conjectures \ref{conj:HTaction} and \ref{conj:eigenmap}. Thus, the conjecture will hold in type
$A$. Proposition \ref{prop:dotsFromEigenmaps} is in essence a proof of the unit axiom for the Frobenius algebra.

% In general we conjecture that $B_d$ has the structure of a graded Frobenius algebra object in $\SBim$ whenever $d$ is a distinguished involution.
% \begin{conjecture}\label{conj:frobalgAndInvolutions}
% If $d$ is a distinguished involution in a Coxeter group $W$, then $B_d$ has the structure of a graded Frobenius algebra object in $\SBim$.  In other words, there are maps
% \[
% \xi: R\rightarrow B_d(\ell),\qquad  \mu: B_d(\ell)\otimes B_d(\ell)\rightarrow B_d(\ell),
% \]
% with duals
% \[
%  \xi^\ast : B_d(-\ell)\rightarrow R, \qquad \mu^\ast:B_d(-\ell) \rightarrow B_d(-\ell)\otimes B_d(-\ell),
% \]
% where $\ell=\ab(\l)$, such that
% \begin{enumerate}
% \item $(B_d(\ell),\xi,\mu)$ is an algebra in $\SBim$,
% \item $(B_d(-\ell), \xi^\ast, \mu^\ast)$  is a coalgebra in $\SBim$,
% \item the Frobenius relation holds (omitting shifts):
% \[
% (\mu\otimes \Id_{B_d})\circ (\Id_{B_d}\otimes \mu^\ast)=\mu^\ast\circ \mu = (\Id_{B_d}\otimes \mu)\circ (\mu^\ast\otimes \Id_{B_d})% \qquad\qquad (\text{omitting shifts}).
% \]
% \end{enumerate}
% \end{conjecture}
% We will consider this structure in future work.

