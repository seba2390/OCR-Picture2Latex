%%%%%%%%%%%%%%%%%%%%%%
\section{Relative cell theory and relative diagonalization in type $A$}
\label{sec:diag}
%%%%%%%%%%%%%%%%%%%%%%

In the last section we constructed maps $\a_\l$ for each two-sided cell of $S_n$, and proved that $\FT_n$ was categorically prediagonalizable. However, we can not directly conclude that
$\FT_n$ is categorically diagonalizable. The issue is that there may exist distinct partitions $\l$ with the same shift $\Sigma_\l$, and which precludes an application of the categorical
diagonalization theorem \cite[Theorem 8.1]{ElHog17a}. To circumvent this problem, it is necessary to use the more technical relative diagonalization theorem \cite[Theorem 8.2]{ElHog17a}
(relative to an inductively constructed diagonalization of $\FT_{n-1}$).


%========================
\subsection{Diagonalization of $\FT_n$}
%========================

The purpose of this section is to state and understand some basic consequences of our main theorem.

\begin{theorem}\label{thm:typeAdiag}
There exist complexes $\PB_\l\in \KC^-(\SBim_n)$ indexed by partitions $\l$ of $n$, such that $\{(\PB_\l, \a_{\l})\}_{\PC(n)^{\op}}$ is a diagonalization of $\FT_n$, where $\a_\l$ are the maps from Theorem \ref{thm:lambdaMaps}.  In particular, we have a decomposition of identity 
\begin{equation}\label{eq:resofId}
\one \simeq \Big(\bigoplus_{\l} \PB_\l, \sum_{\mu,\l} d_{\mu\l}\Big)
\end{equation}
with $\PB_\l\otimes \Cone(\a_\l) \simeq \Cone(\a_\l)\otimes \PB_\l$.  Furthermore, $\PB_\l$ interacts with the cell theory of $\SBim_n$ in the following way:
\begin{enumerate}\setlength{\itemsep}{2pt}
\item the chain bimodules of $\PB_\l$ are in cells $\leq \l$.
\item $\PB_\l$ annihilates bimodules in cells $\not\geq \l$.
\end{enumerate}
\end{theorem}

The proof is an induction on $n$, carried out in \S \ref{subsec:proof}.  But first we develop some consequences which will be helpful in the induction step.  If $(\Omega,\leq)$ is a poset, recall that a subset $I\subset \Omega$ is an \emph{ideal} if $i\leq j$ and $j\in I$ implies $i\in I$.  A subset $K\subset \Omega$ is \emph{convex} if $i\leq j\leq k$ and $i,k\in K$ implies $j\in K$.  Any convex set $K$ can be written as $K=J\setminus I$ for some poset ideals $I\subset J\subset \Omega$.

\begin{definition}\label{def:PK}
For each convex set $K\subset \PC(n)$, we have subquotient complex
\[
\PB_K:= \Big(\bigoplus_{\l\in K}\PB_\l, \sum_{\mu,\l\in K} d_{\mu\l}\Big).
\]
In particular $\PB_{\PC(n)}$ denotes the right-hand side of \eqref{eq:resofId}. 
\end{definition}

The differential $d_{\mu\l}\in \Homc^{1}(\PB_\l,\PB_\mu)$ is nonzero only if $\l\geq \mu$.  Thus, if $I\subset \PC(n)$ is a poset ideal, then $\PB_I$ is a subcomplex of $\PB_{\PC(n)}$, and the inclusion $\e:\PB_I\rightarrow \PB_{\PC(n)}\simeq \one$ satisfies $\Cone(\e)\simeq \PB_{I^c}$ where $I^c = \PC(n)\setminus I$.  This gives us a distinguished triangle
\[
\PB_I\rightarrow \one\rightarrow \PB_{I^c}\rightarrow \PB_I[1].
\]
The orthogonality of the $\PB_\l$ with respect to $\otimes$ implies that $\PB_I \otimes \PB_{I^c}\simeq 0 \simeq \PB_{I^c} \ot \PB_I$, hence $\PB_I$ has the structure of a counital idempotent in $\KC^-(\SBim_n)$ with complementary unital idempotent $\PB_{I^c}$.  The same argument shows more generally that if $I\subset J\subset \PC(n)$ are poset ideals and $K=J\setminus I$, then we have a distinguished triangle
\[
\PB_I\rightarrow \PB_J\rightarrow \PB_K\rightarrow \PB_I[1],\qquad\qquad \PB_I\otimes \PB_K\simeq 0 \simeq \PB_K\otimes\PB_I.
\]
That is, $\PB_I$ is a counital idempotent relative to $\PB_J$, with relative complement $\PB_{J\setminus I}$.   We collect some important properties of the $\PB_K$ below.

\begin{lemma}\label{lemma:PK}
The idempotents $\PB_K$, indexed by convex subsets $K\subset \PC(n)$, satisfy
\begin{enumerate}
\item $\PB_{\PC(n)}\simeq \one$ and $\PB_\emptyset = 0$.
\item $\PB_K\otimes \PB_L\simeq \PB_{K\cap L}$.
\end{enumerate}
In particular each $\PB_K$ is idempotent with respect to $\otimes$, up to homotopy.
\end{lemma}

The counital idempotents $\PB_I$, indexed by poset ideals $I\subset \PC(n)$, are often easier to deal with in practice. 
\begin{proposition}\label{prop:counitalChar}
If $I\subset \PC(n)$ is a poset ideal then $\PB_I$ satisfies
\begin{enumerate}
\item $\PB_I$ is homotopy equivalent to a complex whose chain objects are in cells $\l\in I$.
\item there exists a chain map $\e:\PB_I\rightarrow \one$ such that $\Cone(\e)\otimes B\simeq 0 \simeq B\otimes \Cone(\e)$ for all $B$ in cell $\l$ with $\l\in I$.
\end{enumerate}
Furthermore,
\begin{enumerate}
\item[(3)] a complex $C\in \KC^-(\SBim_n)$ satisfies $\PB_I\otimes C\simeq C$ if and only if $C$ is homotopy equivalent to a complex whose chain bimodules lie in cells $\l \in I$.
\item[(4)] properties (1) and (2) characterize the pair $(\PB_I,\e)$ up to canonical isomorphism in $\KC^-(\SBim_n)$.
\end{enumerate}
\end{proposition}
\begin{proof}
Property (1) follows from property (1) of Theorem \ref{thm:typeAdiag}, given that $\PB_I$ is a convolution of complexes $\PB_\l$ with $\l\in I$.


We choose $\e:\PB_I\rightarrow \one$ to be the first map in a distinguished triangle $\PB_I\rightarrow \one\rightarrow \PB_{I^c}\rightarrow \PB_I[1]$, so that $\PB_{I^c} \simeq \Cone(\e)$.  If $B$ is a bimodule in cell $\l\in I$, then $\PB_{I^c}\otimes B\simeq 0\simeq B\otimes \PB_{I^c}$ by property (2) of Theorem \ref{thm:typeAdiag}.  Here we are using the fact that $\PB_{I^c}$ is a convolution of $\PB_\l$ with $\l\in I^c$, and no element of $I$ can be larger or equal to an element of $I^c$.  This proves (2).

Now, let $C\in \KC^-(\SBim_n)$ be given.  Then each chain bimodule of $\PB_I\otimes C$ in a cell $\l\in I$, since the same is true of $\PB_I$.  This proves the ``only if'' direction of (3).  Conversely, if each summand of each chain bimodule of $C$ is in a cell $\l\in I$, then $\PB_{I^c}\otimes C\simeq 0\simeq C\otimes \PB_{I^c}$ by (2). The existence of the distinguished triangle $\PB_I\rightarrow \one\rightarrow \PB_{I^c}\rightarrow \PB_I[1]$ then implies that $\PB_I\otimes C\simeq C\simeq C\otimes \PB_I$.  Finally, the uniqueness statement (4) follows by elementary properties of categorical idempotents, see Theorem 4.28 in \cite{Hog17a}. 
\end{proof}

\begin{lemma}\label{lemma:projcentral}
Each $\PB_K$ is central in $\KC^-(\SBim_n)$.
\end{lemma}
\begin{proof}
Any convex set $K$ can be expressed as $K=J\setminus I$ for some poset ideals $I\subset J\subset \PC(n)$.  In this case $\PB_K$ can be expressed as
\[
\PB_K \simeq \PB_J\otimes \PB_{I^c}.
\]
A counital idempotent is central if and only if its unital complement is central (Theorem 4.23 in \cite{Hog17a}) hence we are reduced immediately to the case when $K=I$ is a poset ideal. 
This, in turn, follows from general arguments.  Let $\EB$ be a unital or counital idempotent in any triangulated monoidal category.  Let $\AC \EB\subset \AC$ denote the full subcategory on the set of objects $A$ with $A\otimes \EB\simeq A$.  Then $\EB$ is central in $\AC$ if and only if $\AC \EB$ is a two-sided tensor ideal.   The lemma now follows directly from the last statement of Proposition \ref{prop:counitalChar}.
\end{proof}

Note that the singleton $K=\{\l\}$ is automatically a convex subset of $\PC(n)$, hence $\PB_\l =\PB_{\{\l\}}$ is central as well.


%\begin{definition}\label{def:eigenEndomorphisms}
%Let $\frac{\a_\mu}{\a_\l}\in \Endgg(\PB_\l)$ denote the shift functor $\Sigma_\l\inv$ applied to the composition
%\[
%\begin{diagram}
%\Sigma_\mu(\one) \otimes \PB_\l & \rTo^{\a_\mu\otimes \PB_\l} &\FT_n\otimes \PB_\l & \rTo^{(\a_\l\otimes \PB_\l)\inv} & \Sigma_\l(\one)\otimes \PB_\l \end{diagram},
%\]
%thought of as a morphism of $\Sigma_\l\Sigma_\mu(\PB_\l)\rightarrow \PB_\l$.
%\end{definition}


%========================
\subsection{Adding a strand}
%========================

Suppose that we have proven Theorem \ref{thm:typeAdiag} for $n-1$. Then for any $\mu \in \PC(n-1)$, one has an idempotent complex $\PB_{\mu}$ with certain properties. The key to our inductive argument will be understanding how the complex $(\PB_{\mu} \sqcup \one_1)$ acts on $\KC^b(\SBim_n)$. We begin with some elementary observations.

\begin{lemma}[Projector sliding]\label{lemma:projsliding1}
We have $F(\sigma_1\sigma_2\cdots\sigma_{n-1})\otimes (\PB_K\sqcup \one_1)\simeq (\one_1\sqcup \PB_K)\otimes F(\sigma_1\sigma_2\cdots \sigma_{n-1})$ inside $\KC^b(\SBim_{n})$, for any convex subset $K$ of $\PC(n-1)$.
\end{lemma}
We refer to the the statement of this lemma colloquially as ``projectors slide past strands.''
\begin{proof}
We prove this first in case $K=I$ is a poset ideal, then for $K=I^c$, the complement of an ideal, then for general $K$.

Denote $\XB:=F(\sigma_1\sigma_2\cdots \sigma_{n-1})$.  Abuse notation by writing $\PB_K = \PB_K\sqcup \one_1$ and $\PB_K'=\one_1\sqcup \PB_K$.  We also will omit the tensor product symbol for aesthetic reasons.  We may write $K=J\setminus I$ for some poset ideals $I\subset J\subset \PC(n-1)$, and $\PB_K \simeq \PB_J\otimes \PB_{I^c}$. Thus, the desired equivalence $\XB \PB_K\simeq \PB_K'\XB$ reduces immediately to the case when $K=I$ or $K=I^c$.

If $I\subset \PC(n-1)$ is an ideal, then there is a distinguished triangle
\[
\PB_I\rightarrow \one \rightarrow \PB_{I^c}\rightarrow \PB_I[1],
\]
where the chain bimodules of $\PB_{I}$ are in cells $\l\in I$ and $\PB_{I^c}$ kills all bimodules of $\SBim_{n-1}$ in cells  $\l\in I$.  This yields a distinguished triangle
\[
\PB_I' \XB \PB_I\rightarrow \PB_I' \XB \rightarrow \PB_I'\XB \PB_{I^c}\rightarrow \PB_I'\XB \PB_I[1].
\]
The third term is contractible since each chain bimodule of $\PB_{I}'$ is in cells $\l\in I$, slides past $\XB$ by Proposition \ref{prop:slidingBimodules}, and is annihilated by $\PB_{I^c}$.  This implies that
\[
\PB_I'\XB\PB_I \simeq \PB_I'\XB.
\]
A similar argument shows that
\[
\PB_I'\XB\PB_I \simeq \XB\PB_I,
\]
from which it follows that $\XB\PB_I\simeq \PB_I\XB$, as claimed.  A similar argument shows that $\XB\PB_{I^c}\simeq \PB_{I^c}\XB$.   This completes the proof.
\end{proof}

\begin{lemma}\label{lemma:projectorSliding}
We have $(\PB_{\mu}\sqcup \one_1)\otimes \FT_n \simeq \FT_n\otimes (\PB_{\mu} \sqcup \one_1)$.
\end{lemma}
\begin{proof}
This follows immediately from Lemmas \ref{lemma:projcentral} and \ref{lemma:projsliding1}.
\end{proof}

\begin{corollary} \label{cor:projectorconecommute}
$\PB_{\mu}\sqcup \one_1$ commutes with $\Cone(\a_\l)$ for all $\l \in \PC(n)$ and all $\mu \in \PC(n-1)$.
\end{corollary}
\begin{proof}
See Lemma 6.19 in \cite{ElHog17a} and its proof.
\end{proof}

Now that these basic properties are in place, we must ask the harder questions. The idempotent $\PB_\mu$ interacts with the cell theory of $\SBim_{n-1}$ in a particular way; how does the
idempotent $\PB_\mu \sqcup \one_1$ interact with the cell theory of $\SBim_n$? We will eventually show that tensoring with $\PB_\mu \sqcup \one_1$ will allow us to restrict our attention
to those $\l \in \PC(n)$ which are obtained from $\mu$ by adding a box, which is a totally ordered set. For this purpose, we now digress and discuss relative cell theory.

%========================
\subsection{Parabolic subgroups and subtableaux}
\label{subsec:subtableaux}
%========================

The Hecke algebra $\HB(S_k)\cong \HB(S_k\times S_1^{\ell})$ embeds into $\HB(S_{k+\ell})$ using the external product. The goal of this section and the next will be to generalize \eqref{eq:dacts}, taking the operator $v^{\rbb(\l)} b_{T,T} \in \HB(S_k)$ and seeing how it acts on $\HB(S_{k+\ell})$.

Let $W_I$ be a parabolic subgroup of $W$. Each right coset of $W_I$ has a unique element of minimal length; let $Y_I$ denote this set of minimal length right coset representatives.
Similarly, let $X_I$ denote the set of minimal length left coset representatives. Each $w \in W$ has a unique representation as $uy$ for $u \in W_I$ and $y \in Y_I$, or as $xt$ for $t
\in W_I$ and $x \in X_I$, and moreover $\ell(w) = \ell(u) + \ell(y) = \ell(t) + \ell(x)$.

Our first goal is to relate the language of cosets to the language of tableaux. Given a tableau $P$ with $n$ boxes and $k<n$, let $P^k$ denote the tableau with $k$ boxes obtained by
remembering the first $k$ boxes of $P$.

\begin{lemma} \label{lem:cosetvtab} Let $W = S_{k+\ell}$ and $W_I = S_k=S_k\times S_1\times \cdots \times S_1$. Let $w = uy = xt$ for $w \in W$, $x \in X_I$, $y \in Y_I$, and $t, u \in W_I$. Suppose that $w$ corresponds to
$(P,Q)$ under the Robinson-Schensted correspondence. Then $u$ corresponds to $(P^k,A)$ for some tableau $A$, and $t$ corresponds to $(B,Q^k)$ for some tableau $B$. \end{lemma}

\begin{remark} The result for right cosets and the result for left cosets imply each other, under taking inverses. Thus, we need prove only the second statement, that $t$ corresponds to
$(B,Q^k)$. Because of the asymmetry between $P$-symbols and $Q$-symbols in the Robinson-Schensted algorithm, we found left cosets easier to use than right cosets. \end{remark}

\begin{proof} The permutation $t \in S_k$ is easily described as the permutation which puts $\{1, \ldots, k\}$ in
the same order as $\{w(1), \ldots, w(k)\}$. More precisely, let $Z$ be the image of the set $\{1, \ldots, k\}$ under $w$, and let $\xi \co Z \to \{1, \ldots, k\}$ denote the unique
order-preserving bijection. Then $x(i) = \xi \circ w(i)$ for all $1\leq i\leq k$.%where the domain of $w$ is restricted to $\{1, \ldots, k\}$.

In the Robinson-Schensted algorithm, one constructs the tableau-pair $(P,Q)$ associated to $w$ one box at a time, by inserting $w(1)$, then $w(2)$, etcetera. At the $k$-th step in this
process, one has a pair $(P_k, Q_k)$ where $P_k$ is not a standard tableau (its entries are $Z$, not $\{1, \ldots, k\}$), while $Q_k$ is standard and agrees with $Q^k$. The only
difference between performing the first $k$ steps of this algorithm for $w$ and for $t$ is that the boxes in $P_k$ are relabeled via $\xi$; the recording tableau $Q_k$ is unchanged, and
only depends on the relative order of $w(1), \ldots, w(k)$. Hence $Q^k$ agrees with the recording tableau for $t$, as desired. \end{proof}

\begin{remark} In particular, this proof also implies that the tableau $B$ can be described in a straightforward way from $(P,Q)$ as well: by ``unbumping'' the extraneous boxes of $P$ in
the order determined by $Q$ to recover the non-standard tableau $P_k$, and then relabeling it via $\xi$. \end{remark}

\begin{notation} For a tableau $T$ with shape $\l$, we write $\shape(T) = \l$. For $k \le n$ and $w \in S_n$, write $\shape_{L,k}(w) = \shape(P^k)$, where $w$ corresponds to $(P,Q)$. Write $\shape_{R,k}(w) = \shape(Q^k)$. These are the shapes of $u$ and $t$ respectively. \end{notation}
	
In the opposite direction, there is an inclusion map $S_k \to S_{k+\ell}$ which sends $w \mapsto w \sqcup 1_{\ell}$. If $w$ corresponds to $(P,Q)$, then $w \sqcup 1_{\ell}$ corresponds to $(U,V)$, where $U$ is obtained from $P$ by adding the boxes $k+1, k+2, \ldots, k+\ell$ to the first row, and likewise for $V$ and $Q$.


%========================
\subsection{Relative KL cells and the relative action of involutions}
\label{subsec:relativecells}
%========================

Meinolf Geck in \cite{GeckRelative} laid down the framework for the study of relative cells in $\HB(W)$, relative to a parabolic subgroup $W_I$. We continue to use the notation $X_I$ and $Y_I$ from the previous section.

Geck defines a new partial order $\le_{L,I}$ on $W$, or on the set of KL basis elements $\{b_w\}_{w \in W}$, which is analogous to the left cell order $\le_L$ (see
\S\ref{subsec:algCells}) but for left multiplication by elements of the subalgebra $\HB(W_I) \subset \HB(W)$. Namely\footnote{Again, one takes the transitive closure of this relation.}, \[ b_x \le_{L,I} b_y \quad \textrm{if} \quad b_x \babysumset h \cdot
b_y \textrm{ for some } h \in \HB(W_I). \] One defines $\le_{R,I}$ and $\le_{LR,I}$ analogously.

We can equip $W_I$ with its usual partial order $\le_L$; this will not be confused with the corresponding order on $W$, since we are only interested in $\le_{L,I}$ in this section. We may equip $X_I$ and $Y_I$ with the partial order $\le$ induced from the Bruhat order. We state results for left multiplication by $W_I$ (hence right cosets); the results on the other side are analogous.

\begin{lemma} \label{lem:Geck1}(This is \cite[Proposition 4.4]{GeckRelative}) For $t,u \in W_I$ and $x,y \in Y_I$ one has
\begin{equation} \label{eq:Geck1} tx \le_{L,I} uy \quad \implies \quad t \le_{LR} u \quad \textrm{and} \quad x \le y. \end{equation}
Thus, if $tx \sim_{L,I} uy$ then $t \sim_{LR} u$ and $x = y$. \end{lemma}

One also has an analog of Proposition \ref{prop:leftincomp}: that left cells within a given two-sided cell are imcomparable.

\begin{lemma} \label{lem:Geck2} (This is \cite[Theorem 4.8]{GeckRelative}) For $t,u \in W_I$ and $x,y \in Y_I$, if $tx \le_{L,I} uy$ and $t \sim_{LR} u$ then $x = y$ and $t \sim_L u$.
\end{lemma}

We will not use these lemmas directly, but it gives an idea of how Geck's relative cells function. The key result we will use from Geck is the following. As before, let $c_{x,y}^z$ denote\footnote{This is denoted $h_{x,y,z}$ in \cite{GeckRelative}.} the coefficient of $b_z$ in the product $b_x b_y$.

\begin{lemma} \label{lem:Geck3} (This is \cite[Lemma 4.7]{GeckRelative}) For $t,u,w \in W_I$ and $x, y \in Y_I$ one has \begin{enumerate}
\item If $c_{w, ty}^{ux} \ne 0$ then $u \le_{LR} t$ and $x \le y$. (This is a restatement of Lemma \ref{lem:Geck1}.)
\item If $x=y$ then $c_{w,ty}^{uy} = c_{w,t}^{u}$.
\item Assume that $t \sim_{LR} u$ inside $W_I$, both living in cell $\mu$. The coefficient of $v^{-a}$ in $c_{w,ty}^{ux}$ is zero for any $a > \rbb(\mu)$, and if the coefficient of $v^{-\rbb(\mu)}$ is nonzero, then $x=y$. \footnote{The statement of \cite[Lemma 4.7]{GeckRelative} is slightly weaker than this, but the proof suffices to show this stronger result. Geck also considers positive exponents of $v$ rather than negative exponents, but these polynomials are self-dual.}
\end{enumerate} \end{lemma}

Let us rephrase this result using the language of tableau, in the special case when $W = S_n$ for $n = k + \ell$, and $W_I = S_k=S_k\times (S_1)^\ell$. First, let us introduce notation.

\begin{notation} \label{not:relideal} Let $k \le n$ and $\mu \in \PC(k)$. Let $\HB_{L,<\mu} \subset \HB = \HB(S_n)$ denote the span of those $b_w$ for which $\shape_{L,k}(w) < \mu$.
Define $\HB_{L,\le \mu}$ similarly. Let $\HB_{L,\mu}$ denote the span of those $b_w$ for which $\shape_{L,k}(w) = \mu$. Let $\HB^+_{L,\mu}$ denote the $\Z[v]$-span of $vb_w$ where
$\shape_{L,k}(w) = \mu$. \end{notation}

Note that, while $\HB_{L,<\mu}$ is not a left ideal, it is preserved under the left action of the subalgebra $\HB(S_k)$, thanks to Lemma \ref{lem:Geck1}.

Now let $w \in S_n$, and $w = ty$ for $t \in W_I$ and $y \in Y_I$. We now name the corresponding tableaux: let $w = w(P,Q,\l)$, and $t = w(P^k,A,\mu)$, so that $\mu = \shape_{L,k}(w)$. Finally, fix any other element $z \in W_I$ with shape $\mu$, so that $z = w(U,V,\mu)$. Let us compute $b_z b_w$.

\begin{prop} \label{prop:relaction} Fix $k < n$. Let $A, \l, \mu, P, Q, t, U, V, w, y, z$ be as above. Let $g = w(U,A,\mu) \in S_k$. Then
\begin{equation} v^{\rbb(\mu)} b_z b_w = \delta_{V,P^k} b_{gy} + \HB^+_{L,\mu} + \HB_{L,< \mu}.\end{equation}
\end{prop}

\begin{proof} By definition,
\begin{equation} b_z b_{ty} = \sum_{u \in W_I, x \in Y_I} c_{z,ty}^{ux} b_{ux}. \end{equation}
By part (1) of Lemma \ref{lem:Geck3}, the only terms appearing in this sum will have $u \le_{LR} t$ and $x \le y$. In particular, $\shape(u) = \rho \in \PC(k)$ with $\rho \le \mu$. Let us ignore those terms with $\rho < \mu$, and consider only those with $\rho = \mu$, i.e. $u \sim_{LR} t$.  Then no $b_{ux}$ appears with $v^{-a}$ for $a > \rbb(\mu)$, and $v^{-\rbb(\mu)}$ will appear only if $x = y$. Moreover, when $x = y$, $c_{z,ty}^{ux} = c_{z,t}^{u}$. But $b_z b_t = b_{U,V} b_{P^k,A} = \phi(V,P^k) b_{U,A}$, so that $c_{z,t}^{u} = 0$ unless $u = w(U,A,\mu) = g$. Now we can use \eqref{eq:howPQmult} to say that a coefficient with $v^{-\rbb(\mu)}$ will appear if and only if $V = P^k$.
\end{proof}

This leads to our relative version of \eqref{eq:dacts}.

\begin{cor} \label{cor:relactioninvolution} Fix $k < n$, and let $w \in S_n$ correspond to $(P,Q)$. Let $P^k$ have shape $\mu$, and fix another tableau $V \in \SYT(\mu)$. Then one has
\begin{equation} \label{eq:dactsrel} v^{\rbb(\mu)} (b_{V,V} \sqcup 1_{n-k}) b_w = \delta_{V,P^k} b_w + \HB^+_{L,\mu} + \HB_{L,< \mu}. \end{equation} \end{cor}
	
Multiplication formulas for KL basis elements in the Hecke algebra give rise to decompositions of tensor products in the category of Soergel bimodules.

\begin{cor}\label{cor:relativeUnit}
Suppose $w=w(P,Q,\l) \in S_n$, and fix $k < n$.  Let $\mu = \shape_{L,k}(w)$. Then $B_w$ is isomorphic to a direct summand of $(B_{P^k,P^k}(\rbb(\mu)) \sqcup \one_{n-k})\otimes B_w$. \qed
\end{cor}

\begin{proof} This follows immediately from \eqref{eq:dactsrel}. \end{proof}

Finally, we need one other result of Geck, whose proof uses separate ideas (mixing the KL basis and the standard basis) which we choose not to recall.

\begin{lemma} \label{lem:itsarightideal}
The subspace $\HB_{L,\le \mu}$ from Notation \ref{not:relideal} is a right ideal in $\HB(W)$. \end{lemma}

\begin{proof} This subspace appears in \cite[Corollary 3.4]{GeckInduction}, where it is shown to agree with a space Geck calls $\MC$. In \cite[Lemma 2.2]{GeckInduction}, it is proven that $\MC$ is a right\footnote{Geck is using left cosets rather than right cosets, so he obtains a left ideal.} ideal. \end{proof}

The consequence in the category of Soergel bimodules is that those indecomposables $B_w$ where $\shape_{L,k}(w) \le \mu$ (i.e. the categorification of $\HB_{L,\le
\mu}$) form a right tensor ideal. 

%========================
\subsection{Implications for induced projectors}
\label{subsec:implications}
%========================

Assuming that Theorem \ref{thm:typeAdiag} is proven for $S_k$, for any $\mu \in \PC(k)$ we refer to the idempotent $\PB_\mu \sqcup \one_{n-k}$ as an induced projector. Now we use the
results of the previous section to study induced projectors and their interaction with cell theory. 

\begin{lemma} \label{lem:relprojkills} Fix $k < n$. Assume Theorem \ref{thm:typeAdiag} is proven for $k$, and fix $\mu \in \PC(k)$. Then \begin{enumerate}
	\item the chain bimodules of $\PB_\mu \sqcup \one_{n-k}$ are direct sums of shifts of $B_w$ for $w \in S_n$ with $\shape_{L,k}(w) \le \mu$.
	\item $\PB_\mu \sqcup \one_{n-k}$ annihilates bimodules $B_w$ for $w \in S_n$ with $\shape_{L,k}(w) \ngeq \mu$.
\end{enumerate}
\end{lemma}

\begin{proof} We know by Theorem \ref{thm:typeAdiag} that $\PB_\mu$ annihilates indecomposable objects of $\SBim_{k}$ in cells $\ngeq \mu$. Let $w = w(P,Q,\l)$ with $\shape_{L,k}(\l) \ngeq \mu$. Thus
\begin{equation} (\PB_\mu \sqcup \one_{n-k}) \ot (B_{P^k,P^k} \sqcup \one_{n-k}) \ot B_w \simeq 0, \end{equation}
because the shape of $P^k$ is $\ngeq \mu$. However, $B_w$ is a summand of $(B_{P^k,P^k} \sqcup \one_{n-k}) \ot B_w$, by Corollary \ref{cor:relativeUnit}, so $(\PB_\mu \sqcup \one_{n-k}) \ot B_w \simeq 0$.

Now, every chain object of $\PB_\mu$ is in cells $\le \mu$ in $\SBim_k$. In particular, every indecomposable summand of a chain object of $\PB_\mu \sqcup \one_{n-k}$ is $B_w$ for $w \in S_k \subset S_n$ in cell $\nu \le \mu$. This is a much stronger result than merely asserting that $\shape_{L,k}(w) \le \mu$. Of course, $\nu = \shape_{L,k}(w)$ for such $w$.  \end{proof}

Now we prove the key lemma for our inductive proof of Theorem \ref{thm:typeAdiag}. We set up notation. Assume Theorem \ref{thm:typeAdiag} is proven for $n-1$, and fix $\mu \in \PC(n-1)$. Let $\l^1 \le \l^2 \le \ldots \le \l^r$ be the partitions obtained from $\mu$ by adding a box, ordered via the dominance order, so that $\l^r$ is obtained by adding a box to the first row, and $\l^1$ is obtained by adding a box to the first column. (Here we have used superscripts to avoid confusion with the parts of a partition.) We write $\l \supset \mu$ to indicate that $\l = \l^i$ for some $i$. Recall that the maps $\a_{\l^i}$ have been defined already in Theorem \ref{thm:lambdaMaps}. 

\begin{lemma} \label{lem:relprojcones} Assume Theorem \ref{thm:typeAdiag} is proven for $n-1$, and fix $\mu \in \PC(n-1)$. With notation as in the previous paragraph, the tensor product
\begin{equation}\label{eq:reltensorprod}
(\PB_{\mu}\sqcup \one_1) \otimes \bigotimes_{\l\supset \mu} \Cone(\a_\l) \simeq 0
\end{equation}
is contractible. \end{lemma}

Since the cones commute, this tensor product does not depend on the ordering of the factors. Note that $\PB$ is an idempotent, and commutes with each of these cones by Corollary \ref{cor:projectorconecommute}.


\begin{proof}
 Let us write $\PB$ for $(\PB_{\mu}\sqcup \one_1)$ during this proof.
\begin{subequations}
Let \begin{equation} \XB^r = \PB \ot \Cone(\a_{\l^r}), \end{equation} \begin{equation} \XB^{r-1} = \PB \ot \Cone(\a_{\l^r}) \ot \Cone(\a_{\l^{r-1}}), \end{equation} etcetera, so that $\XB^1$ is just the tensor product in \eqref{eq:reltensorprod}.
\end{subequations}

Given a collection of complexes $C_1,\ldots,C_r\in \KC^-(\SBim_n)$, let $\ip{C_1,\ldots,C_r} \subset \KC^-(\SBim_n)$ denote smallest full subcategory containing the $C_i$ and closed under grading shifts $(\pm 1)$, $[\pm 1]$, and locally finite convolutions.  Note that $\ip{B_w\:|\: \shape_{L,n-1}(w)\leq \mu}$ is a right tensor ideal in $\KC^-(\SBim_n)$ by Lemma \ref{lem:itsarightideal}, and $\PB\in \ip{B_w\:|\: \shape_{L,n-1}(w)\leq \mu}$ by Lemma \ref{lem:relprojkills}.  Thus every complex under consideration for the remainder of the proof will be in $\ip{B_w\:|\: w\in \shape_{L,n-1}(w)\leq \mu}$.


We prove by descending induction that
\begin{equation}\label{eq:thegoal1}
\XB^i\in \ip{\PB \ot B_w\:|\: \text{ $\shape_{L,n-1}(w)\leq \mu$ and $w$ is in some cell $< \l^i$}}.
\end{equation}
First observe that $\PB$ annihilates $B_w$ when $\shape_{L,k}(w) \ngeq \mu$, hence we need only consider those $w$ with $\sh_{L,n-1}=\mu$.  But this implies that $w$ is in cell $\l^j$ for some $j$, since removing a box yields $\mu$. Then $\l^j<\l^i$ implies $j<i$.  Thus \eqref{eq:thegoal1} is equivalent to
\begin{equation}\label{eq:thegoal2}
\XB^i\in \ip{\PB \ot B_w\:|\: \text{ $\shape_{L,n-1}(w) = \mu$ and $w$ is in some cell $\l^j$ for $j<i$}}
\end{equation}
We remark that if $i=1$, then this reduces to $\XB^1\simeq 0$, which is what we wish to prove.


For any indecomposable bimodule $B \in \SBim_{n-1}$ in cell $\mu$, $B \sqcup \one_1$ is in cell $\l^r$. Thus the chain objects of $\PB$ lie in cells $\leq
\l^r$. Tensoring with $\Cone(\a_{\l^r})$ takes objects in cells $\le \l^r$ to complexes in cells $< \l^r$. This proves that \eqref{eq:thegoal1} (equivalently \eqref{eq:thegoal2}) is satisfied for $\XB^r$.


Assume by induction that \eqref{eq:thegoal2} is satisfied by $\XB^{i+1}$.  Then
\[
\XB^i\in\ip{\PB\otimes B_w\otimes \Cone(\a_{\l^i})\:|\: \text{ $w$ is in some cell $\l^j$ for $j \le i$}}.
\]
Tensoring with $\Cone(\a_{\l^i})$ takes complexes  in cells $\le \l^i$ to complexes in cells $< \l^i$. We deduce that
\[
\XB^i\in\ip{\PB\otimes B_w\:|\: \text{$w$ is in some cell $<\l^j$ for $j \le i$}}.
\]
As remarked above, $\ip{\PB\otimes B_w\:|\: \shape_{L,n-1}(w)\leq \mu}$  is a right tensor ideal, hence \eqref{eq:thegoal1} (equivalently \eqref{eq:thegoal2}) is satisfied for $\XB^i$. 

Thus, $\XB^1$ satisfies \eqref{eq:thegoal1} by induction. There are no elements $w$ in cells $< \l^1$ for which $\shape_{L,k}(w) = \mu$, hence $\XB^1 \simeq 0$.  \end{proof}




% \begin{comment}
% \begin{proof} Let us write $\PB$ for $(\PB_{\mu}\sqcup \one_1)$ during this proof.
% \begin{subequations}
% Let \begin{equation} \XB^r = \PB \ot \Cone(\a_{\mu^r}), \end{equation} \begin{equation} \XB^{r-1} = \PB \ot \Cone(\a_{\mu^r}) \ot \Cone(\a_{\mu^{r-1}}), \end{equation} etcetera, so that $\XB^1$ is just the tensor product in \eqref{eq:reltensorprod}.
% \end{subequations}
%
% We prove by descending induction that $\XB^i$ can be built as a locally finite convolution of complexes of the form $\PB \ot B_w(a)[b]$, where $w$ satisfies two properties:
% \begin{enumerate} \item[(1)] $w$ is in some cell $< \mu^i$, and \item[(2)] $\shape_{L,k}(w) \le \l$. \end{enumerate} Before proving this, some preliminaries.
%
% We claim that, for any complex $\XB$, the condition that it is built from $\PB \ot B_w(a)[b]$ where $w$ satisfies (1) and (2) above is equivalent to saying that it is built from $\PB \ot B_w(a)[b]$ where $w$ satisfies the following two properties:
% \begin{enumerate} \item[(3)] $w$ is in cell $\mu^j$ for $j < i$, and \item[(4)] $\shape_{L,k}(w) = \l$. \end{enumerate}
%
% Replacing (2) with (4) is easy, since Lemma \ref{lem:relprojkills} states that $\PB$ annihilates $B_w$ when $\shape_{L,k}(w) \ngeq \l$. However, if $w$ is in cell $\nu$ and
% $\shape_{L,k}(w) = \l$, then $\nu \supset \l$ so that $\nu = \mu^j$ for some $j$. Since $\nu < \mu^i$ we must have $j < i$. Hence (3) replaces (1).
%
% These conditions are equivalent to proving that $\XB^i$ is built as a convolution of one-term complexes $B_w(a)[b]$, where $w$ satisfies properties (1) and (2). For sake of clarity, we
% call these conditions (1') and (2'). This is because $\PB$ is idempotent, so that $\PB \ot \XB^i \simeq \XB^i$, and being built from $B_w(a)[b]$ is equivalent to being built from $\PB
% \ot B_w(a)[b]$. However, it is not equivalent to prove that $\XB^i$ is built from $B_w(a)[b]$ where $w$ satisfies conditions (3) and (4), and this is false. After all, when
% $\shape_{L,k}(w) = \l$, $\PB \ot B_w$ has plenty of terms $B_z$ with $\shape_{L,k}(z) < \l$.
%
% Now, condition (2') is immediate, because $\PB$ itself is built from $B_w$ with $\shape_{L,k}(w) \le \l$ by Lemma \ref{lem:relprojkills}, and such indecomposables form a right tensor ideal by Lemma \ref{lem:itsarightideal}.
%
% For any indecomposable bimodule $B \in \SBim_{n-1}$ in cell $\l$, $B \sqcup \one_1$ is in cell $\mu^r$. Thus the chain objects of $\PB$ lie in cells $\leq
% \mu^r$. Tensoring with $\Cone(\a_{\mu^r})$ takes objects in cells $\le \mu^r$ to complexes in cells $< \mu^r$. Thus $\XB^r$ satisfies condition (1').
%
% Suppose properties (1) and (2), and hence (3) and (4), are satisfied for $\XB^{i+1}$. Then $\XB^i$ is built as a convolution of complexes of the form
% \begin{equation} \PB \ot B_w(a)[b] \ot \Cone(\a_{\mu^i}), \end{equation}
% where $\shape_{L,k}(w) = \l$ and $w$ is in a cell $\mu^j$ for $j \le i$. Tensoring with $\Cone(\a_{\mu^i})$ takes objects in cells $\le \mu^i$ to complexes in cells $< \mu^i$. We deduce property (1') for $\XB^i$.
%
% There are no elements $w$ in cells $< \mu^1$ for which $\shape_{L,k}(w) = \l$, implying that $\XB^1 \simeq 0$. \end{proof}
% \end{comment}

%========================
\subsection{Proof of Theorem \ref{thm:typeAdiag}}
\label{subsec:proof}
%========================


The proof is by induction on $n$.  The base case $n=1$ is trivial.  Assume by induction that we have constructed $\PB_{\mu}\in \KC^-(\SBim_{n-1})$ as in the statement.  In particular,
\[
\one_{n-1}\simeq \bigoplus_{\mu \in \PC(n-1)} \PB_{\mu} \qquad \text{ with twisted differential}.
\]
The twisted differential respects the opposite of the usual dominance order on partitions, meaning that the component of the differential from $\PB_\mu$ to $\PB_{\mu'}$ is zero unless $\mu\geq \mu'$.

Let us match the notation of the Relative Diagonalization Theorem \cite[Theorem 8.2]{ElHog17a}. Let $F = \FT_n$. Let $\XC$ (resp $\YC$) denote the set $\PC(n-1)$ (resp. $\PC(n)$) with its opposite poset structure. For $\mu \in \XC$, let $\YC_\mu \subset \YC$ denote the set $\{ \l \in \PC(n) \mid \l \supset \mu\}$. Then, Lemma \ref{lem:relprojcones} together with Lemma \ref{lemma:projectorSliding} and Theorem \ref{thm:lambdaMaps} give all the assumptions of the Relative Diagonalization Theorem.


Then \cite[Theorem 8.2 and following remarks]{ElHog17a} implies the following theorem.

\begin{theorem} \label{thm:relDiag1less} There is a diagonalization $\{(\PB_\l, \a_\l)\}$ of $\FT_n$ indexed by $\YC$, which is built by reassociating an idempotent decomposition $\{\PB_{\mu,\l}\}$ indexed by $\XC \times \YC$. We have
\begin{enumerate}\setlength{\itemsep}{2pt}
\item $\PB_\l \simeq \bigoplus_{\mu \in \XC} \PB_{\mu,\l}$ with twisted differential.
\item $\PB_{\mu,\l}\simeq 0$ unless $\mu\subset \l$.
\item If $\mu\subset \l$, then $\PB_{\mu,\l}$ is constructed as a locally finite convolution of shifts of complexes of the form
\begin{equation} \label{eq:doubleidemp}
(\PB_{\mu}\sqcup \one_1)\otimes \bigotimes_{\nu}\Cone(\a_\nu)
\end{equation}
where $\nu$ ranges over partitions containing $\mu$ but not equal to $\l$.
\end{enumerate}
Furthermore, the relative diagonalization is always \emph{relatively tight} in the sense that
\begin{enumerate}
\item[(4)] if $C\in \KC^-(\SBim_n)$ satisfies $(\PB_{\mu}\sqcup \one_1)\otimes C$ and $\Cone(\a_\l)\otimes C\simeq 0$ for $\mu\subset \l$, then $\PB_{\mu,\l}\otimes C\simeq C$ (and conversely).
\end{enumerate}
\end{theorem}

\begin{lemma} \label{lem:doubleidempcell}
$\PB_{\mu,\l}$ lives in cells $\leq \l$ and kills complexes in cells $\not\geq \l$.
\end{lemma}

\begin{proof}
We need only prove the same result for the tensor product in \eqref{eq:doubleidemp}, which we temporarily call $\XB_{\mu,\l}$

Let us resume the notation of the proof of Lemma \ref{lem:relprojcones}.  In particular, $\PB:=\PB_\mu\sqcup \one_1$.  Let $\l = \l^i$. By \eqref{eq:thegoal2},
\[ \XB^{i+1} \in \ip{\PB \ot B_w\:|\: \text{ $w$ is in some cell $\le \l$}}. \] In particular, this implies that
\[ \XB^{i+1} \in \ip{B_w\:|\: \text{ $w$ is in some cell $\le \l$}}. \] Since $\XB_{\mu,\l}$ is obtained by taking $\XB^{i+1}$ and tensoring with additional cones, it is also in cells $\le \l$, as desired.

Because $\XB_{\mu,\l} \simeq \XB_{\mu,\l} \ot \PB$, we need to prove that $\XB_{\mu,\l} \ot \PB \ot B_w \simeq 0$ whenever $w$ is in cell $\nu$ with $\nu \ngeq \l$. In fact, we can assume that
$\shape_{L,k}(w) \le \mu$. This is because any indecomposable summand in $\PB \ot B_w$ is $B_z$ with $\shape_{L,k}(z) \le \mu$, by Lemma \ref{lem:relprojkills}, and $\PB \ot B_w \simeq
\PB \ot \PB \ot B_w$. But by Lemma \ref{lem:relprojkills}, we may assume $\shape_{L,k}(w) \ge \mu$ or else $\PB \ot B_w \simeq 0$. Hence $\shape_{L,k}(w) = \mu$, meaining that $\nu =
\l^j$ for some $j$. Then $\nu \ngeq \l$ implies $j < i$. The same inductive argument as Lemma \ref{lem:relprojcones} will prove that the cones $\Cone(\a_{\l^k})$ for $k \le j$ will
suffice to kill $\PB \ot B_w$, meaning that $\XB_{\mu,\l} \ot \PB \ot B_w \simeq 0$ as well. \end{proof}


Finally, since $\PB_\l$ is obtained by associating together the complexes $\PB_{\mu,\l}$ with $\mu\subset \l$, it shares the properties stated in Lemma \ref{lem:doubleidempcell}. This
completes the proof of Theorem \ref{thm:typeAdiag}.

%========================
\subsection{The primitive idempotents}
\label{subsec:primitives}
%========================

\begin{definition} \label{defn:PS}
Let $S=(\l_S^{1},\ldots,\l_S^{n})\in \PC(1)\times \cdots \times \PC(n)$ be a sequence of partitions with $\l_S^{i}\in\PC(i)$.  Set
\[
\PB_S := \bigotimes_{i=1}^n (\PB_{\l_S^i}\sqcup \one_{n-i}).
\]
Note that $\PB_{\l_S^1}=\one$, hence can be removed from this tensor product with no effect.
\end{definition}

Each $\PB_{\l_S^i}$ is central in $\KC^b(\SBim_i)$, so while $(\PB_{\l_S^i}\sqcup \one_{n-i})$ is not central, it commutes with $(\PB_{\l_S^j}\sqcup \one_{n-j})$ for $j<i$. Hence the entire family commutes, and the ordering of the above tensor product is irrelevant up to homotopy.

Let $\SYT(n)$ denote the set of standard tableaux with $n$ boxes. Given $T \in \SYT(n)$, we have already defined the tableau $T^k \in \SYT(k)$ in
\S\ref{subsec:subtableaux}. Let $\l_T^k = \shape(T^k)$. Hence $T$ gives rise to a sequence of partitions $(\l_T^1, \l_T^2, \ldots, \l_T^n)$, which determines $T$ uniquely. Using this
injective map $\SYT(n) \to \PC(1) \times \cdots \times \PC(n)$, we define $\PB_T$ as above. Conversely, a sequence of partitions corresponds to a tableau if and only if $\l^i \subset
\l^{i+1}$ for all $i$. Note that we do not yet assume in Definition \ref{defn:PS} that $S$ corresponds to a standard tableau.

\begin{definition}
Given $S,S'\in \PC(1)\times \cdots \times \PC(n)$ we write $S\leq T$ if $\l_S^k\leq \l_{S'}^k$ in the dominance order, for all $1\leq k\leq n$.  We refer to the induced partial order on $\SYT(n)$ as the dominance order on standard tableaux.
\end{definition}

Note that the dominance order on standard tableaux does not agree with the Kazhdan-Lusztig left cell order (or right cell order), when one identifies a tableau $T$ with the corresponding
left cell $(-,T)$.

\begin{example} The first example where the dominance order on tableaux disagrees with the KL left cell order is in $S_3$, for the tableau $ ((1,3),(2)) < ((1,2),(3)) $. These tableaux have the same shape, but comparable left cells can not exist in the same two-sided cell. \end{example}

\begin{example} The first example where the dominance order on tableaux disagrees with the KL left cell order for left cells in distinct two-sided cells is in $S_4$, for the tableaux $P
= ((1,4),(2),(3)) < ((1,2,3),(4)) = Q$. The left cell of $Q$ is generated by $b_u$, and any element $b_w$ in this cell will have $u$ in its right descent set. The left cell of $P$ is
generated by $b_{sts}$, and any element $b_w$ in this cell will have $s$ and $t$ in its right descent set. Therefore, these left cells intersect only in $b_{w_0}$, which is in a strictly
smaller left cell than both of them. Hence the left cells are incomparable. \end{example}



\begin{remark} In \cite{Taskin} various partial orders on $\SYT(n)$ are compared, including the cell order. None of the orders in that paper is the dominance order we consider here; the dominance order in \cite{Taskin} also involves comparing skew tableau within a tableau. No two tableaux of the same shape can be comparable in any of the orders in \cite{Taskin}, while it is important that they be comparable here. \end{remark}

\begin{theorem}\label{thm:PTprops}
We have:
\begin{enumerate}\setlength{\itemsep}{2pt}
\item $\PB_S\not \simeq 0$ if and only if $S$ is a standard tableau.
\item If $T$ is a standard tableau, then
\begin{enumerate}\setlength{\itemsep}{2pt}
\item  $\PB_T$ is constructed from bimodules $B_{P,Q}$ with $P,Q\leq T$.
\item  $B_{P,Q}\otimes \PB_T \simeq 0$ unless $Q\geq T$.
 \item $\PB_T\otimes B_{P,Q}\simeq 0$ unless $P\geq T$.
 \end{enumerate}
 \item There is a decomposition of identity $\one\simeq (\bigoplus_T \PB_T, d)$ in which the differential respects the reverse of the dominance order.
 \item A complex $C\in \KC^-(\SBim_n)$ satisfies $\Cone(\a_{T^k})\otimes C\simeq 0$ for all $1\leq k\leq n$ if and only if $\PB_T\otimes C\simeq C$ (and similarly for tensoring with $C$ on the left).  Thus $\PB_T$ projects onto the joint $(\a_{T^1},\ldots,\a_{T^n})$-eigencategory of $\FT_1,\ldots,\FT_n$.
\end{enumerate}
\end{theorem}

Statement (3) deserves a bit more explanation.  The claim is that there is a differential $d$ on $\bigoplus_T \PB_T$ such that the component $d_{T,U}$ from $\PB_U$ to $\PB_T$ vanishes unless $U\geq T$, $d_{T,T}$ is the given differential on $\PB_T$, and the resulting complex is homotopy equivalent to the monoidal identity.

\begin{proof} 
Recall the relative idempotents $\PB_{\mu,\l}\in \KC^-(\SBim_n)$, indexed by pairs $(\mu,\l)\in \PC(n-1)\times \PC(n)$ with $\mu\subset \l$ (\S \ref{subsec:proof}).  Then $\PB_{\mu}\sqcup \one_1$ is a convolution of idempotents of the form $\PB_{\mu,\l}$ where $\l$ ranges over the partitions of $n$ containing $\mu$, while $\PB_\l$ is a convolution of idempotents $\PB_{\mu,\l}$ where $\mu$ ranges over the paritions of $n-1$ contained in $\l$.  If $\l'\in \PC(n)$ does not contain $\mu\in \PC(n-1)$, then $(\PB_{\mu}\sqcup \one_1) \otimes \PB_{\l'}$ is contractible, since it is a convolution of complexes $\PB_{\mu,\l}\otimes \PB_{\mu',\l'}$ with $\mu\neq \mu'$, and these idempotents are orthogonal.  This proves the ``only if'' direction of (1).

Statement (2) is an immediate consequence of Lemma \ref{lem:relprojkills}.  Finally, statement (3) follows from tensoring together the decompositions of identity $\one\simeq (\bigoplus_{\mu\in \PC(k)} \PB_{\mu}\sqcup \one_{n-k},d)$ for $1\leq k\leq n$, and contracting the terms which are contractible via the ``only if'' direction of (1).

Fix $T \in \SYT(n)$. If there were a complex $C \in \KC^-(\SBim_n)$ such that $\PB_U \otimes C \simeq 0$ for all $U \ne T$, then one must have $\PB_T \otimes C \simeq C$, using the decomposition of identity in (3).  If $C\not \simeq 0$, then this would imply $\PB_T\not\simeq 0$, giving the ``if" direction of (1). The bounded quasi-idempotent complex $\KB_T$, constructed in the next section, will serve as such a complex $C$.

Statement (4) is an implication of statement (4) in Theorem \ref{thm:relDiag1less}.
\end{proof}

The observations made following the statement of Theorem \ref{thm:typeAdiag} have analogues here as well: for each convex subset of $K\subset \SYT(n)$ we may define a subquotient idempotent $\PB_K$, just as in Definition \ref{def:PK}.  If $I\subset \SYT(n)$ is a poset ideal, then $\PB_I$ is has the structure of a counital idempotent with complementary unital idempotent $\PB_{I^c}$, where $I^c=\SYT(n)\setminus I$.  We have the following characterization of the counital idempotents constructed in this way.  We only state it only for $\PB_{\leq T}$, to avoid a potentially confusing conflict of notation with Proposition \ref{prop:counitalChar}.


\begin{proposition}\label{prop:PTleqchar}
Let $T\subset \SYT(n)$ be given. The idempotent $\PB_{\leq T}$ satisfies:
\begin{enumerate}
\item $\PB_{\leq T}$ is constructed from bimodules $B_{P,Q}$ with $P,Q\leq T$.
\item there is a chain map $\e:\PB_{\leq T}\rightarrow \one$ such that $\Cone(\e)\otimes B_{P,Q}\simeq 0$ when $P\leq T$ and $B_{P,Q}\otimes \Cone(\e)\simeq 0$ when $Q\leq T$.
\end{enumerate}
Furthermore,
\begin{enumerate}
\item[(3)] a complex $C\in \KC^-(\SBim_n)$ satisfies $\PB_{\leq T}\otimes C\simeq C$ (resp.~$C\otimes \PB_{\leq T}\simeq C$) if and only if $C\in \ip{B_{P,Q}\:|\: P\leq T}$ (resp.~$C\in \ip{B_{P,Q}\:|\: Q\leq T}$).
\item[(4)] properties (1) and (2) characterize the pair $(\PB_{\leq T},\e)$ up to canonical isomorphism in $\KC^-(\SBim_n)$.
\end{enumerate}
\end{proposition}

\begin{proof}
Essentially identical to the proof of Proposition \ref{prop:counitalChar}. 
\end{proof}
 
The idempotent $\PB_T$ itself can be characterized as the ``difference'' between $\PB_{\leq T}$ and $\PB_{<T}$ (see \S 5.2 of \cite{Hog17a}).

We now say a few words on the construction of $\PB_T$.  First, if $\mu,\l\in \PC(n)$ are comparable, recall the complexes $\CB_{\mu,\l}=\CB_{\a_\mu,\a_\l}(\FT_n)\in \KC^-(\SBim_n)$ from, e.g., see \S 7.2 of \cite{ElHog17a}.  We remind the reader that $\l$ and $\mu$ denote scalar objects /functors in \cite{ElHog17a}, and a shifted copy of $\one$ in $\KC^-(\SBim_n)$ is \emph{small} if it is supported in strictly negative homological degrees.  We continue to let $\l,\mu$ denote partitions, and $\Sigma_\l,\Sigma_\mu$ denote the corresponding scalar objects (shifts of the identity).  If we have partitions $\mu<\l$ in the dominance order then the shift $\Sigma_\l\Sigma_\mu\inv$ is small.  The important properties of the complexes $\CB_{\mu,\l}$ are the existence of distinguished triangles
\begin{subequations}
\begin{equation}
\CB_{\mu,\l}\rightarrow \one\rightarrow \CB_{\l,\mu}\rightarrow \CB_{\mu,\l}[1],\qquad\qquad (\mu<\l)
\end{equation}
\begin{equation}\label{eq:ctriang}
\Sigma_\l \CB_{\l,\mu}\rightarrow \Sigma_\mu \CB_{\l,\mu} \rightarrow \Cone(\a_{\l}) \rightarrow \Sigma_\l \CB_{\l,\mu}[1].
\end{equation}
\end{subequations}

The following is a direct application of Lemma 8.1 in \cite{ElHog17a}.
\begin{lemma}\label{lemma:PTinduction}
Fix $T\in \SYT(n)$ be a tableaux with shape $\l\in \PC(n)$, and let $U = T^{n-1} \in \SYT(n-1)$ with shape $\mu$.  Then
\[
\PB_T \simeq (\PB_U\sqcup \one_1)\otimes \bigotimes_{\nu\neq \l} \CB_{\nu,\l}.
\]
Here the tensor product is over all partitions $\nu\in \PC(n)$ with $\mu\subset \nu$ and $\nu\neq \l$.\qed
\end{lemma}
We remark that the complexes $\CB_{\nu,\l}$ commute with one another, given the vanishing of the obstructions in the family of eigenmaps $\{\a_\l\}_{\l\in \PC(n)}$ and Proposition A.17 in \cite{ElHog17a}.

This relationship between $\PB_T$ and $\PB_U$ may be viewed as a categorification of a well known relationship between $p_T$ and $p_U$ in the Hecke algebra $\HB_n$.  See \cite[Equation (11)]{IMO}.

%========================
\subsection{Finite quasi-idempotents}
\label{subsec:quasiidemp}
%========================

Now we consider a collection of finite complexes $\KB_T$ which are closely related to the idempotents $\PB_T$. They are defined in analogy with Lemma \ref{lemma:PTinduction} which (given this definition) will imply that $\PB_T$ is a locally finite convolution constructed from shifted copies of $\KB_T$.

\begin{definition}\label{def:KT}
Define complexes $\KB_T\in \KC^b(\SBim_n)$, $T\in \SYT(n)$, inductively by
\begin{subequations}
\begin{equation}\label{eq:K1}
\KB_{\square}=\one_1,
\end{equation}
\begin{equation}
\KB_T := (\KB_U\sqcup \one_1)\otimes \bigotimes_{\nu\neq \l} \Cone(\a_{\nu}).
\end{equation}
\end{subequations}
Here, $\shape(T) = \l\in \PC(n)$, $U = T^{n-1} \in \SYT(n-1)$, and $\nu$ runs over the partitions of $n$ such that $\mu\subset \nu$ and $\nu\neq \l$. 
\end{definition}

One should think that $\KB_T$ is the tensor product of all the $\Cone(\a_\nu)$ for partitions one could have taken but did not, along the path in the Young lattice which constructs $T$. That is,
\begin{equation} \label{eq:KTdef} \KB_T \simeq \bigotimes_{k=2}^n \bigotimes_{\substack{\nu \in \PC(k) \\ \nu \supset \l_T^{k-1} \\ \nu \ne \l_T^k}} \Cone(\a_{\nu}). \end{equation}

\begin{example} Let $T = {\: \Yvcentermath1 \young(12,34)}$. Then \[ \KB_T \cong \Cone(\a_{\yoo}) \ot \Cone(\a_{\yh}) \ot \Cone(\a_{\yho}) \ot \Cone(\a_{\ytoo}).\] \end{example}

\begin{lemma}\label{lemma:KabsorbsP}
We have $\PB_T\otimes \KB_T\simeq \KB_T\simeq \KB_T\otimes \PB_T$.
\end{lemma}
\begin{proof}
We focus on the equivalence $\KB_T\simeq \KB_T\otimes \PB_T$; the other is similar.  By (4) of Theorem \ref{thm:PTprops} we must show that $\KB_T\otimes \Cone(\a_{T^k})\simeq 0$ for all $1\leq k\leq n$.  This we prove by induction.  Let $U = T^{n-1}$. Then $\PB_U\sqcup\one_1$ commutes with $\Cone(\a_\l)$ for all $\l\in \PC(n)$ by Corollary \ref{cor:projectorconecommute}. From the definition of $\KB_T$ and the fact that $\PB_U\sqcup \one_1$ is idempotent it follows that
\begin{equation}
\KB_T \otimes (\PB_U\sqcup \one_1)\simeq \KB_T.
\end{equation}
Thus $\KB_T\otimes(-)$ annihilates $\Cone(\a_{T^k})$ for $1\leq k\leq n-1$, since the same is true of $\PB_U\sqcup \one_1$.  The fact that $\KB_T\otimes \Cone(\a_{T^n})\simeq 0$ follows from Lemma \ref{lem:relprojcones}.  This completes the proof.
\end{proof}

\begin{cor} \label{cor:KBtoo} Part (2) of Theorem \ref{thm:PTprops} also applies with $\KB_T$ replacing $\PB_T$. \qed \end{cor}

\begin{cor} If $U \ne T$ in $\SYT(n)$ then $\PB_U \ot \KB_T \simeq 0 \simeq \KB_T \ot \PB_U$. \qed \end{cor}

\begin{example}
If $T\in \SYT(n)$ is the unique one-row tableau, then $\PB_T=\PB_n$ is the categorified Jones-Wenzl idempotent and $\KB_T=\KB_n$ is the associated finite quasi-idempotent, both constructed in \cite{Hog15} (see also \cite{ElHog16a}).  Both of these complexes annihilate $B_w$ for $w\neq 1$.
\end{example}

\begin{example} \label{ex:Konecolumn}
If $T\in \SYT(n)$ is the unique one-column tableau, $\PB_T=\PB_{1^n}$ is the idempotent considered in \cite{AbHog17}.  The finite quasi-idempotent $\KB_T$ is homotopy equivalent to the $n-1$ dimensional Koszul complex associated to the action of $\a_1\otimes 1-1\otimes \a_1,\ldots,\a_{n-1}\otimes 1 -1\otimes \a_{n-1}$ on $B_{w_0}\in \SBim_n$, where $\a_i=x_i-x_{i+1}$.  We don't need or prove this fact, though it can be deduced from results in \cite{AbHog17}.  In any case, both $\PB_{1^n}$ and $\KB_{1^n}$ are built from $B_{w_0}$.
\end{example}


We will soon prove statements about $\PB_T$ in the Grothendieck group. This is subtle, because one must choose an appropriate subcategory of $\KC^-(\SBim_n)$ to get a non-zero Grothendieck group. However, $\KB_T$ is a bounded complex, so it has a well-defined image in the triangulated Grothendieck group $[\KC^b(\SBim_n)] \cong \HB_n$, which we now discuss.

Let us quickly recall how quasi-idempotents and idempotents are constructed when diagonalizing an operator. If $f$ is a linear operator with eigenvalues $\{\k_i\}_{i=0}^r$, and $\prod_i (f - \k_i) = 0$, then the projection to the $\k_i$ eigenspace is given by the formula
\begin{equation} p_i = \prod_{j \ne i} \frac{f - \k_j}{\k_i - \k_j}. \end{equation}
In particular, the element
\begin{equation} k_i = \prod_{j \ne i} (f - \k_j) \end{equation}
is a quasi-idempotent, and the scalar $\g = \prod_{j \ne i} (\k_i - \k_j)$ gives the proportionality between $k_i$ and $p_i$.

For example, \cite[Equation (11)]{IMO} uses this idea to inductively construct projections $p_T$ (which they denote $E_T$). One can inductively construct $k_T$ analogously, using only
the numerators in \cite[Equation (11)]{IMO}, and it is the obvious decategorification of Definition \ref{def:KT}. In particular, $[\KB_T] \mapsto k_T$ under the identification of the
Grothendieck group of $\KC^b(\SBim_n)$ with $\HB_n$.

\begin{cor} For all $T \in \SYT(n)$, $\KB_T$ is not contractible, and its image under the isomorphism of Grothendieck groups $[\KC^b(\SBim_n)]\cong \HB$ is a scalar multiple of $p_T$. \qed \end{cor}

%========================
\subsection{Grothendieck group considerations}
\label{subsec:groth}
%========================


If $C\in \KC(\SBim_n)$ is a minimal complex of Soergel bimodules, let $\supp(C)\subset \Z\times \Z$ denote the set of $(i,j)$ such that the chain bimodule $C^i$ has a summand of the form $B_w(-j)$.  Note that $\supp(C)[k](l) = (-k,-l)+\supp(C)$. If $C$ is not a minimal complex, we let $\supp(C):=\supp(C')$, where $C'\simeq C$ is minimal.  Since minimal complexes are unique up to isomorphism, this is well-defined.  Observe that
\[
\supp(\Sigma_\l)=\{(-2\cbb(\l),2\xbb(\l))\}.
\]

Let $\l,\nu\in \PC(n)$ be a pair of distinct partitions such that there exists $\mu\in \PC(n-1)$ with $\mu\subset \l$ and $\mu\subset \nu$.  In other words, $\l$ and $\nu$ become equal after deleting a single box.  Then $\l$ and $\nu$ are automatically comparable. Assume $\l>\nu$.  The set of such pairs $(\l,\nu)$ will be denoted $\QC(n)$.  If $(\l,\nu)\in \QC(n)$ then the box $\l\setminus \mu$ has larger content and is in a larger column than the box $\nu\setminus \mu$.   Thus, the vector
\[
\vb_{\l,\nu}:= (-\cbb(\l),\xbb(\l))-(-\cbb(\nu),\xbb(\nu))\in\Q\times \Q
\]
lies in the interior of the second quadrant.  Let $D(n)\subset \Q\times \Q$ denote the smallest convex subset containing $\Q_{\geq 0}\vb_{\l,\nu}$ for each $(\l,\nu)\in \QC(k)$ with $2\leq k\leq n$.  Note that $D(n)$ is an ``angle shaped'' region, bounded by two rays from the origin.

For each pair of subsets $E,E'\subset \Q\times \Q$, we have $E+E':=\{\eb+\eb'\:|\: \eb\in E, \eb'\in E'\}$ as usual.  For a subset $D\subset \Q\times \Q$ we write $E\subset \OC(D)$ if there is a finite set $E'$ such that $E\subset E'+D$.  Let $\KC^{\angle}(\SBim_n)\subset\KC^-(\SBim_n)$ denote the full subcategory consisting of complexes $C$ such that $\supp(C)\subset \OC(D(n))$.

Note that $\KC^{\angle}(\SBim_n)$ contains $\KC^b(\SBim_n)$ as a full subcategory.

\begin{lemma}
The category $\KC^{\angle}(\SBim_n)$ is closed under tensor products, direct sums, mapping cones, and shifts.
\end{lemma}
\begin{proof}
Closure under tensor products follows from the (easy) fact that $\supp(C\otimes C')\subset \OC(\supp(C)+\supp(C'))$ for all $C,C'\in \KC^-(\SBim_n)$ and $D(n)$ is closed under addition.  The other properties are clear.
\end{proof}

\begin{lemma}
The idempotents $\PB_T$ and $\PB_\l$ are in $\KC^{\angle}(\SBim_n)$ for all $T\in \SYT(n)$ and all $\l\in \PC(n)$.
\end{lemma}
\begin{proof}
Since $\PB_\l$ is a finite convolution involving the complexes $\PB_T$, it suffices to show that $\PB_T\in \KC^{\angle}(\SBim_n)$.  Recall the construction \ref{lemma:PTinduction}, which states that $\PB_T$ is a tensor product of $(\PB_U \sqcup \one_1)$ and of $\CB_{\nu,\l},\CB_{\l,\nu}\in \KC^-(\SBim_n)$ for various $(\l,\nu)\in\QC(k)$.  Thus, it suffices to show each of these is in $\KC^{\angle}(\SBim_n)$. By induction, $\PB_U \in \KC^{\angle}(\SBim_n)$. There is a distinguished triangle \eqref{eq:ctriang} relating $\CB_{\nu,\l}$,$\CB_{\l,\nu}$, and $\one$, so it suffices to show that $\CB_{\l,\nu}\in\KC^{\angle}(\SBim_n)$ for all $(\l,\nu)\in \QC(k)$ with $2\leq k\leq n$.

 Now we refer the explicit description of these complexes given in \cite{ElHog17a} (see Lemma 7.15).  If $\l>\nu$ then recall that $\Sigma_\l\Sigma_\mu\inv$ is supported in strictly negative homological degrees, hence is small in $\KC^-(\SBim_n)$, in the language of \cite{ElHog17a}.  Then
 \[
 \CB_{\l,\nu} = \Sigma_\nu \Cone(\a_\l)\otimes \bigoplus_{k\geq 0} (\Sigma_\l\Sigma_\nu\inv)^k \qquad\text{ with twisted differential}.
 \]
 It follows that $\CB_{\l,\nu}\in \KC^{\angle}(\SBim_n)$.  This completes the proof.
\end{proof}

\begin{lemma} \label{lem:grothPvK}
For each $T\in \SYT(n)$ there exists a scalar $\g_T\in \Z[v,v\inv]$ such that $p_T = (\g_T)\inv k_T\in \HB_n^{\Q(v)}$ and $[\PB_T] = (\g_T)\inv [\KB_T]$ in $[\KC^{\angle}(\SBim_n)]$.
\end{lemma}

\begin{proof}
Fix a tableau $T\in \SYT(n)$.  Define a scalar $\g_T$ inductively by $\g_{\square}=1$ and 
\[
\g_T =  \g_U \cdot \prod_{\nu\neq \l}(v^{2\xbb(\l)} - v^{2\xbb(\nu)})
\]
where $\l$ is the shape of $T$, $U=T^{n-1}$, and $\nu$ ranges over the partitions of $n$ with $\mu\subset \nu$ and $\nu\neq \l$. 

Let $\nu^k\in \PC(k)$ be a partition.   Then we have the eigenmap $\a_{\nu^k}:\one_n\rightarrow \FT_k\sqcup \one_{n-k}$ and its mapping cone:
\[
(\Sigma_{\nu^k}[-1]\rightarrow \FT_k\sqcup \one_{n-k}).
\]
Tensoring (say, on the right) with $\PB_T$ gives a complex
\[
(\Sigma_{\nu^k}\PB_T[-1]\rightarrow \Sigma_{\l_T^k}\PB_T),
\]
which can be viewed as the (the shift $\Sigma_{\l_T^k}$ applied to the) mapping cone on an endomorphism of $\PB_T$ of degree $\Sigma_{\nu^k}\Sigma_{\l_T^k}\inv$.  This endomorphism will be denoted by $\frac{\a_{\nu^k}}{\a_{\l_T^k}}\in \Endgg(\PB_T)$, as in \cite[\S 7.4]{ElHog17a}.

To avoid copious notation, we will let the partitions $\nu^k \in \PC(k)$ below come from the same set as in the tensor product of \eqref{eq:KTdef}. We have
\begin{eqnarray*}
\KB_T &\simeq & \KB_T\otimes \PB_T\\
& =& \left(\bigotimes_{k=2}^n \bigotimes_{\nu^k} \Cone(\a_{\nu^k})\right)\otimes \PB_T\\
& \simeq & \left(\bigotimes_{k=2}^n \bigotimes_{\nu^k} \Cone(\a_{\nu^k})\otimes \PB_T\right)\\
& \simeq & \bigotimes_{k=2}^n \bigotimes_{\nu^k} \Sigma_{\l_T^k}\Cone\left(\frac{\a_{\nu^k}}{\a_{\l_T^k}}\right).\\
\end{eqnarray*}
In the first equivalence we used Lemma \ref{lemma:KabsorbsP}.  The second is by \eqref{eq:KTdef}.  The third uses the fact that $\PB_T$ is idempotent and commutes with each $\Cone(\a_{\nu^k})$, and the last equivalence follows from the comments made above.  It follows from this that
\[
[\KB_T]=\g_T [\PB_T].
\]
\end{proof}

\begin{proposition}\label{prop:groth}
The completed Grothendieck group \cite{AchStr} of $\KC^{\angle}(\SBim_n)$ is isomorphic to $\HB_n\otimes_{\Z[v,v\inv]} \Z[v]\llbracket v\inv \rrbracket$.  Under this isomorphism the class of $\PB_T$ gets sent to $p_T$.
\end{proposition}

\begin{proof}
In the completed Grothendieck group, the class of a complex is determined by its Euler characteristic, viewed as a Laurent series with coefficients in the usual Grothendieck group $[\KC^b(\SBim_n)]\cong \HB_n$.  This gives the first statement, modulo standard details.  The second statement follows from Lemma \ref{lem:grothPvK}.
\end{proof}

