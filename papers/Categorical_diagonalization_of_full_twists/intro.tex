%%%%%%%%%%%%%%%%%%%%%%%%%
\section{Introduction}
\label{sec:intro}
%%%%%%%%%%%%%%%%%%%%%%%%%

In this paper we categorify some of the most important objects in the representation theory of symmetric groups: the Young idempotents. The classical Young idempotents are a family of
idempotent elements $p_T\in \Q[S_n]$, indexed by standard Young tableaux $T$ with $n$ boxes. These idempotents feature prominently in the representation theory of $S_n$, as they
decompose the regular representation into irreducible representations. In addition, these idempotents have deformations (also denoted $p_T$) in the Hecke algebra which are essential in
the construction of the colored HOMFLY-PT polynomial for links.

We categorify $p_T$ using categorical linear algebra. Namely, we apply the theory of categorical diagonalization, developed in \cite{ElHog17a}, to the full twist Rouquier complexes in
the homotopy category of Soergel bimodules.

Let us now motivate and explain these results, and their conjectural generalizations to other Coxeter groups.

%========================
\subsection{Representation theory of symmetric groups}
%========================

Fix a natural number $n$. For $\l$ a partition of $n$, let $\SYT(\l)$ denote the set of standard Young tableaux of shape $\l$. The irreducible representations $\SM^\l$ of the symmetric
group $S_n$ over $\Q$ are indexed by $\PC(n)$, and the dimension of $\SM^\l$ is equal to the size of $\SYT(\l)$. These irreducibles $\SM^\l$ are often called \emph{Specht modules}, after
Specht \cite{Specht35} (see also Peel \cite{Peel75}) who constructed integral forms of these representations (i.e. modules over $\Z[S_n]$ which become irreducible after base change to
$\Q$).

Young constructed combinatorially a family of operators $k_T \in \Z[S_n]$ indexed by standard Young tableaux $T$ with $n$ boxes. These operators are quasi-idempotent, in that $k_T^2 =
\g_T k_T$ for some scalar $\g_T \in \Z$. Dividing by $\g_T$, one obtains idempotents $p_T \in \Q[S_n]$ which are called \emph{Young idempotents} or \emph{Young symmetrizers}, and satisfy
$\Q[S_n] p_T \cong \SM^\l$ when $T \in \SYT(\l)$. These form a complete family of primitive orthogonal idempotents in $\Q[S_n]$. See \cite{FultonTab} for more details.

The presence of Young idempotents indicates some special feature of the symmetric groups. Using the theory of semisimple algebras, one has canonical central idempotents $p_\l \in \Q[S_n]$ which project to the isotypic component of $\SM^\l$ in $\Q[S_n]$.  That is, the image of $p_\l$ is $\#\SYT(\l)$ copies of $\SM^\l$; said another way,
\begin{equation}
p_\l = \sum_{T \in \SYT(\l)} p_T.
\end{equation}
Projection to a single irreducible component within this isotypic component is not canonical, and in the context of semi-simple algebras it is atypical to have a preferred choice of such projections (e.g. $p_T$).  However, symmetric groups are peculiar in that they fit in a tower
\begin{equation} S_1 \subset S_2 \subset \cdots \subset S_n \end{equation}
of groups, from which one obtains inclusions of group algebras.  A tableau $T \in \SYT(\l)$ corresponds to a tower $\l^1 \subset \cdots \subset \l^n = \l$ of partitions, with $\l^k$ a partition of $k$ for each $k < n$. Then
\begin{equation} p_T = \prod_{k=1}^n p_{\l^k}. \end{equation}

Although the ideas are older (see e.g. \cite{Jucys, Murphy,DipperJamesIdemp}), an influential modern perspective on these projections can be found in the work of Okounkov-Vershik \cite{OV96}.
The centralizer of $\Q[S_k]$ inside $\Q[S_{k+1}]$ is generated by a single element, the \emph{Young-Jucys-Murphy (YJM) element} $j_{k+1}$. It follows that, when inducing an irreducible
representation of $S_k$ to $S_{k+1}$, projection to its isotypic components agrees with projection to the eigenspaces of $j_{k+1}$ (and moreover the isotypic components are irreducible).
By simultaneously diagonalizing the commuting family $\{j_1, j_2, \ldots, j_n\}$, one can construct the projectors $p_T$. The Okounkov-Vershik paper gives a particularly nice proof that the simultaneous eigenvalues of the YJM operators are in bijection with standard tabelaux. The eigenvalue of $j_k$ on $p_T$ is the \emph{content number} (i.e. column
number minus row number) of the $k$-th box of $T$.

Let us quickly restate the construction of $p_T$ using the previous paragraph. Suppose one has constructed idempotents $p_U$ for standard tableaux with $n-1$ boxes. Then letting $p_U \sqcup 1 \in \Q[S_n]$ denote its image under the inclusion $\Q[S_{n-1}] \to \Q[S_n]$, we have
\begin{equation} p_U \sqcup 1 = \sum_{U \subset T} p_T \end{equation}
where the sum is over tableau obtained from $U$ by adding a box. This identity is a concrete realization of the ``branching rule," the decomposition of $\Q[S_n](p_U \sqcup 1)\cong \Ind_{S_{n-1}}^{S_n}(\SM^{\mu})$ into isotypic components, where $\mu$ is the shape of $U$. The individual idempotents $p_T$ are projections to the eigenspaces of $j_{n}$ on the image of $p_U \sqcup 1$.

% Alternately, we have the following inductive construction of the $p_T$.  The multiplicity-free branching rule for inducing representations from $S_{n-1}$ to $S_n$ implies that If $p_{U}\in \Q[S_{n-1}]$ is defined for $U\in \SYT(\mu)$, then the induced element $p_U\sqcup 1\in \Q[S_n]$ splits as a sum of canonically defined idempotents.  These idempotents project the induced representation $\Q[S_n](p_U\sqcup 1)\cong \Ind_{S_{n-1}}^{S_n}(\SM^\mu)$ onto its $\SM^{\l}$-isotypic summand, where $\l$ ranges over the partitions obtained from $\mu$ by adding a box.  The idempotents obtained in this way are primitive by construction and, letting $U$ vary, they are naturally indexed by SYT on $n$ boxes.

These ideas reduce the representation theory of symmetric groups to (relatively fancy) linear algebra; the projectors $p_\l$ and $p_T$ can be constructed by diagonalizing certain
operators, whose set of joint eigenvalues are well-understood.

If one had a central element $z \in \Q[S_n]$ which acted diagonalizably, and acted with distinct eigenvalues on each representation $\SM^\l$, then one could construct $p_\l$ as a
projection to an eigenspace of $z$. The sum $e_1 = \sum_{k=1}^n j_k$ is central, and fulfills this role on a philosophical level, but not in reality. It acts on $\SM^\l$ with eigenvalue
given by the \emph{(total) content number} of $\l$, the sum of the contents of each box, which we denote in this paper by $\xbb(\l)$. Unfortunately, there are distinct tableau $\l \ne
\mu$ with $\xbb(\l) = \xbb(\mu)$, meaning that $e_1$ can not tell these irreducible representations apart.

%========================
\subsection{Lifting to the Hecke algebra}
%========================


The Hecke algebra $\Hecke_n$ (in type $A$) is a $\Z[v,v\inv]$-deformation of $\Z[S_n]$, for a formal parameter $v$. By setting $v=1$ there is an isomorphism $\Hecke_n / (v-1) \cong
\Z[S_n]$. There are quasi-idempotents $k_T \in \Hecke_n$ and idempotents $p_T \in \Hecke_n \ot_{Z[v,v\inv]} \Q(v)$ analogous to those constructed by Young, which we continue (abusively) to denote $k_T$ and $p_T$. See e.g. \cite[\S 5]{DipperJamesIdemp}.

The YJM elements $\{j_k\}$ of the symmetric group deform in a naive way to the \emph{additive Jucys-Murphy elements} of $\Hecke_n$.  Dipper and James used analogous techniques to construct Young idempotents in $\Hecke_n$ by projecting to joint eigenspaces of additive Jucys-Murphy elements, see \cite[p75]{DipperJamesIdemp}.

The Hecke algebra also contains more subtle analogs of the $\{j_k\}$ called \emph{multiplicative Jucys-Murphy (JM) elements}, denoted $\{y_k\}$. We learned of these from work of Isaev-Molev-Oskin \cite{IMO}, which was inspired by ideas of Cherednik, see the references therein. If one sets $v=1$ then $y_k$ is sent to $1 \in \Z[S_n]$. Instead, the relationship is better encoded as a derivative at $v=1$: more precisely (see \cite[p2]{IMO})
\begin{equation} y_k = \frac{j_k - 1}{v - v^{-1}}|_{v = 1}.\end{equation}
One can also imitate the proofs of Okounkov-Vershik \cite{OV96} for these multiplicative JM elements\footnote{This is a straightforward and very worthwhile exercise. We would love to know of good references for this material.}. The eigenvalue of $y_k$ on $p_T$ is $v^{2x}$, where $x$ is the content number of the $k$-th box in $T$.

There are many reasons to prefer multiplicative JM elements over additive ones. For one, the elements $y_k$ are invertible, being in the image of the map from the braid group to $\Hecke_n$. Others will become evident soon. We will only be considering multiplicative JM elements in this paper.

The operator $\fT_n = \prod_{k=1}^n y_k \in \Hecke_n$, called the \emph{full twist}, is the multiplicative analog of the element $e_1 = \sum_{k=1}^n j_k \in \Z[S_n]$, in the sense that its ``derivative at $v=1$'' is $e_1$. It is central in $\Hecke_n$ (being the image of a central braid), and acts diagonalizably with eigenvalue on $\SM^\l$ given by $v^{2\xbb(\l)}$.

So, one can construct projections $p_T$ and $p_\l$ by simultaneously diagonalizing the commuting family $\{y_1, \ldots, y_n\}$. Alternatively, one can simultaneously diagonalize the
commuting family of \emph{(partial) full twists} $\{\fT_1, \ldots, \fT_n\}$, where $\fT_i = \prod_{k=1}^i y_k$.

%========================
\subsection{The categorification}
%========================

The category $\SBim_n$ of Soergel bimodules (for $S_n$) is a full subcategory of the category of graded bimodules for the ring $R = \RM[x_1, \ldots, x_n]$. It is closed under direct sums
and direct summands, grading shifts, and tensor products, so its split Grothendieck group $[\SBim_n]$ is naturally a free $\Z[v,v\inv]$-algebra with basis $\{[B_w]\}$ indexed by the
symbols of the indecomposable Soergel bimodules (up to isomorphism and grading shift). Soergel proved that the indecomposables are indexed by $w \in S_n$, and that there is a
natural isomorphism $\Hecke_n \to [\SBim_n]$. See \S\ref{subsec:SoergelCat} for additional background on this topic.

To lift elements of the braid group (or their images in $\Hecke_n$) to this categorical world, one should work with complexes of Soergel bimodules. The homotopy category (i.e. complexes
modulo nulhomotopic chain maps) of Soergel bimodules is often called the \emph{Hecke category}, and is a triangulated categorification of $\Hecke_n$. Rouquier constructed a bounded
complex of Soergel bimodules for each braid. Thus, the bounded homotopy category $\KC^b(\SBim_n)$ contains objects $Y_k$ associated to JM operators, and $\FT_k$ associated to the
partial full twists.


\begin{remark} It is not known if there is a reasonable categorification of the additive Jucys-Murphy elements. \end{remark}

%========================
\subsection{Categorical linear algebra}
%========================

In previous work \cite{ElHog17a}, the authors introduced a notion of categorical diagonalization, the diagonalization of functors. Let us give a brief synopsis.


Let $\VC$ be an additive category, let $\AC$ be an additive monoidal category which acts on $\VC$ by additive endofunctors. For example one could take $\AC$ to be the category of
additive endofunctors of $\VC$. We often choose $\AC$ first, and let it act on itself $\VC=\AC$ on the left. Any complex $F \in \KC^b(\AC)$ acts by a triangulated endofunctor on
$\KC^b(\VC)$. The analogous situation in linear algebra is an element $f$ in an algebra $A$, which acts on a vector space $V$. In our main example, $\AC = \SBim_n$ and $F = \FT_n$,
analogous to $\fT_n$ acting in the regular representation of $\Hecke_n$.


\begin{remark} In linear algebra any statement regarding eigenvalues and diagonalizability of linear operators on a vector space $V$ can be stated entirely in terms of the algebra of endomorphisms $A=\End(V)$.  We adopt a similar point of view for the categorification.  In particular we prefer to view $F\in \KC^b(\AC)$ as the main object of study, rather than the functor $F\ot(-) \co \KC^b(\VC) \to \KC^b(\VC)$.

The main reason for doing this is that the category of triangulated endofunctors of a triangulated category is typically not triangulated, due to the usual inadequacies of abstract triangulated categories (formation of mapping cones is not functorial).   On the other hand, any action of $\AC$ on $\VC$ extends to an action of $\KC^b(\AC)$ on $\KC^b(\VC)$ by triangulated endofunctors, via the usual rules for tensoring complexes.  Thus if $F,G\in \KC^b(\AC)$, then any mapping cone $\Cone(F\rightarrow G)$ is a well defined object of $\KC^b(\AC)$, hence acts on $\KC^b(\VC)$.  In other words, when taking mapping cones of functors, we do so in the triangulated category $\KC^b(\AC)$, instead of the (not necessarily triangulated) category $\End(\KC^b(\VC))$. \end{remark}

Let us also assume that $\AC$ and $\VC$ are equipped with additional ``grading shift" functors. For simplicity in the introduction we will say that $\AC$ is $\Z$-graded with shift
functors denoted $(k)$, $k\in \Z$. An \emph{invertible scalar object} $\Sigma$ in $\KC^b(\AC)$ is\footnote{In this paper this definition suffices. For more see \S
\ref{subsec:gradedcats} and \cite{ElHog17a}.} an object of the form $\one(a)[b]$, where $[b]$ denotes the homological shift. A \emph{scalar object} is by definition a finite direct sum
of invertible scalar objects. One thinks of tensoring with scalar objects as analogous to multiplication by scalars in linear algebra.

A \emph{weak eigenobject} would be an nonzero object $M \in \KC^b(\VC)$ such that $F\otimes M \simeq \Sigma \ot M$, analogous to $fm = \k m$. A better notion of an eigenobject
would give more control over this isomorphism. With this in mind, let $\a \co \Sigma \to F$ be a chain map in $\KC^b(\AC)$. We say that $M$ is an \emph{$\a$-eigenobject} if $M$ is
nonzero and $\a\otimes \Id_M$ is a homotopy equivalence $\Sigma \otimes M \buildrel\simeq\over \to F \otimes M$. In this case we call $\a$ an \emph{eigenmap}. Eigenmaps are a new
categorical structure which have no analogue in linear algebra: they represent the ``relationship'' between the operator $f$ and its eigenvalues.

One should think of the cone $\Cone(\a)\in \KC^b(\AC)$ as a categorification of $(f - \k)$. The $\a$-eigenobjects can alternately be described as those nonzero complexes $M\in
\KC^b(\VC)$ which are annihilated by $\Cone(\a)$, analogous to $(f-\k)m=0$. The full subcategory of $\a$-eigenobjects is the \emph{$\a$-eigencategory}; it is a full triangulated
subcategory of $\KC^b(\VC)$.


We call $F$ \emph{prediagonalizable} if there exist maps $\a_\l \co \Sigma_\l \to F$ from various scalar objects $\Sigma_\l$, such that the tensor product \begin{equation}\label{eq:introBigOT} \bigotimes \Cone(\a_\l) \simeq 0 \end{equation} is nulhomotopic\footnote{There are some additional technical conditions. The cones in \eqref{eq:introBigOT} may not tensor commute, and one must either be more precise in \eqref{eq:introBigOT}, or impose additional assumptions to ensure that they do commute.}. This is analogous to the familiar statement \begin{equation} \prod (f - \k_\l) = 0 \end{equation} of diagonalizability in linear algebra. Here is a loose statement of the Diagonalization Theorem from \cite{ElHog17a}.

\begin{theorem} \label{thm:DiagThmIntro} (See \cite[Theorem 1.22]{ElHog17a} for a precise version) Suppose that $F$ is invertible, and prediagonalizable with eigenmaps $\a_\l \co \Sigma_\l \to F$, such
that the scalar objects $\Sigma_\l$ have distinct homological shifts. Then for each $\l$ there is an explicit construction of an idempotent complex $\PB_\l$ inside the
bounded-above homotopy category $\KC^-(\AC)$, which projects to the eigencategory for $\a_\l$. The monoidal identity has a filtration whose subquotients are the $\PB_\l$, analogous to the
fact that the identity is the sum of the eigenprojections in linear algebra: \[1 = \sum p_\l.\] \end{theorem}

In other words, the category $\KC^-(\AC)$ has a filtration\footnote{precisely, a semi-orthogonal decomposition} by eigencategories, just as $A$ has a decomposition into eigenspaces. It is the existence of the projectors $\PB_\l$ which most
interests us, so when we can construct them we say that we have \emph{diagonalized} the functor $F$. One should think of the passage from $\KC^b(\AC)$ to $\KC^-(\AC)$ as analogous base change from $\Z[v,v\inv]$ to $\Q(v)$, or better yet, to Laurent power series $\Z((v))$, as finite complexes allow for infinite sums.


\begin{remark} We have said $F$ is analogous to $f$, $\AC$ is analogous to $A$, etcetera, without saying explicitly ``$F$ categorifies $f$.'' To be precise requires some subtlety, since
the Grothendieck group of $\KC^-(\AC)$ is zero by the Eilenberg swindle (see \cite[Remark 4.9]{ElHog17a} for further discussion). Instead, one must show that all the complexes involved
live in a suitable full subcategory of $\KC^-(\AC)$ whose Grothendieck group is a nonzero extension of $[\AC]$, as was done e.g.~in \cite{CK12a,AchStr}. Choosing this subcategory is context-dependent.  In \S \ref{subsec:groth} we define a category $\KC^{\angle}(\SBim_n)$ which contains all the complexes considered here and has a well-behaved Grothendieck group.  This justifies statements like ``$\PB_\l$ categorifies $p_\l$'' below.
\end{remark}

\begin{remark} In fact, our Diagonalization Theorem also proves the existence of quasi-idempotent complexes $\KB_\l$ which live in the bounded homotopy category, such that $\PB_\l$ is built from of infinitely many copies of $\KB_\l$. The $\KB_\l$ enjoy a number of desirable properties, and there is an easily expressed relationship between $\KB_\l$ and $\PB_\l$ which mirrors the relationship (Koszul duality) between exterior and polynomial algebras, see \cite[\S 7.3]{ElHog17a}. \end{remark}

%========================
\subsection{The main theorem}
%========================

Now let us apply this to our current situation, with $\AC = \SBim_n$. Our first major result is this.

\begin{theorem} \label{thm:lambdaMapsIntro} (See Theorem \ref{thm:lambdaMaps}) Let $\FT_n\in \KC^b(\SBim_n)$ be the full twist Rouquier complex. Then $\FT_n$ is categorically
prediagonalizable in the sense of \cite[Definition 6.13]{ElHog17a}, with one eigenmap $\a_\l \co \Sigma_\l \to \FT_n$ for each partition of $n$.  The scalar object $\Sigma_\l$ is
$\one(2\xbb(\l))[-2\cbb(\l)]$, where $\xbb(\l)$ is the total content number as above, and $\cbb(\l)$ is the total column number, obtained by adding together the column number of each box
in $\l$, see Definition \ref{defn:xcr}. \end{theorem}

\begin{remark} Ironically, although JM elements $y_k$ are preferred in the literature to partial full twists $\ft_k$, the JM complexes $Y_k$ will not be prediagonalizable. They do not admit enough maps to or from scalar objects. Hence, we stick to discussing full twists henceforth. \end{remark}

Note that the grading shift $2\xbb(\l)$ corresponded to the eigenvalue $v^{2\xbb(\l)}$ of $\fT_n$ on $\SM^\l$. Meanwhile, the homological shift $-2\cbb(\l)$ is invisible in the
Grothendieck group, but plays a much more significant role in this paper.

Unfortunately, Theorem \ref{thm:lambdaMapsIntro} does not allow one to apply the Diagonalization Theorem, because the scalar objects $\Sigma_\l$ do not have distinct homological shifts.
That is, for any $n \ge 6$ there are distinct partitions $\l \ne \mu$ with $\cbb(\l) = \cbb(\mu)$, and even $\xbb(\l) = \xbb(\mu)$. Thus we again meet the problem that $\fT_n$ can not
distinguish between distinct irreducibles, and again we solve this problem by studying the entire family of full twists. In both cases, the saving grace is the observation that, once one
fixes a partition $\l$ with $n-1$ boxes, and restricts ones attention to partitions of $n$ which contain $\l$, these partitions have distinct values of $\xbb$ and $\cbb$. Ultimately,
this will allow us to use our Relative Diagonalization Theorem \cite[Theorem 8.2]{ElHog17a} to diagonalize $\FT_n$ if we have already diagonalized $\FT_{n-1}$.

Thus our main theorem, see Theorems \ref{thm:typeAdiag} and \ref{thm:PTprops}, states that one can simultaneously diagonalize the full twists $\{\FT_1, \ldots,
\FT_n\}$, producing complexes $\PB_\l$ for each partition of $n$, and $\PB_T$ for each standard tableaux with $n$ boxes, which categorify $p_\l$ and $p_T$ respectively.

\begin{theorem}\label{thm:introFTdiag}
Let $\FT_n\in \KC^b(\SBim_n)$ be the full twist Rouquier complex.  There exists a diagonalization $\{(\PB_\l, \a_\l)\}$ of $\FT_n$, indexed by partitions $\l$ of $n$, in the sense of \cite[Definition 6.16]{ElHog17a}.  The idempotents $\PB_\l$ are central in the homotopy category $\KC^-(\SBim_n)$, and they categorify the central idempotents in $\HB_n$.  If $\l^{(k)}$ is a partition of $k$ for $1 \le k \le n$, then
\[
\PB_{(\l^{(1)},\ldots,\l^{(n)})}:=\bigotimes_{k=1}^n \PB_{\l^{(k)}}
\]
is contractible unless $T=(\l^{(1)},\ldots,\l^{(n)})$ is a standard Young tableaux (that is, $\l^{(k)}\subset \l^{(k+1)}$ for all $1\leq k\leq n-1$).  The idempotents $\PB_T$ describe a simultaneous diagonalization of the complexes $\FT_k$, $1\leq k\leq n$; they categorify the primitive Young idempotents $p_T$ in $\Hecke_n$.
\end{theorem}

Examples of these projectors $\PB_T$ and related finite complexes $\KB_T$ can be found in \S \ref{sec:examples}.

\begin{remark} There are many other related features of the representation theory of $\Hecke_n$, for example, the construction of a particularly nice eigenbasis of $\SM^\l$ called the Young (semi)normal form. In a sequel to this paper, we plan to categorify the proofs from \cite{OV96} of some of these features. \end{remark}

%========================
\subsection{Representations and cells}
%========================

In order to explain how our theorem is proven, we must approach the representation theory of the Hecke algebra from an entirely different perspective: that of cells, and the
Kazhdan-Lusztig basis. Let us give a brief and unusual exposition of this topic, allowing that we already know the Hecke algebra is categorified by Soergel bimodules. A more standard
exposition is given in the body of the paper.

Soergel \cite{Soer90} proved that under the isomorphism $\Hecke_n \to [\SBim_n]$, the indecomposable Soergel bimodules $[B_w]$ are sent to the so-called \emph{Kazhdan-Lusztig (KL) basis}
$\{b_w\}$, defined in \cite{KazLus79} by Kazhdan and Lusztig. This is a difficult theorem, and relies in an essential way on the fact that the base ring $R = \RM[x_1, \ldots, x_n]$ is
over a field of characteristic zero. In finite characteristic, the indecomposable Soergel bimodules are sent to a different basis of $\Hecke_n$, and most of the arguments in this paper
will fail.

A consequence is that plethysm (the decomposition of tensor products into indecomposable objects) in $\SBim_n$ is determined by multiplication of KL basis
elements. That is, if \begin{equation} \label{eq:structurecoeffsintro} b_w b_x = \sum c^y_{w,x} b_y \end{equation} for coefficients $c^y_{w,x} \in \Z{v,v\inv}$, then \begin{equation} B_w
\ot B_x \cong \bigoplus B_y^{\oplus c^y_{w,x}},\end{equation} and in particular the coefficients $c^y_{w,x}$ are non-negative. Here, $B^{\oplus v^k}$ is shorthand for the grading shift
$B(k)$.


\begin{remark} In contrast, plethysm in $\KC^b(\SBim_n)$ is not determined in the Hecke algebra! The class of a complex in the Grothendieck group only remembers its Euler characteristic (and certainly forgets the all-important differential).  Much of this paper consists of lifting formulas in the Hecke algebra to $\KC^b(\SBim_n)$ in a particularly nice way. \end{remark}

%\begin{remark} In contrast, plethysm in $\KC^b(\SBim_n)$ is not determined in the Hecke algebra! The symbol of a complex in the Grothendieck group only remembers the chain objects, but not the all-important differentials. Much of this paper consists of lifting formulas in the Hecke algebra to $\KC^b(\SBim_n)$ in a particularly nice way. \end{remark}

Let $\AC$ be an additive, idempotent complete, monoidal category, and let $\ind \AC$ denote the set of indecomposable objects up to isomorphism. Define a transitive relation $\le_L$ on
$\ind \AC$ by declaring $B_1 \le_L B_2$ if $B_1$ is isomorphic to a direct summand of $C \ot B_2$ for some $C \in \AC$. Equivalence classes with respect to this relation are called
\emph{left cells}. Similarly, one can define \emph{right cells} and \emph{two-sided cells}. The set of cells is a poset under the relation. If $\AC$ is graded, then, $B$ and $B(k)$ are
in the same left (resp. right, two-sided) cell for any grading shift $k$, because $\one(k)$ is invertible. More generally, tensoring with scalar objects will preserve cells. We describe this theory in more
detail in \S\ref{sec:cells}.

Applying this to $\AC = \SBim_n$, we obtain a notion of left, right, and two-sided cells in the symmetric group $S_n$, which is the usual notion of cells attached to the KL basis. It
follows from the original work of Kazhdan-Lusztig that these cells match up with the various data associated to a permutation in the Robinson-Schensted correspondence. The two-sided
cells are in bijection with the partitions of $n$, with the dominance order giving the poset structure. Henceforth, the word cell without any adjectives refers to a two-sided cell.

Given a (two-sided) cell $\l$, the full subcategory $\AC_{\le \l}$ with objects $\{B_w(k)\}$ for $k \in \Z$ and $w$ in cells $\le \l$ is a thick two-sided tensor ideal. Equivalently, the
$\Z[v,v\inv]$-span of $\{b_w\}$ for $w$ in cells $\le \l$ is a two-sided ideal in $\HB = \Hecke_n$, which we denote $\HB_{\le \l}$. This equips the Hecke algebra with a filtration by
ideals, whose subquotients are spanned by the images of $b_w$ for $w$ in cell $\l$. This subquotient, viewed as a left module for $\HB$, splits further into modules for each left cell
inside $\l$, which are called \emph{cell modules}. The crucial point is that, in type $A$, each cell module within the two-sided cell $\l$ is isomorphic to the Specht module $\SM^\l$,
and this gives an alternate construction of the irreducible representations of $\Hecke_n$.

\begin{remark} Later we will discuss the case of general Coxeter groups, where the cell modules need not be irreducible. This discussion leaves out many such subtleties. \end{remark}
	
%========================
\subsection{Cell filtrations and prediagonalization}
%========================

Let us examine how the cell theory construction of $\SM^\l$ interfaces with the eigentheory of the full twist. There is an element $\hT_n$ in the Hecke algebra, known as the \emph{half twist} and also the image of a braid, such that $\hT_n^2 = \fT_n$. By a result of Graham (see Mathas \cite{Mathas96}), for $w \in S_n$ in cell $\l$ one has
\begin{equation} \label{eq:htactionintro} \hT_n b_w \equiv (-1)^{\cbb(\l)} v^{\xbb(\l)} b_{\Schu_L(w)} + \HB_{<\l}. \end{equation} 
Here $\Schu_L$, the left Sch\"utzenberger dual, is an involution on $S_n$ that preserves each left cell. Consequently
\begin{equation} \label{eq:ftactionintro} \fT_n b_w \equiv (-1)^{2 \cbb(\l)} v^{2 \xbb(\l)} b_{w} + \HB_{<\l}. \end{equation}
We recommend that the reader skim through \S\ref{subsec:HT3} now, to see how these formulas play out for $S_3$ in the categorification $\KC^b(\SBim_3)$.


This illustrates a crucial point. It is difficult to explicitly find any eigenvectors for $\fT_n$. However, it is easy to find eigenvectors modulo lower terms in the cell
filtration. The same is true in the categorification: an eigenobject for $\FT_n$ is typically an interesting complex, while any indecomposable object $B_w$ is an ``eigenobject modulo lower
cells'' (as we prove). This motivates us to shift our goalposts: instead of looking for eigenmaps, we look for ``eigenmaps modulo lower cells.''

Let $F$ be a complex in $\KC^b(\AC)$ for an additive monoidal category $\AC$. Suppose that the action of $F$ agrees with that of a scalar object $\Sigma_\l$ on the cellular subquotient
$\KC^b(\AC_{\le \l}/\AC_{< \l})$; in other words, if $B$ is an indecomposable in cell $\l$, then $F \ot B$ is a complex which is built from $\Sigma_\l B$ and various indecomposables in
cells $< \l$. We call a map $\a \co \Sigma_\l \to F$ a \emph{$\l$-equivalence} if it induces the isomorphism $\Sigma_\l B \to F \ot B$ in the cellular subquotient, for all $B$ in cell
$\l$. That is, $\a$ is like an eigenmap but only modulo lower terms.

We are not sure to what extent a naive theory of $\l$-equivalences can be pursued. By making some additional (strong) assumptions, one can prove that $F \ot B$ is built from $\Sigma_\l
B$ and indecomposables in cell $< \l$, where all the lower terms also appear in smaller homological degrees! In this case, the map $\Sigma_\l B \to F \ot B$ induced by a $\l$-equivalence
would just be the inclusion of the last nonzero homological degree. When this happens, it becomes much easier to prove categorical statements. We only discuss $\l$-equivalences in this
more rigid context.

There is a simple linear algebra statement which reads as follows: suppose one has an operator $f$ on a vector space $V$. Suppose $V$ is filtered by a poset $\PC$, and for each $\l \in
\PC$, $f$ acts on $V_{\le \l} / V_{< \l}$ as scalar multiplication by $\k_\l$. Suppose also that if $\mu$ and $\l$ are comparable then $\k_{\mu}\neq \k_{\l}$. Then $f$ is diagonalizable on $V$, with the same eigenvalues $\{\k_\l\}_{\l \in \PC}$. In other words,
``eigenvalues modulo lower terms'' are actually eigenvalues. In Proposition \ref{prop:lequivmeansdiag} we prove an analogous categorical statement, under the aforementioned rigidifying
assumptions. If $F$ admits a $\l$-equivalence $\a_\l$ for each cell $\l$, then $\bigotimes \Cone(\a_\l) \simeq 0$. Thus $F$ is prediagonalizable, and the $\l$-equivalences are actually eigenmaps.

Our proof that $\FT_n$ is diagonalizable can be divided into three major steps. The first is proving that $\FT_n$ satisfies the assumptions of Proposition \ref{prop:lequivmeansdiag},
which we discuss shortly. The second is constructing a family of $\l$-equivalences. The third is congealing these results into a form suitable for the Relative Diagonalization Theorem of
\cite{ElHog17a}.

%========================
\subsection{Bounding the action of the full twist}
\label{subsec:boundingintro}
%========================

For $\l$-equivalences to behave well, we need to prove that $\FT_n \ot B$ should be built from $\Sigma_\l B$ and indecomposables in cell $< \l$, where all the lower terms also appear in
smaller homological degrees. In other words, \eqref{eq:htactionintro} and \eqref{eq:ftactionintro} should be categorified in the best possible way. We believe these results have
independent interest.

\begin{prop} \label{prop:sharpIntro} Let $\FT = \FT_n$, and let $\HT = \HT_n$ be the Rouquier complex for the half twist $\hT_n$. For $w$ in cell $\l$,
\begin{enumerate}
\item $\HT\otimes B_w$ is homotopy equivalent to a complex $X^0\rightarrow \cdots \rightarrow X^{\cbb(\l)}$ where $X^{\cbb(\l)}=B_{\Schu_L(w)}(\xbb(\l))$ and $X^k$ is in cells strictly less than $\l$ for $k<\cbb(\l)$.
\item $\FT\otimes B_w$ is homotopy equivalent to a complex $Y^0\rightarrow \cdots \rightarrow Y^{2\cbb(\l)}$ where $Y^{2\cbb(\l)}=B_{w}(2\xbb(\l))$ and $Y^k$ is in cells strictly less than $\l$ for $k<2\cbb(\l)$.
\end{enumerate} \end{prop}

In \S\ref{subsec:twists} and \S\ref{subsec:sharpconj}, Proposition \ref{prop:sharpIntro} is reduced to a simpler statement, which is proven in Theorem \ref{thm:HTactionA}. The proof of Theorem \ref{thm:HTactionA}, and the exposition to make it possible, comprises a significant bulk of this paper. It requires a fairly thorough knowledge of Lusztig's
asymptotic Hecke algebra.

When one speaks of asymptotic data in the Hecke algebra, one refers to the lowest degree power of $v$ with a nonzero coefficient in some formula; categorically,
this is the minimal degree of a morphism between Soergel bimodules. For example, letting $c_{x,y}^z$ be as in \eqref{eq:structurecoeffsintro}, one can define\footnote{This is denoted
$\abb(z)$ by Lusztig.} $\rbb(z)$ for $z \in S_n$ to be the integer such that $v^{-\rbb(z)}$ is the minimal power of $v$ appearing in $c_{x,y}^z$ for all $x$ and $y$. It turns out that
$\rbb(z)$ only depends on the two-sided cell $\l$ of $z$, and is equal to the row number $\rbb(\l)$ of the corresponding partition.

\begin{remark} We have now defined three numerical statistics associated to a partition: the row number $\rbb(\l)$, the column number $\cbb(\l)$, and the content number $\xbb(\l)$. One
has $\xbb(\l) = \cbb(\l) - \rbb(\l)$, and $\cbb(\l) = \rbb(\l^t)$, where $\l^t$ is the transpose partition, also obtained from the two-sided cell $\l$ by multiplication by the longest
element $w_0 \in S_n$. \end{remark}

Lusztig \cite[Chapter 14]{LuszUnequal14} has a list of famous properties (P1-P15) of this statistic $\rbb(z)$ and its interaction with cells, which he proves for the symmetric group
using the fact that all Kazhdan-Lusztig polynomials are positive. We use most of these properties in the setup of the proof of Theorem \ref{thm:HTactionA}. Said another way, our results
in this paper put a homological spin on Lusztig's asymptotic Hecke algebra (c.f. Remark \ref{rmk:syzygy}).

We use these Grothendieck group considerations to reduce the proof of Proposition \ref{prop:sharpIntro} to the case when $w$ is the longest element of a parabolic subgroup $S_{k_1}
\times \ldots \times S_{k_r} \subset S_n$. We are able to prove this directly in \S\ref{subsec:bounding} and \S\ref{subsec:boundinglemma}, by a nasty computation.

%========================
\subsection{Constructing $\l$-equivalences}
\label{subsec:constructionIntro}
%========================

Two ingredients go into our construction of $\l$-equivalences. The first is a criterion for the existence of a $\l$-equivalence.

Recall that $R = \RM[x_1, \ldots, x_n]$ is the base ring of the construction of Soergel bimodules, and also is the endomorphism ring $\End(\one)$ of the monoidal identity. Using
asymptotic data once more, one can construct homogeneous polynomials $\tbarb{d} \in R$ (well-defined up to a scalar in $\RM_{>0}$) associated to involutions $d \in S_n$, by composing the
morphisms of minimal degree \[\one \to B_d \to \one.\] The span of these polynomials, as $d$ ranges over all involutions in a given cell $\l$, is proven to be an $S_n$-representation
inside $R$. In fact, we prove it is isomorphic to the Specht module $\SM^\l$! So long as there exists some involution $d$ in cell $\l$ and some map $\a$ which is not annihilated by
precomposition with $\tbarb{d}$, we are able to bootstrap the existence of some $\l$-equivalence (not necessarily $\a$ itself). This proof, accomplished in
\S\ref{subsec:lambdaEquivsufficient}, uses the irreducibility of the Specht module.

The second key ingredient in our construction of $\l$-equivalences comes from recent progress in computing the triply graded homology of torus links.

Given complexes $C,D\in \KC(\SBim_n)$, let $\Hom_{\KC(\SBim_n)}^{\Z\times \Z}(C,D)$ denote the bigraded space of homogeneous chain maps modulo homotopy:
\[
\Hom_{\KC(\SBim_n)}^{\Z\times \Z}(C,D) = \bigoplus_{i,j\in \Z}\Hom_{\KC(\SBim_n)}(C,D(i)[j]).
\]
In \cite{ElHog16a} the authors introduced a technique for the computation of $\Hom^{\Z\times \Z}(\one,C)$ when $C$ is a Rouquier complex of a particular class of braids\footnote{We compute the entire triply graded homology of these braids, of which the zeroth Hochschild degree is this bigraded vector space $\Hom^{\Z\times \Z}(\one,C)$.}. As an application, we compute $\Hom^{\Z\times \Z}(\one,\FT_n)$ as a bigraded vector space. In work of the second author, this was extended to a computation of $\Hom^{\Z\times \Z}(\one,\FT_n^{\otimes k})$ for all $n,k\geq 0$. Our computation is inherently inductive, using the embeddings $S_k \subset S_n$ for $k < n$.

In fact, $\Hom^{\Z\times \Z}(\one,\FT_n)$ is an $R$-bimodule on which the right and left actions agree. Our description does not make the $R$-module structure clear, only
the action of $R$ in the associated graded for some filtration. This is because we describe $\Hom^{\Z\times \Z}(\one,\FT_n)$ using a (degenerate) spectral sequence.

The precise answer to the computation is not worth recalling here, though more details are given in \S\ref{subsec:FTHHH}. In \S\ref{subsec:eigenmaptheorem} we justify that this
description of $\Hom^{\Z\times \Z}(\one,\FT_n)$ produces certain specific maps $\a_\l \co \one(2\xbb(\l))[-2\cbb(\l)] \to \FT_n$, which are candidates for $\l$-equivalences. By examining
the action of $R$ on $\a_\l$ in the associated graded, we deduce that a particular polynomial $\tbarb{d} \in R$ will not kill $\a_\l$. This is sufficient to deduce the existence of some $\l$-equivalence, by the criterion mentioned above.

\begin{remark} In fact, using very recent work of the second author and Gorsky \cite{GorHog17}, we can prove that $\a_\l$ itself is a $\l$-equivalence. We include this as an addendum, see \S\ref{subsec:symgrp}. \end{remark}


%========================
\subsection{Ensuring relative diagonalization}
%========================

Finally, we need to prove that $\FT_n$ can be diagonalized, relative to an inductively defined diagonalization of $\FT_{n-1}$. Fix a partition $\mu$ of size $n-1$, and let $\PB_\mu$ be the corresponding central idempotent complex in $\KC^-(\SBim_{n-1})$, which we can also view as a non-central idempotent complex in $\KC^-(\SBim_n)$. Let $\{\l^1, \ldots, \l^r\}$ denote the partitions of $n$ which are obtained from $\mu$ by adding a single box. In order to use our Relative Diagonalization Theorem \cite[Theorem 8.2]{ElHog17a}, we need to prove that
\begin{equation}\label{eq:relBigOTintro} \PB_\mu \ot \bigotimes \Cone(\a_{\l^i}) \simeq 0. \end{equation}


The corresponding statement in the Hecke algebra says that the eigenvalues of $\fT_n$ on the induction of $\SM^\mu$ to $\Hecke_n$ are those associated to $\l^i$. This is a key computation in Okounkov-Vershik \cite{OV96}. A categorical proof is slightly more difficult, as we must prove this result using cell theory instead in order to use the techniques (e.g.~$\l$-equivalences) that we have developed. To this end, we use several results of Meinolf Geck \cite{GeckInduction,GeckRelative} on induction of Kazhdan-Lusztig cells and on relative Kazhdan-Lusztig cells, which we recall in \S\ref{subsec:relativecells}.

Beyond this, the major technical difficulties involved are mostly in the Relative Diagonalization Theorem itself, and this justifies a great deal of the work done in \cite{ElHog17a}.  The resulting idempotents $\PB_T$ satisfy an inductive relationship which categorifies \cite[Equation (11)]{IMO}.

%========================
\subsection{Tightness}
%========================

We have noted that there are distinct partitions $\l \ne \mu$ with the same eigenvalue of $\fT_n$, so that $\fT_n$ can not tell their irreducibles apart. Yet $\FT_n$ has distinct (and in fact, linearly independent) eigenmaps $\a_\l$ and $\a_\mu$, so it seems that $\FT_n$ should be able to tell their categorical cell modules apart, which would be an upgrade. However, this is a very subtle point which we do not prove here.

In our definition of diagonalization \cite[Definition 6.16]{ElHog17a}, the essential image of the projection $\PB_\l\otimes (-)$ must lie inside the eigencategory of $\a_\l$, but it need not be the entire eigencategory! When the essential image of $\PB_\l\otimes (-)$ and the eigencategory of $\a_\l$ agree for all $\l$, the diagonalization is called \emph{tight}. Non-tight diagonalizations are important for applications in algebraic geometry, though a tight diagonalization seems preferable in some contexts. Non-tightness is hard to avoid when eigenvalues coincide, that is, when distinct eigenmaps $\a_\l \ne \a_\mu$ have the same scalar object $\Sigma_\l \cong \Sigma_\mu$. When eigenvalues coincide, it is possible for there to be a nonzero object $M$ which is an eigenobject for both $\a_\l$ and $\a_\mu$; if the diagonalization were tight then $M$ is preserved by both $\PB_\l$ and $\PB_\mu$, but then $M \cong \PB_\l \PB_\mu M \cong 0$ by
orthogonality of projectors, so this possibility is precluded.

To illustrate the subtlely, consider the incomparable partitions $\l=(3,1,1,1)$ and $\mu=(2,2,2)$, which have the same values of $\rbb=6$, $\cbb=3$, and $\xbb=-3$. Both $\a_\l$ and
$\a_\mu$ are elements of $V = \Hom(\one(-6)[6], \FT_6)$, and $\{\a_\l,\a_\mu\}$ is a basis for this two-dimensional space. However, as observed in \cite[\S 2]{ElHog17a} or in the proof
of Lemma \ref{lem:criterion}, being a $\l$-equivalence is an open condition in this vector space $V$, the complement of a union $V_\l$ of hyperplanes. Hence almost every linear
combination of $\a_\l$ and $\a_\mu$ is both a $\l$-equivalence and a $\mu$-equivalence. If $\a_\mu$ happens to be a $\l$-equivalence (the generic situation) then there is a joint eigenobject for both $\a_\l$ and $\a_\mu$.  The question of distinguishing the $\a_\mu$ and $\a_\l$ eigencategories essentially boils down to whether it is possible to construct a $\l$-equivalence $\a_\l$ which is not a $\mu$-equivalence, and vice versa. It certainly would be quite special for $V_\mu$ to be contained in $V_\l$ for some $\l\neq \mu$, so generically one would expect the following to be true.


\begin{conjecture}\label{conj:tightness0}
The $\l$-equivalences $\a_\l$ can be chosen so that $\a_\l$ is not a $\mu$-equivalence for any $\mu\neq \l$.
\end{conjecture}


This is certainly a prerequisite for tightness.

\begin{conjecture}\label{conj:tightness2}
There exist $\l$-equivalences $\a_\l$ for which the diagonalization of $\FT_n$ from Theorem \ref{thm:typeAdiag} is tight.
\end{conjecture}

It is beyond the scope of this paper to tackle either of these conjectures.  We expect Conjecture \ref{conj:tightness2} to be a straightforward consequence of Conjecture \ref{conj:tightness0}, but a proof Conjecture \ref{conj:tightness0} will require a deeper understanding of the $R$-module structure on $\Homgg(\one,\FT_n)$.  Our main tool for studying $\Hom^{\Z \times \Z}(\one,\FT_n)$ is the spectral sequence from \S\ref{subsec:FTHHH}, which only encodes the $R$-action on the associated graded, while our main tool for proving that a map is a $\l$-equivalence uses the true $R$-action (see \S\ref{subsec:constructionIntro}).  One can use this to prove maps are $\l$-equivalences, because being not being annihilated by an element of $R$ in the associated graded implies not being annihilated by the same polynomial in the whole space.  The converse of this is false, hence we do not show that our $\l$-equivalence is not also a $\mu$-equivalence for $\mu\neq \l$.

Finally we mention a closely related conjecture regarding the categorical analogue of the minimal polynomial for $\FT_n$.

\begin{conjecture}\label{conj:tightness1}
The $\l$-equivalences $\a_\l$ can be chosen such that
\begin{equation} \bigotimes_{\l\in \PC(n)} \Cone(\a_\l) \simeq 0, \end{equation}
with none of the factors being redundant.
\end{conjecture}
If the $\mu$-equivalence $\a_\mu$ were also a $\l$-equivalence for some $\l\neq \mu$, then the factor $\Cone(\a_\l)$ would be redundant.  Thus, Conjecture \ref{conj:tightness1} implies Conjecture \ref{conj:tightness0}.


%========================
\subsection{Other Coxeter groups}
%========================

The symmetric group is one family (type $A$) inside a broad class of groups generated by reflections, known as \emph{Coxeter groups}. Given a Coxeter group $W$ with a set of simple
reflections $S$, one has an associated Hecke algebra $\HB(W)$, a category of Soergel bimodules $\SBim(W)$, and Rouquier complexes associated with elements of the braid group of $W$. When
$W$ is finite, it has a longest element $w_0$, which lifts to an element $\hT$ inside the braid group called the \emph{half twist}. This in turn squares to a central element $\fT$ which
we call the \emph{full twist}. Mathas \cite{Mathas96} has proven an analog of \eqref{eq:htactionintro}. All the pieces are in place for a potential diagonalization of $\FT$, the Rouquier
complex of $\fT$.

Below, let $\rbb(\l)$ be defined as in \S\ref{subsec:boundingintro}; this number is commonly called Lusztig's $\abb$-function. Let $\cbb(\l) := \rbb(\l^t)$, where $\l^t$ is the two-sided cell $w_0 \l$, set $\xbb(\l) := \cbb(\l) - \rbb(\l)$. 


\begin{conj} \label{conj:coxIntro} The Rouquier complex $\FT$ associated with the full twist $\fT$ is categorically diagonalizable, for any finite Coxeter group $W$. It has one eigenmap
$\a_\l \co \Sigma_\l \to \FT$ for each two-sided cell $\l$. The scalar object $\Sigma_\l$ is given by $\one(2\xbb(\l))[-2\cbb(\l)]$.\end{conj}

 
In a sequel to this paper, we will prove this conjecture for dihedral groups (which have only three two-sided cells) by direct computation.  

To facilitate this general conjecture, we give our exposition in terms of a general Coxeter group, until \S\ref{sec:typeAcells} when we focus on the case of the symmetric group. It should not make the exposition significantly harder to read, as various constructions (Lusztig's asymptotic Hecke algebra, Sch\"utzenberger duality) are tricky anyway, and not significantly less tricky in type $A$. The main simplification in type $A$ is that every involution is distinguished. The reader unfamiliar with Coxeter groups beyond the symmetric group can just assume $W = S_n$ throughout.

Let us discuss which arguments in this paper go through for general Coxeter groups, and which (enormous) gaps remain in the proof.

Our first step was proving Proposition \ref{prop:sharpIntro}, which said that \eqref{eq:htactionintro} was categorified in the best possible way. In \S\ref{subsec:twists} we develop some general results, which reduce this question to proving that $\HT \ot B_w$ is concentrated in homological degrees $\le \cbb(\l)$ for a single $w$ in each cell $\l$. This is the content of Conjecture \ref{conj:HTaction}. Our proof for a particular $w$ in type $A$ is the content of \S\ref{subsec:bounding}, and is one of the nastiest bits in this paper. The proof has almost no hope of being generalizable beyond type $A$.

The second step was to construct a family of $\l$-equivalences; that they exist is Conjecture \ref{conj:eigenmap}. This in turn had two ingredients. The first was a sufficient condition for a $\l$-equivalence to exist using non-annihilation by polynomials in the Specht module. This analog of the Specht module does exist for arbitrary Coxeter groups, and it seems not to have been studied previously! We do not know what its properties might be, but it might be irreducible in general, which would enable us to reproduce the same sufficient condition.

The second ingredient was an explicit construction of particular chain maps via our computation of triply graded link homology, and in particular, our computation of $\Hom^{\Z\times \Z}(\one,\FT_n)$.  Any computation of $\Hom^{\Z\times \Z}(\one,\FT)$ for other finite Coxeter groups will likely require new ideas (or inhuman tenacity), though we will compute it for the dihedral group in the sequel.

The final step was to apply the Relative Diagonalization Theorem. This step was required because $\FT_n$ might have distinct cells with the same scalar object, and smaller full twists were required to tell these cells apart. Essentially, this step (as well as our computation of $\Hom^{\Z\times \Z}(\one,\FT_n)$ ) relies on the existence of a well-understood tower of Coxeter groups $S_1 \subset S_2 \subset \ldots \subset S_n$, with well-understood relative cell theory. We believe this is a topic which, already in the Grothendieck group, deserves far more study than it has received.


%========================
\subsection{Relation to other work}
\label{subsec:otherwork}
%========================


The idempotents $\PB_T$ and $\PB_\l$ constructed here recover several other idempotents constructed already in other contexts.  We keep the details to a minimum, but hope this will be a useful reference for experts. The categorified Jones-Wenzl idempotents \cite{CK12a,Roz10a,Rose14,Hog15} are obtained from $\PB_T$ when $T$ is the one-row tableau.  The idempotent $\PB_n$ from \cite{Hog15} is in fact isomorphic to $\PB_T$, while the others are obtained from $\PB_T$ by applying a monoidal functor from $\SBim_n$ to an appropriate additive monoidal category (such as $\sl_N$ matrix factorizations or foams).  A category $\OC$ version of the categorified Jones-Wenzl idempotent \cite{FSS12} is conjecturally related to these by a version of Koszul duality \cite{SS12}. 

In \cite{CH12} the second author and Ben Cooper categorified a complete collection of idempotents in the Temperley-Lieb algebra.  The central idempotents $P_{n,k}$ can be obtained from $\PB_{\l}$ for $\l$ a 2-row partition, while the primitive idempotents $P_\e$ can be obtained from $\PB_T$ with $T$ a 2-row standard tableau.  As further special cases, we obtain Rozansky's ``bottom projectors'' $P_{2n,0}$ \cite{Roz10b}.  Applying functors to categories of $\sl_N$ matrix factorizations, our work here generates $\sl_N$ versions of these, indexed by $N$-row partitions / tableau.  These should be closely related to Cautis' idempotents \cite{CautisClasp}.  This connection deserves further exploration and will certainly involve an adaptation of the techniques here to the setting of singular Soergel bimodules.

If $T$ is the one-column partition, then $\PB_T$ is isomorphic to $P_{1^n}$ constructed by the second author and Michael Abel \cite{AbHog17}.

Finally, let us point out the beautiful work \cite{GNR16}.  After sharing early details of this work to Gorsky-Negu\cb{t}-Rasmussen, it was realized that categorical diagonalization provides a natural home for an emerging connection between knot homology and Hilbert schemes.  In particular, the endomorphism rings $\Endgg(\PB_T)$ are conjectured to be isomorphic to some explicit quotients of polynomial rings, which are shown to be isomorphic to the ring of functions on an open chart of the flag Hilbert scheme $\FHilb^n(\C^2)$ in \cite{GNR16}.  The most important ingredient in our story, the eigenmaps, do indeed yield an action of a polynomial ring on $\PB_T$ in which the generators have the predicted degrees.  We refer to \cite{GNR16} for details, and also to \cite{Hog15,AbHog17} for proofs of this conjecture for the one-row and one-column tableaux.  The polynomial action induced by eigenmaps follows from general principals (see for instance the proof of Lemma \ref{lem:grothPvK}).


%========================
\subsection{Structure of the paper}
%========================

The reader may have noticed that we prove our theorem by throwing the whole book at it. We use a huge amount of Hecke algebra theory (Kazhdan-Lusztig theory, Soergel bimodules, cells and
asymptotics, cellularity in type $A$, Robinson-Schensted, specialized results of Mathas and Geck, and even pattern avoidance) and representation theory (Specht modules, Okounkov-Vershik)
together with very fancy homological algebra (convolutions, categorical diagonalization, our recent computation of knot homology). Our positive spin is that we're not really throwing the
book, we're instead reinterpreting the book homologically, taking properties of the Hecke algebra and upgrading them to properties of the full twist complex.

In \S\ref{sec:heckeandsbim} we give background information on Coxeter groups, the Hecke algebra, the Kazhdan-Lusztig basis, Soergel bimodules, and Rouquier complexes. We also describe
minimal complexes and Gaussian elimination, two key concepts in the homological algebra of additive categories. In addition to the basic definitions and familiar properties, we emphasize
several points which may not be familiar to the expert: \begin{enumerate} \item Rouquier's theorem about the canonicity of Rouquier complexes, and its implications for conjugation by
braids. \S\ref{subsec:RouqCanon}. \item How braids act by conjugation on polynomials. \S\ref{subsec:conjugate}. \item Minimal complexes of Rouquier complexes, and in particular, the half twist. \S\ref{subsec:minimalRouq}.
\end{enumerate}

In \S\ref{sec:cells} we give an introduction to the abstract theory of cells in monoidal categories. Then, starting in \S\ref{subsec:twists}, we study the theory of twists, which are invertible complexes sharing some of the key properties of $\HT$ and $\FT$. Sections \S\ref{subsec:twists} and \S\ref{subsec:celltri} are entirely new material.

In \S\ref{sec:cellsplus} we introduce the advanced and asymptotic cell theory of the Hecke algebra. We recall the conjectures (P1-P15) of Lusztig, and we also recall the aforementioned
result of Mathas on the action of the half twist and Sch\"utzenberger duality. Then we discuss some categorical lifts of these ideas. In \S\ref{subsec:sharpconj} we state our main
conjectures for arbitrary finite Coxeter groups: Conjecture \ref{conj:HTaction} stating that the half twist is sharp, implying that \eqref{eq:htactionintro} is categorified in the best
possible way, and Conjecture \ref{conj:eigenmap} stating that there exists a family of $\l$-equivalences. We prove some consequences of these conjecture, in particular that $\FT$ is
prediagonalizable. In \S\ref{subsec:dots} we discuss the existence of certain dot maps in $\SBim$, which are maps of minimal degree from $\one$ to $B_d$ for a distinguished involution
$d$. In future work, we will prove that $B_d$ is a Frobenius algebra object, assuming our conjectures. In \S\ref{subsec:dots}, we prove the unit axiom.

In \S\ref{sec:typeAcells} we restrict our attention to type $A$, and discuss once again the various facts proven about cells and asymptotics, as they apply to the symmetric group. In \S\ref{sec:typeAtwists} we prove Conjecture \ref{conj:HTaction} in type $A$, by a long explicit computation.

In \S\ref{sec:constructing} we prove Conjecture \ref{conj:eigenmap} in type $A$. After some reminders and elementary results in \S\ref{subsec:lambdaEquiv}, we use the dot maps to give an
alternate construction of the Specht module in \S\ref{subsec:specht} in type $A$, and of a ``new" representation of $W$ in other types. We use the irreducibility of the Specht module to
prove a criterion for the existence of a $\l$-equivalence in \S\ref{subsec:lambdaEquivsufficient}. In \S\ref{subsec:FTHHH} we recall our earlier results from \cite{ElHog16a} on the triply
graded homology of the full twist, and in \S\ref{subsec:eigenmaptheorem} we combine these results to prove the existence of $\l$-equivalences.

In \S\ref{sec:diag} we finally prove Theorem \ref{thm:introFTdiag}. After discussing some general consequences of diagonalization, we focus in \S\ref{subsec:subtableaux} and \S\ref{subsec:relativecells} on the interactions between $\HB(S_k)$ and $\HB(S_n)$ for $k < n$, recalling results of Geck and how they apply in type $A$. In \S\ref{subsec:implications} we use these results to prove \eqref{eq:relBigOTintro}, from which the proof of our main theorem follows relatively easily in \S\ref{subsec:proof}. 

%========================
\subsection{Acknowledgments}
%========================

We are indebted to Meinolf Geck for answering many questions by email, and for having answered so many more years ago in his papers! The first author would also like to thank Victor
Ostrik, Sasha Kleshchev, and Geordie Williamson, for a variety of useful conversations and pointers to the literature. The first author would also like to thank Benjamin Young, whose
computations ruled out some overly optimistic ideas, see Remark \ref{rmk:notametric}.

Both authors thank Eugene Gorsky, Andrei Negu\cb{t}, and Jacob Rasmussen for many enlightening conversations, and we thank Mikhail Khovanov for getting our collaboration started.  We would also like to thank Weiquiang Wang for pointing out the very nice proof of Lemma \ref{lem:ftIsCentral_decat}.

The first author was supported by NSF CAREER grant DMS-1553032, and by the Sloan Foundation. The second author was supported by NSF grant DMS-1702274.




%
% \begin{conjecture}[Eigenmap conjecture]\label{conj:introEigenmap}
% Let $W$ be a finite Coxeter group.  For each two-sided cell $\l$ in $W$, let $\ab(\l)\in \Z_{\geq 0}$ denote Lusztig's $\ab$-function, and define the grading shift functor $\Sigma_\l = [-2\ab(\l^t)](2\ab(\l^t)-2\ab(\l))$.  Then for each $\l$ there exists a chain map $\a_\l:\Sigma_\l(\one)\rightarrow \FT$ such that if $x$ is in cell $\l$ then $\Cone(\a_\l)\otimes B_x$ and $B_x\otimes \Cone(\a_\l)$ is homotopy equivalent to a complex in cells strictly less than $\l$.  Here $\l^t$ is the two-sided cell contaning $w_0x$ whenever $x$ is in two-sided cell $\l$, and $\ab(\cdots)$ is Lusztig's $\ab$-function.
% \end{conjecture}
%
%
% \begin{remark}
% Conjecture \ref{conj:introEigenmap} implies that if $B_x$ is in cell $\l$, then $\FT\otimes B_x\simeq \Sigma_\l(B_x)\simeq B_x\otimes \FT$ modulo lower cells, but is stronger in the sense that there exists a map $\a_\l:\Sigma_\l(\one)\rightarrow \FT$ which becomes this ``equivalence modulo lower cells'' upon tensoring with any bimodule in cell $\l$.   This is how one is supposed to interpret our earlier imprecise statement that $\FT$ is categorically diagonalizable in the associated graded of the cell filtration\footnote{Here one thinks of $\Cone(\a_\l)$ as the categorial analague of the matrix of an operator minus the scalar matrix associated to an eigenvalue.}.
% \end{remark}
%
% \begin{remark}
% In case $W=S_n$, the $\ab$-function has an elementary definition: for a partition $\l\vdash n$ we have $\ab(\l)=\sum_i(i-1)\l_i$.  This number is denoted by $\rb(\l)$ throughout the text, with other abbreviations $\cbb(\l)=\ab(\l^t)$ and $\xbb(\l)=\cbb(\l)=\rbb(\l)$.
% \end{remark}
%
% %We prove this conjecture in type $A$, and leave other types for future work.
%
% In order to motivate this conjecture we relate to some ``believable'' properties of the half twists.
% \begin{conjecture}\label{conj:introHTtwist}
% Let $W$ be a finite Coxeter group and $\SBim$ the associated category of Soergel bimodules.
% \begin{enumerate}
% \item Conjugating by $\HT\in \KC^b(\SBim)$ preserves the subcategory of complexes supported in degree zero, up to equivalence.  In the language of \S \ref{}, we would say $\HT$ is \emph{twist-like}.  In fact $\HT\otimes B_x\otimes \HT\inv\simeq B_{w_0x}$ for all $x\in W$.
% \item If $x\in W$ is in cell $\l$, then $\HT\otimes B_x$ is homotopy equivalent to a complex in degrees $0,1,\ldots,\ab(\l^t)$.
% \end{enumerate}
% \end{conjecture}
%
% The following are consequences of this conjecture.
%
% \begin{proposition}
% If conjecture \ref{} is true then
% \begin{enumerate}
% \item $\HT\otimes B_x$ is homotopy equivalent to a complex $X^0\rightarrow \cdots \rightarrow X^{\ab(\l^t)}$ where $X^{\ab(\l^t)}=B_{w_0x}(\ab(\l^t)-\ab(\l))$ and $X^k$ is in cells strictly less than $\l$ for $k<\ab(\l^t)$.
% \item $\FT\otimes B_x$ is homotopy equivalent to a complex $Y^0\rightarrow \cdots \rightarrow Y^{2\ab(\l^t)}$ where $Y^{2\ab(\l^t)}=B_{x}(2\ab(\l^t)-2\ab(\l))$ and $Y^k$ is in cells strictly less than $\l$ for $k<\ab(\l^t)$.
% \end{enumerate}
% \end{proposition}
% %Thus, conjecture \ref{conj:introEigenmap} is reasonable.
%
%
% %We prove this conjecture in type $A$, and leave other types for future work.
% In \S \ref{sec:typeAcells} we prove the following.
%
% \begin{theorem}\label{thm:introEigenmap}
% Conjectures \ref{conj:introHTtwist} and \ref{conj:introEigenmap} are true for $W=S_n$.  Furthermore, the maps $\{\a_\l\}_{\l\vdash n}$ are obstruction free in the sense of \cite{}, so in particular the complexes $\Cone(\a_\l)$ tensor commute with one another up to homotopy.
% \end{theorem}
%
% It follows from this that $\bigotimes_{\l\vdash n}\Cone(\a_\l)$ is contractible, for any ordering of the factors.  Then Theorem \ref{thm:introFTdiag} is a consequence of Theorem \ref{thm:introEigenmap} and the relative diagonalization theorem (Theorem XXX in \cite{}).  To use the relative diagonalization theorem involves the ``relative cell theory'', regarding the action of $\SBim_{n-1}$ on $\SBim_n$.
%
%



\subsection{Notation for complexes} 

Let $\kbbm$ be a commutative ring and let $\AC$ be a $\kbbm$-linear additive category.  We let $\KC(\AC)$ denote the homotopy category of complexes over $\AC$.  We prefer the cohomological convention for differentials, hence complexes will be denoted
\[
\cdots \buildrel d\over \rightarrow C^k \buildrel d\over \rightarrow C^{k+1}\buildrel d\over \rightarrow \cdots.
\]
We call $C^k$ the $k$-th chain object of $C$. We write $C \simeq D$ when $C$ and $D$ are homotopy equivalent complexes. Let $\KC^b(\AC),\KC^-(\AC),\KC^+(\AC)\subset \KC(\AC)$ denote the full subcategories on the complexes which are bounded, respectively bounded from the right, respectively bounded from the left.

For $i \in \Z$ we let $[i]:\KC(\AC)\rightarrow \KC(\AC)$ denote the functor which shifts complexes $i$ units ``to the left.''  More precisely, $C[i]$ denotes the complex $C[i]^k=C^{k+i}$, $d_{C[i]}=(-1)^i d_{C}$.  By a morphism $C \to D$ of degree $i$, we mean a chain map $f:C[-i]\to D$, or the equivalent data of a chain map $C \to D[i]$.

If $C,D\in \KC(\AC)$ are complexes, then we let $\Homc(C,D)$ denote the hom complex.  The $k$-th chain group of $\Homc(C,D)$ is the set of $\Bbbk$-linear maps (not necessarily chain maps) from $C$ to $D$ of degree $k$, and the differential is given by the super-commutator
\[
f\mapsto d_D\circ f  - (-1)^k f\circ d_C
\]
when $f$ has degree $k$. The cohomology of $\Homc(C,D)$ yields the morphisms in the homotopy category: we have isomorphisms of graded $\kbbm$-modules
\[
H^k(\Homc(C,D))\cong \Hom_{\KC(\AC)}(C[-k],D).
\]

\subsection{Graded categories}
\label{subsec:gradedcats}
Let $\Gamma$ be an abelian group.  We say that $\Gamma$ acts on $\AC$ strictly if we are given functors $\Sigma_x:\AC\rightarrow \AC$, $x\in \Gamma$, such that $\Sigma_x\circ \Sigma_y = \Sigma_{x+y}$ for all $x,y\in \Gamma$ and $\Sigma_0=\Id_{\AC}$.  Given a strict action of $\Gamma$ on $\AC$ and objects $A,B\in \AC$, we have the \emph{graded hom space}
\[
\Hom^\Gamma_{\AC}(A,B):=\bigoplus_{x\in \Gamma}\Hom_{\AC}(\Gamma_x(A),B),
\]
which is a $\Gamma$-graded $\Bbbk$-module. It can also be canonically identified with $\bigoplus_{x\in \Gamma}\Hom_{\AC}(A,\Sigma_{-x}(B))$.

The most common examples of strict actions occur when $\Gamma = \Z^d$, and $\Sigma_{i_1,\ldots,i_d}$ is some sort of grading shift functor.  For instance, the homotopy category $\KC(\AC)$ always  has a strict $\Z$ action given by $\Sigma_i =[-i]$, and
\[
\Hom^\Z_{\KC(\AC)}(C,D) = H^{\bullet}(\Homc_{\KC(\AC)}(C,D)).
\]
More generally, if $\AC$ has the structure of a strict $\Gamma$-action, then $\KC(\AC)$ has the structure of a strict $\Gamma\times \Z$ action, via
\[
\Sigma_{x,i}:\KC(\AC)\rightarrow \KC(\AC),\qquad\qquad \Sigma_{x,i}(C) = \Sigma_x(C)[-i].
\]
Given $C,D\in \KC(\AC)$, we also have a complex of homs $\Homc^\Gamma(C,D)$, whose homology is
\[
H(\Homc^{\Gamma}_{\KC(\AC)}(C,D)) \cong \Hom_{\KC(\AC)}^{\Gamma\times\Z}(C,D).
\]

Suppose $\AC$ is additive, monoidal, and comes equipped with a strict action of $\Gamma$ and natural isomorphisms
\[
\Sigma_x(\one)\otimes A\cong \Sigma_x(A)\cong A\otimes \Sigma_x(\one)
\]
for all $x\in \Gamma$ and all $A\in \AC$.  To be more precise we should fix such a family of natural isomorphisms and require certain coherence conditions.  If we are given such data we will regard $\Sigma_x:\AC\rightarrow \AC$ as an \emph{invertible scalar functor}.  More generally, any direct sum of functors $\Sigma_x$ will be called a scalar functor. The object $\Sigma_x(\one)$ is called an \emph{invertible scalar object}.

