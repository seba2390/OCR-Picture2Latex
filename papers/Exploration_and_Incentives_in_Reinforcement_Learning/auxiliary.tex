%!TEX root = main-arxiv.tex

%\NeedsTeXFormat{LaTeX2e}
%\ProvidesPackage{auxiliary}
\newcommand\hmmax{0}
\newcommand\bmmax{0}
\usepackage{verbatim}
\usepackage{booktabs} % For formal tables
\usepackage[utf8]{inputenc}
\usepackage{bm,subfigure}
\usepackage{epsfig,amstext,xspace}
\usepackage{algorithm}
\usepackage[noend]{algpseudocode}
\usepackage[utf8]{inputenc} % allow utf-8 input
\usepackage[T1]{fontenc}    % use 8-bit T1 fonts
\usepackage{url}            % simple URL typesetting
\usepackage{booktabs}       % professional-quality tables
\usepackage{amsfonts}       % blackboard math symbols
\usepackage{bbm}
\iftoggle{acmformat}
{

}
{
  \usepackage{inconsolata}
  \usepackage{geometry}
  \usepackage{amsthm}
  \usepackage{hyperref}
  \usepackage{microtype}
  \usepackage{times}
  \renewcommand{\indent}{\hspace*{\tindent}}

}
\usepackage{upgreek}
\let\widebar\undefined
% Implementation of better \overline alternative from https://tex.stackexchange.com/questions/16337/can-i-get-a-widebar-without-using-the-mathabx-package.

\usepackage{amsmath}
\makeatletter
\let\save@mathaccent\mathaccent
\newcommand*\if@single[3]{%
  \setbox0\hbox{${\mathaccent"0362{#1}}^H$}%
  \setbox2\hbox{${\mathaccent"0362{\kern0pt#1}}^H$}%
  \ifdim\ht0=\ht2 #3\else #2\fi
  }
%The bar will be moved to the right by a half of \macc@kerna, which is computed by amsmath:
\newcommand*\rel@kern[1]{\kern#1\dimexpr\macc@kerna}
%If there's a superscript following the bar, then no negative kern may follow the bar;
%an additional {} makes sure that the superscript is high enough in this case:
\newcommand*\widebar[1]{\@ifnextchar^{{\wide@bar{#1}{0}}}{\wide@bar{#1}{1}}}
%Use a separate algorithm for single symbols:
\newcommand*\wide@bar[2]{\if@single{#1}{\wide@bar@{#1}{#2}{1}}{\wide@bar@{#1}{#2}{2}}}
\newcommand*\wide@bar@[3]{%
  \begingroup
  \def\mathaccent##1##2{%
%Enable nesting of accents:
    \let\mathaccent\save@mathaccent
%If there's more than a single symbol, use the first character instead (see below):
    \if#32 \let\macc@nucleus\first@char \fi
%Determine the italic correction:
    \setbox\z@\hbox{$\macc@style{\macc@nucleus}_{}$}%
    \setbox\tw@\hbox{$\macc@style{\macc@nucleus}{}_{}$}%
    \dimen@\wd\tw@
    \advance\dimen@-\wd\z@
%Now \dimen@ is the italic correction of the symbol.
    \divide\dimen@ 3
    \@tempdima\wd\tw@
    \advance\@tempdima-\scriptspace
%Now \@tempdima is the width of the symbol.
    \divide\@tempdima 10
    \advance\dimen@-\@tempdima
%Now \dimen@ = (italic correction / 3) - (Breite / 10)
    \ifdim\dimen@>\z@ \dimen@0pt\fi
%The bar will be shortened in the case \dimen@<0 !
    \rel@kern{0.6}\kern-\dimen@
    \if#31
      \overline{\rel@kern{-0.6}\kern\dimen@\macc@nucleus\rel@kern{0.4}\kern\dimen@}%
      \advance\dimen@0.4\dimexpr\macc@kerna
%Place the combined final kern (-\dimen@) if it is >0 or if a superscript follows:
      \let\final@kern#2%
      \ifdim\dimen@<\z@ \let\final@kern1\fi
      \if\final@kern1 \kern-\dimen@\fi
    \else
      \overline{\rel@kern{-0.6}\kern\dimen@#1}%
    \fi
  }%
  \macc@depth\@ne
  \let\math@bgroup\@empty \let\math@egroup\macc@set@skewchar
  \mathsurround\z@ \frozen@everymath{\mathgroup\macc@group\relax}%
  \macc@set@skewchar\relax
  \let\mathaccentV\macc@nested@a
%The following initialises \macc@kerna and calls \mathaccent:
  \if#31
    \macc@nested@a\relax111{#1}%
  \else
%If the argument consists of more than one symbol, and if the first token is
%a letter, use that letter for the computations:
    \def\gobble@till@marker##1\endmarker{}%
    \futurelet\first@char\gobble@till@marker#1\endmarker
    \ifcat\noexpand\first@char A\else
      \def\first@char{}%
    \fi
    \macc@nested@a\relax111{\first@char}%
  \fi
  \endgroup
}
\makeatother




% Includes \newtheorem*, \theoremstyle

% DeclarePairedDelimiter
\usepackage{mathtools}
\usepackage{thm-restate}
\usepackage{cleveref}

\crefname{lemma}{lemma}{lemmas}
\Crefname{lemma}{Lemma}{Lemmas}

% DeclareMathOperator
\usepackage{amsmath}
\usepackage{amsfonts}



% \newif\ifpx
% \DeclareOption{px}{\pxtrue}

% \ProcessOptions

% \ifpx
%   \let\iint\relax
%   \let\iiint\relax
%   \let\iiint\relax
%   \let\idotsint\relax
%   \usepackage[varg]{pxfonts}
% \fi
% \DeclareOption{tx}{
%   \let\iint\relax
%   \let\iiint\relax
%   \let\iiint\relax
%   \let\idotsint\relax
%   \usepackage[varg]{txfonts}
% }

% Math delimiters
\DeclarePairedDelimiter{\abs}{\lvert}{\rvert} %
\DeclarePairedDelimiter{\brk}{[}{]}
\DeclarePairedDelimiter{\crl}{\{}{\}}
\DeclarePairedDelimiter{\prn}{(}{)}
\DeclarePairedDelimiter{\nrm}{\|}{\|}
\DeclarePairedDelimiter{\tri}{\langle}{\rangle}
\DeclarePairedDelimiter{\dtri}{\llangle}{\rrangle}

\DeclarePairedDelimiter{\ceil}{\lceil}{\rceil}
\DeclarePairedDelimiter{\floor}{\lfloor}{\rfloor}

% \DeclareMathOperator{\E}{\mathbb{E}} %expecation
\let\Pr\undefined
\let\P\undefined
\DeclareMathOperator*{\En}{\mathbb{E}}
\DeclareMathOperator{\P}{P}
\DeclareMathOperator{\Pr}{\mathbb{P}}

% Arg<x>
\DeclareMathOperator*{\argmin}{arg\,min} % * Places subscript directly under operator
\DeclareMathOperator*{\argmax}{arg\,max}
\DeclareMathOperator*{\arginf}{arg\,inf}
\DeclareMathOperator*{\argsup}{arg\,sup}

% Sets


\iftoggle{acmformat}
{
    \newtheorem{remark}{Remark}[section]
    \newtheorem{claim}{Claim}[section]
    \newtheorem{fact}{Fact}[section]
}
{

% AS: copied from slivkins-theorems.sty
\theoremstyle{plain}
\newtheorem{theorem}{Theorem}[section]
\newtheorem{proposition}[theorem]{Proposition}
\newtheorem{lemma}[theorem]{Lemma}
\newtheorem{fact}[theorem]{Fact}
\newtheorem{claim}[theorem]{Claim}
\newtheorem{construction}[theorem]{Construction}
\newtheorem{corollary}[theorem]{Corollary}
\newtheorem{reduction}[theorem]{Reduction}
\newtheorem{invariant}[theorem]{Invariant}
\newtheorem{extension}[theorem]{Extension}
\newtheorem{assumption}[theorem]{Assumption}

\newtheorem*{theorem*}{Theorem}
\newtheorem*{lemma*}{Lemma}
\newtheorem{claim*}{Claim}

\theoremstyle{definition}
\newtheorem*{definition*}{Definition}
\newtheorem{definition}[theorem]{Definition}
\newtheorem{conjecture}[theorem]{Conjecture}
\newtheorem{example}[theorem]{Example}

\theoremstyle{remark}
\newtheorem{remark}[theorem]{Remark}
\newtheorem{discussion}[theorem]{Discussion}
\newtheorem*{remark*}{Remark}
\newtheorem*{property*}{Property}
\newtheorem{exercise}{Exercise}[section]
\newtheorem*{problem*}{Problem formulation}

}


 \newcommand{\R}{\mathbb{R}}
 \newcommand{\N}{\mathbb{N}}
% \newcommand{\Z}{\mathbb{Z}}
% \newcommand{\Q}{\mathbb{Q}}
% \newcommand{\J}{\mathbb{J}}
% \newcommand{\C}{\mathbb{C}}

\newcommand{\ind}{\mathbbm{1}}    %Indicator


\newcommand{\eps}{\epsilon}
\newcommand{\defeq}{\coloneqq}

\newcommand{\xr}[1][n]{x_{1:#1}}
\newcommand{\yr}[1][n]{y_{1:#1}}
\newcommand{\zr}[1][n]{z_{1:#1}}

\newcommand{\bigO}{\mathcal{O}}

% misc stuff
\newcommand{\fb}{{\bf{}f}}

\newcommand{\mc}[1]{\mathcal{#1}}
\newcommand{\reg}{\mc{R}}
\newcommand{\breg}{D_{\reg}}

\newcommand{\regret}{\emph{Reg}}
\newcommand{\apx}{\emph{ApxReg}}

%Thodoris' additions
\newcommand{\cost}{\ensuremath{\mathit{cost}}}
\newcommand{\opt}{\text{\textsc{Opt}} }
\newcommand{\obj}{\text{\textsc{Obj}} }
\newcommand{\val}{\text{\textsc{val}} }
\newcommand{\sol}{\text{\textsc{sol}} }


%\newenvironment{rtheorem}[3][]{

%\bigskip

%\noindent \ifthenelse{\equal{#1}{}}{\bf #2 #3}{\bf #2 #3 (#1)}
%\begin{it}
%}{\end{it}}

%Figures
\usepackage{tikz}
\usetikzlibrary{arrows}

\usepackage{nicefrac}

\newcommand{\xhdr}[1]{\vspace{1mm} \noindent{\bf #1}}

