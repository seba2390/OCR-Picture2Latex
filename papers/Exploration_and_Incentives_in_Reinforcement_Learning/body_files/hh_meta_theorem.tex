%!TEX root = ../main.tex

\section{Analysis of Hidden Hallucination}


Given a signal space $\sigspace$, a random signal $\sigvar$ taking values in $\sigspace$, and a subset $\sigspace' \subset \sigspace$, an event $\calE$, and a subset of policies $\Pi \subset \Pitotal$, we say that $\sigvar$ \emph{scouts} $\Pi$ under  $\{\sigspace'\mid\calE\}$ written $\sigvar \Wscouts[\sigspace' \mid \calE] \Pi$, if
\begin{align}
 \forall \signal \in \sigspace', \quad \argmax_{\pi \in \Pitotal} \valuemid{\pi}{\sigvar = \signal,\calE} \subseteq \Pi 
\end{align}
Observe that  $ \inf_{\signal \in \sigspace'} \Delta(\Pi \mid \sigvar = \signal) > 0$ implies that $\sigvar \Wscouts \Pi$ (and the converse holds if $\sigspace'$ is finite). Given $\calW \subset (\modtotal,\sigspace)$, we define the projection of $\calW$ onto $\sigspace$
\begin{align}\Proj(\calW \mid \sigspace) = \{\signal \in \sigspace : \exists \model \in \modtotal \text{ s.t. } \}.
\end{align}
and say $\sigvar$ scouts $\Pi$ under $\calW$ iff $\sigvar \Wscouts[\Proj(\calW \mid \sigspace)] \Pi$.



We now begin our analysis of the hidden hallucination algorithm. Our analysis is founded on a very general hidden hallucination principle, which we state presently


\newcommand{\asri}{\textsf{ASRI}}

\newcommand{\trusttext}{tru}
\newcommand{\prhal}{\kernel_{\mathrm{hal}}}

\newcommand{\Prtrust}{\mathbb{P}_{\mathrm{\trusttext}}}
\newcommand{\PrStrust}{\mathbb{P}_{\sigvartrust}}
\newcommand{\sigtrust}{\signal_{\mathrm{\trusttext}}}


\newcommand{\asrHH}{\textsf{ASR}\text{-}\textsf{HH}}
\newcommand{\Zhh}{\calW_{\mathrm{hh}}}
\subsection{General Hidden Hallucination}
\newcommand{\Prhal}{\mathbb{P}_{\mathrm{hal}}}

\newcommand{\Bhide}{\mathbf{B}_{\mathrm{hide}}}

\newcommand{\hhdrawvar}{\boldsf{Draw}_{\mathrm{hh}}}

\newcommand{\Dhh}{\mathcal{D}_{\mathrm{hh}}}

\newcommand{\DelZ}{\Delta_{\mathcal{Z}}}
\newcommand{\boldcalW}{\bm{\mathcal{W}}}
\newcommand{\sigvarcens}{\sigvar_{\mathrm{cens}}}
\newcommand{\sigcens}{\signal_{\mathrm{cens}}}
\newcommand{\sigspacecens}{\sigspace_{\mathrm{cens}}}

\newcommand{\fhh}{f_{\mathrm{hh}}}

\newcommand{\Subsets}{\mathrm{Subsets}}

At present, we establish a generic hidden hallucination lemma. We consider two abstract signals: a \emph{trusted signal} $\sigvartrust \in \sigspace$, and an \emph{censored signal} $\sigvarcens \in \sigspacecens$, random variables jointly distributed with the true model $\modelst \in \modtotal$. Given a realization $\sigcens$ of $\sigvarcens$ we consider the \emph{hallucination probability} $p\in(0,1)$ and \emph{hallucination set} $\calW\subset \modtotal \times \sigspace$
as a function of $\sigcens$; we write this 
as  $(\calW,p)  =\fhh(\sigcens)$ for some 
function $\fhh$ called \emph{HH-function}. 


 %In addition, we have a  function $\fhh: \sigspacecens \to \Subsets(\modtotal \times \sigspace) \times (0,1)$. 


 We first construct a \emph{hallucinated signal} $\sigvarhal \in \sigspace$, a random variable independent of $(\modvarst,\sigvarhal)$ given $\sigvarcens$,  drawn from the distribution $\Prhal[\cdot]$:
\begin{align}
\Prhal[ \signal \mid \sigcens] := \Pr[\sigvartrust = \signal \mid (\sigvartrust,\modvarst) \in \calW\mid\sigvarcens = \sigcens] \label{eq:Prhal}
\end{align}
Finally, the \emph{hidden hallucination signal}  $\sigvarhh \in \sigspace$ induced by $(\modvarst,\sigvartrust,\sigvarcens)$ and $\fhh$ is the signal  which is equal to $\sigvarhal$ with probability $p$, and equal to $\sigvartrust$ otherwise. This is formalized in \Cref{alg:abstract_HH}.
 %We define the induced distribution on $\sigvarhh$ $\Dhh(\Prtrust,\calW,p)$. Recall the conditional value
%Recall For any event measurable $\calE$ with respect to $\Pr$, we define
\newcommand{\sighh}{\signal_{\mathrm{hh}}}
\newcommand{\sighal}{\signal_{\mathrm{hal}}}
\newcommand{\histcens}{\hist^{\censkind}}

%Similarly, we extend the above to conditioning on random variables. 

By selecting $p \in (0,1)$ and the subset $\calW$ effectively, our goal is to incentive a Bayes rational agent to scout some set of exploratory policies $\Pi$, under the set . We now present a general guarantee under which the ``scouting'' condition holds, in terms of the $\gap$ operator defined above. 

\newcommand{\boldp}{\mathbf{p}}
\begin{theorem} 
Let $(\modvarst,\sigvartrust,\sigvarcens)\in \modtotal \times \sigspace \times \sigspacecens$ be a triple of random variables, trusted signal, and censored signal,  let  $\fhh$ be a hidden hallucination function, and let $\sigvarhh$ be the induced HH-signal, with hallucination probability $p$ and hallucination set $\calW \subset \modtotal \times \sigspace$.

Fix a policy set $\Pi\subset \Pitotal$. Then $\sigvarhh$ scouts $\Pi$ under $(\calW \mid \sigvarcens = \sigcens)$
whenever
\begin{align}
 0 < p \le \frac{\Pr[ \calW \mid  \sigcens]\cdot \gap(\Pi \mid \calW;\sigcens)}{3B},
 \end{align} 
where $B$ is a fixed upper bound on  $\valuef{\pi}{\modelst}$ and 
\begin{align}
\gap[\Pi \mid \calW;\sigcens] &:= \inf_{\signal \in \Proj(\calW \mid \sigtot) } \gap[\Pi \mid \sigvartrust = \signal,\sigvarcens = \sigcens ]\\
\Pr[ \calW \mid  \sigcens] &:= \Pr[(\modvarst,\sigvartrust) \in \calW \mid \sigvarcens = \sigcens]
\end{align}  
\end{theorem}
Here, $\gap[\Pi \mid \calW;\sigcens] $ measures the minimal difference between the posterior value of any posterior value of any policy in $\Pi$ and any policy in its complement, given any signal consistent with $\calW$. If all signals consistent with $\calW$ ensure that policies in $\Pi$ are substantially preferable, and if the hallucination probability is small relative to this gap, then whenever \mscomment{\dots}.



\begin{algorithm}[h]
  	\begin{algorithmic}[1]
  	\State{\textbf{Input:}} Realizations
  	$(\sighal,\sigcens)$ of $(\sigvarhal,\sigvarcens)$.
  	\Statex{}~~\quad\qquad HH-function $\fhh$.
  	\State{}\textbf{Choose:} hallucination probability $p \in (0,1)$ and hallucination set  $\calW \subseteq  \modtotal \times \sigspace$ 
  	\Statex{}\algcomment{\%
  	  as a deterministic  function  $(\calW,p) = \fhh(\sigcens)$}
  	\State{\textbf{Define: }} $\sighal \sim \Prhal[\cdot \mid \sigcens]$, where $\Prhal$ is defined in \Cref{eq:Prhal}.
   	\State{\textbf{With Probability $p$} (over independent randomness)}
   	\State\qquad{Return } $\sighh \gets \sighal$
   \State{\textbf{Else:}}
   	\State{\qquad Return } $\sighh\gets \sigtrust$
  \end{algorithmic}
  \caption{Abstract Single-Round Hidden Hallucination }
  \label{alg:abstract_HH}
	\end{algorithm}


\newcommand{\fnew}{f_{\mathrm{new}}}
\newcommand{\fhal}{f_{\mathrm{hal}}}

\subsection{Analyzing The Meta-Hallucination Algorithm}
\mscomment{cannonicall histories}


To state the second assumption, we say that a (raw or censored) history is \emph{valid} if it is consistent with some $\model \in \modelst$.


\begin{definition}[Censored Ledgers] We say the censor. Censoring respects span. 
\end{definition}
\mscomment{Remark}

We define a ledger a



\begin{definition} We say that a pair 
\end{definition}

\begin{theorem} Suppose that the environment exploration dimension $d$, and select $N \ge$ such that, for any cannonical censored history $\hist$, we have 
where $\modclass = \fhal(\hist)$ and $\Pinew = \fnew(\Pi[H]]$. Consider instantiating \Cref{alg:meta_HH}  with $\modclass_{\ell} = \fhal(\histcens_{\calK_{\ell}})$ and $N$. Then, with probability $1$, \Cref{alg:meta_HH} model-explores $\modelst$ after at most $dN$-episodes.
\end{theorem}

\mscomment{only depends on the cannonicall }






