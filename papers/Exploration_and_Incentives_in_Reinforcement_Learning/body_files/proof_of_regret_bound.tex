%!TEX root = ../main-arxiv.tex
\newcommand{\calV}{\mathcal{V}}
\newcommand{\Vvis}{\calV_{\mathrm{vis}}}
\newcommand{\Vreach}{\calV_{\mathrm{rea}}}
\newcommand{\modtil}{\tilde{\model}}
\newcommand{\valuefrest}[2]{\valuename_{\mathrm{rstr}}(#1;#2)}

In this appendix, we restate and prove \Cref{prop:revelation}.
%We restate \Cref{prop:revelation}.
\proprevelation*

At a high-level, the proof requires two steps. First, we show that, with high probability, any model $\model$ drawn from the posterior given the signal $\signal$ has similar rewards and transitions to those of $\modst$ under all triples which are $\rho$-reachable under $\modst$. This step invokes a Bayesian Chernoff argument similar in spirit to those in \Cref{sec:proof_hall_good_event}. In the second step, we argue that similarity on $\rho$-reachable triples implies uniformly that for all Markovian policies $\pi$, $\valuef{\pi}{\model}$ and $\valuef{\pi}{\modst}$ are close by. As a consequence, we conclude that any BIC policy (one which optimizes $\Exp[\valuef{\pi}{\modst} \mid \signal]$) must be near optimal for $\modst$.

\begin{remark} Because assume the ledger $\ledger$ contain \emph{all} trajectories collected by the mechanism, all posteriors are cannonical.
\end{remark}

\paragraph{Preliminaries.}
We begin with a remark on notation. In the majority of the paper, we were concerned with sets of \emph{undervisited} triples $\xah$, notated $\calU$. In this section of the paper, we are concered more with sets of triples $\calV$ which we wish are sufficiently visited, or which we  wish to be so. The two sets of interest are
\begin{definition}[Reachable and Visited Set] Given $\rho > 0$ and a model $\modst$, $\Vreach(\rho,\modst)$ denote the sets of triples $\xah$ which are  $\rho$-reachable under $\modst$. We define $\Vreach(\modst)$ as the set of all reachabile triples $\xah$ for any postive $\rho$; i.e. $\Vreach(\modst) = \bigcup_{\rho > 0}\Vreach(\rho,\modst)$. \footnote{In other words, $\Vreach(>0,\modst)$ is the compliment of the set of triples which cannot be reached by \emph{any} policy under $\modst$.} Given $n \ge 0$ and ledger $\ledger_K$, we let $\Vvis(n)$ denote the set of triples $\xah$ which have been visited at least $n$ times in ledger $k$.
\end{definition}

We now recall the definition of transition-similiarity, modified to be stated in terms of $\calV$-sets.
\begin{restatable}[Transition-Similar]{definition}{defnsimilarV} \label{defn:similarity_V} Let $\|\cdot\|_{\ell_1}$ denote the $\ell_1$-distance between probability distributions.  Given $\calV \subset [S] \times [A] \times [H]$, we say two models $(\model,\modst)$ are $\varepsilon$-transition-similar on $\calV$ if (i) $\|\sfp_{\model}(\cdot \mid 0 ) - \sfp_{\modst}(\cdot \mid 0)\|_{\ell_1} \le \varepsilon $ \emph{(closeness of initial state distribution)}, and (ii) for each $(x,a,h) \in \calV$, $\|\sfp_{\model}(\cdot \mid x,a,h) - \sfp_{\modst}(\cdot\mid x,a,h)\|_{\ell_1} \le \varepsilon$ \emph{(closeness of transitions on $\calV$)}.
\end{restatable}

We also introduce the analogous notion of \emph{reward similarity}
\begin{definition}[Reward Similar]\label{defn:similarity_R}  Given $\calV \subset [S] \times [A] \times [H]$, we say two models $(\model,\modst)$ are $\varepsilon$-reward-similar on $\calV$ if for each $(x,a,h) \in \calV$, $|\sfr_{\model}(x,a,h) - \sfr_{\modst}( x,a,h)| \le \varepsilon$.
\end{definition}



\paragraph{Bayesian Concentration}
We begin by arguing that there exist accurate estimators $\thetar$ and $\thetap$ of the rewards and transitions which are well defined for all states visited at least $n$ times. The following is a modification of \Cref{lem:conc_bounds}, whose proof is similar and omitted in the interest of brevity.
\begin{lemma}[Chernoff Concentration Bounds]\label{lem:conc_bounds_two} Given an $n \ge 0$ and let $\ledger_K$ be an uncensored ledger containing at least $K \ge n$ trajectories. Define the error bounds
\begin{align*}
\textstyle \epsr(n,\delta) := \sqrt{ \frac{2\log(1/\delta)}{n}}, \quad \text{ and } \epsp(n,\delta):= 2\sqrt{ \frac{2(S\log(5) + \log(1/\delta))}{n}}.
\end{align*}
Then, there exist estimators $\theta_r(x,a,h)$, $\theta_{\sfp}(x,a,h)$, $\theta_\sfp(\cdot \mid 0)$,  of the rewards,  transition probabilities, initial state distribution, which are functions of the ledger $\ledger_K$  such that
\begin{align*}
&\Pr[\xah \in \Vreach(n) \cap \{|\theta_r(x,a,h) - \rmodst(x,a,h)| \ge \epsr(n,\delta)\} ] \le \delta \quad \text{and}\\
&\Pr[\xah \in \Vreach(n) \cap \{\|\theta_{\sfp}(x,a,h) - \pmodst(\cdot \mid x,a,h)\|_{\ell_1} \ge \epsp(n,\delta)\} ] \le \delta\\
&\Pr[\|\thetap(\cdot \mid 0) - \pmodst(\cdot \mid 0)\|_{\ell_1} \ge \epsp(n,\delta)] \le \delta.
\end{align*}
\end{lemma}
We now invoke the Bayesian concentration argument due to \cite{Selke-PoIE-ec21}. Let $\model'$ be a drawn from the posterior conditioned on $\ledger$, $\model' \sim \Pr[\cdot \mid \ledger]$. Then, $(\model',\ledger)$ and $(\modst,\ledger)$ have the same distribution. Hence, the estimators $\theta_r(\cdot)$ and $\theta_p(\cdot)$ (a function of only the ledger $\ledger$) also concentrate around $\sfr_{\model'}$ and $\sfp_{\model'}$, in the sense of \Cref{lem:conc_bounds_two}. Thus, by unions bounds and applications of the triangle inequality, it holds that
\begin{align*}
&\Pr[\xah \in \Vreach(n)\cap \{|\sfr_{\model'}(x,a,h) - \rmodst(x,a,h)| \ge 2\epsr(n,\delta)\} ] \le 2\delta \quad \text{and}\\
&\Pr[\xah \in \Vreach(n) \cap \{\|\sfp_{\model'}(x,a,h) - \pmodst(\cdot \mid x,a,h)\|_{\ell_1} \ge 2\epsp(n,\delta)\} ] \le 2\delta\\
&\Pr[\|\sfp_{\model'}(\cdot \mid 0) - \pmodst(\cdot \mid 0)\|_{\ell_1} \ge 2\epsp(n,\delta)] \le 2\delta.
\end{align*}
Recalling the definitions of transition- and reward-similarity, another union bound yields the following lemma:
\newcommand{\Esim}{\mathcal{E}_{\mathrm{sim}}}
\newcommand{\Ebarsim}{\bar{\mathcal{E}}_{\mathrm{sim}}}
\begin{lemma}\label{lem:Esim} Let $\modst$ denote the true model, and consider a sample $\model' \sim \Pr[\cdot \mid \ledger]$. Then for any $\delta \in (0,1)$, the following event $\Esim(\delta)$ holds with probability $1 - 6SAH\delta$ over all randomness in $(\modst,\model',\ledger_K)$:
\begin{align}
\Esim(n,\delta) := \left\{(\modst,\model') \text{ are }\,\, \left\{\begin{matrix} 2\epsr(n,\delta)\text{-reward-similar and}\\
2\epsp(n,\delta)\text{-transition-similar}\\
\end{matrix}\right\} \text{ on } \Vreach(n).\right\}
\end{align}
\end{lemma}

Lastly, we convert \Cref{lem:Esim} into a slightly more useful form to reason about sampling from the posterior conditioned on a fixed true model $\modst$ and ledger $\ledger_K$.
\begin{lemma}\label{lem:Esim_two} Fix a $\delta_1,\delta_2 \in (0,1)$, and define the event
\begin{align*}
\Ebarsim(n,\delta_1,\delta_2) := \{\Pr[\Esim(n,\delta_1) \mid \ledger_K,\modst] \ge 1 - \delta_2 \}
\end{align*}
Then, $\Pr[\Ebarsim(\delta_1,\delta_2)] \ge 1 - \frac{6SAH\delta_1}{\delta_2}$.
\end{lemma}
\begin{proof} We apply Markov's inequality.
\begin{align*}
\Pr[\Esim(\delta_1)^c] &= \Exp[\Pr[\Esim(\delta_1)^c \mid \ledger_K,\modst] ]\\
&\ge  \Exp[\delta_2 \cdot\ind\{\Pr[\Esim(\delta_1)^c \mid \ledger_K,\modst] \ge - \delta_2\}]\\
&:=  \Exp[\delta_2 \cdot \ind\{\Ebarsim(\delta_1,\delta_2)^c\}]\\
&= \delta_2 \cdot\Pr[\Ebarsim(\delta_1,\delta_2)^c].
\end{align*}
From \Cref{lem:Esim}, we know that $\Pr[\Esim(\delta_1)^c] \le 6\delta_1$. Therefore, $\Pr[\Ebarsim(\delta_1,\delta_2)^c] \le 6\delta_1/\delta_2$. The bound follows.
\end{proof}

\paragraph{Similarity implies close values.} With the above preliminaries in place, we first show that the simulation lemma which states that if two models $(\model,\modst)$ are both transition-similiar and reward-similar on $\Vreach(\rho,\modst)$, then \emph{all} policies have similar value.

Our first step is to show that the set of non-$\rho$-reachable triples $\Vreach(\rho,\modst)^c$ for $\modst$ is hard to reach under any $\model$ which is $\epsp$-transition-similar to $\modst$ on the $\rho$-reachable triples $\Vreach(\rho,\modst)$.
\begin{lemma}\label{lem:mass_of_reachable} Let $(\model,\modst)$ be $\epsp$-similar on $\Vreach(\rho,\modst)$. Then for policy $\pi \in \Pimarkov$,
\begin{align*}
\sfP^\pi_{\modst}[\exists (\bmx_h,\bma_h,h) \in \Vreach(\rho,\modst)^c] &\le \rho SH.\\
\sfP^\pi_{\model}[\exists h : (\bmx_h,\bma_h,h) \in \Vreach(\rho,\modst)^c] &\le \rho SH + H^2\epsp.
\end{align*}
\end{lemma}
\begin{proof} This first inequality follows from a union bound,
\begin{align*}
\sfP^\pi_{\modst}[\exists h : (\bmx_h,\bma_h,h) \in \Vreach(\rho,\modst)^c] &\le \sum_{x,a,h} \ind\{a = \pi(x,h)\} \ind\{(x,a,h) \in \Vreach(\rho,\modst)^c\}\sfP^\pi_{\modst}[ (\bmx_h,\bma_h,h) = (x,a,h)]\\
&\le \sum_{x,a,h} \ind\{a = \pi(x,h)\} \ind\{(x,a,h) \in \Vreach(\rho,\modst)^c\} \rho\\
&\le \rho SH.
\end{align*}
Moreover, by $\epsp$-similarity on $\Vreach(\rho,\modst)$, \Cref{lem:visitation_comparison_general} entails the following inequality, which proves our desired bound:
\begin{align*}
\left|\sfP^\pi_{\modst}[\exists h : (\bmx_h,\bma_h,h) \in \Vreach(\rho,\modst)^c] - \sfP^\pi_{\model}[\exists h : (\bmx_h,\bma_h,h) \in \Vreach(\rho,\modst)^c]\right| \le \binom{H}{2}\epsp \le H^2 \epsp. \quad\qedhere
\end{align*}
\end{proof}

\newcommand{\indreach}{\ind_{\mathrm{reach}}}
Using the above, we establish closeness of values:
\begin{lemma}[Simulation on Reachable Set]\label{lem:reachable_simulation} Fix $\rho > 0$, and suppose that  $(\model,\modst)$ are  $\epsp$-transition-similar and $\epsr$-reward-similar on $\Vreach(\rho,\modst)$. Introduce the indicatior
\begin{align*}
\indreach :=\begin{cases}0 & \Vreach(\rho,\modst) = \Vreach(\modst) = \Vreach(\model)\\
1 & \text{otherwise},
\end{cases}
\end{align*}
which is equal to $1$ unless the set of $\rho$-reachable triples under $\modst$ coincide with the set of reachable (for any $\rho'$) triples under either $\modst$ or $\model$).  Then, for any $\pi \in \Pimarkov$,
\begin{align}
\left|\valuef{\pi}{\model} - \valuef{\pi}{\modst}\right| \le  H^2 S \rho \cdot\indreach  + 2 H^3 \epsp + H\epsr. \label{eq:first_bound_reach}
\end{align}
\end{lemma}
\begin{proof}

Our strategy is to invoke the simulation lemma, \Cref{lem:visitation_comparison_general}, twice.  For $h \in [H]$, introduce the $\sfP$-events $\scrE_h := \{(\bmx_\tau,\bma_{\tau},\tau) \in \Vreach(\rho,\modst),~ \forall \tau < h\}$, and let us define the \emph{restricted value function} for any model $\model'$ via
\begin{align}
\valuefrest{\pi}{\model'} := \sfE^{\pi}_{\model'}\left[\sum_{h=1}^H \sfr_{\model'}(\bmx_h,\bma_h,h) \ind\{\scrE_{h+1}\}\right],
\end{align}
which only counts rewards accumulated on trajectories which remain on $\rho$-reachable states $\Vreach(\rho,\modst)$ up until time and \emph{including} step $h$. We can observe then that, for any model $\model'$,
\begin{align*}
\valuefrest{\pi}{\model'} \le \valuef{\pi}{\model'} &\le \valuefrest{\pi}{\model'}  +  H\sfP^{\pi}_{\model'}[\exists h: (\bmx_h,\bma_h,h) \in \Vreach(\rho,\modst)^c ]\\
&= \valuefrest{\pi}{\model'}  +  H\sfP^{\pi}_{\model'}[\exists h: (\bmx_h,\bma_h,h) \in \Vreach(\rho,\modst)^c ] \cdot \indreach,
\end{align*}
where we can multiply by the indicator $\indreach$ since if $\indreach = 0$,  $\sfP^{\pi}_{\model'}[\exists h: (\bmx_h,\bma_h,h) \in \Vreach(\rho,\modst)^c ]  = 0$. In light of \Cref{lem:mass_of_reachable}, we find then that $\modst$ and $\model$ satisfy,
\begin{align*}
\valuefrest{\pi}{\modst} \le \valuef{\pi}{\modst} \le \valuefrest{\pi}{\modst}  +  H^2 S \rho \cdot \indreach \\
\valuefrest{\pi}{\model} \le \valuef{\pi}{\model} \le \valuefrest{\pi}{\model}  +  H^2 S \rho \cdot \indreach + H^3 \epsp.
\end{align*}
Together, these bounds imply that
\begin{align}\label{eq:value_diff_inter_rest}
 |\valuef{\pi}{\model}  - \valuef{\pi}{\modst}| \le  H^2 S \rho \cdot \indreach + H^3 \epsp + |\valuefrest{\pi}{\modst} - \valuefrest{\pi}{\model}|.
\end{align}

It remains to bound the difference in restricted values $|\valuefrest{\pi}{\modst} - \valuefrest{\pi}{\model}|$. To do so, introduce an interpolating model $\modtil$ whose transitions $\modtil$ are the same as those in $\model$ ($\sfp_{\modtil} = \sfp_{\model}$), but whose rewards are the same as those in $\modst$ ($\sfr_{\modtil} = \sfr_{\modst}$). By the triangle inequality and \Cref{eq:value_diff_inter_rest},
\begin{equation}\label{eq:pennultimate_value_bound}
\begin{aligned}
 &|\valuef{\pi}{\model}  - \valuef{\pi}{\modst}| \le  H^2 S \rho \cdot \indreach + H^3 \epsp \\
 &\qquad+ \underbrace{|\valuefrest{\pi}{\modst} - \valuefrest{\pi}{\modtil}|}_{(i)} + \underbrace{|\valuefrest{\pi}{\model} - \valuefrest{\pi}{\modtil}|}_{(ii)}.
\end{aligned}
\end{equation}
To bound term $(i)$, define the event $\bar{\scrE}_{h} := \ind\{(x,a,h) \in \Vreach(\rho,\modst)\}$, so that $\scrE_{h+1} = \bar{\scrE}_{h} \cap \scrE_{h}$. Defining the reward $\rtil(x,a,h) := \ind\{\bar{\scrE}_h\}\sfr_{\modst}\xah$, we then have
\begin{align}
\ind\{\scrE_h\}\rtil(x,a,h) = \ind\{\scrE_h\}\ind\{\bar{\scrE}_h\}\sfr_{\modst}\xah = \ind\{\scrE_{h+1}\}\sfr_{\modst},
\end{align}
and since $\modst$ and $\modtil$ have the same reward function $\ind\{\scrE_h\}\rtil(x,a,h)= \sfr_{\modtil}(\bmx_h,\bma_h,h) \ind\{\scrE_{h+1}\}$. Therefore,
\begin{align*}
\valuefrest{\pi}{\modst} - \valuefrest{\pi}{\modtil} &= \sfE^{\pi}_{\modst}\left[\sum_{h=1}^H \sfr_{\modst}(\bmx_h,\bma_h,h) \ind\{\scrE_{h+1}\}\right] - \sfE^{\pi}_{\modtil}\left[\sum_{h=1}^H \sfr_{\modtil}(\bmx_h,\bma_h,h) \ind\{\scrE_{h+1}\}\right]\\
 &= \sfE^{\pi}_{\modst}\left[\sum_{h=1}^H \rtil(\bmx_h,\bma_h,h) \ind\{\scrE_{h}\}\right] - \sfE^{\pi}_{\modtil}\left[\sum_{h=1}^H \rtil(\bmx_h,\bma_h,h) \ind\{\scrE_{h}\}\right].
\end{align*}
Hence, the difference $\valuefrest{\pi}{\modst} - \valuefrest{\pi}{\modtil}$ takes the form of precisely the quantity bounded by \Cref{lem:visitation_comparison_general}, implying that (with the crude bound $\binom{H}{2} \le H^2$)
\begin{align}
(i) \le |\valuefrest{\pi}{\modst} - \valuefrest{\pi}{\modtil}| \le H^2 \epsp.
\end{align}
Next, we bound term $(ii)$, which, due to the fact that $\modtil$ and $\model$ have the ssame transitions, (and again $\modtil$ has the same rewards as $\modst$) takes the form
\begin{align*}
\valuefrest{\pi}{\model} - \valuefrest{\pi}{\modtil} &= \sfE^{\pi}_{\model}\left[\sum_{h=1}^H \sfr_{\model}(\bmx_h,\bma_h,h) \ind\{\scrE_{h+1}\}\right] - \sfE^{\pi}_{\modtil}\left[\sum_{h=1}^H \sfr_{\modtil}(\bmx_h,\bma_h,h) \ind\{\scrE_{h+1}\}\right]\\
 &=\sfE^{\pi}_{\model}\left[\sum_{h=1}^H (\sfr_{\model}(\bmx_h,\bma_h,h) - \sfr_{\modst}(\bmx_h,\bma_h,h))\ind\{\scrE_{h+1}\}\right].
\end{align*}
We further observe that $\scrE_{h+1} = 0$ unless $(\bmx_h,\bma_h,h) \in \Vreach(\rho,\modst)$, and when this occurs, $\epsr$-reward-similarly implies that $|(\sfr_{\model}(\bmx_h,\bma_h,h) - \sfr_{\modst}(\bmx_h,\bma_h,h))| \le \epsr$. Hence,
\begin{align*}
|\valuefrest{\pi}{\model} - \valuefrest{\pi}{\modtil}| \le  \sfE^{\pi}_{\model}\left[\sum_{h=1}^H \ind\{\scrE_{h+1}\}\epsr\right] \le H\epsr.
\end{align*}
In summary, we have bounded term $(i)$ by $H^2 \epsp$ and term $(ii)$ by $H\epsr$. From \Cref{eq:pennultimate_value_bound},
\begin{align*}
 &|\valuef{\pi}{\model}  - \valuef{\pi}{\modst}| &\le  H^2 S \rho \cdot \indreach+ H^3 \epsp + H \epsr + H^2 \epsp \\
 &\le H^2 S \rho \cdot \indreach + 2 H^3 \epsp + H\epsr. \qquad\qedhere
\end{align*}
\end{proof}

\paragraph{Concluding the proof. }
\newcommand{\Etrav}{\mathcal{E}_{\mathrm{trave}}}
To conclude, suppose that that the following two events hold:
\begin{itemize}
\item The true model $\modst$ has be $(\rho,n)$-traversed, i.e. $\Etrav := \{\Vvis(n) \supset \Vreach(\rho,\modst)\}$ holds.
\item Recall the event $\Ebarsim(n,\delta_1,\delta_2)$ from \Cref{lem:Esim_two}. In words, this is the event that, with probability $1 - \delta_2$, a draw $\model' \sim \Pr[\cdot \mid \ledger_K,\modst]$ is both $2\epsr(n,\delta_1)$-reward-similar and $2\epsp(n,\delta_1)$-transition-similar to $\modst$ on the visited triples $\Vvis(n)$. We take $\delta_2 = 1/n$ and $\delta_1 = \delta_2\cdot \delta/6SAH = \epsilon \delta/6SAH n$, where $\delta \in (0,1)$ is our target failure probability.
\end{itemize}
Observe that
\begin{align}
\Pr[\Etrav \cap \Ebarsim(n,\delta_1,\delta_2)] \ge  1 - \Pr[\Etrav] - \Pr[\Ebarsim] \overset{(i)}{\ge} 1 - \delta - \frac{6SAHn \delta_1}{\delta_2} \overset{(ii)}{\ge} 1 - 2\delta,
\end{align}
where $(i)$ uses that our algorithm satisfies $(\rho,n,\delta,K_0)$-\traversal and \Cref{lem:Esim_two}, and $(ii)$ replaes our chose of $\delta_1,\delta_2$.


To conclude, we assume $\Etrav \cap \Ebarsim(n,\delta_1,\delta_2)$ holds, and bound  $ |\valuef{\hat{\pi}}{\modst} - \valuef{\pi_{\star}}{\modst}|$,  where $\hat{\pi} \in \argmax_{\pi \in \Pimarkov}\Exp'[\valuef{\pi}{\model'} ]$, and $\pi_{\star} \in \argmax_{\pi \in \Pimarkov}\Exp[\valuef{\pi}{\modst} ]$.


To this end, $\Pr'[\cdot], \Exp'[\cdot]$ denote a shorthand expectation over a model $\model' \sim \Pr[\modst \in \cdot \mid \ledger_K]$ (treating $\modst$ and $\ledger_K$ as fixed).  Then on their intersection $\Etrav \cap \Ebarsim(n,\delta_1,\delta_2)$,
\begin{align}\label{eq:pEst}
\calE_{\star} := \left\{ \begin{matrix} \model' \text{ and }\modst \text{ are } 2\epsr(n,\delta_1)\text{-reward-similar and }\\
 2\epsp(n,\delta_1)\text{-transition-similar} \text{ on } \Vreach(\rho,\modst)\end{matrix}\right\} \text{ has } \Pr'[\calE_{\star}] \ge 1 - \delta_2.
\end{align}
On $\calE_{\star}$, \Cref{lem:reachable_simulation} implies that for any policy $\pi \in \Pi$,
\begin{equation}\label{eq:eprime_diff}
\begin{aligned}
|\valuef{\pi}{\model'}  - \valuef{\pi}{\modst}| &\le H^2 S \rho + 4 H^3 \epsp(n,\delta_1) + 2H\epsr(n,\delta_1) \\
&\le H^2 S \rho + 6H^3 \epsp(n,\delta_1),
\end{aligned}
\end{equation}
where we use $\epsp(n,\delta_1) \ge \epsr(n,\delta_1)$ as defined above.

Therefore, since $\Pr'[\calE_{\star}] \ge 1 - \delta_2$ on $\Etrav \cap \Ebarsim(n,\delta_1,\delta_2)$, it holds that any policy $\pi \in \Pimarkov$ and reachability lower bound $\rho_{\min}$,
\begin{align*}
&|\Exp'[\valuef{\pi}{\model'} ] - \valuef{\pi}{\modst}| \\
&= |\Exp'[\valuef{\pi}{\model'} - \valuef{\pi}{\modst}]| \\
&\le \Exp'[|\valuef{\pi}{\model'} - \valuef{\pi}{\modst}|]\\
&\le H\Exp'[\ind\{\calE_\star\}] + H\Exp'[\ind\{\calE_{\star}\}|\valuef{\pi}{\model'} - \valuef{\pi}{\modst}|]\\
&\overset{(i)}{\le}  H\delta_2 + H^2 S \rho \cdot \ind_{\{\rho > \rho_{\min}\}}+ 6H^3 \epsp(n,\delta_1)\\
&\overset{(ii)}{\le}H^2 S \rho \cdot \ind_{\{\rho > \rho_{\min}\}}+ 7 H^3 \epsp(n,\delta_1) := \bar{\varepsilon}.
\end{align*}
where $(i)$ last step uses \Cref{eq:pEst,eq:eprime_diff}, together with the fact that if $\rho_{\min}$ is a reachability bound, then for any $\rho < \rho_{\min}$, $\Vreach(\rho,\modst) = \Vreach(\modst) = \Vreach(\model')$ for all $\modst,\model'$ in the support of the prior $\prior$. In addition, $(ii)$ uses that $H\delta_2 = H/n \le 7 H^3 \epsp(n,\delta_1)$.

In particular, if $\hat{\pi} \in \argmax_{\pi \in \Pimarkov}\Exp'[\valuef{\pi}{\model'} ]$, and $\pi_{\star} \in \argmax_{\pi \in \Pimarkov}\Exp[\valuef{\pi}{\modst} ]$, we conclude that on $\calE_{\star}$.
\begin{align*}
 |\valuef{\hat{\pi}}{\modst} - \valuef{\pi_{\star}}{\modst}| &\le 2\bar{\varepsilon} \\
 &= {\textstyle \mathcal{O}(H^2)\cdot\left(S \rho \cdot \ind_{\{\rho > \rho_{\min}\}}+ H\sqrt{\frac{S + \log(SAHn/\delta)}{n}}\right)}.\qquad\qed
\end{align*}



\begin{comment}


Under the assumptions of the lemma (and for our choices of $\delta_1,\delta_2)$, we have $\rho \le \max\{\frac{\epsilon}{8SH^2},\rho_{\min}\}$, $\delta_2 = \frac{\epsilon}{8H}$, and  $\epsp(n,\delta_1) \le \frac{\epsilon}{24H^3}$ (as one can check)\footnote{Recall $\epsp(n,\delta):= 2\sqrt{ \frac{2(S\log(5) + \log(1/\delta))}{n}}$. Plugging in $\delta \gets \delta_1 = \epsilon \delta/48H,$, we need
$ \epsp(n,\delta_1):= 2\sqrt{ \frac{2(S\log(5) + \log(\frac{48H}{\epsilon\delta} ))}{n}} \le  \frac{\epsilon}{24H^3}.$
This holds for $n \ge 1152H^3\epsilon^{-2}(S\log 5 + \log \tfrac{18 H}{\epsilon\delta} )$, the condition stated in the proposition.} Thus,
\begin{align*}
|\Exp'[\valuef{\pi}{\model'} ] - \valuef{\pi}{\modst}| \le \frac{\epsilon}{2}
\end{align*}
In particular, if $\pi \in \argmax_{\pi' \in \Pimarkov}\Exp'[\valuef{\pi'}{\model'} ]$ is BIC, then an application of the triangle inequality
\begin{align*}
\valuef{\pi}{\modst} \ge \max_{\pi'}\valuef{\pi}{\modst} - 2\cdot\frac{\epsilon}{2} = \OPT(\modst) -\epsilon.
\end{align*}
\end{comment}
