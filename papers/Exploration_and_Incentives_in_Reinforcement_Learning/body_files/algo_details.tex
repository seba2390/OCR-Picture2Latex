%!TEX root = ../main.tex

\section{Detailed Specification of the Algorithm}
\label{app:detailed-spec}

This section provides a formal detailed specification of our main algorithm \Cref{alg:MDP_HH}. \Cref{alg:MDP_HH_app} contains the corresponding pseudocode, which refers to the equations defining the relevant objects of interest. We then expose the pseudocode further in a series of bullets below. 
\begin{algorithm}[h]
  	\begin{algorithmic}[1]
  	\State{}\textbf{Input: } Phase length $\nphase$, target sample size $\nlearn$, tolerance $\epspunish$
    \For{phase $\ell= 1,2,\dots$:}
    \Statex{}\algcomment{\% Phase $\ell$ consists of the next $\nphase$ episodes}
    \State{}$\Phase_{\ell} = (\ell-1) \nphase  + [\nphase] \subset \N$
    \State{}Let $\ledrawl$ and $\ledcensl$ denote the raw and totally censored ledgers
    \Statex{}\qquad\qquad from previous hallucination episodes $\calK_{\ell} := \{ \kexpl[1],\dots,\kexpl[\ell-1]\}$.
    \State{}Construct set $\calU_{\ell}$ of {triples $\xah$ which are not fully-explored} %visited less than $\nlearn$ times
     (\Cref{eq:calU_def}).
    %\Comment{phase $\ell$ lasts for the next $N$ episodes}
    \State{}Draw ``hallucination episode" $\kexpl$ uniformly from $\Phase_{\ell}$
    %\State{}Draw episode $\kexpl$  uniformly from $\PhaseEpisodes_{\ell}$
    %\State{}Draw $\kexpl \unifsim N(\ell - 1) + [N]$
    \For{each episode $k \in \Phase_{\ell}$}
    \If{$k = \kexpl$}
    \algcomment{\qquad\% hallucination}
     \State{}Compute hallucination set $\modclasshall(\epspunish)$ as in \Cref{eq:modclass_hall}.
     \State{}Draw $\modelhall\sim \Pr[\modst \in \cdot \mid \ledcensl, \modst \in \modclasshall]$. \algcomment{\qquad\% hallucinated MDP}
    \State{}Reveal hallucinated $\ledrevk  \gets \ledhall$ by obtained by drawing rewards
    \Statex{}\qquad\qquad\qquad from $\modhall$ (\Cref{eq:hallucinate_rewards,eq:hallucinate_trajectories,eq:hallucinate_ledger})
    \Else{} \algcomment{\qquad\% exploitation}
     \State{}Reveal honest ledger $\ledrevk \gets \ledhonl := \cens(\ledrawl;\calU_{\ell})$
     \algcomment{\qquad\% true raw ledger}
    \EndIf
    \State{}Observe the trajectory $\traj_k$ from this episode.
    \EndFor
    \EndFor
  \end{algorithmic}
  \caption{Hidden Hallucination for MDPs (\mdphh)}
  \label{alg:MDP_HH_app}
	\end{algorithm}


\bluepar{Specification of Algorithm}
\begin{itemize}
\item A  each phase $\ell = 1,2,\dots$, we allocate $\nphase$-episodes, whose set is denoted  $\Phase_{\ell} :=  (\ell - 1) \nphase  + [\nphase] \subset \N$, and select a \emph{hallucination episode} $\kexpl$ uniformly at random;  we let $\kexpl = \ell$ for the initial phases $\ell \in [\nlearn]$. We define $\calK_{\ell} := \{\kexpl[\ell']: \ell' \in [\ell-1]\}$ as the set of all hallucination episodes up to that of phase $\ell$.
\item We define phase-$\ell$ censoring set $\calU_{\ell}$, considering of all triples $(x,a,h)$ which {are not fully-explored} by phase $\ell$, that is, triples which have appeared fewer than $\nlearn$-times in the hallucinations episodes $\calK_{\ell}$:
\begin{align}
&\calU_{\ell} := \{(x,a,h) \in [S] \times [A] \times [H] : N_{\ell}(x,a,h) < \nlearn\}, ~ \label{eq:calU_def}\\
&\text{ where } N_{\ell}(x,a,h) := \textstyle\sum_{k \in \calK_{\ell}}^{\ell-1} \I\{(x,a,h) \in \traj_{k}\} \nn.
\end{align}
Because triples $\xah \in \calU_{\ell}^c$ have been observed sufficiently many (i.e. $\nlearn$) times, the trajectories in $\calK_{\ell}$ provide accurate information about both their transitions $\sfp_{\modst}(\cdot \mid x,a,h)$, and the rewards $r_{\modst}\xah$.
\item Let $\ledrawl$ denote the raw ledger consisting of all trajectories on hallucination episodes: $\ledrawl = (\traj_{k} : k \in \calK_{\ell})$. Define its totally-censored version $\ledcensl := \cens(\ledrawl)$.
\item At each episode, the agent reveals a revealed ledger $\ledrevk$. On non-hallucination episodes in phase $\ell$, $k \in \Phase_{\ell} : k \ne \kexpl$, we reveal the \emph{honest ledger}  $\ledhonl := \cens(\ledrawl ; \calU_{\ell})$, which censors the true raw ledger obtained by omitting rewards from the under-visited states $\calU_{\ell}$.
\item On hallucination episodes $k = \kexpl$, we reveal a \emph{hallucinated ledger}  $\ledhall$, whose construction is defined below.   Note that each agent does not recieve explicit confirmation that they are in a hallucination ($k = \kexpl$) or non-hallucination episode, so prior to seeing the revealed ledger, each agent believes there is an only $1/\nphase$ chance of being in a hallucination episode.  \footnote{Though, through knowledge of the principals mechanism, they \emph{can} deduce the phase number $\ell$.}
\item The hallucinated ledger $\ledhall$ is constructed in two parts. Recall from \Cref{eq:modclass_hall} $\modclasshall(\epspunish)$  denote the set of all models whose rewards are smaller than $\epspunish$ for all $\xah$ in the phase-$\ell$ censoring set $\calU_{\ell}$. We hallucinate a model from the canonical posterior given the totally-censored ledger $\ledcensl$, and conditioned on the event $\modvarhat \in \modclasshall$ lies in this punished set:
\begin{align}
\modhall \sim \Prcan[\modvarhat \in \cdot \mid \ledcensl , \modclasshall(\epspunish) ]. \label{eq:modhalll}
\end{align}
\item We conclude the contruction of $\ledhall$ by hallucinating reward for all $\xah$ triples on past  hallucination episodes $k \in \calK_{\ell}$ using the model $\modhall$ (\Cref{eq:hallucinate_rewards}), and append these rewards to form trajectories (\Cref{eq:hallucinate_trajectories}), and uses these trajectories to fabricate a hallucianted $\calU_{\ell}$-censored \Cref{eq:hallucinate_ledger}:
\begin{align}
 \tilde{r}_{k;h}^{(\ell)} &\overset{\mathrm{indep.}}{\sim} \mathcal{R}_{\model}(x_{k;h},a_{k;h},h), \quad \forall k \in \calK_{\ell}, h \in {[H]}, \text{ where } \model = \modhall  \label{eq:hallucinate_rewards} \\
 \tilde{\traj}^{(\ell)}_{k} &= (x_{k;h},a_{k;h},h, \tilde{r}_{k;h}^{(\ell)})_{h=1}^H  \label{eq:hallucinate_trajectories} \\
\ledhall &= \cens((\tilde{\traj}^{(\ell)}_{k} , \pi_{k})_{k \in \calK_{\ell}};\calU_{\ell}). \label{eq:hallucinate_ledger}
\end{align}
Above, recall that $\mathcal{R}_{\model}$ is the reward distribution under the model $\model$.
\end{itemize}




% AS: one way or another, we need to say the below para in the body

%In reality, (a) the agent actually does know that $\ledhall$ is fabricated, but (b) does not know whether a given episode $k$ is a hallucination episode. Hence, the effectiveness of the mechanism requires showing that the agent will essentially \emph{believe} $\ledhall$ as the true ledger when they are shown it, because, if $\nphase$ is large, than hallucination-episodes are considerably more rare than their alternatives. This is the key insight behind the Hidden Exploration mechanism (\HE) due to \cite{mansour2020bayesian}.
