%!TEX root = main.tex

% macro for thm statements
\newcommand{\OPT}{\mathtt{OPT}} % generic for notions of opt
\newcommand{\REW}{\mathtt{REW}} % total realized reward
\newcommand{\REG}{\ttbf{regret}}

% macro for general results
\newcommand{\qpunish}{q_{\mathrm{pun}}}
\newcommand{\ralt}{r_{\mathrm{alt}}}
\newcommand{\Expsimhh}{\Expop_{\ledhall \sim \mathrm{hh}}}
\newcommand{\Prsimhh}{\Prop_{\ledhall \sim \mathrm{hh}}}

\newcommand{\Dhall}{\mathcal{D}_{\mathrm{hal};\ell}}
\newcommand{\Dhalnol}{\mathcal{D}_{\mathrm{hal}}}
\newcommand{\ledcens}{\ledger_{\mathrm{cens}}}
\newcommand{\Utot}{\mathcal{U}_{\mathrm{all}}}
\newcommand{\modpunish}{\mathcal{M}_{\mathrm{pun}}}
%%%%%%%%%%%%

% BEGIN: macro for algorithm and results
\newcommand{\reach}{\mathsf{Reach}}
%\newcommand{\mdphh}{\texttt{MDP}\text{-}\texttt{HH}}
\newcommand{\mdphh}{{\normalfont \texttt{HiddenHallucination}}\xspace} % our algorithm
\newcommand{\HHandExploit}{{\normalfont \texttt{HH\&Exploit}}\xspace} % main algo followed by exploitation
\newcommand{\reachability}{{\normalfont \texttt{REACHABILITY}}\xspace} % reachability objective

\newcommand{\gapcan}{\gap_{\mathrm{can}}}
\newcommand{\Expcan}{\Exp_{\mathrm{can}}}
\newcommand{\ledhat}{\hat{\ledger}}
\newcommand{\Piseq}{\vec{\Pi}}

\newcommand{\ledvar}{\bm{\ledger}}
\newcommand{\Ledraw}{\Led_{\mathrm{raw}}}
\newcommand{\nullr}{\circ} %expectation

\newcommand{\ledrawl}[1][\ell]{\ledger_{\mathrm{raw};#1}}
\newcommand{\ledcensl}[1][\ell]{\ledger_{\mathrm{cens};#1}}
\newcommand{\ledul}[1][\ell]{\ledger_{\mathcal{U}_{\ell};\ell}}
\newcommand{\modhall}{\model_{\mathrm{hal};\ell}}
\newcommand{\ledrevk}[1][k]{\ledger_{#1}}

\newcommand{\ledhonl}{\ledger_{\mathrm{hon};\ell}}

\newcommand{\ledcenslhat}[1][\ell]{\hat{\bm{\ledger}}_{\mathrm{cens};#1}}
\newcommand{\ledhonlhat}[1][\ell]{\hat{\bm{\ledger}}_{\mathrm{hon};#1}}
\newcommand{\ledhallhat}[1][\ell]{\hat{\bm{\ledger}}_{\mathrm{hal};#1}}

%%%%%%%%%%%%%


%%%%% BEGIN: Alex general macro

%OneLiners
\newcounter{myLISTctr}
\newcommand{\initOneLiners}{%
         \setlength{\itemsep}{0pt}
        \setlength{\parsep }{0pt}
          \setlength{\topsep }{0pt}
%      \usecounter{myLISTctr}
}
\newenvironment{OneLiners}[1][\ensuremath{\bullet}]
    {\begin{list}
        {#1}
        {\initOneLiners}}
    {\end{list}}

\newcommand{\EqComment}[1]{\text{\emph{(#1)}}}

%%% grammar
\newcommand{\ie}{{\em i.e.,~\xspace}}
\newcommand{\Ie}{{\em I.e.,~\xspace}}
\newcommand{\eg}{{\em e.g.,~\xspace}}
\newcommand{\Eg}{{\em E.g.,~\xspace}}

% brackets
\newcommand{\rbr}[1]{\left(\,#1\,\right)}
\newcommand{\sbr}[1]{\left[\,#1\,\right]}
\newcommand{\cbr}[1]{\left\{\,#1\,\right\}}

% math
\newcommand{\LDOTS}{\, ,\ \ldots\ ,}     % smart "..."
\newcommand{\poly}{\operatornamewithlimits{poly}}

%%% END: Alex macro

\numberwithin{equation}{section}

\newcounter{protocol}

\makeatletter
\newenvironment{protocolAlg}[1][htb]{%
  \let\c@algorithm\c@protocol
   \refstepcounter{protocol}
  \renewcommand{\ALG@name}{Protocol}% Update algorithm name
  \begin{algorithm}[#1]%
  }{\end{algorithm}
}

\makeatletter
\newenvironment{constructionAlg}[1][htb]{%
    \renewcommand{\ALG@name}{Construction}% Update algorithm name
   \begin{algorithm}[#1]%
  }{\end{algorithm}}
\makeatother


\newcommand{\polylog}{\textrm{polylog}}
\newcommand{\gapcomplexity}{\mathtt{GapComplexity}}
\DeclareMathAlphabet\mathbfcal{OMS}{cmsy}{b}{n}

\Crefname{claim}{Claim}{Claims}
\Crefname{equation}{Eq.}{Eqs.}
\Crefname{protocol}{Protocol}{Protocols}
\Crefname{asm}{Assumption}{Assumptions}

\Crefname{condition}{Condition}{Conditions}

\newcommand{\calA}{\mathcal{A}}
\newcommand{\calC}{\mathcal{C}}
\newcommand{\calX}{\mathcal{X}}


\newcommand{\nipsvspace}[1]{\iftoggle{nips}{\vspace{1}}{}}
\newcommand{\onespace}{\vspace{.1in}}



%comments

%ceiling

%theorems



%probability
\renewcommand{\Pr}{\mathbb{P}}
\newcommand{\Exp}{\mathbb{E}}
\newcommand{\Var}{\mathrm{Var}}
\newcommand{\Cov}{\mathrm{Cov}}

\newcommand{\Law}{\mathrm{Law}}
\newcommand{\Leb}{\mathrm{Lebesgue}}

\newcommand{\unifsim}{\overset{\mathrm{unif}}{\sim}}
\newcommand{\iidsim}{\overset{\mathrm{i.i.d.}}{\sim}}
\newcommand{\probto}{\overset{\mathrm{prob}}{\to}}
\newcommand{\ltwoto}{\overset{L_2}{\to}}

\newcommand{\info}{\mathbf{i}}
\newcommand{\KL}{\mathrm{KL}}
\newcommand{\Ent}{\mathrm{Ent}}
\newcommand{\TV}{\mathrm{TV}}
\newcommand{\rmd}{\mathrm{d}}

%math operators
\newcommand{\median}{\mathrm{Median}}
\newcommand{\range}{\mathrm{range}}
\newcommand{\im}{\mathrm{im }}
\newcommand{\tr}{\mathrm{tr}}
\newcommand{\rank}{\mathrm{rank}}
\newcommand{\spec}{\mathrm{spec}}
\newcommand{\diag}{\mathrm{diag}}
\newcommand{\Diag}{\mathrm{Diag}}
\newcommand{\sign}{\mathrm{sign\ }}


%norms
\newcommand{\op}{\mathrm{op}}
\newcommand{\F}{\mathrm{F}}
\newcommand{\dist}{\mathrm{dist}}

% logical
\renewcommand{\implies}{\text{ implies }}
\renewcommand{\iff}{\text{ iff }}




\renewcommand{\Im}{\mathfrak{Im }}
\renewcommand{\Re}{\mathfrak{Re }}

\newcommand{\Z}{\mathbb{Z}}

\newcommand{\I}{\mathbf{1}}
\newcommand{\Q}{\mathbb{Q}}

%Vector Spaces
\newcommand{\PD}{\mathbb{S}_{++}^d}
\newcommand{\sphered}{\mathcal{S}^{d-1}}




% Asymptotics

\newcommand{\BigThetaTil}[1]{\widetilde{\BigTm}\left({#1}\right)}
\newcommand{\BigOmega}[1]{\BigWm\left({#1}\right)}

\newcommand{\calF}{\mathcal{F}}
\newcommand{\eventadm}{\calE^{\mathrm{adm}}}



\newcommand{\algcomment}[1]{\textcolor{blue!70!black}{\footnotesize{\texttt{\textbf{#1}}}}}

\newcommand{\getModel}{\mathsf{getModel}}
\newcommand{\getBonus}{\mathsf{getBonus}}
\newcommand{\getModelBonus}{\mathsf{getModelAndBonus}}
\newcommand{\datacorrupt}{\mathcal{T}^{\,\mathrm{corrupt}}}
\newcommand{\Vsthpl}{V^{\star}_{h+1}}
\newcommand{\calS}{\mathcal{S}}
\newcommand{\supp}{\mathrm{supp}}
\newcommand{\xah}{(x,a,h)}
\newcommand{\rtil}{\tilde{r}}
\newcommand{\muhat}{\widehat{\mu}}

\newcommand{\ttbf}[1]{{\normalfont \texttt{\textbf{#1}}}}
\newcommand{\boldsf}[1]{\boldsymbol{\mathsf{#1}}}
\newcommand{\opfont}[1]{\ttbf{#1}}
\newcommand{\varfont}[1]{\mathbf{#1}}
%\newcommand{\bluepar}[1]{\paragraph{\textcolor{blue!70!black}{#1}}}
\newcommand{\bluepar}[1]{\vspace{1mm}\noindent\textbf{\textcolor{blue!70!black}{#1}}}
\newcommand{\Proj}{\ttbf{proj}}
\newcommand{\gapcann}{\gap_{\mathrm{cann}}}
\newcommand{\sigvar}{\bm{\upsigma}}
\newcommand{\modvar}{\bm{\upmu}}
\newcommand{\modvarst}{\modvar_{\star}}
\newcommand{\ledgvar}{\bm{\lambda}}
\newcommand{\width}{\mathrm{width}}
\newcommand{\ledgervar}{\bm{\ledger}}
\newcommand{\EvPun}{\calE_{\mathrm{pun},\,\ell}}

\newcommand{\gap}{\mathrm{Gap}}
\newcommand{\valuename}{\ttbf{value}}
\newcommand{\scrE}{\texttt{E}}

\newcommand{\nlearn}{n_{\mathrm{lrn}}}
\newcommand{\nphase}{n_{\mathrm{ph}}} % phase length


\newcommand{\Pinew}{\Pi_{\mathrm{new}}}
\newcommand{\prior}{\ttbf{p}}
\newcommand{\posteriorhall}{\prior_{\mathrm{hal},\ell}} % hallucinated posterior


\newcommand{\orac}{\opfont{orac}}
\newcommand{\spn}{\opfont{span}}
\newcommand{\spnc}{\opfont{span}^c}
\newcommand{\envir}{\opfont{envir}}



\newcommand{\Led}{\opfont{led}}
\newcommand{\cens}{\opfont{cens}}
\newcommand{\Phase}{\opfont{phase}}



\newcommand{\modclass}{\mathcal{M}}
\newcommand{\modtotal}{\modclass_{\mathrm{mdp}}}
\newcommand{\modelst}{\model_{\star}}
\newcommand{\calW}{\mathcal{W}}


\newcommand{\sigspace}{\mathcal{S}}
\newcommand{\sigtot}{\widebar{\mathcal{S}}}

\newcommand{\kernel}{\mathfrak{q}}
\newcommand{\valuef}[2]{\valuename(#1;#2)}

\newcommand{\valuemid}[2]{\valuename[#1\mid#2]}
\newcommand{\valuemidhh}[2]{\valuename_{\mathrm{hh}}(#1\mid#2)}

\newcommand{\valuecan}[2]{\valuename_{\mathrm{can}}\sbr{#1\mid#2}} % canonical value


\newcommand{\poltotal}{\Pi_{\mathrm{total}}}
\newcommand{\Mbayes}{\check{\model}}
\newcommand{\signalbayes}{\check{\signal}}
\newcommand{\calMst}{\model_{\star}}
\newcommand{\Vpost}[1][\pi]{V^{#1,\mathrm{post}}}
\newcommand{\Pitotal}{\Pi_{\mathrm{mkv}}}
\newcommand{\Piold}{\Pi^{\mathrm{old}}}
\newcommand{\diffV}{\Delta}

\newcommand{\modst}{\model_{\star}}
\newcommand{\sfE}{\mathsf{E}}
\newcommand{\sfEbayes}{\check{\sfE}}
\newcommand{\signalbase}{\signal_{\mathrm{base}}}
\newcommand{\frakD}{\mathfrak{D}}
\newcommand{\basespace}{\sigspace_{\mathrm{base}}}
\newcommand{\Ehh}{\mathbb{E}_{\mathrm{hh}}}
\newcommand{\Mimp}{\model_{\mathrm{imp}}}
\newcommand{\Mhh}{\model_{\mathrm{hh}}}
\newcommand{\calE}{\mathcal{E}}


\newcommand{\calK}{\mathcal{K}}



\newcommand{\trajspace}{\mathcal{T}}
\newcommand{\trajspacetot}{\trajspace^{\cens}}



\newcommand{\priorl}{\ttbf{p}}
\newcommand{\kertrustl}{\kernel_{\mathrm{trust},\ell}}
\newcommand{\priormidexp}{\prior_{\mid \mathrm{exp}}}
\newcommand{\distequals}{\overset{\mathrm{dist}}{=}}
\newcommand{\kerexptotot}{\kernel_{\mathrm{exp}\to \mathrm{tot}}}
\newcommand{\indsim}{\overset{\mathrm{indep.}}{\sim}}

  \newcommand{\modelhall}{\model_{\mathrm{hal},\ell}}
  \newcommand{\modclasshhl}{\modclass_{\mathrm{hh},\ell}}

\newcommand{\signalbayesbase}{\signalbayes_{\mathrm{base}}}
\newcommand{\Qhh}{Q_{\mathrm{\Mhh}}}
\newcommand{\modelhal}{\model_{\mathrm{hal}}}

\newcommand{\kerhhl}{\kernel_{\mathrm{hh},\ell}}

\newcommand{\Pioldl}{\Pi_{\mathrm{old},\ell}}

\newcommand{\ledhall}[1][\ell]{\ledger_{\mathrm{hal};#1}}

\newcommand{\Expop}{\operatornamewithlimits{\ensuremath{\mathbb{E}}}}   % expectation
\newcommand{\Prop}{\operatornamewithlimits{\ensuremath{\mathbb{P}}}}    % probability
\newcommand{\minop}{\operatornamewithlimits{\ensuremath{\min}}}         % min
\newcommand{\infop}{\operatornamewithlimits{\ensuremath{\inf}}}         % inf
\newcommand{\supop}{\operatornamewithlimits{\ensuremath{\sup}}}         % sup



\newcommand{\Prhh}{\Pr_{\mathrm{hh}}}
\newcommand{\Exphh}{\Exp_{\mathrm{hh}}}

\begin{comment}
\subsection{Hidden hallucination via Projected Data: Single Round}

\begin{enumerate}
    \item My agent has a prior $\prior_0$ over models $\model \in \modclass$.
    \item Principal recieves trajectories $(\trajtot_{\pi},\pi)_{\pi \in \Pi}$ over a finite class $\Pi$.
    \item The principle conducts a ``projection'' $(\trajpro_{\pi},\pi)_{\pi \in \Pi} =(\projexp(\trajtot_{\pi}),\pi)_{\pi \in \Pi}$
    \item Explain
\end{enumerate}
\end{comment}



\newcommand{\trajbar}{\overline{\traj}}
\newcommand{\trajrealbar}{\overline{\trajreal}}
\newcommand{\trajrealpro}{\trajreal^{\mathrm{pro}}}
\newcommand{\trajreal}{\boldsf{T}}
\newcommand{\trajspacebar}{\overline{\trajspace}}
\newcommand{\seedreal}{\boldsf{seed}}
  \newcommand{\pireal}{\boldsf{\pi}}
  \newcommand{\kertraj}{\kernel_{\mathrm{traj}}}


  \newcommand{\histvarraw}{\histvar^{\mathrm{raw}}}
  \newcommand{\histvar}{\boldsf{H}}
  \newcommand{\histspace}{\mathcal{H}}
  \newcommand{\histspaceraw}{\histspace^{\mathrm{raw}}}
  \newcommand{\histspacecens}{\histspace^{\mathrm{cens}}}
  \newcommand{\histvarcens}{\histvar^{\mathrm{cens}}}

\newcommand{\Hraw}{\mathsf{H}^{\mathrm{raw}}}
\newcommand{\Hcens}{\mathsf{H}^{\mathrm{cens}}}
\newcommand{\Hhal}{\mathsf{H}^{\mathrm{hal}}}

\newcommand{\pihat}{\widehat{\pi}}
\newcommand{\pist}{\pi_{\star}}
\newcommand{\calD}{\mathcal{D}}
\newcommand{\trajspacecens}{\trajspace_{\censicon}}
\newcommand{\laws}{\mathrm{laws}}

\newcommand{\Wscouts}[1][\calW]{\triangleright_{#1}}
\newcommand{\Escouts}[1][\calE]{\triangleright_{#1}}



  \newcommand{\Prcan}{\Pr_{\mathrm{can}}}
  \newcommand{\Kreal}{\mathbfcal{K}}
  \newcommand{\sfH}{\mathsf{H}}
  \newcommand{\hist}{\mathsf{H}}
  \newcommand{\ledgespace}{\mathcal{L}}
  \newcommand{\ledgespacetot}{\widebar{\ledgespace}}

  \newcommand{\pols}[1]{\Pi_{#1}}


\newcommand{\kind}{\mathrm{kind}}
\newcommand{\rawkind}{\mathrm{raw}}
\newcommand{\censkind}{\mathrm{cens}}
\newcommand{\histvarkind}{\histvar^\kind}
\newcommand{\histspacekind}{\histspace^\kind}

\newcommand{\getModClass}{\mathsf{getModClass}}
\newcommand{\modclasshall}[1][\ell]{\modclass_{#1}}
\newcommand{\kexpl}[1][\ell]{k^{\mathrm{hal}}_{#1}}
\newcommand{\Kexp}[1][\ell]{\calK^{\mathrm{hal}}_{#1}}

\newcommand{\histvarhal}{\histvar^{\mathrm{hal}}}
\newcommand{\Ledhal}{\ledger^{\mathrm{hal}}}
\newcommand{\metaHH}{\textsf{Meta}\text{-}\textsf{HH}}

\newcommand{\censicon}{\mathrm{cs}}
\newcommand{\trajcens}{\traj_{\censicon}}
\newcommand{\trajcenssub}[1][\pi]{\traj_{\censicon;#1}}
\newcommand{\trajvarcenssub}[1][\pi]{\bm{\traj}_{\censicon;#1}}
\newcommand{\ledgevarcens}{\ledgervar_{\censicon}}

\newcommand{\ledgercens}{\ledger_{\censicon}}
\newcommand{\Ledcens}{\Led_{\censicon}}
\newcommand{\ledgespacecens}{\ledgespace_{\censicon}}
\newcommand{\ledgespacecenstot}{\ledgespacetot_{\censicon}}
\newcommand{\ledgespacecenstotm}{\ledgespacetot_{\censicon;m}}

\newcommand{\Ledcenssub}[1][\calK]{\lambda_{\censicon}[#1]}
\newcommand{\Ledsub}[1][\calK]{\ledger[#1]}

\newcommand{\ledgespaceN}{\ledgespace^{\N}}



\newcommand{\uncens}{\cens^{-1}}
\newcommand{\ledgecens}{\ledgercens}
\newcommand{\trajspacecenstot}{\widebar{\trajspace}_{\censicon}}

\newcommand{\pst}{p_{\star}}
\newcommand{\varp}{\mathbf{p}}
\newcommand{\varphall}{\varp_{\mathrm{hall};\ell}}
\newcommand{\epsp}{\varepsilon_{\mathsf{p}}}
\newcommand{\epsr}{\varepsilon_{r}}

\newcommand{\clearnp}{c_{\mathsf{p}}}
\newcommand{\clearnr}{c_{r}}
\newcommand{\ellearn}{\ell_{\mathrm{lrn}}}
\newcommand{\Klearn}{\mathcal{K}_{\mathrm{lrn}}}

\newcommand{\bmx}{\mathbf{x}}
\newcommand{\bmr}{\mathbf{r}}

\newcommand{\bma}{\mathbf{a}}

\newcommand{\pmodst}{\sfp_{\modst}}
\newcommand{\rmodst}{r_{\modst}}
\newcommand{\lonenorm}[1]{\|#1\|_{\ell_1}}
\newcommand{\calN}{\mathcal{N}}

\newcommand{\sfP}{\mathsf{P}}

\newcommand{\br}{\mathbf{r}}
\newcommand{\bo}{\mathbf{o}}
\newcommand{\brhat}{\hat{\mathbf{r}}}
\newcommand{\btheta}{\bm{\theta}}
\newcommand{\modvarhat}{\hat{\model}}

\newcommand{\sfp}{\mathsf{p}}
\newcommand{\ptilsf}{\widetilde{\sfp}}
\newcommand{\sigvarhat}{\hat{\sigvar}}

\newcommand{\Mtil}{\widetilde{M}}
\newcommand{\thetar}{\theta_r}
\newcommand{\thetap}{\theta_{\sfp}}
\newcommand{\epspunish}{\varepsilon_{\mathrm{pun}}}


\newcommand{\Khal}{\calK^{\mathrm{hal}}}
\newcommand{\signalhal}{\signal^{\mathrm{hal}}}
\newcommand{\khal}{k^{\mathrm{hal}}}
\newcommand{\Ledm}{\Led_m}
\newcommand{\Ledcensm}{\Led_{\censicon;m}}
\newcommand{\ledgespacetotm}{ \ledgespacetot_m}
\newcommand{\trajvar}{\bm{\traj}}
\newcommand{\Pimarkov}{\Pi_{\mathrm{mkv}}}

\newcommand{\censorac}{\orac_{\censicon}}
\newcommand{\subsets}{\mathrm{subsets}}
\newcommand{\censinv}{\cens^{-1}}
\newcommand{\calU}{\mathcal{U}}
\newcommand{\sfR}{\mathsf{R}}
\newcommand{\kh}{_{k;h}}
\newcommand{\nn}{\nonumber}
\newcommand{\Expmech}{\mathbb{E}_{\mathrm{mech}}}
\newcommand{\Prmech}{\mathbb{P}_{\mathrm{mech}}}
\newcommand{\sigreal}{\hat{\bm{\signal}}}

%\newcommand{\signaltrust}{\signal_{\mathrm{trust}}}
%\newcommand{\signalhh}{\signal_{\mathrm{hh}}}
%\newcommand{\impute}{\mathrm{imp}}
%\newcommand{\frakDtrust}{\sigspace_{\mathrm{trust}}}
%\newcommand{\signalimp}{\signal_{\mathrm{imp}}}
%\newcommand{\impspace}{\sigspace_{\mathrm{imp}}}
%\newcommand{\kertrust}{\kernel_{\mathrm{trust}}}
%\newcommand{\kerbase}{\kernel_{\mathrm{base}}}
%\newcommand{\kerhh}{\kernel_{\mathrm{hh}}}
%\newcommand{\sigvartrust}{\sigvar_{\mathrm{\trusttext}}}
%\newcommand{\sigvarhal}{\sigvar_{\mathrm{hal}}}
%\newcommand{\sigvarhh}{\sigvar_{\mathrm{hh}}}
%\newcommand{\hhclass}{\modclass_{\mathrm{hh}}}

\newcommand{\HE}{\texttt{HE}}
\newcommand{\HH}{\texttt{HE}}
\newcommand{\RLE}{\texttt{RL}\text{-}\texttt{E}}
\newcommand{\IE}{\texttt{IE}}
\newcommand{\Ecubed}{\texttt{E3}}

% the letter for ...
\newcommand{\ledger}{\uplambda} % ... ledgers
\newcommand{\signal}{\upsigma}  % ... signals
\newcommand{\traj}{{\uptau}}    % ... trajectories
\newcommand{\model}{\upmu}      % ... models
\newcommand{\signals}{\Upsigma} % set of all possible signals

\newcommand{\underexplored}[1][\ell]{\calU^{\mathrm{und}}_{#1}} % set of under-explored triples
%\newcommand{\explored}[1][\ell]{\calU^{\mathrm{full}}_{#1}} % set of fully-explored triples
