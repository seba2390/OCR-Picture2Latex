\documentclass[Colective.tex]{revtex4-1}

%

\begin{document}

\setcounter{affil}{0}
\renewcommand{\thefigure}{S\arabic{figure}}
\setcounter{figure}{0}
\renewcommand{\theequation}{S\arabic{equation}}
\setcounter{equation}{0}
\renewcommand\thesection{S\arabic{section}}
\setcounter{section}{0}

\title{Supplemental material for ``Characterization of collective excitations in weakly-coupled disordered superconductors"}
\label{SI}

%\author{Bo Fan}
%\email{bo.fan@sjtu.edu.cn}
%\affiliation{Shanghai Center for Complex Physics, 
%	School of Physics and Astronomy, Shanghai Jiao Tong
%	University, Shanghai 200240, China}
%\author{Abhisek Samanta}\email{abhiseks@campus.technion.ac.il}
%\affiliation{ Physics Department, Technion, Haifa 32000, Israel}
%\author{Antonio M. Garc\'ia-Garc\'ia}
%\email{amgg@sjtu.edu.cn}
%\affiliation{Shanghai Center for Complex Physics, 
%	School of Physics and Astronomy, Shanghai Jiao Tong
%	University, Shanghai 200240, China}

\begin{abstract}
\noindent We present the technical details involving the calculation of the full current-current correlator. Next we study the behavior of the collective modes in the clean limit, as well as in presence of disorder. Finally we discuss the amplitude and phase fluctuation correlation functions and the optical conductivity.
\end{abstract}
\maketitle
%%
\onecolumngrid
%%
\section{I.\,\,\,\,\,\,     Current-current correlator in disordered superconductors}
%%
We consider a disordered attractive Hubbard model on a square lattice in presence of onsite random potential $V_i$ ($V_i\in [-V,V]$). The model Hamiltonian is given by,
%%
\begin{equation}
H = -t\sum_{\langle ij\rangle\sigma} c^\dagger_{i\sigma} c_{j\sigma} - U\sum_i n_{i\uparrow}n_{i\downarrow} + \sum_i V_i n_i
\end{equation} 
%%
where $t$ is the hopping amplitude between two nearest neighbors, and $U$ is the attractive interaction responsible for the Cooper pairing. We invoke an inhomogeneous mean-field theory (Bogoliubov de-Gennes theory) with two mean-field parameters: local superconducting order parameter $\Delta(r_i)$ and local density $n(r_i)$~\cite{Ghosal2001, ghosal1998}.  We then use the following Bogoliubov transformation,
%%
\begin{equation}
c_{i\sigma} = \sum_n \lt(u_n(i)\gamma_{n\sigma} - \sigma v^*_n(i)\gamma^\dagger_{n\bar\sigma}\rt)
\end{equation}
%%
which diagonalizes the effective mean-field Hamiltonianin in fermionic quasi-particle basis ($\gamma$),
%%
\begin{equation}
H_{MF} = \sum_{n\sigma} E_n \gamma^\dagger_{n\sigma}\gamma_{n\sigma},
\end{equation}
%%
where $n$ runs over the positive eigenvalues i.e. $E_n>0$.

Next we study the effect of disorder on the optical response of the system. The dynamical correlation function is defind as \cite{cea2014}
\begin{equation}
\chi_{ij}(\phi,\phi')=-i \int dt e^{i\omega t} \langle \left[\phi_i(t),\ \phi'_j(0)\right] \rangle
\label{eq.cor_fun}
\end{equation} 
%%
where $\phi$ corresponds to the fluctuation components and the current operators, which are given by \cite{cea2014}
\begin{equation}
\begin{aligned}
&\delta\Delta_i &=\ &c_{i\downarrow}c_{i\uparrow} - \langle c_{i\downarrow}c_{i\uparrow} \rangle \\
&\delta\Delta_i^\dagger &=\ &c_{i\uparrow}^\dagger c_{i\downarrow}^\dagger - \langle c_{i\uparrow}^\dagger c_{i\downarrow}^\dagger \rangle \\
&\delta n_i &=\ &\sum_{\sigma} \left( c_{i\sigma}^\dagger c_{i\sigma} - \langle c_{i\sigma}^\dagger c_{i\sigma} \rangle \right) \\
&j_i^\alpha &=\ &it \sum_{\sigma} \left( c_{i+\alpha,\sigma}^\dagger c_{i\sigma} - c_{i\sigma}^\dagger c_{i+\alpha,\sigma} \right). \\
\end{aligned}
\label{eq.fluct_curr}
\end{equation} 
%%
Here $\delta\Delta_i$ is the fluctuation in local superconducting order parameter, and $\delta n_i$ is the fluctuation in local density. $\langle\cdots\rangle$ corresponds to the expectation value of the operator in the inhomogeneous BdG eigenstate. The amplitude fluctuation $A_i$ and the phase fluctuation $\Phi_i$ of the superconducting order parameter are given by 
\begin{align}
A_i=(\delta\Delta_i+\delta\Delta_i^\dagger)/\sqrt{2} \label{eq.amp}\\
\Phi_i=i(\delta\Delta_i-\delta\Delta_i^\dagger)/\sqrt{2}
\label{eq.pha}
\end{align}
%
Note that all the eigenvectors $(u_n, v_n)$ in our case are real, and hence we can express the dynamical correlation functions in terms of $u_n, v_n$ and $E_n$ only. Now we present the detailed formulae for different correlation functions at zero temperature.

%%
The bare current-current correlation function is given by~\cite{Ghosal2001},
\begin{eqnarray}
\nonumber \chi^0_{ij}(j^x,j^x) 
&= -2t^2\sum_{nm}\left(v_{m}(j+\hat x)u_{n}(j) + v_{n}(j+\hat x)u_{m}(j)\right) \left({\frac{u_n(i+\hat x)v_m(i) - u_n(i)v_m(i+\hat x)}{i\omega_p+E_n+E_m} } - {\frac{v_n(i)u_m(i+\hat x) - v_n(i+\hat x)u_{m}(i)}{i\omega_p-E_n-E_m} }\right) \no\\ 
&~~~~~~~~~~~~~~~~~~~~~~~~~~~~~~~~~~~~~~~~~~~~~~~~~~~~~~~~~~~
+u\leftrightarrow v
\label{eq.bare_jj}
\end{eqnarray}
%%
where $i\omega_p$ is the Bosonic Matsubara frequency, given by $\omega_p=2\pi p/\beta$. \\

The vertex correction to the bare current-current correlation function can be calculated by introducing the following correlations. The correlation functions between the current operator and the pair fluctuations ($\delta\Delta^\dagger,\delta\Delta$), or the charge density fluctuations ($\delta n$) are given by,
%%
\begin{eqnarray}
\chi_{ij}(j^x,\delta\Delta) &=& 2it \sum_{nm} 
{\left(u_n(i+\hat x)v_m(i) - u_n(i)v_m(i+\hat x)\right)} \left( {u_{m}(j)u_{n}(j) \over i\omega_p +E_n+E_m} - {v_{m}(j)v_{n}(j) \over i\omega_p -E_n-E_m} \right) \\
%%
\chi_{ij}(j^x,\delta\Delta^\dagger) &=& -2it \sum_{nm} {\left(u_n(i+\hat x)v_m(i)-u_n(i)v_m(i+\hat x)\right)} \left( { v_{m}(j)v_{n}(j) \over i\omega_p +E_n+E_m} - {u_{m}(j)u_{n}(j) \over i\omega_p -E_n-E_m} \right) \\
%%
\chi_{ij}(j^x,\delta n) &=&  2it \sum_{nm} 
-{\left(u_n(i+\hat x)v_m(i)-u_n(i)v_m(i+\hat x)\right) \left(v_{m}(j)u_{n}(j) + v_{n}(j)u_{m}(j)\right) \over i\omega_p +E_n+E_m} \nonumber\\
&& ~~~~~~~~~~~~~~~~~~~~~~~~ + {\left(v_n(i+\hat x)u_m(i)-v_n(i)u_m(i+\hat x)\right) \left(u_{m}(j)v_{n}(j)+u_{n}(j)v_{m}(j)\right)  \over i\omega_p-E_n-E_m}
\end{eqnarray}
%%

The bare correlation functions between pair fluctuations ($\delta\Delta^\dagger,\delta\Delta$) and charge density fluctuations ($\delta n$) are given by,
\begin{align}
%%
\chi_{ij}(\delta\Delta,\delta\Delta) &= \sum_{nm} -{u_n(i)u_m(i) v_m(j)v_n(j) \over i\omega_p-E_n-E_m}
+ {v_n(i)v_m(i) u_m(j)u_n(j) \over i\omega_p+E_n+E_m} \\
%%
\chi_{ij}(\delta\Delta,\delta\Delta^\dagger) &= \sum_{nm} {u_n(i)u_m(i) u_m(j)u_n(j) \over i\omega_p-E_n-E_m}
- {v_n(i)v_m(i) v_m(j)v_n(j) \over i\omega_p+E_n+E_m} \\
%%
\chi_{ij}(\delta\Delta^\dagger,\delta\Delta) &= \sum_{nm} {v_n(i)v_m(i) v_m(j)v_n(j) \over i\omega_p-E_n-E_m}
- {u_n(i)u_m(i) u_m(j)u_n(j) \over i\omega_p+E_n+E_m}\\
%%
\chi_{ij}(\delta\Delta,\delta n) &= -2 \sum_{nm}  {u_n(i)u_m(i)u_{m}(j)v_{n}(j) \over i\omega_p-E_n-E_m}
+ {v_n(i)v_m(i) v_{m}(j)u_{n}(j) \over i\omega_p+E_n+E_m} \\
%%
\chi_{ij}(\delta n,\delta \Delta) &= 2\sum_{nm} {u_n(i)v_m(i) u_{m}(j)u_{n}(j) \over i\omega_p+E_n+E_m}
+ {v_n(i)u_m(i) v_{m}(j)v_{n}(j) \over i\omega_p-E_n-E_m} \\
%%
\chi_{ij}(\delta n,\delta n) &= 2\sum_{nm} 
{ \left(v_{m}(j)u_{n}(j) + v_{n}(j)u_{m}(j)\right)} \left( -{u_n(i)v_m(i)\over i\omega_p+E_n+E_m} +  {v_m(i)u_n(i)\over i\omega_p-E_n-E_m} \right)
\end{align}
% \indent In the clean limit, the quasi-particle excitation in the normal states
% and SC states are $\xi_{\mathbf{k}}=-2t(cos k_x + cos k_y)-\mu$ and $E_{\mathbf{k}}=\sqrt{\xi_{\mathbf{k}}^2+\Delta_0^2}$, respectively, where $\mu$ is the chemical potential and $\Delta_0$ is the order parameter in the clean limits. The wave function is then defined as $u_{\mathbf{k}}^2 =1/2(1+\xi_{\mathbf{k}}/E_{\mathbf{k}}) = 1-v_{\mathbf{k}}^2 $ from the standard BCS theory of clean superconductor. XXXX\\
%%
\indent  The remaining correlation functions can be obtained using symmetry,
\begin{eqnarray}
\chi_{ij}(\Delta,j^x) &= &-\chi_{ji}(j^x,\Delta^\dagger) \no\\
\chi_{ij}(\Delta^\dagger,j^x) &= &-\chi_{ji}(j^x,\Delta) \no\\
\chi_{ij}(n,j^x) &= &-\chi_{ji}(j^x,n) \\
\chi_{ij}(\Delta^\dagger,\Delta^\dagger) &= &\chi_{ji}(\Delta,\Delta) \no\\
\chi_{ij}(\Delta^\dagger,n) &= &\chi_{ji}(n,\Delta) \no\\
\chi_{ij}(n,\Delta^\dagger) &= &\chi_{ji}(\Delta,n) \no
\label{eq.G1}
\end{eqnarray}

With the definitions of amplitude and phase fluctuations given in Eqn.~\eqref{eq.amp} and ~\eqref{eq.pha}, we can write the correlation functions between current and amplitude or phase fluctuations in the following way,
\begin{eqnarray}
\chi_{ij}(j^x,A) &=\frac{1}{\sqrt{2}} \left( \chi_{ij}(j^x,\delta\Delta) + \chi_{ij}(j^x,\delta\Delta^\dagger) \right)\\
\chi_{ij}(A,j^x) &=\frac{1}{\sqrt{2}} \left( \chi_{ij}(\delta\Delta,j^x) + \chi_{ij}(\delta\Delta^\dagger,j^x) \right)\\
\chi_{ij}(j^x,\Phi) &=\frac{i}{\sqrt{2}} \left( \chi_{ij}(j^x,\delta\Delta) - \chi_{ij}(j^x,\delta\Delta^\dagger) \right)\\
\chi_{ij}(\Phi,j^x) &=\frac{i}{\sqrt{2}} \left( \chi_{ij}(\delta\Delta,j^x) - \chi_{ij}(\delta\Delta^\dagger,j^x) \right),
\label{eq.4}
\end{eqnarray} 
and the correlation functions between amplitude fluctuation, phase fluctuation and density fluctuation are 
\begin{eqnarray}
&\chi_{ij}(A,A)&= \frac{1}{2} \left[ \chi_{ij}(\delta \Delta,\delta \Delta) + \chi_{ij}(\delta \Delta,\delta \Delta^\dagger) + \chi_{ij}(\delta \Delta^\dagger,\delta \Delta) + \chi_{ij}(\delta \Delta^\dagger,\delta \Delta^\dagger) \right]\\
&\chi_{ij}(A,\Phi)&= \frac{i}{2} \left[ \chi_{ij}(\delta \Delta,\delta \Delta) - \chi_{ij}(\delta \Delta,\delta \Delta^\dagger) + \chi_{ij}(\delta \Delta^\dagger,\delta \Delta) - \chi_{ij}(\delta \Delta^\dagger,\delta \Delta^\dagger) \right]\\
&\chi_{ij}(A,\delta n)&= \frac{1}{\sqrt{2}} \left[ \chi_{ij}(\delta \Delta,\delta n) + \chi_{ij}(\delta \Delta^\dagger,\delta n) \right]\\
&\chi_{ij}(\Phi,A)&= \frac{i}{2} \left[ \chi_{ij}(\delta \Delta,\delta \Delta) + \chi_{ij}(\delta \Delta,\delta \Delta^\dagger) - \chi_{ij}(\delta \Delta^\dagger,\delta \Delta) - \chi_{ij}(\delta \Delta^\dagger,\delta \Delta^\dagger) \right]\\
&\chi_{ij}(\Phi,\Phi)&=\frac{i^2}{2} \left[ \chi_{ij}(\delta \Delta,\delta \Delta) - \chi_{ij}(\delta \Delta,\delta \Delta^\dagger) - \chi_{ij}(\delta \Delta^\dagger,\delta \Delta) + \chi_{ij}(\delta \Delta^\dagger,\delta \Delta^\dagger) \right]\\
&\chi_{ij}(\Phi,\delta n)&= \frac{i}{\sqrt{2}} \left[ \chi_{ij}(\delta \Delta,\delta n) - \chi_{ij}(\delta \Delta^\dagger,\delta n) \right]\\
&\chi_{ij}(\delta n,A)&= \frac{1}{\sqrt{2}} \left[ \chi_{ij}(\delta n,\delta \Delta) + \chi_{ij}(\delta n,\delta \Delta^\dagger) \right]\\
&\chi_{ij}(\delta n,\Phi)&= \frac{i}{\sqrt{2}} \left[ \chi_{ij}(\delta n,\delta \Delta) - \chi_{ij}(\delta n,\delta \Delta^\dagger) \right]
\label{eq.5}
\end{eqnarray} 
%%
\indent The full gauge invariant current-current correlation function (including the vertex corrections) is given by,
\begin{align}
\chi_{ij}\left(j^x,j^x\right) = \chi^0_{ij}\left(j^x,j^x\right) + \Lambda_{ip}\mathbb{V}_{pl} \left(\mathbb{I}_{3N\times 3N}-\chi^B \mathbb{V}\right)^{-1}_{ls}\bar{\Lambda}_{sj}
\label{eq.fullchi_SM}
\end{align}
%%
where $\chi^0$ is the bare current-current correlation function, and $\Lambda$ couples the current with one of the fluctuation components. We note that we have three possible types of fluctuations i.e. $A$, $\Phi$ and $\delta n$ which correspond to amplitude, phase and charge density fluctuations respectively,
\begin{align}
\Lambda&=\left(~\chi(j^x,A)~~\chi(j^x,\Phi)~~\chi(j^x,\delta n)~\right)\\
\bar{\Lambda}&=\left(~\chi(A,j^x)~~\chi(\Phi,j^x)~~\chi(\delta n,j^x)~\right)^T.
\end{align}
$\chi^B$ is the bare mean-field susceptibility and $\mathbb V$ is the effective local interaction, defined by $3\times3$ matrices in the basis of fluctuations:
%%
\begin{equation}
\chi^B = 
\left(\begin{array}{ccc}
\chi^{AA} & \chi^{A\Phi} & \chi^{A\delta n} \\
\chi^{\Phi A} & \chi^{\Phi \Phi} & \chi^{\Phi \delta n} \\
\chi^{\delta n A} & \chi^{\delta n\Phi} & \chi^{\delta n\delta n}
\end{array}\right)
\end{equation}
%%
and 
%%
\begin{equation}
\mathbb{V} = 
\left(\begin{array}{ccc}
-|U| & 0 & 0 \\
0 & -|U| & 0 \\
0 & 0 & -|U|/2
\end{array}\right)
\end{equation}
%%
Therefore, in Eqn.~\eqref{eq.fullchi_SM}, for example, $\chi^B(A,A)=\chi^{AA}$, $\mathbb{V}^A=-|U|\mathbb{I_{N\times N}}$ and so on ... Note that all of them are $N\times N$ matrices in real space.\\
%%
\begin{figure}[H]
	\begin{center}
		\subfigure[Amplitude]{\label{fig.amp_U5}%%
			\includegraphics[width=6.cm]{./density_map/amp_U5_n0p875_L200_bcs.pdf}}
		\subfigure[Phase]{\label{fig.pha_U5}%%
			\includegraphics[width=6.cm]{./density_map/pha_U5_n0p875_L200_bcs.pdf}}\\
		\subfigure[Amplitude]{\label{fig.amp_U2}%%
			\includegraphics[width=6.cm]{./density_map/amp_U2p0_n0p875_L200_bcs.pdf}}
		\subfigure[Phase]{\label{fig.pha_U2}%%
			\includegraphics[width=6.cm]{./density_map/pha_U2p0_n0p875_L200_bcs.pdf}}
		\caption{Amplitude ($P^A(q,\omega)$) and phase ($P^\Phi(q,\omega)$) spectral functions for two different couplings $U$ in the clean limit i.e. $V= 0$, as a function of momenta $q$ (i.e. $\Gamma (0,0)$, $X (\pi,0)$, and $M (\pi,\pi)$ in the momentum space). The figures show the low-energy dispersing collective modes, as well as the two-particle continuum above the spectral gap $\omega_g$ (the red horizontal line). The upper panel is for $U=5$, and the lower panel is for $U=2$. The system size is $200\times200$ and the average density is $\langle n \rangle =0.875$.} \label{Fig.amp_pha_U5_2}
	\end{center}
\end{figure}


\section{II.\,\,\,\,\,\,      Collective modes in clean and disordered cases}
To study the collective modes, we construct the following matrix in real space
%%
\begin{align}
\tilde{\chi}^B = \left(\mathbb{I}_{3N\times 3N}-\chi^B \mathbb{V}\right)^{-1}\chi^B = 
\left(\begin{array}{ccc}
\tilde{\chi}^{AA} & \hat{\chi}^{A\Phi} & \tilde{\chi}^{A\delta n} \\
\tilde{\chi}^{\Phi A} & \tilde{\chi}^{\Phi \Phi} & \tilde{\chi}^{\Phi \delta n} \\
\tilde{\chi}^{\delta n A} & \tilde{\chi}^{\delta n\Phi} & \tilde{\chi}^{\delta n\delta n}
\end{array}\right)
\label{eq.RPA_SM}
\end{align}
%%
to obtain the amplitude spectral function $P_{ij}^A(\omega) = - \frac{1}{\pi}\tilde{\chi}_{ij}^{AA}(\omega) $ and phase spectral function $P_{ij}^\Phi(\omega) = - \frac{1}{\pi}\tilde{\chi}^{\Phi \Phi}_{ij}(\omega) $. Then, we do Fourier transformation to get the $P^A(q,\omega)$ and $P^\Phi(q,\omega)$, to observe the behaviour of amplitude and phase collective modes in momentum space.

%%
In the clean limit, we clearly see the dispersing collective modes in the phase sector, which corresponds to the gapless Goldstone mode, see Fig.~\ref{Fig.amp_pha_U5_2}. However, unlike the Goldstone mode, collective modes in the amplitude sector (the Higgs mode) have a finite gap $\omega_{H}$ \cite{abhisek2020}. In the $q\rightarrow 0$ limit, the gap $\omega_{H}$ is same with the two-particle gap $\omega_g$. In presence of strong coupling, the collective modes are well below the two-particle gap, however as the coupling $|U|$ decreases, the two-particle gap also comes down making the collective modes difficult to observe. The same conclusion also holds as the average density $\langle n\rangle$ decreases.

%In the clean limit we see the gapless Goldstone mode, see Fig.~\ref{Fig.amp_pha_U5_2}. However, unlike the Goldstone mode, collective modes related to the amplitude fluctuations (the Higgs mode) have a finite gap $\omega_{H}$ \cite{abhisek2020}. As the coupling constant $|U|$ decreases, $\omega_H$ moves towards the two-particle spectral gap $\omega_g$. In the weak-coupling and low density regime, $\omega_{H}$ becomes comparable or larger than $\omega_g$ and therefore it is not easily observable. 

In presence of weak disorder, in addition to the dispersing collective modes, a non-dispersive mode appears in the amplitude sector at finite energy below two-particle continuum, which has been identified as the \textit{disorder-induced Higgs mode}~\cite{abhisek2020}, see Fig.~\ref{Fig.amp_pha_U5}. On the other hand, the phase mode remains dispersing, but gets broadened. In the strong coupling limit, these modifications are easier to observe as the collective mode structure remains well below the two-particle continuum. When disorder becomes large, the \textit{disorder-induced Higgs mode} gets broadened in energy and hence gets mixed with the incoherent spectral weight coming down from two-particle continuum, and therefore it can not be separately identified. The Goldstone mode is comparatively robust with disorder. However, in presence of sufficiently strong disorder, the sound velocity (related to the slope of the linearly dispersing Goldstone mode) decreases with disorder, and hence the Goldstone mode also ceases to show up as a sharp mode.  

%In the strong coupling limit, the evolution of the collective modes with disorder is somewhat simpler. As disorder becomes stronger, the collective mode dispersion relation gets broaden and mixed. Besides, when disorder $V=1.5$, the collective modes move to lower energy at $\Gamma$ point and move to higher energy at $M$ point, even above the two-particle spectral gap.
%%
\begin{figure}[H]
	\begin{center}
		\subfigure[Amplitude]{\label{fig.amp_V0p25}%%
			\includegraphics[width=6.cm]{./density_map/amp_full_spec_L20_U5p0_V0p25_n0p875.pdf}}
		\subfigure[Phase]{\label{fig.pha_V0p25}%%
			\includegraphics[width=6.cm]{./density_map/pha_full_spec_L20_U5p0_V0p25_n0p875.pdf}}\\
		\subfigure[Amplitude]{\label{fig.amp_V1p5}%%
			\includegraphics[width=6.cm]{./density_map/amp_full_spec_L20_U5p0_V1p5_n0p875.pdf}}
		\subfigure[Phase]{\label{fig.pha_V1p5}%%
			\includegraphics[width=6.cm]{./density_map/pha_full_spec_L20_U5p0_V1p5_n0p875.pdf}}
		\caption{ Amplitude ($P^A(q,\omega)$) and phase ($P^\Phi(q,\omega)$) spectral functions for a constant coupling $U=5$ in presence of two disorder values, as a function of momenta $q$. The upper panel is for the weak disorder $V=0.25$, and the lower panel is for $V=1.5$. The red horizontal line corresponds to the two-particle spectral gap $\omega_g.$ The system size is $20\times20 $ and the average density is $\langle n \rangle =0.875$.
		} \label{Fig.amp_pha_U5}
	\end{center}
\end{figure}
%%

\section{III.\,\,\,\,\,\,      Fluctuation correlation functions and optical conductivity}
%%
We now study the amplitude correlation function, which is defined as $C(r)=\langle\tilde{\chi}^{AA}(r,\omega)\rangle/\langle\tilde{\chi}^{AA}(0,\omega)\rangle$. This is shown in Fig.~{\ref{Fig.pha_cor}\subref{fig.amp_corU5}} for the clean system. {The correlation function of amplitude fluctuations decays to $0$ monotonously with a rather short typical length which suggests that its role in the conductivity is limited.}

Next we define the phase fluctuation correlation function, defined as $C(r)=\langle\tilde{\chi}^{\Phi\Phi}(r,\omega)\rangle/\langle\tilde{\chi}^{\Phi\Phi}(0,\omega)\rangle$. The phase fluctuation correlation function is rather interesting, which shows a damped oscillation around $C(r)=0$ with increasing distance. We present them in Fig.~{\ref{Fig.pha_cor}\subref{fig.pha_corU5}} and~{\ref{Fig.pha_cor}\subref{fig.pha_corU2}} for two different couplings $U$, where the excitation energy is always below the two-particle spectral gap. {In the strong coupling limit $U=5$, phase fluctuations are excited for even very small energies $\omega \sim 0.04\omega_g$ with a rich oscillating pattern for a broad range of subgap energies. In Fig.~{\ref{Fig.cond_U5_2}\subref{fig.cond_U5}}, the optical conductivity peak is around $\omega \sim 0.18\omega_g$ with some broadening when $V=0.25$. 
For $U=2$ and $\langle n \rangle = 0.875$, the phase fluctuation excites around $\omega \sim 0.32\omega_g$, which is moving to the two particle spectral gap $\omega_g$. The optical conductivity peak is also around $0.6\omega_g$ at weak disorder regime in Fig.~{\ref{Fig.cond_U5_2}\subref{fig.cond_U2}}. }
%%
\begin{figure}[H]
	\begin{center}
		\subfigure[~Amplitude]{\label{fig.amp_corU5}%%
			\includegraphics[width=5.5cm]{./correlation/amp_cor_L30_U5_n0p875_V0.pdf}}
		\subfigure[~Phase]{\label{fig.pha_corU5}%%
			\includegraphics[width=5.5cm]{./correlation/pha_cor_L30_U5_n0p875_V0.pdf}}
		\subfigure[~Phase]{\label{fig.pha_corU2}%%
			\includegraphics[width=5.5cm]{./correlation/pha_cor_L30_U2_n0p875_V0.pdf}}
		\caption{ \subref{fig.amp_corU5}. The amplitude fluctuation correlation function in presence of a strong coupling $U=5$. \subref{fig.pha_corU5} and \subref{fig.pha_corU2} are the phase fluctuation correlations. \subref{fig.pha_corU5} is for the same strong coupling $U=5$, while \subref{fig.pha_corU2} is for weak coupling $U=2$. The other parameters are $V=0, L=30$, and $\langle n \rangle = 0.875$.} \label{Fig.pha_cor}
	\end{center}
\end{figure}
%%
\begin{figure}[H]
	\begin{center}
		\subfigure[]{\label{fig.cond_U5}%%
			\includegraphics[width=7.cm]{./Cond/Cond_L20_U5p0_n0p875_Dinf.pdf}}
		\subfigure[]{\label{fig.cond_U2}%%
			\includegraphics[width=7.cm]{./Cond/Cond_L24_U2p0_n0p875_Dinf.pdf}}
		\caption{The optical conductivity $\sigma(\omega)$ in units of $\sigma_0=\frac{e^2}{\hbar}$. The vertical coloured lines are the corresponding two-particle gaps $\omega_g$. } \label{Fig.cond_U5_2}
	\end{center}
\end{figure}
	
\end{document}
