
%uit main.tex: \usepackage{hyperref}

%uitbibtemplate.tex: \usepackage[bookmarks={false},colorlinks={false},hyperref={false}]{hyperref}

abbreviated to \textbf{Cetal's algorithm}

Koekoek\cite{bellman1962dynamic}.

Cheeseman's research
Depth-first strategies

\section{The Hamiltonian Phase Transition}

\section{Acknowledgements}
Many thanks to Richard Karp, Edward Reingold, Andreas Bj\"orklund and Joseph Culberson for answering questions over email.


\section{Geparkeerde stukken tekst}

The hardest instances for \textbf{Cetal's algorithm} are graphs of $v$ vertices that have $\frac{1}{2}(v-1)(v-2)$ edges and consist of a single disconnected vertex and a fully connected clique of size $v-1$. By design, Cetal's algorithm starts at the highest-degree vertex which lies inside the clique from which the backtracking procedure constantly \textit{almost} finds a Hamiltonian cycle. It persistently recurses to almost-maximum search tree depth while branching along almost-maximum degree on every vertex before terminating at a "no solution" answer. The larger the $v$, the closer the algorithm approaches its complexity, or worst-case runtime, of $O(v!)$.

The likeliness of such graphs in the space of possibilities can be calculated (from ˜\cite{erdos1960evolution} or ˜\cite{HopcroftKannan2014DataScience}). The probability of a zero-degree vertex in graph with $v$ vertices and $\frac{1}{2}(v-1)(v-2)$ edges is

\begin{equation} 
N_i = \frac{a}{b}(c \cdot d).
\end{equation}

It turns out these graphs are nearly non-existent in our data with probabilities of $2.8 \cdot 10ˆ{-14}$ for 16-vertex graphs, and $1.5 \cdot 10ˆ{-25}$ and $4.7 \cdot 10ˆ{-38}$ for 24 and 32 edges. This means that even if these hardest graph force Cetal's algorithm ever closer to the worst-case runtime of $O(v!)$, the probability of such graph existing decreases so much faster that hopelidoppie.

\section{Results}
\begin{figure*}[htbp]
%\begin{figure*}[!tb]
\centering%
\includegraphics[width=\linewidth]{results_distri_iter.png} 
%\fbox{\texttt{Insert figure stuff here.}}
\caption{Caption caption caption caption caption caption caption caption caption caption caption caption caption caption caption caption caption caption caption caption caption caption caption caption caption caption caption caption caption caption caption caption caption caption caption caption caption caption caption caption caption caption caption .}\label{figpham}
\end{figure*}



@misc{videoATSP1,
  author = {van den Berg, Daan},
  title = {Video showing where Cetal's results on ATSP are flawed (part1)},
  year = {2020},
  publisher = {YouTube},
  howpublished = {\url{https://www.youtube.com/watch?v=zfqmiLkbCRc}}
}


@misc{videoATSP2,
  author = {van den Berg, Daan},
  title = {Video showing where Cetal's results on ATSP are flawed (part2)},
  year = {2020},
  publisher = {YouTube},
  howpublished = {\url{https://www.youtube.com/watch?v=l8W_GVzqDnk}}
}


@misc{dashed,
  author = {Markus Kr{\"o}tzsch},
  title = {Complexity Theory, Lecture 6: Nondeterministic Polynomial Time},
  year = {2019},
  publisher = {Technische Universit{\"a}t Dresden},
  howpublished = {\url{https://iccl.inf.tu-dresden.de/w/images/0/08/CT2019-Lecture-06-overlay.pdf}}
}

@misc{sourcecodejoeri,
  author = {Sleegers, Joeri},
  title = {Source Code},
  year = {2020},
  publisher = {GitHub},
  journal = {GitHub repository},
  howpublished = {\url{https://github.com/Joeri1324/What-s-Difficult-About-the-Hamilton-Cycle-Poblem-}}
}

@misc{Gijsinteractivesite,
  author = {van Horn, Gijs},
  title = {Interactively viewable Hamiltonian Cycle Hardness},
  year = {2020},
  howpublished = {\url{https://hamiltoncycle.gijsvanhorn.nl/}},
}
