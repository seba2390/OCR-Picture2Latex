\documentclass{amsart}
%
%%%%%%%%%%%%%%%%%%%%
%\usepackage[notref,notcite]{showkeys}
\usepackage{amsmath}
\usepackage{amssymb}
\usepackage{amsthm}
\usepackage{dsfont}
%\usepackage{subfig}
\usepackage{lineno}
\usepackage{subfigure}
\usepackage{graphicx}
\usepackage{multirow}
\usepackage{color}
\usepackage{xspace}
\usepackage{pdfsync}
\usepackage{hyperref} 
\usepackage{algorithmicx}
\usepackage{algpseudocode}
\usepackage{algorithm}
\usepackage{mathtools}
\usepackage{enumitem}
%\usepackage[pdftex]{hyperref} 
%\usepackage{showlabels}
\graphicspath{{Figures/}}

%%%%%%%%%%%%%%%%%%%%%%%%%
% Macros
%%%%%%%%%%%%%%%%%%%%%%%%%%


%\newcommand{\argmin}{argmin}
\DeclareMathOperator*{\argmin}{argmin}
\DeclareMathOperator*{\argmax}{argmax}
\DeclarePairedDelimiter\ceil{\lceil}{\rceil}
\DeclarePairedDelimiter\floor{\lfloor}{\rfloor}
\newcommand{\bq}{\begin{equation}}
\newcommand{\eq}{\end{equation}}
\newcommand{\R}{\mathbb{R}}
\newcommand{\Z}{\mathbb{Z}}
\newcommand{\abs}[1]{\left\vert#1\right\vert}
\newcommand{\norm}[1]{\left\Vert#1\right\Vert}
\newcommand{\blue}[1]{{\color{blue}{#1}}}
\newcommand{\red}[1]{{\color{red}{#1}}}
\newcommand{\grad}{\nabla}
\newcommand{\ex}[1]{\times 10^{- #1}}
\newcommand{\G}{\mathcal{G}}
\newcommand{\bO}{\mathcal{O}}
\newcommand{\dist}{\text{dist}}
\newcommand{\one}{\mathds{1}}
\newcommand{\Dt}{\mathcal{D}}
\newcommand{\Prop}{\mathcal{P}}
\newcommand{\F}{\mathcal{F}}
\newcommand{\C}{\mathcal{C}}
\newcommand{\Sf}{\mathbb{S}^{2}}
\newcommand{\Tf}{\mathcal{T}}
\newcommand{\Zf}{\mathcal{Z}}
\newcommand{\Af}{\mathcal{A}}
\newcommand{\Bf}{\mathcal{B}}
\newcommand{\Lf}{\mathcal{L}}
\newcommand{\Mf}{\mathcal{M}}
\newcommand{\Nf}{\mathcal{N}}
\newcommand{\Ef}{\mathcal{E}}
\newcommand{\cp}{\text{cp}}
\newcommand{\MA}{Monge-Amp\`ere\xspace}
\newcommand{\test}{\mathcal{T}}
\newcommand{\interp}{\mathcal{P}}

\newcommand*{\avint}{\mathop{\ooalign{$\int$\cr$-$}}}

\algnewcommand{\LineComment}[1]{\State \(\triangleright\) #1}
\newcommand{\specialcell}[2][c]{%
  \begin{tabular}[#1]{@{}c@{}}#2\end{tabular}}


%%% Theorem environments
\newtheorem{theorem}{Theorem}
\newtheorem{thm}{Theorem}

%
\theoremstyle{lemma}
\newtheorem{lemma}[theorem]{Lemma}
\newtheorem{lem}[theorem]{Lemma}
\newtheorem{corollary}[theorem]{Corollary}
\newtheorem{cor}[theorem]{Corollary}
\newtheorem{definition}[theorem]{Definition}
\newtheorem{proposition}[theorem]{Proposition}
\newtheorem{conjecture}[theorem]{Conjecture}
\newtheorem{remark}[theorem]{Remark}
\newtheorem{rem}[theorem]{Remark}
\newtheorem{hypothesis}[theorem]{Hypothesis}

%
\theoremstyle{remark}
\newtheorem{examp}{Example}


\makeatletter
\newcommand\appendix@section[1]{%
\refstepcounter{section}%
\orig@section*{Appendix \@Alph\c@section: #1}%
%\addcontentsline{toc}{section}{Appendix \@Alph\c@section: #1}%
}
\let\orig@section\section
\g@addto@macro\appendix{\let\section\appendix@section}
\makeatother


%%%%%%%%%%%%%%%%%
% Formatted for Article class
%%%%%%%%%%%%%%%%%

% This is for article style

\begin{document}

\title[Reflector Antenna Problem via Optimal Transport]{A Convergent Numerical Method for the Reflector Antenna Problem via Optimal Transport on the Sphere}


\author{Brittany Froese Hamfeldt}
\address{Department of Mathematical Sciences, New Jersey Institute of Technology, University Heights, Newark, NJ 07102}
\email{bdfroese@njit.edu}
\author{Axel G. R. Turnquist}
\address{Department of Mathematical Sciences, New Jersey Institute of Technology, University Heights, Newark, NJ 07102}
\email{agt6@njit.edu}


\thanks{The first author was partially supported by NSF DMS-1619807 and NSF DMS-1751996. The second author was partially supported by  an NSF GRFP}

\begin{abstract}
We consider a PDE approach to numerically solving the reflector antenna problem by solving an Optimal Transport problem on the unit sphere with cost function $c(x,y) = -2\log \left\Vert x - y \right\Vert$. At each point on the sphere, we replace the surface PDE with a generalized Monge-Amp\`ere type equation posed on the local tangent plane. We then utilize a provably convergent finite difference scheme to approximate the solution and construct the reflector. The method is easily adapted to take into account highly nonsmooth data and solutions, which makes it particularly well adapted to real-world optics problems.  Computational examples demonstrate the success of this method in computing reflectors for a range of challenging problems including discontinuous intensities and intensities supported on complicated geoemtries.
\end{abstract}


\date{\today}    
\maketitle
\section{Introduction}\label{sec:intro}
Advances in light emitting diode (LED) technology in recent years have allowed for more flexibility in the engineering of freeform lenses using plastics in light illumination problems. In this article, we focus on the reflector antenna problem, which involves designing a reflector to reshape a point source onto a prescribed output in the far-field.  
On the theoretical side, a major advance in understanding freeform geometric optics problems has been gained by reformulating the problem as a fully nonlinear partial differential equation (PDE) of Monge-Amp\`{e}re type. In the particular case of the reflector antenna problem, this PDE is posed on the sphere.  The curved geometry, nonlinearity of the equation, and singular terms within the PDE make this a challenging problem to solve numerically.  

In this article, we propose a new method for the design of the reflector surface that relies on recent advances by the authors in the numerical approximation and analysis of \MA type equations on the sphere~\cite{HT_OTonSphere2}.  We emphasize that this new method comes with theoretical guarantees of convergence, even in settings involving very non-smooth output intensities~\cite{HT_OTonSphere}.

Computational approaches to solving optical design problems can be roughly divided into three basic categories: (1) techniques that use a ray-mapping to design the optical surface, (2) methods that approximate the optical surfaces by supporting quadrics, and (3) methods that represent the optical surface through the solution to an optimal transportation problem.

The ray-mapping approach generally involves a two-step procedure.  In the first step, a ray mapping is produced between the input and output intensities. In the second step, the laws of reflection and/or refraction are employed to construct a surface that achieves this ray mapping as nearly as possible. Several methods based on this general approach are available including~\cite{Bruneton_lens,Desnijder_raymapping,FFL_optics,Fournier_reflector,Parkyn_lenses}.  A downside to this general approach is that it can be difficult to theoretically justify the existence of an optical surface that exactly produces the desired ray mapping.

Oliker's method of supporting quadrics involves representing the optical surface via supporting ellipsoids or hyperboloids~\cite{Oliker_nearfield,Oliker_SQM}. The simple optical properties of these quadrics is used to produce a pixelated version of the desired target.  This approach has the advantage of being theoretically well-founded, but can be costly to implement in practice.

Finally, the solution to many optical design problems can be obtained directly through the solution of a corresponding optimal transportation problem.  That is, if $f_1$ represents the input intensity and $f_2$ the desired output intensity, it is necessary to solve a problem of the form
\bq\label{eq:OT}
\min\limits_{T_\# f_1 = f_2} \int_{\text{supp}(f_1)} c(x,T(x)) f_1(x) dx.
\eq
where $c(x,y)$ is the cost of transporting a unit of mass from $x$ to $y$ and $T_\# f_1 = f_2$ indicates that
\bq\label{eq:massCons} \int_A f_1(x)\,dS(x) = \int_{T(A)} f_2(y)\,dS(y) \eq
for every measurable $A \subset \Sf$.
%Here $c(x,y) = -2\log \left\Vert x - y \right\Vert$ is the cost of transporting a unit of mass from $x$ to $y$ and $T_\# f_1 = f_2$ indicates that
%\[ \int_A f_1(x)\,dS(x) = \int_{T(A)} f_2(y)\,dS(y) \]
%for every measurable $A \subset \Sf$.


Many optical inverse problems have yielded fruitful interpretations via optimal transport by deriving an appropriate cost function $c(x,y)$~\cite{YadavThesis}. To give a simple example, a parallel-in, far-field out setup yields the cost function $c(x,y) = \frac{1}{2}\left\Vert x - y \right\Vert^2$, where $x,y \in \mathbb{R}^2$.  The reflector antenna problem considered in this article
 has a slightly more challenging set-up in that the cost function $c(x,y) = -2\log \left\Vert x - y \right\Vert$ is unbounded and the intensity functions $f_1, f_2$ are supported on $\mathbb{S}^2$ (the unit $2$-sphere), as opposed to subsets of Euclidean space~\cite{GangboOliker,OlikerNewman,Wang_Reflector,Wang_Reflector2}. 

One approach to solving optimal transport problems in optical design is to use optimization techniques, including linear assignment~\cite{Doskolovich_farfield} and linear programming~\cite{GlimmOliker_SingleReflector}.  This approach has the advantage of being theoretically well-understood.  However, the optimization problems typically involve a very large number of constraints and the resulting methods are computationally complex. 

In many cases, the solution to the optimal transport problem can also be obtained through the solution of a fully nonlinear partial differential equation of \MA type, which has the general form
\bq\label{eq:MA}
\det(D_{xx}^2(u(x)+A(x,\nabla u(x)))) = H(x,\nabla u(x))
\eq
subject to the constraint that
\bq\label{eq:cconvex}
D_{xx}^2(u(x)+A(x,\nabla u(x))) \geq 0,
\eq
where $M\geq 0$ means that $M$ is positive semi-definite.  In the case of a point source lens or reflector design problem, this PDE is posed on the unit sphere $\Sf$.

Recently, several methods have been proposed for solving optical design problems involving a point source via the solution of a \MA type equation.  These methods replace the PDE on the sphere with a corresponding equation on the plane by representing subsets of the unit sphere using spherical coordinates~\cite{Wu_lensdesign}, a vertical projection of coordinates onto the plane~\cite{Brix_MAOptics}, or stereographic projection~\cite{RomijnSphere}.  As the numerical solution of these \MA type equations is a very new field, many of the numerical methods used in optical design problems are not yet equipped with theoretical guarantees of convergence.

In the present article, the solution to the reflector antenna problem is obtained by solving a \MA type equation directly on the sphere.  This has the advantage of allowing for intensity distributions supported on complicated subsets of the sphere or even the entire sphere.  Moreover, the approach is intrinsic and thus the solution to the problem will not depend on such details as the choice of the north pole.  Finally, the numerical method we use is theoretically well-justified and can be proven to converge to the correct solution of the \MA equation in a wide variety of challenging settings~\cite{HT_OTonSphere, HT_OTonSphere2}.







\section{Mathematical Approach}\label{sec:background}
%\subsection{Reflector antenna problem}

Here we briefly summarize the derivation of the reflector antenna problem and its connection to optimal transport on the sphere, which leads to an equation of \MA type that can be solved using techniques from numerical PDEs. 

We begin by following the physical derivation in~\cite{Wang_Reflector, Wang_Reflector2}. We start with a light source or detector $\mu$ located at the origin, which is a probability measure indicating directional intensity and is supported on a set $\Omega \subset \mathbb{S}^2$. Next we consider a reflector surface $\Sigma$, which is a radial graph over the domain $\Omega$ and can be represented as
\begin{equation}
\Sigma = \left\{ x\rho(x) \mid x \in \Omega, \quad \rho>0 \right\}
\end{equation}
where $\rho: \Omega \rightarrow \mathbb{R}$ is a non-negative function indicating the distance between the reflector surface and the origin. The light from the source $\mu$ in the direction $x$ bounces off the reflector $\Sigma$ without any refraction or absorption and travels in the direction $T$ following the law of reflection. Over all directions this produces the far-field intensity $\nu$, which is also a probability measure indicating directional intensity and is supported on some target domain $\Omega^{*} \subset \Sf$. See Figure~\ref{fig:reflectorantenna} for a schematic of the setup.

\begin{figure}[htp]
\includegraphics[height=8cm]{reflectorantenna2}
\caption{Reflector antenna with source/detector $\mu$, reflector $\Sigma$ and target far-field intensity $\nu$. The directional vectors $x$ and $T(x)$ are unit vectors.}\label{fig:reflectorantenna}
\end{figure}

The reflector antenna problem is thus: given source and target intensity probability distributions $\mu$ and $\nu$, respectively, find the shape of the reflector $\Sigma$ that transmits the light from the source to the target while satisfying conservation of energy.
We make the assumption that the probability densities $\mu$ and $\nu$ have density functions $f_1$ and $f_2$ respectively (so that $d\mu(x) = f_1(x)dS(x), d\nu(y) = f_2(y)dS(y)$).  Now we seek a PDE that will allow us to determine the reflector height function $\rho(x)$, which fully determines the reflector surface, in terms of the prescribed intensity functions $f_1$ and $f_2$.

The first of the two physical laws that will be used to derive the governing PDE for this setup is the well known geometric law of reflection, which yields the optical map
\begin{equation}\label{eq:mapping}
T(x) = x - 2\left\langle x, n(x) \right\rangle n(x)
\end{equation}
where $n(x)$ is the outward normal to $\Sigma$ at the point $z = x \rho(x)$, $x \in \Omega$.  See Figure~\ref{fig:reflection}.  We emphasize here that this is the geometric optics limit.

\begin{figure}[htp]
\includegraphics[height=4.5cm]{reflection}
\caption{Incident light direction $x$, reflector $\Sigma$, outward normal $n$, and outward light ray $T$}\label{fig:reflection}
\end{figure}

%From~\eqref{eq:mapping} an expression for the mapping $T$ in terms of the function $\rho$ can be derived. The second physical law that completes the problem is the energy conservation law:

The second physical law that completes the problem is the law of conservation of energy:
\begin{equation}\label{eq:conservation}
\int_{{T}^{-1}(E)} f_1(x)\,dS(x) = \int_{E} f_2(y)\,dS(y).
\end{equation}
for any Borel set $E \subset \Omega^{*}$.
 %By the explicit formulation of the mapping $T$~\ref{eq:mapping} and the change of variables formula applied to~\ref{eq:conservation}, one derives the PDE~\cite{Wang_Reflector2} in local coordinates:

By introducing local coordinates on the sphere, Wang~\cite{Wang_Reflector} observes that the unit normal $n$ can be given by
\bq\label{eq:n}
n(x) = \frac{\nabla\rho(x) - x\rho(x)}{\sqrt{\rho(x)^2+\norm{\nabla\rho(x)}^2}}.
\eq
Then the law of reflection~\eqref{eq:mapping} yields the mapping
\bq\label{eq:mapping2}
T(x) = \frac{2\rho(x)\nabla\rho(x)+\left(-\rho(x)^2+\norm{\nabla\rho(x)}^2\right)x}{\rho(x)^2+\norm{\nabla\rho(x)}^2}.
\eq
Applying the change of variables formula to the conservation of energy constraint~\eqref{eq:conservation} produces an equation of the form
\bq\label{eq:detEquation}
\det(\nabla T(x)) = f_1(x)/f_2(T(x)).
\eq
Combining these conditions yields the PDE
\begin{equation}\label{eq:opticsPDE}
\eta^{-2} \det\left( -\nabla_{i} \nabla_{j} \rho + 2 \rho^{-1} \nabla_{i} \rho \nabla_{j} \rho + (\rho - \eta)\delta_{ij} \right) = f_1(x)/f_2(T(x))
\end{equation}
where $\eta = \left( \left\vert \nabla \rho \right\vert^2 + \rho^2 \right)/2\rho$ and $\delta_{ij}$ is the usual Kronecker delta. We recognize this PDE as an equation of Monge-Amp\`{e}re type, with the usual second boundary value condition~\cite{Urbas}
\begin{equation}
T(\Omega) = \Omega^{*}.
\end{equation}

Unfortunately, there are few direct results in the literature that answer the kind of questions of existence and regularity results that are needed to design a convergent numerical method for~\eqref{eq:opticsPDE}. Instead, we extract a problem with more structure via the change of variables
\begin{equation}\label{eq:cov}
\rho = e^{-u}.
\end{equation}
Wang~\cite{Wang_Reflector2} shows that under an equivalent change of variables (modulo a sign change), the function $u$ solves the dual formulation of the optimal transport problem with cost function $\tilde{c}(x,y) = -\log(1-x\cdot y)$.

As an alternative approach, we notice that under this change of variables, the optical mapping~\eqref{eq:mapping2} becomes
\bq\label{eq:mappingu}
T(x) = \frac{-2\nabla u(x) + \left(\norm{\nabla u(x)}^2-1\right)x}{\norm{\nabla u(x)}^2+1}.
\eq
As in~\cite{HT_OTonSphere}, we regard this mapping as a function of the two variables $(x,\nabla u(x))$ and recognize it as a solution of the system
\bq\label{eq:optimalMap}
\begin{cases}
\nabla_x c(x,T(x,p)) = -p, & x \in \Sf\\
T(x,p) \in \Sf
\end{cases}
\eq
with the cost function 
%
%%
%Using the solution $u$ of the optimal transport problem on the sphere with a logarithmic cost function, we can solve for the reflector $\Sigma$ by computing $\rho = e^{u}$. The convergence guarantees for numerically solving the optimal transport problem in~\cite{HT_OTonSphere, HT_OTonSphere2} then translate to a numerical convergence proof for the shape of the reflector $\Sigma$ solving the reflector antenna problem.
%
%%\subsection{Optimal transport on the sphere}
%
%We consider solving the reflector antenna via solving the Optimal Transport problem with logarithmic cost with a PDE formulation on the sphere. We consider points $x,y$ lying on a unit sphere $\Sf$ centered at the origin and the cost function:
\bq\label{eq:logCost}
c(x,y) = -2\log \norm{ x-y }.
\eq
This is precisely the optimality condition for the optimal transport problem on the sphere~\cite{Loeper_OTonSphere}.  Combined with the conservation of energy condition~\eqref{eq:conservation}, we can conclude that the optical mapping $T(x)$ is a solution of the optimal transport problem~\eqref{eq:OT}-\eqref{eq:massCons} with cost~\eqref{eq:logCost}.  Moreover, this interpretation opens up many existence, regularity, and numerical approximation results that can be used in determining the reflector surface $\Sigma$.

Loeper has studied this problem in detail~\cite{Loeper_OTonSphere}. Under mild conditions on the intensity distributions $f_1$ and $f_2$, the function $u$ (which fully determines the reflector surface) can be uniquely obtained as the solution of the following \MA type equation.
\bq\label{eq:OTPDE}
\begin{cases}
\det(D^2u + A(x,\nabla u)) = H(x,\nabla u), & x \in \Sf\\
D^2u + A(x,\nabla u) \geq 0.
\end{cases}
\eq
Here
\bq\label{eq:PDETerms}
\begin{split}
A(x,p) &= D_{xx}^2c \left( x,T(x,p) \right)\\
H(x,p) &= \abs{\det{D_{xy}^2c \left( x,T(x,p) \right)}}f_1(x)/f_2 \left( T(x,p) \right).
\end{split}
\eq
and the statement $M \geq 0$ means that $M$ is positive semi-definite.  This constraint (related to the so-called $c$-convexity of the optimal map $T$) is needed to ensure that the PDE has a unique solution (up to additive constants) and that this solution corresponds to the desired optical mapping $T$. 

We remark that the above equation describes a nonlinear relationship between the surface gradient and Hessian on the sphere.  In light of our goal of solving this equation numerically, perhaps the most challenging term is the mixed Hessian $D_{xy}^2c(x,y)$, which involves derivatives with respect to two different variables located at different points on the sphere.  However, following the derivation in~\cite{HT_OTonSphere2}, we can obtain a very simple explicit expression for this term by interpreting it as a change of area formula:
\begin{equation}\label{eq:mixedHessian}
\abs{\det{D_{xy}^2c \left( x,T(x,p) \right)}} = \frac{\left(\left\Vert p \right\Vert^2 + 1 \right)^2}{4}.
\end{equation}

%This problem was studied by Loeper~\cite{Loeper_OTonSphere}, who showed that under mild regularity requirements on the data, the optimal transport problem on the sphere admits a smooth ($C^3$) solution $u$. Weak ($C^1$) solutions are also possible for discontinuous density functions that are only in $L^p$ ($1 \leq p < \infty$).  These regularity results will be important in constructing convergent numerical schemes. Moreover, the solution is unique up to additive constants.


A second challenge associated with the nonlinear \MA type equation~\eqref{eq:OTPDE} is that it requires the enforcement of an additional constraint that $D^2u + A(x,\nabla u) \geq 0$, which makes it difficult to directly apply standard techniques for approximating PDEs.  However, we succeed at absorbing this constraint into the PDE itself by relying on the following characterization of a positive semi-definite $n\times n$ matrix $M$~\cite{FO_MATheory}:
\bq\label{eq:matrix}
\begin{split}
\det(M) &= \min\limits_{\nu_i^T\nu_k=\delta_{ik}}\prod\limits_{j=1}^n \nu_j^TM\nu_j \\
  &= \min\limits_{\nu_i^T\nu_k=\delta_{ik}}\prod\limits_{j=1}^n \max\{\nu_j^TM\nu_j, 0\}.
\end{split}
\eq
Here $\delta_{ij}$ denotes the Kronecker delta function and this involves a minimization over all orthogonal coordinate frames for $\R^n$.  By observing that $\nu_j^TM\nu_j \geq 0$ for any positive semi-definite matrix $M$, we can include this condition directly in the operator instead of requiring it to be specified as a separate constraint.  This allows us to reformulate the system~\eqref{eq:OTPDE}-\eqref{eq:PDETerms} as the following unconstrained PDE.
\bq\label{eq:PDEReformulated}
\begin{split}
F(x,\nabla u(x), D^2u(x)) &\equiv
\min\limits_{\nu_1\cdot\nu_2 = 0}\prod\limits_{j=1}^2 \left. \max\left\{\frac{\partial^2(u(x)-2\log\norm{x-y})}{\partial\nu_j^2},0\right\} \right|_{y = T(x,\nabla u(x))}\\ &\phantom{=}- \frac{\left(\norm{\nabla u(x)}^2+1\right)^2 f_1(x)}{4f_2(T(x,\nabla u(x))}\\
 &= 0.
\end{split}
\eq




\section{Numerical Method}\label{sec:method}
We now describe the algorithm we use to construct the reflector surface $\Sigma$.  The algorithm hinges on the numerical solution of the nonlinear PDE~\eqref{eq:PDEReformulated}.  For fully nonlinear PDEs, it is well known that consistent and stable numerical methods may nevertheless fail to compute the correct solution. In fact, because the function $u$ is unique only up to additive constants, even fairly sophisticated numerical methods can fail to find any solution at all.  The method we describe here is inspired by a numerical scheme recently designed by the authors, which is equipped with a proof of convergence to the physically meaningful solution of the optimal transport problem.  We summarize the scheme here, and refer to~\cite{HT_OTonSphere,HT_OTonSphere2} for complete details and analysis.

\subsection{Algorithm}
We begin with a high-level overview of the algorithm.  Details will be expanded on in the following subsections.

Our starting point is a finite set of $N$ grid points $\G \subset \Sf$ that discretize the unit sphere, and the intensity distributions $f_1$ and $f_2$ that are supported on domains $\Omega\subset\Sf$ and $\Omega^*\subset\Sf$ respectively.  We let $d_{\Sf}(x,y)$ denote the usual geodesic distance between points $x,y$ on the sphere.

To the grid $\G$, we associate a number $h$ that indicates the overall spacing of grid points.  More precisely,
\begin{equation}\label{eq:h}
h = \sup\limits_{x\in\Sf}\min\limits_{y\in\G^h} d_{\Sf}(x,y) = \bO\left(N^{-1/2}\right).
\end{equation}
In particular, this guarantees that any ball of radius $h$ on the sphere will contain at least one discretization point.

Now we seek a finite difference approximation of the form
\bq\label{eq:fd}
F^h(x,u;f_1,f_2) = 0, \quad x \in \G
\eq
that approximates the original PDE~\eqref{eq:PDEReformulated}.  Our goal is to construct an approximation with the properties that (1) a solution $u^h$ exists and (2) the solution is close to the solution $u$ of the original PDE.  Our earlier work~\cite{HT_OTonSphere,HT_OTonSphere2} provides a framework for doing this.  In the most challenging settings, this requires some initial preprocessing of the data $f_1, f_2$, but then provides us with an algorithm that is guaranteed to produce a reflector surface $\Sigma^h$ that is close to the desired reflector $\Sigma$.
See Algorithm~\ref{alg:reflector}.

\begin{algorithm}[h]
\caption{Computing the reflector surface $\Sigma$}
\label{alg:reflector}
\begin{algorithmic}[1]
\State Preprocess data \[f_2^\epsilon \leftarrow \text{Regularize}(f_2).\]
\State Iterate
\[ u^h_{n+1} = u^h_n + k \left(F^h(x,u^h_n;f_1,{f_2^\epsilon}) - \sqrt{h}u^h_n(x) \right) \]
to steady state.
\State Normalize solution
\[ u^h(x) \leftarrow u^h(x) - \avint_{\phantom{==}\Sf}u^h(x)\,dS(x). \]
\State Construct reflector
\[ \Sigma^h = \left\{xe^{-u^h(x)} \mid x \in \Omega \cap \G\right\}. \]
\end{algorithmic}
\end{algorithm}

\subsection{Discretization}
We now consider a fixed grid point $x_0 \in \G$ and a grid functions $u:\G\to\R$ and explain how we obtain the value of $F^h(x_0,u;f_1,f_2)$; we refer to~\cite{HT_OTonSphere2} for further details.

We begin by projecting grid points close to $x_0$ onto the tangent plane at $x_0$.  That is, we consider the set of relevant discretization points
\bq\label{eq:neighbourCandidates}
\Zf(x_0) = \left\{z = \text{Proj}(x;x_0) \mid x \in \G, d_{\Sf}(x,x_0) \leq \sqrt{h}\right\}.
\eq
The projection is accomplished using geodesic normal coordinates, which are chosen to preserve the distance from $x_0$ (i.e. $d_{\Sf}(x,x_0) = \|x_0 -  \text{Proj}(x;x_0)\|$).  This prevents any distortions that would affect the second order terms in the PDE~\eqref{eq:PDEReformulated}.
\bq\label{eq:projection}
\text{Proj}(x;x_0) = x_0\left(1-d_{\Sf}(x_0,x)\cot d_{\Sf}(x_0,x)\right) + x \left(d_{\Sf}(x_0,x)\csc d_{\Sf}(x_0,x)\right).
\eq

The form of~\eqref{eq:PDEReformulated} indicates that we will need to approximate derivatives along various directions $\nu$.  We will consider the following finite set of possible directions,
\bq\label{eq:directions}
V = \left\{\left\{(\cos(jd\theta),\sin(jd\theta)),(-\sin(jd\theta),\cos(jd\theta))\right\} \mid j=1,\ldots,\frac{\pi}{2d\theta}\right\},
\eq
where the angular resolution $d\theta = \dfrac{\pi}{2\floor{\pi/(2\sqrt{h})}}$.

For each $\nu\in V$, we need to select four grid points $x_j\in\Zf(x_0)$, $j=1, \ldots, 4$, which will be used to construct the directional derivatives in this direction.  To accomplish this, we let $\nu^\perp$ be a unit vector orthogonal to $\nu$ and represent points in $x\in\Zf(x_0)$ using (rotated) polar coordinates $(r,\theta)$ centred at $x_0$ via
\[ x = x_0 + r(\nu\cos\theta + \nu^\perp\sin\theta), x \in \Zf(x_0). \]
Then we select four points, each in a different quadrant ($Q_1, \ldots, Q_4$), that are well-aligned with the direction of $\nu$ via
\bq\label{eq:neighbours}
x_j \in \argmin\limits_{x\in\Zf(x_0)}\left\{\abs{\sin\theta} \mid \abs{\sin\theta} \geq d\theta, r \geq \sqrt{h}-2h, x\in Q_j\right\}
\eq
where $\cos\theta \geq 0$ for points in $Q_1$ or $Q_4$ and $\sin\theta \geq 0$ for points in $Q_1$ or $Q_2$.  

From here, we construct approximations of second directional derivatives (and first directional derivatives for the usual coordinate directions $(1,0)$ and $(0,1)$) of the form
\bq\label{eq:dnu}
\begin{split}
\Dt_{\nu\nu}u(x_0) &= \sum\limits_{j=1}^4 a_j(u(x_j)-u(x_0)) \approx \frac{\partial^2u(x_0)}{\partial\nu^2}\\
\Dt_{\nu}u(x_0) &= \sum\limits_{j=1}^4 b_j(u(x_j)-u(x_0)) \approx \frac{\partial u(x_0)}{\partial\nu}.
\end{split}
\eq
The coefficients in these finite difference approximations are given explicitly by
\bq\label{eq:coeffs}
\begin{split}
a_{1} &= \frac{2\sin\theta_4(\cos\theta_3\sin\theta_2-\cos\theta_2\sin\theta_3)}{r_1\det(A)}\\
a_{2} &= \frac{2\sin\theta_3(\cos\theta_1\sin\theta_4-\cos\theta_4\sin\theta_1)}{r_2\det(A)}\\
a_{3} &= \frac{-2\sin\theta_2(\cos\theta_1\sin\theta_4-\cos\theta_4\sin\theta_1)}{r_3\det(A)}\\
a_{4} &= \frac{-2\sin\theta_1(\cos\theta_3\sin\theta_2-\cos\theta_2\sin\theta_3)}{r_4\det(A)}\\
b_{1} &= \frac{\sin\theta_4 (r_2\sin\theta_3 \cos^2\theta_2 - r_3\sin\theta_2 \cos^2\theta_3)}{r_1\det(A)} \\
b_{2} &= -\frac{\sin\theta_3 (r_1\sin\theta_4 \cos^2\theta_1 - r_4\sin\theta_1 \cos^2\theta_4)}{r_2\det(A)} \\
b_{3} &= \frac{\sin\theta_2 (r_1\sin\theta_4 \cos^2\theta_1 - r_4\sin\theta_1 \cos^2\theta_4)}{r_3\det(A)} \\
b_{4} &= -\frac{\sin\theta_1 (r_2\sin\theta_3 \cos^2\theta_2 - r_3\sin\theta_2 \cos^2\theta_3)}{r_4\det(A)}
\end{split}
\eq
where 
\bq\label{eq:detA}\begin{split}\det(A) = &(\cos\theta_3\sin\theta_2-\cos\theta_2\sin\theta_3)(r_1\cos^2\theta_1\sin\theta_4-r_4\cos^2\theta_4\sin\theta_1)\\&-(\cos\theta_1\sin\theta_4-\cos\theta_4\sin\theta_1)(r_3\cos^2\theta_3\sin\theta_2-r_2\cos^2\theta_2\sin\theta_3).\end{split}\eq

Equation~\eqref{eq:PDEReformulated} contains several functions of the gradient.  We introduce the shorthand notation
\bq\label{eq:gradFuns}
g_1(p;\nu) = \left. -2\Dt_{\nu\nu}\log\norm{x_0-y}\right|_{y=T(x_0,p)}, \quad g_2(p) = \frac{\left(\norm{p}^2+1\right)^2}{4f_2(T(x_0,p))},
\eq
denote by $L_g$ the Lipschitz constant of the function $g$, and for each function define the small parameter
\bq\label{eq:epsilon}
\epsilon_g = L_g\max\limits_{j=1,\ldots,4}\frac{\abs{b_j}}{\abs{a_j}} = \bO(\sqrt{h}).
\eq
Then all functions of the gradient can be discretized using a Laplacian regularization via
\bq\label{eq:discGrad}
g^{\pm}\left(\nabla^h u(x_0)\right) = g\left(\Dt_{(1,0)}u(x_0),\Dt_{(0,1)}u(x_0)\right) \mp \epsilon_g\left(\Dt_{(1,0),(1,0)}u(x_0) + \Dt_{(0,1),(0,1)}u(x_0)\right).
\eq

This regularization allows for the construction of a monotone scheme, which is necessary for the convergence theorem in~\cite{HT_OTonSphere}. Finally, we can combine these different operators to obtain the approximation
\bq\label{eq:approx}
\begin{split}
F^h&(x_0,u;f_1,f_2) = \\ &\min\limits_{\{\nu_1,\nu_2\}\in V} \prod\limits_{j=1}^2 \max\left\{\Dt_{\nu_j\nu_j}u(x_0) + g^-_{1,\nu_j}(\nabla^h u(x_0)), 0\right\} - f_1(x_0)g_2^+\left(\nabla^h u(x_0)\right).
\end{split}
\eq



\begin{remark}
The method of~\cite{HT_OTonSphere2} in principal involves solving a problem with this approximation, verifying that the solution satisfies required Lipschitz bounds, then if necessary solving a second discrete problem to enforce the Lipschitz condition.  However, we have never seen the verification step fail in practice, and hence never actually need to solve a second discrete system. 
\end{remark}

\subsection{Computational Complexity}
Let $N$ be the total number of grid points.  
At each point $x_0\in\G$, evaluating the operator $F^h$ involves computing a minimum over the $\bO\left(1/d\theta\right) = \bO\left(1/\sqrt{h}\right)=\bO\left(N^{1/4}\right)$ pairs of vectors in $V$.

Each pair of vectors $\{\nu_1,\nu_2\}\in V$ requires the construction of two finite difference operators of the form $\Dt_{\nu\nu}$.  Computing each of these requires identifying the four neighbors $x_1, x_2, x_3, x_4$ in the stencil.

We note that selecting each of these neighboring points $x_j$ as in~\eqref{eq:neighbours} involves searching a region whose area scales like $\bO(h^2)$.  From the definition of $h$, this is guaranteed to contain at least one point, and expected to contain $\bO(1)$ points total. Thus identification of these four neighboring points can be done in $\bO(1)$ time.

Thus, given a grid function $u$, the total computational cost of evaluating the operator $F^h$ at all points in the grid is $\bO\left(N^{5/4}\right)$.

\subsection{Preprocessing of data}\label{preprocessing}
Stability and convergence of the numerical method requires at least one of the densities (denoted by $f_2$) to be strictly positive.  This is easily accomplished by choosing $\epsilon>0$ and letting
%\bq\label{eq:f2pos}
%\tilde{f}_2^\epsilon = \frac{\max\{f_2,\epsilon\}}{ \avint_{\Sf}\max\{f_2,\epsilon\}\,dS}.
%\eq
%although sometimes this will be accomplished by the following regularization:
\bq\label{eq:f2pos}
\tilde{f}_2^\epsilon = (1-\epsilon)f_2 + \frac{\epsilon}{4\pi}.
\eq
As $\epsilon\to0$, the mapping of the regularized optimal transport problem converges in measure to the solution of the given problem~\cite{Villani1}, and thus we recover the desired reflector surface.

The numerical method further requires this density function to be smoothed in order to have a (discrete) Lipschitz constant that is at most $\bO\left(h^{-1/4}\right)$.  We accomplish this via a short-time evolution of the heat equation.  That is, we solve
\bq\label{eq:heat}
\begin{cases}
v_t(x,t) = \Delta v(x,t), &(x,t)\in\Sf\times(0,\sqrt{h}]\\
v(x,0) = \tilde{f}_2^\epsilon(x), &x\in\Sf
\end{cases}
\eq
where $\Delta$ is the Laplace-Beltrami operator.  We then set
\bq\label{eq:f2reg}
f_2^\epsilon(x) = v(x,\sqrt{h}).
\eq

The Laplace-Beltrami operator can be discretized using the finite difference schemes~\eqref{eq:dnu} as
\bq\label{eq:Laplace}
\Delta^h = \Dt_{(1,0),(1,0)} + \Dt_{(0,1),(0,1)}
\eq
and evolved using forward Euler
\bq\label{eq:heatDisc}
v^{n+1} = v^n + k \Delta^hv^n.
\eq
The wide stencil nature of the finite difference stencils ($\norm{x_j-x_0} = \bO(\sqrt{h})$) means that this is stable for a time step $k \leq 1/\sum\limits_j a_j = \bO(h)$.  Thus a total of $\bO\left(h^{-1/2}\right)$ time steps are needed, which leads to an overall cost of $\bO\left(N^{5/4}\right)$ that is similar to the cost of discretization.

This regularization procedure can also be applied to unbounded densities, but requires evolving the heat equation to a stopping time of $t=h^{1/6}$ to achieve the required Lipschitz bound.

\subsection{Parabolic solvers}
After discretization, we are left with the task of solving the nonlinear algebraic system
\bq\label{eq:system}
F^h(x,u;f,g) = 0, \quad x \in \G.
\eq
Here, we use an explicit  parabolic scheme of the form
\bq\label{eq:parabolic}
u_{n+1}^h(x) = u_n^h(x) + k F^h(x,u^h_n;f,g).
\eq

As discussed in~\cite{ObermanSINUM}, we can require the time step $k$ to satisfy a nonlinear CFL condition in order to guarantee convergence. In particular, choosing $k < 1/L_{F^{h}} = \bO(h^{-2})$ is sufficient, where $L_{F^{h}}$ is the Lipschitz constant of $F^h$ with respect to the arguments $u^h$. However, in practice these parabolic schemes are sped up using techniques from~\cite{SchaefferHou}, which allows for potentially much larger time steps to be chosen on the fly and preserves convergence guarantees.



%Finally, the solutions of the Optimal Transport PDE satisfy \emph{a priori} Lipschitz bounds $\norm{\nabla u} < R$ for any $R>C$ (see~\cite[Proposition~6.1]{Loeper_OTonSphere} for details).  These bounds can be explicitly built into the PDE through a further modification
%\begin{equation}\label{eq:modifiedPDE}
%G(x,\nabla u(x),D^2u(x)) \equiv \max \left\{ F^{+}(x,\nabla u(x),D^2u(x)), \norm{ \nabla u(x) } - R \right\} = 0.
%\end{equation}
%While this modification enforces a condition that is automatically satisfied by solutions at the continuous level, it also improves the stability of approximation schemes.

%\subsubsection{Convergence results}

%The computational cost of constructing the discretization is $\mathcal{O}(N^{3/2})$. The discrete operator $G^h_i$ at a point $x_i$ involves discrete differential operators which are functions of the maximum operator over all points in computational neighborhoods. That is, at the point $x_i$, we have a computational neighborhood $N_i$, which has radius $\mathcal{O}(\sqrt{h})$ and the operator $G^h_i$, depends on values at $x_j \forall j \in N_i$. Each neighborhood $N_i$ contains $\mathcal{O}(h) \cdot N = \mathcal{O}(\sqrt{N})$ points. Then, since the discretization $G^h$ is an algebraic function of these terms, the computational cost of constructing the discretization is $N \cdot \mathcal{O}(\sqrt{N}) = \mathcal{O}(N^{3/2})$.
%
%
%
%
%
%Since the regularization of $f_2$ is a precomputation, the overall computational complexity of the discretization is still $\mathcal{O}(N^{3/2})$.


\section{Computational Results}\label{sec:numerics}
Here we demonstrate the effectiveness of our method with several computational examples. These include reflector design problems involving an omnidirectional source, discontinuous intensity distributions, and intensity distributions supported on sets with complicated geometries.  In each example, we use Algorithm~\ref{alg:reflector} to construct an approximate reflector $\Sigma^h$.  

%Two of the examples involve source and target masses on the entire sphere, which is not entirely realistic as far as the engineering of a reflector antenna is concerned. However, it demonstrates the ability of the numerical scheme to deal with complicated examples with full support on the sphere. 

%\begin{enumerate}
%\item[1.] The ``globe'' example, involves a discontinuous source density of a map of the world $f_1$ to a constant target mass $f_2$.
%\item[2.] The ``peanut reflector'' example, is derived from the ideal headlight intensity pattern of a vehicle and is inspired by the work done in~\cite{RomijnSphere}. The source intensity $f_1$ is the headlight intensity pattern mapped onto the sphere and the target intensity $f_2$ is a constant.
%%\item[3.] The ``smooth'' example, is a simpler computation of a smooth source mass $f_1$ concentrated at two ends of the sphere and a target mass density $f_2$ which is a constant. 
%%\item[4.] The ``disjoint'' example, shows a source and target masses located in the northern and southern hemispheres, respectively. They have disjoint support (except for the fact that $f_2 \geq \epsilon>0$, for small $\epsilon$ as stipulated by the optimal transport theory~\cite{Loeper_OTonSphere}. This example, of course, corresponds more naturally to the actual optical setup.
%\item[3.] The ``donut'' example, where a source and target densities approximate the situation where both have support on sets that are non $c$-convex (except, again that $f_2\geq \epsilon>0$).
%\item[4.] The ``triangle'' example, where a geodesic triangle source gets mapped to a smoothened target mass supported on the northern hemisphere
%\end{enumerate}

%Here, the numerical solution $u^h$ is computed and used to calculate the numerical reflector height function $\rho^h$, which is then used to construct the numerical reflector shape $\Sigma^h = x_i \rho^h$. To reemphasize, convergence is guaranteed for $u^h$ and therefore $\rho^h$ as $h \rightarrow 0$.


In order to validate our results, we first use the law of reflection~\eqref{eq:mapping2} to perform approximate (forward or inverse) ray-tracing.  We then construct the resulting intensity patterns via approximation of the conservation of energy equation equation~\eqref{eq:conservation} by \[f_1(x_i) \Delta x_i \approx f_2(y_i) \Delta y_i,\]\label{econs} where $\Delta x_i$ and $\Delta y_i$ are the areas of the Voronoi regions containing $x_i$ and $y_i = T(x_i)$ respectively. 
%This provides an approximate ray traces which can be used to validate the performance of the scheme, in the absence of convergence guarantees for $T^h$. 

After performing ray tracing, the presence of numerical artifacts may require that the data be post-processed to show the results clearly. This is done by rescaling the colorbars to cut off a very small number of the highest values. Any numerical artifacts are presented in plots of the difference between the desired and ray-traced intensities.
%The presence of numerical artifacts for $T^h$ is expected, since the ideal result would be $T^h \rightarrow T$ in measure as in standard optimal transport theory~\cite{Villani1}, and a future objective is to develop a framework to derive convergence results for the map $T^h$, as it has a lot of importance in some applications beyond just validation, see~\cite{HT_OTonSphere, HT_OTonSphere}.

All computations were performed on a 13-inch MacBook Pro, 2.3 GHz Intel Core i5 with 16GB 2133 MHz LPPDDR3 using Matlab R2017b. Each computation utilized around $N\approx 20,000$ points on the sphere.  Where applicable, regularization was performed using $\epsilon=0.3$. The precomputation step of approximating all directional derivatives for $N\approx20,000$ points took about $10$ minutes. Solving the parabolic scheme to find the solution took around $30$ minutes. Ongoing work will develop faster, more accurate versions of this method.  We see therefore that the proposed numerical method can certainly accommodate higher precision computations if necessitated by real-world applications.

\subsection{Peanut Reflector}

Following the example of~\cite{RomijnSphere}, we consider a source density coming from an ideal headlight intensity emitting from a vehicle's high beams. This headlight intensity pattern is then mapped to the sphere, and inverted, which becomes the source intensity $f_1$. The target density $f_2$ is constant. 
%Here we do not show the ray trace inverse before rescaling as we did for the globe example. 
The computation yields a peanut-shaped oblong reflector lens; see Figure~\ref{fig:peanutresult}. Despite the fact that we anticipate error in the reverse ray trace due to the approximate conservation of energy equation~\eqref{econs}, we see that the absolute error performs quite well in this smooth example.  The average error in the reconstruction is 11\% of the maximum intensity.

%\begin{figure}
%	\subfigure[Peanut lens in 2D showing the headlight pattern]{\includegraphics[height=5.2cm]{peanut_lens}}
%	\subfigure[Peanut lens as mapped to the sphere]{\includegraphics[height=5.2cm]{peanutlens}}
%	\caption{Peanut Lens in 2D and on the sphere}\label{fig:peanutlens}
%\end{figure}

\begin{figure}
	\subfigure[Headlight intensity $f_1$ to constant intensity $f_2$]{\includegraphics[width=0.45\textwidth]{peanutlensfandg}}
	%\subfigure[Numerical solution $u^h$]{\includegraphics[height=5.7cm]{peanutsolution}}
		\subfigure[Computed reflector]{\includegraphics[width=0.45\textwidth]{peanutreflector2}}
	\subfigure[Inverse ray-traced intensity]{\includegraphics[width=0.45\textwidth]{peanutraytraceinverse}}
	\subfigure[Difference between $f_1$ and inverse ray-traced intensity, with average error of $0.0137$.]{\includegraphics[width=0.45\textwidth]{peanuterror}}
	\caption{``Peanut" reflector}\label{fig:peanutresult}
\end{figure}

\subsection{Discontinuous intensities}
Next, we demonstrate the effectiveness of our method in dealing with discontinuities and complicated densities. In this example, a discontinuous source mass $f_1$ resembling an inverted map of the world is mapped to a constant density $f_2$; see Figure~\ref{fig:globe}. This is a particularly challenging example given the very complicated structure of the discontinuities.  Nevertheless, we achieve a reconstruction that visually agrees with the world map, with an average error of 19\% of the maximum intensity.

\begin{figure}
	\subfigure[Intensities $f_1$ and $f_2$]{\includegraphics[width=0.45\textwidth]{fandgglobe}}
	%\subfigure[Solution $u^h$]{\includegraphics[height=4.7cm]{globe14261}}
	\subfigure[Computed reflector]{\includegraphics[width=0.45\textwidth]{globereflector3}}
	%\subfigure[Globe inverse ray trace showing presence of a few negligible numerical artifacts before rescaling]{\includegraphics[height=4.5cm]{globeafter1}}
	\subfigure[Inverse ray-traced intensity]{\includegraphics[width=0.45\textwidth]{globeraytraceinterp}}
	\subfigure[Difference between $f_1$ and inverse ray-traced intensity, with average $L^1$ error of $0.0206$]{\includegraphics[width=0.45\textwidth]{globeerror}}
	\caption{Discontinuous intensities}\label{fig:globe}
\end{figure}




%\subsection{Smooth example}

%Here we perform computations for the case where the source and target densities are smooth, so the solution to the PDE~\eqref{eq:OTPDE} is known to be \textit{a priori} smooth. The densities chosen are:

%\begin{align*}
%f_1(x, y, z) &= \frac{1-\epsilon}{5.8735} \left( \arccos(x) - \pi/2 \right)^2 + \frac{\epsilon}{4\pi} \\
%f_2(x,y,z) &= \frac{1}{4\pi}
%\end{align*}
%where $\epsilon = 0.6$. The resulting figures are shown in the figure~\ref{fig:smoothresult}:

%\begin{figure}
%	\subfigure[Smooth density $f_1$ to constant density $f_2$]{\includegraphics[height=5.8cm]{smoothfandg}}
%	\subfigure[Numerical solution $u^h$]{\includegraphics[height=5.9cm]{smoothsolution}}
%	\subfigure[Rescaled inverse ray traced density $f_1$]{\includegraphics[height=5.2cm]{smoothraytraceinverse1}}
%	\subfigure[Shape of the reflector]{\includegraphics[height=5.7cm]{smoothreflector2}}
%	\subfigure[$L^{\infty}$ error betweeen $f_1$ and the inverse ray trace]{\includegraphics[height=5.7cm]{smootherror}}
%	\caption{Smooth example computation}\label{fig:smoothresult}
%\end{figure}


%\subsection{Disjoint Support}

%Inspired by the original reflector antenna problem, we compute the case where the source and target masses have ``disjoint" support and are supported in the northern and southern hemispheres, respectively. Of course, our convergence framework and numerical scheme only have convergence guarantees if the target mass is bounded strictly away from zero, so there is a very small amount of mass added to account for this. The density functions used are:

%\begin{equation}
%f_1(x, y, z) =
%\begin{cases}
%\frac{1}{2\pi}, & z\geq 0 \\
%0, & \text{otherwise}
%\end{cases}
%\end{equation}
%and

%\begin{equation}
%f_2(x, y, z) =
%\begin{cases}
%\frac{\epsilon}{4\pi}, & z\geq 0 \\
%\frac{\epsilon-1}{\pi}z + \frac{\epsilon}{4 \pi} & \text{otherwise}
%\end{cases}
%\end{equation}
%where $\epsilon = 0.3$. The resulting figures are shown in the figure~\ref{fig:disjointresult}:

%\begin{figure}
%	\subfigure[Mapping ``disjoint" source and target masses]{\includegraphics[height=5cm]{disjointfandg}}
%	\subfigure[Forward ray trace rescaled to account for small amount of numerical artifacts]{\includegraphics[height=4.7cm]{disjointraytrace2}}
%	\subfigure[$L^{\infty}$ error between $f_2$ and the forward ray trace]{\includegraphics[height=4.1cm]{disjointerror2}}
%	\subfigure[Shape of the reflector]{\includegraphics[height=4.9cm]{disjointreflector2}}
%	\caption{``Disjoint" source and target mass}\label{fig:disjointresult}
%\end{figure}


\subsection{Donut intensities}
To further demonstrate the flexibility of our method, we consider the source and target intensities propagating in a donut shape, with a dark region in the center.  These are given by
\begin{equation}
f_1(x,y,z) =
\begin{cases}
\frac{1}{(4\pi/15)(\sqrt{2} + 2)} \left(-4\sqrt{x^2 + y^2} z^3 + 4(x^2 + y^2)^{3/2}z \right), & \sqrt{2}/2 \geq z\geq 0 \\
0, & \text{otherwise}
\end{cases}
\end{equation}
and
\begin{equation}
f_2(x,y,z) =
\begin{cases}
\frac{1}{(4\pi/15)(\sqrt{2} + 2)} \left(-4\sqrt{x^2 + y^2} z^3 + 4(x^2 + y^2)^{3/2}z \right), & 0 \geq z \geq -\sqrt{2}/2 \\
0, & \text{otherwise}
\end{cases}
\end{equation}

These intensities have very complicated support containing holes, which is particularly challenging numerically.  Indeed, this challenge is inherent in the theory of the optimal transport problem.  We note that the $c$-convexity constraint~\eqref{eq:cconvex} requires the domain $\Omega$ to be $c$-convex in order to guarantee construction of the physically relevant solution of the PDE~\eqref{eq:OTPDE}.  Consequently, PDE based methods that are posed only on the support $\Omega$ of the intensity (rather than being extended into the dark regions) will not be assured of producing the correct reflector.  This issue is handled naturally by our method, which is posed on the entire sphere.  Despite the difficulty of this example, our method performs very well, as evidenced in the results of the ray-tracing.  See Figure~\ref{fig:donutresult}.  Average error is 9\% of the maximal intensity.

\begin{figure}
	\subfigure[Intensities $f_1$ and $f_2$]{\includegraphics[width=0.45\textwidth]{donuts2fandg}}
	%\subfigure[Solution $u^h$]{\includegraphics[height=4.4cm]{donuts2solution}}
	\subfigure[Computed reflector]{\includegraphics[width=0.45\textwidth]{donuts2reflector}}
	\subfigure[Forward ray-traced intensity]{\includegraphics[width=0.45\textwidth]{donuts2raytrace}}
	\subfigure[Difference between $f_2$ and forward ray-traced intensity expressed as a percentage of the maximum of $f_2$. Average $L^1$ error of $0.0304$]{\includegraphics[width=0.45\textwidth]{donuts2error}}
	\caption{``Donut" intensities}\label{fig:donutresult}
\end{figure}


\subsection{Singular reflector}
We conclude with an example of a hemispheric light source (here designated as $f_2$) that is to be reshaped into a geodesic triangle on the sphere (here designated as $f_1$).  We remark that given the complicated (non $c$-convex) support of this target, we are not even guaranteed the existence of a smooth ($C^1$) reflector; see~\cite{Loeper_OTonSphere}. 

The intensities are defined as follows.  We begin by forming a geodesic triangle $T_{\theta} \subset \Sf$ from the three vertices $(t_{0, \theta}, t_{1, \theta}, t_{2, \theta})$, where we define $t_{j, \theta} = \left( \sin \theta \cos (2\pi j/3), \sin \theta \sin (2\pi j/3), \cos \theta \right)$ for $\pi/2\leq \theta <\pi$. The geodesic triangle is formed by the small region enclosed by the three vertices $t_i$, which are connected by geodesics on the sphere. That is, a point $x_0 \in T_{\theta}$ if $x_0$ satisfies the following three inequalities:
\begin{align*}
x_0 \cdot \left( t_{1,\theta} \times t_{2,\theta} \right) &\leq 0 \\
x_0 \cdot \left( t_{2,\theta} \times t_{3,\theta} \right) &\leq 0 \\
x_0 \cdot \left( t_{3,\theta} \times t_{1,\theta} \right) &\leq 0
\end{align*}

Then the triangular intensity is defined by
\begin{equation}
f_1(x,y,z) = 
\begin{cases}
1/A , \ \ \ &(x,y,z) \in T_{\theta} \\
0, &(x,y,z) \notin T_{\theta}
\end{cases}
\end{equation}
where $A$ is the area of the geodesic triangle $T_{\theta}$ and $\theta = 2.1$.

The second intensity is a smoothed version of the identity function on the northern hemisphere:
\begin{equation}
f_2(x,y,z) = 
\begin{cases}
\frac{2\pi  \log \left( \cosh(a) \right)}{a}\tanh(a z), &z \geq 0 \\
0, & z<0
\end{cases}
\end{equation}
where $a=10$.

For ease of implementation, we perform pre-processing to bound both $f_1$ and $f_2$ away from zero.

%\begin{equation}
%f_1(x,y,z) = 
%\begin{cases}
%(1-\epsilon)/A , \ \ \ &(x,y,z) \in T_{\theta} \\
%\epsilon/4\pi, &(x,y,z) \notin T_{\theta}
%\end{cases}
%\end{equation}

%\begin{equation}
%f_2(x,y,z) = 
%\begin{cases}
%\frac{2\pi (1-\epsilon) \log \left( \cosh(a) \right)}{a}\tanh(a z), z \geq 0 \\
%\epsilon/4\pi, & z<0
%\end{cases}
%\end{equation}
%for $\epsilon = 0.3$ and $a = 10$. We get the resulting figures~\ref{fig:triangle}

Results are presented in Figure~\ref{fig:triangle}.  In the computed reflector, and resulting ray-traced intensity, we observe an approximate triangle shape as expected. In this case, there are notable artifacts present near the boundary of the triangle. However, to some extent these are a limitation of the physics rather than of our method.  We remark that there is no reason to expect the reflector we are approximating to be continuously differentiable, so the accuracy of the ray-tracing verification test is itself rather suspect here.  Nevertheless, the absolute error as compared with the ray trace from the approximate conservation of energy equation mostly performs well, with an average error of 16\% of the maximal intensity.

In a challenging problem like this, where the physics itself may not allow for the existence of a reflector with nice properties (from the perspective of manufacturing and outcome), it may also be useful to view our method as a robust way of obtaining a good approximation of the desired reflector.  This could then be used to initialize an end-game method, not based on optimal transport, that would optimize the reflector surface and enforce any desired smoothness. 

\begin{figure}
	\subfigure[Intensities $f_1$ and $f_2$ from below]{\includegraphics[width=0.49\textwidth]{triangle2bottom}}
	\subfigure[Intensities $f_1$ and $f_2$ from side]{\includegraphics[width=0.49\textwidth]{triangle2fandg}}
	\subfigure[Solution $u^h$]{\includegraphics[width=0.45\textwidth]{triangle2solution}}
	\subfigure[Computed reflector]{\includegraphics[width=0.45\textwidth]{triangle2reflector}}
	\subfigure[Forward ray-traced intensity]{\includegraphics[width=0.45\textwidth]{triangle3raytrace}}
	\subfigure[Difference between $f_2$ and forward ray-traced intensity expressed as a percentage of the maximum of $f_2$. Average $L^1$ error of $0.0288$]{\includegraphics[width=0.45\textwidth]{triangle3error}}
	\caption{Singular reflector}\label{fig:triangle}
\end{figure}




\section{Conclusion}\label{sec:conclusion}
We have introduced a new numerical method for solving the reflector antenna design problem.  The method is based on the reformulation of this design problem as an optimal transport problem on the sphere.  This allows the reflector to be described in terms of the solution to a fully nonlinear elliptic PDE of \MA type, posed on the unit sphere.  We describe a provably convergent finite difference method for solving this PDE, which in turn guarantees that the method will correctly approximate the desired reflector.  The method is robust: convergence guarantees hold even for non-smooth data and reflectors.

We validate this new method through several challenging examples, which include intensities that have complicated discontinuities, that propagate over complicated geometries, or that contain a mix of light and dark regions.  The method performs well even in a final example where the physics does not guarantee the existence of a smooth ($C^1$) reflector.

This new finite difference method provides a rigorous foundation upon which faster and more accurate solvers can be designed.  The idea of pairing slower, more robust approximations (to be used in the most singular regions of the domain) with more traditional high-order methods has been successfully applied to the \MA equation in Euclidean space~\cite{FO_FilteredSchemes}.  In the future, we hope to adapt these techniques to the reflector antenna problem in order to produce higher-quality approximations to the desired reflector surface.



\bibliographystyle{plain}
\bibliography{OTonSphere3}

%{\appendix
%\chapter{Supplementary Material}
\label{appendix}

In this appendix, we present supplementary material for the techniques and
experiments presented in the main text.

\section{Baseline Results and Analysis for Informed Sampler}
\label{appendix:chap3}

Here, we give an in-depth
performance analysis of the various samplers and the effect of their
hyperparameters. We choose hyperparameters with the lowest PSRF value
after $10k$ iterations, for each sampler individually. If the
differences between PSRF are not significantly different among
multiple values, we choose the one that has the highest acceptance
rate.

\subsection{Experiment: Estimating Camera Extrinsics}
\label{appendix:chap3:room}

\subsubsection{Parameter Selection}
\paragraph{Metropolis Hastings (\MH)}

Figure~\ref{fig:exp1_MH} shows the median acceptance rates and PSRF
values corresponding to various proposal standard deviations of plain
\MH~sampling. Mixing gets better and the acceptance rate gets worse as
the standard deviation increases. The value $0.3$ is selected standard
deviation for this sampler.

\paragraph{Metropolis Hastings Within Gibbs (\MHWG)}

As mentioned in Section~\ref{sec:room}, the \MHWG~sampler with one-dimensional
updates did not converge for any value of proposal standard deviation.
This problem has high correlation of the camera parameters and is of
multi-modal nature, which this sampler has problems with.

\paragraph{Parallel Tempering (\PT)}

For \PT~sampling, we took the best performing \MH~sampler and used
different temperature chains to improve the mixing of the
sampler. Figure~\ref{fig:exp1_PT} shows the results corresponding to
different combination of temperature levels. The sampler with
temperature levels of $[1,3,27]$ performed best in terms of both
mixing and acceptance rate.

\paragraph{Effect of Mixture Coefficient in Informed Sampling (\MIXLMH)}

Figure~\ref{fig:exp1_alpha} shows the effect of mixture
coefficient ($\alpha$) on the informed sampling
\MIXLMH. Since there is no significant different in PSRF values for
$0 \le \alpha \le 0.7$, we chose $0.7$ due to its high acceptance
rate.


% \end{multicols}

\begin{figure}[h]
\centering
  \subfigure[MH]{%
    \includegraphics[width=.48\textwidth]{figures/supplementary/camPose_MH.pdf} \label{fig:exp1_MH}
  }
  \subfigure[PT]{%
    \includegraphics[width=.48\textwidth]{figures/supplementary/camPose_PT.pdf} \label{fig:exp1_PT}
  }
\\
  \subfigure[INF-MH]{%
    \includegraphics[width=.48\textwidth]{figures/supplementary/camPose_alpha.pdf} \label{fig:exp1_alpha}
  }
  \mycaption{Results of the `Estimating Camera Extrinsics' experiment}{PRSFs and Acceptance rates corresponding to (a) various standard deviations of \MH, (b) various temperature level combinations of \PT sampling and (c) various mixture coefficients of \MIXLMH sampling.}
\end{figure}



\begin{figure}[!t]
\centering
  \subfigure[\MH]{%
    \includegraphics[width=.48\textwidth]{figures/supplementary/occlusionExp_MH.pdf} \label{fig:exp2_MH}
  }
  \subfigure[\BMHWG]{%
    \includegraphics[width=.48\textwidth]{figures/supplementary/occlusionExp_BMHWG.pdf} \label{fig:exp2_BMHWG}
  }
\\
  \subfigure[\MHWG]{%
    \includegraphics[width=.48\textwidth]{figures/supplementary/occlusionExp_MHWG.pdf} \label{fig:exp2_MHWG}
  }
  \subfigure[\PT]{%
    \includegraphics[width=.48\textwidth]{figures/supplementary/occlusionExp_PT.pdf} \label{fig:exp2_PT}
  }
\\
  \subfigure[\INFBMHWG]{%
    \includegraphics[width=.5\textwidth]{figures/supplementary/occlusionExp_alpha.pdf} \label{fig:exp2_alpha}
  }
  \mycaption{Results of the `Occluding Tiles' experiment}{PRSF and
    Acceptance rates corresponding to various standard deviations of
    (a) \MH, (b) \BMHWG, (c) \MHWG, (d) various temperature level
    combinations of \PT~sampling and; (e) various mixture coefficients
    of our informed \INFBMHWG sampling.}
\end{figure}

%\onecolumn\newpage\twocolumn
\subsection{Experiment: Occluding Tiles}
\label{appendix:chap3:tiles}

\subsubsection{Parameter Selection}

\paragraph{Metropolis Hastings (\MH)}

Figure~\ref{fig:exp2_MH} shows the results of
\MH~sampling. Results show the poor convergence for all proposal
standard deviations and rapid decrease of AR with increasing standard
deviation. This is due to the high-dimensional nature of
the problem. We selected a standard deviation of $1.1$.

\paragraph{Blocked Metropolis Hastings Within Gibbs (\BMHWG)}

The results of \BMHWG are shown in Figure~\ref{fig:exp2_BMHWG}. In
this sampler we update only one block of tile variables (of dimension
four) in each sampling step. Results show much better performance
compared to plain \MH. The optimal proposal standard deviation for
this sampler is $0.7$.

\paragraph{Metropolis Hastings Within Gibbs (\MHWG)}

Figure~\ref{fig:exp2_MHWG} shows the result of \MHWG sampling. This
sampler is better than \BMHWG and converges much more quickly. Here
a standard deviation of $0.9$ is found to be best.

\paragraph{Parallel Tempering (\PT)}

Figure~\ref{fig:exp2_PT} shows the results of \PT sampling with various
temperature combinations. Results show no improvement in AR from plain
\MH sampling and again $[1,3,27]$ temperature levels are found to be optimal.

\paragraph{Effect of Mixture Coefficient in Informed Sampling (\INFBMHWG)}

Figure~\ref{fig:exp2_alpha} shows the effect of mixture
coefficient ($\alpha$) on the blocked informed sampling
\INFBMHWG. Since there is no significant different in PSRF values for
$0 \le \alpha \le 0.8$, we chose $0.8$ due to its high acceptance
rate.



\subsection{Experiment: Estimating Body Shape}
\label{appendix:chap3:body}

\subsubsection{Parameter Selection}
\paragraph{Metropolis Hastings (\MH)}

Figure~\ref{fig:exp3_MH} shows the result of \MH~sampling with various
proposal standard deviations. The value of $0.1$ is found to be
best.

\paragraph{Metropolis Hastings Within Gibbs (\MHWG)}

For \MHWG sampling we select $0.3$ proposal standard
deviation. Results are shown in Fig.~\ref{fig:exp3_MHWG}.


\paragraph{Parallel Tempering (\PT)}

As before, results in Fig.~\ref{fig:exp3_PT}, the temperature levels
were selected to be $[1,3,27]$ due its slightly higher AR.

\paragraph{Effect of Mixture Coefficient in Informed Sampling (\MIXLMH)}

Figure~\ref{fig:exp3_alpha} shows the effect of $\alpha$ on PSRF and
AR. Since there is no significant differences in PSRF values for $0 \le
\alpha \le 0.8$, we choose $0.8$.


\begin{figure}[t]
\centering
  \subfigure[\MH]{%
    \includegraphics[width=.48\textwidth]{figures/supplementary/bodyShape_MH.pdf} \label{fig:exp3_MH}
  }
  \subfigure[\MHWG]{%
    \includegraphics[width=.48\textwidth]{figures/supplementary/bodyShape_MHWG.pdf} \label{fig:exp3_MHWG}
  }
\\
  \subfigure[\PT]{%
    \includegraphics[width=.48\textwidth]{figures/supplementary/bodyShape_PT.pdf} \label{fig:exp3_PT}
  }
  \subfigure[\MIXLMH]{%
    \includegraphics[width=.48\textwidth]{figures/supplementary/bodyShape_alpha.pdf} \label{fig:exp3_alpha}
  }
\\
  \mycaption{Results of the `Body Shape Estimation' experiment}{PRSFs and
    Acceptance rates corresponding to various standard deviations of
    (a) \MH, (b) \MHWG; (c) various temperature level combinations
    of \PT sampling and; (d) various mixture coefficients of the
    informed \MIXLMH sampling.}
\end{figure}


\subsection{Results Overview}
Figure~\ref{fig:exp_summary} shows the summary results of the all the three
experimental studies related to informed sampler.
\begin{figure*}[h!]
\centering
  \subfigure[Results for: Estimating Camera Extrinsics]{%
    \includegraphics[width=0.9\textwidth]{figures/supplementary/camPose_ALL.pdf} \label{fig:exp1_all}
  }
  \subfigure[Results for: Occluding Tiles]{%
    \includegraphics[width=0.9\textwidth]{figures/supplementary/occlusionExp_ALL.pdf} \label{fig:exp2_all}
  }
  \subfigure[Results for: Estimating Body Shape]{%
    \includegraphics[width=0.9\textwidth]{figures/supplementary/bodyShape_ALL.pdf} \label{fig:exp3_all}
  }
  \label{fig:exp_summary}
  \mycaption{Summary of the statistics for the three experiments}{Shown are
    for several baseline methods and the informed samplers the
    acceptance rates (left), PSRFs (middle), and RMSE values
    (right). All results are median results over multiple test
    examples.}
\end{figure*}

\subsection{Additional Qualitative Results}

\subsubsection{Occluding Tiles}
In Figure~\ref{fig:exp2_visual_more} more qualitative results of the
occluding tiles experiment are shown. The informed sampling approach
(\INFBMHWG) is better than the best baseline (\MHWG). This still is a
very challenging problem since the parameters for occluded tiles are
flat over a large region. Some of the posterior variance of the
occluded tiles is already captured by the informed sampler.

\begin{figure*}[h!]
\begin{center}
\centerline{\includegraphics[width=0.95\textwidth]{figures/supplementary/occlusionExp_Visual.pdf}}
\mycaption{Additional qualitative results of the occluding tiles experiment}
  {From left to right: (a)
  Given image, (b) Ground truth tiles, (c) OpenCV heuristic and most probable estimates
  from 5000 samples obtained by (d) MHWG sampler (best baseline) and
  (e) our INF-BMHWG sampler. (f) Posterior expectation of the tiles
  boundaries obtained by INF-BMHWG sampling (First 2000 samples are
  discarded as burn-in).}
\label{fig:exp2_visual_more}
\end{center}
\end{figure*}

\subsubsection{Body Shape}
Figure~\ref{fig:exp3_bodyMeshes} shows some more results of 3D mesh
reconstruction using posterior samples obtained by our informed
sampling \MIXLMH.

\begin{figure*}[t]
\begin{center}
\centerline{\includegraphics[width=0.75\textwidth]{figures/supplementary/bodyMeshResults.pdf}}
\mycaption{Qualitative results for the body shape experiment}
  {Shown is the 3D mesh reconstruction results with first 1000 samples obtained
  using the \MIXLMH informed sampling method. (blue indicates small
  values and red indicates high values)}
\label{fig:exp3_bodyMeshes}
\end{center}
\end{figure*}

\clearpage



\section{Additional Results on the Face Problem with CMP}

Figure~\ref{fig:shading-qualitative-multiple-subjects-supp} shows inference results for reflectance maps, normal maps and lights for randomly chosen test images, and Fig.~\ref{fig:shading-qualitative-same-subject-supp} shows reflectance estimation results on multiple images of the same subject produced under different illumination conditions. CMP is able to produce estimates that are closer to the groundtruth across different subjects and illumination conditions.

\begin{figure*}[h]
  \begin{center}
  \centerline{\includegraphics[width=1.0\columnwidth]{figures/face_cmp_visual_results_supp.pdf}}
  \vspace{-1.2cm}
  \end{center}
	\mycaption{A visual comparison of inference results}{(a)~Observed images. (b)~Inferred reflectance maps. \textit{GT} is the photometric stereo groundtruth, \textit{BU} is the Biswas \etal (2009) reflectance estimate and \textit{Forest} is the consensus prediction. (c)~The variance of the inferred reflectance estimate produced by \MTD (normalized across rows).(d)~Visualization of inferred light directions. (e)~Inferred normal maps.}
	\label{fig:shading-qualitative-multiple-subjects-supp}
\end{figure*}


\begin{figure*}[h]
	\centering
	\setlength\fboxsep{0.2mm}
	\setlength\fboxrule{0pt}
	\begin{tikzpicture}

		\matrix at (0, 0) [matrix of nodes, nodes={anchor=east}, column sep=-0.05cm, row sep=-0.2cm]
		{
			\fbox{\includegraphics[width=1cm]{figures/sample_3_4_X.png}} &
			\fbox{\includegraphics[width=1cm]{figures/sample_3_4_GT.png}} &
			\fbox{\includegraphics[width=1cm]{figures/sample_3_4_BISWAS.png}}  &
			\fbox{\includegraphics[width=1cm]{figures/sample_3_4_VMP.png}}  &
			\fbox{\includegraphics[width=1cm]{figures/sample_3_4_FOREST.png}}  &
			\fbox{\includegraphics[width=1cm]{figures/sample_3_4_CMP.png}}  &
			\fbox{\includegraphics[width=1cm]{figures/sample_3_4_CMPVAR.png}}
			 \\

			\fbox{\includegraphics[width=1cm]{figures/sample_3_5_X.png}} &
			\fbox{\includegraphics[width=1cm]{figures/sample_3_5_GT.png}} &
			\fbox{\includegraphics[width=1cm]{figures/sample_3_5_BISWAS.png}}  &
			\fbox{\includegraphics[width=1cm]{figures/sample_3_5_VMP.png}}  &
			\fbox{\includegraphics[width=1cm]{figures/sample_3_5_FOREST.png}}  &
			\fbox{\includegraphics[width=1cm]{figures/sample_3_5_CMP.png}}  &
			\fbox{\includegraphics[width=1cm]{figures/sample_3_5_CMPVAR.png}}
			 \\

			\fbox{\includegraphics[width=1cm]{figures/sample_3_6_X.png}} &
			\fbox{\includegraphics[width=1cm]{figures/sample_3_6_GT.png}} &
			\fbox{\includegraphics[width=1cm]{figures/sample_3_6_BISWAS.png}}  &
			\fbox{\includegraphics[width=1cm]{figures/sample_3_6_VMP.png}}  &
			\fbox{\includegraphics[width=1cm]{figures/sample_3_6_FOREST.png}}  &
			\fbox{\includegraphics[width=1cm]{figures/sample_3_6_CMP.png}}  &
			\fbox{\includegraphics[width=1cm]{figures/sample_3_6_CMPVAR.png}}
			 \\
	     };

       \node at (-3.85, -2.0) {\small Observed};
       \node at (-2.55, -2.0) {\small `GT'};
       \node at (-1.27, -2.0) {\small BU};
       \node at (0.0, -2.0) {\small MP};
       \node at (1.27, -2.0) {\small Forest};
       \node at (2.55, -2.0) {\small \textbf{CMP}};
       \node at (3.85, -2.0) {\small Variance};

	\end{tikzpicture}
	\mycaption{Robustness to varying illumination}{Reflectance estimation on a subject images with varying illumination. Left to right: observed image, photometric stereo estimate (GT)
  which is used as a proxy for groundtruth, bottom-up estimate of \cite{Biswas2009}, VMP result, consensus forest estimate, CMP mean, and CMP variance.}
	\label{fig:shading-qualitative-same-subject-supp}
\end{figure*}

\clearpage

\section{Additional Material for Learning Sparse High Dimensional Filters}
\label{sec:appendix-bnn}

This part of supplementary material contains a more detailed overview of the permutohedral
lattice convolution in Section~\ref{sec:permconv}, more experiments in
Section~\ref{sec:addexps} and additional results with protocols for
the experiments presented in Chapter~\ref{chap:bnn} in Section~\ref{sec:addresults}.

\vspace{-0.2cm}
\subsection{General Permutohedral Convolutions}
\label{sec:permconv}

A core technical contribution of this work is the generalization of the Gaussian permutohedral lattice
convolution proposed in~\cite{adams2010fast} to the full non-separable case with the
ability to perform back-propagation. Although, conceptually, there are minor
differences between Gaussian and general parameterized filters, there are non-trivial practical
differences in terms of the algorithmic implementation. The Gauss filters belong to
the separable class and can thus be decomposed into multiple
sequential one dimensional convolutions. We are interested in the general filter
convolutions, which can not be decomposed. Thus, performing a general permutohedral
convolution at a lattice point requires the computation of the inner product with the
neighboring elements in all the directions in the high-dimensional space.

Here, we give more details of the implementation differences of separable
and non-separable filters. In the following, we will explain the scalar case first.
Recall, that the forward pass of general permutohedral convolution
involves 3 steps: \textit{splatting}, \textit{convolving} and \textit{slicing}.
We follow the same splatting and slicing strategies as in~\cite{adams2010fast}
since these operations do not depend on the filter kernel. The main difference
between our work and the existing implementation of~\cite{adams2010fast} is
the way that the convolution operation is executed. This proceeds by constructing
a \emph{blur neighbor} matrix $K$ that stores for every lattice point all
values of the lattice neighbors that are needed to compute the filter output.

\begin{figure}[t!]
  \centering
    \includegraphics[width=0.6\columnwidth]{figures/supplementary/lattice_construction}
  \mycaption{Illustration of 1D permutohedral lattice construction}
  {A $4\times 4$ $(x,y)$ grid lattice is projected onto the plane defined by the normal
  vector $(1,1)^{\top}$. This grid has $s+1=4$ and $d=2$ $(s+1)^{d}=4^2=16$ elements.
  In the projection, all points of the same color are projected onto the same points in the plane.
  The number of elements of the projected lattice is $t=(s+1)^d-s^d=4^2-3^2=7$, that is
  the $(4\times 4)$ grid minus the size of lattice that is $1$ smaller at each size, in this
  case a $(3\times 3)$ lattice (the upper right $(3\times 3)$ elements).
  }
\label{fig:latticeconstruction}
\end{figure}

The blur neighbor matrix is constructed by traversing through all the populated
lattice points and their neighboring elements.
% For efficiency, we do this matrix construction recursively with shared computations
% since $n^{th}$ neighbourhood elements are $1^{st}$ neighborhood elements of $n-1^{th}$ neighbourhood elements. \pg{do not understand}
This is done recursively to share computations. For any lattice point, the neighbors that are
$n$ hops away are the direct neighbors of the points that are $n-1$ hops away.
The size of a $d$ dimensional spatial filter with width $s+1$ is $(s+1)^{d}$ (\eg, a
$3\times 3$ filter, $s=2$ in $d=2$ has $3^2=9$ elements) and this size grows
exponentially in the number of dimensions $d$. The permutohedral lattice is constructed by
projecting a regular grid onto the plane spanned by the $d$ dimensional normal vector ${(1,\ldots,1)}^{\top}$. See
Fig.~\ref{fig:latticeconstruction} for an illustration of the 1D lattice construction.
Many corners of a grid filter are projected onto the same point, in total $t = {(s+1)}^{d} -
s^{d}$ elements remain in the permutohedral filter with $s$ neighborhood in $d-1$ dimensions.
If the lattice has $m$ populated elements, the
matrix $K$ has size $t\times m$. Note that, since the input signal is typically
sparse, only a few lattice corners are being populated in the \textit{slicing} step.
We use a hash-table to keep track of these points and traverse only through
the populated lattice points for this neighborhood matrix construction.

Once the blur neighbor matrix $K$ is constructed, we can perform the convolution
by the matrix vector multiplication
\begin{equation}
\ell' = BK,
\label{eq:conv}
\end{equation}
where $B$ is the $1 \times t$ filter kernel (whose values we will learn) and $\ell'\in\mathbb{R}^{1\times m}$
is the result of the filtering at the $m$ lattice points. In practice, we found that the
matrix $K$ is sometimes too large to fit into GPU memory and we divided the matrix $K$
into smaller pieces to compute Eq.~\ref{eq:conv} sequentially.

In the general multi-dimensional case, the signal $\ell$ is of $c$ dimensions. Then
the kernel $B$ is of size $c \times t$ and $K$ stores the $c$ dimensional vectors
accordingly. When the input and output points are different, we slice only the
input points and splat only at the output points.


\subsection{Additional Experiments}
\label{sec:addexps}
In this section, we discuss more use-cases for the learned bilateral filters, one
use-case of BNNs and two single filter applications for image and 3D mesh denoising.

\subsubsection{Recognition of subsampled MNIST}\label{sec:app_mnist}

One of the strengths of the proposed filter convolution is that it does not
require the input to lie on a regular grid. The only requirement is to define a distance
between features of the input signal.
We highlight this feature with the following experiment using the
classical MNIST ten class classification problem~\cite{lecun1998mnist}. We sample a
sparse set of $N$ points $(x,y)\in [0,1]\times [0,1]$
uniformly at random in the input image, use their interpolated values
as signal and the \emph{continuous} $(x,y)$ positions as features. This mimics
sub-sampling of a high-dimensional signal. To compare against a spatial convolution,
we interpolate the sparse set of values at the grid positions.

We take a reference implementation of LeNet~\cite{lecun1998gradient} that
is part of the Caffe project~\cite{jia2014caffe} and compare it
against the same architecture but replacing the first convolutional
layer with a bilateral convolution layer (BCL). The filter size
and numbers are adjusted to get a comparable number of parameters
($5\times 5$ for LeNet, $2$-neighborhood for BCL).

The results are shown in Table~\ref{tab:all-results}. We see that training
on the original MNIST data (column Original, LeNet vs. BNN) leads to a slight
decrease in performance of the BNN (99.03\%) compared to LeNet
(99.19\%). The BNN can be trained and evaluated on sparse
signals, and we resample the image as described above for $N=$ 100\%, 60\% and
20\% of the total number of pixels. The methods are also evaluated
on test images that are subsampled in the same way. Note that we can
train and test with different subsampling rates. We introduce an additional
bilinear interpolation layer for the LeNet architecture to train on the same
data. In essence, both models perform a spatial interpolation and thus we
expect them to yield a similar classification accuracy. Once the data is of
higher dimensions, the permutohedral convolution will be faster due to hashing
the sparse input points, as well as less memory demanding in comparison to
naive application of a spatial convolution with interpolated values.

\begin{table}[t]
  \begin{center}
    \footnotesize
    \centering
    \begin{tabular}[t]{lllll}
      \toprule
              &     & \multicolumn{3}{c}{Test Subsampling} \\
       Method  & Original & 100\% & 60\% & 20\%\\
      \midrule
       LeNet &  \textbf{0.9919} & 0.9660 & 0.9348 & \textbf{0.6434} \\
       BNN &  0.9903 & \textbf{0.9844} & \textbf{0.9534} & 0.5767 \\
      \hline
       LeNet 100\% & 0.9856 & 0.9809 & 0.9678 & \textbf{0.7386} \\
       BNN 100\% & \textbf{0.9900} & \textbf{0.9863} & \textbf{0.9699} & 0.6910 \\
      \hline
       LeNet 60\% & 0.9848 & 0.9821 & 0.9740 & 0.8151 \\
       BNN 60\% & \textbf{0.9885} & \textbf{0.9864} & \textbf{0.9771} & \textbf{0.8214}\\
      \hline
       LeNet 20\% & \textbf{0.9763} & \textbf{0.9754} & 0.9695 & 0.8928 \\
       BNN 20\% & 0.9728 & 0.9735 & \textbf{0.9701} & \textbf{0.9042}\\
      \bottomrule
    \end{tabular}
  \end{center}
\vspace{-.2cm}
\caption{Classification accuracy on MNIST. We compare the
    LeNet~\cite{lecun1998gradient} implementation that is part of
    Caffe~\cite{jia2014caffe} to the network with the first layer
    replaced by a bilateral convolution layer (BCL). Both are trained
    on the original image resolution (first two rows). Three more BNN
    and CNN models are trained with randomly subsampled images (100\%,
    60\% and 20\% of the pixels). An additional bilinear interpolation
    layer samples the input signal on a spatial grid for the CNN model.
  }
  \label{tab:all-results}
\vspace{-.5cm}
\end{table}

\subsubsection{Image Denoising}

The main application that inspired the development of the bilateral
filtering operation is image denoising~\cite{aurich1995non}, there
using a single Gaussian kernel. Our development allows to learn this
kernel function from data and we explore how to improve using a \emph{single}
but more general bilateral filter.

We use the Berkeley segmentation dataset
(BSDS500)~\cite{arbelaezi2011bsds500} as a test bed. The color
images in the dataset are converted to gray-scale,
and corrupted with Gaussian noise with a standard deviation of
$\frac {25} {255}$.

We compare the performance of four different filter models on a
denoising task.
The first baseline model (`Spatial' in Table \ref{tab:denoising}, $25$
weights) uses a single spatial filter with a kernel size of
$5$ and predicts the scalar gray-scale value at the center pixel. The next model
(`Gauss Bilateral') applies a bilateral \emph{Gaussian}
filter to the noisy input, using position and intensity features $\f=(x,y,v)^\top$.
The third setup (`Learned Bilateral', $65$ weights)
takes a Gauss kernel as initialization and
fits all filter weights on the train set to minimize the
mean squared error with respect to the clean images.
We run a combination
of spatial and permutohedral convolutions on spatial and bilateral
features (`Spatial + Bilateral (Learned)') to check for a complementary
performance of the two convolutions.

\label{sec:exp:denoising}
\begin{table}[!h]
\begin{center}
  \footnotesize
  \begin{tabular}[t]{lr}
    \toprule
    Method & PSNR \\
    \midrule
    Noisy Input & $20.17$ \\
    Spatial & $26.27$ \\
    Gauss Bilateral & $26.51$ \\
    Learned Bilateral & $26.58$ \\
    Spatial + Bilateral (Learned) & \textbf{$26.65$} \\
    \bottomrule
  \end{tabular}
\end{center}
\vspace{-0.5em}
\caption{PSNR results of a denoising task using the BSDS500
  dataset~\cite{arbelaezi2011bsds500}}
\vspace{-0.5em}
\label{tab:denoising}
\end{table}
\vspace{-0.2em}

The PSNR scores evaluated on full images of the test set are
shown in Table \ref{tab:denoising}. We find that an untrained bilateral
filter already performs better than a trained spatial convolution
($26.27$ to $26.51$). A learned convolution further improve the
performance slightly. We chose this simple one-kernel setup to
validate an advantage of the generalized bilateral filter. A competitive
denoising system would employ RGB color information and also
needs to be properly adjusted in network size. Multi-layer perceptrons
have obtained state-of-the-art denoising results~\cite{burger12cvpr}
and the permutohedral lattice layer can readily be used in such an
architecture, which is intended future work.

\subsection{Additional results}
\label{sec:addresults}

This section contains more qualitative results for the experiments presented in Chapter~\ref{chap:bnn}.

\begin{figure*}[th!]
  \centering
    \includegraphics[width=\columnwidth,trim={5cm 2.5cm 5cm 4.5cm},clip]{figures/supplementary/lattice_viz.pdf}
    \vspace{-0.7cm}
  \mycaption{Visualization of the Permutohedral Lattice}
  {Sample lattice visualizations for different feature spaces. All pixels falling in the same simplex cell are shown with
  the same color. $(x,y)$ features correspond to image pixel positions, and $(r,g,b) \in [0,255]$ correspond
  to the red, green and blue color values.}
\label{fig:latticeviz}
\end{figure*}

\subsubsection{Lattice Visualization}

Figure~\ref{fig:latticeviz} shows sample lattice visualizations for different feature spaces.

\newcolumntype{L}[1]{>{\raggedright\let\newline\\\arraybackslash\hspace{0pt}}b{#1}}
\newcolumntype{C}[1]{>{\centering\let\newline\\\arraybackslash\hspace{0pt}}b{#1}}
\newcolumntype{R}[1]{>{\raggedleft\let\newline\\\arraybackslash\hspace{0pt}}b{#1}}

\subsubsection{Color Upsampling}\label{sec:color_upsampling}
\label{sec:col_upsample_extra}

Some images of the upsampling for the Pascal
VOC12 dataset are shown in Fig.~\ref{fig:Colour_upsample_visuals}. It is
especially the low level image details that are better preserved with
a learned bilateral filter compared to the Gaussian case.

\begin{figure*}[t!]
  \centering
    \subfigure{%
   \raisebox{2.0em}{
    \includegraphics[width=.06\columnwidth]{figures/supplementary/2007_004969.jpg}
   }
  }
  \subfigure{%
    \includegraphics[width=.17\columnwidth]{figures/supplementary/2007_004969_gray.pdf}
  }
  \subfigure{%
    \includegraphics[width=.17\columnwidth]{figures/supplementary/2007_004969_gt.pdf}
  }
  \subfigure{%
    \includegraphics[width=.17\columnwidth]{figures/supplementary/2007_004969_bicubic.pdf}
  }
  \subfigure{%
    \includegraphics[width=.17\columnwidth]{figures/supplementary/2007_004969_gauss.pdf}
  }
  \subfigure{%
    \includegraphics[width=.17\columnwidth]{figures/supplementary/2007_004969_learnt.pdf}
  }\\
    \subfigure{%
   \raisebox{2.0em}{
    \includegraphics[width=.06\columnwidth]{figures/supplementary/2007_003106.jpg}
   }
  }
  \subfigure{%
    \includegraphics[width=.17\columnwidth]{figures/supplementary/2007_003106_gray.pdf}
  }
  \subfigure{%
    \includegraphics[width=.17\columnwidth]{figures/supplementary/2007_003106_gt.pdf}
  }
  \subfigure{%
    \includegraphics[width=.17\columnwidth]{figures/supplementary/2007_003106_bicubic.pdf}
  }
  \subfigure{%
    \includegraphics[width=.17\columnwidth]{figures/supplementary/2007_003106_gauss.pdf}
  }
  \subfigure{%
    \includegraphics[width=.17\columnwidth]{figures/supplementary/2007_003106_learnt.pdf}
  }\\
  \setcounter{subfigure}{0}
  \small{
  \subfigure[Inp.]{%
  \raisebox{2.0em}{
    \includegraphics[width=.06\columnwidth]{figures/supplementary/2007_006837.jpg}
   }
  }
  \subfigure[Guidance]{%
    \includegraphics[width=.17\columnwidth]{figures/supplementary/2007_006837_gray.pdf}
  }
   \subfigure[GT]{%
    \includegraphics[width=.17\columnwidth]{figures/supplementary/2007_006837_gt.pdf}
  }
  \subfigure[Bicubic]{%
    \includegraphics[width=.17\columnwidth]{figures/supplementary/2007_006837_bicubic.pdf}
  }
  \subfigure[Gauss-BF]{%
    \includegraphics[width=.17\columnwidth]{figures/supplementary/2007_006837_gauss.pdf}
  }
  \subfigure[Learned-BF]{%
    \includegraphics[width=.17\columnwidth]{figures/supplementary/2007_006837_learnt.pdf}
  }
  }
  \vspace{-0.5cm}
  \mycaption{Color Upsampling}{Color $8\times$ upsampling results
  using different methods, from left to right, (a)~Low-resolution input color image (Inp.),
  (b)~Gray scale guidance image, (c)~Ground-truth color image; Upsampled color images with
  (d)~Bicubic interpolation, (e) Gauss bilateral upsampling and, (f)~Learned bilateral
  updampgling (best viewed on screen).}

\label{fig:Colour_upsample_visuals}
\end{figure*}

\subsubsection{Depth Upsampling}
\label{sec:depth_upsample_extra}

Figure~\ref{fig:depth_upsample_visuals} presents some more qualitative results comparing bicubic interpolation, Gauss
bilateral and learned bilateral upsampling on NYU depth dataset image~\cite{silberman2012indoor}.

\subsubsection{Character Recognition}\label{sec:app_character}

 Figure~\ref{fig:nnrecognition} shows the schematic of different layers
 of the network architecture for LeNet-7~\cite{lecun1998mnist}
 and DeepCNet(5, 50)~\cite{ciresan2012multi,graham2014spatially}. For the BNN variants, the first layer filters are replaced
 with learned bilateral filters and are learned end-to-end.

\subsubsection{Semantic Segmentation}\label{sec:app_semantic_segmentation}
\label{sec:semantic_bnn_extra}

Some more visual results for semantic segmentation are shown in Figure~\ref{fig:semantic_visuals}.
These include the underlying DeepLab CNN\cite{chen2014semantic} result (DeepLab),
the 2 step mean-field result with Gaussian edge potentials (+2stepMF-GaussCRF)
and also corresponding results with learned edge potentials (+2stepMF-LearnedCRF).
In general, we observe that mean-field in learned CRF leads to slightly dilated
classification regions in comparison to using Gaussian CRF thereby filling-in the
false negative pixels and also correcting some mis-classified regions.

\begin{figure*}[t!]
  \centering
    \subfigure{%
   \raisebox{2.0em}{
    \includegraphics[width=.06\columnwidth]{figures/supplementary/2bicubic}
   }
  }
  \subfigure{%
    \includegraphics[width=.17\columnwidth]{figures/supplementary/2given_image}
  }
  \subfigure{%
    \includegraphics[width=.17\columnwidth]{figures/supplementary/2ground_truth}
  }
  \subfigure{%
    \includegraphics[width=.17\columnwidth]{figures/supplementary/2bicubic}
  }
  \subfigure{%
    \includegraphics[width=.17\columnwidth]{figures/supplementary/2gauss}
  }
  \subfigure{%
    \includegraphics[width=.17\columnwidth]{figures/supplementary/2learnt}
  }\\
    \subfigure{%
   \raisebox{2.0em}{
    \includegraphics[width=.06\columnwidth]{figures/supplementary/32bicubic}
   }
  }
  \subfigure{%
    \includegraphics[width=.17\columnwidth]{figures/supplementary/32given_image}
  }
  \subfigure{%
    \includegraphics[width=.17\columnwidth]{figures/supplementary/32ground_truth}
  }
  \subfigure{%
    \includegraphics[width=.17\columnwidth]{figures/supplementary/32bicubic}
  }
  \subfigure{%
    \includegraphics[width=.17\columnwidth]{figures/supplementary/32gauss}
  }
  \subfigure{%
    \includegraphics[width=.17\columnwidth]{figures/supplementary/32learnt}
  }\\
  \setcounter{subfigure}{0}
  \small{
  \subfigure[Inp.]{%
  \raisebox{2.0em}{
    \includegraphics[width=.06\columnwidth]{figures/supplementary/41bicubic}
   }
  }
  \subfigure[Guidance]{%
    \includegraphics[width=.17\columnwidth]{figures/supplementary/41given_image}
  }
   \subfigure[GT]{%
    \includegraphics[width=.17\columnwidth]{figures/supplementary/41ground_truth}
  }
  \subfigure[Bicubic]{%
    \includegraphics[width=.17\columnwidth]{figures/supplementary/41bicubic}
  }
  \subfigure[Gauss-BF]{%
    \includegraphics[width=.17\columnwidth]{figures/supplementary/41gauss}
  }
  \subfigure[Learned-BF]{%
    \includegraphics[width=.17\columnwidth]{figures/supplementary/41learnt}
  }
  }
  \mycaption{Depth Upsampling}{Depth $8\times$ upsampling results
  using different upsampling strategies, from left to right,
  (a)~Low-resolution input depth image (Inp.),
  (b)~High-resolution guidance image, (c)~Ground-truth depth; Upsampled depth images with
  (d)~Bicubic interpolation, (e) Gauss bilateral upsampling and, (f)~Learned bilateral
  updampgling (best viewed on screen).}

\label{fig:depth_upsample_visuals}
\end{figure*}

\subsubsection{Material Segmentation}\label{sec:app_material_segmentation}
\label{sec:material_bnn_extra}

In Fig.~\ref{fig:material_visuals-app2}, we present visual results comparing 2 step
mean-field inference with Gaussian and learned pairwise CRF potentials. In
general, we observe that the pixels belonging to dominant classes in the
training data are being more accurately classified with learned CRF. This leads to
a significant improvements in overall pixel accuracy. This also results
in a slight decrease of the accuracy from less frequent class pixels thereby
slightly reducing the average class accuracy with learning. We attribute this
to the type of annotation that is available for this dataset, which is not
for the entire image but for some segments in the image. We have very few
images of the infrequent classes to combat this behaviour during training.

\subsubsection{Experiment Protocols}
\label{sec:protocols}

Table~\ref{tbl:parameters} shows experiment protocols of different experiments.

 \begin{figure*}[t!]
  \centering
  \subfigure[LeNet-7]{
    \includegraphics[width=0.7\columnwidth]{figures/supplementary/lenet_cnn_network}
    }\\
    \subfigure[DeepCNet]{
    \includegraphics[width=\columnwidth]{figures/supplementary/deepcnet_cnn_network}
    }
  \mycaption{CNNs for Character Recognition}
  {Schematic of (top) LeNet-7~\cite{lecun1998mnist} and (bottom) DeepCNet(5,50)~\cite{ciresan2012multi,graham2014spatially} architectures used in Assamese
  character recognition experiments.}
\label{fig:nnrecognition}
\end{figure*}

\definecolor{voc_1}{RGB}{0, 0, 0}
\definecolor{voc_2}{RGB}{128, 0, 0}
\definecolor{voc_3}{RGB}{0, 128, 0}
\definecolor{voc_4}{RGB}{128, 128, 0}
\definecolor{voc_5}{RGB}{0, 0, 128}
\definecolor{voc_6}{RGB}{128, 0, 128}
\definecolor{voc_7}{RGB}{0, 128, 128}
\definecolor{voc_8}{RGB}{128, 128, 128}
\definecolor{voc_9}{RGB}{64, 0, 0}
\definecolor{voc_10}{RGB}{192, 0, 0}
\definecolor{voc_11}{RGB}{64, 128, 0}
\definecolor{voc_12}{RGB}{192, 128, 0}
\definecolor{voc_13}{RGB}{64, 0, 128}
\definecolor{voc_14}{RGB}{192, 0, 128}
\definecolor{voc_15}{RGB}{64, 128, 128}
\definecolor{voc_16}{RGB}{192, 128, 128}
\definecolor{voc_17}{RGB}{0, 64, 0}
\definecolor{voc_18}{RGB}{128, 64, 0}
\definecolor{voc_19}{RGB}{0, 192, 0}
\definecolor{voc_20}{RGB}{128, 192, 0}
\definecolor{voc_21}{RGB}{0, 64, 128}
\definecolor{voc_22}{RGB}{128, 64, 128}

\begin{figure*}[t]
  \centering
  \small{
  \fcolorbox{white}{voc_1}{\rule{0pt}{6pt}\rule{6pt}{0pt}} Background~~
  \fcolorbox{white}{voc_2}{\rule{0pt}{6pt}\rule{6pt}{0pt}} Aeroplane~~
  \fcolorbox{white}{voc_3}{\rule{0pt}{6pt}\rule{6pt}{0pt}} Bicycle~~
  \fcolorbox{white}{voc_4}{\rule{0pt}{6pt}\rule{6pt}{0pt}} Bird~~
  \fcolorbox{white}{voc_5}{\rule{0pt}{6pt}\rule{6pt}{0pt}} Boat~~
  \fcolorbox{white}{voc_6}{\rule{0pt}{6pt}\rule{6pt}{0pt}} Bottle~~
  \fcolorbox{white}{voc_7}{\rule{0pt}{6pt}\rule{6pt}{0pt}} Bus~~
  \fcolorbox{white}{voc_8}{\rule{0pt}{6pt}\rule{6pt}{0pt}} Car~~ \\
  \fcolorbox{white}{voc_9}{\rule{0pt}{6pt}\rule{6pt}{0pt}} Cat~~
  \fcolorbox{white}{voc_10}{\rule{0pt}{6pt}\rule{6pt}{0pt}} Chair~~
  \fcolorbox{white}{voc_11}{\rule{0pt}{6pt}\rule{6pt}{0pt}} Cow~~
  \fcolorbox{white}{voc_12}{\rule{0pt}{6pt}\rule{6pt}{0pt}} Dining Table~~
  \fcolorbox{white}{voc_13}{\rule{0pt}{6pt}\rule{6pt}{0pt}} Dog~~
  \fcolorbox{white}{voc_14}{\rule{0pt}{6pt}\rule{6pt}{0pt}} Horse~~
  \fcolorbox{white}{voc_15}{\rule{0pt}{6pt}\rule{6pt}{0pt}} Motorbike~~
  \fcolorbox{white}{voc_16}{\rule{0pt}{6pt}\rule{6pt}{0pt}} Person~~ \\
  \fcolorbox{white}{voc_17}{\rule{0pt}{6pt}\rule{6pt}{0pt}} Potted Plant~~
  \fcolorbox{white}{voc_18}{\rule{0pt}{6pt}\rule{6pt}{0pt}} Sheep~~
  \fcolorbox{white}{voc_19}{\rule{0pt}{6pt}\rule{6pt}{0pt}} Sofa~~
  \fcolorbox{white}{voc_20}{\rule{0pt}{6pt}\rule{6pt}{0pt}} Train~~
  \fcolorbox{white}{voc_21}{\rule{0pt}{6pt}\rule{6pt}{0pt}} TV monitor~~ \\
  }
  \subfigure{%
    \includegraphics[width=.18\columnwidth]{figures/supplementary/2007_001423_given.jpg}
  }
  \subfigure{%
    \includegraphics[width=.18\columnwidth]{figures/supplementary/2007_001423_gt.png}
  }
  \subfigure{%
    \includegraphics[width=.18\columnwidth]{figures/supplementary/2007_001423_cnn.png}
  }
  \subfigure{%
    \includegraphics[width=.18\columnwidth]{figures/supplementary/2007_001423_gauss.png}
  }
  \subfigure{%
    \includegraphics[width=.18\columnwidth]{figures/supplementary/2007_001423_learnt.png}
  }\\
  \subfigure{%
    \includegraphics[width=.18\columnwidth]{figures/supplementary/2007_001430_given.jpg}
  }
  \subfigure{%
    \includegraphics[width=.18\columnwidth]{figures/supplementary/2007_001430_gt.png}
  }
  \subfigure{%
    \includegraphics[width=.18\columnwidth]{figures/supplementary/2007_001430_cnn.png}
  }
  \subfigure{%
    \includegraphics[width=.18\columnwidth]{figures/supplementary/2007_001430_gauss.png}
  }
  \subfigure{%
    \includegraphics[width=.18\columnwidth]{figures/supplementary/2007_001430_learnt.png}
  }\\
    \subfigure{%
    \includegraphics[width=.18\columnwidth]{figures/supplementary/2007_007996_given.jpg}
  }
  \subfigure{%
    \includegraphics[width=.18\columnwidth]{figures/supplementary/2007_007996_gt.png}
  }
  \subfigure{%
    \includegraphics[width=.18\columnwidth]{figures/supplementary/2007_007996_cnn.png}
  }
  \subfigure{%
    \includegraphics[width=.18\columnwidth]{figures/supplementary/2007_007996_gauss.png}
  }
  \subfigure{%
    \includegraphics[width=.18\columnwidth]{figures/supplementary/2007_007996_learnt.png}
  }\\
   \subfigure{%
    \includegraphics[width=.18\columnwidth]{figures/supplementary/2010_002682_given.jpg}
  }
  \subfigure{%
    \includegraphics[width=.18\columnwidth]{figures/supplementary/2010_002682_gt.png}
  }
  \subfigure{%
    \includegraphics[width=.18\columnwidth]{figures/supplementary/2010_002682_cnn.png}
  }
  \subfigure{%
    \includegraphics[width=.18\columnwidth]{figures/supplementary/2010_002682_gauss.png}
  }
  \subfigure{%
    \includegraphics[width=.18\columnwidth]{figures/supplementary/2010_002682_learnt.png}
  }\\
     \subfigure{%
    \includegraphics[width=.18\columnwidth]{figures/supplementary/2010_004789_given.jpg}
  }
  \subfigure{%
    \includegraphics[width=.18\columnwidth]{figures/supplementary/2010_004789_gt.png}
  }
  \subfigure{%
    \includegraphics[width=.18\columnwidth]{figures/supplementary/2010_004789_cnn.png}
  }
  \subfigure{%
    \includegraphics[width=.18\columnwidth]{figures/supplementary/2010_004789_gauss.png}
  }
  \subfigure{%
    \includegraphics[width=.18\columnwidth]{figures/supplementary/2010_004789_learnt.png}
  }\\
       \subfigure{%
    \includegraphics[width=.18\columnwidth]{figures/supplementary/2007_001311_given.jpg}
  }
  \subfigure{%
    \includegraphics[width=.18\columnwidth]{figures/supplementary/2007_001311_gt.png}
  }
  \subfigure{%
    \includegraphics[width=.18\columnwidth]{figures/supplementary/2007_001311_cnn.png}
  }
  \subfigure{%
    \includegraphics[width=.18\columnwidth]{figures/supplementary/2007_001311_gauss.png}
  }
  \subfigure{%
    \includegraphics[width=.18\columnwidth]{figures/supplementary/2007_001311_learnt.png}
  }\\
  \setcounter{subfigure}{0}
  \subfigure[Input]{%
    \includegraphics[width=.18\columnwidth]{figures/supplementary/2010_003531_given.jpg}
  }
  \subfigure[Ground Truth]{%
    \includegraphics[width=.18\columnwidth]{figures/supplementary/2010_003531_gt.png}
  }
  \subfigure[DeepLab]{%
    \includegraphics[width=.18\columnwidth]{figures/supplementary/2010_003531_cnn.png}
  }
  \subfigure[+GaussCRF]{%
    \includegraphics[width=.18\columnwidth]{figures/supplementary/2010_003531_gauss.png}
  }
  \subfigure[+LearnedCRF]{%
    \includegraphics[width=.18\columnwidth]{figures/supplementary/2010_003531_learnt.png}
  }
  \vspace{-0.3cm}
  \mycaption{Semantic Segmentation}{Example results of semantic segmentation.
  (c)~depicts the unary results before application of MF, (d)~after two steps of MF with Gaussian edge CRF potentials, (e)~after
  two steps of MF with learned edge CRF potentials.}
    \label{fig:semantic_visuals}
\end{figure*}


\definecolor{minc_1}{HTML}{771111}
\definecolor{minc_2}{HTML}{CAC690}
\definecolor{minc_3}{HTML}{EEEEEE}
\definecolor{minc_4}{HTML}{7C8FA6}
\definecolor{minc_5}{HTML}{597D31}
\definecolor{minc_6}{HTML}{104410}
\definecolor{minc_7}{HTML}{BB819C}
\definecolor{minc_8}{HTML}{D0CE48}
\definecolor{minc_9}{HTML}{622745}
\definecolor{minc_10}{HTML}{666666}
\definecolor{minc_11}{HTML}{D54A31}
\definecolor{minc_12}{HTML}{101044}
\definecolor{minc_13}{HTML}{444126}
\definecolor{minc_14}{HTML}{75D646}
\definecolor{minc_15}{HTML}{DD4348}
\definecolor{minc_16}{HTML}{5C8577}
\definecolor{minc_17}{HTML}{C78472}
\definecolor{minc_18}{HTML}{75D6D0}
\definecolor{minc_19}{HTML}{5B4586}
\definecolor{minc_20}{HTML}{C04393}
\definecolor{minc_21}{HTML}{D69948}
\definecolor{minc_22}{HTML}{7370D8}
\definecolor{minc_23}{HTML}{7A3622}
\definecolor{minc_24}{HTML}{000000}

\begin{figure*}[t]
  \centering
  \small{
  \fcolorbox{white}{minc_1}{\rule{0pt}{6pt}\rule{6pt}{0pt}} Brick~~
  \fcolorbox{white}{minc_2}{\rule{0pt}{6pt}\rule{6pt}{0pt}} Carpet~~
  \fcolorbox{white}{minc_3}{\rule{0pt}{6pt}\rule{6pt}{0pt}} Ceramic~~
  \fcolorbox{white}{minc_4}{\rule{0pt}{6pt}\rule{6pt}{0pt}} Fabric~~
  \fcolorbox{white}{minc_5}{\rule{0pt}{6pt}\rule{6pt}{0pt}} Foliage~~
  \fcolorbox{white}{minc_6}{\rule{0pt}{6pt}\rule{6pt}{0pt}} Food~~
  \fcolorbox{white}{minc_7}{\rule{0pt}{6pt}\rule{6pt}{0pt}} Glass~~
  \fcolorbox{white}{minc_8}{\rule{0pt}{6pt}\rule{6pt}{0pt}} Hair~~ \\
  \fcolorbox{white}{minc_9}{\rule{0pt}{6pt}\rule{6pt}{0pt}} Leather~~
  \fcolorbox{white}{minc_10}{\rule{0pt}{6pt}\rule{6pt}{0pt}} Metal~~
  \fcolorbox{white}{minc_11}{\rule{0pt}{6pt}\rule{6pt}{0pt}} Mirror~~
  \fcolorbox{white}{minc_12}{\rule{0pt}{6pt}\rule{6pt}{0pt}} Other~~
  \fcolorbox{white}{minc_13}{\rule{0pt}{6pt}\rule{6pt}{0pt}} Painted~~
  \fcolorbox{white}{minc_14}{\rule{0pt}{6pt}\rule{6pt}{0pt}} Paper~~
  \fcolorbox{white}{minc_15}{\rule{0pt}{6pt}\rule{6pt}{0pt}} Plastic~~\\
  \fcolorbox{white}{minc_16}{\rule{0pt}{6pt}\rule{6pt}{0pt}} Polished Stone~~
  \fcolorbox{white}{minc_17}{\rule{0pt}{6pt}\rule{6pt}{0pt}} Skin~~
  \fcolorbox{white}{minc_18}{\rule{0pt}{6pt}\rule{6pt}{0pt}} Sky~~
  \fcolorbox{white}{minc_19}{\rule{0pt}{6pt}\rule{6pt}{0pt}} Stone~~
  \fcolorbox{white}{minc_20}{\rule{0pt}{6pt}\rule{6pt}{0pt}} Tile~~
  \fcolorbox{white}{minc_21}{\rule{0pt}{6pt}\rule{6pt}{0pt}} Wallpaper~~
  \fcolorbox{white}{minc_22}{\rule{0pt}{6pt}\rule{6pt}{0pt}} Water~~
  \fcolorbox{white}{minc_23}{\rule{0pt}{6pt}\rule{6pt}{0pt}} Wood~~ \\
  }
  \subfigure{%
    \includegraphics[width=.18\columnwidth]{figures/supplementary/000010868_given.jpg}
  }
  \subfigure{%
    \includegraphics[width=.18\columnwidth]{figures/supplementary/000010868_gt.png}
  }
  \subfigure{%
    \includegraphics[width=.18\columnwidth]{figures/supplementary/000010868_cnn.png}
  }
  \subfigure{%
    \includegraphics[width=.18\columnwidth]{figures/supplementary/000010868_gauss.png}
  }
  \subfigure{%
    \includegraphics[width=.18\columnwidth]{figures/supplementary/000010868_learnt.png}
  }\\[-2ex]
  \subfigure{%
    \includegraphics[width=.18\columnwidth]{figures/supplementary/000006011_given.jpg}
  }
  \subfigure{%
    \includegraphics[width=.18\columnwidth]{figures/supplementary/000006011_gt.png}
  }
  \subfigure{%
    \includegraphics[width=.18\columnwidth]{figures/supplementary/000006011_cnn.png}
  }
  \subfigure{%
    \includegraphics[width=.18\columnwidth]{figures/supplementary/000006011_gauss.png}
  }
  \subfigure{%
    \includegraphics[width=.18\columnwidth]{figures/supplementary/000006011_learnt.png}
  }\\[-2ex]
    \subfigure{%
    \includegraphics[width=.18\columnwidth]{figures/supplementary/000008553_given.jpg}
  }
  \subfigure{%
    \includegraphics[width=.18\columnwidth]{figures/supplementary/000008553_gt.png}
  }
  \subfigure{%
    \includegraphics[width=.18\columnwidth]{figures/supplementary/000008553_cnn.png}
  }
  \subfigure{%
    \includegraphics[width=.18\columnwidth]{figures/supplementary/000008553_gauss.png}
  }
  \subfigure{%
    \includegraphics[width=.18\columnwidth]{figures/supplementary/000008553_learnt.png}
  }\\[-2ex]
   \subfigure{%
    \includegraphics[width=.18\columnwidth]{figures/supplementary/000009188_given.jpg}
  }
  \subfigure{%
    \includegraphics[width=.18\columnwidth]{figures/supplementary/000009188_gt.png}
  }
  \subfigure{%
    \includegraphics[width=.18\columnwidth]{figures/supplementary/000009188_cnn.png}
  }
  \subfigure{%
    \includegraphics[width=.18\columnwidth]{figures/supplementary/000009188_gauss.png}
  }
  \subfigure{%
    \includegraphics[width=.18\columnwidth]{figures/supplementary/000009188_learnt.png}
  }\\[-2ex]
  \setcounter{subfigure}{0}
  \subfigure[Input]{%
    \includegraphics[width=.18\columnwidth]{figures/supplementary/000023570_given.jpg}
  }
  \subfigure[Ground Truth]{%
    \includegraphics[width=.18\columnwidth]{figures/supplementary/000023570_gt.png}
  }
  \subfigure[DeepLab]{%
    \includegraphics[width=.18\columnwidth]{figures/supplementary/000023570_cnn.png}
  }
  \subfigure[+GaussCRF]{%
    \includegraphics[width=.18\columnwidth]{figures/supplementary/000023570_gauss.png}
  }
  \subfigure[+LearnedCRF]{%
    \includegraphics[width=.18\columnwidth]{figures/supplementary/000023570_learnt.png}
  }
  \mycaption{Material Segmentation}{Example results of material segmentation.
  (c)~depicts the unary results before application of MF, (d)~after two steps of MF with Gaussian edge CRF potentials, (e)~after two steps of MF with learned edge CRF potentials.}
    \label{fig:material_visuals-app2}
\end{figure*}


\begin{table*}[h]
\tiny
  \centering
    \begin{tabular}{L{2.3cm} L{2.25cm} C{1.5cm} C{0.7cm} C{0.6cm} C{0.7cm} C{0.7cm} C{0.7cm} C{1.6cm} C{0.6cm} C{0.6cm} C{0.6cm}}
      \toprule
& & & & & \multicolumn{3}{c}{\textbf{Data Statistics}} & \multicolumn{4}{c}{\textbf{Training Protocol}} \\

\textbf{Experiment} & \textbf{Feature Types} & \textbf{Feature Scales} & \textbf{Filter Size} & \textbf{Filter Nbr.} & \textbf{Train}  & \textbf{Val.} & \textbf{Test} & \textbf{Loss Type} & \textbf{LR} & \textbf{Batch} & \textbf{Epochs} \\
      \midrule
      \multicolumn{2}{c}{\textbf{Single Bilateral Filter Applications}} & & & & & & & & & \\
      \textbf{2$\times$ Color Upsampling} & Position$_{1}$, Intensity (3D) & 0.13, 0.17 & 65 & 2 & 10581 & 1449 & 1456 & MSE & 1e-06 & 200 & 94.5\\
      \textbf{4$\times$ Color Upsampling} & Position$_{1}$, Intensity (3D) & 0.06, 0.17 & 65 & 2 & 10581 & 1449 & 1456 & MSE & 1e-06 & 200 & 94.5\\
      \textbf{8$\times$ Color Upsampling} & Position$_{1}$, Intensity (3D) & 0.03, 0.17 & 65 & 2 & 10581 & 1449 & 1456 & MSE & 1e-06 & 200 & 94.5\\
      \textbf{16$\times$ Color Upsampling} & Position$_{1}$, Intensity (3D) & 0.02, 0.17 & 65 & 2 & 10581 & 1449 & 1456 & MSE & 1e-06 & 200 & 94.5\\
      \textbf{Depth Upsampling} & Position$_{1}$, Color (5D) & 0.05, 0.02 & 665 & 2 & 795 & 100 & 654 & MSE & 1e-07 & 50 & 251.6\\
      \textbf{Mesh Denoising} & Isomap (4D) & 46.00 & 63 & 2 & 1000 & 200 & 500 & MSE & 100 & 10 & 100.0 \\
      \midrule
      \multicolumn{2}{c}{\textbf{DenseCRF Applications}} & & & & & & & & &\\
      \multicolumn{2}{l}{\textbf{Semantic Segmentation}} & & & & & & & & &\\
      \textbf{- 1step MF} & Position$_{1}$, Color (5D); Position$_{1}$ (2D) & 0.01, 0.34; 0.34  & 665; 19  & 2; 2 & 10581 & 1449 & 1456 & Logistic & 0.1 & 5 & 1.4 \\
      \textbf{- 2step MF} & Position$_{1}$, Color (5D); Position$_{1}$ (2D) & 0.01, 0.34; 0.34 & 665; 19 & 2; 2 & 10581 & 1449 & 1456 & Logistic & 0.1 & 5 & 1.4 \\
      \textbf{- \textit{loose} 2step MF} & Position$_{1}$, Color (5D); Position$_{1}$ (2D) & 0.01, 0.34; 0.34 & 665; 19 & 2; 2 &10581 & 1449 & 1456 & Logistic & 0.1 & 5 & +1.9  \\ \\
      \multicolumn{2}{l}{\textbf{Material Segmentation}} & & & & & & & & &\\
      \textbf{- 1step MF} & Position$_{2}$, Lab-Color (5D) & 5.00, 0.05, 0.30  & 665 & 2 & 928 & 150 & 1798 & Weighted Logistic & 1e-04 & 24 & 2.6 \\
      \textbf{- 2step MF} & Position$_{2}$, Lab-Color (5D) & 5.00, 0.05, 0.30 & 665 & 2 & 928 & 150 & 1798 & Weighted Logistic & 1e-04 & 12 & +0.7 \\
      \textbf{- \textit{loose} 2step MF} & Position$_{2}$, Lab-Color (5D) & 5.00, 0.05, 0.30 & 665 & 2 & 928 & 150 & 1798 & Weighted Logistic & 1e-04 & 12 & +0.2\\
      \midrule
      \multicolumn{2}{c}{\textbf{Neural Network Applications}} & & & & & & & & &\\
      \textbf{Tiles: CNN-9$\times$9} & - & - & 81 & 4 & 10000 & 1000 & 1000 & Logistic & 0.01 & 100 & 500.0 \\
      \textbf{Tiles: CNN-13$\times$13} & - & - & 169 & 6 & 10000 & 1000 & 1000 & Logistic & 0.01 & 100 & 500.0 \\
      \textbf{Tiles: CNN-17$\times$17} & - & - & 289 & 8 & 10000 & 1000 & 1000 & Logistic & 0.01 & 100 & 500.0 \\
      \textbf{Tiles: CNN-21$\times$21} & - & - & 441 & 10 & 10000 & 1000 & 1000 & Logistic & 0.01 & 100 & 500.0 \\
      \textbf{Tiles: BNN} & Position$_{1}$, Color (5D) & 0.05, 0.04 & 63 & 1 & 10000 & 1000 & 1000 & Logistic & 0.01 & 100 & 30.0 \\
      \textbf{LeNet} & - & - & 25 & 2 & 5490 & 1098 & 1647 & Logistic & 0.1 & 100 & 182.2 \\
      \textbf{Crop-LeNet} & - & - & 25 & 2 & 5490 & 1098 & 1647 & Logistic & 0.1 & 100 & 182.2 \\
      \textbf{BNN-LeNet} & Position$_{2}$ (2D) & 20.00 & 7 & 1 & 5490 & 1098 & 1647 & Logistic & 0.1 & 100 & 182.2 \\
      \textbf{DeepCNet} & - & - & 9 & 1 & 5490 & 1098 & 1647 & Logistic & 0.1 & 100 & 182.2 \\
      \textbf{Crop-DeepCNet} & - & - & 9 & 1 & 5490 & 1098 & 1647 & Logistic & 0.1 & 100 & 182.2 \\
      \textbf{BNN-DeepCNet} & Position$_{2}$ (2D) & 40.00  & 7 & 1 & 5490 & 1098 & 1647 & Logistic & 0.1 & 100 & 182.2 \\
      \bottomrule
      \\
    \end{tabular}
    \mycaption{Experiment Protocols} {Experiment protocols for the different experiments presented in this work. \textbf{Feature Types}:
    Feature spaces used for the bilateral convolutions. Position$_1$ corresponds to un-normalized pixel positions whereas Position$_2$ corresponds
    to pixel positions normalized to $[0,1]$ with respect to the given image. \textbf{Feature Scales}: Cross-validated scales for the features used.
     \textbf{Filter Size}: Number of elements in the filter that is being learned. \textbf{Filter Nbr.}: Half-width of the filter. \textbf{Train},
     \textbf{Val.} and \textbf{Test} corresponds to the number of train, validation and test images used in the experiment. \textbf{Loss Type}: Type
     of loss used for back-propagation. ``MSE'' corresponds to Euclidean mean squared error loss and ``Logistic'' corresponds to multinomial logistic
     loss. ``Weighted Logistic'' is the class-weighted multinomial logistic loss. We weighted the loss with inverse class probability for material
     segmentation task due to the small availability of training data with class imbalance. \textbf{LR}: Fixed learning rate used in stochastic gradient
     descent. \textbf{Batch}: Number of images used in one parameter update step. \textbf{Epochs}: Number of training epochs. In all the experiments,
     we used fixed momentum of 0.9 and weight decay of 0.0005 for stochastic gradient descent. ```Color Upsampling'' experiments in this Table corresponds
     to those performed on Pascal VOC12 dataset images. For all experiments using Pascal VOC12 images, we use extended
     training segmentation dataset available from~\cite{hariharan2011moredata}, and used standard validation and test splits
     from the main dataset~\cite{voc2012segmentation}.}
  \label{tbl:parameters}
\end{table*}

\clearpage

\section{Parameters and Additional Results for Video Propagation Networks}

In this Section, we present experiment protocols and additional qualitative results for experiments
on video object segmentation, semantic video segmentation and video color
propagation. Table~\ref{tbl:parameters_supp} shows the feature scales and other parameters used in different experiments.
Figures~\ref{fig:video_seg_pos_supp} show some qualitative results on video object segmentation
with some failure cases in Fig.~\ref{fig:video_seg_neg_supp}.
Figure~\ref{fig:semantic_visuals_supp} shows some qualitative results on semantic video segmentation and
Fig.~\ref{fig:color_visuals_supp} shows results on video color propagation.

\newcolumntype{L}[1]{>{\raggedright\let\newline\\\arraybackslash\hspace{0pt}}b{#1}}
\newcolumntype{C}[1]{>{\centering\let\newline\\\arraybackslash\hspace{0pt}}b{#1}}
\newcolumntype{R}[1]{>{\raggedleft\let\newline\\\arraybackslash\hspace{0pt}}b{#1}}

\begin{table*}[h]
\tiny
  \centering
    \begin{tabular}{L{3.0cm} L{2.4cm} L{2.8cm} L{2.8cm} C{0.5cm} C{1.0cm} L{1.2cm}}
      \toprule
\textbf{Experiment} & \textbf{Feature Type} & \textbf{Feature Scale-1, $\Lambda_a$} & \textbf{Feature Scale-2, $\Lambda_b$} & \textbf{$\alpha$} & \textbf{Input Frames} & \textbf{Loss Type} \\
      \midrule
      \textbf{Video Object Segmentation} & ($x,y,Y,Cb,Cr,t$) & (0.02,0.02,0.07,0.4,0.4,0.01) & (0.03,0.03,0.09,0.5,0.5,0.2) & 0.5 & 9 & Logistic\\
      \midrule
      \textbf{Semantic Video Segmentation} & & & & & \\
      \textbf{with CNN1~\cite{yu2015multi}-NoFlow} & ($x,y,R,G,B,t$) & (0.08,0.08,0.2,0.2,0.2,0.04) & (0.11,0.11,0.2,0.2,0.2,0.04) & 0.5 & 3 & Logistic \\
      \textbf{with CNN1~\cite{yu2015multi}-Flow} & ($x+u_x,y+u_y,R,G,B,t$) & (0.11,0.11,0.14,0.14,0.14,0.03) & (0.08,0.08,0.12,0.12,0.12,0.01) & 0.65 & 3 & Logistic\\
      \textbf{with CNN2~\cite{richter2016playing}-Flow} & ($x+u_x,y+u_y,R,G,B,t$) & (0.08,0.08,0.2,0.2,0.2,0.04) & (0.09,0.09,0.25,0.25,0.25,0.03) & 0.5 & 4 & Logistic\\
      \midrule
      \textbf{Video Color Propagation} & ($x,y,I,t$)  & (0.04,0.04,0.2,0.04) & No second kernel & 1 & 4 & MSE\\
      \bottomrule
      \\
    \end{tabular}
    \mycaption{Experiment Protocols} {Experiment protocols for the different experiments presented in this work. \textbf{Feature Types}:
    Feature spaces used for the bilateral convolutions, with position ($x,y$) and color
    ($R,G,B$ or $Y,Cb,Cr$) features $\in [0,255]$. $u_x$, $u_y$ denotes optical flow with respect
    to the present frame and $I$ denotes grayscale intensity.
    \textbf{Feature Scales ($\Lambda_a, \Lambda_b$)}: Cross-validated scales for the features used.
    \textbf{$\alpha$}: Exponential time decay for the input frames.
    \textbf{Input Frames}: Number of input frames for VPN.
    \textbf{Loss Type}: Type
     of loss used for back-propagation. ``MSE'' corresponds to Euclidean mean squared error loss and ``Logistic'' corresponds to multinomial logistic loss.}
  \label{tbl:parameters_supp}
\end{table*}

% \begin{figure}[th!]
% \begin{center}
%   \centerline{\includegraphics[width=\textwidth]{figures/video_seg_visuals_supp_small.pdf}}
%     \mycaption{Video Object Segmentation}
%     {Shown are the different frames in example videos with the corresponding
%     ground truth (GT) masks, predictions from BVS~\cite{marki2016bilateral},
%     OFL~\cite{tsaivideo}, VPN (VPN-Stage2) and VPN-DLab (VPN-DeepLab) models.}
%     \label{fig:video_seg_small_supp}
% \end{center}
% \vspace{-1.0cm}
% \end{figure}

\begin{figure}[th!]
\begin{center}
  \centerline{\includegraphics[width=0.7\textwidth]{figures/video_seg_visuals_supp_positive.pdf}}
    \mycaption{Video Object Segmentation}
    {Shown are the different frames in example videos with the corresponding
    ground truth (GT) masks, predictions from BVS~\cite{marki2016bilateral},
    OFL~\cite{tsaivideo}, VPN (VPN-Stage2) and VPN-DLab (VPN-DeepLab) models.}
    \label{fig:video_seg_pos_supp}
\end{center}
\vspace{-1.0cm}
\end{figure}

\begin{figure}[th!]
\begin{center}
  \centerline{\includegraphics[width=0.7\textwidth]{figures/video_seg_visuals_supp_negative.pdf}}
    \mycaption{Failure Cases for Video Object Segmentation}
    {Shown are the different frames in example videos with the corresponding
    ground truth (GT) masks, predictions from BVS~\cite{marki2016bilateral},
    OFL~\cite{tsaivideo}, VPN (VPN-Stage2) and VPN-DLab (VPN-DeepLab) models.}
    \label{fig:video_seg_neg_supp}
\end{center}
\vspace{-1.0cm}
\end{figure}

\begin{figure}[th!]
\begin{center}
  \centerline{\includegraphics[width=0.9\textwidth]{figures/supp_semantic_visual.pdf}}
    \mycaption{Semantic Video Segmentation}
    {Input video frames and the corresponding ground truth (GT)
    segmentation together with the predictions of CNN~\cite{yu2015multi} and with
    VPN-Flow.}
    \label{fig:semantic_visuals_supp}
\end{center}
\vspace{-0.7cm}
\end{figure}

\begin{figure}[th!]
\begin{center}
  \centerline{\includegraphics[width=\textwidth]{figures/colorization_visuals_supp.pdf}}
  \mycaption{Video Color Propagation}
  {Input grayscale video frames and corresponding ground-truth (GT) color images
  together with color predictions of Levin et al.~\cite{levin2004colorization} and VPN-Stage1 models.}
  \label{fig:color_visuals_supp}
\end{center}
\vspace{-0.7cm}
\end{figure}

\clearpage

\section{Additional Material for Bilateral Inception Networks}
\label{sec:binception-app}

In this section of the Appendix, we first discuss the use of approximate bilateral
filtering in BI modules (Sec.~\ref{sec:lattice}).
Later, we present some qualitative results using different models for the approach presented in
Chapter~\ref{chap:binception} (Sec.~\ref{sec:qualitative-app}).

\subsection{Approximate Bilateral Filtering}
\label{sec:lattice}

The bilateral inception module presented in Chapter~\ref{chap:binception} computes a matrix-vector
product between a Gaussian filter $K$ and a vector of activations $\bz_c$.
Bilateral filtering is an important operation and many algorithmic techniques have been
proposed to speed-up this operation~\cite{paris2006fast,adams2010fast,gastal2011domain}.
In the main paper we opted to implement what can be considered the
brute-force variant of explicitly constructing $K$ and then using BLAS to compute the
matrix-vector product. This resulted in a few millisecond operation.
The explicit way to compute is possible due to the
reduction to super-pixels, e.g., it would not work for DenseCRF variants
that operate on the full image resolution.

Here, we present experiments where we use the fast approximate bilateral filtering
algorithm of~\cite{adams2010fast}, which is also used in Chapter~\ref{chap:bnn}
for learning sparse high dimensional filters. This
choice allows for larger dimensions of matrix-vector multiplication. The reason for choosing
the explicit multiplication in Chapter~\ref{chap:binception} was that it was computationally faster.
For the small sizes of the involved matrices and vectors, the explicit computation is sufficient and we had no
GPU implementation of an approximate technique that matched this runtime. Also it
is conceptually easier and the gradient to the feature transformations ($\Lambda \mathbf{f}$) is
obtained using standard matrix calculus.

\subsubsection{Experiments}

We modified the existing segmentation architectures analogous to those in Chapter~\ref{chap:binception}.
The main difference is that, here, the inception modules use the lattice
approximation~\cite{adams2010fast} to compute the bilateral filtering.
Using the lattice approximation did not allow us to back-propagate through feature transformations ($\Lambda$)
and thus we used hand-specified feature scales as will be explained later.
Specifically, we take CNN architectures from the works
of~\cite{chen2014semantic,zheng2015conditional,bell2015minc} and insert the BI modules between
the spatial FC layers.
We use superpixels from~\cite{DollarICCV13edges}
for all the experiments with the lattice approximation. Experiments are
performed using Caffe neural network framework~\cite{jia2014caffe}.

\begin{table}
  \small
  \centering
  \begin{tabular}{p{5.5cm}>{\raggedright\arraybackslash}p{1.4cm}>{\centering\arraybackslash}p{2.2cm}}
    \toprule
		\textbf{Model} & \emph{IoU} & \emph{Runtime}(ms) \\
    \midrule

    %%%%%%%%%%%% Scores computed by us)%%%%%%%%%%%%
		\deeplablargefov & 68.9 & 145ms\\
    \midrule
    \bi{7}{2}-\bi{8}{10}& \textbf{73.8} & +600 \\
    \midrule
    \deeplablargefovcrf~\cite{chen2014semantic} & 72.7 & +830\\
    \deeplabmsclargefovcrf~\cite{chen2014semantic} & \textbf{73.6} & +880\\
    DeepLab-EdgeNet~\cite{chen2015semantic} & 71.7 & +30\\
    DeepLab-EdgeNet-CRF~\cite{chen2015semantic} & \textbf{73.6} & +860\\
  \bottomrule \\
  \end{tabular}
  \mycaption{Semantic Segmentation using the DeepLab model}
  {IoU scores on the Pascal VOC12 segmentation test dataset
  with different models and our modified inception model.
  Also shown are the corresponding runtimes in milliseconds. Runtimes
  also include superpixel computations (300 ms with Dollar superpixels~\cite{DollarICCV13edges})}
  \label{tab:largefovresults}
\end{table}

\paragraph{Semantic Segmentation}
The experiments in this section use the Pascal VOC12 segmentation dataset~\cite{voc2012segmentation} with 21 object classes and the images have a maximum resolution of 0.25 megapixels.
For all experiments on VOC12, we train using the extended training set of
10581 images collected by~\cite{hariharan2011moredata}.
We modified the \deeplab~network architecture of~\cite{chen2014semantic} and
the CRFasRNN architecture from~\cite{zheng2015conditional} which uses a CNN with
deconvolution layers followed by DenseCRF trained end-to-end.

\paragraph{DeepLab Model}\label{sec:deeplabmodel}
We experimented with the \bi{7}{2}-\bi{8}{10} inception model.
Results using the~\deeplab~model are summarized in Tab.~\ref{tab:largefovresults}.
Although we get similar improvements with inception modules as with the
explicit kernel computation, using lattice approximation is slower.

\begin{table}
  \small
  \centering
  \begin{tabular}{p{6.4cm}>{\raggedright\arraybackslash}p{1.8cm}>{\raggedright\arraybackslash}p{1.8cm}}
    \toprule
    \textbf{Model} & \emph{IoU (Val)} & \emph{IoU (Test)}\\
    \midrule
    %%%%%%%%%%%% Scores computed by us)%%%%%%%%%%%%
    CNN &  67.5 & - \\
    \deconv (CNN+Deconvolutions) & 69.8 & 72.0 \\
    \midrule
    \bi{3}{6}-\bi{4}{6}-\bi{7}{2}-\bi{8}{6}& 71.9 & - \\
    \bi{3}{6}-\bi{4}{6}-\bi{7}{2}-\bi{8}{6}-\gi{6}& 73.6 &  \href{http://host.robots.ox.ac.uk:8080/anonymous/VOTV5E.html}{\textbf{75.2}}\\
    \midrule
    \deconvcrf (CRF-RNN)~\cite{zheng2015conditional} & 73.0 & 74.7\\
    Context-CRF-RNN~\cite{yu2015multi} & ~~ - ~ & \textbf{75.3} \\
    \bottomrule \\
  \end{tabular}
  \mycaption{Semantic Segmentation using the CRFasRNN model}{IoU score corresponding to different models
  on Pascal VOC12 reduced validation / test segmentation dataset. The reduced validation set consists of 346 images
  as used in~\cite{zheng2015conditional} where we adapted the model from.}
  \label{tab:deconvresults-app}
\end{table}

\paragraph{CRFasRNN Model}\label{sec:deepinception}
We add BI modules after score-pool3, score-pool4, \fc{7} and \fc{8} $1\times1$ convolution layers
resulting in the \bi{3}{6}-\bi{4}{6}-\bi{7}{2}-\bi{8}{6}
model and also experimented with another variant where $BI_8$ is followed by another inception
module, G$(6)$, with 6 Gaussian kernels.
Note that here also we discarded both deconvolution and DenseCRF parts of the original model~\cite{zheng2015conditional}
and inserted the BI modules in the base CNN and found similar improvements compared to the inception modules with explicit
kernel computaion. See Tab.~\ref{tab:deconvresults-app} for results on the CRFasRNN model.

\paragraph{Material Segmentation}
Table~\ref{tab:mincresults-app} shows the results on the MINC dataset~\cite{bell2015minc}
obtained by modifying the AlexNet architecture with our inception modules. We observe
similar improvements as with explicit kernel construction.
For this model, we do not provide any learned setup due to very limited segment training
data. The weights to combine outputs in the bilateral inception layer are
found by validation on the validation set.

\begin{table}[t]
  \small
  \centering
  \begin{tabular}{p{3.5cm}>{\centering\arraybackslash}p{4.0cm}}
    \toprule
    \textbf{Model} & Class / Total accuracy\\
    \midrule

    %%%%%%%%%%%% Scores computed by us)%%%%%%%%%%%%
    AlexNet CNN & 55.3 / 58.9 \\
    \midrule
    \bi{7}{2}-\bi{8}{6}& 68.5 / 71.8 \\
    \bi{7}{2}-\bi{8}{6}-G$(6)$& 67.6 / 73.1 \\
    \midrule
    AlexNet-CRF & 65.5 / 71.0 \\
    \bottomrule \\
  \end{tabular}
  \mycaption{Material Segmentation using AlexNet}{Pixel accuracy of different models on
  the MINC material segmentation test dataset~\cite{bell2015minc}.}
  \label{tab:mincresults-app}
\end{table}

\paragraph{Scales of Bilateral Inception Modules}
\label{sec:scales}

Unlike the explicit kernel technique presented in the main text (Chapter~\ref{chap:binception}),
we didn't back-propagate through feature transformation ($\Lambda$)
using the approximate bilateral filter technique.
So, the feature scales are hand-specified and validated, which are as follows.
The optimal scale values for the \bi{7}{2}-\bi{8}{2} model are found by validation for the best performance which are
$\sigma_{xy}$ = (0.1, 0.1) for the spatial (XY) kernel and $\sigma_{rgbxy}$ = (0.1, 0.1, 0.1, 0.01, 0.01) for color and position (RGBXY)  kernel.
Next, as more kernels are added to \bi{8}{2}, we set scales to be $\alpha$*($\sigma_{xy}$, $\sigma_{rgbxy}$).
The value of $\alpha$ is chosen as  1, 0.5, 0.1, 0.05, 0.1, at uniform interval, for the \bi{8}{10} bilateral inception module.


\subsection{Qualitative Results}
\label{sec:qualitative-app}

In this section, we present more qualitative results obtained using the BI module with explicit
kernel computation technique presented in Chapter~\ref{chap:binception}. Results on the Pascal VOC12
dataset~\cite{voc2012segmentation} using the DeepLab-LargeFOV model are shown in Fig.~\ref{fig:semantic_visuals-app},
followed by the results on MINC dataset~\cite{bell2015minc}
in Fig.~\ref{fig:material_visuals-app} and on
Cityscapes dataset~\cite{Cordts2015Cvprw} in Fig.~\ref{fig:street_visuals-app}.


\definecolor{voc_1}{RGB}{0, 0, 0}
\definecolor{voc_2}{RGB}{128, 0, 0}
\definecolor{voc_3}{RGB}{0, 128, 0}
\definecolor{voc_4}{RGB}{128, 128, 0}
\definecolor{voc_5}{RGB}{0, 0, 128}
\definecolor{voc_6}{RGB}{128, 0, 128}
\definecolor{voc_7}{RGB}{0, 128, 128}
\definecolor{voc_8}{RGB}{128, 128, 128}
\definecolor{voc_9}{RGB}{64, 0, 0}
\definecolor{voc_10}{RGB}{192, 0, 0}
\definecolor{voc_11}{RGB}{64, 128, 0}
\definecolor{voc_12}{RGB}{192, 128, 0}
\definecolor{voc_13}{RGB}{64, 0, 128}
\definecolor{voc_14}{RGB}{192, 0, 128}
\definecolor{voc_15}{RGB}{64, 128, 128}
\definecolor{voc_16}{RGB}{192, 128, 128}
\definecolor{voc_17}{RGB}{0, 64, 0}
\definecolor{voc_18}{RGB}{128, 64, 0}
\definecolor{voc_19}{RGB}{0, 192, 0}
\definecolor{voc_20}{RGB}{128, 192, 0}
\definecolor{voc_21}{RGB}{0, 64, 128}
\definecolor{voc_22}{RGB}{128, 64, 128}

\begin{figure*}[!ht]
  \small
  \centering
  \fcolorbox{white}{voc_1}{\rule{0pt}{4pt}\rule{4pt}{0pt}} Background~~
  \fcolorbox{white}{voc_2}{\rule{0pt}{4pt}\rule{4pt}{0pt}} Aeroplane~~
  \fcolorbox{white}{voc_3}{\rule{0pt}{4pt}\rule{4pt}{0pt}} Bicycle~~
  \fcolorbox{white}{voc_4}{\rule{0pt}{4pt}\rule{4pt}{0pt}} Bird~~
  \fcolorbox{white}{voc_5}{\rule{0pt}{4pt}\rule{4pt}{0pt}} Boat~~
  \fcolorbox{white}{voc_6}{\rule{0pt}{4pt}\rule{4pt}{0pt}} Bottle~~
  \fcolorbox{white}{voc_7}{\rule{0pt}{4pt}\rule{4pt}{0pt}} Bus~~
  \fcolorbox{white}{voc_8}{\rule{0pt}{4pt}\rule{4pt}{0pt}} Car~~\\
  \fcolorbox{white}{voc_9}{\rule{0pt}{4pt}\rule{4pt}{0pt}} Cat~~
  \fcolorbox{white}{voc_10}{\rule{0pt}{4pt}\rule{4pt}{0pt}} Chair~~
  \fcolorbox{white}{voc_11}{\rule{0pt}{4pt}\rule{4pt}{0pt}} Cow~~
  \fcolorbox{white}{voc_12}{\rule{0pt}{4pt}\rule{4pt}{0pt}} Dining Table~~
  \fcolorbox{white}{voc_13}{\rule{0pt}{4pt}\rule{4pt}{0pt}} Dog~~
  \fcolorbox{white}{voc_14}{\rule{0pt}{4pt}\rule{4pt}{0pt}} Horse~~
  \fcolorbox{white}{voc_15}{\rule{0pt}{4pt}\rule{4pt}{0pt}} Motorbike~~
  \fcolorbox{white}{voc_16}{\rule{0pt}{4pt}\rule{4pt}{0pt}} Person~~\\
  \fcolorbox{white}{voc_17}{\rule{0pt}{4pt}\rule{4pt}{0pt}} Potted Plant~~
  \fcolorbox{white}{voc_18}{\rule{0pt}{4pt}\rule{4pt}{0pt}} Sheep~~
  \fcolorbox{white}{voc_19}{\rule{0pt}{4pt}\rule{4pt}{0pt}} Sofa~~
  \fcolorbox{white}{voc_20}{\rule{0pt}{4pt}\rule{4pt}{0pt}} Train~~
  \fcolorbox{white}{voc_21}{\rule{0pt}{4pt}\rule{4pt}{0pt}} TV monitor~~\\


  \subfigure{%
    \includegraphics[width=.15\columnwidth]{figures/supplementary/2008_001308_given.png}
  }
  \subfigure{%
    \includegraphics[width=.15\columnwidth]{figures/supplementary/2008_001308_sp.png}
  }
  \subfigure{%
    \includegraphics[width=.15\columnwidth]{figures/supplementary/2008_001308_gt.png}
  }
  \subfigure{%
    \includegraphics[width=.15\columnwidth]{figures/supplementary/2008_001308_cnn.png}
  }
  \subfigure{%
    \includegraphics[width=.15\columnwidth]{figures/supplementary/2008_001308_crf.png}
  }
  \subfigure{%
    \includegraphics[width=.15\columnwidth]{figures/supplementary/2008_001308_ours.png}
  }\\[-2ex]


  \subfigure{%
    \includegraphics[width=.15\columnwidth]{figures/supplementary/2008_001821_given.png}
  }
  \subfigure{%
    \includegraphics[width=.15\columnwidth]{figures/supplementary/2008_001821_sp.png}
  }
  \subfigure{%
    \includegraphics[width=.15\columnwidth]{figures/supplementary/2008_001821_gt.png}
  }
  \subfigure{%
    \includegraphics[width=.15\columnwidth]{figures/supplementary/2008_001821_cnn.png}
  }
  \subfigure{%
    \includegraphics[width=.15\columnwidth]{figures/supplementary/2008_001821_crf.png}
  }
  \subfigure{%
    \includegraphics[width=.15\columnwidth]{figures/supplementary/2008_001821_ours.png}
  }\\[-2ex]



  \subfigure{%
    \includegraphics[width=.15\columnwidth]{figures/supplementary/2008_004612_given.png}
  }
  \subfigure{%
    \includegraphics[width=.15\columnwidth]{figures/supplementary/2008_004612_sp.png}
  }
  \subfigure{%
    \includegraphics[width=.15\columnwidth]{figures/supplementary/2008_004612_gt.png}
  }
  \subfigure{%
    \includegraphics[width=.15\columnwidth]{figures/supplementary/2008_004612_cnn.png}
  }
  \subfigure{%
    \includegraphics[width=.15\columnwidth]{figures/supplementary/2008_004612_crf.png}
  }
  \subfigure{%
    \includegraphics[width=.15\columnwidth]{figures/supplementary/2008_004612_ours.png}
  }\\[-2ex]


  \subfigure{%
    \includegraphics[width=.15\columnwidth]{figures/supplementary/2009_001008_given.png}
  }
  \subfigure{%
    \includegraphics[width=.15\columnwidth]{figures/supplementary/2009_001008_sp.png}
  }
  \subfigure{%
    \includegraphics[width=.15\columnwidth]{figures/supplementary/2009_001008_gt.png}
  }
  \subfigure{%
    \includegraphics[width=.15\columnwidth]{figures/supplementary/2009_001008_cnn.png}
  }
  \subfigure{%
    \includegraphics[width=.15\columnwidth]{figures/supplementary/2009_001008_crf.png}
  }
  \subfigure{%
    \includegraphics[width=.15\columnwidth]{figures/supplementary/2009_001008_ours.png}
  }\\[-2ex]




  \subfigure{%
    \includegraphics[width=.15\columnwidth]{figures/supplementary/2009_004497_given.png}
  }
  \subfigure{%
    \includegraphics[width=.15\columnwidth]{figures/supplementary/2009_004497_sp.png}
  }
  \subfigure{%
    \includegraphics[width=.15\columnwidth]{figures/supplementary/2009_004497_gt.png}
  }
  \subfigure{%
    \includegraphics[width=.15\columnwidth]{figures/supplementary/2009_004497_cnn.png}
  }
  \subfigure{%
    \includegraphics[width=.15\columnwidth]{figures/supplementary/2009_004497_crf.png}
  }
  \subfigure{%
    \includegraphics[width=.15\columnwidth]{figures/supplementary/2009_004497_ours.png}
  }\\[-2ex]



  \setcounter{subfigure}{0}
  \subfigure[\scriptsize Input]{%
    \includegraphics[width=.15\columnwidth]{figures/supplementary/2010_001327_given.png}
  }
  \subfigure[\scriptsize Superpixels]{%
    \includegraphics[width=.15\columnwidth]{figures/supplementary/2010_001327_sp.png}
  }
  \subfigure[\scriptsize GT]{%
    \includegraphics[width=.15\columnwidth]{figures/supplementary/2010_001327_gt.png}
  }
  \subfigure[\scriptsize Deeplab]{%
    \includegraphics[width=.15\columnwidth]{figures/supplementary/2010_001327_cnn.png}
  }
  \subfigure[\scriptsize +DenseCRF]{%
    \includegraphics[width=.15\columnwidth]{figures/supplementary/2010_001327_crf.png}
  }
  \subfigure[\scriptsize Using BI]{%
    \includegraphics[width=.15\columnwidth]{figures/supplementary/2010_001327_ours.png}
  }
  \mycaption{Semantic Segmentation}{Example results of semantic segmentation
  on the Pascal VOC12 dataset.
  (d)~depicts the DeepLab CNN result, (e)~CNN + 10 steps of mean-field inference,
  (f~result obtained with bilateral inception (BI) modules (\bi{6}{2}+\bi{7}{6}) between \fc~layers.}
  \label{fig:semantic_visuals-app}
\end{figure*}


\definecolor{minc_1}{HTML}{771111}
\definecolor{minc_2}{HTML}{CAC690}
\definecolor{minc_3}{HTML}{EEEEEE}
\definecolor{minc_4}{HTML}{7C8FA6}
\definecolor{minc_5}{HTML}{597D31}
\definecolor{minc_6}{HTML}{104410}
\definecolor{minc_7}{HTML}{BB819C}
\definecolor{minc_8}{HTML}{D0CE48}
\definecolor{minc_9}{HTML}{622745}
\definecolor{minc_10}{HTML}{666666}
\definecolor{minc_11}{HTML}{D54A31}
\definecolor{minc_12}{HTML}{101044}
\definecolor{minc_13}{HTML}{444126}
\definecolor{minc_14}{HTML}{75D646}
\definecolor{minc_15}{HTML}{DD4348}
\definecolor{minc_16}{HTML}{5C8577}
\definecolor{minc_17}{HTML}{C78472}
\definecolor{minc_18}{HTML}{75D6D0}
\definecolor{minc_19}{HTML}{5B4586}
\definecolor{minc_20}{HTML}{C04393}
\definecolor{minc_21}{HTML}{D69948}
\definecolor{minc_22}{HTML}{7370D8}
\definecolor{minc_23}{HTML}{7A3622}
\definecolor{minc_24}{HTML}{000000}

\begin{figure*}[!ht]
  \small % scriptsize
  \centering
  \fcolorbox{white}{minc_1}{\rule{0pt}{4pt}\rule{4pt}{0pt}} Brick~~
  \fcolorbox{white}{minc_2}{\rule{0pt}{4pt}\rule{4pt}{0pt}} Carpet~~
  \fcolorbox{white}{minc_3}{\rule{0pt}{4pt}\rule{4pt}{0pt}} Ceramic~~
  \fcolorbox{white}{minc_4}{\rule{0pt}{4pt}\rule{4pt}{0pt}} Fabric~~
  \fcolorbox{white}{minc_5}{\rule{0pt}{4pt}\rule{4pt}{0pt}} Foliage~~
  \fcolorbox{white}{minc_6}{\rule{0pt}{4pt}\rule{4pt}{0pt}} Food~~
  \fcolorbox{white}{minc_7}{\rule{0pt}{4pt}\rule{4pt}{0pt}} Glass~~
  \fcolorbox{white}{minc_8}{\rule{0pt}{4pt}\rule{4pt}{0pt}} Hair~~\\
  \fcolorbox{white}{minc_9}{\rule{0pt}{4pt}\rule{4pt}{0pt}} Leather~~
  \fcolorbox{white}{minc_10}{\rule{0pt}{4pt}\rule{4pt}{0pt}} Metal~~
  \fcolorbox{white}{minc_11}{\rule{0pt}{4pt}\rule{4pt}{0pt}} Mirror~~
  \fcolorbox{white}{minc_12}{\rule{0pt}{4pt}\rule{4pt}{0pt}} Other~~
  \fcolorbox{white}{minc_13}{\rule{0pt}{4pt}\rule{4pt}{0pt}} Painted~~
  \fcolorbox{white}{minc_14}{\rule{0pt}{4pt}\rule{4pt}{0pt}} Paper~~
  \fcolorbox{white}{minc_15}{\rule{0pt}{4pt}\rule{4pt}{0pt}} Plastic~~\\
  \fcolorbox{white}{minc_16}{\rule{0pt}{4pt}\rule{4pt}{0pt}} Polished Stone~~
  \fcolorbox{white}{minc_17}{\rule{0pt}{4pt}\rule{4pt}{0pt}} Skin~~
  \fcolorbox{white}{minc_18}{\rule{0pt}{4pt}\rule{4pt}{0pt}} Sky~~
  \fcolorbox{white}{minc_19}{\rule{0pt}{4pt}\rule{4pt}{0pt}} Stone~~
  \fcolorbox{white}{minc_20}{\rule{0pt}{4pt}\rule{4pt}{0pt}} Tile~~
  \fcolorbox{white}{minc_21}{\rule{0pt}{4pt}\rule{4pt}{0pt}} Wallpaper~~
  \fcolorbox{white}{minc_22}{\rule{0pt}{4pt}\rule{4pt}{0pt}} Water~~
  \fcolorbox{white}{minc_23}{\rule{0pt}{4pt}\rule{4pt}{0pt}} Wood~~\\
  \subfigure{%
    \includegraphics[width=.15\columnwidth]{figures/supplementary/000008468_given.png}
  }
  \subfigure{%
    \includegraphics[width=.15\columnwidth]{figures/supplementary/000008468_sp.png}
  }
  \subfigure{%
    \includegraphics[width=.15\columnwidth]{figures/supplementary/000008468_gt.png}
  }
  \subfigure{%
    \includegraphics[width=.15\columnwidth]{figures/supplementary/000008468_cnn.png}
  }
  \subfigure{%
    \includegraphics[width=.15\columnwidth]{figures/supplementary/000008468_crf.png}
  }
  \subfigure{%
    \includegraphics[width=.15\columnwidth]{figures/supplementary/000008468_ours.png}
  }\\[-2ex]

  \subfigure{%
    \includegraphics[width=.15\columnwidth]{figures/supplementary/000009053_given.png}
  }
  \subfigure{%
    \includegraphics[width=.15\columnwidth]{figures/supplementary/000009053_sp.png}
  }
  \subfigure{%
    \includegraphics[width=.15\columnwidth]{figures/supplementary/000009053_gt.png}
  }
  \subfigure{%
    \includegraphics[width=.15\columnwidth]{figures/supplementary/000009053_cnn.png}
  }
  \subfigure{%
    \includegraphics[width=.15\columnwidth]{figures/supplementary/000009053_crf.png}
  }
  \subfigure{%
    \includegraphics[width=.15\columnwidth]{figures/supplementary/000009053_ours.png}
  }\\[-2ex]




  \subfigure{%
    \includegraphics[width=.15\columnwidth]{figures/supplementary/000014977_given.png}
  }
  \subfigure{%
    \includegraphics[width=.15\columnwidth]{figures/supplementary/000014977_sp.png}
  }
  \subfigure{%
    \includegraphics[width=.15\columnwidth]{figures/supplementary/000014977_gt.png}
  }
  \subfigure{%
    \includegraphics[width=.15\columnwidth]{figures/supplementary/000014977_cnn.png}
  }
  \subfigure{%
    \includegraphics[width=.15\columnwidth]{figures/supplementary/000014977_crf.png}
  }
  \subfigure{%
    \includegraphics[width=.15\columnwidth]{figures/supplementary/000014977_ours.png}
  }\\[-2ex]


  \subfigure{%
    \includegraphics[width=.15\columnwidth]{figures/supplementary/000022922_given.png}
  }
  \subfigure{%
    \includegraphics[width=.15\columnwidth]{figures/supplementary/000022922_sp.png}
  }
  \subfigure{%
    \includegraphics[width=.15\columnwidth]{figures/supplementary/000022922_gt.png}
  }
  \subfigure{%
    \includegraphics[width=.15\columnwidth]{figures/supplementary/000022922_cnn.png}
  }
  \subfigure{%
    \includegraphics[width=.15\columnwidth]{figures/supplementary/000022922_crf.png}
  }
  \subfigure{%
    \includegraphics[width=.15\columnwidth]{figures/supplementary/000022922_ours.png}
  }\\[-2ex]


  \subfigure{%
    \includegraphics[width=.15\columnwidth]{figures/supplementary/000025711_given.png}
  }
  \subfigure{%
    \includegraphics[width=.15\columnwidth]{figures/supplementary/000025711_sp.png}
  }
  \subfigure{%
    \includegraphics[width=.15\columnwidth]{figures/supplementary/000025711_gt.png}
  }
  \subfigure{%
    \includegraphics[width=.15\columnwidth]{figures/supplementary/000025711_cnn.png}
  }
  \subfigure{%
    \includegraphics[width=.15\columnwidth]{figures/supplementary/000025711_crf.png}
  }
  \subfigure{%
    \includegraphics[width=.15\columnwidth]{figures/supplementary/000025711_ours.png}
  }\\[-2ex]


  \subfigure{%
    \includegraphics[width=.15\columnwidth]{figures/supplementary/000034473_given.png}
  }
  \subfigure{%
    \includegraphics[width=.15\columnwidth]{figures/supplementary/000034473_sp.png}
  }
  \subfigure{%
    \includegraphics[width=.15\columnwidth]{figures/supplementary/000034473_gt.png}
  }
  \subfigure{%
    \includegraphics[width=.15\columnwidth]{figures/supplementary/000034473_cnn.png}
  }
  \subfigure{%
    \includegraphics[width=.15\columnwidth]{figures/supplementary/000034473_crf.png}
  }
  \subfigure{%
    \includegraphics[width=.15\columnwidth]{figures/supplementary/000034473_ours.png}
  }\\[-2ex]


  \subfigure{%
    \includegraphics[width=.15\columnwidth]{figures/supplementary/000035463_given.png}
  }
  \subfigure{%
    \includegraphics[width=.15\columnwidth]{figures/supplementary/000035463_sp.png}
  }
  \subfigure{%
    \includegraphics[width=.15\columnwidth]{figures/supplementary/000035463_gt.png}
  }
  \subfigure{%
    \includegraphics[width=.15\columnwidth]{figures/supplementary/000035463_cnn.png}
  }
  \subfigure{%
    \includegraphics[width=.15\columnwidth]{figures/supplementary/000035463_crf.png}
  }
  \subfigure{%
    \includegraphics[width=.15\columnwidth]{figures/supplementary/000035463_ours.png}
  }\\[-2ex]


  \setcounter{subfigure}{0}
  \subfigure[\scriptsize Input]{%
    \includegraphics[width=.15\columnwidth]{figures/supplementary/000035993_given.png}
  }
  \subfigure[\scriptsize Superpixels]{%
    \includegraphics[width=.15\columnwidth]{figures/supplementary/000035993_sp.png}
  }
  \subfigure[\scriptsize GT]{%
    \includegraphics[width=.15\columnwidth]{figures/supplementary/000035993_gt.png}
  }
  \subfigure[\scriptsize AlexNet]{%
    \includegraphics[width=.15\columnwidth]{figures/supplementary/000035993_cnn.png}
  }
  \subfigure[\scriptsize +DenseCRF]{%
    \includegraphics[width=.15\columnwidth]{figures/supplementary/000035993_crf.png}
  }
  \subfigure[\scriptsize Using BI]{%
    \includegraphics[width=.15\columnwidth]{figures/supplementary/000035993_ours.png}
  }
  \mycaption{Material Segmentation}{Example results of material segmentation.
  (d)~depicts the AlexNet CNN result, (e)~CNN + 10 steps of mean-field inference,
  (f)~result obtained with bilateral inception (BI) modules (\bi{7}{2}+\bi{8}{6}) between
  \fc~layers.}
\label{fig:material_visuals-app}
\end{figure*}


\definecolor{city_1}{RGB}{128, 64, 128}
\definecolor{city_2}{RGB}{244, 35, 232}
\definecolor{city_3}{RGB}{70, 70, 70}
\definecolor{city_4}{RGB}{102, 102, 156}
\definecolor{city_5}{RGB}{190, 153, 153}
\definecolor{city_6}{RGB}{153, 153, 153}
\definecolor{city_7}{RGB}{250, 170, 30}
\definecolor{city_8}{RGB}{220, 220, 0}
\definecolor{city_9}{RGB}{107, 142, 35}
\definecolor{city_10}{RGB}{152, 251, 152}
\definecolor{city_11}{RGB}{70, 130, 180}
\definecolor{city_12}{RGB}{220, 20, 60}
\definecolor{city_13}{RGB}{255, 0, 0}
\definecolor{city_14}{RGB}{0, 0, 142}
\definecolor{city_15}{RGB}{0, 0, 70}
\definecolor{city_16}{RGB}{0, 60, 100}
\definecolor{city_17}{RGB}{0, 80, 100}
\definecolor{city_18}{RGB}{0, 0, 230}
\definecolor{city_19}{RGB}{119, 11, 32}
\begin{figure*}[!ht]
  \small % scriptsize
  \centering


  \subfigure{%
    \includegraphics[width=.18\columnwidth]{figures/supplementary/frankfurt00000_016005_given.png}
  }
  \subfigure{%
    \includegraphics[width=.18\columnwidth]{figures/supplementary/frankfurt00000_016005_sp.png}
  }
  \subfigure{%
    \includegraphics[width=.18\columnwidth]{figures/supplementary/frankfurt00000_016005_gt.png}
  }
  \subfigure{%
    \includegraphics[width=.18\columnwidth]{figures/supplementary/frankfurt00000_016005_cnn.png}
  }
  \subfigure{%
    \includegraphics[width=.18\columnwidth]{figures/supplementary/frankfurt00000_016005_ours.png}
  }\\[-2ex]

  \subfigure{%
    \includegraphics[width=.18\columnwidth]{figures/supplementary/frankfurt00000_004617_given.png}
  }
  \subfigure{%
    \includegraphics[width=.18\columnwidth]{figures/supplementary/frankfurt00000_004617_sp.png}
  }
  \subfigure{%
    \includegraphics[width=.18\columnwidth]{figures/supplementary/frankfurt00000_004617_gt.png}
  }
  \subfigure{%
    \includegraphics[width=.18\columnwidth]{figures/supplementary/frankfurt00000_004617_cnn.png}
  }
  \subfigure{%
    \includegraphics[width=.18\columnwidth]{figures/supplementary/frankfurt00000_004617_ours.png}
  }\\[-2ex]

  \subfigure{%
    \includegraphics[width=.18\columnwidth]{figures/supplementary/frankfurt00000_020880_given.png}
  }
  \subfigure{%
    \includegraphics[width=.18\columnwidth]{figures/supplementary/frankfurt00000_020880_sp.png}
  }
  \subfigure{%
    \includegraphics[width=.18\columnwidth]{figures/supplementary/frankfurt00000_020880_gt.png}
  }
  \subfigure{%
    \includegraphics[width=.18\columnwidth]{figures/supplementary/frankfurt00000_020880_cnn.png}
  }
  \subfigure{%
    \includegraphics[width=.18\columnwidth]{figures/supplementary/frankfurt00000_020880_ours.png}
  }\\[-2ex]



  \subfigure{%
    \includegraphics[width=.18\columnwidth]{figures/supplementary/frankfurt00001_007285_given.png}
  }
  \subfigure{%
    \includegraphics[width=.18\columnwidth]{figures/supplementary/frankfurt00001_007285_sp.png}
  }
  \subfigure{%
    \includegraphics[width=.18\columnwidth]{figures/supplementary/frankfurt00001_007285_gt.png}
  }
  \subfigure{%
    \includegraphics[width=.18\columnwidth]{figures/supplementary/frankfurt00001_007285_cnn.png}
  }
  \subfigure{%
    \includegraphics[width=.18\columnwidth]{figures/supplementary/frankfurt00001_007285_ours.png}
  }\\[-2ex]


  \subfigure{%
    \includegraphics[width=.18\columnwidth]{figures/supplementary/frankfurt00001_059789_given.png}
  }
  \subfigure{%
    \includegraphics[width=.18\columnwidth]{figures/supplementary/frankfurt00001_059789_sp.png}
  }
  \subfigure{%
    \includegraphics[width=.18\columnwidth]{figures/supplementary/frankfurt00001_059789_gt.png}
  }
  \subfigure{%
    \includegraphics[width=.18\columnwidth]{figures/supplementary/frankfurt00001_059789_cnn.png}
  }
  \subfigure{%
    \includegraphics[width=.18\columnwidth]{figures/supplementary/frankfurt00001_059789_ours.png}
  }\\[-2ex]


  \subfigure{%
    \includegraphics[width=.18\columnwidth]{figures/supplementary/frankfurt00001_068208_given.png}
  }
  \subfigure{%
    \includegraphics[width=.18\columnwidth]{figures/supplementary/frankfurt00001_068208_sp.png}
  }
  \subfigure{%
    \includegraphics[width=.18\columnwidth]{figures/supplementary/frankfurt00001_068208_gt.png}
  }
  \subfigure{%
    \includegraphics[width=.18\columnwidth]{figures/supplementary/frankfurt00001_068208_cnn.png}
  }
  \subfigure{%
    \includegraphics[width=.18\columnwidth]{figures/supplementary/frankfurt00001_068208_ours.png}
  }\\[-2ex]

  \subfigure{%
    \includegraphics[width=.18\columnwidth]{figures/supplementary/frankfurt00001_082466_given.png}
  }
  \subfigure{%
    \includegraphics[width=.18\columnwidth]{figures/supplementary/frankfurt00001_082466_sp.png}
  }
  \subfigure{%
    \includegraphics[width=.18\columnwidth]{figures/supplementary/frankfurt00001_082466_gt.png}
  }
  \subfigure{%
    \includegraphics[width=.18\columnwidth]{figures/supplementary/frankfurt00001_082466_cnn.png}
  }
  \subfigure{%
    \includegraphics[width=.18\columnwidth]{figures/supplementary/frankfurt00001_082466_ours.png}
  }\\[-2ex]

  \subfigure{%
    \includegraphics[width=.18\columnwidth]{figures/supplementary/lindau00033_000019_given.png}
  }
  \subfigure{%
    \includegraphics[width=.18\columnwidth]{figures/supplementary/lindau00033_000019_sp.png}
  }
  \subfigure{%
    \includegraphics[width=.18\columnwidth]{figures/supplementary/lindau00033_000019_gt.png}
  }
  \subfigure{%
    \includegraphics[width=.18\columnwidth]{figures/supplementary/lindau00033_000019_cnn.png}
  }
  \subfigure{%
    \includegraphics[width=.18\columnwidth]{figures/supplementary/lindau00033_000019_ours.png}
  }\\[-2ex]

  \subfigure{%
    \includegraphics[width=.18\columnwidth]{figures/supplementary/lindau00052_000019_given.png}
  }
  \subfigure{%
    \includegraphics[width=.18\columnwidth]{figures/supplementary/lindau00052_000019_sp.png}
  }
  \subfigure{%
    \includegraphics[width=.18\columnwidth]{figures/supplementary/lindau00052_000019_gt.png}
  }
  \subfigure{%
    \includegraphics[width=.18\columnwidth]{figures/supplementary/lindau00052_000019_cnn.png}
  }
  \subfigure{%
    \includegraphics[width=.18\columnwidth]{figures/supplementary/lindau00052_000019_ours.png}
  }\\[-2ex]




  \subfigure{%
    \includegraphics[width=.18\columnwidth]{figures/supplementary/lindau00027_000019_given.png}
  }
  \subfigure{%
    \includegraphics[width=.18\columnwidth]{figures/supplementary/lindau00027_000019_sp.png}
  }
  \subfigure{%
    \includegraphics[width=.18\columnwidth]{figures/supplementary/lindau00027_000019_gt.png}
  }
  \subfigure{%
    \includegraphics[width=.18\columnwidth]{figures/supplementary/lindau00027_000019_cnn.png}
  }
  \subfigure{%
    \includegraphics[width=.18\columnwidth]{figures/supplementary/lindau00027_000019_ours.png}
  }\\[-2ex]



  \setcounter{subfigure}{0}
  \subfigure[\scriptsize Input]{%
    \includegraphics[width=.18\columnwidth]{figures/supplementary/lindau00029_000019_given.png}
  }
  \subfigure[\scriptsize Superpixels]{%
    \includegraphics[width=.18\columnwidth]{figures/supplementary/lindau00029_000019_sp.png}
  }
  \subfigure[\scriptsize GT]{%
    \includegraphics[width=.18\columnwidth]{figures/supplementary/lindau00029_000019_gt.png}
  }
  \subfigure[\scriptsize Deeplab]{%
    \includegraphics[width=.18\columnwidth]{figures/supplementary/lindau00029_000019_cnn.png}
  }
  \subfigure[\scriptsize Using BI]{%
    \includegraphics[width=.18\columnwidth]{figures/supplementary/lindau00029_000019_ours.png}
  }%\\[-2ex]

  \mycaption{Street Scene Segmentation}{Example results of street scene segmentation.
  (d)~depicts the DeepLab results, (e)~result obtained by adding bilateral inception (BI) modules (\bi{6}{2}+\bi{7}{6}) between \fc~layers.}
\label{fig:street_visuals-app}
\end{figure*}

%}

\end{document}
