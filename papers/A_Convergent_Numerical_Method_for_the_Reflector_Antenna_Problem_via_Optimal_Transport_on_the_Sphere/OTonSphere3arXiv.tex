\documentclass{amsart}
%
%%%%%%%%%%%%%%%%%%%%
%\usepackage[notref,notcite]{showkeys}
\usepackage{amsmath}
\usepackage{amssymb}
\usepackage{amsthm}
\usepackage{dsfont}
%\usepackage{subfig}
\usepackage{lineno}
\usepackage{subfigure}
\usepackage{graphicx}
\usepackage{multirow}
\usepackage{color}
\usepackage{xspace}
\usepackage{pdfsync}
\usepackage{hyperref} 
\usepackage{algorithmicx}
\usepackage{algpseudocode}
\usepackage{algorithm}
\usepackage{mathtools}
\usepackage{enumitem}
%\usepackage[pdftex]{hyperref} 
%\usepackage{showlabels}
\graphicspath{{Figures/}}

%%%%%%%%%%%%%%%%%%%%%%%%%
% Macros
%%%%%%%%%%%%%%%%%%%%%%%%%%


%\newcommand{\argmin}{argmin}
\DeclareMathOperator*{\argmin}{argmin}
\DeclareMathOperator*{\argmax}{argmax}
\DeclarePairedDelimiter\ceil{\lceil}{\rceil}
\DeclarePairedDelimiter\floor{\lfloor}{\rfloor}
\newcommand{\bq}{\begin{equation}}
\newcommand{\eq}{\end{equation}}
\newcommand{\R}{\mathbb{R}}
\newcommand{\Z}{\mathbb{Z}}
\newcommand{\abs}[1]{\left\vert#1\right\vert}
\newcommand{\norm}[1]{\left\Vert#1\right\Vert}
\newcommand{\blue}[1]{{\color{blue}{#1}}}
\newcommand{\red}[1]{{\color{red}{#1}}}
\newcommand{\grad}{\nabla}
\newcommand{\ex}[1]{\times 10^{- #1}}
\newcommand{\G}{\mathcal{G}}
\newcommand{\bO}{\mathcal{O}}
\newcommand{\dist}{\text{dist}}
\newcommand{\one}{\mathds{1}}
\newcommand{\Dt}{\mathcal{D}}
\newcommand{\Prop}{\mathcal{P}}
\newcommand{\F}{\mathcal{F}}
\newcommand{\C}{\mathcal{C}}
\newcommand{\Sf}{\mathbb{S}^{2}}
\newcommand{\Tf}{\mathcal{T}}
\newcommand{\Zf}{\mathcal{Z}}
\newcommand{\Af}{\mathcal{A}}
\newcommand{\Bf}{\mathcal{B}}
\newcommand{\Lf}{\mathcal{L}}
\newcommand{\Mf}{\mathcal{M}}
\newcommand{\Nf}{\mathcal{N}}
\newcommand{\Ef}{\mathcal{E}}
\newcommand{\cp}{\text{cp}}
\newcommand{\MA}{Monge-Amp\`ere\xspace}
\newcommand{\test}{\mathcal{T}}
\newcommand{\interp}{\mathcal{P}}

\newcommand*{\avint}{\mathop{\ooalign{$\int$\cr$-$}}}

\algnewcommand{\LineComment}[1]{\State \(\triangleright\) #1}
\newcommand{\specialcell}[2][c]{%
  \begin{tabular}[#1]{@{}c@{}}#2\end{tabular}}


%%% Theorem environments
\newtheorem{theorem}{Theorem}
\newtheorem{thm}{Theorem}

%
\theoremstyle{lemma}
\newtheorem{lemma}[theorem]{Lemma}
\newtheorem{lem}[theorem]{Lemma}
\newtheorem{corollary}[theorem]{Corollary}
\newtheorem{cor}[theorem]{Corollary}
\newtheorem{definition}[theorem]{Definition}
\newtheorem{proposition}[theorem]{Proposition}
\newtheorem{conjecture}[theorem]{Conjecture}
\newtheorem{remark}[theorem]{Remark}
\newtheorem{rem}[theorem]{Remark}
\newtheorem{hypothesis}[theorem]{Hypothesis}

%
\theoremstyle{remark}
\newtheorem{examp}{Example}


\makeatletter
\newcommand\appendix@section[1]{%
\refstepcounter{section}%
\orig@section*{Appendix \@Alph\c@section: #1}%
%\addcontentsline{toc}{section}{Appendix \@Alph\c@section: #1}%
}
\let\orig@section\section
\g@addto@macro\appendix{\let\section\appendix@section}
\makeatother


%%%%%%%%%%%%%%%%%
% Formatted for Article class
%%%%%%%%%%%%%%%%%

% This is for article style

\begin{document}

\title[Reflector Antenna Problem via Optimal Transport]{A Convergent Numerical Method for the Reflector Antenna Problem via Optimal Transport on the Sphere}


\author{Brittany Froese Hamfeldt}
\address{Department of Mathematical Sciences, New Jersey Institute of Technology, University Heights, Newark, NJ 07102}
\email{bdfroese@njit.edu}
\author{Axel G. R. Turnquist}
\address{Department of Mathematical Sciences, New Jersey Institute of Technology, University Heights, Newark, NJ 07102}
\email{agt6@njit.edu}


\thanks{The first author was partially supported by NSF DMS-1619807 and NSF DMS-1751996. The second author was partially supported by  an NSF GRFP}

\begin{abstract}
We consider a PDE approach to numerically solving the reflector antenna problem by solving an Optimal Transport problem on the unit sphere with cost function $c(x,y) = -2\log \left\Vert x - y \right\Vert$. At each point on the sphere, we replace the surface PDE with a generalized Monge-Amp\`ere type equation posed on the local tangent plane. We then utilize a provably convergent finite difference scheme to approximate the solution and construct the reflector. The method is easily adapted to take into account highly nonsmooth data and solutions, which makes it particularly well adapted to real-world optics problems.  Computational examples demonstrate the success of this method in computing reflectors for a range of challenging problems including discontinuous intensities and intensities supported on complicated geoemtries.
\end{abstract}


\date{\today}    
\maketitle
\section{Introduction}\label{sec:intro}
Advances in light emitting diode (LED) technology in recent years have allowed for more flexibility in the engineering of freeform lenses using plastics in light illumination problems. In this article, we focus on the reflector antenna problem, which involves designing a reflector to reshape a point source onto a prescribed output in the far-field.  
On the theoretical side, a major advance in understanding freeform geometric optics problems has been gained by reformulating the problem as a fully nonlinear partial differential equation (PDE) of Monge-Amp\`{e}re type. In the particular case of the reflector antenna problem, this PDE is posed on the sphere.  The curved geometry, nonlinearity of the equation, and singular terms within the PDE make this a challenging problem to solve numerically.  

In this article, we propose a new method for the design of the reflector surface that relies on recent advances by the authors in the numerical approximation and analysis of \MA type equations on the sphere~\cite{HT_OTonSphere2}.  We emphasize that this new method comes with theoretical guarantees of convergence, even in settings involving very non-smooth output intensities~\cite{HT_OTonSphere}.

Computational approaches to solving optical design problems can be roughly divided into three basic categories: (1) techniques that use a ray-mapping to design the optical surface, (2) methods that approximate the optical surfaces by supporting quadrics, and (3) methods that represent the optical surface through the solution to an optimal transportation problem.

The ray-mapping approach generally involves a two-step procedure.  In the first step, a ray mapping is produced between the input and output intensities. In the second step, the laws of reflection and/or refraction are employed to construct a surface that achieves this ray mapping as nearly as possible. Several methods based on this general approach are available including~\cite{Bruneton_lens,Desnijder_raymapping,FFL_optics,Fournier_reflector,Parkyn_lenses}.  A downside to this general approach is that it can be difficult to theoretically justify the existence of an optical surface that exactly produces the desired ray mapping.

Oliker's method of supporting quadrics involves representing the optical surface via supporting ellipsoids or hyperboloids~\cite{Oliker_nearfield,Oliker_SQM}. The simple optical properties of these quadrics is used to produce a pixelated version of the desired target.  This approach has the advantage of being theoretically well-founded, but can be costly to implement in practice.

Finally, the solution to many optical design problems can be obtained directly through the solution of a corresponding optimal transportation problem.  That is, if $f_1$ represents the input intensity and $f_2$ the desired output intensity, it is necessary to solve a problem of the form
\bq\label{eq:OT}
\min\limits_{T_\# f_1 = f_2} \int_{\text{supp}(f_1)} c(x,T(x)) f_1(x) dx.
\eq
where $c(x,y)$ is the cost of transporting a unit of mass from $x$ to $y$ and $T_\# f_1 = f_2$ indicates that
\bq\label{eq:massCons} \int_A f_1(x)\,dS(x) = \int_{T(A)} f_2(y)\,dS(y) \eq
for every measurable $A \subset \Sf$.
%Here $c(x,y) = -2\log \left\Vert x - y \right\Vert$ is the cost of transporting a unit of mass from $x$ to $y$ and $T_\# f_1 = f_2$ indicates that
%\[ \int_A f_1(x)\,dS(x) = \int_{T(A)} f_2(y)\,dS(y) \]
%for every measurable $A \subset \Sf$.


Many optical inverse problems have yielded fruitful interpretations via optimal transport by deriving an appropriate cost function $c(x,y)$~\cite{YadavThesis}. To give a simple example, a parallel-in, far-field out setup yields the cost function $c(x,y) = \frac{1}{2}\left\Vert x - y \right\Vert^2$, where $x,y \in \mathbb{R}^2$.  The reflector antenna problem considered in this article
 has a slightly more challenging set-up in that the cost function $c(x,y) = -2\log \left\Vert x - y \right\Vert$ is unbounded and the intensity functions $f_1, f_2$ are supported on $\mathbb{S}^2$ (the unit $2$-sphere), as opposed to subsets of Euclidean space~\cite{GangboOliker,OlikerNewman,Wang_Reflector,Wang_Reflector2}. 

One approach to solving optimal transport problems in optical design is to use optimization techniques, including linear assignment~\cite{Doskolovich_farfield} and linear programming~\cite{GlimmOliker_SingleReflector}.  This approach has the advantage of being theoretically well-understood.  However, the optimization problems typically involve a very large number of constraints and the resulting methods are computationally complex. 

In many cases, the solution to the optimal transport problem can also be obtained through the solution of a fully nonlinear partial differential equation of \MA type, which has the general form
\bq\label{eq:MA}
\det(D_{xx}^2(u(x)+A(x,\nabla u(x)))) = H(x,\nabla u(x))
\eq
subject to the constraint that
\bq\label{eq:cconvex}
D_{xx}^2(u(x)+A(x,\nabla u(x))) \geq 0,
\eq
where $M\geq 0$ means that $M$ is positive semi-definite.  In the case of a point source lens or reflector design problem, this PDE is posed on the unit sphere $\Sf$.

Recently, several methods have been proposed for solving optical design problems involving a point source via the solution of a \MA type equation.  These methods replace the PDE on the sphere with a corresponding equation on the plane by representing subsets of the unit sphere using spherical coordinates~\cite{Wu_lensdesign}, a vertical projection of coordinates onto the plane~\cite{Brix_MAOptics}, or stereographic projection~\cite{RomijnSphere}.  As the numerical solution of these \MA type equations is a very new field, many of the numerical methods used in optical design problems are not yet equipped with theoretical guarantees of convergence.

In the present article, the solution to the reflector antenna problem is obtained by solving a \MA type equation directly on the sphere.  This has the advantage of allowing for intensity distributions supported on complicated subsets of the sphere or even the entire sphere.  Moreover, the approach is intrinsic and thus the solution to the problem will not depend on such details as the choice of the north pole.  Finally, the numerical method we use is theoretically well-justified and can be proven to converge to the correct solution of the \MA equation in a wide variety of challenging settings~\cite{HT_OTonSphere, HT_OTonSphere2}.







\section{Mathematical Approach}\label{sec:background}
%\subsection{Reflector antenna problem}

Here we briefly summarize the derivation of the reflector antenna problem and its connection to optimal transport on the sphere, which leads to an equation of \MA type that can be solved using techniques from numerical PDEs. 

We begin by following the physical derivation in~\cite{Wang_Reflector, Wang_Reflector2}. We start with a light source or detector $\mu$ located at the origin, which is a probability measure indicating directional intensity and is supported on a set $\Omega \subset \mathbb{S}^2$. Next we consider a reflector surface $\Sigma$, which is a radial graph over the domain $\Omega$ and can be represented as
\begin{equation}
\Sigma = \left\{ x\rho(x) \mid x \in \Omega, \quad \rho>0 \right\}
\end{equation}
where $\rho: \Omega \rightarrow \mathbb{R}$ is a non-negative function indicating the distance between the reflector surface and the origin. The light from the source $\mu$ in the direction $x$ bounces off the reflector $\Sigma$ without any refraction or absorption and travels in the direction $T$ following the law of reflection. Over all directions this produces the far-field intensity $\nu$, which is also a probability measure indicating directional intensity and is supported on some target domain $\Omega^{*} \subset \Sf$. See Figure~\ref{fig:reflectorantenna} for a schematic of the setup.

\begin{figure}[htp]
\includegraphics[height=8cm]{reflectorantenna2}
\caption{Reflector antenna with source/detector $\mu$, reflector $\Sigma$ and target far-field intensity $\nu$. The directional vectors $x$ and $T(x)$ are unit vectors.}\label{fig:reflectorantenna}
\end{figure}

The reflector antenna problem is thus: given source and target intensity probability distributions $\mu$ and $\nu$, respectively, find the shape of the reflector $\Sigma$ that transmits the light from the source to the target while satisfying conservation of energy.
We make the assumption that the probability densities $\mu$ and $\nu$ have density functions $f_1$ and $f_2$ respectively (so that $d\mu(x) = f_1(x)dS(x), d\nu(y) = f_2(y)dS(y)$).  Now we seek a PDE that will allow us to determine the reflector height function $\rho(x)$, which fully determines the reflector surface, in terms of the prescribed intensity functions $f_1$ and $f_2$.

The first of the two physical laws that will be used to derive the governing PDE for this setup is the well known geometric law of reflection, which yields the optical map
\begin{equation}\label{eq:mapping}
T(x) = x - 2\left\langle x, n(x) \right\rangle n(x)
\end{equation}
where $n(x)$ is the outward normal to $\Sigma$ at the point $z = x \rho(x)$, $x \in \Omega$.  See Figure~\ref{fig:reflection}.  We emphasize here that this is the geometric optics limit.

\begin{figure}[htp]
\includegraphics[height=4.5cm]{reflection}
\caption{Incident light direction $x$, reflector $\Sigma$, outward normal $n$, and outward light ray $T$}\label{fig:reflection}
\end{figure}

%From~\eqref{eq:mapping} an expression for the mapping $T$ in terms of the function $\rho$ can be derived. The second physical law that completes the problem is the energy conservation law:

The second physical law that completes the problem is the law of conservation of energy:
\begin{equation}\label{eq:conservation}
\int_{{T}^{-1}(E)} f_1(x)\,dS(x) = \int_{E} f_2(y)\,dS(y).
\end{equation}
for any Borel set $E \subset \Omega^{*}$.
 %By the explicit formulation of the mapping $T$~\ref{eq:mapping} and the change of variables formula applied to~\ref{eq:conservation}, one derives the PDE~\cite{Wang_Reflector2} in local coordinates:

By introducing local coordinates on the sphere, Wang~\cite{Wang_Reflector} observes that the unit normal $n$ can be given by
\bq\label{eq:n}
n(x) = \frac{\nabla\rho(x) - x\rho(x)}{\sqrt{\rho(x)^2+\norm{\nabla\rho(x)}^2}}.
\eq
Then the law of reflection~\eqref{eq:mapping} yields the mapping
\bq\label{eq:mapping2}
T(x) = \frac{2\rho(x)\nabla\rho(x)+\left(-\rho(x)^2+\norm{\nabla\rho(x)}^2\right)x}{\rho(x)^2+\norm{\nabla\rho(x)}^2}.
\eq
Applying the change of variables formula to the conservation of energy constraint~\eqref{eq:conservation} produces an equation of the form
\bq\label{eq:detEquation}
\det(\nabla T(x)) = f_1(x)/f_2(T(x)).
\eq
Combining these conditions yields the PDE
\begin{equation}\label{eq:opticsPDE}
\eta^{-2} \det\left( -\nabla_{i} \nabla_{j} \rho + 2 \rho^{-1} \nabla_{i} \rho \nabla_{j} \rho + (\rho - \eta)\delta_{ij} \right) = f_1(x)/f_2(T(x))
\end{equation}
where $\eta = \left( \left\vert \nabla \rho \right\vert^2 + \rho^2 \right)/2\rho$ and $\delta_{ij}$ is the usual Kronecker delta. We recognize this PDE as an equation of Monge-Amp\`{e}re type, with the usual second boundary value condition~\cite{Urbas}
\begin{equation}
T(\Omega) = \Omega^{*}.
\end{equation}

Unfortunately, there are few direct results in the literature that answer the kind of questions of existence and regularity results that are needed to design a convergent numerical method for~\eqref{eq:opticsPDE}. Instead, we extract a problem with more structure via the change of variables
\begin{equation}\label{eq:cov}
\rho = e^{-u}.
\end{equation}
Wang~\cite{Wang_Reflector2} shows that under an equivalent change of variables (modulo a sign change), the function $u$ solves the dual formulation of the optimal transport problem with cost function $\tilde{c}(x,y) = -\log(1-x\cdot y)$.

As an alternative approach, we notice that under this change of variables, the optical mapping~\eqref{eq:mapping2} becomes
\bq\label{eq:mappingu}
T(x) = \frac{-2\nabla u(x) + \left(\norm{\nabla u(x)}^2-1\right)x}{\norm{\nabla u(x)}^2+1}.
\eq
As in~\cite{HT_OTonSphere}, we regard this mapping as a function of the two variables $(x,\nabla u(x))$ and recognize it as a solution of the system
\bq\label{eq:optimalMap}
\begin{cases}
\nabla_x c(x,T(x,p)) = -p, & x \in \Sf\\
T(x,p) \in \Sf
\end{cases}
\eq
with the cost function 
%
%%
%Using the solution $u$ of the optimal transport problem on the sphere with a logarithmic cost function, we can solve for the reflector $\Sigma$ by computing $\rho = e^{u}$. The convergence guarantees for numerically solving the optimal transport problem in~\cite{HT_OTonSphere, HT_OTonSphere2} then translate to a numerical convergence proof for the shape of the reflector $\Sigma$ solving the reflector antenna problem.
%
%%\subsection{Optimal transport on the sphere}
%
%We consider solving the reflector antenna via solving the Optimal Transport problem with logarithmic cost with a PDE formulation on the sphere. We consider points $x,y$ lying on a unit sphere $\Sf$ centered at the origin and the cost function:
\bq\label{eq:logCost}
c(x,y) = -2\log \norm{ x-y }.
\eq
This is precisely the optimality condition for the optimal transport problem on the sphere~\cite{Loeper_OTonSphere}.  Combined with the conservation of energy condition~\eqref{eq:conservation}, we can conclude that the optical mapping $T(x)$ is a solution of the optimal transport problem~\eqref{eq:OT}-\eqref{eq:massCons} with cost~\eqref{eq:logCost}.  Moreover, this interpretation opens up many existence, regularity, and numerical approximation results that can be used in determining the reflector surface $\Sigma$.

Loeper has studied this problem in detail~\cite{Loeper_OTonSphere}. Under mild conditions on the intensity distributions $f_1$ and $f_2$, the function $u$ (which fully determines the reflector surface) can be uniquely obtained as the solution of the following \MA type equation.
\bq\label{eq:OTPDE}
\begin{cases}
\det(D^2u + A(x,\nabla u)) = H(x,\nabla u), & x \in \Sf\\
D^2u + A(x,\nabla u) \geq 0.
\end{cases}
\eq
Here
\bq\label{eq:PDETerms}
\begin{split}
A(x,p) &= D_{xx}^2c \left( x,T(x,p) \right)\\
H(x,p) &= \abs{\det{D_{xy}^2c \left( x,T(x,p) \right)}}f_1(x)/f_2 \left( T(x,p) \right).
\end{split}
\eq
and the statement $M \geq 0$ means that $M$ is positive semi-definite.  This constraint (related to the so-called $c$-convexity of the optimal map $T$) is needed to ensure that the PDE has a unique solution (up to additive constants) and that this solution corresponds to the desired optical mapping $T$. 

We remark that the above equation describes a nonlinear relationship between the surface gradient and Hessian on the sphere.  In light of our goal of solving this equation numerically, perhaps the most challenging term is the mixed Hessian $D_{xy}^2c(x,y)$, which involves derivatives with respect to two different variables located at different points on the sphere.  However, following the derivation in~\cite{HT_OTonSphere2}, we can obtain a very simple explicit expression for this term by interpreting it as a change of area formula:
\begin{equation}\label{eq:mixedHessian}
\abs{\det{D_{xy}^2c \left( x,T(x,p) \right)}} = \frac{\left(\left\Vert p \right\Vert^2 + 1 \right)^2}{4}.
\end{equation}

%This problem was studied by Loeper~\cite{Loeper_OTonSphere}, who showed that under mild regularity requirements on the data, the optimal transport problem on the sphere admits a smooth ($C^3$) solution $u$. Weak ($C^1$) solutions are also possible for discontinuous density functions that are only in $L^p$ ($1 \leq p < \infty$).  These regularity results will be important in constructing convergent numerical schemes. Moreover, the solution is unique up to additive constants.


A second challenge associated with the nonlinear \MA type equation~\eqref{eq:OTPDE} is that it requires the enforcement of an additional constraint that $D^2u + A(x,\nabla u) \geq 0$, which makes it difficult to directly apply standard techniques for approximating PDEs.  However, we succeed at absorbing this constraint into the PDE itself by relying on the following characterization of a positive semi-definite $n\times n$ matrix $M$~\cite{FO_MATheory}:
\bq\label{eq:matrix}
\begin{split}
\det(M) &= \min\limits_{\nu_i^T\nu_k=\delta_{ik}}\prod\limits_{j=1}^n \nu_j^TM\nu_j \\
  &= \min\limits_{\nu_i^T\nu_k=\delta_{ik}}\prod\limits_{j=1}^n \max\{\nu_j^TM\nu_j, 0\}.
\end{split}
\eq
Here $\delta_{ij}$ denotes the Kronecker delta function and this involves a minimization over all orthogonal coordinate frames for $\R^n$.  By observing that $\nu_j^TM\nu_j \geq 0$ for any positive semi-definite matrix $M$, we can include this condition directly in the operator instead of requiring it to be specified as a separate constraint.  This allows us to reformulate the system~\eqref{eq:OTPDE}-\eqref{eq:PDETerms} as the following unconstrained PDE.
\bq\label{eq:PDEReformulated}
\begin{split}
F(x,\nabla u(x), D^2u(x)) &\equiv
\min\limits_{\nu_1\cdot\nu_2 = 0}\prod\limits_{j=1}^2 \left. \max\left\{\frac{\partial^2(u(x)-2\log\norm{x-y})}{\partial\nu_j^2},0\right\} \right|_{y = T(x,\nabla u(x))}\\ &\phantom{=}- \frac{\left(\norm{\nabla u(x)}^2+1\right)^2 f_1(x)}{4f_2(T(x,\nabla u(x))}\\
 &= 0.
\end{split}
\eq




\section{Numerical Method}\label{sec:method}
We now describe the algorithm we use to construct the reflector surface $\Sigma$.  The algorithm hinges on the numerical solution of the nonlinear PDE~\eqref{eq:PDEReformulated}.  For fully nonlinear PDEs, it is well known that consistent and stable numerical methods may nevertheless fail to compute the correct solution. In fact, because the function $u$ is unique only up to additive constants, even fairly sophisticated numerical methods can fail to find any solution at all.  The method we describe here is inspired by a numerical scheme recently designed by the authors, which is equipped with a proof of convergence to the physically meaningful solution of the optimal transport problem.  We summarize the scheme here, and refer to~\cite{HT_OTonSphere,HT_OTonSphere2} for complete details and analysis.

\subsection{Algorithm}
We begin with a high-level overview of the algorithm.  Details will be expanded on in the following subsections.

Our starting point is a finite set of $N$ grid points $\G \subset \Sf$ that discretize the unit sphere, and the intensity distributions $f_1$ and $f_2$ that are supported on domains $\Omega\subset\Sf$ and $\Omega^*\subset\Sf$ respectively.  We let $d_{\Sf}(x,y)$ denote the usual geodesic distance between points $x,y$ on the sphere.

To the grid $\G$, we associate a number $h$ that indicates the overall spacing of grid points.  More precisely,
\begin{equation}\label{eq:h}
h = \sup\limits_{x\in\Sf}\min\limits_{y\in\G^h} d_{\Sf}(x,y) = \bO\left(N^{-1/2}\right).
\end{equation}
In particular, this guarantees that any ball of radius $h$ on the sphere will contain at least one discretization point.

Now we seek a finite difference approximation of the form
\bq\label{eq:fd}
F^h(x,u;f_1,f_2) = 0, \quad x \in \G
\eq
that approximates the original PDE~\eqref{eq:PDEReformulated}.  Our goal is to construct an approximation with the properties that (1) a solution $u^h$ exists and (2) the solution is close to the solution $u$ of the original PDE.  Our earlier work~\cite{HT_OTonSphere,HT_OTonSphere2} provides a framework for doing this.  In the most challenging settings, this requires some initial preprocessing of the data $f_1, f_2$, but then provides us with an algorithm that is guaranteed to produce a reflector surface $\Sigma^h$ that is close to the desired reflector $\Sigma$.
See Algorithm~\ref{alg:reflector}.

\begin{algorithm}[h]
\caption{Computing the reflector surface $\Sigma$}
\label{alg:reflector}
\begin{algorithmic}[1]
\State Preprocess data \[f_2^\epsilon \leftarrow \text{Regularize}(f_2).\]
\State Iterate
\[ u^h_{n+1} = u^h_n + k \left(F^h(x,u^h_n;f_1,{f_2^\epsilon}) - \sqrt{h}u^h_n(x) \right) \]
to steady state.
\State Normalize solution
\[ u^h(x) \leftarrow u^h(x) - \avint_{\phantom{==}\Sf}u^h(x)\,dS(x). \]
\State Construct reflector
\[ \Sigma^h = \left\{xe^{-u^h(x)} \mid x \in \Omega \cap \G\right\}. \]
\end{algorithmic}
\end{algorithm}

\subsection{Discretization}
We now consider a fixed grid point $x_0 \in \G$ and a grid functions $u:\G\to\R$ and explain how we obtain the value of $F^h(x_0,u;f_1,f_2)$; we refer to~\cite{HT_OTonSphere2} for further details.

We begin by projecting grid points close to $x_0$ onto the tangent plane at $x_0$.  That is, we consider the set of relevant discretization points
\bq\label{eq:neighbourCandidates}
\Zf(x_0) = \left\{z = \text{Proj}(x;x_0) \mid x \in \G, d_{\Sf}(x,x_0) \leq \sqrt{h}\right\}.
\eq
The projection is accomplished using geodesic normal coordinates, which are chosen to preserve the distance from $x_0$ (i.e. $d_{\Sf}(x,x_0) = \|x_0 -  \text{Proj}(x;x_0)\|$).  This prevents any distortions that would affect the second order terms in the PDE~\eqref{eq:PDEReformulated}.
\bq\label{eq:projection}
\text{Proj}(x;x_0) = x_0\left(1-d_{\Sf}(x_0,x)\cot d_{\Sf}(x_0,x)\right) + x \left(d_{\Sf}(x_0,x)\csc d_{\Sf}(x_0,x)\right).
\eq

The form of~\eqref{eq:PDEReformulated} indicates that we will need to approximate derivatives along various directions $\nu$.  We will consider the following finite set of possible directions,
\bq\label{eq:directions}
V = \left\{\left\{(\cos(jd\theta),\sin(jd\theta)),(-\sin(jd\theta),\cos(jd\theta))\right\} \mid j=1,\ldots,\frac{\pi}{2d\theta}\right\},
\eq
where the angular resolution $d\theta = \dfrac{\pi}{2\floor{\pi/(2\sqrt{h})}}$.

For each $\nu\in V$, we need to select four grid points $x_j\in\Zf(x_0)$, $j=1, \ldots, 4$, which will be used to construct the directional derivatives in this direction.  To accomplish this, we let $\nu^\perp$ be a unit vector orthogonal to $\nu$ and represent points in $x\in\Zf(x_0)$ using (rotated) polar coordinates $(r,\theta)$ centred at $x_0$ via
\[ x = x_0 + r(\nu\cos\theta + \nu^\perp\sin\theta), x \in \Zf(x_0). \]
Then we select four points, each in a different quadrant ($Q_1, \ldots, Q_4$), that are well-aligned with the direction of $\nu$ via
\bq\label{eq:neighbours}
x_j \in \argmin\limits_{x\in\Zf(x_0)}\left\{\abs{\sin\theta} \mid \abs{\sin\theta} \geq d\theta, r \geq \sqrt{h}-2h, x\in Q_j\right\}
\eq
where $\cos\theta \geq 0$ for points in $Q_1$ or $Q_4$ and $\sin\theta \geq 0$ for points in $Q_1$ or $Q_2$.  

From here, we construct approximations of second directional derivatives (and first directional derivatives for the usual coordinate directions $(1,0)$ and $(0,1)$) of the form
\bq\label{eq:dnu}
\begin{split}
\Dt_{\nu\nu}u(x_0) &= \sum\limits_{j=1}^4 a_j(u(x_j)-u(x_0)) \approx \frac{\partial^2u(x_0)}{\partial\nu^2}\\
\Dt_{\nu}u(x_0) &= \sum\limits_{j=1}^4 b_j(u(x_j)-u(x_0)) \approx \frac{\partial u(x_0)}{\partial\nu}.
\end{split}
\eq
The coefficients in these finite difference approximations are given explicitly by
\bq\label{eq:coeffs}
\begin{split}
a_{1} &= \frac{2\sin\theta_4(\cos\theta_3\sin\theta_2-\cos\theta_2\sin\theta_3)}{r_1\det(A)}\\
a_{2} &= \frac{2\sin\theta_3(\cos\theta_1\sin\theta_4-\cos\theta_4\sin\theta_1)}{r_2\det(A)}\\
a_{3} &= \frac{-2\sin\theta_2(\cos\theta_1\sin\theta_4-\cos\theta_4\sin\theta_1)}{r_3\det(A)}\\
a_{4} &= \frac{-2\sin\theta_1(\cos\theta_3\sin\theta_2-\cos\theta_2\sin\theta_3)}{r_4\det(A)}\\
b_{1} &= \frac{\sin\theta_4 (r_2\sin\theta_3 \cos^2\theta_2 - r_3\sin\theta_2 \cos^2\theta_3)}{r_1\det(A)} \\
b_{2} &= -\frac{\sin\theta_3 (r_1\sin\theta_4 \cos^2\theta_1 - r_4\sin\theta_1 \cos^2\theta_4)}{r_2\det(A)} \\
b_{3} &= \frac{\sin\theta_2 (r_1\sin\theta_4 \cos^2\theta_1 - r_4\sin\theta_1 \cos^2\theta_4)}{r_3\det(A)} \\
b_{4} &= -\frac{\sin\theta_1 (r_2\sin\theta_3 \cos^2\theta_2 - r_3\sin\theta_2 \cos^2\theta_3)}{r_4\det(A)}
\end{split}
\eq
where 
\bq\label{eq:detA}\begin{split}\det(A) = &(\cos\theta_3\sin\theta_2-\cos\theta_2\sin\theta_3)(r_1\cos^2\theta_1\sin\theta_4-r_4\cos^2\theta_4\sin\theta_1)\\&-(\cos\theta_1\sin\theta_4-\cos\theta_4\sin\theta_1)(r_3\cos^2\theta_3\sin\theta_2-r_2\cos^2\theta_2\sin\theta_3).\end{split}\eq

Equation~\eqref{eq:PDEReformulated} contains several functions of the gradient.  We introduce the shorthand notation
\bq\label{eq:gradFuns}
g_1(p;\nu) = \left. -2\Dt_{\nu\nu}\log\norm{x_0-y}\right|_{y=T(x_0,p)}, \quad g_2(p) = \frac{\left(\norm{p}^2+1\right)^2}{4f_2(T(x_0,p))},
\eq
denote by $L_g$ the Lipschitz constant of the function $g$, and for each function define the small parameter
\bq\label{eq:epsilon}
\epsilon_g = L_g\max\limits_{j=1,\ldots,4}\frac{\abs{b_j}}{\abs{a_j}} = \bO(\sqrt{h}).
\eq
Then all functions of the gradient can be discretized using a Laplacian regularization via
\bq\label{eq:discGrad}
g^{\pm}\left(\nabla^h u(x_0)\right) = g\left(\Dt_{(1,0)}u(x_0),\Dt_{(0,1)}u(x_0)\right) \mp \epsilon_g\left(\Dt_{(1,0),(1,0)}u(x_0) + \Dt_{(0,1),(0,1)}u(x_0)\right).
\eq

This regularization allows for the construction of a monotone scheme, which is necessary for the convergence theorem in~\cite{HT_OTonSphere}. Finally, we can combine these different operators to obtain the approximation
\bq\label{eq:approx}
\begin{split}
F^h&(x_0,u;f_1,f_2) = \\ &\min\limits_{\{\nu_1,\nu_2\}\in V} \prod\limits_{j=1}^2 \max\left\{\Dt_{\nu_j\nu_j}u(x_0) + g^-_{1,\nu_j}(\nabla^h u(x_0)), 0\right\} - f_1(x_0)g_2^+\left(\nabla^h u(x_0)\right).
\end{split}
\eq



\begin{remark}
The method of~\cite{HT_OTonSphere2} in principal involves solving a problem with this approximation, verifying that the solution satisfies required Lipschitz bounds, then if necessary solving a second discrete problem to enforce the Lipschitz condition.  However, we have never seen the verification step fail in practice, and hence never actually need to solve a second discrete system. 
\end{remark}

\subsection{Computational Complexity}
Let $N$ be the total number of grid points.  
At each point $x_0\in\G$, evaluating the operator $F^h$ involves computing a minimum over the $\bO\left(1/d\theta\right) = \bO\left(1/\sqrt{h}\right)=\bO\left(N^{1/4}\right)$ pairs of vectors in $V$.

Each pair of vectors $\{\nu_1,\nu_2\}\in V$ requires the construction of two finite difference operators of the form $\Dt_{\nu\nu}$.  Computing each of these requires identifying the four neighbors $x_1, x_2, x_3, x_4$ in the stencil.

We note that selecting each of these neighboring points $x_j$ as in~\eqref{eq:neighbours} involves searching a region whose area scales like $\bO(h^2)$.  From the definition of $h$, this is guaranteed to contain at least one point, and expected to contain $\bO(1)$ points total. Thus identification of these four neighboring points can be done in $\bO(1)$ time.

Thus, given a grid function $u$, the total computational cost of evaluating the operator $F^h$ at all points in the grid is $\bO\left(N^{5/4}\right)$.

\subsection{Preprocessing of data}\label{preprocessing}
Stability and convergence of the numerical method requires at least one of the densities (denoted by $f_2$) to be strictly positive.  This is easily accomplished by choosing $\epsilon>0$ and letting
%\bq\label{eq:f2pos}
%\tilde{f}_2^\epsilon = \frac{\max\{f_2,\epsilon\}}{ \avint_{\Sf}\max\{f_2,\epsilon\}\,dS}.
%\eq
%although sometimes this will be accomplished by the following regularization:
\bq\label{eq:f2pos}
\tilde{f}_2^\epsilon = (1-\epsilon)f_2 + \frac{\epsilon}{4\pi}.
\eq
As $\epsilon\to0$, the mapping of the regularized optimal transport problem converges in measure to the solution of the given problem~\cite{Villani1}, and thus we recover the desired reflector surface.

The numerical method further requires this density function to be smoothed in order to have a (discrete) Lipschitz constant that is at most $\bO\left(h^{-1/4}\right)$.  We accomplish this via a short-time evolution of the heat equation.  That is, we solve
\bq\label{eq:heat}
\begin{cases}
v_t(x,t) = \Delta v(x,t), &(x,t)\in\Sf\times(0,\sqrt{h}]\\
v(x,0) = \tilde{f}_2^\epsilon(x), &x\in\Sf
\end{cases}
\eq
where $\Delta$ is the Laplace-Beltrami operator.  We then set
\bq\label{eq:f2reg}
f_2^\epsilon(x) = v(x,\sqrt{h}).
\eq

The Laplace-Beltrami operator can be discretized using the finite difference schemes~\eqref{eq:dnu} as
\bq\label{eq:Laplace}
\Delta^h = \Dt_{(1,0),(1,0)} + \Dt_{(0,1),(0,1)}
\eq
and evolved using forward Euler
\bq\label{eq:heatDisc}
v^{n+1} = v^n + k \Delta^hv^n.
\eq
The wide stencil nature of the finite difference stencils ($\norm{x_j-x_0} = \bO(\sqrt{h})$) means that this is stable for a time step $k \leq 1/\sum\limits_j a_j = \bO(h)$.  Thus a total of $\bO\left(h^{-1/2}\right)$ time steps are needed, which leads to an overall cost of $\bO\left(N^{5/4}\right)$ that is similar to the cost of discretization.

This regularization procedure can also be applied to unbounded densities, but requires evolving the heat equation to a stopping time of $t=h^{1/6}$ to achieve the required Lipschitz bound.

\subsection{Parabolic solvers}
After discretization, we are left with the task of solving the nonlinear algebraic system
\bq\label{eq:system}
F^h(x,u;f,g) = 0, \quad x \in \G.
\eq
Here, we use an explicit  parabolic scheme of the form
\bq\label{eq:parabolic}
u_{n+1}^h(x) = u_n^h(x) + k F^h(x,u^h_n;f,g).
\eq

As discussed in~\cite{ObermanSINUM}, we can require the time step $k$ to satisfy a nonlinear CFL condition in order to guarantee convergence. In particular, choosing $k < 1/L_{F^{h}} = \bO(h^{-2})$ is sufficient, where $L_{F^{h}}$ is the Lipschitz constant of $F^h$ with respect to the arguments $u^h$. However, in practice these parabolic schemes are sped up using techniques from~\cite{SchaefferHou}, which allows for potentially much larger time steps to be chosen on the fly and preserves convergence guarantees.



%Finally, the solutions of the Optimal Transport PDE satisfy \emph{a priori} Lipschitz bounds $\norm{\nabla u} < R$ for any $R>C$ (see~\cite[Proposition~6.1]{Loeper_OTonSphere} for details).  These bounds can be explicitly built into the PDE through a further modification
%\begin{equation}\label{eq:modifiedPDE}
%G(x,\nabla u(x),D^2u(x)) \equiv \max \left\{ F^{+}(x,\nabla u(x),D^2u(x)), \norm{ \nabla u(x) } - R \right\} = 0.
%\end{equation}
%While this modification enforces a condition that is automatically satisfied by solutions at the continuous level, it also improves the stability of approximation schemes.

%\subsubsection{Convergence results}

%The computational cost of constructing the discretization is $\mathcal{O}(N^{3/2})$. The discrete operator $G^h_i$ at a point $x_i$ involves discrete differential operators which are functions of the maximum operator over all points in computational neighborhoods. That is, at the point $x_i$, we have a computational neighborhood $N_i$, which has radius $\mathcal{O}(\sqrt{h})$ and the operator $G^h_i$, depends on values at $x_j \forall j \in N_i$. Each neighborhood $N_i$ contains $\mathcal{O}(h) \cdot N = \mathcal{O}(\sqrt{N})$ points. Then, since the discretization $G^h$ is an algebraic function of these terms, the computational cost of constructing the discretization is $N \cdot \mathcal{O}(\sqrt{N}) = \mathcal{O}(N^{3/2})$.
%
%
%
%
%
%Since the regularization of $f_2$ is a precomputation, the overall computational complexity of the discretization is still $\mathcal{O}(N^{3/2})$.


\section{Computational Results}\label{sec:numerics}
Here we demonstrate the effectiveness of our method with several computational examples. These include reflector design problems involving an omnidirectional source, discontinuous intensity distributions, and intensity distributions supported on sets with complicated geometries.  In each example, we use Algorithm~\ref{alg:reflector} to construct an approximate reflector $\Sigma^h$.  

%Two of the examples involve source and target masses on the entire sphere, which is not entirely realistic as far as the engineering of a reflector antenna is concerned. However, it demonstrates the ability of the numerical scheme to deal with complicated examples with full support on the sphere. 

%\begin{enumerate}
%\item[1.] The ``globe'' example, involves a discontinuous source density of a map of the world $f_1$ to a constant target mass $f_2$.
%\item[2.] The ``peanut reflector'' example, is derived from the ideal headlight intensity pattern of a vehicle and is inspired by the work done in~\cite{RomijnSphere}. The source intensity $f_1$ is the headlight intensity pattern mapped onto the sphere and the target intensity $f_2$ is a constant.
%%\item[3.] The ``smooth'' example, is a simpler computation of a smooth source mass $f_1$ concentrated at two ends of the sphere and a target mass density $f_2$ which is a constant. 
%%\item[4.] The ``disjoint'' example, shows a source and target masses located in the northern and southern hemispheres, respectively. They have disjoint support (except for the fact that $f_2 \geq \epsilon>0$, for small $\epsilon$ as stipulated by the optimal transport theory~\cite{Loeper_OTonSphere}. This example, of course, corresponds more naturally to the actual optical setup.
%\item[3.] The ``donut'' example, where a source and target densities approximate the situation where both have support on sets that are non $c$-convex (except, again that $f_2\geq \epsilon>0$).
%\item[4.] The ``triangle'' example, where a geodesic triangle source gets mapped to a smoothened target mass supported on the northern hemisphere
%\end{enumerate}

%Here, the numerical solution $u^h$ is computed and used to calculate the numerical reflector height function $\rho^h$, which is then used to construct the numerical reflector shape $\Sigma^h = x_i \rho^h$. To reemphasize, convergence is guaranteed for $u^h$ and therefore $\rho^h$ as $h \rightarrow 0$.


In order to validate our results, we first use the law of reflection~\eqref{eq:mapping2} to perform approximate (forward or inverse) ray-tracing.  We then construct the resulting intensity patterns via approximation of the conservation of energy equation equation~\eqref{eq:conservation} by \[f_1(x_i) \Delta x_i \approx f_2(y_i) \Delta y_i,\]\label{econs} where $\Delta x_i$ and $\Delta y_i$ are the areas of the Voronoi regions containing $x_i$ and $y_i = T(x_i)$ respectively. 
%This provides an approximate ray traces which can be used to validate the performance of the scheme, in the absence of convergence guarantees for $T^h$. 

After performing ray tracing, the presence of numerical artifacts may require that the data be post-processed to show the results clearly. This is done by rescaling the colorbars to cut off a very small number of the highest values. Any numerical artifacts are presented in plots of the difference between the desired and ray-traced intensities.
%The presence of numerical artifacts for $T^h$ is expected, since the ideal result would be $T^h \rightarrow T$ in measure as in standard optimal transport theory~\cite{Villani1}, and a future objective is to develop a framework to derive convergence results for the map $T^h$, as it has a lot of importance in some applications beyond just validation, see~\cite{HT_OTonSphere, HT_OTonSphere}.

All computations were performed on a 13-inch MacBook Pro, 2.3 GHz Intel Core i5 with 16GB 2133 MHz LPPDDR3 using Matlab R2017b. Each computation utilized around $N\approx 20,000$ points on the sphere.  Where applicable, regularization was performed using $\epsilon=0.3$. The precomputation step of approximating all directional derivatives for $N\approx20,000$ points took about $10$ minutes. Solving the parabolic scheme to find the solution took around $30$ minutes. Ongoing work will develop faster, more accurate versions of this method.  We see therefore that the proposed numerical method can certainly accommodate higher precision computations if necessitated by real-world applications.

\subsection{Peanut Reflector}

Following the example of~\cite{RomijnSphere}, we consider a source density coming from an ideal headlight intensity emitting from a vehicle's high beams. This headlight intensity pattern is then mapped to the sphere, and inverted, which becomes the source intensity $f_1$. The target density $f_2$ is constant. 
%Here we do not show the ray trace inverse before rescaling as we did for the globe example. 
The computation yields a peanut-shaped oblong reflector lens; see Figure~\ref{fig:peanutresult}. Despite the fact that we anticipate error in the reverse ray trace due to the approximate conservation of energy equation~\eqref{econs}, we see that the absolute error performs quite well in this smooth example.  The average error in the reconstruction is 11\% of the maximum intensity.

%\begin{figure}
%	\subfigure[Peanut lens in 2D showing the headlight pattern]{\includegraphics[height=5.2cm]{peanut_lens}}
%	\subfigure[Peanut lens as mapped to the sphere]{\includegraphics[height=5.2cm]{peanutlens}}
%	\caption{Peanut Lens in 2D and on the sphere}\label{fig:peanutlens}
%\end{figure}

\begin{figure}
	\subfigure[Headlight intensity $f_1$ to constant intensity $f_2$]{\includegraphics[width=0.45\textwidth]{peanutlensfandg}}
	%\subfigure[Numerical solution $u^h$]{\includegraphics[height=5.7cm]{peanutsolution}}
		\subfigure[Computed reflector]{\includegraphics[width=0.45\textwidth]{peanutreflector2}}
	\subfigure[Inverse ray-traced intensity]{\includegraphics[width=0.45\textwidth]{peanutraytraceinverse}}
	\subfigure[Difference between $f_1$ and inverse ray-traced intensity, with average error of $0.0137$.]{\includegraphics[width=0.45\textwidth]{peanuterror}}
	\caption{``Peanut" reflector}\label{fig:peanutresult}
\end{figure}

\subsection{Discontinuous intensities}
Next, we demonstrate the effectiveness of our method in dealing with discontinuities and complicated densities. In this example, a discontinuous source mass $f_1$ resembling an inverted map of the world is mapped to a constant density $f_2$; see Figure~\ref{fig:globe}. This is a particularly challenging example given the very complicated structure of the discontinuities.  Nevertheless, we achieve a reconstruction that visually agrees with the world map, with an average error of 19\% of the maximum intensity.

\begin{figure}
	\subfigure[Intensities $f_1$ and $f_2$]{\includegraphics[width=0.45\textwidth]{fandgglobe}}
	%\subfigure[Solution $u^h$]{\includegraphics[height=4.7cm]{globe14261}}
	\subfigure[Computed reflector]{\includegraphics[width=0.45\textwidth]{globereflector3}}
	%\subfigure[Globe inverse ray trace showing presence of a few negligible numerical artifacts before rescaling]{\includegraphics[height=4.5cm]{globeafter1}}
	\subfigure[Inverse ray-traced intensity]{\includegraphics[width=0.45\textwidth]{globeraytraceinterp}}
	\subfigure[Difference between $f_1$ and inverse ray-traced intensity, with average $L^1$ error of $0.0206$]{\includegraphics[width=0.45\textwidth]{globeerror}}
	\caption{Discontinuous intensities}\label{fig:globe}
\end{figure}




%\subsection{Smooth example}

%Here we perform computations for the case where the source and target densities are smooth, so the solution to the PDE~\eqref{eq:OTPDE} is known to be \textit{a priori} smooth. The densities chosen are:

%\begin{align*}
%f_1(x, y, z) &= \frac{1-\epsilon}{5.8735} \left( \arccos(x) - \pi/2 \right)^2 + \frac{\epsilon}{4\pi} \\
%f_2(x,y,z) &= \frac{1}{4\pi}
%\end{align*}
%where $\epsilon = 0.6$. The resulting figures are shown in the figure~\ref{fig:smoothresult}:

%\begin{figure}
%	\subfigure[Smooth density $f_1$ to constant density $f_2$]{\includegraphics[height=5.8cm]{smoothfandg}}
%	\subfigure[Numerical solution $u^h$]{\includegraphics[height=5.9cm]{smoothsolution}}
%	\subfigure[Rescaled inverse ray traced density $f_1$]{\includegraphics[height=5.2cm]{smoothraytraceinverse1}}
%	\subfigure[Shape of the reflector]{\includegraphics[height=5.7cm]{smoothreflector2}}
%	\subfigure[$L^{\infty}$ error betweeen $f_1$ and the inverse ray trace]{\includegraphics[height=5.7cm]{smootherror}}
%	\caption{Smooth example computation}\label{fig:smoothresult}
%\end{figure}


%\subsection{Disjoint Support}

%Inspired by the original reflector antenna problem, we compute the case where the source and target masses have ``disjoint" support and are supported in the northern and southern hemispheres, respectively. Of course, our convergence framework and numerical scheme only have convergence guarantees if the target mass is bounded strictly away from zero, so there is a very small amount of mass added to account for this. The density functions used are:

%\begin{equation}
%f_1(x, y, z) =
%\begin{cases}
%\frac{1}{2\pi}, & z\geq 0 \\
%0, & \text{otherwise}
%\end{cases}
%\end{equation}
%and

%\begin{equation}
%f_2(x, y, z) =
%\begin{cases}
%\frac{\epsilon}{4\pi}, & z\geq 0 \\
%\frac{\epsilon-1}{\pi}z + \frac{\epsilon}{4 \pi} & \text{otherwise}
%\end{cases}
%\end{equation}
%where $\epsilon = 0.3$. The resulting figures are shown in the figure~\ref{fig:disjointresult}:

%\begin{figure}
%	\subfigure[Mapping ``disjoint" source and target masses]{\includegraphics[height=5cm]{disjointfandg}}
%	\subfigure[Forward ray trace rescaled to account for small amount of numerical artifacts]{\includegraphics[height=4.7cm]{disjointraytrace2}}
%	\subfigure[$L^{\infty}$ error between $f_2$ and the forward ray trace]{\includegraphics[height=4.1cm]{disjointerror2}}
%	\subfigure[Shape of the reflector]{\includegraphics[height=4.9cm]{disjointreflector2}}
%	\caption{``Disjoint" source and target mass}\label{fig:disjointresult}
%\end{figure}


\subsection{Donut intensities}
To further demonstrate the flexibility of our method, we consider the source and target intensities propagating in a donut shape, with a dark region in the center.  These are given by
\begin{equation}
f_1(x,y,z) =
\begin{cases}
\frac{1}{(4\pi/15)(\sqrt{2} + 2)} \left(-4\sqrt{x^2 + y^2} z^3 + 4(x^2 + y^2)^{3/2}z \right), & \sqrt{2}/2 \geq z\geq 0 \\
0, & \text{otherwise}
\end{cases}
\end{equation}
and
\begin{equation}
f_2(x,y,z) =
\begin{cases}
\frac{1}{(4\pi/15)(\sqrt{2} + 2)} \left(-4\sqrt{x^2 + y^2} z^3 + 4(x^2 + y^2)^{3/2}z \right), & 0 \geq z \geq -\sqrt{2}/2 \\
0, & \text{otherwise}
\end{cases}
\end{equation}

These intensities have very complicated support containing holes, which is particularly challenging numerically.  Indeed, this challenge is inherent in the theory of the optimal transport problem.  We note that the $c$-convexity constraint~\eqref{eq:cconvex} requires the domain $\Omega$ to be $c$-convex in order to guarantee construction of the physically relevant solution of the PDE~\eqref{eq:OTPDE}.  Consequently, PDE based methods that are posed only on the support $\Omega$ of the intensity (rather than being extended into the dark regions) will not be assured of producing the correct reflector.  This issue is handled naturally by our method, which is posed on the entire sphere.  Despite the difficulty of this example, our method performs very well, as evidenced in the results of the ray-tracing.  See Figure~\ref{fig:donutresult}.  Average error is 9\% of the maximal intensity.

\begin{figure}
	\subfigure[Intensities $f_1$ and $f_2$]{\includegraphics[width=0.45\textwidth]{donuts2fandg}}
	%\subfigure[Solution $u^h$]{\includegraphics[height=4.4cm]{donuts2solution}}
	\subfigure[Computed reflector]{\includegraphics[width=0.45\textwidth]{donuts2reflector}}
	\subfigure[Forward ray-traced intensity]{\includegraphics[width=0.45\textwidth]{donuts2raytrace}}
	\subfigure[Difference between $f_2$ and forward ray-traced intensity expressed as a percentage of the maximum of $f_2$. Average $L^1$ error of $0.0304$]{\includegraphics[width=0.45\textwidth]{donuts2error}}
	\caption{``Donut" intensities}\label{fig:donutresult}
\end{figure}


\subsection{Singular reflector}
We conclude with an example of a hemispheric light source (here designated as $f_2$) that is to be reshaped into a geodesic triangle on the sphere (here designated as $f_1$).  We remark that given the complicated (non $c$-convex) support of this target, we are not even guaranteed the existence of a smooth ($C^1$) reflector; see~\cite{Loeper_OTonSphere}. 

The intensities are defined as follows.  We begin by forming a geodesic triangle $T_{\theta} \subset \Sf$ from the three vertices $(t_{0, \theta}, t_{1, \theta}, t_{2, \theta})$, where we define $t_{j, \theta} = \left( \sin \theta \cos (2\pi j/3), \sin \theta \sin (2\pi j/3), \cos \theta \right)$ for $\pi/2\leq \theta <\pi$. The geodesic triangle is formed by the small region enclosed by the three vertices $t_i$, which are connected by geodesics on the sphere. That is, a point $x_0 \in T_{\theta}$ if $x_0$ satisfies the following three inequalities:
\begin{align*}
x_0 \cdot \left( t_{1,\theta} \times t_{2,\theta} \right) &\leq 0 \\
x_0 \cdot \left( t_{2,\theta} \times t_{3,\theta} \right) &\leq 0 \\
x_0 \cdot \left( t_{3,\theta} \times t_{1,\theta} \right) &\leq 0
\end{align*}

Then the triangular intensity is defined by
\begin{equation}
f_1(x,y,z) = 
\begin{cases}
1/A , \ \ \ &(x,y,z) \in T_{\theta} \\
0, &(x,y,z) \notin T_{\theta}
\end{cases}
\end{equation}
where $A$ is the area of the geodesic triangle $T_{\theta}$ and $\theta = 2.1$.

The second intensity is a smoothed version of the identity function on the northern hemisphere:
\begin{equation}
f_2(x,y,z) = 
\begin{cases}
\frac{2\pi  \log \left( \cosh(a) \right)}{a}\tanh(a z), &z \geq 0 \\
0, & z<0
\end{cases}
\end{equation}
where $a=10$.

For ease of implementation, we perform pre-processing to bound both $f_1$ and $f_2$ away from zero.

%\begin{equation}
%f_1(x,y,z) = 
%\begin{cases}
%(1-\epsilon)/A , \ \ \ &(x,y,z) \in T_{\theta} \\
%\epsilon/4\pi, &(x,y,z) \notin T_{\theta}
%\end{cases}
%\end{equation}

%\begin{equation}
%f_2(x,y,z) = 
%\begin{cases}
%\frac{2\pi (1-\epsilon) \log \left( \cosh(a) \right)}{a}\tanh(a z), z \geq 0 \\
%\epsilon/4\pi, & z<0
%\end{cases}
%\end{equation}
%for $\epsilon = 0.3$ and $a = 10$. We get the resulting figures~\ref{fig:triangle}

Results are presented in Figure~\ref{fig:triangle}.  In the computed reflector, and resulting ray-traced intensity, we observe an approximate triangle shape as expected. In this case, there are notable artifacts present near the boundary of the triangle. However, to some extent these are a limitation of the physics rather than of our method.  We remark that there is no reason to expect the reflector we are approximating to be continuously differentiable, so the accuracy of the ray-tracing verification test is itself rather suspect here.  Nevertheless, the absolute error as compared with the ray trace from the approximate conservation of energy equation mostly performs well, with an average error of 16\% of the maximal intensity.

In a challenging problem like this, where the physics itself may not allow for the existence of a reflector with nice properties (from the perspective of manufacturing and outcome), it may also be useful to view our method as a robust way of obtaining a good approximation of the desired reflector.  This could then be used to initialize an end-game method, not based on optimal transport, that would optimize the reflector surface and enforce any desired smoothness. 

\begin{figure}
	\subfigure[Intensities $f_1$ and $f_2$ from below]{\includegraphics[width=0.49\textwidth]{triangle2bottom}}
	\subfigure[Intensities $f_1$ and $f_2$ from side]{\includegraphics[width=0.49\textwidth]{triangle2fandg}}
	\subfigure[Solution $u^h$]{\includegraphics[width=0.45\textwidth]{triangle2solution}}
	\subfigure[Computed reflector]{\includegraphics[width=0.45\textwidth]{triangle2reflector}}
	\subfigure[Forward ray-traced intensity]{\includegraphics[width=0.45\textwidth]{triangle3raytrace}}
	\subfigure[Difference between $f_2$ and forward ray-traced intensity expressed as a percentage of the maximum of $f_2$. Average $L^1$ error of $0.0288$]{\includegraphics[width=0.45\textwidth]{triangle3error}}
	\caption{Singular reflector}\label{fig:triangle}
\end{figure}




\section{Conclusion}\label{sec:conclusion}
We have introduced a new numerical method for solving the reflector antenna design problem.  The method is based on the reformulation of this design problem as an optimal transport problem on the sphere.  This allows the reflector to be described in terms of the solution to a fully nonlinear elliptic PDE of \MA type, posed on the unit sphere.  We describe a provably convergent finite difference method for solving this PDE, which in turn guarantees that the method will correctly approximate the desired reflector.  The method is robust: convergence guarantees hold even for non-smooth data and reflectors.

We validate this new method through several challenging examples, which include intensities that have complicated discontinuities, that propagate over complicated geometries, or that contain a mix of light and dark regions.  The method performs well even in a final example where the physics does not guarantee the existence of a smooth ($C^1$) reflector.

This new finite difference method provides a rigorous foundation upon which faster and more accurate solvers can be designed.  The idea of pairing slower, more robust approximations (to be used in the most singular regions of the domain) with more traditional high-order methods has been successfully applied to the \MA equation in Euclidean space~\cite{FO_FilteredSchemes}.  In the future, we hope to adapt these techniques to the reflector antenna problem in order to produce higher-quality approximations to the desired reflector surface.



\bibliographystyle{plain}
\bibliography{OTonSphere3}

%{\appendix
%\onecolumn


% \tableofcontents{}

% \newpage

\section*{Supplementary Material}
\addcontentsline{toc}{section}{Supplementary Material}


Throughout this discussion, 
we will make frequently use 
of the following standard results
concerning the exponential concentration 
of random variables:

\begin{lemma}[Hoeffding's inequality for independent RVs~\citep{hoeffding1994probability}] Let $Z_1, Z_2, \ldots, Z_n$ be independent bounded random variables with $Z_i \in [a,b]$ for all $i$, then 
    \begin{align*}
        \prob\left( \frac{1}{n} \sum_{i=1}^n (Z_i - \Expo{Z_i}) \ge t \right) \le \exp{\left( -\frac{2nt^2}{(b-a)^2} \right) }
    \end{align*} 
    and 
    \begin{align*}
        \prob\left( \frac{1}{n} \sum_{i=1}^n (Z_i - \Expo{Z_i}) \le -t \right) \le \exp{\left( -\frac{2nt^2}{(b-a)^2} \right) }
    \end{align*} 
    for all $t \ge 0$. 
\end{lemma}

\begin{lemma}[Hoeffding's inequality for sampling with replacement~\citep{hoeffding1994probability}] \label{lem:hoeffding_sampling} Let $\calZ = (Z_1, Z_2, \ldots, Z_N)$ be a finite population of $N$ points with $Z_i \in [a.b]$ for all $i$. Let $X_1, X_2, \ldots X_n$ be a random sample drawn without replacement from $\calZ$. Then for all $t \ge 0$, we have 
    \begin{align*}
        \prob\left( \frac{1}{n} \sum_{i=1}^n (X_i - \mu ) \ge t \right) \le \exp{\left( -\frac{2nt^2}{(b-a)^2} \right) }
    \end{align*} 
    and 
    \begin{align*}
        \prob\left( \frac{1}{n} \sum_{i=1}^n (X_i - \mu ) \le -t \right) \le \exp{\left( -\frac{2nt^2}{(b-a)^2} \right) } \,,
    \end{align*} 
    where $\mu = \frac{1}{N} \sum_{i=1}^{N} Z_i$. 
\end{lemma}

We now discuss one condition that generalizes the exponential concentration to dependent random variables.
\begin{condition}[Bounded difference inequality] \label{cond:BDC} Let $\calZ$ be some set and $\phi: \calZ^n \to \Real$. We say that $\phi$ satisfies the bounded difference assumption if 
there exists $c_1, c_2, \ldots c_n \ge 0$ s.t. for all $i$, we have 
\begin{align*}
    \sup_{Z_1,Z_2, \ldots,Z_n, Z_i^\prime \in \calZ^{n+1} } \abs{\phi (Z_1, \ldots, Z_i, \ldots, Z_n ) - \phi (Z_1, \ldots, Z_i^\prime, \ldots, Z_n ) } \le c_i \,.
\end{align*} 
\end{condition}

\begin{lemma}[McDiarmid’s inequality~\citep{mcdiarmid1989}] \label{lem:McDiarmid} Let $Z_1, Z_2, \ldots, Z_n$ be independent random variables on set $\calZ$ and $\phi : \calZ^n \to \Real$ satisfy bounded difference inequality (\codref{cond:BDC}). Then for all $t>0$, we have 
    \begin{align*}
        \prob\left( \phi(Z_1, Z_2, \ldots, Z_n) - \Expo{\phi(Z_1, Z_2, \ldots, Z_n)} \ge t \right) \le \exp{\left( -\frac{2t^2}{\sum_{i=1}^n c_i^2} \right) } 
    \end{align*} 
    and 
    \begin{align*}
        \prob\left( \phi(Z_1, Z_2, \ldots, Z_n) - \Expo{\phi(Z_1, Z_2, \ldots, Z_n)} \le -t \right) \le \exp{\left( -\frac{2t^2}{\sum_{i=1}^n c_i^2} \right) } \,.
    \end{align*} 
\end{lemma}


\section{Proofs from \secref{sec:ERM_training}}\label{app:proof_erm}

\textbf{Additional notation {} {}} Let $m_1$ be the number of mislabeled points ($\wt S_M$) and $m_2$ be the number of correctly labeled points ($\wt S_C$). Note $m_1 + m_2 = m$. 


\subsection{Proof of \thmref{thm:error_ERM}}


\begin{proof}[Proof of \lemref{lem:fit_mislabeled}] 
    The main idea of our proof is to regard 
    the clean portion of the data 
    ($S \cup \wt S_C$) as fixed.   
    Then, there exists an (unknown) classifier $f^*$ 
    that minimizes the expected risk
    calculated on the (fixed) clean data
    and (random draws of) the mislabeled data $\wt S_M$. 
    % 
    % 
    Formally, 
    \begin{align}
    f^* \defeq \argmin_{f \in \calF} \error_{\widecheck {\calD}} (f) \,, \label{eq:modified_ERM}
    \end{align}
    where $$\widecheck \calD = \frac{n}{m+n} \calS + \frac{m_2}{m+n} \wt \calS_C  + \frac{m_1}{m+n}\calDm \,.$$ 
    Note here that $\widecheck \calD$ is a combination 
    of the \emph{empirical distribution} 
    over correctly labeled data $S \cup \wt S_C$
    and the (population) distribution 
    over mislabeled data $\calDm$.
    Recall that 
    \begin{align}
    \wh f \defeq \argmin_{f \in \calF} \error_{\calS \cup \wt S} (f) \,. \label{eq:orig_ERM}
    \end{align}
    % 
    % 
    Since, $\widehat f$ minimizes 0-1 error 
    on $S \cup \wt S$, using ERM optimality on \eqref{eq:orig_ERM},  
    we have 
    \begin{align}
        \error_{\calS \cup \wt \calS}(\widehat f) \le \error_{
            \calS \cup \wt \calS}(f^*) \,.    \label{eq:step1}
    \end{align}
    Moreover, since $f^*$ is independent of $\wt S_M$, using Hoeffding's bound,
    % \footnote{For a fully rigorous argument,
    % refer to the complete proof in App.~\ref{app:proof_erm}.} 
    we have with probability at least $1-\delta$ that
    \begin{align}
      \error_{\wt \calS_M}(f^*) \le \error_{ \calDm}(f^*) +  \sqrt{\frac{\log(1/\delta)}{2 m_1}} \,. \label{eq:step2} 
    \end{align}
    %$ 
    %for some constant $c_1\le 1/2$. 
    Finally, since $f^*$ is the optimal classifier on $\widecheck \calD$, 
    we have 
    \begin{align}
        \error_{\widecheck \calD}(f^*) \le \error_{\widecheck \calD}(\widehat f) \,. \label{eq:step3}
    \end{align}
    Now to relate \eqref{eq:step1} and \eqref{eq:step3}, we multiply \eqref{eq:step2} by $\frac{m_1}{m+n}$ and add $\frac{n}{m+n} \error_{\calS} (f)  + \frac{m_2}{m+n} \error_{\wt \calS_C} (f)$ both the sides. Hence, 
    we can rewrite \eqref{eq:step2} as follows: 
    \begin{align}
        \error_{\calS \cup \wt\calS}(f^*) \le \error_{ \widecheck \calD}(f^*) +  \frac{m_1}{m+n}\sqrt{\frac{\log(1/\delta)}{2 m_1}} \,. \label{eq:step4} 
    \end{align}
    Now we combine equations \eqref{eq:step1}, \eqref{eq:step4}, and \eqref{eq:step3}, to get 
    \begin{align}
        \error_{\calS \cup \wt \calS}(\wh f) \le \error_{\widecheck \calD}(\wh f) +  \frac{m_1}{m+n}\sqrt{\frac{\log(1/\delta)}{2 m_1}} \,, 
    \end{align}
    which implies 
    \begin{align}
        \error_{ \wt \calS_M}(\wh f) \le \error_{\calDm}(\wh f) + \sqrt{\frac{\log(1/\delta)}{2 m_1}} \,. \label{eq:lemma1_final}
    \end{align}
    Since $\wt S$ is obtained by randomly labeling an unlabeled dataset, we assume $2m_1 \approx m$ \footnote{Formally, with probability at least $1-\delta$, we have  $(m - 2m_1)\le \sqrt{m\log(1/\delta)/2}$.}. Moreover, using $\error_{\calDm} = 1 - \error_{\calD}$ we obtain the desired result.   
    % Combining the above steps and using the fact 
    % that $\error_\calD = 1- \error_{\calDm} $, 
    % we obtain the desired result.
\end{proof}

\begin{proof}[Proof of \lemref{lem:mislabeled_error}]
    Recall $\error_{\wt S} (f) = \frac{m_1}{m} \error_{\wt S_M}(f) + \frac{m_2}{m} \error_{\wt S_C}(f)$. Hence, we have 
    \begin{align}
        2\error_{\wt S}(f) - \error_{\wt S_M}(f) - \error_{\wt S_C}(f) &= \left(\frac{2m_1}{m} \error_{\wt S_M}(f) - \error_{\wt S_M}(f)\right) + \left(\frac{2m_2}{m} \error_{\wt S_C}(f) - \error_{\wt S_C}(f)\right) \\ &= \left(\frac{2m_1}{m} - 1\right) \error_{\wt S_M}(f) + \left(\frac{2m_2}{m} - 1 \right)\error_{\wt S_C} (f) \,.
    \end{align} 
    Since the dataset is labeled uniformly at random, with probability at least $1-\delta$, we have  $\left(\frac{2m_1}{m} - 1\right) \le \sqrt{\frac{\log(1/\delta)}{2m}}$. Similarly, we have with probability at least $1-\delta$, $\left(\frac{2m_2}{m} - 1\right) \le \sqrt{\frac{\log(1/\delta)}{2m}}$. Using union bound, with probability at least $1-\delta$, we have
    % \begin{align}
    %     2\error_{\wt S} - \error_{\wt S_M}(f) - \error_{\wt S_C}(f) \le \sqrt{\frac{\log(2/\delta)}{2m}} \left(\error_{\wt S_M}(f) + \error_{\wt S_C}(f) \right) \le 2\sqrt{\frac{\log(2/\delta)}{2m}} \,. \label{eq:lemma2_final}
    % \end{align}
    \begin{align}
        2\error_{\wt S} - \error_{\wt S_M}(f) - \error_{\wt S_C}(f) \le \sqrt{\frac{\log(2/\delta)}{2m}} \left(\error_{\wt S_M}(f) + \error_{\wt S_C}(f) \right) \,. \label{eq:lemma2_prefinal}
    \end{align}
    With re-arranging $\error_{\wt S_M}(f) + \error_{\wt S_C}(f)$ and using the inequality $ 1- a\le \frac{1}{1+a} $, we have  
    \begin{align}
        2\error_{\wt S} - \error_{\wt S_M}(f) - \error_{\wt S_C}(f) \le 2\error_{\wt \calS} \sqrt{\frac{\log(2/\delta)}{2m}}  \,. \label{eq:lemma2_final}
    \end{align}

    % We obtain the desired result by using 
\end{proof}

\begin{proof}[Proof of \lemref{lem:clear_error}]
% Recall 0-1 error on each point  $(x,y) \in S \cup \wt S$ is given by $\I{ f(x)\ne y}$.
In the set of correctly labeled points $S \cup \wt S_C$, we have $S$ as a random subset of $S \cup \wt S_C$. Hence, using Hoeffding's inequality for sampling without replacement (\lemref{lem:hoeffding_sampling}), we have with probability at least $1-\delta$
\begin{align}
    \error_{\wt \calS_C} (\wh f)- \error_{\calS \cup \wt \calS_C}( \wh f) \le  \sqrt{\frac{\log(1/\delta)}{2m_2}} \,.
\end{align}
Re-writing $\error_{\calS \cup \wt \calS_C}( \wh f)$ as $\frac{m_2}{m_2 + n} \error_{\wt \calS_C }(\wh f) + \frac{n}{m_2 + n} \error_{\calS }(\wh f)$, we have with probability at least $1-\delta$
\begin{align}
   \left(\frac{n}{n+m_2}\right) \left(\error_{\wt \calS_C} (\wh f)- \error_{\calS}( \wh f) \right) \le  \sqrt{\frac{\log(1/\delta)}{2m_2}} \,.
\end{align}
As before, assuming $2m_2 \approx m$, we have with probability at least $1-\delta$ 
\begin{align}
    \error_{\wt \calS_C} (\wh f)- \error_{\calS}( \wh f) \le \left(1+\frac{m_2}{n}\right)  \sqrt{\frac{\log(1/\delta)}{m}} \le \left(1 + \frac{m}{2n}\right) \sqrt{\frac{\log(1/\delta)}{m}} \,. \label{eq:lemma3_final}
\end{align} 
\end{proof}

\begin{proof}[Proof of \thmref{thm:error_ERM}] 
    Having established these core intermediate results, we can now combine above three lemmas to prove the main result. 
    In particular, we bound the population error on clean data ($\error_\calD(\wh f)$) as follows:  
    \begin{enumerate}[(i)]
        \item First, use \eqref{eq:lemma1_final}, to obtain an upper bound on the population error on clean data, i.e., with probability at least $1-\delta/4$, we have
        \begin{align}
            \error_{ \calD} (\wh f) \le 1 - \error_{ \wt \calS_M}(\wh f) + \sqrt{\frac{\log(4/\delta)}{m}} \,. 
        \end{align}
        \item  Second, use \eqref{eq:lemma2_final}, to relate the error on the mislabeled fraction with error on clean portion of randomly labeled data and error on whole randomly labeled dataset, i.e., with probability at least $1-\delta/2$, we have 
        \begin{align}
            - \error_{\wt S_M}(f) \le \error_{\wt S_C}(f) - 2\error_{\wt S}  + 2\error_{\wt S} \sqrt{\frac{\log(4/\delta)}{2m}}  \,. 
        \end{align} 
        \item Finally, use \eqref{eq:lemma3_final} to relate the error on the clean portion of randomly labeled data and error on clean training data, i.e., with probability $1-\delta/4$, we have 
        \begin{align}
            \error_{\wt \calS_C} (\wh f)\le - \error_{\calS}( \wh f) + \left(1 + \frac{m}{2n} \right) \sqrt{\frac{\log(4/\delta)}{m}} \,. 
        \end{align} 
    \end{enumerate}

    Using union bound on the above three steps, we have with probability at least $1-\delta$: 
    \begin{align}
        \error_\calD (\wh f) \le \error_{\calS}(\wh f)   + 1 - 2\error_{\wt \calS}(\wh f)   + \left(\sqrt{2} \error_{\wt S} + 2 + \frac{m}{2n}\right)  \sqrt{\frac{\log(4/\delta)}{m}} \,.
    \end{align}
    % Note that $(1/\sqrt{2} + 2.5)$ is a loose constant. In experiments, we use the ratio $\frac{m}{n}$
    %  the exact error $\error_{\wt \calS}(\wh f)$ 
    % to evaluate R.H.S.    
\end{proof}

\subsection{Proof of \propref{prop:rademacher}}

\begin{proof}[Proof of \propref{prop:rademacher}]
    For a classifier $ f: \calX \to \{-1, 1\}$, we have $1 - 2\,\indict{ f(x) \ne y} = y \cdot f(x)$. Hence, by definition of $\error$, we have 
    \begin{align}
        1 -2\error_{\wt \calS}(f) = \frac{1}{m}\sum_{i=1}^m y_i \cdot f(x_i) \le \sup_{f \in \calF} \, \frac{1}{m} \sum_{i=1}^m y_i \cdot f(x_i)  \,. \label{eq:error_rademacher}
    \end{align}
    Note that for fixed inputs $(x_1, x_2, \ldots, x_m)$ in $\wt S$, $(y_1, y_2, \ldots y_m)$ are random labels. Define $\phi_1 (y_1, y_2, \ldots, y_m) \defeq \sup_{f \in \calF} \, \frac{1}{m} \sum_{i=1}^m y_i \cdot f(x_i)$. We have the following bounded difference condition on $\phi_1$. For all i, 
    \begin{align}
        \sup_{y_1, \ldots y_m, y_i^\prime \in \{-1, 1\}^{m+1} } \abs{ \phi_1 (y_1,\ldots, y_i, \ldots, y_m) - \phi_1 (y_1,\ldots, y_i^\prime, \ldots, y_m)  } \le 1/m \,. \label{cond1_rademacher}
    \end{align} 
    
    Similarly, we define $\phi_2 (x_1, x_2, \ldots, x_m) \defeq \Expt{ y_i \sim_U \{-1, 1\}  }{ \sup_{f \in \calF} \, \frac{1}{m}  \sum_{i=1}^m y_i \cdot f(x_i)}$. We have the following bounded difference condition on $\phi_2$. 
    For all i,
    \begin{align}
        \sup_{x_1, \ldots x_m, x_i^\prime \in \calX^{m+1} } \abs{ \phi_2 (x_1,\ldots, x_i, \ldots, x_m) - \phi_1 (x_1,\ldots, x_i^\prime, \ldots, x_m)  } \le 1/m \,. \label{cond2_rademacher}
    \end{align}
    Using McDiarmid’s inequality (\lemref{lem:McDiarmid}) twice 
    with Condition \eqref{cond1_rademacher} and \eqref{cond2_rademacher}, 
    with probability at least $1-\delta$, we have
    \begin{align}
        \sup_{f \in \calF} \, \frac{1}{m} \sum_{i=1}^m y_i \cdot f(x_i)  - \Expt{x,y}{\sup_{f \in \calF} \, \frac{1}{m} \sum_{i=1}^m y_i \cdot f(x_i) } \le \sqrt{\frac{2\log(2/\delta)}{m}} \,. \label{eq:final_rademacher}
    \end{align} 
    Combining \eqref{eq:error_rademacher} and \eqref{eq:final_rademacher}, we obtain the desired result. 
\end{proof}


\subsection{Proof of \thmref{thm:error_regularized_ERM}}

Proof of \thmref{thm:error_regularized_ERM} follows similar to the proof of \thmref{thm:error_ERM}. Note that the same results in \lemref{lem:fit_mislabeled}, \lemref{lem:mislabeled_error}, and \lemref{lem:clear_error} hold in the regularized ERM case. However, the arguments in the proof of \lemref{lem:fit_mislabeled} change slightly. Hence, we state the lemma for regularized ERM and prove it here for completeness. 

\begin{lemma} \label{lem:lemma1_reg}
    Assume the same setup as \thmref{thm:error_regularized_ERM}. 
    Then for any $\delta >0$, with probability at least  $1-\delta$ 
    over the random draws of mislabeled data $\wt S_M$, we have 
    \begin{align}
        \error_\calD(\widehat f)  \le 1 -\error_{\wt \calS_M}(\widehat f) + \sqrt{\frac{\log(1/\delta)}{m}}\,. 
    \end{align} 
\end{lemma}
\begin{proof}
    The main idea of the proof remains the same, i.e. regard 
    the clean portion of the data 
    ($S \cup \wt S_C$) as fixed.   
    Then, there exists a classifier $f^*$ 
    that is optimal over draws 
    of the mislabeled data $\wt S_M$. 

    
    Formally, 
    \begin{align}
    f^* \defeq \argmin_{f \in \calF} \error_{\widecheck {\calD}} (f)  + \lambda R(f) \,, \label{eq:modified_ERM_reg}
    \end{align}
    where $$\widecheck \calD = \frac{n}{m+n} \calS + \frac{m_1}{m+n} \wt \calS_C  + \frac{m_2}{m+n}\calDm \,.$$ That is, $\widecheck \calD$ a combination of 
    the \emph{empirical distribution} 
    over correctly labeled data $S \cup \wt S_C$
    % in $S\cup \wt S$ 
    and the (population) distribution 
    over mislabeled data $\calDm$.
    Recall that 
    \begin{align}
    \wh f \defeq \argmin_{f \in \calF} \error_{\calS \cup \wt S} (f) + \lambda R(f) \,. \label{eq:orig_ERM_reg}
    \end{align}
    % 
    % 
    Since, $\widehat f$ minimizes 0-1 error 
    on $S \cup \wt S$, using ERM optimality on \eqref{eq:orig_ERM},  
    we have 
    \begin{align}
        \error_{\calS \cup \wt \calS}(\widehat f) + \lambda R(\wh f) \le \error_{
            \calS \cup \wt \calS}(f^*) + \lambda R(f^*) \,.    \label{eq:step1_reg}
    \end{align}
    Moreover, since $f^*$ is independent of $\wt S_M$, using Hoeffding's bound,
    % \footnote{For a fully rigorous argument,
    % refer to the complete proof in App.~\ref{app:proof_erm}.} 
    we have with probability at least $1-\delta$ that
    \begin{align}
      \error_{\wt \calS_M}(f^*) \le \error_{ \calDm}(f^*) +  \sqrt{\frac{\log(1/\delta)}{2 m_1}} \,. \label{eq:step2_reg} 
    \end{align}
    %$ 
    %for some constant $c_1\le 1/2$. 
    Finally, since $f^*$ is the optimal classifier on $\widecheck \calD$, 
    we have 
    \begin{align}
        \error_{\widecheck \calD}(f^*) + \lambda R(f^*) \le \error_{\widecheck \calD}(\widehat f) + \lambda R(\wh f) \,. \label{eq:step3_reg}
    \end{align}
     Now to relate \eqref{eq:step1_reg} and \eqref{eq:step3_reg}, we can re-write the \eqref{eq:step2_reg} as follows: 
    \begin{align}
        \error_{\calS \cup \wt\calS}(f^*) \le \error_{ \widecheck \calD}(f^*) +  \frac{m_1}{m+n}\sqrt{\frac{\log(1/\delta)}{2 m_1}} \,. \label{eq:step4_reg} 
    \end{align}
    After adding $\lambda R(f^*)$ on both sides in \eqref{eq:step4_reg}, we combine equations \eqref{eq:step1_reg}, \eqref{eq:step4_reg}, and \eqref{eq:step3_reg}, to get 
    \begin{align}
        \error_{\calS \cup \wt \calS}(\wh f) \le \error_{\widecheck \calD}(\wh f) +  \frac{m_1}{m+n}\sqrt{\frac{\log(1/\delta)}{2 m_1}} \,, 
    \end{align}
    which implies 
    \begin{align}
        \error_{ \wt \calS_M}(\wh f) \le \error_{\calDm}(\wh f) + \sqrt{\frac{\log(1/\delta)}{2 m_1}} \,. \label{eq:lemma_reg_final}
    \end{align}
    Similar as before, since $\wt S$ is obtained by randomly labeling an unlabeled dataset, we assume 
    $2m_1 \approx m$. Moreover, using $\error_{\calDm} = 1 - \error_{\calD}$ we obtain the desired result. 
\end{proof}
% \begin{proof}[Proof of ]
    
% \end{proof}

\subsection{Proof of \thmref{thm:multiclass_ERM}}

To prove our results in the multiclass case,
we first state and prove lemmas
parallel to those
% We first state and prove lemmas 
% parallel 
% to the three lemmas 
used in the proof of balanced binary case. 
We then combine these results 
% in the three lemmas 
to obtain the result in \thmref{thm:multiclass_ERM}. 

Before stating the result, 
we define mislabeled distribution $\calDm$ for any $\calD$.
While $\calDm$ and $\calD$ share 
the same marginal distribution over inputs $\calX$,
the conditional distribution over labels $y$ 
given an input $x\sim \calD_\calX$ is changed as follows:
For any $x$, the Probability Mass Function (PMF) over $y$ is defined as:  
$p_{\calDm} (\cdot \vert x) \defeq \frac{1 - p_{\calD}(\cdot \vert x)}{k - 1}$, where $ p_{\calD}(\cdot \vert x)$ is the PMF over $y$ for the distribution $\calD$. 

\begin{lemma} \label{lem:fit_mislabeled_multi}
    Assume the same setup as \thmref{thm:multiclass_ERM}. 
    Then for any $\delta >0$, with probability at least  $1-\delta$ 
    over the random draws of mislabeled data $\wt S_M$, we have 
    \begin{align}
        \error_\calD(\widehat f)  \le (k-1)\left(1 -\error_{\wt \calS_M}(\widehat f)\right) + (k-1)\sqrt{\frac{\log(1/\delta)}{m}}\,. \label{eq:lemma1_multi}
    \end{align}   
\end{lemma} 

\begin{proof}
   
    The main idea of the proof remains the same.
    We begin by regarding the clean portion of the data 
    ($S \cup \wt S_C$) as fixed. 
    Then, there exists a classifier $f^*$ 
    that is optimal over draws 
    of the mislabeled data $\wt S_M$. 
    
    However, in the multiclass case,
    we cannot as easily relate the population error on mislabeled data 
    to the population accuracy on clean data.   
    While for binary classification, 
    % we could upper bound $\error_{\wt \calS_M}$ 
    % with $1-\error_\calD$ 
    we could lower bound the population accuracy $1-\error_\calD$
    with the empirical error on mislabeled data $\error_{\wt \calS_M}$ 
    (in the proof of \lemref{lem:fit_mislabeled}), 
    for multiclass classification, 
    error on the mislabeled data 
    and accuracy on the clean data 
    in the population 
    are not so directly related.  
    To establish \eqref{eq:lemma1_multi},
    we break the error on the 
    (unknown) mislabeled data 
    into two parts: one term corresponds 
    to predicting the true label on mislabeled data, 
    and the other corresponds to predicting 
    neither the true label 
    nor the assigned (mis-)label.  
    Finally, we relate these errors to their
    population counterparts to establish \eqref{eq:lemma1_multi}. 
    
    Formally, 
    \begin{align}
    f^* \defeq \argmin_{f \in \calF} \error_{\widecheck {\calD}} (f)  + \lambda R(f) \,, \label{eq:modified_ERM_reg2}
    \end{align}
    where $$\widecheck \calD = \frac{n}{m+n} \calS + \frac{m_1}{m+n} \wt \calS_C  + \frac{m_2}{m+n}\calDm \,.$$ 
    That is, $\widecheck \calD$ is a combination 
    of the \emph{empirical distribution} 
    over correctly labeled data $S \cup \wt S_C$
    % in $S\cup \wt S$ 
    and the (population) distribution 
    over mislabeled data $\calDm$.
    Recall that 
    \begin{align}
    \wh f \defeq \argmin_{f \in \calF} \error_{\calS \cup \wt S} (f) + \lambda R(f) \,. \label{eq:orig_ERM_reg2}
    \end{align}
    % 
    % 
    Following the exact steps from the proof of \lemref{lem:lemma1_reg}, 
    with probability at least $1-\delta$, we have  
    \begin{align}
        \error_{ \wt \calS_M}(\wh f) \le \error_{\calDm}(\wh f) + \sqrt{\frac{\log(1/\delta)}{2 m_1}} \,. \label{eq:lemma1_final_multi_prev}
    \end{align}
    Similar to before, since $\wt S$ is obtained 
    by randomly labeling an unlabeled dataset, 
    we assume 
    $\frac{k}{k-1} m_1 \approx m$. 
    
    Now we will relate $\error_{\calDm} (\wh f)$ with $\error_{\calD}(\wh f)$. 
    Let $y^T$ denote the (unknown) true label 
    for a mislabeled point $(x, y)$ 
    (i.e., label before replacing it with a mislabel). 
    \begin{align*}    
         \Expt{(x, y) \in \sim \calDm}{\indict{ \wh f(x) \ne y }}  &= \underbrace{\Expt{(x, y) \in \sim \calDm}{\indict{ \wh f(x) \ne y \land \wh f(x) \ne y^T}}}_{\RN{1}} \\ &\qquad \qquad + \underbrace{\Expt{(x, y) \in \sim \calDm}{\indict{ \wh f(x) \ne y \land \wh f(x) = y^T}}}_{\RN{2}} \,. \numberthis \label{eq:excess_term}
    \end{align*}
    Clearly, term 2 is one minus the accuracy 
    on the clean unseen data, i.e.,
    \begin{align}
        \RN{2} = 1 - \Expt{{x,y} \sim \calD}{ \indict{ \wh f(x) \ne y}} = 1- \error_{\calD}(\wh f) \,. \label{eq:term1}    
    \end{align}
    Next, we relate term 1 with the error on the unseen clean data. 
    We show that term 1 is equal to the error on the unseen clean data 
    scaled by $\frac{k-2}{k-1}$,
    where $k$ is the number of labels.
    Using the definition of mislabeled distribution $\calDm$,  
    we have 
    \begin{align}
        \RN{1} = \frac{1}{k-1} \left( \Expt{(x, y) \in \sim \calD}{ \sum_{i \in \calY \land i\ne y}  \indict{ \wh f(x) \ne i \land \wh f(x) \ne y}} \right) = \frac{k-2}{k-1} \error_{\calD}(\wh f) \,.\label{eq:term2}
    \end{align}    

    Combining the result in \eqref{eq:term1}, \eqref{eq:term2} and \eqref{eq:excess_term}, we have 
    \begin{align}
        \error_{\calDm}(\wh f) = 1- \frac{1}{k-1} \error_{\calD}(\wh f) \,.\label{eq:combine_terms}
    \end{align}
    Finally, combining the result in \eqref{eq:combine_terms} 
    with equation \eqref{eq:lemma1_final_multi_prev}, 
    we have with probability $1-\delta$, 
    \begin{align}
      \error_{\calD}(\wh f) \le  (k-1) \left( 1- \error_{ \wt \calS_M}(\wh f) \right)  + (k-1) \sqrt{\frac{k \log(1/\delta)}{ 2(k-1)m}} \,. \label{eq:lemma1_final_multi}
    \end{align}
\end{proof}

\begin{lemma} \label{lem:mislabeled_error_multi}
    Assume the same setup as \thmref{thm:multiclass_ERM}. 
    Then for any $\delta >0$, 
    with probability at least $1-\delta$ 
    over the random draws of $\wt S$, we have  
    % \begin{align}
        $$\abs{k\error_{\wt \calS}(\widehat f) - \error_{\wt \calS_C}(\widehat f) -  (k-1)\error_{\wt \calS_M}(\widehat f) } \le  2k\sqrt{\frac{\log(4/\delta)}{2m}}\,. $$ % \label{eq:lemma2}
    % \end{align}   
    %  for some constant $c_3 \le 1.0\,$.
\end{lemma} 


\begin{proof}
    Recall $\error_{\wt S} (f) = \frac{m_1}{m} \error_{\wt S_M}(f) + \frac{m_2}{m} \error_{\wt S_C}(f)$. Hence, we have 
    \begin{align*}
        k\error_{\wt S}(f) - (k-1)\error_{\wt S_M}(f) - \error_{\wt S_C}(f) &= (k-1)\left(\frac{k m_1}{(k-1) m} \error_{\wt S_M}(f) - \error_{\wt S_M}(f)\right) \\ & \qquad \qquad + \left(\frac{km_2}{m} \error_{\wt S_C}(f) - \error_{\wt S_C}(f)\right) \\ &= k \left[ \left(\frac{m_1}{m} - \frac{k-1}{k}\right) \error_{\wt S_M}(f) + \left(\frac{m_2}{m} - \frac{1}{k} \right) \error_{\wt S_C} (f) \right] \,.
    \end{align*} 
    Since the dataset is randomly labeled, 
    we have with probability at least $1-\delta$, 
    $\left(\frac{m_1}{m} - \frac{k-1}{k}\right) \le \sqrt{\frac{\log(1/\delta)}{2m}}$. 
    Similarly, we have with probability at least $1-\delta$, 
    $\left(\frac{m_2}{m} - \frac{1}{k}\right) \le \sqrt{\frac{\log(1/\delta)}{2m}}$. 
    Using union bound, we have with probability at least $1-\delta$
    % \begin{align}
    %     2\error_{\wt S} - \error_{\wt S_M}(f) - \error_{\wt S_C}(f) \le \sqrt{\frac{\log(2/\delta)}{2m}} \left(\error_{\wt S_M}(f) + \error_{\wt S_C}(f) \right) \le 2\sqrt{\frac{\log(2/\delta)}{2m}} \,. \label{eq:lemma2_final}
    % \end{align}
    \begin{align}
        k\error_{\wt S}(f) - (k-1)\error_{\wt S_M}(f) - \error_{\wt S_C}(f)  \le k \sqrt{\frac{\log(2/\delta)}{2m}} \left(\error_{\wt S_M}(f) + \error_{\wt S_C}(f) \right) \,. \label{eq:lemma2_final_multi}
    \end{align}

    % We obtain the desired result by using 
\end{proof}

\begin{lemma} \label{lem:clear_error_multi}
    Assume the same setup as \thmref{thm:multiclass_ERM}. 
    Then for any $\delta >0$, with probability at least $1-\delta$ 
    over the random draws of $\wt S_C$ and $S$, we have 
    % \begin{align}
        $$\abs{\error_{\wt \calS_C}(\widehat f) - \error_{\calS}(\widehat f) } \le 1.5 \sqrt{\frac{k\log(2/\delta)}{2m}}\,.$$ %\label{eq:lemma3}
    % \end{align}   
    % for some constant $c_2 \le 1.2\,$.
\end{lemma} 
\begin{proof}
    % Recall 0-1 error on each point  $(x,y) \in S \cup \wt S$ is given by $\I{ f(x)\ne y}$.
    In the set of correctly labeled points $S \cup \wt S_C$,
    we have $S$ as a random subset of $S \cup \wt S_C$. 
    Hence, using Hoeffding's inequality 
    for sampling without replacement 
    (\lemref{lem:hoeffding_sampling}), 
    we have with probability at least $1-\delta$
    \begin{align}
        \error_{\wt \calS_c} (\wh f)- \error_{\calS \cup \wt \calS_C}( \wh f) \le  \sqrt{\frac{\log(1/\delta)}{2m_2}} \,.
    \end{align}
    Re-writing $\error_{\calS \cup \wt \calS_C}( \wh f)$ 
    as $\frac{m_2}{m_2 + n} \error_{\wt \calS_C }(\wh f) + \frac{n}{m_2 + n} \error_{\calS }(\wh f)$, 
    we have with probability at least $1-\delta$
    \begin{align}
       \left(\frac{n}{n+m_2}\right) \left(\error_{\wt \calS_c} (\wh f)- \error_{\calS}( \wh f) \right) \le  \sqrt{\frac{\log(1/\delta)}{2m_2}} \,.
    \end{align}
    As before, assuming $km_2 \approx m$, 
    we have with probability at least $1-\delta$ 
    \begin{align}
        \error_{\wt \calS_c} (\wh f)- \error_{\calS}( \wh f) \le \left(1+\frac{m_2}{n}\right)  \sqrt{\frac{k\log(1/\delta)}{2m}} \le \left( 1 + \frac{1}{k}\right) \sqrt{\frac{k\log(1/\delta)}{2m}} \,. \label{eq:lemma3_final_multi}
    \end{align} 
\end{proof}

\begin{proof}[Proof of \thmref{thm:multiclass_ERM}] 
    Having established these core intermediate results, 
    we can now combine above three lemmas. 
    In particular, we bound the population error 
    on clean data ($\error_\calD(\wh f)$) as follows:  
    \begin{enumerate}[(i)]
        \item First, use \eqref{eq:lemma1_final_multi}, 
        to obtain an upper bound on the population error on clean data, 
        i.e., with probability at least $1-\delta/4$, we have
        \begin{align}
            \error_{ \calD} (\wh f) \le (k-1)\left(1 - \error_{ \wt \calS_M}(\wh f) \right) + (k-1) \sqrt{\frac{k\log(4/\delta)}{2(k-1)m}} \,. 
        \end{align}
        \item  Second, use \eqref{eq:lemma2_final_multi}
        to relate the error on the mislabeled fraction 
        with error on clean portion of randomly labeled data 
        and error on whole randomly labeled dataset, 
        i.e., with probability at least $1-\delta/2$, we have 
        \begin{align}
            - (k-1)\error_{\wt S_M}(f) \le \error_{\wt S_C}(f) - k\error_{\wt S}  + k\sqrt{\frac{\log(4/\delta)}{2m}}  \,. 
        \end{align} 
        \item Finally, use \eqref{eq:lemma3_final_multi} 
        to relate the error on the clean portion of randomly labeled data 
        and error on clean training data, 
        i.e., with probability $1-\delta/4$, we have 
        \begin{align}
            \error_{\wt \calS_C} (\wh f)\le - \error_{\calS}( \wh f) + \left(1 + \frac{m}{kn} \right) \sqrt{\frac{k\log(4/\delta)}{2m}} \,. 
        \end{align} 
    \end{enumerate}

    Using union bound on the above three steps, 
    we have with probability at least $1-\delta$: 
    \begin{align}
        \error_\calD (\wh f) \le \error_{\calS}(\wh f) + (k-1) - k\error_{\wt \calS}(\wh f)   + (\sqrt{k(k-1)} + k + \sqrt{k} + \frac{m}{n\sqrt{k}})  \sqrt{\frac{\log(4/\delta)}{2m}} \,.\label{eq:multiclass_ERM_final}
    \end{align}
    Simplifying the term in RHS of \eqref{eq:multiclass_ERM_final}, 
    we get the desired result. 
    % Note that since $\frac{m}{n\sqrt{k}}$ 
    % is much smaller than the sum of the other terms
    % the other terms in summation, 
    % we ignore $\frac{m}{n\sqrt{k}}$  
    % Z: ??? --- great
    % that 
    % them
    in the final bound. 
    % we ignore that in the final bound. 
    % Note that $(1/\sqrt{2} + 2.5)$ is a loose constant. In experiments, we use the ratio $\frac{m}{n}$
    %  the exact error $\error_{\wt \calS}(\wh f)$ 
    % to evaluate R.H.S.    
\end{proof}

\newpage
\section{Proofs from \secref{sec:linear_models}}\label{app:proof_gd}
We suppose that the parameters of the linear function 
are obtained via gradient descent on 
the following $L_2$ regularized problem: 
\begin{align}
    % n in denominator is avoided deliberately
    \calL_S(w; \lambda) \defeq \sum_{i=1}^n{(w^Tx_i - y_i)^2} + \lambda \norm{w}{2}^2 \,, \label{eq:l2_MSE_app}   
\end{align}
where $\lambda\ge0$ is a regularization parameter. 
We assume access to a clean dataset 
$S = \{(x_i, y_i)\}_{i=1}^n \sim \calD^n$ 
and randomly labeled dataset 
$\wt S = \{(x_i, y_i)\}_{i=n+1}^{n+m} \sim \wt \calD^m$. 
Let $\bX = [x_1, x_2, \cdots, x_{m+n}]$ 
and $\by = [y_1, y_2, \cdots, y_{m+n}]$. 
Fix a positive learning rate $\eta$ such that 
$\eta \le 1/\left(\norm{\bX^T\bX}{\text{op}} + \lambda^2\right)$ 
and an initialization $w_0 = 0$. 
% \todos{Assumption made for simplicty}. 
Consider the following gradient descent iterates 
to minimize objective \eqref{eq:l2_MSE_app} on $S \cup \wt S$:
\begin{align}
w_t = w_{t-1} - \eta \grad_w \calL_{S \cup \wt S} (w_{t-1}; \lambda) \quad \forall t=1,2,\ldots \label{eq:GD_iterates_app}
\end{align} 
Then we have $\{ w_t\}$ converge to the limiting solution 
$\wh w = \left( \bX^T\bX+\lambda \boldsymbol{I}\right)^{-1}\bX^T\by$. Define $\widehat f (x) \defeq f(x ; \wh w) $.  

% \subsection{\textcolor{red}{Errata}}

% We wish to correct the following error in the body:
% \codref{cond:error_stability} is not enough 
% to guarantee the result in \thmref{thm:linear}. 
% We now present a slightly stronger condition 
% called \emph{hypothesis stability} 
% under which we obtain a result 
% similar to \thmref{thm:linear}. 

% This error doesn't change the main arguments of the proof,
% where we show that the empirical train error 
% is less than or equal to the leave-one-out error.
% We need a stronger condition to relate leave-one-out error 
% with the population error of the original classifier. 
% Specifically, while \codref{cond:error_stability} 
% relates the average population error of leave-one-out classifiers 
% with the population error of the original classifier, 
% we need the new condition to show the concentration 
% of the empirical leave-one-out error 
% and average population error of leave-one-out classifiers. 
% main takeaway 

% Note that the new condition, 
% while being stronger than the previous one, 
% still doesn't imply generalization \citep{bousquet2002stability,elisseeff2003leave,abou2019exponential}. 
% Overall, the main results in \secref{sec:ERM_training} 
% and takeaways of the paper remain unaffected by the error.  

% We now present the new condition 
% and a corrected statement of \thmref{thm:linear}. 
% Recall, for a given training set $S \sim \calD^n $, 
% we use $S_{(i)}$ to denote the training set $S$ 
% with the $i^{\text{th}}$ point removed.

% \begin{condition}[Hypothesis Stability] 
%     \label{cond:hypothesis_stability}
%     We have $\beta$ hypothesis stability 
%     if our training algorithm $\calA$ satisfies the following: 
%     \begin{align*}
%     % ${\sum_{i=1}^n \frac{\error_{\calD}( f(\calA, S_{(i)}))}{n} - \error_\calD(f(\calA, S))} \le \beta\,$.
%     \forall i \in \{1,2,\ldots, n\}, \quad  \Expt{\calS, (x,y) \in \calD}{ \abs{\error\left( f(x) ,y  \right) - \error\left( f_{(i)}(x), y \right) }} \le \frac{\beta}{n} \,,
%     \end{align*}
%     where $f_{(i)} \defeq f(\calA, S_{(i)})$ and $ f \defeq f(\calA, S)$.
% \end{condition}

% \begin{theorem}[Correct statement of \thmref{thm:linear}] \label{thm:new_linear}
%     Assume that this gradient descent algorithm satisfies \codref{cond:hypothesis_stability}
%     with $\beta=\calO(1)$.  
%     Then for any $\delta >0$, with probability at least $1-\delta$ 
%     over the random draws of datasets $\wt S$ and $S$, we have:
%     \begin{align}
%         \error_\calD(\widehat f) \le \error_\calS(\widehat f) + 1 - 2 \error_{\wt\calS}(\widehat f) + \left(\frac{1}{\sqrt{2}} + 1.5 \right) \sqrt{\frac{\log(4/\delta)}{m}} + \sqrt{\frac{4}{\delta}\left(\frac{1}{m} +\frac{3\beta}{m+n} \right)}  \,. \label{eq:gd_error}
%     \end{align} 
%     % for some constant $c\le 3.2$.
% \end{theorem}

\subsection{Proof of \thmref{thm:linear}}
We use a standard result from linear algebra, 
namely the Shermann-Morrison formula 
\citep{sherman1950adjustment} for matrix inversion:  

\begin{lemma}[\citet{sherman1950adjustment}] \label{lem:sherman}
    Suppose $\bA \in \Real^{n \times n}$ 
    is an invertible square matrix 
    and $u,v \in \Real^n$ are column vectors. 
    Then $\bA + uv^T$ is invertible iff $1 + v^T \bA u \ne 0$ 
    and in particular
    \begin{align}
        (\bA + u v^T)^{-1} = \bA^{-1}  - \frac{\bA^{-1} uv^T \bA^{-1} }{ 1 + v^T \bA^{-1} u} \,.
    \end{align}   
\end{lemma}
\newcommand\byy[1]{\by_{\left(#1\right)}}
\newcommand\bXX[1]{\bX_{\left(#1\right)}}
\newcommand\ff[1]{\wh f_{\left(#1\right)}}

For a given training set $S \cup \wt S_C$, 
define leave-one-out error 
on mislabeled points in the training data 
as $$\error_{\text{LOO}(\wt S_M) } = \frac{\sum_{(x_i, y_i) \in \wt S_M} \error( f_{(i)}( x_i), y_i)}{ \abs{\wt S_M }} \,, $$
where $f_{(i)} \defeq f(\calA, (S \cup \wt S)_{(i)})$. 
To relate empirical leave-one-out error and population error 
with hypothesis stability condition, 
we use the following lemma:   

\begin{lemma}[\citet{bousquet2002stability}] \label{lem:stability_error}
    For the leave-one-out error, we have
    \begin{align}
        \Expo{ \left( \error_{\calDm}(\wh f) -\error_{\text{LOO}(\wt S_M) } \right)^2 } \le \frac{1}{2m_1}+  \frac{3\beta}{n + m}\,.
    \end{align}   
    % where $ f \defeq f(\calA, S \cup \wt S) $.
\end{lemma}

Proof of the above lemma is similar 
to the proof of Lemma 9 in \citet{bousquet2002stability} 
and can be found in \appref{app:proof_lem_error}. 
% 
% Before presenting the result, we introduce some notation. 
Before presenting the proof of \thmref{thm:linear}, 
we introduce some more notation. 
Let $\bX_{(i)}$ denote the matrix of covariates 
with the $i^{\text{th}}$ point removed. 
Similarly, let $\by_{(i)}$ be the array of responses 
with the $i^{\text{th}}$ point removed. 
Define the corresponding regularized GD solution 
as $\wh w_{(i)} = \left( \bXX{i}^T\bXX{i}+\lambda \boldsymbol{I}\right)^{-1}\bXX{i}^T\byy{i}$. 
Define $\ff{i}(x) \defeq f(x ; \wh w_{(i)}) $.

\begin{proof}[Proof of \thmref{thm:linear}]
    Because squared loss minimization does not imply 0-1 error minimization, 
    we cannot use arguments from \lemref{lem:fit_mislabeled}. 
    This is the main technical difficulty. 
    To compare the 0-1 error at a train point with an unseen point, 
    we use the closed-form expression for $\widehat{w}$ 
    and Shermann-Morrison formula 
    to upper bound training error 
    with leave-one-out cross validation error. 
    
    The proof is divided into three parts: 
    In part one, we show that 0-1 error 
    on mislabeled points in the training set 
    is lower than the error obtained 
    by leave-one-out error at those points. 
    In part two, we relate this leave-one-out error 
    with the population error on mislabeled distribution
    using \codref{cond:hypothesis_stability}.
    While the empirical leave-one-out error is an unbiased estimator 
    of the average population error of leave-one-out classifiers, 
    we need hypothesis stability 
    to control the variance 
    of empirical leave-one-out error. 
    Finally, in part three, we show 
    that the error on the mislabeled training points 
    can be estimated with just the randomly labeled 
    and clean training data (as in proof of \thmref{thm:error_ERM}).  

    \textbf{Part 1 {} {}} First we relate training error with leave-one-out error.        
    For any training point $(x_i, y_i)$ in $\wt S \cup S$, we have 
    \begin{align}
        \error(\wh f(x_i), y_i ) &= \indict{ y_i \cdot x_i^T \wh w < 0 } = \indict{ y_i \cdot x_i^T \left( \bX^T\bX+\lambda \boldsymbol{I}\right)^{-1}\bX^T\by < 0 } \\
        &= \indict{ y_i \cdot x_i^T \underbrace{\left( \bXX{i}^T\bXX{i} + x_i ^T x_i +\lambda \boldsymbol{I}\right)^{-1}}_{\RN{1}} (\bXX{i}^T\byy{i} + y_i \cdot x_i) < 0 } \,.
    \end{align}
    Letting $\bA = \left(\bXX{i}^T\bXX{i} +\lambda \boldsymbol{I}\right)$ 
    and using \lemref{lem:sherman} on term 1, we have 
    \begin{align}
        \error(\wh f(x_i), y_i ) &= \indict{ y_i \cdot x_i^T \left[\bA^{-1} -  \frac{\bA^{-1} x_i x_i^T \bA^{-1}}{ 1 + x_i ^T \bA^{-1} x_i } \right] (\bXX{i}^T\byy{i} + y_i \cdot x_i) < 0 } \\
        &= \indict{ y_i \cdot\left[ \frac{ x_i^T \bA^{-1} ( 1 + x_i ^T \bA^{-1} x_i ) -  x_i^T \bA^{-1} x_i x_i^T \bA^{-1}}{ 1 + x_i ^T \bA ^{-1}x_i } \right] (\bXX{i}^T\byy{i} + y_i \cdot x_i) < 0 } \\
        &= \indict{ y_i \cdot\left[ \frac{ x_i^T \bA^{-1}}{ 1 + x_i ^T \bA ^{-1}x_i } \right] (\bXX{i}^T\byy{i} + y_i \cdot x_i) < 0 } \,.
    \end{align}

    Since $1 + x_i^T \bA^{-1} x_i > 0$, we have 
    \begin{align}
        \error(\wh f(x_i), y_i ) &= \indict{ y_i \cdot x_i^T \bA^{-1} (\bXX{i}^T\byy{i} + y_i \cdot x_i) < 0 } \\
        &= \indict{ x_i^T \bA^{-1} x_i +  y_i \cdot x_i^T \bA^{-1} (\bXX{i}^T\byy{i}) < 0 } \\
        &\le \indict{ y_i \cdot x_i^T \bA^{-1} (\bXX{i}^T\byy{i}) < 0 } = \error(\ff{i}(x_i), y_i ) \,.\label{eq:LOO_error}
    \end{align}

    Using \eqref{eq:LOO_error}, we have 
    \begin{align}
        \error_{\wt \calS_M } (\wh f) \le \error_{\text{LOO} (\wt S_M)} \defeq \frac{\sum_{(x_i, y_i) \in \wt S_M} \error(\ff{i}(x_i), y_i ) }{\abs{\wt \calS_M}}\label{eq:LOO_error_final} \,.
    \end{align}
    \textbf{Part 2 {}{}} We now relate RHS in \eqref{eq:LOO_error_final} 
    with the population error on mislabeled distribution. 
    To do this, we leverage \codref{cond:hypothesis_stability} 
    and \lemref{lem:stability_error}. 
    In particular, we have 

    \begin{align}
        \Expt{\calS \cup \wt \calS_M }{ \left(\error_{\calDm}(\wh f) - \error_{\text{LOO} (\wt S_M)}\right)^2 } \le \frac{1}{2m_1} + \frac{3\beta}{m+n} \,.
    \end{align}

    Using Chebyshev's inequality, with probability at least $1-\delta$, we have 
    \begin{align}
        \error_{\text{LOO} (\wt S_M)} \le  \error_{\calDm}(\wh f)   + \sqrt{\frac{1}{\delta}\left(\frac{1}{2m_1} +\frac{3\beta}{m+n} \right)} \,. \label{eq:final_mislabeled_linear}
    \end{align}
    

    \textbf{Part 3 {}{}} Combining \eqref{eq:final_mislabeled_linear} and \eqref{eq:LOO_error_final}, we have 

    \begin{align}
        \error_{\wt \calS_M } (\wh f) \le \error_{\calDm}(\wh f)   + \sqrt{\frac{1}{\delta}\left(\frac{1}{2m_1} +\frac{3\beta}{m+n} \right)} \,. \label{eq:linear_parallel_lem1}
    \end{align}

    Compare \eqref{eq:linear_parallel_lem1} with \eqref{eq:lemma1_final} 
    in the proof of \lemref{lem:fit_mislabeled}. 
    We obtain a similar relationship 
    between $\error_{\wt \calS_M }$ and $\error_{\calDm}$ 
    but with a polynomial concentration 
    instead of exponential concentration. 
    In addition, since we just use concentration arguments 
    to relate mislabeled error to the errors
    on the clean and unlabeled portions 
    of the randomly labeled data, 
    we can directly use the results 
    in \lemref{lem:mislabeled_error} and \lemref{lem:clear_error}. 
    Therefore, combining results in \lemref{lem:mislabeled_error}, \lemref{lem:clear_error}, and \eqref{eq:linear_parallel_lem1} with union bound, 
    we have with probability at least $1-\delta$
    \begin{align}
        \error_\calD(\widehat f) \le \error_\calS(\widehat f) + 1 - 2 \error_{\wt\calS}(\widehat f) + \left(\sqrt{2}\error_{\wt\calS}(\widehat f) + 1 + \frac{m}{2n} \right) \sqrt{\frac{\log(4/\delta)}{m}} + \sqrt{\frac{4}{\delta}\left(\frac{1}{m} +\frac{3\beta}{m+n} \right)}  \,.
    \end{align}
    

       
\end{proof}

\subsection{Extension to multiclass classification} \label{app:multiclass_linear}
For multiclass problems with squared loss minimization, as standard practice, we consider one-hot encoding for the underlying label, i.e., a class label $c \in [k]$ is treated as $(0, \cdot, 0,1,0, \cdot, 0) \in \Real^k$ (with $c$-th coordinate being 1).  As before, we suppose that the parameters of the linear function 
are obtained via gradient descent on the following $L_2$ regularized problem: 
\begin{align}
    % n in denominator is avoided deliberately
    \calL_S(w; \lambda) \defeq \sum_{i=1}^n\norm{w^Tx_i - y_i}{2}^2 + \lambda \sum_{j=1}^k \norm{w_j}{2}^2 \,, \label{eq:l2_multiclass_MSE_app}   
\end{align}
where $\lambda\ge0$ is a regularization parameter. 
We assume access to a clean dataset 
$S = \{(x_i, y_i)\}_{i=1}^n \sim \calD^n$ 
and randomly labeled dataset 
$\wt S = \{(x_i, y_i)\}_{i=n+1}^{n+m} \sim \wt \calD^m$. 
Let $\bX = [x_1, x_2, \cdots, x_{m+n}]$ 
and $\by = [e_{y_1}, e_{y_2}, \cdots, e_{y_{m+n}}]$. 
Fix a positive learning rate $\eta$ such that 
$\eta \le 1/\left(\norm{\bX^T\bX}{\text{op}} + \lambda^2\right)$ 
and an initialization $w_0 = 0$. 
% \todos{Assumption made for simplicty}. 
Consider the following gradient descent iterates 
to minimize objective \eqref{eq:l2_MSE_app} on $S \cup \wt S$:
\begin{align}
{w_j}^t = {w_j}^{t-1} - \eta \grad_{w_j} \calL_{S \cup \wt S} (w^{t-1}; \lambda) \quad \forall t=1,2,\ldots \text{ and } j=1,2,\ldots,k  \,. \label{eq:GD_multi_iterates_app}
\end{align} 
Then we have $\{ {w_j}^t\}$ for all $j =1,2,\cdots, k$ converge to the limiting solution 
$\wh w_j = \left( \bX^T\bX+\lambda \boldsymbol{I}\right)^{-1}\bX^T\by_j$. Define $\widehat f (x) \defeq f(x ; \wh w) $.  

\begin{theorem}\label{thm:multi_linear}
    Assume that this gradient descent algorithm satisfies \codref{cond:hypothesis_stability}
    with $\beta=\calO(1)$.  
    Then for a multiclass classification problem wth $k$ classes, for any $\delta >0$, with probability at least $1-\delta$, we have:
    \begin{align*}
        \error_\calD(\widehat f) \le \error_\calS(\widehat f) &+ (k-1)\left(1 - \frac{k}{k-1} \error_{\wt\calS}(\widehat f) \right) \\ &+ \left(k + \sqrt{k} + \frac{m}{n\sqrt{k}} \right) \sqrt{\frac{\log(4/\delta)}{2m}} + \sqrt{k(k-1)} \sqrt{\frac{4}{\delta}\left(\frac{1}{m} +\frac{3\beta}{m+n} \right)}  \,. \numberthis \label{eq:gd_multi_error}
    \end{align*} 
    % for some constant $c\le 3.2$.
\end{theorem}
\begin{proof}
    The proof of this theorem is divided into two parts. In the first part, we relate the error on the mislabeled samples with the population error on the mislabeled data. Similar to the proof of \thmref{thm:linear}, we use Shermann-Morrison formula to upper bound training error with leave-one-out error on each $\wh w^j$. Second part of the proof follows entirely from the proof of \thmref{thm:multiclass_ERM}. In essence, the first part derives an equivalent of \eqref{eq:lemma1_final_multi_prev} for GD training with squared loss and then the second part follows from the proof  of \thmref{thm:multiclass_ERM}. 
    
    \textbf{Part-1:} Consider a training point $(x_i,y_i)$ in $\wt S \cup S $. For simplicity, we use $c_i$ to denote the class of $i$-th point and use $y_i$ as the corresponding one-hot embedding. Recall error in multiclass point is given by $\error(\wh f(x_i), y_i ) = \indict{ c_i \not \in \argmax x_i^T \wh w }$. Thus, there exists a $j \ne c_i \in [k]$, such that we have
     \begin{align}
        \error(\wh f(x_i), y_i ) &= \indict{ c_i \not \in \argmax x_i^T \wh w } = \indict{ x_i^T \wh w_{c_i} < x_i^T \wh w_{j}  } \\ &= \indict{ x_i^T \left( \bX^T\bX+\lambda \boldsymbol{I}\right)^{-1}\bX^T\by_{c_i} < x_i^T \left( \bX^T\bX+\lambda \boldsymbol{I}\right)^{-1}\bX^T\by_{j} } \\
        &= \indict{ x_i^T \underbrace{\left( \bXX{i}^T\bXX{i} + x_i ^T x_i +\lambda \boldsymbol{I}\right)^{-1}}_{\RN{1}} \left(\bXX{i}^T{\by_{c_i}}_{(i)} + x_i - \bXX{i}^T{\by_{j}}_{(i)}\right) < 0 } \,.
    \end{align}
    Letting $\bA = \left(\bXX{i}^T\bXX{i} +\lambda \boldsymbol{I}\right)$ 
    and using \lemref{lem:sherman} on term 1, we have 
    \begin{align}
        \error(\wh f(x_i), y_i ) &= \indict{ x_i^T \left[\bA^{-1} -  \frac{\bA^{-1} x_i x_i^T \bA^{-1}}{ 1 + x_i ^T \bA^{-1} x_i } \right]  \left(\bXX{i}^T{\by_{c_i}}_{(i)} + x_i - \bXX{i}^T{\by_{j}}_{(i)}\right) < 0 } \\
        &= \indict{ \left[ \frac{ x_i^T \bA^{-1} ( 1 + x_i ^T \bA^{-1} x_i ) -  x_i^T \bA^{-1} x_i x_i^T \bA^{-1}}{ 1 + x_i ^T \bA ^{-1}x_i } \right]  \left(\bXX{i}^T{\by_{c_i}}_{(i)} + x_i - \bXX{i}^T{\by_{j}}_{(i)}\right) < 0 } \\
        &= \indict{ \left[ \frac{ x_i^T \bA^{-1}}{ 1 + x_i ^T \bA ^{-1}x_i } \right]  \left(\bXX{i}^T{\by_{c_i}}_{(i)} + x_i - \bXX{i}^T{\by_{j}}_{(i)}\right) < 0} \,.
    \end{align}
    Since $1 + x_i^T \bA^{-1} x_i > 0$, we have 
    \begin{align}
        \error(\wh f(x_i), y_i ) &= \indict{ x_i^T \bA^{-1}  \left(\bXX{i}^T{\by_{c_i}}_{(i)} + x_i - \bXX{i}^T{\by_{j}}_{(i)}\right) < 0 } \\
        &= \indict{ x_i^T \bA^{-1} x_i +  x_i^T \bA^{-1}  \bXX{i}^T{\by_{c_i}}_{(i)}  - x_i^T\bA^{-1}  \bXX{i}^T{\by_{j}}_{(i)} < 0 } \\
        &\le \indict{  x_i^T \bA^{-1}  \bXX{i}^T{\by_{c_i}}_{(i)}  - x_i^T\bA^{-1}  \bXX{i}^T{\by_{j}}_{(i)} < 0  } = \error(\ff{i}(x_i), y_i ) \,.\label{eq:LOO_error_multi}
    \end{align}
    Using \eqref{eq:LOO_error_multi}, we have 
    \begin{align}
        \error_{\wt \calS_M } (\wh f) \le \error_{\text{LOO} (\wt S_M)} \defeq \frac{\sum_{(x_i, y_i) \in \wt S_M} \error(\ff{i}(x_i), y_i ) }{\abs{\wt \calS_M}}\label{eq:LOO_error_multi_final} \,.
    \end{align}
    
    We now relate RHS in \eqref{eq:LOO_error_final} 
    with the population error on mislabeled distribution. 
    Similar as before, to do this, we leverage \codref{cond:hypothesis_stability} 
    and \lemref{lem:stability_error}. Using  \eqref{eq:final_mislabeled_linear} and \eqref{eq:LOO_error_multi_final}, we have 
    \begin{align}
        \error_{\wt \calS_M } (\wh f) \le \error_{\calDm}(\wh f)   + \sqrt{\frac{1}{\delta}\left(\frac{1}{2m_1} +\frac{3\beta}{m+n} \right)} \,. \label{eq:linear_multi_parallel_lem1}
    \end{align}
    
    We have now derived a parallel to \eqref{eq:lemma1_final_multi_prev}. Using the same arguments in the proof of \lemref{lem:fit_mislabeled_multi}, we have 
    \begin{align}
      \error_{\calD}(\wh f) \le  (k-1) \left( 1- \error_{ \wt \calS_M}(\wh f) \right)  + (k-1)\sqrt{\frac{k}{\delta(k-1)}\left(\frac{1}{2m_1} +\frac{3\beta}{m+n} \right)}  \,. \label{eq:lemma1_linear_final_multi}
    \end{align}
    
    \textbf{Part-2:} We now combine the results in \lemref{lem:mislabeled_error_multi} and \lemref{lem:clear_error_multi} to obtain the final inequality in terms of quantities that can be computed from just the randomly labeled and clean data. Similar to the binary case, we obtained a polynomial concentration instead of exponential concentration. Combining \eqref{eq:lemma1_linear_final_multi} with \lemref{lem:mislabeled_error_multi} and \lemref{lem:clear_error_multi}, we have with probability at least $1-\delta$
    \begin{align*}
        \error_\calD(\widehat f) \le \error_\calS(\widehat f) &+ (k-1)\left(1 - \frac{k}{k-1} \error_{\wt\calS}(\widehat f) \right) \\ &+ \left(k + \sqrt{k} + \frac{m}{n\sqrt{k}} \right) \sqrt{\frac{\log(4/\delta)}{2m}} + \sqrt{k(k-1)} \sqrt{\frac{4}{\delta}\left(\frac{1}{m} +\frac{3\beta}{m+n} \right)}  \,. \numberthis \label{eq:gd_multi_error_proof}
    \end{align*} 
\end{proof}

\subsection{Discussion on \codref{cond:hypothesis_stability}} \label{app:discuss_cond1}
The quantity in LHS of \codref{cond:hypothesis_stability} 
measures how much the function learned by the algorithm 
(in terms of error on unseen point) will change 
when one point in the training set is removed. 
% Discussion on exponential concentration and stronger condition. 
% Notice that hypothesis stability implies error stability, i.e., \codref{cond:error_stability} \citep{bousquet2002stability}.  
% In summary, while error stability allowed us 
% to relate the average population error 
% of the leave-one-out classifiers 
% with the population error of the original classifier, 
We need hypothesis stability condition 
to control the variance of the empirical leave-one-out error to show concentration of average leave-one-error with the population error. 

Additionally, we note that while the dominating term in the RHS of \thmref{thm:linear} matches with the dominating term in ERM bound in \thmref{thm:error_ERM}, there is a polynomial concentration term 
(dependence on $1/\delta$ instead of $\log(\sqrt{1/\delta})$) 
in \thmref{thm:linear}. 
Since with hypothesis stability, 
we just bound the variance, 
the polynomial concentration is due 
to the use of Chebyshev's inequality 
instead of an exponential tail inequality
(as in \lemref{lem:fit_mislabeled}).
Recent works have highlighted that 
a slightly stronger condition than hypothesis stability 
can be used to obtain an exponential concentration 
for leave-one-out error \citep{abou2019exponential},
but we leave this for future work for now. 
% We leave 
% However, the constants 

% we also want to highlight  

\subsection{Formal statement and proof of \propref{prop:early_stop}} \label{app:formal_early_stop}

Before formally presenting the result, 
we will introduce some notation.  
By $\calL_{S}(w)$, we denote 
the objective in \eqref{eq:l2_MSE_app} with $\lambda=0$. 
Assume Singular Value Decomposition (SVD) of $\bX$
as $\sqrt{n} \bU \bS^{1/2} \bV^T$. 
Hence $\bX^T \bX = \bV \bS \bV^T$.
Consider the GD iterates defined in \eqref{eq:GD_iterates_app}. 
% 
We now derive closed form expression 
for the $t^\text{th}$ iterate of gradient descent:  
% 
\begin{align}
    w_t = w_{t-1} + \eta \cdot \bX^T (\by - \bX w_{t-1}) = (\bI - \eta \bV \bS \bV^T )w_{k-1} + \eta \bX^T \by \,.
\end{align}
Rotating by $\bV^T$, we get 
\begin{align}
    \wt w_t = (\bI - \eta\bS )\wt w_{k-1} + \eta \wt \by \label{eq:GD_recur},
\end{align}
where $\wt w_t = \bV^T w_t $ and $\wt \by = \bV^T \bX^T \by$. 
Assuming the initial point $w_0 = 0$ 
and applying the recursion in \eqref{eq:GD_recur}, we get
\begin{align}
    \wt w_t = \bS ^{-1} ( \bI - (\bI - \eta \bS)^k ) \wt \by \,, 
\end{align} 
Projecting solution back to the original space, we have 
\begin{align}
     w_t = \bV \bS ^{-1} ( \bI - (\bI - \eta \bS)^k ) \bV^T \bX^T \by \,. 
\end{align} 
% We will work with this GD solution at any iterate $t$ in the next proposition. 
Define $f_t(x) \defeq f(x;w_t)$ 
as the solution at the $t^{\text{th}}$ iterate. 
Let $\wt w_{\lambda} = \argmin_{w} \calL_\calS (w;\lambda) = (\bX^T \bX + \lambda \bI)^{-1} \bX^T \by = \bV (\bS + \lambda \bI )^{-1} \bV^T \bX^T \by $. 
% ) \,,$ for all $t=1,2,\ldots\,.$ 
and define $\wt f_\lambda(x) \defeq f(x;\wt w_\lambda)$ as the regularized solution. 
Assume $\kappa$ be the condition number 
of the population covariance matrix 
and let $s_\text{min}$ be the minimum positive 
singular value of the empirical covariance matrix. 
Our proof idea is inspired from recent work 
on relating gradient flow solution 
and regularized solution 
for regression problems \citep{ali2018continuous}. 
We will use the following lemma in the proof: 
\begin{lemma} \label{lem:ineq_soln}
    For all $x \in [0,1]$ and for all $ k \in \mathbb{N}$, 
    we have (a) $ \frac{kx}{1+kx} \le 1- (1-x)^k$ 
    and (b) $ 1- (1-x)^k \le 2 \cdot \frac{kx}{kx+1} $.
    %  where $g(c)$ is a constant dependent on $c$. For $c = 1$, $g(c) = 2.0$.   
\end{lemma}
\begin{proof}
    % [Proof of \lemref{lem:ineq_soln}]
    % Part (a) is easy. 
    Using $ (1-x)^k \le \frac{1}{1+kx}$, we have part (a). 
    For part (b), we numerically maximize 
    $\frac{ (1+kx ) (1 - (1-x)^k) }{kx}$ 
    for all $k\ge 1$ and for all $x \in [0, 1]$.  
\end{proof}

% 
% Next, 

\begin{prop}[Formal statement of \propref{prop:early_stop}] \label{prop:formal_early_stop}
Let $\lambda = \frac{1}{t\eta}$. 
For a training point $x$, we have 
\begin{align*}
    \Expt{x \sim \calS}{(f_t(x) - \wt f_\lambda(x))^2} &\le c(t,\eta) \cdot \Expt{x \sim \calS}{f_t(x)^2} \,, %\label{eq:early_stop}
\end{align*}
where $c(t, \eta) \defeq \min( 0.25, \frac{1}{s_\text{min}^2 t^2 \eta^2})$. 
Similarly for a test point, we have 
\begin{align*}
    \Expt{x \sim \calD_\calX}{(f_t(x) - \wt f_\lambda(x))^2} &\le \kappa \cdot c(t,\eta) \cdot \Expt{x \sim \calD_\calX}{f_t(x)^2} \,. %\label{eq:early_stop}
\end{align*}
\end{prop} 

\begin{proof}
    %%%%%%%%%%%%% 
    We want to analyze the expected squared difference output 
    of regularized linear regression 
    with regularization constant $\lambda = \frac{1}{\eta t}$ 
    and the gradient descent solution at the $t^\text{th}$ iterate. 
    We separately expand the algebraic expression 
    for squared difference at a training point and a test point. 
    % We start by considering the difference  
    Then the main step is to show that 
    $\left[ \bS ^{-1} ( \bI - (\bI - \eta \bS)^k )  - (\bS + \lambda \bI )^{-1}\right] \preceq c(\eta, t) \cdot \bS ^{-1} ( \bI - (\bI - \eta \bS)^k ) $.

    %%%%%%%%%%%%%
    
   \textbf{Part 1 {} {}} 
    First, we will analyze the squared difference 
    of the output at a training point 
    (for simplicity, we refer to $S \cup \wt S$ as $S$), i.e., 
    \begin{align}
        \Expt{ x \sim \calS }{\left(f_t(x) - \wt f_\lambda (x)\right)^2} &= \norm{\bX w_t - \bX \wt w_\lambda}{2}^2\\ &=   \norm{\bX \bV \bS ^{-1} ( \bI - (\bI - \eta \bS)^t ) \bV^T \bX^T \by - \bX \bV (\bS + \lambda \bI )^{-1} \bV^T \bX^T \by }{2}^2 \\
        &= \norm{\bX \bV \left(\bS ^{-1} ( \bI - (\bI - \eta \bS)^t ) - (\bS + \lambda \bI )^{-1} \right) \bV^T \bX^T \by  }{2} \\
        &=  \by^T \bV \bX \left( \underbrace{\bS ^{-1} ( \bI - (\bI - \eta \bS)^t ) - (\bS + \lambda \bI )^{-1}}_{\RN{1}} \right)^2 \bS \bV^T \bX^T \by \label{eq:train_GD_rel} \,.
        %  (\bX \bV \bS ^{-1} ( \bI - (\bI - \eta \bS)^k ) \bV^T \bX^T \by)^T \bX \bV \bS ^{-1} ( \bI - (\bI - \eta \bS)^k ) \bV^T \bX^T \by
    \end{align}
    We now separately consider term 1. 
    Substituting $\lambda = \frac{1}{t \eta}$, 
    we get
    \begin{align}
        \bS ^{-1} ( \bI - (\bI - \eta \bS)^t ) - (\bS + \lambda \bI )^{-1} &= \bS^{-1} \left( ( \bI - (\bI - \eta \bS)^t ) - (\bI + \bS^{-1} \lambda )^{-1}\right) \\
        &= \underbrace{\bS^{-1} \left( ( \bI - (\bI - \eta \bS)^t ) - (\bI + ( \bS t \eta)^{-1}  )^{-1}\right)}_{\bA} \,.
    \end{align}

    We now separately bound the diagonal entries in matrix $\bA$. 
    With $s_i$, we denote $i^{\text{th}}$ diagonal entry of $\bS$.
    Note that since $ \eta\le 1/\norm{S}{\text{op}}$, 
    for all $i$, $\eta s_i  \le 1$.  
    Consider $i^{\text{th}}$ diagonal term (which is non-zero) 
    of the diagonal matrix $\bA$, we have 
    \begin{align}
        \bA_{ii} = \frac{1}{s_i} \left(  1 - (1 - s_i \eta)^t - \frac{t \eta s_i}{1 + t \eta s_i } \right) &=  \frac{1 - (1 - s_i \eta)^t}{s_i} \left( \underbrace{ 1 - \frac{t \eta s_i}{(1 + t \eta s_i)(1 - (1 - s_i \eta)^t)}}_{\RN{2}} \right) \\ 
         &\le \frac{1}{2}\left[ \frac{1 - (1 - s_i \eta)^t}{ s_i} \right] \tag*{(Using \lemref{lem:ineq_soln} (b))} \,.
    \end{align} 
    Additionally, we can also show the following upper bound on term 2: 
    \begin{align}
         1 - \frac{t \eta s_i}{(1 + t \eta s_i)(1 - (1 - s_i \eta)^t)} &= \frac{(1 + t \eta s_i)(1 - (1 - s_i \eta)^t) - t \eta s_i }{(1 + t \eta s_i)(1 - (1 - s_i \eta)^t)} \\
         & \le  \frac{ 1 -  (1 - s_i \eta)^t - t \eta s_i (1 - s_i \eta)^t}{(1 + t \eta s_i)(1 - (1 - s_i \eta)^t)} \\
         & \le \frac{1}{t\eta s_i} \,. \tag{Using \lemref{lem:ineq_soln} (a)}
        %  &\le \frac{1}{2}\left[ \frac{1 - (1 - s_i \eta)^t}{ s_i} \right] \tag*{(Using \lemref{lem:ineq_soln})} \,.
    \end{align} 

    Combining both the upper bounds 
    on each diagonal entry $\bA_{ii}$, we have 
    \begin{align}
    \bA \preceq c_1(\eta, t) \cdot \bS^{-1} ( \bI - (\bI - \eta \bS)^t ) \,, \label{eq:upperbound_diagonal}
    \end{align}
    where $c_1(\eta, t ) = \min(0.5, \frac{1}{t s_i \eta })$. Plugging this into \eqref{eq:train_GD_rel}, we have 
    \begin{align}
        \Expt{ x \sim \calS }{\left(f_t(x) - \wt f_\lambda (x)\right)^2} &\le c(\eta, t) \cdot \by^T \bV \bX  \left( \bS^{-1} ( \bI - (\bI - \eta \bS)^t ) \right)^2 \bS \bV^T \bX^T \by \\
        &=   c(\eta, t) \cdot \by^T \bV \bX  \left( \bS^{-1} ( \bI - (\bI - \eta \bS)^t ) \right) \bS \left( \bS^{-1} ( \bI - (\bI - \eta \bS)^t ) \right) \bV^T \bX^T \by \\
        & =  c(\eta, t) \cdot \norm{\bX w_t}{2}^2 \\
        &= c(\eta, t) \cdot  \Expt{ x \sim \calS }{\left(f_t(x) \right)^2} \,,
    \end{align}
    where $c(\eta, t ) = \min(0.25, \frac{1}{t^2 s^2_i \eta^2 })$.

    \textbf{Part 2 {} {}} With $\bSigma$, 
    we denote the underlying true covariance matrix. 
    We now consider the squared difference of output at an unseen point: 
    \begin{align}
        \Expt{ x \sim \calD_{\calX} }{\left(f_t(x) - \wt f_\lambda (x)\right)^2} &= \Expt{x \sim \calD_{\calX}}{\norm{x^T w_t - x^T \wt w_\lambda}{2}} \\
        &=   \norm{x^T \bV \bS ^{-1} ( \bI - (\bI - \eta \bS)^t ) \bV^T \bX^T \by - x^T \bV (\bS + \lambda \bI )^{-1} \bV^T \bX^T \by }{2} \\
        &= \norm{x^T \bV \left(\bS ^{-1} ( \bI - (\bI - \eta \bS)^t ) - (\bS + \lambda \bI )^{-1} \right) \bV^T \bX^T \by  }{2} \\
        &= \by^T \bV \bX \left( \bS ^{-1} ( \bI - (\bI - \eta \bS)^t ) - (\bS + \lambda \bI )^{-1} \right) \bV^T \bSigma \bV \\ &\qquad \qquad \qquad \qquad \qquad \left( (\bI - (\bI - \eta \bS)^t ) - (\bS + \lambda \bI )^{-1} \right) \bV^T \bX^T \by \\
        &\le \sigma_{\text{max}} \cdot \by^T \bV \bX \left( \underbrace{\bS ^{-1} ( \bI - (\bI - \eta \bS)^t ) - (\bS + \lambda \bI )^{-1}}_{\RN{1}} \right)^2 \bV^T \bX^T \by \,, \label{eq:test_GD_rel}
        %  (\bX \bV \bS ^{-1} ( \bI - (\bI - \eta \bS)^k ) \bV^T \bX^T \by)^T \bX \bV \bS ^{-1} ( \bI - (\bI - \eta \bS)^k ) \bV^T \bX^T \by
    \end{align}
    where $\sigma_{\text{max}}$ is the maximum eigenvalue 
    of the underlying covariance matrix $\bSigma$. 
    Using the upper bound on term 1 in \eqref{eq:upperbound_diagonal}, 
    we have 
    \begin{align}
        \Expt{ x \sim \calD_{\calX} }{\left(f_t(x) - \wt f_\lambda (x)\right)^2} &\le \sigma_{\text{max}} \cdot c(\eta, t) \cdot \by^T \bV \bX  \left( \bS^{-1} ( \bI - (\bI - \eta \bS)^t ) \right)^2 \bV^T \bX^T \by \\
        &=   \kappa \cdot c(\eta, t) \cdot \sigma_{\text{min}}\cdot \norm{\bV \left( \bS^{-1} ( \bI - (\bI - \eta \bS)^t ) \right) \bV^T \bX^T \by}{2}^2 \\
        &\le \kappa \cdot c(\eta, t) \cdot \left[ \bV \left( \bS^{-1} ( \bI - (\bI - \eta \bS)^t ) \right) \bV^T \bX^T \right]^T \bSigma \\
        &\qquad \qquad \qquad \qquad \qquad \left[ \bV \left( \bS^{-1} ( \bI - (\bI - \eta \bS)^t ) \right) \bV^T \bX^T \right] \by \\
        & = \kappa \cdot c(\eta, t) \cdot \Expt{x \sim \calD_{\calX}}{\norm{x^T w_t}{2}} \,.
    \end{align}
% 
% 
    % Since $ \eta\le 1/\norm{S}{\text{op}}$, invoking \lemref{lem:ineq_soln} to upper bound term 1 with
\end{proof}

\subsection{Extension to deep learning} \label{appsubsec:ext_DL}
Under \asmpref{appsubsec:justifying_assumption1}, we present the formal result parallel to \thmref{thm:multiclass_ERM}. 
\begin{theorem} \label{thm:multiclass_ERM_algoA}
    Consider a multiclass classification problem 
    with $k$ classes. Under \asmpref{asmp:deep_models}, 
    for any $\delta >0$, with probability at least $1-\delta$,
    we have
    \vspace{-10pt}
    \begin{align*}
        \error_\calD(\widehat f)  \le \error_\calS(\widehat f) + (k-1) \left(1 - \tfrac{k}{k-1} \error_{\wt\calS}(\widehat f)\right) + c\sqrt{\frac{\log(\frac{4}{\delta})}{2m}} \,,\numberthis \label{eq:multiclass_ERM_deep}
    % \vspace{-20pt}
    \end{align*}
    for some constant $c \le ((c+1) k+\sqrt{k} + \frac{m}{n\sqrt{k}})$.
\end{theorem}

The proof follows exactly as in step (i) to (iii) in \thmref{thm:multiclass_ERM}.  

\subsection{Justifying~\asmpref{asmp:deep_models}} \label{appsubsec:justifying_assumption1}

Motivated by the analysis on linear models, we now discuss alternate (and weaker) conditions that imply \asmpref{asmp:deep_models}. 
We need hypothesis stability (\codref{cond:hypothesis_stability}) and the following assumption relating training error and leave-one-error: 

\begin{assumption} \label{asmp:loo_error}
Let $\wh f$ be a model obtained by training with algorithm $\calA$ on a mixture of clean $S$ and randomly labeled data $\wt S$. Then we assume we have 
\begin{align*}
    \error_{\wt \calS_M} (\wh f) \le  \error_{\text{LOO} (\wt S_M)} \,, 
\end{align*}
for all $(x_i, y_i) \in  \wt S_M$ where $\wh f_{(i)} \defeq f(\calA, S \cup {{}\wt S_M}_{(i)})$ and  $\error_{\text{LOO} (\wt S_M)} \defeq  \frac{\sum_{(x_i, y_i) \in \wt S_M} \error(\ff{i}(x_i), y_i ) }{\abs{\wt \calS_M}}$.  
\end{assumption}

% we assume this to extend our result (parallel to \thmref{thm:multi_linear}) for deep models. 
Intuitively, this assumption states that the error on a (mislabeled) datum $(x,y)$ included in the training set is less than the error on that datum $(x,y)$ obtained by a model trained on the training set $S - \{(x,y)\}$. We proved this for linear models trained with GD in the proof of \thmref{thm:multi_linear}. 
% 
\codref{cond:hypothesis_stability} with $\beta = \calO(1)$ and \asmpref{asmp:loo_error} together with \lemref{lem:stability_error} implies \asmpref{asmp:deep_models} with a polynomial residual term (instead of logarithmic in $1/\delta$): 
\begin{align}
     \error_{\calS_M} (\wh f) \le  \error_{\calDm}(\wh f)   + \sqrt{\frac{1}{\delta}\left(\frac{1}{m} +\frac{3\beta}{m+n} \right)} \,.
\end{align}
% Note that this  

\newpage 
\section{Additional experiments and details}\label{app:exp}
\newcommand\tab[1][1cm]{\hspace*{#1}}

\subsection{Datasets} \label{sec:app_dataset}

\textbf{Toy Dataset {} {}} Assume fixed constants $\mu$ and $\sigma$. For a given label $y$, we simulate features $x$ in our toy classification setup as follows: 
\begin{align*}
    x \defeq \texttt{concat} \left[ x_1, x_2\right] \quad \text{where} \quad  x_1 \sim  \calN( y \cdot \mu, \sigma^2 I_{d \times d}) \ \  \text{and} \ \  x_1 \sim  \calN( 0, \sigma^2 I_{d \times d}) \,.
\end{align*}  
% where $y$ is the true label and $x$ is the corresponding feature vector. 
In experiements throughout the paper, we fix dimention $d=100$, $\mu = 1.0 $, and $\sigma = \sqrt{d}$. Intuitively, $x_1$ carries the information about the underlying label and $x_2$ is additional noise independent of the underlying label. 

\textbf{CV datasets {} {}} We use MNIST~\citep{lecun1998mnist} and CIFAR10~\cite{krizhevsky2009learning}. 
% For binary tasks, 
We produce a binary variant from the multiclass classification problem by mapping classes $\{0,1,2,3,4\}$ to label $1$ and $\{ 5,6,7,8,9\}$ to label $-1$. For CIFAR dataset, we also use the standard data augementation of random crop and horizontal flip. PyTorch code is as follows: 

\texttt{(transforms.RandomCrop(32, padding=4),\\
\tab transforms.RandomHorizontalFlip())}

\textbf{NLP dataset {} {}} We use IMDb Sentiment analysis~\citep{maas2011learning} corpus.  

\subsection{Architecture Details} 

All experiments were run on NVIDIA GeForce RTX 2080 Ti GPUs. We used PyTorch~\citep{NEURIPS2019a9015} and Keras with Tensorflow~\citep{abadi2016tensorflow} backend for experiments. 
% , ELMo embeddings~\citep{Peters:2018}, and Hugging Face Transformers~\citep{wolf-etal-2020-transformers}. 

\textbf{Linear model {} {}} For the toy dataset, we simulate a linear model with scalar output and the same number of parameters as the number of dimensions.   

\textbf{Wide nets {} {}} To simulate the NTK regime, we experiment with $2-$layered wide nets. The PyTorch code for 2-layer wide MLP is as follows: 


\texttt{ nn.Sequential( \\
\tab     nn.Flatten(),\\
\tab    nn.Linear(input\_dims, 200000, bias=True),\\
\tab    nn.ReLU(),\\
\tab    nn.Linear(200000, 1, bias=True)\\
\tab     )}


We experiment both (i) with the second layer fixed at random initialization; (ii)  and updating both layers' weights.     

\textbf{Deep nets for CV tasks {} {}} We consider a 4-layered MLP. The PyTorch code for 4-layer MLP is as follows: 

\texttt{ nn.Sequential(nn.Flatten(), \\
\tab        nn.Linear(input\_dim, 5000, bias=True),\\
\tab        nn.ReLU(),\\
\tab        nn.Linear(5000, 5000, bias=True),\\
\tab        nn.ReLU(),\\
\tab        nn.Linear(5000, 5000, bias=True),\\
\tab        nn.ReLU(),\\
% \tab        nn.Linear(5000, 5000, bias=True),\\
% \tab        nn.ReLU(),\\
\tab        nn.Linear(1024, num\_label, bias=True)\\
\tab        )}

For MNIST, we use $1000$ nodes instead of $5000$ nodes in the hidden layer. 
% 
We also experiment with convolutional nets. In particular, we use ResNet18 \citep{he2016deep}. Implementation adapted from:  \url{https://github.com/kuangliu/pytorch-cifar.git}. 

\textbf{Deep nets for NLP {} {}} We use a simple LSTM model with embeddings intialized with ELMo embeddings~\citep{Peters:2018}. Code adapted from: \url{https://github.com/kamujun/elmo_experiments/blob/master/elmo_experiment/notebooks/elmo_text_classification_on_imdb.ipynb} 

We also evaluate our bounds with a BERT model. In particular, we fine-tune an off-the-shelf uncased BERT model~\citep{devlin2018bert}. Code adapted from Hugging Face Transformers~\citep{wolf-etal-2020-transformers}: \url{https://huggingface.co/transformers/v3.1.0/custom_datasets.html}. 


\subsection{Additonal experiments}

\textbf{Results with SGD on underparameterized linear models {} {}} 

\begin{figure*}[h]
    \centering 
    % \vspace{-15pt}
    % \includegraphics[width=0.9\linewidth]{example-image-a}
    \includegraphics[width=0.3\linewidth]{figures/lowdim-Gaussian-SGD.pdf}
    % \includegraphics[width=0.9\linewidth]{figures/{CIFAR10_rn=0.1_lr=0.2_wd=0.005}.png}
    \vspace{-5pt}
    \caption{ 
    % Predicted lower bound 
    % on different
    We plot the accuracy and corresponding bound 
    (RHS in \eqref{eq:erm}) at $\delta = 0.1$
    for toy binary classification task. 
    Results aggregated over $3$ seeds. 
    % i.e., $1-\error$ where $\error$ is the term in the RHS of \eqref{eq:erm}
    Accuracy vs fraction of unlabeled data (w.r.t clean data) 
    in the toy setup with a linear model trained with SGD. Results parallel to \figref{fig:error_binary}(a) with SGD.  }
    \label{fig:error_binary_linear}
    \vspace{-5pt}
\end{figure*}

\textbf{Results with wide nets on binary MNIST {} {}}

\begin{figure*}[h]
    \centering 
    % \vspace{-15pt}
    % \includegraphics[width=0.9\linewidth]{example-image-a}
    \subfigure[GD with MSE loss]{\includegraphics[width=0.3\linewidth]{figures/MNIST-GD_MSE.pdf}} \hfil
    \subfigure[SGD with CE loss]{\includegraphics[width=0.3\linewidth]{figures/MNIST-SGD_CE.pdf}}
    \subfigure[SGD with MSE loss]{\includegraphics[width=0.3\linewidth]{figures/MNIST-SGD_MSE-first-layer.pdf}}
    % \includegraphics[width=0.9\linewidth]{figures/{CIFAR10_rn=0.1_lr=0.2_wd=0.005}.png}
    \vspace{-5pt}
    \caption{ 
    % Predicted lower bound 
    % on different
    We plot the accuracy and corresponding bound 
    (RHS in \eqref{eq:erm}) at $\delta = 0.1$ 
    for binary MNIST classification. 
    Results aggregated over $3$ seeds. 
    % i.e., $1-\error$ where $\error$ is the term in the RHS of \eqref{eq:erm}
    Accuracy vs fraction of unlabeled data 
    for a 2-layer wide network on binary MNIST with both the layers training in (a,b) and only first layer training in (c). 
    Results parallel to \figref{fig:error_binary}(b) .  }
    \label{fig:error_binary_MNIST}
    \vspace{-5pt}
\end{figure*}

% \begin{figure*}[h]
%     \centering 
%     % \vspace{-15pt}
%     % \includegraphics[width=0.9\linewidth]{example-image-a}
%     \subfigure[GD with MSE loss]{\includegraphics[width=0.3\linewidth]{figures/MNIST.pdf}} \hfil
    
%     \subfigure[SGD with CE loss]{\includegraphics[width=0.3\linewidth]{figures/MNIST.pdf}}
%     % \includegraphics[width=0.9\linewidth]{figures/{CIFAR10_rn=0.1_lr=0.2_wd=0.005}.png}
%     \vspace{-5pt}
%     \caption{ 
%     % Predicted lower bound 
%     % on different
%     We plot the accuracy and corresponding bound 
%     (RHS in \eqref{eq:erm}) at $\delta = 0.1$
%     for binary MNIST classification. 
%     Results aggregated over $3$ seeds. 
%     % i.e., $1-\error$ where $\error$ is the term in the RHS of \eqref{eq:erm}
%     Accuracy vs fraction of unlabeled data 
%     for a 2-layer wide network on binary MNIST with just the first layer training. 
%     Results parallel to \figref{fig:error_binary}(b) with only the first layer training.  }
%     \label{fig:error_binary_MNIST}
%     \vspace{-5pt}
% \end{figure*}

\textbf{Results on CIFAR 10 and MNIST {} {}} 
% 
We plot epoch wise error curve for results in \tabref{table:multiclass}(\figref{fig:error_epoch_CIFAR10} and \figref{fig:error_epoch_MNIST}). We observe the same trend as in \figref{fig:error_CIFAR10}. Additionally, we plot an \emph{oracle bound} obtained by tracking the error on mislabeled data which nevertheless were predicted as true label. To obtain an exact emprical value of the oracle bound, we need underlying true labels for the randomly labeled data. 
% Note that our bound in \thmref{thm:multiclass_ERM}, lower bounds the accuracy as predicted by the oracle bound. 
While with just access to extra unlabeled data we cannot calculate oracle bound, we note that the oracle bound is very tight and never violated in practice underscoring an importamt aspect of generalization in multiclass problems. This highlight that even a stronger conjecture may hold in multiclass classification, i.e., error on mislabeled data (where nevertheless true label was predicted) lower bounds the population error on the distribution of mislabeled data and hence, the error on (a specific) mislabeled portion predicts the population accuracy on clean data. 
% 
On the other hand, the dominating term of in \thmref{thm:multiclass_ERM} is loose when compared with the oracle bound. The main reason, we believe is the pessimistic upper bound in \eqref{eq:lemma1_final_multi_prev} in the proof of \lemref{lem:fit_mislabeled_multi}. We leave an investigation on this gap for future. 
% of fit 

% However, oracle bound highlights two . One,  



\begin{figure}[h]
    \centering 
    % \vspace{-15pt}
    % \includegraphics[width=0.9\linewidth]{example-image-a}
    \subfigure[MLP]{\includegraphics[width=0.3\linewidth]{figures/CIFAR10-FNN.pdf}} \hfil
    \subfigure[ResNet]{\includegraphics[width=0.3\linewidth]{figures/CIFAR10-Resnet.pdf}}
    % \includegraphics[width=0.9\linewidth]{figures/{CIFAR10_rn=0.1_lr=0.2_wd=0.005}.png}
    % \vspace{-10pt}
    \caption{ Per epoch curves for CIFAR10 corresponding results in \tabref{table:multiclass}. As before, we just plot the dominating term in the RHS of \eqref{eq:multiclass_ERM} as predicted bound. Additionally, we also plot the predicted lower bound by the error on mislabeled data which nevertheless were predicted as true label. We refer to this as ``Oracle bound''. See text for more details. 
    % 
    % except for the stopping point. 
    % The bound predicted by RATT (RHS in \eqref{eq:multiclass_ERM}) is vacuous. 
    }\label{fig:error_epoch_CIFAR10}
    % \vspace{-15pt}
\end{figure}


\begin{figure}[h]
    \centering 
    % \vspace{-15pt}
    % \includegraphics[width=0.9\linewidth]{example-image-a}
    \subfigure[MLP]{\includegraphics[width=0.3\linewidth]{figures/MNIST-FNN.pdf}} \hfil
    \subfigure[ResNet]{\includegraphics[width=0.3\linewidth]{figures/MNIST-Resnet.pdf}}
    % \includegraphics[width=0.9\linewidth]{figures/{CIFAR10_rn=0.1_lr=0.2_wd=0.005}.png}
    % \vspace{-10pt}
    \caption{ Per epoch curves for MNIST corresponding results in \tabref{table:multiclass}. As before, we just plot the dominating term in the RHS of \eqref{eq:multiclass_ERM} as predicted bound. Additionally, we also plot the predicted lower bound by the error on mislabeled data which nevertheless were predicted as true label. We refer to this as ``Oracle bound''. See text for more details. 
    % 
    % except for the stopping point. 
    % The bound predicted by RATT (RHS in \eqref{eq:multiclass_ERM}) is vacuous. 
    }\label{fig:error_epoch_MNIST}
    % \vspace{-15pt}
\end{figure}

\textbf{Results on CIFAR 100 {} {}} 
% 
On CIFAR100, our bound in \eqref{eq:multiclass_ERM} yields vacous bounds. However, the oracle bound as explained above yields tight guarantees in the initial phase of the learning (i.e., when learning rate is less than $0.1$) (\figref{fig:error_CIFAR100}).  

\begin{figure}[h]
    \centering 
    % \vspace{-15pt}
    % \includegraphics[width=0.9\linewidth]{example-image-a}
    \includegraphics[width=0.3\linewidth]{figures/CIFAR100-Resnet.pdf}
    % \includegraphics[width=0.9\linewidth]{figures/{CIFAR10_rn=0.1_lr=0.2_wd=0.005}.png}
    % \vspace{-10pt}
    \caption{ Predicted lower bound by the error on mislabeled data which nevertheless were predicted as true label with ResNet18 on CIFAR100. We refer to this as ``Oracle bound''. See text for more details. 
    % 
    % except for the stopping point. 
    The bound predicted by RATT (RHS in \eqref{eq:multiclass_ERM}) is vacuous. 
    }\label{fig:error_CIFAR100}
    % \vspace{-15pt}
\end{figure}


% \paragraph{Experiments on CIFAR100} 


% \subsection{Model Selection using RATT}


\subsection{Hyperparameter Details}


\textbf{\figref{fig:error_CIFAR10} {} {}} We use clean training dataset of size $40,000$. We fix the amount of unlabeled data at $20\%$ of the clean size, i.e. we include additional $8,000$ points with randomly assigned labels. We use test set of $10,000$ points. For both MLP and ResNet, we use SGD with an initial learning rate of $0.1$ and momentum $0.9$. We fix the weight decay parameter at $5\times 10^{-4}$. After $100$ epochs, we decay the learning rate to $0.01$. We use SGD batch size of $100$. 

\textbf{\figref{fig:error_binary} (a) {} {}} We obtain a toy dataset according to the process described in \secref{sec:app_dataset}. We fix $d=100$ and create a dataset of $50,000$ points with balanced classes. Moreover, we sample additional covariates with the same procedure to create randomly labeled dataset. For both SGD and GD training, we use a fixed learning rate $0.1$.    

\textbf{\figref{fig:error_binary} (b) {} {}} Similar to binary CIFAR, we use clean training dataset of size $40,000$ and fix the amount of unlabeled data at $20\%$ of the clean dataset size. To train wide nets, we use a fixed learning of $0.001$ with GD and SGD. We decide the weight decay parameter and the early stopping point that maximizes our generalization bound (i.e. without peeking at unseen data ).  We use SGD batch size of $100$. 

\textbf{\figref{fig:error_binary} (c) {} {}} With IMDb dataset, we use a clean dataset of size $20,000$ and as before, fix the amount of unlabeled data at $20\%$ of the clean data. To train ELMo model, we use Adam optimizer with a fixed learning rate $0.01$ and weight decay $10^{-6}$ to minimize cross entropy loss. We train with batch size $32$ for 3 epochs. To fine-tune BERT model, we use Adam optimizer with learning rate $5\times 10^{-5}$ to minimize cross entropy loss. We train with a batch size of $16$ for 1 epoch.    

\textbf{\tabref{table:multiclass} {} {}} For multiclass datasets, we train both MLP and ResNet with the same hyperparameters as described before. We sample a clean training dataset of size $40,000$ and fix the amount of unlabeled data at $20\%$ of the clean size. We use SGD with an initial learning rate of $0.1$ and momentum $0.9$. We fix the weight decay parameter at $5\times 10^{-4}$. After $30$ epochs for ResNet and after $50$ epochs for MLP, we decay the learning rate to $0.01$.  We use SGD with batch size $100$. 
For \figref{fig:error_CIFAR100}, we use the same hyperparameters as 
CIFAR10 training, except we now decay learning rate after $100$ epochs. 


In all experiments, to identify the best possible accuracy on just the clean data, we use the exact same set of hyperparamters except the stopping point. We choose a stopping point that maximizes test performance. 

\subsection{Summary of experiments }

\begin{center}
    \begin{table}[H] 
        \centering
        \begin{tabular}{|c|c|c|c|} 
        \hline
        Classification type & Model category & Model & Dataset  \\ [0.5ex] 
        \hline
        \hline
        \multirow{10}{*}{Binary} & Low dimensional & Linear model & Toy Gaussain dataset  \\
                        \cline{2-4}
                         & Overparameterized 
                        %  & Linear model & Toy Gaussain dataset \\
                        %  \cline{3-4}
                        %  & & 2-layer wide net& Toy Gaussain dataset \\
                        %  \cline{3-4}
                         & \multirow{2}{*}{2-layer wide net} & \multirow{2}{*}{Binary MNIST} \\
                         & linear nets & &  
                         \\
                         \cline{2-4}                 
                         & \multirow{6}{*}{Deep nets} & \multirow{2}{*}{MLP} & Binary MNIST \\
                         \cline{4-4}
                         & &  & Binary CIFAR \\
                         \cline{3-4}
                         &  & \multirow{2}{*}{ResNet} & Binary MNIST \\
                         \cline{4-4}
                         & &  & Binary CIFAR \\
                         \cline{3-4}
                         &  & ELMo-LSTM model & IMDb Sentiment Analysis \\
                         \cline{3-4}
                         & & BERT pre-trained model & IMDb Sentiment Analysis \\
        \hline
        \multirow{5}{*}{Multiclass} & \multirow{5}{*}{Deep nets} & \multirow{2}{*}{MLP} & MNIST \\
                        \cline{4-4} 
                        & & & CIFAR10 \\                   
                        \cline{3-4}
                         &   & \multirow{3}{*}{ResNet} & MNIST \\
                         \cline{4-4}
                         &   & & CIFAR10 \\
                         \cline{4-4}
                         &   & & CIFAR100 \\
        \hline
        \end{tabular}
        % \caption{Summary of experiments performed} \label{table:experiments}
    \end{table}    
    % \footnotetext[6]{We use both MSE loss and cross-entropy loss.}
    % \footnotetext[6]{We try 2 variants: one with a fixed first layer and the other with both layers trainable.}
\end{center}

\newpage
\section{Proof of \lemref{lem:stability_error}} \label{app:proof_lem_error}

\begin{proof}[Proof of \lemref{lem:stability_error}]
    Recall, we have a training set $S \cup \wt S_C$. We defined leave-one-out error on mislabeled points as $$\error_{\text{LOO}(\wt S_M) } = \frac{\sum_{(x_i, y_i) \in \wt S_M} \error( f_{(i)}( x_i), y_i)}{ \abs{\wt S_M }} \,, $$
    where $f_{(i)} \defeq f(\calA, (S \cup \wt S)_{(i)})$. Define $S^\prime \defeq S \cup \wt S$. Assume $(x,y)$ and $(x^\prime,y^\prime)$ as i.i.d. samples from ${\calDm}$. 
    Using Lemma 25 in \citet{bousquet2002stability}, we have
    \begin{align*}
        \Expo{ \left( \error_{\calDm}(\wh f) -\error_{\text{LOO}(\wt S_M) } \right)^2 } \le & \Expt{ S^\prime, (x,y), (x^\prime,y^\prime) }{ \error(\wh f(x), y ) \error(\wh f(x^\prime), y^\prime )} - 2 \Expt{ S^\prime, (x,y) }{ \error(\wh f(x), y ) \error(f_{(i)}(x_i), y_i )} \\
        & + \frac{m_1-1}{m_1}\Expt{ S^\prime }{  \error(f_{(i)}(x_i), y_i )  \error(f_{(j)}(x_j), y_j )} + \frac{1}{m_1} \Expt{ S^\prime }{  \error(f_{(i)}(x_i), y_i ) } \,. \numberthis \label{eq:main_reln}
    \end{align*}
    We can rewrite the equation above as : 
    \begin{align*}
        \Expo{ \left( \error_{\calDm}(\wh f) -\error_{\text{LOO}(\wt S_M) } \right)^2 } \le &  \, \underbrace{\Expt{ S^\prime, (x,y), (x^\prime,y^\prime) }{ \error(\wh f(x), y ) \error(\wh f(x^\prime), y^\prime ) - \error(\wh f(x), y ) \error(f_{(i)}(x_i), y_i )}}_{\RN{1}} \\
        & + \underbrace{\Expt{ S^\prime }{  \error(f_{(i)}(x_i), y_i )  \error(f_{(j)}(x_j), y_j ) -  \error(\wh f(x), y ) \error(f_{(i)}(x_i), y_i )}}_{\RN{2}} \\ &+ \underbrace{\frac{1}{m_1} \Expt{ S^\prime }{  \error(f_{(i)}(x_i), y_i ) - \error(f_{(i)}(x_i), y_i )  \error(f_{(j)}(x_j), y_j ) }}_{\RN{3}} \,. \numberthis \label{eq:main_reln2}
    \end{align*}
    
    We will now bound term $\RN{3}$.  Using Cauchy-Schwarz's inequality, we have
    
    \begin{align}
        \Expt{ S^\prime }{  \error(f_{(i)}(x_i), y_i ) - \error(f_{(i)}(x_i), y_i )  \error(f_{(j)}(x_j), y_j ) }^2 &\le  \Expt{ S^\prime }{  \error(f_{(i)}(x_i), y_i ) }^2 \Expt{S^\prime}{1 -   \error(f_{(j)}(x_j), y_j ) }^2 \\
        &\le \frac{1}{4} \,.\label{eq:term1_lem12}
    \end{align}
    
    Note that since $(x_i,y_i)$, $(x_j ,y_j )$, $(x,y)$, and $(x^\prime, y^\prime)$ are all from same distribution $\calDm$, we directly incorporate the bounds on term $\RN{1}$ and $\RN{2}$ from the proof of Lemma 9 in \citet{bousquet2002stability}. Combining that with \eqref{eq:term1_lem12} and our definition of hypothesis stability in \codref{cond:hypothesis_stability}, we have the required claim. 
    
    
    % We now re-write term $\RN{1}$ as
    % \begin{align*}
    %         &\Expt{S^\prime, (x,y), (x^\prime,y^\prime) }{ \error(\wh f(x), y ) \error(\wh f(x^\prime), y^\prime ) - \error(\wh f(x), y ) \error(f_{(i)}(x_i), y_i )} \\ & \qquad = \Expt{ S^\prime, (x,y), (x^\prime,y^\prime) }{ \error(\wh f(x), y ) \error(\wh f  (x^\prime), y^\prime ) - \error(\wh f ^\prime(x), y ) \error(f_{(i)}(x^\prime), y^\prime )} \tag{Exchanging $(x_i, y_i)$ with $(x^\prime, y^\prime)$ in the second term} \\
    %         & \qquad = \Expt{ S^\prime, (x,y), (x^\prime,y^\prime) }{  \left(\error(\wh f(x), y )-  \error(f_{(i)}(x), y ) \right) \error(\wh f  (x^\prime), y^\prime )  } \\
    %         & \qquad  + \Expt{ S^\prime, (x,y), (x^\prime,y^\prime) }{  \left(\error(f_{(i)}(x), y ) -\error(\wh f ^\prime(x), y ) \right) \error(\wh f  (x^\prime), y^\prime )}  \\
    %         & \qquad +\Expt{ S^\prime, (x,y), (x^\prime,y^\prime) }{  \left( \error(\wh f  (x^\prime), y^\prime ) -  \error(f_{(i)}(x^\prime), y^\prime ) \right) \error(\wh f ^\prime(x), y ) }  \,, \numberthis \label{eq:term1_final}
    % \end{align*}
    % where $\wh f^\prime$ is the classifier obtained by training on $ S^\prime_{(i)} \cup \{ (x^\prime, y^\prime) \} $. Similarly we can re-write term $\RN{2}$ as 
    % \begin{align*}
    %     & \Expt{ S^\prime }{  \error(f_{(i)}(x_i), y_i )  \error(f_{(j)}(x_j), y_j ) -  \error(\wh f(x), y ) \error(f_{(i)}(x_i), y_i )} \\
    %     &\quad  = \Expt{ S^\prime, (x,y), (x^\prime,y^\prime)}{  \error(f^{\prime\prime}_{(i)}(x), y )  \error(f_{(j)}^{\prime}(x^\prime), y^\prime ) -  \error(\wh f(x), y ) \error(f_{(i)}(x_i), y_i )} \tag{Exchanging $(x_i, y_i)$ with $(x, y)$ and $(x_j, y_j)$ with $(x^\prime, y^\prime)$ in the first term}\\
    %     &\quad = \Expt{ S^\prime, (x,y), (x^\prime,y^\prime)}{  \error(f^{\prime\prime}_{(j)}(x), y )  \error(f_{(i)}^{\prime}(x^\prime), y^\prime ) -  \error(\wh f^\prime (x), y ) \error(f^\prime_{(j)}(x^\prime), y^\prime )} \tag{Exchanging $(x_i, y_i)$ and $(x_j, y_j)$ and then replacing $(x_j, y_j)$ with $(x^\prime, y^\prime)$ in the second term} \\
    %     & \quad = \Expt{ S^\prime, (x,y), (x^\prime,y^\prime) }{  \left( \error(f_{(i)}^{\prime}(x^\prime), y^\prime )   -  \error(\wh f^{\prime\prime}  (x^\prime), y^\prime ) \right)  \error(f^{\prime\prime}_{(j)}(x), y )   } \\
    %     & \quad  + \Expt{ S^\prime, (x,y), (x^\prime,y^\prime) }{  \left( \error(f^{\prime\prime}_{(j)}(x), y )  -\error(\wh f ^\prime(x), y ) \right) \error(\wh f^{\prime\prime}  (x^\prime), y^\prime )  }  \\
    %     & \quad+ \Expt{ S^\prime, (x,y), (x^\prime,y^\prime) }{  \left( \error(\wh f^{\prime\prime}  (x^\prime), y^\prime )  -  \error(f^\prime_{(j)}(x^\prime), y^\prime ) \right)  \error(\wh f^\prime (x), y ) }   \\
    %     & \quad = \Expt{ S^\prime, (x,y), (x^\prime,y^\prime) }{  \left( \error(f_{(i)}^{\prime}(x^\prime), y^\prime )   -  \error(\wh f (x^\prime), y^\prime ) \right)  \error(f_{(i)}(x_j), y_j )   } \\
    %     & \quad  + \Expt{ S^\prime, (x,y), (x^\prime,y^\prime) }{  \left( \error(f^{\prime\prime}_{(j)}(x), y )  -\error(\wh f (x), y ) \right) \error(\wh f^{\prime\prime}  (x_j), y_j )  }  \\
    %     & \quad+ \Expt{ S^\prime, (x,y), (x^\prime,y^\prime) }{  \left( \error(\wh f^{\prime\prime}  (x^\prime), y^\prime )  -  \error(f^\prime_{(j)}(x^\prime), y^\prime ) \right)  \error(\wh f^\prime (x^\prime), y^\prime ) }  \,, \numberthis \label{eq:term2_final}
    % \end{align*}
    % where $f^{\prime\prime}_{(j)}$ is trained on $S^\prime_{(j,i)} \cup {(x,y)}$, $f^{\prime}_{(i)}$ is trained on $S^\prime_{(j,i)} \cup {(x^\prime,y^\prime)}$, and $\wh f^{\prime\prime} $ is trained on $S^\prime_{(j)} \cup {(x,y)}$. Note in the last line we replaced $(x,y)$ by $(x_j, y_j)$ in the first term, replaced $(x^\prime,y^\prime)$ by $(x_j, y_j)$ in the second term and exchanged $(x_i,y_i)$ with $(x_j,y_j)$ and also $(x,y)$ and $(x^\prime, y^\prime)$
    
    
\end{proof}


% 
% 16th Century Version Control 
% 

% \onecolumn

% \section*{Supplementary Material}
% We will be using the following standard results
% on exponential concentration of random variables 
% all throughout the discussion:

% \begin{lemma}[Hoeffding's inequality for independent RVs~\citep{hoeffding1994probability}] Let $Z_1, Z_2, \ldots, Z_n$ be independent bounded random variables with $Z_i \in [a,b]$ for all $i$, then 
%     \begin{align*}
%         \prob\left( \frac{1}{n} \sum_{i=1}^n (Z_i - \Expo{Z_i}) \ge t \right) \le \exp{\left( -\frac{2nt^2}{(b-a)^2} \right) }
%     \end{align*} 
%     and 
%     \begin{align*}
%         \prob\left( \frac{1}{n} \sum_{i=1}^n (Z_i - \Expo{Z_i}) \le -t \right) \le \exp{\left( -\frac{2nt^2}{(b-a)^2} \right) }
%     \end{align*} 
%     for all $t \ge 0$. 
% \end{lemma}

% \begin{lemma}[Hoeffding's inequality for sampling with replacement~\citep{hoeffding1994probability}] \label{lem:hoeffding_sampling} Let $\calZ = (Z_1, Z_2, \ldots, Z_N)$ be a finite population of $N$ points with $Z_i \in [a.b]$ for all $i$. Let $X_1, X_2, \ldots X_n$ be a random sample drawn without replacement from $\calZ$. Then for all $t \ge 0$, we have 
%     \begin{align*}
%         \prob\left( \frac{1}{n} \sum_{i=1}^n (X_i - \mu ) \ge t \right) \le \exp{\left( -\frac{2nt^2}{(b-a)^2} \right) }
%     \end{align*} 
%     and 
%     \begin{align*}
%         \prob\left( \frac{1}{n} \sum_{i=1}^n (X_i - \mu ) \le -t \right) \le \exp{\left( -\frac{2nt^2}{(b-a)^2} \right) } \,,
%     \end{align*} 
%     where $\mu = \frac{1}{N} \sum_{i=1}^{N} Z_i$. 
% \end{lemma}

% We now discuss one condition that generalizes the exponential concentration to dependent random variables.
% \begin{condition}[Bounded difference inequality] \label{cond:BDC} Let $\calZ$ be some set and $\phi: \calZ^n \to \Real$. We say that $\phi$ satisfies the bounded difference assumption if 
% there exists $c_1, c_2, \ldots c_n \ge 0$ s.t. for all $i$, we have 
% \begin{align*}
%     \sup_{Z_1,Z_2, \ldots,Z_n, Z_i^\prime in \calZ^{n+1} } \abs{\phi (Z_1, \ldots, Z_i, \ldots, Z_n ) - \phi (Z_1, \ldots, Z_i^\prime, \ldots, Z_n ) } \le c_i \,.
% \end{align*} 
% \end{condition}

% \begin{lemma}[McDiarmid’s inequality~\citep{mcdiarmid1989}] \label{lem:McDiarmid} Let $Z_1, Z_2, \ldots, Z_n$ be independent random variables on set $\calZ$ and $\phi : \calZ^n \to \Real$ satisfy bounded difference assumption (\codref{cond:BDC}). Then for all $t>0$, we have 
%     \begin{align*}
%         \prob\left( \phi(Z_1, Z_2, \ldots, Z_n) - \Expo{\phi(Z_1, Z_2, \ldots, Z_n)} \ge t \right) \le \exp{\left( -\frac{2t^2}{\sum_{i=1}^n c_i^2} \right) } 
%     \end{align*} 
%     and 
%     \begin{align*}
%         \prob\left( \phi(Z_1, Z_2, \ldots, Z_n) - \Expo{\phi(Z_1, Z_2, \ldots, Z_n)} \le -t \right) \le \exp{\left( -\frac{2t^2}{\sum_{i=1}^n c_i^2} \right) } \,
%     \end{align*} 
% \end{lemma}


% \section{Proofs from \secref{sec:ERM_training}}\label{app:proof_erm}

% \textbf{Additional notation {} {}} Let $m_1$ be the number of mislabeled points ($\wt S_M$) and $m_2$ be the number of correctly labeled points ($\wt S_C$). Note $m_1 + m_2 = m$. 


% \subsection{Proof of \thmref{thm:error_ERM}}


% \begin{proof}[Proof of \lemref{lem:fit_mislabeled}] 
%     The main idea of our proof is to regard 
%     the clean portion of the data 
%     ($S \cup \wt S_C$) as fixed.   
%     Then, there exists a classifier $f^*$ 
%     that is optimal over draws 
%     of the mislabeled data $\wt S_M$. 
% % 
%     % 
%     Formally, 
%     \begin{align}
%     f^* \defeq \argmin_{f \in \calF} \error_{\widecheck {\calD}} (f) \,, \label{eq:modified_ERM}
%     \end{align}
%     where $$\widecheck \calD = \frac{n}{m+n} \calS + \frac{m_1}{m+n} \wt \calS_C  + \frac{m_2}{m+n}\calDm \,.$$ That is, $\widecheck \calD$ a combination of 
%     the \emph{empirical distribution} 
%     over correctly labeled data $S \cup \wt S_C$
%     % in $S\cup \wt S$ 
%     and the (population) distribution 
%     over mislabeled data $\calDm$.
%     Recall that 
%     \begin{align}
%     \wh f \defeq \argmin_{f \in \calF} \error_{\calS \cup \wt S} (f) \,. \label{eq:orig_ERM}
%     \end{align}
%     % 
%     % 
%     Since, $\widehat f$ minimizes 0-1 error 
%     on $S \cup \wt S$, using ERM optimality on \eqref{eq:orig_ERM},  
%     we have 
%     \begin{align}
%         \error_{\calS \cup \wt \calS}(\widehat f) \le \error_{
%             \calS \cup \wt \calS}(f^*) \,.    \label{eq:step1}
%     \end{align}
%     Moreover, since $f^*$ is independent of $\wt S_M$, using Hoeffding's bound,
%     % \footnote{For a fully rigorous argument,
%     % refer to the complete proof in App.~\ref{app:proof_erm}.} 
%     we have with probability at least $1-\delta$ that
%     \begin{align}
%       \error_{\wt \calS_M}(f^*) \le \error_{ \calDm}(f^*) +  \sqrt{\frac{\log(1/\delta)}{2 m_1}} \,. \label{eq:step2} 
%     \end{align}
%     %$ 
%     %for some constant $c_1\le 1/2$. 
%     Finally, since $f^*$ is the optimal classifier on $\widecheck \calD$, 
%     we have 
%     \begin{align}
%         \error_{\widecheck \calD}(f^*) \le \error_{\widecheck \calD}(\widehat f) \label{eq:step3}
%     \end{align}
%      Now to relate \eqref{eq:step1} and \eqref{eq:step3}, we can re-write the \eqref{eq:step2} as follows: 
%     \begin{align}
%         \error_{\calS \cup \wt\calS}(f^*) \le \error_{ \widecheck \calD}(f^*) +  \frac{m_1}{m+n}\sqrt{\frac{\log(1/\delta)}{2 m_1}} \,. \label{eq:step4} 
%     \end{align}
%     Now we combine equations \eqref{eq:step1}, \eqref{eq:step4}, and \eqref{eq:step3}, to get 
%     \begin{align}
%         \error_{\calS \cup \wt \calS}(\wh f) \le \error_{\widecheck \calD}(\wh f) +  \frac{m_1}{m+n}\sqrt{\frac{\log(1/\delta)}{2 m_1}} \,, 
%     \end{align}
%     which implies 
%     \begin{align}
%         \error_{ \wt \calS_M}(\wh f) \le \error_{\calDm}(\wh f) + \sqrt{\frac{\log(1/\delta)}{2 m_1}} \,. \label{eq:lemma1_final}
%     \end{align}
%     Since $\wt S$ is obtained by randomly labeling an unlabeled dataset, we assume $2m_1 \approx m$ \footnote{Formally, with probability at least $1-\delta$, we have  $(m - 2m_1)\le \sqrt{m\log(1/\delta)/2}$ }. Moreover, using $\error_{\calDm} = 1 - \error_{\calD}$ we obtain the desired result.   
%     % Combining the above steps and using the fact 
%     % that $\error_\calD = 1- \error_{\calDm} $, 
%     % we obtain the desired result.
% \end{proof}

% \begin{proof}[Proof of \lemref{lem:mislabeled_error}]
%     Recall $\error_{\wt S} (f) = \frac{m_1}{m} \error_{\wt S_M}(f) + \frac{m_2}{m} \error_{\wt S_C}(f)$. Hence, we have 
%     \begin{align}
%         2\error_{\wt S}(f) - \error_{\wt S_M}(f) - \error_{\wt S_C}(f) &= \left(\frac{2m_1}{m} \error_{\wt S_M}(f) - \error_{\wt S_M}(f)\right) + \left(\frac{2m_2}{m} \error_{\wt S_C}(f) - \error_{\wt S_C}(f)\right) \\ &= \left(\frac{2m_1}{m} - 1\right) \error_{\wt S_M}(f) + \left(\frac{2m_2}{m} - 1 \right)\error_{\wt S_C} (f) \,.
%     \end{align} 
%     Since the dataset is randomly labeled, with probability at least $1-\delta$, we have  $\left(\frac{2m_1}{m} - 1\right) \le \sqrt{\frac{\log(1/\delta)}{2m}}$. Similarly, we have with probability at least $1-\delta$, $\left(\frac{2m_2}{m} - 1\right) \le \sqrt{\frac{\log(1/\delta)}{2m}}$. Using union bound, we have with probability at least $1-\delta$
%     % \begin{align}
%     %     2\error_{\wt S} - \error_{\wt S_M}(f) - \error_{\wt S_C}(f) \le \sqrt{\frac{\log(2/\delta)}{2m}} \left(\error_{\wt S_M}(f) + \error_{\wt S_C}(f) \right) \le 2\sqrt{\frac{\log(2/\delta)}{2m}} \,. \label{eq:lemma2_final}
%     % \end{align}
%     \begin{align}
%         2\error_{\wt S} - \error_{\wt S_M}(f) - \error_{\wt S_C}(f) \le \sqrt{\frac{\log(2/\delta)}{2m}} \left(\error_{\wt S_M}(f) + \error_{\wt S_C}(f) \right) \,. \label{eq:lemma2_prefinal}
%     \end{align}
%     With re-arranging $\error_{\wt S_M}(f) + \error_{\wt S_C}(f)$ and using the inequality $ 1- a\le \frac{1}{1+a} $, we have  
%     \begin{align}
%         2\error_{\wt S} - \error_{\wt S_M}(f) - \error_{\wt S_C}(f) \le 2\error_{\wt \calS} \sqrt{\frac{\log(2/\delta)}{2m}}  \,. \label{eq:lemma2_final}
%     \end{align}

%     % We obtain the desired result by using 
% \end{proof}

% \begin{proof}[Proof of \lemref{lem:clear_error}]
% % Recall 0-1 error on each point  $(x,y) \in S \cup \wt S$ is given by $\I{ f(x)\ne y}$.
% In the set of correctly labeled points $S \cup \wt S_C$, we have $S$ as a random subset of $S \cup \wt S_C$. Hence, using Hoeffding's inequality for sampling without replacement (\lemref{lem:hoeffding_sampling}), we have with probability at least $1-\delta$
% \begin{align}
%     \error_{\wt \calS_c} (\wh f)- \error_{\calS \cup \wt \calS_C}( \wh f) \le  \sqrt{\frac{\log(1/\delta)}{2m_2}} \,.
% \end{align}
% Re-writing $\error_{\calS \cup \wt \calS_C}( \wh f)$ as $\frac{m_2}{m_2 + n} \error_{\wt \calS_C }(\wh f) + \frac{n}{m_2 + n} \error_{\calS }(\wh f)$, we have with probability at least $1-\delta$
% \begin{align}
%   \left(\frac{n}{n+m_2}\right) \left(\error_{\wt \calS_c} (\wh f)- \error_{\calS}( \wh f) \right) \le  \sqrt{\frac{\log(1/\delta)}{2m_2}} \,.
% \end{align}
% As before, assuming $2m_2 \approx m$, we have with probability at least $1-\delta$ 
% \begin{align}
%     \error_{\wt \calS_c} (\wh f)- \error_{\calS}( \wh f) \le \left(1+\frac{m_2}{n}\right)  \sqrt{\frac{\log(1/\delta)}{m}} \le 1.5 \sqrt{\frac{\log(1/\delta)}{m}} \,. \label{eq:lemma3_final}
% \end{align} 
% \end{proof}

% \begin{proof}[Proof of \thmref{thm:error_ERM}] 
%     Having established these core intermediate results, we can now combine above three lemmas to prove the main result. 
%     In particular, we bound the population error on clean data ($\error_\calD(\wh f)$) as follows:  
%     \begin{enumerate}[(i)]
%         \item First, use \eqref{eq:lemma1_final}, to obtain an upper bound on the population error on clean data, i.e., with probability at least $1-\delta/4$, we have
%         \begin{align}
%             \error_{ \calD} (\wh f) \le 1 - \error_{ \wt \calS_M}(\wh f) + \sqrt{\frac{\log(4/\delta)}{m}} \,. 
%         \end{align}
%         \item  Second, use \eqref{eq:lemma2_final}, to relate the error on the mislabeled fraction with error on clean portion of randomly labeled data and error on whole randomly labeled dataset, i.e., with probability at least $1-\delta/2$, we have 
%         \begin{align}
%             - \error_{\wt S_M}(f) \le \error_{\wt S_C}(f) - 2\error_{\wt S}  + \sqrt{\frac{\log(4/\delta)}{2m}}  \,. 
%         \end{align} 
%         \item Finally, use \eqref{eq:lemma3_final} to relate the error on the clean portion of randomly labeled data and error on clean training data, i.e., with probability $1-\delta/4$, we have 
%         \begin{align}
%             \error_{\wt \calS_C} (\wh f)\le - \error_{\calS}( \wh f) + \left(1 + \frac{m}{2n} \right) \sqrt{\frac{\log(4/\delta)}{m}} \,. 
%         \end{align} 
%     \end{enumerate}

%     Using union bound on the above three steps, we have with probability at least $1-\delta$: 
%     \begin{align}
%         \error_\calD (\wh f) \le \error_{\calS}(\wh f)   + 1 - 2\error_{\wt \calS}(\wh f)   + (1/\sqrt{2} + 2.5)  \sqrt{\frac{\log(4/\delta)}{m}} \,.
%     \end{align}
%     Note that $(1/\sqrt{2} + 2.5)$ is a loose constant. In experiments, we use the ratio $\frac{m}{n}$
%     %  the exact error $\error_{\wt \calS}(\wh f)$ 
%     to evaluate R.H.S.    
% \end{proof}

% \subsection{Proof of \propref{prop:rademacher}}

% \begin{proof}[Proof of \propref{prop:rademacher}]
%     For a classifier $ f: \calX \to \{-1, 1\}$, we have $1 - 2\,\indict{ f(x) \ne y} = y \cdot f(x)$. Hence, by definition of $\error$, we have 
%     \begin{align}
%         1 -2\error_{\wt \calS}(f) = \frac{1}{m}\sum_{i=1}^m y_i \cdot f(x_i) \le \sup_{f \in \calF} \, \frac{1}{m} \sum_{i=1}^m y_i \cdot f(x_i)  \,. \label{eq:error_rademacher}
%     \end{align}
%     Note that for fixed inputs $(x_1, x_2, \ldots, x_m)$ in $\wt S$, $(y_1, y_2, \ldots y_m)$ are random labels. Define $\phi_1 (y_1, y_2, \ldots, y_m) \defeq \sup_{f \in \calF} \, \frac{1}{m} \sum_{i=1}^m y_i \cdot f(x_i)$. We have the following bounded difference condition on $\phi_1$. For all i, 
%     \begin{align}
%         \sup_{y_1, \ldots y_m, y_i^\prime \in \{-1, 1\}^{m+1} } \abs{ \phi_1 (y_1,\ldots, y_i, \ldots, y_m) - \phi_1 (y_1,\ldots, y_i^\prime, \ldots, y_m)  } \le 1/m \,. \label{cond1_rademacher}
%     \end{align} 
    
%     Similarly define $\phi_2 (x_1, x_2, \ldots, x_m) \defeq \Expt{ y_i \sim_U \{-1, 1\}  }{ \sup_{f \in \calF} \, \frac{1}{m}  \sum_{i=1}^m y_i \cdot f(x_i)}$. We have the following bounded difference condition on $\phi_2$. For all i,
%     \begin{align}
%         \sup_{x_1, \ldots x_m, x_i^\prime \in \calX^{m+1} } \abs{ \phi_2 (x_1,\ldots, x_i, \ldots, x_m) - \phi_1 (x_1,\ldots, x_i^\prime, \ldots, x_m)  } \le 1/m \,. \label{cond2_rademacher}
%     \end{align}
%     Using McDiarmid’s inequality (\lemref{lem:McDiarmid}) twice with Condition \eqref{cond1_rademacher} and \eqref{cond2_rademacher}, with probability at least $1-\delta$, we have
%     \begin{align}
%         \sup_{f \in \calF} \, \frac{1}{m} \sum_{i=1}^m y_i \cdot f(x_i)  - \Expt{x,y}{\sup_{f \in \calF} \, \frac{1}{m} \sum_{i=1}^m y_i \cdot f(x_i) } \le \sqrt{\frac{2\log(2/\delta)}{m}} \label{eq:final_rademacher}
%     \end{align} 
%     Combining \eqref{eq:error_rademacher} and \eqref{eq:final_rademacher}, we obtain the desired result. 
% \end{proof}


% \subsection{Proof of \thmref{thm:error_regularized_ERM}}

% Proof of \thmref{thm:error_regularized_ERM} follows similar to the proof of \thmref{thm:error_ERM}. Note that the same results in \lemref{lem:fit_mislabeled}, \lemref{lem:mislabeled_error}, and \lemref{lem:clear_error} hold in the regularized ERM case. However, the arguments in the proof of \lemref{lem:fit_mislabeled} changes slightly. Hence, we state and prove a lemma parallel to \lemref{lem:fit_mislabeled} for completeness. 

% \begin{lemma} \label{lem:lemma1_reg}
%     Assume the same setup as \thmref{thm:error_regularized_ERM}. 
%     Then for any $\delta >0$, with probability at least  $1-\delta$ 
%     over the random draws of mislabeled data $\wt S_M$, we have 
%     \begin{align}
%         \error_\calD(\widehat f)  \le 1 -\error_{\wt \calS_M}(\widehat f) + \sqrt{\frac{\log(1/\delta)}{m}}\,. 
%     \end{align} 
% \end{lemma}
% \begin{proof}
%     The main idea of the proof remains the same, i.e. regard 
%     the clean portion of the data 
%     ($S \cup \wt S_C$) as fixed.   
%     Then, there exists a classifier $f^*$ 
%     that is optimal over draws 
%     of the mislabeled data $\wt S_M$. 

    
%     Formally, 
%     \begin{align}
%     f^* \defeq \argmin_{f \in \calF} \error_{\widecheck {\calD}} (f)  + \lambda R(f) \,, \label{eq:modified_ERM_reg}
%     \end{align}
%     where $$\widecheck \calD = \frac{n}{m+n} \calS + \frac{m_1}{m+n} \wt \calS_C  + \frac{m_2}{m+n}\calDm \,.$$ That is, $\widecheck \calD$ a combination of 
%     the \emph{empirical distribution} 
%     over correctly labeled data $S \cup \wt S_C$
%     % in $S\cup \wt S$ 
%     and the (population) distribution 
%     over mislabeled data $\calDm$.
%     Recall that 
%     \begin{align}
%     \wh f \defeq \argmin_{f \in \calF} \error_{\calS \cup \wt S} (f) + \lambda R(f) \,. \label{eq:orig_ERM_reg}
%     \end{align}
%     % 
%     % 
%     Since, $\widehat f$ minimizes 0-1 error 
%     on $S \cup \wt S$, using ERM optimality on \eqref{eq:orig_ERM},  
%     we have 
%     \begin{align}
%         \error_{\calS \cup \wt \calS}(\widehat f) + \lambda R(\wh f) \le \error_{
%             \calS \cup \wt \calS}(f^*) + \lambda R(f^*) \,.    \label{eq:step1_reg}
%     \end{align}
%     Moreover, since $f^*$ is independent of $\wt S_M$, using Hoeffding's bound,
%     % \footnote{For a fully rigorous argument,
%     % refer to the complete proof in App.~\ref{app:proof_erm}.} 
%     we have with probability at least $1-\delta$ that
%     \begin{align}
%       \error_{\wt \calS_M}(f^*) \le \error_{ \calDm}(f^*) +  \sqrt{\frac{\log(1/\delta)}{2 m_1}} \,. \label{eq:step2_reg} 
%     \end{align}
%     %$ 
%     %for some constant $c_1\le 1/2$. 
%     Finally, since $f^*$ is the optimal classifier on $\widecheck \calD$, 
%     we have 
%     \begin{align}
%         \error_{\widecheck \calD}(f^*) + \lambda R(f^*) \le \error_{\widecheck \calD}(\widehat f) + \lambda R(\wh f) \label{eq:step3_reg}
%     \end{align}
%      Now to relate \eqref{eq:step1_reg} and \eqref{eq:step3_reg}, we can re-write the \eqref{eq:step2_reg} as follows: 
%     \begin{align}
%         \error_{\calS \cup \wt\calS}(f^*) \le \error_{ \widecheck \calD}(f^*) +  \frac{m_1}{m+n}\sqrt{\frac{\log(1/\delta)}{2 m_1}} \,. \label{eq:step4_reg} 
%     \end{align}
%     After adding $\lambda R(f^*)$ on both sides in \eqref{eq:step4_reg}, we combine equations \eqref{eq:step1_reg}, \eqref{eq:step4_reg}, and \eqref{eq:step3_reg}, to get 
%     \begin{align}
%         \error_{\calS \cup \wt \calS}(\wh f) \le \error_{\widecheck \calD}(\wh f) +  \frac{m_1}{m+n}\sqrt{\frac{\log(1/\delta)}{2 m_1}} \,, 
%     \end{align}
%     which implies 
%     \begin{align}
%         \error_{ \wt \calS_M}(\wh f) \le \error_{\calDm}(\wh f) + \sqrt{\frac{\log(1/\delta)}{2 m_1}} \,. \label{eq:lemma_reg_final}
%     \end{align}
%     Similar as before, since $\wt S$ is obtained by randomly labeling an unlabeled dataset, we assume 
%     $2m_1 \approx m$. Moreover, using $\error_{\calDm} = 1 - \error_{\calD}$ we obtain the desired result. 
% \end{proof}
% % \begin{proof}[Proof of ]
    
% % \end{proof}

% \subsection{Proof of \thmref{thm:multiclass_ERM}}

% We first state and prove lemmas parallel to three lemmas used in the proof of balanced binary case. Then we combine the results in the three lemmas to obtain the result in \thmref{thm:multiclass_ERM}. 

% Before stating the result, we define mislabeled distribution $\calDm$ for any $\calD$. While $\calDm$ and $\calD$ share 
% the same marginal distribution over $\calX$, 
% the distribution over labels $y$ 
% given an input $x\sim \calD_\calX$ is changed.
% In particular, for any $x$, the pdf over $y$ is changed to:  
% $p_{\calDm} (\cdot \vert x) \defeq \frac{1 - p_{\calD}(\cdot \vert x)}{k - 1}$.

% \begin{lemma} \label{lem:fit_mislabeled_multi}
%     Assume the same setup as \thmref{thm:multiclass_ERM}. 
%     Then for any $\delta >0$, with probability at least  $1-\delta$ 
%     over the random draws of mislabeled data $\wt S_M$, we have 
%     \begin{align}
%         \error_\calD(\widehat f)  \le (k-1)\left(1 -\error_{\wt \calS_M}(\widehat f)\right) + (k-1)\sqrt{\frac{\log(1/\delta)}{m}}\,. \label{eq:lemma1_multi}
%     \end{align}   
% \end{lemma} 

% \begin{proof}
%     The main idea of the proof remains the same, i.e. regard 
%     the clean portion of the data 
%     ($S \cup \wt S_C$) as fixed. 
%     Then, there exists a classifier $f^*$ 
%     that is optimal over draws 
%     of the mislabeled data $\wt S_M$. 
    
%     However, we need to be careful while relating population error on mislabeled data with population accuracy on clean data.   
%     While for binary classification,  we could upper bound $\error_{\wt \calS_M}$ 
%     with $1-\error_\calD$  (in the proof of \lemref{lem:fit_mislabeled}), 
%     for multiclass classification, 
%     error on the mislabeled data 
%     and accuracy on the clean data 
%     in the population 
%     are not so directly related.  
%     To establish \eqref{eq:lemma1_multi},
%     we break the error on the 
%     (unknown) mislabeled data 
%     into two parts: one term corresponds 
%     to predicting the true label on mislabeled data, 
%     and the other corresponds to predicting 
%     neither the true label 
%     nor the assigned (mis-)label.  
%     Finally, we relate these errors to their
%     population counterparts to establish \eqref{eq:lemma1_multi}. 
    
%     Formally, 
%     \begin{align}
%     f^* \defeq \argmin_{f \in \calF} \error_{\widecheck {\calD}} (f)  + \lambda R(f) \,, \label{eq:modified_ERM_reg2}
%     \end{align}
%     where $$\widecheck \calD = \frac{n}{m+n} \calS + \frac{m_1}{m+n} \wt \calS_C  + \frac{m_2}{m+n}\calDm \,.$$ That is, $\widecheck \calD$ a combination of 
%     the \emph{empirical distribution} 
%     over correctly labeled data $S \cup \wt S_C$
%     % in $S\cup \wt S$ 
%     and the (population) distribution 
%     over mislabeled data $\calDm$.
%     Recall that 
%     \begin{align}
%     \wh f \defeq \argmin_{f \in \calF} \error_{\calS \cup \wt S} (f) + \lambda R(f) \,. \label{eq:orig_ERM_reg2}
%     \end{align}
%     % 
%     % 
%     Following the exact steps from the proof of \lemref{lem:lemma1_reg}, with probability at least $1-\delta$, we have  
%     \begin{align}
%         \error_{ \wt \calS_M}(\wh f) \le \error_{\calDm}(\wh f) + \sqrt{\frac{\log(1/\delta)}{2 m_1}} \,. \label{eq:lemma1_final_multi_prev}
%     \end{align}
%     Similar to before, since $\wt S$ is obtained by randomly labeling an unlabeled dataset, we assume 
%     $\frac{k}{k-1} m_1 \approx m$. 
    
%     Now we will relate $\error_\calDm (\wh f)$ with $\error_{\calD}(\wh f)$. Let $y^T$ denote the (unknown) true label for a mislabeled point $(x, y)$ (i.e., label before replacing it with a mislabel). 
%     \begin{align}    
%          \Expt{(x, y) \in \sim \calDm}{\indict{ \wh f(x) \ne y }}  &= \underbrace{\Expt{(x, y) \in \sim \calDm}{\indict{ \wh f(x) \ne y \land \wh f(x) \ne y^T}}}_{\RN{1}} + \underbrace{\Expt{(x, y) \in \sim \calDm}{\indict{ \wh f(x) \ne y \land \wh f(x) = y^T}}}_{\RN{2}} \,. \label{eq:excess_term}
%     \end{align}
%     Clearly, term 2 is one minus the accuracy on the clean unseen data, i.e. 
%     \begin{align}
%         \RN{2} = 1 - \Expt{{x,y} \sim \calD}{ \indict{ \wh f(x) \ne y}} = 1- \error_{\calD}(\wh f) \,. \label{eq:term1}    
%     \end{align}
%     Next, we  relate term 1 with the error on the unseen clean data. We show that term 1 is equal to the error on the unseen clean data scaled by $\frac{k-2}{k-1}$ where $k$ is the number of labels. Using the definition of mislabeled distribution $\calDm$,  we have 
%     \begin{align}
%         \RN{1} = \frac{1}{k-1} \left( \Expt{(x, y) \in \sim \calD}{ \sum_{i \in \calY \land i\ne y}  \indict{ \wh f(x) \ne i \land \wh f(x) \ne y}} \right) = \frac{k-2}{k-1} \error_{\calD}(\wh f) \,.\label{eq:term2}
%     \end{align}    

%     Combining the result in \eqref{eq:term1}, \eqref{eq:term2} and \eqref{eq:excess_term}, we have 
%     \begin{align}
%         \error_{\calDm}(\wh f) = 1- \frac{1}{k-1} \error_{\calD}(\wh f) \,.\label{eq:combine_terms}
%     \end{align}
%     Finally, combining the result in \eqref{eq:combine_terms} with equation \eqref{eq:lemma1_final_multi_prev}, we have with probability $1-\delta$, 
%     \begin{align}
%       \error_{\calD}(\wh f) \le  (k-1) \left( 1- \error_{ \wt \calS_M}(\wh f) \right)  + (k-1) \sqrt{\frac{k \log(1/\delta)}{ 2(k-1)m}} \,. \label{eq:lemma1_final_multi}
%     \end{align}
% \end{proof}

% \begin{lemma} \label{lem:mislabeled_error_multi}
%     Assume the same setup as \thmref{thm:multiclass_ERM}.  Then for any $\delta >0$, with probability at least $1-\delta$ over the random draws of $\wt S$, we have  
%     % \begin{align}
%         $$\abs{k\error_{\wt \calS}(\widehat f) - \error_{\wt \calS_C}(\widehat f) -  (k-1)\error_{\wt \calS_M}(\widehat f) } \le  2k\sqrt{\frac{\log(4/\delta)}{2m}}\,. $$ % \label{eq:lemma2}
%     % \end{align}   
%     %  for some constant $c_3 \le 1.0\,$.
% \end{lemma} 


% \begin{proof}
%     Recall $\error_{\wt S} (f) = \frac{m_1}{m} \error_{\wt S_M}(f) + \frac{m_2}{m} \error_{\wt S_C}(f)$. Hence, we have 
%     \begin{align}
%         k\error_{\wt S}(f) - (k-1)\error_{\wt S_M}(f) - \error_{\wt S_C}(f) &= (k-1)\left(\frac{k m_1}{(k-1) m} \error_{\wt S_M}(f) - \error_{\wt S_M}(f)\right) + \left(\frac{km_2}{m} \error_{\wt S_C}(f) - \error_{\wt S_C}(f)\right) \\ &= k \left[ \left(\frac{m_1}{m} - \frac{k-1}{k}\right) \error_{\wt S_M}(f) + \left(\frac{m_2}{m} - \frac{1}{k} \right) \error_{\wt S_C} (f) \right] \,.
%     \end{align} 
%     Since the dataset is randomly labeled, we have with probability at least $1-\delta$, $\left(\frac{m_1}{m} - \frac{k-1}{k}\right) \le \sqrt{\frac{\log(1/\delta)}{2m}}$. Similarly, we have with probability at least $1-\delta$, $\left(\frac{m_2}{m} - \frac{1}{k}\right) \le \sqrt{\frac{\log(1/\delta)}{2m}}$. Using union bound, we have with probability at least $1-\delta$
%     % \begin{align}
%     %     2\error_{\wt S} - \error_{\wt S_M}(f) - \error_{\wt S_C}(f) \le \sqrt{\frac{\log(2/\delta)}{2m}} \left(\error_{\wt S_M}(f) + \error_{\wt S_C}(f) \right) \le 2\sqrt{\frac{\log(2/\delta)}{2m}} \,. \label{eq:lemma2_final}
%     % \end{align}
%     \begin{align}
%         k\error_{\wt S}(f) - (k-1)\error_{\wt S_M}(f) - \error_{\wt S_C}(f)  \le k \sqrt{\frac{\log(2/\delta)}{2m}} \left(\error_{\wt S_M}(f) + \error_{\wt S_C}(f) \right) \,. \label{eq:lemma2_final_multi}
%     \end{align}

%     % We obtain the desired result by using 
% \end{proof}

% \begin{lemma} \label{lem:clear_error_multi}
%     Assume the same setup as \thmref{thm:multiclass_ERM}. 
%     Then for any $\delta >0$, with probability at least $1-\delta$ 
%     over the random draws of $\wt S_C$ and $S$, we have 
%     % \begin{align}
%         $$\abs{\error_{\wt \calS_C}(\widehat f) - \error_{\calS}(\widehat f) } \le 1.5 \sqrt{\frac{k\log(2/\delta)}{2m}}\,.$$ %\label{eq:lemma3}
%     % \end{align}   
%     % for some constant $c_2 \le 1.2\,$.
% \end{lemma} 
% \begin{proof}
%     % Recall 0-1 error on each point  $(x,y) \in S \cup \wt S$ is given by $\I{ f(x)\ne y}$.
%     In the set of correctly labeled points $S \cup \wt S_C$, we have $S$ as a random subset of $S \cup \wt S_C$. Hence, using Hoeffding's inequality for sampling without replacement (\lemref{lem:hoeffding_sampling}), we have with probability at least $1-\delta$
%     \begin{align}
%         \error_{\wt \calS_c} (\wh f)- \error_{\calS \cup \wt \calS_C}( \wh f) \le  \sqrt{\frac{\log(1/\delta)}{2m_2}} \,.
%     \end{align}
%     Re-writing $\error_{\calS \cup \wt \calS_C}( \wh f)$ as $\frac{m_2}{m_2 + n} \error_{\wt \calS_C }(\wh f) + \frac{n}{m_2 + n} \error_{\calS }(\wh f)$, we have with probability at least $1-\delta$
%     \begin{align}
%       \left(\frac{n}{n+m_2}\right) \left(\error_{\wt \calS_c} (\wh f)- \error_{\calS}( \wh f) \right) \le  \sqrt{\frac{\log(1/\delta)}{2m_2}} \,.
%     \end{align}
%     As before, assuming $km_2 \approx m$, we have with probability at least $1-\delta$ 
%     \begin{align}
%         \error_{\wt \calS_c} (\wh f)- \error_{\calS}( \wh f) \le \left(1+\frac{m_2}{n}\right)  \sqrt{\frac{k\log(1/\delta)}{2m}} \le \left( 1 + \frac{1}{k}\right) \sqrt{\frac{k\log(1/\delta)}{2m}} \,. \label{eq:lemma3_final_multi}
%     \end{align} 
% \end{proof}

% \begin{proof}[Proof of \thmref{thm:multiclass_ERM}] 
%     Having established these core intermediate results, we can now combine above three lemmas. 
%     In particular, we bound the population error on clean data ($\error_\calD(\wh f)$) as follows:  
%     \begin{enumerate}[(i)]
%         \item First, use \eqref{eq:lemma1_final_multi}, to obtain an upper bound on the population error on clean data, i.e., with probability at least $1-\delta/4$, we have
%         \begin{align}
%             \error_{ \calD} (\wh f) \le (k-1)\left(1 - \error_{ \wt \calS_M}(\wh f) \right) + (k-1) \sqrt{\frac{k\log(4/\delta)}{2(k-1)m}} \,. 
%         \end{align}
%         \item  Second, use \eqref{eq:lemma2_final_multi}, to relate the error on the mislabeled fraction with error on clean portion of randomly labeled data and error on whole randomly labeled dataset, i.e., with probability at least $1-\delta/2$, we have 
%         \begin{align}
%             - (k-1)\error_{\wt S_M}(f) \le \error_{\wt S_C}(f) - k\error_{\wt S}  + k\sqrt{\frac{\log(4/\delta)}{2m}}  \,. 
%         \end{align} 
%         \item Finally, use \eqref{eq:lemma3_final_multi} to relate the error on the clean portion of randomly labeled data and error on clean training data, i.e., with probability $1-\delta/4$, we have 
%         \begin{align}
%             \error_{\wt \calS_C} (\wh f)\le - \error_{\calS}( \wh f) + \left(1 + \frac{m}{kn} \right) \sqrt{\frac{k\log(4/\delta)}{2m}} \,. 
%         \end{align} 
%     \end{enumerate}

%     Using union bound on the above three steps, we have with probability at least $1-\delta$: 
%     \begin{align}
%         \error_\calD (\wh f) \le \error_{\calS}(\wh f) + (k-1) - k\error_{\wt \calS}(\wh f)   + (\sqrt{k(k-1)} + k + \sqrt{k} + \frac{m}{n\sqrt{k}})  \sqrt{\frac{\log(4/\delta)}{2m}} \,.
%     \end{align}
%     % Note that $\frac{m}{n\sqrt{k}}$ is much smaller than the other terms in addition. Hence, we ignore this in the final bound. 
%     % Note that $(1/\sqrt{2} + 2.5)$ is a loose constant. In experiments, we use the ratio $\frac{m}{n}$
%     %  the exact error $\error_{\wt \calS}(\wh f)$ 
%     % to evaluate R.H.S.    
% \end{proof}

% \newpage
% \section{Proofs from \secref{sec:linear_models}}\label{app:proof_gd}

% We suppose that the parameters of the linear function 
% are obtained via gradient descent on 
% the following $L_2$ regularized problem: 
% \begin{align}
%     % n in denominator is avoided deliberately
%     \calL_S(w; \lambda) \defeq \sum_{i=1}^n{(w^Tx_i - y_i)^2} + \lambda \norm{w}{2}^2 \,, \label{eq:l2_MSE_app}   
% \end{align}
% where $\lambda\ge0$ is a regularization parameter. 
% We assume access to a clean dataset 
% $S = \{(x_i, y_i)\}_{i=1}^n \sim \calD^n$ 
% and randomly labeled dataset 
% $\wt S = \{(x_i, y_i)\}_{i=n+1}^{n+m} \sim \wt \calD^m$. 
% Let $\bX = [x_1, x_2, \cdots, x_{m+n}]$ 
% and $\by = [y_1, y_2, \cdots, y_{m+n}]$. 
% Fix a positive learning rate $\eta$ such that 
% $\eta \le 1/\left(\norm{\bX^T\bX}{\text{op}} + \lambda^2\right)$ 
% and an initialization $w_0 = 0$. 
% % \todos{Assumption made for simplicty}. 
% Consider the following gradient descent iterates 
% to minimize objective \eqref{eq:l2_MSE_app} on $S \cup \wt S$:
% \begin{align}
% w_t = w_{t-1} - \eta \grad_w \calL_{S \cup \wt S} (w_{t-1}; \lambda) \quad \forall t=1,2,\ldots \label{eq:GD_iterates_app}
% \end{align} 
% Then we have $\{ w_t\}$ converge to the limiting solution 
% $\wh w = \left( \bX^T\bX+\lambda \boldsymbol{I}\right)^{-1}\bX^T\by$. Define $\widehat f (x) \defeq f(x ; \wh w) $.  

% \subsection{\textcolor{red}{Errata}}

% We wish to correct the following error in the body: \codref{cond:error_stability} is not enough to guarantee the result in \thmref{thm:linear}. We now present a slightly stronger condition called \emph{hypothesis stability} under which we obtain a result similar to \thmref{thm:linear}. 

% This error doesn't change the main arguments of the proof where we show that the empirical train error is less than or equal to the leave-one-out error. We need a stronger condition to relate leave-one-out error with the population error of the original classifier. Specifically, while \codref{cond:error_stability} relates the average population error of leave-one-out classifiers with the population error of the original classifier, we need the new condition to show the concentration of the empirical leave-one-out error and  average population error of leave-one-out classifiers. 
% % main takeaway 

% Note that the new condition, while being stronger than the previous one, still doesn't imply generalization~\cite{bousquet2002stability,elisseeff2003leave,abou2019exponential}. Overall, the main results in \secref{sec:ERM_training} and takeaways of the paper remain unaffected by the error.  

% We now present the new condition and a corrected statement of \thmref{thm:linear}. Recall, for a given training set $S \sim \calD^n $, 
% we use $S_{(i)}$ to denote the training set $S$ 
% with the $i^{\text{th}}$ point removed.

% \begin{condition}[Hypothesis Stability] 
%     \label{cond:hypothesis_stability}
%     We have $\beta$ hypothesis stability 
%     if our training algorithm $\calA$ satisfies the following: 
%     \begin{align*}
%     % ${\sum_{i=1}^n \frac{\error_{\calD}( f(\calA, S_{(i)}))}{n} - \error_\calD(f(\calA, S))} \le \beta\,$.
%     \forall i \in \{1,2,\ldots, n\}, \quad  \Expt{\calS, (x,y) \in \calD}{ \abs{\error\left( f(x) ,y  \right) - \error\left( f_{(i)}(x), y \right) }} \le \frac{\beta}{n} \,,
%     \end{align*}
%     where $f_{(i)} \defeq f(\calA, S_{(i)})$ and $ f \defeq f(\calA, S)$.
% \end{condition}

% \begin{theorem}[Correct statement of \thmref{thm:linear}] \label{thm:new_linear}
%     Assume that this gradient descent algorithm satisfies \codref{cond:hypothesis_stability}
%     with $\beta=\calO(1)$.  
%     Then for any $\delta >0$, with probability at least $1-\delta$ 
%     over the random draws of datasets $\wt S$ and $S$, we have:
%     \begin{align}
%         \error_\calD(\widehat f) \le \error_\calS(\widehat f) + 1 - 2 \error_{\wt\calS}(\widehat f) + \left(\frac{1}{\sqrt{2}} + 1.5 \right) \sqrt{\frac{\log(4/\delta)}{m}} + \sqrt{\frac{4}{\delta}\left(\frac{1}{m} +\frac{3\beta}{m+n} \right)}  \,. \label{eq:gd_error}
%     \end{align} 
%     % for some constant $c\le 3.2$.
% \end{theorem}

% \subsection{Proof of \thmref{thm:new_linear}}
% We use a standard result from linear algebra, namely Shermann-Morrison formula~\citep{sherman1950adjustment} for matrix inversion:  

% \begin{lemma}[\citet{sherman1950adjustment}] \label{lem:sherman}
%     Suppose $\bA \in \Real^{n \times n}$ is an invertible square matrix and $u,v \in \Real^n$ are column vectors. Then $\bA + uv^T$ is invertible iff $1 + v^T \bA u \ne 0$ and in particular
%     \begin{align}
%         (\bA + u v^T)^{-1} = \bA^{-1}  - \frac{\bA^{-1} uv^T \bA^{-1} }{ 1 + v^T \bA^{-1} u} \,.
%     \end{align}   
% \end{lemma}
% \newcommand\byy[1]{\by_{\left(#1\right)}}
% \newcommand\bXX[1]{\bX_{\left(#1\right)}}
% \newcommand\ff[1]{\wh f_{\left(#1\right)}}

% For a given training set $S \cup \wt S_C$, define leave-one-out error on mislabeled points in the training data as $$\error_{\text{LOO}(\wt S_M) } = \frac{\sum_{(x_i, y_i) \in \wt S_M} \error( f_{(i)}( x_i), y_i)}{ \abs{\wt S_M }} \,, $$
% where $f_{(i)} \defeq f(\calA, (S \cup \wt S)_{(i)})$. To relate empirical leave-one-out error and population error with hypothesis stability condition, we use the following lemma:   

% \begin{lemma}[\citet{bousquet2002stability}] \label{lem:stability_error}
%     For the leave-one-out error, we have
%     \begin{align}
%         \Expo{ \left( \error_{\calDm}(\wh f) -\error_{\text{LOO}(\wt S_M) } \right)^2 } \le \frac{1}{2m_1}+  \frac{3\beta}{n + m}\,.
%     \end{align}   
%     % where $ f \defeq f(\calA, S \cup \wt S) $.
% \end{lemma}

% Proof of the above lemma is similar to the proof of  Lemma 9 in \citet{bousquet2002stability} and can be found in \appref{app:proof_lem_error}. 
% % 
% % Before presenting the result, we introduce some notation. 
% Before presenting the proof of \thmref{thm:new_linear}, we introduce some more notation. Let $\bX_{(i)}$ denote the matrix of covariates with $i^{\text{th}}$ point removed. Similarly let $\by_{(i)}$ be the array of responses with $i^{\text{th}}$ point removed. Define the corresponding regularized GD solution as $\wh w_{(i)} = \left( \bXX{i}^T\bXX{i}+\lambda \boldsymbol{I}\right)^{-1}\bXX{i}^T\byy{i}$. Define $\ff{i}(x) \defeq f(x ; \wh w_{(i)}) $.

% \begin{proof}[Proof of \thmref{thm:new_linear}]
%     Because squared loss minimization does not imply 0-1 error minimization, we cannot use arguments from \lemref{lem:fit_mislabeled}. This is the main technical difficulty. To compare the 0-1 error at a train point with an unseen point, 
%     we use the closed-form expression for $\widehat{w}$ and Shermann-Morrison formula to upper bound training error with leave-one-out cross validation error. 
    
%     The proof is divided into three parts: In part one, we show that 0-1 error on mislabeled points in the training set is lower than the error obtained by leave-one-out error at those points. In part two, we relate this leave-one-out error with the population error on mislabeled distribution using \codref{cond:hypothesis_stability}. While the empirical leave-one-out error is unbiased estimator of the average population error of leave-one-out classifiers, we need hypothesis stability to control the variance of empirical leave-one-out error. Finally in part three, we show that the error on the mislabeled training points can be estimated with just the randomly labeled and  clean training data (as in proof of \thmref{thm:error_ERM}).  

%     \textbf{Part 1 {} {}} First we relate training error with leave-one-out error.        
%     For any 
%     training point $(x_i, y_i)$ in $\wt S \cup S$, we have 
%     \begin{align}
%         \error(\wh f(x_i), y_i ) &= \indict{ y_i \cdot x_i^T \wh w < 0 } = \indict{ y_i \cdot x_i^T \left( \bX^T\bX+\lambda \boldsymbol{I}\right)^{-1}\bX^T\by < 0 } \\
%         &= \indict{ y_i \cdot x_i^T \underbrace{\left( \bXX{i}^T\bXX{i} + x_i ^T x_i +\lambda \boldsymbol{I}\right)^{-1}}_{\RN{1}} (\bXX{i}^T\byy{i} + y \cdot x_i) < 0 }
%     \end{align}
%     Letting $\bA = \left(\bXX{i}^T\bXX{i} +\lambda \boldsymbol{I}\right)$ and using \lemref{lem:sherman} on term 1, we have 
%     \begin{align}
%         \error(\wh f(x_i), y_i ) &= \indict{ y_i \cdot x_i^T \left[\bA^{-1} -  \frac{\bA^{-1} x_i x_i^T \bA^{-1}}{ 1 + x_i ^T \bA^{-1} x_i } \right] (\bXX{i}^T\byy{i} + y \cdot x_i) < 0 } \\
%         &= \indict{ y_i \cdot\left[ \frac{ x_i^T \bA^{-1} ( 1 + x_i ^T \bA^{-1} x_i ) -  x_i^T \bA^{-1} x_i x_i^T \bA^{-1}}{ 1 + x_i ^T \bA ^{-1}x_i } \right] (\bXX{i}^T\byy{i} + y \cdot x_i) < 0 } \\
%         &= \indict{ y_i \cdot\left[ \frac{ x_i^T \bA^{-1}}{ 1 + x_i ^T \bA ^{-1}x_i } \right] (\bXX{i}^T\byy{i} + y \cdot x_i) < 0 } \,.
%     \end{align}

%     Since $1 + x_i^T \bA^{-1} x_i > 0$, we have 
%     \begin{align}
%         \error(\wh f(x_i), y_i ) &= \indict{ y_i \cdot x_i^T \bA^{-1} (\bXX{i}^T\byy{i} + y \cdot x_i) < 0 } \\
%         &= \indict{ x_i^T \bA^{-1} x_i +  y_i \cdot x_i^T \bA^{-1} (\bXX{i}^T\byy{i}) < 0 } \\
%         &\le \indict{ y_i \cdot x_i^T \bA^{-1} (\bXX{i}^T\byy{i}) < 0 } = \error(\ff{i}(x_i), y_i ) \,.\label{eq:LOO_error}
%     \end{align}

%     Using \eqref{eq:LOO_error}, we have 
%     \begin{align}
%         \error_{\wt \calS_M } (\wh f) \le \error_{\text{LOO} (S_M)} \defeq \frac{\sum_{(x_i, y_i) \in \wt S_M} \error(\ff{i}(x_i), y_i ) }{\abs{\wt \calS_M}}\label{eq:LOO_error_final}
%     \end{align}
%     \textbf{Part 2 {}{}} We now relate RHS in \eqref{eq:LOO_error_final} with the population error on mislabeled distribution. To do this, we leverage \codref{cond:hypothesis_stability} and \lemref{lem:stability_error}. In particular, we have 

%     \begin{align}
%         \Expt{\calS \cup \wt \calS_M }{ \left(\error_{\calDm}(\wh f) - \error_{\text{LOO} (S_M)}\right)^2 } \le \frac{1}{2m_1} + \frac{3\beta}{m+n} \,.
%     \end{align}

%     Using Chebyshev's inequality, with probability at least $1-\delta$, we have 
%     \begin{align}
%         \error_{\text{LOO} (S_M)} \le  \error_{\calDm}(\wh f)   + \sqrt{\frac{1}{\delta}\left(\frac{1}{2m_1} +\frac{3\beta}{m+n} \right)} \,. \label{eq:final_mislabeled_linear}
%     \end{align}
    

%     \textbf{Part 3 {}{}} Combining \eqref{eq:final_mislabeled_linear} and \eqref{eq:LOO_error_final}, we have 

%     \begin{align}
%         \error_{\wt \calS_M } (\wh f) \le \error_{\calDm}(\wh f)   + \sqrt{\frac{1}{\delta}\left(\frac{1}{2m_1} +\frac{3\beta}{m+n} \right)} \,. \label{eq:linear_parallel_lem1}
%     \end{align}

%     Compare \eqref{eq:linear_parallel_lem1}, with \eqref{eq:lemma1_final} in the proof of \lemref{lem:fit_mislabeled}. We obtain a similar relationship between $\error_{\wt \calS_M }$ and $\error_{\calDm}$ but with a polynomial concentration instead of exponential concentration. 
%     In addition, since we just use concentration arguments to relate mislabeled error with the error on clean portion and unlabeled portion, we can directly use the results in \lemref{lem:mislabeled_error} and \lemref{lem:clear_error}. Therefore, combining results in \lemref{lem:mislabeled_error}, \lemref{lem:clear_error}, and \eqref{eq:linear_parallel_lem1} with union bound, we have with probability at least $1-\delta$

%     \begin{align}
%         \error_\calD(\widehat f) \le \error_\calS(\widehat f) + 1 - 2 \error_{\wt\calS}(\widehat f) + \left(\frac{1}{\sqrt{2}} + 1.5 \right) \sqrt{\frac{\log(4/\delta)}{m}} + \sqrt{\frac{4}{\delta}\left(\frac{1}{m} +\frac{3\beta}{m+n} \right)}  \,.
%     \end{align}
    

       
% \end{proof}

% \subsection{Discussion on \codref{cond:hypothesis_stability}}

% The quantity in LHS of \codref{cond:hypothesis_stability} measures how much the function learned by the algorithm (in terms of error on unseen point) will change when one point in the training set is removed. 
% % Discussion on exponential concentration and stronger condition. 
% Notice that hypothesis stability implies error stability, i.e., \codref{cond:error_stability} ~\cite{bousquet2002stability}.  In summary, while error stability allowed us to relate the average population error of the leave-one-out classifiers with the population error of the original classifier, we need hypothesis stability condition to control the variance of the empirical leave-one-out error. 

% Additionally, we note that while the dominating term in the RHS of \thmref{thm:new_linear} matches with the dominating term in ERM bound in \thmref{thm:error_ERM}, there is a polynomial concentration term (dependence on $1/\delta$ instead of $\log(\sqrt{1/\delta})$) in  \thmref{thm:new_linear}. 
% Since with hypothesis stability, we just bound the variance,  the polynomial concentration is due to the use of Chebyshev's inequality instead of an exponential tail inequality (as in \lemref{lem:fit_mislabeled}).
% Recent works have highlighted that slightly stronger condition than hypothesis stability can be used to obtained an exponential concentration for leave-one-out error~\citep{abou2019exponential}, but we leave this for future work for now. 
% % We leave 
% % However, the constants 

% % we also want to highlight  

% \subsection{Formal statement and proof of  of \propref{prop:early_stop}}

% Before formally presenting the result, we will introduce some notation.  By $\calL_{S}(w)$, we denote 
% the objective in \eqref{eq:l2_MSE_app} with $\lambda=0$. 
% Assume Singular Value Decomposition (SVD) of $\bX$  as $\sqrt{n} \bU \bS^{1/2} \bV^T$. Hence $\bX^T \bX = \bV \bS \bV^T$.
% Consider the GD iterates defined in \eqref{eq:GD_iterates_app}. 
% % 
% We now derive closed form expression for the $t^\text{th}$ iterate of gradient descent:  
% % 
% \begin{align}
%     w_t = w_{t-1} + \eta \cdot \bX^T (\by - \bX w_{t-1}) = (\bI - \eta \bV \bS \bV^T )w_{k-1} + \eta \bX^T \by \,.
% \end{align}
% Rotating by $\bV^T$, we get 
% \begin{align}
%     \wt w_t = (\bI - \eta\bS )\wt w_{k-1} + \eta \wt \by \,, \label{eq:GD_recur}
% \end{align}
% where $\wt w_t = \bV^T w_t $ and $\wt \by = \bV^T \bX^T \by$. Assuming the initial point $w_0 = 0$ and applying the recursion in \eqref{eq:GD_recur}, we get
% \begin{align}
%     \wt w_t = \bS ^{-1} ( \bI - (\bI - \eta \bS)^k ) \wt \by \,, 
% \end{align} 
% Projecting solution back to the original space, we have 
% \begin{align}
%      w_t = \bV \bS ^{-1} ( \bI - (\bI - \eta \bS)^k ) \bV^T \bX^T \by \,, 
% \end{align} 
% % We will work with this GD solution at any iterate $t$ in the next proposition. 
% Define $f_t(x) \defeq f(x;w_t)$ as the solution at the $t^{\text{th}}$ iterate. 
% Let $\wt w_{\lambda} = \argmin_{w} \calL_\calS (w;\lambda) = (\bX^T \bX + \lambda \bI)^{-1} \bX^T \by = \bV (\bS + \lambda \bI )^{-1} \bV^T \bX^T \by $. 
% % ) \,,$ for all $t=1,2,\ldots\,.$ 
% and define $\wt f_\lambda(x) \defeq f(x;\wt w_\lambda)$ as the regularized solution. 
% Assume $\kappa$ be the condition number of the population covariance matrix 
% and 
% let $s_\text{min}$ be the minimum positive singular value of the empirical covariance matrix. Our proof idea is inspired from recent work on relating gradient flow solution and regularized solution for regression problems \citep{ali2018continuous}. We will use the following lemma in the proof: 
% \begin{lemma} \label{lem:ineq_soln}
%     For all $x \in [0,1]$ and for all $ k \in \mathbb{N}$, we have (a) $ \frac{kx}{1+kx} \le 1- (1-x)^k$ and (b) $ 1- (1-x)^k \le 2 \cdot \frac{kx}{kx+1} $.
%     %  where $g(c)$ is a constant dependent on $c$. For $c = 1$, $g(c) = 2.0$.   
% \end{lemma}
% \begin{proof}
%     % [Proof of \lemref{lem:ineq_soln}]
%     % Part (a) is easy. 
%     Using $ (1-x)^k \le \frac{1}{1+kx}$, we have part (a). For part (b), we numerically maximize $\frac{ (1+kx ) (1 - (1-x)^k) }{kx}$ for all $k\ge 1$ and for all $x \in [0, 1]$.  
% \end{proof}

% % 
% % Next, 

% \begin{prop}[Formal statement of \propref{prop:early_stop}] \label{prop:formal_early_stop}
% Let $\lambda = \frac{1}{t\eta}$. For a training point $x$, we have 
% \begin{align*}
%     \Expt{x \sim \calS}{(f_t(x) - \wt f_\lambda(x))^2} &\le c(t,\eta) \cdot \Expt{x \sim \calS}{f_t(x)^2} \,, %\label{eq:early_stop}
% \end{align*}
% where $c(t, \eta) \defeq \min( 0.25, \frac{1}{s_\text{min}^2 t^2 \eta^2})$. Similarly for a test point, we have 
% \begin{align*}
%     \Expt{x \sim \calD_\calX}{(f_t(x) - \wt f_\lambda(x))^2} &\le \kappa \cdot c(t,\eta) \cdot \Expt{x \sim \calD_\calX}{f_t(x)^2} \,. %\label{eq:early_stop}
% \end{align*}
% \end{prop} 

% \begin{proof}
%     %%%%%%%%%%%%% 
%     We want to analyze the expected squared difference output of regularized linear regression with regularization constant $\lambda = \frac{1}{\eta t}$ and gradient descent solution at $t^\text{th}$ iterate. We separately expand the algebraic expression for squared difference at a training point and a test point. 
%     % We start by considering the difference  
%     Then the main step is to show that  $\left[ \bS ^{-1} ( \bI - (\bI - \eta \bS)^k )  - (\bS + \lambda \bI )^{-1}\right] \preceq c(\eta, t) \cdot \bS ^{-1} ( \bI - (\bI - \eta \bS)^k ) $.

%     %%%%%%%%%%%%%
    
%   \textbf{Part 1 {} {}} 
%     First, we will analyze the squared difference of output at a training point (for simplicity, we refer to $S \cup \wt S$ as $S$), i.e. 
%     \begin{align}
%         \Expt{ x \sim \calS }{\left(f_t(x) - \wt f_\lambda (x)\right)^2} &= \norm{\bX w_t - \bX \wt w_\lambda}{2}^2 =   \norm{\bX \bV \bS ^{-1} ( \bI - (\bI - \eta \bS)^t ) \bV^T \bX^T \by - \bX \bV (\bS + \lambda \bI )^{-1} \bV^T \bX^T \by }{2}^2 \\
%         &= \norm{\bX \bV \left(\bS ^{-1} ( \bI - (\bI - \eta \bS)^t ) - (\bS + \lambda \bI )^{-1} \right) \bV^T \bX^T \by  }{2} \\
%         &=  \by^T \bV \bX \left( \underbrace{\bS ^{-1} ( \bI - (\bI - \eta \bS)^t ) - (\bS + \lambda \bI )^{-1}}_{\RN{1}} \right)^2 \bS \bV^T \bX^T \by \label{eq:train_GD_rel}
%         %  (\bX \bV \bS ^{-1} ( \bI - (\bI - \eta \bS)^k ) \bV^T \bX^T \by)^T \bX \bV \bS ^{-1} ( \bI - (\bI - \eta \bS)^k ) \bV^T \bX^T \by
%     \end{align}
%     We now separately consider term 1. Substituting $\lambda = \frac{1}{t \eta}$, we get
%     \begin{align}
%         \bS ^{-1} ( \bI - (\bI - \eta \bS)^t ) - (\bS + \lambda \bI )^{-1} &= \bS^{-1} \left( ( \bI - (\bI - \eta \bS)^t ) - (\bI + \bS^{-1} \lambda )^{-1}\right) \\
%         &= \underbrace{\bS^{-1} \left( ( \bI - (\bI - \eta \bS)^t ) - (\bI + ( \bS t \eta)^{-1}  )^{-1}\right)}_{\bA}
%     \end{align}

%     We now separately bound the diagonal entries in matrix $\bA$. 
%     With $s_i$, we denote $i^{\text{th}}$ diagonal entry of $\bS$. Note that since $ \eta\le 1/\norm{S}{\text{op}}$, for all $i$, $\eta s_i  \le 1$.  Consider $i^{\text{th}}$ diagonal term (which is non-zero) of the diagonal matrix $\bA$, we have 
%     \begin{align}
%         \bA_{ii} = \frac{1}{s_i} \left(  1 - (1 - s_i \eta)^t - \frac{t \eta s_i}{1 + t \eta s_i } \right) &=  \frac{1 - (1 - s_i \eta)^t}{s_i} \left( \underbrace{ 1 - \frac{t \eta s_i}{(1 + t \eta s_i)(1 - (1 - s_i \eta)^t)}}_{\RN{2}} \right) \\ 
%          &\le \frac{1}{2}\left[ \frac{1 - (1 - s_i \eta)^t}{ s_i} \right] \tag*{(Using \lemref{lem:ineq_soln} (b))} \,.
%     \end{align} 
%     Additionally, we can also show the following upper bound on term 2: 
%     \begin{align}
%          1 - \frac{t \eta s_i}{(1 + t \eta s_i)(1 - (1 - s_i \eta)^t)} &= \frac{(1 + t \eta s_i)(1 - (1 - s_i \eta)^t) - t \eta s_i }{(1 + t \eta s_i)(1 - (1 - s_i \eta)^t)} \\
%          & \le  \frac{ 1 -  (1 - s_i \eta)^t - t \eta s_i (1 - s_i \eta)^t}{(1 + t \eta s_i)(1 - (1 - s_i \eta)^t)} \\
%          & \le \frac{1}{t\eta s_i} \,. \tag{Using \lemref{lem:ineq_soln} (a)}
%         %  &\le \frac{1}{2}\left[ \frac{1 - (1 - s_i \eta)^t}{ s_i} \right] \tag*{(Using \lemref{lem:ineq_soln})} \,.
%     \end{align} 

%     Combining both the upper bounds on each diagonal entry $\bA_{ii}$, we have 
%     \begin{align}
%     \bA \preceq c_1(\eta, t) \cdot \bS^{-1} ( \bI - (\bI - \eta \bS)^t ) \,, \label{eq:upperbound_diagonal}
%     \end{align}
%     where $c_1(\eta, t ) = \min(0.5, \frac{1}{t s_i \eta })$. Plugging this into \eqref{eq:train_GD_rel}, we have 
%     \begin{align}
%         \Expt{ x \sim \calS }{\left(f_t(x) - \wt f_\lambda (x)\right)^2} &\le c(\eta, t) \cdot \by^T \bV \bX  \left( \bS^{-1} ( \bI - (\bI - \eta \bS)^t ) \right)^2 \bS \bV^T \bX^T \by \\
%         &=   c(\eta, t) \cdot \by^T \bV \bX  \left( \bS^{-1} ( \bI - (\bI - \eta \bS)^t ) \right) \bS \left( \bS^{-1} ( \bI - (\bI - \eta \bS)^t ) \right) \bV^T \bX^T \by \\
%         & =  c(\eta, t) \cdot \norm{\bX w_t}{2}^2 \\
%         &= c(\eta, t) \cdot  \Expt{ x \sim \calS }{\left(f_t(x) \right)^2} \,,
%     \end{align}
%     where $c(\eta, t ) = \min(0.25, \frac{1}{t^2 s^2_i \eta^2 })$.

%     \textbf{Part 2 {} {}} With $\bSigma$, we denote the underlying true covariance matrix. We now consider the squared difference of output at an unseen point: 
%     \begin{align}
%         \Expt{ x \sim \calD_{\calX} }{\left(f_t(x) - \wt f_\lambda (x)\right)^2} &= \Expt{x \sim \calD_{\calX}}{\norm{x^T w_t - x^T \wt w_\lambda}{2}} \\
%         &=   \norm{x^T \bV \bS ^{-1} ( \bI - (\bI - \eta \bS)^t ) \bV^T \bX^T \by - x^T \bV (\bS + \lambda \bI )^{-1} \bV^T \bX^T \by }{2} \\
%         &= \norm{x^T \bV \left(\bS ^{-1} ( \bI - (\bI - \eta \bS)^t ) - (\bS + \lambda \bI )^{-1} \right) \bV^T \bX^T \by  }{2} \\
%         &= \by^T \bV \bX \left( \bS ^{-1} ( \bI - (\bI - \eta \bS)^t ) - (\bS + \lambda \bI )^{-1} \right) \bV^T \bSigma \bV \\ &\qquad \qquad \qquad \qquad \qquad \left( (\bI - (\bI - \eta \bS)^t ) - (\bS + \lambda \bI )^{-1} \right) \bV^T \bX^T \by \\
%         &\le \sigma_{\text{max}} \cdot \by^T \bV \bX \left( \underbrace{\bS ^{-1} ( \bI - (\bI - \eta \bS)^t ) - (\bS + \lambda \bI )^{-1}}_{\RN{1}} \right)^2 \bV^T \bX^T \by \,, \label{eq:test_GD_rel}
%         %  (\bX \bV \bS ^{-1} ( \bI - (\bI - \eta \bS)^k ) \bV^T \bX^T \by)^T \bX \bV \bS ^{-1} ( \bI - (\bI - \eta \bS)^k ) \bV^T \bX^T \by
%     \end{align}
%     where $\sigma_{\text{max}}$ is the maximum eigenvalue of the underlying covariance matrix $\bSigma$. Using the upper bound on term 1 in \eqref{eq:upperbound_diagonal}, we have 
%     \begin{align}
%         \Expt{ x \sim \calD_{\calX} }{\left(f_t(x) - \wt f_\lambda (x)\right)^2} &\le \sigma_{\text{max}} \cdot c(\eta, t) \cdot \by^T \bV \bX  \left( \bS^{-1} ( \bI - (\bI - \eta \bS)^t ) \right)^2 \bV^T \bX^T \by \\
%         &=   \kappa \cdot c(\eta, t) \cdot \sigma_{\text{min}}\cdot \norm{\bV \left( \bS^{-1} ( \bI - (\bI - \eta \bS)^t ) \right) \bV^T \bX^T \by}{2}^2 \\
%         &\le \kappa \cdot c(\eta, t) \cdot \left[ \bV \left( \bS^{-1} ( \bI - (\bI - \eta \bS)^t ) \right) \bV^T \bX^T \right]^T \bSigma \\
%         &\qquad \qquad \qquad \qquad \qquad \left[ \bV \left( \bS^{-1} ( \bI - (\bI - \eta \bS)^t ) \right) \bV^T \bX^T \right] \by \\
%         & = \kappa \cdot c(\eta, t) \cdot \Expt{x \sim \calD_{\calX}}{\norm{x^T w_t}{2}} \,.
%     \end{align}
% % 
% % 
%     % Since $ \eta\le 1/\norm{S}{\text{op}}$, invoking \lemref{lem:ineq_soln} to upper bound term 1 with
% \end{proof}


% \newpage
% \section{Additional experiments and details}\label{app:exp}
% \newcommand\tab[1][1cm]{\hspace*{#1}}

% \subsection{Datasets} \label{sec:app_dataset}

% \textbf{Toy Dataset {} {}} Assume fixed constants $\mu$ and $\sigma$. For a given label $y$, we simulate features $x$ in our toy classification setup as follows: 
% \begin{align*}
%     x \defeq \texttt{concat} \left[ x_1, x_2\right] \quad \text{where} \quad  x_1 \sim  \calN( y \cdot \mu, \sigma^2 I_{d \times d}) \ \  \text{and} \ \  x_1 \sim  \calN( 0, \sigma^2 I_{d \times d}) \,.
% \end{align*}  
% % where $y$ is the true label and $x$ is the corresponding feature vector. 
% In experiements throughout the paper, we fix dimention $d=100$, $\mu = 1.0 $, and $\sigma = \sqrt{d}$. Intuitively, $x_1$ carries the information about the underlying label and $x_2$ is additional noise independent of the underlying label. 

% \textbf{CV datasets {} {}} We use MNIST~\citep{lecun1998mnist} and CIFAR10~\cite{krizhevsky2009learning}. 
% % For binary tasks, 
% We produce a binary variant from the multiclass classification problem by mapping classes $\{0,1,2,3,4\}$ to label $1$ and $\{ 5,6,7,8,9\}$ to label $-1$. For CIFAR dataset, we also use the standard data augementation of random crop and horizontal flip. PyTorch code is as follows: 

% \texttt{(transforms.RandomCrop(32, padding=4),\\
% \tab transforms.RandomHorizontalFlip())}

% \textbf{NLP dataset {} {}} We use IMDb Sentiment analysis~\citep{maas2011learning} corpus.  

% \subsection{Architecture Details} 

% All experiments were run on NVIDIA GeForce RTX 2080 Ti GPUs. We used PyTorch~\citep{NEURIPS2019a9015} and Keras with Tensorflow~\citep{abadi2016tensorflow} backend for experiments. 
% % , ELMo embeddings~\citep{Peters:2018}, and Hugging Face Transformers~\citep{wolf-etal-2020-transformers}. 

% \textbf{Linear model {} {}} For the toy dataset, we simulate a linear model with scalar output and the same number of parameters as the number of dimensions.   

% \textbf{Wide nets {} {}} To simulate the NTK regime, we experiment with $2-$layered wide nets. The PyTorch code for 2-layer wide MLP is as follows: 


% \texttt{ nn.Sequential( \\
% \tab     nn.Flatten(),\\
% \tab    nn.Linear(input\_dims, 200000, bias=True),\\
% \tab    nn.ReLU(),\\
% \tab    nn.Linear(200000, 1, bias=True)\\
% \tab     )}


% We experiment both (i) with the first layer fixed at random initialization; (ii)  and updating both layers' weights.     

% \textbf{Deep nets for CV tasks {} {}} We consider a 4-layered MLP. The PyTorch code for 4-layer MLP is as follows: 

% \texttt{ nn.Sequential(nn.Flatten(), \\
% \tab        nn.Linear(input\_dim, 5000, bias=True),\\
% \tab        nn.ReLU(),\\
% \tab        nn.Linear(5000, 5000, bias=True),\\
% \tab        nn.ReLU(),\\
% \tab        nn.Linear(5000, 5000, bias=True),\\
% \tab        nn.ReLU(),\\
% % \tab        nn.Linear(5000, 5000, bias=True),\\
% % \tab        nn.ReLU(),\\
% \tab        nn.Linear(1024, num\_label, bias=True)\\
% \tab        )}

% For MNIST, we use $1000$ nodes instead of $5000$ nodes in the hidden layer. 
% % 
% We also experiment with convolutional nets. In particular, we use ResNet18 \citep{he2016deep}. Implementation adapted from:  \url{https://github.com/kuangliu/pytorch-cifar.git}. 

% \textbf{Deep nets for NLP {} {}} We use a simple LSTM model with embeddings intialized with ELMo embeddings~\citep{Peters:2018}. Code adapted from: \url{https://github.com/kamujun/elmo_experiments/blob/master/elmo_experiment/notebooks/elmo_text_classification_on_imdb.ipynb} 

% We also evaluate our bounds with a BERT model. In particular, we fine-tune an off-the-shelf uncased BERT model~\citep{devlin2018bert}. Code adapted from Hugging Face Transformers~\citep{wolf-etal-2020-transformers}: \url{https://huggingface.co/transformers/v3.1.0/custom_datasets.html}. 


% \subsection{Additonal experiments}

% 1. SGD with linear models on cross entropy and MSE loss. 

% 2. CE loss and SGD. GD with MSE loss 

% 3. Binary MNIST with MLP. multiclass MNIST  

% \textbf{Results on CIFAR 10 {} {}} 
% % 
% We plot epoch wise error curve for results in \tabref{table:multiclass}. We observe the same trend as in \figref{fig:error_CIFAR10}. Additionally, we plot an \emph{oracle bound} obtained by tracking the error on mislabeled data which nevertheless were predicted as true label. To obtain an exact emprical value of the oracle bound, we need underlying true labels for the randomly labeled data. 
% % Note that our bound in \thmref{thm:multiclass_ERM}, lower bounds the accuracy as predicted by the oracle bound. 
% While with just access to extra unlabeled data we cannot calculate oracle bound, we note that the oracle bound is very tight and never violated in practice underscoring an importamt aspect of generalization in multiclass problems. This highlight that even a stronger conjecture may hold in multiclass classification, i.e., error on mislabeled data (where nevertheless true label was predicted) lower bounds the population error on the distribution of mislabeled data and hence, the error on (a specific) mislabeled portion predicts the population accuracy on clean data. 
% % 
% On the other hand, the dominating term of in \thmref{thm:multiclass_ERM} is loose when compared with the oracle bound. The main reason, we believe is the pessimistic upper bound in \eqref{eq:lemma1_final_multi_prev} in the proof of \lemref{lem:fit_mislabeled_multi}. We leave an investigation on this gap for future. 
% % of fit 

% % However, oracle bound highlights two . One,  



% \begin{figure}[h]
%     \centering 
%     % \vspace{-15pt}
%     % \includegraphics[width=0.9\linewidth]{example-image-a}
%     \includegraphics[width=0.4\linewidth]{figures/CIFAR10-FNN.pdf} \hfil
%     \includegraphics[width=0.4\linewidth]{figures/CIFAR10-Resnet.pdf}
%     % \includegraphics[width=0.9\linewidth]{figures/{CIFAR10_rn=0.1_lr=0.2_wd=0.005}.png}
%     % \vspace{-10pt}
%     \caption{ Per epoch curves for CIFAR10 corresponding results in \tabref{table:multiclass}. As before, we just plot the dominating term in the RHS of \eqref{eq:multiclass_ERM} as predicted bound. Additionally, we also plot the predicted lower bound by the error on mislabeled data which nevertheless were predicted as true label. We refer to this as ``Oracle bound''. See text for more details. 
%     % 
%     % except for the stopping point. 
%     % The bound predicted by RATT (RHS in \eqref{eq:multiclass_ERM}) is vacuous. 
%     }\label{fig:error_epoch_CIFAR10}
%     % \vspace{-15pt}
% \end{figure}


% \textbf{Results on CIFAR 100 {} {}} 
% % 
% On CIFAR100, our bound in \eqref{eq:multiclass_ERM} yields vacous bounds. However, the oracle bound as explained above yields tight guarantees in the initial phase of the learning (i.e., when learning rate is less than $0.1$). 

% \begin{figure}[h]
%     \centering 
%     % \vspace{-15pt}
%     % \includegraphics[width=0.9\linewidth]{example-image-a}
%     \includegraphics[width=0.4\linewidth]{figures/CIFAR100-Resnet.pdf}
%     % \includegraphics[width=0.9\linewidth]{figures/{CIFAR10_rn=0.1_lr=0.2_wd=0.005}.png}
%     % \vspace{-10pt}
%     \caption{ Predicted lower bound by the error on mislabeled data which nevertheless were predicted as true label with ResNet18 on CIFAR100. We refer to this as ``Oracle bound''. See text for more details. 
%     % 
%     % except for the stopping point. 
%     The bound predicted by RATT (RHS in \eqref{eq:multiclass_ERM}) is vacuous. 
%     }\label{fig:error_CIFAR100}
%     % \vspace{-15pt}
% \end{figure}


% % \paragraph{Experiments on CIFAR100} 



% \subsection{Hyperparameter Details}


% \textbf{\figref{fig:error_CIFAR10} {} {}} We use clean training dataset of size $40,000$. We fix the amount of unlabeled data at $20\%$ of the clean size, i.e. we include additional $8,000$ points with randomly assigned labels. We use test set of $10,000$ points. For both MLP and ResNet, we use SGD with an initial learning rate of $0.1$ and momentum $0.9$. We fix the weight decay parameter at $5\times 10^{-4}$. After $100$ epochs, we decay the learning rate to $0.01$. We use SGD batch size of $100$. 

% \textbf{\figref{fig:error_binary} (a) {} {}} We obtain a toy dataset according to the process described in \secref{sec:app_dataset}. We fix $d=100$ and create a dataset of $50,000$ points with balanced classes. Moreover, we sample additional covariates with the same procedure to create randomly labeled dataset. For both SGD and GD training, we use a fixed learning rate $0.1$.    

% \textbf{\figref{fig:error_binary} (b) {} {}} Similar to binary CIFAR, we use clean training dataset of size $40,000$ and fix the amount of unlabeled data at $20\%$ of the clean dataset size. To train wide nets, we use a fixed learning of $0.001$ with GD and SGD. We decide the weight decay parameter and the early stopping point that maximizes our generalization bound (i.e. without peeking at unseen data ).  We use SGD batch size of $100$. 

% \textbf{\figref{fig:error_binary} (c) {} {}} With IMDb dataset, we use a clean dataset of size $20,000$ and as before, fix the amount of unlabeled data at $20\%$ of the clean data. To train ELMo model, we use Adam optimizer with a fixed learning rate $0.01$ and weight decay $10^{-6}$ to minimize cross entropy loss. We train with batch size $32$ for 3 epochs. To fine-tune BERT model, we use Adam optimizer with learning rate $5\times 10^{-5}$ to minimize cross entropy loss. We train with a batch size of $16$ for 1 epoch.    

% \textbf{\tabref{table:multiclass} {} {}} For multiclass datasets, we train both MLP and ResNet with the same hyperparameters as described before. We sample a clean training dataset of size $40,000$ and fix the amount of unlabeled data at $20\%$ of the clean size. We use SGD with an initial learning rate of $0.1$ and momentum $0.9$. We fix the weight decay parameter at $5\times 10^{-4}$. After $30$ epochs for ResNet and after $50$ epochs for MLP, we decay the learning rate to $0.01$.  We use SGD with batch size $100$. 
% For \figref{fig:error_CIFAR100}, we use the same hyperparameters as 
% CIFAR10 training, except we now decay learning rate after $100$ epochs. 


% In all experiments, to identify the best possible accuracy on just the clean data, we use the exact same set of hyperparamters except the stopping point. We choose a stopping point that maximizes test performance. 

% \subsection{Summary of experiments }

% \begin{center}
%     \begin{table}[H] 
%         \centering
%         \begin{tabular}{|c|c|c|c|} 
%         \hline
%         Classification type & Model category & Model & Dataset  \\ [0.5ex] 
%         \hline
%         \hline
%         \multirow{9}{*}{Binary} & Low dimensional & Linear model & Toy Gaussain dataset  \\
%                         \cline{2-4}
%                          & \multirow{1}{*}{Overparameterized linear nets} 
%                         %  & Linear model & Toy Gaussain dataset \\
%                         %  \cline{3-4}
%                         %  & & 2-layer wide net& Toy Gaussain dataset \\
%                         %  \cline{3-4}
%                          & 2-layer wide net & Binary MNIST \\
%                          \cline{2-4}                 
%                          & \multirow{6}{*}{Deep nets} & \multirow{2}{*}{MLP} & Binary MNIST \\
%                          \cline{4-4}
%                          & &  & Binary CIFAR \\
%                          \cline{3-4}
%                          &  & \multirow{2}{*}{ResNet} & Binary MNIST \\
%                          \cline{4-4}
%                          & &  & Binary CIFAR \\
%                          \cline{3-4}
%                          &  & ELMo-LSTM model & IMDb Sentiment Analysis \\
%                          \cline{3-4}
%                          & & BERT pre-trained model & IMDb Sentiment Analysis \\
%         \hline
%         \multirow{5}{*}{Multiclass} & \multirow{5}{*}{Deep nets} & \multirow{2}{*}{MLP} & MNIST \\
%                         \cline{4-4} 
%                         & & & CIFAR10 \\                   
%                         \cline{3-4}
%                          &   & \multirow{3}{*}{ResNet} & MNIST \\
%                          \cline{4-4}
%                          &   & & CIFAR10 \\
%                          \cline{4-4}
%                          &   & & CIFAR100 \\
%         \hline
%         \end{tabular}
%         % \caption{Summary of experiments performed} \label{table:experiments}
%     \end{table}    
%     % \footnotetext[6]{We use both MSE loss and cross-entropy loss.}
%     % \footnotetext[6]{We try 2 variants: one with a fixed first layer and the other with both layers trainable.}
% \end{center}

% \newpage
% \section{Proof of \lemref{lem:stability_error}} \label{app:proof_lem_error}

% \begin{proof}[Proof of \lemref{lem:stability_error}]
%     Recall, we have a training set $S \cup \wt S_C$. We defined leave-one-out error on mislabeled points as $$\error_{\text{LOO}(\wt S_M) } = \frac{\sum_{(x_i, y_i) \in \wt S_M} \error( f_{(i)}( x_i), y_i)}{ \abs{\wt S_M }} \,, $$
%     where $f_{(i)} \defeq f(\calA, (S \cup \wt S)_{(i)})$. Define $S^\prime \defeq S \cup \wt S$. Assume $(x,y)$ and $(x^\prime,y^\prime)$ as i.i.d. samples from ${\calDm}$. 
%     Using Lemma 25 in \citet{bousquet2002stability}, we have
%     \begin{align*}
%         \Expo{ \left( \error_{\calDm}(\wh f) -\error_{\text{LOO}(\wt S_M) } \right)^2 } \le & \Expt{ S^\prime, (x,y), (x^\prime,y^\prime) }{ \error(\wh f(x), y ) \error(\wh f(x^\prime), y^\prime )} - 2 \Expt{ S^\prime, (x,y) }{ \error(\wh f(x), y ) \error(f_{(i)}(x_i), y_i )} \\
%         & + \frac{m_1-1}{m_1}\Expt{ S^\prime }{  \error(f_{(i)}(x_i), y_i )  \error(f_{(j)}(x_j), y_j )} + \frac{1}{m_1} \Expt{ S^\prime }{  \error(f_{(i)}(x_i), y_i ) } \,. \numberthis \label{eq:main_reln}
%     \end{align*}
%     We can rewrite the equation above as : 
%     \begin{align*}
%         \Expo{ \left( \error_{\calDm}(\wh f) -\error_{\text{LOO}(\wt S_M) } \right)^2 } \le &  \, \underbrace{\Expt{ S^\prime, (x,y), (x^\prime,y^\prime) }{ \error(\wh f(x), y ) \error(\wh f(x^\prime), y^\prime ) - \error(\wh f(x), y ) \error(f_{(i)}(x_i), y_i )}}_{\RN{1}} \\
%         & + \underbrace{\Expt{ S^\prime }{  \error(f_{(i)}(x_i), y_i )  \error(f_{(j)}(x_j), y_j ) -  \error(\wh f(x), y ) \error(f_{(i)}(x_i), y_i )}}_{\RN{2}} \\ &+ \underbrace{\frac{1}{m_1} \Expt{ S^\prime }{  \error(f_{(i)}(x_i), y_i ) - \error(f_{(i)}(x_i), y_i )  \error(f_{(j)}(x_j), y_j ) }}_{\RN{3}} \,. \numberthis \label{eq:main_reln2}
%     \end{align*}
    
%     We will now bound term $\RN{3}$.  Using Schwarz's inequality, we have
    
%     \begin{align}
%         \Expt{ S^\prime }{  \error(f_{(i)}(x_i), y_i ) - \error(f_{(i)}(x_i), y_i )  \error(f_{(j)}(x_j), y_j ) }^2 &\le  \Expt{ S^\prime }{  \error(f_{(i)}(x_i), y_i ) }^2 \Expt{S^\prime}{1 -   \error(f_{(j)}(x_j), y_j ) }^2 \\
%         &\le \frac{1}{4} \label{eq:term1_lem12}
%     \end{align}
    
%     Note that since $(x_i,y_i)$, $(x_j ,y_j )$, $(x,y)$, and $(x^\prime, y^\prime)$ are all from same distribution $\calDm$, we directly incorporate the bounds on term $\RN{1}$ and $\RN{2}$ from proof of Lemma 9 in \citet{bousquet2002stability}. Combining that with \eqref{eq:term1_lem12} and our definition of hypothesis stability in \codref{cond:hypothesis_stability}, we have the required claim. 
    
    
%     % We now re-write term $\RN{1}$ as
%     % \begin{align*}
%     %         &\Expt{S^\prime, (x,y), (x^\prime,y^\prime) }{ \error(\wh f(x), y ) \error(\wh f(x^\prime), y^\prime ) - \error(\wh f(x), y ) \error(f_{(i)}(x_i), y_i )} \\ & \qquad = \Expt{ S^\prime, (x,y), (x^\prime,y^\prime) }{ \error(\wh f(x), y ) \error(\wh f  (x^\prime), y^\prime ) - \error(\wh f ^\prime(x), y ) \error(f_{(i)}(x^\prime), y^\prime )} \tag{Exchanging $(x_i, y_i)$ with $(x^\prime, y^\prime)$ in the second term} \\
%     %         & \qquad = \Expt{ S^\prime, (x,y), (x^\prime,y^\prime) }{  \left(\error(\wh f(x), y )-  \error(f_{(i)}(x), y ) \right) \error(\wh f  (x^\prime), y^\prime )  } \\
%     %         & \qquad  + \Expt{ S^\prime, (x,y), (x^\prime,y^\prime) }{  \left(\error(f_{(i)}(x), y ) -\error(\wh f ^\prime(x), y ) \right) \error(\wh f  (x^\prime), y^\prime )}  \\
%     %         & \qquad +\Expt{ S^\prime, (x,y), (x^\prime,y^\prime) }{  \left( \error(\wh f  (x^\prime), y^\prime ) -  \error(f_{(i)}(x^\prime), y^\prime ) \right) \error(\wh f ^\prime(x), y ) }  \,, \numberthis \label{eq:term1_final}
%     % \end{align*}
%     % where $\wh f^\prime$ is the classifier obtained by training on $ S^\prime_{(i)} \cup \{ (x^\prime, y^\prime) \} $. Similarly we can re-write term $\RN{2}$ as 
%     % \begin{align*}
%     %     & \Expt{ S^\prime }{  \error(f_{(i)}(x_i), y_i )  \error(f_{(j)}(x_j), y_j ) -  \error(\wh f(x), y ) \error(f_{(i)}(x_i), y_i )} \\
%     %     &\quad  = \Expt{ S^\prime, (x,y), (x^\prime,y^\prime)}{  \error(f^{\prime\prime}_{(i)}(x), y )  \error(f_{(j)}^{\prime}(x^\prime), y^\prime ) -  \error(\wh f(x), y ) \error(f_{(i)}(x_i), y_i )} \tag{Exchanging $(x_i, y_i)$ with $(x, y)$ and $(x_j, y_j)$ with $(x^\prime, y^\prime)$ in the first term}\\
%     %     &\quad = \Expt{ S^\prime, (x,y), (x^\prime,y^\prime)}{  \error(f^{\prime\prime}_{(j)}(x), y )  \error(f_{(i)}^{\prime}(x^\prime), y^\prime ) -  \error(\wh f^\prime (x), y ) \error(f^\prime_{(j)}(x^\prime), y^\prime )} \tag{Exchanging $(x_i, y_i)$ and $(x_j, y_j)$ and then replacing $(x_j, y_j)$ with $(x^\prime, y^\prime)$ in the second term} \\
%     %     & \quad = \Expt{ S^\prime, (x,y), (x^\prime,y^\prime) }{  \left( \error(f_{(i)}^{\prime}(x^\prime), y^\prime )   -  \error(\wh f^{\prime\prime}  (x^\prime), y^\prime ) \right)  \error(f^{\prime\prime}_{(j)}(x), y )   } \\
%     %     & \quad  + \Expt{ S^\prime, (x,y), (x^\prime,y^\prime) }{  \left( \error(f^{\prime\prime}_{(j)}(x), y )  -\error(\wh f ^\prime(x), y ) \right) \error(\wh f^{\prime\prime}  (x^\prime), y^\prime )  }  \\
%     %     & \quad+ \Expt{ S^\prime, (x,y), (x^\prime,y^\prime) }{  \left( \error(\wh f^{\prime\prime}  (x^\prime), y^\prime )  -  \error(f^\prime_{(j)}(x^\prime), y^\prime ) \right)  \error(\wh f^\prime (x), y ) }   \\
%     %     & \quad = \Expt{ S^\prime, (x,y), (x^\prime,y^\prime) }{  \left( \error(f_{(i)}^{\prime}(x^\prime), y^\prime )   -  \error(\wh f (x^\prime), y^\prime ) \right)  \error(f_{(i)}(x_j), y_j )   } \\
%     %     & \quad  + \Expt{ S^\prime, (x,y), (x^\prime,y^\prime) }{  \left( \error(f^{\prime\prime}_{(j)}(x), y )  -\error(\wh f (x), y ) \right) \error(\wh f^{\prime\prime}  (x_j), y_j )  }  \\
%     %     & \quad+ \Expt{ S^\prime, (x,y), (x^\prime,y^\prime) }{  \left( \error(\wh f^{\prime\prime}  (x^\prime), y^\prime )  -  \error(f^\prime_{(j)}(x^\prime), y^\prime ) \right)  \error(\wh f^\prime (x^\prime), y^\prime ) }  \,, \numberthis \label{eq:term2_final}
%     % \end{align*}
%     % where $f^{\prime\prime}_{(j)}$ is trained on $S^\prime_{(j,i)} \cup {(x,y)}$, $f^{\prime}_{(i)}$ is trained on $S^\prime_{(j,i)} \cup {(x^\prime,y^\prime)}$, and $\wh f^{\prime\prime} $ is trained on $S^\prime_{(j)} \cup {(x,y)}$. Note in the last line we replaced $(x,y)$ by $(x_j, y_j)$ in the first term, replaced $(x^\prime,y^\prime)$ by $(x_j, y_j)$ in the second term and exchanged $(x_i,y_i)$ with $(x_j,y_j)$ and also $(x,y)$ and $(x^\prime, y^\prime)$
    
    
% \end{proof}
%}

\end{document}
