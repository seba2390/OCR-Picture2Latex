\documentclass[10pt,twocolumn,letterpaper]{article}

\usepackage{iccv}
\usepackage{times}
\usepackage{epsfig}
\usepackage{graphicx}
\usepackage{amsmath}
\usepackage{amssymb}
\usepackage{bbm}
\usepackage{enumitem}
\usepackage{tabularx}
\usepackage{rotating}
\usepackage{xcolor}
\usepackage{subcaption}
\usepackage{multirow}
\usepackage{booktabs}
\usepackage{url}
\usepackage[font=small]{caption}
\newcommand\todo[1]{\textcolor{red}{#1}}
%\usepgfplotslibrary{external} 
%\tikzexternalize
\newcommand*\rot{\rotatebox{90}}
\newcommand\boldhead[1]{\vspace{0.03in}\noindent\textbf{#1: }}
% Include other packages here, before hyperref.

% Using baseline stretch. If possible get rid of it by making more space.
% -Saurabh
\renewcommand{\baselinestretch}{0.99} 

% If you comment hyperref and then uncomment it, you should delete
% egpaper.aux before re-running latex.  (Or just hit 'q' on the first latex
% run, let it finish, and you should be clear).
%\usepackage[pagebackref=true,breaklinks=true,letterpaper=true,colorlinks,bookmarks=false]{hyperref}

\iccvfinalcopy % *** Uncomment this line for the final submission

\def\iccvPaperID{1997} % *** Enter the ICCV Paper ID here
\def\httilde{\mbox{\tt\raisebox{-.5ex}{\symbol{126}}}}

% Pages are numbered in submission mode, and unnumbered in camera-ready
\ificcvfinal\pagestyle{empty}\fi
\begin{document}

%%%%%%%%% TITLE
\title{Aligned Image-Word Representations Improve Inductive Transfer Across Vision-Language Tasks}
%\title{Improving Inductive Transfer across Vision-Language Tasks through Aligned Image-Word Representations}
%\title{Learning Image-Word Representations from Complementary and Hierarchical Vision-Language Tasks}
% Keywords - Shared/rich/generalizable representation, multiple vision-language tasks, inductive transfer, intermediate supervision 
% Sharing Vision-Language representations across multiple tasks for inductive transfer   
% What does sharing Vision-Language representation across tasks give us?
% Learning image and language representations from diverse and complementary experience
% Learning image and language representations from diverse and complementary task hierarchy
%\author{First Author\\
%Institution1\\
%Institution1 address\\
%{\tt\small firstauthor@i1.org}
% For a paper whose authors are all at the same institution,
% omit the following lines up until the closing ``}''.
% Additional authors and addresses can be added with ``\and'',
% just like the second author.
% To save space, use either the email address or home page, not both
%\and
%Second Author\\
%Institution2\\
%First line of institution2 address\\
%{\tt\small secondauthor@i2.org}
%}
\author{Tanmay Gupta$^1$ \hspace{0,5cm} Kevin Shih$^1$ \hspace{0,5cm} Saurabh Singh$^2$ \hspace{0,5cm} Derek Hoiem$^1$\\
$^1$University of Illinois, Urbana-Champaign  \hspace{0,5cm} $^2$Google Inc.\\
{\tt\small \{tgupta6, kjshih2, dhoiem\}@illinois.edu} \hspace{0,5cm} {\tt\small saurabhsingh@google.com}}

\maketitle
%\thispagestyle{empty}

%%%%%%%%% ABSTRACT
\begin{abstract}
   An important goal of computer vision is to build systems that learn visual representations over time that can be applied to many tasks. In this paper, we investigate a vision-language embedding as a core representation and show that it leads to better cross-task transfer than standard multi-task learning. In particular, the task of visual recognition is aligned to the task of visual question answering by forcing each to use the same word-region embeddings. We show this leads to greater inductive transfer from recognition to VQA than standard multitask learning. Visual recognition also improves, especially for categories that have relatively few recognition training labels but appear often in the VQA setting.  Thus, our paper takes a small step towards creating more general vision systems by showing the benefit of interpretable, flexible, and trainable core representations.  
   
  
   %Applying the knowledge learned while solving one task to a different but related task is a challenging problem. We propose to address this challenge for vision-language problems by sharing a region and word representation module across models for different tasks. The domain specific models exploit the alignment between the image-region and word representations produced by this shared vision-language representation (SVLR) module. Using visual recognition and visual question answering as an example of related vision-language tasks, we show how to incorporate the SVLR module into models for these tasks. When trained jointly on all tasks, the SVLR module receives diverse but consistent supervision to produce rich representations, improving the performance of individual models through inductive transfer. Our formulation leads to interpretable predictions for VQA and allows for zero-shot VQA. 
   
   
   %Representation plays a crucial role in many vision-language problems such as visual recognition (VR), visual question answering (VQA), image captioning etc. Models for these tasks learn image and language features via end-to-end training. However, these models often ignore the complementary and hierarchical nature of supervision available across datasets for different tasks. For instance, while VR datasets (such as ImageNet) provide a direct mapping between uncluttered images and class label words, VQA datasets provide an alignment between real-world images and the complex semantics of a question-answer (QA) pair. 
   
   %To learn rich image-region and word representations that generalize across tasks, we propose a shared vision-language representation (SVLR) module that is shared across models for different tasks. Using VR and VQA as an example of a complementary task hierarchy, we show how to incorporate SVLR module into popular models for these tasks. We demonstrate that when trained jointly on all tasks, SVLR accumulates diverse supervision to produce rich representations that can boost performance of individual models through inductive transfer. Our formulation also leads to more interpretable predictions for VQA. 
\end{abstract}

%%%%%%%%% BODY TEXT
\vspace{-5mm}

\section{Introduction}

% !TEX root = ../arxiv.tex

Unsupervised domain adaptation (UDA) is a variant of semi-supervised learning \cite{blum1998combining}, where the available unlabelled data comes from a different distribution than the annotated dataset \cite{Ben-DavidBCP06}.
A case in point is to exploit synthetic data, where annotation is more accessible compared to the costly labelling of real-world images \cite{RichterVRK16,RosSMVL16}.
Along with some success in addressing UDA for semantic segmentation \cite{TsaiHSS0C18,VuJBCP19,0001S20,ZouYKW18}, the developed methods are growing increasingly sophisticated and often combine style transfer networks, adversarial training or network ensembles \cite{KimB20a,LiYV19,TsaiSSC19,Yang_2020_ECCV}.
This increase in model complexity impedes reproducibility, potentially slowing further progress.

In this work, we propose a UDA framework reaching state-of-the-art segmentation accuracy (measured by the Intersection-over-Union, IoU) without incurring substantial training efforts.
Toward this goal, we adopt a simple semi-supervised approach, \emph{self-training} \cite{ChenWB11,lee2013pseudo,ZouYKW18}, used in recent works only in conjunction with adversarial training or network ensembles \cite{ChoiKK19,KimB20a,Mei_2020_ECCV,Wang_2020_ECCV,0001S20,Zheng_2020_IJCV,ZhengY20}.
By contrast, we use self-training \emph{standalone}.
Compared to previous self-training methods \cite{ChenLCCCZAS20,Li_2020_ECCV,subhani2020learning,ZouYKW18,ZouYLKW19}, our approach also sidesteps the inconvenience of multiple training rounds, as they often require expert intervention between consecutive rounds.
We train our model using co-evolving pseudo labels end-to-end without such need.

\begin{figure}[t]%
    \centering
    \def\svgwidth{\linewidth}
    \input{figures/preview/bars.pdf_tex}
    \caption{\textbf{Results preview.} Unlike much recent work that combines multiple training paradigms, such as adversarial training and style transfer, our approach retains the modest single-round training complexity of self-training, yet improves the state of the art for adapting semantic segmentation by a significant margin.}
    \label{fig:preview}
\end{figure}

Our method leverages the ubiquitous \emph{data augmentation} techniques from fully supervised learning \cite{deeplabv3plus2018,ZhaoSQWJ17}: photometric jitter, flipping and multi-scale cropping.
We enforce \emph{consistency} of the semantic maps produced by the model across these image perturbations.
The following assumption formalises the key premise:

\myparagraph{Assumption 1.}
Let $f: \mathcal{I} \rightarrow \mathcal{M}$ represent a pixelwise mapping from images $\mathcal{I}$ to semantic output $\mathcal{M}$.
Denote $\rho_{\bm{\epsilon}}: \mathcal{I} \rightarrow \mathcal{I}$ a photometric image transform and, similarly, $\tau_{\bm{\epsilon}'}: \mathcal{I} \rightarrow \mathcal{I}$ a spatial similarity transformation, where $\bm{\epsilon},\bm{\epsilon}'\sim p(\cdot)$ are control variables following some pre-defined density (\eg, $p \equiv \mathcal{N}(0, 1)$).
Then, for any image $I \in \mathcal{I}$, $f$ is \emph{invariant} under $\rho_{\bm{\epsilon}}$ and \emph{equivariant} under $\tau_{\bm{\epsilon}'}$, \ie~$f(\rho_{\bm{\epsilon}}(I)) = f(I)$ and $f(\tau_{\bm{\epsilon}'}(I)) = \tau_{\bm{\epsilon}'}(f(I))$.

\smallskip
\noindent Next, we introduce a training framework using a \emph{momentum network} -- a slowly advancing copy of the original model.
The momentum network provides stable, yet recent targets for model updates, as opposed to the fixed supervision in model distillation \cite{Chen0G18,Zheng_2020_IJCV,ZhengY20}.
We also re-visit the problem of long-tail recognition in the context of generating pseudo labels for self-supervision.
In particular, we maintain an \emph{exponentially moving class prior} used to discount the confidence thresholds for those classes with few samples and increase their relative contribution to the training loss.
Our framework is simple to train, adds moderate computational overhead compared to a fully supervised setup, yet sets a new state of the art on established benchmarks (\cf \cref{fig:preview}).



\begin{figure*}[t]
\begin{center}
%\fbox{\rule{0pt}{2in} \rule{.6\linewidth}{0pt}}
\includegraphics[width=\linewidth]{images/joint_training.pdf}
\vspace{-0.9cm}
\end{center}
   \caption{\textbf{Joint Training on Visual Recognition(VR) and Visual Question Answering(VQA) with the proposed SVLR Module:} The figure depicts sharing of image and word representations through the SVLR module during joint training on object recognition, attribute recognition, and VQA. The recognition tasks use object and attribute labelled regions from Visual Genome while VQA uses images annotated with questions and answers from the VQA dataset. The benefit of joint training is that while the VQA dataset does not provide region groundings of nouns and adjectives in the QA (e.g. ``fluffy",``dog"), this complementary supervision is provided by the Genome recognition dataset. Models for each task involve image and word embeddings produced by SVLR module or their inner products (See Fig~\ref{fig:system} for VQA model architecture).
   }
   \vspace{-6mm}
\label{fig:features}
\end{figure*}

%  \caption{\textbf{Joint Training on Visual Recognition(VR) and Visual Question Answering(VQA) with the proposed SVLR Module:} The figure depicts sharing of image and word representations through the SVLR module during joint training on object recognition, attribute recognition, and VQA. The recognition tasks use object and attribute labelled regions from Visual Genome while VQA uses images annotated with questions and answers from the VQA dataset. The benefit of joint training is that while the VQA dataset does not provide region groundings of nouns and adjectives in the QA (eg. 'fluffy','dog'), this complementary supervision is provided by the Genome recognition dataset. Models for each task involve image and word embeddings produced by SVLR module or their inner products (See Fig~\ref{fig:system} for VQA model architecture). The parameters of the SVLR module and Answer Scoring module in VQA are trained using stochastic gradient descent. As a baseline, we replace $g(y;\theta_{svlr})$, a function of word2vec representation of $y$, with learnable weight vector $h_y$ in the VR models. This is equivalent to using a 1000 way classification layer like the last layers of AlexNet, VGG, or ResNet instead of using our language embeddings for classification. This enables comparison of inductive transfer due to joint training both with and without sharing language representations.}


%-------------------------------------------------------------------------
\section{Related Work}
%\mz{We lack a comparison to this paper: https://arxiv.org/abs/2305.14877}
%\anirudh{refine to be more on-topic?}
\iffalse
\paragraph{In-Context Learning} As language models have scaled, the ability to learn in-context, without any weight updates, has emerged. \cite{brown2020language}. While other families of large language models have emerged, in-context learning remains ubiquitous \cite{llama, bloom, gptneo, opt}. Although such as HELM \cite{helm} have arisen for systematic evaluation of \emph{models}, there is no systematic framework to our knowledge for evaluating \emph{prompting methods}, and validating prompt engineering heuristics. The test-suite we propose will ensure that progress in the field of prompt-engineering is structured and objectively evaluated. 

\paragraph{Prompt Engineering Methods} Researchers are interested in the automatic design of high performing instructions for downstream tasks. Some focus on simple heuristics, such as selecting instructions that have the lowest perplexity \cite{lowperplexityprompts}. Other methods try to use large language models to induce an instruction when provided with a few input-output pairs \cite{ape}. Researchers have also used RL objectives to create discrete token sequences that can serve as instructions \cite{rlprompt}. Since the datasets and models used in these works have very little intersection, it is impossible to compare these methods objectively and glean insights. In our work, we evaluate these three methods on a diverse set of tasks and models, and analyze their relative performance. Additionally, we recognize that there are many other interesting angles of prompting that are not covered by instruction engineering \cite{weichain, react, selfconsistency}, but we leave these to future work.

\paragraph{Analysis of Prompting Methods} While most prompt engineering methods focus on accuracy, there are many other interesting dimensions of performance as well. For instance, researchers have found that for most tasks, the selection of demonstrations plays a large role in few-shot accuracy \cite{whatmakesgoodicexamples, selectionmachinetranslation, knnprompting}. Additionally, many researchers have found that even permuting the ordering of a fixed set of demonstrations has a significant effect on downstream accuracy \cite{fantasticallyorderedprompts}. Prompts that are sensitive to the permutation of demonstrations have been shown to also have lower accuracies \cite{relationsensitivityaccuracy}. Especially in low-resource domains, which includes the large public usage of in-context learning, these large swings in accuracy make prompting less dependable. In our test-suite we include sensitivity metrics that go beyond accuracy and allow us to find methods that are not only performant but reliable.

\paragraph{Existing Benchmarks} We recognize that other holistic in-context learning benchmarks exist. BigBench is a large benchmark of 204 tasks that are beyond the capabilities of current LLMs. BigBench seeks to evaluate the few-shot abilities of state of the art large language models, focusing on performance metrics such as accuracy \cite{bigbench}. Similarly, HELM is another benchmark for language model in-context learning ability. Rather than only focusing on performance, HELM branches out and considers many other metrics such as robustness and bias \cite{helm}. Both BigBench and HELM focus on ranking different language model, while fix a generic instruction and prompt format. We instead choose to evaluate instruction induction / selection methods over a fixed set of models. We are the first ever evaluation script that compares different prompt-engineering methods head to head. 
\fi

\paragraph{In-Context Learning and Existing Benchmarks} As language models have scaled, in-context learning has emerged as a popular paradigm and remains ubiquitous among several autoregressive LLM families \cite{brown2020language, llama, bloom, gptneo, opt}. Benchmarks like BigBench \cite{bigbench} and HELM \cite{helm} have been created for the holistic evaluation of these models. BigBench focuses on few-shot abilities of state-of-the-art large language models, while HELM extends to consider metrics like robustness and bias. However, these benchmarks focus on evaluating and ranking \emph{language models}, and do not address the systematic evaluation of \emph{prompting methods}. Although contemporary work by \citet{yang2023improving} also aims to perform a similar systematic analysis of prompting methods, they focus on simple probability-based prompt selection while we evaluate a broader range of methods including trivial instruction baselines, curated manually selected instructions, and sophisticated automated instruction selection.

\paragraph{Automated Prompt Engineering Methods} There has been interest in performing automated prompt-engineering for target downstream tasks within ICL. This has led to the exploration of various prompting methods, ranging from simple heuristics such as selecting instructions with the lowest perplexity \cite{lowperplexityprompts}, inducing instructions from large language models using a few annotated input-output pairs \cite{ape}, to utilizing RL objectives to create discrete token sequences as prompts \cite{rlprompt}. However, these works restrict their evaluation to small sets of models and tasks with little intersection, hindering their objective comparison. %\mz{For paragraphs that only have one work in the last line, try to shorten the paragraph to squeeze in context.}

\paragraph{Understanding in-context learning} There has been much recent work attempting to understand the mechanisms that drive in-context learning. Studies have found that the selection of demonstrations included in prompts significantly impacts few-shot accuracy across most tasks \cite{whatmakesgoodicexamples, selectionmachinetranslation, knnprompting}. Works like \cite{fantasticallyorderedprompts} also show that altering the ordering of a fixed set of demonstrations can affect downstream accuracy. Prompts sensitive to demonstration permutation often exhibit lower accuracies \cite{relationsensitivityaccuracy}, making them less reliable, particularly in low-resource domains.

Our work aims to bridge these gaps by systematically evaluating the efficacy of popular instruction selection approaches over a diverse set of tasks and models, facilitating objective comparison. We evaluate these methods not only on accuracy metrics, but also on sensitivity metrics to glean additional insights. We recognize that other facets of prompting not covered by instruction engineering exist \cite{weichain, react, selfconsistency}, and defer these explorations to future work. 



%-------------------------------------------------------------------------


\section{Method}
We propose an SVLR module to facilitate greater inductive transfer across vision-language tasks. As shown in Fig.~\ref{fig:features}, the word and region representations required for object recognition, attribute recognition, and VQA are computed through the SVLR module. By specifically formulating each task in terms of inner products of word and region representations and training on all tasks jointly, we ensure each task provides a consistent, non-conflicting training signal for aligning words and region representations. During training, the joint-task model is fed batches containing training examples from each task's dataset.

\vspace{-1mm}

\subsection{Shared Vision Language Representation}
\label{sec:embedding}
The SVLR module converts words and image-regions into feature representations that are aligned to each other and \textit{shared} across tasks.  \\

\noindent\textbf{Word Representations:} The representation $g(w)$ for a word $w$ is constructed by applying two fully connected layers (with 300 output units each) to pretrained word2vec representation~\cite{word2vec} of $w$ with ReLU after the first layer. \\

\noindent\textbf{Region Representations:} A region $R$ is represented using two $300$ dimensional feature vectors $f_o(R)$ and $f_a(R)$ that separately encode the objects and attributes contained. We used two representations instead of one to encourage disentangling of these two factors of variation. For example, we do not expect ``red" to be similar to ``apple", but we expect $f_o(R)$ and $f_a(R)$ to be similar to $g(``red")$ and $g(``apple")$ if $R$ depicts a red apple. The features are constructed by extracting the average pooled features from Resnet~\cite{he2015deep} pretrained on ImageNet and then passing through separate object and attribute networks. Both networks consist of two fully connected layers (with 2048 and 300 output units) with batch normalization~\cite{batchnorm} and ReLU activations. 

\subsection{Visual Recognition using SVLR}
\label{sec:vr}
\subsubsection{Inference}\label{sec:recog_inference}
The visual recognition task is to classify image regions into one or more object and attribute categories. The classification score for region $R$ and object category $w$ is $f_o^T(R)g(w)$. The classification score for an attribute category $v$ is $f_a^T(R)g(v)$. Attributes may include adjectives and adverbs (e.g., ``standing''). Though our recognition dataset has a limited set of object categories $\mathcal{O}$ and attribute categories $\mathcal{T}$,  our model can produce classification scores for any object or attribute label given its word2vec representation. In experiments, the $\mathcal{O}$ and $\mathcal{T}$ consist of 1000 most frequent object and attribute categories in the Visual Genome dataset~\cite{krishna2016visual}. 

\vspace{-3mm}
\subsubsection{Training}\label{sec:recog_learn}
Our VR model is trained using the Visual Genome dataset which provides image regions annotated with object and attribute labels. VR uses only the parameters for the embedding functions $f_o, f_a$ and $g$ that are part of the SVLR module. The parameters of $f_o$ receive gradients from the object loss while those of $f_a$ receive gradients from the attribute loss. The parameters of word embedding model $g$ receive gradients from both losses.\\

\noindent 
\textbf{Object loss:} %We use a multi-label classification loss as object classes may not be mutually exclusive due to hypernyms (e.g., ``man'' \emph{is a} ``person'') and synonyms.  
We use a multi-label loss as object classes may not be mutually exclusive (e.g., ``man'' \emph{is a} ``person''). For a region $R_j$, we denote the set of annotated object categories and their hypernyms extracted from WordNet \cite{miller1995acm} by $\mathcal{H}_j$. The object loss forces the true labels and their hypernyms to score higher than all other object labels by a margin $\eta_{obj}$. For a batch of $M$ samples $\{(R_j,\mathcal{H}j)\}_{j=1}^{M}$ the object loss is:
\vspace{-5mm}
\begin{multline}
\mathcal{L}_{obj} = \frac{1}{M}\sum_{j=1}^{M}\frac{1}{|\mathcal{H}j|}\sum_{l \in \mathcal{H}j} \frac{1}{|\mathcal{O}|}
\sum_{k \in \mathcal{O}\setminus \mathcal{H}j} \\
\max\{0,\eta_{obj} + f_o^T(R_j)g(k) - f_o^T(R_j)g(l)\}
\end{multline}

\noindent
\textbf{Attribute Loss:} The attribute loss is a multi-label classification loss with two differences from object classification. Attribute labels are even less likely to be mutually exclusive than object labels. As such, we predict each attribute with independent cross entropy losses. We also weigh the samples based on fraction of positive labels in the batch to balance the positive and negative labels in the dataset. For a batch with M samples $\{(R_j,\mathcal{T}_j)\}_{j=1}^{M}$ where $\mathcal{T}_j$ is the set of attributes annotated for region $R_j$, the attribute loss is:
\vspace{-4mm}
\begin{multline}
\mathcal{L}_{atr} = \frac{1}{M}
\sum_{j=1}^{M} 
\sum_{t \in \mathcal{T}} \\
\mathbbm{1}\left[t \in \mathcal{T}_j\right](1-\Gamma(t))\log\left[\sigma(f_a^T(R_j)g(t))\right] + \\
\mathbbm{1}\left[t \notin \mathcal{T}_j\right]\Gamma(t)\log\left[1-\sigma(f_a^T(R_j)g(t))\right]
\end{multline}    
where $\sigma$ is a sigmoid activation function and $\Gamma(t)$ is the fraction of positive samples for attribute $t$ in the batch.

\begin{figure*}[ht]
\begin{center}
%\fbox{\rule{0pt}{2in} \rule{0.9\linewidth}{0pt}}
\includegraphics[width=\linewidth]{images/vqa_model.pdf}
\vspace{-1.1cm}
\end{center}
  \caption{\textbf{Inference in our VQA model:} The image is first broken down into Edge Box region proposals\cite{zitnick2014edge}. Each region $R$ is represented by visual category scores $s(R) = [s_o(R), s_a(R)]$ obtained using the visual recognition model. Using the SVLR module, the regions are also assigned an attention score using the inner products of region features with representations of nouns and adjectives in the question and answer. The region features are then pooled using the relevance scores as weights to construct the \textit{attended} image representation. Finally, the image and question/answer representations are combined and passed through a neural network to produce a score for the input question-image-answer triplet.}
  \vspace{-5mm}
\label{fig:system}
\end{figure*}
\vspace{-1mm}
\subsection{Visual Question Answering using SVLR} \label{sec:vqa}

Our VQA model is illustrated in Fig.~\ref{fig:system}.
%\subsection{Inference} \label{sec:vqa_inference}
The input to our VQA model is an image, a question, and a candidate answer. Regions are extracted from the image using Edge Boxes~\cite{zitnick2014edge}.  The same SVLR module used by VR (Sec.~\ref{sec:vr}) is explicitly applied to VQA for attention and answer scoring.  Our system assigns attention scores to each region according to how well it matches words in the question/answer, then scores each answer based on the question, answer, and attention-weighted scores for all objects ($\mathcal{O}$) and attributes ($\mathcal{T}$).

\boldhead{Attention Scoring}\label{sec:relevance} Unlike other attention models ~\cite{yang2015stacked,lu2016hierarchical} that are free to learn any correlation between regions and question/answers, our attention model encodes an explicit notion of vision-language grounding.  Let $\mathcal{R}$ be the set of region proposals extracted from the image, and $\mathcal{N}$ and $\mathcal{J}$ denote the set of nouns and adjectives in the $(Q,A)$ pair. Each region $R\in \mathcal{R}(I)$ is assigned an attention score $a(R)$ as follows:\\
\vspace{-5mm}
\begin{align}
a'(R) &= \max_{n \in \mathcal{N}} f_o^T(R)g(n) + \max_{j \in \mathcal{J}} f_a^T(R)g(j) \label{eqn:relevance_unnormalized}\\
a(R)&= \frac{\exp(a'(R))}{\sum_{R' \in \mathcal{R}(I)}\exp(a'(R'))}
\label{eqn:relevance}
\end{align}

Thus, a region's attention score is the sum of maximum adjective and noun scores for words mentioned in the question or answer (which need not be in sets $\mathcal{O}$  and $\mathcal{T}$).   

%The image representation is computed by averaging these VR scores using the attention scores as weights. This image representation is combined with the QA representation using bimodal pooling layers, and the resulting encoding of the image-question-answer triplet is scored using a fully connected layer. The answer with the highest score is chosen. \\

\boldhead{Image Representation} To score an answer, the content of region $R$ is encoded using the VR scores for all objects and attributes in $\mathcal{O}$ and $\mathcal{T}$, as presence of unmentioned objects or attributes may help answer the question. The image representation is an attention-weighted average of these scores across all regions: 
%To capture the visual content of a region $R$, we concatenate the object and attribute scores for visual categories $\mathcal{O}$ and $\mathcal{T}$ into vectors $s_o(R) \in \mathbb{R}^{1000}$ and $s_a(R) \in \mathbb{R}^{1000}$. The final 2000 dimensional, $(Q,A)$ specific, attended image representation is constructed by averaging features $s_o$ and $s_a$ across all regions using relevance scores as weights.
\begin{equation}
f(I) = \sum_{R\in\mathcal{R}(I)}a(R)
\begin{bmatrix}
s_o(R) \\
s_a(R) \\
\end{bmatrix}
\label{eqn:weightedvisualfeats}
\end{equation}
where $I$ is the image, $s_o(R)$ are the scores for 1000 objects in $\mathcal{O}$ for each image region $R$, $s_a(R)$ are the scores for 1000 attributes in $\mathcal{T}$, and $a(R)$ is the attention score.

\boldhead{Question/Answer Representation}  To construct representations $q(Q)$ and $a(A)$ for the question and answer, we follow Shih et al.~\cite{shih2016look}, dividing question words into 4 bins, averaging word representations in each bin, and concatenating the bin representations resulting in a 1200 ($=300\times4$) dimensional vector $q(Q)$. The answer representation $a(A)\in\mathbb{R}^{300}$ is obtained by averaging the word representations of all answer words. The word representations used here are produced by the SVLR module.

\boldhead{Answer Scoring} We combine the image and Q/A representations to jointly score the $(Q,I,A)$ triplet. %Since some answers like $\{0,1,2,3,yes,no\}$ may not be well represented using vector representations, we experiment with appending binary features for these answers in $a(A)$.

To ensure equal contribution of language and visual features, we apply batch normalization~\cite{batchnorm} on linear transformations of these features before adding them together to get a bimodal representation $\beta(Q,I,A)\in\mathbb{R}^{2500}$:
\begin{multline}\label{eq:bimodal_pool}
\beta(Q,I,A) = \;\mathcal{B}_1(W_1f(I)) 
+ \;\mathcal{B}_2\left(W_2 
\begin{bmatrix}
q(Q) \\
a(A) \\
\end{bmatrix}
\right)
\end{multline}
Here, $\mathcal{B}_1,\mathcal{B}_2$ denote batch normalization and $W_1\in\mathbb{R}^{2500\times2000}$ and $W_2\in\mathbb{R}^{2500\times1500}$ define the linear transformations.
\noindent
The bimodal representation is: 
\begin{equation}
\mathcal{S}(Q,I,A) = W_3 \; \text{ReLU}(\beta(Q,I,A))
\end{equation}
with  ${W_3\in\mathbb{R}^{1\times2500}}$.


\boldhead{Training}\label{sec:vqa_learn}
We use the VQA dataset~\cite{antol2015vqa} for training parameters of our VQA model: $W_1, W_2, W_3$, and scales and offsets of batch normalization layers. In addition, the VQA loss backpropagates into $f_o, f_a$, and $g$ which are part of the SVLR module. Each sample in the dataset consists of a question $Q$ about an image $I$ with list of answer options including a positive answer $A^{+}$ and $N$ negative answers $\{A^{-}(i) | i=1,\cdots, N\}$. 

The VQA loss encourages the correct answer $A^{+}$ to be scored higher than all incorrect answer options $\{A^{-}(i) | i=1,\cdots, N\}$ by a margin $\eta_{ans}$. Given batch samples $\{(Q_j,I_j,A_j)\}_{j=1}^{P}$, the loss is written as 
\vspace{-2mm}
\begin{multline}
\mathcal{L}_{ans} = \frac{1}{NP}\sum_{j=1}^{P}
\sum_{i=1}^{N} \max\{0,\\\;\eta_{ans} + \mathcal{S}(Q_j,I_j,A_j^{-}(i)) - \mathcal{S}(Q_j,I_j,A_j^{+})\}
\end{multline}

\begin{figure*}[t]
\begin{center}
\includegraphics[width=0.95\linewidth]{images/relevance_qual_2_row.png}
\vspace{-0.7cm}
\end{center}
  \caption{\textbf{Interpretable inference in VQA:} Our model produces interpretable intermediate computation for region relevance and object/attribute predictions for the most relevant regions. Our region relevance explicitly grounds nouns and adjectives from the Q/A input in the image. We also show object and attribute predictions for the most relevant region identified for a few correctly answered questions. The relevance masks are generated from relevance scores projected back to their source pixels locations.}
  \vspace{-5mm}
\label{fig:rel_qual}
\end{figure*}

\subsection{Zero-Shot VQA} 

%In the introduction, we claim the ability to perform a new vision-language task without retraining as evidence that our SVLR transfers knowledge across tasks without the need to re-learn a feature to task mapping.  
The representations produced by SVLR module should be directly usable in related vision-language tasks without any additional learning. To demonstrate this \textit{zero-shot cross-task transfer}, we train the SVLR module using Genome VR data only and apply to VQA. Since bimodal pooling and scoring layers cannot be learned without VQA data, we use a proxy scoring function constructed using region-word scores only.  For each region, we compute $p_q(R)$ as the sum of its scores for the maximally aligned question nouns and question adjectives (Eq.~\ref{eqn:relevance_unnormalized} with only question words). A score $p_a(R)$ is similarly computed using answer nouns and adjectives. The final score for the answer is defined by
\begin{equation}
    S(Q,I,A)=\sum_{R\in\mathcal{R}}a(R)\min(p_q(R),p_a(R))
\end{equation}
where $a$ is the attention score computed using Eq.~\ref{eqn:relevance}. Therefore, the highest score is given to QA pairs where question as well as answer nouns and adjectives can be localized in the image. Note that the since the model is not trained on even a \textit{single question} from VQA, the zero-shot VQA task also shows that our model does use the image to answer questions instead of solely relying on the language prior which is a common concern with most VQA models \cite{agrawal2016analyzing,goyal2016arxiv}. 

\section{Implementation and Training Details}
We use 100 region proposals resized to $224 \times 224$ for all experiments. Resnet-50 was used for image feature extraction in all experiments except those in Tab.~\ref{tab:state_art} which used Resnet-152. The nouns and adjectives are extracted from the $(Q,A)$  and lemmatized using the part-of-speech tagger and WordNet lemmatizer in NLTK \cite{bird2009book}. We use the Stanford Dependency Parser \cite{de2006lrec} to parse the question into bins as detailed in~\cite{shih2016look}.  All models are implemented and trained using TensorFlow \cite{tensorflow2015software}. We train the model jointly for the recognition and VQA tasks by minimizing the following loss function using Adam~\cite{adamoptimizer}:
\begin{equation}
\mathcal{L} = \alpha_{ans}\mathcal{L}_{ans} + \alpha_{obj}\mathcal{L}_{obj} + \alpha_{atr}\mathcal{L}_{atr}
\end{equation}
We observe that values of $\alpha_{obj}$ and $\alpha_{atr}$ relative to $\alpha_{ans}$ can be used to trade-off performance between visual recognition and VQA tasks. For experiments that analyze the effect of transfer from VR to VQA (Sec.~\ref{sec:vr2vqa}), we set ${\alpha_{ans} = 1, \alpha_{obj} = 0.1}$, and ${\alpha_{atr}=0.1}$. For VQA only and Genome only baselines, we set the corresponding $\alpha$ to 1 and others to 0.  For experiments dealing with transfer in the other direction (Sec.~\ref{sec:vqa2vr}), we set ${\alpha_{ans} = 0.1, \alpha_{obj} = 1}$, and ${\alpha_{atr}=1}$.  The margins used for object and answer losses are $\eta_{ans}=\eta_{obj}=1$. The object and attribute losses are computed for the same set of Visual Genome regions with a batch size of $M=200$. The answer loss is computed for a batch size of $P=50$ questions sampled from VQA. We use an exponentially decaying learning rate schedule with an initial learning rate of $10^{-3}$ and decay rate of 0.5 every 24000 iterations. Weight decay is used on all trainable variables with a coefficient of $10^{-5}$. All the variables are Xavier initialized \cite{glorot2010aistats}.


In this section we conduct comprehensive experiments to emphasise the effectiveness of DIAL, including evaluations under white-box and black-box settings, robustness to unforeseen adversaries, robustness to unforeseen corruptions, transfer learning, and ablation studies. Finally, we present a new measurement to test the balance between robustness and natural accuracy, which we named $F_1$-robust score. 


\subsection{A case study on SVHN and CIFAR-100}
In the first part of our analysis, we conduct a case study experiment on two benchmark datasets: SVHN \citep{netzer2011reading} and CIFAR-100 \cite{krizhevsky2009learning}. We follow common experiment settings as in \cite{rice2020overfitting, wu2020adversarial}. We used the PreAct ResNet-18 \citep{he2016identity} architecture on which we integrate a domain classification layer. The adversarial training is done using 10-step PGD adversary with perturbation size of 0.031 and a step size of 0.003 for SVHN and 0.007 for CIFAR-100. The batch size is 128, weight decay is $7e^{-4}$ and the model is trained for 100 epochs. For SVHN, the initial learinnig rate is set to 0.01 and decays by a factor of 10 after 55, 75 and 90 iteration. For CIFAR-100, the initial learning rate is set to 0.1 and decays by a factor of 10 after 75 and 90 iterations. 
%We compared DIAL to \cite{madry2017towards} and TRADES \citep{zhang2019theoretically}. 
%The evaluation is done using Auto-Attack~\citep{croce2020reliable}, which is an ensemble of three white-box and one black-box parameter-free attacks, and various $\ell_{\infty}$ adversaries: PGD$^{20}$, PGD$^{100}$, PGD$^{1000}$ and CW$_{\infty}$ with step size of 0.003. 
Results are averaged over 3 restarts while omitting one standard deviation (which is smaller than 0.2\% in all experiments). As can be seen by the results in Tables~\ref{black-and_white-svhn} and \ref{black-and_white-cifar100}, DIAL presents consistent improvement in robustness (e.g., 5.75\% improved robustness on SVHN against AA) compared to the standard AT 
%under variety of attacks 
while also improving the natural accuracy. More results are presented in Appendix \ref{cifar100-svhn-appendix}.


\begin{table}[!ht]
  \caption{Robustness against white-box, black-box attacks and Auto-Attack (AA) on SVHN. Black-box attacks are generated using naturally trained surrogate model. Natural represents the naturally trained (non-adversarial) model.
  %and applied to the best performing robust models.
  }
  \vskip 0.1in
  \label{black-and_white-svhn}
  \centering
  \small
  \begin{tabular}{l@{\hspace{1\tabcolsep}}c@{\hspace{1\tabcolsep}}c@{\hspace{1\tabcolsep}}c@{\hspace{1\tabcolsep}}c@{\hspace{1\tabcolsep}}c@{\hspace{1\tabcolsep}}c@{\hspace{1\tabcolsep}}c@{\hspace{1\tabcolsep}}c@{\hspace{1\tabcolsep}}c@{\hspace{1\tabcolsep}}c}
    \toprule
    & & \multicolumn{4}{c}{White-box} & \multicolumn{4}{c}{Black-Box}  \\
    \cmidrule(r){3-6} 
    \cmidrule(r){7-10}
    Defense Model & Natural & PGD$^{20}$ & PGD$^{100}$  & PGD$^{1000}$  & CW$^{\infty}$ & PGD$^{20}$ & PGD$^{100}$ & PGD$^{1000}$  & CW$^{\infty}$ & AA \\
    \midrule
    NATURAL & 96.85 & 0 & 0 & 0 & 0 & 0 & 0 & 0 & 0 & 0 \\
    \midrule
    AT & 89.90 & 53.23 & 49.45 & 49.23 & 48.25 & 86.44 & 86.28 & 86.18 & 86.42 & 45.25 \\
    % TRADES & 90.35 & 57.10 & 54.13 & 54.08 & 52.19 & 86.89 & 86.73 & 86.57 & 86.70 &  49.50 \\
    $\DIAL_{\kl}$ (Ours) & 90.66 & \textbf{58.91} & \textbf{55.30} & \textbf{55.11} & \textbf{53.67} & 87.62 & 87.52 & 87.41 & 87.63 & \textbf{51.00} \\
    $\DIAL_{\ce}$ (Ours) & \textbf{92.88} & 55.26  & 50.82 & 50.54 & 49.66 & \textbf{89.12} & \textbf{89.01} & \textbf{88.74} & \textbf{89.10} &  46.52  \\
    \bottomrule
  \end{tabular}
\end{table}


\begin{table}[!ht]
  \caption{Robustness against white-box, black-box attacks and Auto-Attack (AA) on CIFAR100. Black-box attacks are generated using naturally trained surrogate model. Natural represents the naturally trained (non-adversarial) model.
  %and applied to the best performing robust models.
  }
  \vskip 0.1in
  \label{black-and_white-cifar100}
  \centering
  \small
  \begin{tabular}{l@{\hspace{1\tabcolsep}}c@{\hspace{1\tabcolsep}}c@{\hspace{1\tabcolsep}}c@{\hspace{1\tabcolsep}}c@{\hspace{1\tabcolsep}}c@{\hspace{1\tabcolsep}}c@{\hspace{1\tabcolsep}}c@{\hspace{1\tabcolsep}}c@{\hspace{1\tabcolsep}}c@{\hspace{1\tabcolsep}}c}
    \toprule
    & & \multicolumn{4}{c}{White-box} & \multicolumn{4}{c}{Black-Box}  \\
    \cmidrule(r){3-6} 
    \cmidrule(r){7-10}
    Defense Model & Natural & PGD$^{20}$ & PGD$^{100}$  & PGD$^{1000}$  & CW$^{\infty}$ & PGD$^{20}$ & PGD$^{100}$ & PGD$^{1000}$  & CW$^{\infty}$ & AA \\
    \midrule
    NATURAL & 79.30 & 0 & 0 & 0 & 0 & 0 & 0 & 0 & 0 & 0 \\
    \midrule
    AT & 56.73 & 29.57 & 28.45 & 28.39 & 26.6 & 55.52 & 55.29 & 55.26 & 55.40 & 24.12 \\
    % TRADES & 58.24 & 30.10 & 29.66 & 29.64 & 25.97 & 57.05 & 56.71 & 56.67 & 56.77 & 24.92 \\
    $\DIAL_{\kl}$ (Ours) & 58.47 & \textbf{31.19} & \textbf{30.50} & \textbf{30.42} & \textbf{26.91} & 57.16 & 56.81 & 56.80 & 57.00 & \textbf{25.87} \\
    $\DIAL_{\ce}$ (Ours) & \textbf{60.77} & 27.87 & 26.66 & 26.61 & 25.98 & \textbf{59.48} & \textbf{59.06} & \textbf{58.96} & \textbf{59.20} & 23.51  \\
    \bottomrule
  \end{tabular}
\end{table}


% \begin{table}[!ht]
%   \caption{Robustness comparison of DIAL to Madry et al. and TRADES defense models on the SVHN dataset under different PGD white-box attacks and the ensemble Auto-Attack (AA).}
%   \label{svhn}
%   \centering
%   \begin{tabular}{llllll|l}
%     \toprule
%     \cmidrule(r){1-5}
%     Defense Model & Natural & PGD$^{20}$ & PGD$^{100}$ & PGD$^{1000}$ & CW$_{\infty}$ & AA\\
%     \midrule
%     $\DIAL_{\kl}$ (Ours) & $\mathbf{90.66}$ & $\mathbf{58.91}$ & $\mathbf{55.30}$ & $\mathbf{55.12}$ & $\mathbf{53.67}$  & $\mathbf{51.00}$  \\
%     Madry et al. & 89.90 & 53.23 & 49.45 & 49.23 & 48.25 & 45.25  \\
%     TRADES & 90.35 & 57.10 & 54.13 & 54.08 & 52.19 & 49.50 \\
%     \bottomrule
%   \end{tabular}
% \end{table}


\subsection{Performance comparison on CIFAR-10} \label{defence-settings}
In this part, we evaluate the performance of DIAL compared to other well-known methods on CIFAR-10. 
%To be consistent with other methods, 
We follow the same experiment setups as in~\cite{madry2017towards, wang2019improving, zhang2019theoretically}. When experiment settings are not identical between tested methods, we choose the most commonly used settings, and apply it to all experiments. This way, we keep the comparison as fair as possible and avoid reporting changes in results which are caused by inconsistent experiment settings \citep{pang2020bag}. To show that our results are not caused because of what is referred to as \textit{obfuscated gradients}~\citep{athalye2018obfuscated}, we evaluate our method with same setup as in our defense model, under strong attacks (e.g., PGD$^{1000}$) in both white-box, black-box settings, Auto-Attack ~\citep{croce2020reliable}, unforeseen "natural" corruptions~\citep{hendrycks2018benchmarking}, and unforeseen adversaries. To make sure that the reported improvements are not caused by \textit{adversarial overfitting}~\citep{rice2020overfitting}, we report best robust results for each method on average of 3 restarts, while omitting one standard deviation (which is smaller than 0.2\% in all experiments). Additional results for CIFAR-10 as well as comprehensive evaluation on MNIST can be found in Appendix \ref{mnist-results} and \ref{additional_res}.
%To further keep the comparison consistent, we followed the same attack settings for all methods.


\begin{table}[ht]
  \caption{Robustness against white-box, black-box attacks and Auto-Attack (AA) on CIFAR-10. Black-box attacks are generated using naturally trained surrogate model. Natural represents the naturally trained (non-adversarial) model.
  %and applied to the best performing robust models.
  }
  \vskip 0.1in
  \label{black-and_white-cifar}
  \centering
  \small
  \begin{tabular}{cccccccc@{\hspace{1\tabcolsep}}c}
    \toprule
    & & \multicolumn{3}{c}{White-box} & \multicolumn{3}{c}{Black-Box} \\
    \cmidrule(r){3-5} 
    \cmidrule(r){6-8}
    Defense Model & Natural & PGD$^{20}$ & PGD$^{100}$ & CW$^{\infty}$ & PGD$^{20}$ & PGD$^{100}$ & CW$^{\infty}$ & AA \\
    \midrule
    NATURAL & 95.43 & 0 & 0 & 0 & 0 & 0 & 0 &  0 \\
    \midrule
    TRADES & 84.92 & 56.60 & 55.56 & 54.20 & 84.08 & 83.89 & 83.91 &  53.08 \\
    MART & 83.62 & 58.12 & 56.48 & 53.09 & 82.82 & 82.52 & 82.80 & 51.10 \\
    AT & 85.10 & 56.28 & 54.46 & 53.99 & 84.22 & 84.14 & 83.92 & 51.52 \\
    ATDA & 76.91 & 43.27 & 41.13 & 41.01 & 75.59 & 75.37 & 75.35 & 40.08\\
    $\DIAL_{\kl}$ (Ours) & 85.25 & $\mathbf{58.43}$ & $\mathbf{56.80}$ & $\mathbf{55.00}$ & 84.30 & 84.18 & 84.05 & \textbf{53.75} \\
    $\DIAL_{\ce}$ (Ours)  & $\mathbf{89.59}$ & 54.31 & 51.67 & 52.04 &$ \mathbf{88.60}$ & $\mathbf{88.39}$ & $\mathbf{88.44}$ & 49.85 \\
    \midrule
    $\DIAL_{\awp}$ (Ours) & $\mathbf{85.91}$ & $\mathbf{61.10}$ & $\mathbf{59.86}$ & $\mathbf{57.67}$ & $\mathbf{85.13}$ & $\mathbf{84.93}$ & $\mathbf{85.03}$  & \textbf{56.78} \\
    $\TRADES_{\awp}$ & 85.36 & 59.27 & 59.12 & 57.07 & 84.58 & 84.58 & 84.59 & 56.17 \\
    \bottomrule
  \end{tabular}
\end{table}



\paragraph{CIFAR-10 setup.} We use the wide residual network (WRN-34-10)~\citep{zagoruyko2016wide} architecture. %used in the experiments of~\cite{madry2017towards, wang2019improving, zhang2019theoretically}. 
Sidelong this architecture, we integrate a domain classification layer. To generate the adversarial domain dataset, we use a perturbation size of $\epsilon=0.031$. We apply 10 of inner maximization iterations with perturbation step size of 0.007. Batch size is set to 128, weight decay is set to $7e^{-4}$, and the model is trained for 100 epochs. Similar to the other methods, the initial learning rate was set to 0.1, and decays by a factor of 10 at iterations 75 and 90. 
%For being consistent with other methods, the natural images are padded with 4-pixel padding with 32-random crop and random horizontal flip. Furthermore, all methods are trained using SGD with momentum 0.9. For $\DIAL_{\kl}$, we balance the robust loss with $\lambda=6$ and the domains loss with $r=4$. For $\DIAL_{\ce}$, we balance the robust loss with $\lambda=1$ and the domains loss with $r=2$. 
%We also introduce a version of our method that incorporates the AWP double-perturbation mechanism, named DIAL-AWP.
%which is trained using the same learning rate schedule used in ~\cite{wu2020adversarial}, where the initial 0.1 learning rate decays by a factor of 10 after 100 and 150 iterations. 
See Appendix \ref{cifar10-additional-setup} for additional details.

\begin{table}[ht]
  \caption{Black-box attack using the adversarially trained surrogate models on CIFAR-10.}
  \vskip 0.1in
  \label{black-box-cifar-adv}
  \centering
  \small
  \begin{tabular}{ll|c}
    \toprule
    \cmidrule(r){1-2}
    Surrogate (source) model & Target model & robustness \% \\
    % \midrule
    \midrule
    TRADES & $\DIAL_{\ce}$ & $\mathbf{67.77}$ \\
    $\DIAL_{\ce}$ & TRADES & 65.75 \\
    \midrule
    MART & $\DIAL_{\ce}$ & $\mathbf{70.30}$ \\
    $\DIAL_{\ce}$ & MART & 64.91 \\
    \midrule
    AT & $\DIAL_{\ce}$ & $\mathbf{65.32}$ \\
    $\DIAL_{\ce}$ & AT  & 63.54 \\
    \midrule
    ATDA & $\DIAL_{\ce}$ & $\mathbf{66.77}$ \\
    $\DIAL_{\ce}$ & ATDA & 52.56 \\
    \bottomrule
  \end{tabular}
\end{table}

\paragraph{White-box/Black-box robustness.} 
%We evaluate all defense models using Auto-Attack, PGD$^{20}$, PGD$^{100}$, PGD$^{1000}$ and CW$_{\infty}$ with step size 0.003. We constrain all attacks by the same perturbation $\epsilon=0.031$. 
As reported in Table~\ref{black-and_white-cifar} and Appendix~\ref{additional_res}, our method achieves better robustness compared to the other methods. Specifically, in the white-box settings, our method improves robustness over~\citet{madry2017towards} and TRADES by 2\% 
%using the common PGD$^{20}$ attack 
while keeping higher natural accuracy. We also observe better natural accuracy of 1.65\% over MART while also achieving better robustness over all attacks. Moreover, our method presents significant improvement of up to 15\% compared to the the domain invariant method suggested by~\citet{song2018improving} (ATDA).
%in both natural and robust accuracy. 
When incorporating 
%the double-perturbation mechanism of 
AWP, our method improves the results of $\TRADES_{\awp}$ by almost 2\%.
%and reaches state-of-the-art results for robust models with no additional data. 
% Additional results are available in Appendix~\ref{additional_res}.
When tested on black-box settings, $\DIAL_{\ce}$ presents a significant improvement of more than 4.4\% over the second-best performing method, and up to 13\%. In Table~\ref{black-box-cifar-adv}, we also present the black-box results when the source model is taken from one of the adversarially trained models. %Then, we compare our model to each one of them both as the source model and target model. 
In addition to the improvement in black-box robustness, $\DIAL_{\ce}$ also manages to achieve better clean accuracy of more than 4.5\% over the second-best performing method.
% Moreover, based on the auto-attack leader-board \footnote{\url{https://github.com/fra31/auto-attack}}, our method achieves the 1st place among models without additional data using the WRN-34-10 architecture.

% \begin{table}
%   \caption{White-box robustness on CIFAR-10 using WRN-34-10}
%   \label{white-box-cifar-10}
%   \centering
%   \begin{tabular}{lllll}
%     \toprule
%     \cmidrule(r){1-2}
%     Defense Model & Natural & PGD$^{20}$ & PGD$^{100}$ & PGD$^{1000}$ \\
%     \midrule
%     TRADES ~\cite{zhang2019theoretically} & 84.92  & 56.6 & 55.56 & 56.43  \\
%     MART ~\cite{wang2019improving} & 83.62  & 58.12 & 56.48 & 56.55  \\
%     Madry et al. ~\cite{madry2017towards} & 85.1  & 56.28 & 54.46 & 54.4  \\
%     Song et al. ~\cite{song2018improving} & 76.91 & 43.27 & 41.13 & 41.02  \\
%     $\DIAL_{\ce}$ (Ours) & $ \mathbf{90}$  & 52.12 & 48.88 & 48.78  \\
%     $\DIAL_{\kl}$ (Ours) & 85.25 & $\mathbf{58.43}$ & $\mathbf{56.8}$ & $\mathbf{56.73}$ \\
%     \midrule
%     $\DIAL_{\kl}$+AWP (Ours) & $\mathbf{85.91}$ & $\mathbf{61.1}$ & - & -  \\
%     TRADES+AWP \cite{wu2020adversarial} & 85.36 & 59.27 & 59.12 & -  \\
%     % MART + AWP & 84.43 & 60.68 & 59.32 & -  \\
%     \bottomrule
%   \end{tabular}
% \end{table}


% \begin{table}
%   \caption{White-box robustness on MNIST}
%   \label{white-box-mnist}
%   \centering
%   \begin{tabular}{llllll}
%     \toprule
%     \cmidrule(r){1-2}
%     Defense Model & Natural & PGD$^{40}$ & PGD$^{100}$ & PGD$^{1000}$ \\
%     \midrule
%     TRADES ~\cite{zhang2019theoretically} & 99.48 & 96.07 & 95.52 & 95.22 \\
%     MART ~\cite{wang2019improving} & 99.38  & 96.99 & 96.11 & 95.74  \\
%     Madry et al. ~\cite{madry2017towards} & 99.41  & 96.01 & 95.49 & 95.36 \\
%     Song et al. ~\cite{song2018improving}  & 98.72 & 96.82 & 96.26 & 96.2  \\
%     $\DIAL_{\kl}$ (Ours) & 99.46 & 97.05 & 96.06 & 95.99  \\
%     $\DIAL_{\ce}$ (Ours) & $\mathbf{99.49}$  & $\mathbf{97.38}$ & $\mathbf{96.45}$ & $\mathbf{96.33}$ \\
%     \bottomrule
%   \end{tabular}
% \end{table}


% \paragraph{Attacking MNIST.} For consistency, we use the same perturbation and step sizes. For MNIST, we use $\epsilon=0.3$ and step size of $0.01$. The natural accuracy of our surrogate (source) model is 99.51\%. Attacks results are reported in Table~\ref{black-and_white-mnist}. It is worth noting that the improvement margin is not conclusive on MNIST as it is on CIFAR-10, which is a more complex task.

% \begin{table}
%   \caption{Black-box robustness on MNIST and CIFAR-10 using naturally trained surrogate model and best performing robust models}
%   \label{black-box-mnist-cifar}
%   \centering
%   \begin{tabular}{lllllll}
%     \toprule
%     & \multicolumn{3}{c}{MNIST} & \multicolumn{3}{c}{CIFAR-10} \\
%     \cmidrule(r){2-4} 
%     \cmidrule(r){5-7}  
%     Defense Model & PGD$^{40}$ & PGD$^{100}$ & PGD$^{1000}$ & PGD$^{20}$ & PGD$^{100}$ & PGD$^{1000}$ \\
%     \midrule
%     TRADES ~\cite{zhang2019theoretically} & 98.12 & 97.86 & 97.81 & 84.08 & 83.89 & 83.8 \\
%     MART ~\cite{wang2019improving} & 98.16 & 97.96 & 97.89  & 82.82 & 82.52 & 82.47 \\
%     Madry et al. ~\cite{madry2017towards}  & 98.05 & 97.73 & 97.78 & 84.22 & 84.14 & 83.96 \\
%     Song et al. ~\cite{song2018improving} & 97.74 & 97.28 & 97.34 & 75.59 & 75.37 & 75.11 \\
%     $\DIAL_{\kl}$ (Ours) & 98.14 & 97.83 & 97.87  & 84.3 & 84.18 & 84.0 \\
%     $\DIAL_{\ce}$ (Ours)  & $\mathbf{98.37}$ & $\mathbf{98.12}$ & $\mathbf{98.05}$  & $\mathbf{89.13}$ & $\mathbf{88.89}$ & $\mathbf{88.78}$ \\
%     \bottomrule
%   \end{tabular}
% \end{table}



% \subsubsection{Ensemble attack} In addition to the white-box and black-box settings, we evaluate our method on the Auto-Attack ~\citep{croce2020reliable} using $\ell_{\infty}$ threat model with perturbation $\epsilon=0.031$. Auto-Attack is an ensemble of parameter-free attacks. It consists of three white-box attacks: APGD-CE which is a step size-free version of PGD on the cross-entropy ~\citep{croce2020reliable}. APGD-DLR which is a step size-free version of PGD on the DLR loss ~\citep{croce2020reliable} and FAB which  minimizes the norm of the adversarial perturbations, and one black-box attack: square attack which is a query-efficient black-box attack ~\citep{andriushchenko2020square}. Results are presented in Table~\ref{auto-attack}. Based on the auto-attack leader-board \footnote{\url{https://github.com/fra31/auto-attack}}, our method achieves the 1st place among models without additional data using the WRN-34-10 architecture.

%Additional results can be found in Appendix ~\ref{additional_res}.

% \begin{table}
%   \caption{Auto-Attack (AA) on CIFAR-10 with perturbation size $\epsilon=0.031$ with $\ell_{\infty}$ threat model}
%   \label{auto-attack}
%   \centering
%   \begin{tabular}{lll}
%     \toprule
%     \cmidrule(r){1-2}
%     Defense Model & AA \\
%     \midrule
%     TRADES ~\cite{zhang2019theoretically} & 53.08  \\
%     MART ~\cite{wang2019improving} & 51.1  \\
%     Madry et al. ~\cite{madry2017towards} & 51.52    \\
%     Song et al.   ~\cite{song2018improving} & 40.18 \\
%     $\DIAL_{\ce}$ (Ours) & 47.33  \\
%     $\DIAL_{\kl}$ (Ours) & $\mathbf{53.75}$ \\
%     \midrule
%     DIAL-AWP (Ours) & $\mathbf{56.78}$ \\
%     TRADES-AWP \cite{wu2020adversarial} & 56.17 \\
%     \bottomrule
%   \end{tabular}
% \end{table}


% \begin{table}[!ht]
%   \caption{Auto-Attack (AA) Robustness (\%) on CIFAR-10 with $\epsilon=0.031$ using an $\ell_{\infty}$ threat model}
%   \label{auto-attack}
%   \centering
%   \begin{tabular}{cccccc|cc}
%     \toprule
%     % \multicolumn{8}{c}{Defence Model}  \\
%     % \cmidrule(r){1-8} 
%     TRADES & MART & Madry & Song & $\DIAL_{\ce}$ & $\DIAL_{\kl}$ & DIAL-AWP  & TRADES-AWP\\
%     \midrule
%     53.08 & 51.10 & 51.52 &  40.08 & 47.33  & $\mathbf{53.75}$ & $\mathbf{56.78}$ & 56.17 \\

%     \bottomrule
%   \end{tabular}
% \end{table}

% \begin{table}[!ht]
% \caption{$F_1$-robust measurement using PGD$^{20}$ attack in white-box and black-box settings on CIFAR-10}
%   \label{f1-robust}
%   \centering
%   \begin{tabular}{ccccccc|cc}
%     \toprule
%     % \multicolumn{8}{c}{Defence Model}  \\
%     % \cmidrule(r){1-8} 
%     Defense Model & TRADES & MART & Madry & Song & $\DIAL_{\kl}$ & $\DIAL_{\ce}$ & DIAL-AWP  & TRADES-AWP\\
%     \midrule
%     White-box & 0.659 & 0.666 & 0.657 & 0.518 & $\mathbf{0.675}$  & 0.643 & $\mathbf{0.698}$ & 0.682 \\
%     Black-box & 0.844 & 0.831 & 0.846 & 0.761 & 0.847 & $\mathbf{0.895}$ & $\mathbf{0.854}$ &  0.849 \\
%     \bottomrule
%   \end{tabular}
% \end{table}

\subsubsection{Robustness to Unforeseen Attacks and Corruptions}
\paragraph{Unforeseen Adversaries.} To further demonstrate the effectiveness of our approach, we test our method against various adversaries that were not used during the training process. We attack the model under the white-box settings with $\ell_{2}$-PGD, $\ell_{1}$-PGD, $\ell_{\infty}$-DeepFool and $\ell_{2}$-DeepFool \citep{moosavi2016deepfool} adversaries using Foolbox \citep{rauber2017foolbox}. We applied commonly used attack budget 
%(perturbation for PGD adversaries and overshot for DeepFool adversaries) 
with 20 and 50 iterations for PGD and DeepFool, respectively.
Results are presented in Table \ref{unseen-attacks}. As can be seen, our approach  gains an improvement of up to 4.73\% over the second best method under the various attack types and an average improvement of 3.7\% over all threat models.


\begin{table}[ht]
  \caption{Robustness on CIFAR-10 against unseen adversaries under white-box settings.}
  \vskip 0.1in
  \label{unseen-attacks}
  \centering
%   \small
  \begin{tabular}{c@{\hspace{1.5\tabcolsep}}c@{\hspace{1.5\tabcolsep}}c@{\hspace{1.5\tabcolsep}}c@{\hspace{1.5\tabcolsep}}c@{\hspace{1.5\tabcolsep}}c@{\hspace{1.5\tabcolsep}}c@{\hspace{1.5\tabcolsep}}c}
    \toprule
    Threat Model & Attack Constraints & $\DIAL_{\kl}$ & $\DIAL_{\ce}$ & AT & TRADES & MART & ATDA \\
    \midrule
    \multirow{2}{*}{$\ell_{2}$-PGD} & $\epsilon=0.5$ & 76.05 & \textbf{80.51} & 76.82 & 76.57 & 75.07 & 66.25 \\
    & $\epsilon=0.25$ & 80.98 & \textbf{85.38} & 81.41 & 81.10 & 80.04 & 71.87 \\\midrule
    \multirow{2}{*}{$\ell_{1}$-PGD} & $\epsilon=12$ & 74.84 & \textbf{80.00} & 76.17 & 75.52 & 75.95 & 65.76 \\
    & $\epsilon=7.84$ & 78.69 & \textbf{83.62} & 79.86 & 79.16 & 78.55 & 69.97 \\
    \midrule
    $\ell_{2}$-DeepFool & overshoot=0.02 & 84.53 & \textbf{88.88} & 84.15 & 84.23 & 82.96 & 76.08 \\\midrule
    $\ell_{\infty}$-DeepFool & overshoot=0.02 & 68.43 & \textbf{69.50} & 67.29 & 67.60 & 66.40 & 57.35 \\
    \bottomrule
  \end{tabular}
\end{table}


%%%%%%%%%%%%%%%%%%%%%%%%% conference version %%%%%%%%%%%%%%%%%%%%%%%%%%%%%%%%%%%%%
\paragraph{Unforeseen Corruptions.}
We further demonstrate that our method consistently holds against unforeseen ``natural'' corruptions, consists of 18 unforeseen diverse corruption types proposed by \citet{hendrycks2018benchmarking} on CIFAR-10, which we refer to as CIFAR10-C. The CIFAR10-C benchmark covers noise, blur, weather, and digital categories. As can be shown in Figure \ref{corruption}, our method gains a significant and consistent improvement over all the other methods. Our method leads to an average improvement of 4.7\% with minimum improvement of 3.5\% and maximum improvement of 5.9\% compared to the second best method over all unforeseen attacks. See Appendix \ref{corruptions-apendix} for the full experiment results.


\begin{figure}[h]
 \centering
  \includegraphics[width=0.4\textwidth]{figures/spider_full.png}
%   \caption{Summary of accuracy over all unforeseen corruptions compared to the second and third best performing methods.}
  \caption{Accuracy comparison over all unforeseen corruptions.}
  \label{corruption}
\end{figure}


%%%%%%%%%%%%%%%%%%%%%%%%% conference version %%%%%%%%%%%%%%%%%%%%%%%%%%%%%%%%%%%%%

%%%%%%%%%%%%%%%%%%%%%%%%% Arxiv version %%%%%%%%%%%%%%%%%%%%%%%%%%%%%%%%%%%%%
% \newpage
% \paragraph{Unforeseen Corruptions.}
% We further demonstrate that our method consistently holds against unforeseen "natural" corruptions, consists of 18 unforeseen diverse corruption types proposed by \cite{hendrycks2018benchmarking} on CIFAR-10, which we refer to as CIFAR10-C. The CIFAR10-C benchmark covers noise, blur, weather, and digital categories. As can be shown in Figure  \ref{spider-full-graph}, our method gains a significant and consistent improvement over all the other methods. Our approach leads to an average improvement of 4.7\% with minimum improvement of 3.5\% and maximum improvement of 5.9\% compared to the second best method over all unforeseen attacks. Full accuracy results against unforeseen corruptions are presented in Tables \ref{corruption-table1} and \ref{corruption-table2}. 

% \begin{table}[!ht]
%   \caption{Accuracy (\%) against unforeseen corruptions.}
%   \label{corruption-table1}
%   \centering
%   \tiny
%   \begin{tabular}{lcccccccccccccccccc}
%     \toprule
%     Defense Model & brightness & defocus blur & fog & glass blur & jpeg compression & motion blur & saturate & snow & speckle noise  \\
%     \midrule
%     TRADES & 82.63 & 80.04 & 60.19 & 78.00 & 82.81 & 76.49 & 81.53 & 80.68 & 80.14 \\
%     MART & 80.76 & 78.62 & 56.78 & 76.60 & 81.26 & 74.58 & 80.74 & 78.22 & 79.42 \\
%     AT &  83.30 & 80.42 & 60.22 & 77.90 & 82.73 & 76.64 & 82.31 & 80.37 & 80.74 \\
%     ATDA & 72.67 & 69.36 & 45.52 & 64.88 & 73.22 & 63.47 & 72.07 & 68.76 & 72.27 \\
%     DIAL (Ours)  & \textbf{87.14} & \textbf{84.84} & \textbf{66.08} & \textbf{81.82} & \textbf{87.07} & \textbf{81.20} & \textbf{86.45} & \textbf{84.18} & \textbf{84.94} \\
%     \bottomrule
%   \end{tabular}
% \end{table}


% \begin{table}[!ht]
%   \caption{Accuracy (\%) against unforeseen corruptions.}
%   \label{corruption-table2}
%   \centering
%   \tiny
%   \begin{tabular}{lcccccccccccccccccc}
%     \toprule
%     Defense Model & contrast & elastic transform & frost & gaussian noise & impulse noise & pixelate & shot noise & spatter & zoom blur \\
%     \midrule
%     TRADES & 43.11 & 79.11 & 76.45 & 79.21 & 73.72 & 82.73 & 80.42 & 80.72 & 78.97 \\
%     MART & 41.22 & 77.77 & 73.07 & 78.30 & 74.97 & 81.31 & 79.53 & 79.28 & 77.8 \\
%     AT & 43.30 & 79.58 & 77.53 & 79.47 & 73.76 & 82.78 & 80.86 & 80.49 & 79.58 \\
%     ATDA & 36.06 & 67.06 & 62.56 & 70.33 & 64.63 & 73.46 & 72.28 & 70.50 & 67.31 \\
%     DIAL (Ours) & \textbf{48.84} & \textbf{84.13} & \textbf{81.76} & \textbf{83.76} & \textbf{78.26} & \textbf{87.24} & \textbf{85.13} & \textbf{84.84} & \textbf{83.93}  \\
%     \bottomrule
%   \end{tabular}
% \end{table}


% \begin{figure}[!ht]
%   \centering
%   \includegraphics[width=9cm]{figures/spider_full.png}
%   \caption{Accuracy comparison with all tested methods over unforeseen corruptions.}
%   \label{spider-full-graph}
% \end{figure}
% %%%%%%%%%%%%%%%%%%%%%%%%% Arxiv version %%%%%%%%%%%%%%%%%%%%%%%%%%%%%%%%%%%%%
%%%%%%%%%%%%%%%%%%%%%%%%% Arxiv version %%%%%%%%%%%%%%%%%%%%%%%%%%%%%%%%%%%%%

\subsubsection{Transfer Learning}
Recent works \citep{salman2020adversarially,utrera2020adversarially} suggested that robust models transfer better on standard downstream classification tasks. In Table \ref{transfer-res} we demonstrate the advantage of our method when applied for transfer learning across CIFAR10 and CIFAR100 using the common linear evaluation protocol. see Appendix \ref{transfer-learning-settings} for detailed settings.

\begin{table}[H]
  \caption{Transfer learning results comparison.}
  \vskip 0.1in
  \label{transfer-res}
  \centering
  \small
\begin{tabular}{c|c|c|c}
\toprule

\multicolumn{2}{l}{} & \multicolumn{2}{c}{Target} \\
\cmidrule(r){3-4}
Source & Defence Model & CIFAR10 & CIFAR100 \\
\midrule
\multirow{3}{*}{CIFAR10} & DIAL & \multirow{3}{*}{-} & \textbf{28.57} \\
 & AT &  & 26.95  \\
 & TRADES &  & 25.40  \\
 \midrule
\multirow{3}{*}{CIFAR100} & DIAL & \textbf{73.68} & \multirow{3}{*}{-} \\
 & AT & 71.41 & \\
 & TRADES & 71.42 &  \\
%  \midrule
% \multirow{3}{}{SVHN} & DIAL &  &  & \multirow{3}{}{-} \\
%  & Madry et al. &  &  &  \\
%  & TRADES &  &  &  \\ 
\bottomrule
\end{tabular}
\end{table}


\subsubsection{Modularity and Ablation Studies}

We note that the domain classifier is a modular component that can be integrated into existing models for further improvements. Removing the domain head and related loss components from the different DIAL formulations results in some common adversarial training techniques. For $\DIAL_{\kl}$, removing the domain and related loss components results in the formulation of TRADES. For $\DIAL_{\ce}$, removing the domain and related loss components results in the original formulation of the standard adversarial training, and for $\DIAL_{\awp}$ the removal results in $\TRADES_{\awp}$. Therefore, the ablation studies will demonstrate the effectiveness of combining DIAL on top of different adversarial training methods. 

We investigate the contribution of the additional domain head component introduced in our method. Experiment configuration are as in \ref{defence-settings}, and robust accuracy is based on white-box PGD$^{20}$ on CIFAR-10 dataset. We remove the domain head from both $\DIAL_{\kl}$, $\DIAL_{\awp}$, and $\DIAL_{\ce}$ (equivalent to $r=0$) and report the natural and robust accuracy. We perform 3 random restarts and omit one standard deviation from the results. Results are presented in Figure \ref{ablation}. All DIAL variants exhibits stable improvements on both natural accuracy and robust accuracy. $\DIAL_{\ce}$, $\DIAL_{\kl}$, and $\DIAL_{\awp}$ present an improvement of 1.82\%, 0.33\%, and 0.55\% on natural accuracy and an improvement of 2.5\%, 1.87\%, and 0.83\% on robust accuracy, respectively. This evaluation empirically demonstrates the benefits of incorporating DIAL on top of different adversarial training techniques.
% \paragraph{semi-supervised extensions.} Since the domain classifier does not require the class labels, we argue that additional unlabeled data can be leveraged in future work.
%for improved results. 

\begin{figure}[ht]
  \centering
  \includegraphics[width=0.35\textwidth]{figures/ablation_graphs3.png}
  \caption{Ablation studies for $\DIAL_{\kl}$, $\DIAL_{\ce}$, and $\DIAL_{\awp}$ on CIFAR-10. Circle represent the robust-natural accuracy without using DIAL, and square represent the robust-natural accuracy when incorporating DIAL.
  %to further investigate the impact of the domain head and loss on natural and robust accuracy.
  }
  \label{ablation}
\end{figure}

\subsubsection{Visualizing DIAL}
To further illustrate the superiority of our method, we visualize the model outputs from the different methods on both natural and adversarial test data.
% adversarial test data generated using PGD$^{20}$ white-box attack with step size 0.003 and $\epsilon=0.031$ on CIFAR-10. 
Figure~\ref{tsne1} shows the embedding received after applying t-SNE ~\citep{van2008visualizing} with two components on the model output for our method and for TRADES. DIAL seems to preserve strong separation between classes on both natural test data and adversarial test data. Additional illustrations for the other methods are attached in Appendix~\ref{additional_viz}. 

\begin{figure}[h]
\centering
  \subfigure[\textbf{DIAL} on natural logits]{\includegraphics[width=0.21\textwidth]{figures/domain_ce_test.png}}
  \hspace{0.03\textwidth}
  \subfigure[\textbf{DIAL} on adversarial logits]{\includegraphics[width=0.21\textwidth]{figures/domain_ce_adversarial.png}}
  \hspace{0.03\textwidth}
    \subfigure[\textbf{TRADES} on natural logits]{\includegraphics[width=0.21\textwidth]{figures/trades_test.png}}
    \hspace{0.03\textwidth}
    \subfigure[\textbf{TRADES} on adversarial logits]{\includegraphics[width=0.21\textwidth]{figures/trades_adversarial.png}}
  \caption{t-SNE embedding of model output (logits) into two-dimensional space for DIAL and TRADES using the CIFAR-10 natural test data and the corresponding PGD$^{20}$ generated adversarial examples.}
  \label{tsne1}
\end{figure}


% \begin{figure}[ht]
% \centering
%   \begin{subfigure}{4cm}
%     \centering\includegraphics[width=3.3cm]{figures/domain_ce_test.png}
%     \caption{\textbf{DIAL} on nat. examples}
%   \end{subfigure}
%   \begin{subfigure}{4cm}
%     \centering\includegraphics[width=3.3cm]{figures/domain_ce_adversarial.png}
%     \caption{\textbf{DIAL} on adv. examples}
%   \end{subfigure}
  
%   \begin{subfigure}{4cm}
%     \centering\includegraphics[width=3.3cm]{figures/trades_test.png}
%     \caption{\textbf{TRADES} on nat. examples}
%   \end{subfigure}
%   \begin{subfigure}{4cm}
%     \centering\includegraphics[width=3.3cm]{figures/trades_adversarial.png}
%     \caption{\textbf{TRADES} on adv. examples}
%   \end{subfigure}
%   \caption{t-SNE embedding of model output (logits) into two-dimensional space for DIAL and TRADES using the CIFAR-10 natural test data and the corresponding adversarial examples.}
%   \label{tsne1}
% \end{figure}



% \begin{figure}[ht]
% \centering
%   \begin{subfigure}{6cm}
%     \centering\includegraphics[width=5cm]{figures/domain_ce_test.png}
%     \caption{\textbf{DIAL} on nat. examples}
%   \end{subfigure}
%   \begin{subfigure}{6cm}
%     \centering\includegraphics[width=5cm]{figures/domain_ce_adversarial.png}
%     \caption{\textbf{DIAL} on adv. examples}
%   \end{subfigure}
  
%   \begin{subfigure}{6cm}
%     \centering\includegraphics[width=5cm]{figures/trades_test.png}
%     \caption{\textbf{TRADES} on nat. examples}
%   \end{subfigure}
%   \begin{subfigure}{6cm}
%     \centering\includegraphics[width=5cm]{figures/trades_adversarial.png}
%     \caption{\textbf{TRADES} on adv. examples}
%   \end{subfigure}
%   \caption{t-SNE embedding of model output (logits) into two-dimensional space for DIAL and TRADES using the CIFAR-10 natural test data and the corresponding adversarial examples.}
%   \label{tsne1}
% \end{figure}



\subsection{Balanced measurement for robust-natural accuracy}
One of the goals of our method is to better balance between robust and natural accuracy under a given model. For a balanced metric, we adopt the idea of $F_1$-score, which is the harmonic mean between the precision and recall. However, rather than using precision and recall, we measure the $F_1$-score between robustness and natural accuracy,
using a measure we call
%We named it
the
\textbf{$\mathbf{F_1}$-robust} score.
\begin{equation}
F_1\text{-robust} = \dfrac{\text{true\_robust}}
{\text{true\_robust}+\frac{1}{2}
%\cdot
(\text{false\_{robust}}+
\text{false\_natural})},
\end{equation}
where $\text{true\_robust}$ are the adversarial examples that were correctly classified, $\text{false\_{robust}}$ are the adversarial examples that were miss-classified, and $\text{false\_natural}$ are the natural examples that were miss-classified.
%We tested the proposed $F_1$-robust score using PGD$^{20}$ on CIFAR-10 dataset in white-box and black-box settings. 
Results are presented in Table~\ref{f1-robust} and demonstrate that our method achieves the best $F_1$-robust score in both settings, which supports our findings from previous sections.

% \begin{table}[!ht]
%   \caption{$F_1$-robust measurement using PGD$^{20}$ attack in white and black box settings on CIFAR-10}
%   \label{f1-robust}
%   \centering
%   \begin{tabular}{lll}
%     \toprule
%     \cmidrule(r){1-2}
%     Defense Model & White-box & Black-box \\
%     \midrule
%     TRADES & 0.65937  & 0.84435 \\
%     MART & 0.66613  & 0.83153  \\
%     Madry et al. & 0.65755 & 0.84574   \\
%     Song et al. & 0.51823 & 0.76092  \\
%     $\DIAL_{\ce}$ (Ours) & 0.65318   & $\mathbf{0.88806}$  \\
%     $\DIAL_{\kl}$ (Ours) & $\mathbf{0.67479}$ & 0.84702 \\
%     \midrule
%     \midrule
%     DIAL-AWP (Ours) & $\mathbf{0.69753}$  & $\mathbf{0.85406}$  \\
%     TRADES-AWP & 0.68162 & 0.84917 \\
%     \bottomrule
%   \end{tabular}
% \end{table}

\begin{table}[ht]
\small
  \caption{$F_1$-robust measurement using PGD$^{20}$ attack in white and black box settings on CIFAR-10.}
  \vskip 0.1in
  \label{f1-robust}
  \centering
%   \small
  \begin{tabular}{c
  @{\hspace{1.5\tabcolsep}}c @{\hspace{1.5\tabcolsep}}c @{\hspace{1.5\tabcolsep}}c @{\hspace{1.5\tabcolsep}}c
  @{\hspace{1.5\tabcolsep}}c @{\hspace{1.5\tabcolsep}}c @{\hspace{1.5\tabcolsep}}|
  @{\hspace{1.5\tabcolsep}}c
  @{\hspace{1.5\tabcolsep}}c}
    \toprule
    % \cmidrule(r){8-9}
     & TRADES & MART & AT & ATDA & $\DIAL_{\ce}$ & $\DIAL_{\kl}$ & $\DIAL_{\awp}$ & $\TRADES_{\awp}$ \\
    \midrule
    White-box & 0.659 & 0.666 & 0.657 & 0.518 & 0.660 & \textbf{0.675} & \textbf{0.698} & 0.682 \\
    Black-box & 0.844 & 0.831 & 0.845 & 0.761 & \textbf{0.890} & 0.847 & \textbf{0.854} & 0.849 \\ 
    \bottomrule
  \end{tabular}
\end{table}

\section{Conclusion}
%We propose an effective means of transferring knowledge and representations between related vision-language tasks by formulating each tasks' model around a shared image-to-text embedding space (SVLR). By sharing knowledge across tasks in the form of alignments between visual features and text, the shared representation has the same interpretation across all tasks, allowing supervision in one task to directly benefit the other by means of improving the shared embedding space. 

Humans learn new skills by building upon existing knowledge and experiences. We attempt to apply this behavior to AI models by demonstrating cross-task learning for the class of vision-language problems using VQA and VR. To enhance inductive transfer, we propose sharing core vision and language representations across all tasks in a way that exploits the word-region alignment. We plan to extend our method to larger sets of vision-language tasks. 

%We designed an interpretable Vision-Language representation space that can be reused across multiple tasks. We show learning and application of the shared representation space through visual recognition and the VQA tasks. We also formulate the VQA task in terms of the recognition task which allows more efficient utilization of recognition supervision for VQA and weak supervision of visual recognition through VQA. As future work, we plan to extend the approach to other tasks such as activity recognition and image-caption retrieval. For VQA, we plan to model pairwise relations explicitly to answer questions involving spatial reasoning. 

\section{Acknowledgements}
This work is supported in part by NSF Awards 14-46765 and 10-53768 and ONR MURI N000014-16-1-2007.

{\small
\bibliographystyle{ieee}
\bibliography{egbib}
}

\end{document}
