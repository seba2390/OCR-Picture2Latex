\documentclass[10pt,twocolumn,letterpaper]{article}

\usepackage{iccv}
\usepackage{times}
\usepackage{epsfig}
\usepackage{graphicx}
\usepackage{amsmath}
\usepackage{amssymb}
\usepackage{bbm}
\usepackage{enumitem}
\usepackage{tabularx}
\usepackage{rotating}
\usepackage{xcolor}
\usepackage{subcaption}
\usepackage{multirow}
\usepackage{booktabs}
\usepackage{url}
\usepackage[font=small]{caption}
\newcommand\todo[1]{\textcolor{red}{#1}}
%\usepgfplotslibrary{external} 
%\tikzexternalize
\newcommand*\rot{\rotatebox{90}}
\newcommand\boldhead[1]{\vspace{0.03in}\noindent\textbf{#1: }}
% Include other packages here, before hyperref.

% Using baseline stretch. If possible get rid of it by making more space.
% -Saurabh
\renewcommand{\baselinestretch}{0.99} 

% If you comment hyperref and then uncomment it, you should delete
% egpaper.aux before re-running latex.  (Or just hit 'q' on the first latex
% run, let it finish, and you should be clear).
%\usepackage[pagebackref=true,breaklinks=true,letterpaper=true,colorlinks,bookmarks=false]{hyperref}

\iccvfinalcopy % *** Uncomment this line for the final submission

\def\iccvPaperID{1997} % *** Enter the ICCV Paper ID here
\def\httilde{\mbox{\tt\raisebox{-.5ex}{\symbol{126}}}}

% Pages are numbered in submission mode, and unnumbered in camera-ready
\ificcvfinal\pagestyle{empty}\fi
\begin{document}

%%%%%%%%% TITLE
\title{Aligned Image-Word Representations Improve Inductive Transfer Across Vision-Language Tasks}
%\title{Improving Inductive Transfer across Vision-Language Tasks through Aligned Image-Word Representations}
%\title{Learning Image-Word Representations from Complementary and Hierarchical Vision-Language Tasks}
% Keywords - Shared/rich/generalizable representation, multiple vision-language tasks, inductive transfer, intermediate supervision 
% Sharing Vision-Language representations across multiple tasks for inductive transfer   
% What does sharing Vision-Language representation across tasks give us?
% Learning image and language representations from diverse and complementary experience
% Learning image and language representations from diverse and complementary task hierarchy
%\author{First Author\\
%Institution1\\
%Institution1 address\\
%{\tt\small firstauthor@i1.org}
% For a paper whose authors are all at the same institution,
% omit the following lines up until the closing ``}''.
% Additional authors and addresses can be added with ``\and'',
% just like the second author.
% To save space, use either the email address or home page, not both
%\and
%Second Author\\
%Institution2\\
%First line of institution2 address\\
%{\tt\small secondauthor@i2.org}
%}
\author{Tanmay Gupta$^1$ \hspace{0,5cm} Kevin Shih$^1$ \hspace{0,5cm} Saurabh Singh$^2$ \hspace{0,5cm} Derek Hoiem$^1$\\
$^1$University of Illinois, Urbana-Champaign  \hspace{0,5cm} $^2$Google Inc.\\
{\tt\small \{tgupta6, kjshih2, dhoiem\}@illinois.edu} \hspace{0,5cm} {\tt\small saurabhsingh@google.com}}

\maketitle
%\thispagestyle{empty}

%%%%%%%%% ABSTRACT
\begin{abstract}
   An important goal of computer vision is to build systems that learn visual representations over time that can be applied to many tasks. In this paper, we investigate a vision-language embedding as a core representation and show that it leads to better cross-task transfer than standard multi-task learning. In particular, the task of visual recognition is aligned to the task of visual question answering by forcing each to use the same word-region embeddings. We show this leads to greater inductive transfer from recognition to VQA than standard multitask learning. Visual recognition also improves, especially for categories that have relatively few recognition training labels but appear often in the VQA setting.  Thus, our paper takes a small step towards creating more general vision systems by showing the benefit of interpretable, flexible, and trainable core representations.  
   
  
   %Applying the knowledge learned while solving one task to a different but related task is a challenging problem. We propose to address this challenge for vision-language problems by sharing a region and word representation module across models for different tasks. The domain specific models exploit the alignment between the image-region and word representations produced by this shared vision-language representation (SVLR) module. Using visual recognition and visual question answering as an example of related vision-language tasks, we show how to incorporate the SVLR module into models for these tasks. When trained jointly on all tasks, the SVLR module receives diverse but consistent supervision to produce rich representations, improving the performance of individual models through inductive transfer. Our formulation leads to interpretable predictions for VQA and allows for zero-shot VQA. 
   
   
   %Representation plays a crucial role in many vision-language problems such as visual recognition (VR), visual question answering (VQA), image captioning etc. Models for these tasks learn image and language features via end-to-end training. However, these models often ignore the complementary and hierarchical nature of supervision available across datasets for different tasks. For instance, while VR datasets (such as ImageNet) provide a direct mapping between uncluttered images and class label words, VQA datasets provide an alignment between real-world images and the complex semantics of a question-answer (QA) pair. 
   
   %To learn rich image-region and word representations that generalize across tasks, we propose a shared vision-language representation (SVLR) module that is shared across models for different tasks. Using VR and VQA as an example of a complementary task hierarchy, we show how to incorporate SVLR module into popular models for these tasks. We demonstrate that when trained jointly on all tasks, SVLR accumulates diverse supervision to produce rich representations that can boost performance of individual models through inductive transfer. Our formulation also leads to more interpretable predictions for VQA. 
\end{abstract}

%%%%%%%%% BODY TEXT
\vspace{-5mm}

\section{Introduction}

Reinforcement learning has achieved great success in areas such as Game-playing \citep{silver2018general,vinyals2019grandmaster}, robotics \cite{kober2013reinforcement}, large language models \citep{ouyang2022training}, etc.
However, due to safety concerns or physical limitations, in some real-world reinforcement learning problems, we must consider additional constraints that may influence the optimal policy and the learning process \citep{garcia2015comprehensive}.
% For example, a robotic arm must not take actions that may cause harm to itself or the environments.
A standard framework to handle such cases is the constrained Markov Decision Process (CMDP) \citep{altman1999constrained}.
Within the CMDP framework, the agent has to maximize
the expected cumulative reward while
obeying a finite number of constraints, which are usually in the form of expected cumulative cost criteria.

However, we are sometimes concerned with the problem with a continuum of constraints.
For example,
the constraints we meet might be time-evolving or subject to uncertain parameters, which
cannot be formulated as an ordinary CMDP
(see Examples \ref{Example_Time_Evolving} and  \ref{Example_Uncertain}).
In this paper we would study a generalized CMDP  
to address the above problem.  Because the constraints are not only infinite-number but also lie
in a continuous set,
the generalization is not trivial. Fortunately, we find that we can borrow the idea behind semi-infinite programming (SIP) \citep{remez1934determination, hettich1993semi} to deal with the semi-infinite constraints.
Accordingly, we propose \emph{semi-infinitely constrained Markov decision processes} (SICMDPs)
as a novel complement to the ordinary CMDP framework.
%More specifically,  an SICMDP model %, we consider 
%contains a continuum of constraints whereas an ordinary CMDP contains a finite number of constraints. 

%This generalization is natural but not trivial. However, we can brows the idea  
%The idea is quite natural and can be backtracked
%to the practice of extending linear programming to linear semi-infinite programming (LSIP) %\cite{remez1934determination, GobernaLSIO1998}.
%In addition, 
%As a complementary approach to the ordinary CMDP framework, 
%SICMDP can be used to model these problems  which cannot be described by a finite number of constraints
%that are not covered by .
%For example,
%the restrictions we consider can be time-evolving or subject to uncertain parameters
%, thus
%cannot be described by a finite number of constraints but a continuum of constraints 
%(see Examples \ref{Example_Time_Evolving} and  \ref{Example_Uncertain}).

We also present two reinforcement learning algorithms to solve SICMDPs called SI-CRL and SI-CPO, respectively.
SI-CRL is a model-based reinforcement learning algorithm designed for tabular cases, and SI-CPO is a policy optimization algorithm for non-tabular cases.
% and analyze its performance both theoretically and empirically.
The main challenge is that we need to deal with a continuum of constraints, thus reinforcement learning algorithms for ordinary CMDPs do not work anymore.
In SI-CRL, we tackle this difficulty by first transforming the reinforcement learning problem to an equivalent LSIP problem, which can then be solved using methods in the LSIP literature like the dual exchange methods \citep{Hu1990,reemtsen1998numerical}.
In SI-CPO, we resort to the idea of cooperative stochastic approximation developed in \cite{lan2020algorithms, wei2020comirror}.
As far as we know, we are the first to introduce tools from semi-infinitely programming (SIP) into the reinforcement learning community for solving constrained reinforcement learning problems.

% To the best of our knowledge, we are the first to apply tools from semi-infinitely programming (SIP) to solve reinforcement learning problems.
Furthermore, we give theoretical analysis for both SI-CRL and SI-CPO.
We decompose the error of SI-CRL into two parts: the statistical error from approximating the true SICMDP with an offline dataset and the optimization error due to the fact that the solution of the LSIP problem obtained by the dual exchange method is inexact.
On the optimization side, we show that the iteration complexity of SI-CRL is $O\left(\left\{\mathrm{diam}(Y)L\sqrt{|\gS|^2|\gA|m}/\left[(1-\gamma)\epsilon\right]\right\}^m\right)$.
On the statistical side, we show that the sample complexity of SI-CRL is $\widetilde O\left(\frac{|S|^2|A|^2}{\epsilon^2(1-\gamma)^3}\right)$ if the offline dataset is generated by a generative model, and $\widetilde O\left(\frac{|S||A|}{\nu_{\min} \epsilon^2(1-\gamma)^3}\right)$ if the dataset is generated by a probability measure $\nu$ as considered in \cite{chen2019information}.
Here $\widetilde O$ means that all logarithm terms are discarded.
For SI-CPO, things become a little more complicated because other than the statistical error and the optimization error, we also need to consider the function approximation error, which comes from imperfect policy parametrizations.
It is shown if the function approximation error can be controlled to $O(\epsilon)$ order, the iteration complexity of SI-CPO is $\widetilde{O}\left(\frac{1}{\epsilon^2(1-\gamma)^6}\right)$ and the sample complexity of SI-CPO is $\widetilde{O}(\frac{1}{\epsilon^4(1-\gamma)^{10}})$.
Here our iteration complexity bound is equivalent to a typical $\widetilde O(1/\sqrt{T})$ global convergence rate.

We perform a set of numerical experiments to illustrate the SICMDP model and validate our proposed algorithms.
Specifically, we examine two numerical examples, namely the discharge of sewage and ship route planning.
Through the discharge of sewage example, we show the advantage of the SICMDP framework over the CMDP baseline obtained by naive discretization in modeling realistic sequential decision-making problems.
Moreover, we demonstrate the effectiveness of the SI-CRL and SI-CPO algorithms in such tabular environments. 
In the ship route planning example, we illustrate the benefits of the SICMDP framework and the ability of the SI-CPO algorithm to address complex continuous control tasks involving continuous state spaces with modern deep reinforcement learning techniques.

% In summary, our contributions are listed as follows.
% First, we present the SICMDP model, which can be viewed as a generalization of the ordinary CMDP model.
% Second, we propose an algorithm to perform reinforcement learning for SICMDPs, which is called SI-CRL, and we believe that we are the first to apply tools from SIP
% to solve reinforcement learning problems.
% Third, we give a theoretical analysis of SI-CRL and identify both its sample complexity and iteration complexity.
% In addition, we perform numerical experiments to illustrate the SICMDP model and validate the SI-CRL algorithm.
% \{This paragraph can be removed!!! \}







\begin{figure*}[t]
\begin{center}
%\fbox{\rule{0pt}{2in} \rule{.6\linewidth}{0pt}}
\includegraphics[width=\linewidth]{images/joint_training.pdf}
\vspace{-0.9cm}
\end{center}
   \caption{\textbf{Joint Training on Visual Recognition(VR) and Visual Question Answering(VQA) with the proposed SVLR Module:} The figure depicts sharing of image and word representations through the SVLR module during joint training on object recognition, attribute recognition, and VQA. The recognition tasks use object and attribute labelled regions from Visual Genome while VQA uses images annotated with questions and answers from the VQA dataset. The benefit of joint training is that while the VQA dataset does not provide region groundings of nouns and adjectives in the QA (e.g. ``fluffy",``dog"), this complementary supervision is provided by the Genome recognition dataset. Models for each task involve image and word embeddings produced by SVLR module or their inner products (See Fig~\ref{fig:system} for VQA model architecture).
   }
   \vspace{-6mm}
\label{fig:features}
\end{figure*}

%  \caption{\textbf{Joint Training on Visual Recognition(VR) and Visual Question Answering(VQA) with the proposed SVLR Module:} The figure depicts sharing of image and word representations through the SVLR module during joint training on object recognition, attribute recognition, and VQA. The recognition tasks use object and attribute labelled regions from Visual Genome while VQA uses images annotated with questions and answers from the VQA dataset. The benefit of joint training is that while the VQA dataset does not provide region groundings of nouns and adjectives in the QA (eg. 'fluffy','dog'), this complementary supervision is provided by the Genome recognition dataset. Models for each task involve image and word embeddings produced by SVLR module or their inner products (See Fig~\ref{fig:system} for VQA model architecture). The parameters of the SVLR module and Answer Scoring module in VQA are trained using stochastic gradient descent. As a baseline, we replace $g(y;\theta_{svlr})$, a function of word2vec representation of $y$, with learnable weight vector $h_y$ in the VR models. This is equivalent to using a 1000 way classification layer like the last layers of AlexNet, VGG, or ResNet instead of using our language embeddings for classification. This enables comparison of inductive transfer due to joint training both with and without sharing language representations.}


%-------------------------------------------------------------------------
The industry standard for pose edition is to create rigs, a collection of pieces of software designed to manipulate a character's skeleton. The rig describes the skeleton's bones, how they relate to each other, are constrained in their possible motion and are deformed. These rules are loosely specified and creating a good rig requires a detailed understanding of physics and anatomy, as well as technical and artistic skills. Rigging is thus a time consuming task even for experienced animators, and even more so in large scale productions which often require a different in-depth rig for each character in the cast.
Previous work has helped alleviate this difficulty by providing efficient tools to speed up/and or ease the rigging process, relying on inverse kinematics or data-driven methods.
\subsection{Character pose design}
\subsubsection{Inverse Kinematics (IK)}
IK solvers are a family of methods commonly used in robotics, engineering and computer graphics, in which the parameterization of a kinematic chain is determined from the position of its end effector.
They are a staple tool in pose design software, ensuring the respect of elementary constraints during pose edition. Their de-facto role is to guarantee the length of the limbs, and in some cases to enforce the orientation angle range of a joint.
Many IK solutions have been studied over the years \cite{aristidou_inverse_2018}; usually revolving around approximated linearizations or heuristics. 

Numerical methods require a set of iterations to achieve a satisfactory solution formulated by a cost function to be minimized.
IK solutions can generally be divided into three sub-categories: Jacobian \cite{Siciliano_Handbook_Robot_2007}, Newtonians \cite{cohen_ik_1996} and Heuristics. Most software implement heuristic methods such as Cyclic Coordinate Descent (CCD) \cite{wang_ccd_1991} or 
Forward-Backward Reaching IK (FABRIK) \cite{aristidou_fabrik:_2011} due to their simplicity and extensibility. 

The main drawback of 
these solvers is that they manipulate kinematic chains without taking into account many morphological aspects that make a pose more or less plausible. They offer a first level of help to users but are not sufficient to guarantee a realistic pose. Many joints constraints are dependent on each other and require subjective, human-made approximations.

\subsubsection{Data-driven pose edition}
Data-driven methods offer promising opportunities to solve these approximations. Using real-life data can help in modelling the complex inter-dependencies of skeletons and providing users with smarter edition tools.
While it is still an early field of research, some solutions have been studied. Wu \etal \cite{wu_posing_2009} propose a method for natural character posing from a large motion database. It employs adaptive KD-clustering to select a representative frame from a database and sparse approximations to accelerate training and posing. 
Huang \etal in \cite{Huang_IK_MGDM_2017} present a method based on the formulation of multi-variate Gaussian distribution models (MGDMs), which learn the joint constraints of a kinematic skeleton from motion capture data. 

Some work has also been dedicated to finding new editing interfaces. \modify{}{Instead of the usual setup manipulating joints directly, Guay \etal \cite{guay_line_2013} articulate a framework based on the conceptual "line of action" which describes the overall pose dynamics. They provide a mathematical definition of the line of action, and a interface in which the software modifies the pose to follow a user-provided line. In the same line of though} Garcia \etal \cite{garcia_sketching_2019} propose \modify{a method transforming doodle of trajectories (position and orientation over time) }{a virtual reality-based interface where the user's hands motion (position and orientation over time) are transformed} into sequences of actions and then into detailed character animations using a dataset of parametrized motion clips automatically fitted to the trajectory. 

% ==> DL et Latent Space. 
\subsection{Neural modelling of human motion}
Neural networks have received a great amount of attention over the last decade and shown impressive result in modelling complex data. Human motion has not been spared and deep learning methods have proven their capability of generating realistic motion in a number of difficult cases. 

The literature in neural-based animation include example in user-controlled character navigation \cite{Holden2017} and interactions with the environment \cite{starke_neural_2019}. 
Holden \etal \cite{Holden2020} also show that neural networks can be used to replace parts of existing data-driven methods, improving their scalability potential.
More recently, some work has also focused on improving smaller parts of the animation pipeline rather than replacing it completely. Berson et al. \cite{berson_intuitive_2020} leverage neural networks to provide an interactive system to edit facial animation. 

% Wrap up
Data-driven IK and pose editing can relieve animators from time-consuming, back-and-forth pose adjustments by applying constraints extracted from real-world data. Recently, neural-network-based approaches have demonstrated their ability to model the intricacies of human motion while scaling to large amount of data and retaining a fast inference time. In this paper we seek to take advantage of these properties to create an efficient posing tool, intuitively usable even by a inexperienced user.



%-------------------------------------------------------------------------


\section{Method}
We propose an SVLR module to facilitate greater inductive transfer across vision-language tasks. As shown in Fig.~\ref{fig:features}, the word and region representations required for object recognition, attribute recognition, and VQA are computed through the SVLR module. By specifically formulating each task in terms of inner products of word and region representations and training on all tasks jointly, we ensure each task provides a consistent, non-conflicting training signal for aligning words and region representations. During training, the joint-task model is fed batches containing training examples from each task's dataset.

\vspace{-1mm}

\subsection{Shared Vision Language Representation}
\label{sec:embedding}
The SVLR module converts words and image-regions into feature representations that are aligned to each other and \textit{shared} across tasks.  \\

\noindent\textbf{Word Representations:} The representation $g(w)$ for a word $w$ is constructed by applying two fully connected layers (with 300 output units each) to pretrained word2vec representation~\cite{word2vec} of $w$ with ReLU after the first layer. \\

\noindent\textbf{Region Representations:} A region $R$ is represented using two $300$ dimensional feature vectors $f_o(R)$ and $f_a(R)$ that separately encode the objects and attributes contained. We used two representations instead of one to encourage disentangling of these two factors of variation. For example, we do not expect ``red" to be similar to ``apple", but we expect $f_o(R)$ and $f_a(R)$ to be similar to $g(``red")$ and $g(``apple")$ if $R$ depicts a red apple. The features are constructed by extracting the average pooled features from Resnet~\cite{he2015deep} pretrained on ImageNet and then passing through separate object and attribute networks. Both networks consist of two fully connected layers (with 2048 and 300 output units) with batch normalization~\cite{batchnorm} and ReLU activations. 

\subsection{Visual Recognition using SVLR}
\label{sec:vr}
\subsubsection{Inference}\label{sec:recog_inference}
The visual recognition task is to classify image regions into one or more object and attribute categories. The classification score for region $R$ and object category $w$ is $f_o^T(R)g(w)$. The classification score for an attribute category $v$ is $f_a^T(R)g(v)$. Attributes may include adjectives and adverbs (e.g., ``standing''). Though our recognition dataset has a limited set of object categories $\mathcal{O}$ and attribute categories $\mathcal{T}$,  our model can produce classification scores for any object or attribute label given its word2vec representation. In experiments, the $\mathcal{O}$ and $\mathcal{T}$ consist of 1000 most frequent object and attribute categories in the Visual Genome dataset~\cite{krishna2016visual}. 

\vspace{-3mm}
\subsubsection{Training}\label{sec:recog_learn}
Our VR model is trained using the Visual Genome dataset which provides image regions annotated with object and attribute labels. VR uses only the parameters for the embedding functions $f_o, f_a$ and $g$ that are part of the SVLR module. The parameters of $f_o$ receive gradients from the object loss while those of $f_a$ receive gradients from the attribute loss. The parameters of word embedding model $g$ receive gradients from both losses.\\

\noindent 
\textbf{Object loss:} %We use a multi-label classification loss as object classes may not be mutually exclusive due to hypernyms (e.g., ``man'' \emph{is a} ``person'') and synonyms.  
We use a multi-label loss as object classes may not be mutually exclusive (e.g., ``man'' \emph{is a} ``person''). For a region $R_j$, we denote the set of annotated object categories and their hypernyms extracted from WordNet \cite{miller1995acm} by $\mathcal{H}_j$. The object loss forces the true labels and their hypernyms to score higher than all other object labels by a margin $\eta_{obj}$. For a batch of $M$ samples $\{(R_j,\mathcal{H}j)\}_{j=1}^{M}$ the object loss is:
\vspace{-5mm}
\begin{multline}
\mathcal{L}_{obj} = \frac{1}{M}\sum_{j=1}^{M}\frac{1}{|\mathcal{H}j|}\sum_{l \in \mathcal{H}j} \frac{1}{|\mathcal{O}|}
\sum_{k \in \mathcal{O}\setminus \mathcal{H}j} \\
\max\{0,\eta_{obj} + f_o^T(R_j)g(k) - f_o^T(R_j)g(l)\}
\end{multline}

\noindent
\textbf{Attribute Loss:} The attribute loss is a multi-label classification loss with two differences from object classification. Attribute labels are even less likely to be mutually exclusive than object labels. As such, we predict each attribute with independent cross entropy losses. We also weigh the samples based on fraction of positive labels in the batch to balance the positive and negative labels in the dataset. For a batch with M samples $\{(R_j,\mathcal{T}_j)\}_{j=1}^{M}$ where $\mathcal{T}_j$ is the set of attributes annotated for region $R_j$, the attribute loss is:
\vspace{-4mm}
\begin{multline}
\mathcal{L}_{atr} = \frac{1}{M}
\sum_{j=1}^{M} 
\sum_{t \in \mathcal{T}} \\
\mathbbm{1}\left[t \in \mathcal{T}_j\right](1-\Gamma(t))\log\left[\sigma(f_a^T(R_j)g(t))\right] + \\
\mathbbm{1}\left[t \notin \mathcal{T}_j\right]\Gamma(t)\log\left[1-\sigma(f_a^T(R_j)g(t))\right]
\end{multline}    
where $\sigma$ is a sigmoid activation function and $\Gamma(t)$ is the fraction of positive samples for attribute $t$ in the batch.

\begin{figure*}[ht]
\begin{center}
%\fbox{\rule{0pt}{2in} \rule{0.9\linewidth}{0pt}}
\includegraphics[width=\linewidth]{images/vqa_model.pdf}
\vspace{-1.1cm}
\end{center}
  \caption{\textbf{Inference in our VQA model:} The image is first broken down into Edge Box region proposals\cite{zitnick2014edge}. Each region $R$ is represented by visual category scores $s(R) = [s_o(R), s_a(R)]$ obtained using the visual recognition model. Using the SVLR module, the regions are also assigned an attention score using the inner products of region features with representations of nouns and adjectives in the question and answer. The region features are then pooled using the relevance scores as weights to construct the \textit{attended} image representation. Finally, the image and question/answer representations are combined and passed through a neural network to produce a score for the input question-image-answer triplet.}
  \vspace{-5mm}
\label{fig:system}
\end{figure*}
\vspace{-1mm}
\subsection{Visual Question Answering using SVLR} \label{sec:vqa}

Our VQA model is illustrated in Fig.~\ref{fig:system}.
%\subsection{Inference} \label{sec:vqa_inference}
The input to our VQA model is an image, a question, and a candidate answer. Regions are extracted from the image using Edge Boxes~\cite{zitnick2014edge}.  The same SVLR module used by VR (Sec.~\ref{sec:vr}) is explicitly applied to VQA for attention and answer scoring.  Our system assigns attention scores to each region according to how well it matches words in the question/answer, then scores each answer based on the question, answer, and attention-weighted scores for all objects ($\mathcal{O}$) and attributes ($\mathcal{T}$).

\boldhead{Attention Scoring}\label{sec:relevance} Unlike other attention models ~\cite{yang2015stacked,lu2016hierarchical} that are free to learn any correlation between regions and question/answers, our attention model encodes an explicit notion of vision-language grounding.  Let $\mathcal{R}$ be the set of region proposals extracted from the image, and $\mathcal{N}$ and $\mathcal{J}$ denote the set of nouns and adjectives in the $(Q,A)$ pair. Each region $R\in \mathcal{R}(I)$ is assigned an attention score $a(R)$ as follows:\\
\vspace{-5mm}
\begin{align}
a'(R) &= \max_{n \in \mathcal{N}} f_o^T(R)g(n) + \max_{j \in \mathcal{J}} f_a^T(R)g(j) \label{eqn:relevance_unnormalized}\\
a(R)&= \frac{\exp(a'(R))}{\sum_{R' \in \mathcal{R}(I)}\exp(a'(R'))}
\label{eqn:relevance}
\end{align}

Thus, a region's attention score is the sum of maximum adjective and noun scores for words mentioned in the question or answer (which need not be in sets $\mathcal{O}$  and $\mathcal{T}$).   

%The image representation is computed by averaging these VR scores using the attention scores as weights. This image representation is combined with the QA representation using bimodal pooling layers, and the resulting encoding of the image-question-answer triplet is scored using a fully connected layer. The answer with the highest score is chosen. \\

\boldhead{Image Representation} To score an answer, the content of region $R$ is encoded using the VR scores for all objects and attributes in $\mathcal{O}$ and $\mathcal{T}$, as presence of unmentioned objects or attributes may help answer the question. The image representation is an attention-weighted average of these scores across all regions: 
%To capture the visual content of a region $R$, we concatenate the object and attribute scores for visual categories $\mathcal{O}$ and $\mathcal{T}$ into vectors $s_o(R) \in \mathbb{R}^{1000}$ and $s_a(R) \in \mathbb{R}^{1000}$. The final 2000 dimensional, $(Q,A)$ specific, attended image representation is constructed by averaging features $s_o$ and $s_a$ across all regions using relevance scores as weights.
\begin{equation}
f(I) = \sum_{R\in\mathcal{R}(I)}a(R)
\begin{bmatrix}
s_o(R) \\
s_a(R) \\
\end{bmatrix}
\label{eqn:weightedvisualfeats}
\end{equation}
where $I$ is the image, $s_o(R)$ are the scores for 1000 objects in $\mathcal{O}$ for each image region $R$, $s_a(R)$ are the scores for 1000 attributes in $\mathcal{T}$, and $a(R)$ is the attention score.

\boldhead{Question/Answer Representation}  To construct representations $q(Q)$ and $a(A)$ for the question and answer, we follow Shih et al.~\cite{shih2016look}, dividing question words into 4 bins, averaging word representations in each bin, and concatenating the bin representations resulting in a 1200 ($=300\times4$) dimensional vector $q(Q)$. The answer representation $a(A)\in\mathbb{R}^{300}$ is obtained by averaging the word representations of all answer words. The word representations used here are produced by the SVLR module.

\boldhead{Answer Scoring} We combine the image and Q/A representations to jointly score the $(Q,I,A)$ triplet. %Since some answers like $\{0,1,2,3,yes,no\}$ may not be well represented using vector representations, we experiment with appending binary features for these answers in $a(A)$.

To ensure equal contribution of language and visual features, we apply batch normalization~\cite{batchnorm} on linear transformations of these features before adding them together to get a bimodal representation $\beta(Q,I,A)\in\mathbb{R}^{2500}$:
\begin{multline}\label{eq:bimodal_pool}
\beta(Q,I,A) = \;\mathcal{B}_1(W_1f(I)) 
+ \;\mathcal{B}_2\left(W_2 
\begin{bmatrix}
q(Q) \\
a(A) \\
\end{bmatrix}
\right)
\end{multline}
Here, $\mathcal{B}_1,\mathcal{B}_2$ denote batch normalization and $W_1\in\mathbb{R}^{2500\times2000}$ and $W_2\in\mathbb{R}^{2500\times1500}$ define the linear transformations.
\noindent
The bimodal representation is: 
\begin{equation}
\mathcal{S}(Q,I,A) = W_3 \; \text{ReLU}(\beta(Q,I,A))
\end{equation}
with  ${W_3\in\mathbb{R}^{1\times2500}}$.


\boldhead{Training}\label{sec:vqa_learn}
We use the VQA dataset~\cite{antol2015vqa} for training parameters of our VQA model: $W_1, W_2, W_3$, and scales and offsets of batch normalization layers. In addition, the VQA loss backpropagates into $f_o, f_a$, and $g$ which are part of the SVLR module. Each sample in the dataset consists of a question $Q$ about an image $I$ with list of answer options including a positive answer $A^{+}$ and $N$ negative answers $\{A^{-}(i) | i=1,\cdots, N\}$. 

The VQA loss encourages the correct answer $A^{+}$ to be scored higher than all incorrect answer options $\{A^{-}(i) | i=1,\cdots, N\}$ by a margin $\eta_{ans}$. Given batch samples $\{(Q_j,I_j,A_j)\}_{j=1}^{P}$, the loss is written as 
\vspace{-2mm}
\begin{multline}
\mathcal{L}_{ans} = \frac{1}{NP}\sum_{j=1}^{P}
\sum_{i=1}^{N} \max\{0,\\\;\eta_{ans} + \mathcal{S}(Q_j,I_j,A_j^{-}(i)) - \mathcal{S}(Q_j,I_j,A_j^{+})\}
\end{multline}

\begin{figure*}[t]
\begin{center}
\includegraphics[width=0.95\linewidth]{images/relevance_qual_2_row.png}
\vspace{-0.7cm}
\end{center}
  \caption{\textbf{Interpretable inference in VQA:} Our model produces interpretable intermediate computation for region relevance and object/attribute predictions for the most relevant regions. Our region relevance explicitly grounds nouns and adjectives from the Q/A input in the image. We also show object and attribute predictions for the most relevant region identified for a few correctly answered questions. The relevance masks are generated from relevance scores projected back to their source pixels locations.}
  \vspace{-5mm}
\label{fig:rel_qual}
\end{figure*}

\subsection{Zero-Shot VQA} 

%In the introduction, we claim the ability to perform a new vision-language task without retraining as evidence that our SVLR transfers knowledge across tasks without the need to re-learn a feature to task mapping.  
The representations produced by SVLR module should be directly usable in related vision-language tasks without any additional learning. To demonstrate this \textit{zero-shot cross-task transfer}, we train the SVLR module using Genome VR data only and apply to VQA. Since bimodal pooling and scoring layers cannot be learned without VQA data, we use a proxy scoring function constructed using region-word scores only.  For each region, we compute $p_q(R)$ as the sum of its scores for the maximally aligned question nouns and question adjectives (Eq.~\ref{eqn:relevance_unnormalized} with only question words). A score $p_a(R)$ is similarly computed using answer nouns and adjectives. The final score for the answer is defined by
\begin{equation}
    S(Q,I,A)=\sum_{R\in\mathcal{R}}a(R)\min(p_q(R),p_a(R))
\end{equation}
where $a$ is the attention score computed using Eq.~\ref{eqn:relevance}. Therefore, the highest score is given to QA pairs where question as well as answer nouns and adjectives can be localized in the image. Note that the since the model is not trained on even a \textit{single question} from VQA, the zero-shot VQA task also shows that our model does use the image to answer questions instead of solely relying on the language prior which is a common concern with most VQA models \cite{agrawal2016analyzing,goyal2016arxiv}. 

\section{Implementation and Training Details}
We use 100 region proposals resized to $224 \times 224$ for all experiments. Resnet-50 was used for image feature extraction in all experiments except those in Tab.~\ref{tab:state_art} which used Resnet-152. The nouns and adjectives are extracted from the $(Q,A)$  and lemmatized using the part-of-speech tagger and WordNet lemmatizer in NLTK \cite{bird2009book}. We use the Stanford Dependency Parser \cite{de2006lrec} to parse the question into bins as detailed in~\cite{shih2016look}.  All models are implemented and trained using TensorFlow \cite{tensorflow2015software}. We train the model jointly for the recognition and VQA tasks by minimizing the following loss function using Adam~\cite{adamoptimizer}:
\begin{equation}
\mathcal{L} = \alpha_{ans}\mathcal{L}_{ans} + \alpha_{obj}\mathcal{L}_{obj} + \alpha_{atr}\mathcal{L}_{atr}
\end{equation}
We observe that values of $\alpha_{obj}$ and $\alpha_{atr}$ relative to $\alpha_{ans}$ can be used to trade-off performance between visual recognition and VQA tasks. For experiments that analyze the effect of transfer from VR to VQA (Sec.~\ref{sec:vr2vqa}), we set ${\alpha_{ans} = 1, \alpha_{obj} = 0.1}$, and ${\alpha_{atr}=0.1}$. For VQA only and Genome only baselines, we set the corresponding $\alpha$ to 1 and others to 0.  For experiments dealing with transfer in the other direction (Sec.~\ref{sec:vqa2vr}), we set ${\alpha_{ans} = 0.1, \alpha_{obj} = 1}$, and ${\alpha_{atr}=1}$.  The margins used for object and answer losses are $\eta_{ans}=\eta_{obj}=1$. The object and attribute losses are computed for the same set of Visual Genome regions with a batch size of $M=200$. The answer loss is computed for a batch size of $P=50$ questions sampled from VQA. We use an exponentially decaying learning rate schedule with an initial learning rate of $10^{-3}$ and decay rate of 0.5 every 24000 iterations. Weight decay is used on all trainable variables with a coefficient of $10^{-5}$. All the variables are Xavier initialized \cite{glorot2010aistats}.



\section{Experiments}\label{sec:experiments}
We validate our approach using multiple datasets containing real-life data from the fields of criminal risk assessment, credit, lending, and college admissions. In each of the datasets we select a binary feature and treat it as the protected attribute (e.g., race or gender), which is the feature we require our trained classifier to behave fairly upon. Our proposed method performs well on all of these datasets, succeeding in removing unfairness almost entirely, at a very modest price in terms of accuracy.


\begin{table*}[h]
\centering
\resizebox{\textwidth}{!}{
\def\arraystretch{1.2}

\begin{tabular}{c c c | c | c | c || c | c | c || c | c | c |}

\cline{4-12}
&&&
\multicolumn{9}{ c| }{\textbf{COMPAS Dataset}}
\\ \cline{4-12}
&&&
\multicolumn{3}{ c|| }{\textbf{FPR Considerations}}&
\multicolumn{3}{ c|| }{\textbf{FNR Considerations}}&
\multicolumn{3}{ c| }{\textbf{Both Considerations}}
\\ \cline{4-12}
&&&
 $\mathbf{Acc.}$ &  $\mathbf{D_{FPR}}$ &  $\mathbf{D_{FNR}}$ &  $\mathbf{Acc.}$ &  $\mathbf{D_{FPR}}$ &  $\mathbf{D_{FNR}}$ &  $\mathbf{Acc.}$ &  $\mathbf{D_{FPR}}$ &  $\mathbf{D_{FNR}}$
\\  \cline{4-12}
\vspace*{-0.5ex}
\\ \cline{1-2} \cline{4-12}
\multicolumn{1}{ |c  }{} &
\multicolumn{1}{ c|  }{  \textbf{Our Method (AVD Penalizers)}}  &&
$\mathbf{0.660}$    &  $\mathbf{0.01}$  &  $0.04$ &
$\mathbf{0.653}$    &  $0.02$   &  $\mathbf{0.04}$ &
$\mathbf{0.654}$    &  $\mathbf{0.02}$  &  $\mathbf{0.04}$
\\ \cline{1-2} \cline{4-12}
\multicolumn{1}{ |c  }{} &
\multicolumn{1}{ c|  }{  \textbf{Our Method (SD Penalizers)}}  &&
$\mathbf{0.664}$    &  $\mathbf{0.02}$  &  $0.09$ &
$\mathbf{0.661}$    &  $0.05$   &  $\mathbf{0.03}$ &
$\mathbf{0.661}$    &  $\mathbf{0.02}$  &  $\mathbf{0.03}$
\\ \cline{1-2} \cline{4-12}
\multicolumn{1}{ |c  }{} &
\multicolumn{1}{ c|  }{  Zafar et al.~(\citeyear{disparatemistreatment})}  &&
$0.660$    &   $0.06$    &   $0.14$  &
$0.662$    &   $0.03$    &   $0.10$  &
$0.661$    &   $0.03$    &   $0.11$
\\ \cline{1-2} \cline{4-12}
\multicolumn{1}{ |c  }{} &
\multicolumn{1}{ c|  }{  Zafar et al. Baseline~(\citeyear{disparatemistreatment})}  &&
$0.643$    &   $0.03$    &   $0.11$  &
$0.660$    &   $0.00$    &   $0.07$  &
$0.660$    &   $0.01$    &   $0.09$
\\ \cline{1-2} \cline{4-12}
\multicolumn{1}{ |c  }{} &
\multicolumn{1}{ c|  }{  Hardt et al.~(\citeyear{hardt})}  &&
$0.659$    &  $0.02$    &   $0.08$  &
$0.653$    &  $0.06$   &    $0.01$  &
$0.645$    &  $0.01$   &    $0.01$
\\ \cline{1-2} \cline{4-12}
\multicolumn{1}{ |c  }{} &
\multicolumn{1}{ c|  }{  \textbf{Vanilla Regularized Logistic Regression}}  &&
$\mathbf{0.672}$    &   $\mathbf{0.20}$    &   $\mathbf{0.30}$  &
$\mathbf{0.672}$    &   $\mathbf{0.20}$    &   $\mathbf{0.30}$  &
$\mathbf{0.672}$    &   $\mathbf{0.20}$    &   $\mathbf{0.30}$
\\ \cline{1-2} \cline{4-12}
\end{tabular}
}
\vspace{3mm}
\caption{Performance comparison on the COMPAS dataset. For the approaches in bold -- Accuracy, FPR difference and FNR difference are evaluated on the test set, averaging over five runs and using a 70-30 training/test split. The performance of the remaining three approaches is stated as reported in Zafar et al.~(\citeyear{disparatemistreatment}).} \label{table:comparison_results}
\end{table*}



\begin{figure*}[b]
  \includegraphics[scale=0.6]{compas0-400.png}
  \caption{COMPAS Dataset. Accuracy, FPR difference ($\mathbf{D_{FPR}}$), and FNR difference ($\mathbf{D_{FNR}}$) (all evaluated on the test set) of the learned classifier, as a function of the weight $c=c_1 = c_2 \geq 0$ placed on the fairness penalizer terms. On the left we use the Absolute Value Difference (AVD) penalizer, and the Squared Difference (SD) penalizer on the right, both as presented in Section~\ref{regularization}. ``Relaxed FPR/FNR Diff.'' plots the value of the relevant penalization term.} %In this particular run, parameters chosen for the absolute value relaxation were: $c=80, q_c=60$, and for the squared relaxation: $c=220, q_c=30$.}
  \label{fig:compas}
\end{figure*}


\subsection{Implementation}
\textbf{Our method} 
%We instantiate our method in the following way: Given dataset $Q$, we split it randomly into a training set $S$ (which we will use for learning) and a test set $T$ (which we will only use for reporting performance). 
For the purpose of comparison with  Zafar et al.~(\citeyear{disparatemistreatment}) and Hardt et al.~\cite{hardt} on the COMPAS data, we use a parameter $c$ to induce three possible combinations of weights on the FPR and FNR penalization terms: $c = c_1$ and $c_2 = 0$; $c_1 = 0$ and $c = c_2$; and $c = c_1 = c_2$. For the other three datasets, we consider only $c = c_1 = c_2$.\footnote{The reason for varying the values of $c$ in the training phase is since we shifted to a proxy problem, in which we rely on the distance from the decision boundary rather the actual classifications. 
%Our hope is that there is no need for a worst-case cross validation between all of the combinations of $c_1, c_2, c_3$, and that the training scheme we propose is sufficient. 
It is possible, of course, that even better results are attainable using our scheme with other combinations of $c_1, c_2$, and $q$.} To explore the accuracy/fairness trade-off curve for the relaxed optimization problem~(\ref{eq:2}), we train for different values of $c$, starting at $c=0$ (which is just standard logistic regression), and growing gradually.



Given a dataset $Q$ and fixing a $d_1, d_2 \in \{0, 1\}$ of interest, we use the following training scheme:
\begin{enumerate}
\item Split $Q$ at random into training set $S$ and test set $T$.
\item For each $c$, perform cross-validation on $S$ to select the corresponding best value $q_c$ for the regularization parameter.
\item For each $(c,q_c)$, let $\theta_c = \argmin\limits_{\theta} \text{Proxy}(\theta;S,c,c,q_c)$.
\item Select $\theta^* \in \argmin\limits_{\theta_c} \text{Objective}(\theta_c;S,d_1,d_2)$.
\item Evaluate performance using $\theta^*$ on test set $T$.
\end{enumerate}
We report the average of five such runs, each with a fresh training-test split.




%We instantiate our method by solving the relaxed optimization problem~(\ref{eq:2}), in place of the original, non-convex problem~(\ref{eq:1}).  
%We test our approach with three different combinations of weights on the penalization terms:
%\katrina{What are the $d$, and how are they related to the $c$s?}
%\begin{enumerate}
%\item FPR considerations only: $d_1 = 1, d_2 = 0$.
%\item FNR considerations only: $d_1 = 0, d_2 = 1$.
%\item Both FPR, FNR considerations, assigned similar significance: $d_1 = 1, d_2 = 1$.
%\end{enumerate}
%One could, of course, pick any other combination of the FPR and FNR penalty weights.

%\katrina{I don't understand how the below is distinct from the list above}
%Learning is done by training the parameters of a logistic regressor to solve~\ref{eq:2}, while picking the value of $c_1, %c_2$ as the following:
%\begin{enumerate}
%\item FPR considerations only: $c_1 = c \geq 0$, $c_2 = 0$.
%\item FNR considerations only: $c_1 = 0$, $c_2 = c \geq 0$.
%\item Both FPR, FNR considerations, assigned similar significance: $c_1 = c_2 = c \geq 0$
%\end{enumerate}



% We then cross-validate to pick the best $c_3$ (the weight on the standard $\ell_2$-regularization term) given $c$.\footnote{The reason for varying the values of $c$ in the training phase is since we shifted to a proxy problem, in which we rely on the distance from the decision boundary rather the actual classifications. 
%Our hope is that there is no need for a worst-case cross validation between all of the combinations of $c_1, c_2, c_3$, and that the training scheme we propose is sufficient. 
%It is possible, of course, that even better results are attainable using our scheme with other combinations of $c_1, c_2, c_3$.} For each such combination, we report results as the averages of multiple \katrina{how many?} different runs, each time splitting data randomly into training and test sets.
%\yahav{We need to shorten this description.}

We solve the relaxed convex optimization problem using the CVXPY solver. Due to stability issues with large training sets, we use a train/test split of 30-70 on the larger datasets, rather than 70-30 as on the COMPAS dataset\footnote{The code implementing our method can be found at https://github.com/jjgold012/lab-project-fairness}.

%
%
%We then report the results (as evaluated on the test set) attained by a regressor $\theta \in \mathbb{R}^d$ that minimizes (on the training set $S$) a weighted combination of the $0$-$1$ loss and the differences in FPR and FNR across populations:
%\begin{equation*}
%\begin{aligned}
%&\underset{\theta}{\text{argmin}}
%& & L_{S}^{0\text{-}1}(\theta) \\
%&&& + d_1|FPR_{A=0}(\theta;S)-FPR_{A=1}(\theta;S)| \\
%&&& + d_2|FNR_{A=0}(\theta;S)-FNR_{A=1}(\theta;S)|
%\end{aligned}
%\end{equation*}
%
%\katrina{What is $d_1$ vs. $c_1$ etc.?}



%For classification, we decided use a standard cut-off threshold of $c=0.5$. There are of course, further possible interactions between the FPR, FNR considerations, and picking a certain cut-off level. These are not straightforward, since  these interactions are data-specific. 



%allows for flexibility in picking the values of $c_1, c_2$, which reflect the significance we wish to place on the objectives of achieving accuracy, equal FPR, and equal FNR. As for $c_3$, we will want to find the value of it that achieves the best results, for any combined objective of accuracy and fairness defined by a specific selection of $c_1,c_2$. Therefore, given a specific selection of $c_1, c_2$, we apply cross-validation to select the value of $c_3$. 




We briefly describe the other algorithmic approaches to which we compare:\\
\textbf{Zafar et al.}~(\citeyear{disparatemistreatment}) performs optimization by considering a proxy for the bias: the covariance between the samples' sensitive attributes and the signed distance between the feature vectors of misclassified users and the classifier decision boundary.\\
\textbf{Zafar et al. Baseline}~(\citeyear{disparatemistreatment}) tries to enforce equal FP/FN rates on the different groups by introducing different penalties for misclassified data points with different sensitive attribute values during the training phase.\\
\textbf{Hardt et al.}~(\citeyear{hardt}) performs post-processing on a standard trained (unfair) logistic regressor, picking different decision thresholds for different groups, and possibly adding randomization.


\subsection{Experimental Results}

In what follows, we use the following notation, given a trained classifier $\hat{Y}$:
\begin{align*}
\mathbf{D_{FPR}}&=\left|FPR_{A=0}(\hat{Y})-FPR_{A=1}(\hat{Y})\right| \\ 
\mathbf{D_{FNR}}&=\left|FNR_{A=0}(\hat{Y})-FNR_{A=1}(\hat{Y})\right|
\end{align*}
The values $FPR_{A=0}(\hat{Y})$, $FPR_{A=1}(\hat{Y})$, $FNR_{A=0}(\hat{Y})$, $FNR_{A=1}(\hat{Y})$ are reported as evaluated on the test set.

\paragraph{The COMPAS Dataset\footnote{https://github.com/propublica/compas-analysis}} The Correctional Offender Management Profiling for Alternative Sanctions (COMPAS) records from Broward County, Florida 2013-2014, made available online by ProPublica, are perhaps the best-studied data in the context of fairness.  The goal in this scenario is to successfully predict recidivism within two years, based on features such as age, gender, race, number of prior offenses, and charge degree. The dataset contains 5,278 samples. The protected attribute in this scenario is race, where $A$ indicates black or white. We filtered the dataset using the same features as Zafar et al.~(\citeyear{disparatemistreatment}), to allow for comparison.

%\begin{table}[h]
%\centering
%\begin{tabularx}{\columnwidth}{c|c|c|c}
%\hline
%  &  Recid. ($y = 1$)        & No Recid.  ($y = 0$)       & Total \\ \hline
%Black &  $ 1661   $ & $ 1514 $ &  $ 3175 $ \\ \hline
%White &  $ 822   $  & $1281  $ &  $ 2103 $ \\ \hline
%Total &  $ 2483  $  & $2795 $ &  $ 5278 $ \\\hline
%\end{tabularx}
%\caption{Statistics of the ProPublica COMPAS data.} \label{table:compas-stats}
%\label{tab:stats}
%\end{table}
%\vspace{-1em}

%\begin{table}[h]
%\centering
%\begin{tabularx}{\columnwidth}{c|c}
%\hline
%Feature  &  Description \\ \hline
%Age Category &  $<25$, between $25$ and $45$, $>45$ \\
%Gender &  Male or Female \\
%Race &  White or Black \\
%Priors Count &  0--37 \\
%Charge Degree &  Misconduct or Felony \\
%\hline
%2-year-recid. & Whether or not the  \\
%(target feature)  & defendant recidivated within two years
%\end{tabularx}
%\caption{Description of features used from ProPublica COMPAS data.} \label{table:compas-features}
%\label{tab:features}
%\end{table}




\begin{table*}[t]
\centering
\caption{A description of the datasets used, along with parameters of the training procedure used for each.}
\label{table:datasets_description}
\begin{adjustbox}{max width=\textwidth}
\begin{tabular}{|l|l|l|l|l|l|l|l|}
\hline
\textbf{Dataset} & \textbf{No. Samples} & \textbf{No. Features} & \textbf{Train/Test Split} & \textbf{No. Repetitions} & \textbf{No. Folds in CV} & \textbf{Protected Feature} & \textbf{Target Variable} \\ \hline
COMPAS           & 5,278                     & 5                          & 70-30                     & 5                        & 5                                 & Race                       & 2-Year-Recidivism        \\ \hline
Adult            & 30,162                    & 10                         & 30-70                     & 5                        & 5                                 & Gender                     & Income Over/Under 50K    \\ \hline
Default          & 30,000                    & 23                         & 30-70                     & 5                        & 3                                 & Gender                     & Defaulting On Payments   \\ \hline
Admissions       & 20,839                    & 17                         & 30-70                     & 5                        & 3                                 & Race                       & Passing Bar Exam         \\ \hline
\end{tabular}
\end{adjustbox}
\end{table*}


\begin{table*}[t]
\centering
\resizebox{\textwidth}{!}{
\def\arraystretch{1.2}

\begin{tabular}{c c c | c | c | c || c | c | c || c | c | c |}

\cline{4-12}
&&&
\multicolumn{3}{ c|| }{\textbf{Adult Dataset}}&
\multicolumn{3}{ c|| }{\textbf{Default Dataset}}&
\multicolumn{3}{ c| }{\textbf{Admissions Dataset}}
\\ \cline{4-12}
%&&&
%\multicolumn{3}{ c|| }{\textbf{Both Considerations}}&
%\multicolumn{3}{ c|| }{\textbf{Both Considerations}}&
%\multicolumn{3}{ c| }{\textbf{Both Considerations}}
%\\ \cline{4-12}
&&&
 $\mathbf{Acc.}$ &  $\mathbf{D_{FPR}}$ &  $\mathbf{D_{FNR}}$ &  $\mathbf{Acc.}$ &  $\mathbf{D_{FPR}}$ &  $\mathbf{D_{FNR}}$ &  $\mathbf{Acc.}$ &  $\mathbf{D_{FPR}}$ &  $\mathbf{D_{FNR}}$
\\  \cline{4-12}
\vspace*{-0.5ex}
\\ \cline{1-2} \cline{4-12}
\multicolumn{1}{ |c  }{} &
\multicolumn{1}{ c|  }{  \textbf{Our Method (AVD Penalizers)}}  &&
$\mathbf{0.776}$    &  $\mathbf{0.00}$  &  $\mathbf{0.04}$ &
$\mathbf{0.807}$    &  $\mathbf{0.00}$   &  $\mathbf{0.01}$ &
$\mathbf{0.950}$    &  $\mathbf{0.01}$  &  $\mathbf{0.00}$
\\ \cline{1-2} \cline{4-12}
\multicolumn{1}{ |c  }{} &
\multicolumn{1}{ c|  }{  \textbf{Our Method (SD Penalizers)}}  &&
$\mathbf{0.783}$    &  $\mathbf{0.00}$  &  $\mathbf{0.09}$ &
$\mathbf{0.806}$    &  $\mathbf{0.01}$   &  $\mathbf{0.02}$ &
$\mathbf{0.950}$    &  $\mathbf{0.00}$  &  $\mathbf{0.00}$
\\ \cline{1-2} \cline{4-12}
\multicolumn{1}{ |c  }{} &
\multicolumn{1}{ c|  }{  \textbf{Vanilla Regularized Logistic Regression}}  &&
$\mathbf{0.800}$    &   $\mathbf{0.08}$    &   $\mathbf{0.39}$  &
$\mathbf{0.807}$    &   $\mathbf{0.01}$    &   $\mathbf{0.05}$  &
$\mathbf{0.951}$    &   $\mathbf{0.16}$    &   $\mathbf{0.02}$
\\ \cline{1-2} \cline{4-12}
\end{tabular}
}
\vspace{3mm}
\caption{Performance on the Adult, Loan Default, and Admissions datasets, penalizing for both FPR and FNR difference. Accuracy, FPR difference and FNR difference are evaluated on the test set, averaging over five runs and using a 30-70 training/test split.} \label{table:comparison_results_rest}
\end{table*}


In Table~\ref{table:comparison_results}, we compare the performance of our approach with that of three other techniques from the literature. Each method was trained based on logistic regression.  As a basis for comparison, we also present the performance of vanilla logistic regression, absent fairness considerations, with the regularization parameter selected via cross-validation.\footnote{Zafar et al.~(\citeyear{disparatemistreatment}) do not incorporate regularization in any of the approaches they report.}
%Results are reported as the averages of 5 different runs \katrina{Is that still correct?}, each time splitting data evenly and randomly into training and test sets. 
Results for Zafar et al., Zafar et al. baseline, and Hardt et al. appear here as reported in Zafar et al.~(\citeyear{disparatemistreatment}).\footnote{Our method selects the classifier based on the training set only and reports its performance over the test set. Results for the three other approaches, reported by Zafar et al.~(\citeyear{disparatemistreatment}), are based on tuning parameters after seeing the trade-off curve over the test set, and reporting according to the best selection of these parameters.}
%\katrina{Perhaps here is the right place for a footnote about the discrepancy with the Zafar baseline}

We find that the vanilla logistic regressor (absent fairness considerations) results in significant unfairness, as $\mathbf{D_{FPR}}=0.20$, and $\mathbf{D_{FNR}}=0.30$. The overall accuracy of this classifier measured on the test set was $0.672$.\footnote{Zafar et al.~(\citeyear{disparatemistreatment}) report a slightly different baseline of: Accuracy = 0.668, $\mathbf{D_{FPR}}=0.18$, $\mathbf{D_{FNR}}=0.30$.} Our SD penalization approach empirically achieves approximately the same accuracy as the Zafar et al.~(\citeyear{disparatemistreatment}) approach, with significantly better fairness. It is difficult to compare fairness-accuracy tradeoffs with the Hardt et al.~(\citeyear{hardt}) approach, since their accuracy is significantly lower than ours. A more direct comparison is possible by noting that our learned classifier can be post-processed to improve its fairness at a direct cost to accuracy. Hence, we can achieve accuracy of $0.659$ with $\mathbf{D_{FPR}} = \mathbf{D_{FNR}} = 0.01$, which compares very favorably with the Hardt et al. accuracy rate of 0.645 given the same FPR and FNR rates.\footnote{For completeness, we note that using a 50-50 training-test split (again not using the test set for parameter selection), our method (SD, both considerations) produces a classifier that provides: Accuracy = 0.659, $\mathbf{D_{FPR}} = 0.01, \mathbf{D_{FNR}} = 0.05$. This classifier can be post-processed to achieve rates of: Accuracy = 0.655, $\mathbf{D_{FPR}} = \mathbf{D_{FNR}} = 0.01$.}

Figure \ref{fig:compas} illustrates the accuracy/fairness trade-offs achievable using our scheme. Increasing the weight $c$ on the proxy fairness penalizers results in reducing their magnitude. The figure also illustrates how our relaxed penalizers succeed in tracking the real FPR and FNR differences. 
%
%
%\katrina{Must rewrite the following paragraph}
%We observe that our method succeeds in eliminating unfairness almost completely on the COMPAS dataset, while retaining most of the accuracy, when compared to the vanilla logistic regression. We achieve very low difference rates when penalizing for achieving each of the FPR and FNR criteria individually, and also for both. We achieve preferable results comparing to Zafar et al. and Zafar et al. baseline in all 3 scenarios, and also comparing to Hardt et al. in the settings of false positive/false negative considerations only. In the setting of both considerations - The Hardt et al. method removes a larger portion of the unfairness, however it results in major accuracy loss as it achieves accuracy rate of 0.645 in comparison to our method which results in accuracy of 0.665, retaining most of the original accuracy rate while removing most of the unfairness.




%The Hardt et al.~\cite{hardt} approach as reported removes a smaller portion of the bias in the different scenarios, however for FP/FN constraints alone, it provides higher accuracy rates. The Zafar et al.~(\citeyear{disparatemistreatment}) approach as reported retains significant bias (in most cases), but in some cases  achieves slightly superior accuracy rates to the methods above. 

%These performance comparisons are incomplete in the sense that each of the compared techniques has the potential to trade off between accuracy and fairness, using some degree of parameter tuning; what we report here is only one point on the achievable trade-off frontier for each algorithm. The ``correct'' trade-off, and, in particular, the best manner in which to weigh unfairness in the FPR against unfairness in the FNR, are matters of opinion. We have chosen to report our method's performance under parameters designed to very aggressively mitigate unfairness, at some cost to the accuracy.

%It would certainly be desirable to evaluate these and other approaches to fair learning on other datasets and on different tasks, particularly on larger datasets, which might afford both greater accuracy and better bias-reduction. The present empirical evaluations, however, suggest that our regularization-based approach provides a new tool worthy of consideration---we succeed in almost entirely eliminating bias on the hold-out set, at a modest price in terms of accuracy.

%Due to the fact that our true objective includes the original non-convex penalization terms, our approach does not carry any formal guarantees. However, the ease of implementation, generality, and empirical results are encouraging. Figure~\ref{fig:test1} illustrates the rate of convergence to a fair, accurate classifier on this dataset.
%In terms of computation costs, given that at each iteration we must calculate the gradient according to the FPR and FNR regularizers, we are required to predict the labels for the entire training set at each step. 
%However, this does not pose a computational burden, as it is already required by the (classic) gradient descent algorithm in our logistic regressor fitting scheme. Furthermore, when given a sufficiently large dataset (one or two orders of magnitude larger than the one currently available for the COMPAS scores data), this could be relaxed to sampling only a mini-batch of samples from the training data set at each iteration (much as is done in stochastic gradient descent).






\subsection{Additional Datasets}


Table~\ref{table:datasets_description} provides summary statistics on each of the datasets on which we tested our approach. We also briefly describe the datasets below. 


{\bf The Adult Dataset}\footnote{http://archive.ics.uci.edu/ml/datasets/Adult} is based on 1994 US Census data. The task we consider is to predict whether the income of each individual is over or under 50K dollars per year, based on features such as occupation, marital status, and education. The protected attribute selected in this task is gender. 

{\bf The Loan Default Dataset}\footnote{{\scriptsize https://archive.ics.uci.edu/ml/datasets/default+of+credit+card+clients}}
contains data regrading Taiwanese credit card users. The task we consider is to predict whether an individual will default on payments, based on features such as history of past payments, age, and the amount of given credit. The protected attribute is gender.

{\bf The Admissions Dataset}\footnote{http://www2.law.ucla.edu/sander/Systemic/Data.htm}
contains records of law school students who went on to take the bar exam. The task we consider is to predict whether a student will pass the exam based on features such as LSAT score, undergraduate GPA, and family income. The protected attribute is set to race.

Table~\ref{table:comparison_results_rest} describes the performance of our approach on these datasets, and Figures~\ref{fig:adult},~\ref{fig:default}, and~\ref{fig:lawschool} illustrate the fairness-accuracy trade-offs we achieve in each context. Overall, we see that unfairness is nearly eliminated while accuracy remains quite high. The dataset on which accuracy suffers most under our approach is the Adult dataset, which is also the dataset on which the vanilla regression is the most unfair.


\begin{figure*}[]
  \includegraphics[scale=0.6]{adult0-800.png}
  \caption{Adult Dataset. Fairness-Accuracy tradeoffs, as in Figure~\ref{fig:compas}.}
  \label{fig:adult}  
\end{figure*}



\begin{figure*}[]
  \includegraphics[scale=0.6]{default0-50.png}
  \caption{Loan Default Dataset. Fairness-Accuracy tradeoffs, as in Figure~\ref{fig:compas}.}
  \label{fig:default}
\end{figure*}



\begin{figure*}[]
  \includegraphics[scale=0.6]{admissions0-400.png}
  \caption{Admissions Dataset. Fairness-Accuracy tradeoffs, as in Figure~\ref{fig:compas}.}
  \label{fig:lawschool}
\end{figure*}



\section{Conclusion}
%We propose an effective means of transferring knowledge and representations between related vision-language tasks by formulating each tasks' model around a shared image-to-text embedding space (SVLR). By sharing knowledge across tasks in the form of alignments between visual features and text, the shared representation has the same interpretation across all tasks, allowing supervision in one task to directly benefit the other by means of improving the shared embedding space. 

Humans learn new skills by building upon existing knowledge and experiences. We attempt to apply this behavior to AI models by demonstrating cross-task learning for the class of vision-language problems using VQA and VR. To enhance inductive transfer, we propose sharing core vision and language representations across all tasks in a way that exploits the word-region alignment. We plan to extend our method to larger sets of vision-language tasks. 

%We designed an interpretable Vision-Language representation space that can be reused across multiple tasks. We show learning and application of the shared representation space through visual recognition and the VQA tasks. We also formulate the VQA task in terms of the recognition task which allows more efficient utilization of recognition supervision for VQA and weak supervision of visual recognition through VQA. As future work, we plan to extend the approach to other tasks such as activity recognition and image-caption retrieval. For VQA, we plan to model pairwise relations explicitly to answer questions involving spatial reasoning. 

\section{Acknowledgements}
This work is supported in part by NSF Awards 14-46765 and 10-53768 and ONR MURI N000014-16-1-2007.

{\small
\bibliographystyle{ieee}
\bibliography{egbib}
}

\end{document}
