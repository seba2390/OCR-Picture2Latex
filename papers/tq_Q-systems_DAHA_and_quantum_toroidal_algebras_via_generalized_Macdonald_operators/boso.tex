\label{secboso}
%\color{red}
%Problem of the non-independence of the $p_k$ as $N$ is finite.
%\color{black}
In this section, we show that the action of the currents $\e(z),\f(z)$ on the space of symmetric functions in $N$ variables can be expressed in terms of plethysms, which act naturally on functions expressed in terms of the power sum symmetric functions. In the infinite rank limit $N\to\infty$, the power sum symmetric functions become algebraically independent, and together with derivatives with respect to these functions, they generate an infinite-dimensional Heisenberg algebra. Thus the plethystic expression becomes, in this limit, a formula for the bosonization of the action of the currents on the space of symmetric functions. This formulation will be compared to recent work of Bergeron et al \cite{BGLX}. 

%In this section, we present a bosonization of the currents
%${\mathfrak e(z)}$ and ${\mathfrak f(z)}$ of \eqref{highercur} with $\al=1$, introduced in the previous section. 
%The idea is to express these currents as functions of two types of operators: those of multiplication by the power sum symmetric functions,
%$p_k=\sum_{i=1}^N x_i^k$ for $k\in \Z^*$, and those of differentiation with respect to $p_k$.  In the limit where $N\to\infty$, these generate of a Heisenberg algebra, hence the term bosonization. 
%When $N$ is finite, $\mathfrak e(z)$ and ${\mathfrak f(z)}$ are expressed in terms of the power sum symmetric functions with a finite number of variables.


\subsection{Power sum action and plethysms}

Let $\{p_k\}_{k\in \Z\setminus\{0\}}$ be an infinite set of algebraically independent variables, and consider 
the space $\mathcal P$ of functions that can be expressed as formal
power series of the $p_k$'s. For any collection $X$ of variables, e.g. $X=(x_1,x_2,...,x_N)$,
we may evaluate functions in $\mathcal P$ by using the substitution 
$p_k\mapsto p_k[X]:=\sum_{i=1}^N x_i^k$, namely by interpreting the $p_k$'s as power sums of the $x$'s, for $k\in \Z^*$, in which case any $F\in {\mathcal P}$ may be evaluated 
as $F[X]$ by taking all $p_k= p_k[X]$ in its power series. If the number of variables $N$ is finite, the resulting space is a quotient of the original space as the functions $p_k[X]$s are not algebraically independent. Note that $F[X]$ is a symmetric function. 

Given the collection $X$, we
%Consider the subspace ${\mathcal P}_N\subset \mathcal{S}_N$ consisting of symmetric functions in the variables $\{x_1,...,x_N\}$ which can be expressed as power series of the {\it power sum} variables 
%$p_k=\sum_{i=1}^N x_i^k$, $k\in \Z^*$. (When there is a finite number $N$ of variables, the power sum symmetric functions are not all algebraically independent.)
 consider the action of the generalized Macdonald operators ${\mathcal D}^{q,t}_{1;n}$ and
${\mathcal D}^{q^{-1},t^{-1}}_{1;n}$ on the space of functions of the form $F[X]$, for $F\in {\mathcal P}$, which we denote by ${\mathcal P}[X]$. First, we present some commutation relations which hold for arbitrary $N$.

\begin{lemma}\label{compk}
For any $N$ and $i\leq N$, and given the collection $X=(x_1,x_2,...,x_N)$, the following commutation relations hold:
%\begin{equation}
$$
[p_k[X], {\mathcal D}^{q,t}_{1;n}]=(1-q^k) \, {\mathcal D}^{q,t}_{1;n+k},\quad {\rm and}\quad 
[p_k[X], {\mathcal D}^{q^{-1},t^{-1}}_{1;n}]=(1-q^{-k}) \, {\mathcal D}^{q^{-1},t^{-1}}_{1;n+k}.
%\end{equation}
$$
\end{lemma}
\begin{proof} By direct computation, using $[p_k[X],\Gamma_i]=(1-q^k)\,x_i^k\,\Gamma_i$, with $\Gamma_i$ as in \eqref{shift}.\end{proof}


\begin{cor}\label{efpleth} For any $N$, $k\in \Z^*$, and $X$ as above,
\begin{eqnarray*}
{\mathfrak e}(z)\, p_k[X]&=&\left( p_k[X]+\frac{q^{k/2}-q^{-k/2}}{z^k} \right) {\mathfrak e}(z),\\
{\mathfrak f}(z)\, p_k[X]&=&\left( p_k[X]-\frac{q^{k/2}-q^{-k/2}}{z^k} \right) {\mathfrak f}(z).
\end{eqnarray*}
\end{cor}

%Note here that despite the fact that all $p_k$, $k\in \Z^*$ are not independent (as the number of variables $x_i$ is finite),
%the commutation relations above are valid for {\it all} $k\in \Z^*$.
Therefore, up to a scalar multiple coming from the action of the currents on the constant function 1, the action of the currents ${\mathfrak e}$, ${\mathfrak f}$ on ${\mathcal P}[X]$
is by substitutions on all the power sums of the form $p_k[X]\mapsto p_k[X]+\mu_k$ for some specific sequences $\mu_k$.

Such substitutions are {\em plethysms} (see the notes \cite{Haiman} for a detailed exposition). In $\lambda$-ring notations, one writes as above
$X=(x_1,x_2,...)$ for the collection of variables (an alphabet), for which power sums are $p_k[X]=\sum_i x_i^k$.
For two alphabets $X$, $Y$, the sum $X+Y$ refers to the concatenation of the two alphabets, with power
sums $p_k[X+Y]=p_k[X]+p_k[Y]$, while for any scalar $\lambda$ we have $p_k[\lambda X]=\lambda^k p_k[X]$. 
Note finally that for a single variable alphabet $\mu$, we have $p_k[X+\mu]=p_k[X]+\mu^k$.

In the plethystic notation, Corollary \ref{efpleth} can be written as
\begin{equation}\label{plet} {\mathfrak e}(z) \,F[X]= F\left[X+\frac{q^{1/2}-q^{-1/2}}{z}\right]\,\e(z),\qquad 
 {\mathfrak f}(z)\,F[X]=F\left[X-\frac{q^{1/2}-q^{-1/2}}{z}\right]\,\f(z)  \end{equation}
for any $F\in {\mathcal P}$, 
where the alphabet $X$ is extended by two variables.
 
%The action of the currents on the right is fixed up to a multiplicative factor, the action on the constant function $1$. 
Define the two functions in ${\mathcal P}[X]$
\begin{eqnarray}
C(z)&=&\prod_{i=1}^{N}(1-z \,x_i)= e^{-\sum_{k=1}^\infty p_k[X] \frac{z^k}{k} }\label{Cdef} ,\\
{\widetilde C}(z)&=&\prod_{i=1}^{N}(1-z \,x_i^{-1})=e^{-\sum_{k=1}^\infty p_{-k}[X] \frac{z^k}{k} }\label{Ctdef}.
\end{eqnarray}
In terms of these functions we can write
\begin{thm}\label{pletbo}
The currents ${\mathfrak e}$ and ${\mathfrak f}$ act as the following plethysms on symmetric functions $F[X]\in {\mathcal P}[X]$:
\begin{eqnarray*}
{\mathfrak e}(z)\, F[X] &=&\frac{q^{1/2}}{1-q} \frac{t^{-1/2}}{1-t^{-1}}\left\{ t^{\frac{N}{2}}\frac{C(q^{1/2}t^{-1}z)}{C(q^{1/2}z)}-t^{-\frac{N}{2}}\frac{{\widetilde C}(q^{-1/2}tz^{-1})}{{\widetilde C}(q^{-1/2}z^{-1})}\right\}
F\left[ X+\frac{q^{1/2}-q^{-1/2}}{z}\right],\\
{\mathfrak f}(z)\, F[X] &=&\frac{q^{-1/2}}{1-q^{-1}} \frac{t^{1/2}}{1-t}\left\{t^{-\frac{N}{2}}\frac{C(q^{-1/2}tz)}{C(q^{-1/2}z)}- t^{\frac{N}{2}}\frac{{\widetilde C}(q^{1/2}t^{-1}z^{-1})}{{\widetilde C}(q^{1/2}z^{-1})}\right\}
F\left[ X-\frac{q^{1/2}-q^{-1/2}}{z}\right].
\end{eqnarray*}
\end{thm}
\begin{proof}
The plethystic part of the action was derived in \eqref{plet}. To fix the overall factors, apply the
currents to the constant $F[X]=1$. For example,
\begin{eqnarray*}
{\mathfrak e}(z)\cdot 1&=& \frac{q^{1/2}}{1-q} \sum_{i=1}^N \delta(q^{1/2}z x_i) \prod_{j\neq i} \frac{\theta x_i -\theta^{-1} x_j}{x_i-x_j}\\
&=&\frac{q^{1/2}}{1-q}\sum_{i=1}^N \left\{ t^{\frac{N-1}{2}} \frac{1}{1-q^{1/2}z x_i} \prod_{j\neq i} \frac{ x_i -t^{-1} x_j}{x_i-x_j}
+t^{-\frac{N-1}{2}} \frac{1}{1-q^{-1/2}z^{-1} x_i^{-1}} \prod_{j\neq i} \frac{ t x_i -x_j}{x_i-x_j}\right. \\
&&\qquad \qquad \qquad \qquad \qquad \qquad \qquad -\left.
\prod_{j\neq i} \frac{\theta x_i -\theta^{-1} x_j}{x_i-x_j}\right\}.
\end{eqnarray*}

First, we have
\begin{equation}\label{firstcn}
\sum_{i=1}^N\prod_{j\neq i} \frac{\theta x_i -\theta^{-1} x_j}{x_i-x_j} =\frac{t^{\frac{N-1}{2}}-t^{-\frac{N+1}{2}}}{1-t^{-1}}
=-\left\{  \frac{t^{-\frac{N+1}{2}}}{1-t^{-1}}+ \frac{t^{\frac{N+1}{2}}}{1-t}\right\},
\end{equation}
by noting that the left hand side of this equation is a symmetric rational function, whose common denominator is the Vandermonde determinant
$\prod_{i<j}x_i-x_j$, and which has
total degree $0$, hence must be a constant $c_N$. The constant can be found, for example, by taking the limit $x_1\to\infty$:
$$c_N=\theta^{N-1}+\theta^{-1} \sum_{i=2}^{N} \prod_{j\neq i\atop j>1} \frac{\theta x_i -\theta^{-1} x_j}{x_i-x_j}=
\theta^{N-1}+\theta^{-1} c_{N-1} ,$$
which, together with $c_0=0$, yields $c_N=\frac{\theta^N-\theta^{-N}}{\theta-\theta^{-1}}$, and \eqref{firstcn} follows.

Next, by simple fraction decomposition,
$$\frac{C(q^{1/2}t^{-1}z)}{C(q^{1/2}z)} =\prod_{i=1}^N \frac{1-q^{1/2}t^{-1}zx_i}{1-q^{1/2}zx_i}=t^{-N}+(1-t^{-1}) 
\sum_{i=1}^N \frac{1}{1-q^{1/2}z x_i} \prod_{j\neq i} \frac{ x_i -t^{-1} x_j}{x_i-x_j}.$$
Therefore
$$\frac{t^{\frac{N-1}{2}}}{1-t^{-1}} \frac{C(q^{1/2}t^{-1}z)}{C(q^{1/2}z)}=
\frac{t^{-\frac{N+1}{2}}}{1-t^{-1}}+t^{\frac{N-1}{2}}
\sum_{i=1}^N \frac{1}{1-q^{1/2}z x_i} \prod_{j\neq i} \frac{ x_i -t^{-1} x_j}{x_i-x_j}.$$

By using the change of variables $(q,t,z,x_i)\to (q^{-1},t^{-1},z^{-1},x_i^{-1})$ we get:
$$\frac{t^{-\frac{N-1}{2}}}{1-t} \frac{{\widetilde C}(q^{-1/2}tz^{-1})}{{\widetilde C}(q^{-1/2}z^{-1})}=
\frac{t^{\frac{N+1}{2}}}{1-t}+t^{-\frac{N-1}{2}}\sum_{i=1}^N\frac{1}{1-q^{-1/2}z^{-1} x_i^{-1}} 
\prod_{j\neq i} \frac{ t x_i -x_j}{x_i-x_j}.$$
Summing the two above contributions yields the full action of ${\mathfrak e}(z)$ on $1$. 
That of ${\mathfrak f}(z)$
follows immediately by sending $(q,t)\to (q^{-1},t^{-1})$.
\end{proof}

Moreover, the Cartan currents act as scalars on symmetric functions of $x_1,x_2,...,x_N$:
$$
\psi^{+}(z)=\frac{C(q^{-1/2}t z)C(q^{1/2}t^{-1} z)}{C(q^{-1/2} z)C(q^{1/2} z)} , \quad 
\psi^{-}(z)=\frac{{\widetilde C}(q^{-1/2}t z^{-1}){\widetilde C}(q^{1/2}t^{-1} z^{-1})}{{\widetilde C}(q^{-1/2} z^{-1}){\widetilde C}(q^{1/2} z^{-1})}\ .
$$

\begin{remark}
The plethystic formulas above for ${\mathfrak e}$ and $-{\mathfrak f}$, as well as $\psi^+$ and $\psi^-$, 
are exchanged under the involution $x_i\mapsto 1/x_i$
for all $i$, and $z\mapsto z^{-1}$, under which $C(z)\mapsto {\widetilde C}(z^{-1})$, 
and the one-variable plethysm $[X+\mu]\mapsto [X+\mu^{-1}]$. This is in agreement with Remark \ref{remef} for $\al=1$.
\end{remark}
 
 
%\subsection{Bosonization formulas}
%
%%The bosonization of the currents uses the operators $p_k$ and $\frac{\partial}{\partial p_k}$, $k\in \Z^*$.
%
%In this section, we shall assume that the functions $\{p_k: k\in \Z^*\}$ are algebraically independent. This is in order to make a connection with bosonization formulas, where the number of variables $\{x_i\}$ is infinite. (Although $N$ is an explicit parameter in our formulas for the currents $\e(z), \f(z)$, we keep it as a parameter at first). The commutation relations
%$$
%\left[ \frac{\partial}{\partial p_j}, p_k\right] = \delta_{jk}
%$$
%hold in this case, and we call the expression of the currents in terms of these elements a bosonization.
%
%When $\{p_k, k\in\Z^*\}$ are independent, a plethysm which adds one additional variable to the alphabet can be written as
%$$F[X+\mu]=\exp\left\{ \sum_{k\neq 0} \mu^k \frac{\partial}{\partial p_k}\right\} F[X]. $$
%In this case, Theorem \ref{pletbo} can be written as
%\begin{thm}
%The Macdonald currents ${\mathfrak e}$, ${\mathfrak f}$ are bosonized as follows:
%\begin{eqnarray*}
%{\mathfrak e}(z)&=&\frac{q^{1/2}}{1-q} \frac{t^{-1/2}}{1-t^{-1}} \left\{ t^{\frac{N}{2}}
%e^{\sum_{k>0} p_k \frac{q^{k/2}(1-t^{-k})z^k}{k}}-t^{-\frac{N}{2}} 
%e^{\sum_{k>0}p_{-k}\frac{q^{-k/2}(1-t^k)z^{-k}}{k} } \right\}\\
%&&\qquad \qquad \qquad \qquad\times\  e^{\sum_{k\neq 0} \frac{q^{k/2}-q^{-k/2}}{z^k}\frac{\partial}{\partial p_k}}, \nonumber \\
%{\mathfrak f}(z)&=&\frac{q^{-1/2}}{1-q^{-1}} \frac{t^{-1/2}}{1-t^{-1}}
%\left\{t^{\frac{N}{2}} e^{\sum_{k>0}p_{-k}\frac{q^{k/2}(1-t^{-k})z^{-k}}{k} }-t^{-\frac{N}{2}}
%e^{\sum_{k>0}p_{k}\frac{q^{-k/2}(1-t^k)z^{k}}{k} }\right\}\\
%&&\qquad \qquad\qquad \qquad\times\   e^{-\sum_{k\neq 0} \frac{q^{k/2}-q^{-k/2}}{z^k}\frac{\partial}{\partial p_k}}. \nonumber
%\end{eqnarray*}
%\end{thm}
%Note that we keep $N$ as a parameter in the formulas at this point.
%
%Using the expressions (\ref{Cdef}-\ref{Ctdef}), we arrive at the following expressions for the Cartan currents:
%\begin{eqnarray*}
%\psi^+(z)&=&e^{-\sum_{k>0} p_k(t^{k/2}-t^{-k/2})(t^{k/2}q^{-k/2}-t^{-k/2}q^{k/2})\frac{z^k}{k}}\\ 
%\psi^-(z)&=&e^{-\sum_{k>0} p_{-k}(t^{k/2}-t^{-k/2})(t^{k/2}q^{-k/2}-t^{-k/2}q^{k/2})\frac{z^{-k}}{k}} .
%\end{eqnarray*}
%Here, the parameter $N$ does not appear.

\subsection{Bosonization formulas in the $N\to\infty$ limit}
As mentioned at the beginning of the section, in the infinite rank $N\to\infty$ limit the  power sum symmetric functions $p_k$ become algebraically independent for all $k\in \Z\setminus\{0\},$ and together with the derivatives with respect to these functions, form a Heisenberg algebra.

\subsubsection{Plethystic formulas for the $N\to\infty$ limit}

The limit of an infinite number of variables is obtained by taking $N\to\infty$, whereas the collection $X=(x_1,x_2,...)$ becomes infinite. 
In view of the plethystic formulas of Theorem \ref{pletbo}, it is natural to define:
${\mathfrak e}_\infty(z) :=\lim_{N\to \infty} t^{\frac{1-N}{2}}\, {\mathfrak e}(z)$, where we implicitly assumed that $|t|>1$.
Accordingly, we define the limiting functions:
$$C_\infty(z):=\prod_{i=1}^\infty (1-z x_i),\qquad 
{\widetilde C}_\infty(z):=\prod_{i=1}^\infty (1-z x_i^{-1})$$
and therefore the limiting Cartan current:
\begin{equation}
\label{psip}
\psi_\infty^{+}(z)=\frac{C_\infty(q^{-1/2}t z)C_\infty(q^{1/2}t^{-1} z)}{C_\infty(q^{-1/2} z)C_\infty(q^{1/2} z)}
\end{equation}
which is a power series of $z$.

To define the limiting ${\mathfrak f}$ current,
recall there are two equivalent ways of defining the current ${\mathfrak f}(z)$ in terms of ${\mathfrak e}(z)$ for
finite number of variables $N$:
$$(1)\ {\mathfrak f}(z)=-S{\mathfrak e}(z^{-1})S\qquad (2) \  {\mathfrak f}(z)={\mathfrak e}(z)\vert_{q\to q^{-1},t\to t^{-1}}$$
where we used (2) as the original definition, and (1) comes from Remark \ref{remef}.
It turns out that applying these to ${\mathfrak e}_\infty(z)$ leads to two different definitions for ${\mathfrak f}_\infty(z)$.

Indeed, it is easy to see that (1) leads to 
${\mathfrak f}^{(1)}_\infty(z) :=\lim_{N\to \infty} t^{\frac{1-N}{2}}\, {\mathfrak f}(z)$,
where it is still assumed that $|t|>1$. However, this naive choice leads to a trivial commutation relation, as:
$$[{\mathfrak e}_\infty(z),{\mathfrak f}^{(1)}_\infty(w)]=
\lim_{N\to\infty} t^{1-N}\, [{\mathfrak e}(z),{\mathfrak f}(w)]=\lim_{N\to\infty} t^{1-N}\, \frac{\delta(z/w)}{g(1,1)}(\psi^+(z)-\psi^-(z))=0$$
as $\psi^\pm$ tend to well-defined infinite products.

We shall therefore rather use the second definition (2), and write
${\mathfrak f}_\infty(z)={\mathfrak e}_\infty(z)\vert_{q\to q^{-1},t\to t^{-1}}$.

%By taking the $N\to\infty$ limit of the result of Theorem \ref{pletbo}, we obtain the following:
\begin{thm}\label{bobothm}
The limiting currents ${\mathfrak e}_\infty(z),{\mathfrak f}_\infty(z)$ act on functions 
$F[X]$ as:
\begin{eqnarray*}
{\mathfrak e}_\infty(z)\, F[X] 
&=&\frac{q^{1/2}}{(1-q)(1-t^{-1})} \,\frac{C_\infty(q^{1/2}t^{-1}z)}{C_\infty(q^{1/2}z)} \,
F\left[ X+\frac{q^{1/2}-q^{-1/2}}{z}\right],\\
{\mathfrak f}_\infty(z)\, F[X] 
&=&\frac{q^{-1/2}}{(1-q^{-1})(1-t)}\,\frac{C_\infty(q^{-1/2}t z)}{C_\infty(q^{-1/2}z)} \,
F\left[ X-\frac{q^{1/2}-q^{-1/2}}{z}\right].
\end{eqnarray*}
Moreover we have the commutation relations:
\begin{eqnarray*}[{\mathfrak e}_\infty(z),{\mathfrak f}_\infty(w)] &=&\frac{1}{g(1,1)}\left\{ \delta(z/w)
\psi_\infty^+(z)-\delta(q t^{-1} z/w) \psi_\infty^-(z)\right\}\\
{\mathfrak e}_\infty(z)\,{\mathfrak e}_\infty(w)&=&-\frac{g(w,z)}{g(z,w)}\, {\mathfrak e}_\infty(w)\, {\mathfrak e}_\infty(z),\ \ {\mathfrak f}_\infty(z)\,{\mathfrak f}_\infty(w)=-\frac{g(z,w)}{g(w,z)}\, {\mathfrak f}_\infty(w)\, {\mathfrak f}_\infty(z)\\
\psi_\infty^\pm(z)\,{\mathfrak e}_\infty(w)&=& -\frac{g(w,z)}{g(z,w)}\, {\mathfrak e}_\infty(w)\, \psi_\infty^\pm(z), \ \ 
\psi_\infty^+(z)\,{\mathfrak f}_\infty(w)=-\frac{g(z,w)}{g(w,z)} \, {\mathfrak f}_\infty(w)\, \psi_\infty^+(z)\\
\psi_\infty^-(z)\,{\mathfrak f}_\infty(w)&=&-\frac{g(qt^{-1}z,w)}{g(w,qt^{-1}z)} {\mathfrak f}_\infty(w)\, \psi_\infty^-(z), \ \
\psi_\infty^-(z)\,\psi_\infty^+(w)=\frac{g(w,z)g(qt^{-1}z,w)}{g(z,w)g(w,qt^{-1}z)}\, \psi_\infty^+(w)\,\psi_\infty^-(z)
\end{eqnarray*}
where $\psi_\infty^+(z)$ is defined in \eqref{psip}, and  $\psi_\infty^-(z)$ is a power series of $z^{-1}$
acting on symmetric functions $F[X]$ as:
$$\psi_\infty^-(z) F[X]=F\Big[X+\frac{(q^{1/2}-q^{-1/2})(1-t q^{-1})}{z}\Big] $$
In addition, the "Serre relations" of Theorem \ref{thserrre} still hold with 
${\mathfrak e},{\mathfrak f}\to {\mathfrak e}_\infty, {\mathfrak f}_\infty$.
\end{thm}

\begin{remark}
The algebra satisfied by the currents ${\mathfrak e}_\infty(z),{\mathfrak f}_\infty(z)$ and 
$\psi^\pm_\infty(z)\in \C[[z^{\pm 1}]]$ is a particular case of
the general quantum toroidal algebra of Def. \ref{qtorgendef}. 
%The latter has the generating currents $x^{\pm}(z)$ and 
%$\psi^\pm(z)\in \C[[z^{\mp 1}]]$ 
%and two central elements $\hat \gamma$ and $\hat \delta$, subject to 
%the following relations, best expressed in terms of the rational function
%$$G(x)=-\frac{g(1,x)}{g(x,1)}=\frac{(1-q x)(1-t^{-1}x)(1-q^{-1}t x)}{(1-q^{-1} x)(1-t x)(1-qt^{-1} x)}$$
%and read:
%\begin{eqnarray*}
%[\psi^\pm(z),\psi^\pm(w)]&=&0,\qquad \psi^+(z)\,\psi^-(w)=\frac{G({\hat \gamma}w/z)}{G({\hat \gamma}^{-1}w/z)}\,
%\psi^-(w)\,\psi^+(z)\\
%\psi^+(z)\, x^\pm(w)&=&G({\hat \gamma}^{\mp 1}w/z)^{\mp 1}\, x^\pm(w)\,\psi^+(z),\quad 
%\psi^-(z)\, x^\pm(w)=G({\hat \gamma}^{\mp 1}w/z)^{\pm 1}\, x^\pm(w)\,\psi^-(z)\\
%x^\pm(z)\, x^\pm(w)&=&G(z/w)^{\pm 1} \, x^\pm(w)\, x^\pm(z), \qquad \psi_0^\pm={\hat \delta}^{\mp 1},\\
%{[} x^+(z), x^-(w) {] }&=& \frac{(1-q)(1-t^{-1})}{(1-q t^{-1})} \left\{ \delta({\hat \gamma}^{-1}z/w)\psi^+({\hat \gamma}^{-1/2}z)
%-\delta({\hat \gamma} z/w)\psi^-({\hat \gamma}^{1/2}z)\right\} \\
%&&{\rm Sym}_{z_1,z_2,z_3}\left( \frac{z_2}{z_3} {\Big[}{\mathfrak x^\pm}(z_1),{[}{\mathfrak x^\pm}(z_2),{\mathfrak x^\pm}(z_3){]}{\Big]}\right)
%=0
%\end{eqnarray*}
%A particular class of representations \cite{FHHSY} indexed by $(\ell_1,\ell_2)\in \Z^2$ corresponds to diagonal actions of the central elements ${\hat \gamma},{\hat \delta}$
%with respective eigenvalues $\gamma^{\ell_1},\gamma^{\ell_2}$, where $\gamma=(t q^{-1})^{1/2}$.
In this language, the relations of Theorem \ref{bobothm} take place in the so-called horizontal representation
$(\ell_1,\ell_2)=(1,0)$, namely with ${\hat \gamma}=\gamma=(tq^{-1})^{1/2}$ and $\hat \delta=1$,
and with the correspondence (see \eqref{dictio}):
\begin{eqnarray}
&&x^+(z)= \frac{(1-q)(1-t^{-1})}{q^{1/2}}{\mathfrak e}_\infty(q^{-1/2}z),\quad x^-(z)
=\frac{(1-q^{-1})(1-t)}{q^{-1/2}}{\mathfrak f}_\infty(t^{-1/2}z),\nonumber \\ 
&&\varphi^+(z)= \psi_\infty^-(q^{-3/4}t^{1/4}z^),\qquad \varphi^-(z)= \psi_\infty^+(q^{-1/4}t^{-1/4}z)
\label{corresp}
\end{eqnarray}
\end{remark}

Let us now turn to the proof of the theorem.
\begin{proof}
The expressions for ${\mathfrak e}_\infty(z)$ and ${\mathfrak f}_\infty(z)$ follow from Theorem \ref{pletbo}
and the definition (2). To compute the commutator, we must commute the plethysms through the prefactors.
Using:
\begin{eqnarray*}
\frac{C_\infty(q^{-1/2}t w)}{C_\infty(q^{-1/2}w)}\left[ X+\frac{q^{1/2}-q^{-1/2}}{z}\right]&=&
\frac{(1-tw/z)(1-q^{-1}w/z)}{1-w/z)(1-q^{-1}tw/z)}\frac{C_\infty(q^{-1/2}t w)}{C_\infty(q^{-1/2}w)}\\
\frac{C_\infty(q^{1/2}t^{-1}z)}{C_\infty(q^{1/2}z)}\left[ X-\frac{q^{1/2}-q^{-1/2}}{w}\right]&=&
\frac{(1-t^{-1}z/w)(1-qz/w)}{1-z/w)(1-qt^{-1}z/w)}\frac{C_\infty(q^{1/2}t^{-1}z)}{C_\infty(q^{1/2}z)}
\end{eqnarray*}
we arrive at:
\begin{eqnarray*}
[{\mathfrak e}_\infty(z),{\mathfrak f}_\infty(w)]&=& \frac{1}{(1-q)(1-q^{-1})(1-t)(1-t^{-1})}\frac{C_\infty(q^{1/2}t^{-1}z)C_\infty(q^{-1/2}t w)}{C_\infty(q^{1/2}z)C_\infty(q^{-1/2}w)}\\
&&\quad \times \left\{ \frac{(1-tw/z)(1-q^{-1}w/z)}{1-w/z)(1-q^{-1}tw/z)}-\frac{(1-t^{-1}z/w)(1-qz/w)}{1-z/w)(1-qt^{-1}z/w)}\right\}\\
&&\qquad \qquad \times F\left[ X+(q^{1/2}-q^{-1/2})(\frac{1}{z}-\frac{1}{w})\right]\\
&=&\frac{1}{g(1,1)} (\delta(z/w)-\delta(qt^{-1}z/w) )\frac{C_\infty(q^{1/2}t^{-1}z)C_\infty(q^{-1/2}t w)}{C_\infty(q^{1/2}z)C_\infty(q^{-1/2}w)}\\
&&\qquad \qquad \times F\left[ X+(q^{1/2}-q^{-1/2})(\frac{1}{z}-\frac{1}{w})\right]
\end{eqnarray*}
The other relations are obtained in a similar way,
and the theorem follows.
\end{proof}
%\subsubsection{Reformulation in terms of the ${\mathfrak m}$ currents}
\subsubsection{Comparison with the plethystic operators of Bergeron et al \cite{BGLX}}

To simplify the comparison with the operators of \cite{BGLX}, let us introduce the generating functions:
\begin{eqnarray}{\mathfrak m}(z)&:=&\sum_{n\in \Z} z^n\, {\mathcal M}_n= \frac{1-q}{q^{1/2}} \,t^{\frac{N-1}{2}}\, {\mathfrak e}(q^{-1/2}z)\label{mcurdefone}, \\
\widetilde{\mathfrak m}(z)&:=&{\mathfrak m}(q z)\vert_{q\to q^{-1},t\to t^{-1}}=\frac{1-q^{-1}}{q^{-1/2}}\,t^{-\frac{N-1}{2}}\, {\mathfrak f}(q^{-1/2}z).\label{mcurdeftwo}\end{eqnarray}
In the infinite rank limit, we write
$${\mathfrak m}_\infty(z):=\lim_{N\to\infty} t^{1-N}\,  {\mathfrak m}(z)
=\frac{1-q}{q^{1/2}}{\mathfrak e}_\infty(q^{-1/2}z),\quad 
\widetilde{\mathfrak m}_\infty(z)={\mathfrak m}_\infty(q z)\vert_{q\to q^{-1},t\to t^{-1}}
=\frac{1-q^{-1}}{q^{-1/2}}{\mathfrak f}_\infty(q^{-1/2}z) .$$
In this limit, the currents ${\mathfrak m}_\infty(z)$ and $\widetilde{\mathfrak m}_\infty$ act on symmetric functions as follows:
\begin{eqnarray}
{\mathfrak m}_\infty(z)\, F[X] &=&
\frac{1}{1-t^{-1}} \frac{C_\infty(t^{-1}z)}{C_\infty(z)} F\left[ X+\frac{q-1}{z}\right],\label{bosom}\\
\widetilde{\mathfrak m}_\infty(z)\, F[X] &=&
\frac{1}{1-t} \frac{C_\infty(t q^{-1}z)}{C_\infty(q^{-1}z)} F\left[ X-\frac{q-1}{z}\right].\label{bosomt}
\end{eqnarray}


Note that as the prefactor only involves the power sums $p_k$ with $k>0$ (via the function $C_\infty$), we may restrict the action to power series $F\in {\mathcal P}_+$,
where ${\mathcal P}_+$ is the space of formal power series of the $\{p_k\}_{k\in \Z_{>0}}$.



The action (\ref{bosom}-\ref{bosomt}) can now be compared with the definition of the difference operators $D_k,D_k^*$ of \cite {BGLX}. These were 
defined for $k\in \Z_+$ only, via a plethystic formulation. However, it is easy to extend the definition to
all $k\in \Z$, by considering the  
generating currents $D(z):=\sum_{k\in \Z} z^k D_k$ and $D^*(z):=\sum_{k\in \Z} z^k D_k^*$. 
Their action on symmetric functions $F[X]\in {\mathcal P}_+[X]$, $X$
an alphabet
of infinitely many variables $x_1,x_2,...$, is:
\begin{eqnarray}
D(z)\, F[X]&=&C_\infty(z)\, F\left[X+\frac{(1-t)(1-q)}{z}\right] \label{fromBG}, \\
D^*(z)\, F[X]&=&\frac{1}{C_\infty(z)}\, F\left[X-\frac{(1-t^{-1})(1-q^{-1})}{z}\right] .\label{fromBGstar} 
\end{eqnarray}
In \cite{BGLX}, the commutation relations between the $D_k$'s and the $D_k^*$'s are derived. 
They extend to the following commutator of currents:
\begin{equation}\label{BGcurrent}
[D(z),D^*(\frac{w}{q t})]F[X]=
\frac{(1-t)(1-q)}{q t-1} \left\{\delta(z/w)\frac{C_\infty(z)}{C_\infty(\frac{z}{q t})}F[X]-
\delta(q tz/w)F\Big[ X+\frac{(1-t)(1-q)(1-\frac{1}{qt})}{z}\Big] \right\}
\end{equation}
easily derived from the commutation relation of Theorem \ref{bobothm}.
In components, this reads:
$$[D_a,D_b^*]F[X]=\frac{(1-t)(1-q)}{q t-1} \left\{(qt)^b h_{a+b}[X(\frac{1}{qt}-1)]F[X]-
F\Big[ X+\frac{(1-t)(1-q)(1-\frac{1}{qt})}{z}\Big]\vert_{z^{a+b}} \right\}$$
where the notation $F(z)\vert_{z^{a+b}}$ stands for the coefficient of $z^{a+b}$ in the current $F$,
and $h_n[X]$ are the complete symmetric functions of the alphabet $X$.
Note that the case $a,b\geq 0$ agrees with the commutator $[D_a,D_b^*]$ of \cite{BGLX}.


\begin{thm}\label{BGconnect}
Let $\Sigma$ be the operator acting by the plethysm $[X]\mapsto [X/(t-1)]$ 
(accordingly $\Sigma^{-1}$ acts by $[X]\mapsto [X(t-1)]$). 
Then we have the following identities between operators acting on ${\mathcal P}_+[X]$:
\begin{eqnarray*} 
{\mathfrak m}_\infty(z)&=& \frac{1}{1-t^{-1}} \left(\Sigma^{-1} D(z) \Sigma\right)\Big\vert_{t\to t^{-1}}, \\
\widetilde{\mathfrak m}_\infty(z)&=& \frac{1}{1-t} \left(\Sigma^{-1} D^*\left(\frac{z}{q t}\right) \Sigma\right)\Big\vert_{t\to t^{-1}} .
\end{eqnarray*}
\end{thm}
\begin{proof}
The bosonized expressions are matched using \eqref{fromBG} and \eqref{fromBGstar} and 
writing  $F=\Sigma G$. Then, we have:
$$F\left[X\pm \frac{(1-t)(1-q)}{u}\right] =G\left[\frac{X}{t-1}\pm \frac{q-1}{u}\right],$$
while
$$\Sigma^{-1} \, C_\infty(u)\, \Sigma=\frac{C_{\infty}(t u)}{C_{\infty}(u)}, \quad 
\Sigma^{-1} \, \frac{1}{C_\infty(\frac{u}{q t})}\, \Sigma=\frac{C_{\infty}(t^{-1}q^{-1}u)}{C_{\infty}(q^{-1} u)} .$$
Hence 
\begin{eqnarray*}
\Sigma^{-1}D(u)\Sigma \,G[X]&=&\Sigma^{-1}D(u)F[X]= \frac{C_{\infty}(t u)}{C_{\infty}(u)} \, G\left[X + \frac{q-1}{u}\right],\\
\Sigma^{-1}D^*(\frac{u}{q t})\Sigma \,G[X]&=&\Sigma^{-1}D^*(\frac{u}{q t})F[X]=\frac{C_{\infty}(t^{-1}q^{-1}u)}{C_{\infty}(q^{-1}u)}  \, G\left[X - \frac{q-1}{u}\right],
\end{eqnarray*}
where the action of $\Sigma^{-1}$ on $C_\infty(u)^{\pm 1}G\left[\frac{X}{t-1}\pm \frac{q-1}{u}\right]$ is expressed
by considering the latter as a function of $[X]$.
The Theorem follows by taking $t\to t^{-1}$ in the above, and comparing with eqns. (\ref{bosom}-\ref{bosomt}).
\end{proof}

%Note that the commutation relation \eqref{BGcurrent} is equivalent to the $N\to\infty$ limit of
%the commutation relation of Theorem \ref{commuef}, once expressed in terms of the ${\mathfrak m}$ currents.

\begin{remark}\label{betternablarem}
Theorem \ref{BGconnect} allows to refine Remark \ref{nablarem} as follows.
The transformation relating the Macdonald polynomials $P_\lambda$ to the modified Macdonald polynomials
${\widetilde H}_\lambda$ is the following \cite{macdo}:
$${\widetilde H}_\lambda[X]=\phi_\lambda(t) \, P_\lambda\left[ \frac{X}{t-1}\right]\Big\vert_{t\to t^{-1}}
=\phi_\lambda(t) \, (\Sigma \, P_\lambda)\vert_{t\to t^{-1}}, $$
where $\phi_\lambda(t)$ is a normalization factor independent of the $x_i$'s.
The $\nabla$ operator of \cite{BG} has eigenvectors ${\widetilde H}_\lambda$
with eigenvalues $T_\lambda:=t^{n(\lambda)}q^{n(\lambda')}$,
i.e.  $\nabla\, {\widetilde H}_\lambda=T_\lambda \, {\widetilde H}_\lambda$. We deduce that
$$(\Sigma^{-1}\, \nabla\, \Sigma)\vert_{t\to t^{-1}} \, P_\lambda= T_{\lambda}\vert_{t\to t^{-1}}\, P_\lambda
=t^{-n(\lambda)}q^{n(\lambda')}\, P_\lambda. $$
We finally identify
\begin{equation}\label{nabnab}
\eta^{-1}=\nabla^{(N)}=C_N\, (t^{\frac{N-1}{2}}q^{\frac{1}{2}})^{d}\, (\Sigma^{-1}\, \nabla\, \Sigma)\vert_{t\to t^{-1}},
\end{equation}
where $d$ acts on Macdonald polynomials as $d\, P_\lambda =|\lambda| \, P_\lambda$, $C_N$ as in Remark \ref{nablarem},
and $\nabla$ is restricted to act on symmetric functions of $x_1,x_2,...,x_N$. The element $\eta$ in the completion of the DAHA was defined in Equation \eqref{etadefn}.
\end{remark}

\subsubsection{Bosonization formulas for ${\mathfrak e}_\infty(z),{\mathfrak f}_\infty(z)$
in the $N\to\infty$ limit}

The so-called bosonization of the currents uses the operators $p_k$ and $\frac{\partial}{\partial p_k}$, $k\in \Z^*$. 
%For an infinite collection $X$, the functions $\{p_k[X]: k\in \Z^*\}$ are algebraically 
%independent, so we may consider the $p_k$'s as independent variables.
They obey the following commutation relations
$$
\left[ \frac{\partial}{\partial p_j}, p_k\right] = \delta_{jk}
$$
interpreted as independent harmonic oscillator relations. These relations still hold when evaluated on the infinite collection $X=(x_1,x_2,...)$, as the $p_k[X]$ remain independent variables.
We call the expression of the currents in terms of these elements evaluated on the infinite collection $X$, a bosonization.

As $\{p_k[X], k\in\Z^*\}$ are independent, a plethysm which adds one additional variable to the alphabet can be written as
$$F[X+\mu]=\exp\left\{ \sum_{k\neq 0} \mu^k \frac{\partial}{\partial p_k[X]}\right\} F[X], $$
as readily seen from multiple Taylor expansion.
This allows to rewrite Theorem \ref{bobothm} as
\begin{thm}
The limiting Macdonald currents ${\mathfrak e}_\infty(z)$, ${\mathfrak f}_\infty(z)$ and Cartan currents $\psi_\infty^\pm(z)$
action on ${\mathcal P}[X]$ can be expressed in terms of the Heisenberg algebra generators as follows:
\begin{eqnarray*}
{\mathfrak e}_\infty(z)&=&\frac{q^{1/2}}{(1-q)(1-t^{-1})} \, 
e^{\sum_{k> 0} p_k[X] \frac{q^{k/2}(1-t^{-k})z^k}{k}} \, 
e^{\sum_{k\neq 0} \frac{q^{k/2}-q^{-k/2}}{z^k}\frac{\partial}{\partial p_k[X]}}, \nonumber \\
{\mathfrak f}_\infty(z)&=&\frac{q^{-1/2}}{(1-q^{-1})(1-t^{-1})}
e^{\sum_{k>0}p_{k}[X]\frac{q^{-k/2}(1-t^{k})z^{k}}{k} }\,   
e^{-\sum_{k\neq 0} \frac{q^{k/2}-q^{-k/2}}{z^k}\frac{\partial}{\partial p_k[X]}}, \nonumber \\
\psi_\infty^+(z)&=& e^{\sum_{k>0} p_{k}[X]\frac{(t^{k/2}-t^{-k/2})((qt^{-1})^{k/2}-(qt^{-1})^{-k/2})z^k}{k}},\nonumber \\
\psi_\infty^-(z)&=& e^{\sum_{k\neq 0} \frac{(q^{k/2}-q^{-k/2})(1-(t q^{-1})^{k})}{z^{k}}\frac{\partial}{\partial p_k[X]}}.
\nonumber
\end{eqnarray*}
\end{thm}

\begin{remark}
When restricted to ${\mathcal P}_+[X]$ (i.e. dropping all $k<0$ summations), these expressions are identical to the bosonized expressions for the level $(1,0)$ representation of the quantum toroidal algebra \cite{FHHSY} up to simple redefinitions of the generators. This non-trivial central charge is a feature of the $N\to\infty$ limit. 
More precisely, we have the correspondence (see \eqref{corresp} above):
\begin{eqnarray*}
&&\eta(z)=\frac{(1-q)(1-t^{-1})}{q^{1/2}}{\mathfrak e}_\infty(q^{-1/2}z),\ 
\xi(z)=\frac{(1-q^{-1})(1-t^{-1})}{q^{-1/2}}{\mathfrak f}_\infty(t^{-1/2}z),\\ 
&&\varphi^+(z)=\psi_\infty^-(q^{-3/4}t^{1/4}z^{-1}),
\ \varphi^-(z)=\psi_\infty^+(q^{-1/4}t^{-1/4}z^{-1})
\end{eqnarray*}
while the oscillator modes $a_k$, $k\in \Z_{>0}$ correspond to:
$$ a_{-k}=p_k[X],\quad a_k=k\frac{1-q^k}{1-t^k}\,\frac{\partial}{\partial p_k[X]}$$
Note that for finite $N$, the representation $(0,0)$ was completely different, and corresponded to taking 
{\it two} mutually commuting families of harmonic oscillators, namely $a_k=p_k$ for $ k>0$, and $a_k=p_k$ for $k<0$
together with their respective adjoints $\partial/\partial p_k$ for $k>0$ and for $k<0$. 
This is related to the fact that for $\ell_1=0$,
i.e. ${\hat \gamma}=1$, the modes $a_k$ and $a_{-\ell}$ commute for all $k,\ell>0$, leading to commuting $\psi^\pm$
whereas they don't when $\ell_1=1$, and $\psi^\pm$ don't commute either.
\end{remark}


\subsection{Plethystic formulas in the $t\to\infty$ limit}

We now investigate the dual Whittaker limit $t\to\infty$ of the plethystic expressions for the currents. To this end, we use the definition of the limiting currents
${\mathfrak e}^{(\infty)},{\mathfrak f}^{(\infty)},\psi^{\pm(\infty)}$ as in \eqref{limefpsi}.

First, using \eqref{psilim}, we are led to introduce the notation $A$ for the following symmetric function:
$$A:=x_1x_2\cdots x_N.$$
The function ${\rm Log}\, A$ can be understood as a renormalized version of the power sum $p_0[X]$, namely:
$${\rm Log}\, A= \lim_{\epsilon\to 0} \frac{p_\epsilon[X] -N}{\epsilon} .$$
By a slight abuse of notation we shall write $A=e^{p_0}$.
In the dual Whittaker limit,
\begin{eqnarray*}
\psi^{+(\infty)}(z)&=&\frac{(-q^{-1/2}z)^N\,  A}{C(q^{-1/2}z)C(q^{1/2}z)}=
(-q^{-1/2}z)^N\, e^{p_0+\sum_{k>0} p_k(q^{k/2}+q^{-k/2})\frac{z^k}{k}},\\
\psi^{-(\infty)}(z)&=&\frac{(-q^{-1/2}z^{-1})^N\,  A^{-1}}{{\widetilde C}(q^{-1/2}z){\widetilde C}(q^{1/2}z)}=
(-q^{-1/2}z^{-1})^N\, e^{-p_0+\sum_{k>0} p_{-k}(q^{k/2}+q^{-k/2})\frac{z^{-k}}{k}}.
\end{eqnarray*}

The necessity of the introduction of the quantity $A$ (or $p_0$) imposes on us to consider it as an independent
variable, so that plethysms will act on the space ${\mathcal P}_0$ of functions that are power series of all $p_k$, $k\in \Z$ (including $k=0$). The plethysms must therefore acquire a $p_0$ dependence as well.
More precisely, we must take into account an extra commutation relation in addition to those of Lemma \ref{compk}.
\begin{lemma}
We have the relations:
\begin{equation}
{\mathcal D}^{q,t}_{1;n}\, p_0[X]=(p_0[X]+{\rm Log}\, q)  \, {\mathcal D}^{q,t}_{1;n},\quad {\rm and}\quad 
{\mathcal D}^{q^{-1},t^{-1}}_{1;n}\, p_0[X]=(p_0[X]-{\rm Log}\, q)\, {\mathcal D}^{q^{-1},t^{-1}}_{1;n}.
\end{equation}
\end{lemma}
\begin{proof}
We use the $q$-commutation relations:
\begin{equation}
{\mathcal D}^{q,t}_{1;n}\, A=q \, A\, {\mathcal D}^{q,t}_{1;n},\quad {\rm and}\quad 
{\mathcal D}^{q^{-1},t^{-1}}_{1;n}\, A=q^{-1} A \, {\mathcal D}^{q^{-1},t^{-1}}_{1;n},
\end{equation}
due to $\Gamma_i^{\pm 1} \, A= q^{\pm 1} A \, \Gamma_i$.
\end{proof}

We now extend plethysms to functions of the $p_k$, $k\in \Z$ so that, in the case of addition of one variable,
we have:
$$p_k[X+\mu]=\left\{ \begin{matrix} 
p_k[X]+\mu^k & {\rm if}\ k\neq 0\\
p_0[X]+{\rm Log}\, \mu & {\rm if}\  k=0
\end{matrix} \right. \ .$$
As a check, 
$$p_k\left[X+\frac{q^{1/2}-q^{-1/2}}{z}\right]=\left\{ \begin{matrix} 
p_k[X]+\frac{q^{k/2}-q^{-k/2}}{z^k}  & {\rm if}\ k\neq 0\\
p_0[X]+{\rm Log}\, q & {\rm if}\  k=0
\end{matrix} \right. \ .$$
Note that the notation $\left[X+\mu \right]$ now refers to the full plethysm involving all the $p_k$, $k\in \Z$.
With this notation, we easily get the action of the currents on functions 
$F[X]\in {\mathcal P}_0[X]$, i.e. formal power series of all 
$p_k[X]$, $k\in \Z$:
\begin{eqnarray*}
{\mathfrak e}^{(\infty)}(z)\, F[X]&=&\frac{q^{1/2}}{1-q} \left\{
\frac{1}{C(q^{1/2}z)}- 
\frac{(-q^{-1/2}z^{-1})^N A^{-1}}{{\widetilde C}(q^{1/2}z^{-1})}\right\} 
F\left[X+\frac{q^{1/2}-q^{-1/2}}{z} \right], \nonumber \\
{\mathfrak f}^{(\infty)}(z)\, F[X]&=&\frac{q^{-1/2}}{1-q^{-1}} \left\{
\frac{1}{{\widetilde C}(q^{1/2}z^{-1})}- 
\frac{(-q^{-1/2}z )^N A}{C(q^{-1/2}z)}\right\} 
F\left[X-\frac{q^{1/2}-q^{-1/2}}{z} \right]. \nonumber
\end{eqnarray*}

%or equivalently:
%\begin{eqnarray*}
%{\mathfrak e}^{(\infty)}(z)&=&\frac{q^{1/2}}{1-q} \left\{
%e^{\sum_{k>0} p_k \frac{q^{k/2}z^k}{k}}-
%e^{-p_0+\sum_{k>0}p_{-k}\frac{q^{-k/2}z^{-k}}{k} } \right\} \, q^{\frac{\partial}{\partial p_0}}\,
%e^{\sum_{k\neq 0} \frac{q^{k/2}-q^{-k/2}}{z^k}\frac{\partial}{\partial p_k}} \nonumber \\
%{\mathfrak f}^{(\infty)}(z)&=&\frac{q^{-1/2}}{1-q^{-1}}
%\left\{e^{\sum_{k>0}p_{-k}\frac{q^{k/2}z^{-k}}{k} }-
%e^{p_0+\sum_{k>0}p_{k}\frac{q^{-k/2}z^{k}}{k} }\right\}\, q^{-\frac{\partial}{\partial p_0}}\,
%e^{-\sum_{k\neq 0} \frac{q^{k/2}-q^{-k/2}}{z^k}\frac{\partial}{\partial p_k}} \nonumber
%\end{eqnarray*}



