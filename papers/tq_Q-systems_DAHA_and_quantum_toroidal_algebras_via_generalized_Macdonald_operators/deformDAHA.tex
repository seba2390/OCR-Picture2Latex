%DAHA
%Realization of difference operators in terms of DAHA

In this section, we find the explicit elements in the spherical double affine Hecke algebra (sDAHA) whose functional representation are the generalized Macdonald operators $\mathcal M_{\al;n}$ of Equation \eqref{genmacdop}. The standard definitions and properties in this section can be found in Cherednik's book
\cite{Cheredbook}.

\subsection{The $A_{N-1}$ DAHA: Definition and relations}

\subsubsection{Generators and relations}\label{defdahasec}
Let $q$ and $\theta$ be indeterminates, where $\theta=t^{\half}$. 
The $A_{N-1}$ double affine Hecke algebra is the algebra generated 
over $\C(q,t)$ by the generators $\{X_i,Y_i,T_j: i\in [1,N], j\in [1,N-1]\}$ ,
%where $\{T_j\}_{j\in[1,n-1]}$ generate the finite Hecke algebra, 
subject to the following relations:
\begin{eqnarray}
 T_i\, T_{i+1}\,T_i&=&T_{i+1}\, T_i\,T_{i+1}; \label{braid}\\
  (T_i-\theta)(T_i+\theta^{-1})&=&0; \nonumber \\
 T_i\, X_i\,T_i&=&X_{i+1}; \qquad T_i^{-1}\,Y_i\,T_i^{-1}=Y_{i+1},\qquad (1\leq i\leq N-1);\label{TY}\\
 T_i\, X_j&=&X_j\, T_i; \qquad
  T_i\, Y_j =Y_j\, T_i ,\quad (j\neq i,i+1);\nonumber\\
 X_1\,Y_{2}&=&Y_2\,T_1^{2}\,X_1;\nonumber \\
X_i X_j &=& X_j X_i;\qquad Y_i Y_j = Y_j Y_i; \nonumber \\ 
{Y_1\cdots Y_N}\, X_j&=&q \, X_j\, {Y_1\cdots Y_N};\nonumber \\
 {X_1\cdots X_N}\, Y_j&=&q^{-1}\, Y_j\, {X_1\cdots X_N} .\label{prodXy}
\end{eqnarray}

\subsubsection{Other useful relations}
We list here several useful relations among the generators of the DAHA which follow from the definition above.
Some of them can be found in \cite{Cheredbook} (see Sect. {\bf Some relations} on pp 103-104, eqs. (1.4.64-69)).

The following relations give a way to reorder $X_i$ and $Y_{j}$:
\begin{eqnarray*}
X_i\,Y_{i+1}&=&Y_{i+1}\,T_i^{2}\,X_i=Y_{i+1}\,T_i \,X_{i+1}\,T_i^{-1}=
T_i^{-1}\,Y_i\, T_i \, X_i,\qquad (i=1,2,...,N-1);\label{firstxy}\\
Y_i \, X_{i+1}&=&X_{i+1} \, T_i^{-2} \, Y_i=X_{i+1} \, T_i^{-1}\, Y_{i+1}\,T_i=T_i\,X_i\,T_i^{-1}\,Y_i,\qquad (i=1,2,...,N-1).\label{secxy}
\end{eqnarray*}
%The latter is readily obtained from the former by multiplying from the left and right by $T_i$.
More generally, let $1\leq i\leq j \leq N$. Then
\begin{eqnarray*} X_i \,Y_{j+1}&=&Y_{j+1}(T_jT_{j-1}\cdots T_{i+1} T_i^2 T_{i+1}^{-1}T_{i+2}^{-1}\cdots T_j^{-1})X_i\\
&=&Y_{j+1}(T_jT_{j-1}\cdots T_i)X_{i+1}(T_i^{-1}T_{i+1}^{-1}\cdots T_j^{-1});\\
Y_i \,X_{j+1}&=&X_{j+1}(T_j^{-1}T_{j-1}^{-1}\cdots T_{i+1}^{-1} T_i^{-2} T_{i+1}T_{i+2}\cdots T_j) Y_i\\
&=&X_{j+1}(T_j^{-1}T_{j-1}^{-1}\cdots T_i^{-1})Y_{i+1}(T_iT_{i+1}\cdots T_j).
\end{eqnarray*}
Moreover,
\begin{eqnarray*}
(X_j X_{j-1}\cdots X_i )\, Y_{j+1}
&=&Y_{j+1} (T_jT_{j-1}\cdots T_{i+1}T_i^2T_{i+1}T_{i+2}\cdots T_{j}) (X_j X_{j-1}\cdots X_i)\nonumber \\
&=& Y_{j+1} (T_jT_{j-1}\cdots T_i) (X_{j+1} X_{j}\cdots X_{i+1} ) (T_i^{-1}T_{i+1}^{-1}\cdots T_j^{-1})\label{comXY} .
\end{eqnarray*}
This equation can be iterated to obtain
\begin{equation}\label{gencoXY}
(X_j X_{j-1}\cdots X_i )^n\, Y_{j+1}
=Y_{j+1} (T_jT_{j-1}\cdots T_i) (X_{j+1} X_{j}\cdots X_{i+1})^n (T_i^{-1}T_{i+1}^{-1}\cdots T_j^{-1}).
\end{equation}

%The next Lemma will also be used below, and is a consequence of the defining relations.
\begin{lemma}\label{comXT}
For all $1\leq i\leq j\leq N$, 
$$(X_iX_{i+1}\cdots X_j)(T_i^{-1}T_{i+1}^{-1}\cdots T_{j-1}^{-1})
=(T_i^{-1}T_{i+1}^{-1}\cdots T_{j-1}^{-1})(X_iX_{i+1}\cdots X_j).$$
\end{lemma}
\begin{proof}
For all $1\leq i\leq j\leq N$, we have:
\begin{eqnarray*}(X_iX_{i+1}\cdots X_j)(T_i^{-1}T_{i+1}^{-1}\cdots T_{j-1}^{-1})
&=&(X_iX_{i+1}\cdots X_{j-1})(T_i^{-1}T_{i+1}^{-1}\cdots T_{j-2}^{-1})T_{j-1}X_{j-1}\\
&=&X_i (T_iX_i \cdots T_{j-2}X_{j-2}T_{j-1}X_{j-1})\\
&=&T_i^{-1}X_{i+1}X_i(T_{i+1}X_{i+1}\cdots T_{j-1}X_{j-1})\\
&=&T_i^{-1}X_{i+1}(T_{i+1}X_{i+1}\cdots T_{j-1}X_{j-1})X_i\\
&=&T_i^{-1}T_{i+1}^{-1}X_{i+2}(T_{i+2}X_{i+1}\cdots T_{j-1}X_{j-1})X_iX_{i+1}\\
&=&(T_i^{-1}T_{i+1}^{-1}\cdots T_{j-1}^{-1})(X_iX_{i+1}\cdots X_j).
\end{eqnarray*}
The lemma follows.
\end{proof}

\subsubsection{The generator $\pi$}

Define
$$\pi=Y_1^{-1}T_1T_2\cdots T_{N-1} . $$
Using $Y_{i+1}=T_i^{-1}\,Y_i\,T_i^{-1}$, we may express each of the $Y_i$s as:
$$Y_i=T_{i}T_{i+1}\cdots T_{N-1} \pi^{-1} T_1^{-1}T_2^{-1}\cdots T_{i-1}^{-1}\qquad (i=1,2...,N).$$
In other words, we may express $\pi$ in $N$ different manners:
\begin{equation}\label{piti}
\pi=T_1^{-1}T_2^{-1}\cdots T_{i-1}^{-1} Y_i^{-1} T_iT_{i+1}\cdots T_{N-1} \qquad (i=1,2,...,N).
\end{equation}

The following two Lemmas show that $\pi$ acts as a translation operator on $T_i$s and $X_i$s:
\begin{lemma}
$$\pi T_i=T_{i+1}\pi \qquad (i=1,2,...,N-2)$$
\end{lemma}
\begin{proof}
Using the $i$th expression for $\pi$ \eqref{piti}, we compute:
\begin{eqnarray*}\pi\, T_i&=&(T_1^{-1}T_2^{-1}\cdots T_{i-1}^{-1}) Y_i^{-1} (T_iT_{i+1}T_i)( T_{i+2}\cdots T_{N-1})\\
&=&(T_1^{-1}T_2^{-1}\cdots T_{i-1}^{-1}) Y_i^{-1}T_{i+1}(T_iT_{i+1}\cdots T_{N-1})\\
&=&T_{i+1}(T_1^{-1}T_2^{-1}\cdots T_{i-1}^{-1}) Y_i^{-1}(T_iT_{i+1}\cdots T_{N-1})=T_{i+1}\pi
\end{eqnarray*}
where we have first used the braid relations \eqref{braid} and the commutation relations\eqref{TY}.
\end{proof}



\begin{lemma}
$$\pi \, X_i=X_{i+1}\,\pi \qquad (i=1,2,...,N-1)\quad {\rm and} \quad \pi X_{N}=q^{-1}\, X_1\,\pi$$
\end{lemma}
\begin{proof}
Using
$Y_i X_{i+1}=X_{i+1} T_i^{-2} Y_i$,
wich implies $Y_i^{-1}X_{i+1} T_i^{-2}=X_{i+1}Y_i^{-1}$,
and the expression \eqref{piti} for $\pi$, we compute:
\begin{eqnarray*}
\pi X_i&=&(T_1^{-1}T_2^{-1}\cdots T_{i-1}^{-1}) Y_i^{-1} (T_i T_{i+1}\cdots T_{N-1})  X_i=
(T_1^{-1}T_2^{-1}\cdots T_{i-1}^{-1}) Y_i^{-1}T_i X_i (T_{i+1}\cdots T_{N-1})\\
&=&(T_1^{-1}T_2^{-1}\cdots T_{i-1}^{-1}) Y_i^{-1}X_{i+1}T_i^{-1} (T_{i+1}\cdots T_{N-1})\\
&=&X_{i+1}(T_1^{-1}T_2^{-1}\cdots T_{i-1}^{-1}) Y_i^{-1}T_i^{-1} (T_{i+1}\cdots T_{N-1})=X_{i+1}\pi
\end{eqnarray*}
The last relation is obtained by using 
$$\pi {X_1\cdots X_N}=q^{-1}{X_1\cdots X_N}\pi,$$ 
obtained from the relation \eqref{prodXy}.
\end{proof}


\subsection{The functional representation of the DAHA}\label{secpol}

Since the variables $X_1,...,X_N$ commute among themselves, we can define the functional representation $\rho$ of the DAHA acting on $V=C_{q,t}(x_1,...,x_N)$ as follows:
\begin{eqnarray*}
\rho(X_i)\, f(x_1,x_2,...,x_N)&=& x_i\, f(x_1,x_2,...,x_N), \quad f\in V;\\
\rho(s_i)\, f(x_1,...,x_i,x_{i+1},...,x_N) &=&f(x_1,...,x_{i+1},x_{i},...,x_N),\quad f\in V; \\
\rho(T_i)&=&\theta \rho(s_i) +\frac{\theta-\theta^{-1}}{x_ix_{i+1}^{-1}-1} (\rho(s_i)-1)\\
&=&
\frac{\theta x_i-\theta^{-1}x_{i+1}}{x_i-x_{i+1}}\rho(s_i)-x_{i+1}\frac{\theta-\theta^{-1}}{x_i-x_{i+1}};\\
\rho(T_i^{-1})&=&\rho(T_i)-\theta+\theta^{-1};\\&=&
\frac{\theta x_i-\theta^{-1}x_{i+1}}{x_i-x_{i+1}}\rho(s_i)-x_{i}\frac{\theta-\theta^{-1}}{x_i-x_{i+1}};\\
\rho(\pi)\, f(x_1,x_2,...,x_{N})&=& f(x_2,x_3,...,x_{N},q^{-1}x_1), \quad f\in V; \\
\rho(Y_i)&=&\rho(T_{i}T_{i+1}\cdots T_{N-1} \pi^{-1} T_1^{-1}T_2^{-1}\cdots T_{i-1}^{-1})\qquad (i=1,2,...,N).\\
\end{eqnarray*}

%In other words, $\pi$ acts on functions of $x_1,...,x_{N}$ via the substitutions:
%\begin{eqnarray*} \pi:(x_1,x_2,...,x_{N})&\mapsto& (x_2,x_3,...,x_{N},q^{-1}x_1)\\
%\pi^{-1}:(x_1,x_2,...,X_{N})&\mapsto& (qx_{N},x_1,x_2...,x_{N-1})
%\end{eqnarray*}
We see that the $q$-shift operators $\Gamma_i$ of \eqref{shift} are a representation of the following element in the DAHA:
\begin{equation} \label{defgamma}
\Gamma_i=\rho(s_i s_{i+1} \cdots s_{N-2}s_{N-1} \pi^{-1}s_1 s_2\cdots s_{i-1}),\quad i=1,2,...,N.
\end{equation}


%\section{Generalized Macdonald difference operators}

\subsection{Macdonald difference operators}

The operators $Y_1,...,Y_{N}$ commute among themselves. Therefore one can define
the elementary symmetric functions $e_m(Y_1,...,Y_{N})$ unambiguously. 
\begin{defn} The Macdonald operators are
\begin{equation}\label{orimac}
{D}_\al:=e_\al(Y_1,...,Y_{N}),\qquad (\al=0,1,2,...,N). \end{equation}
\end{defn}
Equivalently one can write
$\sum_{\al=0}^{N} z^\al {D}_\al=\prod_{i=1}^{N} (1+z Y_i)$.

It is well-known that the operators  $\rho({D}_\al)$ act on the space ${\mathcal S}_N$  of symmetric {\it functions} in the variables $x_1,...,x_{N}$.
The following is a standard result of Macdonald theory:
\begin{thm}\label{macdopthm}
The restriction of the operators $\rho({D}_\al)$ to ${\mathcal S}_N$ is
$$\rho({D}_\al)\vert_{{\mathcal S}_N}=:{\mathcal D}_\al=\theta^{-\al(N-\al)}\, \widetilde{\mathcal D}_{\al}$$
where $\widetilde{\mathcal D}_\al$ are the Macdonald operators \eqref{macdop}.
\end{thm}

%The difference operators ${\mathcal D}_\al$ are the Macdonald difference operators 
%(in Cherednik's normalization).
%Equivalently we also have:
%$$\sum_{\al=0}^{r+1} z^\al {\mathcal D}_\al=\prod_{i=1}^{r+1} (1+z Y_i)$$
%expressed in the spherical DAHA polynomial representation.

\subsection{More commuting operators}

In this section, we introduce families of commuting operators $\{Y_{i,n}\}_{i\in [1,N]}$ for each $n\in \Z$. These are related
to Cherednik's $SL_2(\Z)$ action on $Y_i$ by $n$ iterations of the generator $\tau_+$.

\subsubsection{Definition and commutation}

\begin{defn}
We introduce the family of operators:
%\begin{equation} 
$$
{Y}_{i,n} = (X_1 X_2 \cdots X_{i-1})^{-n}\, Y_i \, (X_1 X_2 \cdots X_i)^n ,\qquad (i=1,2,...,N;n\in \Z).
$$
%\end{equation}
\end{defn}

In particular, $Y_{i,0}=Y_i$, and $Y_{1,n}=Y_1 X_1^n$. We also see that
$$Y_{N,n}=(X_1\cdots X_{N-1})^{-n}Y_{N}\,(X_1\cdots X_N)^n=q^nX_N^n Y_N.$$

\begin{lemma}\label{commuYin}
For fixed $n\in \Z$, the elements $\{Y_{i,n}: i\in[1,N]\}$ commute among themselves:
$$Y_{i,n}\, Y_{j,n}=Y_{j,n}\, Y_{i,n}\qquad \forall\, i,j\in[1,N].$$
\end{lemma}
\begin{proof}
Writing $j=i+k$, $k>0$, we have:
\begin{eqnarray*}
(X_1\cdots X_{i-1})^nY_{i,n}&& \!\!\!\!\!\!\!\!\!\!\!\!\!\! Y_{i+k,n}(X_1\cdots X_{i})^{-n}\\
&=&Y_i (X_{i+1}\cdots X_{i+k-1})^{-n}Y_{i+k} (X_{i+1}\cdots X_{i+k})^n\\
&=& Y_{i+k}Y_i  (T_{i+k-1}\cdots T_{i+1})(X_{i+2}\cdots X_{i+k})^{-n} 
(T_{i+1}^{-1}\cdots T_{i+k-1}^{-1})(X_{i+1}\cdots X_{i+k})^n\\
&=&  Y_{i+k}(T_{i+k-1}\cdots T_{i+1}) (Y_i  X_{i+1}^n) (T_{i+1}^{-1}\cdots T_{i+k-1}^{-1})\\
&=&  Y_{i+k}(T_{i+k-1}\cdots T_i)X_i^n(T_i^{-1}\cdots T_{i+k-1}^{-1})Y_i 
\end{eqnarray*}
where we have first used the relation \eqref{gencoXY}:
$$(X_{i+1}\cdots X_{i+k-1})^{-n}Y_{i+k}=Y_{i+k}(T_{i+k-1}\cdots T_{i+1})(X_{i+2}\cdots X_{i+k})^{-n} 
(T_{i+1}^{-1}\cdots T_{i+k-1}^{-1})$$
then Lemma \ref{comXT}:
$$(T_{i+1}^{-1}\cdots T_{i+k-1}^{-1})(X_{i+1}\cdots X_{i+k})=(X_{i+1}\cdots X_{i+k})(T_{i+1}^{-1}\cdots T_{i+k-1}^{-1})$$
and finally $Y_i  X_{i+1}^n=T_iX_i^nT_i^{-1} Y_i$ by iteration of \eqref{secxy}.
Likewise, we have:
\begin{eqnarray*}
(X_1\cdots X_{i-1})^nY_{i+k,n}&& \!\!\!\!\!\!\!\!\!\!\!\!\!\! Y_{i,n}(X_1\cdots X_{i})^{-n}\\
&=&(X_{i}\cdots X_{i+k-1})^{-n}Y_{i+k} (X_{i}\cdots X_{i+k})^{n}Y_i\\
&=&Y_{i+k} (T_{i+k-1}\cdots T_i)(X_{i+1}\cdots X_{i+k})^{-n}(T_i^{-1}\cdots T_{i+k-1}^{-1})(X_{i}\cdots X_{i+k})^{n}Y_i\\
&=&Y_{i+k} (T_{i+k-1}\cdots T_i)X_i^n (T_i^{-1}\cdots T_{i+k-1}^{-1})Y_i\\
\end{eqnarray*}
by use of the relations
\begin{eqnarray*}
(X_{i}\cdots X_{i+k-1})^{-n}Y_{i+k}&=& Y_{i+k}(T_{i+k-1}\cdots T_{i})(X_{i+1}\cdots X_{i+k})^{-n} 
(T_{i}^{-1}\cdots T_{i+k-1}^{-1})\\
(T_i^{-1}\cdots T_{i+k-1}^{-1})(X_{i}\cdots X_{i+k})&=& (X_{i}\cdots X_{i+k})(T_i^{-1}\cdots T_{i+k-1}^{-1})
\end{eqnarray*}
The lemma follows.
\end{proof}

\subsubsection{Expression in the functional representation}

From this point on, we will work in the functional representation $\rho$ of Section~\ref{secpol}.
We introduce the following symmetric function of $X_1,...,X_{N}$:
\begin{equation}\label{gammadefn}
\gamma:=\exp\left\{ \sum_{i=1}^{N} \frac{{\rm Log}(X_i)^2}{2 {\rm Log}(q)}\right\} 
\end{equation}
This element does not belong to the DAHA, but to a suitable completion (see \cite{Cheredbook}). Nevertheless, it has some useful commutation relations with elements of the DAHA.
%Alternatively, writing $X_i=q^{\xi_i}$, we may write $\gamma=q^{\sum_{i=1}^{N}\frac{\xi_i^2}{2}}$.
%We have the following:

\begin{lemma}\label{piga}
$$\pi^{-1} \, \gamma=q^{\frac{1}{2}}X_{N} \,  \gamma \, \pi^{-1} $$
\end{lemma}
\begin{proof}
Using $\pi^{-1}X_i=X_{i-1}\pi^{-1},$ for $i\in[2,N]$ and $\pi^{-1}X_1=q X_{N}\pi^{-1}$,
we compute
$$\pi^{-1} \left(\sum_{j=1}^{r+1} \frac{{\rm Log}(X_j)^2}{2 {\rm Log}(q)}\right)=
\left(\sum_{j=1}^{r+1} \frac{{\rm Log}(X_j)^2}{2 {\rm Log}(q)}+{\rm Log}(X_{r+1})+\frac{{\rm Log}(q)}{2}\right)\pi^{-1}$$
and the Lemma follows.
\end{proof}

%For later use, we also have the following:

\begin{lemma}\label{gaga}
$$\Gamma_i \, \rho(\gamma)=q^{\frac{1}{2}}x_i \,  \rho(\gamma) \, \Gamma_i $$
\end{lemma}
\begin{proof}
We simply note that 
$$\Gamma_i \left(\sum_{j=1}^{r+1} \frac{{\rm Log}(x_j)^2}{2 {\rm Log}(q)}\right)=
\left(\sum_{j=1}^{r+1} \frac{{\rm Log}(x_j)^2}{2 {\rm Log}(q)}+{\rm Log}(x_i)+\frac{{\rm Log}(q)}{2}\right)\Gamma_i$$
\end{proof}

\begin{thm}\label{conjthm}
For all $n\in \Z$, we have:
$$Y_{i,n}=q^{\frac{n}{2}}\,\gamma^{-n}\,Y_i \, \gamma^n.$$
\end{thm}
\begin{proof}
As $\gamma$ is a symmetric function of the $X_i$'s, it commutes with $s_i$,
and with all the $T_i$ in the functional representation. 
We compute:
\begin{eqnarray*}
\gamma^{-n}\,Y_i \, \gamma^n&=&\gamma^{-n}\,T_i... T_{N-1} \pi^{-1}T_1^{-1}...T_{i-1}^{-1}\, \gamma^n\\
&=&T_i... T_{N-1}\, \gamma^{-n}\,\pi^{-1}\, \gamma^n\, T_1^{-1}...T_{i-1}^{-1}\\
&=& q^{\frac{n}{2}}T_i... T_{N-1}\, X_{N}^n\,\pi^{-1}\,  T_1^{-1}...T_{i-1}^{-1}\\
&=&q^{-\frac{n}{2}}T_i... T_{N-1}\, (X_1...X_{i-1})^{-n} (X_1...X_{i-1})^n\,\pi^{-1}\, X_{1}^n T_1^{-1}...T_{i-1}^{-1}\\
&=&q^{-\frac{n}{2}}(X_1...X_{i-1})^{-n}T_i... T_{N-1}\, \pi^{-1}(X_1...X_{i})^nT_1^{-1}...T_{i-1}^{-1}\\
&=&q^{-\frac{n}{2}}(X_1...X_{i-1})^{-n}Y_i(X_1...X_i)^n=q^{-\frac{n}{2}}\, Y_{i,n}.
\end{eqnarray*}
where we have first used the fact that $\gamma$ is a symmetric function of $X_1,...,X_{N}$ 
and therefore commutes with $T_j$ for all $j$, then we have used Lemma \ref{piga}, and 
finally the commutations between the $X$'s and the $T$'s,
in particular that the symmetric function $X_1...X_i$ of the variables $X_1,...,X_i$ commutes with $T_j$
for $j=1,2,...,i-1$, and also that $X_1...X_{i-1}$ commutes with $T_j$ for $j=i,i+1,...,N-1$.
\end{proof}

\begin{remark}
Theorem \ref{conjthm} above implies immediately the commutation  of the operators $Y_{i,n}$ for any fixed $n$.
However, the element $\gamma$ \eqref{gammadefn} only belongs to a completion of the DAHA, as it involves infinite power series of the generators $X_i$. The direct proof of Lemma \ref{commuYin} bypasses this complication.
\end{remark}

\subsubsection{Comparison with the standard $SL(2,\Z)$ action on DAHA}


Theorem \ref{conjthm} allows to identify the conjugation w.r.t. $\gamma^{-1}$ as the action 
of the generator $\tau_+$ of the standard $SL(2,\Z)$ action on DAHA \cite{Cheredbook}. 
Indeed, using the definition\footnote{This definition is in fact dual to that of \cite{Cheredbook},
and corresponds to the definitions of Chapter 1.}:
\begin{eqnarray*}&&\tau_+(X_i)=X_i, \quad \tau_+(T_i)=T_i, \quad \tau_+(q)=q, \quad \tau_+(t)=t, \\
&& \tau_+(Y_1Y_2\cdots Y_i)=q^{-i/2}\, (Y_1 Y_2 \cdots Y_i)( X_1 X_2 \cdots X_i) 
\end{eqnarray*}
This leads to the expression $Y_{i,n}=q^{n/2}\, \tau_+^n (Y_i)$, which allows to finally identify:

\begin{lemma}\label{taupluslemma}
The generator $\tau_+$ of the standard $SL(2,\Z)$ action on DAHA reads:
$$ \tau_+={\rm ad}_{\gamma^{-1}} $$
namely it acts by conjugation w.r.t. $\gamma^{-1}$ of \eqref{gammadefn}.
\end{lemma}

The second generator $\tau_-$ of the standard $SL(2,\Z)$ action on DAHA is obtained by use of the
anti-involution $\epsilon$ of the DAHA acting on generators and parameters as:
$$\epsilon:\, (X_i,Y_i,T_i,q,t)\, \mapsto \, (Y_i,X_i,T_i^{-1},q^{-1},t^{-1})$$
and such that 
$$\tau_-=\epsilon \, \tau_+\, \epsilon $$
This leads to the following:

\begin{lemma}\label{taumoinslemma}
The generator $\tau_-$ corresponds to the conjugation w.r.t. the element $\eta^{-1}$, where:
\begin{equation}\label{etadefn}
\eta:=\exp\left\{ -\sum_{i=1}^{N} \frac{{\rm Log}(Y_i)^2}{2 {\rm Log}(q)}\right\} 
\end{equation}
namely
$$ \tau_-={\rm ad}_{\eta^{-1}} $$
\end{lemma}
\begin{proof}
Apply the anti-involution $\epsilon$ to $\gamma$, and note that $\epsilon(\gamma)=\eta$.
\end{proof}

\begin{remark}\label{nablarem}
The quantity $\eta^{-1}$ is very similar to the nabla operator $\nabla$ 
of \cite{BG} in a version suitable for the case of $N$ variables. To avoid confusion, we write $\eta^{-1}=\nabla^{(N)}$.
It is known \cite{Cheredbook} that the Macdonald polynomial $P_\lambda(x_1,...,x_N)$ for any 
partition $\lambda=(\lambda_1\geq \lambda_2 \geq \cdots \geq \lambda_N)$ (or equivalently Young diagram
with $\lambda_i$ boxes in row $i$), is an eigenvector
in the functional representation of any symmetric function $f(\{Y_i\}_{i=1}^N)$, with
eigenvalue $f(\{t^{\frac{N+1}{2}-i} q^{\lambda_i}\}_{i=1}^N)$. In particular, this holds for $\nabla^{(N)}$,
with the result:
\begin{eqnarray*}
{\nabla^{(N)}}\, P_\lambda&=& \exp\left\{ \frac{1}{2\,{\rm Log}(q)}\sum_{i=1}^{N} \left( (\frac{N+1}{2}-i)\,{\rm Log}(t)+\lambda_i\, {\rm Log}(q)\right)^2\right\}\, P_\lambda\\
&=&\left(C_N\, \prod_{i=1}^{N} q^{\frac{\lambda_i^2}{2}}\, t^{(\frac{N+1}{2}-i)\lambda_i}\right)\, P_\lambda
=C_N\, u_\lambda\, P_\lambda
\end{eqnarray*}
where ${\rm Log}(C_N)=\frac{N(N^2-1)}{24} \frac{{\rm Log}(t)^2}{{\rm Log}(q)}$ and
$u_\lambda=t^{\frac{N-1}{2}|\lambda|-n(\lambda)}q^{\frac{1}{2}|\lambda|+n(\lambda')} $, where
$n(\lambda)=\sum_i (i-1)\lambda_i$, and $\lambda'$ is the usual reflected diagram, with  $1$ box in the bottommost
$\lambda_1-\lambda_2$ rows, $2$ boxes in the next $\lambda_2-\lambda_3$ rows, etc, such that 
$n(\lambda')=\sum_i \lambda_i(\lambda_i-1)/2$.
In \cite{BG}, the $\nabla$ operator is defined to have eigenvalue $t^{n(\lambda)}q^{n(\lambda')}$ on the {\em modified}
Macdonald polynomials ${\widetilde H}_\lambda$, obtained from $P_\lambda$ by a certain transformation. 
We see that $\nabla^{(N)}$ is an analogue of the operator $\nabla$, acting instead on the $P_\lambda$.
\end{remark}


\subsection{Generalized Macdonald difference operators}

\begin{defn}
We define operators:
\begin{equation}\label{defnewmac} 
{D}_{\al;n}\equiv {D}_{\al;n}^{q,t} := q^{-\al n}\sum_{1\leq i_1<i_2<\cdots <i_\al \leq N}  
{Y}_{i_1,n}{Y}_{i_2,n} \cdots {Y}_{i_\al,n} \qquad (\al=0,1,...,N)
\end{equation}
Equivalently, we have:
$\sum_{\al=0}^{r+1}z^\al q^{n\al}\,{D}_{\al;n}=\prod_{i=1}^{N}(1+z Y_{i,n})$.
\end{defn}

Theorem \ref{conjthm} allows to rewrite immediately:
\begin{lemma}\label{dalga}
We have the following identity in the $A_{N-1}$ DAHA functional representation:
$$ \rho({D}_{\al;n}) =q^{-\frac{\al n}{2}}\rho(\gamma)^{-n} \rho({D}_{\al;0})\rho(\gamma)^{n}\ ,$$
where $ {D}_{\al;0}\equiv  {D}_{\al}$ are given by \eqref{orimac}.
\end{lemma}


\begin{thm}\label{genmacthm}
The operators $\rho({D}_{\al;n})$ leave the space ${\mathcal S}_N$ of symmetric functions of the $x$'s invariant. 
They take the following form:
\begin{equation}\label{genmac}
\rho({D}_{\al;n})\vert_{{\mathcal S}_N}=:{\mathcal D}_{\al;n}=\theta^{-\al(N-\al)}\, {\mathcal M}_{\al;n}
\end{equation}
where ${\mathcal M}_{\al;n}$ are the generalized Macdonald operators \eqref{genmacdop}, and ${ \mathcal D}_{\al;n}$
their slightly renormalized version:
\begin{equation}\label{defcald}
{ \mathcal D}_{\al;n} 
=\sum_{|I|=\al,\, I\subset [1,N]} (x_I)^n \prod_{i\in I \atop j\not \in I} 
\frac{\theta x_i -\theta^{-1} x_j}{x_i-x_j}\, \Gamma_I 
\end{equation}
where $x_I=\prod_{i\in I}x_i$ and $\Gamma_I=\prod_{i\in I}\Gamma_i$, with ${\mathcal D}_{\al;0}=\widetilde{\mathcal D}_\al$.
\end{thm}
\begin{proof}
We use Lemma \ref{gaga} to write for any index subset $I$ of cardinality $\al$:
$$ \Gamma_I\, \rho(\gamma)=(\prod_{i\in I}\Gamma_i) \, \rho(\gamma) =q^{\frac{\al}{2}}x_I \, \rho(\gamma) \, \Gamma_I $$
%with $x_I=\prod_{i\in I}x_i $ and $\Gamma_I=\prod_{i\in I}\Gamma_i$.
Starting from the expression of Lemma \ref{dalga}, we may now conjugate the formula of Theorem \ref{macdop} for 
$\rho(\widetilde{\mathcal D}_\al)=\rho({\mathcal D}_{\al;0})$ 
with $\rho(\gamma)^n$ as follows:
\begin{eqnarray*}\rho({\mathcal D}_{\al;n})=q^{-\frac{\al n}{2}}\rho(\gamma)^{-n}\rho(\widetilde{\mathcal D}_\al)\rho(\gamma)^n
&=&q^{-\frac{\al n}{2}}\sum_{|I|=\al,\, I\subset [1,N]}  \prod_{i\in I \atop j\not \in I} 
\frac{\theta x_i -\theta^{-1} x_j}{x_i-x_j}\, \rho(\gamma)^{-n} \Gamma_I \rho(\gamma)^n\\
&=&\sum_{|I|=\al,\, I\subset [1,N]} (x_I)^n 
\prod_{i\in I \atop j\not \in I} \frac{\theta x_i -\theta^{-1} x_j}{x_i-x_j}\,  \Gamma_I
\end{eqnarray*}
where we have used Lemma \ref{gaga}.
The Theorem follows.
\end{proof}

%Note that ${\mathcal D}_{\al,0}={\mathcal D}_{\al}$ of Theorem \ref{macdopthm}.


