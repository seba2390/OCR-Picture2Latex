%intro
%summary of results and main references
\subsection{Introduction}

The aim in this paper is to make explicit the relation between the algebra satisfied by the generators of the $A_{N-1}$ quantum $Q$-system, 
or more precisely, their $t$-deformation \cite{DFK16}, and the following algebras, whose relation to 
each other is better known: the $A_{N-1}$ spherical Double Affine Hecke Algebra (sDAHA) 
\cite{Cheredbook}, the level-$(0,0)$ and level-$(1,0)$ quantum toroidal 
algebras of $\widehat{\mathfrak gl}_1$ \cite{Miki07,FJMM,AFS}, the shuffle algebra \cite{FO,NegutShuffle} and the 
Elliptic Hall Algebra (EHA) \cite{SHIVAS,BS}. 

Quantum $Q$-systems arise naturally as the quantization \cite{BZ} of the cluster algebras \cite{FZ} associated with 
the classical Q-system \cite{Ke07,DFKnoncom}. The latter is a recursion relation for characters of Kirillov-Reshetikhin modules \cite{KR} of quantum affine algebras, and it is directly connected \cite{HKOTY,DFK08} with fermionic formulas for the characters of the tensor products of KR-modules. The quantum $Q$-system is similarly connected with the graded characters of the Feigin-Loktev fusion product \cite{FL} of the same spaces \cite{qKR}, where the quantum parameter $q$
is associated with the homogeneous grading. 

Graded characters of tensor products of KR-modules 
can be written as iterated, ordered products of difference operators, which satisfy the quantum $Q$-system, acting on the polynomial 1. That is, there is a representation of the algebra satisfied by the generators of the quantum $Q$-system in terms of difference operators acting on the space of symmetric functions \cite{qKR,DFK15}.

In Ref. \cite{DFK16}, we gave the relation between this algebra and a non-standard level-$0$ version of the quantum affine algebra of $\widehat{\mathfrak sl}_2$ in the Drinfeld presentation. For finite $N$, it is a rank-dependent quotient of a quantum affine algebra.
The difference operators introduced in \cite{DFK15} were identified as a generalization of the the dual Whittaker limit $t\to \infty$ of the celebrated Macdonald 
difference operators \cite{macdo}. 

This observation naturally lead  \cite{DFK16} to the introduction of a $t$-deformation of these operators. The present paper is a continuation of this work, and its purpose is to give the algebraic framework of this deformation. To do this, we define a new set of difference operators.
%In this context, we start by considering a natural further generalization of the Macdonald difference operators of \cite{DFK16}.

\subsection{Generalized Macdonald operators}

Let $\C_{q,t}:=\C(q,t)$ be the field of rational functions in the formal variables $q$ and $t$.
We denote by $\mathcal F_N$ the space of rational symmetric functions of $N$ variables over $\C_{q,t}$:
\begin{equation}{\mathcal F}_N:=\C_{q,t}(x_1,x_2,...,x_N)^{S_N}.\end{equation} 
It is a subpace of ${\mathcal S}_N$, symmetric functions of $N$ variables over the same field. 
Here, the symmetric group $S_N$ acts naturally on $\{x_1,...,x_N\}$ by permutations. 

%\begin{defn}
Given any rational function $f(x_1,x_2,...,x_N)$, let ${\rm Sym}(f)$ denote its symmetrization, and
% of the function, defined as
%\begin{equation}{\rm Sym}( f(x_1,x_2,...,x_N))
%=\sum_{\sigma\in S_N} f(x_{\sigma(1)},x_{\sigma(2)},...,x_{\sigma(N)})\ .
%\end{equation}
%\end{defn}
%Furthermore, 
let $\Gamma_i$ be the multiplicative shift operator, acting on functions in $\S_N$ as follows:
\begin{equation}\label{shift}
\Gamma_i\, f(x_1,...,x_N)=f(x_1,...,x_{i-1},q x_i,x_{i+1},...,x_N) \qquad (1\leq i\leq N).
\end{equation}
Alternatively, $\Gamma_i=q^{\delta_i}$ where $\delta_i$ is the additive shift operator, 
$\delta_i: \ x_j\mapsto x_j+\delta_{i,j}$.

Consider
%The main definition of this section is that of a 
the following $q$-difference operator, which is a generalization of the Macdonald operator, acting on 
%on symmetric functions in 
$\F_N$:
%and is a simple generalization of the Macdonald operator acting on the same space.
\begin{defn}\label{gmacdef} 
Let $\al\in [1,N]$ and $P\in \F_\al$. Define the associated generalized 
Macdonald operator to be the difference operator ${\mathcal D}_\al(P)$ acting on $\F_N$ as
\begin{equation}\label{symdiffop}
{\mathcal D}_\al(P):=\frac{1}{\al!\,(N-\al)!}{\rm Sym}\left( P(x_1,x_2,...,x_\al) 
\prod_{1\leq i\leq  \al<j \leq N} \frac{tx_i-x_j }{x_i-x_j}\, 
\Gamma_1\Gamma_2\cdots \Gamma_\al\right).\end{equation}
\end{defn}

The following special examples of the operator in Definition \ref{gmacdef}:
% are of interest to us in this paper:
\begin{enumerate}
\item The original Macdonald difference operators \cite{macdo} correspond to $P=1\in \F_\al$:
\begin{equation}\label{macdop}
\widetilde{\mathcal D}_\al:={\mathcal D}_\al(1)= \sum_{I\subset [1,N]\atop |I|=\al } 
\prod_{i\in I\atop j\not\in I} \frac{t x_i -x_j}{x_i-x_j} \, \prod_{i\in I} \Gamma_i .
\end{equation}

\item The generalized Macdonald operators, introduced in \cite{DFK16}, correspond to the special case of $P=(x_1x_2\cdots x_\al)^n\in \F_\al$, $n\in \Z$:
\begin{equation}\label{genmacdop}
{\mathcal M}_{\al;n}:={\mathcal D}_\al\Big( (x_1x_2\cdots x_\al)^n\Big)=
\sum_{I\subset [1,N]\atop |I|=\al } \prod_{i\in I}(x_i)^n\,\prod_{i\in I\atop j\not\in I} 
\frac{t x_i -x_j}{x_i-x_j} \, \prod_{i\in I} \Gamma_i.
\end{equation}
These operators were inspired by the functional representation of the quantum Q-system. Note that ${\mathcal M}_{\al;0}=\widetilde{\mathcal D}_\al$ are the original Macdonald operators.

\item More generally,
%In this paper, we will focus on a more general case of Definition \ref{gmacdef}, obtained 
%
Let $P$ be
a {\em generalized Schur function} $s_{a_1,...,a_\al}\in \F_\al$ with $a_i\in \Z$. This is the natural 
generalization of Schur polynomials to Schur Laurent polynomials:
\begin{defn}
For any $(a_1,a_2,...,a_\al) \in \Z^\al$, define the generalized Schur function:
\begin{equation}\label{defschur}
s_{a_1,...,a_\al}(x_1,x_2,...,x_\al)
:=\frac{\det_{1\leq i,j\leq \al}\left( x_i^{a_j+\al-j}\right)}{\prod_{1\leq i<j\leq \al} x_i-x_j}={\rm Sym}\left(
\frac{\prod_{i=1}^\al x_i^{a_i+\al-i}}{\prod_{i<j} x_i-x_j} \right) .
\end{equation}
\end{defn}
\noindent By construction, $s_{a_1,...,a_\al}(x_1,x_2,...,x_\al)$ is a Laurent polynomial in $\F_\al$.
We denote the difference operators corresponding to generalized Schur functions by
\begin{eqnarray}
{\mathcal M}_{a_1,a_2,...,a_\al}&:=&{\mathcal D}_\al\Big(s_{a_1,...,a_\al}(x_1,x_2,...,x_\al)\Big) \nonumber \\
&=&\sum_{I\subset [1,N]\atop |I|=\al } s_{a_1,...,a_\al}(\bx_I) \,\prod_{i\in I\atop j\not\in I} \frac{t x_i -x_j}{x_i-x_j} 
\, \prod_{i\in I} \Gamma_i \label{schurmacdo}
\end{eqnarray}
where $\bx_I$ stands for the ordered collection of variables $(x_i)_{i\in I}$.

As a particular case, we recover ${\mathcal M}_{\al;n}={\mathcal M}_{n,n,...,n}$, which when $n>0$
corresponds to the usual Schur function $s_{n,n,...,n}(x_1,...,x_\al)=(x_1x_2\cdots x_\al)^n$, 
with rectangular Young diagram $\al\times n$. 
\end{enumerate}

%The latter is known to be the classical character of the irreducible module of $A_{N-1}$ with rectangular 
%Young diagram $\al\times n$, also identified with the Kirillov-Reshetikhin module $KR_{\al;n}$. As such, it satisfies 
%the classical $Q$-system for $A_{N-1}$. It is interesting to note 
%that ${\mathcal D}_\al$ maps this classical character to the ``quantum character" ${\mathcal M}_{\al;n}$.

\subsection{Main results and outline}
The paper is organized as follows.

Our first task is to show in Section \ref{dahasec} how the operators ${\mathcal M}_{\al;n}$
of \eqref{genmacdop} appear naturally in the 
context of the functional representation of the spherical DAHA. To this end, we introduce 
families of commuting operators $Y_{i,n}= (X_1 X_2 \cdots X_{i-1})^{-n}\, Y_i \, (X_1 X_2 \cdots X_i)^n$ 
in terms of the standard generators $X_i,Y_i$ of the DAHA (see Sect. \ref{defdahasec} for definitions).
Next we show in Theorem \ref{genmacthm} that the operators \eqref{genmacdop} correspond to their elementary
symmetric functions, in the same way as usual Macdonald operators are obtained from elementary symmetric
functions of the $Y_i$-generators of the DAHA. We find that $Y_{i,n}$ is proportional to the $n$-th iterate of the action
on $Y_i$ of one particular generator of the $SL_2(\Z)$ action on DAHA. This clarifies the algebraic origin of the
operators ${\mathcal M}_{\al;n}$.

%In the rest of the paper, the main tool used is that of generating functions.

In Section \ref{qtorsec}, we show (Theorem \ref{gentoro}) that the generating functions for ${\mathcal M}_n={\mathcal M}_{1;n}$, 
for all operators with $\al=1$ is an element of the functional (level-$(0,0)$) representation of the quantum 
toroidal algebra of $\widehat{\mathfrak gl}_1$ \cite{Miki99,FJMM}, or more precisely a particular quotient 
thereof that imposes the finite number $N$ of variables (see \eqref{elipquo} and \eqref{defpsipm}). We call these the ``fundamental currents" 
for reasons which will become clear in Section 4.
This section explains a posteriori one of the findings of Ref. \cite{DFK16}, which corresponds to
the limit $t\to\infty$, where these generating functions are part of a non-standard representation 
of the quantum enveloping algebra of the affine algebra $\widehat{\mathfrak sl}_2$.

In Section \ref{bososec}, we give the plethystic formulation of the above currents, equivalent to their 
so-called bosonization in the limit of an infinite number of variables $N\to\infty$ (Theorem \ref{bobothm}). 
It describes their action on formal power series of the power sum functions 
$p_k=\sum_i x_i^k$ for $k \in \Z^*$. 
We show how this is related to a construction of Bergeron et al. \cite{BGLX}. One feature of this $N\to\infty$ limit
is the fact that the limiting currents pertain to a non-trivial representation of the quantum toroidal algebra,
with levels $(1,0)$ (as opposed to level $(0,0)$ for finite $N$).

In Section \ref{shufflesec}, we give an alternative definition ${\mathcal M}_\al(P)$ (see Definition \ref{malphadef})
for the generalized Macdonald operators  ${\mathcal D}_\al(P)$ of  Definition \ref{gmacdef}, which expresses
them as  a multiple constant term involving products of the above fundamental currents. The coincidence of 
these two definitions is the subject of Theorem \ref{mainthm}. The latter allows to write the multiple generating function
for the operators ${\mathcal M}_{a_1,...,a_\al}$ of \eqref{schurmacdo} in terms solely of those of the ${\mathcal M}_{n}$'s
(Theorem \ref{Mofm}). We conjecture (Conjecture \ref{polyconj}) that these may be reduced to polynomial expressions 
modulo the relations of the quantum toroidal algebra, and give the proof in the case $\al=2$ (Theorem \ref{polynomialitythm}). Such polynomials play the role of $(q,t)-$determinants (see the expression \eqref{mtwopol} for $\al=2$). 
In Section \ref{shuprosec}, we show that
the definition of ${\mathcal M}_\al(P)$ is naturally compatible with a suitably defined non-commutative product 
$*: \F_\al\times \F_\beta\to \F_{\al+\beta}$, 
$(P,P')\mapsto P*P'$ (the shuffle product \cite{FO,NegutShuffle}), 
which satisfies the morphism  property: ${\mathcal M}_{\al+\beta}(P*P')={\mathcal M}_\al(P){\mathcal M}_\beta(P')$
(see Thorem \ref{shufmac}).
We may therefore translate relations between the generalized Macdonald operators into shuffle product identities, 
which sometimes are easier to prove (see examples in Sections \ref{appsecone}-\ref{appsecthree}).

Section \ref{EHAsec} presents a functional representation of the EHA in terms of our generalized 
Macdonald operators. The established connection between spherical DAHA and EHA in the case of $A_\infty$ 
(infinite number of variables) \cite{SHIVAS} extends to the quotient corresponding to $A_{N-1}$ (finite
number $N$ of variables). This connection allows to derive new formulas for the operators 
${\mathcal M}_{\al;n}$ of \eqref{genmacdop} as {\it polynomials} of the fundamental operators with $\al=1$, thus proving 
Conjecture \ref{polyconj} for $a_1=a_2=\cdots =a_\al=n$ (Theorem \ref{polpol}).

In Section \ref{whitaklimsec}, we explore the dual Whittaker limit $t\to\infty$
of the constructions of this paper. In particular, we relate the finite $t$ Macdonald operators 
${\mathcal M}_{n}$ to their $t\to\infty$ limit $M_n$. We also find an explicit formula for the
$t\to\infty$ limit  of the operators ${\mathcal M}_{a_1,a_2,...,a_\al}$ as a quantum determinant, which involves
a summation over $\al\times \al$ Alternating Sign Matrices (Theorem \ref{qdethm}).
%More generally, the operators
%${\mathcal M}_{\al;n}$ are variables
%in a $t$-deformation of the quantum cluster algebra associated to the $Q$-system for $A_{N-1}$. 
By considering the $t\to\infty$ limit of the 
shuffle product, we find an alternative shuffle expression for the quantum cluster algebra relations (Theorem \ref{msyshuf}).

Section \ref{concsec}
gathers a few concluding remarks on the $(q,t)$-determinant that expresses the operator ${\mathcal M}_{a_1,...,a_\al}$
as a polynomial of the ${\mathcal M}_n$'s, and suggest that the $A_{N-1}$ 
EHA quotient corresponding to a finite number $N$ of variables is the natural $t$-deformation of the quantum
$Q$-system algebra.
%\color{red}
%Deformed commutation relations. $q,t$- Laurent phenomenon. Quantum determinant. Flat connection. Yang-Baxter.
%Compare to \cite{FHHSY}.
%\color{black}

\vskip.2in

\noindent{\bf Acknowledgments.}  We thank O. Babelon, J.-E. Bourgine, I. Cherednik, M. Jimbo, Y. Matsuo,
A. Negut, V. Pasquier, O. Schiffmann for
valuable discussions and especially F. Bergeron for bringing Ref. \cite{BGLX} to our attention and providing us 
with detailed explanations. R.K.�s research is supported by NSF grant DMS-1404988 and the conference travel grant NSF DMS-1643027.
P.D.F. acknowldeges support from the NSF grant DMS-1301636 and the Morris and 
Gertrude Fine endowment. We thank the Institut de Physique Th\'eorique (IPhT) of Saclay, France, 
and the Institut Henri Poincar\'e, Paris program on ``Combinatorics and Interactions", France for hospitality during various stages of this work.



