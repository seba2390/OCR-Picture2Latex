%toroidal
%quantum toroidal, t->infty limit, connection to elliptic Hall, Schiffman Vasserot

\subsection{Definitions and results}

In this section, we show that the generalized Macdonald operators $\{{D}_{1;n}: n\in \Z\}$ satisfy the relations of the quantum toroidal algebra \cite{Miki07,FJMM} at level $(0,0)$. We call the generating functions of these generators the fundamental currents, and in Section \ref{EHAsec} (Theorem \ref{polpol}) we show, by use of the elliptic Hall algebra, that the generators $D_{\al;n}$ are polynomials in the fundamental generators $\{D_{1;m}\}$.

\subsubsection{Quantum toroidal algebra $\widehat{\gl}_1$}

For generic parameters $q,t \in \C^\star$ let us introduce the functions:
\begin{equation}\label{defofg}
g(z,w):= (z-q w)(z-t^{-1}w)(z-q^{-1}t w),\quad 
G(x):=-\frac{g(1,x)}{g(x,1)}=\frac{(1-q x)(1-t^{-1}x)(1-q^{-1}t x)}{(1-q^{-1} x)(1-t x)(1-qt^{-1} x)}
\end{equation}
We will use the formal delta function
\begin{equation}\label{formaldet} \delta(u)=\sum_{n\in \Z} u^n \end{equation}
with the property that $\delta(u) f(u)=\delta(u) f(1)$ for any function $f$ which is non-singular at $u=1$.
The following definitions are borrowed from Ref.\cite{AFS}.

\begin{defn}\label{qtorgendef}
The quantum toroidal algebra of $\widehat{\gl}_1$ is defined by generators and relations. The generators are
the modes of the currents $x^\pm(z)=\sum_{n\in \Z} x_n^{\pm} z^{-n}$ and the series
$\varphi^\pm(z)=\sum_{n\geq 0} \varphi_n^{\pm} z^{\mp n}\in \C[[z^{\mp 1}]]$, together with two central elements
${\hat \gamma},\ {\hat \delta}$, and the relations read:
\begin{eqnarray*}
[\varphi^\pm(z),\psi^\pm(w)]&=&0,\qquad \varphi^+(z)\,\psi^-(w)=\frac{G({\hat \gamma}w/z)}{G({\hat \gamma}^{-1}w/z)}\,
\varphi^-(w)\,\varphi^+(z)\\
\varphi^+(z)\, x^\pm(w)&=&G({\hat \gamma}^{\mp 1}w/z)^{\mp 1}\, x^\pm(w)\,\varphi^+(z),\quad 
\varphi^-(z)\, x^\pm(w)=G({\hat \gamma}^{\mp 1}w/z)^{\pm 1}\, x^\pm(w)\,\varphi^-(z)\\
x^\pm(z)\, x^\pm(w)&=&G(z/w)^{\pm 1} \, x^\pm(w)\, x^\pm(z), \qquad \psi_0^\pm={\hat \delta}^{\mp 1},\\
{[} x^+(z), x^-(w) {] }&=& \frac{(1-q)(1-t^{-1})}{(1-q t^{-1})} \left\{ \delta({\hat \gamma}^{-1}z/w)\varphi^+({\hat \gamma}^{-1/2}z)
-\delta({\hat \gamma} z/w)\varphi^-({\hat \gamma}^{1/2}z)\right\}  \\
0&=&{\rm Sym}_{z_1,z_2,z_3}\left( \frac{z_2}{z_3} {\Big[}{ x^\pm}(z_1),{[}{x^\pm}(z_2),{ x^\pm}(z_3){]}{\Big]}\right)
\end{eqnarray*}
\end{defn}

A particular class of representations \cite{FHHSY} indexed by integers $(\ell_1,\ell_2)\in \Z_+^2$ (referred to as levels)
corresponds to diagonal actions of the central elements ${\hat \gamma},\ {\hat \delta}$
with respective eigenvalues $\gamma^{\ell_1},\gamma^{\ell_2}$, where $\gamma=(t q^{-1})^{1/2}$.

In the following, we adopt sligthtly different conventions for the naming of the generators (which, up to a change of variables
amounts to applying the anti-automorphism $\omega$ of the algebra, that sends $(x^\pm,\varphi^\pm,{\hat \gamma},{\hat \delta})$
to $(x^\mp,\varphi^\mp,{\hat \gamma}^{-1},{\hat \delta}^{-1})$, namely we set:
\begin{eqnarray}
&&x^+(z)=\frac{(1-q)(1-t^{-1})}{q^{1/2}}{\mathfrak e}(q^{-1/2}z),\quad x^-(z)
=\frac{(1-q^{-1})(1-t)}{q^{-1/2}}{\mathfrak f}(q^{-1/2} {\hat \gamma}^{-1}z),\nonumber \\ 
&&\varphi^+(z)=\psi^-(q^{-1/2}{\hat \gamma}^{1/2}z),\qquad \varphi^-(z)=\psi^+(q^{-1/2}{\hat \gamma}^{-1/2}z)
\label{dictio}
\end{eqnarray}



\subsubsection{Level $(0,0)$ quantum toroidal $\widehat{\gl}_1$}

Unless otherwise mentioned, we will mainly concentrate on the level-$(0,0)$ representations corresponding to 
$\hat \gamma=\hat \delta=1$. In our set of generators with currents
${\mathfrak e}(z)=\sum_{n\in \Z} z^n\, e_n$, ${\mathfrak f}(z)=\sum_{n\in \Z} z^n\, f_n$,
$\psi^\pm(z)=\sum_{n\geq 0} z^{\pm n}\, \psi^{\pm}_n\in \C[[z^{\pm 1}]]$, the corresponding relations read:
%The quantum toroidal algebra of $\widehat{\gl}_1$ is defined by generators and relations. The generators are $\{ e_n,f_n, \psi^{\pm}_m: n\in \Z, m\in \Z_+\}$, and the relations
%are most easily expressed in terms of generating functions:
%\begin{equation*}{\mathfrak e}(z)=\sum_{n\in \Z} z^n\, e_n\ , \quad {\mathfrak f}(z)=\sum_{n\in \Z} z^n\, f_n\ , \quad
%\psi^{\pm}(z) = \sum_{n\geq 0} z^{\pm n}\, \psi^{\pm}_n ,
%\end{equation*}
%where $z$ is a formal variable.
%In general, the algebra has two central charges $c_1,c_2$ explicitly appearing in the defining relations. In the functional representation which appears in our context, we have $c_1=c_2=0$, also referred to as level 0. 

\begin{defn}\label{qtordef}
The level $(0,0)$ quantum toroidal $\widehat{\gl}_1$ is the algebra generated by $\{ e_n,f_n, \psi^{\pm}_m: \ n\in \Z, \ m\in \Z_+\}$ with relations:
\begin{eqnarray*}
&&g(z,w){\mathfrak e}(z){\mathfrak e}(w)+g(w,z){\mathfrak e}(w){\mathfrak e}(z)=0, \ \ g(w,z){\mathfrak f}(z){\mathfrak f}(w)+g(z,w){\mathfrak f}(w){\mathfrak f}(z)=0, \\
&&g(z,w)\psi^\pm(z)\,{\mathfrak e}(w)+g(w,z){\mathfrak e}(w)\,\psi^\pm(z)=0, \ \ g(w,z)\psi^\pm(z)\,{\mathfrak f}(w)+g(z,w){\mathfrak f}(w)\,\psi^\pm(z)=0, \\
&&{[} {\mathfrak e}(z),{\mathfrak f}(w) {]}=\frac{\delta(z/w)}{g(1,1)}\, (\psi^+(z)-\psi^-(z)),\\
&&{\rm Sym}_{z_1,z_2,z_3}\left( \frac{z_2}{z_3} {\Big[}{\mathfrak e}(z_1),{[}{\mathfrak e}(z_2),{\mathfrak e}(z_3){]}{\Big]}\right)
=0 ,\ \ 
{\rm Sym}_{z_1,z_2,z_3}\left( \frac{z_2}{z_3} {\Big[}{\mathfrak f}(z_1),{[}{\mathfrak f}(z_2),{\mathfrak f}(z_3){]}{\Big]}\right)
=0 ,
\end{eqnarray*}
with $\psi_0^{\pm}=1$ and $\psi_{n}^{\pm}$ mutually commuting for $n\in \Z_+$.
\end{defn}


Note that when $t\to \infty$, we may set $g_0(z,w)=\lim_{t\to\infty} t^{-1}g(z,w)=z-qw$, and the first relations become relations in the quantum affine algebra of $\sl_2$ in the Drinfeld presentation (with a non-standard deformation parameter $\sqrt{q}$, as in the Hall algebra of \cite{spherical_hall}).
The last two identities are Serre-type relations and distinguish the quantum toroidal algebra from the original Ding-Iohara algebra \cite{DI}.

\subsubsection{Macdonald currents}
We claim that the generalized Macdonald operators introduced in Equation \eqref{defcald} are elements of a functional representation of a quotient of the quantum toroidal algebra. In order to make the comparison explicit, we first define generating currents for the generalized Macdonald operators \eqref{defcald}
for $\al=1,2,...,N$:
\begin{equation}\label{highercur}
{\mathfrak e}_\al(z):=\frac{q^\frac{\al}{2}}{(1-q)^\al}\sum_{n\in \Z} q^{n\al/2}\, z^n\, {\mathcal D}_{\al;n}^{q,t} , \qquad 
{\mathfrak f}_\al(z):=\frac{q^{-\frac{\al}{2}}}{(1-q^{-1})^\al}\sum_{n\in \Z} q^{-n\al/2}\, z^n\, {\mathcal D}_{\al;n}^{q^{-1},t^{-1}} .
\end{equation}
where we have indicated the $q,t$ dependence as superscripts, so that:
\begin{equation}\label{defdtilde}
{ \mathcal D}_{\al;n}^{q^{-1},t^{-1}} 
=\sum_{|I|=\al,\, I\subset [1,N]} (x_I)^n \prod_{i\in I \atop j\not \in I} 
\frac{\theta^{-1} x_i -\theta x_j}{x_i-x_j}\, \Gamma_I^{-1}
\end{equation}
and $\Gamma_i^{-1}$ acts on functions of $x_1,...,x_{N}$ by multiplying the $i$-th variable $x_i$  by $q^{-1}$.

\begin{remark}\label{remef}
Note also that if $S$ denotes the involution acting on functions of $(x_1,...,x_{N})$ by sending $x_i\mapsto x_i^{-1}$ 
for all $i$, then we have $S\Gamma_IS=\Gamma_I^{-1}$ and 
${ \mathcal D}_{\al;-n}^{q^{-1},t^{-1}} =S{ \mathcal D}_{\al;n}^{q,t}S$, so that 
$$S\, {\mathfrak e}_\al(z)\, S=(-1)^\al \, {\mathfrak f}_\al(z^{-1})\ .$$
\end{remark}

The currents \eqref{highercur} can be explicitly expressed as
\begin{eqnarray*}
{\mathfrak e}_\al(z)&=& \frac{q^\frac{\al}{2}}{(1-q)^\al}\sum_{|I|=\al,\, I\subset [1,N]} \delta(q^{\al/2}z x_I) 
\prod_{i\in I \atop j\not \in I} \frac{\theta x_i -\theta^{-1} x_j}{x_i-x_j}\, \Gamma_I, \\
{\mathfrak f}_\al(z)&=&\frac{q^{-\frac{\al}{2}}}{(1-q^{-1})^\al} \sum_{|I|=\al,\, I\subset [1,N]} \delta(q^{-\al/2}z x_I) 
\prod_{i\in I \atop j\not \in I} \frac{\theta^{-1} x_i -\theta x_j}{x_i-x_j}\, \Gamma_I^{-1},
\end{eqnarray*}
by use of the formal $\delta$ function \eqref{formaldet}.

Note that the finite number $N$ of variables implies the vanishing relations:
\begin{equation}\label{elipquo} {\mathfrak e}_{N+1}(z)=0 \ \ {\rm and} \ \ {\mathfrak f}_{N+1}(z)=0 .
\end{equation}


\subsubsection{Main result}
We call ${\mathfrak e}(z):={\mathfrak e}_1(z)$ and ${\mathfrak f}(z):={\mathfrak f}_1(z)$ the fundamental currents. Our main result is that these satisfy relations in the quantum toroidal algebra, as suggested by the notation. We will see later that the vanishing condition \eqref{elipquo} implies a certain quotient of the algebra by polynomials of degree $N+1$ in the fundamental generators.

\begin{thm}\label{gentoro}
The Macdonald currents ${\mathfrak e}(z):={\mathfrak e}_1(z)$ and ${\mathfrak f}(z):={\mathfrak f}_1(z)$ \eqref{highercur},
together with the series:
\begin{equation}\label{defpsipm}
\psi^{\pm}(z):=\prod_{i=1}^{N} 
\frac{(1-q^{-\frac{1}{2}}t (z x_i)^{\pm 1})(1-q^{\frac{1}{2}}t^{-1} (z x_i)^{\pm 1})}{(1-q^{-\frac{1}{2}} (z x_i)^{\pm 1})(1-q^{\frac{1}{2}} (z x_i)^{\pm 1})} \in \C[[z^{\pm 1}]]
\end{equation}
satisfy the level $(0,0)$ quantum toroidal $\widehat{\gl}_1$ algebra relations of Def.~\ref{qtordef}.
\end{thm}

We will prove this theorem in several steps in the following sections.

Note that $\psi^\pm_n$ are explicit symmetric polynomials of $x_1,...,x_N$, with $\psi_0^+=\psi_0^-=1$, and in particular 
\begin{eqnarray}
\psi_1^+&=&(t^{\frac{1}{2}}-t^{-\frac{1}{2}})(q^{\frac{1}{2}}t^{-\frac{1}{2}}-q^{-\frac{1}{2}}t^{\frac{1}{2}})
(x_1+x_2+\cdots+x_{N})\nonumber \\
\psi_1^-&=&(t^{\frac{1}{2}}-t^{-\frac{1}{2}})(q^{\frac{1}{2}}t^{-\frac{1}{2}}-q^{-\frac{1}{2}}t^{\frac{1}{2}})(x_1^{-1}+x_2^{-1}+\cdots+x_{N}^{-1})\label{psiones}
\end{eqnarray}

The finite number $N$ of variables clearly imposes algebraic relations between the coefficients of the series $\psi^\pm$.
The Macdonald currents generate a quotient of the quantum toroidal algebra by the relations \eqref{elipquo}
and those relations.


\subsubsection{The dual Whittaker limit $t\to\infty$}
Before proceeding to the proof of Theorem \ref{gentoro}, we make the connection with our previous work in \cite{DFK16} regarding the quantum Q-system generators.
In that paper, we showed that the $A_{N-1}$ quantum $Q$-system algebra can be realized as a
quotient, depending on $N$, of the algebra of nilpotent currents in the quantum enveloping algebra of $\widehat{\mathfrak sl}_2$, 
$U_{\sqrt{q}}({\mathfrak n}[u,u^{-1}])$. Furthermore, we were able to define currents ${\mathfrak e}$, ${\mathfrak f}$
in terms of fundamental quantum $Q$-system solutions which satisfied non-standard $U_{\sqrt{q}}(\widehat{\mathfrak sl}_2)$
with truncated Cartan currents. The functional representation of these was constructed
in terms of the degenerate generalized Macdonald operators, corresponding to the so-called dual Whittaker
($t\to\infty$) limit of the $t$-deformation considered in this paper.

These observations become more transparent when we take the $t\to \infty$ limit of Theorem \ref{gentoro}.
Note that it is crucial that $N$ be finite in order for this limit to make sense. 
Defining the limits of the renormalized currents:
\begin{equation}
\label{limefpsi}
{\mathfrak e}^{(\infty)}(u):= \lim_{t\to\infty} t^{-\frac{N-1}{2}} \, {\mathfrak e}(u),\quad {\mathfrak f}^{(\infty)}(u):= \lim_{t\to\infty} t^{-\frac{N-1}{2}} \, {\mathfrak f}(u), \quad \psi^{\pm(\infty)}(z):= \lim_{t\to\infty} t^{-N} \psi^{\pm}(z),
\end{equation}
and $g^{(\infty)}(z,w):=\lim_{t\to\infty} \frac{-qt^{-1}}{z w}\, g(z,w)= z-q w$, we find that the limiting Cartan currents become:
\begin{equation}\label{psilim}
\psi^{\pm(\infty)}(z)=(-q^{-\frac{1}{2}}z^{\pm 1})^N\, \prod_{i=1}^{N} 
\frac{(x_i)^{\pm 1}}{(1-q^{-\frac{1}{2}} (z x_i)^{\pm 1})(1-q^{\frac{1}{2}} (z x_i)^{\pm 1})}  ,
\end{equation}
whereas the other relations reduce to those of $U_{\sqrt{q}}(\widehat{\mathfrak sl}_2)$ (the Serre relation is 
automatically satisfied). Note that the Cartan currents have a non-standard valuation of $z^{\pm N}$ in this limit, and vanish when $N\to\infty$, whereas in the Drinfeld presentation of the quantum affine algebra, the zero modes of the Cartan currents are invertible elements.



\subsection{Proof of Theorem \ref{gentoro}}

For readability, we decompose the proof of the Theorem into several steps, Theorems \ref{efrela}, \ref{commuef}, 
\ref{commuefpsi} and \ref{thserrre} below.

\begin{thm}\label{efrela}
We have the following relations:
\begin{eqnarray*}
g(z,w){\mathfrak e}(z){\mathfrak e}(w)+g(w,z){\mathfrak e}(w){\mathfrak e}(z)&=&0\\
g(w,z){\mathfrak f}(z){\mathfrak f}(w)+g(z,w){\mathfrak f}(w){\mathfrak f}(z)&=&0
\end{eqnarray*}
\end{thm}
\begin{proof}
We start with the computation of
\begin{eqnarray*}
\frac{(1-q)^2}{q}\, {\mathfrak e}(z){\mathfrak e}(w)&=&\sum_{i=1}^{r+1} \delta(q^{\frac{1}{2}}z x_i)\prod_{k\neq i}\frac{\theta x_i-\theta^{-1}x_k}{x_i-x_k}\Gamma_i \sum_{j=1}^{r+1} \delta(q^{\frac{1}{2}}w x_j)\prod_{k\neq j}\frac{\theta x_j-\theta^{-1}x_k}{x_j-x_k}\Gamma_j\\
&=&\sum_{1\leq i\neq j\leq r+1}\delta(q^{\frac{1}{2}}z x_i)\delta(q^{\frac{1}{2}}w x_j)\frac{\theta x_i-\theta^{-1}x_j}{x_i-x_j}
\frac{\theta x_j-q\theta^{-1}x_i}{x_j-qx_i}\\
&&\qquad\qquad \times
\prod_{k\neq i,j} \frac{\theta x_i-\theta^{-1}x_k}{x_i-x_k}\frac{\theta x_j-\theta^{-1}x_k}{x_j-x_k}\Gamma_i\Gamma_j\\
&&\quad +\sum_{i=1}^{r+1}\delta(q^{\frac{1}{2}}z x_i)\delta(q^{\frac{3}{2}} w x_i)\prod_{k\neq i} \frac{\theta x_i-\theta^{-1}x_k}{x_i-x_k}\frac{q\theta x_i-\theta^{-1}x_k}{q x_i-x_k}\Gamma_i^2.\\
\end{eqnarray*}
The second term is proportional to $\delta(z/(q w))$, and since $(z-q w)\delta(z/(q w))=0$, we have:
\begin{eqnarray}
&&\frac{(1-q)^2}{q}\,(z-q w){\mathfrak e}(z){\mathfrak e}(w)=\sum_{1\leq i\neq j\leq r+1} \delta(q^{\frac{1}{2}}z x_i)\delta(q^{\frac{1}{2}}w x_j)\times \nonumber \\
&& \qquad \times \frac{(z-qw)(\theta x_i-\theta^{-1}x_j)(\theta x_j-q\theta^{-1}x_i)}{(x_i-x_j)(x_j-qx_i)}
\prod_{k\neq i,j} \frac{(\theta x_i-\theta^{-1}x_k)(\theta x_j-\theta^{-1}x_k)}{(x_i-x_k)(x_j-x_k)}\Gamma_i\Gamma_j
\nonumber \\
&&\qquad \qquad = \frac{(z-tw)(z-qt^{-1}w)}{z-w}\sum_{1\leq i<j\leq r+1} (\delta(q^{\frac{1}{2}}z x_i)\delta(q^{\frac{1}{2}}w x_j)+\delta(q^{\frac{1}{2}}z x_j)\delta(q^{\frac{1}{2}}w x_i))\times \nonumber \\
&&\qquad \qquad \times
\prod_{k\neq i,j}\frac{(\theta x_i-\theta^{-1}x_k)(\theta x_j-\theta^{-1}x_k)}{(x_i-x_k)(x_j-x_k)}\Gamma_i\Gamma_j
\label{pree}\end{eqnarray}
This makes the quantity $(z-q w)(z-t^{-1}w)(z-t q^{-1}w){\mathfrak e}(z){\mathfrak e}(w)$ 
manifestly skew-symmetric under the interchange $z\leftrightarrow w$.
The relation for ${\mathfrak f}$ follows by taking $(q,t)\to (q^{-1},t^{-1})$, under which $g(z,w)\to g(w,z)$.
\end{proof}
This shows that the fundamental generalized Macdonald currents satisfy the first two relations of the quantum toroidal algebra. We now consider the third relation.

\begin{thm}\label{commuef}
We have the commutation relation between Macdonald currents:
$$[{\mathfrak e}(z),{\mathfrak f}(w)]=\frac{\delta(z/w)}{g(1,1)}\, (\psi^+(z)-\psi^-(z))$$
where $\psi^\pm(z)$ are the power series \eqref{defpsipm} of $z^{\pm 1}$.
\end{thm}
\begin{proof}
Let us first compute
\begin{eqnarray*}
(1-q)(1-q^{-1}){\mathfrak e}(z){\mathfrak f}(w)&=& 
\sum_{i=1}^{r+1} \delta(q^{\frac{1}{2}}z x_i)\prod_{k\neq i}\frac{\theta x_i-\theta^{-1}x_k}{x_i-x_k}\Gamma_i \,
\sum_{j=1}^{r+1} \delta(q^{-\frac{1}{2}}w x_j)\prod_{k\neq j}\frac{\theta^{-1} x_j-\theta x_k}{x_j-x_k}\Gamma_j^{-1}\\
&=&\sum_{1\leq i\neq j\leq r+1} \delta(q^{\frac{1}{2}}z x_i)\delta(q^{-\frac{1}{2}}w x_j)\frac{\theta x_i-\theta^{-1}x_j}{x_i-x_j}
\frac{\theta^{-1} x_j-q\theta x_i}{x_j-qx_i}\\
&&\qquad\qquad\qquad \times \prod_{k\neq i,j} \frac{(\theta x_i-\theta^{-1}x_k)(\theta^{-1} x_j-\theta x_k)}{(x_i-x_k)(x_j-x_k)}\Gamma_i\Gamma_j^{-1}\\
&&\quad +\sum_{i=1}^{r+1} \delta(q^{\frac{1}{2}}z x_i)\delta(q^{\frac{1}{2}} w x_i)\prod_{k\neq i}\frac{\theta x_i-\theta^{-1}x_k}{x_i-x_k}\frac{\theta^{-1} q x_i-\theta x_k}{q x_i-x_k},
\end{eqnarray*}
and
\begin{eqnarray*}
(1-q)(1-q^{-1}){\mathfrak f}(w){\mathfrak e}(z)&=& \sum_{j=1}^{r+1} \delta(q^{-\frac{1}{2}}w x_j)\prod_{k\neq j}\frac{\theta^{-1} x_j-\theta x_k}{x_j-x_k}\Gamma_j^{-1}\sum_{i=1}^{r+1} \delta(q^{\frac{1}{2}}z x_i)\prod_{k\neq i}\frac{\theta x_i-\theta^{-1}x_k}{x_i-x_k}\Gamma_i\\
&=&\sum_{1\leq i\neq j\leq r+1} \delta(q^{\frac{1}{2}}z x_i)\delta(q^{-\frac{1}{2}}w x_j)\frac{\theta^{-1} x_j-\theta x_i}{x_j-x_i}\frac{\theta x_i-q^{-1}\theta^{-1}x_j}{x_i-q^{-1}x_j}\\
&&\qquad\qquad\qquad \times 
\prod_{k\neq i,j} \frac{(\theta x_i-\theta^{-1}x_k)(\theta^{-1} x_j-\theta x_k)}{(x_i-x_k)(x_j-x_k)}\Gamma_i\Gamma_j^{-1}\\
&&\quad +\sum_{i=1}^{r+1} \delta(q^{-\frac{1}{2}}z x_i)\delta(q^{-\frac{1}{2}}w x_i)\prod_{k\neq i}\frac{q^{-1}\theta x_i-\theta^{-1}x_k}{q^{-1}x_i-x_k}\frac{\theta^{-1} x_i-\theta x_k}{x_i-x_k}.
\end{eqnarray*}
In the commutator $[{\mathfrak e}(z),{\mathfrak f}(w)]$, the terms corresponding to $i\neq j$ cancel out, 
while the remaining terms are proportional to $\delta(z/w)$, so that we are left
with:
\begin{eqnarray*}
[{\mathfrak e}(z),{\mathfrak f}(w)]&=&\frac{\delta(z/w)}{(1-q)(1-q^{-1})}\sum_{i=1}^{r+1}\left\{
 \delta(q^{\frac{1}{2}}z x_i)\prod_{k\neq i}\frac{\theta x_i-\theta^{-1}x_k}{x_i-x_k}\frac{\theta^{-1} q x_i-\theta x_k}{q x_i-x_k}\right.\\
&&\qquad\qquad  - \left. \delta(q^{-\frac{1}{2}}z x_i)\prod_{k\neq i}\frac{q^{-1}\theta x_i-\theta^{-1}x_k}{q^{-1}x_i-x_k}\frac{\theta^{-1} x_i-\theta x_k}{x_i-x_k}\right\}.
\end{eqnarray*}
Recalling that $\delta(z)=\sum_{n\in\Z} z^n$, this is in agreement with the partial fraction decomposition of \eqref{defpsipm},
which reads:
\begin{eqnarray}
&&\qquad \quad \psi^+(z)=1+\frac{g(1,1)}{(1-q)(1-q^{-1})}\sum_{i=1}^{r+1}\left\{\frac{1}{1-q^{\frac{1}{2}}z x_i}\prod_{k\neq i}\frac{\theta x_i-\theta^{-1}x_k}{x_i-x_k}\frac{\theta^{-1} q x_i-\theta x_k}{q x_i-x_k}\right. \label{psiplusX}\\
&&\qquad\qquad\qquad\qquad\qquad \left. - \frac{1}{1-q^{-\frac{1}{2}}z x_i}\prod_{k\neq i}\frac{q^{-1}\theta x_i-\theta^{-1}x_k}{q^{-1}x_i-x_k}\frac{\theta^{-1} x_i-\theta x_k}{x_i-x_k}\right\}\nonumber  ,\\
&&\qquad \quad  \psi^-(z)=1-\frac{g(1,1)}{(1-q)(1-q^{-1})}\sum_{i=1}^{r+1}\left\{\frac{1}{1-q^{-\frac{1}{2}}z^{-1} x_i^{-1}}\prod_{k\neq i}\frac{\theta x_i-\theta^{-1}x_k}{x_i-x_k}\frac{\theta^{-1} q x_i-\theta x_k}{q x_i-x_k}\right.\label{psiminusX}\\
&&\qquad\qquad\qquad\qquad\qquad \left. - \frac{1}{1-q^{\frac{1}{2}}z^{-1} x_i^{-1}}\prod_{k\neq i}\frac{q^{-1}\theta x_i-\theta^{-1}x_k}{q^{-1}x_i-x_k}\frac{\theta^{-1} x_i-\theta x_k}{x_i-x_k}\right\}\nonumber ,
\end{eqnarray}
as power series of $z$ and $z^{-1}$, respectively. Here, we have identified the constant
$g(1,1)/((1-q)(1-q^{-1}))=(1-t^{-1})(1-t q^{-1})/(1-q^{-1})$.
%To show this, we note that e.g. $\psi^+(z)$
%is a rational fraction of $z$ with single poles at $z=q^{\pm \frac{1}{2}}x_i^{-1}$, $i=1,2,...,r$, and is bounded at $z\to\infty$,
%so it has the form: $\psi^+(z)=P(z)/Q(z)$, with $P$ a polynomial of degree at most $2r+2$,
%and $Q(z)=\prod_{i=1}^{r+1}(1-q^{-\frac{1}{2}}z x_i)(1-q^{\frac{1}{2}}z x_i)$. We then check that $\psi^+(z)$ vanishes
%at $z=(q\theta^{-2})^{\pm \frac{1}{2}}x_i^{-1}$, $i=1,2,...,r+1$. Finally, $P$ is entirely fixed by its zeros and the value
%$P(0)=\psi^+_0$.  
%The values \eqref{psiones}
%are easily obtained by expanding (\ref{psiplus}-\ref{psiminus}) at first order in $z$, $z^{-1}$
%respectively.
\end{proof}


We proceed with the next two relations of the level $(0,0)$ quantum toroidal algebra.

\begin{thm}\label{commuefpsi}
We have the following relations between the Macdonald currents and the series \eqref{defpsipm}:
\begin{eqnarray*}
g(z,w)\psi^\pm(z)\,{\mathfrak e}(w)+g(w,z){\mathfrak e}(w)\,\psi^\pm(z)&=&0\\
g(w,z)\psi^\pm(z)\,{\mathfrak f}(w)+g(z,w){\mathfrak f}(w)\,\psi^\pm(z)&=&0
\end{eqnarray*}
\end{thm}
\begin{proof} By explicit calculation,
\begin{eqnarray*}
\frac{1-q}{q^{\frac{1}{2}}}\psi^+(z)\,{\mathfrak e}(w)&=&
\prod_{j=1}^{r+1} 
\frac{(1-q^{-\frac{1}{2}}\theta^2 z x_j)(1-q^{\frac{1}{2}}\theta^{-2} z x_j)}{(1-q^{-\frac{1}{2}} z x_j)(1-q^{\frac{1}{2}} z x_j)} 
\sum_{i=1}^{r+1} \delta(q^{\frac{1}{2}}w x_i)\prod_{k\neq i}\frac{\theta x_i-\theta^{-1}x_k}{x_i-x_k}\Gamma_i \\
&=&\sum_{i=1}^{r+1} \delta(q^{\frac{1}{2}}w x_i)\frac{(w-q^{-1}t z)(w-t^{-1} z)}{(w-q^{-1} z)(w-z)}\\
&&\qquad \times
\prod_{k\neq i}\frac{(1-q^{-\frac{1}{2}}\theta^2 z x_k)(1-q^{\frac{1}{2}}\theta^{-2} z x_k)}{(1-q^{-\frac{1}{2}} z x_k)(1-q^{\frac{1}{2}} z x_k)}\frac{\theta x_i-\theta^{-1}x_k}{x_i-x_k}\Gamma_i ,
\end{eqnarray*}
\begin{eqnarray*}
\frac{1-q}{q^{\frac{1}{2}}}{\mathfrak e}(w)\,\psi^+(z)&=&
\sum_{i=1}^{r+1} \delta(q^{\frac{1}{2}}w x_i)\prod_{k\neq i}\frac{\theta x_i-\theta^{-1}x_k}{x_i-x_k}\Gamma_i \prod_{j=1}^{r+1} 
\frac{(1-q^{-\frac{1}{2}}\theta^2 z x_j)(1-q^{\frac{1}{2}}\theta^{-2} z x_j)}{(1-q^{-\frac{1}{2}} z x_j)(1-q^{\frac{1}{2}} z x_j)}  \\
&=&\sum_{i=1}^{r+1} \delta(q^{\frac{1}{2}}w x_i)
\frac{(w-t z)(w-q t^{-1} z)}{(w- z)(w-q z)}\\
&&\qquad \times
\prod_{k\neq i}\frac{\theta x_i-\theta^{-1}x_k}{x_i-x_k}
\frac{(1-q^{-\frac{1}{2}}\theta^2 z x_k)(1-q^{\frac{1}{2}}\theta^{-2} z x_k)}{(1-q^{-\frac{1}{2}} z x_k)(1-q^{\frac{1}{2}} z x_k)}\Gamma_i .
\end{eqnarray*}
Using
$$ g(z,w)\frac{(w-q^{-1}t z)(w-t^{-1} z)}{(w-q^{-1} z)(w-z)}=-g(w,z)\frac{(w-t z)(w-q t^{-1} z)}{(w- z)(w-q z)}$$
we see that $g(z,w)\psi^+(z)\,{\mathfrak e}(w)+g(w,z){\mathfrak e}(w)\,\psi^+(z)=0$. The derivation of the $\psi^-$
equation follows a similar calculation.
\end{proof}

The final two relations are the cubic, Serre-type relations for the fundamental currents.
\begin{thm}\label{thserrre}
The fundamental currents satisfy the cubic relations
\begin{eqnarray}
{\rm Sym}_{z_1,z_2,z_3}\left( \frac{z_2}{z_3} {\Big[}{\mathfrak e}(z_1),{[}{\mathfrak e}(z_2),{\mathfrak e}(z_3){]}{\Big]}\right)&=&0  \label{serre1}, \\
{\rm Sym}_{z_1,z_2,z_3}\left( \frac{z_2}{z_3} {\Big[}{\mathfrak f}(z_1),{[}{\mathfrak f}(z_2),{\mathfrak f}(z_3){]}{\Big]}\right)&=&0 \label{serre2}.
\end{eqnarray}
\end{thm}
\begin{proof}
It is sufficient to prove the statement of the theorem for
${\mathfrak e}$, eq. \eqref{serre1}, as
that for ${\mathfrak f}$ follows by the substitution $(q,t)\to (q^{-1},t^{-1})$.
The proof of \eqref{serre1} is straightforward but extremely tedious. 
We prefer to postpone it to Section \ref{shuprosec}, where it reduces
to a suffle product identity (Theorem \ref{shufserre}) which is considerably easier to prove.
%As above, we only prove the statement \eqref{serre1} for the current $\mathfrak e(z)$, as the relations for $\mathfrak f(z)$, \eqref{serre2}
%follow by taking $(q,t)\to (q^{-1},t^{-1})$. Let $\mu=(1-q)^3 q^{-3/2}$.
%\begin{eqnarray*}
%&&\mu\, {\mathfrak e}(z_1){\mathfrak e}(z_2){\mathfrak e}(z_3)=
%\sum_{i_1,i_2,i_3\atop i_1\neq i_2 \neq i_3}\delta(q^{\frac{1}{2}}z_1x_{i_1})\delta(q^{\frac{1}{2}}z_2x_{i_2})\delta(q^{\frac{1}{2}}z_3x_{i_3})\frac{\theta x_{i_1}-\theta^{-1}x_{i_2}}{x_{i_1}-x_{i_2}}\frac{\theta x_{i_1}-\theta^{-1}x_{i_3}}{x_{i_1}-x_{i_3}}
%\\
%&&\qquad \times\quad \frac{\theta x_{i_2}-q \theta^{-1}x_{i_1}}{x_{i_2}-qx_{i_1}}\frac{\theta x_{i_2}-\theta^{-1}x_{i_3}}{x_{i_2}-x_{i_3}}
%\frac{\theta x_{i_3}-q\theta^{-1}x_{i_1}}{x_{i_3}-qx_{i_1}}\frac{\theta x_{i_3}-q\theta^{-1}x_{i_2}}{x_{i_3}-qx_{i_2}}\\
%&&\qquad \times \quad \left(
%\prod_{k\neq i_1,i_2,i_3}
%\frac{\theta x_{i_1}-\theta^{-1}x_k}{x_{i_1}-x_k}\frac{\theta x_{i_2}-\theta^{-1}x_k}{x_{i_2}-x_k}\frac{\theta x_{i_3}-\theta^{-1}x_k}{x_{i_3}-x_k}\right)\, \Gamma_{i_1}\Gamma_{i_2}\Gamma_{i_3}\\
%&&+\sum_{i_1=i_2\neq i_3}\delta(q^{\frac{1}{2}}z_1x_{i_1})\delta(q^{\frac{3}{2}}z_2x_{i_1})\delta(q^{\frac{1}{2}}z_3x_{i_3})\frac{\theta x_{i_1}-\theta^{-1}x_{i_3}}{x_{i_1}-x_{i_3}}\frac{\theta q x_{i_1}-\theta^{-1}x_{i_3}}{q x_{i_1}-x_{i_3}}\frac{\theta x_{i_3}-q^2\theta^{-1}x_{i_1}}{x_{i_3}-q^2x_{i_1}}\\
%&&\qquad \times \quad \left(\prod_{k\neq i_1,i_3}
%\frac{\theta x_{i_1}-\theta^{-1}x_k}{x_{i_1}-x_k}\frac{\theta q x_{i_1}-\theta^{-1}x_k}{q x_{i_1}-x_k}\frac{\theta x_{i_3}-\theta^{-1}x_k}{x_{i_3}-x_k}\right)\, \Gamma_{i_1}^2\Gamma_{i_3}\\
%&&+\sum_{i_1\neq i_2=i_3}\delta(q^{\frac{1}{2}}z_1x_{i_1})\delta(q^{\frac{1}{2}}z_2x_{i_2})\delta(q^{\frac{3}{2}}z_3x_{i_2})\frac{\theta x_{i_1}-\theta^{-1}x_{i_2}}{x_{i_1}-x_{i_2}}\frac{\theta  x_{i_2}-q \theta^{-1}x_{i_1}}{x_{i_2}-q x_{i_1}}\frac{\theta q x_{i_2}-q \theta^{-1}x_{i_1}}{q x_{i_2}-q x_{i_1}}\\
%&&\qquad \times \quad \left(\prod_{k\neq i_1,i_2}
%\frac{\theta x_{i_1}-\theta^{-1}x_k}{x_{i_1}-x_k}\frac{\theta x_{i_2}-\theta^{-1}x_k}{x_{i_2}-x_k}\frac{\theta q x_{i_2}-\theta^{-1}x_k}{q x_{i_2}-x_k}\right)\, \Gamma_{i_1}\Gamma_{i_2}^2\\
%&&+\sum_{i_1=i_3\neq i_2}\delta(q^{\frac{1}{2}}z_1x_{i_1})\delta(q^{\frac{1}{2}}z_2x_{i_2})\delta(q^{\frac{3}{2}}z_3x_{i_1})\frac{\theta x_{i_1}-\theta^{-1}x_{i_2}}{x_{i_1}-x_{i_2}}\frac{\theta  x_{i_2}-q \theta^{-1}x_{i_1}}{x_{i_2}-q x_{i_1}}\frac{\theta q x_{i_1}-q \theta^{-1}x_{i_2}}{q x_{i_1}-q x_{i_2}}\\
%&&\qquad \times \quad \left(\prod_{k\neq i_1,i_2}
%\frac{\theta x_{i_1}-\theta^{-1}x_k}{x_{i_1}-x_k}\frac{\theta x_{i_2}-\theta^{-1}x_k}{x_{i_2}-x_k}\frac{\theta q x_{i_1}-\theta^{-1}x_k}{q x_{i_1}-x_k}\right)\, \Gamma_{i_1}\Gamma_{i_2}\Gamma_{i_1}\\
%&&+\sum_{i_1=i_2=i_3}\delta(q^{\frac{1}{2}}z_1x_{i_1})\delta(q^{\frac{3}{2}}z_2x_{i_1})\delta(q^{\frac{5}{2}}z_3x_{i_1})\\
%&&\qquad \times \quad \left(\prod_{k\neq i_1}
%\frac{\theta x_{i_1}-\theta^{-1}x_k}{x_{i_1}-x_k}\frac{\theta q x_{i_1}-\theta^{-1}x_k}{q x_{i_1}-x_k}\frac{\theta q^2x_{i_1}-\theta^{-1}x_k}{q^2x_{i_1}-x_k}\right)\, \Gamma_{i_1}^3, 
%\end{eqnarray*}
%which may be rewritten as:
%\begin{eqnarray*}
%&&\mu\, {\mathfrak e}(z_1){\mathfrak e}(z_2){\mathfrak e}(z_3)=t^3\frac{z_{1}-t^{-1}z_{2}}{z_{1}-z_{2}}\frac{z_{1}-t^{-1}z_{3}}{z_{1}-z_{3}}\frac{z_{2}-qt^{-1}z_{1}}{z_{2}-q z_{1}}\frac{z_{2}-t^{-1}z_{3}}{z_{2}-z_{3}}\\
%&&\qquad\times \quad \frac{z_{3}-qt^{-1}z_{1}}{z_{3}-q z_{1}}\frac{z_{3}-qt^{-1}z_{2}}{z_{3}-q z_{2}}
%\sum_{i_1,i_2,i_3\atop i_1\neq i_2 \neq i_3}\delta(q^{\frac{1}{2}}z_1x_{i_1})\delta(q^{\frac{1}{2}}z_2x_{i_2})\delta(q^{\frac{1}{2}}z_3x_{i_3}) \\
%&&\qquad \times \quad \left(
%\prod_{k\neq i_1,i_2,i_3}
%\frac{\theta -q^{1/2}\theta^{-1}z_1x_k}{1-q^{1/2}z_1x_k}\frac{\theta -q^{1/2}\theta^{-1}z_2x_k}{1-q^{1/2}z_2x_k}\frac{\theta -q^{1/2}\theta^{-1}z_3x_k}{1-q^{1/2}z_3x_k}\right)\, \Gamma_{i_1}\Gamma_{i_2}\Gamma_{i_3}\\
%&&+\theta^3\delta\left(\frac{z_1}{q z_2}\right)\frac{z_3-q t^{-1}z_2}{z_3-q z_2}\frac{z_3-t^{-1}z_2}{ z_3-z_2}\frac{z_2-q t^{-1}z_3}{z_2-q z_3}\sum_{i_1=i_2\neq i_3}\delta(q^{\frac{1}{2}}z_1x_{i_1})\delta(q^{\frac{1}{2}}z_3x_{i_3})\\
%&&\qquad \times \quad \left(\prod_{k\neq i_1,i_3}
%\frac{\theta -q^{1/2}\theta^{-1}z_1x_k}{1-q^{1/2}z_1x_k}\frac{\theta -q^{1/2}\theta^{-1}z_2x_k}{1-q^{1/2}z_2x_k}\frac{\theta -q^{1/2}\theta^{-1}z_3x_k}{1-q^{1/2}z_3x_k}\right)\, \Gamma_{i_1}^2\Gamma_{i_3}\\
%&&+\theta^3\delta\left(\frac{z_2}{q z_3}\right)\frac{z_2-t^{-1}z_1}{z_2-z_1}\frac{z_1-q t^{-1}z_2}{z_1-q z_2}\frac{z_1-t^{-1}z_2}{z_1-z_2}\sum_{i_1\neq i_2=i_3}\delta(q^{\frac{1}{2}}z_1x_{i_1})\delta(q^{\frac{1}{2}}z_2x_{i_2})\\
%&&\qquad \times \quad \left(\prod_{k\neq i_1,i_2}
%\frac{\theta -q^{1/2}\theta^{-1}z_1x_k}{1-q^{1/2}z_1x_k}\frac{\theta -q^{1/2}\theta^{-1}z_2x_k}{1-q^{1/2}z_2x_k}\frac{\theta -q^{1/2}\theta^{-1}z_3x_k}{1-q^{1/2}z_3x_k}\right)\, \Gamma_{i_1}\Gamma_{i_2}^2\\
%&&+\theta^3\delta\left(\frac{z_1}{q z_3}\right)\frac{z_2-t^{-1}z_1}{z_2-z_1}\frac{z_1-q t^{-1}z_2}{z_1-q z_2}\frac{z_2-t^{-1}z_1}{z_2-z_1}\sum_{i_1=i_3\neq i_2}\delta(q^{\frac{1}{2}}z_1x_{i_1})\delta(q^{\frac{1}{2}}z_2x_{i_2})\\
%&&\qquad \times \quad \left(\prod_{k\neq i_1,i_2}
%\frac{\theta -q^{1/2}\theta^{-1}z_1x_k}{1-q^{1/2}z_1x_k}\frac{\theta -q^{1/2}\theta^{-1}z_2x_k}{1-q^{1/2}z_2x_k}\frac{\theta -q^{1/2}\theta^{-1}z_3x_k}{1-q^{1/2}z_3x_k}\right)\, \Gamma_{i_1}\Gamma_{i_2}\Gamma_{i_1}\\
%&&+\delta\left(\frac{z_1}{q z_2}\right)\delta\left(\frac{z_2}{q z_3}\right)\sum_{i_1=i_2=i_3}\delta(q^{\frac{1}{2}}z_1x_{i_1})\\
%&&\qquad \times \quad \left(\prod_{k\neq i_1}
%\frac{\theta -q^{1/2}\theta^{-1}z_1x_k}{1-q^{1/2}z_1x_k}\frac{\theta -q^{1/2}\theta^{-1}z_2x_k}{1-q^{1/2}z_2x_k}\frac{\theta -q^{1/2}\theta^{-1}z_3x_k}{1-q^{1/2}z_3x_k}\right)\, \Gamma_{i_1}^3.
%\end{eqnarray*}
%Using the notations:
%\begin{eqnarray*}
%F_a&=&\sum_{i_1}\left(\delta(q^{\frac{1}{2}}z_ax_{i_1})\prod_{k\neq i_1}
%\prod_{j=1}^3\frac{1 -q^{1/2}t^{-1}z_jx_k}{1-q^{1/2}z_jx_k} \right)\, \Gamma_{i_1}^3,\\
%F_{a,b}&=&\sum_{i_1\neq i_2}\left(
%\delta(q^{\frac{1}{2}}z_ax_{i_1})\delta(q^{\frac{1}{2}}z_bx_{i_2})\prod_{j=1}^3\prod_{k\neq i_1,i_2}
%\frac{1 -q^{1/2}t^{-1}z_jx_k}{1-q^{1/2}z_jx_k}\right)\, \Gamma_{i_1}^2\Gamma_{i_2},\\
%F_{1,2,3}&=&\sum_{i_1\neq i_2\neq i_3}\left(
%\prod_{j=1}^3\delta(q^{\frac{1}{2}}z_jx_{i_j})\prod_{k\neq i_1,i_2,i_3}
%\frac{1 -q^{1/2}t^{-1}z_jx_k}{1-q^{1/2}z_jx_k}\right)\, \Gamma_{i_1}\Gamma_{i_2}\Gamma_{i_3},\\
%\sigma_{a,b}&=&\frac{z_a-t^{-1}z_b}{z_a-z_b},\qquad \tau_{a,b}=\frac{z_a-qt^{-1}z_b}{z_a-qz_b},\\
%\delta_{i,j}&=&\delta\left(\frac{z_i}{q z_j}\right),
%\end{eqnarray*}
%we may rewrite the above result as:
%\begin{eqnarray*}
%\mu\, t^{-3r/2}\,{\mathfrak e}(z_1){\mathfrak e}(z_2){\mathfrak e}(z_3)&=&\sigma_{1,2}\sigma_{1,3}\tau_{2,1}\sigma_{2,3}\tau_{3,1}\tau_{3,2} \,F_{1,2,3} +
%\tau_{2,3}\tau_{3,2}\sigma_{3,2} \, \delta_{1,2}\,F_{1,3}\\
%&&\qquad +\tau_{1,2}\sigma_{2,1}\left( \delta_{2,3}\,\sigma_{1,2}\, F_{2,1}+\delta_{1,3}\, \sigma_{2,1}\, F_{1,2}\right)+\delta_{1,2}\delta_{2,3} \, F_1.
%\end{eqnarray*}
%Note that $F_{1,2,3}$ is symmetric in $z_1,z_2,z_3$. We have:
%\begin{eqnarray*}
%\mu\, t^{-3r/2}\big[{\mathfrak e}(z_1),[{\mathfrak e}(z_2),{\mathfrak e}(z_3)]\big]&=&(\tau_{1,2}\tau_{1,3}\sigma_{2,1}\sigma_{3,1}-\sigma_{1,2}\sigma_{1,3}\tau_{2,1}\tau_{3,1})(\tau_{2,3}\sigma_{3,2}-\sigma_{2,3}\tau_{3,2})\, F_{1,2,3}\\
%&&+\delta_{1,2}\,F_{1,3} (\tau_{2,3}\tau_{3,2}\sigma_{3,2}- \tau_{1,3}\sigma_{3,1}^2)
%+\delta_{3,2}\, F_{3,1}(\tau_{2,1}\tau_{1,2}\sigma_{1,2}-\tau_{1,3}\sigma_{3,1}\sigma_{1,3})\\
%&&\!\!\!\!\!\!+\delta_{1,3}\,F_{1,2}(\tau_{1,2}\sigma_{2,1}^2-\tau_{2,3}\tau_{3,2}\sigma_{2,3})
%+\delta_{2,3}\, F_{2,1}( \tau_{1,2}\sigma_{2,1}\sigma_{1,2}-\tau_{3,1}\tau_{1,3}\sigma_{1,3})\\
%&&+ \delta_{2,1}\, F_{2,3}\sigma_{3,2}(\tau_{3,2}\sigma_{2,3}-\tau_{2,3}\sigma_{3,2})
%+\delta_{3,1}\, F_{3,2}\sigma_{2,3}(\tau_{3,2}\sigma_{2,3}-\tau_{2,3}\sigma_{3,2})\\
%%&&\quad +\tau_{2,3}\tau_{3,2}\left(\sigma_{3,2} \, \delta_{1,2}\,F_{1,3}-\sigma_{2,3} \, \delta_{1,3}\,F_{1,2}\right)\\
%%&&\quad + \tau_{1,2}\sigma_{2,1}\left( \delta_{2,3}\,\sigma_{1,2}\, F_{2,1}+\delta_{1,3}\, \sigma_{2,1}\, F_{1,2}\right)\\
%%&&\quad - \tau_{1,3}\sigma_{3,1}\left( \delta_{3,2}\,\sigma_{1,3}\, F_{3,1}+\delta_{1,2}\, \sigma_{3,1}\, F_{1,3}\right)\\
%%&&\quad -\tau_{3,1}\tau_{1,3}\sigma_{1,3} \, \delta_{2,3}\,F_{2,1}+\tau_{2,1}\tau_{1,2}\sigma_{1,2} \, \delta_{3,2}\,F_{3,1}\\
%%&&-\tau_{2,3}\sigma_{3,2}\left( \delta_{3,1}\,\sigma_{2,3}\, F_{3,2}+\delta_{2,1}\, \sigma_{3,2}\, F_{2,3}\right)\\
%%&&+\tau_{3,2}\sigma_{2,3}\left( \delta_{2,1}\,\sigma_{3,2}\, F_{2,3}+\delta_{3,1}\, \sigma_{2,3}\, F_{3,2}\right)\\
%&&\quad +(\delta_{1,2}\delta_{2,3}-\delta_{1,3}\delta_{3,2}) \, F_1-\delta_{2,3}\delta_{3,1} \, F_2+\delta_{3,2}\delta_{2,1} \, F_3.
%\end{eqnarray*}
%Finally, we compute
%${\rm Sym}_{z_1,z_2,z_3}\left( \frac{z_2}{z_3} {\Big[}{\mathfrak e}(z_1),{[}{\mathfrak e}(z_2),{\mathfrak e}(z_3){]}{\Big]}\right)$
%by collecting the contributions proportional to $F_{1,2,3}$, $F_{i,j}$ for $1\leq i\neq j\leq 3$ and $F_i$ for $i=1,2,3$, 
%which are all found to vanish. For instance, the coefficient of $F_1$ reads:
%$$(\delta_{1,2}\delta_{2,3}-\delta_{1,3}\delta_{3,2})+(\delta_{1,3}\delta_{3,2}-\delta_{1,2}\delta_{2,3})-\delta_{1,3}\delta_{3,2}-\delta_{1,2}\delta_{2,3}+\delta_{1,2}\delta_{2,3}+\delta_{1,3}\delta_{3,2}=0$$
%while that of $F_{1,2}$ is:
%\begin{eqnarray*}&&\!\!\!\!\!\!\!\!\!\!\delta_{1,3}(\tau_{1,2}\sigma_{2,1}^2-\tau_{2,3}\tau_{3,2}\sigma_{2,3})
%+\delta_{1,3}(\tau_{3,2}\tau_{2,3}\sigma_{2,3}- \tau_{1,2}\sigma_{2,1}^2)
%+\delta_{1,3}(\tau_{3,2}\tau_{2,3}\sigma_{2,3}-\tau_{2,1}\sigma_{1,2}\sigma_{2,1})\\
%&&\!\!\!\!\!\!\!\!\!\!+\delta_{1,3}( \tau_{2,1}\sigma_{1,2}\sigma_{2,1}-\tau_{3,2}\tau_{2,3}\sigma_{2,3})
%+ \delta_{1,3}\sigma_{2,1}(\tau_{2,1}\sigma_{1,2}-\tau_{1,2}\sigma_{2,1})
%+\delta_{1,3}\sigma_{2,1}(\tau_{1,2}\sigma_{2,1}-\tau_{2,1}\sigma_{1,2})=0.
%\end{eqnarray*}
%The Theorem follows. 
\end{proof}
We conclude that the currents ${\mathfrak e}(z)$, ${\mathfrak f}(z)$, $\psi^\pm(z)$ satisfy all the relations of the
level $(0,0)$ quantum toroidal algebra ${\mathfrak gl}_1$, and Theorem \ref{gentoro} follows.



