In this section, we use the known relation between DAHA and Elliptic Hall Algebra
to obtain new relations between our generalized Macdonald operators. 

\subsection{Elliptic Hall Algebra: definition and isomorphisms}

In this section, we recall known results from \cite{BS,SCHIFFEHA}.
Let
\begin{equation}\label{defalpha}
\al_k:=\frac{1}{k}(1-q^k)(1-t^{-k})(1-q^{-k}t^k) ,\qquad (k\in \Z^*) .
\end{equation}

\begin{defn}\label{EHAdefn}
The Elliptic Hall Algebra (EHA) has generators $u_{a,b},\theta_{a,b}$, $a,b\in \Z^*$,
subject to the commutations:
$$\!\!\!\!\!\!\!\!\!\!(E1)\quad [u_{c,d},u_{a,b}]=0 \qquad \hbox{\rm if (0,0), (a,b), (c,d)  are aligned}
$$
and
$$\qquad\qquad (E2)\quad  [u_{c,d},u_{a,b}]=\frac{\epsilon_{a,b;c,d}}{\al_1}\theta_{a+c,b+d}, \quad {\rm if}\, gcd(a,b)=1\quad 
{\rm and}\quad \Delta_{a,b;c,d}=\emptyset$$
where $\Delta_{a,b;c,d}$ is the intersection with $\Z^2$ of the strict interior of the triangle $(0,0),(a,b),(a+c,b+d)$,
and $\epsilon_{a,b;c,d}={\rm sgn}(ad-bc)$.
The generators $\theta$ and $u$ for non-coprime indices are further related via:
$$(E3)\quad 1+\sum_{n=1}^\infty \theta_{n(a,b)} z^n =e^{\sum_{k=1}^\infty \al_k u_{k(a,b)} z^k}\qquad (gcd(a,b)=1)$$
and in particular $\theta_{n,p}=\al_1\,u_{n,p}$ whenever $gcd(n,p)=1$.
\end{defn}

The EHA is isomorphic to the quantum toroidal algebra of Section \ref{qtorsec} via the following
assignments:
\begin{equation}\label{ehatoqtor}
e(z)=\sum_{n\in\Z} u_{1,n} \,z^n,\quad f(z)=\sum_{n\in\Z} u_{-1,n} \,z^n,\quad 
\psi_{\pm}(z)=1+\sum_{n\geq 1} \theta_{0,\pm n}\, z^{\pm n}
\end{equation}

There is a natural action of $SL(2,\Z)$ on the generators $u_{a,b}$ given by:
$$ \begin{pmatrix} a_0 & a_1 \\ a_2 & a_3 \end{pmatrix}\cdot u_{a,b}=u_{a_0 a+a_2 b,a_1 a+a_3 b}$$
For further use, we single out the two generators:
\begin{equation}\label{defTU}
T=\begin{pmatrix} 1 & 1 \\ 0 & 1 \end{pmatrix}\qquad {\rm and}\qquad U=\begin{pmatrix} 1 & 0 \\ 1 & 1 \end{pmatrix}
\end{equation}
acting on the EHA generators respectively as $T \, u_{a,b}=u_{a,a+b}$ and $U\, u_{a,b}=u_{a+b,b}$.

In \cite{SHIVAS}, the authors constructed an isomorphism between the EHA and the Spherical DAHA 
(in infinitely many variables $X_i$, $i\in \N$). In particular, the natural $SL(2,\Z)$ action on the EHA
maps onto the natural $SL(2,\Z)$ action on the DAHA \cite{Cheredbook}, namely: 
$$T\mapsto \tau_+={\rm ad}_{\gamma^{-1}}, \quad {\rm and}\quad U\mapsto \tau_-={\rm ad}_{\eta^{-1}}$$
with $\gamma$ and $\eta$ as in \eqref{gammadefn} and \eqref{etadefn}, and Lemmas \ref{taupluslemma} and
\ref{taumoinslemma} respectively.

The isomorphism maps the generators $u_{k,0}$
to the power sums $P_k:=\sum_i (Y_i)^k$ operators in the spherical version of DAHA,
which is respected by the functional representation of Sect. \ref{secpol} above.

\subsection{EHA representation via generalized Macdonald operators}

We are now ready to complete the identification of generators of the EHA in terms of generalized 
Macdonald operators.

%\color{red} check involutive isomorphisms of DAHA in Cherednik's book like:
%$$\begin{pmatrix}\end{pmatrix}:\ (X_i\to X_i, Y_i\to Y_i^{-1},q\to q^{-1},t\to t^{-1})$$
%$$(X_i\to X_i^{-1}, Y_i\to Y_i^{-1},T_i\to T_i^{-1},q\to q,t\to t)$$
%$$\begin{pmatrix}0 & -1 \\ -1 & 0\end{pmatrix}:\ (X_i\to Y_i, Y_i\to X_i,T_i\to T_i^{-1},q\to q^{-1},t\to t^{-1})$$
%\color{black}

Comparing \eqref{ehatoqtor}
with the representation of Theorem \ref{gentoro} involving generalized Macdonald operators, 
we easily deduce the following representation of the EHA, or rather a quotient thereof corresponding 
to the finite number $N$ of variables\footnote{These assignments first appeared in \cite{Miki07} 
Proposition 3.3, with different notations $(q,\gamma,y_i)\mapsto (\theta=t^{1/2},q/\theta,x_i)$.}:
\begin{equation} u_{1,n}=\frac{q^{1/2}}{1-q} q^{\frac{n}{2}}\,{\mathcal D}_{1;n}^{q,t}
=\frac{q^{1/2}}{1-q} q^{\frac{n}{2}}t^{\frac{1-N}{2}}\,{\mathcal M}_{n},\qquad   
u_{-1,n}=\frac{q^{-1/2}}{1-q^{-1}} q^{-\frac{n}{2}}\,{\mathcal D}_{1;n}^{q^{-1},t^{-1}},
\end{equation}
whereas the relation:
$$
1+\sum_{n\geq 1} \theta_{0,\pm n}\, z^{\pm n}
=e^{\sum_{k\geq 1} \frac{z^{\pm k}}{k}p_{\pm k} q^{\frac{k}{2}}(1-t^{-k})(1-t^kq^{-k})}
$$
fixes the values:
\begin{equation} u_{0,\pm k}= \frac{q^{\frac{k}{2}}}{(1-q^k)} p_{\pm k}\qquad (k\in \Z_{>0})
\end{equation}

Moreover, from the spherical DAHA homomorphism, it is easy to identify the $u_{k,0}$ from
the DAHA power sum operators, which in the functional representation are:
$${\mathcal P}_k := \sum_{i=1}^N (Y_i)^k \vert_{{\mathcal S}_N} $$
and are related for $k>0$ to the Macdonald operators ${\mathcal D}_\al\equiv {\mathcal D}_{\al;0}^{q,t}$ via:
\begin{equation}\label{posiP}\prod_{i=1}^N(1-z Y_i)\vert_{{\mathcal S}_N}=\sum_{\al=0}^N (-z)^\al {\mathcal D}_\al =
e^{-\sum_{k\geq 1} {\mathcal P}_k \frac{z^k}{k}} 
\end{equation}
while for $k<0$ they are expressed in terms of the dual operators 
$\bar{\mathcal D}_\al\equiv {\mathcal D}_{\al;0}^{q^{-1},t^{-1}}$ via:
\begin{equation}\label{negaP}\sum_{\al=0}^N (-z)^\al \bar{\mathcal D}_\al =
e^{-\sum_{k\geq 1} {\mathcal P}_{-k} \frac{z^k}{k}} 
\end{equation}

Finally from the action of $-\epsilon$ which maps $u_{0,k}$ to $u_{k,0}$, we have the identification:
\begin{equation} u_{\pm k,0}= \frac{q^{\frac{k}{2}}}{(1-q^k)} {\mathcal P}_{\pm k}\qquad (k\in \Z_{>0})
\end{equation}

\begin{remark}
We continue the comparison with the work of \cite{BGLX} started in Remarks \ref{nablarem} and
\ref{betternablarem}. In this paper a connection to the positive quadrant of the Elliptic Hall Algebra
(with generators $u_{a,b}\to Q_{a,b}$ with $a,b\geq 0$) was found by setting
$$Q_{1,k}= D_k \quad (k\geq 0), \quad {\rm and} \quad Q_{0,1}=-e_1=-p_1 .$$
for $k\geq 0$,
where the $D_k$'s are defined via their generating current in \eqref{fromBG}, and extending the identification
via the $SL(2,Z)$ action on the EHA. 
By Theorem \ref{BGconnect}, we have the relations
\begin{eqnarray*}
\lim_{N\to \infty} t^{1-N}\, {\mathcal M}_{n}&=& \frac{1}{1-t^{-1}}(\Sigma^{-1}\, D_n\, \Sigma)\vert_{t\to t^{-1}} \\
\lim_{N\to \infty} -p_1(x_1,x_2,...,x_N) &=&-p_1(x_1,x_2,...)
\end{eqnarray*}
Comparing with our functional representation of EHA in the limit of infinite number of variables of $x_1,x_2,...$, we find:
$$ \lim_{N\to \infty} t^{\frac{1-N}{2}}\, u_{1,k}=
\frac{q^{\frac{k+1}{2}}}{(1-q)(1-t^{-1})}\,(\Sigma^{-1}\, Q_{1,k}\, \Sigma)\vert_{t\to t^{-1}}$$
and
$$\lim_{N\to \infty} u_{0,1}=\frac{q^{\frac{1}{2}}}{(1-q)(1-t^{-1})}\, (\Sigma^{-1}\, (-p_1)\, \Sigma)\vert_{t\to t^{-1}}$$
Moreover, by using \eqref{nabnab}, and $\nabla^{(N)}u_{a,b}{\nabla^{(N)}}^{-1}=u_{a+b,b}$ we have:
\begin{eqnarray*}
&&\!\!\!\!\!\!\!\!\!\!\!\! \lim_{N\to \infty} t^{\frac{(1-N)(k+1)}{2}} \nabla^{(N)}u_{1,k}{\nabla^{(N)}}^{-1}=
\lim_{N\to \infty} t^{\frac{(1-N)(k+1)}{2}} u_{k+1,k}\\
&=&\lim_{N\to \infty}  t^{\frac{(1-N)(k+1)}{2}}(t^{\frac{N-1}{2}}q^{\frac{1}{2}})^{d}\, (\Sigma^{-1}\, \nabla\, \Sigma)\vert_{t\to t^{-1}} u_{1,k} (\Sigma^{-1}\, \nabla^{-1}\, \Sigma)\vert_{t\to t^{-1}}\,(t^{\frac{N-1}{2}}q^{\frac{1}{2}})^{-d}\\
&=&\frac{q^{\frac{k+1}{2}}}{(1-q)(1-t^{-1})}\lim_{N\to \infty} t^{\frac{(1-N)(k)}{2}} (t^{\frac{N-1}{2}}q^{\frac{1}{2}})^{d}\, (\Sigma^{-1}\, \nabla\, Q_{1,k}\, \nabla^{-1} \Sigma)\vert_{t\to t^{-1}} 
\,(t^{\frac{N-1}{2}}q^{\frac{1}{2}})^{-d}
\end{eqnarray*}
Note that the action of $Q_{1,k}$ on ${\tilde H}_\lambda$ is a linear combination of terms in which $k$ boxes are added
to $\lambda$, and similarly for $u_{1,k}$  acting on $P_\lambda$. We deduce that the conjugation
by $(t^{\frac{N-1}{2}}q^{\frac{1}{2}})^{d}$ amounts to a factor $(t^{\frac{N-1}{2}}q^{\frac{1}{2}})^{k}$, and using
$\nabla Q_{a,b}\nabla^{-1}=Q_{a+b,b}$ (from \cite{BGLX}) we finally get:
$$\lim_{N\to \infty} t^{\frac{(1-N)(k+1)}{2}}u_{k+1,k}=\frac{q^{\frac{2k+1}{2}}}{(1-q)(1-t^{-1})}
(\Sigma^{-1}\, Q_{k+1,k} \Sigma)\vert_{t\to t^{-1}} 
$$
Repeating the inductive proof of \cite{BGLX}, we arrive in general at:
$$ \lim_{N\to \infty} t^{\frac{(1-N)a}{2}}u_{a,b}=\frac{q^{\frac{a+b}{2}}}{(1-q)(1-t^{-1})}
(\Sigma^{-1}\, Q_{a,b} \Sigma)\vert_{t\to t^{-1}} $$
for any coprime $(a,b)$.
%In particular, taking $a=0$, $b=k$, we identify
%$$\lim_{N\to \infty} p_k= \frac{1-q^k}{(1-q)(1-t^{-1})}\,  (\Sigma^{-1}\, Q_{0,k} \Sigma)\vert_{t\to t^{-1}}$$
\end{remark}

\subsection{EHA and relations between generalized Macdonald operators}

The EHA representation provides us with an alternative way of finding relations 
between the generalized Macdonald operators. We first concentrate on an alternative version
of Theorem \ref{Mofm}, aimed at expressing the generalized Macdonald operator ${\mathcal M}_{\al;n}$
as an explicit polynomial of the ${\mathcal M}_i$'s.

%The main tool is the identification of the $SL(2,\Z)$ generator $T\equiv {\rm ad}_{\gamma^{-1}}$.

\begin{thm}\label{polpol}
The operator ${\mathcal D}_{\al;n}$ is expressible as a homogeneous polynomial of degree $\al$ in the variables
${\mathcal D}_{1;n},{\mathcal D}_{1;n\pm 1}$, with coefficients in $\C(q,t)$.
\end{thm}
\begin{proof}
Let us first use the definition of EHA to compute $\theta_{n,0}$ as an iterated commutator. We first note that
\begin{eqnarray*}u_{n-1,1}&=&[u_{n-2,1},u_{1,0}]=\cdots=\big[ [\cdots [ u_{1,1},u_{1,0}],u_{1,0}]\cdots , u_{1,0}\big]\\
&=&\frac{q^{\frac{n}{2}}}{(1-q)^{n-1}}
\big[ [\cdots [ {\mathcal D}_{1;1},{\mathcal D}_{1;0}],{\mathcal D}_{1;0}]\cdots , {\mathcal D}_{1;0}\big]
\end{eqnarray*}
Moreover, we have:
\begin{equation}\label{thetanzero}
\theta_{n,0}=\al_1\, [u_{n-1,1},u_{1,-1}]=\frac{\al_1\, q^{\frac{n}{2}}}{(1-q)^n}\,
\big[\cdots [{\mathcal D}_{1;1},{\mathcal D}_{1;0}],{\mathcal D}_{1;0}],\ldots ,{\mathcal D}_{1;0}],{\mathcal D}_{1;-1}\big]
\end{equation}
for $n\geq 2$, while $\theta_{1,0}=\al_1 u_{1,0}=\frac{\al_1\, q^{\frac{1}{2}}}{(1-q)}\,{\mathcal D}_{1;0}$.
On the other hand, $\theta_{n,0}$ is related to the $u_{k,0}$ via the relation $(E3)$ of Definition \ref{EHAdefn} above:
\begin{equation}\label{Ptotheta}
1+\sum_{n>0} \theta_{n,0} z^n=e^{\sum_{k>0} \al_k u_{k,0} z^k} =
e^{\sum_{k>0} q^{k/2} (1-t^{-k})(1-q^{-k}t^k){\mathcal P}_k\frac{z^k}{k}}
\end{equation}
Eliminating ${\mathcal P}_k$ between this and \eqref{posiP}, we are left with an algebraic relation of the form:
$$ {\mathcal D}_{\al;0}=\varphi(\theta_{1,0},\theta_{2,0},...,\theta_{\al,0})$$
where $\varphi$ is a quasi-homogeneous polynomial of total degree $\al$ 
(assuming $\theta_{i,0}$ has degree $i$), with coefficients in $\C(q,t)$. 
Combining this with \eqref{thetanzero}, we deduce that 
${\mathcal D}_{\al;0}$ is a homogeneous polynomial of  degree $\al$ in the three variables
$({\mathcal D}_{1;1},{\mathcal D}_{1;0},{\mathcal D}_{1;-1})$, with coefficients in $\C(q,t)$. 
This proves the Theorem for $n=0$. 
To get to arbitrary $n$, we simply have to iteratively conjugate by $\gamma^{-1}$ and use  ${\mathcal D}_{\al;n}=
q^{\al n/2} \gamma^{-n} {\mathcal D}_{\al;0}\gamma^n$.
\end{proof}

\begin{cor}\label{polpolcor}
The operator ${\mathcal M}_{\al;n}$ is expressible as a homogeneous polynomial of degree $\al$ in the variables
${\mathcal M}_{n},{\mathcal M}_{n\pm 1}$, with coefficients in $\C(q,t)$.
\end{cor}
\begin{proof}
The result for ${\mathcal D}_{\al;n}$ of Theorem \ref{polpol}
immediately translates into that for ${\mathcal M}_{\al;n}$, due to the relations
${\mathcal D}_{\al;n}=t^{\al(\al-N)/2}{\mathcal M}_{\al;n}$.
\end{proof}

\begin{example}
Let us write the explicit polynomial relation of Theorem \ref{polpol} above for $\al=2,3$.
We start by expressing ${\mathcal D}_{i;0}$'s in terms of ${\mathcal P}_i$'s via \eqref{posiP}, which gives:
\begin{equation}\label{initexe}
 {\mathcal D}_{1;0}={\mathcal P}_1,\quad {\mathcal D}_{2;0}=\frac{{\mathcal P}_1^2-{\mathcal P}_2}{2},
\quad {\mathcal D}_{3;0}=\frac{{\mathcal P}_1^3-3 {\mathcal P}_1{\mathcal P}_2+2{\mathcal P}_3}{6} \ .
\end{equation}
Next, we express the ${\mathcal P}_i$'s in terms of the $\theta_{i,0}$'s via \eqref{Ptotheta}, which gives:
$${\mathcal P}_1=\frac{\theta_{1,0}}{q^{1/2}(1-t^{-1})(1-q^{-1}t)}, \ 
{\mathcal P}_2=\frac{2\theta_{2,0}-\theta_{1,0}^2}{q(1-t^{-2})(1-q^{-2}t^2)}, \ 
{\mathcal P}_3=\frac{3\theta_{3,0}-3\theta_{1,0}\theta_{2,0}+\theta_{1,0}^3}{q^{3/2}(1-t^{-3})(1-q^{-3}t^3)}\ .$$
These are reexpressed in terms of ${\mathcal D}_{1;1},{\mathcal D}_{1;0},{\mathcal D}_{1;-1}$ by using:
$$\theta_{1,0}=\al_1 \frac{q^{1/2}}{1-q}{\mathcal D}_{1;0}, \ 
\theta_{2,0}=\al_1  \frac{q}{(1-q)^2}[{\mathcal D}_{1;1},{\mathcal D}_{1;-1}],\ 
\theta_{3,0}=\al_1  \frac{q^{3/2}}{(1-q)^3}\big[ [{\mathcal D}_{1;1},{\mathcal D}_{1;0}],{\mathcal D}_{1;-1}\big],$$
as:
\begin{eqnarray*}
{\mathcal P}_1&=&{\mathcal D}_{1;0}\\
{\mathcal P}_2&=&\frac{\al_1}{(1-q)^2(1-t^{-2})(1-q^{-2}t^2)} 
\left(2 [{\mathcal D}_{1;1},{\mathcal D}_{1;-1}]
-\al_1 {\mathcal D}_{1;0}^2 \right)\\
{\mathcal P}_3&=&\frac{\al_1}{(1-q)^3(1-t^{-3})(1-q^{-3}t^3)}
\left(3\big[ [{\mathcal D}_{1;1},{\mathcal D}_{1;0}],{\mathcal D}_{1;-1}\big] -3\al_1{\mathcal D}_{1;0}[{\mathcal D}_{1;1},{\mathcal D}_{1;-1}] +\al_1^2 {\mathcal D}_{1;0}^3 \right)
\end{eqnarray*}
Substituting this into \eqref{initexe} gives the polynomial relations:
\begin{eqnarray*}
{\mathcal D}_{2;0}&=&\frac{t}{(1+t)(q+t)} \left\{(1+q){\mathcal D}_{1;0}^2-\frac{q}{1-q} \,[{\mathcal D}_{1;1},{\mathcal D}_{1;-1}] \right\}\\
{\mathcal D}_{3;0}&=&\frac{t^2}{(1+t)(q+t)(1+t+t^2) (q^2+q t+t^2)}\left\{ (q(q+t^2)+t(1+q+q^2+q^3)){\mathcal D}_{1;0}^3 \right.\\
&& \qquad \left. \!\!\!\!\!\!\!\!\!\!\!\!\!\!\!\!\!\!\!+\frac{q((1+q)(1+t)(q+t)+t(1-q+q^2))}{1-q} 
{\mathcal D}_{1;0}[{\mathcal D}_{1;1},{\mathcal D}_{1;-1}]+\frac{q^2}{(1-q)^2} 
\big[ [{\mathcal D}_{1;1},{\mathcal D}_{1;0}],{\mathcal D}_{1;-1}\big]  \right\} 
\end{eqnarray*}
Finally conjugating $n$ times w.r.t. $\gamma^{-1}$ gives:
\begin{eqnarray*}
{\mathcal D}_{2;n}&=&\frac{t}{(1+t)(q+t)} \left\{(1+q){\mathcal D}_{1;n}^2-\frac{q}{1-q} \,[{\mathcal D}_{1;n+1},{\mathcal D}_{1;n-1}] \right\}\\
{\mathcal D}_{3;n}&=&\frac{t^2}{(1+t)(q+t)(1+t+t^2) (q^2+q t+t^2)}
\left\{ (q(q+t^2)+t(1+q+q^2+q^3)){\mathcal D}_{1;n}^3 \right.\\
&& \left.  \!\!\!\!\!\!\!\!\!\!\!\!\!\!\!\!\!\!\!\!\!\!\!+\frac{q((1+q)(1+t)(q+t)+t(1-q+q^2))}{1-q} {\mathcal D}_{1;n}[{\mathcal D}_{1;n+1},{\mathcal D}_{1;n-1}]+\frac{q^2}{(1-q)^2}
\big[ [{\mathcal D}_{1;n+1},{\mathcal D}_{1;n}],{\mathcal D}_{1;n-1}\big]  \right\} 
\end{eqnarray*}
Note that the expression for ${\mathcal D}_{2;n}$ above is in agreement with \eqref{Mnn}, upon identifying
${\mathcal D}_{2;n}=t^{2-N}{\mathcal M}_{n,n}$ and ${\mathcal D}_{1;n}=t^{\frac{1-N}{2}}{\mathcal M}_{n}$.
Moreover, using also ${\mathcal D}_{3;n}=t^{3(3-N)/2}{\mathcal M}_{n,n,n}$, we get:
\begin{eqnarray*}
{\mathcal M}_{3;n}&=&{\mathcal M}_{n,n,n}=\frac{t^{-1}}{(1+t)(q+t)(1+t+t^2) (q^2+q t+t^2)}
\left\{ (q(q+t^2)+t(1+q+q^2+q^3)){\mathcal M}_{n}^3 \right.\\
&& \left. \!\!\!\!\!\!\!\!\!\!\!\!\!\!\!\!\!\!\!+\frac{q((1+q)(1+t)(q+t)+t(1-q+q^2))}{1-q}
{\mathcal M}_{n}[{\mathcal M}_{n+1},{\mathcal M}_{n-1}]+\frac{q^2}{(1-q)^2}
\big[ [{\mathcal M}_{n+1},{\mathcal M}_{n}],{\mathcal M}_{n-1}\big]  \right\} 
\end{eqnarray*}
\end{example}

