\subsection{A constant term identity for generalized Macdonald operators}
In this section we reformulate the relations between the
generating functions of the previous sections in terms of shuffle products.

\subsubsection{Constant term identities and generating currents}

We use the generating currents ${\mathfrak m}_\al(u)$ for the generalized Macdonald operators \eqref{genmacdop}:
\begin{equation}\label{defm}
{\mathfrak m}_\al(u):=\sum_{n\in \Z} u^n\, {\mathcal M}_{\al;n}
=\sum_{I\subset [1,N]\atop |I|=\al} \delta(u x_I) \prod_{i\in I\atop j\not\in I}
\frac{t x_i-x_j}{x_i - x_j} \, \Gamma_I.
\end{equation}
\begin{remark}
The currents ${\mathfrak m}_\al(u)$ and ${\mathfrak e}_\al(u)$ are related via
\begin{equation}\label{emrela}
{\mathfrak e}_\al(u)=\frac{q^{\frac{\al}{2}}t^{-\frac{\al(N-\al)}{2}}}{(1-q)^\al}\, {\mathfrak m}_\al(q^{\al/2}u)
\end{equation}
\end{remark}
Note that ${\mathfrak m}_1(u)={\mathfrak m}(u)$ of Equation \eqref{mcurdefone}. 
Theorem \ref{efrela} implies the exchange relation
\begin{equation}\label{exchange}
g(u,v) \, {\mathfrak m}(u)\, {\mathfrak m}(v)+g(v,u)\, {\mathfrak m}(v)\, {\mathfrak m}(u)=0,
\end{equation}
where $g$ is as in \eqref{defofg}. 

In terms of components, this exchange relation is equivalent to the relation between components:
\begin{equation}\label{cubicM}
\mu_{a,b}:=q t {\mathcal M}_{a-3}{\mathcal M}_{b}-(t^2+q^2 t+q){\mathcal M}_{a-2}{\mathcal M}_{b-1}+(qt^2+t+q^2){\mathcal M}_{a-1}{\mathcal M}_{b-2}-q t{\mathcal M}_{a}{\mathcal M}_{b-3}=-\mu_{b,a}
\end{equation}
for all $a,b\in\Z$.

\begin{defn}
We define the multiple constant term of any rational symmetric function $f(u_1,...,u_\al)\in {\mathcal F}_\al$ as:
\begin{equation}\label{CTdef}
CT_{u_1,...,u_\al}\left( f(u_1,...,u_\al)\right):=\prod_{i=1}^\al \oint \frac{du_i}{2i\pi\, u_i} f(u_1,...,u_\al)
\end{equation}
where the contour integral picks up the residues at $u_i=0$.
\end{defn}
In particular, we have $CT_v( f(v)\delta(u/v))=f(u)$.

\begin{defn}\label{malphadef}
To each symmetric rational function $P(x_1,x_2,...,x_\al)\in {\mathcal F}_\al$ we associate the difference operator
${\mathcal M}_\al(P)$:
\begin{equation}
{\mathcal M}_\al(P):=\frac{1}{\al!}CT_\bu\left( P(u_1^{-1},u_2^{-1},...,u_\al^{-1}) \prod_{1\leq i<j\leq \al} 
\frac{(u_i-u_j)(u_i-q u_j)}{(u_i-t u_j)(t u_i-q u_j)}\, \prod_{i=1}^\al {\mathfrak m}(u_i)\right) \label{ctdiffop}
\end{equation}
\end{defn}

We have the following remarkable result:

\begin{thm}\label{mainthm}
For any symmetric rational function $P(x_1,...,x_\al)\in {\mathcal F}_\al$, $1\leq \al\leq N$, we have the identity:
\begin{equation}
{\mathcal M}_\al(P)= {\mathcal D}_\al(P) \label{ctPqt}
\end{equation}
with ${\mathcal D}_\al(P)$ as in Definition \ref{gmacdef}.
\end{thm}
\begin{proof}
Let us compute:
\begin{eqnarray*}
&&\prod_{i<j}  \frac{u_i-u_j}{u_i-t u_j}\frac{u_i-q u_j}{t u_i-q u_j} \, \prod_{i=1}^\al {\mathfrak m}(u_i)\\
&=& \prod_{i<j} \frac{u_i-u_j}{u_i-t u_j}\frac{u_i-q u_j}{t u_i-q u_j} \sum_{i_1,i_2,...,i_\al} 
\prod_{k=1}^\al \left(\delta(u_k x_{i_k}) 
\prod_{j_k\neq i_k}\frac{t x_{i_k}-x_{j_k}}{x_{i_k}-x_{j_k}} \Gamma_{i_k}\right) \\
&=& \prod_{i<j} \frac{u_i-u_j}{u_i-t u_j}\frac{u_i-q u_j}{t u_i-q u_j} \times \\
&&\qquad \times\, \sum_{i_1\neq i_2\neq ...\neq i_\al} 
\prod_{j=1}^\al \delta(u_j x_{i_j}) 
\prod_{k<\ell }\frac{t x_{i_k}- x_{i_\ell}}{x_{i_k}- x_{i_\ell}}\frac{t x_{i_\ell}-q x_{i_k}}{x_{i_\ell}- q x_{i_k}} 
\prod_{j=1}^\al \prod_{i\neq i_1,...,i_\al}\frac{t x_{i_j}-x_i}{x_{i_j}-x_i}\, \Gamma_{i_1}\cdots \Gamma_{i_\al}
\\
&=&\sum_{i_1\neq i_2\neq ...\neq i_\al} 
\prod_{j=1}^\al \delta(u_j x_{i_j}) \prod_{i\neq i_1,...,i_\al} \frac{t x_{i_k}-x_i}{x_{i_k}-x_i}
\, \Gamma_{i_1}\cdots \Gamma_{i_\al}\\
&=&\al!\, \sum_{i_1<i_2< ...<i_\al} 
\prod_{j=1}^\al \delta(u_j x_{i_j}) \prod_{i\neq i_1,...,i_\al} \frac{t x_{i_k}-x_i}{x_{i_k}-x_i}
\, \Gamma_{i_1}\cdots \Gamma_{i_\al}\\
&=&\al!\,  \frac{1}{\al!\,(N-\al)!}{\rm Sym}\left( \prod_{k=1}^\al \delta(u_k x_{k})  
\prod_{1\leq i\leq \al<j\leq N} \frac{t x_{i}-x_j}{x_{i}-x_j} \ 
\Gamma_1 \cdots \Gamma_\al \right)
\end{eqnarray*}
We have first noted that terms with any two identical $i_k=i_\ell$, $k<\ell$, in the sum must vanish. 
This is due to the prefactor $(u_{k}-q u_{\ell})$ which when multiplying the delta function 
$\delta(u_k x_{i_k})\delta(u_\ell q x_{i_k})$ yields a zero result.
Comparing with Definition \ref{symdiffop}, 
the constant term \eqref{ctPqt} follows immediately.
\end{proof}


As a by-product of the proof of Theorem \ref{mainthm} above, we note that if $\al>N$, then there are no terms in
which all $i_k$ are distinct (as there are at most $N$ of them), thus causing the result to vanish. 
We deduce the following:

\begin{cor}\label{vanicor}
For any symmetric rational function $P(x_1,...,x_\al)\in {\mathcal F}_\al$, $\al>N$, we have:
\begin{equation}\label{vanimac}   {\mathcal M}_\al(P)= {\mathcal D}_\al(P) =0\qquad \forall\ \al>N .
\end{equation}
\end{cor}

This implies in particular that 
\begin{equation}\label{macvani}
{\mathcal M}_{N+1;n}=0 \quad \forall\ n\in \Z .
\end{equation}
in agreement with \eqref{elipquo}.

Recalling the definition \eqref{schurmacdo}, 
Theorem \ref{mainthm} has also the following immediate application to $P=s_{a_1,...,a_\al}(x_1,...,x_\al)$:

\begin{cor}\label{corctMqt}
We have:
\begin{equation}\label{ctMqt}
{\mathcal M}_{a_1,...,a_\al}=\frac{1}{\al!}CT_\bu\left( s_{a_1,...,a_\al}(\bu^{-1}) \prod_{1\leq i<j\leq \al} \frac{(u_i-u_j)
(u_i-q u_j)}{(u_i-t u_j)(t u_i-q u_j)}
\, \prod_{i=1}^\al {\mathfrak m}(u_i)\right)
\end{equation}
\end{cor}

In particular, this implies:
\begin{equation}\label{notcur}
{\mathcal M}_{\al;n}=\frac{1}{\al!}CT_\bu\left( (u_1 u_2 \cdots u_\al)^{-n} \prod_{1\leq i<j\leq \al} \frac{(u_i-u_j)
(u_i-q u_j)}{(u_i-t u_j)(t u_i-q u_j)}
\, \prod_{i=1}^\al {\mathfrak m}(u_i)\right)
\end{equation}
or equivalently in terms of the currents ${\mathfrak m}_\al(z)$ or ${\mathfrak e}_\al(z)$
of \eqref{highercur}:
\begin{cor}\label{ealine}
We have the following constant term identities:
\begin{eqnarray}
\qquad {\mathfrak m}_{\al}(z)&=&\frac{1}{\al!}CT_\bu\left( \delta(u_1 u_2 \cdots u_\al/z)\prod_{1\leq i<j\leq \al} \frac{(u_i-u_j)
(u_i-q u_j)}{(u_i-t u_j)(t u_i-q u_j)}
\, \prod_{i=1}^\al {\mathfrak m}(u_i)\right)\\
\qquad {\mathfrak e}_{\al}(z)&=&\frac{1}{\al!}CT_\bu\left( \delta(u_1 u_2 \cdots u_\al/z)\prod_{1\leq i<j\leq \al} \frac{(u_i-u_j)
(u_i-q u_j)}{(u_i-t u_j)(u_i-q/t u_j)}
\, \prod_{i=1}^\al {\mathfrak e}(u_i)\right)
\end{eqnarray}
\end{cor}
\begin{proof} The first equation is the current form of \eqref{notcur} obtained by multiplying by $z^n$ 
and summing over $n\in \Z$.
The second equation results from the change of variables $u_i\mapsto q^{1/2} u_i$ for all $i$ and
$z\mapsto q^{\al/2} \, z$ in the previous multiple 
constant term residue integral, while using the relation \eqref{emrela}.
\end{proof}



The result of Corollary \ref{corctMqt} may be rephrased in terms of generating currents as follows.
We consider the generating multi-current with argument $\bv=(v_1,v_2,...,v_\al)$:
\begin{equation}\label{defMal}
{\mathfrak M}_\al(\bv):=\sum_{a_1,...,a_\al\in \Z} {\mathcal M}_{a_1,...,a_\al}\, v_1^{a_1}v_2^{a_2}\cdots v_{\al}^{a_\al}=
 \frac{1}{\prod_{j=1}^\al v_j^{\al-j}}{\mathcal D}_\al \left(\frac{\det\left(\left( \delta(x_i \, v_j) \right)_{1\leq i,j\leq \al}\right)}{\prod_{1\leq i<j\leq \al}(x_i-x_j)} \right)
\end{equation}

\begin{thm}\label{Mofm}
The generating current for the generalized Macdonald operators \eqref{schurmacdo} reads:
\begin{equation}
{\mathfrak M}_\al(\bv)=\prod_{1\leq i<j\leq \al} \frac{(v_i-q v_j)}{(t -v_i v_j^{-1})(t v_i-q  v_j)}
\, \prod_{i=1}^\al {\mathfrak m}(v_i)
\end{equation}
\end{thm}
\begin{proof}
Using the identity \eqref{ctMqt}, and the expression \eqref{defschur} for the generalized Schur function, we compute:
\begin{eqnarray*}
{\mathfrak M}_\al(\bv)&=&\frac{1}{\al!}CT_\bu\left(\sum_{a_1,...,a_\al\in \Z} \det\left( u_i^{-a_j-\al+j}\right)
v_1^{a_1}v_2^{a_2}\cdots v_{\al}^{a_\al} \right. \times \\
&&\qquad \qquad \qquad \times \left. \prod_{1\leq i<j\leq \al} \frac{(u_i-q u_j)}{(t u_i^{-1}-u_j^{-1})(t u_i-q u_j)}
\, \prod_{i=1}^\al {\mathfrak m}(u_i)\right)\\
&=&\frac{1}{\al!}CT_\bu\left(\det\left( \delta(v_j/u_i)\, v_j^{j-\al}\right)
 \prod_{1\leq i<j\leq \al} \frac{(u_i-q u_j)}{(t u_i^{-1}-u_j^{-1})(t u_i-q  u_j)}
\, \prod_{i=1}^\al {\mathfrak m}(u_i)\right)\\
&=&\frac{1}{\al!}CT_\bu\left(\det\left( \delta(v_j/u_i)\right)
 \prod_{1\leq i<j\leq \al} \frac{(u_i-q u_j)}{(t -u_i u_j^{-1})(t u_i-q u_j)}
\, \prod_{i=1}^\al {\mathfrak m}(u_i)\right)\\
&=&\prod_{1\leq i<j\leq \al} \frac{(v_i-q v_j)}{(t -v_i v_j^{-1})(t v_i-q v_j)}
\, \prod_{i=1}^\al {\mathfrak m}(v_i)
\end{eqnarray*}
where in the last step we have used the skew-symmetry of both the determinant and the quantity next to it, 
to see that each of the $\al!$ terms in the expansion of the determinant contributes the same as the diagonal 
term $\prod \delta(v_i/u_i) u_i^{i-\al}$.
\end{proof}


\begin{cor}\label{corCT}
We have the following alternative expression for the generalized Macdonald operators  \eqref{schurmacdo}:
\begin{equation}\label{ctsimp}
{\mathcal M}_{a_1,...,a_\al}=CT_\bu\left(\prod_{i=1}^\al u_i^{-a_i}
 \prod_{1\leq i<j\leq \al} \frac{(u_i-q u_j)}{(t -u_i u_j^{-1})(t u_i-q u_j)}
\, \prod_{i=1}^\al {\mathfrak m}(u_i)\right)
\end{equation}
\end{cor}
\begin{proof}
The constant term \eqref{ctsimp} picks up the coefficient of 
$u_1^{a_1}u_2^{a_2}\cdots u_\al^{a_\al}$ in ${\mathfrak M}_\al(\bu)$.
\end{proof}

We also have the corresponding current version, when $a_1=a_2=\cdots =a_\al=n$:
\begin{eqnarray}
\qquad\quad  {\mathfrak m}_{\al}(z)&=&CT_\bu\left( \delta(u_1 u_2 \cdots u_\al/z) 
\prod_{1\leq i<j\leq \al} \frac{(u_i-q u_j)}{(t -u_i u_j^{-1})(t u_i-q u_j)}\, \prod_{i=1}^\al {\mathfrak m}(u_i)\right)\label{mbo}\\
\qquad\quad {\mathfrak e}_{\al}(z)&=&CT_\bu\left( \delta(u_1 u_2 \cdots u_\al/z) 
\prod_{1\leq i<j\leq \al} \frac{(u_i-q u_j)}{(t -u_i u_j^{-1})(u_i-q/t u_j)}\, \prod_{i=1}^\al {\mathfrak e}(u_i)\right)\label{ebo}
\end{eqnarray}

\subsubsection{Polynomiality and the $(q,t)$-determinant}

In this section, we state the following:

\begin{conj}\label{polyconj}
The generalized Macdonald operators ${\mathcal M}_{a_1,...,a_\al}$ may be expressed as {\it polynomials} of 
finitely many ${\mathcal M}_{p}$'s. These polynomials are $t$-deformation of the quantum determinant expression \eqref{ctsimplim} below.
\end{conj}

Note that if they exist, such polynomials are not necessarily unique, as they can be modified 
by use of the exchange relations \eqref{cubicM}.


We first give the proof of the conjecture in the case $\al=2$ for arbitrary ${\mathcal M}_{a,b}$, $a,b\in \Z$,
by deriving an explicit polynomial expression (see Theorem \ref{polynomialitythm} below), and the sketch 
of the proof in the case $\al=3$ (Theorem \ref{polthree}). 
Further evidence will be derived in Section \ref{EHAsec}, where the connection to Elliptic Hall 
algebra leads naturally to a polynomial expression for 
${\mathcal M}_{\al;n}={\mathcal M}_{n,n,...,n}$ for all $\al,n$, as a function of solely 
${\mathcal M}_n,{\mathcal M}_{n\pm 1}$.

\begin{thm}\label{polynomialitythm}
For all $a,b\in \Z$, the operator ${\mathcal M}_{a,b}$ can be expressed as an explicit quadratic polynomial of the 
${\mathcal M}_{n}$'s, with coefficients in $\C(q,t)$. More precisely, we have:
\begin{equation}\label{mtwopol}
{\mathcal M}_{a,b}=\frac{q(q+t^2)\nu_{a,b}-t(1+q) \nu_{a+1,b-1}+(q+t^2)\nu_{b-1,a+1}
-q t(1+q) \nu_{b-2,a+2}}{(q-1)(q^2-t^2)(1-t^2)}
\end{equation}
where $\nu_{a,b}$ stands for the following ``quantum determinant":
\begin{equation}\label{qdetwo}
\nu_{a,b}=\left\vert
\begin{matrix}
{\mathcal M}_a & {\mathcal M}_{b-1}\\
{\mathcal M}_{a+1} & {\mathcal M}_{b}
\end{matrix} \right\vert_q:={\mathcal M}_a\, {\mathcal M}_b- q\, {\mathcal M}_{a+1}\, {\mathcal M}_{b-1}
\end{equation}
\end{thm}
\begin{proof}
We start from the current relation of Theorem \ref{Mofm}:
\begin{equation}\label{mtwo}
{\mathfrak M}_2(v_1,v_2)=\frac{(v_1-q v_2)}{(t -v_1 v_2^{-1})(t v_1-q  v_2)}
\,{\mathfrak m}(v_1)\,{\mathfrak m}(v_2)
\end{equation}
We note that the exchange relation \eqref{exchange} can be rewritten as
$$\frac{(v_1-q v_2)}{(t v_2 -v_1)(t v_1-q  v_2)}
\,{\mathfrak m}(v_1)\,{\mathfrak m}(v_2)+\frac{(v_2-q v_1)}{(t v_1 -v_2)(t v_2-q  v_1)}
\,{\mathfrak m}(v_2)\,{\mathfrak m}(v_1)=0$$
As a consequence, the renormalized double current ${\mathfrak N}_2(v_1,v_2):={\mathfrak M}_2(v_1,v_2)/v_2$
is skew-symmetric, and we may rewrite \eqref{mtwo} as:
\begin{equation}\label{none}
\delta_{1,2}\,  {\mathfrak N}_2(v_1,v_2)
=\frac{v_1-q v_2}{v_1v_2} \,{\mathfrak m}(v_1)\,{\mathfrak m}(v_2)=:\mu_2(v_1,v_2) 
\end{equation}
where we use the notation
$$\delta_{i,j}:= \Big(t-\frac{v_i}{v_j}\Big)\Big(t-q \frac{v_j}{v_i}\Big)$$
Using the skew-symmetry of ${\mathfrak N}_2$, let us apply the transposition $(12)$ that interchanges 
$v_1\leftrightarrow v_2$, with the result:
\begin{equation}\label{ntwo}
\delta_{2,1}\,  {\mathfrak N}_2(v_1,v_2)=-\mu_2(v_2,v_1) 
\end{equation}
Next, we eliminate the prefactors by using the ``inversion" relation:
\begin{equation}\label{inverela}
\eta_{1,2}\delta_{1,2}-\theta_{1,2}\delta_{2,1}=1,\quad \eta_{i,j}:=\frac{q(q+t^2)-t(1+q)\frac{v_j}{v_i}}{(q-1)(1-t^2)(q^2-t^2)},
\quad \theta_{i,j}:=\frac{(q+t^2)-qt(1+q)\frac{v_j}{v_i}}{(q-1)(1-t^2)(q^2-t^2)}
\end{equation}
which is easily derived by decomposing $1/(\delta_{1,2}\delta_{2,1})$ into simple fractions. 
With the above choice, we conclude that:
$${\mathfrak M}_2(v_1,v_2)= v_2 {\mathfrak N}_2(v_1,v_2)
=v_2(\eta_{1,2}\, \mu_2(v_1,v_2)+\theta_{1,2}\, \mu_2(v_2,v_1))$$
Defining $\nu_2(v_1,v_2):=v_2 \mu_2(v_1,v_2)=\left(1-q\frac{v_2}{v_1}\right) \,{\mathfrak m}(v_1)\,{\mathfrak m}(v_2)$,
we finally get:
\begin{equation}\label{soltwo}
{\mathfrak M}_2(v_1,v_2)=\eta_{1,2}\, \nu_2(v_1,v_2)+\frac{v_2}{v_1}\,\theta_{1,2}\, \nu_2(v_2,v_1)
\end{equation}
Introducing the mode expansion: $\nu_2(v_1,v_2)=\sum_{a,b\in \Z} \nu_{a,b} v_1^a v_2^b$, with
$\nu_{a,b}$ as in \eqref{qdetwo}, the formula \eqref{mtwopol} follows from the mode expansion of \eqref{soltwo}.
\end{proof}

Note that the expression \eqref{mtwopol} expresses ${\mathcal M}_{a,b}$ in terms of more variables that 
just ${\mathcal M}_a,{\mathcal M}_{a+1},{\mathcal M}_{b-1},{\mathcal M}_{b}$. However, if we consider the limit
when $t\to \infty$ of this expression, after defining $M_{a,b}=\lim_{t\to \infty} t^{-2(N-2)}{\mathcal M}_{a,b}$,
$M_a=\lim_{t\to \infty} t^{-(N-1)} {\mathcal M}_a$, and $n_{a,b}=\lim_{t\to \infty} t^{-2(N-1)} \nu_{a,b}$, we obtain:
$$M_{a,b}=\frac{q \ n_{a,b}+n_{b-1,a+1}}{q-1}=n_{a,b}=M_aM_b-q M_{a+1}M_{b-1}$$
where we have used the $t\to \infty$ exchange relation. 
Indeed, for finite $t$ the expression \eqref{mtwopol} is not unique: 
it is unique only up to the exchange relation \eqref{cubicM}
for the ${\mathcal M}_n$'s. Here is a simple example.
Revisiting the proof of the Theorem, we note that there is another 
inverting pair $(\eta_{1,2}',\theta_{1,2}')=(-\theta_{2,1},-\eta_{2,1})$ obtained by acting with the transposition (12) 
on the inversion relation \eqref{inverela}, namely we also have:
$$\eta_{1,2}'\, \delta_{1,2}-\theta_{1,2}' \, \delta_{2,1}=1$$
This choice leads to alternative expressions:
\begin{eqnarray}
{\mathfrak M}_2(v_1,v_2)&=& v_2 {\mathfrak N}_2(v_1,v_2)=-v_2\Big(\theta_{2,1}\, \mu_2(v_1,v_2)+\eta_{2,1}\, \mu_2(v_2,v_1)\Big)\nonumber \\
&=& -\theta_{2,1} \, \nu_2(v_1,v_2)-\frac{v_2}{v_1}\,\eta_{2,1}\, \nu_2(v_2,v_1)\nonumber \\
{\mathcal M}_{a,b}&=&
\frac{q t(1+q)\nu_{a-1,b+1}-(q+t^2)\nu_{a,b}+t(1+q) \nu_{b,a}-q(q+t^2)\nu_{b-1,a+1}}{(q-1)(q^2-t^2)(1-t^2)}
\label{alterM}
\end{eqnarray}
The difference between the two expressions \eqref{mtwopol} and \eqref{alterM} is proportional to:
\begin{equation}\label{difference}
(q+t^2)(\nu_{a,b}+\nu_{b-1,a+1})-t(\nu_{a+1,b-1}+\nu_{b,a})-qt(\nu_{a-1,b+1}+\nu_{b-2,a+2})
\end{equation}
Rewriting the exchange relation \eqref{cubicM} as:
\begin{equation}\label{rewexch}\varphi_{a,b}:=qt \nu_{a-3,b}-(q+t^2)\nu_{a-2,b-1}+t \nu_{a-1,b-2}=-\varphi_{b,a}
\end{equation}
we see that \eqref{difference} is nothing but 
$-\varphi_{a+2,b+1}-\varphi_{b+1,a+2}=0$, as a direct consequence of the exchange relation.
This gives a simple example of equivalence of two polynomial expressions for ${\mathcal M}_{a,b}$
modulo the exchange relations of the algebra.

\begin{thm}\label{polthree}
The conjecture \ref{polyconj} holds for $\al=3$.
\end{thm}
\begin{proof}
Sketch of the proof. Use simple fraction decomposition of the quantity
$1/(\delta_{1,2}\delta_{2,1}\delta_{1,3}\delta_{3,1}\delta_{2,3}\delta_{3,2})$ to obtain a relation of the form:
$$\sum_{\sigma\in S_3} {\rm sgn}(\sigma)\,A_\sigma(v_1,v_2,v_3) \, 
\delta_{\sigma(1),\sigma(2)}\delta_{\sigma(1),\sigma(3)}\delta_{\sigma(2),\sigma(3)}=1$$
with explicit Laurent polynomials $A_\sigma(v_1,...,v_\al)$. This allows to express the skew-symmetric 
current ${\mathfrak N}_\al={\mathfrak M}_\al/(v_2 v_3^2)$ as a sum over the symmetric group $S_3$ 
of Laurent polynomials of the $v$'s times permuted products of the fundamental currents 
${\mathfrak m}(v_i)$'s. The polynomiality property of the coefficients follows.
\end{proof}


More generally, one could try to generalize the above argument. Defining the skew-symmetric function
${\mathfrak N}_\al={\mathfrak M}_\al/(v_2 v_3^2\cdots v_\al^{\al-1})$, we wish to invert the relation
$$\left(\prod_{1\leq i<j\leq \al} \delta_{i,j} \right)\, {\mathfrak N}_\al(v_1,...,v_\al)
=\prod_{1\leq i<j\leq \al}\frac{v_i-q v_j}{v_i v _j}\,
{\mathfrak m}(v_1)\cdots {\mathfrak m}(v_\al)
=:\mu_\al(v_1,...,v_\al)$$
Acting with the permutation group of the $v$'s, we have accordingly for all $\sigma \in S_\al$:
$$\prod_{1\leq i<j\leq \al} \delta_{\sigma(i),\sigma(j)} \, {\mathfrak N}_\al(v_1,...,v_\al)
={\rm sgn}(\sigma)\, \mu_\al(v_{\sigma(1)},...,v_{\sigma(\al)})$$
Inverting the system could be done by looking for Laurent polynomials $A_\sigma(v_1,...,v_\al)$ such that
$$\sum_{\sigma\in S_\al} {\rm sgn}(\sigma)\,A_\sigma(v_1,...,v_\al) \,\prod_{1\leq i<j\leq \al} \delta_{\sigma(i),\sigma(j)}=1$$
If such $A_\sigma$'s existed, then we could write
$${\mathfrak M}_\al=v_2 v_3^2\cdots v_\al^{\al-1}\sum_{\sigma\in S_\al} 
A_\sigma(v_1,...,v_\al)  \, \mu_\al(v_{\sigma(1)},...,v_{\sigma(\al)})$$
and polynomiality would follow.





%\begin{example} \label{exampletwo}
%Rewriting the result of Theorem \ref{Mofm} for $\al=2$ as:
%$$(t -v_1 v_2^{-1})(t -q  v_2v_1^{-1})\, {\mathfrak M}_2(v_1,v_2)=(1-q v_2v_1^{-1})\, {\mathfrak m}(v_1){\mathfrak m}(v_2)\ ,
%$$
%and picking the coefficient of $v_1^nv_2^p$, we get the equivalent recursion relation:
%\begin{equation}
%\label{recualphatwo}
%-qt {\mathcal M}_{n+1,p-1}+(q+t^2) {\mathcal M}_{n,p}-t  {\mathcal M}_{n-1,p+1}={\mathcal M}_{n}\,{\mathcal M}_{p}
%-q \, {\mathcal M}_{n+1}\,{\mathcal M}_{p-1}
%\end{equation}
%Note also the skew-symmetry formula ${\mathcal M}_{a,b}=-{\mathcal M}_{b-1,a+1}$ as a direct consequence of the
%skew-symmetry of the determinantal definition \eqref{defschur} of the generalized Schur function $s_{a,b}=-s_{b-1,a+1}$.
%Together with the recursion relation \eqref{recualphatwo}, it allows to express inductively all the operators ${\mathcal M}_{i,j}$ as quadratic polynomials of the ${\mathcal M}_{n}$'s 
%(see Theorems \ref{oddMthm} and \ref{evenMthm} and Lemma \ref{evenMlemma}
%below for a simpler proof using shuffle relations). 
%\end{example}

\subsection{Plethystic formulation}

Using the plethystic formulas of Section \ref{secboso}, we derive a plethystic formula for the higher currents ${\mathfrak e}_{\al}$.

\begin{thm}\label{bosoethm}
The current ${\mathfrak e}_{\al}(z)$ acts on functions $F[X]$ as follows:
\begin{eqnarray*}
{\mathfrak e}_{\al}(z)\cdot F[X]&=& \left(\frac{q^{1/2}t^{-1/2}}{(1-q)(1-t^{-1})}\right)^\al
{\rm CT}_\bu\left(\prod_{1\leq i<j\leq \al} \frac{(u_i-u_j)}{(u_i-t u_j)(u_i-t^{-1}u_j)} \right. \\
&&\quad  \times \prod_{i=1}^\al \left\{ t^{\frac{N}{2}} \frac{C(q^{1/2}t^{-1}u_i)}{C(q^{1/2} u_i)} 
-t^{-\frac{N}{2}} \frac{{\tilde C}(q^{-1/2}t u_i^{-1})}{{\tilde C}(q^{-1/2} u_i^{-1})} \right\} \\
&&\qquad \qquad \qquad \delta(u_1u_2 \cdots u_\al/z)\cdot \left.
F\left[X+(q^{1/2}-q^{-1/2})\sum_{i=1}^\al \frac{1}{u_i} \right]\right)
\end{eqnarray*}
\end{thm}
\begin{proof}
We start from the bosonized formula for ${\mathfrak e}(z)$ of Theorem \ref{pletbo}, and substitute it into
\eqref{ebo}. We need to compute the action of the plethysm $[X+\frac{q^{1/2}-q^{-1/2}}{z}]$ on ${\mathfrak e}(w)$:
it sends respectively
\begin{eqnarray*}
\frac{C(q^{1/2}t^{-1}w)}{C(q^{1/2}w)}&\mapsto& \frac{(z-q t^{-1} w)(z-w)}{(z-t^{-1}w)(z-q w)} 
\,\frac{C(q^{1/2}t^{-1}w)}{C(q^{1/2}w)}\\
\frac{{\tilde C}(q^{-1/2}t w^{-1})}{{\tilde C}(q^{-1/2}w^{-1})}&\mapsto& \frac{(w-q^{-1} t z)(w-z)}{(w-t z)(w-q^{-1} z)} 
\, \frac{{\tilde C}(q^{-1/2}t w^{-1})}{{\tilde C}(q^{-1/2}w^{-1})}
\end{eqnarray*}
hence both terms are mapped identically, and the theorem follows by repeated applications until all $u_i$'s are
exhausted.
\end{proof}

Similarly, we have the following bosonization of the multi-current ${\mathfrak M}_\al(\bv)$ of \eqref{defMal}.
\begin{thm}
The multi-current ${\mathfrak M}_\al(\bv)$  for the generalized Macdonald operators acts on functions $F[X]$ as follows.
\begin{eqnarray*}
{\mathfrak M}_\al(\bv)\cdot F[X]&=&\left(\frac{t^{\frac{N}{2}}}{t-1}\right)^\al
\prod_{1\leq i<j\leq \al} \frac{(v_i-v_j)}{(t -v_i v_j^{-1})(t v_i-v_j)}\times \\
&&\quad \times\prod_{i=1}^\al \left\{t^{\frac{N}{2}} \frac{C(t^{-1}v_i)}{C(v_i)} 
-t^{-\frac{N}{2}} \frac{{\tilde C}(t v_i^{-1})}{{\tilde C}(v_i^{-1})} \right\} \cdot  F\left[X+(q-1)\sum_{i=1}^\al \frac{1}{v_i} \right]
\end{eqnarray*}
\end{thm}
\begin{proof}
Starting from \eqref{defMal}, we substitute ${\mathfrak m}(v_i)= q^{-1/2}(1-q)t^{\frac{N-1}{2}}{\mathfrak e}(q^{-1/2}v_i)$,
and then use the proof of Theorem \ref{bosoethm} to rearrange the prefactors of the plethysms.
\end{proof}

This immediately implies the following corollary for the limit of infinite alphabet $N\to \infty$:
\begin{cor}
Defining the limiting multi-current ${\mathfrak M}_\al^{\infty}(\bv):=\lim_{N\to \infty} t^{-N \al}\,{\mathfrak M}_\al(\bv)$, we 
have the following plethystic formula:
$$
{\mathfrak M}_\al^{\infty}(\bv)F[X]=\frac{1}{(t-1)^\al}\prod_{1\leq i<j\leq \al} \frac{(v_i-v_j)}{(t -v_i v_j^{-1})(t v_i-v_j)}\prod_{i=1}^\al \frac{C_\infty(t^{-1}v_i)}{C_\infty(v_i)}\cdot  F\left[X+(q-1)\sum_{i=1}^\al \frac{1}{v_i} \right]
$$
\end{cor}

\subsection{Shuffle product}\label{shuprosec}

The shuffle product is a non-commutative product 
$*:{\mathcal F}_\al \times {\mathcal F}_\beta \to {\mathcal F}_{\al+\beta}$, sending $(P,P')\mapsto Q=P*P'$. 
It allows to express the product 
of any two difference operators ${\mathcal D}_\al(P)$ and ${\mathcal D}_\beta(P')$
for arbitrary rational functions $P\in {\mathcal F}_\al $ and $P'\in {\mathcal F}_\beta$, 
as a difference operator of the form
${\mathcal D}_{\al+\beta}(Q)$, for some $Q\in {\mathcal F}_{\al+\beta}$.
In the following sections, we first define the shuffle product, and then present various applications, among which 
alternative proofs of the exchange relation \eqref{exchange}, and of Theorems \ref{Mofm},  \ref{thserrre}
and \ref{polynomialitythm}.

\subsubsection{Definition and main result}

Let us introduce the quantity:
$$\zeta(x):=\frac{1-t x}{1-x}\frac{t-q x}{1-q x}$$

\begin{defn}\label{shufdef}
The shuffle product $P*P'\in {\mathcal F}_{\al+\beta}$ of $P\in {\mathcal F}_\al $ and $P'\in {\mathcal F}_\beta$ 
is defined as the symmetrized expression:
\begin{equation}\label{defshuf}
P*P'(x_1,...,x_{\al+\beta}):=\frac{1}{\al!\, \beta!} {\rm Sym}\left(  P(x_1,...,x_\al)P'(x_{\al+1},...,x_{\al+\beta}) 
\prod_{1\leq i\leq \al<j\leq \al+\beta} 
\zeta(x_i/x_j) \right)
\end{equation}
where the symmetrization $\rm Sym$ is over $x_1,x_2,...,x_{\al+\beta}$.
\end{defn}

This product is in general non-commutative by construction, but associative.

\begin{thm}\label{shufmac}
For any rational functions $P\in {\mathcal F}_\al $ and $P'\in {\mathcal F}_\beta$, we have the relations
\begin{equation}\label{prodshuf}
{\mathcal D}_\al(P)\, {\mathcal D}_\beta(P')={\mathcal D}_{\al+\beta}(P * P'), \qquad 
{\mathcal M}_\al(P)\, {\mathcal M}_\beta(P')={\mathcal M}_{\al+\beta}(P * P')
\end{equation}
with the shuffle product $P*P'\in {\mathcal F}_{\al+\beta}$ defined in \eqref{defshuf}.
\end{thm}
\begin{proof}
We use Theorem \ref{mainthm} and compute, for $\bu=(u_1,u_2,...,u_{\al+\beta})$:
\begin{eqnarray*}
{\mathcal M}_\al(P)\, {\mathcal M}_\beta(P')&=&CT_{\bu}\left( \frac{P(u_1^{-1},...,u_\al^{-1})P'(u_{\al+1}^{-1},...,u_{\al+\beta}^{-1})}{\al!\, \beta!}\times \right. \\
&&\qquad \qquad \qquad \qquad \left. \times \prod_{i<j\,  \in [1,\al]\,
{\rm or} \, [\al+1,\al+\beta] }\zeta(u_j/u_i)^{-1} \prod_{k=1}^{\al+\beta} {\mathfrak m}(u_k)\right) \\
&=&CT_{\bu}\left(  \frac{P(u_1^{-1},...,u_\al^{-1})P'(u_{\al+1}^{-1},...,u_{\al+\beta}^{-1})}{\al!\, \beta!}\times \right. \\
&&\qquad \qquad \qquad \qquad \left. \times\prod_{1\leq i\leq \al<j\leq \al+\beta} \zeta(u_j/u_i)
\prod_{i<j \,\in [1,\al+\beta]}\zeta(u_j/u_i)^{-1}
\prod_{k=1}^{\al+\beta} {\mathfrak m}(u_k)\right)\\
&=& \frac{1}{(\al+\beta)!}CT_{\bu}\left( P*P'(u_1^{-1},...,u_{\al+\beta}^{-1})
\prod_{i<j \,\in [1,\al+\beta]}\zeta(u_j/u_i)^{-1}
\prod_{k=1}^{\al+\beta} {\mathfrak m}(u_k)\right)\\
&=&{\mathcal M}_{\al+\beta}(P*P')
\end{eqnarray*}
where we have used the symmetry of the last factor in the second line, and the fact that the 
constant term is preserved under symmetrization, namely 
$CT_{u_1,...,u_m}\big({\rm Sym}(f(u_1,...,u_{m}))\big)/m! =CT_{u_1,...,u_m}\big(f(u_1,...,u_{m})\big)$.
The Theorem follows.
\end{proof}

In the next subsections, we explore applications of Theorem \ref{shufmac}.

\subsubsection{Application I: Macdonald current relations}\label{appsecone}
Note that
$${\mathfrak m}(v)={\mathcal D}_1(\delta(v x_1))$$
By iterated use of Theorem \ref{shufmac},
we may express any product ${\mathfrak m}(v_1){\mathfrak m}(v_2)\cdots {\mathfrak m}(v_\al)$ as:
$${\mathfrak m}(v_1){\mathfrak m}(v_2)\cdots {\mathfrak m}(v_\al)={\mathcal D}_\al\Big(
\delta(v_1 x_1)*\delta(v_2 x_1)* \cdots *\delta(v_\al x_1)\Big)$$
where we used associativity of the shuffle product to drop parentheses.
In particular, the exchange relation \eqref{exchange} boils down to the following shuffle identity on ${\mathcal F}_2$.

\begin{thm}
We have the relation:
$$g(u,v)\, \delta(u x_1)*\delta(v x_1)+g(v,u)\, \delta(v x_1)*\delta(u x_1)=0$$
\end{thm}
\begin{proof}
Use Definition~\ref{shufdef} to compute:
\begin{eqnarray*}
g(u,v)\, \delta(u x_1)*\delta(v x_1)&=&
g(u,v){\rm Sym}\left( \delta(u x_1)\,\delta(v x_2)\frac{(x_2-tx_1)(t x_2-q x_1)}{(x_2-x_1)(x_2-q x_1)} \right)\\
&=& g(u,v)\frac{(u-tv)(t u-q v)}{(u-v)(u-q v)}{\rm Sym}\left( \delta(u x_1)\,\delta(v x_2) \right)\\
&=& \frac{(v-t u)(u-tv)(t v-q u)(t u-q v)}{q t (u-v)}\left( \delta(u x_1)\,\delta(v x_2)+\delta(v x_1)\,\delta(u x_2)\right)
\end{eqnarray*}
which is manifestly skew-symmetric in $(u,v)$.
\end{proof}

More generally, Theorem \ref{Mofm} translates into the following shuffle identity.
\begin{thm}\label{detshuf}
We have the relation:
$$\frac{\det\left(\left(\delta(x_i v_j)\right)_{1\leq i,j\leq \al}\right)}{\prod_{1\leq i<j\leq \al} (x_i-x_j)}
=\prod_{i=1}^\al v_j^{\al-1}\,
\prod_{1\leq i<j\leq \al} \frac{v_i-q v_j}{(t v_j -v_i)(t v_i-q v_j)}\, \delta(v_1x_1)*\delta(v_2 x_1)*\cdots *\delta(v_\al x_1)$$
\end{thm}
\begin{proof}
We compute:
\begin{eqnarray*}
&&\delta(v_1x_1)*\delta(v_2 x_1)*\cdots *\delta(v_\al x_1)={\rm Sym}\left(
\delta(v_1x_1)\delta(v_2 x_2)\cdots\delta(v_\al x_\al)\prod_{1\leq i<j\leq \al}\frac{(x_j-tx_i)(t x_j-q x_i)}{(x_j-x_i)(x_j-q x_i)}\right)\\
&&\qquad\qquad =\frac{1}{(\prod_{i=1}^\al v_j)^{\al-1}} \prod_{1\leq i<j\leq \al}\frac{(tv_j-v_i)(t v_i-q v_j)}{(v_i-q v_j)} {\rm Sym}\left(
\frac{\delta(v_1x_1)\delta(v_2 x_2)\cdots\delta(v_\al x_\al)}{\prod_{1\leq i<j\leq \al} (x_i-x_j)}\right)
\end{eqnarray*}
and the result follows, as the symmetrization produces the desired determinant.
\end{proof}

Finally let us revisit Theorem \ref{thserrre} for ${\mathfrak m}$, namely the identity 
$${\rm Sym}_{v_1,v_2,v_3}\left( 
\frac{v_2}{v_3} {\Big[}{\mathfrak m}(v_1),{[}{\mathfrak m}(v_2),{\mathfrak m}(v_3){]}{\Big]}
\right)=0$$
The corresponding shuffle identity is the following.
\begin{thm}\label{shufserre}
We have the identity:
\begin{eqnarray*}
&&{\rm Sym}_{v_1,v_2,v_3}\left\{ 
\frac{v_2}{v_3}\Big(\delta(v_1x_1)*\big(\delta(v_2 x_1)*\delta(v_3x_1)-\delta(v_3 x_1)*\delta(v_2x_1)\big)\right. \\
&&\qquad\qquad\qquad \left. -
\big(\delta(v_2 x_1)*\delta(v_3x_1)-\delta(v_3 x_1)*\delta(v_2x_1)\big)*\delta(v_1x_1)\Big)\right\}=0
\end{eqnarray*}
\end{thm}
\begin{proof}
Using the proof of Theorem \ref{detshuf} for $\al=3$, we compute:
$$\delta(v_1x_1)*\delta(v_2 x_1)*\delta(v_3 x_1)=
\prod_{1\leq i<j\leq 3}\frac{(tv_j-v_i)(t v_i-q v_j)}{(v_i-q v_j)(v_j-v_i)} {\rm Sym}_{x_1,x_2,x_3}\left(
\delta(v_1x_1)\delta(v_2 x_2)\delta(v_3 x_3)\right)$$
Noting that the symmetrized term is also symmetric in $(v_1,v_2,v_3)$, the statement of the Theorem
boils down to:
$${\rm Sym}_{v_1,v_2,v_3}\left(\frac{v_2}{v_3}\big(1-(23)-(123)+(13)\big)
\prod_{1\leq i<j\leq 3}\frac{(tv_j-v_i)(t v_i-q v_j)}{(v_i-q v_j)(v_j-v_i)}\right)=0$$
where the permutations on the left act by permuting the $v$'s. This latter identity is easily checked.
\end{proof}




\subsubsection{Application II: commuting difference operators}\label{appsectwo}

The Macdonald operators form a commuting family whose common eigenfunctions 
are the celebrated Macdonald polynomials \cite{macdo}.
Their commutativity boils down via the relations \eqref{prodshuf} to the identity
$$1_\al*1_\beta=1_\beta*1_\al$$
where we denote by $1_\al$ the constant function $1\in {\mathcal F}_\al$.
This identity amounts to the following:
\begin{lemma}\label{comac}
The following identity in ${\mathcal F}_{\al+\beta}$ holds.
\begin{equation}
{\rm Sym}\left(\prod_{1\leq i\leq \al<j\leq \al+\beta} \zeta(x_i/x_j)\right)=
{\rm Sym}\left(\prod_{1\leq i'\leq \beta<j'\leq \al+\beta} \zeta(x_{i'}/x_{j'})\right)
\end{equation}
\end{lemma}
\begin{proof}
The symmetrized function is by definition invariant under any permutation of the $x$'s. 
Consider the permutation $\sigma\in S_{\al+\beta}$ such that $\sigma(i)=\al+\beta+1-i$
for all $i\in \{1,2,...,\al+\beta\}$.
The permutation $\sigma$ clearly maps the range $1\leq i\leq \al<j\leq \al+\beta$ to
$1\leq \sigma(j)\leq \beta< \sigma(i)\leq \al+\beta$, and the identity follows by noting that
the result is invariant under $x_i\mapsto x_i^{-1}$ for all $i$.
\end{proof}

Likewise, using the definition ${\mathcal M}_{\al;n}={\mathcal D}_\al((x_1\cdots x_\al)^n)$,
the commutation of the operators ${\mathcal M}_{\al;n}$ for any fixed $n$ and $\al=1,2,...,N$
is a consequence of the commutation:
$$(x_1\cdots x_\al)^n * (x_1\cdots x_\beta)^n =(x_1\cdots x_\beta)^n * (x_1\cdots x_\al)^n $$
The latter is a trivial consequence of Lemma  \ref{comac}, as the product
$(x_1\cdots x_\al)^n (x_{\al+1}\cdots x_{\al+\beta})^n$ is symmetric in the $x$'s, and therefore 
drops out of the symmetrization.

More generally, we could try to find commuting families of similar difference operators, by looking
for families of shuffle-commuting rational functions. In fact, any factorized choice for
$P(x_1,...,x_\al)=f(x_1)f(x_2)\cdots f(x_\al)$ and similarly for $P'(x_1,...,x_\beta)=f(x_1)f(x_2)\cdots f(x_\beta)$
will trivially lead to commuting operators from $P*P'=P'*P$,
as the product $f(x_1)\cdots f(x_{\al+\beta})$ drops out of the 
symmetrization and leaves us with the identity of Lemma \ref{comac}. 
The previous case corresponds to $f(x)=x^n$. 

Consider now the function $f(x)=1+a x$ for some fixed arbitrary coefficient $a$. Note that
$$\prod_{i=1}^\al f(x_i)=\sum_{k=0}^\al a^k s_{1^k,0^{\al-k}}(x_1,...,x_\al)$$
where the notation $1^k0^{\al-k}$ stands for $k$ 1's followed by $\al-k$ 0's. We deduce that
$${\mathcal D}_\al(f(x_1)\cdots f(x_\al))=\sum_{k=0}^\al a^k {\mathcal M}_{1^k,0^{\al-k}}$$
Expressing the commutation relation 
$[{\mathcal D}_\al(f(x_1)\cdots f(x_\al)),{\mathcal D}_\beta(f(x_1)\cdots f(x_\beta))]=0$
and identifying the coefficient of $a^k$ yields the following relations between the ${\mathcal M}$'s:
$$\sum_{\ell={\rm Max}(0,k-\beta)}^{{\rm Min}(k,\al)} 
[{\mathcal M}_{1^\ell,0^{\al-\ell}},{\mathcal M}_{1^{k-\ell},0^{\beta-k+\ell}}]=0\qquad (0\leq k\leq \al+\beta)$$
For $\al=1$, $\beta=2$, this reduces for instance to:
$$[{\mathcal M}_0,{\mathcal M}_{0,0}]=0, 
\ \ [{\mathcal M}_0,{\mathcal M}_{1,0}]+[{\mathcal M}_1,{\mathcal M}_{0,0}]=0, \ \ 
[{\mathcal M}_0,{\mathcal M}_{1,1}]+[{\mathcal M}_1,{\mathcal M}_{1,0}]=0, 
\ \  [{\mathcal M}_1,{\mathcal M}_{1,1}]=0\ .
$$

\subsubsection{Application III: proving relations between the difference operators}\label{appsecthree}

Another application of Theorem \ref{shufmac} consists in using shuffle identities to prove identities between 
difference operators, the former being much simpler than the latter. Let us illustrate this on the following
examples, which allow to express any operator ${\mathcal M}_{n,p}$ as a quadratic polynomial 
of the ${\mathcal M}_{n}$'s (an alternative expression to that of Theorem \ref{polynomialitythm}). 
We denote by $[x,y]_q=xy-q yx$ the $q$-commutator of $x,y$.

We shall proceed in several steps. First we note that, due to the skew-symmetry of the determinantal 
definition of the generalized Schur function \eqref{defschur}, we have $s_{n,p}=-s_{p-1,n+1}$, and therefore
${\mathcal M}_{n,p}=-{\mathcal M}_{p-1,n+1}$. This allows to restrict ourselves to $n\geq p$, as
${\mathcal M}_{n,n+1}=0$. In the two following Theorems \ref{oddMthm} and \ref{evenMthm}, we find explicit 
expressions for ${\mathcal M}_{n+k,n}$,
respectively for $k>0$ odd and even. Finally, we express ${\mathcal M}_{n,n}$ in Lemma \ref{Mnn}, which allows to
complete the Theorem, with explicit quadratic expressions for all ${\mathcal M}_{n,p}$.

\begin{thm}\label{oddMthm}
We have the relation:
\begin{equation}\label{theone}
(1-q)t \,{\mathcal M}_{n+2k+1,n}= \sum_{\ell=0}^k \nu_{n+2\ell,n+2k-2\ell+1}
%{\mathcal M}_{n+2k-2\ell}\,{\mathcal M}_{n+2\ell+1}-q\,
%{\mathcal M}_{n+2k-2\ell+1}\,{\mathcal M}_{n+2\ell} \label{theone} 
=\sum_{\ell=0}^k \left[ {\mathcal M}_{n+2\ell},{\mathcal M}_{n+2k-2\ell+1}\right]_q 
\end{equation}
\end{thm}
\begin{proof}
The shuffle relation corresponding to \eqref{theone} reads, in ${\mathcal F}_2$:
\begin{eqnarray*}
\sum_{\ell=0}^k (x_1)^{n+2k-2\ell} * (x_1)^{n+2\ell+1}-q (x_1)^{n+2k-2\ell+1} * (x_1)^{n+2\ell}
&=&(1-q)t \,s_{n+2k+1,n}(x_1,x_2)\\
&=&(1-q)t\,  (x_1x_2)^n\, s_{2k+1,0}(x_1,x_2)
\end{eqnarray*}
Explicitly, we write
\begin{eqnarray*}
&&{\rm Sym}\left( \Big(\sum_{\ell=0}^k (x_1)^{n+2k-2\ell} (x_2)^{n+2\ell+1}-q (x_1)^{n+2k-2\ell+1} (x_2)^{n+2\ell}\Big)\frac{(x_2-t x_1)(t x_2-q x_1)}{(x_2-x_1)(x_2-q x_1)}\right) \\
&&\qquad =\, 
(x_1x_2)^n {\rm Sym}\left(\frac{(x_2^2)^{k+1}-(x_1^2)^{k+1}}{x_2^2-x_1^2} 
\frac{(x_2-t x_1)(t x_2-q  x_1)}{x_2-x_1}\right)\\
&&\qquad =\,\frac{(x_1x_2)^n}{(x_2-x_1)(x_2^2-x_1^2)} 
{\rm Sym}\left(((x_2)^{2k+2}-(x_1)^{2k+2})(x_2-t x_1)(t x_2-q  x_1)\right)\\
&&\qquad =(1-q)t \, (x_1x_2)^n\, s_{2k+1,0}(x_1,x_2)
\end{eqnarray*}
where we identified $s_{a,0}(x_1,x_2)=(x_1^{a+1}-x_2^{a+1})/(x_1-x_2)$.
The Theorem follows.
\end{proof}

This gives an explicit expression for ${\mathcal M}_{n+2k+1,n}$ for $k\geq 0$.

\begin{thm}\label{evenMthm}
We have the relations, for all $k\geq 0$:
\begin{eqnarray}
(1-q)t\, ({\mathcal M}_{n+4k,n}-{\mathcal M}_{n+2k,n+2k})
&=&
\sum_{\ell=0}^{k-1} \nu_{n+2\ell,n+4k-2\ell}+\nu_{n+4k-1-2\ell,n+2\ell+1}
%\sum_{\ell=0}^{k-1} \left\{ {\mathcal M}_{n+4k-1-2\ell} {\mathcal M}_{n+2\ell+1}
%-q {\mathcal M}_{n+4k-2\ell}{\mathcal M}_{n+2\ell} \right.\nonumber  \\
%&&\qquad \left. +{\mathcal M}_{n+2\ell}{\mathcal M}_{n+4k-2\ell} 
%-q {\mathcal M}_{n+2\ell+1}{\mathcal M}_{n+4k-1-2\ell}\right\}
\nonumber \\
&=&
\sum_{\ell=0}^{k-1} \left[ {\mathcal M}_{n+2\ell},{\mathcal M}_{n+4k-2\ell} \right]_q+
\left[{\mathcal M}_{n+4k-1-2\ell}, {\mathcal M}_{n+2\ell+1}\right]_q \label{zeromodfour} \\
(1-q)t\, ({\mathcal M}_{n+4k+2,n}+{\mathcal M}_{n+2k+1,n+2k+1})
&=&
\sum_{\ell=0}^{k} \nu_{n+2\ell,n+4k+2-2\ell}+\nu_{n+4k+1-2\ell,n+2\ell+1}
%\sum_{\ell=0}^{k}\left\{  {\mathcal M}_{n+4k+1-2\ell}\,{\mathcal M}_{n+2\ell+1}
%-q \, {\mathcal M}_{n+4k+2-2\ell}\,{\mathcal M}_{n+2\ell}\right. \nonumber \\
%&&\qquad \left. +{\mathcal M}_{n+2\ell}{\mathcal M}_{n+4k+2-2\ell} 
%-q \, {\mathcal M}_{n+2\ell+1}{\mathcal M}_{n+4k+1-2\ell} \right\}
\nonumber \\
&=& \sum_{\ell=0}^k \left[{\mathcal M}_{n+2\ell},{\mathcal M}_{n+4k+2-2\ell}\right]_q
+\left[{\mathcal M}_{n+4k+1-2\ell},{\mathcal M}_{n+2\ell+1}\right]_q \label{twomodfour}
\end{eqnarray}
\end{thm}
\begin{proof}
The shuffle relations corresponding to \eqref{zeromodfour} and \eqref{twomodfour} read respectively in $\cF_2$:
\begin{eqnarray*}
&&\sum_{\ell=0}^{k-1} \left\{ (x_1)^{n+4k-1-2\ell} * (x_1)^{n+2\ell+1}-q (x_1)^{n+4k-2\ell} * (x_1)^{n+2\ell}
\right.\\
&&\qquad \qquad \left. 
+(x_1)^{n+2\ell} * (x_1)^{n+4k-2\ell}-q (x_1)^{n+2\ell+1} * (x_1)^{n+4k-1-2\ell} \right\}\\
&&\qquad \qquad\qquad\qquad\qquad\qquad\qquad = (1-q)t \big(s_{n+4k,n}(x_1,x_2)-s_{n+2k,n+2k}(x_1,x_2)\big)\\
&&\sum_{\ell=0}^{k} \left\{ (x_1)^{n+4k+1-2\ell} * (x_1)^{n+2\ell+1}-q (x_1)^{n+4k+2-2\ell} * (x_1)^{n+2\ell}\right.\\
&&\qquad \qquad \left. +(x_1)^{n+2\ell} * (x_1)^{n+4k+2-2\ell}-q (x_1)^{n+2\ell+1} * (x_1)^{n+4k+1-2\ell} \right\}\\
&&\qquad \qquad\qquad\qquad\qquad\qquad\qquad =(1-q)t \big(s_{n+4k+2,n}(x_1,x_2)+s_{n+2k+1,n+2k+1}(x_1,x_2)\big)
\end{eqnarray*}
Explicitly, we write:
\begin{eqnarray*}
&&{\rm Sym}\left( \Big(\sum_{\ell=0}^{k-1} (x_1)^{n+4k-1-2\ell} (x_2)^{n+2\ell}+(x_1)^{n+2\ell}(x_2)^{n+4k-1-2\ell}\Big)
\frac{(x_2-t x_1)(t x_2-q x_1)}{x_2-x_1}\right) \\
&&=\, 
t(1-q)(x_1+x_2) (x_1x_2)^{n} \Big(\sum_{\ell=0}^{k-1} (x_1)^{4k-1-2\ell} (x_2)^{2\ell}
+(x_1)^{2\ell}(x_2)^{4k-1-2\ell}\Big)\\
&&=\, 
t(1-q)(x_1x_2)^{n} \Big(\sum_{\ell=0}^{k-1} (x_1)^{4k-2\ell} (x_2)^{2\ell}+(x_1)^{2\ell}(x_2)^{4k-2\ell} \\
&&\qquad \qquad\qquad\qquad\qquad+(x_1)^{2\ell+1}(x_2)^{4k-1-2\ell}+(x_1)^{4k-1-2\ell}(x_2)^{2\ell+1}\Big)\\
&&\qquad \qquad =t(1-q)\, (x_1x_2)^{n}  \Big(\sum_{\ell=0}^{4k}(x_1)^{4k-\ell} (x_2)^{\ell} -(x_1x_2)^{2k}\Big)\\
&&\qquad \qquad =t(1-q)\, \Big(s_{n+4k,n}(x_1,x_2)-s_{n+2k,n+2k}(x_1,x_2)\Big)
\end{eqnarray*}
Analogously, we have:
\begin{eqnarray*}
&&{\rm Sym}\left( \Big(\sum_{\ell=0}^{k} (x_1)^{n+4k+1-2\ell} (x_2)^{n+2\ell}+(x_1)^{n+2\ell}(x_2)^{n+4k+1-2\ell}\Big)
\frac{(x_2-t x_1)(t x_2-q x_1)}{x_2-x_1}\right) \\
&&=\, 
t(1-q)(x_1+x_2) (x_1x_2)^{n} \Big(\sum_{\ell=0}^{k} (x_1)^{4k+1-2\ell} (x_2)^{2\ell}
+(x_1)^{2\ell}(x_2)^{4k+1-2\ell}\Big)\\
&&=\, 
t(1-q)(x_1x_2)^{n} \Big(\sum_{\ell=0}^{k} (x_1)^{4k+2-2\ell} (x_2)^{2\ell}+(x_1)^{4k+1-2\ell} (x_2)^{2\ell+1}\\
&&\qquad \qquad\qquad\qquad\qquad+(x_1)^{2\ell+1}(x_2)^{4k+1-2\ell}+(x_1)^{2\ell}(x_2)^{4k+2-2\ell}\Big)\\
&&\qquad \qquad =t(1-q)\, (x_1x_2)^{n}  \Big(\sum_{\ell=0}^{4k+2}(x_1)^{4k+2-\ell} (x_2)^{\ell} +(x_1x_2)^{2k+1}\Big)\\
&&\qquad \qquad =t(1-q)\, \Big(s_{n+4k+2,n}(x_1,x_2)+s_{n+2k+1,n+2k+1}(x_1,x_2)\Big)
\end{eqnarray*}
This completes the proof of the shuffle relations and the Theorem follows.
\end{proof}

To get an expression for ${\mathcal M}_{n+2k,n}$ for all $k>0$, we also need to compute ${\mathcal M}_{n,n}$.


\begin{lemma}\label{evenMlemma}
We have the relations:
\begin{eqnarray}
({\mathcal M}_n)^2 -q\, {\mathcal M}_{n+1}{\mathcal M}_{n-1}
&=&(q+t+t^2)\,{\mathcal M}_{n,n}-q t\, {\mathcal M}_{n+1,n-1} \label{Mevenone}\\
({\mathcal M}_n)^2 -q^{-1} {\mathcal M}_{n-1}{\mathcal M}_{n+1}
&=&(1+t+q^{-1}t^2)\,{\mathcal M}_{n,n}-q^{-1} t \,{\mathcal M}_{n+1,n-1} \label{Meventwo}
\end{eqnarray}
Equivalently, we have:
\begin{equation}\label{Mnn}  {\mathcal M}_{n,n}
=\frac{(1-q^2)({\mathcal M}_{n})^2+q \left[{\mathcal M}_{n-1},{\mathcal M}_{n+1}\right]}{(1-q)(1+t)(q+t)}
\end{equation}
and
\begin{equation}\label{Mnnplus}  {\mathcal M}_{n+1,n-1}
=\frac{(1-q)(q+t^2)({\mathcal M}_{n})^2+(1+t)(q+t)\left[{\mathcal M}_{n-1},{\mathcal M}_{n+1}\right]_q
-t q \left[{\mathcal M}_{n-1},{\mathcal M}_{n+1}\right]}{(1-q)(1+t)(q+t)}
\end{equation}
\end{lemma}
\begin{proof}
The shuffle relations corresponding to \eqref{Mevenone} and \eqref{Meventwo} read
in ${\mathcal F}_2$:
\begin{eqnarray*} 
(x_1)^n * (x_1)^{n}-q (x_1)^{n+1} * (x_1)^{n-1}&=& (q+t+t^2)s_{n,n}(x_1,x_2)-q t \,s_{n+1,n-1}(x_1,x_2)\\
(x_1)^n * (x_1)^{n}-q (x_1)^{n-1} * (x_1)^{n+1}&=& (1+t+q^{-1}t^2)s_{n,n}(x_1,x_2)-q^{-1} t \,s_{n+1,n-1}(x_1,x_2)
\end{eqnarray*}
%with $s_{n+1,n-1}(x_1,x_2)=(x_1x_2)^{n-1}(x_1^2+x_1x_2+x_2^2)$ and $s_{n,n}=(x_1x_2)^n$.
These read respectively:
\begin{eqnarray*}&&{\rm Sym}\left( ( x_1^n x_2^{n}-q x_1^{n+1}x_{2}^{n-1})
\frac{(x_2-t x_1)(t x_2-q x_1)}{(x_2-x_1)(x_2-q x_1)} \right)\\
&&\qquad=(x_1x_2)^{n-1}{\rm Sym}\left(\frac{x_1(x_2-t x_1)(t x_2-q x_1)}{x_2-x_1}\right)\\
&&\qquad=(x_1x_2)^{n-1}((q+t+t^2)x_1x_2-q t (x_1^2+x_1 x_2+x_2^2))
\end{eqnarray*}
and
\begin{eqnarray*}&&{\rm Sym}\left( ( x_1^n x_2^{n}-q^{-1} x_1^{n-1}x_{2}^{n+1})
\frac{(x_2-t x_1)(t x_2-q x_1)}{(x_2-x_1)(x_2-q x_1)} \right)\\
&&\qquad=-q^{-1}(x_1x_2)^{n-1}{\rm Sym}\left(\frac{x_2(x_2-t x_1)(t x_2-q x_1)}{x_2-x_1}\right)\\
&&\qquad=-q^{-1}(x_1x_2)^{n-1}(t (x_1^2+x_1 x_2+x_2^2)-(q+qt+t^2)x_1x_2)
\end{eqnarray*}
and eqs.~\ref{Mevenone} and \ref{Meventwo} follow from the values $s_{n,n}=(x_1x_2)^n$ and $s_{n+1,n-1}=(x_1x_2)^{n-1}(x_1^2+x_1x_2+x_2^2)$. Finally eq.~\eqref{Mnn} follows from the combination \eqref{Mevenone}$- q\, $\eqref{Meventwo}.
\end{proof}

Substitution of the expression \eqref{Mnn} into the relations (\ref{zeromodfour}-\ref{twomodfour}) yields alternative polynomial expressions to those of Theorem \ref{polynomialitythm}.

%\begin{example}
%Theorem \ref{oddMthm} for $k=0,1$ gives:
%\begin{eqnarray}
%(1-q)t\, {\mathcal M}_{n+1,n}&=&\nu_{n,n+1}\label{nnplusone}\\
%(1-q)t\, {\mathcal M}_{n+3,n}&=&\nu{n,n+3}+\nu_{n+2,n+1}\label{nnplusthree}
%\end{eqnarray}
%with $\nu_{a,b}$ as in \eqref{qdetwo}.
%\end{example}
%Theorem \ref{evenMthm} for $k=0,1$ gives:
%\begin{eqnarray}
%(1-q)t\, {\mathcal M}_{n+2,n}&=&\nu_{n,n+2}+\nu_{n+1,n+1}-(1-q)t\, {\mathcal M}_{n+1,n+1}\label{nnplustwo}\\
%(1-q)t\, {\mathcal M}_{n+4,n}&=&\nu{n,n+4}+\nu_{n+3,n+1}-(1-q)t\, {\mathcal M}_{n+2,n+2}\label{nnplusfour}
%\end{eqnarray}
%where we have used \eqref{Mnn}.
