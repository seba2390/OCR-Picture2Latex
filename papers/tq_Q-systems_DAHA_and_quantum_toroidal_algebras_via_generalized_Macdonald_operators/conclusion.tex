\subsection{$(q,t)$-quantum determinant}

One of the goals of this paper can be understood as the construction of a $t$-deformation of the quantum determinant
of \cite{DFK15} defined in the context of the quantum $Q$-system for $A_{N-1}$. 
Concretely, we conjectured that the quantity
${\mathcal M}_{a_1,a_2,...,a_\al}$ 
is a polynomial of the ${\mathcal M}_n$'s, and proved the conjecture in the case $\al=2$ (Theorem \ref{polynomialitythm}),
and in the case when $a_1=a_2=\cdots =a_\al=n$ (Theorem \ref{polpol}, and Corollary \ref{polpolcor}). 
The latter case uses explicitly the relations in the Elliptic Hall Algebra.
We would like to interpret this polynomial as the $(q,t)$-determinant of the matrix
$\left( {\mathcal M}_{a_j+i-j}\right)_{1\leq i,j\leq \al}$.

The main difference with the usual and the quantum determinant is that, in both of the cases above, the polynomial expression for ${\mathcal M}_{a_1,a_2,...,a_\al}$ depends, in general, on more than just the
matrix elements $\{{\mathcal M}_{a_j+i-j}$, $1\leq i, j \leq \al\}$ (on which it depends in the quantum determinant case). This polynomial is unique modulo the relations of
the quantum toroidal algebra, namely the exchange relation \eqref{exchange} and the Serre relations \eqref{serre1}, 
expressed respectively as quadratic and cubic relations between the ${\mathcal M}_n$'s. 
In that respect, the expressions \eqref{theone} of Theorem \ref{oddMthm} for $k=0$,
 \eqref{Mnn} derived from Lemma \ref{evenMlemma}, and the alternative expression \eqref{alterM}
 for $a=n+2$, $b=n$, all have the property to be polynomials of just the matrix elements 
 $\{{\mathcal M}_{a_j+i-j}$, $1\leq i, j \leq \al\}$, as we may write for them:
\begin{eqnarray*}
{\mathcal M}_{n+1,n}&=&\left\vert \begin{matrix} {\mathcal M}_n & {\mathcal M}_{n} \\
{\mathcal M}_{n+1} & {\mathcal M}_{n+1}
\end{matrix}\right\vert_{q,t}=\frac{1}{(1-q)t} \left[ {\mathcal M}_n,  {\mathcal M}_{n+1}\right]_q\\
{\mathcal M}_{n,n}&=&\left\vert \begin{matrix} {\mathcal M}_n & {\mathcal M}_{n-1} \\
{\mathcal M}_{n+1} & {\mathcal M}_{n}
\end{matrix}\right\vert_{q,t}= \frac{1}{(1-q)(1+t)(q+t)} 
\left( (q^{-1}-q){\mathcal M}_n^2 +\left[{\mathcal M}_{n-1},{\mathcal M}_{n+1}\right]\right) \\
{\mathcal M}_{n+2,n}&=&\left\vert \begin{matrix} {\mathcal M}_{n+2} & {\mathcal M}_{n-1} \\
{\mathcal M}_{n+3} & {\mathcal M}_{n}
\end{matrix}\right\vert_{q,t}=\frac{1}{(q-1)(1-t^2)(q^2-t^2)} \times \\
&&\times
\left( t(1+q)\left[{\mathcal M}_{n},{\mathcal M}_{n+2}\right]_{q^2}-(q+t^2)\left[{\mathcal M}_{n+2},{\mathcal M}_{n}\right]_{q^2}-q(q+t^2)\left[{\mathcal M}_{n-1},{\mathcal M}_{n+3}\right]\right)
\end{eqnarray*}
The property still holds for ${\mathcal M}_{n+3,n}$, but breaks down for ${\mathcal M}_{n+4,n}$ which can at best be expressed as a polynomial
of ${\mathcal M}_{n+4},{\mathcal M}_{n},{\mathcal M}_{n+5},{\mathcal M}_{n-1},{\mathcal M}_{n+3},{\mathcal M}_{n+1}$.

The $(q,t)$-determinant is therefore a subtle deformation of the  quantum determinant. However, it is possible that
one of the expressions for this quantity has a nice combinatorial expression, generalizing eq. \eqref{ASMqdet}
of Theorem \ref{qdethm}.



\subsection{EHA as $t$-deformed quantum cluster algebra}

In this paper, we have constructed a representation of the EHA for finitely many variables $x_1,x_2,...,x_N$.
The algebra itself admits a quotient by the ideal generated by the relations ${\mathcal M}_{N+1;n}=0$, $n\in \Z$,
and the relations expressing that $\psi^{\pm}\propto \sum_{k>0} u_{0,\pm k} z^{\pm k}$ 
are series expansions of finite products. In particular, the condition ${\mathcal M}_{N+1;n}=0$
expresses that a degree $N+1$ polynomial of ${\mathcal M}_n,{\mathcal M}_{n\pm 1}$ 
must vanish (by use of Theorem \ref{polpol} and Corollary \ref{polpolcor}).

This allows to view the EHA as a natural $t$-deformation of the quantum $A_{N-1}$ M-system algebra, 
which corresponds to the limit $t\to\infty$.
We expect the defining relations of the M-system to be explicitly $t$-deformed.
For instance, the first q-commutation relation:
\begin{equation}\label{Msysone}M_nM_{n+1}-qM_{n+1}M_n=0
\end{equation}
is $t$-deformed into:
$$ {\mathcal M}_n \,{\mathcal M}_{n+1}-q\, {\mathcal M}_{n+1}\, {\mathcal M}_n=t(1-q){\mathcal M}_{n+1,n} \ ,$$
obtained from eq.~\eqref{theone} of Theorem \ref{oddMthm} for $k=0$.
Indeed, as ${\mathcal M}_n\sim t^{N-1}$ and ${\mathcal M}_{n,p}\sim t^{2(N-2)}$ for large $t$, we write:
$$ t^{1-N}{\mathcal M}_n \,t^{1-N} {\mathcal M}_{n+1}-q\,t^{1-N} {\mathcal M}_{n+1}\, t^{1-N}{\mathcal M}_n=
t^{-1}(1-q)t^{2(N-2)}{\mathcal M}_{n+1,n} $$
which shows that the r.h.s. is subleading at $t\to \infty$, and we recover the M-system relation \eqref{Msysone} in this limit.

Similarly, the M-system relation 
\begin{equation}\label{Msystwo}
M_{2;n}=M_n^2-q\, M_{n+1}M_{n-1}
\end{equation}
is deformed into \eqref{Mevenone},
namely:
$${\mathcal M}_n^2-q\, {\mathcal M}_{n+1}{\mathcal M}_{n-1}=(q+t+t^2){\mathcal M}_{2;n}-q t {\mathcal M}_{n+1,n-1}.$$
Repeating the scaling analysis, we get
$$ t^{2(1-N)}{\mathcal M}_n^2-q\,  t^{1-N}{\mathcal M}_{n+1} t^{1-N}{\mathcal M}_{n-1}=(1+t^{-1}+q t^{-2}) t^{2(2-N)}{\mathcal M}_{2;n}-q t^{-1}  t^{2(2-N)}{\mathcal M}_{n+1,n-1}$$
and we recover \eqref{Msystwo} in the $t\to\infty$ limit, by neglecting the subleading terms $O(t^{-1}),O(t^{-2})$.
%Note that the dual $Q$-system relation obtained by conjugating \eqref{Msystwo} by $M_{n-1}$:
%\begin{equation}\label{Msysdutwo}
%M_{2;n}=q M_n^2-M_{n-1}M_{n+1}
%\end{equation}
%is $t$-deformed into \eqref{Meventwo}, namely:
%$$q({\mathcal M}_n)^2 -{\mathcal M}_{n-1}{\mathcal M}_{n+1}
%=(q+qt+t^2)\,{\mathcal M}_{2;n}- t \,{\mathcal M}_{n+1,n-1} $$
%and the same scaling analysis as before shows that this tends to \eqref{Msysdutwo} as $t\to \infty$.

More generally, the quantum M-system relations were shown in \cite{DFK16} to be solved by the quantum determinant
\eqref{qdetmal} expressions for $M_{\al;n}$ as polynomials of $\{M_{1;n+i}\}_{|i|<\al}$, with the 
condition that $M_{N+1;n}=0$ for all $n\in \Z$. More generally, we may consider the quantum determinant expression \eqref{ctsimplim}
for $M_{a_1,a_2,...,a_\al}$. Let us rewrite the $(q,t)$ relation of Theorem \ref{Mofm} in the following manner:
\begin{equation}\label{tdefor}
\prod_{1\leq i<j\leq \al}  \left(1-t^{-1}\frac{v_i}{v_j}\right)\left(1-t^{-1}q \frac{v_j}{v_i}\right) \, t^{-\al(N-\al)}{\mathfrak M}_{\al}(\bv)
=\prod_{1\leq i<j\leq \al}\left(1-q \frac{v_j}{v_i}\right) \, \prod_{i=1}^\al t^{-(N-1)} {\mathfrak m}(v_i)
\end{equation}
where $\lim_{t\to\infty}t^{-\al(N-\al)}{\mathfrak M}_{\al}(\bv)=M_\al(\bv)$, and accordingly
$\lim_{t\to\infty}t^{-(N-1)}{\mathfrak m}(v)=m(v)$.
The r.h.s. of \eqref{tdefor} is simply the generating function for the quantum determinants 
$\vert (t^{1-N}{\mathcal M}_{a_j+j-i})_{1\leq i,j\leq \al}\vert_q$ which tend to 
$\vert ({M}_{a_j+j-i})_{1\leq i,j\leq \al}\vert_q$ when $t\to \infty$.
Expanding the l.h.s. in powers of $t^{-1}$ at large $t$, we see that the dominant term
is $t^{-\al(N-\al)}{\mathfrak M}_{\al}(\bv)$ and all other terms are of stricly smaller order. 
This displays explicitly in which sense this
relation is a $t$-deformation of the quantum determinant relation \eqref{ctsimplim}.

\begin{example}
For $\al=2$, \eqref{tdefor} gives in components:
$$t^{2(2-N)}\left((1+qt^{-2}){\mathcal M}_{a_1,a_2} -t^{-1}{\mathcal M}_{a_1-1,a_2+1}-t^{-1}q{\mathcal M}_{a_1+1,a_2-1}\right)
=t^{2(1-N)}\left( {\mathcal M}_{a_1}{\mathcal M}_{a_2}-q {\mathcal M}_{a_1+1}{\mathcal M}_{a_2-1}\right).$$
\end{example}

%More generally, the relations of the quantum $Q$-system are quantum generalizations of the Desnanot-Jacobi identity
%that relates the minors of any square matrix, when applied to the matrix $({M}_{n+j-i})_{1\leq i,j\leq \al}$.
%We expect the $(q,t)$-determinants of the matrices $({\mathcal M}_{a_j+j-i})_{1\leq i,j\leq \al}$ introduced above to satisfy more involved relations, that would be the natural $t$-deformations of the quantum $Q$-system relations.

%We also expect other mutations of the $Q$-system quantum cluster algebra to give rise to new operators, 
%and it would be interesting to investigate their $t$-deformations as well.

\subsection{Relation to graded characters}

The difference operators $M_{\al;n}=\lim_{t\to\infty} t^{\al(\al-N)} {\mathcal M}_{\al;n}$ 
were introduced in \cite{DFK15} to generate graded characters of tensor products of Kirillov-Reshetikhin modules
by iterated action on the constant function $1$.
In particular, any expression of the form $\prod_{i=k}^1 \prod_{\al=1}^{N-1} (M_{\al;i})^{n_{\al,i}} \cdot 1$
 for $n_{\al,i}\in \Z_+$
is Schur positive, namely decomposes onto Schur functions with graded multiplicities in $\Z_+[q]$.
This is not the case for the $t$-deformed version. As an example, it is easy to see that
${\mathcal M}_2\cdot 1=t^{N-1} s_2 -t^{N-2} s_{1,1}$, so Schur positivity is lost. It would be interesting to understand the geometric or representation-theoretical meaning of this $t$-deformation of the $q$-graded characters.
%However, the $t$-deformed expressions $\prod_{i=k}^1 \prod_{\al=1}^{N-1} ({\mathcal M}_{\al;i})^{n_{\al,i}} \cdot 1$
%have a Schur decomposition with coefficients in $\Z[t,q]$, which might still have combinatorial interpretations.

\subsection{Possible generalizations}

The objects and structures of this paper are all linked to the type $A$ groups/algebras. However, many
of them can be extended to other types: on the one hand, DAHA was defined for other types \cite{Cheredbook};
on the other hand (quantum) cluster algebras and $Q$-sytems have been defined for other types as well. 
Preliminary results indicate that similar constructions to those of this paper should exist for the other types. 

Another interesting direction, even in the $A_{N-1}$ case, is to try to understand the meaning of the other
cluster variables (not of the form $M_{\al;n}$) in the quantum cluster algebra of the $Q$-system. 
Preliminary explorations show that those
other variables are also difference operators. Understanding these
could be a step in the direction of fully comprehending the $t$-deformation of the quantum cluster algebra.

