
In this section, we describe the $t\to \infty$ limit of the constructions of this paper, in particular the 
degenerations of our generalized Macdonald operators, and of the shuffle product.

\subsection{Quantum $M$-system and quantum determinant}

\subsubsection{Quantum $M$-system}
The generalized Macdonald operators ${\mathcal M}_{\al;n}$ \eqref{genmacdop} were introduced in \cite{DFK15} 
as the natural $t$-deformation of the difference operators:
\begin{equation}\label{oldmac}
M_{\al;n}:=\lim_{t\to \infty} t^{-\al(N-\al)}\, {\mathcal M}_{\al;n}=\sum_{I\subset [1,N]\atop |I|=\al} x_I^n \prod_{i\in I\atop j\not\in I}
\frac{x_i}{x_i-x_j} \, \Gamma_I
\end{equation}

The difference operators $M_{\al;n}$, together with the quantities $\Delta=\Gamma_1\Gamma_2\cdots \Gamma_N$
and $A=x_1x_2\cdots x_N$ satisfy the quantum $M$-system relations, inherited from the (graded) quantum cluster algebra
associated to the $A_{N-1}$ quantum $Q$-system with a coefficient \cite{DFK15}.
These relations read for $\al,\beta =1,2,...,N$:
\begin{eqnarray*}
M_{\al;n}\, M_{\beta;p}&=& q^{\min(\al,\beta)(p-n)}\, M_{\beta;p}\, M_{\al;n}\qquad (|n-p|\leq |\al-\beta|+1)\\
q^\al \, M_{\al;n+1} \, M_{\al;n-1}&=& (M_{\al;n})^2-M_{\al+1;n}\, M_{\al-1;n} \qquad (1\leq \al \leq N)\\
M_{0;n}&=&1 , \qquad M_{N+1;n}=0, \qquad M_{N,n}=A^n\,\Delta \\
\Delta\, M_{\al;n}&=& q^{\al\, n}\, M_{\al;n}\, \Delta, \quad M_{\al;n}\, A= q^\al \, A \, M_{\al;n},\quad 
\Delta\, A= q^N\, A\, \Delta
\end{eqnarray*}

\subsection{Currents}

There is a simple linear relation between the $t$-deformed ${\mathcal M}_{1;n}$'s and the $M_{1;n}$'s.
\begin{thm}\label{DofM}
We have the relation:
\begin{equation}\label{calMofM}
{\mathcal M}_{1;n}=\frac{t^{N}}{t-1} \sum_{j=0}^{N}(-t^{-1})^j 
\,  e_j(x_1,...,x_{N})\, M_{1;n-j}
\end{equation}
where $e_i$ are the elementary symmetric functions.
\end{thm}
\begin{proof}
Evaluating
\begin{equation}\label{evaone}
\prod_{i=1}^{N} (t x-x_i)=\sum_{j=0}^{N} (-1)^j t^{N-j} x^{N-j}  e_j(x_1,...,x_{N})
\end{equation}
at $x=x_1$, we get:
\begin{equation}\label{evatwo}
\prod_{i=2}^{N} (t x_1-x_i)=\frac{t^{N}}{t-1} \sum_{j=0}^{N}(-t^{-1})^j 
x_1^{N-1-j}  e_j(x_1,...,x_{N})
\end{equation}
%Taking $t=1$ and $x=x_1$ in \eqref{evaone}, we also get:
%\begin{equation}\label{evathree}
%0 = \sum_{j=0}^{N}(-1)^j  
%x_1^{N-1-j}  e_j(x_1,...,x_{N})
%\end{equation}
%Combining \eqref{evatwo} and \eqref{evathree} we arrive at:
%$$ \prod_{i=2}^{N} (t x_1-x_i)= 
%
%When $\theta\to 1$, this tends to a finite limit, thanks to the relation $\sum_{j=0}^{r+1}(-z)^j \,
%e_j(X_1,...,X_{r+1})=\prod_{i=1}^{r+1}(1-z X_i)$, which vanishes at $z=X_1^{-1}$.
%Subtracting $\theta^{-r-1}/(\theta-\theta^{-1})$ times this from the above leads to:
%$$\prod_{i=2}^{r+1} (\theta X_1-\theta^{-1} X_i)=\theta^{-r}\sum_{j=0}^{r+1}(-1)^j \frac{\theta^{2(r+1-j)}-1}{\theta^2-1} 
%X_1^{r-j}  e_j(X_1,...,X_{r+1})$$
We deduce that
\begin{equation}
x_1^n\,\prod_{i=2}^{N}  \frac{t x_1-x_i}{x_1-x_i}=\frac{t^{N}}{t-1} \sum_{j=0}^{N}(-t^{-1})^j \,e_j(x_1,...,x_{N})\,
x_1^{n-j} \,\prod_{i=2}^{N} \frac{x_1}{x_1-x_i} ,
\end{equation}
and the lemma follows by substituting this and its images under the interchanges $x_1 \leftrightarrow x_i$
for $i=2,3,...,N$ into the expression \eqref{genmacdop} for $\al=1$.
\end{proof}

In terms of currents, defining $m(z):=\sum_{n\in \Z} z^n\, M_{1;n}$, 
Theorem \ref{DofM} translates into the following relation.

\begin{cor}
The $t$-deformed Macdonald current ${\mathfrak m}(u)$ is expressed in terms of the $t\to\infty$ limit
$m(u)$ as:
\begin{equation}
{\mathfrak m}(u)=\frac{t^{N}}{t-1}\, C(t^{-1}u)\, m(u)
\end{equation}
with $C$ as in \eqref{Cdef}.
\end{cor}
\begin{proof}
Note that by definition $C(t^{-1} u)=\sum_{j=0}^N (-t^{-1} u)^j e_j(x_1,...,x_N)$ and compute
the generating currents for both sides of \eqref{calMofM}.
\end{proof}

With the obvious definition of limiting currents 
$$m_\al(z):=\sum_{n\in \Z} z^n M_{\al;n}=\lim_{t\to\infty}t^{-\al(N-\al)} {\mathfrak m}_\al(z)\ ,$$
it is easy to recover from eq.\eqref{mbo} the following quantum determinant expression for $m_\al(z)$
(see \cite{DFK16}, Theorem 2.10):
\begin{equation}\label{qdetmal} m_\al(z)={\rm CT}_\bu\left(\delta(u_1\cdots u_\al/z) 
\left(\prod_{1\leq i<j\leq \al} 1-q \frac{u_j}{u_i}\right) \prod_{i=1}^\al m(u_i)\right) 
\end{equation}
Note also that the limit of the result of Corollary \ref{ealine} yields the following alternative formula:
$$ m_\al(z)=\frac{1}{\al!}{\rm CT}_\bu\left(\delta(u_1\cdots u_\al/z) 
\prod_{1\leq i<j\leq \al} (u_i^{-1}-u_j^{-1})(u_i-q u_j)\,  \prod_{i=1}^\al m(u_i)\right) $$

Similarly, we may consider the limiting difference operators:
\begin{equation}\label{limitDM}
D_\al(P):=\lim_{t\to\infty} t^{-\al(N-\al)}\, {\mathcal D}_\al(P)= 
M_\al(P):=\lim_{t\to\infty} t^{-\al(N-\al)}\, {\mathcal M}_\al(P)\ ,
\end{equation}
where
\begin{eqnarray}
D_\al(P)&=&\frac{1}{\al!(N-\al)!}{\rm Sym}\left(P(x_1,...,x_\al) 
\prod_{1\leq i\leq \al<j\leq N}\frac{x_i}{x_i-x_j} \Gamma_1\cdots \Gamma_\al \right)\label{limDal}\\
M_\al(P)&=&\frac{1}{\al!}{\rm CT}_\bu \left(P(u_1^{-1},...,u_\al^{-1}) \prod_{1\leq i<j \leq \al}(u_i^{-1}-u_j^{-1})(u_i-q u_j)
\prod_{i=1}^\al m(u_i) \right) \label{limMal}
\end{eqnarray}
as well as 
$$M_{a_1,...,a_\al}:=\lim_{t\to \infty} t^{-\al(N-\al)} {\mathcal M}_{a_1,...,a_\al}=M_\al(s_{a_1,...,a_\al})\ ,$$
which is a polynomial of degree $\al$ in the $M_\ell$'s, as a direct consequence of formula \eqref{limMal} and the fact that
$s_{a_1,...,a_\al}$ is a Laurent polynomial of $x_1,...,x_\al$. 

\subsubsection{Quantum determinants and Alternating Sign Matrices}
The formula of Corollary \ref{ctsimp} also gives the following
alternative ``quantum determinant" expression:
\begin{equation}\label{ctsimplim}
{M}_{a_1,...,a_\al}=CT_\bu\left(\prod_{i=1}^\al u_i^{-a_i}
 \left( \prod_{1\leq i<j\leq \al}1-q \frac{u_j}{u_i}\right)
\, \prod_{i=1}^\al {m}(u_i)\right)=: \left\vert \left( M_{a_j+i-j}\right)_{1\leq i,j\leq \al} \right\vert_q
\end{equation}
or equivalently the generating multi-current expression:
\begin{equation}\label{geneM}
M_\al(v_1,...,v_\al):=\sum_{a_1,...,a_\al\in\Z}{M}_{a_1,...,a_\al} v_1^{a_1}v_2^{a_2}\cdots v_\al^{a_\al}
=  \left(\prod_{1\leq i<j\leq \al}1-q \frac{v_j}{v_i}\right)\, \prod_{i=1}^\al {m}(v_i)
\end{equation}

There is a very nice expression of the quantum determinant \eqref{ctsimplim}, involving a sum 
over Alternating Sign Matrices (ASM). 
This is because the quantity $\prod_{i<j}v_i+\lambda v_j$ is the $\lambda$-determinant $\lambda\!\det( V_n)$
(as defined by Robbins and Rumsey \cite{RR})
of the Vandermonde matrix $V_n:=(v_i^{n-j})_{1\leq i,j\leq n}$.
Recall that an $n\times n$ ASM $A$ has elements $a_{i,j}\in \{0,1,-1\}$
such that each row and column sum is $1$, and the non-zero entries alternate in sign along each row and column.
We denote by $ASM_n$ the set of such matrices.
We need a few more definitions.
The inversion number of an ASM is the quantity $I(A)=\sum_{i>k,j<\ell} A_{i,j}A_{k,\ell}$. Moreover, for any ASM $A$,
the total number of entries equal to $-1$, which we call the $-1$ number, is denoted by $N(A)$. 
Let us also define the column vector $v=(n-1,n-2,..,1,0)^t$,
and for each ASM $A$ we denote by $m_i(A):= (Av)_i$.
Then we have the explicit formula, obtained by taking $\lambda =-q$ for the $\lambda$-determinant of the 
$\al\times \al$ Vandermonde matrix $V_\al$:
$$\prod_{1\leq i<j\leq \al}v_i-q v_j=\sum_{A\in ASM_n} (-q)^{I(A)-N(A)} (1-q)^{N(A)} \prod_{i=1}^n v_i^{m_i(A)} $$
Combining this with \eqref{ctsimplim}, we deduce the following compact expression for the quantum determinant:

\begin{thm}\label{qdethm}
The quantum determinant of the matrix $\left( M_{a_j+i-j}\right)_{1\leq i,j\leq \al}$ reads:
\begin{equation}\label{ASMqdet}
\left\vert \left( M_{a_j+i-j}\right)_{1\leq i,j\leq \al} \right\vert_q=\sum_{A\in ASM_\al}
(-q)^{I(A)-N(A)}(1-q)^{N(A)} \, \prod_{i=1}^\al M_{ a_i+\al-i-m_i(A)}
\end{equation}
\end{thm}

\begin{example}
For $\al=2$, we have two ASMs:
$$\begin{pmatrix} 1 & 0 \\ 0 & 1\end{pmatrix} \qquad {\rm and}\qquad \begin{pmatrix} 0 & 1 \\ 1 & 0\end{pmatrix}$$
with respective inversion and $-1$ numbers $I(A)=0,1$ and $N(A)=0,0$, and with $(m_1(A),m_2(A))=(1,0),(0,1)$. 
The formula \eqref{ASMqdet}  gives:
$$M_{a_1,a_2}=\left\vert \begin{matrix}M_{a_1} & M_{a_2-1}\\ M_{a_1+1} & M_{a_2}\end{matrix}\right\vert_q
:=M_{a_1}M_{a_2}-qM_{a_1+1}M_{a_2-1}$$

For $\al=3$, we have seven ASMs:
$$ \begin{pmatrix} 1 & 0 & 0\\ 0 & 1 & 0\\ 0& 0& 1\end{pmatrix},
\begin{pmatrix} 0 & 1 & 0\\ 1 & 0 & 0\\ 0& 0& 1\end{pmatrix},
\begin{pmatrix} 1 & 0 & 0\\ 0 & 0 & 1\\ 0& 1& 0\end{pmatrix},
\begin{pmatrix} 0 & 0 & 1\\ 0 & 1 & 0\\ 1& 0& 0\end{pmatrix},
\begin{pmatrix} 0 & 0 & 1\\ 1 & 0 & 0\\ 0& 1& 0\end{pmatrix},
\begin{pmatrix} 0 & 1 & 0\\ 0 & 0 & 1\\ 1& 0& 0\end{pmatrix},
\begin{pmatrix} 0 & 1 & 0\\ 1 & -1 & 1\\ 0& 1& 0\end{pmatrix}
$$
with respective inversion and $-1$ numbers $I(A)=0,1,1,3,2,2,2$ and $N(A)=0,0,0,0,0,0,1$,
and $(m_1(A),m_2(A),m_3(A))=(2,1,0),(1,2,0),(2,0,1),(0,1,2),(0,2,1),(1,0,2),(1,1,1)$.
The formula \eqref{ASMqdet} gives:
\begin{eqnarray*}&&\!\!\!\!\!\!\!\!\!\!\! M_{a_1,a_2,a_3}=\left\vert \begin{matrix}
M_{a_1} & M_{a_2-1}& M_{a_3-2}\\ 
M_{a_1+1} & M_{a_2}& M_{a_3-1}\\
M_{a_1+2} & M_{a_2+1}& M_{a_3}
\end{matrix}\right\vert_q
:=M_{a_1}M_{a_2}M_{a_3}-qM_{a_1+1}M_{a_2-1}M_{a_3}-qM_{a_1}M_{a_2+1}M_{a_3-1}\\
&&\!\!\!\!\!\!\!\!\!\!\!  -q^3 M_{a_1+2}M_{a_2}M_{a_3-2}+ q^2 M_{a_1+2}M_{a_2-1}M_{a_3-1}
+q^2 M_{a_1+1}M_{a_2+1}M_{a_3-2}-q(1-q)M_{a_1+1}M_{a_2}M_{a_3-1}
\end{eqnarray*}
\end{example}

\subsection{The $t\to \infty$ limit of the shuffle product and the quantum M-system}

It is instructive to consider the limiting definition $\star$ of the shuffle product $*$, 
compatible with the limiting difference operators \eqref{limitDM}.

Define for $(P,P')\in {\mathcal F}_\al\times {\mathcal F}_\beta$ the product:
$$P\star P'(x_1,...,x_{\al+\beta}):=\frac{1}{\al!\, \beta!}{\rm Sym}\left( 
\frac{P(x_1,...,x_\al)P'(x_{\al+1},...,x_{\al+\beta})}{\prod_{1\leq i\leq \al<j\leq \al+\beta} (x_j^{-1}-x_i^{-1})(x_j-q x_i)}\right)$$
It is given by the limit 
$$P\star P'(x_1,...,x_{\al+\beta})=\lim_{t\to\infty} t^{-2\al\beta}\, P*P'(x_1,...,x_{\al+\beta})$$
The compatibility:
$$D_\al(P)D_\beta(P')=D_{\al+\beta}(P\star P')$$
simply follows by computing the limit
$$\lim_{t\to\infty} t^{-\al(N-\al)}\, {\mathcal D}_\al(P)\, t^{-\beta(N-\beta)}\,{\mathcal D}_\beta(P')
=\lim_{t\to\infty} t^{-(\al+\beta)(N-\al-\beta)}\,{\mathcal D}_{\al+\beta}(t^{-2\al\beta}\,P*P')$$.

Recall that $M_{\al;n}=D_\al((x_1x_2\cdots x_\al)^n)$. The quantum $M$-system relations
boil down to relations in the $\star$ shuffle algebra, namely:

\begin{thm}\label{msyshuf}
We have the following relations:
\begin{eqnarray*}
&&(x_1x_2\cdots x_\al)^n\star (x_1x_2\cdots x_\beta)^p\\
&&\qquad\qquad=q^{Min(\al,\beta)(p-n)}
(x_1x_2\cdots x_\beta)^p\star (x_1x_2\cdots x_\al)^n\quad (n,p\in \Z,\ |p-n|\leq |\al-\beta|+1)\\
&& q^\al (x_1x_2\cdots x_\al)^{n+1}\star (x_1x_2\cdots x_\al)^{n-1}\\
&&\qquad\qquad=
(x_1x_2\cdots x_\al)^n\star (x_1x_2\cdots x_\al)^n -
(x_1x_2\cdots x_{\al+1})^n\star (x_1x_2\cdots x_{\al-1})^n\quad (n\in \Z)
\end{eqnarray*}
which hold respectively in ${\mathcal F}_{\al+\beta}$ and ${\mathcal F}_{2\al}$, with $\al,\beta\in [1,N-1]$.
\end{thm}

\begin{example}
For $\al=\beta=1$, we have in ${\mathcal F}_{2}$:
$$x_1^n\star x_1^{n+1}=q\,x_1^{n+1}\star x_1^n,\quad (x_1x_2)^n=x_1^n\star x_1^n-q \,x_1^{n+1}\star x_1^{n-1}$$
These follow from respectively
\begin{eqnarray*}
{\rm Sym}\left( \frac{(x_1x_2)^n}{(x_2^{-1}-x_1^{-1})} \right)=
(x_1x_2)^{n+1}\,{\rm Sym}\left( \frac{1}{(x_1-x_2)} \right)&=& 0\\
{\rm Sym}\left( \frac{x_1(x_1x_2)^{n-1}}{(x_2^{-1}-x_1^{-1})} \right)=
(x_1x_2)^{n}\,{\rm Sym}\left( \frac{x_1}{(x_1-x_2)} \right)&=& (x_1x_2)^{n}
\end{eqnarray*}
where ${\rm Sym}$ denotes the symmetrization in $x_1,x_2$. The ``higher" identities:
$${\rm Sym}\left( \frac{x_1^{k+1}(x_1x_2)^{n-1}}{(x_2^{-1}-x_1^{-1})} \right)=
(x_1x_2)^{n}\,{\rm Sym}\left( \frac{x_1^{k+1}}{(x_1-x_2)} \right)=(x_1x_2)^{n}\,s_{k,0}(x_1,x_2)=s_{n+k,n}(x_1,x_2)$$
amount to:
$$ s_{n+k,n}(x_1,x_2)=x_1^{k+n}\star x_1^{n}-q \,x_1^{k+n+1}\star x_1^{n-1}$$
which amounts to the relation
$$M_{n+k,n}=M_{n+k}\, M_n-q \, M_{n+k+1}\, M_{n-1}=
\left\vert \begin{matrix}M_{n+k} & M_{n-1}\\ M_{n+k+1} & M_{n}\end{matrix}\right\vert_q$$
This is nothing but the $v_1^{n+k}v_2^n$ coefficient of $M_2(v_1,v_2)$ \eqref{geneM}.
\end{example}



