\section{Weak formulation}\label{sec:weak_form}


In this section, we introduce the weak formulation of the problem, which is the foundation for constructing numerical solutions. The link between 
the classical solution and the weak solution of this model will be analyzed as well.


%Spatial discretization involves constructing a weak formulation of the problem \eqref{eq:problem1} over a given domain. 
For simplicity, we choose a finite interval $[V_{\min}, V_F]$ as the computation domain and suppose $V_{\min}$ is small enough such that the density function $p(v,t)$ for  $v < V_{\min}$ is negligible. Then the semi-unbounded problem \eqref{eq:problem1} can be truncated to boundary value problem as follow:
\begin{equation}
    \label{eq:problem2}
    \begin{cases}
        \partial_{t}p+\partial_{v}(hp)-a\partial_{v v}p=0,\qquad v\in[V_{\min},V_F]/\{V_R\},\\
        p(v,0)=p^0(v),\qquad p(V_{\min},t)=p(V_F,t)=0,\\
        p(V^-_R,t)=p(V^+_R,t),\quad \partial _vp(V^-_R,t)=\partial _vp(V^+_R,t)+\frac{N(t)}{a}.\\
    \end{cases}
\end{equation}
The truncated equation \eqref{eq:problem2} should still satisfy the mass conservation, i.e.
\begin{equation}
    \label{eq:mass_conservation}
    \int_{V_{\min}}^{V_{F}} p(v, t) d v=\int_{V_{\min}}^{V_{F}} p^{0}(v) d v=1.
\end{equation}
By integrating \eqref{eq:problem2} and using the boundary conditions therein, this conversation implies the following boundary condition.
\begin{equation}
    \label{eq:leftd}
    \frac{\partial}{\partial v}p(V_{\min},t)=0.
\end{equation}
In fact, \eqref{eq:leftd} is never precisely satisfied, but as long as $V_{\min}$ is chosen properly, $\partial_vp(V_{\min},t)$ is negligible.


%Before defining the weak solution, it is necessary to have a clear interpretation of the classical solution. 
 We adopt the definition of the classical solution in \cite{carrillo2013classical}\cite{liu2022rigorous} for the truncated problem.
\begin{definition}[classical solution] \label{class_solution}
    For any given $0<T<+\infty$, $p(v,t)$ is a classical solution of \eqref{eq:problem2} in the time interval $(0, T]$  in the following sense:
    \begin{itemize}
        \item[1.] $N(t)=-a\partial_vp(V_F^-,t)$ is a continuous function for $t\in [0,T]$,
        \item[2.] $p(v,t)$ is continuous in the region $\{(v,t):V_{\min}<v<V_F, t\in[0,T]\}$,
        \item[3.] $p_{vv}$ and $p_t$ are well defined in the region $\{(v,t): v\in [V_{\min},V_R)\cup(V_R,V_F], t \in (0,T]\}$,
        \item[4.] $p_v(V_R^-,t)$ and $p_v(V_R^+,t)$ are well defined for $t\in(0,T]$,
        \item[5.] For $t\in (0,T]$, equation \eqref{eq:problem2} is satisfied,
        \item[6.] $p(v,0)=p^0(v)$ for $v\in [V_{\min},V_R)\cup(V_R,V_F] $.
    \end{itemize}
\end{definition}
In this paper, we consider classical solutions of \eqref{eq:problem2} which additionally satisfy \eqref{eq:leftd}. Having explicitly defined the classical solution of \eqref{eq:problem2}, we can now move on to discuss the weak solution.


If $p(v,t)$ is the classical solution of \eqref{eq:problem2} , weak formulation of \eqref{eq:problem2} is obtained by multiplying \eqref{eq:problem2} with some test function $\phi \in C^{\infty}([V_{\min}, V_F])$ and integrating over $[V_{\min}, V_F]$
\begin{equation}
    \label{variational_1}
    \int_{V_{\min}}^{V_{F}} \left(
    \partial_{t}p+\partial_{v}(hp)-a\partial_{v v}p\right)\phi dv =0.
\end{equation}
Integrating by parts in intervals $[V_{\min},V_R]$ and $[V_R,V_F]$ respectively, we obtain 
\begin{equation}
    \label{eq:int_by_part}
\begin{aligned}
    &\int_{V_{\min}}^{V_{F}} \left(\partial_tp \phi-hp\partial_v\phi+a\partial_vp\partial_v\phi\right) dv\\
    +&\left(hp\phi|_{V_{\min}}^{V_R^-}+hp\phi|_{V_R^+}^{V_F}\right)-\left(a\partial_vp\phi|_{V_{\min}}^{V_R^-}+a\partial_vp\phi|_{V_{V_R^+}}^{V_F}\right)=0.
\end{aligned}
\end{equation}
By substituting the boundary conditions in \eqref{eq:problem2} and \eqref{eq:leftd}, \eqref{eq:int_by_part} can be simplified as
\begin{equation}
    \label{variational_2}
    \int_{V_{\min}}^{V_{F}} \left(\partial_tp \phi-hp\partial_v\phi+a\partial_vp\partial_v\phi\right) dv+a\partial_vp(V_F)\left(\phi(V_R)-\phi(V_F)\right)=0.
\end{equation}
The above derivation helps to formally introduce the definition of the weak solution of \eqref{eq:problem2}.
\begin{definition}[weak solution] \label{weak_solution}
    The variational space appropriate for the present case is
    \begin{equation}
        \label{eq:variational_space}
        \mathbb{H}^{1}_0(V_{\min},V_F)=\{p\in \mathbb{H}^{1}(V_{\min},V_F):p|_{V_{\min}}=p|_{V_F}=0\}.
    \end{equation}
    We say  $p(v,t)\in  C^{1}([0,T];\, \mathbb{H}^{1}_0(V_{\min},V_F)) $ is a weak solution of \eqref{eq:problem2}  if for any test function $\phi(v) \in \mathbb{H}^{1}(V_{\min},V_F)$, \eqref{variational_2} holds for $\forall t\in (0,T]$ and $p(v,0)=p^0(v)$. 
\end{definition}
The weak solution in Definition \ref{weak_solution} still inherits the essence of the original problem \eqref{eq:problem2}, and the relation between the weak solution and the classical solution is established in the following.
\begin{theorem} [Relation with the classical solution]
If $p(v,t)$ is a classical solution of \eqref{eq:problem2} in the time interval $(0, T]$ which also satisfies \eqref{eq:leftd}, then it is a weak solution of \eqref{eq:problem2} in the same time interval. Conversely, if $p(v,t)$ is a weak solution of \eqref{eq:problem2} in the time interval $(0, T]$ and additionally we assume that $$p(v,t) \in C^{1}\left((0,T];\,C^{2}\left([V_{\min},V_R)\cup(V_R,V_F]\right)\right) $$ satisfies $p(V_R^-,t)=p(V_R^+,t)$ and the one-sided derivatives of $p(v,t)$ exist at each side of $V_R$ for all $t\in(0,T] $, then it is a classical solution of \eqref{eq:problem2} in the same time interval and it satisfies \eqref{eq:leftd}.
\end{theorem}
\begin{proof}
The first part of the theorem can obviously be proved by the derivation of the weak solution. 

For the other direction, let $p(v,t)$ be a weak solution of \eqref{eq:problem2} in the time interval $(0, T]$, and $p$ satisfies all the additional assumptions in the statement. We aim to prove that $p(v,t)$ is a classical solution in Definition \ref{class_solution}, and satisfies \eqref{eq:leftd}. 


By the definition of the weak solution and the additional conditions it satisfies, it is straightforward to show that the solution $p$ meets the first four and the last criteria laid out in Definition \ref{class_solution}. In particular, the smoothness assumption at $V_F$ (from the left-hand side) implies the continuity of $N(t)$.  In the following, we will thoroughly demonstrate that $p$ conforms to the fifth item of Definition \ref{class_solution} and \eqref{eq:leftd}. By integration by parts, \eqref{variational_2} can be rewritten as 
\begin{equation}
\label{eq:int_by_part2}
\begin{aligned}
    &\int_{V_{\min}}^{V_{F}} \left(\partial_{t}p+\partial_{v}(hp)-a\partial_{v v}p\right)\phi dv\\ 
    -&\left(hp\phi|_{V_{\min}}^{V_R^-}+hp\phi|_{V_R^+}^{V_F}\right)+\left(a\partial_vp\phi|_{V_{\min}}^{V_R^-}+a\partial_vp\phi|_{V_{V_R^+}}^{V_F})+a\partial_vp(V_F)(\phi(V_R)-\phi(V_F)\right)=0.
\end{aligned}
\end{equation}
The definition of $\mathbb{H}^{1}_0(V_{\min},V_F)$ states that $p(V_{\min})=p(V_F)=0$, thus
\begin{equation}
    \label{eq:int_by_part3}
    \begin{aligned}
        &\int_{V_{\min}}^{V_{F}} \left(\partial_{t}p+\partial_{v}(hp)-a\partial_{v v}p\right)\phi dv-h(V_R)\phi(V_R)\left(p(V_R^-)-p(V_R^+)\right)\\
        +&a\phi(V_R)\left(\partial_vp(V_R^-)-\partial_vp(V_R^+)+\partial_vp(V_F)\right)-a\partial_vp(V_{\min})\phi(V_{\min})=0,
    \end{aligned}
\end{equation}
The key to the proof is selecting different test function spaces to simplify \eqref{eq:int_by_part3}, such that the equations identified in \eqref{eq:problem2} and the boundary conditions delineated in \eqref{eq:problem2} and \eqref{eq:leftd} are successively established. First, taking the test functions $\phi \in \mathbb{V}_1(V_{\min},V_F)=\{\phi \in \mathbb{H}^1(V_{\min},V_F): \phi(V_R)=\phi(V_{\min})=0\}$, \eqref{eq:int_by_part3} reduce to
\begin{equation}
    \int_{V_{\min}}^{V_{F}} \left(\partial_{t}p+\partial_{v}(hp)-a\partial_{v v}p\right)\phi dv=0.
\end{equation}
 Since $p\in C^{2}\left([V_{\min},V_R)\cup(V_R,V_F]\right)$, $\partial_{t}p+\partial_{v}(hp)-a\partial_{v v}p$ is continuous on the interval $(V_{\min},V_R)\cup(V_R,V_F)$, and it can be inferred from the arbitrariness of $\phi$ that $p$ satisfies
\begin{equation}
    \label{pde_equation}
    \partial_{t}p+\partial_{v}(hp)-a\partial_{v v}p=0, \quad \forall v\in (V_{\min},V_R)\cup(V_R,V_F).
\end{equation}
This verifies that the equation in \eqref{eq:problem2} holds for  $p(v,t)$ within the interval, and the next step in the proof is that the weak solution satisfies the boundary conditions in \eqref{eq:problem2} and \eqref{eq:leftd}. By the definition of trial function space $\mathbb{H}^{1}_0(V_{\min},V_F)$, it is easy to see that $p(v,t)$ satisfies the following boundary conditions
\begin{equation}\label{eq:Dirichlet_boundary}
\begin{aligned}
    &p(V_{\min},t)=p(V_F,t)=0.\\
\end{aligned}
\end{equation}
Changing the test functions $\phi \in \mathbb{V}_2(V_{\min},V_F)=\{\phi \in \mathbb{H}^1(V_{\min},V_F): \phi(V_R)=0\}$ and using \eqref{pde_equation}, \eqref{eq:int_by_part3} can be written as
\begin{equation}
    a\partial_vp(V_{\min})\phi(V_{\min})=0.
\end{equation}
Since the arbitrariness of $\phi(V_{\min})$, we obtain
\begin{equation}\label{eq:left_boundary}
    \partial_vp(V_{\min})=0.
\end{equation}
Similarly, changing the test functions $\phi \in \mathbb{V}_3(V_{\min},V_F)=\{\phi \in \mathbb{H}^1(V_{\min},V_F): \phi(V_{\min})=0\}$ again and using \eqref{pde_equation}, \eqref{eq:int_by_part3} is reduced into
\begin{equation}
    -h(V_R)\phi(V_R)\left(p(V_R^-)-p(V_R^+)\right)+a\phi(V_R)\left(\partial_vp(V_R^-)-\partial_vp(V_R^+)+\partial_vp(V_F)\right)=0.
\end{equation}
Since $p(V^-_R)=p(V^+_R) $ and the arbitrariness of $\phi(V_R)$, we deduce
\begin{equation}
    \partial_vp(V_R^-)-\partial_vp(V_R^+)+\partial_vp(V_F)=0.
\end{equation}
By the definition of trace, $p(v,t)$ satisfies boundary conditions in \eqref{eq:problem2} and \eqref{eq:leftd}. Now, we have proved that $p(v,t)$ satisfies the fifth item in Definition \ref{class_solution}.  To conclude, we have shown that $p(v,t)$ is a classical solution of equation \eqref{eq:problem2}


\end{proof}