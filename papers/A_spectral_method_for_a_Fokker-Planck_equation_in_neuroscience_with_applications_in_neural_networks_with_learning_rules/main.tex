\documentclass[a4paper, 10pt]{article}
\usepackage{srcltx,graphicx,epstopdf}
\usepackage{amsmath, amssymb, amsthm}
\usepackage{color}
\usepackage{xcolor}
\usepackage{multirow}
\usepackage{subfig} % it is suggested to use subfloat rather than subfigure
\usepackage[margin=1in]{geometry}
%%%%%% cleveref: use \cref to replace ``Figure \ref'', ``Section \ref'', etc.
\usepackage[T1]{fontenc}
\usepackage{cleveref}
\usepackage{cite}
\crefname{section}{Section}{Sections}
\crefname{subsection}{Subsection}{Subsections}
\crefname{appendix}{Appendix}{Appendix}
\crefname{figure}{Figure}{Figures}
\crefname{table}{Table}{Tables}
\crefname{property}{Property}{Properties}
\crefname{theorem}{Theorem}{Theorem}
\usepackage{booktabs}
\usepackage{tabularx}
\usepackage{threeparttable}
\usepackage{enumerate}
\usepackage[normalem]{ulem}
\graphicspath{{graphh/}}

\newtheorem{theorem}{Theorem}
\newtheorem{lemma}[theorem]{Lemma}
\newtheorem{deduction}[theorem]{Deduction}
\newtheorem{property}[theorem]{Property}
\newtheorem{proposition}{Proposition}
\newtheorem{definition}[theorem]{Definition}
\newtheorem{comment}{Comment}
\newtheorem{example}[theorem]{Example}

{\theoremstyle{remark} \newtheorem{remark}{Remark}}
\newtheorem{corollary}{Corollary}



\renewcommand{\ULthickness}{2pt}

\newcommand\pd[2]{\dfrac{\partial {#1}}{\partial {#2}}}
\newcommand\od[2]{\dfrac{\dd {#1}}{\dd {#2}}}
\newcommand\odd[2]{\dfrac{\mathrm{D} {#1}}{\mathrm{D} {#2}}}
\def\bsxi{\boldsymbol{\xi}}
\newcommand\bxi{\boldsymbol{\xi}}
\newcommand\bA{{\bf A}}
\newcommand\bbA{\bar{\bf A}}
\newcommand\bbB{\bar{\bf B}}

\newcommand\bI{{\bf I}}
\newcommand\bbI{\bar{{\bf I}}}

\newcommand\bB{{\bf B}}
\newcommand\mE{\mathcal{E}}
\newcommand\rC[2]{{\rm{C}}_{{#1},{#2}}}
\newcommand\bN{\boldsymbol{N}}
\newcommand\bx{\boldsymbol{x}}
\newcommand\bn{\boldsymbol{n}}
\newcommand\bu{\boldsymbol{u}}
\newcommand\bv{\boldsymbol{v}}
\newcommand\bc{\boldsymbol{c}}
\newcommand\bC{\boldsymbol{C}}
\newcommand\bM{{\bf M}}
\newcommand\bD{{\bf D}}
\newcommand\ba{\boldsymbol{a}}
\newcommand\bomega{\boldsymbol{\omega}}
\newcommand{\imag}{\mathrm{i}}
\newcommand\bg{\boldsymbol{g}}
\newcommand\bR{\boldsymbol{R}}
\newcommand\bs{\boldsymbol{s}}
\newcommand\bh{\boldsymbol{h}}
\newcommand\bbR{\mathbb{R}}
\newcommand\bbN{\mathbb{N}}
\newcommand\bbZ{\mathbb{Z}}
\newcommand\bbS{\mathbb{S}}
\newcommand\bdeta{\boldsymbol{\eta}}
\newcommand\htheta{\hat{\theta}}
\newcommand\bq{\boldsymbol{q}}
\newcommand\bPhi{\boldsymbol{\Phi}}
\newcommand\dd{\,\mathrm{d}}
\newcommand\Kn{\mathit{Kn}}
\newcommand\mQ{\mathcal{Q}}
\newcommand\mM{\mathcal{M}}
\newcommand\mH{\mathcal{H}}
\newcommand\mF{\mathcal{F}}
\newcommand\mG{\mathcal{G}}
\newcommand\mS{\mathcal{S}}


\newcommand\mI{\mathcal{I}}
\newcommand\bw{\boldsymbol{w}}
\newcommand\He{\mathit{He}}
\def\bd{\boldsymbol{d}}
\def\bx{\boldsymbol{x}}
\def\bbR{\mathbb{R}}
\def\bu{\boldsymbol{u}}
\def\bq{\boldsymbol{q}}
\def\bsigma{\boldsymbol{\sigma}}
\def\bslambda{\boldsymbol{\lambda}}
\def\bE{\boldsymbol{E}}
\def\bF{\boldsymbol{F}}
\def\Identity{\boldsymbol{I}}
\def\bg{\boldsymbol{g}}
\def\bn{\boldsymbol{n}}
\def\bm{\boldsymbol{m}}
\def\bsalpha{\boldsymbol{\alpha}}
\def\bsOmega{\boldsymbol{\Omega}}
\def\bsmu{\boldsymbol{\mu}}
\numberwithin{equation}{section}

\newcommand\note[2]{{{\bf #1}\color{red} [ {\it #2} ]}}
\definecolor{electricpurple}{rgb}{0.75,0.0,1.0}
\definecolor{darkred}{rgb}{0.65,0,0}
\definecolor{green}{rgb}{0.0, 0.5, 0.0}

% \newcommand\fy[1]{{\color{red}FY: #1}}
 \newcommand\addi[1]{{\color{darkred} #1}}
 \newcommand\addii[1]{{\color{blue} #1}}
% \newcommand\addiii[1]{{\color{electricpurple}#1}}
% \newcommand\addall[1]{{\color{green}#1}}
\newcommand\wyl[1]{{\color{red} #1}}
\newcommand\zzn[1]{{\color{blue} #1}}
 \newcommand\deletei{\bgroup\markoverwith{\textcolor{darkred}{\rule[0.5ex]{1pt}{1pt}}}\ULon}
 \newcommand\deleteii{\bgroup\markoverwith{\textcolor{blue}{\rule[0.5ex]{2pt}{2pt}}}\ULon}
% \newcommand\deleteiii{\bgroup\markoverwith{\textcolor{electricpurple}{\rule[0.5ex]{2pt}{2pt}}}\ULon}
% \newcommand\deleteall{\bgroup\markoverwith{\textcolor{green}{\rule[0.5ex]{2pt}{2pt}}}\ULon}


\title {A spectral method for a Fokker-Planck equation in neuroscience with applications in neural networks with learning rules }

%\author{Pei Zhang\thanks{Beijing Computational Science Research Center, Beijing, China, 100193, email: {\tt zhangpei@csrc.ac.cn}}, ~~Zhennan Zhou\thanks{Beijing International Center for Mathematical Research, Peking University, Beijing, China, 100871, email: {\tt zhennan@bicmr.pku.edu.cn}}, ~~Yanli Wang\thanks{Beijing Computational Science Research Center, Beijing, China, 100193, email: {\tt ylwang@csrc.ac.cn}}.}
\author{Pei Zhang\thanks{Beijing Computational Science Research Center, Beijing, China, 100193, email: {\tt zhangpei@csrc.ac.cn}},
   ~~Yanli Wang\thanks{Beijing Computational Science Research Center, Beijing, China, 100193, email: {\tt ylwang@csrc.ac.cn}},
   ~~Zhennan Zhou\thanks{Beijing International Center for Mathematical Research, Peking University, Beijing, China, 100871, email: {\tt zhennan@bicmr.pku.edu.cn}}.
    }

\begin{document}
\maketitle
%\tableofcontents
%\clearpage 
% \leavevmode
% \\
% \\
% \\
% \\
% \\
\section{Introduction}
\label{introduction}

AutoML is the process by which machine learning models are built automatically for a new dataset. Given a dataset, AutoML systems perform a search over valid data transformations and learners, along with hyper-parameter optimization for each learner~\cite{VolcanoML}. Choosing the transformations and learners over which to search is our focus.
A significant number of systems mine from prior runs of pipelines over a set of datasets to choose transformers and learners that are effective with different types of datasets (e.g. \cite{NEURIPS2018_b59a51a3}, \cite{10.14778/3415478.3415542}, \cite{autosklearn}). Thus, they build a database by actually running different pipelines with a diverse set of datasets to estimate the accuracy of potential pipelines. Hence, they can be used to effectively reduce the search space. A new dataset, based on a set of features (meta-features) is then matched to this database to find the most plausible candidates for both learner selection and hyper-parameter tuning. This process of choosing starting points in the search space is called meta-learning for the cold start problem.  

Other meta-learning approaches include mining existing data science code and their associated datasets to learn from human expertise. The AL~\cite{al} system mined existing Kaggle notebooks using dynamic analysis, i.e., actually running the scripts, and showed that such a system has promise.  However, this meta-learning approach does not scale because it is onerous to execute a large number of pipeline scripts on datasets, preprocessing datasets is never trivial, and older scripts cease to run at all as software evolves. It is not surprising that AL therefore performed dynamic analysis on just nine datasets.

Our system, {\sysname}, provides a scalable meta-learning approach to leverage human expertise, using static analysis to mine pipelines from large repositories of scripts. Static analysis has the advantage of scaling to thousands or millions of scripts \cite{graph4code} easily, but lacks the performance data gathered by dynamic analysis. The {\sysname} meta-learning approach guides the learning process by a scalable dataset similarity search, based on dataset embeddings, to find the most similar datasets and the semantics of ML pipelines applied on them.  Many existing systems, such as Auto-Sklearn \cite{autosklearn} and AL \cite{al}, compute a set of meta-features for each dataset. We developed a deep neural network model to generate embeddings at the granularity of a dataset, e.g., a table or CSV file, to capture similarity at the level of an entire dataset rather than relying on a set of meta-features.
 
Because we use static analysis to capture the semantics of the meta-learning process, we have no mechanism to choose the \textbf{best} pipeline from many seen pipelines, unlike the dynamic execution case where one can rely on runtime to choose the best performing pipeline.  Observing that pipelines are basically workflow graphs, we use graph generator neural models to succinctly capture the statically-observed pipelines for a single dataset. In {\sysname}, we formulate learner selection as a graph generation problem to predict optimized pipelines based on pipelines seen in actual notebooks.

%. This formulation enables {\sysname} for effective pruning of the AutoML search space to predict optimized pipelines based on pipelines seen in actual notebooks.}
%We note that increasingly, state-of-the-art performance in AutoML systems is being generated by more complex pipelines such as Directed Acyclic Graphs (DAGs) \cite{piper} rather than the linear pipelines used in earlier systems.  
 
{\sysname} does learner and transformation selection, and hence is a component of an AutoML systems. To evaluate this component, we integrated it into two existing AutoML systems, FLAML \cite{flaml} and Auto-Sklearn \cite{autosklearn}.  
% We evaluate each system with and without {\sysname}.  
We chose FLAML because it does not yet have any meta-learning component for the cold start problem and instead allows user selection of learners and transformers. The authors of FLAML explicitly pointed to the fact that FLAML might benefit from a meta-learning component and pointed to it as a possibility for future work. For FLAML, if mining historical pipelines provides an advantage, we should improve its performance. We also picked Auto-Sklearn as it does have a learner selection component based on meta-features, as described earlier~\cite{autosklearn2}. For Auto-Sklearn, we should at least match performance if our static mining of pipelines can match their extensive database. For context, we also compared {\sysname} with the recent VolcanoML~\cite{VolcanoML}, which provides an efficient decomposition and execution strategy for the AutoML search space. In contrast, {\sysname} prunes the search space using our meta-learning model to perform hyperparameter optimization only for the most promising candidates. 

The contributions of this paper are the following:
\begin{itemize}
    \item Section ~\ref{sec:mining} defines a scalable meta-learning approach based on representation learning of mined ML pipeline semantics and datasets for over 100 datasets and ~11K Python scripts.  
    \newline
    \item Sections~\ref{sec:kgpipGen} formulates AutoML pipeline generation as a graph generation problem. {\sysname} predicts efficiently an optimized ML pipeline for an unseen dataset based on our meta-learning model.  To the best of our knowledge, {\sysname} is the first approach to formulate  AutoML pipeline generation in such a way.
    \newline
    \item Section~\ref{sec:eval} presents a comprehensive evaluation using a large collection of 121 datasets from major AutoML benchmarks and Kaggle. Our experimental results show that {\sysname} outperforms all existing AutoML systems and achieves state-of-the-art results on the majority of these datasets. {\sysname} significantly improves the performance of both FLAML and Auto-Sklearn in classification and regression tasks. We also outperformed AL in 75 out of 77 datasets and VolcanoML in 75  out of 121 datasets, including 44 datasets used only by VolcanoML~\cite{VolcanoML}.  On average, {\sysname} achieves scores that are statistically better than the means of all other systems. 
\end{itemize}


%This approach does not need to apply cleaning or transformation methods to handle different variances among datasets. Moreover, we do not need to deal with complex analysis, such as dynamic code analysis. Thus, our approach proved to be scalable, as discussed in Sections~\ref{sec:mining}.
\section{Weak formulation}\label{sec:weak_form}


In this section, we introduce the weak formulation of the problem, which is the foundation for constructing numerical solutions. The link between 
the classical solution and the weak solution of this model will be analyzed as well.


%Spatial discretization involves constructing a weak formulation of the problem \eqref{eq:problem1} over a given domain. 
For simplicity, we choose a finite interval $[V_{\min}, V_F]$ as the computation domain and suppose $V_{\min}$ is small enough such that the density function $p(v,t)$ for  $v < V_{\min}$ is negligible. Then the semi-unbounded problem \eqref{eq:problem1} can be truncated to boundary value problem as follow:
\begin{equation}
    \label{eq:problem2}
    \begin{cases}
        \partial_{t}p+\partial_{v}(hp)-a\partial_{v v}p=0,\qquad v\in[V_{\min},V_F]/\{V_R\},\\
        p(v,0)=p^0(v),\qquad p(V_{\min},t)=p(V_F,t)=0,\\
        p(V^-_R,t)=p(V^+_R,t),\quad \partial _vp(V^-_R,t)=\partial _vp(V^+_R,t)+\frac{N(t)}{a}.\\
    \end{cases}
\end{equation}
The truncated equation \eqref{eq:problem2} should still satisfy the mass conservation, i.e.
\begin{equation}
    \label{eq:mass_conservation}
    \int_{V_{\min}}^{V_{F}} p(v, t) d v=\int_{V_{\min}}^{V_{F}} p^{0}(v) d v=1.
\end{equation}
By integrating \eqref{eq:problem2} and using the boundary conditions therein, this conversation implies the following boundary condition.
\begin{equation}
    \label{eq:leftd}
    \frac{\partial}{\partial v}p(V_{\min},t)=0.
\end{equation}
In fact, \eqref{eq:leftd} is never precisely satisfied, but as long as $V_{\min}$ is chosen properly, $\partial_vp(V_{\min},t)$ is negligible.


%Before defining the weak solution, it is necessary to have a clear interpretation of the classical solution. 
 We adopt the definition of the classical solution in \cite{carrillo2013classical}\cite{liu2022rigorous} for the truncated problem.
\begin{definition}[classical solution] \label{class_solution}
    For any given $0<T<+\infty$, $p(v,t)$ is a classical solution of \eqref{eq:problem2} in the time interval $(0, T]$  in the following sense:
    \begin{itemize}
        \item[1.] $N(t)=-a\partial_vp(V_F^-,t)$ is a continuous function for $t\in [0,T]$,
        \item[2.] $p(v,t)$ is continuous in the region $\{(v,t):V_{\min}<v<V_F, t\in[0,T]\}$,
        \item[3.] $p_{vv}$ and $p_t$ are well defined in the region $\{(v,t): v\in [V_{\min},V_R)\cup(V_R,V_F], t \in (0,T]\}$,
        \item[4.] $p_v(V_R^-,t)$ and $p_v(V_R^+,t)$ are well defined for $t\in(0,T]$,
        \item[5.] For $t\in (0,T]$, equation \eqref{eq:problem2} is satisfied,
        \item[6.] $p(v,0)=p^0(v)$ for $v\in [V_{\min},V_R)\cup(V_R,V_F] $.
    \end{itemize}
\end{definition}
In this paper, we consider classical solutions of \eqref{eq:problem2} which additionally satisfy \eqref{eq:leftd}. Having explicitly defined the classical solution of \eqref{eq:problem2}, we can now move on to discuss the weak solution.


If $p(v,t)$ is the classical solution of \eqref{eq:problem2} , weak formulation of \eqref{eq:problem2} is obtained by multiplying \eqref{eq:problem2} with some test function $\phi \in C^{\infty}([V_{\min}, V_F])$ and integrating over $[V_{\min}, V_F]$
\begin{equation}
    \label{variational_1}
    \int_{V_{\min}}^{V_{F}} \left(
    \partial_{t}p+\partial_{v}(hp)-a\partial_{v v}p\right)\phi dv =0.
\end{equation}
Integrating by parts in intervals $[V_{\min},V_R]$ and $[V_R,V_F]$ respectively, we obtain 
\begin{equation}
    \label{eq:int_by_part}
\begin{aligned}
    &\int_{V_{\min}}^{V_{F}} \left(\partial_tp \phi-hp\partial_v\phi+a\partial_vp\partial_v\phi\right) dv\\
    +&\left(hp\phi|_{V_{\min}}^{V_R^-}+hp\phi|_{V_R^+}^{V_F}\right)-\left(a\partial_vp\phi|_{V_{\min}}^{V_R^-}+a\partial_vp\phi|_{V_{V_R^+}}^{V_F}\right)=0.
\end{aligned}
\end{equation}
By substituting the boundary conditions in \eqref{eq:problem2} and \eqref{eq:leftd}, \eqref{eq:int_by_part} can be simplified as
\begin{equation}
    \label{variational_2}
    \int_{V_{\min}}^{V_{F}} \left(\partial_tp \phi-hp\partial_v\phi+a\partial_vp\partial_v\phi\right) dv+a\partial_vp(V_F)\left(\phi(V_R)-\phi(V_F)\right)=0.
\end{equation}
The above derivation helps to formally introduce the definition of the weak solution of \eqref{eq:problem2}.
\begin{definition}[weak solution] \label{weak_solution}
    The variational space appropriate for the present case is
    \begin{equation}
        \label{eq:variational_space}
        \mathbb{H}^{1}_0(V_{\min},V_F)=\{p\in \mathbb{H}^{1}(V_{\min},V_F):p|_{V_{\min}}=p|_{V_F}=0\}.
    \end{equation}
    We say  $p(v,t)\in  C^{1}([0,T];\, \mathbb{H}^{1}_0(V_{\min},V_F)) $ is a weak solution of \eqref{eq:problem2}  if for any test function $\phi(v) \in \mathbb{H}^{1}(V_{\min},V_F)$, \eqref{variational_2} holds for $\forall t\in (0,T]$ and $p(v,0)=p^0(v)$. 
\end{definition}
The weak solution in Definition \ref{weak_solution} still inherits the essence of the original problem \eqref{eq:problem2}, and the relation between the weak solution and the classical solution is established in the following.
\begin{theorem} [Relation with the classical solution]
If $p(v,t)$ is a classical solution of \eqref{eq:problem2} in the time interval $(0, T]$ which also satisfies \eqref{eq:leftd}, then it is a weak solution of \eqref{eq:problem2} in the same time interval. Conversely, if $p(v,t)$ is a weak solution of \eqref{eq:problem2} in the time interval $(0, T]$ and additionally we assume that $$p(v,t) \in C^{1}\left((0,T];\,C^{2}\left([V_{\min},V_R)\cup(V_R,V_F]\right)\right) $$ satisfies $p(V_R^-,t)=p(V_R^+,t)$ and the one-sided derivatives of $p(v,t)$ exist at each side of $V_R$ for all $t\in(0,T] $, then it is a classical solution of \eqref{eq:problem2} in the same time interval and it satisfies \eqref{eq:leftd}.
\end{theorem}
\begin{proof}
The first part of the theorem can obviously be proved by the derivation of the weak solution. 

For the other direction, let $p(v,t)$ be a weak solution of \eqref{eq:problem2} in the time interval $(0, T]$, and $p$ satisfies all the additional assumptions in the statement. We aim to prove that $p(v,t)$ is a classical solution in Definition \ref{class_solution}, and satisfies \eqref{eq:leftd}. 


By the definition of the weak solution and the additional conditions it satisfies, it is straightforward to show that the solution $p$ meets the first four and the last criteria laid out in Definition \ref{class_solution}. In particular, the smoothness assumption at $V_F$ (from the left-hand side) implies the continuity of $N(t)$.  In the following, we will thoroughly demonstrate that $p$ conforms to the fifth item of Definition \ref{class_solution} and \eqref{eq:leftd}. By integration by parts, \eqref{variational_2} can be rewritten as 
\begin{equation}
\label{eq:int_by_part2}
\begin{aligned}
    &\int_{V_{\min}}^{V_{F}} \left(\partial_{t}p+\partial_{v}(hp)-a\partial_{v v}p\right)\phi dv\\ 
    -&\left(hp\phi|_{V_{\min}}^{V_R^-}+hp\phi|_{V_R^+}^{V_F}\right)+\left(a\partial_vp\phi|_{V_{\min}}^{V_R^-}+a\partial_vp\phi|_{V_{V_R^+}}^{V_F})+a\partial_vp(V_F)(\phi(V_R)-\phi(V_F)\right)=0.
\end{aligned}
\end{equation}
The definition of $\mathbb{H}^{1}_0(V_{\min},V_F)$ states that $p(V_{\min})=p(V_F)=0$, thus
\begin{equation}
    \label{eq:int_by_part3}
    \begin{aligned}
        &\int_{V_{\min}}^{V_{F}} \left(\partial_{t}p+\partial_{v}(hp)-a\partial_{v v}p\right)\phi dv-h(V_R)\phi(V_R)\left(p(V_R^-)-p(V_R^+)\right)\\
        +&a\phi(V_R)\left(\partial_vp(V_R^-)-\partial_vp(V_R^+)+\partial_vp(V_F)\right)-a\partial_vp(V_{\min})\phi(V_{\min})=0,
    \end{aligned}
\end{equation}
The key to the proof is selecting different test function spaces to simplify \eqref{eq:int_by_part3}, such that the equations identified in \eqref{eq:problem2} and the boundary conditions delineated in \eqref{eq:problem2} and \eqref{eq:leftd} are successively established. First, taking the test functions $\phi \in \mathbb{V}_1(V_{\min},V_F)=\{\phi \in \mathbb{H}^1(V_{\min},V_F): \phi(V_R)=\phi(V_{\min})=0\}$, \eqref{eq:int_by_part3} reduce to
\begin{equation}
    \int_{V_{\min}}^{V_{F}} \left(\partial_{t}p+\partial_{v}(hp)-a\partial_{v v}p\right)\phi dv=0.
\end{equation}
 Since $p\in C^{2}\left([V_{\min},V_R)\cup(V_R,V_F]\right)$, $\partial_{t}p+\partial_{v}(hp)-a\partial_{v v}p$ is continuous on the interval $(V_{\min},V_R)\cup(V_R,V_F)$, and it can be inferred from the arbitrariness of $\phi$ that $p$ satisfies
\begin{equation}
    \label{pde_equation}
    \partial_{t}p+\partial_{v}(hp)-a\partial_{v v}p=0, \quad \forall v\in (V_{\min},V_R)\cup(V_R,V_F).
\end{equation}
This verifies that the equation in \eqref{eq:problem2} holds for  $p(v,t)$ within the interval, and the next step in the proof is that the weak solution satisfies the boundary conditions in \eqref{eq:problem2} and \eqref{eq:leftd}. By the definition of trial function space $\mathbb{H}^{1}_0(V_{\min},V_F)$, it is easy to see that $p(v,t)$ satisfies the following boundary conditions
\begin{equation}\label{eq:Dirichlet_boundary}
\begin{aligned}
    &p(V_{\min},t)=p(V_F,t)=0.\\
\end{aligned}
\end{equation}
Changing the test functions $\phi \in \mathbb{V}_2(V_{\min},V_F)=\{\phi \in \mathbb{H}^1(V_{\min},V_F): \phi(V_R)=0\}$ and using \eqref{pde_equation}, \eqref{eq:int_by_part3} can be written as
\begin{equation}
    a\partial_vp(V_{\min})\phi(V_{\min})=0.
\end{equation}
Since the arbitrariness of $\phi(V_{\min})$, we obtain
\begin{equation}\label{eq:left_boundary}
    \partial_vp(V_{\min})=0.
\end{equation}
Similarly, changing the test functions $\phi \in \mathbb{V}_3(V_{\min},V_F)=\{\phi \in \mathbb{H}^1(V_{\min},V_F): \phi(V_{\min})=0\}$ again and using \eqref{pde_equation}, \eqref{eq:int_by_part3} is reduced into
\begin{equation}
    -h(V_R)\phi(V_R)\left(p(V_R^-)-p(V_R^+)\right)+a\phi(V_R)\left(\partial_vp(V_R^-)-\partial_vp(V_R^+)+\partial_vp(V_F)\right)=0.
\end{equation}
Since $p(V^-_R)=p(V^+_R) $ and the arbitrariness of $\phi(V_R)$, we deduce
\begin{equation}
    \partial_vp(V_R^-)-\partial_vp(V_R^+)+\partial_vp(V_F)=0.
\end{equation}
By the definition of trace, $p(v,t)$ satisfies boundary conditions in \eqref{eq:problem2} and \eqref{eq:leftd}. Now, we have proved that $p(v,t)$ satisfies the fifth item in Definition \ref{class_solution}.  To conclude, we have shown that $p(v,t)$ is a classical solution of equation \eqref{eq:problem2}


\end{proof}

\section{Numerical scheme and analysis}\label{sec:scheme}
In this section, we present a spectral approximation for the weak solution to the Fokker-Planck equation \eqref{eq:problem2} and construct a fully discrete numerical scheme. Numerical solutions are sought in a specific function space in which the functions satisfy the boundary conditions and can be determined by solving the derived equation system after specifying the test function space.
\subsection{A fully discrete numerical scheme based on Legendre spectral method}\label{sec:fully_discrete_scheme}
In this part, we construct the numerical scheme of the Fokker-Planck equation \eqref{eq:problem2}, which is implemented in two steps. First, the spectral approximation is used for space discretization, resulting in a system of ordinary differential equations; second, a semi-implicit scheme is applied for time discretization. The spectral method is established such that the numerical solution inherently satisfies the boundary conditions. Formally, the approximate variational problem is
\begin{equation}
    \label{var_problem}
    \begin{cases}
        \text{Find } p\in \mathrm{W} \text{ such that}\\
        \int_{V_{\min}}^{V_{F}} \left(
    \partial_{t}p+\partial_{v}(hp)-a\partial_{v v}p\right)\phi =0,\quad \forall \phi \in \mathrm{V},
    \end{cases}
\end{equation}
where $\mathrm{W}$ is the trial function space and $\mathrm{V}$ is the test function space. Compared to Definition \ref{weak_solution}, the variational problem \eqref{var_problem} requires a more complex trial function space, which will be further described below. The specific form of the test function space will be introduced in Section \ref{sec:stability}.
\subsubsection{Construction of trial function space and space discretization}
A challenging aspect of the spectral method is constructing the trial function space $\mathrm{W}$ so as to satisfy the complex boundary conditions, including the discontinuous derivative of the density function and the dynamic boundary. To that end, the trial function space should be a subset of $\mathbb{H}^1_0$ wherein strong boundary derivatives can be defined. Specifically, the polynomial space that fulfills the boundary conditions in \eqref{eq:problem2} and \eqref{eq:leftd} can be used as the trial function space. That is $\mathrm{W} \subseteq \mathrm{P}_{\infty}(V_{\min}, V_R) + \mathrm{P}_{\infty}(V_{R}, V_F)$ and for all $ p\in \mathrm{W}$, it holds that
\begin{equation}
    \label{boudary_condition}
    \begin{cases}
        p(V_{\min})=\partial _vp(V_{\min})=0,\\
        p(V_F)=0,\\
        p(V^-_R)=p(V^+_R),\\
        \partial _vp(V^-_R)=\partial _vp(V^+_R)+\partial _vp(V_F),
    \end{cases}
\end{equation}
where $\mathrm{P}_{\infty}(a,b)$ is the set of all real polynomials defined on the interval $(a, b)$. 
With integration by parts and the boundary conditions \eqref{boudary_condition} in the trial function space, the solution to the above variational problem \eqref{var_problem} agrees with the weak solution specified in Definition \ref{weak_solution}. 


Let $\{\psi_k\}_{k=0}^{\infty}$ be a set of basis functions of $\mathrm{W}$. The approximate solution of problem \eqref{eq:problem2} can be expanded as
\begin{equation}
    \label{eq:approximate_solution1}
    p(v,t)=\sum_{k=0}^{\infty}\hat{u}_k(t)\psi_k(v).
\end{equation}
The essence of constructing the trial function space $\mathrm{W}$ is to determine the specific form of its basis functions $\{\psi_k\}_{k=0}^{\infty}$. This is accomplished by dividing the interval into two segments by the discontinuity point $V_R$, utilizing a fixed number of basis functions to meet the dynamic boundary conditions, and employing basis functions with homogeneous boundary conditions for each segment to improve accuracy. That is
\begin{equation}
    \mathrm{W}=\mathrm{W}_1+\mathrm{W}_2,
\end{equation}
where $\mathrm{W}_1$ is a finite-dimensional space that handles the conditions in \eqref{boudary_condition}, and $\mathrm{W}_2$ enhances accuracy within the interval and satisfies the homogeneous conditions of points $V_{\min}$, $V_R$, and $V_F$, which is
\begin{equation}
    \label{eq:W2condition}
    \begin{cases}
        p(V_{\min})=\partial _vp(V_{\min})=0,\\
         p(V_R^-)=\partial _vp(V_R^-)=0,\\
         p(V_R^+)=\partial _vp(V_R^+)=0,\\
        p(V_F)=\partial _vp(V_F)=0,\\
    \end{cases}\qquad \forall p \in \mathrm{W}_2,
\end{equation}

For simplicity, it is preferable to keep the dimension of $\mathrm{W}_1$ as low as possible. In the case of taking into account the function value and first derivative value, there are eight degrees of freedom at the boundary, comprising of the function value and derivative value at $V_{\min}$, $V_F$, and both sides of $V_R$. Since the five conditions in \eqref{boudary_condition} have to be satisfied, there are three degrees of freedom remaining. Therefore, $\mathrm{W}_1$ can be spanned by three basis functions
\begin{equation}
    \mathrm{W}_1=\text{span}\{g_1,g_2,g_3\},
\end{equation}
where
\begin{equation}
    g_1\Rightarrow
    \begin{gathered}
        \begin{cases}
            g_1(V_{\min})=0,\\
            g_1(V_R)=1,\\
            \partial_v g_1(V_{\min})=0,\\
            \partial_v g_1(V_R)=0,
        \end{cases}\quad v\in(V_{\min},V_R),\qquad
        \begin{cases}
            g_1(V_R)=1,\\
            g_1(V_F)=0,\\
            \partial_v g_1(V_R)=0,\\
            \partial_v g_1(V_F)=0,
        \end{cases}\quad v\in(V_R,V_F).
    \end{gathered}
\end{equation}
\begin{equation}
    g_2\Rightarrow
    \begin{gathered}
        \begin{cases}
            g_2(V_{\min})=0,\\
            g_2(V_R)=0,\\
            \partial_v g_2(V_{\min})=0,\\
            \partial_v g_2(V_R)=1,
        \end{cases}\quad v\in(V_{\min},V_R), \qquad
        \begin{cases}
            g_2(V_R)=0,\\
            g_2(V_F)=0,\\
            \partial_v g_2(V_R)=1,\\
            \partial_v g_2(V_F)=0,
        \end{cases}\quad v\in(V_R,V_F).
    \end{gathered}
\end{equation}
\begin{equation}
    g_3\Rightarrow
    \begin{gathered}
    \begin{cases}
            g_3(V_{\min})=0,\\
            g_3(V_R)=0,\\
            \partial_v g_3(V_{\min})=0,\\
            \partial_v g_3(V_R)=0,
        \end{cases}\quad v\in(V_{\min},V_R), \qquad
    \begin{cases}
        g_3(V_R)=0,\\
        g_3(V_F)=0,\\
        \partial_v g_3(V_R)=1,\\
        \partial_v g_3(V_F)=1,
    \end{cases}\qquad v\in(V_R,V_F).\\
    \end{gathered}
\end{equation}
 The specific form of the basis functions in $\mathrm{W}_1$ are presented in Appendix \ref{app:basis}.

\begin{figure}[!htb]
    \centering
        \begin{minipage}[c]{0.8\textwidth}
            \centering
            \includegraphics[width=1\textwidth]{g.eps}
        \end{minipage}
         \caption{The basis functions of $p_3$ with Equation parameters $V_{\min}=-1,V_R=0,V_F=1$. Here, $g_1$ and $g_2$ are measured using the left axis, while $g_3$ is measured with the right axis.}
        \label{fig:g}
\end{figure}
The specific illustration of $g_1,g_2,g_3$ are shown in Figure \ref{fig:g}. After the basis function of $\mathrm{W}_1$ is determined in this way, we set
\begin{equation}
    \label{eq:w1_expansion}
    p_3=\sum_{k=1}^3 \lambda_kg_k \quad \in \mathrm{W}_1.
\end{equation}
The boundary conditions can be well satisfied by adjusting the coefficients of $g_1,g_2,g_3$ in the following way,
\begin{equation}
    \label{eq:lambda}
    \begin{gathered}
        \begin{cases}
            p(V_R^-)=p_3(V_R^-)=\lambda_1,\\
            p(V_R^-)=p_3(V_R^+)=\lambda_1,
        \end{cases}\qquad
        \begin{cases}
            \partial_vp(V_R^-)=\partial_vp_3(V_R^-)=\lambda_2,\\
            \partial_vp(V_R^-)=\partial_vp_3(V_R^+)=\lambda_2+\lambda_3,\\
            \partial_vp(V_R^-)=\partial_vp_3(V_F^-)=\lambda_3,
        \end{cases}
    \end{gathered}
\end{equation}
where $p$ denotes the numerical solution mentioned in \eqref{eq:approximate_solution1}. 

The construction of the $\mathrm{W}_2$ space is motivated by spectral methods for solving general homogeneous boundary value problems.  According to \eqref{eq:W2condition}, the interval $(V_{\min},V_F) $ is divided into two segments by $V_R$ naturally. Assuming $I_L=(V_{\min},V_R)$ and $I_R=(V_R,V_F)$, we further denote
\begin{equation}
    \label{eq:X_ab}
        \mathrm{X}_{(a,b)}=\left\{\varphi \in P_{\infty}(a,b) : \varphi(a)=\varphi(b)=\varphi'(a)=\varphi'(b)=0   \right\}.\\
\end{equation}
$\mathrm{X}_{(a,b)}$ represents the set of real polynomials defined on the interval $(a, b)$, where the function value and derivative are zero at boundary points. So we can divide $\mathrm{W}_2$ into two parts 
\begin{equation}
    \mathrm{W}_2=\mathrm{X}_{(V_{\min},V_R)}+ \mathrm{X}_{(V_R,V_F)}.
\end{equation}

In the spectral methods, in order to minimize the interaction of basis functions in the frequency space, the basis functions should take the form of adjacent orthogonal polynomials \cite{shen1994efficient}. Therefore, it is reasonable to use a compact combination of Legendre polynomials as basis functions of $\mathrm{X}_{(a,b)}$, namely,
\begin{equation}
    \hat{h}_k=\mathcal{H}_k+\alpha_k\mathcal{H}_{k+1}+\beta_k\mathcal{H}_{k+2}+\gamma_k\mathcal{H}_{k+3}+\eta_k\mathcal{H}_{k+4},\quad k=0,1,2,...,
\end{equation}
where $\mathcal{H}_k$ is the scaling of the kth-degree Legendre polynomial $L_k$
\begin{equation}
    \mathcal{H}_k(v)=L_k(x),\qquad x=\frac{v-\left(\frac{a+b}{2}\right)}{\frac{b-a}{2}},
\end{equation}
and the parameter $\{\alpha_k,\beta_k,\gamma_k,\eta_k\}$ are chosen to satisfy the boundary conditions in \eqref{eq:X_ab}
\begin{equation}
    \alpha_k=0,\, \beta_k=-\frac{4k+10}{2k+7},\,\gamma_k=0,\,\eta_k=\frac{2k+3}{2k+7}.
\end{equation}




After constructing the trial function space, spatial discretization will be discussed, yielding the system of ordinary differential equations for the coefficients. Let $\{h_k\}_{k=0}^{\infty}$ be the basis functions of $\mathrm{W}_2$. Then the approximate solution \eqref{eq:approximate_solution1} can be rewritten as
\begin{equation}
    \label{eq:approximate_solution2}
    p(v,t)=\sum_{k=0}^{\infty}u_k(t)h_k(v)+\sum_{k=1}^3\lambda_k(t)g_k(v).
\end{equation}
The basal functions in \eqref{eq:approximate_solution2} correspond to ones in \eqref{eq:approximate_solution1} in the following way
\begin{equation}
    \{\psi_k\}_{k=0}^{\infty}=\{g_k\}_{k=1}^{3}+\{h_k\}_{k=0}^{\infty}.
\end{equation}
And the expansion coefficients $\{u_k(t)\}_{k=0}^{\infty},\{\lambda_k(t)\}_{k=1}^3$ are to be determined.
Assuming the initial value is to satisfy the boundary conditions \eqref{boudary_condition}, the initial expansion coefficients $\{u_k(0)\}_{k=0}^{\infty},\{\lambda_k(0)\}_{k=1}^3$ can be obtained by the best squares approximation,
\begin{equation}
    \label{eq:initial_vector}
    \int_{V_{\min}}^{V_F} \left(\sum_{k=0}^{\infty}u_k(0)h_k(v)+\sum_{k=1}^3\lambda_k(t)g_k(v)\right) \phi_j dv=\int_{V_{\min}}^{V_F} p^0(v)\phi_j dv, \qquad \forall \phi_j \in \mathrm{V}.
\end{equation}
For a properly defined test function space, the solvability of the \eqref{eq:initial_vector} is guaranteed by the Gram-Schmidt orthogonalization of the basis functions. Note again that the specific form of the test function space is discussed in Section \ref{sec:stability}. We denote the initial value vector as
\begin{equation}
    \label{eq:initial_value}
    \mathbf{P^0}=(\lambda_1(0),\lambda_2(0),\lambda_3(0),u_1(0),u_2(0),...)^T.
\end{equation}

It should be noted that while constructing the basis functions, we assume that the value of $N(t)$ is already known. In fact, $N(t)$ is self-consistently determined in the dynamic process, and $N(t)$ is part of the degrees of freedom of the solution. It follows from \eqref{eq:lambda} that
\begin{equation}
    \partial_vp(V_F,t)=\lambda_3(t).
\end{equation}
One can rewrite the mean firing rate using \eqref{eq:ha} and \eqref{eq:Nt}
\begin{equation}
    \label{Nt1}
    N(t)=-\frac{a_0\lambda_3(t)}{1+a_1\lambda_3(t)}.
\end{equation}
Define
\begin{equation}
    \label{eq:operator}
    \mathcal{L}p(v,t)=\partial_{t}p-\partial_v(vp)-\left(b\frac{a_0\lambda_3(t)}{1+a_1\lambda_3(t)}\right)\partial_{v}p-\left(a_0-a_1\frac{a_0\lambda_3(t)}{1+a_1\lambda_3(t)}\right)\partial_{v v}p.
\end{equation}
The expansion coefficients $\{u_k(t)\}_{k=0}^{\infty},\,\{\lambda_k(t)\}_{k=1}^{3}(t>0)$ in \eqref{eq:approximate_solution2} can be determined by variational problem \eqref{var_problem} with using the mean firing rate $N(t)$ in \eqref{Nt1}:
\begin{equation}
    \label{eq:nonlinear_system1}
    \begin{cases}
        \text{Find } p\in  \mathrm{W} \text{ such that}\\
        (\mathcal{L}p,\phi_j)=0,\quad \forall \phi_j \in \mathrm{V},
    \end{cases}
\end{equation}
where $(\cdot,\cdot)$ is the inner product of the usual $L^2$ space.

The nonlinear system of ordinary differential equations of the above scheme is obtained by substituting \eqref{eq:approximate_solution2} into \eqref{eq:nonlinear_system1}. More precisely, setting
\begin{equation}
    \begin{aligned}
        &\mathbf{P}=(\lambda_1(t),\lambda_2(t),\lambda_3(t),u_1(t),u_2(t),...)^T,\\
        &s_{jk}=\begin{cases}
            (g_k,\phi_j),\qquad &1\leq k \leq3,\\
            (h_{k-4},\phi_j),& k\geq4.
        \end{cases}, &S=(s_{jk})_{j,k=1,2,...},\\
        &a_{jk}=\begin{cases}
            (\partial_v(vg_k),\phi_j),\qquad &1\leq k \leq3,\\
            (\partial_v(vh_{k-4}),\phi_j),& k\geq4.
        \end{cases}, &A=(a_{jk})_{j,k=1,2,...},\\
        &b_{jk}=\begin{cases}
            (\partial_vg_k,\phi_j),\qquad &1\leq k \leq3,\\
            (\partial_vh_{k-4},\phi_j),& k\geq4.
        \end{cases}, &B=(b_{jk})_{j,k=1,2,...},\\
        &c_{jk}=\begin{cases}
            (\partial_{vv}g_k,\phi_j),\qquad &1\leq k \leq3,\\
            (\partial_{vv}h_{k-4},\phi_j),& k\geq4.
        \end{cases}, &C=(c_{jk})_{j,k=1,2,...}.\\
    \end{aligned}
\end{equation}
The nonlinear system \eqref{eq:nonlinear_system1} becomes
\begin{equation}
    \label{eq:nonlinear_sde}
    S\partial_t\mathbf{P}=\left(A+\left(b\frac{a_0\lambda_3(t)}{1+a_1\lambda_3(t)}\right)B+\left(a_0-a_1\frac{a_0\lambda_3(t)}{1+a_1\lambda_3(t)}\right)C\right)\mathbf{P}.
\end{equation}
After the spatial discretization, the solution of problem \eqref{eq:problem2} converts into the solution of the nonlinear ordinary differential equation system of initial value problem \eqref{eq:nonlinear_sde}\eqref{eq:initial_value}.

\subsubsection{Fully discrete numerical scheme}

To finish the construction of the numerical scheme, we need to truncate the approximate solution \eqref{eq:approximate_solution2} to a finite-dimensional one and perform time discretization. The finite-dimensional form of \eqref{eq:X_ab} is denoted as
\begin{equation}
\label{eq:X_N}
    \mathrm{X}_{N(a,b)}=\left\{\varphi \in P_{N+3}(a,b) : \varphi(a)=\varphi(b)=\varphi'(a)=\varphi'(b)=0   \right\},
\end{equation}
where $P_N(a,b)$ is the set of all real polynomials of degree no more than $N$ and the dimension of $P_N{(a,b)}$ is $N+1$. It is evident that a  non-trivial polynomial with the homogeneous boundary conditions in \eqref{eq:X_N} must be of at least fourth degree, thus leading to a reduced dimension of the set in \eqref{eq:X_N}. The polynomial space in \eqref{eq:X_N} is selected as $P_{N+3}(a,b)$ so that the dimension of the $\mathrm{X}_{N(a,b)}$ space is $N$. Then the trial function space can be truncated as

\begin{equation}
    \label{eq:trial_function}
    \mathrm{W}_N=\mathrm{X}_{N(V_{\min},V_R)} + \mathrm{X}_{N(V_R,V_F)} + \mathrm{W}_1.
\end{equation}
Assuming $\{h^L_k\}_{k=0}^{N-1}$ is a set
of basis functions of $X_{N(V_{\min},V_R)}$ and $\{h^R_k\}_{k=0}^{N-1}$ is a set
of basis functions of $X_{N(V_R,V_F)}$. $\{\psi_k\}_{k=1}^{2N+3}=\{h^L_0,...,h^L_{N-1},h^R_0,...,h^R_{N-1},g_1,g_2,g_3\}$ is a basis of $\mathrm{W}_N$. Then the numerical solution $p_N(v,t)$ can be expressed as
\begin{equation}
    \label{eq:approximate_solution3}
    p_N(v,t)=\sum_{k=0}^{N-1} u_k^L(t)h^L_k(v)+\sum_{k=0}^{N-1} u_k^R(t)h^R_k(v)+\sum_{k=1}^3 \lambda_k(t)g_k(v)=\sum_{k=1}^{2N+3}\hat{u}_k(t)\psi_k(v).
\end{equation}
 The initial condition for the expansion coefficients $\{\hat{u}_{k}(0)\}_{k=0}^{2N+3}$ can be obtained by the least square approximation,
\begin{equation}
    \label{eq:initial_vector2}
    \int_{V_{\min}}^{V_F} \sum_{k=1}^{2N+3}\hat{u}_k(0)\psi_k(v) \phi_j dv=\int_{V_{\min}}^{V_F} p^0(v)\phi_j dv, \qquad \forall \phi_j \in \mathrm{V}_N.
\end{equation}

Suppose the truncated test function space is denoted by $\mathrm{V}_N$, which shall be specified later. The expansion coefficients $\{\hat{u}_k(t)\}_{k=0}^{2N+3}(t>0)$ can be determined by the semi-discrete variational formulation
\begin{equation}
    \label{eq:variational_form2}
    \begin{cases}
        \text{Find } p_N\in \mathrm{W}_{N} \text{ such that}\\
        (\mathcal{L}p_N,\phi_j)=0,\quad \forall \phi_j \in \mathrm{V}_N.
    \end{cases}
\end{equation}


For time discretization, we use a semi-implicit method. The interval $[0,T_{\text{max}}]$ is divided  into $n_t$ equal sub-intervals with size
\begin{equation}
    \Delta t=\frac{T_{\text{max}}}{n_t},
\end{equation}
and the grid points can be represented as follows
\begin{equation}
    t^{n}=n \Delta t, \qquad n=0,1,2, \cdots, n_{t}.
\end{equation}


The semi-implicit scheme of \eqref{eq:operator} is denoted by
\begin{equation}
    \label{eq:semi_implicit}
\begin{aligned}
    \tilde{\mathcal{L}}p_N(v,t^{n+1})&=\frac{p_N(v,t^{n+1})-p_N(v,t^{n})}{\Delta t}-\partial_v(vp_N(v,t^{n+1}))+bN(t^n)\partial_vp_N(v,t^{n+1})\\
    &-a(N(t^n))\partial_{vv}p_N(v,t^{n+1})=0,\qquad\qquad\qquad n=1,2,...,n_t.
\end{aligned}
\end{equation}
Note that, the mean firing rate $N(t^n)$ is treated explicitly, but the rest of the terms are implicit. Such a time discretization naturally avoids the use of a nonlinear solver. Then we can obtain the fully discrete scheme of the variational formulation \eqref{eq:variational_form2}: for each time step
\begin{equation}
    \label{eq:variational_form3}
    \begin{cases}
       \text{Find } p_N\in \mathrm{W}_{N} \text{ such that}\\
        (\tilde{\mathcal{L}}p_N,\phi_j)=0,\quad \forall \phi_j \in \mathrm{V}_N.
    \end{cases}
\end{equation}

More precisely, setting
\begin{equation}
    \label{eq:Matrix2}
    \begin{aligned}
        &\hat{\mathbf{P}}^n=(\hat{u}_1(t^n),\hat{u}_2(t^n),...,\hat{u}_{2N+3}(t^n))^T,\\
        &\hat{s}_{jk}=(\psi_k,\phi_j),\quad \hat{S}=(\hat{s}_{jk})_{k=1,...,2N+3}\\
        &\hat{a}_{jk}=(\partial_v(v\psi_k),\phi_j),\quad \hat{A}=(\hat{a}_{jk})_{k=1,...,2N+3}\\
        &\hat{b}_{jk}=(\partial_{v}\psi_k,\phi_j),\quad \hat{B}=(\hat{b}_{jk})_{k=1,...,2N+3}\\
        &\hat{c}_{jk}=(\partial_{vv}\psi_k,\phi_j),\quad \hat{C}=(\hat{c}_{jk})_{k=1,...,2N+3},
    \end{aligned}
\end{equation}
the  variational formulation \eqref{eq:variational_form3} reduces to
\begin{equation}
\label{eq:system2}
    \left(\frac{\hat{S}}{\Delta t}-\hat{A}+bN(t^n)\hat{B}-a(N(t^n))\hat{C}\right)\hat{\mathbf{P}}^{n+1}=\frac{\hat{S}}{\Delta t}\hat{\mathbf{P}}^n.
\end{equation}











\subsection{Stability and the choice of test functions} \label{sec:stability}

The dynamic boundary conditions also give rise to challenges in choosing proper finite-dimensional test function spaces.  As we shall elaborate below, the construction of the trial function is so delicate that we can not simply choose the test functions only out of accuracy. Our goal is to find test functions that result in a stable evolution system in the discrete setting, and we hope the total mass is conserved with satisfactory accuracy. 


%The conservation property is the basic property of the model, and stability is an essential factor in ensuring the success of the numerical scheme. Having proposed the fully discrete numerical scheme in Section \ref{sec:fully_discrete_scheme}, it is necessary to determine an appropriate test function space, as it affects the properties of the numerical solution. 

To this end, two propositions are introduced that relate the test functions to the conservation and stability of the semi-discrete scheme \eqref{eq:variational_form2} in the linear case. However, the spectral method is often not able to completely ensure the conservation of mass, therefore it is not serving as a rigid criterion. The stability of the numerical solution is instead analyzed through its long-term asymptotic behavior in the linear regime, which will be discussed in greater detail below. Following this, three different test function spaces are analyzed respectively.


When analyzing the impact of the test function space, we are to consider the semi-discrete system \eqref{eq:variational_form2}. Thanks to the definition in \eqref{eq:Matrix2}, the system can reduce to
\begin{equation}
    \label{eq:SDE_system2}
    \hat{S}\partial_t\hat{\mathbf{P}}=\left(\hat{A}-\left(bN(t)\right)\hat{B}+\left(a(N(t))\right)\hat{C}\right)\hat{\mathbf{P}},\quad \hat{\mathbf{P}}(0)=\hat{\mathbf{P}}^0,
\end{equation}
where $\hat{\mathbf{P}}=(\hat{u}_1(t),\hat{u}_2(t),...,\hat{u}_{2N+3}(t))^T,\,\hat{\mathbf{P}}^0=(\hat{u}_1(0),\hat{u}_2(0),...,\hat{u}_{2N+3}(0))^T$. For simplicity, we study the case of a linear equation that is $b=0$ and $a(N)=1$. Then the nonlinear system \eqref{eq:SDE_system2} becomes a linear system
\begin{equation}
    \label{eq:linear_system2}
    \hat{S}\partial_t \hat{\mathbf{P}}=(\hat{A}+\hat{C})\hat{\mathbf{P}}, \quad \hat{\mathbf{P}}(0)=\hat{\mathbf{P}}^0.
\end{equation}
Considering the unique solvability of ordinary differential equations, we assume that the matrices $\hat{S}$, $\hat{A}$, and $\hat{C}$ are square matrices of order $2N+3$ and the matrix $\hat{S}$ is invertible. Let $\hat{K}=\hat{S}^{-1}(\hat{A}+\hat{C})$, then the system \eqref{eq:linear_system2} can be rewritten as
\begin{equation}
\label{eq:linear_system3}
    \hat{\mathbf{P}}_t=\hat{K}\hat{\mathbf{P}}.
\end{equation}
Let $\hat{\mathbf{P}}^{\infty}=(\hat{u}_1^{\infty},\hat{u}_2^{\infty},...,\hat{u}_{2N+3}^{\infty})^T$ be the steady-state solution of the equation. It holds that
\begin{equation}
    \label{eq:steady}
    \hat{K}\hat{\mathbf{P}}^{\infty}=0.
\end{equation}
The steady-state equation \eqref{eq:steady} has a nonzero solution if and only if the matrix $\hat{K}$ has at least one zero eigenvalue. With a prescribed test function space, the properties of the scheme can be assessed by inspecting the elements of matrix $\hat{K}$, allowing us to fully characterize the system's behavior. The following propositions serve to elucidate this connection.

\begin{proposition}[mass conservation]\label{prop1}
    Consider the Fokker-Planck equation \eqref{eq:problem2} with $a=1, b=0$ and the semi-discrete scheme \eqref{eq:linear_system3} where the dimension of test function space $V_N$ is $2N+3$. The following relations hold:
\begin{enumerate}
    \item Matrix $\hat{K}$ has zero eigenvalue if and only if the test function space $\mathrm{V}_N$ contains constant functions.
    \item If the matrix $\hat{K}$ has zero eigenvalue, then the total mass of the numerical solution solved by system \eqref{eq:linear_system2} does not change with time. That is
    \begin{equation}
        \int_{V_{\min}}^{V_F}\partial_tp_N(v,t) dv=0,
    \end{equation}
    where $p_N(v,t)$ is defined in \eqref{eq:approximate_solution3}.
\end{enumerate}
\end{proposition}
\begin{proof}
\textbf{Proof of (1)}. 
If the test function space $\mathrm{V}_N$ contains constants, without loss of generality, let $\phi_j=1$. Using the definition in \eqref{eq:Matrix2}, $\forall \varphi_i \in \mathrm{W}_N$, we can derive that
\begin{equation}
    \label{eq:integral}
    \begin{aligned}
        &\int_{V_{\min}}^{V_F}\partial_v(v\varphi_i) dv=v\varphi_i |_{V_{\min}}^{V_R^-}+v\varphi_i |_{V_R^+}^{V_F}=0,\\
        &\int_{V_{\min}}^{V_F}\partial_{vv}\varphi_i dv=\partial_v\varphi_i |_{V_{\min}}^{V_R^-}+\partial_v\varphi_i |_{V_R^+}^{V_F}=0.
    \end{aligned}
\end{equation}
So the elements of the $j$th row of the matrices $A$ and $C$ are all zeros. Since $\hat{S}$ is invertible, $S^{-1}$ is a full-rank matrix. Then
\begin{equation}
    \text{rank}(\hat{K})=\text{rank}(\hat{S}^{-1}(\hat{A}+\hat{C}))=\text{rank}(\hat{A}+\hat{C})<2N+3.
\end{equation}
So $\hat{K}$ has a zero eigenvalue.


If matrix $\hat{K}$ has a zero eigenvalue, then matrix $(\hat{A}+\hat{C})$ has zero eigenvalue for $\hat{S}$ is invertible. Therefore, the matrix $(\hat{A}+\hat{C})$ can make the elements in the jth row all zero through the matrix transformation. Notice that, performing matrix row transformation on matrix $(\hat{A}+\hat{C})$ corresponds to replacing the test function in linear system \eqref{eq:linear_system2}  with the linear combination of the original test function. Without loss of generality, we assume that the elements in the jth row of matrix $(\hat{A}+\hat{C})$ are all zeros and the corresponding test function is $\phi$. Using the definition in \eqref{eq:Matrix2}, $\forall \varphi_i \in \mathrm{W}_N$,  it holds that
\begin{equation}
    \begin{aligned}
        \int_{V_{\min}}^{V_F}\partial_v(v\varphi_i) +\partial_{vv}\varphi_i dv=-\int_{V_{\min}}^{V_F} (v\varphi_i+\partial_v\varphi_i)\partial_v\phi dv=0.
    \end{aligned}
\end{equation}
Since the above formula holds for all $\varphi_i \in \mathrm{W}_N$, so $\partial_v\phi=0$, that is $\phi= \text{constant}$.

\textbf{Proof of (2)}.
From (1), we know that when the matrix $\hat{K}$ has a zero eigenvalue,  the constant function $C_1 \in V_N$. Substituting $\phi_j=1$ into \eqref{eq:variational_form2} with $a=1,b=0$,
\begin{equation}
    \int_{V_{\min}}^{V_F} \partial_t p_N dv-\int_{V_{\min}}^{V_F}(\partial(vp_N)+\partial_{vv}p_N) dv\overset{\eqref{eq:integral}}{=}\partial_t \int_{V_{\min}}^{V_F} p_N dv=0.
\end{equation}
So the total mass does not change over time.
\end{proof}

\begin{proposition}[Stability]\label{prop2}
    Consider the Fokker-Planck equation \eqref{eq:problem2} with $a=1, b=0$ and the semi-discrete scheme \eqref{eq:linear_system3} . A necessary condition for the stability of the method is that all the eigenvalues of the matrix $\hat{K}$ are non-positive.
\end{proposition}
Note that a modified stability criterion is proposed here because the traditional stability conclusion cannot be applied due to the complexity of the equation. In the linear case, the equation has a unique steady state \cite{caceres2011analysis}, and the solution of the equation will converge exponentially to the steady state, so the discretized kinetic equation can only have non-positive eigenvalues. When there are positive eigenvalues, it means that the numerical scheme is unstable.

Following the theoretical analysis, we can now discuss the specific test function space. Our goal is to select suitable test function spaces such that the constructed numerical scheme is stable and preserves the original properties of the Fokker-Planck equation \eqref{eq:problem2} to the greatest extent, such as mass conservation. The Galerkin method is widely used in spectral methods \cite{shen2011spectral}. Consequently, Legendre-Galerkin Method is proposed below.
\paragraph{Legendre-Galerkin Method (LGM)}
The test function space is chosen to be the same as the trial function space. Applied to the semi-discrete method \eqref{eq:variational_form2} or its fully discrete version \eqref{eq:variational_form3} as 
\begin{equation}
    \mathrm{V}_N=\mathrm{W}_N,
\end{equation}
where $\mathrm{V}_N$ is test function space, and $\mathrm{W}_N$ is trial function space defined in \eqref{eq:trial_function}, we obtain a Legendre-Galerkin Method (LGM for short) for the model \eqref{eq:problem1}. 

The LGM method is numerically stable but total mass is not well conserved in dynamics. When constructing the trial function space, some low-order polynomials, especially constants, are discarded in order to satisfy the boundary conditions. For the LGM, the test function space does not contain constants, which fails to ensure mass conservation as stated in Proposition \ref{prop1}. Hence, this method can be used for  finite-time simulations, yet it is unsuitable for capturing long-time behavior or multiscale problems.


 According to Proposition \ref{prop1}, to improve the mass conservation property of the LGM, it seems that we may replace one of the basis functions in  the test function space with the constant function $1$. Say, we may consider the modified test function space
\begin{equation} \label{tildeV}
    \tilde{\mathrm{V}}_N=\mathrm{W}_N-\{\psi_k\} +\{1\},
\end{equation}
where $\{\psi_k\}(k=1,...,2N+3)$ is the basis function of the $\mathrm{W}_N$ space.

In this case, the mass of the numerical solution appears invariant. However, as shown in Figure \ref{fig:stable}, the matrix $\hat{K}$ in \eqref{eq:linear_system3} has positive eigenvalues for some $N$, which makes the method unstable, which agrees with Proposition \ref{prop2}. In fact, Figure \ref{fig:stable} shows that the maximum eigenvalue of the matrix $\hat{K}$ is significantly positive large  for when $N$ is odd and when the modified test function space $\tilde{\mathrm{V}}_N$ is used. Hence, we need to resort to other strategies for enhancing mass conservation. 
\begin{figure}[!htb]
    \centering
        \begin{minipage}[c]{0.49\textwidth}
            \centering
            \includegraphics[width=1\textwidth]{Eig.eps}
        \end{minipage}
        \begin{minipage}[c]{0.49\textwidth}
            \centering
            \includegraphics[width=1\textwidth]{3step.eps}
        \end{minipage}
        \caption{When the test function space is $\tilde{\mathrm{V}}_N$ \eqref{tildeV}, the numerical method might be unstable. Left: The maximum eigenvalue of matrix $\hat{K}$ at different $N$. Right: A typical unstable solution. Equation parameters $a=1, b=0$ with Gaussian initial condition $v_0=-1, \sigma_0^2=0.5$ and $N=11,\Delta t=0.001$.}
        \label{fig:stable}
\end{figure}



\paragraph{Modified Petrov-Galerkin Method (MPGM)} We propose an alternative formulation of the test function space by extending the test function space with one additional basis function $1$. As a consequence, the dimension of the test function space is larger than that of the trial function space, which results in an overdetermined system, and we solve such a system using the Least-Squares method. 

More precisely,  for the semi-discrete method \eqref{eq:variational_form2} or its fully discrete version \eqref{eq:approximate_solution3}, constants are added to  form an augmented test function space
\begin{equation}
    \mathrm{V}_N=\mathrm{W}_N +\{1\}.
\end{equation}
where $\mathrm{W}_N$ is trial function space defined in \eqref{eq:trial_function}, and we thus obtain the modified Petrov-Galerkin Method (MPGM for short).

Note that, the dimension of the test function space is higher than that of the trial function space by $1$. Multiplying \eqref{eq:linear_system2} from the left by $\hat{S}^T$ , the least square solution satisfies
\begin{equation}
    \hat{S}^T\hat{S}\partial_t \hat{\mathbf{P}}=\hat{S}^T(\hat{A}+\hat{C})\hat{\mathbf{P}}.
\end{equation}
The matrix $\hat{K}$ in \eqref{eq:linear_system3} can be written as
\begin{equation}
    \hat{K}=(\hat{S}^T\hat{S})^{-1}(\hat{S}^T(\hat{A}+\hat{C})).
\end{equation}
The numerical solution is not completely mass-conserving due to the use of the least-squares method. But compared with the LGM, the mass of the numerical solution of the MPGM changes very little over time, as shown in Figure \ref{mass}. With extensive tests, the matrix $\hat{K}$ has no positive eigenvalues, and the MPGM is numerically stable.
\begin{figure}[!htb]
    \centering
        \begin{minipage}[c]{0.49\textwidth}
            \centering
            \includegraphics[width=7cm]{mass1.eps}
        \end{minipage}
        \begin{minipage}[c]{0.49\textwidth}
            \centering
            \includegraphics[width=7cm]{mass2.eps}
        \end{minipage}
        \caption{Equation parameters $a=1, b=0$ with Gaussian initial condition $v_0=-1, \sigma_0^2=0.5$ and $N=10,\Delta t=0.001,T_\text{max}=5$. Left: Variation of total mass with time by the LGM. Right: Variation of total mass with time by the MPGM.}
        \label{mass}
\end{figure}



In conclusion, we have proposed two methods, i.e. the LGM and the MPGM, for the model problem \eqref{eq:problem1},  each possessing different advantages and therefore should be used in a flexible manner. The MPGM is preferred for simulating long-time behavior and testing the asymptotic preserving properties of the model, as the mass of the numerical solution from the LGM is significantly diminished in a long time. On the other hand, the LGM can be utilized for $\mathcal O(1)$ time simulations with verifiable order of convergence, and it does not involve the error due to the least square approximation.





%
\section{Case Studies}
\label{sec:case_studies}
In this section, we present a case study of Facebook posts from an Australian public page.
The page shifts between early 2020 (\emph{2019-2020 Australian bushfire season}) and late 2020 (\emph{COVID-19 crises}) from being a moderate-right group for discussion around climate change to a far-right extremist group for conspiracy theories.


\begin{figure*}[!tbp]
	\begin{subfigure}{0.21\textwidth}
		\includegraphics[width=\textwidth]{images/facebook1.png}
		\caption{}
		\label{subfig:first-posting}
		\includegraphics[width=0.9\textwidth]{images/facebook3.jpg}
		\caption{}
		\label{subfig:comment-post-1}
	\end{subfigure}
    \begin{subfigure}{0.28\textwidth}
		\includegraphics[width=\textwidth]{images/facebook2.jpg}
		\caption{}
		\label{subfig:second-posting}
	\end{subfigure}
    \begin{subfigure}{0.23\textwidth}
		\includegraphics[width=\textwidth]{images/facebook4.jpg}
		\caption{}
		\label{subfig:comment-post-2a}
	\end{subfigure}
    \begin{subfigure}{0.23\textwidth}
		\includegraphics[width=\textwidth]{images/facebook5.jpg}
		\caption{}
		\label{subfig:comment-post-2b}
	\end{subfigure}
	\caption{
		Examples of postings and comment threads from a public Facebook page from two periods of time early 2020 (a) and late 2020 (b)-(e), which show a shift from climate change debates to extremist and far-right messaging.
	}
	\label{fig:facebook}
\end{figure*}

We focus on a sample of 2 postings and commenting threads from one Australian Facebook page we classified as ``far-right'' based on the content on the page. 
We have anonymized the users in \Cref{fig:facebook} to avoid re-identification.
The first posting and comment thread (see \Cref{subfig:first-posting}) was collected on Jan 10, 2020, and responded to the Australian bushfire crisis that began in late 2019 and was still ongoing in January 2020. It contains an ambivalent text-based provocation that references disputes in the community regarding the validity of climate change and climate science. 

The second posting and comment thread (see \Cref{subfig:second-posting}) was collected from the same page in September 2020, months after the bushfire crisis had abated.
At that time, a new crisis was energizing and connecting the far-right groups in our dataset --- i.e., the COVID-19 pandemic and the government interventions to curb the spread of the virus. 
The post is different in style compared to the first.
It is image-based instead of text-based and highly emotive, with a photo collage bringing together images of prison inmates with iron masks on their faces (top row) juxtaposed to people wearing face masks during COVID-19 (bottom row). 
The image references the public health orders issued during Melbourne's second lockdown and suggests that being ordered to wear masks is an infringement of citizen rights and freedoms, similar to dehumanizing restraints used on prisoners.

To analyze reactions to the posts, two researchers used a deductive analytical approach to separately code and to analyze the commenting threads --- see \Cref{subfig:comment-post-1} for comments of the first posting, and \Cref{subfig:comment-post-2a,subfig:comment-post-2b} for comments on the second posting. 
Conversations were also inductively coded for emerging themes. 
During the analysis, we observed qualitative differences in the types of content users posted, interactions between commenters, tone and language of debate, linked media shared in the commenting thread, and the opinions expressed.
The rest of this section further details these differences.
To ensure this was not a random occurrence, we tested the exemplar threads against field notes collected on the group during the entire study.
We also used Facebook's search function within pages to find a sample of posts from the same period and which dealt with similar topics. 
After this analysis, we can confidently say that key changes occurred in the group between the bushfire crisis and COVID-19, that we detail next.

\subsubsection*{Exemplar 1 --- climate change skepticism.}
To explore this transformation in more depth, we analyzed comments scraped on the first posting --- \cref{sub@subfig:comment-post-1} shows a small sample of these comments.
The language used was similar to comments that we observed on numerous far-right nationalist pages at the time of the bushfires.
These comments are usually text-based, employing emojis to denote emotions, and sometimes being mocking or provocative in tone. 
Noteworthy for this commenting thread is the 50/50 split in the number of members posting in favor of action on climate change (on one side) and those who posted anti-Greens and anti-climate change science posts and memes (on the other side).
The two sides aligned strongly with political partisanship --- either with Liberal/National coalition (climate change deniers) or Labor/Green (climate change believers) parties. 
This is rather unusual for pages classified as far-right. 

We observed trolling practices between the climate change deniers and believers, which often descend into \emph{flame wars} --- i.e., online ``firefights that take place between disembodied combatants on electronic bulletin boards''~\citep{bukatman1994flame}.
The result is a boosted engagement on the post but also the frustration and confusion of community members and lurkers who came to the discussions to become informed or debate rationally on key differences between the two positions.
They often even become targeted, victimized, and baited by trolls on both sides of the partisan divide. 
The opinions expressed by deniers in commenting sections range from skepticism regarding climate change science to plain denial.
Deniers also regard a range of targets as embroiled in a climate change conspiracy to deceive the public, such as The Greens and their environmental policy, in some cases the government, the United Nations, and climate change celebrities like David Attenborough and Greta Thunberg. 
These figures are blamed for either exaggerating risks of climate change or creating a climate change hoax to increase the influence of the UN on domestic governments or to increase domestic governments' social control over citizens. 

Both coders noted that flame wars between these opposing personas contained very few links to external media. 
Where links were added, they often seemed disconnected from the rest of the conversation and were from users whose profiles suggested they believed in more radical conspiracy theories.
One such example is ``geo-engineering'' (see \cref{sub@subfig:comment-post-1}).
Its adherents believe that solar geo-engineering programs designed to combat climate change are secretly used by a global elite to depopulate the world through sterilization or to control and weaponize the weather.

Nonetheless, apart from the random comments that hijack the thread, redirecting users to external ``alternative'' news sites and Twitter, and the trolls who seem to delight in victimizing unsuspecting victims, the discussion was pretty healthy.
There are many questions, rational inquiries, and debates between users of different political persuasion and views on climate change.
This, however, changes in the span of only a couple of months.

\subsubsection*{Exemplar 2 --- posting and commenting thread.}
We observe a shift in the comment section of the post collected during the second wave of the COVID pandemic (\Cref{sub@subfig:second-posting}) --- which coincided with government laws mandating the public to wear masks and stay at home in Victoria, Australia.
There emerges much more extreme far-right content that converges with anti-vaccination opinions and content.
We also note a much higher prevalence of conspiracy theories often implicating racialized targets.
This is exemplified in the comments on the second post (\Cref{sub@subfig:comment-post-2a,sub@subfig:comment-post-2b}) where Islamophobia and antisemitism are confidently asserted alongside anti-mask rhetoric.
These comments consider face masks similar to the religious head coverings worn by some Muslim women, which users describe as ``oppressive'' and ``silencing''. 
In this way, anti-maskers cast women as a distinct, sympathetic marginalized demographic.
However, this is enacted alongside the racialization and demonization of Islam as an oppressive religion. 

Given the extreme racialization of anti-mask rhetoric, some commenters contest these positions, arguing that the page is becoming less an anti-Scott Morrison page (Australia's Prime Minister at the time) and changing into a page that harbors ``far-right dickheads''.
This questioning is actively challenged by far-right commenters and conspiracy theorists on the page, who regarded pro-mask users and the Scott Morrison government as ``puppets'' being manipulated by higher forces (see \Cref{sub@subfig:comment-post-2b}). 

This indicates a significant change on the page's membership towards the extreme-right, who employs more extreme forms of racialized imagery, with more extreme opinion being shared.
Conspiracy theorists become more active and vocal, and they consistently challenge the opinions of both center conservative and left-leaning users. 
This is evident in the final two comments in \Cref{subfig:comment-post-2b}, which reflect QAnon style conspiracy theories and language.
Public health orders to wear masks are being connected to a conspiracy that all of these decisions are directed by a secret network of global Jewish elites, who manipulate the pandemic to increase their power and control. 
This rhetoric intersects with the contemporary ``QAnon'' conspiracy theory, which evolved from the ``Pizzagate'' conspiracy theory.
They also heavily draw on well-established antisemitic blood libel conspiracy theories, which foster beliefs that a powerful global elite is controlling the decisions of organizations such as WHO and are responsible for the vaccine rollout and public health orders related to the pandemic.
The QAnon conspiracy is also influenced by Bill Gates' Microchips conspiracy theory, i.e., the theory that the WHO and the Bill Gates Foundation global vaccine programs are used to inject tracking microchips into people.

These conspiracy theories have, since COVID-19, connected formerly separate communities and discourses, uniting existing anti-vaxxer communities, older demographics who are mistrustful of technology, far-right communities suspicious of global and national left-wing agendas, communities protesting against 5G mobile networks (for fear that they will brainwash, control, or harm people), as well as generating its own followers out of those anxious during the 2020 onset of the COVID-19 pandemic.
We detect and describe some of these opinion dynamics in the next section.

\section{NNLIF with learning rules}\label{sec:lr}
In this Section, we consider the NNLIF model with a learning rule which is an extension of the Fokker-Planck equation \eqref{eq:problem1}, involving synaptic weights and the Hebbian learning rule. This is a novel and intriguing model and the dynamics of the membrane potential $v$ and the synaptic weight $w$ are on different time scales, making numerical simulation far more challenging. In order to better understand this model and verify the generality of the method proposed in Section \ref{sec:scheme}, we further explore this model from a numerical perspective. 

\subsection{Model introduction}


Compared with the simplest form of NNLIF model, the NNLIF model with learning rules introduces a new variable, the synaptic weight $w$, which is also the connectivity of the network $b$ mentioned in \eqref{eq:ha}. Furthermore, an external input function $I(w,t)$ is added to the drift coefficient $h$
\begin{equation}
    h(w,N(t))=-v+I(w,t)+w\sigma(N(t)).
\end{equation}
The function $\sigma(\cdot)$ represents the response of the network to the total activity, usually taking $\sigma(N)=N$. Then the Fokker-Planck equation without learning rules can be written as 
\begin{equation}
    \label{eq:problem3}
    \begin{cases}
        \partial_{t}p+\partial_{v}((-v+I(w,t)+w\sigma(\bar{N}(t)))p)-a\partial_{v v}p=0,\qquad v\in(-\infty,V_F]/\{V_R\},\\
        p(v,w,0)=p^0(v,w),\qquad p(-\infty,w,t)=p(V_F,w,t)=0,\\
        p(V^-_R,w,t)=p(V^+_R,w,t),\quad \partial _vp(V^-_R,w,t)=\partial _vp(V^+_R,w,t)+\frac{N(w,t)}{a},\\
    \end{cases}
\end{equation}
where, $p(v,w,t)$ describes the probability of finding a neuron at voltage $v$, synaptic weight $w$ and given time $t$. The diffusion coefficient $a$ is the same as in \eqref{eq:ha}. The subnetwork activity $N(w, t)$ and total activity
$\bar{N}(t)$ are defined as
\begin{equation}
    N(w,t)=-a\frac{\partial p}{\partial v}(V_F,w,t)\geq 0,\quad \bar{N}(t)=\int_{-\infty}^{+\infty}N(w,t)dw.
\end{equation}
Then we define the probability density of finding a neuron at synaptic weight $w$ and given time $t$ by 
\begin{equation}
    H(w, t)=\int_{-\infty}^{V_{F}} p(v, w, t) d v, \quad \int_{-\infty}^{\infty} H(w, t) d w=1 .
\end{equation}


In this case of no learning rule, the function $H(w,t)$ is time-independent because the distribution of synaptic weights in \eqref{eq:problem3} is fixed. The input signal $I(w,t)$ can be reflected by an output signal related to network activity $N(w,t)$. Next, we employ the learning rules of \cite{perthame2017distributed} to modulate the distribution of synaptic weights $H$, enabling the network to discriminate specific input signals $I$ by choosing an apposite synaptic weight distribution $H$ that is adapted to the signal $I$.


In \cite{perthame2017distributed}, the authors choose learning rules inspired by the seminal Hebbian rule and assume synaptic weights described with a single parameter $w$ and the subnetworks interact only via the total rate $\bar{N}$. They elucidate that all subnetworks parameterized by $w$ can vary their intrinsic synaptic weights $w$ according to a function $\Phi$ that is based on the intrinsic activity $N(w)$ of the network and the total activity of the network $\bar{N}$. Then, they give the generalization choice of Hebbian rule
\begin{equation}
    \Phi(N(w), \bar{N})=\bar{N} N(w) K(w),
\end{equation}
where $K(\cdot)$ represents the learning strength of the subnetwork with synaptic weight $w$. Adding the above choice of learning rule, the Fokker-Planck equation with learning rules is given by
\begin{equation}
    \label{eq:problem40}
    \frac{\partial p}{\partial t}+\frac{\partial}{\partial v}[(-v+I(w,t)+w \sigma(\bar{N}(t))) p]+\varepsilon \frac{\partial}{\partial w}[(\Phi-w) p]-a \frac{\partial^{2} p}{\partial v^{2}}=N(w, t) \delta\left(v-V_{R}\right).
\end{equation}
In order to better apply the numerical scheme and study the learning behavior of the model, we consider the equation \eqref{eq:problem40} for time rescaling $t \rightarrow t / \varepsilon$ and convert $\delta$-function to dynamic boundary condition such as:
\begin{equation}
    \label{eq:problem4}
    \begin{cases}
        \displaystyle
        \frac{\partial p}{\partial t}+\frac{\partial}{\partial w}[(\bar{N}(t)N(w,t)K(w)-w)p]
        =\frac{1}{\varepsilon}\left\{a\frac{\partial^2p}{\partial v^2}-\frac{\partial}{\partial v}[(-v+I(w,t)+w\sigma(\bar{N}(t)))p]\right\},\\
        p(v,w,0)=p^0(v,w),p(V_F,w,t)=p(-\infty,w,t)=p(v,\pm \infty,t)=0,\\
        p(V_R^-,w,t)=p(V_R^+,w,t),\qquad \frac{\partial}{\partial v}p(V^-_R,w,t)=\frac{\partial}{\partial v}p(V^+_R,w,t)+\frac{N(w,t)}{a}.
    \end{cases}
\end{equation}

Here, $p^0(v,w)$ is initial condition and the probability density
function p(v, t) should satisfy the condition of conservation of mass
\begin{equation}
    \int_{-\infty}^{\infty} \int_{-\infty}^{V_{F}} p(v, w, t) d v d w=\int_{-\infty}^{\infty} \int_{-\infty}^{V_{F}} p^{0}(v, w) d v d w=1.
\end{equation}



Despite some research on model \eqref{eq:problem4} as indicated by the theoretical properties presented in \cite{perthame2017distributed} and the numerical analysis and experiments in \cite{he2022structure}, it is still a relatively new model with limited established knowledge. In this paper, the numerical method proposed in Section \ref{sec:scheme} is used to further investigate the learning behaviors of this model numerically.

\subsection{Numerical scheme}
Now, we describe the numerical scheme for \eqref{eq:problem4}. We choose the calculation interval as $[V_{\text{min}},V_F]\times [W_{\text{min}},W_{\text{max}}]\times [0,T_{\text{max}}]$ and suppose the density function is practically negligible out of this region.
We use spectral methods for v-wise discretization and Differential method for w-wise and t-wise discretization. So we divide the interval $[W_{\text{min}},W_{\text{max}}],[0,T_{\text{max}}]$ into $n_w,n_t$ equal sub-intervals with size
\begin{equation}
    \Delta w=\frac{W_{\text{max}}-W_{\text{min}}}{n_w},\Delta t=\frac{T_{\text{max}}}{n_t}.
\end{equation}
Then the grid points can be represented as follows
\begin{equation}
    \begin{aligned}
        &w_{j}=W_{\text{min} }+j \Delta w, & j=0,1,2, \cdots, n_{w} \\
        &t^{n}=n \Delta t, & n=0,1,2, \cdots, n_{t}
    \end{aligned}
\end{equation}
For the v-direction discretization, we take the same scheme as in Section \ref{sec:fully_discrete_scheme}. The approximate solution is expended as 
\begin{equation}
    \label{eq:approximate_solution4}
    p_N(v,w,t)=\sum_{k=1}^{2N+3}\hat{u}_k(w,t)\psi_k(v).
\end{equation}
The initial condition for the expansion coefficients $\{\hat{u}_{k}(w,0)\}_{k=0}^{2N+3}$ can be obtained by the least square approximation,
\begin{equation}
    \label{eq:initial_vector3}
    \int_{V_{\min}}^{V_F} \sum_{k=1}^{2N+3}\hat{u}_k(w_j,0)\psi_k(v) \phi_i dv=\int_{V_{\min}}^{V_F} p^0(w_j,v)\phi_i dv, \quad j=0,1,2, \cdots, n_{w} \quad \forall \phi_i \in \mathrm{V}_N.
\end{equation}
From the properties of the basis functions \eqref{eq:lambda}, subnetwork activity $N(w,t)$ can be expressed as
\begin{equation}
    N^n_j=N(w_j,t^n)=-a\hat{u}_{2N+3}(w_j,t^n).
\end{equation}
And we apply the simplest rectangular numerical integration rule to discretize the total activity $\bar{N}(t)$
\begin{equation}
    \bar{N}^n=\Delta w \sum_{j=0}^{n_w}N^n_j.
\end{equation}
For the w-direction discretization, we inherit the idea form \cite{he2022structure} which takes the following explicit flux construction adapted from Godunov's Method 
 \begin{equation}
    \Phi_{i, j+\frac{1}{2}}^{n}=
        \begin{cases}
            \begin{cases}
                \min \left\{\Phi_{i, j}^{n}, \Phi_{i, j+1}^{n}\right\} \qquad &\hat{P}_{i, j}^{n} \leq \hat{P}_{i, j+1}^{n} \\
                \max \left\{\Phi_{i, j}^{n}, \Phi_{i, j+1}^{n}\right\}  &\hat{P}_{i, j}^{n}>\hat{P}_{i, j+1}^{n} \\
            \end{cases}&j=0, \cdots, n_{w}-1\\
            0  &j=-1, n_{w}
        \end{cases}
    \end{equation}
where
\begin{equation}
    \Phi_{i, j}^{n}=\left(\bar{N}^{n} N_{j}^{n} K\left(w_{j}\right)-w_{j}\right) \hat{P}_{i, j}^{n} \quad \text { for } \quad j=0, \cdots, n_{w}.
\end{equation}
$\hat{P}_{i,j}^n$ is the coefficients of the basis functions in \eqref{eq:approximate_solution4}
\begin{equation}
    \hat{P}_{i,j}^n=\hat{u}_i(w_j,t^n).
\end{equation}
Define 
\begin{equation}
\begin{aligned}
    &p_{N,j}^{n}=\sum_{k=1}^{2N+3}\hat{u}_k(w_j,t^n)\psi_k(v),\\
    &q_{N,j+\frac{1}{2}}^n=\sum_{k=1}^{2N+3} \Phi_{k,j+\frac{1}{2}}^{n}\psi_k(v).
\end{aligned}
\end{equation}
After using a semi-implicit method for time discretization, we obtain the fully discrete scheme as follows:
\begin{equation}
    \frac{p_{N,j}^{n+1}-p_{N,j}^{n}}{\Delta t}+\frac{q_{N,j+\frac{1}{2}}^n-q_{N,j-\frac{1}{2}}^n}{\Delta w}=\frac{1}{\varepsilon}\left\{a\frac{\partial^2p_{N,j}^{n+1}}{\partial v^2}-\frac{\partial}{\partial v}\left[(-v+I(w_j)+w_j\sigma(\bar{N}(t^n)))p_{N,j}^{n+1}\right]\right\}.
\end{equation}
When the test function space $\mathrm{V}_N$ is given, the coefficients of the approximate solution \eqref{eq:approximate_solution4} for each $t$ and $w$ step can be obtained by the following linear system
\begin{equation}
    \begin{aligned}
        &\frac{\hat{S}(\hat{\mathbf{P}}^{n+1}_{j}-\hat{\mathbf{P}}^n_{j})}{\Delta t}+\frac{\hat{S}(\mathbf{\Phi}_{j+\frac{1}{2}}^n-\mathbf{\Phi}_{j-\frac{1}{2}}^n)}{\Delta w}\\
        +&\frac{1}{\varepsilon}\left\{-\hat{A}\hat{\mathbf{P}}^{n+1}_{j}+\left(I(w_{j},t^n)+w_{j}\sigma(\bar{N}(t^n))\right)\hat{B}\hat{\mathbf{P}}^{n+1}_{j}-a\hat{C}\hat{\mathbf{P}}^{n+1}_{j}\right\}=0,
    \end{aligned}
\end{equation}
where 
\begin{equation}
    \begin{aligned}
    &\hat{\mathbf{P}}^n_j=\left(\hat{u}_1(w_j,t^n),\hat{u}_2(w_j,t^n),...,\hat{u}_{2N+3}(w_j,t^n)\right)^T,\\
        &\mathbf{\Phi}_{j+\frac{1}{2}}^n=(\Phi_{1, j+\frac{1}{2}}^{n},\Phi_{2, j+\frac{1}{2}}^{n},...,\Phi_{2N+3, j+\frac{1}{2}}^{n})^T,
    \end{aligned}
\end{equation}
 and the matrix $\hat{S},\hat{A},\hat{B},\hat{C}$ are defined in \eqref{eq:Matrix2}.

 
This numerical scheme is conserved naturally in the $w$ direction, however, strict conservation of mass in the $v$ direction is not achieved when the test function space is selected based on Section \ref{sec:stability}. When $\varepsilon$ is small enough, the asymptotic preserving properties of the model can only be verified through the use of MPGM.

 
\section{Numerical test} \label{sec:numerical_test}

In this section, we give  numerical tests to verify the properties of the proposed schemes and demonstrate some explorations of the model. Numerical solutions for the initial three subsections are obtained by LGM; results for the MPGM approach are similar except for Section \ref{sec:Convergence}, which are thus omitted, and numerical solutions for Section \ref{sec:Learning_testing} are obtained by MPGM, as variations in the time scale require the scheme to be asymptotic preserving.

The tests are structured as follows. In Section \ref{sec:Convergence}, the convergence order of the method is tested in both the NNLIF model and the NNLIF model with learning rules. In Section \ref{sec:time_saving}, we validate the efficiency of the spectral method by comparing it to existing methods. In Section \ref{sec:blow_up}, we test a few properties of the NNLIF model. In Section \ref{sec:Learning_testing}, we test the learning and discrimination abilities of NNLIF model with learning rules for the periodic input function.


\subsection{Order of accuracy}\label{sec:Convergence}
In this part, we test the order of accuracy of the proposed scheme based on the NNLIF model and the NNLIF model with learning rules. Since the exact solution is unavailable, we choose the numerical solution $p_e$ of the finite difference method \cite{hu2021structure} with sufficient accuracy to replace the exact solution.



For NNLIF model \eqref{eq:problem2}, we choose $V_F=2,V_R=1,V_{\text{min}}=-4,a=1, b=3$ and the Gaussian distribution
\begin{equation}
    p_G(v)=\frac{1}{\sqrt{2 \pi} \sigma_{0} M_0} e^{-\frac{\left(v-v_{0}\right)^{2}}{2 \sigma_{0}^{2}}},
\end{equation}
as the initial condition with $v_0=-1$ and $\sigma_0^2=0.5$, $M_0$ is a normalization factor such that
\begin{equation}
    \int_{V_{\text{min}}}^{V_F} p_G(v) dv=1.
\end{equation}
 The numerical solution is computed till time $t=0.2$. Errors in both $L^{\infty}$ and $L^2$ norm are examined with fixed $N=12$ and different $\Delta t$ in Table \ref{convergence1}. It should be noted that the number of basis functions is not $N$, but rather $2N+3$, as shown in equation \eqref{eq:approximate_solution3}. 
\begin{table}[!htb]
	\centering
	\begin{tabularx}{10cm}{ccccc}
	\toprule
	$\Delta t$ & $\left\| p_N-p_e \right\|_{L^{\infty}}$& {$O_{\tau,L^{\infty}}$}& $\left\| p_N-p_e \right\|_{L^2}$& {$O_{\tau,L^2}$}\\ 
	\midrule
	0.04 & 3.880E-03 &0.9520 & 1.868E-03 &0.9508 \\
	0.02 & 2.005E-03 & 0.9792 & 9.662E-04 &0.9617 \\
	0.01 & 1.017E-03 & 0.9926 &4.961E-06&0.9287 \\
    0.005 & 5.111E-04 & - &2.606E-04& - \\
	\bottomrule
    \end{tabularx}
    \caption{Error and order of accuracy of the proposed numerical scheme for NNLIF model with different temporal sizes. The parameter $N$ is fixed as $N=12$.}
    \label{convergence1}
\end{table}

For the order of accuracy in the $v$ direction, we choose the time step size $\Delta t=10^{-5}$. Errors in the $L^2$ norm are examined with different $N$. The logarithm of the error versus $N$ is plotted in Figure \ref{convergence2}. We remark that when testing the order of spatial convergence, the results present a zig-zag decreasing profile as $N$ increases, which is a common phenomenon for spectral methods. We thus plot the errors for odd and even numbers of $N$, respectively. For each scenario, we clearly observe the spectral convergence as the number of spatial basis functions increases.


\begin{figure}[!htb]
    \centering
        \begin{minipage}[c]{0.49\textwidth}
            \centering
            \includegraphics[width=7cm]{O_N1.eps}
        \end{minipage}
        \begin{minipage}[c]{0.49\textwidth}
            \centering
            \includegraphics[width=7cm]{O_N2.eps}
        \end{minipage}
		\caption{Logarithm of the error of the proposed numerical scheme for NNLIF model with learning rules with different $N$. The temporal size is fixed as $\Delta t=10^{-5}$. Left: $N$ is odd; Right: $N$ is even.}
  \label{convergence2}
\end{figure}


For NNLIF with learning rules model \eqref{eq:problem4}, we choose $V_F=2,V_R=1,V_{\text{min}}=-4,a=1,\varepsilon=0.5,W_{\text{min}}=-1.1,W_{\text{max}}=0.1,\sigma(\bar{N})=\bar{N}, I(w)=0$ and initial condition
\begin{equation}
    p_{\text{init}}=\begin{cases}
        \frac{1}{\sqrt{2 \pi} \sigma_{0} } e^{-\frac{\left(v-v_{0}\right)^{2}}{2 \sigma_{0}^{2}}}\text{sin}^2(\pi w) \qquad &-1<w<0,\\
        0 &\text{otherwise},
    \end{cases}
\end{equation}
with $v_0=-1$ and $\sigma_0^2=0.5$.

The numerical solution is computed till time $t=0.1$. For $t$ direction and $w$ direction, we fix $\frac{\Delta w}{\Delta t}=1, N=16$. Considering that both the $t$ direction and the $w$ direction are theoretically first-order accurate, as well as the stability factor, it is reasonable to jointly test the order of accuracy. Errors in both $L^{1}$ and $L^2$ norm are examined with different $\Delta t$ and $\Delta w$ in Table \ref{convergence3}. For $v$ direction, we fix ${\Delta w}={\Delta t}=10^{-5}$. Errors in the $L^2$ norm are examined with different $N$. The logarithm of the error versus $N$ is plotted in Figure \ref{convergence4}.

\begin{table}[!htb]
	\centering
	\begin{tabularx}{10cm}{cccccc}
	\toprule
	$\Delta t$ &$\Delta w$ & $\left\| p_N-p_e \right\|_{L^{1}}$& $O_{\tau,L^{1}}$& $\left\| p_N-p_e \right\|_{L^2}$& {$O_{\tau,L^2}$}\\ 
	\midrule
	0.02 &0.02 & 1.599E-03 &1.04 & 3.234E-03 &1.07 \\
	0.01 &0.01 & 7.755E-04 & 0.99 & 1.536E-03 &0.97 \\
	0.005 &0.005 &3.893E-04 & 1.01 &7.812E-04&1.05 \\
    0.0025 & 0.0025 & 1.926E-04 & - &3.757E-04& - \\
	\bottomrule
    \end{tabularx}
    \caption{Error and order of accuracy of the proposed numerical scheme for NNLIF model with learning rules with different ${\Delta w}$ and ${\Delta t}$. The parameter $N$ is fixed as $N=16$.}
    \label{convergence3}
\end{table}
\begin{figure}[!htb]
    \centering
        \begin{minipage}[c]{0.49\textwidth}
            \centering
            \includegraphics[width=7cm]{O2_N1.eps}
        \end{minipage}
        \begin{minipage}[c]{0.49\textwidth}
            \centering
            \includegraphics[width=7cm]{O2_N2.eps}
        \end{minipage}
		\caption{Logarithm of the error of the proposed numerical scheme for NNLIF model with learning rules with different $N$. The temporal size is fixed as $\Delta t=10^{-5}$. Left: $N$ is odd; Right: $N$ is even}
  \label{convergence4}
\end{figure}

The results indicate that the scheme shows first-order accuracy in time and exponential convergence in space for the NNLIF model; first-order accuracy in the $w$, $t$ direction and exponential convergence in the v direction for the NNLIF model with learning rules.






\subsection{Simulation time comparison}\label{sec:time_saving}
In this part, we compare the CPU time between the proposed spectral method and the finite difference method \cite{hu2021structure}, to show that our scheme has a significant computational time advantage with the same level of accuracy.


We choose NNLIF model with parameters $a=1,b=0.5,\Delta t=5\times 10^{-4}$ and the Gaussian initial condition with $v_0=0,\sigma_0^2=0.25$. The numerical solution is computed till time $t=0.5$. The results of the spectral method and the finite difference method are shown in Table \ref{error1} and Table \ref{error2}.
\begin{table}[!htb]
	\centering
	\begin{tabularx}{8cm}{cccc}
	\toprule
	$N$& $\left\| \cdot \right\|_{\infty}$&$\left\| \cdot \right\|_{1}$& CPU Time (s) \\ 
	\midrule
	5 & 5.15e-02 & 1.58e-02& 0.026 \\
    10 & 3.33e-03 & 2.85e-04 & 0.030  \\
    15 & 9.71e-05 & 2.51e-05& 0.053 \\
    20 & 1.30e-06 & 3.76e-07 & 0.071  \\
	\bottomrule
    \end{tabularx}
    \caption{Errors using the spectral method with different numbers of basis functions.}
    \label{error1}
\end{table}

\begin{table}[!htb]
	\centering
	\begin{tabularx}{8cm}{cccc}
	\toprule
	$h$& $\left\| \cdot \right\|_{\infty}$&$\left\| \cdot \right\|_{1}$& CPU Time (s) \\ 
	\midrule
    ${1/4}$ & 3.01e-03 & 7.09e-04& 0.031 \\
    ${1/8}$ & 9.69e-04 & 2.18e-04& 0.073 \\
    ${1/16}$ & 2.79e-04 & 6.18e-05 & 0.157  \\
    ${1/32}$ & 7.54e-05 & 1.64e-05& 0.348 \\
    ${1/64}$ & 1.97e-05 & 4.21e-06 & 2.801  \\
    ${1/128}$ & 4.41e-06 & 1.12e-06 & 11.971  \\
	\bottomrule
    \end{tabularx}
    \caption{Errors using the finite difference method  with different spatial grid sizes}
    \label{error2}
\end{table}
These tables clearly indicate that to achieve the same level of accuracy, the spectral method is more efficient in terms of the simulation time, and the advantage is more noticeable when the accuracy level is higher.



\subsection{Global solution and blow-up in NNLIF model}\label{sec:blow_up}
\subsubsection{Blow up}


In \cite{caceres2011analysis}, the authors find the solution may blow up in finite time with the suitable initial conditions for the excitatory network. They show that whenever the value of $b>0$ is, if the initial data is concentrated enough around $v=V_F$, then the defined weak solution in Definition 2.1 of \cite{caceres2011analysis} does not exist for all times. Figure \ref{fig:blowup1} and Figure \ref{fig:blowup2} show this phenomenon. It can be seen that when the blow-up phenomenon is about to occur, the density function $p(v,t)$ is increasingly concentrated and sharp at reset point $V_R$ and the firing rate $N(t)$ is growing rapidly. 
\begin{figure}[!htb]
    \centering
        \begin{minipage}[c]{0.49\textwidth}
            \centering
            \includegraphics[width=7cm]{blowupNt1.eps}
        \end{minipage}
        \begin{minipage}[c]{0.49\textwidth}
            \centering
            \includegraphics[width=7cm]{blowup1.eps}
        \end{minipage}
        \caption{Equation parameters $a=1, b=3$ with Gaussian initial condition $v_0=-1, \sigma_0^2=0.5$. Left: evolution of firing rate $N(t)$. Right: density function $p(v, t)$ at $t=2.95,3.15,3.35$.}
        \label{fig:blowup1}
\end{figure}
\begin{figure}[!htb]
        \begin{minipage}[c]{0.49\textwidth}
            \centering
            \includegraphics[width=7cm]{blowupNt2.eps}
        \end{minipage}
        \begin{minipage}[c]{0.49\textwidth}
            \centering
            \includegraphics[width=7cm]{blowup2.eps}
        \end{minipage}
        \caption{Equation parameters $a=1, b=1.5$ with Gaussian initial condition $v_0=1.5, \sigma_0^2=0.005$.Left: evolution of firing rate $N(t)$. Right: density function $p(v, t)$ at $t=0.0325,0.0365,0.0405$.}
        \label{fig:blowup2}
\end{figure}


For spectral methods, the approximate solution of the density function is dependent on the coefficients of the basis functions. We aim to further investigate how the coefficients change when the blow-up phenomenon is about to occur. We choose $a=1, b=1.5$ in equation and $N=20,\Delta t=10^{-5}$.
\begin{figure}[!htb]
    \centering
        \begin{minipage}[c]{0.49\textwidth}
            \centering
            \includegraphics[width=7cm]{coeff.eps}
        \end{minipage}
        \begin{minipage}[c]{0.49\textwidth}
            \centering
            \includegraphics[width=7cm]{coeff2.eps}
        \end{minipage}
    \caption{Changes of the coefficients of the first few terms in the expansion formula \eqref{eq:approximate_solution3} during blow up. Left: evolution of the coefficients $\{p_k,f_k\}_{k=0}^2$. Right:evolution of the coefficients $\{\lambda_k\}_{k=1}^3$.  }
    \label{fig:coeff}
\end{figure}


Recall that
\begin{equation}
    \lambda_3(t)=\partial_vp(V_F,t)=-\frac{N(t)}{a},\qquad \partial_vp(V_R^+,t)=\lambda_2(t)+\lambda_3(t).
\end{equation}
Therefore, $\lambda_2$ and $\lambda_3$ are directly influenced by the firing rate. Due to the use of global basis functions, as the firing rate $N(t)$ increases, all the basis functions are affected. In response to the change of $\lambda_3$, $\lambda_2$ and the coefficients of the basis functions in $\mathrm{W}_2$ change accordingly, respectively controlling the derivative value on both sides of point $V_R$ and the function value in the interval. Figure \ref{fig:coeff} show the change of coefficients $\{p_k,f_k\}_{k=0}^2$, $\{\lambda_k\}_{k=1}^3$ in \eqref{eq:approximate_solution3} as time involves. It can be seen from the figure that the changes in $\lambda_2$ and $\lambda_3$ are most obvious, while the coefficients of all basis functions in $\mathrm{W}_2$ space are affected but the changes are relatively small.



\subsubsection{Relative entropy}

As we have mentioned, since little is known about the properties of the solutions of the Fokker-Planck equation \eqref{eq:problem1}, there  is a lack of complete understanding of the long-time asymptotic behavior in the continuous case. In \cite{caceres2011analysis}, they studied relative entropy theory for linear problem $a_1=b=0$, which implies exponential convergence to equilibrium. The relative entropy is given by
\begin{equation}
    I_e=\int_{-\infty}^{V_{F}} G\left(\frac{p(v, t)}{p^{\infty}(v)}\right) p_{\infty}(v) d v,
\end{equation}
which can be shown to be decreasing in time, where $G(\cdot)$ is a smooth convex function and $p^{\infty}(v)$ represents the stationary solution. In this part, we numerically verify the relative entropy theory. The numerical relative entropy is given by
\begin{equation}
    S(t)=\int_{V_L}^{V_{F}} G\left(\frac{p_N(v, t)}{p^{\infty}(v)}\right) p_{\infty}(v) d v.
\end{equation}

We consider nonlinear cases with $a_0=1,a_1=0,b=-0.5$ and $a_0=1,a_1=0.1,b=0$. We choose the numerical solution of a sufficiently long time as the stationary solution $p^{\infty}(v)$ and the Gaussian initial condition $v_0=-1, \sigma_0^2=0.5$. Figure \ref{fig:relative_entropy2} \ref{fig:relative_entropy3} show the time evolution of the firing rate and the numerical relative entropy for these cases.
\begin{figure}[!htb]
    \centering
    \begin{minipage}[c]{0.49\textwidth}
        \centering
        \includegraphics[width=1\textwidth]{Nt2.eps}
    \end{minipage}
    \begin{minipage}[c]{0.49\textwidth}
        \centering
        \includegraphics[width=1\textwidth]{relative2.eps}
    \end{minipage}
    \caption{Equation parameters $a=1,b=-0.5$ with Gaussian initial condition $v_0=-1, \sigma_0^2=0.5$. Left: evolution of firing rate $N(t)$. Right: evolution of relative entropy $S(t)$ with $G(x)=\frac{(x-1)^2}{2}$.}
     \label{fig:relative_entropy2}
\end{figure}
\begin{figure}[!htb]
    \centering
    \begin{minipage}[c]{0.49\textwidth}
        \centering
        \includegraphics[width=1\textwidth]{Nt3.eps}
    \end{minipage}
    \begin{minipage}[c]{0.49\textwidth}
        \centering
        \includegraphics[width=1\textwidth]{relative3.eps}
    \end{minipage}
    \caption{Equation parameters $a_0=1,a_1=0.1,b=0$ with Gaussian initial condition $v_0=-1, \sigma_0^2=0.5$. Left: evolution of firing rate $N(t)$. Right: evolution of relative entropy $S(t)$ with $G(x)=\frac{(x-1)^2}{2}$.}
     \label{fig:relative_entropy3}
\end{figure}


As shown in \cite{caceres2011analysis}, there may be two stationary solutions for the system of $b>0$. For example, when $ a(N(t)) = 1$ and $b =1.5$, there are two different steady states whose firing rates are $N^{\infty}=2.319$ and $N^{\infty}=0.1924$. Given the firing rate $N^{\infty}$, the expression of $p^{\infty}(v)$is given by
\begin{equation}
    p^{\infty}(v)=\frac{N^{\infty}}{a\left(N^{\infty}\right)} e^{-\frac{h\left(v, N^{\infty}\right)^{2}}{2 a\left(N^{\infty}\right)}} \int_{\max \left\{v, V_{R}\right\}}^{V_{F}} e^{\frac{h\left(\omega, N^{\infty}\right)^{2}}{2 a\left(N^{\infty}\right)}} d \omega ,
\end{equation}
which is the stationary solution when we calculate the relative entropy for multiple steady-state problems. The results are shown in Figure \ref{fig:relative_entropy4}, where the steady state with a larger firing rate $N^{\infty}=2.319$ is unstable while the stationary solution with a lower firing rate $N^{\infty}=0.1915$ is stable. We see that the relative entropy decreases with time for the stable state, while the other one does not.
\begin{figure}[!htb]
    \centering
    \begin{minipage}[c]{0.49\textwidth}
        \centering
        \includegraphics[width=1\textwidth]{st1.eps}
    \end{minipage}
    \begin{minipage}[c]{0.49\textwidth}
        \centering
        \includegraphics[width=1\textwidth]{st2.eps}
    \end{minipage}
    \caption{Equation parameters $a=1,b=1.5$  with Gaussian initial condition $v_0=-1, \sigma_0^2=0.5$. In this case, the model has two stationary states with firing rates $N^{\infty}=0.1924$ and $N^{\infty}=2.319$. Left: evolution of relative entropy $S(t)$ with $G(x)=\frac{(x-1)^2}{2}$ for stable state with $N^{\infty}=0.1924$. Right: evolution of relative entropy $S(t)$ with $G(x)=\frac{(x-1)^2}{2}$ for unstable state with $N^{\infty}=2.319$ }
     \label{fig:relative_entropy4}
\end{figure}



\subsection{Learning and testing in NNLIF model with learning rules}\label{sec:Learning_testing}
In this part, we consider the learning and discrimination abilities in NNLIF model with learning rules. In \cite{perthame2017distributed}, the authors proposed a two-phase test to illustrate the discrimination property:\medskip


\textbf{Learning phase}
\smallskip

1. An heterogeneous input $I(w)$ is presented to the system, when the learning process is active. The initial data is supported on inhibitory weights and the learning rule is determined for the present weights by $-N(w)\bar{N}$ by taking $K(w) = -1$ if $w \leq 0$.


2. After some time, the synaptic weight distribution $H(w, t)$ converges to an equilibrium distribution $H^*_
I(w)$, which depends on $I$.\medskip



\textbf{Testing phase}
\smallskip

1. The learning process is now switched off, i.e. there is no w-direction convection, and a new input $J(w)$ is presented to the system.


2. After some time, the solution $p_J (v,w, t)$ reaches an equilibrium $p^*_J (v,w)$, which is characterized  by the output signal $N^*_J(w)$ which is the neural activity distribution across the heterogeneous populations.\medskip


Some numerical explorations of the learning behavior and discriminative properties of the model have been done in \cite{perthame2017distributed}\cite{he2022structure}. When the learning phase is over, in addition to the synaptic weight distribution $H(w, t)$, the equilibrium state $N_I(w)$ of the sub-network activity $N(w,t)$ can also be obtained, which we call the \textbf{prediction signal}. In the previous work on the time-independent input function $I(w)$ for the learning phase \cite{perthame2017distributed}\cite{he2022structure}, the prediction signal $N_I(w)$ is like a triangle depending on the input function $I(w)$ of the learning phase.  After the testing phase when the learning input $I(w)$ and testing input $J(w)$ are the same, the output signal $N_J^*(w)$ is like a triangle that matches the prediction signal $N_I(w)$; but when $I(w)$ and $J(w)$ are different, the output signal is not in a regular shape. 


They explore learning and discriminative power in the model only if the input function is constant in time. In our work, we plan to explore how the model would react to a time-varying input signal through numerical experiments, and there have been studies in the field of neuroscience surrounding time-varying input \cite{isidori1990output}. Especially, we consider input functions that are time-periodic and explore the effect of oscillation periods on the learning ability of the model. To this aim, we have designed $4$ sets of experiments, progressively revealing the nature of its learning behavior.

\paragraph{Test 1. Synchronizing with oscillating inputs.}
We choose the testing input functions
\begin{equation}
    \begin{aligned}
        I_{1}&=\pi^{-\frac{1}{4}} e^{-\frac{1}{2}(10 w+5)^{2}}+2 \\
        I_{2}&=\pi^{-\frac{1}{4}} \sqrt{2}(10 w+5) e^{-\frac{1}{2}(10 w+5)^{2}}+2,
    \end{aligned}
\end{equation}
and the learning input function is periodically switching between those two
\begin{equation}
    \label{input}
    I(w,t)=a(t)I_1(w)+b(t)I_2(w),
\end{equation}
where
\begin{equation}
    \label{eq:input_coff}
    \begin{aligned}
        a(t)&=\frac{1+\cos(\frac{2\pi t}{D})}{2},\\
        b(t)&=1-a(t).
    \end{aligned}
\end{equation}

For other parameters, we choose $V_F=2,V_R=1,V_{\text{min}}=1,a=1,\varepsilon=0.1,W_{\text{min}}=-1.1,W_{\text{max}}=0.1,T_{\text{max}}=4,\sigma(\bar{N})=\bar{N},\Delta t=2.5\times 10^{-4},\Delta w=0.01$ and the initial condition 
\begin{equation}
    p_{\text{init}}=\begin{cases}
        \text{sin}^2(\pi v)\text{sin}^2(\pi w) \qquad &-1<w<0\text{ and }-1<v<1,\\
        0 &\text{otherwise}.
    \end{cases}
\end{equation}


In the learning phase, the input function changes periodically in time; the smaller the period is, the greater the rate of change of the input function is. The total network activity $\bar{N}$ is an intuitive response to the input function, so we first observe the change in the total network activity. First, we choose period $D=1,0.5,0.2$. 
\begin{figure}[!htb]
    \centering
        \begin{minipage}[c]{0.3\textwidth}
            \centering
            \includegraphics[width=1\textwidth]{Nbar1.eps}
        \end{minipage}
        \begin{minipage}[c]{0.3\textwidth}
            \centering
            \includegraphics[width=1\textwidth]{Nbar2.eps}
        \end{minipage}
        \begin{minipage}[c]{0.3\textwidth}
            \centering
            \includegraphics[width=1\textwidth]{Nbar3.eps}
        \end{minipage}
    \caption{Equation parameters $a=1$ and $\varepsilon=0.1$. The evolution of total firing rate $\bar{N}$. Left: the input function period $D=1$. Middle: the input function period $D=0.5$. Right: the input function period $D=0.2$.}
    \label{fig:Nbar}
\end{figure}

Figure \ref{fig:Nbar} shows the evolution of the total firing rate at different periods. As we expected, except for the initial transient evolutionary phase, the total activity of the network changes periodically over time and its period is the same as the input function.



\paragraph{Test 2. Adapting to fast oscillating inputs.}
Since the prediction signal is determined by the learning input function and reflects the model's learning of the learning input function $I(w,t)$, observing the prediction signal in different periods helps us explore the learning behavior of the model. We compare numerical results for different periods $D=4,0.4,0.2,0.1,0.01$.  In this case, the last input function learned by the model is $I(w,t_\text{max})=I_1$.



\begin{figure}[!htb]
    \centering
        \begin{minipage}[c]{0.3\textwidth}
            \centering
            \includegraphics[width=1\textwidth]{D=4.eps}
        \end{minipage}
        \begin{minipage}[c]{0.3\textwidth}
            \centering
            \includegraphics[width=1\textwidth]{D=0.4.eps}
        \end{minipage}
        \begin{minipage}[c]{0.3\textwidth}
            \centering
            \includegraphics[width=1\textwidth]{D=0.2.eps}
        \end{minipage}
        \begin{minipage}[c]{0.3\textwidth}
            \centering
            \includegraphics[width=1\textwidth]{D=0.1.eps}
        \end{minipage}
        \begin{minipage}[c]{0.3\textwidth}
            \centering
            \includegraphics[width=1\textwidth]{D=0.05.eps}
        \end{minipage}
        \begin{minipage}[c]{0.3\textwidth}
            \centering
            \includegraphics[width=1\textwidth]{D=0.01.eps}
        \end{minipage}
    \caption{Equation parameters $a=1$ and $\varepsilon=0.1$. The prediction signal at $t=4$ with different input function periods. Top: the input function period $D=4,0.4,0.2$ from left to right. Bottom: the input function period $0.1,0.05,0.01$ from left to right.}
    \label{fig:prediction}
\end{figure}

Figure \ref{fig:prediction} shows the prediction signal at different periods. When the period is large, the prediction signal is like a triangle. As the period gets smaller, the shape of the prediction signal is getting more and more irregular. However, as the period is getting further smaller, the shape of the prediction signal is becoming triangular again. In previous experiments \cite{perthame2017distributed}\cite{he2022structure}, for the time-independent learning input signal $I(w)$, the test signal always resembles a triangle. So we speculate from Figure \ref{fig:prediction} that for sufficiently large or sufficiently small periods, the predicted signal looks like a triangle, and the model has effectively learned a signal of a certain form.




\paragraph{Test 3. Learning from oscillating inputs.}

In order to verify the above conjecture, we choose a relatively large period with $D=4$ and a small period with $D=0.01$ in \eqref{eq:input_coff}.  In the testing phase, we choose testing input functions $J=I_1$, $J=I_2$, and $J=\frac{I_1+I_2}{2}$.
\begin{figure}[!htb]
    \centering
        \begin{minipage}[c]{0.3\textwidth}
            \centering
            \includegraphics[width=1\textwidth]{D=4_I1test.eps}
        \end{minipage}
        \begin{minipage}[c]{0.3\textwidth}
            \centering
            \includegraphics[width=1\textwidth]{D=4_I2test.eps}
        \end{minipage}
        \begin{minipage}[c]{0.3\textwidth}
            \centering
            \includegraphics[width=1\textwidth]{D=4_I12test.eps}
        \end{minipage}
    \caption{(Output signal for the large period learning input) The
final firing rate $N(w)$ with different testing input $J(w)$. Equation parameters $a=1$ and $\varepsilon=0.1$, and the period of the input function in the learning phase is $D=4$. Left: Output signal with testing input function $J=I_1$. Middle: Output signal with testing input function $J=I_2$. Right: Output signal with testing input function $J=\frac{I_1+I_2}{2}$.}
    \label{fig:largeD}
\end{figure}
\begin{figure}[!htb]
    \centering
        \begin{minipage}[c]{0.3\textwidth}
            \centering
            \includegraphics[width=1\textwidth]{D=0.01_I1test.eps}
        \end{minipage}
        \begin{minipage}[c]{0.3\textwidth}
            \centering
            \includegraphics[width=1\textwidth]{D=0.01_I2test.eps}
        \end{minipage}
        \begin{minipage}[c]{0.3\textwidth}
            \centering
            \includegraphics[width=1\textwidth]{D=0.01_I12test.eps}
        \end{minipage}
    \caption{Output signal for the small period learning input) The
final firing rate $N(w)$ with different testing input $J(w)$. Equation parameters $a=1$ and $\varepsilon=0.1$, and the period of the input function in the learning phase is $D=0.01$. Left: Output signal with testing input function $J=I_1$. Middle: Output signal with testing input function $J=I_2$. Right: Output signal with testing input function $J=\frac{I_1+I_2}{2}$.}
    \label{fig:shortD}
\end{figure}

Figure \ref{fig:largeD} shows the output signal of period $D=4$ with testing input function $J=I_1$, $J=I_2$ and $J=\frac{I_1+I_2}{2}$. When $J=I_1=I(w,t_\text{max})$, the output signal $N_J^*(w)$ is like a triangle. Figure \ref{fig:shortD} shows the output signal of period $D=0.01$ with testing input function $J=I_1$, $J=I_2$ and $J=\frac{I_1+I_2}{2}$. When $J=\frac{I_1+I_2}{2}$, the output signal $N_J^*(w)$ is like a triangle. Numerical results show that when the period is relatively large, the signal learned by the model matches $I_1$, and when the period is relatively small, it matches $\frac{I_1+I_2}{2}$. 

The experimental results can be interpreted as follows. When the period is large, the model has enough time to learn, so the learned signal is the input function at the last moment. And when the period is small, neither $I_1$ nor $I_2$ can be learned well, but the result of learning is the average of the two.  Because when the switching process is too fast, the effect of the model on the learning of either $I_1$ or $I_2$ is poor. Instead, the average signal $\frac{I_1+I_2}{2}$ is captured by the time averaging of the learning process. 


 \paragraph{Test 4. Phase diagram for leaning.} There are multiple typical time scales in this model: the time scale for the voltage activities, the time scale for learning by redistributing the synaptic weights and the time period in the external input. When introducing the model, we perform a time rescaling for \eqref{eq:problem4}, where the parameter $\varepsilon$ reflects the ratio between the time scales of voltage activities and learning. In the next numerical experiment, we choose $\varepsilon=1,0.5,0.25,0.125$ and periods $D=2^2,2^1,\dots,2^{-7}$ to compare the results of the output signal under different parameters. After the testing phase, we choose the total activity $\bar{N}(t)$ to quantify the output signal:
\begin{equation}
    \label{judge_tool}
    E^{\varepsilon,D}_J=\left| \bar{N}^{\varepsilon,D}_J -\bar{N}^{\varepsilon}_J \right|.
\end{equation}
Here, $\bar{N}^{\varepsilon,D}_J$ denotes the total activity when the equation parameter is $\varepsilon$, the learning input function is given by \eqref{input} with period $D$, while the testing input function is $J$. $\bar{N}^{\varepsilon}_J$ represents the total activity where the equation parameter is $\varepsilon$ and both the learning input function and the testing input function are $J$. $E^{\varepsilon,D}_J$ can roughly measure the output signal. The closer the value of $E^{\varepsilon,D}_J$ is to zero, the superior the model's learning efficacy.

\begin{figure}[!htb]
    \centering
        \begin{minipage}[c]{0.49\textwidth}
            \centering
            \includegraphics[width=1\textwidth]{eps_test1.eps}
        \end{minipage}
        \begin{minipage}[c]{0.49\textwidth}
            \centering
            \includegraphics[width=1\textwidth]{eps_test2.eps}
        \end{minipage}
    \caption{Equation parameter $a=1$. Left: The value of $E^{\varepsilon,D}_J$ under different $D$ and $\varepsilon$ with input function $J=I_1$. Right: The value of $E^{\varepsilon,D}_J$ under different $D$ and $\varepsilon$ with input function $J=\frac{I_1+I_2}{2}$.}
    \label{fig:output3}
\end{figure}

 As shown in Figure \ref{fig:output3}, as the period becomes smaller, the testing indicator becomes less significant with respect to the testing input function $J=I_1$, and the test indicator becomes more significant with respect to the testing input function $J=\frac{I_1+I_2}{2}$. Besides, the numerical results also suggest that when epsilon is small, the transition in learning takes place at a smaller time period, whereas such a trend is not prominent. Although the experiments are not fully conclusive yet, they show a lot of promise for using the proposed numerical method to simulate large-scale tests.

% \vspace{-0.5em}
\section{Conclusion}
% \vspace{-0.5em}
Recent advances in multimodal single-cell technology have enabled the simultaneous profiling of the transcriptome alongside other cellular modalities, leading to an increase in the availability of multimodal single-cell data. In this paper, we present \method{}, a multimodal transformer model for single-cell surface protein abundance from gene expression measurements. We combined the data with prior biological interaction knowledge from the STRING database into a richly connected heterogeneous graph and leveraged the transformer architectures to learn an accurate mapping between gene expression and surface protein abundance. Remarkably, \method{} achieves superior and more stable performance than other baselines on both 2021 and 2022 NeurIPS single-cell datasets.

\noindent\textbf{Future Work.}
% Our work is an extension of the model we implemented in the NeurIPS 2022 competition. 
Our framework of multimodal transformers with the cross-modality heterogeneous graph goes far beyond the specific downstream task of modality prediction, and there are lots of potentials to be further explored. Our graph contains three types of nodes. While the cell embeddings are used for predictions, the remaining protein embeddings and gene embeddings may be further interpreted for other tasks. The similarities between proteins may show data-specific protein-protein relationships, while the attention matrix of the gene transformer may help to identify marker genes of each cell type. Additionally, we may achieve gene interaction prediction using the attention mechanism.
% under adequate regulations. 
% We expect \method{} to be capable of much more than just modality prediction. Note that currently, we fuse information from different transformers with message-passing GNNs. 
To extend more on transformers, a potential next step is implementing cross-attention cross-modalities. Ideally, all three types of nodes, namely genes, proteins, and cells, would be jointly modeled using a large transformer that includes specific regulations for each modality. 

% insight of protein and gene embedding (diff task)

% all in one transformer

% \noindent\textbf{Limitations and future work}
% Despite the noticeable performance improvement by utilizing transformers with the cross-modality heterogeneous graph, there are still bottlenecks in the current settings. To begin with, we noticed that the performance variations of all methods are consistently higher in the ``CITE'' dataset compared to the ``GEX2ADT'' dataset. We hypothesized that the increased variability in ``CITE'' was due to both less number of training samples (43k vs. 66k cells) and a significantly more number of testing samples used (28k vs. 1k cells). One straightforward solution to alleviate the high variation issue is to include more training samples, which is not always possible given the training data availability. Nevertheless, publicly available single-cell datasets have been accumulated over the past decades and are still being collected on an ever-increasing scale. Taking advantage of these large-scale atlases is the key to a more stable and well-performing model, as some of the intra-cell variations could be common across different datasets. For example, reference-based methods are commonly used to identify the cell identity of a single cell, or cell-type compositions of a mixture of cells. (other examples for pretrained, e.g., scbert)


%\noindent\textbf{Future work.}
% Our work is an extension of the model we implemented in the NeurIPS 2022 competition. Now our framework of multimodal transformers with the cross-modality heterogeneous graph goes far beyond the specific downstream task of modality prediction, and there are lots of potentials to be further explored. Our graph contains three types of nodes. while the cell embeddings are used for predictions, the remaining protein embeddings and gene embeddings may be further interpreted for other tasks. The similarities between proteins may show data-specific protein-protein relationships, while the attention matrix of the gene transformer may help to identify marker genes of each cell type. Additionally, we may achieve gene interaction prediction using the attention mechanism under adequate regulations. We expect \method{} to be capable of much more than just modality prediction. Note that currently, we fuse information from different transformers with message-passing GNNs. To extend more on transformers, a potential next step is implementing cross-attention cross-modalities. Ideally, all three types of nodes, namely genes, proteins, and cells, would be jointly modeled using a large transformer that includes specific regulations for each modality. The self-attention within each modality would reconstruct the prior interaction network, while the cross-attention between modalities would be supervised by the data observations. Then, The attention matrix will provide insights into all the internal interactions and cross-relationships. With the linearized transformer, this idea would be both practical and versatile.

% \begin{acks}
% This research is supported by the National Science Foundation (NSF) and Johnson \& Johnson.
% \end{acks}
\chapter{Supplementary Material}
\label{appendix}

In this appendix, we present supplementary material for the techniques and
experiments presented in the main text.

\section{Baseline Results and Analysis for Informed Sampler}
\label{appendix:chap3}

Here, we give an in-depth
performance analysis of the various samplers and the effect of their
hyperparameters. We choose hyperparameters with the lowest PSRF value
after $10k$ iterations, for each sampler individually. If the
differences between PSRF are not significantly different among
multiple values, we choose the one that has the highest acceptance
rate.

\subsection{Experiment: Estimating Camera Extrinsics}
\label{appendix:chap3:room}

\subsubsection{Parameter Selection}
\paragraph{Metropolis Hastings (\MH)}

Figure~\ref{fig:exp1_MH} shows the median acceptance rates and PSRF
values corresponding to various proposal standard deviations of plain
\MH~sampling. Mixing gets better and the acceptance rate gets worse as
the standard deviation increases. The value $0.3$ is selected standard
deviation for this sampler.

\paragraph{Metropolis Hastings Within Gibbs (\MHWG)}

As mentioned in Section~\ref{sec:room}, the \MHWG~sampler with one-dimensional
updates did not converge for any value of proposal standard deviation.
This problem has high correlation of the camera parameters and is of
multi-modal nature, which this sampler has problems with.

\paragraph{Parallel Tempering (\PT)}

For \PT~sampling, we took the best performing \MH~sampler and used
different temperature chains to improve the mixing of the
sampler. Figure~\ref{fig:exp1_PT} shows the results corresponding to
different combination of temperature levels. The sampler with
temperature levels of $[1,3,27]$ performed best in terms of both
mixing and acceptance rate.

\paragraph{Effect of Mixture Coefficient in Informed Sampling (\MIXLMH)}

Figure~\ref{fig:exp1_alpha} shows the effect of mixture
coefficient ($\alpha$) on the informed sampling
\MIXLMH. Since there is no significant different in PSRF values for
$0 \le \alpha \le 0.7$, we chose $0.7$ due to its high acceptance
rate.


% \end{multicols}

\begin{figure}[h]
\centering
  \subfigure[MH]{%
    \includegraphics[width=.48\textwidth]{figures/supplementary/camPose_MH.pdf} \label{fig:exp1_MH}
  }
  \subfigure[PT]{%
    \includegraphics[width=.48\textwidth]{figures/supplementary/camPose_PT.pdf} \label{fig:exp1_PT}
  }
\\
  \subfigure[INF-MH]{%
    \includegraphics[width=.48\textwidth]{figures/supplementary/camPose_alpha.pdf} \label{fig:exp1_alpha}
  }
  \mycaption{Results of the `Estimating Camera Extrinsics' experiment}{PRSFs and Acceptance rates corresponding to (a) various standard deviations of \MH, (b) various temperature level combinations of \PT sampling and (c) various mixture coefficients of \MIXLMH sampling.}
\end{figure}



\begin{figure}[!t]
\centering
  \subfigure[\MH]{%
    \includegraphics[width=.48\textwidth]{figures/supplementary/occlusionExp_MH.pdf} \label{fig:exp2_MH}
  }
  \subfigure[\BMHWG]{%
    \includegraphics[width=.48\textwidth]{figures/supplementary/occlusionExp_BMHWG.pdf} \label{fig:exp2_BMHWG}
  }
\\
  \subfigure[\MHWG]{%
    \includegraphics[width=.48\textwidth]{figures/supplementary/occlusionExp_MHWG.pdf} \label{fig:exp2_MHWG}
  }
  \subfigure[\PT]{%
    \includegraphics[width=.48\textwidth]{figures/supplementary/occlusionExp_PT.pdf} \label{fig:exp2_PT}
  }
\\
  \subfigure[\INFBMHWG]{%
    \includegraphics[width=.5\textwidth]{figures/supplementary/occlusionExp_alpha.pdf} \label{fig:exp2_alpha}
  }
  \mycaption{Results of the `Occluding Tiles' experiment}{PRSF and
    Acceptance rates corresponding to various standard deviations of
    (a) \MH, (b) \BMHWG, (c) \MHWG, (d) various temperature level
    combinations of \PT~sampling and; (e) various mixture coefficients
    of our informed \INFBMHWG sampling.}
\end{figure}

%\onecolumn\newpage\twocolumn
\subsection{Experiment: Occluding Tiles}
\label{appendix:chap3:tiles}

\subsubsection{Parameter Selection}

\paragraph{Metropolis Hastings (\MH)}

Figure~\ref{fig:exp2_MH} shows the results of
\MH~sampling. Results show the poor convergence for all proposal
standard deviations and rapid decrease of AR with increasing standard
deviation. This is due to the high-dimensional nature of
the problem. We selected a standard deviation of $1.1$.

\paragraph{Blocked Metropolis Hastings Within Gibbs (\BMHWG)}

The results of \BMHWG are shown in Figure~\ref{fig:exp2_BMHWG}. In
this sampler we update only one block of tile variables (of dimension
four) in each sampling step. Results show much better performance
compared to plain \MH. The optimal proposal standard deviation for
this sampler is $0.7$.

\paragraph{Metropolis Hastings Within Gibbs (\MHWG)}

Figure~\ref{fig:exp2_MHWG} shows the result of \MHWG sampling. This
sampler is better than \BMHWG and converges much more quickly. Here
a standard deviation of $0.9$ is found to be best.

\paragraph{Parallel Tempering (\PT)}

Figure~\ref{fig:exp2_PT} shows the results of \PT sampling with various
temperature combinations. Results show no improvement in AR from plain
\MH sampling and again $[1,3,27]$ temperature levels are found to be optimal.

\paragraph{Effect of Mixture Coefficient in Informed Sampling (\INFBMHWG)}

Figure~\ref{fig:exp2_alpha} shows the effect of mixture
coefficient ($\alpha$) on the blocked informed sampling
\INFBMHWG. Since there is no significant different in PSRF values for
$0 \le \alpha \le 0.8$, we chose $0.8$ due to its high acceptance
rate.



\subsection{Experiment: Estimating Body Shape}
\label{appendix:chap3:body}

\subsubsection{Parameter Selection}
\paragraph{Metropolis Hastings (\MH)}

Figure~\ref{fig:exp3_MH} shows the result of \MH~sampling with various
proposal standard deviations. The value of $0.1$ is found to be
best.

\paragraph{Metropolis Hastings Within Gibbs (\MHWG)}

For \MHWG sampling we select $0.3$ proposal standard
deviation. Results are shown in Fig.~\ref{fig:exp3_MHWG}.


\paragraph{Parallel Tempering (\PT)}

As before, results in Fig.~\ref{fig:exp3_PT}, the temperature levels
were selected to be $[1,3,27]$ due its slightly higher AR.

\paragraph{Effect of Mixture Coefficient in Informed Sampling (\MIXLMH)}

Figure~\ref{fig:exp3_alpha} shows the effect of $\alpha$ on PSRF and
AR. Since there is no significant differences in PSRF values for $0 \le
\alpha \le 0.8$, we choose $0.8$.


\begin{figure}[t]
\centering
  \subfigure[\MH]{%
    \includegraphics[width=.48\textwidth]{figures/supplementary/bodyShape_MH.pdf} \label{fig:exp3_MH}
  }
  \subfigure[\MHWG]{%
    \includegraphics[width=.48\textwidth]{figures/supplementary/bodyShape_MHWG.pdf} \label{fig:exp3_MHWG}
  }
\\
  \subfigure[\PT]{%
    \includegraphics[width=.48\textwidth]{figures/supplementary/bodyShape_PT.pdf} \label{fig:exp3_PT}
  }
  \subfigure[\MIXLMH]{%
    \includegraphics[width=.48\textwidth]{figures/supplementary/bodyShape_alpha.pdf} \label{fig:exp3_alpha}
  }
\\
  \mycaption{Results of the `Body Shape Estimation' experiment}{PRSFs and
    Acceptance rates corresponding to various standard deviations of
    (a) \MH, (b) \MHWG; (c) various temperature level combinations
    of \PT sampling and; (d) various mixture coefficients of the
    informed \MIXLMH sampling.}
\end{figure}


\subsection{Results Overview}
Figure~\ref{fig:exp_summary} shows the summary results of the all the three
experimental studies related to informed sampler.
\begin{figure*}[h!]
\centering
  \subfigure[Results for: Estimating Camera Extrinsics]{%
    \includegraphics[width=0.9\textwidth]{figures/supplementary/camPose_ALL.pdf} \label{fig:exp1_all}
  }
  \subfigure[Results for: Occluding Tiles]{%
    \includegraphics[width=0.9\textwidth]{figures/supplementary/occlusionExp_ALL.pdf} \label{fig:exp2_all}
  }
  \subfigure[Results for: Estimating Body Shape]{%
    \includegraphics[width=0.9\textwidth]{figures/supplementary/bodyShape_ALL.pdf} \label{fig:exp3_all}
  }
  \label{fig:exp_summary}
  \mycaption{Summary of the statistics for the three experiments}{Shown are
    for several baseline methods and the informed samplers the
    acceptance rates (left), PSRFs (middle), and RMSE values
    (right). All results are median results over multiple test
    examples.}
\end{figure*}

\subsection{Additional Qualitative Results}

\subsubsection{Occluding Tiles}
In Figure~\ref{fig:exp2_visual_more} more qualitative results of the
occluding tiles experiment are shown. The informed sampling approach
(\INFBMHWG) is better than the best baseline (\MHWG). This still is a
very challenging problem since the parameters for occluded tiles are
flat over a large region. Some of the posterior variance of the
occluded tiles is already captured by the informed sampler.

\begin{figure*}[h!]
\begin{center}
\centerline{\includegraphics[width=0.95\textwidth]{figures/supplementary/occlusionExp_Visual.pdf}}
\mycaption{Additional qualitative results of the occluding tiles experiment}
  {From left to right: (a)
  Given image, (b) Ground truth tiles, (c) OpenCV heuristic and most probable estimates
  from 5000 samples obtained by (d) MHWG sampler (best baseline) and
  (e) our INF-BMHWG sampler. (f) Posterior expectation of the tiles
  boundaries obtained by INF-BMHWG sampling (First 2000 samples are
  discarded as burn-in).}
\label{fig:exp2_visual_more}
\end{center}
\end{figure*}

\subsubsection{Body Shape}
Figure~\ref{fig:exp3_bodyMeshes} shows some more results of 3D mesh
reconstruction using posterior samples obtained by our informed
sampling \MIXLMH.

\begin{figure*}[t]
\begin{center}
\centerline{\includegraphics[width=0.75\textwidth]{figures/supplementary/bodyMeshResults.pdf}}
\mycaption{Qualitative results for the body shape experiment}
  {Shown is the 3D mesh reconstruction results with first 1000 samples obtained
  using the \MIXLMH informed sampling method. (blue indicates small
  values and red indicates high values)}
\label{fig:exp3_bodyMeshes}
\end{center}
\end{figure*}

\clearpage



\section{Additional Results on the Face Problem with CMP}

Figure~\ref{fig:shading-qualitative-multiple-subjects-supp} shows inference results for reflectance maps, normal maps and lights for randomly chosen test images, and Fig.~\ref{fig:shading-qualitative-same-subject-supp} shows reflectance estimation results on multiple images of the same subject produced under different illumination conditions. CMP is able to produce estimates that are closer to the groundtruth across different subjects and illumination conditions.

\begin{figure*}[h]
  \begin{center}
  \centerline{\includegraphics[width=1.0\columnwidth]{figures/face_cmp_visual_results_supp.pdf}}
  \vspace{-1.2cm}
  \end{center}
	\mycaption{A visual comparison of inference results}{(a)~Observed images. (b)~Inferred reflectance maps. \textit{GT} is the photometric stereo groundtruth, \textit{BU} is the Biswas \etal (2009) reflectance estimate and \textit{Forest} is the consensus prediction. (c)~The variance of the inferred reflectance estimate produced by \MTD (normalized across rows).(d)~Visualization of inferred light directions. (e)~Inferred normal maps.}
	\label{fig:shading-qualitative-multiple-subjects-supp}
\end{figure*}


\begin{figure*}[h]
	\centering
	\setlength\fboxsep{0.2mm}
	\setlength\fboxrule{0pt}
	\begin{tikzpicture}

		\matrix at (0, 0) [matrix of nodes, nodes={anchor=east}, column sep=-0.05cm, row sep=-0.2cm]
		{
			\fbox{\includegraphics[width=1cm]{figures/sample_3_4_X.png}} &
			\fbox{\includegraphics[width=1cm]{figures/sample_3_4_GT.png}} &
			\fbox{\includegraphics[width=1cm]{figures/sample_3_4_BISWAS.png}}  &
			\fbox{\includegraphics[width=1cm]{figures/sample_3_4_VMP.png}}  &
			\fbox{\includegraphics[width=1cm]{figures/sample_3_4_FOREST.png}}  &
			\fbox{\includegraphics[width=1cm]{figures/sample_3_4_CMP.png}}  &
			\fbox{\includegraphics[width=1cm]{figures/sample_3_4_CMPVAR.png}}
			 \\

			\fbox{\includegraphics[width=1cm]{figures/sample_3_5_X.png}} &
			\fbox{\includegraphics[width=1cm]{figures/sample_3_5_GT.png}} &
			\fbox{\includegraphics[width=1cm]{figures/sample_3_5_BISWAS.png}}  &
			\fbox{\includegraphics[width=1cm]{figures/sample_3_5_VMP.png}}  &
			\fbox{\includegraphics[width=1cm]{figures/sample_3_5_FOREST.png}}  &
			\fbox{\includegraphics[width=1cm]{figures/sample_3_5_CMP.png}}  &
			\fbox{\includegraphics[width=1cm]{figures/sample_3_5_CMPVAR.png}}
			 \\

			\fbox{\includegraphics[width=1cm]{figures/sample_3_6_X.png}} &
			\fbox{\includegraphics[width=1cm]{figures/sample_3_6_GT.png}} &
			\fbox{\includegraphics[width=1cm]{figures/sample_3_6_BISWAS.png}}  &
			\fbox{\includegraphics[width=1cm]{figures/sample_3_6_VMP.png}}  &
			\fbox{\includegraphics[width=1cm]{figures/sample_3_6_FOREST.png}}  &
			\fbox{\includegraphics[width=1cm]{figures/sample_3_6_CMP.png}}  &
			\fbox{\includegraphics[width=1cm]{figures/sample_3_6_CMPVAR.png}}
			 \\
	     };

       \node at (-3.85, -2.0) {\small Observed};
       \node at (-2.55, -2.0) {\small `GT'};
       \node at (-1.27, -2.0) {\small BU};
       \node at (0.0, -2.0) {\small MP};
       \node at (1.27, -2.0) {\small Forest};
       \node at (2.55, -2.0) {\small \textbf{CMP}};
       \node at (3.85, -2.0) {\small Variance};

	\end{tikzpicture}
	\mycaption{Robustness to varying illumination}{Reflectance estimation on a subject images with varying illumination. Left to right: observed image, photometric stereo estimate (GT)
  which is used as a proxy for groundtruth, bottom-up estimate of \cite{Biswas2009}, VMP result, consensus forest estimate, CMP mean, and CMP variance.}
	\label{fig:shading-qualitative-same-subject-supp}
\end{figure*}

\clearpage

\section{Additional Material for Learning Sparse High Dimensional Filters}
\label{sec:appendix-bnn}

This part of supplementary material contains a more detailed overview of the permutohedral
lattice convolution in Section~\ref{sec:permconv}, more experiments in
Section~\ref{sec:addexps} and additional results with protocols for
the experiments presented in Chapter~\ref{chap:bnn} in Section~\ref{sec:addresults}.

\vspace{-0.2cm}
\subsection{General Permutohedral Convolutions}
\label{sec:permconv}

A core technical contribution of this work is the generalization of the Gaussian permutohedral lattice
convolution proposed in~\cite{adams2010fast} to the full non-separable case with the
ability to perform back-propagation. Although, conceptually, there are minor
differences between Gaussian and general parameterized filters, there are non-trivial practical
differences in terms of the algorithmic implementation. The Gauss filters belong to
the separable class and can thus be decomposed into multiple
sequential one dimensional convolutions. We are interested in the general filter
convolutions, which can not be decomposed. Thus, performing a general permutohedral
convolution at a lattice point requires the computation of the inner product with the
neighboring elements in all the directions in the high-dimensional space.

Here, we give more details of the implementation differences of separable
and non-separable filters. In the following, we will explain the scalar case first.
Recall, that the forward pass of general permutohedral convolution
involves 3 steps: \textit{splatting}, \textit{convolving} and \textit{slicing}.
We follow the same splatting and slicing strategies as in~\cite{adams2010fast}
since these operations do not depend on the filter kernel. The main difference
between our work and the existing implementation of~\cite{adams2010fast} is
the way that the convolution operation is executed. This proceeds by constructing
a \emph{blur neighbor} matrix $K$ that stores for every lattice point all
values of the lattice neighbors that are needed to compute the filter output.

\begin{figure}[t!]
  \centering
    \includegraphics[width=0.6\columnwidth]{figures/supplementary/lattice_construction}
  \mycaption{Illustration of 1D permutohedral lattice construction}
  {A $4\times 4$ $(x,y)$ grid lattice is projected onto the plane defined by the normal
  vector $(1,1)^{\top}$. This grid has $s+1=4$ and $d=2$ $(s+1)^{d}=4^2=16$ elements.
  In the projection, all points of the same color are projected onto the same points in the plane.
  The number of elements of the projected lattice is $t=(s+1)^d-s^d=4^2-3^2=7$, that is
  the $(4\times 4)$ grid minus the size of lattice that is $1$ smaller at each size, in this
  case a $(3\times 3)$ lattice (the upper right $(3\times 3)$ elements).
  }
\label{fig:latticeconstruction}
\end{figure}

The blur neighbor matrix is constructed by traversing through all the populated
lattice points and their neighboring elements.
% For efficiency, we do this matrix construction recursively with shared computations
% since $n^{th}$ neighbourhood elements are $1^{st}$ neighborhood elements of $n-1^{th}$ neighbourhood elements. \pg{do not understand}
This is done recursively to share computations. For any lattice point, the neighbors that are
$n$ hops away are the direct neighbors of the points that are $n-1$ hops away.
The size of a $d$ dimensional spatial filter with width $s+1$ is $(s+1)^{d}$ (\eg, a
$3\times 3$ filter, $s=2$ in $d=2$ has $3^2=9$ elements) and this size grows
exponentially in the number of dimensions $d$. The permutohedral lattice is constructed by
projecting a regular grid onto the plane spanned by the $d$ dimensional normal vector ${(1,\ldots,1)}^{\top}$. See
Fig.~\ref{fig:latticeconstruction} for an illustration of the 1D lattice construction.
Many corners of a grid filter are projected onto the same point, in total $t = {(s+1)}^{d} -
s^{d}$ elements remain in the permutohedral filter with $s$ neighborhood in $d-1$ dimensions.
If the lattice has $m$ populated elements, the
matrix $K$ has size $t\times m$. Note that, since the input signal is typically
sparse, only a few lattice corners are being populated in the \textit{slicing} step.
We use a hash-table to keep track of these points and traverse only through
the populated lattice points for this neighborhood matrix construction.

Once the blur neighbor matrix $K$ is constructed, we can perform the convolution
by the matrix vector multiplication
\begin{equation}
\ell' = BK,
\label{eq:conv}
\end{equation}
where $B$ is the $1 \times t$ filter kernel (whose values we will learn) and $\ell'\in\mathbb{R}^{1\times m}$
is the result of the filtering at the $m$ lattice points. In practice, we found that the
matrix $K$ is sometimes too large to fit into GPU memory and we divided the matrix $K$
into smaller pieces to compute Eq.~\ref{eq:conv} sequentially.

In the general multi-dimensional case, the signal $\ell$ is of $c$ dimensions. Then
the kernel $B$ is of size $c \times t$ and $K$ stores the $c$ dimensional vectors
accordingly. When the input and output points are different, we slice only the
input points and splat only at the output points.


\subsection{Additional Experiments}
\label{sec:addexps}
In this section, we discuss more use-cases for the learned bilateral filters, one
use-case of BNNs and two single filter applications for image and 3D mesh denoising.

\subsubsection{Recognition of subsampled MNIST}\label{sec:app_mnist}

One of the strengths of the proposed filter convolution is that it does not
require the input to lie on a regular grid. The only requirement is to define a distance
between features of the input signal.
We highlight this feature with the following experiment using the
classical MNIST ten class classification problem~\cite{lecun1998mnist}. We sample a
sparse set of $N$ points $(x,y)\in [0,1]\times [0,1]$
uniformly at random in the input image, use their interpolated values
as signal and the \emph{continuous} $(x,y)$ positions as features. This mimics
sub-sampling of a high-dimensional signal. To compare against a spatial convolution,
we interpolate the sparse set of values at the grid positions.

We take a reference implementation of LeNet~\cite{lecun1998gradient} that
is part of the Caffe project~\cite{jia2014caffe} and compare it
against the same architecture but replacing the first convolutional
layer with a bilateral convolution layer (BCL). The filter size
and numbers are adjusted to get a comparable number of parameters
($5\times 5$ for LeNet, $2$-neighborhood for BCL).

The results are shown in Table~\ref{tab:all-results}. We see that training
on the original MNIST data (column Original, LeNet vs. BNN) leads to a slight
decrease in performance of the BNN (99.03\%) compared to LeNet
(99.19\%). The BNN can be trained and evaluated on sparse
signals, and we resample the image as described above for $N=$ 100\%, 60\% and
20\% of the total number of pixels. The methods are also evaluated
on test images that are subsampled in the same way. Note that we can
train and test with different subsampling rates. We introduce an additional
bilinear interpolation layer for the LeNet architecture to train on the same
data. In essence, both models perform a spatial interpolation and thus we
expect them to yield a similar classification accuracy. Once the data is of
higher dimensions, the permutohedral convolution will be faster due to hashing
the sparse input points, as well as less memory demanding in comparison to
naive application of a spatial convolution with interpolated values.

\begin{table}[t]
  \begin{center}
    \footnotesize
    \centering
    \begin{tabular}[t]{lllll}
      \toprule
              &     & \multicolumn{3}{c}{Test Subsampling} \\
       Method  & Original & 100\% & 60\% & 20\%\\
      \midrule
       LeNet &  \textbf{0.9919} & 0.9660 & 0.9348 & \textbf{0.6434} \\
       BNN &  0.9903 & \textbf{0.9844} & \textbf{0.9534} & 0.5767 \\
      \hline
       LeNet 100\% & 0.9856 & 0.9809 & 0.9678 & \textbf{0.7386} \\
       BNN 100\% & \textbf{0.9900} & \textbf{0.9863} & \textbf{0.9699} & 0.6910 \\
      \hline
       LeNet 60\% & 0.9848 & 0.9821 & 0.9740 & 0.8151 \\
       BNN 60\% & \textbf{0.9885} & \textbf{0.9864} & \textbf{0.9771} & \textbf{0.8214}\\
      \hline
       LeNet 20\% & \textbf{0.9763} & \textbf{0.9754} & 0.9695 & 0.8928 \\
       BNN 20\% & 0.9728 & 0.9735 & \textbf{0.9701} & \textbf{0.9042}\\
      \bottomrule
    \end{tabular}
  \end{center}
\vspace{-.2cm}
\caption{Classification accuracy on MNIST. We compare the
    LeNet~\cite{lecun1998gradient} implementation that is part of
    Caffe~\cite{jia2014caffe} to the network with the first layer
    replaced by a bilateral convolution layer (BCL). Both are trained
    on the original image resolution (first two rows). Three more BNN
    and CNN models are trained with randomly subsampled images (100\%,
    60\% and 20\% of the pixels). An additional bilinear interpolation
    layer samples the input signal on a spatial grid for the CNN model.
  }
  \label{tab:all-results}
\vspace{-.5cm}
\end{table}

\subsubsection{Image Denoising}

The main application that inspired the development of the bilateral
filtering operation is image denoising~\cite{aurich1995non}, there
using a single Gaussian kernel. Our development allows to learn this
kernel function from data and we explore how to improve using a \emph{single}
but more general bilateral filter.

We use the Berkeley segmentation dataset
(BSDS500)~\cite{arbelaezi2011bsds500} as a test bed. The color
images in the dataset are converted to gray-scale,
and corrupted with Gaussian noise with a standard deviation of
$\frac {25} {255}$.

We compare the performance of four different filter models on a
denoising task.
The first baseline model (`Spatial' in Table \ref{tab:denoising}, $25$
weights) uses a single spatial filter with a kernel size of
$5$ and predicts the scalar gray-scale value at the center pixel. The next model
(`Gauss Bilateral') applies a bilateral \emph{Gaussian}
filter to the noisy input, using position and intensity features $\f=(x,y,v)^\top$.
The third setup (`Learned Bilateral', $65$ weights)
takes a Gauss kernel as initialization and
fits all filter weights on the train set to minimize the
mean squared error with respect to the clean images.
We run a combination
of spatial and permutohedral convolutions on spatial and bilateral
features (`Spatial + Bilateral (Learned)') to check for a complementary
performance of the two convolutions.

\label{sec:exp:denoising}
\begin{table}[!h]
\begin{center}
  \footnotesize
  \begin{tabular}[t]{lr}
    \toprule
    Method & PSNR \\
    \midrule
    Noisy Input & $20.17$ \\
    Spatial & $26.27$ \\
    Gauss Bilateral & $26.51$ \\
    Learned Bilateral & $26.58$ \\
    Spatial + Bilateral (Learned) & \textbf{$26.65$} \\
    \bottomrule
  \end{tabular}
\end{center}
\vspace{-0.5em}
\caption{PSNR results of a denoising task using the BSDS500
  dataset~\cite{arbelaezi2011bsds500}}
\vspace{-0.5em}
\label{tab:denoising}
\end{table}
\vspace{-0.2em}

The PSNR scores evaluated on full images of the test set are
shown in Table \ref{tab:denoising}. We find that an untrained bilateral
filter already performs better than a trained spatial convolution
($26.27$ to $26.51$). A learned convolution further improve the
performance slightly. We chose this simple one-kernel setup to
validate an advantage of the generalized bilateral filter. A competitive
denoising system would employ RGB color information and also
needs to be properly adjusted in network size. Multi-layer perceptrons
have obtained state-of-the-art denoising results~\cite{burger12cvpr}
and the permutohedral lattice layer can readily be used in such an
architecture, which is intended future work.

\subsection{Additional results}
\label{sec:addresults}

This section contains more qualitative results for the experiments presented in Chapter~\ref{chap:bnn}.

\begin{figure*}[th!]
  \centering
    \includegraphics[width=\columnwidth,trim={5cm 2.5cm 5cm 4.5cm},clip]{figures/supplementary/lattice_viz.pdf}
    \vspace{-0.7cm}
  \mycaption{Visualization of the Permutohedral Lattice}
  {Sample lattice visualizations for different feature spaces. All pixels falling in the same simplex cell are shown with
  the same color. $(x,y)$ features correspond to image pixel positions, and $(r,g,b) \in [0,255]$ correspond
  to the red, green and blue color values.}
\label{fig:latticeviz}
\end{figure*}

\subsubsection{Lattice Visualization}

Figure~\ref{fig:latticeviz} shows sample lattice visualizations for different feature spaces.

\newcolumntype{L}[1]{>{\raggedright\let\newline\\\arraybackslash\hspace{0pt}}b{#1}}
\newcolumntype{C}[1]{>{\centering\let\newline\\\arraybackslash\hspace{0pt}}b{#1}}
\newcolumntype{R}[1]{>{\raggedleft\let\newline\\\arraybackslash\hspace{0pt}}b{#1}}

\subsubsection{Color Upsampling}\label{sec:color_upsampling}
\label{sec:col_upsample_extra}

Some images of the upsampling for the Pascal
VOC12 dataset are shown in Fig.~\ref{fig:Colour_upsample_visuals}. It is
especially the low level image details that are better preserved with
a learned bilateral filter compared to the Gaussian case.

\begin{figure*}[t!]
  \centering
    \subfigure{%
   \raisebox{2.0em}{
    \includegraphics[width=.06\columnwidth]{figures/supplementary/2007_004969.jpg}
   }
  }
  \subfigure{%
    \includegraphics[width=.17\columnwidth]{figures/supplementary/2007_004969_gray.pdf}
  }
  \subfigure{%
    \includegraphics[width=.17\columnwidth]{figures/supplementary/2007_004969_gt.pdf}
  }
  \subfigure{%
    \includegraphics[width=.17\columnwidth]{figures/supplementary/2007_004969_bicubic.pdf}
  }
  \subfigure{%
    \includegraphics[width=.17\columnwidth]{figures/supplementary/2007_004969_gauss.pdf}
  }
  \subfigure{%
    \includegraphics[width=.17\columnwidth]{figures/supplementary/2007_004969_learnt.pdf}
  }\\
    \subfigure{%
   \raisebox{2.0em}{
    \includegraphics[width=.06\columnwidth]{figures/supplementary/2007_003106.jpg}
   }
  }
  \subfigure{%
    \includegraphics[width=.17\columnwidth]{figures/supplementary/2007_003106_gray.pdf}
  }
  \subfigure{%
    \includegraphics[width=.17\columnwidth]{figures/supplementary/2007_003106_gt.pdf}
  }
  \subfigure{%
    \includegraphics[width=.17\columnwidth]{figures/supplementary/2007_003106_bicubic.pdf}
  }
  \subfigure{%
    \includegraphics[width=.17\columnwidth]{figures/supplementary/2007_003106_gauss.pdf}
  }
  \subfigure{%
    \includegraphics[width=.17\columnwidth]{figures/supplementary/2007_003106_learnt.pdf}
  }\\
  \setcounter{subfigure}{0}
  \small{
  \subfigure[Inp.]{%
  \raisebox{2.0em}{
    \includegraphics[width=.06\columnwidth]{figures/supplementary/2007_006837.jpg}
   }
  }
  \subfigure[Guidance]{%
    \includegraphics[width=.17\columnwidth]{figures/supplementary/2007_006837_gray.pdf}
  }
   \subfigure[GT]{%
    \includegraphics[width=.17\columnwidth]{figures/supplementary/2007_006837_gt.pdf}
  }
  \subfigure[Bicubic]{%
    \includegraphics[width=.17\columnwidth]{figures/supplementary/2007_006837_bicubic.pdf}
  }
  \subfigure[Gauss-BF]{%
    \includegraphics[width=.17\columnwidth]{figures/supplementary/2007_006837_gauss.pdf}
  }
  \subfigure[Learned-BF]{%
    \includegraphics[width=.17\columnwidth]{figures/supplementary/2007_006837_learnt.pdf}
  }
  }
  \vspace{-0.5cm}
  \mycaption{Color Upsampling}{Color $8\times$ upsampling results
  using different methods, from left to right, (a)~Low-resolution input color image (Inp.),
  (b)~Gray scale guidance image, (c)~Ground-truth color image; Upsampled color images with
  (d)~Bicubic interpolation, (e) Gauss bilateral upsampling and, (f)~Learned bilateral
  updampgling (best viewed on screen).}

\label{fig:Colour_upsample_visuals}
\end{figure*}

\subsubsection{Depth Upsampling}
\label{sec:depth_upsample_extra}

Figure~\ref{fig:depth_upsample_visuals} presents some more qualitative results comparing bicubic interpolation, Gauss
bilateral and learned bilateral upsampling on NYU depth dataset image~\cite{silberman2012indoor}.

\subsubsection{Character Recognition}\label{sec:app_character}

 Figure~\ref{fig:nnrecognition} shows the schematic of different layers
 of the network architecture for LeNet-7~\cite{lecun1998mnist}
 and DeepCNet(5, 50)~\cite{ciresan2012multi,graham2014spatially}. For the BNN variants, the first layer filters are replaced
 with learned bilateral filters and are learned end-to-end.

\subsubsection{Semantic Segmentation}\label{sec:app_semantic_segmentation}
\label{sec:semantic_bnn_extra}

Some more visual results for semantic segmentation are shown in Figure~\ref{fig:semantic_visuals}.
These include the underlying DeepLab CNN\cite{chen2014semantic} result (DeepLab),
the 2 step mean-field result with Gaussian edge potentials (+2stepMF-GaussCRF)
and also corresponding results with learned edge potentials (+2stepMF-LearnedCRF).
In general, we observe that mean-field in learned CRF leads to slightly dilated
classification regions in comparison to using Gaussian CRF thereby filling-in the
false negative pixels and also correcting some mis-classified regions.

\begin{figure*}[t!]
  \centering
    \subfigure{%
   \raisebox{2.0em}{
    \includegraphics[width=.06\columnwidth]{figures/supplementary/2bicubic}
   }
  }
  \subfigure{%
    \includegraphics[width=.17\columnwidth]{figures/supplementary/2given_image}
  }
  \subfigure{%
    \includegraphics[width=.17\columnwidth]{figures/supplementary/2ground_truth}
  }
  \subfigure{%
    \includegraphics[width=.17\columnwidth]{figures/supplementary/2bicubic}
  }
  \subfigure{%
    \includegraphics[width=.17\columnwidth]{figures/supplementary/2gauss}
  }
  \subfigure{%
    \includegraphics[width=.17\columnwidth]{figures/supplementary/2learnt}
  }\\
    \subfigure{%
   \raisebox{2.0em}{
    \includegraphics[width=.06\columnwidth]{figures/supplementary/32bicubic}
   }
  }
  \subfigure{%
    \includegraphics[width=.17\columnwidth]{figures/supplementary/32given_image}
  }
  \subfigure{%
    \includegraphics[width=.17\columnwidth]{figures/supplementary/32ground_truth}
  }
  \subfigure{%
    \includegraphics[width=.17\columnwidth]{figures/supplementary/32bicubic}
  }
  \subfigure{%
    \includegraphics[width=.17\columnwidth]{figures/supplementary/32gauss}
  }
  \subfigure{%
    \includegraphics[width=.17\columnwidth]{figures/supplementary/32learnt}
  }\\
  \setcounter{subfigure}{0}
  \small{
  \subfigure[Inp.]{%
  \raisebox{2.0em}{
    \includegraphics[width=.06\columnwidth]{figures/supplementary/41bicubic}
   }
  }
  \subfigure[Guidance]{%
    \includegraphics[width=.17\columnwidth]{figures/supplementary/41given_image}
  }
   \subfigure[GT]{%
    \includegraphics[width=.17\columnwidth]{figures/supplementary/41ground_truth}
  }
  \subfigure[Bicubic]{%
    \includegraphics[width=.17\columnwidth]{figures/supplementary/41bicubic}
  }
  \subfigure[Gauss-BF]{%
    \includegraphics[width=.17\columnwidth]{figures/supplementary/41gauss}
  }
  \subfigure[Learned-BF]{%
    \includegraphics[width=.17\columnwidth]{figures/supplementary/41learnt}
  }
  }
  \mycaption{Depth Upsampling}{Depth $8\times$ upsampling results
  using different upsampling strategies, from left to right,
  (a)~Low-resolution input depth image (Inp.),
  (b)~High-resolution guidance image, (c)~Ground-truth depth; Upsampled depth images with
  (d)~Bicubic interpolation, (e) Gauss bilateral upsampling and, (f)~Learned bilateral
  updampgling (best viewed on screen).}

\label{fig:depth_upsample_visuals}
\end{figure*}

\subsubsection{Material Segmentation}\label{sec:app_material_segmentation}
\label{sec:material_bnn_extra}

In Fig.~\ref{fig:material_visuals-app2}, we present visual results comparing 2 step
mean-field inference with Gaussian and learned pairwise CRF potentials. In
general, we observe that the pixels belonging to dominant classes in the
training data are being more accurately classified with learned CRF. This leads to
a significant improvements in overall pixel accuracy. This also results
in a slight decrease of the accuracy from less frequent class pixels thereby
slightly reducing the average class accuracy with learning. We attribute this
to the type of annotation that is available for this dataset, which is not
for the entire image but for some segments in the image. We have very few
images of the infrequent classes to combat this behaviour during training.

\subsubsection{Experiment Protocols}
\label{sec:protocols}

Table~\ref{tbl:parameters} shows experiment protocols of different experiments.

 \begin{figure*}[t!]
  \centering
  \subfigure[LeNet-7]{
    \includegraphics[width=0.7\columnwidth]{figures/supplementary/lenet_cnn_network}
    }\\
    \subfigure[DeepCNet]{
    \includegraphics[width=\columnwidth]{figures/supplementary/deepcnet_cnn_network}
    }
  \mycaption{CNNs for Character Recognition}
  {Schematic of (top) LeNet-7~\cite{lecun1998mnist} and (bottom) DeepCNet(5,50)~\cite{ciresan2012multi,graham2014spatially} architectures used in Assamese
  character recognition experiments.}
\label{fig:nnrecognition}
\end{figure*}

\definecolor{voc_1}{RGB}{0, 0, 0}
\definecolor{voc_2}{RGB}{128, 0, 0}
\definecolor{voc_3}{RGB}{0, 128, 0}
\definecolor{voc_4}{RGB}{128, 128, 0}
\definecolor{voc_5}{RGB}{0, 0, 128}
\definecolor{voc_6}{RGB}{128, 0, 128}
\definecolor{voc_7}{RGB}{0, 128, 128}
\definecolor{voc_8}{RGB}{128, 128, 128}
\definecolor{voc_9}{RGB}{64, 0, 0}
\definecolor{voc_10}{RGB}{192, 0, 0}
\definecolor{voc_11}{RGB}{64, 128, 0}
\definecolor{voc_12}{RGB}{192, 128, 0}
\definecolor{voc_13}{RGB}{64, 0, 128}
\definecolor{voc_14}{RGB}{192, 0, 128}
\definecolor{voc_15}{RGB}{64, 128, 128}
\definecolor{voc_16}{RGB}{192, 128, 128}
\definecolor{voc_17}{RGB}{0, 64, 0}
\definecolor{voc_18}{RGB}{128, 64, 0}
\definecolor{voc_19}{RGB}{0, 192, 0}
\definecolor{voc_20}{RGB}{128, 192, 0}
\definecolor{voc_21}{RGB}{0, 64, 128}
\definecolor{voc_22}{RGB}{128, 64, 128}

\begin{figure*}[t]
  \centering
  \small{
  \fcolorbox{white}{voc_1}{\rule{0pt}{6pt}\rule{6pt}{0pt}} Background~~
  \fcolorbox{white}{voc_2}{\rule{0pt}{6pt}\rule{6pt}{0pt}} Aeroplane~~
  \fcolorbox{white}{voc_3}{\rule{0pt}{6pt}\rule{6pt}{0pt}} Bicycle~~
  \fcolorbox{white}{voc_4}{\rule{0pt}{6pt}\rule{6pt}{0pt}} Bird~~
  \fcolorbox{white}{voc_5}{\rule{0pt}{6pt}\rule{6pt}{0pt}} Boat~~
  \fcolorbox{white}{voc_6}{\rule{0pt}{6pt}\rule{6pt}{0pt}} Bottle~~
  \fcolorbox{white}{voc_7}{\rule{0pt}{6pt}\rule{6pt}{0pt}} Bus~~
  \fcolorbox{white}{voc_8}{\rule{0pt}{6pt}\rule{6pt}{0pt}} Car~~ \\
  \fcolorbox{white}{voc_9}{\rule{0pt}{6pt}\rule{6pt}{0pt}} Cat~~
  \fcolorbox{white}{voc_10}{\rule{0pt}{6pt}\rule{6pt}{0pt}} Chair~~
  \fcolorbox{white}{voc_11}{\rule{0pt}{6pt}\rule{6pt}{0pt}} Cow~~
  \fcolorbox{white}{voc_12}{\rule{0pt}{6pt}\rule{6pt}{0pt}} Dining Table~~
  \fcolorbox{white}{voc_13}{\rule{0pt}{6pt}\rule{6pt}{0pt}} Dog~~
  \fcolorbox{white}{voc_14}{\rule{0pt}{6pt}\rule{6pt}{0pt}} Horse~~
  \fcolorbox{white}{voc_15}{\rule{0pt}{6pt}\rule{6pt}{0pt}} Motorbike~~
  \fcolorbox{white}{voc_16}{\rule{0pt}{6pt}\rule{6pt}{0pt}} Person~~ \\
  \fcolorbox{white}{voc_17}{\rule{0pt}{6pt}\rule{6pt}{0pt}} Potted Plant~~
  \fcolorbox{white}{voc_18}{\rule{0pt}{6pt}\rule{6pt}{0pt}} Sheep~~
  \fcolorbox{white}{voc_19}{\rule{0pt}{6pt}\rule{6pt}{0pt}} Sofa~~
  \fcolorbox{white}{voc_20}{\rule{0pt}{6pt}\rule{6pt}{0pt}} Train~~
  \fcolorbox{white}{voc_21}{\rule{0pt}{6pt}\rule{6pt}{0pt}} TV monitor~~ \\
  }
  \subfigure{%
    \includegraphics[width=.18\columnwidth]{figures/supplementary/2007_001423_given.jpg}
  }
  \subfigure{%
    \includegraphics[width=.18\columnwidth]{figures/supplementary/2007_001423_gt.png}
  }
  \subfigure{%
    \includegraphics[width=.18\columnwidth]{figures/supplementary/2007_001423_cnn.png}
  }
  \subfigure{%
    \includegraphics[width=.18\columnwidth]{figures/supplementary/2007_001423_gauss.png}
  }
  \subfigure{%
    \includegraphics[width=.18\columnwidth]{figures/supplementary/2007_001423_learnt.png}
  }\\
  \subfigure{%
    \includegraphics[width=.18\columnwidth]{figures/supplementary/2007_001430_given.jpg}
  }
  \subfigure{%
    \includegraphics[width=.18\columnwidth]{figures/supplementary/2007_001430_gt.png}
  }
  \subfigure{%
    \includegraphics[width=.18\columnwidth]{figures/supplementary/2007_001430_cnn.png}
  }
  \subfigure{%
    \includegraphics[width=.18\columnwidth]{figures/supplementary/2007_001430_gauss.png}
  }
  \subfigure{%
    \includegraphics[width=.18\columnwidth]{figures/supplementary/2007_001430_learnt.png}
  }\\
    \subfigure{%
    \includegraphics[width=.18\columnwidth]{figures/supplementary/2007_007996_given.jpg}
  }
  \subfigure{%
    \includegraphics[width=.18\columnwidth]{figures/supplementary/2007_007996_gt.png}
  }
  \subfigure{%
    \includegraphics[width=.18\columnwidth]{figures/supplementary/2007_007996_cnn.png}
  }
  \subfigure{%
    \includegraphics[width=.18\columnwidth]{figures/supplementary/2007_007996_gauss.png}
  }
  \subfigure{%
    \includegraphics[width=.18\columnwidth]{figures/supplementary/2007_007996_learnt.png}
  }\\
   \subfigure{%
    \includegraphics[width=.18\columnwidth]{figures/supplementary/2010_002682_given.jpg}
  }
  \subfigure{%
    \includegraphics[width=.18\columnwidth]{figures/supplementary/2010_002682_gt.png}
  }
  \subfigure{%
    \includegraphics[width=.18\columnwidth]{figures/supplementary/2010_002682_cnn.png}
  }
  \subfigure{%
    \includegraphics[width=.18\columnwidth]{figures/supplementary/2010_002682_gauss.png}
  }
  \subfigure{%
    \includegraphics[width=.18\columnwidth]{figures/supplementary/2010_002682_learnt.png}
  }\\
     \subfigure{%
    \includegraphics[width=.18\columnwidth]{figures/supplementary/2010_004789_given.jpg}
  }
  \subfigure{%
    \includegraphics[width=.18\columnwidth]{figures/supplementary/2010_004789_gt.png}
  }
  \subfigure{%
    \includegraphics[width=.18\columnwidth]{figures/supplementary/2010_004789_cnn.png}
  }
  \subfigure{%
    \includegraphics[width=.18\columnwidth]{figures/supplementary/2010_004789_gauss.png}
  }
  \subfigure{%
    \includegraphics[width=.18\columnwidth]{figures/supplementary/2010_004789_learnt.png}
  }\\
       \subfigure{%
    \includegraphics[width=.18\columnwidth]{figures/supplementary/2007_001311_given.jpg}
  }
  \subfigure{%
    \includegraphics[width=.18\columnwidth]{figures/supplementary/2007_001311_gt.png}
  }
  \subfigure{%
    \includegraphics[width=.18\columnwidth]{figures/supplementary/2007_001311_cnn.png}
  }
  \subfigure{%
    \includegraphics[width=.18\columnwidth]{figures/supplementary/2007_001311_gauss.png}
  }
  \subfigure{%
    \includegraphics[width=.18\columnwidth]{figures/supplementary/2007_001311_learnt.png}
  }\\
  \setcounter{subfigure}{0}
  \subfigure[Input]{%
    \includegraphics[width=.18\columnwidth]{figures/supplementary/2010_003531_given.jpg}
  }
  \subfigure[Ground Truth]{%
    \includegraphics[width=.18\columnwidth]{figures/supplementary/2010_003531_gt.png}
  }
  \subfigure[DeepLab]{%
    \includegraphics[width=.18\columnwidth]{figures/supplementary/2010_003531_cnn.png}
  }
  \subfigure[+GaussCRF]{%
    \includegraphics[width=.18\columnwidth]{figures/supplementary/2010_003531_gauss.png}
  }
  \subfigure[+LearnedCRF]{%
    \includegraphics[width=.18\columnwidth]{figures/supplementary/2010_003531_learnt.png}
  }
  \vspace{-0.3cm}
  \mycaption{Semantic Segmentation}{Example results of semantic segmentation.
  (c)~depicts the unary results before application of MF, (d)~after two steps of MF with Gaussian edge CRF potentials, (e)~after
  two steps of MF with learned edge CRF potentials.}
    \label{fig:semantic_visuals}
\end{figure*}


\definecolor{minc_1}{HTML}{771111}
\definecolor{minc_2}{HTML}{CAC690}
\definecolor{minc_3}{HTML}{EEEEEE}
\definecolor{minc_4}{HTML}{7C8FA6}
\definecolor{minc_5}{HTML}{597D31}
\definecolor{minc_6}{HTML}{104410}
\definecolor{minc_7}{HTML}{BB819C}
\definecolor{minc_8}{HTML}{D0CE48}
\definecolor{minc_9}{HTML}{622745}
\definecolor{minc_10}{HTML}{666666}
\definecolor{minc_11}{HTML}{D54A31}
\definecolor{minc_12}{HTML}{101044}
\definecolor{minc_13}{HTML}{444126}
\definecolor{minc_14}{HTML}{75D646}
\definecolor{minc_15}{HTML}{DD4348}
\definecolor{minc_16}{HTML}{5C8577}
\definecolor{minc_17}{HTML}{C78472}
\definecolor{minc_18}{HTML}{75D6D0}
\definecolor{minc_19}{HTML}{5B4586}
\definecolor{minc_20}{HTML}{C04393}
\definecolor{minc_21}{HTML}{D69948}
\definecolor{minc_22}{HTML}{7370D8}
\definecolor{minc_23}{HTML}{7A3622}
\definecolor{minc_24}{HTML}{000000}

\begin{figure*}[t]
  \centering
  \small{
  \fcolorbox{white}{minc_1}{\rule{0pt}{6pt}\rule{6pt}{0pt}} Brick~~
  \fcolorbox{white}{minc_2}{\rule{0pt}{6pt}\rule{6pt}{0pt}} Carpet~~
  \fcolorbox{white}{minc_3}{\rule{0pt}{6pt}\rule{6pt}{0pt}} Ceramic~~
  \fcolorbox{white}{minc_4}{\rule{0pt}{6pt}\rule{6pt}{0pt}} Fabric~~
  \fcolorbox{white}{minc_5}{\rule{0pt}{6pt}\rule{6pt}{0pt}} Foliage~~
  \fcolorbox{white}{minc_6}{\rule{0pt}{6pt}\rule{6pt}{0pt}} Food~~
  \fcolorbox{white}{minc_7}{\rule{0pt}{6pt}\rule{6pt}{0pt}} Glass~~
  \fcolorbox{white}{minc_8}{\rule{0pt}{6pt}\rule{6pt}{0pt}} Hair~~ \\
  \fcolorbox{white}{minc_9}{\rule{0pt}{6pt}\rule{6pt}{0pt}} Leather~~
  \fcolorbox{white}{minc_10}{\rule{0pt}{6pt}\rule{6pt}{0pt}} Metal~~
  \fcolorbox{white}{minc_11}{\rule{0pt}{6pt}\rule{6pt}{0pt}} Mirror~~
  \fcolorbox{white}{minc_12}{\rule{0pt}{6pt}\rule{6pt}{0pt}} Other~~
  \fcolorbox{white}{minc_13}{\rule{0pt}{6pt}\rule{6pt}{0pt}} Painted~~
  \fcolorbox{white}{minc_14}{\rule{0pt}{6pt}\rule{6pt}{0pt}} Paper~~
  \fcolorbox{white}{minc_15}{\rule{0pt}{6pt}\rule{6pt}{0pt}} Plastic~~\\
  \fcolorbox{white}{minc_16}{\rule{0pt}{6pt}\rule{6pt}{0pt}} Polished Stone~~
  \fcolorbox{white}{minc_17}{\rule{0pt}{6pt}\rule{6pt}{0pt}} Skin~~
  \fcolorbox{white}{minc_18}{\rule{0pt}{6pt}\rule{6pt}{0pt}} Sky~~
  \fcolorbox{white}{minc_19}{\rule{0pt}{6pt}\rule{6pt}{0pt}} Stone~~
  \fcolorbox{white}{minc_20}{\rule{0pt}{6pt}\rule{6pt}{0pt}} Tile~~
  \fcolorbox{white}{minc_21}{\rule{0pt}{6pt}\rule{6pt}{0pt}} Wallpaper~~
  \fcolorbox{white}{minc_22}{\rule{0pt}{6pt}\rule{6pt}{0pt}} Water~~
  \fcolorbox{white}{minc_23}{\rule{0pt}{6pt}\rule{6pt}{0pt}} Wood~~ \\
  }
  \subfigure{%
    \includegraphics[width=.18\columnwidth]{figures/supplementary/000010868_given.jpg}
  }
  \subfigure{%
    \includegraphics[width=.18\columnwidth]{figures/supplementary/000010868_gt.png}
  }
  \subfigure{%
    \includegraphics[width=.18\columnwidth]{figures/supplementary/000010868_cnn.png}
  }
  \subfigure{%
    \includegraphics[width=.18\columnwidth]{figures/supplementary/000010868_gauss.png}
  }
  \subfigure{%
    \includegraphics[width=.18\columnwidth]{figures/supplementary/000010868_learnt.png}
  }\\[-2ex]
  \subfigure{%
    \includegraphics[width=.18\columnwidth]{figures/supplementary/000006011_given.jpg}
  }
  \subfigure{%
    \includegraphics[width=.18\columnwidth]{figures/supplementary/000006011_gt.png}
  }
  \subfigure{%
    \includegraphics[width=.18\columnwidth]{figures/supplementary/000006011_cnn.png}
  }
  \subfigure{%
    \includegraphics[width=.18\columnwidth]{figures/supplementary/000006011_gauss.png}
  }
  \subfigure{%
    \includegraphics[width=.18\columnwidth]{figures/supplementary/000006011_learnt.png}
  }\\[-2ex]
    \subfigure{%
    \includegraphics[width=.18\columnwidth]{figures/supplementary/000008553_given.jpg}
  }
  \subfigure{%
    \includegraphics[width=.18\columnwidth]{figures/supplementary/000008553_gt.png}
  }
  \subfigure{%
    \includegraphics[width=.18\columnwidth]{figures/supplementary/000008553_cnn.png}
  }
  \subfigure{%
    \includegraphics[width=.18\columnwidth]{figures/supplementary/000008553_gauss.png}
  }
  \subfigure{%
    \includegraphics[width=.18\columnwidth]{figures/supplementary/000008553_learnt.png}
  }\\[-2ex]
   \subfigure{%
    \includegraphics[width=.18\columnwidth]{figures/supplementary/000009188_given.jpg}
  }
  \subfigure{%
    \includegraphics[width=.18\columnwidth]{figures/supplementary/000009188_gt.png}
  }
  \subfigure{%
    \includegraphics[width=.18\columnwidth]{figures/supplementary/000009188_cnn.png}
  }
  \subfigure{%
    \includegraphics[width=.18\columnwidth]{figures/supplementary/000009188_gauss.png}
  }
  \subfigure{%
    \includegraphics[width=.18\columnwidth]{figures/supplementary/000009188_learnt.png}
  }\\[-2ex]
  \setcounter{subfigure}{0}
  \subfigure[Input]{%
    \includegraphics[width=.18\columnwidth]{figures/supplementary/000023570_given.jpg}
  }
  \subfigure[Ground Truth]{%
    \includegraphics[width=.18\columnwidth]{figures/supplementary/000023570_gt.png}
  }
  \subfigure[DeepLab]{%
    \includegraphics[width=.18\columnwidth]{figures/supplementary/000023570_cnn.png}
  }
  \subfigure[+GaussCRF]{%
    \includegraphics[width=.18\columnwidth]{figures/supplementary/000023570_gauss.png}
  }
  \subfigure[+LearnedCRF]{%
    \includegraphics[width=.18\columnwidth]{figures/supplementary/000023570_learnt.png}
  }
  \mycaption{Material Segmentation}{Example results of material segmentation.
  (c)~depicts the unary results before application of MF, (d)~after two steps of MF with Gaussian edge CRF potentials, (e)~after two steps of MF with learned edge CRF potentials.}
    \label{fig:material_visuals-app2}
\end{figure*}


\begin{table*}[h]
\tiny
  \centering
    \begin{tabular}{L{2.3cm} L{2.25cm} C{1.5cm} C{0.7cm} C{0.6cm} C{0.7cm} C{0.7cm} C{0.7cm} C{1.6cm} C{0.6cm} C{0.6cm} C{0.6cm}}
      \toprule
& & & & & \multicolumn{3}{c}{\textbf{Data Statistics}} & \multicolumn{4}{c}{\textbf{Training Protocol}} \\

\textbf{Experiment} & \textbf{Feature Types} & \textbf{Feature Scales} & \textbf{Filter Size} & \textbf{Filter Nbr.} & \textbf{Train}  & \textbf{Val.} & \textbf{Test} & \textbf{Loss Type} & \textbf{LR} & \textbf{Batch} & \textbf{Epochs} \\
      \midrule
      \multicolumn{2}{c}{\textbf{Single Bilateral Filter Applications}} & & & & & & & & & \\
      \textbf{2$\times$ Color Upsampling} & Position$_{1}$, Intensity (3D) & 0.13, 0.17 & 65 & 2 & 10581 & 1449 & 1456 & MSE & 1e-06 & 200 & 94.5\\
      \textbf{4$\times$ Color Upsampling} & Position$_{1}$, Intensity (3D) & 0.06, 0.17 & 65 & 2 & 10581 & 1449 & 1456 & MSE & 1e-06 & 200 & 94.5\\
      \textbf{8$\times$ Color Upsampling} & Position$_{1}$, Intensity (3D) & 0.03, 0.17 & 65 & 2 & 10581 & 1449 & 1456 & MSE & 1e-06 & 200 & 94.5\\
      \textbf{16$\times$ Color Upsampling} & Position$_{1}$, Intensity (3D) & 0.02, 0.17 & 65 & 2 & 10581 & 1449 & 1456 & MSE & 1e-06 & 200 & 94.5\\
      \textbf{Depth Upsampling} & Position$_{1}$, Color (5D) & 0.05, 0.02 & 665 & 2 & 795 & 100 & 654 & MSE & 1e-07 & 50 & 251.6\\
      \textbf{Mesh Denoising} & Isomap (4D) & 46.00 & 63 & 2 & 1000 & 200 & 500 & MSE & 100 & 10 & 100.0 \\
      \midrule
      \multicolumn{2}{c}{\textbf{DenseCRF Applications}} & & & & & & & & &\\
      \multicolumn{2}{l}{\textbf{Semantic Segmentation}} & & & & & & & & &\\
      \textbf{- 1step MF} & Position$_{1}$, Color (5D); Position$_{1}$ (2D) & 0.01, 0.34; 0.34  & 665; 19  & 2; 2 & 10581 & 1449 & 1456 & Logistic & 0.1 & 5 & 1.4 \\
      \textbf{- 2step MF} & Position$_{1}$, Color (5D); Position$_{1}$ (2D) & 0.01, 0.34; 0.34 & 665; 19 & 2; 2 & 10581 & 1449 & 1456 & Logistic & 0.1 & 5 & 1.4 \\
      \textbf{- \textit{loose} 2step MF} & Position$_{1}$, Color (5D); Position$_{1}$ (2D) & 0.01, 0.34; 0.34 & 665; 19 & 2; 2 &10581 & 1449 & 1456 & Logistic & 0.1 & 5 & +1.9  \\ \\
      \multicolumn{2}{l}{\textbf{Material Segmentation}} & & & & & & & & &\\
      \textbf{- 1step MF} & Position$_{2}$, Lab-Color (5D) & 5.00, 0.05, 0.30  & 665 & 2 & 928 & 150 & 1798 & Weighted Logistic & 1e-04 & 24 & 2.6 \\
      \textbf{- 2step MF} & Position$_{2}$, Lab-Color (5D) & 5.00, 0.05, 0.30 & 665 & 2 & 928 & 150 & 1798 & Weighted Logistic & 1e-04 & 12 & +0.7 \\
      \textbf{- \textit{loose} 2step MF} & Position$_{2}$, Lab-Color (5D) & 5.00, 0.05, 0.30 & 665 & 2 & 928 & 150 & 1798 & Weighted Logistic & 1e-04 & 12 & +0.2\\
      \midrule
      \multicolumn{2}{c}{\textbf{Neural Network Applications}} & & & & & & & & &\\
      \textbf{Tiles: CNN-9$\times$9} & - & - & 81 & 4 & 10000 & 1000 & 1000 & Logistic & 0.01 & 100 & 500.0 \\
      \textbf{Tiles: CNN-13$\times$13} & - & - & 169 & 6 & 10000 & 1000 & 1000 & Logistic & 0.01 & 100 & 500.0 \\
      \textbf{Tiles: CNN-17$\times$17} & - & - & 289 & 8 & 10000 & 1000 & 1000 & Logistic & 0.01 & 100 & 500.0 \\
      \textbf{Tiles: CNN-21$\times$21} & - & - & 441 & 10 & 10000 & 1000 & 1000 & Logistic & 0.01 & 100 & 500.0 \\
      \textbf{Tiles: BNN} & Position$_{1}$, Color (5D) & 0.05, 0.04 & 63 & 1 & 10000 & 1000 & 1000 & Logistic & 0.01 & 100 & 30.0 \\
      \textbf{LeNet} & - & - & 25 & 2 & 5490 & 1098 & 1647 & Logistic & 0.1 & 100 & 182.2 \\
      \textbf{Crop-LeNet} & - & - & 25 & 2 & 5490 & 1098 & 1647 & Logistic & 0.1 & 100 & 182.2 \\
      \textbf{BNN-LeNet} & Position$_{2}$ (2D) & 20.00 & 7 & 1 & 5490 & 1098 & 1647 & Logistic & 0.1 & 100 & 182.2 \\
      \textbf{DeepCNet} & - & - & 9 & 1 & 5490 & 1098 & 1647 & Logistic & 0.1 & 100 & 182.2 \\
      \textbf{Crop-DeepCNet} & - & - & 9 & 1 & 5490 & 1098 & 1647 & Logistic & 0.1 & 100 & 182.2 \\
      \textbf{BNN-DeepCNet} & Position$_{2}$ (2D) & 40.00  & 7 & 1 & 5490 & 1098 & 1647 & Logistic & 0.1 & 100 & 182.2 \\
      \bottomrule
      \\
    \end{tabular}
    \mycaption{Experiment Protocols} {Experiment protocols for the different experiments presented in this work. \textbf{Feature Types}:
    Feature spaces used for the bilateral convolutions. Position$_1$ corresponds to un-normalized pixel positions whereas Position$_2$ corresponds
    to pixel positions normalized to $[0,1]$ with respect to the given image. \textbf{Feature Scales}: Cross-validated scales for the features used.
     \textbf{Filter Size}: Number of elements in the filter that is being learned. \textbf{Filter Nbr.}: Half-width of the filter. \textbf{Train},
     \textbf{Val.} and \textbf{Test} corresponds to the number of train, validation and test images used in the experiment. \textbf{Loss Type}: Type
     of loss used for back-propagation. ``MSE'' corresponds to Euclidean mean squared error loss and ``Logistic'' corresponds to multinomial logistic
     loss. ``Weighted Logistic'' is the class-weighted multinomial logistic loss. We weighted the loss with inverse class probability for material
     segmentation task due to the small availability of training data with class imbalance. \textbf{LR}: Fixed learning rate used in stochastic gradient
     descent. \textbf{Batch}: Number of images used in one parameter update step. \textbf{Epochs}: Number of training epochs. In all the experiments,
     we used fixed momentum of 0.9 and weight decay of 0.0005 for stochastic gradient descent. ```Color Upsampling'' experiments in this Table corresponds
     to those performed on Pascal VOC12 dataset images. For all experiments using Pascal VOC12 images, we use extended
     training segmentation dataset available from~\cite{hariharan2011moredata}, and used standard validation and test splits
     from the main dataset~\cite{voc2012segmentation}.}
  \label{tbl:parameters}
\end{table*}

\clearpage

\section{Parameters and Additional Results for Video Propagation Networks}

In this Section, we present experiment protocols and additional qualitative results for experiments
on video object segmentation, semantic video segmentation and video color
propagation. Table~\ref{tbl:parameters_supp} shows the feature scales and other parameters used in different experiments.
Figures~\ref{fig:video_seg_pos_supp} show some qualitative results on video object segmentation
with some failure cases in Fig.~\ref{fig:video_seg_neg_supp}.
Figure~\ref{fig:semantic_visuals_supp} shows some qualitative results on semantic video segmentation and
Fig.~\ref{fig:color_visuals_supp} shows results on video color propagation.

\newcolumntype{L}[1]{>{\raggedright\let\newline\\\arraybackslash\hspace{0pt}}b{#1}}
\newcolumntype{C}[1]{>{\centering\let\newline\\\arraybackslash\hspace{0pt}}b{#1}}
\newcolumntype{R}[1]{>{\raggedleft\let\newline\\\arraybackslash\hspace{0pt}}b{#1}}

\begin{table*}[h]
\tiny
  \centering
    \begin{tabular}{L{3.0cm} L{2.4cm} L{2.8cm} L{2.8cm} C{0.5cm} C{1.0cm} L{1.2cm}}
      \toprule
\textbf{Experiment} & \textbf{Feature Type} & \textbf{Feature Scale-1, $\Lambda_a$} & \textbf{Feature Scale-2, $\Lambda_b$} & \textbf{$\alpha$} & \textbf{Input Frames} & \textbf{Loss Type} \\
      \midrule
      \textbf{Video Object Segmentation} & ($x,y,Y,Cb,Cr,t$) & (0.02,0.02,0.07,0.4,0.4,0.01) & (0.03,0.03,0.09,0.5,0.5,0.2) & 0.5 & 9 & Logistic\\
      \midrule
      \textbf{Semantic Video Segmentation} & & & & & \\
      \textbf{with CNN1~\cite{yu2015multi}-NoFlow} & ($x,y,R,G,B,t$) & (0.08,0.08,0.2,0.2,0.2,0.04) & (0.11,0.11,0.2,0.2,0.2,0.04) & 0.5 & 3 & Logistic \\
      \textbf{with CNN1~\cite{yu2015multi}-Flow} & ($x+u_x,y+u_y,R,G,B,t$) & (0.11,0.11,0.14,0.14,0.14,0.03) & (0.08,0.08,0.12,0.12,0.12,0.01) & 0.65 & 3 & Logistic\\
      \textbf{with CNN2~\cite{richter2016playing}-Flow} & ($x+u_x,y+u_y,R,G,B,t$) & (0.08,0.08,0.2,0.2,0.2,0.04) & (0.09,0.09,0.25,0.25,0.25,0.03) & 0.5 & 4 & Logistic\\
      \midrule
      \textbf{Video Color Propagation} & ($x,y,I,t$)  & (0.04,0.04,0.2,0.04) & No second kernel & 1 & 4 & MSE\\
      \bottomrule
      \\
    \end{tabular}
    \mycaption{Experiment Protocols} {Experiment protocols for the different experiments presented in this work. \textbf{Feature Types}:
    Feature spaces used for the bilateral convolutions, with position ($x,y$) and color
    ($R,G,B$ or $Y,Cb,Cr$) features $\in [0,255]$. $u_x$, $u_y$ denotes optical flow with respect
    to the present frame and $I$ denotes grayscale intensity.
    \textbf{Feature Scales ($\Lambda_a, \Lambda_b$)}: Cross-validated scales for the features used.
    \textbf{$\alpha$}: Exponential time decay for the input frames.
    \textbf{Input Frames}: Number of input frames for VPN.
    \textbf{Loss Type}: Type
     of loss used for back-propagation. ``MSE'' corresponds to Euclidean mean squared error loss and ``Logistic'' corresponds to multinomial logistic loss.}
  \label{tbl:parameters_supp}
\end{table*}

% \begin{figure}[th!]
% \begin{center}
%   \centerline{\includegraphics[width=\textwidth]{figures/video_seg_visuals_supp_small.pdf}}
%     \mycaption{Video Object Segmentation}
%     {Shown are the different frames in example videos with the corresponding
%     ground truth (GT) masks, predictions from BVS~\cite{marki2016bilateral},
%     OFL~\cite{tsaivideo}, VPN (VPN-Stage2) and VPN-DLab (VPN-DeepLab) models.}
%     \label{fig:video_seg_small_supp}
% \end{center}
% \vspace{-1.0cm}
% \end{figure}

\begin{figure}[th!]
\begin{center}
  \centerline{\includegraphics[width=0.7\textwidth]{figures/video_seg_visuals_supp_positive.pdf}}
    \mycaption{Video Object Segmentation}
    {Shown are the different frames in example videos with the corresponding
    ground truth (GT) masks, predictions from BVS~\cite{marki2016bilateral},
    OFL~\cite{tsaivideo}, VPN (VPN-Stage2) and VPN-DLab (VPN-DeepLab) models.}
    \label{fig:video_seg_pos_supp}
\end{center}
\vspace{-1.0cm}
\end{figure}

\begin{figure}[th!]
\begin{center}
  \centerline{\includegraphics[width=0.7\textwidth]{figures/video_seg_visuals_supp_negative.pdf}}
    \mycaption{Failure Cases for Video Object Segmentation}
    {Shown are the different frames in example videos with the corresponding
    ground truth (GT) masks, predictions from BVS~\cite{marki2016bilateral},
    OFL~\cite{tsaivideo}, VPN (VPN-Stage2) and VPN-DLab (VPN-DeepLab) models.}
    \label{fig:video_seg_neg_supp}
\end{center}
\vspace{-1.0cm}
\end{figure}

\begin{figure}[th!]
\begin{center}
  \centerline{\includegraphics[width=0.9\textwidth]{figures/supp_semantic_visual.pdf}}
    \mycaption{Semantic Video Segmentation}
    {Input video frames and the corresponding ground truth (GT)
    segmentation together with the predictions of CNN~\cite{yu2015multi} and with
    VPN-Flow.}
    \label{fig:semantic_visuals_supp}
\end{center}
\vspace{-0.7cm}
\end{figure}

\begin{figure}[th!]
\begin{center}
  \centerline{\includegraphics[width=\textwidth]{figures/colorization_visuals_supp.pdf}}
  \mycaption{Video Color Propagation}
  {Input grayscale video frames and corresponding ground-truth (GT) color images
  together with color predictions of Levin et al.~\cite{levin2004colorization} and VPN-Stage1 models.}
  \label{fig:color_visuals_supp}
\end{center}
\vspace{-0.7cm}
\end{figure}

\clearpage

\section{Additional Material for Bilateral Inception Networks}
\label{sec:binception-app}

In this section of the Appendix, we first discuss the use of approximate bilateral
filtering in BI modules (Sec.~\ref{sec:lattice}).
Later, we present some qualitative results using different models for the approach presented in
Chapter~\ref{chap:binception} (Sec.~\ref{sec:qualitative-app}).

\subsection{Approximate Bilateral Filtering}
\label{sec:lattice}

The bilateral inception module presented in Chapter~\ref{chap:binception} computes a matrix-vector
product between a Gaussian filter $K$ and a vector of activations $\bz_c$.
Bilateral filtering is an important operation and many algorithmic techniques have been
proposed to speed-up this operation~\cite{paris2006fast,adams2010fast,gastal2011domain}.
In the main paper we opted to implement what can be considered the
brute-force variant of explicitly constructing $K$ and then using BLAS to compute the
matrix-vector product. This resulted in a few millisecond operation.
The explicit way to compute is possible due to the
reduction to super-pixels, e.g., it would not work for DenseCRF variants
that operate on the full image resolution.

Here, we present experiments where we use the fast approximate bilateral filtering
algorithm of~\cite{adams2010fast}, which is also used in Chapter~\ref{chap:bnn}
for learning sparse high dimensional filters. This
choice allows for larger dimensions of matrix-vector multiplication. The reason for choosing
the explicit multiplication in Chapter~\ref{chap:binception} was that it was computationally faster.
For the small sizes of the involved matrices and vectors, the explicit computation is sufficient and we had no
GPU implementation of an approximate technique that matched this runtime. Also it
is conceptually easier and the gradient to the feature transformations ($\Lambda \mathbf{f}$) is
obtained using standard matrix calculus.

\subsubsection{Experiments}

We modified the existing segmentation architectures analogous to those in Chapter~\ref{chap:binception}.
The main difference is that, here, the inception modules use the lattice
approximation~\cite{adams2010fast} to compute the bilateral filtering.
Using the lattice approximation did not allow us to back-propagate through feature transformations ($\Lambda$)
and thus we used hand-specified feature scales as will be explained later.
Specifically, we take CNN architectures from the works
of~\cite{chen2014semantic,zheng2015conditional,bell2015minc} and insert the BI modules between
the spatial FC layers.
We use superpixels from~\cite{DollarICCV13edges}
for all the experiments with the lattice approximation. Experiments are
performed using Caffe neural network framework~\cite{jia2014caffe}.

\begin{table}
  \small
  \centering
  \begin{tabular}{p{5.5cm}>{\raggedright\arraybackslash}p{1.4cm}>{\centering\arraybackslash}p{2.2cm}}
    \toprule
		\textbf{Model} & \emph{IoU} & \emph{Runtime}(ms) \\
    \midrule

    %%%%%%%%%%%% Scores computed by us)%%%%%%%%%%%%
		\deeplablargefov & 68.9 & 145ms\\
    \midrule
    \bi{7}{2}-\bi{8}{10}& \textbf{73.8} & +600 \\
    \midrule
    \deeplablargefovcrf~\cite{chen2014semantic} & 72.7 & +830\\
    \deeplabmsclargefovcrf~\cite{chen2014semantic} & \textbf{73.6} & +880\\
    DeepLab-EdgeNet~\cite{chen2015semantic} & 71.7 & +30\\
    DeepLab-EdgeNet-CRF~\cite{chen2015semantic} & \textbf{73.6} & +860\\
  \bottomrule \\
  \end{tabular}
  \mycaption{Semantic Segmentation using the DeepLab model}
  {IoU scores on the Pascal VOC12 segmentation test dataset
  with different models and our modified inception model.
  Also shown are the corresponding runtimes in milliseconds. Runtimes
  also include superpixel computations (300 ms with Dollar superpixels~\cite{DollarICCV13edges})}
  \label{tab:largefovresults}
\end{table}

\paragraph{Semantic Segmentation}
The experiments in this section use the Pascal VOC12 segmentation dataset~\cite{voc2012segmentation} with 21 object classes and the images have a maximum resolution of 0.25 megapixels.
For all experiments on VOC12, we train using the extended training set of
10581 images collected by~\cite{hariharan2011moredata}.
We modified the \deeplab~network architecture of~\cite{chen2014semantic} and
the CRFasRNN architecture from~\cite{zheng2015conditional} which uses a CNN with
deconvolution layers followed by DenseCRF trained end-to-end.

\paragraph{DeepLab Model}\label{sec:deeplabmodel}
We experimented with the \bi{7}{2}-\bi{8}{10} inception model.
Results using the~\deeplab~model are summarized in Tab.~\ref{tab:largefovresults}.
Although we get similar improvements with inception modules as with the
explicit kernel computation, using lattice approximation is slower.

\begin{table}
  \small
  \centering
  \begin{tabular}{p{6.4cm}>{\raggedright\arraybackslash}p{1.8cm}>{\raggedright\arraybackslash}p{1.8cm}}
    \toprule
    \textbf{Model} & \emph{IoU (Val)} & \emph{IoU (Test)}\\
    \midrule
    %%%%%%%%%%%% Scores computed by us)%%%%%%%%%%%%
    CNN &  67.5 & - \\
    \deconv (CNN+Deconvolutions) & 69.8 & 72.0 \\
    \midrule
    \bi{3}{6}-\bi{4}{6}-\bi{7}{2}-\bi{8}{6}& 71.9 & - \\
    \bi{3}{6}-\bi{4}{6}-\bi{7}{2}-\bi{8}{6}-\gi{6}& 73.6 &  \href{http://host.robots.ox.ac.uk:8080/anonymous/VOTV5E.html}{\textbf{75.2}}\\
    \midrule
    \deconvcrf (CRF-RNN)~\cite{zheng2015conditional} & 73.0 & 74.7\\
    Context-CRF-RNN~\cite{yu2015multi} & ~~ - ~ & \textbf{75.3} \\
    \bottomrule \\
  \end{tabular}
  \mycaption{Semantic Segmentation using the CRFasRNN model}{IoU score corresponding to different models
  on Pascal VOC12 reduced validation / test segmentation dataset. The reduced validation set consists of 346 images
  as used in~\cite{zheng2015conditional} where we adapted the model from.}
  \label{tab:deconvresults-app}
\end{table}

\paragraph{CRFasRNN Model}\label{sec:deepinception}
We add BI modules after score-pool3, score-pool4, \fc{7} and \fc{8} $1\times1$ convolution layers
resulting in the \bi{3}{6}-\bi{4}{6}-\bi{7}{2}-\bi{8}{6}
model and also experimented with another variant where $BI_8$ is followed by another inception
module, G$(6)$, with 6 Gaussian kernels.
Note that here also we discarded both deconvolution and DenseCRF parts of the original model~\cite{zheng2015conditional}
and inserted the BI modules in the base CNN and found similar improvements compared to the inception modules with explicit
kernel computaion. See Tab.~\ref{tab:deconvresults-app} for results on the CRFasRNN model.

\paragraph{Material Segmentation}
Table~\ref{tab:mincresults-app} shows the results on the MINC dataset~\cite{bell2015minc}
obtained by modifying the AlexNet architecture with our inception modules. We observe
similar improvements as with explicit kernel construction.
For this model, we do not provide any learned setup due to very limited segment training
data. The weights to combine outputs in the bilateral inception layer are
found by validation on the validation set.

\begin{table}[t]
  \small
  \centering
  \begin{tabular}{p{3.5cm}>{\centering\arraybackslash}p{4.0cm}}
    \toprule
    \textbf{Model} & Class / Total accuracy\\
    \midrule

    %%%%%%%%%%%% Scores computed by us)%%%%%%%%%%%%
    AlexNet CNN & 55.3 / 58.9 \\
    \midrule
    \bi{7}{2}-\bi{8}{6}& 68.5 / 71.8 \\
    \bi{7}{2}-\bi{8}{6}-G$(6)$& 67.6 / 73.1 \\
    \midrule
    AlexNet-CRF & 65.5 / 71.0 \\
    \bottomrule \\
  \end{tabular}
  \mycaption{Material Segmentation using AlexNet}{Pixel accuracy of different models on
  the MINC material segmentation test dataset~\cite{bell2015minc}.}
  \label{tab:mincresults-app}
\end{table}

\paragraph{Scales of Bilateral Inception Modules}
\label{sec:scales}

Unlike the explicit kernel technique presented in the main text (Chapter~\ref{chap:binception}),
we didn't back-propagate through feature transformation ($\Lambda$)
using the approximate bilateral filter technique.
So, the feature scales are hand-specified and validated, which are as follows.
The optimal scale values for the \bi{7}{2}-\bi{8}{2} model are found by validation for the best performance which are
$\sigma_{xy}$ = (0.1, 0.1) for the spatial (XY) kernel and $\sigma_{rgbxy}$ = (0.1, 0.1, 0.1, 0.01, 0.01) for color and position (RGBXY)  kernel.
Next, as more kernels are added to \bi{8}{2}, we set scales to be $\alpha$*($\sigma_{xy}$, $\sigma_{rgbxy}$).
The value of $\alpha$ is chosen as  1, 0.5, 0.1, 0.05, 0.1, at uniform interval, for the \bi{8}{10} bilateral inception module.


\subsection{Qualitative Results}
\label{sec:qualitative-app}

In this section, we present more qualitative results obtained using the BI module with explicit
kernel computation technique presented in Chapter~\ref{chap:binception}. Results on the Pascal VOC12
dataset~\cite{voc2012segmentation} using the DeepLab-LargeFOV model are shown in Fig.~\ref{fig:semantic_visuals-app},
followed by the results on MINC dataset~\cite{bell2015minc}
in Fig.~\ref{fig:material_visuals-app} and on
Cityscapes dataset~\cite{Cordts2015Cvprw} in Fig.~\ref{fig:street_visuals-app}.


\definecolor{voc_1}{RGB}{0, 0, 0}
\definecolor{voc_2}{RGB}{128, 0, 0}
\definecolor{voc_3}{RGB}{0, 128, 0}
\definecolor{voc_4}{RGB}{128, 128, 0}
\definecolor{voc_5}{RGB}{0, 0, 128}
\definecolor{voc_6}{RGB}{128, 0, 128}
\definecolor{voc_7}{RGB}{0, 128, 128}
\definecolor{voc_8}{RGB}{128, 128, 128}
\definecolor{voc_9}{RGB}{64, 0, 0}
\definecolor{voc_10}{RGB}{192, 0, 0}
\definecolor{voc_11}{RGB}{64, 128, 0}
\definecolor{voc_12}{RGB}{192, 128, 0}
\definecolor{voc_13}{RGB}{64, 0, 128}
\definecolor{voc_14}{RGB}{192, 0, 128}
\definecolor{voc_15}{RGB}{64, 128, 128}
\definecolor{voc_16}{RGB}{192, 128, 128}
\definecolor{voc_17}{RGB}{0, 64, 0}
\definecolor{voc_18}{RGB}{128, 64, 0}
\definecolor{voc_19}{RGB}{0, 192, 0}
\definecolor{voc_20}{RGB}{128, 192, 0}
\definecolor{voc_21}{RGB}{0, 64, 128}
\definecolor{voc_22}{RGB}{128, 64, 128}

\begin{figure*}[!ht]
  \small
  \centering
  \fcolorbox{white}{voc_1}{\rule{0pt}{4pt}\rule{4pt}{0pt}} Background~~
  \fcolorbox{white}{voc_2}{\rule{0pt}{4pt}\rule{4pt}{0pt}} Aeroplane~~
  \fcolorbox{white}{voc_3}{\rule{0pt}{4pt}\rule{4pt}{0pt}} Bicycle~~
  \fcolorbox{white}{voc_4}{\rule{0pt}{4pt}\rule{4pt}{0pt}} Bird~~
  \fcolorbox{white}{voc_5}{\rule{0pt}{4pt}\rule{4pt}{0pt}} Boat~~
  \fcolorbox{white}{voc_6}{\rule{0pt}{4pt}\rule{4pt}{0pt}} Bottle~~
  \fcolorbox{white}{voc_7}{\rule{0pt}{4pt}\rule{4pt}{0pt}} Bus~~
  \fcolorbox{white}{voc_8}{\rule{0pt}{4pt}\rule{4pt}{0pt}} Car~~\\
  \fcolorbox{white}{voc_9}{\rule{0pt}{4pt}\rule{4pt}{0pt}} Cat~~
  \fcolorbox{white}{voc_10}{\rule{0pt}{4pt}\rule{4pt}{0pt}} Chair~~
  \fcolorbox{white}{voc_11}{\rule{0pt}{4pt}\rule{4pt}{0pt}} Cow~~
  \fcolorbox{white}{voc_12}{\rule{0pt}{4pt}\rule{4pt}{0pt}} Dining Table~~
  \fcolorbox{white}{voc_13}{\rule{0pt}{4pt}\rule{4pt}{0pt}} Dog~~
  \fcolorbox{white}{voc_14}{\rule{0pt}{4pt}\rule{4pt}{0pt}} Horse~~
  \fcolorbox{white}{voc_15}{\rule{0pt}{4pt}\rule{4pt}{0pt}} Motorbike~~
  \fcolorbox{white}{voc_16}{\rule{0pt}{4pt}\rule{4pt}{0pt}} Person~~\\
  \fcolorbox{white}{voc_17}{\rule{0pt}{4pt}\rule{4pt}{0pt}} Potted Plant~~
  \fcolorbox{white}{voc_18}{\rule{0pt}{4pt}\rule{4pt}{0pt}} Sheep~~
  \fcolorbox{white}{voc_19}{\rule{0pt}{4pt}\rule{4pt}{0pt}} Sofa~~
  \fcolorbox{white}{voc_20}{\rule{0pt}{4pt}\rule{4pt}{0pt}} Train~~
  \fcolorbox{white}{voc_21}{\rule{0pt}{4pt}\rule{4pt}{0pt}} TV monitor~~\\


  \subfigure{%
    \includegraphics[width=.15\columnwidth]{figures/supplementary/2008_001308_given.png}
  }
  \subfigure{%
    \includegraphics[width=.15\columnwidth]{figures/supplementary/2008_001308_sp.png}
  }
  \subfigure{%
    \includegraphics[width=.15\columnwidth]{figures/supplementary/2008_001308_gt.png}
  }
  \subfigure{%
    \includegraphics[width=.15\columnwidth]{figures/supplementary/2008_001308_cnn.png}
  }
  \subfigure{%
    \includegraphics[width=.15\columnwidth]{figures/supplementary/2008_001308_crf.png}
  }
  \subfigure{%
    \includegraphics[width=.15\columnwidth]{figures/supplementary/2008_001308_ours.png}
  }\\[-2ex]


  \subfigure{%
    \includegraphics[width=.15\columnwidth]{figures/supplementary/2008_001821_given.png}
  }
  \subfigure{%
    \includegraphics[width=.15\columnwidth]{figures/supplementary/2008_001821_sp.png}
  }
  \subfigure{%
    \includegraphics[width=.15\columnwidth]{figures/supplementary/2008_001821_gt.png}
  }
  \subfigure{%
    \includegraphics[width=.15\columnwidth]{figures/supplementary/2008_001821_cnn.png}
  }
  \subfigure{%
    \includegraphics[width=.15\columnwidth]{figures/supplementary/2008_001821_crf.png}
  }
  \subfigure{%
    \includegraphics[width=.15\columnwidth]{figures/supplementary/2008_001821_ours.png}
  }\\[-2ex]



  \subfigure{%
    \includegraphics[width=.15\columnwidth]{figures/supplementary/2008_004612_given.png}
  }
  \subfigure{%
    \includegraphics[width=.15\columnwidth]{figures/supplementary/2008_004612_sp.png}
  }
  \subfigure{%
    \includegraphics[width=.15\columnwidth]{figures/supplementary/2008_004612_gt.png}
  }
  \subfigure{%
    \includegraphics[width=.15\columnwidth]{figures/supplementary/2008_004612_cnn.png}
  }
  \subfigure{%
    \includegraphics[width=.15\columnwidth]{figures/supplementary/2008_004612_crf.png}
  }
  \subfigure{%
    \includegraphics[width=.15\columnwidth]{figures/supplementary/2008_004612_ours.png}
  }\\[-2ex]


  \subfigure{%
    \includegraphics[width=.15\columnwidth]{figures/supplementary/2009_001008_given.png}
  }
  \subfigure{%
    \includegraphics[width=.15\columnwidth]{figures/supplementary/2009_001008_sp.png}
  }
  \subfigure{%
    \includegraphics[width=.15\columnwidth]{figures/supplementary/2009_001008_gt.png}
  }
  \subfigure{%
    \includegraphics[width=.15\columnwidth]{figures/supplementary/2009_001008_cnn.png}
  }
  \subfigure{%
    \includegraphics[width=.15\columnwidth]{figures/supplementary/2009_001008_crf.png}
  }
  \subfigure{%
    \includegraphics[width=.15\columnwidth]{figures/supplementary/2009_001008_ours.png}
  }\\[-2ex]




  \subfigure{%
    \includegraphics[width=.15\columnwidth]{figures/supplementary/2009_004497_given.png}
  }
  \subfigure{%
    \includegraphics[width=.15\columnwidth]{figures/supplementary/2009_004497_sp.png}
  }
  \subfigure{%
    \includegraphics[width=.15\columnwidth]{figures/supplementary/2009_004497_gt.png}
  }
  \subfigure{%
    \includegraphics[width=.15\columnwidth]{figures/supplementary/2009_004497_cnn.png}
  }
  \subfigure{%
    \includegraphics[width=.15\columnwidth]{figures/supplementary/2009_004497_crf.png}
  }
  \subfigure{%
    \includegraphics[width=.15\columnwidth]{figures/supplementary/2009_004497_ours.png}
  }\\[-2ex]



  \setcounter{subfigure}{0}
  \subfigure[\scriptsize Input]{%
    \includegraphics[width=.15\columnwidth]{figures/supplementary/2010_001327_given.png}
  }
  \subfigure[\scriptsize Superpixels]{%
    \includegraphics[width=.15\columnwidth]{figures/supplementary/2010_001327_sp.png}
  }
  \subfigure[\scriptsize GT]{%
    \includegraphics[width=.15\columnwidth]{figures/supplementary/2010_001327_gt.png}
  }
  \subfigure[\scriptsize Deeplab]{%
    \includegraphics[width=.15\columnwidth]{figures/supplementary/2010_001327_cnn.png}
  }
  \subfigure[\scriptsize +DenseCRF]{%
    \includegraphics[width=.15\columnwidth]{figures/supplementary/2010_001327_crf.png}
  }
  \subfigure[\scriptsize Using BI]{%
    \includegraphics[width=.15\columnwidth]{figures/supplementary/2010_001327_ours.png}
  }
  \mycaption{Semantic Segmentation}{Example results of semantic segmentation
  on the Pascal VOC12 dataset.
  (d)~depicts the DeepLab CNN result, (e)~CNN + 10 steps of mean-field inference,
  (f~result obtained with bilateral inception (BI) modules (\bi{6}{2}+\bi{7}{6}) between \fc~layers.}
  \label{fig:semantic_visuals-app}
\end{figure*}


\definecolor{minc_1}{HTML}{771111}
\definecolor{minc_2}{HTML}{CAC690}
\definecolor{minc_3}{HTML}{EEEEEE}
\definecolor{minc_4}{HTML}{7C8FA6}
\definecolor{minc_5}{HTML}{597D31}
\definecolor{minc_6}{HTML}{104410}
\definecolor{minc_7}{HTML}{BB819C}
\definecolor{minc_8}{HTML}{D0CE48}
\definecolor{minc_9}{HTML}{622745}
\definecolor{minc_10}{HTML}{666666}
\definecolor{minc_11}{HTML}{D54A31}
\definecolor{minc_12}{HTML}{101044}
\definecolor{minc_13}{HTML}{444126}
\definecolor{minc_14}{HTML}{75D646}
\definecolor{minc_15}{HTML}{DD4348}
\definecolor{minc_16}{HTML}{5C8577}
\definecolor{minc_17}{HTML}{C78472}
\definecolor{minc_18}{HTML}{75D6D0}
\definecolor{minc_19}{HTML}{5B4586}
\definecolor{minc_20}{HTML}{C04393}
\definecolor{minc_21}{HTML}{D69948}
\definecolor{minc_22}{HTML}{7370D8}
\definecolor{minc_23}{HTML}{7A3622}
\definecolor{minc_24}{HTML}{000000}

\begin{figure*}[!ht]
  \small % scriptsize
  \centering
  \fcolorbox{white}{minc_1}{\rule{0pt}{4pt}\rule{4pt}{0pt}} Brick~~
  \fcolorbox{white}{minc_2}{\rule{0pt}{4pt}\rule{4pt}{0pt}} Carpet~~
  \fcolorbox{white}{minc_3}{\rule{0pt}{4pt}\rule{4pt}{0pt}} Ceramic~~
  \fcolorbox{white}{minc_4}{\rule{0pt}{4pt}\rule{4pt}{0pt}} Fabric~~
  \fcolorbox{white}{minc_5}{\rule{0pt}{4pt}\rule{4pt}{0pt}} Foliage~~
  \fcolorbox{white}{minc_6}{\rule{0pt}{4pt}\rule{4pt}{0pt}} Food~~
  \fcolorbox{white}{minc_7}{\rule{0pt}{4pt}\rule{4pt}{0pt}} Glass~~
  \fcolorbox{white}{minc_8}{\rule{0pt}{4pt}\rule{4pt}{0pt}} Hair~~\\
  \fcolorbox{white}{minc_9}{\rule{0pt}{4pt}\rule{4pt}{0pt}} Leather~~
  \fcolorbox{white}{minc_10}{\rule{0pt}{4pt}\rule{4pt}{0pt}} Metal~~
  \fcolorbox{white}{minc_11}{\rule{0pt}{4pt}\rule{4pt}{0pt}} Mirror~~
  \fcolorbox{white}{minc_12}{\rule{0pt}{4pt}\rule{4pt}{0pt}} Other~~
  \fcolorbox{white}{minc_13}{\rule{0pt}{4pt}\rule{4pt}{0pt}} Painted~~
  \fcolorbox{white}{minc_14}{\rule{0pt}{4pt}\rule{4pt}{0pt}} Paper~~
  \fcolorbox{white}{minc_15}{\rule{0pt}{4pt}\rule{4pt}{0pt}} Plastic~~\\
  \fcolorbox{white}{minc_16}{\rule{0pt}{4pt}\rule{4pt}{0pt}} Polished Stone~~
  \fcolorbox{white}{minc_17}{\rule{0pt}{4pt}\rule{4pt}{0pt}} Skin~~
  \fcolorbox{white}{minc_18}{\rule{0pt}{4pt}\rule{4pt}{0pt}} Sky~~
  \fcolorbox{white}{minc_19}{\rule{0pt}{4pt}\rule{4pt}{0pt}} Stone~~
  \fcolorbox{white}{minc_20}{\rule{0pt}{4pt}\rule{4pt}{0pt}} Tile~~
  \fcolorbox{white}{minc_21}{\rule{0pt}{4pt}\rule{4pt}{0pt}} Wallpaper~~
  \fcolorbox{white}{minc_22}{\rule{0pt}{4pt}\rule{4pt}{0pt}} Water~~
  \fcolorbox{white}{minc_23}{\rule{0pt}{4pt}\rule{4pt}{0pt}} Wood~~\\
  \subfigure{%
    \includegraphics[width=.15\columnwidth]{figures/supplementary/000008468_given.png}
  }
  \subfigure{%
    \includegraphics[width=.15\columnwidth]{figures/supplementary/000008468_sp.png}
  }
  \subfigure{%
    \includegraphics[width=.15\columnwidth]{figures/supplementary/000008468_gt.png}
  }
  \subfigure{%
    \includegraphics[width=.15\columnwidth]{figures/supplementary/000008468_cnn.png}
  }
  \subfigure{%
    \includegraphics[width=.15\columnwidth]{figures/supplementary/000008468_crf.png}
  }
  \subfigure{%
    \includegraphics[width=.15\columnwidth]{figures/supplementary/000008468_ours.png}
  }\\[-2ex]

  \subfigure{%
    \includegraphics[width=.15\columnwidth]{figures/supplementary/000009053_given.png}
  }
  \subfigure{%
    \includegraphics[width=.15\columnwidth]{figures/supplementary/000009053_sp.png}
  }
  \subfigure{%
    \includegraphics[width=.15\columnwidth]{figures/supplementary/000009053_gt.png}
  }
  \subfigure{%
    \includegraphics[width=.15\columnwidth]{figures/supplementary/000009053_cnn.png}
  }
  \subfigure{%
    \includegraphics[width=.15\columnwidth]{figures/supplementary/000009053_crf.png}
  }
  \subfigure{%
    \includegraphics[width=.15\columnwidth]{figures/supplementary/000009053_ours.png}
  }\\[-2ex]




  \subfigure{%
    \includegraphics[width=.15\columnwidth]{figures/supplementary/000014977_given.png}
  }
  \subfigure{%
    \includegraphics[width=.15\columnwidth]{figures/supplementary/000014977_sp.png}
  }
  \subfigure{%
    \includegraphics[width=.15\columnwidth]{figures/supplementary/000014977_gt.png}
  }
  \subfigure{%
    \includegraphics[width=.15\columnwidth]{figures/supplementary/000014977_cnn.png}
  }
  \subfigure{%
    \includegraphics[width=.15\columnwidth]{figures/supplementary/000014977_crf.png}
  }
  \subfigure{%
    \includegraphics[width=.15\columnwidth]{figures/supplementary/000014977_ours.png}
  }\\[-2ex]


  \subfigure{%
    \includegraphics[width=.15\columnwidth]{figures/supplementary/000022922_given.png}
  }
  \subfigure{%
    \includegraphics[width=.15\columnwidth]{figures/supplementary/000022922_sp.png}
  }
  \subfigure{%
    \includegraphics[width=.15\columnwidth]{figures/supplementary/000022922_gt.png}
  }
  \subfigure{%
    \includegraphics[width=.15\columnwidth]{figures/supplementary/000022922_cnn.png}
  }
  \subfigure{%
    \includegraphics[width=.15\columnwidth]{figures/supplementary/000022922_crf.png}
  }
  \subfigure{%
    \includegraphics[width=.15\columnwidth]{figures/supplementary/000022922_ours.png}
  }\\[-2ex]


  \subfigure{%
    \includegraphics[width=.15\columnwidth]{figures/supplementary/000025711_given.png}
  }
  \subfigure{%
    \includegraphics[width=.15\columnwidth]{figures/supplementary/000025711_sp.png}
  }
  \subfigure{%
    \includegraphics[width=.15\columnwidth]{figures/supplementary/000025711_gt.png}
  }
  \subfigure{%
    \includegraphics[width=.15\columnwidth]{figures/supplementary/000025711_cnn.png}
  }
  \subfigure{%
    \includegraphics[width=.15\columnwidth]{figures/supplementary/000025711_crf.png}
  }
  \subfigure{%
    \includegraphics[width=.15\columnwidth]{figures/supplementary/000025711_ours.png}
  }\\[-2ex]


  \subfigure{%
    \includegraphics[width=.15\columnwidth]{figures/supplementary/000034473_given.png}
  }
  \subfigure{%
    \includegraphics[width=.15\columnwidth]{figures/supplementary/000034473_sp.png}
  }
  \subfigure{%
    \includegraphics[width=.15\columnwidth]{figures/supplementary/000034473_gt.png}
  }
  \subfigure{%
    \includegraphics[width=.15\columnwidth]{figures/supplementary/000034473_cnn.png}
  }
  \subfigure{%
    \includegraphics[width=.15\columnwidth]{figures/supplementary/000034473_crf.png}
  }
  \subfigure{%
    \includegraphics[width=.15\columnwidth]{figures/supplementary/000034473_ours.png}
  }\\[-2ex]


  \subfigure{%
    \includegraphics[width=.15\columnwidth]{figures/supplementary/000035463_given.png}
  }
  \subfigure{%
    \includegraphics[width=.15\columnwidth]{figures/supplementary/000035463_sp.png}
  }
  \subfigure{%
    \includegraphics[width=.15\columnwidth]{figures/supplementary/000035463_gt.png}
  }
  \subfigure{%
    \includegraphics[width=.15\columnwidth]{figures/supplementary/000035463_cnn.png}
  }
  \subfigure{%
    \includegraphics[width=.15\columnwidth]{figures/supplementary/000035463_crf.png}
  }
  \subfigure{%
    \includegraphics[width=.15\columnwidth]{figures/supplementary/000035463_ours.png}
  }\\[-2ex]


  \setcounter{subfigure}{0}
  \subfigure[\scriptsize Input]{%
    \includegraphics[width=.15\columnwidth]{figures/supplementary/000035993_given.png}
  }
  \subfigure[\scriptsize Superpixels]{%
    \includegraphics[width=.15\columnwidth]{figures/supplementary/000035993_sp.png}
  }
  \subfigure[\scriptsize GT]{%
    \includegraphics[width=.15\columnwidth]{figures/supplementary/000035993_gt.png}
  }
  \subfigure[\scriptsize AlexNet]{%
    \includegraphics[width=.15\columnwidth]{figures/supplementary/000035993_cnn.png}
  }
  \subfigure[\scriptsize +DenseCRF]{%
    \includegraphics[width=.15\columnwidth]{figures/supplementary/000035993_crf.png}
  }
  \subfigure[\scriptsize Using BI]{%
    \includegraphics[width=.15\columnwidth]{figures/supplementary/000035993_ours.png}
  }
  \mycaption{Material Segmentation}{Example results of material segmentation.
  (d)~depicts the AlexNet CNN result, (e)~CNN + 10 steps of mean-field inference,
  (f)~result obtained with bilateral inception (BI) modules (\bi{7}{2}+\bi{8}{6}) between
  \fc~layers.}
\label{fig:material_visuals-app}
\end{figure*}


\definecolor{city_1}{RGB}{128, 64, 128}
\definecolor{city_2}{RGB}{244, 35, 232}
\definecolor{city_3}{RGB}{70, 70, 70}
\definecolor{city_4}{RGB}{102, 102, 156}
\definecolor{city_5}{RGB}{190, 153, 153}
\definecolor{city_6}{RGB}{153, 153, 153}
\definecolor{city_7}{RGB}{250, 170, 30}
\definecolor{city_8}{RGB}{220, 220, 0}
\definecolor{city_9}{RGB}{107, 142, 35}
\definecolor{city_10}{RGB}{152, 251, 152}
\definecolor{city_11}{RGB}{70, 130, 180}
\definecolor{city_12}{RGB}{220, 20, 60}
\definecolor{city_13}{RGB}{255, 0, 0}
\definecolor{city_14}{RGB}{0, 0, 142}
\definecolor{city_15}{RGB}{0, 0, 70}
\definecolor{city_16}{RGB}{0, 60, 100}
\definecolor{city_17}{RGB}{0, 80, 100}
\definecolor{city_18}{RGB}{0, 0, 230}
\definecolor{city_19}{RGB}{119, 11, 32}
\begin{figure*}[!ht]
  \small % scriptsize
  \centering


  \subfigure{%
    \includegraphics[width=.18\columnwidth]{figures/supplementary/frankfurt00000_016005_given.png}
  }
  \subfigure{%
    \includegraphics[width=.18\columnwidth]{figures/supplementary/frankfurt00000_016005_sp.png}
  }
  \subfigure{%
    \includegraphics[width=.18\columnwidth]{figures/supplementary/frankfurt00000_016005_gt.png}
  }
  \subfigure{%
    \includegraphics[width=.18\columnwidth]{figures/supplementary/frankfurt00000_016005_cnn.png}
  }
  \subfigure{%
    \includegraphics[width=.18\columnwidth]{figures/supplementary/frankfurt00000_016005_ours.png}
  }\\[-2ex]

  \subfigure{%
    \includegraphics[width=.18\columnwidth]{figures/supplementary/frankfurt00000_004617_given.png}
  }
  \subfigure{%
    \includegraphics[width=.18\columnwidth]{figures/supplementary/frankfurt00000_004617_sp.png}
  }
  \subfigure{%
    \includegraphics[width=.18\columnwidth]{figures/supplementary/frankfurt00000_004617_gt.png}
  }
  \subfigure{%
    \includegraphics[width=.18\columnwidth]{figures/supplementary/frankfurt00000_004617_cnn.png}
  }
  \subfigure{%
    \includegraphics[width=.18\columnwidth]{figures/supplementary/frankfurt00000_004617_ours.png}
  }\\[-2ex]

  \subfigure{%
    \includegraphics[width=.18\columnwidth]{figures/supplementary/frankfurt00000_020880_given.png}
  }
  \subfigure{%
    \includegraphics[width=.18\columnwidth]{figures/supplementary/frankfurt00000_020880_sp.png}
  }
  \subfigure{%
    \includegraphics[width=.18\columnwidth]{figures/supplementary/frankfurt00000_020880_gt.png}
  }
  \subfigure{%
    \includegraphics[width=.18\columnwidth]{figures/supplementary/frankfurt00000_020880_cnn.png}
  }
  \subfigure{%
    \includegraphics[width=.18\columnwidth]{figures/supplementary/frankfurt00000_020880_ours.png}
  }\\[-2ex]



  \subfigure{%
    \includegraphics[width=.18\columnwidth]{figures/supplementary/frankfurt00001_007285_given.png}
  }
  \subfigure{%
    \includegraphics[width=.18\columnwidth]{figures/supplementary/frankfurt00001_007285_sp.png}
  }
  \subfigure{%
    \includegraphics[width=.18\columnwidth]{figures/supplementary/frankfurt00001_007285_gt.png}
  }
  \subfigure{%
    \includegraphics[width=.18\columnwidth]{figures/supplementary/frankfurt00001_007285_cnn.png}
  }
  \subfigure{%
    \includegraphics[width=.18\columnwidth]{figures/supplementary/frankfurt00001_007285_ours.png}
  }\\[-2ex]


  \subfigure{%
    \includegraphics[width=.18\columnwidth]{figures/supplementary/frankfurt00001_059789_given.png}
  }
  \subfigure{%
    \includegraphics[width=.18\columnwidth]{figures/supplementary/frankfurt00001_059789_sp.png}
  }
  \subfigure{%
    \includegraphics[width=.18\columnwidth]{figures/supplementary/frankfurt00001_059789_gt.png}
  }
  \subfigure{%
    \includegraphics[width=.18\columnwidth]{figures/supplementary/frankfurt00001_059789_cnn.png}
  }
  \subfigure{%
    \includegraphics[width=.18\columnwidth]{figures/supplementary/frankfurt00001_059789_ours.png}
  }\\[-2ex]


  \subfigure{%
    \includegraphics[width=.18\columnwidth]{figures/supplementary/frankfurt00001_068208_given.png}
  }
  \subfigure{%
    \includegraphics[width=.18\columnwidth]{figures/supplementary/frankfurt00001_068208_sp.png}
  }
  \subfigure{%
    \includegraphics[width=.18\columnwidth]{figures/supplementary/frankfurt00001_068208_gt.png}
  }
  \subfigure{%
    \includegraphics[width=.18\columnwidth]{figures/supplementary/frankfurt00001_068208_cnn.png}
  }
  \subfigure{%
    \includegraphics[width=.18\columnwidth]{figures/supplementary/frankfurt00001_068208_ours.png}
  }\\[-2ex]

  \subfigure{%
    \includegraphics[width=.18\columnwidth]{figures/supplementary/frankfurt00001_082466_given.png}
  }
  \subfigure{%
    \includegraphics[width=.18\columnwidth]{figures/supplementary/frankfurt00001_082466_sp.png}
  }
  \subfigure{%
    \includegraphics[width=.18\columnwidth]{figures/supplementary/frankfurt00001_082466_gt.png}
  }
  \subfigure{%
    \includegraphics[width=.18\columnwidth]{figures/supplementary/frankfurt00001_082466_cnn.png}
  }
  \subfigure{%
    \includegraphics[width=.18\columnwidth]{figures/supplementary/frankfurt00001_082466_ours.png}
  }\\[-2ex]

  \subfigure{%
    \includegraphics[width=.18\columnwidth]{figures/supplementary/lindau00033_000019_given.png}
  }
  \subfigure{%
    \includegraphics[width=.18\columnwidth]{figures/supplementary/lindau00033_000019_sp.png}
  }
  \subfigure{%
    \includegraphics[width=.18\columnwidth]{figures/supplementary/lindau00033_000019_gt.png}
  }
  \subfigure{%
    \includegraphics[width=.18\columnwidth]{figures/supplementary/lindau00033_000019_cnn.png}
  }
  \subfigure{%
    \includegraphics[width=.18\columnwidth]{figures/supplementary/lindau00033_000019_ours.png}
  }\\[-2ex]

  \subfigure{%
    \includegraphics[width=.18\columnwidth]{figures/supplementary/lindau00052_000019_given.png}
  }
  \subfigure{%
    \includegraphics[width=.18\columnwidth]{figures/supplementary/lindau00052_000019_sp.png}
  }
  \subfigure{%
    \includegraphics[width=.18\columnwidth]{figures/supplementary/lindau00052_000019_gt.png}
  }
  \subfigure{%
    \includegraphics[width=.18\columnwidth]{figures/supplementary/lindau00052_000019_cnn.png}
  }
  \subfigure{%
    \includegraphics[width=.18\columnwidth]{figures/supplementary/lindau00052_000019_ours.png}
  }\\[-2ex]




  \subfigure{%
    \includegraphics[width=.18\columnwidth]{figures/supplementary/lindau00027_000019_given.png}
  }
  \subfigure{%
    \includegraphics[width=.18\columnwidth]{figures/supplementary/lindau00027_000019_sp.png}
  }
  \subfigure{%
    \includegraphics[width=.18\columnwidth]{figures/supplementary/lindau00027_000019_gt.png}
  }
  \subfigure{%
    \includegraphics[width=.18\columnwidth]{figures/supplementary/lindau00027_000019_cnn.png}
  }
  \subfigure{%
    \includegraphics[width=.18\columnwidth]{figures/supplementary/lindau00027_000019_ours.png}
  }\\[-2ex]



  \setcounter{subfigure}{0}
  \subfigure[\scriptsize Input]{%
    \includegraphics[width=.18\columnwidth]{figures/supplementary/lindau00029_000019_given.png}
  }
  \subfigure[\scriptsize Superpixels]{%
    \includegraphics[width=.18\columnwidth]{figures/supplementary/lindau00029_000019_sp.png}
  }
  \subfigure[\scriptsize GT]{%
    \includegraphics[width=.18\columnwidth]{figures/supplementary/lindau00029_000019_gt.png}
  }
  \subfigure[\scriptsize Deeplab]{%
    \includegraphics[width=.18\columnwidth]{figures/supplementary/lindau00029_000019_cnn.png}
  }
  \subfigure[\scriptsize Using BI]{%
    \includegraphics[width=.18\columnwidth]{figures/supplementary/lindau00029_000019_ours.png}
  }%\\[-2ex]

  \mycaption{Street Scene Segmentation}{Example results of street scene segmentation.
  (d)~depicts the DeepLab results, (e)~result obtained by adding bilateral inception (BI) modules (\bi{6}{2}+\bi{7}{6}) between \fc~layers.}
\label{fig:street_visuals-app}
\end{figure*}

\bibliographystyle{plain}
\bibliography{reference}
\end{document}
