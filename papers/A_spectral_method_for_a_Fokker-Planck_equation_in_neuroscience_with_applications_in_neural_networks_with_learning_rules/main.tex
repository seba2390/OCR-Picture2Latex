\documentclass[a4paper, 10pt]{article}
\usepackage{srcltx,graphicx,epstopdf}
\usepackage{amsmath, amssymb, amsthm}
\usepackage{color}
\usepackage{xcolor}
\usepackage{multirow}
\usepackage{subfig} % it is suggested to use subfloat rather than subfigure
\usepackage[margin=1in]{geometry}
%%%%%% cleveref: use \cref to replace ``Figure \ref'', ``Section \ref'', etc.
\usepackage[T1]{fontenc}
\usepackage{cleveref}
\usepackage{cite}
\crefname{section}{Section}{Sections}
\crefname{subsection}{Subsection}{Subsections}
\crefname{appendix}{Appendix}{Appendix}
\crefname{figure}{Figure}{Figures}
\crefname{table}{Table}{Tables}
\crefname{property}{Property}{Properties}
\crefname{theorem}{Theorem}{Theorem}
\usepackage{booktabs}
\usepackage{tabularx}
\usepackage{threeparttable}
\usepackage{enumerate}
\usepackage[normalem]{ulem}
\graphicspath{{graphh/}}

\newtheorem{theorem}{Theorem}
\newtheorem{lemma}[theorem]{Lemma}
\newtheorem{deduction}[theorem]{Deduction}
\newtheorem{property}[theorem]{Property}
\newtheorem{proposition}{Proposition}
\newtheorem{definition}[theorem]{Definition}
\newtheorem{comment}{Comment}
\newtheorem{example}[theorem]{Example}

{\theoremstyle{remark} \newtheorem{remark}{Remark}}
\newtheorem{corollary}{Corollary}



\renewcommand{\ULthickness}{2pt}

\newcommand\pd[2]{\dfrac{\partial {#1}}{\partial {#2}}}
\newcommand\od[2]{\dfrac{\dd {#1}}{\dd {#2}}}
\newcommand\odd[2]{\dfrac{\mathrm{D} {#1}}{\mathrm{D} {#2}}}
\def\bsxi{\boldsymbol{\xi}}
\newcommand\bxi{\boldsymbol{\xi}}
\newcommand\bA{{\bf A}}
\newcommand\bbA{\bar{\bf A}}
\newcommand\bbB{\bar{\bf B}}

\newcommand\bI{{\bf I}}
\newcommand\bbI{\bar{{\bf I}}}

\newcommand\bB{{\bf B}}
\newcommand\mE{\mathcal{E}}
\newcommand\rC[2]{{\rm{C}}_{{#1},{#2}}}
\newcommand\bN{\boldsymbol{N}}
\newcommand\bx{\boldsymbol{x}}
\newcommand\bn{\boldsymbol{n}}
\newcommand\bu{\boldsymbol{u}}
\newcommand\bv{\boldsymbol{v}}
\newcommand\bc{\boldsymbol{c}}
\newcommand\bC{\boldsymbol{C}}
\newcommand\bM{{\bf M}}
\newcommand\bD{{\bf D}}
\newcommand\ba{\boldsymbol{a}}
\newcommand\bomega{\boldsymbol{\omega}}
\newcommand{\imag}{\mathrm{i}}
\newcommand\bg{\boldsymbol{g}}
\newcommand\bR{\boldsymbol{R}}
\newcommand\bs{\boldsymbol{s}}
\newcommand\bh{\boldsymbol{h}}
\newcommand\bbR{\mathbb{R}}
\newcommand\bbN{\mathbb{N}}
\newcommand\bbZ{\mathbb{Z}}
\newcommand\bbS{\mathbb{S}}
\newcommand\bdeta{\boldsymbol{\eta}}
\newcommand\htheta{\hat{\theta}}
\newcommand\bq{\boldsymbol{q}}
\newcommand\bPhi{\boldsymbol{\Phi}}
\newcommand\dd{\,\mathrm{d}}
\newcommand\Kn{\mathit{Kn}}
\newcommand\mQ{\mathcal{Q}}
\newcommand\mM{\mathcal{M}}
\newcommand\mH{\mathcal{H}}
\newcommand\mF{\mathcal{F}}
\newcommand\mG{\mathcal{G}}
\newcommand\mS{\mathcal{S}}


\newcommand\mI{\mathcal{I}}
\newcommand\bw{\boldsymbol{w}}
\newcommand\He{\mathit{He}}
\def\bd{\boldsymbol{d}}
\def\bx{\boldsymbol{x}}
\def\bbR{\mathbb{R}}
\def\bu{\boldsymbol{u}}
\def\bq{\boldsymbol{q}}
\def\bsigma{\boldsymbol{\sigma}}
\def\bslambda{\boldsymbol{\lambda}}
\def\bE{\boldsymbol{E}}
\def\bF{\boldsymbol{F}}
\def\Identity{\boldsymbol{I}}
\def\bg{\boldsymbol{g}}
\def\bn{\boldsymbol{n}}
\def\bm{\boldsymbol{m}}
\def\bsalpha{\boldsymbol{\alpha}}
\def\bsOmega{\boldsymbol{\Omega}}
\def\bsmu{\boldsymbol{\mu}}
\numberwithin{equation}{section}

\newcommand\note[2]{{{\bf #1}\color{red} [ {\it #2} ]}}
\definecolor{electricpurple}{rgb}{0.75,0.0,1.0}
\definecolor{darkred}{rgb}{0.65,0,0}
\definecolor{green}{rgb}{0.0, 0.5, 0.0}

% \newcommand\fy[1]{{\color{red}FY: #1}}
 \newcommand\addi[1]{{\color{darkred} #1}}
 \newcommand\addii[1]{{\color{blue} #1}}
% \newcommand\addiii[1]{{\color{electricpurple}#1}}
% \newcommand\addall[1]{{\color{green}#1}}
\newcommand\wyl[1]{{\color{red} #1}}
\newcommand\zzn[1]{{\color{blue} #1}}
 \newcommand\deletei{\bgroup\markoverwith{\textcolor{darkred}{\rule[0.5ex]{1pt}{1pt}}}\ULon}
 \newcommand\deleteii{\bgroup\markoverwith{\textcolor{blue}{\rule[0.5ex]{2pt}{2pt}}}\ULon}
% \newcommand\deleteiii{\bgroup\markoverwith{\textcolor{electricpurple}{\rule[0.5ex]{2pt}{2pt}}}\ULon}
% \newcommand\deleteall{\bgroup\markoverwith{\textcolor{green}{\rule[0.5ex]{2pt}{2pt}}}\ULon}


\title {A spectral method for a Fokker-Planck equation in neuroscience with applications in neural networks with learning rules }

%\author{Pei Zhang\thanks{Beijing Computational Science Research Center, Beijing, China, 100193, email: {\tt zhangpei@csrc.ac.cn}}, ~~Zhennan Zhou\thanks{Beijing International Center for Mathematical Research, Peking University, Beijing, China, 100871, email: {\tt zhennan@bicmr.pku.edu.cn}}, ~~Yanli Wang\thanks{Beijing Computational Science Research Center, Beijing, China, 100193, email: {\tt ylwang@csrc.ac.cn}}.}
\author{Pei Zhang\thanks{Beijing Computational Science Research Center, Beijing, China, 100193, email: {\tt zhangpei@csrc.ac.cn}},
   ~~Yanli Wang\thanks{Beijing Computational Science Research Center, Beijing, China, 100193, email: {\tt ylwang@csrc.ac.cn}},
   ~~Zhennan Zhou\thanks{Beijing International Center for Mathematical Research, Peking University, Beijing, China, 100871, email: {\tt zhennan@bicmr.pku.edu.cn}}.
    }

\begin{document}
\maketitle
%\tableofcontents
%\clearpage 
\section{Introduction}  \label{sec:introduction}

\newcommand\inexpIntro[3]{#1?(#2,#3).}
\newcommand\rinexpIntro[3]{*#1?(#2,#3).}
\newcommand\outexpIntro[3]{#1!(#2,#3).}
\newcommand\outatomIntro[3]{#1!(#2,#3)}

We propose a fully automated method for proving termination of \(\pi\)-calculus processes.
Although there have been a lot of studies on termination analysis for the \(\pi\)-calculus
and related calculi~\cite{Deng06IC,Demangeon07,SangiorgiTermination,KobayashiHybrid,Yoshida04IC,DBLP:journals/jlp/DemangeonHS10,Venet98SAS}, most of them have been rather theoretical,
and there have been surprisingly little efforts in developing  fully automated termination
verification methods and tools based on them. To our knowledge,
Kobayashi's \typical{}~\cite{TyPiCal,KobayashiHybrid} is the only exception that
can prove termination of \(\pi\)-calculus processes (extended with natural numbers)
fully automatically, but its termination analysis is quite limited (see Section~\ref{sec:relatedwork}).

Our method is based on a reduction to termination analysis for sequential programs:
we translate a \(\pi\)-calculus process \(P\) to a sequential program \(S_P\), so that
if \(S_P\) is terminating, so is \(P\). The reduction allows us to use
powerful, mature methods and tools
for termination analysis of sequential programs~\cite{heizmann2016ultimate,freqterm,DBLP:conf/lics/PodelskiR04,Kuwahara2014Termination,DBLP:journals/cacm/CookPR11}.

The idea of the translation is to convert a chain of communications on replicated input
channels to a chain of recursive function calls of the target sequential program.
Let us consider the following Fibonacci process:
\begin{align*}
    & \rinexpIntro{\fib}{n}{r}
        \ifexp{n<2}{ \soutatom{r}{1} \\ &\quad}
                   { \nuexp{s_1} \nuexp{s_2} (\outatomIntro{\fib}{n-1}{s_1} \PAR \outatomIntro{\fib}{n-2}{s_2} \PAR \sinexp{s_1}{x}\sinexp{s_2}{y}\soutatom{r}{x+y}) \\}
    & \PAR \outatomIntro{\fib}{m}{r}
\end{align*}
Here, the process
$\rinexpIntro{\fib}{n}{r} \ldots$ is a function server that computes the \(n\)-th Fibonacci number
in parallel and returns the result to \(r\),
and $\outatom{\fib}{m}{r}$ sends a request for computing the \(m\)-th Fibonacci number;
those who are not familiar with the syntax of the \(\pi\)-calculus may wish to consult
Section~\ref{sec:targetlanguage} first.
To prove that the process above is terminating for any integer \(m\),
it suffices to show that there is no infinite chain of communications on $\fib$:
\[
    \fib(m,r) \to \fib(m_1,r_1) \to \fib(m_2,r_2) \to \cdots.
\]
We convert the process above to the following program:\footnote{The actual translation
  given later is a little more complex.}
\begin{verbatim}
 let rec fib(n) = if n<2 then () else (fib(n-1) [] fib(n-2)) in
 fib(m)
\end{verbatim}
Here, \texttt{[]} represents the non-deterministic choice.
Note that, although the calculation of Fibonacci numbers is not preserved,
for each chain of communications on \texttt{fib}, there is a corresponding
sequence of recursive calls:
\[
\mathtt{fib}(m) \to \mathtt{fib}(m_1) \to \mathtt{fib}(m_2) \to \cdots.
\]
Thus, the termination of the sequential program above implies the termination of
the original process.
As shown in the example above, (i) each communication on a replicated input channel
is converted to a function call, (ii) each communication on a non-replicated input
channel is just removed (or, in the actual translation, replaced by a call of
a trivial function defined by \(f(\seq{x})=(\,)\)), and (iii) parallel composition
is replaced by a non-deterministic choice.
We formalize the translation outlined above and prove its correctness.

The basic translation sketched above sometimes loses too much information.
For example, consider the following process:
\begin{align*}
    & \rinexpIntro{\pre}{n}{r} \soutatom{r}{n-1} \\
    & \PAR \rinexpIntro{f}{n}{r} \ifexp{n<0}{ \soutatom{r}{1} }
                                       { \nuexp{s} (\outatomIntro{\pre}{n}{s} \PAR \sinexp{s}{x}\outatomIntro{f}{x}{r}) } \\
    & \PAR \outatomIntro{f}{m}{r}
\end{align*}
The translation sketched above would yield:
\begin{verbatim}
  let pred(n) = n-1 in
  let rec f(n) = if n<0 then () else (pred(n) [] f(*)) in
  f(m)
\end{verbatim}
Here, \texttt{*} represents a non-deterministic integer: since we have removed
the input $\sinatom{s}{x}$, we do not have information about the value of \( x \).
As a result, the sequential program above is non-terminating, although the original
process is terminating.
To remedy this problem, we also refine the basic translation above by using a refinement
type system for the \(\pi\)-calculus. Using the refinement type system,
we can infer that the value of \(x\) in the original process is less than \(n\),
so that we can refine the definition of \texttt{f} to:
\begin{verbatim}
 let rec f(n) = ... else (pred(n) [] let x=* in assume(x<n);f(x))
\end{verbatim}
The target program is now terminating, from which
we can deduce that the original process is also terminating.
We have implemented an automated tool based on the refined translation above.

The contributions of this paper are summarized as follows.
\begin{itemize}
\item The formalization of the basic translation from the \(\pi\)-calculus
  (extended with integers) to sequential programs, and a proof of its correctness.
\item The formalization of a refined translation based on a refinement type system.
\item An implementation of the refined translation, including automated refinement type
  inference based on CHC solving, and experiments to evaluate the effectiveness of
  our method.
\end{itemize}

The rest of this paper is structured as follows.
Section~\ref{sec:targetlanguage} introduces the source and target languages
of our translation.
Section~\ref{sec:approach} 
formalizes the basic translation, and proves its correctness.
Section~\ref{sec:refinement} refines the basic translation by using a refinement type system.
Section~\ref{sec:implementation} reports an implementation and experiments.
Section~\ref{sec:relatedwork} discusses related work,
and Section~\ref{sec:conclusion} concludes the paper.

\section{Weak formulation}\label{sec:weak_form}


In this section, we introduce the weak formulation of the problem, which is the foundation for constructing numerical solutions. The link between 
the classical solution and the weak solution of this model will be analyzed as well.


%Spatial discretization involves constructing a weak formulation of the problem \eqref{eq:problem1} over a given domain. 
For simplicity, we choose a finite interval $[V_{\min}, V_F]$ as the computation domain and suppose $V_{\min}$ is small enough such that the density function $p(v,t)$ for  $v < V_{\min}$ is negligible. Then the semi-unbounded problem \eqref{eq:problem1} can be truncated to boundary value problem as follow:
\begin{equation}
    \label{eq:problem2}
    \begin{cases}
        \partial_{t}p+\partial_{v}(hp)-a\partial_{v v}p=0,\qquad v\in[V_{\min},V_F]/\{V_R\},\\
        p(v,0)=p^0(v),\qquad p(V_{\min},t)=p(V_F,t)=0,\\
        p(V^-_R,t)=p(V^+_R,t),\quad \partial _vp(V^-_R,t)=\partial _vp(V^+_R,t)+\frac{N(t)}{a}.\\
    \end{cases}
\end{equation}
The truncated equation \eqref{eq:problem2} should still satisfy the mass conservation, i.e.
\begin{equation}
    \label{eq:mass_conservation}
    \int_{V_{\min}}^{V_{F}} p(v, t) d v=\int_{V_{\min}}^{V_{F}} p^{0}(v) d v=1.
\end{equation}
By integrating \eqref{eq:problem2} and using the boundary conditions therein, this conversation implies the following boundary condition.
\begin{equation}
    \label{eq:leftd}
    \frac{\partial}{\partial v}p(V_{\min},t)=0.
\end{equation}
In fact, \eqref{eq:leftd} is never precisely satisfied, but as long as $V_{\min}$ is chosen properly, $\partial_vp(V_{\min},t)$ is negligible.


%Before defining the weak solution, it is necessary to have a clear interpretation of the classical solution. 
 We adopt the definition of the classical solution in \cite{carrillo2013classical}\cite{liu2022rigorous} for the truncated problem.
\begin{definition}[classical solution] \label{class_solution}
    For any given $0<T<+\infty$, $p(v,t)$ is a classical solution of \eqref{eq:problem2} in the time interval $(0, T]$  in the following sense:
    \begin{itemize}
        \item[1.] $N(t)=-a\partial_vp(V_F^-,t)$ is a continuous function for $t\in [0,T]$,
        \item[2.] $p(v,t)$ is continuous in the region $\{(v,t):V_{\min}<v<V_F, t\in[0,T]\}$,
        \item[3.] $p_{vv}$ and $p_t$ are well defined in the region $\{(v,t): v\in [V_{\min},V_R)\cup(V_R,V_F], t \in (0,T]\}$,
        \item[4.] $p_v(V_R^-,t)$ and $p_v(V_R^+,t)$ are well defined for $t\in(0,T]$,
        \item[5.] For $t\in (0,T]$, equation \eqref{eq:problem2} is satisfied,
        \item[6.] $p(v,0)=p^0(v)$ for $v\in [V_{\min},V_R)\cup(V_R,V_F] $.
    \end{itemize}
\end{definition}
In this paper, we consider classical solutions of \eqref{eq:problem2} which additionally satisfy \eqref{eq:leftd}. Having explicitly defined the classical solution of \eqref{eq:problem2}, we can now move on to discuss the weak solution.


If $p(v,t)$ is the classical solution of \eqref{eq:problem2} , weak formulation of \eqref{eq:problem2} is obtained by multiplying \eqref{eq:problem2} with some test function $\phi \in C^{\infty}([V_{\min}, V_F])$ and integrating over $[V_{\min}, V_F]$
\begin{equation}
    \label{variational_1}
    \int_{V_{\min}}^{V_{F}} \left(
    \partial_{t}p+\partial_{v}(hp)-a\partial_{v v}p\right)\phi dv =0.
\end{equation}
Integrating by parts in intervals $[V_{\min},V_R]$ and $[V_R,V_F]$ respectively, we obtain 
\begin{equation}
    \label{eq:int_by_part}
\begin{aligned}
    &\int_{V_{\min}}^{V_{F}} \left(\partial_tp \phi-hp\partial_v\phi+a\partial_vp\partial_v\phi\right) dv\\
    +&\left(hp\phi|_{V_{\min}}^{V_R^-}+hp\phi|_{V_R^+}^{V_F}\right)-\left(a\partial_vp\phi|_{V_{\min}}^{V_R^-}+a\partial_vp\phi|_{V_{V_R^+}}^{V_F}\right)=0.
\end{aligned}
\end{equation}
By substituting the boundary conditions in \eqref{eq:problem2} and \eqref{eq:leftd}, \eqref{eq:int_by_part} can be simplified as
\begin{equation}
    \label{variational_2}
    \int_{V_{\min}}^{V_{F}} \left(\partial_tp \phi-hp\partial_v\phi+a\partial_vp\partial_v\phi\right) dv+a\partial_vp(V_F)\left(\phi(V_R)-\phi(V_F)\right)=0.
\end{equation}
The above derivation helps to formally introduce the definition of the weak solution of \eqref{eq:problem2}.
\begin{definition}[weak solution] \label{weak_solution}
    The variational space appropriate for the present case is
    \begin{equation}
        \label{eq:variational_space}
        \mathbb{H}^{1}_0(V_{\min},V_F)=\{p\in \mathbb{H}^{1}(V_{\min},V_F):p|_{V_{\min}}=p|_{V_F}=0\}.
    \end{equation}
    We say  $p(v,t)\in  C^{1}([0,T];\, \mathbb{H}^{1}_0(V_{\min},V_F)) $ is a weak solution of \eqref{eq:problem2}  if for any test function $\phi(v) \in \mathbb{H}^{1}(V_{\min},V_F)$, \eqref{variational_2} holds for $\forall t\in (0,T]$ and $p(v,0)=p^0(v)$. 
\end{definition}
The weak solution in Definition \ref{weak_solution} still inherits the essence of the original problem \eqref{eq:problem2}, and the relation between the weak solution and the classical solution is established in the following.
\begin{theorem} [Relation with the classical solution]
If $p(v,t)$ is a classical solution of \eqref{eq:problem2} in the time interval $(0, T]$ which also satisfies \eqref{eq:leftd}, then it is a weak solution of \eqref{eq:problem2} in the same time interval. Conversely, if $p(v,t)$ is a weak solution of \eqref{eq:problem2} in the time interval $(0, T]$ and additionally we assume that $$p(v,t) \in C^{1}\left((0,T];\,C^{2}\left([V_{\min},V_R)\cup(V_R,V_F]\right)\right) $$ satisfies $p(V_R^-,t)=p(V_R^+,t)$ and the one-sided derivatives of $p(v,t)$ exist at each side of $V_R$ for all $t\in(0,T] $, then it is a classical solution of \eqref{eq:problem2} in the same time interval and it satisfies \eqref{eq:leftd}.
\end{theorem}
\begin{proof}
The first part of the theorem can obviously be proved by the derivation of the weak solution. 

For the other direction, let $p(v,t)$ be a weak solution of \eqref{eq:problem2} in the time interval $(0, T]$, and $p$ satisfies all the additional assumptions in the statement. We aim to prove that $p(v,t)$ is a classical solution in Definition \ref{class_solution}, and satisfies \eqref{eq:leftd}. 


By the definition of the weak solution and the additional conditions it satisfies, it is straightforward to show that the solution $p$ meets the first four and the last criteria laid out in Definition \ref{class_solution}. In particular, the smoothness assumption at $V_F$ (from the left-hand side) implies the continuity of $N(t)$.  In the following, we will thoroughly demonstrate that $p$ conforms to the fifth item of Definition \ref{class_solution} and \eqref{eq:leftd}. By integration by parts, \eqref{variational_2} can be rewritten as 
\begin{equation}
\label{eq:int_by_part2}
\begin{aligned}
    &\int_{V_{\min}}^{V_{F}} \left(\partial_{t}p+\partial_{v}(hp)-a\partial_{v v}p\right)\phi dv\\ 
    -&\left(hp\phi|_{V_{\min}}^{V_R^-}+hp\phi|_{V_R^+}^{V_F}\right)+\left(a\partial_vp\phi|_{V_{\min}}^{V_R^-}+a\partial_vp\phi|_{V_{V_R^+}}^{V_F})+a\partial_vp(V_F)(\phi(V_R)-\phi(V_F)\right)=0.
\end{aligned}
\end{equation}
The definition of $\mathbb{H}^{1}_0(V_{\min},V_F)$ states that $p(V_{\min})=p(V_F)=0$, thus
\begin{equation}
    \label{eq:int_by_part3}
    \begin{aligned}
        &\int_{V_{\min}}^{V_{F}} \left(\partial_{t}p+\partial_{v}(hp)-a\partial_{v v}p\right)\phi dv-h(V_R)\phi(V_R)\left(p(V_R^-)-p(V_R^+)\right)\\
        +&a\phi(V_R)\left(\partial_vp(V_R^-)-\partial_vp(V_R^+)+\partial_vp(V_F)\right)-a\partial_vp(V_{\min})\phi(V_{\min})=0,
    \end{aligned}
\end{equation}
The key to the proof is selecting different test function spaces to simplify \eqref{eq:int_by_part3}, such that the equations identified in \eqref{eq:problem2} and the boundary conditions delineated in \eqref{eq:problem2} and \eqref{eq:leftd} are successively established. First, taking the test functions $\phi \in \mathbb{V}_1(V_{\min},V_F)=\{\phi \in \mathbb{H}^1(V_{\min},V_F): \phi(V_R)=\phi(V_{\min})=0\}$, \eqref{eq:int_by_part3} reduce to
\begin{equation}
    \int_{V_{\min}}^{V_{F}} \left(\partial_{t}p+\partial_{v}(hp)-a\partial_{v v}p\right)\phi dv=0.
\end{equation}
 Since $p\in C^{2}\left([V_{\min},V_R)\cup(V_R,V_F]\right)$, $\partial_{t}p+\partial_{v}(hp)-a\partial_{v v}p$ is continuous on the interval $(V_{\min},V_R)\cup(V_R,V_F)$, and it can be inferred from the arbitrariness of $\phi$ that $p$ satisfies
\begin{equation}
    \label{pde_equation}
    \partial_{t}p+\partial_{v}(hp)-a\partial_{v v}p=0, \quad \forall v\in (V_{\min},V_R)\cup(V_R,V_F).
\end{equation}
This verifies that the equation in \eqref{eq:problem2} holds for  $p(v,t)$ within the interval, and the next step in the proof is that the weak solution satisfies the boundary conditions in \eqref{eq:problem2} and \eqref{eq:leftd}. By the definition of trial function space $\mathbb{H}^{1}_0(V_{\min},V_F)$, it is easy to see that $p(v,t)$ satisfies the following boundary conditions
\begin{equation}\label{eq:Dirichlet_boundary}
\begin{aligned}
    &p(V_{\min},t)=p(V_F,t)=0.\\
\end{aligned}
\end{equation}
Changing the test functions $\phi \in \mathbb{V}_2(V_{\min},V_F)=\{\phi \in \mathbb{H}^1(V_{\min},V_F): \phi(V_R)=0\}$ and using \eqref{pde_equation}, \eqref{eq:int_by_part3} can be written as
\begin{equation}
    a\partial_vp(V_{\min})\phi(V_{\min})=0.
\end{equation}
Since the arbitrariness of $\phi(V_{\min})$, we obtain
\begin{equation}\label{eq:left_boundary}
    \partial_vp(V_{\min})=0.
\end{equation}
Similarly, changing the test functions $\phi \in \mathbb{V}_3(V_{\min},V_F)=\{\phi \in \mathbb{H}^1(V_{\min},V_F): \phi(V_{\min})=0\}$ again and using \eqref{pde_equation}, \eqref{eq:int_by_part3} is reduced into
\begin{equation}
    -h(V_R)\phi(V_R)\left(p(V_R^-)-p(V_R^+)\right)+a\phi(V_R)\left(\partial_vp(V_R^-)-\partial_vp(V_R^+)+\partial_vp(V_F)\right)=0.
\end{equation}
Since $p(V^-_R)=p(V^+_R) $ and the arbitrariness of $\phi(V_R)$, we deduce
\begin{equation}
    \partial_vp(V_R^-)-\partial_vp(V_R^+)+\partial_vp(V_F)=0.
\end{equation}
By the definition of trace, $p(v,t)$ satisfies boundary conditions in \eqref{eq:problem2} and \eqref{eq:leftd}. Now, we have proved that $p(v,t)$ satisfies the fifth item in Definition \ref{class_solution}.  To conclude, we have shown that $p(v,t)$ is a classical solution of equation \eqref{eq:problem2}


\end{proof}

\section{Numerical scheme and analysis}\label{sec:scheme}
In this section, we present a spectral approximation for the weak solution to the Fokker-Planck equation \eqref{eq:problem2} and construct a fully discrete numerical scheme. Numerical solutions are sought in a specific function space in which the functions satisfy the boundary conditions and can be determined by solving the derived equation system after specifying the test function space.
\subsection{A fully discrete numerical scheme based on Legendre spectral method}\label{sec:fully_discrete_scheme}
In this part, we construct the numerical scheme of the Fokker-Planck equation \eqref{eq:problem2}, which is implemented in two steps. First, the spectral approximation is used for space discretization, resulting in a system of ordinary differential equations; second, a semi-implicit scheme is applied for time discretization. The spectral method is established such that the numerical solution inherently satisfies the boundary conditions. Formally, the approximate variational problem is
\begin{equation}
    \label{var_problem}
    \begin{cases}
        \text{Find } p\in \mathrm{W} \text{ such that}\\
        \int_{V_{\min}}^{V_{F}} \left(
    \partial_{t}p+\partial_{v}(hp)-a\partial_{v v}p\right)\phi =0,\quad \forall \phi \in \mathrm{V},
    \end{cases}
\end{equation}
where $\mathrm{W}$ is the trial function space and $\mathrm{V}$ is the test function space. Compared to Definition \ref{weak_solution}, the variational problem \eqref{var_problem} requires a more complex trial function space, which will be further described below. The specific form of the test function space will be introduced in Section \ref{sec:stability}.
\subsubsection{Construction of trial function space and space discretization}
A challenging aspect of the spectral method is constructing the trial function space $\mathrm{W}$ so as to satisfy the complex boundary conditions, including the discontinuous derivative of the density function and the dynamic boundary. To that end, the trial function space should be a subset of $\mathbb{H}^1_0$ wherein strong boundary derivatives can be defined. Specifically, the polynomial space that fulfills the boundary conditions in \eqref{eq:problem2} and \eqref{eq:leftd} can be used as the trial function space. That is $\mathrm{W} \subseteq \mathrm{P}_{\infty}(V_{\min}, V_R) + \mathrm{P}_{\infty}(V_{R}, V_F)$ and for all $ p\in \mathrm{W}$, it holds that
\begin{equation}
    \label{boudary_condition}
    \begin{cases}
        p(V_{\min})=\partial _vp(V_{\min})=0,\\
        p(V_F)=0,\\
        p(V^-_R)=p(V^+_R),\\
        \partial _vp(V^-_R)=\partial _vp(V^+_R)+\partial _vp(V_F),
    \end{cases}
\end{equation}
where $\mathrm{P}_{\infty}(a,b)$ is the set of all real polynomials defined on the interval $(a, b)$. 
With integration by parts and the boundary conditions \eqref{boudary_condition} in the trial function space, the solution to the above variational problem \eqref{var_problem} agrees with the weak solution specified in Definition \ref{weak_solution}. 


Let $\{\psi_k\}_{k=0}^{\infty}$ be a set of basis functions of $\mathrm{W}$. The approximate solution of problem \eqref{eq:problem2} can be expanded as
\begin{equation}
    \label{eq:approximate_solution1}
    p(v,t)=\sum_{k=0}^{\infty}\hat{u}_k(t)\psi_k(v).
\end{equation}
The essence of constructing the trial function space $\mathrm{W}$ is to determine the specific form of its basis functions $\{\psi_k\}_{k=0}^{\infty}$. This is accomplished by dividing the interval into two segments by the discontinuity point $V_R$, utilizing a fixed number of basis functions to meet the dynamic boundary conditions, and employing basis functions with homogeneous boundary conditions for each segment to improve accuracy. That is
\begin{equation}
    \mathrm{W}=\mathrm{W}_1+\mathrm{W}_2,
\end{equation}
where $\mathrm{W}_1$ is a finite-dimensional space that handles the conditions in \eqref{boudary_condition}, and $\mathrm{W}_2$ enhances accuracy within the interval and satisfies the homogeneous conditions of points $V_{\min}$, $V_R$, and $V_F$, which is
\begin{equation}
    \label{eq:W2condition}
    \begin{cases}
        p(V_{\min})=\partial _vp(V_{\min})=0,\\
         p(V_R^-)=\partial _vp(V_R^-)=0,\\
         p(V_R^+)=\partial _vp(V_R^+)=0,\\
        p(V_F)=\partial _vp(V_F)=0,\\
    \end{cases}\qquad \forall p \in \mathrm{W}_2,
\end{equation}

For simplicity, it is preferable to keep the dimension of $\mathrm{W}_1$ as low as possible. In the case of taking into account the function value and first derivative value, there are eight degrees of freedom at the boundary, comprising of the function value and derivative value at $V_{\min}$, $V_F$, and both sides of $V_R$. Since the five conditions in \eqref{boudary_condition} have to be satisfied, there are three degrees of freedom remaining. Therefore, $\mathrm{W}_1$ can be spanned by three basis functions
\begin{equation}
    \mathrm{W}_1=\text{span}\{g_1,g_2,g_3\},
\end{equation}
where
\begin{equation}
    g_1\Rightarrow
    \begin{gathered}
        \begin{cases}
            g_1(V_{\min})=0,\\
            g_1(V_R)=1,\\
            \partial_v g_1(V_{\min})=0,\\
            \partial_v g_1(V_R)=0,
        \end{cases}\quad v\in(V_{\min},V_R),\qquad
        \begin{cases}
            g_1(V_R)=1,\\
            g_1(V_F)=0,\\
            \partial_v g_1(V_R)=0,\\
            \partial_v g_1(V_F)=0,
        \end{cases}\quad v\in(V_R,V_F).
    \end{gathered}
\end{equation}
\begin{equation}
    g_2\Rightarrow
    \begin{gathered}
        \begin{cases}
            g_2(V_{\min})=0,\\
            g_2(V_R)=0,\\
            \partial_v g_2(V_{\min})=0,\\
            \partial_v g_2(V_R)=1,
        \end{cases}\quad v\in(V_{\min},V_R), \qquad
        \begin{cases}
            g_2(V_R)=0,\\
            g_2(V_F)=0,\\
            \partial_v g_2(V_R)=1,\\
            \partial_v g_2(V_F)=0,
        \end{cases}\quad v\in(V_R,V_F).
    \end{gathered}
\end{equation}
\begin{equation}
    g_3\Rightarrow
    \begin{gathered}
    \begin{cases}
            g_3(V_{\min})=0,\\
            g_3(V_R)=0,\\
            \partial_v g_3(V_{\min})=0,\\
            \partial_v g_3(V_R)=0,
        \end{cases}\quad v\in(V_{\min},V_R), \qquad
    \begin{cases}
        g_3(V_R)=0,\\
        g_3(V_F)=0,\\
        \partial_v g_3(V_R)=1,\\
        \partial_v g_3(V_F)=1,
    \end{cases}\qquad v\in(V_R,V_F).\\
    \end{gathered}
\end{equation}
 The specific form of the basis functions in $\mathrm{W}_1$ are presented in Appendix \ref{app:basis}.

\begin{figure}[!htb]
    \centering
        \begin{minipage}[c]{0.8\textwidth}
            \centering
            \includegraphics[width=1\textwidth]{g.eps}
        \end{minipage}
         \caption{The basis functions of $p_3$ with Equation parameters $V_{\min}=-1,V_R=0,V_F=1$. Here, $g_1$ and $g_2$ are measured using the left axis, while $g_3$ is measured with the right axis.}
        \label{fig:g}
\end{figure}
The specific illustration of $g_1,g_2,g_3$ are shown in Figure \ref{fig:g}. After the basis function of $\mathrm{W}_1$ is determined in this way, we set
\begin{equation}
    \label{eq:w1_expansion}
    p_3=\sum_{k=1}^3 \lambda_kg_k \quad \in \mathrm{W}_1.
\end{equation}
The boundary conditions can be well satisfied by adjusting the coefficients of $g_1,g_2,g_3$ in the following way,
\begin{equation}
    \label{eq:lambda}
    \begin{gathered}
        \begin{cases}
            p(V_R^-)=p_3(V_R^-)=\lambda_1,\\
            p(V_R^-)=p_3(V_R^+)=\lambda_1,
        \end{cases}\qquad
        \begin{cases}
            \partial_vp(V_R^-)=\partial_vp_3(V_R^-)=\lambda_2,\\
            \partial_vp(V_R^-)=\partial_vp_3(V_R^+)=\lambda_2+\lambda_3,\\
            \partial_vp(V_R^-)=\partial_vp_3(V_F^-)=\lambda_3,
        \end{cases}
    \end{gathered}
\end{equation}
where $p$ denotes the numerical solution mentioned in \eqref{eq:approximate_solution1}. 

The construction of the $\mathrm{W}_2$ space is motivated by spectral methods for solving general homogeneous boundary value problems.  According to \eqref{eq:W2condition}, the interval $(V_{\min},V_F) $ is divided into two segments by $V_R$ naturally. Assuming $I_L=(V_{\min},V_R)$ and $I_R=(V_R,V_F)$, we further denote
\begin{equation}
    \label{eq:X_ab}
        \mathrm{X}_{(a,b)}=\left\{\varphi \in P_{\infty}(a,b) : \varphi(a)=\varphi(b)=\varphi'(a)=\varphi'(b)=0   \right\}.\\
\end{equation}
$\mathrm{X}_{(a,b)}$ represents the set of real polynomials defined on the interval $(a, b)$, where the function value and derivative are zero at boundary points. So we can divide $\mathrm{W}_2$ into two parts 
\begin{equation}
    \mathrm{W}_2=\mathrm{X}_{(V_{\min},V_R)}+ \mathrm{X}_{(V_R,V_F)}.
\end{equation}

In the spectral methods, in order to minimize the interaction of basis functions in the frequency space, the basis functions should take the form of adjacent orthogonal polynomials \cite{shen1994efficient}. Therefore, it is reasonable to use a compact combination of Legendre polynomials as basis functions of $\mathrm{X}_{(a,b)}$, namely,
\begin{equation}
    \hat{h}_k=\mathcal{H}_k+\alpha_k\mathcal{H}_{k+1}+\beta_k\mathcal{H}_{k+2}+\gamma_k\mathcal{H}_{k+3}+\eta_k\mathcal{H}_{k+4},\quad k=0,1,2,...,
\end{equation}
where $\mathcal{H}_k$ is the scaling of the kth-degree Legendre polynomial $L_k$
\begin{equation}
    \mathcal{H}_k(v)=L_k(x),\qquad x=\frac{v-\left(\frac{a+b}{2}\right)}{\frac{b-a}{2}},
\end{equation}
and the parameter $\{\alpha_k,\beta_k,\gamma_k,\eta_k\}$ are chosen to satisfy the boundary conditions in \eqref{eq:X_ab}
\begin{equation}
    \alpha_k=0,\, \beta_k=-\frac{4k+10}{2k+7},\,\gamma_k=0,\,\eta_k=\frac{2k+3}{2k+7}.
\end{equation}




After constructing the trial function space, spatial discretization will be discussed, yielding the system of ordinary differential equations for the coefficients. Let $\{h_k\}_{k=0}^{\infty}$ be the basis functions of $\mathrm{W}_2$. Then the approximate solution \eqref{eq:approximate_solution1} can be rewritten as
\begin{equation}
    \label{eq:approximate_solution2}
    p(v,t)=\sum_{k=0}^{\infty}u_k(t)h_k(v)+\sum_{k=1}^3\lambda_k(t)g_k(v).
\end{equation}
The basal functions in \eqref{eq:approximate_solution2} correspond to ones in \eqref{eq:approximate_solution1} in the following way
\begin{equation}
    \{\psi_k\}_{k=0}^{\infty}=\{g_k\}_{k=1}^{3}+\{h_k\}_{k=0}^{\infty}.
\end{equation}
And the expansion coefficients $\{u_k(t)\}_{k=0}^{\infty},\{\lambda_k(t)\}_{k=1}^3$ are to be determined.
Assuming the initial value is to satisfy the boundary conditions \eqref{boudary_condition}, the initial expansion coefficients $\{u_k(0)\}_{k=0}^{\infty},\{\lambda_k(0)\}_{k=1}^3$ can be obtained by the best squares approximation,
\begin{equation}
    \label{eq:initial_vector}
    \int_{V_{\min}}^{V_F} \left(\sum_{k=0}^{\infty}u_k(0)h_k(v)+\sum_{k=1}^3\lambda_k(t)g_k(v)\right) \phi_j dv=\int_{V_{\min}}^{V_F} p^0(v)\phi_j dv, \qquad \forall \phi_j \in \mathrm{V}.
\end{equation}
For a properly defined test function space, the solvability of the \eqref{eq:initial_vector} is guaranteed by the Gram-Schmidt orthogonalization of the basis functions. Note again that the specific form of the test function space is discussed in Section \ref{sec:stability}. We denote the initial value vector as
\begin{equation}
    \label{eq:initial_value}
    \mathbf{P^0}=(\lambda_1(0),\lambda_2(0),\lambda_3(0),u_1(0),u_2(0),...)^T.
\end{equation}

It should be noted that while constructing the basis functions, we assume that the value of $N(t)$ is already known. In fact, $N(t)$ is self-consistently determined in the dynamic process, and $N(t)$ is part of the degrees of freedom of the solution. It follows from \eqref{eq:lambda} that
\begin{equation}
    \partial_vp(V_F,t)=\lambda_3(t).
\end{equation}
One can rewrite the mean firing rate using \eqref{eq:ha} and \eqref{eq:Nt}
\begin{equation}
    \label{Nt1}
    N(t)=-\frac{a_0\lambda_3(t)}{1+a_1\lambda_3(t)}.
\end{equation}
Define
\begin{equation}
    \label{eq:operator}
    \mathcal{L}p(v,t)=\partial_{t}p-\partial_v(vp)-\left(b\frac{a_0\lambda_3(t)}{1+a_1\lambda_3(t)}\right)\partial_{v}p-\left(a_0-a_1\frac{a_0\lambda_3(t)}{1+a_1\lambda_3(t)}\right)\partial_{v v}p.
\end{equation}
The expansion coefficients $\{u_k(t)\}_{k=0}^{\infty},\,\{\lambda_k(t)\}_{k=1}^{3}(t>0)$ in \eqref{eq:approximate_solution2} can be determined by variational problem \eqref{var_problem} with using the mean firing rate $N(t)$ in \eqref{Nt1}:
\begin{equation}
    \label{eq:nonlinear_system1}
    \begin{cases}
        \text{Find } p\in  \mathrm{W} \text{ such that}\\
        (\mathcal{L}p,\phi_j)=0,\quad \forall \phi_j \in \mathrm{V},
    \end{cases}
\end{equation}
where $(\cdot,\cdot)$ is the inner product of the usual $L^2$ space.

The nonlinear system of ordinary differential equations of the above scheme is obtained by substituting \eqref{eq:approximate_solution2} into \eqref{eq:nonlinear_system1}. More precisely, setting
\begin{equation}
    \begin{aligned}
        &\mathbf{P}=(\lambda_1(t),\lambda_2(t),\lambda_3(t),u_1(t),u_2(t),...)^T,\\
        &s_{jk}=\begin{cases}
            (g_k,\phi_j),\qquad &1\leq k \leq3,\\
            (h_{k-4},\phi_j),& k\geq4.
        \end{cases}, &S=(s_{jk})_{j,k=1,2,...},\\
        &a_{jk}=\begin{cases}
            (\partial_v(vg_k),\phi_j),\qquad &1\leq k \leq3,\\
            (\partial_v(vh_{k-4}),\phi_j),& k\geq4.
        \end{cases}, &A=(a_{jk})_{j,k=1,2,...},\\
        &b_{jk}=\begin{cases}
            (\partial_vg_k,\phi_j),\qquad &1\leq k \leq3,\\
            (\partial_vh_{k-4},\phi_j),& k\geq4.
        \end{cases}, &B=(b_{jk})_{j,k=1,2,...},\\
        &c_{jk}=\begin{cases}
            (\partial_{vv}g_k,\phi_j),\qquad &1\leq k \leq3,\\
            (\partial_{vv}h_{k-4},\phi_j),& k\geq4.
        \end{cases}, &C=(c_{jk})_{j,k=1,2,...}.\\
    \end{aligned}
\end{equation}
The nonlinear system \eqref{eq:nonlinear_system1} becomes
\begin{equation}
    \label{eq:nonlinear_sde}
    S\partial_t\mathbf{P}=\left(A+\left(b\frac{a_0\lambda_3(t)}{1+a_1\lambda_3(t)}\right)B+\left(a_0-a_1\frac{a_0\lambda_3(t)}{1+a_1\lambda_3(t)}\right)C\right)\mathbf{P}.
\end{equation}
After the spatial discretization, the solution of problem \eqref{eq:problem2} converts into the solution of the nonlinear ordinary differential equation system of initial value problem \eqref{eq:nonlinear_sde}\eqref{eq:initial_value}.

\subsubsection{Fully discrete numerical scheme}

To finish the construction of the numerical scheme, we need to truncate the approximate solution \eqref{eq:approximate_solution2} to a finite-dimensional one and perform time discretization. The finite-dimensional form of \eqref{eq:X_ab} is denoted as
\begin{equation}
\label{eq:X_N}
    \mathrm{X}_{N(a,b)}=\left\{\varphi \in P_{N+3}(a,b) : \varphi(a)=\varphi(b)=\varphi'(a)=\varphi'(b)=0   \right\},
\end{equation}
where $P_N(a,b)$ is the set of all real polynomials of degree no more than $N$ and the dimension of $P_N{(a,b)}$ is $N+1$. It is evident that a  non-trivial polynomial with the homogeneous boundary conditions in \eqref{eq:X_N} must be of at least fourth degree, thus leading to a reduced dimension of the set in \eqref{eq:X_N}. The polynomial space in \eqref{eq:X_N} is selected as $P_{N+3}(a,b)$ so that the dimension of the $\mathrm{X}_{N(a,b)}$ space is $N$. Then the trial function space can be truncated as

\begin{equation}
    \label{eq:trial_function}
    \mathrm{W}_N=\mathrm{X}_{N(V_{\min},V_R)} + \mathrm{X}_{N(V_R,V_F)} + \mathrm{W}_1.
\end{equation}
Assuming $\{h^L_k\}_{k=0}^{N-1}$ is a set
of basis functions of $X_{N(V_{\min},V_R)}$ and $\{h^R_k\}_{k=0}^{N-1}$ is a set
of basis functions of $X_{N(V_R,V_F)}$. $\{\psi_k\}_{k=1}^{2N+3}=\{h^L_0,...,h^L_{N-1},h^R_0,...,h^R_{N-1},g_1,g_2,g_3\}$ is a basis of $\mathrm{W}_N$. Then the numerical solution $p_N(v,t)$ can be expressed as
\begin{equation}
    \label{eq:approximate_solution3}
    p_N(v,t)=\sum_{k=0}^{N-1} u_k^L(t)h^L_k(v)+\sum_{k=0}^{N-1} u_k^R(t)h^R_k(v)+\sum_{k=1}^3 \lambda_k(t)g_k(v)=\sum_{k=1}^{2N+3}\hat{u}_k(t)\psi_k(v).
\end{equation}
 The initial condition for the expansion coefficients $\{\hat{u}_{k}(0)\}_{k=0}^{2N+3}$ can be obtained by the least square approximation,
\begin{equation}
    \label{eq:initial_vector2}
    \int_{V_{\min}}^{V_F} \sum_{k=1}^{2N+3}\hat{u}_k(0)\psi_k(v) \phi_j dv=\int_{V_{\min}}^{V_F} p^0(v)\phi_j dv, \qquad \forall \phi_j \in \mathrm{V}_N.
\end{equation}

Suppose the truncated test function space is denoted by $\mathrm{V}_N$, which shall be specified later. The expansion coefficients $\{\hat{u}_k(t)\}_{k=0}^{2N+3}(t>0)$ can be determined by the semi-discrete variational formulation
\begin{equation}
    \label{eq:variational_form2}
    \begin{cases}
        \text{Find } p_N\in \mathrm{W}_{N} \text{ such that}\\
        (\mathcal{L}p_N,\phi_j)=0,\quad \forall \phi_j \in \mathrm{V}_N.
    \end{cases}
\end{equation}


For time discretization, we use a semi-implicit method. The interval $[0,T_{\text{max}}]$ is divided  into $n_t$ equal sub-intervals with size
\begin{equation}
    \Delta t=\frac{T_{\text{max}}}{n_t},
\end{equation}
and the grid points can be represented as follows
\begin{equation}
    t^{n}=n \Delta t, \qquad n=0,1,2, \cdots, n_{t}.
\end{equation}


The semi-implicit scheme of \eqref{eq:operator} is denoted by
\begin{equation}
    \label{eq:semi_implicit}
\begin{aligned}
    \tilde{\mathcal{L}}p_N(v,t^{n+1})&=\frac{p_N(v,t^{n+1})-p_N(v,t^{n})}{\Delta t}-\partial_v(vp_N(v,t^{n+1}))+bN(t^n)\partial_vp_N(v,t^{n+1})\\
    &-a(N(t^n))\partial_{vv}p_N(v,t^{n+1})=0,\qquad\qquad\qquad n=1,2,...,n_t.
\end{aligned}
\end{equation}
Note that, the mean firing rate $N(t^n)$ is treated explicitly, but the rest of the terms are implicit. Such a time discretization naturally avoids the use of a nonlinear solver. Then we can obtain the fully discrete scheme of the variational formulation \eqref{eq:variational_form2}: for each time step
\begin{equation}
    \label{eq:variational_form3}
    \begin{cases}
       \text{Find } p_N\in \mathrm{W}_{N} \text{ such that}\\
        (\tilde{\mathcal{L}}p_N,\phi_j)=0,\quad \forall \phi_j \in \mathrm{V}_N.
    \end{cases}
\end{equation}

More precisely, setting
\begin{equation}
    \label{eq:Matrix2}
    \begin{aligned}
        &\hat{\mathbf{P}}^n=(\hat{u}_1(t^n),\hat{u}_2(t^n),...,\hat{u}_{2N+3}(t^n))^T,\\
        &\hat{s}_{jk}=(\psi_k,\phi_j),\quad \hat{S}=(\hat{s}_{jk})_{k=1,...,2N+3}\\
        &\hat{a}_{jk}=(\partial_v(v\psi_k),\phi_j),\quad \hat{A}=(\hat{a}_{jk})_{k=1,...,2N+3}\\
        &\hat{b}_{jk}=(\partial_{v}\psi_k,\phi_j),\quad \hat{B}=(\hat{b}_{jk})_{k=1,...,2N+3}\\
        &\hat{c}_{jk}=(\partial_{vv}\psi_k,\phi_j),\quad \hat{C}=(\hat{c}_{jk})_{k=1,...,2N+3},
    \end{aligned}
\end{equation}
the  variational formulation \eqref{eq:variational_form3} reduces to
\begin{equation}
\label{eq:system2}
    \left(\frac{\hat{S}}{\Delta t}-\hat{A}+bN(t^n)\hat{B}-a(N(t^n))\hat{C}\right)\hat{\mathbf{P}}^{n+1}=\frac{\hat{S}}{\Delta t}\hat{\mathbf{P}}^n.
\end{equation}











\subsection{Stability and the choice of test functions} \label{sec:stability}

The dynamic boundary conditions also give rise to challenges in choosing proper finite-dimensional test function spaces.  As we shall elaborate below, the construction of the trial function is so delicate that we can not simply choose the test functions only out of accuracy. Our goal is to find test functions that result in a stable evolution system in the discrete setting, and we hope the total mass is conserved with satisfactory accuracy. 


%The conservation property is the basic property of the model, and stability is an essential factor in ensuring the success of the numerical scheme. Having proposed the fully discrete numerical scheme in Section \ref{sec:fully_discrete_scheme}, it is necessary to determine an appropriate test function space, as it affects the properties of the numerical solution. 

To this end, two propositions are introduced that relate the test functions to the conservation and stability of the semi-discrete scheme \eqref{eq:variational_form2} in the linear case. However, the spectral method is often not able to completely ensure the conservation of mass, therefore it is not serving as a rigid criterion. The stability of the numerical solution is instead analyzed through its long-term asymptotic behavior in the linear regime, which will be discussed in greater detail below. Following this, three different test function spaces are analyzed respectively.


When analyzing the impact of the test function space, we are to consider the semi-discrete system \eqref{eq:variational_form2}. Thanks to the definition in \eqref{eq:Matrix2}, the system can reduce to
\begin{equation}
    \label{eq:SDE_system2}
    \hat{S}\partial_t\hat{\mathbf{P}}=\left(\hat{A}-\left(bN(t)\right)\hat{B}+\left(a(N(t))\right)\hat{C}\right)\hat{\mathbf{P}},\quad \hat{\mathbf{P}}(0)=\hat{\mathbf{P}}^0,
\end{equation}
where $\hat{\mathbf{P}}=(\hat{u}_1(t),\hat{u}_2(t),...,\hat{u}_{2N+3}(t))^T,\,\hat{\mathbf{P}}^0=(\hat{u}_1(0),\hat{u}_2(0),...,\hat{u}_{2N+3}(0))^T$. For simplicity, we study the case of a linear equation that is $b=0$ and $a(N)=1$. Then the nonlinear system \eqref{eq:SDE_system2} becomes a linear system
\begin{equation}
    \label{eq:linear_system2}
    \hat{S}\partial_t \hat{\mathbf{P}}=(\hat{A}+\hat{C})\hat{\mathbf{P}}, \quad \hat{\mathbf{P}}(0)=\hat{\mathbf{P}}^0.
\end{equation}
Considering the unique solvability of ordinary differential equations, we assume that the matrices $\hat{S}$, $\hat{A}$, and $\hat{C}$ are square matrices of order $2N+3$ and the matrix $\hat{S}$ is invertible. Let $\hat{K}=\hat{S}^{-1}(\hat{A}+\hat{C})$, then the system \eqref{eq:linear_system2} can be rewritten as
\begin{equation}
\label{eq:linear_system3}
    \hat{\mathbf{P}}_t=\hat{K}\hat{\mathbf{P}}.
\end{equation}
Let $\hat{\mathbf{P}}^{\infty}=(\hat{u}_1^{\infty},\hat{u}_2^{\infty},...,\hat{u}_{2N+3}^{\infty})^T$ be the steady-state solution of the equation. It holds that
\begin{equation}
    \label{eq:steady}
    \hat{K}\hat{\mathbf{P}}^{\infty}=0.
\end{equation}
The steady-state equation \eqref{eq:steady} has a nonzero solution if and only if the matrix $\hat{K}$ has at least one zero eigenvalue. With a prescribed test function space, the properties of the scheme can be assessed by inspecting the elements of matrix $\hat{K}$, allowing us to fully characterize the system's behavior. The following propositions serve to elucidate this connection.

\begin{proposition}[mass conservation]\label{prop1}
    Consider the Fokker-Planck equation \eqref{eq:problem2} with $a=1, b=0$ and the semi-discrete scheme \eqref{eq:linear_system3} where the dimension of test function space $V_N$ is $2N+3$. The following relations hold:
\begin{enumerate}
    \item Matrix $\hat{K}$ has zero eigenvalue if and only if the test function space $\mathrm{V}_N$ contains constant functions.
    \item If the matrix $\hat{K}$ has zero eigenvalue, then the total mass of the numerical solution solved by system \eqref{eq:linear_system2} does not change with time. That is
    \begin{equation}
        \int_{V_{\min}}^{V_F}\partial_tp_N(v,t) dv=0,
    \end{equation}
    where $p_N(v,t)$ is defined in \eqref{eq:approximate_solution3}.
\end{enumerate}
\end{proposition}
\begin{proof}
\textbf{Proof of (1)}. 
If the test function space $\mathrm{V}_N$ contains constants, without loss of generality, let $\phi_j=1$. Using the definition in \eqref{eq:Matrix2}, $\forall \varphi_i \in \mathrm{W}_N$, we can derive that
\begin{equation}
    \label{eq:integral}
    \begin{aligned}
        &\int_{V_{\min}}^{V_F}\partial_v(v\varphi_i) dv=v\varphi_i |_{V_{\min}}^{V_R^-}+v\varphi_i |_{V_R^+}^{V_F}=0,\\
        &\int_{V_{\min}}^{V_F}\partial_{vv}\varphi_i dv=\partial_v\varphi_i |_{V_{\min}}^{V_R^-}+\partial_v\varphi_i |_{V_R^+}^{V_F}=0.
    \end{aligned}
\end{equation}
So the elements of the $j$th row of the matrices $A$ and $C$ are all zeros. Since $\hat{S}$ is invertible, $S^{-1}$ is a full-rank matrix. Then
\begin{equation}
    \text{rank}(\hat{K})=\text{rank}(\hat{S}^{-1}(\hat{A}+\hat{C}))=\text{rank}(\hat{A}+\hat{C})<2N+3.
\end{equation}
So $\hat{K}$ has a zero eigenvalue.


If matrix $\hat{K}$ has a zero eigenvalue, then matrix $(\hat{A}+\hat{C})$ has zero eigenvalue for $\hat{S}$ is invertible. Therefore, the matrix $(\hat{A}+\hat{C})$ can make the elements in the jth row all zero through the matrix transformation. Notice that, performing matrix row transformation on matrix $(\hat{A}+\hat{C})$ corresponds to replacing the test function in linear system \eqref{eq:linear_system2}  with the linear combination of the original test function. Without loss of generality, we assume that the elements in the jth row of matrix $(\hat{A}+\hat{C})$ are all zeros and the corresponding test function is $\phi$. Using the definition in \eqref{eq:Matrix2}, $\forall \varphi_i \in \mathrm{W}_N$,  it holds that
\begin{equation}
    \begin{aligned}
        \int_{V_{\min}}^{V_F}\partial_v(v\varphi_i) +\partial_{vv}\varphi_i dv=-\int_{V_{\min}}^{V_F} (v\varphi_i+\partial_v\varphi_i)\partial_v\phi dv=0.
    \end{aligned}
\end{equation}
Since the above formula holds for all $\varphi_i \in \mathrm{W}_N$, so $\partial_v\phi=0$, that is $\phi= \text{constant}$.

\textbf{Proof of (2)}.
From (1), we know that when the matrix $\hat{K}$ has a zero eigenvalue,  the constant function $C_1 \in V_N$. Substituting $\phi_j=1$ into \eqref{eq:variational_form2} with $a=1,b=0$,
\begin{equation}
    \int_{V_{\min}}^{V_F} \partial_t p_N dv-\int_{V_{\min}}^{V_F}(\partial(vp_N)+\partial_{vv}p_N) dv\overset{\eqref{eq:integral}}{=}\partial_t \int_{V_{\min}}^{V_F} p_N dv=0.
\end{equation}
So the total mass does not change over time.
\end{proof}

\begin{proposition}[Stability]\label{prop2}
    Consider the Fokker-Planck equation \eqref{eq:problem2} with $a=1, b=0$ and the semi-discrete scheme \eqref{eq:linear_system3} . A necessary condition for the stability of the method is that all the eigenvalues of the matrix $\hat{K}$ are non-positive.
\end{proposition}
Note that a modified stability criterion is proposed here because the traditional stability conclusion cannot be applied due to the complexity of the equation. In the linear case, the equation has a unique steady state \cite{caceres2011analysis}, and the solution of the equation will converge exponentially to the steady state, so the discretized kinetic equation can only have non-positive eigenvalues. When there are positive eigenvalues, it means that the numerical scheme is unstable.

Following the theoretical analysis, we can now discuss the specific test function space. Our goal is to select suitable test function spaces such that the constructed numerical scheme is stable and preserves the original properties of the Fokker-Planck equation \eqref{eq:problem2} to the greatest extent, such as mass conservation. The Galerkin method is widely used in spectral methods \cite{shen2011spectral}. Consequently, Legendre-Galerkin Method is proposed below.
\paragraph{Legendre-Galerkin Method (LGM)}
The test function space is chosen to be the same as the trial function space. Applied to the semi-discrete method \eqref{eq:variational_form2} or its fully discrete version \eqref{eq:variational_form3} as 
\begin{equation}
    \mathrm{V}_N=\mathrm{W}_N,
\end{equation}
where $\mathrm{V}_N$ is test function space, and $\mathrm{W}_N$ is trial function space defined in \eqref{eq:trial_function}, we obtain a Legendre-Galerkin Method (LGM for short) for the model \eqref{eq:problem1}. 

The LGM method is numerically stable but total mass is not well conserved in dynamics. When constructing the trial function space, some low-order polynomials, especially constants, are discarded in order to satisfy the boundary conditions. For the LGM, the test function space does not contain constants, which fails to ensure mass conservation as stated in Proposition \ref{prop1}. Hence, this method can be used for  finite-time simulations, yet it is unsuitable for capturing long-time behavior or multiscale problems.


 According to Proposition \ref{prop1}, to improve the mass conservation property of the LGM, it seems that we may replace one of the basis functions in  the test function space with the constant function $1$. Say, we may consider the modified test function space
\begin{equation} \label{tildeV}
    \tilde{\mathrm{V}}_N=\mathrm{W}_N-\{\psi_k\} +\{1\},
\end{equation}
where $\{\psi_k\}(k=1,...,2N+3)$ is the basis function of the $\mathrm{W}_N$ space.

In this case, the mass of the numerical solution appears invariant. However, as shown in Figure \ref{fig:stable}, the matrix $\hat{K}$ in \eqref{eq:linear_system3} has positive eigenvalues for some $N$, which makes the method unstable, which agrees with Proposition \ref{prop2}. In fact, Figure \ref{fig:stable} shows that the maximum eigenvalue of the matrix $\hat{K}$ is significantly positive large  for when $N$ is odd and when the modified test function space $\tilde{\mathrm{V}}_N$ is used. Hence, we need to resort to other strategies for enhancing mass conservation. 
\begin{figure}[!htb]
    \centering
        \begin{minipage}[c]{0.49\textwidth}
            \centering
            \includegraphics[width=1\textwidth]{Eig.eps}
        \end{minipage}
        \begin{minipage}[c]{0.49\textwidth}
            \centering
            \includegraphics[width=1\textwidth]{3step.eps}
        \end{minipage}
        \caption{When the test function space is $\tilde{\mathrm{V}}_N$ \eqref{tildeV}, the numerical method might be unstable. Left: The maximum eigenvalue of matrix $\hat{K}$ at different $N$. Right: A typical unstable solution. Equation parameters $a=1, b=0$ with Gaussian initial condition $v_0=-1, \sigma_0^2=0.5$ and $N=11,\Delta t=0.001$.}
        \label{fig:stable}
\end{figure}



\paragraph{Modified Petrov-Galerkin Method (MPGM)} We propose an alternative formulation of the test function space by extending the test function space with one additional basis function $1$. As a consequence, the dimension of the test function space is larger than that of the trial function space, which results in an overdetermined system, and we solve such a system using the Least-Squares method. 

More precisely,  for the semi-discrete method \eqref{eq:variational_form2} or its fully discrete version \eqref{eq:approximate_solution3}, constants are added to  form an augmented test function space
\begin{equation}
    \mathrm{V}_N=\mathrm{W}_N +\{1\}.
\end{equation}
where $\mathrm{W}_N$ is trial function space defined in \eqref{eq:trial_function}, and we thus obtain the modified Petrov-Galerkin Method (MPGM for short).

Note that, the dimension of the test function space is higher than that of the trial function space by $1$. Multiplying \eqref{eq:linear_system2} from the left by $\hat{S}^T$ , the least square solution satisfies
\begin{equation}
    \hat{S}^T\hat{S}\partial_t \hat{\mathbf{P}}=\hat{S}^T(\hat{A}+\hat{C})\hat{\mathbf{P}}.
\end{equation}
The matrix $\hat{K}$ in \eqref{eq:linear_system3} can be written as
\begin{equation}
    \hat{K}=(\hat{S}^T\hat{S})^{-1}(\hat{S}^T(\hat{A}+\hat{C})).
\end{equation}
The numerical solution is not completely mass-conserving due to the use of the least-squares method. But compared with the LGM, the mass of the numerical solution of the MPGM changes very little over time, as shown in Figure \ref{mass}. With extensive tests, the matrix $\hat{K}$ has no positive eigenvalues, and the MPGM is numerically stable.
\begin{figure}[!htb]
    \centering
        \begin{minipage}[c]{0.49\textwidth}
            \centering
            \includegraphics[width=7cm]{mass1.eps}
        \end{minipage}
        \begin{minipage}[c]{0.49\textwidth}
            \centering
            \includegraphics[width=7cm]{mass2.eps}
        \end{minipage}
        \caption{Equation parameters $a=1, b=0$ with Gaussian initial condition $v_0=-1, \sigma_0^2=0.5$ and $N=10,\Delta t=0.001,T_\text{max}=5$. Left: Variation of total mass with time by the LGM. Right: Variation of total mass with time by the MPGM.}
        \label{mass}
\end{figure}



In conclusion, we have proposed two methods, i.e. the LGM and the MPGM, for the model problem \eqref{eq:problem1},  each possessing different advantages and therefore should be used in a flexible manner. The MPGM is preferred for simulating long-time behavior and testing the asymptotic preserving properties of the model, as the mass of the numerical solution from the LGM is significantly diminished in a long time. On the other hand, the LGM can be utilized for $\mathcal O(1)$ time simulations with verifiable order of convergence, and it does not involve the error due to the least square approximation.





%% \vspace{-0.20in}
%\subsection{Analysis:}


\textbf{Use of Multiple Projection Heads:} The use of different projection heads for each view on OpenImages classification gives us a boost of $1.1$ mAP on Obj-Obj+Dilate crop. Pre-training on COCO and finetuning on VOC dataset for object-detection task gives a boost of $0.4$ mAP. Hence using multiple projection heads results in a consistent improvement. 

\textbf{Varying Dilation Parameter:} Table 3 (appendix) shows the effect of varying the dilation parameter. A sweet spot exists at a moderate dilation value of $\delta=0.1$ for COCO object detection. 

% \textbf{Computational Cost:} BING adds negligible time to the pre-training. Generating object proposals takes ~29 mins for the full OHMS dataset (one-time cost) and ~16 mins for COCO. Instead of pre-generating, adding the BING operator to the data loader pipeline has a trivial overhead (+$0.1\%$). %As an example, the wall-clock time taken for 1 epoch of training is 1'46'' for the Dense-CL baseline and 1'45'' for our method.
% \textbf{}



%Between two views, we measure the number of common pixels; and then measure the fraction of these common pixels that overlap with a ground truth bounding box (object). We find that this fraction for COCO is $99\%$ for object-scene crops and $92.1\%$ for the scene-scene crop. In the case of OpenImages-Subset, the numbers are, respectively, $99.1\%$ and $87.3\%$. This is another way of seeing that OpenImages-Subset can benefit more from object-scene crops, borne out by the numbers in Tables \ref{tab:ssl_comparison_classification} and \ref{tab:coco_detection}. 


% \as{Shlok: could you please make this description a little better and clear?}
% We find the overlapping pixels between two crops ($C_{int} = C_1 \cap C_2$). Next we calculate intersection of $C_{int}$  with the most overlapping ground truth object ($O$) and calculate the score $\frac{C_{int} \cap O}{C_{int}}$ for each image and average it. 
% To do this, we calculate the \% intersection of the most overlapping ground truth object with the inter
% Next we try to find the probability of an actual ground truth object co-occuring in between two crops. We find  object-overlap between both Scene-Object crops and Scene-Scene crops. To do this we firstly calculate the overlapping region between two crops. Overlapping region is the area of overlap between two crops before the resize operation. Then for all the ground truth objects present in the original image  we find the object with maximum overlap in the overlapping region. Intuitively for a object to have high overlap, the object should be present in both the crops. 

% Similarly instead of taking an crop with maximum overlap we calculate average of all the crops that are present in the image. We find this average probability to be 65.12 \% for Object-Scene crop and 73.47 \% for Scene-Scene crop. 
% This is consistent with the findings of the InfoMin \cite{tian2020contrastive} that there is a tradeoff between how much information views can share.  

% Similarly in the case of OpenImages we can see from Fig \ref{fig:radius_openimages} that as we increase the radius of the object-object crops the performance firstly increases and then decreases, suggesting there is a sweet point on mutual information on OpenImages dataset as well.
% \\

% \textbf{Performance on 5 classes per image images?}









\section{NNLIF with learning rules}\label{sec:lr}
In this Section, we consider the NNLIF model with a learning rule which is an extension of the Fokker-Planck equation \eqref{eq:problem1}, involving synaptic weights and the Hebbian learning rule. This is a novel and intriguing model and the dynamics of the membrane potential $v$ and the synaptic weight $w$ are on different time scales, making numerical simulation far more challenging. In order to better understand this model and verify the generality of the method proposed in Section \ref{sec:scheme}, we further explore this model from a numerical perspective. 

\subsection{Model introduction}


Compared with the simplest form of NNLIF model, the NNLIF model with learning rules introduces a new variable, the synaptic weight $w$, which is also the connectivity of the network $b$ mentioned in \eqref{eq:ha}. Furthermore, an external input function $I(w,t)$ is added to the drift coefficient $h$
\begin{equation}
    h(w,N(t))=-v+I(w,t)+w\sigma(N(t)).
\end{equation}
The function $\sigma(\cdot)$ represents the response of the network to the total activity, usually taking $\sigma(N)=N$. Then the Fokker-Planck equation without learning rules can be written as 
\begin{equation}
    \label{eq:problem3}
    \begin{cases}
        \partial_{t}p+\partial_{v}((-v+I(w,t)+w\sigma(\bar{N}(t)))p)-a\partial_{v v}p=0,\qquad v\in(-\infty,V_F]/\{V_R\},\\
        p(v,w,0)=p^0(v,w),\qquad p(-\infty,w,t)=p(V_F,w,t)=0,\\
        p(V^-_R,w,t)=p(V^+_R,w,t),\quad \partial _vp(V^-_R,w,t)=\partial _vp(V^+_R,w,t)+\frac{N(w,t)}{a},\\
    \end{cases}
\end{equation}
where, $p(v,w,t)$ describes the probability of finding a neuron at voltage $v$, synaptic weight $w$ and given time $t$. The diffusion coefficient $a$ is the same as in \eqref{eq:ha}. The subnetwork activity $N(w, t)$ and total activity
$\bar{N}(t)$ are defined as
\begin{equation}
    N(w,t)=-a\frac{\partial p}{\partial v}(V_F,w,t)\geq 0,\quad \bar{N}(t)=\int_{-\infty}^{+\infty}N(w,t)dw.
\end{equation}
Then we define the probability density of finding a neuron at synaptic weight $w$ and given time $t$ by 
\begin{equation}
    H(w, t)=\int_{-\infty}^{V_{F}} p(v, w, t) d v, \quad \int_{-\infty}^{\infty} H(w, t) d w=1 .
\end{equation}


In this case of no learning rule, the function $H(w,t)$ is time-independent because the distribution of synaptic weights in \eqref{eq:problem3} is fixed. The input signal $I(w,t)$ can be reflected by an output signal related to network activity $N(w,t)$. Next, we employ the learning rules of \cite{perthame2017distributed} to modulate the distribution of synaptic weights $H$, enabling the network to discriminate specific input signals $I$ by choosing an apposite synaptic weight distribution $H$ that is adapted to the signal $I$.


In \cite{perthame2017distributed}, the authors choose learning rules inspired by the seminal Hebbian rule and assume synaptic weights described with a single parameter $w$ and the subnetworks interact only via the total rate $\bar{N}$. They elucidate that all subnetworks parameterized by $w$ can vary their intrinsic synaptic weights $w$ according to a function $\Phi$ that is based on the intrinsic activity $N(w)$ of the network and the total activity of the network $\bar{N}$. Then, they give the generalization choice of Hebbian rule
\begin{equation}
    \Phi(N(w), \bar{N})=\bar{N} N(w) K(w),
\end{equation}
where $K(\cdot)$ represents the learning strength of the subnetwork with synaptic weight $w$. Adding the above choice of learning rule, the Fokker-Planck equation with learning rules is given by
\begin{equation}
    \label{eq:problem40}
    \frac{\partial p}{\partial t}+\frac{\partial}{\partial v}[(-v+I(w,t)+w \sigma(\bar{N}(t))) p]+\varepsilon \frac{\partial}{\partial w}[(\Phi-w) p]-a \frac{\partial^{2} p}{\partial v^{2}}=N(w, t) \delta\left(v-V_{R}\right).
\end{equation}
In order to better apply the numerical scheme and study the learning behavior of the model, we consider the equation \eqref{eq:problem40} for time rescaling $t \rightarrow t / \varepsilon$ and convert $\delta$-function to dynamic boundary condition such as:
\begin{equation}
    \label{eq:problem4}
    \begin{cases}
        \displaystyle
        \frac{\partial p}{\partial t}+\frac{\partial}{\partial w}[(\bar{N}(t)N(w,t)K(w)-w)p]
        =\frac{1}{\varepsilon}\left\{a\frac{\partial^2p}{\partial v^2}-\frac{\partial}{\partial v}[(-v+I(w,t)+w\sigma(\bar{N}(t)))p]\right\},\\
        p(v,w,0)=p^0(v,w),p(V_F,w,t)=p(-\infty,w,t)=p(v,\pm \infty,t)=0,\\
        p(V_R^-,w,t)=p(V_R^+,w,t),\qquad \frac{\partial}{\partial v}p(V^-_R,w,t)=\frac{\partial}{\partial v}p(V^+_R,w,t)+\frac{N(w,t)}{a}.
    \end{cases}
\end{equation}

Here, $p^0(v,w)$ is initial condition and the probability density
function p(v, t) should satisfy the condition of conservation of mass
\begin{equation}
    \int_{-\infty}^{\infty} \int_{-\infty}^{V_{F}} p(v, w, t) d v d w=\int_{-\infty}^{\infty} \int_{-\infty}^{V_{F}} p^{0}(v, w) d v d w=1.
\end{equation}



Despite some research on model \eqref{eq:problem4} as indicated by the theoretical properties presented in \cite{perthame2017distributed} and the numerical analysis and experiments in \cite{he2022structure}, it is still a relatively new model with limited established knowledge. In this paper, the numerical method proposed in Section \ref{sec:scheme} is used to further investigate the learning behaviors of this model numerically.

\subsection{Numerical scheme}
Now, we describe the numerical scheme for \eqref{eq:problem4}. We choose the calculation interval as $[V_{\text{min}},V_F]\times [W_{\text{min}},W_{\text{max}}]\times [0,T_{\text{max}}]$ and suppose the density function is practically negligible out of this region.
We use spectral methods for v-wise discretization and Differential method for w-wise and t-wise discretization. So we divide the interval $[W_{\text{min}},W_{\text{max}}],[0,T_{\text{max}}]$ into $n_w,n_t$ equal sub-intervals with size
\begin{equation}
    \Delta w=\frac{W_{\text{max}}-W_{\text{min}}}{n_w},\Delta t=\frac{T_{\text{max}}}{n_t}.
\end{equation}
Then the grid points can be represented as follows
\begin{equation}
    \begin{aligned}
        &w_{j}=W_{\text{min} }+j \Delta w, & j=0,1,2, \cdots, n_{w} \\
        &t^{n}=n \Delta t, & n=0,1,2, \cdots, n_{t}
    \end{aligned}
\end{equation}
For the v-direction discretization, we take the same scheme as in Section \ref{sec:fully_discrete_scheme}. The approximate solution is expended as 
\begin{equation}
    \label{eq:approximate_solution4}
    p_N(v,w,t)=\sum_{k=1}^{2N+3}\hat{u}_k(w,t)\psi_k(v).
\end{equation}
The initial condition for the expansion coefficients $\{\hat{u}_{k}(w,0)\}_{k=0}^{2N+3}$ can be obtained by the least square approximation,
\begin{equation}
    \label{eq:initial_vector3}
    \int_{V_{\min}}^{V_F} \sum_{k=1}^{2N+3}\hat{u}_k(w_j,0)\psi_k(v) \phi_i dv=\int_{V_{\min}}^{V_F} p^0(w_j,v)\phi_i dv, \quad j=0,1,2, \cdots, n_{w} \quad \forall \phi_i \in \mathrm{V}_N.
\end{equation}
From the properties of the basis functions \eqref{eq:lambda}, subnetwork activity $N(w,t)$ can be expressed as
\begin{equation}
    N^n_j=N(w_j,t^n)=-a\hat{u}_{2N+3}(w_j,t^n).
\end{equation}
And we apply the simplest rectangular numerical integration rule to discretize the total activity $\bar{N}(t)$
\begin{equation}
    \bar{N}^n=\Delta w \sum_{j=0}^{n_w}N^n_j.
\end{equation}
For the w-direction discretization, we inherit the idea form \cite{he2022structure} which takes the following explicit flux construction adapted from Godunov's Method 
 \begin{equation}
    \Phi_{i, j+\frac{1}{2}}^{n}=
        \begin{cases}
            \begin{cases}
                \min \left\{\Phi_{i, j}^{n}, \Phi_{i, j+1}^{n}\right\} \qquad &\hat{P}_{i, j}^{n} \leq \hat{P}_{i, j+1}^{n} \\
                \max \left\{\Phi_{i, j}^{n}, \Phi_{i, j+1}^{n}\right\}  &\hat{P}_{i, j}^{n}>\hat{P}_{i, j+1}^{n} \\
            \end{cases}&j=0, \cdots, n_{w}-1\\
            0  &j=-1, n_{w}
        \end{cases}
    \end{equation}
where
\begin{equation}
    \Phi_{i, j}^{n}=\left(\bar{N}^{n} N_{j}^{n} K\left(w_{j}\right)-w_{j}\right) \hat{P}_{i, j}^{n} \quad \text { for } \quad j=0, \cdots, n_{w}.
\end{equation}
$\hat{P}_{i,j}^n$ is the coefficients of the basis functions in \eqref{eq:approximate_solution4}
\begin{equation}
    \hat{P}_{i,j}^n=\hat{u}_i(w_j,t^n).
\end{equation}
Define 
\begin{equation}
\begin{aligned}
    &p_{N,j}^{n}=\sum_{k=1}^{2N+3}\hat{u}_k(w_j,t^n)\psi_k(v),\\
    &q_{N,j+\frac{1}{2}}^n=\sum_{k=1}^{2N+3} \Phi_{k,j+\frac{1}{2}}^{n}\psi_k(v).
\end{aligned}
\end{equation}
After using a semi-implicit method for time discretization, we obtain the fully discrete scheme as follows:
\begin{equation}
    \frac{p_{N,j}^{n+1}-p_{N,j}^{n}}{\Delta t}+\frac{q_{N,j+\frac{1}{2}}^n-q_{N,j-\frac{1}{2}}^n}{\Delta w}=\frac{1}{\varepsilon}\left\{a\frac{\partial^2p_{N,j}^{n+1}}{\partial v^2}-\frac{\partial}{\partial v}\left[(-v+I(w_j)+w_j\sigma(\bar{N}(t^n)))p_{N,j}^{n+1}\right]\right\}.
\end{equation}
When the test function space $\mathrm{V}_N$ is given, the coefficients of the approximate solution \eqref{eq:approximate_solution4} for each $t$ and $w$ step can be obtained by the following linear system
\begin{equation}
    \begin{aligned}
        &\frac{\hat{S}(\hat{\mathbf{P}}^{n+1}_{j}-\hat{\mathbf{P}}^n_{j})}{\Delta t}+\frac{\hat{S}(\mathbf{\Phi}_{j+\frac{1}{2}}^n-\mathbf{\Phi}_{j-\frac{1}{2}}^n)}{\Delta w}\\
        +&\frac{1}{\varepsilon}\left\{-\hat{A}\hat{\mathbf{P}}^{n+1}_{j}+\left(I(w_{j},t^n)+w_{j}\sigma(\bar{N}(t^n))\right)\hat{B}\hat{\mathbf{P}}^{n+1}_{j}-a\hat{C}\hat{\mathbf{P}}^{n+1}_{j}\right\}=0,
    \end{aligned}
\end{equation}
where 
\begin{equation}
    \begin{aligned}
    &\hat{\mathbf{P}}^n_j=\left(\hat{u}_1(w_j,t^n),\hat{u}_2(w_j,t^n),...,\hat{u}_{2N+3}(w_j,t^n)\right)^T,\\
        &\mathbf{\Phi}_{j+\frac{1}{2}}^n=(\Phi_{1, j+\frac{1}{2}}^{n},\Phi_{2, j+\frac{1}{2}}^{n},...,\Phi_{2N+3, j+\frac{1}{2}}^{n})^T,
    \end{aligned}
\end{equation}
 and the matrix $\hat{S},\hat{A},\hat{B},\hat{C}$ are defined in \eqref{eq:Matrix2}.

 
This numerical scheme is conserved naturally in the $w$ direction, however, strict conservation of mass in the $v$ direction is not achieved when the test function space is selected based on Section \ref{sec:stability}. When $\varepsilon$ is small enough, the asymptotic preserving properties of the model can only be verified through the use of MPGM.

 
\section{Numerical test} \label{sec:numerical_test}

In this section, we give  numerical tests to verify the properties of the proposed schemes and demonstrate some explorations of the model. Numerical solutions for the initial three subsections are obtained by LGM; results for the MPGM approach are similar except for Section \ref{sec:Convergence}, which are thus omitted, and numerical solutions for Section \ref{sec:Learning_testing} are obtained by MPGM, as variations in the time scale require the scheme to be asymptotic preserving.

The tests are structured as follows. In Section \ref{sec:Convergence}, the convergence order of the method is tested in both the NNLIF model and the NNLIF model with learning rules. In Section \ref{sec:time_saving}, we validate the efficiency of the spectral method by comparing it to existing methods. In Section \ref{sec:blow_up}, we test a few properties of the NNLIF model. In Section \ref{sec:Learning_testing}, we test the learning and discrimination abilities of NNLIF model with learning rules for the periodic input function.


\subsection{Order of accuracy}\label{sec:Convergence}
In this part, we test the order of accuracy of the proposed scheme based on the NNLIF model and the NNLIF model with learning rules. Since the exact solution is unavailable, we choose the numerical solution $p_e$ of the finite difference method \cite{hu2021structure} with sufficient accuracy to replace the exact solution.



For NNLIF model \eqref{eq:problem2}, we choose $V_F=2,V_R=1,V_{\text{min}}=-4,a=1, b=3$ and the Gaussian distribution
\begin{equation}
    p_G(v)=\frac{1}{\sqrt{2 \pi} \sigma_{0} M_0} e^{-\frac{\left(v-v_{0}\right)^{2}}{2 \sigma_{0}^{2}}},
\end{equation}
as the initial condition with $v_0=-1$ and $\sigma_0^2=0.5$, $M_0$ is a normalization factor such that
\begin{equation}
    \int_{V_{\text{min}}}^{V_F} p_G(v) dv=1.
\end{equation}
 The numerical solution is computed till time $t=0.2$. Errors in both $L^{\infty}$ and $L^2$ norm are examined with fixed $N=12$ and different $\Delta t$ in Table \ref{convergence1}. It should be noted that the number of basis functions is not $N$, but rather $2N+3$, as shown in equation \eqref{eq:approximate_solution3}. 
\begin{table}[!htb]
	\centering
	\begin{tabularx}{10cm}{ccccc}
	\toprule
	$\Delta t$ & $\left\| p_N-p_e \right\|_{L^{\infty}}$& {$O_{\tau,L^{\infty}}$}& $\left\| p_N-p_e \right\|_{L^2}$& {$O_{\tau,L^2}$}\\ 
	\midrule
	0.04 & 3.880E-03 &0.9520 & 1.868E-03 &0.9508 \\
	0.02 & 2.005E-03 & 0.9792 & 9.662E-04 &0.9617 \\
	0.01 & 1.017E-03 & 0.9926 &4.961E-06&0.9287 \\
    0.005 & 5.111E-04 & - &2.606E-04& - \\
	\bottomrule
    \end{tabularx}
    \caption{Error and order of accuracy of the proposed numerical scheme for NNLIF model with different temporal sizes. The parameter $N$ is fixed as $N=12$.}
    \label{convergence1}
\end{table}

For the order of accuracy in the $v$ direction, we choose the time step size $\Delta t=10^{-5}$. Errors in the $L^2$ norm are examined with different $N$. The logarithm of the error versus $N$ is plotted in Figure \ref{convergence2}. We remark that when testing the order of spatial convergence, the results present a zig-zag decreasing profile as $N$ increases, which is a common phenomenon for spectral methods. We thus plot the errors for odd and even numbers of $N$, respectively. For each scenario, we clearly observe the spectral convergence as the number of spatial basis functions increases.


\begin{figure}[!htb]
    \centering
        \begin{minipage}[c]{0.49\textwidth}
            \centering
            \includegraphics[width=7cm]{O_N1.eps}
        \end{minipage}
        \begin{minipage}[c]{0.49\textwidth}
            \centering
            \includegraphics[width=7cm]{O_N2.eps}
        \end{minipage}
		\caption{Logarithm of the error of the proposed numerical scheme for NNLIF model with learning rules with different $N$. The temporal size is fixed as $\Delta t=10^{-5}$. Left: $N$ is odd; Right: $N$ is even.}
  \label{convergence2}
\end{figure}


For NNLIF with learning rules model \eqref{eq:problem4}, we choose $V_F=2,V_R=1,V_{\text{min}}=-4,a=1,\varepsilon=0.5,W_{\text{min}}=-1.1,W_{\text{max}}=0.1,\sigma(\bar{N})=\bar{N}, I(w)=0$ and initial condition
\begin{equation}
    p_{\text{init}}=\begin{cases}
        \frac{1}{\sqrt{2 \pi} \sigma_{0} } e^{-\frac{\left(v-v_{0}\right)^{2}}{2 \sigma_{0}^{2}}}\text{sin}^2(\pi w) \qquad &-1<w<0,\\
        0 &\text{otherwise},
    \end{cases}
\end{equation}
with $v_0=-1$ and $\sigma_0^2=0.5$.

The numerical solution is computed till time $t=0.1$. For $t$ direction and $w$ direction, we fix $\frac{\Delta w}{\Delta t}=1, N=16$. Considering that both the $t$ direction and the $w$ direction are theoretically first-order accurate, as well as the stability factor, it is reasonable to jointly test the order of accuracy. Errors in both $L^{1}$ and $L^2$ norm are examined with different $\Delta t$ and $\Delta w$ in Table \ref{convergence3}. For $v$ direction, we fix ${\Delta w}={\Delta t}=10^{-5}$. Errors in the $L^2$ norm are examined with different $N$. The logarithm of the error versus $N$ is plotted in Figure \ref{convergence4}.

\begin{table}[!htb]
	\centering
	\begin{tabularx}{10cm}{cccccc}
	\toprule
	$\Delta t$ &$\Delta w$ & $\left\| p_N-p_e \right\|_{L^{1}}$& $O_{\tau,L^{1}}$& $\left\| p_N-p_e \right\|_{L^2}$& {$O_{\tau,L^2}$}\\ 
	\midrule
	0.02 &0.02 & 1.599E-03 &1.04 & 3.234E-03 &1.07 \\
	0.01 &0.01 & 7.755E-04 & 0.99 & 1.536E-03 &0.97 \\
	0.005 &0.005 &3.893E-04 & 1.01 &7.812E-04&1.05 \\
    0.0025 & 0.0025 & 1.926E-04 & - &3.757E-04& - \\
	\bottomrule
    \end{tabularx}
    \caption{Error and order of accuracy of the proposed numerical scheme for NNLIF model with learning rules with different ${\Delta w}$ and ${\Delta t}$. The parameter $N$ is fixed as $N=16$.}
    \label{convergence3}
\end{table}
\begin{figure}[!htb]
    \centering
        \begin{minipage}[c]{0.49\textwidth}
            \centering
            \includegraphics[width=7cm]{O2_N1.eps}
        \end{minipage}
        \begin{minipage}[c]{0.49\textwidth}
            \centering
            \includegraphics[width=7cm]{O2_N2.eps}
        \end{minipage}
		\caption{Logarithm of the error of the proposed numerical scheme for NNLIF model with learning rules with different $N$. The temporal size is fixed as $\Delta t=10^{-5}$. Left: $N$ is odd; Right: $N$ is even}
  \label{convergence4}
\end{figure}

The results indicate that the scheme shows first-order accuracy in time and exponential convergence in space for the NNLIF model; first-order accuracy in the $w$, $t$ direction and exponential convergence in the v direction for the NNLIF model with learning rules.






\subsection{Simulation time comparison}\label{sec:time_saving}
In this part, we compare the CPU time between the proposed spectral method and the finite difference method \cite{hu2021structure}, to show that our scheme has a significant computational time advantage with the same level of accuracy.


We choose NNLIF model with parameters $a=1,b=0.5,\Delta t=5\times 10^{-4}$ and the Gaussian initial condition with $v_0=0,\sigma_0^2=0.25$. The numerical solution is computed till time $t=0.5$. The results of the spectral method and the finite difference method are shown in Table \ref{error1} and Table \ref{error2}.
\begin{table}[!htb]
	\centering
	\begin{tabularx}{8cm}{cccc}
	\toprule
	$N$& $\left\| \cdot \right\|_{\infty}$&$\left\| \cdot \right\|_{1}$& CPU Time (s) \\ 
	\midrule
	5 & 5.15e-02 & 1.58e-02& 0.026 \\
    10 & 3.33e-03 & 2.85e-04 & 0.030  \\
    15 & 9.71e-05 & 2.51e-05& 0.053 \\
    20 & 1.30e-06 & 3.76e-07 & 0.071  \\
	\bottomrule
    \end{tabularx}
    \caption{Errors using the spectral method with different numbers of basis functions.}
    \label{error1}
\end{table}

\begin{table}[!htb]
	\centering
	\begin{tabularx}{8cm}{cccc}
	\toprule
	$h$& $\left\| \cdot \right\|_{\infty}$&$\left\| \cdot \right\|_{1}$& CPU Time (s) \\ 
	\midrule
    ${1/4}$ & 3.01e-03 & 7.09e-04& 0.031 \\
    ${1/8}$ & 9.69e-04 & 2.18e-04& 0.073 \\
    ${1/16}$ & 2.79e-04 & 6.18e-05 & 0.157  \\
    ${1/32}$ & 7.54e-05 & 1.64e-05& 0.348 \\
    ${1/64}$ & 1.97e-05 & 4.21e-06 & 2.801  \\
    ${1/128}$ & 4.41e-06 & 1.12e-06 & 11.971  \\
	\bottomrule
    \end{tabularx}
    \caption{Errors using the finite difference method  with different spatial grid sizes}
    \label{error2}
\end{table}
These tables clearly indicate that to achieve the same level of accuracy, the spectral method is more efficient in terms of the simulation time, and the advantage is more noticeable when the accuracy level is higher.



\subsection{Global solution and blow-up in NNLIF model}\label{sec:blow_up}
\subsubsection{Blow up}


In \cite{caceres2011analysis}, the authors find the solution may blow up in finite time with the suitable initial conditions for the excitatory network. They show that whenever the value of $b>0$ is, if the initial data is concentrated enough around $v=V_F$, then the defined weak solution in Definition 2.1 of \cite{caceres2011analysis} does not exist for all times. Figure \ref{fig:blowup1} and Figure \ref{fig:blowup2} show this phenomenon. It can be seen that when the blow-up phenomenon is about to occur, the density function $p(v,t)$ is increasingly concentrated and sharp at reset point $V_R$ and the firing rate $N(t)$ is growing rapidly. 
\begin{figure}[!htb]
    \centering
        \begin{minipage}[c]{0.49\textwidth}
            \centering
            \includegraphics[width=7cm]{blowupNt1.eps}
        \end{minipage}
        \begin{minipage}[c]{0.49\textwidth}
            \centering
            \includegraphics[width=7cm]{blowup1.eps}
        \end{minipage}
        \caption{Equation parameters $a=1, b=3$ with Gaussian initial condition $v_0=-1, \sigma_0^2=0.5$. Left: evolution of firing rate $N(t)$. Right: density function $p(v, t)$ at $t=2.95,3.15,3.35$.}
        \label{fig:blowup1}
\end{figure}
\begin{figure}[!htb]
        \begin{minipage}[c]{0.49\textwidth}
            \centering
            \includegraphics[width=7cm]{blowupNt2.eps}
        \end{minipage}
        \begin{minipage}[c]{0.49\textwidth}
            \centering
            \includegraphics[width=7cm]{blowup2.eps}
        \end{minipage}
        \caption{Equation parameters $a=1, b=1.5$ with Gaussian initial condition $v_0=1.5, \sigma_0^2=0.005$.Left: evolution of firing rate $N(t)$. Right: density function $p(v, t)$ at $t=0.0325,0.0365,0.0405$.}
        \label{fig:blowup2}
\end{figure}


For spectral methods, the approximate solution of the density function is dependent on the coefficients of the basis functions. We aim to further investigate how the coefficients change when the blow-up phenomenon is about to occur. We choose $a=1, b=1.5$ in equation and $N=20,\Delta t=10^{-5}$.
\begin{figure}[!htb]
    \centering
        \begin{minipage}[c]{0.49\textwidth}
            \centering
            \includegraphics[width=7cm]{coeff.eps}
        \end{minipage}
        \begin{minipage}[c]{0.49\textwidth}
            \centering
            \includegraphics[width=7cm]{coeff2.eps}
        \end{minipage}
    \caption{Changes of the coefficients of the first few terms in the expansion formula \eqref{eq:approximate_solution3} during blow up. Left: evolution of the coefficients $\{p_k,f_k\}_{k=0}^2$. Right:evolution of the coefficients $\{\lambda_k\}_{k=1}^3$.  }
    \label{fig:coeff}
\end{figure}


Recall that
\begin{equation}
    \lambda_3(t)=\partial_vp(V_F,t)=-\frac{N(t)}{a},\qquad \partial_vp(V_R^+,t)=\lambda_2(t)+\lambda_3(t).
\end{equation}
Therefore, $\lambda_2$ and $\lambda_3$ are directly influenced by the firing rate. Due to the use of global basis functions, as the firing rate $N(t)$ increases, all the basis functions are affected. In response to the change of $\lambda_3$, $\lambda_2$ and the coefficients of the basis functions in $\mathrm{W}_2$ change accordingly, respectively controlling the derivative value on both sides of point $V_R$ and the function value in the interval. Figure \ref{fig:coeff} show the change of coefficients $\{p_k,f_k\}_{k=0}^2$, $\{\lambda_k\}_{k=1}^3$ in \eqref{eq:approximate_solution3} as time involves. It can be seen from the figure that the changes in $\lambda_2$ and $\lambda_3$ are most obvious, while the coefficients of all basis functions in $\mathrm{W}_2$ space are affected but the changes are relatively small.



\subsubsection{Relative entropy}

As we have mentioned, since little is known about the properties of the solutions of the Fokker-Planck equation \eqref{eq:problem1}, there  is a lack of complete understanding of the long-time asymptotic behavior in the continuous case. In \cite{caceres2011analysis}, they studied relative entropy theory for linear problem $a_1=b=0$, which implies exponential convergence to equilibrium. The relative entropy is given by
\begin{equation}
    I_e=\int_{-\infty}^{V_{F}} G\left(\frac{p(v, t)}{p^{\infty}(v)}\right) p_{\infty}(v) d v,
\end{equation}
which can be shown to be decreasing in time, where $G(\cdot)$ is a smooth convex function and $p^{\infty}(v)$ represents the stationary solution. In this part, we numerically verify the relative entropy theory. The numerical relative entropy is given by
\begin{equation}
    S(t)=\int_{V_L}^{V_{F}} G\left(\frac{p_N(v, t)}{p^{\infty}(v)}\right) p_{\infty}(v) d v.
\end{equation}

We consider nonlinear cases with $a_0=1,a_1=0,b=-0.5$ and $a_0=1,a_1=0.1,b=0$. We choose the numerical solution of a sufficiently long time as the stationary solution $p^{\infty}(v)$ and the Gaussian initial condition $v_0=-1, \sigma_0^2=0.5$. Figure \ref{fig:relative_entropy2} \ref{fig:relative_entropy3} show the time evolution of the firing rate and the numerical relative entropy for these cases.
\begin{figure}[!htb]
    \centering
    \begin{minipage}[c]{0.49\textwidth}
        \centering
        \includegraphics[width=1\textwidth]{Nt2.eps}
    \end{minipage}
    \begin{minipage}[c]{0.49\textwidth}
        \centering
        \includegraphics[width=1\textwidth]{relative2.eps}
    \end{minipage}
    \caption{Equation parameters $a=1,b=-0.5$ with Gaussian initial condition $v_0=-1, \sigma_0^2=0.5$. Left: evolution of firing rate $N(t)$. Right: evolution of relative entropy $S(t)$ with $G(x)=\frac{(x-1)^2}{2}$.}
     \label{fig:relative_entropy2}
\end{figure}
\begin{figure}[!htb]
    \centering
    \begin{minipage}[c]{0.49\textwidth}
        \centering
        \includegraphics[width=1\textwidth]{Nt3.eps}
    \end{minipage}
    \begin{minipage}[c]{0.49\textwidth}
        \centering
        \includegraphics[width=1\textwidth]{relative3.eps}
    \end{minipage}
    \caption{Equation parameters $a_0=1,a_1=0.1,b=0$ with Gaussian initial condition $v_0=-1, \sigma_0^2=0.5$. Left: evolution of firing rate $N(t)$. Right: evolution of relative entropy $S(t)$ with $G(x)=\frac{(x-1)^2}{2}$.}
     \label{fig:relative_entropy3}
\end{figure}


As shown in \cite{caceres2011analysis}, there may be two stationary solutions for the system of $b>0$. For example, when $ a(N(t)) = 1$ and $b =1.5$, there are two different steady states whose firing rates are $N^{\infty}=2.319$ and $N^{\infty}=0.1924$. Given the firing rate $N^{\infty}$, the expression of $p^{\infty}(v)$is given by
\begin{equation}
    p^{\infty}(v)=\frac{N^{\infty}}{a\left(N^{\infty}\right)} e^{-\frac{h\left(v, N^{\infty}\right)^{2}}{2 a\left(N^{\infty}\right)}} \int_{\max \left\{v, V_{R}\right\}}^{V_{F}} e^{\frac{h\left(\omega, N^{\infty}\right)^{2}}{2 a\left(N^{\infty}\right)}} d \omega ,
\end{equation}
which is the stationary solution when we calculate the relative entropy for multiple steady-state problems. The results are shown in Figure \ref{fig:relative_entropy4}, where the steady state with a larger firing rate $N^{\infty}=2.319$ is unstable while the stationary solution with a lower firing rate $N^{\infty}=0.1915$ is stable. We see that the relative entropy decreases with time for the stable state, while the other one does not.
\begin{figure}[!htb]
    \centering
    \begin{minipage}[c]{0.49\textwidth}
        \centering
        \includegraphics[width=1\textwidth]{st1.eps}
    \end{minipage}
    \begin{minipage}[c]{0.49\textwidth}
        \centering
        \includegraphics[width=1\textwidth]{st2.eps}
    \end{minipage}
    \caption{Equation parameters $a=1,b=1.5$  with Gaussian initial condition $v_0=-1, \sigma_0^2=0.5$. In this case, the model has two stationary states with firing rates $N^{\infty}=0.1924$ and $N^{\infty}=2.319$. Left: evolution of relative entropy $S(t)$ with $G(x)=\frac{(x-1)^2}{2}$ for stable state with $N^{\infty}=0.1924$. Right: evolution of relative entropy $S(t)$ with $G(x)=\frac{(x-1)^2}{2}$ for unstable state with $N^{\infty}=2.319$ }
     \label{fig:relative_entropy4}
\end{figure}



\subsection{Learning and testing in NNLIF model with learning rules}\label{sec:Learning_testing}
In this part, we consider the learning and discrimination abilities in NNLIF model with learning rules. In \cite{perthame2017distributed}, the authors proposed a two-phase test to illustrate the discrimination property:\medskip


\textbf{Learning phase}
\smallskip

1. An heterogeneous input $I(w)$ is presented to the system, when the learning process is active. The initial data is supported on inhibitory weights and the learning rule is determined for the present weights by $-N(w)\bar{N}$ by taking $K(w) = -1$ if $w \leq 0$.


2. After some time, the synaptic weight distribution $H(w, t)$ converges to an equilibrium distribution $H^*_
I(w)$, which depends on $I$.\medskip



\textbf{Testing phase}
\smallskip

1. The learning process is now switched off, i.e. there is no w-direction convection, and a new input $J(w)$ is presented to the system.


2. After some time, the solution $p_J (v,w, t)$ reaches an equilibrium $p^*_J (v,w)$, which is characterized  by the output signal $N^*_J(w)$ which is the neural activity distribution across the heterogeneous populations.\medskip


Some numerical explorations of the learning behavior and discriminative properties of the model have been done in \cite{perthame2017distributed}\cite{he2022structure}. When the learning phase is over, in addition to the synaptic weight distribution $H(w, t)$, the equilibrium state $N_I(w)$ of the sub-network activity $N(w,t)$ can also be obtained, which we call the \textbf{prediction signal}. In the previous work on the time-independent input function $I(w)$ for the learning phase \cite{perthame2017distributed}\cite{he2022structure}, the prediction signal $N_I(w)$ is like a triangle depending on the input function $I(w)$ of the learning phase.  After the testing phase when the learning input $I(w)$ and testing input $J(w)$ are the same, the output signal $N_J^*(w)$ is like a triangle that matches the prediction signal $N_I(w)$; but when $I(w)$ and $J(w)$ are different, the output signal is not in a regular shape. 


They explore learning and discriminative power in the model only if the input function is constant in time. In our work, we plan to explore how the model would react to a time-varying input signal through numerical experiments, and there have been studies in the field of neuroscience surrounding time-varying input \cite{isidori1990output}. Especially, we consider input functions that are time-periodic and explore the effect of oscillation periods on the learning ability of the model. To this aim, we have designed $4$ sets of experiments, progressively revealing the nature of its learning behavior.

\paragraph{Test 1. Synchronizing with oscillating inputs.}
We choose the testing input functions
\begin{equation}
    \begin{aligned}
        I_{1}&=\pi^{-\frac{1}{4}} e^{-\frac{1}{2}(10 w+5)^{2}}+2 \\
        I_{2}&=\pi^{-\frac{1}{4}} \sqrt{2}(10 w+5) e^{-\frac{1}{2}(10 w+5)^{2}}+2,
    \end{aligned}
\end{equation}
and the learning input function is periodically switching between those two
\begin{equation}
    \label{input}
    I(w,t)=a(t)I_1(w)+b(t)I_2(w),
\end{equation}
where
\begin{equation}
    \label{eq:input_coff}
    \begin{aligned}
        a(t)&=\frac{1+\cos(\frac{2\pi t}{D})}{2},\\
        b(t)&=1-a(t).
    \end{aligned}
\end{equation}

For other parameters, we choose $V_F=2,V_R=1,V_{\text{min}}=1,a=1,\varepsilon=0.1,W_{\text{min}}=-1.1,W_{\text{max}}=0.1,T_{\text{max}}=4,\sigma(\bar{N})=\bar{N},\Delta t=2.5\times 10^{-4},\Delta w=0.01$ and the initial condition 
\begin{equation}
    p_{\text{init}}=\begin{cases}
        \text{sin}^2(\pi v)\text{sin}^2(\pi w) \qquad &-1<w<0\text{ and }-1<v<1,\\
        0 &\text{otherwise}.
    \end{cases}
\end{equation}


In the learning phase, the input function changes periodically in time; the smaller the period is, the greater the rate of change of the input function is. The total network activity $\bar{N}$ is an intuitive response to the input function, so we first observe the change in the total network activity. First, we choose period $D=1,0.5,0.2$. 
\begin{figure}[!htb]
    \centering
        \begin{minipage}[c]{0.3\textwidth}
            \centering
            \includegraphics[width=1\textwidth]{Nbar1.eps}
        \end{minipage}
        \begin{minipage}[c]{0.3\textwidth}
            \centering
            \includegraphics[width=1\textwidth]{Nbar2.eps}
        \end{minipage}
        \begin{minipage}[c]{0.3\textwidth}
            \centering
            \includegraphics[width=1\textwidth]{Nbar3.eps}
        \end{minipage}
    \caption{Equation parameters $a=1$ and $\varepsilon=0.1$. The evolution of total firing rate $\bar{N}$. Left: the input function period $D=1$. Middle: the input function period $D=0.5$. Right: the input function period $D=0.2$.}
    \label{fig:Nbar}
\end{figure}

Figure \ref{fig:Nbar} shows the evolution of the total firing rate at different periods. As we expected, except for the initial transient evolutionary phase, the total activity of the network changes periodically over time and its period is the same as the input function.



\paragraph{Test 2. Adapting to fast oscillating inputs.}
Since the prediction signal is determined by the learning input function and reflects the model's learning of the learning input function $I(w,t)$, observing the prediction signal in different periods helps us explore the learning behavior of the model. We compare numerical results for different periods $D=4,0.4,0.2,0.1,0.01$.  In this case, the last input function learned by the model is $I(w,t_\text{max})=I_1$.



\begin{figure}[!htb]
    \centering
        \begin{minipage}[c]{0.3\textwidth}
            \centering
            \includegraphics[width=1\textwidth]{D=4.eps}
        \end{minipage}
        \begin{minipage}[c]{0.3\textwidth}
            \centering
            \includegraphics[width=1\textwidth]{D=0.4.eps}
        \end{minipage}
        \begin{minipage}[c]{0.3\textwidth}
            \centering
            \includegraphics[width=1\textwidth]{D=0.2.eps}
        \end{minipage}
        \begin{minipage}[c]{0.3\textwidth}
            \centering
            \includegraphics[width=1\textwidth]{D=0.1.eps}
        \end{minipage}
        \begin{minipage}[c]{0.3\textwidth}
            \centering
            \includegraphics[width=1\textwidth]{D=0.05.eps}
        \end{minipage}
        \begin{minipage}[c]{0.3\textwidth}
            \centering
            \includegraphics[width=1\textwidth]{D=0.01.eps}
        \end{minipage}
    \caption{Equation parameters $a=1$ and $\varepsilon=0.1$. The prediction signal at $t=4$ with different input function periods. Top: the input function period $D=4,0.4,0.2$ from left to right. Bottom: the input function period $0.1,0.05,0.01$ from left to right.}
    \label{fig:prediction}
\end{figure}

Figure \ref{fig:prediction} shows the prediction signal at different periods. When the period is large, the prediction signal is like a triangle. As the period gets smaller, the shape of the prediction signal is getting more and more irregular. However, as the period is getting further smaller, the shape of the prediction signal is becoming triangular again. In previous experiments \cite{perthame2017distributed}\cite{he2022structure}, for the time-independent learning input signal $I(w)$, the test signal always resembles a triangle. So we speculate from Figure \ref{fig:prediction} that for sufficiently large or sufficiently small periods, the predicted signal looks like a triangle, and the model has effectively learned a signal of a certain form.




\paragraph{Test 3. Learning from oscillating inputs.}

In order to verify the above conjecture, we choose a relatively large period with $D=4$ and a small period with $D=0.01$ in \eqref{eq:input_coff}.  In the testing phase, we choose testing input functions $J=I_1$, $J=I_2$, and $J=\frac{I_1+I_2}{2}$.
\begin{figure}[!htb]
    \centering
        \begin{minipage}[c]{0.3\textwidth}
            \centering
            \includegraphics[width=1\textwidth]{D=4_I1test.eps}
        \end{minipage}
        \begin{minipage}[c]{0.3\textwidth}
            \centering
            \includegraphics[width=1\textwidth]{D=4_I2test.eps}
        \end{minipage}
        \begin{minipage}[c]{0.3\textwidth}
            \centering
            \includegraphics[width=1\textwidth]{D=4_I12test.eps}
        \end{minipage}
    \caption{(Output signal for the large period learning input) The
final firing rate $N(w)$ with different testing input $J(w)$. Equation parameters $a=1$ and $\varepsilon=0.1$, and the period of the input function in the learning phase is $D=4$. Left: Output signal with testing input function $J=I_1$. Middle: Output signal with testing input function $J=I_2$. Right: Output signal with testing input function $J=\frac{I_1+I_2}{2}$.}
    \label{fig:largeD}
\end{figure}
\begin{figure}[!htb]
    \centering
        \begin{minipage}[c]{0.3\textwidth}
            \centering
            \includegraphics[width=1\textwidth]{D=0.01_I1test.eps}
        \end{minipage}
        \begin{minipage}[c]{0.3\textwidth}
            \centering
            \includegraphics[width=1\textwidth]{D=0.01_I2test.eps}
        \end{minipage}
        \begin{minipage}[c]{0.3\textwidth}
            \centering
            \includegraphics[width=1\textwidth]{D=0.01_I12test.eps}
        \end{minipage}
    \caption{Output signal for the small period learning input) The
final firing rate $N(w)$ with different testing input $J(w)$. Equation parameters $a=1$ and $\varepsilon=0.1$, and the period of the input function in the learning phase is $D=0.01$. Left: Output signal with testing input function $J=I_1$. Middle: Output signal with testing input function $J=I_2$. Right: Output signal with testing input function $J=\frac{I_1+I_2}{2}$.}
    \label{fig:shortD}
\end{figure}

Figure \ref{fig:largeD} shows the output signal of period $D=4$ with testing input function $J=I_1$, $J=I_2$ and $J=\frac{I_1+I_2}{2}$. When $J=I_1=I(w,t_\text{max})$, the output signal $N_J^*(w)$ is like a triangle. Figure \ref{fig:shortD} shows the output signal of period $D=0.01$ with testing input function $J=I_1$, $J=I_2$ and $J=\frac{I_1+I_2}{2}$. When $J=\frac{I_1+I_2}{2}$, the output signal $N_J^*(w)$ is like a triangle. Numerical results show that when the period is relatively large, the signal learned by the model matches $I_1$, and when the period is relatively small, it matches $\frac{I_1+I_2}{2}$. 

The experimental results can be interpreted as follows. When the period is large, the model has enough time to learn, so the learned signal is the input function at the last moment. And when the period is small, neither $I_1$ nor $I_2$ can be learned well, but the result of learning is the average of the two.  Because when the switching process is too fast, the effect of the model on the learning of either $I_1$ or $I_2$ is poor. Instead, the average signal $\frac{I_1+I_2}{2}$ is captured by the time averaging of the learning process. 


 \paragraph{Test 4. Phase diagram for leaning.} There are multiple typical time scales in this model: the time scale for the voltage activities, the time scale for learning by redistributing the synaptic weights and the time period in the external input. When introducing the model, we perform a time rescaling for \eqref{eq:problem4}, where the parameter $\varepsilon$ reflects the ratio between the time scales of voltage activities and learning. In the next numerical experiment, we choose $\varepsilon=1,0.5,0.25,0.125$ and periods $D=2^2,2^1,\dots,2^{-7}$ to compare the results of the output signal under different parameters. After the testing phase, we choose the total activity $\bar{N}(t)$ to quantify the output signal:
\begin{equation}
    \label{judge_tool}
    E^{\varepsilon,D}_J=\left| \bar{N}^{\varepsilon,D}_J -\bar{N}^{\varepsilon}_J \right|.
\end{equation}
Here, $\bar{N}^{\varepsilon,D}_J$ denotes the total activity when the equation parameter is $\varepsilon$, the learning input function is given by \eqref{input} with period $D$, while the testing input function is $J$. $\bar{N}^{\varepsilon}_J$ represents the total activity where the equation parameter is $\varepsilon$ and both the learning input function and the testing input function are $J$. $E^{\varepsilon,D}_J$ can roughly measure the output signal. The closer the value of $E^{\varepsilon,D}_J$ is to zero, the superior the model's learning efficacy.

\begin{figure}[!htb]
    \centering
        \begin{minipage}[c]{0.49\textwidth}
            \centering
            \includegraphics[width=1\textwidth]{eps_test1.eps}
        \end{minipage}
        \begin{minipage}[c]{0.49\textwidth}
            \centering
            \includegraphics[width=1\textwidth]{eps_test2.eps}
        \end{minipage}
    \caption{Equation parameter $a=1$. Left: The value of $E^{\varepsilon,D}_J$ under different $D$ and $\varepsilon$ with input function $J=I_1$. Right: The value of $E^{\varepsilon,D}_J$ under different $D$ and $\varepsilon$ with input function $J=\frac{I_1+I_2}{2}$.}
    \label{fig:output3}
\end{figure}

 As shown in Figure \ref{fig:output3}, as the period becomes smaller, the testing indicator becomes less significant with respect to the testing input function $J=I_1$, and the test indicator becomes more significant with respect to the testing input function $J=\frac{I_1+I_2}{2}$. Besides, the numerical results also suggest that when epsilon is small, the transition in learning takes place at a smaller time period, whereas such a trend is not prominent. Although the experiments are not fully conclusive yet, they show a lot of promise for using the proposed numerical method to simulate large-scale tests.


\begin{comment}
\begin{figure}
\includegraphics[width=\linewidth]{figs/beyond_tss_lesion.pdf}
\caption[]{End-to-End runtime lesion study of the entire MNIST dataset and the FMA featurized music dataset. Each of DROP's contributions provides a runtime improvement.}
\label{fig:beyond_lesion}
\end{figure}
\end{comment}



\section{Conclusion}
\label{sec:conclusion}

Advanced data analytics techniques must scale to rising data volumes. 
DR techniques offer a powerful toolkit when processing these datasets, with PCA frequently outperforming popular techniques in exchange for high computational cost. 
In response, we propose DROP, a new dimensionality reduction optimizer. 
DROP combines progressive sampling, progress estimation, and online aggregation to identify high quality low dimensional bases via PCA without processing the entire dataset by balancing the runtime of downstream tasks and achieved dimensionality. 
Thus, DROP provides a first step in bridging the gap between quality and efficiency in end-to-end DR for downstream \red{analytics}. 

%We revisit canonical operators for time series dimensionality reduction and the measurement study of~\cite{keogh-study}, and show that PCA is more effective than popular alternatives in the data mining literature often by a margin of over $2\times$ on average on gold-standard time series benchmark data sets with respect to output data dimension. More surprisingly, we empirically demonstrate that a small number of samples are sufficient to accurately characterize directions of maximum variance and obtain a high-quality low-dimensional transformation.



\onecolumn


% \tableofcontents{}

% \newpage

\section*{Supplementary Material}
\addcontentsline{toc}{section}{Supplementary Material}


Throughout this discussion, 
we will make frequently use 
of the following standard results
concerning the exponential concentration 
of random variables:

\begin{lemma}[Hoeffding's inequality for independent RVs~\citep{hoeffding1994probability}] Let $Z_1, Z_2, \ldots, Z_n$ be independent bounded random variables with $Z_i \in [a,b]$ for all $i$, then 
    \begin{align*}
        \prob\left( \frac{1}{n} \sum_{i=1}^n (Z_i - \Expo{Z_i}) \ge t \right) \le \exp{\left( -\frac{2nt^2}{(b-a)^2} \right) }
    \end{align*} 
    and 
    \begin{align*}
        \prob\left( \frac{1}{n} \sum_{i=1}^n (Z_i - \Expo{Z_i}) \le -t \right) \le \exp{\left( -\frac{2nt^2}{(b-a)^2} \right) }
    \end{align*} 
    for all $t \ge 0$. 
\end{lemma}

\begin{lemma}[Hoeffding's inequality for sampling with replacement~\citep{hoeffding1994probability}] \label{lem:hoeffding_sampling} Let $\calZ = (Z_1, Z_2, \ldots, Z_N)$ be a finite population of $N$ points with $Z_i \in [a.b]$ for all $i$. Let $X_1, X_2, \ldots X_n$ be a random sample drawn without replacement from $\calZ$. Then for all $t \ge 0$, we have 
    \begin{align*}
        \prob\left( \frac{1}{n} \sum_{i=1}^n (X_i - \mu ) \ge t \right) \le \exp{\left( -\frac{2nt^2}{(b-a)^2} \right) }
    \end{align*} 
    and 
    \begin{align*}
        \prob\left( \frac{1}{n} \sum_{i=1}^n (X_i - \mu ) \le -t \right) \le \exp{\left( -\frac{2nt^2}{(b-a)^2} \right) } \,,
    \end{align*} 
    where $\mu = \frac{1}{N} \sum_{i=1}^{N} Z_i$. 
\end{lemma}

We now discuss one condition that generalizes the exponential concentration to dependent random variables.
\begin{condition}[Bounded difference inequality] \label{cond:BDC} Let $\calZ$ be some set and $\phi: \calZ^n \to \Real$. We say that $\phi$ satisfies the bounded difference assumption if 
there exists $c_1, c_2, \ldots c_n \ge 0$ s.t. for all $i$, we have 
\begin{align*}
    \sup_{Z_1,Z_2, \ldots,Z_n, Z_i^\prime \in \calZ^{n+1} } \abs{\phi (Z_1, \ldots, Z_i, \ldots, Z_n ) - \phi (Z_1, \ldots, Z_i^\prime, \ldots, Z_n ) } \le c_i \,.
\end{align*} 
\end{condition}

\begin{lemma}[McDiarmid’s inequality~\citep{mcdiarmid1989}] \label{lem:McDiarmid} Let $Z_1, Z_2, \ldots, Z_n$ be independent random variables on set $\calZ$ and $\phi : \calZ^n \to \Real$ satisfy bounded difference inequality (\codref{cond:BDC}). Then for all $t>0$, we have 
    \begin{align*}
        \prob\left( \phi(Z_1, Z_2, \ldots, Z_n) - \Expo{\phi(Z_1, Z_2, \ldots, Z_n)} \ge t \right) \le \exp{\left( -\frac{2t^2}{\sum_{i=1}^n c_i^2} \right) } 
    \end{align*} 
    and 
    \begin{align*}
        \prob\left( \phi(Z_1, Z_2, \ldots, Z_n) - \Expo{\phi(Z_1, Z_2, \ldots, Z_n)} \le -t \right) \le \exp{\left( -\frac{2t^2}{\sum_{i=1}^n c_i^2} \right) } \,.
    \end{align*} 
\end{lemma}


\section{Proofs from \secref{sec:ERM_training}}\label{app:proof_erm}

\textbf{Additional notation {} {}} Let $m_1$ be the number of mislabeled points ($\wt S_M$) and $m_2$ be the number of correctly labeled points ($\wt S_C$). Note $m_1 + m_2 = m$. 


\subsection{Proof of \thmref{thm:error_ERM}}


\begin{proof}[Proof of \lemref{lem:fit_mislabeled}] 
    The main idea of our proof is to regard 
    the clean portion of the data 
    ($S \cup \wt S_C$) as fixed.   
    Then, there exists an (unknown) classifier $f^*$ 
    that minimizes the expected risk
    calculated on the (fixed) clean data
    and (random draws of) the mislabeled data $\wt S_M$. 
    % 
    % 
    Formally, 
    \begin{align}
    f^* \defeq \argmin_{f \in \calF} \error_{\widecheck {\calD}} (f) \,, \label{eq:modified_ERM}
    \end{align}
    where $$\widecheck \calD = \frac{n}{m+n} \calS + \frac{m_2}{m+n} \wt \calS_C  + \frac{m_1}{m+n}\calDm \,.$$ 
    Note here that $\widecheck \calD$ is a combination 
    of the \emph{empirical distribution} 
    over correctly labeled data $S \cup \wt S_C$
    and the (population) distribution 
    over mislabeled data $\calDm$.
    Recall that 
    \begin{align}
    \wh f \defeq \argmin_{f \in \calF} \error_{\calS \cup \wt S} (f) \,. \label{eq:orig_ERM}
    \end{align}
    % 
    % 
    Since, $\widehat f$ minimizes 0-1 error 
    on $S \cup \wt S$, using ERM optimality on \eqref{eq:orig_ERM},  
    we have 
    \begin{align}
        \error_{\calS \cup \wt \calS}(\widehat f) \le \error_{
            \calS \cup \wt \calS}(f^*) \,.    \label{eq:step1}
    \end{align}
    Moreover, since $f^*$ is independent of $\wt S_M$, using Hoeffding's bound,
    % \footnote{For a fully rigorous argument,
    % refer to the complete proof in App.~\ref{app:proof_erm}.} 
    we have with probability at least $1-\delta$ that
    \begin{align}
      \error_{\wt \calS_M}(f^*) \le \error_{ \calDm}(f^*) +  \sqrt{\frac{\log(1/\delta)}{2 m_1}} \,. \label{eq:step2} 
    \end{align}
    %$ 
    %for some constant $c_1\le 1/2$. 
    Finally, since $f^*$ is the optimal classifier on $\widecheck \calD$, 
    we have 
    \begin{align}
        \error_{\widecheck \calD}(f^*) \le \error_{\widecheck \calD}(\widehat f) \,. \label{eq:step3}
    \end{align}
    Now to relate \eqref{eq:step1} and \eqref{eq:step3}, we multiply \eqref{eq:step2} by $\frac{m_1}{m+n}$ and add $\frac{n}{m+n} \error_{\calS} (f)  + \frac{m_2}{m+n} \error_{\wt \calS_C} (f)$ both the sides. Hence, 
    we can rewrite \eqref{eq:step2} as follows: 
    \begin{align}
        \error_{\calS \cup \wt\calS}(f^*) \le \error_{ \widecheck \calD}(f^*) +  \frac{m_1}{m+n}\sqrt{\frac{\log(1/\delta)}{2 m_1}} \,. \label{eq:step4} 
    \end{align}
    Now we combine equations \eqref{eq:step1}, \eqref{eq:step4}, and \eqref{eq:step3}, to get 
    \begin{align}
        \error_{\calS \cup \wt \calS}(\wh f) \le \error_{\widecheck \calD}(\wh f) +  \frac{m_1}{m+n}\sqrt{\frac{\log(1/\delta)}{2 m_1}} \,, 
    \end{align}
    which implies 
    \begin{align}
        \error_{ \wt \calS_M}(\wh f) \le \error_{\calDm}(\wh f) + \sqrt{\frac{\log(1/\delta)}{2 m_1}} \,. \label{eq:lemma1_final}
    \end{align}
    Since $\wt S$ is obtained by randomly labeling an unlabeled dataset, we assume $2m_1 \approx m$ \footnote{Formally, with probability at least $1-\delta$, we have  $(m - 2m_1)\le \sqrt{m\log(1/\delta)/2}$.}. Moreover, using $\error_{\calDm} = 1 - \error_{\calD}$ we obtain the desired result.   
    % Combining the above steps and using the fact 
    % that $\error_\calD = 1- \error_{\calDm} $, 
    % we obtain the desired result.
\end{proof}

\begin{proof}[Proof of \lemref{lem:mislabeled_error}]
    Recall $\error_{\wt S} (f) = \frac{m_1}{m} \error_{\wt S_M}(f) + \frac{m_2}{m} \error_{\wt S_C}(f)$. Hence, we have 
    \begin{align}
        2\error_{\wt S}(f) - \error_{\wt S_M}(f) - \error_{\wt S_C}(f) &= \left(\frac{2m_1}{m} \error_{\wt S_M}(f) - \error_{\wt S_M}(f)\right) + \left(\frac{2m_2}{m} \error_{\wt S_C}(f) - \error_{\wt S_C}(f)\right) \\ &= \left(\frac{2m_1}{m} - 1\right) \error_{\wt S_M}(f) + \left(\frac{2m_2}{m} - 1 \right)\error_{\wt S_C} (f) \,.
    \end{align} 
    Since the dataset is labeled uniformly at random, with probability at least $1-\delta$, we have  $\left(\frac{2m_1}{m} - 1\right) \le \sqrt{\frac{\log(1/\delta)}{2m}}$. Similarly, we have with probability at least $1-\delta$, $\left(\frac{2m_2}{m} - 1\right) \le \sqrt{\frac{\log(1/\delta)}{2m}}$. Using union bound, with probability at least $1-\delta$, we have
    % \begin{align}
    %     2\error_{\wt S} - \error_{\wt S_M}(f) - \error_{\wt S_C}(f) \le \sqrt{\frac{\log(2/\delta)}{2m}} \left(\error_{\wt S_M}(f) + \error_{\wt S_C}(f) \right) \le 2\sqrt{\frac{\log(2/\delta)}{2m}} \,. \label{eq:lemma2_final}
    % \end{align}
    \begin{align}
        2\error_{\wt S} - \error_{\wt S_M}(f) - \error_{\wt S_C}(f) \le \sqrt{\frac{\log(2/\delta)}{2m}} \left(\error_{\wt S_M}(f) + \error_{\wt S_C}(f) \right) \,. \label{eq:lemma2_prefinal}
    \end{align}
    With re-arranging $\error_{\wt S_M}(f) + \error_{\wt S_C}(f)$ and using the inequality $ 1- a\le \frac{1}{1+a} $, we have  
    \begin{align}
        2\error_{\wt S} - \error_{\wt S_M}(f) - \error_{\wt S_C}(f) \le 2\error_{\wt \calS} \sqrt{\frac{\log(2/\delta)}{2m}}  \,. \label{eq:lemma2_final}
    \end{align}

    % We obtain the desired result by using 
\end{proof}

\begin{proof}[Proof of \lemref{lem:clear_error}]
% Recall 0-1 error on each point  $(x,y) \in S \cup \wt S$ is given by $\I{ f(x)\ne y}$.
In the set of correctly labeled points $S \cup \wt S_C$, we have $S$ as a random subset of $S \cup \wt S_C$. Hence, using Hoeffding's inequality for sampling without replacement (\lemref{lem:hoeffding_sampling}), we have with probability at least $1-\delta$
\begin{align}
    \error_{\wt \calS_C} (\wh f)- \error_{\calS \cup \wt \calS_C}( \wh f) \le  \sqrt{\frac{\log(1/\delta)}{2m_2}} \,.
\end{align}
Re-writing $\error_{\calS \cup \wt \calS_C}( \wh f)$ as $\frac{m_2}{m_2 + n} \error_{\wt \calS_C }(\wh f) + \frac{n}{m_2 + n} \error_{\calS }(\wh f)$, we have with probability at least $1-\delta$
\begin{align}
   \left(\frac{n}{n+m_2}\right) \left(\error_{\wt \calS_C} (\wh f)- \error_{\calS}( \wh f) \right) \le  \sqrt{\frac{\log(1/\delta)}{2m_2}} \,.
\end{align}
As before, assuming $2m_2 \approx m$, we have with probability at least $1-\delta$ 
\begin{align}
    \error_{\wt \calS_C} (\wh f)- \error_{\calS}( \wh f) \le \left(1+\frac{m_2}{n}\right)  \sqrt{\frac{\log(1/\delta)}{m}} \le \left(1 + \frac{m}{2n}\right) \sqrt{\frac{\log(1/\delta)}{m}} \,. \label{eq:lemma3_final}
\end{align} 
\end{proof}

\begin{proof}[Proof of \thmref{thm:error_ERM}] 
    Having established these core intermediate results, we can now combine above three lemmas to prove the main result. 
    In particular, we bound the population error on clean data ($\error_\calD(\wh f)$) as follows:  
    \begin{enumerate}[(i)]
        \item First, use \eqref{eq:lemma1_final}, to obtain an upper bound on the population error on clean data, i.e., with probability at least $1-\delta/4$, we have
        \begin{align}
            \error_{ \calD} (\wh f) \le 1 - \error_{ \wt \calS_M}(\wh f) + \sqrt{\frac{\log(4/\delta)}{m}} \,. 
        \end{align}
        \item  Second, use \eqref{eq:lemma2_final}, to relate the error on the mislabeled fraction with error on clean portion of randomly labeled data and error on whole randomly labeled dataset, i.e., with probability at least $1-\delta/2$, we have 
        \begin{align}
            - \error_{\wt S_M}(f) \le \error_{\wt S_C}(f) - 2\error_{\wt S}  + 2\error_{\wt S} \sqrt{\frac{\log(4/\delta)}{2m}}  \,. 
        \end{align} 
        \item Finally, use \eqref{eq:lemma3_final} to relate the error on the clean portion of randomly labeled data and error on clean training data, i.e., with probability $1-\delta/4$, we have 
        \begin{align}
            \error_{\wt \calS_C} (\wh f)\le - \error_{\calS}( \wh f) + \left(1 + \frac{m}{2n} \right) \sqrt{\frac{\log(4/\delta)}{m}} \,. 
        \end{align} 
    \end{enumerate}

    Using union bound on the above three steps, we have with probability at least $1-\delta$: 
    \begin{align}
        \error_\calD (\wh f) \le \error_{\calS}(\wh f)   + 1 - 2\error_{\wt \calS}(\wh f)   + \left(\sqrt{2} \error_{\wt S} + 2 + \frac{m}{2n}\right)  \sqrt{\frac{\log(4/\delta)}{m}} \,.
    \end{align}
    % Note that $(1/\sqrt{2} + 2.5)$ is a loose constant. In experiments, we use the ratio $\frac{m}{n}$
    %  the exact error $\error_{\wt \calS}(\wh f)$ 
    % to evaluate R.H.S.    
\end{proof}

\subsection{Proof of \propref{prop:rademacher}}

\begin{proof}[Proof of \propref{prop:rademacher}]
    For a classifier $ f: \calX \to \{-1, 1\}$, we have $1 - 2\,\indict{ f(x) \ne y} = y \cdot f(x)$. Hence, by definition of $\error$, we have 
    \begin{align}
        1 -2\error_{\wt \calS}(f) = \frac{1}{m}\sum_{i=1}^m y_i \cdot f(x_i) \le \sup_{f \in \calF} \, \frac{1}{m} \sum_{i=1}^m y_i \cdot f(x_i)  \,. \label{eq:error_rademacher}
    \end{align}
    Note that for fixed inputs $(x_1, x_2, \ldots, x_m)$ in $\wt S$, $(y_1, y_2, \ldots y_m)$ are random labels. Define $\phi_1 (y_1, y_2, \ldots, y_m) \defeq \sup_{f \in \calF} \, \frac{1}{m} \sum_{i=1}^m y_i \cdot f(x_i)$. We have the following bounded difference condition on $\phi_1$. For all i, 
    \begin{align}
        \sup_{y_1, \ldots y_m, y_i^\prime \in \{-1, 1\}^{m+1} } \abs{ \phi_1 (y_1,\ldots, y_i, \ldots, y_m) - \phi_1 (y_1,\ldots, y_i^\prime, \ldots, y_m)  } \le 1/m \,. \label{cond1_rademacher}
    \end{align} 
    
    Similarly, we define $\phi_2 (x_1, x_2, \ldots, x_m) \defeq \Expt{ y_i \sim_U \{-1, 1\}  }{ \sup_{f \in \calF} \, \frac{1}{m}  \sum_{i=1}^m y_i \cdot f(x_i)}$. We have the following bounded difference condition on $\phi_2$. 
    For all i,
    \begin{align}
        \sup_{x_1, \ldots x_m, x_i^\prime \in \calX^{m+1} } \abs{ \phi_2 (x_1,\ldots, x_i, \ldots, x_m) - \phi_1 (x_1,\ldots, x_i^\prime, \ldots, x_m)  } \le 1/m \,. \label{cond2_rademacher}
    \end{align}
    Using McDiarmid’s inequality (\lemref{lem:McDiarmid}) twice 
    with Condition \eqref{cond1_rademacher} and \eqref{cond2_rademacher}, 
    with probability at least $1-\delta$, we have
    \begin{align}
        \sup_{f \in \calF} \, \frac{1}{m} \sum_{i=1}^m y_i \cdot f(x_i)  - \Expt{x,y}{\sup_{f \in \calF} \, \frac{1}{m} \sum_{i=1}^m y_i \cdot f(x_i) } \le \sqrt{\frac{2\log(2/\delta)}{m}} \,. \label{eq:final_rademacher}
    \end{align} 
    Combining \eqref{eq:error_rademacher} and \eqref{eq:final_rademacher}, we obtain the desired result. 
\end{proof}


\subsection{Proof of \thmref{thm:error_regularized_ERM}}

Proof of \thmref{thm:error_regularized_ERM} follows similar to the proof of \thmref{thm:error_ERM}. Note that the same results in \lemref{lem:fit_mislabeled}, \lemref{lem:mislabeled_error}, and \lemref{lem:clear_error} hold in the regularized ERM case. However, the arguments in the proof of \lemref{lem:fit_mislabeled} change slightly. Hence, we state the lemma for regularized ERM and prove it here for completeness. 

\begin{lemma} \label{lem:lemma1_reg}
    Assume the same setup as \thmref{thm:error_regularized_ERM}. 
    Then for any $\delta >0$, with probability at least  $1-\delta$ 
    over the random draws of mislabeled data $\wt S_M$, we have 
    \begin{align}
        \error_\calD(\widehat f)  \le 1 -\error_{\wt \calS_M}(\widehat f) + \sqrt{\frac{\log(1/\delta)}{m}}\,. 
    \end{align} 
\end{lemma}
\begin{proof}
    The main idea of the proof remains the same, i.e. regard 
    the clean portion of the data 
    ($S \cup \wt S_C$) as fixed.   
    Then, there exists a classifier $f^*$ 
    that is optimal over draws 
    of the mislabeled data $\wt S_M$. 

    
    Formally, 
    \begin{align}
    f^* \defeq \argmin_{f \in \calF} \error_{\widecheck {\calD}} (f)  + \lambda R(f) \,, \label{eq:modified_ERM_reg}
    \end{align}
    where $$\widecheck \calD = \frac{n}{m+n} \calS + \frac{m_1}{m+n} \wt \calS_C  + \frac{m_2}{m+n}\calDm \,.$$ That is, $\widecheck \calD$ a combination of 
    the \emph{empirical distribution} 
    over correctly labeled data $S \cup \wt S_C$
    % in $S\cup \wt S$ 
    and the (population) distribution 
    over mislabeled data $\calDm$.
    Recall that 
    \begin{align}
    \wh f \defeq \argmin_{f \in \calF} \error_{\calS \cup \wt S} (f) + \lambda R(f) \,. \label{eq:orig_ERM_reg}
    \end{align}
    % 
    % 
    Since, $\widehat f$ minimizes 0-1 error 
    on $S \cup \wt S$, using ERM optimality on \eqref{eq:orig_ERM},  
    we have 
    \begin{align}
        \error_{\calS \cup \wt \calS}(\widehat f) + \lambda R(\wh f) \le \error_{
            \calS \cup \wt \calS}(f^*) + \lambda R(f^*) \,.    \label{eq:step1_reg}
    \end{align}
    Moreover, since $f^*$ is independent of $\wt S_M$, using Hoeffding's bound,
    % \footnote{For a fully rigorous argument,
    % refer to the complete proof in App.~\ref{app:proof_erm}.} 
    we have with probability at least $1-\delta$ that
    \begin{align}
      \error_{\wt \calS_M}(f^*) \le \error_{ \calDm}(f^*) +  \sqrt{\frac{\log(1/\delta)}{2 m_1}} \,. \label{eq:step2_reg} 
    \end{align}
    %$ 
    %for some constant $c_1\le 1/2$. 
    Finally, since $f^*$ is the optimal classifier on $\widecheck \calD$, 
    we have 
    \begin{align}
        \error_{\widecheck \calD}(f^*) + \lambda R(f^*) \le \error_{\widecheck \calD}(\widehat f) + \lambda R(\wh f) \,. \label{eq:step3_reg}
    \end{align}
     Now to relate \eqref{eq:step1_reg} and \eqref{eq:step3_reg}, we can re-write the \eqref{eq:step2_reg} as follows: 
    \begin{align}
        \error_{\calS \cup \wt\calS}(f^*) \le \error_{ \widecheck \calD}(f^*) +  \frac{m_1}{m+n}\sqrt{\frac{\log(1/\delta)}{2 m_1}} \,. \label{eq:step4_reg} 
    \end{align}
    After adding $\lambda R(f^*)$ on both sides in \eqref{eq:step4_reg}, we combine equations \eqref{eq:step1_reg}, \eqref{eq:step4_reg}, and \eqref{eq:step3_reg}, to get 
    \begin{align}
        \error_{\calS \cup \wt \calS}(\wh f) \le \error_{\widecheck \calD}(\wh f) +  \frac{m_1}{m+n}\sqrt{\frac{\log(1/\delta)}{2 m_1}} \,, 
    \end{align}
    which implies 
    \begin{align}
        \error_{ \wt \calS_M}(\wh f) \le \error_{\calDm}(\wh f) + \sqrt{\frac{\log(1/\delta)}{2 m_1}} \,. \label{eq:lemma_reg_final}
    \end{align}
    Similar as before, since $\wt S$ is obtained by randomly labeling an unlabeled dataset, we assume 
    $2m_1 \approx m$. Moreover, using $\error_{\calDm} = 1 - \error_{\calD}$ we obtain the desired result. 
\end{proof}
% \begin{proof}[Proof of ]
    
% \end{proof}

\subsection{Proof of \thmref{thm:multiclass_ERM}}

To prove our results in the multiclass case,
we first state and prove lemmas
parallel to those
% We first state and prove lemmas 
% parallel 
% to the three lemmas 
used in the proof of balanced binary case. 
We then combine these results 
% in the three lemmas 
to obtain the result in \thmref{thm:multiclass_ERM}. 

Before stating the result, 
we define mislabeled distribution $\calDm$ for any $\calD$.
While $\calDm$ and $\calD$ share 
the same marginal distribution over inputs $\calX$,
the conditional distribution over labels $y$ 
given an input $x\sim \calD_\calX$ is changed as follows:
For any $x$, the Probability Mass Function (PMF) over $y$ is defined as:  
$p_{\calDm} (\cdot \vert x) \defeq \frac{1 - p_{\calD}(\cdot \vert x)}{k - 1}$, where $ p_{\calD}(\cdot \vert x)$ is the PMF over $y$ for the distribution $\calD$. 

\begin{lemma} \label{lem:fit_mislabeled_multi}
    Assume the same setup as \thmref{thm:multiclass_ERM}. 
    Then for any $\delta >0$, with probability at least  $1-\delta$ 
    over the random draws of mislabeled data $\wt S_M$, we have 
    \begin{align}
        \error_\calD(\widehat f)  \le (k-1)\left(1 -\error_{\wt \calS_M}(\widehat f)\right) + (k-1)\sqrt{\frac{\log(1/\delta)}{m}}\,. \label{eq:lemma1_multi}
    \end{align}   
\end{lemma} 

\begin{proof}
   
    The main idea of the proof remains the same.
    We begin by regarding the clean portion of the data 
    ($S \cup \wt S_C$) as fixed. 
    Then, there exists a classifier $f^*$ 
    that is optimal over draws 
    of the mislabeled data $\wt S_M$. 
    
    However, in the multiclass case,
    we cannot as easily relate the population error on mislabeled data 
    to the population accuracy on clean data.   
    While for binary classification, 
    % we could upper bound $\error_{\wt \calS_M}$ 
    % with $1-\error_\calD$ 
    we could lower bound the population accuracy $1-\error_\calD$
    with the empirical error on mislabeled data $\error_{\wt \calS_M}$ 
    (in the proof of \lemref{lem:fit_mislabeled}), 
    for multiclass classification, 
    error on the mislabeled data 
    and accuracy on the clean data 
    in the population 
    are not so directly related.  
    To establish \eqref{eq:lemma1_multi},
    we break the error on the 
    (unknown) mislabeled data 
    into two parts: one term corresponds 
    to predicting the true label on mislabeled data, 
    and the other corresponds to predicting 
    neither the true label 
    nor the assigned (mis-)label.  
    Finally, we relate these errors to their
    population counterparts to establish \eqref{eq:lemma1_multi}. 
    
    Formally, 
    \begin{align}
    f^* \defeq \argmin_{f \in \calF} \error_{\widecheck {\calD}} (f)  + \lambda R(f) \,, \label{eq:modified_ERM_reg2}
    \end{align}
    where $$\widecheck \calD = \frac{n}{m+n} \calS + \frac{m_1}{m+n} \wt \calS_C  + \frac{m_2}{m+n}\calDm \,.$$ 
    That is, $\widecheck \calD$ is a combination 
    of the \emph{empirical distribution} 
    over correctly labeled data $S \cup \wt S_C$
    % in $S\cup \wt S$ 
    and the (population) distribution 
    over mislabeled data $\calDm$.
    Recall that 
    \begin{align}
    \wh f \defeq \argmin_{f \in \calF} \error_{\calS \cup \wt S} (f) + \lambda R(f) \,. \label{eq:orig_ERM_reg2}
    \end{align}
    % 
    % 
    Following the exact steps from the proof of \lemref{lem:lemma1_reg}, 
    with probability at least $1-\delta$, we have  
    \begin{align}
        \error_{ \wt \calS_M}(\wh f) \le \error_{\calDm}(\wh f) + \sqrt{\frac{\log(1/\delta)}{2 m_1}} \,. \label{eq:lemma1_final_multi_prev}
    \end{align}
    Similar to before, since $\wt S$ is obtained 
    by randomly labeling an unlabeled dataset, 
    we assume 
    $\frac{k}{k-1} m_1 \approx m$. 
    
    Now we will relate $\error_{\calDm} (\wh f)$ with $\error_{\calD}(\wh f)$. 
    Let $y^T$ denote the (unknown) true label 
    for a mislabeled point $(x, y)$ 
    (i.e., label before replacing it with a mislabel). 
    \begin{align*}    
         \Expt{(x, y) \in \sim \calDm}{\indict{ \wh f(x) \ne y }}  &= \underbrace{\Expt{(x, y) \in \sim \calDm}{\indict{ \wh f(x) \ne y \land \wh f(x) \ne y^T}}}_{\RN{1}} \\ &\qquad \qquad + \underbrace{\Expt{(x, y) \in \sim \calDm}{\indict{ \wh f(x) \ne y \land \wh f(x) = y^T}}}_{\RN{2}} \,. \numberthis \label{eq:excess_term}
    \end{align*}
    Clearly, term 2 is one minus the accuracy 
    on the clean unseen data, i.e.,
    \begin{align}
        \RN{2} = 1 - \Expt{{x,y} \sim \calD}{ \indict{ \wh f(x) \ne y}} = 1- \error_{\calD}(\wh f) \,. \label{eq:term1}    
    \end{align}
    Next, we relate term 1 with the error on the unseen clean data. 
    We show that term 1 is equal to the error on the unseen clean data 
    scaled by $\frac{k-2}{k-1}$,
    where $k$ is the number of labels.
    Using the definition of mislabeled distribution $\calDm$,  
    we have 
    \begin{align}
        \RN{1} = \frac{1}{k-1} \left( \Expt{(x, y) \in \sim \calD}{ \sum_{i \in \calY \land i\ne y}  \indict{ \wh f(x) \ne i \land \wh f(x) \ne y}} \right) = \frac{k-2}{k-1} \error_{\calD}(\wh f) \,.\label{eq:term2}
    \end{align}    

    Combining the result in \eqref{eq:term1}, \eqref{eq:term2} and \eqref{eq:excess_term}, we have 
    \begin{align}
        \error_{\calDm}(\wh f) = 1- \frac{1}{k-1} \error_{\calD}(\wh f) \,.\label{eq:combine_terms}
    \end{align}
    Finally, combining the result in \eqref{eq:combine_terms} 
    with equation \eqref{eq:lemma1_final_multi_prev}, 
    we have with probability $1-\delta$, 
    \begin{align}
      \error_{\calD}(\wh f) \le  (k-1) \left( 1- \error_{ \wt \calS_M}(\wh f) \right)  + (k-1) \sqrt{\frac{k \log(1/\delta)}{ 2(k-1)m}} \,. \label{eq:lemma1_final_multi}
    \end{align}
\end{proof}

\begin{lemma} \label{lem:mislabeled_error_multi}
    Assume the same setup as \thmref{thm:multiclass_ERM}. 
    Then for any $\delta >0$, 
    with probability at least $1-\delta$ 
    over the random draws of $\wt S$, we have  
    % \begin{align}
        $$\abs{k\error_{\wt \calS}(\widehat f) - \error_{\wt \calS_C}(\widehat f) -  (k-1)\error_{\wt \calS_M}(\widehat f) } \le  2k\sqrt{\frac{\log(4/\delta)}{2m}}\,. $$ % \label{eq:lemma2}
    % \end{align}   
    %  for some constant $c_3 \le 1.0\,$.
\end{lemma} 


\begin{proof}
    Recall $\error_{\wt S} (f) = \frac{m_1}{m} \error_{\wt S_M}(f) + \frac{m_2}{m} \error_{\wt S_C}(f)$. Hence, we have 
    \begin{align*}
        k\error_{\wt S}(f) - (k-1)\error_{\wt S_M}(f) - \error_{\wt S_C}(f) &= (k-1)\left(\frac{k m_1}{(k-1) m} \error_{\wt S_M}(f) - \error_{\wt S_M}(f)\right) \\ & \qquad \qquad + \left(\frac{km_2}{m} \error_{\wt S_C}(f) - \error_{\wt S_C}(f)\right) \\ &= k \left[ \left(\frac{m_1}{m} - \frac{k-1}{k}\right) \error_{\wt S_M}(f) + \left(\frac{m_2}{m} - \frac{1}{k} \right) \error_{\wt S_C} (f) \right] \,.
    \end{align*} 
    Since the dataset is randomly labeled, 
    we have with probability at least $1-\delta$, 
    $\left(\frac{m_1}{m} - \frac{k-1}{k}\right) \le \sqrt{\frac{\log(1/\delta)}{2m}}$. 
    Similarly, we have with probability at least $1-\delta$, 
    $\left(\frac{m_2}{m} - \frac{1}{k}\right) \le \sqrt{\frac{\log(1/\delta)}{2m}}$. 
    Using union bound, we have with probability at least $1-\delta$
    % \begin{align}
    %     2\error_{\wt S} - \error_{\wt S_M}(f) - \error_{\wt S_C}(f) \le \sqrt{\frac{\log(2/\delta)}{2m}} \left(\error_{\wt S_M}(f) + \error_{\wt S_C}(f) \right) \le 2\sqrt{\frac{\log(2/\delta)}{2m}} \,. \label{eq:lemma2_final}
    % \end{align}
    \begin{align}
        k\error_{\wt S}(f) - (k-1)\error_{\wt S_M}(f) - \error_{\wt S_C}(f)  \le k \sqrt{\frac{\log(2/\delta)}{2m}} \left(\error_{\wt S_M}(f) + \error_{\wt S_C}(f) \right) \,. \label{eq:lemma2_final_multi}
    \end{align}

    % We obtain the desired result by using 
\end{proof}

\begin{lemma} \label{lem:clear_error_multi}
    Assume the same setup as \thmref{thm:multiclass_ERM}. 
    Then for any $\delta >0$, with probability at least $1-\delta$ 
    over the random draws of $\wt S_C$ and $S$, we have 
    % \begin{align}
        $$\abs{\error_{\wt \calS_C}(\widehat f) - \error_{\calS}(\widehat f) } \le 1.5 \sqrt{\frac{k\log(2/\delta)}{2m}}\,.$$ %\label{eq:lemma3}
    % \end{align}   
    % for some constant $c_2 \le 1.2\,$.
\end{lemma} 
\begin{proof}
    % Recall 0-1 error on each point  $(x,y) \in S \cup \wt S$ is given by $\I{ f(x)\ne y}$.
    In the set of correctly labeled points $S \cup \wt S_C$,
    we have $S$ as a random subset of $S \cup \wt S_C$. 
    Hence, using Hoeffding's inequality 
    for sampling without replacement 
    (\lemref{lem:hoeffding_sampling}), 
    we have with probability at least $1-\delta$
    \begin{align}
        \error_{\wt \calS_c} (\wh f)- \error_{\calS \cup \wt \calS_C}( \wh f) \le  \sqrt{\frac{\log(1/\delta)}{2m_2}} \,.
    \end{align}
    Re-writing $\error_{\calS \cup \wt \calS_C}( \wh f)$ 
    as $\frac{m_2}{m_2 + n} \error_{\wt \calS_C }(\wh f) + \frac{n}{m_2 + n} \error_{\calS }(\wh f)$, 
    we have with probability at least $1-\delta$
    \begin{align}
       \left(\frac{n}{n+m_2}\right) \left(\error_{\wt \calS_c} (\wh f)- \error_{\calS}( \wh f) \right) \le  \sqrt{\frac{\log(1/\delta)}{2m_2}} \,.
    \end{align}
    As before, assuming $km_2 \approx m$, 
    we have with probability at least $1-\delta$ 
    \begin{align}
        \error_{\wt \calS_c} (\wh f)- \error_{\calS}( \wh f) \le \left(1+\frac{m_2}{n}\right)  \sqrt{\frac{k\log(1/\delta)}{2m}} \le \left( 1 + \frac{1}{k}\right) \sqrt{\frac{k\log(1/\delta)}{2m}} \,. \label{eq:lemma3_final_multi}
    \end{align} 
\end{proof}

\begin{proof}[Proof of \thmref{thm:multiclass_ERM}] 
    Having established these core intermediate results, 
    we can now combine above three lemmas. 
    In particular, we bound the population error 
    on clean data ($\error_\calD(\wh f)$) as follows:  
    \begin{enumerate}[(i)]
        \item First, use \eqref{eq:lemma1_final_multi}, 
        to obtain an upper bound on the population error on clean data, 
        i.e., with probability at least $1-\delta/4$, we have
        \begin{align}
            \error_{ \calD} (\wh f) \le (k-1)\left(1 - \error_{ \wt \calS_M}(\wh f) \right) + (k-1) \sqrt{\frac{k\log(4/\delta)}{2(k-1)m}} \,. 
        \end{align}
        \item  Second, use \eqref{eq:lemma2_final_multi}
        to relate the error on the mislabeled fraction 
        with error on clean portion of randomly labeled data 
        and error on whole randomly labeled dataset, 
        i.e., with probability at least $1-\delta/2$, we have 
        \begin{align}
            - (k-1)\error_{\wt S_M}(f) \le \error_{\wt S_C}(f) - k\error_{\wt S}  + k\sqrt{\frac{\log(4/\delta)}{2m}}  \,. 
        \end{align} 
        \item Finally, use \eqref{eq:lemma3_final_multi} 
        to relate the error on the clean portion of randomly labeled data 
        and error on clean training data, 
        i.e., with probability $1-\delta/4$, we have 
        \begin{align}
            \error_{\wt \calS_C} (\wh f)\le - \error_{\calS}( \wh f) + \left(1 + \frac{m}{kn} \right) \sqrt{\frac{k\log(4/\delta)}{2m}} \,. 
        \end{align} 
    \end{enumerate}

    Using union bound on the above three steps, 
    we have with probability at least $1-\delta$: 
    \begin{align}
        \error_\calD (\wh f) \le \error_{\calS}(\wh f) + (k-1) - k\error_{\wt \calS}(\wh f)   + (\sqrt{k(k-1)} + k + \sqrt{k} + \frac{m}{n\sqrt{k}})  \sqrt{\frac{\log(4/\delta)}{2m}} \,.\label{eq:multiclass_ERM_final}
    \end{align}
    Simplifying the term in RHS of \eqref{eq:multiclass_ERM_final}, 
    we get the desired result. 
    % Note that since $\frac{m}{n\sqrt{k}}$ 
    % is much smaller than the sum of the other terms
    % the other terms in summation, 
    % we ignore $\frac{m}{n\sqrt{k}}$  
    % Z: ??? --- great
    % that 
    % them
    in the final bound. 
    % we ignore that in the final bound. 
    % Note that $(1/\sqrt{2} + 2.5)$ is a loose constant. In experiments, we use the ratio $\frac{m}{n}$
    %  the exact error $\error_{\wt \calS}(\wh f)$ 
    % to evaluate R.H.S.    
\end{proof}

\newpage
\section{Proofs from \secref{sec:linear_models}}\label{app:proof_gd}
We suppose that the parameters of the linear function 
are obtained via gradient descent on 
the following $L_2$ regularized problem: 
\begin{align}
    % n in denominator is avoided deliberately
    \calL_S(w; \lambda) \defeq \sum_{i=1}^n{(w^Tx_i - y_i)^2} + \lambda \norm{w}{2}^2 \,, \label{eq:l2_MSE_app}   
\end{align}
where $\lambda\ge0$ is a regularization parameter. 
We assume access to a clean dataset 
$S = \{(x_i, y_i)\}_{i=1}^n \sim \calD^n$ 
and randomly labeled dataset 
$\wt S = \{(x_i, y_i)\}_{i=n+1}^{n+m} \sim \wt \calD^m$. 
Let $\bX = [x_1, x_2, \cdots, x_{m+n}]$ 
and $\by = [y_1, y_2, \cdots, y_{m+n}]$. 
Fix a positive learning rate $\eta$ such that 
$\eta \le 1/\left(\norm{\bX^T\bX}{\text{op}} + \lambda^2\right)$ 
and an initialization $w_0 = 0$. 
% \todos{Assumption made for simplicty}. 
Consider the following gradient descent iterates 
to minimize objective \eqref{eq:l2_MSE_app} on $S \cup \wt S$:
\begin{align}
w_t = w_{t-1} - \eta \grad_w \calL_{S \cup \wt S} (w_{t-1}; \lambda) \quad \forall t=1,2,\ldots \label{eq:GD_iterates_app}
\end{align} 
Then we have $\{ w_t\}$ converge to the limiting solution 
$\wh w = \left( \bX^T\bX+\lambda \boldsymbol{I}\right)^{-1}\bX^T\by$. Define $\widehat f (x) \defeq f(x ; \wh w) $.  

% \subsection{\textcolor{red}{Errata}}

% We wish to correct the following error in the body:
% \codref{cond:error_stability} is not enough 
% to guarantee the result in \thmref{thm:linear}. 
% We now present a slightly stronger condition 
% called \emph{hypothesis stability} 
% under which we obtain a result 
% similar to \thmref{thm:linear}. 

% This error doesn't change the main arguments of the proof,
% where we show that the empirical train error 
% is less than or equal to the leave-one-out error.
% We need a stronger condition to relate leave-one-out error 
% with the population error of the original classifier. 
% Specifically, while \codref{cond:error_stability} 
% relates the average population error of leave-one-out classifiers 
% with the population error of the original classifier, 
% we need the new condition to show the concentration 
% of the empirical leave-one-out error 
% and average population error of leave-one-out classifiers. 
% main takeaway 

% Note that the new condition, 
% while being stronger than the previous one, 
% still doesn't imply generalization \citep{bousquet2002stability,elisseeff2003leave,abou2019exponential}. 
% Overall, the main results in \secref{sec:ERM_training} 
% and takeaways of the paper remain unaffected by the error.  

% We now present the new condition 
% and a corrected statement of \thmref{thm:linear}. 
% Recall, for a given training set $S \sim \calD^n $, 
% we use $S_{(i)}$ to denote the training set $S$ 
% with the $i^{\text{th}}$ point removed.

% \begin{condition}[Hypothesis Stability] 
%     \label{cond:hypothesis_stability}
%     We have $\beta$ hypothesis stability 
%     if our training algorithm $\calA$ satisfies the following: 
%     \begin{align*}
%     % ${\sum_{i=1}^n \frac{\error_{\calD}( f(\calA, S_{(i)}))}{n} - \error_\calD(f(\calA, S))} \le \beta\,$.
%     \forall i \in \{1,2,\ldots, n\}, \quad  \Expt{\calS, (x,y) \in \calD}{ \abs{\error\left( f(x) ,y  \right) - \error\left( f_{(i)}(x), y \right) }} \le \frac{\beta}{n} \,,
%     \end{align*}
%     where $f_{(i)} \defeq f(\calA, S_{(i)})$ and $ f \defeq f(\calA, S)$.
% \end{condition}

% \begin{theorem}[Correct statement of \thmref{thm:linear}] \label{thm:new_linear}
%     Assume that this gradient descent algorithm satisfies \codref{cond:hypothesis_stability}
%     with $\beta=\calO(1)$.  
%     Then for any $\delta >0$, with probability at least $1-\delta$ 
%     over the random draws of datasets $\wt S$ and $S$, we have:
%     \begin{align}
%         \error_\calD(\widehat f) \le \error_\calS(\widehat f) + 1 - 2 \error_{\wt\calS}(\widehat f) + \left(\frac{1}{\sqrt{2}} + 1.5 \right) \sqrt{\frac{\log(4/\delta)}{m}} + \sqrt{\frac{4}{\delta}\left(\frac{1}{m} +\frac{3\beta}{m+n} \right)}  \,. \label{eq:gd_error}
%     \end{align} 
%     % for some constant $c\le 3.2$.
% \end{theorem}

\subsection{Proof of \thmref{thm:linear}}
We use a standard result from linear algebra, 
namely the Shermann-Morrison formula 
\citep{sherman1950adjustment} for matrix inversion:  

\begin{lemma}[\citet{sherman1950adjustment}] \label{lem:sherman}
    Suppose $\bA \in \Real^{n \times n}$ 
    is an invertible square matrix 
    and $u,v \in \Real^n$ are column vectors. 
    Then $\bA + uv^T$ is invertible iff $1 + v^T \bA u \ne 0$ 
    and in particular
    \begin{align}
        (\bA + u v^T)^{-1} = \bA^{-1}  - \frac{\bA^{-1} uv^T \bA^{-1} }{ 1 + v^T \bA^{-1} u} \,.
    \end{align}   
\end{lemma}
\newcommand\byy[1]{\by_{\left(#1\right)}}
\newcommand\bXX[1]{\bX_{\left(#1\right)}}
\newcommand\ff[1]{\wh f_{\left(#1\right)}}

For a given training set $S \cup \wt S_C$, 
define leave-one-out error 
on mislabeled points in the training data 
as $$\error_{\text{LOO}(\wt S_M) } = \frac{\sum_{(x_i, y_i) \in \wt S_M} \error( f_{(i)}( x_i), y_i)}{ \abs{\wt S_M }} \,, $$
where $f_{(i)} \defeq f(\calA, (S \cup \wt S)_{(i)})$. 
To relate empirical leave-one-out error and population error 
with hypothesis stability condition, 
we use the following lemma:   

\begin{lemma}[\citet{bousquet2002stability}] \label{lem:stability_error}
    For the leave-one-out error, we have
    \begin{align}
        \Expo{ \left( \error_{\calDm}(\wh f) -\error_{\text{LOO}(\wt S_M) } \right)^2 } \le \frac{1}{2m_1}+  \frac{3\beta}{n + m}\,.
    \end{align}   
    % where $ f \defeq f(\calA, S \cup \wt S) $.
\end{lemma}

Proof of the above lemma is similar 
to the proof of Lemma 9 in \citet{bousquet2002stability} 
and can be found in \appref{app:proof_lem_error}. 
% 
% Before presenting the result, we introduce some notation. 
Before presenting the proof of \thmref{thm:linear}, 
we introduce some more notation. 
Let $\bX_{(i)}$ denote the matrix of covariates 
with the $i^{\text{th}}$ point removed. 
Similarly, let $\by_{(i)}$ be the array of responses 
with the $i^{\text{th}}$ point removed. 
Define the corresponding regularized GD solution 
as $\wh w_{(i)} = \left( \bXX{i}^T\bXX{i}+\lambda \boldsymbol{I}\right)^{-1}\bXX{i}^T\byy{i}$. 
Define $\ff{i}(x) \defeq f(x ; \wh w_{(i)}) $.

\begin{proof}[Proof of \thmref{thm:linear}]
    Because squared loss minimization does not imply 0-1 error minimization, 
    we cannot use arguments from \lemref{lem:fit_mislabeled}. 
    This is the main technical difficulty. 
    To compare the 0-1 error at a train point with an unseen point, 
    we use the closed-form expression for $\widehat{w}$ 
    and Shermann-Morrison formula 
    to upper bound training error 
    with leave-one-out cross validation error. 
    
    The proof is divided into three parts: 
    In part one, we show that 0-1 error 
    on mislabeled points in the training set 
    is lower than the error obtained 
    by leave-one-out error at those points. 
    In part two, we relate this leave-one-out error 
    with the population error on mislabeled distribution
    using \codref{cond:hypothesis_stability}.
    While the empirical leave-one-out error is an unbiased estimator 
    of the average population error of leave-one-out classifiers, 
    we need hypothesis stability 
    to control the variance 
    of empirical leave-one-out error. 
    Finally, in part three, we show 
    that the error on the mislabeled training points 
    can be estimated with just the randomly labeled 
    and clean training data (as in proof of \thmref{thm:error_ERM}).  

    \textbf{Part 1 {} {}} First we relate training error with leave-one-out error.        
    For any training point $(x_i, y_i)$ in $\wt S \cup S$, we have 
    \begin{align}
        \error(\wh f(x_i), y_i ) &= \indict{ y_i \cdot x_i^T \wh w < 0 } = \indict{ y_i \cdot x_i^T \left( \bX^T\bX+\lambda \boldsymbol{I}\right)^{-1}\bX^T\by < 0 } \\
        &= \indict{ y_i \cdot x_i^T \underbrace{\left( \bXX{i}^T\bXX{i} + x_i ^T x_i +\lambda \boldsymbol{I}\right)^{-1}}_{\RN{1}} (\bXX{i}^T\byy{i} + y_i \cdot x_i) < 0 } \,.
    \end{align}
    Letting $\bA = \left(\bXX{i}^T\bXX{i} +\lambda \boldsymbol{I}\right)$ 
    and using \lemref{lem:sherman} on term 1, we have 
    \begin{align}
        \error(\wh f(x_i), y_i ) &= \indict{ y_i \cdot x_i^T \left[\bA^{-1} -  \frac{\bA^{-1} x_i x_i^T \bA^{-1}}{ 1 + x_i ^T \bA^{-1} x_i } \right] (\bXX{i}^T\byy{i} + y_i \cdot x_i) < 0 } \\
        &= \indict{ y_i \cdot\left[ \frac{ x_i^T \bA^{-1} ( 1 + x_i ^T \bA^{-1} x_i ) -  x_i^T \bA^{-1} x_i x_i^T \bA^{-1}}{ 1 + x_i ^T \bA ^{-1}x_i } \right] (\bXX{i}^T\byy{i} + y_i \cdot x_i) < 0 } \\
        &= \indict{ y_i \cdot\left[ \frac{ x_i^T \bA^{-1}}{ 1 + x_i ^T \bA ^{-1}x_i } \right] (\bXX{i}^T\byy{i} + y_i \cdot x_i) < 0 } \,.
    \end{align}

    Since $1 + x_i^T \bA^{-1} x_i > 0$, we have 
    \begin{align}
        \error(\wh f(x_i), y_i ) &= \indict{ y_i \cdot x_i^T \bA^{-1} (\bXX{i}^T\byy{i} + y_i \cdot x_i) < 0 } \\
        &= \indict{ x_i^T \bA^{-1} x_i +  y_i \cdot x_i^T \bA^{-1} (\bXX{i}^T\byy{i}) < 0 } \\
        &\le \indict{ y_i \cdot x_i^T \bA^{-1} (\bXX{i}^T\byy{i}) < 0 } = \error(\ff{i}(x_i), y_i ) \,.\label{eq:LOO_error}
    \end{align}

    Using \eqref{eq:LOO_error}, we have 
    \begin{align}
        \error_{\wt \calS_M } (\wh f) \le \error_{\text{LOO} (\wt S_M)} \defeq \frac{\sum_{(x_i, y_i) \in \wt S_M} \error(\ff{i}(x_i), y_i ) }{\abs{\wt \calS_M}}\label{eq:LOO_error_final} \,.
    \end{align}
    \textbf{Part 2 {}{}} We now relate RHS in \eqref{eq:LOO_error_final} 
    with the population error on mislabeled distribution. 
    To do this, we leverage \codref{cond:hypothesis_stability} 
    and \lemref{lem:stability_error}. 
    In particular, we have 

    \begin{align}
        \Expt{\calS \cup \wt \calS_M }{ \left(\error_{\calDm}(\wh f) - \error_{\text{LOO} (\wt S_M)}\right)^2 } \le \frac{1}{2m_1} + \frac{3\beta}{m+n} \,.
    \end{align}

    Using Chebyshev's inequality, with probability at least $1-\delta$, we have 
    \begin{align}
        \error_{\text{LOO} (\wt S_M)} \le  \error_{\calDm}(\wh f)   + \sqrt{\frac{1}{\delta}\left(\frac{1}{2m_1} +\frac{3\beta}{m+n} \right)} \,. \label{eq:final_mislabeled_linear}
    \end{align}
    

    \textbf{Part 3 {}{}} Combining \eqref{eq:final_mislabeled_linear} and \eqref{eq:LOO_error_final}, we have 

    \begin{align}
        \error_{\wt \calS_M } (\wh f) \le \error_{\calDm}(\wh f)   + \sqrt{\frac{1}{\delta}\left(\frac{1}{2m_1} +\frac{3\beta}{m+n} \right)} \,. \label{eq:linear_parallel_lem1}
    \end{align}

    Compare \eqref{eq:linear_parallel_lem1} with \eqref{eq:lemma1_final} 
    in the proof of \lemref{lem:fit_mislabeled}. 
    We obtain a similar relationship 
    between $\error_{\wt \calS_M }$ and $\error_{\calDm}$ 
    but with a polynomial concentration 
    instead of exponential concentration. 
    In addition, since we just use concentration arguments 
    to relate mislabeled error to the errors
    on the clean and unlabeled portions 
    of the randomly labeled data, 
    we can directly use the results 
    in \lemref{lem:mislabeled_error} and \lemref{lem:clear_error}. 
    Therefore, combining results in \lemref{lem:mislabeled_error}, \lemref{lem:clear_error}, and \eqref{eq:linear_parallel_lem1} with union bound, 
    we have with probability at least $1-\delta$
    \begin{align}
        \error_\calD(\widehat f) \le \error_\calS(\widehat f) + 1 - 2 \error_{\wt\calS}(\widehat f) + \left(\sqrt{2}\error_{\wt\calS}(\widehat f) + 1 + \frac{m}{2n} \right) \sqrt{\frac{\log(4/\delta)}{m}} + \sqrt{\frac{4}{\delta}\left(\frac{1}{m} +\frac{3\beta}{m+n} \right)}  \,.
    \end{align}
    

       
\end{proof}

\subsection{Extension to multiclass classification} \label{app:multiclass_linear}
For multiclass problems with squared loss minimization, as standard practice, we consider one-hot encoding for the underlying label, i.e., a class label $c \in [k]$ is treated as $(0, \cdot, 0,1,0, \cdot, 0) \in \Real^k$ (with $c$-th coordinate being 1).  As before, we suppose that the parameters of the linear function 
are obtained via gradient descent on the following $L_2$ regularized problem: 
\begin{align}
    % n in denominator is avoided deliberately
    \calL_S(w; \lambda) \defeq \sum_{i=1}^n\norm{w^Tx_i - y_i}{2}^2 + \lambda \sum_{j=1}^k \norm{w_j}{2}^2 \,, \label{eq:l2_multiclass_MSE_app}   
\end{align}
where $\lambda\ge0$ is a regularization parameter. 
We assume access to a clean dataset 
$S = \{(x_i, y_i)\}_{i=1}^n \sim \calD^n$ 
and randomly labeled dataset 
$\wt S = \{(x_i, y_i)\}_{i=n+1}^{n+m} \sim \wt \calD^m$. 
Let $\bX = [x_1, x_2, \cdots, x_{m+n}]$ 
and $\by = [e_{y_1}, e_{y_2}, \cdots, e_{y_{m+n}}]$. 
Fix a positive learning rate $\eta$ such that 
$\eta \le 1/\left(\norm{\bX^T\bX}{\text{op}} + \lambda^2\right)$ 
and an initialization $w_0 = 0$. 
% \todos{Assumption made for simplicty}. 
Consider the following gradient descent iterates 
to minimize objective \eqref{eq:l2_MSE_app} on $S \cup \wt S$:
\begin{align}
{w_j}^t = {w_j}^{t-1} - \eta \grad_{w_j} \calL_{S \cup \wt S} (w^{t-1}; \lambda) \quad \forall t=1,2,\ldots \text{ and } j=1,2,\ldots,k  \,. \label{eq:GD_multi_iterates_app}
\end{align} 
Then we have $\{ {w_j}^t\}$ for all $j =1,2,\cdots, k$ converge to the limiting solution 
$\wh w_j = \left( \bX^T\bX+\lambda \boldsymbol{I}\right)^{-1}\bX^T\by_j$. Define $\widehat f (x) \defeq f(x ; \wh w) $.  

\begin{theorem}\label{thm:multi_linear}
    Assume that this gradient descent algorithm satisfies \codref{cond:hypothesis_stability}
    with $\beta=\calO(1)$.  
    Then for a multiclass classification problem wth $k$ classes, for any $\delta >0$, with probability at least $1-\delta$, we have:
    \begin{align*}
        \error_\calD(\widehat f) \le \error_\calS(\widehat f) &+ (k-1)\left(1 - \frac{k}{k-1} \error_{\wt\calS}(\widehat f) \right) \\ &+ \left(k + \sqrt{k} + \frac{m}{n\sqrt{k}} \right) \sqrt{\frac{\log(4/\delta)}{2m}} + \sqrt{k(k-1)} \sqrt{\frac{4}{\delta}\left(\frac{1}{m} +\frac{3\beta}{m+n} \right)}  \,. \numberthis \label{eq:gd_multi_error}
    \end{align*} 
    % for some constant $c\le 3.2$.
\end{theorem}
\begin{proof}
    The proof of this theorem is divided into two parts. In the first part, we relate the error on the mislabeled samples with the population error on the mislabeled data. Similar to the proof of \thmref{thm:linear}, we use Shermann-Morrison formula to upper bound training error with leave-one-out error on each $\wh w^j$. Second part of the proof follows entirely from the proof of \thmref{thm:multiclass_ERM}. In essence, the first part derives an equivalent of \eqref{eq:lemma1_final_multi_prev} for GD training with squared loss and then the second part follows from the proof  of \thmref{thm:multiclass_ERM}. 
    
    \textbf{Part-1:} Consider a training point $(x_i,y_i)$ in $\wt S \cup S $. For simplicity, we use $c_i$ to denote the class of $i$-th point and use $y_i$ as the corresponding one-hot embedding. Recall error in multiclass point is given by $\error(\wh f(x_i), y_i ) = \indict{ c_i \not \in \argmax x_i^T \wh w }$. Thus, there exists a $j \ne c_i \in [k]$, such that we have
     \begin{align}
        \error(\wh f(x_i), y_i ) &= \indict{ c_i \not \in \argmax x_i^T \wh w } = \indict{ x_i^T \wh w_{c_i} < x_i^T \wh w_{j}  } \\ &= \indict{ x_i^T \left( \bX^T\bX+\lambda \boldsymbol{I}\right)^{-1}\bX^T\by_{c_i} < x_i^T \left( \bX^T\bX+\lambda \boldsymbol{I}\right)^{-1}\bX^T\by_{j} } \\
        &= \indict{ x_i^T \underbrace{\left( \bXX{i}^T\bXX{i} + x_i ^T x_i +\lambda \boldsymbol{I}\right)^{-1}}_{\RN{1}} \left(\bXX{i}^T{\by_{c_i}}_{(i)} + x_i - \bXX{i}^T{\by_{j}}_{(i)}\right) < 0 } \,.
    \end{align}
    Letting $\bA = \left(\bXX{i}^T\bXX{i} +\lambda \boldsymbol{I}\right)$ 
    and using \lemref{lem:sherman} on term 1, we have 
    \begin{align}
        \error(\wh f(x_i), y_i ) &= \indict{ x_i^T \left[\bA^{-1} -  \frac{\bA^{-1} x_i x_i^T \bA^{-1}}{ 1 + x_i ^T \bA^{-1} x_i } \right]  \left(\bXX{i}^T{\by_{c_i}}_{(i)} + x_i - \bXX{i}^T{\by_{j}}_{(i)}\right) < 0 } \\
        &= \indict{ \left[ \frac{ x_i^T \bA^{-1} ( 1 + x_i ^T \bA^{-1} x_i ) -  x_i^T \bA^{-1} x_i x_i^T \bA^{-1}}{ 1 + x_i ^T \bA ^{-1}x_i } \right]  \left(\bXX{i}^T{\by_{c_i}}_{(i)} + x_i - \bXX{i}^T{\by_{j}}_{(i)}\right) < 0 } \\
        &= \indict{ \left[ \frac{ x_i^T \bA^{-1}}{ 1 + x_i ^T \bA ^{-1}x_i } \right]  \left(\bXX{i}^T{\by_{c_i}}_{(i)} + x_i - \bXX{i}^T{\by_{j}}_{(i)}\right) < 0} \,.
    \end{align}
    Since $1 + x_i^T \bA^{-1} x_i > 0$, we have 
    \begin{align}
        \error(\wh f(x_i), y_i ) &= \indict{ x_i^T \bA^{-1}  \left(\bXX{i}^T{\by_{c_i}}_{(i)} + x_i - \bXX{i}^T{\by_{j}}_{(i)}\right) < 0 } \\
        &= \indict{ x_i^T \bA^{-1} x_i +  x_i^T \bA^{-1}  \bXX{i}^T{\by_{c_i}}_{(i)}  - x_i^T\bA^{-1}  \bXX{i}^T{\by_{j}}_{(i)} < 0 } \\
        &\le \indict{  x_i^T \bA^{-1}  \bXX{i}^T{\by_{c_i}}_{(i)}  - x_i^T\bA^{-1}  \bXX{i}^T{\by_{j}}_{(i)} < 0  } = \error(\ff{i}(x_i), y_i ) \,.\label{eq:LOO_error_multi}
    \end{align}
    Using \eqref{eq:LOO_error_multi}, we have 
    \begin{align}
        \error_{\wt \calS_M } (\wh f) \le \error_{\text{LOO} (\wt S_M)} \defeq \frac{\sum_{(x_i, y_i) \in \wt S_M} \error(\ff{i}(x_i), y_i ) }{\abs{\wt \calS_M}}\label{eq:LOO_error_multi_final} \,.
    \end{align}
    
    We now relate RHS in \eqref{eq:LOO_error_final} 
    with the population error on mislabeled distribution. 
    Similar as before, to do this, we leverage \codref{cond:hypothesis_stability} 
    and \lemref{lem:stability_error}. Using  \eqref{eq:final_mislabeled_linear} and \eqref{eq:LOO_error_multi_final}, we have 
    \begin{align}
        \error_{\wt \calS_M } (\wh f) \le \error_{\calDm}(\wh f)   + \sqrt{\frac{1}{\delta}\left(\frac{1}{2m_1} +\frac{3\beta}{m+n} \right)} \,. \label{eq:linear_multi_parallel_lem1}
    \end{align}
    
    We have now derived a parallel to \eqref{eq:lemma1_final_multi_prev}. Using the same arguments in the proof of \lemref{lem:fit_mislabeled_multi}, we have 
    \begin{align}
      \error_{\calD}(\wh f) \le  (k-1) \left( 1- \error_{ \wt \calS_M}(\wh f) \right)  + (k-1)\sqrt{\frac{k}{\delta(k-1)}\left(\frac{1}{2m_1} +\frac{3\beta}{m+n} \right)}  \,. \label{eq:lemma1_linear_final_multi}
    \end{align}
    
    \textbf{Part-2:} We now combine the results in \lemref{lem:mislabeled_error_multi} and \lemref{lem:clear_error_multi} to obtain the final inequality in terms of quantities that can be computed from just the randomly labeled and clean data. Similar to the binary case, we obtained a polynomial concentration instead of exponential concentration. Combining \eqref{eq:lemma1_linear_final_multi} with \lemref{lem:mislabeled_error_multi} and \lemref{lem:clear_error_multi}, we have with probability at least $1-\delta$
    \begin{align*}
        \error_\calD(\widehat f) \le \error_\calS(\widehat f) &+ (k-1)\left(1 - \frac{k}{k-1} \error_{\wt\calS}(\widehat f) \right) \\ &+ \left(k + \sqrt{k} + \frac{m}{n\sqrt{k}} \right) \sqrt{\frac{\log(4/\delta)}{2m}} + \sqrt{k(k-1)} \sqrt{\frac{4}{\delta}\left(\frac{1}{m} +\frac{3\beta}{m+n} \right)}  \,. \numberthis \label{eq:gd_multi_error_proof}
    \end{align*} 
\end{proof}

\subsection{Discussion on \codref{cond:hypothesis_stability}} \label{app:discuss_cond1}
The quantity in LHS of \codref{cond:hypothesis_stability} 
measures how much the function learned by the algorithm 
(in terms of error on unseen point) will change 
when one point in the training set is removed. 
% Discussion on exponential concentration and stronger condition. 
% Notice that hypothesis stability implies error stability, i.e., \codref{cond:error_stability} \citep{bousquet2002stability}.  
% In summary, while error stability allowed us 
% to relate the average population error 
% of the leave-one-out classifiers 
% with the population error of the original classifier, 
We need hypothesis stability condition 
to control the variance of the empirical leave-one-out error to show concentration of average leave-one-error with the population error. 

Additionally, we note that while the dominating term in the RHS of \thmref{thm:linear} matches with the dominating term in ERM bound in \thmref{thm:error_ERM}, there is a polynomial concentration term 
(dependence on $1/\delta$ instead of $\log(\sqrt{1/\delta})$) 
in \thmref{thm:linear}. 
Since with hypothesis stability, 
we just bound the variance, 
the polynomial concentration is due 
to the use of Chebyshev's inequality 
instead of an exponential tail inequality
(as in \lemref{lem:fit_mislabeled}).
Recent works have highlighted that 
a slightly stronger condition than hypothesis stability 
can be used to obtain an exponential concentration 
for leave-one-out error \citep{abou2019exponential},
but we leave this for future work for now. 
% We leave 
% However, the constants 

% we also want to highlight  

\subsection{Formal statement and proof of \propref{prop:early_stop}} \label{app:formal_early_stop}

Before formally presenting the result, 
we will introduce some notation.  
By $\calL_{S}(w)$, we denote 
the objective in \eqref{eq:l2_MSE_app} with $\lambda=0$. 
Assume Singular Value Decomposition (SVD) of $\bX$
as $\sqrt{n} \bU \bS^{1/2} \bV^T$. 
Hence $\bX^T \bX = \bV \bS \bV^T$.
Consider the GD iterates defined in \eqref{eq:GD_iterates_app}. 
% 
We now derive closed form expression 
for the $t^\text{th}$ iterate of gradient descent:  
% 
\begin{align}
    w_t = w_{t-1} + \eta \cdot \bX^T (\by - \bX w_{t-1}) = (\bI - \eta \bV \bS \bV^T )w_{k-1} + \eta \bX^T \by \,.
\end{align}
Rotating by $\bV^T$, we get 
\begin{align}
    \wt w_t = (\bI - \eta\bS )\wt w_{k-1} + \eta \wt \by \label{eq:GD_recur},
\end{align}
where $\wt w_t = \bV^T w_t $ and $\wt \by = \bV^T \bX^T \by$. 
Assuming the initial point $w_0 = 0$ 
and applying the recursion in \eqref{eq:GD_recur}, we get
\begin{align}
    \wt w_t = \bS ^{-1} ( \bI - (\bI - \eta \bS)^k ) \wt \by \,, 
\end{align} 
Projecting solution back to the original space, we have 
\begin{align}
     w_t = \bV \bS ^{-1} ( \bI - (\bI - \eta \bS)^k ) \bV^T \bX^T \by \,. 
\end{align} 
% We will work with this GD solution at any iterate $t$ in the next proposition. 
Define $f_t(x) \defeq f(x;w_t)$ 
as the solution at the $t^{\text{th}}$ iterate. 
Let $\wt w_{\lambda} = \argmin_{w} \calL_\calS (w;\lambda) = (\bX^T \bX + \lambda \bI)^{-1} \bX^T \by = \bV (\bS + \lambda \bI )^{-1} \bV^T \bX^T \by $. 
% ) \,,$ for all $t=1,2,\ldots\,.$ 
and define $\wt f_\lambda(x) \defeq f(x;\wt w_\lambda)$ as the regularized solution. 
Assume $\kappa$ be the condition number 
of the population covariance matrix 
and let $s_\text{min}$ be the minimum positive 
singular value of the empirical covariance matrix. 
Our proof idea is inspired from recent work 
on relating gradient flow solution 
and regularized solution 
for regression problems \citep{ali2018continuous}. 
We will use the following lemma in the proof: 
\begin{lemma} \label{lem:ineq_soln}
    For all $x \in [0,1]$ and for all $ k \in \mathbb{N}$, 
    we have (a) $ \frac{kx}{1+kx} \le 1- (1-x)^k$ 
    and (b) $ 1- (1-x)^k \le 2 \cdot \frac{kx}{kx+1} $.
    %  where $g(c)$ is a constant dependent on $c$. For $c = 1$, $g(c) = 2.0$.   
\end{lemma}
\begin{proof}
    % [Proof of \lemref{lem:ineq_soln}]
    % Part (a) is easy. 
    Using $ (1-x)^k \le \frac{1}{1+kx}$, we have part (a). 
    For part (b), we numerically maximize 
    $\frac{ (1+kx ) (1 - (1-x)^k) }{kx}$ 
    for all $k\ge 1$ and for all $x \in [0, 1]$.  
\end{proof}

% 
% Next, 

\begin{prop}[Formal statement of \propref{prop:early_stop}] \label{prop:formal_early_stop}
Let $\lambda = \frac{1}{t\eta}$. 
For a training point $x$, we have 
\begin{align*}
    \Expt{x \sim \calS}{(f_t(x) - \wt f_\lambda(x))^2} &\le c(t,\eta) \cdot \Expt{x \sim \calS}{f_t(x)^2} \,, %\label{eq:early_stop}
\end{align*}
where $c(t, \eta) \defeq \min( 0.25, \frac{1}{s_\text{min}^2 t^2 \eta^2})$. 
Similarly for a test point, we have 
\begin{align*}
    \Expt{x \sim \calD_\calX}{(f_t(x) - \wt f_\lambda(x))^2} &\le \kappa \cdot c(t,\eta) \cdot \Expt{x \sim \calD_\calX}{f_t(x)^2} \,. %\label{eq:early_stop}
\end{align*}
\end{prop} 

\begin{proof}
    %%%%%%%%%%%%% 
    We want to analyze the expected squared difference output 
    of regularized linear regression 
    with regularization constant $\lambda = \frac{1}{\eta t}$ 
    and the gradient descent solution at the $t^\text{th}$ iterate. 
    We separately expand the algebraic expression 
    for squared difference at a training point and a test point. 
    % We start by considering the difference  
    Then the main step is to show that 
    $\left[ \bS ^{-1} ( \bI - (\bI - \eta \bS)^k )  - (\bS + \lambda \bI )^{-1}\right] \preceq c(\eta, t) \cdot \bS ^{-1} ( \bI - (\bI - \eta \bS)^k ) $.

    %%%%%%%%%%%%%
    
   \textbf{Part 1 {} {}} 
    First, we will analyze the squared difference 
    of the output at a training point 
    (for simplicity, we refer to $S \cup \wt S$ as $S$), i.e., 
    \begin{align}
        \Expt{ x \sim \calS }{\left(f_t(x) - \wt f_\lambda (x)\right)^2} &= \norm{\bX w_t - \bX \wt w_\lambda}{2}^2\\ &=   \norm{\bX \bV \bS ^{-1} ( \bI - (\bI - \eta \bS)^t ) \bV^T \bX^T \by - \bX \bV (\bS + \lambda \bI )^{-1} \bV^T \bX^T \by }{2}^2 \\
        &= \norm{\bX \bV \left(\bS ^{-1} ( \bI - (\bI - \eta \bS)^t ) - (\bS + \lambda \bI )^{-1} \right) \bV^T \bX^T \by  }{2} \\
        &=  \by^T \bV \bX \left( \underbrace{\bS ^{-1} ( \bI - (\bI - \eta \bS)^t ) - (\bS + \lambda \bI )^{-1}}_{\RN{1}} \right)^2 \bS \bV^T \bX^T \by \label{eq:train_GD_rel} \,.
        %  (\bX \bV \bS ^{-1} ( \bI - (\bI - \eta \bS)^k ) \bV^T \bX^T \by)^T \bX \bV \bS ^{-1} ( \bI - (\bI - \eta \bS)^k ) \bV^T \bX^T \by
    \end{align}
    We now separately consider term 1. 
    Substituting $\lambda = \frac{1}{t \eta}$, 
    we get
    \begin{align}
        \bS ^{-1} ( \bI - (\bI - \eta \bS)^t ) - (\bS + \lambda \bI )^{-1} &= \bS^{-1} \left( ( \bI - (\bI - \eta \bS)^t ) - (\bI + \bS^{-1} \lambda )^{-1}\right) \\
        &= \underbrace{\bS^{-1} \left( ( \bI - (\bI - \eta \bS)^t ) - (\bI + ( \bS t \eta)^{-1}  )^{-1}\right)}_{\bA} \,.
    \end{align}

    We now separately bound the diagonal entries in matrix $\bA$. 
    With $s_i$, we denote $i^{\text{th}}$ diagonal entry of $\bS$.
    Note that since $ \eta\le 1/\norm{S}{\text{op}}$, 
    for all $i$, $\eta s_i  \le 1$.  
    Consider $i^{\text{th}}$ diagonal term (which is non-zero) 
    of the diagonal matrix $\bA$, we have 
    \begin{align}
        \bA_{ii} = \frac{1}{s_i} \left(  1 - (1 - s_i \eta)^t - \frac{t \eta s_i}{1 + t \eta s_i } \right) &=  \frac{1 - (1 - s_i \eta)^t}{s_i} \left( \underbrace{ 1 - \frac{t \eta s_i}{(1 + t \eta s_i)(1 - (1 - s_i \eta)^t)}}_{\RN{2}} \right) \\ 
         &\le \frac{1}{2}\left[ \frac{1 - (1 - s_i \eta)^t}{ s_i} \right] \tag*{(Using \lemref{lem:ineq_soln} (b))} \,.
    \end{align} 
    Additionally, we can also show the following upper bound on term 2: 
    \begin{align}
         1 - \frac{t \eta s_i}{(1 + t \eta s_i)(1 - (1 - s_i \eta)^t)} &= \frac{(1 + t \eta s_i)(1 - (1 - s_i \eta)^t) - t \eta s_i }{(1 + t \eta s_i)(1 - (1 - s_i \eta)^t)} \\
         & \le  \frac{ 1 -  (1 - s_i \eta)^t - t \eta s_i (1 - s_i \eta)^t}{(1 + t \eta s_i)(1 - (1 - s_i \eta)^t)} \\
         & \le \frac{1}{t\eta s_i} \,. \tag{Using \lemref{lem:ineq_soln} (a)}
        %  &\le \frac{1}{2}\left[ \frac{1 - (1 - s_i \eta)^t}{ s_i} \right] \tag*{(Using \lemref{lem:ineq_soln})} \,.
    \end{align} 

    Combining both the upper bounds 
    on each diagonal entry $\bA_{ii}$, we have 
    \begin{align}
    \bA \preceq c_1(\eta, t) \cdot \bS^{-1} ( \bI - (\bI - \eta \bS)^t ) \,, \label{eq:upperbound_diagonal}
    \end{align}
    where $c_1(\eta, t ) = \min(0.5, \frac{1}{t s_i \eta })$. Plugging this into \eqref{eq:train_GD_rel}, we have 
    \begin{align}
        \Expt{ x \sim \calS }{\left(f_t(x) - \wt f_\lambda (x)\right)^2} &\le c(\eta, t) \cdot \by^T \bV \bX  \left( \bS^{-1} ( \bI - (\bI - \eta \bS)^t ) \right)^2 \bS \bV^T \bX^T \by \\
        &=   c(\eta, t) \cdot \by^T \bV \bX  \left( \bS^{-1} ( \bI - (\bI - \eta \bS)^t ) \right) \bS \left( \bS^{-1} ( \bI - (\bI - \eta \bS)^t ) \right) \bV^T \bX^T \by \\
        & =  c(\eta, t) \cdot \norm{\bX w_t}{2}^2 \\
        &= c(\eta, t) \cdot  \Expt{ x \sim \calS }{\left(f_t(x) \right)^2} \,,
    \end{align}
    where $c(\eta, t ) = \min(0.25, \frac{1}{t^2 s^2_i \eta^2 })$.

    \textbf{Part 2 {} {}} With $\bSigma$, 
    we denote the underlying true covariance matrix. 
    We now consider the squared difference of output at an unseen point: 
    \begin{align}
        \Expt{ x \sim \calD_{\calX} }{\left(f_t(x) - \wt f_\lambda (x)\right)^2} &= \Expt{x \sim \calD_{\calX}}{\norm{x^T w_t - x^T \wt w_\lambda}{2}} \\
        &=   \norm{x^T \bV \bS ^{-1} ( \bI - (\bI - \eta \bS)^t ) \bV^T \bX^T \by - x^T \bV (\bS + \lambda \bI )^{-1} \bV^T \bX^T \by }{2} \\
        &= \norm{x^T \bV \left(\bS ^{-1} ( \bI - (\bI - \eta \bS)^t ) - (\bS + \lambda \bI )^{-1} \right) \bV^T \bX^T \by  }{2} \\
        &= \by^T \bV \bX \left( \bS ^{-1} ( \bI - (\bI - \eta \bS)^t ) - (\bS + \lambda \bI )^{-1} \right) \bV^T \bSigma \bV \\ &\qquad \qquad \qquad \qquad \qquad \left( (\bI - (\bI - \eta \bS)^t ) - (\bS + \lambda \bI )^{-1} \right) \bV^T \bX^T \by \\
        &\le \sigma_{\text{max}} \cdot \by^T \bV \bX \left( \underbrace{\bS ^{-1} ( \bI - (\bI - \eta \bS)^t ) - (\bS + \lambda \bI )^{-1}}_{\RN{1}} \right)^2 \bV^T \bX^T \by \,, \label{eq:test_GD_rel}
        %  (\bX \bV \bS ^{-1} ( \bI - (\bI - \eta \bS)^k ) \bV^T \bX^T \by)^T \bX \bV \bS ^{-1} ( \bI - (\bI - \eta \bS)^k ) \bV^T \bX^T \by
    \end{align}
    where $\sigma_{\text{max}}$ is the maximum eigenvalue 
    of the underlying covariance matrix $\bSigma$. 
    Using the upper bound on term 1 in \eqref{eq:upperbound_diagonal}, 
    we have 
    \begin{align}
        \Expt{ x \sim \calD_{\calX} }{\left(f_t(x) - \wt f_\lambda (x)\right)^2} &\le \sigma_{\text{max}} \cdot c(\eta, t) \cdot \by^T \bV \bX  \left( \bS^{-1} ( \bI - (\bI - \eta \bS)^t ) \right)^2 \bV^T \bX^T \by \\
        &=   \kappa \cdot c(\eta, t) \cdot \sigma_{\text{min}}\cdot \norm{\bV \left( \bS^{-1} ( \bI - (\bI - \eta \bS)^t ) \right) \bV^T \bX^T \by}{2}^2 \\
        &\le \kappa \cdot c(\eta, t) \cdot \left[ \bV \left( \bS^{-1} ( \bI - (\bI - \eta \bS)^t ) \right) \bV^T \bX^T \right]^T \bSigma \\
        &\qquad \qquad \qquad \qquad \qquad \left[ \bV \left( \bS^{-1} ( \bI - (\bI - \eta \bS)^t ) \right) \bV^T \bX^T \right] \by \\
        & = \kappa \cdot c(\eta, t) \cdot \Expt{x \sim \calD_{\calX}}{\norm{x^T w_t}{2}} \,.
    \end{align}
% 
% 
    % Since $ \eta\le 1/\norm{S}{\text{op}}$, invoking \lemref{lem:ineq_soln} to upper bound term 1 with
\end{proof}

\subsection{Extension to deep learning} \label{appsubsec:ext_DL}
Under \asmpref{appsubsec:justifying_assumption1}, we present the formal result parallel to \thmref{thm:multiclass_ERM}. 
\begin{theorem} \label{thm:multiclass_ERM_algoA}
    Consider a multiclass classification problem 
    with $k$ classes. Under \asmpref{asmp:deep_models}, 
    for any $\delta >0$, with probability at least $1-\delta$,
    we have
    \vspace{-10pt}
    \begin{align*}
        \error_\calD(\widehat f)  \le \error_\calS(\widehat f) + (k-1) \left(1 - \tfrac{k}{k-1} \error_{\wt\calS}(\widehat f)\right) + c\sqrt{\frac{\log(\frac{4}{\delta})}{2m}} \,,\numberthis \label{eq:multiclass_ERM_deep}
    % \vspace{-20pt}
    \end{align*}
    for some constant $c \le ((c+1) k+\sqrt{k} + \frac{m}{n\sqrt{k}})$.
\end{theorem}

The proof follows exactly as in step (i) to (iii) in \thmref{thm:multiclass_ERM}.  

\subsection{Justifying~\asmpref{asmp:deep_models}} \label{appsubsec:justifying_assumption1}

Motivated by the analysis on linear models, we now discuss alternate (and weaker) conditions that imply \asmpref{asmp:deep_models}. 
We need hypothesis stability (\codref{cond:hypothesis_stability}) and the following assumption relating training error and leave-one-error: 

\begin{assumption} \label{asmp:loo_error}
Let $\wh f$ be a model obtained by training with algorithm $\calA$ on a mixture of clean $S$ and randomly labeled data $\wt S$. Then we assume we have 
\begin{align*}
    \error_{\wt \calS_M} (\wh f) \le  \error_{\text{LOO} (\wt S_M)} \,, 
\end{align*}
for all $(x_i, y_i) \in  \wt S_M$ where $\wh f_{(i)} \defeq f(\calA, S \cup {{}\wt S_M}_{(i)})$ and  $\error_{\text{LOO} (\wt S_M)} \defeq  \frac{\sum_{(x_i, y_i) \in \wt S_M} \error(\ff{i}(x_i), y_i ) }{\abs{\wt \calS_M}}$.  
\end{assumption}

% we assume this to extend our result (parallel to \thmref{thm:multi_linear}) for deep models. 
Intuitively, this assumption states that the error on a (mislabeled) datum $(x,y)$ included in the training set is less than the error on that datum $(x,y)$ obtained by a model trained on the training set $S - \{(x,y)\}$. We proved this for linear models trained with GD in the proof of \thmref{thm:multi_linear}. 
% 
\codref{cond:hypothesis_stability} with $\beta = \calO(1)$ and \asmpref{asmp:loo_error} together with \lemref{lem:stability_error} implies \asmpref{asmp:deep_models} with a polynomial residual term (instead of logarithmic in $1/\delta$): 
\begin{align}
     \error_{\calS_M} (\wh f) \le  \error_{\calDm}(\wh f)   + \sqrt{\frac{1}{\delta}\left(\frac{1}{m} +\frac{3\beta}{m+n} \right)} \,.
\end{align}
% Note that this  

\newpage 
\section{Additional experiments and details}\label{app:exp}
\newcommand\tab[1][1cm]{\hspace*{#1}}

\subsection{Datasets} \label{sec:app_dataset}

\textbf{Toy Dataset {} {}} Assume fixed constants $\mu$ and $\sigma$. For a given label $y$, we simulate features $x$ in our toy classification setup as follows: 
\begin{align*}
    x \defeq \texttt{concat} \left[ x_1, x_2\right] \quad \text{where} \quad  x_1 \sim  \calN( y \cdot \mu, \sigma^2 I_{d \times d}) \ \  \text{and} \ \  x_1 \sim  \calN( 0, \sigma^2 I_{d \times d}) \,.
\end{align*}  
% where $y$ is the true label and $x$ is the corresponding feature vector. 
In experiements throughout the paper, we fix dimention $d=100$, $\mu = 1.0 $, and $\sigma = \sqrt{d}$. Intuitively, $x_1$ carries the information about the underlying label and $x_2$ is additional noise independent of the underlying label. 

\textbf{CV datasets {} {}} We use MNIST~\citep{lecun1998mnist} and CIFAR10~\cite{krizhevsky2009learning}. 
% For binary tasks, 
We produce a binary variant from the multiclass classification problem by mapping classes $\{0,1,2,3,4\}$ to label $1$ and $\{ 5,6,7,8,9\}$ to label $-1$. For CIFAR dataset, we also use the standard data augementation of random crop and horizontal flip. PyTorch code is as follows: 

\texttt{(transforms.RandomCrop(32, padding=4),\\
\tab transforms.RandomHorizontalFlip())}

\textbf{NLP dataset {} {}} We use IMDb Sentiment analysis~\citep{maas2011learning} corpus.  

\subsection{Architecture Details} 

All experiments were run on NVIDIA GeForce RTX 2080 Ti GPUs. We used PyTorch~\citep{NEURIPS2019a9015} and Keras with Tensorflow~\citep{abadi2016tensorflow} backend for experiments. 
% , ELMo embeddings~\citep{Peters:2018}, and Hugging Face Transformers~\citep{wolf-etal-2020-transformers}. 

\textbf{Linear model {} {}} For the toy dataset, we simulate a linear model with scalar output and the same number of parameters as the number of dimensions.   

\textbf{Wide nets {} {}} To simulate the NTK regime, we experiment with $2-$layered wide nets. The PyTorch code for 2-layer wide MLP is as follows: 


\texttt{ nn.Sequential( \\
\tab     nn.Flatten(),\\
\tab    nn.Linear(input\_dims, 200000, bias=True),\\
\tab    nn.ReLU(),\\
\tab    nn.Linear(200000, 1, bias=True)\\
\tab     )}


We experiment both (i) with the second layer fixed at random initialization; (ii)  and updating both layers' weights.     

\textbf{Deep nets for CV tasks {} {}} We consider a 4-layered MLP. The PyTorch code for 4-layer MLP is as follows: 

\texttt{ nn.Sequential(nn.Flatten(), \\
\tab        nn.Linear(input\_dim, 5000, bias=True),\\
\tab        nn.ReLU(),\\
\tab        nn.Linear(5000, 5000, bias=True),\\
\tab        nn.ReLU(),\\
\tab        nn.Linear(5000, 5000, bias=True),\\
\tab        nn.ReLU(),\\
% \tab        nn.Linear(5000, 5000, bias=True),\\
% \tab        nn.ReLU(),\\
\tab        nn.Linear(1024, num\_label, bias=True)\\
\tab        )}

For MNIST, we use $1000$ nodes instead of $5000$ nodes in the hidden layer. 
% 
We also experiment with convolutional nets. In particular, we use ResNet18 \citep{he2016deep}. Implementation adapted from:  \url{https://github.com/kuangliu/pytorch-cifar.git}. 

\textbf{Deep nets for NLP {} {}} We use a simple LSTM model with embeddings intialized with ELMo embeddings~\citep{Peters:2018}. Code adapted from: \url{https://github.com/kamujun/elmo_experiments/blob/master/elmo_experiment/notebooks/elmo_text_classification_on_imdb.ipynb} 

We also evaluate our bounds with a BERT model. In particular, we fine-tune an off-the-shelf uncased BERT model~\citep{devlin2018bert}. Code adapted from Hugging Face Transformers~\citep{wolf-etal-2020-transformers}: \url{https://huggingface.co/transformers/v3.1.0/custom_datasets.html}. 


\subsection{Additonal experiments}

\textbf{Results with SGD on underparameterized linear models {} {}} 

\begin{figure*}[h]
    \centering 
    % \vspace{-15pt}
    % \includegraphics[width=0.9\linewidth]{example-image-a}
    \includegraphics[width=0.3\linewidth]{figures/lowdim-Gaussian-SGD.pdf}
    % \includegraphics[width=0.9\linewidth]{figures/{CIFAR10_rn=0.1_lr=0.2_wd=0.005}.png}
    \vspace{-5pt}
    \caption{ 
    % Predicted lower bound 
    % on different
    We plot the accuracy and corresponding bound 
    (RHS in \eqref{eq:erm}) at $\delta = 0.1$
    for toy binary classification task. 
    Results aggregated over $3$ seeds. 
    % i.e., $1-\error$ where $\error$ is the term in the RHS of \eqref{eq:erm}
    Accuracy vs fraction of unlabeled data (w.r.t clean data) 
    in the toy setup with a linear model trained with SGD. Results parallel to \figref{fig:error_binary}(a) with SGD.  }
    \label{fig:error_binary_linear}
    \vspace{-5pt}
\end{figure*}

\textbf{Results with wide nets on binary MNIST {} {}}

\begin{figure*}[h]
    \centering 
    % \vspace{-15pt}
    % \includegraphics[width=0.9\linewidth]{example-image-a}
    \subfigure[GD with MSE loss]{\includegraphics[width=0.3\linewidth]{figures/MNIST-GD_MSE.pdf}} \hfil
    \subfigure[SGD with CE loss]{\includegraphics[width=0.3\linewidth]{figures/MNIST-SGD_CE.pdf}}
    \subfigure[SGD with MSE loss]{\includegraphics[width=0.3\linewidth]{figures/MNIST-SGD_MSE-first-layer.pdf}}
    % \includegraphics[width=0.9\linewidth]{figures/{CIFAR10_rn=0.1_lr=0.2_wd=0.005}.png}
    \vspace{-5pt}
    \caption{ 
    % Predicted lower bound 
    % on different
    We plot the accuracy and corresponding bound 
    (RHS in \eqref{eq:erm}) at $\delta = 0.1$ 
    for binary MNIST classification. 
    Results aggregated over $3$ seeds. 
    % i.e., $1-\error$ where $\error$ is the term in the RHS of \eqref{eq:erm}
    Accuracy vs fraction of unlabeled data 
    for a 2-layer wide network on binary MNIST with both the layers training in (a,b) and only first layer training in (c). 
    Results parallel to \figref{fig:error_binary}(b) .  }
    \label{fig:error_binary_MNIST}
    \vspace{-5pt}
\end{figure*}

% \begin{figure*}[h]
%     \centering 
%     % \vspace{-15pt}
%     % \includegraphics[width=0.9\linewidth]{example-image-a}
%     \subfigure[GD with MSE loss]{\includegraphics[width=0.3\linewidth]{figures/MNIST.pdf}} \hfil
    
%     \subfigure[SGD with CE loss]{\includegraphics[width=0.3\linewidth]{figures/MNIST.pdf}}
%     % \includegraphics[width=0.9\linewidth]{figures/{CIFAR10_rn=0.1_lr=0.2_wd=0.005}.png}
%     \vspace{-5pt}
%     \caption{ 
%     % Predicted lower bound 
%     % on different
%     We plot the accuracy and corresponding bound 
%     (RHS in \eqref{eq:erm}) at $\delta = 0.1$
%     for binary MNIST classification. 
%     Results aggregated over $3$ seeds. 
%     % i.e., $1-\error$ where $\error$ is the term in the RHS of \eqref{eq:erm}
%     Accuracy vs fraction of unlabeled data 
%     for a 2-layer wide network on binary MNIST with just the first layer training. 
%     Results parallel to \figref{fig:error_binary}(b) with only the first layer training.  }
%     \label{fig:error_binary_MNIST}
%     \vspace{-5pt}
% \end{figure*}

\textbf{Results on CIFAR 10 and MNIST {} {}} 
% 
We plot epoch wise error curve for results in \tabref{table:multiclass}(\figref{fig:error_epoch_CIFAR10} and \figref{fig:error_epoch_MNIST}). We observe the same trend as in \figref{fig:error_CIFAR10}. Additionally, we plot an \emph{oracle bound} obtained by tracking the error on mislabeled data which nevertheless were predicted as true label. To obtain an exact emprical value of the oracle bound, we need underlying true labels for the randomly labeled data. 
% Note that our bound in \thmref{thm:multiclass_ERM}, lower bounds the accuracy as predicted by the oracle bound. 
While with just access to extra unlabeled data we cannot calculate oracle bound, we note that the oracle bound is very tight and never violated in practice underscoring an importamt aspect of generalization in multiclass problems. This highlight that even a stronger conjecture may hold in multiclass classification, i.e., error on mislabeled data (where nevertheless true label was predicted) lower bounds the population error on the distribution of mislabeled data and hence, the error on (a specific) mislabeled portion predicts the population accuracy on clean data. 
% 
On the other hand, the dominating term of in \thmref{thm:multiclass_ERM} is loose when compared with the oracle bound. The main reason, we believe is the pessimistic upper bound in \eqref{eq:lemma1_final_multi_prev} in the proof of \lemref{lem:fit_mislabeled_multi}. We leave an investigation on this gap for future. 
% of fit 

% However, oracle bound highlights two . One,  



\begin{figure}[h]
    \centering 
    % \vspace{-15pt}
    % \includegraphics[width=0.9\linewidth]{example-image-a}
    \subfigure[MLP]{\includegraphics[width=0.3\linewidth]{figures/CIFAR10-FNN.pdf}} \hfil
    \subfigure[ResNet]{\includegraphics[width=0.3\linewidth]{figures/CIFAR10-Resnet.pdf}}
    % \includegraphics[width=0.9\linewidth]{figures/{CIFAR10_rn=0.1_lr=0.2_wd=0.005}.png}
    % \vspace{-10pt}
    \caption{ Per epoch curves for CIFAR10 corresponding results in \tabref{table:multiclass}. As before, we just plot the dominating term in the RHS of \eqref{eq:multiclass_ERM} as predicted bound. Additionally, we also plot the predicted lower bound by the error on mislabeled data which nevertheless were predicted as true label. We refer to this as ``Oracle bound''. See text for more details. 
    % 
    % except for the stopping point. 
    % The bound predicted by RATT (RHS in \eqref{eq:multiclass_ERM}) is vacuous. 
    }\label{fig:error_epoch_CIFAR10}
    % \vspace{-15pt}
\end{figure}


\begin{figure}[h]
    \centering 
    % \vspace{-15pt}
    % \includegraphics[width=0.9\linewidth]{example-image-a}
    \subfigure[MLP]{\includegraphics[width=0.3\linewidth]{figures/MNIST-FNN.pdf}} \hfil
    \subfigure[ResNet]{\includegraphics[width=0.3\linewidth]{figures/MNIST-Resnet.pdf}}
    % \includegraphics[width=0.9\linewidth]{figures/{CIFAR10_rn=0.1_lr=0.2_wd=0.005}.png}
    % \vspace{-10pt}
    \caption{ Per epoch curves for MNIST corresponding results in \tabref{table:multiclass}. As before, we just plot the dominating term in the RHS of \eqref{eq:multiclass_ERM} as predicted bound. Additionally, we also plot the predicted lower bound by the error on mislabeled data which nevertheless were predicted as true label. We refer to this as ``Oracle bound''. See text for more details. 
    % 
    % except for the stopping point. 
    % The bound predicted by RATT (RHS in \eqref{eq:multiclass_ERM}) is vacuous. 
    }\label{fig:error_epoch_MNIST}
    % \vspace{-15pt}
\end{figure}

\textbf{Results on CIFAR 100 {} {}} 
% 
On CIFAR100, our bound in \eqref{eq:multiclass_ERM} yields vacous bounds. However, the oracle bound as explained above yields tight guarantees in the initial phase of the learning (i.e., when learning rate is less than $0.1$) (\figref{fig:error_CIFAR100}).  

\begin{figure}[h]
    \centering 
    % \vspace{-15pt}
    % \includegraphics[width=0.9\linewidth]{example-image-a}
    \includegraphics[width=0.3\linewidth]{figures/CIFAR100-Resnet.pdf}
    % \includegraphics[width=0.9\linewidth]{figures/{CIFAR10_rn=0.1_lr=0.2_wd=0.005}.png}
    % \vspace{-10pt}
    \caption{ Predicted lower bound by the error on mislabeled data which nevertheless were predicted as true label with ResNet18 on CIFAR100. We refer to this as ``Oracle bound''. See text for more details. 
    % 
    % except for the stopping point. 
    The bound predicted by RATT (RHS in \eqref{eq:multiclass_ERM}) is vacuous. 
    }\label{fig:error_CIFAR100}
    % \vspace{-15pt}
\end{figure}


% \paragraph{Experiments on CIFAR100} 


% \subsection{Model Selection using RATT}


\subsection{Hyperparameter Details}


\textbf{\figref{fig:error_CIFAR10} {} {}} We use clean training dataset of size $40,000$. We fix the amount of unlabeled data at $20\%$ of the clean size, i.e. we include additional $8,000$ points with randomly assigned labels. We use test set of $10,000$ points. For both MLP and ResNet, we use SGD with an initial learning rate of $0.1$ and momentum $0.9$. We fix the weight decay parameter at $5\times 10^{-4}$. After $100$ epochs, we decay the learning rate to $0.01$. We use SGD batch size of $100$. 

\textbf{\figref{fig:error_binary} (a) {} {}} We obtain a toy dataset according to the process described in \secref{sec:app_dataset}. We fix $d=100$ and create a dataset of $50,000$ points with balanced classes. Moreover, we sample additional covariates with the same procedure to create randomly labeled dataset. For both SGD and GD training, we use a fixed learning rate $0.1$.    

\textbf{\figref{fig:error_binary} (b) {} {}} Similar to binary CIFAR, we use clean training dataset of size $40,000$ and fix the amount of unlabeled data at $20\%$ of the clean dataset size. To train wide nets, we use a fixed learning of $0.001$ with GD and SGD. We decide the weight decay parameter and the early stopping point that maximizes our generalization bound (i.e. without peeking at unseen data ).  We use SGD batch size of $100$. 

\textbf{\figref{fig:error_binary} (c) {} {}} With IMDb dataset, we use a clean dataset of size $20,000$ and as before, fix the amount of unlabeled data at $20\%$ of the clean data. To train ELMo model, we use Adam optimizer with a fixed learning rate $0.01$ and weight decay $10^{-6}$ to minimize cross entropy loss. We train with batch size $32$ for 3 epochs. To fine-tune BERT model, we use Adam optimizer with learning rate $5\times 10^{-5}$ to minimize cross entropy loss. We train with a batch size of $16$ for 1 epoch.    

\textbf{\tabref{table:multiclass} {} {}} For multiclass datasets, we train both MLP and ResNet with the same hyperparameters as described before. We sample a clean training dataset of size $40,000$ and fix the amount of unlabeled data at $20\%$ of the clean size. We use SGD with an initial learning rate of $0.1$ and momentum $0.9$. We fix the weight decay parameter at $5\times 10^{-4}$. After $30$ epochs for ResNet and after $50$ epochs for MLP, we decay the learning rate to $0.01$.  We use SGD with batch size $100$. 
For \figref{fig:error_CIFAR100}, we use the same hyperparameters as 
CIFAR10 training, except we now decay learning rate after $100$ epochs. 


In all experiments, to identify the best possible accuracy on just the clean data, we use the exact same set of hyperparamters except the stopping point. We choose a stopping point that maximizes test performance. 

\subsection{Summary of experiments }

\begin{center}
    \begin{table}[H] 
        \centering
        \begin{tabular}{|c|c|c|c|} 
        \hline
        Classification type & Model category & Model & Dataset  \\ [0.5ex] 
        \hline
        \hline
        \multirow{10}{*}{Binary} & Low dimensional & Linear model & Toy Gaussain dataset  \\
                        \cline{2-4}
                         & Overparameterized 
                        %  & Linear model & Toy Gaussain dataset \\
                        %  \cline{3-4}
                        %  & & 2-layer wide net& Toy Gaussain dataset \\
                        %  \cline{3-4}
                         & \multirow{2}{*}{2-layer wide net} & \multirow{2}{*}{Binary MNIST} \\
                         & linear nets & &  
                         \\
                         \cline{2-4}                 
                         & \multirow{6}{*}{Deep nets} & \multirow{2}{*}{MLP} & Binary MNIST \\
                         \cline{4-4}
                         & &  & Binary CIFAR \\
                         \cline{3-4}
                         &  & \multirow{2}{*}{ResNet} & Binary MNIST \\
                         \cline{4-4}
                         & &  & Binary CIFAR \\
                         \cline{3-4}
                         &  & ELMo-LSTM model & IMDb Sentiment Analysis \\
                         \cline{3-4}
                         & & BERT pre-trained model & IMDb Sentiment Analysis \\
        \hline
        \multirow{5}{*}{Multiclass} & \multirow{5}{*}{Deep nets} & \multirow{2}{*}{MLP} & MNIST \\
                        \cline{4-4} 
                        & & & CIFAR10 \\                   
                        \cline{3-4}
                         &   & \multirow{3}{*}{ResNet} & MNIST \\
                         \cline{4-4}
                         &   & & CIFAR10 \\
                         \cline{4-4}
                         &   & & CIFAR100 \\
        \hline
        \end{tabular}
        % \caption{Summary of experiments performed} \label{table:experiments}
    \end{table}    
    % \footnotetext[6]{We use both MSE loss and cross-entropy loss.}
    % \footnotetext[6]{We try 2 variants: one with a fixed first layer and the other with both layers trainable.}
\end{center}

\newpage
\section{Proof of \lemref{lem:stability_error}} \label{app:proof_lem_error}

\begin{proof}[Proof of \lemref{lem:stability_error}]
    Recall, we have a training set $S \cup \wt S_C$. We defined leave-one-out error on mislabeled points as $$\error_{\text{LOO}(\wt S_M) } = \frac{\sum_{(x_i, y_i) \in \wt S_M} \error( f_{(i)}( x_i), y_i)}{ \abs{\wt S_M }} \,, $$
    where $f_{(i)} \defeq f(\calA, (S \cup \wt S)_{(i)})$. Define $S^\prime \defeq S \cup \wt S$. Assume $(x,y)$ and $(x^\prime,y^\prime)$ as i.i.d. samples from ${\calDm}$. 
    Using Lemma 25 in \citet{bousquet2002stability}, we have
    \begin{align*}
        \Expo{ \left( \error_{\calDm}(\wh f) -\error_{\text{LOO}(\wt S_M) } \right)^2 } \le & \Expt{ S^\prime, (x,y), (x^\prime,y^\prime) }{ \error(\wh f(x), y ) \error(\wh f(x^\prime), y^\prime )} - 2 \Expt{ S^\prime, (x,y) }{ \error(\wh f(x), y ) \error(f_{(i)}(x_i), y_i )} \\
        & + \frac{m_1-1}{m_1}\Expt{ S^\prime }{  \error(f_{(i)}(x_i), y_i )  \error(f_{(j)}(x_j), y_j )} + \frac{1}{m_1} \Expt{ S^\prime }{  \error(f_{(i)}(x_i), y_i ) } \,. \numberthis \label{eq:main_reln}
    \end{align*}
    We can rewrite the equation above as : 
    \begin{align*}
        \Expo{ \left( \error_{\calDm}(\wh f) -\error_{\text{LOO}(\wt S_M) } \right)^2 } \le &  \, \underbrace{\Expt{ S^\prime, (x,y), (x^\prime,y^\prime) }{ \error(\wh f(x), y ) \error(\wh f(x^\prime), y^\prime ) - \error(\wh f(x), y ) \error(f_{(i)}(x_i), y_i )}}_{\RN{1}} \\
        & + \underbrace{\Expt{ S^\prime }{  \error(f_{(i)}(x_i), y_i )  \error(f_{(j)}(x_j), y_j ) -  \error(\wh f(x), y ) \error(f_{(i)}(x_i), y_i )}}_{\RN{2}} \\ &+ \underbrace{\frac{1}{m_1} \Expt{ S^\prime }{  \error(f_{(i)}(x_i), y_i ) - \error(f_{(i)}(x_i), y_i )  \error(f_{(j)}(x_j), y_j ) }}_{\RN{3}} \,. \numberthis \label{eq:main_reln2}
    \end{align*}
    
    We will now bound term $\RN{3}$.  Using Cauchy-Schwarz's inequality, we have
    
    \begin{align}
        \Expt{ S^\prime }{  \error(f_{(i)}(x_i), y_i ) - \error(f_{(i)}(x_i), y_i )  \error(f_{(j)}(x_j), y_j ) }^2 &\le  \Expt{ S^\prime }{  \error(f_{(i)}(x_i), y_i ) }^2 \Expt{S^\prime}{1 -   \error(f_{(j)}(x_j), y_j ) }^2 \\
        &\le \frac{1}{4} \,.\label{eq:term1_lem12}
    \end{align}
    
    Note that since $(x_i,y_i)$, $(x_j ,y_j )$, $(x,y)$, and $(x^\prime, y^\prime)$ are all from same distribution $\calDm$, we directly incorporate the bounds on term $\RN{1}$ and $\RN{2}$ from the proof of Lemma 9 in \citet{bousquet2002stability}. Combining that with \eqref{eq:term1_lem12} and our definition of hypothesis stability in \codref{cond:hypothesis_stability}, we have the required claim. 
    
    
    % We now re-write term $\RN{1}$ as
    % \begin{align*}
    %         &\Expt{S^\prime, (x,y), (x^\prime,y^\prime) }{ \error(\wh f(x), y ) \error(\wh f(x^\prime), y^\prime ) - \error(\wh f(x), y ) \error(f_{(i)}(x_i), y_i )} \\ & \qquad = \Expt{ S^\prime, (x,y), (x^\prime,y^\prime) }{ \error(\wh f(x), y ) \error(\wh f  (x^\prime), y^\prime ) - \error(\wh f ^\prime(x), y ) \error(f_{(i)}(x^\prime), y^\prime )} \tag{Exchanging $(x_i, y_i)$ with $(x^\prime, y^\prime)$ in the second term} \\
    %         & \qquad = \Expt{ S^\prime, (x,y), (x^\prime,y^\prime) }{  \left(\error(\wh f(x), y )-  \error(f_{(i)}(x), y ) \right) \error(\wh f  (x^\prime), y^\prime )  } \\
    %         & \qquad  + \Expt{ S^\prime, (x,y), (x^\prime,y^\prime) }{  \left(\error(f_{(i)}(x), y ) -\error(\wh f ^\prime(x), y ) \right) \error(\wh f  (x^\prime), y^\prime )}  \\
    %         & \qquad +\Expt{ S^\prime, (x,y), (x^\prime,y^\prime) }{  \left( \error(\wh f  (x^\prime), y^\prime ) -  \error(f_{(i)}(x^\prime), y^\prime ) \right) \error(\wh f ^\prime(x), y ) }  \,, \numberthis \label{eq:term1_final}
    % \end{align*}
    % where $\wh f^\prime$ is the classifier obtained by training on $ S^\prime_{(i)} \cup \{ (x^\prime, y^\prime) \} $. Similarly we can re-write term $\RN{2}$ as 
    % \begin{align*}
    %     & \Expt{ S^\prime }{  \error(f_{(i)}(x_i), y_i )  \error(f_{(j)}(x_j), y_j ) -  \error(\wh f(x), y ) \error(f_{(i)}(x_i), y_i )} \\
    %     &\quad  = \Expt{ S^\prime, (x,y), (x^\prime,y^\prime)}{  \error(f^{\prime\prime}_{(i)}(x), y )  \error(f_{(j)}^{\prime}(x^\prime), y^\prime ) -  \error(\wh f(x), y ) \error(f_{(i)}(x_i), y_i )} \tag{Exchanging $(x_i, y_i)$ with $(x, y)$ and $(x_j, y_j)$ with $(x^\prime, y^\prime)$ in the first term}\\
    %     &\quad = \Expt{ S^\prime, (x,y), (x^\prime,y^\prime)}{  \error(f^{\prime\prime}_{(j)}(x), y )  \error(f_{(i)}^{\prime}(x^\prime), y^\prime ) -  \error(\wh f^\prime (x), y ) \error(f^\prime_{(j)}(x^\prime), y^\prime )} \tag{Exchanging $(x_i, y_i)$ and $(x_j, y_j)$ and then replacing $(x_j, y_j)$ with $(x^\prime, y^\prime)$ in the second term} \\
    %     & \quad = \Expt{ S^\prime, (x,y), (x^\prime,y^\prime) }{  \left( \error(f_{(i)}^{\prime}(x^\prime), y^\prime )   -  \error(\wh f^{\prime\prime}  (x^\prime), y^\prime ) \right)  \error(f^{\prime\prime}_{(j)}(x), y )   } \\
    %     & \quad  + \Expt{ S^\prime, (x,y), (x^\prime,y^\prime) }{  \left( \error(f^{\prime\prime}_{(j)}(x), y )  -\error(\wh f ^\prime(x), y ) \right) \error(\wh f^{\prime\prime}  (x^\prime), y^\prime )  }  \\
    %     & \quad+ \Expt{ S^\prime, (x,y), (x^\prime,y^\prime) }{  \left( \error(\wh f^{\prime\prime}  (x^\prime), y^\prime )  -  \error(f^\prime_{(j)}(x^\prime), y^\prime ) \right)  \error(\wh f^\prime (x), y ) }   \\
    %     & \quad = \Expt{ S^\prime, (x,y), (x^\prime,y^\prime) }{  \left( \error(f_{(i)}^{\prime}(x^\prime), y^\prime )   -  \error(\wh f (x^\prime), y^\prime ) \right)  \error(f_{(i)}(x_j), y_j )   } \\
    %     & \quad  + \Expt{ S^\prime, (x,y), (x^\prime,y^\prime) }{  \left( \error(f^{\prime\prime}_{(j)}(x), y )  -\error(\wh f (x), y ) \right) \error(\wh f^{\prime\prime}  (x_j), y_j )  }  \\
    %     & \quad+ \Expt{ S^\prime, (x,y), (x^\prime,y^\prime) }{  \left( \error(\wh f^{\prime\prime}  (x^\prime), y^\prime )  -  \error(f^\prime_{(j)}(x^\prime), y^\prime ) \right)  \error(\wh f^\prime (x^\prime), y^\prime ) }  \,, \numberthis \label{eq:term2_final}
    % \end{align*}
    % where $f^{\prime\prime}_{(j)}$ is trained on $S^\prime_{(j,i)} \cup {(x,y)}$, $f^{\prime}_{(i)}$ is trained on $S^\prime_{(j,i)} \cup {(x^\prime,y^\prime)}$, and $\wh f^{\prime\prime} $ is trained on $S^\prime_{(j)} \cup {(x,y)}$. Note in the last line we replaced $(x,y)$ by $(x_j, y_j)$ in the first term, replaced $(x^\prime,y^\prime)$ by $(x_j, y_j)$ in the second term and exchanged $(x_i,y_i)$ with $(x_j,y_j)$ and also $(x,y)$ and $(x^\prime, y^\prime)$
    
    
\end{proof}


% 
% 16th Century Version Control 
% 

% \onecolumn

% \section*{Supplementary Material}
% We will be using the following standard results
% on exponential concentration of random variables 
% all throughout the discussion:

% \begin{lemma}[Hoeffding's inequality for independent RVs~\citep{hoeffding1994probability}] Let $Z_1, Z_2, \ldots, Z_n$ be independent bounded random variables with $Z_i \in [a,b]$ for all $i$, then 
%     \begin{align*}
%         \prob\left( \frac{1}{n} \sum_{i=1}^n (Z_i - \Expo{Z_i}) \ge t \right) \le \exp{\left( -\frac{2nt^2}{(b-a)^2} \right) }
%     \end{align*} 
%     and 
%     \begin{align*}
%         \prob\left( \frac{1}{n} \sum_{i=1}^n (Z_i - \Expo{Z_i}) \le -t \right) \le \exp{\left( -\frac{2nt^2}{(b-a)^2} \right) }
%     \end{align*} 
%     for all $t \ge 0$. 
% \end{lemma}

% \begin{lemma}[Hoeffding's inequality for sampling with replacement~\citep{hoeffding1994probability}] \label{lem:hoeffding_sampling} Let $\calZ = (Z_1, Z_2, \ldots, Z_N)$ be a finite population of $N$ points with $Z_i \in [a.b]$ for all $i$. Let $X_1, X_2, \ldots X_n$ be a random sample drawn without replacement from $\calZ$. Then for all $t \ge 0$, we have 
%     \begin{align*}
%         \prob\left( \frac{1}{n} \sum_{i=1}^n (X_i - \mu ) \ge t \right) \le \exp{\left( -\frac{2nt^2}{(b-a)^2} \right) }
%     \end{align*} 
%     and 
%     \begin{align*}
%         \prob\left( \frac{1}{n} \sum_{i=1}^n (X_i - \mu ) \le -t \right) \le \exp{\left( -\frac{2nt^2}{(b-a)^2} \right) } \,,
%     \end{align*} 
%     where $\mu = \frac{1}{N} \sum_{i=1}^{N} Z_i$. 
% \end{lemma}

% We now discuss one condition that generalizes the exponential concentration to dependent random variables.
% \begin{condition}[Bounded difference inequality] \label{cond:BDC} Let $\calZ$ be some set and $\phi: \calZ^n \to \Real$. We say that $\phi$ satisfies the bounded difference assumption if 
% there exists $c_1, c_2, \ldots c_n \ge 0$ s.t. for all $i$, we have 
% \begin{align*}
%     \sup_{Z_1,Z_2, \ldots,Z_n, Z_i^\prime in \calZ^{n+1} } \abs{\phi (Z_1, \ldots, Z_i, \ldots, Z_n ) - \phi (Z_1, \ldots, Z_i^\prime, \ldots, Z_n ) } \le c_i \,.
% \end{align*} 
% \end{condition}

% \begin{lemma}[McDiarmid’s inequality~\citep{mcdiarmid1989}] \label{lem:McDiarmid} Let $Z_1, Z_2, \ldots, Z_n$ be independent random variables on set $\calZ$ and $\phi : \calZ^n \to \Real$ satisfy bounded difference assumption (\codref{cond:BDC}). Then for all $t>0$, we have 
%     \begin{align*}
%         \prob\left( \phi(Z_1, Z_2, \ldots, Z_n) - \Expo{\phi(Z_1, Z_2, \ldots, Z_n)} \ge t \right) \le \exp{\left( -\frac{2t^2}{\sum_{i=1}^n c_i^2} \right) } 
%     \end{align*} 
%     and 
%     \begin{align*}
%         \prob\left( \phi(Z_1, Z_2, \ldots, Z_n) - \Expo{\phi(Z_1, Z_2, \ldots, Z_n)} \le -t \right) \le \exp{\left( -\frac{2t^2}{\sum_{i=1}^n c_i^2} \right) } \,
%     \end{align*} 
% \end{lemma}


% \section{Proofs from \secref{sec:ERM_training}}\label{app:proof_erm}

% \textbf{Additional notation {} {}} Let $m_1$ be the number of mislabeled points ($\wt S_M$) and $m_2$ be the number of correctly labeled points ($\wt S_C$). Note $m_1 + m_2 = m$. 


% \subsection{Proof of \thmref{thm:error_ERM}}


% \begin{proof}[Proof of \lemref{lem:fit_mislabeled}] 
%     The main idea of our proof is to regard 
%     the clean portion of the data 
%     ($S \cup \wt S_C$) as fixed.   
%     Then, there exists a classifier $f^*$ 
%     that is optimal over draws 
%     of the mislabeled data $\wt S_M$. 
% % 
%     % 
%     Formally, 
%     \begin{align}
%     f^* \defeq \argmin_{f \in \calF} \error_{\widecheck {\calD}} (f) \,, \label{eq:modified_ERM}
%     \end{align}
%     where $$\widecheck \calD = \frac{n}{m+n} \calS + \frac{m_1}{m+n} \wt \calS_C  + \frac{m_2}{m+n}\calDm \,.$$ That is, $\widecheck \calD$ a combination of 
%     the \emph{empirical distribution} 
%     over correctly labeled data $S \cup \wt S_C$
%     % in $S\cup \wt S$ 
%     and the (population) distribution 
%     over mislabeled data $\calDm$.
%     Recall that 
%     \begin{align}
%     \wh f \defeq \argmin_{f \in \calF} \error_{\calS \cup \wt S} (f) \,. \label{eq:orig_ERM}
%     \end{align}
%     % 
%     % 
%     Since, $\widehat f$ minimizes 0-1 error 
%     on $S \cup \wt S$, using ERM optimality on \eqref{eq:orig_ERM},  
%     we have 
%     \begin{align}
%         \error_{\calS \cup \wt \calS}(\widehat f) \le \error_{
%             \calS \cup \wt \calS}(f^*) \,.    \label{eq:step1}
%     \end{align}
%     Moreover, since $f^*$ is independent of $\wt S_M$, using Hoeffding's bound,
%     % \footnote{For a fully rigorous argument,
%     % refer to the complete proof in App.~\ref{app:proof_erm}.} 
%     we have with probability at least $1-\delta$ that
%     \begin{align}
%       \error_{\wt \calS_M}(f^*) \le \error_{ \calDm}(f^*) +  \sqrt{\frac{\log(1/\delta)}{2 m_1}} \,. \label{eq:step2} 
%     \end{align}
%     %$ 
%     %for some constant $c_1\le 1/2$. 
%     Finally, since $f^*$ is the optimal classifier on $\widecheck \calD$, 
%     we have 
%     \begin{align}
%         \error_{\widecheck \calD}(f^*) \le \error_{\widecheck \calD}(\widehat f) \label{eq:step3}
%     \end{align}
%      Now to relate \eqref{eq:step1} and \eqref{eq:step3}, we can re-write the \eqref{eq:step2} as follows: 
%     \begin{align}
%         \error_{\calS \cup \wt\calS}(f^*) \le \error_{ \widecheck \calD}(f^*) +  \frac{m_1}{m+n}\sqrt{\frac{\log(1/\delta)}{2 m_1}} \,. \label{eq:step4} 
%     \end{align}
%     Now we combine equations \eqref{eq:step1}, \eqref{eq:step4}, and \eqref{eq:step3}, to get 
%     \begin{align}
%         \error_{\calS \cup \wt \calS}(\wh f) \le \error_{\widecheck \calD}(\wh f) +  \frac{m_1}{m+n}\sqrt{\frac{\log(1/\delta)}{2 m_1}} \,, 
%     \end{align}
%     which implies 
%     \begin{align}
%         \error_{ \wt \calS_M}(\wh f) \le \error_{\calDm}(\wh f) + \sqrt{\frac{\log(1/\delta)}{2 m_1}} \,. \label{eq:lemma1_final}
%     \end{align}
%     Since $\wt S$ is obtained by randomly labeling an unlabeled dataset, we assume $2m_1 \approx m$ \footnote{Formally, with probability at least $1-\delta$, we have  $(m - 2m_1)\le \sqrt{m\log(1/\delta)/2}$ }. Moreover, using $\error_{\calDm} = 1 - \error_{\calD}$ we obtain the desired result.   
%     % Combining the above steps and using the fact 
%     % that $\error_\calD = 1- \error_{\calDm} $, 
%     % we obtain the desired result.
% \end{proof}

% \begin{proof}[Proof of \lemref{lem:mislabeled_error}]
%     Recall $\error_{\wt S} (f) = \frac{m_1}{m} \error_{\wt S_M}(f) + \frac{m_2}{m} \error_{\wt S_C}(f)$. Hence, we have 
%     \begin{align}
%         2\error_{\wt S}(f) - \error_{\wt S_M}(f) - \error_{\wt S_C}(f) &= \left(\frac{2m_1}{m} \error_{\wt S_M}(f) - \error_{\wt S_M}(f)\right) + \left(\frac{2m_2}{m} \error_{\wt S_C}(f) - \error_{\wt S_C}(f)\right) \\ &= \left(\frac{2m_1}{m} - 1\right) \error_{\wt S_M}(f) + \left(\frac{2m_2}{m} - 1 \right)\error_{\wt S_C} (f) \,.
%     \end{align} 
%     Since the dataset is randomly labeled, with probability at least $1-\delta$, we have  $\left(\frac{2m_1}{m} - 1\right) \le \sqrt{\frac{\log(1/\delta)}{2m}}$. Similarly, we have with probability at least $1-\delta$, $\left(\frac{2m_2}{m} - 1\right) \le \sqrt{\frac{\log(1/\delta)}{2m}}$. Using union bound, we have with probability at least $1-\delta$
%     % \begin{align}
%     %     2\error_{\wt S} - \error_{\wt S_M}(f) - \error_{\wt S_C}(f) \le \sqrt{\frac{\log(2/\delta)}{2m}} \left(\error_{\wt S_M}(f) + \error_{\wt S_C}(f) \right) \le 2\sqrt{\frac{\log(2/\delta)}{2m}} \,. \label{eq:lemma2_final}
%     % \end{align}
%     \begin{align}
%         2\error_{\wt S} - \error_{\wt S_M}(f) - \error_{\wt S_C}(f) \le \sqrt{\frac{\log(2/\delta)}{2m}} \left(\error_{\wt S_M}(f) + \error_{\wt S_C}(f) \right) \,. \label{eq:lemma2_prefinal}
%     \end{align}
%     With re-arranging $\error_{\wt S_M}(f) + \error_{\wt S_C}(f)$ and using the inequality $ 1- a\le \frac{1}{1+a} $, we have  
%     \begin{align}
%         2\error_{\wt S} - \error_{\wt S_M}(f) - \error_{\wt S_C}(f) \le 2\error_{\wt \calS} \sqrt{\frac{\log(2/\delta)}{2m}}  \,. \label{eq:lemma2_final}
%     \end{align}

%     % We obtain the desired result by using 
% \end{proof}

% \begin{proof}[Proof of \lemref{lem:clear_error}]
% % Recall 0-1 error on each point  $(x,y) \in S \cup \wt S$ is given by $\I{ f(x)\ne y}$.
% In the set of correctly labeled points $S \cup \wt S_C$, we have $S$ as a random subset of $S \cup \wt S_C$. Hence, using Hoeffding's inequality for sampling without replacement (\lemref{lem:hoeffding_sampling}), we have with probability at least $1-\delta$
% \begin{align}
%     \error_{\wt \calS_c} (\wh f)- \error_{\calS \cup \wt \calS_C}( \wh f) \le  \sqrt{\frac{\log(1/\delta)}{2m_2}} \,.
% \end{align}
% Re-writing $\error_{\calS \cup \wt \calS_C}( \wh f)$ as $\frac{m_2}{m_2 + n} \error_{\wt \calS_C }(\wh f) + \frac{n}{m_2 + n} \error_{\calS }(\wh f)$, we have with probability at least $1-\delta$
% \begin{align}
%   \left(\frac{n}{n+m_2}\right) \left(\error_{\wt \calS_c} (\wh f)- \error_{\calS}( \wh f) \right) \le  \sqrt{\frac{\log(1/\delta)}{2m_2}} \,.
% \end{align}
% As before, assuming $2m_2 \approx m$, we have with probability at least $1-\delta$ 
% \begin{align}
%     \error_{\wt \calS_c} (\wh f)- \error_{\calS}( \wh f) \le \left(1+\frac{m_2}{n}\right)  \sqrt{\frac{\log(1/\delta)}{m}} \le 1.5 \sqrt{\frac{\log(1/\delta)}{m}} \,. \label{eq:lemma3_final}
% \end{align} 
% \end{proof}

% \begin{proof}[Proof of \thmref{thm:error_ERM}] 
%     Having established these core intermediate results, we can now combine above three lemmas to prove the main result. 
%     In particular, we bound the population error on clean data ($\error_\calD(\wh f)$) as follows:  
%     \begin{enumerate}[(i)]
%         \item First, use \eqref{eq:lemma1_final}, to obtain an upper bound on the population error on clean data, i.e., with probability at least $1-\delta/4$, we have
%         \begin{align}
%             \error_{ \calD} (\wh f) \le 1 - \error_{ \wt \calS_M}(\wh f) + \sqrt{\frac{\log(4/\delta)}{m}} \,. 
%         \end{align}
%         \item  Second, use \eqref{eq:lemma2_final}, to relate the error on the mislabeled fraction with error on clean portion of randomly labeled data and error on whole randomly labeled dataset, i.e., with probability at least $1-\delta/2$, we have 
%         \begin{align}
%             - \error_{\wt S_M}(f) \le \error_{\wt S_C}(f) - 2\error_{\wt S}  + \sqrt{\frac{\log(4/\delta)}{2m}}  \,. 
%         \end{align} 
%         \item Finally, use \eqref{eq:lemma3_final} to relate the error on the clean portion of randomly labeled data and error on clean training data, i.e., with probability $1-\delta/4$, we have 
%         \begin{align}
%             \error_{\wt \calS_C} (\wh f)\le - \error_{\calS}( \wh f) + \left(1 + \frac{m}{2n} \right) \sqrt{\frac{\log(4/\delta)}{m}} \,. 
%         \end{align} 
%     \end{enumerate}

%     Using union bound on the above three steps, we have with probability at least $1-\delta$: 
%     \begin{align}
%         \error_\calD (\wh f) \le \error_{\calS}(\wh f)   + 1 - 2\error_{\wt \calS}(\wh f)   + (1/\sqrt{2} + 2.5)  \sqrt{\frac{\log(4/\delta)}{m}} \,.
%     \end{align}
%     Note that $(1/\sqrt{2} + 2.5)$ is a loose constant. In experiments, we use the ratio $\frac{m}{n}$
%     %  the exact error $\error_{\wt \calS}(\wh f)$ 
%     to evaluate R.H.S.    
% \end{proof}

% \subsection{Proof of \propref{prop:rademacher}}

% \begin{proof}[Proof of \propref{prop:rademacher}]
%     For a classifier $ f: \calX \to \{-1, 1\}$, we have $1 - 2\,\indict{ f(x) \ne y} = y \cdot f(x)$. Hence, by definition of $\error$, we have 
%     \begin{align}
%         1 -2\error_{\wt \calS}(f) = \frac{1}{m}\sum_{i=1}^m y_i \cdot f(x_i) \le \sup_{f \in \calF} \, \frac{1}{m} \sum_{i=1}^m y_i \cdot f(x_i)  \,. \label{eq:error_rademacher}
%     \end{align}
%     Note that for fixed inputs $(x_1, x_2, \ldots, x_m)$ in $\wt S$, $(y_1, y_2, \ldots y_m)$ are random labels. Define $\phi_1 (y_1, y_2, \ldots, y_m) \defeq \sup_{f \in \calF} \, \frac{1}{m} \sum_{i=1}^m y_i \cdot f(x_i)$. We have the following bounded difference condition on $\phi_1$. For all i, 
%     \begin{align}
%         \sup_{y_1, \ldots y_m, y_i^\prime \in \{-1, 1\}^{m+1} } \abs{ \phi_1 (y_1,\ldots, y_i, \ldots, y_m) - \phi_1 (y_1,\ldots, y_i^\prime, \ldots, y_m)  } \le 1/m \,. \label{cond1_rademacher}
%     \end{align} 
    
%     Similarly define $\phi_2 (x_1, x_2, \ldots, x_m) \defeq \Expt{ y_i \sim_U \{-1, 1\}  }{ \sup_{f \in \calF} \, \frac{1}{m}  \sum_{i=1}^m y_i \cdot f(x_i)}$. We have the following bounded difference condition on $\phi_2$. For all i,
%     \begin{align}
%         \sup_{x_1, \ldots x_m, x_i^\prime \in \calX^{m+1} } \abs{ \phi_2 (x_1,\ldots, x_i, \ldots, x_m) - \phi_1 (x_1,\ldots, x_i^\prime, \ldots, x_m)  } \le 1/m \,. \label{cond2_rademacher}
%     \end{align}
%     Using McDiarmid’s inequality (\lemref{lem:McDiarmid}) twice with Condition \eqref{cond1_rademacher} and \eqref{cond2_rademacher}, with probability at least $1-\delta$, we have
%     \begin{align}
%         \sup_{f \in \calF} \, \frac{1}{m} \sum_{i=1}^m y_i \cdot f(x_i)  - \Expt{x,y}{\sup_{f \in \calF} \, \frac{1}{m} \sum_{i=1}^m y_i \cdot f(x_i) } \le \sqrt{\frac{2\log(2/\delta)}{m}} \label{eq:final_rademacher}
%     \end{align} 
%     Combining \eqref{eq:error_rademacher} and \eqref{eq:final_rademacher}, we obtain the desired result. 
% \end{proof}


% \subsection{Proof of \thmref{thm:error_regularized_ERM}}

% Proof of \thmref{thm:error_regularized_ERM} follows similar to the proof of \thmref{thm:error_ERM}. Note that the same results in \lemref{lem:fit_mislabeled}, \lemref{lem:mislabeled_error}, and \lemref{lem:clear_error} hold in the regularized ERM case. However, the arguments in the proof of \lemref{lem:fit_mislabeled} changes slightly. Hence, we state and prove a lemma parallel to \lemref{lem:fit_mislabeled} for completeness. 

% \begin{lemma} \label{lem:lemma1_reg}
%     Assume the same setup as \thmref{thm:error_regularized_ERM}. 
%     Then for any $\delta >0$, with probability at least  $1-\delta$ 
%     over the random draws of mislabeled data $\wt S_M$, we have 
%     \begin{align}
%         \error_\calD(\widehat f)  \le 1 -\error_{\wt \calS_M}(\widehat f) + \sqrt{\frac{\log(1/\delta)}{m}}\,. 
%     \end{align} 
% \end{lemma}
% \begin{proof}
%     The main idea of the proof remains the same, i.e. regard 
%     the clean portion of the data 
%     ($S \cup \wt S_C$) as fixed.   
%     Then, there exists a classifier $f^*$ 
%     that is optimal over draws 
%     of the mislabeled data $\wt S_M$. 

    
%     Formally, 
%     \begin{align}
%     f^* \defeq \argmin_{f \in \calF} \error_{\widecheck {\calD}} (f)  + \lambda R(f) \,, \label{eq:modified_ERM_reg}
%     \end{align}
%     where $$\widecheck \calD = \frac{n}{m+n} \calS + \frac{m_1}{m+n} \wt \calS_C  + \frac{m_2}{m+n}\calDm \,.$$ That is, $\widecheck \calD$ a combination of 
%     the \emph{empirical distribution} 
%     over correctly labeled data $S \cup \wt S_C$
%     % in $S\cup \wt S$ 
%     and the (population) distribution 
%     over mislabeled data $\calDm$.
%     Recall that 
%     \begin{align}
%     \wh f \defeq \argmin_{f \in \calF} \error_{\calS \cup \wt S} (f) + \lambda R(f) \,. \label{eq:orig_ERM_reg}
%     \end{align}
%     % 
%     % 
%     Since, $\widehat f$ minimizes 0-1 error 
%     on $S \cup \wt S$, using ERM optimality on \eqref{eq:orig_ERM},  
%     we have 
%     \begin{align}
%         \error_{\calS \cup \wt \calS}(\widehat f) + \lambda R(\wh f) \le \error_{
%             \calS \cup \wt \calS}(f^*) + \lambda R(f^*) \,.    \label{eq:step1_reg}
%     \end{align}
%     Moreover, since $f^*$ is independent of $\wt S_M$, using Hoeffding's bound,
%     % \footnote{For a fully rigorous argument,
%     % refer to the complete proof in App.~\ref{app:proof_erm}.} 
%     we have with probability at least $1-\delta$ that
%     \begin{align}
%       \error_{\wt \calS_M}(f^*) \le \error_{ \calDm}(f^*) +  \sqrt{\frac{\log(1/\delta)}{2 m_1}} \,. \label{eq:step2_reg} 
%     \end{align}
%     %$ 
%     %for some constant $c_1\le 1/2$. 
%     Finally, since $f^*$ is the optimal classifier on $\widecheck \calD$, 
%     we have 
%     \begin{align}
%         \error_{\widecheck \calD}(f^*) + \lambda R(f^*) \le \error_{\widecheck \calD}(\widehat f) + \lambda R(\wh f) \label{eq:step3_reg}
%     \end{align}
%      Now to relate \eqref{eq:step1_reg} and \eqref{eq:step3_reg}, we can re-write the \eqref{eq:step2_reg} as follows: 
%     \begin{align}
%         \error_{\calS \cup \wt\calS}(f^*) \le \error_{ \widecheck \calD}(f^*) +  \frac{m_1}{m+n}\sqrt{\frac{\log(1/\delta)}{2 m_1}} \,. \label{eq:step4_reg} 
%     \end{align}
%     After adding $\lambda R(f^*)$ on both sides in \eqref{eq:step4_reg}, we combine equations \eqref{eq:step1_reg}, \eqref{eq:step4_reg}, and \eqref{eq:step3_reg}, to get 
%     \begin{align}
%         \error_{\calS \cup \wt \calS}(\wh f) \le \error_{\widecheck \calD}(\wh f) +  \frac{m_1}{m+n}\sqrt{\frac{\log(1/\delta)}{2 m_1}} \,, 
%     \end{align}
%     which implies 
%     \begin{align}
%         \error_{ \wt \calS_M}(\wh f) \le \error_{\calDm}(\wh f) + \sqrt{\frac{\log(1/\delta)}{2 m_1}} \,. \label{eq:lemma_reg_final}
%     \end{align}
%     Similar as before, since $\wt S$ is obtained by randomly labeling an unlabeled dataset, we assume 
%     $2m_1 \approx m$. Moreover, using $\error_{\calDm} = 1 - \error_{\calD}$ we obtain the desired result. 
% \end{proof}
% % \begin{proof}[Proof of ]
    
% % \end{proof}

% \subsection{Proof of \thmref{thm:multiclass_ERM}}

% We first state and prove lemmas parallel to three lemmas used in the proof of balanced binary case. Then we combine the results in the three lemmas to obtain the result in \thmref{thm:multiclass_ERM}. 

% Before stating the result, we define mislabeled distribution $\calDm$ for any $\calD$. While $\calDm$ and $\calD$ share 
% the same marginal distribution over $\calX$, 
% the distribution over labels $y$ 
% given an input $x\sim \calD_\calX$ is changed.
% In particular, for any $x$, the pdf over $y$ is changed to:  
% $p_{\calDm} (\cdot \vert x) \defeq \frac{1 - p_{\calD}(\cdot \vert x)}{k - 1}$.

% \begin{lemma} \label{lem:fit_mislabeled_multi}
%     Assume the same setup as \thmref{thm:multiclass_ERM}. 
%     Then for any $\delta >0$, with probability at least  $1-\delta$ 
%     over the random draws of mislabeled data $\wt S_M$, we have 
%     \begin{align}
%         \error_\calD(\widehat f)  \le (k-1)\left(1 -\error_{\wt \calS_M}(\widehat f)\right) + (k-1)\sqrt{\frac{\log(1/\delta)}{m}}\,. \label{eq:lemma1_multi}
%     \end{align}   
% \end{lemma} 

% \begin{proof}
%     The main idea of the proof remains the same, i.e. regard 
%     the clean portion of the data 
%     ($S \cup \wt S_C$) as fixed. 
%     Then, there exists a classifier $f^*$ 
%     that is optimal over draws 
%     of the mislabeled data $\wt S_M$. 
    
%     However, we need to be careful while relating population error on mislabeled data with population accuracy on clean data.   
%     While for binary classification,  we could upper bound $\error_{\wt \calS_M}$ 
%     with $1-\error_\calD$  (in the proof of \lemref{lem:fit_mislabeled}), 
%     for multiclass classification, 
%     error on the mislabeled data 
%     and accuracy on the clean data 
%     in the population 
%     are not so directly related.  
%     To establish \eqref{eq:lemma1_multi},
%     we break the error on the 
%     (unknown) mislabeled data 
%     into two parts: one term corresponds 
%     to predicting the true label on mislabeled data, 
%     and the other corresponds to predicting 
%     neither the true label 
%     nor the assigned (mis-)label.  
%     Finally, we relate these errors to their
%     population counterparts to establish \eqref{eq:lemma1_multi}. 
    
%     Formally, 
%     \begin{align}
%     f^* \defeq \argmin_{f \in \calF} \error_{\widecheck {\calD}} (f)  + \lambda R(f) \,, \label{eq:modified_ERM_reg2}
%     \end{align}
%     where $$\widecheck \calD = \frac{n}{m+n} \calS + \frac{m_1}{m+n} \wt \calS_C  + \frac{m_2}{m+n}\calDm \,.$$ That is, $\widecheck \calD$ a combination of 
%     the \emph{empirical distribution} 
%     over correctly labeled data $S \cup \wt S_C$
%     % in $S\cup \wt S$ 
%     and the (population) distribution 
%     over mislabeled data $\calDm$.
%     Recall that 
%     \begin{align}
%     \wh f \defeq \argmin_{f \in \calF} \error_{\calS \cup \wt S} (f) + \lambda R(f) \,. \label{eq:orig_ERM_reg2}
%     \end{align}
%     % 
%     % 
%     Following the exact steps from the proof of \lemref{lem:lemma1_reg}, with probability at least $1-\delta$, we have  
%     \begin{align}
%         \error_{ \wt \calS_M}(\wh f) \le \error_{\calDm}(\wh f) + \sqrt{\frac{\log(1/\delta)}{2 m_1}} \,. \label{eq:lemma1_final_multi_prev}
%     \end{align}
%     Similar to before, since $\wt S$ is obtained by randomly labeling an unlabeled dataset, we assume 
%     $\frac{k}{k-1} m_1 \approx m$. 
    
%     Now we will relate $\error_\calDm (\wh f)$ with $\error_{\calD}(\wh f)$. Let $y^T$ denote the (unknown) true label for a mislabeled point $(x, y)$ (i.e., label before replacing it with a mislabel). 
%     \begin{align}    
%          \Expt{(x, y) \in \sim \calDm}{\indict{ \wh f(x) \ne y }}  &= \underbrace{\Expt{(x, y) \in \sim \calDm}{\indict{ \wh f(x) \ne y \land \wh f(x) \ne y^T}}}_{\RN{1}} + \underbrace{\Expt{(x, y) \in \sim \calDm}{\indict{ \wh f(x) \ne y \land \wh f(x) = y^T}}}_{\RN{2}} \,. \label{eq:excess_term}
%     \end{align}
%     Clearly, term 2 is one minus the accuracy on the clean unseen data, i.e. 
%     \begin{align}
%         \RN{2} = 1 - \Expt{{x,y} \sim \calD}{ \indict{ \wh f(x) \ne y}} = 1- \error_{\calD}(\wh f) \,. \label{eq:term1}    
%     \end{align}
%     Next, we  relate term 1 with the error on the unseen clean data. We show that term 1 is equal to the error on the unseen clean data scaled by $\frac{k-2}{k-1}$ where $k$ is the number of labels. Using the definition of mislabeled distribution $\calDm$,  we have 
%     \begin{align}
%         \RN{1} = \frac{1}{k-1} \left( \Expt{(x, y) \in \sim \calD}{ \sum_{i \in \calY \land i\ne y}  \indict{ \wh f(x) \ne i \land \wh f(x) \ne y}} \right) = \frac{k-2}{k-1} \error_{\calD}(\wh f) \,.\label{eq:term2}
%     \end{align}    

%     Combining the result in \eqref{eq:term1}, \eqref{eq:term2} and \eqref{eq:excess_term}, we have 
%     \begin{align}
%         \error_{\calDm}(\wh f) = 1- \frac{1}{k-1} \error_{\calD}(\wh f) \,.\label{eq:combine_terms}
%     \end{align}
%     Finally, combining the result in \eqref{eq:combine_terms} with equation \eqref{eq:lemma1_final_multi_prev}, we have with probability $1-\delta$, 
%     \begin{align}
%       \error_{\calD}(\wh f) \le  (k-1) \left( 1- \error_{ \wt \calS_M}(\wh f) \right)  + (k-1) \sqrt{\frac{k \log(1/\delta)}{ 2(k-1)m}} \,. \label{eq:lemma1_final_multi}
%     \end{align}
% \end{proof}

% \begin{lemma} \label{lem:mislabeled_error_multi}
%     Assume the same setup as \thmref{thm:multiclass_ERM}.  Then for any $\delta >0$, with probability at least $1-\delta$ over the random draws of $\wt S$, we have  
%     % \begin{align}
%         $$\abs{k\error_{\wt \calS}(\widehat f) - \error_{\wt \calS_C}(\widehat f) -  (k-1)\error_{\wt \calS_M}(\widehat f) } \le  2k\sqrt{\frac{\log(4/\delta)}{2m}}\,. $$ % \label{eq:lemma2}
%     % \end{align}   
%     %  for some constant $c_3 \le 1.0\,$.
% \end{lemma} 


% \begin{proof}
%     Recall $\error_{\wt S} (f) = \frac{m_1}{m} \error_{\wt S_M}(f) + \frac{m_2}{m} \error_{\wt S_C}(f)$. Hence, we have 
%     \begin{align}
%         k\error_{\wt S}(f) - (k-1)\error_{\wt S_M}(f) - \error_{\wt S_C}(f) &= (k-1)\left(\frac{k m_1}{(k-1) m} \error_{\wt S_M}(f) - \error_{\wt S_M}(f)\right) + \left(\frac{km_2}{m} \error_{\wt S_C}(f) - \error_{\wt S_C}(f)\right) \\ &= k \left[ \left(\frac{m_1}{m} - \frac{k-1}{k}\right) \error_{\wt S_M}(f) + \left(\frac{m_2}{m} - \frac{1}{k} \right) \error_{\wt S_C} (f) \right] \,.
%     \end{align} 
%     Since the dataset is randomly labeled, we have with probability at least $1-\delta$, $\left(\frac{m_1}{m} - \frac{k-1}{k}\right) \le \sqrt{\frac{\log(1/\delta)}{2m}}$. Similarly, we have with probability at least $1-\delta$, $\left(\frac{m_2}{m} - \frac{1}{k}\right) \le \sqrt{\frac{\log(1/\delta)}{2m}}$. Using union bound, we have with probability at least $1-\delta$
%     % \begin{align}
%     %     2\error_{\wt S} - \error_{\wt S_M}(f) - \error_{\wt S_C}(f) \le \sqrt{\frac{\log(2/\delta)}{2m}} \left(\error_{\wt S_M}(f) + \error_{\wt S_C}(f) \right) \le 2\sqrt{\frac{\log(2/\delta)}{2m}} \,. \label{eq:lemma2_final}
%     % \end{align}
%     \begin{align}
%         k\error_{\wt S}(f) - (k-1)\error_{\wt S_M}(f) - \error_{\wt S_C}(f)  \le k \sqrt{\frac{\log(2/\delta)}{2m}} \left(\error_{\wt S_M}(f) + \error_{\wt S_C}(f) \right) \,. \label{eq:lemma2_final_multi}
%     \end{align}

%     % We obtain the desired result by using 
% \end{proof}

% \begin{lemma} \label{lem:clear_error_multi}
%     Assume the same setup as \thmref{thm:multiclass_ERM}. 
%     Then for any $\delta >0$, with probability at least $1-\delta$ 
%     over the random draws of $\wt S_C$ and $S$, we have 
%     % \begin{align}
%         $$\abs{\error_{\wt \calS_C}(\widehat f) - \error_{\calS}(\widehat f) } \le 1.5 \sqrt{\frac{k\log(2/\delta)}{2m}}\,.$$ %\label{eq:lemma3}
%     % \end{align}   
%     % for some constant $c_2 \le 1.2\,$.
% \end{lemma} 
% \begin{proof}
%     % Recall 0-1 error on each point  $(x,y) \in S \cup \wt S$ is given by $\I{ f(x)\ne y}$.
%     In the set of correctly labeled points $S \cup \wt S_C$, we have $S$ as a random subset of $S \cup \wt S_C$. Hence, using Hoeffding's inequality for sampling without replacement (\lemref{lem:hoeffding_sampling}), we have with probability at least $1-\delta$
%     \begin{align}
%         \error_{\wt \calS_c} (\wh f)- \error_{\calS \cup \wt \calS_C}( \wh f) \le  \sqrt{\frac{\log(1/\delta)}{2m_2}} \,.
%     \end{align}
%     Re-writing $\error_{\calS \cup \wt \calS_C}( \wh f)$ as $\frac{m_2}{m_2 + n} \error_{\wt \calS_C }(\wh f) + \frac{n}{m_2 + n} \error_{\calS }(\wh f)$, we have with probability at least $1-\delta$
%     \begin{align}
%       \left(\frac{n}{n+m_2}\right) \left(\error_{\wt \calS_c} (\wh f)- \error_{\calS}( \wh f) \right) \le  \sqrt{\frac{\log(1/\delta)}{2m_2}} \,.
%     \end{align}
%     As before, assuming $km_2 \approx m$, we have with probability at least $1-\delta$ 
%     \begin{align}
%         \error_{\wt \calS_c} (\wh f)- \error_{\calS}( \wh f) \le \left(1+\frac{m_2}{n}\right)  \sqrt{\frac{k\log(1/\delta)}{2m}} \le \left( 1 + \frac{1}{k}\right) \sqrt{\frac{k\log(1/\delta)}{2m}} \,. \label{eq:lemma3_final_multi}
%     \end{align} 
% \end{proof}

% \begin{proof}[Proof of \thmref{thm:multiclass_ERM}] 
%     Having established these core intermediate results, we can now combine above three lemmas. 
%     In particular, we bound the population error on clean data ($\error_\calD(\wh f)$) as follows:  
%     \begin{enumerate}[(i)]
%         \item First, use \eqref{eq:lemma1_final_multi}, to obtain an upper bound on the population error on clean data, i.e., with probability at least $1-\delta/4$, we have
%         \begin{align}
%             \error_{ \calD} (\wh f) \le (k-1)\left(1 - \error_{ \wt \calS_M}(\wh f) \right) + (k-1) \sqrt{\frac{k\log(4/\delta)}{2(k-1)m}} \,. 
%         \end{align}
%         \item  Second, use \eqref{eq:lemma2_final_multi}, to relate the error on the mislabeled fraction with error on clean portion of randomly labeled data and error on whole randomly labeled dataset, i.e., with probability at least $1-\delta/2$, we have 
%         \begin{align}
%             - (k-1)\error_{\wt S_M}(f) \le \error_{\wt S_C}(f) - k\error_{\wt S}  + k\sqrt{\frac{\log(4/\delta)}{2m}}  \,. 
%         \end{align} 
%         \item Finally, use \eqref{eq:lemma3_final_multi} to relate the error on the clean portion of randomly labeled data and error on clean training data, i.e., with probability $1-\delta/4$, we have 
%         \begin{align}
%             \error_{\wt \calS_C} (\wh f)\le - \error_{\calS}( \wh f) + \left(1 + \frac{m}{kn} \right) \sqrt{\frac{k\log(4/\delta)}{2m}} \,. 
%         \end{align} 
%     \end{enumerate}

%     Using union bound on the above three steps, we have with probability at least $1-\delta$: 
%     \begin{align}
%         \error_\calD (\wh f) \le \error_{\calS}(\wh f) + (k-1) - k\error_{\wt \calS}(\wh f)   + (\sqrt{k(k-1)} + k + \sqrt{k} + \frac{m}{n\sqrt{k}})  \sqrt{\frac{\log(4/\delta)}{2m}} \,.
%     \end{align}
%     % Note that $\frac{m}{n\sqrt{k}}$ is much smaller than the other terms in addition. Hence, we ignore this in the final bound. 
%     % Note that $(1/\sqrt{2} + 2.5)$ is a loose constant. In experiments, we use the ratio $\frac{m}{n}$
%     %  the exact error $\error_{\wt \calS}(\wh f)$ 
%     % to evaluate R.H.S.    
% \end{proof}

% \newpage
% \section{Proofs from \secref{sec:linear_models}}\label{app:proof_gd}

% We suppose that the parameters of the linear function 
% are obtained via gradient descent on 
% the following $L_2$ regularized problem: 
% \begin{align}
%     % n in denominator is avoided deliberately
%     \calL_S(w; \lambda) \defeq \sum_{i=1}^n{(w^Tx_i - y_i)^2} + \lambda \norm{w}{2}^2 \,, \label{eq:l2_MSE_app}   
% \end{align}
% where $\lambda\ge0$ is a regularization parameter. 
% We assume access to a clean dataset 
% $S = \{(x_i, y_i)\}_{i=1}^n \sim \calD^n$ 
% and randomly labeled dataset 
% $\wt S = \{(x_i, y_i)\}_{i=n+1}^{n+m} \sim \wt \calD^m$. 
% Let $\bX = [x_1, x_2, \cdots, x_{m+n}]$ 
% and $\by = [y_1, y_2, \cdots, y_{m+n}]$. 
% Fix a positive learning rate $\eta$ such that 
% $\eta \le 1/\left(\norm{\bX^T\bX}{\text{op}} + \lambda^2\right)$ 
% and an initialization $w_0 = 0$. 
% % \todos{Assumption made for simplicty}. 
% Consider the following gradient descent iterates 
% to minimize objective \eqref{eq:l2_MSE_app} on $S \cup \wt S$:
% \begin{align}
% w_t = w_{t-1} - \eta \grad_w \calL_{S \cup \wt S} (w_{t-1}; \lambda) \quad \forall t=1,2,\ldots \label{eq:GD_iterates_app}
% \end{align} 
% Then we have $\{ w_t\}$ converge to the limiting solution 
% $\wh w = \left( \bX^T\bX+\lambda \boldsymbol{I}\right)^{-1}\bX^T\by$. Define $\widehat f (x) \defeq f(x ; \wh w) $.  

% \subsection{\textcolor{red}{Errata}}

% We wish to correct the following error in the body: \codref{cond:error_stability} is not enough to guarantee the result in \thmref{thm:linear}. We now present a slightly stronger condition called \emph{hypothesis stability} under which we obtain a result similar to \thmref{thm:linear}. 

% This error doesn't change the main arguments of the proof where we show that the empirical train error is less than or equal to the leave-one-out error. We need a stronger condition to relate leave-one-out error with the population error of the original classifier. Specifically, while \codref{cond:error_stability} relates the average population error of leave-one-out classifiers with the population error of the original classifier, we need the new condition to show the concentration of the empirical leave-one-out error and  average population error of leave-one-out classifiers. 
% % main takeaway 

% Note that the new condition, while being stronger than the previous one, still doesn't imply generalization~\cite{bousquet2002stability,elisseeff2003leave,abou2019exponential}. Overall, the main results in \secref{sec:ERM_training} and takeaways of the paper remain unaffected by the error.  

% We now present the new condition and a corrected statement of \thmref{thm:linear}. Recall, for a given training set $S \sim \calD^n $, 
% we use $S_{(i)}$ to denote the training set $S$ 
% with the $i^{\text{th}}$ point removed.

% \begin{condition}[Hypothesis Stability] 
%     \label{cond:hypothesis_stability}
%     We have $\beta$ hypothesis stability 
%     if our training algorithm $\calA$ satisfies the following: 
%     \begin{align*}
%     % ${\sum_{i=1}^n \frac{\error_{\calD}( f(\calA, S_{(i)}))}{n} - \error_\calD(f(\calA, S))} \le \beta\,$.
%     \forall i \in \{1,2,\ldots, n\}, \quad  \Expt{\calS, (x,y) \in \calD}{ \abs{\error\left( f(x) ,y  \right) - \error\left( f_{(i)}(x), y \right) }} \le \frac{\beta}{n} \,,
%     \end{align*}
%     where $f_{(i)} \defeq f(\calA, S_{(i)})$ and $ f \defeq f(\calA, S)$.
% \end{condition}

% \begin{theorem}[Correct statement of \thmref{thm:linear}] \label{thm:new_linear}
%     Assume that this gradient descent algorithm satisfies \codref{cond:hypothesis_stability}
%     with $\beta=\calO(1)$.  
%     Then for any $\delta >0$, with probability at least $1-\delta$ 
%     over the random draws of datasets $\wt S$ and $S$, we have:
%     \begin{align}
%         \error_\calD(\widehat f) \le \error_\calS(\widehat f) + 1 - 2 \error_{\wt\calS}(\widehat f) + \left(\frac{1}{\sqrt{2}} + 1.5 \right) \sqrt{\frac{\log(4/\delta)}{m}} + \sqrt{\frac{4}{\delta}\left(\frac{1}{m} +\frac{3\beta}{m+n} \right)}  \,. \label{eq:gd_error}
%     \end{align} 
%     % for some constant $c\le 3.2$.
% \end{theorem}

% \subsection{Proof of \thmref{thm:new_linear}}
% We use a standard result from linear algebra, namely Shermann-Morrison formula~\citep{sherman1950adjustment} for matrix inversion:  

% \begin{lemma}[\citet{sherman1950adjustment}] \label{lem:sherman}
%     Suppose $\bA \in \Real^{n \times n}$ is an invertible square matrix and $u,v \in \Real^n$ are column vectors. Then $\bA + uv^T$ is invertible iff $1 + v^T \bA u \ne 0$ and in particular
%     \begin{align}
%         (\bA + u v^T)^{-1} = \bA^{-1}  - \frac{\bA^{-1} uv^T \bA^{-1} }{ 1 + v^T \bA^{-1} u} \,.
%     \end{align}   
% \end{lemma}
% \newcommand\byy[1]{\by_{\left(#1\right)}}
% \newcommand\bXX[1]{\bX_{\left(#1\right)}}
% \newcommand\ff[1]{\wh f_{\left(#1\right)}}

% For a given training set $S \cup \wt S_C$, define leave-one-out error on mislabeled points in the training data as $$\error_{\text{LOO}(\wt S_M) } = \frac{\sum_{(x_i, y_i) \in \wt S_M} \error( f_{(i)}( x_i), y_i)}{ \abs{\wt S_M }} \,, $$
% where $f_{(i)} \defeq f(\calA, (S \cup \wt S)_{(i)})$. To relate empirical leave-one-out error and population error with hypothesis stability condition, we use the following lemma:   

% \begin{lemma}[\citet{bousquet2002stability}] \label{lem:stability_error}
%     For the leave-one-out error, we have
%     \begin{align}
%         \Expo{ \left( \error_{\calDm}(\wh f) -\error_{\text{LOO}(\wt S_M) } \right)^2 } \le \frac{1}{2m_1}+  \frac{3\beta}{n + m}\,.
%     \end{align}   
%     % where $ f \defeq f(\calA, S \cup \wt S) $.
% \end{lemma}

% Proof of the above lemma is similar to the proof of  Lemma 9 in \citet{bousquet2002stability} and can be found in \appref{app:proof_lem_error}. 
% % 
% % Before presenting the result, we introduce some notation. 
% Before presenting the proof of \thmref{thm:new_linear}, we introduce some more notation. Let $\bX_{(i)}$ denote the matrix of covariates with $i^{\text{th}}$ point removed. Similarly let $\by_{(i)}$ be the array of responses with $i^{\text{th}}$ point removed. Define the corresponding regularized GD solution as $\wh w_{(i)} = \left( \bXX{i}^T\bXX{i}+\lambda \boldsymbol{I}\right)^{-1}\bXX{i}^T\byy{i}$. Define $\ff{i}(x) \defeq f(x ; \wh w_{(i)}) $.

% \begin{proof}[Proof of \thmref{thm:new_linear}]
%     Because squared loss minimization does not imply 0-1 error minimization, we cannot use arguments from \lemref{lem:fit_mislabeled}. This is the main technical difficulty. To compare the 0-1 error at a train point with an unseen point, 
%     we use the closed-form expression for $\widehat{w}$ and Shermann-Morrison formula to upper bound training error with leave-one-out cross validation error. 
    
%     The proof is divided into three parts: In part one, we show that 0-1 error on mislabeled points in the training set is lower than the error obtained by leave-one-out error at those points. In part two, we relate this leave-one-out error with the population error on mislabeled distribution using \codref{cond:hypothesis_stability}. While the empirical leave-one-out error is unbiased estimator of the average population error of leave-one-out classifiers, we need hypothesis stability to control the variance of empirical leave-one-out error. Finally in part three, we show that the error on the mislabeled training points can be estimated with just the randomly labeled and  clean training data (as in proof of \thmref{thm:error_ERM}).  

%     \textbf{Part 1 {} {}} First we relate training error with leave-one-out error.        
%     For any 
%     training point $(x_i, y_i)$ in $\wt S \cup S$, we have 
%     \begin{align}
%         \error(\wh f(x_i), y_i ) &= \indict{ y_i \cdot x_i^T \wh w < 0 } = \indict{ y_i \cdot x_i^T \left( \bX^T\bX+\lambda \boldsymbol{I}\right)^{-1}\bX^T\by < 0 } \\
%         &= \indict{ y_i \cdot x_i^T \underbrace{\left( \bXX{i}^T\bXX{i} + x_i ^T x_i +\lambda \boldsymbol{I}\right)^{-1}}_{\RN{1}} (\bXX{i}^T\byy{i} + y \cdot x_i) < 0 }
%     \end{align}
%     Letting $\bA = \left(\bXX{i}^T\bXX{i} +\lambda \boldsymbol{I}\right)$ and using \lemref{lem:sherman} on term 1, we have 
%     \begin{align}
%         \error(\wh f(x_i), y_i ) &= \indict{ y_i \cdot x_i^T \left[\bA^{-1} -  \frac{\bA^{-1} x_i x_i^T \bA^{-1}}{ 1 + x_i ^T \bA^{-1} x_i } \right] (\bXX{i}^T\byy{i} + y \cdot x_i) < 0 } \\
%         &= \indict{ y_i \cdot\left[ \frac{ x_i^T \bA^{-1} ( 1 + x_i ^T \bA^{-1} x_i ) -  x_i^T \bA^{-1} x_i x_i^T \bA^{-1}}{ 1 + x_i ^T \bA ^{-1}x_i } \right] (\bXX{i}^T\byy{i} + y \cdot x_i) < 0 } \\
%         &= \indict{ y_i \cdot\left[ \frac{ x_i^T \bA^{-1}}{ 1 + x_i ^T \bA ^{-1}x_i } \right] (\bXX{i}^T\byy{i} + y \cdot x_i) < 0 } \,.
%     \end{align}

%     Since $1 + x_i^T \bA^{-1} x_i > 0$, we have 
%     \begin{align}
%         \error(\wh f(x_i), y_i ) &= \indict{ y_i \cdot x_i^T \bA^{-1} (\bXX{i}^T\byy{i} + y \cdot x_i) < 0 } \\
%         &= \indict{ x_i^T \bA^{-1} x_i +  y_i \cdot x_i^T \bA^{-1} (\bXX{i}^T\byy{i}) < 0 } \\
%         &\le \indict{ y_i \cdot x_i^T \bA^{-1} (\bXX{i}^T\byy{i}) < 0 } = \error(\ff{i}(x_i), y_i ) \,.\label{eq:LOO_error}
%     \end{align}

%     Using \eqref{eq:LOO_error}, we have 
%     \begin{align}
%         \error_{\wt \calS_M } (\wh f) \le \error_{\text{LOO} (S_M)} \defeq \frac{\sum_{(x_i, y_i) \in \wt S_M} \error(\ff{i}(x_i), y_i ) }{\abs{\wt \calS_M}}\label{eq:LOO_error_final}
%     \end{align}
%     \textbf{Part 2 {}{}} We now relate RHS in \eqref{eq:LOO_error_final} with the population error on mislabeled distribution. To do this, we leverage \codref{cond:hypothesis_stability} and \lemref{lem:stability_error}. In particular, we have 

%     \begin{align}
%         \Expt{\calS \cup \wt \calS_M }{ \left(\error_{\calDm}(\wh f) - \error_{\text{LOO} (S_M)}\right)^2 } \le \frac{1}{2m_1} + \frac{3\beta}{m+n} \,.
%     \end{align}

%     Using Chebyshev's inequality, with probability at least $1-\delta$, we have 
%     \begin{align}
%         \error_{\text{LOO} (S_M)} \le  \error_{\calDm}(\wh f)   + \sqrt{\frac{1}{\delta}\left(\frac{1}{2m_1} +\frac{3\beta}{m+n} \right)} \,. \label{eq:final_mislabeled_linear}
%     \end{align}
    

%     \textbf{Part 3 {}{}} Combining \eqref{eq:final_mislabeled_linear} and \eqref{eq:LOO_error_final}, we have 

%     \begin{align}
%         \error_{\wt \calS_M } (\wh f) \le \error_{\calDm}(\wh f)   + \sqrt{\frac{1}{\delta}\left(\frac{1}{2m_1} +\frac{3\beta}{m+n} \right)} \,. \label{eq:linear_parallel_lem1}
%     \end{align}

%     Compare \eqref{eq:linear_parallel_lem1}, with \eqref{eq:lemma1_final} in the proof of \lemref{lem:fit_mislabeled}. We obtain a similar relationship between $\error_{\wt \calS_M }$ and $\error_{\calDm}$ but with a polynomial concentration instead of exponential concentration. 
%     In addition, since we just use concentration arguments to relate mislabeled error with the error on clean portion and unlabeled portion, we can directly use the results in \lemref{lem:mislabeled_error} and \lemref{lem:clear_error}. Therefore, combining results in \lemref{lem:mislabeled_error}, \lemref{lem:clear_error}, and \eqref{eq:linear_parallel_lem1} with union bound, we have with probability at least $1-\delta$

%     \begin{align}
%         \error_\calD(\widehat f) \le \error_\calS(\widehat f) + 1 - 2 \error_{\wt\calS}(\widehat f) + \left(\frac{1}{\sqrt{2}} + 1.5 \right) \sqrt{\frac{\log(4/\delta)}{m}} + \sqrt{\frac{4}{\delta}\left(\frac{1}{m} +\frac{3\beta}{m+n} \right)}  \,.
%     \end{align}
    

       
% \end{proof}

% \subsection{Discussion on \codref{cond:hypothesis_stability}}

% The quantity in LHS of \codref{cond:hypothesis_stability} measures how much the function learned by the algorithm (in terms of error on unseen point) will change when one point in the training set is removed. 
% % Discussion on exponential concentration and stronger condition. 
% Notice that hypothesis stability implies error stability, i.e., \codref{cond:error_stability} ~\cite{bousquet2002stability}.  In summary, while error stability allowed us to relate the average population error of the leave-one-out classifiers with the population error of the original classifier, we need hypothesis stability condition to control the variance of the empirical leave-one-out error. 

% Additionally, we note that while the dominating term in the RHS of \thmref{thm:new_linear} matches with the dominating term in ERM bound in \thmref{thm:error_ERM}, there is a polynomial concentration term (dependence on $1/\delta$ instead of $\log(\sqrt{1/\delta})$) in  \thmref{thm:new_linear}. 
% Since with hypothesis stability, we just bound the variance,  the polynomial concentration is due to the use of Chebyshev's inequality instead of an exponential tail inequality (as in \lemref{lem:fit_mislabeled}).
% Recent works have highlighted that slightly stronger condition than hypothesis stability can be used to obtained an exponential concentration for leave-one-out error~\citep{abou2019exponential}, but we leave this for future work for now. 
% % We leave 
% % However, the constants 

% % we also want to highlight  

% \subsection{Formal statement and proof of  of \propref{prop:early_stop}}

% Before formally presenting the result, we will introduce some notation.  By $\calL_{S}(w)$, we denote 
% the objective in \eqref{eq:l2_MSE_app} with $\lambda=0$. 
% Assume Singular Value Decomposition (SVD) of $\bX$  as $\sqrt{n} \bU \bS^{1/2} \bV^T$. Hence $\bX^T \bX = \bV \bS \bV^T$.
% Consider the GD iterates defined in \eqref{eq:GD_iterates_app}. 
% % 
% We now derive closed form expression for the $t^\text{th}$ iterate of gradient descent:  
% % 
% \begin{align}
%     w_t = w_{t-1} + \eta \cdot \bX^T (\by - \bX w_{t-1}) = (\bI - \eta \bV \bS \bV^T )w_{k-1} + \eta \bX^T \by \,.
% \end{align}
% Rotating by $\bV^T$, we get 
% \begin{align}
%     \wt w_t = (\bI - \eta\bS )\wt w_{k-1} + \eta \wt \by \,, \label{eq:GD_recur}
% \end{align}
% where $\wt w_t = \bV^T w_t $ and $\wt \by = \bV^T \bX^T \by$. Assuming the initial point $w_0 = 0$ and applying the recursion in \eqref{eq:GD_recur}, we get
% \begin{align}
%     \wt w_t = \bS ^{-1} ( \bI - (\bI - \eta \bS)^k ) \wt \by \,, 
% \end{align} 
% Projecting solution back to the original space, we have 
% \begin{align}
%      w_t = \bV \bS ^{-1} ( \bI - (\bI - \eta \bS)^k ) \bV^T \bX^T \by \,, 
% \end{align} 
% % We will work with this GD solution at any iterate $t$ in the next proposition. 
% Define $f_t(x) \defeq f(x;w_t)$ as the solution at the $t^{\text{th}}$ iterate. 
% Let $\wt w_{\lambda} = \argmin_{w} \calL_\calS (w;\lambda) = (\bX^T \bX + \lambda \bI)^{-1} \bX^T \by = \bV (\bS + \lambda \bI )^{-1} \bV^T \bX^T \by $. 
% % ) \,,$ for all $t=1,2,\ldots\,.$ 
% and define $\wt f_\lambda(x) \defeq f(x;\wt w_\lambda)$ as the regularized solution. 
% Assume $\kappa$ be the condition number of the population covariance matrix 
% and 
% let $s_\text{min}$ be the minimum positive singular value of the empirical covariance matrix. Our proof idea is inspired from recent work on relating gradient flow solution and regularized solution for regression problems \citep{ali2018continuous}. We will use the following lemma in the proof: 
% \begin{lemma} \label{lem:ineq_soln}
%     For all $x \in [0,1]$ and for all $ k \in \mathbb{N}$, we have (a) $ \frac{kx}{1+kx} \le 1- (1-x)^k$ and (b) $ 1- (1-x)^k \le 2 \cdot \frac{kx}{kx+1} $.
%     %  where $g(c)$ is a constant dependent on $c$. For $c = 1$, $g(c) = 2.0$.   
% \end{lemma}
% \begin{proof}
%     % [Proof of \lemref{lem:ineq_soln}]
%     % Part (a) is easy. 
%     Using $ (1-x)^k \le \frac{1}{1+kx}$, we have part (a). For part (b), we numerically maximize $\frac{ (1+kx ) (1 - (1-x)^k) }{kx}$ for all $k\ge 1$ and for all $x \in [0, 1]$.  
% \end{proof}

% % 
% % Next, 

% \begin{prop}[Formal statement of \propref{prop:early_stop}] \label{prop:formal_early_stop}
% Let $\lambda = \frac{1}{t\eta}$. For a training point $x$, we have 
% \begin{align*}
%     \Expt{x \sim \calS}{(f_t(x) - \wt f_\lambda(x))^2} &\le c(t,\eta) \cdot \Expt{x \sim \calS}{f_t(x)^2} \,, %\label{eq:early_stop}
% \end{align*}
% where $c(t, \eta) \defeq \min( 0.25, \frac{1}{s_\text{min}^2 t^2 \eta^2})$. Similarly for a test point, we have 
% \begin{align*}
%     \Expt{x \sim \calD_\calX}{(f_t(x) - \wt f_\lambda(x))^2} &\le \kappa \cdot c(t,\eta) \cdot \Expt{x \sim \calD_\calX}{f_t(x)^2} \,. %\label{eq:early_stop}
% \end{align*}
% \end{prop} 

% \begin{proof}
%     %%%%%%%%%%%%% 
%     We want to analyze the expected squared difference output of regularized linear regression with regularization constant $\lambda = \frac{1}{\eta t}$ and gradient descent solution at $t^\text{th}$ iterate. We separately expand the algebraic expression for squared difference at a training point and a test point. 
%     % We start by considering the difference  
%     Then the main step is to show that  $\left[ \bS ^{-1} ( \bI - (\bI - \eta \bS)^k )  - (\bS + \lambda \bI )^{-1}\right] \preceq c(\eta, t) \cdot \bS ^{-1} ( \bI - (\bI - \eta \bS)^k ) $.

%     %%%%%%%%%%%%%
    
%   \textbf{Part 1 {} {}} 
%     First, we will analyze the squared difference of output at a training point (for simplicity, we refer to $S \cup \wt S$ as $S$), i.e. 
%     \begin{align}
%         \Expt{ x \sim \calS }{\left(f_t(x) - \wt f_\lambda (x)\right)^2} &= \norm{\bX w_t - \bX \wt w_\lambda}{2}^2 =   \norm{\bX \bV \bS ^{-1} ( \bI - (\bI - \eta \bS)^t ) \bV^T \bX^T \by - \bX \bV (\bS + \lambda \bI )^{-1} \bV^T \bX^T \by }{2}^2 \\
%         &= \norm{\bX \bV \left(\bS ^{-1} ( \bI - (\bI - \eta \bS)^t ) - (\bS + \lambda \bI )^{-1} \right) \bV^T \bX^T \by  }{2} \\
%         &=  \by^T \bV \bX \left( \underbrace{\bS ^{-1} ( \bI - (\bI - \eta \bS)^t ) - (\bS + \lambda \bI )^{-1}}_{\RN{1}} \right)^2 \bS \bV^T \bX^T \by \label{eq:train_GD_rel}
%         %  (\bX \bV \bS ^{-1} ( \bI - (\bI - \eta \bS)^k ) \bV^T \bX^T \by)^T \bX \bV \bS ^{-1} ( \bI - (\bI - \eta \bS)^k ) \bV^T \bX^T \by
%     \end{align}
%     We now separately consider term 1. Substituting $\lambda = \frac{1}{t \eta}$, we get
%     \begin{align}
%         \bS ^{-1} ( \bI - (\bI - \eta \bS)^t ) - (\bS + \lambda \bI )^{-1} &= \bS^{-1} \left( ( \bI - (\bI - \eta \bS)^t ) - (\bI + \bS^{-1} \lambda )^{-1}\right) \\
%         &= \underbrace{\bS^{-1} \left( ( \bI - (\bI - \eta \bS)^t ) - (\bI + ( \bS t \eta)^{-1}  )^{-1}\right)}_{\bA}
%     \end{align}

%     We now separately bound the diagonal entries in matrix $\bA$. 
%     With $s_i$, we denote $i^{\text{th}}$ diagonal entry of $\bS$. Note that since $ \eta\le 1/\norm{S}{\text{op}}$, for all $i$, $\eta s_i  \le 1$.  Consider $i^{\text{th}}$ diagonal term (which is non-zero) of the diagonal matrix $\bA$, we have 
%     \begin{align}
%         \bA_{ii} = \frac{1}{s_i} \left(  1 - (1 - s_i \eta)^t - \frac{t \eta s_i}{1 + t \eta s_i } \right) &=  \frac{1 - (1 - s_i \eta)^t}{s_i} \left( \underbrace{ 1 - \frac{t \eta s_i}{(1 + t \eta s_i)(1 - (1 - s_i \eta)^t)}}_{\RN{2}} \right) \\ 
%          &\le \frac{1}{2}\left[ \frac{1 - (1 - s_i \eta)^t}{ s_i} \right] \tag*{(Using \lemref{lem:ineq_soln} (b))} \,.
%     \end{align} 
%     Additionally, we can also show the following upper bound on term 2: 
%     \begin{align}
%          1 - \frac{t \eta s_i}{(1 + t \eta s_i)(1 - (1 - s_i \eta)^t)} &= \frac{(1 + t \eta s_i)(1 - (1 - s_i \eta)^t) - t \eta s_i }{(1 + t \eta s_i)(1 - (1 - s_i \eta)^t)} \\
%          & \le  \frac{ 1 -  (1 - s_i \eta)^t - t \eta s_i (1 - s_i \eta)^t}{(1 + t \eta s_i)(1 - (1 - s_i \eta)^t)} \\
%          & \le \frac{1}{t\eta s_i} \,. \tag{Using \lemref{lem:ineq_soln} (a)}
%         %  &\le \frac{1}{2}\left[ \frac{1 - (1 - s_i \eta)^t}{ s_i} \right] \tag*{(Using \lemref{lem:ineq_soln})} \,.
%     \end{align} 

%     Combining both the upper bounds on each diagonal entry $\bA_{ii}$, we have 
%     \begin{align}
%     \bA \preceq c_1(\eta, t) \cdot \bS^{-1} ( \bI - (\bI - \eta \bS)^t ) \,, \label{eq:upperbound_diagonal}
%     \end{align}
%     where $c_1(\eta, t ) = \min(0.5, \frac{1}{t s_i \eta })$. Plugging this into \eqref{eq:train_GD_rel}, we have 
%     \begin{align}
%         \Expt{ x \sim \calS }{\left(f_t(x) - \wt f_\lambda (x)\right)^2} &\le c(\eta, t) \cdot \by^T \bV \bX  \left( \bS^{-1} ( \bI - (\bI - \eta \bS)^t ) \right)^2 \bS \bV^T \bX^T \by \\
%         &=   c(\eta, t) \cdot \by^T \bV \bX  \left( \bS^{-1} ( \bI - (\bI - \eta \bS)^t ) \right) \bS \left( \bS^{-1} ( \bI - (\bI - \eta \bS)^t ) \right) \bV^T \bX^T \by \\
%         & =  c(\eta, t) \cdot \norm{\bX w_t}{2}^2 \\
%         &= c(\eta, t) \cdot  \Expt{ x \sim \calS }{\left(f_t(x) \right)^2} \,,
%     \end{align}
%     where $c(\eta, t ) = \min(0.25, \frac{1}{t^2 s^2_i \eta^2 })$.

%     \textbf{Part 2 {} {}} With $\bSigma$, we denote the underlying true covariance matrix. We now consider the squared difference of output at an unseen point: 
%     \begin{align}
%         \Expt{ x \sim \calD_{\calX} }{\left(f_t(x) - \wt f_\lambda (x)\right)^2} &= \Expt{x \sim \calD_{\calX}}{\norm{x^T w_t - x^T \wt w_\lambda}{2}} \\
%         &=   \norm{x^T \bV \bS ^{-1} ( \bI - (\bI - \eta \bS)^t ) \bV^T \bX^T \by - x^T \bV (\bS + \lambda \bI )^{-1} \bV^T \bX^T \by }{2} \\
%         &= \norm{x^T \bV \left(\bS ^{-1} ( \bI - (\bI - \eta \bS)^t ) - (\bS + \lambda \bI )^{-1} \right) \bV^T \bX^T \by  }{2} \\
%         &= \by^T \bV \bX \left( \bS ^{-1} ( \bI - (\bI - \eta \bS)^t ) - (\bS + \lambda \bI )^{-1} \right) \bV^T \bSigma \bV \\ &\qquad \qquad \qquad \qquad \qquad \left( (\bI - (\bI - \eta \bS)^t ) - (\bS + \lambda \bI )^{-1} \right) \bV^T \bX^T \by \\
%         &\le \sigma_{\text{max}} \cdot \by^T \bV \bX \left( \underbrace{\bS ^{-1} ( \bI - (\bI - \eta \bS)^t ) - (\bS + \lambda \bI )^{-1}}_{\RN{1}} \right)^2 \bV^T \bX^T \by \,, \label{eq:test_GD_rel}
%         %  (\bX \bV \bS ^{-1} ( \bI - (\bI - \eta \bS)^k ) \bV^T \bX^T \by)^T \bX \bV \bS ^{-1} ( \bI - (\bI - \eta \bS)^k ) \bV^T \bX^T \by
%     \end{align}
%     where $\sigma_{\text{max}}$ is the maximum eigenvalue of the underlying covariance matrix $\bSigma$. Using the upper bound on term 1 in \eqref{eq:upperbound_diagonal}, we have 
%     \begin{align}
%         \Expt{ x \sim \calD_{\calX} }{\left(f_t(x) - \wt f_\lambda (x)\right)^2} &\le \sigma_{\text{max}} \cdot c(\eta, t) \cdot \by^T \bV \bX  \left( \bS^{-1} ( \bI - (\bI - \eta \bS)^t ) \right)^2 \bV^T \bX^T \by \\
%         &=   \kappa \cdot c(\eta, t) \cdot \sigma_{\text{min}}\cdot \norm{\bV \left( \bS^{-1} ( \bI - (\bI - \eta \bS)^t ) \right) \bV^T \bX^T \by}{2}^2 \\
%         &\le \kappa \cdot c(\eta, t) \cdot \left[ \bV \left( \bS^{-1} ( \bI - (\bI - \eta \bS)^t ) \right) \bV^T \bX^T \right]^T \bSigma \\
%         &\qquad \qquad \qquad \qquad \qquad \left[ \bV \left( \bS^{-1} ( \bI - (\bI - \eta \bS)^t ) \right) \bV^T \bX^T \right] \by \\
%         & = \kappa \cdot c(\eta, t) \cdot \Expt{x \sim \calD_{\calX}}{\norm{x^T w_t}{2}} \,.
%     \end{align}
% % 
% % 
%     % Since $ \eta\le 1/\norm{S}{\text{op}}$, invoking \lemref{lem:ineq_soln} to upper bound term 1 with
% \end{proof}


% \newpage
% \section{Additional experiments and details}\label{app:exp}
% \newcommand\tab[1][1cm]{\hspace*{#1}}

% \subsection{Datasets} \label{sec:app_dataset}

% \textbf{Toy Dataset {} {}} Assume fixed constants $\mu$ and $\sigma$. For a given label $y$, we simulate features $x$ in our toy classification setup as follows: 
% \begin{align*}
%     x \defeq \texttt{concat} \left[ x_1, x_2\right] \quad \text{where} \quad  x_1 \sim  \calN( y \cdot \mu, \sigma^2 I_{d \times d}) \ \  \text{and} \ \  x_1 \sim  \calN( 0, \sigma^2 I_{d \times d}) \,.
% \end{align*}  
% % where $y$ is the true label and $x$ is the corresponding feature vector. 
% In experiements throughout the paper, we fix dimention $d=100$, $\mu = 1.0 $, and $\sigma = \sqrt{d}$. Intuitively, $x_1$ carries the information about the underlying label and $x_2$ is additional noise independent of the underlying label. 

% \textbf{CV datasets {} {}} We use MNIST~\citep{lecun1998mnist} and CIFAR10~\cite{krizhevsky2009learning}. 
% % For binary tasks, 
% We produce a binary variant from the multiclass classification problem by mapping classes $\{0,1,2,3,4\}$ to label $1$ and $\{ 5,6,7,8,9\}$ to label $-1$. For CIFAR dataset, we also use the standard data augementation of random crop and horizontal flip. PyTorch code is as follows: 

% \texttt{(transforms.RandomCrop(32, padding=4),\\
% \tab transforms.RandomHorizontalFlip())}

% \textbf{NLP dataset {} {}} We use IMDb Sentiment analysis~\citep{maas2011learning} corpus.  

% \subsection{Architecture Details} 

% All experiments were run on NVIDIA GeForce RTX 2080 Ti GPUs. We used PyTorch~\citep{NEURIPS2019a9015} and Keras with Tensorflow~\citep{abadi2016tensorflow} backend for experiments. 
% % , ELMo embeddings~\citep{Peters:2018}, and Hugging Face Transformers~\citep{wolf-etal-2020-transformers}. 

% \textbf{Linear model {} {}} For the toy dataset, we simulate a linear model with scalar output and the same number of parameters as the number of dimensions.   

% \textbf{Wide nets {} {}} To simulate the NTK regime, we experiment with $2-$layered wide nets. The PyTorch code for 2-layer wide MLP is as follows: 


% \texttt{ nn.Sequential( \\
% \tab     nn.Flatten(),\\
% \tab    nn.Linear(input\_dims, 200000, bias=True),\\
% \tab    nn.ReLU(),\\
% \tab    nn.Linear(200000, 1, bias=True)\\
% \tab     )}


% We experiment both (i) with the first layer fixed at random initialization; (ii)  and updating both layers' weights.     

% \textbf{Deep nets for CV tasks {} {}} We consider a 4-layered MLP. The PyTorch code for 4-layer MLP is as follows: 

% \texttt{ nn.Sequential(nn.Flatten(), \\
% \tab        nn.Linear(input\_dim, 5000, bias=True),\\
% \tab        nn.ReLU(),\\
% \tab        nn.Linear(5000, 5000, bias=True),\\
% \tab        nn.ReLU(),\\
% \tab        nn.Linear(5000, 5000, bias=True),\\
% \tab        nn.ReLU(),\\
% % \tab        nn.Linear(5000, 5000, bias=True),\\
% % \tab        nn.ReLU(),\\
% \tab        nn.Linear(1024, num\_label, bias=True)\\
% \tab        )}

% For MNIST, we use $1000$ nodes instead of $5000$ nodes in the hidden layer. 
% % 
% We also experiment with convolutional nets. In particular, we use ResNet18 \citep{he2016deep}. Implementation adapted from:  \url{https://github.com/kuangliu/pytorch-cifar.git}. 

% \textbf{Deep nets for NLP {} {}} We use a simple LSTM model with embeddings intialized with ELMo embeddings~\citep{Peters:2018}. Code adapted from: \url{https://github.com/kamujun/elmo_experiments/blob/master/elmo_experiment/notebooks/elmo_text_classification_on_imdb.ipynb} 

% We also evaluate our bounds with a BERT model. In particular, we fine-tune an off-the-shelf uncased BERT model~\citep{devlin2018bert}. Code adapted from Hugging Face Transformers~\citep{wolf-etal-2020-transformers}: \url{https://huggingface.co/transformers/v3.1.0/custom_datasets.html}. 


% \subsection{Additonal experiments}

% 1. SGD with linear models on cross entropy and MSE loss. 

% 2. CE loss and SGD. GD with MSE loss 

% 3. Binary MNIST with MLP. multiclass MNIST  

% \textbf{Results on CIFAR 10 {} {}} 
% % 
% We plot epoch wise error curve for results in \tabref{table:multiclass}. We observe the same trend as in \figref{fig:error_CIFAR10}. Additionally, we plot an \emph{oracle bound} obtained by tracking the error on mislabeled data which nevertheless were predicted as true label. To obtain an exact emprical value of the oracle bound, we need underlying true labels for the randomly labeled data. 
% % Note that our bound in \thmref{thm:multiclass_ERM}, lower bounds the accuracy as predicted by the oracle bound. 
% While with just access to extra unlabeled data we cannot calculate oracle bound, we note that the oracle bound is very tight and never violated in practice underscoring an importamt aspect of generalization in multiclass problems. This highlight that even a stronger conjecture may hold in multiclass classification, i.e., error on mislabeled data (where nevertheless true label was predicted) lower bounds the population error on the distribution of mislabeled data and hence, the error on (a specific) mislabeled portion predicts the population accuracy on clean data. 
% % 
% On the other hand, the dominating term of in \thmref{thm:multiclass_ERM} is loose when compared with the oracle bound. The main reason, we believe is the pessimistic upper bound in \eqref{eq:lemma1_final_multi_prev} in the proof of \lemref{lem:fit_mislabeled_multi}. We leave an investigation on this gap for future. 
% % of fit 

% % However, oracle bound highlights two . One,  



% \begin{figure}[h]
%     \centering 
%     % \vspace{-15pt}
%     % \includegraphics[width=0.9\linewidth]{example-image-a}
%     \includegraphics[width=0.4\linewidth]{figures/CIFAR10-FNN.pdf} \hfil
%     \includegraphics[width=0.4\linewidth]{figures/CIFAR10-Resnet.pdf}
%     % \includegraphics[width=0.9\linewidth]{figures/{CIFAR10_rn=0.1_lr=0.2_wd=0.005}.png}
%     % \vspace{-10pt}
%     \caption{ Per epoch curves for CIFAR10 corresponding results in \tabref{table:multiclass}. As before, we just plot the dominating term in the RHS of \eqref{eq:multiclass_ERM} as predicted bound. Additionally, we also plot the predicted lower bound by the error on mislabeled data which nevertheless were predicted as true label. We refer to this as ``Oracle bound''. See text for more details. 
%     % 
%     % except for the stopping point. 
%     % The bound predicted by RATT (RHS in \eqref{eq:multiclass_ERM}) is vacuous. 
%     }\label{fig:error_epoch_CIFAR10}
%     % \vspace{-15pt}
% \end{figure}


% \textbf{Results on CIFAR 100 {} {}} 
% % 
% On CIFAR100, our bound in \eqref{eq:multiclass_ERM} yields vacous bounds. However, the oracle bound as explained above yields tight guarantees in the initial phase of the learning (i.e., when learning rate is less than $0.1$). 

% \begin{figure}[h]
%     \centering 
%     % \vspace{-15pt}
%     % \includegraphics[width=0.9\linewidth]{example-image-a}
%     \includegraphics[width=0.4\linewidth]{figures/CIFAR100-Resnet.pdf}
%     % \includegraphics[width=0.9\linewidth]{figures/{CIFAR10_rn=0.1_lr=0.2_wd=0.005}.png}
%     % \vspace{-10pt}
%     \caption{ Predicted lower bound by the error on mislabeled data which nevertheless were predicted as true label with ResNet18 on CIFAR100. We refer to this as ``Oracle bound''. See text for more details. 
%     % 
%     % except for the stopping point. 
%     The bound predicted by RATT (RHS in \eqref{eq:multiclass_ERM}) is vacuous. 
%     }\label{fig:error_CIFAR100}
%     % \vspace{-15pt}
% \end{figure}


% % \paragraph{Experiments on CIFAR100} 



% \subsection{Hyperparameter Details}


% \textbf{\figref{fig:error_CIFAR10} {} {}} We use clean training dataset of size $40,000$. We fix the amount of unlabeled data at $20\%$ of the clean size, i.e. we include additional $8,000$ points with randomly assigned labels. We use test set of $10,000$ points. For both MLP and ResNet, we use SGD with an initial learning rate of $0.1$ and momentum $0.9$. We fix the weight decay parameter at $5\times 10^{-4}$. After $100$ epochs, we decay the learning rate to $0.01$. We use SGD batch size of $100$. 

% \textbf{\figref{fig:error_binary} (a) {} {}} We obtain a toy dataset according to the process described in \secref{sec:app_dataset}. We fix $d=100$ and create a dataset of $50,000$ points with balanced classes. Moreover, we sample additional covariates with the same procedure to create randomly labeled dataset. For both SGD and GD training, we use a fixed learning rate $0.1$.    

% \textbf{\figref{fig:error_binary} (b) {} {}} Similar to binary CIFAR, we use clean training dataset of size $40,000$ and fix the amount of unlabeled data at $20\%$ of the clean dataset size. To train wide nets, we use a fixed learning of $0.001$ with GD and SGD. We decide the weight decay parameter and the early stopping point that maximizes our generalization bound (i.e. without peeking at unseen data ).  We use SGD batch size of $100$. 

% \textbf{\figref{fig:error_binary} (c) {} {}} With IMDb dataset, we use a clean dataset of size $20,000$ and as before, fix the amount of unlabeled data at $20\%$ of the clean data. To train ELMo model, we use Adam optimizer with a fixed learning rate $0.01$ and weight decay $10^{-6}$ to minimize cross entropy loss. We train with batch size $32$ for 3 epochs. To fine-tune BERT model, we use Adam optimizer with learning rate $5\times 10^{-5}$ to minimize cross entropy loss. We train with a batch size of $16$ for 1 epoch.    

% \textbf{\tabref{table:multiclass} {} {}} For multiclass datasets, we train both MLP and ResNet with the same hyperparameters as described before. We sample a clean training dataset of size $40,000$ and fix the amount of unlabeled data at $20\%$ of the clean size. We use SGD with an initial learning rate of $0.1$ and momentum $0.9$. We fix the weight decay parameter at $5\times 10^{-4}$. After $30$ epochs for ResNet and after $50$ epochs for MLP, we decay the learning rate to $0.01$.  We use SGD with batch size $100$. 
% For \figref{fig:error_CIFAR100}, we use the same hyperparameters as 
% CIFAR10 training, except we now decay learning rate after $100$ epochs. 


% In all experiments, to identify the best possible accuracy on just the clean data, we use the exact same set of hyperparamters except the stopping point. We choose a stopping point that maximizes test performance. 

% \subsection{Summary of experiments }

% \begin{center}
%     \begin{table}[H] 
%         \centering
%         \begin{tabular}{|c|c|c|c|} 
%         \hline
%         Classification type & Model category & Model & Dataset  \\ [0.5ex] 
%         \hline
%         \hline
%         \multirow{9}{*}{Binary} & Low dimensional & Linear model & Toy Gaussain dataset  \\
%                         \cline{2-4}
%                          & \multirow{1}{*}{Overparameterized linear nets} 
%                         %  & Linear model & Toy Gaussain dataset \\
%                         %  \cline{3-4}
%                         %  & & 2-layer wide net& Toy Gaussain dataset \\
%                         %  \cline{3-4}
%                          & 2-layer wide net & Binary MNIST \\
%                          \cline{2-4}                 
%                          & \multirow{6}{*}{Deep nets} & \multirow{2}{*}{MLP} & Binary MNIST \\
%                          \cline{4-4}
%                          & &  & Binary CIFAR \\
%                          \cline{3-4}
%                          &  & \multirow{2}{*}{ResNet} & Binary MNIST \\
%                          \cline{4-4}
%                          & &  & Binary CIFAR \\
%                          \cline{3-4}
%                          &  & ELMo-LSTM model & IMDb Sentiment Analysis \\
%                          \cline{3-4}
%                          & & BERT pre-trained model & IMDb Sentiment Analysis \\
%         \hline
%         \multirow{5}{*}{Multiclass} & \multirow{5}{*}{Deep nets} & \multirow{2}{*}{MLP} & MNIST \\
%                         \cline{4-4} 
%                         & & & CIFAR10 \\                   
%                         \cline{3-4}
%                          &   & \multirow{3}{*}{ResNet} & MNIST \\
%                          \cline{4-4}
%                          &   & & CIFAR10 \\
%                          \cline{4-4}
%                          &   & & CIFAR100 \\
%         \hline
%         \end{tabular}
%         % \caption{Summary of experiments performed} \label{table:experiments}
%     \end{table}    
%     % \footnotetext[6]{We use both MSE loss and cross-entropy loss.}
%     % \footnotetext[6]{We try 2 variants: one with a fixed first layer and the other with both layers trainable.}
% \end{center}

% \newpage
% \section{Proof of \lemref{lem:stability_error}} \label{app:proof_lem_error}

% \begin{proof}[Proof of \lemref{lem:stability_error}]
%     Recall, we have a training set $S \cup \wt S_C$. We defined leave-one-out error on mislabeled points as $$\error_{\text{LOO}(\wt S_M) } = \frac{\sum_{(x_i, y_i) \in \wt S_M} \error( f_{(i)}( x_i), y_i)}{ \abs{\wt S_M }} \,, $$
%     where $f_{(i)} \defeq f(\calA, (S \cup \wt S)_{(i)})$. Define $S^\prime \defeq S \cup \wt S$. Assume $(x,y)$ and $(x^\prime,y^\prime)$ as i.i.d. samples from ${\calDm}$. 
%     Using Lemma 25 in \citet{bousquet2002stability}, we have
%     \begin{align*}
%         \Expo{ \left( \error_{\calDm}(\wh f) -\error_{\text{LOO}(\wt S_M) } \right)^2 } \le & \Expt{ S^\prime, (x,y), (x^\prime,y^\prime) }{ \error(\wh f(x), y ) \error(\wh f(x^\prime), y^\prime )} - 2 \Expt{ S^\prime, (x,y) }{ \error(\wh f(x), y ) \error(f_{(i)}(x_i), y_i )} \\
%         & + \frac{m_1-1}{m_1}\Expt{ S^\prime }{  \error(f_{(i)}(x_i), y_i )  \error(f_{(j)}(x_j), y_j )} + \frac{1}{m_1} \Expt{ S^\prime }{  \error(f_{(i)}(x_i), y_i ) } \,. \numberthis \label{eq:main_reln}
%     \end{align*}
%     We can rewrite the equation above as : 
%     \begin{align*}
%         \Expo{ \left( \error_{\calDm}(\wh f) -\error_{\text{LOO}(\wt S_M) } \right)^2 } \le &  \, \underbrace{\Expt{ S^\prime, (x,y), (x^\prime,y^\prime) }{ \error(\wh f(x), y ) \error(\wh f(x^\prime), y^\prime ) - \error(\wh f(x), y ) \error(f_{(i)}(x_i), y_i )}}_{\RN{1}} \\
%         & + \underbrace{\Expt{ S^\prime }{  \error(f_{(i)}(x_i), y_i )  \error(f_{(j)}(x_j), y_j ) -  \error(\wh f(x), y ) \error(f_{(i)}(x_i), y_i )}}_{\RN{2}} \\ &+ \underbrace{\frac{1}{m_1} \Expt{ S^\prime }{  \error(f_{(i)}(x_i), y_i ) - \error(f_{(i)}(x_i), y_i )  \error(f_{(j)}(x_j), y_j ) }}_{\RN{3}} \,. \numberthis \label{eq:main_reln2}
%     \end{align*}
    
%     We will now bound term $\RN{3}$.  Using Schwarz's inequality, we have
    
%     \begin{align}
%         \Expt{ S^\prime }{  \error(f_{(i)}(x_i), y_i ) - \error(f_{(i)}(x_i), y_i )  \error(f_{(j)}(x_j), y_j ) }^2 &\le  \Expt{ S^\prime }{  \error(f_{(i)}(x_i), y_i ) }^2 \Expt{S^\prime}{1 -   \error(f_{(j)}(x_j), y_j ) }^2 \\
%         &\le \frac{1}{4} \label{eq:term1_lem12}
%     \end{align}
    
%     Note that since $(x_i,y_i)$, $(x_j ,y_j )$, $(x,y)$, and $(x^\prime, y^\prime)$ are all from same distribution $\calDm$, we directly incorporate the bounds on term $\RN{1}$ and $\RN{2}$ from proof of Lemma 9 in \citet{bousquet2002stability}. Combining that with \eqref{eq:term1_lem12} and our definition of hypothesis stability in \codref{cond:hypothesis_stability}, we have the required claim. 
    
    
%     % We now re-write term $\RN{1}$ as
%     % \begin{align*}
%     %         &\Expt{S^\prime, (x,y), (x^\prime,y^\prime) }{ \error(\wh f(x), y ) \error(\wh f(x^\prime), y^\prime ) - \error(\wh f(x), y ) \error(f_{(i)}(x_i), y_i )} \\ & \qquad = \Expt{ S^\prime, (x,y), (x^\prime,y^\prime) }{ \error(\wh f(x), y ) \error(\wh f  (x^\prime), y^\prime ) - \error(\wh f ^\prime(x), y ) \error(f_{(i)}(x^\prime), y^\prime )} \tag{Exchanging $(x_i, y_i)$ with $(x^\prime, y^\prime)$ in the second term} \\
%     %         & \qquad = \Expt{ S^\prime, (x,y), (x^\prime,y^\prime) }{  \left(\error(\wh f(x), y )-  \error(f_{(i)}(x), y ) \right) \error(\wh f  (x^\prime), y^\prime )  } \\
%     %         & \qquad  + \Expt{ S^\prime, (x,y), (x^\prime,y^\prime) }{  \left(\error(f_{(i)}(x), y ) -\error(\wh f ^\prime(x), y ) \right) \error(\wh f  (x^\prime), y^\prime )}  \\
%     %         & \qquad +\Expt{ S^\prime, (x,y), (x^\prime,y^\prime) }{  \left( \error(\wh f  (x^\prime), y^\prime ) -  \error(f_{(i)}(x^\prime), y^\prime ) \right) \error(\wh f ^\prime(x), y ) }  \,, \numberthis \label{eq:term1_final}
%     % \end{align*}
%     % where $\wh f^\prime$ is the classifier obtained by training on $ S^\prime_{(i)} \cup \{ (x^\prime, y^\prime) \} $. Similarly we can re-write term $\RN{2}$ as 
%     % \begin{align*}
%     %     & \Expt{ S^\prime }{  \error(f_{(i)}(x_i), y_i )  \error(f_{(j)}(x_j), y_j ) -  \error(\wh f(x), y ) \error(f_{(i)}(x_i), y_i )} \\
%     %     &\quad  = \Expt{ S^\prime, (x,y), (x^\prime,y^\prime)}{  \error(f^{\prime\prime}_{(i)}(x), y )  \error(f_{(j)}^{\prime}(x^\prime), y^\prime ) -  \error(\wh f(x), y ) \error(f_{(i)}(x_i), y_i )} \tag{Exchanging $(x_i, y_i)$ with $(x, y)$ and $(x_j, y_j)$ with $(x^\prime, y^\prime)$ in the first term}\\
%     %     &\quad = \Expt{ S^\prime, (x,y), (x^\prime,y^\prime)}{  \error(f^{\prime\prime}_{(j)}(x), y )  \error(f_{(i)}^{\prime}(x^\prime), y^\prime ) -  \error(\wh f^\prime (x), y ) \error(f^\prime_{(j)}(x^\prime), y^\prime )} \tag{Exchanging $(x_i, y_i)$ and $(x_j, y_j)$ and then replacing $(x_j, y_j)$ with $(x^\prime, y^\prime)$ in the second term} \\
%     %     & \quad = \Expt{ S^\prime, (x,y), (x^\prime,y^\prime) }{  \left( \error(f_{(i)}^{\prime}(x^\prime), y^\prime )   -  \error(\wh f^{\prime\prime}  (x^\prime), y^\prime ) \right)  \error(f^{\prime\prime}_{(j)}(x), y )   } \\
%     %     & \quad  + \Expt{ S^\prime, (x,y), (x^\prime,y^\prime) }{  \left( \error(f^{\prime\prime}_{(j)}(x), y )  -\error(\wh f ^\prime(x), y ) \right) \error(\wh f^{\prime\prime}  (x^\prime), y^\prime )  }  \\
%     %     & \quad+ \Expt{ S^\prime, (x,y), (x^\prime,y^\prime) }{  \left( \error(\wh f^{\prime\prime}  (x^\prime), y^\prime )  -  \error(f^\prime_{(j)}(x^\prime), y^\prime ) \right)  \error(\wh f^\prime (x), y ) }   \\
%     %     & \quad = \Expt{ S^\prime, (x,y), (x^\prime,y^\prime) }{  \left( \error(f_{(i)}^{\prime}(x^\prime), y^\prime )   -  \error(\wh f (x^\prime), y^\prime ) \right)  \error(f_{(i)}(x_j), y_j )   } \\
%     %     & \quad  + \Expt{ S^\prime, (x,y), (x^\prime,y^\prime) }{  \left( \error(f^{\prime\prime}_{(j)}(x), y )  -\error(\wh f (x), y ) \right) \error(\wh f^{\prime\prime}  (x_j), y_j )  }  \\
%     %     & \quad+ \Expt{ S^\prime, (x,y), (x^\prime,y^\prime) }{  \left( \error(\wh f^{\prime\prime}  (x^\prime), y^\prime )  -  \error(f^\prime_{(j)}(x^\prime), y^\prime ) \right)  \error(\wh f^\prime (x^\prime), y^\prime ) }  \,, \numberthis \label{eq:term2_final}
%     % \end{align*}
%     % where $f^{\prime\prime}_{(j)}$ is trained on $S^\prime_{(j,i)} \cup {(x,y)}$, $f^{\prime}_{(i)}$ is trained on $S^\prime_{(j,i)} \cup {(x^\prime,y^\prime)}$, and $\wh f^{\prime\prime} $ is trained on $S^\prime_{(j)} \cup {(x,y)}$. Note in the last line we replaced $(x,y)$ by $(x_j, y_j)$ in the first term, replaced $(x^\prime,y^\prime)$ by $(x_j, y_j)$ in the second term and exchanged $(x_i,y_i)$ with $(x_j,y_j)$ and also $(x,y)$ and $(x^\prime, y^\prime)$
    
    
% \end{proof}
\bibliographystyle{plain}
\bibliography{reference}
\end{document}
