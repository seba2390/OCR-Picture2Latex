
\begin{abstract}
    In this work, we consider the Fokker-Planck equation of the Nonlinear Noisy Leaky Integrate-and-Fire (NNLIF) model for neuron networks. Due to the firing events of neurons at the microscopic level, this Fokker-Planck equation contains dynamic boundary conditions involving specific internal points. To efficiently solve this problem and explore the properties of the unknown, we construct a flexible numerical scheme for the Fokker-Planck equation in the framework of spectral methods that can accurately handle the dynamic boundary condition. This numerical scheme is stable with suitable choices of test function spaces, and asymptotic preserving, and it is easily extendable to variant models with multiple time scales. We also present extensive numerical examples to verify the scheme properties, including order of convergence and time efficiency, and explore unique properties of the model, including blow-up phenomena for the NNLIF model and learning and discriminative properties for the NNLIF model with learning rules.
\end{abstract}

\vspace*{4mm}
  \noindent {\bf Key words:}  Integrate-and-Fire model; Fokker-Planck equation; neuron network; spectral methods.
  \noindent {\bf Mathematics Subject Classification:} 35Q92; 65M70; 92B20
  

\section{Introduction}\label{sec:introduction}

In recent years, there has been a growing interest in studying large-scale neuron network models, e.g. \cite{nykamp2000population}\cite{caceres2011analysis}\cite{renart2004mean}\cite{delarue2015global}, placing a greater emphasis on the proper mathematical tools for analyzing and simulating the dynamics of such networks. The underlying formulation of these models is based on deterministic or stochastic differential equations which describe the activities of neuron ensembles.

In this article, we consider the Nonlinear Noisy Leaky Integrate-and-Fire (NNLIF)  model, which was originally proposed in the pioneering works \cite{brunel1999fast}\cite{brunel2000dynamics}, and it is one of the fundamental models in computational neuroscience. In the microscopic perspective, this model takes the membrane potential $v$ of neurons as the state variable, which is restricted by a given threshold value $V_F$ \cite{renart2004mean}\cite{sirovich2006dynamics}\cite{omurtag2000simulation}\cite{mattia2002population}. A defining characteristic of this model is the inclusion of firing events, which are described by a reset mechanism: when the membrane potential $v$ reaches the threshold value of $V_F$, a spike occurs, and the membrane potential is then reset to a lower value $V_R$. Moreover, the neurons within an ensemble interact with each other only through spikes. In the macroscopic perspective, this model is related to the Fokker-Planck equation \cite{nykamp2000population}\cite{liu2022rigorous}\cite{Liu2021investigating}, as follows:
\begin{equation}
    \label{eq:problem1}
    \begin{cases}
        \partial_{t}p+\partial_{v}(hp)-a\partial_{v v}p=0,\qquad v\in(-\infty,V_F]/\{V_R\},\\
        p(v,0)=p^0(v),\qquad p(-\infty,t)=p(V_F,t)=0,\\
        p(V^-_R,t)=p(V^+_R,t),\quad \partial _vp(V^-_R,t)=\partial _vp(V^+_R,t)+\frac{N(t)}{a},\\
    \end{cases}
\end{equation}
where the probability density function $p(v,t)$ represents the probability of finding a neuron at voltage $v$ and given time $t$, and $p^0(v)$ is the initial condition. The spiking behavior is described by the mean firing rate $N (t)$, which is implicitly given by
\begin{equation}
    \label{eq:Nt}
    N(t)=-a(N(t))\frac{\partial p}{\partial v}(V_F,t).
\end{equation}
The drift coefficient $h$ and diffusion coefficient $a$ are typically expressed as functions of the mean firing rate $N(t)$
\begin{equation}
    \label{eq:ha}
    h(v,N)=-v+bN,\qquad a(N)=a_0+a_1N,
\end{equation}
where $-v$ models the leaky mechanism and $b$ represents the connectivity of the network: $b > 0$ for excitatory-average networks and $b < 0$ for inhibitory-average networks. The connectivity of the network has an essential effect on the properties of \eqref{eq:problem1}, such as its steady states and blow-up phenomenon. Besides, $a$ stands for the amplitude of the noise, where $a_0 > 0$ and $a_1 \geq 0$. 
The probability density function $p(v, t)$  should satisfy the condition of conservation of mass
\begin{equation}
    \int_{-\infty}^{V_{F}} p(v, t) d v=\int_{-\infty}^{V_{F}} p^{0}(v) d v=1.
\end{equation}


In recent years, there have been significant progress in the numerical and analytical studies of the NNLIF models.
\cite{delarue2015global}\cite{delarue2015particle}\cite{caceres2020understanding} analyze the stability and asymptotic behavior from the point of view of microscopic stochastic differential equation (SDE). From the macroscopic perspective, \cite{caceres2011analysis}\cite{carrillo2013classical}\cite{carrillo2015qualitative} establish the existence theory of the Fokker-Planck equation \eqref{eq:problem1}, and the classical solution exists only when  the firing rate N(t) does not diverge. In \cite{caceres2011analysis}, the authors analyze the model's steady states and blow-up phenomenon. In \cite{dou2022dilating}, aiming at investigating the solution structure in the presence of the blow-up phenomenon, a new notion of generalized solution is proposed by introducing the dilated time scale.

Some variants of the model \eqref{eq:problem1} have been studied to incorporate more biological ingredients \cite{caceres2018analysis}, including multi-species populations (excitatory and inhibitory) \cite{caceres2016blow}, the refractory state \cite{caceres2014beyond}\cite{sharma2020discontinuous}, the transmission delay between neurons \cite{caceres2019global}\cite{hu2021structure}\cite{sharma2020discontinuous} and the age of the neuron \cite{dumont2016noisy}. Besides, there have also been multi-scale models with additional state variables. For example, the kinetic voltage-conductance model for neuron networks has been explored in \cite{caceres2011numerical}\cite{dou2022bounds}\cite{carrillo2022simplified}.  In \cite{he2022structure}\cite{perthame2017distributed}, the authors consider the learning behavior of the NNLIF model which is structured by the synaptic weights, and the distribution of the weights evolve according to the Hebbian learning rule. 

In this paper, we focus on the Fokker-Planck equation \eqref{eq:problem1}, and the primary goal is to investigate its efficient numerical approximation and applications to other variants. This Fokker-Planck equation is distinguished from other classical kinetic models due to its complex nonlinearity through the boundary flux and the dynamic boundary conditions. In spite of the existing results, the properties of this model are far from being fully understood, necessitating further numerical experiments to gain a deeper understanding. In \cite{caceres2011numerical}, the authors propose a numerical scheme combining the WENO-finite differences and the Chang-Cooper method. The numerical tests are mainly concerned with the blow-up phenomenon and the steady states. In \cite{hu2021structure}\cite{he2022structure}, the authors propose conservative and conditionally positivity-preserving schemes and show that the corresponding discrete entropy is dissipating in time. Besides, the finite element method and discontinuous Galerkin method are also applied to solve the Fokker-Planck type equations \cite{sharma2019numerical}\cite{sharma2020discontinuous}. Properly addressing the flux shift terms in the Fokker-Planck equation is of essential significance in the numerical approximation. In most cases, the flux offset term is implicitly included in the equation, requiring the modification of dynamic boundary conditions into equations with $\delta$ source terms for better implementation of numerical methods. Then the original problem \eqref{eq:problem1} is transformed into the following boundary value problem:
\begin{equation}
\label{eq:problem_delta}
    \begin{cases}
        \partial_{t}p+\partial_{v}(hp)-a\partial_{v v}p=N(t)\delta(v-V_R),\qquad v\in(-\infty,V_F],\\
        p(v,0)=p^0(v),\qquad p(-\infty,t)=p(V_F,t)=0,\\
    \end{cases}
\end{equation}
where $\delta(v)$ stands for the Delta function. Although  this transformation facilitates the construction of numerical schemes, it also causes certain restrictions due to the particularity of the $\delta$ function, such as requiring $V_R$ to fall on the grid point.


To achieve improved efficiency while properly handling the dynamic boundary conditions, we aim to construct a spectral approximation for the Fokker-Planck equation \eqref{eq:problem1}. The spectral method that we shall present relies on semi-globally differentiable and integrable basis functions, which accurately capture the dynamic boundary conditions in \eqref{eq:problem1} rather than complicating the equation with a $\delta$ source as in \eqref{eq:problem_delta}. 

There are two key factors that enable our scheme to meet the desired properties. First, the construction of the basis functions enforces the approximate solution to satisfy the dynamic boundary conditions exactly. Second, the time evolution of the approximate solution is determined by the Galerkin method, and there exist suitable choices of the test function spaces  that make the method stable and asymptotic preserving.


Beyond that, we perform systematic numerical tests to verify the convergence order of the method and to investigate the diverse solution properties such as the blow-up phenomenon and discrete relative entropy. Taking advantage of the scheme's flexibility, we apply it to the NNLIF model with learning rules proposed in \cite{perthame2017distributed}, testing the discrimination property and further exploring the learning behavior of the model.


The paper is structured as follows. In Section \ref{sec:weak_form}, we define the weak solution of \eqref{eq:problem1} and establish its relationship to the classical solution. In Section \ref{sec:scheme}, we introduce the numerical scheme for the NNLIF model based on spectral methods in detail and analyze the choices of different test functions in constructing the numerical solution. In Section \ref{sec:lr}, we introduce the NNLIF model with learning rules and apply the proposed method to the model. In Section \ref{sec:numerical_test}, we perform some numerical experiments, including the convergence order of the scheme, comparison with existing numerical methods, and some other numerical explorations.