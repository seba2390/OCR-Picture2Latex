\section{Conclusion} \label{sec:conclusion}


In this work, we have proposed a numerical scheme for approximating the Fokker-Planck equation using spectral methods for spatial discretization and successfully applied it to models with multiple time scales. Specific trial function space can guarantee that dynamic bounds are always satisfied, and suitable test function space is essential for ensuring stability and asymptotic-preserving. This method shows clear advantages regarding computational efficiency for high accuracy compared to existing methods. Besides essential convergence tests and assessments of computational efficiency, we execute assorted numerical examples exploring the response of various solutions. Subsequent to simulating the blow up phenomenon and relative entropy decay of the NNLIF model, we have studied the learning and discriminating behavior of the NNLIF model with learning rules when the input signal is time-dependent. The experimental results demonstrate that the learning behavior of the model obeys the general trend and provides clues for further research. Moving forward, we can still probe new numerical schemes for unbounded domains as well as explore more complex Fokker-Planck equations from neuroscience.



\section*{Acknowledgements}
The work of Z.Zhou is partially supported by the National Key R\&D Program
of China (Project No. 2021YFA1001200, 2020YFA0712000), and the National Natural Science Foundation of China (Grant No. 12031013, 12171013).
This work of Y.Wang is partially supported by the National Natural Science Foundation of China (Grant No. 12171026, U2230402 and 12031013), and Foundation of President of China Academy of Engineering Physics (YZJJZQ2022017).