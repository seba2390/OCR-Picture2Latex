\documentclass{article}

% if you need to pass options to natbib, use, e.g.:
\PassOptionsToPackage{numbers}{natbib}
% before loading nips_2018

% ready for submission
\usepackage[final]{neurips_2018}
% to compile a preprint version, e.g., for submission to arXiv, add
% add the [preprint] option:
% \usepackage[preprint]{nips_2018}

% to compile a camera-ready version, add the [final] option, e.g.:
% \usepackage[final]{nips_2018}

% to avoid loading the natbib package, add option nonatbib:
% \usepackage[nonatbib]{nips_2018}

\usepackage[utf8]{inputenc} % allow utf-8 input
\usepackage[T1]{fontenc}    % use 8-bit T1 fonts
\usepackage{hyperref}       % hyperlinks
\usepackage{url}            % simple URL typesetting
\usepackage{booktabs}       % professional-quality tables
\usepackage{amsfonts}       % blackboard math symbols
\usepackage{nicefrac}       % compact symbols for 1/2, etc.
\usepackage{microtype}      % microtypography
% ours 
\usepackage{epsfig}
\usepackage{graphicx}
\usepackage{amsmath}
\usepackage{amssymb}
\usepackage{subcaption}
\usepackage{float}
\usepackage{multirow}
% \usepackage{color}
\usepackage{comment}
\usepackage{wrapfig}

\usepackage[usenames, dvipsnames]{color}

\title{Dialog-based Interactive Image Retrieval}

% The \author macro works with any number of authors. There are two
% commands used to separate the names and addresses of multiple
% authors: \And and \AND.
%
% Using \And between authors leaves it to LaTeX to determine where to
% break the lines. Using \AND forces a line break at that point. So,
% if LaTeX puts 3 of 4 authors names on the first line, and the last
% on the second line, try using \AND instead of \And before the third
% author name.
\begin{comment}
\author{
  %% examples of more authors
  Xiaoxiao Guo$^{*1}$ \thanks{$^{*}$Contributed equally. 
  } \phantom{a} Hui Wu$^{*1}$ \phantom{a} Yu Cheng$^{1}$ \phantom{a}  Steven Rennie$^{2}$ \phantom{a} Rogerio Feris$^{1}$  \\
  $^{1}$IBM Research AI \phantom{AND} $^{2}$Fusemachines Inc. 
  %% Coauthor \\
  %% Affiliation \\
  %% Address \\
  %% \texttt{email} \\
  %% \And
  %% Coauthor \\
  %% Affiliation \\
  %% Address \\
  %% \texttt{email} \\
}
\end{comment}
\author{
  %% examples of more authors
  Xiaoxiao Guo$^{\dagger}$ \thanks{$^{\dagger}$These two authors contributed equally to this work. } \\
  IBM Research AI\\
  \texttt{xiaoxiao.guo@ibm.com}
  \And
  Hui Wu$^{\dagger}$ \\
  IBM Research AI\\
  \texttt{wuhu@us.ibm.com}
  \And
  Yu Cheng \\
  IBM Research AI\\
  \texttt{chengyu@us.ibm.com}
  \And
  Steven Rennie \\
  Fusemachines Inc.  \\
  \texttt{srennie@gmail.com}
  \And
 Gerald Tesauro \\
  IBM Research AI\\
  \texttt{gtesauro@us.ibm.com}
  \And
  Rogerio Schmidt Feris \\
  IBM Research AI\\
  \texttt{rsferis@us.ibm.com}
}

\renewcommand\footnotemark{}
\begin{document}
%\nipsfinalcopy
%is no longer used

\maketitle

\begin{abstract}
\begin{comment}
Inspired by the enormous growth of huge online media collections of many types (e.g. images, audio, video, e-books, etc.), and the paucity of intelligent retrieval systems, 
this paper introduces a novel approach to interactive visual content 
retrieval.  The proposed retrieval framework is guided by free-form natural language feedback from users, allowing for more natural and effective communication. 
Such a framework constitutes a multi-modal dialog protocol where in each dialog turn, a user submits a natural language request to a retrieval agent, which then attempts to retrieve the optimal object.
We formulate the retrieval task
as a reinforcement learning problem, and reward the dialog system for improving the 
rank of the target object during each dialog turn. This
framework can be applied to a variety of visual media types (images, videos, graphics, etc.),
and in this paper, we study in-depth its application on the task of interactive image retrieval.
To avoid the cumbersome and costly process of collecting human-machine conversations 
as the dialog system learns, we train the dialog system with 
a user simulator, which is itself trained to describe the differences between 
target and retrieved images.  The efficacy of our approach is demonstrated in a footwear image retrieval application.  Extensive experiments on both simulated and real-world data show that: 1) our proposed learning framework achieves better accuracy than other supervised and reinforcement learning baselines; and 2) user feedback based on natural language rather than pre-specified attributes leads to more effective retrieval results, and a more natural and expressive communication interface. 
\end{comment}

Existing methods for interactive image retrieval have demonstrated the merit of integrating
user feedback, improving retrieval results. However, most current systems
rely on restricted forms of user feedback, such as binary relevance responses, or
feedback based on a fixed set of relative attributes, which limits their impact. In this
paper, we introduce a new approach to interactive image search that enables users
to provide feedback via natural language, allowing for more natural and effective
interaction. We formulate the task of dialog-based interactive image retrieval as a
reinforcement learning problem, and reward the dialog system for improving the
rank of the target image during each dialog turn. To mitigate the cumbersome and
costly process of collecting human-machine conversations as the dialog system
learns, we train our system with a user simulator, which is itself trained to describe
the differences between target and candidate images. The efficacy of our approach
is demonstrated in a footwear retrieval application. Experiments on
both simulated and real-world data show that 1) our proposed learning framework
achieves better accuracy than other supervised and reinforcement learning baselines
and 2) user feedback based on natural language rather than pre-specified
attributes leads to more effective retrieval results, and a more natural and expressive
communication interface.
 
\end{abstract}

Reinforcement learning has achieved great success in areas such as Game-playing \citep{silver2018general,vinyals2019grandmaster}, robotics \cite{kober2013reinforcement}, large language models \citep{ouyang2022training}, etc.
However, due to safety concerns or physical limitations, in some real-world reinforcement learning problems, we must consider additional constraints that may influence the optimal policy and the learning process \citep{garcia2015comprehensive}.
% For example, a robotic arm must not take actions that may cause harm to itself or the environments.
A standard framework to handle such cases is the constrained Markov Decision Process (CMDP) \citep{altman1999constrained}.
Within the CMDP framework, the agent has to maximize
the expected cumulative reward while
obeying a finite number of constraints, which are usually in the form of expected cumulative cost criteria.

However, we are sometimes concerned with the problem with a continuum of constraints.
For example,
the constraints we meet might be time-evolving or subject to uncertain parameters, which
cannot be formulated as an ordinary CMDP
(see Examples \ref{Example_Time_Evolving} and  \ref{Example_Uncertain}).
In this paper we would study a generalized CMDP  
to address the above problem.  Because the constraints are not only infinite-number but also lie
in a continuous set,
the generalization is not trivial. Fortunately, we find that we can borrow the idea behind semi-infinite programming (SIP) \citep{remez1934determination, hettich1993semi} to deal with the semi-infinite constraints.
Accordingly, we propose \emph{semi-infinitely constrained Markov decision processes} (SICMDPs)
as a novel complement to the ordinary CMDP framework.
%More specifically,  an SICMDP model %, we consider 
%contains a continuum of constraints whereas an ordinary CMDP contains a finite number of constraints. 

%This generalization is natural but not trivial. However, we can brows the idea  
%The idea is quite natural and can be backtracked
%to the practice of extending linear programming to linear semi-infinite programming (LSIP) %\cite{remez1934determination, GobernaLSIO1998}.
%In addition, 
%As a complementary approach to the ordinary CMDP framework, 
%SICMDP can be used to model these problems  which cannot be described by a finite number of constraints
%that are not covered by .
%For example,
%the restrictions we consider can be time-evolving or subject to uncertain parameters
%, thus
%cannot be described by a finite number of constraints but a continuum of constraints 
%(see Examples \ref{Example_Time_Evolving} and  \ref{Example_Uncertain}).

We also present two reinforcement learning algorithms to solve SICMDPs called SI-CRL and SI-CPO, respectively.
SI-CRL is a model-based reinforcement learning algorithm designed for tabular cases, and SI-CPO is a policy optimization algorithm for non-tabular cases.
% and analyze its performance both theoretically and empirically.
The main challenge is that we need to deal with a continuum of constraints, thus reinforcement learning algorithms for ordinary CMDPs do not work anymore.
In SI-CRL, we tackle this difficulty by first transforming the reinforcement learning problem to an equivalent LSIP problem, which can then be solved using methods in the LSIP literature like the dual exchange methods \citep{Hu1990,reemtsen1998numerical}.
In SI-CPO, we resort to the idea of cooperative stochastic approximation developed in \cite{lan2020algorithms, wei2020comirror}.
As far as we know, we are the first to introduce tools from semi-infinitely programming (SIP) into the reinforcement learning community for solving constrained reinforcement learning problems.

% To the best of our knowledge, we are the first to apply tools from semi-infinitely programming (SIP) to solve reinforcement learning problems.
Furthermore, we give theoretical analysis for both SI-CRL and SI-CPO.
We decompose the error of SI-CRL into two parts: the statistical error from approximating the true SICMDP with an offline dataset and the optimization error due to the fact that the solution of the LSIP problem obtained by the dual exchange method is inexact.
On the optimization side, we show that the iteration complexity of SI-CRL is $O\left(\left\{\mathrm{diam}(Y)L\sqrt{|\gS|^2|\gA|m}/\left[(1-\gamma)\epsilon\right]\right\}^m\right)$.
On the statistical side, we show that the sample complexity of SI-CRL is $\widetilde O\left(\frac{|S|^2|A|^2}{\epsilon^2(1-\gamma)^3}\right)$ if the offline dataset is generated by a generative model, and $\widetilde O\left(\frac{|S||A|}{\nu_{\min} \epsilon^2(1-\gamma)^3}\right)$ if the dataset is generated by a probability measure $\nu$ as considered in \cite{chen2019information}.
Here $\widetilde O$ means that all logarithm terms are discarded.
For SI-CPO, things become a little more complicated because other than the statistical error and the optimization error, we also need to consider the function approximation error, which comes from imperfect policy parametrizations.
It is shown if the function approximation error can be controlled to $O(\epsilon)$ order, the iteration complexity of SI-CPO is $\widetilde{O}\left(\frac{1}{\epsilon^2(1-\gamma)^6}\right)$ and the sample complexity of SI-CPO is $\widetilde{O}(\frac{1}{\epsilon^4(1-\gamma)^{10}})$.
Here our iteration complexity bound is equivalent to a typical $\widetilde O(1/\sqrt{T})$ global convergence rate.

We perform a set of numerical experiments to illustrate the SICMDP model and validate our proposed algorithms.
Specifically, we examine two numerical examples, namely the discharge of sewage and ship route planning.
Through the discharge of sewage example, we show the advantage of the SICMDP framework over the CMDP baseline obtained by naive discretization in modeling realistic sequential decision-making problems.
Moreover, we demonstrate the effectiveness of the SI-CRL and SI-CPO algorithms in such tabular environments. 
In the ship route planning example, we illustrate the benefits of the SICMDP framework and the ability of the SI-CPO algorithm to address complex continuous control tasks involving continuous state spaces with modern deep reinforcement learning techniques.

% In summary, our contributions are listed as follows.
% First, we present the SICMDP model, which can be viewed as a generalization of the ordinary CMDP model.
% Second, we propose an algorithm to perform reinforcement learning for SICMDPs, which is called SI-CRL, and we believe that we are the first to apply tools from SIP
% to solve reinforcement learning problems.
% Third, we give a theoretical analysis of SI-CRL and identify both its sample complexity and iteration complexity.
% In addition, we perform numerical experiments to illustrate the SICMDP model and validate the SI-CRL algorithm.
% \{This paragraph can be removed!!! \}






The industry standard for pose edition is to create rigs, a collection of pieces of software designed to manipulate a character's skeleton. The rig describes the skeleton's bones, how they relate to each other, are constrained in their possible motion and are deformed. These rules are loosely specified and creating a good rig requires a detailed understanding of physics and anatomy, as well as technical and artistic skills. Rigging is thus a time consuming task even for experienced animators, and even more so in large scale productions which often require a different in-depth rig for each character in the cast.
Previous work has helped alleviate this difficulty by providing efficient tools to speed up/and or ease the rigging process, relying on inverse kinematics or data-driven methods.
\subsection{Character pose design}
\subsubsection{Inverse Kinematics (IK)}
IK solvers are a family of methods commonly used in robotics, engineering and computer graphics, in which the parameterization of a kinematic chain is determined from the position of its end effector.
They are a staple tool in pose design software, ensuring the respect of elementary constraints during pose edition. Their de-facto role is to guarantee the length of the limbs, and in some cases to enforce the orientation angle range of a joint.
Many IK solutions have been studied over the years \cite{aristidou_inverse_2018}; usually revolving around approximated linearizations or heuristics. 

Numerical methods require a set of iterations to achieve a satisfactory solution formulated by a cost function to be minimized.
IK solutions can generally be divided into three sub-categories: Jacobian \cite{Siciliano_Handbook_Robot_2007}, Newtonians \cite{cohen_ik_1996} and Heuristics. Most software implement heuristic methods such as Cyclic Coordinate Descent (CCD) \cite{wang_ccd_1991} or 
Forward-Backward Reaching IK (FABRIK) \cite{aristidou_fabrik:_2011} due to their simplicity and extensibility. 

The main drawback of 
these solvers is that they manipulate kinematic chains without taking into account many morphological aspects that make a pose more or less plausible. They offer a first level of help to users but are not sufficient to guarantee a realistic pose. Many joints constraints are dependent on each other and require subjective, human-made approximations.

\subsubsection{Data-driven pose edition}
Data-driven methods offer promising opportunities to solve these approximations. Using real-life data can help in modelling the complex inter-dependencies of skeletons and providing users with smarter edition tools.
While it is still an early field of research, some solutions have been studied. Wu \etal \cite{wu_posing_2009} propose a method for natural character posing from a large motion database. It employs adaptive KD-clustering to select a representative frame from a database and sparse approximations to accelerate training and posing. 
Huang \etal in \cite{Huang_IK_MGDM_2017} present a method based on the formulation of multi-variate Gaussian distribution models (MGDMs), which learn the joint constraints of a kinematic skeleton from motion capture data. 

Some work has also been dedicated to finding new editing interfaces. \modify{}{Instead of the usual setup manipulating joints directly, Guay \etal \cite{guay_line_2013} articulate a framework based on the conceptual "line of action" which describes the overall pose dynamics. They provide a mathematical definition of the line of action, and a interface in which the software modifies the pose to follow a user-provided line. In the same line of though} Garcia \etal \cite{garcia_sketching_2019} propose \modify{a method transforming doodle of trajectories (position and orientation over time) }{a virtual reality-based interface where the user's hands motion (position and orientation over time) are transformed} into sequences of actions and then into detailed character animations using a dataset of parametrized motion clips automatically fitted to the trajectory. 

% ==> DL et Latent Space. 
\subsection{Neural modelling of human motion}
Neural networks have received a great amount of attention over the last decade and shown impressive result in modelling complex data. Human motion has not been spared and deep learning methods have proven their capability of generating realistic motion in a number of difficult cases. 

The literature in neural-based animation include example in user-controlled character navigation \cite{Holden2017} and interactions with the environment \cite{starke_neural_2019}. 
Holden \etal \cite{Holden2020} also show that neural networks can be used to replace parts of existing data-driven methods, improving their scalability potential.
More recently, some work has also focused on improving smaller parts of the animation pipeline rather than replacing it completely. Berson et al. \cite{berson_intuitive_2020} leverage neural networks to provide an interactive system to edit facial animation. 

% Wrap up
Data-driven IK and pose editing can relieve animators from time-consuming, back-and-forth pose adjustments by applying constraints extracted from real-world data. Recently, neural-network-based approaches have demonstrated their ability to model the intricacies of human motion while scaling to large amount of data and retaining a fast inference time. In this paper we seek to take advantage of these properties to create an efficient posing tool, intuitively usable even by a inexperienced user.

The proposed segmentation-by-detection framework, as depicted in Figure \ref{fig:framework}, consists of a detection module and a segmentation module.
In detection stage, 2D slices (layered box) from the input volume are fed to the RPN. Based on the region proposals obtained from RPN, an attention model (block in orange) is formed. The input volume as well as the attention model are further processed in segmentation stage to get the refined anatomical segmentation. 
\vspace{1em} 

\begin{figure}[t]
\centering
\includegraphics[width=0.95\linewidth]{fig/framework.pdf}
\caption{Schematic representation of the segmentation-by-detection framework. The left part is the detection module while the segmentation module is followed on the right. The blue block denotes the input volume which is 3D ultrasound scan of femoral head. The output segmentation is in red.}
\label{fig:framework}
\end{figure}
% dana could you improve the figure. we can try to think together of better ways 

\noindent\textbf{Detection Module:} 
% dana : here you have to make the clarification that you have ground truth on the boxes (in implementation part)
The detection module follows an RPN architecture, a fully convolutional network which takes image slice as input and outputs object region candidates. 
We use the VGG-16 model as the backbone \cite{simonyan2014very} to learn convolutional features and an $3 \times 3$ spatial window to generate region proposals. At each sliding-window location, 9 anchors are predicted associated with different scales and aspect ratios. The last layer consists of a box-regression (reg) layer and a box-classification (cls) layer in parallel. The reg layer outputs 4 regression offsets, $ t = (t_x,t_y,t_w,t_h)$, denoting a scale-invariant translation as well as log-space height and width shift, where $x,y,w$ and $h$ specify two coordinates of the box center, width and height. The cls layer outputs two scores by softmax, related to probabilities of object and background for each proposal. We assign a positive label (of being object) to candidate which has an Intersection-over-Union (IoU) ratio higher than 0.7 with ground truth box. Note that an image slice may contain multiple object regions or none. 

The loss function of RPN follows the multi-task loss \cite{ren2015faster} which is defined as $L = L_{reg} + L_{cls}$. The regression loss, $L_{reg} = -\log p_{obj}$ is log loss and the classification loss,
\begin{equation} \label{eq:loss}
L_{cls} = \sum_{i \in \{x,y,w,h\}} smooth_{L_1} (t_i - t_i^*)
\end{equation}
is smooth $L_1$ loss where $t_i^*$ denotes the ground truth box for the target object. 
\vspace{1em}

\noindent\textbf{Segmentation Module:}
3D U-Net \cite{cciccek20163d} is utilized in the segmentation module as its outstanding performance in medical image segmentation. The u-shaped architecture consists of two paths: a contracting path, where each layer contains two $3\times3\times3$ convolutions followed by a rectified linear unit (ReLU) and then a max pooling, provides high resolution features. While, the symmetric expanding path for semantically richer features replaces max pooling with a upconvolution $2\times2\times2$ with stride of 2 in each dimension, and then two $3\times3\times3$ convolutions each followed by a ReLU. Skip connections between layers of equal resolution in the contracting path and the expanding path enables context information as well as precise localization.

Different from 3D U-Net, to incorporate the attention model detected by the RPN, our architecture takes as input both the volumetric image data and the candidate RoIs proposed by the RPN, concatenated as 3D volume. 
% dana not sure what you like to say below
% densely annotated
The attention model makes the network to focus on the potential RoIs and can reduce the interference of the surrounding noise.
The anatomical segmentation is then generated from a $1\times1\times1$ convolution which reduces the number of feature maps to the number of labels.  The energy function is computed by a pixel-wise softmax combined with the cross entropy loss.
% dana equation ??

\subsection{System and implementation Details}
The segmentation-by-detection approach adopts a cascade structure with two stages: detection and segmentation. The two networks are trained separately in an end-to-end manner. All the new layers are randomly initialized from zero-mean Gaussian distribution with standard deviations 0.01. Biases are initialized to 0. We use Caffe \cite{jia2014caffe} for the implementation and an NVIDIA Titan X GPU for training.

In the detection stage, we initialize the VGG-16 model by the pre-trained model for ImageNet classification \cite{russakovsky2015imagenet} and further fine-tune the model for our detection task. The input fed to the network are image slices with a fixed size of $184\times96$ and the corresponding ground truth boxes are generated from the annotation in the format of tight bounding boxes surrounding the segmentation contour (as illustrated in Figure \ref{fig:hip} (b), the boundary of white area). To optimize the energy function, stochastic gradient descent (SGD) is used. The global learning rate is set to 0.001, while a momentum of 0.9 and a weight decay of 0.0005 are used. The batch size is set to 256 and each mini-batch only contains the positive anchors for training. The region proposals are obtained from the reg path for each image slice. The attention model is then formed by concatenating all the detected regions, as binary masks, into a volume.

In the segmentation stage, we use the Adam optimizer \cite{kingma2014adam} to learn the network parameters. A global learning rate is set to 0.001 while the two momentum coefficients are set to 0.9 and 0.999 respectively. A batch size of 1 is used due to the memory constraints of the GPU. The network takes the volume data as well as the attention model as input. We train the network for a maximum of 30K iterations and reserve the learned weights with the best performance from every 1K iterations. 
\vspace{1em}

\noindent\textbf{Inference:}
At test time, the 2D slices from an input volume are first fed to the detection module. The attention model is obtained based on the output. Then the volume data as well as the attention model are fed to the segmentation module to get the pixel-wise prediction.




\section{Dataset: Relative Captioning}
\label{sec:dataset}

% In this section, we provide information on the dataset used for training the user simulator. 
Our user simulator aims 
to capture the rich and flexible language describing visual 
differences of any given image pair. The relative captioning dataset 
thus needs this property. We situated the data collection procedure
in a scenario of a shopping chatting session between a shopping assistant and a customer. The annotator was asked to take the role of the customer 
and provide a natural expression to inform the shopping assistant about the desired
product item. 
To promote more regular, specific, and relative user feedback, we provided a sentence prefix for the annotator to complete when composing their response to a retrieved
image. Otherwise the annotator response is completely free-form: no other constraints on the response were imposed.
We used Amazon Mechanical Turk %~\cite{buhrmester2011} 
to crowdsource the relative expressions. After manually removing erroneous annotations, 
we collected in total $10,751$ captions, with one caption per pair of images.

Interestingly, we observed that when the target image and the reference image are sufficiently different, users often directly describe the visual appearance of the target image, rather than using relative expressions (c.f. fourth example in Figure 7(b), Appendix A). This behavior mirrors the \emph{discriminative captioning} problem considered in \cite{vedantam2017}, where a method must take in two images and produce a caption that refers only to one of them. Relative and discriminative captioning are complementary, and in practice, both strategies are used, and so we augmented our dataset by pairing 3600 captions that were discriminative with additional dissimilar images. Our captioner and dialog-based interactive retriever are thus trained on both discriminative and relative captions, so as to be respectively more representative of and responsive to real users. Additional details about the dataset collection procedure and the analysis on dataset statistics are included in Appendix~\ref{sec:app_data} and Appendix~\ref{sec:dataset_analysis}.
% To facilitate future research on relative captioning and reproducible results, we will make our dataset and codes available for public use.

\newcommand{\twomoons}{{\tt Twomoons}}
\newcommand{\gauss}{{\tt Gauss}}
\newcommand{\sculpture}{{\tt Sculpture}}
\newcommand{\baseline}{{\tt Baseline}}
\newcommand{\MM}{{\tt MsgPassing}}
\newcommand{\blackboard}{{\tt Blackboard}}
\newcommand{\ncut}{\text{ncut}}
\newcommand{\chensays}[2][]{\textcolor{blue} {\textsc{Jiecao #1:} \emph{#2}}}

\section{Experiments}
In this section we present experimental results for  graph clustering in the message passing and blackboard models. We will compare the following three algorithms. (1) \baseline: each site sends all the data to the coordinator directly; (2) \MM: our algorithm in the message passing model (Section~\ref{sec:gcmessage}); (3) 
\blackboard: our algorithm in  the blackboard model (Section~\ref{sec:bb}).


%Since both of our algorithms are crucially based on the use of spectral scarification, our main focus in the experiments is to investigate to what extend the quality of the spectral clustering algorithms will be affected by using spectral sparsification, the saving of communication costs by using spectral sparsificaion, ...
%
%
%The goal of this experiment is not to demonstrate the effectiveness of the spectral clustering algorithm. We mainly want to investigate the following, 
%\begin{itemize}
%\item to what extend the quality of clustered results will be affected by using spectral sparsification.
%\item saving of communication costs by using spectral sparsifier.
%\item the affect of constants in algorithms of the message passing/blackboard model.
%\end{itemize}
%
%
%\subsection{The Setup}
%\paragraph{Reference Algorithms}
%We compare different algorithms in our experiment.

%Note that we can also run \MM~ in the blackboard model.

Besides giving the visualized results of these algorithms on various datasets, we also measure the qualities of the results via the {\em normalized cut}, defined as 
\[
\ncut(A_1, \ldots, A_{k}) = \frac{1}{2}\sum_{i\in[k]}\frac{w(A_i, V\backslash A_i)}{\vol(A_i)},
\]
 which is a standard objective function to be minimized for spectral clustering algorithms. 
%We will compare the communication costs of these algorithms in different settings.

%We also compare the total communication costs of different algorithms/models. As the unit does not matter in our case, we normalize all communication costs by the cost of \baseline.  Whenever possible, we will visualize the clustered results.

We implemented the algorithms using multiple languages, including Matlab, Python and C++. Our experiments were conducted on an IBM NeXtScale nx360 M4 server, which is equipped with 2 Intel Xeon E5-2652 v2 8-core processors, 32GB RAM and 250GB local storage.


\subsection{Datasets.}
We test the algorithms in the following real and synthetic datasets, which is visualized in \figref{visualization}.


\begin{figure}[h]
     \centering
     \subfigure[\twomoons]{\includegraphics[width=0.23\textwidth]{twomoons-14000-original.png}\label{fig:twomoons}}
     ~~
     \subfigure[\gauss]{\includegraphics[width=0.23\textwidth]{gauss-10000-original.png}\label{fig:gauss}}
     ~~
     \subfigure[\sculpture]{\includegraphics[width=0.13\textwidth,height=0.16\textwidth]{sculpture-11680-original.jpg}\label{fig:sculpture}}
     \caption{Visualization of the datasets for our experiments.}
     \label{fig:visualization}
\end{figure}



\vspace{-1mm}
\begin{itemize}
\item \twomoons : this dataset contains $n=14,000$ coordinates in $\mathbb{R}^2$. We consider each point to be a vertex. For any two vertices $u, v$, we add an edge with weight $w(u,v) = \exp\{-\|u-v\|_2^2/\sigma^2\}$ with $\sigma = 0.1$ when one vertex is among the $7000$-nearest points of the other.  This construction results in a graph with about $110,000,000$ edges.

\item  \gauss : this dataset contains $n = 10,000$ points in $\mathbb{R}^2$. There are $4$ clusters in this dataset, each generated using a Gaussian distribution. We construct a complete graph as the similarity graph.  For any two vertices $u, v$, we define the weight $w(u,v) = \exp\{-\|u-v\|_2^2/\sigma^2\}$ with $\sigma = 1$. The resulting graph has about $100,000,000$ edges.

\item \sculpture : a photo of \textit{The Greek Slave}~\footnote{Available in e.g., \url{http://artgallery.yale.edu/collections/objects/14794}}. We use an $80\times 150$ version of this photo where each pixel is viewed as a vertex. To construct a similarity graph, we map each pixel to a point in $\mathbb{R}^5$, i.e., $(x, y, r, g, b)$, where the latter three coordinates are the RGB values. For any two vertices $u, v$, we  put an edge between $u, v$ with weight $w(u,v) = \exp\{-\|u-v\|_2^2/\sigma^2\}$ with $\sigma = 0.5$ if one of $u, v$ is among the $5000$-nearest points of the other. This results in a graph with about $70,000,000$ edges.
\end{itemize}
\vspace{-1mm}
In the distributed model edges are randomly partitioned across $s$ sites. 

%\vspace{-1.5mm}



\subsection{Results on clustering quality}
%{\em Quality.} \
\begin{figure*}[ht]
     \centering
     \subfigure[\baseline]{\includegraphics[width=0.2\textwidth]{twomoons-14000-original-clustered.png}\label{fig:twomoons-clustered-original}}
     \subfigure[\MM]{\includegraphics[width=0.2\textwidth]{twomoons-14000-sparsify-clustered-15.png}\label{fig:twomoons-clustered-sparsify}}
     \subfigure[\blackboard]{\includegraphics[width=0.2\textwidth]{twomoons-14000-chain-clustered.png}\label{fig:twomoons-clustered-chain}}
     \caption*{\twomoons, $k = 2$;}

\subfigure[\baseline]{\includegraphics[width=0.2\textwidth]{gauss-10000-original-clustered.png}\label{fig:gauss-clustered-original}}
     \subfigure[\MM]{\includegraphics[width=0.2\textwidth]{gauss-10000-sparsify-clustered-15.png}\label{fig:gauss-clustered-sparsify}}
     \subfigure[\blackboard]{\includegraphics[width=0.2\textwidth]{gauss-10000-chain-clustered.png}\label{fig:gauss-clustered-chain}}
     \caption*{\gauss, $k = 4$}


     \subfigure[\baseline]{\includegraphics[width=0.2\textwidth,height=0.2\textwidth]{sculpture-11680-original-clustered.png}\label{fig:sculpture-clustered-original}}  
     \subfigure[\MM]{\includegraphics[width=0.2\textwidth,height=0.2\textwidth]{sculpture-11680-sparsify-clustered-15.png}\label{fig:sculpture-clustered-sparsify}}
     \subfigure[\blackboard]{\includegraphics[width=0.2\textwidth,height=0.2\textwidth]{sculpture-11680-chain-clustered.png}\label{fig:sculpture-clustered-chain}}
     \caption*{\sculpture, $k = 3$. }


     
     \caption{Visualization of the results on \twomoons, \gauss\ and \sculpture. In the message passing model each site samples $5 n$ edges; in the blackboard model all sites jointly sample $10n$ edges (in \twomoons~ and \gauss) or $20n$ edges (in \sculpture) and the chain has length $18$. $s = 15$.}
     \label{fig:quality-1}
\end{figure*}

We visualize the clustered results for 
the \twomoons, \gauss\ and \sculpture\ in Figure~\ref{fig:quality-1}.
% and visualize the clustered results for \gauss\ and \sculpture in Figure~\ref{fig:quality-2}.
It can be seen that \baseline, \MM\ and \blackboard\ give results of very similar qualities.  For simplicity, here we only present the visualization for $s=15$. Similar results were observed when we varied the values of $s$.  
%\he{To Qin: Do you plan to have two titles (Results \& Quality)?}


% \begin{figure*}[h]
%      \centering
% \subfigure[\baseline]{\includegraphics[width=0.3\textwidth]{gauss-10000-original-clustered.png}\label{fig:gauss-clustered-original}}
%      \subfigure[\MM]{\includegraphics[width=0.3\textwidth]{gauss-10000-sparsify-clustered-15.png}\label{fig:gauss-clustered-sparsify}}
%      \subfigure[\blackboard]{\includegraphics[width=0.3\textwidth]{gauss-10000-chain-clustered.png}\label{fig:gauss-clustered-chain}}
%      \caption*{\gauss, $k = 4$}


%      \subfigure[\baseline]{\includegraphics[width=0.2\textwidth]{sculpture-11680-original-clustered.png}\label{fig:sculpture-clustered-original}}  
%      \subfigure[\MM]{\includegraphics[width=0.2\textwidth]{sculpture-11680-sparsify-clustered-15.png}\label{fig:sculpture-clustered-sparsify}}
%      \subfigure[\blackboard]{\includegraphics[width=0.2\textwidth]{sculpture-11680-chain-clustered.png}\label{fig:sculpture-clustered-chain}}
%      \caption*{\sculpture, $k = 3$. }

%      \caption{Visualization of results on \gauss\ and \sculpture; in the message passing model each site samples $5 n$ edges; in the blackboard model all sites jointly sample $10n$ (in \gauss) or $20n$ (in \sculpture) edges and the chain has length $18$.}
%      \label{fig:quality-2}
% \end{figure*}


We also compare the normalized cut (ncut) values of the clustering results of different algorithms.  The results are presented in Figure \ref{fig:quality}. In all datasets, the ncut values of different algorithms are very close. The ncut value of \MM\ slightly decreases when we increase the value of $s$, while the ncut value of \blackboard\ is independent of $s$.
%We comment that in general, it is difficult to compare \MM\ and \blackboard\ directly because they are affected by different parameters.


\begin{figure*}[!ht]
  \centering
  \subfigure[\twomoons]{\includegraphics[width=0.33\textwidth]{twomoons-14000-ncut.png}\label{fig:twomoons-quality}}\hspace*{-1.1em}
  \subfigure[\gauss]{\includegraphics[width=0.31\textwidth]{gauss-10000-ncut.png}\label{fig:gauss-quality}}\hspace*{-1.1em}
  \subfigure[\sculpture]{\includegraphics[width=0.31\textwidth]{sculpture-11680-ncut.png}\label{fig:sculpture-quality}}\hspace*{-1.1em}
  \subfigure{\includegraphics[width=0.14\textwidth]{legend.png}}
     \caption{Comparisons on normalized cuts. In the message passing model, each site samples $5n$ edges; in each round of the algorithm in the blackboard model, all sites jointly sample $10n$ edges (in \twomoons~and \gauss) or $20n$ edges (in \sculpture) edges and the chain has length $18$.}
     \label{fig:quality}
\end{figure*}

%\textcolor{red}{To Jiecao: Can you put the color lines indicating baseline, message passing, and blackboard within one row in Pic 2? Withthis we can save some space.}

%\vspace{-1.5mm}

\subsection{Results on communication costs} 
\begin{figure*}[!ht]
     \centering
     \subfigure[\twomoons]{\includegraphics[width=0.3\textwidth]{twomoons-14000-communication.png}\label{fig:twomoons-communication}}
     \subfigure[\gauss]{\includegraphics[width=0.3\textwidth]{gauss-10000-communication.png}\label{fig:gauss-communication}}
     \subfigure[\sculpture]{\includegraphics[width=0.3\textwidth]{sculpture-11680-communication.png}\label{fig:sculpture-communication}}


     \subfigure[\twomoons]{\includegraphics[width=0.32\textwidth]{twomoons-14000-communication-2.png}\label{fig:twomoons-communication-2}}
     \subfigure[\gauss]{\includegraphics[width=0.32\textwidth]{gauss-10000-communication-2.png}\label{fig:gauss-communication-2}}
     \subfigure[\sculpture]{\includegraphics[width=0.32\textwidth]{sculpture-11680-communication-2.png}\label{fig:sculpture-communication-2}}
     \caption{Comparisons on communication costs. In the message passing model, each site samples $5n$ edges; in each round of the algorithm in the blackboard model, all sites jointly sample $10n$ (in \twomoons~and \gauss) or $20n$ (in \sculpture) edges and the chain has length $18$. }
     \label{fig:communication}
\end{figure*}

We compare the communication costs of different algorithms in Figure \ref{fig:communication}. We observe that while achieving similar clustering qualities as \baseline, both \MM\ and \blackboard\ are significantly more communication-efficient (by one or two orders of magnitudes in our experiments). We also notice that the value of $s$ does not affect the communication cost of \blackboard, while the communication cost of \MM\ grows almost linearly with $s$; when $s$ is large, \MM\ uses significantly more communication than \blackboard. These confirm our theory.  %In Figure~\ref{fig:mm-const} and Figure~\ref{fig:blackboard-const}   in Appendix~\ref{sec:parameters} we present how the performance of \MM\ and \blackboard\ are affected by their parameters.

%
%
%\vspace{-1.5mm}
%\paragraph{Summary.}  From our experimental results we conclude that \MM\ and \blackboard\ achieve similar clustering quality as the native algorithm \baseline, while significantly reduce the communication cost.  When the number of sites is large, \blackboard\ is more communication efficient than \MM, as predicted by our theory.



\subsection{Parameters in \MM\ and \blackboard}
\label{sec:parameters}

Figure \ref{fig:mm-const} shows in \MM how the value of ncut is affected by the number of sites and the number of edges sampled in each site. 
Here, each site samples $cn$ edges. 
When $c=3$ and $s=1$, the ncut value diverges in all datasets. This is because with such a small $c$, the algorithm does not generate a valid sparsifier. In general, increasing $c$ or $s$ will slightly decrease the ncut value. But once they are above some thresholds, the ncut values of \MM\ and \baseline\ become very close.

Figure \ref{fig:blackboard-const} shows in \blackboard  how the ncut value is affected by the number of iterations and the number of edges sampled. When the number of iterations is set to be $5$, ncut values diverge in all datasets. This is because we cannot expect to generate a valid sparsifier by using such few iterations. It can be seen from \ref{fig:bb-gauss-constant} that for a fixed $c$, performing more iterations will help to reduce ncut values. From the same figure, one can also conclude that for fixed iterations, increasing $c$ also helps to reduce the ncut values.



\begin{figure*}[h!t]
     \centering
     \subfigure[\twomoons]{\includegraphics[width=0.3\textwidth]{twomoons-c.png}\label{fig:mm-twomoons-constant}}
     \subfigure[\gauss~dataset]{\includegraphics[width=0.3\textwidth]{gauss-c.png}\label{fig:mm-gauss-constant}}
     \subfigure[\sculpture]{\includegraphics[width=0.3\textwidth]{sculpture-c.png}\label{fig:mm-sculpture-constant}}
     \caption{The pictures above show the $\ncut$ values with respect to the values of $c$ and $s$ for the \MM\ algorithm. Here  
 each site samples $c n$ edges.}
     \label{fig:mm-const}
\end{figure*}


\begin{figure*}[h!t]
     \centering
     \subfigure[\twomoons]{\includegraphics[width=0.3\textwidth]{twomoons-iter.png}\label{fig:bb-twomoons-constant}}
     \subfigure[\gauss]{\includegraphics[width=0.3\textwidth]{gauss-iter.png}\label{fig:bb-gauss-constant}}
     \subfigure[\sculpture]{\includegraphics[width=0.3\textwidth]{sculpture-iter.png}\label{fig:bb-sculpture-constant}}
     \caption{The pictures above show how the $\ncut$ values are affected by the number of iterations and the value of $c$ for the \blackboard\ algorithm. Here 
all sites jointly sample $c n$ edges. }
     \label{fig:blackboard-const}
\end{figure*}






\section{Conclusions}
This paper introduced a novel and practical task residing at the intersection 
of computer vision and language understanding: %relative captioning, and 
dialog-based
interactive image retrieval.
Ultimately, techniques that are successful on such tasks will form the basis 
for the high fidelity,  multi-modal, intelligent 
conversational systems of the future, and thus represent important milestones in this quest.  
We demonstrated the value of %these tasks and 
the proposed learning architecture on the application
of interactive fashion footwear retrieval.
Our approach, 
enabling users to provide natural language feedback,
significantly outperforms
traditional methods relying on a pre-defined vocabulary of relative attributes, while offering more natural communication.
As future work, we plan to leverage side information, such as textual descriptions associated with images of product items,
and to develop user models that are
conditioned on dialog histories, enabling more realistic interactions.
We are also optimistic that our approach for image retrieval can be extended to other media types such as audio, video, and e-books, given the 
performance of deep learning on tasks such as speech recognition, machine translation, and activity recognition. 


\textbf{Acknowledgement.}
We would like to give special thanks to Professor Kristen Grauman for helpful discussions. 


\begingroup
    %\setlength{\bibsep}{10pt}
   	\small
    \bibliographystyle{unsrt}
	\bibliography{nips2018}
\endgroup

% \bibliographystyle{unsrt}
% \bibliography{nips2018}

\clearpage 
\newpage
\begin{center}
{\bf {\Large Supplemental Material: Dialog-based Interactive Image Retrieval \\} }
\end{center}
\appendix
\section{Data Collection}
\label{sec:app_data}

In the following, we explain the details on how we collected the relative 
captioning dataset for training the user simulator and provide insights 
on the dataset properties. Unlike existing datasets which aim to capture 
the visual differences purely using ``more" or ``less" relations on visual attributes~\cite{kovashka2012}, we want to collect data which 
captures comparative visual differences that are 
hard to describe merely using a pre-defined set of attributes. 
As shown in Figure~\ref{fig:amtInterface}, we designed the data collection 
interface in the context of fashion footwear retrieval, where 
a conversational shopping assistant interacts with a customer and whose goal
is to efficiently retrieve and present the product that matches 
the user's mental image of the desired item. 

\begin{figure*}[h]
\centering
	\begin{subfigure}[t]{0.4\textwidth}
        \centering
        \includegraphics[width=\linewidth]{figs/wordCount.pdf}
        \caption{}
    \end{subfigure}
    \begin{subfigure}[t]{0.55\textwidth}
        \centering
        \includegraphics[width=\linewidth]{figs/examples.pdf}
        \caption{}
    \end{subfigure}%
    \caption{Length distribution of the relative captioning dataset (a),
    and examples of relative captions collected in the dataset (b). 
    The leading phrase ``\emph{Unlike the provided image, the ones I want}" 
    is omitted for brevity.}
    \label{fig:relativeExample}
\end{figure*}
  
\textbf{Collecting Relative Expressions.} 
The desired annotation for relative captioning should be free-form and
introduce minimum constraints on how a user might construct the feedback
sentence. On the other hand, we want the collected feedback to be concise and 
relevant for retrieval and avoid casual and non-informative 
phrases (such as ``\emph{thank you}", ``\emph{oh, well}"). 
Bearing the two goals in mind, we designed a data collection interface as
shown in Figure~\ref{fig:amtInterface}, which provided the beginning phrase 
of the user's response (``\emph{Unlike the provided ...}'') and the annotators only 
needed to complete the sentence by giving an informative relative expression. 
This way, we can achieve a balance between sufficient lexical flexibility and 
avoiding irrelevant and casual phrases. After manual data cleaning, we are left
with $10,751$ relative expressions with one annotation per image pair. 

\begin{figure}
\begin{center}
\includegraphics[width=10cm]{figs/interface.pdf}
\caption{AMT annotation interface. Annotators need to assume the role of the customer and
 complete the rest of the response message. The collected captions are concise,
and only contain phrases that are useful for image retrieval.}
\label{fig:amtInterface}
\end{center}
\end{figure} 

\textbf{Augmenting Dataset with Single-Image Captions.} 
During our data collection procedure for relative expressions, 
we observed that when the target image and the reference image are 
visually distinct (fourth example in Figure~\ref{fig:relativeExample}(b)), 
users often only implicitly use the reference image by directly describing
the visual appearance of the target image. 
Inspired by this, we asked annotators to give 
direct descriptions on $3600$ images without the use of reference images. 
We then paired each image in this set with 
multiple visually distinct reference images (selected 
using deep feature similarity). This data augmentation procedure further 
boosted the size of our dataset at a relatively low annotation cost. 
%In total, our dataset contains $28,751$ pairs of images with one relative
%expression per image pair. 

\begin{figure*}
\centering
\includegraphics[width=\textwidth]{figs/word_chart.pdf}
\caption{Visualization of the rich vocabulary discovered from the 
relative captioning dataset. The size of each rectangle is proportional to the word count of the corresponding
word. }
\label{fig:wordDistr}

\end{figure*}

\section{Dataset analysis}
\label{sec:dataset_analysis}

Figure~\ref{fig:relativeExample}(a) shows the length distribution of the collected
captions. Most captions are very concise (between 4 to 8 words), yet composing
a large body of highly rich vocabularies as shown in Figure~\ref{fig:wordDistr}
\footnote{A few high-frequency words are removed from this chart, 
including "has/have", "is/are", "a", "with".} .
Interestingly, although annotators have 
the freedom to give feedback in terms of comparison on 
a single visual attribute (such as ``\emph{is darker}", 
``\emph{is more formal}"), most feedback expressions consist of 
compositions of multiple phrases that often include spatial or structural details (Table~\ref{tab:phrases}). 

Examples of the collected relative expressions are shown in 
Figure~\ref{fig:relativeExample}(b). We observed that, in some cases, 
users apply a concise phrase to describe the key visual difference (first example); 
but most often, users adopt more complicated phrases (second and third examples). 
The benefit of using free-form feedback can be seen
in the second example: when the two shoes are exactly the same on most attributes 
(white color, flat heeled, clog shoes), the user resorts to using composition of 
a fine-grained visual attribute (``\emph{holes}") with spatial reference (``\emph{on the top}"). Without free-form dialog based feedback, this intricate visual 
difference would be hard to convey. 



\begin{table*}
\begin{center}
\small
  \begin{tabular}{p{4cm} | p{4cm} | p{4.4cm}}
    \hline
    {\bf Single Phrase } & \bf{Composition of Phrases } & \bf{Propositional Phrases} \\ 
    {\bf (36\%)} & \bf{(63\%)} & \bf{(40\%)} \\\hline\hline
    
 are brownish & is \textcolor{NavyBlue}{more athletic} and is \textcolor{NavyBlue}{white} & is lower \textcolor{NavyBlue}{on the ankle} and blue \\ \hline
    have a zebra print & has \textcolor{NavyBlue}{a larger sole} and is \textcolor{NavyBlue}{not a high top} & have rhinestones \textcolor{NavyBlue}{across the toe} and a strap \\\hline
    have a thick foot sheath & has \textcolor{NavyBlue}{lower heel} and \textcolor{NavyBlue}{exposes more foot and toe} & are brown \textcolor{NavyBlue}{with a side cut out} \\\hline
    are low-top canvas sneakers& is \textcolor{NavyBlue}{white}, and \textcolor{NavyBlue}{has high heels, not platforms} & is in neutrals \textcolor{NavyBlue}{with buckled strap} and flatter toe \\\hline
    have polka dot linings & is \textcolor{NavyBlue}{alligator}, \textcolor{NavyBlue}{not snake print}, and \textcolor{NavyBlue}{a pointy tip} & is more rugged \textcolor{NavyBlue}{with textured sole} \\%and leather construction
    \hline
  \end{tabular}
\end{center}
 \caption{Examples of relative expressions. Around two thirds of the collected expressions 
contain composite feedback on more than one types of visual feature. And 40\%
of the expressions contain propositional phrases that provide information 
containing spatial or structural details.}
\label{tab:phrases}
\end{table*}

\section{Human Evaluation of Relative Captioning Results}
\label{sec:app_relative}
\begin{figure}
\begin{center}
\includegraphics[width=.6\linewidth]{figs/rc_comparison.pdf}
\caption{Ratings of relative captions provided by 
humans and different relative captioner
models. The raters were asked to give a score from 1 to 4 on the
quality of the captions: no errors (4), minor errors (3), 
somewhat related (2) and unrelated (1). }
\label{fig:human}
\end{center}
\end{figure}

We tested a variety of relative captioning models based on different 
choices of feature fusion and the use of attention mechanism. 
Specifically, we tested one
{\em Show and Tell}~\cite{vinyals2015show} based model,
\textbf{RC-FC} (using concatenated deep features as input), 
and three {\em Show, Attend and Tell}~\cite{icml2015_xuc15} based models,
including \textbf{RC-FCA} (feature concatenation), \textbf{RC-LNA} (feature fusion using 
a linear layer) and \textbf{RC-CNA} (feature fusion using a convolutional
layer). For all methods, we adopted the architecture of ResNet101~\cite{He2015}
pre-trained on ImageNet to extract deep feature representation. 

\begin{figure}
\begin{center}
\includegraphics[width=\linewidth]{figs/more_captions.pdf}
\caption{Examples of generated relative captions using \textbf{RC-FCA}. Red fonts highlight
inaccurate or redundant descriptions.}
\label{fig:more_captions}
\end{center}
\end{figure}

We report several common quantitative metrics to compare the quality of generated
captions in Table~\ref{tab:scores}. Given the intrinsic flexibility in describing visual differences between two images, and the lack of comprehensive variations 
of human annotations for each pair of images, we found that common image captioning 
metrics does not provide reliable evaluation of the 
actual quality of the generated captions.
Therefore, to better evaluate the caption quality, 
we directly conducted human evaluation, following the same rating scheme used in
\cite{vinyals2015show}. We collected user ratings on relative captions generated by
each model and those provided by humans on $1000$ image pairs. 
Both quantitative results and human evaluation (Figure~\ref{fig:human}) suggest that
all relative captioning models produced similar performance with \textbf{RC-CNA} 
exhibiting marginally better performance. It is also noticeable that there is a gap 
between human provided descriptions and all automatically generated captions,
and we observed some captions with incorrect attribute descriptions or are not
entirely sensible to humans, as shown in Figure~\ref{fig:more_captions}.
This indicates the inherent complexity of task of relative image
captioning and room for improvement of the user simulator,
which will lead to more robust and generalizable dialog agents. 

\begin{table}
  \centering
   \caption{Quantitative metrics of generated relative captions on Shoes dataset.} 
  \begin{tabular}{l|c|c|c}
  \hline
  & BLEU-1 & BLEU-4 & ROUGE  \\
  \hline
  RC-CNA & 32.5 & 11.2 & 45.4 \\
  RC-LNA & 30.7 & 10.7 & 43.2 \\
  RC-FCA & 29.6 & 10.3 & 42.9 \\
  RC-FC & 26.3 & 8.8 & 40.4 \\
  \hline
  \end{tabular}
 \label{tab:scores}
\end{table}


\section{Experimental Configurations}
\label{sec:config}
Since no official training and testing data split was reported on {\em Shoes} 
dataset, we randomly selected $10,000$ images as the training set, 
and the rest $4,658$ images as the held-out testing set. The user simulator adopts
the same training and testing data split as our dialog manager: it
was trained using image pairs sampled from the training set with no overlap 
with the testing images. Since the four models for relative image captioning produced
similar qualitative results in the user study, we selected \textbf{RC-FCA} model as our user simulator since it leads to more efficient training time for the dialog manager 
than the \textbf{RC-CNA} model. 
The baseline method, \textbf{RL-SCST}, uses the same network architecture
and the same supervised pre-training step as our dialog manager 
and also utilizes the user simulator for training. The idea of \textbf{RL-SCST} is to 
use test-time inference reward as the baseline for policy gradient learning 
by encouraging policies performing above 
the baseline while suppressing policies under-performing the baseline. 
Given the trained user simulator, we can easily compute the test-time 
rewards for \textbf{RL-SCST} by greedy decoding rather than stochastically sampling 
the image to return at each dialog turn. 

For all methods, the embedding dimensionality of the feature space is set to $D= 256$;
the MLP layer of the image encoder is finetuned using the single image captions 
to better capture the domain-specific image features. For \textbf{SL} training, 
we used the ADAM optimizer with an initial learning rate of $0.001$ and the 
margin parameter $m$ is set to $0.1$. For all reinforcement 
learning based methods, we employed the RMSprop optimizer with an initial learning 
rate of $10^{-5}$, and the discount factor is set to $1$. 
For our dialog manager, we set the number of nearest neighbors
as $3$ for the Candidate Generator. 
 
\begin{figure*}
\centering
\includegraphics[width=.95\textwidth]{figs/qualitative.pdf}
\caption{Examples of users interacting with the proposed dialog manager system.
User feedbacks are shown below the corresponding images. ``\emph{Unlike the provided image, the ones I want"} is omitted from each sentence for brevity.}
\label{fig:qualitative}
\end{figure*}
\section{Discussions on the Dialog Manager}
\label{sec:qualitative}
In this section, we provide more discussions on the proposed dialog 
manager framework and point out a few directions for improvement. 

\textbf{Dialog-based User Interaction.} 

\begin{wrapfigure}{r}{7cm}
\centering
\includegraphics[width=.4\textwidth]{figs/intuition.png}
\caption{Illustration of the triple loss objective and the ranking objective. }
\label{fig:intuition}
\end{wrapfigure}

Figure~\ref{fig:qualitative} shows more examples of the dialog interactions 
on human users. In all examples, the target image reached a final
ranking within the top $100$ images (about $97\%$ in ranking percentile) within
five dialog turns. These examples indicate that, visible 
improvement of retrieval results often comes from a flexible combination of 
direct reference to distinctive visual attributes of the target image, 
and comparison to the candidate image based on relative attributes. 
Ideally, feedback based on a pre-defined attribute set can achieve 
similar performance if the attribute vocabulary is sufficiently 
comprehensive and descriptive (which often consists of hundreds of 
words as in our footwear retrieval application). 
But in practice, it is infeasible to ask the user to scroll through a list of 
hundreds of attribute words and select the optimal one to provide feedback on. 



Further, we observe that the system tends to be less responsive to 
certain low-frequency words generated by the use simulator (such as 
``slouchy'' in the third example). This is as expected, since the dialog manager 
is trained on the user simulator, which in itself has limitations
(such as the fixed size of vocabulary after being trained, and the lack of memory for
dialog history). We are interested in finetuning the dialog manager on 
real users, so that it can directly adapt to new vocabularies from the user. 
In summary, results on real users demonstrated that free-form dialog feedback 
is able to capture various types of visual differences with great 
lexical flexibility and can potentially result in valuable applications
in real-world image retrieval systems. 

\textbf{Dialog Manager Learning Framework.} One main advantage of the proposed 
RL based framework is to train the agent end-to-end
with a non-differentiable objective function (the target image rank).
While triplet loss based objective makes it efficient to pre-train the dialog
manager, it still deviates from the ranking objective. 
As illustrated in Figure~\ref{fig:intuition}: two examples exhibit similar triplet loss 
objectives, but the target image ranks differ greatly. 


We noticed that the dialog manager based on the current learning architecture 
sometimes forgets information from past 
turns. For example, in the second example of Figure~\ref{fig:qualitative}, the
second turn imposes a ``yellow accents'' requirement to the target image. 
While this feedback is reflected in the immediate next turn, it is missing
from the later turns of the dialog. We think that model architectures which 
better incorporates the dialog history is able to alleviate this issue.
We could in principle investigate more variations of the network 
design to further improve its performance. Overall, the proposed network 
architecture is effective in demonstrating the applicability of dialog-based 
interactive image retrieval.









\end{document}


