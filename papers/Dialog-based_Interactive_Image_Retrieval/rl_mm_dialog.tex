\documentclass{article}

% if you need to pass options to natbib, use, e.g.:
\PassOptionsToPackage{numbers}{natbib}
% before loading nips_2018

% ready for submission
\usepackage[final]{neurips_2018}
% to compile a preprint version, e.g., for submission to arXiv, add
% add the [preprint] option:
% \usepackage[preprint]{nips_2018}

% to compile a camera-ready version, add the [final] option, e.g.:
% \usepackage[final]{nips_2018}

% to avoid loading the natbib package, add option nonatbib:
% \usepackage[nonatbib]{nips_2018}

\usepackage[utf8]{inputenc} % allow utf-8 input
\usepackage[T1]{fontenc}    % use 8-bit T1 fonts
\usepackage{hyperref}       % hyperlinks
\usepackage{url}            % simple URL typesetting
\usepackage{booktabs}       % professional-quality tables
\usepackage{amsfonts}       % blackboard math symbols
\usepackage{nicefrac}       % compact symbols for 1/2, etc.
\usepackage{microtype}      % microtypography
% ours 
\usepackage{epsfig}
\usepackage{graphicx}
\usepackage{amsmath}
\usepackage{amssymb}
\usepackage{subcaption}
\usepackage{float}
\usepackage{multirow}
% \usepackage{color}
\usepackage{comment}
\usepackage{wrapfig}

\usepackage[usenames, dvipsnames]{color}

\title{Dialog-based Interactive Image Retrieval}

% The \author macro works with any number of authors. There are two
% commands used to separate the names and addresses of multiple
% authors: \And and \AND.
%
% Using \And between authors leaves it to LaTeX to determine where to
% break the lines. Using \AND forces a line break at that point. So,
% if LaTeX puts 3 of 4 authors names on the first line, and the last
% on the second line, try using \AND instead of \And before the third
% author name.
\begin{comment}
\author{
  %% examples of more authors
  Xiaoxiao Guo$^{*1}$ \thanks{$^{*}$Contributed equally. 
  } \phantom{a} Hui Wu$^{*1}$ \phantom{a} Yu Cheng$^{1}$ \phantom{a}  Steven Rennie$^{2}$ \phantom{a} Rogerio Feris$^{1}$  \\
  $^{1}$IBM Research AI \phantom{AND} $^{2}$Fusemachines Inc. 
  %% Coauthor \\
  %% Affiliation \\
  %% Address \\
  %% \texttt{email} \\
  %% \And
  %% Coauthor \\
  %% Affiliation \\
  %% Address \\
  %% \texttt{email} \\
}
\end{comment}
\author{
  %% examples of more authors
  Xiaoxiao Guo$^{\dagger}$ \thanks{$^{\dagger}$These two authors contributed equally to this work. } \\
  IBM Research AI\\
  \texttt{xiaoxiao.guo@ibm.com}
  \And
  Hui Wu$^{\dagger}$ \\
  IBM Research AI\\
  \texttt{wuhu@us.ibm.com}
  \And
  Yu Cheng \\
  IBM Research AI\\
  \texttt{chengyu@us.ibm.com}
  \And
  Steven Rennie \\
  Fusemachines Inc.  \\
  \texttt{srennie@gmail.com}
  \And
 Gerald Tesauro \\
  IBM Research AI\\
  \texttt{gtesauro@us.ibm.com}
  \And
  Rogerio Schmidt Feris \\
  IBM Research AI\\
  \texttt{rsferis@us.ibm.com}
}

\renewcommand\footnotemark{}
\begin{document}
%\nipsfinalcopy
%is no longer used

\maketitle

\begin{abstract}
\begin{comment}
Inspired by the enormous growth of huge online media collections of many types (e.g. images, audio, video, e-books, etc.), and the paucity of intelligent retrieval systems, 
this paper introduces a novel approach to interactive visual content 
retrieval.  The proposed retrieval framework is guided by free-form natural language feedback from users, allowing for more natural and effective communication. 
Such a framework constitutes a multi-modal dialog protocol where in each dialog turn, a user submits a natural language request to a retrieval agent, which then attempts to retrieve the optimal object.
We formulate the retrieval task
as a reinforcement learning problem, and reward the dialog system for improving the 
rank of the target object during each dialog turn. This
framework can be applied to a variety of visual media types (images, videos, graphics, etc.),
and in this paper, we study in-depth its application on the task of interactive image retrieval.
To avoid the cumbersome and costly process of collecting human-machine conversations 
as the dialog system learns, we train the dialog system with 
a user simulator, which is itself trained to describe the differences between 
target and retrieved images.  The efficacy of our approach is demonstrated in a footwear image retrieval application.  Extensive experiments on both simulated and real-world data show that: 1) our proposed learning framework achieves better accuracy than other supervised and reinforcement learning baselines; and 2) user feedback based on natural language rather than pre-specified attributes leads to more effective retrieval results, and a more natural and expressive communication interface. 
\end{comment}

Existing methods for interactive image retrieval have demonstrated the merit of integrating
user feedback, improving retrieval results. However, most current systems
rely on restricted forms of user feedback, such as binary relevance responses, or
feedback based on a fixed set of relative attributes, which limits their impact. In this
paper, we introduce a new approach to interactive image search that enables users
to provide feedback via natural language, allowing for more natural and effective
interaction. We formulate the task of dialog-based interactive image retrieval as a
reinforcement learning problem, and reward the dialog system for improving the
rank of the target image during each dialog turn. To mitigate the cumbersome and
costly process of collecting human-machine conversations as the dialog system
learns, we train our system with a user simulator, which is itself trained to describe
the differences between target and candidate images. The efficacy of our approach
is demonstrated in a footwear retrieval application. Experiments on
both simulated and real-world data show that 1) our proposed learning framework
achieves better accuracy than other supervised and reinforcement learning baselines
and 2) user feedback based on natural language rather than pre-specified
attributes leads to more effective retrieval results, and a more natural and expressive
communication interface.
 
\end{abstract}

% !TEX root = ../arxiv.tex

Unsupervised domain adaptation (UDA) is a variant of semi-supervised learning \cite{blum1998combining}, where the available unlabelled data comes from a different distribution than the annotated dataset \cite{Ben-DavidBCP06}.
A case in point is to exploit synthetic data, where annotation is more accessible compared to the costly labelling of real-world images \cite{RichterVRK16,RosSMVL16}.
Along with some success in addressing UDA for semantic segmentation \cite{TsaiHSS0C18,VuJBCP19,0001S20,ZouYKW18}, the developed methods are growing increasingly sophisticated and often combine style transfer networks, adversarial training or network ensembles \cite{KimB20a,LiYV19,TsaiSSC19,Yang_2020_ECCV}.
This increase in model complexity impedes reproducibility, potentially slowing further progress.

In this work, we propose a UDA framework reaching state-of-the-art segmentation accuracy (measured by the Intersection-over-Union, IoU) without incurring substantial training efforts.
Toward this goal, we adopt a simple semi-supervised approach, \emph{self-training} \cite{ChenWB11,lee2013pseudo,ZouYKW18}, used in recent works only in conjunction with adversarial training or network ensembles \cite{ChoiKK19,KimB20a,Mei_2020_ECCV,Wang_2020_ECCV,0001S20,Zheng_2020_IJCV,ZhengY20}.
By contrast, we use self-training \emph{standalone}.
Compared to previous self-training methods \cite{ChenLCCCZAS20,Li_2020_ECCV,subhani2020learning,ZouYKW18,ZouYLKW19}, our approach also sidesteps the inconvenience of multiple training rounds, as they often require expert intervention between consecutive rounds.
We train our model using co-evolving pseudo labels end-to-end without such need.

\begin{figure}[t]%
    \centering
    \def\svgwidth{\linewidth}
    \input{figures/preview/bars.pdf_tex}
    \caption{\textbf{Results preview.} Unlike much recent work that combines multiple training paradigms, such as adversarial training and style transfer, our approach retains the modest single-round training complexity of self-training, yet improves the state of the art for adapting semantic segmentation by a significant margin.}
    \label{fig:preview}
\end{figure}

Our method leverages the ubiquitous \emph{data augmentation} techniques from fully supervised learning \cite{deeplabv3plus2018,ZhaoSQWJ17}: photometric jitter, flipping and multi-scale cropping.
We enforce \emph{consistency} of the semantic maps produced by the model across these image perturbations.
The following assumption formalises the key premise:

\myparagraph{Assumption 1.}
Let $f: \mathcal{I} \rightarrow \mathcal{M}$ represent a pixelwise mapping from images $\mathcal{I}$ to semantic output $\mathcal{M}$.
Denote $\rho_{\bm{\epsilon}}: \mathcal{I} \rightarrow \mathcal{I}$ a photometric image transform and, similarly, $\tau_{\bm{\epsilon}'}: \mathcal{I} \rightarrow \mathcal{I}$ a spatial similarity transformation, where $\bm{\epsilon},\bm{\epsilon}'\sim p(\cdot)$ are control variables following some pre-defined density (\eg, $p \equiv \mathcal{N}(0, 1)$).
Then, for any image $I \in \mathcal{I}$, $f$ is \emph{invariant} under $\rho_{\bm{\epsilon}}$ and \emph{equivariant} under $\tau_{\bm{\epsilon}'}$, \ie~$f(\rho_{\bm{\epsilon}}(I)) = f(I)$ and $f(\tau_{\bm{\epsilon}'}(I)) = \tau_{\bm{\epsilon}'}(f(I))$.

\smallskip
\noindent Next, we introduce a training framework using a \emph{momentum network} -- a slowly advancing copy of the original model.
The momentum network provides stable, yet recent targets for model updates, as opposed to the fixed supervision in model distillation \cite{Chen0G18,Zheng_2020_IJCV,ZhengY20}.
We also re-visit the problem of long-tail recognition in the context of generating pseudo labels for self-supervision.
In particular, we maintain an \emph{exponentially moving class prior} used to discount the confidence thresholds for those classes with few samples and increase their relative contribution to the training loss.
Our framework is simple to train, adds moderate computational overhead compared to a fully supervised setup, yet sets a new state of the art on established benchmarks (\cf \cref{fig:preview}).


\section{Related Work}
%\mz{We lack a comparison to this paper: https://arxiv.org/abs/2305.14877}
%\anirudh{refine to be more on-topic?}
\iffalse
\paragraph{In-Context Learning} As language models have scaled, the ability to learn in-context, without any weight updates, has emerged. \cite{brown2020language}. While other families of large language models have emerged, in-context learning remains ubiquitous \cite{llama, bloom, gptneo, opt}. Although such as HELM \cite{helm} have arisen for systematic evaluation of \emph{models}, there is no systematic framework to our knowledge for evaluating \emph{prompting methods}, and validating prompt engineering heuristics. The test-suite we propose will ensure that progress in the field of prompt-engineering is structured and objectively evaluated. 

\paragraph{Prompt Engineering Methods} Researchers are interested in the automatic design of high performing instructions for downstream tasks. Some focus on simple heuristics, such as selecting instructions that have the lowest perplexity \cite{lowperplexityprompts}. Other methods try to use large language models to induce an instruction when provided with a few input-output pairs \cite{ape}. Researchers have also used RL objectives to create discrete token sequences that can serve as instructions \cite{rlprompt}. Since the datasets and models used in these works have very little intersection, it is impossible to compare these methods objectively and glean insights. In our work, we evaluate these three methods on a diverse set of tasks and models, and analyze their relative performance. Additionally, we recognize that there are many other interesting angles of prompting that are not covered by instruction engineering \cite{weichain, react, selfconsistency}, but we leave these to future work.

\paragraph{Analysis of Prompting Methods} While most prompt engineering methods focus on accuracy, there are many other interesting dimensions of performance as well. For instance, researchers have found that for most tasks, the selection of demonstrations plays a large role in few-shot accuracy \cite{whatmakesgoodicexamples, selectionmachinetranslation, knnprompting}. Additionally, many researchers have found that even permuting the ordering of a fixed set of demonstrations has a significant effect on downstream accuracy \cite{fantasticallyorderedprompts}. Prompts that are sensitive to the permutation of demonstrations have been shown to also have lower accuracies \cite{relationsensitivityaccuracy}. Especially in low-resource domains, which includes the large public usage of in-context learning, these large swings in accuracy make prompting less dependable. In our test-suite we include sensitivity metrics that go beyond accuracy and allow us to find methods that are not only performant but reliable.

\paragraph{Existing Benchmarks} We recognize that other holistic in-context learning benchmarks exist. BigBench is a large benchmark of 204 tasks that are beyond the capabilities of current LLMs. BigBench seeks to evaluate the few-shot abilities of state of the art large language models, focusing on performance metrics such as accuracy \cite{bigbench}. Similarly, HELM is another benchmark for language model in-context learning ability. Rather than only focusing on performance, HELM branches out and considers many other metrics such as robustness and bias \cite{helm}. Both BigBench and HELM focus on ranking different language model, while fix a generic instruction and prompt format. We instead choose to evaluate instruction induction / selection methods over a fixed set of models. We are the first ever evaluation script that compares different prompt-engineering methods head to head. 
\fi

\paragraph{In-Context Learning and Existing Benchmarks} As language models have scaled, in-context learning has emerged as a popular paradigm and remains ubiquitous among several autoregressive LLM families \cite{brown2020language, llama, bloom, gptneo, opt}. Benchmarks like BigBench \cite{bigbench} and HELM \cite{helm} have been created for the holistic evaluation of these models. BigBench focuses on few-shot abilities of state-of-the-art large language models, while HELM extends to consider metrics like robustness and bias. However, these benchmarks focus on evaluating and ranking \emph{language models}, and do not address the systematic evaluation of \emph{prompting methods}. Although contemporary work by \citet{yang2023improving} also aims to perform a similar systematic analysis of prompting methods, they focus on simple probability-based prompt selection while we evaluate a broader range of methods including trivial instruction baselines, curated manually selected instructions, and sophisticated automated instruction selection.

\paragraph{Automated Prompt Engineering Methods} There has been interest in performing automated prompt-engineering for target downstream tasks within ICL. This has led to the exploration of various prompting methods, ranging from simple heuristics such as selecting instructions with the lowest perplexity \cite{lowperplexityprompts}, inducing instructions from large language models using a few annotated input-output pairs \cite{ape}, to utilizing RL objectives to create discrete token sequences as prompts \cite{rlprompt}. However, these works restrict their evaluation to small sets of models and tasks with little intersection, hindering their objective comparison. %\mz{For paragraphs that only have one work in the last line, try to shorten the paragraph to squeeze in context.}

\paragraph{Understanding in-context learning} There has been much recent work attempting to understand the mechanisms that drive in-context learning. Studies have found that the selection of demonstrations included in prompts significantly impacts few-shot accuracy across most tasks \cite{whatmakesgoodicexamples, selectionmachinetranslation, knnprompting}. Works like \cite{fantasticallyorderedprompts} also show that altering the ordering of a fixed set of demonstrations can affect downstream accuracy. Prompts sensitive to demonstration permutation often exhibit lower accuracies \cite{relationsensitivityaccuracy}, making them less reliable, particularly in low-resource domains.

Our work aims to bridge these gaps by systematically evaluating the efficacy of popular instruction selection approaches over a diverse set of tasks and models, facilitating objective comparison. We evaluate these methods not only on accuracy metrics, but also on sensitivity metrics to glean additional insights. We recognize that other facets of prompting not covered by instruction engineering exist \cite{weichain, react, selfconsistency}, and defer these explorations to future work. 










\section{Proposed Approach} \label{sec:method}

Our goal is to create a unified model that maps task representations (e.g., obtained using task2vec~\cite{achille2019task2vec}) to simulation parameters, which are in turn used to render synthetic pre-training datasets for not only tasks that are seen during training, but also novel tasks.
This is a challenging problem, as the number of possible simulation parameter configurations is combinatorially large, making a brute-force approach infeasible when the number of parameters grows. 

\subsection{Overview} 

\cref{fig:controller-approach} shows an overview of our approach. During training, a batch of ``seen'' tasks is provided as input. Their task2vec vector representations are fed as input to \ours, which is a parametric model (shared across all tasks) mapping these downstream task2vecs to simulation parameters, such as lighting direction, amount of blur, background variability, etc.  These parameters are then used by a data generator (in our implementation, built using the Three-D-World platform~\cite{gan2020threedworld}) to generate a dataset of synthetic images. A classifier model then gets pre-trained on these synthetic images, and the backbone is subsequently used for evaluation on specific downstream task. The classifier's accuracy on this task is used as a reward to update \ours's parameters. 
Once trained, \ours can also be used to efficiently predict simulation parameters in {\em one-shot} for ``unseen'' tasks that it has not encountered during training. 


\subsection{\ours Model} 


Let us denote \ours's parameters with $\theta$. Given the task2vec representation of a downstream task $\bs{x} \in \mc{X}$ as input, \ours outputs simulation parameters $a \in \Omega$. The model consists of $M$ output heads, one for each simulation parameter. In the following discussion, just as in our experiments, each simulation parameter is discretized to a few levels to limit the space of possible outputs. Each head outputs a categorical distribution $\pi_i(\bs{x}, \theta) \in \Delta^{k_i}$, where $k_i$ is the number of discrete values for parameter $i \in [M]$, and $\Delta^{k_i}$, a standard $k_i$-simplex. The set of argmax outputs $\nu(\bs{x}, \theta) = \{\nu_i | \nu_i = \argmax_{j \in [k_i]} \pi_{i, j} ~\forall i \in [M]\}$ is the set of simulation parameter values used for synthetic data generation. Subsequently, we drop annotating the dependence of $\pi$ and $\nu$ on $\theta$ and $\bs{x}$ when clear.

\subsection{\ours Training} 


Since Task2Sim aims to maximize downstream accuracy after pre-training, we use this accuracy as the reward in our training optimization\footnote{Note that our rewards depend only on the task2vec input and the output action and do not involve any states, and thus our problem can be considered similar to a stateless-RL or contextual bandits problem \cite{langford2007epoch}.}.
Note that this downstream accuracy is a non-differentiable function of the output simulation parameters (assuming any simulation engine can be used as a black box) and hence direct gradient-based optimization cannot be used to train \ours. Instead, we use REINFORCE~\cite{williams1992simple}, to approximate gradients of downstream task performance with respect to model parameters $\theta$. 

\ours's outputs represent a distribution over ``actions'' corresponding to different values of the set of $M$ simulation parameters. $P(a) = \prod_{i \in [M]} \pi_i(a_i)$ is the probability of picking action $a = [a_i]_{i \in [M]}$, under policy $\pi = [\pi_i]_{i \in [M]}$. Remember that the output $\pi$ is a function of the parameters $\theta$ and the task representation $\bs{x}$. To train the model, we maximize the expected reward under its policy, defined as
\begin{align}
    R = \E_{a \in \Omega}[R(a)] = \sum_{a \in \Omega} P(a) R(a)
\end{align}
where $\Omega$ is the space of all outputs $a$ and $R(a)$ is the reward when parameter values corresponding to action $a$ are chosen. Since reward is the downstream accuracy, $R(a) \in [0, 100]$.  
Using the REINFORCE rule, we have
\begin{align}
    \nabla_{\theta} R 
    &= \E_{a \in \Omega} \left[ (\nabla_{\theta} \log P(a)) R(a) \right] \\
    &= \E_{a \in \Omega} \left[ \left(\sum_{i \in [M]} \nabla_{\theta} \log \pi_i(a_i) \right) R(a) \right]
\end{align}
where the 2nd step comes from linearity of the derivative. In practice, we use a point estimate of the above expectation at a sample $a \sim (\pi + \epsilon)$ ($\epsilon$ being some exploration noise added to the Task2Sim output distribution) with a self-critical baseline following \cite{rennie2017self}:
\begin{align} \label{eq:grad-pt-est}
    \nabla_{\theta} R \approx \left(\sum_{i \in [M]} \nabla_{\theta} \log \pi_i(a_i) \right) \left( R(a) - R(\nu) \right) 
\end{align}
where, as a reminder $\nu$ is the set of the distribution argmax parameter values from the \name{} model heads.

A pseudo-code of our approach is shown in \cref{alg:train}.  Specifically, we update the model parameters $\theta$ using minibatches of tasks sampled from a set of ``seen'' tasks. Similar to \cite{oh2018self}, we also employ self-imitation learning biased towards actions found to have better rewards. This is done by keeping track of the best action encountered in the learning process and using it for additional updates to the model, besides the ones in \cref{ln:update} of \cref{alg:train}. 
Furthermore, we use the test accuracy of a 5-nearest neighbors classifier operating on features generated by the pretrained backbone as a proxy for downstream task performance since it is computationally much faster than other common evaluation criteria used in transfer learning, e.g., linear probing or full-network finetuning. Our experiments demonstrate that this proxy evaluation measure indeed correlates with, and thus, helps in final downstream performance with linear probing or full-network finetuning. 






\begin{algorithm}
\DontPrintSemicolon
 \textbf{Input:} Set of $N$ ``seen'' downstream tasks represented by task2vecs $\mc{T} = \{\bs{x}_i | i \in [N]\}$. \\
 Given initial Task2Sim parameters $\theta_0$ and initial noise level $\epsilon_0$\\
 Initialize $a_{max}^{(i)} | i \in [N]$ the maximum reward action for each seen task \\
 \For{$t \in [T]$}{
 Set noise level $\epsilon = \frac{\epsilon_0}{t} $ \\
 Sample minibatch $\tau$ of size $n$ from $\mc{T}$  \\
 Get \ours output distributions $\pi^{(i)} | i \in [n]$ \\
 Sample outputs $a^{(i)} \sim \pi^{(i)} + \epsilon$ \\
 Get Rewards $R(a^{(i)})$ by generating a synthetic dataset with parameters $a^{(i)}$, pre-training a backbone on it, and getting the 5-NN downstream accuracy using this backbone \\
 Update $a_{max}^{(i)}$ if $R(a^{(i)}) > R(a_{max}^{(i)})$ \\
 Get point estimates of reward gradients $dr^{(i)}$ for each task in minibatch using \cref{eq:grad-pt-est} \\
 $\theta_{t,0} \leftarrow \theta_{t-1} + \frac{\sum_{i \in [n]} dr^{(i)}}{n}$ \label{ln:update} \\
 \For{$j \in [T_{si}]$}{ 
    \tcp{Self Imitation}
    Get reward gradient estimates $dr_{si}^{(i)}$ from \cref{eq:grad-pt-est} for $a \leftarrow a_{max}^{(i)}$ \\
    $\theta_{t, j}  \leftarrow \theta_{t, j-1} + \frac{\sum_{i \in [n]} dr_{si}^{(i)}}{n}$
 }
 $\theta_{t} \leftarrow \theta_{t, T_{si}}$
 }
 \textbf{Output}: Trained model with parameters $\theta_T$. 
 \caption{Training Task2Sim}
 \label{alg:train}  
\end{algorithm}


\section{Dataset: Relative Captioning}
\label{sec:dataset}

% In this section, we provide information on the dataset used for training the user simulator. 
Our user simulator aims 
to capture the rich and flexible language describing visual 
differences of any given image pair. The relative captioning dataset 
thus needs this property. We situated the data collection procedure
in a scenario of a shopping chatting session between a shopping assistant and a customer. The annotator was asked to take the role of the customer 
and provide a natural expression to inform the shopping assistant about the desired
product item. 
To promote more regular, specific, and relative user feedback, we provided a sentence prefix for the annotator to complete when composing their response to a retrieved
image. Otherwise the annotator response is completely free-form: no other constraints on the response were imposed.
We used Amazon Mechanical Turk %~\cite{buhrmester2011} 
to crowdsource the relative expressions. After manually removing erroneous annotations, 
we collected in total $10,751$ captions, with one caption per pair of images.

Interestingly, we observed that when the target image and the reference image are sufficiently different, users often directly describe the visual appearance of the target image, rather than using relative expressions (c.f. fourth example in Figure 7(b), Appendix A). This behavior mirrors the \emph{discriminative captioning} problem considered in \cite{vedantam2017}, where a method must take in two images and produce a caption that refers only to one of them. Relative and discriminative captioning are complementary, and in practice, both strategies are used, and so we augmented our dataset by pairing 3600 captions that were discriminative with additional dissimilar images. Our captioner and dialog-based interactive retriever are thus trained on both discriminative and relative captions, so as to be respectively more representative of and responsive to real users. Additional details about the dataset collection procedure and the analysis on dataset statistics are included in Appendix~\ref{sec:app_data} and Appendix~\ref{sec:dataset_analysis}.
% To facilitate future research on relative captioning and reproducible results, we will make our dataset and codes available for public use.

\section{Experimental Evaluation}
\label{sec:experiment}
To demonstrate the viability of our modeling methodology, we show experimentally how through the deliberate combination and configuration of parallel FREEs, full control over 2DOF spacial forces can be achieved by using only the minimum combination of three FREEs.
To this end, we carefully chose the fiber angle $\Gamma$ of each of these actuators to achieve a well-balanced force zonotope (Fig.~\ref{fig:rigDiagram}).
We combined a contracting and counterclockwise twisting FREE with a fiber angle of $\Gamma = 48^\circ$, a contracting and clockwise twisting FREE with $\Gamma = -48^\circ$, and an extending FREE with $\Gamma = -85^\circ$.
All three FREEs were designed with a nominal radius of $R$ = \unit[5]{mm} and a length of $L$ = \unit[100]{mm}.
%
\begin{figure}
    \centering
    \includegraphics[width=0.75\linewidth]{figures/rigDiagram_wlabels10.pdf}
    \caption{In the experimental evaluation, we employed a parallel combination of three FREEs (top) to yield forces along and moments about the $z$-axis of an end effector.
    The FREEs were carefully selected to yield a well-balanced force zonotope (bottom) to gain full control authority over $F^{\hat{z}_e}$ and $M^{\hat{z}_e}$.
    To this end, we used one extending FREE, and two contracting FREEs which generate antagonistic moments about the end effector $z$-axis.}
    \label{fig:rigDiagram}
\end{figure}


\subsection{Experimental Setup}
To measure the forces generated by this actuator combination under a varying state $\vec{x}$ and pressure input $\vec{p}$, we developed a custom built test platform (Fig.~\ref{fig:rig}). 
%
\begin{figure}
    \centering
    \includegraphics[width=0.9\linewidth]{figures/photos/rig_labeled.pdf}
    \caption{\revcomment{1.3}{This experimental platform is used to generate a targeted displacement (extension and twist) of the end effector and to measure the forces and torques created by a parallel combination of three FREEs. A linear actuator and servomotor impose an extension and a twist, respectively, while the net force and moment generated by the FREEs is measured with a force load cell and moment load cell mounted in series.}}
    \label{fig:rig}
\end{figure}
%
In the test platform, a linear actuator (ServoCity HDA 6-50) and a rotational servomotor (Hitec HS-645mg) were used to impose a 2-dimensional displacement on the end effector. 
A force load cell (LoadStar  RAS1-25lb) and a moment load cell (LoadStar RST1-6Nm) measured the end-effector forces $F^{\hat{z_e}}$ and moments $M^{\hat{z_e}}$, respectively.
During the experiments, the pressures inside the FREEs were varied using pneumatic pressure regulators (Enfield TR-010-g10-s). 

The FREE attachment points (measured from the end effector origin) were measured to be:
\begin{align}
    \vec{d}_1 &= \bmx 0.013 & 0 & 0 \emx^T  \text{m}\\
    \vec{d}_2 &= \bmx -0.006 & 0.011 & 0 \emx^T  \text{m}\\
    \vec{d}_3 &= \bmx -0.006 & -0.011 & 0 \emx^T \text{m}
%    \vec{d}_i &= \bmx 0 & 0 & 0 \emx^T , && \text{for } i = 1,2,3
\end{align}
All three FREEs were oriented parallel to the end effector $z$-axis:
\begin{align}
    \hat{a}_i &= \bmx 0 & 0 & 1 \emx^T, \hspace{20pt} \text{for } i = 1,2,3
\end{align}
Based on this geometry, the transformation matrices $\bar{\mathcal{D}}_i$ were given by:
\begin{align}
    \bar{\mathcal{D}}_1 &= \bmx 0 & 0 & 1 & 0 & -0.013 & 0 \\ 0 & 0 & 0 & 0 & 0 & 1 \emx^T  \\
    \bar{\mathcal{D}}_2 &= \bmx 0 & 0 & 1 & 0.011 & 0.006 & 0 \\ 0 & 0 & 0 & 0 & 0 & 1 \emx^T  \\
    \bar{\mathcal{D}}_3 &= \bmx 0 & 0 & 1 & -0.011 & 0.006 & 0 \\ 0 & 0 & 0 & 0 & 0 & 1 \emx^T 
%    \bar{\mathcal{D}}_i &= \bmx 0 & 0 & 1 & 0 & 0 & 0 \\ 0 & 0 & 0 & 0 & 0 & 1 \emx^T , && \text{for } i = 1,2,3
\end{align}
These matrices were used in equation \eqref{eq:zeta} to yield the state-dependent fluid Jacobian $\bar{J}_x$ and to compute the resulting force zontopes.
%while using measured values of $\vec{\zeta}^{\,\text{meas}} (\vec{q}, \vec{P})$ and $\vec{\zeta}^{\,\text{meas}} (\vec{q}, 0)$ in \eqref{eq:fiberIso} yields the empirical measurements of the active force.



\subsection{Isolating the Active Force}
To compare our model force predictions (which focus only on the active forces induced by the fibers)
to those measured empirically on a physical system, we had to remove the elastic force components attributed to the elastomer. 
Under the assumption that the elastomer force is merely a function of the displacement $\vec{x}$ and independent of pressure $\vec{p}$ \cite{bruder2017model}, this force component can be approximated by the measured force at a pressure of $\vec{p}=0$. 
That is: 
\begin{align}
    \vec{f}_{\text{elast}} (\vec{x}) = \vec{f}_{\text{\,meas}} (\vec{x}, 0)
\end{align}
With this, the active generalized forces were measured indirectly by subtracting off the force generated at zero pressure:
\begin{align}
    \vec{f} (\vec{x}, \vec{p})  &= \vec{f}_{\text{meas}} (\vec{x}, \vec{p}) - \vec{f}_{\text{meas}} (\vec{x}, 0)     \label{eq:fiberIso}
\end{align}


%To validate our parallel force model, we compare its force predictions, $\vec{\zeta}_{\text{pred}}$, to those measured empirically on a physical system, $\vec{\zeta}_\text{meas}$. 
%From \eqref{eq:Z} and \eqref{eq:zeta}, the force at the end effector is given by:
%\begin{align}
%    \vec{\zeta}(\vec{q}, \vec{P}) &= \sum_{i=1}^n \bar{\mathcal{D}}_i \left( {\bar{J}_V}_i^T(\vec{q_i}) P_i + \vec{Z}_i^{\text{elast}} (\vec{q_i}) \right) \\
%    &= \underbrace{\sum_{i=1}^n \bar{\mathcal{D}}_i {\bar{J}_V}_i^T(\vec{q_i}) P_i}_{\vec{\zeta}^{\,\text{fiber}} (\vec{q}, \vec{P})} + \underbrace{\sum_{i=1}^n \bar{\mathcal{D}}_i \vec{Z}_i^{\text{elast}} (\vec{q_i})}_{\vec{\zeta}^{\text{elast}} (\vec{q})}   \label{eq:zetaSplit}
%     &= \vec{\zeta}^{\,\text{fiber}} (\vec{q}, \vec{P}) + \vec{\zeta}^{\text{elast}} (\vec{q})
%\end{align}
%\Dan{These will need to reflect changes made to previous section.}
%The model presented in this paper does not specify the elastomer forces, $\vec{\zeta}^{\text{elast}}$, therefore we only validate its predictions %of the fiber forces, $\vec{\zeta}^{\,\text{fiber}}$. 
%We isolate the fiber forces by noting that $\vec{\zeta}^{\text{elast}} (\vec{q}) = \vec{\zeta}(\vec{q}, 0)$ and rearranging \eqref{eq:zetaSplit}
%\begin{align}
%    \vec{\zeta}^{\,\text{fiber}} (\vec{q}, \vec{P})  &= \vec{\zeta}(\vec{q}, \vec{P}) - \vec{\zeta}(\vec{q}, 0)     \label{eq:fiberIso}
%%    \vec{\zeta}^{\,\text{fiber}}_{\text{emp}} (\vec{q}, \vec{P})  &= \vec{\zeta}_{\text{emp}}(\vec{q}, \vec{P}) - %\vec{\zeta}_{\text{emp}}(\vec{q}, 0)
%\end{align}
%Thus we measure the fiber forces indirectly by subtracting off the forces generated at zero pressure.  


\subsection{Experimental Protocol}
The force and moment generated by the parallel combination of FREEs about the end effector $z$-axis  was measured in four different geometric configurations: neutral, extended, twisted, and simultaneously extended and twisted (see Table \ref{table:RMSE} for the exact deformation amounts). 
At each of these configurations, the forces were measured at all pressure combinations in the set
\begin{align}
    \mathcal{P} &= \left\{ \bmx \alpha_1 & \alpha_2 & \alpha_3 \emx^T p^{\text{max}} \, : \, \alpha_i = \left\{ 0, \frac{1}{4}, \frac{1}{2}, \frac{3}{4}, 1 \right\} \right\}
\end{align}
with $p^{\text{max}}$ = \unit[103.4]{kPa}. 
\revcomment{3.2}{The experiment was performed twice using two different sets of FREEs to observe how fabrication variability might affect performance. The results from Trial 1 are displayed in Fig.~\ref{fig:results} and the error for both trials is given in Table \ref{table:RMSE}.}



\subsection{Results}

\begin{figure*}[ht]
\centering

\def\picScale{0.08}    % define variable for scaling all pictures evenly
\def\plotScale{0.2}    % define variable for scaling all plots evenly
\def\colWidth{0.22\linewidth}

\begin{tikzpicture} %[every node/.style={draw=black}]
% \draw[help lines] (0,0) grid (4,2);
\matrix [row sep=0cm, column sep=0cm, style={align=center}] (my matrix) at (0,0) %(2,1)
{
& \node (q1) {(a) $\Delta l = 0, \Delta \phi = 0$}; & \node (q2) {(b) $\Delta l = 5\text{mm}, \Delta \phi = 0$}; & \node (q3) {(c) $\Delta l = 0, \Delta \phi = 20^\circ$}; & \node (q4) {(d) $\Delta l = 5\text{mm}, \Delta \phi = 20^\circ$};

\\

&
\node[style={anchor=center}] {\includegraphics[width=\colWidth]{figures/photos/s0w0pic_colored.pdf}}; %\fill[blue] (0,0) circle (2pt);
&
\node[style={anchor=center}] {\includegraphics[width=\colWidth]{figures/photos/s5w0pic_colored.pdf}}; %\fill[blue] (0,0) circle (2pt);
&
\node[style={anchor=center}] {\includegraphics[width=\colWidth]{figures/photos/s0w20pic_colored.pdf}}; %\fill[blue] (0,0) circle (2pt);
&
\node[style={anchor=center}] {\includegraphics[width=\colWidth]{figures/photos/s5w20pic_colored.pdf}}; %\fill[blue] (0,0) circle (2pt);

\\

\node[rotate=90] (ylabel) {Moment, $M^{\hat{z}_e}$ (N-m)};
&
\node[style={anchor=center}] {\includegraphics[width=\colWidth]{figures/plots3/s0w0.pdf}}; %\fill[blue] (0,0) circle (2pt);
&
\node[style={anchor=center}] {\includegraphics[width=\colWidth]{figures/plots3/s5w0.pdf}}; %\fill[blue] (0,0) circle (2pt);
&
\node[style={anchor=center}] {\includegraphics[width=\colWidth]{figures/plots3/s0w20.pdf}}; %\fill[blue] (0,0) circle (2pt);
&
\node[style={anchor=center}] {\includegraphics[width=\colWidth]{figures/plots3/s5w20.pdf}}; %\fill[blue] (0,0) circle (2pt);

\\

& \node (xlabel1) {Force, $F^{\hat{z}_e}$ (N)}; & \node (xlabel2) {Force, $F^{\hat{z}_e}$ (N)}; & \node (xlabel3) {Force, $F^{\hat{z}_e}$ (N)}; & \node (xlabel4) {Force, $F^{\hat{z}_e}$ (N)};

\\
};
\end{tikzpicture}

\caption{For four different deformed configurations (top row), we compare the predicted and the measured forces for the parallel combination of three FREEs (bottom row). 
\revcomment{2.6}{Data points and predictions corresponding to the same input pressures are connected by a thin line, and the convex hull of the measured data points is outlined in black.}
The Trial 1 data is overlaid on top of the theoretical force zonotopes (grey areas) for each of the four configurations.
Identical colors indicate correspondence between a FREE and its resulting force/torque direction.}
\label{fig:results}
\end{figure*}






% & \node (a) {(a)}; & \node (b) {(b)}; & \node (c) {(c)}; & \node (d) {(d)};


For comparison, the measured forces are superimposed over the force zonotope generated by our model in Fig.~\ref{fig:results}a-~\ref{fig:results}d.
To quantify the accuracy of the model, we defined the error at the $j^{th}$ evaluation point as the difference between the modeled and measured forces
\begin{align}
%    \vec{e}_j &= \left( {\vec{\zeta}_{\,\text{mod}}} - {\vec{\zeta}_{\,\text{emp}}} \right)_j
%    e_j &= \left( F/M_{\,\text{mod}} - F/M_{\,\text{emp}} \right)_j
    e^F_j &= \left( F^{\hat{z}_e}_{\text{pred}, j} - F^{\hat{z}_e}_{\text{meas}, j} \right) \\
    e^M_j &= \left( M^{\hat{z}_e}_{\text{pred}, j} - M^{\hat{z}_e}_{\text{meas}, j} \right)
\end{align}
and evaluated the error across all $N = 125$ trials of a given end effector configuration.
% using the following metrics:
% \begin{align}
%     \text{RMSE} &= \sqrt{ \frac{\sum_{j=1}^{N} e_j^2}{N} } \\
%     \text{Max Error} &= \max \{ \left| e_j \right| : j = 1, ... , N \}
% \end{align}
As shown in Table \ref{table:RMSE}, the root-mean-square error (RMSE) is less than \unit[1.5]{N} (\unit[${8 \times 10^{-3}}$]{Nm}), and the maximum error is less than \unit[3]{N}  (\unit[${19 \times 10^{-3}}$]{Nm}) across all trials and configurations.

\begin{table}[H]
\centering
\caption{Root-mean-square error and maximum error}
\begin{tabular}{| c | c || c | c | c | c|}
    \hline
     & \rule{0pt}{2ex} \textbf{Disp.} & \multicolumn{2}{c |}{\textbf{RMSE}} & \multicolumn{2}{c |}{\textbf{Max Error}} \\ 
     \cline{2-6}
     & \rule{0pt}{2ex} (mm, $^\circ$) & F (N) & M (Nm) & F (N) & M (Nm) \\
     \hline
     \multirow{4}{*}{\rotatebox[origin=c]{90}{\textbf{Trial 1}}}
     & 0, 0 & 1.13 & $3.8 \times 10^{-3}$ & 2.96 & $7.8 \times 10^{-3}$ \\
     & 5, 0 & 0.74 & $3.2 \times 10^{-3}$ & 2.31 & $7.4 \times 10^{-3}$ \\
     & 0, 20 & 1.47 & $6.3 \times 10^{-3}$ & 2.52 & $15.6 \times 10^{-3}$\\
     & 5, 20 & 1.18 & $4.6 \times 10^{-3}$ & 2.85 & $12.4 \times 10^{-3}$ \\  
     \hline
     \multirow{4}{*}{\rotatebox[origin=c]{90}{\textbf{Trial 2}}}
     & 0, 0 & 0.93 & $6.0 \times 10^{-3}$ & 1.90 & $13.3 \times 10^{-3}$ \\
     & 5, 0 & 1.00 & $7.7 \times 10^{-3}$ & 2.97 & $19.0 \times 10^{-3}$ \\
     & 0, 20 & 0.77 & $6.9 \times 10^{-3}$ & 2.89 & $15.7 \times 10^{-3}$\\
     & 5, 20 & 0.95 & $5.3 \times 10^{-3}$ & 2.22 & $13.3 \times 10^{-3}$ \\  
     \hline
\end{tabular}
\label{table:RMSE}
\end{table}

\begin{figure}
    \centering
    \includegraphics[width=\linewidth]{figures/photos/buckling.pdf}
    \caption{At high fluid pressure the FREE with fiber angle of $-85^\circ$ started to buckle.  This effect was less pronounced when the system was extended along the $z$-axis.}
    \label{fig:buckling}
\end{figure}

%Experimental precision was limited by unmodeled material defects in the FREEs, as well as sensor inaccuracy. While the commercial force and moment sensors used have a quoted accuracy of 0.02\% for the force sensor and 0.2\% for the moment sensor (LoadStar Sensors, 2015), a drifting of up to 0.5 N away from zero was noticed on the force sensor during testing.

It should be noted, that throughout the experiments, the FREE with a fiber angle of $-85^\circ$ exhibited noticeable buckling behavior at pressures above $\approx$ \unit[50]{kPa} (Fig.~\ref{fig:buckling}). 
This behavior was more pronounced during testing in the non-extended configurations (Fig.~\ref{fig:results}a~and~\ref{fig:results}c). 
The buckling might explain the noticeable leftward offset of many of the points in Fig.~\ref{fig:results}a and Fig.~\ref{fig:results}c, since it is reasonable to assume that buckling reduces the efficacy of of the FREE to exert force in the direction normal to the force sensor. 

\begin{figure}
    \centering
    \includegraphics[width=\linewidth]{figures/zntp_vs_x4.pdf}
    \caption{A visualization of how the \emph{force zonotope} of the parallel combination of three FREEs (see Fig.~\ref{fig:rig}) changes as a function of the end effector state $x$. One can observe that the change in the zonotope ultimately limits the work-space of such a system.  In particular the zonotope will collapse for compressions of more than \unit[-10]{mm}.  For \revcomment{2.5}{scale and comparison, the convex hulls of the measured points from Fig.~\ref{fig:results}} are superimposed over their corresponding zonotope at the configurations that were evaluated experimentally.}
    % \marginnote{\#2.5}
    \label{fig:zntp_vs_x}
\end{figure}
\section{Conclusions}
This paper introduced a novel and practical task residing at the intersection 
of computer vision and language understanding: %relative captioning, and 
dialog-based
interactive image retrieval.
Ultimately, techniques that are successful on such tasks will form the basis 
for the high fidelity,  multi-modal, intelligent 
conversational systems of the future, and thus represent important milestones in this quest.  
We demonstrated the value of %these tasks and 
the proposed learning architecture on the application
of interactive fashion footwear retrieval.
Our approach, 
enabling users to provide natural language feedback,
significantly outperforms
traditional methods relying on a pre-defined vocabulary of relative attributes, while offering more natural communication.
As future work, we plan to leverage side information, such as textual descriptions associated with images of product items,
and to develop user models that are
conditioned on dialog histories, enabling more realistic interactions.
We are also optimistic that our approach for image retrieval can be extended to other media types such as audio, video, and e-books, given the 
performance of deep learning on tasks such as speech recognition, machine translation, and activity recognition. 


\textbf{Acknowledgement.}
We would like to give special thanks to Professor Kristen Grauman for helpful discussions. 


\begingroup
    %\setlength{\bibsep}{10pt}
   	\small
    \bibliographystyle{unsrt}
	\bibliography{nips2018}
\endgroup

% \bibliographystyle{unsrt}
% \bibliography{nips2018}

\clearpage 
\newpage
\begin{center}
{\bf {\Large Supplemental Material: Dialog-based Interactive Image Retrieval \\} }
\end{center}
\appendix
\section{Data Collection}
\label{sec:app_data}

In the following, we explain the details on how we collected the relative 
captioning dataset for training the user simulator and provide insights 
on the dataset properties. Unlike existing datasets which aim to capture 
the visual differences purely using ``more" or ``less" relations on visual attributes~\cite{kovashka2012}, we want to collect data which 
captures comparative visual differences that are 
hard to describe merely using a pre-defined set of attributes. 
As shown in Figure~\ref{fig:amtInterface}, we designed the data collection 
interface in the context of fashion footwear retrieval, where 
a conversational shopping assistant interacts with a customer and whose goal
is to efficiently retrieve and present the product that matches 
the user's mental image of the desired item. 

\begin{figure*}[h]
\centering
	\begin{subfigure}[t]{0.4\textwidth}
        \centering
        \includegraphics[width=\linewidth]{figs/wordCount.pdf}
        \caption{}
    \end{subfigure}
    \begin{subfigure}[t]{0.55\textwidth}
        \centering
        \includegraphics[width=\linewidth]{figs/examples.pdf}
        \caption{}
    \end{subfigure}%
    \caption{Length distribution of the relative captioning dataset (a),
    and examples of relative captions collected in the dataset (b). 
    The leading phrase ``\emph{Unlike the provided image, the ones I want}" 
    is omitted for brevity.}
    \label{fig:relativeExample}
\end{figure*}
  
\textbf{Collecting Relative Expressions.} 
The desired annotation for relative captioning should be free-form and
introduce minimum constraints on how a user might construct the feedback
sentence. On the other hand, we want the collected feedback to be concise and 
relevant for retrieval and avoid casual and non-informative 
phrases (such as ``\emph{thank you}", ``\emph{oh, well}"). 
Bearing the two goals in mind, we designed a data collection interface as
shown in Figure~\ref{fig:amtInterface}, which provided the beginning phrase 
of the user's response (``\emph{Unlike the provided ...}'') and the annotators only 
needed to complete the sentence by giving an informative relative expression. 
This way, we can achieve a balance between sufficient lexical flexibility and 
avoiding irrelevant and casual phrases. After manual data cleaning, we are left
with $10,751$ relative expressions with one annotation per image pair. 

\begin{figure}
\begin{center}
\includegraphics[width=10cm]{figs/interface.pdf}
\caption{AMT annotation interface. Annotators need to assume the role of the customer and
 complete the rest of the response message. The collected captions are concise,
and only contain phrases that are useful for image retrieval.}
\label{fig:amtInterface}
\end{center}
\end{figure} 

\textbf{Augmenting Dataset with Single-Image Captions.} 
During our data collection procedure for relative expressions, 
we observed that when the target image and the reference image are 
visually distinct (fourth example in Figure~\ref{fig:relativeExample}(b)), 
users often only implicitly use the reference image by directly describing
the visual appearance of the target image. 
Inspired by this, we asked annotators to give 
direct descriptions on $3600$ images without the use of reference images. 
We then paired each image in this set with 
multiple visually distinct reference images (selected 
using deep feature similarity). This data augmentation procedure further 
boosted the size of our dataset at a relatively low annotation cost. 
%In total, our dataset contains $28,751$ pairs of images with one relative
%expression per image pair. 

\begin{figure*}
\centering
\includegraphics[width=\textwidth]{figs/word_chart.pdf}
\caption{Visualization of the rich vocabulary discovered from the 
relative captioning dataset. The size of each rectangle is proportional to the word count of the corresponding
word. }
\label{fig:wordDistr}

\end{figure*}

\section{Dataset analysis}
\label{sec:dataset_analysis}

Figure~\ref{fig:relativeExample}(a) shows the length distribution of the collected
captions. Most captions are very concise (between 4 to 8 words), yet composing
a large body of highly rich vocabularies as shown in Figure~\ref{fig:wordDistr}
\footnote{A few high-frequency words are removed from this chart, 
including "has/have", "is/are", "a", "with".} .
Interestingly, although annotators have 
the freedom to give feedback in terms of comparison on 
a single visual attribute (such as ``\emph{is darker}", 
``\emph{is more formal}"), most feedback expressions consist of 
compositions of multiple phrases that often include spatial or structural details (Table~\ref{tab:phrases}). 

Examples of the collected relative expressions are shown in 
Figure~\ref{fig:relativeExample}(b). We observed that, in some cases, 
users apply a concise phrase to describe the key visual difference (first example); 
but most often, users adopt more complicated phrases (second and third examples). 
The benefit of using free-form feedback can be seen
in the second example: when the two shoes are exactly the same on most attributes 
(white color, flat heeled, clog shoes), the user resorts to using composition of 
a fine-grained visual attribute (``\emph{holes}") with spatial reference (``\emph{on the top}"). Without free-form dialog based feedback, this intricate visual 
difference would be hard to convey. 



\begin{table*}
\begin{center}
\small
  \begin{tabular}{p{4cm} | p{4cm} | p{4.4cm}}
    \hline
    {\bf Single Phrase } & \bf{Composition of Phrases } & \bf{Propositional Phrases} \\ 
    {\bf (36\%)} & \bf{(63\%)} & \bf{(40\%)} \\\hline\hline
    
 are brownish & is \textcolor{NavyBlue}{more athletic} and is \textcolor{NavyBlue}{white} & is lower \textcolor{NavyBlue}{on the ankle} and blue \\ \hline
    have a zebra print & has \textcolor{NavyBlue}{a larger sole} and is \textcolor{NavyBlue}{not a high top} & have rhinestones \textcolor{NavyBlue}{across the toe} and a strap \\\hline
    have a thick foot sheath & has \textcolor{NavyBlue}{lower heel} and \textcolor{NavyBlue}{exposes more foot and toe} & are brown \textcolor{NavyBlue}{with a side cut out} \\\hline
    are low-top canvas sneakers& is \textcolor{NavyBlue}{white}, and \textcolor{NavyBlue}{has high heels, not platforms} & is in neutrals \textcolor{NavyBlue}{with buckled strap} and flatter toe \\\hline
    have polka dot linings & is \textcolor{NavyBlue}{alligator}, \textcolor{NavyBlue}{not snake print}, and \textcolor{NavyBlue}{a pointy tip} & is more rugged \textcolor{NavyBlue}{with textured sole} \\%and leather construction
    \hline
  \end{tabular}
\end{center}
 \caption{Examples of relative expressions. Around two thirds of the collected expressions 
contain composite feedback on more than one types of visual feature. And 40\%
of the expressions contain propositional phrases that provide information 
containing spatial or structural details.}
\label{tab:phrases}
\end{table*}

\section{Human Evaluation of Relative Captioning Results}
\label{sec:app_relative}
\begin{figure}
\begin{center}
\includegraphics[width=.6\linewidth]{figs/rc_comparison.pdf}
\caption{Ratings of relative captions provided by 
humans and different relative captioner
models. The raters were asked to give a score from 1 to 4 on the
quality of the captions: no errors (4), minor errors (3), 
somewhat related (2) and unrelated (1). }
\label{fig:human}
\end{center}
\end{figure}

We tested a variety of relative captioning models based on different 
choices of feature fusion and the use of attention mechanism. 
Specifically, we tested one
{\em Show and Tell}~\cite{vinyals2015show} based model,
\textbf{RC-FC} (using concatenated deep features as input), 
and three {\em Show, Attend and Tell}~\cite{icml2015_xuc15} based models,
including \textbf{RC-FCA} (feature concatenation), \textbf{RC-LNA} (feature fusion using 
a linear layer) and \textbf{RC-CNA} (feature fusion using a convolutional
layer). For all methods, we adopted the architecture of ResNet101~\cite{He2015}
pre-trained on ImageNet to extract deep feature representation. 

\begin{figure}
\begin{center}
\includegraphics[width=\linewidth]{figs/more_captions.pdf}
\caption{Examples of generated relative captions using \textbf{RC-FCA}. Red fonts highlight
inaccurate or redundant descriptions.}
\label{fig:more_captions}
\end{center}
\end{figure}

We report several common quantitative metrics to compare the quality of generated
captions in Table~\ref{tab:scores}. Given the intrinsic flexibility in describing visual differences between two images, and the lack of comprehensive variations 
of human annotations for each pair of images, we found that common image captioning 
metrics does not provide reliable evaluation of the 
actual quality of the generated captions.
Therefore, to better evaluate the caption quality, 
we directly conducted human evaluation, following the same rating scheme used in
\cite{vinyals2015show}. We collected user ratings on relative captions generated by
each model and those provided by humans on $1000$ image pairs. 
Both quantitative results and human evaluation (Figure~\ref{fig:human}) suggest that
all relative captioning models produced similar performance with \textbf{RC-CNA} 
exhibiting marginally better performance. It is also noticeable that there is a gap 
between human provided descriptions and all automatically generated captions,
and we observed some captions with incorrect attribute descriptions or are not
entirely sensible to humans, as shown in Figure~\ref{fig:more_captions}.
This indicates the inherent complexity of task of relative image
captioning and room for improvement of the user simulator,
which will lead to more robust and generalizable dialog agents. 

\begin{table}
  \centering
   \caption{Quantitative metrics of generated relative captions on Shoes dataset.} 
  \begin{tabular}{l|c|c|c}
  \hline
  & BLEU-1 & BLEU-4 & ROUGE  \\
  \hline
  RC-CNA & 32.5 & 11.2 & 45.4 \\
  RC-LNA & 30.7 & 10.7 & 43.2 \\
  RC-FCA & 29.6 & 10.3 & 42.9 \\
  RC-FC & 26.3 & 8.8 & 40.4 \\
  \hline
  \end{tabular}
 \label{tab:scores}
\end{table}


\section{Experimental Configurations}
\label{sec:config}
Since no official training and testing data split was reported on {\em Shoes} 
dataset, we randomly selected $10,000$ images as the training set, 
and the rest $4,658$ images as the held-out testing set. The user simulator adopts
the same training and testing data split as our dialog manager: it
was trained using image pairs sampled from the training set with no overlap 
with the testing images. Since the four models for relative image captioning produced
similar qualitative results in the user study, we selected \textbf{RC-FCA} model as our user simulator since it leads to more efficient training time for the dialog manager 
than the \textbf{RC-CNA} model. 
The baseline method, \textbf{RL-SCST}, uses the same network architecture
and the same supervised pre-training step as our dialog manager 
and also utilizes the user simulator for training. The idea of \textbf{RL-SCST} is to 
use test-time inference reward as the baseline for policy gradient learning 
by encouraging policies performing above 
the baseline while suppressing policies under-performing the baseline. 
Given the trained user simulator, we can easily compute the test-time 
rewards for \textbf{RL-SCST} by greedy decoding rather than stochastically sampling 
the image to return at each dialog turn. 

For all methods, the embedding dimensionality of the feature space is set to $D= 256$;
the MLP layer of the image encoder is finetuned using the single image captions 
to better capture the domain-specific image features. For \textbf{SL} training, 
we used the ADAM optimizer with an initial learning rate of $0.001$ and the 
margin parameter $m$ is set to $0.1$. For all reinforcement 
learning based methods, we employed the RMSprop optimizer with an initial learning 
rate of $10^{-5}$, and the discount factor is set to $1$. 
For our dialog manager, we set the number of nearest neighbors
as $3$ for the Candidate Generator. 
 
\begin{figure*}
\centering
\includegraphics[width=.95\textwidth]{figs/qualitative.pdf}
\caption{Examples of users interacting with the proposed dialog manager system.
User feedbacks are shown below the corresponding images. ``\emph{Unlike the provided image, the ones I want"} is omitted from each sentence for brevity.}
\label{fig:qualitative}
\end{figure*}
\section{Discussions on the Dialog Manager}
\label{sec:qualitative}
In this section, we provide more discussions on the proposed dialog 
manager framework and point out a few directions for improvement. 

\textbf{Dialog-based User Interaction.} 

\begin{wrapfigure}{r}{7cm}
\centering
\includegraphics[width=.4\textwidth]{figs/intuition.png}
\caption{Illustration of the triple loss objective and the ranking objective. }
\label{fig:intuition}
\end{wrapfigure}

Figure~\ref{fig:qualitative} shows more examples of the dialog interactions 
on human users. In all examples, the target image reached a final
ranking within the top $100$ images (about $97\%$ in ranking percentile) within
five dialog turns. These examples indicate that, visible 
improvement of retrieval results often comes from a flexible combination of 
direct reference to distinctive visual attributes of the target image, 
and comparison to the candidate image based on relative attributes. 
Ideally, feedback based on a pre-defined attribute set can achieve 
similar performance if the attribute vocabulary is sufficiently 
comprehensive and descriptive (which often consists of hundreds of 
words as in our footwear retrieval application). 
But in practice, it is infeasible to ask the user to scroll through a list of 
hundreds of attribute words and select the optimal one to provide feedback on. 



Further, we observe that the system tends to be less responsive to 
certain low-frequency words generated by the use simulator (such as 
``slouchy'' in the third example). This is as expected, since the dialog manager 
is trained on the user simulator, which in itself has limitations
(such as the fixed size of vocabulary after being trained, and the lack of memory for
dialog history). We are interested in finetuning the dialog manager on 
real users, so that it can directly adapt to new vocabularies from the user. 
In summary, results on real users demonstrated that free-form dialog feedback 
is able to capture various types of visual differences with great 
lexical flexibility and can potentially result in valuable applications
in real-world image retrieval systems. 

\textbf{Dialog Manager Learning Framework.} One main advantage of the proposed 
RL based framework is to train the agent end-to-end
with a non-differentiable objective function (the target image rank).
While triplet loss based objective makes it efficient to pre-train the dialog
manager, it still deviates from the ranking objective. 
As illustrated in Figure~\ref{fig:intuition}: two examples exhibit similar triplet loss 
objectives, but the target image ranks differ greatly. 


We noticed that the dialog manager based on the current learning architecture 
sometimes forgets information from past 
turns. For example, in the second example of Figure~\ref{fig:qualitative}, the
second turn imposes a ``yellow accents'' requirement to the target image. 
While this feedback is reflected in the immediate next turn, it is missing
from the later turns of the dialog. We think that model architectures which 
better incorporates the dialog history is able to alleviate this issue.
We could in principle investigate more variations of the network 
design to further improve its performance. Overall, the proposed network 
architecture is effective in demonstrating the applicability of dialog-based 
interactive image retrieval.









\end{document}


