
\begin{table}
\begin{tabular}{ |p{0.1cm}||p{3.5cm}|p{1.5cm}|p{0.3cm}|p{0.3cm}| }
        \hline
        & \textbf{Dataset} & $n$ & $d$ & $c$ \\
        \hline
        a & Phone Accelerometer & 13,062,475 & 3 & 7 \\
        \hline
        b & Phone Gyroscope & 13,932,632 & 3 & 7 \\
        \hline
        c & Watch Accelerometer & 3,540,962 & 3 & 7 \\
        \hline
        d & Watch Gyroscope & 3,205,431 & 3 & 7 \\
        \hline
        e & Still & 949,983 & 3 & 6 \\
        \hline
        f & Skin & 245,057 & 3 & 2 \\
        \hline
        g & Iris & 150 & 4 & 3 \\
        \hline
        h & Lupus & 87 & 3 & 2 \\
        \hline
        i & Confidence & 72 & 3 & 2 \\
        \hline
        j & Geyser & 22 & 2 & 2 \\
        \hline
        k & Balance Scale & 625 & 4 & 3 \\
        \hline
        l & Vinnie & 380 & 2 & 2 \\
        \hline
        m & Sleep Data & 1,024 & 2 & 2 \\
        \hline
        n & Transplant & 131 & 3 & 2 \\
        \hline
        o & Slope & 44 & 3 & 2 \\
        \hline
        p & PRNN & 250 & 2 & 2 \\
        \hline
        q & Wall Robot & 5,456 & 4 & 4 \\
        \hline
        r & User Knowledge & 403 & 5 & 5 \\
        \hline
    \end{tabular}
    \vspace{0.3cm}
    \caption{\label{fig:datasetsummary_small}\textit{Summary of datasets used.} Includes dataset size ($n$), number of features ($d$), and number of clusters ($c$).}
\end{table}

\begin{table}
\footnotesize
\begin{tabular}{ |p{2.2cm}||p{0.95cm}|p{1.0cm}||p{0.95cm}|p{1.0cm}| }
        \cline{2-5}
        \multicolumn{1}{c}{} & \multicolumn{2}{|c||}{ARI} & \multicolumn{2}{c|}{AMI}  \\
        \cline{2-5}
        \multicolumn{1}{c}{} & \multicolumn{1}{|c|}{MS++} & \multicolumn{1}{c||}{MS} & \multicolumn{1}{c|}{MS++} & \multicolumn{1}{c|}{MS} \\
        \hline
        a) Phone & \textbf{0.0897} & \textit{DNF} & \textbf{0.1959} & \textit{DNF} \\
        Accelerometer & 29m 59s & $>$24h & 49m 19s & $>$24h \\
        \hline
        b) Phone & \textbf{0.2354} & \textit{DNF} & \textbf{0.1835} & \textit{DNF} \\
        Gyroscope & 1h 35m & $>$24h & 32m 32s & $>$24h \\
        \hline
        c) Watch & \textbf{0.0913} & \textit{DNF} & \textbf{0.2309} & \textit{DNF} \\
        Accelerometer & 17m 52s & $>$24h & 43m 32s & $>$24h \\
        \hline
        d) Watch & \textbf{0.1595} & \textit{DNF} & \textbf{0.1336} & \textit{DNF} \\
        Gyroscope & 24m 45s & $>$24h & 9m 3s & $>$24h \\
        \hline
        e) Still & \textbf{0.7900} & \textit{DNF} & \textbf{0.8551} & \textit{DNF} \\
        & 13.12s & $>$24h & 8.58s & $>$24h \\
        \hline
        f) Skin & \textbf{0.3270} & 0.3255 & \textbf{0.4240} &0.3975 \\
        & 16.44s & 3h 41m & 13.07s & 3h 41m \\
        \hline
        g) Iris & 0.5681 & \textbf{0.6832} & \textbf{0.7316} & 0.6970 \\
        & $<$0.01s & 6.35s & $<$0.01s & 2.36s \\
        \hline
        h) Lupus & \textbf{0.1827} & 0.1399 & \textbf{0.2134} & 0.2042 \\
        & $<$0.01s & 3.82s & $<$0.01s & 3.82s \\
        \hline
        i) Confidence & \textbf{0.2080} & 0.2059 & \textbf{0.2455} & 0.2215 \\
        & 0.02s & 0.70s & $<$0.01s & 0.98s \\
        \hline
        j) Geyser & \textbf{0.1229} & 0.0886 & \textbf{0.2409} & 0.2198 \\
        & $<$0.01s & 2.88s & $<$0.01s & 2.88s \\
        \hline
        k) Balance Scale & \textbf{0.0883} & 0.0836 & \textbf{0.2268} & 0.2166 \\
        & 0.09s & 16.02s & 0.09s & 16.02s \\
        \hline
        l) Vinnie & \textbf{0.4594} & 0.4383 & 0.3666 & \textbf{0.3671} \\
        & 0.01s & 16.85s & 0.01s & 16.85s \\
        \hline
        m) Sleep Data & 0.1181 & \textbf{0.1242} & \textbf{0.1028} & 0.0998 \\
        & 0.02s & 45.25s & 0.02s & 45.25s \\
        \hline
        n) Transplant & \textbf{0.7687} & 0.6328 & \textbf{0.7175} & 0.7018 \\
        & $<$0.01s & 4.22s & $<$0.01s & 4.22s \\
        \hline
        o) Slope & \textbf{0.2777} & 0.2715 & \textbf{0.3877} & 0.3630 \\
        & $<$0.01s & 0.43s & $<$0.01s & 0.43s \\
        \hline
        p) PRNN & \textbf{0.2093} & 0.1872 & \textbf{0.2912} & 0.2590 \\
        & 0.02s & 10.72s & $<$0.01s & 10.72s \\
        \hline
        q) Wall Robot & \textbf{0.1788} & 0.1706 & 0.3239 & \textbf{0.3246} \\
        & 0.69s & 4m37s & 0.88s & 2m30s \\
        \hline
        r) User Knowledge & \textbf{0.3398} & 0.2140 & \textbf{0.4086} & 0.3278 \\
        & 0.06s & 7.62s & 0.06s & 7.62s \\
        \hline
    \end{tabular}
    \vspace{0.3cm}
    \caption{\label{fig:datasetsummary_large}\textit{Summary of clustering performances.} MeanShift++'s and MeanShift's best results for 19 real-world datasets after tuning bandwidth. Datasets from the UCI Machine Learning Repository \cite{Dua:2019} and OpenML \cite{OpenML2013}. In cases where the original target variable is continuous, binarized versions of the datasets were used. These experiments were run on a local machine with a 1.2 GHz Intel Core M processor and 8 GB memory. MeanShift did not finish (DNF) within 24 hours for the top five largest datasets. ARI, AMI, and runtime are reported for each run, and the highest score obtained for that metric and dataset is bolded.}
\end{table}

\begin{figure*}
\begin{center}
\includegraphics[width=0.9\linewidth]{experiments_small.png}
\end{center}
   \caption{\label{fig:experiments}\textit{Comparison of MeanShift++ and MeanShift on four real-world datasets across a wide range of hyperparameter.} Datasets are shown here to illustrate how both algorithms were tuned over an appropriate range of bandwidth. Adjusted RAND index (ARI), adjusted mutual information score (AMI), and runtime are reported for each run. MeanShift++ consistently performs as well or better than MeanShift despite being up to 1000x faster. Additional experiments are shown in the Appendix.}
\end{figure*}


\begin{figure*}
\begin{center}
\includegraphics[width=\linewidth]{image_segmentation_small.png}
\end{center}
   \caption{\label{fig:image_segmentation}\textit{Comparison of six image segmentation algorithms.} We show the results of MeanShift++, MeanShift, Quickshift++, and three other popular image segmentation algorithms from the Scikit-Image library\cite{van2014scikit}. MeanShift returns qualitatively good results on image segmentation but takes very long to run. MeanShift++ returns segmentations that are the most similar to MeanShift with an up to 10,000x speedup. We expect the speedup to be far greater for high resolution images--the images shown here are low resolution (under 200k pixels).}
\end{figure*}


\begin{figure*}
\begin{center}
\includegraphics[width=1.0\linewidth]{tracking.png}
\end{center}
   \caption{\label{fig:object_tracking}\textit{Comparison of MeanShift++ and MeanShift on object tracking.} Unlike MeanShift++, MeanShift is too slow to generate masks for real-time object tracking. In practice, the user manually provides a color range that they want to track, which is often incomplete, inaccurate, or biased. Here, we initialize both MeanShift++ and MeanShift with a mask from clustering results generated by MeanShift++ to save time. For MeanShift, we use OpenCV's \cite{bradski2008learning} implementation of color histograms to track the object in question. For MeanShift++, we naturally use the grid cells that are returned from the clustering step. We find that MeanShift is more likely to get distracted by backgrounds, foregrounds, and other objects in the scene. {\bf First scene}: MeanShift returns less accurate object centers and search windows. {\bf Second scene}: MeanShift fails to find the object altogether due to an abundance of similar colors in the frame that cannot be decoupled from the object of interest. {\bf Third scene}: MeanShift starts tracking similar objects nearby when the original objective moves out of frame. In contrast, MeanShift++ stops tracking when it finds the center of mass in the search window disappear. {\bf Fourth scene}: MeanShift loses the skater faster than MeanShift++ and fails to find him again (instead it starts to track another skater altogether).}
\end{figure*}

\begin{figure}
\begin{center}
\includegraphics[width=\linewidth]{rebuttal_ari_fm.jpg}
\begin{tabular}{ |p{2.2cm}||p{3.8cm}| }
    \hline
    & Average Runtime ($\mu s$) \\
    \hline
    MeanShift++ & 2,675,100 \\
    \hline
    MeanShift & 1,765,462,893 \\
    \hline
    SLIC & 56,266 \\
    \hline
    Quickshift & 31,717,386 \\
    \hline
\end{tabular}
\end{center}
   \caption{\label{fig:bsds500}\textit{Comparison of MeanShift++, MeanShift, SLIC, and QuickShift on BSDS500 using the ARI and FM clustering metrics.} Performance metrics are averaged over 500 images. For each baseline, we plot MS++ wins, baseline wins, and ties (where the two algorithms score within 1\% of each other). MS++ performs on par or better compared to baseline algorithms. In order for MeanShift++ to finish running, we sample the images down by an order of 2. MeanShift++ is still around 1000x faster than MeanShift.}
\label{fig:short}
\end{figure}