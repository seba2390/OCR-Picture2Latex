\section{Image Segmentation}

We compare MeanShift++ to a number of baselines for unsupervised image segmentation in Figure~\ref{fig:image_segmentation}. We include Felzenszwalb \cite{felzenszwalb2004efficient}, QuickShift \cite{vedaldi2008quick}, and $k$-means, three popular image segmentation procedures from the Python Scikit-Image library \cite{van2014scikit}, as well as Quickshift++ \cite{jiang2018quickshift++}, a recent algorithm shown to be an improvement over Quickshift on image segmentation. We also include MeanShift, which often produces qualitatively better clusters than the other baselines, but runs for so much longer that it is impractical for high-resolution image segmentation. 

For image segmentation, we run each algorithm on a preprocessed image with each pixel represented in a 3D RGB color channel space, with the exception of Quickshift++, which takes $(r, g, b, x, y)$ color and spatial coordinates. MeanShift was run with both $(r, g, b)$ (shown in Figure~\ref{fig:image_segmentation}) and $(r, g, b, x, y)$ inputs, but we did not see a difference in segmentation quality or runtime. For each algorithm, the returned clusters are taken as the segments.

Our image segmentation experiments in Figure~\ref{fig:image_segmentation} show that MeanShift++ is able to produce segmentations that are nearly identical to that of MeanShift with an up to 10,000x speedup. We capped the sizes of our images at 187,500 pixels to allow MeanShift to finish running, so this speedup would surely be greater on even higher resolution images. 

Multiple attempts have been made to speed up MeanShift for image segmentation at the cost of quality, but MeanShift++ does not seem to trade off segmentation quality despite running in a sub-fraction of the time.

For a more quantitative comparison, we ran experiments using the Berkeley Segmentation Dataset Benchmark (BSDS500) of $500$ images with $6$ human-labeled segmentations each. We ran MeanShift++, MeanShift, SLIC, and QuickShift on each image and used the adjusted RAND index \cite{hubert1985comparing} (ARI) and Fowlkes-Mallows \cite{fowlkes1983method} (FM) scores to compare the clusters. Scores were averaged over the $6$ ground truth segmentations. We found that MeanShift++ performed on par or better than baselines despite being faster than MeanShift by 1,000x on average (Figure \ref{fig:bsds500}). \\