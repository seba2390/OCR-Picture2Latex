\onecolumn

\section{Experiments}
\begin{figure}[H]
\begin{center}
\includegraphics[width=0.95\linewidth]{experiments.png}
\end{center}
   \caption{\label{fig:additional_experiments}\textit{Additional experiments.} MeanShift++ and MeanShift performances on 10 more real-world datasets, for a total of 14 small datasets. Again, we illustrate how both algorithms were tuned over an appropriate range of bandwidth, and ARI, AMI, and runtime are reported for each run. MeanShift++ consistently performs as well or better than Meanshift in a fraction of the time, besting MeanShift in 12 out of these 14 select small datasets in ARI and AMI.}
\end{figure}

\section{Proofs}

\begin{proof}[Proof of Theorem~\ref{theorem}]
We first give an upper and lower bound on $\mathcal{P}(G)$ for $G \in \mathcal{G}_h$, where $\mathcal{P}(G)$ denotes the total probability mass of $G$ w.r.t. $\mathcal{P}$.
Since $\mathcal{X}$ is bounded in $\mathbb{R}^d$, there exists a constant $C_\X$ such that for all $h > 0$, we have $|\mathcal{G}_h| \le C_\X \cdot h^{-d}$.
Next, we have by the assumptions that for any cell $G \in \mathcal{G}_h$, for some $C'$ depending on $p$:
\begin{align*}
    \mathcal{P}(G) &\ge \min_{x \in G\cap \mathcal{X}} p(x) \cdot \text{Vol}(G \cap \mathcal{X})\\
    &\ge C' (\lambda_0 - C_\alpha\cdot h^\alpha)\cdot h^d \ge \frac{1}{2} C' \lambda_0 \cdot h^d.
\end{align*}
Next, since $\mathcal{X}$ is compact, there exists $p_{\text{max}}$ such that $\sup_{x\in\X} p(x) = p_{\text{max}} < \infty$. Thus,
$\mathcal{P}(G)\le  p_{\text{max}}\cdot h^d$.


We now bound $\sup_{x \in \X} |\frac{\mathcal{P}(G(x))}{h^d} - \widehat{p}_h(x)|$. We have that the event a sample drawn according to $x$ lies in $G$ is a Bernoulli random variable of probability $\mathcal{P}(G)$. From the above, we have that this variance is upper and lower bounded by $O(h^{d})$, and let us denote the number of samples in $\X_{[n]}$ that lie in $G$ as $\mathcal{P}_n(G)$. 
Therefore, by Hoeffding-Chernoff inequality (i.e. Theorem 1.3 of \cite{phillips2012chernoff}), we have for some $C''$ depending on $p$ that
\begin{align*}
    \mathbb{P}\left(\left|\cdot \mathcal{P}(G(x)) - \cdot \mathcal{P}_n(G(x))\right| \ge \frac{t}{n}\right) \le \exp\left(-\frac{t^2}{4C''\cdot n\cdot h^d}\right),
\end{align*}
for $n \cdot h^d$ sufficiently large depending on $p$.
Now choosing $t = 2\sqrt{n\cdot h^d}\cdot \sqrt{  C''\cdot\log(C_\mathcal{X}\cdot h^{-d}/\delta)}$, we have
\begin{align*}
    \mathbb{P}\left(\left|\frac{\mathcal{P}(G(x))}{h^d} - \widehat{p}_h(x)\right| \ge \frac{2\sqrt{C''\cdot \log( C_\mathcal{X}\cdot h^{-d}/\delta)}}{\sqrt{n\cdot h^d}}\right) \le \frac{\delta}{|\mathcal{G}_h|}.
\end{align*}
Thus, by union bound, we have the following holds:
\begin{align*}
    \mathbb{P}\left(\sup_{x\in \mathcal{X}}\left|\frac{\mathcal{P}(G(x))}{h^d} - \widehat{p}_h(x)\right| \ge \frac{2\sqrt{C''\cdot\log( C_\mathcal{X}\cdot h^{-d}/\delta)}}{\sqrt{n\cdot h^d}}\right) \le \delta.
\end{align*}
Next, we have by the smoothness assumption that
\begin{align*}
    \sup_{x\in \X} \left|p(x) - \frac{\mathcal{P}(G(x))}{h^d} \right| \le C_\alpha h^\alpha.
\end{align*}
The result follows by triangle inequality.
\end{proof}
