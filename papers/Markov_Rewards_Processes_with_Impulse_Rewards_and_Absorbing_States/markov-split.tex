\pdfoutput=1
%\documentclass[journal,draftclsnofoot,onecolumn,12pt,twoside]{IEEEtranTCOM}
\documentclass[journal,onecolumn,12pt,twoside]{IEEEtranTCOM}
%\documentclass[conference]{IEEEtran}
%\documentclass[peerreview]{IEEEtran}
%\documentclass[journal]{IEEEtran}
\usepackage{etex}
%\usepackage{appendix}

\normalsize

% Some very useful LaTeX packages include:
%
\usepackage{cite}
\usepackage{setspace}
%\usepackage[nomessages]{fp}


% *** GRAPHICS RELATED PACKAGES ***
%
\usepackage{graphicx}
\usepackage{color}
%\usepackage{pstricks}
\usepackage{pst-sigsys}
\ifCLASSINFOpdf
  % \usepackage[pdftex]{graphicx}
  % declare the path(s) where your graphic files are
  % \graphicspath{{../pdf/}{../jpeg/}}
  % and their extensions so you won't have to specify these with
  % every instance of \includegraphics
  % \DeclareGraphicsExtensions{.pdf,.jpeg,.png}
\else
  % or other class option (dvipsone, dvipdf, if not using dvips). graphicx
  % will default to the driver specified in the system graphics.cfg if no
  % driver is specified.
  % \usepackage[dvips]{graphicx}
  % declare the path(s) where your graphic files are
  % \graphicspath{{../eps/}}
  % and their extensions so you won't have to specify these with
  % every instance of \includegraphics
  % \DeclareGraphicsExtensions{.eps}
\fi

% graphicx was written by David Carlisle and Sebastian Rahtz. It is
% required if you want graphics, photos, etc. graphicx.sty is already
% installed on most LaTeX systems. The latest version and documentation can
% be obtained at:
% http://www.ctan.org/tex-archive/macros/latex/required/graphics/
% Another good source of documentation is "Using Imported Graphics in
% LaTeX2e" by Keith Reckdahl which can be found as epslatex.ps or
% epslatex.pdf at: http://www.ctan.org/tex-archive/info/
%
% latex, and pdflatex in dvi mode, support graphics in encapsulated
% postscript (.eps) format. pdflatex in pdf mode supports graphics
% in .pdf, .jpeg, .png and .mps (metapost) formats. Users should ensure
% that all non-photo figures use a vector format (.eps, .pdf, .mps) and
% not a bitmapped formats (.jpeg, .png). IEEE frowns on bitmapped formats
% which can result in "jaggedy"/blurry rendering of lines and letters as
% well as large increases in file sizes.
%
% You can find documentation about the pdfTeX application at:
% http://www.tug.org/applications/pdftex
%\usepackage[textwidth=6.5cm]{geometry}




% *** MATH PACKAGES ***
%
\usepackage[cmex10]{amsmath}
\usepackage{amsmath,dsfont}
\usepackage{verbatim}
%\usepackage[fleqn]{amsmath}
\usepackage{amssymb}
\usepackage{amsthm}
\usepackage{nccmath}
\usepackage{enumerate}
%\usepackage{tikz}
  \usepackage{pgfplots}
%  \usepackage{soul}
  \usepackage[normalem]{ulem}
  \usepackage{cases}
  \usepackage{bbm}  
  \usepackage{mathtools}
\usepackage[final]{hyperref}
%  \usepackage{hyperref}
%  \hypersetup{final}
  \usepackage{cleveref}
%\usetikzlibrary{plotmarks}
%  \pgfplotsset{compat=newest} 
\pgfplotsset{plot coordinates/math parser=false}
%\usepackage{verbatim}
%\usepackage{graphicx,amssymb,amstext,amsmath,cite,amsthm}
%\usepackage{amstext}
\renewcommand{\descriptionlabel}[1]{%
  \hspace\labelsep \upshape{ \textit{\underline{#1:}}}%
}

\newtheorem{mydef}{Definition}
\newtheorem{definition}{Definition}
\newtheorem{theorem}{Theorem}
\newtheorem{lemma}[theorem]{Lemma}
\newtheorem{claim}[theorem]{Claim}
\newtheorem{corollary}[theorem]{Corollary}
\newtheorem{remark}{Remark}
\newcommand{\Rmnum}[1]{\MakeUppercase{\romannumeral #1}}
\newcommand{\mybfs}[1]{\ensuremath{\mathbf{S}_{#1}}}
\newcommand{\mybfx}[1]{\ensuremath{\mathbf{X}_{#1}}}
\newcommand{\mybfp}[1]{\ensuremath{\mathbf{P}_{#1}}}
\DeclareMathOperator*{\argmin}{arg\,min}



\newcommand{\nRepi}[1]{\ensuremath{M_{#1}}}
\newcommand{\vecnRep}{\ensuremath{\underline{M}}}
\newcommand{\indi}[1]{\ensuremath{\mathbbm{1}_{#1}}}

%\newcommand{\Nretransmit}{\ensuremath{\bar{N}_{0}}}
\newcommand{\Nretransmit}{\ensuremath{N_{0}}}
\newcommand{\ENretransmit}{\ensuremath{\mathbb{E}\Nretransmit}}

%\newcommand{\astar}{\ensuremath{\hat{a}}}
%\newcommand{\bstar}{\ensuremath{\hat{b}}}
\newcommand{\astar}{\ensuremath{c^{\dagger}}}
\newcommand{\bstar}{\ensuremath{d^{\dagger}}}
\newcommand{\cstar}{\ensuremath{a^{\dagger}}}
\newcommand{\mydstar}{\ensuremath{b^{\dagger}}}

\newcommand{\omegaparam}{\ensuremath{\theta}}
\newcommand{\gammaparam}{\ensuremath{\gamma}}

\newcommand{\omegastar}{\ensuremath{\omegaparam^{*}}}
\newcommand{\gammahat}{\ensuremath{\hat{\gammaparam}}}
\newcommand{\gammatilde}{\ensuremath{\gammaparam^{*}}}

\newcommand{\Fset}{\ensuremath{F}}
\newcommand{\Cset}{\ensuremath{C}}
\newcommand{\Aset}{\ensuremath{A}}
\newcommand{\AsetC}{\ensuremath{A^{\complement}}}
\newcommand{\Bset}{\ensuremath{B}}
\newcommand{\BsetOmega}{\ensuremath{B_{\omegaparam}}}
\newcommand{\BsetOmegaC}{\ensuremath{B_{\overline{\omegaparam}}}}
%\newcommand{\bOmegaC}{\ensuremath{b_{\overline{\omegaparam}}}}
\newcommand{\bOmegaC}{\ensuremath{b}}
\newcommand{\bOmegaPrime}{\ensuremath{b'}}
\newcommand{\bOmegaT}{\ensuremath{b_{\overline{\omegaparam}}(t)}}

\newcommand{\nFUnknown}{\ensuremath{N_{1}}}
\newcommand{\nBUnknown}{\ensuremath{N_{2}}}

\newcommand{\LHSfunc}{\ensuremath{\mathcal{L}}}
\newcommand{\LHSfuncparam}{\ensuremath{\LHSfunc(\gammaparam, \omegaparam)}}
\newcommand{\LHSfuncprime}{\ensuremath{\LHSfunc(\gammaparam', \omegaparam')}}

\newcommand{\kgamma}{\ensuremath{k_1}}
\newcommand{\komega}{\ensuremath{k_2}}
\newcommand{\kk}{\ensuremath{k_3}}

\newcommand{\wplusd}{\ensuremath{w^{+}(d_1, d_2)}}
\newcommand{\wid}{\ensuremath{w_{i}^{*}(d_i)}}
\newcommand{\wi}{\ensuremath{w_{i}^{*}}}
\newcommand{\wtwo}{\ensuremath{w_{2}^{*}(d_2)}}
\newcommand{\wone}{\ensuremath{w_{1}^{*}(d_1)}}

\newcommand{\depstwo}{\ensuremath{d_2/\epsilon_2}}
\newcommand{\depsone}{\ensuremath{d_1/\epsilon_1}}

\newcommand{\epsot}{\ensuremath{\epsilon_{12}}}

\newcommand{\donedagg}{\ensuremath{d_{1}^{\dagger}}}
\newcommand{\doneddagg}{\ensuremath{d_{1}^{\ddagger}}}

\newcommand{\donedaggall}{\ensuremath{\epsilon_1 (2\epsot - 1)/\epsot}}
\newcommand{\doneddaggall}{\ensuremath{\epsilon_1\epsot}}
\newcommand{\chainsumOne}{\ensuremath{\sum_{k_1=1}^{\sizeS} \sum_{k_2 = 1}^{\sizeS} \ldots \sum_{k_{n-2} = 1}^{\sizeS}}}
\newcommand{\chainsum}{\ensuremath{\sum_{k_1=1}^{\sizeS} \sum_{k_2 = 1}^{\sizeS} \ldots \sum_{k_{n-1} = 1}^{\sizeS}}}

% States
%\newcommand{\sizeS}{\ensuremath{|\mathcal{S}|}}
\newcommand{\sizeS}{\ensuremath{|\myState|}}
\newcommand{\Si}[1]{\ensuremath{S_{#1}}}


% Transition Probabilities
\newcommand{\trans}[2]{\ensuremath{p_{#1#2}}}
\newcommand{\transM}{\ensuremath{P}}
\newcommand{\transMsup}[1]{\ensuremath{\transM^{#1}}}
\newcommand{\transMsupij}[3]{\ensuremath{\transMsup{#1}(#2, #3)}}
\newcommand{\transR}{\ensuremath{R}}
\newcommand{\transQ}{\ensuremath{Q}}

% Absorbing Markov Chains
\newcommand{\fundMat}{\ensuremath{N}}

% Reward Definitions
\newcommand{\rew}[2]{\ensuremath{\rho_{#1#2}}}
%\newcommand{\Rew}{\ensuremath{\Thorn}}
\newcommand{\Rew}{\ensuremath{\Theta}}
\newcommand{\RewSubAbsAbs}{\ensuremath{\Theta_{4}}}
\newcommand{\RewSubAbsTrans}{\ensuremath{\Theta_{3}}}
\newcommand{\RewSubTransAbs}{\ensuremath{\Theta_{2}}}
\newcommand{\RewSubTransTrans}{\ensuremath{\Theta_{1}}}

% Markov Chain submatricies
\newcommand{\RewSubCi}[1]{\ensuremath{B_{#1}}}
\newcommand{\RewSubDi}[1]{\ensuremath{A_{#1}}}
\newcommand{\RewSubC}{\ensuremath{\RewSubCi{n}}}
\newcommand{\RewSubD}{\ensuremath{\RewSubDi{n}}}

\newcommand{\HtransRew}{\ensuremath{H}}
\newcommand{\HSubAbsAbs}{\ensuremath{H_{4}}}
\newcommand{\HSubAbsTrans}{\ensuremath{H_{3}}}
\newcommand{\HSubTransAbs}{\ensuremath{H_{2}}}
\newcommand{\HSubTransTrans}{\ensuremath{H_{1}}}
\newcommand{\HtransRewij}[2]{\ensuremath{H(#1, #2)}}

% Expected Rewards
\newcommand{\Rni}[1]{\ensuremath{R_{#1}}}
\newcommand{\Rn}{\ensuremath{\Rni{n}}}
\newcommand{\Rnij}{\ensuremath{\Rn(i, j)}}

\newcommand{\barR}{\ensuremath{\bar{R}_{n}}}
\newcommand{\barRij}[2]{\ensuremath{\bar{R}_{n}(#1, #2)}}
\newcommand{\barRsubij}[3]{\ensuremath{\bar{R}_{#1}(#2, #3)}}
\newcommand{\barRinfij}[2]{\ensuremath{\barRsubij{\infty}{#1}{#2}}}
\newcommand{\barRinf}{\ensuremath{\bar{R}_{\infty}}}
\newcommand{\barRi}[1]{\ensuremath{\bar{R}_{n}(#1)}}
\newcommand{\barRinfi}[1]{\ensuremath{\barRi{#1}_{\infty}}}

\newcommand{\hatR}{\ensuremath{\hat{R}_{n}}}
\newcommand{\hatRij}[2]{\ensuremath{\hat{R}_{n}(#1, #2)}}


\newcommand{\hatRsub}[1]{\ensuremath{\hat{R}_{#1}}}
\newcommand{\hatRinf}{\ensuremath{\hatRsub{\infty}}}
\newcommand{\hatRsubij}[3]{\ensuremath{\hat{R}_{#1}(#2, #3)}}


% Markov Chain definitions
\newcommand{\myState}{\ensuremath{\Omega}}
\newcommand{\mystate}{\ensuremath{\Omega}}
\newcommand{\StateAbs}{\ensuremath{\myState_{A}}}
\newcommand{\StateTrans}{\ensuremath{\myState_{T}}}



%\newcommand{\bof}[1]{\textbf{#1}}
%\newcommand{\YN}[1]{\ensuremath{\tilde{Y}_{#1}^{n}}}
%\newcommand{\yN}[1]{\ensuremath{\tilde{y}_{#1}^{n}}}
%\newcommand{\tYN}[1]{\ensuremath{\tilde{Y}_{#1}^{n}}}
%\newcommand{\tyN}[1]{\ensuremath{\tilde{y}_{#1}^{n}}}
%\newcommand{\nN}[1]{\ensuremath{\tilde{n}_{#1}^{n}}}
%\newcommand{\NN}[1]{\ensuremath{\tilde{N}_{#1}^{n}}}
%\newcommand{\SHatK}[1]{\ensuremath{\hat{S}_{#1}^{m}}}
%%\newcommand{\setI}[1]{\ensuremath{\mathcal{I}(n_{#1}^{n})}}
%\newcommand{\setI}[1]{\ensuremath{I_{#1}}}
%\newcommand{\setInN}{\ensuremath{\mathcal{I}(n^n)}}
%\newcommand{\tSM}[1]{\ensuremath{\tilde{S}_{#1}^m}}
%\newcommand{\setURi}{\ensuremath{\mathcal{A}_i}}
%\newcommand{\setUR}{\ensuremath{\mathcal{A}}}
%\newcommand{\UN}{\ensuremath{U^n}}
%\newcommand{\uN}{\ensuremath{u^n}}
%\newcommand{\setIB}[1]{\ensuremath{\mathcal{I}_{#1}(u^n)}}
%\newcommand{\setIBH}[1]{\ensuremath{\mathcal{I}_{#1}(\hat{u}^n)}}
%\newcommand{\Prob}{\ensuremath{\operatorname{Pr}}}
%\newcommand{\Dstar}[1]{\ensuremath{D^{*}(\epsilon_{#1})}}
%%\newcommand{\XC}{\ensuremath{X_{\mathcal{\overline{C}(\nN{i})}}}}
%\newcommand{\oCn}{\ensuremath{\overline{\mathcal{C}}(\nN{i}, \nN{j})}}
%\newcommand{\Cn}{\ensuremath{\mathcal{C}(\nN{i}, \nN{j})}}
%\newcommand{\XC}{\ensuremath{X_{\oCn}}}
%\newcommand{\tUi}[1]{\ensuremath{\tilde{U}_{#1}}}
%\newcommand{\setV}{\ensuremath{\mathcal{V}}}
%\newcommand{\setU}{\ensuremath{\mathcal{U}}}
%\newcommand{\setA}{\ensuremath{\mathcal{A}}}
%\newcommand{\setB}{\ensuremath{\mathcal{B}}}
%\newcommand{\tUNn}{\ensuremath{\tilde{U}^n(\nS)}}
%\newcommand{\tUN}{\ensuremath{\tilde{U}^n}}
%
%\newcommand{\hpB}{\ensuremath{\hat{p}_{\setB}}}
%\newcommand{\pA}{\ensuremath{p_{\setA}}}
%\newcommand{\oCn}{\ensuremath{\overline{\mathcal{C}}(\nN{i}, \nN{j})}}
%\newcommand{\Cn}{\ensuremath{\mathcal{C}(\nN{i}, \nN{j})}}
%\newcommand{\XC}{\ensuremath{X_{\oCn}}}


\newcommand{\norm}[1]{\left\lVert#1\right\rVert}

%\newcommand{\YS}{\ensuremath{Y_{1}^n}}
%\newcommand{\YW}{\ensuremath{Y_{2}^n}}
%\newcommand{\NW}{\ensuremath{N_{2}^n}}
%\newcommand{\NS}{\ensuremath{N_{1}^n}}
%\newcommand{\nS}{\ensuremath{n_{1}^n}}

\newcommand{\chainsetarg}[1]{\ensuremath{\mathcal{C}(#1)}}
\newcommand{\chainsett}{\ensuremath{\chainsetarg{t}}}

\newcommand{\xsetarg}[1]{\ensuremath{\mathcal{T}(#1)}}
\newcommand{\xsett}{\ensuremath{\xsetarg{t}}}

\newcommand{\zall}{\ensuremath{\mathbf{Z}(t)}}

\newcommand{\myrew}{\ensuremath{\rho}}
\newcommand{\Mstar}{\ensuremath{M^{*}}}

\newcommand{\rhosub}[1]{\ensuremath{\myrew_{#1}}}

\newcommand{\rhoj}{\ensuremath{\rhosub{j}}}
\newcommand{\rhok}{\ensuremath{\rhosub{k}}}
\newcommand{\rhoqi}{\ensuremath{\rhosub{Q_{i}}}}
\newcommand{\rhoqj}{\ensuremath{\rhosub{Q_{j}}}}
\newcommand{\rhoqk}{\ensuremath{\rhosub{Q_{k}}}}
\newcommand{\rhoc}{\ensuremath{\rhosub{E}}}
\newcommand{\rhoqstar}{\ensuremath{\rhosub{\Qstar}}}
\newcommand{\Qstar}{\ensuremath{Q^{*}}}

\newcommand{\rhojarg}[2]{\ensuremath{\rhoj(#1, #2)}}
\newcommand{\rhokarg}[2]{\ensuremath{\rhok(#1, #2)}}
\newcommand{\rhoqiarg}[2]{\ensuremath{\rhoqi(#1, #2)}}
\newcommand{\rhoqjarg}[2]{\ensuremath{\rhoqj(#1, #2)}}
\newcommand{\rhoqkarg}[2]{\ensuremath{\rhoqk(#1, #2)}}
\newcommand{\rhocarg}[2]{\ensuremath{\rhoc(#1, #2)}}

\newcommand{\orhoe}{\ensuremath{\bar{\myrew}_{E}}}
\newcommand{\rholcarg}[2]{\ensuremath{\orhoe(#1, #2)}}

\newcommand{\orhoj}{\ensuremath{\bar{\myrew}_{j}}}
\newcommand{\orhok}{\ensuremath{\bar{\myrew}_{k}}}
\newcommand{\orhoqi}{\ensuremath{\bar{\myrew}_{Q_{i}}}}
\newcommand{\orhoqj}{\ensuremath{\bar{\myrew}_{Q_{j}}}}
\newcommand{\orhoqk}{\ensuremath{\bar{\myrew}_{Q_{k}}}}
\newcommand{\orhoqstar}{\ensuremath{\bar{\myrew}_{\Qstar}}}

\newcommand{\myRew}{\ensuremath{R}}
\newcommand{\oRhojl}{\ensuremath{\bar{\myRew}_{u, l}}}
\newcommand{\oRhoel}{\ensuremath{\bar{\myRew}_{E, l}}}

%\newcommand{\rhoc}[2]{\ensuremath{\rhosub{c}(#1, #2)}}

%			$\rho_{i, l \to m}$ & Reward accumulated by user~$j$ after transition from state $l$ to $m$ \\
%			$\rho_{2, l \to m}$ &Reward accumulated by user~$k$ after transition from state $l$ to $m$ \\ 
%			$\rho_{c, l \to m}$ &Reward accumulated by user~$k$ after transition from state $l$ to $m$ \\ 

\newlength\figureheight 
\newlength\figurewidth 

%% -------------------------------------------------------------
%% *** Converse Chapter Theorems ***
%% -------------------------------------------------------------
\newtheorem{myTheorem}{Theorem}
%\newtheorem{mydef}{Definition}
\newtheorem{myLemma}{Lemma}
\newtheorem{myExample}{Example}
\newtheorem{myprop}{Proposition}
\newtheorem{mycor}{Corollary}

%\newtheorem{fact}[theorem]{Fact}
\newtheorem{fact}{Fact}
\newcommand{\Q}[1]{\ensuremath{Q_{\{#1\}}}}
\newcommand{\tQ}[1]{\ensuremath{\tilde{Q}_{\{#1\}}}}
\newcommand{\tQo}[1]{\ensuremath{\tilde{Q}_{#1}}}
\newcommand{\Qp}[3]{\ensuremath{Q_{#1}^{+}(#2,#3)}}
\newcommand{\myd}[2]{\ensuremath{\delta_{#1}^{#2}}}
\newcommand{\bT}[1]{\ensuremath{\bar{T}_{#1}}}
\newcommand{\bTs}[1]{\ensuremath{\bar{T}_{#1}^{*}}}
\newcommand{\dvec}{\ensuremath{(d_1, d_2, d_3)}}

\newcounter{cnt}
\newcounter{mymagicrownumbers}
\newcommand\myrownumber{\stepcounter{mymagicrownumbers}\arabic{mymagicrownumbers}.}
%\DeclareMathOperator*{\argmin}{argmin}

\newcommand{\bof}[1]{\textbf{#1}}
\newcommand{\YN}[1]{\ensuremath{\tilde{Y}_{#1}^{n}}}
\newcommand{\yN}[1]{\ensuremath{\tilde{y}_{#1}^{n}}}
\newcommand{\tYN}[1]{\ensuremath{\tilde{Y}_{#1}^{n}}}
\newcommand{\tyN}[1]{\ensuremath{\tilde{y}_{#1}^{n}}}
\newcommand{\nN}[1]{\ensuremath{\tilde{n}_{#1}^{n}}}
\newcommand{\NN}[1]{\ensuremath{\tilde{N}_{#1}^{n}}}
\newcommand{\SHatK}[1]{\ensuremath{\hat{S}_{#1}^{m}}}
%\newcommand{\setI}[1]{\ensuremath{\mathcal{I}(n_{#1}^{n})}}
\newcommand{\setI}[1]{\ensuremath{I_{#1}}}
\newcommand{\setInN}{\ensuremath{\mathcal{I}(\etaN)}}
\newcommand{\tSM}[1]{\ensuremath{\tilde{S}_{#1}^m}}
\newcommand{\setURi}{\ensuremath{\mathcal{A}_i}}
\newcommand{\setUR}{\ensuremath{\mathcal{A}}}
\newcommand{\UN}{\ensuremath{U^n}}
\newcommand{\uN}{\ensuremath{u^n}}
\newcommand{\setIB}[1]{\ensuremath{\mathcal{I}_{#1}(u^n)}}
\newcommand{\setIBH}[1]{\ensuremath{\mathcal{I}_{#1}(\hat{u}^n)}}
\newcommand{\Prob}{\ensuremath{\operatorname{Pr}}}
\newcommand{\Dstar}[1]{\ensuremath{D^{*}(\epsilon_{#1}; b)}}
%\newcommand{\XC}{\ensuremath{X_{\mathcal{\overline{C}(\nN{i})}}}}
\newcommand{\oCn}{\ensuremath{\overline{\mathcal{C}}(\nN{i}, \nN{j})}}
\newcommand{\Cn}{\ensuremath{\mathcal{C}(\nN{i}, \nN{j})}}
\newcommand{\XC}{\ensuremath{X_{\oCn}}}
\newcommand{\tUi}[1]{\ensuremath{\tilde{U}_{#1}}}
\newcommand{\XCn}{\ensuremath{X_{\oCn}}}
\newcommand{\Keps}{\ensuremath{K(\epsilon_1, \epsilon_2; b)}}
\newcommand{\sethV}{\ensuremath{\mathcal{\hat{V}}(p^n)}}

\newcommand{\setV}{\ensuremath{\mathcal{V}}}
\newcommand{\setU}{\ensuremath{\mathcal{U}}}
\newcommand{\setA}{\ensuremath{\mathcal{A}}}
\newcommand{\setB}{\ensuremath{\mathcal{B}}}
\newcommand{\tUNn}{\ensuremath{\tilde{U}^n(\pN)}}
\newcommand{\tUN}{\ensuremath{\tilde{U}^n}}
\newcommand{\etaN}{\ensuremath{\eta^n}}
\newcommand{\tUNPN}{\ensuremath{\hUN}}

\newcommand{\tqN}{\ensuremath{\tilde{q}^n}}
\newcommand{\tQN}{\ensuremath{\tilde{Q}^n}}
\newcommand{\tQi}[1]{\ensuremath{\tilde{Q}_{#1}}}


\newcommand{\setIpN}{\ensuremath{\mathcal{I}(p^n)}}
\newcommand{\setIqN}{\ensuremath{\mathcal{I}(q^n)}}
\newcommand{\hpB}{\ensuremath{p(\setB; \hat{q})}}
\newcommand{\pq}{\ensuremath{p(q^n; \hat{q})}}
\newcommand{\pA}{\ensuremath{p({\setA; \hat{p})}}}
\newcommand{\pI}{\ensuremath{p({\setInN; \hat{p})}}}
\newcommand{\pN}{\ensuremath{p^n}}
\newcommand{\PN}{\ensuremath{P^n}}
\newcommand{\setW}[1]{\ensuremath{\mathcal{W}_{#1}(\uN)}}
\newcommand{\setWU}[1]{\ensuremath{\mathcal{W}_{#1}(\hUN)}}
\newcommand{\hUN}{\ensuremath{\hat{U}^n}}
\newcommand{\hUi}[1]{\ensuremath{\hat{U}_{#1}}}

%\newcommand{\oCn}{\ensuremath{\overline{\mathcal{C}}(\nN{i}, \nN{j})}}
%\newcommand{\Cn}{\ensuremath{\mathcal{C}(\nN{i}, \nN{j})}}
%\newcommand{\XC}{\ensuremath{X_{\oCn}}}

\newcommand{\YS}{\ensuremath{Y_{1}^n}}
\newcommand{\YW}{\ensuremath{Y_{2}^n}}
\newcommand{\NW}{\ensuremath{N_{2}^n}}
\newcommand{\NS}{\ensuremath{N_{1}^n}}
\newcommand{\nS}{\ensuremath{\eta_{1}^n}}


%% GWW DEFINITIONS AND ABBREVIATIONS

% TeX Defs

                 % subequations environment, etc
%\usepackage{amssymb}  % get, among others, blackboard bold fonts
                      % defines extra symbols like \gtreqless, etc
%\usepackage{verbatim} % get comment environment, + new verbatim
% \usepackage{amsxtra}  % get, eg, \accentedsymbol
%
%\DeclareMathOperator*{\argmax}{arg\,max}
%\DeclareMathOperator*{\argmin}{arg\,min}
%\DeclareMathOperator*{\argsup}{arg\,sup}
%\DeclareMathOperator*{\arginf}{arg\,inf}
%\DeclareMathOperator{\erfc}{erfc}
%\DeclareMathOperator{\diag}{diag}
%\DeclareMathOperator{\cum}{cum}
%\DeclareMathOperator{\sgn}{sgn}
%\DeclareMathOperator{\tr}{tr}
%\DeclareMathOperator{\spn}{span}
%\DeclareMathOperator{\adj}{adj}
%\DeclareMathOperator{\var}{var}
%\DeclareMathOperator{\cov}{cov}
%\DeclareMathOperator{\sech}{sech}
%\DeclareMathOperator{\sinc}{sinc}
%\DeclareMathOperator*{\lms}{l.i.m.\,}
%\newcommand{\varop}[1]{\var\left[{#1}\right]}
%\newcommand{\covop}[2]{\cov\left({#1},{#2}\right)}
%
%\newcommand{\p}{\partial}

% LIST ENVIRONMENTS

\newcounter{actr}
\newenvironment{alist}%
{\begin{list}{(\alph{actr})}{\usecounter{actr}}}{\end{list}}

\newcounter{ictr}
\newenvironment{ilist}%
{\begin{list}{(\roman{ictr})}{\usecounter{ictr}}}{\end{list}}

\iffalse

% SPACING ENVIRONMENTS

\newenvironment{singlespace}%
{\begin{spacing}{1}}{\end{spacing}}

\newenvironment{onehalfspace}% for 11pt font
{\begin{spacing}{1.21}}{\end{spacing}}

\newenvironment{doublespace}% for 11pt font
{\begin{spacing}{1.62}}{\end{spacing}}

\fi

% THEOREM ENVIRONMENTS
\newtheorem{obs}{Observation}
\newtheorem{remark}{Remark}
\newtheorem{thm}{Theorem}
\newtheorem{lemma}{Lemma}
\newtheorem{claim}{Claim}
\newtheorem{corol}{Corollary}
\newtheorem{prop}{Proposition}
\newtheorem{defn}{Definition}
\newtheorem{fact}{Fact}
%\newenvironment{proof}%
%{\noindent{\em Proof: } \begin{singlespace} \small \noindent}%
%{\noindent\qed \end{singlespace}}
\newenvironment{new-proof}[1]
{{\em Proof }:\\}%
{ \noindent\qed }
%
\newcommand{\abs}[1]{\left|#1\right|}
%\newcommand{\comb}[2]{{#1\choose#2}}
\newcommand{\comb}[2]{\binom{#1}{#2}}
\newcommand{\ie}{i.e.}
\newcommand{\eg}{e.g.}
\newcommand{\etc}{etc.}
\newcommand{\viz}{viz.}
\newcommand{\etal}{et al.}
\newcommand{\cf}{cf.}

\newcommand{\vect}[3]{\begin{bmatrix} #1 & #2 & \cdots & #3 \end{bmatrix}^\T}

\newcommand{\dsp}{.5\baselineskip}             % double space amount
\newcommand{\down}{\vspace{\dsp}}              % double space command
\newcommand{\ddown}{\vspace{\baselineskip}}    % quadruple space command
\newcommand{\spec}{\hspace*{1pt}}              % little bit of space
\newcommand{\ds}{\displaystyle}                % abbreviation
\newcommand{\ts}{\textstyle}                % abbreviation
\newcommand{\nin}{\noindent}                   % noindent abbreviation
\newcommand{\cvar}[1]{\mathrm{var_{#1}\,}}
%\newcommand{\qed}{\rule[0.1ex]{1.4ex}{1.6ex}}
\newcommand{\mycap}[2]{\caption{\sl #2 \label{#1}}}
\newcommand{\subcap}[1]{{\begin{center}\sl #1\end{center}}}
\newcommand{\ditem}[1]{\item[#1 \hspace*{\fill}]}
\newcommand{\appfig}{\vspace*{1in}\begin{center} Figure appended to
                       end of manuscript. \end{center} \vspace*{1in}}
\newcommand{\psx}[1]{\centerline{\epsfxsize=6in \epsfbox{#1}}}
\newcommand{\psy}[1]{\centerline{\epsfysize=7in \epsfbox{#1}}}
\newcommand{\psxs}[2]{\centerline{\epsfxsize=#1in \epsfbox{#2}}}
\newcommand{\psxsbb}[3]{\centerline{\epsfxsize=#1in \epsfbox[#3]{#2}}}
\newcommand{\psys}[2]{\centerline{\epsfysize=#1in \epsfbox{#2}}}
\newcommand{\convsamp}[3]{\left.\left\{#1 \ast #2\right\}\right|_{#3}}
\newcommand{\gap}{\qquad}
\newcommand{\order}[1]{\mathcal{O}\left(#1\right)}
\newcommand{\arror}[3]{\begin{cases} #1 & #2 \\
                                     #3 & \text{otherwise} \end{cases}}
\newcommand{\arrorc}[3]{\begin{cases} #1 & #2 \\
                                     #3 & \text{otherwise,} \end{cases}}
\newcommand{\arrorp}[3]{\begin{cases} #1 & #2 \\
                                     #3 & \text{otherwise.} \end{cases}}
\newcommand{\darror}[4]{\begin{cases} #1 & #2 \\ #3 & #4 \end{cases}}
% \newcommand{\defeq}{\stackrel{\triangle}{=}}
\newcommand{\defeq}{\stackrel{\Delta}{=}}
\newcommand{\msconv}{\stackrel{\mathrm{m.s.}}{\longrightarrow}}
\newcommand{\pwaeconv}{\stackrel{\mathrm{p.w.a.e.}}{\longrightarrow}}
\newcommand{\peq}{\stackrel{\mathcal{P}}{=}}
% \newcommand{\glt}{ \begin{array}{c} \Hh=H_1 \\
%  \renewcommand{\arraystretch}{.3}
%  \begin{array}{c} > \\ < \end{array}
%  \renewcommand{\arraystretch}{1} \\ \Hh=H_0 \end{array}}

\hyphenation{or-tho-nor-mal}
\hyphenation{wave-let wave-lets}

\newcommand{\crb}{Cram\'{e}r-Rao}  % obsolete
\newcommand{\CR}{Cram\'{e}r-Rao}
\newcommand{\KL}{Karhunen-Lo\`{e}ve}
\newcommand{\sE}{\sqrt{E_0}}
\newcommand{\pe}{\Pr(\eps)}
\newcommand{\jw}{j\w}
\newcommand{\ejw}{e^{j\w}}
\newcommand{\ejv}{e^{j\nu}}
\newcommand{\wo}{{\w_0}}
\newcommand{\woh}{{\wh_0}}
\newcommand{\sumi}[1]{\sum_{#1=-\infty}^{+\infty}}
\newcommand{\inti}{\int_{-\infty}^{+\infty}}
\newcommand{\intp}{\int_{-\pi}^{\pi}}
\newcommand{\nintp}{\frac{1}{2\pi}\int_{-\pi}^{\pi}}
\newcommand{\inth}{\int_{0}^{\infty}}
\newcommand{\E}[1]{E\left[{#1}\right]}
\newcommand{\bigE}[1]{E\bigl[{#1}\bigr]}
\newcommand{\BigE}[1]{E\Bigl[{#1}\Bigr]}
\newcommand{\biggE}[1]{E\biggl[{#1}\biggr]}
\newcommand{\BiggE}[1]{E\Biggl[{#1}\Biggr]}
\newcommand{\Prob}[1]{\Pr\left[{#1}\right]}
\newcommand{\Pu}[1]{\Pr\left[{#1}\right]} % obsolete; same as \Prob now
\newcommand{\Pc}[2]{\Pr\left[{#1}\mid{#2}\right]}  % obsolete
\newcommand{\Pcb}[2]{\Pr\left[{#1}\Bigm|{#2}\right]} % obsolete
\newcommand{\Q}[1]{\mathcal{Q}\left({#1}\right)}
\newcommand{\FT}[1]{\mathcal{F}\left\{{#1}\right\}}
\newcommand{\LT}[1]{\mathcal{L}\left\{{#1}\right\}}
\newcommand{\ZT}[1]{\mathcal{Z}\left\{{#1}\right\}}
%\newcommand{\reals}{\mathbf{R}}
\newcommand{\reals}{\mathbb{R}}
%\newcommand{\ints}{\mathbf{Z}}
\newcommand{\ints}{\mathbb{Z}}
\newcommand{\compls}{\mathbb{C}}
\newcommand{\nats}{\mathbb{N}}
\newcommand{\rats}{\mathbb{Q}}
\newcommand{\ltwor}{L^2(\reals)}
\newcommand{\ltwoz}{\ell^2(\ints)}
\newcommand{\ltwow}{L^2(\Omega)}
% \newcommand{\ltwo}{\mathbf{L}^2}
% \newcommand{\ltwor}{\mathbf{L}^2 (\reals)}
% \newcommand{\ltwoz}{\mathbf{l}^2 (\ints)}
\newcommand{\sys}[1]{\mathcal{S}\left\{#1\right\}}
\newcommand{\nn}{\nonumber}
\newcommand{\mrm}{\mathrm}
\newcommand{\ip}[2]{\left\langle{#1},{#2}\right\rangle}
\newcommand{\di}[2]{d\left({#1},{#2}\right)}
\newcommand{\ceil}[1]{\lceil{#1}\rceil}
\newcommand{\floor}[1]{\lfloor{#1}\rfloor}
\newcommand{\phase}{\measuredangle}

\newcommand{\Ht}{\mathrm{H}}
\newcommand{\T}{{\mathrm{T}}}
% \newcommand{\R}{\Re\mathit{e}}
% \newcommand{\I}{\Im\mathit{m}}
\DeclareMathOperator{\R}{Re}
\DeclareMathOperator{\I}{Im}

%% ABBREVIATIONS FOR CHARACTERS IN VARIOUS FONTS

% STANDARD CHARACTERS

\newcommand{\ba}{{\mathbf{a}}}
\newcommand{\bah}{{\hat{\ba}}}
\newcommand{\ah}{{\hat{a}}}
\newcommand{\Ah}{{\hat{A}}}
\newcommand{\cA}{{\mathcal{A}}}
\newcommand{\at}{{\tilde{a}}}
\newcommand{\bat}{{\tilde{\ba}}}
\newcommand{\At}{{\tilde{A}}}
\newcommand{\bA}{{\mathbf{A}}}
\newcommand{\ac}{a^{\ast}}

\newcommand{\bb}{{\mathbf{b}}}
\newcommand{\bbt}{{\tilde{\bb}}}
\newcommand{\cB}{{\mathcal{B}}}
\newcommand{\tb}{{\tilde{b}}}
\newcommand{\tB}{{\tilde{B}}}
\newcommand{\hb}{{\hat{b}}}
\newcommand{\hB}{{\hat{B}}}
\newcommand{\bB}{{\mathbf{B}}}

\newcommand{\bc}{{\mathbf{c}}}
\newcommand{\bch}{{\hat{\mathbf{c}}}}
\newcommand{\bC}{{\mathbf{C}}}
\newcommand{\cC}{{\mathcal{C}}}
\newcommand{\ct}{{\tilde{c}}}
\newcommand{\Ct}{{\tilde{C}}}
\newcommand{\ctc}{\ct^{\ast}}

\newcommand{\bd}{{\mathbf{d}}}
\newcommand{\bD}{{\mathbf{D}}}
\newcommand{\cD}{{\mathcal{D}}}
\newcommand{\hd}{{\hat{d}}}  % old: \dh
\newcommand{\dt}{{\tilde{d}}}
\newcommand{\bdt}{{\tilde{\bd}}}
\newcommand{\Dt}{{\tilde{D}}}
\newcommand{\dtc}{\dt^{\ast}}

\newcommand{\et}{{\tilde{e}}}
\newcommand{\bfe}{{\mathbf{e}}}
\newcommand{\bE}{{\mathbf{E}}}
\newcommand{\cE}{{\mathcal{E}}}
\newcommand{\cEt}{{\tilde{\cE}}}
\newcommand{\cEb}{{\bar{\cE}}}
\newcommand{\bcE}{{\mathbf{\cE}}}  % bf cal E doesn't exist

\newcommand{\bff}{{{\mathbf{f}}}}
\newcommand{\bF}{{\mathbf{F}}}
\newcommand{\cF}{{\mathcal{F}}}
\newcommand{\ft}{{\tilde{f}}}
\newcommand{\Ft}{{\tilde{F}}}
\newcommand{\Fh}{{\hat{F}}}
\newcommand{\ftc}{\ft^{\ast}}
\newcommand{\bft}{{\tilde{\bff}}}
\newcommand{\bFt}{{\tilde{\bF}}}
\newcommand{\fh}{{\hat{f}}}

\newcommand{\bg}{{\mathbf{g}}}
\newcommand{\gt}{{\tilde{g}}}
\newcommand{\bgt}{{\tilde{\bg}}}
\newcommand{\bG}{{\mathbf{G}}}
\newcommand{\cG}{{\mathcal{G}}}
\newcommand{\Gt}{{\tilde{\bG}}}
\newcommand{\Ge}{{G_\mathrm{eff}}}


\newcommand{\hti}{{\tilde{h}}}
\newcommand{\Hti}{{\tilde{H}}}
\newcommand{\bh}{{\mathbf{h}}}
\newcommand{\bht}{{\tilde{\bh}}}
\newcommand{\Hh}{{\hat{H}}}
\newcommand{\bH}{{\mathbf{H}}}
\newcommand{\bHh}{{\hat{\mathbf{H}}}}

\newcommand{\ih}{{\hat{\imath}}}
\newcommand{\bI}{{\mathbf{I}}}
\newcommand{\cI}{{\mathcal{I}}}

\newcommand{\jh}{{\hat{\jmath}}}
\newcommand{\bJ}{{\mathbf{J}}}
\newcommand{\cJ}{{\mathcal{J}}}
\newcommand{\Jt}{{\tilde{J}}}

\newcommand{\bk}{{\mathbf{k}}}
\newcommand{\bK}{{\mathbf{K}}}
\newcommand{\Kt}{{\tilde{K}}}
\newcommand{\Kh}{{\hat{K}}}
\newcommand{\cK}{{\mathcal{K}}}

\newcommand{\cl}{\ell}
\newcommand{\bL}{{\mathbf{L}}}
\newcommand{\cL}{{\mathcal{L}}}

\newcommand{\mb}{{\mathbf{m}}}
\newcommand{\mh}{{\hat{m}}}
\newcommand{\bM}{{\mathbf{M}}}
\newcommand{\bm}{{\mathbf{m}}}
\newcommand{\cM}{{\mathcal{M}}}


\newcommand{\cN}{{\mathcal{N}}}
\newcommand{\CN}{{\mathcal{CN}}}
\newcommand{\Nt}{{\tilde{N}}}
\newcommand{\tN}{{\tilde{N}}}  % backward compatibility

\newcommand{\bo}{{\mathbf{o}}}
\newcommand{\cO}{{\mathcal{O}}}

\newcommand{\bp}{{\mathbf{p}}}
\newcommand{\bP}{{\mathbf{P}}}
\newcommand{\cP}{{\mathcal{P}}}
\newcommand{\ph}{{\hat{p}}}
\newcommand{\Ph}{{\hat{P}}}
\newcommand{\Pt}{{\tilde{P}}}
\newcommand{\Ptt}{{\tilde{\tilde{P}}}}
\newcommand{\bq}{{\mathbf{q}}}
\newcommand{\cQ}{{\mathcal{Q}}}
\newcommand{\bQ}{{\mathbf{Q}}}

\newcommand{\br}{{\mathbf{r}}}
\newcommand{\bR}{{\mathbf{R}}}
\newcommand{\cR}{{\mathcal{R}}}
\newcommand{\Rt}{{\tilde{R}}}

\newcommand{\sh}{{\hat{s}}}
\newcommand{\sck}{{\check{s}}}
\newcommand{\shh}{{\Hat{\Hat{s}}}}
\newcommand{\bs}{{\mathbf{s}}}
\newcommand{\bsh}{{\hat{\mathbf{s}}}}
\newcommand{\bsc}{{\check{\mathbf{s}}}}
\newcommand{\bshh}{{\Hat{\Hat{\mathbf{s}}}}}
\newcommand{\bS}{{\mathbf{S}}}
\newcommand{\cS}{{\mathcal{S}}}
\newcommand{\st}{{\tilde{s}}}

\newcommand{\bT}{{\mathbf{T}}}
\newcommand{\cT}{{\mathcal{T}}}
\newcommand{\bu}{{\mathbf{u}}}
\newcommand{\bU}{{\mathbf{U}}}
\newcommand{\bUt}{{\tilde{\bU}}}
\newcommand{\ut}{{\tilde{u}}}
\newcommand{\cU}{{\mathcal{U}}}

\newcommand{\vh}{{\hat{v}}}
\newcommand{\bv}{{\mathbf{v}}}
\newcommand{\bV}{{\mathbf{V}}}
\newcommand{\cV}{{\mathcal{V}}}

\newcommand{\bw}{{\mathbf{w}}}
\newcommand{\bW}{{\mathbf{W}}}
\newcommand{\cW}{{\mathcal{W}}}
\newcommand{\wt}{{\tilde{w}}}

\newcommand{\bx}{{\mathbf{x}}}
\newcommand{\bxt}{{\tilde{\bx}}}
\newcommand{\xt}{{\tilde{x}}}
\newcommand{\Xt}{{\tilde{X}}}
\newcommand{\bX}{{\mathbf{X}}}
\newcommand{\cX}{{\mathcal{X}}}
\newcommand{\bXt}{{\tilde{\bX}}}
\newcommand{\xh}{{\hat{x}}}
\newcommand{\xc}{{\check{x}}}
\newcommand{\xhh}{{\Hat{\Hat{x}}}}
\newcommand{\bxh}{{\hat{\bx}}}
\newcommand{\bxc}{{\check{\bx}}}
\newcommand{\bxhh}{{\Hat{\hat{\bx}}}}

\newcommand{\cY}{{\mathcal{Y}}}
\newcommand{\by}{{\mathbf{y}}}
\newcommand{\byt}{{\tilde{\by}}}
\newcommand{\bY}{{\mathbf{Y}}}
\newcommand{\Yt}{{\tilde{Y}}}
\newcommand{\yt}{{\tilde{y}}}
\newcommand{\yh}{{\hat{y}}}

\newcommand{\zt}{{\tilde{z}}}
\newcommand{\zh}{{\hat{z}}}
\newcommand{\bz}{{\mathbf{z}}}
\newcommand{\bZ}{{\mathbf{Z}}}
\newcommand{\cZ}{{\mathcal{Z}}}

% GREEK CHARACTERS

\newcommand{\al}{\alpha}
\newcommand{\bal}{{\boldsymbol{\al}}}
\newcommand{\balh}{{\hat{\boldsymbol{\al}}}}
\newcommand{\alh}{{\hat{\al}}}
\newcommand{\aln}{{\bar{\al}}}

\newcommand{\bt}{\boldsymbol{t}}
%\newcommand{\bt}{\beta}
\newcommand{\btt}{{\tilde{\bt}}}
\newcommand{\btht}{{\hat{\bt}}}

\newcommand{\g}{\gamma}
\newcommand{\G}{\Gamma}
\newcommand{\bGa}{{\boldsymbol{\Gamma}}}
\newcommand{\gh}{{\hat{\g}}}

\newcommand{\de}{\delta}
\newcommand{\del}{\delta}
\newcommand{\De}{\Delta}
\newcommand{\Deh}{{\hat{\Delta}}}
\newcommand{\bde}{{\boldsymbol{\de}}}
\newcommand{\bDe}{{\boldsymbol{\De}}}

\newcommand{\e}{\epsilon}
\newcommand{\eps}{\varepsilon}

\newcommand{\etah}{{\hat{\eta}}}
\newcommand{\bpi}{{\boldsymbol{\pi}}}

\newcommand{\pht}{{\tilde{\phi}}}
\newcommand{\Pht}{{\tilde{\Phi}}}

\newcommand{\pst}{{\tilde{\psi}}}
\newcommand{\Pst}{{\tilde{\Psi}}}

\newcommand{\s}{\sigma}
\newcommand{\sih}{\hat{\sigma}}

\newcommand{\z}{\zeta}
\newcommand{\ztt}{{\tilde{\z}}}
\newcommand{\ztb}{{\bar{\z}}}

% \newcommand{\th}{\theta} % symbol name used by other latex package
\newcommand{\thh}{{\hat{\theta}}}
\newcommand{\Thh}{{\hat{\Theta}}}
\newcommand{\Th}{\Theta}
\newcommand{\bth}{{\boldsymbol{\theta}}}
\newcommand{\bTh}{{\boldsymbol{\Theta}}}
\newcommand{\bThh}{{\hat{\bTh}}}
\newcommand{\Tht}{{\tilde{\Theta}}}

\newcommand{\la}{\lambda}
\newcommand{\La}{\Lambda}
\newcommand{\lam}{\lambda}  % backward compatibility
\newcommand{\Lam}{\Lambda}  % backward compatibility
\newcommand{\bLa}{{\boldsymbol{\La}}}
\newcommand{\lah}{{\hat{\lam}}}

\newcommand{\bmu}{{\boldsymbol{\mu}}}

\newcommand{\bXi}{{\boldsymbol{\Xi}}}
%\newcommand{\rvx}{{\mathsf{x}}}
%\newcommand{\rvy}{{\mathrm{y}}}
\newcommand{\bPi}{{\boldsymbol{\Pi}}}

\newcommand{\rht}{{\tilde{\rho}}}
\newcommand{\rhc}{{\check{\rho}}}

\newcommand{\bSi}{{\boldsymbol{\Sigma}}}

\newcommand{\ups}{\upsilon}
\newcommand{\Ups}{\Upsilon}
\newcommand{\bUp}{{\boldsymbol{\Ups}}}

\newcommand{\bPs}{{\boldsymbol{\Psi}}}

\newcommand{\w}{\omega}
\newcommand{\wh}{{\hat{\omega}}}
\newcommand{\W}{\Omega}

\newcommand{\dagg}{\dagger}
\newcommand{\dagbh}{{\mathbf{h}}^\dagger}
\newcommand{\dagbH}{{\mathbf{H}}^\dagger}

%\DeclareMathAlphabet{\mathbsf}{OT1}{cmss}{bx}{n}% bold sans serif
\DeclareMathAlphabet{\mathssf}{OT1}{cmss}{m}{sl}% slanted sans serif

% define some useful uppercase Greek letters in regular and bold sf
\DeclareSymbolFont{bsfletters}{OT1}{cmss}{bx}{n}
\DeclareSymbolFont{ssfletters}{OT1}{cmss}{m}{n}
\DeclareMathSymbol{\bsfGamma}{0}{bsfletters}{'000}
\DeclareMathSymbol{\ssfGamma}{0}{ssfletters}{'000}
\DeclareMathSymbol{\bsfDelta}{0}{bsfletters}{'001}
\DeclareMathSymbol{\ssfDelta}{0}{ssfletters}{'001}
\DeclareMathSymbol{\bsfTheta}{0}{bsfletters}{'002}
\DeclareMathSymbol{\ssfTheta}{0}{ssfletters}{'002}
\DeclareMathSymbol{\bsfLambda}{0}{bsfletters}{'003}
\DeclareMathSymbol{\ssfLambda}{0}{ssfletters}{'003}
\DeclareMathSymbol{\bsfXi}{0}{bsfletters}{'004}
\DeclareMathSymbol{\ssfXi}{0}{ssfletters}{'004}
\DeclareMathSymbol{\bsfPi}{0}{bsfletters}{'005}
\DeclareMathSymbol{\ssfPi}{0}{ssfletters}{'005}
\DeclareMathSymbol{\bsfSigma}{0}{bsfletters}{'006}
\DeclareMathSymbol{\ssfSigma}{0}{ssfletters}{'006}
\DeclareMathSymbol{\bsfUpsilon}{0}{bsfletters}{'007}
\DeclareMathSymbol{\ssfUpsilon}{0}{ssfletters}{'007}
\DeclareMathSymbol{\bsfPhi}{0}{bsfletters}{'010}
\DeclareMathSymbol{\ssfPhi}{0}{ssfletters}{'010}
\DeclareMathSymbol{\bsfPsi}{0}{bsfletters}{'011}
\DeclareMathSymbol{\ssfPsi}{0}{ssfletters}{'011}
\DeclareMathSymbol{\bsfOmega}{0}{bsfletters}{'012}
\DeclareMathSymbol{\ssfOmega}{0}{ssfletters}{'012}

\newcommand{\fxfm}{\stackrel{\mathcal{F}}{\longleftrightarrow}}
\newcommand{\lxfm}{\stackrel{\mathcal{L}}{\longleftrightarrow}}
\newcommand{\zxfm}{\stackrel{\mathcal{Z}}{\longleftrightarrow}}

\DeclareMathOperator*{\gltop}{\gtreqless}
\newcommand{\glt}{\;\gltop^{\Hh=\svH_1}_{\Hh=\svH_0}\;}
\newcommand{\glty}{\;\gltop^{\Hh(\svy)=\svH_1}_{\Hh(\svy)=\svH_0}\;}
\newcommand{\gltby}{\;\gltop^{\Hh(\svby)=\svH_1}_{\Hh(\svby)=\svH_0}\;}
\DeclareMathOperator*{\geltop}{\genfrac{}{}{0pt}{}{\ge}{<}}
\newcommand{\gelty}{\;\geltop^{\Hh(\svy)=\svH_1}_{\Hh(\svy)=\svH_0}\;}
\newcommand{\geltby}{\;\geltop^{\Hh(\svby)=\svH_1}_{\Hh(\svby)=\svH_0}\;}
\renewcommand{\pe}{\Pr(e)}
\renewcommand{\defeq}{\triangleq}
\newcommand{\like}{\svlike}
\newcommand{\rvlike}{\mathssf{L}}
\newcommand{\sst}{\cl}
\newcommand{\svlike}{L}
\newcommand{\llike}{\rvllike}
\newcommand{\rvllike}{\cl}
\newcommand{\svllike}{l}
\newcommand{\bllike}{\rvbllike}
\newcommand{\rvbllike}{\boldsymbol{\cl}}
\newcommand{\svbllike}{\mathbf{l}}
\newcommand{\Qb}{\overline{Q}}
\renewcommand{\comb}[2]{\binom{#1}{#2}}


%% Random/sample variable/vector declarations.  Please add in alphabetical
%% order.  First section is for capitals.  Second for lower case.
% Capitals
\newcommand{\rvA}{{\mathssf{A}}}    % A
\newcommand{\svA}{A}
\newcommand{\rvbA}{{\mathbsf{A}}}
\newcommand{\svbA}{{\mathbf{A}}}
\newcommand{\rvD}{{\mathssf{D}}}    % D
\newcommand{\svD}{D}
\newcommand{\rvbD}{{\mathbsf{D}}}
\newcommand{\svbD}{{\mathbf{D}}}
\newcommand{\rvFh}{{\hat{\mathssf{F}}}} % F
\newcommand{\rvF}{{\mathssf{F}}}
\newcommand{\rvHh}{{\hat{\mathssf{H}}}} % H
\newcommand{\rvH}{{\mathssf{H}}}
\newcommand{\svH}{H}
\newcommand{\svHh}{{\hat{\svH}}}
\newcommand{\rvL}{{\mathssf{L}}}    % L
\newcommand{\rvN}{{\mathssf{N}}}    % N
\newcommand{\rvR}{{\mathssf{R}}}    % R
\newcommand{\rvRh}{{\hat{\rvR}}}
\newcommand{\rvS}{{\mathssf{S}}}    % S
\newcommand{\rvSh}{{\hat{\rvS}}}
\newcommand{\rvW}{{\mathssf{W}}}    % W
\newcommand{\rvX}{{\mathssf{X}}}    % X, random variable
\newcommand{\svX}{X}
\newcommand{\rvXt}{{\tilde{\rvX}}}
\newcommand{\rvY}{{\mathssf{Y}}}    % Y
\newcommand{\rvZ}{{\mathssf{Z}}}    % Z

\newcommand{\rva}{{\mathssf{a}}}    % a
\newcommand{\rvah}{{\hat{\rva}}}
\newcommand{\sva}{a}
\newcommand{\svah}{{\hat{\sva}}}
\newcommand{\rvba}{{\mathbsf{a}}}
\newcommand{\svba}{{\mathbf{a}}}
\newcommand{\rvhba}{\hat{{{\mathbsf{{a}}}}}}


\newcommand{\rvb}{{\mathssf{b}}}    % b
\newcommand{\rvbB}{{\mathbsf{B}}}   % b
\newcommand{\rvbb}{{\mathbsf{b}}}
\newcommand{\rvhbb}{\hat{{{\mathbsf{{b}}}}}}
\newcommand{\svbb}{{\mathbf{b}}}

\newcommand{\rvc}{{\mathssf{c}}}    % c
\newcommand{\rvch}{{\hat{\rvc}}}
\newcommand{\svc}{c}
\newcommand{\svch}{{\hat{\svc}}}
\newcommand{\rvbc}{{\mathbsf{c}}}
\newcommand{\svbc}{{\mathbf{c}}}

\newcommand{\rvd}{{\mathssf{d}}}    % d
\newcommand{\rvdh}{{\hat{\rvd}}}
\newcommand{\svd}{d}
\newcommand{\svdh}{{\hat{\svd}}}
\newcommand{\rvbd}{{\mathbsf{d}}}
\newcommand{\svbd}{{\mathbf{d}}}



\newcommand{\rve}{{\mathssf{e}}}    % e
\newcommand{\sve}{e}
\newcommand{\rvbe}{{\mathbsf{e}}}
\newcommand{\svbe}{{\mathbf{e}}}
\newcommand{\rvf}{{\mathssf{f}}}    % f
\newcommand{\rvhf}{{\hat{{\mathssf{f}}}}}    % f

\newcommand{\svf}{f}
\newcommand{\rvbf}{{\mathbsf{f}}}
\newcommand{\svbf}{{\mathbf{f}}}
\newcommand{\rvg}{{\mathssf{g}}}    % g
\newcommand{\svg}{g}
\newcommand{\rvbg}{{\mathbsf{g}}}
\newcommand{\rvbG}{{\mathbsf{G}}}
\newcommand{\svbg}{{\mathbf{g}}}
\newcommand{\rvh}{{\mathssf{h}}}    % h
\newcommand{\svh}{h}
\newcommand{\rvbh}{{\mathbsf{h}}}
\newcommand{\rvbH}{{\mathbsf{H}}}
\newcommand{\svbh}{{\mathbf{h}}}
\newcommand{\rvk}{{\mathssf{k}}}    % k
\newcommand{\rvhk}{{\hat{\mathssf{k}}}}    % k

\newcommand{\svk}{k}
\newcommand{\rvl}{{\mathssf{l}}}    % l

\newcommand{\rvm}{{\mathssf{m}}}    % m
\newcommand{\svm}{m}
\newcommand{\rvbm}{{\mathbsf{m}}}
\newcommand{\svbm}{{\mathbf{m}}}
\newcommand{\rvn}{{\mathssf{n}}}    % n
\newcommand{\rvbn}{{\mathbsf{n}}}
\newcommand{\rvp}{{\mathssf{p}}}    % p
\newcommand{\svp}{p}
\newcommand{\rvq}{{\mathssf{q}}}    % q
\newcommand{\svq}{q}
\newcommand{\svQ}{Q}
\newcommand{\rvr}{{\mathssf{r}}}    % r
\newcommand{\rvbr}{{\mathbsf{r}}}
\newcommand{\svr}{r}
\newcommand{\rvs}{{\mathssf{s}}}    % s
\newcommand{\rvbs}{{\mathbsf{s}}}
\newcommand{\svs}{s}
\newcommand{\svbs}{{\mathbf{s}}}
\newcommand{\rvt}{{\mathssf{t}}}    % t
\newcommand{\rvbt}{{\mathbsf{t}}}
\newcommand{\rvhbt}{\hat{{\mathbsf{t}}}}
\newcommand{\svt}{t}
\newcommand{\svbt}{{\mathbf{t}}}
\newcommand{\rvu}{{\mathssf{u}}}    % u
\newcommand{\svu}{u}
\newcommand{\svuh}{{\hat{\svu}}}
\newcommand{\rvbu}{{\mathbsf{u}}}
\newcommand{\rvbU}{{\mathbsf{U}}}
\newcommand{\svbu}{{\mathbf{u}}}
\newcommand{\rvv}{{\mathssf{v}}}    % v
\newcommand{\svv}{v}
\newcommand{\svvh}{{\hat{\svv}}}
\newcommand{\rvbv}{{\mathbsf{v}}}
\newcommand{\rvbV}{{\mathbsf{V}}}
\newcommand{\svbv}{{\mathbf{v}}}
\newcommand{\rvvh}{{\hat{\rvv}}}
\newcommand{\rvw}{{\mathssf{w}}}    % w
\newcommand{\svw}{w}
\newcommand{\rvwh}{{\hat{\rvw}}}
\newcommand{\svwh}{{\hat{\svw}}}
\newcommand{\rvbw}{{\mathbsf{w}}}
\newcommand{\svbw}{{\mathbf{w}}}
\newcommand{\rvx}{{\mathssf{x}}}    % x, random variable
\newcommand{\rvxh}{{\hat{\rvx}}}
\newcommand{\rvxt}{{\tilde{\rvx}}}
\newcommand{\svx}{x}            % sample value
\newcommand{\svxh}{{\hat{\svx}}}
\newcommand{\svxt}{{\tilde{\svx}}}
\newcommand{\rvbx}{{\mathbsf{x}}}
\newcommand{\rvbxh}{{\hat{\rvbx}}}
\newcommand{\rvbxt}{{\tilde{\rvbx}}}
\newcommand{\svbx}{{\mathbf{\svx}}}
\newcommand{\svbxt}{{\tilde{\svbx}}}
\newcommand{\svbxh}{{\hat{\mathbf{x}}}}
\newcommand{\rvy}{{\mathssf{y}}}    % y
\newcommand{\rvyh}{{\hat{\mathssf{y}}}}
\newcommand{\svy}{y}
\newcommand{\rvyt}{{\tilde{\rvy}}}
\newcommand{\svyt}{{\tilde{\svy}}}
\newcommand{\svyh}{{\hat{\svy}}}
\newcommand{\rvby}{{\mathbsf{y}}}
\newcommand{\rvbyt}{{\tilde{\rvby}}}
\newcommand{\svby}{{\mathbf{y}}}
\newcommand{\svbyt}{{\tilde{\svby}}}
\newcommand{\rvz}{{\mathssf{z}}}    % z
\newcommand{\rvzh}{{\hat{\rvz}}}
\newcommand{\rvzt}{{\tilde{\rvz}}}
\newcommand{\svz}{z}
\newcommand{\svzh}{{\hat{\svz}}}
\newcommand{\rvbz}{{\mathbsf{z}}}
\newcommand{\svbz}{{\mathbf{z}}}

\newcommand{\rvB}{{\mathssf{B}}}
\newcommand{\rvJ}{{\mathssf{J}}}
\newcommand{\rvK}{{\mathssf{K}}}
\newcommand{\rvT}{{\mathssf{T}}}
\newcommand{\rvU}{{\mathssf{U}}}
\newcommand{\rvV}{{\mathssf{V}}}

% Handle uppercase Greek differently
\newcommand{\rvTh}{\ssfTheta}
\newcommand{\svTh}{\Theta}
\newcommand{\rvbTh}{\bsfTheta}
\newcommand{\svbTh}{\boldsymbol{\Theta}}
\newcommand{\rvPh}{\ssfPhi}
\newcommand{\svPh}{\Phi}
\newcommand{\rvbPh}{\bsfPhi}
\newcommand{\svbPh}{\boldsymbol{\Phi}}

\newcommand{\ddx}{\frac{\p}{\p \svx}}
\newcommand{\ddbx}{\frac{\p}{\p\svbx}}

%  --add new macros below this line--

% \newcommand{\iid}{\emph{i.i.d.}}
\newcommand{\iid}{i.i.d.}

\newcommand{\corolref}[1]{Corollary~\mbox{\ref{#1}}}
\newcommand{\thrmref}[1]{Theorem~\mbox{\ref{#1}}}
\newcommand{\lemref}[1]{Lemma~\mbox{\ref{#1}}}
\newcommand{\figref}[1]{Figure~\mbox{\ref{#1}}}
\newcommand{\secref}[1]{Section~\mbox{\ref{#1}}}
\newcommand{\chapref}[1]{Chapter~\mbox{\ref{#1}}}
\newcommand{\appref}[1]{Appendix~\mbox{\ref{#1}}}


% A popular package from the American Mathematical Society that provides
% many useful and powerful commands for dealing with mathematics. If using
% it, be sure to load this package with the cmex10 option to ensure that
% only type 1 fonts will utilized at all point sizes. Without this option,
% it is possible that some math symbols, particularly those within
% footnotes, will be rendered in bitmap form which will result in a
% document that can not be IEEE Xplore compliant!
%
% Also, note that the amsmath package sets \interdisplaylinepenalty to 10000
% thus preventing page breaks from occurring within multiline equations. Use:
%\interdisplaylinepenalty=2500
% after loading amsmath to restore such page breaks as IEEEtran.cls normally
% does. amsmath.sty is already installed on most LaTeX systems. The latest
% version and documentation can be obtained at:
% http://www.ctan.org/tex-archive/macros/latex/required/amslatex/math/





% *** SPECIALIZED LIST PACKAGES ***
%
%\usepackage{algorithmic}
% algorithmic.sty was written by Peter Williams and Rogerio Brito.
% This package provides an algorithmic environment fo describing algorithms.
% You can use the algorithmic environment in-text or within a figure
% environment to provide for a floating algorithm. Do NOT use the algorithm
% floating environment provided by algorithm.sty (by the same authors) or
% algorithm2e.sty (by Christophe Fiorio) as IEEE does not use dedicated
% algorithm float types and packages that provide these will not provide
% correct IEEE style captions. The latest version and documentation of
% algorithmic.sty can be obtained at:
% http://www.ctan.org/tex-archive/macros/latex/contrib/algorithms/
% There is also a support site at:
% http://algorithms.berlios.de/index.html
% Also of interest may be the (relatively newer and more customizable)
% algorithmicx.sty package by Szasz Janos:
% http://www.ctan.org/tex-archive/macros/latex/contrib/algorithmicx/



% *** ALIGNMENT PACKAGES ***
%
%\usepackage{array}
% Frank Mittelbach's and David Carlisle's array.sty patches and improves
% the standard LaTeX2e array and tabular environments to provide better
% appearance and additional user controls. As the default LaTeX2e table
% generation code is lacking to the point of almost being broken with
% respect to the quality of the end results, all users are strongly
% advised to use an enhanced (at the very least that provided by array.sty)
% set of table tools. array.sty is already installed on most systems. The
% latest version and documentation can be obtained at:
% http://www.ctan.org/tex-archive/macros/latex/required/tools/


%\usepackage{mdwmath}
%\usepackage{mdwtab}
% Also highly recommended is Mark Wooding's extremely powerful MDW tools,
% especially mdwmath.sty and mdwtab.sty which are used to format equations
% and tables, respectively. The MDWtools set is already installed on most
% LaTeX systems. The lastest version and documentation is available at:
% http://www.ctan.org/tex-archive/macros/latex/contrib/mdwtools/


% IEEEtran contains the IEEEeqnarray family of commands that can be used to
% generate multiline equations as well as matrices, tables, etc., of high
% quality.


%\usepackage{eqparbox}
% Also of notable interest is Scott Pakin's eqparbox package for creating
% (automatically sized) equal width boxes - aka "natural width parboxes".
% Available at:
% http://www.ctan.org/tex-archive/macros/latex/contrib/eqparbox/





% *** SUBFIGURE PACKAGES ***
%\usepackage[tight,footnotesize]{subfigure}
% subfigure.sty was written by Steven Douglas Cochran. This package makes it
% easy to put subfigures in your figures. e.g., "Figure 1a and 1b". For IEEE
% work, it is a good idea to load it with the tight package option to reduce
% the amount of white space around the subfigures. subfigure.sty is already
% installed on most LaTeX systems. The latest version and documentation can
% be obtained at:
% http://www.ctan.org/tex-archive/obsolete/macros/latex/contrib/subfigure/
% subfigure.sty has been superceeded by subfig.sty.



%\usepackage[caption=false]{caption}
\usepackage{caption}
%\usepackage[font=footnotesize]{subfig}
% subfig.sty, also written by Steven Douglas Cochran, is the modern
% replacement for subfigure.sty. However, subfig.sty requires and
% automatically loads Axel Sommerfeldt's caption.sty which will override
% IEEEtran.cls handling of captions and this will result in nonIEEE style
% figure/table captions. To prevent this problem, be sure and preload
% caption.sty with its "caption=false" package option. This is will preserve
% IEEEtran.cls handing of captions. Version 1.3 (2005/06/28) and later
% (recommended due to many improvements over 1.2) of subfig.sty supports
% the caption=false option directly:
\usepackage[caption=false,font=footnotesize]{subfig}
%
% The latest version and documentation can be obtained at:
% http://www.ctan.org/tex-archive/macros/latex/contrib/subfig/
% The latest version and documentation of caption.sty can be obtained at:
% http://www.ctan.org/tex-archive/macros/latex/contrib/caption/




% *** FLOAT PACKAGES ***
%
%\usepackage{fixltx2e}
% fixltx2e, the successor to the earlier fix2col.sty, was written by
% Frank Mittelbach and David Carlisle. This package corrects a few problems
% in the LaTeX2e kernel, the most notable of which is that in current
% LaTeX2e releases, the ordering of single and double column floats is not
% guaranteed to be preserved. Thus, an unpatched LaTeX2e can allow a
% single column figure to be placed prior to an earlier double column
% figure. The latest version and documentation can be found at:
% http://www.ctan.org/tex-archive/macros/latex/base/



%\usepackage{stfloats}
% stfloats.sty was written by Sigitas Tolusis. This package gives LaTeX2e
% the ability to do double column floats at the bottom of the page as well
% as the top. (e.g., "\begin{figure*}[!b]" is not normally possible in
% LaTeX2e). It also provides a command:
%\fnbelowfloat
% to enable the placement of footnotes below bottom floats (the standard
% LaTeX2e kernel puts them above bottom floats). This is an invasive package
% which rewrites many portions of the LaTeX2e float routines. It may not work
% with other packages that modify the LaTeX2e float routines. The latest
% version and documentation can be obtained at:
% http://www.ctan.org/tex-archive/macros/latex/contrib/sttools/
% Documentation is contained in the stfloats.sty comments as well as in the
% presfull.pdf file. Do not use the stfloats baselinefloat ability as IEEE
% does not allow \baselineskip to stretch. Authors submitting work to the
% IEEE should note that IEEE rarely uses double column equations and
% that authors should try to avoid such use. Do not be tempted to use the
% cuted.sty or midfloat.sty packages (also by Sigitas Tolusis) as IEEE does
% not format its papers in such ways.




% *** PDF, URL AND HYPERLINK PACKAGES ***
%
\usepackage{url}
% url.sty was written by Donald Arseneau. It provides better support for
% handling and breaking URLs. url.sty is already installed on most LaTeX
% systems. The latest version can be obtained at:
% http://www.ctan.org/tex-archive/macros/latex/contrib/misc/
% Read the url.sty source comments for usage information. Basically,
% \url{my_url_here}.





% *** Do not adjust lengths that control margins, column widths, etc. ***
% *** Do not use packages that alter fonts (such as pslatex).         ***
% There should be no need to do such things with IEEEtran.cls V1.6 and later.
% (Unless specifically asked to do so by the journal or conference you plan
% to submit to, of course. )

%\usepackage{enumitem}

% correct bad hyphenation here
\hyphenation{op-tical net-works semi-conduc-tor}


\begin{document}
%
% paper title
% can use linebreaks \\ within to get better formatting as desired
%\title{Lossy Broadcasting to Diverse Users: Rateless Codes and Hybrid Approaches}
%\title{Successive Segmentation-based Coding for Broadcasting over Erasure Channels}
\title{Markov Rewards Processes with Impulse Rewards and Absorbing States}
%
% author names and IEEE memberships
% note positions of commas and nonbreaking spaces ( ~ ) LaTeX will not break
% a structure at a ~ so this keeps an author's name from being broken across
% two lines.
% use \thanks{} to gain access to the first footnote area
% a separate \thanks must be used for each paragraph as LaTeX2e's \thanks
% was not built to handle multiple paragraphs
%

%\author{Michael~Shell,~\IEEEmembership{Member,~IEEE,}
%        John~Doe,~\IEEEmembership{Fellow,~OSA,}
%        and~Jane~Doe,~\IEEEmembership{Life~Fellow,~IEEE}% <-this % stops a space

%\thanks{J. Doe and J. Doe are with Anonymous University.}% <-this % stops a space
%\thanks{TCOM version based on Michael Shell's bare{\textunderscore}jrnl.tex version 1.3.}}
%%%
\author{\IEEEauthorblockN{Louis Tan, Kaveh Mahdaviani and Ashish Khisti}
%\author{\IEEEauthorblockN{Louis Tan, Kaveh Mahdaviani and Ashish Khisti~\IEEEmembership{ Member,~IEEE} and Emina Soljanin~\IEEEmembership{Fellow,~IEEE}.}
%\thanks{L.~Tan and A.~Khisti are with the Dept. of Electrical and Computer Engineering, University of Toronto, Toronto, ON, Canada. Y.~Li is with the Dept.\ of Electrical Engineering, UCLA, Box 951594, Los Angeles, CA 90095. E.~Soljanin is with Bell Labs, Alcatel-Lucent, Murray Hill, NJ 07974, USA. Part of this work was presented at the 2013 Information Theory Workshop in Seville, Spain~\cite{TLKS_ITW13}, and at the 2014 International Symposium on Information Theory in Honolulu, Hawaii~\cite{LTKS_ISIT14}.}%
\thanks{L.~Tan, K.~Mahdaviani and A.~Khisti are with the Dept.\ of Electrical and Computer Engineering, University of Toronto, Toronto, ON, Canada (e-mail: louis.tan@mail.utoronto.ca, mahdaviani@cs.toronto.edu, akhisti@ece.utoronto.ca). 
%Part of this work was presented at the 2013 Information Theory Workshop in Seville, Spain~\cite{TLKS_ITW13}, and at the 2014 International Symposium on Information Theory in Honolulu, Hawaii~\cite{LTKS_ISIT14}.
}%
}%
%%%\\
%%%\IEEEauthorblockA{\IEEEauthorrefmark{1}Dept.\ of Electrical Engineering, UCLA, Box 951594, Los Angeles, CA 90095, USA}%, liyao@ucla.edu}
%%%\IEEEauthorblockA{\IEEEauthorrefmark{2}Dept.\ of Electrical \& Computer Engineering, University of Toronto, Toronto, ON M5S 3G4 Canada} %\{ltan, akhisti\}@comm.utoronto.ca}
%%%\IEEEauthorblockA{\IEEEauthorrefmark{3}Bell Labs, Alcatel-Lucent, Murray Hill, NJ 07974, USA}%, \\emina@alcatel-lucent.com}
%%%\IEEEauthorblockA{Email: liyao@ucla.edu, \{ltan, akhisti\}@comm.utoronto.ca and emina@alcatel-lucent.com}
%%%}

%\IEEEauthorblockA{Dept.\ of Electrical \& Computer Engineering\\ University of Toronto\\
%Toronto, ON M5S 3G4 Canada\\ ltan@comm.utoronto.ca} \and \IEEEauthorblockN{Yao Li} \IEEEauthorblockA{Dept.\ of Electrical Engineering, UCLA\\ Los Angeles CA USA\\
%liyao@ucla.edu} \and \IEEEauthorblockN{Ashish Khisti} \IEEEauthorblockA{Dept.\ of Electrical \& Computer Engineering\\ University of Toronto\\
%Toronto, ON M5S 3G4 Canada\\ akhisti@comm.utoronto.ca} \and \IEEEauthorblockN{Emina Soljanin} \IEEEauthorblockA{Bell Labs, Alcatel-Lucent\\ Murray Hill NJ 07974, USA\\
%emina@alcatel-lucent.com}}


%\author{\IEEEauthorblockN{Louis Tan} \IEEEauthorblockA{Dept.\ of Electrical \& Computer Engineering\\ University of Toronto\\
%Toronto, ON M5S 3G4 Canada\\ ltan@comm.utoronto.ca} \and \IEEEauthorblockN{Yao Li} \IEEEauthorblockA{Dept.\ of Electrical Engineering, UCLA\\ Los Angeles CA USA\\
%liyao@ucla.edu} \and \IEEEauthorblockN{Ashish Khisti} \IEEEauthorblockA{Dept.\ of Electrical \& Computer Engineering\\ University of Toronto\\
%Toronto, ON M5S 3G4 Canada\\ akhisti@comm.utoronto.ca} \and \IEEEauthorblockN{Emina Soljanin} \IEEEauthorblockA{Bell Labs, Alcatel-Lucent\\ Murray Hill NJ 07974, USA\\
%emina@alcatel-lucent.com}}

%\author{\IEEEauthorblockN{Louis Tan and Ashish Khisti} \IEEEauthorblockA{Dept.\ of Electrical \& Computer Engineering\\ University of Toronto\\
%Toronto, ON M5S 3G4 Canada\\ \{ltan, akhisti\}@comm.utoronto.ca} \and \IEEEauthorblockN{Emina Soljanin} \IEEEauthorblockA{Bell Labs, Alcatel-Lucent\\ Murray Hill NJ 07974, USA\\
%emina@alcatel-lucent.com}}

% note the % following the last \IEEEmembership and also \thanks -
% these prevent an unwanted space from occurring between the last author name
% and the end of the author line. i.e., if you had this:
%
% \author{....lastname \thanks{...} \thanks{...} }
%                     ^------------^------------^----Do not want these spaces!
%
% a space would be appended to the last name and could cause every name on that
% line to be shifted left slightly. This is one of those "LaTeX things". For
% instance, "\textbf{A} \textbf{B}" will typeset as "A B" not "AB". To get
% "AB" then you have to do: "\textbf{A}\textbf{B}"
% \thanks is no different in this regard, so shield the last } of each \thanks
% that ends a line with a % and do not let a space in before the next \thanks.
% Spaces after \IEEEmembership other than the last one are OK (and needed) as
% you are supposed to have spaces between the names. For what it is worth,
% this is a minor point as most people would not even notice if the said evil
% space somehow managed to creep in.



% The paper headers
%\markboth{IEEE Journal on Selected Areas in Communications}%
%{Submitted paper}
% The only time the second header will appear is for the odd numbered pages
% after the title page when using the twoside option.
%
% *** Note that you probably will NOT want to include the author's ***
% *** name in the headers of peer review papers.                   ***
% You can use \ifCLASSOPTIONpeerreview for conditional compilation here if
% you desire.




% If you want to put a publisher's ID mark on the page you can do it like
% this:
%\IEEEpubid{0000--0000/00\$00.00~\copyright~2007 IEEE}
% Remember, if you use this you must call \IEEEpubidadjcol in the second
% column for its text to clear the IEEEpubid mark.



% use for special paper notices
%\IEEEspecialpapernotice{(Invited Paper)}



%\vspace{-4em}


% make the title area
\maketitle


\begin{abstract}
%\boldmath
\iffalse

We study a successive segmentation-based coding scheme for broadcasting a binary source over a multi-receiver erasure broadcast channel. Each receiver has a certain demand on the fraction of source symbols to be reconstructed, and its channel is a memoryless erasure channel.  We study the minimum achievable latency at the source required to simultaneously meet all the receiver constraints. We consider a class of schemes that partition the source sequence into multiple segments and apply a systematic erasure code to each segment. We formulate the optimal choice of segment sizes and code-rates in this class of schemes as a linear programming problem and provide an explicit solution. We further show that the optimal solution can be interpreted as a successive segmentation scheme, which naturally adjusts when users are added or deleted from the system. 

Given the solution for the segment sizes, we then consider possible transmission orderings for \emph{individual} user decoding delay considerations.  We provide closed-form expressions for each individual user's excess latency when parity checks are successively transmitted in both increasing and decreasing order of the segment's coded rate.  Finally, we adapt the segmentation-based coding scheme for transmission across the product of two reversely degraded erasure broadcast channels and numerically show that significant gains are achievable compared to baseline separation-based coding schemes.
\fi
%

We study the expected accumulated reward for a discrete-time Markov reward model with absorbing states. The rewards are impulse rewards, where a reward $\rew{i}{j}$ is accumulated when transitioning from state $i$ to state $j$.  We derive an explicit, single-letter expression for the expected accumulated reward as a function of the number of time steps $n$ and include in our analysis the limit in which $n \to \infty$.

%We study the effects of introducing a feedback channel in the erasure source-broadcast problem for the case of three receivers.  In our problem formulation, we wish to transmit a binary source to three users over the erasure broadcast channel when causal feedback is universally available to the transmitter and all three receivers.  Each receiver requires a certain fraction of the source sequence, and we are interested in the minimum latency, or transmission time, required to serve them all.  
%
%%For broadcasting to three users, 
%We propose a queue-based hybrid digital-analog coding scheme that achieves optimal performance for the duration of analog transmissions.  
%We characterize the number of analog transmissions that can be sent as the solution to a linear program, and furthermore give sufficient conditions for which optimal performance can be achieved by all users.  
%%We show that the number of analog transmissions that can be sent is characterized by the solution of a linear program, and furthermore give sufficient conditions for which all users can be optimal.  
%In some cases, we find that users can be point-to-point optimal regardless of their distortion constraints.  
%
%When the analog transmissions are  insufficient in meeting user demands, we propose two subsequent coding schemes.  The first uses a queue preprocessing strategy for channel coding.  The second method is a novel \emph{chaining algorithm}, which involves the transmitter targeting point-to-point optimal performance for two users as if they were the only users in the network. Meanwhile, the third user simultaneously builds ``chains'' of symbols that he may at times be able to decode based on the channel conditions, and the subsequent reception of additional symbols.  In the analysis of the chaining algorithm, we also solve the more general problem of finding the expected accumulated reward before absorption of a Markov rewards process with impulse rewards and absorbing states.

%Finally, we also provide simulations that highlight the benefits of feedback.


%We consider the problem of deriving an outer bound for the source-broadcast problem involving a common equiprobable source that is to be sent to two receivers subject to an erasure distortion constraint.  The system model we study is the same as in Chapter~\ref{chap:no_feedback}, however our derivation makes two additional assumptions.  The first assumption is that we consider the case when $M \triangleq 1/(1 - \epsilon_2)$ is an integer.  The second assumption is that we consider only the class of non-erasure-randomized codes.  As we explain in Section~\ref{sec:virtual_source}, this is the class of codes for which the positions of erasures in the source reconstruction is determined only by the channel noise realization.  

%The outer bound we present is parameterized by the distortion of user~1, the stronger user.  We assume that $D_1$, the distortion achieved by user~1, is given by $D_1 = \Dstar{1} + \delta$, where $\delta \in [0, \epsilon_1 - \Dstar{1})$, and $\Dstar{i}$ is the point-to-point optimal distortion for user~$i$, given by
%
%\begin{equation}
%	\Dstar{i} = 1 - b(1 - \epsilon_i),
%\end{equation}
%%
%where $b = n/m$ is the number of channel uses per source symbol, i.e., the bandwidth expansion factor.  

%Motivated by error correction coding in multimedia applications, we study the problem of broadcasting a single common source to multiple receivers over heterogenous erasure channels. Each receiver is required to partially reconstruct the source sequence by decoding a certain fraction of the source symbols.  We propose a coding scheme that requires only off-the-shelf erasure codes and can be easily adapted as users join and leave the network. Our scheme involves splitting the source sequence into multiple segments and applying a systematic erasure code to each such segment. We formulate the problem of minimizing the transmission latency at the server as a linear programming problem and explicitly characterize an optimal choice for the code-rates and segment sizes. Through numerical comparisons, we demonstrate that our proposed scheme outperforms both separation-based coding schemes, and degree-optimized rateless codes and performs close to a natural outer {(lower)} bound in certain cases.  %\textcolor{brown}{\sout{We also show how our segmentation-based scheme can be naturally extended if the network users instead listened over \emph{parallel} broadcast channels.}}

%We further study \emph{individual} user decoding delays for various orderings of segments in our scheme.  We  provide closed-form expressions for each individual user's excess latency when parity checks are successively transmitted in both increasing and decreasing order of their segment's coded rate and also qualitatively discuss the merits of each order. %\textcolor{red}{Finally, we also present a natural extension of the segmentation-based scheme to parallel broadcast channels.}

\end{abstract}
% IEEEtran.cls defaults to using nonbold math in the Abstract.
% This preserves the distinction between vectors and scalars. However,
% if the journal you are submitting to favors bold math in the abstract,
% then you can use LaTeX's standard command \boldmath at the very start
% of the abstract to achieve this. Many IEEE journals frown on math
% in the abstract anyway.


%\vspace{-2em}

% Note that keywords are not normally used for peerreview papers.
%\begin{IEEEkeywords}
%Rateless Codes, Packet Erasure Channels, Linear Programming, Unequal Error Protection, Multiple Description Coding, eMBMS
%\end{IEEEkeywords}
\begin{IEEEkeywords}
%Application-Layer Error Correction Coding, Broadcast Channels,  Joint Source-Channel Coding, Linear Programming, Multimedia broadcast/multicast services (MBMS), Rateless Codes, Unequal Error Protection.
\end{IEEEkeywords}



% For peer review papers, you can put extra information on the cover
% page as needed:
% \ifCLASSOPTIONpeerreview
% \begin{center} \bfseries EDICS Category: 3-BBND \end{center}
% \fi
%
% For peerreview papers, this IEEEtran command inserts a page break and
% creates the second title. It will be ignored for other modes.
\IEEEpeerreviewmaketitle

\section{Related Work}
%\mz{We lack a comparison to this paper: https://arxiv.org/abs/2305.14877}
%\anirudh{refine to be more on-topic?}
\iffalse
\paragraph{In-Context Learning} As language models have scaled, the ability to learn in-context, without any weight updates, has emerged. \cite{brown2020language}. While other families of large language models have emerged, in-context learning remains ubiquitous \cite{llama, bloom, gptneo, opt}. Although such as HELM \cite{helm} have arisen for systematic evaluation of \emph{models}, there is no systematic framework to our knowledge for evaluating \emph{prompting methods}, and validating prompt engineering heuristics. The test-suite we propose will ensure that progress in the field of prompt-engineering is structured and objectively evaluated. 

\paragraph{Prompt Engineering Methods} Researchers are interested in the automatic design of high performing instructions for downstream tasks. Some focus on simple heuristics, such as selecting instructions that have the lowest perplexity \cite{lowperplexityprompts}. Other methods try to use large language models to induce an instruction when provided with a few input-output pairs \cite{ape}. Researchers have also used RL objectives to create discrete token sequences that can serve as instructions \cite{rlprompt}. Since the datasets and models used in these works have very little intersection, it is impossible to compare these methods objectively and glean insights. In our work, we evaluate these three methods on a diverse set of tasks and models, and analyze their relative performance. Additionally, we recognize that there are many other interesting angles of prompting that are not covered by instruction engineering \cite{weichain, react, selfconsistency}, but we leave these to future work.

\paragraph{Analysis of Prompting Methods} While most prompt engineering methods focus on accuracy, there are many other interesting dimensions of performance as well. For instance, researchers have found that for most tasks, the selection of demonstrations plays a large role in few-shot accuracy \cite{whatmakesgoodicexamples, selectionmachinetranslation, knnprompting}. Additionally, many researchers have found that even permuting the ordering of a fixed set of demonstrations has a significant effect on downstream accuracy \cite{fantasticallyorderedprompts}. Prompts that are sensitive to the permutation of demonstrations have been shown to also have lower accuracies \cite{relationsensitivityaccuracy}. Especially in low-resource domains, which includes the large public usage of in-context learning, these large swings in accuracy make prompting less dependable. In our test-suite we include sensitivity metrics that go beyond accuracy and allow us to find methods that are not only performant but reliable.

\paragraph{Existing Benchmarks} We recognize that other holistic in-context learning benchmarks exist. BigBench is a large benchmark of 204 tasks that are beyond the capabilities of current LLMs. BigBench seeks to evaluate the few-shot abilities of state of the art large language models, focusing on performance metrics such as accuracy \cite{bigbench}. Similarly, HELM is another benchmark for language model in-context learning ability. Rather than only focusing on performance, HELM branches out and considers many other metrics such as robustness and bias \cite{helm}. Both BigBench and HELM focus on ranking different language model, while fix a generic instruction and prompt format. We instead choose to evaluate instruction induction / selection methods over a fixed set of models. We are the first ever evaluation script that compares different prompt-engineering methods head to head. 
\fi

\paragraph{In-Context Learning and Existing Benchmarks} As language models have scaled, in-context learning has emerged as a popular paradigm and remains ubiquitous among several autoregressive LLM families \cite{brown2020language, llama, bloom, gptneo, opt}. Benchmarks like BigBench \cite{bigbench} and HELM \cite{helm} have been created for the holistic evaluation of these models. BigBench focuses on few-shot abilities of state-of-the-art large language models, while HELM extends to consider metrics like robustness and bias. However, these benchmarks focus on evaluating and ranking \emph{language models}, and do not address the systematic evaluation of \emph{prompting methods}. Although contemporary work by \citet{yang2023improving} also aims to perform a similar systematic analysis of prompting methods, they focus on simple probability-based prompt selection while we evaluate a broader range of methods including trivial instruction baselines, curated manually selected instructions, and sophisticated automated instruction selection.

\paragraph{Automated Prompt Engineering Methods} There has been interest in performing automated prompt-engineering for target downstream tasks within ICL. This has led to the exploration of various prompting methods, ranging from simple heuristics such as selecting instructions with the lowest perplexity \cite{lowperplexityprompts}, inducing instructions from large language models using a few annotated input-output pairs \cite{ape}, to utilizing RL objectives to create discrete token sequences as prompts \cite{rlprompt}. However, these works restrict their evaluation to small sets of models and tasks with little intersection, hindering their objective comparison. %\mz{For paragraphs that only have one work in the last line, try to shorten the paragraph to squeeze in context.}

\paragraph{Understanding in-context learning} There has been much recent work attempting to understand the mechanisms that drive in-context learning. Studies have found that the selection of demonstrations included in prompts significantly impacts few-shot accuracy across most tasks \cite{whatmakesgoodicexamples, selectionmachinetranslation, knnprompting}. Works like \cite{fantasticallyorderedprompts} also show that altering the ordering of a fixed set of demonstrations can affect downstream accuracy. Prompts sensitive to demonstration permutation often exhibit lower accuracies \cite{relationsensitivityaccuracy}, making them less reliable, particularly in low-resource domains.

Our work aims to bridge these gaps by systematically evaluating the efficacy of popular instruction selection approaches over a diverse set of tasks and models, facilitating objective comparison. We evaluate these methods not only on accuracy metrics, but also on sensitivity metrics to glean additional insights. We recognize that other facets of prompting not covered by instruction engineering exist \cite{weichain, react, selfconsistency}, and defer these explorations to future work. 
\subsection{Problem Formulation}

A \emph{discrete-time Markov chain} is a sequence of random variables $\Si{0}, \Si{1}, \ldots $, where for every $i = 0, 1, \ldots$, state $\Si{i}$ takes values from the state space $\myState$, i.e., $\Si{i} \in \myState$, and the probability of transitioning from state $i$ at time $n-1$ to state $j$ at time $n$ given previous states $\Si{0}, \Si{1}, \ldots, \Si{n-1}$ satisfies the Markov property 

\begin{equation}
	\Prob(\Si{n} = j | \Si{n-1} = i, \Si{n- 2} = i_{n-2}, \ldots, \Si{1} = i_{1}) = \Prob(\Si{n} = j | \Si{n-1} = i).
\end{equation}
%
Without loss of generality, we assume the state space, $\myState$, is indexed by a set of integers so that $\myState = \{1, 2, \ldots, |\myState|\}$.

We study \emph{time-homogenous} Markov chains where the transition probabilities do not depend on $n$.  Specifically, at any time $n = 0, 1, \ldots$, the probability of transitioning from state $i$ to state $j$ does not depend on $n$, i.e., $\Prob(\Si{n} = j | \Si{n-1} = i) = \trans{i}{j}$.  At any time $n$, the transitions between states can therefore be described by a $|\myState| \times |\myState|$ \emph{transition matrix} $\transM$ whose $(i, j)$th entry is given by $\trans{i}{j}$ for $i, j \in \myState$.

%We consider discrete-time Markov chains characterized by a state space $\myState$ and a time-homogenous transition matrix $\transM$.  
In addition to being time-homogenous, the Markov chains we study are also \emph{absorbing}.

\begin{mydef}
	A state $i \in \myState$ is said to be \emph{absorbing} if for all $j \in \myState \setminus \{i\}$, $\trans{i}{j} = 0$  and $\trans{i}{i} = 1$.
\end{mydef}

\begin{mydef}
	A state is said to be \emph{transient} if it is not an absorbing state.
\end{mydef}

\begin{mydef}
	A discrete-time Markov chain is said to be an \emph{absorbing} Markov chain if it has at least one absorbing state, and it is possible to reach an absorbing state from any transient state within a finite number of transitions.
\end{mydef}

A \emph{Markov reward process} is a Markov chain that incorporates rewards that are accumulated during the evolution of the Markov chain.  The reward process we consider is additionally characterized by an \emph{impulse-reward matrix}.

\begin{mydef}
	The impulse-reward matrix, $\Rew$, is a $\sizeS \times \sizeS$ matrix whose $(i, j)th$ entry, $\rew{i}{j}$, represents the reward accumulated when transitioning from state $i$ to state $j$.
\end{mydef}
%
%During the evolution of the Markov chain we study, there are also rewards accumulated based on the state transitions.
%
%A discrete-time Markov impulse reward process with absorbing states is a Markov chain is characterized by an impulse-reward matrix $\Rew$ and a transition matrix $\transM$ having the form 
%
For any realization of a sequence of $n+1$ states from time step zero to time step $n$, i.e., given $\Si{0} = i_{0}$, $\Si{1} = i_{1}$, \ldots, $\Si{n} = i_{n}$, we define the accumulated reward $\Rn(\Si{0} = i_{0}, \Si{1} = i_{1}, \ldots, \Si{n} = i_{n})$ as 
%
\begin{equation}
	\label{eq:Rn}
	\Rn(\Si{0} = i_{0}, \Si{1} = i_{1}, \ldots, \Si{n} = i_{n}) = \sum_{k = 1}^{n} \rew{i_{k-1}}{i_{k}}.
\end{equation}

%
%For brevity we will write $\Rn(\Si{0} = i_{0}, \Si{1} = i_{1}, \ldots, \Si{n} = i_{n})$ as $\Rn$ when the context is clear.  

%\begin{mydef}
%\label{def:Rn}
%	Let $\Rn$ be a random variable representing the accumulated reward at time step $n$, where the randomness is taken over the random state sequence $\Si{0}, \Si{1}, \ldots, \Si{n}$.   
%\end{mydef}

\begin{mydef}
\label{def:Rn}
	Let $\Rn$ be a random variable representing the accumulated reward at time step $n$ when the state sequence $\Si{0}, \Si{1}, \ldots, \Si{n}$ is taken to be random.   
\end{mydef}

\begin{mydef}
	\label{def:barRij}
	Let $\barRij{i}{j}$ be the expected value of $\Rn$ given initial state $\Si{0} = i$ and final state $\Si{n} = j$, i.e., $\barRij{i}{j} = \mathbb{E}(\Rn | \Si{0} = i, \Si{n} = j)$.  
\end{mydef}

From Definition~\ref{def:barRij}, the expression for $\barRij{i}{j}$ should be clear when $n = 1$.

\begin{mycor}
	\label{cor:barR1}
	$\barRsubij{1}{i}{j} = \rew{i}{j}$
\end{mycor}

\begin{mydef}
	Let $\barR$ be the $|\myState| \times |\myState|$ matrix whose $(i, j)$th entry is given by $\barRij{i}{j}$.
\end{mydef}

\begin{mydef}
	\label{def:hatRij}
	Let $\hatRij{i}{j}$ be a scaled version of $\barRij{i}{j}$ where $\hatRij{i}{j} = \barRij{i}{j} \times \Prob(\Si{n} = j | \Si{0} = i)$.
\end{mydef}

\begin{mydef}
	Let $\hatR$ be the $|\myState| \times |\myState|$ matrix whose $(i, j)$th entry is given by $\hatRij{i}{j}$.
\end{mydef}


We are often also interested in the expected accumulated reward at time $n$ given only initial state $\Si{0} = i$ irrespective of the state at time $n$.

\begin{mydef}
\label{def:barRi}
	Let $\barRi{i}$ be the expected accumulated reward at time $n$, given initial state $\Si{0} = i$.
\end{mydef}

As we will see in the next section, the probability that the Markov reward process eventually reaches an absorbing state is unity, and so the long-term accumulated reward, also known as the reward until absorption, is of interest.

\begin{mydef}
	Let $\hatRinf(i, j) = \lim_{n \to \infty} \hatRij{i}{j}$ be the long-term value of $\hatRij{i}{j}$.
\end{mydef}

\begin{mydef}
	Let $\hatRinf$ be the $|\myState| \times |\myState|$ matrix whose $(i, j)$th entry is given by $\hatRinf(i, j)$.
\end{mydef}

\begin{mydef}
\label{def:barRinfi}
	Let $\barRinf(i) = \lim_{n \to \infty} \barRi{i}$ be the long-term value of $\barRi{i}$.
\end{mydef}

%\begin{mydef}
%	Let $\barRinf$ be the $\mathbb{R}^{|\myState|}$ vector whose $i$th entry is given by $\barRinf(i)$.
%\end{mydef}

In practice, $\barRinf(i)$ is often the most relevant quantity of interest, since we start the Markov reward process in an initial state and wish to know the accumulated reward before absorption.  We therefore define our problem as that of finding an expression for $\barRinf(i)$.  

We do this by first spending the majority of our efforts in Section~\ref{subsec:scaled_rewards} to derive an expression for the scaled accumulated reward variables, which culminates in the derivation of the transient scaled accumulated reward, $\hatR$, in~Lemma~\ref{lem:hatRn_matrix} and the long-term scaled accumulated reward, $\hatRinf$, in Theorem~\ref{thm:hatR_inf}.  In Section~\ref{subsec:unscaled_rewards}, we then show how to use these expressions to derive expressions for the \emph{unscaled} accumulated reward variables.  These variables are conditioned on initial state $\Si{0} = i$, and consist of the transient accumulated reward, $\barRi{i}$, in Theorem~\ref{thm:barRi}, and the accumulated reward before absorption, $\barRinf(i)$, in Corollary~\ref{cor:barRinfi}.  Finally, if a prior distribution over the initial states is known, we show how to calculate the unconditional transient accumulated reward, $\barR$, in Theorem~\ref{thm:barR_prior} and the unconditional accumulated reward before absorption, $\barRinf$, in Corollary~\ref{cor:barRinf}.



%We define our problem as that of finding a single letter expression for $\hatRinf$.  In the process, we will also derive expressions for the other quantities of interest.



\subsection{Background for Absorbing Markov Chains}

A discrete-time absorbing Markov chain has a state space $\myState$ that can be partitioned into a set of absorbing states, $\StateAbs$, and a set of transient states, $\StateTrans$, such that $\myState = \StateAbs \cup \StateTrans$.  Recall our assumption that the state space is indexed by the integers $\{1, 2, \ldots, |\myState|\}$.  We further assume that transient states have \emph{lower} index values than absorbing states.  If this is the case, we can write out the transition matrix in its canonical form 
%For any absorbing state $i \in \StateAbs$
%transition matrix $\transM$ has a specific form.

\begin{equation}
	\label{eq:transM}
	\transM = 
	\begin{bmatrix}
		\transQ & \transR \\ 
		0 & I
	\end{bmatrix},
\end{equation}
%
where
\begin{enumerate}
%	\item $\transQ$ is a $|\StateTrans| \times |\StateTrans|$ matrix such that for $i, j \in \{1, 2, \ldots, |\StateTrans|\}$, the $(i, j)$th element of $\transQ$ gives the probability of transitioning from transient state $i$ to transient state $j$ %$i, j \in \StateTrans$, the $(i, j)$th element of $\transQ$ gives the probability of transitioning from transient state $i$ to transient state $j$
%	\item $\transR$ is a $|\StateTrans| \times |\StateAbs|$ matrix such that for $(i, j) \in \StateTrans \times \StateAbs$, the $(i, j - |\StateAbs|)$th element gives the probability of transitioning from  transient state $i$ to an absorbing state $j$
	\item $\transQ$ is a $|\StateTrans| \times |\StateTrans|$ matrix whose elements represent the probability of transitioning from one transient state to another transient state
	\item $\transR$ is a $|\StateTrans| \times |\StateAbs|$ matrix whose elements represent the probability of transitioning from a transient state to an absorbing state
	\item The matrix $0$ is the $|\StateAbs| \times |\StateTrans|$ zero matrix whose elements represent the impossibility of transitioning from an absorbing state to a transient state
	\item $I$ is the $|\StateAbs| \times |\StateAbs|$ identity matrix whose elements represent the probability of transitioning from one absorbing state to another absorbing state.
\end{enumerate}

\begin{remark}
	\label{rem:Rew}
	Given the assumption that transient states have \emph{lower} index values than absorbing states, we may also assume without loss of generality that the impulse-reward matrix, $\Rew$, has the form 
	\begin{equation}
	\label{eq:Rew_sub}
		\Rew = 
			\begin{bmatrix}
				\RewSubTransTrans & \RewSubTransAbs \\ 
				0 & 0
			\end{bmatrix},
	\end{equation}
	where 
	\begin{enumerate}
		\item $\RewSubTransTrans$ is a $|\StateTrans| \times |\StateTrans|$ matrix whose elements represent the reward accumulated for transitioning from one transient state to another transient state
		\item $\RewSubTransAbs$ is a $|\StateTrans| \times |\StateAbs|$ matrix whose elements represent the reward accumulated for transitioning from a transient state to an absorbing state
		\item the zero matrices have the appropriate dimensions for $\Rew$ to be a $\sizeS \times \sizeS$ matrix.
	\end{enumerate}
\end{remark}

\begin{mydef}
	\label{def:HtransRew}
	Let $\HtransRew$ be the Hadamard (element-wise) product of the reward matrix $\Rew$ and transition matrix $\transM$, i.e., $\HtransRew = \Rew \odot \transM$ so that $\HtransRewij{i}{j} = \rew{i}{j}\trans{i}{j}$
\end{mydef}

\begin{remark}
	\label{rem:HtransRew}
	Similar to Remark~\ref{rem:Rew}, we may assume without loss of generality that~$\HtransRew$ has the form 
	\begin{equation}
	\label{eq:HtransRew_sub}
		\HtransRew = 
			\begin{bmatrix}
				\HSubTransTrans & \HSubTransAbs \\ 
				0 & 0
			\end{bmatrix}.
	\end{equation}
\end{remark}

From Definition~\ref{def:hatRij}, and Corollary~\ref{cor:barR1}, the expression for $\hatRij{i}{j}$ should be clear when $n = 1$.

\begin{mycor}
	\label{cor:hatR1}
	$\hatRsubij{1}{i}{j} = \HtransRewij{i}{j}$
\end{mycor}

At time step $n$, the probability of being in state $j$ given initial state $i$ is given by the $(i, j)$th entry of $\transM^n$, which is given by the following lemma.

\begin{myLemma}
	\label{lem:transM_n}
	For $n = 1, 2, \ldots $, the transition matrix taken to the $n$th power is given by 
	\begin{equation}
	\label{eq:transM_n}
		\transM^n = 
			\begin{bmatrix}
				\transQ^n & \sum_{i = 0}^{n-1} \transQ^{i}\transR \\ 
				0 & I
			\end{bmatrix}.
	\end{equation}
\end{myLemma}

\begin{proof}
	We proceed by induction.
%	\begin{description}
	\begin{LaTeXdescription}
		\item[Base case] It is readily verified by substituting $n = 1$ into \eqref{eq:transM_n} that we may recover the canonical form of the transition matrix in~\eqref{eq:transM}.
		\item[Inductive Hypothesis] We assume that for $n - 1 \geq 1$, 
			\begin{equation}
				\transM^{n-1} = 
					\begin{bmatrix*}[c]
						\transQ^{n-1} & \sum_{i = 0}^{n-2} \transQ^{i} \transR \\ 
						0 & I
					\end{bmatrix*}.
			\end{equation}
		\item[Induction Step] We have that
			\setcounter{cnt}{1}
			\begin{align}	
				\transM^n &= \transM^{n-1} \transM \\
				\label{eq:transM_n_mult}
				&\stackrel{(\alph{cnt})}{=} \begin{bmatrix*}[c]
						\transQ^{n-1} & \sum_{i = 0}^{n-2} \transQ^{i} \transR \\ 
						0 & I
					\end{bmatrix*}
					\begin{bmatrix}
						\transQ & \transR \\ 
						0 & I
					\end{bmatrix}	,
			\end{align}
			where 
			\begin{enumerate}[(a)]
				\item follows from~\eqref{eq:transM} and the inductive hypothesis.
			\end{enumerate}
			After performing the block matrix multiplication in~\eqref{eq:transM_n_mult}, it is straightforward to see that the right-hand sides of~\eqref{eq:transM_n} and~\eqref{eq:transM_n_mult} are equal.
%	\end{description}
	\end{LaTeXdescription}
\end{proof}

\begin{myLemma}[{\hspace{1sp}\cite[Theorem 11.3]{GrinsteadSnell}}]
	\label{lem:transQn_zero}
	In an absorbing Markov chain, the probability that the process will be absorbed is 1~(i.e., $\transQ^n \to 0$ as $n \to \infty$).
\end{myLemma}

\begin{myLemma}[{\hspace{1sp}\cite[Theorem 11.4]{GrinsteadSnell}}]
	\label{lem:fund_mat}
	For an absorbing Markov chain, the matrix $I - \transQ$ has an inverse, $\fundMat$, termed the \emph{fundamental matrix}, and $\fundMat  = I+\transQ+\transQ^2 + \ldots$.
\end{myLemma}

\begin{myLemma}
	\label{lem:transM_infty}
	The steady state probability $\transM^{\infty} = \lim_{n \to \infty} \transM^{n}$ is given by 
	\begin{equation}
	\label{eq:transM_infty}
		\transM^{\infty} = 
			\begin{bmatrix}
				0 & \fundMat \transR \\ 
				0 & I
			\end{bmatrix}.
	\end{equation}
\end{myLemma}
\begin{proof}
	This follows from substituting the results of Lemmas~\ref{lem:transQn_zero} and~\ref{lem:fund_mat} into the expression for $\transM^n$ in~\ref{eq:transM_n}.
\end{proof}

\begin{mydef}[{\hspace{1sp}\cite[Definition 5.6.8]{HornJohnson}}]
	The \emph{spectral radius} $\rho(A)$ of a matrix $A \in \mathbb{R}^{n \times n}$ is 
	\begin{equation}
		\rho(A) \triangleq \max \{|\lambda| : \lambda \textrm{ is an eigenvalue of } A\}.
	\end{equation}
\end{mydef}

The spectral radius is itself not a matrix norm, however the following corollary states that there exists a norm that is arbitrarily close to the spectral radius.

\begin{myLemma}[{\hspace{1sp}\cite[Lemma 5.6.10]{HornJohnson}}]
	\label{lem:norm_bounds}
	Let $A \in \mathbb{R}^{n \times n}$ and $\epsilon > 0$ be given.  There is at least one matrix norm $\norm{\cdot}$ such that $\rho(A) \leq \norm{A} \leq \rho(A) + \epsilon$.
\end{myLemma}

\begin{myLemma}[{\hspace{1sp}\cite[Lemma 5.6.12]{HornJohnson}}]
	\label{lem:spectral}
	Let $A \in \mathbb{R}^{n \times n}$.  Then $\lim_{n \to \infty} A^n = 0$ if and only if $\rho(A) < 1$.
\end{myLemma}

By Lemmas~\ref{lem:transQn_zero} and~\ref{lem:spectral}, we have that $\rho(\transQ) < 1$.  By appropriately defining $\epsilon$ in Lemma~\ref{lem:norm_bounds}, it then follows that there is a matrix norm for which $\norm{\transQ} < 1$.

\begin{mycor}
	\label{cor:transQ_norm}
	For an absorbing markov chain, there exists a matrix norm for which $\norm{\transQ} < 1$.
\end{mycor}

%\section{Expected Rewards}
%\subsection{Scaled Rewards}
\section{Expected Scaled Rewards}
\label{subsec:scaled_rewards}

We first derive Lemma~\ref{lem:barRij_property} to express $\barRij{i}{j}$ in terms of the elements in both the reward matrix and the transition matrix.  We then use this lemma to derive a recurrence relation for the scaled transient accumulated reward, $\hatR$, in~Lemma~\ref{lem:recurrence}.  We use the recurrence relation to derive an actual expression for the transient scaled reward in Lemma~\ref{lem:hatRn_matrix}.  Finally, we use properties of absorbing Markov chains to derive a single letter expression for the long-term scaled reward, $\hatRinf$, in Theorem~\ref{thm:hatR_inf}.

\begin{myLemma}
	\label{lem:barRij_property}
	Let $\barRij{i}{j}$ be defined as in Definition~\ref{def:barRij}.  For $n = 2, 3, \ldots$
	\begin{equation}
		\barRij{i}{j} = \chainsum \left( \rew{i}{k_1} + \rew{k_1}{k_2} + \ldots + \rew{k_{n-1}}{j}\right) \times \frac{\trans{i}{k_1}\trans{k_1}{k_2} \ldots \trans{k_{n-1}}{j}}{\Prob(\Si{n} = j| \Si{0} = i)}.
	\end{equation}
\end{myLemma}


%%\begin{fleqn}
%%	\begin{align}
%%	\label{eq:sum_of_equations}
%%	\mathbb{E}(\Rn | \Si{0} = i, \Si{n} = j) &=  \phantom{\sum_{k_1} \sum_{k_2} \ldots \sum_{k_{n-1}} \mathbb{E}(\Rn | \Si{0} = i, \Si{1} = k_1, \Si{2} = k_2, \ldots, \Si{n-1} = k_{n-1},\Si{n} = j)} \mathllap{\mathbb{E}_{\Si{1}, \Si{2}, \ldots, \Si{n-1}} \mathbb{E}(\Rn | \Si{0} = i, \Si{1}, \Si{2}, \ldots, \Si{n-1}, \Si{n} = j)} \hidewidth\\
%%%	&\stackrel{(\alph{cnt})}{=}& \sum_{i = 0}^{M-1} H(\SHatK{2}(Y_i^{n}), \mathcal{S}_i | N_0^n) -  H( \mathcal{S}_i | \SHatK{2}(Y_i^{n}), N_0^n)\\
%%%		\addtocounter{cnt}{1}
%%%	&\stackrel{(\alph{cnt})}{=}& \sum_{i = 0}^{M-1} H(\mathcal{S}_i | N_0^n) \\
%%%		\addtocounter{cnt}{1}
%%%	&\stackrel{(\alph{cnt})}{=}& H(\mathcal{S}_0, \mathcal{S}_1, \ldots, \mathcal{S}_{M-1}) | N_0^n) + \sum_{i = 0}^{M-1} I(\mathcal{S}_i; \mathcal{S}_0^{i-1} | N_0^n) \\
%%	&\mathllap{\stackrel{(\alph{cnt})}{=} \sum_{k_1} \sum_{k_2} \ldots \sum_{k_{n-1}} &&\mathbb{E}(\Rn | \Si{0} = i, \Si{1} = k_1, \Si{2} = k_2, \ldots, \Si{n-1} = k_{n-1},\Si{n} = j)} \nonumber \\
%%		&&&\quad  \times \Prob(\Si{1} = k_1, \Si{2} = k_2, \ldots, \Si{n-1} = k_{n-1}| \Si{0} = i, \Si{n} = j)\\
%%	&\stackrel{(\alph{cnt})}{=} \sum_{k_1, k_2, \ldots, k_{n-1}} &&\mathbb{E}(\Rn | \Si{0} = i, \Si{1} = k_1, \Si{2} = k_2, \ldots, \Si{n-1} = k_{n-1},\Si{n} = j) \nonumber \\
%%	&&&\quad \times \Prob(\Si{1} = k_1, \Si{2} = k_2, \ldots, \Si{n} = j | \Si{0} = i)/\Prob(\Si{n}| \Si{0})\\
%%	\end{align}
%%\end{fleqn},

%%% Prob only indented after equal sign
%%%\begin{fleqn}
%%%	\begin{align}
%%%	\label{eq:sum_of_equations}
%%%	\mathbb{E}(\Rn | \Si{0} = i, \Si{n} = j) ={}& \mathbb{E}_{\Si{1}, \Si{2}, \ldots, \Si{n-1}} \mathbb{E}(\Rn | \Si{0} = i, \Si{1}, \Si{2}, \ldots, \Si{n-1}, \Si{n} = j)\\
%%%%	&\stackrel{(\alph{cnt})}{=}& \sum_{i = 0}^{M-1} H(\SHatK{2}(Y_i^{n}), \mathcal{S}_i | N_0^n) -  H( \mathcal{S}_i | \SHatK{2}(Y_i^{n}), N_0^n)\\
%%%%		\addtocounter{cnt}{1}
%%%%	&\stackrel{(\alph{cnt})}{=}& \sum_{i = 0}^{M-1} H(\mathcal{S}_i | N_0^n) \\
%%%%		\addtocounter{cnt}{1}
%%%%	&\stackrel{(\alph{cnt})}{=}& H(\mathcal{S}_0, \mathcal{S}_1, \ldots, \mathcal{S}_{M-1}) | N_0^n) + \sum_{i = 0}^{M-1} I(\mathcal{S}_i; \mathcal{S}_0^{i-1} | N_0^n) \\
%%%	\begin{split}
%%%		\stackrel{(\alph{cnt})}{=}{}& \sum_{k_1} \sum_{k_2} \ldots \sum_{k_{n-1}} \mathbb{E}(\Rn | \Si{0} = i, \Si{1} = k_1, \Si{2} = k_2, \ldots, \Si{n-1} = k_{n-1},\Si{n} = j) \\
%%%		& \times \Prob(\Si{1} = k_1, \Si{2} = k_2, \ldots, \Si{n-1} = k_{n-1}| \Si{0} = i, \Si{n} = j)
%%%	\end{split} \\
%%%	\begin{split}
%%%		\stackrel{(\alph{cnt})}{=}{}& \sum_{k_1, k_2, \ldots, k_{n-1}} \mathbb{E}(\Rn | \Si{0} = i, \Si{1} = k_1, \Si{2} = k_2, \ldots, \Si{n-1} = k_{n-1},\Si{n} = j) \\
%%%		& \times \Prob(\Si{1} = k_1, \Si{2} = k_2, \ldots, \Si{n} = j | \Si{0} = i)/\Prob(\Si{n} = j| \Si{0} = i)
%%%	\end{split} \\
%%%	\end{align}
%%%\end{fleqn},

%\begin{dmath}
%		\stackrel{(\alph{cnt})}{=} \sum_{k_1} \sum_{k_2} \ldots \sum_{k_{n-1}} \mathbb{E}(\Rn | \Si{0} = i, \Si{1} = k_1, \Si{2} = k_2, \ldots, \Si{n-1} = k_{n-1},\Si{n} = j)  \times \Prob(\Si{1} = k_1, \Si{2} = k_2, \ldots, \Si{n-1} = k_{n-1}| \Si{0} = i, \Si{n} = j)
%\end{dmath}

% Quad
\begin{proof}
	We calculate 
	\setcounter{cnt}{1}
	\begin{fleqn}
		\begin{alignat}{2}
	%	\begin{align}
%		&\mathbb{E}(\Rn | \Si{0} = i, \Si{n} = j) \nonumber \\
			&\barRij{i}{j} \nonumber \\
			&\stackrel{(\alph{cnt})}{=} \mathbb{E}_{\Si{1}, \Si{2}, \ldots, \Si{n-1}} \mathbb{E}(\Rn | \Si{0} = i, \Si{1}, \Si{2}, \ldots, \Si{n-1}, \Si{n} = j) \\
				\addtocounter{cnt}{1}
			&= \chainsum  \Big\{ \mathbb{E}(\Rn | \Si{0} = i, \Si{1} = k_1, \Si{2} = k_2, \ldots, \Si{n-1} = k_{n-1},\Si{n} = j) \phantom{\Big\}} \nonumber \\
				& \hphantom{=\chainsum} \qquad \times \vphantom{\Big\{} \Prob(\Si{1} = k_1, \Si{2} = k_2, \ldots, \Si{n-1} = k_{n-1}| \Si{0} = i, \Si{n} = j) \Big\} \\
			&\stackrel{(\alph{cnt})}{=} \chainsum  \left\{ \mathbb{E}(\Rn | \Si{0} = i, \Si{1} = k_1, \Si{2} = k_2, \ldots, \Si{n-1} = k_{n-1},\Si{n} = j) \vphantom{\frac{\Prob\Si{1}}{\Prob\Si{2}}} \right. \nonumber \\
				&\hphantom{=\chainsum} \qquad \times \left. \frac{\Prob(\Si{1} = k_1, \Si{2} = k_2, \ldots, \Si{n} = j | \Si{0} = i)}{\Prob(\Si{n} = j| \Si{0} = i)} \right\} \\
		\addtocounter{cnt}{1}
			&\stackrel{(\alph{cnt})}{=} \chainsum \left( \rew{i}{k_1} + \rew{k_1}{k_2} + \ldots + \rew{k_{n-1}}{j}\right) \times \frac{\Prob(\Si{1} = k_1, \Si{2} = k_2, \ldots, \Si{n} = j | \Si{0} = i)}{\Prob(\Si{n} = j| \Si{0} = i)} \\
		\addtocounter{cnt}{1}
			&\stackrel{(\alph{cnt})}{=} \chainsum \left( \rew{i}{k_1} + \rew{k_1}{k_2} + \ldots + \rew{k_{n-1}}{j}\right) \times \frac{\trans{i}{k_1}\trans{k_1}{k_2} \ldots \trans{k_{n-1}}{j}}{\Prob(\Si{n} = j| \Si{0} = i)} 
		\end{alignat}
	\end{fleqn}
	where 
	\begin{enumerate}[(a)]
		\item follows from Definition~\ref{def:barRij} and the law of total expectation
		\item follows from Bayes' Theorem
		\item follows from the fact that we have conditioned on each state from time $t = 0, 1, 2, \ldots, n$, and so the additive rewards for each transition is known (see~\eqref{eq:Rn})
		\item follows from the Markov property
	\end{enumerate}
\end{proof}

\begin{myLemma}
	Let $\hatR$ be defined as in Definition~\ref{def:hatR}.  For the case of $n = 2$, 
	\begin{equation}
		\label{eq:hatR2}
		\hatRsub{2} = \HtransRew \transM + \transM \HtransRew,
	\end{equation}
	where $\HtransRew$ is defined in Definition~\ref{def:HtransRew} and $\transM$ is given by~\eqref{eq:transM}.
\end{myLemma}
\begin{proof}
	\setcounter{cnt}{1}
	We use Definition~\ref{def:hatRij} and substitute $n = 2$ into Lemma~\ref{lem:barRij_property} to get 
	\begin{align}
		\hatRsubij{2}{i}{j} &= \sum_{k_1 = 1}^{\sizeS} (\rew{i}{k_1} + \rew{k_1}{j}) \trans{i}{k_1}\trans{k_1}{j} \\
		&= \sum_{k_1 = 1}^{\sizeS}(\rew{i}{k_1}\trans{i}{k_1}) \trans{k_1}{j} +  \trans{i}{k_1}(\rew{k_1}{j} \trans{k_1}{j} ) \\
		&\stackrel{(\alph{cnt})}{=} \sum_{k_1 = 1}^{\sizeS}\HtransRewij{i}{k_1} \trans{k_1}{j} +  \trans{i}{k_1}\HtransRewij{k_1}{j} \\	
		\addtocounter{cnt}{1}		
		&\stackrel{(\alph{cnt})}{=} [\HtransRew \transM]_{i, j} +  [\transM \HtransRew]_{i, j} 	
	\end{align}
	where 
\begin{enumerate}[(a)]
	\item follows from Definition~\ref{def:HtransRew}
	\item follows from the definition of matrix multiplication
\end{enumerate}
\end{proof}

% --------------------------------------------------------------
% Recurrence Relation
% --------------------------------------------------------------

We now use Lemma~\ref{lem:barRij_property} to derive a recurrence relation for $\hatR$, the scaled transient accumulated reward at time step $n$.

\begin{myLemma}
\label{lem:recurrence}
	Let $\hatR$ be defined as in Definition~\ref{def:hatR}.  For $n = 2, 3,  \ldots $
	\begin{equation}
	\label{eq:lem_recurrence}
		\hatR = \hatRsub{n-1} \transM + \transMsup{n-1}\HtransRew,
	\end{equation}
	where $\HtransRew$ is defined in Definition~\ref{def:HtransRew} and $\transM$ is given by~\eqref{eq:transM}.
\end{myLemma}

\begin{proof}
We first prove the lemma for $n = 3, 4, \ldots $
\setcounter{cnt}{1}
\begin{fleqn}
	\begin{alignat}{2}
		&\hatRij{i}{j} \nonumber \\
			&\stackrel{(\alph{cnt})}{=} \sum_{k_{n-1}=1}^{\sizeS} \left( \left\{\chainsumOne \left( \rew{i}{k_1} + \rew{k_1}{k_2} + \ldots + \rew{k_{n-2}}{k_{n-1}}\right) \times \trans{i}{k_1}\trans{k_1}{k_2} \ldots \trans{k_{n-2}}{k_{n-1}}\right\} \right.  \\
			&\hphantom{=\sum_{k_{n-1} = 1}^{\sizeS} } \qquad \times \trans{k_{n-1}}{j}  + \left.  \chainsumOne  \rew{k_{n-1}}{j} \times \trans{i}{k_1}\trans{k_1}{k_2} \ldots \trans{k_{n-1}}{j}\right) \\
		\addtocounter{cnt}{1}
			&\stackrel{(\alph{cnt})}{=} \sum_{k_{n-1}=1}^{\sizeS} \left(  \Big\{ \hatRsubij{n-1}{i}{k_{n-1}} \Big\} \trans{k_{n-1}}{j} +\chainsumOne  \rew{k_{n-1}}{j} \times \trans{i}{k_1}\trans{k_1}{k_2} \ldots \trans{k_{n-1}}{j} \right) \\
		\addtocounter{cnt}{1}
%			&\hphantom{=\sum_{k_{n-1} = 1}^{\sizeS} }  \qquad + \left. \chainsumOne  \rew{k_{n-1}}{j} \times \trans{i}{k_1}\trans{k_1}{k_2} \ldots \trans{k_{n-1}}{j} \right) \\
			&= \sum_{k_{n-1}=1}^{\sizeS} \left(   \hatRsubij{n-1}{i}{k_{n-1}}  \trans{k_{n-1}}{j} + \rew{k_{n-1}}{j} \trans{k_{n-1}}{j}\chainsumOne   \trans{i}{k_1}\trans{k_1}{k_2} \ldots \trans{k_{n-2}}{k_{n-1}} \right) \\
			&\stackrel{(\alph{cnt})}{=} \sum_{k_{n-1}=1}^{\sizeS}   \hatRsubij{n-1}{i}{k_{n-1}}  \trans{k_{n-1}}{j} + \rew{k_{n-1}}{j} \trans{k_{n-1}}{j}\transMsupij{n-1}{i}{k_{n-1}} \\
		\addtocounter{cnt}{1}
			&\stackrel{(\alph{cnt})}{=} \sum_{k_{n-1}=1}^{\sizeS}   \hatRsubij{n-1}{i}{k_{n-1}}  \trans{k_{n-1}}{j} + \HtransRewij{k_{n-1}}{j} \transMsupij{n-1}{i}{k_{n-1}} \\
		\addtocounter{cnt}{1}
			&\stackrel{(\alph{cnt})}{=}  [\hatRsub{n-1}  \transM]_{i,j} +  [\transMsup{n-1} \HtransRew ]_{i, j}
	\end{alignat}
\end{fleqn}
where 
\begin{enumerate}[(a)]
	\item follows from Definition~\ref{def:hatRij} and rearranging Lemma~\ref{lem:barRij_property}
	\item follows from Definition~\ref{def:hatRij} and the application of Lemma~\ref{lem:barRij_property} for $\barRsubij{n-1}{i}{k_{n-1}}$
	\item follows from Lemma~\ref{lem:transMsup} in Appendix~\ref{sec:matrix_properties}, where we have used the fact that $n \geq 3$
	\item follows from Definition~\ref{def:HtransRew}
	\item follows from the definition of matrix multiplication
\end{enumerate}

We mention that although we derived the lemma assuming $n \in \{3, 4, \ldots\}$, the lemma also holds if $n=2$.  We can see this by using  Corollary~\ref{cor:hatR1} to compare the right-hand-sides of \eqref{eq:hatR2} and \eqref{eq:lem_recurrence} when $n=2$.

\end{proof}

We next use the \emph{recurrence} relation for $\hatR$ in Lemma~\ref{lem:recurrence} to derive an \emph{explicit} expression for $\hatR$, the scaled transient accumulated reward at time step $n$.

% --------------------------------------------------------------
% R for general n
% --------------------------------------------------------------
\begin{myLemma}
	\label{lem:hatRn_matrix}
	Let $\hatR$ be defined as in Definition~\ref{def:hatR}.
	For $n = 1, 2, \ldots $
	\begin{equation}
	\label{eq:hatRn_matrix}
		\hatR = 
		\begin{bmatrix}
			\RewSubD & \RewSubC\\
			0 & 0 
		\end{bmatrix},
	\end{equation}	
	where
	\begin{align}
		\label{eq:RewSubDn}
			\RewSubD &= \sum_{i = 0}^{n-1}\transQ^{i}\HSubTransTrans\transQ^{n - i - 1}, \\
		\label{eq:RewSubCn}
			\RewSubC &= \fundMat(I - \transQ^{n})\HSubTransAbs + \fundMat(I - \transQ^{n-1})\HSubTransTrans\fundMat\transR - \sum_{i = 0}^{n-2}\transQ^{i}\HSubTransTrans\fundMat\transQ^{n - i - 1}\transR,
	\end{align}
	the matrices $\HSubTransTrans$ and $\HSubTransAbs$ are given in Remark~\ref{rem:HtransRew}, $\fundMat$ is the fundamental matrix of Lemma~\ref{lem:fund_mat} and the matrices $\transQ$ and $\transR$ are defined in~\eqref{eq:transM}.
\end{myLemma}

\begin{proof}
	We proceed by induction.
	\begin{LaTeXdescription}
		\item[Base case] We use Corollary~\ref{cor:hatR1} and Remark~\ref{rem:HtransRew} to verify \eqref{eq:hatRn_matrix} for the base case after substituting $n = 1$ into~\eqref{eq:RewSubDn}  and~\eqref{eq:RewSubCn} to get that 
		\setcounter{cnt}{1}
		\begin{align}
			\RewSubDi{1} &= \HSubTransTrans, \\		
			\RewSubCi{1} &= \fundMat(I - \transQ)\HSubTransAbs \\
			&\stackrel{(\alph{cnt})}{=} \HSubTransAbs,
		\end{align} %
		where 
		\begin{enumerate}[(a)]
			\item follows from the definition of the fundamental matrix in Lemma~\ref{lem:fund_mat}.
		\end{enumerate}
		
		\item[Inductive Hypothesis] We assume that for $n - 1 \geq 1$, 
			\begin{align}
				\RewSubDi{n-1} &= \sum_{i = 0}^{n-2}\transQ^{i}\HSubTransTrans\transQ^{n - i - 2}, \\			
				\RewSubCi{n-1} &= \fundMat(I - \transQ^{n-1})\HSubTransAbs + \fundMat(I - \transQ^{n-2})\HSubTransTrans\fundMat\transR - \sum_{i = 0}^{n-3}\transQ^{i}\HSubTransTrans\fundMat\transQ^{n - i - 2}\transR.
			\end{align}
		
		\item[Induction Step] We begin with the recurrence relation in Lemma~\ref{lem:recurrence} from which we get 
			\setcounter{cnt}{1}
			\begin{align}
				\hatR &= \hatRsub{n-1} \transM + \transMsup{n-1}\HtransRew \\
				&\stackrel{(\alph{cnt})}{=} 	
					\begin{bmatrix}
						\RewSubDi{n-1} & \RewSubCi{n-1}\\
						0 & 0
					\end{bmatrix}
					P + \transMsup{n-1}\HtransRew \\
				\addtocounter{cnt}{1}
				&\stackrel{(\alph{cnt})}{=} 	
					\begin{bmatrix}
						\RewSubDi{n-1} & \RewSubCi{n-1}\\
						0 & 0
					\end{bmatrix}
					\begin{bmatrix}
						\transQ & \transR \\ 
						0 & I
					\end{bmatrix} + 
					\begin{bmatrix}
						\transQ^{n-1} & \sum_{i = 0}^{n-2} \transQ^{i}\transR \\ 
						0 & I
					\end{bmatrix} H\\
				\addtocounter{cnt}{1}
				&\stackrel{(\alph{cnt})}{=}
					\label{eq:before_block}
					\begin{bmatrix}
						\RewSubDi{n-1} & \RewSubCi{n-1}\\
						0 & 0
					\end{bmatrix}
					\begin{bmatrix}
						\transQ & \transR \\ 
						0 & I
					\end{bmatrix} + 
					\begin{bmatrix}
						\transQ^{n-1} & \sum_{i = 0}^{n-2} \transQ^{i}\transR \\ 
						0 & I
					\end{bmatrix}
					\begin{bmatrix}
						\HSubTransTrans & \HSubTransAbs\\ 
						0 & 0
					\end{bmatrix} 
				\addtocounter{cnt}{1}					
			\end{align}
			where 
			\begin{enumerate}[(a)]
				\item follows from the inductive hypothesis
				\item follows from Lemma~\ref{lem:transM_n}
				\item follows from Remark~\ref{rem:HtransRew}.
			\end{enumerate}

			We first consider $\RewSubDi{n}$.  We perform the block matrix multiplication in~\eqref{eq:before_block} to get that 
			\begin{align}
				\setcounter{cnt}{1}
				\RewSubDi{n} &=	\RewSubDi{n-1} \transQ + \transQ^{n-1}\HSubTransTrans \\
				&\stackrel{(\alph{cnt})}{=} \left( \sum_{i = 0}^{n-2}\transQ^{i}\HSubTransTrans\transQ^{n - i - 2} \right) \transQ +  \transQ^{n-1}\HSubTransTrans\\
				\addtocounter{cnt}{1}	
				&=  \sum_{i = 0}^{n-2}\transQ^{i}\HSubTransTrans\transQ^{n - i - 1}+  \transQ^{n-1}\HSubTransTrans\\
				\label{eq:RewSubDi_final}
				&=  \sum_{i = 0}^{n-1}\transQ^{i}\HSubTransTrans\transQ^{n - i - 1},
			\end{align}
			where 
			\begin{enumerate}[(a)]
				\item follows from the inductive hypothesis.
			\end{enumerate}
			We conclude that~\eqref{eq:RewSubDn} holds after comparing with~\eqref{eq:RewSubDi_final} in the induction step. Next, for $\RewSubCi{n}$, we again perform the block matrix multiplication in~\eqref{eq:before_block} to get that 
			\begin{align}
				\setcounter{cnt}{1}
				\RewSubCi{n} &= \RewSubDi{n-1} \transR +  \RewSubCi{n-1}  + \transQ^{n-1}\HSubTransAbs \\
				&\stackrel{(\alph{cnt})}{=} \left( \sum_{i = 0}^{n-2}\transQ^{i}\HSubTransTrans\transQ^{n - i - 2} \right) \transR + \left( \fundMat(I - \transQ^{n-1})\HSubTransAbs + \fundMat(I - \transQ^{n-2})\HSubTransTrans\fundMat\transR \vphantom{\sum_{i=0}^{n}} \right. \\ \nonumber
				&\qquad \left. - \sum_{i = 0}^{n-3}\transQ^{i}\HSubTransTrans\fundMat\transQ^{n - i - 2}\transR \right) + \transQ^{n-1}\HSubTransAbs \\
				\addtocounter{cnt}{1}				
				&\stackrel{(\alph{cnt})}{=}  \left( \sum_{i = 0}^{n-2}\transQ^{i}\HSubTransTrans \fundMat (I - \transQ)\transQ^{n - i - 2}  \transR  - \sum_{i = 0}^{n-3}\transQ^{i}\HSubTransTrans\fundMat\transQ^{n - i - 2}\transR \right) \\ \nonumber
				&\qquad +  \fundMat(I - \transQ^{n-1})\HSubTransAbs + \fundMat(I - \transQ^{n-2})\HSubTransTrans\fundMat\transR  + \transQ^{n-1}\HSubTransAbs \\
				\addtocounter{cnt}{1}				
				&=  \left( \sum_{i = 0}^{n-2} \left( \transQ^{i}\HSubTransTrans \fundMat \transQ^{n - i - 2}  \transR -  \transQ^{i}\HSubTransTrans \fundMat\transQ^{n - i - 1} \transR \right) - \sum_{i = 0}^{n-3}\transQ^{i}\HSubTransTrans\fundMat\transQ^{n - i - 2}\transR \right) \\ \nonumber 
				&\qquad +  \fundMat(I - \transQ^{n-1})\HSubTransAbs + \fundMat(I - \transQ^{n-2})\HSubTransTrans\fundMat\transR + \transQ^{n-1}\HSubTransAbs \\
				&=  \left(   \transQ^{n-2} \HSubTransTrans  \fundMat  \transR -  \sum_{i = 0}^{n-2} \transQ^{i}\HSubTransTrans \fundMat\transQ^{n - i - 1} \transR \right) \\ \nonumber
				&\qquad +  \fundMat(I - \transQ^{n-1})\HSubTransAbs + \fundMat(I - \transQ^{n-2})\HSubTransTrans\fundMat\transR + \transQ^{n-1}\HSubTransAbs \\
				&=   \fundMat(I - \transQ^{n-1} + \fundMat^{-1}\transQ^{n-1})\HSubTransAbs + \fundMat(I - \transQ^{n-2} + \fundMat^{-1}\transQ^{n-2})\HSubTransTrans\fundMat\transR  \\ \nonumber
				&\qquad -  \sum_{i = 0}^{n-2} \transQ^{i}\HSubTransTrans \fundMat\transQ^{n - i - 1} \transR   \\
				\label{eq:RewSubCn_final}
				&\stackrel{(\alph{cnt})}{=}\fundMat(I - \transQ^{n})\HSubTransAbs + \fundMat(I - \transQ^{n-1})\HSubTransTrans\fundMat\transR  -  \sum_{i = 0}^{n-2} \transQ^{i}\HSubTransTrans \fundMat\transQ^{n - i - 1} \transR,
			\end{align}
			where 
			\begin{enumerate}[(a)]
				\item follows from the inductive hypothesis
				\item and (c) follow from the definition of the fundamental matrix, i.e., $\fundMat^{-1} = (I - \transQ)$ in Lemma~\ref{lem:fund_mat}.
%				\item follows from the definition of the fundamental matrix, i.e., $\fundMat = (I - \transQ)^{-1}$ in Lemma~\ref{lem:fund_mat}.
			\end{enumerate}
			We conclude the proof after comparing~\eqref{eq:RewSubCn} with~\eqref{eq:RewSubCn_final} in the induction step.
	\end{LaTeXdescription}
\end{proof}

% --------------------------------------------------------------
% Limit R_inf
% --------------------------------------------------------------

Finally, after having found an explicit expression for $\hatR$ at time step $n$, we now use properties of absorbing Markov chains to derive an expression for the long-term scaled reward, $\hatRinf$.

\begin{theorem}
	\label{thm:hatR_inf}
	Let $\hatRinf$ be defined as in Definition~\ref{def:hatRinf}. Let $\RewSubDi{\infty} = \lim_{n \to \infty} \RewSubD$, and $\RewSubCi{\infty} = \lim_{n \to \infty} \RewSubC$ where $\RewSubD$ and $\RewSubC$ are given by~\eqref{eq:RewSubDn} and~\eqref{eq:RewSubCn} respectively.  Then $\RewSubDi{\infty} = 0$ and %and define 
%	Let $\hatRinf = \lim_{n\to \infty} \hatR$.  Then 
%	Let $\RewSubCi{\infty} = \lim_{n \to \infty} \RewSubC$.  Then %and define 
	\begin{equation}
	\label{eq:hatRnInf_matrix}
		\hatRinf = 
		\begin{bmatrix}
			0 & \RewSubCi{\infty} \\
			0 & 0 
		\end{bmatrix},
	\end{equation}	
	where
	\begin{align}
		\label{eq:RewSubCInf}
		\RewSubCi{\infty} &= \fundMat(\HSubTransAbs + \HSubTransTrans\fundMat\transR),
	\end{align}
	the matrices $\HSubTransTrans$ and $\HSubTransAbs$ are given in Remark~\ref{rem:HtransRew}, $\fundMat$ is the fundamental matrix of Lemma~\ref{lem:fund_mat} and $\transR$ is defined in~\eqref{eq:transM}.
\end{theorem}

\begin{proof}	
	We begin by writing
	\setcounter{cnt}{1}
	\begin{align}
		\RewSubCi{\infty} &\stackrel{(\alph{cnt})}{=} \lim_{n \to \infty} \left( \fundMat(I - \transQ^{n})\HSubTransAbs + \fundMat(I - \transQ^{n-1})\HSubTransTrans\fundMat\transR - \sum_{i = 0}^{n-2}\transQ^{i}\HSubTransTrans\fundMat\transQ^{n - i - 1}\transR\right) \\
		\addtocounter{cnt}{1}
		\label{eq:RewSubCInf_summation}
		&\stackrel{(\alph{cnt})}{=}  \fundMat\HSubTransAbs + \fundMat\HSubTransTrans\fundMat\transR - \lim_{n \to \infty}  \sum_{i = 0}^{n-2}\transQ^{i}\HSubTransTrans\fundMat\transQ^{n - i - 1}\transR,
		\addtocounter{cnt}{1}
%		\RewSubDi{\infty} &= \lim_{n \to \infty} \sum_{i = 0}^{n-1}\transQ^{i}\HSubTransTrans\transQ^{n - i - 1} \\
%		&= \lim_{n \to \infty} \left( \transQ^{n-1} \HSubTransTrans + \sum_{i = 0}^{n-2}\transQ^{i}\HSubTransTrans\transQ^{n - i - 1} \right)\\
	\end{align}
	where 
	\begin{enumerate}[(a)]
		\item follows from Lemma~\ref{lem:hatRn_matrix}
		\item follows from Lemma~\ref{lem:transQn_zero}.
	\end{enumerate}
	In order to prove~\eqref{eq:RewSubCInf}, we must now show the last term in~\eqref{eq:RewSubCInf_summation} converges to the zero matrix.  We use Corollary~\ref{cor:transQ_norm} and Lemma~\ref{lem:summation_matrix_zero} in Appendix~\ref{sec:matrix_properties} to conclude that this is indeed the case.  
	
	For $\RewSubDi{\infty}$, we have that 
	\begin{align}
		\RewSubDi{\infty} &= \lim_{n \to \infty} \sum_{i = 0}^{n-1}\transQ^{i}\HSubTransTrans\transQ^{n - i - 1} \\
		\label{eq:RewSubDi_inf_proof}
		&= \lim_{n \to \infty} \left( \transQ^{n-1} \HSubTransTrans + \sum_{i = 0}^{n-2}\transQ^{i}\HSubTransTrans\transQ^{n - i - 1} \right)
		\addtocounter{cnt}{1}.
	\end{align}
	We conclude from Lemma~\ref{lem:transQn_zero} that the first term in~\eqref{eq:RewSubDi_inf_proof} converges to the zero matrix.  Finally we again use Corollary~\ref{cor:transQ_norm} and Lemma~\ref{lem:summation_matrix_zero} in Appendix~\ref{sec:matrix_properties} to conclude that the second term in~\eqref{eq:RewSubDi_inf_proof} also converges to the zero matrix so that $\RewSubDi{\infty} = 0$.
\end{proof}

%\subsection{Unscaled Rewards}
\section{Expected Unscaled Rewards}
\label{subsec:unscaled_rewards}

% --------------------------------------------------------------
% R bar inf
% --------------------------------------------------------------

%\begin{mydef}
%	Let $\barRinfij{i}{j} = \lim_{n \to \infty} \barRij{i}{j}$ be the long-term value of $\barRij{i}{j}$.
%\end{mydef}

%\begin{mydef}
%	Let $\barRinf$ be the $|\myState| \times |\myState|$ matrix whose $(i, j)$th entry is given by $\barRinfij{i}{j}$.
%\end{mydef}

%\begin{theorem}
%	\label{thm:barR_inf}
%	Let $\transM^{\infty}$ be the steady state transition matrix in Lemma~\ref{lem:transM_infty}. Then $\barRinf = \hatRinf \odot \transM^{\infty}$
%\end{theorem}

Having found an expression for $\hatR$ in the previous section, we now show that given initial state $\Si{0} = i$, the expected accumulated reward at time $n$ is given by the sum over all columns of the $i$th row of the matrix $\hatR$.
%give the relationship between $\barRi{i}$ and $\hatRij{i}{j}$ according to the law of total probability as follows

%\begin{myTheorem}
\begin{theorem}
\label{thm:barRi}
	Let $\barRi{i}$ be defined as in Definition~\ref{def:barRi}.  Then
\begin{align}
	\barRi{i} &=  \sum_{j = 1}^{|\myState|} \hatRij{i}{j},
%	\addtocounter{cnt}{1}
%	\addtocounter{cnt}{1}	
%	&\stackrel{(\alph{cnt})}{=}  k - \sum_{i = 0}^{k} \epsilon_{2}^{\nRepi{i}} \\
\end{align}
where $\hatRij{i}{j}$ is defined as in Definition~\ref{def:hatRij} and is given by Lemma~\ref{lem:hatRn_matrix}.
%\end{myTheorem}
\end{theorem}
\begin{proof}
	By the law of total expectation we have that 
	\setcounter{cnt}{1}
	\begin{align}
		\barRi{i} &= \sum_{j = 1}^{|\myState|} \barRij{i}{j} \Prob(\Si{n} = j | \Si{0} = i).
	\end{align}
	We conclude the result in Theorem~\ref{thm:barRi} holds by Definition~\ref{def:hatRij}.
\end{proof}

\begin{mycor}
\label{cor:barRinfi}
	Let $\barRinf(i)$ be defined as in Definition~\ref{def:barRinfi}.  Then
\begin{align}
	\barRinf(i) &=  \sum_{j = 1}^{|\myState|} \hatRinf(i, j),
\end{align}
where $\hatRinf(i, j)$ is defined as in Definition~\ref{def:hatRinfij} and is given by Theorem~\ref{thm:hatR_inf}.
\end{mycor}

Similarly, given $\barRi{i}$ and a prior distribution over initial states, we can use the law of total expectation to calculate the unconditional expected value of $\Rn$. 

%\begin{myTheorem}
\begin{theorem}
\label{thm:barR_prior}
	Let $\Rn$ be defined as in Definition~\ref{def:Rn}. If a prior distribution $\Prob(\Si{0})$ over the initial state $\Si{0}$ is known, then $\mathbb{E}\Rn$, the expected value of $\Rn$ is given by
	\begin{equation}
		\mathbb{E}\Rn = \sum_{i = 1}^{|\myState|} \barRi{i} \Prob(\Si{0} = i),
	\end{equation}
where $\barRi{i}$ is defined as in Definition~\ref{def:barRi} and is given by Theorem~\ref{thm:barRi}.
%\end{myTheorem}
\end{theorem}
\begin{mycor}
	Let $\barRinf = \lim_{n \to \infty} \mathbb{E}\Rn$.  Then 
\label{cor:barRinf}
\begin{align}
	\barRinf &=  \sum_{i= 1}^{|\myState|} \barRinf(i)\Prob(\Si{0} = i),
\end{align}
where $\barRinf(i)$ is defined as in Definition~\ref{def:barRinfi} and is given by Corollary~\ref{cor:barRinfi}.
\end{mycor}

Finally, it may be of interest to know the expected accumulated reward after absorption given initial state~$i$ and absorbing state~$j$.

%\begin{myTheorem}
\begin{theorem}
\label{thm:barRinf_absorb}
	Let $i \in \StateTrans$ and $j \in \StateAbs$.  Let $\barRinf(i, j)$ represent the expected accumulated reward after absorption given initial state~$i$ and absorbing state~$j$.  Then 
	\begin{align}
		\barRinf(i, j) &=  \frac{1}{\transM^{\infty}(i, j)}\hatRinf(i, j), 
	\end{align}
	where $\transM^{\infty}(i, j)$ is the $(i, j)th$ entry of $\transM^{\infty}$ given in Lemma~\ref{lem:transM_infty}.
%\end{myTheorem}
\end{theorem}



%\begin{proof}
%	We show that the norm $\norm{\lim_{n \to \infty} g(n, A, Q)} = 0$.  We have that
%	\setcounter{cnt}{1}
%	\begin{align}
%		\norm{\lim_{n \to \infty} g(n, A, Q)} &= \norm{\lim_{n \to \infty} \sum_{i = 0}^{n} \transQ^{i}A\transQ^{n - i - 1}}\\
%		&\stackrel{(\alph{cnt})}{=} \lim_{n \to \infty} \norm{\sum_{i = 0}^{\infty} \transQ^{i} A \transQ^{n - i - 1}} \\		
%		\addtocounter{cnt}{1}
%		&\stackrel{(\alph{cnt})}{\leq} \lim_{n \to \infty} \sum_{i = 0}^{n} \norm{\transQ^{i} A \transQ^{n - i - 1}} \\
%		\addtocounter{cnt}{1}
%		&\stackrel{(\alph{cnt})}{\leq} \lim_{n \to \infty} \sum_{i = 0}^{n} \norm{\transQ}^{i} \cdot \norm{A} \cdot \norm{\transQ}^{n - i - 1} \\
%		\addtocounter{cnt}{1}
%		&= \lim_{n \to \infty}\sum_{i = 0}^{n} \norm{\transQ}^{n - 1} \cdot \norm{A} \\
%		&= \lim_{n \to \infty}n \norm{\transQ}^{n - 1} \cdot \norm{A} \\		
%		&\stackrel{(\alph{cnt})}{=} 0 
%	\end{align}
%	where 
%	\begin{enumerate}[(a)]
%		\item follows from the fact that $\mathbb{R}^{m \times m}$ is a normed  Banach space
%		\item follows from sub-additive property of the matrix norm
%		\item follows from sub-multiplicative property of the matrix norm	
%		\item follows from L'Hospital's Rule and the assumption that $\norm{Q} < 1$
%	\end{enumerate}
%\end{proof}


\pdfoutput=1
\documentclass{article}
\usepackage[final]{pdfpages}
\begin{document}
\includepdf[pages=1-9]{CVPR18VOlearner.pdf}
\includepdf[pages=1-last]{supp.pdf}
\end{document}

%%% \leavevmode
% \\
% \\
% \\
% \\
% \\
\section{Introduction}
\label{introduction}

AutoML is the process by which machine learning models are built automatically for a new dataset. Given a dataset, AutoML systems perform a search over valid data transformations and learners, along with hyper-parameter optimization for each learner~\cite{VolcanoML}. Choosing the transformations and learners over which to search is our focus.
A significant number of systems mine from prior runs of pipelines over a set of datasets to choose transformers and learners that are effective with different types of datasets (e.g. \cite{NEURIPS2018_b59a51a3}, \cite{10.14778/3415478.3415542}, \cite{autosklearn}). Thus, they build a database by actually running different pipelines with a diverse set of datasets to estimate the accuracy of potential pipelines. Hence, they can be used to effectively reduce the search space. A new dataset, based on a set of features (meta-features) is then matched to this database to find the most plausible candidates for both learner selection and hyper-parameter tuning. This process of choosing starting points in the search space is called meta-learning for the cold start problem.  

Other meta-learning approaches include mining existing data science code and their associated datasets to learn from human expertise. The AL~\cite{al} system mined existing Kaggle notebooks using dynamic analysis, i.e., actually running the scripts, and showed that such a system has promise.  However, this meta-learning approach does not scale because it is onerous to execute a large number of pipeline scripts on datasets, preprocessing datasets is never trivial, and older scripts cease to run at all as software evolves. It is not surprising that AL therefore performed dynamic analysis on just nine datasets.

Our system, {\sysname}, provides a scalable meta-learning approach to leverage human expertise, using static analysis to mine pipelines from large repositories of scripts. Static analysis has the advantage of scaling to thousands or millions of scripts \cite{graph4code} easily, but lacks the performance data gathered by dynamic analysis. The {\sysname} meta-learning approach guides the learning process by a scalable dataset similarity search, based on dataset embeddings, to find the most similar datasets and the semantics of ML pipelines applied on them.  Many existing systems, such as Auto-Sklearn \cite{autosklearn} and AL \cite{al}, compute a set of meta-features for each dataset. We developed a deep neural network model to generate embeddings at the granularity of a dataset, e.g., a table or CSV file, to capture similarity at the level of an entire dataset rather than relying on a set of meta-features.
 
Because we use static analysis to capture the semantics of the meta-learning process, we have no mechanism to choose the \textbf{best} pipeline from many seen pipelines, unlike the dynamic execution case where one can rely on runtime to choose the best performing pipeline.  Observing that pipelines are basically workflow graphs, we use graph generator neural models to succinctly capture the statically-observed pipelines for a single dataset. In {\sysname}, we formulate learner selection as a graph generation problem to predict optimized pipelines based on pipelines seen in actual notebooks.

%. This formulation enables {\sysname} for effective pruning of the AutoML search space to predict optimized pipelines based on pipelines seen in actual notebooks.}
%We note that increasingly, state-of-the-art performance in AutoML systems is being generated by more complex pipelines such as Directed Acyclic Graphs (DAGs) \cite{piper} rather than the linear pipelines used in earlier systems.  
 
{\sysname} does learner and transformation selection, and hence is a component of an AutoML systems. To evaluate this component, we integrated it into two existing AutoML systems, FLAML \cite{flaml} and Auto-Sklearn \cite{autosklearn}.  
% We evaluate each system with and without {\sysname}.  
We chose FLAML because it does not yet have any meta-learning component for the cold start problem and instead allows user selection of learners and transformers. The authors of FLAML explicitly pointed to the fact that FLAML might benefit from a meta-learning component and pointed to it as a possibility for future work. For FLAML, if mining historical pipelines provides an advantage, we should improve its performance. We also picked Auto-Sklearn as it does have a learner selection component based on meta-features, as described earlier~\cite{autosklearn2}. For Auto-Sklearn, we should at least match performance if our static mining of pipelines can match their extensive database. For context, we also compared {\sysname} with the recent VolcanoML~\cite{VolcanoML}, which provides an efficient decomposition and execution strategy for the AutoML search space. In contrast, {\sysname} prunes the search space using our meta-learning model to perform hyperparameter optimization only for the most promising candidates. 

The contributions of this paper are the following:
\begin{itemize}
    \item Section ~\ref{sec:mining} defines a scalable meta-learning approach based on representation learning of mined ML pipeline semantics and datasets for over 100 datasets and ~11K Python scripts.  
    \newline
    \item Sections~\ref{sec:kgpipGen} formulates AutoML pipeline generation as a graph generation problem. {\sysname} predicts efficiently an optimized ML pipeline for an unseen dataset based on our meta-learning model.  To the best of our knowledge, {\sysname} is the first approach to formulate  AutoML pipeline generation in such a way.
    \newline
    \item Section~\ref{sec:eval} presents a comprehensive evaluation using a large collection of 121 datasets from major AutoML benchmarks and Kaggle. Our experimental results show that {\sysname} outperforms all existing AutoML systems and achieves state-of-the-art results on the majority of these datasets. {\sysname} significantly improves the performance of both FLAML and Auto-Sklearn in classification and regression tasks. We also outperformed AL in 75 out of 77 datasets and VolcanoML in 75  out of 121 datasets, including 44 datasets used only by VolcanoML~\cite{VolcanoML}.  On average, {\sysname} achieves scores that are statistically better than the means of all other systems. 
\end{itemize}


%This approach does not need to apply cleaning or transformation methods to handle different variances among datasets. Moreover, we do not need to deal with complex analysis, such as dynamic code analysis. Thus, our approach proved to be scalable, as discussed in Sections~\ref{sec:mining}.
%%\input{prior_work}
%%\section{Problem Formulation} \label{ProblemFormulation}

\begin{figure}[tbp]
\begin{center}
\includegraphics[width=8cm]{probFormulation.png}
\end{center}
\vspace{-0.2 in}
\caption{A) The range of the i-th control point that guarantees convex hull property. B) The problem scaled back to [0, 1]. It suffices to consider the concave upper bound function $f(t)$ and the control points picked along the upper bound with equal time spacing. The poly-line of the control polygon formed with these control points is defined as CPETS (in red poly-line).}
\label{Fig:probFormulation}
\vspace{-0.2 in}
\end{figure}

Considering a convex spatio-temporal corridor defined in $\left[t_{1}, t_{2}\right]$. The upper bound function $f_{ub}(t)$ is concave, while the lower bound function $f_{lb}(t)$ is convex. We pick $n+1$ control points in the corridor to form a scaled Bezier curve ~\cite{ding2019safe} of degree $n$ which is defined in $\left[t_{1}, t_{2}\right]$. The vertical axis label $Y$ is a generic spacial label, which could represent $X$, $Y$, $S$, $L$, or any spacial variable, as shown in Fig. \ref{Fig:probFormulation} A. The vertical coordinate of the i-th control point is noted as $s_{i}$. We want to find a sufficient condition to make sure the Bezier curve lies in the corridor.

\begin{theorem}
If the series $s_{i}$ (i=0,\;1,\;...,\;n) satisfies
\begin{equation}\begin{split}
s_{i} \in \left[f_{lb}(t_{1}+\frac{i}{n}(t_{2}-t_{1})), f_{ub}(t_{1}+\frac{i}{n}(t_{2}-t_{1}))\right]
\label{eq.0}
\end{split}\end{equation}
the scaled Bezier curve generated from control points lies in the corridor.
\label{theorem:baseTheorem}
\end{theorem}

Theorem \ref{theorem:baseTheorem} shows that it suffices to pick the control points with equal time spacing, and the range of each control point is the value of the upper bound/lower bound functions at the corresponding time. 

To simplify the proof of theorem \ref{theorem:baseTheorem}, without loss of generality, we can scale the time interval back to $\left[0, 1\right]$, consider the upper bound function $f(t)$ only, and pick the control points along the upper bound function with equal time spacing (which will generate the highest possible Bezier curve). The vertical coordinate of the i-th control point thus satisfies $s_{i}=f(\frac{i}{n})$. The poly-line of the control polygon formed with these control points is defined as CPETS (control polygon of equal time spacing), as shown in the red poly-line in Fig. \ref{Fig:probFormulation} B. Due to the property of concave functions, we have $\text{CPETS} \leq f(t)$. This yields theorem \ref{theorem:equivalentTheorem}, which is equivalent to theorem \ref{theorem:baseTheorem}.

\begin{theorem}
$\forall t\in\left[0, 1\right]$, the Bezier curve $C(t)=\sum_{i=0}^{n}s_{i}B\textsupsub{$n$}{$i$}(t)$ satisfies $C(t)$$\leq$CPETS$\leq$$f(t)$, where $f(t)$ is a concave function, $s_{i}=f(\frac{i}{n})$, and 
$B\textsupsub{$n$}{$i$}(t)$ is the Bernstein polynomial which is defined as $B\textsupsub{$n$}{$i$}(t)=C\textsupsub{$i$}{$n$}t^{i}(1-t)^{n-i}$.
\label{theorem:equivalentTheorem}
\end{theorem}

We will decompose the proof of theorem \ref{theorem:equivalentTheorem} into several lemmas. In this paper, all terms share the same definition as they are defined in theorem~\ref{theorem:equivalentTheorem}, unless stated otherwise.


%%\input{intuition}
%%\begin{table*}[t]
\centering
  \caption{Quantitative comparisons with the optimization-based and efficient methods. Encoding time means the time cost to obtain the unique/pseudo embedding. Our method achieves optimal results in terms of text-alignment, face similarity, and encoding time.}
  \label{tab:main_result}
  \begin{tabular}{cccc}
    \toprule
    Methods & Text-alignment $\uparrow$ & Face similarity $\uparrow$ & Encoding Time $\downarrow$ \\
    \midrule
    Textual Inversion \cite{gal2022image} & 0.213 & 0.326 & 20 min \\ 
    Dreambooth \cite{ruiz2022dreambooth} & 0.217 & 0.425 & 4 min  \\ 
    E4T \cite{gal2023designing} & 0.220 & 0.420 & 20 s \\ 
    Elite \cite{wei2023elite} & 0.196 & 0.450 & 0.05 s\\ 
    \midrule
    Ours & \textbf{0.228} & \textbf{0.467} & \textbf{0.04 s}\\
    \bottomrule
  \end{tabular}
\end{table*}
%%\input{construction_y}
%%\input{properties_y}
%%\section{Numerical reconstructions}\label{sec:reconstructions} %bk 2021-11-28

In this section we give some details and show some reconstructions
of the algorithms described in section \ref{sec:recon_meth}.
We assume single mode excitation, set $\beta_i=1$ and
break the reconstructions into three separate paradigms as previously indicated
depending on which time ranges are feasible to be measured.

\input recon_figs.tex

We assume single mode excitation, set $\beta_i=1$ and 
break the reconstructions into three separate paradigms as indicated
in Section~\ref{sec:recon_meth} 
depending on which time ranges are feasible to be measured.

\noindent
\subsection{Full time measurements.}

The leftmost graph in Figure \ref{fig:h_and_hath} shows the actual time trace
of $h(t)$ 
over the range $0\leq t\leq 40$
for a damping model with three terms 
$b_i\partial_t^{\alpha_i}$, 
$i=1,2,3$.
Note the damped oscillatory behaviour evident and the range shown
is the one we are labelling as ``full-time'' despite the fact that we
will later look at the important case of 
the very long term
behaviour of the solution where the time values are substantially outside
of this range.  This time range terminology is highly dependent on the
values of critical constants in the model.
These include the size of the domain (which determines the scaling of the
eigenvalues $\{\lambda_n\}$), the wave speed $c$ and of course the strength
of the damping given by the coefficients $\{b_i\}$.
In our reconstructions we will typically take these constants to be of
approximate order unity but note the dependence on these quantities
and the effect that a substantial change would make.

To obtain this $h(t)$ our damping constants are of order 
$\sim 0.1-0.2$ and the combined term 
$\Lambda=c^2\lambda$ 
is taken to be unity.
Changes to the former would modify the time-decrease in $h$ and changes to the 
latter would alter the frequency of the oscillations.

The rightmost graph in Figure~\ref{fig:h_and_hath} shows the logarithm of the
Laplace transform $\hat{h}(s)$ together with the values of $\hat{h}_i(s)$ after
iteration $i$ of the scheme.
%%%%%%%%%%%%%%%%%%%
\begin{figure}[ht]
\hbox to \hsize{\hfill\copy\figureone\hfill\copy\figuretwo\hfill}
\small
\caption{\small {\bf Profiles of $h(t)$ and $\hat{h}(s)$, actual and after
iterations 1 and 2.}}
\label{fig:h_and_hath}
\end{figure}
%%%%%%%%%%%%

This demonstrates two important facts.
First, the very fast convergence of the scheme in the sense of the
convergence of the target Laplace transform function as the parameters
$\{\alpha_i,b_i\}$ are resolved.
The actual $\hat{h}(s)$ is shown by the solid black line, the initial approximation
by the dotted red curve and the first iteration by the dashed green curve
which, at this scale is already almost indistinguishable from the actual $\hat{h}$.
Second, given this it should be quite feasible to obtain reasonably
accurate values the parameters $\{\alpha_i,\,b_i\}$.
In addition, assuming we excite the system with an initial condition
equal to an eigenfunction corresponding to an actual eigenvalue $\lambda$,
we will be able to reconstruct the value of $c^2$ from 
the composite $\Lambda=c^2\lambda$. %bk 2021-11-28

\smallskip
In this situation, to reconstruct the parameters $\{\alpha_i,\,b_i,\, c^2\}$
we work directly with the representation $\hat{h}(s)$ or more exactly with its
logarithm which is a more convenient form for computing the Jacobian
\begin{equation}\label{eqn:log_F(s)}
\log\bigl(\hat{h}(s)\bigr) =
\log\Bigl( s + \sum_1^n b_i s^{\alpha_i-1}\Bigr) -
\log\Bigl( s^2 + \sum_1^n b_i s^{\alpha_i} + c^2\lambda\Bigr).
\end{equation}
Of course, to obtain $\hat{h}(s)$ we must approximate
the integral $\hat{h}(s) = \int_0^\infty e^{-st}h(t)\,dt$ and this indeed does
require full time measurements.
However, we do not actually need the values of the analytic function
$\hat{h}(s)$ for all $s$ for the Newton scheme and so if we only use $s$
values with $s>s_0$ then this allows us to ignore very large time measurements
and indeed we truncated these to $t\leq 40$.

The reconstruction of the components in the function $\hat{h}(s)$ is shown
graphically in Figure~\ref{fig:b_and_alpha} and in tabular form in 
Table~\ref{Table:Full_time_trace}.
The exact values were $\alpha=\{0.25, \, 0.5, \, 0.75\}$, $b=\{0.2,\,0.25,\,0.1\}$, $\Lambda=4$. %bk 2021-11-28
Note that we have included the value of the residual here and keeping track of
this value allows the iteration scheme to terminate when saturation has
occurred.
%%%%%%%%%%%%%%%%%%%%%%%%%%%%%%%%%%%%%%%%%%%%%%%%
\begin{table}[H]
\centering
\small
\footnotesize
\begin{tabular}{|c|c|c|c|c|c|c|c|c|}
\hline
iter & $\alpha_1$ & $\alpha_2$ & $\alpha_3$ & $b_1$ & $b_2$ & $b_3$ & $\Lambda$ & residual\\
\hline
	0 & 0.3000 & 0.6000 & 0.8000 & 0.3000 & 0.3750 & 0.1500 & 3.500 & {} \\
1 & 0.2448 & 0.5590 & 0.7946 & 0.1907 & 0.2612 & 0.1098 & 4.003 & 0.035032 \\
2 & 0.2506 & 0.5254 & 0.7695 & 0.2057 & 0.2747 & 0.1190 & 4.000 & 0.004371 \\
3 & 0.2492 & 0.5253 & 0.7700 & 0.2060 & 0.2784 & 0.1190 & 4.000 & 0.000075 \\
4 & 0.2491 & 0.5254 & 0.7700 & 0.2060 & 0.2748 & 0.1192 & 4.000 & 0.000000 \\
\hline
\end{tabular}
\small
\footnotesize
\caption{\bf Recovery of  damping terms and unknown $\Lambda$ from full time values.}
 \label{Table:Full_time_trace}
\end{table}
%%%%%%%%%%%%%%%%%%%
\begin{figure}[h]
\small
\hbox to \hsize{\hfill\copy\figurefour\hfill\copy\figurethree\hfill}
\caption{\small {\bf reconstructed values of $\{b_i\}$ and $\{\alpha_i\}$.
%The zero index is the initial approximation
The symbols in red are the exact values.}}
\label{fig:b_and_alpha}
\end{figure}
%%%%%%%%%%%%%%%%%%%%%%%%%%%%%%%%%%%%%%%%%%%%%%%%



\subsection{Large time measurements.}

Here we are trying to simulate the asymptotic values of the constituent
powers of $t$ occurring in the data function $h(t)$.
This was achieved by using a sample of points between $t_{\rm min}$
and $t_{\rm max}$.

\revision{\subsubsection*{A Newton based approach}}
We apply Newton's method to recover the constants $\{p_i,c_{k,\ell}\}$ in 
\[
\begin{aligned}
h(t) &= 
c_{1,1} t^{-p_1}
+c_{1,2} t^{-p_2}
+c_{1,3} t^{-p_3}
+c_{2,1} t^{-2p_1}
+c_{2,2} t^{-2p_2}
+c_{2,3} t^{-2p_3}
+c_{2,4} t^{-p_1-p_2}
\\&\quad+c_{2,5} t^{-p_2-p_3}
+c_{2,6} t^{-p_3-p_1}
+c_{3,1} t^{-3p_1}
+c_{3,2} t^{-3p_2}
+c_{3,3} t^{-2p_1-p_2}
+c_{3,4} t^{-p_1-2p_2}
\\&\quad+c_{3,5} t^{-2p_1-p_3}
+c_{3,6} t^{-p_1-2p_3}
+c_{3,7} t^{-2p_2-p_3}
+c_{3,8} t^{-p_2-2p_3}
+O(t^{-3p_3})
\end{aligned}
\]
which results from the expansion \eqref{eqn:hsing} in case of three terms,
and then recover $\alpha_i=p_i$ and $b_i=c_{1,i}\Gamma(1-p_i)$  
In the above we have neglected the term $t^{-3\alpha_3}$ as this would not arise
from the Tauberian theorem in the case that the largest power
$\alpha_3 \geq \frac{1}{3}$.
We may also have to exclude other terms such as  
$t^{-2\alpha_1-\alpha_2}$
if
$2\alpha_1 + \alpha_2 \geq 1$.
In practice, during the iteration process, terms should be included or excluded
in the code depending on this criterion: we did so by checking
if the argument passed to the $\Gamma$ function would be negative in which
case the term is deleted from use for that iteration step.

Values for the table shown below were
$t_{\rm min}=5\times 10^4$ and $t_{\rm max}=2\times 10^5$.
As a general rule, terms with small $\alpha$ values can be resolved
with a smaller value of $t_{\rm max}$, but for, say the recovery of
a pair of damping terms with $\alpha_i>0.8$, a larger value 
of $t_{\rm max}$ with commensurate accuracy will be needed.

The case of three damping terms 
$\{b_i\partial^{\alpha_i}_t\}_{i=1}^3$ 
with
$\alpha = \{\frac{1}{4},\,\frac{1}{3},\,\frac{2}{3}\}$ and $b_i=0.1$
is shown in Table~\ref{Table:large_tt} below.
The initial starting values were taken to be between ten and thirty percent
of the actual.
These are shown in the line corresponding to iteration $0$.

\begin{table}[ht]
\centering
\footnotesize
\begin{tabular}{|c|c|c|c|c|c|c|c|}
\hline
iter & $\alpha_1$ & $\alpha_2$ & $\alpha_3$ & $b_1$ & $b_2$ & $b_3$ & residual\\
\hline
0 & 0.2000 & 0.3000 & 0.6000 & 0.130 & 0.080 & 0.110 &\\			%bk 2021-11-28
1 & 0.2409 & 0.3572 & 0.6168 & 0.112 & 0.089 & 0.107 & 0.548761\\
2 & 0.2477 & 0.3307 & 0.6531 & 0.102 & 0.091 & 0.104 & 0.104843\\
3 & 0.2499 & 0.3346 & 0.6650 & 0.100 & 0.099 & 0.092 & 0.005124\\
4 & 0.2500 & 0.3332 & 0.6668 & 0.100 & 0.099 & 0.090 & 0.003625\\
5 & 0.2500 & 0.3331 & 0.6658 & 0.100 & 0.100 & 0.090 & 0.001134\\
6 & 0.2500 & 0.3332 & 0.6658 & 0.100 & 0.100 & 0.092 & 0.000578\\
7 & 0.2500 & 0.3332 & 0.6660 & 0.100 & 0.100 & 0.093 & 0.000357\\
8 & 0.2500 & 0.3333 & 0.6665 & 0.100 & 0.100 & 0.094 & 0.000240\\
\hline
\end{tabular}
\small
\caption{Large time values with 3 damping terms}
 \label{Table:large_tt}
\end{table}

There are features here that are typical of such reconstructions.
The reconstruction method resolves the lowest fractional power $\alpha_1$
and its coefficient $b_1$ quickly as this term is the most persistent one
for large times:
essential numerical convergence for $\{\alpha_1,\,b_1\}$ is obtained
by the third iteration.
The next lowest power and coefficient lags behind; here $\{\alpha_2,\,b_2\}$ is already at the stated accuracy by the fifth iteration.
In each case the power is resolved faster and more accurately than its
coefficient.
The third term also illustrates this;  the power is essentially resolved by
iteration 8, but in fact its coefficient $b_3$ is not resolved to the third
decimal place until iteration number 30.
This is seen quite clearly in the singular values of the Jacobian:
the largest singular values correspond to the lowest $\alpha$-values
and the smallest to the coefficients of the largest $\alpha$-powers.

As might be expected, resolving terms whose powers are quite close is
in general more difficult.
This is relatively insignificant for low $\alpha$ values.
For example, with $\alpha_1=0.2$ and $0.22<\alpha_2<0.25$ say, correct
resolution will be obtained although the coefficients will take longer to
resolve than indicated in Table~\ref{Table:large_tt}.
On the other hand if, say $\alpha_1=\frac{1}{4}$ and $\alpha_2=0.85$,
$\alpha_3=0.9$, then with the indicated range of time values used the code
will fail to recover this last pair.
If this is sensed and now only a single second power is requested
this will give a good estimate for $\alpha_2$ but its coefficient will
be overestimated.
Also, as indicated previously, $\alpha$ values close to one require
an extended time measurement range to stay closer to the asymptotic regime of $u(t)$.

\revision{
Note that the starting values were at least ten percent away from the exact values and one cannot expect the convergence radius of Newton's methods to be much larger for a problem exhibiting as high nonlinearity as the one at hand.
Still, let us point to the fact that in the single term case we do not need any starting guesses but obtain very good results directly from the asymptotic formulas \eqref{eqn:alpha-larget}, \eqref{eqn:b1-larget}. This might give the idea of using the asymptotics to construct starting guesses in the multi term case and then apply Newton. However, it is unclear how the asymptotics could yield starting values for terms other than the first one. A (theoretical, as it turns out) possibility for exploiting asympotics in the multi term case is described below. 
}
\Margin{Ref 2 (iv)}

\revision{\subsubsection*{Sucessive sequential use of asymptotic formulas}}
A few words are in order about an approach that from the above discussion
might seem a good or even a better alternative.

Since each damping term 
$b_i \partial^{\alpha_i}_t$ 
contributes a time trace term
with large time behaviour $c_i t^{-\alpha_i}$, it is feasible to take
$T$ sufficiently large so that
$c_i t^{-\alpha_i} <\!\!\!< c_1 t^{-\alpha_1}$ for $t>T$, that is, all but the
smallest damping power is negligible, and this can then be recovered.
In successive steps then we subtract this from the data $h$ to
get $h_1(t) = h(t) - c_1t^{-\alpha_1}$ and now seek to recover the next
lowest $\alpha$ power from the large time values of $h_1(t)$  in a range
$(\delta T,T)$ for $0<\delta<1$.
Then these steps can be repeated until there is no discernible signal remaining
in the sample interval $t_{\rm min},t_{\rm max}$.

This indeed works well under the right circumstances
for recovering two $\alpha$ values but the
coefficients $\{b_i\}$ are less well resolved.
It also requires a delicate splitting of the time interval and gives a
much poorer resolution of the two terms in the case where, say
$\alpha_1=0.2$ and $\alpha_2=0.25$  than that recovered from the Newton scheme.
For the recovery of three damping terms this was in general quite
ineffective.

Every time an $\alpha_i$ has been recovered, the remaining signal is significantly smaller than the previous, leading to an equally significant drop in effective accuracy.
Also, even if we just make a small error in the coefficient $b_i$, the relative error that is caused by this becomes completely dominant for large times. 

In short, this is an elegant and seemingly constructive approach to showing uniqueness for a finite number of damping terms. However, it has limited value from a numerical recovery perspective when used under a wide range of parameter values.

\subsection{Small time measurements.}

In this case we are simulating measurements taken over a very limited initial
time range: in fact we take the measurement interval to be $t\in [0, 0.1)$.
The line of attack is to use the known form of $\hat{h}(s)$ for large values of $s$
and convert the powers of $s$ appearing into powers of $t$ for small times
using the Tauberian theorem.
In case of two damping terms
this gives
\begin{equation*}%\label{eqn:small_t-form}
\begin{aligned}
%f_{\tiny\rm{small}}(t) &= 
%- \frac{\lambda}{\Gamma(4)})t^3 + \frac{\lambda^2}{\Gamma(6)}t^5
%-\frac{\lambda^3}{\Gamma(8)}t^7 \\
%&\quad - c_{1,1}t^{3-p_1} - c_{2,1}t^{3-p_2}) - c_{3,1}t^{3-p_3}  
% + c_{1,4}t^{5-p_1} + c_{2,4}t^{5-p_2} + c_{3,4}t^{5-p_3} \\
%&\quad + c_{1,6}t^{5-2p_1} + c_{2,6}t^{5-2p_2} + c_{3,6}t^{5-2p_3} 
% + c_{4,6}t^{5-p_1-p_2} + c_{5,6}t^{5-p_1-p_3}+ c_{6,6}t^{5-p_2-p_3}\\
-c^2\lambda h_{\tiny\rm{small}}(t)-h_{\tiny\rm{small}}''(t) 
&
=c_{1,1} t^{1-\alpha_1}
+c_{1,2} t^{1-\alpha_2}
+c_{2,1} t^{3-\alpha_1}
+c_{2,2} t^{3-\alpha_2}
\\&\quad
+c_{2,3} t^{3-2\alpha_1}
+c_{2,4} t^{3-\alpha_1-\alpha_2}
+c_{2,5} t^{3-2\alpha_2}
\end{aligned}
\end{equation*}
where each term $c_{k,\ell}$ is computed in terms of
$\{\alpha_i\}$, $\{b_i\}$ and $\lambda$, %bk 2021-11-28
cf. \eqref{eqn:smalltime}. 
							         
The values of $\{\alpha_i,c_{k,\ell}\}$ 
were then computed from the data by a Newton
scheme then finally converted back to the derived values of
$\{\alpha_i\}$ and $\{b_i\}$.

The exact values chosen were
$\alpha = \{0.25,\ 0.2\}$, $b = \{0.1,\ 0.1\}$ and the
initial starting guesses were $\alpha = \{0.3,\ 0.16\}$, $b = \{0.08,\ 0.12\}$.
We show the progression of the iteration in 
Table~\ref{tab:smalltime}. 

\begin{table}[H]
\centering
\small
\footnotesize
\begin{tabular}{|c|c|c|c|c|c|}
\hline
iter & $\alpha_1$ & $\alpha_2$ & $b_1$ & $b_2$ & residual\\
\hline
0 & 0.3000 & 0.1600 & 0.0800 & 0.1200 &  \\
1 & 0.2848 & 0.1673 & 0.0821 & 0.1160 & 0.034308 \\
2 & 0.2632 & 0.1944 & 0.0863 & 0.1142 & 0.024472 \\
3 & 0.2554 & 0.2016 & 0.0862 & 0.1140 & 0.002015 \\
4 & 0.2537 & 0.2033 & 0.0861 & 0.1139 & 0.000444 \\
5 & 0.2534 & 0.2036 & 0.0861 & 0.1139 & 0.000059 \\
6 & 0.2534 & 0.2036 & 0.0861 & 0.1139 & 0.000016 \\
7 & 0.2534 & 0.2036 & 0.0861 & 0.1139 & 0.000010 \\
\hline
\end{tabular}
\small
	\caption{{\small\bf Small time values with 2 damping terms.}
\label{tab:smalltime}}
\end{table}

While theory predicts reconstructibility of an arbitrary number of terms in both cases,
there is a clear difference in ability to reconstruct terms between the small time and the large time asymptotics. 
First of all, the method we described effectively only recovers two terms with small time measurements, as compared to three in the large time. 
The $b_i$ coefficients, which are always harder to obtain than the $\alpha_i$ exponents, are much worse than in the small time than in the large time regime.
This is partly explained by the higher degree of ill-posedness due to the necessity of differentiating the data twice. 




%%\input{main_derivation}
%%% \vspace{-0.5em}
\section{Conclusion}
% \vspace{-0.5em}
Recent advances in multimodal single-cell technology have enabled the simultaneous profiling of the transcriptome alongside other cellular modalities, leading to an increase in the availability of multimodal single-cell data. In this paper, we present \method{}, a multimodal transformer model for single-cell surface protein abundance from gene expression measurements. We combined the data with prior biological interaction knowledge from the STRING database into a richly connected heterogeneous graph and leveraged the transformer architectures to learn an accurate mapping between gene expression and surface protein abundance. Remarkably, \method{} achieves superior and more stable performance than other baselines on both 2021 and 2022 NeurIPS single-cell datasets.

\noindent\textbf{Future Work.}
% Our work is an extension of the model we implemented in the NeurIPS 2022 competition. 
Our framework of multimodal transformers with the cross-modality heterogeneous graph goes far beyond the specific downstream task of modality prediction, and there are lots of potentials to be further explored. Our graph contains three types of nodes. While the cell embeddings are used for predictions, the remaining protein embeddings and gene embeddings may be further interpreted for other tasks. The similarities between proteins may show data-specific protein-protein relationships, while the attention matrix of the gene transformer may help to identify marker genes of each cell type. Additionally, we may achieve gene interaction prediction using the attention mechanism.
% under adequate regulations. 
% We expect \method{} to be capable of much more than just modality prediction. Note that currently, we fuse information from different transformers with message-passing GNNs. 
To extend more on transformers, a potential next step is implementing cross-attention cross-modalities. Ideally, all three types of nodes, namely genes, proteins, and cells, would be jointly modeled using a large transformer that includes specific regulations for each modality. 

% insight of protein and gene embedding (diff task)

% all in one transformer

% \noindent\textbf{Limitations and future work}
% Despite the noticeable performance improvement by utilizing transformers with the cross-modality heterogeneous graph, there are still bottlenecks in the current settings. To begin with, we noticed that the performance variations of all methods are consistently higher in the ``CITE'' dataset compared to the ``GEX2ADT'' dataset. We hypothesized that the increased variability in ``CITE'' was due to both less number of training samples (43k vs. 66k cells) and a significantly more number of testing samples used (28k vs. 1k cells). One straightforward solution to alleviate the high variation issue is to include more training samples, which is not always possible given the training data availability. Nevertheless, publicly available single-cell datasets have been accumulated over the past decades and are still being collected on an ever-increasing scale. Taking advantage of these large-scale atlases is the key to a more stable and well-performing model, as some of the intra-cell variations could be common across different datasets. For example, reference-based methods are commonly used to identify the cell identity of a single cell, or cell-type compositions of a mixture of cells. (other examples for pretrained, e.g., scbert)


%\noindent\textbf{Future work.}
% Our work is an extension of the model we implemented in the NeurIPS 2022 competition. Now our framework of multimodal transformers with the cross-modality heterogeneous graph goes far beyond the specific downstream task of modality prediction, and there are lots of potentials to be further explored. Our graph contains three types of nodes. while the cell embeddings are used for predictions, the remaining protein embeddings and gene embeddings may be further interpreted for other tasks. The similarities between proteins may show data-specific protein-protein relationships, while the attention matrix of the gene transformer may help to identify marker genes of each cell type. Additionally, we may achieve gene interaction prediction using the attention mechanism under adequate regulations. We expect \method{} to be capable of much more than just modality prediction. Note that currently, we fuse information from different transformers with message-passing GNNs. To extend more on transformers, a potential next step is implementing cross-attention cross-modalities. Ideally, all three types of nodes, namely genes, proteins, and cells, would be jointly modeled using a large transformer that includes specific regulations for each modality. The self-attention within each modality would reconstruct the prior interaction network, while the cross-attention between modalities would be supervised by the data observations. Then, The attention matrix will provide insights into all the internal interactions and cross-relationships. With the linearized transformer, this idea would be both practical and versatile.

% \begin{acks}
% This research is supported by the National Science Foundation (NSF) and Johnson \& Johnson.
% \end{acks}
%\input{numerical_plot}

%% !TEX root = ../arxiv.tex

Unsupervised domain adaptation (UDA) is a variant of semi-supervised learning \cite{blum1998combining}, where the available unlabelled data comes from a different distribution than the annotated dataset \cite{Ben-DavidBCP06}.
A case in point is to exploit synthetic data, where annotation is more accessible compared to the costly labelling of real-world images \cite{RichterVRK16,RosSMVL16}.
Along with some success in addressing UDA for semantic segmentation \cite{TsaiHSS0C18,VuJBCP19,0001S20,ZouYKW18}, the developed methods are growing increasingly sophisticated and often combine style transfer networks, adversarial training or network ensembles \cite{KimB20a,LiYV19,TsaiSSC19,Yang_2020_ECCV}.
This increase in model complexity impedes reproducibility, potentially slowing further progress.

In this work, we propose a UDA framework reaching state-of-the-art segmentation accuracy (measured by the Intersection-over-Union, IoU) without incurring substantial training efforts.
Toward this goal, we adopt a simple semi-supervised approach, \emph{self-training} \cite{ChenWB11,lee2013pseudo,ZouYKW18}, used in recent works only in conjunction with adversarial training or network ensembles \cite{ChoiKK19,KimB20a,Mei_2020_ECCV,Wang_2020_ECCV,0001S20,Zheng_2020_IJCV,ZhengY20}.
By contrast, we use self-training \emph{standalone}.
Compared to previous self-training methods \cite{ChenLCCCZAS20,Li_2020_ECCV,subhani2020learning,ZouYKW18,ZouYLKW19}, our approach also sidesteps the inconvenience of multiple training rounds, as they often require expert intervention between consecutive rounds.
We train our model using co-evolving pseudo labels end-to-end without such need.

\begin{figure}[t]%
    \centering
    \def\svgwidth{\linewidth}
    \input{figures/preview/bars.pdf_tex}
    \caption{\textbf{Results preview.} Unlike much recent work that combines multiple training paradigms, such as adversarial training and style transfer, our approach retains the modest single-round training complexity of self-training, yet improves the state of the art for adapting semantic segmentation by a significant margin.}
    \label{fig:preview}
\end{figure}

Our method leverages the ubiquitous \emph{data augmentation} techniques from fully supervised learning \cite{deeplabv3plus2018,ZhaoSQWJ17}: photometric jitter, flipping and multi-scale cropping.
We enforce \emph{consistency} of the semantic maps produced by the model across these image perturbations.
The following assumption formalises the key premise:

\myparagraph{Assumption 1.}
Let $f: \mathcal{I} \rightarrow \mathcal{M}$ represent a pixelwise mapping from images $\mathcal{I}$ to semantic output $\mathcal{M}$.
Denote $\rho_{\bm{\epsilon}}: \mathcal{I} \rightarrow \mathcal{I}$ a photometric image transform and, similarly, $\tau_{\bm{\epsilon}'}: \mathcal{I} \rightarrow \mathcal{I}$ a spatial similarity transformation, where $\bm{\epsilon},\bm{\epsilon}'\sim p(\cdot)$ are control variables following some pre-defined density (\eg, $p \equiv \mathcal{N}(0, 1)$).
Then, for any image $I \in \mathcal{I}$, $f$ is \emph{invariant} under $\rho_{\bm{\epsilon}}$ and \emph{equivariant} under $\tau_{\bm{\epsilon}'}$, \ie~$f(\rho_{\bm{\epsilon}}(I)) = f(I)$ and $f(\tau_{\bm{\epsilon}'}(I)) = \tau_{\bm{\epsilon}'}(f(I))$.

\smallskip
\noindent Next, we introduce a training framework using a \emph{momentum network} -- a slowly advancing copy of the original model.
The momentum network provides stable, yet recent targets for model updates, as opposed to the fixed supervision in model distillation \cite{Chen0G18,Zheng_2020_IJCV,ZhengY20}.
We also re-visit the problem of long-tail recognition in the context of generating pseudo labels for self-supervision.
In particular, we maintain an \emph{exponentially moving class prior} used to discount the confidence thresholds for those classes with few samples and increase their relative contribution to the training loss.
Our framework is simple to train, adds moderate computational overhead compared to a fully supervised setup, yet sets a new state of the art on established benchmarks (\cf \cref{fig:preview}).

%\input{SysModel}


%\section{{System Model}}
%\subsection{{}System Model}
%\section{Problem Formulation}
\label{sec:system_model_feedback}

%We first define the problem in Section~\ref{subsec:problem_definition}.  In Section~\ref{subsec:metrics}, we then discuss metrics that are of interest in the evaluation of broadcast codes and show how a solution to the problem in Section~\ref{subsec:problem_definition} can be used to address these metrics.

%\section{Problem Definition}\label{sec:problemdef}

\begin{figure} [t]
	\centering
	\scalebox{1.05}[1.15]
	{
		\resizebox{\linewidth}{!}
		{
			\includegraphics[scale=1.0]{visio_pics/rdf_graph_new.pdf}
		}
	}
	\vspace{-0.1in}
	\caption{A Sample of DBpedia RDF Graph.}
	\label{fig:rdfgraph}
	\vspace{-0.2in}
\end{figure}

In this section, we define our problem and review the terminologies used throughout this paper. As a de facto standard model of knowledge base, RDF represents the assertions by $\langle$subject, predicate, object$\rangle$ triples. An RDF dataset can be represented as a graph naturally, where subjects and objects are vertices and predicates denote directed edges between them. A running example of RDF graph is illustrated in Figure \ref{fig:rdfgraph}. Formally, we have the definition about RDF graph as follows.


%\vspace{-0.1in}
\begin{definition} \label{def:rdfgraph}
	\textbf{ (RDF Graph) }
	An \textit{RDF graph} is denoted as $G(V, E)$, where
	$V$ is the set of entity and class vertices corresponding to subjects and objects of RDF triples,
	and $E$ is the set of directed relation edges with their labels corresponding to predicates of RDF triples.

\end{definition}

Note that RDF triple's object may be literal value, for example, $\langle$res:Alan\_Turing, dbo:deathDate, ``1954-06-07'' $\textasciicircum{}\textasciicircum{}$xms:date$\rangle$. We treat all literal values as entity vertices in RDF graph, and literal types (e.g. xms:date as for ``1954-06-07'') as class vertices. 

SPARQL is the standard structural query language of RDF, which can also be represented as a \emph{query graph} defined as follows.
%Answering SPARQLs is equivalent to evaluate the query by subgraph matching using homomorphism \cite{zou2014gstore}. 

\begin{definition} \textbf{ (Query Graph) }
	A \textit{query graph} is denoted as $Q(V_Q, E_Q)$, where
	$V_Q$ consists of entity vertices, class vertices, and vertex \emph{variables},
	and $E_Q$ consists of relation edges as well as edge \emph{variables}.
\end{definition}

%Although SPARQL provides a systematic approach to accessing RDF graph, the complexity of the SPARQL syntax makes it hard to use. 
In this paper, we study the keyword search problem over RDF graph. Given a keyword token sequence $RQ = \{k_1, k_2, ..., k_m\}$, our problem is to interpret $RQ$ as a \emph{query graph} $Q$. 

%\subsection{Evaluation metrics}
\label{sup:eval}
Performance was evaluated using precision@$k$ and nDCG@$k$ metrics. Performance was also evaluated using propensity scored metrics, namely propensity scored precision@$k$ and nDCG@$k$ (with $k$ = 1, 3 and 5) for extreme classification. The propensity scoring model and values available on The Extreme Classification Repository~\citep{XMLRepo} were used for the publicly available datasets. For the proprietary datasets, the method outlined in \cite{Jain16} was used. For a predicted score vector $\hat{\mathbf{y}} \in R^L$ and ground truth label vector $\mathbf{y} \in \{0, 1\}^L$, the metrics are defined below. In the following, $p_l$ is propensity score of the label $l$ as proposed in~\citep{Jain16}.
\begin{flalign}
P@k&= \frac{1}{k} {\sum_{l \in rank_k(\hat{\mathbf{y}})}} y_l &
PSP@k&= \frac{1}{k} {\sum_{l \in rank_k(\hat{\mathbf{y}})}} \frac{y_l}{p_l} & \nonumber\\
DCG@k&= \frac{1}{k} {\sum_{l \in rank_k(\hat{\mathbf{y}})}} \frac{y_l}{\log(l+1)} \nonumber
& PSDCG@k&= \frac{1}{k} {\sum_{l \in rank_k(\hat{\mathbf{y}})}} \frac{y_l}{p_l \log(l+1)} & \nonumber\\
nDCG@k&= \frac{DCG@k}{\sum_{l=1}^{\min(k, ||\mathbf{y}||_0)} \frac{1}{\log(l +1) }} \nonumber
 &
PSnDCG@k&= \frac{PSDCG@k}{\sum_{l=1}^{k} \frac{1}{\log l +1 }} &\nonumber,
\end{flalign}


%%%\begin{figure*}
%%\begin{figure}
%%	\centering
%%%	%\documentclass{standalone}
%
%\usepackage{tikz}
%%\usepackage[pdf]{pstricks}
%\usepackage{pstricks}
%\usepackage{pst-sigsys}
%
%\begin{document}

\psset{xunit=0.5cm,yunit=0.5cm}

\begin{pspicture}[showgrid=false](0,-15)(15, 0)
%\begin{pspicture}[showgrid=true](0,-15)(15, 0)
%	\psset{xunit=0.5cm,yunit=0.5cm}
	\scriptsize
	\pssignal(0,-7) {S}{{\normalsize$S^{k}$}}
	
	%%%%%%%%%%%%%%%%%%  Encoder %%%%%%%%%%%%%%%%%%
	\psfblock[framesize=2 7.5](3.5,-7){enc}{{\normalsize Encoder}}
	\newcount\cnt
	
	%%%%%%%%%%%%%%%%%%  Decoder %%%%%%%%%%%%%%%%%%
	\psfblock[framesize=1.5 3](15,-2.5){dec1}{{\small Decoder 1}}
	\psfblock[framesize=1.5 4.5](15,-10.25){dec2}{{\small Decoder 2}}

	%%%%%%%%%%%%%%%%%% Multipliers %%%%%%%%%%%%%%%%%%
	\cnt=0
	\psforeach{\ry}{0,-2,-4,-6,-8,-10,-12,-14}{\advance\cnt by 1\relax
		\pscircleop[oplength=.08](10,\ry){op\the\cnt}
		\pnode(5.5,\ry){enc_dot\the\cnt}
		\pnode(13.5,\ry){dec_dot\the\cnt}
	}
	
	\cnt=0
	\psforeach{\ry}{-1,-3,-5,-7,-9,-11,-13,-15}{\advance\cnt by 1\relax
		\pnode(10,\ry){N_arrow\the\cnt}
		\ncline[style=Arrow]{N_arrow\the\cnt}{op\the\cnt}
	}
	
	\pssignal(6.5,0.5){X1}{$X(1)$}
	\pssignal(6.5,-1.5){X2}{$X(2)$}
	\psldots[angle=90,ldotssep=0.1](6.5,-2.7)
	\pssignal(6.5,-3.5){Xn1}{$X(n_{1})$}
	\psldots[angle=90,ldotssep=0.1](8.5,-9)
	\pssignal(7,-11.5){Xn1p1}{$X(n_{1} + 1)$}	
	\pssignal(6.5,-13.5){Xn2}{$X(n_{2})$}
	\psldots[angle=90,ldotssep=0.1](8.5,-13)	
	
	\pssignal(11,-1){Z1_1}{$Z_{1}(1)$}
	\pssignal(11,-3){Z1_2}{$Z_{1}(2)$}
	\pssignal(11,-5){Z1_n1}{$Z_{1}(n_{1})$}
	\pssignal(11,-7){Z2_1}{$Z_{2}(1)$}
	\pssignal(11,-9){Z2_2}{$Z_{2}(2)$}
	\pssignal(11,-11){Z2_n1}{$Z_{2}(n_{1})$}
	\pssignal(11.5,-13){Z2_n1p1}{$Z_{2}(n_{1} + 1)$}
	\pssignal(11,-15){Z2_n2}{$Z_{2}(n_{2})$}
	
	\pssignal(12.5,0.5){Y1_1}{$Y_{1}(1)$}
	\pssignal(12.5,-1.5){Y1_2}{$Y_{1}(2)$}
	\pssignal(12.5,-3.5){Y1_n1}{$Y_{1}(n_{1})$}
	\pssignal(12.5,-5.5){Y2_1}{$Y_{2}(1)$}
	\pssignal(12.5,-7.5){Y2_2}{$Y_{2}(2)$}
	\pssignal(12.5,-9.5){Y2_n1}{$Y_{2}(n_{1})$}
	\pssignal(12,-11.5){Y2_n1p1}{$Y_{2}(n_{1} + 1)$}
	\pssignal(12.5,-13.5){Y2_n2}{$Y_{2}(n_{2})$}
	
	
	%%%%%%%%%%%%%%%%%%  Looped Input Dots  %%%%%%%%%%%%%%%%%%
	% Draw decoder 2's input dots
	\cnt=0
	\psforeach{\ry}{0,-1.75,-2.25, -3.75, -4.25, -6}{\advance\cnt by 1\relax
		\dotnode[dotsize=.08](9,\ry){dot_x1_\the\cnt}
	}	
	\cnt=0
	\psforeach{\ry}{-2,-3.75,-4.25,-8}{\advance\cnt by 1\relax
		\dotnode[dotsize=.08](8,\ry){dot_x2_\the\cnt}
	}
	\cnt=0
	\psforeach{\ry}{-4,-10}{\advance\cnt by 1\relax
		\dotnode[dotsize=.08](7,\ry){dot_xn1_\the\cnt}
	}
	
	\ncline[style=Arrow]{S}{enc}
	
	% X1
	\ncarc[arcangle=-50]{dot_x1_2}{dot_x1_3}
	\ncarc[arcangle=-50]{dot_x1_4}{dot_x1_5}
	\ncline{dot_x1_1}{dot_x1_2}
	\ncline{dot_x1_3}{dot_x1_4}
	\ncline{dot_x1_5}{dot_x1_6}

	% X2
	\ncarc[arcangle=-50]{dot_x2_2}{dot_x2_3}
	\ncarc[arcangle=-50]{dot_x2_4}{dot_x2_5}
	\ncline{dot_x2_1}{dot_x2_2}	
	\ncline{dot_x2_3}{dot_x2_4}	
	\ncline{dot_x2_5}{dot_x2_6}	
		
	\ncline{dot_xn1_1}{dot_xn1_2}
	
	\nclist[style=Arrow]{ncline}[naput]{enc_dot1,op1,dec_dot1}	
	\nclist[style=Arrow]{ncline}[naput]{enc_dot2,op2,dec_dot2}
	\nclist[style=Arrow]{ncline}[naput]{enc_dot3,op3,dec_dot3}
	\nclist[style=Arrow]{ncline}[naput]{dot_x1_6,op4,dec_dot4}	
	\nclist[style=Arrow]{ncline}[naput]{dot_x2_4,op5,dec_dot5}	
	\nclist[style=Arrow]{ncline}[naput]{dot_xn1_2,op6,dec_dot6}
	\nclist[style=Arrow]{ncline}[naput]{enc_dot7,op7,dec_dot7}
	\nclist[style=Arrow]{ncline}[naput]{enc_dot8,op8,dec_dot8}
	

	
\end{pspicture}

%\end{document}
%%%	\includegraphics[scale=0.6]{3/out/fig/schematica}
%%%	\includegraphics[scale=0.75]{out/fig/schematic}
%%%	\includegraphics[scale=0.8]{2_problem_formulation/fig/schematicE}
%%	\includegraphics{2_problem_formulation/fig/schematicE}
%%%	\includegraphics[width=0.6\textwidth]{outer_bound.png}
%%	\caption{Broadcasting an equiprobable binary source over an erasure broadcast channel.}
%%	\label{fig:schematic}
%%%\end{figure*}
%%\end{figure}

%We begin by presenting our joint source-channel coding formulation which is illustrated in Fig.~\ref{fig:schematic}.

%\begin{figure*}
%%%%%\begin{figure}
%%%%%	\centering
%%%%%%	\begin{pspicture}[showgrid=false](0,-8)(9, 0)
%\begin{pspicture}[showgrid=true](0,-8)(9, 0)
	\psset{gratioWh=2}
%	\pssignal(0,0) {S}{$S^{k}$}
	\pnode(0,0){sdot}
	
	%%%%%%%%%%%%%%%%%%  Encoders %%%%%%%%%%%%%%%%%%

	\newcount\cnt

	\psfblock(1.75,0){enc}{Encoder}
	\dotnode(3.5,0){dot1}
	\dotnode(3.5,-2.5){dot2}
	\dotnode(3.5,-5.6){dotn}	
%	\dotnode(7,0){dot2}
%	\dotnode(1,0){dot3}
%	\dotnode(1,-2.5){dot4}
%	\pscircleop[operation=plus](1,-2.5){op3} 
	
%	\pssignal(3,-2.5){W}{$W^{k}$}
%	\pssignal(1,-3.5) {U}{$U^{k}$}
	
	%%%%%%%%%%%%%%%%%% Multipliers %%%%%%%%%%%%%%%%%%
	\cnt=0
	\psforeach{\ry}{0,-2.5, -5.6}{\advance\cnt by 1\relax
		\pscircleop[operation=times](4.5,\ry){op\the\cnt}
	}
	
	%%%%%%%%%%%%%%%%%%  Decoders %%%%%%%%%%%%%%%%%%
	\psfblock(6.5,0){dec1}{Decoder $1$}
	\psfblock(6.5,-2.5){dec2}{Decoder $2$}
	\psfblock(6.5,-5.6){decn}{Decoder $n$}		
	%%%%%%%%%%%%%%%%%%  S hat %%%%%%%%%%%%%%%%%%
%	\pssignal(14,0){b1}{$\hat{S}_{1}^{m}$}
%	\pssignal(14,-2.5){b2}{$\hat{S}_{2}^{m}$}
	\pnode(8.25,0){s1hatdot}
	\pnode(8.25,-2.5){s2hatdot}
	\pnode(8.25,-5.6){snhatdot}

	%%%%%%%%%%%%%%%%%%  Noise symbols %%%%%%%%%%%%%%%%%%
	% Draw N1 noise symbols
	\pssignal(4.5,-1){N1_1}{$N_{1}^{W}$}
	\pssignal(4.5,-3.5){N2_1}{$N_{2}^{W}$}
	\pssignal(4.5,-6.6){Nn_1}{$N_{n}^{W}$}
		
%	\psldots[angle=90,ldotssep=0.1,ldotssize=.1](4.5,-4.25)
%	\psldots[angle=90,ldotssize=.05](5.5,-4.1)
	\psldots[angle=90,ldotssize=.05](6.5,-4.05)
	
	%%%%%%%%%%%%%%%%%%  Connecting blocks 	%%%%%%%%%%%%%%%%%%
	\nclist[style=Arrow]{ncline}[naput]{sdot,enc $S^{N}$,op1 $X^{W}$,dec1 $Y_{1}^{W}$,s1hatdot $\hat{S}_{1}^{N}$}
	\nclist[style=Arrow]{ncline}[naput]{op2,dec2 $Y_{2}^{W}$,s2hatdot $\hat{S}_{2}^{N}$}
	\nclist[style=Arrow]{ncline}[naput]{op3,decn $Y_{n}^{W}$,snhatdot $\hat{S}_{n}^{N}$}
	\ncangle[style=Arrow,angleA=-90,angleB=180]{dot1}{op2}
	\ncangle[style=Arrow,angleA=-90,angleB=180]{dot2}{op3}
%	\ncangle[style=Arrow,angleA=-90,angleB=180]{dotn}{op3}
	\ncline[style=Arrow]{N1_1}{op1}
	\ncline[style=Arrow]{N2_1}{op2}
	\ncline[style=Arrow]{Nn_1}{op3}
%	\ncline[style=Arrow]{W}{op3}
%	\ncline[style=Arrow]{dotn}{op3}
%	\ncline[style=Arrow]{op3}{U}
	
\end{pspicture}
%%%%%%	\includegraphics[scale=0.95]{fig/schematic}
%%%%%	\includegraphics[scale=0.75]{fig/schematic}
%%%%%%	\includegraphics[scale=0.75]{SysModel/fig/schematic_arrow}
%%%%%	\caption{Broadcasting an equiprobable binary source over an erasure broadcast channel.}
%%%%%	\label{fig:schematic}
%%%%%%\end{figure*}
%%%%%\end{figure}

%The problem is illustrated in Fig.~\ref{fig:schematic}.  
We first consider a version of the erasure-source broadcast problem involving universal feedback, after which, we consider the case of one-sided feedback in Section~\ref{sec:one_sided_feedback}. Our problem involves communicating a binary memoryless source $\{S(t)\}_{t=1,2, \ldots}$ to two users over an erasure broadcast channel with feedback.  %Let $\mathcal{S}^{N}$ be the set of all $N$-vectors with components in $\mathcal{S}$, and denote $(S(1), S(2), \ldots , S(N))$ as $S^{N}$.  For convenience, we will also denote the set $\{1,2,\dots,N\}$ as $[N]$.
%
%\begin{figure}
%	\centering
%%	%\documentclass{standalone}
%
%\usepackage{tikz}
%%\usepackage[pdf]{pstricks}
%\usepackage{pstricks}
%\usepackage{pst-sigsys}
%
%\begin{document}

\psset{xunit=0.5cm,yunit=0.5cm}

\begin{pspicture}[showgrid=false](0,-15)(15, 0)
%\begin{pspicture}[showgrid=true](0,-15)(15, 0)
%	\psset{xunit=0.5cm,yunit=0.5cm}
	\scriptsize
	\pssignal(0,-7) {S}{{\normalsize$S^{k}$}}
	
	%%%%%%%%%%%%%%%%%%  Encoder %%%%%%%%%%%%%%%%%%
	\psfblock[framesize=2 7.5](3.5,-7){enc}{{\normalsize Encoder}}
	\newcount\cnt
	
	%%%%%%%%%%%%%%%%%%  Decoder %%%%%%%%%%%%%%%%%%
	\psfblock[framesize=1.5 3](15,-2.5){dec1}{{\small Decoder 1}}
	\psfblock[framesize=1.5 4.5](15,-10.25){dec2}{{\small Decoder 2}}

	%%%%%%%%%%%%%%%%%% Multipliers %%%%%%%%%%%%%%%%%%
	\cnt=0
	\psforeach{\ry}{0,-2,-4,-6,-8,-10,-12,-14}{\advance\cnt by 1\relax
		\pscircleop[oplength=.08](10,\ry){op\the\cnt}
		\pnode(5.5,\ry){enc_dot\the\cnt}
		\pnode(13.5,\ry){dec_dot\the\cnt}
	}
	
	\cnt=0
	\psforeach{\ry}{-1,-3,-5,-7,-9,-11,-13,-15}{\advance\cnt by 1\relax
		\pnode(10,\ry){N_arrow\the\cnt}
		\ncline[style=Arrow]{N_arrow\the\cnt}{op\the\cnt}
	}
	
	\pssignal(6.5,0.5){X1}{$X(1)$}
	\pssignal(6.5,-1.5){X2}{$X(2)$}
	\psldots[angle=90,ldotssep=0.1](6.5,-2.7)
	\pssignal(6.5,-3.5){Xn1}{$X(n_{1})$}
	\psldots[angle=90,ldotssep=0.1](8.5,-9)
	\pssignal(7,-11.5){Xn1p1}{$X(n_{1} + 1)$}	
	\pssignal(6.5,-13.5){Xn2}{$X(n_{2})$}
	\psldots[angle=90,ldotssep=0.1](8.5,-13)	
	
	\pssignal(11,-1){Z1_1}{$Z_{1}(1)$}
	\pssignal(11,-3){Z1_2}{$Z_{1}(2)$}
	\pssignal(11,-5){Z1_n1}{$Z_{1}(n_{1})$}
	\pssignal(11,-7){Z2_1}{$Z_{2}(1)$}
	\pssignal(11,-9){Z2_2}{$Z_{2}(2)$}
	\pssignal(11,-11){Z2_n1}{$Z_{2}(n_{1})$}
	\pssignal(11.5,-13){Z2_n1p1}{$Z_{2}(n_{1} + 1)$}
	\pssignal(11,-15){Z2_n2}{$Z_{2}(n_{2})$}
	
	\pssignal(12.5,0.5){Y1_1}{$Y_{1}(1)$}
	\pssignal(12.5,-1.5){Y1_2}{$Y_{1}(2)$}
	\pssignal(12.5,-3.5){Y1_n1}{$Y_{1}(n_{1})$}
	\pssignal(12.5,-5.5){Y2_1}{$Y_{2}(1)$}
	\pssignal(12.5,-7.5){Y2_2}{$Y_{2}(2)$}
	\pssignal(12.5,-9.5){Y2_n1}{$Y_{2}(n_{1})$}
	\pssignal(12,-11.5){Y2_n1p1}{$Y_{2}(n_{1} + 1)$}
	\pssignal(12.5,-13.5){Y2_n2}{$Y_{2}(n_{2})$}
	
	
	%%%%%%%%%%%%%%%%%%  Looped Input Dots  %%%%%%%%%%%%%%%%%%
	% Draw decoder 2's input dots
	\cnt=0
	\psforeach{\ry}{0,-1.75,-2.25, -3.75, -4.25, -6}{\advance\cnt by 1\relax
		\dotnode[dotsize=.08](9,\ry){dot_x1_\the\cnt}
	}	
	\cnt=0
	\psforeach{\ry}{-2,-3.75,-4.25,-8}{\advance\cnt by 1\relax
		\dotnode[dotsize=.08](8,\ry){dot_x2_\the\cnt}
	}
	\cnt=0
	\psforeach{\ry}{-4,-10}{\advance\cnt by 1\relax
		\dotnode[dotsize=.08](7,\ry){dot_xn1_\the\cnt}
	}
	
	\ncline[style=Arrow]{S}{enc}
	
	% X1
	\ncarc[arcangle=-50]{dot_x1_2}{dot_x1_3}
	\ncarc[arcangle=-50]{dot_x1_4}{dot_x1_5}
	\ncline{dot_x1_1}{dot_x1_2}
	\ncline{dot_x1_3}{dot_x1_4}
	\ncline{dot_x1_5}{dot_x1_6}

	% X2
	\ncarc[arcangle=-50]{dot_x2_2}{dot_x2_3}
	\ncarc[arcangle=-50]{dot_x2_4}{dot_x2_5}
	\ncline{dot_x2_1}{dot_x2_2}	
	\ncline{dot_x2_3}{dot_x2_4}	
	\ncline{dot_x2_5}{dot_x2_6}	
		
	\ncline{dot_xn1_1}{dot_xn1_2}
	
	\nclist[style=Arrow]{ncline}[naput]{enc_dot1,op1,dec_dot1}	
	\nclist[style=Arrow]{ncline}[naput]{enc_dot2,op2,dec_dot2}
	\nclist[style=Arrow]{ncline}[naput]{enc_dot3,op3,dec_dot3}
	\nclist[style=Arrow]{ncline}[naput]{dot_x1_6,op4,dec_dot4}	
	\nclist[style=Arrow]{ncline}[naput]{dot_x2_4,op5,dec_dot5}	
	\nclist[style=Arrow]{ncline}[naput]{dot_xn1_2,op6,dec_dot6}
	\nclist[style=Arrow]{ncline}[naput]{enc_dot7,op7,dec_dot7}
	\nclist[style=Arrow]{ncline}[naput]{enc_dot8,op8,dec_dot8}
	

	
\end{pspicture}

%\end{document}
%	\includegraphics[scale=0.8]{fig/schematicE}
%%%	\includegraphics[width=0.6\textwidth]{outer_bound.png}
%	\caption{Broadcasting an equiprobable binary source over an erasure broadcast channel with individual bandwidth mismatches.}
%	\label{fig:schematicE}
%\end{figure}
%
The source produces equiprobable symbols in $\mathcal{S}=\{0,1\}$ and is communicated by an encoding function that produces the channel input sequence $X^{W} = (X(1),  \dots , X(W))$, where $X(t)$ denotes the $t^{\mathrm{th}}$ channel input taken from the alphabet $\mathcal{X} = \{0, 1\}$.  We assume that 
$X(t)$ is a function of the source as well as the channel outputs of all users prior to time $t$.  
%The encoding function produces the channel input sequence $X^{W} = (X(1),  \dots , X(W))$, where $X(t)$ denotes the $t^{\mathrm{th}}$ channel input taken from the alphabet $\mathcal{X} = \{0, 1\}$.  %Note here that our notation defines $X^{n}$ as the first $n$ channel symbols sent. %which is also taken from the alphabet $\mathcal{S}$.  %

% Original
%Let $Y_{i}(t)$ be the channel output observed by user $i$ on the $t^{\mathrm{th}}$ channel use for $i \in [n]$ and $t \in [W]$.  Our channel model is an erasure channel.  In particular, let $\epsilon_i$ denote the erasure rate of the channel corresponding to user~$i$.  Without loss of generality, we will assume that $0 < \epsilon_{1} < \epsilon_{2} < \ldots < \epsilon_{n} < 1$.  Then $Y_{i}(t)$ exactly reproduces the channel input $X(t)$ with probability $(1 - \epsilon_i)$ and otherwise indicates an erasure event, which happens with probability $\epsilon_i$.  We let $Y_{i}(t)$ take on values in the alphabet $\mathcal{Y} = \{0, 1, \star\}$ so that an erasure event is represented by `$\star$,' the erasure symbol.  We associate user~$i$ with the state sequence $(N_i(t))_{t \in [W]}$, which is defined such that $N_{i}(t) = 1$ if $Y_i(t)$ was erased and $N_i(t) = 0$ otherwise.  The channel we consider is memoryless in the sense that $N_{i}(t)$ is drawn i.i.d.\ from a $\operatorname{Bern}(1 - \epsilon_{i})$ distribution.

Let $Y_{i}(t)$ be the channel output observed by user $i$ on the $t^{\mathrm{th}}$ channel use for $i \in \{1, 2\}$. %$\mathcal{U}= \{1, 2, 3\}$ and $t \in [W] = \{1, \ldots, W\}$.  %Our channel model is a binary erasure broadcast channel. % as shown in Fig.~\ref{fig:schematic}.  
%In particular, let $\epsilon_i$ denote the erasure rate of the channel corresponding to user~$i$, where we assume that $0 < \epsilon_{1} < \epsilon_{2} < \ldots < \epsilon_{n} < 1$.  This is without loss of generality since we can address {{}all} users that experience identical erasure rates by serving the {{}one} with the most stringent distortion requirement.  
%In particular, if $\epsilon_i$ denotes the erasure rate of the channel corresponding to user~$i$, our model specifies that $Y_{i}(t)$ exactly reproduces the channel input $X(t)$ with probability $(1 - \epsilon_i)$ and otherwise indicates an erasure event, which happens with probability $\epsilon_i$.  
We let $Y_{i}(t)$ take on values in the alphabet $\mathcal{Y} = \{0, 1, \star\}$ so that an erasure event is represented by `$\star$'.   
%
For $W \in \mathbb{N}$,  let $[W]$ denote the set $\{1, 2, \ldots, W\}$. We associate user~$i$ with the state sequence $\{Z_i(t)\}_{t \in [W]}$, which represents the noise on user~$i$'s channel, where $Z_i(t) \in \mathcal{Z} \triangleq \{0,1\}$,  and  $Y_i(t)$ will be erased if $Z_{i}(t) = 1$ and $Y_{i}(t) = X(t)$ if $Z_i(t) = 0$.  The channel we consider is memoryless in the sense that $(Z_{1}(t), Z_2(t))$ is drawn i.i.d.\ from the probability mass function given by   
%For $i, j \in \{1, 2\}$, let $p_{E}(i, j) = \textrm{Pr}(Z_1(t) = i - 1, Z_2(t) = j - 1)$, and let $P_{E} \in \mathbb{R}^{2 \times 2}$ be the stochastic matrix such that $P_{E} = [p_{E}(i, j)]$

\begin{subequations}
\label{eq:pmf_z1z2}
\begin{align}
	\textrm{Pr}(Z_1 = 1, Z_2 = 1) &= \epsot \\
	\textrm{Pr}(Z_1 = 1, Z_2 = 0) &= \epsilon_1 - \epsot \\
	\textrm{Pr}(Z_1 = 0, Z_2 = 1) &= \epsilon_2 - \epsot \\
	\textrm{Pr}(Z_1 = 0, Z_2 = 0) &= 1 - \epsilon_1 - \epsilon_2 + \epsot,			
\end{align}
\end{subequations}
where $\epsot\in (0, 1)$ is the probability that an erasure simultaneously occurs on both channels and $\epsilon_i \in (0, 1)$ denotes the erasure rate of the channel corresponding to user~$i$, where we assume $\epsilon_1 < \epsilon_2$.

%\textcolor{blue}{We assume that the erasure rates are known to both the transmitter and receiver. If such channel knowledge is not possible, we can take the standard approach of interpreting a broadcast channel as an abstraction for a \emph{compound channel}~\cite{CT}.  In such a situation, the compound channel is a channel that randomly takes on one of many potential states for the duration of transmission.  We model this by considering a broadcast channel whose constituent channels correspond to these potential channel states.}
%{}{Note that in our setup, the erasure rates for each user are assumed to be known. However, our setup also models the 
%compound channel~\cite{CT}, where an erasure rate is not known and instead belong to a collection of possible states, with each 
%state corresponding to one virtual user in our system.}

%Let $Y_{i}(t)$ be the channel output observed by user $i$ on the $t^{\mathrm{th}}$ channel use for $i \in [n]$ and $t \in [W]$.  Our channel model is an erasure channel.  In particular, $Y_{i}(t)$ either exactly reproduces the channel input $X(t)$ or otherwise indicates an erasure event.  We let $Y_{i}(t)$ take on values in the alphabet $\mathcal{Y} = \{0, 1, \star\}$ so that an erasure event is represented by `$\star$,' the erasure symbol.  Let $\epsilon_i$ denote the erasure rate of the channel corresponding to user~$i$.  Without loss of generality, we assume that $0 < \epsilon_{1} < \epsilon_{2} < \ldots < \epsilon_{n} < 1$.  We associate user~$i$ with the state sequence $(N_i(t))_{t \in [W]}$, where $N_i(t) \in \{0,1\}$.  The state sequence represents the noise on user~$i$'s channel and we will have that $Y_i(t)$ is erased if $N_{i}(t) = 1$ and $Y_{i}(t) = X(t)$ if $N_i(t) = 0$.  The channel we consider is memoryless in the sense that $N_{i}(t)$ is drawn i.i.d.\ from a $\operatorname{Bern}(1 - \epsilon_{i})$ distribution.

%\begin{equation}
%%\label{eq:channel_model}
%	Y_{i}(t) = X(t) \cdot N_{i}(t),
%\end{equation}
%
%where $N_{i}(l)$ is a $\operatorname{Bern}(1 - \epsilon_{i})$ random variable representing the noise at user $i$'s $l^{\mathrm{th}}$ channel output.  The channel is memoryless in the sense that $N_{i}(l)$ is drawn i.i.d.\ from a $\operatorname{Bern}(1 - \epsilon_{i})$ distribution.  %Specifically, the statistics of $N_{i}(l)$ are such that
%
%\begin{equation}
%\label{eq:nstat}
%	N_{i}(l) =
%	\begin{cases}
%		0 & \text{with probability} \ \epsilon_{i}, \\
%		1 & \text{with probability} \ 1 - \epsilon_{i}.
%	\end{cases}
%\end{equation}
%
%Thus, $Y_{i}(l)$ takes on values in the alphabet $\mathcal{Y} = \{-1, 0, +1\}$ and we interpret $Y_{i}(l)$ as the output of an erasure channel, with erasure probability $\epsilon_{i}$, when $X(l)$ is the channel input (see Section~\ref{sec:erased_rv} for a discussion of the erasure channel model).
%
%Thus, $Y_{i}(l)$ takes on values in the alphabet $\mathcal{Y} = \{-1, 0, +1\}$.  If we define `0' as the erasure symbol, we can interpret $Y_{i}(l)$ as the output of an erasure channel with erasure probability $\epsilon_{i}$ when $X(l)$ is the channel input. Without loss of generality, we will assume that $0 < \epsilon_{1} < \epsilon_{2} < \ldots < \epsilon_{n} < 1$. %Users on channels with the same erasure probability are represented by a user of the smallest distortion

%Now although the length-$n$ channel input is broadcasted to both users, we will assume that due to each user's delay constraint, the $i^{\mathrm{th}}$ user reconstructs the source after observing only the first $n_{i}$ channel input symbols, denoted as $X^{n_{i}}$, where $n_{i} \leq n$.  Specifically, we have that $n = \max(n_{1}, n_{2})$.

The problem we consider involves causal feedback  that is universally available.  That is, at time $T$, we assume that $\{Z_1(t), Z_2(t)\}_{t=1, 2, \ldots, T-1}$ is available to the transmitter and all receivers. 
After $W$ channel uses, user $i$ utilizes the feedback and his own channel output to reconstruct the source as a length-$N$ sequence, denoted as $\hat{S}_{i}^{N}$.  We will be interested in a fractional recovery requirement so that each symbol in $\hat{S}_{i}^{N}$ either faithfully recovers the corresponding symbol in $S^{N}$, or otherwise a failure is indicated with an erasure symbol, i.e., we do not allow for any bit flips.

More precisely, we choose the reconstruction alphabet $\mathcal{\hat{S}}$ to be an augmented version of the source alphabet so that $\mathcal{\hat{S}} = \{0, 1, \star\}$, where the additional `$\star$' symbol indicates an erasure symbol.  Let $\mathcal{D} = [0,1]$ and $d_i \in \mathcal{D}$ be the distortion user $i$ requires.  We then express the constraint that an achievable code ensures that each user $i \in \{1, 2\}$ achieves a fractional recovery of $1 - d_{i}$ with the following definition.

%by first defining the set $D_{i}(\hat{S}_{i}^{N}) = \{l \in [N] : \hat{S}_{i}(l) = 0\}$.

%Note that the we have taken the reconstruction alphabet to be an augmented version of the source alphabet by adding the additional `0' symbol, which we interpret as an erasure symbol.  In our reconstruction, we require that $\hat{S}_{i}^{N}$ is such that for $l \in [N]$, $\hat{S}_{i}(l) = S(l)$ or $\hat{S}_{i}(l) = 0$.  In other words, we do not allow any bit flips in our reconstruction so that $\hat{S}_{i}(l)$ either faithfully reproduces $S(l)$ or otherwise outputs an erasure symbol.


%%%On a symbol-by-symbol basis, we measure the reconstruction's fidelity with the erasure distortion $d_{E}: \mathcal{S} \times \hat{\mathcal{S}} \to \{0,1, \infty\}$ given by
%%%%Depending on the distortion measure we use, $\mathcal{\hat{S}}$ can either be a binary source alphabet or the extended alphabet $\mathcal{\hat{S}_{E}} = \{-1, 0, 1\}$, where `0' denotes the source erasure symbol.
%%%
%%%%%%%%%%%%%%%%%%%%%
%%%%%%%%%%%%%%%%%%%%%
%%%%On the other hand, when the output alphabet can include the source erasure symbol as in the second case, then on a symbol-by-symbol comparison, the reconstruction's fidelity is measured with the erasure distortion $d_{E} : \mathcal{S} \times \mathcal{\hat{S}_{E}} \to \{0, 1, \infty\}$, given by
%%%
%%%\begin{equation}
%%%\label{eq:symbol_distortion1}
%%%	d_{E}(s, \hat{s}) =
%%%	\begin{cases}
%%%		0 & \text{if } \hat{s} = s, \\
%%%		1 & \text{if } \hat{s} = 0 \\
%%%		\infty & \text{otherwise}.
%%%	\end{cases}
%%%\end{equation}
%%%%
%%%The per-letter distortion of a vector is then defined as $d(s^{k}, \hat{s}^{k}) = \frac{1}{k} \Sigma_{m=1}^{k} d(s_{m}, \hat{s}_{m})$.
%%%
%%%%\begin{equation}
%%%%\label{eq:vector_distortion1}
%%%%	d(s^{k}, \hat{s}^{k}) = \frac{1}{k} \sum\limits_{i=1}^k d(s_{i}, \hat{s}_{i}).
%%%%\end{equation}
%%%%
%%%%where $d(\cdot, \cdot)$ can be either $d_{H}(\cdot, \cdot)$ or $d_{E}(\cdot, \cdot)$.
%%%
%%%%We point out that with the erasure distortion measure,  a finite distortion value indicates that we are certain about the accuracy of the non-erased symbols in our reconstruction, while the actual value of the distortion is a measure of the proportion of bits that we are unsure about, i.e., that have been erased in the reconstruction.
%%%
%%%%%%%%%%%%%%%%%%%%%
%%%%%%%%%%%%%%%%%%%%
%%%
%%%We point out that with the erasure distortion measure,  a finite distortion value indicates that we are certain about the accuracy of the non-erased symbols in our reconstruction, while the actual value of the distortion is a measure of the proportion of bits that we are unsure about, i.e., that have been erased in the reconstruction.  We now define the components of our problem.

\begin{mydef}
\label{def:code_two_users_feedback}
	An $(N, W, d_{1}, d_{2})$ code for source $S$ on the erasure broadcast channel with \emph{universal} feedback consists of	
	\begin{enumerate}
		\item a sequence of encoding functions $f_{t, N} : \mathcal{S}^{N} \times  \prod_{j = 1}^{2} \mathcal{Z}^{t-1} \to \mathcal{X}$ for $t \in [W]$, such that $X(t) = f_{t, N}(S^{N}, Z_1^{t -1}, Z_2^{t -1})$, and
		
		\item two decoding functions $g_{i,N} : \mathcal{Y}^{W} \times \mathcal{Z}^{2W} \to \mathcal{\hat{S}}^{N}$ s.t.\ for $i \in \{1, 2\}$, $\hat{S}_{i}^{N} = g_{i,N}(Y_{i}^{W}, Z_1^{W}, Z_2^{W})$, and
		\begin{enumerate}
			\item $\hat{S}_{i}^{N}$ is such that for $t \in [N]$, if $\hat{S}_{i}(t) \neq S(t)$, then $\hat{S}_{i}(t) = \star$,
%			\item $\mathbb{E}   \left\vert{D_{i}(\hat{S}_{i}^{N})}\right\vert \leq d_{i}$
			\item $\mathbb{E}   \left\vert{\{t \in [N] \mid \hat{S}_{i}(t) = \star\}}\right\vert \leq N d_{i}$.
		\end{enumerate}
		 	
%	$\mathbb{E}d(S^{k}, g_{i}( X^{n} \cdot N_{i}^{n})) \leq D_{i}$ holds for $i \in \{1,2\}$		
	\end{enumerate}
	
%	where $\mathbb{E}(\cdot)$ is the expectation operation and $\vert A \vert$ denotes the cardinality of set $A$.
\end{mydef}

%\begin{mydef}
%\label{def:code}
%	An $(N, W, d_{1}, d_{2}, d_{3})$ code for source $S$ on the erasure broadcast channel consists of
%	\begin{enumerate}
%		\item a sequence of encoding functions $f_{i, N} : \mathcal{S}^{N} \times  \prod_{j = 1}^{3} \mathcal{N}^{i-1} \to \mathcal{X}$ for $i \in [W]$, such that $X(i) = f_{i, N}(S^{N}, N_1^{i -1}, N_2^{i -1}, N_3^{i - 1})$, and
%		
%		\item three decoding functions $g_{i,N} : \mathcal{Y}^{W} \times  \prod_{j = 1}^{3} \mathcal{N}^{W} \to \mathcal{\hat{S}}^{N}$ s.t.\ $\hat{S}_{i}^{N} = g_{i,N}(Y_{i}^{W}, N_1^{W}, N_2^{W}, N_3^{W})$, $i \in \{1, 2, 3\}$,
%		\begin{enumerate}
%			\item $\hat{S}_{i}^{N}$ is such that for $t \in [N]$, if $\hat{S}_{i}(t) \neq S(t)$, then $\hat{S}_{i}(t) = \star$,
%%			\item $\mathbb{E}   \left\vert{D_{i}(\hat{S}_{i}^{N})}\right\vert \leq d_{i}$
%			\item $\mathbb{E}   \left\vert{\{t \in [N] \mid \hat{S}_{i}(t) = \star\}}\right\vert \leq N d_{i}$,
%		\end{enumerate}
%		 	
%%	$\mathbb{E}d(S^{k}, g_{i}( X^{n} \cdot N_{i}^{n})) \leq D_{i}$ holds for $i \in \{1,2\}$		
%	\end{enumerate}
%	
%%	where $\mathbb{E}(\cdot)$ is the expectation operation and $\vert A \vert$ denotes the cardinality of set $A$.
%\end{mydef}

%A point is now made about the modelling of latencies in our problem.  We define the \emph{latency} or \emph{bandwidth expansion factor} $b \in [0, \infty )$, as the number of channel uses per source symbol that are delivered over the broadcast channel, i.e., $b \triangleq n/k$.  This is to say that $b \cdot k$ channel uses are required before both users can reconstruct $S^{k}$ subject to their distortion constraints.  Our problem is now defined as characterizing the achievable latency region under a given pair of distortion constraints as per the next definition.

%A point is now made about the modelling of delays in our problem.  We define the bandwidth expansion factor $b \in [0, \infty )$, as $b = n/k$.  We interpret $b$ as the normalized latency (in units of ``channel uses per source symbol''), before both users can reconstruct $S^{k}$ subject to their distortion constraints.  Our problem is now defined as characterizing the achievable distortion region under a given bandwidth expansion factor as per the next definition.
%
We again mention that in our problem formulation, we assume that all receivers have causal knowledge of $(Z_{1}^{t-1}, Z_{2}^{t-1})$ at time $t$.  That is, each receiver has causal knowledge of which packets were received.  This can be made possible, for example, through the control plane of a network.

We define the {\it latency} that a given code requires before all users can recover their desired fraction of the source as follows.
\begin{mydef}
\label{def:latency_two_users}
	The latency, $w$, of an~$(N, W, d_{1}, d_{2})$ code is the number of channel uses per source symbol that the code requires to meet all distortion demands, i.e., $w = W/N$.
\end{mydef}
%
Our goal is to characterize the achievable latencies under a prescribed distortion vector,
as per the following definition.
\begin{mydef}
\label{def:achievable_general_feedback}
	Latency $w$ is said to be $(d_{1}, d_{2})$-achievable over the erasure broadcast channel if for every $\delta > 0$, there exists for sufficiently large $N$, an $(N, wN, \hat{d}_{1}, \hat{d}_{2})$ code such that for all $i \in \{1, 2\}$, $d_{i}+\delta \geq \hat{d}_{i}$.
	
%	\begin{equation}
%		D_{i}+\delta \geq d_{i},  \quad i \in \{1, 2\}.
%	\end{equation}	
%	Alternatively, we may occasionally say that the tuple $(b, D_{1}, D_{2})$ is achievable if the latency $b$ is $(D_{1}, D_{2})$-achievable.
	
%The achievable latency region is the set of all achievable latencies under the prescribed distortion vector.

\end{mydef}

\begin{remark}
	We remark that while our definitions have assumed binary source and channel input symbols for simplicity, our results can be easily extended to non-binary alphabets.
\end{remark}

%In the remaining sections, we consider the cases $n=2$ and $n=3$.  We will also make a mention whenever the strategies we use for these cases can be generalized for $n$ users.  
In the next section, we show that the point-to-point Shannon bound can always be achieved for the problem we have just defined.  In Section~\ref{sec:one_sided_feedback}, we present a variation of the problem where only the stronger user has access to a feedback channel.  We further study the case of three receivers in the sequel.  
%While this may seem to be an oversimplification, it is worth noting that to the authors' best knowledge, a solution is also not known for the related channel coding problem involving $n$ users~\cite{GGT}.

%\begin{remark}
%\label{rem:deps}
%Throughout this paper we will assume that for each user $i \in [n]$, we have that $d_i < \epsilon_i$. Any user with $d_i \ge \epsilon_i$ will be trivially satisfied by the systematic portion of our segmentation-based coding scheme.  Furthermore, we will show in Lemma~\ref{lem:bn1} that within our class of coding schemes, such a systematic portion can be transmitted without loss of optimality when at least one user satisfies $d_i < \epsilon_i$. Finally, if every user satisfies $d_i \ge \epsilon_i$, a simple uncoded transmission scheme is easily shown to be optimal.
%\end{remark}
%
%\begin{remark}
%	While our system model has assumed binary alphabets for both the source and channel input sequences, our results can be easily extended to larger alphabet sizes for the purpose of applying our results to packet erasure networks.
%\end{remark}


%For convenience, throughout this paper, for any positive integer $k$, we shorthand the set $\{1,2,\dots,k\}$ as $[k].$
%
%Consider broadcasting a length-$N$ binary sequence $S = \{s(j)\}_{j\in[N]}$ from a source to $n$ receivers over independent memoryless binary erasure channels. For $i\in[n]$, denote the erasure rate of the channel between the source and receiver $i$ as $\epsilon_i$. Applying some encoding scheme to $S$, the source generates and broadcasts the coded binary sequence $X=\{x_j\}_{j\ge1}$. Each receiver $i$ retrieves from the channel a sequence $Y^i = \{y_{j}^i\}_{j\ge1}$ where $y_{j}^i \in \{0,1,e\}$, $e$ indicating an erasure. Denote $Y^{i,W} = \{y_{j}^i\}_{j\in[W]}$ as the truncation of $Y^i$ up to the $W$th symbol. Receiver $i$ reconstructs from $Y^{i,W}$ a copy $\hat S^{i,W}=\{\hat s_{j}^{i,W}\}_{j\in[N]},$ where $\hat s_{j}^{i,W}\in\{0,1,e\}.$
%
%%\subsection{Erasure Distortion}
%We consider a symbol-by-symbol based erasure reconstruction distortion measure $D_e: \{0,1\}\times \{0,1,e\}\mapsto \mathds R$ defined as
%\begin{align*}
%D_E(s,\hat s) = \left\{\begin{array}{lr} 0,& \textnormal{if } \hat s=s\\ 1, & \textnormal{if } \hat s\ne s, \hat s\ne e\\ \infty,& \textnormal{if } \hat s=e \end{array}\right.
%\end{align*}
%%for symbols $s\in\{0,1\}$ and $\hat s\in\{0,1,e\}$,
%and
%\begin{align*}
%D(S,\hat S^{i,W}) =  \frac{1}{N}\sum_{j\in[N]} D_E(s_j,\hat s_{j}^{i,W}).
%\end{align*}
%for the source sequence $S$ and the sequence $\hat S^{i,W}$ reconstructed by user $i$ from the first $W$ symbols of the channel output sequence $Y^{i,W}$.
%
%Now our problem is to upper bound and lower bound the minimum $W$, or $w=W/N,$ which we refer to as the {\it latency}, that guarantees $\mathds E[ D(S,\hat S^{i,W})]\le d_i$ for all $i\in[n]$,
%where $d_i$ is the given distortion requirement of user $i$.
%
%Clearly, $w$ is trivially lower bounded by $\max_{i\in[n]}\{\frac{1-d_i}{1-\epsilon_i}\}$. In the next section, we provide an upper bound of $w$ and the achieving coding scheme. We show that the upper bound is achieved with a coding scheme that sends a part of the source sequence uncoded, partitions the rest of the source sequence into a number of segments, and encodes each segment by a systematic optimal erasure code. The optimal segmentation can be found by solving a linear programming problem. We also show in Sec.~\ref{sec:jscc} that the segmentation-based scheme is equivalent to the Mittal-Phamdo~\cite{} scheme for erasure channels.

%\input{SysModel/individual_code}




%\input{SysModel/JSCC_approach3}









%\section{Background and Motivation}

\subsection{IBM Streams}

IBM Streams is a general-purpose, distributed stream processing system. It
allows users to develop, deploy and manage long-running streaming applications
which require high-throughput and low-latency online processing.

The IBM Streams platform grew out of the research work on the Stream Processing
Core~\cite{spc-2006}.  While the platform has changed significantly since then,
that work established the general architecture that Streams still follows today:
job, resource and graph topology management in centralized services; processing
elements (PEs) which contain user code, distributed across all hosts,
communicating over typed input and output ports; brokers publish-subscribe
communication between jobs; and host controllers on each host which
launch PEs on behalf of the platform.

The modern Streams platform approaches general-purpose cluster management, as
shown in Figure~\ref{fig:streams_v4_v6}. The responsibilities of the platform
services include all job and PE life cycle management; domain name resolution
between the PEs; all metrics collection and reporting; host and resource
management; authentication and authorization; and all log collection. The
platform relies on ZooKeeper~\cite{zookeeper} for consistent, durable metadata
storage which it uses for fault tolerance.

Developers write Streams applications in SPL~\cite{spl-2017} which is a
programming language that presents streams, operators and tuples as
abstractions. Operators continuously consume and produce tuples over streams.
SPL allows programmers to write custom logic in their operators, and to invoke
operators from existing toolkits. Compiled SPL applications become archives that
contain: shared libraries for the operators; graph topology metadata which tells
both the platform and the SPL runtime how to connect those operators; and
external dependencies. At runtime, PEs contain one or more operators. Operators
inside of the same PE communicate through function calls or queues. Operators
that run in different PEs communicate over TCP connections that the PEs
establish at startup. PEs learn what operators they contain, and how to connect
to operators in other PEs, at startup from the graph topology metadata provided
by the platform.

We use ``legacy Streams'' to refer to the IBM Streams version 4 family. The
version 5 family is for Kubernetes, but is not cloud native. It uses the
lift-and-shift approach and creates a platform-within-a-platform: it deploys a
containerized version of the legacy Streams platform within Kubernetes.

\subsection{Kubernetes}

Borg~\cite{borg-2015} is a cluster management platform used internally at Google
to schedule, maintain and monitor the applications their internal infrastructure
and external applications depend on. Kubernetes~\cite{kube} is the open-source
successor to Borg that is an industry standard cloud orchestration platform.

From a user's perspective, Kubernetes abstracts running a distributed
application on a cluster of machines. Users package their applications into
containers and deploy those containers to Kubernetes, which runs those
containers in \emph{pods}. Kubernetes handles all life cycle management of pods,
including scheduling, restarting and migration in case of failures.

Internally, Kubernetes tracks all entities as \emph{objects}~\cite{kubeobjects}.
All objects have a name and a specification that describes its desired state.
Kubernetes stores objects in etcd~\cite{etcd}, making them persistent,
highly-available and reliably accessible across the cluster. Objects are exposed
to users through \emph{resources}. All resources can have
\emph{controllers}~\cite{kubecontrollers}, which react to changes in resources.
For example, when a user changes the number of replicas in a
\code{ReplicaSet}, it is the \code{ReplicaSet} controller which makes sure the
desired number of pods are running. Users can extend Kubernetes through
\emph{custom resource definitions} (CRDs)~\cite{kubecrd}. CRDs can contain
arbitrary content, and controllers for a CRD can take any kind of action.

Architecturally, a Kubernetes cluster consists of nodes. Each node runs a
\emph{kubelet} which receives pod creation requests and makes sure that the
requisite containers are running on that node. Nodes also run a
\emph{kube-proxy} which maintains the network rules for that node on behalf of
the pods. The \emph{kube-api-server} is the central point of contact: it
receives API requests, stores objects in etcd, asks the scheduler to schedule
pods, and talks to the kubelets and kube-proxies on each node. Finally,
\emph{namespaces} logically partition the cluster. Objects which should not know
about each other live in separate namespaces, which allows them to share the
same physical infrastructure without interference.

\subsection{Motivation}
\label{sec:motivation}

Systems like Kubernetes are commonly called ``container orchestration''
platforms. We find that characterization reductive to the point of being
misleading; no one would describe operating systems as ``binary executable
orchestration.'' We adopt the idea from Verma et al.~\cite{borg-2015} that
systems like Kubernetes are ``the kernel of a distributed system.'' Through CRDs
and their controllers, Kubernetes provides state-as-a-service in a distributed
system. Architectures like the one we propose are the result of taking that view 
seriously.

The Streams legacy platform has obvious parallels to the Kubernetes
architecture, and that is not a coincidence: they solve similar problems.
Both are designed to abstract running arbitrary user-code across a distributed
system.  We suspect that Streams is not unique, and that there are many
non-trivial platforms which have to provide similar levels of cluster
management.  The benefits to being cloud native and offloading the platform
to an existing cloud management system are: 
\begin{itemize}
    \item Significantly less platform code.
    \item Better scheduling and resource management, as all services on the cluster are 
        scheduled by one platform.
    \item Easier service integration.
    \item Standardized management, logging and metrics.
\end{itemize}
The rest of this paper presents the design of replacing the legacy Streams 
platform with Kubernetes itself.


%\input{intuition/intuition-L}


%\input{sbc/sbcE-y-L}

%\input{implication/implication}
%\input{numerical_comparisons/numerical_comparisons}

%\vspace{-1.5em}

%\input{rateless_code_comparison/rateless_code_comparison}
%\input{optimality}
%\input{individual_latencies/individual_latencies}
%\color{red}
%
Parallel composition arises when a trader is faced with multiple AMMs,
but wants to treat them them as if they were a single AMM.
In sequential composition,
the composed AMMs exchange ``hidden'' assets .
In parallel composition,
the composed AMMs compete for overlapping assets.

Suppose Alice wants to trade asset $X$ for asset $Y$.
Bob and Carol both offer AMMs to convert from $X$ to $Y$.
Bob's AMM is $B(x,y):=x^2y=\frac{3}{4}$ in state $(1,\frac{3}{4})$,
while Carol's AMM is $C(x,y):=x y = 1$ in state $(1,1)$.
Alice would like to compose the two AMMs and treat them as one AMM.
Bob provides a better initial rate of exchange for small trades,
but Carol provides less slippage for large trades.
One can check that if Alice converts 1 unit of $X$,
she gets more $Y$ assets from Bob than from Carol,
while if she converts 3 units,
she gets more from Carol.

This process is not the same as order-book clearing,
because order-book offers are typically expressed in terms of a fixed amount and a fixed price,
while parallel AMM offers are expressed in terms of price curves.
This type of composition occurs, for example,
when a trader is faced with multiple pools as in Uniswap v3~\cite{uniswapv3}.

A rational Alice will split her assets between
Bob and Carol to maximize her return.
Suppose $A(x,y)$ is in state $(a,f(a))$
and $B(x,y)$ in $(b,g(b))$.
Alice splits her $d$ assets,
transferring $t d$ to Bob's AMM
and $(1-t)d$ to Carol's,
returning
\begin{equation*}
  f(a) - f(a+tx) + g(b) - g(b+(1-t)x)
\end{equation*}
units of $Y$.
Let $h(x) = f(a+tx) + g(b+(1-t)x)$.
Define the \emph{parallel composition} of $B$ and $C$
with respect to $v=(t,1-t)$ to be.
\begin{equation*}
    (B\|C)(x,y) := y - h(x) = 0.
\end{equation*}
\begin{lemma}
  $(B\|C)(x,y)$ is a 2-dimensional AMM.
\end{lemma}
\begin{proof}
  $(B\|C)(x,y)$ is twice-differentiable because $f$ and $g$ are twice-differentiable.
  To check that $(B\|C)(x,y)$ is strictly increasing,
  let $x' \geq x$ and $y' \geq y$ where at least one inequality is strict.
  For the first case, suppose $x' > x$ and $y' \geq y$.
  \begin{align*}
    f(a+tx') &< f(a+tx) \\
    g(b+(1-t)x') &< g(b+(1-t)x) \\
    f(a+tx') + g(b+(1-t)x') &< f(a+tx) + g(b+(1-t)x) \\
    h(x') &< h(x) \\
    y - h(x') &> y-h(x)\\
    y' - h(x') &> y-h(x)
  \end{align*}
  The case where $x' \geq x$ and $y'> y$ is similar.

  To check that $\upper((B\|C)(x,y))$ is strictly convex,
  we can verify $h$ is strictly convex.
  Pick distinct $x,x'$.
  For $s \in (0,1)$,
  \begin{multline*}
    s f(a+t x) + (1-s) f(a+t x')\\
    > f(s (a+t x) + (1-s) (a+t x')) 
  \end{multline*}
  \begin{multline*}
    s g(b+(1-t)x) + (1-s) g(b+(1-t)x') \\
    >g(s (b+(1-t)x) + (1-s) (b+(1-t)x'))
  \end{multline*}
  \begin{multline*}
    s (f(a+t x) + g(b+(1-t)x))\\ + (1-s)(f(a + t x')+g(b+(1-t)x'))\\
    > f(s (a + t x) + (1-s) (a + t x'))\\ + 
    g(s (b+(1-t)x) + (1-s) (b+(1-t)x'))
  \end{multline*}
  \begin{equation*}
    s h(x) + (1-s)h(x') > h(s x + (1-s)x')  \\
  \end{equation*}
  which establishes the claim.
\end{proof}

\begin{lemma}
  Let $A := (x,f(x))$, $B := (y,g(y))$ be two AMMs trading assets $X$ and $Y$,
  such that $(a,f(a))$ and $(b,g(b))$ are their respective stable points
  for the valuation $(v,1-v)$.
  If $h_t(x) = f(a + tx) + g(b + (1-t)x)$,
  then $(0,h_t(0))$ is stable point on $A \| B$ with respect to $(v,1-v)$ for all $t \in \Reals$.
\end{lemma}

\begin{proof}
By assumption we have
\begin{align*}
    v a + (1-v)f(a) &< v(a + tx) + (1-v)f(a + tx)\\
    (1-v)f(a) &< tvx + (1-v)f(a + tx)
\end{align*}
and
\begin{align*}
  vb + (1-v)g(b) &< v(b + (1-t)x) + (1-v)g(b + (1-t)x)\\
  (1-v)g(b) &< (1-t)vx + (1-v)g(b + (1-t)x),
\end{align*}
yielding
\begin{align*}
  v 0 + (1-v)h_t(0)
  &= (1-v)h_t(0) \\
  &= (1-v)f(a) + (1-v)g(b) \\
  &< tvx + (1-v)f(a + tx) + (1-t)vx \\
  &\quad \quad + (1-v)g(b + (1-t)x) \\
  &= vx + (1-v)(f(a + tx) + g(b + (1-t)x)) \\
  &= vx + (1-v)h(x)
\end{align*}
so $(0,h_t(0))$ is a stable point for $(v,(1-v))$.
\end{proof}


Parallel composition is well-defined for any valuation,
but what valuation should a rational Alice pick?
Differentiating with respect to $t$ yields
\begin{align}
  0 &= -x f'(a+t x) + x g'(b+(1-t)x)\nonumber\\
  x f'(a+t x) &= x g'(b+(1-t)x) \nonumber\\
  f'(a+t x) &= g'(b+(1-t)x).\eqnlabel{best-split}
\end{align}
Alice maximizes her return when she splits her
assets so that Bob and Carol end up offering the same rate.
Informally, if Bob had ended up providing a better rate,
then Alice should have given him a larger share.
If there is no $t \in (0,1)$ that satisfies \eqnref{best-split},
Alice should give all her assets to the AMM with the better rate.

How should parallel composition be defined for AMMs with multiple asset types?
Suppose Alice has some combination $\bd$ of assets in $X_1,\ldots,X_p$
that she wants to convert into some combination of assets in $Y_1,\ldots,Y_q$.
Alice has a choice of two alternative AMMs:
$B(x_1,\ldots,x_p,y_1,\ldots,y_q)$ and $C(x_1,\ldots,x_p,y_1,\ldots,y_q)$.
Perhaps the most sensible way to define parallel composition is through
asset virtualization.
Alice's input asset vector $\bd$ induces a valuation $\bd/ \|\bd\|_1$
which can be used to define a virtual asset $X$ from $X_1,\ldots,X_p$.
Alice chooses a valuation $\bv$ for her $Y_1,\ldots,Y_q$ outputs
(perhaps the market valuation)
which can be used to define a virtual asset $Y$ from $Y_1,\ldots,Y_q$.
After asset virtualization, the alternative AMMs have the form:
$\tilde{B}(\bx,t)$ and $\tilde{C}(\bx,t)$,
and the definition of parallel composition proceeds as before.

The next lemma describes properties of stable points for two 2-dimensional AMMs composed in parallel.

If $(\bx,f(\bx))$ and $(\by,g(\by))$ are two $n$-dimensional AMMs in states $(\ba,f(\ba))$ and $(\bb,g(\bb))$,
we can define parallel composition as follows.
For $\bt \in [0,1]^{n-1}$,
let $h_{\bt}(\bx) = f(\ba + \bt * \bx) + g(\bb + (\bone - \bt )* \bx)$,
where $*$ is component-wise multiplication.

\begin{lemma}
  \lemmalabel{parallel-stable}
  Let $(\bv,1-\|\bv\|_1))$ be a valuation,
  and let $A := (\bx,f(\bx))$ and $B := (\by,g(\by))$.
  If $(\ba,f(\ba))$ and $(\bb,g(\bb))$ are both stable points with respect to $(\bv,1-\|\bv\|_1))$.
  then $(\bzero,h_{\bt}(\bzero))$ is the stable point on $A \| B$  for $(\bv,1-\|\bv\|_1)$
  for all $\bt \in \Reals^{n-1}$.
  \end{lemma}

  \begin{proof}
    For $\bt \in \Reals^{n-1}$,
    \begin{multline*}
      \bv \cdot \ba + (1-\|\bv\|_1)f(\ba)\\
      < \bv \cdot (\ba + \bt * \bx) + (1-\|\bv\|_1)f(\ba + \bt * \bx)
    \end{multline*}
    \begin{equation*}
      (1-\|\bv\|_1)f(\ba)
      < (\bt * \bx) \cdot \bv + (1-\|\bv\|_1)f(\ba + \bt * \bx)
    \end{equation*}
    and
    \begin{multline*}
      \bv \cdot \bb + (1-\|\bv\|_1)g(\bb)\\
      < \bv \cdot (\bb + (\bone - \bt) * \by) + (1-\|\bv\|_1)g(\bb + (\bone - \bt) * \by)
    \end{multline*}
    \begin{multline*}
      (1-\|\bv\|_1)g(\bb) \\
      < ((\bone - \bt) * \by) \cdot \bv + (1-\|\bv\|_1)g(\bb + (\bone - \bt) * \by),
     \end{multline*}
implying
\begin{align*}
  \bv \cdot \bzero + &(1-\|\bv\|_1)h_t(\bzero) \\
  &= (1-\|\bv\|_1)h_t(\bzero) \\
  &= (1-\|\bv\|_1)f(\ba) +(1-\|\bv\|_1)g(\bb) \\
  &< (\bt * \bx) \cdot \bv + (1-\|\bv\|_1)f(\ba + \bt * \bx) \\
  &\quad \quad+ (\bone-\bt) * \bx \cdot \bv + \\
  &\quad \quad (1-\|\bv\|_1)g(\bb + (\bone-\bt) * \bx)\\
    &= \bv \cdot \bx + (1-\|\bv\|_1)(f(\ba + \bt \cdot \bx) \\
    &\quad \quad + g(\bb + (\bone-\bt) \cdot \bx))  \\
    &= \bv \cdot \bx + (1-\|\bv\|_1)h(\bx)
\end{align*}
so $(0,h_t(0))$ is a stable point for $(\bv,1-\|\bv\|_1)$.
\end{proof}
Most generally, if we have two AMMs $A(\bx,\bz)$ and $B(\by,\bz')$, and valuation $\bw$,
we can write $A| \bw$ as $(\bx,f(\bx))$ and $B| \bw$ as $(\by,g(\by))$.
We then can define parallel composition as before.

\begin{theorem}
    Let $(\bv,\bv')$ be a valuation, and let $A(\bx,\bz)$ and $B(\by,\bz')$ two AMMs.
    If $(\ba^{*},\bb^{*})$ is the stable point for $A$,
    and $(\bc^{*},\bd^{*})$ the stable point for $B$,
    both with respect to $(\bv,\bv')$,
    then $(\bv,\|\bv'\|_2^2)/\|(\bv,\|\bv'\|_2^2)\|_1$ is the stable point for $(A | \bv' ) \| (B | \bv')$.
\end{theorem}

\begin{proof}\sloppy
  \lemmaref{virtual-stable} implies that $(\ba^{*},f(\ba^{*}))$ and $(\bc^{*},g(\bc^{*}))$
  are stable on $A | \bv'$ and $B | \bv'$,
  both with respect to $(\bv,\|\bv'\|_2^2)$. 
\lemmaref{parallel-stable} implies that $(\bzero,h_{\bt}(\bzero))$
is the stable point for
$(A | \bv' ) \| (B | \bv')$ with respect to valuation $(\bv,\|\bv'\|_2^2)$.
\end{proof}

%\color{black}
%
Parallel composition arises when a trader is faced with multiple AMMs,
but wants to treat them them as if they were a single AMM.
In sequential composition,
the composed AMMs exchange ``hidden'' assets .
In parallel composition,
the composed AMMs compete for overlapping assets.

Suppose Alice wants to trade asset $X$ for asset $Y$.
Bob and Carol both offer AMMs to convert from $X$ to $Y$.
Bob's AMM is $B(x,y):=x^2y=\frac{3}{4}$ in state $(1,\frac{3}{4})$,
while Carol's AMM is $C(x,y):=x y = 1$ in state $(1,1)$.
Alice would like to compose the two AMMs and treat them as one AMM.
Bob provides a better initial rate of exchange for small trades,
but Carol provides less slippage for large trades.
One can check that if Alice converts 1 unit of $X$,
she gets more $Y$ assets from Bob than from Carol,
while if she converts 3 units,
she gets more from Carol.

This process is not the same as order-book clearing,
because order-book offers are typically expressed in terms of a fixed amount and a fixed price,
while parallel AMM offers are expressed in terms of price curves.
This type of composition occurs, for example,
when a trader is faced with multiple pools as in Uniswap v3~\cite{uniswapv3}.

A rational Alice will split her assets between
Bob and Carol to maximize her return.
Suppose $A(x,y)$ is in state $(a,f(a))$
and $B(x,y)$ in $(b,g(b))$.
Alice splits her $d$ assets,
transferring $t d$ to Bob's AMM
and $(1-t)d$ to Carol's,
returning
\begin{equation*}
  f(a) - f(a+tx) + g(b) - g(b+(1-t)x)
\end{equation*}
units of $Y$.
Let $h(x) = f(a+tx) + g(b+(1-t)x)$.
Define the \emph{parallel composition} of $B$ and $C$
with respect to $v=(t,1-t)$ to be.
\begin{equation*}
    (B\|C)(x,y) := y - h(x) = 0.
\end{equation*}
\begin{lemma}
  $(B\|C)(x,y)$ is a 2-dimensional AMM.
\end{lemma}
\begin{proof}
  $(B\|C)(x,y)$ is twice-differentiable because $f$ and $g$ are twice-differentiable.
  To check that $(B\|C)(x,y)$ is strictly increasing,
  let $x' \geq x$ and $y' \geq y$ where at least one inequality is strict.
  For the first case, suppose $x' > x$ and $y' \geq y$.
  \begin{align*}
    f(a+tx') &< f(a+tx) \\
    g(b+(1-t)x') &< g(b+(1-t)x) \\
    f(a+tx') + g(b+(1-t)x') &< f(a+tx) + g(b+(1-t)x) \\
    h(x') &< h(x) \\
    y - h(x') &> y-h(x)\\
    y' - h(x') &> y-h(x)
  \end{align*}
  The case where $x' \geq x$ and $y'> y$ is similar.

  To check that $\upper((B\|C)(x,y))$ is strictly convex,
  we can verify $h$ is strictly convex.
  Pick distinct $x,x'$.
  For $s \in (0,1)$,
  \begin{multline*}
    s f(a+t x) + (1-s) f(a+t x')\\
    > f(s (a+t x) + (1-s) (a+t x')) 
  \end{multline*}
  \begin{multline*}
    s g(b+(1-t)x) + (1-s) g(b+(1-t)x') \\
    >g(s (b+(1-t)x) + (1-s) (b+(1-t)x'))
  \end{multline*}
  \begin{multline*}
    s (f(a+t x) + g(b+(1-t)x))\\ + (1-s)(f(a + t x')+g(b+(1-t)x'))\\
    > f(s (a + t x) + (1-s) (a + t x'))\\ + 
    g(s (b+(1-t)x) + (1-s) (b+(1-t)x'))
  \end{multline*}
  \begin{equation*}
    s h(x) + (1-s)h(x') > h(s x + (1-s)x')  \\
  \end{equation*}
  which establishes the claim.
\end{proof}

\begin{lemma}
  Let $A := (x,f(x))$, $B := (y,g(y))$ be two AMMs trading assets $X$ and $Y$,
  such that $(a,f(a))$ and $(b,g(b))$ are their respective stable points
  for the valuation $(v,1-v)$.
  If $h_t(x) = f(a + tx) + g(b + (1-t)x)$,
  then $(0,h_t(0))$ is stable point on $A \| B$ with respect to $(v,1-v)$ for all $t \in \Reals$.
\end{lemma}

\begin{proof}
By assumption we have
\begin{align*}
    v a + (1-v)f(a) &< v(a + tx) + (1-v)f(a + tx)\\
    (1-v)f(a) &< tvx + (1-v)f(a + tx)
\end{align*}
and
\begin{align*}
  vb + (1-v)g(b) &< v(b + (1-t)x) + (1-v)g(b + (1-t)x)\\
  (1-v)g(b) &< (1-t)vx + (1-v)g(b + (1-t)x),
\end{align*}
yielding
\begin{align*}
  v 0 + (1-v)h_t(0)
  &= (1-v)h_t(0) \\
  &= (1-v)f(a) + (1-v)g(b) \\
  &< tvx + (1-v)f(a + tx) + (1-t)vx \\
  &\quad \quad + (1-v)g(b + (1-t)x) \\
  &= vx + (1-v)(f(a + tx) + g(b + (1-t)x)) \\
  &= vx + (1-v)h(x)
\end{align*}
so $(0,h_t(0))$ is a stable point for $(v,(1-v))$.
\end{proof}


Parallel composition is well-defined for any valuation,
but what valuation should a rational Alice pick?
Differentiating with respect to $t$ yields
\begin{align}
  0 &= -x f'(a+t x) + x g'(b+(1-t)x)\nonumber\\
  x f'(a+t x) &= x g'(b+(1-t)x) \nonumber\\
  f'(a+t x) &= g'(b+(1-t)x).\eqnlabel{best-split}
\end{align}
Alice maximizes her return when she splits her
assets so that Bob and Carol end up offering the same rate.
Informally, if Bob had ended up providing a better rate,
then Alice should have given him a larger share.
If there is no $t \in (0,1)$ that satisfies \eqnref{best-split},
Alice should give all her assets to the AMM with the better rate.

How should parallel composition be defined for AMMs with multiple asset types?
Suppose Alice has some combination $\bd$ of assets in $X_1,\ldots,X_p$
that she wants to convert into some combination of assets in $Y_1,\ldots,Y_q$.
Alice has a choice of two alternative AMMs:
$B(x_1,\ldots,x_p,y_1,\ldots,y_q)$ and $C(x_1,\ldots,x_p,y_1,\ldots,y_q)$.
Perhaps the most sensible way to define parallel composition is through
asset virtualization.
Alice's input asset vector $\bd$ induces a valuation $\bd/ \|\bd\|_1$
which can be used to define a virtual asset $X$ from $X_1,\ldots,X_p$.
Alice chooses a valuation $\bv$ for her $Y_1,\ldots,Y_q$ outputs
(perhaps the market valuation)
which can be used to define a virtual asset $Y$ from $Y_1,\ldots,Y_q$.
After asset virtualization, the alternative AMMs have the form:
$\tilde{B}(\bx,t)$ and $\tilde{C}(\bx,t)$,
and the definition of parallel composition proceeds as before.

The next lemma describes properties of stable points for two 2-dimensional AMMs composed in parallel.

If $(\bx,f(\bx))$ and $(\by,g(\by))$ are two $n$-dimensional AMMs in states $(\ba,f(\ba))$ and $(\bb,g(\bb))$,
we can define parallel composition as follows.
For $\bt \in [0,1]^{n-1}$,
let $h_{\bt}(\bx) = f(\ba + \bt * \bx) + g(\bb + (\bone - \bt )* \bx)$,
where $*$ is component-wise multiplication.

\begin{lemma}
  \lemmalabel{parallel-stable}
  Let $(\bv,1-\|\bv\|_1))$ be a valuation,
  and let $A := (\bx,f(\bx))$ and $B := (\by,g(\by))$.
  If $(\ba,f(\ba))$ and $(\bb,g(\bb))$ are both stable points with respect to $(\bv,1-\|\bv\|_1))$.
  then $(\bzero,h_{\bt}(\bzero))$ is the stable point on $A \| B$  for $(\bv,1-\|\bv\|_1)$
  for all $\bt \in \Reals^{n-1}$.
  \end{lemma}

  \begin{proof}
    For $\bt \in \Reals^{n-1}$,
    \begin{multline*}
      \bv \cdot \ba + (1-\|\bv\|_1)f(\ba)\\
      < \bv \cdot (\ba + \bt * \bx) + (1-\|\bv\|_1)f(\ba + \bt * \bx)
    \end{multline*}
    \begin{equation*}
      (1-\|\bv\|_1)f(\ba)
      < (\bt * \bx) \cdot \bv + (1-\|\bv\|_1)f(\ba + \bt * \bx)
    \end{equation*}
    and
    \begin{multline*}
      \bv \cdot \bb + (1-\|\bv\|_1)g(\bb)\\
      < \bv \cdot (\bb + (\bone - \bt) * \by) + (1-\|\bv\|_1)g(\bb + (\bone - \bt) * \by)
    \end{multline*}
    \begin{multline*}
      (1-\|\bv\|_1)g(\bb) \\
      < ((\bone - \bt) * \by) \cdot \bv + (1-\|\bv\|_1)g(\bb + (\bone - \bt) * \by),
     \end{multline*}
implying
\begin{align*}
  \bv \cdot \bzero + &(1-\|\bv\|_1)h_t(\bzero) \\
  &= (1-\|\bv\|_1)h_t(\bzero) \\
  &= (1-\|\bv\|_1)f(\ba) +(1-\|\bv\|_1)g(\bb) \\
  &< (\bt * \bx) \cdot \bv + (1-\|\bv\|_1)f(\ba + \bt * \bx) \\
  &\quad \quad+ (\bone-\bt) * \bx \cdot \bv + \\
  &\quad \quad (1-\|\bv\|_1)g(\bb + (\bone-\bt) * \bx)\\
    &= \bv \cdot \bx + (1-\|\bv\|_1)(f(\ba + \bt \cdot \bx) \\
    &\quad \quad + g(\bb + (\bone-\bt) \cdot \bx))  \\
    &= \bv \cdot \bx + (1-\|\bv\|_1)h(\bx)
\end{align*}
so $(0,h_t(0))$ is a stable point for $(\bv,1-\|\bv\|_1)$.
\end{proof}
Most generally, if we have two AMMs $A(\bx,\bz)$ and $B(\by,\bz')$, and valuation $\bw$,
we can write $A| \bw$ as $(\bx,f(\bx))$ and $B| \bw$ as $(\by,g(\by))$.
We then can define parallel composition as before.

\begin{theorem}
    Let $(\bv,\bv')$ be a valuation, and let $A(\bx,\bz)$ and $B(\by,\bz')$ two AMMs.
    If $(\ba^{*},\bb^{*})$ is the stable point for $A$,
    and $(\bc^{*},\bd^{*})$ the stable point for $B$,
    both with respect to $(\bv,\bv')$,
    then $(\bv,\|\bv'\|_2^2)/\|(\bv,\|\bv'\|_2^2)\|_1$ is the stable point for $(A | \bv' ) \| (B | \bv')$.
\end{theorem}

\begin{proof}\sloppy
  \lemmaref{virtual-stable} implies that $(\ba^{*},f(\ba^{*}))$ and $(\bc^{*},g(\bc^{*}))$
  are stable on $A | \bv'$ and $B | \bv'$,
  both with respect to $(\bv,\|\bv'\|_2^2)$. 
\lemmaref{parallel-stable} implies that $(\bzero,h_{\bt}(\bzero))$
is the stable point for
$(A | \bv' ) \| (B | \bv')$ with respect to valuation $(\bv,\|\bv'\|_2^2)$.
\end{proof}

%% \vspace{-0.5em}
\section{Conclusion}
% \vspace{-0.5em}
Recent advances in multimodal single-cell technology have enabled the simultaneous profiling of the transcriptome alongside other cellular modalities, leading to an increase in the availability of multimodal single-cell data. In this paper, we present \method{}, a multimodal transformer model for single-cell surface protein abundance from gene expression measurements. We combined the data with prior biological interaction knowledge from the STRING database into a richly connected heterogeneous graph and leveraged the transformer architectures to learn an accurate mapping between gene expression and surface protein abundance. Remarkably, \method{} achieves superior and more stable performance than other baselines on both 2021 and 2022 NeurIPS single-cell datasets.

\noindent\textbf{Future Work.}
% Our work is an extension of the model we implemented in the NeurIPS 2022 competition. 
Our framework of multimodal transformers with the cross-modality heterogeneous graph goes far beyond the specific downstream task of modality prediction, and there are lots of potentials to be further explored. Our graph contains three types of nodes. While the cell embeddings are used for predictions, the remaining protein embeddings and gene embeddings may be further interpreted for other tasks. The similarities between proteins may show data-specific protein-protein relationships, while the attention matrix of the gene transformer may help to identify marker genes of each cell type. Additionally, we may achieve gene interaction prediction using the attention mechanism.
% under adequate regulations. 
% We expect \method{} to be capable of much more than just modality prediction. Note that currently, we fuse information from different transformers with message-passing GNNs. 
To extend more on transformers, a potential next step is implementing cross-attention cross-modalities. Ideally, all three types of nodes, namely genes, proteins, and cells, would be jointly modeled using a large transformer that includes specific regulations for each modality. 

% insight of protein and gene embedding (diff task)

% all in one transformer

% \noindent\textbf{Limitations and future work}
% Despite the noticeable performance improvement by utilizing transformers with the cross-modality heterogeneous graph, there are still bottlenecks in the current settings. To begin with, we noticed that the performance variations of all methods are consistently higher in the ``CITE'' dataset compared to the ``GEX2ADT'' dataset. We hypothesized that the increased variability in ``CITE'' was due to both less number of training samples (43k vs. 66k cells) and a significantly more number of testing samples used (28k vs. 1k cells). One straightforward solution to alleviate the high variation issue is to include more training samples, which is not always possible given the training data availability. Nevertheless, publicly available single-cell datasets have been accumulated over the past decades and are still being collected on an ever-increasing scale. Taking advantage of these large-scale atlases is the key to a more stable and well-performing model, as some of the intra-cell variations could be common across different datasets. For example, reference-based methods are commonly used to identify the cell identity of a single cell, or cell-type compositions of a mixture of cells. (other examples for pretrained, e.g., scbert)


%\noindent\textbf{Future work.}
% Our work is an extension of the model we implemented in the NeurIPS 2022 competition. Now our framework of multimodal transformers with the cross-modality heterogeneous graph goes far beyond the specific downstream task of modality prediction, and there are lots of potentials to be further explored. Our graph contains three types of nodes. while the cell embeddings are used for predictions, the remaining protein embeddings and gene embeddings may be further interpreted for other tasks. The similarities between proteins may show data-specific protein-protein relationships, while the attention matrix of the gene transformer may help to identify marker genes of each cell type. Additionally, we may achieve gene interaction prediction using the attention mechanism under adequate regulations. We expect \method{} to be capable of much more than just modality prediction. Note that currently, we fuse information from different transformers with message-passing GNNs. To extend more on transformers, a potential next step is implementing cross-attention cross-modalities. Ideally, all three types of nodes, namely genes, proteins, and cells, would be jointly modeled using a large transformer that includes specific regulations for each modality. The self-attention within each modality would reconstruct the prior interaction network, while the cross-attention between modalities would be supervised by the data observations. Then, The attention matrix will provide insights into all the internal interactions and cross-relationships. With the linearized transformer, this idea would be both practical and versatile.

% \begin{acks}
% This research is supported by the National Science Foundation (NSF) and Johnson \& Johnson.
% \end{acks}

%\vspace{-2em}

%\appendix

\section{Experimental details and more results}
\label{sec:app_exp}
We run all the experiments on Nvidia RTX 2080 Ti GPUs and V100 GPUs. Table~\ref{tab:app_testbed} summarizes the set of images used in each figure or table in the main paper.  

\captionsetup[table]{font=small}
\begin{table}[H]
    \small
    \centering
    \begin{tabular}{|p{2.5cm}|p{10cm}|}
    \toprule
         {\bf Figure/Table} & {\bf Comments}	\\
    \midrule
        Figure~\ref{fig:BN_var}a & We’ve tuned hyperparams for the attack (see Appendix~\ref{sec:app_hyperparam}) and carried out evaluations on the whole CIFAR-subset. The first sampled batch of size 16 from CIFAR-subset was used in Figure~\ref{fig:BN_var}a to demonstrate the quality of recovery for low-resolution images when BatchNorm statistics are not assumed to be known.  \\
        \midrule
        Figure~\ref{fig:BN_var}b & We’ve tuned hyperparams for the attack (see Appendix~\ref{sec:app_hyperparam}) and carried out evaluations on the whole ImageNet-subset. The best-reconstructed image in ImageNet-subset was used in Figure 1b to demonstrate the quality of recovery for high-resolution images when BatchNorm statistics are not assumed to be known.\\
        \midrule
        Figure~\ref{fig:batch_label_dist} & Percentages of class labels per batch were evaluated over the entire CIFAR10 dataset, for a random seed.	\\
        \midrule
        Figure~\ref{fig:reconstructed_labels} & The first sampled batch of size 16 was used in Figure~\ref{fig:reconstructed_labels} to demonstrate the quality of recovery when labels are not assumed to be known.	\\
        \midrule
        Table~\ref{tab:exp_summary} and Figure~\ref{fig:vis_recon} & We’ve tuned hyperparams for the attack and carried out evaluations on the whole CIFAR-subset. Table~\ref{tab:exp_summary} summarizes the performance of the attack on the whole CIFAR-subset and  Figure~\ref{fig:vis_recon} shows example images.\\
    \bottomrule
    \end{tabular}
    \caption{Summary of experimental testbed for each evaluation.}
    \label{tab:app_testbed}
\end{table}


\subsection{Hyper-parameters}
\label{sec:app_hyperparam}



\paragraph{Training.} For all experiments, we train ResNet-18 for 200 epochs, with a batch size of 128. We use SGD with momentum 0.9 as the optimizer. The initial learning rate is set to 0.1 by default, except for gradient pruning with $p=0.99$ and $p=0.999$. where we set the initial learning rate to 0.02. We decay the learning rate by a factor of 0.1 every 50 epochs.

\paragraph{The attack.}  We report the performance under different $\alpha_{\rm TV}$'s (Figure~\ref{fig:BN_tv_tune}) and $\alpha_{\rm BN}$'s (Figure~\ref{fig:BN_reg_tune}).

\begin{figure}[H]
\captionsetup[subfigure]{labelfont=scriptsize, textfont=tiny}
    \centering
    \subfloat[Original]{\includegraphics[width=0.12\textwidth]{imgs/appendix/TV/original.png}}
    \subfloat[$\alpha_{\rm TV}$=0]{\includegraphics[width=0.12\textwidth]{imgs/appendix/TV/tv_0.png}}
    \subfloat[$\alpha_{\rm TV}$=1e-3]{\includegraphics[width=0.12\textwidth]{imgs/appendix/TV/tv_1e-3.png}}
    \subfloat[$\alpha_{\rm TV}$=5e-3]{\includegraphics[width=0.12\textwidth]{imgs/appendix/TV/tv_5e-3.png}}
    \subfloat[$\alpha_{\rm TV}$=1e-2]{\includegraphics[width=0.12\textwidth]{imgs/appendix/TV/tv_1e-2.png}}
    \subfloat[$\alpha_{\rm TV}$=5e-2]{\includegraphics[width=0.12\textwidth]{imgs/appendix/TV/tv_5e-2.png}}
    \subfloat[$\alpha_{\rm TV}$=1e-1]{\includegraphics[width=0.12\textwidth]{imgs/appendix/TV/tv_1e-1.png}}
    \subfloat[$\alpha_{\rm TV}$=5e-1]{\includegraphics[width=0.12\textwidth]{imgs/appendix/TV/tv_5e-1.png}}
    
    \caption{Attacking a single CIFAR-10 images in $\rm BN_{exact}$ setting, with different coefficients for the total variation regularizer ($\alpha_{\rm TV}$'s). $\alpha_{\rm TV}$=1e-2 gives the best reconstruction.}
    \label{fig:BN_tv_tune}
\end{figure}


\begin{figure}[H]
\vspace{-5mm}
\captionsetup[subfigure]{labelfont=scriptsize, textfont=tiny}
    \centering
    \subfloat[Original]{\includegraphics[width=0.16\textwidth]{imgs/assumptions/BN/original.png}}
    \subfloat[$\alpha_{\rm BN}$=0]{\includegraphics[width=0.16\textwidth]{imgs/assumptions/BN/reconstructed_train_train_bn=0.png}}
    \subfloat[$\alpha_{\rm BN}$=5e-4]{\includegraphics[width=0.16\textwidth]{imgs/assumptions/BN/reconstructed_train_train_bn=5e-4.png}}
    \subfloat[$\alpha_{\rm BN}$=1e-3]{\includegraphics[width=0.16\textwidth]{imgs/assumptions/BN/reconstructed_train_train_bn=1e-3.png}}
    \subfloat[$\alpha_{\rm BN}$=5e-3]{\includegraphics[width=0.16\textwidth]{imgs/assumptions/BN/reconstructed_train_train_bn=5e-3.png}}
    \subfloat[$\alpha_{\rm BN}$=1e-2]{\includegraphics[width=0.16\textwidth ]{imgs/assumptions/BN/reconstructed_train_train_bn=1e-2.png}}
    \caption{Attacking a batch of 16 CIFAR-10 images in $\rm BN_{infer}$ setting, with different coefficients for the BatchNorm regularizer ($\alpha_{\rm BN}$'s). $\alpha_{\rm TV}$=1e-3 gives the best reconstruction.}
    \label{fig:BN_reg_tune}
\end{figure}


\subsection{Details and more results for Section~\ref{sec:assumption}}

\paragraph{Attacking a single ImageNet image.} We launched the attack on ImageNet using the objective function in Eq.~\ref{eq:objective}, where $\alpha_{\rm TV}=0.1$, $\alpha_{\rm BN}=0.001$. We run the attack for 24,000 iterations using Adam optimizer, with initial learning rate 0.1, and decay the learning rate by a factor of $0.1$ at 
$3/8,5/8,7/8$ of training. We rerun the attack 5 times and present the best results measured by LPIPS in Figure~\ref{fig:BN_var}.

\paragraph{Qualitative and quantitative results for a more realistic attack.} We also present results of a more realistic attack in Table~\ref{tab:exp_summary_realistic} and Figure~\ref{fig:vis_recon_realistic}, where the attacker does {\em not} know BatchNorm statistics but knows the private labels. We assume the private labels to be known in this evaluation, because for those batches whose distribution of labels is uniform, the restoration of labels should still be quite accurate~\citep{yin2021see}.
As shown, in the evaluated setting, the attack is no longer effective when the batch size is 32 and Intra-InstaHide with $k=4$ is applied. The accuracy loss to stop the realistic attack is only around $3\%$ (compared to around $7\%$ to stop the strongest attack) .


\begin{figure}[H]
\captionsetup[subfigure]{font=small}
  \centering
  \subfloat{\includegraphics[width=\textwidth]{imgs/Compare_16_32.png}}
  \caption{Reconstruction results under different defenses for a more realistic setting (when the attacker knows private labels but does not know BatchNorm statistics). We also present the results for the strongest attack from Figure~\ref{fig:vis_recon} for comparison. Using Intra-InstaHide with $k=4$ and batch size 32 seems to stop the realistic attack.}
  \label{fig:vis_recon_realistic}
\end{figure}

\captionsetup[table]{font=small}
\begin{table}[H] 
  \scriptsize
  \setlength{\tabcolsep}{2.6pt}
  \renewcommand{\arraystretch}{0.95}
  \begin{tabular}{|l|c|c|c|c|c|c|c|c|c|c|c|c|c|c|c|c|}
  \toprule
   &  \multirow{2}{*}{\bf None} & \multicolumn{6}{c|}{\multirow{2}{*}{\bf GradPrune ($p$)}} & \multicolumn{2}{c|}{\multirow{2}{*}{\bf MixUp ($k$)}} & \multicolumn{2}{c|}{\multirow{2}{*}{\bf Intra-InstaHide ($k$)}} & \multicolumn{2}{c|}{\bf GradPrune ($p=0.9$)}\\
   & & \multicolumn{6}{c|}{} & \multicolumn{2}{c|}{} & \multicolumn{2}{c|}{} & {\bf  + MixUp } & {\bf  + Intra-InstaHide}\\
  \midrule
   {\bf Parameter}  & - & 0.5 & 0.7 & 0.9 & 0.95 & 0.99 & 0.999 & 4 & 6 & 4 & 6 & $k=4$ & $k=4$ \\
   \midrule
   {\bf Test Acc.} & 93.37 & 93.19 & 93.01 & 90.57 & 89.92 & 88.61 & 83.58 &  92.31 & 90.41 & 90.04 & 88.20 & 91.37 & 86.10 \\
   \midrule
  {\bf Time (train)} & $1\times$ & \multicolumn{6}{c|}{$1.04\times$} & \multicolumn{2}{c|}{$1.06\times$} & \multicolumn{2}{c|}{$1.06\times$} & \multicolumn{2}{c|}{$1.10\times$} \\
  \midrule
  \multicolumn{14}{|c|}{\bf Attack batch size $= 16$, the strongest attack} \\
  \midrule
  {\bf Avg. LPIPS $\downarrow$}  & 0.41  & 0.41  & 0.42  & 0.46  & 0.48  & 0.50  & 0.55         & 0.50  & 0.49  & 0.69  & 0.69  & 0.62  & \best{0.73}\\
  {\bf Best LPIPS $\downarrow$}  & 0.21  & 0.22  & 0.27  & 0.29  & 0.30  & 0.29  & 0.48         & 0.31  & 0.28  & 0.56  & 0.56  & 0.37  & \best{0.65}\\
  {(LPIPS std.)}                 & 0.09  & 0.08  & 0.07  & 0.06  & 0.06  & 0.06  & 0.04         & 0.10  & 0.10  & 0.06  & 0.07  & 0.10  & 0.05\\
  \midrule
   \multicolumn{14}{|c|}{\bf Attack batch size $= 16$, attacker knows private labels but does not know BatchNorm statistics} \\
   \midrule
   {\bf Avg. LPIPS $\downarrow$}  & 0.49 & 0.51 & 0.48 & 0.51 & 0.52 & 0.56 & 0.60 & 0.71 & 0.71 & \best{0.75} & \best{0.75} & 0.74 &  0.74\\
   {\bf Best LPIPS $\downarrow$}  & 0.30 & 0.33 & 0.31 & 0.33 & 0.34 & 0.39 & 0.44 & 0.48 & 0.53 & \best{0.65} & 0.63 & 0.61 &  0.63\\
   {(LPIPS std.)}                 & 0.08 & 0.09 & 0.08 & 0.08 & 0.07 & 0.07 & 0.05 & 0.08 & 0.07 & 0.04 & 0.05 & 0.08 &  0.05\\
   \midrule
   \multicolumn{14}{|c|}{\bf Attack batch size $= 32$, the strongest attack} \\
  \midrule
  {\bf Avg. LPIPS $\downarrow$}  & 0.45  & 0.46  & 0.48  & 0.52  & 0.54  & 0.58  & 0.63         & 0.50  & 0.49  & 0.69  & 0.69  & 0.62  & \best{0.73}\\
   {\bf Best LPIPS $\downarrow$}  & 0.18  & 0.18  & 0.22  & 0.31  & 0.43  & 0.48  & 0.54         & 0.31  & 0.28  & 0.56  & 0.56  & 0.37  & \best{0.65}\\
   {(LPIPS std.)}                 & 0.11  & 0.11  & 0.09  & 0.07  & 0.05  & 0.04  & 0.04         & 0.10  & 0.10  & 0.06  & 0.07  & 0.10  & 0.05\\
    \midrule
   \multicolumn{14}{|c|}{\bf Attack batch size $= 32$, attacker knows private labels but does not know BatchNorm statistics} \\
   \midrule
   {\bf Avg. LPIPS $\downarrow$}  & 0.48 & 0.50 & 0.53 & 0.53 & 0.55 & 0.60 & 0.63 & 0.73 & 0.72 & 0.76 & 0.76 & 0.76 & \best{0.77} \\
   {\bf Best LPIPS $\downarrow$}  & 0.29 & 0.32 & 0.32 & 0.31 & 0.40 & 0.41 & 0.55 & 0.63 & 0.60 & \best{0.68} & 0.63 & 0.66 & 0.65\\
   {(LPIPS std.)}                 & 0.08 & 0.07 & 0.07 & 0.08 & 0.08 & 0.06 & 0.04 & 0.06 & 0.06 & 0.04 & 0.05 & 0.06 & 0.05\\
  \bottomrule
  \end{tabular}
  \vspace{2mm}
%   \subfloat{\includegraphics[width=0.98\textwidth]{imgs/Compare_16_32.png}}
  \caption{\small Utility-security trade-off of different defenses for a more realistic setting (when the attacker knows private labels but does not know BatchNorm statistics). We also present the results for the strongest attack from Table~\ref{tab:exp_summary} for comparison. We evaluate the attack on 50 CIFAR-10 images and report the LPIPS score ($\downarrow$: lower values suggest more privacy leakage).
  We mark the least-leakage defense measured by the metric in \best{green}.} 
  \label{tab:exp_summary_realistic}
\end{table}

\iffalse
\paragraph{Qualitative and quantitative results for private labels unknown.} Apart from the example in Figure~\ref{fig:reconstructed_labels} with batch size being 16, we provide another example for how unknown labels affect reconstruction quality in Figure~\ref{fig:assumption2_app}, with batch size being 32. We also provide quantitative measurements in Figure~\ref{tab:assumption2_app1} and~\ref{tab:assumption2_app2}.

\begin{figure}[H]
    \centering
    \subfloat[Reconstructions with and without private labels]{
    \includegraphics[width=0.95\textwidth]{imgs/assumptions/label_known_unknown_32.png}
    \label{fig:assumption2_app}
    }\\
    \subfloat[Batch size = 16]{
        \setlength{\tabcolsep}{4pt}
        \small
        \begin{tabular}[b]{|c|c|c|}
                \toprule
                  & {\bf Labels known} &  {\bf Labela unknown} \\
                \midrule
                %  {\bf Avg. PSNR $\uparrow$} & 12.45 & 12.01  \\
                %  {\bf Best PSNR $\uparrow$} & 17.42 & 14.85    \\
                 {\bf Avg. LPIPS $\downarrow$} & 0.44 & 0.58 \\
                 {\bf Best LPIPS $\downarrow$} & 0.25 & 0.32    \\
                \bottomrule
            \end{tabular}
        \label{tab:assumption2_app1}
        }
        \subfloat[Batch size = 32]{
        \setlength{\tabcolsep}{4pt}
        \small
        \begin{tabular}[b]{|c|c|c|}
                \toprule
                  & {\bf Labels known} &  {\bf Labels unknown} \\
                \midrule
                %  {\bf Avg. PSNR $\uparrow$} & 13.01 & 12.16 \\
                %  {\bf Best PSNR $\uparrow$} & 17.09 & 14.62    \\
                 {\bf Avg. LPIPS $\downarrow$} & 0.41 & 0.62 \\
                 {\bf Best LPIPS $\downarrow$} & 0.21 & 0.39    \\
                \bottomrule
            \end{tabular}
        \label{tab:assumption2_app2}
        }
    \caption{A reconstructed batch of 32 images with and without private labels known (a). We also provide quantitative measurements of reconstructions with batch size 16 (b) and 32 (c) ($\downarrow$: lower values suggest more leakage). The gradient inversion attack is weakened when private labels are not available.}
\end{figure}
\fi


% \iffalse
\subsection{More results for the strongest attack}

\paragraph{Full version of Figure~\ref{fig:vis_recon}.} Figure~\ref{fig:vis_recon_full} provides more examples for reconstructed images by the strongest attack under different defenses and batch sizes. 



\begin{figure}[H]
\captionsetup[subfigure]{font=small}
  \centering
  \vspace{-12mm}
  \subfloat[Batch size $=1$]{\includegraphics[width=\linewidth]{imgs/recon_vis_bs=1_BN_exact_small.png}}\\
  \vspace{-3mm}
  \subfloat[Batch size $=16$]{\includegraphics[width=\linewidth]{imgs/recon_vis_bs=16_BN_exact_small.png}}\\
  \vspace{-3mm}
  \subfloat[Batch size $=32$]{\includegraphics[width=\linewidth]{imgs/recon_vis_bs=32_BN_exact_small.png}}\\
  \vspace{-2mm}
  \caption{Reconstruction results under different defenses with batch size 1 (a), 16 (b) and 32 (c). Full version of Figure~\ref{fig:vis_recon}.}
  \label{fig:vis_recon_full}
  \vspace{-2mm}
\end{figure}


\paragraph{Results with MNIST dataset.} We’ve repeated our main evaluation of defenses and attacks (Table~\ref{tab:exp_summary}) on MNIST dataset~\citep{deng2012mnist} with a simple 6-layer ConvNet model. Note that the simple ConvNet does not contain BatchNorm layers. We evaluate the following defenses on the MNIST dataset with a 6-layer ConvNet architecture against the strongest attack (private labels known):

\begin{itemize}
    \item GradPrune (gradient pruning): gradient pruning sets gradients of small magnitudes to zero. We vary the pruning ratio $p$ in \{0.5, 0.7, 0.9, 0.95, 0.99, 0.999, 0.9999\}.
    \item MixUp: we vary $k$ in \{4,6\}, and set the upper bound of a single coefficient to 0.65 (coefficients sum to 1).
    \item Intra-InstaHide: we vary $k$ in \{4,6\}, and set the upper bound of a single coefficient to 0.65 (coefficients sum to 1). 
    \item A combination of GradPrune and MixUp/Intra-InstaHide.
\end{itemize}

We run the evaluation against the strongest attack and batch size 1 to estimate the upper bound of privacy leakage. Specifically, we assume the attacker knows private labels, as well as the indices of mixed images and mixing coefficients for MixUp and Intra-InstaHide. 

\begin{figure}[t]
    \centering
    \includegraphics[width=0.95\linewidth]{imgs/appendix/recon_vis_MNIST.png}
    \caption{Reconstruction results of MNIST digits under different defenses with the strongest atttack and batch size 1.}
    \label{fig:vis_recon_MNIST}
    \vspace{-5mm}
\end{figure}

For MNIST with a simple 6-layer ConvNet, defending the strongest attack with gradient pruning may require the pruning ratio $p\geq 0.9999$. MixUp with $k=4$ or $k=6$ are not sufficient to defend the gradient inversion attack. Combining MixUp ($k=4$) with gradient pruning ($p=0.99$) improves the defense, however, the reconstructed digits are still highly recognizable. Intra-InstaHide alone ($k=4$ or $k=6$) gives a bit better defending performance than MixUp and GradPrune. Combining InstaHide ($k=4$) with gradient pruning ($p=0.99$) further improves the defense and makes the reconstruction almost unrecognizable. 




\subsection{More results for encoding-based defenses}
We visualize the whole reconstructed dataset under MixUp and Intra-InstaHide defenses with different batch sizes in Figure~\ref{fig:encode_bs1}, \ref{fig:encode_bs16} and \ref{fig:encode_bs32}.  Sample results of the original and the reconstructed batches are provided in Figure~\ref{fig:mixup_vs_instahide}.

\begin{figure}[H]
    \centering
    \includegraphics[width=0.95\textwidth]{imgs/appendix/mixup_vs_instahide.png}
    \caption{Original and reconstructed batches of 16 images under MixUp and Intra-InstaHide defenses. We visualize both the original and the absolute images for the Intra-InstaHide defense. Intra-InstaHide makes pixel-wise matching harder.}
    \label{fig:mixup_vs_instahide}
    \vspace{-5mm}
\end{figure}

\begin{figure}[H]
\captionsetup[subfigure]{labelfont=scriptsize, textfont=tiny}
    \centering
    \subfloat[Original]{\includegraphics[width=0.23\textwidth]{imgs/decode_res/InstaHide/bs1_k4/originals.png}} \hspace{1mm}
    \subfloat[MixUp, $k$=4]{\includegraphics[width=0.23\textwidth]{imgs/decode_res/Mixup/bs1_k4/grad_decode.png}} \hspace{1mm}
    \subfloat[MixUp, $k$=6]{\includegraphics[width=0.23\textwidth]{imgs/decode_res/Mixup/bs1_k6/grad_decode.png}} \hspace{1mm}
    \subfloat[MixUp+GradPrune, $k$=4, $p$=0.9]{\includegraphics[width=0.23\textwidth]{imgs/decode_res/Mixup/bs1_k4_gradprune/grad_decode.png}}
    
    \subfloat[Original]{\includegraphics[width=0.23\textwidth]{imgs/decode_res/InstaHide/bs1_k4/originals.png}} \hspace{1mm}
    \subfloat[InstaHide, $k$=4]{\includegraphics[width=0.23\textwidth]{imgs/decode_res/InstaHide/bs1_k4/grad_decode.png}} \hspace{1mm}
    \subfloat[InstaHide, $k$=6]{\includegraphics[width=0.23\textwidth]{imgs/decode_res/InstaHide/bs1_k6/grad_decode.png}} \hspace{1mm}
    \subfloat[InstaHide+GradPrune, $k$=4, $p$=0.9]{\includegraphics[width=0.23\textwidth]{imgs/decode_res/InstaHide/bs1_k4_gradprune/grad_decode.png}}
    \caption{Reconstrcuted dataset under MixUp and Intra-InstaHide against the strongest attack (batch size is 1).}
    \label{fig:encode_bs1}
    \vspace{-10mm}
\end{figure}


\begin{figure}[H]
\captionsetup[subfigure]{labelfont=scriptsize, textfont=tiny}
    \centering
    \subfloat[Original]{\includegraphics[width=0.23\textwidth]{imgs/decode_res/InstaHide/bs1_k4/originals.png}} \hspace{1mm}
    \subfloat[MixUp, $k$=4]{\includegraphics[width=0.23\textwidth]{imgs/decode_res/Mixup/bs16_k4/grad_decode.png}} \hspace{1mm}
    \subfloat[MixUp, $k$=6]{\includegraphics[width=0.23\textwidth]{imgs/decode_res/Mixup/bs16_k6/grad_decode.png}} \hspace{1mm}
    \subfloat[MixUp+GradPrune, $k$=4, p=0.9]{\includegraphics[width=0.23\textwidth]{imgs/decode_res/Mixup/bs16_k4_gradprune/grad_decode.png}}

    
    \subfloat[Original]{\includegraphics[width=0.23\textwidth]{imgs/decode_res/InstaHide/bs1_k4/originals.png}} \hspace{1mm}
    \subfloat[InstaHide, $k$=4]{\includegraphics[width=0.23\textwidth]{imgs/decode_res/InstaHide/bs16_k4/grad_decode.png}} \hspace{1mm}
    \subfloat[InstaHide, $k$=6]{\includegraphics[width=0.23\textwidth]{imgs/decode_res/InstaHide/bs16_k6/grad_decode.png}} \hspace{1mm}
    \subfloat[InstaHide+GradPrune, $k$=4, $p$=0.9]{\includegraphics[width=0.23\textwidth]{imgs/decode_res/InstaHide/bs16_k4_gradprune/grad_decode.png}}
    \caption{Reconstrcuted dataset under MixUp and Intra-InstaHide against the strongest attack (batch size is 16).}
    \label{fig:encode_bs16}
\end{figure}



\begin{figure}[H]
\captionsetup[subfigure]{labelfont=scriptsize, textfont=tiny}
    \centering
    \subfloat[Original]{\includegraphics[width=0.23\textwidth]{imgs/decode_res/InstaHide/bs1_k4/originals.png}} \hspace{1mm}
    \subfloat[MixUp, $k$=4]{\includegraphics[width=0.23\textwidth]{imgs/decode_res/Mixup/bs32_k4/grad_decode.png}} \hspace{1mm}
    \subfloat[MixUp, $k$=6]{\includegraphics[width=0.23\textwidth]{imgs/decode_res/Mixup/bs32_k6/grad_decode.png}} \hspace{1mm}
    \subfloat[MixUp+GradPrune, $k$=4, $p$=0.9]{\includegraphics[width=0.23\textwidth]{imgs/decode_res/Mixup/bs32_k4_gradprune/grad_decode.png}}

    
    \subfloat[Original]{\includegraphics[width=0.23\textwidth]{imgs/decode_res/InstaHide/bs1_k4/originals.png}} \hspace{1mm}
    \subfloat[InstaHide, $k$=4]{\includegraphics[width=0.23\textwidth]{imgs/decode_res/InstaHide/bs32_k4/grad_decode.png}} \hspace{1mm}
    \subfloat[InstaHide, $k$=6]{\includegraphics[width=0.23\textwidth]{imgs/decode_res/InstaHide/bs32_k6/grad_decode.png}} \hspace{1mm}
    \subfloat[InstaHide+GradPrune, $k$=4, $p$=0.9]{\includegraphics[width=0.23\textwidth]{imgs/decode_res/InstaHide/bs32_k4_gradprune/grad_decode.png}}
    \caption{Reconstrcuted dataset under MixUp and Intra-InstaHide against the strongest attack (batch size is 32).}
    \label{fig:encode_bs32}
\end{figure}






We briefly recall the framework of statistical inference via empirical risk minimization.
Let $(\bbZ, \calZ)$ be a measurable space.
Let $Z \in \bbZ$ be a random element following some unknown distribution $\Prob$.
Consider a parametric family of distributions $\calP_\Theta := \{P_\theta: \theta \in \Theta \subset \reals^d\}$ which may or may not contain $\Prob$.
We are interested in finding the parameter $\theta_\star$ so that the model $P_{\theta_\star}$ best approximates the underlying distribution $\Prob$.
For this purpose, we choose a \emph{loss function} $\score$ and minimize the \emph{population risk} $\risk(\theta) := \Expect_{Z \sim \Prob}[\score(\theta; Z)]$.
Throughout this paper, we assume that
\begin{align*}
     \theta_\star = \argmin_{\theta \in \Theta} L(\theta)
\end{align*}
uniquely exists and satisfies $\theta_\star \in \text{int}(\Theta)$, $\nabla_\theta L(\theta_\star) = 0$, and $\nabla_\theta^2 L(\theta_\star) \succ 0$.

\myparagraph{Consistent loss function}
We focus on loss functions that are consistent in the following sense.

\begin{customasmp}{0}\label{asmp:proper_loss}
    When the model is \emph{well-specified}, i.e., there exists $\theta_0 \in \Theta$ such that $\Prob = P_{\theta_0}$, it holds that $\theta_0 = \theta_\star$.
    We say such a loss function is \emph{consistent}.
\end{customasmp}

In the statistics literature, such loss functions are known as proper scoring rules \citep{dawid2016scoring}.
We give below two popular choices of consistent loss functions.

\begin{example}[Maximum likelihood estimation]
    A widely used loss function in statistical machine learning is the negative log-likelihood $\score(\theta; z) := -\log{p_\theta(z)}$ where $p_\theta$ is the probability mass/density function for the discrete/continuous case.
    When $\Prob = P_{\theta_0}$ for some $\theta_0 \in \Theta$,
    we have $L(\theta) = \Expect[-\log{p_\theta(Z)}] = \kl(p_{\theta_0} \Vert p_\theta) - \Expect[\log{p_{\theta_0}(Z)}]$ where $\kl$ is the Kullback-Leibler divergence.
    As a result, $\theta_0 \in \argmin_{\theta \in \Theta} \kl(p_{\theta_0} \Vert p_\theta) = \argmin_{\theta \in \Theta} L(\theta)$.
    Moreover, if there is no $\theta$ such that $p_\theta \txtover{a.s.}{=} p_{\theta_0}$, then $\theta_0$ is the unique minimizer of $L$.
    We give in \Cref{tab:glms} a few examples from the class of generalized linear models (GLMs) proposed by \citet{nelder1972generalized}.
\end{example}

\begin{example}[Score matching estimation]
    Another important example appears in \emph{score matching} \citep{hyvarinen2005estimation}.
    Let $\bbZ = \reals^\tau$.
    Assume that $\Prob$ and $P_\theta$ have densities $p$ and $p_\theta$ w.r.t the Lebesgue measure, respectively.
    Let $p_\theta(z) = q_\theta(z) / \Lambda(\theta)$ where $\Lambda(\theta)$ is an unknown normalizing constant. We can choose the loss
    \begin{align*}
        \score(\theta; z) := \Delta_z \log{q_\theta(z)} + \frac12 \norm{\nabla_z \log{q_\theta(z)}}^2 + \text{const}.
    \end{align*}
    Here $\Delta_z := \sum_{k=1}^p \partial^2/\partial z_k^2$ is the Laplace operator.
    Since \cite[Thm.~1]{hyvarinen2005estimation}
    \begin{align*}
        L(\theta) = \frac12 \Expect\left[ \norm{\nabla_z q_\theta(z) - \nabla_z p(z)}^2 \right],
    \end{align*}
    we have, when $p = p_{\theta_0}$, that $\theta_0 \in \argmin_{\theta \in \Theta} L(\theta)$.
    In fact, when $q_\theta > 0$ and there is no $\theta$ such that $p_\theta \txtover{a.s.}{=} p_{\theta_0}$, the true parameter $\theta_0$ is the unique minimizer of $L$ \cite[Thm.~2]{hyvarinen2005estimation}.
\end{example}

\myparagraph{Empirical risk minimization}
Assume now that we have an i.i.d.~sample $\{Z_i\}_{i=1}^n$ from $\Prob$.
To learn the parameter $\theta_\star$ from the data, we minimize the empirical risk to obtain the \emph{empirical risk minimizer}
\begin{align*}
    \theta_n \in \argmin_{\theta \in \Theta} \left[ L_n(\theta) := \frac1n \sum_{i=1}^n \score(\theta; Z_i) \right].
\end{align*}
This applies to both maximum likelihood estimation and score matching estimation. 
In \Cref{sec:main_results}, we will prove that, with high probability, the estimator $\theta_n$ exists and is unique under a generalized self-concordance assumption.

\begin{figure}
    \centering
    \includegraphics[width=0.45\textwidth]{graphs/logistic-dikin} %0.4
    \caption{Dikin ellipsoid and Euclidean ball.}
    \label{fig:logistic_dikin}
\end{figure}

\myparagraph{Confidence set}
In statistical inference, it is of great interest to quantify the uncertainty in the estimator $\theta_n$.
In classical asymptotic theory, this is achieved by constructing an asymptotic confidence set.
We review here two commonly used ones, assuming the model is well-specified.
We start with the \emph{Wald confidence set}.
It holds that $n(\theta_n - \theta_\star)^\top H_n(\theta_n) (\theta_n - \theta_\star) \rightarrow_d \chi_d^2$, where $H_n(\theta) := \nabla^2 L_n(\theta)$.
Hence, one may consider a confidence set $\{\theta: n(\theta_n - \theta)^\top H_n(\theta_n) (\theta_n - \theta) \le q_{\chi_d^2}(\delta) \}$ where $q_{\chi_d^2}(\delta)$ is the upper $\delta$-quantile of $\chi_d^2$.
The other is the \emph{likelihood-ratio (LR) confidence set} constructed from the limit $2n [L_n(\theta_\star) - L_n(\theta_n)] \rightarrow_d \chi_d^2$, which is known as the Wilks' theorem \citep{wilks1938large}.
These confidence sets enjoy two merits: 1) their shapes are an ellipsoid (known as the \emph{Dikin ellipsoid}) which is adapted to the optimization landscape induced by the population risk; 2) they are asymptotically valid, i.e., their coverages are exactly $1 - \delta$ as $n \rightarrow \infty$.
However, due to their asymptotic nature, it is unclear how large $n$ should be in order for it to be valid.

Non-asymptotic theory usually focuses on developing finite-sample bounds for the \emph{excess risk}, i.e., $\Prob(L(\theta_n) - L(\theta_\star) \le C_n(\delta)) \ge 1 - \delta$.
To obtain a confidence set, one may assume that the population risk is twice continuously differentiable and $\lambda$-strongly convex.
Consequently, we have $\lambda \norm{\theta_n - \theta_\star}_2^2 / 2 \le L(\theta_n) - L(\theta_\star)$ and thus we can consider the confidence set $\calC_{\text{finite}, n}(\delta) := \{\theta: \norm{\theta_n - \theta}_2^2 \le 2C_n(\delta)/\lambda\}$.
Since it originates from a finite-sample bound, it is valid for fixed $n$, i.e., $\Prob(\theta_\star \in \calC_{\text{finite}, n}(\delta)) \ge 1 - \delta$ for all $n$; however, it is usually conservative, meaning that the coverage is strictly larger than $1 - \delta$.
Another drawback is that its shape is a Euclidean ball which remains the same no matter which loss function is chosen.
We illustrate this phenomenon in \Cref{fig:logistic_dikin}.
Note that a similar observation has also been made in the bandit literature \citep{faury2020improved}.

We are interested in developing finite-sample confidence sets.
However, instead of using excess risk bounds and strong convexity, we construct in \Cref{sec:main_results} the Wald and LR confidence sets in a non-asymptotic fashion, under a generalized self-concordance condition.
These confidence sets have the same shape as their asymptotic counterparts while maintaining validity for fixed $n$.
These new results are achieved by characterizing the critical sample size enough to enter the asymptotic regime.


% An example of a floating figure using the graphicx package.
% Note that \label must occur AFTER (or within) \caption.
% For figures, \caption should occur after the \includegraphics.
% Note that IEEEtran v1.7 and later has special internal code that
% is designed to preserve the operation of \label within \caption
% even when the captionsoff option is in effect. However, because
% of issues like this, it may be the safest practice to put all your
% \label just after \caption rather than within \caption{}.
%
% Reminder: the "draftcls" or "draftclsnofoot", not "draft", class
% option should be used if it is desired that the figures are to be
% displayed while in draft mode.
%
%\begin{figure}[!t]
%\centering
%\includegraphics[width=2.5in]{myfigure}
% where an .eps filename suffix will be assumed under latex,
% and a .pdf suffix will be assumed for pdflatex; or what has been declared
% via \DeclareGraphicsExtensions.
%\caption{Simulation Results}
%\label{fig_sim}
%\end{figure}

% Note that IEEE typically puts floats only at the top, even when this
% results in a large percentage of a column being occupied by floats.


% An example of a double column floating figure using two subfigures.
% (The subfig.sty package must be loaded for this to work.)
% The subfigure \label commands are set within each subfloat command, the
% \label for the overall figure must come after \caption.
% \hfil must be used as a separator to get equal spacing.
% The subfigure.sty package works much the same way, except \subfigure is
% used instead of \subfloat.
%
%\begin{figure*}[!t]
%\centerline{\subfloat[Case I]\includegraphics[width=2.5in]{subfigcase1}%
%\label{fig_first_case}}
%\hfil
%\subfloat[Case II]{\includegraphics[width=2.5in]{subfigcase2}%
%\label{fig_second_case}}}
%\caption{Simulation results}
%\label{fig_sim}
%\end{figure*}
%
% Note that often IEEE papers with subfigures do not employ subfigure
% captions (using the optional argument to \subfloat), but instead will
% reference/describe all of them (a), (b), etc., within the main caption.


% An example of a floating table. Note that, for IEEE style tables, the
% \caption command should come BEFORE the table. Table text will default to
% \footnotesize as IEEE normally uses this smaller font for tables.
% The \label must come after \caption as always.
%
%\begin{table}[!t]
%% increase table row spacing, adjust to taste
%\renewcommand{\arraystretch}{1.3}
% if using array.sty, it might be a good idea to tweak the value of
% \extrarowheight as needed to properly center the text within the cells
%\caption{An Example of a Table}
%\label{table_example}
%\centering
%% Some packages, such as MDW tools, offer better commands for making tables
%% than the plain LaTeX2e tabular whiIEEEtranch is used here.
%\begin{tabular}{|c||c|}
%\hline
%One & Two\\
%\hline
%Three & Four\\
%\hline
%\end{tabular}
%\end{table}


% Note that IEEE does not put floats in the very first column - or typically
% anywhere on the first page for that matter. Also, in-text middle ("here")
% positioning is not used. Most IEEE journals use top floats exclusively.
% Note that, LaTeX2e, unlike IEEE journals, places footnotes above bottom
% floats. This can be corrected via the \fnbelowfloat command of the
% stfloats package.



%\section{Conclusion and Future Work}
%One interesting direction is to identify analytical solutions or approximate solutions to the degree distribution optimization problem, and compare with the theoretical results.

%The authors would like to thank...


% Can use something like this to put references on a page
% by themselves when using endfloat and the captionsoff option.
\ifCLASSOPTIONcaptionsoff
  \newpage
\fi



% trigger a \newpage just before the given reference
% number - used to balance the columns on the last page
% adjust value as needed - may need to be readjusted if
% the document is modified later
%\IEEEtriggeratref{8}
% The "triggered" command can be changed if desired:
%\IEEEtriggercmd{\enlargethispage{-5in}}




% references section

% can use a bibliography generated by BibTeX as a .bbl file
% BibTeX documentation can be easily obtained at:
% http://www.ctan.org/tex-archive/biblio/bibtex/contrib/doc/
% The IEEEtran BibTeX style support page is at:
% http://www.michaelshell.org/tex/ieeetran/bibtex/
%\bibliographystyle{IEEEtranTCOM}
% argument is your BibTeX string definitions and bibliography database(s)
%\bibliography{IEEEabrv,../bib/paper}
%
% <OR> manually copy in the resultant .bbl file
% set second argument of \begin to the number of references
% (used to reserve space for the reference number labels box)
%
%\begin{thebibliography}{1}
%
%\bibitem{IEEEhowto:kopka}
%H.~Kopka and P.~W. Daly, \emph{A Guide to \LaTeX}, 3rd~ed.\hskip 1em plus
%  0.5em minus 0.4em\relax Harlow, England: Addison-Wesley, 1999.
%
%\end{thebibliography}


%\vspace{-2em}
\vspace{-1em}

\bibliographystyle{IEEEtran}
%\bibliographystyle{IEEEtranTCOM}
%\bibliographystyle{hieeetr}
%\bibliography{IEEEabrv,itw2013emina}
\bibliography{IEEEabrv,itw2013emina2}

% biography section
%
% If you have an EPS/PDF photo (graphicx package needed) extra braces are
% needed around the contents of the optional argument to biography to prevent
% the LaTeX parser from getting confused when it sees the complicated
% \includegraphics command within an optional argument. (You could create
% your own custom macro containing the \includegraphics command to make things
% simpler here.)
%\begin{biography}[{\includegraphics[width=1in,height=1.25in,clip,keepaspectratio]{mshell}}]{Michael Shell}
% or if you just want to reserve a space for a photo:

%\begin{IEEEbiographynophoto} {Louis Tan}
%	received his BASc degree in Engineering Science with a major in Computer Engineering from the University of Toronto, Toronto, ON, Canada in 2010, and his MASc degree in Electrical Engineering, also from the University of Toronto, in 2012.  He is currently working towards his PhD degree in the Electrical and Computer Engineering Department at the University of Toronto.  His research interests include joint source-channel coding, multimedia systems, machine learning, and network information theory.
%\end{IEEEbiographynophoto}

%\begin{IEEEbiographynophoto} {Yao Li}
%received her B.Eng.\ degree in Information Engineering and Electronics from Tsinghua University, Beijing, China, and her Ph.D degree in Electrical and Computer Engineering from Rutgers, the State University of New Jersey, in 2012. She was a postdoctoral scholar at the Laboratory for Robust Information Systems (LORIS) at University of California, Los Angeles, between 2012 and 2014. She is now with Akamai Technologies, Inc. Her research interests include coding solutions for content distribution, robust inference in sensor networks, and computing on noisy hardware.
%\end{IEEEbiographynophoto}

%\begin{IEEEbiographynophoto} {Ashish Khisti}
%received his BASc Degree (2002) in Engineering Sciences (Electrical Option) from University of Toronto, and his S.M and Ph.D. Degrees in Electrical Engineering from the Massachusetts Institute of Technology. Between 2009-2015, he was an assistant professor in the Electrical and Computer Engineering department at the University of Toronto. He is presently an associate professor, and holds a Canada Research Chair in the same department. He is a recipient of an Ontario Early Researcher Award, the Hewlett-Packard Innovation Research Award and the Harold H. Hazen teaching assistant award from MIT. He presently serves as an associate editor for IEEE Transactions on Information Theory and is also a guest editor for the Proceedings of the IEEE (Special Issue on Secure Communications via Physical-Layer and Information-Theoretic Techniques). 
%\end{IEEEbiographynophoto}
%
%\begin{IEEEbiographynophoto} {Emina Soljanin}
%is a Professor at Rutgers University. Before joining Rutgers in January 2016, she worked as the Distinguished Member of Technical Staff at Bell Labs, doing research in information, coding, and, more recently, queueing theory. Her interests and expertise are wide. Over the past quarter of the century, she has participated in numerous research and business projects, as diverse as power system optimization, magnetic recording, color space quantization, hybrid ARQ, network coding, data and network security, and quantum information theory and networking. Prof. Soljanin served as the Associate Editor for Coding Techniques, for the IEEE Transactions on Information Theory, on the Information Theory Society Board of Governors, and in various roles on other journal editorial boards and conference program committees. She is a co-organizer of the DIMACS 2001-2005 Special Focus on Computational Information Theory and Coding and 2011-2015 Special Focus on Cybersecurity. Prof. Soljanin is an IEEE Fellow, and serves a distinguished lecturer for the IEEE Information Theory Society in 2015 and 2016. 
%\end{IEEEbiographynophoto}

% You can push biographies down or up by placing
% a \vfill before or after them. The appropriate
% use of \vfill depends on what kind of text is
% on the last page and whether or not the columns
% are being equalized.

\vfill

% Can be used to pull up biographies so that the bottom of the last one
% is flush with the other column.
%\enlargethispage{-5in}



% that's all folks
\end{document}


