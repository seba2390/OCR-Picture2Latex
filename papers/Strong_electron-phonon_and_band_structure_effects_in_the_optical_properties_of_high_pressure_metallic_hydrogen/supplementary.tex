\documentclass[11pt,titlepage,a4paper,twoside]{article}
\usepackage[hmarginratio=1:1]{geometry}
\usepackage{grffile}



\usepackage[USenglish]{babel}
\usepackage[utf8x]{inputenc}
\usepackage[numbers,comma,sort&compress]{natbib}
\usepackage{ucs}



%%%%%%%%%%%%%%%%%%
%% Page Headers %%
%%%%%%%%%%%%%%%%%%



% \usepackage{fancyhdr}
% \pagestyle{fancy}
% \renewcommand{\sectionmark}[1]{\markright{\thesection\ #1}}
% \lhead[\rmfamily \thepage]
%     {\fancyplain{}{\slshape \let\uppercase\relax\rightmark}}
% \rhead[\fancyplain{}{\slshape \let\uppercase\relax\leftmark}]
%     {\rmfamily\thepage}
% \cfoot{}
% \renewcommand{\headrulewidth}{0.5pt}





\usepackage{url}
\usepackage{latexsym}
\usepackage{amsmath}
\usepackage{graphicx}
%\usepackage[pdftex=true,colorlinks=true,plainpages=false]{}
\usepackage{doi}
\usepackage{authblk}
\usepackage[]{hyperref}
\hypersetup{colorlinks=true,urlcolor=blue,citecolor=blue,linkcolor=blue}
\usepackage{floatrow}
\usepackage{braket}
\usepackage{float}
\usepackage{multirow}
\usepackage{amsfonts}
\usepackage{bm}
\usepackage{subfig}
\newcommand{\footnoteremember}[2]{
  \footnote{#2}
  \newcounter{#1}
  \setcounter{#1}{\value{footnote}}
} \newcommand{\footnoterecall}[1]{
  \footnotemark[\value{#1}]
} 
\usepackage{epigraph}
\usepackage[font=footnotesize]{caption}

\floatsetup[table]{font=footnotesize}

\newcommand{\onlinecite}[1]{\hspace{-1 ex} \nocite{#1}\citenum{#1}} 

% \setlength{\parskip}{3mm}
% \setlength{\parindent}{0pt}
% \setlength{\headheight}{26pt}

\begin{document}
\selectlanguage{USenglish}

%%%%%%%%%%%%%%%%%%%
%%     COVER     %%  
%%%%%%%%%%%%%%%%%%%


%\pagestyle{fancy}

\title{ \textbf{Supplementary Material} \\~ \\ Strong electron-phonon and band structure effects in the optical properties of high pressure metallic hydrogen}

\author[1,2]{Miguel Borinaga}
\author[3]{Julen Iba\~nez-Azpiroz}
\author[1,2,4]{Aitor Bergara}
\author[2,5]{Ion Errea}
\affil[1]{Centro de F\'isica de Materiales CFM, CSIC-UPV/EHU, Paseo Manuel de
             Lardizabal 5, 20018 Donostia/San Sebasti\'an, Basque Country, Spain}
\affil[2]{Donostia International Physics Center
             (DIPC), Manuel Lardizabal pasealekua 4, 20018 Donostia/San
             Sebasti\'an, Basque Country, Spain}
\affil[3]{Peter Gr{\"u}nberg Institute and Institute for Advanced Simulation, 
             Forschungszentrum J{\"u}lich \& JARA, D-52425 J{\"u}lich, Germany}             
\affil[4]{Departamento de F\'isica de la Materia Condensada,  University of the Basque Country (UPV/EHU), 48080 Bilbao, 
             Basque Country, Spain}
\affil[5]{Fisika Aplikatua 1 Saila, Bilboko Ingeniaritza Eskola,
             University of the Basque Country (UPV/EHU), Rafael Moreno ``Pitxitxi'' Pasealekua 3, 48013 Bilbao,
             Basque Country, Spain}

\renewcommand\Authands{ and }
\renewcommand*{\Affilfont}{\small\itshape}

\date{\vspace{-5ex}}
\maketitle

\section{Calculation methods and procedure}

\begin{figure}[t]
\subfloat[a][]{\includegraphics[width=0.49\linewidth]{imepsilon}\label{imepsilon}}\hfill% Here is how to import EPS art
%\subfloat[b][]{\includegraphics[height=0.33\linewidth]{transitions_on}\label{bands}}
\subfloat[b][]{\includegraphics[width=0.49\linewidth]{reepsilon}\label{reepsilon}}
\caption{
(a) Different contributions to the imaginary part of $\varepsilon$ of of $\mathrm{I4_1/amd}$ hydrogen at $500$ GPa. The inset shows a zoom into lower energies.
(b) Real part of $\varepsilon$ calculated using the Kramers-Kronig relations.
\label{fig1}}
\end{figure}

The central quantity addressed in this work is the frequency dependent reflectivity,
which for normal incident light in a medium with refractive index $n$ can be written as 
\begin{equation}\label{reflec}
 R(\omega)=\left|\frac{\sqrt{\varepsilon(\omega)}-n}{\sqrt{\varepsilon(\omega)}+n}\right|^2,
\end{equation}
where the relative dielectric function $\varepsilon(\omega)$ can be expressed in terms of the 
optical conductivity $\sigma(\omega)$ as
\begin{equation}\label{eq:dielectric}
 \varepsilon(\omega)=1+i\frac{4\pi\sigma(\omega)}{\omega}.
\end{equation}
The optical conductivity of a metal can be described as 
\begin{equation}
\sigma(\omega)=\sigma_{intra}(\omega)+\sigma_{inter}(\omega)+\sigma_{phonons}(\omega),
\label{opcond}
\end{equation}
where $\sigma_{intra}$ and $\sigma_{inter}$ account, respectively,
for the optical conductivity provided by electronic intraband and interband
transitions, while $\sigma_{phonons}$ accounts for the direct phonon
absorption contribution.
As $\mathrm{I4_1/amd}$ hydrogen lacks of IR active vibrational 
modes~\cite{PhysRevB.93.174308}, we set 
$\sigma_{phonons}=0.$ 

\begin{figure}[t]
 \includegraphics[width=0.7\linewidth]{tauwide}\hspace{0.1cm}
 \caption{\label{tauwide} Drude model frequency-dependent plasma frequency $\omega_p(\omega)$ of $\mathrm{I4_1/amd}$ hydrogen at $500$ GPa for different impurity scattering rates and temperatures. The same curve is also obtained for the case in which the ME formalism is
 not considered (TDDFT only). The inset shows the frequency-dependent impurity scattering rate $\tau^{-1}(\omega)$.
}
\end{figure}

The interband and intraband contributions are computed in two stages. 
We first calculate
the dielectric function within time-dependent DFT~\cite{PhysRevLett.52.997,PhysRevLett.76.1212} (TDDFT),
which realistically incorporates the actual electronic structure
into the dielectric function. 
The dielectric function is calculated by employing  
an interpolation scheme~\cite{PhysRevB.85.054305,PhysRevB.86.085106,PhysRevB.89.085102} of both the Kohn-Sham states
and the matrix elements with the use of maximally localized Wannier functions~\cite{PhysRevB.56.12847,marzari2012}.
The method allows a very fine sampling of the reciprocal space.
In order to avoid numerical problems, a finite but small momentum
is taken for the calculation of the dielectric function. The obtained optical conductivity
from the TDDFT calculation thus contains both interband and intraband
contributions: $\sigma^{TDDFT}(\omega) = \sigma_{intra}^{TDDFT}(\omega)
+ \sigma_{inter}^{TDDFT}(\omega)$.
In order to incorporate
the fine features of the band structure,
we set $\sigma_{inter}(\omega)=\sigma_{inter}^{TDDFT}(\omega)$ in Eq. \eqref{opcond},
which provides a fine description of the reflectivity at high energies.
The low-energy intraband contribution given by
$\sigma_{intra}^{TDDFT}(\omega)$ is affected by the choice
of a finite momentum and completely neglects how
an excited electron can decay 
due to the electron-phonon interaction. Moreover, this regime can also be 
strongly affected in superconductors due to the presence of the 
superconducting gap~\cite{PhysRevLett.59.1958,Rotter1992,Capitani2016}.
In order to incorporate these effects into the reflectivity, the intraband
contribution to the optical conductivity is calculated instead by
solving the isotropic
Migdal-Eliashberg (ME) equations. We thus make
$\sigma_{intra}(\omega)=\sigma^{ME}(\omega)$,
where $\sigma^{ME}(\omega)$ is the optical conductivity obtained
solving ME equations. 
Following Ref.~\cite{PhysRevB.42.67},
these equations are solved 
in the imaginary axis using a Pad\'e approximant for the analytic continuation 
to the real frequency axis.
%The electron-phonon spectral density $\alpha^2F(\omega)$ needed to solve the
%ME equation is 
%calculated within density-functional perturbation theory (DFPT) 
%in the harmonic approach, considering that 
%anharmonicity barely affects it~\cite{PhysRevB.93.174308}. 
The free-electron density used in the 
ME equations is determined so that the partial f-sum rule integral
\begin{equation}
 \int_{0}^{\infty}Re~\sigma^{ME}(\omega)d\omega= \int_{0}^{\infty}Re~\sigma_{intra}^{TDDFT}(\omega)d\omega
\label{fsum}
\end{equation}
is satisfied by $\sigma^{ME}(\omega)$. This guarantees that 
the total optical conductivity in Eq. \eqref{opcond} 
satisfies the f-sum rule and yields the correct electron density.
This procedure can be performed because the interband and intraband
contributions in the TDDFT optical conductivities (real part) are
clearly separated as it can be seen in Fig. \ref{imepsilon} for Im $\varepsilon(\omega)=\omega\mathrm{Re~}\sigma(\omega)/4\pi$. The real part of $\epsilon$ (Fig. \ref{reepsilon})
is obtained afterwards using the Kramers-Kronig relations.   
Performing the f-sum rule integral (Eq. \eqref{fsum}) for each contribution in the TDDFT calculation
we obtain $\omega_p^{intra}=\sqrt{4 \pi N_{intra}}=$ 22.6 eV and $\omega_p^{inter}=\sqrt{4 \pi N_{intra}}=$ 26.3 eV, with
$\omega_p^=\sqrt{4 \pi N}=\sqrt{4 \pi (N_{intra}+N_{inter})}=$ 35.0 eV, with $N_{intra}$ and $N_{inter}$ being the electronic density contributing to the intraband
and interband processes respectively. 


\begin{figure}[t]
 \includegraphics[width=0.70\linewidth]{pressure}\hspace{0.1cm}
 \caption{\label{pressure} Reflectivity of a $\mathrm{I4_1/amd}$ hydrogen/diamond interface at 400,500 and 600 GPa at 5 K and $\tau_{imp}^{-1}=$ 200 meV in the superconducting state.
}
\end{figure}


All ground state DFT calculations are performed
at 400, 500 and 600 GPa, where metallic hydrogen 
is predicted to adopt the $\mathrm{I4_1/amd}$ 
crystal structure~\cite{PhysRevLett.112.165501,PhysRevLett.106.165302}.
We employ the QUANTUM-ESPRESSO package~\cite{0953-8984-21-39-395502}, 
using a plane-wave energy cutoff of 100 Ry and 
an ultrasoft pseudopotential\cite{PhysRevB.41.7892} 
with the Perdew-Wang parametrization of the 
local-density-approximation\cite{PhysRevB.45.13244} for the exchange and correlation potential. 
The wannierization process includes the 40 lowest-lying bands and is performed 
using the WANNIER90 package~\cite{Mostofi20142309}.
It allows us to interpolate the original $20\times20\times20$ \textbf{k}-space mesh
into a fine $60\times60\times60$ mesh for the calculation of the TDDFT 
dielectric function. In the latter crystal local field effects are taken into account by the use of 
two reciprocal lattice shells~\cite{PhysRevB.85.054305,PhysRevB.89.085102}.
The Eliashberg function $\alpha^2F(\omega)$ function needed to solve the
ME equations is calculated as described in Ref. \cite{PhysRevB.93.174308},
but with the use of the same exchange and correlation potential as
for the TDDFT calculation. $\alpha^2F(\omega)$ is calculated
at the harmonic level as anharmonicity barely affects it~\cite{PhysRevB.93.174308}. ME equations were solved
using a Matsubara energy cutoff of 6 times the highest phonon frequency, and the same as for computing the Padé approximant. We
solved the ME equations assuming different $\tau_{imp}^{-1}$ impurity scattering rates as the latter is introduced as a parameter in the equations. 


\begin{figure}[t]
 \includegraphics[width=0.70\linewidth]{fitting}\hspace{0.1cm}
 \caption{\label{fitting} Experimental data at 5 K \cite{Diaseaal1579} fitted with the Drude model (only the two lowest energy data points are fitted). Solid and dashed black curves show our two different fitting results, while the green curve shows the fitting by
 Dias \textit{et al.}\cite{Dias2017}. Red and blue curves show the curves obtained by fitting only $\tau^{-1}$ for $\omega_p$ fixed at 20 and 40 eV respectively. 
}
\end{figure}

\section{Drude model}

% \begin{figure}[t]
%  \includegraphics[width=0.70\linewidth]{taulorentz}\hspace{0.1cm}
%  \caption{\label{taulorentz} Drude model frequency-dependent plasma frequency $\omega_p(\omega)$ of $\mathrm{I4_1/amd}$ hydrogen at $500$ GPa for different impurity scattering rates and temperatures. The same curve is also obtained for the case in which the ME formalism is
%  not considered (TDDFT only). The inset shows the frequency-dependent impurity scattering rate $\tau^{-1}(\omega)$.
% }
% \end{figure}

In Ref. [\onlinecite{Diaseaal1579}] the authors fitted the experimental data with the reflectivity formula following the Drude model. According to this model,
\begin{equation}
\varepsilon(\omega)=1-  \frac{\omega_p^2 \tau}{\omega}\frac{1}{i+\omega\tau}.
\end{equation}

While often the plasma frequency $\omega_p$ and the mean scattering time $\tau$ are defined as fixed parameters, it can also be useful to generalize this formula by making them frequency dependent. This way, for a known
$\varepsilon(\omega)$, one can define $\omega_p(\omega)$ and $\tau(\omega)$ as

\begin{equation}
 \frac{1}{\tau(\omega)}=\frac{\omega \mathrm{Im}~\varepsilon(\omega)}{1-\mathrm{Re}~\varepsilon(\omega)}
\end{equation}

\begin{equation}
\omega_p^2(\omega)=\omega\tau(\omega)\left(\omega^2+\frac{1}{\tau(\omega)^2}\right)\mathrm{Im}~\varepsilon(\omega).
\end{equation}




We have calculated $\tau^{-1}(\omega)$ and $\omega_p(\omega)\equiv+\sqrt{|\omega_p^2(\omega)|}$ using $\varepsilon(\omega)$ values calculated for different impurity scattering rates and temperatures, as well as only considering TDDFT and therefore neglecting
phonon and impurity scattering. In Fig. \ref{tauwide} we can see that for frequencies larger than 15 eV, all $\omega_p$ curves converge in a plateau close to the theoretical $\omega_p=\sqrt{4\pi N}=$ 35.0 eV value, while
$\tau^{-1}$ yields around 7 eV, regardless of whether or not including phonon and impurity scattering. This shows electronic scattering is dominated by electronic band structure effects in this high frequency regime.
For frequencies lower than 5 eV but larger than 0.5 eV approximately, $\omega_p$ yields values around 21 eV, close to the $\omega_p=\sqrt{4\pi N_{intra}}=$ 22.6 eV value, which is expected as the interband transitions occur at higher energies and therefore their corresponding electronic density 
does not contribute to $\varepsilon$. In this low energy regime, $\tau^{-1}$ ranges 0.7-1.5 eV when phonon and impurity scattering is included in the calculation even if for the TDDFT only calculation yields negligible values. This shows phonon and impurities
clearly govern electronic scattering for the photon energies the experiment was performed at.  
In the 5-15 eV interband plasmon region, $\omega_p$ and $\tau^{-1}$ take unrealistic
(even negative for $\tau^{-1}$) values. This is because the Drude formula is not adequate for modeling the optical conductivity close to interband excitations. In order to take into account the interband transitions and their contribution to
$\varepsilon$, one should add a Lorentz Oscillator. The same holds for the very small energy region ($\omega<$ 0.5 eV), as the $\omega$ dependence of the electron-phonon and impurity scattering contribution to $\varepsilon$ is more complex
than the one modeled by the Drude formula.




% In order to check the validity of using the Drude model to analyze the optical properties of $\mathrm{I4_1/amd}$ hydrogen we have also calculated $\tau^{-1}$ for 
% \begin{equation}
% \varepsilon(\omega)=1+i \frac{(\omega_p^{intra})^2 \tau(\omega)}{\omega-i\omega^2\tau(\omega)}+i \frac{(\omega_p^{inter})^2 \tau(\omega)}{\omega-i(\omega^2-\omega_0^2)\tau(\omega)},
% \end{equation}
% which includes a Lorentz oscillator to take into account the electronic interband transitions of the system. We have set $\omega_0=$ 8.2 eV, $\omega_p^{intra}=$ 22.6 eV and $\omega_p^{inter}=$ 26.3 eV.
% It is interesting to analyze both the high and the low energy limits of this model. For $\omega>>\omega_0$ the Lorentz term behaves
% as the Drude one, yielding a total Drude term with $\omega_p^2=(\omega_p^{intra})^2+(\omega_p^{inter})^2=(35.0~ \mathrm{eV})^2$. This is the reason why within the generalized Drude model (Fig.\ref{tauwide}) we get $\omega_p(\omega)\sim$ 35 eV values
% at $\omega>$ 15 eV. For $\omega<<\omega_0$ the Lorentz term goes to zero and we are left just with the Drude term and therefore
% $\omega_p=\omega_p^{intra}=$ 22.6 eV, close to our $\sim$ 21 eV values using the Drude model.   
% In Fig. \ref{taulorentz} we compare our combined Drude-Lorentz $\tau^{-1}$ for $\tau_{imp}^{-1}=$ 0 meV at 5 K with the ones where only the Drude model was used.
% As we can see for high energies $\tau^{-1}$ curves converge showing
% the Drude model is valid to analyze this material close to its free-electron plasma frequency. In the low energy limit a similar conclusion can be extracted and the Drude model yields reasonable results if instead of the total density of the system
% one uses the one corresponding to the intraband transitions only. However, the quantitative differences of the $\tau^{-1}$ values at $\omega\sim$ 0.7-3 eV
% shows the proximity of the interband plasmon does modify the optical spectrum at lower energies and therefore 
% the $\omega<<\omega_0$ condition is not completely fulfilled in the experimental frequency range.   



\section{Fitting of experimental data}

\begin{table}[t]
\begin{tabular}{cccc}
\hline
\hline
Pressure (GPa)    &        400  &    500   &  600 \\
\hline
$\Delta_0$ (meV)  &       64.1  &   61.0   & 67.1 \\
$\omega_0$ (eV)   &        7.6  &    8.2   &  8.2 \\
$\omega_{inter}$ (eV)& 5.5  &    6.5   &  6.6 \\
$\omega_p$  (eV)   &       34.2  &   35.7   & 36.8 \\
\hline
\hline
\end{tabular}
\caption{Superconducting electronic bandgap ($\Delta_0$), interband absorption peak position ($\omega_0$), interband plasmon peak position ($\omega_{inter}$) and total plasma frequency ($\omega_p$) of $\mathrm{I4_1/amd}$ hydrogen at different pressures.
\label{changes}}
\end{table}

In Ref. [\onlinecite{Diaseaal1579}] the authors fitted four experimental data points for each temperature (5 and 83 K), where reflectivity was corrected for taking into account diamond absorption. However, later they claimed this correction procedure
may not be valid and thus only considered the lowest two energy raw (not corrected) data points\cite{Dias2017}, obtaining similar values for $\omega_p$ and $\tau^{-1}$ at 5 K. However, fitting a non-linear formula consisting of two parameters
to only two experimental points results in a non-unique solution for the fitting parameters. By fitting the first two experimental data points at 5 K to the reflectivity formula we have obtained at least two different results: $\omega_p=$ 
60.91 eV and 1.98 eV and with $\tau^{-1}=$ 2.21 eV and 0.085 eV respectively (see Fig. \ref{fitting}). None of the fitted $\omega_p$ values are reasonable, either for too high or too low. 
It is also important to notice we do not obtain the same fitting parameters as in Ref. [\onlinecite{Dias2017}] ($\omega_p=$ 33.2 eV and $\tau^{-1}=$ 1.1 eV). However,if we set $\omega_p=$ 33.2 eV and fit only $\tau^{-1}$ we obtain the same 1.1 eV as 
in Ref. [\onlinecite{Dias2017}].
We have seen that for $\omega_p$ values fixed within 20-40 eV (values close to what is expected for a nearly-free-electron-like metal at these densities) and fitted only $\tau^{-1}$ to the
experimental data, the latter only oscillates within 0.64-1.36 eV yielding a good fitting to the experiments within the error bars.   

\section{Pressure dependence}



Due to the doubts on the pressure calibration~\cite{Eremets2017,Loubeyre2017,Liu2017,Goncharov2017} in Ref. [\onlinecite{Diaseaal1579}] we have calculated the reflectivity of a $\mathrm{I4_1/amd}$ hydrogen/diamond interface at 400 and 600 GPa as well at 5 K and $\tau_{imp}^{-1}=$ 200 meV in the superconducting
state. As we can see in Fig. \ref{pressure}, differences are only quantitative, as the qualitative nature of the curves does not change. The minor quantitative changes of the main parameters are summarized in Table \ref{changes}. 




 
\bibliography{bibliografia}
\bibliographystyle{apsrev4-1}

\end{document}
