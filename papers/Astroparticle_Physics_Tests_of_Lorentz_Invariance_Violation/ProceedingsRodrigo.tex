\documentclass[a4paper]{jpconf}
\usepackage{graphicx}
\usepackage{amsmath}
\newcommand{\lagr}{\mathcal{L}}
\newcommand{\elinha}{\epsilon^{\prime}}
\newcommand{\dpi}{\delta_{\pi}}
\newcommand{\dn}{\delta_{N}}
\bibliographystyle{iopart-num}

\begin{document}
\title{Astroparticle Physics Tests of Lorentz Invariance Violation}

\author{R G Lang and V de Souza}

\address{Instituto de F\'isica de S\~ao Carlos, Universidade de S\~ao Paulo. Avenida Trabalhador S\~ao-Carlense, 400, S\~ao Carlos, SP, Brazil}

\ead{rodrigo.lang@usp.br}

\begin{abstract}
Testing Lorentz invariance is essential as it is one of the pillars of modern physics. Moreover, its violation is foreseen in several popular Quantum Gravity models. Several authors study the effects of Lorentz invariance violation (LIV) in the propagation of ultra-high energy cosmic rays. These particles are the most energetic events ever detected and therefore represent a promising framework to test LIV. In this work we present an analytic calculation of the in-elasticity for any $a+b \rightarrow c+d$ interaction using first order perturbation in the dispersion relation that violates Lorentz invariance. The inelasticity can be calculated by solving a third-order polynomial equation containing: a) the kinematics of the interaction, b) the LIV term for each particle and c) the geometry of the interaction. We use the inelasticity we calculate to investigate the proton propagation in the intergalactic media. The photopion production of the proton interaction with the CMB is taken into account using the inelasticity and the attenuation length in different LIV scenarios. We show how the allowed phase space for the photopion production changes when LIV is considered for the interaction. The calculations presented here are going to be extended in order to calculated the  modified ultra-high energy cosmic rays spectrum and compare it to the data.
\end{abstract}

\section{Introduction}
Relativity is one of the most important theories of modern physics and one of its pillars is the Lorentz invariance, which is proposed to be a fundamental symmetry in Quantum Field Theory. The possibility of Lorentz invariance violation (LIV), however, has been investigated by several quantum gravity (QG) models and is incorporated in high energy models of space-time structure~\cite{Mattingly2005}. Testing LIV is, therefore, very relevant in order to investigate new proposals and also to set the range of validity of relativity.

In this work, we study the possibility of testing LIV with the propagation of ultra-high energy cosmic rays (UHECR). We use a generic first order perturbative term in the free particle Lagrangian~\cite{ColemanGlashow}, which results in a new momentum dependent term in the dispersion relation. This correction is expected to be suppressed by the Planck Scale. Thus, astrophysics is a good framework for testing it, since UHECR have been detected by the Pierre Auger Observatory~\cite{AUGER} with energies three orders of magnitude larger than the events detected in particle accelerator.

Ultra-high energy cosmic rays interact with the low energy photon background as they propagate through the Universe. Protons interact mainly with the Cosmic Microwave Background (CMB) via pion and pair production~\cite{Stecker1968,Blumenthal1970}, heavier nuclei also interact with the Extragalactic Background Light (EBL), undergoing photo-disintegration~\cite{Stecker1999} and energetic photons can produce an electron-positron pair when interacting with the CMB and the Radio Background (RB)~\cite{DeAngelis2013}. The main structures in the UHECR spectrum could be caused by these interactions. The dip model proposes that in a pure proton spectrum, the ankle at $10^{18.7}$ eV appears naturally due to pair production~\cite{Berezinsky2009}. On the other hand, one of the proposed models to explain the suppression at the highest energies could be due to pion production and photo-disintegration, which is called the GZK suppression~\cite{Greisen,ZatsepinKuzmin}. If LIV is considered, the energy losses of the UHECR in the interactions with the CMB are changed and, consequently, the energy spectrum structures too.  For this reason, the UHECR spectrum, currently measured by the Pierre Auger Observatory~\cite{AUGERSpectrum} and by the Telescope Array experiment~\cite{SpectrumTA}, can be a reliable source of information to impose limits in the LIV.


This paper is organized as follows: in Section 2 we present the LIV framework used. In Section 3 we discuss the modifications in the kinematics of interactions of propagating UHECR. In Section 4 we show the results in the photopion production. And, finally, in Section 5 we conclude the paper and discuss other important tests that could be studied.

\section{Lorentz Invariance Violation}

The Coleman and Glashow formalism, proposed in~\cite{ColemanGlashow}, is a formulation which has been used in several works to study cosmic rays propagation with LIV~\cite{Jacobson2002,Stecker2005,Maccione2009,Scully2009}. It proposes a perturbation to the free particle Lagragian:

\begin{equation}
\centering
\lagr = \partial_{\mu} \psi^{*} Z \partial^{\mu} \psi - \psi^{*} M^{2} \psi
\end{equation}

\begin{equation}
\centering
\lagr \rightarrow \lagr + \partial_{i} \psi \delta_{a} \partial^{i} \psi
\end{equation}

This new perturbative term must be calculated for each particles and it leads to a new dispersion relation given by:

\begin{equation}
\centering
E_{a}^{2} = p_{a}^{2} c^{2} + m_{a}^{2} c^{4} + \delta_{a} p_{a}^{2} c^{2}
\end{equation}

which can be rewritten as:

\begin{equation}
\centering
E_{a}^{2} = p_{a}^{2} c^{2} (1+\delta_{a}) + m_{a}^{2} c^{4}
\end{equation}

It is then defined $c_{a} = c \sqrt{1+\delta_{a}}$:

\begin{equation}
\centering
E_{a}^{2} = p_{a}^{2} c_{a}^{2} + m_{a}^{2} \frac{c_{a}^{4}}{(1+\delta_{a})^{2}} \approx p_{a}^{2} c_{a}^{2} + m_{a}^{2} c_{a}^{4}
\end{equation}

therefore, $c_{a}$ is the maximum attainable velocity of a particle or its ``own speed of light''.

The main effect of this perturbation is a shift in the center of mass system (cms) energy of a particle\footnote{We were expressly showing $c$ in the dispersion relation in order to compare with $c_{a}$, but from now on we will use natural units where $c = 1$.}:

\begin{equation}
\label{eq:DispersionRelation}
\centering
\sqrt{s_{a}} = \sqrt{E_{a}^{2} - p_{a}^{2}} = \sqrt{m_{a}^{2} + \delta_{a} p^{2}_{a}}
\end{equation}

The cms energy is crucial for the calculation of the energy threshold of interactions and, consequently, a shift in this quantity results in a shift in the threshold of such interactions.

\section{Kinematics of Propagating UHECR}

In order to calculate the effects of LIV in the energy spectrum of UHECR, it is necessary to calculate the energy losses of propagating cosmic rays considering LIV. In this work we present the calculation for any generic $a + b \rightarrow c + d$ interaction based on the calculations performed by Scully and Stecker \cite{Scully2009}.

Scully and Stecker assume the effects of LIV only in the inelasticity, leaving the cross section unchanged. The same assumptions are used in this work. Their calculations for the inelasticity, however, are performed in the nucleus reference frame (NRF) and, then, a boost to the laboratory frame\footnote{The laboratory frame (LF) is that in which the CMB is isotropic and is the frame in which the energy spectra of cosmic rays are usually presented.} is necessary. Consequently, another assumption, that the boost is not modified by LIV is needed. The following calculations are already performed in the LF and, therefore, this assumption is avoided.

The inelasticity is obtained by imposing that the total rest energy, $\sqrt{s} = \sqrt{E^{2}_{tot} - p^{2}_{tot}}$ is conserved, i.e., $s_{i} = s_{f}$. We can, then, obtain $s_{i}$ and $s_{f}$ in terms of the properties of the particles $a$, $b$, $c$ and $d$:

\begin{equation}
\centering
\begin{cases}
s_{i} = \left(E_{a}+E_{b}\right)^{2} - \left(\vec{p}_{a}+\vec{p}_{b}\right)^{2} = s_{a} + s_{b} + 2 E_{a} E_{b} - 2 p_{a} p_{b} \cos{\theta_{i}} \\
s_{f} = \left(E_{c}+E_{d}\right)^{2} - \left(\vec{p}_{c}+\vec{p}_{d}\right)^{2} = s_{c} + s_{d} + 2 E_{c} E_{d} - 2 p_{c} p_{d} \cos{\theta_{f}}
\end{cases}
\end{equation}

We then rewrite the momentum as a function of the energy and the mass and use that $E^{2} >> s$:

\begin{equation}
\centering
s_{i} = s_{a} + s_{b} + 2 E_{a} E_{b} \left(1 - \cos{\theta_{i}} + \cos{\theta_{i}} \left(\frac{s_{a}}{2 E^{2}_{a}} + \frac{s_{b}}{2 E^{2}_{b}}\right) + \mathcal{O}(2) \right)
\end{equation}

\begin{equation}
\centering
s_{f} = s_{c} + s_{d} + 2 E_{c} E_{d} \left(1 - \cos{\theta_{f}} + \cos{\theta_{f}} \left(\frac{s_{c}}{2 E^{2}_{c}} + \frac{s_{d}}{2 E^{2}_{d}}\right) + \mathcal{O}(2) \right)
\end{equation}

In the regime where $a$ is a relativistic particle interacting with a low energy background and generating two relativistic particles, which is valid for all the interactions of interest\footnote{The pair production produces three particles: an electron, a positron and a photon. The electron-positron pair, however, can be approximated as a single particle with double the mass for the calculations.}, the final angle can be approximated as $\cos{\theta_{f}} \approx 1$, therefore:

\begin{equation}
\centering
s_{f} \approx s_{c} + s_{d} + 2 E_{c} E_{d} \left(\cos{\theta_{f}} \left(\frac{s_{c}}{2 E^{2}_{c}} + \frac{s_{d}}{2 E^{2}_{d}}\right)\right)
\end{equation}

Finally, we want to look at the inelasticity of the interaction that is defined as the fraction of energy lost by the particle $a$ due to the creation of $c$. In this regime, we can write:

\begin{equation}
\centering
K = \frac{E_{c}}{E_{a}} \implies E_{c} = K E_{a}
\end{equation}

\begin{equation}
\centering
K = 1 - \frac{E_{d}}{E_{a}} \implies E_{d} = (1-K) E_{a}
\end{equation}

Substituting $E_{c}$ and $E_{d}$ in $s_{i} = s_{f}$ and writing the cms energy of each particle using Eq.~\ref{eq:DispersionRelation}, we have:

\begin{equation}
\centering
\begin{array}{c}
s_{a} + s_{b} + 2 E_{a} E_{b} \left(1 - \cos{\theta_{i}} + \cos{\theta_{i}} \left(\frac{s_{a}}{2 E_{a}} + \frac{s_{b}}{2 E_{b}}\right)\right) \\ = s_{c} (K) + s_{d} (K) + K (1-K) \left(\cos{\theta_{f}} \left(\frac{s_{c} (K)}{K} + \frac{s_{d} (K)}{(1-K)}\right) \right)
\end{array}
\end{equation}

It is highly desirable to write this expression as a polynomial, for it is easier to solve. Therefore, we first define:

\begin{equation}
\centering
A := s_{i} - m_{c}^{2} - m_{d}^{2}
\end{equation}

and, then, multiply the expression\footnote{This is only possible as we know that $K \neq 0$ and $K \neq 1$. In the first situation there is no interaction and, in the second, only a pion is created, which is physically forbidden.} by $K(1-K)$:

\begin{equation}
\centering
\begin{split}
-m_{c}^{2} + K \left(A+2m_{c}^{2} - \delta_{d} E_{a}^{2}\right) + K^{2} \left(-A-m_{c}^{2}-m_{d}^{2} - \delta_{c} E_{a}^{2}\right) + K^{3} \left(\delta_{c} E_{a}^{2} \right) + \\
- \delta_{d} E_{a}^{2} \sum_{i=0}^{3} K^{i+1}  \binom{3}{i} \left(-1\right)^{i} - \delta_{d} E_{a}^{2} \sum_{i=0}^{2} K^{i} \binom{2}{i} \left(-1\right)^{i+2} = 0
\end{split}
\end{equation}

This third-order polynomial contains all the kinematics, LIV and geometry information and is easily solvable. Nevertheless, only roots in the regions $(0,1)$ are physical solutions for the problem.

Lastly, we use the modified inelasticity to calculate the attenuation length, $\ell$, which is defined as the distance the cosmic ray can travel before losing $1/e$ of its energy:

\begin{equation}
\centering
\frac{1}{\ell (E)} = \frac{1}{E} \frac{dE}{dx} = \int_{0}^{\infty} d\epsilon n(\epsilon) \sigma(\epsilon) \int_{-1}^{1} dcos \left(\theta_{i}\right) K(\epsilon,E,\theta_{i})
\end{equation}

where $n(\epsilon)$ is the background density, $\sigma(\epsilon)$ is the cross section and the threshold appears naturally in the inelasticity, as $K = 0$ for $\epsilon < \epsilon_{th}$.

With this information it is possible to calculate the propagation for any cosmic ray including LIV effects. In the next section we show the results for protons producing pions ($p + \gamma \rightarrow p + \pi$) as an example.

\section{Photopion Production}

The photopion production with LIV has been widely studied \cite{AlfaroPalma,Stecker2005,Scully2009,Xiao-Jun2009,Bietenholz,Saveliev,Cowsik2012,Boncioli2015,Aloisio2000} as this interaction modulates the shape of the spectrum in the highest energies, where the LIV effects could be strong enough to be detected. We have calculated the inelasticity and the attenuation length for this interaction using different values of $\delta_{\pi}$ and $\delta_{p}$ as an example for the calculations presented in the last section.

\begin{figure}[h]
\centering
\begin{minipage}{11.5pc}
\includegraphics[width=11.5pc]{0_0.eps}
\caption{\label{fig:Inelasticity0_0}Inelasticity for the photopion production in a scenario without LIV.}
\end{minipage}\hspace{1.5pc}%
\begin{minipage}{11.5pc}
\includegraphics[width=11.5pc]{10_0.eps}
\caption{\label{fig:Inelasticity10_0}Inelasticity for the photopion production in a scenario with LIV for the pion.}
\end{minipage} \hspace{1.5pc}%
\begin{minipage}{11.5pc}
\includegraphics[width=11.5pc]{0_10.eps}
\caption{\label{fig:Inelasticity0_10}Inelasticity for the photopion production in a scenario with LIV for the proton.}
\end{minipage}
\end{figure}

Figs. \ref{fig:Inelasticity0_0}-\ref{fig:Inelasticity0_10} show the inelasticity for the interaction as a function of the initial proton energy and the background photon energy, both in the LF. These figures also show the threshold energy for the background photon, which is the first energy with non-zero inelasticity, and the phase space of the interaction.

Turning on LIV for the pion reduces the phase space for the interaction as the threshold for the interaction becomes larger when compared to the LI case. This effect gets stronger as the initial proton energy gets larger. The opposite happens for the proton, the threshold is reduced, resulting in a larger phase space.

\begin{figure}[h]
\centering
\begin{minipage}{16pc}
\includegraphics[width=16pc]{deltapi.eps}
\caption{\label{fig:AttLengthPion}Attenuation length for the photopion production with LIV for the pion. Each line show the result for a different LIV term.}
\end{minipage}\hspace{2pc}%
\begin{minipage}{16pc}
\includegraphics[width=16pc]{deltap.eps}
\caption{\label{fig:AttLengthProton}Attenuation length for the photopion production with LIV for the proton. Each line show the result for a different LIV term.}\end{minipage}
\end{figure}


Fig. \ref{fig:AttLengthPion} and Fig. \ref{fig:AttLengthProton} show the attenuation length, i.e., the distance the proton can travel before losing $1/e$ of its energy, for different LIV terms. The changes in the phase space have direct effects in this. For the pion, the attenuation length gets larger and, therefore, the cosmic ray travels further before interacting. For the proton, on the other hand, it gets smaller, which means that the cosmic ray will interact more often.

The LIV term in the dispersion relation can be treated as an effective mass that gets larger with the energy. For the pion, a larger mass would mean more energy needed to create it, consequently shifting up the threshold. A shift in the proton mass, on the other hand, would change the energy both before and after the interaction. But, as the initial proton is more energetic, the shift before the interaction is larger and therefore the energy of the background photon necessary for creating the pion is smaller.

Those changes have direct influence in the UHECR spectrum as it could be recovered or suppressed in the highest energies.

\section{Conclusions}

The propagation of ultra energetic cosmic rays is a reliable source for testing Lorentz Invariance Violation as it leaves footprints in the energy spectrum measured with high statistics by different experiments.

In this work we have presented a generic analytic calculation for the inelasticity in the LF of any $a + b \rightarrow c + d$ interaction. The inelasticity is directly related to the threshold of these interactions and is used to obtain their energy losses. We have shown, as an example, the results of the calculations for the photopion production. The main effect is a change in the phase space of the interaction, which can be larger or smaller, depending on the LIV terms.

The calculations here proposed can be used in several LIV studies. Limits for the LIV terms, for example, could be imposed by performing these calculations for all the interactions present in the propagation and obtaining the spectrum using propagation codes such as CRPropa~\cite{CRPropa3} and SimProp~\cite{SimProp}.

\ack
We would like to thank the S\~ao Paulo Research Foundation (FAPESP) for the financial support through Grant No. 2014/26816-0 and 2010/07359-6 and Conacyt for the support through the Mesoamerican Center for Theoretical Physics, provinding a scholarship to Rodrigo Guedes Lang in order to participate in the school and present this work. We would also like to thank Humberto Martinez for the help provided to fully understand and double check the calculations.


\section*{References}
\bibliography{bibtex.bib}

\end{document}


These guidelines show how to prepare articles for publication in \jpcs\ using \LaTeX\ so they can be published quickly and accurately. Articles will be refereed by the \corg s but the accepted PDF will be published with no editing, proofreading or changes to layout. It is, therefore, the author's responsibility to ensure that the content and layout are correct.  This document has been prepared using \cls\ so serves as a sample document. The class file and accompanying documentation are available from \verb"http://jpcs.iop.org".

\section{Preparing your paper}
\verb"jpconf" requires \LaTeXe\ and  can be used with other package files such
as those loading the AMS extension fonts
\verb"msam" and \verb"msbm" (these fonts provide the
blackboard bold alphabet and various extra maths symbols as well as
symbols useful in figure captions); an extra style file \verb"iopams.sty" is
provided to load these packages and provide extra definitions for bold Greek letters.
\subsection{Headers, footers and page numbers}
Authors should {\it not} add headers, footers or page numbers to the pages of their article---they will
be added by \iopp\ as part of the production process.

\subsection{{\cls\ }package options}
The \cls\ class file has two options `a4paper' and `letterpaper':
\begin{verbatim}
\documentclass[a4paper]{jpconf}
\end{verbatim}

or \begin{verbatim}
\documentclass[letterpaper]{jpconf}
\end{verbatim}

\begin{center}
\begin{table}[h]
\caption{\label{opt}\cls\ class file options.}
%\footnotesize\rm
\centering
\begin{tabular}{@{}*{7}{l}}
\br
Option&Description\\
\mr
\verb"a4paper"&Set the paper size and margins for A4 paper.\\
\verb"letterpaper"&Set the paper size and margins for US letter paper.\\
\br
\end{tabular}
\end{table}
\end{center}

The default paper size is A4 (i.e., the default option is {\tt a4paper}) but this can be changed to Letter by
using \verb"\documentclass[letterpaper]{jpconf}". It is essential that you do not put macros into the text which alter the page dimensions.

\section{The title, authors, addresses and abstract}
The code for setting the title page information is slightly different from
the normal default in \LaTeX\ but please follow these instructions as carefully as possible so all articles within a conference have the same style to the title page.
The title is set in bold unjustified type using the command
\verb"\title{#1}", where \verb"#1" is the title of the article. The
first letter of the title should be capitalized with the rest in lower case.
The next information required is the list of all authors' names followed by
the affiliations. For the authors' names type \verb"\author{#1}",
where \verb"#1" is the
list of all authors' names. The style for the names is initials then
surname, with a comma after all but the last
two names, which are separated by `and'. Initials should {\it not} have
full stops. First names may be used if desired. The command \verb"\maketitle" is not
required.

The addresses of the authors' affiliations follow the list of authors.
Each address should be set by using
\verb"\address{#1}" with the address as the single parameter in braces.
If there is more
than one address then a superscripted number, followed by a space, should come at the start of
each address. In this case each author should also have a superscripted number or numbers following their name to indicate which address is the appropriate one for them.

Please also provide e-mail addresses for any or all of the authors using an \verb"\ead{#1}" command after the last address. \verb"\ead{#1}" provides the text Email: so \verb"#1" is just the e-mail address or a list of emails.

The abstract follows the addresses and
should give readers concise information about the content
of the article and should not normally exceed 200
words. {\bf All articles must include an abstract}. To indicate the start
of the abstract type \verb"\begin{abstract}" followed by the text of the
abstract.  The abstract should normally be restricted
to a single paragraph and is terminated by the command
\verb"\end{abstract}"

\subsection{Sample coding for the start of an article}
\label{startsample}
The code for the start of a title page of a typical paper might read:
\begin{verbatim}
\title{The anomalous magnetic moment of the
neutrino and its relation to the solar neutrino problem}

\author{P J Smith$^1$, T M Collins$^2$,
R J Jones$^{3,}$\footnote[4]{Present address:
Department of Physics, University of Bristol, Tyndalls Park Road,
Bristol BS8 1TS, UK.} and Janet Williams$^3$}

\address{$^1$ Mathematics Faculty, Open University,
Milton Keynes MK7~6AA, UK}
\address{$^2$ Department of Mathematics,
Imperial College, Prince Consort Road, London SW7~2BZ, UK}
\address{$^3$ Department of Computer Science,
University College London, Gower Street, London WC1E~6BT, UK}

\ead{williams@ucl.ac.uk}

\begin{abstract}
The abstract appears here.
\end{abstract}
\end{verbatim}

\section{The text}
The text of the article should should be produced using standard \LaTeX\ formatting. Articles may be divided into sections and subsections, but the length limit provided by the \corg\ should be adhered to.

\subsection{Acknowledgments}
Authors wishing to acknowledge assistance or encouragement from
colleagues, special work by technical staff or financial support from
organizations should do so in an unnumbered Acknowledgments section
immediately following the last numbered section of the paper. The
command \verb"\ack" sets the acknowledgments heading as an unnumbered
section.

\subsection{Appendices}
Technical detail that it is necessary to include, but that interrupts
the flow of the article, may be consigned to an appendix.
Any appendices should be included at the end of the main text of the paper, after the acknowledgments section (if any) but before the reference list.
If there are two or more appendices they will be called Appendix A, Appendix B, etc.
Numbered equations will be in the form (A.1), (A.2), etc,
figures will appear as figure A1, figure B1, etc and tables as table A1,
table B1, etc.

The command \verb"\appendix" is used to signify the start of the
appendixes. Thereafter \verb"\section", \verb"\subsection", etc, will
give headings appropriate for an appendix. To obtain a simple heading of
`Appendix' use the code \verb"\section*{Appendix}". If it contains
numbered equations, figures or tables the command \verb"\appendix" should
precede it and \verb"\setcounter{section}{1}" must follow it.

\section{References}
%%%%%%%%%%%%%%%%%%%%%%%%%%%%%%%%%%%%%%%%%%%
In the online version of \jpcs\ references will be linked to their original source or to the article within a secondary service such as INSPEC or ChemPort wherever possible. To facilitate this linking extra care should be taken when preparing reference lists.

Two different styles of referencing are in common use: the Harvard alphabetical system and the Vancouver numerical system.  For \jpcs, the Vancouver numerical system is preferred but authors should use the Harvard alphabetical system if they wish to do so. In the numerical system references are numbered sequentially throughout the text within square brackets, like this [2], and one number can be used to designate several references.

\subsection{Using \BibTeX}
We highly recommend the {\ttfamily\textbf\selectfont iopart-num} \BibTeX\ package by Mark~A~Caprio~\cite{iopartnum}, which is included with this documentation.

\subsection{Reference lists}
A complete reference should provide the reader with enough information to locate the article concerned, whether published in print or electronic form, and should, depending on the type of reference, consist of:

\begin{itemize}
\item name(s) and initials;
\item date published;
\item title of journal, book or other publication;
\item titles of journal articles may also be included (optional);
\item volume number;
\item editors, if any;
\item town of publication and publisher in parentheses for {\it books};
\item the page numbers.
\end{itemize}

Up to ten authors may be given in a particular reference; where
there are more than ten only the first should be given followed by
`{\it et al}'. If an author is unsure of a particular journal's abbreviated title it is best to leave the title in
full. The terms {\it loc.\ cit.\ }and {\it ibid.\ }should not be used.
Unpublished conferences and reports should generally not be included
in the reference list and articles in the course of publication should
be entered only if the journal of publication is known.
A thesis submitted for a higher degree may be included
in the reference list if it has not been superseded by a published
paper and is available through a library; sufficient information
should be given for it to be traced readily.

\subsection{Formatting reference lists}
Numeric reference lists should contain the references within an unnumbered section (such as \verb"\section*{References}"). The
reference list itself is started by the code
\verb"\begin{thebibliography}{<num>}", where \verb"<num>" is the largest
number in the reference list and is completed by
\verb"\end{thebibliography}".
Each reference starts with \verb"\bibitem{<label>}", where `label' is the label used for cross-referencing. Each \verb"\bibitem" should only contain a reference to a single article (or a single article and a preprint reference to the same article).  When one number actually covers a group of two or more references to different articles, \verb"\nonum"
should replace \verb"\bibitem{<label>}" at
the start of each reference in the group after the first.

For an alphabetic reference list use \verb"\begin{thereferences}" ... \verb"\end{thereferences}" instead of the
`thebibliography' environment and each reference can be start with just \verb"\item" instead of \verb"\bibitem{label}"
as cross referencing is less useful for alphabetic references.

\subsection {References to printed journal articles}
A normal reference to a journal article contains three changes of font (see table \ref{jfonts}) and is constructed as follows:

\begin{itemize}
\item the authors should be in the form surname (with only the first letter capitalized) followed by the initials with no periods after the initials. Authors should be separated by a comma except for the last two which should be separated by `and' with no comma preceding it;
\item the article title (if given) should be in lower case letters, except for an initial capital, and should follow the date;
\item the journal title is in italic and is abbreviated. If a journal has several parts denoted by different letters the part letter should be inserted after the journal in Roman type, e.g. {\it Phys. Rev.} A;
\item the volume number should be in bold type;
\item both the initial and final page numbers should be given where possible. The final page number should be in the shortest possible form and separated from the initial page number by an en rule `-- ', e.g. 1203--14, i.e. the numbers `12' are not repeated.
\end{itemize}

A typical (numerical) reference list might begin

\medskip
\begin{thebibliography}{9}
\item Strite S and Morkoc H 1992 {\it J. Vac. Sci. Technol.} B {\bf 10} 1237
\item Jain S C, Willander M, Narayan J and van Overstraeten R 2000
{\it J. Appl. Phys}. {\bf 87} 965
\item Nakamura S, Senoh M, Nagahama S, Iwase N, Yamada T, Matsushita T, Kiyoku H
and 	Sugimoto Y 1996 {\it Japan. J. Appl. Phys.} {\bf 35} L74
\item Akasaki I, Sota S, Sakai H, Tanaka T, Koike M and Amano H 1996
{\it Electron. Lett.} {\bf 32} 1105
\item O'Leary S K, Foutz B E, Shur M S, Bhapkar U V and Eastman L F 1998
{\it J. Appl. Phys.} {\bf 83} 826
\item Jenkins D W and Dow J D 1989 {\it Phys. Rev.} B {\bf 39} 3317
\end{thebibliography}
\smallskip

\noindent which would be obtained by typing

\begin{verbatim}
\begin{\thebibliography}{9}
\item Strite S and Morkoc H 1992 {\it J. Vac. Sci. Technol.} B {\bf 10} 1237
\item Jain S C, Willander M, Narayan J and van Overstraeten R 2000
{\it J. Appl. Phys}. {\bf 87} 965
\item Nakamura S, Senoh M, Nagahama S, Iwase N, Yamada T, Matsushita T, Kiyoku H
and 	Sugimoto Y 1996 {\it Japan. J. Appl. Phys.} {\bf 35} L74
\item Akasaki I, Sota S, Sakai H, Tanaka T, Koike M and Amano H 1996
{\it Electron. Lett.} {\bf 32} 1105
\item O'Leary S K, Foutz B E, Shur M S, Bhapkar U V and Eastman L F 1998
{\it J. Appl. Phys.} {\bf 83} 826
\item Jenkins D W and Dow J D 1989 {\it Phys. Rev.} B {\bf 39} 3317
\end{\thebibliography}
\end{verbatim}

\begin{center}
\begin{table}[h]
\centering
\caption{\label{jfonts}Font styles for a reference to a journal article.}
\begin{tabular}{@{}l*{15}{l}}
\br
Element&Style\\
\mr
Authors&Roman type\\
Date&Roman type\\
Article title (optional)&Roman type\\
Journal title&Italic type\\
Volume number&Bold type\\
Page numbers&Roman type\\
\br
\end{tabular}
\end{table}
\end{center}

\subsection{References to \jpcs\ articles}
Each conference proceeding published in \jpcs\ will be a separate {\it volume};
references should follow the style for conventional printed journals. For example:\vspace{6pt}
\numrefs{1}
\item Douglas G 2004 \textit{J. Phys.: Conf. Series} \textbf{1} 23--36
\endnumrefs

%%%%%%%%%%%%%%%%%%%%%%%%%%%%%%%%%%
\subsection{References to preprints}
For preprints there are two distinct cases:
\renewcommand{\theenumi}{\arabic{enumi}}
\begin{enumerate}
\item Where the article has been published in a journal and the preprint is supplementary reference information. In this case it should be presented as:
\medskip
\numrefs{1}
\item Kunze K 2003 T-duality and Penrose limits of spatially homogeneous and inhomogeneous cosmologies {\it Phys. Rev.} D {\bf 68} 063517 ({\it Preprint} gr-qc/0303038)
\endnumrefs
\item Where the only reference available is the preprint. In this case it should be presented as
\medskip
\numrefs{1}
\item Milson R, Coley A, Pravda V and Pravdova A 2004 Alignment and algebraically special tensors {\it Preprint} gr-qc/0401010
\endnumrefs
\end{enumerate}

\subsection{References to electronic-only journals}
In general article numbers are given, and no page ranges, as most electronic-only journals start each article on page 1.

\begin{itemize}
\item For {\it New Journal of Physics} (article number may have from one to three digits)
\numrefs{1}
\item Fischer R 2004 Bayesian group analysis of plasma-enhanced chemical vapour deposition data {\it New. J. Phys.} {\bf 6} 25
\endnumrefs
\item For SISSA journals the volume is divided into monthly issues and these form part of the article number

\numrefs{2}
\item Horowitz G T and Maldacena J 2004 The black hole final state {\it J. High Energy Phys.}  	JHEP02(2004)008
\item Bentivegna E, Bonanno A and Reuter M 2004 Confronting the IR fixed point cosmology 	with 	high-redshift observations {\it J. Cosmol. Astropart. Phys.} JCAP01(2004)001
\endnumrefs
\end{itemize}

\subsection{References to books, conference proceedings and reports}
References to books, proceedings and reports are similar to journal references, but have
only two changes of font (see table~\ref{book}).

\begin{table}
\centering
\caption{\label{book}Font styles for references to books, conference proceedings and reports.}
\begin{tabular}{@{}l*{15}{l}}
\br
Element&Style\\
\mr
Authors&Roman type\\
Date&Roman type\\
Book title (optional)&Italic type\\
Editors&Roman type\\
Place (city, town etc) of publication&Roman type\\
Publisher&Roman type\\
Volume&Roman type\\
Page numbers&Roman type\\
\br
\end{tabular}
\end{table}

Points to note are:
\medskip
\begin{itemize}
\item Book titles are in italic and should be spelt out in full with initial capital letters for all except minor words. Words such as Proceedings, Symposium, International, Conference, Second, etc should be abbreviated to {\it Proc.}, {\it Symp.}, {\it Int.}, {\it Conf.}, {\it 2nd}, respectively, but the rest of the title should be given in full, followed by the date of the conference and the town or city where the conference was held. For Laboratory Reports the Laboratory should be spelt out wherever possible, e.g. {\it Argonne National Laboratory Report}.
\item The volume number, for example vol 2, should be followed by the editors, if any, in a form such as `ed A J Smith and P R Jones'. Use {\it et al} if there are more than two editors. Next comes the town of publication and publisher, within brackets and separated by a colon, and finally the page numbers preceded by p if only one number is given or pp if both the initial and final numbers are given.
\end{itemize}

Examples taken from published papers:
\medskip

\numrefs{99}
\item Kurata M 1982 {\it Numerical Analysis for Semiconductor Devices} (Lexington, MA: Heath)
\item Selberherr S 1984 {\it Analysis and Simulation of Semiconductor Devices} (Berlin: Springer)
\item Sze S M 1969 {\it Physics of Semiconductor Devices} (New York: Wiley-Interscience)
\item Dorman L I 1975 {\it Variations of Galactic Cosmic Rays} (Moscow: Moscow State University Press) p 103
\item Caplar R and Kulisic P 1973 {\it Proc. Int. Conf. on Nuclear Physics (Munich)} vol 1 (Amsterdam: 	North-Holland/American Elsevier) p 517
\item Cheng G X 2001 {\it Raman and Brillouin Scattering-Principles and Applications} (Beijing: Scientific)
\item Szytula A and Leciejewicz J 1989 {\it Handbook on the Physics and Chemistry of Rare Earths} vol 12, ed K A Gschneidner Jr and L Erwin (Amsterdam: Elsevier) p 133
\item Kuhn T 1998 {\it Density matrix theory of coherent ultrafast dynamics Theory of Transport Properties of Semiconductor Nanostructures} (Electronic Materials vol 4) ed E Sch\"oll (London: Chapman and Hall) chapter 6 pp 173--214
\endnumrefs

\section{Tables and table captions}
Tables should be numbered serially and referred to in the text
by number (table 1, etc, {\bf rather than} tab. 1). Each table should be a float and be positioned within the text at the most convenient place near to where it is first mentioned in the text. It should have an
explanatory caption which should be as concise as possible.

\subsection{The basic table format}
The standard form for a table is:
\begin{verbatim}
\begin{table}
\caption{\label{label}Table caption.}
\begin{center}
\begin{tabular}{llll}
\br
Head 1&Head 2&Head 3&Head 4\\
\mr
1.1&1.2&1.3&1.4\\
2.1&2.2&2.3&2.4\\
\br
\end{tabular}
\end{center}
\end{table}
\end{verbatim}

The above code produces table~\ref{ex}.

\begin{table}[h]
\caption{\label{ex}Table caption.}
\begin{center}
\begin{tabular}{llll}
\br
Head 1&Head 2&Head 3&Head 4\\
\mr
1.1&1.2&1.3&1.4\\
2.1&2.2&2.3&2.4\\
\br
\end{tabular}
\end{center}
\end{table}

Points to note are:
\medskip
\begin{enumerate}
\item The caption comes before the table.
\item The normal style is for tables to be centred in the same way as
equations. This is accomplished
by using \verb"\begin{center}" \dots\ \verb"\end{center}".

\item The default alignment of columns should be aligned left.

\item Tables should have only horizontal rules and no vertical ones. The rules at
the top and bottom are thicker than internal rules and are set with
\verb"\br" (bold rule).
The rule separating the headings from the entries is set with
\verb"\mr" (medium rule). These commands do not need a following double backslash.

\item Numbers in columns should be aligned as appropriate, usually on the decimal point;
to help do this a control sequence \verb"\lineup" has been defined
which sets \verb"\0" equal to a space the size of a digit, \verb"\m"
to be a space the width of a minus sign, and \verb"\-" to be a left
overlapping minus sign. \verb"\-" is for use in text mode while the other
two commands may be used in maths or text.
(\verb"\lineup" should only be used within a table
environment after the caption so that \verb"\-" has its normal meaning
elsewhere.) See table~\ref{tabone} for an example of a table where
\verb"\lineup" has been used.
\end{enumerate}

\begin{table}[h]
\caption{\label{tabone}A simple example produced using the standard table commands
and $\backslash${\tt lineup} to assist in aligning columns on the
decimal point. The width of the
table and rules is set automatically by the
preamble.}

\begin{center}
\lineup
\begin{tabular}{*{7}{l}}
\br
$\0\0A$&$B$&$C$&\m$D$&\m$E$&$F$&$\0G$\cr
\mr
\0\023.5&60  &0.53&$-20.2$&$-0.22$ &\01.7&\014.5\cr
\0\039.7&\-60&0.74&$-51.9$&$-0.208$&47.2 &146\cr
\0123.7 &\00 &0.75&$-57.2$&\m---   &---  &---\cr
3241.56 &60  &0.60&$-48.1$&$-0.29$ &41   &\015\cr
\br
\end{tabular}
\end{center}
\end{table}

\section{Figures and figure captions}
Figures must be included in the source code of an article at the appropriate place in the text not grouped together at the end.

Each figure should have a brief caption describing it and, if
necessary, interpreting the various lines and symbols on the figure.
As much lettering as possible should be removed from the figure itself and
included in the caption. If a figure has parts, these should be
labelled ($a$), ($b$), ($c$), etc.
\Tref{blobs} gives the definitions for describing symbols and lines often
used within figure captions (more symbols are available
when using the optional packages loading the AMS extension fonts).

\begin{table}[h]
\caption{\label{blobs}Control sequences to describe lines and symbols in figure
captions.}
\begin{center}
\begin{tabular}{lllll}
\br
Control sequence&Output&&Control sequence&Output\\
\mr
\verb"\dotted"&\dotted        &&\verb"\opencircle"&\opencircle\\
\verb"\dashed"&\dashed        &&\verb"\opentriangle"&\opentriangle\\
\verb"\broken"&\broken&&\verb"\opentriangledown"&\opentriangledown\\
\verb"\longbroken"&\longbroken&&\verb"\fullsquare"&\fullsquare\\
\verb"\chain"&\chain          &&\verb"\opensquare"&\opensquare\\
\verb"\dashddot"&\dashddot    &&\verb"\fullcircle"&\fullcircle\\
\verb"\full"&\full            &&\verb"\opendiamond"&\opendiamond\\
\br
\end{tabular}
\end{center}
\end{table}


Authors should try and use the space allocated to them as economically as possible. At times it may be convenient to put two figures side by side or the caption at the side of a figure. To put figures side by side, within a figure environment, put each figure and its caption into a minipage with an appropriate width (e.g. 3in or 18pc if the figures are of equal size) and then separate the figures slightly by adding some horizontal space between the two minipages (e.g. \verb"\hspace{.2in}" or \verb"\hspace{1.5pc}". To get the caption at the side of the figure add the small horizontal space after the \verb"\includegraphics" command and then put the \verb"\caption" within a minipage of the appropriate width aligned bottom, i.e. \verb"\begin{minipage}[b]{3in}" etc (see code in this file used to generate figures 1--3).

Note that it may be necessary to adjust the size of the figures (using optional arguments to \verb"\includegraphics", for instance \verb"[width=3in]") to get you article to fit within your page allowance or to obtain good page breaks.

\begin{figure}[h]
\begin{minipage}{14pc}
\includegraphics[width=14pc]{name.eps}
\caption{\label{label}Figure caption for first of two sided figures.}
\end{minipage}\hspace{2pc}%
\begin{minipage}{14pc}
\includegraphics[width=14pc]{name.eps}
\caption{\label{label}Figure caption for second of two sided figures.}
\end{minipage}
\end{figure}

\begin{figure}[h]
\includegraphics[width=14pc]{name.eps}\hspace{2pc}%
\begin{minipage}[b]{14pc}\caption{\label{label}Figure caption for a narrow figure where the caption is put at the side of the figure.}
\end{minipage}
\end{figure}

Using the graphicx package figures can be included using code such as:
\begin{verbatim}
\begin{figure}
\begin{center}
\includegraphics{file.eps}
\end{center}
\caption{\label{label}Figure caption}
\end{figure}
\end{verbatim}

\section*{References}
\begin{thebibliography}{9}
\bibitem{iopartnum} IOP Publishing is to grateful Mark A Caprio, Center for Theoretical Physics, Yale University, for permission to include the {\tt iopart-num} \BibTeX package (version 2.0, December 21, 2006) with  this documentation. Updates and new releases of {\tt iopart-num} can be found on \verb"www.ctan.org" (CTAN).
\end{thebibliography}

\end{document}
