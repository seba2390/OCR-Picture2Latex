\section{Integration with custom chip-scale APD}
\begin{comment}
\begin{figure} [ht]
\begin{center}
\begin{tabular}{c} 
%\includegraphics[height=7.5cm]{images/ASTARLab-Setup8.jpg}
\includegraphics[width=\textwidth]{images/APD-arch-ASTAR-1.jpg}
\end{tabular}
\begin{tabular}{p{0.5\textwidth}>{c}p{0.5\textwidth}>{c}}
    (a) & (b) \\
\end{tabular}
\end{center}
\caption[] 
{ \label{fig:astar-apdarch} 
Two physical architectures of our custom chip-scale APD. (a) the wire-bonded package connects 8 out of the 16 on-chip APDs to pads which come out as pins of a standard DIP (dual-in-line package). (b) shows the bare chip with out any external packaging or interconnect. The N and P pads of the APDs can be connected directly with RF probe tips. The thin horizontal lines in both cases are optical waveguides for coupling light into the APDs.}
\end{figure} 
\begin{figure} [ht]
\begin{center}
\begin{tabular}{c} 
%\includegraphics[height=7.5cm]{images/ASTARLab-Setup8.jpg}
\includegraphics[width=\textwidth]{images/DIP-pkg-ASTAR-1.jpg}
\end{tabular}
\begin{tabular}{p{0.33\textwidth}>{c}p{0.33\textwidth}>{c}p{0.33\textwidth}>{c}}
    (a) & & (b) \\
\end{tabular}
\end{center}
\caption[] 
{ \label{fig:astar-apddip} 
(a) shows the interface board developed to host and connect the DIP-based custom APD shown in figure \ref{fig:astar-apdarch}(a). The N and P pads of each APD come out into separate 4-pin headers. (b) shows the test setup with the APD interface board connecting APD 'D12' to channel 2 of the custom active quench board shown in figure \ref{fig:aq-pcb}, which in-turn connects with the commercial SoC board}
\end{figure}
\begin{figure} [ht]
\begin{center}
\begin{tabular}{c} 
%\includegraphics[height=7.5cm]{images/ASTARLab-Setup8.jpg}
\includegraphics[width=\textwidth]{images/ASTARLab-Setup10.jpg}
\end{tabular}
\end{center}
\caption[] 
{ \label{fig:astar-setup} 
The lab set up at IMRE showing the active quench system integrated directly with the chip-scale APD. Inset (i) shows the RF probe interface board used to connect the APDs with Channel 2 of custom active quench board and the SoC. RF probe tips [inset (ii)] connect with the APD pads on the chip and relay the signal via RF probes to the interface board. The APD chip sits on a TEC so it can be cooled down to -20°C. A commercial SAP500-T8 is connected to Channel 1 of the custom board and both channels are operated simultaneously during the experiments}
\end{figure} 
\end{comment}


%\mathbf{*** Inputs needed from IMRE with references/citations - brief description of APD w.r.t. figure \ref{fig:astar-apdarch} - typical breadkdown voltage range (16V-25V?), over-voltage (2V-9V?) for the APDs - device types and approximate number of device variants ***}

\begin{comment}
Prior to testing with the active quench system, the custom APDs had been tested in Geiger mode with a passive quench system and almost all APDs displayed an anomalous behavior w.r.t. the breakdown voltage. For any fixed bias voltage above the APD breakdown voltage, it was observed that the breakdown voltage of the APDs increased during operation and within a few seconds it reached close to the bias voltage, thus reducing the over-voltage to almost 0. Consequently the detected counts also started reducing during the operation due to this effect until it eventually reached 0. Increasing the bias voltage at this point temporarily restores the counts until the breakdown voltage again increases to reach the new bias voltage and the counts again come down to 0 within a few seconds. This is a long-term effect in which the breakdown voltage in not restored even if the APD is kept powered off for several hours. The only way to restore the breakdown voltage to the original value is to manually forward bias the APD with 1V-2V for a few seconds.

\subsection{Nullifying the breakdown voltage anomaly in Geiger mode}
Initial testing was done with the wire-bonded DIP version of the custom APDs. As shown in figure \ref{fig:astar-apddip}(a), we developed a simple interface board which hosts the DIP on a zero-insertion-force (ZIF) socket. It brings out the P and N pins of each APD into a standard 4-pin header so that any of the 8 APDs can be easily connected with the active quench system in a plug-and-play manner. Figure \ref{fig:astar-apddip}(b) shows the test setup in which the APD 'D12' is connected to channel 2 of the custom active quench board shown in figure \ref{fig:aq-pcb}. Test results indicate that the breakdown voltage drift could be prevented with the active quench system depending on the values of bias voltage V\_BIAS and V\_QUENCH (indicated in figure \ref{aq-design}). We observe that when V\_QUENCH$\geq$V\_BIAS, the count rate never dips even when the APD was operated for several minutes continuously. Our hypothesis for this is that, when an avalanche occurs and quenching begins, the APD is forward biased for the entire quench duration under the condition V\_QUENCH$\geq$V\_BIAS which prevents the breakdown voltage of the APD from drifting.
\end{comment}

\begin{figure} [ht]
\begin{center}
\begin{tabular}{c} 
\includegraphics[height=8cm]{images/IPS3-Vbr-DarkcountVsTemperature-23Sep21.pdf}
\end{tabular}
\end{center}
\caption[] 
{ \label{fig:astar-data} 
Variation of breakdown voltage and dark counts w.r.t. temperature for one of the chip-scale APDs in Geiger mode, obtained using our SoC based active quench system. Both data plots were obtained in three consecutive phases in which the APD was first cooled down to -20°C, then warmed up to +50°C and again cooled back to +20°C}
\end{figure} 
After validating the performance of our active quench system with a widely used commercial APD (SAP500-T8), we further use it for testing our custom fabricated chip-scale APDs\cite{imreAPD21} in Geiger mode. These are Si waveguide APDs integrated with on-chip photonic waveguides and are interfaced with the active quench system using RF probes. The active quench system shown in figure \ref{aq-design} was modified to include a second identical active quench channel so that both the SAP500 and custom chip-scale APD can be simultaneously connected and operated in parallel. Both GM-APDs are controlled by only one SoC through replication of the quench/reset generator and the counter modules on the FPGA. The SAP500 was operated at room temperature while the temperature of the chip-scale APD was varied between -20°C and +50°C using a TEC. To rule out the presence of any stray light in the setup, the count rate of the SAP500 was monitored closely during the experiments to ensure that the dark counts always remained at $\approx$20Kcps, which is the expected value for this APD at room temperature when operated at an excess voltage of 10V. The breakdown voltage of the chip-scale APDs are typically between 16V to 25V and their V\_BIAS is derived from a bench-top power supply. To  prevent the breakdown voltage of the chip-scale APDs from drifting \cite{imreAPD21}, V\_QUENCH was maintained between 18V to 27V in conjunction with V\_BIAS, so as to always keep it 2V above V\_BIAS. Although V\_QUENCH for both APDs was derived from the same bench-top power supply, any changes in V\_QUENCH does not affect the operation of the SAP500 since V\_QUENCH is always greater than 15V for all tests. Considering these high voltage values, the durations of quench, reset and deadtime for both channels were set conservatively higher so as to provide sufficient time for the MOSFETs to switch ON/OFF.
%Having a bigger quench duration also ensures the after-pulsing of the APDs stays quite low even at lower temperatures. 
So for both channels, the quench duration was set to 25ns, the reset duration to 15ns and a delay of 10ns between quench and reset, giving a total deadtime of $\approx$65ns (inclusive of initial response time and all other delays) for all experiments. V\_QUENCH and V\_BIAS of the chip-scale APD were changed at run-time using scripts running on a laptop PC, while the other parameters in Table \ref{tab:reconfig-param} remained fixed.

%which can be configured at run-time by commands from the PC over a USB interface. This enabled to automate the process of varying both voltages using scripts running on the PC. The associated driver software for the TEC module was used for changing the temperature, which was also configurable through scripts on the PC. Using the described setup we measured breakdown voltage and dark counts of the chip-scale APD for varying temperatures. 
Figure \ref{fig:astar-data} shows variation in dark count and breakdown voltage w.r.t. temperature for one chip-scale APD. To check for hysteresis in breakdown voltage, the temperature of the APD was changed in three continuous phases. It was first cooled from +20°C to -20°C, then warmed up from -20°C to +50°C and again cooled down from +50°C to +20°C. In each phase temperature was changed in steps of 10°C. At each temperature, V\_BIAS was initially set to a value well below the breakdown voltage and the detected counts were ensured to be 0. V\_BIAS was then increased in steps of 0.05V every 2 seconds until the SoC started measuring counts, which indicated the breakdown voltage. V\_BIAS was then kept constant and the dark counts were recorded. Before each new measurement the APD was reset by switching off V\_BIAS. 
%At the initial temperature of +20°C, the breakdown voltage for this device was measured to be 16.1V and the dark count obtained was $\approx$1.57Mcps. 
It can be observed that the variation of breakdown voltage and dark count w.r.t. temperature follows a well known trend as seen with existing commercial devices.

\subsection{Advantage of using SoC based active quenching for testing chip-scale APDs}
For preliminary tests, only dark counts and breakdown voltage of one chip-scale APD have been recorded. Future tests will involve characterizing the after-pulsing probability (like shown in figure \ref{fig:afterpulse}) of each APD for at least 10 different deadtimes, 5 different temperatures and 3 different excess voltages, which amounts to a minimum of 150 test cases per APD. There are more than 400 variants of our custom chip-scale APDs with each differing in channel length, width, doping concentration etc. So characterizing after-pulse probability for all of them requires 60,000 tests in total. This means the active quench system needs to be re-configured 60,000 times for executing the complete test set. Using hard-wired fixed active quench circuits for this purpose is impractical since it requires a lot of manual intervention for changing parameters, which is both laborious and error prone. Using our SoC based active quench system instead paves way for making these adjustments instantly with script-based automation thus saving both cost and time.

%are fabricated on a single chip containing several variants in which each APD differs from the other in terms of channel length, width, doping concentration, etc. As shown in figure \ref{fig:astar-apdarch}, there exist two physical architectures of our custom chip-scale APDs -  those with wire-bonded dual-in-line packaging (DIP) and those which are bare without any packaging and interconnect, .  
%This is especially true for future testing with our custom chip-scale APDs, for which there exist several hundred variants with each one different from the other in terms of channel length, width, doping concentration, etc.  This results in an exponential increase in the total number of test cases and  