\section{INTRODUCTION}
Single photon detection is used widely in quantum optics technology. Increasing the rate and efficiency of single photon detection is a fundamental way to improve the overall efficiency of optical systems. The most popular method to detect single photons is by using Geiger-mode Avalanche Photo Diodes (GM-APD) in which photon absorption triggers a current due to avalanche multiplication process. This avalanche current needs to be stopped quickly and the APD needs to be restored to its normal operating mode before it can detect the next photon. This process is called $quenching$ and can be done either passively or actively. Although $passive\ quench$ (PQ) \cite{Cova96, wincomparator} circuits are simple to implement they do not reset the diode quickly to its nominal state which limits their detection efficiency. $Passive\ Quench\ Active\ Reset$ (PQAR) \cite{pqar} circuits can provide shorter and well-defined reset times but fail to restrict the after-pulsing probability of the APD within acceptable bounds if the reset times set are too short. $Active\ quench$ (AQ) \cite{Gallivanoni10} circuits on the other hand provide fast response and reset times while limiting the after-pulsing probability of the APDs and hence are most suitable for high detection rate applications.

\subsubsection*{Motivation}
In several active quenching circuit designs, generation of the quench and reset signals for the APD is done using only discrete hardware components \cite{Ghioni96, Stipcevic09, Stipcevic17}. Such designs are cheaper to manufacture but critical parameters such as the quench and reset times are hard bound and cannot be changed dynamically. Changing the quench and reset time can compensate for deviation in APD performance at run-time occurring due to factors such as temperature variation. As an alternative to discrete systems, monolithic integrated \cite{zappa2000, Acconcia16, 37ps-aq, Ceccarelli2019, Zimmerman5ns} active quench circuits have been developed which offer flexibility to change parameters dynamically while also reaching very high count rates and detection efficiency. However fabricating custom ICs is a complex process requiring long development cycles and a very high manufacturing cost. 
%It also creates geographical dependencies on specific fabrication houses which makes the process commercially less viable.

In this paper we describe a hybrid design for active quenching of Geiger-Mode APDs (GM-APD) comprising of minimal discrete components and a commercially available System-on-Chip (SoC) \cite{zedboard} which has both an FPGA and a micro-controller integrated on the same chip. The active quench control logic is implemented on the FPGA which allows to dynamically change typical active quench parameters such as quench width, reset width, deadtime, etc. The microcontroller interfaces with a PC and provides remote access to the FPGA registers which allows to re-calibrate the APDs deployed in the field during run-time. Only a few discrete components need to be placed close to the APD head while the SoC can be placed farther away without compromising the performance and detection efficiency. We present results to show that our system can achieve a deadtime of 35ns despite the APD head located $\approx$15cm away from the SoC, thus allowing to reach count rates of $>$28Mcps. Although our system is mainly designed for active quenching, we show how it can also be reconfigured to work in either passive quench or PQAR configurations.

%At a system level this design has several advantages. It reduces the overall real estate area of the PCB close to the APD head making it potentially suitable for size/volume-constrained applications with strict SWAP (size-weight-area-power) budgets, such as nano-satellites. The use of the microcontroller allows for unique identification and access to each APD attached to the circuit through a connected network such as RF wireless and/or Ethernet. 
%It also enables to run diagnostic self-calibrating algorithms and dynamically correct for drifts in APD performance. This is useful for APDs which are deployed in harsh environments where human access is either difficult or impossible. 

%Our design supports a wide range of values for the tune-able parameters which provides flexibility to directly use it with many APDs from several manufacturers without modifications. 

The tunable parameters in our design can be varied over a wide range allowing operation with APDs from several manufacturers without modifications. We demonstrate this versatility by simultaneously operating our system with a commercial APD \cite{sap500} and our in-house custom fabricated chip-scale APD \cite{imreAPD21}, both of which have different characteristics in terms of breakdown voltage, over-voltage, etc. We present some preliminary results for breakdown voltage and dark counts obtained with the custom fabricated chip-scale APDs which further validates our design. In principle, the flexibility in our system would be useful both in a research setup comprising of multiple device variants and in mass production for doing direct wafer-scale testing. 

%This is especially true for future testing with our custom chip-scale APDs, for which there exist several hundred variants with each one different from the other in terms of channel length, width, doping concentration, etc. Characterizing each of these variants for multiple performance metrics such as dark count, after-pulsing, jitter, linearity etc. would involve dynamic adjustments of several active quench circuit parameters (quench/reset duration, deadtime, bias/quench voltages, etc.). This results in an exponential increase in the total number of test cases and doing them using hard-wired fixed active quench circuits is impractical since it involves a lot of manual intervention which is both laborious and error prone. Using our system instead paves way for automation of such testing and characterization thus saving cost and time.