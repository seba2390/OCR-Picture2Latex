\documentclass[]{spie}  %>>> use for US letter paper
%\documentclass[a4paper]{spie}  %>>> use this instead for A4 paper
%\documentclass[nocompress]{spie}  %>>> to avoid compression of citations

\renewcommand{\baselinestretch}{1.0} % Change to 1.65 for double spacing
 
\usepackage{amsmath,amsfonts,amssymb}
\usepackage{graphicx}
\usepackage{comment}
%\usepackage[colorlinks=true, allcolors=blue]{hyperref}
\usepackage{hyperref}
\hypersetup{colorlinks=true, allcolors=blue}
%\usepackage{array} 
%\usepackage [latin1]{inputenc}
\usepackage{tabularx}

\title{Realizing a robust, reconfigurable active quenching design for multiple architectures of single-photon avalanche detectors}

\author[a]{Subash Sachidananda}
\author[c]{Prithvi Gundlapalli}
\author[c]{Victor Leong}
\author[c]{Leonid Krivitsky}
\author[a,b]{Alexander Ling}
\affil[a]{Centre for Quantum Technologies, National University of Singapore, 3 Science Drive 2, Singapore 117543
}
\affil[b]{Department of Physics, National University of Singapore, 2 Science Drive 3, Singapore 117551}
\affil[c]{Institute of Materials Research and Engineering, Agency for Science, Technology and Research (A*STAR), Singapore
}

\authorinfo{For further information on active quench system: Subash Sachidananda - E-mail: subash.jies@gmail.com\\ For further information on custom fabricated APD: Victor Leong - E-mail: victor\_leong@imre.a-star.edu.sg}

% Option to view page numbers
\pagestyle{empty} % change to \pagestyle{plain} for page numbers   
\setcounter{page}{1} % Set start page numbering at e.g. 301
 
\begin{document} 
\maketitle

\begin{abstract}
Most active quench circuits used for single-photon avalanche detectors are designed either with discrete components which lack the flexibility of dynamically changing the control parameters, or with custom ASICs which require a long development time and high cost. As an alternative, we present a reconfigurable and robust hybrid design implemented using a System-on-Chip (SoC), which integrates both an FPGA and a microcontroller. We take advantage of the FPGA’s speed and configuration capabilities to vary the quench and reset parameters dynamically over a large range, thus allowing our circuit to operate with a wide variety of APDs without having to re-design the system. The microcontroller enables the remote adjustment of control parameters and re-calibration of APDs in the field. The ruggedized design uses components with space heritage, thus making it suitable for space-based applications in the fields of telecommunications and quantum key distribution (QKD). We characterize our circuit with a commercial APD cooled to -20°C, and obtain a deadtime of 35ns while maintaining the after-pulsing probability at close to 3$\%$. We also demonstrate versatility of the circuit by directly testing custom fabricated chip-scale APDs, which paves the way for automated wafer-scale testing and characterization.
\end{abstract}

% Include a list of keywords after the abstract 
\keywords{single photon avalanche detectors , System on chip SoC, FPGA, active quenching, chip-scale APD, reconfigurable quenching circuit, automated wafer-scale testing}

\section{Introduction}
\label{sec:Introduction}


The goal in top-$\size$ recommendation is to recommend to each
consumer a small set of $\size$ items from a large collection of
items~\cite{cremonesi2010performance}.  For example, Netflix may want
to recommend $\size$ appealing movies to each consumer.  Collaborative
Filtering (CF)~\cite{herlocker2002empirical,lee2012comparative} is a
common top-$\size$ recommendation method.  CF infers user interests by
analyzing partially observed user-item interaction data, such as user
ratings on movies or historical purchase
logs~\cite{kanagal2012supercharging}. The main assumption in CF is that
users with similar interaction patterns have similar interests.


Standard CF methods for top-$\size$ recommendation focus on making  suggestions  that accurately reflect the user's preference history. However, as  observed in previous work,  CF recommendations are generally biased toward  popular items, leading to a rich get richer effect~\cite{vargas2014improving,steck2011item}.  The major reasons for this are \textit{popularity bias} and \textit{sparsity} of CF interaction data (detailed in Section~\ref{sec:related-work}). In a nutshell, to maintain  accuracy, recommendations are generated from the dense regions of the data,  where the popular items lie.  

However,  accurately suggesting popular items, may not be satisfactory for the consumers. For example, in Netflix, an accuracy-focused movie recommender may recommend ``Star Wars: The Force Awakens'' to users who have seen ``Star Wars: Rogue One''.  But, those users are probably already aware of ``The Force Awakens''. Considering additional factors, such as novelty of recommendations,  can lead to more effective suggestions~\cite{cremonesi2010performance,Castells2015,zhang2008avoiding,ziegler2005improving,zhang2012auralist}. 
%Second, accuracy-focused models typically achieve a   overall item-space coverage across their recommendations,  whereas high item-space coverage helps providers of the items increase revenue
%, users satisfaction since they are  likely already aware of or can find these items on their own.  

Focusing on popular items also adversely affects the satisfaction of  the providers of the items. This is because  accuracy-focused models typically achieve a  low overall item space coverage across their recommendations, whereas   high item space coverage helps providers of the items increase their revenue~\cite{vargas2014improving,Castells2015,adomavicius2011maximizing,anderson2006thelongtail, yin2012challenging,adomavicius2012improving}.
%accuracy-focused models typically achieve a

In contrast to the relatively small number of popular items, there are copious  {\it long-tail\/} items that have fewer observations (e.g., ratings) available. More precisely,  using the Pareto  principle (i.e.,~the $80/20$ rule),  long-tail items can be defined as items that generate the lower $20\%$ of observations~\cite{yin2012challenging}. Experimentally we found that these items correspond to almost $85\%$ of the items in several datasets (Sections~\ref{sec:Notation} and \ref{sec:Experiments}). %Table~\ref{tab:DatasetStatsticsSmall})


As previously shown, one way to improve the novelty of top-$\size$ sets is to recommend interesting long-tail items~\cite{cremonesi2010performance,ge2010beyond}.  The intuition  is that since they have fewer observations available,  they are more likely to be unseen~\cite{Kaminskas:2016:DSN:3028254.2926720}.  
 %For example, in online commerce,  newly added items are long-tail items that are yet to be discovered.  
Moreover, long-tail item promotion also results in higher overall coverage of the item space%, which increases profits for providers of the items
~\cite{vargas2014improving,Castells2015,zhang2008avoiding,zhang2012auralist,adomavicius2011maximizing,anderson2006thelongtail,yin2012challenging,jambor2010optimizing}. Because long-tail promotion reduces accuracy~\cite{steck2011item}, there are trade-offs to be explored.


%original submitted to ICDE
%This work studies three aspects of top-$\size$ recommendation: accuracy, novelty, and item-space coverage, and examines their trade-offs. In most previous work, predictions of a base recommendation system are re-ranked to handle their trade-offs~\cite{adomavicius2012improving,jambor2010optimizing,zhang2013personalize,wang2009portfolio}. Due to performance considerations, however, these techniques are not customized per user. For example,  parameters that balance the trade-off between novelty and accuracy are cross-validated at a global level.  This can be detrimental since users have varying preferences for  objectives such as long-tail novelty. We explore how to  automatically infer  user  preference for long-tail novelty, and how to leverage  it to correct  the popularity bias in standard recommender models. Our work does not rely on any additional contextual data, although such data, if available, can help promote newly-added long-tail items~\cite{agarwal2009regression,Saveski:2014:ICR:2645710.2645751}.

This work studies three aspects of top-$\size$ recommendation: accuracy, novelty, and item space coverage, and examines their trade-offs. In most previous work, predictions of a base recommendation algorithm are \textit{re-ranked} to handle these trade-offs~\cite{adomavicius2012improving,jambor2010optimizing,zhang2013personalize,wang2009portfolio}. The re-ranking models are computationally efficient but suffer from two drawbacks. First, due to performance considerations,  parameters that balance the trade-off between novelty and accuracy  are not customized per user. Instead they are cross-validated at a global level.  This can be detrimental since users have varying preferences for  objectives such as long-tail novelty. Second,  the re-ranking methods are often limited to a specific base recommender  that may be sensitive to dataset density. 
As a result, the datasets are pruned and the problem is studied in dense settings~\cite{adomavicius2012improving,ho2014likes}; but real world  scenarios are often sparse~\cite{kanagal2012supercharging,liu2017experimental}.   
% Because  dataset density can impact the performance of most base recommenders (like R-SVD), which in turn affects the performance of the re-ranking model, 

\iffalse
We address these limitations by directly inferring  user  preference for long-tail novelty  from interaction data.  This  allows us to customize the re-ranking  per user, and design a \textit{generic} framework, which resolves the second problem. In particular, since the long-tail novelty preferences are estimated independently of any base  recommender model, we can  plug-in an appropriate base recommender w.r.t. the dataset sparsity.% including ones that are more suitable for sparse settings.  

Modelling  user  preference for  long-tail novelty using only item popularity statistics, e.g., the average popularity of rated items as in~\cite{jugovac2017efficient}, disregards additional information like whether the user found the item interesting and the long-tail preferences of other users  of the items. \iffalse To incorporate them, we introduce the notion of  \emph{item long-tail importance}. Both  user long-tail preferences and item long-tail importance are dependent:  a user has high preference for discovering long-tail items if she is interested in important long-tail items, and an item that is associated with many of these kinds of users is likely to be more important.  We propose a joint optimization framework to directly learn,  from interaction data, both the users' long-tail preferences and the  items' long-tail importance. \fi
We propose an optimization approach that  incorporates  this information and  directly learns,  from interaction data, the users' long-tail novelty preferences.

Next, we use these learned preferences  to design a  top-$\size$ recommendation framework thats is generic, and provides customized balance between accuracy, novelty, and coverage. We refer to it as framework as GANC.  Using GANC, we design a novel algorithm, {\it Ordered Sampling-based Locally Greedy (OSLG)\/}, that relies on the learned long-tail novelty preferences  to scalably correct for popularity bias. Our work does not rely on any additional contextual data, although such data, if available, can help promote newly-added long-tail items~\cite{agarwal2009regression,Saveski:2014:ICR:2645710.2645751}. In summary:
\fi

We address the first limitation by directly inferring  user  preference for long-tail novelty  from interaction data.   Estimating these  preferences  using only item popularity statistics, e.g., the average popularity of rated items as in~\cite{jugovac2017efficient}, disregards additional information, like whether the user found the item interesting or the long-tail preferences of other users  of the items. We propose an approach that  incorporates  this information and  learns the users' long-tail novelty preferences from interaction data.

This approach allows us to customize the re-ranking  per user, and  design a \textit{generic} re-ranking framework, which resolves the second limitation of prior work. In particular, since the long-tail novelty preferences are estimated independently of any base recommender, we can  plug-in an appropriate one w.r.t. different factors, such as the dataset sparsity.

Our top-$\size$ recommendation framework, \textbf{GANC}, is \textbf{G}eneric, and provides customized balance between \textbf{A}ccuracy, \textbf{N}ovelty, and \textbf{C}overage. % Moreover, based on the learned long-tail novelty preferences, we also design a novel algorithm, {\it Ordered Sampling-based Locally Greedy (OSLG)\/}, that relies on the learned long-tail novelty preferences  to scalably correct for popularity bias. 
Our work does not rely on any additional contextual data, although such data, if available, can help promote newly-added long-tail items~\cite{agarwal2009regression,Saveski:2014:ICR:2645710.2645751}. In summary:

%Consider  the following toy example:
\vspace{-0.2cm}
\begin{table}[htb]
\centering
\scriptsize
%\small
\begin{tabular}{ccccccc} 
%\toprule
%&\multirow{2}{*}{}&\multicolumn{7}{c}{Ratings}\\
& & \cellcolor{blue!35}$w_1$ &\cellcolor{blue!18} $w_2$ & $\dots$ &\cellcolor{blue!8} $w_{89}$  &\cellcolor{blue!8} $w_{99}$   
\\
&   &$i_1$&$i_2$&$\dots$&$i_{89}$&$i_{90}$\\ 
\cmidrule(r){3-7} 	 
%\midrule
\cellcolor{red!35}$\theta_1$  &$u_1 $   &5 &   & $\dots$ &  &   \\
\cellcolor{red!28}$\theta_2$  &$u_2$     &5 &    & $\dots$ &  &  \\
 $\theta_3=?$  &$\bf u_3$  &5 &  &   $\dots$ &  &  \\
\cellcolor{red!10}$\theta_4$ & $u_4$  &  &5   & $\dots$ & &\\ 
\cellcolor{red!10}$\theta_5$ & $u_5$  &  & 5  & $\dots$ & &\\ 
$\theta_6=?$  & $\bf u_6$ & &5  &      $\dots$& &  \\ 
 & & $\hdots$  &$\hdots$   &$\hdots$   &$\hdots$   &$\hdots$  \\
%\midrule 
\cmidrule(r){3-7} 	 
\multicolumn{2}{c}{item pop.}  & 3  & 3  & $\dots$ &50&60\\  
%\bottomrule
%$ f_i$    &3  &3  &1  &3  &1  &2  \\  \hline
\end{tabular}
%#.
\caption{Simplified user-item interaction data. The user long-tail novelty preference ($\theta_u$), item long-tail importance weight ($w_i$) are highlighted. Darker colors indicate larger values. } \label{tab:example}
\end{table} 
\vspace{-0.2cm}
\begin{example}  
In Table~\ref{tab:example}, we are interested in estimating $\theta_3$ and $\theta_6$,  the long-tail preference of users $u_3$ and $u_6$ who have each rated a single movie. Additional ratings for other users  are not included here.  Considering only rating information, we observe $i_1$ and $i_2$ are  equally popular $|\mathcal{U}_{i_1}^{\trainset}| = |\mathcal{U}_{i_2}^{\trainset}|=3$, and $r_{31}=5$ and $r_{62}=5$. Using Eq.~\ref{eq:tfidf-risk}  we have $\theta_3 = \theta_6$. However, if we were given the long-tail preferences of the each item's user set, specifically that $u_1$ and $u_2$ have high long-tail preference (darker red), while $u_4$ and $u_5$ have lower long-tail preference (lighter red), we could conclude $i_1$ is a more important long-tail item compared to $i_2$ (indicated by a darker blue shade for $w_1$), and we expect  $\theta_3 \geq \theta_6$.

% On the other hand, if we knew that $u_4$ and $u_5$ have lower long-tail preference, we could conclude $i_2$ is a  less significant long-tail item. Therefore, However, if we  consider the long-tail preferences of other users, we may reason differently.    We need another variable $w_i$ which captures this information. 
%we would conclude that $u_3$ has higher long-tail preference compared to $u_6$, since the users $i_1$ is a more prominent long-tail item. 

% Relying only  on item popularity information, we would  conclude   $u_3$ and $u_6$ have equal long-tail preference, since $i_1$ and $i_2$ are  equally popular. However, considering  the second column,  long-tail preference of users,  long-tail importance for each item,  which captures the long-tail preference of its users. Since  that  both users of $i_1$ have high long-tail preference while  the users of $i_2$ have lower preference,  we may conclude $i_1$ is a more important long-tail item compared to $i_2$. Therefore, $u_3$'s long-tail preference should be at least as large as $u_6$'s preference. Specifically, consider two  items $i_1$ and $i_2$, with the following rating data: $i_1=\{u_1:5, u_2:5, u_3:5 \}$, $i_2=\{u_4:5, u_5:5, u_6:5\}$.  

%Table~\ref{tab:example} shows  simplified rating data. We want an estimate of the long-tail preference of $u_3$ and $u_6$, who have each  rated a single movie.  Relying only  on movie popularity information, we would  conclude   $u_3$ and $u_6$ have similar long-tail preference, since $m_1$ and $m_2$ are  equally popular. However, considering the long-tail preferences of other users of those movies, we may reason differently: since $u_1$ and $u_2$ have high long-tail preference, and $u_4$ and $u_5$ have low long-tail preference, $m_1$ is a more prominent long-tail item compared to $m_2$. Therefore, it is likely that $u_3$ has higher long-tail preference compared to $u_6$.considering the long-tail preferences of other users of those movies, we may reason differently.  For example, 
\label{ex:running}
\end{example}



%------------------------------

\iffalse
\begin{example}
Table~\ref{tab:example} shows rating data for a simplified system. %Note the user-item interaction matrix is sparse.
For this example, we define popular movies as those that have received  three or more ratings; $\{m_1, m_2, m_4\}$ are popular and  $\{m_3, m_5, m_6\}$ are niche movies. We observe $u_1$ and $u_3$  have rated relatively popular movies (risk-averse) while $u_2$ and $u_4$ have rated niche movies (risk-loving). 
\label{ex:running}
\end{example}

\begin{table}[htb]
\centering
\scriptsize
\begin{tabular}{ccccccc} 
\toprule
			&$m_1$ &$m_2$   &$m_3$    &$m_4$   &$m_5$ &$m_6$  \\ \hline 
$u_1 $ &5  &4  & - &-  &-  &-   \\
$u_2$  &-  &-  &-  &-  &5  &5   \\
$u_3$  &-  &4  &-  &5  &-  &-   \\
$u_4$  &-  &-  &3  &-  &-  &4   \\ 
$u_5$  &5  &-  &-  &3  &-  &-   \\ 
$u_6$  &4  &2  &-  &4  &-  &-   \\ 
\bottomrule
%$ f_i$    &3  &3  &1  &3  &1  &2  \\  \hline
\end{tabular}
\caption{User-Movie rating data} \label{tab:example}
\end{table}

It is essential to consider consumer characteristics in designing recommender systems so that they promote long-tail items to the right group of users and spread demand evenly between hit and niche items.  

\fi





%------------------------------
\iffalse
\begin{table}[htb]
\centering
\scriptsize
\begin{tabular}{ccccccc} 
\toprule
			&$m_1$ &$m_2$   &$m_3$    &$m_4$   &$m_5$ &$m_6$  \\ \hline 
$u_1 $ &\textbf{5}  & \textbf{4}  &\textcolor{gray}{ 1.2} &-  &-  &-   \\
$u_2$  &-  &-  &-  &-  & \textbf{5}  &\textbf{5}   \\
$u_3$  &-  &\textbf{4}  &-  &\textbf{5}  &-  &-   \\
$u_4$  &-  &-  &\textbf{3}  &-  &-  &\textbf{4}   \\ 
$u_5$  &\textbf{5}  &-  &-  &\textbf{3}  &-  &-   \\ 
$u_6$  &\textbf{4}  &\textbf{2}  &-  &\textbf{4}  &-  &-   \\ 
\bottomrule
%$ f_i$    &3  &3  &1  &3  &1  &2  \\  \hline
\end{tabular}
\caption{User-Movie rating data} \label{tab:example}
\end{table}
% $\mathcal{P}^1= \{ \mathcal{P}_1^1 \{i_1,i_2,i_3\}, \mathcal{P}_2^1:\{i_2,i_3,i_5\}  \}$
 %$\mathcal{P}^2= \{ \mathcal{P}_1^2: \{i_1,i_2,i_3\}, \mathcal{P}_2^2:\{i_2,i_5,i_6\}  \}$
 %$\mathcal{P}^3= \{ \mathcal{P}_1^3: \{i_7,i_8,i_9\}, \mathcal{P}_2^3:\{i_{10},i_{11},i_{12}\}  \}$
\begin{table}[htb]
\centering
\tiny
\begin{tabular}{ccc} 
\toprule
		&$u_1$&$u_2$  \\ \hline 
$\mathcal{P}^1 $ & $\{i_1,i_2,i_3\}$ & $\{i_2,i_3,i_5\} $ \\
$\mathcal{P}^2$ & $\{i_1,i_2,i_3\}$ & $\{i_2,i_5,i_6\} $ \\
$\mathcal{P}^3$ & $\{i_7,i_8,i_9\}$ & $\{i_{10},i_{11},i_{12} \}$ \\
\bottomrule
%$ f_i$    &3  &3  &1  &3  &1  &2  \\  \hline
\end{tabular}
\caption{Top-$\size$ allocations to users.} \label{tab:paretoExamples}
\end{table}
\fi


\iffalse
When considering long-tail items, it is important to consider consumers' willingness  to explore niche or unpopular items and their propensity towards similar items. In particular, they can be characterized by their  {\it risk degree\/} and {\it focusing degree\/}, respectively.  We compute these estimates  based on historical rating information. The following example further describes these notions in the context of movie rating data. 

\begin{example}  
Table~\ref{tab:example} shows rating data for a simplified system with $6$ users, $6$ movies, and $3$ genres. $m_i^{j}$ implies that movie $m_i$ belongs to genre $j$. Note the user-item interaction matrix is sparse. 
  For this setting, we define popular movies as those that have received  three or more ratings; $\{m_1, m_2, m_4\}$ are popular and  $\{m_3, m_5, m_6\}$ are niche movies. We now profile the users according to their risk and focusing degree. E.g., $u_1$ has rated relatively popular movies belonging to the same genre (risk-averse, high focusing degree); $u_2$ has rated niches movies in the same genre (risk-loving, high focusing degree); $u_3$ has rated popular movies in two different genres (risk-averse, low focusing degree), and $u_4$ has rated niches movies in two different genres (risk-loving, low focusing degree). 
\label{ex:running}
\end{example}
\begin{table}[htb]
\centering
\tiny
\begin{tabular}{ccccccc} 
\toprule
			&$m_1^{1}$ &$m_2^{1}$   &$m_3^{2}$    &$m_4^{3}$   &$m_5^{3}$ &$m_6^{3}$  \\ \hline 
$u_1 $ &5  &4  &-  &-  &-  &-   \\
$u_2$  &-  &-  &-  &-  &5  &5   \\
$u_3$  &-  &4  &-  &5  &-  &-   \\
$u_4$  &-  &-  &3  &-  &-  &4   \\ 
$u_5$  &5  &-  &-  &3  &-  &-   \\ 
$u_6$  &4  &2  &-  &4  &-  &-   \\ 
\bottomrule
%$ f_i$    &3  &3  &1  &3  &1  &2  \\  \hline
\end{tabular}
\caption{User-Movie rating data} \label{tab:example}
\end{table}
It is essential to consider these consumer characteristics in designing recommender systems so that they promote long-tail items to the right group of users and spread demand evenly between the hit and niche items.  
\fi
\iffalse
\begin{center}
\begin{figure*}[tp]
%\scalebox{0.5}{%
\resizebox{1\textwidth}{!}{%
%\small%\addtolength{\tabcolsep}{5pt}% below sums to 8
\begin{tabularx}{1.5\textwidth}{>{\hsize=2.5\hsize}X>{\hsize=2.5\hsize}X>{\hsize=0.5\hsize}X>{\hsize=0.5\hsize}X>{\hsize=0.5\hsize}X>{\hsize=0.5\hsize}X>{\hsize=0.5\hsize}X>{\hsize=0.5\hsize}X}
    \multirow{12}{*}{\includegraphics[scale=0.3]{codeForExample/popularity-movie.png}} & \multirow{12}{*}{\includegraphics[scale=0.3]{codeForExample/scatterplot.png}} & & & & & & \\
%   & &               &       &       &       &       &       \\
    & &\multicolumn{1}{l|}{}               &$m_1^{g1}$   	&$m_2^{g1}$    	&$m_3^{g2}$    &$m_4^{g2}$      &$m_5^{g3}$    \\ \cline{3-8}%\hline
    & &\multicolumn{1}{l|}{u1}          &5  &5  &-  &-   &-  \\
    & &\multicolumn{1}{l|}{u2}    		&-  &-  &4  &4  &5  \\
    & &\multicolumn{1}{l|}{u3}   			&1  &2  &1  &-  &-   \\
    & &\multicolumn{1}{l|}{u4}     		&1  &-  &-  &-  &-  \\
    & &               &       &       &       &       &       \\
    & &               &       &       &       &       &       \\
    & &               &       &       &       &       &       \\
    & &               &       &       &       &       &	\\
    \\
\end{tabularx}}
\caption{User-Movie interaction data a) Popularity-Movie histogram b)Movie genres/clusters c) User-Movie rating data} \label{fig:example}
\end{figure*}
\end{center}
\fi



%We propose a novel approach that allows us to  promote long-tail items in a targeted manner, thereby improving the novelty of top-$\size$ sets, the overall item-space coverage across recommendations, while maintaining reasonable levels of accuracy.

%Next, we integrate these learned preferences  in a generic  top-$\size$ recommendation framework to provide customized balance between accuracy and coverage.

%sequentially make recommendations, while adjusting its parameters with regard to the set of top-$\size$ recommendations made so far. However, since  sequential parameter updates  cause  scalability issues, we propose a sampling based algorithm. This variant of our framework, called {\it Ordered Sampling-based Locally Greedy (OSLG)\/},  allows us to  correct for the popularity bias in recommendations with regard to individual user long-tail preferences. 

%ICDE submission
%Our framework differs with  prior work in the following aspects:  unlike~\cite{adomavicius2011maximizing,adomavicius2012improving,zhang2013personalize,ho2014likes},  the long-tail preference personalization in our framework is learned rather than optimized using cross-validation or parameter tuning. In other words, our personalization method is independent of the underlying base  recommendation models.  Moreover, our framework is  generic. This enables us to  plug-in several base recommenders, and evaluate their  effectiveness without requiring  extensive tuning for the accuracy and coverage trade-off. 


%\vspace{-2.8pt}
\begin{itemize}

\item  We examine various measures for estimating user long-tail novelty preference in Section~\ref{sec:lt-pref} and formulate an optimization problem  to directly learn users' preferences for long-tail  items from interaction data in Section~\ref{sec:learning-lt-pref}. %In addition, we introduce several heuristics for measuring the user preference for less common items from historical rating data.% 

\item  We integrate the user preference estimates into GANC %, a generic re-ranking framework that provides customized balance between accuracy, novelty, and coverage 
(Section~\ref{sec:RiskbasedReranking}), and  introduce {\it Ordered Sampling-based Locally Greedy (OSLG)\/}, a scalable algorithm that relies  on user long-tail preferences to correct the popularity bias (Section~\ref{sec:optimizationAlgorithm}).
%We introduce OSLG, a scalable algorithm that relies  on user long-tail preferences to  maximize item space coverage \textcolor{red}{while maintaining acceptable levels of accuracy} (Section~\ref{sec:optimizationAlgorithm}).

\item   We conduct an extensive empirical study and evaluate performance from  accuracy, novelty, and coverage perspectives (Section~\ref{sec:Experiments}).  We use five  datasets with varying density and difficulty levels. %:  Netflix, MovieTweetings, and MovieLens (100K, 1M, 10M). 
  In contrast to most related work,  our evaluation considers realistic settings that include a large number of infrequent  items and users. %This enables us to study the impact of  data density on the performance trade-offs of several  state of the art top-$\size$ recommendation algorithms. %   %,  and use the all-items ranking protocol~\cite{steck2013evaluation,vargas2014improving}, where performance is measured using all items with train data. to evaluate the performance of several  state of the art top-$\size$ recommendation algorithms 
 
\item Our empirical results confirm that the performance of re-ranking models is impacted by the underlying   base recommender and the dataset density. Our generic approach enables us to easily incorporate a suitable base recommender to devise an effective solution for both dense and sparse settings. In dense settings, we use the same base recommender as existing re-ranking approaches, and we outperform them in accuracy and coverage metrics. For sparse settings, we plug-in a more suitable base recommender, and devise an effective solution that is competitive with existing top-$\size$ recommendation methods in accuracy and novelty. 

%Directly estimating the long-tail novelty preferences allows us to customize re-ranking per user, and  devise a generic framework.   
 
\end{itemize}

Section~\ref{sec:related-work} describes related work. Section~\ref{sec:conclusion} concludes.

\section{Design and implementation}
\label{sec:design}
Similar to conventional active quenching circuits, our system operates in two phases in response to the occurrence of an avalanche: 1) $quench$ - reduce the bias voltage of the diode below its breakdown voltage for a few nanoseconds and 2) $reset$ - restore the bias voltage above the breakdown for a few nanoseconds while bypassing the ballast resistor at the same time. After the reset phase, the ballast resistor is reconnected and the APD returns to the nominal state. The main functional blocks of our system are shown in figure \ref{aq-design}. The $ZedBoard$ \cite{zedboard} development kit with a Zynq-7000 based SoC is used for realizing the FPGA and microcontroller modules. The GM-APD is connected with the quench/reset generator on the FPGA using a custom PCB with minimal discrete components. We use the LT6752-3 high speed comparator (COMP) which produces complementary outputs after detecting the avalanche pulse across sense resistor $R_S$. The feedback path for doing the quench comprises of a fast AND gate, P-MOSFET $P_1$ and N-MOSFET $N_1$. Reset is done through N-MOSFET $N_2$. The values of $R_4$ and $C_1$ are set such that the D-FlipFlop (D-FF) trims the comparator output to 20ns pulses (OUT) which are used for characterizing the deadtime and after-pulsing probability. Matsusada high voltage DC-DC converters are used to derive a bias voltage (V\_BIAS) between 0-500V while quench voltage V\_QUENCH is derived from an external power supply.
\begin{figure} [ht]
\begin{center}
\begin{tabular}{c} 
%\includegraphics[height=6cm]{images/HybridAQ-Design8.jpg}
\includegraphics[width=\textwidth]{images/HybridAQ-Design8.jpg}
\end{tabular}
\end{center}
\caption[] 
{ \label{aq-design} 
Simplified circuit diagram of the system with Geiger Mode APD (GM-APD)}
\end{figure}
\begin{figure} [ht]
\begin{center}
\begin{tabular}{c} 
%\includegraphics[height=5cm]{images/HybridAQ-TimingDiagram6.jpg}
\includegraphics[width=0.65\textwidth]{images/HybridAQ-TimingDiagram6.jpg}
\end{tabular}
\end{center}
\caption[] 
{ \label{aq-timingdiagram} 
Timing diagram of critical signals showing the active quench operation after occurrence of an avalanche}
\end{figure}
\subsection{Operation in active quench configuration}
\begin{figure} [ht]
\begin{center}
\begin{tabular}{c} 
%\includegraphics[height=7cm]{images/HybridAQ-Schematic3.jpg}
\includegraphics[width=\textwidth]{images/HybridAQ-APDAnode1.jpg}
\end{tabular}
\end{center}
\caption[] 
{ \label{fig:aq-anode} 
Waveform of the OUT pulse (C1) and GM-APD\_ANODE voltage (C2) captured on the oscilloscope}
\end{figure}
The design and operation is better understood with the timing diagram shown in figure \ref{aq-timingdiagram}. The avalanche current in the APD causes the voltage at the APD anode (GM-APD\_ANODE) to rise and consequently the voltage across $R_S$ exceeds the comparator reference voltage (V\_REF), which is typically set to $\approx$17mV. In response the comparator (COMP) toggles the complementary output pulses $Q$ to high and $\overline{Q}$ to low. Output $Q$ is fed back to one of the inputs of the AND gate while the FPGA keeps the other input of the AND gate (QUENCH\_ENABLE) high before the avalanche occurs. So the AND gate output (QUENCH) goes high to turn ON $N_1$ which subsequently turns ON $P_1$ to initiate quenching. As a result of this, the anode voltage of GM-APD reaches the quench voltage (V\_QUENCH) which effectively reduces the voltage across the APD below its breakdown voltage and brings it into the quench state. Until then the APD is passively quenched and the avalanche current is limited by the ballast resistor ($R_B$). This feedback path is kept as short as possible to reduce the response time and after-pulsing probability\cite{Wayne-AP}.
%Rb=43k and Rs=1k

The other comparator output $\overline{Q}$ is inverted (APD\_PULSE) and fed into the quench/reset logic to register the occurrence of an avalanche and into a 24-bit counter to record the total number of avalanches per second. The duration for which the APD stays in the quench state is configurable and after this duration the FPGA sets the QUENCH\_ENABLE signal to low, as shown in figure \ref{aq-timingdiagram}. This drives the output of the AND gate to low which turns OFF $N_1$ and $P_1$. The FPGA then initiates the reset state by setting the RESET signal high to turn ON $N_2$. The GM-APD\_ANODE voltage then dips to 0 which brings the relative bias voltage of the APD above its breakdown voltage. This also makes the comparator output $Q$ to go low and $\overline{Q}$ to go high after which $N_2$ is turned OFF by setting the RESET signal low. The duration of reset is configurable and typically kept as short as possible. To ensure that both $P_1$ and $N_2$ are not ON at the same time, the FPGA inserts a tiny configurable delay between end of the quench and start of the reset. The total duration from the start of the avalanche to the completion of the reset constitutes the $deadtime$ of the system. After reset the QUENCH\_ENABLE signal is driven high again to keep the system ready for the next detection event.

%The microcontroller software can directly configure registers of the FPGA modules as memory mapped entities over the standard AXI bus. For every parameter/register/flag in each FPGA module, there is a corresponding variable in the software. These can be initialized and changed by the microcontroller software at run-time. The microcontroller also provides a bi-directional $command-response$ interface and connects with a PC (laptop/desktop) using a serial communication port. So the active quench parameters (such as quench, reset, delay, deadtime, counts, etc.) can be accessed during the experiment from the PC either directly through user-entered commands or by automated scripts which can read and parse these commands from a configuration file. This also enables for high precision monitoring and control of all APDs used during the experiment to ensure that the parameters like quantum bit error rate (QBER) is within agreeable limits at all times.

%In addition, the microcontroller software periodically reads the pulse counter on the FPGA to keep track of the number of detection events per second. This is further used for characterization parameters such as dark counts, linearity, etc. of the APD. A copy of the APD\_PULSE is also fed into a separate (off-chip) pulse trimming circuit comprising of a D-Flip Flop  These pulses can be either fed into an oscilloscope  



The operation of our system is verified with a commercial SAP500-TO8 APD \cite{sap500}, biased with an excess voltage of 10V. V\_QUENCH is set to 15V, which potentially allows to operate the APD up to an excess voltage 12V. Figure \ref{fig:aq-anode} shows the GM-APD\_ANODE voltage (C2) captured with a LeCroy Waverunner 610Zi oscilloscope. After the avalanche starts, the activation of quench occurs at $\approx$9ns. Referring to figure \ref{aq-design}, this delay is incurred in the quench feedback path which is the sum of propagation delays through the comparator, AND gate and the switch ON delays for $N_1$ and $P_1$. After quench is activated, the GM-APD\_ANODE voltage rises rapidly to V\_QUENCH within 2ns. The OUT signal (C1) is delayed by $\approx$6ns due to the propagation delays through the comparator, inverting buffer and D-Flipflop (D-FF). 

The minimum quench duration is then decided by the propagation delay of the APD\_PULSE signal from the comparator through the FPGA logic to drive QUENCH\_ENABLE signal and AND gate output to low. In our case this is 10ns despite the physical separation of $\approx$15cm between the APD head and the SoC. The delay between quench and reset is typically set to 5ns to account for switch OFF delays of $N_1$ and $P_1$. So in figure \ref{fig:aq-anode} it can be observed that the reset is activated 15ns after GM-APD\_ANODE reaches V\_QUENCH. The reset pulse duration is kept as small as possible and is typically between 5ns-10ns depending on the value of V\_QUENCH. Table \ref{tab:reconfig-param} shows the list of configurable parameters in our design along with their range. The FPGA module for quench/reset generation runs on an internal 200MHz clock which allows to increase/decrease the timing parameters with a minimum step size of 5ns.
\begin{table}[ht]
\caption{Reconfigurable parameters and their range of supported values} 
\label{tab:reconfig-param}
\begin{center}       
\begin{tabular}{|l|c|c|c|}
\hline
\rule[-1ex]{0pt}{3.5ex}  & Min. & Max. & Units \\
\hline
\rule[-1ex]{0pt}{3.5ex} Quench duration & 10 & 1000 & ns  \\
\hline
\rule[-1ex]{0pt}{3.5ex} Reset duration & 5 & 1000 & ns  \\
\hline
\rule[-1ex]{0pt}{3.5ex} Deadtime & 35 & 1000 & ns  \\
\hline
\rule[-1ex]{0pt}{3.5ex} Quench voltage & 0 & 30 & V  \\
\hline
\rule[-1ex]{0pt}{3.5ex} Bias voltage & 0 & 500 & V  \\
\hline 
\end{tabular}
\end{center}
\end{table}
\subsection{Operation in PQAR and PQ configurations}
Our system operates in PQAR mode when the FPGA is configured to keep the QUENCH\_ENABLE signal low at all times. This disables the AND gate shown in figure \ref{aq-design} and the QUENCH signal is never triggered, keeping $N_1$ and $P_1$ always OFF. So when the APD avalanche occurs it is quenched passively through ballast resistor $R_B$ for a designated duration, after which the FPGA only drives the RESET signal to high and turns ON $N_2$. This restores the relative reverse bias voltage of the GM-APD to V\_BIAS. For passive quenching (PQ) mode, the FPGA is configured to keep both the QUENCH\_ENABLE and RESET signals low at all times. As a result $P_1,\ N_1$ and $N_2$ are never turned ON and so the APD is both quenched and reset passively. However in both modes the comparator still continues to detect APD avalanches and toggles the OUT and APD\_PULSE signals which can be counted by the FPGA.
\section{Performance characterization with commercial APD}
We characterize our system with the SAP500-T8 APD which has a built-in TEC (Thermo-electric cooler) to vary the temperature. For this APD, we present results for the lowest achievable values for deadtime, after-pulsing and operating temperature at an excess voltage of 10V for any given temperature.

%Reducing APD temperature is helpful in reducing dark counts that occur due to thermal excitation. However reducing it too much increases the after-pulsing probability. To reduce after-pulsing, the quench duration of the APD needs to be increased but this increases the deadtime and in-turn reduces the linearity and detection efficiency of the system.   
\begin{figure} [ht]
\begin{center}
\begin{tabular}{c} 
\includegraphics[width=\textwidth]{images/HybridAQ-Deadtime35ns7.jpg}\\
\end{tabular}
\end{center}
\caption[] 
{ \label{fig:deadtime} 
Deadtime of 35ns is observed with accumulated OUT pulses (C1) captured on the oscilloscope in persistent mode at a background rate of $\approx$66Kcps}
\end{figure} 

\subsection{Deadtime}
Once an avalanche is detected, the APD cannot detect another avalanche until the first avalanche is completely quenched and the APD is reset back to its original state. This constitutes the deadtime and as shown in figure \ref{aq-timingdiagram}. 
%An important consideration in our system is the physical separation of $\approx$15cm between the APD head and the quench/reset generator logic block which is located on the SoC.To ensure this does not impact the deadtime, 
We capture the accumulated OUT pulses on the oscilloscope in persistent mode. It can be observed in figure \ref{fig:deadtime} that the deadtime (time interval between the rising edge of the first pulse and the rising edge of the earliest subsequent pulse) is 35ns which in-principle allows for detection rates of $>$28Mcps with our system. 

\subsection{After-pulsing probability and temperature}
%When the SAP500-T8 APD is operated at room temperatures 20°C to 25°C (293K-298K), the dark counts (due to thermal excitation) are higher than 20Kcps for an over-voltage of 10V. For the same over-voltage but at lower temperatures, the dark counts measured are much lesser: $\approx$5Kcps at 0°C (273K) and $\approx$4Kcps at -10°C (263K). 


%which happens when avalanche charge during a photon detection get trapped in localized defects within the APD. These charge carriers are released immediately after resetting the APD above its breakdown voltage, thus triggering another 'false' avalanche.
%This second avalanche is called an 'after-pulse' and does not occur due to an actual photon detection and such pulses are detrimental to the detection efficiency and deadtime of the system. Other than the effectiveness of the quenching circuit, the probability of after-pulsing also depends on the APD characteristics. For instance having a higher over-voltage is desirable for better detection efficiency of the APD but after-pulsing increases with the increase in over-voltage.  
 
\begin{figure} [ht]
\begin{center}
%\begin{tabular}{p{0.5\textwidth}>{c}p{0.5\textwidth}>{c}}  
%\includegraphics[height=5cm]{images/SPIE2-Min10AP-Zynq-43k-1k-17mV-35ns.pdf} & \includegraphics[height=4.65cm]{images/IPS2-AP-Zynq-43k-1k-17mV-35ns.pdf} 
\begin{tabular}{c} 
\includegraphics[width=0.5\textwidth]{images/SPIE1-Min20AP-Zynq-43k-1k-17mV-35ns.pdf}
\includegraphics[width=0.5\textwidth]{images/SPIE2-AP-Zynq-43k-1k-17mV-35ns.pdf}
\end{tabular}
%\begin{tabular}{p{0.5\textwidth}>{c}p{0.5\textwidth}>{c}}
\begin{tabularx}{\linewidth}{>{\centering\arraybackslash}X >{\centering\arraybackslash}X}
    (a) & (b) \\
\end{tabularx}
%\end{tabular}
\end{center}
\caption[] 
{ \label{fig:afterpulse} 
(a) shows the second order auto-correlation $g^{(2)}$ of OUT pulses when recorded with a time-tagger of 2ns time resolution. The number of after-pulses (highlighted) are higher immediately after the deadtime $Dt$ of 35ns (vertical red line) and taper off to reach a floor (horizontal blue line) after $\approx$200ns. (b) shows the variation of after-pulsing probability at different temperatures and deadtime. 
%The increase in deadtime is achieved by only increasing the quench duration while keeping other parameters constant
}
\end{figure}

It is beneficial to operate the APD at lower temperatures as it reduces dark counts (occurring due to thermal excitation) but doing so increases after-pulsing. The probability of after-pulsing also increases when the deadtime of the circuit is reduced. We measure the after-pulsing probability for the SAP500-T8 using our system for different deadtime and temperature. This is done by computing the second order auto-correlation $g^{(2)}$ of the OUT pulses using an external time-tagger module (with 2ns time resolution). One such instance is shown in figure \ref{fig:afterpulse}(a) where the deadtime ($Dt$) is 35ns, the APD temperature is 253K (-20°C) and over-voltage is 10V. The background rate is $\approx$30Kcps and the timestamps are recorded for 10 seconds which gives more than 300,000 data samples ($C_{Total}$) of arrival times in total. The $g^{(2)}$ auto-correlation is then computed at a time resolution of 2ns. In figure \ref{fig:afterpulse}(a) it can be seen that there is a high degree of correlation of arrival times immediately after the deadtime and the counts here constitute after-pulses as highlighted in the graph. After reaching a peak the counts drop exponentially until they flatten out to an average value which we call `floor', beyond which the correlation is weak or nil.
%. The flat region actually continues until 10$\mu$s, but only until 1$\mu$s is shown in figure \ref{fig:afterpulse}(a) because the correlation is weak or nil beyond $\approx$200ns and so these pulses do not count as after-pulses. 
Theoretically, the after-pulse probability $P_{ap}$\cite{Ceccarelli2019} is defined as:
\begin{equation}
    %P_{ap} = \frac{\sum_{\tau=Dt}^{+\infty}(C_{\tau}-floor)}{\sum_{\tau=1}^{\tau=T}C_{\tau}}
    P_{ap} = \frac{\sum_{\tau=Dt}^{+\infty}(C_{\tau}-floor)}{C_{Total}}
    \label{eq:afterpulse} 
\end{equation}
where $C_{Total}$ is the total number of events and $C_{\tau}$ is the count value at each time bin $\tau$. Using this approach, the percentage of after-pulsing probability for SAP500-T8 obtained with our system is shown in figure \ref{fig:afterpulse}(b) for four operating temperatures (-20°C, -10°C, 0°C and +20°C). The deadtime is varied in steps of 5ns by changing only the quench duration while keeping all other parameters constant. W.r.t. Eqn. \ref{eq:afterpulse}, for our data we consider $Dt$ as time bin of the first observed correlation event and 10$\mu$s as the upper limit for $\tau$. It is observed that $P_{ap}$ at the lowest temperature (-20°C) and lowest deadtime (35ns) is 2.95$\pm$0.08\% and remains below this value for all other settings of deadtime and temperature. For most practical applications such as QKD, it is desirable to keep the after-pulsing probability below 5\% \cite{Stipcevic17, Ceccarelli2019}.

%Linearity:
%The advantage with active quench circuits is their high detection rate and it is crucial that the circuit detects most of the incident photons. This is termed as linearity and is typically the ratio of number of detected pulses versus number of incident pulses. In reality, no circuit will be able to detect 100\% of all incident photons, however the higher the number the better is the linearity. The linearity of our implementation was tested with an optical set up that generates correlated photon pairs using Spontaneous parametric down-conversion [3] and the linearity was found to be around 87\% for a detected rate of 1M counts per second. This means that our system could detect 1 million of the 1.15 million incident photons.



 
\section{Integration with custom chip-scale APD}
\begin{comment}
\begin{figure} [ht]
\begin{center}
\begin{tabular}{c} 
%\includegraphics[height=7.5cm]{images/ASTARLab-Setup8.jpg}
\includegraphics[width=\textwidth]{images/APD-arch-ASTAR-1.jpg}
\end{tabular}
\begin{tabular}{p{0.5\textwidth}>{c}p{0.5\textwidth}>{c}}
    (a) & (b) \\
\end{tabular}
\end{center}
\caption[] 
{ \label{fig:astar-apdarch} 
Two physical architectures of our custom chip-scale APD. (a) the wire-bonded package connects 8 out of the 16 on-chip APDs to pads which come out as pins of a standard DIP (dual-in-line package). (b) shows the bare chip with out any external packaging or interconnect. The N and P pads of the APDs can be connected directly with RF probe tips. The thin horizontal lines in both cases are optical waveguides for coupling light into the APDs.}
\end{figure} 
\begin{figure} [ht]
\begin{center}
\begin{tabular}{c} 
%\includegraphics[height=7.5cm]{images/ASTARLab-Setup8.jpg}
\includegraphics[width=\textwidth]{images/DIP-pkg-ASTAR-1.jpg}
\end{tabular}
\begin{tabular}{p{0.33\textwidth}>{c}p{0.33\textwidth}>{c}p{0.33\textwidth}>{c}}
    (a) & & (b) \\
\end{tabular}
\end{center}
\caption[] 
{ \label{fig:astar-apddip} 
(a) shows the interface board developed to host and connect the DIP-based custom APD shown in figure \ref{fig:astar-apdarch}(a). The N and P pads of each APD come out into separate 4-pin headers. (b) shows the test setup with the APD interface board connecting APD 'D12' to channel 2 of the custom active quench board shown in figure \ref{fig:aq-pcb}, which in-turn connects with the commercial SoC board}
\end{figure}
\begin{figure} [ht]
\begin{center}
\begin{tabular}{c} 
%\includegraphics[height=7.5cm]{images/ASTARLab-Setup8.jpg}
\includegraphics[width=\textwidth]{images/ASTARLab-Setup10.jpg}
\end{tabular}
\end{center}
\caption[] 
{ \label{fig:astar-setup} 
The lab set up at IMRE showing the active quench system integrated directly with the chip-scale APD. Inset (i) shows the RF probe interface board used to connect the APDs with Channel 2 of custom active quench board and the SoC. RF probe tips [inset (ii)] connect with the APD pads on the chip and relay the signal via RF probes to the interface board. The APD chip sits on a TEC so it can be cooled down to -20°C. A commercial SAP500-T8 is connected to Channel 1 of the custom board and both channels are operated simultaneously during the experiments}
\end{figure} 
\end{comment}


%\mathbf{*** Inputs needed from IMRE with references/citations - brief description of APD w.r.t. figure \ref{fig:astar-apdarch} - typical breadkdown voltage range (16V-25V?), over-voltage (2V-9V?) for the APDs - device types and approximate number of device variants ***}

\begin{comment}
Prior to testing with the active quench system, the custom APDs had been tested in Geiger mode with a passive quench system and almost all APDs displayed an anomalous behavior w.r.t. the breakdown voltage. For any fixed bias voltage above the APD breakdown voltage, it was observed that the breakdown voltage of the APDs increased during operation and within a few seconds it reached close to the bias voltage, thus reducing the over-voltage to almost 0. Consequently the detected counts also started reducing during the operation due to this effect until it eventually reached 0. Increasing the bias voltage at this point temporarily restores the counts until the breakdown voltage again increases to reach the new bias voltage and the counts again come down to 0 within a few seconds. This is a long-term effect in which the breakdown voltage in not restored even if the APD is kept powered off for several hours. The only way to restore the breakdown voltage to the original value is to manually forward bias the APD with 1V-2V for a few seconds.

\subsection{Nullifying the breakdown voltage anomaly in Geiger mode}
Initial testing was done with the wire-bonded DIP version of the custom APDs. As shown in figure \ref{fig:astar-apddip}(a), we developed a simple interface board which hosts the DIP on a zero-insertion-force (ZIF) socket. It brings out the P and N pins of each APD into a standard 4-pin header so that any of the 8 APDs can be easily connected with the active quench system in a plug-and-play manner. Figure \ref{fig:astar-apddip}(b) shows the test setup in which the APD 'D12' is connected to channel 2 of the custom active quench board shown in figure \ref{fig:aq-pcb}. Test results indicate that the breakdown voltage drift could be prevented with the active quench system depending on the values of bias voltage V\_BIAS and V\_QUENCH (indicated in figure \ref{aq-design}). We observe that when V\_QUENCH$\geq$V\_BIAS, the count rate never dips even when the APD was operated for several minutes continuously. Our hypothesis for this is that, when an avalanche occurs and quenching begins, the APD is forward biased for the entire quench duration under the condition V\_QUENCH$\geq$V\_BIAS which prevents the breakdown voltage of the APD from drifting.
\end{comment}

\begin{figure} [ht]
\begin{center}
\begin{tabular}{c} 
\includegraphics[height=8cm]{images/IPS3-Vbr-DarkcountVsTemperature-23Sep21.pdf}
\end{tabular}
\end{center}
\caption[] 
{ \label{fig:astar-data} 
Variation of breakdown voltage and dark counts w.r.t. temperature for one of the chip-scale APDs in Geiger mode, obtained using our SoC based active quench system. Both data plots were obtained in three consecutive phases in which the APD was first cooled down to -20°C, then warmed up to +50°C and again cooled back to +20°C}
\end{figure} 
After validating the performance of our active quench system with a widely used commercial APD (SAP500-T8), we further use it for testing our custom fabricated chip-scale APDs\cite{imreAPD21} in Geiger mode. These are Si waveguide APDs integrated with on-chip photonic waveguides and are interfaced with the active quench system using RF probes. The active quench system shown in figure \ref{aq-design} was modified to include a second identical active quench channel so that both the SAP500 and custom chip-scale APD can be simultaneously connected and operated in parallel. Both GM-APDs are controlled by only one SoC through replication of the quench/reset generator and the counter modules on the FPGA. The SAP500 was operated at room temperature while the temperature of the chip-scale APD was varied between -20°C and +50°C using a TEC. To rule out the presence of any stray light in the setup, the count rate of the SAP500 was monitored closely during the experiments to ensure that the dark counts always remained at $\approx$20Kcps, which is the expected value for this APD at room temperature when operated at an excess voltage of 10V. The breakdown voltage of the chip-scale APDs are typically between 16V to 25V and their V\_BIAS is derived from a bench-top power supply. To  prevent the breakdown voltage of the chip-scale APDs from drifting \cite{imreAPD21}, V\_QUENCH was maintained between 18V to 27V in conjunction with V\_BIAS, so as to always keep it 2V above V\_BIAS. Although V\_QUENCH for both APDs was derived from the same bench-top power supply, any changes in V\_QUENCH does not affect the operation of the SAP500 since V\_QUENCH is always greater than 15V for all tests. Considering these high voltage values, the durations of quench, reset and deadtime for both channels were set conservatively higher so as to provide sufficient time for the MOSFETs to switch ON/OFF.
%Having a bigger quench duration also ensures the after-pulsing of the APDs stays quite low even at lower temperatures. 
So for both channels, the quench duration was set to 25ns, the reset duration to 15ns and a delay of 10ns between quench and reset, giving a total deadtime of $\approx$65ns (inclusive of initial response time and all other delays) for all experiments. V\_QUENCH and V\_BIAS of the chip-scale APD were changed at run-time using scripts running on a laptop PC, while the other parameters in Table \ref{tab:reconfig-param} remained fixed.

%which can be configured at run-time by commands from the PC over a USB interface. This enabled to automate the process of varying both voltages using scripts running on the PC. The associated driver software for the TEC module was used for changing the temperature, which was also configurable through scripts on the PC. Using the described setup we measured breakdown voltage and dark counts of the chip-scale APD for varying temperatures. 
Figure \ref{fig:astar-data} shows variation in dark count and breakdown voltage w.r.t. temperature for one chip-scale APD. To check for hysteresis in breakdown voltage, the temperature of the APD was changed in three continuous phases. It was first cooled from +20°C to -20°C, then warmed up from -20°C to +50°C and again cooled down from +50°C to +20°C. In each phase temperature was changed in steps of 10°C. At each temperature, V\_BIAS was initially set to a value well below the breakdown voltage and the detected counts were ensured to be 0. V\_BIAS was then increased in steps of 0.05V every 2 seconds until the SoC started measuring counts, which indicated the breakdown voltage. V\_BIAS was then kept constant and the dark counts were recorded. Before each new measurement the APD was reset by switching off V\_BIAS. 
%At the initial temperature of +20°C, the breakdown voltage for this device was measured to be 16.1V and the dark count obtained was $\approx$1.57Mcps. 
It can be observed that the variation of breakdown voltage and dark count w.r.t. temperature follows a well known trend as seen with existing commercial devices.

\subsection{Advantage of using SoC based active quenching for testing chip-scale APDs}
For preliminary tests, only dark counts and breakdown voltage of one chip-scale APD have been recorded. Future tests will involve characterizing the after-pulsing probability (like shown in figure \ref{fig:afterpulse}) of each APD for at least 10 different deadtimes, 5 different temperatures and 3 different excess voltages, which amounts to a minimum of 150 test cases per APD. There are more than 400 variants of our custom chip-scale APDs with each differing in channel length, width, doping concentration etc. So characterizing after-pulse probability for all of them requires 60,000 tests in total. This means the active quench system needs to be re-configured 60,000 times for executing the complete test set. Using hard-wired fixed active quench circuits for this purpose is impractical since it requires a lot of manual intervention for changing parameters, which is both laborious and error prone. Using our SoC based active quench system instead paves way for making these adjustments instantly with script-based automation thus saving both cost and time.

%are fabricated on a single chip containing several variants in which each APD differs from the other in terms of channel length, width, doping concentration, etc. As shown in figure \ref{fig:astar-apdarch}, there exist two physical architectures of our custom chip-scale APDs -  those with wire-bonded dual-in-line packaging (DIP) and those which are bare without any packaging and interconnect, .  
%This is especially true for future testing with our custom chip-scale APDs, for which there exist several hundred variants with each one different from the other in terms of channel length, width, doping concentration, etc.  This results in an exponential increase in the total number of test cases and  
\section{Conclusion}
\label{sec:conclusion}
This paper presents a generic top-$\size$ recommendation framework for  trading-off accuracy, novelty, and coverage. To achieve this, we profile the users according to their preference for long-tail novelty. We examine various measures, and formulate an optimization problem to learn these user preferences from interaction data.  We then integrate the user preference estimates in our generic framework, GANC.  Extensive experiments on several datasets confirm that there are trade-offs between accuracy, coverage, and novelty. Almost all re-ranking models increase coverage and novelty at the cost of accuracy. However, existing re-ranking models typically rely on rating prediction models, and are hence more effective in dense settings. Using a generic approach, we can easily incorporate a suitable base accuracy recommender to devise an effective solution for both sparse and dense settings.  %Our results  also indicate there is no single method that outperforms other methods in all metrics. However our techniques obtain a significant improvement in coverage, while  . 
Although we integrated the  long-tail novelty preference estimates into a re-ranking framework, their use-case is not limited to these frameworks. In  the future, we intend to explore the temporal and topical dynamics of long-tail novelty preference, particularly in settings where contextual information is  available.  
%We achieve these objectives without using any additional contextual information.


\iffalse
While we focused on promoting long-tail items across users, we did not consider diversity of individual top-$\size$ recommendations, a factor that has been shown to positively affect consumer satisfaction. This is one direction for future work. Moreover, the sequential setting  in our work, creates a dependency between different batches, where,  the items recommended to a batch of users, depends on those recommended to previous batches. This dependency is created through the parameter $\mathbf{f}$, that is updated every time a top-$\size$ set  is allocated to a batch of users. A future direction for our work is to estimate a distribution over $\mathbf{f}$ that allows us to independently solve the problem for each user, leading to improvements across all performance metrics, including recommendation time. 

We design algorithms that take advantage of the structure in the value functions to obtain both efficient and scalable solutions. 
We design an algorithm that takes advantage of the structure in the value functions to obtain both efficient and scalable solutions. 

\textcolor{red}{Our  sequential  algorithms can be applied for batch recommendation contexts,~e.g., personalized email marketing, where based on prior interaction data between users and items,  a new round of recommendations must be sent to all users in the system.  However, the independent coverage algorithms lift the sequential setting restrictions and allow it be applied for re-ranking the output of base recommender in any setting. }A future direction for our work is to incorporate explicit diversity metrics in the framework. 
\fi


%We have a presented a submodular maximization framework to systematically trade-off relevance and diversity in recommendations to individual users and coverage across the item-space. This ensures both consumer and producer satisfaction. We model users according to their risk and focusing degrees and promote long-tail items to the right group of consumers. Consequently, we obtain a significant improvement in coverage while maintaining reasonable levels of user satisfaction. Furthermore, our methods are able to achieve a more balanced distribution across the set of recommended items. In the future, we plan to investigate the effect of using alternative base recommender systems. 

%Future Work
%However most of these methods assume that the ratings are missing at random (MAR). Since our method of generating recommendations is based on the completed matrix, assuming MAR might introduce additional bias, we will use methods which assume that the ratings at missing not at random (MNAR),explored in~\cite{steck2010training, icml2014c2_hernandez-lobatob14}. 	 
%Long Tail %Recently, authors in~\cite{cremonesi2010performance} conducted extensive experiments to evaluate the performances of various matrix factorization-based algorithms and neighborhood models on the task of recommending long tail items. Their experimental results show that long tail recommendation leads to a decrease in accuracy for all algorithms. They also showed that for this task, SVD outperforms other state-of-the-art algorithms. 



% References
\bibliography{Bib-ReconfAQ} % bibliography data in report.bib
\bibliographystyle{spiebib} % makes bibtex use spiebib.bst

\end{document} 
