\section{Design and implementation}
\label{sec:design}
Similar to conventional active quenching circuits, our system operates in two phases in response to the occurrence of an avalanche: 1) $quench$ - reduce the bias voltage of the diode below its breakdown voltage for a few nanoseconds and 2) $reset$ - restore the bias voltage above the breakdown for a few nanoseconds while bypassing the ballast resistor at the same time. After the reset phase, the ballast resistor is reconnected and the APD returns to the nominal state. The main functional blocks of our system are shown in figure \ref{aq-design}. The $ZedBoard$ \cite{zedboard} development kit with a Zynq-7000 based SoC is used for realizing the FPGA and microcontroller modules. The GM-APD is connected with the quench/reset generator on the FPGA using a custom PCB with minimal discrete components. We use the LT6752-3 high speed comparator (COMP) which produces complementary outputs after detecting the avalanche pulse across sense resistor $R_S$. The feedback path for doing the quench comprises of a fast AND gate, P-MOSFET $P_1$ and N-MOSFET $N_1$. Reset is done through N-MOSFET $N_2$. The values of $R_4$ and $C_1$ are set such that the D-FlipFlop (D-FF) trims the comparator output to 20ns pulses (OUT) which are used for characterizing the deadtime and after-pulsing probability. Matsusada high voltage DC-DC converters are used to derive a bias voltage (V\_BIAS) between 0-500V while quench voltage V\_QUENCH is derived from an external power supply.
\begin{figure} [ht]
\begin{center}
\begin{tabular}{c} 
%\includegraphics[height=6cm]{images/HybridAQ-Design8.jpg}
\includegraphics[width=\textwidth]{images/HybridAQ-Design8.jpg}
\end{tabular}
\end{center}
\caption[] 
{ \label{aq-design} 
Simplified circuit diagram of the system with Geiger Mode APD (GM-APD)}
\end{figure}
\begin{figure} [ht]
\begin{center}
\begin{tabular}{c} 
%\includegraphics[height=5cm]{images/HybridAQ-TimingDiagram6.jpg}
\includegraphics[width=0.65\textwidth]{images/HybridAQ-TimingDiagram6.jpg}
\end{tabular}
\end{center}
\caption[] 
{ \label{aq-timingdiagram} 
Timing diagram of critical signals showing the active quench operation after occurrence of an avalanche}
\end{figure}
\subsection{Operation in active quench configuration}
\begin{figure} [ht]
\begin{center}
\begin{tabular}{c} 
%\includegraphics[height=7cm]{images/HybridAQ-Schematic3.jpg}
\includegraphics[width=\textwidth]{images/HybridAQ-APDAnode1.jpg}
\end{tabular}
\end{center}
\caption[] 
{ \label{fig:aq-anode} 
Waveform of the OUT pulse (C1) and GM-APD\_ANODE voltage (C2) captured on the oscilloscope}
\end{figure}
The design and operation is better understood with the timing diagram shown in figure \ref{aq-timingdiagram}. The avalanche current in the APD causes the voltage at the APD anode (GM-APD\_ANODE) to rise and consequently the voltage across $R_S$ exceeds the comparator reference voltage (V\_REF), which is typically set to $\approx$17mV. In response the comparator (COMP) toggles the complementary output pulses $Q$ to high and $\overline{Q}$ to low. Output $Q$ is fed back to one of the inputs of the AND gate while the FPGA keeps the other input of the AND gate (QUENCH\_ENABLE) high before the avalanche occurs. So the AND gate output (QUENCH) goes high to turn ON $N_1$ which subsequently turns ON $P_1$ to initiate quenching. As a result of this, the anode voltage of GM-APD reaches the quench voltage (V\_QUENCH) which effectively reduces the voltage across the APD below its breakdown voltage and brings it into the quench state. Until then the APD is passively quenched and the avalanche current is limited by the ballast resistor ($R_B$). This feedback path is kept as short as possible to reduce the response time and after-pulsing probability\cite{Wayne-AP}.
%Rb=43k and Rs=1k

The other comparator output $\overline{Q}$ is inverted (APD\_PULSE) and fed into the quench/reset logic to register the occurrence of an avalanche and into a 24-bit counter to record the total number of avalanches per second. The duration for which the APD stays in the quench state is configurable and after this duration the FPGA sets the QUENCH\_ENABLE signal to low, as shown in figure \ref{aq-timingdiagram}. This drives the output of the AND gate to low which turns OFF $N_1$ and $P_1$. The FPGA then initiates the reset state by setting the RESET signal high to turn ON $N_2$. The GM-APD\_ANODE voltage then dips to 0 which brings the relative bias voltage of the APD above its breakdown voltage. This also makes the comparator output $Q$ to go low and $\overline{Q}$ to go high after which $N_2$ is turned OFF by setting the RESET signal low. The duration of reset is configurable and typically kept as short as possible. To ensure that both $P_1$ and $N_2$ are not ON at the same time, the FPGA inserts a tiny configurable delay between end of the quench and start of the reset. The total duration from the start of the avalanche to the completion of the reset constitutes the $deadtime$ of the system. After reset the QUENCH\_ENABLE signal is driven high again to keep the system ready for the next detection event.

%The microcontroller software can directly configure registers of the FPGA modules as memory mapped entities over the standard AXI bus. For every parameter/register/flag in each FPGA module, there is a corresponding variable in the software. These can be initialized and changed by the microcontroller software at run-time. The microcontroller also provides a bi-directional $command-response$ interface and connects with a PC (laptop/desktop) using a serial communication port. So the active quench parameters (such as quench, reset, delay, deadtime, counts, etc.) can be accessed during the experiment from the PC either directly through user-entered commands or by automated scripts which can read and parse these commands from a configuration file. This also enables for high precision monitoring and control of all APDs used during the experiment to ensure that the parameters like quantum bit error rate (QBER) is within agreeable limits at all times.

%In addition, the microcontroller software periodically reads the pulse counter on the FPGA to keep track of the number of detection events per second. This is further used for characterization parameters such as dark counts, linearity, etc. of the APD. A copy of the APD\_PULSE is also fed into a separate (off-chip) pulse trimming circuit comprising of a D-Flip Flop  These pulses can be either fed into an oscilloscope  


