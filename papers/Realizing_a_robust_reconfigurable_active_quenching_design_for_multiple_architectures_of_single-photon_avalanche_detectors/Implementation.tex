The operation of our system is verified with a commercial SAP500-TO8 APD \cite{sap500}, biased with an excess voltage of 10V. V\_QUENCH is set to 15V, which potentially allows to operate the APD up to an excess voltage 12V. Figure \ref{fig:aq-anode} shows the GM-APD\_ANODE voltage (C2) captured with a LeCroy Waverunner 610Zi oscilloscope. After the avalanche starts, the activation of quench occurs at $\approx$9ns. Referring to figure \ref{aq-design}, this delay is incurred in the quench feedback path which is the sum of propagation delays through the comparator, AND gate and the switch ON delays for $N_1$ and $P_1$. After quench is activated, the GM-APD\_ANODE voltage rises rapidly to V\_QUENCH within 2ns. The OUT signal (C1) is delayed by $\approx$6ns due to the propagation delays through the comparator, inverting buffer and D-Flipflop (D-FF). 

The minimum quench duration is then decided by the propagation delay of the APD\_PULSE signal from the comparator through the FPGA logic to drive QUENCH\_ENABLE signal and AND gate output to low. In our case this is 10ns despite the physical separation of $\approx$15cm between the APD head and the SoC. The delay between quench and reset is typically set to 5ns to account for switch OFF delays of $N_1$ and $P_1$. So in figure \ref{fig:aq-anode} it can be observed that the reset is activated 15ns after GM-APD\_ANODE reaches V\_QUENCH. The reset pulse duration is kept as small as possible and is typically between 5ns-10ns depending on the value of V\_QUENCH. Table \ref{tab:reconfig-param} shows the list of configurable parameters in our design along with their range. The FPGA module for quench/reset generation runs on an internal 200MHz clock which allows to increase/decrease the timing parameters with a minimum step size of 5ns.
\begin{table}[ht]
\caption{Reconfigurable parameters and their range of supported values} 
\label{tab:reconfig-param}
\begin{center}       
\begin{tabular}{|l|c|c|c|}
\hline
\rule[-1ex]{0pt}{3.5ex}  & Min. & Max. & Units \\
\hline
\rule[-1ex]{0pt}{3.5ex} Quench duration & 10 & 1000 & ns  \\
\hline
\rule[-1ex]{0pt}{3.5ex} Reset duration & 5 & 1000 & ns  \\
\hline
\rule[-1ex]{0pt}{3.5ex} Deadtime & 35 & 1000 & ns  \\
\hline
\rule[-1ex]{0pt}{3.5ex} Quench voltage & 0 & 30 & V  \\
\hline
\rule[-1ex]{0pt}{3.5ex} Bias voltage & 0 & 500 & V  \\
\hline 
\end{tabular}
\end{center}
\end{table}
\subsection{Operation in PQAR and PQ configurations}
Our system operates in PQAR mode when the FPGA is configured to keep the QUENCH\_ENABLE signal low at all times. This disables the AND gate shown in figure \ref{aq-design} and the QUENCH signal is never triggered, keeping $N_1$ and $P_1$ always OFF. So when the APD avalanche occurs it is quenched passively through ballast resistor $R_B$ for a designated duration, after which the FPGA only drives the RESET signal to high and turns ON $N_2$. This restores the relative reverse bias voltage of the GM-APD to V\_BIAS. For passive quenching (PQ) mode, the FPGA is configured to keep both the QUENCH\_ENABLE and RESET signals low at all times. As a result $P_1,\ N_1$ and $N_2$ are never turned ON and so the APD is both quenched and reset passively. However in both modes the comparator still continues to detect APD avalanches and toggles the OUT and APD\_PULSE signals which can be counted by the FPGA.