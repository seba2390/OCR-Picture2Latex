\section{Performance characterization with commercial APD}
We characterize our system with the SAP500-T8 APD which has a built-in TEC (Thermo-electric cooler) to vary the temperature. For this APD, we present results for the lowest achievable values for deadtime, after-pulsing and operating temperature at an excess voltage of 10V for any given temperature.

%Reducing APD temperature is helpful in reducing dark counts that occur due to thermal excitation. However reducing it too much increases the after-pulsing probability. To reduce after-pulsing, the quench duration of the APD needs to be increased but this increases the deadtime and in-turn reduces the linearity and detection efficiency of the system.   
\begin{figure} [ht]
\begin{center}
\begin{tabular}{c} 
\includegraphics[width=\textwidth]{images/HybridAQ-Deadtime35ns7.jpg}\\
\end{tabular}
\end{center}
\caption[] 
{ \label{fig:deadtime} 
Deadtime of 35ns is observed with accumulated OUT pulses (C1) captured on the oscilloscope in persistent mode at a background rate of $\approx$66Kcps}
\end{figure} 

\subsection{Deadtime}
Once an avalanche is detected, the APD cannot detect another avalanche until the first avalanche is completely quenched and the APD is reset back to its original state. This constitutes the deadtime and as shown in figure \ref{aq-timingdiagram}. 
%An important consideration in our system is the physical separation of $\approx$15cm between the APD head and the quench/reset generator logic block which is located on the SoC.To ensure this does not impact the deadtime, 
We capture the accumulated OUT pulses on the oscilloscope in persistent mode. It can be observed in figure \ref{fig:deadtime} that the deadtime (time interval between the rising edge of the first pulse and the rising edge of the earliest subsequent pulse) is 35ns which in-principle allows for detection rates of $>$28Mcps with our system. 

\subsection{After-pulsing probability and temperature}
%When the SAP500-T8 APD is operated at room temperatures 20°C to 25°C (293K-298K), the dark counts (due to thermal excitation) are higher than 20Kcps for an over-voltage of 10V. For the same over-voltage but at lower temperatures, the dark counts measured are much lesser: $\approx$5Kcps at 0°C (273K) and $\approx$4Kcps at -10°C (263K). 


%which happens when avalanche charge during a photon detection get trapped in localized defects within the APD. These charge carriers are released immediately after resetting the APD above its breakdown voltage, thus triggering another 'false' avalanche.
%This second avalanche is called an 'after-pulse' and does not occur due to an actual photon detection and such pulses are detrimental to the detection efficiency and deadtime of the system. Other than the effectiveness of the quenching circuit, the probability of after-pulsing also depends on the APD characteristics. For instance having a higher over-voltage is desirable for better detection efficiency of the APD but after-pulsing increases with the increase in over-voltage.  
 
\begin{figure} [ht]
\begin{center}
%\begin{tabular}{p{0.5\textwidth}>{c}p{0.5\textwidth}>{c}}  
%\includegraphics[height=5cm]{images/SPIE2-Min10AP-Zynq-43k-1k-17mV-35ns.pdf} & \includegraphics[height=4.65cm]{images/IPS2-AP-Zynq-43k-1k-17mV-35ns.pdf} 
\begin{tabular}{c} 
\includegraphics[width=0.5\textwidth]{images/SPIE1-Min20AP-Zynq-43k-1k-17mV-35ns.pdf}
\includegraphics[width=0.5\textwidth]{images/SPIE2-AP-Zynq-43k-1k-17mV-35ns.pdf}
\end{tabular}
%\begin{tabular}{p{0.5\textwidth}>{c}p{0.5\textwidth}>{c}}
\begin{tabularx}{\linewidth}{>{\centering\arraybackslash}X >{\centering\arraybackslash}X}
    (a) & (b) \\
\end{tabularx}
%\end{tabular}
\end{center}
\caption[] 
{ \label{fig:afterpulse} 
(a) shows the second order auto-correlation $g^{(2)}$ of OUT pulses when recorded with a time-tagger of 2ns time resolution. The number of after-pulses (highlighted) are higher immediately after the deadtime $Dt$ of 35ns (vertical red line) and taper off to reach a floor (horizontal blue line) after $\approx$200ns. (b) shows the variation of after-pulsing probability at different temperatures and deadtime. 
%The increase in deadtime is achieved by only increasing the quench duration while keeping other parameters constant
}
\end{figure}

It is beneficial to operate the APD at lower temperatures as it reduces dark counts (occurring due to thermal excitation) but doing so increases after-pulsing. The probability of after-pulsing also increases when the deadtime of the circuit is reduced. We measure the after-pulsing probability for the SAP500-T8 using our system for different deadtime and temperature. This is done by computing the second order auto-correlation $g^{(2)}$ of the OUT pulses using an external time-tagger module (with 2ns time resolution). One such instance is shown in figure \ref{fig:afterpulse}(a) where the deadtime ($Dt$) is 35ns, the APD temperature is 253K (-20°C) and over-voltage is 10V. The background rate is $\approx$30Kcps and the timestamps are recorded for 10 seconds which gives more than 300,000 data samples ($C_{Total}$) of arrival times in total. The $g^{(2)}$ auto-correlation is then computed at a time resolution of 2ns. In figure \ref{fig:afterpulse}(a) it can be seen that there is a high degree of correlation of arrival times immediately after the deadtime and the counts here constitute after-pulses as highlighted in the graph. After reaching a peak the counts drop exponentially until they flatten out to an average value which we call `floor', beyond which the correlation is weak or nil.
%. The flat region actually continues until 10$\mu$s, but only until 1$\mu$s is shown in figure \ref{fig:afterpulse}(a) because the correlation is weak or nil beyond $\approx$200ns and so these pulses do not count as after-pulses. 
Theoretically, the after-pulse probability $P_{ap}$\cite{Ceccarelli2019} is defined as:
\begin{equation}
    %P_{ap} = \frac{\sum_{\tau=Dt}^{+\infty}(C_{\tau}-floor)}{\sum_{\tau=1}^{\tau=T}C_{\tau}}
    P_{ap} = \frac{\sum_{\tau=Dt}^{+\infty}(C_{\tau}-floor)}{C_{Total}}
    \label{eq:afterpulse} 
\end{equation}
where $C_{Total}$ is the total number of events and $C_{\tau}$ is the count value at each time bin $\tau$. Using this approach, the percentage of after-pulsing probability for SAP500-T8 obtained with our system is shown in figure \ref{fig:afterpulse}(b) for four operating temperatures (-20°C, -10°C, 0°C and +20°C). The deadtime is varied in steps of 5ns by changing only the quench duration while keeping all other parameters constant. W.r.t. Eqn. \ref{eq:afterpulse}, for our data we consider $Dt$ as time bin of the first observed correlation event and 10$\mu$s as the upper limit for $\tau$. It is observed that $P_{ap}$ at the lowest temperature (-20°C) and lowest deadtime (35ns) is 2.95$\pm$0.08\% and remains below this value for all other settings of deadtime and temperature. For most practical applications such as QKD, it is desirable to keep the after-pulsing probability below 5\% \cite{Stipcevic17, Ceccarelli2019}.

%Linearity:
%The advantage with active quench circuits is their high detection rate and it is crucial that the circuit detects most of the incident photons. This is termed as linearity and is typically the ratio of number of detected pulses versus number of incident pulses. In reality, no circuit will be able to detect 100\% of all incident photons, however the higher the number the better is the linearity. The linearity of our implementation was tested with an optical set up that generates correlated photon pairs using Spontaneous parametric down-conversion [3] and the linearity was found to be around 87\% for a detected rate of 1M counts per second. This means that our system could detect 1 million of the 1.15 million incident photons.



 