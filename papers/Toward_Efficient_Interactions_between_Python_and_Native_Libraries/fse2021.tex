%%
%% This is file `sample-sigconf.tex',
%% generated with the docstrip utility.
%%
%% The original source files were:
%%
%% samples.dtx  (with options: `sigconf')
%% 
%% IMPORTANT NOTICE:
%% 
%% For the copyright see the source file.
%% 
%% Any modified versions of this file must be renamed
%% with new filenames distinct from sample-sigconf.tex.
%% 
%% For distribution of the original source see the terms
%% for copying and modification in the file samples.dtx.
%% 
%% This generated file may be distributed as long as the
%% original source files, as listed above, are part of the
%% same distribution. (The sources need not necessarily be
%% in the same archive or directory.)
%%
%% The first command in your LaTeX source must be the \documentclass command.
%\documentclass[sigconf]{acmart}

\documentclass[sigconf,screen]{acmart}
\pdfpagewidth=8.5in
\pdfpageheight=11in



\acmConference[ESEC/FSE 2021]{The 29th ACM Joint European Software Engineering Conference and Symposium on the Foundations of Software Engineering}{23 - 27 August, 2021}{Athens, Greece}


% remove bottom copyright
%\settopmatter{printacmref=false} % Removes citation information below abstract
%\renewcommand\footnotetextcopyrightpermission[1]{} % removes footnote with conference information in first column
%\pagestyle{plain} % removes running headers


%remove doi
%\makeatletter
%\renewcommand\@formatdoi[1]{\ignorespaces}
%\makeatother

%%
%% \BibTeX command to typeset BibTeX logo in the docs
\AtBeginDocument{%
  \providecommand\BibTeX{{%
    \normalfont B\kern-0.5em{\scshape i\kern-0.25em b}\kern-0.8em\TeX}}}

%% Rights management information.  This information is sent to you
%% when you complete the rights form.  These commands have SAMPLE
%% values in them; it is your responsibility as an author to replace
%% the commands and values with those provided to you when you
%% complete the rights form.

%\settopmatter{printacmref=false}
\setcopyright{none}
%\renewcommand\footsnotetextcopyrightpermission[1]{}
%\pagestyle{plain}


% \setcopyright{acmcopyright}
% \copyrightyear{2018}
% \acmYear{2018}
% \acmDOI{10.1145/1122445.1122456}

%% These commands are for a PROCEEDINGS abstract or paper.
% \acmConference[Woodstock '18]{Woodstock '18: ACM Symposium on Neural
%   Gaze Detection}{June 03--05, 2018}{Woodstock, NY}
% \acmBooktitle{Woodstock '18: ACM Symposium on Neural Gaze Detection,
%   June 03--05, 2018, Woodstock, NY}
% \acmPrice{15.00}
% \acmISBN{978-1-4503-XXXX-X/18/06}


\newcommand{\tool}{\textsc{PieProf}\xspace}
\newcommand{\ff}{native function call\xspace}
\newcommand{\ffs}{native function calls\xspace}
\newcommand{\yu}[1]{\textcolor{blue}{#1}}
\newcommand{\jtan}[1]{\textcolor{orange}{#1}}
\providecommand{\myparab}[1]{\smallskip\noindent\textbf{\textcolor{Red}{#1}} }
\providecommand{\myparabb}[1]{\smallskip\noindent\textbf{#1} }


\usepackage[normalem]{ulem}
\usepackage{xspace}
\usepackage{listings}
\usepackage{graphicx}
\usepackage{adjustbox}
\usepackage{booktabs} % for professional tables

%\usepackage{tabularx}
\usepackage{algorithm}
\usepackage{amsmath}
\usepackage{algpseudocode}
\usepackage{tabto}
\usepackage{caption}
% \usepackage{subfigure}
% \usepackage{subcaption}
\usepackage{subfig}
%\usepackage{subfig}
%\usepackage[font=small,labelfont=bf,textfont=bf]{caption}
\usepackage{caption}
%\usepackage[font={9pt,bf}]{caption}
%\usepackage[font=small,textfont=bf]{caption}
\usepackage{flushend}
\usepackage{multirow}
\usepackage{multicol}
\usepackage{enumitem}
\usepackage{bm}
\usepackage{ulem}
\usepackage{pifont}
%\usepackage{lmodern}
\usepackage{microtype}

\lstset{ %
aboveskip=5pt,
belowskip=0pt,
lineskip= 0pt,
language=Python,                % the language of the code
basicstyle=\scriptsize,       % the size of the fonts that are used for the code
numbers=left,                   % where to put the line-numbers
numberstyle=\scriptsize,      % the size of the fonts that are used for the line-numbers
stepnumber=1,                   % the step between two line-numbers. If it's 1, each line
                                % will be numbered
numbersep=2pt,                  % how far the line-numbers are from the code
backgroundcolor=\color{white},  % choose the background color. You must add \usepackage{color}
showspaces=false,               % show spaces adding particular underscores
stringstyle=\color{red}\ttfamily\scriptsize,
%%%% I added
identifierstyle=\scriptsize,
commentstyle=\color[rgb]{0,0.55,0.27}\ttfamily\scriptsize,
basicstyle=\scriptsize\ttfamily,
%stringstyle=\ttfamily,
showstringspaces=false,         % underline spaces within strings
showtabs=false,                 % show tabs within strings adding particular underscores
frame=t,                   % adds a frame around the code
%framexleftmargin=2mm,
tabsize=2,                      % sets default tabsize to 2 spaces
captionpos=b,                   % sets the caption-position to bottom
floatplacement=t,
breaklines=true,                % sets automatic line breaking
breakatwhitespace=false,        % sets if automatic breaks should only happen at whitespace
title=\lstname,                 % show the filename of files included with \lstinputlisting;
                                % also try caption instead of title
keywordstyle=\color{blue}\ttfamily,
escapechar={@},
%escapeinside={\%*}{*)},         % if you want to add a comment within your code
morekeywords={}            % if you want to add more keywords to the set
emphstyle=\underline
}

%%
%% Submission ID.
%% Use this when submitting an article to a sponsored event. You'll
%% receive a unique submission ID from the organizers
%% of the event, and this ID should be used as the parameter to this command.
%%\acmSubmissionID{123-A56-BU3}

%%
%% The majority of ACM publications use numbered citations and
%% references.  The command \citestyle{authoryear} switches to the
%% "author year" style.
%%
%% If you are preparing content for an event
%% sponsored by ACM SIGGRAPH, you must use the "author year" style of
%% citations and references.
%% Uncommenting
%% the next command will enable that style.
%%\citestyle{acmauthoryear}

%%
%% end of the preamble, start of the body of the document source.


%%% The following is specific to ESEC/FSE '21 and the paper
%%% 'Toward Efficient Interactions between Python and Native Libraries'
%%% by Jialiang Tan, Yu Chen, Zhenming Liu, Bin Ren, Shuaiwen Leon Song, Xipeng Shen, and Xu Liu.
%%%
\copyrightyear{2021} 
\acmYear{2021} 
\setcopyright{acmlicensed}\acmConference[ESEC/FSE '21]{Proceedings of the 29th ACM Joint European Software Engineering Conference and Symposium on the Foundations of Software Engineering}{August 23--27, 2021}{Athens, Greece}
\acmBooktitle{Proceedings of the 29th ACM Joint European Software Engineering Conference and Symposium on the Foundations of Software Engineering (ESEC/FSE '21), August 23--27, 2021, Athens, Greece}
\acmPrice{15.00}
\acmDOI{10.1145/3468264.3468541}
\acmISBN{978-1-4503-8562-6/21/08}




\begin{document}

%%
%% The "title" command has an optional parameter,
%% allowing the author to define a "short title" to be used in page headers.
\title{Toward Efficient Interactions between Python and Native Libraries}

%%
%% The "author" command and its associated commands are used to define
%% the authors and their affiliations.
%% Of note is the shared affiliation of the first two authors, and the
%% "authornote" and "authornotemark" commands
%% used to denote shared contribution to the research.
\author{Jialiang Tan}
\authornote{Both authors contributed equally to this research.}
\authornote{This work is done when Jialiang visits at NCSU.}
\email{jtan02@email.wm.edu}
%\orcid{1234-5678-9012}
\author{Yu Chen}
\authornotemark[1]
\email{ychen39@email.wm.edu}
\affiliation{%
  \institution{William \& Mary}
  \city{Williamsburg}
  \state{Virginia}
  \country{USA}
  %\postcode{43017-6221}
}

\author{Zhenming Liu}
\email{zliu@cs.wm.edu}
\affiliation{%
  \institution{William \& Mary}
  \city{Williamsburg}
  \state{Virginia}
  \country{USA}
  %\postcode{43017-6221}
}

\author{Bin Ren}
\email{bren@cs.wm.edu}
\affiliation{%
  \institution{William \& Mary}
  \city{Williamsburg}
  \state{Virginia}
  \country{USA}
  %\postcode{43017-6221}
}

\author{Shuaiwen Leon Song}
\email{shuaiwen.song@sydney.edu.au}
\affiliation{%
  \institution{University of Sydney}
  \city{Sydney}
  \state{}
  \country{Australia}
  \postcode{}}

\author{Xipeng Shen}
\email{xshen5@ncsu.edu}
\affiliation{%
  \institution{North Carolina State University}
  \city{Raleigh}
  \state{North Carolina}
  \country{USA}
  \postcode{}}
  
\author{Xu Liu}
\email{xliu88@ncsu.edu}
\affiliation{%
  \institution{North Carolina State University}
  \city{Raleigh}
  \state{North Carolina}
  \country{USA}
  \postcode{}}

%%
%% By default, the full list of authors will be used in the page
%% headers. Often, this list is too long, and will overlap
%% other information printed in the page headers. This command allows
%% the author to define a more concise list
%% of authors' names for this purpose.
\renewcommand{\shortauthors}{Tan and Chen, et al.}

%%
%% The abstract is a short summary of the work to be presented in the
%% article.
\begin{abstract}
Python has become a popular programming language because of its excellent programmability. Many modern software packages utilize Python for high-level algorithm design and depend on native libraries written in C/C++/Fortran for efficient computation kernels. Interaction between Python code and native libraries introduces performance losses because of the abstraction lying on the boundary of Python and native libraries. On the one side, Python code, typically run with interpretation, is disjoint from its execution behavior. On the other side, native libraries do not include program semantics to understand algorithm defects.

To understand the interaction inefficiencies, we extensively study a large collection of Python software packages and categorize them according to the root causes of inefficiencies. We extract two inefficiency patterns that are common in interaction inefficiencies. Based on these patterns, we develop \tool{}, a lightweight profiler, to pinpoint interaction inefficiencies in Python applications. The principle of \tool{} is to measure the inefficiencies in the native execution and associate inefficiencies with high-level Python code to provide a holistic view.
Guided by \tool{}, we optimize 17 real-world applications, yielding speedups up to 6.3$\times$ on application level.
\end{abstract}

%%
%% The code below is generated by the tool at http://dl.acm.org/ccs.cfm.
%% Please copy and paste the code instead of the example below.
%%
% \begin{CCSXML}
% <ccs2012>
%  <concept>
%   <concept_id>10010520.10010553.10010562</concept_id>
%   <concept_desc>Computer systems organization~Embedded systems</concept_desc>
%   <concept_significance>500</concept_significance>
%  </concept>
%  <concept>
%   <concept_id>10010520.10010575.10010755</concept_id>
%   <concept_desc>Computer systems organization~Redundancy</concept_desc>
%   <concept_significance>300</concept_significance>
%  </concept>
%  <concept>
%   <concept_id>10010520.10010553.10010554</concept_id>
%   <concept_desc>Computer systems organization~Robotics</concept_desc>
%   <concept_significance>100</concept_significance>
%  </concept>
%  <concept>
%   <concept_id>10003033.10003083.10003095</concept_id>
%   <concept_desc>Networks~Network reliability</concept_desc>
%   <concept_significance>100</concept_significance>
%  </concept>
% </ccs2012>
% \end{CCSXML}

% \ccsdesc[500]{Computer systems organization~Embedded systems}
% \ccsdesc[300]{Computer systems organization~Redundancy}
% \ccsdesc{Computer systems organization~Robotics}
% \ccsdesc[100]{Networks~Network reliability}



%%
%% Keywords. The author(s) should pick words that accurately describe
%% the work being presented. Separate the keywords with commas.
\keywords{Python, profiling, PMU, debug register}

\begin{CCSXML}
<ccs2012>
<concept>
<concept_id>10002944.10011123.10011674</concept_id>
<concept_desc>General and reference~Performance</concept_desc>
<concept_significance>500</concept_significance>
</concept>
<concept>
<concept_id>10002944.10011123.10011124</concept_id>
<concept_desc>General and reference~Metrics</concept_desc>
<concept_significance>500</concept_significance>
</concept>
<concept>
<concept_id>10011007.10011006.10011073</concept_id>
<concept_desc>Software and its engineering~Software maintenance tools</concept_desc>
<concept_significance>500</concept_significance>
</concept>
</ccs2012>
\end{CCSXML}

\ccsdesc[500]{General and reference~Performance}
\ccsdesc[500]{General and reference~Metrics}
\ccsdesc[500]{Software and its engineering~Software maintenance tools}

%% A "teaser" image appears between the author and affiliation
%% information and the body of the document, and typically spans the
%% page.

%%
%% This command processes the author and affiliation and title
%% information and builds the first part of the formatted document.
\maketitle

\section{Introduction}
In recent years, Python has become the most prominent programming language for data modeling and library development, especially in the area of machine learning, thanks to its elegant design that offers high-level abstraction, and its powerful interoperability with native libraries that delivers heavy numeric computations. Decoupling data analysis and modeling logics from operation logics is the singular mechanism guiding the remarkable improvements in developers’ productivity in the past decade. Python enables small teams to build sophisticated model~\cite{meta} that were barely imaginable a few years ago, and enables large teams of modelers and numeric developers to seamlessly collaborate and develop highly influential frameworks such as Tensorflow~\cite{tensorflow2015-whitepaper} and Pytorch~\cite{paszke2017automatic}. %\textcolor{red}{Tied to ML? I am not sure if we should make this more general.} {\color{red} ZLiu: non-ml people dont really use Python (with native libraries)?}

While high-level languages to articulate business logics and native libraries to deliver efficient computation is not a new paradigm, downstream developers have not always understood the details of native libraries, and have implemented algorithms that interacted poorly with native codes. A well-known example of the \emph{interaction inefficiency} problem occurs when developers, who fail to recognize that certain matrix operations can be vectorized, write significantly slower loop-based solutions. MATLAB and Mathematica can alleviate the problem since these languages usually are locked with a fixed set of native libraries over a long time, and developers can establish simple best practice guidelines to eliminate most interaction inefficiencies (MATLAB contains the command, “try to vectorize whenever possible”). 


In the Python ecosystem, native libraries and downstream application codes evolve rapidly so they can interact in numerous and unexpected ways. Therefore, building a list to exhaust all interaction inefficiencies becomes infeasible. We seek a solution that will automatically identify the blocks of Python code that lead to inefficient interactions, through closing the knowledge gap between Python and native code. Existing profiling tools cannot address this issue. Python profiles~\cite{cProfile, guppy3, py-spy, pyflame, pyinstrument, pycallgraph, pprofile, memoryprofiler, austin} cannot step in native code so they do not know  execution details. Native profiling tools~\cite{reinders2005vtune, de2010new, nistor2013toddler, adhianto2010hpctoolkit, chabbi2012deadspy, wen2017redspy, loadspy, wen2018watching} can identify hotspots, which sometimes offer leads to problematic code blocks. But because these tools do not have knowledge about Python code's semantic, they cannot render detailed root cause and thus often make debugging remarkably challenging. 

%This problem is subtly different from mere performance tuning. For example, most profiling tools can identify hotspots, and with sufficient manual effort, a developer potentially can fix a misusage of native libraries. But tuning performance from hotspots deteriorates Python’s abstraction principle and drags model developers back to the days of excessive worry about details in lower-level software stacks.

%\textcolor{red}{Not very clear reference. which approach? some references may also help} Using hotspot-based profiling tools relies heavily on the assumption that an inefficient interaction directly translates to observable hotspots in benchmarks. More important, the approach deteriorates Python’s abstraction principle and drags model developers back to the days of excessive worry about details in lower-level software stacks.

We propose \tool, the first lightweight, insightful profiler to pinpoint interaction inefficiencies in Python programs. \tool works for production Python software packages running in commodity CPU processors without modifying the software stacks. 
Its backbones algorithmic module is a recently proposed technique based on hardware performance monitoring units (PMUs) and debug registers to efficiently identify redundant memory accesses (hereafter, referred to as CL-algorithm\footnote{Chabbi-Liu Algorithm.}~\cite{wen2018watching, su2019pinpointing}). CL-algorithm intelligently chooses a small collection of memory cells and uses hardware to track accesses to these cells at a fine granularity. For example, when the technique detects two consecutive writes of the same value to the same cell, it determines that the second write is unnecessary, and flags the responsible statement/function for further inspection. The developer can clearly see where a non-opt memory access occurs and why.  The technique already shows its potential for eliminating inefficiencies in monolithic codebases that use one programming language.

%CL-algorithm can be used in our setting because the most pronounced symptom of inefficient interactions is redundant memory accesses. \textcolor{red}{Check the grammar of the following sentence:}Nevertheless, correctly applying 

\tool leverages the CL-algorithm in a substantially more complex multi-languages environment, in which a dynamic and (predominantly) interpretation-based language Python is used to govern the semantics and native libraries compiled from C, C++, Fortran are used to execute high-performance computation. Doing so requires us to address three major challenges that crosscuts Python and native code. 

%We need tackle challenges that arise at three fronts. \textcolor{red}{Is this work only an application of an existing algorithm? may consider improving the presentation of the above two paragraphs.}

At the measurement front, we need to suppress false positives and avoid tracking irrelevant memory operations produced from Python interpreter and Python-native interactions. For example, memory accesses performed by Python interpreters may ``bait'' the CL-algorithm to waste resources (i.e., debug registers) on irrelevant variables such as reference counters. At the infrastructure front, we need to penetrate entire software stacks: it cannot see execution details (i.e, how memory is accessed) with only Python runtime information, or cannot understand program semantics with only native library knowledge. Our main task here is to compactly implement lock-free calling context trees that span both Python code and native libraries, and retain a large amount of information to effectively correlate redundant memory accesses with inefficient interactions. At the memory/safety front, we need to avoid unexpected behaviors and errors caused by Python runtime. For example, Python’s garbage collection (GC) may
reclaim memory that our tool is tracking. So delicate coordination between \tool and Python interpreter is needed to avoid unexpected behaviors and errors. 


% {\color{red} ZLiu: remove?}
%\yu{\tool takes the redundant memory access across native function calls as the indicator, collects instruction information of native functions and inspects Python runtime states on the fly, resulting in a holistic view for interaction inefficiency diagnosis. \tool organizes the profiling data with a novel lock-free calling context tree, to reduce the heavy memory consumption caused by recording dynamic information of Python runtime. \tool applies multiple methods to make sure the profiling procedure works well under the interfering of Python runtime.}


\iffalse
We propose \tool, the first lightweight, insightful profiler to pinpoint interaction inefficiencies in Python programs. \tool works for production Python software packages in commodity CPU processors without modifying the software stacks. Its barebones algorithmic module is a recently proposed technique based on hardware performance monitoring units (PMUs) and debug registers to efficiently identify redundant memory access. The technique intelligently chooses a small collection of memory cells and uses hardware to track accesses to these cells at a fine granularity. For example, when the technique detects two consecutive writes of the same value to the same cell, it determines that the second write is unnecessary, and flags the responsible statement/function for further inspection by the developer. The developer can clearly see where an inefficient memory access occurs and why.  The technique already shows early signs of potential in monolithic codebases that use one programming language (Java CITE and C++ CITE). 

\yu{(we need to emphasize the relationship: redundant memory access->redundant native function calls-> interaction inefficiency)}

\yu{We aim to apply this technique for a more complex goal: diagnosing interaction inefficiencies under a multi-language environment, in which a dynamic and (predominantly) interpretation-based language Python is used to govern the semantics and a static and compilation-based language(C, C++ or Fortran)is used to execute high-performance computation. Such a methodology brings us three major challenges.} 
% We aim to deploy this technique to a substantially more complex multi-language environment, in which a dynamic and (predominantly) interpretation-based language Python is used to govern the semantics and a static and compilation-based language \yu{ (C, C++ or Fortran)} is used to execute high-performance computation. We need to address two major challenges.  {\color{red} bring up interaction inefficiencies.}

\yu{First, it requires to accurately identify redundant memory access related to interaction inefficiencies. Either the noise triggered from software stack's upper-level such as the interfering of the Python garbage collector(GC), or the lower-level such as the memory inefficiencies inside native libraries, can be barriers to locate interaction inefficiencies. Second, diagnosing interaction inefficiencies needs to penetrate the entire software stacks: it cannot see execution details (i.e, how memory is accessed) with only Python runtime information, or cannot understand program semantics with only native library knowledge. The major task here is to compactly implement calling context trees that span both Python code and native libraries and retain a considerably larger amount of information compared to existing ones CITE. Third, coordinating upper/lower-level behaviors of software stack is a basic building block to deploy the hardware technique (PMUs and hardware debug registers) for detecting interaction inefficiencies. Without such a mechanism, both of layers in software stack can interfere each other, such as GC frees the memory location which the debug register is monitoring, or the PMUs interrupt the Python runtime in a improper moment, leading to unexpected errors.}
\fi



% First, PieProf needs to penetrate the entire software stacks implemented in different language. PieProf cannot see execution details (i.e, how memory is accessed) if implemented at Python interpreter, and cannot understand program semantics if implemented at the native library level. Our major task is to compactly implement calling context trees that span both Python \yu{code and native libraries} and retain a considerably larger amount of information compared to existing ones CITE. \yu{Specifically, PieProf not only requires to collect instruction information of native functions, but also needs to inspect Python runtime states on the fly, resulting in a holistic view for interaction inefficiency diagnosis.}


% Second, PieProf needs to circumvent nuances produced by non-interacting code. It needs to avoid investing resources to track redundant memory access within the same native function calls, or within pure Python code (which can be found by existing solutions CITE CHECK). It also needs to safely handle operations triggered by Python interpreters. For example, garbage collectors \jtan{(GC)} could frequently change reference counters or deallocate the memory that we are tracking. Thus, PieProf needs to carefully interact with GC to avoid tracking irrelevant variables (reference counters) or avoid segmentation fault.  

We note that while most of the downstream applications we examined are machine learning related, \tool is a generic tool that can be used in any codebase that requires Python-native library interactions. 



\begin{comment}
{\color{red}Xu: I think we should separate contribtion and outlines.} We designed and implemented \tool, and ran it over a large set of codebases. In our paper we explain four aspects of the problem/solution. 


\noindent{\emph{Section 3}}: We provide the first characterization of interaction inefficiencies, which reveals that (i) interaction inefficiencies widely  even in highly popular open source packages, demonstrating the importance of eliminating the problem, and (ii) professionals, not amateurs, developed a significant fraction of the inefficiencies we discovered. Our findings confirm the challenges of writing high quality code under tight engineering and time resource constraints, and highlights the need to build automatic tools to detect inefficiencies. 


\noindent{\emph{Section 4}}: We explain the design and implementation of \tool, and how how we address the aforementioned challenges. 


\noindent{\emph{Section 5}}: We use \tool to study a collection of highly ranked Python applications, including widely used ones like Scikit-learn \jtan{cite here or in the table}, Pytorch, and highly specialized ones like Meta-heuristics. By running only sample code provided by the applications, we identify actionable interaction inefficiencies from 17 applications with {\color{red}inconsequential} overheads. Moderate refactoring effort leads to significant performance improvement (e.g., half of them have 100\% or more improvement at the functional level). This is quite unexpected because happy paths (provided sample code) in high-profile projects usually are carefully optimized and have low coverage of codebase. %{\color{red} low overheads.}


%We use \tool to study more than 100 highly ranked Python applications in Github. We identify interaction inefficiencies in 17 real-world applications and optimize them for nontrivial speedups. \yu{with low oeverhead}


\noindent{\emph{Section 6}}: We compare \tool with existing tools in XXXgive numberXXX case studies and discuss the findings. 







Both existing Python and native profilers fail to identify the interaction inefficiencies. Motivated by the need to obtain holistic profiles from both Python code and native libraries, we propose \tool (\underline{P}ython \underline{I}nteraction in\underline{E}fficiency \underline{PROF}iler), a lightweight, insightful profiler to pinpoint interaction inefficiencies in Python programs. The key novelty is to leverage PMUs and other hardware facilities available in commodity CPU processors to monitor native execution and associate the analysis with Python semantics. 

\paragraph{Scope.}
First, we target only interaction inefficiencies between Python codes and native libraries, and measuring inefficiencies in pure Python or pure native codes is out of the scope. Second, we design \tool{} as a dynamic profiler that pinpoints inefficiencies in codes, but it is the responsibility of human programmers to investigate the profilers and optimize the codes. Third, \tool{} is input dependent; to ensure that it produces representative profiles, we recommend using typical inputs to study the given Python application. 
\end{comment}

\myparabb{\textbf{Contributions.}}
We make three contributions.
\begin{itemize}
\item We are the first to thoroughly study the interaction inefficiencies between Python codes and native libraries. We categorize the interaction inefficiencies by their root causes.

\item We design and implement \tool, the first profiler to identify interaction inefficiencies and provide intuitive optimization guidance, by carefully stepping through Python runtimes and native binaries. \tool works for production Python software packages in commodity CPU processors without modifying the software stacks.

\item Following the guidance of \tool, we examine a wide range of influential codebases and identify interaction inefficiencies in 17 real-world applications and optimize them for nontrivial speedups.
\end{itemize}


%\textcolor{red}{We usually omit this to save space. We can highlight some evaluation results here instead.}

\myparabb{\textbf{Organization.}}
Section~\ref{background} reviews the background and related work. Section~\ref{characterization} characterizes the interaction inefficiencies. Section~\ref{design} describes the design and implementation of \tool. Section~\ref{evaluation} explains the evaluation. Section~\ref{casestudy} presents case studies. Section~\ref{validity} discusses some threats to validity. Section~\ref{conclusions} presents some conclusions. 
 







% % \leavevmode
% \\
% \\
% \\
% \\
% \\
\section{Introduction}
\label{introduction}

AutoML is the process by which machine learning models are built automatically for a new dataset. Given a dataset, AutoML systems perform a search over valid data transformations and learners, along with hyper-parameter optimization for each learner~\cite{VolcanoML}. Choosing the transformations and learners over which to search is our focus.
A significant number of systems mine from prior runs of pipelines over a set of datasets to choose transformers and learners that are effective with different types of datasets (e.g. \cite{NEURIPS2018_b59a51a3}, \cite{10.14778/3415478.3415542}, \cite{autosklearn}). Thus, they build a database by actually running different pipelines with a diverse set of datasets to estimate the accuracy of potential pipelines. Hence, they can be used to effectively reduce the search space. A new dataset, based on a set of features (meta-features) is then matched to this database to find the most plausible candidates for both learner selection and hyper-parameter tuning. This process of choosing starting points in the search space is called meta-learning for the cold start problem.  

Other meta-learning approaches include mining existing data science code and their associated datasets to learn from human expertise. The AL~\cite{al} system mined existing Kaggle notebooks using dynamic analysis, i.e., actually running the scripts, and showed that such a system has promise.  However, this meta-learning approach does not scale because it is onerous to execute a large number of pipeline scripts on datasets, preprocessing datasets is never trivial, and older scripts cease to run at all as software evolves. It is not surprising that AL therefore performed dynamic analysis on just nine datasets.

Our system, {\sysname}, provides a scalable meta-learning approach to leverage human expertise, using static analysis to mine pipelines from large repositories of scripts. Static analysis has the advantage of scaling to thousands or millions of scripts \cite{graph4code} easily, but lacks the performance data gathered by dynamic analysis. The {\sysname} meta-learning approach guides the learning process by a scalable dataset similarity search, based on dataset embeddings, to find the most similar datasets and the semantics of ML pipelines applied on them.  Many existing systems, such as Auto-Sklearn \cite{autosklearn} and AL \cite{al}, compute a set of meta-features for each dataset. We developed a deep neural network model to generate embeddings at the granularity of a dataset, e.g., a table or CSV file, to capture similarity at the level of an entire dataset rather than relying on a set of meta-features.
 
Because we use static analysis to capture the semantics of the meta-learning process, we have no mechanism to choose the \textbf{best} pipeline from many seen pipelines, unlike the dynamic execution case where one can rely on runtime to choose the best performing pipeline.  Observing that pipelines are basically workflow graphs, we use graph generator neural models to succinctly capture the statically-observed pipelines for a single dataset. In {\sysname}, we formulate learner selection as a graph generation problem to predict optimized pipelines based on pipelines seen in actual notebooks.

%. This formulation enables {\sysname} for effective pruning of the AutoML search space to predict optimized pipelines based on pipelines seen in actual notebooks.}
%We note that increasingly, state-of-the-art performance in AutoML systems is being generated by more complex pipelines such as Directed Acyclic Graphs (DAGs) \cite{piper} rather than the linear pipelines used in earlier systems.  
 
{\sysname} does learner and transformation selection, and hence is a component of an AutoML systems. To evaluate this component, we integrated it into two existing AutoML systems, FLAML \cite{flaml} and Auto-Sklearn \cite{autosklearn}.  
% We evaluate each system with and without {\sysname}.  
We chose FLAML because it does not yet have any meta-learning component for the cold start problem and instead allows user selection of learners and transformers. The authors of FLAML explicitly pointed to the fact that FLAML might benefit from a meta-learning component and pointed to it as a possibility for future work. For FLAML, if mining historical pipelines provides an advantage, we should improve its performance. We also picked Auto-Sklearn as it does have a learner selection component based on meta-features, as described earlier~\cite{autosklearn2}. For Auto-Sklearn, we should at least match performance if our static mining of pipelines can match their extensive database. For context, we also compared {\sysname} with the recent VolcanoML~\cite{VolcanoML}, which provides an efficient decomposition and execution strategy for the AutoML search space. In contrast, {\sysname} prunes the search space using our meta-learning model to perform hyperparameter optimization only for the most promising candidates. 

The contributions of this paper are the following:
\begin{itemize}
    \item Section ~\ref{sec:mining} defines a scalable meta-learning approach based on representation learning of mined ML pipeline semantics and datasets for over 100 datasets and ~11K Python scripts.  
    \newline
    \item Sections~\ref{sec:kgpipGen} formulates AutoML pipeline generation as a graph generation problem. {\sysname} predicts efficiently an optimized ML pipeline for an unseen dataset based on our meta-learning model.  To the best of our knowledge, {\sysname} is the first approach to formulate  AutoML pipeline generation in such a way.
    \newline
    \item Section~\ref{sec:eval} presents a comprehensive evaluation using a large collection of 121 datasets from major AutoML benchmarks and Kaggle. Our experimental results show that {\sysname} outperforms all existing AutoML systems and achieves state-of-the-art results on the majority of these datasets. {\sysname} significantly improves the performance of both FLAML and Auto-Sklearn in classification and regression tasks. We also outperformed AL in 75 out of 77 datasets and VolcanoML in 75  out of 121 datasets, including 44 datasets used only by VolcanoML~\cite{VolcanoML}.  On average, {\sysname} achieves scores that are statistically better than the means of all other systems. 
\end{itemize}


%This approach does not need to apply cleaning or transformation methods to handle different variances among datasets. Moreover, we do not need to deal with complex analysis, such as dynamic code analysis. Thus, our approach proved to be scalable, as discussed in Sections~\ref{sec:mining}.
\section{Background and Motivation}

\subsection{IBM Streams}

IBM Streams is a general-purpose, distributed stream processing system. It
allows users to develop, deploy and manage long-running streaming applications
which require high-throughput and low-latency online processing.

The IBM Streams platform grew out of the research work on the Stream Processing
Core~\cite{spc-2006}.  While the platform has changed significantly since then,
that work established the general architecture that Streams still follows today:
job, resource and graph topology management in centralized services; processing
elements (PEs) which contain user code, distributed across all hosts,
communicating over typed input and output ports; brokers publish-subscribe
communication between jobs; and host controllers on each host which
launch PEs on behalf of the platform.

The modern Streams platform approaches general-purpose cluster management, as
shown in Figure~\ref{fig:streams_v4_v6}. The responsibilities of the platform
services include all job and PE life cycle management; domain name resolution
between the PEs; all metrics collection and reporting; host and resource
management; authentication and authorization; and all log collection. The
platform relies on ZooKeeper~\cite{zookeeper} for consistent, durable metadata
storage which it uses for fault tolerance.

Developers write Streams applications in SPL~\cite{spl-2017} which is a
programming language that presents streams, operators and tuples as
abstractions. Operators continuously consume and produce tuples over streams.
SPL allows programmers to write custom logic in their operators, and to invoke
operators from existing toolkits. Compiled SPL applications become archives that
contain: shared libraries for the operators; graph topology metadata which tells
both the platform and the SPL runtime how to connect those operators; and
external dependencies. At runtime, PEs contain one or more operators. Operators
inside of the same PE communicate through function calls or queues. Operators
that run in different PEs communicate over TCP connections that the PEs
establish at startup. PEs learn what operators they contain, and how to connect
to operators in other PEs, at startup from the graph topology metadata provided
by the platform.

We use ``legacy Streams'' to refer to the IBM Streams version 4 family. The
version 5 family is for Kubernetes, but is not cloud native. It uses the
lift-and-shift approach and creates a platform-within-a-platform: it deploys a
containerized version of the legacy Streams platform within Kubernetes.

\subsection{Kubernetes}

Borg~\cite{borg-2015} is a cluster management platform used internally at Google
to schedule, maintain and monitor the applications their internal infrastructure
and external applications depend on. Kubernetes~\cite{kube} is the open-source
successor to Borg that is an industry standard cloud orchestration platform.

From a user's perspective, Kubernetes abstracts running a distributed
application on a cluster of machines. Users package their applications into
containers and deploy those containers to Kubernetes, which runs those
containers in \emph{pods}. Kubernetes handles all life cycle management of pods,
including scheduling, restarting and migration in case of failures.

Internally, Kubernetes tracks all entities as \emph{objects}~\cite{kubeobjects}.
All objects have a name and a specification that describes its desired state.
Kubernetes stores objects in etcd~\cite{etcd}, making them persistent,
highly-available and reliably accessible across the cluster. Objects are exposed
to users through \emph{resources}. All resources can have
\emph{controllers}~\cite{kubecontrollers}, which react to changes in resources.
For example, when a user changes the number of replicas in a
\code{ReplicaSet}, it is the \code{ReplicaSet} controller which makes sure the
desired number of pods are running. Users can extend Kubernetes through
\emph{custom resource definitions} (CRDs)~\cite{kubecrd}. CRDs can contain
arbitrary content, and controllers for a CRD can take any kind of action.

Architecturally, a Kubernetes cluster consists of nodes. Each node runs a
\emph{kubelet} which receives pod creation requests and makes sure that the
requisite containers are running on that node. Nodes also run a
\emph{kube-proxy} which maintains the network rules for that node on behalf of
the pods. The \emph{kube-api-server} is the central point of contact: it
receives API requests, stores objects in etcd, asks the scheduler to schedule
pods, and talks to the kubelets and kube-proxies on each node. Finally,
\emph{namespaces} logically partition the cluster. Objects which should not know
about each other live in separate namespaces, which allows them to share the
same physical infrastructure without interference.

\subsection{Motivation}
\label{sec:motivation}

Systems like Kubernetes are commonly called ``container orchestration''
platforms. We find that characterization reductive to the point of being
misleading; no one would describe operating systems as ``binary executable
orchestration.'' We adopt the idea from Verma et al.~\cite{borg-2015} that
systems like Kubernetes are ``the kernel of a distributed system.'' Through CRDs
and their controllers, Kubernetes provides state-as-a-service in a distributed
system. Architectures like the one we propose are the result of taking that view 
seriously.

The Streams legacy platform has obvious parallels to the Kubernetes
architecture, and that is not a coincidence: they solve similar problems.
Both are designed to abstract running arbitrary user-code across a distributed
system.  We suspect that Streams is not unique, and that there are many
non-trivial platforms which have to provide similar levels of cluster
management.  The benefits to being cloud native and offloading the platform
to an existing cloud management system are: 
\begin{itemize}
    \item Significantly less platform code.
    \item Better scheduling and resource management, as all services on the cluster are 
        scheduled by one platform.
    \item Easier service integration.
    \item Standardized management, logging and metrics.
\end{itemize}
The rest of this paper presents the design of replacing the legacy Streams 
platform with Kubernetes itself.


\section{Designing a Watermark}
\label{sec:taxonomy}

After the quick overview of watermarking schemes in \cref{sec:background}, we now provide more details 
about the watermarking design space. We created a unifying taxonomy under which all previous schemes 
can be expressed. We first discuss the requirements then the building blocks of a text watermark. 
%
%We provide a modular implementation of all schemes, so any of the building blocks can be combined.
%
\cref{fig:design-figure} summarizes the current design space.

\subsection{Requirements}

A useful watermarking scheme must detect watermarked texts, without falsely flagging human-generated text and without impairing the original model's performance.
%
More precisely, we want watermarks to have the following properties.
% \begin{itemize}[leftmargin=\itemlm,itemsep=2pt]
\begin{enumerate}[leftmargin=\itemlm,itemsep=2pt]
    \item \textbf{High Recall}. $\Pr[\mathcal{V}_k(T) = \texttt{True}]$ is large if $T$ is a watermarked text generated using the marking procedure $\mathcal{W}$ and secret key $k$.
    %
    \item \textbf{High Precision}. For a random key $k$, $\Pr[\mathcal{V}_k(\Tilde{T}) = \texttt{False}]$ is large if $\Tilde{T}$ is a human-generated (\emph{non-watermarked}) text.
    %
    \item \textbf{Quality}. The watermarked model should perform similarly to the original model. 
    It should be useful for the same tasks and generate similar quality text.
    %
    \item \textbf{Robustness}. A good watermark should be robust to small changes to the watermarked text (potentially caused by an adversary), 
    meaning if a sample $T$ is watermarked with key $k$, then for any text $\Tilde{T}$ that is semantically close to $T$, $\mathcal{V}_k(\Tilde{T})$ should evaluate to \text{True}.
\end{enumerate}

\noindent
A desireable (but optional) property for watermarks is diversity. 
In some settings, such as creative tasks like story-telling, users might want the model to have the ability to generate 
multiple different outputs in response to the same prompt (so they can select their favorite).
We would like watermarked outputs to preserve this capability.
% \noindent
% In addition to these properties, another desirable property for a watermark is to 
% preserve a model's diversity. Language models tend to have diverse generated text distributions: 
% they are able to generate different responses to a same prompt. This is useful in many settings, 
% such as creative tasks like story telling, so the user can  their favorite output.

% The notion of \emph{undetectability} has been defined in previous work~\citep{christ_undetectable_2023}:
Another useful property is \emph{undetectability}, also called \emph{indistinguishability}:
%
no feasible adversary should be able to distinguish watermarked text from non-watermarked text, without knowledge of the secret key~\citep{christ_undetectable_2023}. 
%
A watermark is considered undetectable if the maximum advantage at distinguishing is very small.
%
This notion is appealing; for instance, undetectability implies that watermarking does not degrade the model's quality.
%
However, we find in practice that undetectability is not necessary and may be overly restrictive:
%
minor changes to the model's output distribution are not always detrimental to its quality.

In this paper we focus on symmetric-key watermarking, where both the watermarking and verification procedures share a secret key.
%
This is most suitable for proprietary language models that served via an API.
%
We imagine that the vendor would watermark all outputs, and also provide a second API to query the verification procedure.
%
Alternatively, one could publish the key, enabling anyone to run the verification procedure.
%
\begin{figure*}
    \begin{center}
    \begin{tikzpicture}
    
    \draw[draw=black] (0,15) rectangle ++(17.5,1) node[pos=0.5, align=center] {\Large{Watermarking Taxonomy}};
    \draw[draw=black] (0,12.75) rectangle ++(8.375,2) node[pos=0.5, align=left] 
    {\\
    \\
    \textbf{Parameters:} Key $k$, Sampling $\mathcal{C}$, Randomness $\mathcal{R}$\\
    \textbf{Inputs:} Probs $\mathcal{D}_n = \{\lambda^n_1,\, \cdots, \lambda^n_d\}$, Tokens $\{T_i\}_{i < n}$\\
    \textbf{Output:} Next token 
    $T_n \leftarrow \mathcal{C}(\mathcal{R}_k( \{T_i\}_{i < n}), \mathcal{D}_n)$};
    \draw[draw=black] (9.125,12.75) rectangle ++(8.375,2) node[pos=0.5, align=left] 
    {\\
    \\
    \textbf{Parameters:} Key $k$, Score $\mathcal{S}$, Threshold $p$\\
    \textbf{Inputs:} Text $T$\\
    \textbf{Output:} Decision $\mathcal{V} \leftarrow \text{P}_{0}\left( \mathcal{S} < \mathcal{S}_k(T)\right) < p$};
    \draw (8.75,13.75) circle (0.25) node {+};
    \draw[draw=none] (0,14.25) rectangle ++(8.375,.5) node[pos=0.5, align=left] {\large{Marking $\mathcal{W}$}};
    \draw[draw=none] (9.125,14.25) rectangle ++(8.375,.5) node[pos=0.5, align=left] {\large{Verification $\mathcal{V}$}};
    
    %%%
    
    \draw[draw=black,dashed] (0,8.75) rectangle ++(17.5,3.75);
    \draw[draw=none] (0,11) rectangle ++(17.5,1.75) node[pos=0.5, align=center] {\large{Randomness Source $\mathcal{R}$}\\
    \textbf{Inputs:} Tokens $\{T_i\}_{i < n}$\,
    \textbf{Output:} Random value $r_n = \mathcal{R}_k(\{T_i\}_{i < n})$};
    \draw[draw=black] (0.25,9) rectangle ++(11.25,2.35) node[pos=0, anchor=south west] {\textbf{Text-dependent.} Hash function $h$. Context length H};
    \draw[draw=black] (0.5,9.6) rectangle ++(10.75,0.625) node[anchor=north west] at (0.5, 10.225) {\textbf{(R2) Min Hash}} node[pos=1, anchor=north east, align=left] {
    $r_n = \text{min} \left( h\left( T_{n-1} \mathbin\Vert k\right), \, \cdots, h\left( T_{n-H} \mathbin\Vert k\right) \right)$\\
    };
    \draw[draw=black] (0.5,10.475) rectangle ++(10.75,0.625) node[anchor=north west] at (0.5, 11.1) {\textbf{(R1) Sliding Window}} node[pos=1, anchor=north east, align=left] {
    $r_n = h\left( T_{n-1} \mathbin\Vert \, \cdots \mathbin\Vert T_{n-H} \mathbin\Vert k\right)$\\
    };
    \draw[draw=black] (11.75,9) rectangle ++(5.5,2.35) node[pos=0, anchor=south west] {\textbf{(R3) Fixed}} node[pos=0.5, align=left] {Key length L. Expand $k$ to\\ pseudo-random sequence $\{r^k_i\}_{i<L}$.\\ 
    $r_n = r^k_{n \text{ (mod L)}}$ \\ \\ };
    
    %%%
    
    \draw[draw=black,dashed] (0,3.25) rectangle ++(17.5,5.25);
    \draw[draw=none] (0,6.85) rectangle ++(17.5,1.75) node[pos=0.5, align=center] {\large{Sampling algorithm $\mathcal{C}$ \& Per-token statistic $s$}\\
    \textbf{Inputs:} Random value $r_n = \mathcal{R}_k( \{T_i\}_{i < n})$, Probabilities $\mathcal{D}_n = \{\lambda^n_1,\, \cdots, \lambda^n_d\}$, Logits $\mathcal{L}_n = \{l^n_1,\,\cdots,l^n_d\}$\\};
    
    %
    
    \draw[draw=black] (11,4.75) rectangle ++(6.25,2.5) node[pos=0, anchor=south west] {\textbf{(C3) Binary}} node[pos=0.5, align=left] {Binary alphabet.\\ 
    $T_n \leftarrow 0$ if $r_n < \lambda^n_0$, else $1$. \\
    $s(T_n, r) = \begin{cases} -\log(r) \text{ if } T_n = 1\\
          -\log(1-r) \text{ if } T_n = 0\\\end{cases} $};
    
    \draw[draw=black] (5,4.75) rectangle ++(5.75,2.5) node[pos=0, anchor=south west] {\textbf{(C2) Inverse Transform}} node[pos=0.5, align=left] 
    {$\pi$ keyed permutation. $\eta$ scaling func.\\
    $T_n \leftarrow \pi_k \left( \min\limits_{ j \leq d } \sum\limits_{i=1}^j \lambda^n_{\pi_k (i)} \geq r_n \right)$ \\
    $s(T_n, r) = | r - \eta \left( \pi^{-1}_k(T_n) \right) | $\\};
    
    \draw[draw=black] (0.25,4.75) rectangle ++(4.5,2.5) node[pos=0, anchor=south west] {\textbf{(C1) Exponential}} node[pos=0.5, align=left] 
    {$h$ keyed hash function. \\
    $T_n \leftarrow \argmax\limits_{i \leq d} \left\{ \frac{\log \left( h_{r_n}\left( i \right) \right)}{\lambda^n_i} \right\}$ \\
    $s(T_n, r) = -\log(1 \! - \! h_r(T_n))$\\};
    
    % 
    
    \draw[draw=black] (0.25,3.5) rectangle ++(17,1) node[pos=0, anchor=south west] {\textbf{(C4) Distribution-shift}} node[pos=0.5, align=right] {Bias $\delta$, Greenlist size $\gamma$. Keyed permutation $\pi$. $T_n$ sampled from $\widetilde{\mathcal{L}}_n = \{l^n_i + \delta \text{ if } \pi_{r_n}(i) < \gamma d \text{ else } l^n_i\, , 1 \leq i \leq d\}$\\
    $s(T_n, r) = 1 \text{ if } \pi_{r}(T_n) < \gamma d \text{ else } 0$};
    
    %%% 
    
    \draw[draw=black,dashed] (0,0) rectangle ++(17.5,3);
    \draw[draw=none] (0,1.75) rectangle ++(17.5,1.25) node[pos=0.5, align=center] {\large{Score $\mathcal{S}$}\\
    \textbf{Inputs:} Per-token statistics $s_{i,j} = s(T_i, r_j)$, where $r_j = \mathcal{R}_k( \{T_l\}_{l < j}))$. \# Tokens $N$.};
    
    % 
    
    \draw[draw=black] (8.15,0.25) rectangle ++(9.1,1.5) node[pos=0, anchor=south west] {\textbf{(S3) Edit Score}}
    
    node[pos=0.5, align=left] {
    $\mathcal{S}_{\text{edit}}^\psi = s^\psi(N,N)$,
    $
        s^\psi (i,j) = \min \begin{cases}
          s^\psi (i-1, j-1) + s_{i,j}\\
          s^\psi (i-1, j) + \psi\\
          s^\psi (i, j-1) + \psi\\
        \end{cases} 
    $};
    \draw[draw=black] (0.25,0.25) rectangle ++(2.6,1.5) node[pos=0, anchor=south west] {\textbf{(S1) Sum Score}} node[pos=0.5, align=left] {$\mathcal{S}_{\text{sum}}\! = \! \sum_{i=1}^N s_{i,i}$ \\};
    \draw[draw=black] (3.1,0.25) rectangle ++(4.8,1.5) node[pos=0, anchor=south west] {\textbf{(S2) Align Score}} node[pos=0.5, align=left] {$\mathcal{S}_{\text{align}} \!= \!\min\limits_{0 \leq j < N} \sum\limits_{i=1}^N s_{i, (i+j) \text{ mod}(N)}$ \\ \\ };
    
    \end{tikzpicture}
    \caption{Watermarking design blocks. There are three main components: randomness source, sampling algorithm (and associated per-token statistics), and score function. Each solid box within each of these three components (dashed) denotes a design choice. The choice for each component is independent and offers different trade-offs.}\label{fig:design-figure}
    \end{center}
    \end{figure*}

\subsection{Watermark Design Space}
\label{sec:watermark-design}

Designing a good watermark is a balancing act.
% 
For instance, replacing every word of the output with [WATERMARK] would achieve high recall but destroy the utility of the model.
%
%Conversely, sampling from the original distribution preserves quality but makes it impossible to watermark. 

Existing proposals have cleverly crafted marking procedures that are meant to preserve quality, provide high precision and recall, and achieve a degree of robustness.
%
Despite their apparent differences, we realized they can all be expressed within a unified framework:

\begin{itemize}[leftmargin=\itemlm,itemsep=2pt]
    \item The marking procedure $\mathcal{W}$ contains a randomness source $\mathcal{R}$ and a sampling algorithm $\mathcal{C}$.
    %
    The randomness source $\mathcal{R}$ produces a (pseudo-random) value $r_n$ for each new token, based on the secret key $k$ and the previous tokens $T_0,\cdots,T_{n-1}$.
    %
    The sampling algorithm $\mathcal{C}$ uses $r_n$ and the model's next token distribution $\mathcal{D}$ to  a token.
    \item The verification procedure $\mathcal{V}$ is a one-tailed significance test that computes a $p$-value for the null hypothesis that the text is not watermarked.
    %
    The procedure compares this $p$-value to a threshold, which enables control over the watermark's precision and recall.
    %
    % This test is done using a \emph{score function} $\mathcal{S}$ based on a per-token variable that depends on the ed sampling algorithm.
    % We call the value of this per-token test statistic $s_n$, which only depends on the random value $r_n$ and the ed token $T_n$: $s_n = s(T_n, r_n)$.
    In particular, we compute a per-token score $s_{n,m} \coloneqq s(T_n, r_m)$ for each token $T_n$ and randomness $r_m$, aggregate them to obtain an overall score $\mathcal{S}$, and compute a $p$-value from this score.
    We consider all overlaps $s_{n,m}$ instead of only $s_{n,n}$ to support scores that consider misaligned randomness and text after perturbation. 
    %the test computes \emph{score function} $\mathcal{S}$ which takes as input per-token test statistics $s_{n,m} \coloneqq s(T_n, r_m)$ for a token $T_n$ and a random value $r_m$, $\forall n,m \in [N]$.
    %
    %$s_{n,m}$ depends on the sampling algorithm (see \cref{fig:design-figure} for examples).
    %
    % \dave{I believe $s(T_n, r_m)$ is incorrect and it should be $s(T_n, r_n)$.  Also I think the score should be $s_n$ rather than $s_{n,m}$.}
    % \jp{Depending on the alignment between the key string and the text, there are times we want to refer to the score for key at position m and token at poistion n (for instance, for both the align and edit scores). I'll add some explanation for this.}
    
\end{itemize}
% \dave{I find the sheer number of fonts inelegant (blackboard bold, mathcal, mathbf, typewritter, italics, bold, etc.). In some places, algorithms are denoted by mathcal (W,V), in other places by mathbf (R,C,S).  I suggest picking one and being consistent.  I prefer mathcal.  Lots of bold feels distracting to my eyes, as does lots of font changes.}
% \jp{I changed a bunch of fonts to make it more consistent, and removed bold fonts}

Next, we show how each scheme we consider falls within this framework, each with its own choices for $\mathcal{R},\mathcal{C},\mathcal{S}$.
%Given this template, previous work introduced their own variants of the building blocks, which we will now detail. 
% \chawin{I would have liked to see a summary of which design choices belong to which paper. Maybe we can add a shorthand notation denoting each paper in \cref{fig:design-figure} or have a separate table.}
% \jp{I agree that's a good idea. A table is probably the right way to represent this.}

\subsubsection{Randomness source $\mathcal{R}$}\label{ssec:randomness}
% \textbf{Randomness source $\mathcal{R}$.}
%
% \chawin{Maybe others?} \jp{Yeah but all the other papers i've seen seem to attribute it to one of these two.}
We distinguish two main ways of generating the random values $r_n$, \emph{text-dependent} (computed as a deterministic function of the prior tokens) vs \emph{fixed} (computed as a function of the token index).
Both approaches use the standard heuristic of applying a keyed function (typically, a PRF) to some data, to produce pseudorandom values that can be treated as effectively random but can also be reproduced by the verification procedure.

\citet{aaronson_watermarking_2022} and \citet{kirchenbauer_watermark_2023}
use text-dependent randomness: $r_n = f\left(T_0,\,\cdots,T_{n-1},k\right)$.
%
This scheme has two parameters: the length of the token context window (which we call the window size H) and the aggregation function $f$.
%
\citet{aaronson_watermarking_2022} proposed using the hash of the concatenation of previous tokens, $f := h\left( T_{n-1} \mathbin\Vert \, \cdots \mathbin\Vert T_{n-H} \mathbin\Vert k\right)$; we call this (R1) sliding window.
%
\citet{kirchenbauer_watermark_2023} used this with a window size of $ H = 1$ and also introduced an alternate aggregation function $f := \text{min} \left( h\left( T_{n-1} \mathbin\Vert k\right), \, \cdots, h\left( T_{n-H} \mathbin\Vert k\right) \right)$.
%
We call this last aggregation function (R2) min hash.
%
While these two schemes propose specific choices of $H$, other values are possible. 
We use \benchmarkname{} to evaluate a range of values of $H$ with each candidate aggregation function.

% \smallskip\noindent\textbf{(R3) Fixed}
\citet{kuditipudi_robust_2023} use fixed randomness:
$r_n = f_k(n)$, where $n$ is the index (position) of the token.
We call this (R3) fixed.
%
In practice, they propose using a fixed string of length $L$ (the key length), which is repeated across the generation.
% r_n = f_k(n \bmod L)$ where $L$ is the key length.
% \dave{I don't think we need this level of detail.  I suggest deleting the preceding sentence.}
% \jp{Since we look at the impact of the key length on generations we still need to introduce the idea that the key is repeated, but I canwrite that in english for it to be more digestable}
%
We test the choice of key length in ~\cref{ssec:param_tuning}
%
In the extreme case where $L=1$ or $H=0$, both sources are identical, as $r_n$ will be the same value for every token. \citet{zhao2023provable} explored this option using the same sampling algorithm as~\citet{kirchenbauer_watermark_2023}.

\label{ssec:binary}
\citet{christ_undetectable_2023} proposed setting a target entropy for the context window instead of fixing a window size.
%
This allows to set a lower bound on the security parameter for the model's undetectability.
%
However, setting a fixed entropy makes for a less efficient detector since all context window lengths must be tried in order to detect a watermark.
%
Furthermore, in practice, provable undetectability is not needed to achieve optimal quality: we chose to keep using a fixed-size window for increased efficiency.

\subsubsection{Sampling algorithm \(\mathcal{C}\)}\label{ssec:sampling}
% \textbf{sampling algorithm $\mathcal{C}$.}
%
\noindent
We now give more details about the four sampling algorithms initially presented in~\cref{tab:marking-algorithms}.

\smallskip\noindent\textbf{(C1) Exponential}.
%
Introduced by \citet{aaronson_watermarking_2022} and also used by \citet{kuditipudi_robust_2023}. It relies on the Gumbel-max trick.
%
Let $\mathcal{D}_n = \left\{\lambda^n_i\,, 1 \leq i \leq d\right\}$ be the distribution of the language model over the next token. %(obtained after passing the logits through a softmax and applying a temperature adjustment).
%
The exponential scheme will select the next token as:
\begin{align}
    T_{n} = \argmax\limits_{i \leq d}\left\{ \frac{\log \left( h_{r_n}\left( i \right) \right)}{\lambda^n_i} \right\}
\end{align}
where $h$ is a keyed hash function using $r_n$ as its key.
%
The per-token variable used in the statistical test is either $s_n = h_{r_n}(T_n)$ or $s_n = -\log \left( 1-h_{r_n}(T_n)\right)$.
%
\citet{aaronson_watermarking_2022} and \citet{kuditipudi_robust_2023} both use the latter quantity.
%
We argue the first variable provides the same results, and unlike the log-based variable, the distribution of watermarked variables can be expressed analytically (see~\cref{app:ssec:pseudorandom-proofs} for more details).
%
We align with previous work and use the $\log$ for \benchmarkname{}.

\smallskip\noindent\textbf{(C2) Inverse transform}.
%
\citet{kuditipudi_robust_2023} introduce inverse transform sampling.
%
They derive a random permutation using the secret key $\pi_k$. The next token is selected as follows:
\begin{align}
    T_{n} = \pi_k \left( \min\limits_{ j \leq d } \sum\limits_{i=1}^j \lambda^n_{\pi_k (i)} \geq r_n \right)
\end{align}
which is the smallest index in the inverse permutation such that the CDF of the next token distribution is at least $r_n$.
%
\citet{kuditipudi_robust_2023} propose to use $s_n = | r_n - \eta \left( \pi^{-1}_k(T_n) \right) |$ as a the test variable, where $\eta$ normalizes the token index to the $[0,1]$ range.
%
% We call this scheme the \textit{inverse transform} scheme.

\smallskip\noindent\textbf{(C3) Binary}.
%
\citet{christ_undetectable_2023} propose a different sampling scheme for binary token alphabets --- however, it can be applied to any model by using a bit encoding of the tokens.
%
In our implementation, we rely on a Huffman encoding of the token set, using frequencies derived from a large corpus of natural text.
%
In this case, the distribution over the next token reduces to a single probability $p_n$ that token ``0'' is ed next, and $1-p$ that ``1'' is ed.
%
The sampling rule s 0 if $r_n < p$, and 1 otherwise. The test variable for this case is $s_n = -\log \left( T_n r_n + (1-T_n) (1-r_n) \right)$.
%
% We call this scheme the \textit{binary} scheme.
%
% At first glance, it can seem like this scheme is identical to the exponential scheme. However, because it uses a binary alphabet, the distribution of the test variable is different for both schemes.
%
% However, we show in Appendix \jp{ref} that this is not the case: the distribution of the test variable is different for both schemes.
% %
% \jp{Maybe I'll remove this if I don't have time to show it.}

\smallskip\noindent\textbf{(C4) Distribution-shift}.
%
\citet{kirchenbauer_watermark_2023} propose the distribution-shift scheme. 
%
It produces a modified distribution $D_n$ from which the next token is sampled.
%
Let $\delta > 0$ and $\gamma \in [0,1]$ be two system parameters, and $d$ be the number of tokens.
%
The scheme constructs a permutation $\pi_{r_n}$, seeded by the random value $r_n$, which is used to define a ``green list,'' containing tokens $T$ such that $\pi_{r_n} (T) < \delta d$. It then adds $\delta$ to green-list logits.
%
This modified distribution is then used by the model to sample the next token. The test variable $s_n$ is a bit equal to ``1'' if $T_n$ is in the green list defined by $\pi_{r_n}$, and ``0'' if not.
%
% We call this scheme the \textit{distribution-shift} scheme.

The advantage of this last scheme over the others is that it preserves the model's diversity: 
for a given key, the model will still generate diverse outputs.
In contrast, for a given secret key and a given prompt, the first three sampling strategies 
will always produce the same result, since the randomness value $r_n$ will be the same.
\citet{kuditipudi_robust_2023} tackles this by randomly offseting the key sequence of 
fixed randomness for each generation. We introcude a skip probability $p$ for the 
same effect on text-dependent randomness. Each token is selected without the marking 
strategy with probability $p$. In the interest of space, we leave a detailed discussion 
of generation diversity in~\cref{app:ssec:diverse}.

Another advantage of the distribution-shift scheme is that it can also be used 
at any temperature, by applying the temperature scaling \emph{after} using the 
scheme to modify the logits. Other models apply temperature before watermarking.

However the distribution-shift scheme is not indistinguishable from the original model, 
as discussed earlier in~\cref{ssec:watermark-design}.

\subsubsection{Score Function $\mathcal{S}$}\label{ssec:score}

% \paragraph{Verification procedure $\mathcal{V}.$}

% The distribution of the per-token test statistic is different for watermarked text and non-watermarked text: this is what makes detection possible. Depending on the scheme, it is either higher or lower on average in the watermarked case. Without loss of generality, we assume it is always lower for this discussion.

To determine whether an $N$-token text is watermarked, we compute a score over per-token statistics.
%
This score is then subject to a one-tailed statistical test where the null hypothesis is that the text is not watermarked.
%
In other words, if its $p$-value is under a fixed threshold, the text is watermarked.
%
Different works propose different scores.

\smallskip\noindent\textbf{(S1) Sum score}.
%
\citet{aaronson_watermarking_2022} and \citet{kirchenbauer_watermark_2023} take the sum of all individual per-token statistics:
\begin{align}
    \mathcal{S}_{\text{sum}}=\sum_{i=1}^N s_i = \sum_{i=1}^N s(T_i, r_i).
\end{align}
%
This score requires the random values $r_i$ and the tokens $T_i$ to be aligned.
%
% \chawin{Maybe this goes into limitation or discussion or appendix}
This is not a problem when using text-dependent randomness, since the random values are directly obtained from the tokens.
%
However, this score is not suited for fixed randomness: removing one token at the start of the text will offset the values of $r_i$ for the rest of the text and remove the watermark.
%
The use of the randomness shift to increase diversity will have the same effect. 

\smallskip\noindent\textbf{(S2) Alignment score}.
Proposed by \citet{kuditipudi_robust_2023}, the alignment score aims to mitigate the misalignment issue mentioned earlier.
% \citet{kuditipudi_robust_2023} proposes two alternative scores to deal with this issue.
%
% In keeping with their work, we name these scores the alignment score and the edit score.
Given the sequence of random values $r_i$ and the sequence of tokens $T_i$, the verification process now computes different versions of the per-token test statistic for each possible overlap of both sequences $s_{i,j} = s(T_i, r_j)$.
%
The alignment score is defined as:
\begin{align}
   \mathcal{S}_{\text{align}}  = \min\limits_{0 \leq j < N} \sum\limits_{i=1}^N s_{i, (i+j) \text{ mod}(N)}
\end{align}

\smallskip\noindent\textbf{(S3) Edit score}.
Similar to the alignment score, \citet{kuditipudi_robust_2023} propose the edit score as an alternative for dealing with the misalignment issue.
%
It comes with an additional parameter $\psi$ and is defined as $\mathcal{S}_{\text{edit}}^\psi = s^\psi(N,N)$, where
\begin{align}
    s^\psi (i,j) &= \min \begin{cases}
      s^\psi (i-1, j-1) + s_{i,j}\\
      s^\psi (i-1, j) + \psi\\
      s^\psi (i, j-1) + \psi\\
    \end{cases} 
\end{align}

In all three cases, the average value of the score for watermarked text will be lower than for non-watermarked text.
%
% In the case of the sum score, we can often derive the distribution of the score under the null hypothesis, allowing us to use a $z$-test to determine if the text is watermarked.
In the case of the sum score, the previous works use the $z$-test on the score to determine whether the text is watermarked, but it is also possible, or even better in certain situations, to use a different statistical test according to \citet{fernandez_three_2023}.
%
When possible, we derive the exact distribution of the scores under the null hypothesis (see \cref{app:ssec:exact_dist}) which is more precise than the $z$-test. When it is not, we rely on an empirical T-test, as proposed by \citet{kuditipudi_robust_2023}
%
% This allows one to compute 

\subsection{Limitations of the Building Blocks}\label{ssec:limit_blocks}

While we design the blocks to be as independent as possible, some combinations of the scheme and specific parameters are obviously sub-optimal.
%
Here, we list a few of these subpar block combinations as a guide for practitioners.
% Even though any of the three scores can be used with any scheme and randomness source, in practice not all combinations are useful.
\begin{itemize}[leftmargin=\itemlm,itemsep=2pt]
    \item The sum score (S1) is not robust for fixed randomness (R3).
    \item The alignment score (S2) does not make sense for the text-dependent randomness (R1, R2) since misalignment is not an issue.
    \item The edit score (S3) has a robustness benefit since it can support local misalignment caused by token insertion, deletion, or swapping. However, using it with text-dependent randomness (R1, R2) only makes sense for a window size of 1: for longer window sizes, swapping, adding, or removing tokens would actually change the random values themselves, and not just misalign them.
    \item Finally, in the corner case when a window size of 0 for the text-dependent randomness (R1, R2) or when a random sequence length of 1 for the fixed randomness (R3), both the alignment score (S2) and the edit score (S3) are unnecessary since all random values are the same and misalignment is not possible.
\end{itemize}

In our experiments (\cref{sec:experiments}), we test all reasonable configurations of the randomness source, 
the sampling protocol, and the verification score, along with their parameters. 
We list the evaluated combinations in~\cref{tab:design_space_combinations}. 
The edit score is too inefficient 
to be run on all configurations, instead we rely on the sum and align scores.
%
We hope to not only fairly compare the prior works but also investigate previously unexplored combinations in the 
design space that can produce a better result.

% \chawin{We need a table or a tree that lists all the combinations we test.}\

\begin{table}[h!]
    \centering
    \caption{Tested combinations in the design space, using notations from~\cref{fig:design-figure}.\\
    We only tested the edit score {\bf S3} on a subset of watermarks.\\
    The distribution of non-watermarked scores is known for \textcolor{orange}{orange} configurations and 
    unknown for \textcolor{blue}{blue} configuration. We rely on empirical T-tests~\cite{kuditipudi_robust_2023} for blue configurations.
    }
    \label{tab:design_space_combinations}
    \normalsize
    \begin{tabular}{|l||c|c|c|c|} 
    \hline
     & \makecell[tc]{{\bf C4}\\{\small Distribution}\\{\small Shift}} & \makecell[tc]{{\bf C1}\\{\small Exponential}} & \makecell[tc]{{\bf C2}\\{\small Binary}} & \makecell[tc]{{\bf C3}\\{\small Inverse}\\{\small Transform}} \\
    \hline
    \hline
    \makecell{{\bf S1}+{\bf R1}}  & \textcolor{orange}{X} & \textcolor{orange}{X} & \textcolor{orange}{X} & \textcolor{blue}{X} \\
    \hline
    \makecell{{\bf S1}+{\bf R2}}  & \textcolor{orange}{X} & \textcolor{orange}{X} & \textcolor{orange}{X} & \textcolor{blue}{X} \\
    \hline
    \makecell{{\bf S2}+{\bf R3}}  & \textcolor{blue}{X} & \textcolor{blue}{X} & \textcolor{blue}{X} & \textcolor{blue}{X} \\
    \hline
    \makecell{{\bf S3}+{\bf R3}}  & \textcolor{blue}{X} &  &  &  \\
    \hline
    \end{tabular}
\end{table}
    

\subsection{Analysis of the edit score.} 
\label{ssec:editscore}
We analyzed the tamper-resistance of the edit score on a subset of watermarks 
(distribution-shift with $\delta=2.5$ at a temperature of 1, for key lengths between 1 and 1024). 
We tried various $\psi$ values between 0 and 1 for the edit distance, and compared the tamper-resistance 
and watermark size of the resulting verification procedures to the align score. 
Using an edit distance does improve tamper-resistance for key lengths under 32, but at a large efficiency cost: 
for key lengths above 8, the edit score size is at least twice that of the align score. 
We do not recommend using an edit score on low entropy models such as Llama-2 chat.



\section{Implementation: Ring Abstraction}
\label{sec:implement}
\subsection{Distributed \mbox{$G_t$} in QMC Solver}
\label{distributedG4}
Before introducing the communication phase of the ring abstraction layer,
it is important to understand how the authors distributed the large device array $G_t$ across MPI ranks.
%
Original $G_t$ was compared, and $G^d_t$ versions were distributed
(Figure~\ref{fig:compare_original_distributed_g4}). 


In the original $G_t$ implementation, the measurements---which were computed by matrix-matrix multiplication---are distributed statically and independently over the MPI ranks to avoid
inter-node communications. Each MPI rank keeps its partial copy of $G_{t,i}$ to accumulate 
measurements within a rank, where $i$ is the rank index. 
After all the measurements are finished, a reduction step is 
taken to accumulate $G_{t,i}$ across all MPI ranks into a final and complete
$G_t$ in the root MPI rank. The size of the $G_{t,i}$ in each rank is 
the same size as the final and complete $G_t$. 

With the distributed $G^d_t$ implementation, this large device array 
$G_t$ was evenly partitioned across all MPI ranks; each portion of it is local to each MPI rank.
%
Instead of keeping its partial copy of $G_t$, 
each rank now keeps an instance of $G^d_{t,i}$ to accumulate measurements 
of a portion or sub-slice of the final and complete $G_t$, where the notation
$d$ in $G^d_t$  refers to the distributed version, and $i$ means the $i$-th rank.
%
The $G^d_{t,i}$ size in each rank is 
reduced to $1/p$ of the size of the final and complete $G_t$, comparing the same configuration 
in original $G_t$ implementation, where $p$ is the number of MPI ranks used. 
%
For example, in Figure~\ref{fig:distributed_g4}, there are four ranks, and rank $i$
now only keeps $G^d_{t,i}$, which is one-fourth the size of the original $G_t$ array size.
% and contains values indexing from range of $[0, ..., N/4)$ in original $G_t$ array where $N$ is the 
% number of entries in  $G_t$  when viewed as a one-dimensional array.

To compute the final and complete $G^d_{t,i}$ for the distributed $G^d_t$ implementation, 
each rank must see every $G_{\sigma,i}$ from all ranks. 
%
In other words, each rank must broadcast the
locally generated $G_{\sigma,i}$ to the remainder of the other ranks at every measurement step. 
%
To efficiently perform this ``all-to-all'' broadcast, a ring abstraction layer was built (Section. \ref{section:ring_algorithm}), which circulates
all $G_{\sigma,i}$ across all ranks.

% over high-speed GPUs interconnect (GPUDirect RDMA) to facilitate the communication phase.

% \begin{figure}
% \centering
% \subfloat[Original $G_t$ implementation.]
%     {\includegraphics[width=\columnwidth]{original_g4.pdf}}\label{fig:original_g4}

% \subfloat[Distributed $G_t$ implementation.]
%     {\includegraphics[width=0.99\columnwidth]{distributed_g4.pdf} \label{fig:distributed_g4}}

% \caption{Comparison of the original $G_t$ vs. the distributed $G^d_t$ implementation. Each rank contains one GPU resource.}
% \label{fig:compare_original_distributed_g4} 
% \end{figure} 

\begin{figure}
\centering
     \begin{subfigure}[b]{\columnwidth}
         \centering
         \includegraphics[width=\textwidth]{images/original_g4.pdf}
         \caption{Original $G_t$ implementation.}
         \label{fig:original_g4}
     \end{subfigure}
     
    \begin{subfigure}[b]{\columnwidth}
         \centering
         \includegraphics[width=\textwidth]{images/distributed_g4.pdf}
         \caption{Distributed $G_t$ implementation.}
         \label{fig:distributed_g4}
     \end{subfigure}
     
\caption{Comparison of the original $G_t$ vs. the distributed $G^d_t$ implementation. Each rank contains one GPU resource.}
\label{fig:compare_original_distributed_g4}
\end{figure}

\subsection{Pipeline Ring Algorithm}
\label{section:ring_algorithm}
A pipeline ring algorithm was implemented that broadcasts the $G_{\sigma}$ 
array circularly during every measurement. 
%
The algorithm (Algorithm \ref{alg:ring_algorithm_code}) is 
visualized in Figure~\ref{fig:ring_algorithm_figure}.

\begin{algorithm}
\SetAlgoLined
    generateGSigma(gSigmaBuf)\; \label{lst:line:generateG2}
    updateG4(gSigmaBuf)\;       \label{lst:line:updateG4}
    %\texttt{\\}
    {$i\leftarrow 0$}\;         \label{lst:line:initStart}
    {$myRank \leftarrow worldRank$}\;          \label{lst:line:initRankId}
    {$ringSize \leftarrow mpiWorldSize$}\;      \label{lst:line:initRingSize}
    {$leftRank \leftarrow (myRank - 1 + ringSize) \: \% \: ringSize $}\;
    {$rightRank \leftarrow (myRank + 1 + ringSize) \: \% \: ringSize $}\;
    sendBuf.swap(gSigmaBuf)\;           \label{lst:line:initEnd}
    \While{$i< ringSize$}{
        MPI\_Irecv(recvBuf, source=leftRank, tag = recvTag, recvRequest)\; \label{lst:line:Irecv}
        MPI\_Isend(sendBuf, source=rightRank, tag = sendTag, sendRequest)\; \label{lst:line:Isend}
        MPI\_Wait(recvRequest)\;        \label{lst:line:recevBuffWait}
        
        updateG4(recvBuf)\;             \label{lst:line:updateG4_loop}
        
        MPI\_Wait(sendRequest)\;        \label{lst:line:sendBuffWait}
        
        sendBuf.swap(recvBuf)\;         \label{lst:line:swapBuff}
        i++\;
    }
\caption{Pipeline ring algorithm}
\label{alg:ring_algorithm_code}
\end{algorithm}

\begin{figure}
	\centering
	\includegraphics[width=\columnwidth, trim=0 5cm 0 0, clip]{images/ring_algorithm.pdf}
	\caption{Workflow of ring algorithm per iteration. }
	\label{fig:ring_algorithm_figure}
\end{figure}

At the start of every new measurement, a single-particle Green's function $G_{\sigma}$
 (Line~\ref{lst:line:generateG2}) is generated 
and then used to update $G^d_{t,i}$ (Line~\ref{lst:line:updateG4})
via the formula in Eq.~(\ref{eq:G4}).
%
% Different from original method that performs 
% full matrix-matrix multiplication (Equation~(\ref{eq:G4})), the current ring algorithm only performs partial
% matrix-matrix multiplication that contributes to $G^d_{t,i}$, a subslice of the final and complete $G_t$.
%
Between Lines \ref{lst:line:initStart} to \ref{lst:line:initEnd}, the algorithm 
initializes the indices
of left and right neighbors and prepares the sending message buffer from the
previously generated $G_{\sigma}$ buffer. 
%
The processes are organized as a ring so that the first and last rank are considered to be neighbors to each other. 
%
A \textit{swap} operation is used to avoid unnecessary memory copies for \textit{sendBuf} preparation.
%
A walker-accumulator thread allocates an additional \textit{recvBuf} buffer of the same size 
as \textit{gSigmaBuf} to hold incoming 
\textit{gSigmaBuf} buffer from \textit{leftRank}. 

The \textit{while} loop is the core part of the pipeline ring algorithm. 
%
For every iteration, each thread in a rank 
receives a $G_{\sigma}$ buffer from its left neighbor rank and sends a $G_{\sigma}$ buffer to its right neighbor rank. 
A synchronization step (Line~\ref{lst:line:recevBuffWait}) is performed
afterward to ensure that each rank receives a new buffer to update the 
local $G^d_{t,i}$ (Line~\ref{lst:line:updateG4_loop}). 
%
Another synchronization step
follows to ensure that all send requests are finalized 
(Line~\ref{lst:line:sendBuffWait}). Lastly, another \textit{swap} operation is used to exchange
content pointers between \textit{sendBuf} and \textit{recvBuf} to avoid unnecessary memory copy and prepare
for the next iteration of communication.
%
In the multi-threaded version (Section~\ref{subsec:multi-thread}), the thread of index, \textit{i}, only communicates with
	threads of index, \textit{i}, in neighbor ranks, and each thread allocates two buffers: \textit{sendBuff} and \textit{recvBuff}.

The \textit{while} loop will be terminated after $\mbox{\textit{ringSize}} - 1$ steps. By that time, 
each locally generated $G_{\sigma,i}$ will have traveled across all MPI ranks and
updated $G^d_{t,i}$ in all ranks. Eventually, each $G_{\sigma,i}$ reaches
to the left neighbor of its birth rank. For example, $G_{\sigma,0}$ generated from rank $0$ will end 
in last rank in the ring communicator.

Additionally, if the $G_t$ is too large to be stored in one node, 
it is optional to accumulate all $G^d_{t,i}$
at the end of all measurements. 
%
Instead, a parallel write into the file system could be taken.

\subsubsection{Sub-Ring Optimization.}

A sub-ring optimization strategy is further proposed to reduce message communication
times if the large device array $G_t$ can fit in fewer devices. 
%
The sub-ring algorithm is visualized in Figure~\ref{fig:subring_algorithm_figure}.

For the ring algorithm (Section~\ref{section:ring_algorithm}), the size of the ring communicator
(\textit{mpiWorldSize}) is set to the same size of the global \mbox{\texttt{MPI\_COMM\_WORLD}}, and thus the size of $G_t$ is equally 
distributed across all MPI ranks.

However, to complete the update to $G^d_{t,i}$ in one measurement, 
one $G_{\sigma,i}$
must travel \textit{mpiWorldSize} ranks. In total, 
there are \textit{mpiWorldSize} numbers of $G_{\sigma,i}$
being sent and received concurrently in one measurement 
in the global
\mbox{\texttt{MPI\_COMM\_WORLD}} 
communicator. If the size of $G^d_{t,i}$ is relatively small per rank, then this will cause high communication overhead.

If $G_t$ can be distributed and fitted in fewer devices, then a shorter travel distance is required 
for $G_{\sigma,i}$, thus reducing the communication overhead. One reduction
step was performed at the end of all measurements to accumulate $G^d_{t,s_i}$, 
where $s_i$ means $i$-th rank on the $s$-th sub-ring.

At the beginning of MPI initialization, the global \mbox{\texttt{MPI\_COMM\_WORLD}} was partitioned  into several new sub-ring communicators by using \mbox{\texttt{MPI\_Comm\_split}}. 
% where each new communicator represents conceptually a subring. 
The new
communicator information was passed to the DCA++ concurrency class by substituting the original global 
\mbox{\texttt{MPI\_COMM\_WORLD}} with this new communicator. Now, only a few minor modifications
are needed to transform the ring algorithm (Algorithm~\ref{alg:ring_algorithm_code})
to sub-ring Algorithm~\ref{alg:sub_ring_algorithm}. In Line~\ref{lst:line:initRankId}, \textit{myRank} is 
initialized to \textit{subRingRank} instead of \textit{worldRank}, where 
\textit{subRingRank} is the rank index in the local sub-ring communicator. 
%
In Line~\ref{lst:line:initRingSize}, \textit{ringSize} is initialized to \textit{subRingSize}
instead of \textit{mpiWorldSize}, where \textit{subRingSize} is the
size of the new communicator.
%
The general ring algorithm is a special case for the sub-ring algorithm because the
\textit{subRingSize} of the general ring algorithm is equal to \textit{mpiWorldSize}, and
there is only one sub-ring group throughout all MPI ranks.


\LinesNumberedHidden
\begin{algorithm}
    {$\mbox{\textit{myRank}} \leftarrow \mbox{\textit{subRingRank}}$}\;         
    {$\mbox{\textit{ringSize}} \leftarrow \mbox{\textit{subRingSize}}$}\;      
\caption{Modified ring algorithm to support sub-ring communication}
\label{alg:sub_ring_algorithm}
\end{algorithm}


\begin{figure}
	\centering
	\includegraphics[width=\columnwidth, trim=0 5cm 0 0, clip]{images/subring_alg.pdf}
	\caption{Workflow of sub-ring algorithm per iteration. Every consecutive $S$ rank forms a sub-ring communicator, 
	and no communication occurs between sub-ring communicators until all measurements are finished. Here, $S$ is the number of ranks in a sub-ring.}
	\label{fig:subring_algorithm_figure}
\end{figure}

\subsubsection{Multi-Threaded Ring Communication.}
\label{subsec:multi-thread}
To take advantage of the multi-threaded QMC model already in DCA++, 
multi-threaded ring communication support was further implemented in the ring algorithm.
%
Figure~\ref{fig:dca_overview} shows that in the original DCA++ method,
each walker-accumulator
thread in a rank is independent of each other, and all the threads in a 
rank synchronize only after all rank-local measurements are finished.
%
Moreover, during every measurement, each walker-accumulator thread
generates its own thread-private $G_{\sigma, i}$ to update $G_t$. 
%

The multi-threaded ring algorithm now allows concurrent message exchange so that threads of same rank-local thread index exchange their thread-private $G_{\sigma, i}$. 
%
Conceptually, there are $k$ parallel and independent rings, where $k$ 
is number of threads per rank, because threads of the same local thread ID
form a closed ring. 
%
For example, a thread of index $0$ in rank $0$ will send its $G_\sigma$ to 
the thread of index $0$ in rank $1$ and receive another $G_\sigma$ from thread index of $0$ 
from last rank in the ring algorithm.
%

The only changes in the ring algorithm are offsetting the tag values 
(\texttt{recvTag} and \texttt{sendTag}) by the thread index value. For example,
Lines~\ref{lst:line:Irecv} and ~\ref{lst:line:Isend} from 
Algorithm~\ref{alg:ring_algorithm_code} are modified to Algorithm~\ref{alg:multi_threaded_ring}.

\LinesNumberedHidden
\begin{algorithm}
        MPI\_Irecv(recvBuf, source=leftRank, tag = recvTag + threadId, recvRequest)\; 
        MPI\_Isend(sendBuf, source=rightRank, tag = sendTag + threadId, sendRequest)\;
\caption{Modified ring algorithm to support multi-threaded ring}
\label{alg:multi_threaded_ring}
\end{algorithm}

To efficiently send and receive $G_\sigma$, each thread
will allocate one additional \textit{recvBuff} to hold incoming 
\textit{gSigmaBuf} buffer from \textit{leftRank} and perform send/receive efficiently.
%
In the original DCA++ method, there are $k$ numbers of buffers of $G_\sigma$ 
size per rank, and in the multi-threaded ring method, there are $2k$
numbers of buffers of $G_\sigma$ size per rank, where $k$ is number of 
threads per rank.

\section{Evaluation}
\label{sec:evaluation}
\begin{table*}[!t]
\begin{center}
%\small
\caption {Benchmarks and applications for the study of the application-level resilience}
\vspace{-5pt}
\label{tab:benchmark}
\tiny
\begin{tabular}{|p{1.7cm}|p{7.5cm}|p{4cm}|p{2.5cm}|}
\hline
\textbf{Name} 	& \textbf{Benchmark description} 		& \textbf{Execution phase for evaluation}  			& \textbf{Target data objects}             \\ \hline \hline
CG (NPB)             & Conjugate Gradient, irregular memory access (input class S)   & The routine conj\_grad in the main computation loop  & The arrays $r$ and $colidx$     \\\hline
MG (NPB)    	       & Multi-Grid on a sequence of meshes (input class S)             & The routine mg3P in the main computation loop & The arrays $u$ and $r$ 	\\ \hline
FT (NPB)             & Discrete 3D fast Fourier Transform (input class S)            & The routine fftXYZ in the main computation loop  & The arrays $plane$ and $exp1$    \\ \hline
BT (NPB)             & Block Tri-diagonal solver (input class S)         		& The routine x\_solve in the main computation loop & The arrays $grid\_points$ and $u$	\\ \hline
SP (NPB)             & Scalar Penta-diagonal solver (input class S)         		& The routine x\_solve in the main computation loop & The arrays $rhoi$ and $grid\_points$  \\ \hline
LU (NPB)            & Lower-Upper Gauss-Seidel solver (input class S)        	& The routine ssor 	& The arrays $u$ and $rsd$ \\ \hline \hline
LULESH~\cite{IPDPS13:LULESH} & Unstructured Lagrangian explicit shock hydrodynamics (input 5x5x5) & 
The routine CalcMonotonicQRegionForElems 
& The arrays $m\_elemBC$ and $m\_delv\_zeta$ \\ \hline
AMG2013~\cite{anm02:amg} & An algebraic multigrid solver for linear systems arising from problems on unstructured grids (we use  GMRES(10) with AMG preconditioner). We use a compact version from LLNL with input matrix $aniso$. & The routine hypre\_GMRESSolve & The arrays $ipiv$ and $A$   \\ \hline
%$hierarchy.levels[0].R.V$ \\ \hline
\end{tabular}
\end{center}
\vspace{-5pt}
\end{table*}

%We evaluate the effectiveness of ARAT, and 
%We use ARAT to study the application-level resilience.
%The goal is to demonstrate 
%that aDVF can be a very useful metric to quantify the resilience of data objects
%at the application level. 
We study 12 data objects from six benchmarks of the NAS parallel benchmark (NPB) suite (we use SNU\_NPB-1.0.3) and 4 data objects from two scientific applications. 
%which is a c version of NPB 3.3, but ARAT can work for Fortran.
Those data objects are chosen to be representative: they have various data access patterns and participate in various execution phases.  
%For the benchmarks, we use CLASS S as the input problems and use the default compiler options of NPB.
For those benchmarks and applications, we use their default compiler options, and use gcc 4.7.3 and LLVM 3.4.2 for trace generation.
To count the algorithm-level fault masking, we use the default convergence thresholds (or the fault tolerance levels) for those benchmarks.
Table~\ref{tab:benchmark} gives 
%for->on by anzheng
detailed information on the benchmarks and applications.
The maximum fault propagation path for aDVF analysis is set to 10 by default.
%the value shadowing threshold is set as 0.01 (except for BT, we use $1 \times 10^{-6}$).
%These value shadowing thresholds are chosen such that any error corruption
%that results in the operand's value variance less than 1\% (for the threshold 0.01) or 0.0001\% (for the threshold $1 \times 10^{-6}$) during the 
%trace analysis does not impact the outcome correctness of six benchmarks.
%LU: check the newton-iteration residuals against the tolerance levels
%SP: check the newton-iteration residuals against the tolerance levels
%BT: check the newton-iteration residuals against the tolerance levels

\subsection{Resilience Modeling Results}
%We use ARAT to calculate aDVF values of 16 data objects. 
Figure~\ref{fig:aDVF_3tiers_profiling}
shows the aDVF results and breaks them down into the three levels 
(i.e., the operation-level, fault propagation level, and algorithm-level).
Figure~\ref{fig:aDVF_3classes_profiling} shows the 
%for->of by anzheng
results for the analyses at the levels of the operation and fault propagation,
and further breaks down the results into 
the three classes (i.e., the value overwriting, logical and comparison operations,
and value shadowing). %based on the reasons of the fault masking.
We have multiple interesting findings from the results.

\begin{figure*}
	\centering
        \includegraphics[width=0.8\textwidth]{three_tiers_gray.pdf}
% * <azguolu@gmail.com> 2017-03-23T03:20:28.808Z:
%
% ^.
        \vspace{-5pt}
        \caption{The breakdown of aDVF results based on the three level analysis. The $x$ axis is the data object name.}
        \vspace{-8pt}
        \label{fig:aDVF_3tiers_profiling}
\end{figure*}


\begin{figure*}
	\centering
	\includegraphics[width=0.8\textwidth]{three_types_gray.pdf}
	\vspace{-5pt}
	\caption{The breakdown of aDVF results based on the three classes of fault masking. The $x$ axis is the data object name. \textit{zeta} and \textit{elemBC} in LULESH are \textit{m\_delv\_zeta} and \textit{m\_elemBC} respectively.} % Anzheng
	\vspace{-5pt}
	\label{fig:aDVF_3classes_profiling}
    %\vspace{-5pt}
\end{figure*}

(1) Fault masking is common across benchmarks and applications.
Several data objects (e.g., $r$ in CG, and $exp1$ and $plane$ in FT)
have aDVF values close to 1 in Figure~\ref{fig:aDVF_3tiers_profiling}, 
which indicates that most of operations working on these data objects
have fault masking.
However, a couple of data objects have much less intensive fault masking.
For example, the aDVF value of $colidx$ in CG is 0.28 (Figure~\ref{fig:aDVF_3tiers_profiling}). 
Further study reveals that $colidx$ is an array to store column indexes of sparse matrices, and there is few operation-level or fault propagation-level fault masking  (Figure~\ref{fig:aDVF_3classes_profiling}).
The corruption of it can easily cause segmentation fault caught by the
algorithm-level analysis. 
$grid\_points$ in SP and BT also have a relatively small aDVF value (0.14 and 0.38 for SP and BT respectively in Figure~\ref{fig:aDVF_3tiers_profiling}).
Further study reveals that $grid\_points$ defines input problems for SP and BT. 
A small corruption of $grid\_points$ 
%change->changes by anzheng
can easily cause major changes in computation
caught by the fault propagation analysis. 

The data object $u$ in BT also has a relatively small aDVF value (0.82 in Figure~\ref{fig:aDVF_3tiers_profiling}).
Further study reveals that $u$ is read-only in our target code region
for matrix factorization and Jacobian, neither of which is friendly
for fault masking.
Furthermore, the major fault masking for $u$ comes from value shadowing,
and value shadowing only happens in a couple of the least significant bits 
of the operands that reference $u$, which further reduces the value of aDVF.
%also reduces fault masking.

(2) The data type is strongly correlated with fault masking.
Figure~\ref{fig:aDVF_3tiers_profiling} reveals that the integer data objects ($colidx$ in CG, $grid\_points$ in BT and SP, $m\_elemBC$ in LULESH) appear to be 
more sensitive to faults than the floating point data objects 
($u$ and $r$ in MG, $exp1$ and $plane$ in FT, $u$ and $rsd$ in LU, $m\_delv\_zeta$ in LULESH, and $rhoi$ in SP).
In HPC applications, the integer data objects are commonly employed to
define input problems and bound computation boundaries (e.g., $colidx$ in CG and $grid\_points$ in BT), 
or track computation status (e.g., $m\_elemBC$ in LULESH). Their corruption 
%these integer data objects
is very detrimental to the application correctness. 

(3) Operation-level fault masking is very common.
For many data objects, the operation-level fault masking contributes 
more than 70\% of the aDVF values. For $r$ in CG, $exp1$ in FT, and $rhoi$ in SP,
the contribution of the operation-level fault masking is close to 99\% (Figure~\ref{fig:aDVF_3tiers_profiling}).

Furthermore, the value shadowing is a very common operation level fault masking,
especially for floating point data objects (e.g., $u$ and $r$ in BT, $m\_delv\_zeta$ in LULESH, and $rhoi$ in SP in Figure~\ref{fig:aDVF_3classes_profiling}).
This finding has a very important indication for studying the application resilience.
In particular, the values of a data object can be different across different input problems. If the values of the data object are different, 
then the number of fault masking events due to the value shadowing will be different. 
Hence, we deduce that the application resilience
can be correlated with the input problems,
because of the correlation between the value shadowing and input problems. 
We must consider the input problems when studying the application resilience.
This conclusion is consistent with a very recent work~\cite{sc16:guo}.

(4) The contribution of the algorithm-level fault masking to the application resilience can be nontrivial.
For example, the algorithm-level fault masking contributes 19\% of the aDVF value for $u$ in MG and 27\% for $plane$ in FT (Figure~\ref{fig:aDVF_3tiers_profiling}).
The large contribution of algorithm-level fault masking in MG is consistent with
the results of existing work~\cite{mg_ics12}. 
For FT (particularly 3D FFT), the large contribution of algorithm-level fault masking in $plane$ (Figure~\ref{fig:aDVF_3tiers_profiling})
comes from frequent transpose and 1D FFT computations that average out 
or overwrite the data corruption.
CG, as an iterative solver, is known to have the algorithm-level fault masking
because of the iterative nature~\cite{2-shantharam2011characterizing}.
Interestingly, the algorithm-level fault masking in CG contributes most to the resilience of $colidx$ which is a vulnerable integer data object (Figure~\ref{fig:aDVF_3tiers_profiling}).

%Our study reveals the algorithm-level fault masking of CG from
%two perspectives. First, $a$ in CG, which is an array for intermediate results,
%has few algorithm-level fault masking (0.008\%);
%Second, $x$ in CG, which is a result vector, has 5.4\% of the aDVF value coming from the algorithm-level fault masking.
%This result indicates that the effects of the algorithm-level fault masking
%are not uniform across data objects. 

(5) Fault masking at the fault propagation level is small.
For all data objects, the contribution of the fault masking at the level of fault propagation is less than 5\% (Figure~\ref{fig:aDVF_3tiers_profiling}).
For 6 data objects ($r$ and $colidx$ in CG, $grid\_points$ and $u$ in BT, and 
$grid\_points$ and $rhoi$ in SP),  there is no fault masking at the level of fault propagation.
In combination with the finding 4, we conclude that once the fault
is propagated, it is difficult to mask it because of the contamination of
more data objects after fault propagation, and only the algorithm semantics can tolerate  propagated faults well. 
%This finding is consistent with our sensitivity analysis. 

(6) Fault masking by logical and comparison operations is small,
%For all data objects, the fault masking contributions due to logical and comparison operations are very small, 
comparing with the contributions of value shadowing and overwriting (Figure~\ref{fig:aDVF_3classes_profiling}). 
Among all data objects, 
the logical and comparison operations in $grid\_points$ in BT contribute the most (25\% contribution in Figure~\ref{fig:aDVF_fine_profiling}), 
because of intensive ICmp operations (integer comparison). %logical OR and SHL (left shifting).


(7) The resilience varies across data objects. %within the same application.
This fact is especially pronounced in two data objects $colidx$ and $r$ in CG (Figure~\ref{fig:aDVF_3tiers_profiling}).
 $colidx$ has aDVF much smaller than $r$, which means $colidx$ is much less resilient than $r$ (see finding 1 for a detailed analysis on $colidx$). 
Furthermore, $colidx$ and $r$ have different algorithm-level
fault masking (see finding 4 for a detailed analysis).

\begin{comment}
\textbf{Finding 7: The resilience of the same data objects varies across different applications.}
This fact is especially pronounced in BT and SP.
BT and SP address the same numerical problem but with different algorithms.
BT and SP have the same data objects, $qs$ and $rhoi$, but
$qs$ manifests different resilience in BT and SP.
This result is interesting, because it indicates that by using
different algorithms, we have opportunities to
improve the resilience of data objects.
\end{comment}

To further investigate the reasons for fault masking, 
we break down the aDVF results at the granularity of LLVM instructions,
based on the analyses at the levels of operation and fault propagation.
The results are shown in Figure~\ref{fig:aDVF_fine_profiling}.
%Because of the space limitation, 
%we only show one data object per benchmark, but each selected data object has the most diverse fault masking events within the corresponding benchmark.
%Based on Figure~\ref{fig:aDVF_fine_profiling}, we have another interesting finding.

(8) Arithmetic operations make a lot of contributions to fault masking.
%For $r$ in CG, $r$ in MG, $exp1$ in FT, $u$ in BT, $qs$ in SP, and $u$ in LU,
%the arithmetic operations, FMul (100\%), Add (16\%), FMul (85\%), 
%FMul (94\%), FMul (28\%), and FAdd (50\%)
For $r$ in CG, $u$ in BT, $plane$ and $exp1$ in FT, $m\_elemBC$ in LULESH, 
arithmetic operations (addition, multiplication, and division) contribute to almost 100\% of the fault masking (Figure~\ref{fig:aDVF_fine_profiling}).  
%(at the operation level and the fault propagation level).
%For $qs$ in SP and $u$ in LU, the store operation also makes
%important contributions as the arithmetic operations because of value overwriting.

\begin{figure*}
	\centering
	\includegraphics[width=0.77\textheight, height=0.23\textheight]{pie_chart.pdf}
	\vspace{-10pt}
	\caption{Breakdown of the aDVF results based on the analyses at the levels of operation and fault propagation}
    \vspace{-10pt}
	\label{fig:aDVF_fine_profiling}
\end{figure*}


\subsection{Sensitivity Study}
\label{sec:eval_sen}
%\textbf{change the fault propagation threshold and study the sensitivity of analysis to the threshold}
ARAT uses 10 as the default fault propagation analysis threshold. 
The fault propagation analysis will not go beyond 10 operations. Instead,
we will use deterministic fault injection after 10 operations. 
In this section, we study the impact of this threshold on the modeling accuracy. We use a range of threshold values and examine how the aDVF value varies and whether
the identification of fault masking varies. 
Figure~\ref{fig:sensitivity_error_propagation} shows the results for 
%add , after BT by anzheng
multiple data objects in CG, BT, and SP.
We perform the sensitivity study for all 16 data objects.
%in six benchmarks and two applications.
Due to the page space limitation, we only show the results for three data objects,
but we summarize the sensitivity study results for all data objects in this section.
%but other data objects in all benchmarks have the same trend.

Our results reveal that the identification of fault masking by tracking fault propagation is not significantly 
affected by the fault propagation analysis threshold. Even if we use a rather large threshold (50), 
the variation of aDVF values is 4.48\% on average among all data objects,
and the variation at each of the three levels of analysis (the operation level, fault propagation level,  and algorithm level) is less than 5.2\% on average. 
In fact, using a threshold value of 5 is sufficiently accurate in most of the cases (14 out of 16 data objects).
This result is consistent with our finding 5 (i.e., fault masking at the fault propagation level is small). %in most benchmarks).
However, we do find a data object ($m\_elementBC$ in LULESH) %and $exp1$ in FT) 
showing relatively high-sensitive (up to 15\% variation) to the threshold. For this uncommon data object, using 50 as the fault propagation path is sufficient. 

%In other words, even though using a larger threshold value can identify more error masking by tracking error 
%propagation, the implicit error masking induced by the error propagation is very limited.

\begin{figure}
		\begin{center}
		\includegraphics[width=0.48\textwidth,height=0.11\textheight]{sensi_study_gray.pdf}
		\vspace{-15pt}
		\caption{Sensitivity study for fault propagation threshold}
		\label{fig:sensitivity_error_propagation}
		\end{center}
\vspace{-15pt}
\end{figure}


\begin{comment}
\subsection{Comparison with the Traditional Random Fault Injection}
%\textbf{compare with the traditional fault injection to verify accuracy}
To show the effectiveness of our resilience modeling, we compare traditional random fault injection
and our analytical modeling. Figure~\ref{fig:comparison_fi} and Table~\ref{tab:comparison} show the results.
The figure shows the success rate of all random fault injection. The ``success'' means the application
outcome is verified successfully by the benchmarks and the execution does not have any segfault. The success rate is used as a metric
to evaluate the application resilience.

We use a data-oriented approach to perform random fault injection.
In particular, given a data object, for each fault injection test we trigger a bit flip
in an operand of a random instruction, and this operand must be a reference to the
target data object. We develop a tool based on PIN~\cite{pintool} to implement the above fault injection functionality.
For each data object, we conduct five sets of random fault injection tests, 
and each set has 200 tests (in total 1000 tests per data object). 
We show the results for CG and FT in this section, but we find that
the conclusions we draw from CG and FT are also valid for the other four benchmarks.


%\begin{table*}
%\label{tab:success_rate}
%\begin{centering}
%\renewcommand\arraystretch{1.1}
%\begin{tabular}{|c|c|c|c|c|c|c|}
%\hline 
%Success Rate (Difference) & Test set 1 & Test set 2 & Test set 3 & Test set 4 & Test set 5 & Average\tabularnewline
%\hline 
%\hline 
%CG-a & 66.1\% (11.7\%) & 68.5\% (15.7\%) & 56.7\% (4.21\%) & 61.3\% (3.57\%) & 43.3\% (26.8\%) & 59.2\%\tabularnewline
%\hline 
%CG-x & 99.2\% (2.2\%) & 98.6\% (1.5\%) & 96.5\% (0.63\%) & 97.8\% (0.64\%) & 93.6\% (3.7\%) & 97.1\%\tabularnewline
%\hline 
%CG-colidx & 36.8\% (12.7\%) & 49.6\% (17.8\%) & 40.2\% (4.6\%) & 52.6\% (24.9\%) & 31.4\% (25.4\%) & 42.1\%\tabularnewline
%\hline 
%FT-exp1 & 52.7\% (1.4\%) & 22.6\% (56.5\%) & 78.5\% (51.0\%) & 60.7\% (16.7\%) & 45.4\% (12.7\%) & 51.9\%\tabularnewline
%\hline 
%FT-plane & 82.1\% (2.5\%) & 79.3\% (5.6\%) & 99.5\% (18.2\%) & 93.2\% (10.7\%) & 66.8\% (20.6\%) & 84.2\%\tabularnewline
%\hline 
%\end{tabular}
%\par\end{centering}
%\caption{XXXXX}
%\end{table*}


\begin{table*}
\begin{centering}
\caption{\small The results for random fault injection. The numbers in parentheses for each set of tests (200 tests per set) are the success rate difference from the average success rate of 1000 fault injection tests.}
\label{tab:comparison}
\renewcommand\arraystretch{1.1}
\begin{tabular}{|c|p{2.2cm}|p{2.2cm}|p{2.2cm}|p{2.2cm}|p{2.2cm}|p{1.8cm}|}
\hline 
       %& Test set 1 & Test set 2 & Test set 3 & Test set 4 & Test set 5 & Average\tabularnewline
       & \hspace{13pt} Test set 1 \hspace{1pt}/  & \hspace{13pt} Test set 2 \hspace{1pt}/ & \hspace{13pt} Test set 3 \hspace{1pt}/ & \hspace{13pt} Test set 4 \hspace{1pt}/ & \hspace{13pt} Test set 5 \hspace{1pt}/ & Ave. of all test / \\
       & success rate (diff.) & success rate (diff.) & success rate (diff.) & success rate (diff.) & success rate (diff.) & \hspace{5pt} success rate \\
\hline 
\hline 
CG-a & 66.1\% (6.9\%) & 68.5\% (9.3\%) & 56.7\% (-2.5\%) & 61.3\% (2.1\%) & 43.3\% (-15.9\%) & 59.2\%\tabularnewline
\hline 
CG-x & 99.2\% (2.1\%) & 98.6\% (1.5\%) & 96.5\% (-0.6\%) & 97.8\% (0.7\%) & 93.6\% (-3.5\%) & 97.1\%\tabularnewline
\hline 
CG-colidx & 36.8\% (-5.3\%) & 49.6\% (7.5\%) & 40.2\% (-2.0\%) & 52.6\% (10.5\%) & 31.4\% (-10.7\%) & 42.1\%\tabularnewline
\hline 
FT-exp1 & 52.7\% (0.8\%) & 22.6\% (-29.3\%) & 78.5\% (26.6\%) & 60.7\% (8.8\%) & 45.4\% (-6.5\%) & 51.9\%\tabularnewline
\hline 
FT-plane & 82.1\% (-2.1\%) & 79.3\% (-4.9\%) & 99.5\% (15.3\%) & 93.2\% (9.0\%) & 66.8\% (-17.4\%) & 84.2\%\tabularnewline
\hline 
\end{tabular}
\par\end{centering}
\vspace{-0.4cm}
\end{table*}

\begin{figure}
	\begin{center}
		\includegraphics[width=0.48\textwidth,keepaspectratio]{verifi-study.png}
		\caption{The traditional random fault injection vs. ARAT}
		\label{fig:comparison_fi}
	\end{center}
\vspace{-0.7cm}
\end{figure}


We first notice from Table~\ref{tab:comparison} that 
%across 5 sets of random fault injection tests, there are big variances (up to 55.9\% in $exp1$ of FT) in terms of the success rate. 
the results of 5 test sets can be quite different from each other and from 1000 random fault inject tests (up to 29.3\%).
1000 fault injection tests provide better statistical significance than 200 fault injection tests.
We expect 1000 fault injection tests potentially provide higher accuracy to quantify the application resilience.
The above result difference is clearly an indication to the randomness of fault injection, and there
is no guarantee on the random fault injection accuracy.

%In Figure~\ref{fig:comparison_fi}, 
We compare the success rate of 1000 fault inject tests with the aDVF value (Fig.~\ref{fig:comparison_fi}). 
We find that the order of the success rate of the three data objects in CG (i.e., $colidx < a < x$) and the two data objects in FT 
(i.e., $exp1 < plane$) is the same as the order of the aDVF values of these data objects. 
%In fact, 1000 fault injection tests
%account for \textcolor{blue}{\textbf{xxx\%}} of total memory references to the data object,
%and provide better resilience quantification than 200 fault injection tests.
The same order (or the same resilience trend)
%between our approach and the random fault injection based on a large number of tests 
is a demonstration of the effectiveness of our approach.
Note that the values of the aDVF and success rate %for a data object
cannot be exactly the same (even if we have sufficiently large numbers of random fault injection), 
because aDVF and random fault injection quantify
the resilience based on different metrics.
Also, the random fault injection can miss some fault masking events that can be captured by our approach.

\end{comment}
%\mySection{Related Works and Discussion}{}
\label{chap3:sec:discussion}

In this section we briefly discuss the similarities and differences of the model presented in this chapter, comparing it with some related work presented earlier (Chapter \ref{chap1:artifact-centric-bpm}). We will mention a few related studies and discuss directly; a more formal comparative study using qualitative and quantitative metrics should be the subject of future work.

Hull et al. \citeyearpar{hull2009facilitating} provide an interoperation framework in which, data are hosted on central infrastructures named \textit{artifact-centric hubs}. As in the work presented in this chapter, they propose mechanisms (including user views) for controlling access to these data. Compared to choreography-like approach as the one presented in this chapter, their settings has the advantage of providing a conceptual rendezvous point to exchange status information. The same purpose can be replicated in this chapter's approach by introducing a new type of agent called "\textit{monitor}", which will serve as a rendezvous point; the behaviour of the agents will therefore have to be slightly adapted to take into account the monitor and to preserve as much as possible the autonomy of agents.

Lohmann and Wolf \citeyearpar{lohmann2010artifact} abandon the concept of having a single artifact hub \cite{hull2009facilitating} and they introduce the idea of having several agents which operate on artifacts. Some of those artifacts are mobile; thus, the authors provide a systematic approach for modelling artifact location and its impact on the accessibility of actions using a Petri net. Even though we also manipulate mobile artifacts, we do not model artifact location; rather, our agents are equipped with capabilities that allow them to manipulate the artifacts appropriately (taking into account their location). Moreover, our approach considers that artifacts can not be remotely accessed, this increases the autonomy of agents.

The process design approach presented in this chapter, has some conceptual similarities with the concept of \textit{proclets} proposed by Wil M. P. van der Aalst et al. \citeyearpar{van2001proclets, van2009workflow}: they both split the process when designing it. In the model presented in this chapter, the process is split into execution scenarios and its specification consists in the diagramming of each of them. Proclets \cite{van2001proclets, van2009workflow} uses the concept of \textit{proclet-class} to model different levels of granularity and cardinality of processes. Additionally, proclets act like agents and are autonomous enough to decide how to interact with each other.

The model presented in this chapter uses an attributed grammar as its mathematical foundation. This is also the case of the AWGAG model by Badouel et al. \citeyearpar{badouel14, badouel2015active}. However, their model puts stress on modelling process data and users as first class citizens and it is designed for Adaptive Case Management.

To summarise, the proposed approach in this chapter allows the modelling and decentralized execution of administrative processes using autonomous agents. In it, process management is very simply done in two steps. The designer only needs to focus on modelling the artifacts in the form of task trees and the rest is easily deduced. Moreover, we propose a simple but powerful mechanism for securing data based on the notion of accreditation; this mechanism is perfectly composed with that of artifacts. The main strengths of our model are therefore : 
\begin{itemize}
	\item The simplicity of its syntax (process specification language), which moreover (well helped by the accreditation model), is suitable for administrative processes;
	\item The simplicity of its execution model; the latter is very close to the blockchain's execution model \cite{hull2017blockchain, mendling2018blockchains}. On condition of a formal study, the latter could possess the same qualities (fault tolerance, distributivity, security, peer autonomy, etc.) that emanate from the blockchain;
	\item Its formal character, which makes it verifiable using appropriate mathematical tools;
	\item The conformity of its execution model with the agent paradigm and service technology.
\end{itemize}
In view of all these benefits, we can say that the objectives set for this thesis have indeed been achieved. However, the proposed model is perfectible. For example, it can be modified to permit agents to respond incrementally to incoming requests as soon as any prefix of the extension of a bud is produced. This makes it possible to avoid the situation observed on figure \ref{chap3:fig:execution-figure-4} where the associated editor is informed of the evolution of the subtree resulting from $C$ only when this one is closed. All the criticisms we can make of the proposed model in particular, and of this thesis in general, have been introduced in the general conclusion (page \pageref{chap5:general-conclusion}) of this manuscript.




% \vspace{-0.5em}
\section{Conclusion}
% \vspace{-0.5em}
Recent advances in multimodal single-cell technology have enabled the simultaneous profiling of the transcriptome alongside other cellular modalities, leading to an increase in the availability of multimodal single-cell data. In this paper, we present \method{}, a multimodal transformer model for single-cell surface protein abundance from gene expression measurements. We combined the data with prior biological interaction knowledge from the STRING database into a richly connected heterogeneous graph and leveraged the transformer architectures to learn an accurate mapping between gene expression and surface protein abundance. Remarkably, \method{} achieves superior and more stable performance than other baselines on both 2021 and 2022 NeurIPS single-cell datasets.

\noindent\textbf{Future Work.}
% Our work is an extension of the model we implemented in the NeurIPS 2022 competition. 
Our framework of multimodal transformers with the cross-modality heterogeneous graph goes far beyond the specific downstream task of modality prediction, and there are lots of potentials to be further explored. Our graph contains three types of nodes. While the cell embeddings are used for predictions, the remaining protein embeddings and gene embeddings may be further interpreted for other tasks. The similarities between proteins may show data-specific protein-protein relationships, while the attention matrix of the gene transformer may help to identify marker genes of each cell type. Additionally, we may achieve gene interaction prediction using the attention mechanism.
% under adequate regulations. 
% We expect \method{} to be capable of much more than just modality prediction. Note that currently, we fuse information from different transformers with message-passing GNNs. 
To extend more on transformers, a potential next step is implementing cross-attention cross-modalities. Ideally, all three types of nodes, namely genes, proteins, and cells, would be jointly modeled using a large transformer that includes specific regulations for each modality. 

% insight of protein and gene embedding (diff task)

% all in one transformer

% \noindent\textbf{Limitations and future work}
% Despite the noticeable performance improvement by utilizing transformers with the cross-modality heterogeneous graph, there are still bottlenecks in the current settings. To begin with, we noticed that the performance variations of all methods are consistently higher in the ``CITE'' dataset compared to the ``GEX2ADT'' dataset. We hypothesized that the increased variability in ``CITE'' was due to both less number of training samples (43k vs. 66k cells) and a significantly more number of testing samples used (28k vs. 1k cells). One straightforward solution to alleviate the high variation issue is to include more training samples, which is not always possible given the training data availability. Nevertheless, publicly available single-cell datasets have been accumulated over the past decades and are still being collected on an ever-increasing scale. Taking advantage of these large-scale atlases is the key to a more stable and well-performing model, as some of the intra-cell variations could be common across different datasets. For example, reference-based methods are commonly used to identify the cell identity of a single cell, or cell-type compositions of a mixture of cells. (other examples for pretrained, e.g., scbert)


%\noindent\textbf{Future work.}
% Our work is an extension of the model we implemented in the NeurIPS 2022 competition. Now our framework of multimodal transformers with the cross-modality heterogeneous graph goes far beyond the specific downstream task of modality prediction, and there are lots of potentials to be further explored. Our graph contains three types of nodes. while the cell embeddings are used for predictions, the remaining protein embeddings and gene embeddings may be further interpreted for other tasks. The similarities between proteins may show data-specific protein-protein relationships, while the attention matrix of the gene transformer may help to identify marker genes of each cell type. Additionally, we may achieve gene interaction prediction using the attention mechanism under adequate regulations. We expect \method{} to be capable of much more than just modality prediction. Note that currently, we fuse information from different transformers with message-passing GNNs. To extend more on transformers, a potential next step is implementing cross-attention cross-modalities. Ideally, all three types of nodes, namely genes, proteins, and cells, would be jointly modeled using a large transformer that includes specific regulations for each modality. The self-attention within each modality would reconstruct the prior interaction network, while the cross-attention between modalities would be supervised by the data observations. Then, The attention matrix will provide insights into all the internal interactions and cross-relationships. With the linearized transformer, this idea would be both practical and versatile.

% \begin{acks}
% This research is supported by the National Science Foundation (NSF) and Johnson \& Johnson.
% \end{acks}

\begin{acks}
We thank the anonymous reviewers for their valuable comments. We thank Denys Poshyvanyk for his feedback to the paper. This work is supported in part by NSF grants CNS-2050007, CRII-1755769, OAC-1835821, IIS-2008557, CCF-1703487, CCF-2028850 and CCF-2047516, a Department of Energy (DOE) grant DE-SC0013700. %Jialiang thanks her beloved border collie Kunkun, for her loyalty and companion.
\end{acks}


%%
%% The next two lines define the bibliography style to be used, and
%% the bibliography file.
\bibliographystyle{ACM-Reference-Format}
\bibliography{fse2021}

%%
%% If your work has an appendix, this is the place to put it.
\appendix

%\section{Research Methods}

\end{document}
\endinput
%%
%% End of file `sample-sigconf.tex'.
