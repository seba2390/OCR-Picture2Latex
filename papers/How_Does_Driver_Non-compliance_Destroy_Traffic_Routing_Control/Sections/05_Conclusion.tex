%%%%%%%%%%%%%%%%%%%%%%%%%%%%%%%%%%%%%%%%%%%%%%%%%%%%%%%%%%%%%%%%%%%%%%%%%%%%%%%%
\section{Concluding remarks}
\label{sec_conclusion}

In this paper, we considered the traffic stability and throughput of a parallel-link network subject to non-compliant traffic flows. We formulated a Markov chain that captures the traffic evolution under a dynamic routing strategy and in the face of a state-dependent non-compliance rate of drivers. Using Lyapunov methods, we derived stability conditions for several typical settings with or without traffic spillbacks. We also used the results to analytically quantify the impact of driver non-compliance on network throughput.
Possible future directions include extension of the results to general single-origin-single-destination networks with cyclic structures and multi-commodity scenarios.


\appendix
\subsection{Proof of Lemma~\ref{lmm_1}}
\label{app_pf_lmm1}

By Proposition 7.1.4 in  \cite{meyn2012markov}, we see that the system \eqref{eq_inf_1}-\eqref{eq_inf_2} is forward accessible. Then Assumption~4.1 indicates that the Markov chain~\eqref{eq_markovchain_1} is $\varphi$-irreducible by Theorem 7.2.6 in \cite{meyn2012markov}. 

Since we assume the flow functions are continuous (see Assumptions~1-2), there exists a neighborhood such that \eqref{eq_asm4_1}-\eqref{eq_asm4_2} hold. It implies that the system \eqref{eq_fin_1}-\eqref{eq_fin_3} is forward accessible at leat over the neighborhood. Then the Markov chain \eqref{eq_markovchain_2} is also $\varphi$-irreducible.