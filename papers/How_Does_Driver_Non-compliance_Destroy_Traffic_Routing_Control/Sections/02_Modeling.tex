%%%%%%%%%%%%%%%%%%%%%%%%%%%%%%%%%%%%%%%%%%%%%%%%%%%%%%%%%%%%%%%%%%%%%%%%%%%%%%%%
\section{Modeling and formulation}
\label{sec_modeling}

Consider the two-link network in Fig.~\ref{fig_twolink}: one is the major link $e_1$, typically with a higher free-flow speed or capacity, and the other is the minor link $e_2$. We suppose that the system operator tries to route part of flows to the minor link $e_2$ to reduce congestion in the major link $e_1$.

We denote by $X_e(t)\in\mathbb{R}_{\geq0}$ traffic density of link $e\in\{e_1,e_2\}$ at time $t$. Each link $e\in\{e_1, e_2\}$ is associated with a sending flow $f_e(x_e):\mathbb{R}_{\geq0}\to\mathbb{R}_{\geq0}$ and a receiving flow $r_e(x_e):\mathbb{R}_{\geq0}\to\mathbb{R}_{\geq0}$. Here the sending flow $f_e(x_e)$ indicates the desired outflow from link $e$ given traffic density $x_e$, and the receiving flow $r_e(x_e)$ stands for the maximum flow allowed into link $e$. We assume that the flow functions satisfy:
\begin{asm}[Sending \& receiving flows]
\label{asm_1}
\quad

\begin{enumerate}
    \item[1.1] Sending flows: For link $e$, $f_e(x_e)$ is Lipschitz continuous and $\mathrm{d}f_e(x_e)/\mathrm{d}x_e\geq0$ almost everywhere (a.e.). Moreover, $f_e(0)=0$ and $\sup_{x_e}f_e(x_e)<\infty$.
    %
    \item[1.2] Receiving flows: For link $e$ with a finite buffer size $x_e^{\max}<\infty$, $r_e(x_e)$ is Lipschitz continuous and $\mathrm{d}r_e(x_e)/\mathrm{d}x_e\leq0$ a.e.. Moreover, $r_e(x_e^{\max})=0$ and $\sup_{x_e}r_e(x_e)<\infty$. For link $e$ with an infinite buffer size, $r_e=\infty$. 
\end{enumerate}
\end{asm}

The assumptions above follow the conventional modeling of road traffic. We also define \emph{link capacity} as
\begin{equation}
    Q_e := \sup_{x_e} \min\{f_e(x_e), r_e(x_e)\},
\end{equation}
which denotes an upper bound of sustainable discharging flow from link $e$.

Note that Assumption~\ref{asm_1}.2 implies that it is reasonable to only consider $X_e(t)\in[0, x_e^{\max}]$ for link $e$ with limited storage. Compared with supposing finite buffer sizes, the assumption of infinite buffer sizes seems a little unrealistic, but it helps understand and design routing control, even on complex networks. In this paper, we discuss both of them.

For demand modeling, we consider an independent and identically distributed (i.i.d.) stochastic process $\{D(t):t\geq0\}$ with a distribution $\Gamma^d$, $\mathbb{E}[D(t)]=\alpha$ and $D(t)\in\mathcal{D}$ for $t\geq 0$, where $\mathcal{D}$ is a compact set. This is based on the observation that during rush hours, of interest to traffic management, traveling demands are relatively stationary and only fluctuate within certain bounds \cite{adot}. Obviously, we require 
\begin{equation}
    \mathbb{E}[D(t)]=\alpha < Q_{e_1} + Q_{e_2}, \label{eq_necess}
\end{equation}
otherwise the traffic densities must blow up.

Next, we introduce routing control. Let $\beta_e(x):\mathbb{R}^2_{\geq0}\to[0,1]$ denote a proportion of traffic routed to link $e$. We assume the routing policies to satisfy:
\begin{asm}[Routing control] 
The routing proportions $\beta_{e_1}(x_{e_1}, x_{e_2})$ and $\beta_{e_2}(x_{e_1}, x_{e_2})$ are continuous and have the following monotonicity a.e.:
\begin{enumerate}
    \item[2.1] $\frac{\partial}{\partial x_{e_1}}\beta_{e_1}(x_{e_1},x_{e_2}) \leq 0$ and  $\frac{\partial}{\partial x_{e_2}}\beta_{e_1}(x_{e_1},x_{e_2}) \geq 0$;
    %
    \item[2.2] $\frac{\partial}{\partial x_{e_1}}\beta_{e_2}(x_{e_1},x_{e_2}) \geq 0$ and  $\frac{\partial}{\partial x_{e_2}}\beta_{e_2}(x_{e_1},x_{e_2}) \leq 0$.
\end{enumerate}
\end{asm}

The assumption above implies that the routing proportion $\beta_{e}(x_{e_1},x_{e_2})$ tends to decrease (resp. increase) as link $e$ (resp. the other link) becomes more congested. It holds true for typical routing policies, such as logit routing \cite{como2011robust}.

Recalling that routing proportions could be compromised due to heterogeneous drivers' choice behavior, we denote by $C(t)\in[0, 1]$ the compliance rate of drivers' routed to the minor link $e_2$ at time $t$. Then the compromised routing ratios, denoted by $\tilde{\beta}_e(x_{e_1},x_{e_2},c):\mathbb{R}^2_{\geq0}\times[0, 1]\to[0,1]$, are given by
\begin{subequations}
    \begin{align}
        &\tilde{\beta}_{e_1}(X_{e_1}(t), X_{e_2}(t), C(t)) \nonumber \\
        &=\beta_{e_1}(X_{e_1}(t), X_{e_2}(t)) + \beta_{e_2}(X_{e_1}(t), X_{e_2}(t))(1-C(t)), \label{eq_comp_1} \\
        &\tilde{\beta}_{e_2}(X_{e_1}(t),X_{e_2}(t), C(t)) \nonumber \\
        &= \beta_{e_2}(X_{e_1}(t), X_{e_2}(t))C(t). \label{eq_comp_2}
    \end{align}
\end{subequations}
Note that \eqref{eq_comp_1}-\eqref{eq_comp_2} imply the compliance rate of drivers routed to the major link $e_1$ equals one. This is because in our setting drivers are assumed to prefer the major link $e_1$ while the system operator tries to route some of them to the minor link $e_2$. The assumption is not necessary, just for simplifying the problem. In fact, we can introduce the second compliance rate, and apply our method to obtain stability and instability criteria, which are more complicated.

We consider that $C(t+1)$ depends on $X_{e_1}(t)=x_{e_1}$ and $X_{e_2}(t)=x_{e_2}$ with a distribution $\Gamma^c_{x_{e_1},x_{e_2}}$. For convenience of analysis, we assume that the distributions $\Gamma^c_{x_{e_1},x_{e_2}}(c)$, for any $x_{e_1}$ and $x_{e_2}$, have lower semi-continuous densities with the same support $\mathcal{C}\subseteq [0,1]$. We define $\mathbb{E}_{x_{e_1},x_{e_2}}[C]:=\mathbb{E}[C(t+1)|X_{e_1}(t)=x_{e_1},X_{e_2}(t)=x_{e_2}]$ and assume it to satisfy:
\begin{asm}[Drivers' compliance] The expected compliance rate has the following monotonicity a.e.:
\begin{equation}
    \frac{\partial}{\partial x_{e_1}} \mathbb{E}_{x_{e_1},x_{e_2}}[C] \geq0\text{, and } \frac{\partial}{\partial x_{e_2}} \mathbb{E}_{x_{e_1},x_{e_2}}[C] \leq0. 
\end{equation}
\end{asm}

Clearly, the assumption implies that more drivers follow the routing advise to the minor link $e_2$ if the major link $e_1$ becomes more congested or the minor link $e_2$ becomes less congested.

The following specifies the inflows into links $e_1$ and $e_2$. Given an upstream flow $F(t)$, we denote by $q_e^{\mathrm{in}}:\mathbb{R}_{\geq0}^3\times[0,1]\to\mathbb{R}_{\geq0}$ the inflow into link $e\in\{e_1,e_2\}$:
\begin{align}
&q_e^{\mathrm{in}}(F(t), X_{e_1}(t), X_{e_2}(t), C(t)) \nonumber \\
&= \min\{\tilde{\beta}_e(X_{e_1}(t), X_{e_2}(t), C(t)) F(t), r_{e}(X_{e}(t))\}.
\end{align}

Supposing $r_{e_1}=r_{e_2}=\infty$, we have the following network dynamics:
\begin{subequations}
    \begin{align}
        \Delta X_{e_1}(t) =& \frac{\delta}{l_{e_1}}\Big(q_{e_1}^{\mathrm{in}}(D(t), X_{e_1}(t), X_{e_2}(t), C(t)) \nonumber \\ 
        & - f_{e_1}(X_{e_1}(t))\Big), \label{eq_inf_1} \\
        %
        \Delta X_{e_2}(t) =& \frac{\delta}{l_{e_2}}\Big(q_{e_2}^{\mathrm{in}}(D(t), X_{e_1}(t), X_{e_2}(t), C(t)) \nonumber \\ 
        & - f_{e_2}(X_{e_2}(t))\Big). \label{eq_inf_2}
    \end{align}
\end{subequations}
where $\Delta X_e(t):=X_e(t+1)-X_e(t)$ for any link $e$, $\delta$ denotes the time step size and $l_e$ denotes length of link $e$.

Clearly, if links $e_1$ and $e_2$ have finite space, congestion could block the inflows. For the sake of analysis, we consider another link $e_0$ upstream of links $e_1$ and $e_2$, satisfying $Q_{e_0}\geq Q_{e_1}+Q_{e_2}$ and $r_{e_0}=\infty$, to accept inflows. It leads to the network dynamics as follows:
\begin{subequations}
    \begin{align}
        &\Delta X_{e_0}(t) = \frac{\delta}{l_{e_0}}\Big(D(t) \nonumber \\
        &-\sum_{e\in\{e_1,e_2\}} q_e^{\mathrm{in}}(f_{e_0}(X_{e_0}(t)), X_{e_1}(t), X_{e_2}(t), C(t)) \Big) \label{eq_fin_1} \\
        %
        &\Delta X_{e_1}(t) = \frac{\delta}{l_{e_1}}\Big(q_{e_1}^{\mathrm{in}}(f_{e_0}(X_{e_0}(t)), X_{e_1}(t), X_{e_2}(t), C(t)) \nonumber \\
        & - f_{e_1}(X_{e_1}(t))\Big), \label{eq_fin_2}  \\
         %
        &\Delta X_{e_2}(t) = \frac{\delta}{l_{e_2}}\Big(q_{e_2}^{\mathrm{in}}(f_{e_0}(X_{e_0}(t)), X_{e_1}(t), X_{e_2}(t), C(t)) \nonumber \\
        &-f_{e_2}(X_{e_2}(t))\Big). \label{eq_fin_3} 
    \end{align}
\end{subequations}
For notional convenience, we assume $\delta/l_e$  to be the same for any link $e$ and ignore them in the following analysis.

Then, \eqref{eq_inf_1}-\eqref{eq_inf_2} indicate that
\begin{equation}
    \Phi_1:=\{(X_{e_1}(t), X_{e_2}(t), D(t), C(t)):t\geq0\} \label{eq_markovchain_1} 
\end{equation}
is a Markov chain with a state space $\mathbb{R}_{\geq0}\times\mathbb{R}_{\geq0}\times\mathcal{D}\times\mathcal{C}$, and \eqref{eq_fin_1}-\eqref{eq_fin_3} indicate 
\begin{equation}
    \Phi_2:=\{(X_{e_0}(t), X_{e_1}(t), X_{e_2}(t), D(t), C(t)):t\geq0\} \label{eq_markovchain_2}
\end{equation}
is also a Markov chain with a state space $\mathbb{R}_{\geq0}\times\mathcal{X}_{e_1}\times\mathcal{X}_{e_2}\times\mathcal{D}\times\mathcal{C}$. Note that $\mathcal{X}_{e_1}\subseteq[0, x_{e_1}^{\max}]$ and $\mathcal{X}_{e_2}\subseteq[0, x_{e_2}^{\max}]$ are bounded sets.

We make the last assumption as follows:
\begin{asm}
\quad

\begin{enumerate}
    %
    \item[4.1] For the system \eqref{eq_inf_1}-\eqref{eq_inf_2}, there exists $c\in\mathcal{C}$ and $d\in\mathcal{D}$ such that $\lim_{t\to\infty} X_e(t)=x_e^*<\infty$, $e\in\{e_1,e_2\}$, given $C(t)\equiv c$ and $D(t)\equiv d$;
    %
    \item[4.2] For the system \eqref{eq_fin_1}-\eqref{eq_fin_3}, there exists $c\in\mathcal{C}$ and $d\in\mathcal{D}$ such that $\lim_{t\to\infty} X_e(t)=x_e^*<\infty$, $e\in\{e_0,e_1,e_2\}$, given $C(t)\equiv c$ and $D(t)\equiv d$. Moreover, 
    \begin{subequations}
        \begin{align}
            \tilde{\beta}_{e_1}((x_{e_1}^*, x_{e_2}^*), c)f_{e_0}(x_{e_0}^*) <& r_{e_1}(x_{e_1}^*), \label{eq_asm4_1} \\
            %
            \tilde{\beta}_{e_2}((x_{e_1}^*, x_{e_2}^*), c)f_{e_0}(x_{e_0}^*) <& r_{e_2}(x_{e_2}^*). \label{eq_asm4_2} 
        \end{align}
    \end{subequations}
\end{enumerate}
\end{asm}

The above assumption essentially states that there exists $c$ and $d$ such that the systems \eqref{eq_inf_1}-\eqref{eq_inf_2} and \eqref{eq_fin_1}-\eqref{eq_fin_3} are stable. Note that \eqref{eq_asm4_1}-\eqref{eq_asm4_2} are mild technical assumptions. The system \eqref{eq_inf_1}-\eqref{eq_inf_2} does not require them due to $r_{e_1}=r_{e_2}=\infty$. The equations \eqref{eq_asm4_1}-\eqref{eq_asm4_2} imply that the inflows into links $e_1$ and $e_2$ are strictly fewer than the corresponding receiving flows. That is, the inflow can smoothly pass links $e_1$ and $e_2$ when there is no congestion. By noting \eqref{eq_necess},  \eqref{eq_asm4_1}-\eqref{eq_asm4_2} are easy to achieve for appropriate routing polices.

We have the following lemma proved in Appendix~\ref{app_pf_lmm1}:
\begin{lmm} 
\label{lmm_1}
Given Assumption~4.1, the Markov chain \eqref{eq_markovchain_1} is $\varphi$-irreducible; and given Assumption~4.2, the Markov chain \eqref{eq_markovchain_2} is $\varphi$-irreducible.
\end{lmm}

Here $\varphi$ is a certain measure. The $\varphi$-irreducibility means that any set with positive measure can be reached by the Markov chain given any initial state. It implies that any large set can be reached from any initial condition and thus the state space is indecomposable. It is a prerequisite of discussing stability of Markov chains.

Finally, we define the stability of interest below:
\begin{dfn}[Stability \& Instability]
A stochastic process $\{Y(t):t\geq0\}$ with a state space $\mathcal{Y}$ is \emph{stable} if there exists a scalar $Z<\infty$ such that for any initial condition $y\in\mathcal{Y}$
\begin{equation}\label{eq_bounded}
  \limsup_{t\to\infty}\frac{1}{t}\sum_{\tau=0}^t\mathbb {E}[|Y(\tau)|] \le Z,
\end{equation}
where $|Y(\tau)|$ denotes 1-norm of $Y(\tau)$. The network is \emph{unstable} if there does not exist $Z<\infty$ such that \eqref{eq_bounded} holds for any initial condition $y\in\mathcal{Y}$.
\end{dfn}

The notion of stability follows a classical definition \cite{dai1995stability} and is widely used in studying traffic control \cite{barman2023throughput}. Practically, if the time-average traffic density in all links are bounded, the network is stable; otherwise, it is unstable.






%\addtolength{\textheight}{-3cm}   
% This command serves to balance the column lengths on the last page of the document manually. It shortens the textheight of the last page by a suitable amount. This command does not take effect until the next page so it should come on the page before the last. Make sure that you do not shorten the textheight too much.

%%%%%%%%%%%%%%%%%%%%%%%%%%%%%%%%%%%%%%%%%%%%%%%%%%%%%%%%%%%%%%%%%%%%%%%%%%%%%%%%