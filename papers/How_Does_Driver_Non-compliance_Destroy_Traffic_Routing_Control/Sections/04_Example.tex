\section{Stability analysis of the network with congestion propagation}
\label{sec_example}

We state the main results as follows:
\begin{thm}
\label{thm_2}
Given Assumptions~\ref{asm_1}-4, the Markov chain \eqref{eq_markovchain_2} with the state space $\mathbb{R}_{\geq0}\times\mathcal{X}_{e_1}\times\mathcal{X}_{e_2}\times\mathcal{D}\times\mathcal{C}$ is stable if there exists a vector $\theta:=[\theta_{e_1},\theta_{e_2}]^{\mathrm{T}}\in[0, 1]^2$ and a positive scalar $\gamma>0$ such that
\begin{align}
    &\alpha- \sum_{e\in\{e_1,e_2\}} (1-\theta_{e})\mathbb{E}_{x_{e_1},x_{e_2}}[q_{e}^{\mathrm{in}}(f_{e_0}(x_{e_0}^c), x_{e_1},x_{e_2}, C)]- \nonumber \\
    & - \sum_{e\in\{e_1,e_2\}}\theta_{e} f_{e}(x_{e}) < -\gamma, ~\forall (x_{e_1},x_{e_2})\in\mathcal{X}_{e_1}\times\mathcal{X}_{e_2}, 
    \label{eq_thm2_1}
\end{align}
where $x_{e_0}^c:=\inf\{x_{e_0}|f_{e_0}(x_{e_0})=Q_{e_0}\}$ and
\begin{align}
    &\mathbb{E}_{x_{e_1},x_{e_2}}[q_{e}^{\mathrm{in}}(f_{e_0}(x_{e_0}), x_{e_1},x_{e_2}, C)] \nonumber \\
    &:= \int_{\mathcal{C}} q_{e}^{\mathrm{in}}(f_{e_0}(x_{e_0}), x_{e_1}, x_{e_2}, c)) \Gamma_{x_{e_1,e_2}}(\mathrm{d}c).
\end{align}
\end{thm}

\begin{thm}
\label{thm_3}
Given Assumptions~\ref{asm_1}-4, the Markov chain \eqref{eq_markovchain_2} with the state space $\mathbb{R}_{\geq0}\times\mathcal{X}_{e_1}\times\mathcal{X}_{e_2}\times\mathcal{D}\times\mathcal{C}$ is unstable if there exists a vector $\theta:=[\theta_{e_1},\theta_{e_2}]^{\mathrm{T}}\in[0,1]^2$ and a non-negative scalar $\gamma\geq0$ such that
\begin{align}
    &\alpha- \sum_{e\in\{e_1,e_2\}} (1-\theta_{e})\mathbb{E}_{x_{e_1},x_{e_2}}[q_{e}^{\mathrm{in}}(f_{e_0}(\bar{x}_{e_0}), x_{e_1},x_{e_2}, C)]- \nonumber \\
    & - \sum_{e\in\{e_1,e_2\}}\theta_{e} f_{e}(x_{e}) \geq \gamma, ~\forall (x_{e_1},x_{e_2})\in\mathcal{X}_{e_1}\times\mathcal{X}_{e_2}, \label{eq_thm3_1}
\end{align}
where $\bar{x}_{e_0}:=\infty$.
\end{thm}

Note that $x_{e_0}^c$ defined in Theorem~\ref{thm_2} is usually interpreted as \emph{critical density} since link $e_0$ with $x_{e_0}>x_{e_0}^c$ is considered as ``congested'' in practice. Theorem~\ref{thm_2} indicates that though link $e_0$ could be congested with extremely high traffic densities, we only need to check the critical density. Besides, Theorem~\ref{thm_3} says that we need to check $x_{e_0}=\infty$, namely to consider $\sup f_{e_0}$. 

One can implement Theorem~\ref{thm_2} by solving the following semi-infinite programming (SIP \cite{stein2012solve}):
\begin{equation}
    (P_1)~ \min_{\theta_1,\theta_2,\gamma}~\gamma~s.t.~\eqref{eq_thm2_1},
\end{equation}
and Theorem~\ref{thm_3} by solving the SIP: 
\begin{equation}
    (P_2)~ \max_{\theta_1,\theta_2,\gamma}~\gamma~s.t.~\eqref{eq_thm3_1}.
\end{equation}
We conclude the Markov chain \eqref{eq_markovchain_2} is stable if the optimal value of $P_1$ is positive and unstable if the optimal value of $P_2$ is non-positive. The programmings $P_1$ and $P_2$ belong to SIPs because they have infinite constraints over the continuous set $\mathcal{X}_{e_1}\times\mathcal{X}_{e_2}$. But noting $\mathcal{X}_{e_1}$ and $\mathcal{X}_{e_2}$ are bounded, we have efficient algorithms to solve $P_1$ and $P_2$ \cite{stein2012solve}.

Note that Theorem~\ref{thm_2} is proved based on the Lyapunov function $\tilde V:\mathbb{R}_{\geq0}^3\to\mathbb{R}_{\geq0}$:
\begin{equation}
    \tilde{V}(x_{e_0}, x_{e_1},x_{e_2}) = x_{e_0}(\frac{1}{2}x_{e_0} + \theta_{e_1}x_{e_1} + \theta_{e_2}x_{e_2}). \label{eq_lyaunov_2}
\end{equation}
and Theorem~\ref{thm_3} is based on the test function $\tilde{W}:\mathbb{R}_{\geq0}^3\to\mathbb{R}_{\geq0}$:
\begin{equation}
    \tilde{W}(x_{e_0},x_{e_1},x_{e_2}) = \xi_1 - \frac{1}{x_{e_0}+ \theta_{e_1}x_{e_1} + \theta_{e_2}x_{e_2} + \xi_2}, \label{eq_lyapunov_3}
\end{equation}
where $\xi_1$ and $\xi_2$ are sufficiently large numbers. Clearly, we can further improve the stability and instability conditions by considering more sophisticated Lyapunov/test functions, such as those replacing the linear term $\theta_{e_1}x_{e_1}+\theta_{e_2}x_{e_2}$ with non-linear terms. However, it incurs more computational costs. 

The following section presents a numerical example. The proofs of Theorems~2 and 3 are omitted since they are similar to those of Theorem~1, except for different Lyapunov/test functions.

\subsection{Numerical example}
Besides the setting in Section~\ref{sec_num_inf}, we introduce the receiving flows
\begin{equation}
    r_e(x_e) = R_e - w_ex_e
\end{equation}
with $R_{e_1}=1.2$, $R_{e_2}=0.8$, $w_{e_1}=0.5$ and $w_{e_2}=0.4$, as illusrated in Fig.~\ref{fig_receivingflow}. It follows $x_{e_1}^{\max}=2.4$ and $x_{e_2}^{\max}=2$. For the upstream link $e_0$, we suppose that its sending flow  with $v_{e_0}=1$ and $Q_{e_0}=1$, which indicates the critical density $x_{e_0}^c=1$.
\begin{figure}[htbp]
    \centering
    \includegraphics[width=0.6\linewidth]{Figures/receivingflow.pdf}
    \caption{Receiving flows of major and minor links}
    \label{fig_receivingflow}
\end{figure}

Analyzing the Markov chain \eqref{eq_markovchain_2} is more difficult since traffic dynamics involving congestion spillback is more complicated. We consider the technique of invariant sets \cite{blanchini1999set} and focus on our analysis of the Markov chain \eqref{eq_markovchain_2} on the state space 
\begin{equation}
    [\underline{d}, \infty)\times [\underline{x}_{e_1}, \bar{x}_{e_1}]\times[0, \bar{x}_{e_2}]\times [\underline{d}, 1.2]\times[0, \bar{c}], \label{eq_invariance}
\end{equation}
where the boundaries $\underline x_{e_1}$, $\bar{x}_{e_1}$ and $\bar{x}_{e_2}$ satisfy 
\begin{subequations}
\begin{align}
    \beta_{e_1}(\underline{x}_{e_1}, 0)\underline{x}_{e_1}\underline{d}+\beta_{e_2}(\underline{x}_{e_1},0)(1-\bar{c})\underline{d} =& f_{e_1}(\underline{x}_{e_1}), \\
    r_{e_1}(\bar{x}_{e_1}) =& f_{e_1}(\bar{x}_{e_1}), \\
    \beta_2(\bar{x}_{e_1}, \bar{x}_{e_2}) f_{e_0}(x_{e_0}^c)\bar{c}=& f_{e_2}(\bar{x}_{e_2}).
\end{align}
\end{subequations}
Note that restricting analysis on the state space \eqref{eq_invariance} does not lose any generality. In fact, we can prove that the set
\begin{equation}
    \tilde{\mathcal{X}}:=[\underline{d}, \infty)\times [\underline{x}_{e_1}, \bar{x}_{e_1}]\times[0, \bar{x}_{e_2}]
\end{equation}
is positively invariant and globally attracting. That is, for any initial condition $(X_{e_0}(0), X_{e_1}(0), X_{e_2}(0))\in\tilde{\mathcal{X}}$, $(X_{e_0}(t), X_{e_1}(t), X_{e_2}(t))\in\tilde{\mathcal{X}}$ for $t\geq0$ given any $(D(t),C(t))\in[\underline{d},1.2]\times[0,\bar{c}]$. Besides, for any initial condition $(X_{e_0}(0), X_{e_1}(0), X_{e_2}(0))\in\mathbb{R}_{\geq0}\times[0,x_{e_1}^{\max}]\times[0, x_{e_2}^{\max}]$, $(X_{e_0}(t), X_{e_1}(t), X_{e_2}$ enters $\tilde{\mathcal{X}}$ almost surely. 

\begin{figure}[htbp]
    \centering
    \begin{subfigure}{0.45\linewidth}
    \centering
    \includegraphics[width=\linewidth]{Figures/region_fin.pdf}  
  \caption{Stability region.}
  \label{fig_fin_stability}
\end{subfigure}
\begin{subfigure}{0.45\linewidth}
  \centering
  \includegraphics[width=\linewidth]{Figures/throughput_fin.pdf}  
  \caption{Throughput.}
  \label{fig_fin_throughput}
\end{subfigure}
    \caption{Analysis of stability and throughput given links $e_1$ and $e_2$ with finite buffer sizes.}
    \label{fig_finite}
\end{figure}

Fig.~\ref{fig_fin_stability} presents the time-average of traffic densities after $5\times10^5$ steps and discloses the stability and instability regions. We have two observations. First, there exists a gap between the stability and instability regions. This is because our stability and instability criteria are only sufficient. Second, the stability region in Fig.~\ref{fig_fin_stability} shrinks significantly, compared with that in Fig.~\ref{fig_region_inf}. It indicates that congestion spillback may not be neglected in analyzing real-world scenarios.

Fig.~\ref{fig_fin_throughput} shows upper and lower bounds of throughput. We note that the gap tends to be enlarged as the compliance rate increases. Per our discussion on Theorems~\ref{thm_2} and \ref{thm_3}, it is possible to narrow down the gap by considering more advanced Lyapunov/test functions.

%\subsection{Proof of Theorem~\ref{thm_2}}

%We still apply the Foster-Lyapunov criterion to prove the stability condition. Given the Lyapunov function \eqref{eq_lyaunov_2}, we obtain
%\begin{align*}
%    & \mathbb{E}[\tilde{V}(X(t+1))|X(t)=x] - \tilde{V}(x) \\
%    \leq& \Big(\alpha- \sum_{e\in\{e_1,e_2\}} (1-\theta_{e})\mathbb{E}_{x_{e_1},x_{e_2}}[q_{e}^{\mathrm{in}}(f_{e_0}(x_{e_0}^c), x_{e_1},x_{e_2}, C)] \\
%    & - \sum_{e\in\{e_1,e_2\}}\theta_{e} f_{e}(x_{e})\Big)x_{e_0} + n,
%\end{align*}
%where $n<\infty$. By noting that $\theta_{e_1},\theta_{e_2}\in[0, 1]$ and that 
%$$\mathbb{E}_{x_{e_1},x_{e_2}}[q_{e}^{\mathrm{in}}(f_{e_0}(x_{e_0}), x_{e_1},x_{e_2}, C)]$$ is non-decreasing in $x_{e_0}$, we conclude that
%$$\mathbb{E}[\tilde{V}(X(t+1))|X(t)=x] - \tilde{V}(x)<-mx_{e_0}+n$$ holds in the whole state space. This completes the proof.

%\subsection{Proof of Theorem~\ref{thm_3}}
%We prove the instability condition by showing the Markov chain \eqref{eq_markovchain_2} is transient given \eqref{eq_thm3_1}. For the test function \eqref{eq_lyapunov_3}, we apply the same technique in proving the necessary condition in Theorem~\ref{thm_1} and obtain
%\begin{align*}
%    & \mathbb{E}[\tilde{W}(X(t+1))|X(t)=x] - \tilde{W}(x) \\
%    \geq& \frac{1}{Z(x_{e_0})}\Big(\mathbb{E}[D(t)]
%    -\sum_{e\in\{e_1,e_2\}}\theta_ef_e(x_e)
%    \\
%    &-\sum_{e\in\{e_1,e_2\}}(1-\theta_e)\int_{\mathcal{C}} q_{e}^{\mathrm{in}}(f_{e_0}(x_{e_0}), x_{e_1}, x_{e_2}, c)) \Gamma_{x_{e_1,e_2}}(\mathrm{d}c)\Big) \\
%    %
%    =& \frac{1}{Z(x_{e_0})}\Big(\alpha - \sum_{e\in\{e_1,e_2\}}\theta_ef_e(x_e) \\
%    &- \sum_{e\in\{e_1,e_2\}} (1-\theta_{e}) \mathbb{E}_{x_{e_1},x_{e_2}}[q_{e}^{\mathrm{in}}(f_{e_0}(x_{e_0}), x_{e_1},x_{e_2}, C)]\Big),
%\end{align*}
%where $Z(x_{e_0}):\mathbb{R}_{\geq0}\to\mathbb{R}_{\geq0}$ is a certain positive function. By noting  $\theta_{e_1},\theta_{e_2}\in[0, 1]$ and 
%$$\mathbb{E}_{x_{e_1},x_{e_2}}[q_{e}^{\mathrm{in}}(f_{e_0}(x_{e_0}), x_{e_1},x_{e_2}, C)]$$ is non-decreasing in $x_{e_0}$, we conclude that
%$$\mathbb{E}[\tilde{W}(X(t+1))|X(t)=x] - \tilde{W}(x)\geq0$$ holds in the whole state space, which completes the proof.