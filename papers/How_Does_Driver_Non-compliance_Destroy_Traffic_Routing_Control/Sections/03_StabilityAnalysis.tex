\section{Stability analysis of the network without congestion propagation}
\label{sec_stability}


We state the main result as follows:
\begin{thm}
\label{thm_1}
The Markov chain \eqref{eq_markovchain_1} with the state space $\mathbb{R}_{\geq0}\times\mathbb{R}_{\geq0}\times\mathcal{D}\times\mathcal{C}$ is stable if and only if there exists a vector $\theta:=[\theta_{e_1},\theta_{e_2}]^{\mathrm{T}}\in\mathbb{R}_{\geq0}^2$ such that
\begin{subequations}
    \begin{align}
        \Big(\beta_{e_1}(\theta) + \beta_{e_2}(\theta)\mathbb{E}_\theta[1-C]\Big)\alpha -f_{e_1}(\theta_{e_1}) &< 0, \label{eq_thm1_1} \\
        %
        \beta_{e_2}(\theta)\mathbb{E}_\theta[C]\alpha  -f_{e_2}(\theta_{e_2}) &< 0. \label{eq_thm1_2}
    \end{align}
\end{subequations}
\end{thm}

Note that the stability condition is sufficient and necessary. Thus we can use it to derive exact values of throughput. In the following sections, we first present a numerical example and then prove Theorem~\ref{thm_1}.

\subsection{Numerical example}
\label{sec_num_inf}

First, we set $\delta = 0.1$ and  $l_{e_1}=l_{e_2}=1$. We consider the sending flows 
\begin{equation}
    f_e(x_e) = \min\{v_ex_e, Q_e\}, ~e\in\{e_1,e_2\} \label{eq_sending}
\end{equation}
with $v_{e_1}=1$, $v_{e_2}=0.8$, $Q_{e_1}=0.6$, $Q_{e_2}=0.4$, as illustrated in Fig.~\ref{fig_sendingflow}.

\begin{figure}[htbp]
    \centering
    \includegraphics[width=0.6\linewidth]{Figures/sendingflow.pdf}
    \caption{Sending flows of major and minor links.}
    \label{fig_sendingflow}
\end{figure}

For the purpose of routing, we adopt the classical logit routing as follows:
\begin{equation}
     \beta_e(x) = \frac{e^{-\nu_e x_e}}{e^{-\nu_{e_1}x_{e_1}}+e^{-\nu_{e_2}x_{e_2}}}, ~e\in\{e_1,e_2\},
\end{equation}
where $\nu_{e_1}=1$ and $\nu_{e_2}=2$ are routing parameters. 

We assume the demands $D(t)\in[\underline{d}, 1.2]$, $t\geq0$, are independent and identically distributed (i.i.d.) uniform random variables. It follows $\mathbb{E}[D(t)]=\underline{d}/2+0.6$.  We also assume the routing compliance rates $C(t)\in[0,\bar{c}]$, $t\geq0$, are i.i.d. uniform random variables, along with $\mathbb{E}[C(t)]=\bar{c}/2$. It indicates that in our numerical example the compliance rates are independent of traffic states. It should be noted that this independence is not necessary for our approach. Here we assume it just for simplification. However, we still have non-trivial observations in this case.

We first analyze the stability and instability of scenarios with different $\underline d$ and compliance rate $\bar c$. Fig.~\ref{fig_region_inf} shows the time-average traffic densities after $5\times10^5$ steps and reveals the stability and instability regions. We observe a non-linear boundary: given moderate traffic demands, improvements of compliance rates can stabilize the network; but given a high demand close to the network capacity, we hardly see the effect of improving compliance rate.

Then we compute the throughput, the maximum expected demand under which the network can be stabilized. It is interesting to find that we can achieve a relatively high throughput (around 0.987) when $\mathbb{E}[C(t)]=0.395$. Further improvement is marginal when $\mathbb{E}[C(t)]$ exceed 0.395.  
\begin{figure}[htbp]
    \centering
    \begin{subfigure}{0.45\linewidth}
    \centering
    \includegraphics[width=\linewidth]{Figures/region_inf.pdf}  
  \caption{Stability region.}
  \label{fig_region_inf}
\end{subfigure}
\begin{subfigure}{0.45\linewidth}
  \centering
  \includegraphics[width=\linewidth]{Figures/throughput_inf.pdf}  
  \caption{Throughput.}
  \label{fig_throu_inf}
\end{subfigure}
    \caption{Analysis of stability and throughput given links $e_1$ and $e_2$ with infinite buffer sizes.}
    \label{fig_infinite_analysis}
\end{figure}


\subsection{Proof of Theorem~\ref{thm_1}}
We first prove the sufficiency by the Foster-Lyapunov criterion \cite{meyn2012markov}:
\begin{namedthm*}{Foster-Lyapunov criterion}
Consider a $\varphi$-irreducible Markov chain $\{Y(t); t\ge0\}$ with a state space $\mathcal Y$, an infinitesimal generator $\mathscr L$, and a Lyapunov function $V:\mathcal Y\to\mathbb R_{\ge0}$. If there exist constants $m>0$, $n<\infty$, a function $g:\mathcal Y\to\mathbb R_{\ge0}$ and a compact set $\mathcal{E}$ such that for any $y\in\mathcal{Y}$
\begin{align*}
    \mathbb{E}[V(Y(t+1))|Y(t)=y] - V(y) \leq -mg(x) + n\mathbf{1}_{\mathcal{E}}(y),
\end{align*}
where $\mathbf{1}_{\mathcal{E}}(y)$ is an indicator function, then, for each initial condition $y(0)\in\mathcal Y$,
$$
\limsup_{t\to\infty}\frac1t\sum_{\tau=0}^t\mathrm E[g(Y(\tau))] \le m/n.
$$
\end{namedthm*}

To proceed, we consider the following Lyapunov function
\begin{equation}
    V(x) = \begin{cases}
    0 & x\in\mathcal{X}^1, \\
    \frac{1}{2}(x_{e_1}-\theta_{e_1})_+^2 & x\in\mathcal{X}^2, \\
    \frac{1}{2}(x_{e_2}-\theta_{e_2})_+^2 & x\in\mathcal{X}^3, \\
    \frac{1}{2}((x_{e_1}-\theta_{e_1})_+ + (x_{e_2}-\theta_{e_2})_+)^2 & x\in\mathcal{X}^4,
    \end{cases}
\end{equation}
%\begin{equation}
%    V(x) = \frac{1}{2}\Big((x_{e_1}-%\theta_{e_1})_+ + (x_{e_2}-%\theta_{e_2})_+\Big)^2,
%\end{equation}
where $(\cdot)_+:=\max\{\cdot, 0\}$, $\mathcal{X}_{e_1}:=[0,\theta_{e_1}]\times[0,\theta_{e_2}]$, $\mathcal{X}_{e_2}:=(\theta_{e_1}, \infty)\times[0,\theta_{e_2}]$, $\mathcal{X}_{e_3}:=[0,\theta_{e_1}]\times(\theta_{e_2}, \infty)$ and $\mathcal{X}_{e_4}:=(\theta_{e_1}, \infty)\times(\theta_{e_2}, \infty)$.

The rest is devoted to show that there exist constants $m'>0$ and $n'<\infty$ such that for every $x\in\mathbb{R}_{\geq0}^2$
\begin{align}
 &\mathbb{E}[V(X(t+1))|X(t)=x] - V(x) \nonumber \\
 \leq & -m' \Big((x_{e_1}-\theta_{e_1})_+ + (x_{e_2}-\theta_{e_2})_+\Big) + n'.  \label{eq_proof_thm1_1}  
\end{align}
If \eqref{eq_proof_thm1_1} holds, we must have $0<m<m'$, $n<\infty$, and a compact set $\mathcal{E}=[0, M]\times[0,M]$ such that
\begin{align}
 &\mathbb{E}[V(X(t+1))|X(t)=x] - V(x) \nonumber \\
 \leq & -m \Big((x_{e_1}-\theta_{e_1})_+ + (x_{e_2}-\theta_{e_2})_+\Big) + n\mathbf{1}_{\mathcal{E}}(x),  \label{eq_proof_thm1_2}  
\end{align}
which indicates that $X_{e_1}(t)$ and $X_{e_2}(t)$ and thus concludes the stability.

To show \eqref{eq_proof_thm1_1}, we need to discuss whether $X_{e_1}(t)$ is larger than $\theta_{e_1}$ and whether $X_{e_2}(t)$ is larger than $\theta_{e_2}$, up to four cases. Here we present the proofs for the typical cases and the remaining can be proved in a similar way.

When $X_{e_1}(t)\leq\theta_{e_1}$ and $X_{e_2}(t)\leq\theta_{e_2}$, the proof is trivial by noting that $X_{e_1}(t+1)$ and $X_{e_2}(t+1)$ must be bounded a sufficiently large number.


Now we assume $X_{e_1}(t)>\theta_{e_1}$ and $X_{e_2}(t)\leq\theta_{e_2}$. It follows
\begin{align*}
    &\mathbb{E}[V(X(t+1))|X(t)=x]-V(x) \\
    \leq& \frac{1}{2}\Big(\int (x_{e_1}+G_{e_1}-\theta_{e_1})_+^2 - (x_{e_1}-\theta_{e_1})^2\Big)  \\
    %
    \leq& \frac{1}{2}\Big(\int (x_{e_1}+G_{e_1}-\theta_{e_1})^2 - (x_{e_1}-\theta_{e_1})^2\Big) \\
    %
    \leq& \Big(\Big(\beta_{e_1}(x) + \beta_{e_2}(x)\mathbb{E}_x[1-C]\Big)\alpha -f_{e_1}(x_{e_1}) \Big)x_{e_1} + n,
\end{align*}
where $n$ is a sufficiently large number and
\begin{equation}
    G_{e_1}:=q_{e_1}^{\mathrm{in}}(D(t), X_{e_1}(t), X_{e_2}(t), C(t)) - f_{e_1}(X_{e_1}(t)). \label{eq_G_e1}
\end{equation}
Note that we omit $\delta/l_{e_1}$. By Assumptions~1-3,
$$\Big(\beta_{e_1}(x_{e_1}, x_{e_2}) + \beta_{e_2}(x_{e_1}, x_{e_2})\mathbb{E}_x[1-C]\Big)\alpha -f_{e_1}(x_{e_1})$$
is non-increasing in $x_{e_1}$ and non-decreasing in $x_{e_2}$. Thus \eqref{eq_thm1_1} indicates that there exists $m_1'>0$ such that for any $x_{e_1}>\theta_{e_1}$ and $x_{e_2}\leq\theta_{e_2}$,
\begin{align*}
    &\mathbb{E}[V(X(t+1))|X(t)=x]-V(x) \\
    \leq& -m_1'x_{e_1}+n \\
    =& -m_1'\Big((x_{e_1}-\theta_{e_1})_++(x_{e_2}-\theta_{e_2})_+\Big) + (n-m_1'\theta_{e_1}).
\end{align*}
Since $n$ can be sufficiently large, we must have 
$$n-m_1'\theta_{e_1}>0.$$

Finally, we consider $X_{e_1}(t)>\theta_{e_1}$ and $X_{e_2}(t)>\theta_{e_2}$. It turns out that we obtain
\begin{align}
    &\mathbb{E}[V(X(t+1))|X(t)=x]-V(x) \nonumber \\
    \leq & \Big(\alpha - f_{e_1}(x_{e_1}) - f_{e_2}(x_{e_2})\Big)(x_{e_1}+x_{e_2}) + n. \nonumber
\end{align}
Note the $\alpha - f_{e_1}(x_{e_1})) - f_{e_2}(x_{e_2}$ is non-increasing in both $x_{e_1}$ and $x_{e_2}$. Combing \eqref{eq_thm1_1}-\eqref{eq_thm1_2} indicates that there exists $m_2'>0$ such that for any $x_{e_1}>\theta_{e_1}$ and $x_{e_2}>\theta_{e_2}$,
\begin{align*}
    &\mathbb{E}[V(X(t+1))|X(t)=x]-V(x) \\
    \leq& -m_2'(x_{e_1}+x_{e_2}) + n \\
    =& -m_2'\Big((x_{e_1}-\theta_{e_1})_++(x_{e_2}-\theta_{e_2})_+\Big) \\
    & + (n-m_2'\theta_{e_1}-m_2'\theta_{e_2}).
\end{align*}
We can similarly conclude that $n-m_2'\theta_{e_1}-m_2'\theta_{e_2}>0$. Thus we finish proving the sufficient condition.

The following proves the necessary condition. That is, we show that if there does not exist a vector $\theta$ satisfying \eqref{eq_thm1_1}-\eqref{eq_thm1_2}, the system is unstable. We prove the instability by showing the Markov chain \eqref{eq_markovchain_1} is transient, which indicates that some state tends towards infinity in the long term, and thus transience can be as instability \cite{meyn1993survey}. We consider the following transience criterion \cite{meyn2012markov}:
\begin{namedthm*}{Transience criterion}
Consider a $\varphi$-irreducible Markov chain $\{Y(t): t\ge0\}$ with a state space $\mathcal Y$. Then $\{Y(t): t\geq0\}$ is transient if there exists a bounded function $V:\mathcal{Y}\to\mathbb{R}_{\geq0}$ and a sublevel set of $V$, denoted by $S$, such that 
\begin{enumerate}
    \item[(i)] $\varphi(S)>0$ and $ \varphi(\mathcal{Y}\setminus S)>0$;
    \item[(ii)] $\mathbb{E}[V(Y(t+1))|Y(t)=y]-V(y)\geq0,~\forall y\in \mathcal{Y}\setminus S$.
\end{enumerate}
\end{namedthm*}

Note that our Markov chain \eqref{eq_markovchain_1} is $\varphi$-irreducible, stated by Lemma~\ref{lmm_1}. To proceed, we first assume that for any $\theta\in\mathbb{R}_{\geq0}^2$
\begin{equation}
    \Big(\beta_{e_1}(\theta) + \beta_{e_2}(\theta)\mathbb{E}_\theta[1-C]\Big)\alpha -f_{e_1}(\theta_{e_1}) \geq 0. \label{eq_pf_thm1_transience}
\end{equation}
We consider a bounded test function $W:\mathbb{R}_{\geq0}\to\mathbb{R}_{\geq0}$:
\begin{equation}
    W(x_{e_1}) = \xi_1 - \frac{1}{x_{e_1} + \xi_2}
\end{equation}
where $\xi_1$ and $\xi_2$ are sufficiently large numbers.
Then, we obtain
\begin{align*}
     &\mathbb{E}[W(X_{e_1}(t+1))|X_{e_1}(t)=x_{e_1}] - f_{e_1}(x_{e_1}) \\
    =& \frac{1}{x_{e_1}+\xi_2} - \int \frac{1}{x_{e_1}+G_{e_1}+\xi_2} \\
    =& \int\frac{G_{e_1}}{(x_{e_1}+\xi_2)(x_{e_1}+G_{e_1}+\xi_2)},
    %\frac{
    %\Big(\beta_{e_1}(x) + \beta_2(x)\mathbb{E}_x[1-C]\Big)\alpha -f_{e_1}(x_{e_1})
    %}{(x_{e_1}+\xi)(\mathbb{E}[X_{e_1}(t+1)|X_{e_1}(t)=x_{e_1}]+\xi)} \geq  0.
\end{align*}
where $G_{e_1}$ is given by \eqref{eq_G_e1}. Note that $G_{e_1}$ is bounded because flow functions are bounded. With loss of generality, we assume 
\begin{equation*}
    G_{e_1}^{\mathrm{min}} \leq G_{e_1} \leq G_{e_1}^{\mathrm{max}}.
\end{equation*}
If follows
\begin{align*}
     &\mathbb{E}[W(X_{e_1}(t+1))|X_{e_1}(t)=x_{e_1}] - f_{e_1}(x_{e_1}) \\
    =& \int_{G_{e_1}\leq0}\frac{G_{e_1}}{(x_{e_1}+\xi_2)(x_{e_1}+G_{e_1}+\xi_2)} \\
     & + \int_{G_{e_1}>0}\frac{G_{e_1}}{(x_{e_1}+\xi_2)(x_{e_1}+G_{e_1}+\xi_2)} \\
     %
    \geq & \int_{G_{e_1}\leq0}\frac{G_{e_1}}{(x_{e_1}+\xi_2)(x_{e_1}+G_{e_1}^{\mathrm{min}}+\xi_2)} \\
     & + \int_{G_{e_1}>0}\frac{G_{e_1}}{(x_{e_1}+\xi_2)(x_{e_1}+G_{e_1}^{\mathrm{max}}+\xi_2)} \\
     =& \frac{1}{(x_{e_1}+\xi_2)(x_{e_1}+G_{e_1}^{\mathrm{min}}+\xi_2)}\int_{G_{e_1}\leq0} G_{e_1} \\
     &+ \frac{1}{(x_{e_1}+\xi_2)(x_{e_1}+G_{e_1}^{\mathrm{max}}+\xi_2)}\int_{G_{e_1}>0} G_{e_1} \\
     =& \frac{1}{(x_{e_1}+\xi_2)(x_{e_1}+G_{e_1}^{\mathrm{min}}+\xi_2)(x_{e_1}+G_{e_1}^{\mathrm{max}}+\xi_2)}\Big(\\
     &(x_{e_1}+\xi_2)\int G_{e_1}+ G_{e_1}^{\max}\int_{G_{e_1}\leq0}G_{e_1} \\
     & + G_{e_1}^{\min}\int_{G_{e_1}>0}G_{e_1} \Big).
\end{align*}
Obviously, given 
$$\int G_{e_1} = \Big(\beta_{e_1}(\theta) + \beta_{e_2}(\theta)\mathbb{E}_\theta[1-C]\Big)\alpha -f_{e_1}(\theta_{e_1}) \geq 0,$$
we must have 
$$\mathbb{E}[W(X_{e_1}(t+1))|X_{e_1}(t)=x_{e_1}] - f_{e_1}(x_{e_1})\geq0$$
by letting $\xi_2$ be sufficiently large.

Note that the above inequality holds over $\mathbb{R}_{\geq0}^2$. It indicates that the conditions (i)-(ii) above are satisfied. Thus we conclude the Markov chain \eqref{eq_markovchain_1} is transient given \eqref{eq_pf_thm1_transience}.

Then we assume that for any $\theta\in\mathbb{R}_{\geq0}^2$
\begin{equation}
    \beta_{e_2}(\theta)\mathbb{E}_\theta[C]\alpha  -f_{e_2}(\theta_{e_2}) \geq 0.
\end{equation}
We can prove in a similar way that the Markov chain \eqref{eq_markovchain_1} is transient in this case. Finally, we conclude that if there does not exist a vector $\theta$ satisfying \eqref{eq_thm1_1}-\eqref{eq_thm1_2}, the system is unstable.

