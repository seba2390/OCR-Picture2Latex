\section*{Appendix}
\label{sec:appendix}

\subsection*{Proof of Proposition \ref{prop:meancovcondsignal}}
\label{subsec:proofmeancovcond}

\begin{proof}
Firstly, the probability response curve can be obtained by 
\begin{equation}
    \begin{split}
       \text{Pr}(\mathbf{Y}_{\boldsymbol{\tau}}=\mathbf{1}\mid\mathbf{Z}_{\boldsymbol{\tau}},\sigma_{\epsilon}^2)&=\int p(\mathbf{Y}_{\boldsymbol{\tau}}=\mathbf{1}\mid\boldsymbol{\mathcal{Z}}_{\boldsymbol{\tau}},\mathbf{Z}_{\boldsymbol{\tau}},\sigma_{\epsilon}^2)p(\boldsymbol{\mathcal{Z}}_{\boldsymbol{\tau}}\mid\mathbf{Z}_{\boldsymbol{\tau}},\sigma_{\epsilon}^2)d\boldsymbol{\mathcal{Z}}_{\boldsymbol{\tau}}\\
       &=\int \varphi(\boldsymbol{\mathcal{Z}}_{\boldsymbol{\tau}})N(\boldsymbol{\mathcal{Z}}_{\boldsymbol{\tau}}\mid\mathbf{Z}_{\boldsymbol{\tau}},\sigma_{\epsilon}^2\mathbf{I})d\boldsymbol{\mathcal{Z}}_{\boldsymbol{\tau}}=\text{E}(\varphi(\boldsymbol{\mathcal{Z}}_{\boldsymbol{\tau}})\mid \mathbf{Z}_{\boldsymbol{\tau}},\sigma_{\epsilon}^2).
    \end{split}
    \label{eq:probcurvecondsignal}
\end{equation}

 
Then, to find the diagonal and off-diagonal elements for the covariance matrix of $\mathbf{Y}_{\boldsymbol{\tau}}$, we use the law of total variance/covariance. For the diagonal elements,
\begin{equation}
    \begin{split}
        \text{Var}(Y_{\tau}\mid\mathbf{Z}_{\boldsymbol{\tau}},\sigma_{\epsilon}^2)&=\text{Var}[\text{E}(Y_{\tau}\mid \mathcal{Z}_{\boldsymbol{\tau}})\mid\mathbf{Z}_{\boldsymbol{\tau}},\sigma_{\epsilon}^2]+\text{E}[\text{Var}(Y_{\tau}\mid \mathcal{Z}_{\boldsymbol{\tau}})\mid\mathbf{Z}_{\boldsymbol{\tau}},\sigma_{\epsilon}^2]\\
        &=\text{Var}[\varphi(\mathcal{Z}_{\tau})\mid\mathbf{Z}_{\boldsymbol{\tau}},\sigma_{\epsilon}^2]+\text{E}[\varphi(\mathcal{Z}_{\tau})(1-\varphi(\mathcal{Z}_{\tau}))\mid \mathbf{Z}_{\boldsymbol{\tau}},\sigma_{\epsilon}^2]\\
        &=\text{E}[\varphi(\mathcal{Z}_{\tau})\mid \mathbf{Z}_{\boldsymbol{\tau}},\sigma_{\epsilon}^2]-\text{E}^2[\varphi(\mathcal{Z}_{\tau})\mid \mathbf{Z}_{\boldsymbol{\tau}},\sigma_{\epsilon}^2].
    \end{split}
    \label{eq:varcondsignal}
\end{equation}
As for the off-diagonal entries, similarly,
\begin{equation}
    \begin{split}
        \text{Cov}(Y_{\tau},Y_{\tau^{\prime}}\mid\mathbf{Z}_{\boldsymbol{\tau}},\sigma_{\epsilon}^2)&=\text{Cov}[\text{E}(Y_{\tau}\mid \boldsymbol{\mathcal{Z}}_{\boldsymbol{\tau}}),\text{E}(Y_{\tau^{\prime}}\mid \boldsymbol{\mathcal{Z}}_{\boldsymbol{\tau}})\mid\mathbf{Z}_{\boldsymbol{\tau}},\sigma_{\epsilon}^2]+\text{E}[\text{Cov}(Y_{\tau},Y_{\tau^{\prime}}\mid \boldsymbol{\mathcal{Z}}_{\boldsymbol{\tau}})\mid\mathbf{Z}_{\boldsymbol{\tau}},\sigma_{\epsilon}^2]\\
        &=\text{Cov}[\varphi(\mathcal{Z}_{\tau}),\varphi(\mathcal{Z}_{\tau^{\prime}})\mid\mathbf{Z}_{\boldsymbol{\tau}},\sigma_{\epsilon}^2]
    \end{split}
    \label{eq:covcondsignal}
\end{equation}
\end{proof}


\subsection*{Proof of Proposition \ref{prop:deltaapproxlogitnormal}}
\label{subsec:proofdeltaapprox}

\begin{proof}
Write $\mathbf{Z}=\boldsymbol{\mu}+\boldsymbol{\zeta}$, where $\boldsymbol{\zeta}\sim N(0,\boldsymbol{\Sigma})$.
By Taylor expansion around the mean, 
\begin{equation}
    \varphi(Z_i)\approx \varphi(\mu_i)+\zeta_i\varphi^{\prime}(\mu_i)+\frac{\zeta^2}{2}\varphi^{\prime\prime}(\mu_i).
\label{eq:taylorlogitnomralmean}
\end{equation}
Then taking expectation yields $\text{E}(\varphi(Z_i))\approx \varphi(\mu_i)+\frac{\sigma^2}{2}\varphi^{\prime\prime}(\mu_i)$, $i=1,2$.

The expectation of $\varphi^2(Z_i)$ can be derived using the same technique,
\begin{equation}
    \varphi^2(Z_i)\approx \varphi^2(\mu_i)+2\zeta_i\varphi(\mu_i)\varphi^{\prime}(\mu_i)+\zeta_i^2[(\varphi^{\prime}(\mu_i))^2+\varphi(\mu_i)\varphi^{\prime\prime}(\mu_i)].
    \label{eq:taylorlogitnormalvar}
\end{equation}
Taking expectation with respect to $\zeta_i$ again, we arrive at the result. 

As for $E(\varphi(Z_1)\varphi(Z_2))$, consider the function $f(\mathbf{Z})=\varphi(Z_1)\varphi(Z_2)$, using the bivariate version of Taylor expansion,
\begin{equation}
f(\mathbf{Z})\approx f(\boldsymbol{\mu})+\bigtriangledown f(\boldsymbol{\mu})^{\top}\boldsymbol{\zeta}+\frac{1}{2}\boldsymbol{\zeta}^{\top}\bigtriangledown^2f(\boldsymbol{\mu})\boldsymbol{\zeta}.
\label{eq:taylorlogitnormalcov}    
\end{equation}
Similarly, taking expectation with respect to $\boldsymbol{\zeta}$, we can obtain the result. 

\end{proof}



\subsection*{Proof of Proposition \ref{prop:marginalsignal}}
\label{subsec:proofmarginalsignal}

\begin{proof}
The result is proved by considering the corresponding f.d.d.s. on any finite grids $\boldsymbol{\tau}$. Let the bold letter denote the corresponding process evaluated at $\boldsymbol{\tau}$. From the model assumption, 
\begin{equation}
    p(\mathbf{Z})=\int\int p(\mathbf{Z}\mid \boldsymbol{\mu},\boldsymbol{\Sigma})p(\boldsymbol{\mu}\mid \boldsymbol{\Sigma})p(\boldsymbol{\Sigma})d\boldsymbol{\mu}d\boldsymbol{\Sigma}
    \label{eq:marginalsignal}
\end{equation}
We first notice that marginalizing over its mean $\boldsymbol{\mu}$,  $\mathbf{Z}\sim N(\boldsymbol{\mu}_0,(\nu-2)\boldsymbol{\Sigma})$. Based on that, 
\begin{equation}
    \begin{split}
        p(\mathbf{Z})&=\int p(\mathbf{Z}\mid \boldsymbol{\Sigma})p(\boldsymbol{\Sigma})d\boldsymbol{\Sigma}\\
        &\propto \int \frac{\exp\{-\frac{1}{2}\text{Tr}[(\boldsymbol{\Psi}_{\boldsymbol{\phi}}+\frac{(\mathbf{Z}-\boldsymbol{\mu}_0)(\mathbf{Z}-\boldsymbol{\mu}_0)^{\top}}{\nu-2})\boldsymbol{\Sigma}^{-1}]\}}{|\boldsymbol{\Sigma}|^{(\nu+|\boldsymbol{\tau}|+1)/2}}d\boldsymbol{\Sigma}\\
        &\propto [1+\frac{(\mathbf{Z}-\boldsymbol{\mu}_0)^{\top}\boldsymbol{\Psi}_{\boldsymbol{\phi}}^{-1}(\mathbf{Z}-\boldsymbol{\mu}_0)}{\nu-2}]^{-(\nu+|\boldsymbol{\tau}|)/2},
   \end{split}
   \label{eq:marginalizescalematrix}
\end{equation}
which can be recognized as the kernel of a MVT distribution. Therefore, the result holds. 

\end{proof}

