\documentclass[12pt]{article}
\usepackage[utf8]{inputenc}
\usepackage[T1]{fontenc}

%\addtolength{\oddsidemargin}{-.5in}%
%\addtolength{\evensidemargin}{-1in}%
%\addtolength{\textwidth}{1in}%
%\addtolength{\textheight}{1.7in}%
\addtolength{\oddsidemargin}{-.5in}%
\addtolength{\evensidemargin}{-.5in}%
\addtolength{\textwidth}{1in}%
\addtolength{\textheight}{1.3in}%
\addtolength{\topmargin}{-.8in}%
%\addtolength{\topmargin}{-1in}%
%\addtolength{\topmargin}{-0.9in}%
%\addtolength{\bottommargin}{-0.7in}%
%\addtolength{\leftmargin}{-0.8in}%
%\addtolength{\rightmargin}{-1in}%

%\usepackage[letterpaper, left=0.8in, top=0.9in, right=1.0in, bottom=0.7in,nohead,includefoot, verbose, ignoremp]{geometry}
%\usepackage[top=1in, bottom=1in, left=1.25in, right=1in]{geometry}
%\usepackage[onehalfspacing]{setspace}
%\usepackage{setspace}
%\onehalfspacing

\usepackage{graphicx}
\graphicspath{ {./plot/} }

\usepackage{authblk}

\usepackage{caption}
% \usepackage[font=scriptsize]{caption}

\usepackage{float}
\usepackage{subcaption}
\usepackage{amsmath, amssymb, amsthm}
\usepackage{mathtools}
\usepackage{bbm}
\usepackage{bm}
\usepackage{mathrsfs}
\usepackage{tikz}
\usepackage{pgfplots}
\usepgfplotslibrary{dateplot}
\usepackage{yhmath}
\usepackage{multirow} % multiple row in a table

\usepackage{hyperref}
\hypersetup{hidelinks}
\hypersetup{
%colorlinks=false,
colorlinks=true,
linkcolor=blue,
citecolor=blue
}
% reference different files
\usepackage{xr}

% multi-columns
% \usepackage{multicol}

\usepackage{enumerate, enumitem}
\usepackage{fancyhdr, graphicx, proof, comment, multicol}
\usepackage[none]{hyphenat} % This command prevents hyphenation of words
%\binoppenalty=\maxdimen % This command and the next prevent in-line equation breaks
%\relpenalty=\maxdimen


\usepackage{microtype} % Modifies spacing between letters and words
% \usepackage{mathpazo} % Modifies font. Optional package.
\usepackage{mdframed} % Required for boxed problems.
%\usepackage{parskip} % Left justifies new paragraphs.

% citation 
\usepackage[round]{natbib}
\setcitestyle{authoryear,open={(},close={)}}
%\bibliographystyle{jasa}


\DeclarePairedDelimiter\ceil{\lceil}{\rceil}
\DeclarePairedDelimiter\floor{\lfloor}{\rfloor}


%\linespread{1.8} 
%\maxdeadcycles=200

% custom definitions ...%
\newtheorem{thm}{Theorem}
\newtheorem{lemma}{Lemma}
\newtheorem{proposition}{Proposition}
\newtheorem{corollary}{Corollary}
\newtheorem{remark}{Remark}
\newtheorem{definition}{Definition}


\begin{document}
\def\spacingset#1{\renewcommand{\baselinestretch}%
{#1}\small\normalsize} \spacingset{1}

%\bibliographystyle{natbib}

%\def\spacingset#1{\renewcommand{\baselinestretch}%
%{#1}\small\normalsize} \spacingset{1}

%\bibliographystyle{natbib}

%\def\spacingset#1{\renewcommand{\baselinestretch}%
%{#1}\small\normalsize} \spacingset{1}


%%%%%%%%%%%%%%%%%%%%%%%%%%%%%%%%%%%%%%%%%%%%%%%%%%%%%%%%%%%%%%%%%%%%%%%%%%%%%%


\title{\bf Flexible Bayesian Modeling for Longitudinal Binary and Ordinal Responses}
%
\author{Jizhou Kang and Athanasios Kottas\thanks{Jizhou Kang (jkang37@ucsc.edu)
is Ph.D. student, and Athanasios Kottas (thanos@soe.ucsc.edu) is Professor, Department 
of Statistics, University of California, Santa Cruz.} \\
  Department of Statistics, University of California, Santa Cruz\\
}
%
%  \author{}
\maketitle  



\bigskip
\begin{abstract}

%In longitudinal studies with binary responses, the primary focus is on the temporal evolution of the probability of positive response. Traditional solutions attempt the problem under the generalized mixed effects modeling framework, which is restrictive due to the assumptions on the expectation and covariance structure of the binary responses. We tackle the problem from a functional data analysis perspective, treating the observations for each subject as realizations from subject-specific stochastic processes at the measured times. We assume the stochastic processes have binomial marginal distributions. Leveraging the logits representation, we can model the discrete space processes through continuous space processes, whose mean and covariance kernel are nonparametrically and simultaneously through a Gaussian process prior and an Inverse-Wishart process prior, respectively. These prior assumptions result in flexible inference for the evolution and correlation of binary responses, while allowing for borrowing of strength across subjects. We illustrate the methodology with synthetic data examples and an analysis of college students’ mental health status data.

Longitudinal studies with binary or ordinal responses are widely encountered in 
various disciplines, where the primary focus is on the temporal evolution of the 
probability of each response category. Traditional approaches build from  
the generalized mixed effects modeling framework. Even amplified with nonparametric 
priors placed on the fixed or random effects, such models are restrictive due to the 
implied assumptions on the marginal expectation and covariance structure of the responses. 
We tackle the problem from a functional data analysis perspective, treating the 
observations for each subject as realizations from subject-specific stochastic processes 
at the measured times. %The primary designs of our proposed model focus on the 
We develop the methodology focusing initially on binary responses, for which we assume 
the stochastic processes have Binomial marginal distributions. Leveraging the logits 
representation, we model the discrete space processes through sequences of continuous 
space processes. We utilize a hierarchical framework to model the mean and covariance 
kernel of the continuous space processes nonparametrically and simultaneously through 
a Gaussian process prior and an Inverse-Wishart process prior, respectively. The prior 
structure results in flexible inference for the evolution and correlation of binary 
responses, while allowing for borrowing of strength across all subjects. The modeling 
approach can be naturally extended to ordinal responses. Here, the continuation-ratio 
logits factorization of the multinomial distribution is key for efficient modeling 
and inference, including a practical way of dealing with unbalanced longitudinal data. 
The methodology is illustrated with synthetic data examples and an analysis of 
college students' mental health status data.

%Synthetic and real data examples demonstrate that, in comparison with alternative methods, the proposed Bayesian approach achieves better accuracy in both the probability response curves and covariance structure estimates. 


\end{abstract}

\noindent%
{\it Keywords:} Bayesian hierarchical modeling; %Bayesian nonparametrics; 
Continuation-ratio logits; Functional data analysis; Markov chain Monte Carlo; 
Student-t process.
%Blocked Gibbs sampler. 
%\vfill

\newpage
\spacingset{1.5} %% use 1.7 for JASA, can change to 1.5 for JRSSB



\section{Introduction}
\label{sec:intro}

% background
Recent years have witnessed a rapid growth of longitudinal studies with binary and 
ordinal responses in several disciplines, including econometrics, and the health 
and social sciences. In such studies, of primary importance are the probability 
response curves, i.e., the probabilities of the response categories that evolve
dynamically over time.
%known as the probability response curves. Because of its potential to address 
%relevant scientific and practical questions, inference for these curves is of 
%primary importance. 
This article aims at developing a hierarchical framework, 
customized to longitudinal settings, that allows flexible inference for the probability 
response curves. In addition, the defining characteristic of longitudinal data is that 
repeated measurements on the same subject induce dependence. Hence, a further objective 
is to %accurately identify 
flexibly model lead-lag correlations among repeated measurements. 


% research objective, also mention that we focus on the case without covariates/with categorical covariates

%The defining characteristic of a longitudinal study is that subjects are measured repeatedly through time \citep{DiggleBook2002}. With longitudinal data, one can differentiate the changes over time within subjects (aging effects) from the changes between subjects at baseline (cohort effects). The objectives of analysis include assessing the fluctuation within a subject over time and identifying lead-lag correlations among repeated measurements. In certain applications where features associated with the measurements are also available, regressing longitudinal ordinal outcomes on these fixed covariates is also of interest.  In this article, we focus particularly on developing flexible modeling approach for estimate the probability response curves, which can be viewed as the regression relationship with time as the only covariates. Nonetheless, we discuss including other categorical covariates as a natural extension of the proposed method. 

%Nonetheless, the benefit comes at a price of modeling challenges. The major challenges arise from the two objectives of analyzing longitudinal data: capturing the fluctuation and dependency among the repeated measurements. Consequently, more sophisticated statistical methods are needed. Note that usually repeated measurements come along with features of the subjects, forming up regression problems.  

% parametric GMM modeling framework
%The development of statistical methods for longitudinal binary data stems from models for longitudinal continuous responses. Since the most natural way to view binary data is to postulate the existence of a latent continuous variable associated with each response, such models can be extended to deal with binary data through appropriate link function. A plethora of literature have covered the topic of longitudinal continuous data analysis \citep{HedekerGibbons2006,Verbeke2009,Fitzmaurice2011}, with the majority of them under the mixed effects modeling framework. Suppose we have repeated measurements on $n$ subjects, denoted as $\mathbf{Z}_i=(Z_{i1},\cdots,Z_{iT_i})^T$, $i=1,\cdots,n$. Typically the mixed effects model assumes $\mathbf{Z}_i\sim N(\mathbf{u},\mathbf{V}_i)$, where $\mathbf{u}$ denote the common fixed effect across subjects, and $\mathbf{V}_i$ is the random subject effect that depicts the influence of subjects on their repeated observations \citep{Diggle1988}. The crucial question for the mixed effects model is the choice of fixed and random effects' structure, which has various options \citep{Lindstrom1990,Shi1996,Zhang2001}. These models have been extended to deal with dichotomous data by adopting a link function (logit, probit, etc.), resulting in the generalized mixed models (GMM). For a comprehensive review about GMM, we refer to \citet{Hedeker2008}.
%These models have been extended to deal with dichotomous and ordinal data by adopting a link function (logit, probit, etc.), and a specific representation of response probabilities (proportional odds, adjacent-categories, continuation-ratio, etc.), resulting in the generalized mixed models (GMM). For a comprehensive review about GMM, we refer to \citet{Hedeker2008}. 

% nonparametric modeling approaches
%Seeking for more flexibility, one option is to assume the fixed or random effects of the latent variable have more sophisticated, nonparametric structure. Popular choices include modeling the fixed effect through smoothing techniques \citep{HeZhu2002,WuZhang2006}, applying a matrix stick-breaking process for a residual covariance structure \citep{Das2014}, capturing subject random effect by stochastic process \cite{ZhangLin1998}, putting a Dirichlet process (DP) prior \citep{LiLinMuller2010} or mixture of Pólya trees \citep{Ghosh2010} for random effect distributions, and using a combination of stochastic processes and DP mixture model for the subject random effect \citep{Quintana2016}. Admittedly, such models can be embedded in a hierarchical framework to deal with discrete responses. For instance, under the GMM framework which incorporate a DP mixture of normal prior as probability model for the latent variables,  \citet{Jara2007} and \citet{TangDuan2012} consider binary responses and \citet{Kunihama2019} handles mixed-scale data consisted by continuous and binary responses. Nonetheless, these approaches may not be applicable beyond dichotomous responses because of the computational burden. Concerning multivariate longitudinal ordinal responses, \citet{Tran2021} proposes to use a latent factor model within the GMM framework where the random effect is modeled by Ornstein-Uhlenbeck process. Albeit its good performance, the proposed model is restrictive in the sense that it depends on pre-specified covariance structure. In conclusion, these models are trading between flexibility and computational difficulty. Even though they achieved the balance for their specific purpose respectively, they are not the ideal option for the specific problem we considered. Besides, they examine the mean and covariance structure of the latent variable separately, which may neglect their dependence. 

% review transition modeling framework
%To evaluate the fixed and random effects jointly, one option is to follow the transition modeling framework \citep[see][chap.~10]{DiggleBook2002}. Under a transition model, the past value explicitly affect the present observation, inducing the aging effects. Therefore, the proposed models under this framework typically assume autoregressive transition dynamics between the distribution of responses at adjacent time.  \citet{DiLucca2013} develop a class of Bayesian nonparametric autoregression models. They model a sequence of continuous outcomes though a dependent DP with an additional normal kernel as a prior for the regression on lagged terms in an autoregression. It can be extended to handle binary and ordinal outcomes by treating the them as the discretized version of the continuous outcomes. In addition, motivated by a specific application in fisheries science, \citet{MariaJASA2018} propose a Bayesian nonparametric model for ordinal regression relationships that evolve in discrete time. Their proposed methodology is built on the well-studied DP mixture model for cross-sectional ordinal regression \citep{DeYoreoKottas2018} as the marginal at a specific time, while introduce temporal dependence in the weights and atoms of the DP constructive definition. Although autoregressive structure brings benefits such as interpretability, there is no natural way to treat missing data, hindering the application in unbalanced longitudinal studies.    

%Under the GMM framework, the mean and covariance structure of the latent variable are examined separately, which may neglect their dependence. To incorporate the possible dependence among them, one option is to follow the transition modeling framework \citep[see][chap.~10]{DiggleBook2002}. Under a transition model, the past value explicitly affect the present observation, inducing the aging effects. The proposed models under this framework typically assume autoregressive transition dynamics between the distribution of responses at adjacent time. Although autoregressive structure brings benefits such as interpretability, there is no natural way to treat missing data, hindering the application in unbalanced longitudinal studies.


The development of statistical methods for longitudinal binary and ordinal data stems from 
models for longitudinal continuous responses, postulating the generalized linear model framework. 
Analogous to the continuous case, a specific model is formulated under one of three broad
approaches pertaining to marginal models, conditional models, or subject-specific models. 
Marginal models provide alternative modeling options when likelihood-based approaches are 
difficult to implement.
%A marginal model captures the responses by their marginal distribution given the covariates. 
A conditional model describes the distribution of responses conditional on the covariates 
and also on part of the other components of the responses. In a subject-specific model, 
the effects of a subset of covariates are allowed to vary randomly from one individual 
to another. In the absence of predictor variables, functions of the observation time are 
usually used as covariates. We refer to \citet{Molenberghs2006} for a comprehensive review. 
In Section \ref{subsec:literaturereview}, we elaborate on the connection of our 
proposed modeling approach with existing methods.


%our approach
In this article, we introduce a novel viewpoint for longitudinal binary and ordinal data 
analysis. %which directs a path for new discoveries. 
We begin with the model construction for longitudinal binary responses. The critical insight 
that distinguishes our methodology from the majority of the existing literature is functional 
data analysis. We treat the subjects' measurements as stochastic process realizations at the 
corresponding time points. The benefits are twofold. First, the models can incorporate 
unbalanced data from longitudinal studies in a unified scheme; directly inferring the 
stochastic process provides a well-defined probabilistic model for the missing values. 
Secondly, we can exploit the power of Bayesian hierarchical modeling for continuous 
functional data \citep[e.g.,][]{Yang2016}. To that end, we adopt the Binomial distribution 
with the logit link that connects binary responses to continuous signals, which, subject to 
%a common level of 
additive measurement error, are then modeled as (conditionally) independent and 
identically distributed (i.i.d.) realizations from a Gaussian process (GP) with 
random mean and covariance function. We place an Inverse-Wishart process (IWP) prior 
on the covariance function, and conditional on it, use a GP prior for the mean function. 
Therefore, the two essential ingredients in longitudinal modeling, the trend and the 
covariance structure, are modeled simultaneously and nonparametrically. 


The hierarchical structure allows borrowing of strength across the subjects' trajectories. 
We apply a specific setting of hyperpriors for the GP and IWP priors, such that marginalizing 
over them, the latent continuous functions have a Student-t process (TP) prior. The 
TP enhances the flexibility of the GP \citep[e.g.,][]{Shah2014}. It retains attractive GP 
properties, such as analytic marginal and predictive distributions, and it yields 
predictive covariance that, unlike the GP, explicitly depends on the observed values. 
% what are the benefit?
For inferential purposes, we represent the joint posterior distribution in multivariate form 
through evaluating the functions on the pooled grid, resulting in the common 
normal-inverse-Wishart conditional conjugacy. In conjunction with the Pólya-Gamma data 
augmentation technique \citep{Polson2013}, we develop a relatively simple and effective posterior 
simulation algorithm, circumventing the need for specialized techniques or tuning of 
Metropolis-Hastings steps. 


To extend the model for ordinal responses, we utilize the continuation-ratio logits 
representation of the multinomial distribution. Such representation features an %interchangeable 
encoding of an ordinal response with $C$ categories as a sequence of $C-1$ binary 
indicators, in which the $j$-th indicator signifies whether the ordinal response belongs 
to the $j$-th category or to one of the higher categories. 
%As shown in Proposition \ref{prop:fitseparate}, 
We show that fitting a multinomial model for the ordinal responses is equivalent 
to fitting separately the aforementioned model on the binary indicators. 
%As a consequence, the model for longitudinal ordinal responses inherits the features of 
%the model for binary responses. Furthermore, 
Hence, we can conduct posterior simulation for each response category in a parallel fashion, 
leading to significant computational efficiency gains in model implementation.       


% about the missingness
In modern longitudinal studies, it is common that the complete vector of repeated measurements 
is not collected on all subjects. As a specific example, in ecological momentary assessment 
(EMA) studies, emotions and behaviors are repeatedly measured for a cohort of participants, 
through wearable electronic devices \citep{EMABook2018}. For instance, in the \textit{StudentLife} 
study \citep{StudentLife2014}, researchers monitored the students' mental status through 
pop-up questionnaires on their smartphones that prompted multiple times at pseudorandom 
intervals during the study period. Since the data collection process is based on the 
participants' conscious responding to prompted questions several times a day, non-response 
is inevitable. Missing values are typically considered to be a nuisance rather than a 
characteristic of EMA time series. Parametric and nonparametric Bayesian methods have been 
developed to handle longitudinal data with missingness; see \citet{Daniels2020} for a review. 
The common issue is that one has to bear the drawbacks of making either structured or 
unstructured assumptions to manage missingness. The unstructured approach leads to flexibility, 
yet it may result in difficulties due to a large number of parameters relative to the sample 
size. Besides, the majority of the existing literature on longitudinal studies with missingness 
focuses on the scenario with continuous responses, and the extension to discrete responses is not trivial. 


% summary of our contribution
Accordingly, our contributions can be summarized as follows: (i) we model the mean and 
covariance jointly and nonparametrically, avoiding potential biases caused by a pre-specified 
model structure; (ii) we unify the toolbox for balanced and unbalanced longitudinal studies; 
(iii) the model encourages borrowing of strength, preserving systematic patterns that are 
common across all subject responses; (iv) we develop a computationally efficient posterior 
simulation method by taking advantage of conditional conjugacy; (v) the model facilitates 
applications for ordinal responses with a moderate to large number of categories. 
% and parallel computing.    


% organization of the paper
The rest of the paper is organized as follows. 
%We start from emphasizing the pivotal model designs and benefits with longitudinal 
%binary responses in Section \ref{sec:binarymodel} and Section \ref{sec:realapp}, 
%as it lays the foundation for the extension to ordinal responses. In more detail, 
Section \ref{sec:binarymodel} develops the methodology for binary responses, including
model formulation, study of model properties, and the computational approach to 
inference and prediction.
%In Section \ref{sec:simstudy}, we assess the model by applying it to carefully designed 
%simulation scenarios that reflect our main contributions. 
Section \ref{sec:realapp} illustrates the modeling approach through an EMA study that 
focuses on analyzing students' mental health through binary outcomes. The modeling extension 
for longitudinal ordinal responses is presented in Section \ref{sec:polyordinalmodel}, 
including an illustration involving an ordinal outcome from the same EMA study. Finally, 
Section \ref{sec:summary} concludes with a summary.



 
\section{Incremental scaling and squaring}
\label{sec:scaling_and_squaring}

Since the set of conformally partitioned block triangular matrices
forms an algebra, and $\exp(G_n)$ is a polynomial in $G_n$, the matrix
$\exp(G_n)$ has the same block upper triangular structure as $G_n$, that is,
\begin{equation*}
    \exp(G_n) =
    \begin{bmatrix}
        \exp(G_{0,0}) & \ast          & \cdots & \ast \\
                      & \exp(G_{1,1}) & \ddots & \vdots \\
                      &               & \ddots & \ast  \\
                      &               &        & \exp(G_{n,n}) \\
    \end{bmatrix}
    \in \R^{d_n \times d_n}.
\end{equation*}
As outlined in the introduction, we aim at computing $\exp(G_n)$ block
column by block column, from left to right.  Our algorithm is based on
the scaling and squaring methodology, which we briefly summarize
next.

\subsection{Summary of the scaling and squaring method}
\label{sec:scaling_squaring}

The scaling and squaring method uses a rational function to
approximate the exponential function, and typically involves three
steps. Denote by $r_{k,m}(z) = \frac{p_{k,m}(z)}{q_{k,m}(z)}$ the
$(k,m)$-Pad\'e approximant to the exponential function, meaning that
the numerator is a polynomial of degree $k$, and the denominator is a
polynomial of degree $m$. These Pad\'e approximants are very accurate close
to the origin, and in a first step the input matrix $G$ is therefore scaled by a
power of two, so that $\norm{2^{-s}G}$ is small enough to guarantee an
accurate approximation $r_{k,m}(2^{-s}G) \approx \exp(2^{-s} G)$.

The second step consists of evaluating the rational approximation
$r_{k,m}(2^{-s}G)$, and, finally, an approximation
to $\exp(G)$ is obtained in a third step by repeatedly squaring the
result, i.e.,
\begin{equation*}
    \exp(G) \approx r_{k,m}(2^{-s}G)^{2^s}.
\end{equation*}

Different choices of the scaling parameter $s$, and of the
approximation degrees $k$ and $m$ yield methods of different
characteristics. The choice of these parameters is critical for
the approximation quality, and for the computational efficiency,
see~\cite[chapter~10]{Higham2008}.

In what follows we describe techniques that allow for an incremental
evaluation of the matrix exponential of the block triangular
matrix~\eqref{eq:G_n}, using scaling and squaring.  These techniques
can be used with any choice for the actual underlying scaling and
squaring method, defined through the parameters $s$, $k$, and $m$.


% The idea developed in our algorithm can be employed for every type of
% scaling and squaring method. In fact, our algorithm is mainly based on
% the modification of the basic operations performed in the scaling and
% squaring method. For instance, the computational effort of the
% algorithm consists in building the powers of the scaled input matrix
% $2^{-s}G$ (when we compute $p_{k,m}(2^{-s}G)$ and $q_{k,m}(2^{-s}G)$);
% in evaluating the inversion $p_{k,m}(2^{-s}G)^{-1}q_{k,m}(2^{-s}G)$
% and in the squaring phase. In each of these operations the structure
% of the input matrix is exploited and past stored quantities are
% reused.

\subsection{Tools for the incremental computation of exponentials}
\label{sec:tools}
Before explaining the algorithm, we first introduce some notation that
is used throughout.  The matrix $G_n$ from~\eqref{eq:G_n} can be written as
\begin{equation}
    \label{eq:G_twobytwo}
    G_n =
\left[
    \begin{array}{ccc|c}
        G_{0,0} &  \cdots &G_{0,n-1}& G_{0,n} \\
                &  \ddots & \vdots & \vdots  \\ 
                &   & G_{n-1,n-1}& G_{n-1,n} \\ \hline
                &         & & G_{n,n} \\
    \end{array}
\right]
    \bydef
    \begin{bmatrix}
        G_{n-1} & g_n \\
        0       & G_{n,n}
    \end{bmatrix}
\end{equation}
where $G_{n-1} \in \mathbb{R}^{d_{n-1} \times d_{n-1}}$, $G_{n,n}\in
\mathbb{R}^{b_n \times b_n}$, so that $g_n \in \R^{d_{n-1} \times
b_n}$.  Let $s$ be the scaling parameter, and $r = \frac{p}{q}$ the
rational function used in the approximation (for simplicity we will often omit the indices $k$ and $m$).  We denote the scaled
matrix by $\tilde{G}_n \defby 2^{-s} G_n$ and we partition it as
in~\eqref{eq:G_twobytwo}.\

The starting point of the algorithm consists in computing the Pad\'e approximant of the exponential $\exp(G_0)= \exp(G_{0,0})$, using a scaling and squaring method. Then, the sequence of matrix exponentials~\eqref{eq:exp_sequence} is incrementally computed by reusing at each step previously obtained quantities. So more generally, assume that $\exp(G_{n-1})$ has been approximated by using a scaling and squaring
method.  The three main computational steps for obtaining the Pad\'e approximant of $\exp(G_n)$ are
\begin{inparaenum}[(i)]
\item evaluating the polynomials $p(\tilde{G}_n)$, $q(\tilde{G}_n)$,
\item evaluating $p(\tilde{G}_n)^{-1} q(\tilde{G}_n)$, and
\item repeatedly squaring it.
\end{inparaenum}
We now discuss each of these steps separately, noting the quantities to keep at every iteration.

\subsubsection{Evaluating $p(\tilde{G}_n)$, $q(\tilde{G}_n)$ from $p(\tilde{G}_{n-1})$, $q(\tilde{G}_{n-1})$}
Similarly to \eqref{eq:G_twobytwo}, we start by writing $P_n\defby p(\tilde{G}_n)$ and $Q_n\defby q(\tilde{G}_n)$ as
\begin{equation*}
    P_n = 
    \begin{bmatrix}
        P_{n-1} & p_n \\
        0       & P_{n,n}
    \end{bmatrix},
    \quad
    Q_n = 
    \begin{bmatrix}
        Q_{n-1} & q_n \\
        0       & Q_{n,n}
    \end{bmatrix}.
\end{equation*}
In order to evaluate $P_{n}$, we first need to compute monomials of $\tilde{G}_{n}$, which for $l = 1, \dotsc, k$, can be written as
\begin{equation*}
    \tilde{G}_{n}^l =
    \begin{bmatrix}
        \tilde{G}_{n-1}^l & \sum_{j=0}^{l-1}
            \tilde{G}_{n-1}^j \tilde{g}_{n} \tilde{G}_{n,n}^{l-j-1} \\
        &                   \tilde{G}_{n,n}^l
    \end{bmatrix}.
\end{equation*}
Denote by $X_l \defby \sum_{j=0}^{l-1} \tilde{G}_{n-1}^j
\tilde{g}_{n} \tilde{G}_{n,n}^{l-j-1}$ the upper off diagonal block of
$\tilde{G}_n^l$, then we have the relation
\begin{equation*}
    X_{l} = \tilde{G}_{n-1}X_{l-1} + \tilde{g}_n \tilde{G}_{n,n}^{l-1}, \quad \text{for } l = 2, \cdots, k,
\end{equation*}
with $X_{1} \defby \tilde{g}_n$, so that all the monomials $\tilde{G}_n^l$, $l = 1, \dotsc, k$, can be
computed in $\bigO(b_n^3 + d_{n-1} b_n^2 + d_{n-1}^2 b_n)$.  Let $p(z)
= \sum_{l=0}^k \alpha_l z^l$ be the numerator polynomial of $r$, then
we have that
\begin{equation}
    \label{eq:P_n}
    P_n = 
    \begin{bmatrix}
        P_{n-1} & \sum_{l=0}^{k} \alpha_l X_l\\
                & p(\tilde{G}_{n,n})
    \end{bmatrix},
\end{equation}
which can be assembled in $\bigO(b_n^2 + d_{n-1} b_n)$, since only the
last block column needs to be computed. The complete evaluation of $P_n$ is summarized in Algorithm~\ref{alg:P_n}.
\begin{algorithm}[ht]
    \caption{Evaluation of $P_n$, using $P_{n-1}$
    \label{alg:P_n}}
    \begin{algorithmic}[1]
        \REQUIRE $G_{n-1}, G_{n,n}, g_n, P_{n-1}$, Pad\'e coefficients $\alpha_l, l = 0, \cdots, k$.
        \ENSURE $P_n$.
        \STATE $\tilde{g}_n \leftarrow 2^{-s} g_n$,
            $\tilde{G}_{n,n} \leftarrow 2^{-s} G_{n,n}$,
	 $\tilde{G}_{n-1} \leftarrow 2^{-s} G_{n-1}$
        \STATE $X_1 \leftarrow \tilde{g}_n$
        \FOR {$l=2, 3,  \cdots, k$}
	\STATE Compute $\tilde{G}_{n,n}^l$
         	\STATE $X_l = \tilde{G}_{n-1} X_{l-1} + \tilde{g}_n \tilde{G}_{n,n}^{l-1}$
        \ENDFOR
        \STATE $X_0 \leftarrow \mathbf{0}_{d_{n-1} \times b_n}$
        \STATE Compute off diagonal block of $P_n$:  $\sum_{l=0}^{k} \alpha_l X_l$.
        \STATE Compute $p(\tilde{G}_{n,n}) = \sum_{l=0}^{k} \alpha_l \tilde{G}_{n,n}^l$
        \STATE Assemble $P_n$ as in $\eqref{eq:P_n}$
    \end{algorithmic}
\end{algorithm}

Similarly, one computes $Q_n$ from $Q_{n-1}$, using again the matrices
$X_l$.
\subsubsection{Evaluating $Q_n^{-1} P_n$}

With the matrices $P_n$, $Q_n$ at hand, we now need to compute the
rational approximation $Q_n^{-1} P_n$. 
We assume that $Q_n$ is well
conditioned, in particular non-singular, which is ensured by the choice of the 
scaling parameter and of the Pad\'e approximation, see, e.g.,~\cite{Higham2009}.
We focus on the computational cost.
For simplicity, we introduce the notation
\begin{equation*}
    \tilde{F}_n = 
    \begin{bmatrix}
        \tilde{F}_{0,0} &  \cdots & \tilde{F}_{0,n} \\
                        &  \ddots & \vdots  \\
                        &         & \tilde{F}_{n,n} \\
    \end{bmatrix}
    \defby Q_n^{-1} P_n,
    \quad
    F_n = 
    \begin{bmatrix}
        F_{0,0} &  \cdots & F_{0,n} \\
                &  \ddots & \vdots  \\
                &         & F_{n,n} \\
    \end{bmatrix}
    \defby \tilde{F}_n^{2^s},
\end{equation*}
and we see that
\begin{equation}
    \label{eq:F_n_tilde}
    \begin{split}
    \tilde{F}_n & = Q_n^{-1} P_n =
    \begin{bmatrix}
        Q_{n-1}^{-1} & - Q_{n-1}^{-1} q_n Q_{n,n}^{-1} \\
        0            & Q_{n,n}^{-1}
    \end{bmatrix}
    \begin{bmatrix}
        P_{n-1} & p_n \\
        0       & P_{n,n}
    \end{bmatrix}\\
    & =
    \begin{bmatrix}
        \tilde{F}_{n-1} & Q_{n-1}^{-1} ( p_n - q_n Q_{n,n}^{-1} P_{n,n} ) \\
        0               & Q_{n,n}^{-1} P_{n,n}
    \end{bmatrix}.
    \end{split}
\end{equation}

To solve the linear system $Q_{n,n}^{-1} P_{n,n}$ we compute an LU
decomposition with partial pivoting for
$Q_{n,n}$, requiring $\bigO(b_n^3)$ operations.
This LU decomposition is saved for future use, and hence we may assume
that we have available the LU decompositions for all diagonal
blocks from previous computations:
\begin{equation}
    \label{eq:store_lu}
    \Pi_l Q_{l,l}= L_l U_l, \quad l=0, \dotsc, n-1.
\end{equation}
Here, $\Pi_l \in \mathbb{R}^{b_l \times b_l},$ $l=0, \dotsc, n-1$ are permutation matrices; 
$L_l\in \mathbb{R}^{b_l \times b_l},$ $l=0, \dotsc, n-1$ are lower triangular matrices and $U_l\in \mathbb{R}^{b_l \times b_l},$ $l=0, \dotsc, n-1$ are upper triangular matrices.

Set $Y_n \defby p_n - q_n Q_{n,n}^{-1} P_{n,n} \in \R^{d_{n-1} \times
b_n}$, and partition it as
\begin{equation*}
    Y_n = 
    \begin{bmatrix}
      Y_{0,n} \\
      \vdots  \\
      Y_{n-1,n} \\
    \end{bmatrix}.
\end{equation*}
Then we compute $Q_{n-1}^{-1} Y_n$ by block backward
substitution, using the decompositions of the diagonal blocks.  The
total number of operations for this computation is hence $\bigO(d_{n-1}^2 b_n + d_{n-1} b_n^2)$, so that the number of
operations for computing $\tilde{F}_n$ is $\bigO(b_n^3 + d_{n-1}^2
b_n + d_{n-1} b_n^2)$.
Algorithm~\ref{alg:tilde_F_n} describes the complete procedure to compute $\tilde{F}_n$.

\begin{algorithm}[ht]
    \caption{Evaluation of $\tilde{F}_n = Q_n^{-1} P_n$}
    \label{alg:tilde_F_n}
    \begin{algorithmic}[1]
        \REQUIRE $Q_n, P_n$ and quantities \eqref{eq:store_lu}
        \ENSURE $\tilde{F}_n = Q_n^{-1} P_n$ and LU decomposition of $Q_{n,n}$.
        \STATE Compute $\Pi_n Q_{n,n}= L_n U_n$ and keep it for future use \eqref{eq:store_lu}
        \STATE Compute  $\tilde{F}_{n,n} \defby Q_{n,n}^{-1} P_{n,n}$
        \STATE $Y_n = p_n - q_n Q_{n,n}^{-1} P_{n,n}$
        \STATE $\tilde{F}_{n-1,n} = U_{n-1}^{-1} L_{n-1}^{-1} \Pi_{n-1} Y_{n-1,n} $
        \FOR {$l = n - 2, n - 3,  \cdots, 0$}
        \STATE $\tilde{F}_{l, n} = U_{l}^{-1}L_{l}^{-1} \Pi_{l}(Y_{l, n}-\sum_{j=l+1}^{n-1} Q_{l,j}\tilde{F}_{j, n}) $
        \ENDFOR
        \STATE Assemble $\tilde{F}_n$ as in \eqref{eq:F_n_tilde}
    \end{algorithmic}
\end{algorithm}
\subsubsection{The squaring phase}

Having computed $\tilde{F}_n$, which we write as
\begin{equation*}
    \tilde{F}_n =
    \begin{bmatrix}
        \tilde{F}_{n-1} & \tilde{f}_n \\
                        & \tilde{F}_{n,n}
    \end{bmatrix},
\end{equation*}
we now need to compute $s$ repeated squares of that matrix, i.e.,
\begin{equation}
    \label{eq:squares}
    \tilde{F}_{n}^{2^l} =
    \begin{bmatrix}
        \tilde{F}_{n-1}^{2^l} & \sum_{j=0}^{l-1}
            \tilde{F}_{n-1}^{2^{l - 1 + j}} \tilde{f}_n \tilde{F}_{n,n}^{2^j}\\
                              & \tilde{F}_{n,n}^{2^l}
   \end{bmatrix}, \quad l = 1, \dotsc, s,
\end{equation}
so that $F_n = \tilde{F}_{n}^{2^s}$.  Setting $Z_l \defby
\sum_{j=0}^{l-1} \tilde{F}_{n-1}^{2^{l - 1 + j}} \tilde{f}_j
\tilde{F}_{n,n}^{2^j}$, we have the recurrence
\begin{equation*}
    Z_l = \tilde{F}_{n-1}^{2^{l-1}} Z_{l-1} + Z_{l-1} \tilde{F}_{n,n}^{2^{l-1}},
\end{equation*}
with $Z_0 \defby \tilde{f}_n$.  Hence, if we have stored the
intermediate squares from the computation of $F_{n-1}$, i.e.,
\begin{equation}
    \label{eq:store_squares}
    \tilde{F}_{n-1}^{2^l},  \quad l=1, \dotsc, s
\end{equation}
we can compute all the quantities $Z_l$, $l=1, \dotsc, s$ in
$\bigO(d_{n-1}^2 b_n + d_{n-1} b_n^2)$, so that the total cost for
computing $F_n$ (and the intermediate squares of $\tilde{F}_n$) is
$\bigO(d_{n-1}^2 b_n + d_{n-1} b_n^2 + b_n^3)$. Again, we summarize the squaring phase in the following algorithm.

\begin{algorithm}[ht]
    \caption{Evaluation of $F_n = \tilde{F}_n^{2^s}$
    \label{alg:F_n}}
    \begin{algorithmic}[1]
        \REQUIRE $\tilde{F}_{n-1}, \tilde{f}_n, \tilde{F}_{n,n}$, quantities \eqref{eq:store_squares}.
        \ENSURE $F_n$ and updated intermediates.
        \STATE $Z_0 \leftarrow \tilde{f}_n$
        \FOR {$l=1, 2,  \cdots, s$}
          \STATE Compute $\tilde{F}_{n,n}^{2^l}$
          \STATE $Z_l = \tilde{F}_{n-1}^{2^{l-1}} Z_{l-1} + Z_{l-1} \tilde{F}_{n,n}^{2^{l-1}}$
          \STATE Assemble $\tilde{F}_n^{2^l}$ as in \eqref{eq:squares} and save it
        \ENDFOR
          \STATE $F_n \leftarrow \tilde{F}_{n}^{2^s}$
    \end{algorithmic}
\end{algorithm}
\subsection{Overall Algorithm}
\label{sec:overall}

Using the techniques from the previous section, we now give a concise
description of the overall algorithm.  We assume that the quantities
listed in equations~\eqref{eq:store_lu}
and~\eqref{eq:store_squares} are stored in memory, with a space requirement of $\bigO(d_{n-1}^2)$. 

In view of this, we assume that $F_{n-1}$ and the aforementioned intermediate
quantities have been computed.  Algorithm~\ref{alg:step} describes the
overall procedure to compute $F_n$, and to update the intermediates;
we continue to use the notation introduced in~\eqref{eq:G_twobytwo}.

\begin{algorithm}[ht]
    \caption{Computation of $F_n \approx \exp(G_n)$, using $F_{n-1}$
    \label{alg:step}}
    \begin{algorithmic}[1]
        \REQUIRE Block column $g_n$, diagonal block $G_{n,n}$, quantities \eqref{eq:store_lu},
        and~\eqref{eq:store_squares}.
        \ENSURE $F_n$, and updated intermediates.
        \STATE Extend $P_{n-1}$ to $P_n$ using Algorithm~\ref{alg:P_n},
        and form analogously $Q_n$
        \STATE Compute $\tilde{F}_n$ using Algorithm~\ref{alg:tilde_F_n}
        \STATE Evaluate $F_n = \tilde{F}_n^{2^s}$
        using Algorithm~\ref{alg:F_n}
    \end{algorithmic}
\end{algorithm}

As explained in the previous section, the number of operations for
each step in Algorithm~\ref{alg:step} is $\bigO(d_{n-1}^2 b_n +
d_{n-1} b_n^2 + b_n^3)$, using the notation at the beginning of
section~\ref{sec:tools}.  If $F_n$ were simply computed from scratch, without
the use of the intermediates, the number of operations for scaling and
squaring would be $\bigO((d_{n-1} + b_n)^3)$.  In the typical
situation where $d_{n-1} \gg b_n$, the dominant term in the latter
complexity bound is $d_{n-1}^3$, which is absent from the complexity
bound of Algorithm~\ref{alg:step}.

In order to solve our original problem, the computation of the
sequence $\exp(G_0)$, $\exp(G_1)$, $\exp(G_2)$, $\dotsc$, we use
Algorithm~\ref{alg:step} repeatedly; the resulting procedure is shown
in Algorithm~\ref{alg:full}.

\begin{algorithm}[ht]
    \caption{Approximation of $\exp(G_0), \exp(G_1), \dotsc$
    \label{alg:full}}
    \begin{algorithmic}[1]
        \REQUIRE Pad\'e approximation parameters $k$, $m$, and $s$
        \ENSURE $F_0 \approx \exp(G_0)$, $F_1 \approx \exp(G_1), \dotsc$
        \STATE Compute $F_0$ using scaling and squaring, store
            intermediates for Algorithm~\ref{alg:step}
        \FOR{$n=1,2,\dotsc$}
            \STATE Compute $F_{n}$ from $F_{n-1}$ using Algorithm~\ref{alg:step}
            \IF{termination criterion is satisfied} \RETURN
            \ENDIF
        \ENDFOR
    \end{algorithmic}
\end{algorithm}

We now derive a complexity bound for the number of operations
spent in Algorithm~\ref{alg:full}.  For simplicity of notation we
consider the case where all diagonal blocks are of equal size, i.e.,
$b_k \equiv b \in \N$, so that $d_k = (k+1)b$.  At iteration $k$ the
number of operations spent within Algorithm~\ref{alg:step} is thus 
$\bigO(k^2 b^3)$.  Assume that the termination criterion used in
Algorithm~\ref{alg:full} effects to stop the procedure after the
computation of $F_n$.  The overall complexity bound for the number of
operations until termination is $\bigO(\sum_{k=0}^n k^2 b^3 ) =
\bigO(n^3 b^3)$, which matches the complexity bound of applying
scaling and squaring only to $G_n \in \R^{(n+1)b \times (n+1)b}$,
which is also $\bigO( (nb)^3 )$.

In summary the number of operations needed to compute $F_n$ by
Algorithm~\ref{alg:full} is asymptotically the same as applying the
same scaling and squaring setting \emph{only} to compute $\exp(G_n)$,
while Algorithm~\ref{alg:full} incrementally reveals \emph{all}
exponentials $\exp(G_0)$, $\dotsc$, $\exp(G_n)$ in the course of the
iteration, satisfying our requirements outlined in the introduction.

\subsection{Adaptive scaling}

In Algorithms~\ref{alg:step} and~\ref{alg:full} we have assumed
that the scaling power $s$ is given as input parameter, and that it is
fixed throughout the computation of $\exp(G_0), \dotsc, \exp(G_n)$.
This is in contrast to what is usually intented in the scaling and squaring method, see Section~\ref{sec:scaling_squaring}.  On the one hand $s$
must be sufficiently large so that $r_{k,m}(2^{-s} G_l) \approx \exp(2^{-s}
G_l)$, for $0 \le l \le n$.  If, on the other hand, $s$ is chosen
\emph{too large}, then the evaluation of $r_{k,m}(2^{-s}G_l)$ may become
inaccurate, due to \emph{overscaling}.
So if $s$ is fixed, and the norms $\norm{G_l}$ grow with
increasing $l$, as one would normally expect, an accurate approximation cannot
be guaranteed for all $l$.

Most scaling and squaring designs hence choose $s$ in dependence of the
norm of the input matrix~\cite{Moler2003,Guttel2016,Higham2009}.  For
example, in the algorithm of Higham described in~\cite{Higham2009}, it
is the smallest integer satisfying
\begin{equation}
    \label{eqn:theta}
    \norm{2^{-s} G_l}_1 \le \theta \approx 5.37... .
\end{equation}
In order to combine our incremental evaluation techniques with this
scaling and squaring design, the scaling power $s$ must thus be chosen
dynamically in the course of the evaluation.  Assume that $s$
satisfies the criterion~\eqref{eqn:theta} at step $l-1$, but not at step
$l$.  We then simply discard all accumulated data structures from
Algorithm~\ref{alg:step}, increase $s$ to match the
bound~\eqref{eqn:theta} for $G_{l}$, and start
Algorithm~\ref{alg:full} anew with the \emph{repartitioned} input matrix
\begin{equation}
    \label{eqn:repart}
G_n = \left[
    \begin{array}{ccc|c|c|c}
        G_{0,0} &  \cdots &G_{0,l}  & G_{0,l+1}  & \cdots & G_{0,n}\\
                &  \ddots & \vdots  & \vdots     &        & \vdots\\ 
                &         & G_{l,l} & G_{l,l+1}  & \cdots & G_{l,n}\\ \hline
                &         &         & G_{l+1,l+1}& \cdots & G_{l+1,n}\\
                &         &         &            & \ddots & \vdots   \\
                &         &         &            &        & G_{n,n}  \\
    \end{array}
\right]
    =
    \underbrace{
\left[
    \begin{array}{c|c|c|c}
        \hat{G}_{0,0} & \hat{G}_{0,1}    & \cdots & \hat{G}_{0,n-l}\\ \hline
                      & \hat{G}_{1,1}    & \cdots & \hat{G}_{1,n-l}\\
                      &                  & \ddots & \vdots   \\
                      &                  &        & \hat{G}_{n-l,n-l}  \\
    \end{array}
\right]}_{\bydef \hat{G}_{n-l}}.
\end{equation}
The procedure is summarized in Algorithm~\ref{alg:adaptive}.

\begin{algorithm}[t]
    \caption{Approximation of $\exp(G_0), \exp(G_1), \dotsc$ with
    adaptive scaling
    \label{alg:adaptive}}
    \begin{algorithmic}[1]
        \REQUIRE Pad\'e approximation parameters $k$, $m$, norm
        bound $\theta$.
        \ENSURE $F_0 \approx \exp(G_0)$, $F_1 \approx \exp(G_1), \dotsc$
        \STATE $s \leftarrow \max\{0, \log(\norm{G_0}_1)\}$
        \STATE Compute $F_0$ using scaling and squaring, store
            intermediates for Algorithm~\ref{alg:step}
        \FOR{$l=1,2,\dotsc$}
            \IF{ $\norm{G_l}_1 > \theta$ }
                \STATE Repartition $G_n = \hat{G}_{n-l}$ as in~\eqref{eqn:repart}
                \STATE Restart algorithm with $\hat{G}_{n-l}$.
            \ENDIF
            \STATE Compute $F_{l}$ from $F_{l-1}$ using Algorithm~\ref{alg:step}
            \IF{termination criterion is satisfied} \RETURN
            \ENDIF
        \ENDFOR
    \end{algorithmic}
\end{algorithm}

It turns out that the computational overhead induced by this
restarting procedure is quite modest.  In the notation introduced for
the complexity discussion in Section~\ref{sec:overall}, the number of
operations for computing $\exp(G_n)$ by Higham's scaling and squaring
method is $\bigO(\log(\norm{G_n}_1) (nb)^3)$.  Since there are at
most $\log(\norm{G_n}_1)$ restarts in Algorithm~\ref{alg:adaptive},
the total number of operations for incrementally computing all
exponentials $\exp(G_0), \dotsc, \exp(G_n)$ can be bounded by a
function in $\bigO(\log(\norm{G_n}_1)^2 (nb)^3)$.  We assess the
actual performance of Algorithm~\ref{alg:adaptive} in
Section~\ref{sec:numerical_experiments}.

In our application from option pricing it turns out that the norms of
the matrices $G_l$ do not grow dramatically (see
Sections~\ref{sec:jacobix} and~\ref{sec:hestonx}) and quite accurate
approximations to all the matrix exponentials can be computed even if
the scaling factor is fixed (see Section~\ref{sec:exp_jacobi}).

%\section{Option pricing in polynomial models}\label{sct3}

The main purpose of this section is to explain how certain option pricing techniques require the sequential computation of matrix exponentials for block triangular matrices. The description will necessarily be rather brief; we refer to, e.g., the textbook~\cite{Elliot2005} for more details.

Because we are evaluating at initial time $t=0$, the price of a certain option expiring at time $\tau>0$ consists of computing an expression of the form
\begin{equation}\label{priceexpectation}
e^{-r\tau} \mathbb{E}[f(X_\tau)],
\end{equation}
where $(X)_{0 \leq t \leq \tau}$ is a $d$-dimensional stochastic process modelling the price of financial assets over the time interval $[0,\tau]$, $f : \mathbb{R}^d \to \mathbb{R}$ is the so-called payoff function and $r$ represents a fixed interest rate.  In the following, we consider stochastic processes described by an SDE of the form
\begin{align}\label{SDE}
dX_t=b(X_t)dt+\Sigma(X_t)dW_t,
\end{align}
where $W$ denotes a $d$-dimensional Brownian motion, $b : \mathbb{R}^d \mapsto \mathbb{R}^{d}$, and $\Sigma : \mathbb{R}^d \mapsto \mathbb{R}^{d \times d}$. 



\subsection{Polynomial diffusion models}

During the last years, polynomial diffusion models have become a versatile tool in financial applications, including option pricing. In the following, we provide a short summary and refer to the paper by Filipovi\'c and Larsson \cite{filipovic2016polynomial} for the mathematical foundations. 

For a polynomial diffusion process one assumes that the coefficients of the vector $b$ in~\eqref{SDE} and the matrix $A : = \Sigma \Sigma^T$ satisfy
\begin{equation}\label{polynomial}
A_{ij} \in \text{Pol}_2(\mathbb{R}^d), \qquad b_i \in \text{Pol}_1(\mathbb{R}^d)  \quad \text{for} \quad i,j = 1,\ldots,d.
\end{equation}
Here, $\text{Pol}_n(\mathbb{R}^d)$ represents the set of $d$-variate polynomials of total degree at most $n$, that is,
\begin{equation*}
\text{Pol}_n(\mathbb{R}^d):= \Big\{\sum_{0 \le |\textbf{k}|\le n} \alpha_{\textbf{k}} x^{\bf{k}}| x \in \mathbb{R}^d, \alpha_{\textbf{k}} \in \mathbb{R}\Big\},
\end{equation*}
where we use multi-index notation: $\mathbf{k}=(k_1, \dots, k_d) \in \mathbb{N}_0^d$, $|\mathbf{k}|:=k_1+\dots+k_d$ and $x^{\bf{k}}:=x_1^{k_1}\dots x_d^{k_d}$. In the following, $\text{Pol}(\mathbb{R}^d)$ represents the set of all multivariate polynomials on $\mathbb{R}^d$.\

Associated with $A$ and $b$ we define the partial differential operator $\mathcal{G}$ by 
\begin{equation}\label{generator}
\mathcal{G}f=\frac{1}{2}\tr(A \nabla^2 f)+b^T \nabla f.
\end{equation}
which represents the so called generator
%\footnote{
%The generator of a stochastic process $(X_t)_{0 \leq t \leq T}$ is defined as $\mathcal{G}f(x):= \lim_{t \to 0}\frac{\mathbb{E}^x[f(X_t)]-f(x)}{t}$, where the expectation is taken with respect to the law $\mathbb{P}^x$ of $X$ given $X_0=x$. See for more details.}
for~\eqref{SDE}, see~\cite{Oksendal2003}. It can be directly verified that~\eqref{polynomial} implies that $\text{Pol}_n(\mathbb{R}^d$) is invariant under $\mathcal{G}$  for any $n \in \mathbb{N}$, that is, 
\begin{equation}\label{PreservingProperty}
\mathcal{G}\text{Pol}_n(\mathbb{R}^d) \subseteq \text{Pol}_n(\mathbb{R}^d).
\end{equation}
\begin{remark}
    \begin{rm}
In many applications, one is interested in solutions to $\eqref{SDE}$ that lie on a state space $E \subseteq \mathbb{R}^d$ to incorporate, for example, nonnegativity. This problem is largely studied in~\cite{filipovic2016polynomial}, where existence and uniqueness of solutions to \eqref{SDE} on several types of state spaces $E \subseteq \mathbb{R}^d$ and for large classes of $A$ and $b$ is shown.
    \end{rm}
\end{remark}
Let us now fix a basis of polynomials $\mathcal{H}_n=\{h_1, \dots, h_N\}$ for $\text{Pol}_n(\mathbb{R}^d)$, where $N= \dim \text{Pol}_n(\mathbb{R}^d) = \binom{n+d}{n}$, and write
\begin{equation*}
H_n(x)=(h_1(x), \dots, h_N(x))^T.
\end{equation*}
Let $G_n$ denote the matrix representation with respect to $\mathcal{H}$ of the linear operator $\mathcal{G}$ restricted to $\text{Pol}_n(\mathbb{R}^d)$. By definition,
\begin{equation*}
\mathcal{G}p(x)=H_n(x)^T G_n \vec{p}.
\end{equation*}
for any $p\in \text{Pol}_n(\mathbb{R}^d)$ with coordinate vector $\vec{p} \in \R^N$ with respect to $\mathcal{H}_n$.
By Theorem 3.1 in \cite{filipovic2016polynomial}, the  corresponding polynomial moment can be computed from
\begin{equation}\label{condmoments}
\mathbb{E}[p(X_\tau)]=H_n(X_0)^Te^{\tau G_n} \vec{p}.
\end{equation}
% \begin{remark}
% To be more precise, Theorem 3.1 in~\cite{filipovic2016polynomial} provides a more general formula for the computation of conditional polynomial moments at any time $t \leq T$. However, in our applications we are always interested in the special case $t=0$ and we use \eqref{condmoments}. 
% \end{remark}

The setting discussed above corresponds to the scenario described in the introduction. We have a sequence of subspaces
\[ \text{Pol}_0(\mathbb{R}^d) \subseteq \text{Pol}_1(\mathbb{R}^d) \subseteq \text{Pol}_2(\mathbb{R}^d) \subseteq \cdots \subseteq \text{Pol}(\mathbb{R}^d)\]
and the polynomial preserving property~\eqref{PreservingProperty} implies that the matrix representation $G_n$ is block upper triangular with $n+1$ square diagonal blocks of size \[1, d, \binom{1+d}{2}, \ldots, \binom{n+d-1}{n}.\]

In the rest of this section we introduce two different pricing techniques that require the incremental computation of polynomial moments of the form~\eqref{condmoments}. %This needs the iterative extension of the vector space $\text{Pol}_n(\mathbb{R}^d)$ (increasing $n$) and the computation of the sequence of matrix exponentials \eqref{eq:exp_sequence}.
%Our algorithm described in chapter 2 can be used in this context.
%%%%%%%%%%%%%%%%%%%%%%%%%%%%%%%%%%%%%%%%%%%%%%%%%%%%%%%%%%%%

\subsection{Moment-based option pricing for Jacobi models} \label{sec:jacobix}
The Jacobi stochastic volatility model is a special case of a polynomial diffusion model and it is characterized by the SDE
\begin{align*}
&dY_t=(r-V_t/2)dt + \rho \sqrt{Q(V_t)} dW_{1t}+\sqrt{V_t-\rho^2Q(V_t)}dW_{2t},\\
&dV_t=\kappa(\theta -V_t)dt+\sigma \sqrt{Q(V_t)}dW_{1t},
\end{align*}
where 
\begin{equation*}
Q(v)=\frac{(v-v_{\min})(v_{\max}-v)}{(\sqrt{v_{\max}}-\sqrt{v_{\min}})^2},
\end{equation*}
for some $0 \leq v_{\min} < v_{\max}$. Here, $W_{1t}$ and $W_{2t}$ are independent standard Brownian motions and the model parameters satisfy the conditions $\kappa \geq 0$, $\theta \in [v_{\min},v_{\max}]$, $\sigma >0$, $r \geq 0$, $\rho \in [-1,1]$.\
In their paper, Ackerer et al.~\cite{ackerer2016jacobi} use this model in the context of option pricing where the price of the asset is specified by $S_t \defby e^{Y_t}$ and $V_t$ represents the squared stochastic volatility. In the following, we briefly introduce the pricing technique they propose and  explain how it involves the incremental computation of polynomial moments. 

Under the Jacobi model with the discounted payoff function $f$ of an European claim, 
the option price~\eqref{priceexpectation} at initial time $t=0$ can be expressed as 
\begin{equation}\label{price}
\sum_{n \geq 0} f_n l_n,
\end{equation}
where $\{f_n, n\geq 0\}$ are the Fourier coefficients of $f$ and $\{l_n, n\geq 0\}$ are Hermite moments. As explained in~\cite{ackerer2016jacobi}, the Fourier coefficients can be conveniently computed in a recursive manner. The Hermite moments are computed using~\eqref{condmoments}. Specifically,
consider the monomial basis of $\text{Pol}_n(\R^2)$:
\begin{equation}\label{eq:basisJacobi}
H_n(y,v) \defby (1,y,v,y^2,yv,v^2,\dots,y^n,y^{n-1}v,\dots,v^n)^T.
\end{equation}
Then 
\begin{equation} \label{eq:hermitemoments}
l_n = H_n(Y_0,V_0)^Te^{\tau G_n} \vec{h}_n,
\end{equation}
where $\vec{h}_n$ contains the coordinates with respect to~\eqref{eq:basisJacobi} of 
\begin{equation*}
 \frac{1}{\sqrt{n!}} h_n \left(\frac{y-\mu_w}{\sigma_w}\right),
\end{equation*}
with real parameters $\sigma_w, \mu_w$ and the $n$th Hermite polynomial ${h}_n$.

Truncating the sum \eqref{price} after a finite number of terms allows
one to obtain an approximation of the option price.
Algorithm~\ref{alg:jacobi} describes a heuristic to selecting the truncation based on the absolute value of the summands, using Algorithm~\ref{alg:full} for computing the required moments incrementally. 
\begin{algorithm}[ht]
\caption{Option pricing for the European call option under the Jacobi stochastic volatility model}
\begin{algorithmic}[1]\label{alg:jacobi}
\REQUIRE Model and payoff parameters, tolerance $\epsilon$
\ENSURE Approximate option price
\STATE $n=0$
\STATE Compute $l_0$, $f_0$; set $\text{Price} = l_0 f_0$.
\WHILE{$|l_n f_n| > \epsilon \cdot \text{Price}$}
\STATE $n=n+1$
\STATE \label{line:jacobiexp} Compute $\exp(\tau G_n)$ using Algorithm~\ref{alg:step}.
\STATE \text{Compute Hermite moment $l_n$ using~\eqref{eq:hermitemoments}. }
\STATE \text{Compute Fourier coefficient $f_n$ as described in~\cite{ackerer2016jacobi}.}
\STATE $\text{Price}=\text{Price}+l_n f_n$;
\ENDWHILE
\end{algorithmic}
\end{algorithm}
%In chapter 4 we will show some numerical examples based on algorithm \ref{algJacobi} where we consider the European call option.\

As discussed in Section~\ref{sec:scaling_and_squaring}, a norm estimate for $G_n$ is instrumental for choosing a priori the scaling parameter in the scaling and squaring method. The following lemma provides such an estimate for the model under consideration. 
\begin{lemma} \label{lemmanormJ}
Let $G_n$ be the matrix representation of the operator $\mathcal{G}$ defined in~\eqref{generator}, with respect to the basis~\eqref{eq:basisJacobi} of $\mathrm{Pol}_{n}(\mathbb{R}^2)$.  Define 
\begin{equation*}
\alpha:=\frac{\sigma (1+v_{\min}v_{\max}+v_{\max}+v_{\min})}{2(\sqrt{v_{\max}}-\sqrt{v_{\min}})^2}.
\end{equation*}
Then the matrix 1-norm of $G_n$ is bounded by
\begin{equation*}
n( r + \kappa + \kappa\theta - \sigma \alpha ) + \frac12 n^2 ( 1 + |\rho| \alpha + 2 \sigma \alpha).
\end{equation*}
\begin{proof}
The operator $\mathcal{G}$ in the Jacobi model takes the form
\begin{equation*}
\mathcal{G}f(y,v)=\frac{1}{2}\tr(A(v) \nabla^2 f(y,v))+b(v)^\top \nabla f(y,v),
\end{equation*}
where 
\begin{equation*}
b(v)=\begin{bmatrix}
  r-v/2 \\
  \kappa(\theta -v)
\end{bmatrix},
 \quad 
A(v)=\begin{bmatrix}
  v & \rho \sigma Q(v)\\
 \rho \sigma Q(v) &  \sigma^2 Q(v)
\end{bmatrix}.
\end{equation*} 
Setting $S\defby (\sqrt{v_{\max}}-\sqrt{v_{\min}})^2$, we consider the action of the generator $\mathcal{G}$ on a basis element $y^pv^q$:
\begin{align*}
\mathcal{G} y^p v^q=&y^{p-2} v^{q+1}p \frac{p-1}{2} - y^{p-1}v^{q+1} p \Big(\frac{1}{2}+\frac{q\rho \sigma}{S}\Big)+y^{p-1} v^qp \Big(r+q \rho \sigma \frac{v_{\max} + v_{\min}}{S}\Big)\\
& -y^{p-1}v^{q-1}\frac{pq \rho \sigma v_{\max} v_{\min}}{S} - y^p v^q q \Big(\kappa + \frac{q-1}{2} \frac{\sigma^2}{S} \Big) \\
& - y^p v^{q-2} q \frac{q-1}{2} \frac{\sigma^2 v_{\max}v_{\min}}{S} +y^p v^{q-1} q \Big(\kappa \theta + \frac{q-1}{2} \sigma^2 \frac{ v_{\max} + v_{\min}}{S} \Big). 
\end{align*}
For the matrix 1-norm of $G_n$, one needs to determine the values of $(p,q) \in \mathcal{M}:= \{(p,q) \in \mathbb{N}_0 \times \mathbb{N}_0 | p+q \leq n\}$ for which the $1$-norm of the coordinate vector of $\mathcal{G} y^p v^q$ becomes maximal. Taking into account the nonnegativity of the involved model parameters and replacing $\rho$ by $|\rho|$, we obtain an upper bound as follows:
\begin{align*}
& p \frac{p-1}{2} + p \Big(\frac{1}{2}+\frac{q|\rho|\sigma}{S}\Big)+ p \Big(r+q |\rho|\sigma \frac{v_{\max} + v_{\min}}{S}\Big)+ \frac{pq |\rho|\sigma v_{\max} v_{\min}}{S}\\
&+ q \Big(\kappa + \frac{q-1}{2} \frac{\sigma^2}{S} \Big) + q \frac{q-1}{2} \frac{\sigma^2 v_{\max}v_{\min}}{S} + q \Big(\kappa \theta + \frac{q-1}{2} \sigma^2 \frac{ v_{\max} + v_{\min}}{S} \Big) \\
=& pr + q\kappa(\theta+1) + \frac12 p^2 + 2pq|\rho| \alpha +  q (q-1) \sigma \alpha  \\
\le & n( r + \kappa + \kappa\theta ) + \frac12 n^2 + 2pq|\rho| \alpha + n(n-1)\sigma \alpha. 
\end{align*}
This completes the proof, noting that the maximum of $pq$ on $\mathcal{M}$ is bounded by $n^2 / 4$
over $\mathcal{M}$.
\end{proof}
\end{lemma}

The result of Lemma~\ref{lemmanormJ} predicts that the norm of $G_n$ grows, in general, quadratically. This prediction is confirmed numerically for parameter settings of practical relevance.
% If the scaling parameter is to be chosen by using the 2-norm, above lemma allows us to estimate $\norm{G_n}_2$ by means of the norm inequality
% \begin{equation}\label{estimation}
% \norm{G_n}_2 \leq \sqrt{M} \norm{G_n}_{1},
% \end{equation}
% where $M$ is the size of the matrix $G_n$. Consider for example the set of parameters
% \begin{equation*}
% \kappa=0.5, \quad \theta=0.04, \quad  \sigma=0.15, \quad  \rho=-0.5, \quad  v_{\min}=0.01,\quad  v_{\max}=1,\quad  r=0.
% \end{equation*}
% Figure \ref{fig:normJacobi} shows (left) the norm of the matrices $G_l, l=1,\cdots,n$ and the estimation performed by using the inequality 
% \eqref{estimation}. On the right side, one can see the scaling parameter we would choose for all different values of $l$.
% \begin{figure}[t]
% \centering 
% \includegraphics[height=4.8 cm]{Figures/Norm_estimation_Jacobi}
% \includegraphics[height=4.8 cm]{Figures/scaling_parameters_Jacobi}
% \caption{\textsl{Left:} Norm estimations \eqref{estimation}. \textsl{Right:} Corresponding scaling parameters.}
% \label{fig:normJacobi}
% \end{figure}
%%%%%%%%%%%%%%%%%%%%%%%%%%%%%%%%%%%%%%%%%%%%%%%%%%%%%%%%%%%%
\subsection{Moment-based option pricing for Heston models} \label{sec:hestonx}

The Heston model is another special case of a polynomial diffusion model, characterized by the SDE 
\begin{align*}
&dY_t=(r-V_t/2)d_t + \rho \sqrt{V_t} dW_{1t}+\sqrt{V_t}\sqrt{1-\rho^2}dW_{2t},\\
&dV_t=\kappa(\theta -V_t)dt+\sigma \sqrt{V_t}dW_{1t},
\end{align*}
with model parameters satisfying the conditions $\kappa \geq 0$, $\theta \geq 0$, $\sigma >0$, $r \geq 0$, $\rho \in [-1,1]$. As before, the asset price is modeled via $S_t \defby e^{Y_t}$, while $V_t$ represents the squared stochastic volatility.

Lasserre et al.~\cite{Lasserre2006} developed a general option pricing technique based on moments and semidefinite programming (SDP). In the following we briefly explain the main steps and in which context an incremental computation of moments is needed. In doing so, we restrict ourselves to the specific case of the Heston model and European call options.

Consider the payoff function $f(y) \defby (e^{y}-e^K)^+$ for a certain log strike value $K$. Let $\nu(dy)$ be the $Y_\tau$-marginal distribution of the joint distribution of the random variable $(Y_\tau,V_\tau)$. Define the restricted measures $\nu_1$ and $\nu_2$ as $\nu_1=\nu |_{(- \infty , K]}$ and $\nu_2=\nu |_{[K, \infty)}$. By approximating the exponential in the payoff function with a Taylor series truncated after $n$ terms, the option price~\eqref{priceexpectation} can be written as a certain linear function $L$ in the moments of $\nu_1$ and $\nu_2$, i.e., 
\begin{equation*}
\mathbb{E}[f(Y_\tau)]= L(n,\nu_1^0, \cdots, \nu_1^n,\nu_2^0, \cdots, \nu_2^n),
\end{equation*}
where $\nu_i^m$ represents the $m$th moment of the $i$th measure.\

A lower / upper bound of the option price can then be computed by solving the optimization problems 
 \begin{align} \label{optiproblem}
 SDP_n \defby \left\{
                \begin{array}{ll}
                  \min/ \max \hspace{0.1 cm}&L(n,\nu_1^0, \cdots, \nu_1^n,\nu_2^0, \cdots, \nu_2^n)\\
\text{subject to   }\hspace{0.1 cm}  &\nu_1^j+\nu_2^j=\nu^j, \quad j=0, \cdots, n\\ 
                  &\nu_1 \text{ is a Borel measure on } (- \infty , K],\\
                  &\nu_2 \text{ is a Borel measure on } [K,\infty).\\
		\end{array}
              \right.
\end{align}
Two SDP arise when writing the last two conditions in \eqref{optiproblem} via moment and localizing matrices, corresponding to the so-called truncated Stieltjes moment problem.

Formula \eqref{condmoments} is used in this setting to compute the moments $\nu^j$. Increasing the relaxation order $n$ iteratively allows us to find sharper bounds (this is trivial because increasing $n$ adds more constraints). One stops as soon as the bounds are sufficiently close. Algorithm~\ref{algoSDP} summarizes the resulting pricing algorithm.
\begin{algorithm}[ht]
\caption{Option pricing for European options based on SDP and moments relaxation}
\begin{algorithmic}[1]\label{algoSDP}
\REQUIRE Model and payoff parameters, tolerance $\epsilon$
\ENSURE Approximate option price
\STATE   $n=1$, $\mathrm{gap}=1$
\WHILE{\text{$\mathrm{gap} > \epsilon$}}
\STATE Compute $\exp(\tau G_n)$ using Algorithm~\ref{alg:step}
\STATE Compute moments of order $n$ using \eqref{condmoments}
\STATE Solve corresponding $SDP_n$ to get $LowerBound$ and $UpperBound$ 
\STATE  $\mathrm{gap} = |UpperBound - LowerBound|$
\STATE  $n=n+1$
\ENDWHILE
\end{algorithmic}
\end{algorithm}

The following lemma extends the result of Lemma~\ref{lemmanormJ} to the Heston model.
\begin{lemma}
    \label{lem:heston_est}
Let $G_n$ be the matrix representation of the operator $\mathcal{G}$ introduced above with respect to the basis~\eqref{eq:basisJacobi} of $\mathrm{Pol}_{n}(\mathbb{R}^2)$.  
Then the matrix 1-norm of $G_n$ is bounded by
\begin{equation*}
n( r + \kappa + \kappa\theta - \frac{ \sigma^2}{2} ) + \frac12 n^2 ( 1 + |\rho| \frac{\sigma}{2} + \sigma^2).
\end{equation*}
\begin{proof}
Similar to the proof of Lemma \ref{lemmanormJ}.
\end{proof}
\end{lemma}

%TODO:
%\begin{itemize}
% \item Explain Heston model as special case of polynomial diffusion.
% \item Short paragraph indicating how the polynomial moments are used in option pricing.
% \item Move algorithm from numerical experiments here
% \item Include lemma on block norms of $G$
%\end{itemize}


\chapter{Consensus Message Passing}
\label{chap:cmp}

\tikzset{
  cluster/.style={rectangle, minimum height=0.4cm, minimum width=0.4cm, inner sep=0.05cm, draw},
  mylatent/.style={circle, minimum height=0.5cm, minimum width=0.5cm, inner sep=0.05cm, draw},
  myfactor/.style={rectangle, minimum height=0.05cm, minimum width=0.05cm, fill=black, scale=0.75},
  myinitfactor/.style={rectangle, minimum height=0.1cm, minimum width=0.1cm, fill=red, scale=0.75},
}

\newcommand{\METHOD}{Consensus Message Passing\@\xspace}
\newcommand{\Method}{Consensus message passing\@\xspace}
\newcommand{\method}{consensus message passing\@\xspace}
\newcommand{\MTD}{CMP\@\xspace}

\makeatletter
\renewcommand{\thesubfigure}{\alph{subfigure}}
\renewcommand{\@thesubfigure}{(\thesubfigure)\hskip\subfiglabelskip}
\makeatother

In the last chapter, we have proposed a technique to speed up the sampling process
for inverse graphics. Despite having faster convergence with techniques
like informed sampler, sampling based inference is often too slow for practical applications.
An alternative inference approach in vision, which is often faster, is
message-passing in factor graph models (see Section~\ref{sec:pgm}).
Generative models in computer vision tend to be large, loopy and layered as discussed in Section~\ref{sec:pgm}.
We find that widely-used, general-purpose message passing inference algorithms such as Expectation Propagation (EP) and Variational Message Passing (VMP) fail on the simplest of vision models. With these models in mind, we introduce a modification to message passing that learns to exploit their layered structure by passing \textit{consensus} messages that guide inference towards good solutions. Experiments on a variety of problems show that the proposed technique leads to significantly more accurate inference results, not only when compared to standard EP and VMP, but also when compared to competitive bottom-up discriminative models. Refer to Section~\ref{sec:pgm} for an overview
of factor graphs and message-passing inference.


\section{Introduction}
\label{sec:introduction}

As discussed in Section~\ref{sec:gen-adv-limits}, perhaps, the most significant challenge of the generative modeling framework is that inference can be very hard. Sampling-based methods run the risk of slow convergence, while message passing-based methods (which are the focus of this chapter) can converge slowly, converge to bad solutions, or fail to converge at all. Whilst significant efforts have been made to improve the accuracy of message passing algorithms (\eg by using structured variational approximations), many challenges remain, including difficulty of implementation, the problem of computational cost and the question of how the structured approximation should be chosen. The work
in this chapter aims to alleviate these problems for general-purpose message-passing algorithms.

Our initial observation is that general purpose message passing inference algorithms (\eg EP and VMP; \cite{Minka2001,Winn2005}) fail on even the simplest of computer vision models. We claim that in these models the failure can be attributed to the algorithms' inability to determine the values of a relatively small number of influential variables which we call `global' variables. Without accurate estimation of these global variables, it can be very difficult for message passing to make meaningful progress on the other variables in the model.

Latent variables in vision models are often organized in a layered structure (also discussed in
Section~\ref{sec:pgm}), where the observed image pixels are at the bottom and high-level scene parameters are at
the top. Additionally, knowledge about the values of the variables at level $l$ is sufficient to reason about
any global variable at layer $l+1$. With these properties in mind, we develop a method called \emph{\METHOD}
(\MTD) that learns to exploit such layered structures and estimate global variables during the early stages of
inference.

Experimental results on a variety of problems show that \MTD leads to significantly more accurate inference results while preserving the computational efficiency of standard message passing. The implication of this work is twofold. First, it adds a useful tool to the toolbox of techniques for improving general-purpose inference, and second, in doing so it overcomes a bottleneck that has restricted the use of model-based machine learning in computer vision.

This chapter is organized as follows.~In Section~\ref{sec:related-work-chap4},
we discuss related work and explain our CMP approach in Section~\ref{sec:method-chap4}.
Then we present experimental analysis of
CMP with three diverse generative vision models in Section~\ref{sec:experiments-chap4}.
We conclude with a discussion in Section~\ref{sec:discussion-chap4}.

\section{Related Work}
\label{sec:related-work-chap4}

Inspiration for \MTD stems from the kinds of distinctions that have been made for decades between so-called `intuitive', bottom-up, fast inference techniques, and iterative `rational' inference techniques~\citep{Hinton1990}. \MTD can be seen as an implementation of such ideas in the context of message passing, where the consensus messages form the `intuitive' part of inference and the following standard message passing forms the `rational' part. Analogues to intuitive and rational inference also exist for sampling, where bottom-up techniques are used to compute proposals for MCMC, leading to significant speedup in inference~\citep{tu2002image, stuhlmueller2013nips} (Chapter~\ref{chap:infsampler}).
The works of~\cite{rezende2014stochastic} and~\cite{kingma2013auto} proposed techniques for learning the parameters of both the generative model and the corresponding recognition model.

The idea of `learning to infer' also has a long history. Early examples include~\cite{Hinton1995}, where a dedicated set of `recognition' parameters are learned to drive inference. In more modern instances of such ideas \citep{Munoz2010, Ross2011, Domke2011, shapovalov2013spatial, Munoz2013}, message passing is performed  by a sequence of predictions defined by a graphical model, and the predictors are jointly trained to ensure that the system produces coherent labelings. However, in these techniques, the resulting inference procedure no longer corresponds to the original (or perhaps to any) graphical model. An important distinction of \MTD is that the predictors fit completely within the framework of message passing and final inference results correspond to valid fixed points in the original model of interest.

Finally, we note recent works of~\cite{Heess2013} and~\cite{Eslami2014} that make use of regressors (neural networks and random forests, respectively) to learn to pass EP messages. These works are concerned with reducing the computational cost of computing individual messages and do not make any attempt to change the accuracy or rate of convergence in message passing inference as a whole. In contrast, \MTD learns to pass messages specifically with the aim of reducing the total number of iterations required for accurate inference in a given generative model.


\section{\METHOD}
\label{sec:method-chap4}

\Method exploits the layered characteristic of vision models in order to overcome the aforementioned inference challenges. For illustration, two layers of latent variables of such a model are shown in Fig.~\ref{fig:types-a} using factor graph notation (black). Here the latent variables below ($\mathbf{h}^b = \{ h^b_k\}$) are a function of the latent variables above ($\mathbf{h}^a = \{ h^a_k\}$) and the global variables $p$ and $q$, where $k$ ranges over pixels (in this case $|k|=3$). As we will see in the experiments that follow, this is a recurring pattern that appears in many models of interest in vision. For example, in the case of face modeling, the $\mathbf{h}^a$ variables correspond to the normals $\mathbf{n}_i$, the global variable $p$ to the light vector $\mathbf{l}$, and $\mathbf{h}^b$ to the shading intensities $s_i$ (see Fig.~\ref{fig:shading-model}).

Our reasoning follows a recursive structure. Assume for a moment that in Fig.~\ref{fig:types-a}, the messages from the layer below to the inter-layer factors (blue) are both informative and accurate (\eg\ due to being close to the observed pixels). We will refer to these messages collectively as \textit{contextual messages}. It would be desirable, for purposes of both speed and accuracy, that we could ensure that the messages sent to the layer above ($\mathbf{h}^a$) are also accurate and informative. If we had access to an oracle that could give us the correct belief for the global variables ($p$ and $q$) for the image, we could send accurate initial messages from $p$ and $q$ and then compute informative and accurate messages from the inter-layer factors to the layer above.

In practice, however, we do not have access to such an oracle. In this work we train regressors to \textit{predict} the values of the global variables given all the messages from the layer below. Should this prediction be good enough, the messages to the layer above will be informative and accurate, and the inductive argument will hold for further layers (if any) above in the factor graph. We describe how these regressors are trained in Section~\ref{sec:training-chap4}. The approach consists of the following two components:

\begin{enumerate}

\item Before inference, for each global variable in different layers of the model, we train a regressor to predict some oracle's value for the target variable given the values of all the messages from the layer below (\ie the \textit{contextual messages}, Fig.~\ref{fig:types-a}, blue),

\item During inference, each regressor sends this belief in the form of a \textit{consensus message} (Fig.~\ref{fig:types-a}, red) to its target variable.

\end{enumerate}

In some models, it will be useful to employ a second type of \MTD, illustrated graphically in Fig.~\ref{fig:types-b}, where global layer variables are absent and loops in the graphical model are due to global variables in other layers. In this case, a consensus message is sent to each variable in the latent layer above, given all the contextual messages.

\begin{figure}[t]
	\centering
	\subfigure[Type A]{
		\begin{tikzpicture}
			\matrix at (0, 0.2) [matrix, column sep=0.1cm, row sep=0.21cm,ampersand replacement=\&]
			{
				\node { }; \&
				\node (y1e) { $\vdots$ }; \&
				\node (y2e) { $\vdots$ }; \&
				\node (y3e) { $\vdots$ }; \&
				\node { }; \\

				\node (a) [mylatent] { $p$ }; \&
				\node (y1) [mylatent] { $h^a_1$ }; \&
				\node (y2) [mylatent] { $h^a_2$ }; \&
				\node (y3) [mylatent] { $h^a_3$ }; \&
				\node (b) [mylatent] { $q$ }; \\

				\& \& \& \& \\

				\node { }; \&
				\node (cl1) [myfactor] {  }; \&
				\node (cl2) [myfactor] {  }; \&
				\node (cl3) [myfactor] {  }; \&
				\node { }; \& \\

				\& \& \& \& \\
				\& \& \& \& \\
				\& \& \& \& \\

				\node { }; \&
				\node (x1) [mylatent] { $h^b_1$ }; \&
				\node (x2) [mylatent] { $h^b_2$ }; \&
				\node (x3) [mylatent] { $h^b_3$ }; \&
				\node { }; \\

				\node { }; \&
				\node (x1e) { $\vdots$ }; \&
				\node (x2e) { $\vdots$ }; \&
				\node (x3e) { $\vdots$ }; \&
				\node { }; \\
			};

			\draw[red!20] (-1.9, 0) -- (1.8, 0);
			\draw[red, dotted] (-1.9, 0) -- (1.8, 0);

			\fill (a.south) ++ (0, -1.1) circle (3pt) [fill=red] { };
			\fill (b.south) ++ (0, -1.1) circle (3pt) [fill=red] { };

			\node[yshift=-1.45cm] at (a.south) { $\textcolor{red}{\Delta^p}$ };
			\node[yshift=-1.45cm] at (b.south) { $\textcolor{red}{\Delta^q}$ };

			\draw[-stealth, cyan] (0.1, -0.2) -- (0.1, 0.2);
			\draw[-stealth, cyan] (0.99, -0.2) -- (0.99, 0.2);
			\draw[-stealth, cyan] (-0.8, -0.2) -- (-0.8, 0.2);

			\fill (0.1, 0) circle (1pt) [fill=cyan] { };
			\fill (0.99, 0) circle (1pt) [fill=cyan] { };
			\fill (-0.8, 0) circle (1pt) [fill=cyan] { };

			\draw[-stealth, red] (1.6, 0.25) -- (1.6, 0.9);
			\draw[-stealth, red] (-1.68, 0.25) -- (-1.68, 0.9);

			\draw (y1.north) -- (y1e);
			\draw (y2.north) -- (y2e);
			\draw (y3.north) -- (y3e);

			\draw (a.south) -- (cl1.north);
			\draw (a.south) -- (cl2.north);
			\draw (a.south) -- (cl3.north);
			\draw (b.south) -- (cl1.north);
			\draw (b.south) -- (cl2.north);
			\draw (b.south) -- (cl3.north);
			\draw (y1.south) -- (cl1);
			\draw (y2.south) -- (cl2);
			\draw (y3.south) -- (cl3);
			\draw [->] (cl1) -- (x1.north);
			\draw [->] (cl2) -- (x2.north);
			\draw [->] (cl3) -- (x3.north);

			\draw (x1.south) -- (x1e);
			\draw (x2.south) -- (x2e);
			\draw (x3.south) -- (x3e);

		\end{tikzpicture}
		\label{fig:types-a}
	}
	% \hfill
	\subfigure[Type B]{
		\begin{tikzpicture}

			\matrix at (0, 0.2) [matrix, column sep=0.4cm, row sep=0.217cm,ampersand replacement=\&]
			{
				\node (y1e) { $\vdots$ }; \&
				\node (y2e) { $\vdots$ }; \&
				\node (y3e) { $\vdots$ }; \\

				\node (y1) [mylatent] { $h^a_1$ }; \&
				\node (y2) [mylatent] { $h^a_2$ }; \&
				\node (y3) [mylatent] { $h^a_3$ }; \\

				\& \& \\

				\node (cl1) [myfactor] {  }; \&
				\node (cl2) [myfactor] {  }; \&
				\node (cl3) [myfactor] {  }; \\

				\& \& \\
				\& \& \\
				\& \& \\

				\node (x1) [mylatent] { $h^b_1$ }; \&
				\node (x2) [mylatent] { $h^b_2$ }; \&
				\node (x3) [mylatent] { $h^b_3$ }; \\

				\node (x1e) { $\vdots$ }; \&
				\node (x2e) { $\vdots$ }; \&
				\node (x3e) { $\vdots$ }; \\
			};

			\draw[red!20] (-1.9, 0) -- (1.6, 0);
			\draw[red, dotted] (-1.9, 0) -- (1.6, 0);

			\fill (y1.south) ++ (-0.5, -0.98) circle (3pt) [fill=red] { };
			\fill (y2.south) ++ (-0.5, -0.98) circle (3pt) [fill=red] { };
			\fill (y3.south) ++ (-0.5, -0.98) circle (3pt) [fill=red] { };

			\node[xshift=-0.5cm, yshift=-1.3cm] at (y1.south) { $\textcolor{red}{\Delta^1}$ };
			\node[xshift=-0.5cm, yshift=-1.3cm] at (y2.south) { $\textcolor{red}{\Delta^2}$ };
			\node[xshift=-0.5cm, yshift=-1.3cm] at (y3.south) { $\textcolor{red}{\Delta^3}$ };

			\draw[-stealth, cyan] (0.17, -0.2) -- (0.17, 0.2);
			\draw[-stealth, cyan] (1.35, -0.2) -- (1.35, 0.2);
			\draw[-stealth, cyan] (-1.03, -0.2) -- (-1.03, 0.2);

			\fill (0.17, 0) circle (1pt) [fill=cyan] { };
			\fill (1.35, 0) circle (1pt) [fill=cyan] { };
			\fill (-1.03, 0) circle (1pt) [fill=cyan] { };

			\draw[-stealth, red] (-0.46, 0.23) -- (-0.13, 0.87);
			\draw[-stealth, red] (-1.65, 0.23) -- (-1.32, 0.87);
			\draw[-stealth, red] (0.71, 0.23) -- (1.04, 0.87);

			\draw (y1.north) -- (y1e);
			\draw (y2.north) -- (y2e);
			\draw (y3.north) -- (y3e);

			\draw (y1.south) -- (cl1);
			\draw (y2.south) -- (cl2);
			\draw (y3.south) -- (cl3);
			\draw [->] (cl1) -- (x1.north);
			\draw [->] (cl2) -- (x2.north);
			\draw [->] (cl3) -- (x3.north);

			\draw (x1.south) -- (x1e);
			\draw (x2.south) -- (x2e);
			\draw (x3.south) -- (x3e);

		\end{tikzpicture}
		\label{fig:types-b}
	}
	\mycaption{\Method}{Vision models tend to be large, layered and loopy. (a)~Two adjacent layers of the latent variables of a model of this kind (black). In \MTD, consensus messages (red) are computed from contextual messages (blue) and sent to global variables ($p$ and $q$), guiding inference in the layer. (b)~\Method of a different kind for situations where loops in the graphical model are due to global variables in other layers.}
	\label{fig:types}
\end{figure}

Any message passing schedule can be used subject to the constraint that the consensus messages are given maximum priority within a layer and that they are sent bottom up. Naturally, a consensus message can only be sent after its contextual messages have been computed. It is desirable to be able to ensure that the fixed point (result at convergence) reached under this scheme is also a fixed point of standard message passing in the model. One approach for this is to reduce the certainty of the consensus messages over the course of inference, or to only pass them in the first few iterations. In our experiments we found that even passing consensus messages only in the first iteration led to accurate inference, and therefore we follow this strategy for the remainder of the chapter. It is worth emphasizing that message-passing equations remain unchanged and we used the same scheduling scheme in all our experiments (\ie no need for manual tuning).

It is important to highlight a crucial difference between \method and heuristic \textit{initialization}. In the latter, predictions are made from the \textit{observations} no matter how high up in the hierarchy the target variable is, whereas in \MTD predictions are made using \textit{messages} that are sent from variables immediately below the target variables of interest. The CMP prediction task is much simpler, since the relationship between the target variables and the variables in the layer immediately below is much less complex than the relationship between the target variables and the observations. Furthermore, we know from the layered structure of the model that all relevant information from the observations is contained in the variables in the layer below.  This is because target variables at layer $l+1$ are conditionally independent of all layers $l-1$ and below, given the values of layer $l$.

One final note on the capacity of the regressors. Of course it is true that an infinite capacity regressor can make perfect predictions given enough data (whether using \MTD or heuristic initialization). However, we are interested in practical ways of obtaining accurate results for models of increasing complexity, where lack of capable regressors and unlimited data is inevitable. One important feature of \MTD is that it makes use of predictors in a scalable way, since regressions are only made between adjacent latent layers.

\subsection{Predicting Messages for CMP}
\label{sec:training-chap4}

To recap, the goal is to perform inference in a layered model of observed variables $\obs$ with latent variables $\mathbf{h}$. Each predictor $\Delta^t$ (with target $t$) is a function of a collection of its contextual messages $\mathbf{c} = \{ c_k \}$ (incoming from the latent layer below $\mathbf{h}^b$), that produces the consensus message $m$, \ie $m = \Delta^t(\mathbf{c}).$

We adopt an approach in which we \textit{learn} a function for this task that is parameterized by $\bm{\theta}$, \ie $\overline{m} \equiv \func(\mathbf{c}|\bm{\theta}).$ This can be seen as an instance of the canonical regression task. For a given family of regressors $\func$, the goal of training is to find parameters $\bm{\theta}$ that capture the relationship between context and consensus message pairs $\{ (\mathbf{c}_d, m_d) \}_{d=1...D}$ in some set of training examples.

\subsubsection{Choice of Predictor Training Data}

First we discuss how this training data is obtained. There can be at least three different sources:

\textbf{1.\,\,Beliefs at Convergence.} Standard message passing inference is run in the model for a large number of iterations until convergence and for a collection of different observations $\{ \obs_d \}$. Message passing is scheduled in precisely the same way as it would be if \MTD were present, however no consensus messages are sent. For each observation $\obs_d$, the collection of the marginals of the latent variables in the layer below the predictor ($\mathbf{h}^b_d = \{ h^b_{dk} \}$, (see \eg Fig.~\ref{fig:types-a}) at the \textit{first} iteration of message passing is considered to be the context $\mathbf{c}_d$, and the marginal of the target variable $t$ at the \textit{last} iteration of message passing is considered to be the oracle message $m_d$. The intuition is that during inference on new problems, a predictor trained in this way would send messages that \textit{accelerate}
convergence to the fixed-point that message passing would have reached by itself anyway. This technique is only useful if standard message passing works but is slow.

\textbf{2.\,\,Samples from the Model.} First a collection of samples from the model is generated, giving us both the observation $\obs_d$ and its corresponding latent variables $\mathbf{h}_d$ for each sample. Standard message passing inference is then run on the observations $\{ \obs_d \}$ only for a single iteration. Message passing is scheduled as before. For each observation $\obs_d$, the marginals of the latent variables in the layer below $\mathbf{h}^b_d$ at the \textit{first} iteration of message passing is the context $\mathbf{c}_d$, and the oracle message $m_d$ is considered to be a point-mass centered at the sampled value of the target variable $t$. The intuition is that during inference on new problems, a predictor trained in this way would send messages that guide inference to a fixed-point in which the marginal of the target variable $t$ is close to its sampled value. This technique is useful if standard message passing fails to reach good fixed points no matter how long it is run for.

\textbf{3.\,\,Labelled Data.} As above, except the latent variables of interest $\mathbf{h}_d$ are set from real data instead of being sampled from the model. The oracle message $m_d$ is therefore a point-mass centered at the label provided for the target variable $t$ for observation $\mathbf{X}_d$. The aim is that during inference on new problems, a predictor trained in this way would send messages that guide inference to a fixed-point in which the marginal of the target variable $t$ is close to its labelled value, even in the presence of a degree of model mismatch. We demonstrate each of the strategies in the experiments in Section~\ref{sec:experiments-chap4}.

\subsubsection{Random Regression Forests for CMP}

Our goal is to learn a mapping $\func$ from contextual messages $\mathbf{c}$ to the consensus message $m$ from training data $\{ (\mathbf{c}_d, m_d) \}_{d=1...D}$. This is challenging since the inputs and outputs of the regression problem are both messages (\ie distributions), and special care needs to be taken to account for this fact. We follow closely the methodology of~\cite{Eslami2014}, which use random forests to predict outgoing messages from a factor given the incoming messages to it. Please refer to Section~\ref{sec:forests} for a brief review of random forests.

In approximate message passing (\eg EP~\cite{Minka2001} and VMP~\cite{Winn2005}), messages can be represented using only a few numbers, \eg a Gaussian message can be represented by its natural parameters. We represent the contextual messages $\mathbf{c}$ collectively, in two different ways: first as a concatenation of the parameters of its constituent messages which we refer to as `regression parameterization' and denote by $\mathbf{r}_\textrm{c}$; and second as a vector of features computed on the set which we refer to as `tree parameterization' and denote by $\mathbf{t}_\textrm{c}$. This parametrization typically contains features of the set as a whole (\eg moments of their means). We represent the outgoing message $m$ by a vector of real valued numbers $\mathbf{r}_\textrm{m}$.

\textbf{Prediction Model.} Each leaf node is associated with a subset of the labelled training data. During testing, a previously unseen set of contextual messages represented by $\mathbf{t}_\textrm{c}$ traverses the tree until it reaches a leaf which by construction is likely to contain similar training examples. Therefore, we use the statistics of the data gathered in that leaf to predict the consensus message with a multivariate regression model of the form: $\mathbf{r}_\textrm{m} = \mathbf{W} \cdot \mathbf{r}_\textrm{c} + \epsilon$ where $\epsilon$ is a vector of normal error terms. We use the learned matrix of coefficients $\mathbf{W}$ at test time to make predictions $\overline{\mathbf{r}}_\textrm{m}$ for a given $\mathbf{r}_\textrm{c}$. To recap, $\mathbf{t}_\textrm{c}$ is used to traverse the contextual messages down to leaves, and $\mathbf{r}_\textrm{c}$ is used by a linear regressor to predict the parameters $\mathbf{r}_\textrm{m}$ of the consensus message.

\textbf{Training Objective Function.} Recall the training procedure of random forests from Section~\ref{sec:forests}. Each node is in a tree represents a partition of feature space and
split function in each node is chosen in a greedy manner minimizing a splitting criterion $E$.
A common split criterion, which we also use here, is the sum of data likelihood in the node's
left and right child clusters (see Eq.~\ref{eqn:forest_energy}). We use the `fit residual' as
defined in~\citep{Eslami2014} as the likelihood (model fit) function for optimizing splits at
each node. In other words, this objective function splits the training data at each node in a way that the relationship between the incoming and outgoing messages is well captured by the regression model in each child.

\textbf{Ensemble Model.} During testing, a set of contextual messages simultaneously traverses every tree in the forest from their roots until it reaches their leaves. Combining the predictions into a single forest prediction might be done by averaging the parameters $\overline{\mathbf{r}}_\textrm{m}^t$ of the predicted messages $\overline{m}^t$ by each tree $t$, however this would be sensitive to the chosen parameterization for the messages. Instead we compute the moment average $\overline{m}$ of the distributions $\{ \overline{m}^t \}$ by averaging the first few moments of the predictions across trees, and solving for the distribution parameters which match the averaged moments (see \eg~\cite{Grosse2013}).

\section{Experiments}
\label{sec:experiments-chap4}

We first illustrate the application of \MTD to two diagnostic models: one of circles and a second of squares. We then use the approach to improve inference in a more challenging vision model: intrinsic images of faces. In the first experiment, the predictors are trained on beliefs at convergence, in the second on samples from the model, and in the third on annotated labels, showcasing various use-cases of \MTD. We show that in all cases, the proposed technique leads to significantly more accurate inference results whilst preserving the computational efficiency of message passing. The experiments were performed using Infer.NET~\cite{InferNET2012} with default settings, unless stated otherwise. For random forest predictors, we set the number of trees in each forest to 8.

\subsection{A Generative Model of Circles}
\label{sec:circle}

\begin{figure}[t]
	\centering
	\subfigure[]{
		\setlength\fboxsep{-0.3mm}
		\setlength\fboxrule{0.5pt}
		\parbox[b]{3.1cm}{
			\fbox{\includegraphics[width=\linewidth]{figures/Rotate4_Random_XData_1}} \\
			\vspace{1mm}
			\fbox{\includegraphics[width=\linewidth]{figures/Rotate4_Random_XData_3}}
		}
		\hspace{0.1cm}
		\label{fig:circle-data}
	}
	\subfigure[]{
		\begin{tikzpicture}

			\draw[red!20] (-2.3, 2.5) -- (2, 2.5);
			\draw[red, dotted] (-2.3, 2.5) -- (2, 2.5);

			\node[obs] 											(x)			{$\mathbf{x}_i$}; %
			\node[latent, above=of x, yshift=-2mm]                			(z)			{$\mathbf{z}_i$}; %

			\factor[above=of x]				 					{noise}		{left:Gaussian} {} {}; %
			\factor[above=of z, yshift=10mm] 					{sum}		{left:Sum} {} {}; %

			\node[latent, right=of sum]                			(c)			{$\mathbf{c}$}; %
			\node[latent, above=of sum, yshift=-7mm]   			(p)			{$\mathbf{p}_i$}; %

			\factor[above=of p] 								{circle}	{above:Circle\,\,\,\,\,} {} {}; %
			\factor[above=of c] 								{pc}		{} {} {}; %

			\node[latent, left=of circle]                		(a)			{$a_i$}; %
			\node[latent, right=of circle]                		(r)			{$r$}; %

			\factor[above=of a] 								{pa}		{} {} {}; %
			\factor[above=of r] 								{pr}		{} {} {}; %

			\factoredge {z} 		{noise} 		{x}; %
			\factoredge {p} 		{sum} 			{}; %
			\factoredge {a} 		{circle} 		{p}; %
			\factoredge {r} 		{circle} 		{}; %
			\factoredge {c} 		{sum} 			{z}; %
			\factoredge {} 			{pc} 			{c}; %
			\factoredge {} 			{pa} 			{a}; %
			\factoredge {} 			{pr} 			{r}; %

			\plate {} {(pa) (a) (p) (z) (x)} {}; %

			\fill (c.south) ++ (0, -0.52) circle (3pt) [fill=red] { };

			\draw [-stealth, red] (1.45, 2.65) -- (1.45, 2.95);

			\node[yshift=-0.9cm] at (c.south) { $\textcolor{red}{\Delta^c}$ };)

		\end{tikzpicture}
		\label{fig:circle-model}
	}
	\mycaption{The circle problem}{(a)~Given a sample of points on a circle (black), we wish to infer the circle's center (red) and its radius. Two sets of samples are shown. (b)~The graphical model for this problem.}
	\label{fig:circle}
\end{figure}

\begin{figure}[t]
	\centering
	\subfigure[Center]{
		\includegraphics[width=0.46\linewidth]{figures/Rotate4_Random_Center_Distance_10}
	}
	\subfigure[Radius]{
		\includegraphics[width=0.46\linewidth]{figures/Rotate4_Random_Radius_Distance_10}
	}
	\mycaption{Accelerated inference using \MTD for the circle problem}{(a)~Distance of the mean of the marginal posterior of center $c$ from its true value as a function of number of inference iterations (Forest: direct prediction, MP: standard VMP, \MTD: VMP with consensus). \Method significantly accelerates convergence. (b)~Similar plot for radius $r$. }
	\label{fig:circle-results}
\end{figure}

We begin by studying the behavior of standard message passing on a simplified Gauss and Ceres problem~\citep{Teets1999}.
Given a noisy sample of points $\obs = \{ \mathbf{x}_i \}_{i=1...N}$ on a circle in the 2D plane (Fig.~\ref{fig:circle-data}, black, $\mathcal{N}(0,0.01)$ noise on each axis), the aim is to infer the coordinates of the circle's center $\mathbf{c}$ (Fig.~\ref{fig:circle-data}, red) and its radius $r$. We can express the data generation process using a graphical model (Fig.~\ref{fig:circle-model}). The Cartesian point $(0, r)$ is rotated $a_i$ radians to generate $\mathbf{p}_i$, then translated by $\mathbf{c}$ to generate the latent $\mathbf{z}_i$, which finally produces the noisy observation $\mathbf{x}_i$.  This model can be expressed in a few lines of code in Infer.NET. The circle model is interesting for our purposes ,since it is both layered (the $\mathbf{z}_i$s, $\mathbf{p}_i$s and $a_i$s each form a layer) and loopy (due to the presence of two variables outside the plate).

We use this example to highlight the fact that although inference may require many iterations of message passing, message initialization can have a significant effect on the speed of convergence, and to demonstrate how this can be done automatically using \MTD.

Vanilla message passing inference in this model can take a surprisingly large number of iterations to converge. We draw 10 points $\{ \mathbf{x}_i \}$ from circles with random centers and radii, run VMP and record the accuracy of the marginals of the latent variables at each iteration. We repeat the experiment 50 times and plot results in Fig.~\ref{fig:circle-results} (dashed black). As can be seen from the figure, the marginals contain significant errors even after 50 iterations of message passing.

We then experiment with \method. A predictor $\Delta^c$ is trained to send a consensus message to $\mathbf{c}$ in the initial stages of inference, given the messages coming up from all of the $\mathbf{z}_i$ (indicated graphically in Fig.~\ref{fig:circle-model}, red). The predictor is trained on final beliefs at 100 iterations of standard message passing on $D=500$ sample problems.

As can be seen in Fig.~\ref{fig:circle-results} (red), this single consensus message has the effect of significantly increasing the rate of convergence (as indicated by slope) and also inference robustness (as indicated by error bars). For comparison, we also plot how well a regressor of the same capacity as the one used by \MTD can directly estimate the latent variables without using the graphical model in Fig.~\ref{fig:circle-results} (blue). \Method gives us the best of both worlds in this example: speed that is more comparable to one-shot bottom-up prediction and the accuracy of message passing inference in a good model for the problem.

\begin{figure}[t]
	\centering
	\subfigure[]{
		\setlength\fboxsep{-0.2mm}
		\setlength\fboxrule{0.7pt}
		\parbox[b]{2.3cm}{
			\fbox{\includegraphics[width=0.45\linewidth]{figures/Translate1_XData_1}} \hspace*{0mm}
			\fbox{\includegraphics[width=0.45\linewidth]{figures/Translate1_XData_2}} \vspace*{-2.4mm} \\
			\fbox{\includegraphics[width=0.45\linewidth]{figures/Translate1_XData_3}} \hspace*{0mm}
			\fbox{\includegraphics[width=0.45\linewidth]{figures/Translate1_XData_4}} \vspace*{-2.4mm} \\
			\fbox{\includegraphics[width=0.45\linewidth]{figures/Translate1_XData_5}} \hspace*{0mm}
			\fbox{\includegraphics[width=0.45\linewidth]{figures/Translate1_XData_6}} \vspace*{-2.4mm} \\
			\fbox{\includegraphics[width=0.45\linewidth]{figures/Translate1_XData_7}} \hspace*{0mm}
			\fbox{\includegraphics[width=0.45\linewidth]{figures/Translate1_XData_8}} \vspace*{-2.4mm} \\
			\fbox{\includegraphics[width=0.45\linewidth]{figures/Translate1_XData_9}} \hspace*{0mm}
			\fbox{\includegraphics[width=0.45\linewidth]{figures/Translate1_XData_10}} \vspace*{-2.4mm} \\
			\fbox{\includegraphics[width=0.45\linewidth]{figures/Translate1_XData_11}} \hspace*{0mm}
			\fbox{\includegraphics[width=0.45\linewidth]{figures/Translate1_XData_12}} \vspace*{-2.4mm} \\
			\fbox{\includegraphics[width=0.45\linewidth]{figures/Translate1_XData_13}} \hspace*{0mm}
			\fbox{\includegraphics[width=0.45\linewidth]{figures/Translate1_XData_14}}
		}
		\hspace{0.2cm}
		\label{fig:square-data}
	}
% 	\hfill
	\subfigure[]{
		\begin{tikzpicture}

			\draw[red!20] (-2, 2.7) -- (2, 2.7);
			\draw[red, dotted] (-2, 2.7) -- (2, 2.7);

			\draw[red!20] (-2, 5.5) -- (2, 5.5);
			\draw[red, dotted] (-2, 5.5) -- (2, 5.5);

			\node[obs] 											(x)			{$\mathbf{x}_i$}; %
			\node[latent, above=of x]                			(z)			{$\mathbf{z}_i$}; %

			\factor[above=of x, yshift=1mm] 					{noise}		{left:Gaussian} {} {}; %
			\factor[above=of z, yshift=10mm] 					{gate}		{below left:Gate} {} {}; %

			\node[latent, right=of gate]                		(fg)		{$\mathrm{\mathbf{fg}}$}; %
			\node[latent, left=of gate]                			(bg)		{$\mathrm{\mathbf{bg}}$}; %
			\node[latent, above=of gate, yshift=-5mm]			(s)			{$s_i$}; %

			\factor[above=of s, yshift=10mm] 					{insq}		{below left, yshift=-2mm:Square} {} {}; %
			\factor[above=of fg] 								{pfg}		{} {} {}; %
			\factor[above=of bg] 								{pbg}		{} {} {}; %

			\node[latent, left=of insq]                			(c)			{$\mathbf{c}$}; %
			\node[latent, right=of insq]                		(l)			{$l$}; %
			\node[latent, above=of insq, yshift=-5mm, draw=none](p)			{$p_i$}; %

			\factor[above=of c] 								{pc}		{} {} {}; %
			\factor[above=of l] 								{pr}		{} {} {}; %

			\factoredge {z} 		{noise} 		{x}; %
			\factoredge {s} 		{gate} 			{z}; %
			\factoredge {c} 		{insq} 			{}; %
			\factoredge {l} 		{insq} 			{}; %
			\factoredge {p} 		{insq} 			{s}; %
			\factoredge {fg} 		{gate} 			{}; %
			\factoredge {bg} 		{gate} 			{}; %
			\factoredge {} 			{pfg} 			{fg}; %
			\factoredge {} 			{pbg} 			{bg}; %
			\factoredge {} 			{pc} 			{c}; %
			\factoredge {} 			{pr} 			{l}; %

			\plate {} {(p) (x)} {}; %

			\fill (bg.south) ++ (0, -0.52) circle (3pt) [fill=red] { };
			\fill (fg.south) ++ (0, -0.52) circle (3pt) [fill=red] { };

			\fill (l.south) ++ (0, -0.52) circle (3pt) [fill=red] { };

			\draw [-stealth, red] (-1.45, 2.85) -- (-1.45, 3.15);
			\draw [-stealth, red] (1.45, 2.85) -- (1.45, 3.15);

			\draw [-stealth, red] (1.45, 5.65) -- (1.45, 5.95);

			\node[yshift=-0.9cm] at (fg.south) { $\textcolor{red}{\Delta^{\mathrm{\mathbf{fg}}}}$ };
			\node[yshift=-0.9cm] at (bg.south) { $\textcolor{red}{\Delta^{\mathrm{\mathbf{bg}}}}$ };

			\node[yshift=-0.9cm] at (l.south) { $\textcolor{red}{\Delta^l}$ };)

		\end{tikzpicture}
		\hspace{0.1cm}
		\label{fig:square-model}
	}
	\mycaption{The square problem}{(a)~We wish to infer the square's center and its side length. (b)~A graphical model for this problem. $s_i$ is a boolean variable indicating the square's presence at position $p_i$. Depending on the value of $s_i$, the gate copies the appropriate color ($\mathrm{\mathbf{fg}}$ or $\mathrm{\mathbf{bg}}$) to $\mathbf{z}_i$.}
	\label{fig:square}
\end{figure}

\begin{figure}[t]
	\centering
	\subfigure[Center]{
		\includegraphics[width=0.4\linewidth]{figures/Translate4_Bullseye}
		\label{fig:square-results-bullseye}
	}
	\subfigure[Center]{
		\includegraphics[width=0.4\linewidth]{figures/Translate4_Center_Distance}
		\label{fig:square-results-center}
	}
	\subfigure[Side length]{
		\includegraphics[width=0.4\linewidth]{figures/Translate4_Radius_Distance}
		\label{fig:square-results-radius}
	}
	\subfigure[BG color]{
		\includegraphics[width=0.4\linewidth]{figures/Translate4_BGColor_Distance}
		\label{fig:square-results-bgcolor}
	}
	\mycaption{Robustified inference using \MTD for the square problem}{(a)~Position of inferred centers relative to ground-truth. Image boundaries shown in blue for scale. (b,c,d)~Distance of the mean of the posterior of $\mathbf{c}$, $l$ and $\mathrm{\mathbf{bg}}$ from their true values. \MTD consistently increases inference accuracy. Results have been averaged over 50 different problems. 1 stage \MTD only makes use of the lower predictors $\Delta^\mathrm{\mathbf{fg}}$ and $\Delta^\mathrm{\mathbf{bg}}$.}
	\label{fig:square-results}
\end{figure}


\subsection{A Generative Model of Squares}
\label{sec:square}

Next, we turn our attention to a more challenging problem for which even the best message passing scheme that we could devise frequently finds completely inaccurate solutions. The task is to infer the center $\mathbf{c}$ and side length $r$ of a square in an image (Fig.~\ref{fig:square-data}). Unlike the previous problem where we knew that all points belonged to the circle, here we must first determine which pixels belong to the square and which do not. To do so we might also wish to reason about the color of the foreground $\mathrm{\mathbf{fg}}$ and background $\mathrm{\mathbf{bg}}$, making the task of inference significantly harder. The graphical model for this problem is shown in Fig.~\ref{fig:square-model}. Let $\mathbf{c}$ and $l$ denote square center and side length respectively. At each pixel position $p_i$, $s_i$ is a boolean variable indicating the square's presence. Depending on the value of $s_i$, the gate copies the appropriate color ($\mathrm{\mathbf{fg}}$ or $\mathrm{\mathbf{bg}}$) to $\mathbf{z}_i$.

We experiment with 50 test images (themselves samples from the model), perform inference using EP and with a sequential schedule, recording the accuracy of the marginals of the latent variables at each iteration. We additionally place damping with step size 0.95 on messages from the square factor to the center $\mathbf{c}$. We found these choices led to the best performing standard message passing algorithm. Despite this, we observed inference accuracy to be disappointingly poor (see Fig.~\ref{fig:square-results}). In Fig.~\ref{fig:square-results-bullseye} we see that, for many images, message passing converges to highly inaccurate marginals for the center. The low quality of inference can also be seen in quantitative results of Figs.~\ref{fig:square-results}(b-d).

We implement \MTD predictors at two different layers of the model (see Fig.~\ref{fig:square-model}, red). In the first layer, $\Delta^\mathrm{\mathbf{fg}}$ and $\Delta^\mathrm{\mathbf{bg}}$ send consensus messages to $\mathrm{\mathbf{fg}}$ and $\mathrm{\mathbf{bg}}$ respectively, given the messages coming up from all of the $\mathbf{z}_i$ which take the form of independent Gaussians centered at the appearances of the observed pixels (we use a Gaussian noise model). Therefore $\Delta^\mathrm{\mathbf{fg}}$ and $\Delta^\mathrm{\mathbf{bg}}$ effectively make initial guesses of the values of the foreground and background colors in the image given the observed image. Split features in the internal nodes of the regression forest are designed to test for equality of two randomly chosen pixel positions, and sparse regressors are used at the leaves to prevent overfitting.

In the second layer, $\Delta^l$ sends a consensus message to $l$ given the messages coming up from all of the $s_i$. The messages from $s_i$ take the form of independent Bernoullis indicating the algorithm's current beliefs about the presence of the square at each pixel. Therefore, the predictor's job is to predict the square's side length from this probabilistic segmentation map. Note that it is much easier to implement a regressor to perform this task (effectively one only needs to count) than it is to do so using the original observed image pixels $x_i$. We find these predictors to be sufficient for stable inference and so we do not implement a fourth predictor for $\mathbf{c}$. We experiment with single stage \MTD, where only the lower predictors $\Delta^\mathrm{\mathbf{fg}}$ and $\Delta^\mathrm{\mathbf{bg}}$ are active, and with two stage \MTD, where all three predictors are active. The predictors are trained on $D=500$ samples from the model.

The results of these experiments are shown in Fig.~\ref{fig:square-results}. We observe that \MTD significantly improves the accuracy of inference for the center $\mathbf{c}$ (Figs.~\ref{fig:square-results-bullseye}, \ref{fig:square-results-center}) but also for the other latent variables (Figs.~\ref{fig:square-results-radius}, \ref{fig:square-results-bgcolor}). Note that single stage \MTD appears to be insufficient for guiding message passing to good solutions. Whereas in circle example \MTD accelerated convergence, this example demonstrates how it can make inference possible in models that were outside the capabilities of standard message passing.

\subsection{A Generative Model of Faces}
\label{sec:shading}

\begin{figure}[t]
	\centering
	\subfigure[]{
		\setlength\fboxsep{-0.3mm}
		\setlength\fboxrule{0pt}
		\parbox[b]{3cm}{
			\centering
			\small
			\fbox{\includegraphics[width=1.3cm]{figures/sample_N.png}}\\
			Normal $\{\mathbf{n}_i \}$ \vspace{4.5mm} \\
			\fbox{\includegraphics[width=1.3cm]{figures/sample_S.png}} \\
			Shading $\{ s_i \}$ \vspace{4.5mm} \\
			\fbox{\includegraphics[width=1.3cm]{figures/sample_R.png}} \\
			Reflectance $\{ r_i \}$ \vspace{4.5mm} \\
			\fbox{\includegraphics[width=1.3cm]{figures/sample_X.png}} \\
			Observed image $\{ x_i \}$\\
		}
		\hspace{0.1cm}
		\label{fig:shading-data}
	}
	% \hfill
	\subfigure[]{
		\begin{tikzpicture}

			\draw[red!20] (-2.4, 2.7) -- (2, 2.7);
			\draw[red, dotted] (-2.4, 2.7) -- (2, 2.7);

			\draw[red!20] (-2.4, 5.5) -- (2, 5.5);
			\draw[red, dotted] (-2.4, 5.5) -- (2, 5.5);

			\node[obs] 											(x)			{$x_i$}; %
			\node[latent, above=of x]                			(z)			{$z_i$}; %

			\factor[above=of z, yshift=10mm] 					{times}		{below left: $\times$} {} {}; %

			\node[latent, above=of times, yshift=-5mm]			(s)			{$s_i$}; %

			\factor[above=of x, yshift=1mm] 					{noise1}	{left:Gaussian} {} {}; %

			\node[latent, left=of times]                		(r)			{$r_i$}; %

			\factor[above=of r] 								{pr}		{} {} {}; %

			\factor[above=of s, yshift=10mm] 					{inner}		{left:Product} {} {}; %

			\node[latent, above=of inner, yshift=-5mm] 			(n)			{$\mathbf{n_i}$}; %
			\node[latent, right=of inner]                		(l)			{$\mathbf{l}$}; %

			\factor[above=of l] 								{pl}		{} {} {}; %

			\factor[above=of n] 								{pn}		{} {} {}; %

			\factoredge {z} 		{noise1} 		{x}; %
			\factoredge {s} 		{times} 		{z}; %
			\factoredge {r} 		{times} 		{}; %
			\factoredge {n} 		{inner} 		{s}; %
			\factoredge {l} 		{inner} 		{}; %
			\factoredge {} 			{pn} 			{n}; %
			\factoredge {} 			{pl} 			{l}; %
			\factoredge {} 			{pr} 			{r}; %

			\plate {} {(pn) (x) (r)} {}; %

			\fill (r.south) ++ (0, -0.52) circle (3pt) [fill=red] { };

			\fill (l.south) ++ (0, -0.52) circle (3pt) [fill=red] { };

			\draw [-stealth, red] (-1.45, 2.85) -- (-1.45, 3.15);

			\draw [-stealth, red] (1.45, 5.65) -- (1.45, 5.95);

			\node[yshift=-0.9cm] at (r.south) { $\textcolor{red}{\Delta_i^\mathbf{r}}$ };

			\node[yshift=-0.9cm] at (l.south) { $\textcolor{red}{\Delta^\mathbf{l}}$ };

			\node[yshift=0.35cm, xshift=-0.83cm] at (inner) { Inner };

		\end{tikzpicture}
		\label{fig:shading-model}
	}
	\mycaption{The face problem}{(a)~We observe an image and wish to infer the corresponding reflectance map and normal map (visualized here as 3D shape). (b)~A graphical model for this problem. Symmetry priors not shown.}
	\label{fig:shading}
	\vspace*{-4mm}
\end{figure}

In this sectin, we also investigate a more realistic application: face modeling. The estimation of reflectance and shape from a single image of a human face is a well-studied problem in computer vision (see \eg \citep{Georghiades2001, Lee2005, Wang2009, Kemelmacher2011, Tang2012}). A primary motivation for this task is that reflectance and shape are invariant to confounding light effects, and are therefore useful for downstream tasks such as recognition. The problem is ill-posed and modern approaches make use of prior knowledge in order to obtain good solutions, \eg in the form of average reflectance and normal statistics \citep{Biswas2009, Biswas2010} or morphable 3D models \citep{Zhang2006, Wang2009}.

\textbf{Model.} Given an observation of image pixels $\mathbf{x} = \{x_i\}$, the aim is to infer the reflectance value $r_i$ and normal vector $\mathbf{n_i}$ for each pixel $i$ (see Fig.~\ref{fig:shading-data}). In Fig.~\ref{fig:shading-model}, a model is shown for these variables that represents the following image formation process: $x_i = (\mathbf{n_i} \cdot \mathbf{l}) \times r_i + \epsilon$, thereby assuming Lambertian reflection and an infinitely distant directional light source with variable intensity. We place Gaussian priors over reflectances $\{ r_i \}$, normals $\{ \mathbf{n_i} \}$, and the light $\mathbf{l}$; and set the parameters of the priors using training data. We additionally place a soft symmetry prior on the $\{ r_i \}$ (the reflectance value on one side of the face should be close to its value on the other side) and on the $\{ \mathbf{n_i} \}$ (normal vectors on each side should be approximately symmetric), reflecting our prior knowledge about faces. These symmetry priors can be added to the model in just a few lines of code, illustrating the way in which model-based methods lend themselves to rapid prototyping and experimentation.

Although this model is only a crude approximation to the true image formation process (\eg it does not account for shadows or specularities), similar approximations have been found to be useful in prior work \citep{Biswas2009, Biswas2010, Kemelmacher2011}. Additionally, if we can successfully develop algorithms that perform accurate and reliable inference in this class of models, we would then be able to increase its usefulness by updating it to reflect the true image formation process more accurately. Note that even for a relatively small image of size $96 \times 84$, the model contains over 48,000 latent variables and 56,000 factors, and as we will show below, standard message passing in the model routinely fails to converge to accurate solutions.

\begin{figure*}
\begin{center}
\centerline{\includegraphics[width=1.0\columnwidth]{figures/face_cmp_visual_results.pdf}}
\vspace{-1.2cm}
\end{center}
	\mycaption{A visual comparison of inference results for the face problem}{For 4 randomly chosen test images, we show inference results obtained by competing methods. (a)~Observed images. (b)~Inferred reflectance maps. \textit{GT} is the stereo estimate which we use as a proxy for ground-truth, \textit{BU} is the bottom-up reflectance estimate of Biswas \etal (2009), \textit{MP} refers to standard variational message passing, \textit{Forest} is the consensus prediction and \textit{CMP} is the proposed consensus message passing technique. (c)~The variance of the inferred reflectance estimate produced by \MTD (normalized across rows). High variance regions correlate strongly with cast shadows. (d)~Visualization of inferred light. (e)~Inferred normal maps.}
	\label{fig:shading-qualitative-multiple-subjects}
\end{figure*}

\textbf{Consensus Message Passing.} We use predictors at two levels in the model (see Fig.~\ref{fig:shading-model}) to tackle this problem. The first sends consensus messages to \textit{each} reflectance pixel $r_i$, making it an instance of type B of \MTD as described in Fig.~\ref{fig:types-b}. Here, each consensus message is predicted using information from all the contextual messages from the $z_i$. We denote each of these predictors by $\Delta_i^\mathbf{r}$. The second predictor sends a consensus message to $\mathbf{l}$ using information from all the messages from the $s_i$ and is denoted by $\Delta^\mathbf{l}$. The first level of predictors effectively make a guess of the reflectance image from the denoised observation, and the second layer predictor produces an estimate of the light from the shading image (which is likely to be easier to do than directly from the observation). The reflectance predictors $\{ \Delta_i^\mathbf{r} \}$ are all powered by a single random forest, however the pixel position $i$ is used as a feature that it can exploit to create location specific behaviour. The tree parameterization of the contextual messages $\mathbf{c}$ for use in the reflectance predictor $\Delta_i^\mathbf{r}$ also includes 16 features such as mean, median, max, min and gradients of a $21 \times 21$ patch around the pixel. The tree parameterization of the contextual messages for use in the lighting predictor $\Delta^\mathbf{l}$ consists of means of the mean of the shading messages in $12 \times 12$ blocks. We deliberately use simple features to maintain generality but one could imagine the use of more specialized regressors for maximal performance.

\begin{figure}
	\centering
	\setlength\fboxsep{0.2mm}
	\setlength\fboxrule{0pt}
	\begin{tikzpicture}

		\matrix at (0, 0) [matrix of nodes, nodes={anchor=east}, column sep=-0.05cm, row sep=-0.2cm]
		{
			\fbox{\includegraphics[width=1cm]{figures/sample_3_1_X.png}} &
			\fbox{\includegraphics[width=1cm]{figures/sample_3_1_GT.png}} &
			\fbox{\includegraphics[width=1cm]{figures/sample_3_1_BISWAS.png}}  &
			\fbox{\includegraphics[width=1cm]{figures/sample_3_1_VMP.png}}  &
			\fbox{\includegraphics[width=1cm]{figures/sample_3_1_FOREST.png}}  &
			\fbox{\includegraphics[width=1cm]{figures/sample_3_1_CMP.png}}  &
			\fbox{\includegraphics[width=1cm]{figures/sample_3_1_CMPVAR.png}}
			 \\

			\fbox{\includegraphics[width=1cm]{figures/sample_3_2_X.png}} &
			\fbox{\includegraphics[width=1cm]{figures/sample_3_2_GT.png}} &
			\fbox{\includegraphics[width=1cm]{figures/sample_3_2_BISWAS.png}}  &
			\fbox{\includegraphics[width=1cm]{figures/sample_3_2_VMP.png}}  &
			\fbox{\includegraphics[width=1cm]{figures/sample_3_2_FOREST.png}}  &
			\fbox{\includegraphics[width=1cm]{figures/sample_3_2_CMP.png}}  &
			\fbox{\includegraphics[width=1cm]{figures/sample_3_2_CMPVAR.png}}
			 \\

			\fbox{\includegraphics[width=1cm]{figures/sample_3_3_X.png}} &
			\fbox{\includegraphics[width=1cm]{figures/sample_3_3_GT.png}} &
			\fbox{\includegraphics[width=1cm]{figures/sample_3_3_BISWAS.png}} &
			\fbox{\includegraphics[width=1cm]{figures/sample_3_3_VMP.png}}  &
			\fbox{\includegraphics[width=1cm]{figures/sample_3_3_FOREST.png}}  &
			\fbox{\includegraphics[width=1cm]{figures/sample_3_3_CMP.png}}  &
			\fbox{\includegraphics[width=1cm]{figures/sample_3_3_CMPVAR.png}}
			 \\
	     };


	     \node at (-3.85, -2.0) {\small Observed};
	     \node at (-2.55, -2.0) {\small `GT'};
	     \node at (-1.27, -2.0) {\small BU};
	     \node at (0.0, -2.0) {\small MP};
	     \node at (1.27, -2.0) {\small Forest};
	     \node at (2.55, -2.0) {\small \textbf{CMP}};
	     \node at (3.85, -2.0) {\small Variance};

	\end{tikzpicture}
	\mycaption{Robustness to varying illumination}{Left to right: observed image, photometric stereo estimate (proxy for ground-truth), \cite{Biswas2009} estimate, VMP result, consensus forest estimate, CMP mean, and CMP variance.}
	\label{fig:shading-qualitative-same-subject}
  \vspace{-0.3cm}
\end{figure}

\begin{figure}[t]
	\centering
	\subfigure[Without shadows]{
		\includegraphics[width=0.4\linewidth]{figures/Shading_Recognition_Rate_Synthetic}
	}
	\subfigure[With shadows]{
		\includegraphics[width=0.4\linewidth]{figures/Shading_Recognition_Rate_Real}
	}
	\mycaption{Reflectance inference accuracy demonstrated through recognition accuracy}{\MTD allows us to make use of the full potential of the generative model, thereby outperforming the competitive bottom-up method of \cite{Biswas2009}.}
	\label{fig:shading-quantitative-reflectance}

	\subfigure[Without shadows]{
		\includegraphics[width=0.4\linewidth]{figures/Shading_Light_Angle_Error_Synthetic}
	}
	\subfigure[With shadows]{
		\includegraphics[width=0.4\linewidth]{figures/Shading_Light_Angle_Error_Real}
	}
	\mycaption{Light inference accuracy}{The presence of cast shadows makes the direct prediction task easier, however \MTD is accurate even in their absence.}
	\label{fig:shading-quantitative-light}
\end{figure}

\textbf{Datasets.} We experiment with the `Yale B' and `Extended Yale B' datasets~\citep{Georghiades2001, Lee2005}. Together, they contain images of 38 subjects each with 64 illumination directions. We remove images taken with extreme light angles (azimuth or elevation $\ge 85$ degrees) that are almost entirely in shadow, leaving around 45 images for each subject. Images are down-sampled to $96 \times 84$. There are no ground-truth normals or reflectances for this dataset, however it is common practice to create proxy ground-truths using photometric stereo, which we obtain using the code of~\cite{Queau2013}. We use images from 22 subjects for training and test on the remaining 16 subjects.

\textbf{Results.} We begin by qualitatively assessing the different inference schemes. In Fig.~\ref{fig:shading-qualitative-multiple-subjects} we show inference results for reflectance maps, normal maps and lights that are obtained after 100 iterations of message passing (VMP). For reflectance (Fig.~\ref{fig:shading-qualitative-multiple-subjects}b), we would like inference to produce estimates that match closely the ground-truth produced by photometric stereo (GT). We also display the reflectance estimates produced by the strong baseline (BU) of \cite{Biswas2009} for reference. We note that the baseline achieves excellent accuracy in regions with strong lighting, however it produces blurry estimates in regions under shadow.

As can be seen in Fig.~\ref{fig:shading-qualitative-multiple-subjects}b (MP), standard variational message passing finds solutions that are highly inaccurate with continued presence of illumination and artifacts in areas of cast show. In contrast, inference using \MTD produces artefact-free results that much more closely resemble the stereo ground-truths. Arguably \MTD also improves over the baseline \citep{Biswas2009}, since its estimates are not blurry in regions with cast shadows. This can be attributed to the presence of symmetry priors in the model. Additionally, we note that the variance of the \MTD inference for reflectance (Fig.~\ref{fig:shading-qualitative-multiple-subjects}c) correlates strongly with cast shadows in the observed images (\ie the model is uncertain where it should be) suggesting that in future work it would be fruitful to have the notion of cast shadows explicitly built into the model. Figs.~\ref{fig:shading-qualitative-multiple-subjects}d and \ref{fig:shading-qualitative-multiple-subjects}e show analogous results for lighting and normal maps, and Fig.~\ref{fig:shading-qualitative-same-subject} demonstrates \MTD's ability to robustly infer reflectance maps for images of a single subject taken under varying lighting conditions. More visual results are shown in the supplementary at the end of this thesis
(Figs.~\ref{fig:shading-qualitative-multiple-subjects-supp},~\ref{fig:shading-qualitative-same-subject}).

We use the task of subject recognition (using estimated reflectance) as a quantitative measure of inference accuracy, as it can be difficult to measure in more direct ways (\eg RMSE strongly favors blurry predictions). The reflectance estimate produced by each algorithm is compared to all training subjects' ground-truth reflectances and is assigned the label of its closest match. We have found this evaluation to reflect the quality of inference estimates. Fig.~\ref{fig:shading-quantitative-reflectance} shows the result of this experiment, both for real images and also synthetic images that were produced by taking the stereo ground-truths and adding artificial lighting (but with no cast shadows). We show analogous results for light in Fig.~\ref{fig:shading-quantitative-light}, where error is defined to be the cosine angle distance between the estimated light and the photometric stereo reference. First, we note that standard variational message passing (MP) performs poorly, producing reflectance estimates that are much less useful for recognition than those from \cite{Biswas2009}. Second, we note that \MTD in the same model (both 1 stage and 2 stage versions) produces inferences that are significantly more useful downstream. The horizontal line labelled `Forest' represents the accuracy of the consensus messages without any message passing, showing that the model-based fine-tuning provides a significant benefit. Finally, we highlight the fact that initializing light directly from the image and running message passing (Fig.~\ref{fig:shading-quantitative-reflectance}, Init+MP) leads to worse estimates than \MTD demonstrating the use of layered predictions as opposed to direct predictions from the observations. These results demonstrate that \MTD helps message passing find better fixed points even in the presence of model mis-match (shadows) and make use of the full potential of the generative model.

\section{Discussion and Conclusions}
\label{sec:discussion-chap4}

We have presented \METHOD and shown that it is a computationally efficient technique that can be used to improve the accuracy of message passing inference in a variety of vision models. The crux of the approach is to recognize the importance of global variables, and to take advantage of layered model structures commonly seen in vision to make rough estimates of their values.

The success of \MTD depends on the accuracy of the random forest predictors. The design of forest features is not yet completely automated, but we took care in this work to use generic features that can be applied to a broad class of problems. Our forests are implemented in an extensible manner, and we envisage building a library of them that one can choose from, simply by inspecting the data types of the contextual and target variables.

In future work, we would like to exploit the benefits of the \MTD framework by applying it to more challenging problems from computer vision. Each of the examples in Section~\ref{sec:experiments-chap4} can be extended in various ways, \eg by making considerations for multiple objects, incorporating occlusion in the squares example and cast shadows in the faces example, or by developing more realistic priors. We are also seeking to understand in what other domains the application of our ideas may be fruitful.

More broadly, a major challenge in machine learning is that of enriching models in a scalable way. We continually seek to ask our models to provide interpretations of increasingly complicated, heterogeneous data sources. Graphical models provide an appealing framework to manage this complexity, but the difficulty of inference has long been a barrier to achieving these goals. The \MTD framework takes us one step in the direction of overcoming this barrier.

\section{Summary and future work}


We have presented techniques for scaling and squaring algorithms that
allow for the incremental computation of block triangular matrix
exponentials.  We combined these techniques with an adaptive scaling
strategy that allows for both fast and accurate computation of each
matrix exponential in this sequence (Algorithm~\ref{alg:adaptive}).
For our application in polynomial diffusion models, the run time can
be further reduced by using fixed scaling parameter, determined
through the estimation techniques in Lemmas~\ref{lemmanormJ}
and~\ref{lem:heston_est}.

We observed in our numerical experiments that accurate approximations
to these matrix exponentials can be obtained even for quite small,
fixed scaling parameters.  For the case of two-by-two block triangular
matrices, the results of Dieci and Papini~\cite{Dieci2000,Dieci2001}
support this finding, but an extension of these results to cover a
more general setting would be appreciable.





\chapter{Video Propagation Networks}
\label{chap:vpn}

In this chapter, we leverage the learnable bilateral filters developed in the
previous chapter, and develop a novel neural network architecture for inference
in video data. We focus on the task of propagating information across video
frames. Standard CNNs are poor candidates for filtering video data.
Standard spatial CNNs have fixed receptive fields whereas the video content
changes differently in different videos depending on the type of scene and camera
motion. So, filters with video adaptive receptive fields are better candidates
for video filtering.

Based on this observation, we adapt the bilateral convolution layers (BCL) proposed
in the previous chapter for filtering video data.
By stacking several BCL and standard spatial convolutional
layers, we develop a neural network architecture for video information propagation
which we call `Video Propagation Network' (VPN). We evaluate VPN on different
tasks of video object segmentation and semantic video segmentation
and show increased performance comparing to the best previous task-specific methods,
while having favorable runtime. Additionally we demonstrate our approach on an example
regression task of propagating color in a grayscale video.

\section{Introduction}

% Why information propagation in videos?
We focus on the problem of propagating structured information across video frames in this chapter.
~This problem appears in many forms (e.g., semantic segmentation or depth estimation) and is a pre-requisite for many applications.~An example instance is shown in Fig.~\ref{fig:illustration-vpn}.
Given an accurate object mask for the first frame, the problem is to propagate this mask forward
through the entire video sequence.~Propagation of semantic information through time and video
colorization are other problem instances.

% What are the main challenges?
Videos pose both technical and representational challenges.
The presence of scene and camera motion lead to the difficult association problem of optical flow.
Video data is computationally more demanding than static images. A naive per-frame approach would scale at least linear with frames.
These challenges complicate the use of standard convolutional neural networks (CNNs) for video processing.
As a result, many previous works for video propagation use slow optimization based techniques.


\begin{figure}[t!]
\begin{center}
\centerline{\includegraphics[width=\columnwidth]{figures/teaser_network.pdf}}
  \mycaption{Video Propagation with VPNs} {The end-to-end trained VPN network is composed
  of a bilateral network followed by a standard spatial network and can be used for
  propagating information across frames. Shown here is an example result
  of foreground mask from the 1$^{st}$ frame to other video frames.}
  \label{fig:illustration-vpn}
  \vspace{-0.3cm}
\end{center}
\end{figure}

% What we do in this work? And also briefly about VPNs
We propose a generic neural network architecture that propagates information across
video frames. The main innovation is the use of image adaptive convolutional operations that automatically
adapt to
the video stream content.~This allows the network to adapt to the changing content of the video stream.
It can be applied to several types of information, e.g. labels, colors, etc. and runs online, that is, only requiring current and previous frames.

% Briefly about VPNs
Our architecture is composed of two components (see Fig.~\ref{fig:illustration-vpn}).
A temporal \textit{bilateral network} that performs image-adaptive spatio-temporal dense filtering.
This part allows to connect densely all pixels from current and previous frames and to propagate associated pixel information to the current frame.
The bilateral network allows the specification of a metric between video pixels and allows a straight-forward integration of temporal information.
This is followed by a standard \textit{spatial CNN} on the filter output to refine and predict for the present video frame.
We call this combination a \textit{Video Propagation Network (VPN)}.
In effect we are combining a filtering technique with rather small spatial CNNs which leads to a favorable runtime compared to many previous approaches.

VPNs have the following suitable properties for video processing:

\vspace{-0.5cm}
\paragraph{General applicability:} VPNs can be used for propagating any
type of information content i.e., both discrete
(e.g., semantic labels) and continuous (e.g. color) information across video frames.
\vspace{-0.5cm}
\paragraph{Online propagation:} The method needs no future frames
and so can be used for online video analysis.
\vspace{-0.5cm}
\paragraph{Long-range and image adaptive:} VPNs can efficiently handle a large
number of input frames and are adaptive to the video.
\vspace{-0.5cm}
\paragraph{End-to-end trainable:} VPNs can be trained end-to-end, so they
can be used in other deep network architectures.
\vspace{-0.5cm}
\paragraph{Favorable runtime:} VPNs have favorable runtime in comparison to several current
best methods, also making them amenable for learning with
large datasets.

Empirically we show that VPNs, despite being generic,
perform better or on-par with current best approaches on video object segmentation
and semantic label propagation while being faster.
VPNs can easily be integrated into sequential per-frame approaches
and require only a small fine-tuning step that can be performed separately.

\section{Related Work}
\label{sec:related}

The literature on propagating information across video frames contains a vast and varied number of approaches.
Here, we only discuss those works that are related to our technique and applications.

\vspace{-0.3cm}
\paragraph{General propagation techniques}
Techniques for propagating content across
image or video pixels are predominantly
optimization based or filtering techniques. Optimization
based techniques typically formulate the propagation as an energy minimization problem
on a graph constructed across video pixels or frames.
A classic example is the color propagation technique from~\cite{levin2004colorization} which uses
graph structure that encodes prior knowledge about pixel colors in a local neighborhood.
Although efficient closed-form
solutions~\cite{levin2008closed} exists for certain scenarios,
optimization tends to be slow due to either large graph structures for videos and/or the use of
complex connectivity resulting in the use of iterative optimization schemes. Fully-connected conditional
random fields (CRFs)~\cite{krahenbuhl2012efficient} open a way for incorporating dense
and long-range pixel connections while retaining fast inference.

Filtering techniques~\cite{kopf2007joint,chang2015propagated,he2013guided} aim to propagate
information with the use of image or video filters resulting in fast runtimes compared
to optimization techniques. Bilateral filtering~\cite{aurich1995non,tomasi1998bilateral} is
one of the popular filters for long-range information propagation.
We have already discussed bilateral filtering and its generalization in the previous
chapter. A popular application, that is also discussed in previous chapter, is joint
bilateral up-sampling~\cite{kopf2007joint} that up-samples a low-resolution signal with
the use of a high-resolution guidance image.
Chapter~\ref{chap:bnn} and the works
of~\cite{li2014mean,domke2013learning,zheng2015conditional,schwing2015fully,barron2015bilateral}
showed that one can back-propagate through the bilateral filtering operation for
learning filter parameters (Chapter~\ref{chap:bnn}) or doing optimization in the bilateral
space~\cite{barron2015bilateral,barron2015defocus}.
Recently, several works proposed to do upsampling
in images by learning CNNs that mimics edge-aware filtering~\cite{xu2015deep} or
that directly learns to up-sample~\cite{li2016deep,hui2016depth}.
Most of these works are confined to images and are either not extendible or computationally
too expensive for videos. We leverage some of these previous works and propose a
scalable yet robust neural network based approach for video content propagation.

\vspace{-0.3cm}
\paragraph{Video object segmentation}

Prior work on video object segmentation can be broadly categorized into two types:
Semi-supervised methods that require manual annotation to define what is foreground
object and unsupervised methods that does segmentation completely automatically.
Unsupervised techniques such as
~\cite{faktor2014video,li2013video,lee2011key,papazoglou2013fast,wang2015saliency,
zhang2013video,taylor2015causal,dondera2014interactive}
use some prior information about the foreground objects such as
distinctive motion, saliency etc. And, they typically fail if these
assumptions do not hold in a video.

In this work, we focus on the semi-supervised task of propagating the foreground
mask from the first frame to the entire video. Existing works
predominantly use graph-based optimization frameworks that perform graph-cuts~\cite{boykov2001fast,
boykov2001interactive,shi2000normalized} on video data.
Several of these works~\cite{reso2014interactive,
li2005video,price2009livecut,wang2005interactive,kohli2007dynamic,jain2014supervoxel}
aim to reduce the complexity of graph structure with
clustering techniques such as spatio-temporal superpixels and
optical flow~\cite{tsaivideo}.
Another direction was to estimate correspondence between different frame
pixels~\cite{agarwala2004keyframe,bai2009video,lang2012practical} by using
nearest neighbor fields~\cite{fan2015jumpcut} or optical flow~\cite{chuang2002video}
and then refine the propagated masks with the use of local classifiers.
Closest to our technique are the works of~\cite{perazzi2015fully} and~\cite{marki2016bilateral}.
~\cite{perazzi2015fully} proposed to use fully-connected CRF over the
refined object proposals across the video frames.~\cite{marki2016bilateral} proposed a
graph-cut in the bilateral space. Our approach is similar in the regard that we also
use a bilateral space embedding. Instead of graph-cuts, we learn
propagation filters in the high-dimensional bilateral space with CNNs.
This results in a more generic architecture and allows integration into other deep learning frameworks.

Two contemporary works~\cite{caelles2016one,khoreva2016learning} proposed CNN based
approaches for video object segmentation. Both works rely on fine-tuning a deep network
using the first frame annotation of a given test sequence. This could potentially result
in overfitting to the background.
In contrast, the proposed approach relies only on offline training and thus can be easily adapted
to different problem scenarios.

\vspace{-0.3cm}
\paragraph{Semantic video segmentation}
Earlier methods such as~\cite{brostow2008segmentation,sturgess2009combining} use structure from motion
on video frames to compute geometrical and/or motion features.
More recent works~\cite{ess2009segmentation,chen2011temporally,de2012line,miksik2013efficient,tripathi2015semantic,
kundu2016feature} construct large graphical models on videos and enforce temporal consistency across frames. \cite{chen2011temporally} used dynamic temporal links in their CRF energy formulation.
\cite{de2012line} proposes to use Perturb-and-MAP random field model with spatio-temporal energy terms
based on Potts model and \cite{miksik2013efficient} propagate predictions across time by learning
a similarity function between pixels of consecutive frames.

In the recent years, there is a big leap in the performance of semantic image
segmentation~\cite{long2014fully,chen2014semantic} with the use of CNNs but mostly
applied to images. Recently,~\cite{shelhamer2016clockwork}
proposed to retain the intermediate CNN representations while sliding the image based
CNN across the frames. Another approach, which inspired our work, is to take unary
predictions from CNN and then propagate semantic information across the frames. A recent
prominent approach in this direction is of~\cite{kundu2016feature} which proposes
a technique for optimizing feature spaces for fully-connected CRF.

\section{Video Propagation Networks}
\label{sec:vpn}

We aim to adapt the bilateral filtering operation to predict information forward in time, across video frames.
Formally, we work on a sequence of $n$ (color or grayscale) images $\obs = \{\obs_1, \obs_2, \cdots, \obs_n\}$
and denote with $\target = \{\target_1, \target_2, \cdots, \target_n\}$ a sequence of outputs, one per frame.
Consider as an example, a sequence $\target_1,\ldots,\target_n$ of foreground masks for a moving object in the
scene. Our goal is to develop an online propagation method, that is, a function that has no access to the future frames.
Formally we predict $\target_t$, having observed the video up to frame $t$ and possibly previous $\target_{1,\cdots,t-1}$
\begin{equation}
\mathcal{F}(\target_{t-1}, \target_{t-2}, \cdots; \obs_t, \obs_{t-1}, \obs_{t-2},\cdots) = \target_t.
\end{equation}

\begin{figure*}[t!]
\begin{center}
\centerline{\includegraphics[width=\textwidth]{figures/net_illustration.pdf}}
  \mycaption{Computation Flow of Video Propagation Network} {Bilateral networks (BNN) consist of a series of bilateral filterings interleaved with ReLU non-linearities. The filtered information from BNN is then passed into a spatial network (CNN) which refines the features with convolution layers interleaved with ReLU
  non-linearities, resulting in the prediction for the current frame.}
  \label{fig:net_illustration}
\end{center}
\end{figure*}


If training examples $(\obs,\target)$ with full or partial knowledge of $\target$ are available, it is possible to learn $\mathcal{F}$ and for a complex and unknown relationship between input and output, a deep CNN is a natural design
choice. However, any learning based method has to face the main challenge: the scene and camera motion and its
effect on $\target$. Since no motion in two different videos is the same, fixed-sized static receptive fields of CNN units are insufficient.
We propose to resolve this with video-adaptive convolutional component, an adaption of the bilateral filtering to videos.
Our Bilateral Network (Section~\ref{sec:bilateralnetwork}) has a connectivity that adapts to video sequences, its output is then fed into a common Spatial Network (Section~\ref{sec:spatialcnn}) that further refines the desired output.
The combined network layout of this Video Propagation Network is depicted in Fig.~\ref{fig:net_illustration}.
It is a sequence of learnable bilateral and spatial filters that is efficient, trainable end-to-end and adaptive to the video input.

\begin{figure*}[t!]
\begin{center}
  \centerline{\includegraphics[width=\textwidth]{figures/permutohedral_illustration_2.pdf}}
    \mycaption{Schematic of Fast Bilateral Filtering for Video Processing}
    {Mask probabilities from previous frames $V_{1,\cdots,t-1}$ are splatted on to the
    lattice positions defined by the image features $f_{I_{1}},f_{I_2},\cdots,f_{I_{t-1}}$.
    The splatted result is convolved with a $1 \times 1$ filter $B$, and the filtered
    result is sliced back to the original image space to get $V_t$ for the present frame.
    Input and output need not be $V_t$, but can also be an intermediate neural network representation.
    $B$ is learned via back-propagation through these operations.}
    \label{fig:filter_illustration}
\end{center}
\end{figure*}

\subsection{Bilateral Network (BNN)}\label{sec:bilateralnetwork}
In this section, we describe the extension of the learnable bilateral filtering, proposed in
Chapter~\ref{chap:bnn} to video data.
Several properties of bilateral filtering make it a perfect candidate for information propagation in videos.
In particular, our method is inspired by two main ideas that we extend in this work: joint bilateral up-sampling~\cite{kopf2007joint} and learnable bilateral filters (Chapter~\ref{chap:bnn}). Although,
bilateral filtering has been used for filtering video data before~\cite{paris2008edge},
its use has been limited to fixed filter weights (say, Gaussian).

{\bf Fast Bilateral Up-sampling across Frames} The idea of joint bilateral up-sampling~\cite{kopf2007joint}
is to view up-sampling as a filtering operation.
A high resolution guidance image is used to up-sample a low-resolution result.
In short, a smaller number of input points $\target_i$ and the corresponding features $\f_i$
are given $\target_i,\f_i; i=1,\ldots,N_{in}$, for example a segmentation result $\target_i$ at a lower resolution.
This is then scaled to a larger number of output points $\f_j;j=1,\ldots,N_{out}$ using the
bilateral filtering operation, that is to compute the following bilateral filtering equation:

\begin{equation}
  \target'_i = \sum_{j=1}^n \bw_{\f_i,\f_j} \target_j
  \label{eq:bilateral2}
\end{equation}

where the sum runs over all $N_{in}$ points and the output is computed for all $N_{out}$ positions.
We will use this idea to propagate content from previous frames to the current frame (all of which have the same dimensions), using the current frame as a guidance image.
This is illustrated in Fig.~\ref{fig:filter_illustration}. We take all previous frame results $\target_{1,\cdots,t-1}$
 and splat them into a lattice using the features computed on video frames $\obs_{1,\cdots,t-1}$.
A filtering (described below) is applied to every lattice point and the result is then sliced back using the
current frame $\obs_t$.
This result need not be the final $\target_t$, in fact we compute a filter bank of responses and continue with further
processing as will be discussed.

For videos, we need to extend bilateral filtering to temporal data, and there are two natural choices.
First, one can simply attach a frame index $t$ as an additional time dimension to the input data, yielding a six dimensional feature vector $\f=(x,y,r,g,b,t)^{\top}$ for every pixel in every frame.
The summation in Eq.~\ref{eq:bilateral2} now runs over \emph{all} previous frames and pixels.
Imagine a video where an object moves to reveal some background.
Pixels of the object and background will be close spatially $(x,y)$ and temporally $(t)$ but likely be of different color $(r,g,b)$.
Therefore they will have no strong influence on each other (being splatted to distant positions in the six-dimensional bilateral space).
In summary, one can understand the filter to be adaptive to color changes across frames, only pixels that are static and have similar color have a strong influence on each other (end up nearby in the lattice space).
The second possibility is to use optical flow. If the perfect flow is available, the video frames could be warped into a common frame of reference. This would resolve the corresponding problem and make information propagation much easier.
We can make use of an optical flow estimate by warping pixel positions $(x,y)$ by their displacement vector $(u_x,u_y)$ to $(x+u_x,y+u_y)$.

Another property of permutohedral filtering that we exploit
is that the \emph{inputs points need not lie on a regular grid} since the filtering
is done in the high-dimensional lattice. Instead of splatting millions of pixels on to the
lattice, we randomly sample or use superpixels and perform filtering using these sampled
points as input to the filter. In practice, we observe that this results in big computational
gains with minor drop in performance (more in Sec.~\ref{sec:videoseg}).

{\bf Learnable Bilateral Filters} The property of propagating information forward using a guidance image through filtering solves the problem of pixel association.
But a Gaussian filter may be insufficient and further, we would like to increase the capacity by using a filter bank instead of a single fixed filter.
We propose to use the technique proposed in previous chapter
to learn the filter values in the permutohedral lattice using back-propagation.

The process works as follows.
A input video is used to determine the positions in the bilateral space to splat the input points
$\target(i)\in\mathbb{R}^D$ i.e. the features $\f$ (e.g. $(x,y,r,g,b,t)$) define the splatting matrix $S_{splat}$.~This leads to a number of vectors $\target_{splatted} = S_{splat}\target$, that lie on the permutohedral lattice, with dimensionality $\target_{splatted}\in\mathbb{R}^D$.
In effect, the splatting operation groups points that are close together, that is, they have similar $\f_i,\f_j$.
All lattice points are now filtered using a filter bank $B\in\mathbb{R}^{F\times D}$ which results in $F$ dimensional vectors on the lattice points.
These are sliced back to the $N_{out}$ points of interest (present video frame).
The values of $B$ are learned by back-propagation.
General parameterization of $B$ from previous chapter allows to have
any neighborhood size for the filters. Since constructing the neighborhood structure in
high-dimensions is time consuming, we choose to use $1 \times 1$ filters for speed reasons.
This makes up one \emph{Bilateral Convolution Layer (BCL)} which we will stack and concatenate to form a Bilateral Network. See Fig.~\ref{fig:filter_illustration} for an illustration of a BCL.

{\bf BNN Architecture} The Bilateral Network (BNN) is illustrated in the green box of
Fig.~\ref{fig:net_illustration}.
The input is a video sequence $\obs$ and the corresponding predictions $\target$ up to frame
$t$. Those are filtered using two BCLs with $32$ filters each.
For both BCLs, we use the same features $\f$ but scale them with different diagonal matrices
$\f_a=\Lambda_a\f,\f_b=\Lambda_b\f$. The feature scales are found by cross-validation.
The two $32$ dimensional outputs are concatenated, passed through a ReLU non-linearity and passed to a
second layer of two separate BCL filters that uses same feature spaces $\f_a,\f_b$.
The output of the second filter bank is then reduced using a $1\times 1$ spatial filter (C-1) to map to
the original dimension of $\target$.
We investigated scaling frame inputs with an exponential time decay and found that, when processing
frame $t$, a re-weighting with $(\alpha \target_{t-1}, \alpha^2 \target_{t-2}, \alpha^3 \target_{t-3} \cdots)$ with
$0\le\alpha\le 1$ improved the performance a little bit.

In the experiments, we also included a simple BNN variant,
where no filters are applied inside the permutohedral space, just splatting and slicing
with the two layers $BCL_a$ and $BCL_b$ and adding the results.
We will refer to this model as \emph{BNN-Identity},
it corresponds to an image adaptive smoothing of the inputs $\target$.
We found this filtering to have a positive effect and include it as a baseline in our experiments.

\vspace{-0.1cm}
\subsection{Spatial Network}\label{sec:spatialcnn}

The BNN was designed to propagate the information from the previous frames, respecting the scene and object motion.
We then add a small spatial CNN with 3 layers, each with $32$ filters of size $3\times 3$,
interleaved with ReLU non-linearities.
The final result is then mapped to the desired output of $\target_t$ using a $1\times 1$
convolution.
The main role of this spatial CNN is to refine the information in frame $t$.
Depending on the problem and the size of the available training data, other network designs are
conceivable. We use the same network architecture shown in Fig.~\ref{fig:net_illustration}
for all the experiments to demonstrate the generality of VPNs.

\vspace{-0.1cm}
\section{Experiments}
\label{sec:exps}

We evaluated VPN on three different propagation tasks: foreground masks, semantic
labels and color information in videos. Our implementation runs in Caffe~\cite{jia2014caffe} using standard settings. We used Adam~\cite{kingma2014adam} stochastic optimization for training VPNs, multinomial-logistic loss for label propagation networks and Euclidean loss for training
color propagation networks. Runtime computations were performed using a
Nvidia TitanX GPU and a 6 core Intel i7-5820K CPU clocked at 3.30GHz machine.
We will make available all the code and experimental results.

\subsection{Video Object Segmentation}
\label{sec:videoseg}

The task of class-agnostic video object segmentation aims to segment foreground objects in
videos. Since the semantics of the foreground object is not pre-defined, this problem is
usually addressed in a semi-supervised manner. The goal is to propagate a
given foreground mask of the first frame to the entire video frames.
Object segmentation in videos is useful for several high level tasks such
as video editing, summarization, rotoscoping etc.

\vspace{-0.5cm}
\paragraph{Dataset} We use the recently published DAVIS dataset~\cite{Perazzi2016}
for experiments on this task. The
DAVIS dataset consists of 50 high-quality (1080p resolution) unconstrained videos
with number of frames in each video ranging from 25 to 104. All the frames come with
high-quality per-pixel annotation of the foreground object. The videos for this
dataset are carefully chosen to contain motion blur, occlusions, view-point changes
and other occurrences of object segmentation challenges.
For robust evaluation and to get results on all the dataset videos,
we evaluate our technique using 5-fold cross-validation.
We randomly divided the data into
5 folds, where in each fold, we used 35 images for training, 5 for validation and
the remaining 10 for the testing. For the evaluation, we used the 3 metrics that
are proposed in~\cite{Perazzi2016}: Intersection over Union (IoU) score, Contour
accuracy ($\mathcal{F}$) score and temporal instability ($\mathcal{T}$) score. The widely
used IoU score is defined as $TP/(TP+FN+FP)$, where TP: True positives; FN: False negatives
and FP: False positives. Please refer to~\cite{Perazzi2016} for the definition of the contour
accuracy and temporal instability scores. We are aware of some other datasets for
this task such as JumpCut~\cite{fan2015jumpcut} and
SegTrack~\cite{tsai2012motion}, but we note that the number of videos in these datasets is too
small for a learning based approach.

\begin{table}[t]
    % \scriptsize
    % \fontsize{5}{3.2}\selectfont
    \centering
    \begin{tabular}{p{3.0cm}>{\centering\arraybackslash}p{1.2cm}>{\centering\arraybackslash}
      p{1.2cm}>{\centering\arraybackslash}p{1.2cm}>{\centering\arraybackslash}p{1.2cm}>{\centering\arraybackslash}p{1.2cm}
      >{\centering\arraybackslash}p{0.6cm}}
        \toprule
        \scriptsize
        & Fold-1 & Fold-2 & Fold-3 & Fold-4 & Fold-5 & All\\ [0.1cm]
        \midrule
        BNN-Identity & 56.4 & 74.0 & 66.1 & 72.2 & 66.5 & 67.0 \\
        VPN-Stage1 & 58.2 & 77.7 & 70.4 & 76.0 & 68.1 & 70.1 \\
        VPN-Stage2 & \textbf{60.9} & \textbf{78.7} & \textbf{71.4} & \textbf{76.8} & \textbf{69.0} & \textbf{71.3} \\

        \bottomrule
        \\
    \end{tabular}
    \mycaption{5-Fold Validation on DAVIS Video Segmentation Dataset}
    {Average IoU scores for different models on the 5 folds.}
    \label{tbl:davis-folds}
\end{table}

\begin{table}[t]
    % \scriptsize
    % \fontsize{5}{3.2}\selectfont
    \centering
    \begin{tabular}{p{3.0cm}>{\centering\arraybackslash}p{1.2cm}>{\centering\arraybackslash}
      p{1.2cm}>{\centering\arraybackslash}p{1.2cm}>{\centering\arraybackslash}p{2.3cm}}
      \toprule
      \scriptsize
      & \textit{IoU$\uparrow$} & $\mathcal{F}\uparrow$ & $\mathcal{T}\downarrow$ & \textit{Runtime}(s) \\ [0.1cm]
      \midrule
      BNN-Identity & 67.0 & 67.1 & 36.3 & 0.21\\
      VPN-Stage1 & 70.1 & 68.4 & 30.1 & 0.48\\
      VPN-Stage2 & 71.3 & 68.9 & 30.2 & 0.75\\
      \midrule
      \multicolumn{4}{l}{\emph{With pre-trained models}} & \\
      DeepLab & 57.0 & 49.9 & 47.8 & 0.15 \\
      VPN-DeepLab & \textbf{75.0} & \textbf{72.4} & 29.5 & 0.63 \\
      \midrule
      OFL~\cite{tsaivideo} & 71.1 & 67.9 & 22.1 & $>$60\\
      BVS~\cite{marki2016bilateral} & 66.5 & 65.6 & 31.6 &  0.37\\
      NLC~\cite{faktor2014video} & 64.1 & 59.3 & 35.6 & 20\\
      FCP~\cite{perazzi2015fully} & 63.1 & 54.6 & 28.5 & 12\\
      JMP~\cite{fan2015jumpcut} & 60.7 & 58.6 & \textbf{13.2} & 12\\
      HVS~\cite{grundmann2010efficient} & 59.6 & 57.6 & 29.7 & 5\\
      SEA~\cite{ramakanth2014seamseg} & 55.6 & 53.3 & 13.7 & 6\\
      \bottomrule
        \\
    \end{tabular}
    \mycaption{Results of Video Object Segmentation on DAVIS dataset}
    {Average IoU score, contour accuracy ($\mathcal{F}$),
    temporal instability ($\mathcal{T}$) scores, and average runtimes (in seconds)
    per frame for different VPN models along with recent published
    techniques for this task. VPN runtimes also include superpixel computation (10ms).
    Runtimes of other methods are taken from~\cite{marki2016bilateral,perazzi2015fully,tsaivideo}
    and only indicative and are not directly comparable to our runtimes. Runtime of VPN-Stage1 includes
    the runtime of BNN-Identity which is in-turn included in the runtime of VPN-Stage2. Runtime
    of VPN-DeepLab model includes the runtime of DeepLab.}
    \label{tbl:davis-main}
    \vspace{-0.7cm}
\end{table}

\begin{figure}[t!]
\begin{center}
  \centerline{\includegraphics[width=0.5\columnwidth]{figures/acc_points_plots.pdf}}
    \mycaption{Random Sampling of Input Points vs. IoU}
    {The effect of randomly sampling points from input video frames on object
    segmentation IoU of BNN-Identity on DAVIS dataset.
    The points sampled are out of $\approx$2 million points from the previous 5 frames.}
    \label{fig:acc_vs_points}
\end{center}
\vspace{-0.8cm}
\end{figure}

\begin{figure}[th!]
\begin{center}
  \centerline{\includegraphics[width=0.65\columnwidth]{figures/video_seg_visuals.pdf}}
    \mycaption{Video Object Segmentation}
    {Shown are the different frames in example videos with the corresponding
    ground truth (GT) masks, predictions from BVS~\cite{marki2016bilateral},
    OFL~\cite{tsaivideo}, VPN (VPN-Stage2) and VPN-DLab (VPN-DeepLab) models.}
    \label{fig:video_seg_visuals}
\end{center}
\vspace{-1.0cm}
\end{figure}

\vspace{-0.5cm}
\paragraph{VPN and Results} In this task, we only have access to foregound mask
for the first frame $V_1$.
For the ease of training VPN, we obtain initial set of predictions with
\emph{BNN-Identity}. We sequentially apply \emph{BNN-Identity} at each frame
and obtain an initial set of foreground masks for the entire video.
These BNN-Identity propagated masks are then used as inputs to train a VPN to
predict the refined masks at each frame. We refer to this
VPN model as \emph{VPN-Stage1}. Once VPN-Stage1 is trained, its refined training
mask predictions are in-turn used as inputs to train another VPN model which we
refer to as \emph{VPN-Stage2}. This resulted in further refinement of foreground
masks. Training further stages did not result in any improvements.

Following the recent work of~\cite{marki2016bilateral} on video object segmentation,
we used scaled features $\f=(x,y,Y,Cb,Cr,t)$ with YCbCr color features for bilateral filtering.
To be comparable with the one of the fastest state-of-the-art technique~\cite{marki2016bilateral},
we do not use any optical flow information. First, we analyze the performance of BNN-Identity by changing the number of randomly sampled input points. Figure~\ref{fig:acc_vs_points} shows how the segmentation IoU changes with the increase
in the number of sampled points (out of 2 million points) from the previous frames.
The IoU levels out after sampling 25\% of points. For
further computational efficiency, we used superpixel sampling instead of random
sampling. Usage of superpixels reduced the IoU slightly (0.5\%), while reducing the
number of input points by a factor of 10 in comparison to a large number
of randomly sampled points. We used 12000 SLIC~\cite{achanta2012slic} superpixels from each frame
computed using the fast GPU implementation from~\cite{gSLICr_2015}. For predictions
at each frame, we input mask probabilities of previous 9 frames into VPN as we observe
no significant improvements with more frames. We set $\alpha$ to $0.5$ and the
feature scales for bilateral filtering are presented in Tab.~\ref{tbl:parameters_supp}.

Table~\ref{tbl:davis-folds} shows the IoU scores for each of the 5 folds and
Tab.~\ref{tbl:davis-main} shows the overall scores and runtimes of different VPN
models along with the best performing segmentation techniques.
The performance improved consistently across all 5 folds with the addition of new VPN stages.~BNN-Identity already performed reasonably well.
And with 1-stage and 2-stage VPNs, we outperformed the present fastest
BVS method~\cite{marki2016bilateral} by a significant margin on all
the performance measures of IoU, contour accuracy and temporal instability scores,
while being comparable in runtime. We perform marginally better than OFL method~\cite{tsaivideo}
while being at least 80$\times$ faster and OFL relies on optical flow whereas we
obtain similar performance without using any optical flow.
~Further, VPN has the advantage of doing online processing
as it looks only at previous frames whereas BVS processes entire video at once.
One can obtain better VPN performance with using better superpixels and
also incorporating optical flow, but this increases runtime as well.
Figure~\ref{fig:video_seg_visuals} shows some qualitative results and more are present
in Figs.~\ref{fig:video_seg_pos_supp}. A couple of
failure cases are shown in Fig.~\ref{fig:video_seg_neg_supp}. Visual results indicate that learned VPN is able to retain foreground masks even with large variations in viewpoint and object size.

\paragraph{Augmenation of Pre-trained Models:} One of the main advantages of the proposed
VPN architecture is that it is end-to-end trainable and can be easily integrated into
other deep neural network architectures. To demonstrate this, we augmented VPN architecture
with standard DeepLab segmentation architecture from~\cite{chen2014semantic}.
We replaced the last classification layer of DeepLab-LargeFOV model
from~\cite{chen2014semantic} to output 2 classes (foreground and background)
in our case and bi-linearly up-sampled the resulting low-resolution probability map to
the original image dimension. 5-fold fine-tuning of the DeepLab model on DAVIS dataset
resulted in the IoU of 57.0 and other scores are shown in Tab.~\ref{tbl:davis-main}.
Then, we combine the VPN and DeepLab models in the following way: The output from
the DeepLab network and the bilateral network are concatenated and then passed on to the spatial network.
In other words, the bilateral network propagates label information from previous frames to the present
frame, whereas the DeepLab network does the prediction for the present frame. The results
of both are then combined and refined by the spatial network in the VPN architecture.
We call this `VPN-DeepLab' model. We trained this model end-to-end and observed big
improvements in performance. As shown in Tab.~\ref{tbl:davis-main}, the VPN-DeepLab
model has the IoU score of 75.0 and contour accuracy score of 72.4 resulting
in significant improvements over the published results. Since DeepLab has also fast
runtime, the total runtime of VPN-DeepLab is only 0.63s which makes this also one of
the fastest video segmentation systems. A couple of visual results of
VPN-DeepLab model are shown in Fig.~\ref{fig:video_seg_visuals}
and more are present in
Figs.~\ref{fig:video_seg_pos_supp} and~\ref{fig:video_seg_neg_supp}.

\vspace{-0.2cm}
\subsection{Semantic Video Segmentation}

A semantic video segmentation assigns a semantic label to every video pixel.
Since the semantics between adjacent frames does not change
radically, intuitively, propagating semantic information across frames should improve
the segmentation quality of each frame. Unlike mask propagation in the previous
section where the ground-truth mask for the first frame is given, we approach
semantic video segmentation in a fully automatic fashion. Specifically, we start
with the unary predictions of standard CNNs and use VPN for propagating semantics across the frames.

\begin{table}[t]
    % \scriptsize
    % \fontsize{5}{3.2}\selectfont
    \centering
    \begin{tabular}{p{5.0cm}>{\centering\arraybackslash}p{2.4cm}>{\centering\arraybackslash}p{3.5cm}}
        \toprule
        \scriptsize
        & \textit{IoU} & \textit{Runtime}(s) \\ [0.1cm]
        \midrule
        CNN from ~\cite{yu2015multi} & 65.3 & 0.38\\
        + FSO-CRF~\cite{kundu2016feature} & 66.1 & \textbf{$>$}10\\
        + BNN-Identity  & 65.3 & 0.31\\
        + BNN-Identity-Flow  & 65.5 & 0.33\\
        + VPN (Ours) & 66.5 & 0.35\\
        + VPN-Flow (Ours) & \textbf{66.7} & 0.37\\
        \midrule
        CNN from ~\cite{richter2016playing} & 68.9 & 0.30\\
        + VPN-Flow (Ours) & \textbf{69.5} & 0.38\\
        \bottomrule
        \\
    \end{tabular}
    \mycaption{Results of Semantic Segmentation on the CamVid Dataset}{
    Average IoU and runtimes (in seconds)
    per frame of different models on \textit{test} split.
    Runtimes exclude CNN computations which are shown separately.
    VPN and BNN-Identity runtimes include superpixel computation which
    takes up large portion of computation time (0.23s).}
    \label{tbl:camvid}
    \vspace{-0.5cm}
\end{table}

\vspace{-0.4cm}
\paragraph{Dataset} We use the CamVid dataset~\cite{brostow2009semantic} that contains 4 high
quality videos captured at 30Hz while the semantically labelled 11-class ground truth is
provided at 1Hz. While the original dataset comes at a resolution of 960$\times$720, similar to
previous works~\cite{yu2015multi,kundu2016feature}, we operate on a resolution of 640$\times$480.
We use the same splits proposed in~\cite{sturgess2009combining} resulting in
367, 100 and 233 frames with ground-truth for training, validation and testing.
Following common practice, we report the IoU scores for evaluation.

\begin{figure}[th!]
\begin{center}
  \centerline{\includegraphics[width=0.9\columnwidth]{figures/semantic_visuals.pdf}}
    \mycaption{Semantic Video Segmentation}
    {Input video frames and the corresponding ground truth (GT)
    segmentation together with the predictions of CNN~\cite{yu2015multi} and with
    VPN-Flow.}
    \label{fig:semantic_visuals}
\end{center}
\vspace{-0.7cm}
\end{figure}

\vspace{-0.5cm}
\paragraph{VPN and Results} Since we already have CNN predictions for every
frame, we train a VPN that takes the CNN predictions of previous \emph{and} present
frames as input and predicts the refined predictions for the present frame.
We compare with the state-of-the-art CRF approach for this problem~\cite{kundu2016feature}
which we refer to as `FSO-CRF'. Following~\cite{kundu2016feature}, we also experimented with
optical flow in our framework and refer that model as \emph{VPN-Flow}.
We used the fast optical flow method that uses dense inverse search
~\cite{kroeger2016fast} to compute flows and modify the positional features of previous frames.
We used the superpixels method of Dollar et al.~\cite{DollarICCV13edges} for this dataset as
gSLICr~\cite{gSLICr_2015} has introduced artifacts.

We experimented with predictions from two different CNNs:
One is with dilated convolutions~\cite{yu2015multi} (CNN-1) and another one~\cite{richter2016playing} (CNN-2)
is trained with the additional data obtained from a video game,
which is the present state-of-the-art on this dataset.
For CNN-1 and CNN-2, using 2 and 3 previous frames respectively as input
to VPN is found to be optimal. Other parameters of the bilateral network are presented
in Tab.~\ref{tbl:parameters_supp}. Table~\ref{tbl:camvid} shows quantitative results on this dataset.
Using BNN-Identity only slightly improved the CNN performance whereas training the
entire VPN significantly improved the CNN performance by over 1.2\% IoU, with both
VPN and VPN-Flow networks. Moreover, VPN is at least 25$\times$ faster, and simpler to use
compared to the optimization based FSO-CRF which relies on
LDOF optical flow~\cite{brox2009large}, long-term tacks~\cite{sundaram2010dense} and
edges~\cite{dollar2015fast}.
We further improved the performance of the state-of-the-art CNN~\cite{richter2016playing}
with the use of VPN-Flow model. Using better optical flow estimation
might give even better results. Figure~\ref{fig:semantic_visuals} shows some qualitative
results and more are presented in Fig.~\ref{fig:semantic_visuals_supp}.

\subsection{Video Color Propagation}

We also evaluate VPNs on a different kind of information and
experimented with propagating color information in a grayscale video. Given the
color image for the first video frame, the task is to propagate the color to the entire
video. Note that this task is fundamentally different from automatic colorization of images
for which recent CNN based based methods have become popular.
For experiments on this task, we again used the DAVIS dataset~\cite{Perazzi2016} with the
first 25 frames from each video. We randomly divided the dataset into 30 train,
5 validation and 15 test videos.

\begin{table}[t]
    % \scriptsize
    % \fontsize{5}{3.2}\selectfont
    \centering
    \begin{tabular}{p{4.0cm}>{\centering\arraybackslash}p{2.6cm}>{\centering\arraybackslash}p{3.5cm}}
        \toprule
        \scriptsize
        & \textit{PSNR} & \textit{Runtime}(s) \\ [0.1cm]
        \midrule
        BNN-Identity & 27.89 & 0.29\\
        VPN-Stage1 & \textbf{28.15} & 0.90\\
        \midrule
        Levin et al.~\cite{levin2004colorization} & 27.11 & 19\\
        \bottomrule
        \\
    \end{tabular}
    \mycaption{Results of Video Color Propagation}{Average PSNR results and runtimes of
    different methods for video color propagation on images from DAVIS dataset.}
    \label{tbl:color}
    \vspace{-0.5cm}
\end{table}

\begin{figure}[th!]
\begin{center}
  \centerline{\includegraphics[width=0.9\columnwidth]{figures/colorization_visuals.pdf}}
    \mycaption{Video Color Propagation}
    {Input grayscale video frames and corresponding ground-truth (GT) color images
    together with color predictions of Levin et al.~\cite{levin2004colorization} and VPN-Stage1 models.}
    \label{fig:color_visuals}
\end{center}
\vspace{-1.0cm}
\end{figure}

We work with YCbCr representation of images and propagate CbCr values from previous
frames with pixel intensity, position and time features as guidance for VPN.
The same strategy as in object segmentation is used, where an initial
set of color propagated results was obtained with BNN-Identity and then used to trained a VPN-Stage1 model.
Training further VPN stages did not improve the performance.
Table~\ref{tbl:color} shows the PSNR results.
We use 300K radomly sampled points from previous 3 frames as input
to the VPN network. We also show a baseline result of~\cite{levin2004colorization} that
does graph based optimization and uses optical flow. We used fast
DIS optical flow~\cite{kroeger2016fast} in the baseline method~\cite{levin2004colorization}
and we did not observe significant differences with using LDOF optical flow~\cite{brox2009large}.
Figure~\ref{fig:color_visuals} shows a visual result with more
in Fig.~\ref{fig:color_visuals_supp}.
From the results,
VPN works reliably better than~\cite{levin2004colorization} while being 20$\times$ faster.
The method of~\cite{levin2004colorization} relies heavily on optical flow
and so the color drifts away with incorrect flow. We observe that our method also bleeds color
in some regions especially when there are large viewpoint changes.
We could not compare against recent video color propagation techniques such as
~\cite{heu2009image,sheng2014video} as their codes are not available online.
This application shows general applicability of VPNs in propagating different
kinds of information.

\vspace{-0.3cm}
\section{Discussion and Conclusions}
\label{sec:conclusion}

We proposed a fast, scalable and generic neural network based learning approach
for propagating information across video frames.~The video propagation network uses
bilateral network for long-range video-adaptive propagation of information from previous
frames to the present frame which is then refined by a standard spatial network.
Experiments on diverse tasks show that VPNs, despite being generic, outperformed
the current state-of-the-art task-specific methods. At the core of our technique
is the exploitation and modification of learnable bilateral filtering for the use
in video processing. We used a simple and fixed network architecture for all the
tasks for showcasing the generality of the approach. Depending on the type of
problems and the availability of data, using more filters and deeper layers
would result in better performance. In this work, we manually tuned the feature scales which
could be amendable to learning. Finding optimal yet fast-to-compute bilateral features for
videos together with the learning of their scales is an important future
research direction.



\section*{Supplementary material}

The Supplementary Material includes details for the MCMC algorithm, proofs of the propositions, and additional results for the data examples. %Code and data for the numerical study of this paper is available at the \href{https://github.com/jkang37/LSBP-Ordinal-Regression}{Github page}.

%% this Github page is currently made private.


%\newpage
\appendix
\section{Pricing equations}
\subsection{Credit default swap}
\label{CDS_pricing}
A credit default swap (CDS) is a contract designed to exchange credit risk of a Reference Name (RN) between a Protection Buyer (PB) and a Protection Seller (PS). PB makes periodic coupon payments to PS conditional on no default of RN, up to the nearest payment date, in the exchange for receiving from PS the loss given RN's default.

Consider a CDS contract written on the first bank (RN), denote its price $C_1(t, x)$.\footnote{For the CDS contracts written on the second bank, the similar expression could be provided by analogy.} We assume that the coupon is paid continuously and equals to $c$. Then, the value of a standard CDS contract can be given (\cite{BieleckiRutkowski}) by the solution of  (\ref{kolm_1})--(\ref{kolm_2})  with $\chi(t, x) = c$ and terminal condition
\begin{equation*}
	\psi(x) = 
	\begin{cases}
		1 - \min(R_1, \tilde{R}_1(1)), \quad (x_1, x_2) \in D_2, \\
		1 - \min(R_1, \tilde{R}_1(\omega_2)), \quad (x_1, x_2) \in D_{12}, \\		
	\end{cases}
\end{equation*}
where $\omega_2 = \omega_2(x)$ is defined in (\ref{term_cond}) and 
\begin{equation*}
	\tilde{R}_1(\omega_2) = \min \left[1, \frac{A_1(T) +  \omega_2 L_{2 1}(T)}{L_1(T) + \omega_2 L_{12}(T)}\right].
\end{equation*}
Thus, the pricing problem for CDS contract on the first bank is
\begin{equation}
\begin{aligned}
		& \frac{\partial}{\partial t} C_1(t, x) + \mathcal{L} C_1(t, x) = c, \\
		& C_1(t, 0, x_2) = 1 - R_1, \quad C_1(t, \infty, x_2) = -c(T-t), \\
		& C_1(t, x_1, 0) = \Xi(t, x_1) = 
		\begin{cases}
			c_{1,0}(t, x_1), & x_1 \ge \tilde{\mu}_1, \\
			1-R_1, & x_1 < \tilde{\mu}_i,
		\end{cases} \quad C_1(t, x_1, \infty) = c_{1,\infty}(t, x_1),\\
		& C_1(T, x) = \psi(x) = 
	\begin{cases}
		1 - \min(R_1, \tilde{R}_1(1)), \quad (x_1, x_2) \in D_2, \\
		1 - \min(R_1, \tilde{R}_1(\omega_2)), \quad (x_1, x_2) \in D_{12}, \\		
	\end{cases}
\end{aligned}
\end{equation}
where $c_{1,0}(t, x_1)$ is the solution of the following boundary value problem:
\begin{equation}
\begin{aligned}
		& \frac{\partial}{\partial t} c_{1, 0}(t, x_1) + \mathcal{L}_1 c_{1, 0}(t, x_1) = c, \\
		& c_{1, 0}(t, \tilde{\mu}_1^{<}) = 1 - R_1, \quad c_{1, 0}(t, \infty) = -c(T-t), \\
		& c_{1, 0}(T, x_1) = (1 - R_1) \mathbbm{1}_{\{\tilde{\mu}_1^{<} \le x_1 \le \tilde{\mu}_1^{=}\}}, 
\end{aligned}
\end{equation}
and $c_{1,\infty}(t, x_1)$ is the solution of the following boundary value problem
\begin{equation}
\begin{aligned}
		& \frac{\partial}{\partial t} c_{1, \infty}(t, x_1) + \mathcal{L}_1 c_{1, \infty}(t, x_1) = c, \\
		& c_{1, \infty}(t, 0) = 1 - R_1, \quad c_{1, \infty}(t, \infty) = -c(T-t), \\
		& c_{1, \infty}(T, x_1) = (1 - R_1) \mathbbm{1}_{\{x_1 \le \mu_1^{=}\}}.
\end{aligned}
\end{equation}

\subsection{First-to-default swap}
An FTD contract refers to a basket of reference names (RN). Similar to a regular CDS, the Protection Buyer (PB) pays a regular coupon payment $c$ to the Protection Seller (PS) up to the first default of any of the RN in the basket or maturity time $T$. In return, PS compensates PB the loss caused by the first default.

Consider the FTD contract referenced on $2$ banks, and denote its price $F(t, x)$. We assume that the coupon is paid continuously and equals to $c$. Then, the value of FTD contract can be given (\cite{LiptonItkin2015}) by the solution of  (\ref{kolm_1})--(\ref{kolm_2})  with $\chi(t, x) = c$ and terminal condition
\begin{equation*}
	\psi(x) = \beta_0  \mathbbm{1}_{\{x \in D_{12}\}} + \beta_1 \mathbbm{1}_{\{x \in D_{1}\}} + \beta_2 \mathbbm{1}_{\{x \in D_{2}\}},
\end{equation*}
where
\begin{equation*}
	\begin{aligned}
		\beta_0 = 1 - \min[\min(R_1, \tilde{R}_1(\omega_2), \min(R_2, \tilde{R}_2(\omega_1)], \\
		\beta_1 = 1 - \min(R_2, \tilde{R}_2(1)), \quad \beta_2 = 1 - \min(R_1, \tilde{R}_1(1)),
	\end{aligned}
\end{equation*}
and
\begin{equation*}
	\tilde{R}_1(\omega_2) = \min \left[1, \frac{A_1(T) +  \omega_2 L_{2 1}(T)}{L_1(T) + \omega_2 L_{12}(T)}\right], \quad \tilde{R}_2(\omega_1) = \min \left[1, \frac{A_2(T) +  \omega_1 L_{1 2}(T)}{L_2(T) + \omega_1 L_{21}(T)}\right].
\end{equation*}
with $\omega_1 = \omega_1(x)$ and $\omega_2 = \omega_2(x)$ defined in (\ref{term_cond}).

Thus, the pricing problem for a FTD contract is
\begin{equation}
\begin{aligned}
		& \frac{\partial}{\partial t} F(t, x) + \mathcal{L} F(t, x) = c, \\
		& F(t, x_1, 0) = 1 - R_2,  \quad F(t, 0, x_2) = 1 - R_1, \\
		& F(t, x_1, \infty) = f_{2,\infty}(t, x_1), \quad F(t, \infty, x_2) = f_{1,\infty}(t, x_2), \\
		& F(T, x) = \beta_0  \mathbbm{1}_{\{x \in D_{12}\}} + \beta_1 \mathbbm{1}_{\{x \in D_{1}\}} + \beta_2 \mathbbm{1}_{\{x \in D_{2}\}},
\end{aligned}
\end{equation}
where $f_{1,\infty}(t, x_1)$ and $f_{2,\infty}(t, x_2)$ are the solutions of the following boundary value problems
\begin{equation}
\begin{aligned}
		& \frac{\partial}{\partial t} f_{i, \infty}(t, x_i) + \mathcal{L}_i f_{i, \infty}(t, x_i) = c, \\
		& f_{i, \infty}(t, 0) = 1 - R_i, \quad f_{i, \infty}(t, \infty) = -c(T-t), \\
		& f_{1, \infty}(T, x_i) = (1 - R_i) \mathbbm{1}_{\{x_i \le \mu_i^{=}\}}.
\end{aligned}
\end{equation}

\subsection{Credit and Debt Value Adjustments for CDS}

Credit Value Adjustment and Debt Value Adjustment can be considered either unilateral or bilateral. For unilateral counterparty risk, we need to consider only two banks (RN, and PS for CVA and PB for DVA), and a two-dimensional problem can be formulated, while bilateral counterparty risk requires a three-dimensional problem, where Reference Name, Protection Buyer, and Protection Seller are all taken into account. We follow \cite{LiptonSav} for the pricing problem formulation but include jumps and mutual liabilities, which affects the boundary conditions.

\paragraph{Unilateral CVA and DVA}
The Credit Value Adjustment represents the additional price associated with the possibility of a counterparty's default. Then, CVA can be defined as
\begin{equation}
	V^{CVA} = (1- R_{PS}) \mathbb{E}[\mathbbm{1}_{\{\tau^{PS} < \min(T, \tau^{RN}) \}} (V_{\tau^{PS}}^{CDS})^{+} \, | \mathcal{F}_t],
\end{equation}
where $R_{PS}$ is the recovery rate of PS, $\tau^{PS}$ and $\tau^{RN}$ are the default times of PS and RN, and $V_t^{CDS}$ is the price of a CDS without counterparty credit risk.

We associate $x_1$ with the Protection Seller and $x_2$ with the Reference Name, then CVA can be given by the solution of  (\ref{kolm_1})--(\ref{kolm_2})  with $\chi(t, x) = 0$ and $\psi(x) = 0$. Thus,
\begin{equation}
\begin{aligned}
		& \frac{\partial}{\partial t} V^{CVA}+ \mathcal{L} V^{CVA} = 0, \\
		& V^{CVA}(t, 0, x_2) = (1 - R_{PS}) V^{CDS}(t, x_2)^{+}, \quad V^{CVA}(t, x_1, 0) = 0, \\
		& V^{CVA}(T, x_1, x_2) = 0.
\end{aligned}
\end{equation}

Similar, Debt Value Adjustment represents the additional price associated with the default and defined as
\begin{equation}
	V^{DVA} = (1- R_{PB}) \mathbb{E}[\mathbbm{1}_{\{\tau^{PB} < \min(T, \tau^{RN}) \}} (V_{\tau^{PB}}^{CDS})^{-} \, | \mathcal{F}_t],
\end{equation}
where $R_{PB}$ and $\tau^{PB}$ are the recovery rate and default time of the protection buyer.

Here, we associate $x_1$ with the Protection Buyer and $x_2$ with the Reference Name, then, similar to CVA,  DVA can be given by the solution of  (\ref{kolm_1})--(\ref{kolm_2}),
\begin{equation}
\begin{aligned}
		& \frac{\partial}{\partial t} V^{DVA}+ \mathcal{L} V^{DVA} = 0, \\
		& V^{DVA}(t, 0, x_2) = (1 - R_{PB}) V^{CDS}(t, x_2)^{-}, \quad V^{DVA}(t, x_1, 0) = 0, \\
		& V^{DVA}(T, x_1, x_2) = 0.
\end{aligned}
\end{equation}

\paragraph{Bilateral CVA and DVA}

When we defined unilateral CVA and DVA, we assumed that either protection  buyer, or protection seller are risk-free. Here we assume that they are both risky. Then, 
The Credit Value Adjustment represents the additional price associated with the possibility of counterparty's default and defined as
\begin{equation}
	V^{CVA} = (1 - R_{PS}) \mathbb{E}[\mathbbm{1}_{\{\tau^{PS} < \min(\tau^{PB}, \tau^{RN}, T)\}} (V^{CDS}_{\tau^{PS}})^{+} \, | \mathcal{F}_t],
\end{equation} 

Similar, for DVA
\begin{equation}
	V^{DVA} = (1 - R_{PB}) \mathbb{E}[\mathbbm{1}_{\{\tau^{PB} < \min(\tau^{PS}, \tau^{RN}, T)\}} (V^{CDS}_{\tau^{PB}})^{-} \, | \mathcal{F}_t],
\end{equation} 


We associate $x_1$ with protection seller, $x_2$ with protection buyer, and $x_3$ with reference name. Here, we have a three-dimensional process. Applying three-dimensional version of (\ref{kolm_1})--(\ref{kolm_2}) with $\psi(x) = 0, \chi(t, x) = 0$, we get
\begin{equation}
	\label{CVA_pde}
\begin{aligned}
		& \frac{\partial}{\partial t} V^{CVA} + \mathcal{L}_3 V^{CVA} = 0, \\
		& V^{CVA}(t, 0, x_2, x_3) = (1 - R_{PS}) V^{CDS}(t, x_3)^{+}, \\
		& V^{CVA}(t, x_1, 0, x_3 ) = 0, \quad V^{CVA}(t, x_1, x_2, 0)  = 0, \\
		& V^{CVA}(T, x_1, x_2, x_3) = 0,
\end{aligned}
\end{equation}
and
\begin{equation}
\label{DVA_pde}
\begin{aligned}
		& \frac{\partial}{\partial t} V^{DVA} + \mathcal{L}_3 V^{DVA} = 0, \\
		& V^{DVA}(t, 0, x_2, x_3) = (1 - R_{PB}) V^{CDS}(t, x_3)^{-}, \\
		& V^{DVA}(t, x_1, 0, x_3 ) = 0, \quad V^{DVA}(t, x_1, x_2, 0)  = 0, \\
		& V^{DVA}(T, x_1, x_2, x_3) = 0,
\end{aligned}
\end{equation}
where $\mathcal{L}_3 f$ is the three-dimensional infinitesimal generator.





%\section{BibTeX}

%We hope you've chosen to use BibTeX!\ If you have, please feel free to use the package natbib with any bibliography style you're comfortable with. The .bst file agsm has been included here for your convenience. 

\bibliographystyle{jasa3}
\bibliography{sample}


\section{Convergence of multi-electron states}
The convergence of multi-electron states calculated using FCI depends on the number of single-electron states taken into the basis. As in this work, we investigate the effect of superexchange, i.e. the energy difference $\Delta E$ between outer-dot singlet and triplet states, $S^l$ and $T^l_0$ respectively, we use $\Delta E$ also to determine the convergence of our calculated four-electron states. We gradually increase the number of single-electron basis states in FCI calculations and observe when the changes in $\Delta E$ are negligible within some numerical tolerance. In Figure \ref{fig:convergence} we show examples of such analysis for donors separated in both [100] and [110] crystal directions. We find that in most of the cases considered in the paper, it is sufficient to use 56 basis states, i.e. 28 valley-orbital states, with two-fold spin degeneracy. In a few cases, we have used up to 72 basis states to reach convergence in the energy difference. 

\begin{figure}[htb!]
    \centering
    \includegraphics[width=0.8\textwidth]{All_convergence.pdf}
    \caption[Convergence of superexchange as a function of a number of single-electron basis states]{\textbf{Convergence of superexchange as a function of a number of single-electron basis states.} Plots (a,c) show results for equidistant dots with $r_1=r_2=r_M=18a_0$ along the [100] direction where $a_0 = 0.5431$ nm and $r_1=r_2=r_M=24a$ along the [110] direction where $a = a_0/\sqrt{2}$, respectively. Plots (b,d) show results for non-equidistant cases $r_1=r_2=20a_0$, $r_M=14a_0$ in [100] and $r_1=r_2=26a$, $r_M=20a$ in [110], respectively.}
    \label{fig:convergence}
\end{figure}



\section{Analysis of four-electron eigenstates}

% As discussed in the main text, the Schrieffer-Wolff approximation of the spin Hamiltonian holds only when $j_1,j_2\ll j_M$. In this regime, the lowest energy manifold is characterized by a singlet state formed within the two middle dots and is well separated in energy from higher energy states. In that case, the superexchange is well-defined and the coherent coupling of the outer-dots spins is possible. 


% In this section, we analyze the 4-electron states in terms of contributions from different Slater determinants, i.e. consisting of different 1-electron basis states. This will help us to determine the regimes of $j_1,j_2,j_M$ where the eigenstates can be confidently described in terms of singlet-triplet symmetries.


% \begin{figure}[htbp]
%     \centering
%     \includegraphics[scale=0.6]{spin_model_eval_contrib.pdf}
%     \caption{(a) Energies of 4-electron eigenstates calculated with effective Hamiltonian $H_{eff}$ as a function of middle donors exchange $j_M$, for constant $j_1=j_2=0.1$. (b) Contribution of inner-singlet and inner-triplet basis states in the ground state of $H_{eff}$ for the same $j$ values as in (a).}
%     \label{fig:effective}
% \end{figure}

% First, we look at the problem from the effective spin Hamiltonian perspective. As mentioned in the main text we restrict ourselves to the basis consisting only of $S_z=0$ states, i.e. $\ket{\uparrow S \downarrow}, \ket{\downarrow S \uparrow}$, $\ket{\uparrow T_0 \downarrow}, \ket{\downarrow T_0 \uparrow}$, $\ket{\uparrow T_- \uparrow}, \ket{\downarrow T_+ \downarrow}$. For $j_1,j_2\ll j_M$ the two lowest-energy states can be well described by $S^l$ and $T^l_0$, as defined in the main text. However, when we move closer to the $j_1,j_2 \approx j_M$ region $S^l$ and $T^l_0$ states begin to acquire some inner-dots triplet admixtures. This can be seen in Fig. \ref{fig:effective}. In Fig. \ref{fig:effective}(a) we plot energies of all the eigenstates of the effective Hamiltonian $H_{eff}$ for $j_1=j_1=0.1$ and $j_M$ varying from 0 to 1. We can see that for $j_M=1$ the $S^l$ and $T^l_0$ states are well separated in energy from all excited states. In \ref{fig:effective}(b) in blue we can see $S^l$ contributions from inner-dots singlet states (i.e. $\ket{\uparrow S \downarrow}$ and $\ket{\downarrow S \uparrow}$) and in red contributions from inner-dots triplet states (all the remaining basis states). We can see the singlet contribution reaches 99.1\% for $j_M=1$. When decreasing $j_M$ the energies of $S^l$ and $T^l_0$ are closing to the excited states and around $j_1=j_2=j_M$ we can see clear anticrossing between lower and upper energy manifolds. In this region, the inner-dots singlet and triplet contributions in $S_l$ are both approximately 50\%. In this regime, we do no longer have well defined two-level system of $S^l$ and $T^l_0$, separated by superexchange which can be used to manipulate the states coherently. The specific value of $j_M/j_{1,2}$ ratio required for the coherent manipulation in real device applications depends strongly on the desired fidelity and the details of the experiment.


\begin{figure}[htbp]
    \centering
    \includegraphics[width=\columnwidth]{wf16_v2.pdf}
    \caption{\textbf{Single electron basis states calculated using an atomistic tight-binding approach.} The wavefunctions of the first 16 valley-orbit molecular states are plotted here on a logarithmic scale. The titles indicate the binding energy of each of the states.}
    \label{fig:basis_states}
\end{figure}

In this section, we analyze the 4-electron states in terms of their contributions from different Slater determinants, i.e. consisting of different 1-electron atomistic tight-binding basis states. This will help us to determine the regimes of $j_1,j_2,j_M$ where the eigenstates can be confidently described in terms of singlet-triplet symmetries. First, we focus on [100] crystal direction and the equidistant case $r_1=r_2=r_M=18a_0$. In Figure \ref{fig:basis_states} we plot several of the lowest-energy single electron states. 

\begin{itemize}
    \item 
We can see that the ground and first-excited orbitals (states numbered 1 and 3 in the figure) are localized mainly in the two middle dots. This is because superposing the four donors confining potentials in TB calculations results in the total potential being slightly deeper within the middle dots than the outer ones. 
    \item 
The next two excited orbitals (states 5 and 7) are localized mainly within two outer dots. 
    \item
We consider the even-numbered states (2,4,6,8...) to be of the same orbital as odd (1,3,5,7...) but of opposite spin. 
    \item
The 4-electron ground state, labelled by us as $S_l$, has main contributions from following configurations: 14.7\% $\ket{1,3,6,8}$, 14.7\% $\ket{2,4,5,7}$, 10.7\% $\ket{1,2,5,6}$, 9.5\% $\ket{1,2,7,8}$, 8.9\% $\ket{3,4,7,8}$, 7.8\% $\ket{3,4,5,6}$, and smaller contributions from other configurations.
    \item
The first two contributions can be interpreted as inner-triplet outer-triplet states ($\ket{1,3,6,8}$ is $T_-$ in inner dots and $T_+$ in outer dots,  $\ket{2,4,5,7}$ is opposite to that). 
    \item
Other configurations are inner-singlet outer-singlet states. 
    \item
We can then see that a significant part of the ground state is described as an inner-dot triplet, which is specific for $j_{1,2}\approx j_M$ regime and has been expected by us following the above discussion on effective Hamiltonian states. 
    \item
To obtain a rough estimate of the percentage of singlet and triplet contributions within the middle dots we sum up contributions from states 
$\{\ket{1,2,i,j},\ket{3,4,i,j}\}$ 
for singlet and $\{\ket{1,3,i,j},\ket{1,4,i,j}$,
$\ket{2,3,i,j}$,
$\ket{2,4,i,j}\}$ for triplet. 
\end{itemize}
We show the results in Table \ref{tab:contrib}. As expected from the effective Hamiltonian model here both contributions are close to 50\%. The presented numbers are, however, just an estimate, not a precise evaluation of singlet and triplet admixtures in the middle dots as i) we do not sum up all the higher-order Slater determinant probabilities and ii) the orbitals 1 and 3 have small but non-zero probabilities also in the outer dots.


Next, we look at the non-equidistant case, with $r_1=r_2>r_M$ which guarantees $j_1=j_2<j_M$. In Table \ref{tab:contrib} we compare inner-singlet and inner-triplet contributions in $S_l$ for 3 values of $r_M$, i.e. $18a_0$, $16a_0$ and $14a_0$, while keeping the total four-donor separation $R$ constant at $54a_0$. We can see the inner-singlet (inner-triplet) contribution is significantly increasing (decreasing) when the middle donors are pulled closer together.  


\begin{table}[htb!]
    \centering
    \begin{tabular}{|c|c|c|}
    \hline
     Separation ($a_0$) & Inner singlet contribution & Inner triplet contribution  \\ \hline
     18 & 0.4927 & 0.483  \\ \hline
     16 & 0.9209 & 0.06706 \\ \hline
     14 & 0.97904 & 0.00238 \\ \hline
\end{tabular} 
    \caption{\textbf{Approximate contributions of the inner-singlet and inner-triplet configurations in the FCI ground state of a 4-donor chain separated along the [100] direction} Here we show 3 different middle donor separations $18a_0$, $16a_0$ and $14a_0$ with a constant outer donor separation of $54a_0$. We see that the inner singlet contribution to the ground state dramatically increases from $\sim49\%$ to $\sim98\%$ as the separation decreases by $4a_0$.}
    \label{tab:contrib}
\end{table}




\end{document}
