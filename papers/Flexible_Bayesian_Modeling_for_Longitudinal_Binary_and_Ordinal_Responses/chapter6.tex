\section{Summary}
%\section{Summary}
\label{sec:summary}

We have developed a novel Bayesian hierarchical model for analyzing longitudinal binary 
data. We approach the problem from a functional data analysis perspective, resulting in 
a method that is suitable for either regularly or irregularly spaced longitudinal data. 
The modeling approach achieves flexibility and computational efficiency in full posterior 
inference. With regard to the former, the key model feature is the joint and nonparametric 
modeling of the mean and covariance structure. As illustrated by the data application, 
%that involves assessing students' affects, 
our approach enables interpretable inference with coherent uncertainty quantification, 
and provides improvement over the GLMM approach.  
The model formulation enables a natural extension to incorporate ordinal responses, 
which is accomplished by leveraging the continuation-ratio logits representation of 
the multinomial distribution. This representation leads to a factorization of the 
multinomial model into separate binomial models, on which the modeling approach for 
binary responses can be applied. The computational benefit is retained, since we can 
utilize parallel computing across response categories. 
%We illustrate the implementation of the proposed model in dealing with ordinal 
%responses with the four levels arousal score data. 

 
%In the current implementation of our approach, we focus on binary responses. We could also 
%incorporate %multinomial 

%
%In the data application we considered, we have not addressed the correlation between 
%the valence score and the arousal score from the same subject. To account for the correlation 
%of a subject's multiple responses, models for multivariate ordinal responses are required. 
%One potential challenge is designing a model structure that can capture various types of 
%the multivariate ordinal responses' correlation. A further model extension involves taking 
%into account pre-determined covariates associated with each subject, which can be 
%incorporated through the prior placed on the mean function of the signal process. 
%We will report on these modeling extensions in a future manuscript.
%

%
%One aspect of longitudinal data analysis problem that we have not addressed in our model is 
%the presence of covariates. It is common for longitudinal studies to have pre-determined 
%covariates associated with each subject, or time-varying covariates corresponding to each 
%observation. The pre-determined covariates can be incorporate in the proposed model through 
%the prior placed on the mean function of the signal process. We plan to report on this modeling 
%extension in a future manuscript. The potentially more challenge task is to account for 
%time-varying covariates, which should be treated as functional data as well. One possible 
%solution is using the functional linear model.
%
   