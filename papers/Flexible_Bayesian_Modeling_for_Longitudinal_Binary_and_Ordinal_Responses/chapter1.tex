

\section{Introduction}
\label{sec:intro}

% background
Recent years have witnessed a rapid growth of longitudinal studies with binary and 
ordinal responses in several disciplines, including econometrics, and the health 
and social sciences. In such studies, of primary importance are the probability 
response curves, i.e., the probabilities of the response categories that evolve
dynamically over time.
%known as the probability response curves. Because of its potential to address 
%relevant scientific and practical questions, inference for these curves is of 
%primary importance. 
This article aims at developing a hierarchical framework, 
customized to longitudinal settings, that allows flexible inference for the probability 
response curves. In addition, the defining characteristic of longitudinal data is that 
repeated measurements on the same subject induce dependence. Hence, a further objective 
is to %accurately identify 
flexibly model lead-lag correlations among repeated measurements. 


% research objective, also mention that we focus on the case without covariates/with categorical covariates

%The defining characteristic of a longitudinal study is that subjects are measured repeatedly through time \citep{DiggleBook2002}. With longitudinal data, one can differentiate the changes over time within subjects (aging effects) from the changes between subjects at baseline (cohort effects). The objectives of analysis include assessing the fluctuation within a subject over time and identifying lead-lag correlations among repeated measurements. In certain applications where features associated with the measurements are also available, regressing longitudinal ordinal outcomes on these fixed covariates is also of interest.  In this article, we focus particularly on developing flexible modeling approach for estimate the probability response curves, which can be viewed as the regression relationship with time as the only covariates. Nonetheless, we discuss including other categorical covariates as a natural extension of the proposed method. 

%Nonetheless, the benefit comes at a price of modeling challenges. The major challenges arise from the two objectives of analyzing longitudinal data: capturing the fluctuation and dependency among the repeated measurements. Consequently, more sophisticated statistical methods are needed. Note that usually repeated measurements come along with features of the subjects, forming up regression problems.  

% parametric GMM modeling framework
%The development of statistical methods for longitudinal binary data stems from models for longitudinal continuous responses. Since the most natural way to view binary data is to postulate the existence of a latent continuous variable associated with each response, such models can be extended to deal with binary data through appropriate link function. A plethora of literature have covered the topic of longitudinal continuous data analysis \citep{HedekerGibbons2006,Verbeke2009,Fitzmaurice2011}, with the majority of them under the mixed effects modeling framework. Suppose we have repeated measurements on $n$ subjects, denoted as $\mathbf{Z}_i=(Z_{i1},\cdots,Z_{iT_i})^T$, $i=1,\cdots,n$. Typically the mixed effects model assumes $\mathbf{Z}_i\sim N(\mathbf{u},\mathbf{V}_i)$, where $\mathbf{u}$ denote the common fixed effect across subjects, and $\mathbf{V}_i$ is the random subject effect that depicts the influence of subjects on their repeated observations \citep{Diggle1988}. The crucial question for the mixed effects model is the choice of fixed and random effects' structure, which has various options \citep{Lindstrom1990,Shi1996,Zhang2001}. These models have been extended to deal with dichotomous data by adopting a link function (logit, probit, etc.), resulting in the generalized mixed models (GMM). For a comprehensive review about GMM, we refer to \citet{Hedeker2008}.
%These models have been extended to deal with dichotomous and ordinal data by adopting a link function (logit, probit, etc.), and a specific representation of response probabilities (proportional odds, adjacent-categories, continuation-ratio, etc.), resulting in the generalized mixed models (GMM). For a comprehensive review about GMM, we refer to \citet{Hedeker2008}. 

% nonparametric modeling approaches
%Seeking for more flexibility, one option is to assume the fixed or random effects of the latent variable have more sophisticated, nonparametric structure. Popular choices include modeling the fixed effect through smoothing techniques \citep{HeZhu2002,WuZhang2006}, applying a matrix stick-breaking process for a residual covariance structure \citep{Das2014}, capturing subject random effect by stochastic process \cite{ZhangLin1998}, putting a Dirichlet process (DP) prior \citep{LiLinMuller2010} or mixture of Pólya trees \citep{Ghosh2010} for random effect distributions, and using a combination of stochastic processes and DP mixture model for the subject random effect \citep{Quintana2016}. Admittedly, such models can be embedded in a hierarchical framework to deal with discrete responses. For instance, under the GMM framework which incorporate a DP mixture of normal prior as probability model for the latent variables,  \citet{Jara2007} and \citet{TangDuan2012} consider binary responses and \citet{Kunihama2019} handles mixed-scale data consisted by continuous and binary responses. Nonetheless, these approaches may not be applicable beyond dichotomous responses because of the computational burden. Concerning multivariate longitudinal ordinal responses, \citet{Tran2021} proposes to use a latent factor model within the GMM framework where the random effect is modeled by Ornstein-Uhlenbeck process. Albeit its good performance, the proposed model is restrictive in the sense that it depends on pre-specified covariance structure. In conclusion, these models are trading between flexibility and computational difficulty. Even though they achieved the balance for their specific purpose respectively, they are not the ideal option for the specific problem we considered. Besides, they examine the mean and covariance structure of the latent variable separately, which may neglect their dependence. 

% review transition modeling framework
%To evaluate the fixed and random effects jointly, one option is to follow the transition modeling framework \citep[see][chap.~10]{DiggleBook2002}. Under a transition model, the past value explicitly affect the present observation, inducing the aging effects. Therefore, the proposed models under this framework typically assume autoregressive transition dynamics between the distribution of responses at adjacent time.  \citet{DiLucca2013} develop a class of Bayesian nonparametric autoregression models. They model a sequence of continuous outcomes though a dependent DP with an additional normal kernel as a prior for the regression on lagged terms in an autoregression. It can be extended to handle binary and ordinal outcomes by treating the them as the discretized version of the continuous outcomes. In addition, motivated by a specific application in fisheries science, \citet{MariaJASA2018} propose a Bayesian nonparametric model for ordinal regression relationships that evolve in discrete time. Their proposed methodology is built on the well-studied DP mixture model for cross-sectional ordinal regression \citep{DeYoreoKottas2018} as the marginal at a specific time, while introduce temporal dependence in the weights and atoms of the DP constructive definition. Although autoregressive structure brings benefits such as interpretability, there is no natural way to treat missing data, hindering the application in unbalanced longitudinal studies.    

%Under the GMM framework, the mean and covariance structure of the latent variable are examined separately, which may neglect their dependence. To incorporate the possible dependence among them, one option is to follow the transition modeling framework \citep[see][chap.~10]{DiggleBook2002}. Under a transition model, the past value explicitly affect the present observation, inducing the aging effects. The proposed models under this framework typically assume autoregressive transition dynamics between the distribution of responses at adjacent time. Although autoregressive structure brings benefits such as interpretability, there is no natural way to treat missing data, hindering the application in unbalanced longitudinal studies.


The development of statistical methods for longitudinal binary and ordinal data stems from 
models for longitudinal continuous responses, postulating the generalized linear model framework. 
Analogous to the continuous case, a specific model is formulated under one of three broad
approaches pertaining to marginal models, conditional models, or subject-specific models. 
Marginal models provide alternative modeling options when likelihood-based approaches are 
difficult to implement.
%A marginal model captures the responses by their marginal distribution given the covariates. 
A conditional model describes the distribution of responses conditional on the covariates 
and also on part of the other components of the responses. In a subject-specific model, 
the effects of a subset of covariates are allowed to vary randomly from one individual 
to another. In the absence of predictor variables, functions of the observation time are 
usually used as covariates. We refer to \citet{Molenberghs2006} for a comprehensive review. 
In Section \ref{subsec:literaturereview}, we elaborate on the connection of our 
proposed modeling approach with existing methods.


%our approach
In this article, we introduce a novel viewpoint for longitudinal binary and ordinal data 
analysis. %which directs a path for new discoveries. 
We begin with the model construction for longitudinal binary responses. The critical insight 
that distinguishes our methodology from the majority of the existing literature is functional 
data analysis. We treat the subjects' measurements as stochastic process realizations at the 
corresponding time points. The benefits are twofold. First, the models can incorporate 
unbalanced data from longitudinal studies in a unified scheme; directly inferring the 
stochastic process provides a well-defined probabilistic model for the missing values. 
Secondly, we can exploit the power of Bayesian hierarchical modeling for continuous 
functional data \citep[e.g.,][]{Yang2016}. To that end, we adopt the Binomial distribution 
with the logit link that connects binary responses to continuous signals, which, subject to 
%a common level of 
additive measurement error, are then modeled as (conditionally) independent and 
identically distributed (i.i.d.) realizations from a Gaussian process (GP) with 
random mean and covariance function. We place an Inverse-Wishart process (IWP) prior 
on the covariance function, and conditional on it, use a GP prior for the mean function. 
Therefore, the two essential ingredients in longitudinal modeling, the trend and the 
covariance structure, are modeled simultaneously and nonparametrically. 


The hierarchical structure allows borrowing of strength across the subjects' trajectories. 
We apply a specific setting of hyperpriors for the GP and IWP priors, such that marginalizing 
over them, the latent continuous functions have a Student-t process (TP) prior. The 
TP enhances the flexibility of the GP \citep[e.g.,][]{Shah2014}. It retains attractive GP 
properties, such as analytic marginal and predictive distributions, and it yields 
predictive covariance that, unlike the GP, explicitly depends on the observed values. 
% what are the benefit?
For inferential purposes, we represent the joint posterior distribution in multivariate form 
through evaluating the functions on the pooled grid, resulting in the common 
normal-inverse-Wishart conditional conjugacy. In conjunction with the Pólya-Gamma data 
augmentation technique \citep{Polson2013}, we develop a relatively simple and effective posterior 
simulation algorithm, circumventing the need for specialized techniques or tuning of 
Metropolis-Hastings steps. 


To extend the model for ordinal responses, we utilize the continuation-ratio logits 
representation of the multinomial distribution. Such representation features an %interchangeable 
encoding of an ordinal response with $C$ categories as a sequence of $C-1$ binary 
indicators, in which the $j$-th indicator signifies whether the ordinal response belongs 
to the $j$-th category or to one of the higher categories. 
%As shown in Proposition \ref{prop:fitseparate}, 
We show that fitting a multinomial model for the ordinal responses is equivalent 
to fitting separately the aforementioned model on the binary indicators. 
%As a consequence, the model for longitudinal ordinal responses inherits the features of 
%the model for binary responses. Furthermore, 
Hence, we can conduct posterior simulation for each response category in a parallel fashion, 
leading to significant computational efficiency gains in model implementation.       


% about the missingness
In modern longitudinal studies, it is common that the complete vector of repeated measurements 
is not collected on all subjects. As a specific example, in ecological momentary assessment 
(EMA) studies, emotions and behaviors are repeatedly measured for a cohort of participants, 
through wearable electronic devices \citep{EMABook2018}. For instance, in the \textit{StudentLife} 
study \citep{StudentLife2014}, researchers monitored the students' mental status through 
pop-up questionnaires on their smartphones that prompted multiple times at pseudorandom 
intervals during the study period. Since the data collection process is based on the 
participants' conscious responding to prompted questions several times a day, non-response 
is inevitable. Missing values are typically considered to be a nuisance rather than a 
characteristic of EMA time series. Parametric and nonparametric Bayesian methods have been 
developed to handle longitudinal data with missingness; see \citet{Daniels2020} for a review. 
The common issue is that one has to bear the drawbacks of making either structured or 
unstructured assumptions to manage missingness. The unstructured approach leads to flexibility, 
yet it may result in difficulties due to a large number of parameters relative to the sample 
size. Besides, the majority of the existing literature on longitudinal studies with missingness 
focuses on the scenario with continuous responses, and the extension to discrete responses is not trivial. 


% summary of our contribution
Accordingly, our contributions can be summarized as follows: (i) we model the mean and 
covariance jointly and nonparametrically, avoiding potential biases caused by a pre-specified 
model structure; (ii) we unify the toolbox for balanced and unbalanced longitudinal studies; 
(iii) the model encourages borrowing of strength, preserving systematic patterns that are 
common across all subject responses; (iv) we develop a computationally efficient posterior 
simulation method by taking advantage of conditional conjugacy; (v) the model facilitates 
applications for ordinal responses with a moderate to large number of categories. 
% and parallel computing.    


% organization of the paper
The rest of the paper is organized as follows. 
%We start from emphasizing the pivotal model designs and benefits with longitudinal 
%binary responses in Section \ref{sec:binarymodel} and Section \ref{sec:realapp}, 
%as it lays the foundation for the extension to ordinal responses. In more detail, 
Section \ref{sec:binarymodel} develops the methodology for binary responses, including
model formulation, study of model properties, and the computational approach to 
inference and prediction.
%In Section \ref{sec:simstudy}, we assess the model by applying it to carefully designed 
%simulation scenarios that reflect our main contributions. 
Section \ref{sec:realapp} illustrates the modeling approach through an EMA study that 
focuses on analyzing students' mental health through binary outcomes. The modeling extension 
for longitudinal ordinal responses is presented in Section \ref{sec:polyordinalmodel}, 
including an illustration involving an ordinal outcome from the same EMA study. Finally, 
Section \ref{sec:summary} concludes with a summary.



 