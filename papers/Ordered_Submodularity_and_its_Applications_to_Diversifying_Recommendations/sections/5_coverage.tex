% \documentclass[../main.tex]{subfiles}

% \begin{document}
% !TEX root = main.tex

\section{Application 1: Diversification for user coverage}\label{sec:coverage}

Suppose we are designing a movie recommender system which produces a single list of recommendations for a large number of users. Every user has some amount of \emph{patience}, representing the fact that users are only willing to scroll down so far before deciding that the list is unsatisfactory. We say that the system \emph{covers} a user if the user is able to find a movie that interests them before their patience expires; otherwise the user gives up on the system and simply walks away. The goal of the designer is to diversify the list of recommendations in order to maximize the number of users covered by the system. In this section, we formally define an objective function for this problem and show that it is ordered-submodular, allowing us to conclude that the greedy algorithm gives a factor of $2$ approximation for the coverage problem.


\subsection{Mathematical formulation}
In a realistic application, we may not expect to exactly know each individual user that will ever use the recommendation system; instead, we may only know a probability distribution over the types of users who will use the system. We may also not know with complete certainty that a movie will or will not interest a given user; we may only have an estimated probability that a movie interests a user of a certain type. To generalize our model to this randomized setting, we seek to maximize the expected number of users covered by the system, or equivalently, the probability that a randomly chosen user is covered.

Let $\pi$ represent the probability distribution over user types (so that $\pi_u$ is the probability that a random user has type $u$). Denote the probability that movie $m$ interests user type $u$ by $p_{m,u}$. Define $\theta_u$, the \emph{patience} of type $u$, as the number of recommendations that a user of type $u$ will consider before leaving the system (e.g., if $\theta_u = 2$, the system will cover $u$ only if they are interested by the first or second movie in the list). 
Then, the probability that the recommendation list $S=s_1 s_2\dots s_k$ covers a randomly chosen user from $\pi$ is $$f(S) = \sum_u \pi_u \left(1 - \prod_{j=1}^{\min \{\theta_u, |S|\}} (1 - p_{s_j, u}) \right),$$ where the inner expression is obtained as the complement of the probability that a user of type $u$ is \emph{not} satisfied before their patience expires or they reach the end of the list, whichever comes first. This is the objective function that we now seek to maximize.

\subsection{Demonstration of ordered submodularity}
The objective function is of the form $f(S) = \sum_u \pi_u f_u(S)$, where $$f_u(S) = 1 - \prod_{j=1}^{\min \{\theta_u, |S|\}} (1 - p_{s_j, u}).$$ Thus by Lemma~\ref{lem:linear_combo}, to show ordered submodularity, it suffices to fix $u$ and show that $f_u$ is ordered-submodular. But now observe that $f_u$ is a function of the set of the first $\theta_u$ elements only (since multiplication is commutative, and any elements indexed above $\theta_u$ are not included in the product). Further, the coverage expression on the right hand side is a submodular set function of the type studied by~\cite{agrawal2009}. So $f_u$ is a sequence function defined by imposing a threshold $\theta_u$ on a submodular set function $h$, which is ordered-submodular by Lemma~\ref{lem:threshold_submodular}. Therefore, we conclude that the overall function $f$ is ordered-submodular. 

\begin{thm}
The user coverage function parametrized by user probability distribution $\pi$, movie satisfaction probabilities $\{p_{m,u}\}$, and patience values $\{\theta_u\}$, $$f(S) = \sum_u \pi_u \left(1 - \prod_{j=1}^{\min \{\theta_u, |S|\}} (1 - p_{s_j, u}) \right),$$ is ordered-submodular. Thus, the greedy algorithm produces a ranked list covering at least $\frac{1}{2}$ as many users as the optimal ranked list.
\end{thm}


\subsection{Greedy approximation ratio of 2 is tight}
A simple example in this setting shows that we can do no better than a factor of $2$ approximation using the greedy algorithm.

Suppose there are two user types, $1$ and $2$, with $(\pi_1, \pi_2) = \left(\frac{1}{2}, \frac{1}{2}\right)$, $\theta_1 = 1$, and $\theta_2 = 2$. There are also two movies, $s_1$ and $s_2$, with $p_{s_1, 1} = p_{s_2, 2} = 1$, $p_{s_1, 2} = p_{s_2, 1} = 0$. We seek to generate a recommendation list of length $2$ (i.e., to rank the two movies in order).

Since $f(s_1) = \pi_1 = \frac{1}{2}$, $f(s_2) = \pi_2 = \frac{1}{2}$, the greedy algorithm may choose arbitrarily between $s_1$ and $s_2$; suppose it chooses $s_2$ first.\footnote{We may also perturb the probabilities by an arbitrarily small amount $\varepsilon$ so that $\pi_1 < \pi_2$, but we make the standard assumption of arbitrary tiebreaking for a cleaner proof of the same result.} It then chooses $s_1$ in the second step, but obtains no additional value since $s_1$ only interests user type $1$, but user type $1$ will not look at the second movie in the list. Then $$ALG = f(s_2 s_1) = \pi_1 \cdot 0 + \pi_2 \cdot 1 = \frac{1}{2}.$$
But the optimal list would place $s_1$ ahead of $s_2$, which first covers user type $1$ before their patience expires, then covers user type $2$, giving 
\begin{align*}
    OPT = f(s_1 s_2) &= \pi_1 \cdot 1 + \pi_2 \cdot 1 = 1,
\end{align*}
so $ALG/OPT = \frac{1}{2}$ exactly.

This example can be extended to a recommendation list of arbitrary length $k$ by defining $k$ user types with $\pi_i = \frac{1}{k}$, $\theta_i = i$ (for $i = 1, 2, \dots, k$) and $k$ movies $s_j$ (for $j = 1, 2, \dots, k$) with $p_{s_j, i} = 1$ if $j = i$ and $p_{s_j, i} = 0$ otherwise.

The optimal list is $s_1 s_2 \dots s_k$, which covers each user type exactly before their patience expires, giving $OPT = 1$. Meanwhile, via induction on the iterations we see that the greedy algorithm can
choose the movies in reverse order, producing the list $s_k s_{k-1} \dots s_1$. Then only movies $s_k$ through $s_{k/2 + 1}$ will be able to interest their corresponding user type (for simplicity suppose $k$ is even); for movies $s_{k/2}$ through $s_1$, their corresponding user type will walk away before they are covered. So we have $$ALG = \sum_{i=1}^{k/2} \pi_i \cdot 0 + \sum_{i=k/2 + 1}^k \pi_i \cdot 1 = \frac{k}{2} \cdot \frac{1}{k} = \frac{1}{2}.$$
Again, $ALG/OPT = \frac{1}{2}$ exactly, establishing that the greedy approximation ratio of $2$ is tight.

\begin{thm}
There exist instances of ordered-submodular optimization problems on which the greedy algorithm achieves exactly $\frac{1}{2}$ of the optimal value. Thus, the 2-approximation performance bound is tight.
\end{thm}

% \end{document}