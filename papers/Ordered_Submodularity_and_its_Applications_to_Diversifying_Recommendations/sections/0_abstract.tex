% !TEX root = main.tex
%\documentclass[../main.tex]{subfiles}


%\begin{document}
A fundamental task underlying many important optimization problems, from influence maximization to sensor placement to content recommendation, is to select the optimal group of $k$ items from a larger set. Submodularity has been very effective in allowing approximation algorithms for such subset selection problems. However, in several applications, we are interested not only in the elements of a set, but also the \emph{order} in which they appear, breaking the assumption that all selected items receive equal consideration. One such category of applications involves the presentation of search results, product recommendations, news articles, and other content, due to the well-documented phenomenon that humans pay greater attention to higher-ranked items. As a result, optimization in content presentation for diversity, user coverage, calibration, or other objectives more accurately represents a sequence selection problem, to which traditional submodularity approximation results no longer apply. Although extensions of submodularity to sequences have been proposed, none is designed to model settings where items contribute based on their position in a ranked list, and hence they are not able to express these types of optimization problems. In this paper, we aim to address this modeling gap.

Here, we propose a new formalism of \emph{ordered submodularity} that captures these ordering problems in content presentation, and more generally a category of optimization problems over ranked sequences in which different list positions contribute differently to the objective function. We analyze the natural ordered analogue of the greedy algorithm and show that it provides a $2$-approximation. We also show that this bound is tight, establishing that our new framework is conceptually and quantitatively distinct from previous formalisms of set and sequence submodularity.


%The fundamental task underlying many important optimization problems, from influence maximization to sensor placement to content recommendation, is to select the optimal group of $k$ items from a larger ground set. When the $k$ items are treated symmetrically, this is a subset selection problem, for which the notion of submodularity has been widely used. However, in several applications, we are interested not only in the elements of a set, but also the \emph{order} in which they appear, breaking the assumption that all items receive equal consideration. One such category of applications involves the presentation of search results, product recommendations, news articles, and other content, due to the well-documented phenomenon that humans pay greater attention to higher-ranked items. As a result, optimization in content presentation for diversity, user coverage, calibration, or other objectives more accurately represents a sequence selection problem, to which traditional submodularity approximation results no longer apply. Although extensions of submodularity to sequences have been proposed, none adequately describes this type of setting where items contribute based on their position in a ranked list. In this paper, we aim to address this modeling gap.
%\end{document}