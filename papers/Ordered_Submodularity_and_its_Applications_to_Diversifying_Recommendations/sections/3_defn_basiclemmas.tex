% !TEX root = main.tex
% \documentclass[../main.tex]{subfiles}

% \begin{document}

\section{Definition of ordered submodularity}\label{sec:ordsub_defn}

In this section we define our extension of submodularity to ordered sets. To simplify notation, for two sequences $A$ and $B$ we will use $A||B$ to denote their concatenation. For a single element $s$ we will use $A||s$ to denote $s$ added at the end of the list $A$.


\begin{defn}[Ordered submodularity]
A sequence function $f$ is \textbf{ordered-submodular} if for all sequences $A$ and $B$, the following property holds for all elements $s$ and $\bar s$: $$f(A||s)-f(A)\ge f(A|| s||B)-f(A||\bar s||B).$$
\end{defn}

Ordered submodularity can be viewed as a generalization of monotonicity and submodularity for set functions. For functions $f$ that depend only on the set of elements in the input sequence and not their order, setting $\bar s=s$ implies $f(A||s)-f(A)\ge 0$, corresponding to monotonicity, and setting $\bar s$ to the ``null'' element implies 
\begin{align*}
    &f(A||s)-f(A) \ge f(A||s||B)-f(A||B),
\end{align*}
corresponding to submodularity.

On the other hand, any monotone submodular set function $f$ when viewed as a function on sequences, that does not depend on the order of the elements satisfies
\begin{align*}
f(A||s)-f(A)\ge f(A||s||B)-f(A||B)\ge f(A||\bar s||B)
\end{align*}
where the first inequality is due to submodularity and the second inequality is due to monotonicity. This is exactly ordered submodularity when $f$ is interpreted as a sequence function, so we see that ordered submodularity is indeed a very natural and well-motivated property in the sequential setting.

We now demonstrate a few basic ways of constructing ordered-submodular functions from other submodular and ordered-submodular functions.

\begin{lem} \label{lem:linear_combo}
If $f$ and $g$ are ordered-submodular, then $\alpha f+ \beta g$ is also ordered-submodular for any $\alpha, \beta \ge 0$.
\end{lem}
\begin{proof}
We simply multiply and add the two inequalities from the definition of ordered submodularity:
\begin{align*}
    \alpha \left[f(A || s) - f(A)\right] &\ge \alpha \left[f(A || s || B) - f(A || \bar{s} || B) \right] \\
    \beta \left[g(A || s) - g(A)\right] &\ge \beta \left[g(A || s || B) - g(A || \bar{s} || B) \right] \\
    \implies (\alpha f+ \beta g)(A||s) - (\alpha f+ \beta g)(A) &\ge (\alpha f+ \beta g)f(A || s || B) - (\alpha f+ \beta g)f(A || \bar{s}_i || B).
\end{align*}
\end{proof}

\begin{lem} \label{lem:threshold_submodular}
Suppose $h$ is a monotone submodular set function. Then the function $f$ constructed by evaluating $h$ on the set of the first $t$ elements of $S$, that is, 
$$f(S) = \begin{cases} h(S) & \text{if } |S| \le t \\
h(S_t) &\text{if } |S| > t \end{cases}$$
is ordered-submodular.
\end{lem}
Here, it is useful to think of $t$ as a \emph{threshold} beyond which additional elements contribute nothing to the value of $f$. Once again, $S_i$ denotes the sequence of the first $i$ elements of the sequence $S$, and for a sequence $S$ we use $h(S)$ to denote the value of the submodular function on the set of elements in $S$, independent of order.

\begin{proof}
We seek to show that for all sequences $A$ and $B$ and elements $s$ and $\bar{s}$, $$f(A||s)-f(A)\ge f(A|| s||B)-f(A||\bar s||B).$$ We take two cases based on $|A|$.

\paragraph{Case 1:} $|A| \ge t$.
Then $f(A || s) = f(A) = f(A||s||B) = f(A||\bar{s}||B) = h(A_t)$, so $$f(A||s) - f(A) = 0 = f(A||s||B) - f(A||\bar{s}||B).$$

\paragraph{Case 2:} $|A| < t$.

Let $j = t - |A| - 1$. Now, observe that we have
\begin{align*}
    f(A||s) - f(A) = h(A||s) - h(A) &\ge h(A||s||B_j) - h(A||B_j) \\
    &\ge h(A||s||B_j) - h(A||\bar{s}||B_j) = f(A||s||B) - f(A||\bar{s}||B),
\end{align*}
where the first inequality is due to submodularity of $h$ and the second is due to monotonicity of $h$.
\end{proof}

\begin{lem} \label{lem:weighted_submodular}
Suppose $h$ is a monotone submodular set function and $\{g_i\}$ is a sequence of monotonically decreasing weights (i.e., $g_i \ge g_j$ if $i < j$). Then the sequence function defined by
$$f(S) = \sum_{i=1}^k g_i \cdot [ h(S_i) - h(S_{i-1}) ],$$ 
where $k = |S|$, is ordered-submodular. 
\end{lem}
Here we use $S_i$ to denote the sequence of the first $i$ elements of the sequence $S$, and for a sequence $S$ we use $h(S)$ to denote the value of the submodular function on the set of elements in $S$, independent of the order of the sequence.
\begin{proof}
Define $g_i' = g_i - g_{i+1}$ (where we use an additional term, $g_{k+1} = 0$, for notational convenience) and the sequence functions $h_i(S) = h(S_i)$, so that we can write $f(S) = \sum_{i=1}^k g_i' \cdot h_i(S)$. By monotonicity, $g_i' \ge 0$ for all $i$, so by Lemma~\ref{lem:linear_combo} it suffices to show that each $h_i(S)$ is ordered-submodular. But $h_i(S)$ is just a monotone submodular set function $h$ evaluated on a threshold of the first $i$ elements of $S$, so it is ordered-submodular by Lemma~\ref{lem:threshold_submodular}. Thus we conclude that $f$ is ordered-submodular.
\end{proof}




%\end{document}