% \documentclass[../main.tex]{subfiles}

% \begin{document}
% !TEX root = main.tex

\section{Discussion} \label{sec:discussion}

In this paper, we have presented a new definition of \emph{ordered submodularity}, which extends the traditional notion of set submodularity to a class of optimization problems in which the order of elements matters, because elements contribute differently based on their position in the sequence. In particular, our formalism models coverage and calibration of ranked lists, two standard problems in the design of content recommendation systems. We have also shown that greedy ordered-submodular maximization gives a $2$-approximation and that this bound is tight on simple instances of the coverage problem. This quantitative result establishes our framework as qualitatively distinct from previous formalisms of set and sequence submodularity, and thus our work has provided the first performance guarantee for approximate optimization of this type.

It is interesting to consider the greedy algorithm in the calibration problem and ask whether the factor of 2 is tight here too, or if the greedy algorithm always performs better in this specific context. Another potential direction for further investigation is parametrizing worst-case instances of the calibration problem, since we found the greedy solution to be very close to optimal across many randomly generated instances. More generally, we pose the natural open question: Does there exist a polynomial time approximation algorithm for ordered submodular maximization achieving a constant factor better than $2$? Or does the analogy to set submodularity continue to hold, in that the greedy algorithm provides the best approximation guarantee possible? Further understanding the approximability of this class of problems is a key next step in the continued development and application of our framework. 

% \end{document}