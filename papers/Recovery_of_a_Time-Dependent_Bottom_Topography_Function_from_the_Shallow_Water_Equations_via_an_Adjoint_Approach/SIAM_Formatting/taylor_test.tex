\documentclass{article}

\usepackage{bm}
\usepackage{graphicx}
\usepackage{xcolor}
\usepackage{amsmath}
\usepackage{hyperref}
\usepackage{enumitem}

\begin{document}

\noindent \textbf{Taylor Test:}

From \url{http://www.dolfin-adjoint.org/en/latest/documentation/verification.html}:

``The fundamental tool used in verification of gradients is the Taylor remainder convergence test. Let $\hat{J}(m)$ be the functional, considered as a pure function of the parameter of interest, let $\nabla \hat{J}$ be its gradient, and let $\delta m$ be a perturbation to m. This test is based on the observation that

\begin{equation}
    |\hat{J}(m + h \delta m) - \hat{J}(m) | \rightarrow 0 \qquad \text{at } O(h),
\end{equation}
but that
\begin{equation}
    |\hat{J}(m + h \delta m) - \hat{J}(m) - h \nabla \hat{J} \cdot \delta m| \rightarrow 0 \qquad \text{at } O(h^2),
\end{equation}
by Taylor’s theorem. The quantity $|\hat{J}(m + h \delta m) - \hat{J}(m) |$ is called the first-order Taylor remainder (because it’s supposed to be first-order), and the quantity $|\hat{J}(m + h \delta m) - \hat{J}(m) - h \nabla \hat{J} \cdot \delta m|$ is called the second-order Taylor remainder.

Suppose someone gives you two functions $\hat{J}$ and $d\hat{J}$, and claims that $d\hat{J}$ is the gradient of $\hat{J}$. Then their claim can be rigorously verified by computing the second-order Taylor remainder for some choice of $h$ and $\delta m$, then repeatedly halving h and checking that the result decreases by a factor of 4.''
\\\\\\
Thus for our problem, is the following correct?

\begin{itemize}
    \item $m = p(t)$
    \item $m + h \delta m = p(t) + k \delta p(t)$ 
    \\ (I use $k$ here instead of $h$ so as to not confuse it with one of the forward solution variables)
    \item $\hat{J}(m) 
    = \bm{J}_0(p(t)) 
    = \int_0^T \frac{1}{2}\left|\left[\mathcal{E}(p)\right](x_0,t)\right|^2 + \frac{1}{2}\left|\left[\mathcal{E}(p)\right](x_L,t)\right|^2 dt $ 
    \begin{itemize}[label=$\circ$]
        \item where $\left[\mathcal{E}(p)\right](x,t) = \Lambda(B(p)) (x,t)-\hat{\Lambda}^{\text{noisy}}(x,t)$ 
        \item with $\Lambda(B(p)) (x,t) = \begin{pmatrix} h(x,t) \\ q(x,t) \end{pmatrix}$
        \item and $\hat{\Lambda}^{\text{noisy}}(x,t) = \begin{pmatrix} \hat{h}^{\text{noisy}}(x,t) \\ \hat{q}^{\text{noisy}}(x,t) \end{pmatrix}$
    \end{itemize}
    \item $\hat{J}(m + h \delta m) 
    = \bm{J}_0(p(t) + k \delta p(t)) 
    = \int_0^T \frac{1}{2}\left|\left[\mathcal{E}(p+k\delta p)\right](x_0,t)\right|^2 + \frac{1}{2}\left|\left[\mathcal{E}(p+k\delta p)\right](x_L,t)\right|^2 dt .$
    \begin{itemize}[label=$\circ$]
        \item where $\left[\mathcal{E}(p+k\delta p)\right](x,t) = \Lambda(B(p+k\delta p)) (x,t)-\hat{\Lambda}^{\text{noisy}}(x,t)$ 
        \item with $\Lambda(B(p+k\delta p)) (x,t)$ which means $h$ and $q$ get evaluated when the bottom is $B(p+k\delta p) = B_0 + (p(t) + k \delta p(t))B_1$
        \textcolor{red}{Is this correct? If so, does that mean that for each RK step I need to calculate both h \& q evaluated when $B = B_0 + p(t)B_1$ and when evaluated for $B = B_0 + (p + k \delta p)B_0?$}
        \item  \textcolor{red}{I don't need to perturb the noisy measured data here, do I? I just use $\hat{\Lambda}^{\text{noisy}}(x,t)$?}
    \end{itemize}
    \item $\nabla \hat{J} 
    = \nabla \bm{J}_0 
    = \int_{x_0}^{x_L} -g \sigma_2(x,t) h(x,t) \partial_x b_1(x) \ dx$
\end{itemize}
So the second order Taylor remainder would be:
\begin{equation}\label{taylor_1}
\begin{split}
    \Bigg|\int_0^T  & \bigg[ \frac{1}{2}\left|\left[\mathcal{E}(p+k \delta p)\right](x_0,t)\right|^2 
    + \frac{1}{2}\left|\left[\mathcal{E}(p+k \delta p)\right](x_L,t)\right|^2 \\
    & - \frac{1}{2}\left|\left[\mathcal{E}(p)\right](x_0,t)\right|^2 
    - \frac{1}{2}\left|\left[\mathcal{E}(p)\right](x_L,t)\right|^2 \bigg] \ dt \\
    & - k \int_0^T\left(\int_{x_0}^{x_L} -g \sigma_2(x,t) h(x,t) \partial_x b_1(x) \ dx \cdot \delta p(t) \ dt \right)\Bigg| 
\end{split}
\end{equation}
% Question: since $m=p(t)$ (i.e. a function) does the term $-h\nabla \hat{J} \cdot \delta m = -k \nabla \hat{J} \cdot \delta p(t)$ in our problem or does it have some chain rule component to it?

\textcolor{red}{Question: In the derivations from our paper the $\delta p(t)$ (or $\tilde{p}$ in the manuscript) always cancelled itself out so I did not need to implement/code it anywhere. When I implement this Taylor test, what do I do for $\delta p$?}


\end{document}