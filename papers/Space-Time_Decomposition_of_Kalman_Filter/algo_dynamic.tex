
\documentclass[10pt]{article}
\usepackage[explicit]{titlesec}
\setlength{\parindent}{0pt}
\setlength{\parskip}{1em}
\usepackage{hyphenat}
\usepackage{ragged2e}
\usepackage{amsfonts,latexsym}
\usepackage{epsfig}
\usepackage{amsmath}
\usepackage{graphicx}
\usepackage{caption}
\usepackage{subfig}
\usepackage{sidecap}
\usepackage{caption}
\usepackage{amsmath}
\usepackage{amsfonts}
\usepackage{amssymb}
\usepackage{mathtools}
\DeclarePairedDelimiter{\ceil}{\lceil}{\rceil}
\RaggedRight

% These commands change the font. If you do not have Garamond on your computer, you will need to install it.
%\usepackage{garamondx}
\usepackage[T1]{fontenc}
\usepackage{amsmath, amsthm}
\usepackage{graphicx}

% This adjusts the underline to be in keeping with word processors.
\usepackage{soul}
\usepackage{mathtools}
\setul{.6pt}{.4pt}



% The following sets margins to 1 in. on top and bottom and .75 in on left and right, and remove page numbers.
\usepackage{geometry}
\geometry{vmargin={1in,1in}, hmargin={.75in, .75in}}
\usepackage{fancyhdr}
\pagestyle{fancy}
\pagenumbering{gobble}
\renewcommand{\headrulewidth}{0.0pt}
\renewcommand{\footrulewidth}{0.0pt}

% These Commands create the label style for tables, figures and equations.
\usepackage[labelfont={footnotesize,bf} , textfont=footnotesize]{caption}
\captionsetup{labelformat=simple, labelsep=period}
\newcommand\num{\addtocounter{equation}{1}\tag{\theequation}}
\renewcommand{\theequation}{\arabic{equation}}
\makeatletter
\renewcommand\tagform@[1]{\maketag@@@ {\ignorespaces {\footnotesize{\textbf{Equation}}} #1.\unskip \@@italiccorr }}
\makeatother
\setlength{\intextsep}{10pt}
\setlength{\abovecaptionskip}{2pt}
\setlength{\belowcaptionskip}{-10pt}

\renewcommand{\textfraction}{0.10}
\renewcommand{\topfraction}{0.85}
\renewcommand{\bottomfraction}{0.85}
\renewcommand{\floatpagefraction}{0.90}

% These commands set the paragraph and line spacing
\titleformat{\section}
  {\normalfont}{\thesection}{1em}{\MakeUppercase{\textbf{#1}}}
\titlespacing\section{0pt}{0pt}{-10pt}
\titleformat{\subsection}
  {\normalfont}{\thesubsection}{1em}{\textit{#1}}
\titlespacing\subsection{0pt}{0pt}{-8pt}
\renewcommand{\baselinestretch}{1.15}

% This designs the title display style for the maketitle command
\makeatletter
\newcommand\sixteen{\@setfontsize\sixteen{16pt}{6}}
\renewcommand{\maketitle}{\bgroup\setlength{\parindent}{0pt}
\begin{flushleft}
\vspace{-.375in}
\sixteen\bfseries \@title
\medskip
\end{flushleft}
\textit{\@author}
\egroup}
\makeatother

% This styles the bibliography and citations.
%\usepackage[biblabel]{cite}
\usepackage[sort&compress]{natbib}
\setlength\bibindent{2em}
\makeatletter
\renewcommand\@biblabel[1]{\textbf{#1.}\hfill}
\makeatother
%\renewcommand{\citenumfont}[1]{\textbf{#1}}
\bibpunct{}{}{,~}{s}{,}{,}
\setlength{\bibsep}{0pt plus 0.3ex}




%%%%%%%%%%%%%%%%%%%%%%%%%%%%%%%%%%%%%%%%%%%%%%%%%

% Authors: Add additional packages and new commands here.  
% Limit your use of new commands and special formatting.

% Place your title below. Use Title Capitalization.
\begin{document}

\begin{table}[ht!]
\label{tab}
\begin{tabular}{l}
\hline
\textbf{Procedure Dynamic load balancing}(in: $p$, $\Omega$, out: $l_1$,\ldots,$l_p$) \\

\%Procedure Dynamic load balancing allows to balance observations between adjacent subdomains \\
\% i.e. equal number of observations in each subdomains.\\ 
\% The domain $\Omega$ must be able to be decomposed in $p$ adjacent subdomains and it is not necessary \\ \% that all subdomains have observations.\\
\% The procedure is composed by: DD step, Scheduling step and Migration Step.\\ 
\% DD step partitions the domain $\Omega$ in subdomains and if some subdomains haven't observations,\\ 
\% repartitions the adjacent subdomain with maximum load in 2 subdomains and redefines the subdomains. \\ 
\% The Scheduling step
associates a processor to a subdomain and computes exact amount \\ 
\% of observations that it should send to (or receive from) its neighbouring subdomains\\
\% (by using processors graph). During the Migration step each processor decides which\\ 
\%observations it should send to or receive from it neighbouring subdomains. \\
\% The Update DD step redefines the subdomains.

\\

\textbf{DD step}\\ 
\% DD step partition the domani $\Omega$ in 
$\Omega\leftarrow (\Omega_{1},\Omega_{2},\ldots,\Omega_{p})$  \% initial partition of domain\\
\textbf{Define} number of adjacent subdomains: $n_{i}$ \\
\textbf{for} $i=1,p$\\
\textbf{Define} amount of observations in $\Omega_i$: $l_i$\\
\textbf{repeat}\\
\% identification of the adjacent observations subdomain  $\Omega_j$ with the maximum load\\
\textbf{Compute} maximum amount of observations: $l_m=max_{j=1,\ldots, n_{i}}\ (l_{j})$\\ 

\textbf{Define} DD of subdomain $\Omega_m$ in 2 subdomains: $\Omega_m\leftarrow (\Omega_m^{1},\Omega_m^{2})$  \\
\textbf{Redefine} subdomain $\Omega_i$\\
\textbf{until} ($l_i\neq 0$)\\
\textbf{endfor}\\
\textbf{end DD Step}\\
\textbf{Begin Scheduling step}\\
\textbf{Define} processor graph associated with initial partition $G$: $i\leftarrow Y_i$ \% vertex $i$ corresponds to processor $P_i$\\
\textbf{Distribute} amount of observations $l_i$ on each processor $P_i$\\
\textbf{Define} degree of node $i$ of $G$: $deg(i)=n_i$ \\
\textbf{repeat}\\
\textbf{Compute} average load : $\bar{l}=\frac{\sum_{i=1}^{p}l_i}{p}$\\
\textbf{Compute} load imbalance: $b={(l_i-\bar{l})_{i=1,\ldots,p}}$\\
\textbf{Compute} Laplacian matrix of processor graph $G$: $L$\\
\textbf{Call} Par-solve(in:$L,b$, out:$\lambda_i$) \% parallel implementation of algorithm to solve linear system\\
\textbf{Send} Lagrange multiplier $\lambda_i$ to adjacent subdomains $\Omega_j$, $j=1,\ldots,n_i$\\
\textbf{Receive} Lagrange multipliers $\lambda_i$ form adjacent subdomains $\Omega_j$, $j=1,\ldots,n_i$\\
\textbf{Compute} amount of observations to send to adjacent subdomains $\delta_{i,j}=\ceil[\big]{(\lambda_i-\lambda_j)}$\\ \% i.e. the smallest integer $\ge (\lambda_i-\lambda_j)_{i=1,\ldots,p}$\\
\textbf{Define} number of subdomains to send to $P_i$ and to receive from $P_i$: $n_{s_i},n_{r_i}$\\ 
\textbf{Update} new amount of observations of $P_i$:  $l_i=l_i-\sum_{j=1}^{n_{s_i}}\delta_{i,j}+\sum_{j=1}^{n_{r_i}}\delta_{j,i}$\\
\textbf{until} $(max |l_i-\bar{l}|=\frac{deg(i)}{2})$ \% i.e. maximum load-difference is $deg(i)/2$\\
\textbf{end Scheduling step}\\
\textbf{Begin Migration Step}\\
\textbf{Send and Receive} amount of observations from processor $P_i$ to processor $P_j$ and inversely.\\
\textbf{end Migration Step}\\
\textbf{Update} DD of $\Omega$\\
\textbf{end Procedure  Dynamic load balancing}\\
\hline
\end{tabular}
\end{table}


%state TRUE indicates that we should not update the local solution on $\Omega_i$

\end{document}