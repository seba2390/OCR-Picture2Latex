\section{The Vortex Lattice Method}
\label{sec:vlm}
The vortex lattice method (VLM) is a numerical method used to solve the three-dimensional potential-flow lifting-surface problem. Its primary advantage over simpler methods such as the Weissinger method or lifting-line theory is that it represents the actual camber shape of the wing, and therefore can be used for a wide variety of cambered wings and planforms. The implementation used in this work is described in detail by Katz and Plotkin \cite{katz_plotkin_book_1991}.

The geometry is first condensed into a camber surface, which is then discretized into a lattice of panels in the chordwise and spanwise directions. A vortex ring element is assigned to each panel, such that the leading segment of the vortex ring lies at the quarter-chord point of the panel. The trailing segment coincides with the leading segment of the next vortex ring. A collocation point is defined at the three-quarter chord line of the panel. The normal vector to the camber surface at the collocation point ($\hat{n}$) is used to enforce the boundary condition of zero normal flow through the surface. In the current implementation, a steady wake model is used -- the wake is assumed to be flat (no roll-up) and fixed (no change due to angle of attack), and extends to downstream infinity. This wake shape is modeled using horseshoe vortices to discretize the wake. An illustration of the discretization is shown in \fref{fig:vlm-panels}.

\begin{figure}[!htbp]
    \centering
    \includegraphics{figs/eps_fig/vlm-discretization.eps}
    \caption{The lattice of vortex ring elements used to discretize a lifting surface in the VLM}
    \label{fig:vlm-panels}
\end{figure}

At each collocation point, the %Dirichlet
boundary condition of zero normal flow through the camber surface is imposed. This boundary condition can be expressed as:
\begin{align}
    \left(\sum \vec{V}\right) \cdot \hat{n} = 0 \label{eqn:boundary-condition}
\end{align}


\noindent where $\sum \vec{V}$ is the vector sum of all velocities acting at the collocation point. This includes the incoming wind ($\vec{V}_\infty$), the velocities induced at the collocation point due to the vorticity bound to the wing and shed in the wake, velocities induced by any free vortices or vorticity bound to a different surface, and any velocity due to rotation of the body itself.
%
% \subsection{Vortex-Induced Velocity }
% Consider the finite vortex segment shown in \fref{fig:vortex-induced-vel}, extending from the starting point $\vec{s}$ to the endpoint $\vec{e}$, having a strength $\Gamma$.
% The velocity induced at a point $\vec{p}$ by this vortex is given by
% \begin{align}
%     \vec{V}_{p,\text{seg}} &= \frac{\Gamma}{4 \pi} \frac{\vec{g} \times \vec{h}}{\left\vert \vec{g} \times \vec{h} \right\vert^2 } \left(|\vec{g}| + |\vec{h}|\right) \left(1 - \frac{\vec{g} \cdot \vec{h}}{|\vec{g}| |\vec{h}|}\right) \\
%     &= \frac{\Gamma}{4 \pi} \vec{A}_{p,\text{seg}}
% \end{align}

% \noindent where $\vec{g}$ and $\vec{h}$ are the position vectors of the vortex start- and end-points with respect to the point at which the induced velocity is to be calculated.

% \begin{align}
%     \vec{g} &= \vec{s} - \vec{p} \\
%     \vec{h} &= \vec{e} - \vec{p}
% \end{align}

% \begin{figure}[!h]
%     \centering
%     \includegraphics{figs/eps_fig/vortex-vel.eps}
%     \caption{Velocity induced at a point $\vec{p}$ by a finite vortex segment}
%     \label{fig:vortex-induced-vel}
% \end{figure}

% The velocity induced at a point by a vortex ring of strength $\Gamma$ is simply the sum of the velocities induced by each leg of the vortex ring.
% \begin{align}
%     \vec{V}_{p,\text{ring}} &= \sum_{i=1}^{n_\text{seg}} \vec{V}_{p, i} \\
%     &= \frac{\Gamma}{4 \pi} \sum_{i=1}^{n_\text{seg}} \vec{A}_{p, i} = \frac{\Gamma}{4 \pi} \vec{\boldsymbol{A}}_{p,\text{ring}}
% \end{align}

% % q1 = (
% %         axb
% %         / vectorop.rowwise_dot(axb, axb)
% %         * (nA + nB)
% %         * (1 - vectorop.rowwise_dot(rA, rB) / (nA * nB))
% %     )

% \subsection{Calculation of Aerodynamic Influence Coefficients}
% \label{ssec:aic-calc}
% To simplify the calculation of velocity induced due to bound and wake vorticity at every collocation point, the problem may be expressed using aerodynamic influence coefficients. The velocity induced at the $i$th collocation point by the $j$th vortex ring by given by $V_{i,j} = \Gamma_j/4\pi \cdot \vec{\boldsymbol{A}}_{i,j}$. The contribution of this induced velocity to the normal velocity at the collocation point is given by:

% \begin{align}
%     V_{i,j}^\text{norm} &= \vec{V}_{i,j} \cdot \hat{n}_i = \frac{\Gamma_j}{4 \pi} \vec{\boldsymbol{A}}_{i,j} \cdot \hat{n}_i \\
%     &= \Gamma_j~a_{i,j}
% \end{align}

% Here, $a_{i,j} = (\vec{\boldsymbol{A}}_{i,j} \cdot \hat{n}_i) / (4 \pi)$ is the influence coefficient of the $j$th vortex ring at the $i$th collocation point. The normal components of the induced velocities at all $K$ collocation points due to every vortex ring can then be written in matrix form. This matrix is called the aerodynamic influence coefficient matrix, denoted by $[AIC]$.

% \begin{align}
%     \{V_\text{ind}^\text{norm}\} &= \left( \begin{array}{cccc}
%         a_{1,1} & a_{1,2} & \cdots & a_{1,K}  \\
%         a_{2,1} & a_{2,2} & \cdots & a_{2,K}  \\
%         \vdots & \vdots & \ddots & \vdots \\
%         a_{K,1} & a_{K,2} & \cdots & a_{K,K}  \\
%     \end{array} \right) \left\{ \begin{array}{c}
%          \Gamma_1  \\
%          \Gamma_2 \\
%          \vdots \\
%          \Gamma_K
%     \end{array} \right\} \\
%     &= [AIC] \{\Gamma\}
% \end{align}

% \subsubsection{Kutta Condition}

% The Kutta condition states that the flow leaving a sharp trailing edge of an airfoil or wing must leave parallel to the surface since it is not physical for a fluid to turn around a sharp edge. In terms of circulation, this implies that there must be no vorticity at the sharp trailing edge, i.e. the strength of the vortex segment at the trailing edge must be zero. This is achieved in steady flow by setting the strengths of the horseshoe vortices in the wake to be equal to the strengths of the last chordwise bound vortex rings. Therefore, the influence of the wake vortices can be included in the AIC matrix by adding the influence coefficient of each wake vortex to the influence coefficient of its corresponding bound vortex.

% \subsection{Linear Set of Equations}

% Now, the boundary condition given in \eref{eqn:boundary-condition} can be written as a set of linear equations given by
% \begin{align}
%     \left( \begin{array}{cccc}
%         a_{1,1} & a_{1,2} & \cdots & a_{1,K}  \\
%         a_{2,1} & a_{2,2} & \cdots & a_{2,K}  \\
%         \vdots & \vdots & \ddots & \vdots \\
%         a_{K,1} & a_{K,2} & \cdots & a_{K,K}  \\
%     \end{array} \right) \left\{ \begin{array}{c}
%          \Gamma_1  \\
%          \Gamma_2 \\
%          \vdots \\
%          \Gamma_K
%     \end{array} \right\} + \left(\vec{V}_\infty + \vec{V}_\text{rot} + \vec{V}_\text{misc} \right) \cdot \left\{ \begin{array}{c}
%          \hat{n}_1  \\
%          \hat{n}_2 \\
%          \vdots \\
%          \hat{n}_K
%     \end{array} \right\} = 0
% \end{align}

% \noindent where $\vec{V}_\text{misc}$ denotes known velocities from any other sources acting at the collocation points. This linear system can be solved to obtain the bound and wake vortex strengths for a given geometry when the right-hand-side of \eref{eqn:vlm-linsys} is known.

% \begin{align}
%     [AIC] \{\Gamma\} = - \left(\vec{V}_\infty + \vec{V}_\text{rot} + \vec{V}_\text{misc}\right) \cdot \{ \hat{n} \} \label{eqn:vlm-linsys}
% \end{align}

% \subsection{Calculation of Forces}
% Similar to the AIC matrix described earlier, an induced velocity matrix can be set up to easily calcuate the induced velocities $w_{ind}$ along each coordinate axis at the vortex segments that make up the vortex rings.
% \begin{align}
%     \left\{ \begin{array}{c}
%          w_{1,ind}  \\
%          w_{2,ind}  \\
%          \vdots \\
%          w_{4K,ind}  \\
%     \end{array} \right\}_{(x|y|z)} &= \left( \begin{array}{cccc}
%          b_{1,1} & b_{1,2} & \cdots & b_{1,K}  \\
%          b_{2,1} & b_{2,2} & \cdots & b_{2,K}  \\
%          \vdots & \vdots & \ddots & \vdots \\
%          b_{4K,1} & b_{4K,2} & \cdots & b_{4K,K}  \\
%     \end{array} \right)_{(x|y|z)} \left\{ \begin{array}{c}
%          \Gamma_1  \\
%          \Gamma_2 \\
%          \vdots \\
%          \Gamma_K
%     \end{array} \right\}
% \end{align}

% In potential flow, the vector force per unit length $\vec{F}$ acting on a vortex segment having circulation $\Gamma$ is given by the Kutta-Joukowski theorem.
% \begin{align}
%     \vec{F} = \rho \vec{V} \times \vec{\Gamma}
% \end{align}
% \noindent where $\vec{V}$ is the velocity passing through the vortex segment, and the vector $\vec{\Gamma}$ is a vector of magnitude $\Gamma$ in the direction of the vortex segment.
% Therefore, the force on a segment $k$ is given by
% \begin{align}
%     \vec{F}_k &= \rho \left(\vec{V}_\infty + \vec{V}_\text{rot} + \vec{V}_\text{misc} + \vec{w}_{k, ind} \right) \times \vec{\Gamma}_k
% \end{align}

% The vector sum of these forces gives the total force on the wing. The moment about a point can be obtained by taking the sum of each force multiplied by its moment arm. The appropriate components of the net forces and moments can then be nondimensionalized using the dynamic pressure and planform area to obtain the coefficients of lift, drag, and moment.

Since the geometry remains unchanged even with varying angles of attack, the calculation can be simplified using an aerodynamic influence coefficient matrix, $[AIC]$. With the geometry discretized into $M$ chordwise and $N$ spanwise panels, the AIC matrix becomes a square matrix of order ($M \times N$), with each element $a_{i,j}$ specifying the influence on the collocation point of the $i$th panel of the $j$th bound vortex ring.

\begin{align}
    [AIC] \left\{ \Gamma \right\} &= \left\{ RHS \right\} \label{eqn:vortex-lattice-eqn}
\end{align}

The RHS is a known column vector, with each element $r_{i}$ denoting the velocity due to all velocities not arising due to vortex-lattice influences, including $V_\infty$, rotational velocities. Solving \eref{eqn:vortex-lattice-eqn} yields a vector containing the circulation strengths of each vortex ring, which can be used to calculate the loads on the wing using the Kutta-Joukowski theorem.
The lift calculated using the VLM is corrected for thickness effects using an empirical equation given by Katz and Plotkin \cite{katz_plotkin_book_1991}.
\begin{align}
    C_{l, \text{corrected}} = \left[1 + 0.77 (t/c)_\text{max}\right] C_{l}
\end{align}

\section{The Decambering Method}
\label{sec:decambering}
% Potential-flow methods accurately predict the lift and moment characteristics of an airfoil at low angles of attack. As the angle of attack increases, viscous effects start to dominate and potential-flow methods no longer predict the loads correctly. At high angles of attack, the formation of an adverse pressure gradient on the upper surface causes the boundary layer to thicken and then separate from the surface.
The %resulting
change in the shape of the effective body due to boundary-layer separation at high angles of attack causes a reduction in the camber of the airfoil (``decambering''), leading to a drop in lift and moment associated with stall.
The decambering method \cite{Mukherjee_poststall_2006} developed in previous research at NCSU models the reduction in camber at high angles of attack using a linear ``decambering flap'' at the trailing edge of the airfoil, hinged at a fixed chordwise location, along with another decambering flap hinged at the leading edge.
\revnote{On applying the decambering flap, the zero-lift $\alpha$ of the airfoil is changed, with the lift-curve slope remaining unchanged.}{\#3.9}
A potential flow method can then be used to calculate the loads on the modified airfoil. 

The decambering method is easily applied to three-dimensional wings using a strip-theory approach. The wing is divided into chordwise strips with each strip assumed to behave like an airfoil. Decambering flaps are applied to each strip so as to fulfil the following conditions:

\label{sec:decambering-conditions}
\emph{Condition 1:} There is no normal flow through the strip, achieved by imposing the zero-flow boundary condition  normal to the decambered geometry at the collocation points of the vortex lattice. This condition is enforced by solving the linear system of the VLM to obtain the correct circulation strengths for the bound and wake vortices.

\emph{Condition 2:} The operating points for each strip after decambering, given by ($\aeff, C_l$) and ($\aeff$, $C_m$), fall on the $C_l$-$\alpha$ and $C_m$-$\alpha$ curves of the airfoil. This condition is satisfied by the deflection of the decambering flap.

An iterative process is used to enforce both conditions simultaneously. This approach has been shown to satisfactorily predict the lift generated by finite wings \cite{Mukherjee_poststall_2006,Paul_Gopa_Iteration_Schemes,gopalarathnam_paul_petrilli_ASM_2012}. The following section gives a detailed illustration of the modified nonlinear decambering procedure presented in the current work.
