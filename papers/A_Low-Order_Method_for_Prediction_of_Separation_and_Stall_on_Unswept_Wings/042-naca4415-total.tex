\subsubsection{NACA4415 Wings: Case C}

\fref{fig:n4415-coeffs}
% \multiref{f}{fig:n4415-ar8-coeffs}{fig:n4415-ar16-coeffs}
shows the variation of $C_L, C_D$, and  $C_M$ vs. $\alpha$ for the NACA4415 wings.
At low angles of attack, the lift predictions from the inviscid and viscous low-order methods agree well with CFD results.
As the angle of attack increases, the inviscid method does not model the effects of flow separation, and hence the predicted $C_L$ is unsurprisingly higher than the viscous $C_L$ obtained from CFD solutions.
The $C_L$ results from the viscous LOM, however, match CFD results excellently.
The stall angle and the drop in $C_L$ after stall is predicted well for all aspect ratios.
As the angle of attack is increased further beyond $\alpha = 25\degree$, there is massively separated flow on the upper surface of the wings. The flow at such high angles of attack is inherently unsteady, with large stall cells and leading-edge vortex shedding present in the CFD solutions. The low-order method does not model these phenomena, but the predicted coefficients show acceptable agreement with the CFD solutions.

% \begin{figure}[!h]
%         \centering
%         \includegraphics[width=\figwidth]{figs/eps_fig/naca4415_ar8_swp0_coeffs.eps}
%         \caption{Total coefficients of lift, drag, and pitching moment  for the NACA4415 $\ar{8}$ wing\label{fig:n4415-ar8-coeffs}}
%     \end{figure}%

% % \end{figure*}
% % \begin{figure*}[htb]\ContinuedFloat
% %     \centering
%     \begin{figure}[!h]
%         \centering
%         \includegraphics[width=\figwidth]{figs/eps_fig/naca4415_ar12_swp0_coeffs.eps}
%         \caption{Total coefficients of lift, drag, and pitching moment  for the NACA4415 $\ar{12}$ wing\label{fig:n4415-ar12-coeffs}}
%     \end{figure}%

% % \end{figure*}
% % \begin{figure*}[h]\ContinuedFloat
% %     \centering
%     \begin{figure}[!h]
%         \centering
%         \includegraphics[width=\figwidth]{figs/eps_fig/naca4415_ar16_swp0_coeffs.eps}
%         \caption{Total coefficients of lift, drag, and pitching moment  for the NACA4415 $\ar{16}$ wing\label{fig:n4415-ar16-coeffs}}
%     \end{figure}


% \begin{figure*}
%     \centering
%     \begin{subfigure}[t]{\textwidth}
%         \centering
%         \includegraphics[width=\figwidth]{figs/eps_fig/naca4415_ar8_swp0_coeffs.eps}
%         \caption{$\ar{8}$\label{sfig:n4415-ar8-coeffs}}
%     \end{subfigure}%

% % \end{figure*}
% % \begin{figure*}[htb]\ContinuedFloat
% %     \centering
%     \begin{subfigure}[t]{\textwidth}
%         \centering
%         \includegraphics[width=\figwidth]{figs/eps_fig/naca4415_ar12_swp0_coeffs.eps}
%         \caption{$\ar{12}$\label{sfig:n4415-ar12-coeffs}}
%     \end{subfigure}%

% % \end{figure*}
% % \begin{figure*}[h]\ContinuedFloat
% %     \centering
%     \begin{subfigure}[t]{\textwidth}
%         \centering
%         \includegraphics[width=\figwidth]{figs/eps_fig/naca4415_ar16_swp0_coeffs.eps}
%         \caption{$\ar{16}$\label{sfig:n4415-ar16-coeffs}}
%     \end{subfigure}
%     \caption{Total coefficients of lift, drag, and pitching moment  for the NACA4415 wings (Cases A--C)}
%     \label{fig:n4415-coeffs}
% \end{figure*}
\begin{figure}[!h]
    \centering
    \includegraphics[width=\figwidth]{figs/eps_fig/naca4415_coeffs_all.eps}
    \caption{Total coefficients of lift, drag, and pitching moment  for the NACA4415 wings (Cases C\textsubscript{1}--C\textsubscript{3})}
    \label{fig:n4415-coeffs}
\end{figure}


As seen with the lift comparisons, the drag predictions from both methods match CFD results well at low angles of attack.
Interestingly, the drag for the $\ar{8}$ wing is predicted well by the inviscid and viscous low-order methods even at high $\alpha (\approx 20\degree)$ where significant flow separation exists. This is because induced drag, which is predicted well by the inviscid method, is the major contributor to the total drag for the lower aspect ratios. % where the wingtip vortices affect the flow over a large portion of the wing.
As the aspect ratio increases, the induced drag is supplemented by profile drag. This increase in drag is accurately predicted by the viscous LOM.


There is a significant discrepancy in the prediction from the inviscid method for pitching moment even at low $\alpha$. %\PHNote{explanation?}.
This discrepancy is rectified by the decambering method, and the viscous LOM prediction agrees well with CFD. As the angle of attack increases, the viscous LOM accurately predicts the moment break and the angle at which this occurs.
Comparing the results for the three aspect ratios, it can be observed that the viscous LOM predictions generally improve as the aspect ratio increases. This trend occurs because the behavior of the higher $\ar{}$ wings is closer to that of the airfoil.
