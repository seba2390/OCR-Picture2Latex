\section{Results}
\label{sec:unswept-results}

The low-order method (LOM) described above was tested for multiple unswept wing geometries using viscous airfoil data, obtained from wind-tunnel experiments and from 2D RANS CFD solutions, as input. The \methodname was tested using airfoils of various maximum thickness for rectangular and tapered wing planforms. In addition to steady flight conditions, the \methodname was tested in the quasi-steady regime by applying a small, constant roll-rate to the wings. Results from the \methodabbr are compared against experimental observations from literature and 3D RANS CFD solutions for the geometries and flight conditions listed in \tref{tab:geoms}

\begin{table}[h!]
\centering
\caption{Summary of cases presented}
\label{tab:geoms}
\renewcommand{\arraystretch}{1.5}
\begin{tabular}{  c  c  c  c  c  }
\hline
\hline
 Case & Airfoil & Aspect Ratio & $Re$ & Notes  \\
 \hline
 A\textsubscript{1} & \multirow{3}{*}{NACA 4418} & 6 & \multirow{3}{*}{$0.75\times10^6$} & \multirow{3}{*}{\shortstack{Experimental verification \\[10pt] Source: Ref. \cite{naik_ostowari_nrel} }}\\
 A\textsubscript{2} &  & 9 &  &  \\
 A\textsubscript{3} & & 12 &  &  \\
 \cline{1-5}
 B\textsubscript{1}  & \multirow{2}{*}{NACA 0012} & 8 & \multirow{2}{*}{$3\times10^6$} & {Symmetric airfoil} \\
 B\textsubscript{2}  &  & 12 & & CFD verification \\
 \cline{1-5}
 C\textsubscript{1} & \multirow{3}{*}{NACA 4415} & 8 & \multirow{3}{*}{$3\times10^6$} & \multirow{3}{*}{\shortstack{Cambered airfoil\\[10pt]CFD verification}} \\
 C\textsubscript{2} &  & 12 & & \\
 C\textsubscript{3} &  & 16 & & \\
 \cline{1-5}
%  D & \multirow{2}{*}{NACA 0015} & 8 & \multirow{2}{*}{--} \\
%  E &  & 12 & \\
%  \cline{1-4}
D\textsubscript{1}  & NACA 4415 & 12 & $3\times10^6$ & Tapered wing ($\lambda = 0.5$) \\
 \cline{1-5}
%  E\textsubscript{1}  & \multirow{3}{*}{NACA 4415} & 8 & $pb/2V = 1.369\times{10}^{-2}$ \\
E\textsubscript{1}  & \multirow{2}{*}{NACA 4415} & 12 & \multirow{2}{*}{$3\times10^6$} & Rolling wing; Rectangular planform\\
E\textsubscript{2}  &  & 12 &  & Rollling wing; Tapered planform, $\lambda = 0.5$\\
 \hline
 \hline
\end{tabular}
\end{table}


% \PHNote{Change this paragraph to reflect the new world order}

{The lifting surface for each geometry was calculated and discretized into a lattice of 20 spanwise and 40 chordwise panels. Since the decambering shape is implemented in the \methodabbr by rotating the normal vectors of the panels, insufficient chordwise discretization can cause problems with convergence. Therefore, a fairly large number of chordwise panels (when compared with traditional VLMs) is used to ensure sufficient sensitivity of the method to the decambering shape. The code, implemented in Python 3.6 and optimized using the NumPy 1.16.2 package compiled with the Intel MKL libraries to perform vectorized linear algebra operations, ran on an Apple MacBook Air (2.2Ghz Dual-Core Intel i7) in 8-12 minutes for all cases.}

\secref{sec:exp-results} presents case A, in which results are obtained from the low-order method using the 2D viscous input curves obtained in the wind tunnel experiments by  Ostowari and Naik \cite{naik_ostowari_nrel}.
These predictions are compared against 3D experimental results from the same source.

Results presented in subsequent sections were obtained using 2D viscous input curves obtained from 2D RANS simulations performed using ANSYS Fluent on the NCSU HPC cluster. Details about the CFD simulations are given in \secref{sec:cfd-meth}. %
%
% Results from the \methodabbr are shown along with the CFD solutions for the viscous airfoil curves used as input data in \secref{sec:airfoil-results}
The total wing loads ($C_L, C_D, C_M$) predicted by the \methodabbr are compared against CFD solutions in \secref{sec:total-loads} for a rectangular wing with a symmetric 12\% thick airfoil (Case B) and a cambered 15\% thick airfoil (Case C).
% \methodabbr results show excellent agreement with CFD results.
\secref{sec:spanwise-loads} compares the low-order predictions of spanwise distributions of $C_l$ and $C_m$, and separation lines for these wings against the CFD solutions.
% \PHNote{separation lines not yet plotted.}
The \methodname accounts for the effects of the separated boundary layer by ``decambering'' the wing sections, effectively changing its shape.
The decambered wing-sections are overlaid on the contour of $V/V_\infty$ in \secref{sec:dec-shape} to illustrate the resemblance of the decambered airfoil shape to the shape of the separated boundary layer.
The effectiveness of the \methodname in predicting the characteristics of tapered wings is demonstrated in \secref{sec:tapered-wings} for a wing with taper ratio $\lambda = 0.5$. \secref{sec:rolling-wings} illustrates the utility of the \methodname in quasi-steady cases, such as when a small, constant roll-rate is present, for both rectangular and tapered wings.
Finally, \secref{sec:limitations} presents results for a swept wing and discusses the limitations in the method and motivates efforts to develop a correction for swept geometries.

%\PHNote{Should we show results for the airfoils? Don't think it's necessary}

\subsection{Experimental Validation}
\label{sec:exp-results}
% \PHNote{Rewrite}
The 2D viscous operating curves used as input to the low-order method can be obtained experimentally or from computational solutions. Here, we present the results obtained from the \methodname for rectangular NACA 4418 wings of three different aspect ratios using input curves obtained from Ostowari ans Naik \cite{naik_ostowari_nrel}.
A key requirement for the nonlinear decambering approach is the knowledge of the separation location. Since the separation curve ($f$ vs. $\alpha$) for the airfoil is not usually easily available from experiments, Beddoes' method (\eref{eqn:beddoes-f}) is used to calculate an approximate separation curve.

% \tref{tab:exp-geoms} lists the wing geometries for which results are presented in this section along with the sources for the input (2D) curves and the 3D experimental comparison data.

% \begin{table}[h!]
%     \centering
%     \caption{Summary of cases presented for experimental verification}
%     \label{tab:exp-geoms}
%     \renewcommand{\arraystretch}{1.5}
%     \begin{tabular}{ | c | m{2in} | c | m{2in} | }
%         \hline
%         \hline
%         Case & Airfoil & Aspect Ratio & Notes \\
%         \hline
%         H & NACA 4418 & ... & 2D source: \cite{naik_ostowari_nrel}; 3D source: \cite{naik_ostowari_nrel}; $Re = 0.75\times10^6$ \\
%         \hline
%         I & Root: NACA 4424; Tip: NACA 4412 & 12.06 &
%         Taper ratio: 0.4; Washout: $3\degree$ ; 2D source: \cite{abbott}; 3D source: \cite{McVeigh1971}; $Re = 2.87\times 10^6$ \\
%         \hline
%     \end{tabular}
% \end{table}

Wind tunnel results used in Case A are from the experiments \revnote{which were}{\#1.3} performed by Ostowari and Naik \cite{naik_ostowari_nrel} in the Texas A\&M University wind tunnel. Reflection-plane models of various NACA 44XX family wings were used to obtain force and moment curves at angles of attack ranging from $-10\degrees{}$ to $110\degrees{}$.
Data was obtained at a range of Reynolds numbers for wings of aspect ratios 6, 9, and 12, and for a wing spanning the entire test section ($\ar = \infty$). The results shown here use the viscous curves at $Re = 0.75\times10^6$. The separation curve is calculated using \eref{eqn:beddoes-f} and the lift curve for the NACA 4418 airfoil given in Ref. \cite{naik_ostowari_nrel}.
It was seen that at certain angles of attack, \eref{eqn:beddoes-f} gives a value for $f$ that is greater than 1.
In such cases, the value of $f$ is simply set to 1. The experimentally obtained lift, drag, and moment, and calculated separation curves are shown in \fref{fig:exp-coeffs}.

\begin{figure}[!h]
    \centering
    \includegraphics[width=4in]{figs/eps_fig/naca4418_exp_input.eps}
    \caption{The viscous lift, drag, and moment curves for the NACA 4418 airfoil obtained from experimental tests by Naik and Ostowari \cite{naik_ostowari_nrel}, and the separation curve calculated using Beddoes' model}
    \label{fig:exp-coeffs}
\end{figure}

Using these input curves, the low-order method can predict the loads on 3D wings of various aspect ratios.
The predictions from the low-order method for the NACA 4418 wings of aspect ratios 6, 9, and 12 are shown in \fref{fig:exp-n4418-coeffs} %
%
\begin{figure*}[!h]
    \centering
    \includegraphics[width=\figwidth]{figs/eps_fig/naca4418_exp_coeffs_all.eps}
    \caption{Total coefficients of lift, drag, and pitching moment  for the NACA 4418 wings (Case A\textsubscript{1} -- A\textsubscript{3})}
    \label{fig:exp-n4418-coeffs}
\end{figure*}
% \begin{figure*}[!h]
%     \centering
%     \begin{subfigure}[t]{\textwidth}
%         \centering
%         \includegraphics[width=\figwidth]{figs/eps_fig/naca4418_ar6_swp0_coeffs.eps}
%         \caption{$\ar{6}$\label{sfig:n4418-ar6-coeffs}}
%     \end{subfigure}%

%     \begin{subfigure}[t]{\textwidth}
%         \centering
%         \includegraphics[width=\figwidth]{figs/eps_fig/naca4418_ar9_swp0_coeffs.eps}
%         \caption{$\ar{9}$\label{sfig:n4418-ar9-coeffs}}
%     \end{subfigure}%

%     \begin{subfigure}[t]{\textwidth}
%         \centering
%         \includegraphics[width=\figwidth]{figs/eps_fig/naca4418_ar12_swp0_coeffs.eps}
%         \caption{$\ar{12}$\label{sfig:n4418-ar12-coeffs}}
%     \end{subfigure}%

%     \caption{Total coefficients of lift, drag, and pitching moment  for the NACA4418 wings (Case J)}
%     \label{fig:exp-n4418-coeffs}
% \end{figure*} %
 %
The low-order predictions for lift and drag show excellent agreement with the experimentally obtained values.
The low-order method correctly predicts the increasing lift-curve slope with increase in aspect ratio at low angles of attack.
As the angle of attack increases, the viscous low-order method correctly predicts stall and the associated drop in $C_L$ and rise in $C_D$.
The maximum $C_L$ and stall angle predictions from the viscous low-order method are within 5\% of the experimental values.
The low-order prediction for moment agrees well with experimental result for the $\ar 12$ case.
However, we see that for the smaller aspect ratios, the low-order moment prediction starts to deviate from the experimental result and the error increases with decreasing aspect ratios. This is thought to be due to the interactions of the detached wingtip vortices with the wing, which become important at low aspect ratios but are not modeled by the current low-order method.


%\PHNote{Residual history?}

\subsection{CFD Methodology for 3D Wings}
\label{sec:cfd-meth}
The low-order method does not require any input data from 3-dimensional CFD solutions. However, 3D CFD solutions for each of the wing geometries described above were obtained to evaluate the accuracy of the low-order method. Using the procedure described in Ref. \cite{Jamwal2018}, body-conforming structured meshes having a wall $y^+ = 1$  were generated for each geometry using the multi-blocking Hexa algorithm in ANSYS ICEM-CFD. These meshes, having cell counts ranging from 20M--45M cells, were used to obtain time-accurate solutions at $Re = 3\times10^6$ in ANSYS Fluent. A physical timestep of 0.01s was used for a total of 300 timesteps. The Spalart-Allmaras model was used for turbulence closure.
\revnote{Time-accurate simulations were performed to ensure that the CFD solutions would converge at the high post-stall angles of attack where the inherently unsteady flow prevents steady simulations from converging. The CFD results shown in this work are the mean values of each quantity over one oscillation.}{\#3.14}
A detailed explanation of the CFD methodology is given in Ref. \cite{AbhimanyuThesis}. The total and spanwise load distributions are obtained from the CFD solutions for comparison with low-order predictions. The separation line is obtained from plots of skin-friction lines on the upper surface of the wing.


\newcommand{\clmax}{\ensuremath{C_{l, \text{max}}}}
\newcommand{\CLmax}{\ensuremath{C_{L, \text{max}}}}


% \subsection{Airfoil Behavior}
% \label{sec:airfoil-results}
% The nonlinear decambering method was first tested for geometries with $\ar = 10000$, which are practically identical to a two-dimensional airfoil of infinite aspect ratio. Predictions for the coefficients of lift, drag, and moment are shown in
% % \multiref{f}{fig:n4415-airfoil-coeffs}{fig:n0012-airfoil-coeffs}.
% \fref{fig:airfoil-coeffs}.
% The results presented are obtained from three sources:

% \begin{itemize}
%     \item \emph{CFD:} Results obtained from RANS CFD solutions.
%     \item{\emph{Inv. LOM:} Results obtained from the inviscid vortex lattice method, as described by Katz and Plotkin \cite{katz_plotkin_book_1991}}
%     \item \emph{Visc. LOM:} Results obtained from the VLM augmented with the nonlinear decambering method, which is the low-order method described in this chapter.
% \end{itemize}

% At low angles of attack, the results from the inviscid LOM adequately predict the lift and moment coefficients for the airfoils. Since potential-flow methods are incapable of predicting profile drag, the inviscid method gives $C_d = 0$ for all airfoils for all angles of attack. As the angle of attack is increased, the inviscid prediction starts deviating from the viscous CFD solution. The nonlinear decambering accurately recreates the drop in lift and moment and increase in drag for angles of attack up to 40\degree.


% \begin{figure*}[!h]
%     \centering
%     \begin{subfigure}[t]{\textwidth}
%         \centering
%         \includegraphics[width=\figwidth]{figs/eps_fig/naca4415_airfoil_coeffs.eps}
%         \caption{NACA 4415 airfoil\label{sfig:n4415-airfoil-coeffs}}
%     \end{subfigure}%

% % \end{figure*}
% % \begin{figure*}[htb]\ContinuedFloat
% %     \centering
%     \begin{subfigure}[t]{\textwidth}
%         \centering
%         \includegraphics[width=\figwidth]{figs/eps_fig/naca0015_airfoil_coeffs.eps}
%         \caption{NACA 0015 airfoil\label{sfig:n0015-airfoil-coeffs}}
%     \end{subfigure}%

% % \end{figure*}
% % \begin{figure*}[h]\ContinuedFloat
% %     \centering
%     \begin{subfigure}[t]{\textwidth}
%         \centering
%         \includegraphics[width=\figwidth]{figs/eps_fig/naca0012_airfoil_coeffs.eps}
%         \caption{NACA 0012 airfoil\label{sfig:n0012-airfoil-coeffs}}
%     \end{subfigure}
%     \caption{Coefficients of lift, drag, and pitching moment  for the NACA4415, NACA0015, and NACA0012 airfoils}
%     \label{fig:airfoil-coeffs}
% \end{figure*}


% However, this is not surprising -- since the viscous LOM is given the viscous $C_l$, $C_d$, and $C_m$ CFD curves as input, the accurate match with the input data is simply included as a sanity check. In the following sections, we use the two-dimensional input curves to obtain predictions from the viscous LOM for three-dimensional wings having various aspect ratios, taper ratios, and roll rates.

\subsection{Wing Lift, Drag, and Moment Predictions}
\label{sec:total-loads}

The total coefficients of lift, drag, and pitching moment  up to $\alpha=35\degree$ are presented below. For all these cases, viscous input curves for $C_l$, $C_d$, $C_m$, and $f$ vs. $\alpha$, shown in \fref{fig:cfd-coeffs}, were obtained from 2D CFD solutions at the appropriate Reynolds numbers.

\begin{figure}[!h]
    \centering
    \includegraphics[width=4in]{figs/eps_fig/naca0012_4415_cfd_input.eps}
    \caption{The viscous lift, drag, moment, and separation curves for the NACA 0012 and NACA 4415 airfoils obtained from 2D CFD soluions}
    \label{fig:cfd-coeffs}
\end{figure}

\subsubsection{NACA0012 Wings: Case B}
The low-order predictions of $C_L, C_D, C_M$ vs. $\alpha$ for the NACA0012 wings are shown in
% \multiref{f}{fig:n0012-ar8-coeffs}{fig:n0012-ar12-coeffs}
\fref{fig:n0012-coeffs}.
At low angles of attack, the inviscid low-order method correctly predicts the loads on the wings, and no additional decambering is required. As the angle of attack increases to $16\degrees{}$ and beyond, the wings begin to stall and the inviscid low-order method does not predict the associated drop in lift and moment, and increase in drag. The viscous \methodname is able to correctly predict these effects of stall.
% Both methods (inviscid and viscous) show satisfactory agreement with CFD solutions at low $\alpha$ for all three coefficients. With increasing $\alpha$, the inviscid method starts deviating from the CFD solution, while the results from the viscous method still match excellently.
% It is seen that the viscous LOM slightly overpredicts the stall angle, $C_{L,\text{stall}}$, and $C_{M,\text{stall}}$, but these results are still acceptable and a significant improvement over inviscid predictions. As seen in earlier cases, the viscous LOM shows better agreement with CFD for the higher aspect ratios.

    % \begin{figure}[!h]
    %     \centering
    %     \includegraphics[width=\figwidth]{figs/eps_fig/naca0015_ar8_swp0_coeffs.eps}
    %     \caption{Total coefficients of lift, drag, and pitching moment  for the NACA0015 $\ar{8}$ wing\label{fig:n0015-ar8-coeffs}}
    % \end{figure}%

    % \begin{figure}[!h]
    %     \centering
    %     \includegraphics[width=\figwidth]{figs/eps_fig/naca0015_ar12_swp0_coeffs.eps}
    %     \caption{Total coefficients of lift, drag, and pitching moment  for the NACA0015 $\ar{12}$ wing\label{fig:n0015-ar12-coeffs}}
    % \end{figure}%


% \begin{figure*}
%     \centering
%     \begin{subfigure}[t]{\textwidth}
%         \centering
%         \includegraphics[width=\figwidth]{figs/eps_fig/naca0015_ar8_swp0_coeffs.eps}
%         \caption{$\ar{8}$\label{sfig:n0015-ar8-coeffs}}
%     \end{subfigure}%

%     \begin{subfigure}[t]{\textwidth}
%         \centering
%         \includegraphics[width=\figwidth]{figs/eps_fig/naca0015_ar12_swp0_coeffs.eps}
%         \caption{$\ar{12}$\label{sfig:n0015-ar12-coeffs}}
%     \end{subfigure}%
%     \caption{Total coefficients of lift, drag, and pitching moment  for the NACA0015 wings (Cases D and E)}
%     \label{fig:n0015-coeffs}
% \end{figure*}

% \subsubsection{ }

    % \begin{figure}[!h]
    %     \centering
    %     \includegraphics[width=\figwidth]{figs/eps_fig/naca0012_ar8_swp0_coeffs.eps}
    %     \caption{Total coefficients of lift, drag, and pitching moment  for the NACA0012 $\ar{8}$ wing\label{fig:n0012-ar8-coeffs}}
    % \end{figure}%

    % \begin{figure}[!h]
    %     \centering
    %     \includegraphics[width=\figwidth]{figs/eps_fig/naca0012_ar12_swp0_coeffs.eps}
    %     \caption{Total coefficients of lift, drag, and pitching moment  for the NACA0012 $\ar{12}$ wing\label{fig:n0012-ar12-coeffs}}
    % \end{figure}%

    \begin{figure*}[!h]
        \centering
        \includegraphics[width=\figwidth]{figs/eps_fig/naca0012_coeffs_all.eps}
        \caption{Total coefficients of lift, drag, and pitching moment  for the NACA0012 wings (Cases B\textsubscript{1}--B\textsubscript{2})}
        \label{fig:n0012-coeffs}
    \end{figure*}

% \begin{figure*}[!h]
%     \centering
%     \begin{subfigure}[t]{\textwidth}
%         \centering
%         \includegraphics[width=\figwidth]{figs/eps_fig/naca0012_ar8_swp0_coeffs.eps}
%         \caption{$\ar{8}$\label{sfig:n0012-ar8-coeffs}}
%     \end{subfigure}%

%     \begin{subfigure}[t]{\textwidth}
%         \centering
%         \includegraphics[width=\figwidth]{figs/eps_fig/naca0012_ar12_swp0_coeffs.eps}
%         \caption{$\ar{12}$\label{sfig:n0012-ar12-coeffs}}
%     \end{subfigure}%
%     \caption{Total coefficients of lift, drag, and pitching moment  for the NACA0012 wings (Cases D and E)}
%     \label{fig:n0012-coeffs}
% \end{figure*}

\subsubsection{NACA4415 Wings: Case C}

\fref{fig:n4415-coeffs}
% \multiref{f}{fig:n4415-ar8-coeffs}{fig:n4415-ar16-coeffs}
shows the variation of $C_L, C_D$, and  $C_M$ vs. $\alpha$ for the NACA4415 wings.
At low angles of attack, the lift predictions from the inviscid and viscous low-order methods agree well with CFD results.
As the angle of attack increases, the inviscid method does not model the effects of flow separation, and hence the predicted $C_L$ is unsurprisingly higher than the viscous $C_L$ obtained from CFD solutions.
The $C_L$ results from the viscous LOM, however, match CFD results excellently.
The stall angle and the drop in $C_L$ after stall is predicted well for all aspect ratios.
As the angle of attack is increased further beyond $\alpha = 25\degree$, there is massively separated flow on the upper surface of the wings. The flow at such high angles of attack is inherently unsteady, with large stall cells and leading-edge vortex shedding present in the CFD solutions. The low-order method does not model these phenomena, but the predicted coefficients show acceptable agreement with the CFD solutions.

% \begin{figure}[!h]
%         \centering
%         \includegraphics[width=\figwidth]{figs/eps_fig/naca4415_ar8_swp0_coeffs.eps}
%         \caption{Total coefficients of lift, drag, and pitching moment  for the NACA4415 $\ar{8}$ wing\label{fig:n4415-ar8-coeffs}}
%     \end{figure}%

% % \end{figure*}
% % \begin{figure*}[htb]\ContinuedFloat
% %     \centering
%     \begin{figure}[!h]
%         \centering
%         \includegraphics[width=\figwidth]{figs/eps_fig/naca4415_ar12_swp0_coeffs.eps}
%         \caption{Total coefficients of lift, drag, and pitching moment  for the NACA4415 $\ar{12}$ wing\label{fig:n4415-ar12-coeffs}}
%     \end{figure}%

% % \end{figure*}
% % \begin{figure*}[h]\ContinuedFloat
% %     \centering
%     \begin{figure}[!h]
%         \centering
%         \includegraphics[width=\figwidth]{figs/eps_fig/naca4415_ar16_swp0_coeffs.eps}
%         \caption{Total coefficients of lift, drag, and pitching moment  for the NACA4415 $\ar{16}$ wing\label{fig:n4415-ar16-coeffs}}
%     \end{figure}


% \begin{figure*}
%     \centering
%     \begin{subfigure}[t]{\textwidth}
%         \centering
%         \includegraphics[width=\figwidth]{figs/eps_fig/naca4415_ar8_swp0_coeffs.eps}
%         \caption{$\ar{8}$\label{sfig:n4415-ar8-coeffs}}
%     \end{subfigure}%

% % \end{figure*}
% % \begin{figure*}[htb]\ContinuedFloat
% %     \centering
%     \begin{subfigure}[t]{\textwidth}
%         \centering
%         \includegraphics[width=\figwidth]{figs/eps_fig/naca4415_ar12_swp0_coeffs.eps}
%         \caption{$\ar{12}$\label{sfig:n4415-ar12-coeffs}}
%     \end{subfigure}%

% % \end{figure*}
% % \begin{figure*}[h]\ContinuedFloat
% %     \centering
%     \begin{subfigure}[t]{\textwidth}
%         \centering
%         \includegraphics[width=\figwidth]{figs/eps_fig/naca4415_ar16_swp0_coeffs.eps}
%         \caption{$\ar{16}$\label{sfig:n4415-ar16-coeffs}}
%     \end{subfigure}
%     \caption{Total coefficients of lift, drag, and pitching moment  for the NACA4415 wings (Cases A--C)}
%     \label{fig:n4415-coeffs}
% \end{figure*}
\begin{figure}[!h]
    \centering
    \includegraphics[width=\figwidth]{figs/eps_fig/naca4415_coeffs_all.eps}
    \caption{Total coefficients of lift, drag, and pitching moment  for the NACA4415 wings (Cases C\textsubscript{1}--C\textsubscript{3})}
    \label{fig:n4415-coeffs}
\end{figure}


As seen with the lift comparisons, the drag predictions from both methods match CFD results well at low angles of attack.
Interestingly, the drag for the $\ar{8}$ wing is predicted well by the inviscid and viscous low-order methods even at high $\alpha (\approx 20\degree)$ where significant flow separation exists. This is because induced drag, which is predicted well by the inviscid method, is the major contributor to the total drag for the lower aspect ratios. % where the wingtip vortices affect the flow over a large portion of the wing.
As the aspect ratio increases, the induced drag is supplemented by profile drag. This increase in drag is accurately predicted by the viscous LOM.


There is a significant discrepancy in the prediction from the inviscid method for pitching moment even at low $\alpha$. %\PHNote{explanation?}.
This discrepancy is rectified by the decambering method, and the viscous LOM prediction agrees well with CFD. As the angle of attack increases, the viscous LOM accurately predicts the moment break and the angle at which this occurs.
Comparing the results for the three aspect ratios, it can be observed that the viscous LOM predictions generally improve as the aspect ratio increases. This trend occurs because the behavior of the higher $\ar{}$ wings is closer to that of the airfoil.

\subsection{Comparison of Spanwise Distributions of Lift and Moment}
\label{sec:spanwise-loads}

\fref{fig:n4415-ar12-cldist} shows the spanwise distributions of section lift coefficient ($C_l$) and pitching moment  ($C_m$) for the NACA 4415 $\ar{12}$ wing at angles of attack before stall ($\alpha=10\degree$), slightly post-stall ($\alpha=20\degree$), and well beyond stall ($\alpha=32\degree$) at which the boundary layer separates close to the leading edge on a large portion of the wing.
Before stall, the inviscid prediction is close to the viscous solution from CFD. As the angle of attack is increased, the wing stalls at the root. The drop in lift and moment on the inboard sections is predicted well by the viscous LOM. Well beyond stall, the low-order prediction agrees quite well with the CFD solution.
%The accuracy of the low-order prediction at the wingtips is affected by the wingtip vortices. \PHNote{this sentence...}
The large discrepancy between CFD and the low-order method at the wingtips is attributed to the absence of a model in the low-order method to predict the detachment of the tip vortices at higher angles of attack. This discrepancy becomes more evident for the lower aspect-ratio wings, for which the wingtip vortices affect flow over a considerable portion of the wing.

\begin{figure*}
    \centering
    \begin{subfigure}[t]{0.45\textwidth}
        \centering
        \includegraphics[width=0.9\textwidth]{{figs/eps_fig/spanwise-dist/naca4415_ar12_swp0_a10.0_cldist}.eps}
        \caption{$\alpha=10\degree$\label{sfig:4415-ar12-cldist-a10}}
    \end{subfigure}%
    ~
    \begin{subfigure}[t]{0.45\textwidth}
        \centering
        \includegraphics[width=0.9\textwidth]{{figs/eps_fig/spanwise-dist/naca4415_ar12_swp0_a10.0_cmdist}.eps}
        \caption{$\alpha=10\degree$\label{sfig:4415-ar12-cmdist-a10}}
    \end{subfigure}%

    \begin{subfigure}[t]{0.45\textwidth}
        \centering
        \includegraphics[width=0.9\textwidth]{{figs/eps_fig/spanwise-dist/naca4415_ar12_swp0_a20.0_cldist}.eps}
        \caption{$\alpha=20\degree$\label{sfig:4415-ar12-cldist-a20}}
    \end{subfigure}%
    ~
    \begin{subfigure}[t]{0.45\textwidth}
        \centering
        \includegraphics[width=0.9\textwidth]{{figs/eps_fig/spanwise-dist/naca4415_ar12_swp0_a20.0_cmdist}.eps}
        \caption{$\alpha=20\degree$\label{sfig:4415-ar12-cmdist-a20}}
    \end{subfigure}%

    \begin{subfigure}[t]{0.45\textwidth}
        \centering
        \includegraphics[width=0.9\textwidth]{{figs/eps_fig/spanwise-dist/naca4415_ar12_swp0_a32.0_cldist}.eps}
        \caption{$\alpha=32\degree$\label{sfig:4415-ar12-cldist-a32}}
    \end{subfigure}%
    ~
    \begin{subfigure}[t]{0.45\textwidth}
        \centering
        \includegraphics[width=0.9\textwidth]{{figs/eps_fig/spanwise-dist/naca4415_ar12_swp0_a32.0_cmdist}.eps}
        \caption{$\alpha=32\degree$\label{sfig:4415-ar12-cmdist-a32}}
    \end{subfigure}%

    \caption{Spanwise distributions of $C_l$ and $C_m$ at pre- and post-stall angles of attack from CFD (black), inviscid LOM (red), and viscous LOM (blue) for the NACA4415 $\ar{12}$ wing (Case C\textsubscript{2})}
    \label{fig:n4415-ar12-cldist}
\end{figure*}

\begin{figure}
    \centering
    \includegraphics[width=3in]{figs/eps_fig/spanwise-dist/naca4415_ar12_swp0_allalpha_fdist.eps}
    \caption{Separation line predicted by the LOM (blue) and CFD (black) for angles of attack ranging from pre-stall to post-stall (Case C\textsubscript{2}). Right half of the wing is shown.}
    \label{fig:naca4415_ar12_swp0_allalpha_fdist}
\end{figure}

\fref{fig:naca4415_ar12_swp0_allalpha_fdist} compares the separation line predicted by the \methodabbr with CFD predictions at the same angles of attack as above. At $\alpha=10\degree$ (well before stall), there is only a small amount of separation, indicated by the separation line being close to $f=1$ (trailing-edge) at all sections. The decambering method models the drop in lift using a trailing-edge flap hinged at the separation line. The other decambering parameters at a section ($\delta_l, m$) depend on the value of $f$. If $f \approx 1$, unphysically large values of $\delta_l$ and $m$ are required to model even a small drop in lift. Therefore, in the low-order method, the aft-most location of the hinge for the decambering flap is constrained to $f \leq 0.8$. As the angle of attack increases beyond stall ($20\degree$), we see significant separation over large portions of the wing. The undulations in the separation line from CFD indicate the presence of stall cells on the upper surface of the wing. While these stall cells are not predicted by the low-order method, the overall agreement of the predicted separation line with the CFD solution is remarkably good. As the angle of attack is increased well beyond stall ($\alpha=32\degree$), we see from the CFD solution that most of the wing experiences fully separated flow. This separation-line behavior is again predicted well by the low-order method. %
%
% \PHNote{What about other spanwise distributions? Should I just refer to my dissertation?}
Spanwise $C_l$, $C_m$, and $f$ distributions for the other geometries given in \tref{tab:geoms} agree similarly well with CFD solutions and are included in Ref. \cite{PranavThesis}.

\subsection{Comparison of Decambering Shape with CFD Velocity Contours}
\label{sec:dec-shape}

As the angle of attack increases, the separated boundary layer changes the effective shape of the wing.
The nonlinear decambering method models the effects of a separated boundary layer using a parabolic decambering flap to simultaneously achieve a drop in both lift and moment.
% \multiref{f}{fig:n4415-ar12-y1.5-blpics}{fig:n4415-ar12-y5.1-blpics}
% This observation is applied later in \cref{ch:wakepred} to estimate the location of the wake behind the wing, which is used by the \methodabbr to predict deep stall of downstream surfaces.
%
% \PHNote{Is this size good? or too small?}
\begin{figure}[!ht]
    \centering
    \begin{tabular}{p{0.38in} c c c c}
        %\hline
        %\hline
        %\diagbox[innerwidth=0.35in]{$\alpha$}{$2y/b$}
        $2y/b\rightarrow$ \\ $\alpha\downarrow$ & 0.25  & 0.45 & 0.65 & 0.85 \\
        %\hline
        \hfil$14\degree$ & \input{blplots/a1-y1.tex} & \input{blplots/a1-y2.tex} & \input{blplots/a1-y3.tex} & \begin{subfigure}[b]{0.20\textwidth}
    \centering
    \includegraphics[trim=2.5in 3.05in 2in 2.2in, clip, width=0.9\textwidth]{{figs/decam_lines/bl_vmag_a14_y5.1}.png}
    % \caption{$\alpha=14\degree$\label{sfig:4415-ar12-blvmag-a14-y5.1}}
\end{subfigure}% \\
        \hfil$18\degree$ & \input{blplots/a2-y1.tex} & \input{blplots/a2-y2.tex} & \input{blplots/a2-y3.tex} & \input{blplots/a2-y4.tex} \\
        \hfil$22\degree$ & \begin{subfigure}[b]{0.20\textwidth}
    \centering
    \includegraphics[trim=2.5in 3.05in 2in 2.2in, clip, width=0.9\textwidth]{{figs/decam_lines/bl_vmag_a22_y1.5}.png}
    % \caption{$\alpha=22\degree$\label{sfig:4415-ar12-blvmag-a22-y1.5}}
\end{subfigure}% & \input{blplots/a3-y2.tex} & \input{blplots/a3-y3.tex} & \input{blplots/a3-y4.tex} \\
        \hfil$26\degree$ & \input{blplots/a4-y1.tex} & \input{blplots/a4-y2.tex} & \input{blplots/a4-y3.tex} & \begin{subfigure}[b]{0.20\textwidth}
    \centering
    \includegraphics[trim=2.5in 3.05in 2in 2.2in, clip, width=0.9\textwidth]{{figs/decam_lines/bl_vmag_a26_y5.1}.png}
    % \caption{$\alpha=26\degree$\label{sfig:4415-ar12-blvmag-a26-y5.1}}
\end{subfigure}% \\
        %\hline
        %\hline
    \end{tabular}
    \caption{The decambered camberline (red) at spanwise stations of the NACA 4415 $\ar{12}$ wing (Case C\textsubscript{2}) overlaid on a contour plot of the velocity magnitude $V/V_\infty$ from pre-stall to post-stall angles of attack}
    \label{fig:n4415-blpics}
\end{figure}%
%
% \begin{figure*}[!h]
%     \centering
%     \begin{subfigure}[t]{0.22\textwidth}
%         \centering
%         \includegraphics[trim=2.5in 3.05in 2in 2.2in, clip, width=0.9\textwidth]{{figs/decam_lines/bl_vmag_a14_y1.5}.png}
%         \caption{$\alpha=14\degree$\label{sfig:4415-ar12-blvmag-a14-y1.5}}
%     \end{subfigure}%
%     ~
%     \begin{subfigure}[t]{0.22\textwidth}
%         \centering
%         \includegraphics[trim=2.5in 3.05in 2in 2.2in, clip, width=0.9\textwidth]{{figs/decam_lines/bl_vmag_a18_y1.5}.png}
%         \caption{$\alpha=18\degree$\label{sfig:4415-ar12-blvmag-a18-y1.5}}
%     \end{subfigure}%
%
%     \begin{subfigure}[t]{0.22\textwidth}
%         \centering
%         \includegraphics[trim=2.5in 3.05in 2in 2.2in, clip, width=0.9\textwidth]{{figs/decam_lines/bl_vmag_a22_y1.5}.png}
%         \caption{$\alpha=22\degree$\label{sfig:4415-ar12-blvmag-a22-y1.5}}
%     \end{subfigure}%
%     ~
%     \begin{subfigure}[t]{0.22\textwidth}
%         \centering
%         \includegraphics[trim=2.5in 3.05in 2in 2.2in, clip, width=0.9\textwidth]{{figs/decam_lines/bl_vmag_a26_y1.5}.png}
%         \caption{$\alpha=26\degree$\label{sfig:4415-ar12-blvmag-a26-y1.5}}
%     \end{subfigure}%
%
%     \caption{The decambered camberline (red) and separation point location (blue symbol) for the NACA 4415 $\ar{12}$ wing (Case A) overlaid on a contour plot of the velocity magnitude $V/V_\infty$ at an inboard section ($2y/b=0.25$) from pre-stall to post-stall angles of attack}
%     \label{fig:n4415-ar12-y1.5-blpics}
% \end{figure*}
%
% \begin{figure*}[!h]
%     \centering
%     \begin{subfigure}[t]{0.22\textwidth}
%         \centering
%         \includegraphics[trim=2.5in 3.05in 2in 2.2in, clip, width=0.9\textwidth]{{figs/decam_lines/bl_vmag_a14_y2.7}.png}
%         \caption{$\alpha=14\degree$\label{sfig:4415-ar12-blvmag-a14-y2.7}}
%     \end{subfigure}%
%     ~
%     \begin{subfigure}[t]{0.22\textwidth}
%         \centering
%         \includegraphics[trim=2.5in 3.05in 2in 2.2in, clip, width=0.9\textwidth]{{figs/decam_lines/bl_vmag_a18_y2.7}.png}
%         \caption{$\alpha=18\degree$\label{sfig:4415-ar12-blvmag-a18-y2.7}}
%     \end{subfigure}%
%
%     \begin{subfigure}[t]{0.22\textwidth}
%         \centering
%         \includegraphics[trim=2.5in 3.05in 2in 2.2in, clip, width=0.9\textwidth]{{figs/decam_lines/bl_vmag_a22_y2.7}.png}
%         \caption{$\alpha=22\degree$\label{sfig:4415-ar12-blvmag-a22-y2.7}}
%     \end{subfigure}%
%     ~
%     \begin{subfigure}[t]{0.22\textwidth}
%         \centering
%         \includegraphics[trim=2.5in 3.05in 2in 2.2in, clip, width=0.9\textwidth]{{figs/decam_lines/bl_vmag_a26_y2.7}.png}
%         \caption{$\alpha=26\degree$\label{sfig:4415-ar12-blvmag-a26-y2.7}}
%     \end{subfigure}%
%
%     \caption{The decambered camberline (red) and separation point location (blue symbol) for the NACA 4415 $\ar{12}$ wing (Case A) overlaid on a contour plot of the velocity magnitude $V/V_\infty$ at a middle section ($2y/b=0.45$) from pre-stall to post-stall angles of attack}
%     \label{fig:n4415-ar12-y2.7-blpics}
% \end{figure*}
%
% \begin{figure*}[!h]
%     \centering
%     \begin{subfigure}[t]{0.22\textwidth}
%         \centering
%         \includegraphics[trim=2.5in 3.05in 2in 2.2in, clip, width=0.9\textwidth]{{figs/decam_lines/bl_vmag_a14_y3.9}.png}
%         \caption{$\alpha=14\degree$\label{sfig:4415-ar12-blvmag-a14-y3.9}}
%     \end{subfigure}%
%     ~
%     \begin{subfigure}[t]{0.22\textwidth}
%         \centering
%         \includegraphics[trim=2.5in 3.05in 2in 2.2in, clip, width=0.9\textwidth]{{figs/decam_lines/bl_vmag_a18_y3.9}.png}
%         \caption{$\alpha=18\degree$\label{sfig:4415-ar12-blvmag-a18-y3.9}}
%     \end{subfigure}%
%
%     \begin{subfigure}[t]{0.22\textwidth}
%         \centering
%         \includegraphics[trim=2.5in 3.05in 2in 2.2in, clip, width=0.9\textwidth]{{figs/decam_lines/bl_vmag_a22_y3.9}.png}
%         \caption{$\alpha=22\degree$\label{sfig:4415-ar12-blvmag-a22-y3.9}}
%     \end{subfigure}%
%     ~
%     \begin{subfigure}[t]{0.22\textwidth}
%         \centering
%         \includegraphics[trim=2.5in 3.05in 2in 2.2in, clip, width=0.9\textwidth]{{figs/decam_lines/bl_vmag_a26_y3.9}.png}
%         \caption{$\alpha=26\degree$\label{sfig:4415-ar12-blvmag-a26-y3.9}}
%     \end{subfigure}%
%
%     \caption{The decambered camberline (red) and separation point location (blue symbol) for the NACA 4415 $\ar{12}$ wing (Case A) overlaid on a contour plot of the velocity magnitude $V/V_\infty$ at an outboard section ($2y/b=0.65$) from pre-stall to post-stall angles of attack}
%     \label{fig:n4415-ar12-y3.9-blpics}
% \end{figure*}
%
% \begin{figure*}[!h]
%     \centering
%     \begin{subfigure}[t]{0.22\textwidth}
%         \centering
%         \includegraphics[trim=2.5in 3.05in 2in 2.2in, clip, width=0.9\textwidth]{{figs/decam_lines/bl_vmag_a14_y5.1}.png}
%         \caption{$\alpha=14\degree$\label{sfig:4415-ar12-blvmag-a14-y5.1}}
%     \end{subfigure}%
%     ~
%     \begin{subfigure}[t]{0.22\textwidth}
%         \centering
%         \includegraphics[trim=2.5in 3.05in 2in 2.2in, clip, width=0.9\textwidth]{{figs/decam_lines/bl_vmag_a18_y5.1}.png}
%         \caption{$\alpha=18\degree$\label{sfig:4415-ar12-blvmag-a18-y5.1}}
%     \end{subfigure}%

%     \begin{subfigure}[t]{0.22\textwidth}
%         \centering
%         \includegraphics[trim=2.5in 3.05in 2in 2.2in, clip, width=0.9\textwidth]{{figs/decam_lines/bl_vmag_a22_y5.1}.png}
%         \caption{$\alpha=22\degree$\label{sfig:4415-ar12-blvmag-a22-y5.1}}
%     \end{subfigure}%
%     ~
%     \begin{subfigure}[t]{0.22\textwidth}
%         \centering
%         \includegraphics[trim=2.5in 3.05in 2in 2.2in, clip, width=0.9\textwidth]{{figs/decam_lines/bl_vmag_a26_y5.1}.png}
%         \caption{$\alpha=26\degree$\label{sfig:4415-ar12-blvmag-a26-y5.1}}
%     \end{subfigure}%

%     \caption{The decambered camberline (red) and separation point location (blue symbol) for the NACA 4415 $\ar{12}$ wing (Case A) overlaid on a contour plot of the velocity magnitude $V/V_\infty$ close to the wingtip ($2y/b=0.85$) from pre-stall to post-stall angles of attack}
%     \label{fig:n4415-ar12-y5.1-blpics}
% \end{figure*}
%

\fref{fig:n4415-blpics} compares the geometry of the decambered wing at multiple sections with contour plots showing the ratio of velocity magnitude to the freestream velocity ($V/V_\infty$). At $\alpha=14\degree$, the flow is mostly attached at all sections of the wing.
A small decambering flap is sufficient to accurately model the effective shape change due to the boundary layer. As the angle of attack is increased to $18\degree$, the separation point moves closer to the leading edge, the wing stalls and the boundary layer becomes thicker. The forward movement of the separation point is predicted well by the \methodname at all sections away from the wingtip. The thicker boundary layer is mimicked well by the decambering flap having a larger deflection and trailing-edge height. We also see that the \methodname accurately predicts the tendency of a rectangular wing to stall to the root. Upon increasing the angle of attack further to $22\degree$ and then to $26\degree$, we observe that the separation point at most sections is very close to the leading edge. The decambering flap approximately models the centerline of the thick boundary layer.
This observation was applied in Ref.~\cite{Hosangadi2018} to predict the location of the viscous wake behind the wing, and the velocity profile in the wake without the need for expensive boundary layer calculations. %
% \PHNote{Refer to dissertation?}
Figures showing the comparison of the decambered camberlines for other geometries are included in Ref. \cite{PranavThesis}
\subsection{Predictions for Tapered Wing (Case D)}
\label{sec:tapered-wings}
\fref{fig:n4415-tap-coeff} shows the coefficients of lift, drag, and pitching moment about root-quarter-chord for the tapered wing (Case F in \tref{tab:geoms}). This wing has an aspect ratio $\ar{}=12$, a taper ratio $\lambda = 0.5$, an unswept leading edge, and a root chord $4/3$ times the mean chord. The planform of the wing is shown in \fref{fig:tapered-planforms}.
% \PHNote{CFD and LOM moments are about (0.25,0,0) not root-quarter-chord}

\begin{figure}[!h]
    \centering
    % \begin{subfigure}[t]{\textwidth}
    %     \centering
        \includegraphics[width=\figwidth]{figs/eps_fig/le-unswept-planform-final2.pdf}
    %     \caption{\label{sfig:letap-planform}}
    % \end{subfigure}%

    % \begin{subfigure}[t]{\textwidth}
    %     \centering
    %     \includegraphics[width=\figwidth]{figs/eps_fig/qc-unswept-planform.eps}
    %     \caption{\label{sfig:qctap-planform}}
    % \end{subfigure}%
    % \caption{Planform view of the tapered wings with (a) unswept leading edge and (b) unswept quarter-chord line}
    \caption{Planform view of the tapered wing with  unswept leading edge}
    \label{fig:tapered-planforms}
\end{figure}

\begin{figure*}[!h]
    \centering
    % \begin{subfigure}[t]{\textwidth}
    %     \centering
        \includegraphics[width=\figwidth]{{figs/eps_fig/naca4415_coeffs_taper_all}.eps}
    %     \caption{Unswept leading edge\label{sfig:4415-ar12-letap}}
    % \end{subfigure}%

    % \begin{subfigure}[t]{\textwidth}
    %     \centering
    %     \includegraphics[width=\figwidth]{{figs/eps_fig/naca4415_taper0.5_qc-unswept_coeffs}.eps}
    %     \caption{Unswept Quarter-chord line\label{sfig:4415-ar12-qctap}}
    % \end{subfigure}%
    % \caption{Total lift, drag, and pitching moment vs. $\alpha$ for the tapered wings from CFD (black) and viscous LOM (blue)}
    \caption{Total lift, drag, and pitching moment vs. $\alpha$ for the tapered wing (Case D) from CFD (black) and viscous LOM (blue)}
    \label{fig:n4415-tap-coeff}
\end{figure*}

Viscous low-order results for the tapered wing were obtained using the same 2D viscous curves at each section, i.e. no Reynolds number adjustments were made. The results from the \methodname are shown in \fref{fig:n4415-tap-coeff}.
The viscous LOM accurately predicts the stall angle and the associated drop in lift and moment for the tapered wing.

From \fref{fig:n4415-ar12-le-unswept-cldist}, it can be seen that the viscous low-order method correctly predicts the spanwise distributions of lift and moment at the pre-stall angle of attack ($\alpha = 10\degree$). %
% The magnitudes of lift and nose-down pitching moment are slightly overpredicted throughout the span of the tapered wing.
As the angle of attack increases to stall and beyond ($ \alpha \geq 20\degree$), the drop in lift and moment is correctly modeled by the decambering approach implemented in the viscous low-order method. It is worth noting that the viscous LOM achieves this using viscous input curves obtained at a single Reynolds number ($Re = 3\times10^6$). The separation line is also accurately predicted by the low-order method, as shown in \fref{fig:naca4415_ar12_le-unswept_allalpha_fdist}.

% \PHNote{Spanwise distributions for tapered wings}

\begin{figure*}[!h]
    \centering
    \begin{subfigure}[t]{0.45\textwidth}
        \centering
        \includegraphics[width=0.9\textwidth]{{figs/eps_fig/spanwise-dist/naca4415_ar12_le-unswept_a10.0_cldist}.eps}
        \caption{$\alpha=10\degree$\label{sfig:4415-ar12-le-unswept-cldist-a10}}
    \end{subfigure}%
    ~
    \begin{subfigure}[t]{0.45\textwidth}
        \centering
        \includegraphics[width=0.9\textwidth]{{figs/eps_fig/spanwise-dist/naca4415_ar12_le-unswept_a10.0_cmdist}.eps}
        \caption{$\alpha=10\degree$\label{sfig:4415-ar12-le-unswept-cmdist-a10}}
    \end{subfigure}%

    \begin{subfigure}[t]{0.45\textwidth}
        \centering
        \includegraphics[width=0.9\textwidth]{{figs/eps_fig/spanwise-dist/naca4415_ar12_le-unswept_a22.0_cldist}.eps}
        \caption{$\alpha=20\degree$\label{sfig:4415-ar12-le-unswept-cldist-a22}}
    \end{subfigure}%
    ~
    \begin{subfigure}[t]{0.45\textwidth}
        \centering
        \includegraphics[width=0.9\textwidth]{{figs/eps_fig/spanwise-dist/naca4415_ar12_le-unswept_a22.0_cmdist}.eps}
        \caption{$\alpha=22\degree$\label{sfig:4415-ar12-le-unswept-cmdist-a22}}
    \end{subfigure}%

    \begin{subfigure}[t]{0.45\textwidth}
        \centering
        \includegraphics[width=0.9\textwidth]{{figs/eps_fig/spanwise-dist/naca4415_ar12_le-unswept_a28.0_cldist}.eps}
        \caption{$\alpha=28\degree$\label{sfig:4415-ar12-le-unswept-cldist-a20}}
    \end{subfigure}%
    ~
    \begin{subfigure}[t]{0.45\textwidth}
        \centering
        \includegraphics[width=0.9\textwidth]{{figs/eps_fig/spanwise-dist/naca4415_ar12_le-unswept_a28.0_cmdist}.eps}
        \caption{$\alpha=28\degree$\label{sfig:4415-ar12-le-unswept-cmdist-a28}}
    \end{subfigure}%

    \caption{Spanwise distributions of $C_l$ and $C_m$ at pre- and post-stall angles of attack from CFD (black), inviscid LOM (red), and viscous LOM (blue) for the tapered NACA4415 $\ar{12}$ wing (Case D)}
    \label{fig:n4415-ar12-le-unswept-cldist}
\end{figure*}

\begin{figure}[!h]
    \centering
    \includegraphics[width=3in]{figs/eps_fig/spanwise-dist/naca4415_ar12_le-unswept_allalpha_fdist.eps}
    \caption{Separation line predicted by the LOM (blue) and CFD (black) for angles of attack ranging from pre-stall to post-stall (Case D).}
    \label{fig:naca4415_ar12_le-unswept_allalpha_fdist}
\end{figure}

\subsection{Predictions for Wings Undergoing Rolling Motion (Case E)}
\label{sec:rolling-wings}
Although the underlying vortex lattice method used in this work is a steady code, it is capable of making predictions for quasi-steady flow states, such as for wings undergoing small, constant rates of rotation.
These quasi-steady conditions are typical of those experienced by general aviation and transport aircraft.
% In these quasi-steady states, flow features such as leading-edge vortex shedding which are usually observed in highly unsteady flows, are absent.
The examples presented in this section demonstrate the ability of the viscous LOM to predict the variation of total wing lift, drag, and moment coefficients, and their spanwise distributions, for two wings having a roll rate of $\SI{0.1}{\radian\per\second}$ or $\SI{5.73}{\deg\per\second}$ about the chordwise axis. This roll rate causes the left wing ($2y/b < 0$) to move downwards and see an increased effective angle of attack, while the right wing ($2y/b > 0$) moves upwards and experiences a reduced effective angle of attack. Two geometries, each of aspect ratio 12, are presented in this section: one rectangular and one tapered with a taper ratio $\lambda = 0.5$.
To verify the results from the low-order method, CFD solutions were obtained using ANSYS Fluent at select angles of attack before (0\degree, 10\degree), close to (15\degree, 18\degree), and after (20\degree) stall.

% \multiref{f}{fig:rot-ar8-coeffs}{fig:rot-ar12-coeffs}
\fref{fig:n4415-rot-coeffs}
shows the total wing $C_L$, $C_D$, and $C_M$ vs. $\alpha$ variation predicted by the \methodname. The lift and drag predictions from the viscous LOM agree well with CFD solutions. Predictions for pitching moment are seen to deviate from CFD solutions at higher angles of attack.
\multiref{f}{fig:n4415-ar12-dist-rot}{fig:n4415-tap-dist-rot} show the comparison of the spanwise distribution of $C_l$.
For reference, the spanwise distribution of $C_l$ for the wings without any rotational velocity from CFD is plotted using the dashed black line.
At a low angle of attack (0\degree) shown in Figures \ref{sfig:4415-ar12-cldist-a0-rot} and \ref{sfig:4415-ar12-le-uns-cldist-a0-rot}, the rolling motion of the wing causes an increase in lift on the left side (rolling downwards) and a drop in lift on the right side (rolling upwards).
It is this increase in lift on the descending wing that results in roll damping at unstalled conditions.
At $\alpha=20\degree$, this effect is reversed. The lift produced on the left side of the wing is reduced, whereas the right side of the wing produces more lift than the case without any rotation, as seen from Figures \ref{sfig:4415-ar12-cldist-a20-rot} and \ref{sfig:4415-ar12-le-uns-cldist-a20-rot}.
This post-stall behavior that results in loss of roll damping is captured correctly by the LOM.
The variation of total coefficient of rolling moment with angle of attack for the rectangular and tapered wings is shown in \fref{fig:n4415-croll}.
At low angles of attack, the wing experiences a negative rolling moment, i.e. in the direction opposite to the roll. The negative rolling moment indicates that roll damping is present. As the angle of attack increases, the restoring moment reduces, and after stall, the rolling moment acts in the direction of the rotation. The low-order method correctly predicts the loss of roll damping due to stall.

\begin{figure*}
    \centering
        \includegraphics[width=\figwidth]{{figs/eps_fig/naca4415_coeffs_roll_all}.eps}
    \caption{Total coefficients of lift, drag, and pitching moment for the NACA4415 $\ar{12}$ wings (Cases E\textsubscript{1}--E\textsubscript{2}) experiencing a $0.1\si{\radian\per\second}$ roll-rate}
    \label{fig:n4415-rot-coeffs}
\end{figure*}

% \begin{figure*}
%     \centering
%     \begin{subfigure}[t]{0.45\textwidth}
%         \centering
%         \includegraphics[width=0.9\textwidth]{{figs/eps_fig/spanwise-dist/naca4415_omega0.1_ar8_swp0_a0.0_cldist}.eps}
%         \caption{$\alpha=0\degree$\label{sfig:4415-ar8-cldist-a0-rot}}
%     \end{subfigure}%
%     ~
%     \begin{subfigure}[t]{0.45\textwidth}
%         \centering
%         \includegraphics[width=0.9\textwidth]{{figs/eps_fig/spanwise-dist/naca4415_omega0.1_ar8_swp0_a0.0_cmdist}.eps}
%         \caption{$\alpha=0\degree$\label{sfig:4415-ar8-cmdist-a0-rot}}
%     \end{subfigure}%

%     \begin{subfigure}[t]{0.45\textwidth}
%         \centering
%         \includegraphics[width=0.9\textwidth]{{figs/eps_fig/spanwise-dist/naca4415_omega0.1_ar8_swp0_a20.0_cldist}.eps}
%         \caption{$\alpha=20\degree$\label{sfig:4415-ar8-cldist-a20-rot}}
%     \end{subfigure}%
%     ~
%     \begin{subfigure}[t]{0.45\textwidth}
%         \centering
%         \includegraphics[width=0.9\textwidth]{{figs/eps_fig/spanwise-dist/naca4415_omega0.1_ar8_swp0_a20.0_cmdist}.eps}
%         \caption{$\alpha=20\degree$\label{sfig:4415-ar8-cmdist-a20-rot}}
%     \end{subfigure}%


%     \caption{Spanwise distributions of $C_l$ and $C_m$ at pre- and post-stall angles of attack from CFD (black), inviscid LOM (red), and viscous LOM (blue) for the NACA4415 rectangular $\ar{8}$ wing (Case E) with a $0.1\si{\radian\per\second}$ roll rate}
%     \label{fig:n4415-ar8-dist-rot}
% \end{figure*}


\begin{figure*}
    \centering
    \begin{subfigure}[t]{0.45\textwidth}
        \centering
        \includegraphics[width=0.9\textwidth]{{figs/eps_fig/spanwise-dist/naca4415_omega0.1_ar12_swp0_a0.0_cldist}.eps}
        \caption{$\alpha=0\degree$\label{sfig:4415-ar12-cldist-a0-rot}}
    \end{subfigure}%
    % ~
    % \begin{subfigure}[t]{0.45\textwidth}
    %     \centering
    %     \includegraphics[width=0.9\textwidth]{{figs/eps_fig/spanwise-dist/naca4415_omega0.1_ar12_swp0_a0.0_cmdist}.eps}
    %     \caption{$\alpha=0\degree$\label{sfig:4415-ar12-cmdist-a0-rot}}
    % \end{subfigure}%

    \begin{subfigure}[t]{0.45\textwidth}
        \centering
        \includegraphics[width=0.9\textwidth]{{figs/eps_fig/spanwise-dist/naca4415_omega0.1_ar12_swp0_a20.0_cldist}.eps}
        \caption{$\alpha=20\degree$\label{sfig:4415-ar12-cldist-a20-rot}}
    \end{subfigure}%
    % ~
    % \begin{subfigure}[t]{0.45\textwidth}
    %     \centering
    %     \includegraphics[width=0.9\textwidth]{{figs/eps_fig/spanwise-dist/naca4415_omega0.1_ar12_swp0_a20.0_cmdist}.eps}
    %     \caption{$\alpha=20\degree$\label{sfig:4415-ar12-cmdist-a20-rot}}
    % \end{subfigure}%


    \caption{Spanwise distributions of $C_l$ at pre- and post-stall angles of attack from CFD (black), inviscid LOM (red), and viscous LOM (blue) for the NACA4415 rectangular $\ar{12}$ wing (Case E\textsubscript{1}) with a $0.1\si{\radian\per\second}$ roll rate}
    \label{fig:n4415-ar12-dist-rot}
\end{figure*}




\begin{figure*}
    \centering
    \begin{subfigure}[t]{0.45\textwidth}
        \centering
        \includegraphics[width=0.9\textwidth]{{figs/eps_fig/spanwise-dist/naca4415_ar12_le-unswept_omega0.1_a0.0_cldist}.eps}
        \caption{$\alpha=0\degree$\label{sfig:4415-ar12-le-uns-cldist-a0-rot}}
    \end{subfigure}%
    % ~
    % \begin{subfigure}[t]{0.45\textwidth}
    %     \centering
    %     \includegraphics[width=0.9\textwidth]{{figs/eps_fig/spanwise-dist/naca4415_ar12_le-unswept_omega0.1_a0.0_cmdist}.eps}
    %     \caption{$\alpha=0\degree$\label{sfig:4415-ar12-le-uns-cmdist-a0-rot}}
    % \end{subfigure}%

    \begin{subfigure}[t]{0.45\textwidth}
        \centering
        \includegraphics[width=0.9\textwidth]{{figs/eps_fig/spanwise-dist/naca4415_ar12_le-unswept_omega0.1_a20.0_cldist}.eps}
        \caption{$\alpha=20\degree$\label{sfig:4415-ar12-le-uns-cldist-a20-rot}}
    \end{subfigure}%
    % ~
    % \begin{subfigure}[t]{0.45\textwidth}
    %     \centering
    %     \includegraphics[width=0.9\textwidth]{{figs/eps_fig/spanwise-dist/naca4415_ar12_le-unswept_omega0.1_a20.0_cmdist}.eps}
    %     \caption{$\alpha=20\degree$\label{sfig:4415-ar12-le-uns-cmdist-a20-rot}}
    % \end{subfigure}%


    \caption{Spanwise distributions of $C_l$ at pre- and post-stall angles of attack from CFD (black), inviscid LOM (red), and viscous LOM (blue) for the NACA4415 tapered wing (Case E\textsubscript{2}) with a $0.1\si{\radian\per\second}$ roll rate}
    \label{fig:n4415-tap-dist-rot}
\end{figure*}



\begin{figure*}
    \centering
        \includegraphics{{figs/eps_fig/naca4415_rollmoment_all}.eps}
    \caption{Total rolling moment vs. $\alpha$ on the rectangular and tapered $\ar{12}$ wings from CFD (symbols) and viscous LOM (lines)}
    \label{fig:n4415-croll}
\end{figure*}

% \PHNote{Hmm, no significant difference between rect and tapered from LOM an CFD}

% \clearpage




% \PHNote{Final conclusion}

% At low angles of attack, the wing experiences a negative rolling moment, i.e. in the direction opposite to the roll. As the angle of attack increases, this restoring moment reduces and, after stall, the rolling moment acts in the direction of the rotation. This trend is correctly predicted by the low-order method.

