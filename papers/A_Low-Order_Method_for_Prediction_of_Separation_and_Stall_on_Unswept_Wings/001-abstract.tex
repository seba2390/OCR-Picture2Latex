\begin{abstract}
A low-order method is presented for aerodynamic prediction of wings operating at near-stall and post-stall flight conditions. The method is intended for use in design, modeling, and simulation. In this method, the flow separation due to stall is modeled in a vortex-lattice framework as an effective reduction in the camber, or ``decambering.'' For each section of the wing, a parabolic decambering flap, hinged at the separation location of the section, is calculated through iteration to ensure that the lift and moment coefficients of the section match with the values from the two-dimensional viscous input curves for the effective angle of attack of the section. As an improvement from earlier low-order methods, this method also predicts the separation pattern on the wing. Results from the method, presented for unswept wings having various airfoils, aspect ratios, taper ratios, and small, quasi-steady roll rates, are shown to agree well with experimental results in the literature, and computational solutions obtained as part of the current work.

% Low-order aerodynamic models based on potential-flow theory are well suited to predicting the aerodynamic characteristics of wings and multiple-surface configurations at low angles of attack in the linear region of aerodynamics.
% Increasing the range of these methods beyond the linear region to near-stall and post-stall angles of attack can prove to be valuable in a variety of applications such as preliminary design, flight dynamics characterization, and flight simulation.
% Past efforts to augment the capabilities of potential-flow methods by modeling the effects of a separated boundary layer have seen some success.
% The decambering approach models the effects of the separated boundary layer with a reduction in camber of the lifting surface of a potential-flow method such as the vortex lattice method (VLM).
% The wing is split into sections along the chordwise direction, and each section is assumed to behave like an airfoil.
% The reduction in camber is achieved through a parabolic decambering flap, hinged at the location of flow separation at the section.
% The shape of the decambering flap is calculated to ensure that its operating point moves onto the viscous lift and moment curves of the airfoil, which are easy to obtain using computational fluid dynamics (CFD) or experiments.
% The decambering operation is performed iteratively until this condition is satisfied at all sections of the wing.
% Results from the decambering method show excellent agreement with CFD solutions obtained as a part of this work, and with experimental results found in the literature, for unswept wings of various airfoils, aspect ratios, taper ratios, and experiencing small, quasi-steady roll rates.

\end{abstract}
