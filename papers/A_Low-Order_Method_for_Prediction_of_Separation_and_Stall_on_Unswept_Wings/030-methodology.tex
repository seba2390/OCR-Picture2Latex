\section{Nonlinear Decambering}
\label{sec:nonlin-decambering}
The nonlinear decambering method \cite{Narsipur2018} was developed as an improvement to the linear decambering method, and uses a single parabolic flap as shown in \fref{fig:nld-flap-illustration} to model the reduction in camber due to flow separation at high angles of attack. The nonlinear decambering flap is hinged at the separation location ($f$), and has an inclination ($\delta_l$) at the hinge, and a height ($m$) at the trailing edge. The value of $f$ varies from 0 (separation occurs at the leading edge, i.e. flow is fully separated) to 1 (flow is fully attached, separation occurs at the trailing edge).
In contrast to previously published decambering approaches \cite{Mukherjee_poststall_2006,Paul_Gopa_Iteration_Schemes,gopalarathnam_paul_petrilli_ASM_2012}, this method requires a single flap to account for deviation in both lift and moment from their respective inviscid values.

\begin{figure}[!ht]
    \centering
    \includegraphics[width=3in]{figs/eps_fig/naca4415-decambering-flap.eps}
    \caption{The parabolic flap used by the nonlinear decambering method, hinged at $x/c = f$, with an inclination of $\delta_l$ at the hinge location and a height $m$ at the trailing edge}
    \label{fig:nld-flap-illustration}
\end{figure}


Non-dimensionalizing the $x$ and $z$ coordinates with chord, we write $\bar{x} = x/c$ and $\bar{z} = z/c$.
The shape of the parabolic decambering flap is given by:
\begin{align}
    \bar{z}\left(\bar{x}\right) = \begin{cases}
    0 & \bar{x} < f \\
    A\bar{x}^2 + B\bar{x} + D & \bar{x} \geq f
    \end{cases} \label{eqn:decambering-camberline}
\end{align}

\noindent where,
\begin{align}
    A &= \frac{m - (1 - f)\tan(\delta_l)}{(1 - f)^2} \label{eqn:decambering-A} \\
    B &= \tan(\delta_l) - 2 A f \label{eqn:decambering-B}  \\
    D &= m - (A + B) \label{eqn:decambering-D}
\end{align}

Using thin-airfoil theory, the change in lift and moment caused by a nonlinear decambering flap having the decambering parameters ($f, \delta_l, m$) are given by:
\begin{align}
    \left[ \begin{array}{c}
         \Delta C_l \\
         \Delta C_m
    \end{array}\right] &= \left[ \begin{array}{cc}
        a_1 & b_1 \\
        a_2 & b_2
    \end{array} \right] \left[ \begin{array}{c}
         A  \\
         B
    \end{array} \right] \label{eqn:decambering-dclcm}
\end{align}

\noindent where,
\begin{align*}
    a_1 &= 3 \theta_f - 3 \pi - 4 \sin \theta_f + \frac{1}{2} \sin 2\theta_f \\
    a_2 &= \frac{3}{4} \sin \theta_f - \frac{3}{8} \sin 2\theta_f + \frac{1}{12} \sin 3\theta_f - \frac{\theta_f}{4} + \frac{\pi}{4} \\
    b_1 &= 2\theta_f - 2\pi - 2 \sin \theta_f \\
    b_2 &= \frac{1}{2} \sin \theta_f - \frac{1}{4} \sin 2\theta_f \\
    \theta_f &= \arccos(1 - 2f)
\end{align*}

% \section{Input Data for the Nonlinear Decambering Method}
The decambering parameters ($f, \delta_l, m$) for each flow condition are calculated using viscous lift, moment, and separation location data obtained from steady two-dimensional CFD solutions or experiments. In case the separation location is not available in the input datasets, Beddoes' modification \cite{Beddoes1983} to the Kirchhoff-Helmholtz solution for separated flow over a flat plate can be used to estimate the location of the separation point as follows:
\begin{align}
    f &= \left(2\sqrt{\frac{C_l}{2\pi \sin(\alpha - \alpha_{0L})}} - 1\right)^2 \label{eqn:beddoes-f}
\end{align}

\section{Nonlinear Decambering Applied to Two-Dimensional Airfoils}
\label{sec:nld-2D}

The concept of nonlinear decambering is easily illustrated using the example of 2D flow past an airfoil shown in  \multiref{f}{fig:nld-illustration}{fig:nld-targets}. At low angles of attack, the boundary layer is fully attached to the surface of the airfoil, as seen in \fref{sfig:nld-lowalpha}.
The resulting loads on the airfoil, shown in \fref{fig:nld-targets} are accurately predicted by an inviscid method and no decambering is required.
As the angle of attack increases, the separated boundary layer causes a deviation in lift and moment from the potential-flow predictions.
From the viscous input data supplied, we obtain the location of the separation point ($f$), and the ``target'' $C_l$ and $C_m$ values that the airfoil is known to produce in viscous flow, marked in \fref{fig:nld-targets} by black asterisks. The required change in $C_l$ and $C_m$ are written as $\Delta C_l = C_{l,\text{viscous}} - C_{l,\text{potential}}$ and $\Delta C_m = C_{m,\text{viscous}} - C_{m,\text{potential}}$, respectively. The separation location and required $\Delta C_l$ and $\Delta C_m$ thus found from the viscous input data are plugged in to \multiref{e}{eqn:decambering-A}{eqn:decambering-dclcm} to obtain the values of the decambering parameters. At low angles of attack where the viscous input curves indicate mostly attached flow, the required $\Delta C_l$ and $\Delta C_m$ are small, and $f \approx 1$. In such cases, using the correct value of $f$ leads to large, non-physical decambering flap deflections to achieve even the small deviations in lift and moment. This problem is avoided by restricting the value of $f$ to a maximum of $0.8$. %This is necessary because a higher value of $f$ would require a large, non-physical decambering flap deflection to obtain the even small $\Delta C_l$ and $\Delta C_m$ that are required in such cases.
The camberline of the airfoil is modified according to \eref{eqn:decambering-camberline}, and the potential-flow solution for the modified camberline is seen to fall on the viscous operating curve of the airfoil.
\revnote{From \fref{sfig:nld-highalpha}--\ref{sfig:nld-linhighalpha}, it can be seen that the nonlinear decambering flap mimics the shape of the separated boundary layer more closely than the linear decambering flap described in Ref. \cite{Paul_Gopa_Iteration_Schemes}.}{\#2.2}

\begin{figure*}[ht!]
    \centering
    \begin{subfigure}[t]{0.45\textwidth}
        \centering
        \includegraphics[width=2.5in]{figs/eps_fig/naca4415-2D-decambering-lowalpha.eps}
        \caption{$\alpha = 2\degree$, no decambering\label{sfig:nld-lowalpha}}
    \end{subfigure}
    ~
    \begin{subfigure}[t]{0.45\textwidth}
        \centering
        \includegraphics[width=2.5in]{figs/eps_fig/naca4415-2D-decambering-highalpha.eps}
        \caption{$\alpha = 18\degree$, Nonlinear decambering\label{sfig:nld-highalpha}}
    \end{subfigure}%
    ~
    \begin{subfigure}[t]{0.45\textwidth}
        \centering
        \includegraphics[width=2.5in]{figs/eps_fig/naca4415-2D-lineardecambering-highalpha.eps}
        \mnote{\#2.2}
        \caption{$\alpha = 18\degree$, linear decambering\label{sfig:nld-linhighalpha}}
    \end{subfigure}
    \caption{The original camberline (black, dashed) and the camberline after nonlinear decambering (red) and linear decambering from Ref. \cite{Paul_Gopa_Iteration_Schemes} (blue) for the NACA 4415 airfoil, overlaid on the CFD-predicted contour of $|\vec{V}|/|\vec{V}_\infty|$\label{fig:nld-illustration}}
\end{figure*}

\begin{figure}[h!t]
    \centering
    \includegraphics[width=\figwidth]{figs/eps_fig/naca4415-airfoil-decam-target.eps}
    \caption{Potential-flow and viscous-flow curves (lines) and operating points (symbols) for the NACA 4415 airfoil}
    \label{fig:nld-targets}
\end{figure}

\section{Nonlinear Decambering Applied to Three-Dimensional Wings}
\label{sec:nld-3D}
Similar to the other variations of decambering, nonlinear decambering can be applied to three-dimensional wings using a strip-theory approach. The wing is divided into chordwise strips, with a set of decambering parameters assigned to each strip. These parameters are determined individually for each strip based on the change in lift and moment required at that strip to simultaneously satisfy both conditions mentioned in \secref{sec:decambering} above.

\subsection{Effective Angle of Attack}
\label{sec:aeff}

The angle of attack ($\alpha$) is defined as the angle between the chordline of the airfoil/wing and the incoming wind.
However, the trailing vortex sheet behind the wing induces a downwash at every section of the wing that causes a reduction in the angle of attack experienced by the section. Since the section ``sees'' the incoming wind impinging on itself at this angle, the behavior of the section is governed by this ``effective'' angle of attack ($\aeff$) rather than the wing angle of attack ($\alpha$). Calculation of this effective angle of attack correctly at each section of the wing is crucial to accurate prediction of the aerodynamic loads on the sections, and therefore those on the wing.
%This is achieved by finding the angle of attack at which a two-dimensional section having the same shape and decambering parameters as the wing-section experiences the same normal force as the wing-section.
The effective angle of attack, however, is not an output of a VLM solution.
To calculate the effective angle of attack for a wing section from the VLM solution for a given angle of attack, the same decambering flap applied to the section is also applied to a two-dimensional airfoil. Next, a Newton-Raphson iteration is used to vary the angle of attack of the airfoil until it produces the same normal force ($C_n$) as the wing section. The angle of attack for the 2D airfoil is the effective angle of attack of the wing section.



\subsection{Profile Drag}
Although the VLM can predict induced drag on three-dimensional wings excellently, it is ill-suited to predicting the profile drag of the wing because it is based on potential flow theory. However, the strip theory approach can be used to estimate the profile drag without needing to calculate it by solving the equations of viscous flow. Since the $C_d$ for the airfoil is typically available in viscous airfoil datasets, each strip is assigned a viscous $C_d/C_l$ vs. $\alpha$ curve obtained from the $C_d$ and $C_l$ of the airfoil. When the \methodname solves for the lift distribution on the sections, the $C_d$ for each section is obtained by interpolating the $C_d/C_l$ vs. $\alpha$ curve at the $\aeff$ of the section. The forces and moments resulting from the interpolated $C_d$ for each strip are then added to the total values calculated by the VLM.

\subsection{Decambering Trajectory}
In earlier works \cite{Mukherjee_poststall_2006,Paul_Gopa_Iteration_Schemes}, it was shown that the induced flow at a section is affected by the decambering at every other section.
Using the example wing geometry in \fref{fig:strips-illustration-3D} in which the wing span is discretized into 20 strips, the effective angle of attack at some section (section 10, for example) will depend on the downwash at that section. Because this downwash depends on the lift distribution over the span, the effective angle of attack at section 10 depends on the decambering at all the sections. %As a result, the effective angle of attack of a section depends on the decambering parameters of all other sections in addition to the downwash induced by the wingtip vortices.
This behavior is not seen in the case of a 2-dimensional airfoil.
To illustrate this phenomenon, consider the \revnote{effect of a single iteration of decambering on a}{\#3.11} NACA4415 airfoil at an angle of attack of $18\degree$.
Without decambering, the operating point $(\alpha, C_l)$ falls on the inviscid lift curve ($C_l = 2\pi \sin(\alpha - \alpha_{0L})$) for the airfoil, denoted by the dashed blue line in \fref{sfig:traj-illustration-2D}.
\revnote{In the first iteration of decambering, the values of $f$, $\delta_l$, and $m$ are calculated}{\#3.11} using \multiref{e}{eqn:decambering-A}{eqn:decambering-dclcm} above. The airfoil produces less lift ($C_{l,d}$) than in the case without decambering.
The operating point must fall on the lift curve of the decambered airfoil, denoted by the red line in \fref{fig:traj-illustration}.
Now, the effective angle of attack of the airfoil is found by locating the $\alpha$-coordinate at which the lift curve of the decambered airfoil gives a coefficient of lift equal to $C_{l,d}$.
The operating point of the airfoil is observed to have moved vertically downwards as a consequence of decambering (\fref{sfig:traj-illustration-2D}).

\begin{figure}
    \centering
    \includegraphics[width=4.5in]{figs/eps_fig/strips_illustration_3D.eps}
    \caption{Illustration of the wing geometry divided into strips}
    \label{fig:strips-illustration-3D}
\end{figure}

Now consider the behavior of a single section (section 10) near the root of the three-dimensional wing shown in \fref{fig:strips-illustration-3D}, \revnote{again undergoing a single decambering iteration}{\#3.11}.
Due to induced downwash effects, the section produces less lift than the airfoil even without any decambering applied, which moves the inviscid (no-decambering) operating point to the filled blue circle in \fref{sfig:traj-illustration-3D}. Now, let us apply the calculated decambering parameters to this section only, and no decambering at any other sections of the wing.
\revnote{The loss in lift on that strip due to the decambering results in strong trailing vortices at the edges of the strip (due to sudden increase in lift going from this strip to the adjacent one). These trailing vortices cause upwash which partly negates the effect of the decambering. The result is that the loss in $C_l$ is not as much as that seen with just the 2D decambering. Calculating the new lift and $\aeff$ of this section as described above, we see that the operating point moves along the trajectory labeled ``T1'' in \fref{sfig:traj-illustration-3D} to the point denoted by the filled red triangle.
Instead, if the requisite decambering is applied to all sections of the wing instead of a single section, there are no strong trailing vortices at the edges of the strip. The upwash due to the entire-wing decambering is much smaller than with the single-strip decambering.
Because of this, the loss in $C_l$ is only slightly smaller than that from the 2D decambering. The operating point now follows a different trajectory labeled ``T*'' to the new operating point denoted by the filled red circle.}{\#3.10}

\begin{figure*}[ht!]
    \centering
    \begin{subfigure}[t]{0.40\textwidth}
        \centering
        \includegraphics[width=\textwidth]{figs/eps_fig/traj-illustration-2D.eps}
        \caption{2D airfoil\label{sfig:traj-illustration-2D}}
    \end{subfigure}%
    ~
    \begin{subfigure}[t]{0.40\textwidth}
        \centering
        \includegraphics[width=\textwidth]{figs/eps_fig/traj-illustration-3D.eps}
        \caption{Section of a 3D wing\label{sfig:traj-illustration-3D}}
    \end{subfigure}
    \caption{An illustration of the decambering trajectories for an airfoil and a section of a 3D wing\label{fig:traj-illustration}}
\end{figure*}

It is seen from this exercise that the movement of the operating point of a section in the $C_l$ vs. $\aeff$ space depends on the induced flow at the section, which in turn depends on the decambering at every other section of the wing.
The slopes of the trajectory lines for each section are required in order to obtain accurate target viscous operating points from the viscous input curves.
To obtain these slopes, the inviscid solution is calculated to obtain the starting operating point ($\aeff^0, C_{l}^0$) for each section.
The trajectory for each section is assumed to be vertical, as seen for the airfoil, and the intersection of the vertical trajectory with the viscous input curve gives an initial $C_{l}^\text{target}$ and $C_{m}^\text{target}$ for the decambering method. The decambering required at each section is calculated, and the perturbed operating points ($\aeff^p, C_{l}^p$) are found. Now, the trajectory slope at each section is given by:
\begin{align}
    \diff{C_l}{\aeff} &= \frac{C_l^p - C_l^0}{\aeff^p - \aeff^0}
\end{align}

% \begin{figure}[!h]
%     \centering
%     \includegraphics[width=2.75in]{figs/eps_fig/traj-moreiters.eps}
%     \caption{Decambering trajectories for all sections of a 3-dimensional wing remain fairly constant}
%     \label{fig:traj-moreiters}
% \end{figure}

\begin{figure*}[!h]
    \begin{subfigure}[t]{0.45\textwidth}
        \centering
        \includegraphics[width=2.75in]{figs/eps_fig/n4415-ar12-swp0-traj30.eps}
        \caption{}
        \label{fig:n4415-ar12-swp0-traj30}
    \end{subfigure}%
    ~
    \begin{subfigure}[t]{0.45\textwidth}
        \centering
        \includegraphics[width=2.75in]{figs/eps_fig/traj-moreiters.eps}
        \caption{}
        \label{fig:traj-moreiters}
    \end{subfigure}%
    ~
    \caption{(a) Decambering trajectories calculated at a post-stall angle of attack $\alpha = 30\degree$, (b) Decambering trajectories do not change significantly as the iteration advances}
\end{figure*}


\revnote{It was observed that the actual trajectories of the operating points due to decambering were more or less identical to the decambering trajectories calculated in the first iteration at a high angle of attack where all sections require some decambering.
For all results presented in this paper, the trajectories were calculated at $\alpha = 30\degree$.
These trajectories are shown in \fref{fig:n4415-ar12-swp0-traj30} for the NACA4415 $\ar{12}$ wing.
Since the wing is symmetric about the $y$-plane, trajectories for sections $i$ and $20-i+1$ coincide.
During subsequent iterations of the decambering method, the
operating points appear to move in a straight line in the $C_l$-$\aeff$ space due to the trajectories remaining invariant.
The similarity of decambering trajectories between subsequent iterations is shown in \fref{fig:traj-moreiters}. This observation is in agreement with the results of Paul and Gopalarathnam~\cite{Paul_Gopa_Iteration_Schemes}, and allows the method to select an accurate target viscous operating point immediately after the inviscid solution is calculated.}{\#3.11}



% Additionally, it was found that the trajectories calculated for an angle of attack where at least some separation is experienced by every section were representative of trajectories at most angles of attack.
% \PHNote{revisit this paragraph}


% \begin{figure}[!h]
%     \centering
%     \includegraphics[width=2.75in]{figs/eps_fig/n4415-ar12-swp0-traj30.eps}
%     \caption{Decambering trajectories for the sections of a NACA4415 \ar{12} wing, calculated at $\alpha=30\degree$}
%     \label{fig:n4415-ar12-swp0-traj30}
% \end{figure}


% \subsection{Obtaining Trajectory Intersections with Viscous Curves}
% % \newcommand{\aeffprime}{\ensuremath{\alpha_\text{eff}^{'}}}
% To easily obtain the intersections of the inclined trajectory lines with the viscous curve, the input $C_l$-$\alpha$ curve for each section is subjected to a shear transform parallel to the $\alpha$ axis as follows:
% \begin{align}
%     \left(\begin{array}{c}
%          \alpha'  \\
%          C_l'
%     \end{array}\right)_i &= \left[\begin{array}{cc}
%         1 & -\left(\left.\diff{C_l}{\aeff}\right\vert_i\right)^{-1} \\
%         0 & 1
%     \end{array}\right] \left( \begin{array}{c}
%          \alpha  \\
%          C_l
%     \end{array} \right)
% \end{align}

% This transforms the inclined trajectories into vertical lines, allowing for quick interpolation during the iteration procedure.
% To obtain the intersection of the trajectory line from an operating point given by ($\aeff, C_l$), we first apply the same shear transformation to the operating point.
% \begin{align}
%     \left(\begin{array}{c}
%          \aeffprime  \\
%          C_l
%     \end{array}\right)_i &= \left[\begin{array}{cc}
%         1 & -\left(\left.\diff{C_l}{\aeff}\right\vert_i\right)^{-1} \\
%         0 & 1
%     \end{array}\right] \left( \begin{array}{c}
%          \aeff  \\
%          C_l
%     \end{array} \right)_i
% \end{align}

% Next, the transformed target curve is interpolated at $\alpha = \aeffprime$ to obtain the target coefficient of lift ($C_{l}^\text{target}$). The inverse shear transform is then applied to the target operating point $(\aeffprime, C_{l}^\text{target})_i$ to obtain the actual target operating point.
% \begin{align}
%     \left(\begin{array}{c}
%          \aefftarget  \\
%          C_{l}^\text{target}
%     \end{array}\right)_i &= \left[\begin{array}{cc}
%         1 & \left(\left.\diff{C_l}{\aeff}\right\vert_i\right)^{-1} \\
%         0 & 1
%     \end{array}\right] \left( \begin{array}{c}
%          \aeffprime  \\
%          C_{l}^\text{target}
%     \end{array} \right)_i
% \end{align}


% \begin{figure}
%     \centering
%     \includegraphics[width=5in]{figs/eps_fig/skew-strip-illustration.eps}
%     \caption{Illustration of the shear transformation procedure to find the decambering target at a single strip}
%     \label{fig:skew-drawing}
% \end{figure}

% The procedure described above is illustrated graphically in \fref{fig:skew-drawing} for a single section. Decambering trajectories are calculated only for $C_l$. The target $C_m$ and $C_d$ are defined as the $C_m$ and $C_d$ from the input curves at $\aefftarget$.


\subsection{Iterative Calculation of Decambering Parameters}
\label{sec:iter-nld}
Once the decambering trajectories have been calculated at a preset angle of attack, the \methodname can be used to calculate the wing loads at any $\alpha$. The iterative procedure used to converge on a solution satisfying the conditions from \secref{sec:decambering-conditions} is described below with the aid of the flowchart in \fref {fig:flowchart}.
We start with the decambering parameters for all sections initialized to $f=1, \delta_l = 0, m = 0$, which corresponds to the inviscid solution for flow over the wing.

\tikzstyle{startstop} = [rectangle, rounded corners, minimum width=3cm, minimum height=1cm,text centered, draw=black, fill=white] %
\tikzstyle{process} = [rectangle, minimum width=3cm, minimum height=1.5cm, text centered, draw=black, fill=white]%
\tikzstyle{decision} = [diamond, minimum width=3cm, minimum height=1cm, text centered, draw=black, fill=white, yshift=-0.5cm] %
\tikzstyle{arrow} = [thick,->,>=stealth] %
%
% \begin{figure}[ht]
%     \centering
%     \begin{tikzpicture}[node distance=2cm]
%         \node (start) [startstop] {Start};
%         \node (vlmsolve) [process, below of=start] {Solve VLM};
%         \node (aeff) [process, below of=vlmsolve] {Calculate $\aeff$};
%         \node (targets) [process, below of=aeff] {Obtain viscous targets};
%         \node (converged) [decision, below of=targets, text width=1.5cm] {Solution \\ converged?};
%         \node (modgeom) [process, right of=targets, xshift=2cm] {Modify geometry};
%         \node (calcdec) [process, right of=converged, xshift=2cm, text width=2cm] {Calculate decambering ($f, \delta_l, m$)};
%         \node (stop) [startstop, below of=converged, yshift=-0.5cm] {Stop};
%     \end{tikzpicture}
%     \caption{Flowchart of the \methodname}
%     \label{fig:flowchart}
% \end{figure}
%
\begin{figure}[ht]
    \centering
    \begin{tikzpicture}[node distance=2.1cm, text width=1.5cm]
        \node (start) [startstop] {Start};
        \node (traj) [process, below of=start, yshift=-0.2cm, text width=2cm] {Calculate decambering trajectory slopes (once per wing geometry)};
        \node (vlmsolve) [process, below of=traj, yshift=-0.5cm] {Solve VLM};
        \node (aeff) [process, right of=vlmsolve, xshift=2cm] {Calculate $\aeff$};
        \node (targets) [process, right of=aeff, xshift=2cm, text width=2.2cm] {Obtain viscous $f$, $\Delta C_l$, $\Delta C_m$, $C_d$};
        \node (converged) [decision, below of=targets, text width=1.5cm] {Solution \\ converged?};
        \node (calcdec) [process, left of=converged, xshift=-2cm, text width=2cm] {Calculate decambering ($\delta_l, m$)};
        \node (modgeom) [process, left of=calcdec, xshift=-2cm] {Modify geometry};
        \node (stop) [startstop, below of=converged, yshift=-0.5cm] {Stop};
        \draw [arrow] (start) -- (traj);
        \draw [arrow] (traj) -- (vlmsolve);
        \draw [arrow] (vlmsolve) -- (aeff);
        \draw [arrow] (aeff) -- (targets);
        \draw [arrow] (targets) -- (converged);
        \draw[arrow] (converged) -- node[anchor=south,xshift=0.6cm] {no} (calcdec);
        \draw[arrow] (converged) -- node[anchor=west] {yes} (stop);
        \draw[arrow] (calcdec) -- (modgeom);
        \draw[arrow] (modgeom) -- (vlmsolve);
    \end{tikzpicture}
    \caption{Flowchart of the \methodname}
    \label{fig:flowchart}
\end{figure}%
%

\subsubsection{Potential-Flow Solution}\label{step:vlm-solution}
The RHS of the linear system of the VLM is calculated using the normal vectors of the geometry, the velocity of the incoming wind ($V_\infty$), and the rotational velocity (if any) of the wing. Control surface deflections are accounted for by tilting the appropriate normal vectors by the required angle. For the $i$th panel, the new normal vector is given by
\begin{align}
    \hat{n}_{i}^{'} &= \hat{n}_i + \Delta \hat{n}_{i,\text{decambering}} + \Delta\hat{n}_{i,\text{control}}
\end{align}
\noindent where $\hat{n}_i$ is the original normal vector of the panel, with $\Delta \hat{n}_{i,\text{decambering}}$ and  $\Delta\hat{n}_{i,\text{control}}$ denoting the change in normal vector due to decambering and control surface deflection respectively.
The total velocity $\vec{V}_i$ at the collocation point for this panel depends upon $\vec{V}_\infty$ and the velocity produced at the collocation point $\vec{p}_i$ due to rotation about the center of rotation $\vec{p}_\text{rot}$.
\begin{align}
    \vec{V}_i &= \vec{V}_\infty - \vec{\omega} \times (\vec{p}_i - \vec{p}_\text{rot})
\end{align}
Finally, the normal component of the velocity at the collocation point is obtained from the dot product of the total velocity $\vec{V}_i$ with the effective normal vector $\hat{n}_{i}^{'}$.
\begin{align}
    RHS_i &= -\vec{V}_i \cdot \hat{n}_{i}^{'}
\end{align}

The linear system of the vortex lattice is then solved to obtain the potential flow solution.

\subsubsection{Calculation of Effective Angle of Attack}\label{step:aeff-calc}
For each section of the wing, a 2D discrete vortex solver is initialized with a geometry identical to the wing-section, including any alterations due to control surface deflection or decambering. Additional velocities due to the rotation of the wing are not included, since the effect of these velocities will be included in the calculated $\aeff$.
%\PHNote{Need to think of a better way to explain this}.
A Newton iteration is set up to calculate the $\alpha$ required for the 2D solver to produce the same normal force ($C_n$) as the section of the 3D wing. This calculated $\alpha$ is the $\aeff$ for the section.

\subsubsection{Obtaining Target Viscous Coefficients Using Decambering Trajectories}\label{step:traj-target}
At each section, the target viscous $C_l$ is obtained by finding the intersection of its decambering trajectory line with the pre-calculated slope and the viscous input $C_l$ vs. $\alpha$ curve. \fref{fig:iter0-targets} shows the operating points and decambering targets for the zeroth iteration for each section of the NACA4415 \ar{12} wing at $\alpha=18\degree$.

\begin{figure}[ht]
    \centering
    \includegraphics[width=2.75in]{figs/eps_fig/n4415-ar12-a18-iter0-targets.eps}
    \caption{Decambering targets for each section}
    \label{fig:iter0-targets}
\end{figure}


\subsubsection{Checking for Convergence}\label{step:convergence}
At this stage, the error in $C_l$ and $C_m$ for each section is calculated.

\begin{align}
    \Delta C_{l,i} &= C_{l,i}^\text{target} - C_{l,i} \\
    \Delta C_{m,i} &= C_{m,i}^\text{target} - C_{m,i}
\end{align}

Convergence is said to have been achieved when the mean absolute error in both $C_l$ and $C_m$ are below a specified tolerance, which indicates that the conditions specified in \secref{sec:decambering-conditions} have been fulfilled.
% The results presented in \secref{sec:unswept-results} were obtained using a tolerance of 0.05 for $C_l$ and 0.01 for $C_m$.
Tolerances of 0.05 for $C_l$ and 0.01 for $C_m$ were found to yield sufficiently accurate results, as shown in \secref{sec:unswept-results} below.

\subsubsection{Calculation of Decambering Parameters}\label{step:calc-decam}
If the required tolerance is not met, the increment in decambering needed to achieve the required change in lift and moment is calculated using \multiref{e}{eqn:decambering-A}{eqn:decambering-dclcm}. If a decambering flap already exists at a section prior to the current iteration, the calculated $\delta_l$ and $m$ are added to the preexisting decambering parameters.

\subsubsection{Solution Update}\label{step:sol-update}
Once the decambering parameters have been calculated, the shape of the decambering flap is obtained using \eref{eqn:decambering-camberline}.
%
% The slope of the decambering flap is given by
% \begin{align}
%     \diff{z}{x}(x) &= \begin{cases}
%     0 & x < f \\
%     2 A x + B & x \geq f
%     \end{cases}
% \end{align}
%
% Since the geometry is represented by discrete panels, the normal vector of every panel is tilted by the slope at the location of its collocation point ($\left.\diff{z}{x}\right\vert_{i,\text{colloc}}$).
% \begin{align}
%     m &= \left.\diff{z}{x}\right\vert_{i,\text{colloc}} \\
%     \Delta\vec{n}_{i, \text{decambering}} &= \left\{\sin\left(m\right); 0; \cos\left(m\right)-1 \right\}
% \end{align}
%
% In case the separation location falls within a panel, its inclination is multiplied by the fraction of the panel that encounters separated flow.
%
As illustrated in \fref{fig:decam-in-place}, the normal vectors are merely rotated in their original locations without moving the collocation points to the location of the modified camberline, since doing so would require an expensive recalculation of the aerodynamic influence coefficients for the VLM.
Thereafter, steps 1--6 are repeated until the solution has converged. At higher angles of attack, large decambering flaps are required to model the significant separated flow over the wing. At these angles, typically $\alpha > 25\degree$, the solution from the previous angle of attack can be used as the initial solution to aid convergence.

\begin{figure}[h!t]
    \centering
    \includegraphics[trim={0.1in 0.75in 0.1in 0.75in}, clip,width=0.75\textwidth]{figs/eps_fig/decambering-in-place.eps}
    \caption{An illustration of the decambering implemented in the VLM by rotating normal vectors in situ. The modified normal vectors (red arrows) are normal to the modified camberline (red line), but located on the original camberline (black line).}
    \label{fig:decam-in-place}
\end{figure}