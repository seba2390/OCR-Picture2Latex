
% \lipsum[1-5]

\section{Introduction}

Aircraft normally operate in the ``linear region'' of aerodynamics.
This region, which occurs at low angles of attack, is characterized by mostly attached flow, and a linear variation of lift with angle of attack.
In the linear region, the boundary layer is thin and the flow can be approximated by a potential-flow solution.
The behavior of airfoils and wings at low angles of attack has been thoroughly studied, and extensive data on the forces and moments acting on lifting surfaces in the linear region is available from a variety of experimental, numerical, and theoretical sources \cite{abbott,stuttgarter_profilkatalog_1,lsats_3,xfoil,drela_avl}.%\cite{abbott,stuttgarter_profilkatalog_1,lsats_1,lsats_2,lsats_3,lsats_4,xfoil}.
As the angle of attack increases, an adverse pressure gradient forms on the upper surface of the airfoil/wing. The adverse pressure gradient %
% reduces the velocity of the flow and
causes the boundary layer to thicken and then separate from the surface.
The thick, separated boundary layer changes the effective shape of the body. The flow can no longer be approximated by the potential-flow theory, and
the lift produced drops in comparison to the linear curve.
As the angle of attack increases further, the adverse pressure gradient intensifies, and the location at which the flow separates moves forward towards the leading edge.
Beyond a limiting angle of attack ($\alpha_\text{stall}$), the lift produced starts to decrease with an increasing angle of attack. The drop in lift is accompanied by a significant increase in drag and a drop in pitching moment, and the airfoil/wing is said to have stalled.

Although a majority of applications operate in the linear region, post-stall aerodynamics are commonly experienced by applications such as wind turbines, helicopters, and even some fixed-wing aircraft. A solid understanding of near-stall and post-stall flows is crucial to the success of these applications.
Aerodynamic models that can be used to rapidly predict the loads acting on wings and aircraft configurations have applications in preliminary design, flight dynamics characterization, and flight simulation.
Due to the requirement for rapid predictions, low-order models are especially useful in such applications.
Low-order predictive methods based on potential flow, such as the vortex lattice method (VLM), are well established in predicting the force and moment characteristics, and spanwise distributions of the forces and moments, on wings and multiple-surface configurations at low angles of attack, where the flow can be approximated by potential flow.
The development of the first steady VLM dates back to work done by Hedman in the 1960s
\cite{Hedman_VLM_1965}, with unsteady modifications introduced by Thrasher et al. \cite{Thrasher1977} and Konstadinopoulos et al. \cite{Konstadinopoulos1985} in the 1970s and 80s.
However, the VLM, with various modifications
and enhancements, is used even today for low-order modeling and
engineering applications, with recent examples ranging from flight
dynamics analysis \cite{drela_avl, Obradvoic_Subbarao_2011}, analysis of
yacht sails \cite{FIDDES199635}, calculation of aerodynamic
interference effects \cite{Elzebda_Mook_Nayfeh_1994, Rossow_1995,
Karkehabadi_2004}, post-stall analysis \cite{Mukherjee_poststall_2006,
Rom1993}, flapping-wing analysis
\cite{Nguyen_insect_UVLM_2016, Fritz_Long_2004, Stanford_Beran_2010,Hirato2019},
wind turbines \cite{SIMOES1992129, PESMAJOGLOU20001}, design
optimization \cite{CUSHER201435, Stanford_Beran_2010, Mariens2014} and
aeroelasticity \cite{MURUA201246, Palacios_Murua_Cook_2010,
Murua_Palacios_Graham_2012}.
Modified VLMs have also been extensively used for modeling steady and unsteady flows past delta-wings \cite{Traub1999},
propeller aerodynamics \cite{Kobayakawa1985},
propeller-wing interactions \cite{Witkowski1989},
ground effect and formation flight \cite{Frazier2003, King2005,Han2005,Zhang2017},
compressibility effects and transonic flow over wings \cite{Batina1986,Melin2010},
system identification \cite{Venkataraman2019},
and for rapid performance prediction in adaptive control of aircraft \cite{Kim2010,Menon2013}.
The current work, along similar lines, aims to extend the VLM for modeling separation and stall.

Extensive research has been carried out to extend the range of potential-flow-based methods to obtain aerodynamic predictions beyond the linear region. Some methods \cite{Valarezo1994,Phillips2007} use empirical relations based on the lift curve of the airfoil obtained from experimental or CFD data to obtain maximum wing lift. While these methods can accurately predict $C_{L,\text{max}}$, they do not predict the wing behavior well in the post-stall region.
Another common approach is
to modify the potential flow-based equations of traditional low-order methods to
model the effects of thick and separated boundary layers. {Often, this
modification is achieved using a strip-theory based approach. Strip theory has been widely used to predict the behavior of wings based on the behavior of their airfoils \cite{Rodden1959,PamadiTaylor-1984-JofAC-SpinningAirplane,Liu1988,Cebeci1989,Wang2010,Castellani2017}.} To calculate the loads on the wing, it
is discretized into strips and the behavior of each strip is approximated to \revnote{that}{\#3.16} of the corresponding airfoil.
For each airfoil, viscous input data is supplied, often in the form of airfoil
lift ($C_l$-$\alpha$) curves which form the convergence criteria while solving the
3-D potential flow equations to calculate spanwise loading. Convergence is achieved by iteratively modifying the circulation distribution over the surface
\cite{tani_1934,schairer_1939,sivells_neely_1947,sears_1956,levinsky_1976,piszkin_levinsky_1976,anderson_llt_1980,mccormick_nonlinear_llt_1989}
($\Gamma$-correction methods),
or the effective angle of attack of the strips \cite{Purser1951,Hunton1953,tseng_lan_1988,bruce_owens_nonlinear_weissinger,van_dam_nonlin_wing_2001,wickenheiser_garcia_2011} ($\alpha$-correction methods).
These approaches yield sufficiently accurate results for simple unswept
geometries, providing a significant cost-benefit compared to higher fidelity
approaches such as CFD.
Dias \cite{Dias2016} uses the Kirchhoff-Helmholtz formulation to obtain the coefficient of lift for each section of a wing represented as a lifting line. The equation of the lifting line is modified to include the effect of separation. The location of the separation point, denoted in that work by $X$, is the variable used to change the viscous behavior of each section. The variation of the location of the separation point with angle of attack is specified as an empirical equation derived by fitting experimental observations. Iterations are performed until the change in the effective angle of attack of the sections due to a change in $X$ becomes negligible.
Chreim et al. \cite{Chreim2018} model viscous effects in their implementation of lifting-line theory by moving the location of the collocation points points in the chordwise direction for each section. Changing the location of the collocation points has the effect of changing the lift-curve slope for each section. The method calculates the required lift-curve slope for each section so that its operating point may fall on the viscous lift curve of the airfoil.
Gabor et al. \cite{Gabor2016} apply a $\Gamma$-correction to a vortex lattice method to calculate the circulation distribution required on the surface to change the strip behavior to be identical to that of an airfoil. Corrections to the circulation strength of each vortex ring are obtained using a Jacobian-based Newton iteration.
The work by dos Santos and Marques \cite{Santos2018} uses a $\Gamma$-correction approach to apply viscous corrections to inviscid solutions obtained from a VLM. The elements of the aerodynamic influence coefficient (AIC) matrix are corrected based on Kirchhoff's model for separated flow over a flat plate, where the separation point location is estimated using a semi-empirical model developed by Leishman and Beddoes \cite{Leishman1989}.
Kharlamov et al. \cite{Kharlamov2018} use a 2D URANS solver modified to obtain solutions for ``infinite-swept-wings'', which includes the effects of sweep in the lift-curves of the 2D sections of the wing. An $\alpha$-correction method is used to modify the effective angle of attack of the sections of the wing. The correction to $\alpha$ for each section is based on the change in $C_l$ required at that section and the lift-curve slope.
A similar approach is used by Gallay and Laurendeau \cite{Gallay2016} and Parenteau et al. \cite{Parenteau2018,Parenteau2018b}.

A method developed at NCSU's Applied Aerodynamics Group uses the concept of ``decambering'', wherein the camber of the sections of the wing is reduced at high angles of attack to model the separation of the boundary layer and the accompanying reduction in lift.
As with the other methods described above, %readily available
viscous lift data for the airfoil from experiments or computations is supplied to the decambering method. In contrast to the methods discussed previously, a ``decambering flap'' is used to implement the viscous correction by modifying the shape of the effective body.
The decambering approach provides accurate predictions at high angles of attack \cite{Mukherjee_poststall_2006,Paul_Gopa_Iteration_Schemes,gopalarathnam_paul_petrilli_ASM_2012}.
% Motivated by the success of the decambering approach in predicting the lift behavior for unswept wings at high angles of attack, the current research effort aims to extend this success to predicting the post-stall behavior of the lift, drag, and moment on wings having arbitrary planforms, and on multiple-surface configurations.

This paper describes the concept of decambering and its application to a potential-flow method to obtain viscous load predictions for airfoils and wings experiencing separated flow. A novel decambering approach dubbed ``nonlinear decambering'' is presented. In contrast to previous ``linear'' decambering approaches which used two linear decambering flap deflections hinged at predetermined locations to obtain the required drop in lift and moment associated with boundary-layer separation, the nonlinear decambering approach achieves this using a single parabolic decambering flap. The nonlinear flap for each section is hinged at the predicted location of flow separation, allowing the flap to better approximate the shape of the separated boundary layer.
\revnote{The use of a vortex lattice method allows for calculation of the chordwise distribution of the surface loading over the entire lifting surface as opposed only the spanwise distribution that is obtained using an approach based on lifting-line theory. Additionally, the vortex lattice method can correctly calculate the inviscid circulation distribution over swept wings and more complicated planforms, a capability that lifting-line theory does not possess. This capability allows for prediction of post-stall aerodynamics for swept wings, an early version of which is presented in \cite{Hosangadi2015}.}{\#3.2}
The benefits of the nonlinear decambering over other post-stall low-order methods discussed above, including the linear decambering approaches, are the capability to predict separation patterns along the wing span and cross-sectional separated-flow profiles. These benefits serve as essential stepping stones to extension of the current work to predictions of swept-wing stall and viscous wakes behind stalled wings. \revnote{Although the decambering method is incorporated in a VLM in the current work, it can also be applied to other inviscid prediction methods like surface panel methods.}{\#3.7}

An overview of the underlying vortex lattice method (VLM) implemented in this work is given in \sref{sec:vlm}. \sref{sec:decambering} discusses the background of the decambering method and the main assumptions of the approach. A detailed description of the nonlinear decambering approach is given in \sref{sec:nonlin-decambering}. For flow over an airfoil, the application of nonlinear decambering is relatively straightforward, as described in \sref{sec:nld-2D}.
% \multiref{s}{sec:nld-3D}{sec:iter-nld}
\sref{sec:nld-3D} covers the complications involved in applying nonlinear decambering to a three-dimensional wing. Finally, results from the low-order method for various geometries are compared against experimental results % from Ostowari and Naik \cite{naik_ostowari_nrel}
and against 3D RANS CFD solutions obtained using ANSYS Fluent in \sref{sec:unswept-results}.
