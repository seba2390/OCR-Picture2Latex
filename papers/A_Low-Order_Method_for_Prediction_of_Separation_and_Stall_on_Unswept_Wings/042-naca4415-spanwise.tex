\subsection{Comparison of Spanwise Distributions of Lift and Moment}
\label{sec:spanwise-loads}

\fref{fig:n4415-ar12-cldist} shows the spanwise distributions of section lift coefficient ($C_l$) and pitching moment  ($C_m$) for the NACA 4415 $\ar{12}$ wing at angles of attack before stall ($\alpha=10\degree$), slightly post-stall ($\alpha=20\degree$), and well beyond stall ($\alpha=32\degree$) at which the boundary layer separates close to the leading edge on a large portion of the wing.
Before stall, the inviscid prediction is close to the viscous solution from CFD. As the angle of attack is increased, the wing stalls at the root. The drop in lift and moment on the inboard sections is predicted well by the viscous LOM. Well beyond stall, the low-order prediction agrees quite well with the CFD solution.
%The accuracy of the low-order prediction at the wingtips is affected by the wingtip vortices. \PHNote{this sentence...}
The large discrepancy between CFD and the low-order method at the wingtips is attributed to the absence of a model in the low-order method to predict the detachment of the tip vortices at higher angles of attack. This discrepancy becomes more evident for the lower aspect-ratio wings, for which the wingtip vortices affect flow over a considerable portion of the wing.

\begin{figure*}
    \centering
    \begin{subfigure}[t]{0.45\textwidth}
        \centering
        \includegraphics[width=0.9\textwidth]{{figs/eps_fig/spanwise-dist/naca4415_ar12_swp0_a10.0_cldist}.eps}
        \caption{$\alpha=10\degree$\label{sfig:4415-ar12-cldist-a10}}
    \end{subfigure}%
    ~
    \begin{subfigure}[t]{0.45\textwidth}
        \centering
        \includegraphics[width=0.9\textwidth]{{figs/eps_fig/spanwise-dist/naca4415_ar12_swp0_a10.0_cmdist}.eps}
        \caption{$\alpha=10\degree$\label{sfig:4415-ar12-cmdist-a10}}
    \end{subfigure}%

    \begin{subfigure}[t]{0.45\textwidth}
        \centering
        \includegraphics[width=0.9\textwidth]{{figs/eps_fig/spanwise-dist/naca4415_ar12_swp0_a20.0_cldist}.eps}
        \caption{$\alpha=20\degree$\label{sfig:4415-ar12-cldist-a20}}
    \end{subfigure}%
    ~
    \begin{subfigure}[t]{0.45\textwidth}
        \centering
        \includegraphics[width=0.9\textwidth]{{figs/eps_fig/spanwise-dist/naca4415_ar12_swp0_a20.0_cmdist}.eps}
        \caption{$\alpha=20\degree$\label{sfig:4415-ar12-cmdist-a20}}
    \end{subfigure}%

    \begin{subfigure}[t]{0.45\textwidth}
        \centering
        \includegraphics[width=0.9\textwidth]{{figs/eps_fig/spanwise-dist/naca4415_ar12_swp0_a32.0_cldist}.eps}
        \caption{$\alpha=32\degree$\label{sfig:4415-ar12-cldist-a32}}
    \end{subfigure}%
    ~
    \begin{subfigure}[t]{0.45\textwidth}
        \centering
        \includegraphics[width=0.9\textwidth]{{figs/eps_fig/spanwise-dist/naca4415_ar12_swp0_a32.0_cmdist}.eps}
        \caption{$\alpha=32\degree$\label{sfig:4415-ar12-cmdist-a32}}
    \end{subfigure}%

    \caption{Spanwise distributions of $C_l$ and $C_m$ at pre- and post-stall angles of attack from CFD (black), inviscid LOM (red), and viscous LOM (blue) for the NACA4415 $\ar{12}$ wing (Case C\textsubscript{2})}
    \label{fig:n4415-ar12-cldist}
\end{figure*}

\begin{figure}
    \centering
    \includegraphics[width=3in]{figs/eps_fig/spanwise-dist/naca4415_ar12_swp0_allalpha_fdist.eps}
    \caption{Separation line predicted by the LOM (blue) and CFD (black) for angles of attack ranging from pre-stall to post-stall (Case C\textsubscript{2}). Right half of the wing is shown.}
    \label{fig:naca4415_ar12_swp0_allalpha_fdist}
\end{figure}

\fref{fig:naca4415_ar12_swp0_allalpha_fdist} compares the separation line predicted by the \methodabbr with CFD predictions at the same angles of attack as above. At $\alpha=10\degree$ (well before stall), there is only a small amount of separation, indicated by the separation line being close to $f=1$ (trailing-edge) at all sections. The decambering method models the drop in lift using a trailing-edge flap hinged at the separation line. The other decambering parameters at a section ($\delta_l, m$) depend on the value of $f$. If $f \approx 1$, unphysically large values of $\delta_l$ and $m$ are required to model even a small drop in lift. Therefore, in the low-order method, the aft-most location of the hinge for the decambering flap is constrained to $f \leq 0.8$. As the angle of attack increases beyond stall ($20\degree$), we see significant separation over large portions of the wing. The undulations in the separation line from CFD indicate the presence of stall cells on the upper surface of the wing. While these stall cells are not predicted by the low-order method, the overall agreement of the predicted separation line with the CFD solution is remarkably good. As the angle of attack is increased well beyond stall ($\alpha=32\degree$), we see from the CFD solution that most of the wing experiences fully separated flow. This separation-line behavior is again predicted well by the low-order method. %
%
% \PHNote{What about other spanwise distributions? Should I just refer to my dissertation?}
Spanwise $C_l$, $C_m$, and $f$ distributions for the other geometries given in \tref{tab:geoms} agree similarly well with CFD solutions and are included in Ref. \cite{PranavThesis}.
