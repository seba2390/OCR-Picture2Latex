\subsection{Comparison of Decambering Shape with CFD Velocity Contours}
\label{sec:dec-shape}

As the angle of attack increases, the separated boundary layer changes the effective shape of the wing.
The nonlinear decambering method models the effects of a separated boundary layer using a parabolic decambering flap to simultaneously achieve a drop in both lift and moment.
% \multiref{f}{fig:n4415-ar12-y1.5-blpics}{fig:n4415-ar12-y5.1-blpics}
% This observation is applied later in \cref{ch:wakepred} to estimate the location of the wake behind the wing, which is used by the \methodabbr to predict deep stall of downstream surfaces.
%
% \PHNote{Is this size good? or too small?}
\begin{figure}[!ht]
    \centering
    \begin{tabular}{p{0.38in} c c c c}
        %\hline
        %\hline
        %\diagbox[innerwidth=0.35in]{$\alpha$}{$2y/b$}
        $2y/b\rightarrow$ \\ $\alpha\downarrow$ & 0.25  & 0.45 & 0.65 & 0.85 \\
        %\hline
        \hfil$14\degree$ & \begin{subfigure}[b]{0.20\textwidth}
    \centering
    \includegraphics[trim=2.5in 3.05in 2in 2.2in, clip, width=0.9\textwidth]{{figs/decam_lines/bl_vmag_a14_y1.5}.png}
    % \caption{$\alpha=14\degree$\label{sfig:4415-ar12-blvmag-a14-y1.5}}
\end{subfigure}% & \begin{subfigure}[b]{0.20\textwidth}
    \centering
    \includegraphics[trim=2.5in 3.05in 2in 2.2in, clip, width=0.9\textwidth]{{figs/decam_lines/bl_vmag_a14_y2.7}.png}
    % \caption{$\alpha=14\degree$\label{sfig:4415-ar12-blvmag-a14-y2.7}}
\end{subfigure}% & \begin{subfigure}[b]{0.20\textwidth}
    \centering
    \includegraphics[trim=2.5in 3.05in 2in 2.2in, clip, width=0.9\textwidth]{{figs/decam_lines/bl_vmag_a14_y3.9}.png}
    % \caption{$\alpha=14\degree$\label{sfig:4415-ar12-blvmag-a14-y3.9}}
\end{subfigure}% & \begin{subfigure}[b]{0.20\textwidth}
    \centering
    \includegraphics[trim=2.5in 3.05in 2in 2.2in, clip, width=0.9\textwidth]{{figs/decam_lines/bl_vmag_a14_y5.1}.png}
    % \caption{$\alpha=14\degree$\label{sfig:4415-ar12-blvmag-a14-y5.1}}
\end{subfigure}% \\
        \hfil$18\degree$ & \begin{subfigure}[b]{0.20\textwidth}
    \centering
    \includegraphics[trim=2.5in 3.05in 2in 2.2in, clip, width=0.9\textwidth]{{figs/decam_lines/bl_vmag_a18_y1.5}.png}
    % \caption{$\alpha=18\degree$\label{sfig:4415-ar12-blvmag-a18-y1.5}}
\end{subfigure}% & \begin{subfigure}[b]{0.20\textwidth}
    \centering
    \includegraphics[trim=2.5in 3.05in 2in 2.2in, clip, width=0.9\textwidth]{{figs/decam_lines/bl_vmag_a18_y2.7}.png}
    % \caption{$\alpha=18\degree$\label{sfig:4415-ar12-blvmag-a18-y2.7}}
\end{subfigure}% & \begin{subfigure}[b]{0.20\textwidth}
    \centering
    \includegraphics[trim=2.5in 3.05in 2in 2.2in, clip, width=0.9\textwidth]{{figs/decam_lines/bl_vmag_a18_y3.9}.png}
    % \caption{$\alpha=18\degree$\label{sfig:4415-ar12-blvmag-a18-y3.9}}
\end{subfigure}% & \begin{subfigure}[b]{0.20\textwidth}
    \centering
    \includegraphics[trim=2.5in 3.05in 2in 2.2in, clip, width=0.9\textwidth]{{figs/decam_lines/bl_vmag_a18_y5.1}.png}
    % \caption{$\alpha=18\degree$\label{sfig:4415-ar12-blvmag-a18-y5.1}}
\end{subfigure}% \\
        \hfil$22\degree$ & \begin{subfigure}[b]{0.20\textwidth}
    \centering
    \includegraphics[trim=2.5in 3.05in 2in 2.2in, clip, width=0.9\textwidth]{{figs/decam_lines/bl_vmag_a22_y1.5}.png}
    % \caption{$\alpha=22\degree$\label{sfig:4415-ar12-blvmag-a22-y1.5}}
\end{subfigure}% & \begin{subfigure}[b]{0.20\textwidth}
    \centering
    \includegraphics[trim=2.5in 3.05in 2in 2.2in, clip, width=0.9\textwidth]{{figs/decam_lines/bl_vmag_a22_y2.7}.png}
    % \caption{$\alpha=22\degree$\label{sfig:4415-ar12-blvmag-a22-y2.7}}
\end{subfigure}% & \begin{subfigure}[b]{0.20\textwidth}
    \centering
    \includegraphics[trim=2.5in 3.05in 2in 2.2in, clip, width=0.9\textwidth]{{figs/decam_lines/bl_vmag_a22_y3.9}.png}
    % \caption{$\alpha=22\degree$\label{sfig:4415-ar12-blvmag-a22-y3.9}}
\end{subfigure}% & \begin{subfigure}[b]{0.20\textwidth}
    \centering
    \includegraphics[trim=2.5in 3.05in 2in 2.2in, clip, width=0.9\textwidth]{{figs/decam_lines/bl_vmag_a22_y5.1}.png}
    % \caption{$\alpha=22\degree$\label{sfig:4415-ar12-blvmag-a22-y5.1}}
\end{subfigure}% \\
        \hfil$26\degree$ & \begin{subfigure}[b]{0.20\textwidth}
    \centering
    \includegraphics[trim=2.5in 3.05in 2in 2.2in, clip, width=0.9\textwidth]{{figs/decam_lines/bl_vmag_a26_y1.5}.png}
    % \caption{$\alpha=26\degree$\label{sfig:4415-ar12-blvmag-a26-y1.5}}
\end{subfigure}% & \begin{subfigure}[b]{0.20\textwidth}
    \centering
    \includegraphics[trim=2.5in 3.05in 2in 2.2in, clip, width=0.9\textwidth]{{figs/decam_lines/bl_vmag_a26_y2.7}.png}
    % \caption{$\alpha=26\degree$\label{sfig:4415-ar12-blvmag-a26-y2.7}}
\end{subfigure}% & \begin{subfigure}[b]{0.20\textwidth}
    \centering
    \includegraphics[trim=2.5in 3.05in 2in 2.2in, clip, width=0.9\textwidth]{{figs/decam_lines/bl_vmag_a26_y3.9}.png}
    % \caption{$\alpha=26\degree$\label{sfig:4415-ar12-blvmag-a26-y3.9}}
\end{subfigure}% & \begin{subfigure}[b]{0.20\textwidth}
    \centering
    \includegraphics[trim=2.5in 3.05in 2in 2.2in, clip, width=0.9\textwidth]{{figs/decam_lines/bl_vmag_a26_y5.1}.png}
    % \caption{$\alpha=26\degree$\label{sfig:4415-ar12-blvmag-a26-y5.1}}
\end{subfigure}% \\
        %\hline
        %\hline
    \end{tabular}
    \caption{The decambered camberline (red) at spanwise stations of the NACA 4415 $\ar{12}$ wing (Case C\textsubscript{2}) overlaid on a contour plot of the velocity magnitude $V/V_\infty$ from pre-stall to post-stall angles of attack}
    \label{fig:n4415-blpics}
\end{figure}%
%
% \begin{figure*}[!h]
%     \centering
%     \begin{subfigure}[t]{0.22\textwidth}
%         \centering
%         \includegraphics[trim=2.5in 3.05in 2in 2.2in, clip, width=0.9\textwidth]{{figs/decam_lines/bl_vmag_a14_y1.5}.png}
%         \caption{$\alpha=14\degree$\label{sfig:4415-ar12-blvmag-a14-y1.5}}
%     \end{subfigure}%
%     ~
%     \begin{subfigure}[t]{0.22\textwidth}
%         \centering
%         \includegraphics[trim=2.5in 3.05in 2in 2.2in, clip, width=0.9\textwidth]{{figs/decam_lines/bl_vmag_a18_y1.5}.png}
%         \caption{$\alpha=18\degree$\label{sfig:4415-ar12-blvmag-a18-y1.5}}
%     \end{subfigure}%
%
%     \begin{subfigure}[t]{0.22\textwidth}
%         \centering
%         \includegraphics[trim=2.5in 3.05in 2in 2.2in, clip, width=0.9\textwidth]{{figs/decam_lines/bl_vmag_a22_y1.5}.png}
%         \caption{$\alpha=22\degree$\label{sfig:4415-ar12-blvmag-a22-y1.5}}
%     \end{subfigure}%
%     ~
%     \begin{subfigure}[t]{0.22\textwidth}
%         \centering
%         \includegraphics[trim=2.5in 3.05in 2in 2.2in, clip, width=0.9\textwidth]{{figs/decam_lines/bl_vmag_a26_y1.5}.png}
%         \caption{$\alpha=26\degree$\label{sfig:4415-ar12-blvmag-a26-y1.5}}
%     \end{subfigure}%
%
%     \caption{The decambered camberline (red) and separation point location (blue symbol) for the NACA 4415 $\ar{12}$ wing (Case A) overlaid on a contour plot of the velocity magnitude $V/V_\infty$ at an inboard section ($2y/b=0.25$) from pre-stall to post-stall angles of attack}
%     \label{fig:n4415-ar12-y1.5-blpics}
% \end{figure*}
%
% \begin{figure*}[!h]
%     \centering
%     \begin{subfigure}[t]{0.22\textwidth}
%         \centering
%         \includegraphics[trim=2.5in 3.05in 2in 2.2in, clip, width=0.9\textwidth]{{figs/decam_lines/bl_vmag_a14_y2.7}.png}
%         \caption{$\alpha=14\degree$\label{sfig:4415-ar12-blvmag-a14-y2.7}}
%     \end{subfigure}%
%     ~
%     \begin{subfigure}[t]{0.22\textwidth}
%         \centering
%         \includegraphics[trim=2.5in 3.05in 2in 2.2in, clip, width=0.9\textwidth]{{figs/decam_lines/bl_vmag_a18_y2.7}.png}
%         \caption{$\alpha=18\degree$\label{sfig:4415-ar12-blvmag-a18-y2.7}}
%     \end{subfigure}%
%
%     \begin{subfigure}[t]{0.22\textwidth}
%         \centering
%         \includegraphics[trim=2.5in 3.05in 2in 2.2in, clip, width=0.9\textwidth]{{figs/decam_lines/bl_vmag_a22_y2.7}.png}
%         \caption{$\alpha=22\degree$\label{sfig:4415-ar12-blvmag-a22-y2.7}}
%     \end{subfigure}%
%     ~
%     \begin{subfigure}[t]{0.22\textwidth}
%         \centering
%         \includegraphics[trim=2.5in 3.05in 2in 2.2in, clip, width=0.9\textwidth]{{figs/decam_lines/bl_vmag_a26_y2.7}.png}
%         \caption{$\alpha=26\degree$\label{sfig:4415-ar12-blvmag-a26-y2.7}}
%     \end{subfigure}%
%
%     \caption{The decambered camberline (red) and separation point location (blue symbol) for the NACA 4415 $\ar{12}$ wing (Case A) overlaid on a contour plot of the velocity magnitude $V/V_\infty$ at a middle section ($2y/b=0.45$) from pre-stall to post-stall angles of attack}
%     \label{fig:n4415-ar12-y2.7-blpics}
% \end{figure*}
%
% \begin{figure*}[!h]
%     \centering
%     \begin{subfigure}[t]{0.22\textwidth}
%         \centering
%         \includegraphics[trim=2.5in 3.05in 2in 2.2in, clip, width=0.9\textwidth]{{figs/decam_lines/bl_vmag_a14_y3.9}.png}
%         \caption{$\alpha=14\degree$\label{sfig:4415-ar12-blvmag-a14-y3.9}}
%     \end{subfigure}%
%     ~
%     \begin{subfigure}[t]{0.22\textwidth}
%         \centering
%         \includegraphics[trim=2.5in 3.05in 2in 2.2in, clip, width=0.9\textwidth]{{figs/decam_lines/bl_vmag_a18_y3.9}.png}
%         \caption{$\alpha=18\degree$\label{sfig:4415-ar12-blvmag-a18-y3.9}}
%     \end{subfigure}%
%
%     \begin{subfigure}[t]{0.22\textwidth}
%         \centering
%         \includegraphics[trim=2.5in 3.05in 2in 2.2in, clip, width=0.9\textwidth]{{figs/decam_lines/bl_vmag_a22_y3.9}.png}
%         \caption{$\alpha=22\degree$\label{sfig:4415-ar12-blvmag-a22-y3.9}}
%     \end{subfigure}%
%     ~
%     \begin{subfigure}[t]{0.22\textwidth}
%         \centering
%         \includegraphics[trim=2.5in 3.05in 2in 2.2in, clip, width=0.9\textwidth]{{figs/decam_lines/bl_vmag_a26_y3.9}.png}
%         \caption{$\alpha=26\degree$\label{sfig:4415-ar12-blvmag-a26-y3.9}}
%     \end{subfigure}%
%
%     \caption{The decambered camberline (red) and separation point location (blue symbol) for the NACA 4415 $\ar{12}$ wing (Case A) overlaid on a contour plot of the velocity magnitude $V/V_\infty$ at an outboard section ($2y/b=0.65$) from pre-stall to post-stall angles of attack}
%     \label{fig:n4415-ar12-y3.9-blpics}
% \end{figure*}
%
% \begin{figure*}[!h]
%     \centering
%     \begin{subfigure}[t]{0.22\textwidth}
%         \centering
%         \includegraphics[trim=2.5in 3.05in 2in 2.2in, clip, width=0.9\textwidth]{{figs/decam_lines/bl_vmag_a14_y5.1}.png}
%         \caption{$\alpha=14\degree$\label{sfig:4415-ar12-blvmag-a14-y5.1}}
%     \end{subfigure}%
%     ~
%     \begin{subfigure}[t]{0.22\textwidth}
%         \centering
%         \includegraphics[trim=2.5in 3.05in 2in 2.2in, clip, width=0.9\textwidth]{{figs/decam_lines/bl_vmag_a18_y5.1}.png}
%         \caption{$\alpha=18\degree$\label{sfig:4415-ar12-blvmag-a18-y5.1}}
%     \end{subfigure}%

%     \begin{subfigure}[t]{0.22\textwidth}
%         \centering
%         \includegraphics[trim=2.5in 3.05in 2in 2.2in, clip, width=0.9\textwidth]{{figs/decam_lines/bl_vmag_a22_y5.1}.png}
%         \caption{$\alpha=22\degree$\label{sfig:4415-ar12-blvmag-a22-y5.1}}
%     \end{subfigure}%
%     ~
%     \begin{subfigure}[t]{0.22\textwidth}
%         \centering
%         \includegraphics[trim=2.5in 3.05in 2in 2.2in, clip, width=0.9\textwidth]{{figs/decam_lines/bl_vmag_a26_y5.1}.png}
%         \caption{$\alpha=26\degree$\label{sfig:4415-ar12-blvmag-a26-y5.1}}
%     \end{subfigure}%

%     \caption{The decambered camberline (red) and separation point location (blue symbol) for the NACA 4415 $\ar{12}$ wing (Case A) overlaid on a contour plot of the velocity magnitude $V/V_\infty$ close to the wingtip ($2y/b=0.85$) from pre-stall to post-stall angles of attack}
%     \label{fig:n4415-ar12-y5.1-blpics}
% \end{figure*}
%

\fref{fig:n4415-blpics} compares the geometry of the decambered wing at multiple sections with contour plots showing the ratio of velocity magnitude to the freestream velocity ($V/V_\infty$). At $\alpha=14\degree$, the flow is mostly attached at all sections of the wing.
A small decambering flap is sufficient to accurately model the effective shape change due to the boundary layer. As the angle of attack is increased to $18\degree$, the separation point moves closer to the leading edge, the wing stalls and the boundary layer becomes thicker. The forward movement of the separation point is predicted well by the \methodname at all sections away from the wingtip. The thicker boundary layer is mimicked well by the decambering flap having a larger deflection and trailing-edge height. We also see that the \methodname accurately predicts the tendency of a rectangular wing to stall to the root. Upon increasing the angle of attack further to $22\degree$ and then to $26\degree$, we observe that the separation point at most sections is very close to the leading edge. The decambering flap approximately models the centerline of the thick boundary layer.
This observation was applied in Ref.~\cite{Hosangadi2018} to predict the location of the viscous wake behind the wing, and the velocity profile in the wake without the need for expensive boundary layer calculations. %
% \PHNote{Refer to dissertation?}
Figures showing the comparison of the decambered camberlines for other geometries are included in Ref. \cite{PranavThesis}