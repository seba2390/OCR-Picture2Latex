\section{Limitations and Future Extensions of the Nonlinear Decambering Method}
\label{sec:limitations}
The nonlinear decambering method provides excellent predictions for wings of different airfoils and moderate to high aspect ratios, with and without taper and for wings experiencing small roll rates up to angles of attack well beyond stall. However, the accuracy of predictions from the low-order method suffers beyond $\alpha \approx 35\degree$ and for wings having swept planforms. These limitations are discussed in the following sections.

\subsection{Extremely high angles of attack}
\newcommand{\bigO}{\ensuremath{\mathcal{O}}}
The decambering method has difficulty converging to a solution beyond $\alpha \approx 35\degree$. This difficulty is a consequence of the way decambering changes the shape of the wing section. As shown in \fref{fig:decam-in-place}, the decambering is applied by simply tilting the normal vectors in place. This approximation is used so that the AIC matrix may be calculated once and then reused for a given geometry so as to speed up the time required to obtain a solution. However, at extremely high angles of attack, such as at $\alpha=40\degree$ shown in \fref{fig:decam-alpha40}, a large decambering flap is required to obtain the required drop in lift and moment. The zero-normal-flow boundary condition is enforced at the collocation points in the direction of the normal vectors. The new camberline is no longer approximated satisfactorily by simply rotating the normal vectors at their original collocation points. However, recalculating the AIC matrix as a result of changing the geometry is a computationally expensive process, requiring $\bigO(n^2)$ computations for a vortex lattice with $n$ ring elements. More research is required to develop a useful compromise between speed and accuracy.

\begin{figure}[!h]
    \centering
    \includegraphics[width=0.3\textwidth]{figs/eps_fig/decambering_in_place_a40.eps}
    \caption{An illustration of the decambering implemented in the VLM by rotating normal vectors in situ at a high $\alpha$. The decambered camberline is no longer adequately approximated by simply rotating the normal vectors.}
    \label{fig:decam-alpha40}
\end{figure}


\subsection{Effects of sweep angle}
The primary assumption of the decambering method is that the behavior of the sections of the three-dimensional wing is identical to that of the two-dimensional airfoil.
Based on this assumption, the airfoil $C_l$, $C_d$, and $C_m$ vs. $\alpha$ curves are used to obtain the target operating points for each section.
A spanwise pressure gradient exists on swept wings that causes spanwise transport of the separated boundary layer.
This spanwise transport affects the characteristics of the sections of the swept wing and invalidates the assumption that the behavior of each section matches that of the airfoil.
This causes the decambering method to identify the wrong ($C_l, \aeff$) and ($C_m, \aeff$) operating points as the target for the sections, and significantly overpredict $C_L$ and $C_M$ for the swept wing at stall, as shown in \fref{fig:n4415-swpwing-limitation}.
An early approach to correct for these swept-wing effects is discussed in Ref.~\cite{Hosangadi2015}.
% \cref{ch:sweep-effects}.

\begin{figure}[!h]
    \centering
    \includegraphics[width=\figwidth]{figs/eps_fig/naca4415_coeffs_sweep_airfoilin.eps}
    \caption{Total lift, drag, and pitching moment vs $\alpha$ for the NACA4415 $\ar{16}$ $30\degree$ swept wing from CFD (black) and the viscous LOM using airfoil input curves (blue)}
    \label{fig:n4415-swpwing-limitation}
\end{figure}

\subsection{Limitations of RANS CFD}
The limitations of RANS CFD models in predicting highly separated flows are well known~\cite{Strelets2001,Menter2011}.
The low-order method presented in this paper uses input $C_l$, $C_d$, $C_m$, and $f$ vs. $\alpha$ curves obtained from 2D RANS CFD to make predictions for 3D wings.
While the low-order predictions compare well with 3D RANS CFD solutions, it is prudent to verify the results using 2D input data  and 3D solutions obtained from a higher fidelity computational method such as LES or DES in a future study.
% While the low-order results obtained in this work using 2D RANS solutions as input match excellently with 3D RANS solutions, it would be prudent to verify the results using a higher fidelity computational method such as LES or DES in future studies. For the scope of this study, the

\subsection{Low aspect ratio wings}
\revnote{
As noted in \sref{sec:unswept-results}, the predictions from the low-order
method become less accurate as the aspect ratio of the wing reduces.
At high angles of attack, the tip vortex separates from
the wingtip which affects the forces and moments on the
wing.
While the detached vortex does not significantly affect
the flow over wings of higher aspect ratios,
low aspect ratio wings ($\ar \lessapprox 6$) experience a
greater disruption of
flow and therefore modeling the effects of the modified
wingtip vorticity is essential to accurate modeling of
the flow and loads on such wings.
The current method can be easily augmented with a wingtip
vortex model, such as that proposed by Loewenthal and
Gopalarathnam \cite{loewenthal2019}.}{\#3.3}


\subsection{Airfoils exhibiting sharp stall}
% \PHNote{We could demonstrate what happens with the Merz wing and the NACA0009 wing?}
The current implementation of the \methodabbr also has problems converging at post-stall conditions for wings with airfoils that have abrupt stall behavior (sudden drop in $C_l$ at stall). This convergence problem is a subject of continuing work, and may require improvements to the numerical methods used in this work.

