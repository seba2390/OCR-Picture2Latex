\subsection{Experimental Validation}
\label{sec:exp-results}
% \PHNote{Rewrite}
The 2D viscous operating curves used as input to the low-order method can be obtained experimentally or from computational solutions. Here, we present the results obtained from the \methodname for rectangular NACA 4418 wings of three different aspect ratios using input curves obtained from Ostowari ans Naik \cite{naik_ostowari_nrel}.
A key requirement for the nonlinear decambering approach is the knowledge of the separation location. Since the separation curve ($f$ vs. $\alpha$) for the airfoil is not usually easily available from experiments, Beddoes' method (\eref{eqn:beddoes-f}) is used to calculate an approximate separation curve.

% \tref{tab:exp-geoms} lists the wing geometries for which results are presented in this section along with the sources for the input (2D) curves and the 3D experimental comparison data.

% \begin{table}[h!]
%     \centering
%     \caption{Summary of cases presented for experimental verification}
%     \label{tab:exp-geoms}
%     \renewcommand{\arraystretch}{1.5}
%     \begin{tabular}{ | c | m{2in} | c | m{2in} | }
%         \hline
%         \hline
%         Case & Airfoil & Aspect Ratio & Notes \\
%         \hline
%         H & NACA 4418 & ... & 2D source: \cite{naik_ostowari_nrel}; 3D source: \cite{naik_ostowari_nrel}; $Re = 0.75\times10^6$ \\
%         \hline
%         I & Root: NACA 4424; Tip: NACA 4412 & 12.06 &
%         Taper ratio: 0.4; Washout: $3\degree$ ; 2D source: \cite{abbott}; 3D source: \cite{McVeigh1971}; $Re = 2.87\times 10^6$ \\
%         \hline
%     \end{tabular}
% \end{table}

Wind tunnel results used in Case A are from the experiments \revnote{which were}{\#1.3} performed by Ostowari and Naik \cite{naik_ostowari_nrel} in the Texas A\&M University wind tunnel. Reflection-plane models of various NACA 44XX family wings were used to obtain force and moment curves at angles of attack ranging from $-10\degrees{}$ to $110\degrees{}$.
Data was obtained at a range of Reynolds numbers for wings of aspect ratios 6, 9, and 12, and for a wing spanning the entire test section ($\ar = \infty$). The results shown here use the viscous curves at $Re = 0.75\times10^6$. The separation curve is calculated using \eref{eqn:beddoes-f} and the lift curve for the NACA 4418 airfoil given in Ref. \cite{naik_ostowari_nrel}.
It was seen that at certain angles of attack, \eref{eqn:beddoes-f} gives a value for $f$ that is greater than 1.
In such cases, the value of $f$ is simply set to 1. The experimentally obtained lift, drag, and moment, and calculated separation curves are shown in \fref{fig:exp-coeffs}.

\begin{figure}[!h]
    \centering
    \includegraphics[width=4in]{figs/eps_fig/naca4418_exp_input.eps}
    \caption{The viscous lift, drag, and moment curves for the NACA 4418 airfoil obtained from experimental tests by Naik and Ostowari \cite{naik_ostowari_nrel}, and the separation curve calculated using Beddoes' model}
    \label{fig:exp-coeffs}
\end{figure}

Using these input curves, the low-order method can predict the loads on 3D wings of various aspect ratios.
The predictions from the low-order method for the NACA 4418 wings of aspect ratios 6, 9, and 12 are shown in \fref{fig:exp-n4418-coeffs} %
%
\begin{figure*}[!h]
    \centering
    \includegraphics[width=\figwidth]{figs/eps_fig/naca4418_exp_coeffs_all.eps}
    \caption{Total coefficients of lift, drag, and pitching moment  for the NACA 4418 wings (Case A\textsubscript{1} -- A\textsubscript{3})}
    \label{fig:exp-n4418-coeffs}
\end{figure*}
% \begin{figure*}[!h]
%     \centering
%     \begin{subfigure}[t]{\textwidth}
%         \centering
%         \includegraphics[width=\figwidth]{figs/eps_fig/naca4418_ar6_swp0_coeffs.eps}
%         \caption{$\ar{6}$\label{sfig:n4418-ar6-coeffs}}
%     \end{subfigure}%

%     \begin{subfigure}[t]{\textwidth}
%         \centering
%         \includegraphics[width=\figwidth]{figs/eps_fig/naca4418_ar9_swp0_coeffs.eps}
%         \caption{$\ar{9}$\label{sfig:n4418-ar9-coeffs}}
%     \end{subfigure}%

%     \begin{subfigure}[t]{\textwidth}
%         \centering
%         \includegraphics[width=\figwidth]{figs/eps_fig/naca4418_ar12_swp0_coeffs.eps}
%         \caption{$\ar{12}$\label{sfig:n4418-ar12-coeffs}}
%     \end{subfigure}%

%     \caption{Total coefficients of lift, drag, and pitching moment  for the NACA4418 wings (Case J)}
%     \label{fig:exp-n4418-coeffs}
% \end{figure*} %
 %
The low-order predictions for lift and drag show excellent agreement with the experimentally obtained values.
The low-order method correctly predicts the increasing lift-curve slope with increase in aspect ratio at low angles of attack.
As the angle of attack increases, the viscous low-order method correctly predicts stall and the associated drop in $C_L$ and rise in $C_D$.
The maximum $C_L$ and stall angle predictions from the viscous low-order method are within 5\% of the experimental values.
The low-order prediction for moment agrees well with experimental result for the $\ar 12$ case.
However, we see that for the smaller aspect ratios, the low-order moment prediction starts to deviate from the experimental result and the error increases with decreasing aspect ratios. This is thought to be due to the interactions of the detached wingtip vortices with the wing, which become important at low aspect ratios but are not modeled by the current low-order method.
