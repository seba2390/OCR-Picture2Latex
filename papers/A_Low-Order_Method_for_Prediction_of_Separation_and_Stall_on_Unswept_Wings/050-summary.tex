\section{Conclusions}
\label{sec:ch2-summary}

A traditional vortex lattice method (VLM) provides accurate potential-flow solutions for three-dimensional wings at low angles of attack at which the boundary layer is thin and mostly attached. At higher angles of attack, at which there is significant flow separation, a VLM significantly overpredicts the lift and moment produced by the wing. The nonlinear decambering method presented in this paper is an augmentation to the potential-flow VLM to model the effects of boundary-layer separation at near-stall and post-stall angles of attack. Viscous data for the two-dimensional airfoils, in the form of experimentally- or computationally-obtained lift, drag, and moment curves are provided as inputs to the decambering method. For each wing section, the deviations in the lift and moment coefficients are computed as the differences in values obtained from the viscous input curves at the effective angle of attack for the section and those predicted by the VLM for that section. A parabolic decambering flap is deflected from the separation location at each section with the aim of bringing these deviations to zero for all the wing sections. The iterative procedure to determine the decambering-flap shapes for all the sections converges when the operating point for each section lies on its two-dimensional viscous input curve. 

The total loads predicted by the low-order method agree well with experimental results and computational solutions for a variety of unswept wings. Stall angle, and lift and moment coefficients at stall are slightly overpredicted in some cases, but generally follow the trends seen in the experimental and computational results. The predictions from the method are seen to become less accurate for wings with low aspect ratio, which is attributed to the unmodeled effects of the separated and rolled-up wing tip vortices. Spanwise distributions of lift and moment compare well with CFD solutions even at post-stall conditions. The method correctly predicts the stall characteristics of unswept wings, with root sections stalling before tip sections for rectangular wings, and stall occurring at the outboard sections first on tapered planforms. A unique capability of the current method is to predict the spanwise variation of the flow separation on the wing. While the method is unable to resolve stall cells that occur on unswept wings at high angles of attack, the predicted separation patterns generally agree well with those obtained from skin-friction lines calculated using CFD solutions. The shapes of the decambering flap also closely mimic the shapes of the separated boundary layer at various sections of a stalled wing. For the wings experiencing a small roll rate, the method accurately predicts the roll damping at pre-stall angles of attack, and the loss thereof after stall.

Improvements to the method could focus on improved convergence for airfoils having abrupt stall characteristics, extensions to very high post-stall angles of attack, and the capability to handle swept wings. Nevertheless, even in its current state, the method shows promise for rapid prediction of stall behavior of unswept wings in steady flight and with quasi-steady roll rates, providing useful capability for design, modeling, and simulation at post-stall conditions.

% % that the rectangular wing stalls at the root.

% The decambering variables ($f$, $\delta_l$, and $m$) calculated by the \methodabbr{} can be used to obtain a predicted separation line and boundary-layer shape
% While the \methodabbr{} is unable to resolve stall cells that occur on unswept wings at high angles of attack, the predicted separation line generally agrees well with that obtained from skin-friction lines calculated using CFD solutions. The shape of the decambering flap closely mimics the shape of the separated boundary layer, and can be used in a rapid estimation of the wake behind the wing as demonstrated in Ref. \cite{Hosangadi2018}.

% For the wings with a small roll rate, the LOM accurately predicts the roll damping at pre-stall angles of attack, and the loss thereof after stall. These quasi-steady predictions are obtained using only steady 2D input data.



% A traditional vortex lattice method (VLM) provides a potential-flow solution that accurately predicts loads on three-dimensional wings at low angles of attack where the boundary layer is thin and mostly attached. At higher angles of attack, the potential-flow solution obtained from the VLM significantly overpredicts the lift and moment produced by the wing.
% The nonlinear decambering method presented in this paper is an augmentation to the potential-flow VLM to model the effects of boundary-layer separation at high angles of attack.
% Input data, in the form of experimentally- or computationally-obtained 2D $C_l$, $C_d$, $C_m$, and $f$ vs. $\alpha$ curves are provided to the decambering method.
% Using a strip-theory approach, the required drop in lift and moment from the inviscid solution is calculated at each section of a 3D wing from 2D input data.
% A parabolic decambering flap is deflected from the separation location at each section so as to effect this change in lift and moment. This process is repeated until the operating point for each section lies on its viscous 2D curve.

% The total loads predicted by the low-order method agree well with experimental observations and 3D RANS CFD solutions. Stall angle, $C_l$, and $C_m$ at stall are slightly overpredicted in some cases, but are significantly improved compared to inviscid predictions. The predictions from the \methodabbr{} are seen to improve with increasing aspect ratio, as wingtip effects become less dominant.
% \PHNote{Not super happy with this sentence: Given the appropriate 2D input data, the \methodabbr{} accurately predicts the loads on wings of different thickness and camberline shapes}

% Spanwise distributions of lift and moment compare well with CFD solutions.
% Based on the lift distributions, the low-order method correctly predicts the stall characteristics of unswept wings, with root sections stalling before tip sections, and of tapered wings, with stall occuring at the outboard sections first.
% % that the rectangular wing stalls at the root.

% The decambering variables ($f$, $\delta_l$, and $m$) calculated by the \methodabbr{} can be used to obtain a predicted separation line and boundary-layer shape
% While the \methodabbr{} is unable to resolve stall cells that occur on unswept wings at high angles of attack, the predicted separation line generally agrees well with that obtained from skin-friction lines calculated using CFD solutions. The shape of the decambering flap closely mimics the shape of the separated boundary layer, and can be used in a rapid estimation of the wake behind the wing as demonstrated in Ref. \cite{Hosangadi2018}.

% For the wings with a small roll rate, the LOM accurately predicts the roll damping at pre-stall angles of attack, and the loss thereof after stall. These quasi-steady predictions are obtained using only steady 2D input data.


% The current \methodabbr is unable to achieve satisfactory convergence in case of airfoils exhibiting abrupt stall behavior. This shortcoming is the subject of continuing work. Additionally, the ability of the low-order method to converge to a solution starts to break down at extremely high angles of attack ($\alpha > 35\degree$). This occurs because the decambering is implemented by simply rotating the panel-normal vectors in place on the original camberline instead of moving them to the decambered camberline in the interest of computational efficiency and speed. At high angles of attack, a large decambering flap is required and the decambered camberline is not close to the original camberline. In such cases, rotating the normal vectors in place does not produce the required drop in lift and moment.
% It is important to note that by using identical 2D input curves for each section, the low-order method assumes that the flow over the wing sections is largely two-dimensional. However, this assumption breaks down in the case of swept wings, where a significant spanwise pressure gradient exists that induces a spanwise velocity in the separated flow. A method to correct the input curves for the LOM for swept wings is under development.

% \PHNote{Need help writing a nice final paragraph, I don't think that up there is a nice way to finish.}

% The Vortex Lattice Method is an inviscid low-order method well-suited to predicting the loads on three-dimensional wings at low angles of attack, where the boundary layer is thin and mostly attached.
% As the angle of attack increases, the adverse pressure gradient induces boundary-layer separation, changing the effective shape of the wing.
% The decambering approach mimics this change in effective shape by deflecting a ``decambering flap'' from the trailing edge of the airfoil. The flow over this modified shape can  be simulated using a potential-flow method to simulate the effects of the thick separated boundary layer.
% The decambering method is applied to a three-dimensional wing using a strip-theory approach, where the wing is divided into spanwise sections, and each section is assumed to behave like an airfoil in viscous flow based on the local effective angle of attack at that section.
% Two conditions must be satisfied by the solution:
% \begin{enumerate*}[label=(\roman*)]
%   \item there is no normal flow through the lifting surface, and
%   \item the operating point for each section lies on the viscous curve of the airfoil.
% \end{enumerate*}
% Interactions between the sections cause complications in the calculation of the decambering parameters, and an iterative procedure is required to obtain the decambering so as to satisfy both conditions mentioned above.
% The total and spanwise distributions of lift and pitching moment are obtained by solving the VLM with the decambered geometry. The profile drag on each section is obtained by interpolating the $C_d$ vs. $\alpha$ curve for the airfoil, and integrated to find the profile drag on the wing.
% The VLM thus augmented by the decambering method gives excellent predictions for unswept symmetric and cambered geometries at angles of attack as high as 35\degree.
% The shape of the decambering flap calculated by the low-order method matches well with the separated boundary layer as seen from contour plots of $V/V_\infty$.
% The \methodname also adequately predicts loads on geometries experiencing small rotational rates and tapered geometries.
% Beyond a wing-angle-of-attack of about $35\degree$, a large decambering flap is required to induce the required drop in lift and moment. As a result, the decambered camberline is not approximated well by simply rotating the normal vectors at the location of the original collocation points, causing difficulties in convergence at extremely high angles of attack.
% Additionally, there are large discrepancies between CFD and low-order predictions for swept wings, since the underlying assumption of the decambering method is violated in the case of swept-wing flows.
% % The following chapter describes modifications needed in order to account for the effects of sweep in the decambering method.

