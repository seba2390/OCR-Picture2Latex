% These three packages allow latex to use a nicer font
% That doesn't pixelate when zoomed in
\usepackage[utf8]{inputenc}
\usepackage[T1]{fontenc}
% \usepackage{lmodern}

\usepackage{graphicx}
%\usepackage{caption} % Caption is deprecated. Use subcaption.
\usepackage{subcaption}
	\graphicspath{{./figs/}}
% \usepackage{amsmath}
% \usepackage{amsfonts}
% \usepackage{amssymb}
\usepackage{float}
\usepackage{epstopdf}

\usepackage{wrapfig}% embedding figures/tables in text (i.e., Galileo style)
\usepackage{threeparttable}% tables with footnotes
\usepackage{dcolumn}% decimal-aligned tabular math columns
	\newcolumntype{d}{D{.}{.}{-1}}
\usepackage{nomencl}% automatic nomenclature generation via makeindex
	\makeglossary

% \usepackage{subfigure}% subcaptions for subfigures
% \usepackage{subfigmat}% matrices of similar subfigures, aka small mulitples

\usepackage{fancyvrb}% extended verbatim environments
	\fvset{fontsize=\footnotesize,xleftmargin=2em}

\usepackage{ifthen}
\usepackage{xstring}


\usepackage{lettrine}% dropped capital at beginning of paragraph
%%%%\usepackage[colorlinks=true]{hyperref} % Link citations, label refs, color links.


% Makes citations blue, because nobody likes
% light green text on white paper
\hypersetup{
	colorlinks=true,
	citecolor=blue
}

% Don't use me. Ever.
\newcommand{\expandref}[2][]{%
	\IfStrEq{#2}{e}{Equation#1}{%
	\IfStrEq{#2}{f}{Figure#1}{%
	\IfStrEq{#2}{c}{Chapter#1}{%
	\IfStrEq{#2}{s}{Section#1}{%
	\IfStrEq{#2}{t}{Table#1}{%
	\IfStrEq{#2}{fs}{Fig#1.}{%
	\IfStrEq{#2}{ss}{Sec.}{%
	\IfStrEq{#2}{es}{Eq#1.}{%
	\IfStrEq{#2}{ap}{Appendix#1}{%
	}}}}}}}}}
}

\newcommand{\uv}[1]{\ensuremath{\mathbf{\hat{#1}}}} % unit vector
\newcommand{\eref}[1]{\hyperref[#1]{\expandref{e}(\ref*{#1})}} % equation ref
\newcommand{\fref}[1]{\hyperref[#1]{\expandref{f}\ref*{#1}}} % figure ref
\newcommand{\tref}[1]{\hyperref[#1]{\expandref{t}\ref*{#1}}} % table ref
\newcommand{\cref}[1]{\hyperref[#1]{\expandref{c}\ref*{#1}}} % chapter ref
\newcommand{\sref}[1]{\hyperref[#1]{\expandref{s}\ref*{#1}}} % section ref
\newcommand{\erefn}[1]{(\ref{#1})} % Just the equation number
\newcommand{\sfref}[1]{\hyperref[#1]{(\protect\subref*{#1})}} % subfig ref
\newcommand{\figref}[1]{\hyperref[#1]{\expandref{fs}\ref*{#1}}} % fig. ref
\newcommand{\secref}[1]{\hyperref[#1]{\expandref{ss}\ref*{#1}}} % sec. ref
\newcommand{\eqnref}[1]{\hyperref[#1]{\expandref{es}\ref*{#1}}} % eq. ref
\newcommand{\appref}[1]{\hyperref[#1]{\expandref{ap}\ref*{#1}}} % eq. ref

% refer to multiple labels of the same type.
% Usage: \multiref{e}{eqn:one}{eqn:two}
\newcommand{\multiref}[3]{%
\IfStrEq{#1}{e}{\expandref[s]{#1}(\ref{#2}) -- (\ref{#3})}{%
\expandref[s]{#1}\ref{#2} -- \ref{#3}
}}

\newcommand{\quotes}[1]{``#1''} % Enclose something in quotes nicely

\newcommand{\note}[1]{\textbf{\color{red}{#1}}} % Bold red text stands out
\newcommand{\degreei}{\ensuremath{^\circ}} % Internal command for the degree symbol
\newcommand{\degree}{\degreei{}} % Degree symbol. Use this
\newcommand{\degrees}{\ensuremath{^\circ ~}} % Degree symbol with a forced nbsp
\newcommand{\numberthis}{\addtocounter{equation}{1}\tag{\theequation}} % Number an equation in a non-numbered float (align* / equation*)

% Make enclused text upper / lower case.
% Usage: \makecase{u}{make me uppercase}
\newcommand{\makecase}[2]{ %
\IfStrEq{#1}{u}{\texorpdfstring{\MakeUppercase #2}{#2}}{%
\IfStrEq{#1}{l}{\texorpdfstring{\MakeLowercase #2}{#2}}{%
}}
}

% Define figwidth to be 80% of textwidth. Use in
% \includegraphics[width=\figwidth] to have standard-width figures everywhere
\newcommand{\figwidth}{0.8\textwidth}

\usepackage{ar} % The aspect ratio package for the AR symbol.
\usepackage{siunitx} % A standard way to include SI units in text

\newcommand{\ar}{\AR} % Just an alias to the aspect ratio command because I can never remember if it's \ar or \AR. Now it's both

\newcommand{\aeffi}{\ensuremath{\alpha_\text{eff}}}
\newcommand{\aeff}{\aeffi{}}
\newcommand{\azli}{\ensuremath{\alpha_{0L}}}
\newcommand{\azl}{\azli{}}
\newcommand{\dl}{\ensuremath{\delta_l}}

\newcommand{\PHNote}[1]{\textbf{PH: #1}}

\newcommand{\RHS}{\text{RHS}}
\newcommand{\LHS}{\text{LHS}}

\usepackage{mathtools}

\newcommand{\tf}{\therefore}
\newcommand{\LESP}{\mathit{LESP}}

\def\etal.{et\penalty50\ al.}

\newcommand{\diff}[2]{\frac{\text{d}#1}{\text{d}#2}}
\newcommand{\pdiff}[2]{\frac{\partial #1}{\partial #2}}

\newcommand{\rednote}[1]{{\color{red}#1}}
