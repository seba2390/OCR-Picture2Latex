\subsection{Predictions for Tapered Wing (Case D)}
\label{sec:tapered-wings}
\fref{fig:n4415-tap-coeff} shows the coefficients of lift, drag, and pitching moment about root-quarter-chord for the tapered wing (Case F in \tref{tab:geoms}). This wing has an aspect ratio $\ar{}=12$, a taper ratio $\lambda = 0.5$, an unswept leading edge, and a root chord $4/3$ times the mean chord. The planform of the wing is shown in \fref{fig:tapered-planforms}.
% \PHNote{CFD and LOM moments are about (0.25,0,0) not root-quarter-chord}

\begin{figure}[!h]
    \centering
    % \begin{subfigure}[t]{\textwidth}
    %     \centering
        \includegraphics[width=\figwidth]{figs/eps_fig/le-unswept-planform-final2.pdf}
    %     \caption{\label{sfig:letap-planform}}
    % \end{subfigure}%

    % \begin{subfigure}[t]{\textwidth}
    %     \centering
    %     \includegraphics[width=\figwidth]{figs/eps_fig/qc-unswept-planform.eps}
    %     \caption{\label{sfig:qctap-planform}}
    % \end{subfigure}%
    % \caption{Planform view of the tapered wings with (a) unswept leading edge and (b) unswept quarter-chord line}
    \caption{Planform view of the tapered wing with  unswept leading edge}
    \label{fig:tapered-planforms}
\end{figure}

\begin{figure*}[!h]
    \centering
    % \begin{subfigure}[t]{\textwidth}
    %     \centering
        \includegraphics[width=\figwidth]{{figs/eps_fig/naca4415_coeffs_taper_all}.eps}
    %     \caption{Unswept leading edge\label{sfig:4415-ar12-letap}}
    % \end{subfigure}%

    % \begin{subfigure}[t]{\textwidth}
    %     \centering
    %     \includegraphics[width=\figwidth]{{figs/eps_fig/naca4415_taper0.5_qc-unswept_coeffs}.eps}
    %     \caption{Unswept Quarter-chord line\label{sfig:4415-ar12-qctap}}
    % \end{subfigure}%
    % \caption{Total lift, drag, and pitching moment vs. $\alpha$ for the tapered wings from CFD (black) and viscous LOM (blue)}
    \caption{Total lift, drag, and pitching moment vs. $\alpha$ for the tapered wing (Case D) from CFD (black) and viscous LOM (blue)}
    \label{fig:n4415-tap-coeff}
\end{figure*}

Viscous low-order results for the tapered wing were obtained using the same 2D viscous curves at each section, i.e. no Reynolds number adjustments were made. The results from the \methodname are shown in \fref{fig:n4415-tap-coeff}.
The viscous LOM accurately predicts the stall angle and the associated drop in lift and moment for the tapered wing.

From \fref{fig:n4415-ar12-le-unswept-cldist}, it can be seen that the viscous low-order method correctly predicts the spanwise distributions of lift and moment at the pre-stall angle of attack ($\alpha = 10\degree$). %
% The magnitudes of lift and nose-down pitching moment are slightly overpredicted throughout the span of the tapered wing.
As the angle of attack increases to stall and beyond ($ \alpha \geq 20\degree$), the drop in lift and moment is correctly modeled by the decambering approach implemented in the viscous low-order method. It is worth noting that the viscous LOM achieves this using viscous input curves obtained at a single Reynolds number ($Re = 3\times10^6$). The separation line is also accurately predicted by the low-order method, as shown in \fref{fig:naca4415_ar12_le-unswept_allalpha_fdist}.

% \PHNote{Spanwise distributions for tapered wings}

\begin{figure*}[!h]
    \centering
    \begin{subfigure}[t]{0.45\textwidth}
        \centering
        \includegraphics[width=0.9\textwidth]{{figs/eps_fig/spanwise-dist/naca4415_ar12_le-unswept_a10.0_cldist}.eps}
        \caption{$\alpha=10\degree$\label{sfig:4415-ar12-le-unswept-cldist-a10}}
    \end{subfigure}%
    ~
    \begin{subfigure}[t]{0.45\textwidth}
        \centering
        \includegraphics[width=0.9\textwidth]{{figs/eps_fig/spanwise-dist/naca4415_ar12_le-unswept_a10.0_cmdist}.eps}
        \caption{$\alpha=10\degree$\label{sfig:4415-ar12-le-unswept-cmdist-a10}}
    \end{subfigure}%

    \begin{subfigure}[t]{0.45\textwidth}
        \centering
        \includegraphics[width=0.9\textwidth]{{figs/eps_fig/spanwise-dist/naca4415_ar12_le-unswept_a22.0_cldist}.eps}
        \caption{$\alpha=20\degree$\label{sfig:4415-ar12-le-unswept-cldist-a22}}
    \end{subfigure}%
    ~
    \begin{subfigure}[t]{0.45\textwidth}
        \centering
        \includegraphics[width=0.9\textwidth]{{figs/eps_fig/spanwise-dist/naca4415_ar12_le-unswept_a22.0_cmdist}.eps}
        \caption{$\alpha=22\degree$\label{sfig:4415-ar12-le-unswept-cmdist-a22}}
    \end{subfigure}%

    \begin{subfigure}[t]{0.45\textwidth}
        \centering
        \includegraphics[width=0.9\textwidth]{{figs/eps_fig/spanwise-dist/naca4415_ar12_le-unswept_a28.0_cldist}.eps}
        \caption{$\alpha=28\degree$\label{sfig:4415-ar12-le-unswept-cldist-a20}}
    \end{subfigure}%
    ~
    \begin{subfigure}[t]{0.45\textwidth}
        \centering
        \includegraphics[width=0.9\textwidth]{{figs/eps_fig/spanwise-dist/naca4415_ar12_le-unswept_a28.0_cmdist}.eps}
        \caption{$\alpha=28\degree$\label{sfig:4415-ar12-le-unswept-cmdist-a28}}
    \end{subfigure}%

    \caption{Spanwise distributions of $C_l$ and $C_m$ at pre- and post-stall angles of attack from CFD (black), inviscid LOM (red), and viscous LOM (blue) for the tapered NACA4415 $\ar{12}$ wing (Case D)}
    \label{fig:n4415-ar12-le-unswept-cldist}
\end{figure*}

\begin{figure}[!h]
    \centering
    \includegraphics[width=3in]{figs/eps_fig/spanwise-dist/naca4415_ar12_le-unswept_allalpha_fdist.eps}
    \caption{Separation line predicted by the LOM (blue) and CFD (black) for angles of attack ranging from pre-stall to post-stall (Case D).}
    \label{fig:naca4415_ar12_le-unswept_allalpha_fdist}
\end{figure}

\subsection{Predictions for Wings Undergoing Rolling Motion (Case E)}
\label{sec:rolling-wings}
Although the underlying vortex lattice method used in this work is a steady code, it is capable of making predictions for quasi-steady flow states, such as for wings undergoing small, constant rates of rotation.
These quasi-steady conditions are typical of those experienced by general aviation and transport aircraft.
% In these quasi-steady states, flow features such as leading-edge vortex shedding which are usually observed in highly unsteady flows, are absent.
The examples presented in this section demonstrate the ability of the viscous LOM to predict the variation of total wing lift, drag, and moment coefficients, and their spanwise distributions, for two wings having a roll rate of $\SI{0.1}{\radian\per\second}$ or $\SI{5.73}{\deg\per\second}$ about the chordwise axis. This roll rate causes the left wing ($2y/b < 0$) to move downwards and see an increased effective angle of attack, while the right wing ($2y/b > 0$) moves upwards and experiences a reduced effective angle of attack. Two geometries, each of aspect ratio 12, are presented in this section: one rectangular and one tapered with a taper ratio $\lambda = 0.5$.
To verify the results from the low-order method, CFD solutions were obtained using ANSYS Fluent at select angles of attack before (0\degree, 10\degree), close to (15\degree, 18\degree), and after (20\degree) stall.

% \multiref{f}{fig:rot-ar8-coeffs}{fig:rot-ar12-coeffs}
\fref{fig:n4415-rot-coeffs}
shows the total wing $C_L$, $C_D$, and $C_M$ vs. $\alpha$ variation predicted by the \methodname. The lift and drag predictions from the viscous LOM agree well with CFD solutions. Predictions for pitching moment are seen to deviate from CFD solutions at higher angles of attack.
\multiref{f}{fig:n4415-ar12-dist-rot}{fig:n4415-tap-dist-rot} show the comparison of the spanwise distribution of $C_l$.
For reference, the spanwise distribution of $C_l$ for the wings without any rotational velocity from CFD is plotted using the dashed black line.
At a low angle of attack (0\degree) shown in Figures \ref{sfig:4415-ar12-cldist-a0-rot} and \ref{sfig:4415-ar12-le-uns-cldist-a0-rot}, the rolling motion of the wing causes an increase in lift on the left side (rolling downwards) and a drop in lift on the right side (rolling upwards).
It is this increase in lift on the descending wing that results in roll damping at unstalled conditions.
At $\alpha=20\degree$, this effect is reversed. The lift produced on the left side of the wing is reduced, whereas the right side of the wing produces more lift than the case without any rotation, as seen from Figures \ref{sfig:4415-ar12-cldist-a20-rot} and \ref{sfig:4415-ar12-le-uns-cldist-a20-rot}.
This post-stall behavior that results in loss of roll damping is captured correctly by the LOM.
The variation of total coefficient of rolling moment with angle of attack for the rectangular and tapered wings is shown in \fref{fig:n4415-croll}.
At low angles of attack, the wing experiences a negative rolling moment, i.e. in the direction opposite to the roll. The negative rolling moment indicates that roll damping is present. As the angle of attack increases, the restoring moment reduces, and after stall, the rolling moment acts in the direction of the rotation. The low-order method correctly predicts the loss of roll damping due to stall.

\begin{figure*}
    \centering
        \includegraphics[width=\figwidth]{{figs/eps_fig/naca4415_coeffs_roll_all}.eps}
    \caption{Total coefficients of lift, drag, and pitching moment for the NACA4415 $\ar{12}$ wings (Cases E\textsubscript{1}--E\textsubscript{2}) experiencing a $0.1\si{\radian\per\second}$ roll-rate}
    \label{fig:n4415-rot-coeffs}
\end{figure*}

% \begin{figure*}
%     \centering
%     \begin{subfigure}[t]{0.45\textwidth}
%         \centering
%         \includegraphics[width=0.9\textwidth]{{figs/eps_fig/spanwise-dist/naca4415_omega0.1_ar8_swp0_a0.0_cldist}.eps}
%         \caption{$\alpha=0\degree$\label{sfig:4415-ar8-cldist-a0-rot}}
%     \end{subfigure}%
%     ~
%     \begin{subfigure}[t]{0.45\textwidth}
%         \centering
%         \includegraphics[width=0.9\textwidth]{{figs/eps_fig/spanwise-dist/naca4415_omega0.1_ar8_swp0_a0.0_cmdist}.eps}
%         \caption{$\alpha=0\degree$\label{sfig:4415-ar8-cmdist-a0-rot}}
%     \end{subfigure}%

%     \begin{subfigure}[t]{0.45\textwidth}
%         \centering
%         \includegraphics[width=0.9\textwidth]{{figs/eps_fig/spanwise-dist/naca4415_omega0.1_ar8_swp0_a20.0_cldist}.eps}
%         \caption{$\alpha=20\degree$\label{sfig:4415-ar8-cldist-a20-rot}}
%     \end{subfigure}%
%     ~
%     \begin{subfigure}[t]{0.45\textwidth}
%         \centering
%         \includegraphics[width=0.9\textwidth]{{figs/eps_fig/spanwise-dist/naca4415_omega0.1_ar8_swp0_a20.0_cmdist}.eps}
%         \caption{$\alpha=20\degree$\label{sfig:4415-ar8-cmdist-a20-rot}}
%     \end{subfigure}%


%     \caption{Spanwise distributions of $C_l$ and $C_m$ at pre- and post-stall angles of attack from CFD (black), inviscid LOM (red), and viscous LOM (blue) for the NACA4415 rectangular $\ar{8}$ wing (Case E) with a $0.1\si{\radian\per\second}$ roll rate}
%     \label{fig:n4415-ar8-dist-rot}
% \end{figure*}


\begin{figure*}
    \centering
    \begin{subfigure}[t]{0.45\textwidth}
        \centering
        \includegraphics[width=0.9\textwidth]{{figs/eps_fig/spanwise-dist/naca4415_omega0.1_ar12_swp0_a0.0_cldist}.eps}
        \caption{$\alpha=0\degree$\label{sfig:4415-ar12-cldist-a0-rot}}
    \end{subfigure}%
    % ~
    % \begin{subfigure}[t]{0.45\textwidth}
    %     \centering
    %     \includegraphics[width=0.9\textwidth]{{figs/eps_fig/spanwise-dist/naca4415_omega0.1_ar12_swp0_a0.0_cmdist}.eps}
    %     \caption{$\alpha=0\degree$\label{sfig:4415-ar12-cmdist-a0-rot}}
    % \end{subfigure}%

    \begin{subfigure}[t]{0.45\textwidth}
        \centering
        \includegraphics[width=0.9\textwidth]{{figs/eps_fig/spanwise-dist/naca4415_omega0.1_ar12_swp0_a20.0_cldist}.eps}
        \caption{$\alpha=20\degree$\label{sfig:4415-ar12-cldist-a20-rot}}
    \end{subfigure}%
    % ~
    % \begin{subfigure}[t]{0.45\textwidth}
    %     \centering
    %     \includegraphics[width=0.9\textwidth]{{figs/eps_fig/spanwise-dist/naca4415_omega0.1_ar12_swp0_a20.0_cmdist}.eps}
    %     \caption{$\alpha=20\degree$\label{sfig:4415-ar12-cmdist-a20-rot}}
    % \end{subfigure}%


    \caption{Spanwise distributions of $C_l$ at pre- and post-stall angles of attack from CFD (black), inviscid LOM (red), and viscous LOM (blue) for the NACA4415 rectangular $\ar{12}$ wing (Case E\textsubscript{1}) with a $0.1\si{\radian\per\second}$ roll rate}
    \label{fig:n4415-ar12-dist-rot}
\end{figure*}




\begin{figure*}
    \centering
    \begin{subfigure}[t]{0.45\textwidth}
        \centering
        \includegraphics[width=0.9\textwidth]{{figs/eps_fig/spanwise-dist/naca4415_ar12_le-unswept_omega0.1_a0.0_cldist}.eps}
        \caption{$\alpha=0\degree$\label{sfig:4415-ar12-le-uns-cldist-a0-rot}}
    \end{subfigure}%
    % ~
    % \begin{subfigure}[t]{0.45\textwidth}
    %     \centering
    %     \includegraphics[width=0.9\textwidth]{{figs/eps_fig/spanwise-dist/naca4415_ar12_le-unswept_omega0.1_a0.0_cmdist}.eps}
    %     \caption{$\alpha=0\degree$\label{sfig:4415-ar12-le-uns-cmdist-a0-rot}}
    % \end{subfigure}%

    \begin{subfigure}[t]{0.45\textwidth}
        \centering
        \includegraphics[width=0.9\textwidth]{{figs/eps_fig/spanwise-dist/naca4415_ar12_le-unswept_omega0.1_a20.0_cldist}.eps}
        \caption{$\alpha=20\degree$\label{sfig:4415-ar12-le-uns-cldist-a20-rot}}
    \end{subfigure}%
    % ~
    % \begin{subfigure}[t]{0.45\textwidth}
    %     \centering
    %     \includegraphics[width=0.9\textwidth]{{figs/eps_fig/spanwise-dist/naca4415_ar12_le-unswept_omega0.1_a20.0_cmdist}.eps}
    %     \caption{$\alpha=20\degree$\label{sfig:4415-ar12-le-uns-cmdist-a20-rot}}
    % \end{subfigure}%


    \caption{Spanwise distributions of $C_l$ at pre- and post-stall angles of attack from CFD (black), inviscid LOM (red), and viscous LOM (blue) for the NACA4415 tapered wing (Case E\textsubscript{2}) with a $0.1\si{\radian\per\second}$ roll rate}
    \label{fig:n4415-tap-dist-rot}
\end{figure*}



\begin{figure*}
    \centering
        \includegraphics{{figs/eps_fig/naca4415_rollmoment_all}.eps}
    \caption{Total rolling moment vs. $\alpha$ on the rectangular and tapered $\ar{12}$ wings from CFD (symbols) and viscous LOM (lines)}
    \label{fig:n4415-croll}
\end{figure*}

% \PHNote{Hmm, no significant difference between rect and tapered from LOM an CFD}

% \clearpage
