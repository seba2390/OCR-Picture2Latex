\newpage
%%%%%%%%%%%%%%%%%%%%%%%%%%%%%%%%%%%%%%%%%%%%%%%%%%%%%
\section{Regular symmetry breaking operators
 $\Atbb \lambda \nu {\delta\varepsilon} {i,j}$ from $I_{\delta}(i,\lambda)$ to $J_{\varepsilon}(j,\nu)$}
\label{sec:holo}
%%%%%%%%%%%%%%%%%%%%%%%%%%%%%%%%%%%%%%%%%%%%%%%%%%%%%

In this chapter we apply the general results
 developed in Chapter \ref{sec:section7}
 on the analytic continuation
 of integral symmetry breaking operators
 $\Atbb \lambda \nu {\delta\varepsilon} {V,W} 
  \colon 
  I_{\delta}(V,\lambda) \to J_{\varepsilon}(W,\nu)$
 to the special setting where 
\begin{equation}
\label{eqn:VWexterior}
  (V,W)=(\Exterior^i({\mathbb{C}}^n), \Exterior^j({\mathbb{C}}^{n-1})), 
\end{equation}
 and construct a holomorphic family
 of (normalized) regular symmetry breaking operators
\[
  \Atbb \lambda \nu {\delta\varepsilon} {i,j}
  \colon 
   I_{\delta}(i,\lambda) \to J_{\varepsilon}(j,\nu), 
\]
 which exist if and only if $j=i-1$ or $i$
 (Theorems \ref{thm:152271} and \ref{thm:1522a}).  
Then the goal of this chapter 
 is to determine
\begin{enumerate}
\item[$\bullet$]
the parameter $(\lambda,\nu)$ 
for which $\Atbb \lambda \nu {\pm} {i,j}$ vanishes
 (Section \ref{subsec:Aijzero});
\item[$\bullet$]
the $(K,K')$-spectrum of $\Atbb \lambda \nu {\pm} {i,j}$
 (Sections \ref{subsec:Kspec}--\ref{subsec:Apm1});
\item[$\bullet$]
functional equations of $\Atbb \lambda \nu {\pm} {i,j}$
 (Sections \ref{subsec:FI}--\ref{subsec:161702}).  
\end{enumerate}
Thus we will complete the proof of Theorem \ref{thm:161243}
 that determines the zeros
 of the normalized operators
 $\Atbb \lambda \nu {\delta \varepsilon} {i,j}$.  
This is the last missing piece 
 in the classification scheme
 (Theorem \ref{thm:VWSBO}), 
 and thus we complete the proof of the classification
 of the space ${\operatorname{Hom}}_{G'}(I_{\delta}(i,\lambda)|_{G'}, J_{\varepsilon}(j,\nu))$ 
 of {\it{all}} symmetry breaking operators
 as stated in Theorems \ref{thm:1.1} and \ref{thm:SBObasis}.  



The $(K,K')$-spectrum resembles eigenvalues
 of a symmetry breaking operator
 (Definition \ref{def:KKspec}), 
 for which we find an integral expression 
 and determine the explicit formula
 for basic $K$- and $K'$-types
 (Theorem \ref{thm:153315}).  



The matrix-valued functional equations 
 among various intertwining operators are determined explicitly 
 in Theorems \ref{thm:TAA} and \ref{thm:ATA}
 by using the formula of the $(K,K')$-spectrum, 
 which in turn will play a crucial role in analyzing the behavior
 of the symmetry breaking operators
 at reducible places
 (Chapter \ref{sec:pfSBrho}).  



Degenerate cases where the normalized operators $\Atbb \lambda \nu {\pm}{i,j}$ 
 vanish will be discussed in Sections \ref{subsec:161702} and \ref{subsec:Rest}.


As an application of the matrix-valued functional 
equations (Theorems \ref{thm:TAA} and \ref{thm:ATA})
 and the residue formul{\ae}
 of $\Atbb \lambda \nu {\pm}{i,j}$
 (Fact \ref{fact:153316}), 
 we determine when the {\it{differential}}
 symmetry breaking operators
 $\Ctbb \lambda \nu{i,j}$
 ($j=i,i-1$)
 are surjective in Section \ref{subsec:ImageC}.  

%%%%%%%%%%%%%%%%%%%%%%%%%%%%%%%%%%%%%%%%%%%%%%%%%%%%%%%%%%%%%%%%%
\subsection{Regular symmetry breaking operators $\Atbb \lambda \nu {\pm}{i,j}$}
\label{subsec:Aij}
%%%%%%%%%%%%%%%%%%%%%%%%%%%%%%%%%%%%%%%%%%%%%%%%%%%%%%%%%%%%%%%%%%
In this section,
 we give the existence condition
 and an explicit construction
 of (generically) regular symmetry breaking operators from 
 $G$-modules $I_{\delta}(V,\lambda)$
 to $G'$-modules $J_{\varepsilon}(W,\nu)$
 in the setting \eqref{eqn:VWexterior}
 by applying the general results
 of Chapters \ref{sec:general} and \ref{sec:section7}, 
 in particular,  
 Theorems \ref{thm:regexist} and \ref{thm:152389}
 and their proofs.  

%%%%%%%%%%%%%%%%%%%%%%%%%%%%%%%%%%%%%%%%%%%%%%%%%%%%%
\subsubsection{Existence condition for regular symmetry breaking operators}
\label{subsec:reg}
%%%%%%%%%%%%%%%%%%%%%%%%%%%%%%%%%%%%%%%%%%%%%%%%%%%%%
We recall from Definition \ref{def:regSBO}
 the notion of {\it{regular}} symmetry breaking operators.~We also recall from \eqref{eqn:Ureg+} and \eqref{eqn:Ureg-}
 the definition of the open dense subsets
 $U_{\pm}^{\operatorname{reg}}$ in ${\mathbb{C}}^2$.  
Then the existence condition of regular symmetry breaking operators
 in the setting \eqref{eqn:VWexterior} is stated as follows.  


\begin{theorem}
\label{thm:152271}
Suppose $0 \le i \le n$
 and $0 \le j \le n-1$.  
Then the following three conditions on the pair $(i,j)$ are equivalent:
\begin{enumerate}
\item[{\rm{(i)}}]
 there exists a nonzero regular symmetry breaking operator from the $G$-module
 $I_{\delta}(i,\lambda)$ to the $G'$-module $J_{\varepsilon}(j,\nu)$
 for some $(\lambda,\nu,\delta,\varepsilon) \in {\mathbb{C}}^2 \times \{ \pm \}^2$;
\item[{\rm{(ii)}}]
 for any $(\delta,\varepsilon) \in \{ \pm \}^2$, 
 there exists a nonzero regular symmetry breaking operator from 
 $I_{\delta}(i,\lambda)$ to $J_{\varepsilon}(j,\nu)$
 for all $(\lambda,\nu) \in U_{\delta \varepsilon}^{{\operatorname{reg}}}$;
\item[{\rm{(iii)}}]
 $j=i$ or $i-1$.  
\end{enumerate}
\end{theorem}

\begin{proof}
As we have seen in the decomposition \eqref{eqn:extij},  
 $[V:W] \ne 0$ in the setting \eqref{eqn:VWexterior}
 if and only if $j=i-1$ or $i$.  
Then Theorem \ref{thm:152271} follows from Theorem \ref{thm:regexist}
 and Proposition \ref{prop:172425}.  
\end{proof}

\subsubsection{Construction of $\Atbb \lambda \nu {\pm}{i,j}$
 for $j \in \{i-1,i\}$}

In this section 
 we apply Theorem \ref{thm:152389}
 about the construction
 of the (generically) regular symmetry breaking operators
 $\Atbb \lambda \nu {\pm}{V,W}$
 in the setting \eqref{eqn:VWexterior}
 with $j=i-1$ or $i$.  
In particular,
 we give concrete formul{\ae}
 of the matrix-valued distribution kernels
 $\Atcal \lambda \nu {\pm}{i,j}$
 for the operators.  



Let $j=i-1$ or $i$.  
We recall from \eqref{eqn:Tii1} and \eqref{eqn:Tii2}
 that the projection $\pr i j \colon \Exterior^i({\mathbb{C}}^n) \to \Exterior^j({\mathbb{C}}^{n-1})$
 defines an element of 
\[
   {\operatorname{Hom}}_{O(n-1)}(V,W)
   =
   {\operatorname{Hom}}_{O(n-1)}({\Exterior}^i({\mathbb{C}}^n),{\Exterior}^j({\mathbb{C}}^{n-1})).  
\]
Denote by $\sigma \equiv \sigma^{(i)}$ 
 the $i$-th exterior representation of $O(n)$ on ${\Exterior}^i({\mathbb{C}}^n)$.  
Then the matrix-valued function $\Rij V W$ 
(see \eqref{eqn:RVW})
 amounts to the following map  
\[
\Rij i j : {\mathbb{R}}^n \setminus \{ 0 \} 
            \to 
           \operatorname{Hom}_{\mathbb{C}}({\Exterior}^i({\mathbb{C}}^n),
                            {\Exterior}^j({\mathbb{C}}^{n-1}) )
\]
given by 
\index{A}{Rij@$\Rij ij$|textbf}
\index{A}{prij@$\pr i j$, projection}
\index{A}{1psin@$\psi_n$}
\begin{equation}
\label{eqn:Rij}
   \Rij ij := \pr i j  \circ \sigma \circ \psi_n
\end{equation}
where we recall from \eqref{eqn:psim}
 that $\psi_n \colon {\mathbb{R}}^n \setminus\{0\} \to O(n)$
 is the map of taking \lq\lq{reflection}\rq\rq.  

\medskip
Applying the general formul{\ae} \eqref{eqn:KVWpt} and \eqref{eqn:KVWmt}
 of the distribution kernels 
 $\Atcal \lambda \nu {\pm}{V,W}$
 in the setting \eqref{eqn:VWexterior}, 
 we obtain 
$
    {\operatorname{Hom}}_{\mathbb{C}}
    (\Exterior^i({\mathbb{C}}^n),\Exterior^j({\mathbb{C}}^{n-1}))
$-valued locally integrable functions
 on ${\mathbb{R}}^n$
 for $\operatorname{Re}\lambda \gg |\operatorname{Re} \nu|$
 as follows.  
\begin{align}
\Atcal {\lambda}{\nu}{+}{i,j}
 :=&
\frac{1}{\Gamma(\frac{\lambda + \nu -n +1}{2})
         \Gamma(\frac{\lambda-\nu}{2})}
(|x|^2+ x_n^2)^{-\nu}
|x_n|^{\lambda+\nu-n}
\Rij ij (x,x_n), 
\label{eqn:Kijtilde}
\\
\Atcal {\lambda}{\nu}{-}{i,j}
 :=&
\frac{1}{\Gamma(\frac{\lambda + \nu -n +2}{2})
         \Gamma(\frac{\lambda-\nu+1}{2})}
(|x|^2 +x_n^2)^{-\nu}
|x_n|^{\lambda+\nu-n}
 {\operatorname{sgn}} x_n
\Rij i j(x,x_n).  
\label{eqn:Kijpmtilde}
\end{align}
Then, 
 as a special case
 of Theorem \ref{thm:152389}, 
 we obtain:
\begin{theorem}
[holomorphic continuation of integral operators]
\label{thm:1522a}
Let $(V,W)$ be as in \eqref{eqn:VWexterior}
 with $j=i,i-1$, 
 and $\delta, \varepsilon \in \{\pm \}$.  
Then the distributions $\Atcal \lambda \nu {\delta\varepsilon} {i,j}$, 
 initially defined as ${\operatorname{Hom}}_{\mathbb{C}}(V,W)$-valued 
 locally integrable functions
 on ${\mathbb{R}}^n$ 
 for $\operatorname{Re}\lambda \gg |\operatorname{Re} \nu|$, 
 extends to 
$
   ({\mathcal{D}}'(G/P, {\mathcal{V}}_{\lambda,\delta}^{\ast}) \otimes W_{\nu,\varepsilon})^{\Delta(P')}
$
 that depends holomorphically on $(\lambda, \nu)$
 in ${\mathbb{C}}^2$.  
Then the matrix-valued distribution kernels
\index{A}{Acts0@$\Atcal \lambda \nu {\pm} {i,j}$}
$
\Atcal \lambda \nu {\delta\varepsilon} {i,j}
$
 induce a family of symmetry breaking operators
\index{A}{Ahtsln0@$\Atbb \lambda \nu {\pm} {i,j}$|textbf}
\begin{equation}
\label{eqn:Aijdef}
  \Atbb \lambda \nu {\delta\varepsilon} {i,j}
  \colon
  I_{\delta}(i,\lambda) \to J_{\varepsilon}(j,\nu)
\end{equation}
 for all $(\lambda, \nu) \in {\mathbb{C}}^2$.  
\end{theorem}

Then $\Atbb \lambda \nu {\pm} {i,j}$ is the normalized (generically) regular 
 symmetry breaking operator (Definition \ref{def:AregSBO})
 in the sense that there exists an open dense subset
 $U_{\gamma}$ in ${\mathbb{C}}^2$
 for $\gamma \in \{\pm\}$
 such that the support of the distribution kernel 
 of $\Atbb \lambda \nu \gamma {i,j}$ equals 
 the whole flag manifold $G/P$
 as far as $(\lambda,\nu) \in U_{\gamma}$, 
 see Proposition \ref{prop:172425}.  
By a little abuse of terminology,
 we say that $\{\Atbb \lambda \nu \pm {i,j}\}$
 is a family of 
\index {B}{regularsymmetrybreakingoperator@regular symmetry breaking operator}
 {\it{normalized regular symmetry breaking operators}}.  



%%%%%%%%%%%%%%%%%%%%%%%%%%%%%%%%%%%%%%%%%%%%%%%%%%%%%%%%%%%%%%%%%%%%%%%%%%%
\subsection{Zeros of $\Atbb \lambda \nu {\pm} {i,j}$ : Proof of Theorem \ref{thm:161243}}
\label{subsec:Aijzero}
%%%%%%%%%%%%%%%%%%%%%%%%%%%%%%%%%%%%%%%%%%%%%%%%%%%%%%%%%%%%%%%%%%%%%%%%%%%
In this section 
 we determine the exact place of the zeros
 of the normalized regular symmetry breaking operators
 $\Atbb \lambda \nu {\delta\varepsilon} {i,j}$, 
 and thus give a proof 
 of Theorem \ref{thm:161243}.  
In particular,
 we see that the Gamma factors
 in the normalization \eqref{eqn:Kijtilde} and \eqref{eqn:Kijpmtilde}
 are optimal 
 in the sense
 that the zeros of $\Atbb \lambda \nu {\gamma} {i,j}$
 are of codimension two in ${\mathbb{C}}^2$, 
 namely,
 form a discrete subset of ${\mathbb{C}}^2$. 
The proof of Theorem \ref{thm:161243}
 consists of the following steps.  
\par\noindent
{\bf{Step 0.}}\enspace
(existence condition)\enspace
Regular symmetry breaking operators from 
 $I_{\delta}(i,\lambda)$ to $J_{\varepsilon}(j,\nu)$ exist
 if and only if $j \in \{i-1,i\}$
 (Theorem \ref{thm:152271}).  
\par\noindent
{\bf{Step 1.}}\enspace
(generically nonzero)\enspace
If $\Atbb \lambda \nu {\delta\varepsilon} {i,j}=0$, 
 then $(\lambda, \nu, \delta, \varepsilon)$ belongs to the set 
 $\Psising$ 
 of special parameters
 (Theorem \ref{thm:genbasis}).  
\par\noindent
{\bf{Step 2.}}\enspace
(residue formula)\enspace
If $(\lambda, \nu, \delta, \varepsilon)\in \Psising$, 
 then $\Atbb \lambda \nu {\delta\varepsilon} {i,j}$ is proportional
 to the {\it{differential}} symmetry breaking operator
 $\Cbb \lambda \nu {i,j}$ 
 with explicit proportional constant
 (Fact \ref{fact:153316}).  

\subsubsection{Residue formula of the regular symmetry breaking operator
 $\Atbb \lambda \nu {\pm}{i,j}$}

Generalizing the residue formula
 of the scalar-valued regular symmetry breaking operators
 $\Atbb \lambda \nu {+}{0,0}$
 for spherical principal series representations
 given in \cite{xkEastwood} (see also \cite[Thm.~12.2]{sbon}), 
 we determined the residue
 of the matrix-valued regular symmetry breaking operators
 $\Atbb \lambda \nu {\pm}{i,j}$
 in \cite{xkresidue}, 
 as follows:
\begin{fact}
[residue formula {\cite[Thm.~1.3]{xkresidue}}]
\label{fact:153316}
\index{B}{residueformula@residue formula|textbf}
Let $\Cbb \lambda \nu {i,j}$ be the differential symmetry breaking operators
 defined in \eqref{eqn:Ciiln}
 and \eqref{eqn:Cijln} for $j=i-1$ or $i$.  
\begin{enumerate}
\item[{\rm{(1)}}]
Suppose $\nu-\lambda=2\ell$ with $\ell \in {\mathbb{N}}$.  
Then, 
\index{A}{Ciiln@$\Cbb \lambda \nu {i,j}$, matrix-valued differential operator}
\begin{equation}
\label{eqn:Aijres1}
\Atbb \lambda \nu {+} {i,j} 
=
\frac{(-1)^{i-j+\ell} \pi^{\frac{n-1}{2}} \ell !}{2^{2\ell-1}\Gamma(\nu+1)}
\Cbb{\lambda}{\nu}{i,j}.  
\end{equation}
\item[{\rm{(2)}}]
Suppose $\nu-\lambda=2\ell+1$ with $\ell \in {\mathbb{N}}$.  
Then, 
\[
\Atbb \lambda \nu {-} {i,j} 
=
\frac{(-1)^{i-j+\ell+1} \pi^{\frac{n-1}{2}} \ell !}{2^{2\ell+2}\Gamma(\nu+1)}
\Cbb{\lambda}{\nu}{i,j}.  
\]
\end{enumerate}
\end{fact}



We may unify the two formul\ae\
 in Fact \ref{fact:153316}
 into one formula:
for $\nu -\lambda \in {\mathbb{N}}$ and $j \in \{i,i-1\}$, 
\begin{equation}
\label{eqn:qAC}
\Atbb \lambda \nu {(-1)^{\nu-\lambda}} {i,j}
=
\frac{2(-1)^{i-j} \pi^{\frac {n-1}2}}{q(\nu-\lambda)\Gamma(\nu+1)}
\Cbb \lambda \nu {i,j}, 
\end{equation}
where we set,
 for $m \in {\mathbb{N}}$, 
\index{A}{qm@$q(m)$|textbf}
\begin{equation}
\label{eqn:qm}
q(m):=
\begin{cases}
\dfrac{(-1)^{\ell} 2^{2\ell}}{\ell!}
\qquad
&\text{if $m=2\ell$}, 
\\
&
\\
\dfrac{(-1)^{\ell+1} 2^{2\ell+3}}{\ell!}
\qquad
&\text{if $m=2\ell+1$}.  
\end{cases}
\end{equation}


\subsubsection{Zeros of $\Atbb \lambda \nu {\pm} {i,j}$}

The zeros of the operators $\Atbb \lambda \nu {\delta\varepsilon} {i,i}$
 for the special parameter
 in 
\index{A}{1psi@$\Psising$,
          special parameter in ${\mathbb{C}}^2 \times \{\pm\}^2$}
 $\Psising$
 (see \eqref{eqn:singset} for the definition)
 were determined in \cite{xkresidue}
 as a corollary of the residue formula
 (Fact \ref{fact:153316}), 
 which we recall now.  
\begin{corollary}
[zeros of $\Atbb \lambda \nu {\pm} {i,i}$
 for $\Psising$, 
{\cite[Thm.~8.1]{xkresidue}}]
\label{cor:Avanish}
\begin{enumerate}
\item[{\rm{(1)}}]
Suppose $\nu - \lambda \in 2 {\mathbb{N}}$.  
\newline
$\Atbb \lambda \nu {+} {i,i}=0$
if and only if
\begin{equation*}
(\lambda, \nu)\in 
\begin{cases}
L_{\operatorname{even}}
&\text{for $i=0$}, 
\\
(L_{\operatorname{even}} \setminus \{\nu =0\})\cup \{(i,i)\}
\quad
&\text{for $1 \le i \le n-1$}.   
\end{cases}
\end{equation*}
$\Atbb \lambda \nu {+} {i,i-1}=0$
if and only if
\begin{equation*}
(\lambda, \nu)\in 
\begin{cases}
(L_{\operatorname{even}} \setminus \{\nu =0\})\cup \{(n-i,n-i)\}
\quad
&\text{for $1 \le i \le n-1$}, 
\\
L_{\operatorname{even}}
&\text{for $i=n$}.   
\end{cases}
\end{equation*}
\item[{\rm{(2)}}]
Suppose $\nu - \lambda \in 2 {\mathbb{N}}+1$.  
\newline
$\Atbb \lambda \nu {-} {i,i}=0$
if and only if
\begin{equation*}
(\lambda, \nu)\in 
\begin{cases}
L_{\operatorname{odd}}
&\text{for $i=0$}, 
\\
L_{\operatorname{odd}} \setminus \{\nu =0\}
\quad
&\text{for $1 \le i \le n-1$}.   
\end{cases}
\end{equation*}
$\Atbb \lambda \nu {-} {i,i-1}=0$
if and only if
\begin{equation*}
(\lambda, \nu)\in 
\begin{cases}
L_{\operatorname{odd}} \setminus \{\nu =0\}
\quad
&\text{for $1 \le i \le n-1$}, 
\\
L_{\operatorname{odd}}
&\text{for $i=n$}.   
\end{cases}
\end{equation*}
\end{enumerate}
\end{corollary}
We are ready to complete the proof
 of Theorem \ref{thm:161243}
 on the zeros of the analytic continuation
 $\Atbb \lambda \nu \gamma {i,j}$
 of regular symmetry breaking operators.  
\begin{proof}
[Proof of Theorem \ref{thm:161243}]
We apply Theorem \ref{thm:genbasis}
 to the exterior representations
 \eqref{eqn:VWexterior},
 and see that  
$\Atbb \lambda \nu \gamma {i,j}=0$
 only if 
\begin{equation}
\label{eqn:AC}
\nu-\lambda \in 2 {\mathbb{N}}
\quad
(\gamma=+)
\quad
\text{or}
\quad
\nu-\lambda \in 2 {\mathbb{N}}+1
\quad
(\gamma=-).  
\end{equation}
Then Theorem \ref{thm:161243} follows from
 Corollary \ref{cor:Avanish}.  
\end{proof}


\subsection{$(K,K')$-spectrum for symmetry breaking operators}
\label{subsec:Kspec}

The second goal of this chapter is to formulate the concept
 of the $(K,K')$-spectrum
 for symmetry breaking operators (Definition \ref{def:KKspec}), 
 and give an explicit formula
 of the $(K,K')$-spectrum
\index{A}{SAln3ij@$S(\Atbb \lambda \nu {\varepsilon} {i,j})$, $(K,K')$-spectrum}\begin{equation}
\label{eqn:SAabcd}
S(\Atbb \lambda \nu \varepsilon {i,j})
=
\begin{pmatrix}
a_{\varepsilon}^{i,j}(\lambda,\nu)
&
b_{\varepsilon}^{i,j}(\lambda,\nu)
\\
c_{\varepsilon}^{i,j}(\lambda,\nu)
&
d_{\varepsilon}^{i,j}(\lambda,\nu)
\end{pmatrix}, 
\end{equation}
 (see \eqref{eqn:Smat}), 
 for the regular symmetry breaking operator
$
    \Atbb \lambda \nu \varepsilon{i,j}
    :
    I_{\delta}(i,\lambda)
    \to 
    J_{\delta \varepsilon}(j,\nu)
$
 with respect to basic $K$-types $\mu^{\natural}(i,\delta)$
 and $K'$-types $\mu^{\natural}(j,\delta \varepsilon)'$
 (see \eqref{eqn:muflat} and \eqref{eqn:musharp})
 for $\natural=\flat$ or $\sharp$.  
We will discuss the $(K,K')$-spectrum
 in Sections \ref{subsec:Kspec}--\ref{subsec:Apm1}.  
The main results are Theorem \ref{thm:153315}
 which will be proved in  Proposition \ref{prop:CApm1} (vanishing results)
 and Theorems \ref{thm:minKscalar}
 and \ref{thm:CApm1}.  




One of the algebraic clues
 that we introduced in the study 
 of symmetry breaking operators $A$ in \cite{sbon}
 was an explicit formula
 of the \lq\lq{eigenvalues}\rq\rq\
 of $A$
 on spherical vectors.  
In the setting of this article,
 there is no spherical vector
 in the principal series representation
 $I_{\delta}(i,\lambda)$ if $i>0$
 or $J_{\varepsilon}(j,\nu)$ if $j>0$.  
In this section,
 we extend the idea of \cite{sbon}
 to the 
\index{B}{KspectrumKprime@$(K,K')$-spectrum|textbf}
$(K,K')$-{\it{spectrum}}
 for symmetry breaking operators
 with focus on 
\index{B}{basicKtype@basic $K$-type}
basic $K$-types.  


\subsubsection{Generalities:
 $(K,K')$-spectrum of symmetry breaking operators}
We begin with a general setup.  
Let $(G,G')$ be a pair of real reductive Lie groups.  
Suppose $\Pi$ is a continuous representation of $G$, 
 and $\pi$ is that of the subgroup $G'$.  
We define a subset of $\widehat K \times \widehat{K'}$
 by 
\[
   {\mathcal{D}}(\Pi,\pi):=
   \{(\mu, \mu') \in \widehat K \times \widehat{K'}:
     [\Pi|_K:\mu], 
 [\pi|_{K'}:\mu'], 
 [\mu|_{K'}:\mu']
 \in \{0,1\}
\}.  
\]

Here is a sufficient condition for ${\mathcal{D}}(\Pi,\pi)$ 
 to be nonempty:
\begin{proposition}
\label{prop:170715}
Let $P=L N$
 and $P'=L' N'$ be parabolic subgroups of $G$ and its subgroup $G'$, 
 respectively.  
Suppose that $\Pi={\operatorname{Ind}}_P^G(\sigma \otimes {\mathbb{C}}_{\lambda})$
 and $\pi={\operatorname{Ind}}_{P'}^{G'}(\tau \otimes {\mathbb{C}}_{\nu})$
 are the induced representations from irreducible
 finite-dimensional representations
 $\sigma \otimes {\mathbb{C}}_{\lambda}$ of $L \simeq P/N$
 and $\tau \otimes {\mathbb{C}}_{\nu}$ of $L' \simeq P'/N'$, 
 respectively.  
\begin{enumerate}
\item[{\rm{(1)}}]
{\rm{(spherical principal series)}}\enspace
If $\sigma$ and $\tau$ are the trivial one-dimensional representations, 
then ${\mathcal{D}}(\Pi,\pi) \ni ({\bf{1}}_K, {\bf{1}}_{K'})$.  
\item[{\rm{(2)}}]
If $(K, L \cap K)$, $(K', L' \cap K')$ and $(K, K')$ are strong Gel'fand pairs, 
 in particular,
 if they are symmetric pairs, 
 then ${\mathcal{D}}(\Pi,\pi) = \widehat K \times \widehat {K'}$.  
\end{enumerate}
\end{proposition}
\begin{proof}
\begin{enumerate}
\item[(1)]
Clear from the Frobenius reciprocity.  
\item[(2)]
Immediate from the multiplicity-free property for strong Gel'fand pairs.  
\end{enumerate}
\end{proof}
The following is an example of Proposition \ref{prop:170715} (2).  
\begin{example}
\label{ex:5.2}
Let $(G,G')=(O(n+1,1),O(n,1))$, 
 and we consider $\Pi= I_{\delta}(V,\lambda)$, 
$\pi=J_{\varepsilon}(W, \nu)$
 for any $(\sigma, V) \in \widehat {O(n)}$
 and any $(\tau, W) \in \widehat {O(n-1)}$.  
Then ${\mathcal{D}}(\Pi,\pi)= \widehat K \times \widehat{K'}$.  
\end{example}

Now we introduce a $(K,K')$-spectrum
 for symmetry breaking operators as follows.  

\begin{definition}
[$(K,K')$-spectrum]
\label{def:KKspec}
Let $(\mu,\mu') \in {\mathcal{D}}(\Pi,\pi)$.  
If 
$[\Pi|_K:\mu]
=[\pi|_{K'}:\mu']
=[\mu|_{K'}:\mu']
=1$, 
 then we fix a nonzero $K$-homomorphism
 $\varphi:\mu \hookrightarrow \Pi$
 and nonzero $K'$-homomorphisms
 $\varphi':\mu' \hookrightarrow \pi$
 and $\iota:\mu' \hookrightarrow \mu$
 that are unique up to scalar multiplication.  
Suppose $A \in {\operatorname{Hom}}_{G'}(\Pi|_{G'}, \pi)$.  
Then by Schur's lemma,
 there exists a constant 
 $S_{\mu,\mu'}(A) \in {\mathbb{C}}$
 such that 
\begin{equation}
\label{eqn:Smumu}
A \circ \varphi \circ \iota
=
S_{\mu,\mu'}(A)\circ \varphi'
\qquad
\text{on $\mu'$}.  
\end{equation}
If one of $[\Pi|_K:\mu]$, 
 $[\pi|_{K'}:\mu']$,
 or $[\mu|_{K'}:\mu']$ is $0$, 
 then we just set
\[
   S_{\mu,\mu'}(A)=0
   \quad
   \text{for any }
   A \in   \operatorname{Hom}_{G'}(\Pi|_{G'}, \pi).  
\]
Thus we have defined a map
\begin{equation}
\label{eqn:KKspec}
 S \colon 
 \operatorname{Hom}_{G'}(\Pi|_{G'}, \pi)
 \times 
 {\mathcal{D}}(\Pi, \pi)
 \to {\mathbb{C}}, 
\qquad
(A, (\mu,\mu')) \mapsto S_{\mu,\mu'}(A).  
\end{equation}
We say $S_{\mu,\mu'}(A)$ is the {\it{$(K,K')$-spectrum}}
 of the symmetry breaking operator $A$
 for $(\mu,\mu') \in \widehat K \times \widehat {K'}$.  
We note that it is independent
 of the choice of the normalizations of $\varphi$, $\varphi'$, and $\iota$
 whether $S_{\mu,\mu'}(A)$ vanishes or not.  
\end{definition}

\subsection{Explicit formula of $(K,K')$-spectrum on basic $K$-types
 for regular symmetry breaking operators
 $\Atbb \lambda \nu {\pm}{i,j}$}
We return to our setting
 where $(G,G')=(O(n+1,1),O(n,1))$, 
 and thus
\[
K=O(n+1) \times O(1)
\supset
K'=O(n) \times O(1).  
\]
We consider a pair of representations
  $\Pi = I_{\delta}(i,\lambda)$
 of $G=O(n+1,1)$
 and $\pi=J_{\varepsilon}(j,\nu)$ of the subgroup $G'=O(n,1)$. 
In this case 
 ${\mathcal{D}}(\Pi,\pi)=\widehat K \times \widehat{K'}$
 as we saw in Example \ref{ex:5.2}, 
 however, 
 the following finite subset 
\[
   {\mathcal{D}}^{\flat, \sharp}
   \equiv
   {\mathcal{D}}^{\flat, \sharp}(\Pi,\pi)
   :=
   \{\mub (i,\delta), \mus (i,\delta)\}
   \times
   \{\mub (j,\varepsilon)', \mus (j,\varepsilon)'\}
   \subset \widehat K \times \widehat {K'}
\]
 will be sufficient for the later analysis of symmetry breaking operators.  
Here we recall from \eqref{eqn:muflat} and \eqref{eqn:musharp}
 that 
\index{A}{0muflat@$\mub(i,\delta)$}
$\mub (i,\delta)$
 and 
\index{A}{0musharp@$\mus(i,\delta)$}
$\mus (i,\delta)$ are 
 \lq\lq{basic $K$-types}\rq\rq\ of the principal series representation
 $I_{\delta}(i,\lambda)$ of $G$
 and that $\mub (j,\varepsilon)'$ and $\mus(j,\varepsilon)'$
 are those for $J_{\varepsilon}(j,\nu)$
 of the subgroup $G'$.  



Then the $(K,K')$-spectrum restricted
 to the subset ${\mathcal{D}}^{\flat,\sharp}$
 is described as a $2 \times 2$ matrix: 
\begin{equation}
\label{eqn:Smat}
S:\operatorname{Hom}_{G'}(I_{\delta}(i,\lambda)|_{G'},J_{\varepsilon}(j,\nu))
  \to 
  M(2,{\mathbb{C}}),
  \qquad
  A 
  \mapsto
  \begin{pmatrix} a & b \\ c & d \end{pmatrix}
\end{equation}
 by taking $a$, $b$, $c$, $d$
 to be $S_{\mu,\mu'}(A)$ as follows:

\begin{equation*}
\begin{tabular}{c|rl|rl}
{$S_{\mu,\mu'}(A)$}
&
&{$\mu$}
&
&{$\mu'$}
\\
\hline
  $a$ 
& $\mub(i,\delta)$
& $=\Exterior^i({\mathbb{C}}^{n+1}) \boxtimes \delta$ 
& $\mub(j,\varepsilon)'$
& $=\Exterior^j({\mathbb{C}}^{n}) \boxtimes \varepsilon$
\\%[-3pt]
$b$ 
& $\mub(i,\delta)$
& $=\Exterior^i({\mathbb{C}}^{n+1}) \boxtimes \delta$ 
& $\mus(j,\varepsilon)'$
& $=\Exterior^{j+1}({\mathbb{C}}^{n}) \boxtimes (-\varepsilon)$
\\%[-3pt]
$c$ 
& $\mus(i,\delta)$
& $=\Exterior^{i+1}({\mathbb{C}}^{n+1}) \boxtimes (-\delta)$
& $\mub(j,\varepsilon)'$
&$=\Exterior^{j}({\mathbb{C}}^{n}) \boxtimes \varepsilon$
\\%[-3pt]
$d$ 
& $\mus(i,\delta)$
& $=\Exterior^{i+1}({\mathbb{C}}^{n+1}) \boxtimes (-\delta)$
\qquad
& $\mus(j,\varepsilon)'$
&=$\Exterior^{j+1}({\mathbb{C}}^{n}) \boxtimes (-\varepsilon)$
\end{tabular}
\end{equation*}
To be more precise,
 we need a normalization
of the map $\varphi$, $\varphi'$
 and $\iota$
 in Definition \ref{def:KKspec} in this setting.  
For this,
 we realize the $K$-types
 $\mub(i,\delta)=\Exterior^i({\mathbb{C}}^{n+1}) \boxtimes \delta$
 and $\mus(i,\delta)=\Exterior^{i+1}({\mathbb{C}}^{n+1}) \boxtimes (-\delta)$
 in $I_{\delta}(i,\lambda)$
 as in Proposition \ref{prop:minKN}.  
Similarly,
 $\mub(j,\varepsilon)'=\Exterior^j({\mathbb{C}}^{n}) \boxtimes \varepsilon$
 and 
 $\mus(j,\varepsilon)'=\Exterior^{j+1}({\mathbb{C}}^{n}) \boxtimes (-\varepsilon)$ 
 are realized in $J_{\varepsilon}(j,\nu)$.  
When $\mu'$ and $\mu$ are representations
 on the exterior tensor spaces $\Exterior^l({\mathbb{C}}^{n})$
 and $\Exterior^k({\mathbb{C}}^{n+1})$
 ($l=k$ or $k-1$) respectively,
 we normalize an $O(n)$-homomorphism 
\[
   \iota_{l \to k}\colon \Exterior^l({\mathbb{C}}^n) 
         \hookrightarrow \Exterior^k({\mathbb{C}}^{n+1})
\]
 such that $\pr k l \circ \iota_{l \to k}= {\operatorname{id}}$, 
 where the projection 
\index{A}{prij@$\pr i j$, projection}
$\pr k l : \Exterior^k({\mathbb{C}}^{n+1}) \to \Exterior^l({\mathbb{C}}^{n})$
 is defined in \eqref{eqn:Tii1} and \eqref{eqn:Tii2}.  
With these normalizations,
 the map \eqref{eqn:Smat} is defined.  
We obtain the following closed formula
 of the $(K,K')$-spectrum for the normalized regular 
 symmetry breaking operators 
\index{A}{Ahtsln0@$\Atbb \lambda \nu {\pm} {i,j}$}
$\Atbb \lambda \nu {\pm}{i,j} \colon I_{\delta}(i,\lambda) \to J_{\pm \delta}(j,\nu)$.  
\begin{theorem}
[$(K,K')$-spectrum for $\Atbb \lambda \nu {\pm} {i,j}$]
\label{thm:153315}
Suppose $(\lambda, \nu)\in {\mathbb{C}}^2$.  
Then the $(K,K')$-spectrum of the analytic continuation
 $\Atbb \lambda \nu {\pm}{i,j}$
 of regular symmetry breaking operators
 takes the following form on basic $K$-types:
\index{A}{SAln3ij@$S(\Atbb \lambda \nu {\varepsilon} {i,j})$, $(K,K')$-spectrum}

\begin{alignat*}{2}
S(\Atbb \lambda \nu {+} {i,i})
=&
\frac{\pi^{\frac{n-1}{2}}}{\Gamma(\lambda+1)}
\begin{pmatrix} \lambda-i & 0 \\ 0 & \nu-i \end{pmatrix}
\quad
&&
\text{for $0 \le i \le n-1$};
\\
S(\Atbb \lambda \nu {-} {i,i})
=&
\frac{\pi^{\frac{n-1}{2}}}{\Gamma(\lambda+1)}
\begin{pmatrix}0 & 0 \\ 2(-1)^{i+1} & 0 \end{pmatrix}
\quad
&&
\text{for $0 \le i \le n-1$};
\\
S(\Atbb \lambda \nu {+} {i,i-1})
=&
\frac{\pi^{\frac{n-1}{2}}}{\Gamma(\lambda+1)}
\begin{pmatrix} n-\nu-i & 0 \\ 0 & \lambda-n+i \end{pmatrix}
\quad
&&
\text{for $1 \le i \le n$};
\\
S(\Atbb \lambda \nu {-} {i,i-1})
=&
\frac{\pi^{\frac{n-1}{2}}}{\Gamma(\lambda+1)}
\begin{pmatrix} 0 & -2 \\ 0 & 0 \end{pmatrix}
\quad
&&
\text{for $1 \le i \le n$}.
\end{alignat*}
\end{theorem}
The vanishing result (an easy part) of Theorem \ref{thm:153315}
 will be shown in Proposition \ref{prop:CApm1}, 
 and the remaining nontrivial part will be proved in Theorems \ref{thm:minKscalar} and \ref{thm:CApm1}.  



\subsection{Proof of vanishing results on $(K,K')$-spectrum}
In this section,
 we formulate and prove vanishing results
 for $(K,K')$-spectrum
 that hold for general symmetry breaking operators.  
\begin{proposition}
\label{prop:CApm1}
Suppose $j \in \{i-1,i\}$, 
$\delta, \varepsilon \in \{\pm\}$, 
 and $\lambda, \nu \in {\mathbb{C}}$.  
Let $A:I_{\delta}(i,\lambda) \to J_{\varepsilon}(j,\nu)$
 be an arbitrary symmetry breaking operator.  
Then the $(K,K')$-spectrum $S(A)$
 for basic $K$-types takes the following form:
\begin{equation*}
\begin{tabular}{c|cccc}
$j$
&$i$
&$i$
&$i-1$
&$i-1$
\\
$\delta\varepsilon$
&$+$
&$-$
&$+$
&$-$
\\
\hline
$S(A)$ 
& $\begin{pmatrix} \ast & 0 \\ 0 & \ast \end{pmatrix}$
& $\begin{pmatrix} 0    & 0 \\ \ast & 0 \end{pmatrix}$
& $\begin{pmatrix} \ast & 0 \\ 0 & \ast \end{pmatrix}$
& $\begin{pmatrix} 0 & \ast \\ 0 & 0 \end{pmatrix}$
\end{tabular}
\end{equation*}
\end{proposition}


\begin{proof}
Without loss of generality,
 we may assume $\delta=+$.  
The $K$-modules $\mub(i,+)$ and $\mus(i,+)$
 (see \eqref{eqn:muflat} and \eqref{eqn:musharp})
 decompose into the sum of irreducible representations
 of the subgroup $K'$:
\begin{alignat*}{4}
 \mub(i,+) &= \Exterior^{i}({\mathbb{C}}^{n+1}) \boxtimes {\bf{1}}
&&\simeq
&& \Exterior^{i}({\mathbb{C}}^{n}) \boxtimes {\bf{1}}
&&    \oplus 
 \Exterior^{i-1}({\mathbb{C}}^{n}) \boxtimes {\bf{1}}, 
\\
 \mus(i,+) &= \Exterior^{i+1}({\mathbb{C}}^{n+1}) \boxtimes {\operatorname{sgn}}
&&\simeq
&& \Exterior^{i+1}({\mathbb{C}}^{n}) \boxtimes {\operatorname{sgn}}
&&    \oplus
    \Exterior^{i}({\mathbb{C}}^{n}) \boxtimes {\operatorname{sgn}}.  
\end{alignat*}

Using the notion $\mu^{\natural}(j,\pm)'$
 with $\natural=\flat$ or $\sharp$
 for $K'$-types,
 we may rewrite these decompositions as
\begin{align}
\mub(i,+)|_{K'}
\simeq & \mub(i,+)' \oplus \mub(i-1,+)'
\label{eqn:smbra1}
\\
\simeq & \mus(i-1,-)' \oplus \mus(i-2,-)', 
\notag
\\
\mus(i,+)|_{K'}
\simeq & \mus(i,+)' \oplus \mus(i-1,+)'
\label{eqn:smbra2}
\\
\simeq & \mub(i+1,-)' \oplus \mub(i,-)'.  
\notag
\end{align}
The second isomorphisms follow from \eqref{eqn:flatsharp}.  



For simplicity,
 we discuss the symmetry breaking operator
 $A\colon I_{\delta}(i,\lambda) \to J_{\varepsilon}(j,\nu)$
 in the case $j=i$, $\delta=+$, and $\varepsilon=-$.  
Then the branching rule \eqref{eqn:smbra1} tells
 that neither the $K'$-type $\mub (i,-)'$ nor $\mus(i,-)'$
 occurs in the $K$-type $\mub (i,+)$ of $I_+(i,\lambda)$.  
Likewise, 
 \eqref{eqn:smbra2} tells
 that the $K'$-type $\mus(i,-)'$
 does not occur in the $K$-type $\mus (i,+)$.  
Hence the matrix $S(A)$ in \eqref{eqn:Smat} must be of the form
 $\begin{pmatrix} 0 & 0 \\ \ast & 0\end{pmatrix}$.  



The vanishing statements in the other cases are proved similarly.  
\end{proof}


%%%%%%%%%%%%%%%%%%%%%%%%%%%%%%%%%%%%%%%%%%%
\subsection{Proof of Theorem \ref{thm:153315}
 on $(K,K')$-spectrum
\\
 for the normalized symmetry breaking operator 
 $\Atbb{\lambda}{\nu}{+}{i,j}\colon
I_{\delta}(i,\lambda) \to J_{\delta}(j,\nu)$}
\label{subsec:Apm1-5}
%%%%%%%%%%%%%%%%%%%%%%%%%%%%%%%%%%%%%%%%%%

In this section,
 we determine the $(K,K')$-spectrum $a_{\varepsilon}^{i,j}(\lambda,\nu)$
 and $d_{\varepsilon}^{i,j}(\lambda,\nu)$
 for $j=i, i-1$ 
 in \eqref{eqn:SAabcd}
 when $\varepsilon=+$.  
The case $\varepsilon=-$ will be discussed separately
 in Section \ref{subsec:Apm1}.  
By definition \eqref{eqn:Smumu}, 
 the constants $a_{\varepsilon}^{i,j}(\lambda,\nu)$
 and $d_{\varepsilon}^{i,j}(\lambda,\nu)$ are characterized 
 by the following equations:



\index{A}{Ahtsln1@$\Atbb \lambda \nu {+} {i,j}$}
\index{A}{0iotalmdd@$\iota_{\lambda}^{\ast}$}
\begin{alignat}{2}
\label{eqn:Aaij}
\Atbb \lambda \nu + {i,j}
\circ
\iota_{\lambda}^{\ast}
\circ
\iota_{j \to i}
=&
a_+^{i,j}(\lambda,\nu)
\iota_{\nu}^{\ast}
\quad
&&\text{on $\Exterior^{j}({\mathbb{C}}^{n})$,}
\\
\label{eqn:Adij}
\Atbb \lambda \nu + {i,j}
\circ
\iota_{\lambda}^{\ast}
\circ
\iota_{j+1 \to i+1}
=&
d_+^{i,j}(\lambda,\nu)
\iota_{\nu}^{\ast}
\quad
&&\text{on $\Exterior^{j+1}({\mathbb{C}}^{n})$}, 
\end{alignat}
where $\Atbb \lambda \nu + {i,j} \colon I_{\delta}(i,\lambda) \to J_{\delta}(j,\nu)$
 is the normalized symmetry breaking operator,
 $\iota_{\lambda}^{\ast}$ is the transform from the $K$-picture
 to the $N$-picture
 (see \eqref{eqn:KtoN}), 
 and 
\index{A}{0iotato@$\iota_{j \to i}$|textbf}
$
\iota_{j \to i} \colon \Exterior^j({\mathbb{C}}^{n}) \to \Exterior^i({\mathbb{C}}^{n+1})
$ is the normalized injective $O(n)$-homomorphism
 such that
\index{A}{prij@$\pr i j$, projection}
 $\pr i j \circ \iota_{j \to i} = {\operatorname{id}}$.  
The main results of this section
 are part of Theorem \ref{thm:153315}, 
 which is given as follows:

\begin{theorem}
\label{thm:minKscalar}
Suppose $\lambda, \nu\in {\mathbb{C}}$.  
\begin{align}
a_+^{i,i}(\lambda,\nu)
=&
\frac{\pi^{\frac{n-1}{2}} (\lambda-i)}
     {\Gamma(\lambda+1)}.  
\notag
\\
a_+^{i,i-1}(\lambda,\nu)
=&
\frac{\pi^{\frac{n-1}{2}} (n-\nu-i)}
     {\Gamma(\lambda+1)}.  
\notag
\\
\label{eqn:pmIi-5}
   d_+^{i,i}(\lambda,\nu)
   =&
   \frac
   {\pi^{\frac {n-1}{2}} (\nu-i)}
   {\Gamma(\lambda+1)}.  
\\
\label{eqn:152361}
d_+^{i,i-1}(\lambda,\nu)=&\frac{\pi^{\frac{n-1}{2}} (\lambda-n+i)}{\Gamma(\lambda+1)}.  
\end{align}
\end{theorem}

\begin{remark}
Theorem \ref{thm:minKscalar} generalizes 
 \cite[Thm~1.10]{sbon}
 in the spherical case
 ($i=j=0$ and $\delta = \varepsilon=+$).  
\end{remark}

The proof of Theorem \ref{thm:minKscalar} is divided
 into the following two steps:
\begin{enumerate}
\item[$\bullet$]
integral expression of 
\index{A}{a1ij@$a_+^{i,j}(\lambda, \nu)$}
$
a_+^{i,j}(\lambda,\nu)
$
 and 
\index{A}{d1ij@$d_+^{i,j}(\lambda, \nu)$}
$d_+^{i,j}(\lambda,\nu)
$
 (Section \ref{subsec:KKspec1});
\item[$\bullet$]
computation of the integral
 (Section \ref{subsec:KKspec2}).  
\end{enumerate}


\subsubsection{Integral expression of $(K,K')$-spectrum}
\label{subsec:KKspec1}

As the first step of the proof, 
 we give an integral expression
 of the $(K,K')$-spectrum $a_+^{i,j}(\lambda,\nu)$ and $d_+^{i,j}(\lambda,\nu)$.  
For $I \in {\mathfrak{I}}_{n,i}$, 
we recall from \eqref{eqn:QI}
 that the quadratic form $Q_I(b)$ is defined
 to be $\sum_{k \in I} {b_k}^2$, 
 and set 
\index{A}{QIb@$Q_I(b)$, quadratic polynomial}
\begin{align}
\alpha_I(b):=&1-\frac{2 Q_I(b)}{(1+|b|^2)|b|^2}, 
\label{eqn:alphaI}
\\
\delta_I(b):=&1-\frac{2 |b|^2}{1+|b|^2}-\frac{2 Q_I(b)}{(1+|b|^2)|b|^2}.  
\label{eqn:deltaI}
\intertext{Consider the following integrals:}
A_I(\lambda,\nu):=&\int_{{\mathbb{R}}^n}\Atcal \lambda \nu + {}(b)(1+|b|^2)^{-\lambda}\alpha_I(b) d b,
\notag
\\
D_I(\lambda,\nu)
:=&\int_{{\mathbb{R}}^n}\Atcal \lambda \nu + {}(b)(1+|b|^2)^{-\lambda}\delta_I(b) d b.
\notag  
\end{align}


Then the $(K,K')$-spectrum $a_+^{i,j}(\lambda,\nu)$ and $d_+^{i,j}(\lambda,\nu)$
 in \eqref{eqn:Aaij} and \eqref{eqn:Adij}, 
 respectively, 
 is given by the integrals
 $A_I(\lambda,\nu)$ and $D_I(\lambda,\nu)$
 as follows:
\begin{proposition}
[integral expression of $(K,K')$-spectrum]
\label{prop:1617113}
\begin{alignat*}{2}
a_+^{i,i}(\lambda, \nu)
=&
A_I(\lambda, \nu)
\qquad
&&\text{for any $I \in {\mathfrak {I}}_{n,i}$ with $n \not\in I$,}
\\
a_+^{i,i-1}(\lambda, \nu)
=&
A_I(\lambda, \nu)
\qquad
&&\text{for any $I \in {\mathfrak {I}}_{n,i}$ with $n \in I$, }
\\
d_+^{i,i}(\lambda, \nu)
=&
D_I(\lambda, \nu)
\qquad
&&\text{for any $I \in {\mathfrak {I}}_{n,i}$ with $n \not\in I$,}
\\
d_+^{i,i-1}(\lambda, \nu)
=&
-D_I(\lambda, \nu)
\qquad
&&\text{for any $I \in {\mathfrak {I}}_{n,i}$ with $n \in I$.}
\end{alignat*}
\end{proposition}



In order to prove Proposition \ref{prop:1617113}, 
 we use the $N$-picture of the principal series representations
 $I_{\delta}(i,\lambda)$ and $J_{\varepsilon}(j,\nu)$.  
By Proposition \ref{prop:minKN}
 for the vectors ${\bf{1}}_{\lambda}^{\mathcal{I}}$
 and $h_{\lambda}^{\mathcal{I}}$ belonging to the basic $K$-types,
 the equation \eqref{eqn:Aaij} means
 that for ${\mathcal{I}} \in {\mathfrak{I}}_{n+1,i}$
\index{A}{01oneIlmd@${\bf{1}}_{\lambda}^{\mathcal{I}}$}
\index{A}{hIlmd@$h_{\lambda}^{{\mathcal{I}}}$}
\begin{alignat*}{2}
\Atbb \lambda \nu {+} {i,i} {\bf{1}}_{\lambda}^{\mathcal{I}}
&=
a_+^{i,i}(\lambda,\nu) {{\bf{1}}_{\nu}'}^{{\mathcal{I}}}
\hphantom{(-1)^{i-}mii}
\quad
&&(n \notin {\mathcal{I}}), 
\\
\Atbb \lambda \nu {+} {i,i-1}
{\bf{1}}_{\lambda}^{\mathcal{I}}
&=
(-1)^{i-1} a_+^{i,i-1}(\lambda,\nu) {{\bf{1}}_{\nu}'}^{\mathcal{I}\setminus \{n\}}
\quad
&&(n\in {\mathcal{I}}).  
\end{alignat*}
The signature in the second formula
 arises from the definition \eqref{eqn:Tii2}
 of the projection $\pr i {i-1}$.  



To compute the constants $a_+^{i,j}(\lambda,\nu)$, 
 we take $I \in {\mathfrak {I}}_{n, i}$
 and set ${\mathcal{I}}:=I$, 
 regarded as an element of ${\mathfrak {I}}_{n+1, i}$, 
 where we recall Convention \ref{conv:index} of index sets.  
Since $0 \not \in {\mathcal{I}}$, 
 it follows from \eqref{eqn:I10} that 
\begin{alignat*}{2}
\Atbb \lambda \nu {+} {i,i}
 ({\bf{1}}_{\lambda}^{I})(0)
 =&
 a_+^{i,i}(\lambda,\nu) e_{I}
\qquad
&&\text{if }\,\,
 n \not \in I, 
\\
\Atbb \lambda \nu {+} {i,i-1}
 ({\bf{1}}_{\lambda}^{I})(0)
 =& 
 (-1)^{i-1} a_+^{i,i-1}(\lambda,\nu) e_{I \setminus \{n\}}
 \qquad
&&\text{if }\,\,
 n \in I.   
\end{alignat*}
Likewise, 
 the equation \eqref{eqn:Adij} means 
that
for ${\mathcal{I}} \in {\mathfrak{I}}_{n+1,i+1}$
\begin{alignat*}{2}
\Atbb \lambda \nu {+} {i,i} h_{\lambda}^{\mathcal{I}}
=&
d_+^{i,i}(\lambda,\nu) {h'}_{\nu}^{{\mathcal{I}}}
\qquad\qquad\qquad
&&(n \notin {\mathcal{I}}), 
\\
\Atbb \lambda \nu {+} {i,i-1}
h_{\lambda}^{\mathcal{I}}
=&(-1)^i d_+^{i,i-1}(\lambda,\nu) {h_{\nu}'}^{\mathcal{I}\setminus\{n\}}
\quad
&&(n\in {\mathcal{I}}).  
\end{alignat*}
In this case, 
 we take $I \in {\mathfrak{I}}_{n,i}$
 and set ${\mathcal{I}}:=I \cup \{0\} \in {\mathfrak {I}}_{n+1, i+1}$.  
Then \eqref{eqn:hI0} implies 
\begin{alignat}{2}
\label{eqn:CApm-5}
(\Atbb \lambda \nu + {i,i} h_{\lambda}^{I \cup \{0\}})(0)
=&
d_+^{i,i}(\lambda,\nu) e_I
\quad
&&\text{if $n \not \in I$}, 
\\
(\Atbb \lambda \nu + {i,i-1} h_{\lambda}^{I \cup \{0\}})(0)
=&
(-1)^i d_+^{i,i-1}(\lambda,\nu) e_{I\setminus \{n\}}
\quad
&&\text{if $n \in I$.  }
\notag
\end{alignat}



Let us compute 
$
 \Atbb \lambda \nu {+} {i,j}
 ({\bf{1}}_{\lambda}^{I})(0)
$
 and 
$
 \Atbb \lambda \nu {+} {i,j}
 (h_{\lambda}^{I \cup \{0\}})(0)
$
 for $j=i$ and $i-1$.  
If $\operatorname{Re} \lambda \gg |\operatorname{Re} \nu|$, 
 then the matrix-valued distribution kernel 
 $\Atcal \lambda \nu + {i,j}$
 (see \eqref{eqn:Kijtilde})
 of the regular symmetry breaking operator
 $\Atbb \lambda \nu +{i,j}$ is decomposed as 
\[
   \Atcal \lambda \nu + {i,j}= \Atcal \lambda \nu + {} \Rij ij, 
\]
where 
\index{A}{Act1@$\Atcal \lambda \nu + {}$}
$\Atcal \lambda \nu + {}$ is the scalar-valued,
 locally integrable function
 defined in \eqref{eqn:KAlnn+}
 and the matrix-valued function 
\index{A}{Rij@$\Rij ij$}
$
\Rij ij \in C^{\infty}({\mathbb{R}}^n \setminus \{0\})
 \otimes 
 {\operatorname{Hom}}_{{\mathbb{C}}}(\Exterior^i({\mathbb{C}}^n), \Exterior^j({\mathbb{C}}^{n-1}))$
 is defined in \eqref{eqn:Rij}.  
Hence,
 we have
\begin{align*}
(\Atbb \lambda \nu +{i,j} \psi)(0)
=&\int_{\mathbb{R}^n} \Atcal \lambda \nu +{}(-b) \Rij ij (-b)\psi(b) d b
\\
=&\int_{\mathbb{R}^n} \Atcal \lambda \nu +{}(b) \Rij ij (b)\psi(b) d b
\end{align*}
in the $N$-picture for any $\psi \in \iota_{\lambda}^{\ast}({\mathcal{E}}^i(S^n))\subset C^{\infty}(\mathbb{R}^n) \otimes \Exterior^i(\mathbb{C}^n)$.  
Thus Proposition \ref{prop:1617113} is a consequence
 of the following two lemmas
 on the computation of $\Rij i j(b) \psi(b) \in \Exterior^j({\mathbb{C}}^{n-1})$
 for $\psi={\bf{1}}_{\lambda}^{I}$
 or $h_{\lambda}^{I \cup \{0\}}$ and 
 for $j=i$ or $i-1$.  
\begin{lemma}
\label{lem:eIRone}
Suppose $I \in {\mathfrak{I}}_{n,i}$.  
\begin{enumerate}
\item[{\rm{(1)}}]
If $n \not \in I$, 
then the coefficient of $e_I$
 in 
$
  \Rij i i (b) {\bf{1}}_{\lambda}^{I}(b)
$
is given by 
\[
  (1+|b|^2)^{-\lambda} \alpha_I(b)
  =
 (1+|b|^2)^{-\lambda}(1-\frac{2Q_I(b)}{(1+|b|^2)|b|^2}), 
\]
where we recall 
 $Q_I(b)=\sum_{l \in I}b_l^2$ from \eqref{eqn:QI}.  
\item[{\rm{(2)}}]
If $n \in I$, 
then the coefficient of $e_{I\setminus \{ n \}}$
 in 
$
  \Rij i {i-1} (b) {\bf{1}}_{\lambda}^{I}(b)
$
is given by 
\[
  (-1)^{i-1}(1+|b|^2)^{-\lambda} \alpha_I(b)
  = 
  (1+|b|^2)^{-\lambda} (1-\frac{2Q_I(b)}{(1+|b|^2)|b|^2}).  
\]
\end{enumerate}
\end{lemma}

\begin{lemma}
\label{lem:eIpm-5}
Suppose $I \in {\mathfrak{I}}_{n,i}$.  
\begin{enumerate}
\item[{\rm{(1)}}]
If $n \not \in I$, 
then the coefficient of $e_I$
 in 
$
  \Rij i i (b) h_{\lambda}^{I \cup \{0\}}(b)
$
is given by 
\[
  (1+|b|^2)^{-\lambda} \delta_I(b)
  =
 (1+|b|^2)^{-\lambda-1}(1-|b|^2-\frac{2Q_I(b)}{|b|^2}).  
\]
\item[{\rm{(2)}}]
If $n \in I$, 
then the coefficient of $e_{I\setminus \{ n \}}$
 in 
$
  \Rij i {i-1} (b) h_{\lambda}^{I \cup \{0\}}(b)
$
is given by 
\[
  (-1)^{i-1}(1+|b|^2)^{-\lambda} \delta_I(b)
  = 
  (-1)^{i-1} (1+|b|^2)^{-\lambda-1} (1-|b|^2-\frac{2Q_I(b)}{|b|^2}).  
\]
\end{enumerate}
\end{lemma}



\begin{proof}
[Proof of Lemma \ref{lem:eIRone}]
Let $\sigma$ be the $i$-th exterior representation
 on $\Exterior^i({\mathbb{C}}^n)$.  
We recall from \eqref{eqn:Rij}
 $\Rij ij=\pr ij \circ \sigma \circ \psi_n$.  
We identify $I \in {\mathfrak{I}}_{n,i}$ with ${\mathcal{I}} \in {\mathfrak{I}}_{n+1,i}$
 such that $n \not \in {\mathcal{I}}$ as usual, 
 and apply the formula \eqref{eqn:minI}
 of ${\bf{1}}_{\lambda}^I$.  
Then we have
\[
  \sigma(\psi_n(b)) {\mathbf{1}}_{\lambda}^{I}(b)
=
(1+|b|^2)^{-\lambda}
\sigma(\psi_n(b))
\sum_{J \in {\mathfrak{I}}_{n,i}} (\det \psi_{n+1}(1,b))_{I J} e_{J}.  
\]
By the formula \eqref{eqn:exrep}
 of the matrix coefficients of the exterior tensor representation, 
 the coefficient of $e_I$ 
 in $\sigma(\psi_n(b)) {\mathbf{1}}_{\lambda}^{I}(b)$
 amounts to 
\[
   (1+|b|^2)^{-\lambda}
   \sum_{J \in {\mathfrak {I}}_{n,i}}
   (\det \psi_{n+1}(1,b))_{I J}
   (\det \psi_n(b))_{I J},   
\]
which is equal to 
\[
  (1+|b|^2)^{-\lambda}(1-\frac{2Q_I(b)}{(1+|b|^2)|b|^2})
   =
   (1+|b|^2)^{-\lambda} \alpha_I(b)
\]
by the minor summation formula \eqref{eqn:kbsum}
 in Proposition \ref{prop:msum}.  
Hence the lemma follows from $\pr ii (e_I) = e_I$ $(n \notin I)$
 and $\pr i{i-1} (e_I) = (-1)^{i-1}e_{I\setminus\{n\}}$ $(n \in I)$
 (see \eqref{eqn:Tii1} and \eqref{eqn:Tii2}).  
\end{proof}



\begin{proof}
[Proof of Lemma \ref{lem:eIpm-5}]
The proof goes in parallel to that of Lemma \ref{lem:eIRone}.  
For the sake of completeness, 
 we give a proof.  


By \eqref{eqn:hIlmd} and \eqref{eqn:exrep}, 
we have
\begin{align*}
&\sigma(\psi_n(b)) h_{\lambda}^{I \cup \{0\}}(b)
\\
=& -(1+|b|^2)^{-\lambda}
  \sigma(\psi_n(b))
  \sum_{J \in {\mathfrak{I}}_{n,i}}
  (\det \psi_{n+1}(1,b))_{I \cup \{0\}, J \cup \{0\}} e_J
\\
=& -(1+|b|^2)^{-\lambda}
  \sum_{J \in {\mathfrak{I}}_{n,i}}
  \sum_{J' \in {\mathfrak{I}}_{n,i}}
  (\det \psi_{n+1}(1,b))_{I \cup \{0\}, J \cup \{0\}}
  \det \psi_n(b)_{J' J}e_{J'}.  
\end{align*}



Hence the coefficient of $e_I$
 in $\sigma(\psi_n(b)) h_{\lambda}^{I \cup \{0\}}(b)$
 is equal to 
\begin{equation*}
  - (1+|b|^2)^{-\lambda}
  \sum_{J \in {\mathfrak{I}}_{n,i}}
  (\det \psi_{n+1}(1,b))_{I \cup \{0\}, J \cup \{0\}}
   \det \psi_n(b)_{I J}, 
\end{equation*}
which amounts to 
\[
  (1+|b|^2)^{-\lambda-1}(1-|b|^2-\frac{2Q_I(b)}{|b|^2})
  =
 (1+|b|^2)^{-\lambda} \delta_I(b)
\]
by the minor summation formula
 in Proposition \ref{prop:msum} (2).  
Thus we have shown the lemma.  
\end{proof}

Therefore we have completed the proof of Proposition \ref{prop:1617113}.  



\subsubsection{Integral formula of the $(K,K')$-spectrum}
\label{subsec:KKspec2}
As the second step,
 we compute the integrals
 $A_I(\lambda,\nu)$ and $D_I(\lambda,\nu)$
 in Section \ref{subsec:KKspec1}.  
We begin with the following integral formul{\ae}:
Denote by $d \omega$ the standard measure
 on the unit sphere $S^{n-1}=\{\omega=(\omega_1, \cdots, \omega_n) \in {\mathbb{R}}^n:
\sum_{j=1}^{n} {\omega_j}^2=1\}$.  



For $a,b \in {\mathbb{C}}$
 with $\operatorname{Re} a, \operatorname{Re} b>-1$, 
 we set
\begin{equation}
\label{eqn:intsphere}
S(a,b) 
\equiv
S_n(a,b)
:=
\int_{S^{n-1}} |\omega_n|^a |\omega_{n-1}|^b d \omega.  
\end{equation}
Then we have 
\begin{equation}
   S_n(a,0) = \int_{S^{n-1}} |\omega_n|^a d \omega
=
\frac{2 \pi^{\frac {n-1}{2}} \Gamma(\frac{a+1}{2})}{\Gamma(\frac{a+n}{2})}, 
\label{eqn:1.11}
\end{equation}
see \cite[Lemma 7.6]{sbon}, 
 for instance.  
More generally,
 we have the following.  
\begin{lemma}
\label{lem:intsphere}
Suppose ${\operatorname{Re}}\, a >-1$ and ${\operatorname{Re}}\, b >-1$.  
Then we have 
\begin{equation}
\label{eqn:Sab}
S(a,b)=
\frac
{2 \pi^{\frac{n-2}{2}}
\Gamma(\frac{a+1}{2})\Gamma(\frac{b+1}{2})}{\Gamma(\frac{a+b+n}{2})}.  
\end{equation}
\end{lemma}
It is convenient to write down the following recurrence relations
 that are derived readily from \eqref{eqn:Sab}:
\begin{align}
\label{eqn:Sab2}
S(a,2)=&
\frac{1}{a+n}S(a,0), 
\\
\label{eqn:Sab3}
S(a+2,0)=&
\frac{a+1}{a+n}S(a,0).  
\end{align}

\begin{proof}
[Proof of Lemma \ref{lem:intsphere}]
For any $f \in C(S^{n-1})$, 
 the polar coordinates give the following expression
 of the integral:
\begin{equation}
\int_{S^{n-1}} f (\omega) d \omega
=\int_{-1}^{1} \int_{S^{n-2}} 
 f(\sqrt{1-t^2} \eta, t) (1-t^2)^{\frac {n-3}{2}} d \eta d t.
\label{eqn:polarS}
\end{equation}
Then we have
\begin{align*}
S(a,b)
=& \int_{-1}^1 \int_{S^{n-2}}
  |\sqrt{1-t^2} \eta_{n-1} |^b |t|^a (1-t^2)^{\frac{n-3}{2}} d\eta d t 
\\
=& \int_{S^{n-2}}
  |\eta_{n-1}|^b d \eta \int_{-1}^1 |t|^a (1-t^2)^{\frac{n+b-3}{2}} d t. 
\end{align*}
The first term equals $S_{n-1}(b,0)$, 
 see \eqref{eqn:1.11}.  
The second term is given by the Beta function:
\begin{equation}
\int_{0}^{1} t^{2A-1}(1-t^2)^{B-1} d t 
=\frac{\Gamma(A)\Gamma(B)}{2\Gamma(A+B)}.  
\label{eqn:Beta}
\end{equation}
Here we get the lemma.  
\end{proof}

\begin{lemma}
\label{lem:1616116}
Let $\Atcal \lambda \nu +{}$ be the 
 (scalar-valued) locally integrable function on ${\mathbb{R}}^n$
 defined in \eqref{eqn:KAlnn+}
 for $\operatorname{Re}\left(\lambda-\nu\right)>0$
 and $\operatorname{Re}\left(\lambda+\nu\right)>n-1$.  
\begin{enumerate}
\item[{\rm{(1)}}]
We have 
\index{A}{Act1@$\Atcal \lambda \nu + {}$}
\[
   \int_{{\mathbb{R}}^n} \Atcal \lambda \nu + {}(b)
                         (1+|b|^2)^{-\lambda} d b
  =
  \frac{\pi^{\frac{n-1}{2}}}{\Gamma(\lambda)}.  
\]
\item[{\rm{(2)}}]
Let $l \in \{1,2,\cdots,n\}$.  
Then we have 
\begin{multline*}
\int_{{\mathbb{R}}^n} \Atcal \lambda \nu + {}(b)
                         (1+|b|^2)^{-\lambda} 
                       \frac{2 {b_{\ell}}^2}{(1+|b|^2)|b|^2} d b
\\
  =
\frac{\pi^{\frac{n-1}{2}}}{\Gamma(\lambda+1)}
\times
\begin{cases}
  1
\quad
&\text{if $1\le \ell \le n-1$}, 
\\
\lambda + \nu -n+1
\quad
&\text{if $\ell=n$}.  
\end{cases}
\end{multline*}
\end{enumerate}
\end{lemma}

\begin{proof}
(1)\enspace
This formula was given in \cite[Prop.~7.4]{sbon}, 
 but we give a proof here 
 in order to illustrate our notation
 for later purpose.  
By \eqref{eqn:Rabnew}, 
 the left-hand side amounts to 
\begin{align*}
&\frac{1}{\Gamma(\frac{\lambda+\nu-n+1}{2})\Gamma(\frac{\lambda-\nu}{2})} 
\int_0^{\infty} r^{\lambda-\nu-1} (1+r^2)^{-\lambda} d r
          \int_{S^{n-1}}|\omega_n|^{\lambda+\nu-n} d \omega
\\
=&
\frac{1}{2\Gamma(\frac{\lambda+\nu-n+1}{2})\Gamma(\frac{\lambda-\nu}{2})}
B(\frac{\lambda-\nu}{2},\frac{\lambda+\nu}2)S(\lambda+\nu-n,0), 
\end{align*}
which equals $\frac{\pi^{\frac {n-1}2}}{\Gamma(\lambda)}$
 by \eqref{eqn:Sab}.  
\par\noindent
(2)\enspace
By a similar computation as above,
 the ratio of the two integrals is given as 
\begin{equation*}
\frac{\text{the left-hand side of (2)}}
      {\text{the left-hand side of (1)}}
 =
 \frac{2 \int_0^{\infty} r^{\lambda-\nu-1} (1+r^2)^{-\lambda-1} d r
       \int_{S^{n-1}}|\omega_n|^{\lambda+\nu-n} |\omega_{\ell}|^2 d \omega}
      {\int_0^{\infty} r^{\lambda-\nu-1} (1+r^2)^{-\lambda} d r
       \int_{S^{n-1}}|\omega_n|^{\lambda+\nu-n} d \omega}.  
\end{equation*}
The right-hand side depends on 
 whether $\ell=n$ or not.  
It amounts to 
\begin{align*}
&\frac{2B(\frac{\lambda-\nu}{2}, \frac{\lambda+\nu}2+1)}
     {B(\frac{\lambda-\nu}{2}, \frac{\lambda+\nu}{2})}
\cdot
\frac{1}{S(\lambda+\nu-n,0)}
\times
\begin{cases}
  S(\lambda+\nu-n,2)
\\
  S(\lambda+\nu-n+2,0)
\end{cases}
\\
=&
\frac{\lambda+\nu}
     {\lambda}
\cdot
\frac{1}{\lambda+\nu}
\times
\begin{cases}
  1
\quad
&\text{if $1\le \ell \le n-1$}, 
\\
\lambda + \nu -n+1
\quad
&\text{if $\ell=n$}
\end{cases}
\end{align*}
by the recurrence relations
 \eqref{eqn:Sab2} and \eqref{eqn:Sab3}.  
\end{proof}


\begin{lemma}
\label{lem:1619112}
$A_I(\lambda,\nu)-D_I(\lambda,\nu)=\frac{\pi^{\frac {n-1}2}(\lambda-\nu)}{\Gamma(\lambda+1)}$.  
\end{lemma}
\begin{proof}
By the definitions \eqref{eqn:alphaI} and \eqref{eqn:deltaI}, 
 we have $\alpha_I(b) - \delta_I(b)=\dfrac{2|b|^2}{1+|b|^2}$.  
Thus we have
\begin{align*}
   A_I(\lambda,\nu)-D_I(\lambda,\nu)
   =&
   2 \int_{{\mathbb{R}}^{n}}
   \Atcal \lambda \nu + {}(b)
   (1+|b|^2)^{-\lambda-1}|b|^2 d b
\\
   =& \frac{B(\frac{\lambda-\nu}2,\frac{\lambda+\nu}2+1)
             S(\lambda+\nu-n,0)}
           {\Gamma(\frac{\lambda-\nu}{2})\Gamma(\frac{\lambda+\nu-n+1}{2})}, 
\end{align*}
as in the proof of Lemma \ref{lem:1616116} (1).  
Thus the lemma follows from \eqref{eqn:Sab}.  
\end{proof}

\begin{proof}
[Proof of Theorem \ref{thm:minKscalar}]
It follows from Lemma \ref{lem:1616116} that 
\[
  A_I(\lambda,\nu)
  =
  \frac{\pi^{\frac{n-1}{2}}}{\Gamma(\lambda+1)}
  \times 
\begin{cases}
\lambda-i
&
\text{if $n \not\in I$, }
\\
\lambda-(i-1)-(\lambda+\nu-n+1)
\quad
&
\text{if $n \in I$, }
\end{cases}
\]
whence the first two formul{\ae} of Theorem \ref{thm:minKscalar}
 are proved 
 by Proposition \ref{prop:1617113}.  

By Lemma \ref{lem:1619112}, 
 we have 
\begin{align*}
  D_I(\lambda,\nu)
  &=
  A_I(\lambda,\nu)
  -
  \frac{\pi^{\frac{n-1}{2}}(\lambda-\nu)}{\Gamma(\lambda+1)}
\\
  &=
  \frac{\pi^{\frac{n-1}{2}}}{\Gamma(\lambda+1)}
  \times
\begin{cases}
(\lambda-i)-(\lambda-\nu)
&
\text{if $n \not\in I$, }
\\
(n-\nu-i)-(\lambda-\nu)
\quad
&
\text{if $n \in I$, }
\end{cases}
\end{align*}
whence the last two formul{\ae}
 of Theorem \ref{thm:minKscalar}
 by Proposition \ref{prop:1617113}.  
\end{proof}

\begin{remark}
\label{rem:KKdual}
Alternatively,
 one could derive the last two formul{\ae}
 of Theorem \ref{thm:minKscalar} from the first two
 by using the duality theorem for symmetry breaking operators
 given in Proposition \ref{prop:SBOdual}.  
\end{remark}

%%%%%%%%%%%%%%%%%%%%%%%%%%%%%%%%%%%%%%%%%%%%%%%%
\subsection{Proof of Theorem \ref{thm:153315}
 on the $(K,K')$-spectrum for $\Atbb{\lambda}{\nu}{-}{i,j}:
I_{\delta}(i,\lambda) \to J_{-\delta}(j,\nu)$}
\label{subsec:Apm1}
%%%%%%%%%%%%%%%%%%%%%%%%%%%%%%%%%%%%%%%%%%%%%%



In this section,
 we determine the $(K,K')$-spectrum
 $b_-^{i,i-1}(\lambda,\nu)$ and  $c_-^{i,i}(\lambda,\nu)$
 in \eqref{eqn:SAabcd}
 for the normalized regular symmetry breaking operators
\index{A}{Ahtsln2@$\Atbb \lambda \nu {-} {i,j}$}
$
\Atbb \lambda {\nu}{-} {i,j}
\colon
I_{\delta}(i,\lambda) \to J_{-\delta}(j,\nu)
$
 with $j \in \{i-1, i\}$.  
By definition,
 these constants 
\index{A}{b2ii-1@$b_-^{i,i-1}(\lambda,\nu)$|textbf}
$
   b_-^{i,i-1}(\lambda,\nu)
$
 and $c_-^{i,i}(\lambda,\nu)$
 are characterized by the following equations:
\begin{alignat}{2}
\label{eqn:Abij}
\Atbb \lambda \nu - {i,i-1}
\circ
\iota_{\lambda}^{\ast}
=&
b_-^{i,i-1}(\lambda,\nu)
\iota_{\nu}^{\ast}
\circ
\pr ii
\quad
&&\text{on $\Exterior^{i}({\mathbb{C}}^{n+1})$, }
\\
\label{eqn:Acij}
\Atbb \lambda \nu - {i,i}
\circ
\iota_{\lambda}^{\ast}
=&
(-1)^{i}
c_-^{i,i}(\lambda,\nu)
\circ
\iota_{\nu}^{\ast}
\circ
\pr {i+1}i
\quad
&&\text{on $\Exterior^{i+1}({\mathbb{C}}^{n+1})$}.  
\end{alignat}
The main results of this section are given as follows:
\begin{theorem}
\label{thm:CApm1}
Suppose $\lambda, \nu \in {\mathbb{C}}$.  
Then we have
\begin{align}
\label{eqn:Cpmii1}
   b_-^{i,i-1}(\lambda,\nu)
   =&
   -
   \frac
   {2\pi^{\frac {n-1}{2}}}
   {\Gamma(\lambda+1)}, 
\\
\label{eqn:pmIi}
   c_-^{i,i}(\lambda,\nu)
   =&
   \frac
   {2 (-1)^{i+1} \pi^{\frac {n-1}{2}}}
   {\Gamma(\lambda+1)}.  
\end{align}
\end{theorem}

This is the remaining part
 of Theorem \ref{thm:153315}, 
 and the proof of Theorem \ref{thm:153315} will be complete
 when Theorem \ref{thm:CApm1} is shown.  
The proof of Theorem \ref{thm:CApm1} is parallel to 
 that of Theorem \ref{thm:minKscalar}, 
 and thus will be discussed briefly.  
We begin with an integral expression
 of the constants 
 $b_-^{i,i-1}(\lambda,\nu)$
 and 
 $c_-^{i,i}(\lambda,\nu)$ 
 as follows.  
\begin{proposition}
[integral expression of $(K,K')$-spectrum]
\label{prop:1617115}
\begin{align*}
b_-^{i,i-1}(\lambda,\nu)=& -2\int_{{\mathbb{R}}^n} \Atcal \lambda \nu - {}(b)
                           (1+|b|^2)^{-\lambda-1}b_n d b, 
\\
c_-^{i,i}(\lambda,\nu)=& 2(-1)^{i+1}\int_{{\mathbb{R}}^n} \Atcal \lambda \nu - {}(b)
                           (1+|b|^2)^{-\lambda-1}b_n d b.  
\end{align*}
\end{proposition}
Admitting Proposition \ref{prop:1617115} for the time being, 
 we complete the proof 
 of Theorem \ref{thm:CApm1}.  
\begin{proof}
[Proof of Theorem \ref{thm:CApm1}]
Theorem \ref{thm:CApm1} is an immediate consequence
 of Proposition \ref{prop:1617115}
 and the following lemma.  
\end{proof}



\begin{lemma}
\label{lem:Abnint}
\index{A}{Act2@$\Atcal \lambda \nu - {}$}
\[
\int_{{\mathbb{R}}^n} \Atcal \lambda \nu -{}(b) (1+|b|^2)^{-\lambda-1}b_n d b
=
\frac{\pi^{\frac{n-1}{2}}}{\Gamma(\lambda+1)}.  
\]
\end{lemma}
\begin{proof}
We use the identity
\[ 
   b_n \Atcal \lambda \nu -{}(b)=\Atcal {\lambda+1} \nu +{}(b).  
\]
Then the lemma follows from Lemma \ref{lem:1616116}.  
\end{proof}
The rest of this section is devoted to the proof
 of Proposition \ref{prop:1617115}.  
In the $N$-picture, 
 the equation \eqref{eqn:Abij} amounts to 
\[
\Atbb \lambda \nu {-} {i,i-1} ({\bf{1}}_{\lambda}^{{\mathcal{I}}})
=
\begin{cases}
b_-^{i,i-1}(\lambda,\nu) {h'}_{\nu}^{{\mathcal{I}}}
\qquad
&\text{if}\ n \notin {\mathcal{I}}, 
\\
0
\qquad
&\text{if}\ n \in {\mathcal{I}}, 
\end{cases}
\]
for all ${\mathcal{I}} \in {\mathfrak{I}}_{n+1,i}$, 
 whereas \eqref{eqn:Acij} amounts to 
\[
\Atbb \lambda \nu {-} {i,i} h_{\lambda}^{\mathcal{I}}
=
\begin{cases}
c_-^{i,i}(\lambda,\nu) {{\bf{1}}'}_{\nu}^{{\mathcal{I}} \setminus \{n\}}
\qquad
&\text{if}\ n \in {\mathcal{I}}, 
\\
0
\qquad
&\text{if}\ n \notin {\mathcal{I}}, 
\end{cases}
\]
for all ${\mathcal{I}} \in {\mathfrak{I}}_{n+1,i+1}$.  
In particular, 
 we have
\begin{alignat}{2}
\label{eqn:CApm2}
\Abb \lambda \nu {-} {i,i-1} 
({\bf{1}}_{\lambda}^{I \cup \{0\}})(0)
=&
b_-^{i,i-1}(\lambda,\nu) e_{I}
\quad
&&\text{for any $I \in {\mathfrak{I}}_{n-1,i-1}$}, 
\\
\label{eqn:CApm}
(\Atbb \lambda \nu -{i,i} h_{\lambda}^{I \cup \{n\}})(0)
=&
c_-^{i,i}(\lambda,\nu) e_I
\quad
&&\text{for any $I \in {\mathfrak{I}}_{n-1,i}$}
\end{alignat}
by \eqref{eqn:I10} and \eqref{eqn:hI0}
 because $0 \notin I$.  



The distribution kernel $\Atcal \lambda \nu -{i,j}$
 of the regular symmetry breaking operator
 $\Atbb \lambda \nu -{i,j}$ is decomposed as
\[
   \Atcal \lambda \nu -{i,j} = \Atcal \lambda \nu -{} \Rij ij, 
\]
 where
\index{A}{Act2@$\Atcal \lambda \nu - {}$}
 $\Atcal \lambda \nu -{}$ is the scalar-valued, 
locally integrable function defined
 in \eqref{eqn:KAlnn-}
 and the matrix-valued function 
\index{A}{Rij@$\Rij ij$}
 $\Rij ij$ is defined in \eqref{eqn:Rij}.  
Then we have
\begin{align*}
(\Atbb \lambda \nu -{i,j} \psi)(0)
=&\int_{\mathbb{R}^n} \Atcal \lambda \nu -{}(-b) \Rij ij (-b)\psi(b) d b
\\
=&-\int_{\mathbb{R}^n} \Atcal \lambda \nu -{}(b) \Rij ij (b)\psi(b) d b
\end{align*}
in the $N$-picture for any $\psi \in \iota_{\lambda}^{\ast}({\mathcal{E}}^i(S^n))$.  
Hence Proposition \ref{prop:1617115} is a consequence
 of \eqref{eqn:CApm2}, \eqref{eqn:CApm}, 
 and of the following two lemmas.  

\begin{lemma}
\label{lem:TeIpm}
Suppose $I \in {\mathfrak{I}}_{n-1,i}$.  
Then the coefficient of $e_{I}$
 in 
$
   \Rij i {i-1}(b) {\bf{1}}_{\lambda}^{I \cup \{0\}}(b)
$
is equal to
\[
   2(1+|b|^2)^{-\lambda-1}b_n.  
\]
\end{lemma}

\begin{proof}
Using  the formula \eqref{eqn:minI}
 of ${\bf{1}}_{\lambda}^{\mathcal{I}}(b)$, 
 we have for $I \in {\mathfrak{I}}_{n,i}$
\begin{align*}
& (1+|b|^2)^{\lambda} \Rij i {i-1}(b) {\bf{1}}_{\lambda}^{I\cup\{0\}}(b)
\\
=& \Rij i{i-1}(b) 
   \sum_{J \in {\mathfrak{I}}_{n,i}}
   \det \psi_{n+1}(1,b)_{I\cup\{0\}, J} e_J
\\
=& \pr i {i-1} 
   \sum_{J \in {\mathfrak{I}}_{n,i}}
   \sum_{J' \in {\mathfrak{I}}_{n,i}}
   \det \psi_{n+1}(1,b)_{I\cup\{0\}, J} 
   \det \psi_{n}(b)_{J' J}
   e_{J'}
\\
=&(-1)^{i-1}\sum_{J \in {\mathfrak{I}}_{n,i}}
   \sum_{J' \in {\mathfrak{I}}_{n,i} \setminus {\mathfrak{I}}_{n-1,i}}
   \det \psi_{n+1}(1,b)_{I\cup\{0\}, J} 
   \det \psi_{n}(b)_{J' J}
   e_{J' \setminus \{n\}}.  
\end{align*}
Here, 
 for $J' \in {\mathfrak{I}}_{n,i}$, 
 we mean by $J' \not \in {\mathfrak{I}}_{n-1,i}$ the condition
 that $n \not \in J'$.  
Hence the coefficient of $e_I$
in 
$
   \Rij i {i-1}(b) {\bf{1}}_{\lambda}^{I \cup \{0\}}(b)
$
amounts to 
\[
(-1)^{i-1}\sum_{J \in {\mathfrak{I}}_{n,i}}
   \det \psi_{n+1}(1,b)_{I\cup\{0\}, J} 
   \det \psi_{n}(b)_{I\cup \{n\}, J}.  
\]
Now the lemma follows from Lemma \ref{lem:S0n}.  
\end{proof}


\begin{lemma}
\label{lem:eIpm}
Suppose $I \in {\mathfrak{I}}_{n-1,i}$.  
The coefficient of $e_I$
 in 
$
  \Rij ii (b) h_{\lambda}^{I \cup \{n\}}(b)
$
is given by 
\[
  2(-1)^i (1+|b|^2)^{-\lambda-1}b_n.  
\]
\end{lemma}

\begin{proof}
By \eqref{eqn:hIlmd} and \eqref{eqn:exrep}, 
we have
\begin{align*}
&\sigma(\psi_n(b)) h_{\lambda}^{I \cup \{n\}}(b)
\\
=&-(1+|b|^2)^{-\lambda}
  \sigma(\psi_n(b))
  \sum_{J \in {\mathfrak{I}}_{n,i}}
  \det \psi_{n+1}(1,b)_{I \cup \{n\}, J \cup \{0\}}e_J
\\
=&-(1+|b|^2)^{-\lambda}
  \sum_{J \in {\mathfrak{I}}_{n,i}}
  \sum_{J' \in {\mathfrak{I}}_{n,i}}
  \det \psi_{n+1}(1,b)_{I \cup \{n\}, J \cup \{0\}} 
  \det \psi_n(b)_{J' J}e_{J'}.  
\end{align*}
Applying the projection 
 $\pr ii:\Exterior^{i}({\mathbb{C}}^{n}) \to \Exterior^{i}({\mathbb{C}}^{n-1})$
 (see \eqref{eqn:Tii1}), 
 we find that the coefficient of $e_I$
 in $\Rij ii (b) h_{\lambda}^{I \cup \{n\}}(b)$
 is equal to 
\[
  -(1+|b|^2)^{-\lambda}
  \sum_{J \in {\mathfrak{I}}_{n,i}}
  \det \psi_{n+1}(1,b)_{I \cup \{n\}, J \cup \{0\}}
  \det \psi_n(b)_{I J}.  
\]
Hence the lemma follows from the minor summation formula
 in Lemma \ref{lem:Sn0}.  
\end{proof}


\subsection{Matrix-valued functional equations}
\label{subsec:FI}
The third goal of this chapter 
 is to obtain explicit matrix-valued functional equations
 for the regular symmetry breaking operators
 $\Atbb \lambda \nu \pm {i,j}$.  
We retain the setting 
 where $(G,G')=(O(n+1,1),O(n,1))$.  
By the 
\index{B}{genericmultiplicityonetheorem@generic multiplicity-one theorem}
generic multiplicity-one theorem
 (Theorem \ref{thm:unique}),
 two symmetry breaking operators from 
 the $G$-module $I_{\delta}(V,\lambda)$
 to the $G'$-module $J_{\varepsilon}(W,\nu)$
 must be proportional to each other
 if $[V:W]\ne0$ and $(\lambda,\nu,\delta,\varepsilon)$ does not belong 
 to the set $\Psising$
 of special parameters.  
In Sections \ref{subsec:FI} and \ref{subsec:161702},
 we consider the case 
\[
    (V,W)=
    (\Exterior^i({\mathbb{C}}^n), \Exterior^j({\mathbb{C}}^{n-1})),
\quad
    j \in \{i-1,i\}, 
\]
 and compare the (normalized) regular symmetry breaking operator
 $\Atbb \lambda \nu \gamma {i,j}$
 with its composition of the Knapp--Stein intertwining operator
 for $G$ or for the subgroup $G'$
 as in the following diagrams:
\begin{eqnarray*}
\xymatrix{
I_{\delta}(i,\lambda)\ar[rr]^{\Atbb \lambda \nu \gamma {i,j}}
\ar[drr]_{\Atbb \lambda{n-1-\nu}\gamma{i,j}}
&& J_{\varepsilon}(j,\nu)\ar[d]^{\Ttbb \nu {n-1-\nu}j}
\\
&&J_{\varepsilon}(j,n-1-\nu),
}
&&
\xymatrix{
I_{\delta}(i,\lambda) \ar[d]_{\Ttbb \lambda {n-\lambda}i}
\ar[drr]^{\Atbb \lambda \nu \gamma {i,j}}
&&
\\
I_{\delta}(i,n-\lambda)\ar[rr]_{\Atbb {n-\lambda}\nu\gamma{i,j}}
&&J_{\varepsilon}(j,\nu)
}
\end{eqnarray*}
where $\gamma = \delta \varepsilon$.  
We obtain closed formul{\ae}
 of the proportional constants 
 for the two operators in each diagram
 in Theorems \ref{thm:TAA} and \ref{thm:ATA}.  
The zeros of the proportional constants provide us 
 crucial information
 on the kernels and the images of the symmetry breaking operators
 $\Atbb \lambda \nu {\delta \varepsilon} {i,j} \colon
 I_{\delta}(i,\lambda) \to J_{\varepsilon}(j,\nu)$
 at reducible places
 of the principal series representations, 
 which will be investigated 
 in Chapter \ref{sec:pfSBrho}.  



\subsubsection{Main results : Functional equations of $\Atbb \lambda \nu {\varepsilon} {i,j}$}
\label{subsec:fiA}
Suppose $j \in \{i-1, i\}$.  
Let $\Atbb \lambda \nu {\delta\varepsilon} {i,j} \colon I_{\delta}(i,\lambda)
 \to J_{\varepsilon}(j,\nu)$ be the normalized symmetry breaking operators
 as defined in \eqref{eqn:Aijdef}, 
 and $\Ttbb \nu {n-1-\nu} {j} \colon J_{\varepsilon}(j,\nu) \to J_{\varepsilon}(j,n-1-\nu)$
 be the normalized 
\index{B}{KnappSteinoperator@Knapp--Stein operator}
Knapp--Stein operators
 as defined in \eqref{eqn:KSii}
 for principal series representations
 of the subgroup $G'$.  
Then we obtain:
\begin{theorem}
[functional equation]
\label{thm:TAA}
Suppose $(\lambda,\nu) \in {\mathbb{C}}^2$
 and $\gamma \in \{ \pm \}$.  
Then 
\begin{alignat*}{2}
\Ttbb {\nu}{n-1-\nu}{i}
\circ
\Atbb \lambda \nu {\gamma} {i,i}
=&
\frac{\pi^{\frac{n-1}{2}} (\nu-i)}{\Gamma(\nu +1)}
\Atbb \lambda {n-1-\nu} {\gamma} {i, i}
\qquad
&&\text{for $0 \le i \le n-1$, }
\\
\Ttbb{\nu}{n-1-\nu}{i-1}
\circ
\Atbb \lambda \nu {\gamma} {i,i-1}
=&
\frac{\pi^{\frac{n-1}{2}} (n-\nu-i)}{\Gamma(\nu +1)}
\Atbb \lambda {n-1-\nu} {\gamma}{i,i-1}
\qquad
&&\text{for $1 \le i \le n$.  }  
\end{alignat*}
\end{theorem}
In the next theorem,
 we use the same letter $\Ttbb \lambda {n-\lambda} i$
 to denote the normalized Knapp--Stein intertwining operators
 $\Ttbb \lambda {n-\lambda} i \colon I_{\delta}(i,\lambda) \to I_{\delta}(i,n-\lambda)$
 for the group $G$.  
Then we obtain:
\begin{theorem}
[functional equation]
\label{thm:ATA}
\index{B}{functionalequation@functional equation}
Suppose $(\lambda,\nu) \in {\mathbb{C}}^2$
 and $\gamma \in \{ \pm \}$.  
Then 
\index{A}{Tiln@$\Ttbb \lambda{n-\lambda}{i}$|textbf}
\begin{alignat*}{2}
\Atbb {n-\lambda} \nu {\gamma} {i,i}
\circ
\Ttbb {\lambda}{n-\lambda}{i}
=&
\frac{\pi^{\frac{n}{2}} (n-\lambda-i)}{\Gamma(n-\lambda +1)}
\Atbb \lambda {\nu} {\gamma} {i,i}
\qquad
&&\text{for $0 \le i \le n-1$, }
\\
\Atbb {n-\lambda} \nu {\gamma} {i,i-1}
\circ
\Ttbb{\lambda}{n-\lambda}{i}
=&
\frac{\pi^{\frac{n}{2}} (\lambda-i)}{\Gamma(n-\lambda +1)}
\Atbb \lambda {\nu} {\gamma} {i,i-1}
\qquad
&&\text{for $1 \le i \le n$.}
\end{alignat*}
\end{theorem}

\begin{remark}
Theorems \ref{thm:TAA} and \ref{thm:ATA} generalize 
 the functional equations 
 which we proved in the scalar case
 \cite[Thm.~8.5]{sbon}.  
Matrix-valued functional identities
\index{B}{factorizationidentities@factorization identity}
 (factorization identities)
 for {\it{differential}} symmetry breaking operators
 were recently proved explicitly
 in \cite[Chap. 13]{KKP}.  
Alternatively,
 we could deduce a large part of the identities 
\cite[Chap.~13]{KKP} from Theorems \ref{thm:TAA}
 and \ref{thm:ATA}
 by using the residue formula 
 of the normalized symmetry breaking operators
 $\Atbb \lambda \nu {\pm} {i,j}$
 given in Fact \ref{fact:153316}, see \cite{xkresidue}.  
\end{remark}


\subsubsection{Proof of functional equations}
\label{subsec:fi}

In this section
 we give a proof 
 of the functional equations
 that are stated
 in Theorems \ref{thm:TAA} and \ref{thm:ATA}.  



We apply Proposition \ref{prop:TminK}
 on the $K'$-spectrum 
 of the Knapp--Stein intertwining operator
 to the subgroup $G'=O(n,1)$.  
Then the $K'$-spectrum of the (normalized) Knapp--Stein intertwining operator
$
\Ttbb \nu{n-1-\nu}j : J_{\varepsilon}(j,\nu) \to J_{\varepsilon}(j,n-1-\nu)
$
 of $G'$ is given by 
\[
   \Ttbb \nu {n-1-\nu}i \circ \iota_{\nu}^{\ast}
  =
  c^{\natural}(j,\nu)' \iota_{n-\nu}^{\ast}
\qquad
\text{on $\mu^{\natural}(j,\varepsilon)'$}
\]
for $\natural =\flat$ or $\sharp$, 
 where 
\[
  c^{\flat}(j,\nu)'=\frac{(\nu-j) \pi^{\frac {n-1} 2}}{\Gamma(\nu+1)}, 
\qquad
c^{\sharp}(j,\nu)'=\frac{(n-1-j-\nu) \pi^{\frac {n-1} 2}}{\Gamma(\nu+1)}.   
\]



\begin{proof}
[Proof of Theorem \ref{thm:TAA}]
For $j=i$ or $j-1$
 and for $(\lambda, \nu)\in {\mathbb{C}}^2$
 with $\nu-\lambda \not\in {\mathbb{N}}$,
 we recall from Theorem \ref{thm:SBObasis}
 and Corollary \ref{cor:160150upper}
 that 
\[
  \operatorname{Hom}_{G'}(I_+(i,\lambda)|_{G'}, J_{\varepsilon}(j,n-1-\nu))
  =
  {\mathbb{C}} \Atbb \lambda {n-1-\nu}{\varepsilon} {i,j}.  
\]
Hence, 
 there exists a constant
$p_A ^{TA}(i,j,\varepsilon;\lambda, \nu) \in {\mathbb{C}}$
 such that
\begin{equation}
\label{eqn:TApA}
  \Ttbb \nu{n-1-\nu} j
  \circ
  \Atbb \lambda \nu \varepsilon {i,j}
  =
  p_A ^{TA}(i,j,\varepsilon;\lambda, \nu)
  \Atbb \lambda {n-1-\nu}{\varepsilon} {i,j}
\end{equation}
 if $n-1-\nu-\lambda \not\in {\mathbb{N}}$.  
We compute $p_A ^{TA}(i,j,\varepsilon;\lambda, \nu)$
 by using the $(K,K')$-spectrum $S_{\mu,\mu'}$
 (see Section \ref{subsec:Kspec})
 for \eqref{eqn:TApA}
 with an appropriate choice of basic $K$-types $\mu \in \widehat K$ and $\mu' \in \widehat K'$.  
We recall from Theorem \ref{thm:153315}
 an explicit formula
 of the $(K,K')$-spectrum
\[
S(\Atbb \lambda \nu \varepsilon {i,j})
=
\begin{pmatrix}
a_{\varepsilon}^{i,j}(\lambda,\nu)
&
b_{\varepsilon}^{i,j}(\lambda,\nu)
\\
c_{\varepsilon}^{i,j}(\lambda,\nu)
&
d_{\varepsilon}^{i,j}(\lambda,\nu)
\end{pmatrix}
\]
for the regular symmetry breaking operator
$\Atbb \lambda \nu \varepsilon {i,j}:
I_+(i,\lambda) \to J_{\varepsilon}(j,\nu)$
 with respect to basic $K$-types.  

\par\noindent
{\bf{Case 1.}}\enspace
$j=i$ and $\varepsilon=+$.  
Take $(\mu,\mu')=(\mub(i,+), \mub(i,+)')$. 
Then the computation of $S_{\mu,\mu'}$
 on the both sides of \eqref{eqn:TApA}
 leads us to the following identity:
\[
   p_A ^{TA}(i,i,+;\lambda, \nu)
   =
   c^{\flat}(i,\nu)'
   \cdot
   \frac{a_+^{i,i}(\lambda, \nu)}{a_+^{i,i}(\lambda, n-1-\nu)}
   =
   \frac{\pi^{\frac{n-1}{2}}(\nu-j)}{\Gamma(\nu+1)}
   \cdot 1.  
\]
\par\noindent
{\bf{Case 2.}}\enspace
$j=i$ and $\varepsilon=-$.  
Take $(\mu,\mu')=(\mus(i,+), \mub(i,-)')$. 
By the same argument as above,
 we have
\[
   p_A ^{TA}(i,i,-;\lambda, \nu)
   =
   c^{\flat}(i,\nu)'
   \cdot
   \frac{c_-^{i,i}(\lambda, \nu)}{c_-^{i,i}(\lambda, n-1-\nu)}
   =
   \frac{\pi^{\frac{n-1}{2}}(\nu-j)}{\Gamma(\nu+1)}
   \cdot 1.  
\]
\par\noindent
{\bf{Case 3.}}\enspace
$j=i-1$ and $\varepsilon=+$.  
Take $(\mu,\mu')=(\mus(i,+), \mus(i-1,+)')$. 
\[
   p_A ^{TA}(i,i-1,+;\lambda, \nu)
   =
   c^{\sharp}(i-1,\nu)'
   \cdot
   \frac{d_+^{i,i-1}(\lambda, \nu)}{d_+^{i,i-1}(\lambda, n-1-\nu)}
   =
   \frac{\pi^{\frac{n-1}{2}}(n-\nu-i)}{\Gamma(\nu+1)}
   \cdot 1.  
\]
\par\noindent
{\bf{Case 4.}}\enspace
$j=i-1$ and $\varepsilon=-$.  
Take $(\mu,\mu')=(\mub(i,+), \mus(i-1,-)')$. 
\[
   p_A ^{TA}(i,i-1,-;\lambda, \nu)
   =
   c^{\sharp}(i-1,\nu)'
   \cdot
   \frac{b_-^{i,i-1}(\lambda, \nu)}{b_-^{i,i-1}(\lambda, n-1-\nu)}
   =
   \frac{\pi^{\frac{n-1}{2}}(n-\nu-i)}{\Gamma(\nu+1)}
   \cdot 1.  
\]
Since both sides of \eqref{eqn:TApA} depend
 holomorphically in the entire $(\lambda,\nu)\in {\mathbb{C}}^2$, 
 the identity \eqref{eqn:TApA} holds
 for all $(\lambda,\nu)\in {\mathbb{C}}^2$.  
Hence Theorem \ref{thm:TAA} is proved.  
\end{proof}



\begin{proof}
[Proof of Theorem \ref{thm:ATA}]
The proof of Theorem \ref{thm:ATA} goes similarly.  
Since $\Atbb {n-\lambda}{\nu} \varepsilon{i,j} \circ
\Ttbb \lambda{n-\lambda}i$
$\in \operatorname{Hom}_{G'}(I_+(i,\lambda)|_{G'}, J_{\varepsilon}(j,\nu))$, 
 there exists a constant 
\[
   p_A^{AT}(i,j,\varepsilon;\lambda,\nu)\in {\mathbb{C}}
\]
 such that
\begin{equation}
\label{eqn:ATpA}
  \Atbb {n-\lambda} \nu \varepsilon {i,j}
  \circ
  \Ttbb \lambda {n-\lambda} i
  =
  p_A ^{AT}(i,j,\varepsilon;\lambda, \nu)
  \Atbb \lambda \nu \varepsilon {i,j}
\end{equation}
by the generic multiplicity-one theorem 
 (Theorem \ref{thm:genbasis})
for $j \in \{i-1,i\}$, 
 $\varepsilon \in \{\pm\}$, 
 and $(\lambda,\nu) \in {\mathbb{C}}^2$
 with $\nu-\lambda \not \in {\mathbb{N}}$.  
\par\noindent
{\bf{Case 1.}}\enspace
$j=i$ and $\varepsilon=+$.  
Take $(\mu,\mu')=(\mus(i,+), \mus(i,+)')$. 
Applying both sides of \eqref{eqn:ATpA}
 to the basic $K'$-type $\mu'=\mus(i,+)'$
 via the inclusion $\mu' \hookrightarrow \mu = \mus(i,+)$, 
 we get the following identities from 
 Proposition \ref{prop:TminK}
 and Theorem \ref{thm:minKscalar}:

\index{A}{cyfsharp@$c^{\sharp}(i,\lambda)$}
\index{A}{d1ij@$d_+^{i,j}(\lambda, \nu)$}
\[
   p_A ^{AT}(i,i,+;\lambda, \nu)
   =
   c^{\sharp}(i,\lambda)
   \cdot
   \frac{d_+^{i,i}(n-\lambda, \nu)}{d_+^{i,i}(\lambda, \nu)}
   =
   \frac{\pi^{\frac{n}{2}}(n-\lambda-i)}{\Gamma(\lambda+1)}
   \cdot 
   \frac{\Gamma(\lambda+1)}{\Gamma(n-\lambda+1)}.  
\]
The other three cases are proved similarly as below.  
\par\noindent
{\bf{Case 2.}}\enspace
$j=i$ and $\varepsilon=-$.  
Take $(\mu,\mu')=(\mus(i,+), \mub(i,-)')$. 
\index{A}{cz2ii@$c_-^{i,i}(\lambda, \nu)$}
\[
   p_A ^{AT}(i,i,-;\lambda, \nu)
   =
   c^{\sharp}(i,\lambda)
   \cdot
   \frac{c_-^{i,i}(n-\lambda, \nu)}{c_-^{i,i}(\lambda, \nu)}
   =
   \frac{\pi^{\frac{n}{2}}(n-\lambda-i)}{\Gamma(\lambda+1)}
   \cdot 
   \frac{\Gamma(\lambda+1)}{\Gamma(n-\lambda+1)}.  
\]
\par\noindent
{\bf{Case 3.}}\enspace
$j=i-1$ and $\varepsilon=+$.  
Take $(\mu,\mu')=(\mub(i,+), \mub(i-1,+)')$. 
\index{A}{cyflat@$c^{\flat}(i,\lambda)$}
\[
   p_A ^{AT}(i,i-1,+;\lambda, \nu)
   =
   c^{\flat}(i,\lambda)
   \cdot
   \frac{a_+^{i,i-1}(n-\lambda, \nu)}{a_+^{i,i-1}(\lambda, \nu)}
   =
   \frac{\pi^{\frac{n}{2}}(\lambda-i)}{\Gamma(\lambda+1)}
   \cdot 
   \frac{\Gamma(\lambda+1)}{\Gamma(n-\lambda+1)}.  
\]
\par\noindent
{\bf{Case 4.}}\enspace
$j=i-1$ and $\varepsilon=-$.  
Take $(\mu,\mu')=(\mub(i,+), \mus(i-1,-)')$. 
\[
   p_A ^{AT}(i,i-1,-;\lambda, \nu)
   =
   c^{\flat}(i,\lambda)
   \cdot
   \frac{b_-^{i,i-1}(n-\lambda, \nu)}{b_-^{i,i-1}(\lambda, \nu)}
   =
   \frac{\pi^{\frac{n}{2}}(\lambda-i)}{\Gamma(\lambda+1)}
   \cdot 
   \frac{\Gamma(\lambda+1)}{\Gamma(n-\lambda+1)}.  
\]
Thus Theorem \ref{thm:ATA} is proved.  
\end{proof}

%%%%%%%%%%%%%%%%%%%%%%%%%%%%%%%%%%%%%%%%%
\subsection{Renormalized symmetry breaking operator $\Attbb \lambda \nu + {i,j}$}
\label{subsec:161702}
%%%%%%%%%%%%%%%%%%%%%%%%%%%%%%%%%%%%%%%%%%
In Theorem \ref{thm:170340}, 
 we constructed a {\it{renormalized}} symmetry breaking operator
 $\Attbb \lambda \nu \pm {V,W}$
 when the normalized regular symmetry breaking operator $\Atbb \lambda \nu \pm {V,W}$
vanishes.~We apply it 
 to the special case
 $(V,W)=(\Exterior^i({\mathbb{C}}^n), \Exterior^j({\mathbb{C}}^{n-1}))$, 
 and obtain for those $(\lambda,\nu)$
 for which $\Atbb \lambda \nu \gamma {i,j}=0$
 the renormalized symmetry breaking operator
 $\Attbb \lambda \nu \gamma {i,j}$
 as the analytic continuation of the following:
\begin{equation}
\label{eqn:Aijrenorm}
\Attbb \lambda \nu \gamma {i,j}
=
\begin{cases}
\Gamma(\frac{\lambda-\nu}{2})\Atbb \lambda \nu + {i,j}
\quad
&\text{if $\gamma=+$}, 
\\
\Gamma(\frac{\lambda-\nu+1}{2})\Atbb \lambda \nu - {i,j}
\quad
&\text{if $\gamma=-$}.
\end{cases}
\end{equation}  
We recall 
 that for $j \in \{i-1,i\}$
 and $\gamma \in \{\pm\}$, 
 we have determined in Theorem \ref{thm:161243}
 precisely the zero set
\[
   \{
    (\lambda, \nu)\in {\mathbb{C}}^2
    :   
    \Atbb \lambda \nu \gamma {i,j}=0
   \}.  
\]
In this section,
 we discuss functional equations
 and $(K,K')$-spectrum
 of the 
\index{B}{renormalized regular symmetry breaking operator@regular symmetry breaking operator, renormalized---}
renormalized operators $\Attbb \lambda \nu \pm{i,j}$
 only in the few cases
 that are necessary for later arguments.  



\subsubsection{Functional equations
 for the renormalized operator $\Attbb \lambda i + {i,i}$}
In this subsection,
 we treat the case $j=i$.  
For $\nu=i(=j)$, 
 $\Atbb \lambda i \gamma {i,i}=0$
 if and only if $\lambda=i \in \{0,1,\cdots,n-1\}$
 and $\gamma=+$ by Theorem \ref{thm:161243}.  
Then the renormalized operator 
$
   \Attbb \lambda i + {i,i} \colon I_{\delta}(i,\lambda) \to J_{\delta}(i,i)
$
 is the analytic continuation 
 of the following:
\begin{equation}
\label{eqn:Aii2tilde}
   \Attbb \lambda i + {i,i}=\Gamma(\frac{\lambda-i}2)\Atbb \lambda i + {i,i}.  
\end{equation}
Then $\Attbb \lambda i + {i,i}\colon I_{\delta}(i,\lambda) \to J_{\delta}(i,i)$
 is a $G'$-homomorphism
 that depends holomorphically on $\lambda$
 in the entire complex plane ${\mathbb{C}}$
 by Theorem \ref{thm:170340} (3). 



We determine 
\index{B}{functionalequation@functional equation}
 functional equations
 and 
\index{B}{KspectrumKprime@$(K,K')$-spectrum}
$(K,K')$-spectrum 
 $S(\Attbb \lambda i + {i,i})$
 (see \eqref{eqn:Smat})
 on basic $K$- and $K'$-types
 for the renormalized operator
\index{A}{Ahttsln1@$\Attbb \lambda \nu {+} {i,j}$}
 $\Attbb \lambda i + {i,i}$ as follows.  



\begin{lemma}
[functional equations and the $(K,K')$-spectrum
 for $\Attbb {n-\lambda} i + {i,i}$]
\label{lem:Aiiren}
~~~\newline
Suppose $0 \le i \le n-1$
 and $\lambda \in {\mathbb{C}}$.  
Then we have 
\begin{align}
\label{eqn:TA2tilde}
\Ttbb i {n-1-i} i \circ \Attbb {\lambda} i + {i,i} 
=&
0, 
\\
\label{eqn:ATAi2}
\Attbb {n-\lambda} i + {i,i} 
\,\,\,
\circ \Ttbb \lambda {n-\lambda} {i}
=&
\frac{2 \pi^{\frac n 2}\Gamma(\frac {n-\lambda-i}2+1)}
     {\Gamma(\frac{\lambda-i}2) \Gamma(n-\lambda+1)} \Attbb \lambda i + {i,i}, 
\\
\label{eqn:SAiitilde}
S(\Attbb \lambda i + {i,i}) \,=& \frac{\pi^{\frac {n-1} 2}}{\Gamma(\lambda+1)}
\begin{pmatrix} 2 & 0 \\ 0 & 0 \end{pmatrix}.  
\end{align}
\end{lemma}

\begin{proof}
Applying Theorem \ref{thm:TAA}
 with $\nu =i$ ($0 \le i \le n$), 
 we have
\[
   \Ttbb i {n-1-i} i \circ \Atbb {\lambda} i + {i,i} 
   =0 
\quad \text{for all $\lambda \in {\mathbb{C}}$.  }
\]
Taking the limit as $\lambda$ tends to $i$
 in the following equation:
\[
   \Ttbb i {n-1-i} i \circ 
   \Gamma(\frac{\lambda-i}{2})
   \Atbb {\lambda} i + {i,i} =0, 
\]
we get the desired formula \eqref{eqn:TA2tilde}
 by the definition \eqref{eqn:Aii2tilde} of the renormalization 
 $\Attbb {\lambda} i + {i,i}$.  



Similarly,
 the formul{\ae} \eqref{eqn:ATAi2} and \eqref{eqn:SAiitilde}
 for the renormalized operator
 $\Attbb {\lambda} i + {i,i}$
 follow from the limit of the corresponding results
 for $\Atbb {\lambda} i + {i,i}$ given in Theorems \ref{thm:ATA}
 and \ref{thm:153315}, 
 respectively.  
\end{proof}

\subsubsection{Functional equations at middle degree for $n$ even}

For $n$ even (say, $n=2m$), 
 at the \lq\lq{middle degree}\rq\rq\ $i=\frac n 2( =m$), 
 we observe
 that the Knapp--Stein operator
$\Ttbb \lambda {2m-\lambda}{m} \colon I_+(m,\lambda) \to I_+(m,2m-\lambda)$
 vanishes 
 if $\lambda =m$
 (see Proposition \ref{prop:Tvanish}), 
 and so the functional equation \eqref{eqn:ATAi2} is trivial.  
Instead we use the
\index{B}{KnappSteinoperatorrenorm@ Knapp--Stein operator, renormalized---}
 renormalized  Knapp--Stein operator
\index{A}{TVttiln@$\Tttbb \lambda{n-\lambda}{\frac n 2}$, renormalized Knapp--Stein intertwining operator}
 $\Tttbb \lambda {2m-\lambda} {m}$
 defined in  \eqref{eqn:Ttilde}
 for another functional equation, 
 see \eqref{eqn:AT2tilde} below.  
We recall from Lemma \ref{lem:161745}
 that $\Tttbb \lambda{2m-\lambda}m$ is an endomorphism
 of $I_{\delta}(m,m)$ 
 when $\lambda=m$, 
 but is not proportional
 to the identity operator
 when $\lambda=m$.  

\begin{lemma}
[functional equation for $\Attbb m m + {i,i}$]
\label{lem:AmmT}
\index{B}{functionalequation@functional equation}
Let $(G,G')=(O(n+1,1),O(n,1))$ with $n=2m$.  
Then we have
\begin{equation}
\label{eqn:AT2tilde}
\Attbb {m}{m} + {m, m}
\circ
\Tttbb m m m
=
\frac{\pi^m}{m!} 
\Attbb {m}{m} + {m, m}.  
\end{equation}
\end{lemma}



\begin{proof}
By Theorem \ref{thm:170340} and \eqref{eqn:Ttilde}, 
\begin{align}
\Attbb {m}{m} + {m, m}
\circ
\Tttbb {m}{m} {m}
=&
(\lim_{\lambda \to m}
  \Gamma(\frac{(2m-\lambda)-m}{2})
  \Atbb {2m-\lambda}{m} + {m, m})
\circ
(\lim_{\lambda \to m}
  \frac{1}{\lambda - m}
   \Ttbb {\lambda}{2m-\lambda} {m})
\notag
\\
=&
\lim_{\lambda \to m}
  \frac{\Gamma(\frac{m-\lambda}{2})}
       {\lambda-m}
  \Atbb {2m-\lambda}{m} + {m, m}
  \circ
  \Ttbb {\lambda}{2m-\lambda} {m}.  
\notag
\end{align}
In turn, the functional equation
 in Theorem \ref{thm:ATA} shows that 
 the right-hand side amounts to
\begin{align}
\lim_{\lambda \to m}
  \frac{\Gamma(\frac{m-\lambda}{2})}
       {\lambda-m}
  \frac{\pi^{m} (2m-\lambda-m)}{\Gamma(2m-\lambda+1)}
  \Atbb {\lambda}{m} + {m, m}
=&
 \frac{-\pi^{m}}{\Gamma(m+1)}
 (\lim_{\lambda \to m} 
  \frac{\Gamma(\frac{m - \lambda}{2})}{\Gamma(\frac{\lambda-m}{2})})\Attbb {m}{m} + {m, m}
\notag
\\
=&
 \frac{\pi^{m}}{\Gamma(m+1)}
 \Attbb {m}{m} + {m, m}.  
\end{align}
Hence the formula \eqref{eqn:AT2tilde} is proved.  
\end{proof}

In contrast to Lemma \ref{lem:AmmT}
 where we needed to treat the renormalized operator $\Attbb {m}{m} + {m, m}$
 because $\Atbb {m}{m} + {m, m}=0$, 
 the normalized operator $\Atbb {m}{m} - {m, m}$ does not vanish
 (Theorem \ref{thm:161243} (3)).  
In this case,
 the functional equations for $\Atbb {m}{m} - {m, m}$ are given as follows:

\begin{lemma}
[functional equation for $\Atbb m m - {m, m}$]
\label{lem:AmmT-}
We retain the setting that $(G,G')=(O(n+1,1),O(n,1))$
 with $n=2m$.  
Then we have 
\begin{align}
\label{eqn:ATAm-}
\Atbb m m - {m,m}\circ \Tttbb m m m = & - \frac {\pi^m}{m!}\Atbb m m - {m,m}, 
\\
\label{eqn:TATm-}
\Ttbb m {m-1} m  \circ \Atbb m m - {m,m} = & 0.  
\end{align}
\end{lemma}

\begin{proof}
By the definition of $\Tttbb \lambda{2m-\lambda}{m,m}$
 in \eqref{eqn:Ttilde}
 and the functional equation in Theorem \ref{thm:ATA}, 
 we have
\begin{align*}
\Atbb m m - {m,m}\circ \Tttbb m m m 
= & \lim_{\lambda \to m} \Atbb {2m-\lambda} m - {m,m} 
    \circ \frac {1}{\lambda-m} \Ttbb \lambda {2m-\lambda} m
\\
= & \lim_{\lambda \to m} \frac {\pi^m (m-\lambda)}{(\lambda-m)\Gamma(2m-\lambda+1)}
  \Atbb {\lambda} m - {m,m} 
\\ 
=&\frac {-\pi^m}{m!} \Atbb m m - {m,m}.  
\end{align*}
Hence the first statement is verified.  
The second statement is a special case
 of Theorem \ref{thm:TAA}.  
\end{proof}



\subsubsection{Functional equations for the renormalized operator
 $\Attbb \lambda {n-i} + {i,i-1}$}
In this subsection, 
 we treat the case $j=i-1$.  
For $j=i-1$ and $\nu=n-i$, 
 $\Atbb \lambda {n-i} \gamma {i,j} =0$
 if and only if $\gamma=+$ and $\lambda =n-i$
 by Theorem \ref{thm:161243}.  
In this case, 
 the renormalized symmetry breaking operator
 $\Attbb \lambda {n-i} + {i,i-1} \colon I_{\delta}(i,\lambda) \to J_{\delta}(i-1, n-i)$ is obtained 
 as the analytic continuation of the following:
\index{A}{Ahttsln1@$\Attbb \lambda \nu {+} {i,j}$}
\[
   \Attbb \lambda {n-i} + {i,i-1}=\Gamma(\frac{\lambda-n+i}2)\Atbb \lambda {n-i} + {i,i-1}, 
\]
see Theorem \ref{thm:170340} (3).  


We determine functional equations
 and $(K,K')$-spectrum $S(\Attbb \lambda {n-i}+{i,i-1})$
 (see \eqref{eqn:Smat})
 on basic $K$- and $K'$-types
 for the renormalized operator $\Attbb \lambda {n-i}+{i,i-1}$
 as follows.  
 
\begin{lemma}
[functional equations and the $(K,K')$-spectrum
 for $\Attbb \lambda {n-i}+{i,i-1}$]
\label{lem:Aii-ren}
~~~\newline
Suppose $1 \le i \le n$ and $\lambda \in {\mathbb{C}}$.  
Then we have 
\begin{align}
\label{eqn:1802119}
\Attbb {n-\lambda} {n-i} + {i,i-1} \circ \Ttbb \lambda {n-\lambda} {i}
=&
-\frac{2\pi^{\frac n 2}\Gamma(\frac{i-\lambda}2+1)}{\Gamma(n-\lambda+1)\Gamma(\frac{\lambda-n+i}{2})}
 \Attbb \lambda {n-i} + {i,i-1}, 
\\
\label{eqn:180294}
\Ttbb {n-i}{i-1} {i-1}\circ
\Attbb {\lambda} {n-i} + {i,i-1} 
=&
0, 
\\
S(\Attbb \lambda {n-i} + {i,i-1}) =& \frac{\pi^{\frac {n-1} 2}}{\Gamma(\lambda+1)}
\begin{pmatrix} 0 & 0 \\ 0 & 2 \end{pmatrix}.  
\notag
\end{align}
\end{lemma}

\begin{proof}
The functional equations follow from Theorems \ref{thm:TAA} and \ref{thm:ATA}.  The formula for the $(K,K')$-spectrum
 is derived from Theorem \ref{thm:153315}.  
\end{proof}


\subsubsection{Functional equations at middle degree for $n$ odd}
For $n$ odd 
 (say, $n=2m+1$), 
 the Knapp--Stein operator $\Ttbb \nu {n-\nu-1} j \colon J_{\varepsilon}(j,\nu) \to J_{\varepsilon}(j,n-1-\nu)$ 
 for the subgroup $G'=O(n,1)$ vanishes
 at the middle degree $j=\frac 1 2 (n-1) (=m)$ 
 if $\nu=m$ by Proposition \ref{prop:Tvanish}.  
We note that the exact sequence in Theorem \ref{thm:LNM20} (1)
 for $G'=O(2m+1,1)$ splits, 
 and we have a direct sum decomposition 
\[
  J_{\varepsilon}(m,m) \simeq \pi_{m,\varepsilon} \oplus \pi_{m+1,-\varepsilon}
\]
 of two irreducible tempered representations of $G'$.  
In this case,
 the functional equations \eqref{eqn:TA2tilde}
 in Lemma \ref{lem:Aiiren}
 and \eqref{eqn:180294} in Lemma \ref{lem:Aii-ren} are trivial,
 and we replace them
 by the following functional equations
 for the {\it{renormalized}} Knapp--Stein operator
\index{A}{TVttiln@$\Tttbb \lambda{n-\lambda}{\frac n 2}$, renormalized Knapp--Stein intertwining operator}
 $\Tttbb m m m$.  

\begin{lemma}
\label{lem:TAT3tilde}
For $(G,G')=(O(n+1,1),O(n,1))$ with $n=2m+1$
 and for $\lambda \in {\mathbb{C}}$, 
 we have
\begin{align}
\label{eqn:TAA3tilde}
   \Tttbb {m}{m} {m} \circ \Attbb {\lambda}{m} + {m, m}
   =&
   \frac{\pi^m}{m!} \Attbb {\lambda}{m} + {m, m}, 
\\
\label{eqn:180293}
   \Tttbb {m}{m} {m} \circ \Attbb {\lambda}{m} + {m+1, m}
   =&
   -\frac{\pi^m}{m!} \Attbb {\lambda}{m} + {m+1, m}.  
\end{align}
\end{lemma}

Lemma \ref{lem:TAT3tilde} tells that 
\begin{alignat*}{3}
&{\operatorname{Image}}
(\Attbb {\lambda}{m} + {m, m}\colon I_{\delta}(m,\lambda) \to J_{\delta}(m,m)) 
&&\subset \pi_{m,\delta}, 
\\
&{\operatorname{Image}}
(\Attbb {\lambda}{m} + {m+1, m}\colon I_{\delta}(m+1,\lambda) 
\to J_{\delta}(m,m)) 
&&\subset \pi_{m+1,-\delta}, 
\end{alignat*}
for all $\lambda \in {\mathbb{C}}$
by Lemma \ref{lem:161745}.  


\begin{proof}
The functional equations in Theorem \ref{thm:TAA} tell
 that 
\begin{align*}
(\frac{1}{\nu-m} \Ttbb \nu {2m-\nu} m) 
\circ 
\Gamma(\frac{\lambda-m}{2}) \Atbb \lambda \nu + {m,m} 
=& 
\frac{\pi^m}{\Gamma(\nu+1)} \Gamma(\frac{\lambda-m}{2}) 
\Atbb \lambda {2m-\nu} + {m,m}, 
\\
(\frac{1}{\nu-m} \Ttbb \nu {2m-\nu} m) 
\circ 
\Gamma(\frac{\lambda-m}{2}) \Atbb \lambda \nu + {m+1,m} 
=& 
-\frac{\pi^m}{\Gamma(\nu+1)} \Gamma(\frac{\lambda-m}{2}) 
\Atbb \lambda {2m-\nu} + {m+1,m}.  
\end{align*}
Taking the limit as $\nu$ tends to $m$, 
 we get Lemma \ref{lem:TAT3tilde}.  
\end{proof}

\subsection{Restriction map $I_{\delta}(i,\lambda) \to J_{\delta}(i,\lambda)$}
\label{subsec:Rest}
The restriction of (smooth) differential forms
 to a submanifold defines an obvious continuous map
 between Fr{\'e}chet spaces,
 which intertwines the conformal representation
 (see \cite[Lem.~8.9]{KKP}).  
We end this section with the most elementary symmetry breaking operator,
 namely,
 the restriction map
 for the pair $(G'/P', G/P) \subset (S^{n-1},S^n)$.  
\begin{lemma}
\label{lem:152268}
The restriction map from $G/P$ to the submanifold $G'/P'$
 induces obvious symmetry breaking operators
\[
{\operatorname{Rest}}_{\lambda,\lambda,+}^{i,i}
:
I_{\delta}(i,\lambda) \to J_{\delta}(i,\lambda).  
\]
Then the $(K,K')$-spectrum
 for basic $K$- and $K'$-types (see \eqref{eqn:Smat})
 is given by
\begin{equation}
\label{eqn:SRest}
  S({\operatorname{Rest}}_{\lambda,\lambda,+}^{i,i})
  =
  \begin{pmatrix} 1 & 0 \\ 0 & 1\end{pmatrix}.
\end{equation}
\end{lemma}
\begin{proof}
We recall from Proposition \ref{prop:minKN}
 that 
\[
   \{ {\bf{1}}_{\lambda}^{\mathcal{I}}
    :
   {\mathcal{I}} \in {\mathfrak {I}}_{n+1,i}
   \}
\]
forms a basis of the basic $K$-type $\mub(i,\delta)$
 of the principal series representation $I_{\delta}(i,\lambda)$
 in the $N$-picture.  
Let $\mathcal{I} \in {\mathfrak {I}}_{n+1,i}$
 and $(x,x_n) \in {\mathbb{R}}^{n-1} \oplus {\mathbb{R}}={\mathbb{R}}^n$.  
By \eqref{eqn:minI}, 
 we have
\[
{\bf{1}}_{\lambda}^{\mathcal{I}}(x,x_n)
=-(1+|x|^2+x_n^2)^{-\lambda-1}
\sum_{J \in {\mathfrak{I}}_{n,i}} S_{{\mathcal{I}}J}(1,x,x_n)e_J.  
\]
Then an elementary computation by using \eqref{eqn:SIJ} shows
\begin{equation*}
{\bf{1}}_{\lambda}^{\mathcal{I}}(x,0)
=
\begin{cases}
{{\bf{1}}_{\lambda}'}^{\mathcal{I}}(x)
\qquad
&\text{if }n \notin {\mathcal{I}} \in {\mathfrak {I}}_{n+1,i}, 
\\
{{\bf{1}}_{\lambda}'}^{\mathcal{I}\setminus \{n\}}(x) \wedge e_n
\qquad
&\text{if }n \in {\mathcal{I}} \in {\mathfrak {I}}_{n+1,i}.  
\end{cases}
\end{equation*}
The case for the basic $K$-type
 $\mus(i,\delta)$ is similar, 
 where we recall from Proposition \ref{prop:minKN}
 that $\{h_{\lambda}^{\mathcal{I}}: {\mathcal{I}} \in {\mathfrak {I}}_{n+1,i+1}\}$ forms its basis, 
 for which we can compute the restriction $x_n=0$.  
Thus Lemma \ref{lem:152268} is shown.  
\end{proof}

\subsection{Image of the differential symmetry breaking operator
 $\Ctbb \lambda \nu {i,j}$}
\label{subsec:ImageC}


In Theorem \ref{thm:imgDSBO}, 
 we have proved 
 that the image of any nonzero differential symmetry breaking operator from
 principal series representation
 is infinite-dimensional.  
As an application of the functional equations
 of the (generically) regular symmetry breaking operators
 $\Atbb \lambda \nu \pm {i,j}$
 (Theorems \ref{thm:TAA} and \ref{thm:ATA})
 and of the residue formul{\ae}
 of $\Atbb \lambda \nu \pm {i,j}$
 (Fact \ref{fact:153316},  see \cite{xkresidue}), 
 we end this chapter
 with a necessary and sufficient condition
 for the renormalized differential symmetry breaking operator
 $\Ctbb \lambda \nu {i,j}$ to be surjective
 when $j=i$, $i-1$, 
see Theorems \ref{thm:180380} and \ref{thm:180386}.  


\subsubsection{Surjectivity condition of $\Ctbb \lambda \nu {i,j}$}
Suppose $j \in \{i,i-1\}$.  
We recall from \eqref{eqn:Ciitilde} and \eqref{eqn:Cii-1tilde}
 that the renormalized differential symmetry 
 breaking operator 
$\Ctbb \lambda \nu {i,j}\colon I_{\delta}(i,\lambda)
 \to J_{\varepsilon}(j,\nu)$
 is defined 
 for $(\lambda, \nu) \in {\mathbb{C}}^2$
 with $\nu-\lambda \in {\mathbb{N}}$
 and $\delta \varepsilon =(-1)^{\nu-\lambda}$.  
Moreover, 
 $\Ctbb \lambda \nu {i,j}$ is nonzero
 for any $(i,j, \lambda,\nu)$
 with $j \in \{i,i-1\}$
 and $\nu-\lambda \in {\mathbb{N}}$.  



In what follows, 
 we shall sometimes encounter the condition that 
\index{A}{Leven@$L_{\operatorname{even}}$|textbf}
\index{A}{Lodd@$L_{\operatorname{odd}}$|textbf}
$(\lambda,n-1-\nu) \in L_{\operatorname{even}} \cup L_{\operatorname{odd}}$, 
 which is equivalent to 
\begin{equation}
\label{eqn:he}
(\lambda,\nu) \in {\mathbb{Z}}^2
\quad
\text{and}
\quad
\lambda+\nu \le n-1 \le \nu.  
\end{equation}
\begin{theorem}
\label{thm:180380}
Suppose $0 \le i \le n-1$, 
 $\nu-\lambda \in {\mathbb{N}}$, 
 and $\delta$, $\varepsilon \in \{\pm\}$
 with $(-1)^{\nu-\lambda} = \delta \varepsilon$.  
Then the following two conditions (i) and (ii) on $(i, \lambda,\nu)$
 are equivalent:
\begin{enumerate}
\item[{\rm{(i)}}]
\index{A}{Ctiiln@$\Ctbb \lambda \nu {i,i}$}
$
   \Ctbb \lambda \nu {i,i} \colon I_{\delta}(i,\lambda) \to J_{\varepsilon}(i,\nu)
$
 is not surjective.  
\item[{\rm{(ii)}}]
One of the following conditions holds:
\begin{enumerate}
\item[{\rm{(ii-a)}}]
$1 \le i \le n-1$, 
$\nu=i$, 
 and ${\mathbb{Z}} \ni \lambda < i;$
\item[{\rm{(ii-b)}}]
$n$ is odd, 
 $i=0$, 
 and \eqref{eqn:he}$;$
\item[{\rm{(ii-c)}}]
$n$ is odd,
 $1 \le i \le n-1$, 
 \eqref{eqn:he}, 
 and $\nu \ne n-1;$
\item[{\rm{(ii-d)}}]
$n$ is odd,
 $1 \le i < \frac 1 2 (n-1)$, 
 $(\lambda, \nu)=(i,n-1-i)$.  
\end{enumerate}
\end{enumerate}
\end{theorem}

\begin{theorem}
\label{thm:180386}
Suppose $1 \le i \le n$, 
 $\nu-\lambda \in {\mathbb{N}}$, 
 and $\delta$, $\varepsilon \in \{\pm\}$
 with $(-1)^{\nu-\lambda} = \delta \varepsilon$.  
Then the following two conditions (i) and (ii)
 on $(i, \lambda,\nu)$ are equivalent:
\begin{enumerate}
\item[{\rm{(i)}}]
\index{A}{Ctiim1ln@$\Ctbb \lambda \nu {i,i-1}$}
$
   \Ctbb \lambda \nu {i,i-1} \colon I_{\delta}(i,\lambda) \to J_{\varepsilon}(i-1,\nu)
$
 is not surjective.  
\item[{\rm{(ii)}}]
One of the following conditions holds:
\begin{enumerate}
\item[{\rm{(ii-a)}}]
$1 \le i \le n-1$, $\nu=n-i$, and ${\mathbb{Z}} \ni \lambda<n-i;$
\item[{\rm{(ii-b)}}]
$n$ is odd,
 $1 \le i \le n-1$, 
 \eqref{eqn:he}, 
 and $\nu \ne n-1;$
\item[{\rm{(ii-c)}}]
$n$ is odd,
 $i=n$, 
 and \eqref{eqn:he}$;$
\item[{\rm{(ii-d)}}]
$n$ is odd,
 $\frac 1 2 (n+1) < i \le n-1$, 
 and $(\lambda,\nu)=(n-i,i-1)$.  
\end{enumerate}
\end{enumerate}
\end{theorem}

For the proof of Theorems \ref{thm:180380} and \ref{thm:180386}, 
 we first derive
 the functional equations for $\Ctbb \lambda \nu {i,j}$
 in Theorem \ref{thm:180389} from those
 for the regular symmetry breaking operators
 $\Atbb \lambda \nu \pm{i,j}$
 in Chapter \ref{sec:holo}
 and from the matrix-valued residue formul{\ae} \cite{xkresidue}.  
The results cover most of the cases
 where the Knapp--Stein intertwining operators
 $\Ttbb \nu {n-1-\nu}j$ do not vanish.  
A special attention is required
 when $\Ttbb \nu {n-1-\nu}j=0$.  
In this case,
 the principal series representation $J_{\varepsilon}(j,\nu)$ splits into the direct sum
 of two irreducible representations
 of the subgroup $G'=O(n,1)$, 
 and we shall treat this case separately 
 in Section \ref{subsec:10.6.3}.  
The proof of Theorems \ref{thm:180380} and \ref{thm:180386} will be completed 
 in Section \ref{subsec:10.6.4}.  

\subsubsection{Functional equation for $\Ctbb \lambda \nu {i,j}$}
Suppose $0 \le i \le n$, 
 $0 \le j \le n-1$,
 and $j=i$ or $i-1$.  
We set $p^{i,j}(\lambda,\nu)$ by 
\begin{align}
\label{eqn:pijlmdnu}
p^{i,i}(\lambda,\nu)
&:=
\begin{cases}
1 
&\text{if $i=0$ or $\lambda = \nu$}, 
\\
\frac 1 2(\nu-i)
\qquad\quad
&\text{otherwise}, 
\end{cases}
\\
\notag
p^{i,i-1}(\lambda,\nu)
&:=
\begin{cases}
1 
\quad
&\text{if $i=n$ or $\lambda = \nu$}, 
\\
\frac 1 2(\nu+i-n)
\quad
&\text{otherwise}.  
\end{cases}
\end{align}

\begin{theorem}
[functional equation for $\Ctbb \lambda \nu {i,j}$]
\label{thm:180389}
Suppose $0 \le i \le n$, 
 $0 \le j \le n-1$, 
 and $j \in \{i,i-1\}$.  
For $(\lambda, \nu) \in {\mathbb{C}}^2$
 with $\nu-\lambda \in {\mathbb{N}}$, 
 we have
\begin{equation}
\label{eqn:180389}
  \Ttbb \nu {n-1-\nu} {j} \circ \Ctbb \lambda \nu {i,j}
=
  q(\nu-\lambda) p^{i,j}(\lambda,\nu) \Atbb \lambda {n-1-\nu} {(-1)^{\nu-\lambda}}{i,j}, 
\end{equation}
 where 
\index{A}{qm@$q(m)$}
 $q(m)$ is a nonzero number defined in \eqref{eqn:qm}.  
\end{theorem}

\begin{proof}
We set
\begin{equation}
\label{eqn:qijnu}
p_{i,j}(\nu)
:=
\begin{cases}
\frac 12(\nu-i) 
\quad
&\text{if $j=i$}, 
\\
\frac 1 2(n-\nu-i)
\quad
&\text{if $j=i-1$}.  
\end{cases}
\end{equation}
By the functional equation 
 for the regular symmetry breaking operator
$\Atbb \lambda \nu \pm {i,j}$ 
given in Theorem \ref{thm:TAA}, 
 we have for $\gamma \in \{\pm\}$
\[
  \Ttbb \nu{n-1-\nu} j \circ \Atbb \lambda \nu \gamma {i,j}
= \frac{2 \pi^{\frac{n-1}{2}} p_{i,j}(\nu)}{\Gamma(\nu+1)}
 \Atbb \lambda {n-1-\nu} \gamma {i,j}.  
\]
Suppose $\nu-\lambda \in {\mathbb{N}}$.  
Applying the residue formula \eqref{eqn:qAC}
 of $\Atbb \lambda \nu {\pm}{i,j}$ to the left-hand side,
 we get 
\begin{equation}
\label{eqn:TCqpAij}
  \Ttbb \nu{n-1-\nu} j \circ \Cbb \lambda \nu {i,j}
= (-1)^{i-j} q(\nu-\lambda) p_{i,j}(\nu)
  \Atbb \lambda {n-1-\nu} {(-1)^{\nu-\lambda}}{i,j}.  
\end{equation}
On the other hand, 
 by using $p^{i,j}(\lambda,\nu)$ and $p_{i,j}(\nu)$, 
 the relation between the unnormalized operators
 $\Cbb \lambda \nu{i,j}$ and 
 the renormalized operators $\Ctbb \lambda \nu {i,j}$
 defined in \eqref{eqn:Ciitilde} and \eqref{eqn:Cii-1tilde}
 are given as the following unified formula:
\[
  p^{i,j}(\lambda,\nu) \Cbb \lambda \nu {i,j} 
 =
  (-1)^{i-j} p_{i,j}(\nu) \Ctbb \lambda \nu {i,j}
  \quad
  \text{for $j \in \{i,i-1\}$.}
\]
Multiplying both sides of the equation \eqref{eqn:TCqpAij}
 by $p^{i,j}(\lambda,\nu)$, 
 we get the desired formula.  
\end{proof}



\begin{proposition}
\label{prop:10.34}
Suppose $\nu-\lambda \in {\mathbb{N}}$, 
 and $\delta$, $\varepsilon \in \{\pm\}$
 with $(-1)^{\nu-\lambda} = \delta \varepsilon$.  
\begin{enumerate}
\item[{\rm{(1)}}]
Suppose $0 \le i \le n-1$.  
Then the following two conditions on $(i, \lambda,\nu)$ are equivalent:
\begin{enumerate}
\item[{\rm{(i)}}]
The image of 
$\hphantom{i}
   \Ctbb \lambda \nu {i,i} \colon I_{\delta}(i,\lambda) \to J_{\varepsilon}(i,\nu)
$
 is contained in ${\operatorname{Ker}}\,(\Ttbb \nu {n-1-\nu}i)$.  
\item[{\rm{(ii)}}]
One of the following conditions holds:
\begin{enumerate}
\item[{\rm{(ii-a)}}]
$1 \le i \le n-1$, $\nu=i$, and ${\mathbb{Z}} \ni \lambda < i;$
\item[{\rm{(ii-b)}}]
$n$ is odd, $i=0$, and \eqref{eqn:he}$;$
\item[{\rm{(ii-c)}}]
$n$ is odd, $1 \le i \le n-1$, \eqref{eqn:he}, 
 and $\nu \ne n-1;$
\item[{\rm{(ii-d)}}]
$n$ is odd, $1 \le i \le \frac 1 2(n-1)$, 
 and $(\lambda,\nu) = (i,n-1-i)$.  
\end{enumerate}
\end{enumerate}
\item[{\rm{(2)}}]
Suppose $1 \le i \le n$.  
Then the following two conditions on $(i, \lambda,\nu)$ are equivalent:
\begin{enumerate}
\item[{\rm{(iii)}}]
The image of 
$
   \Ctbb \lambda \nu {i,i-1} \colon I_{\delta}(i,\lambda) \to J_{\varepsilon}(i-1,\nu)
$
 is contained in ${\operatorname{Ker}}\,(\Ttbb \nu {n-1-\nu}{i-1})$.  
\item[{\rm{(iv)}}]
One of the following holds:
\begin{enumerate}
\item[{\rm{(iv-a)}}]
$1 \le i \le n-1$, $\nu=n-i$, and ${\mathbb{Z}} \ni \lambda < n-i;$
\item[{\rm{(iv-b)}}]
$n$ is odd, $1 \le i \le n-1$, \eqref{eqn:he}, 
 and $\nu \ne n-1;$
\item[{\rm{(iv-c)}}]
$n$ is odd, $i=n$, and \eqref{eqn:he}$;$
\item[{\rm{(iv-d)}}]
$n$ is odd, $\frac 12 (n+1) \le i \le n-1$, 
 and $(\lambda,\nu) = (n-i,i-1)$.  
\end{enumerate}
\end{enumerate}
\end{enumerate}
\end{proposition}
The difference of this proposition from Theorems \ref{thm:180380} and \ref{thm:180386}
 is that the cases $i=\frac 1 2 (n-1)$ in (1)
 and $i=\frac 1 2 (n+1)$ in (2)
 are included in Proposition \ref{prop:10.34}.  
In these cases,
 the Knapp--Stein intertwining operator $\Ttbb \nu {n-1-\nu} j$
 vanishes 
 where $j=i$ in (1)
 and $=i-1$ in (2), 
 and the conditions (i) and (iii) do not provide 
 any information of $\operatorname{Image}(\Ctbb \lambda \nu{i,j})$.  
In these special cases,
 we shall study $\operatorname{Image}(\Ctbb \lambda \nu{i,j})$
 separately
 in Section \ref{subsec:10.6.3}
 by using the renormalized Knapp--Stein operators
 $\Tttbb \nu{n-1-\nu} j$.  

\begin{proof}
By the functional equation \eqref{eqn:180389} in Theorem \ref{thm:180389}, 
 we see that 
\[
 {\operatorname{Image}}\,(\Ctbb \lambda \nu {i,j})
 \subset {\operatorname{Ker}}\,(\Ttbb \nu {n-1-\nu}j)
\]
 if and only if $p^{i,j}(\lambda,\nu)=0$
 or $\Atbb \lambda {n-1-\nu}{(-1)^{\nu-\lambda}}{i,j}=0$.  
Suppose $0 \le i \le n$, $0 \le j \le n-1$, 
 and $j \in \{i,i-1\}$.  
By definition \eqref{eqn:pijlmdnu}, 
\begin{align*}
 p^{i,i}(\lambda,\nu) = 0 &\Leftrightarrow \lambda \ne \nu =i
\quad
\text{and}
\quad
1 \le i \le n-1,
\\
 p^{i,i-1}(\lambda,\nu) = 0 &\Leftrightarrow 
\lambda \ne \nu =n-i
\quad
\text{and}
\quad
1 \le i \le n-1.  
\end{align*}
On the other hand, 
 we claim 
\begin{align*}
\Atbb \lambda{n-1-\nu}{(-1)^{\nu-\lambda}}{i,i}=0
&\Leftrightarrow
\text{(ii-b), (ii-c), or (ii-d) holds};
\\
\Atbb \lambda{n-1-\nu}{(-1)^{\nu-\lambda}}{i,i-1}=0
&\Leftrightarrow
\text{(iv-b), (iv-c), or (iv-d) holds.}
\end{align*}
Let us verify the first equivalence 
 for $1 \le i \le n-1$.  
The vanishing condition 
of $\Atbb \lambda {\nu}{\pm}{i,j}$
 given in Theorem \ref{thm:161243} (1) and (3) shows that
 $\Atbb \lambda{n-1-\nu}{(-1)^{\nu-\lambda}}{i,i}=0$
 if and only if one of the following three conditions holds:
\begin{enumerate}
\item[$\bullet$]
$i=0$,
 $(\lambda,n-1-\nu) \in {\mathbb{Z}}^2$,
 $(\nu+1-n)-\lambda \equiv \nu-\lambda \mod 2$, 
 and $0 \le \nu+1-n \le -\lambda$;
\item[$\bullet$]
$i\ne 0$, $(\lambda,n-1-\nu) \in {\mathbb{Z}}^2$,
 $(\nu+1-n)-\lambda \equiv \nu-\lambda \mod 2$, 
 and $0 < \nu+1-n \le -\lambda$;
\item[$\bullet$]
$i\ne 0$, $\nu-\lambda \in 2{\mathbb{Z}}$, 
 and $(\lambda,n-1-\nu) =(i,i)$.  
\end{enumerate}
These conditions amount to (ii-b), (ii-c), and (ii-d)
 in Proposition \ref{prop:10.34} (1), 
 respectively.  
The second equivalence is shown similarly.  
Hence Proposition \ref{prop:10.34} is proved.  
\end{proof}
\begin{remark}
For $\lambda=\nu$, 
 the above conditions are fulfilled 
 if and only if $(\lambda,\nu)=(\frac 1 2 (n-1),\frac 1 2 (n-1))$
 and $i=\frac 1 2 (n-1)$
 in Proposition \ref{prop:10.34} (1) 
 or $i=\frac 1 2 (n+1)$ in Proposition \ref{prop:10.34} (2).  
This is exactly when $\Ttbb \nu {n-1-\nu}j$
 ($j=i$, $i-1$) vanishes.  
\end{remark}

\subsubsection{The case when $\Ttbb \nu {n-1-\nu}j=0$}
\label{subsec:10.6.3}

By Proposition \ref{prop:Tvanish}, 
 the Knapp--Stein operators
 $\Ttbb \nu {n-1-\nu} {j}$
 for the subgroup $G'=O(n,1)$
 vanishes if and only if $n$ is odd and 
\[
\nu=j=\frac{n-1}2.
\]
We note in this case that 
 $\nu-i=0$ for $i=j$
 and $\nu+i-n=0$ for $i=j+1$, 
 and therefore the definition \eqref{eqn:pijlmdnu} tells
\begin{equation*}
p^{i,j}(\lambda,\nu)
=
\begin{cases}
1 
\quad
&\text{if $\lambda = \nu$}, 
\\
0
\quad
&\text{if $\lambda \ne \nu$}.  
\end{cases}
\end{equation*}

When $n=2m+1$
 and $j= \frac 1 2 (n-1)$ ($=m$), 
 we use the renormalized Knapp--Stein operator, 
 see \eqref{eqn:Ttilde}, 
 given by 
\[
  \Tttbb \nu {2m-\nu}m = \frac 1 {\nu-m} \Ttbb \nu {2m-\nu}m.  
\]



\begin{proposition}
\label{prop:1803105}
Suppose $(G,G')=(O(n+1,1),O(n,1))$ with $n=2m+1$.  
Let $i=m$ or $m+1$.  
Then the composition $\Tttbb \nu {2m-\nu}m \circ \Ctbb \lambda \nu {i,m} \colon I_{\delta}(i,\lambda) \to J_{\varepsilon}(m,2m-\nu)$
 for $\nu-\lambda \in {\mathbb{N}}$
 and $\delta \varepsilon \in (-1)^{\nu-\lambda}$ is given as follows.   
\begin{enumerate}
\item[{\rm{(1)}}]
For $\nu-\lambda \in {\mathbb{N}}_+$, 
\[
  \Tttbb \nu {2m-\nu}m \circ \Ctbb \lambda \nu {i,m}
  = 
  \frac 1 2 q(\nu-\lambda) \Atbb \lambda {2m-\nu} {(-1)^{\nu-\lambda}} {i,m}.  
\]
In particular,
 if $m-\lambda \in {\mathbb{N}}_+$, 
 then 
\[
  \Tttbb m m m \circ \Ctbb \lambda m {i,m}
  = 
  (-1)^{i-m} \frac {\pi^m} {m!} \Ctbb \lambda m {i,m}.  
\]

\item[{\rm{(2)}}]
For $\nu=\lambda=m$, 
\[
     \Tttbb m m m \circ \Ctbb m m {i,m}
  = 
     \Attbb m m + {i,m} 
     + (-1)^{i-m+1} \frac{\pi^m}{m!} \Ctbb m m {i,m}.  
\]
\end{enumerate}
\end{proposition}
\begin{proof}
(1)\enspace
The functional equation \eqref{eqn:180389}
 with $n=2m+1$ and $j=m$ shows
\[
   \Ttbb \nu {2m-\nu} m \circ \Ctbb \lambda \nu {i,m}
  = 
  q(\nu-\lambda) p^{i,m}(\lambda,\nu) \Atbb \lambda {2m-\nu}{(-1)^{\nu-\lambda}} {i,m}.  
\]
By \eqref{eqn:pijlmdnu}, 
 we have for $i \in \{m,m+1\}$ and $\lambda \ne \nu$, 
\[
  p^{i,m}(\lambda,\nu) = \frac 1 2 (\nu-m).  
\]
Hence the first equation is verified.  
For the second statement, 
 we substitute $\nu=m$.  
Then the second equation follows from the residue formula \eqref{eqn:qAC}
 and from the fact 
 $\Ctbb \lambda m {i,m}=\Cbb \lambda m {i,m}$
 when $\lambda \ne m$.  
\par\noindent
(2)\enspace
The case $i=m+1$ will be shown in Lemma \ref{lem:TCAC}.  
The case $i=m$ is similar by using
\[
     \lim_{\nu \to m} \frac 1 {\nu-m} \Atbb \nu{2m-\nu} +{m, m}
     = 
     \Attbb m m + {m,m}
     - \frac{\pi^m}{m!} \Ctbb m m {m,m} 
\]
as it will be explained in \eqref{eqn:AClimit} of Chapter \ref{sec:pfSBrho}.  
\end{proof}


\subsubsection{Proof of Theorems \ref{thm:180380} and \ref{thm:180386}}
\label{subsec:10.6.4}
Suppose $0 \le j \le n-1$.  
Then the principal series representation $J_{\delta}(j,\nu)$
 is reducible as a module of $G'=O(n,1)$
 if and only if
\begin{equation}
\label{eqn:Jreducible}
\nu \in \{j,n-1-j\} \cup (-{\mathbb{N}}_+) \cup (n + {\mathbb{N}}), 
\end{equation}
see Proposition \ref{prop:redIilmd} (1).  
Suppose $\nu$ satisfies \eqref{eqn:Jreducible}.  
Then the proper submodules
 of $J_{\delta}(j,\nu)$ are described as follows:
\par\noindent
{Case 1.}\enspace
$(n,\nu) \ne (2j+1,j)$, 
 equivalently,
 $\Ttbb \nu {n-1-\nu}j \ne 0$.  
\par
In this case, 
 the unique proper submodule
 of $J_{\delta}(j,\nu)$ is given as the kernel 
 of the Knapp--Stein operator
 $\Ttbb \nu {n-1-\nu}j \colon J_{\delta}(j,\nu) \to J_{\delta}(j,n-1-\nu)$.  
\par\noindent
{Case 2.}\enspace
$(n,\nu) = (2j+1,j)$, 
 equivalently,
 $\Ttbb \nu {n-1-\nu}j = 0$.  
\par
In this case, 
 there are two proper submodules
 of $J_{\delta}(j,\nu)$, 
 which are given as the kernel 
 of 
 $\Tttbb j j j \pm \frac{\pi^j}{j!}{\operatorname{id}}
  \in {\operatorname{End}}_{G'}(J_{\delta}(j,j))$, 
 see Lemma \ref{lem:161745}.  
\begin{proof}
[Proof of Theorems \ref{thm:180380} and \ref{thm:180386}]
Assume $(n,\nu) \ne (2j+1,j)$.  
This excludes the case 
 where ${\mathbb{Z}} \ni \lambda \le j$ from the conditions (ii) 
 ($i=j$) and (iv) ($i=j+1$)
 in Proposition \ref{prop:10.34}.  
In this case 
 Theorems \ref{thm:180380} and \ref{thm:180386}
 are immediate consequences of Proposition \ref{prop:10.34}.  



Assume now $(n,\nu,j)=(2m+1,m,m)$ for some $m \in {\mathbb{N}}_+$.  
Then $\Ctbb \lambda m {i,m}$ is not surjective
 if $\lambda < m$, 
 and is surjective if $\lambda=m$
 by Proposition \ref{prop:1803105} (1) and (2), 
 respectively.  
Thus Theorems \ref{thm:180380} and \ref{thm:180386} are proved.  
\end{proof}

