\newpage
%%%%%%%%%%%%%%%%%%%%%%%%%%%%%%%%%%%%%%%%%%%%%%%%%%%%%%%
\section{Symmetry breaking operators
 for irreducible representations
 with infinitesimal character $\rho$ :
 Proof of Theorems \ref{thm:SBOvanish}
 and \ref{thm:SBOone}}
\label{sec:pfSBrho}
%%%%%%%%%%%%%%%%%%%%%%%%%%%%%%%%%%%%%%%%%%%%%%%%%%%%%%%%%%
In the first half of this chapter, 
 we give a proof of Theorems \ref{thm:SBOvanish}
 and \ref{thm:SBOone}
 that determine the dimension of the space of symmetry breaking operators from
 {\it{irreducible}} representations $\Pi$ of $G=O(n+1,1)$
 to {\it{irreducible}} representations $\pi$ of the subgroup $G'=O(n,1)$
 when both $\Pi$ and $\pi$ have the trivial infinitesimal characters $\rho$,
 or equivalently
 by Theorem \ref{thm:LNM20} (2), 
 when 
\index{A}{IrrGrho@${\operatorname{Irr}}(G)_{\rho}$,
 set of irreducible admissible smooth representation of $G$
 with trivial infinitesimal character $\rho$\quad}
\begin{alignat*}{2}
\Pi 
&\in {\operatorname{Irr}}(G)_{\rho} 
&&=\{\Pi_{i,\delta} : 0 \le i \le n+1, \,\delta \in \{\pm\}\}, 
\\
\pi 
&\in {\operatorname{Irr}}(G')_{\rho} 
&&=\{\pi_{j,\varepsilon}: 0 \le j \le n, \,\varepsilon \in \{\pm\}\}.  
\end{alignat*} 
The proofs of Theorems \ref{thm:SBOvanish} and \ref{thm:SBOone} are completed
 in Section \ref{subsec:Pfvanish} and Section \ref{subsec:IJdual}, 
 respectively.  
In the latter half of this chapter,
 we give a concrete construction of such symmetry breaking operators from 
 $\Pi$ to $\pi$.  
We pursue such constructions 
 more than what we need for the proof
 for Theorems \ref{thm:SBOvanish} and \ref{thm:SBOone}:
 some of the results will be used 
 in calculating \lq\lq{periods}\rq\rq\
 in Chapter \ref{sec:period}.  
Our proof uses the symmetry breaking operators
 for {\it{principal series representations}}
 and their basic properties
 that we have developed in the previous chapters.  


%%%%%%%%%%%%%%%%%%%%%%%%%%%%%%%%%%%%%%%%%%%%%%%%%%%%%%%%%%%
\subsection{Proof of the vanishing result (Theorem \ref{thm:SBOvanish})}
\label{subsec:Pfvanish}
%%%%%%%%%%%%%%%%%%%%%%%%%%%%%%%%%%%%%%%%%%%%%%%%%%%%%%%%%
This section gives a proof of the vanishing theorem
 of symmetry breaking operators (Theorem \ref{thm:SBOvanish}).  
In the same circle of the ideas,
 we also give a proof of multiplicity-free results
 (Proposition \ref{prop:mfPipj}).  
In order to study symmetry breaking for irreducible representations
 $\Pi_{i,\delta}$ of $G$, 
 we embed 
$
   {\operatorname{Hom}}_{G'}(\Pi_{i,\delta}|_{G'}, \pi_{j,\varepsilon})
$
 into the space of symmetry breaking operators
 between principal series representations
 as follows:
\begin{lemma}
\label{lem:170519}
Let $\delta, \varepsilon \in \{\pm\}$.  
Then we have natural embeddings:
\begin{enumerate}
\item[{\rm{(1)}}]
for $0 \le i \le n$ and $0 \le j \le n-1$, 
\begin{equation}
{\operatorname{Hom}}_{G'}(\Pi_{i,\delta}|_{G'}, \pi_{j,\varepsilon})
\subset
{\operatorname{Hom}}_{G'}(I_{\delta}(i,n-i)|_{G'}, J_{\varepsilon}(j,j));
\label{eqn:PpIJ1}
\end{equation}
\item[{\rm{(2)}}]
for $1 \le i \le n+1$ and $0 \le j \le n-1$, 
\begin{equation}
{\operatorname{Hom}}_{G'}(\Pi_{i,\delta}|_{G'}, \pi_{j,\varepsilon})
\subset
{\operatorname{Hom}}_{G'}(I_{-\delta}(i-1,i-1)|_{G'}, J_{\varepsilon}(j,j));
\label{eqn:PpIJ2}
\end{equation}
\item[{\rm{(3)}}]
for $0 \le i \le n$ and $1 \le j \le n$,
\begin{equation}
{\operatorname{Hom}}_{G'}(\Pi_{i,\delta}|_{G'}, \pi_{j,\varepsilon})
\subset
{\operatorname{Hom}}_{G'}(I_{\delta}(i,n-i)|_{G'}, J_{-\varepsilon}(j-1,n-j)); 
\label{eqn:PpIJ3}
\end{equation}
\item[{\rm{(4)}}]
for $1 \le i \le n+1$ and $1 \le j \le n$, 
\begin{equation}
{\operatorname{Hom}}_{G'}(\Pi_{i,\delta}|_{G'}, \pi_{j,\varepsilon})
\subset
{\operatorname{Hom}}_{G'}(I_{-\delta}(i-1,i-1)|_{G'}, J_{-\varepsilon}(j-1,n-j)).  
\label{eqn:PpIJ4}
\end{equation}
\end{enumerate}
\end{lemma}
\begin{proof}
We recall from Theorem \ref{thm:LNM20} (1)
 that there are surjective $G$-homomorphisms
\[
  I \twoheadrightarrow \Pi_{i,\delta}
\quad
  \text{for }
  I= I_{\delta}(i,n-i) 
  \text{ or }
  I_{-\delta}(i-1,i-1)
\]
 and injective $G'$-homomorphisms
\[
  \pi_{j,\varepsilon} \hookrightarrow J
 \quad \text{for $J=J_{\varepsilon}(j,j)$ or $J_{-\varepsilon}(j-1,n-j)$}.  
\]
Then the composition 
 $I \twoheadrightarrow \Pi_{i,\delta} \to \pi_{j,\varepsilon} \hookrightarrow J_{\varepsilon}(j,j)$
 yields the embeddings \eqref{eqn:PpIJ1}--\eqref{eqn:PpIJ4}.  
\end{proof}

\begin{proposition}
\label{prop:Ppvanish}
If $j \notin \{i-1,i\}$, 
 then 
$
{\operatorname{Hom}}_{G'}(\Pi_{i,\delta}|_{G'}, \pi_{j,\varepsilon})
=
\{0\}.  
$
\end{proposition}
\begin{proof}
Assume ${\operatorname{Hom}}_{G'}(\Pi_{i,\delta}|_{G'}, \pi_{j,\varepsilon})
\ne
\{0\}$.  



Suppose first $1 \le i \le n$.  
By Theorem \ref{thm:1.1} (1), 
 we get $j \in \{i-3, i-2, i-1,i\}$ from \eqref{eqn:PpIJ2}, 
 and $j \in \{i-1, i, i+1,i+2\}$ from \eqref{eqn:PpIJ3}.  
Hence we conclude $j \in \{i-1,i\}$.  


Suppose next $i=0$ or $n+1$.  
By using Theorem \ref{thm:1.1} (1) again, 
 we get $j \in \{0,1\}$ from \eqref{eqn:PpIJ1} for $i=0$, 
 and $j \in \{n-1,n\}$ from \eqref{eqn:PpIJ2} for $i=n+1$.  
Since $\dim_{\mathbb{C}} \Pi_{0,\delta}= \dim_{\mathbb{C}} \Pi_{n+1,\delta}=1$
 whereas both $\pi_{1,\varepsilon}$ and $\pi_{n-1,\varepsilon}$
 are infinite-dimensional irreducible representations of $G'$
 (Theorem \ref{thm:LNM20} (4)), 
 we have an obvious vanishing result:
\[
{\operatorname{Hom}}_{G'}(\Pi_{i,\delta}|_{G'}, \pi_{j,\varepsilon})
=\{0\}
\quad
\text{ if $(i,j)=(0,1)$ or $(n+1,n-1)$}.  
\]


Hence we conclude $j \in \{i-1,i\}$
 for $i=0$ or $n+1$, too.  
\end{proof}



\begin{proposition}
\label{prop:170526}
If $\delta \varepsilon =-$, 
 then 
$
   {\operatorname{Hom}}_{G'}(\Pi_{i,\delta}|_{G'}, \pi_{j,\varepsilon})
   =
   \{0\}.
$
\end{proposition}
%for arXiv
\newpage
\begin{proof}
We have already proved the assertion in the case 
 $j \not \in \{i-1,i\}$
 in Proposition \ref{prop:Ppvanish}.  
Therefore it suffices to prove 
 the assertion in the case $j=i-1$ and $i$.  
We begin with the case $j=i-1$.  


Suppose $2 \le i \le n$.  
Then by Theorem \ref{thm:1.1} (3), 
\[
   {\operatorname{Hom}}_{G'}(I_{\delta}(i,n-i)|_{G'}, J_{-\varepsilon}(i-2,n-i+1))
   =
   \{0\}
\]
because $\delta (-\varepsilon) =+$.  
This implies 
$
   {\operatorname{Hom}}_{G'}
   (\Pi_{i,\delta}|_{G'}, \pi_{i-1,\varepsilon})
   =
   \{0\}
$ from \eqref{eqn:PpIJ3}.    



For the case
 $(i,j)=(1,0)$, 
 we know from \cite[Thm.~2.5 (1-a)]{sbon}
 that
\[
   {\operatorname{Hom}}_{G'}
   (\Pi_{1,-}|_{G'}, \pi_{0,+})
   =
   \{0\}.  
\]
($F(0)=\pi_{0,+}$ and $T(0)=\Pi_{1,-}$ with the notation therein.)
It then follows from  Proposition \ref{prop:SBOdual}
 that 
$
   {\operatorname{Hom}}_{G'}
   (\Pi_{1,+}|_{G'}, \pi_{0,-})=\{0\}
$.  



For the case $(i,j)=(n+1,n)$, 
 we use the fact 
 that both $\Pi_{i,\delta}$ and $\pi_{j,\varepsilon}$ are 
 one-dimensional.  
In fact,
 we have isomorphisms
 $\Pi_{n+1,\delta} \simeq \chi_{-,\delta}$
 and $\pi_{n,\varepsilon} \simeq \chi_{-,\varepsilon}|_{G'}$
 by Theorem \ref{thm:LNM20} (4).  
Thus the vanishing assertion is straightforward for $j=i-1$
 ($1 \le i \le n+1$).  



The case $j=i$ is derived from the case $j=i-1$
 by duality
 (see Proposition \ref{prop:SBOdual}). 
\end{proof}



By Propositions \ref{prop:Ppvanish} and \ref{prop:170526}, 
 we have completed the proof of Theorem \ref{thm:SBOvanish}.  



%%%%%%%%%%%%%%%%%%%%%%%%%%%%%%%%%%%%%%%%%%%%%%%%%
\subsection{Construction of symmetry breaking operators from 
 $\Pi_{i,\delta}$ to $\pi_{i,\delta}$: 
Proof of Theorem \ref{thm:SBOone}}
\label{subsec:StageCii}
%%%%%%%%%%%%%%%%%%%%%%%%%%%%%%%%%%%%%%%%%%%%%%%%%
In this section we prove
 the existence and the uniqueness 
 (up to scalar multiplication) of symmetry breaking operators from 
 the irreducible $G$-module $\Pi_{i,\delta}$ 
 to the irreducible $G'$-module $\pi_{j,\varepsilon}$
 when $j \in \{i-1,i\}$ and $\delta \varepsilon= +$, 
 and thus complete the proof of Theorem \ref{thm:SBOone}.  
Moreover,
 we investigate their $(K,K')$-spectrum
 for 
\index{B}{minimalKtype@minimal $K$-type}
minimal $K$- and $K'$-types,
 and also give an explicit construction of such operators.  

%%%%%%%%%%%%%%%%%%%%%%%%%%%%%%%%%%%%%%%%%%%%%%%%%
\subsubsection{Generators of symmetry breaking operators
 between principal series representations
 having the trivial infinitesimal character $\rho$}
\label{subsec:StageBii}
%%%%%%%%%%%%%%%%%%%%%%%%%%%%%%%%%%%%%%%%%%%%%%%%%

We have determined explicit generators 
 of symmetry breaking operators
 $I_{\delta}(i,\lambda) \to J_{\varepsilon}(j,\nu)$
 in Theorem \ref{thm:SBObasis}.  
In this subsection,
 we extract some special cases
 which will be used for the proof of Theorem \ref{thm:SBOone}.   



The following lemma is used for the proof
 of the multiplicity-free theorem
 (Proposition \ref{prop:mfPipj} below), 
 and also for an explicit construction of nonzero symmetry breaking operators
 $\Pi \to \pi$ with $\Pi \in {\operatorname{Irr}}(G)_{\rho}$
 and $\pi \in {\operatorname{Irr}}(G')_{\rho}$
 (Proposition \ref{prop:172459}).  


\begin{lemma}
\label{lem:170529}
\begin{enumerate}
\item[{\rm{(1)}}]
Suppose $0 \le i \le n-1$
 and $\delta \varepsilon=+$.  
Then 
\begin{equation*}
{\operatorname{Hom}}_{G'}
     (I_{\delta}(i,n-i)|_{G'}, J_{\delta}(i,i))
\simeq
\begin{cases}
{\mathbb{C}}\Atbb {n-i}i+ {i,i}
\quad
&\text{if $2i \ne n$},  
\\
{\mathbb{C}}\Attbb {n-i}i+ {i,i}
\oplus
{\mathbb{C}}\Ctbb {n-i}i {i,i}
\quad
&\text{if $2i = n$}.  
\end{cases}
\end{equation*}
\item[{\rm{(2)}}]
Suppose $1 \le i \le n-1$
 and $\delta \varepsilon=+$.  
Then 
\begin{equation*}
{\operatorname{Hom}}_{G'}
     (I_{-\delta}(i-1,i-1)|_{G'}, J_{\varepsilon}(i,i))
=
{\mathbb{C}}\Ctbb {i-1} i {i-1,i}.  
\end{equation*}
\item[{\rm{(3)}}]
Suppose $0 \le i \le n-1$
 and $\delta \in \{\pm\}$.  
Then we have
\begin{equation*}
{\operatorname{Hom}}_{G'}
     (I_{\delta}(i,i)|_{G'}, J_{\delta}(i,i))
=
{\mathbb{C}}\Attbb {i}i+ {i,i}
\oplus
{\mathbb{C}} \Ctbb i i{i,i}.    
\end{equation*}
\item[{\rm{(4)}}]
Suppose $1 \le i \le n$
 and $\delta \varepsilon=+$.  
Then 
\begin{multline*}
{\operatorname{Hom}}_{G'}
     (I_{-\delta}(i-1,i-1)|_{G'}, J_{-\varepsilon}(i-1,n-i))
\\
\simeq
\begin{cases}
{\mathbb{C}}\Atbb {i-1}{n-i}+ {i-1,i-1}
\quad
&\text{if $n \ne 2i-1$},  
\\
{\mathbb{C}}\Attbb {i-1}{n-i}+ {i-1,i-1}
\oplus
{\mathbb{C}} \Ctbb {i-1}{n-i}{i-1,i-1}
\quad
&\text{if $n=2i-1$}.  
\end{cases}
\end{multline*}
\end{enumerate}
\end{lemma}
\begin{proof}
We determined the dimension of the left-hand side
 by Theorem \ref{thm:1.1} (2) and (3).  
Then the lemma follows from Theorem \ref{thm:SBObasis}
 for (1), (3), (4);
 and from Fact \ref{fact:3.9} for (2).  
\end{proof}

\begin{remark}
\label{rem:170529}
In the $N$-picture
 where the open Bruhat cells
 for the pair of the real flag manifolds
 $G'/P' \subset G/P$
 are represented by ${\mathbb{R}}^{n-1} \subset {\mathbb{R}}^{n}$, 
 we have $\Ctbb {n-i}{i}{i,i} ={\operatorname{Rest}}_{x_n=0}$
 in Lemma \ref{lem:170529} (1), 
 $\Ctbb {i-1}{i}{i-1,i} ={\operatorname{Rest}}_{x_n=0}
 \circ d_{\mathbb{R}^n}$ in (2), 
$\Ctbb {i}{i}{i,i} ={\operatorname{Rest}}_{x_n=0}$
 in (3), 
 and 
 $\Ctbb {i-1}{i-1}{i-1,i-1} =
{\operatorname{Rest}}_{x_n=0}$
 in (4).  
\end{remark}

The following lemma is used for an alternative construction
 (see Proposition \ref{prop:172459} below)
 of symmetry breaking operators $\Pi_{i,\delta} \to \pi_{i,\delta}$. 
\begin{lemma}
\label{lem:172459}
Suppose $1 \le i \le n$
 and $\delta \in \{\pm\}$.  
Then we have
\begin{equation*}
{\operatorname{Hom}}_{G'}
     (I_{\delta}(i,i)|_{G'}, J_{-\delta}(i-1,n-i))
=
{\mathbb{C}}\Atbb {i}{n-i}- {i,i-1}.  
\end{equation*}
\end{lemma}

\begin{proof}
By Theorem \ref{thm:1.1} (2), 
 $\Atbb {i}{n-i}- {i,i-1} \ne 0$, 
 and therefore the lemma follows from Theorem \ref{thm:SBObasis}.  
\end{proof}



\subsubsection{Multiplicity-free property of symmetry breaking}
In this subsection,
 we prove the following multiplicity-free property:
\begin{proposition}
\label{prop:mfPipj}
For any $0 \le i \le n+1$, 
 $0 \le j \le n$, 
 and $\delta, \varepsilon \in \{\pm\}$, 
 we have
\begin{equation}
\label{eqn:mfPipj}
   \dim_{\mathbb{C}}{\operatorname{Hom}}_{G'}
   (\Pi_{i,\delta}|_{G'},\pi_{j,\varepsilon}) \le 1.  
\end{equation}
\end{proposition}
Proposition \ref{prop:mfPipj} is a very special case 
 of the multiplicity-free theorem 
 which was proved in Sun--Zhu \cite{SunZhu}, 
 however,
 we give a different proof based on Lemmas \ref{lem:170519}
 and \ref{lem:170529}
 because the following short proof illustrates the idea of this chapter.  
\begin{proof}
[Proof of Proposition \ref{prop:mfPipj}]
Owing to the vanishing results (Theorem \ref{thm:SBOvanish}), 
 it suffices to show \eqref{eqn:mfPipj}
when $j \in \{i-1,i\}$ and $\delta \varepsilon=+$.  
Moreover,
 the case $j=i-1$ can be reduced to the case $j=i$
 by the duality
 between the spaces of symmetry breaking operators
 (Proposition \ref{prop:SBOdual}).  
Henceforth, 
 we assume $j=i \in \{0,1,\ldots,n\}$
 and $\delta \varepsilon =+$.  
Then, 
 owing to the embedding results given in Lemma \ref{lem:170519}, 
the multiplicity-free property
 \eqref{eqn:mfPipj} holds for $1 \le i \le n-1$
 by Lemma \ref{lem:170529} (2), 
 and for $i =0$ and $n$ by Lemma \ref{lem:170529} (1) and (4).  
Thus Proposition \ref{prop:mfPipj} is proved.  
\end{proof}

%%%%%%%%%%%%%%%%%%%%%%%%%%%%%%%%%%%%%%%%%%%%%%%%%
\subsubsection{Multiplicity-one property: Proof of Theorem \ref{thm:SBOone}}
\label{subsec:pfSBOone}
%%%%%%%%%%%%%%%%%%%%%%%%%%%%%%%%%%%%%%%%%%%%%%%%%

In proving Theorem \ref{thm:SBOone}, 
 we use the following proposition,
 whose proof is deferred at the next subsection.  
\begin{proposition}
\label{prop:existii}
 ${\operatorname{Hom}}_{G'}(\Pi_{i,\delta}|_{G'}, \pi_{i,\delta}) \ne \{0\}$
 for all $0 \le i \le n$ and $\delta \in \{\pm\}$.  
\end{proposition}
\begin{remark}
\label{rem:existii}
Obviously Proposition \ref{prop:existii} holds for $i=0$
 because $\Pi_{0,\delta}|_{G'} \simeq \pi_{0,\delta}$ as $G'$-modules
 for $\delta \in \{\pm\}$.  
Indeed, 
 the $G$-modules $\Pi_{0,+}$ and $\Pi_{0,-}$ are the one-dimensional representations
 ${\bf{1}}$ and respectively
 $\chi_{+-}$ 
 (Theorem \ref{thm:LNM20} (4)),
 and likewise for the $G'$-modules $\pi_{0,\pm}$.  
\end{remark}

Before giving a proof of Proposition \ref{prop:existii}, 
 we show
 that Proposition \ref{prop:existii} implies Theorem \ref{thm:SBOone}.  
\begin{proof}
[Proof of Theorem \ref{thm:SBOone}]
By the duality among the spaces
 of symmetry breaking operators
 (Proposition \ref{prop:SBOdual}), 
 we may and do assume $j=i$ and $\delta=\varepsilon=+$
 because $\widetilde j:=n-j$
 and $\widetilde i:=n+1-i$
 satisfy $\widetilde j = \widetilde i-1$
 if and only if $j=i$.  
Then Theorem \ref{thm:SBOone} follows from Propositions \ref{prop:mfPipj}
 (uniqueness) and \ref{prop:existii} (existence).  
\end{proof}
For later purpose,
 we need a refinement of Proposition \ref{prop:existii}
 by providing information of $(K,K')$-spectrum
 in Proposition \ref{prop:AiiAq} below.  
For this,
 we fix some terminology:
\begin{definition}
[minimal $K$-type]
\label{def:minK}
We set $m:=[\frac{n+1}2]$.  
Suppose $\mu \in \widehat K$.  
To describe an irreducible finite-dimensional representation $\mu$
 of $K=O(n+1) \times O(1)$, 
 we use the notation in Section \ref{subsec:fdimrep}
 in Appendix I
 rather than the previous one
 in Section \ref{subsec:ONWeyl}, 
 and write
\[
  \mu = \Kirredrep{O(n+1)}{\sigma_1, \cdots,\sigma_m}_{\varepsilon} \boxtimes \delta
\]
for $\sigma =(\sigma_1, \cdots,\sigma_m) \in \Lambda^+(m)$
 and $\varepsilon,\delta \in \{\pm\}$.  
We define $\|\mu\|>0$ by 
\[
\|\mu\|^2 = \sum_{j=1}^m (\sigma_j + n + 1-2j)^2
\quad
(=\|\sigma + 2 \rho_c\|^2), 
\]
where $2 \rho_c =(n-1,n-3,\cdots,n+1-2m)$ is the sum
 of positive root
 for ${\mathfrak{k}}_{\mathbb{C}}={\mathfrak{o}}(n+1,{\mathbb{C}})$
 in the standard coordinates.  
For a nonzero admissible representation $\Pi$ of $G$, 
 the set of {\it{minimal $K$-types}} of $\Pi$ is 
\[
   \{\mu \in \widehat K
   :
   \text{$\mu$ occurs in $\Pi$, 
 and $\|\mu\|$ is minimal with this property}\}, 
\]
see \cite[Chap.~2]{KV}
 or \cite[Def.~5.4.18]{Vogan81}.  
\end{definition}

We then observe:
\begin{remark}
[minimal $K$-type]
\label{rem:minK}
\index{B}{minimalKtype@minimal $K$-type|textbf}
The basic $K$-type 
 (see Definition \ref{def:basicK}) of the principal series representation
 $I_{\delta}(i,\lambda)$
 is the unique minimal $K$-type
 of the irreducible $G$-module $\Pi_{i,\delta}$, 
 as stated in Theorem \ref{thm:LNM20} (3).  
\end{remark}
\begin{proposition}
\label{prop:AiiAq}
Let $(G,G')=(O(n+1,1),O(n,1))$, 
 $0 \le i \le n$
 and $\delta \in \{\pm\}$.  
Then there exists a nonzero symmetry breaking operator
\begin{equation}
\label{eqn:Pipi}
   A_{i,i} \colon \Pi_{i,\delta} \to \pi_{i,\delta}
\end{equation}
 such that its $(K,K')$-spectrum
 for the minimal $K'$- and $K$-types $\mub (i,\delta)' (\hookrightarrow \mub(i,\delta))$
 is nonzero.  
\end{proposition}

Proposition \ref{prop:AiiAq} is an {\it{existence}} theorem,
 however,
 we shall prove it by {\it{constructing}}
 nonzero symmetry breaking operators 
 $\Pi_{i,\delta} \to \pi_{i,\delta}$, 
 see Proposition \ref{prop:172459}
 in the next subsection.  
Alternative constructions are also given in Sections \ref{subsec:irrIJ} and \ref{subsec:3irrIJ}, 
 and thus we construct symmetry breaking operators
 $\Pi_{i,\delta} \to \pi_{i,\delta}$
 in the following three ways:
\begin{center}\begin{tabular}{ll}
$\bullet$\ $\Atbb i{n-i}- {i,i-1} \colon I_{\delta}(i,i) \to J_{-\delta}(i-1,n-i)$, &
(Proposition \ref{prop:172459}), \\[0.5cm]
$\bullet$\ $\Attbb {i}{i}+ {i,i} \colon I_{\delta}(i,i) \to J_{\delta}(i,i)$, &
(Proposition \ref{prop:SBOi4}), \\[0.5cm]
$\bullet$\ $\Atbb {n-i}{i}+ {i,i} \colon I_{\delta}(i,n-i) \to J_{\delta}(i,i)$, &
(Proposition \ref{prop:170530}).  
\end{tabular}\end{center}



\subsubsection{First construction $\Pi_{i,\delta} \to \pi_{i,\delta}$
 ($1 \le i \le n$)}
\label{subsec:IJdual}
In this subsection,
 we construct a nonzero symmetry breaking operator
\[
   \Pi_{i,\delta} \to \pi_{i,\delta}
   \qquad
  \text{for $1 \le i \le n$, $\delta \in \{\pm\}$, }
\]
by using Lemma \ref{lem:172459}.  
\begin{proposition}
\label{prop:172459}
Suppose $1 \le i \le n$
 and $\delta \in \{\pm\}$.  
Then the normalized symmetry breaking operator
\begin{equation*}
    \Atbb {i}{n-i}- {i,i-1}
    \colon 
   I_{\delta}(i,i) \to J_{-\delta}(i-1,n-i)
\]
satisfies the following:
\begin{enumerate}
\item[{\rm{(1)}}]
${\operatorname{Image}}(\Atbb {i}{n-i}- {i,i-1})_{K'}=(\pi_{i,\delta})_{K'}$
 as $({\mathfrak{g}}',K')$-modules;
\item[{\rm{(2)}}]
$\Atbb {i}{n-i}- {i,i-1}|_{\Pi_{i,\delta}} \ne 0$.  
\end{enumerate}
In particular,
 it induces a symmetry breaking operator
 $\Pi_{i,\delta} \to \pi_{i,\delta}$
 as in the diagram below.  
Moreover,
 the $(K,K')$-spectrum
 of the resulting operator 
 for the minimal $K'$- and $K$-types $\mub (i,\delta)'$
 $(\hookrightarrow \mub(i,\delta))$ is nonzero.  
\begin{eqnarray*}
\xymatrix@R=2pt{
I_{\delta}(i,i)\ar[rr]^{\Atbb {i}{n-i}- {i,i-1}\quad}
&& J_{-\delta}(i-1,n-i)
\\
\bigcup
&&\bigcup
\\
\Pi_{i,\delta}
\ar@{-->}[rr]
&&\pi_{i,\delta}
}
\end{eqnarray*}
\end{proposition}

\begin{convention}
\label{conv:Image}
Hereafter,
 by abuse of notation,
 we shall write simply as 
${\operatorname{Image}}(\Atbb {i}{n-i}- {i,i-1})=\pi_{i,\delta}$
 if their underlying $({\mathfrak{g}}',K')$-modules coincide
 ({\it{cf}}. Proposition \ref{prop:172459} (1)).  
\end{convention}
%for arXiv
\newpage
\begin{proof}
[Proof of Proposition \ref{prop:172459}]
(1)\enspace
First we observe
\[
   {\operatorname{Image}}(\Atbb {i}{n-i}- {i,i-1})
  \subset 
   {\operatorname{Ker}}(\Ttbb {n-i}{i-1} {i-1})
\]
because Theorem \ref{thm:TAA} with $\nu=n-i$ tells the functional equation
$
   \Ttbb {n-i}{i-1}{i-1} \circ \Atbb i {n-i}-{i,i-1}=0.  
$



When $n \ne 2i-1$, 
 we conclude ${\operatorname{Image}}(\Atbb i {n-i}-{i,i-1})=\pi_{i,\delta}$ 
 by Proposition \ref{prop:Timage}
 because $\pi_{i,\delta}$ is irreducible as a $G'$-module.  
When $n=2i-1$, 
 the Knapp--Stein operator $\Ttbb {n-i}{i-1}{i-1}$
 vanishes 
 (Proposition \ref{prop:Tvanish}).  
Instead we use the following renormalized Knapp--Stein operator
 (see \eqref{eqn:Ttilde}):
\[
   \Tttbb \nu{n-1-\nu}{i-1}
   =
  \frac {1}{\nu-i+1}\Ttbb \nu{n-1-\nu}{i-1}.  
\]
Then the functional equation given in Theorem \ref{thm:TAA} implies
\[
   \left(
   \Tttbb {i-1}{i-1}{i-1} + \frac{\pi^{i-1}}{(i-1)!}\, {\operatorname{id}}
   \right)
   \circ \Atbb \lambda{i-1}-{i,i-1}=0.  
\]
By Lemma \ref{lem:161745} applied to the subgroup $G'=O(n,1)$ $(=O(2i-1,1))$, 
 we conclude 
 ${\operatorname{Image}}(\Atbb \lambda {n-i}- {i,i-1})
={\operatorname{Image}}(\Atbb \lambda {i-1}- {i,i-1})=\pi_{i,\delta}$
 in the case $n=2i-1$, too.  
\newline\noindent
(2)\enspace
The second statement follows from the fact
 that the $(K,K')$-spectrum of $\Atbb \lambda \nu - {i,i-1}$
 (Theorem \ref{thm:153315})
 for the basic $K$-types
 $(\mu,\mu')=(\mub(i,\delta),\mus(i-1,-\delta)')$
 does not vanish.  
The last assertion is derived from the following observation 
 (see \eqref{eqn:flatsharp}):
there are isomorphisms
 of representations of $K'=O(n) \times O(1)$, 
\[
  \mus (i-1,-\delta)'\simeq \mub(i,\delta)'.  
\]
Hence Proposition \ref{prop:172459} is proved.  
\end{proof}

\begin{proof}
[Proof of Proposition \ref{prop:AiiAq}]
Clear from Proposition \ref{prop:172459}
 and Remark \ref{rem:existii}.  
\end{proof}

Thus,
 the proof of Theorem \ref{thm:SBOone} has been completed.  

\vskip 1pc
For the rest of this chapter,
 we give alternative constructions
 of symmetry breaking operators for later purposes.  

\subsubsection{Second construction $\Pi_{i,\delta} \to \pi_{i,\delta}$
 ($0 \le i \le n-1$)}
\label{subsec:irrIJ}

In this subsection,
 we provide another construction
 of a nonzero symmetry breaking operator
\[
    \Pi_{i,\delta} \to \pi_{i,\delta}
   \quad
   \text{for $0 \le i \le n-1$, $\delta \in \{\pm\}$}, 
\]
by using Lemma \ref{lem:170529} (3).  
\begin{proposition}
\label{prop:SBOi4}
Suppose $0 \le i \le n-1$ and $\delta \in \{\pm\}$.  
Then the renormalized operator
\[
  \Attbb i i + {i,i}\colon I_{\delta}(i,i) \to J_{\delta}(i,i)
\]
 satisfies the following:
\begin{enumerate}
\item[{\rm{(1)}}]
${\operatorname{Image}}(\Attbb i i + {i,i})=\pi_{i,\delta}$;
\item[{\rm{(2)}}]
$\Attbb i i + {i,i}|_{\Pi_{i,\delta}}\ne 0.$
\end{enumerate}
In particular,
 it induces a symmetry breaking operator
 $\Pi_{i,\delta} \to \pi_{i,\delta}$ 
 as in the diagram below.  
Moreover,
 the $(K,K')$-spectrum
 of the resulting operator 
 for the minimal $K'$- and $K$-types $\mub (i,\delta)'$
 $(\hookrightarrow \mub(i,\delta))$ is nonzero.  

\begin{eqnarray*}
\xymatrix@R=2pt{
I_{\delta}(i,i)\ar[rr]^{\Attbb {i}{i}+ {i,i}\quad}
&& J_{\delta}(i,i)
\\
\bigcup
&&\bigcup
\\
\Pi_{i,\delta}
\ar@{-->}[rr]
&&\pi_{i,\delta}
}
\end{eqnarray*}
\end{proposition}
\begin{proof}
[Proof of Proposition \ref{prop:SBOi4}]
(1)\enspace
By the functional equation \eqref{eqn:TA2tilde}, 
we have
\[
  {\operatorname{Image}}(\Attbb \lambda i + {i,i})
 \subset {\operatorname{Ker}} (\Ttbb i {n-1-i}i).  
\]
When $n \ne 2i+1$, 
 we conclude 
 ${\operatorname{Image}}(\Attbb \lambda i + {i,i})=\pi_{i,\delta}$
 by Proposition \ref{prop:Timage}.  


When $n=2i+1$, 
 the Knapp--Stein operator $\Ttbb i {n-1-i}i\equiv \Ttbb i i i$ vanishes 
 (Proposition \ref{prop:Tvanish}).  
Instead we use the functional equation \eqref{eqn:TAA3tilde}
 for the renormalized operators $\Tttbb i i i$ and $\Attbb \lambda i +{i,i}$, 
 which tells
 that
\[
   {\operatorname{Image}}(\Attbb \lambda i + {i,i})
   \subset {\operatorname{Ker}} 
   \left(\Tttbb i i i - \frac{\pi^i}{\Gamma(i+1)}\, {\operatorname{id}}\right).  
\]
By Lemma \ref{lem:161745}, 
 we conclude 
$
 {\operatorname{Image}}(\Attbb \lambda i + {i,i})=\pi_{i,\delta} 
$
 because $\Attbb \lambda i + {i,i}$ is nonzero
 and $\pi_{i,\delta}$ is irreducible as a $G'$-module.  
\newline\noindent
(2)\enspace
The assertion follows readily from the $(K,K')$-spectrum
 of the renormalized operator $\Attbb \lambda i + {i,i}$
 (see \eqref{eqn:SAiitilde})
 for the basic $K$- and $K'$-types
 $(\mu,\mu')=(\mub(i,\delta),\mub(i,\delta)')$.  
\end{proof}





%%%%%%%%%%%%%%%%%%%%%%%%%%%%%%%%%%%%%%%%%%%%%%%%%
\subsubsection{Third construction $\Pi_{i,\delta} \to \pi_{i,\delta}$}
\label{subsec:3irrIJ}
%%%%%%%%%%%%%%%%%%%%%%%%%%%%%%%%%%%%%%%%%%%%%%%%%



We give yet another construction
 of a nonzero symmetry breaking operator
 $\Pi_{i,\delta} \to \pi_{i,\delta}$
 in the case $n \ne 2i$.  
In the case $n=2i$, 
 the normalized operator 
 $\Atbb {n-i} i + {i,i}$ vanishes.  
We shall discuss this case separately
 in Section \ref{subsec:IJmm+}, 
see Proposition \ref{prop:Ammimage}.  

\begin{proposition}
\label{prop:170530}
If $2i \ne n$, 
 then 
$
   \Atbb {n-i}i+ {i,i}
   \in 
   {\operatorname{Hom}}_{G'}
   (I_\delta(i,n-i)|_{G'}, J_\delta(i,i))
$ 
satisfies
\[
   \Atbb {n-i}i+ {i,i}|_{\Pi_{i+1,-\delta}} \equiv 0
\quad
   \text{and}
\quad
   {\operatorname{Image}}(\Atbb {n-i}i+ {i,i})
   = \pi_{i,\delta}.  
\]
Thus it induces a symmetry breaking operator 
$\Pi_{i,\delta} \to \pi_{i,\delta}$
 as in the diagram below.  
Moreover,
 the $(K,K')$-spectrum
 of the resulting operator 
 for the minimal $K'$- and $K$-types $\mub (i,\delta)'$
 $(\hookrightarrow \mub(i,\delta))$ is nonzero.  
\begin{eqnarray*}
\xymatrix@C=20pt@R=2pt{
I_{\delta}(i,n-i)\ar[rr]^{\Atbb {n-i}{i}+ {i,i}\,\,}
\ar[dd]
&& J_{\delta}(i,i)
\\
&&\bigcup
\\
\Pi_{i,\delta} \simeq
I_{\delta}(i,n-i)/\Pi_{i+1,-\delta}
\ar@{-->}[rr]
&&\pi_{i,\delta}
}
\end{eqnarray*}
\end{proposition}

\begin{proof}
Since $\Atbb i i + {i,i} =0$
 by Theorem \ref{thm:161243} (1), 
 the composition 
$
     \Atbb {n-i}i+ {i,i} \circ \Ttbb i{n-i}{i}
$
 vanishes by the functional equation (Theorem \ref{thm:ATA}).  
Thus $\Atbb {n-i}i+ {i,i}$ is identically zero
 on ${\operatorname{Image}} (\Ttbb i{n-i}{i}) \simeq \Pi_{i+1,-\delta}$
 (see Proposition \ref{prop:Timage}).  



For the second assertion,
 we use another functional equation 
 (Theorem \ref{thm:TAA})
 to get $\Ttbb i{n-1-i}{i} \circ \Atbb {n-i}i+ {i,i}=0$.  
Hence
\[
     {\operatorname{Image}}(\Atbb {n-i}i+ {i,i})
     \subset
     {\operatorname{Ker}}(\Ttbb i{n-1-i}i)
     \simeq 
     \pi_{i,\delta}
\]
 by Proposition \ref{prop:Timage}.  
Since $\Atbb {n-i}i+ {i,i} \ne 0$
(see Theorem \ref{thm:161243} (1))
 and since $\pi_{i,\delta}$ is irreducible,
 the underlying $({\mathfrak{g}}',K')$-modules
 of ${\operatorname{Image}}(\Atbb {n-i}i+ {i,i})$
 and $\pi_{i,\delta}$ coincide.  
\end{proof}




\subsection{Splitting of $I_{\delta}(m,m)$ and its symmetry breaking for $(G,G')=(O(2m+1,1),O(2m,1))$}
\label{subsec:2m}

Suppose $n$ is even, 
 say $n=2m$.  
A distinguished feature in this setting
 is that the principal series representation 
 $I_{\delta}(m,\lambda)$ of $G =O(2m+1,1)$
 splits into the direct sum
 of two irreducible $G$-modules
 when $\lambda=m$: for $\delta\in \{\pm\}$, 
\begin{equation}
\label{eqn:I+n2}
I_{\delta}(m,m) \simeq \Pi_{m,\delta} \oplus \Pi_{m+1,-\delta}, 
\end{equation}
both of which are smooth irreducible 
\index{B}{temperedrep@tempered representation}
tempered representations of $G$,
 see Theorem \ref{thm:LNM20} (1) and (8).  
Accordingly, 
 the space of symmetry breaking operators has a direct sum decomposition:
\begin{multline}
\label{eqn:mmdeco}
{\operatorname{Hom}}_{G'}(I_{\delta}(m,m)|_{G'}, J_{\varepsilon}(m, m))
\\
\simeq 
{\operatorname{Hom}}_{G'}(\Pi_{m,\delta}|_{G'}, J_{\varepsilon}(m, m))
\oplus 
{\operatorname{Hom}}_{G'}(\Pi_{m+1,-\delta}|_{G'}, J_{\varepsilon}(m, m)), 
\end{multline}
for each $\varepsilon \in \{\pm\}$.  
The left-hand side of \eqref{eqn:mmdeco} 
 has been understood by the classification
 of symmetry breaking operators
 given in Theorem \ref{thm:SBObasis}
 (see \eqref{eqn:IJmm} as below). 
On the other hand,
 the target space $J_{\varepsilon}(m,m)$ is not irreducible
 as a $G'$-module.  
We recall from Theorem \ref{thm:LNM20} (1)
 that the principal series representation
 $J_\varepsilon(m,\nu)$
 of $G'=O(2m,1)$ at $\nu=m$ has a nonsplitting exact sequence
 of $G'$-modules:
\begin{equation}
\label{eqn:Jmm}
0 \to \pi_{m,\varepsilon} \to J_{\varepsilon}(m,m) \to \pi_{m+1,-\varepsilon} \to 0.  
\end{equation}
With this in mind,
 we shall take a closer look 
 at the right-hand side of \eqref{eqn:mmdeco}
 and determine each summand as follows: 


\begin{equation}
\label{tbl:PiJmm}
\begin{tabular}{c|c|c}
& {$\delta\varepsilon=+$}
&{$\delta\varepsilon=-$}
\\
\hline
  ${\operatorname{Hom}}_{G'}(\Pi_{m,\delta}|_{G'}, J_{\varepsilon}(m, m))$ 
& ${\mathbb{C}}$
& $\{0\}$
\\
${\operatorname{Hom}}_{G'}(\Pi_{m+1,-\delta}|_{G'}, J_{\varepsilon}(m, m))$ 
& ${\mathbb{C}}$
& ${\mathbb{C}}$ 
\end{tabular}
\end{equation}
See Section \ref{subsec:IJmm+}
 for the left column of \eqref{tbl:PiJmm} in detail,
 and for Section \ref{subsec:IJmm-} for the right column.



\subsubsection
{${\operatorname{Hom}}_{G'}(I_{\delta}(m,m)|_{G'}, J_{\varepsilon}(m,m))$
 with $\delta \varepsilon=+$}
\label{subsec:IJmm+}
We begin with the case $\delta\varepsilon =+$.  
Without loss of generality,
 we may and do assume $\delta=\varepsilon=+$.  



Then Lemma \ref{lem:170529} (3) 
 with Remark \ref{rem:170529} tells that 
\begin{equation}
\label{eqn:IJmm}
{\operatorname{Hom}}_{G'}(I_{+}(m,m)|_{G'}, 
                          J_{+}(m,m))
=
{\mathbb{C}} \Attbb {m} {m} + {m,m}
\oplus
{\mathbb{C}} {\operatorname{Rest}}.  
\end{equation}



The first generator $\Attbb {m} {m} + {m,m}$ is defined
 as the renormalization (Theorem \ref{thm:170340})
\begin{equation}
\label{eqn:Amm2tilde}
\Attbb {m} {m} + {m,m}
=
\lim_{\lambda \to m}
\Gamma\left(\frac {\lambda-m}{2}\right)
\Atbb {\lambda} {m} + {m,m}
\end{equation}
of the normalized regular symmetry breaking operator
 $\Atbb {\lambda} {m} + {m,m}$
 which vanishes at $\lambda=m$
 (Theorem \ref{thm:161243}).  
The second generator,
 ${\operatorname{Rest}}\equiv {\operatorname{Rest}}_{x_n=0}$, 
 is the obvious symmetry breaking operator
({\it{cf.}}~Lemma \ref{lem:152268}), 
 given by ${\operatorname{Rest}}_{x_n=0}$
 in the $N$-picture.  
By using the second generator,
 we obtain the following.  



\begin{proposition}
\label{prop:IJmmbase}
Let $(G,G')=(O(2m+1,1),O(2m,1))$.  
Then we have 
\begin{align*}
{\operatorname{Hom}}_{G'}(\Pi_{m,+}|_{G'},J_+(m,m))
&=
{\mathbb{C}} {\operatorname{Rest}}|_{\Pi_{m,+}}, 
\\
{\operatorname{Hom}}_{G'}(\Pi_{m+1,-}|_{G'},J_+(m,m))
&=
{\mathbb{C}} {\operatorname{Rest}}|_{\Pi_{m+1,-}}.  
\end{align*}
\end{proposition}

\begin{proof}
By the direct sum decompositions \eqref{eqn:IJmm} and \eqref{eqn:I+n2}, 
 we have
\begin{align*}
  2 =& \dim_{\mathbb{C}} \operatorname{Hom}_{G'}(I_+(m, m)|_{G'}, J_+(m, m))
\\
    =&\dim_{\mathbb{C}} \operatorname{Hom}_{G'}(\Pi_{m, +}|_{G'}, J_+(m, m))
 + \dim_{\mathbb{C}} \operatorname{Hom}_{G'}(\Pi_{m+1, -}|_{G'}, J_+(m, m)).   
\end{align*}
On the other hand,
 we know from Lemma \ref{lem:152268}
 that ${\operatorname{Rest}}|_{\Pi_{m,+}}\ne 0$
 and ${\operatorname{Rest}}|_{\Pi_{m+1,-}}\ne 0$.  
Hence we have proved the proposition.  
\end{proof}
We have not used the other generator
 $\Attbb m m + {m,m}$ in \eqref{eqn:IJmm}
 for the previous proposition.  
For the sake of completeness,
 we investigate its restriction
 to each of the irreducible components in \eqref{eqn:I+n2}.  


\begin{proposition}
\label{prop:Ammrest}
Retain the notation as in \eqref{eqn:IJmm}.  
\begin{alignat*}{2}
&\Attbb {m}{m} + {m, m}
|_{\Pi_{m+1,-}}
&&
\equiv
0.  
\\
&\Attbb {m}{m} + {m, m}
|_{\Pi_{m,+}}
&&=
\frac{2 \pi^{m-\frac{1}{2}}}{m!} {\operatorname{Rest}}|_{\Pi_{m,+}}.  
\end{alignat*}
\end{proposition}
We also determine
 the image of the nonzero symmetry breaking operators
 $\Attbb m m + {m,m}$ and ${\operatorname{Rest}}$
 on each irreducible summand
in \eqref{eqn:I+n2}.  
\begin{proposition}
\label{prop:Ammimage}
With Convention \ref{conv:Image}, 
 we have 
\begin{align*}
{\operatorname{Image}}(\Attbb {m}{m} + {m, m}
|_{\Pi_{m,+}})
=&
{\operatorname{Image}}
 ({\operatorname{Rest}}|_{\Pi_{m,+}})
=\pi_{m,+},   
\\
{\operatorname{Image}}
 ({\operatorname{Rest}}|_{\Pi_{m+1,-}})
=&
J_+(m, m).  
\end{align*}
\end{proposition}
For the proof of Propositions \ref{prop:Ammrest} and \ref{prop:Ammimage}, 
 we use Lemma \ref{lem:AmmT} about functional equations
 with appropriate renormalizations.  
We set 
\begin{equation}
\label{eqn:cm}
   c(m):=\frac{\pi^{m}}{m!}. 
\end{equation}
\begin{proof}
[Proof of Proposition \ref{prop:Ammrest}]
It follows from the functional equation \eqref{eqn:AT2tilde}
 for the 
\index{B}{KnappSteinoperatorrenorm@ Knapp--Stein operator, renormalized---}
renormalized Knapp--Stein operator
 $\Tttbb m m m$
 that 
\[
  \Attbb {m}{m} + {m, m}
  \circ
  (c(m)\, {\operatorname{id}} - \Tttbb {m}{m}{m})=0.  
\]



On the other hand, 
 Lemma \ref{lem:161745} implies 
 that the renormalized Knapp--Stein operator satisfies 
\[
  c(m)\, {\operatorname{id}} - \Tttbb {m}{m}{m}
  =
  0\,\, {\operatorname{id}}_{\Pi_{m, +}} 
  \oplus
  2\,\, {\operatorname{id}}_{\Pi_{m+1, -}}, 
\]
which implies ${\operatorname{Image}}(c(m)\, {\operatorname{id}} - \Tttbb {m}{m}{m})=\Pi_{m+1, -}$.  
Therefore, 
 $\Attbb {m}{m} + {m, m}$ is 
 identically zero
 on the irreducible $G$-submodule $\Pi_{m+1,-}$.  

To see the second statement,
 we use Proposition \ref{prop:IJmmbase}, 
 which shows that $\Attbb {m}{m}+{m, m}|_{\Pi_{m, +}}$ must be 
 proportional to ${\operatorname{Rest}}|_{\Pi_{m,+}}$.  
Comparing the $(K,K')$-spectrum
 of the two operators $\Attbb {m}{m}+{m, m}$
 and ${\operatorname{Rest}}$
 with respect to basic $K'$- and $K$-types
 $\mub(m, +)' \hookrightarrow \mub(m, +)$ 
 (see the formula \eqref{eqn:SAiitilde} for $\Attbb m m + {m,m}$ and Lemma \ref{lem:152268}
 for ${\operatorname{Rest}}$), 
 we get the second statement.  
\end{proof}

\begin{proof}
[Proof of Proposition \ref{prop:Ammimage}]
By the functional equation \eqref{eqn:TA2tilde}, 
\[
  {\operatorname{Image}}
  (\Attbb m m + {m,m}|_{\Pi_{m,+}})
  \subset
  {\operatorname{Ker}}
  (\Ttbb m {m-1} {m})
  =\pi_{m,+}.  
\]
Since $\Attbb m m + {m,m}|_{\Pi_{m,+}}$ is nonzero, 
 and since $\pi_{m,+}$ is an irreducible $G'$-module, 
 we get the first statement.  
For the second one, 
 we compare the $(K,K')$-spectrum
 of $\Attbb m m + {m,m}$
 (see \eqref{eqn:SAiitilde})
 and that of ${\operatorname{Rest}}$
 (see \eqref{eqn:SRest}) in Lemma \ref{lem:152268}.  
\end{proof}



\subsubsection
{${\operatorname{Hom}}_{G'}(I_\delta(m,m)|_{G'},J_\varepsilon(m,m))$
 with $\delta \varepsilon=-$}
\label{subsec:IJmm-}
The case $\delta \varepsilon=-$ is much simpler 
 because the space of symmetry breaking operators
 is one-dimensional:
\[
   {\operatorname{Hom}}_{G'}(I_\delta(m,m)|_{G'},J_\varepsilon (m,m))
   =
   {\mathbb{C}} \Atbb m m - {m,m}, 
\]
 see Theorem \ref{thm:SBObasis}.  
Without loss of generality,
 we may and do assume 
 $(\delta, \varepsilon)=(+,-)$.  
The restriction of the generator $\Atbb m m - {m,m}$
 to each irreducible component in \eqref{eqn:I+n2}
 is given as follows.  
\begin{proposition}
\label{prop:161756}
Let $(G,G')=(O(2m+1,1),O(2m,1))$.  
Then we have
\begin{align*}
\Atbb m m - {m,m}|_{\Pi_{m,+}}\equiv & 0.  
\\
{\operatorname{Image}}(\Atbb m m - {m,m}|_{\Pi_{m+1,-}}) = & \pi_{m,-}.  
\end{align*}
\end{proposition}

The proof of Proposition \ref{prop:161756} relies 
 on the functional equations given in Lemma \ref{lem:AmmT-}.  

\begin{proof}
The functional equation \eqref{eqn:ATAm-} implies
\[
   \Atbb m m - {m,m} \circ (\Tttbb m m m + c(m){\operatorname{id}})
   =0.   
\]
By Lemma \ref{lem:161745}, 
 ${\operatorname{Image}}(\Tttbb m m m + c(m){\operatorname{id}})
 =\Pi_{m,+}$.  
Hence the first statement is proved.  



The second statement follows from the functional equation \eqref{eqn:TATm-}
 and ${\operatorname{Ker}}(\Ttbb m {m-1} m: J_-(m,m) \to J_-(m,m-1))
 =\pi_{m,-}$
 (see Proposition \ref{prop:Timage}). 
\end{proof}

\subsection{Splitting of $J_{\varepsilon}(m,m)$
 and symmetry breaking operators for 
 $(G,G')=(O(2m+2,1),O(2m+1,1))$}
\label{subsec:2m+1}

Suppose $n$ is odd, 
say $n=2m+1$.  
In contrast to the $n$ even case treated
 in Section \ref{subsec:2m}, 
 a distinguished feature in this setting
 is that the principal series representation
 $J_{\varepsilon}(m,\nu)$ of the subgroup $G' =O(2m+1,1)$
 splits into the direct sum
 of two irreducible 
 tempered representations
 when $\nu=m$: for $\varepsilon\in \{\pm\}$, 
\begin{equation}
\label{eqn:JmmsplitI+n2}
J_{\varepsilon}(m,m) \simeq \pi_{m,\varepsilon} \oplus \pi_{m+1,-\varepsilon}, 
\end{equation}
 see Theorem \ref{thm:LNM20} (1) and (8).  
Accordingly, 
 the space of symmetry breaking operators has a direct sum decomposition:
\begin{multline}
\label{eqn:m+1mdeco}
{\operatorname{Hom}}_{G'}(I_{\delta}(i,\lambda)|_{G'}, J_{\varepsilon}(m, m))
\\
\simeq 
{\operatorname{Hom}}_{G'}(I_{\delta}(i,\lambda)|_{G'}, \pi_{m,\varepsilon})
\oplus 
{\operatorname{Hom}}_{G'}(I_{\delta}(i,\lambda)|_{G'},
 \pi_{m+1,-\varepsilon})
\end{multline}
for any $\lambda \in {\mathbb{C}}$.  
The left-hand side of \eqref{eqn:m+1mdeco} is understood
 via explicit generators
 given in Theorem \ref{thm:SBObasis} (classification).  
In this section,
 we examine the following two cases:
\index{A}{Ciiln@$\Cbb \lambda \nu {i,j}$, matrix-valued differential operator}
\begin{align}
\label{eqn:IJnoddmm}
{\operatorname{Hom}}_{G'}(I_{\delta}(m+1,m)|_{G'}, J_{\delta}(m, m))
= &
{\mathbb{C}}
\Attbb m m + {m+1,m}
\oplus 
{\mathbb{C}}
\Ctbb m m {m+1,m},
\\
\label{eqn:IJnoddmm-}
{\operatorname{Hom}}_{G'}(I_{-\varepsilon}(m,m)|_{G'}, J_{\varepsilon}(m, m))
= &
{\mathbb{C}} \Atbb m m -{m,m}, 
\end{align}
in connection with the decomposition 
 in the right-hand side of \eqref{eqn:IJnoddmm}.  



We retain the notation \eqref{eqn:cm}
 in the previous section,
 that is, 
\[
  c(m) = \frac{\pi^m}{m!}.  
\]
Then the irreducible $G'$-modules 
 $\pi_{m,\varepsilon}$ and $\pi_{m+1,-\varepsilon}$
 in \eqref{eqn:JmmsplitI+n2}
 are the eigenspaces
 of the renormalized Knapp--Stein operator
 $\Tttbb m m m$ for the subgroup $G'$
 with eigenvalues $c(m)$ and $-c(m)$, 
respectively, 
 by Lemma \ref{lem:161745}.   



The case \eqref{eqn:IJnoddmm} will be discussed
 in Section \ref{subsec:IJnoddmm}
 and the case \eqref{eqn:IJnoddmm-}
 in Section \ref{subsec:IJnoddmm-}.  
In particular,
 we shall see in Section \ref{subsec:SBOrho-}, 
 that both 
 ${\mathbb{A}}':= \frac 1 2 \Attbb m m + {m+1,m}+
 c(m) \Ctbb m m {m+1,m}$
 in \eqref{eqn:IJnoddmm}
 and $\frac 1 2 (-1)^{m+1}\Atbb m m - {m,m}$
 in \eqref{eqn:IJnoddmm-}
 yield the same symmetry breaking operator
\[
  A_{m+1,m} \colon \Pi_{m+1,\delta} \to \pi_{m,\delta}, 
\]
 which will be utilized in the construction 
 of nonzero periods in Chapter \ref{sec:period}, 
 see Theorem \ref{thm:period2}.  

\subsubsection
{${\operatorname{Hom}}_{G'}(I_{\delta}(m+1,m)|_{G'}, J_{\delta}(m,m))$
 for $n=2m+1$}
\label{subsec:IJnoddmm}
We recall from Theorem \ref{thm:161243} (2)
 that the regular symmetry breaking operator
 $\Atbb \lambda \nu +{i,j}$ vanishes
 when $(n,i,j, \lambda, \nu)=(2m+1,m+1,m,m,m)$, 
and therefore, 
 the left-hand side of \eqref{eqn:m+1mdeco} 
 at $\lambda=m$ is two-dimensional 
 by Theorem \ref{thm:1.1} (2).  
More precisely, 
 the classification
 of symmetry breaking operators
 given in Theorem \ref{thm:SBObasis} shows \eqref{eqn:IJnoddmm}.  



On the other hand, 
 we recall from Theorem \ref{thm:LNM20} (1)
 that the principal series representation $I_{\delta}(m+1,m)$ has
 a nonsplitting exact sequence of $G$-modules:
\[
 0 \to \Pi_{m+2,-\delta} \to I_{\delta}(m+1,m) \to \Pi_{m+1,\delta} \to 0.  
\]
The irreducible $G$-submodule $\Pi_{m+2,-\delta}$ is the image
 of the Knapp--Stein operator $\Ttbb {m+1}m {m+1}$ for the group $G$.  
With this in mind,
 we shall take a closer look at the right-hand side of 
 \eqref{eqn:m+1mdeco}.  



We introduce the following element in \eqref{eqn:IJnoddmm}:
\begin{equation}
\label{eqn:Aprimm}
  {\mathbb{A}}'
:=
  \frac 1 2 \Attbb m m + {m+1,m}
  +
  c(m) \Ctbb m m {m+1,m}.  
\end{equation}
The main result of this subsection is the following.  


\begin{proposition}
\label{prop:1802103}
Let $(G,G')=(O(2m+2,1),O(2m+1,1))$.  
Then ${\mathbb{A}}' \colon I_{\delta}(m+1,m) \to J_{\delta}(m,m)$ is 
 a symmetry breaking operator satisfying 
\begin{align*}
 {\mathbb{A}}'  \circ \Ttbb {m+1} m {m+1} =&0, 
\\
 \Tttbb m m {m,m} \circ {\mathbb{A}}' 
  =& c(m) {\mathbb{A}}',  
\\
 S({\mathbb{A}}' 
  )=& c(m) \begin{pmatrix} 1 & 0 \\ 0 & 0 \end{pmatrix}.  
\end{align*}
\end{proposition}



Proposition \ref{prop:1802103} follows from the corresponding results
 for the renormalized operator $\Attbb m m + {m+1,m}$
 (Lemma \ref{lem:reAm+1m} below)
 and for the differential operator $\Ctbb m m {m+1,m}$
 (Lemmas \ref{lem:KKspecC}, \ref{lem:TCAC}, and \ref{lem:CTAtilde}).  
We begin with the functional equations
 and the $(K,K')$-spectrum
 of the first generator 
 $\Attbb m m + {m+1,m}$ in \eqref{eqn:IJnoddmm}.  
\begin{lemma}
\label{lem:reAm+1m}
Retain the setting 
 where $(G,G')=(O(2m+2,1),O(2m+1,1))$.  
Then the renormalized regular symmetry breaking operator
 $\Attbb m m + {m+1,m}$ satisfies the following:
\begin{align*}
 \Attbb m m + {m+1,m} \circ \Ttbb {m+1} m {m+1} 
  =& -2 c(m) \Attbb {m+1}m+{m+1,m}, 
\\
 \Tttbb {m} m {m}  \circ \Attbb m m + {m+1,m}
 =& -c(m) \Attbb m m + {m+1,m}, 
\\
 S(\Attbb m m + {m+1,m})=& c(m) \begin{pmatrix} 0 & 0 \\ 0 & 2 \end{pmatrix}.  
\end{align*}
\end{lemma}

\begin{proof}
See Lemma \ref{lem:Aii-ren} for the first and third equalities, 
 and Lemma \ref{lem:TAT3tilde} for the second.  
\end{proof}



For the differential symmetry breaking operator
 $\Ctbb m m {m+1,m}$ in \eqref{eqn:IJnoddmm},
 we recall from \cite[Thm.~1.3]{xkresidue}
 the 
\index{B}{residueformula@residue formula}
 residue formula of the regular symmetry breaking operators
 $\Atbb \lambda \nu {\delta \varepsilon}{i,j}
 \colon I_{\delta}(i,\lambda) \to J_{\varepsilon}(j,\nu)$
 when 
\index{A}{1psi@$\Psising$,
          special parameter in ${\mathbb{C}}^2 \times \{\pm\}^2$}
$(\lambda, \nu,\delta, \varepsilon) \in \Psising$
 for $j=i-1$ and $i$, 
 see Fact \ref{fact:153316}.  
Applying \eqref{eqn:Aijres1} to 
 $(n, i,j,\lambda,\nu)=(2m+1,m+1,m,\lambda,\lambda)$, 
 we obtain 
\begin{equation}
\label{eqn:resAC2}
\Atbb \lambda \lambda + {m+1,m}
=
\frac{(m-\lambda) \pi^{m}}{\Gamma(\lambda+1)}\Ctbb \lambda \lambda {m+1,m}, 
\end{equation}
where we recall from \eqref{eqn:Cijln}
 that the differential symmetry breaking operator
 $\Cbb \lambda \nu {i,i-1}$ vanishes
 for the parameter
 that we are dealing with, 
 namely,
 when $\lambda=\nu=n-i$.  
So we use the renormalized operator 
 $\Ctbb \lambda \nu {i,i-1}$ instead.  
We note that 
$
   \Ctbb \lambda \lambda {i,i-1}
  ={\operatorname{Rest}}_{x_n=0}\circ \iota_{\frac\partial{\partial x_n}}
$.  



\begin{lemma}
\label{lem:KKspecC}
The $(K,K')$-spectrum of $\Ctbb m m {m+1,m}$ is given by
\[
   S(\Ctbb m m {m+1,m})
   = \begin{pmatrix} 1 & 0 \\ 0 & -1 \end{pmatrix}.  
\]
\end{lemma}
\begin{proof}
By the residue formula \eqref{eqn:resAC2}, 
 we have
\[
  \lim_{\lambda \to m}
  \frac 1 {\lambda-m} \Atbb \lambda \lambda + {m+1,m}
  = 
  -c(m)\Ctbb m m {m+1,m}.  
\]
Now the lemma follows from the $(K,K')$-spectrum
 of the regular symmetry breaking operator
 $\Atbb \lambda \nu {\pm} {i,j}$
 given in Theorem \ref{thm:153315}.  
\end{proof}



The symmetry breaking operator $\Atbb \lambda \nu + {m+1,m}$
 vanishes at $(\lambda,\nu)=(m,m)$.  
We recall from Lemma \ref{lem:upsemi}
 and Definition \ref{def:AVWder}
 that 
\[
  (\Atbb m m + {m+1,m})_{k,l}
  :=
   \left.
   \frac{\partial^{k+l}}{\partial \lambda^{k}\partial \nu^{l}}
   \right|
   _{\substack{\lambda=m \\ \nu=m}}
   \Atbb \lambda \nu + {m+1,m}
   \in {\operatorname{Hom}}_{G'}
   (I_{\delta}(m+1,m)|_{G'}, J_{\delta}(m,m))
\]
 for $(k,l)=(1,0)$ and $(0,1)$.  



The base change of the vector space
 ${\operatorname{Hom}}_{G'}
   (I_{\delta}(m+1,m)|_{G'}, J_{\delta}(m,m))$, 
 see \eqref{eqn:IJnoddmm}, 
 is given as follows.  

\begin{lemma}
\label{lem:twobases}
\begin{enumerate}
\item[{\rm{(1)}}]
$2 (\Atbb m m + {m+1,m})_{1,0}= \Attbb m m + {m+1,m}$, 
\item[{\rm{(2)}}]
$(\Atbb m m + {m+1,m})_{1,0} + (\Atbb m m + {m+1,m})_{0,1}
 = -c(m) \Ctbb m m {m+1,m}$.  
\end{enumerate}
\end{lemma}
\begin{proof}
The first assertion is immediate from the definition 
 of the renormalized operator 
 $\Attbb m m + {m+1,m}$, 
 see \eqref{eqn:Ader}.  
The second assertion follows from the residue formula 
 \eqref{eqn:resAC2}.  
\end{proof}
It follows from Lemma \ref{lem:twobases}
 that 
\begin{equation}
\label{eqn:AClimit}
  \lim_{\lambda \to m}
  \frac 1 {\lambda-m} \Atbb \lambda {2m-\lambda} + {m+1,m}
  = 
  \Attbb m m + {m+1,m}
  +c(m) \Ctbb m m {m+1,m}.  
\end{equation}

Now we give functional equations
 of the differential symmetry breaking operators
 $\Ctbb m m {m+1,m}$
 and the (renormalized) Knapp--Stein operators for $G'$ and $G$
 as follows.  


\begin{lemma}
\label{lem:TCAC}
$\Tttbb m m m \circ \Ctbb m m {m+1,m}
=
\Attbb m m + {m+1,m} + c(m) \Ctbb m m {m+1,m}$.  
\end{lemma}
\begin{proof}
By the functional equation in Theorem \ref{thm:TAA}, 
 we have
\[
 \Ttbb \lambda {2m-\lambda} {m}
 \circ 
 \Atbb \lambda \lambda + {m+1,m}
  = 
  \frac{(m-\lambda)\pi^{m}}{\Gamma(\lambda+1)}
  \Atbb \lambda {2m-\lambda} +  {m+1,m}.  
\]

Hence we get from the residue formula \eqref{eqn:resAC2}
\[
  \Ttbb \lambda {2m-\lambda} {m}
 \circ 
  \Ctbb \lambda \lambda {m+1,m}
  = 
  \Atbb \lambda {2m-\lambda} +  {m+1,m}.  
\]
Now Lemma \ref{lem:TCAC} follows from \eqref{eqn:AClimit}.  
\end{proof}



\begin{lemma}
\label{lem:CTAtilde}
$
\Ctbb m m {m+1,m} \circ \Ttbb {m+1} m {m+1} 
=
\Attbb {m+1} m + {m+1,m}.  
$
\end{lemma}
\begin{proof}
By the functional equation in Theorem \ref{thm:ATA}, 
 we have
\[
 \Atbb \lambda \lambda + {m+1,m}
 \circ 
 \Ttbb {2m+1-\lambda}\lambda {m+1}
 = 
  \frac{\pi^{m+\frac 1 2}(m-\lambda)}{\Gamma(\lambda+1)}
  \Atbb {2m+1-\lambda}\lambda +  {m+1,m}.  
\]
By the residue formula \eqref{eqn:resAC2}
 and by analytic continuation, 
 we get 
\[
  \Ctbb \lambda \lambda {m+1,m}
  \circ 
  \Ttbb {2m+1-\lambda} \lambda {m+1}
  = 
  \pi^{\frac 12}\Atbb {2m+1-\lambda}\lambda+{m+1,m}.  
\]
Since $\Attbb {m+1} m + {m+1,m} = \pi^{\frac 1 2} \Atbb {m+1} m + {m+1,m}$
 by the definition \eqref{eqn:Aijrenorm}
 of the renormalized operator $\Attbb {\lambda} \nu \pm {i,j}$, 
 the lemma is proved.  
\end{proof}


\subsubsection
{${\operatorname{Hom}}_{G'}(I_{-\varepsilon}(m,m)|_{G'}, J_{\varepsilon}(m,m))$
 for $n=2m+1$}
\label{subsec:IJnoddmm-}


In this subsection,
 we examine 
\[
  {\operatorname{Hom}}_{G'}
  (I_{-\varepsilon}(m,m)|_{G'}, J_{\varepsilon}(m,m))
 ={\mathbb{C}} \Atbb m m - {m,m}, 
\]
as stated in \eqref{eqn:IJnoddmm-}, 
 which is derived from Theorems \ref{thm:1.1} and \ref{thm:SBObasis}.  
We recall from Theorem \ref{thm:LNM20} (1)
 that there is a nonsplitting exact sequence of $G$-modules:
\[
   0 \to \Pi_{m,-\delta} \to I_{-\delta}(m,m) \to \Pi_{m+1,\delta} \to 0.  
\]
Concerning the regular symmetry breaking operator $\Atbb m m -{m,m}$, 
 we have the following.  
\begin{lemma}
\label{lem:IHP080328}
Let $(G,G')=(O(2m+2,1),O(2m+1,1))$.  
Then we have
\begin{align*}
   \Tttbb m m {m} \circ \Atbb m m - {m,m}
 =& c(m) \Atbb m m - {m,m}, 
\\
   \Atbb m m - {m,m} \circ \Ttbb {m+1} m {m} =&0, 
\\
   S(\Atbb m m - {m,m})=& 2(-1)^{m+1} c(m) \begin{pmatrix} 0 & 0 \\ 1 & 0\end{pmatrix}.  
\end{align*}
\end{lemma}

\begin{proof}
The proof of first formula parallels to that of Lemma \ref{lem:TAT3tilde}, 
 and the second formula is a special case of Theorem \ref{thm:ATA}.  
The third formula follows from Theorem \ref{thm:153315}.  
\end{proof}

\subsection{Symmetry breaking operators from $\Pi_{i,\delta}$ to $\pi_{i-1,\delta}$}
\label{subsec:SBOrho-}

In Sections \ref{subsec:StageCii} and \ref{subsec:2m}, 
 we constructed nontrivial symmetry breaking operators from 
 the irreducible representation $\Pi_{i,\delta}$ of $G=O(n+1,1)$
 to the irreducible one $\pi_{i,\delta}$ of $G'=O(n,1)$.  
This is sufficient for the proof of Theorem \ref{thm:SBOone}
 by the duality theorem
 (Proposition \ref{prop:SBOdual})
 between symmetry breaking operators
 for the indices:
\[
   \text{$(i,j)$ and $(\widetilde i,\widetilde j):=(n+1-i, n-j)$}.  
\]
Nevertheless,
 we give in this section an explicit construction
 of the normalized symmetry breaking operators
 $\Pi_{i,\delta} \to \pi_{j,\varepsilon}$
 also for $j=i-1$, 
 and determine their $(K,K')$-spectrum 
 of symmetry breaking operators from $\Pi_{i,\delta}$ to $\pi_{i-1,\delta}$.  
The results will be used in the computation
 of {\it{periods}} of admissible smooth representations 
 in Chapter \ref{sec:period}.  



We begin with some basic properties
 of the regular symmetry breaking operator
\[
   \Atbb \lambda \nu \delta {i,i-1} \colon
 I_\delta(i,\lambda) \to J_\delta (i-1,\nu)
\]
for $(\lambda,\nu)=(n-i,i-1)$.  

\begin{proposition}
\label{prop:Aii1ker}
Suppose $1 \le i \le n$ and $\delta \in \{\pm\}$.  
\begin{enumerate}
\item[{\rm{(1)}}]
$\Pi_{i+1,-\delta} \subset{\operatorname{Ker}}~(\Atbb {n-i}{i-1} +{i,i-1})$.  
\item[{\rm{(2)}}]
${\operatorname{Image}}~(\Atbb {n-i}{i-1}+{i,i-1}) \simeq \pi_{i-1,\delta}$
 if $n \ne 2i-1 ;$
$\Atbb {n-i}{i-1}+{i,i-1} =0$
 if $n = 2i-1$.  
\end{enumerate}
\end{proposition}
\begin{proof}
(1)\enspace
Applying the functional equation
 given in Theorem \ref{thm:ATA} with $\lambda=i$, 
 we see that the symmetry breaking operator $\Atbb {n-i}{\nu} \gamma{i,i-1}$ vanishes 
 on the image of the Knapp--Stein intertwining operator
 $\Ttbb i {n-i} i \colon I_{\delta}(i,i) \to I_{\delta}(i,n-i)$, 
 namely, 
 on the irreducible submodule $\Pi_{i+1,-\delta}$
 (see Theorem \ref{thm:LNM20} (1)).  
\newline\noindent
{\rm{(2)}}\enspace
By Theorem \ref{thm:161243}, 
 $\Atbb {n-i}{i-1}+{i,i-1}=0$
 if and only if $n=2i-1$.  



Suppose from now 
 that $n \ne 2i-1$.  
Applying the functional equation given in Theorem \ref{thm:TAA} with $(\lambda, \nu, \gamma)=(n-i,i-1,+)$, 
 we see that the composition 
 $\Ttbb {\nu} {n-1-\nu} {i-1} \circ \Atbb {\lambda}{\nu} +{i,i-1}$
 is a scalar multiple of the symmetry breaking operator $\Atbb {n-i}{n-i} +{i,i-1}$, 
 which vanishes by Theorem \ref{thm:161243}.  
In turn, 
 applying Proposition \ref{prop:Timage} to $G'=O(n,1)$, 
 we get
\[
    {\operatorname{Ker}}(\Ttbb \nu{n-1-\nu}{i-1} \colon
                         J_{\delta}(i-1,\nu) \to J_{\delta}(i-1,n-1-\nu))
    \simeq \pi_{i-1,\delta}
\]
 because $i-1 \ne \frac 1 2(n-1)$. 
Hence the second statement is also proved.  
\end{proof}



Since $\Atbb {n-i} {i-1}+ {i,i-1}=0$
 for $n=2i-1$, 
 we treat this case separately as follows.  
Suppose $n=2m+1$.  
We recall that there are a nonsplitting exact sequence of $G$-modules
\[
  0 \to \Pi_{m,-\delta} \to I_{-\delta}(m,m) \to \Pi_{m+1,\delta} \to 0
\]
 and a direct sum decomposition of irreducible $G'$-modules
\[
   J_{\delta}(m,m) \simeq \pi_{m,\delta} \oplus \pi_{m+1,-\delta}.  
\]
We use the following regular symmetry breaking operator
\[
  \Atbb m m - {m,m} \colon I_{-\delta}(m,m) \to J_{\delta}(m,m).  
\]
\begin{proposition}
\label{prop:IHP180328}
Suppose $(G,G')=(O(2m+2,1),O(2m+1,1))$
 and $\delta \in \{\pm\}$.  
\begin{enumerate}
\item[{\rm{(1)}}]
${\operatorname{Ker}}(\Atbb m m - {m,m}) \supset \Pi_{m,-\delta}$.  
\item[{\rm{(2)}}]
${\operatorname{Image}}(\Atbb m m - {m,m})=\pi_{m,\delta}$.  
\end{enumerate}
\end{proposition}
\begin{proof}
The assertions follow from Lemma \ref{lem:IHP080328}.  
\end{proof}

It follows from Proposition \ref{prop:Aii1ker}
 that if $n \ne 2i-1$ then the normalized symmetry breaking operator $\Atbb {n-i}{n-i} +{i,i-1}$
 yields a surjective $G'$-homomorphism
\begin{equation}
\label{eqn:Pipi-}
  A_{{i},{i-1}} \colon \Pi_{i,\delta} \to \pi_{i-1,\delta}
\end{equation}
by the following diagram.  
\begin{eqnarray*}
\xymatrix@C=16pt{
&&I_{\delta}(i,n-i)\ar[rr]^{\Atbb {n-i}{n-i}+{i,i-1}}
\ar@{}[dr]^\circlearrowleft
\ar@{->>}[d]
&&\pi_{i-1,\delta} \ar@{^{(}->}[rr]
&&J_{\delta}(i-1,i-1)
\\
\Pi_{i,\delta}
&&\ar[ll]_{\sim\qquad\quad} I_{\delta}(i,n-i)/\Pi_{i+1,-\delta}\ar@{-->}[urr]
&&
&&
}
\end{eqnarray*}
If $n=2i-1$, 
 we set $(n,i)=(2m+1,m+1)$.  
Then, 
similarly to the case $n \ne 2i-1$, 
 Proposition \ref{prop:1802103} shows
 that the symmetry breaking operator 
 ${\mathbb{A}}' \colon I_{\delta}(m+1,m) \to J_{\delta}(m,m)$ 
 defined in \eqref{eqn:Aprimm}
 yields a surjective $G'$-homomorphism
\begin{equation}
\label{eqn:Pmpm-}
   A_{m+1,m} \colon \Pi_{m+1,\delta} \to \pi_{m,\delta}
\end{equation}
 by the following diagram.  
\begin{eqnarray*}
\xymatrix@C=16pt{
&&I_{\delta}(m+1,m)\ar[rr]^{{\mathbb{A}}'} 
\ar@{->}[d]
\ar@{}[dr]^\circlearrowleft
&&\pi_{m,\delta} 
\subset J_{\delta}(m,m)
\\
\Pi_{m+1,\delta}
&&\ar[ll]_{\sim\quad\quad} I_{\delta}(m+1,m)/\Pi_{m+2,-\delta}\ar@{-->}[urr]
&&
}
\end{eqnarray*}

In order to define the $(K,K')$-spectrum,
 we need to fix an inclusive map from 
 the $K'$-type into the $K$-type,
 see Definition \ref{def:KKspec}.  
In our setting, 
 we use the natural embedding of the minimal $K$- and $K'$-types
\begin{equation}
\label{eqn:KKemb}
  \mub (i,\delta) \hookleftarrow \mub (i-1,\delta)'
\end{equation}
 of the irreducible representations $\Pi_{i,\delta}$ and $\pi_{i-1,\delta}$
 of $G$ and the subgroup $G'$, 
 respectively,
 as in Section \ref{subsec:Apm1-5}.  
Then we get the following formula
 for the 
\index{B}{KspectrumKprime@$(K,K')$-spectrum}
$(K,K')$-spectrum.  

\begin{proposition}
\label{prop:Aidown}
Let $(G, G')=(O(n+1,1), O(n,1))$ and $1 \le i \le n+1$. 
Then the symmetry breaking operator
\[
     A_{i,i-1} \colon \Pi_{i,\delta} \to \pi_{i-1,\delta}
\]
 acts on $\mub(i-1,\delta)'$ $(\hookrightarrow \mub(i,\delta))$
 as the following scalar:
\begin{equation*}
\begin{cases}
 \frac{\pi^{\frac{n-1}{2}}(n-2i+1)}{(n-i)!}
\qquad
 &\text{if $n \ne 2i-1$}, 
\\ 
 \frac{\pi^{\frac{n-1}{2}}}{(n-i)!}
  & \text{if $n=2i-1$}.
\end{cases}
\end{equation*}
\end{proposition}
\begin{proof}
For $n \ne 2i-1$, 
 the assertion follows directly from the $(1,1)$-component
 of the matrix $S(\Atbb{\lambda}{\nu}{+}{i,i-1})$
 in Theorem \ref{thm:153315}
 with $(\lambda,\nu)=(n-i,i-1)$.  
For $n=2i-1$, 
 the (1,1)-component of $S({\mathbb{A}}')$ in Proposition \ref{prop:1802103}
 with $(n,i)=(2m+1,m+1)$ shows the desired formula.  
\end{proof}

\begin{remark}
\label{rem:20180406}
When $n=2i-1$, 
 we set $(n,i)=(2m+1,m+1)$ as above.  
In this case we may use $\Atbb m m - {m,m}$ in Lemma \ref{lem:IHP080328}
 for an alternative construction 
 of $A_{m+1,m} \in {\operatorname{Hom}}_{G'}(\Pi_{m+1,\delta}|_{G'}, \pi_{m,\delta})$.  
To see this, 
 we recall from Section \ref{subsec:IJnoddmm-}
 the following natural inclusion 
\[
{\operatorname{Hom}}_{G'}(\Pi_{m+1,\delta}|_{G'}, \pi_{m,\delta})
\subset
 {\operatorname{Hom}}_{G'}(I_{-\delta}(m,m)|_{G'}, J_{\delta}(m,m))
 ={\mathbb{C}} \Atbb m m - {m,m}, 
\]
and therefore any element
 in 
$
   {\operatorname{Hom}}_{G'}(\Pi_{m+1,\delta}|_{G'}, \pi_{m,\delta})
$
 is proportional to the one
 which is induced from $\Atbb m m - {m,m}$.  
On the other hand, 
 Proposition \ref{prop:IHP180328} tells 
 that the symmetry breaking operator 
 $\Atbb m m - {m,m}$ yields
 a surjective $G'$-homomorphism
 $\Pi_{m+1,\delta} \to \pi_{m,\delta}$
 by the following diagram.  
\begin{eqnarray*}
\xymatrix@C=22pt{
&&I_{-\delta}(m,m)\ar[rr]^{\quad\Atbb m m -{m,m}}
\ar@{->>}[d]
\ar@{}[dr]^\circlearrowleft
&&\pi_{m,\delta} \ar@{^{(}->}[rr]
&&J_{\delta}(m,m)
\\
\Pi_{m+1,\delta}
&&\ar[ll]_{\sim\quad\quad} I_{-\delta}(m,m)/\Pi_{m,-\delta}\ar@{-->}[urr]
&&
&&
}
\end{eqnarray*}
By Lemma \ref{lem:IHP080328}, 
 $\frac 1 2 (-1)^{m+1} \Atbb m m - {m,m}$ has 
 the $(K,K')$-spectrum 
 for the basic $K$- and $K'$-types
\[
  S\left(\frac 1 2 (-1)^{m+1} \Atbb m m - {m,m}\right) = c(m) \begin{pmatrix} 0 & 0 \\ 1 & 0\end{pmatrix}.  
\]
In view of the (2,1)-component, 
 the resulting symmetry breaking operator from $\Pi_{m+1,\delta}$ to $\pi_{m,\delta}$
 has the $(K,K')$-spectrum $c(m)$
 for the embedding 
 of the $K$- and $K'$-types
 $\mub (m+1,\delta) \hookleftarrow \mub (m,\delta)'$.  
This is the same with the $(K,K')$-spectrum of $A_{m+1,m}$
 which is induced from 
 ${\mathbb{A}}' \in {\operatorname{Hom}}_{G'}(I_{\delta}(m+1,m)|_{G'}, J_{\delta}(m,m))$.  
Hence $\frac 1 2 (-1)^{m+1} \Atbb m m - {m,m}$ induces
 the same symmetry breaking operator with $A_{m+1,m}$.  
\end{remark}

