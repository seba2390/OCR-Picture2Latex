\newpage
%%%%%%%%%%%%%%%%%%%%%%%%%%%%%%%%%%%%%%%%%%%%%%%%%%%%%%%%%
\section{Minor summation formul{\ae}
 related to exterior tensor $\Exterior^i({\mathbb{C}}^n)$}
\label{sec:section9}
%%%%%%%%%%%%%%%%%%%%%%%%%%%%%%%%%%%%%%%%%%%%%%%%%%%%%%%%%

This chapter collects some combinatorial formul{\ae}, 
 which will be used in later chapters to compute the $(K,K')$-spectrum
 for symmetry breaking operators
 between differential forms 
 on spheres $S^n$ and $S^{n-1}$, 
 namely,
 between principal series representations
 $I_{\delta}(V,\lambda)$ of $G$
 and $J_{\varepsilon}(W,\nu)$ of its subgroup $G'$
 in the setting where $(V,W)=(\Exterior^i({\mathbb{C}}^n), \Exterior^j({\mathbb{C}}^{n-1}))$.  
%%%%%%%%%%%%%%%%%%%%%%%%%%%%%%%%%%%%%%%%%%%%%%%%%%%%%%%%%
\subsection{Some notation on index sets}
\label{subsec:index}
%%%%%%%%%%%%%%%%%%%%%%%%%%%%%%%%%%%%%%%%%%%%%%%%%%%%%%%%%%

Let $n$ be a positive integer.  
We shall use the following convention 
 of index sets:
\index{A}{Ini@${\mathfrak{I}}_{n,i}$, index set|textbf}
\begin{equation}
\label{eqn:Indexni}
{\mathfrak{I}}_{n,i}
:=\{I \subset \{1,\cdots,n\}
  : \# I =i\}.   
\end{equation}



\begin{convention}
\label{conv:index}
We use calligraphic uppercase letters 
 ${\mathcal {I}}$, ${\mathcal{J}}$
 instead of Roman uppercase letters $I$, $J$
 if the index set may contain 0.  
That is, 
if we write ${\mathcal{I}} \in {\mathfrak{I}}_{n+1,i}$, 
then 
\[
\text{${\mathcal{I}} \subset \{0,1,\cdots,n\}$
\,\, with \,\,$\# {\mathcal{I}} =i$.  
}
\]
\end{convention}
In later applications for symmetry breaking 
 with respect to $(G,G')=(O(n+1,1),O(n,1))$, 
 the notation ${\mathfrak{I}}_{n+1,i}$
 for subsets of $\{0,1,\cdots,n\}$ will be used
 when we describe the basis
 of the basic $K$-types and $K'$-types, 
 whereas the notation ${\mathfrak{I}}_{n,i}$, 
 ${\mathfrak{I}}_{n-1,i}$
 will be used 
 when we discuss representations of $M$ and $M'$, 
respectively.  



%%%%%%%%%%%%%%%%%%%%%%%%%%%%%%%%%%
\subsubsection{Exterior tensors $\Exterior^i({\mathbb{C}}^n)$}
%%%%%%%%%%%%%%%%%%%%%%%%%%%%%%%%%
Let $\{e_1, \cdots, e_n\}$ be the standard basis of ${\mathbb{C}}^n$.  
For $I = \{k_1,k_2,\cdots, k_i\}\in {\mathfrak{I}}_{n,i}$
 with $k_1 < k_2 < \cdots < k_i$,
 we set
\[
   e_I := e_{k_1} \wedge \cdots \wedge e_{k_i} \in {\Exterior}^i({\mathbb{C}}^n).  
\]
Then $\{e_I:I\in {\mathfrak{I}}_{n,i}\}$ forms
 a basis of the exterior tensor space $\Exterior^i({\mathbb{C}}^n)$.  
We define linear maps
\index{A}{prij@$\pr i j$, projection|textbf}
\[
\pr i j \colon {\Exterior}^i({\mathbb{C}}^n) \to {\Exterior}^j({\mathbb{C}}^{n-1}), 
\quad
(j=i-1,i)
\]
by 
\begin{align}
\label{eqn:Tii1}
\pr i i (e_I)=&
\begin{cases}
e_I \hspace{25mm} &\text{if}\ n \notin I, 
\\
0 \qquad &\text{if}\ n \in I, 
\end{cases}
\\
\label{eqn:Tii2}
\pr i {i-1} (e_I)=&
\begin{cases}
0  &\text{if}\ n \notin I, 
\\
(-1)^{i-1} e_{I \setminus \{n\}} \quad &\text{if}\ n \in I.  
\end{cases}
\end{align}

Then we have the direct sum decomposition 
\begin{equation}
\label{eqn:extij}
  {\Exterior}^i({\mathbb{C}}^n) \simeq {\Exterior}^i({\mathbb{C}}^{n-1})
                                     \oplus
                                     {\Exterior}^{i-1}({\mathbb{C}}^{n-1}).  
\end{equation}
%%%%%%%%%%%%%%%%%%%%%%%%%%%%%%%%%%
\subsubsection{Signatures for index sets}
%%%%%%%%%%%%%%%%%%%%%%%%%%%%%%%%%
Let $N \in {\mathbb{N}}_+$.  
In later sections, 
 $N$ will be $n-1$, $n$ or $n+1$.  

For a subset $I \subset \{1,\cdots, N\}$, 
 we define a signature $\varepsilon_I(k)$ by 
\index{A}{0epsilonI@$\varepsilon_I$|textbf}
\[
 \varepsilon_I(k)
 :=
\begin{cases}
 1
 \qquad
  &\text{if}\ k \in I, 
\\
 -1
\qquad
  &\text{if}\ k \not\in I, 
\end{cases}
\]
 and a quadratic polynomial 
\index{A}{QIb@$Q_I(b)$, quadratic polynomial|textbf}
$
Q_I(y)
$
 by 
\begin{equation}
\label{eqn:QI}
Q_I(y):= \sum_{\ell \in I} {y_{\ell}}^2
\qquad
\text{for }\,\, 
 y =(y_1, \cdots, y_N)\in {\mathbb{R}}^N.  
\end{equation}
We note that
\[
  2 Q_I(y) - |y|^2
  = 
  \sum_{k =1}^N \varepsilon_I(k) {y_k}^2.  
\]
For $I , J \subset {\mathfrak{I}}_{N,i}$, 
 we set
\[
|I-J|:= \# I - \#(I \cap J)=\# J - \#(I \cap J).  
\]
By definition,
 $|I-J|=0$ if and only if $I=J$;
 $|I-J|=1$
 if and only if there exist $K \in{\mathfrak{I}}_{N,i-1}$
 and $p$, $q \not \in K$
 with $p \ne q$ 
such that $I = K \cup \{p\}$
 and $J = K \cup \{q\}$.  



\begin{definition}
\label{def:sign}
For $I \subset \{1,2,\cdots,n\}$
 and $p,q \in {\mathbb{N}}$, 
 we set 
\begin{align*}
\index{A}{sgnIp@$\operatorname{sgn}(I;p)$|textbf}
\operatorname{sgn}(I;p)
:=&(-1)^{\# \{r \in I:r < p\}}, 
\\
\index{A}{sgnIpq@$\operatorname{sgn}(I;p,q)$|textbf}
\operatorname{sgn}(I;p,q)
:=&(-1)^{\# \{r \in I:\operatorname{min}(p,q)<r < \operatorname{max}(p,q)\}}.  
\end{align*}
\end{definition}

The following lemma is readily seen from the definition.  
\begin{lemma}
\label{lem:sgn}
For $I \subset \{1,2,\cdots,n\}$
 and $p,q \in {\mathbb{N}}$, 
 we have 
\begin{equation*}
\operatorname{sgn}(I;p)
\operatorname{sgn}(I;q)
=
\begin{cases}
\operatorname{sgn}(I;p,q)
\quad
&
\text{ if }\operatorname{min}(p,q) \notin I, 
\\
-\operatorname{sgn}(I;p,q)
\quad
&
\text{ if }\operatorname{min}(p,q) \in I.  
\end{cases}
\end{equation*}
\end{lemma}


For $y=(y_1,\cdots,y_N) \in {\mathbb{R}}^N$, 
 we define quadratic polynomials 
\index{A}{SIJ@$S_{I J}$|textbf}
$
S_{I J}(y)
$ 
by 
\begin{equation}
\label{eqn:SIJ}
S_{I J}(y)
:=
\begin{cases}
\sum_{k=1}^{N} \varepsilon_I(k) y_k^2
\quad
&\text{ if } I =J, 
\\
2 \operatorname{sgn}(K;p,q) y_p y_q
\quad
&\text{ if } I = K \cup \{p\}, J = K \cup \{q\}, 
\\
0
\quad
&\text{ if } |I- J| \ge 2, 
\\
\end{cases}
\end{equation}
where we write $I=K \cup \{p\}$
 and $J=K \cup \{q\}$
 $(p \ne q)$
 when $|I - J|=1$.  

It is convenient to set

\begin{equation}
\label{eqn:Sempty}
S_{\emptyset \emptyset}(y)=-\sum_{k=1}^N y_k^2.  
\end{equation}

%%%%%%%%%%%%%%%%%%%%%%%%%%%%%%%%%%%%%%%%%%%%%%%%%%%%%%%%%%
\subsection{Minor determinant 
 for $\psi:{\mathbb{R}}^N \setminus \{0\}\to O(N)$
}
\label{subsec:psiN}
%%%%%%%%%%%%%%%%%%%%%%%%%%%%%%%%%%%%%%%%%%%%%%%%%%%%%%%%%%%

We introduce the following map:
\index{A}{1psinl@$\psi_n(\cdot ;\lambda)$}
\begin{equation}
\label{eqn:psilmd}
\psi_N \colon \mathbb{R}^N \times {\mathbb{C}}
        \to M(N,{\mathbb{C}}),
\quad
(y; \lambda) \mapsto I_N - \lambda\, y \, {}^t \! y.  
\end{equation}
Here we have used a similar notation to the map $\psi_N(y)$ defined in \eqref{eqn:psim}.  
In fact,
 the map \eqref{eqn:psilmd} may be thought of as an extension
 of the previous one,
 since its special value 
 at $\lambda = \dfrac{2}{|y|^2}$ recovers \eqref{eqn:psim} by 
\begin{equation}
   \psi_N(y)=\psi_N(y;\frac{2}{|y|^2})
\qquad
\text{for $y \in {\mathbb{R}}^N \setminus \{0\}$}.  
\end{equation}
  



For $I,J \subset \{1,2,\cdots,N\}$
 with $\# I= \# J$, 
 the minor determinant
 of $A=(A_{i j})_{1 \le i, j \le N} \in M(N,{\mathbb{R}})$
 is denoted by 
\[
\det A_{I J}:=\det(A_{i j})_{\substack{i \in I \\ j \in J}}.  
\]
Then the exterior representation
\[
  \sigma:O(N) \to GL_{\mathbb{C}}(\Exterior^k ({\mathbb{C}}^N))
\]
is given by
\begin{equation}
\label{eqn:exrep}
\sigma(A) e_J 
=
\sum_{J' \in {\mathfrak{I}}_{N, k}}
(\det A)_{J' J}e_{J'}.  
\end{equation}
It follows from \eqref{eqn:exrep}
 that for $A, B \in O(N)$
 we have
\begin{equation}
\label{eqn:klminor}
\det(A B)_{J J'}
=\sum_{J'' \in {\mathfrak{I}}_{N, j}}
(\det A)_{J J''}
(\det B)_{J'' J'}
\end{equation}



\begin{lemma}
\label{lem:psidet}
Suppose $I$, $J \subset \{1,\cdots,N\}$
 with $\# I = \# J$.  
\par\noindent
{\rm{(1)}}\enspace
For $(y;\lambda) \in {\mathbb{R}}^N \times {\mathbb{C}}$,
\[
  \det \psi_N(y; \lambda)_{IJ}
  =
  \begin{cases}
  1 -\lambda Q_I(y)
  \quad
  &\text{if } I=J,
  \\
  -\lambda \operatorname{sgn}(K;p,q) y_p y_q
  &\text{if } I=K \cup \{p\}, \, J = K \cup \{q\},
  \\
  0
  &\text{if } |I-J| \ge 2.  
  \end{cases}
\]
\par\noindent
{\rm{(2)}}\enspace
For $y \in {\mathbb{R}}^N \setminus \{0\}$,
\index{A}{1psin@$\psi_n$}
\index{A}{0epsilonI@$\varepsilon_I$}
\index{A}{SIJ@$S_{I J}$}
\begin{align*}
  \det \psi_N(y)_{I J}
=&
  -\frac {1}{|y|^2} S_{I J}(y)
\\
=&
 \frac{1}{|y|^2} \times
 \begin{cases}
  -\sum_{l =1}^{N} \varepsilon_I(l) y_l^2 
 \quad
  &\text{if } I=J,
  \\
  -2 \operatorname{sgn}(K;p,q) y_p y_q
  &\text{if } I=K \cup \{p\}, J=K \cup \{q\},
  \\
  0
  &\text{if } |I-J| \ge 2.  
  \end{cases}
\end{align*}
\end{lemma}

\begin{proof}
(1) \enspace
Suppose $I=J$.  
Since the symmetric matrix $y \ {}^{t\!}y$ is of rank 1,
 its characteristic polynomial has zeros of order $N-1$:
\[
    \det (\mu I_N - y \ {}^{t\!}y)
   =\mu^N-\mu^{N-1}(\operatorname{Trace} y \ {}^{t\!}y)
   =\mu^N-\mu^{N-1}\sum_{j=1}^{N} y_j^2, 
\]
 and therefore
\[
    \det (I_N - \lambda y \ {}^{t\!}y)= 1- \lambda \sum_{j=1}^{N}y_j^2.  
\]
Applying this to the principal minor
 of size $\# I$,
 we get the first formula.  

\par\indent
Next suppose $|I-J|=1$.  
We may write as $I=K \cup \{p\}$, $J=K \cup \{q\}$.  
Then the $q$-th column vector of the minor matrix
 $(I_N - \lambda y \ {}^{t\!}y)_{IJ}$
 is of the form $-\lambda y_q(y_i)_{i\in I}$.  
Adding this vector multiplied by the scalar
 $(- y_j/y_q)$ to the $j$-th column vector
 for $j \in J \setminus \{q\}$,
 we get 
\begin{align*}
  \det \psi_N(y; \lambda)_{I J}
  =&
  \operatorname{sgn}(K;q) 
  \det (-\lambda y_q(y_i)_{i \in I}, (\delta_{i j})_{\substack{i \in I \\ j \in K}})
\\
=&
-\lambda \operatorname{sgn}(K;p)
 \operatorname{sgn}(K;q)
 y_p y_q.  
\end{align*}
Hence the second formula follows from Lemma \ref{lem:sgn}.  
The third one is proved similarly.  
\par\noindent
(2)\enspace
Substitute $\lambda=\frac{2}{|y|^2}$.  
\end{proof}
As a special case of Lemma \ref{lem:psidet} (2) with $N=n+1$, 
 we have the following:
\begin{lemma}
\label{lem:psid}
For ${\mathcal{I}}, {\mathcal{J}} \in {\mathfrak{I}}_{n+1,i}$
 and $b \in {\mathbb{R}}^n$, 
 we have
\[
 \det \psi_{n+1}(1,b)_{{\mathcal{I}} {\mathcal{J}}}
 =
\frac{-1}{1+|b|^2} S_{{\mathcal{I}} {\mathcal{J}}}(1,b).  
\]
Here $(1,b):=(1,b_1, \cdots, b_n) \in {\mathbb{R}}^{n+1}$.  
\end{lemma}




%%%%%%%%%%%%%%%%%%%%%%%%%%%%%%%%%%%%%%%%%%%%%%%%%%%%%%%%%%%%%%%%%%%%%%%%%%%%%%
\subsection{Minor summation formul{\ae}}
\label{subsec:minorsum}
%%%%%%%%%%%%%%%%%%%%%%%%%%%%%%%%%%%%%%%%%%%%%%%%%%%%%%%%%%%%%%%%%%%%%%%%%%%%%%


We collect minor summation formul{\ae} 
 that we shall need
 in computing the $(K,K')$-spectrum of symmetry breaking operators
 for \lq\lq{basic $K$-types}\rq\rq.  



We recall from \eqref{eqn:QI}
 that $Q_I(b)=\sum_{k \in I} b_k^2$.  
\begin{lemma}
\label{lem:msum}
Suppose $I \in {\mathfrak{I}}_{n,i}$.  
For $b \in {\mathbb{R}}^n$ and $s,t \in {\mathbb{C}}$, 
 we have:
\begin{alignat}{2}
\label{eqn:msum}
&{\text{\rm{(1)}}}\,\,
&&\sum_{J \in {\mathfrak{I}}_{n,i}}
\det \psi_n(b;s)_{I J}
\det \psi_n(b;t)_{I J}
=
1-(s+t)Q_I(b) + s t |b|^2 Q_I(b).  
\\
&{\text{\rm{(2)}}}
&&\sum_{J \in {\mathfrak{I}}_{n,i}}
\det \psi_{n+1}(1,b;s)_{I\cup \{0\},  J\cup \{0\}}
\det \psi_n(b;t)_{I J}
\notag
\\
&
&&\qquad\qquad\qquad\qquad
=
1- s -(s+t-st)Q_I(b) + s t |b|^2 Q_I(b).  
\label{eqn:msum2}
\\
\label{eqn:msum3}
&{\text{\rm{(3)}}}
&&
\sum_{J \in {\mathfrak{I}}_{n,i}}
  \det \psi_{n+1}(1,b;s)_{I J}
\,
  \det \psi_n(b;t)_{I J}
  =
   1-(s+t) Q_I(b) + s t |b|^2 Q_I(b).  
\end{alignat}
\end{lemma}
\begin{proof}
(1)\enspace
By Lemma \ref{lem:psidet}, 
the left-hand side is equal
 to 
\begin{align*}
&(1-s Q_I(b))(1-t Q_I(b))
+\sum_{\substack{k \in I\\ l \notin I}} s t b_k^2 b_l^2
\\
=&
1-(s+t)Q_I(b) + s t Q_I(b)^2 + st Q_I(b)(|b|^2-Q_I(b)), 
\end{align*}
whence the equation \eqref{eqn:msum}.  
\par\noindent
(2)\enspace
By Lemma \ref{lem:psidet}, 
the left-hand side is equal
 to 
\[
(1-s (1+Q_I(b)))(1-t Q_I(b))
+s t \sum_{\substack{k \in I\\ l \notin I}} b_k^2 b_l^2, 
\]
whence the equation \eqref{eqn:msum2}.  
\par\noindent
(3)\enspace
By Lemma \ref{lem:psidet}, 
 the left-hand side is equal to 
\begin{align*}
   &(1-s Q_I(b))(1-t Q_I(b))
   + s t \sum_{\substack{k \in I\\ l \notin I}} b_k^2 b_l^2
\\
   =&
   1-(s+t) Q_I(b) + s t Q_I(b)^2 + s t Q_I(b) (|b|^2-Q_I(b)), 
\end{align*}
whence the equation \eqref{eqn:msum3}.  
\end{proof}



The following proposition 
 will be used in obtaining the closed formul{\ae}
 of the $(K,K')$-spectrum
 of the Knapp--Stein intertwining operators (Proposition \ref{prop:TminK})
 and the ones of the regular symmetry breaking operators
 (Theorem \ref{thm:153315}).  
\begin{proposition}
\label{prop:msum}
For $I \in {\mathfrak{I}}_{n,i}$, 
we have:
\begin{alignat}{2}
&\text{\rm{(1)}}
&&\sum_{J \in {\mathfrak{I}}_{n,i}}
\det \psi_n(b;\frac{2}{1+|b|^2})_{I J}
\det \psi_n(b)_{I J}
=
1- \frac{2 Q_I(b)}{(1+|b|^2) |b|^2}.  
\notag
\\
&\text{\rm{(2)}}
&&\sum_{J \in {\mathfrak{I}}_{n,i}}
\det \psi_{n+1}(1,b)_{I\cup \{0\},  J\cup \{0\}}
\det \psi_n(b)_{I J}
=
\frac{-1+|b|^2}{1+|b|^2} + \frac{2 Q_I(b)}{(1+|b|^2) |b|^2}.  
\notag
\\
&\text{\rm{(3)}}
&&\sum_{J \in {\mathfrak{I}}_{n,i}}
\left(\det \psi_n(b;\frac{2}{1+|b|^2})_{I J}
+
\det \psi_{n+1}(1,b)_{I\cup \{0\},  J\cup \{0\}}
\right)
\det \psi_n(b)_{I J}
=
\frac{2 |b|^2}{1+|b|^2}.  
\notag
\\
\label{eqn:kbsum}
&\text{\rm{(4)}}
&&\sum_{J \in {\mathfrak{I}}_{n,i}} \det \psi_{n+1}(1,b)_{I J}
\det \psi_n(b)_{I J}
=
1-\frac{2Q_I(b)}{(1+|b|^2)|b|^2}.  
\end{alignat}
\end{proposition}

\begin{proof}
The assertions (1), (2), and (4) are special cases
 of Lemma \ref{lem:msum} (1), (2), and (3), 
 respectively,
 with $s = \dfrac{2}{1+|b|^2}$ and $t=\dfrac {2}{|b|^2}$.  
The third one follows from the first two.  
\end{proof}

\begin{lemma}
\label{lem:S0n}
For $I \in {\mathfrak{I}}_{n-1,i-1}$, 
\[
  \sum_{J \in {\mathfrak{I}}_{n,i}}
  \det \psi_{n+1}(1,b)_{I \cup \{0\},  J}
  \det \psi_{n}(b)_{I \cup \{n\},  J}
  =
  \frac{2(-1)^{i+1} b_n}{1+|b|^2}.  
\]
\end{lemma}

\begin{proof}
Since $0 \notin J$, 
the summand vanishes
 except for the following two cases:
\par\noindent
Case 1)\enspace $J=I \cup \{ n \}$.  
\par\noindent
Case 2)\enspace $J=I \cup \{ p \}$
  for some $p \in \{1,2,\cdots,n-1\} \setminus I$.  



By Lemma \ref{lem:psidet}, 
 we get 
\begin{align*}
&  (1+|b|^2)|b|^2
  \sum_{J \in {\mathfrak{I}}_{n,i}}
  \det \psi_{n+1}(1,b)_{I \cup \{0\},  J}
\ \det \psi_{n}(b)_{I \cup \{n\},  J}  
\\
=& (-2 \operatorname{sgn}(I;0,n) b_n)
   (|b|^2 -2Q_{I}(b)-2b_n^2)
\\
  &+
  \sum_{p \in \{1,2,\cdots,n-1\}\setminus I}
  (-2 \operatorname{sgn}(I;0,p) b_p)
  (-2 \operatorname{sgn}(I;p,n) b_p b_n)
\\
=&2(-1)^{i+1} b_n(2 Q_{I}(b)+2b_n^2-|b|^2)
  +
  4(-1)^{i+1}b_n (|b|^2 - Q_{I}(b) -b_n^2)
\\
=& 2(-1)^{i+1} |b|^2 b_n.  
\end{align*}
Hence Lemma \ref{lem:S0n} is proved.  
\end{proof}
\begin{lemma}
\label{lem:Sn0}
For $I \in {\mathfrak{I}}_{n-1,i}$, 
\[
  \sum_{J \in {\mathfrak{I}}_{n,i}}
  \det \psi_{n+1}(1,b)_{I \cup \{n\}, J \cup \{0\}} \det \psi_n(b)_{I J}
=
 \frac{2(-1)^{i+1} b_n}{1+|b|^2}.  
\]
\end{lemma}
\begin{proof}
Since $0 \notin I$, 
$| (I \cup \{n\}) - (J \cup \{0\}) |\le 1$
 holds 
 in the following two cases:
\par\noindent
Case 1. \enspace
$I=J$.
\par\noindent
Case 2. \enspace
$I=K \cup \{p\}$
 and $J=K \cup \{n\}$
 for some $K \in {\mathfrak{I}}_{n-1,i-1}$.  
\par
In Case 1, 
\[
   \det \psi_{n+1}(1,b)_{I \cup \{n\}, J \cup \{0\}} 
   \det \psi_n(b)_{I I}
=
 \frac{2(-1)^{i+1} b_n}{1+|b|^2}
 \times 
(1-\frac{2 Q_I(b)}{|b|^2}).  
\]
\par
In Case 2, 
\begin{align*}
  &\det \psi_{n+1}(1,b)_{K \cup \{ p, n\}, K \cup \{ 0, n \}}
   \det \psi_{n}(b)_{K \cup \{ p \}, K \cup \{ n \}}
\\
&= \frac{-2 {\operatorname{sgn}}(K \cup \{n\};0,p)b_p}{1+|b|^2}
   \times 
   \frac{-2 {\operatorname{sgn}}(K;p,n)b_p b_n}{|b|^2}
\\
&= (-1)^{i-1} \frac{4}{(1+|b|^2) |b|^2} b_p^2 b_n.  
\end{align*}
Adding the term in Case 1 and taking the summation of the terms
 over $p \in I$ in Case 2, 
 we get the lemma.  
\end{proof}

