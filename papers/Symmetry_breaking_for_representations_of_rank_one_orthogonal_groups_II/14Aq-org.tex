\newpage
\section{Appendix I: Irreducible representations of $G=O(n+1,1)$, 
$\theta$-stable parameters,  and cohomological induction}
\label{sec:aq}
In Appendix I,
 we give a classification
 of irreducible admissible representations
 of $G=O(n+1,1)$
 in Theorem \ref{thm:irrG}.  
In particular, 
 we give a number of equivalent descriptions
 of irreducible representations
 with integral infinitesimal character
 (Definition \ref{def:intreg})
 by means of Langlands quotients
 (or subrepresentations), 
 coherent continuation starting at
\index{A}{1Piidelta@$\Pi_{i,\delta}$, irreducible representations of $G$}
 $\Pi_{i,\delta}$, 
 and cohomologically induced representations from 
 finite-dimensional representations of $\theta$-stable parabolic subalgebras,
 see Theorem \ref{thm:1808116}.  
Our results include a description
 of the following irreducible representations:
\begin{enumerate}
\item[$\bullet$]
\index{B}{Hassesequence@Hasse sequence}
\lq\lq{Hasse sequence}\rq\rq\ starting with arbitrary finite-dimensional 
irreducible representations
 (Theorems \ref{thm:171471} and \ref{thm:171471b});
\item[$\bullet$]
\index{B}{complementaryseries@complementary series representation}
complementary series representations
 with
\index{B}{singularintegralinfinitesimalcharacter@singular integral infinitesimal character}
 singular integral infinitesimal character
 (Theorem \ref{thm:compint}).  
\end{enumerate}
Since the Lorentz group $G=O(n+1,1)$ has four connected components, 
 we need a careful treatment
 even in dealing with finite-dimensional representations
 because not all of them extend holomophically 
 to $O(n+2,{\mathbb{C}})$.  
Thus Appendix I starts with irreducible finite-dimensional representations
 (Section \ref{subsec:fdimrep}), 
 and then discuss infinite-dimensional admissible representations
 for the rest of the chapter.  


\subsection{Finite-dimensional representations of $O(N-1,1)$}
\label{subsec:fdimrep}
In this section we give a parametrization
 of irreducible finite-dimensional representations
 of the disconnected groups $O(N-1,1)$ and $O(N)$.   
The description here fits well with the $\theta$-stable parameters
 (Definition \ref{def:thetapara})
 for the Hasse sequence, 
 see Theorem \ref{thm:171471}.  
We note
 that the parametrization here 
 for irreducible finite-dimensional representations
 of $O(N)$ is different from what was defined in Section \ref{subsec:ONWeyl},  
 although the \lq\lq{dictionary}\rq\rq\
 is fairly simple,
 see Remark \ref{rem:FOn}.  



There are two connected components in the compact Lie group $O(N)$.  
We recall from Definition \ref{def:type}
 that the set of equivalence classes
 of irreducible finite-dimensional representations
 of the orthogonal group $O(N)$
 can be divided into two types,
 namely,
 type I and II.  
On the other hand,
 there are four connected components in the noncompact Lie group $O(N-1,1)$, 
 and the division into two types 
 is not sufficient for the classification 
 of irreducible finite-dimensional representations of $O(N-1,1)$.  
We observe
 that some of the irreducible finite-dimensional representations
 of $O(N-1,1)$ cannot be extended to holomorphic representations
 of $O(N,{\mathbb{C}})$. 
For example,
 neither the one-dimensional representation $\chi_{+-}$ nor $\chi_{-+}$
 of $O(N-1,1)$ (see \eqref{eqn:chiab}) comes from a holomorphic character
 of $O(N,{\mathbb{C}})$.  
We shall use only representations
 of 
 \lq\lq{type I}\rq\rq\
 and tensoring them with four characters
 $\chi_{ab}$ ($a,b \in \{\pm\}$)
 to describe all irreducible finite-dimensional representations of $O(N-1,1)$.  


First of all, 
 we recall from \eqref{eqn:Lambda}
 that 
\index{A}{0tLambda@$\Lambda^+(N)$, dominant weight}
$\Lambda^+(k)$ is the set of 
 $\lambda \in {\mathbb{Z}}^k$
 with $\lambda_1 \ge \lambda_2 \ge \cdots \ge \lambda_k
\ge 0$.  



Let $N \ge 2$. 
For $\lambda \in \Lambda^+([\frac N 2])$, 
 we extend it to 
\begin{equation}
\label{eqn:lmdtilde}
   \widetilde \lambda 
 :=(\lambda_1, \cdots, \lambda_{[\frac N 2]}, 
    \underbrace{0,\cdots,0}_{[\frac{N+1}{2}]}) \in {\mathbb{Z}}^N, 
\end{equation}
 and define 
\index{A}{FONCl@$\Kirredrep{O(N,{\mathbb{C}})}{\lambda}$|textbf}
\begin{equation}
\label{eqn:Fholohw}
\Kirredrep {O(N,{\mathbb{C}})}{\lambda}_+
\equiv
\Kirredrep{O(N,{\mathbb{C}})}{\widetilde \lambda}, 
\end{equation}
to be the unique irreducible summand
 of $O(N,{\mathbb{C}})$ in the irreducible finite-dimensional representation 
 $\Kirredrep{G L(N,{\mathbb{C}})}{\widetilde \lambda}$
 of $GL(N,{\mathbb{C}})$
 that contains a highest weight vector
 corresponding to $\widetilde \lambda$, 
 see \eqref{eqn:CWOn}.  
Its restriction to the real forms $O(N)$ and $O(N-1,1)$
 will be denoted by $F^{O(N)}(\lambda)_+$ and $F^{O(N-1,1)}(\lambda)_{+,+}$, 
 respectively.  
Then the irreducible $O(N)$-module $\Kirredrep {O(N)}{\lambda}_+$ 
 is a representation of type I.  
We may summarize these notations
 as follows.  
\begin{equation}
\label{eqn:ONCreal}
   \Kirredrep {O(N)}{\lambda}_+
   \underset{{\operatorname{rest}}_{O(N)}}{\overset \sim \longleftarrow}
   \Kirredrep {O(N, {\mathbb{C}})}{\widetilde{\lambda}}
   \underset{{\operatorname{rest}}_{O(N-1,1)}}{\overset \sim \longrightarrow}
   \Kirredrep {O(N-1,1)}{\lambda}_{+,+}.  
\end{equation}
\begin{remark}
\label{rem:FOn}
With the notation as in \eqref{eqn:CWOn}, 
 we have
\[
   \Kirredrep {O(N)}{\lambda}_+
   \simeq
   \Kirredrep {O(N)}{\widetilde{\lambda}}
\]
for $\lambda \in \Lambda^+([\frac N 2])$.  
This is a general form 
 of representations of $O(N)$ of type I
 (Definition \ref{def:type}).  
Then other representations of $O(N)$, 
 {\it{i.e.}}, 
 representations of type II are obtained from the tensor product of those of type I
 with the one-dimensional representation, 
 $\det$, 
 as we recall now.  

Suppose $0 \le 2 \ell \le N$.  
If $\lambda \in \Lambda^+([\frac N 2])$ is of the form 
\[
   \lambda =(\lambda_1, \cdots, \lambda_{\ell}, 
             \underbrace{0, \cdots,0}_{[\frac N 2]-{\ell}})
\]
with $\lambda_{\ell} >0$, 
then by \eqref{eqn:type1to2}, 
 we have an isomorphism
 as representations of $O(N)$:
\[
   \Kirredrep {O(N)}{\lambda}_+ \otimes \det
   \simeq
   \Kirredrep {O(N)}{\underbrace{\lambda_1, \cdots,\lambda_{\ell}}_{\ell}, 
   \underbrace{1, \cdots,1}_{N-2{\ell}}, \underbrace{0, \cdots,0}_\ell}, 
\]
which is of type II
 if $N \ne 2\ell$.  
We shall denote this representation by $\Kirredrep {O(N)}{\lambda}_-$
 as \eqref{eqn:Fnpm} below.  
\end{remark}

Analogously, 
 $\Kirredrep {O(N-1,1)}{\lambda}_{+,+}$ is a general form
 of representations of the Lorentz group $O(N-1,1)$
 of type I in the following sense.  
\begin{definition}
[representation of type I for $O(N-1,1)$]
\label{def:typeone}
An irreducible
\newline
 finite-dimensional representation
 of $O(N-1,1)$ is said
 to be of {\it{type I}}
\index{B}{type1indef@type I, representation of $O(N-1,1)$|textbf}
 if it is obtained as the holomorphic continuation
of an irreducible representation
 of $O(N)$
 of type I
\index{B}{type1@type I, representation of $O(N)$}
 (see Definition \ref{def:type}).  
\end{definition}

We define for $\lambda \in \Lambda^+([\frac N2])$
\index{A}{FONpml@$\Kirredrep{O(N)}{\lambda}_{\pm}$|textbf}
\index{A}{FONpmpml@$\Kirredrep{O(N-1,1)}{\lambda}_{\pm\pm}$|textbf}
\begin{align}
  \Kirredrep {O(N)}{\lambda}_-
:=&
  \Kirredrep{O(N)}{\lambda}_+ \otimes \det, 
\label{eqn:Fnpm}
\\
  \Kirredrep {O(N-1,1)}{\lambda}_{a,b}
:=&
  \Kirredrep {O(N-1,1)}{\lambda}_{+,+} \otimes \chi_{ab}
\quad
(a,b \in \{\pm\}).  
\label{eqn:Fn1ab}
\end{align}
These are irreducible representations
 of $O(N)$ and $O(N-1,1)$, 
respectively.  



With the notation \eqref{eqn:Fnpm} and \eqref{eqn:Fn1ab}, 
 irreducible finite-dimensional representations
 of $O(N)$ and of $O(N-1,1)$, 
 respectively,
 are described as follows:
\begin{lemma}
\label{lem:161612}
\begin{enumerate}
\item[{\rm{(1)}}]
Any irreducible finite-dimensional representation 
 of $O(N)$
 is of the form 
$
    \text{$F^{O(N)}(\lambda)_+$ or $F^{O(N)}(\lambda)_-$ for some $\lambda \in \Lambda^+([\frac N 2]).$}
$
\item[{\rm{(2)}}]
Suppose $N \ge 3$.  
Any irreducible finite-dimensional representation of $O(N-1,1)$
 is of the form
 $\Kirredrep {O(N-1,1)}{\lambda}_{a,b}$ 
for some $\lambda \in \Lambda^+([\frac N 2])$
 and $a, b \in \{\pm\}$.  
\index{A}{1chipmpm@$\chi_{\pm\pm}$, one-dimensional representation of $O(n+1,1)$}
\end{enumerate}
\end{lemma}

The point of Lemma \ref{lem:161612} (2) is 
 that an analogous statement of Weyl's unitary trick may fail
 for the disconnected group $O(N-1,1)$, 
 that is, 
 not all irreducible finite-dimensional representations
 of $O(N-1,1)$ cannot extend to holomorphic representations
 of $O(N, {\mathbb{C}})$.  
\begin{proof}
[Proof of Lemma \ref{lem:161612}]
(1)\enspace
This is a restatement of Weyl's description \eqref{eqn:CWOn}
 of $\widehat{O(N)}$.  
\par\noindent
(2)\enspace
Take any irreducible finite-dimensional representation
 $\sigma$ of $O(N-1,1)$.  
By the Frobenius reciprocity,
 $\sigma$ occurs as an irreducible summand of the induced representation 
 ${\operatorname{Ind}}_{SO_0(N-1,1)}^{O(N-1,1)}(\sigma|_{SO_0(N-1,1)})$.  
Since $N \ge 3$, 
 the fundamental group of $SO(N,{\mathbb{C}})/SO_0(N-1,1)$ is trivial
 because it is homotopic to 
 $SO(N)/SO(N-1) \simeq S^{N-1}$, 
 see \cite[Lem.~6.1]{xkpro89}.  
Hence the irreducible finite-dimensional representation $\tau$
 of $SO_0(N-1,1)$ extends to a holomorphic representation 
 of $SO(N,{\mathbb{C}})$, 
 which we shall denote by $\tau_{\mathbb{C}}$.  



Let $\lambda \in \Lambda^+([\frac N 2])$ be the highest weight
 of the irreducible $SO(N,{\mathbb{C}})$-module $\tau_{\mathbb{C}}$.  
Then $\tau_{\mathbb{C}}$ occurs in the restriction 
 $\Kirredrep {O(N,{\mathbb{C}})} {\widetilde \lambda}|_{SO(N,{\mathbb{C}})}$, 
 and therefore the $SO_0(N-1,1)$-module $\tau$ occurs 
 in the restriction 
 ${\Kirredrep{O(N-1,1)}{\lambda}}_{+,+}|_{SO(N-1,1)}$.  
Hence $\sigma$ occurs as an irreducible summand of the induced representation
\begin{equation}
\label{eqn:IndSO0}
{\operatorname{Ind}}_{SO_0(N-1,1)}^{O(N-1,1)}
({\Kirredrep{O(N-1,1)}{\lambda}}_{+,+}|_{SO_0(N-1,1)}).  
\end{equation}
In light that ${\Kirredrep{O(N-1,1)}{\lambda}}_{+,+}$ is a representation
 of $O(N-1,1)$, 
 we can compute the induced representation \eqref{eqn:IndSO0}
 as follows.  
\begin{align*}
\text{\eqref{eqn:IndSO0}}
\simeq\,\, & {\Kirredrep{O(N-1,1)}{\lambda}}_{+,+} 
         \otimes {\operatorname{Ind}}_{SO_0(N-1,1)}^{O(N-1,1)}({\bf{1}})
\\
\simeq\,\, &{\Kirredrep{O(N-1,1)}{\lambda}}_{+,+} 
         \otimes (\bigoplus_{a, b \in \{\pm\}} \chi_{a b})
\\
\simeq\,\, & \bigoplus_{a,b \in \{\pm\}}{\Kirredrep{O(N-1,1)}{\lambda}}_{a,b}.  
\end{align*}
Thus Lemma \ref{lem:161612} is proved.  
\end{proof}
There are a few overlaps in the expressions \eqref{eqn:Fnpm} for 
$O(N)$-modules and \eqref{eqn:Fn1ab} for $O(N-1,1)$-modules.  
We give a necessary and sufficient condition
 for two expressions, 
 which give the same irreducible representation as follows.  
\begin{lemma}
\label{lem:fdeq}
\begin{enumerate}
\item[{\rm{(1)}}]
The following two conditions on $\lambda, \mu \in \Lambda^+([\frac N 2])$
 and $a, b \in \{\pm\}$ are equivalent:
\begin{enumerate}
\item[{\rm{(i)}}]
$\Kirredrep {O(N)}{\lambda}_a \simeq \Kirredrep {O(N)}{\mu}_b$
 as $O(N)$-modules;
\item[{\rm{(ii)}}]
\lq\lq{$\lambda = \mu$ and $a=b$}\rq\rq\
 or the following condition holds:
\begin{equation}
\label{eqn:ONisom}
\text{
$\lambda=\mu$, 
 $N$ is even, $\lambda_{\frac N 2} >0$, 
 and $a=-b$. 
}
\end{equation}
\end{enumerate}
\item[{\rm{(2)}}]
Suppose $N \ge 2$.  
Then the following two conditions on $\lambda, \mu \in \Lambda^+([\frac N 2])$
 and $a, b, c, d \in \{\pm\}$ are equivalent:
\begin{enumerate}
\item[{\rm{(i)}}]
$\Kirredrep {O(N-1,1)}{\lambda}_{a,b} \simeq \Kirredrep {O(N-1,1)}{\mu}_{c,d}$
 as $O(N-1,1)$-modules;
\item[{\rm{(ii)}}]
\lq\lq{$\lambda = \mu$ and $(a,b)=(c,d)$}\rq\rq\
 or the following condition holds:
\begin{equation}
\label{eqn:ON1isom}
\text{
$\lambda=\mu$, 
 $N$ is even, $\lambda_{\frac N 2}>0$, and $(a,b)=-(c,d)$. 
}
\end{equation}
\end{enumerate}
\end{enumerate}
\end{lemma}



\begin{proof}
(1)\enspace
The $O(N)$-isomorphism $\Kirredrep {O(N)}{\lambda}_{a} \simeq \Kirredrep {O(N)}{\mu}_{b}$ implies an obvious isomorphism
 $\Kirredrep {O(N)}{\lambda}_{a}|_{SO(N)} \simeq \Kirredrep {O(N)}{\mu}_{b}|_{SO(N)}$
 as $SO(N)$-modules,
 whence $\lambda=\mu$ by the classical branching law
 (Lemma \ref{lem:OSO})
 for the restriction $O(N) \downarrow SO(N)$.  
Then the equivalence (i) $\Leftrightarrow$ (ii) follows from the equivalence 
(i) $\Leftrightarrow$ (iii) in Lemma \ref{lem:branchII}.  
\par\noindent
(2)\enspace
Similarly to the proof for the first statement,
 we may and do assume $\lambda=\mu$
 by considering of the restriction $O(N-1,1) \downarrow SO(N-1,1)$.  
Then the proof of the equivalence (i) $\Leftrightarrow$ (ii) for $O(N-1,1)$
 reduces to the case for $O(N,1)$
 and the following lemma.  
\end{proof}

\begin{lemma}
\label{lem:holoON1}
Suppose $\sigma$ is an irreducible finite-dimensional representation of 
 $O(N-1,1)$.  
\begin{enumerate}
\item[{\rm{(1)}}]
Suppose $N \ge 2$.  
If $\sigma$ is extended to a holomorphic representation
 of $O(N,{\mathbb{C}})$, 
 then neither $\sigma \otimes \chi_{+-}$ nor $\sigma \otimes \chi_{-+}$
 can be extended to a holomorphic representation of $O(N,{\mathbb{C}})$.  
\item[{\rm{(2)}}]
Suppose $N \ge 3$.  
If $\sigma$ cannot be extended to a holomorphic representation
 of $O(N,{\mathbb{C}})$, 
 then both $\sigma \otimes \chi_{+-}$ and $\sigma \otimes \chi_{-+}$
 can be extended to a holomorphic representation of $O(N,{\mathbb{C}})$.  
\end{enumerate}
\end{lemma}

\begin{proof}
(1)\enspace
If $\sigma \otimes \chi_{ab}$ extends to a holomorphic representation
 of $O(N,{\mathbb{C}})$, 
 then so does the subrepresentation $\chi_{ab}$ in the tensor product
 $(\sigma \otimes \chi_{ab}) \otimes \sigma^{\vee}$, 
 where $\sigma^{\vee}$ stands for the contragredient representation of $\sigma$.  
Since $\chi_{ab}$ is the restriction of some holomorphic character of $O(N,{\mathbb{C}})$
 if and only if $(a,b)=(+,+)$ or $(-,-)$, 
 the first statement is proved.  
\par\noindent
(2)\enspace
As in the proof of Lemma \ref{lem:161612} (2), 
 we see that at least one element
 in $\{\sigma \otimes \chi_{ab}:a,b \in \{\pm\}\}$
 can be extended to a holomorphic representation
 of $O(N,{\mathbb{C}})$.  
Then the second statement follows from the first one.  
\end{proof}

\begin{example}
\label{ex:exteriorpm}
The natural action of $O(N)$
 on $i$-th exterior algebra
 $\Exterior^i({\mathbb{C}}^N)$ is given as 
\[
\Exterior^i({\mathbb{C}}^N)
\simeq
\begin{cases}
\Kirredrep {O(N)}{1^i, 0^{[\frac N 2]-i}}_{+}
\quad
&\text{if } i \le \frac N 2, 
\\
\Kirredrep {O(N)}{1^{N-i}, 0^{i-[\frac {N+1} 2]}}_{-}
\quad
&\text{if } i \ge \frac N 2, 
\end{cases}
\]
with the notation
 in this section,
 whereas the same representation was described as 
\[
    \Exterior^i({\mathbb{C}}^N)
    \simeq
    \Kirredrep {O(N)}{\underbrace{1,\cdots,1}_{i}, \underbrace{0,\cdots,0}_{N-i}}
\]
with the notation \eqref{eqn:CWOn}
 in Section \ref{subsec:repON}.  
\end{example}
As in the classical branching rule for $O(N) \downarrow O(N-1)$
 given in Fact \ref{fact:ONbranch}, 
 we give the irreducible decomposition of finite-dimensional representations
 of $O(N,1)$
 when restricted to the subgroup $O(N-1,1)$
 as follows:
\begin{theorem}
[branching rule for $O(N,1) \downarrow O(N-1,1)$]
\label{thm:ON1branch}
\index{B}{branchingruleON1@branching rule, for $O(N,1) \downarrow O(N-1,1)$}
Let $N \ge 2$.  
Suppose that $(\lambda_1, \cdots,\lambda_{[\frac{N+1}2]})
 \in \Lambda^+([\frac{N+1}2])$
 and $a,b \in \{\pm\}$.  
Then the irreducible finite-dimensional representation
 $\Kirredrep{O(N,1)}{\lambda_1, \cdots,\lambda_{[\frac{N+1}2]}}_{a, b}$
  of $O(N,1)$ decomposes
 into a multiplicity-free sum
 of irreducible representations
 of $O(N-1,1)$ as follows:
\begin{equation*}
   \Kirredrep{O(N,1)}{\lambda_1, \cdots,\lambda_{[\frac{N+1}2]}}_{a, b}|_{O(N-1,1)}
   \simeq
  \bigoplus
   \Kirredrep{O(N-1,1)}{\nu_1, \cdots, \nu_{[\frac{N}{2}]}}_{a, b}, 
\end{equation*}
where the summation is taken over
 $(\nu_1, \cdots, \nu_{[\frac{N}{2}]}) \in {\mathbb{Z}}^{[\frac N 2]}$
 subject to 
\begin{alignat*}{2}
&\lambda_{1} \ge \nu_{1} \ge \lambda_{2} \ge \cdots \ge \nu_{\frac N 2} \ge 0
\quad
&&\text{for $N$ even}, 
\\
&\lambda_{1} \ge \nu_{1} \ge \lambda_{2} \ge \cdots \ge \nu_{\frac {N-1} 2}
\ge \lambda_{\frac {N+1} 2}
\quad
&&\text{for $N$ odd}.   
\end{alignat*} 
\end{theorem}

\begin{proof}
The assertion follows 
 in the case $(a,b)=(+,+)$ from Fact \ref{fact:ONbranch}.  
The general case follows from the definition \eqref{eqn:Fn1ab}
 and from the observation 
 that the restriction $\chi_{a,b}|_{G'}$
 of the $G$-character $\chi_{a,b}$
 gives the same type of a character for $G'=O(N-1,1)$, 
 see \eqref{eqn:chiabrest}. 
\end{proof}

\subsection{Singular parameters for $V \in \widehat{O(n)}$: $S(V)$ and $S_Y(V)$}

In this section we prepare some notation 
 that describes the parameters
 of {\it{reducible}} principal series representations $I_{\delta}(V,\lambda)$
 of $G=O(n+1,1)$.  



We recall from Lemma \ref{lem:IVchi}
 that both of the following subsets
\begin{align*}
&\{(\delta,V,\lambda)
:
\text{$I_{\delta}(V,\lambda)$ has regular integral ${\mathfrak{Z}}_G({\mathfrak{g}})$-infinitesimal character}
\}, 
\\
&\{(\delta,V,\lambda)
:
\text{$I_{\delta}(V,\lambda)$ is reducible}
\}
\end{align*}
 of $\{\pm\} \times \widehat {O(n)} \times {\mathbb{C}}$
 are preserved under the following transforms:
\begin{align*}
  (\delta,V,\lambda) &\mapsto (-\delta,V,\lambda), 
\\
    (\delta,V,\lambda) &\mapsto (\delta,V \otimes \det,\lambda).  
\end{align*}
Thus we omit 
 the signature $\delta$
 in our notation,
 and focus on the second and third components.  

\begin{definition}
\label{def:RIntRed}
We define two subsets
 of $\widehat {O(n)} \times {\mathbb{C}}$
 (actually, of $\widehat{O(n)} \times {\mathbb{Z}}$)
 by 
\index{A}{RzqSeducible@$\Reducible$ $(\subset \widehat {O(n)} \times {\mathbb{Z}})$|textbf}
\index{A}{RzqSeducibleIn@$\RInt$ $(\subset \widehat {O(n)} \times {\mathbb{Z}})$|textbf}
\begin{align}
\RInt
&:=\{(V,\lambda)
:
\text{$I_{\delta}(V,\lambda)$ has regular integral ${\mathfrak{Z}}_G({\mathfrak{g}})$-infinitesimal character}\}, 
\notag
\\
\label{eqn:reducible}
\Reducible
&:=\{(V,\lambda)
:
\text{$I_{\delta}(V,\lambda)$ is reducible}\}.  
\end{align}
\end{definition}

Both the  sets $\RInt$ and $\Reducible$ are preserved by the transformations
\begin{alignat*}{3}
&(V, \lambda) 
&&\mapsto 
&&(V \otimes \det, \lambda), 
\\
&(V, \lambda) 
&&\mapsto 
&&(V, n-\lambda).  
\end{alignat*}
This is clear for $\RInt$, 
 whereas the assertions for $\Reducible$ follows from
 the $G$-isomorphism $I_{\delta}(V,\lambda) \otimes \chi_{--} \simeq I_{\delta}(V\otimes \det,\lambda)$
 by Lemma \ref{lem:IVchi}
 and from the fact that $I_{\delta}(V,n-\lambda)$
 is isomorphic to the contragredient representation
 of $I_{\delta}(V,\lambda)$.  
We shall introduce two discrete sets 
 $S(V)$ and $S_Y(V)$ for $V \in \widehat {O(n)}$
 in Definition \ref{def:SVSYV} below, 
 and prove in Lemma \ref{lem:regint} and in Theorem \ref{thm:irrIV}
\begin{alignat*}{3}
  &\RInt
  &&=
  &&\{(V,\lambda) \in \widehat {O(n)} \times {\mathbb{Z}}
    :
    \lambda \not \in S(V) \}
\\
  &\hphantom{ii} \cup
  &&
  &&\hphantom{MMMMM}\cup
\\
  &\Reducible
  &&=
  &&\{(V,\lambda) \in \widehat {O(n)} \times {\mathbb{Z}}
    :
    \lambda \not \in S(V) \cup S_Y(V)\},  
\end{alignat*}
 see also Convention \ref{conv:SYsigma}.  


\subsubsection{Infinitesimal character $r(V,\lambda)$ of $I_{\delta}(V,\lambda)$}
\label{subsec:muV}
Suppose that $V \in \widehat {O(n)}$ is given as 
\[
   V=\Kirredrep{O(n)}{\sigma}_{\varepsilon}
  \quad
  \text{for some }
  \sigma \in \Lambda^+\left([\dfrac n 2]\right)
  \text{ and }
  \varepsilon \in \{\pm\}
\]
with the notation
 as in Section \ref{subsec:fdimrep}.  
We define an element of ${\mathfrak {h}}_{\mathbb{C}}^{\ast}
\simeq {\mathbb{C}}^{[\frac n2]+1}$
by 
\index{A}{rVlmd@$r(V,\lambda)$|textbf}
\begin{equation}
\label{eqn:IVZG}
r(V,\lambda)
:=
(\sigma_{1}+\frac n 2-1,\sigma_{2}+\frac n 2-2, \cdots,\sigma_{[\frac n 2]}+
\frac n 2 - [\frac n 2],\lambda-\frac n 2).  
\end{equation}
The ordering in \eqref{eqn:IVZG} will play a crucial role
 in a combinatorial argument
 in later sections,
 whereas,
 up to the action of the Weyl group $W_G$, 
 $r(V,\lambda)$ gives the 
\index{A}{ZGg@${\mathfrak{Z}}_G({\mathfrak{g}})$}
\index{B}{infinitesimalcharacter@infinitesimal character}
\index{B}{ZGginfinitesimalcharacter@${\mathfrak{Z}}_G({\mathfrak{g}})$-infinitesimal character}
 ${\mathfrak{Z}}_G({\mathfrak{g}})$-infinitesimal character
 of the unnormalized induced representation
 $I_{\delta}(V, \lambda)$ of $G=O(n+1,1)$, 
 see \eqref{eqn:ZGinfI}.  



\begin{example}
\label{ex:rhoi}
For $0 \le i \le n$, 
 we set $\ell := \min(i,n-i)$
 and 
\index{A}{1parhoi@$\rho^{(i)}$|textbf}
\begin{align*}
   \rho^{(i)}:=& r(\Exterior^i({\mathbb{C}}^n),i)  
\\
=&(\underbrace{\dfrac n 2, \dfrac n 2-1, \cdots, 
 \dfrac n 2-\ell+1}_\ell, 
 \underbrace{\dfrac n 2-\ell-1, \cdots, 
 \dfrac n 2-[\dfrac n 2]}_{[\frac n 2]-\ell},
 i-\frac n 2)
\\
=&\begin{cases}
(\underbrace{\dfrac n 2, \cdots, 
 \dfrac n 2-i+1}_i, 
 \underbrace{\dfrac n 2-i-1, \cdots, 
 \dfrac n 2-[\dfrac n 2]}_{[\frac n 2]-i},
 i-\frac n 2)
\quad
&\text{for $i \le [\dfrac n 2]$, }
\\[1em]
(\underbrace{\dfrac n 2, \cdots, 
 -\dfrac n 2+i+1}_{n-i}, 
 \underbrace{-\dfrac n 2+i-1, \cdots, 
 \dfrac n 2-[\dfrac n 2]}_{i-[\frac{n+1}2]},
 i-\dfrac n 2)
\quad
&\text{for $[\dfrac {n+1} 2] \le i$. }
\end{cases}
\end{align*}
Here are some elementary properties.  
\begin{enumerate}
\item[{\rm{(1)}}]
The following equations hold:
\begin{align}
\label{eqn:rhoi0}
\rho^{(i)}-\rho^{(0)}
=&
(\underbrace{1,\cdots,1}_{\ell},
 \underbrace{0,\cdots,0}_{[\frac n 2]-\ell},i)
\\
\notag
=&
\begin{cases}
(\underbrace{1,\cdots,1}_{i},
 \underbrace{0,\cdots,0}_{[\frac n 2]-i},
 i)
\quad
\text{for $0 \le i \le [\frac n 2]$, }
\\
(\underbrace{1,\cdots,1}_{n-i},
 \underbrace{0,\cdots,0}_{i-[\frac{n+1}2]},
 i)
\quad
\text{for $[\frac{n+1}2] \le i \le n$.  }
\end{cases}
\end{align}

\item[{\rm{(2)}}]
Let $r(V,\lambda)$ be defined as in \eqref{eqn:IVZG}.  
Then for any $i$ $(0 \le i \le n)$, 
 we have
\begin{align*}
r(V,\lambda)
 =&(\sigma_1, \cdots, \sigma_{[\frac n 2]}, \lambda) + \rho^{(0)}
\\
 =&(\sigma_1-1, \cdots, \sigma_{\ell}-1, \sigma_{\ell+1}, \cdots, \sigma_{[\frac n 2]}, \lambda-i) + \rho^{(i)}, 
\end{align*}
where we retain the notation $\ell = \min (i,n-i)$.  

\item[{\rm{(3)}}]
For all $i$ $(0 \le i \le n)$, 
\index{A}{1parho@$\rho_G$}
\begin{equation}
\label{eqn:rhoiW}
   \rho_G \equiv \rho^{(i)} \mod W_G.  
\end{equation}
\end{enumerate}
\end{example}

\subsubsection{Singular integral parameter:
$S(V)$ and $S_Y(V)$}
Retain the setting as in Section \ref{subsec:muV}.  
Let $G=O(n+1,1)$ and $m=[\frac n 2]$.  
Suppose $V \in \widehat{O(n)}$
 is given as $V = \Kirredrep{O(n)}{\sigma}_{\varepsilon}$
 with $\sigma=(\sigma_1, \cdots, \sigma_m)$ and $\varepsilon \in \{\pm\}$.  
Since $\sigma_1$, $\cdots$, $\sigma_{m} \in {\mathbb{Z}}$,  
 the following three conditions
 on $\lambda \in {\mathbb{C}}$ are equivalent:
\begin{enumerate}
\item[(i)]
The ${\mathfrak{Z}}_G({\mathfrak{g}})$-infinitesimal character
 of $I_{\delta}(V,\lambda)$ is integral 
 in the sense of Definition \ref{def:intreg};
\item[(ii)]
$\langle r(V,\lambda), \alpha^{\vee} \rangle \in {\mathbb{Z}}$ for any $\alpha \in \Delta({\mathfrak{g}}_{\mathbb{C}}, {\mathfrak{h}}_{\mathbb{C}})$;
\item[(iii)]
 $\lambda \in {\mathbb{Z}}$.  
\end{enumerate}



For each $V \in \widehat {O(n)}$, 
 we introduce a subset $S(V)$ in ${\mathbb{Z}}$
 (and a subset $S_Y(V)$ in ${\mathbb{Z}}$ for $V$ of type Y) 
 as follows.  
\begin{definition}
[$S(V)$ and $S_Y(V)$]
\label{def:SVSYV}
Let $m=[\frac n 2]$.  
For $V=\Kirredrep{O(n)}{\sigma}_{\varepsilon}$
 with $\sigma=(\sigma_1, \cdots, \sigma_m) \in \Lambda^+(m)$
 and $\varepsilon \in \{\pm\}$, 
 we define a finite subset of ${\mathbb{Z}}$ by 
\index{A}{Ssigma@$S(V)$|textbf}
\begin{equation}
\label{eqn:singint}
  S(V):=\{j-\sigma_j, n+\sigma_j-j: 1 \le j \le m\}.  
\end{equation}


When the irreducible $O(n)$-module $V$
 is of 
\index{B}{typeY@type Y, representation of ${O(N)}$\quad}
 type Y
 (see Definition \ref{def:OSO}), 
 namely, 
when $n$ is even $(=2m)$ and $\sigma_m > 0$, 
 we define also the following finite set
\index{A}{SsigmaY@$S_Y(V)$|textbf}
\begin{equation}
\label{eqn:SYsigma}
  S_Y(V):=\{\lambda \in {\mathbb{Z}} : 0 < |\lambda-m| <\sigma_m\}.  
\end{equation}
\end{definition}
We note that 
\[
  S(V) \cap S_Y(V) = \emptyset
\]
by definition.  
We shall sometimes adopt the following convention:
\begin{convention}
\label{conv:SYsigma}
When $V$ is of type X (see Definition \ref{def:OSO}), 
 we set
\[
   S_Y(V) = \emptyset.  
\]
\end{convention}



The definitions imply the following lemma.  
\begin{lemma}
\label{lem:regint}
The ${\mathfrak {Z}}_G({\mathfrak{g}})$-infinitesimal character
 of $I_{\delta}(V,\lambda)$ is regular integral 
 (see Definition \ref{def:intreg})
 if and only if $\lambda \in {\mathbb{Z}} \setminus S(V)$.  
Thus, 
 we have
\[
\RInt =\{(V,\lambda) \in \widehat{O(n)} \times {\mathbb{Z}}
:
 \lambda \not \in S(V) \}.  
\]
\end{lemma}
We refer to $S(V)$ 
as the set of 
 {\it{singular integral parameters}}.  
It should be noted
 that $I_{\delta}(V,\lambda)$ has
\index{B}{regularintegralinfinitesimalcharacter@regular integral infinitesimal character}
 regular integral infinitesimal character
 if $\lambda \in S_Y(V)$, 
 since $S_Y(V) \subset {\mathbb{Z}} \setminus S(V)$.  



We shall see in Theorem \ref{thm:irrIV} below
 that the principal series representation $I_{\delta}(V, \lambda)$ is
 irreducible
 if and only if 
 $\lambda \in ({\mathbb{C}} \setminus {\mathbb{Z}}) \cup S(V) \cup S_Y(V)$.  



We end this section with a lemma
 that will be used in Appendix III
 (Chapter \ref{sec:Translation})
 when we discuss translation functors.  

\begin{lemma}
\label{lem:1806115}
Let $V \in \widehat{O(n)}$
 and $\lambda \in {\mathbb{Z}} \setminus S(V)$.  
\begin{enumerate}
\item[{\rm{(1)}}]
Suppose $V$ is of type X (Definition \ref{def:OSO}).  
Then the $W_{\mathfrak{g}}$- and $W_G$-orbits
 through
 $r(V,\lambda) \in {\mathfrak{h}}_{\mathbb{C}}^{\ast} \simeq {\mathbb{C}}^{[\frac n 2]+1}$
 coincide:
\begin{equation}
\label{eqn:WGrV}
W_{\mathfrak{g}}\, r(V,\lambda)
=
W_G\, r(V,\lambda).  
\end{equation}
\item[{\rm{(2)}}]
Suppose $V$ is of type Y.  
Then \eqref{eqn:WGrV} holds
 if and only if $\lambda=\frac n 2$.  
\end{enumerate}
\end{lemma}


\begin{proof}
(1)\enspace 
The assertion is obvious
 when $n$ is odd
 because $W_{\mathfrak{g}}=W_G$
 in this case.  
Suppose $n$ is even, 
say, 
 $n=2m$.  
It is sufficient to show
 that $r(V,\lambda)$ contains zero
 in its entries.  
Since $V$ is of type X, 
 we have $\sigma_m=0$, 
 and therefore, 
 the $m$-th entry of $r(V,\lambda)$ amounts to 
 $\sigma_m + m-m=0$ by the definition \eqref{eqn:IVZG}.  
Thus the lemma is proved.  
\newline
(2)\enspace
Since $V$ is of type Y, 
 $n$ is even $(=2m)$ and $W_G \supsetneqq W_{\mathfrak{g}}$.  
Since $\lambda \not \in S(V)$, 
 $r(V,\lambda)$ is $W_{\mathfrak{g}}$-regular.  
Hence \eqref{eqn:WGrV} holds
 if and only if at least one of the entries in $r(V,\lambda)$ equals zero.  
Since $\sigma_1 \ge \sigma_2 \ge \cdots \ge \sigma_m >0$, 
 this happens only when the $(m+1)$-th entry
 of $r(V,\lambda)$ vanishes, 
 {\it{i.e.}}, $\lambda = \frac n 2 (=m)$.  
Hence Lemma \ref{lem:1806115} is proved.  
\end{proof}



\begin{remark}
\label{rem:zeroreg}
For $n=2m$ (even), 
 if $V$ is of type X
 or if $\lambda=m$, 
 then the ${\mathfrak {Z}}_G({\mathfrak{g}})$-infinitesimal character
 \eqref{eqn:IVZG} is regular 
 for 
\index{A}{Weylgroupg@$W_{\mathfrak{g}}$, Weyl group for ${\mathfrak{g}}_{\mathbb{C}}={\mathfrak{o}}(n+2,{\mathbb{C}})$}
 $W_{\mathfrak{g}}$
 in the sense of Definition \ref{def:intreg}, 
 but is 
\index{B}{singularintegralinfinitesimalcharacter@singular integral infinitesimal character}
\lq\lq{singular}\rq\rq\
 with respect to the Weyl group 
\index{A}{WeylgroupG@$W_G$, Weyl group for $G=O(n+1,1)$}
 $W_G$
 for the {\it{disconnected}} group 
 $G=O(n+1,1)$
 which is not in the Harish-Chandra class.  
\end{remark}







\subsection{Irreducibility condition of $I_{\delta}(V,\lambda)$}
\label{subsec:180595}
We are ready to state a necessary and sufficient condition
 for the principal series representation $I_{\delta}(V,\lambda)$
 of $G=O(n+1,1)$
 to be irreducible.  

We recall from \eqref{eqn:singint} and \eqref{eqn:SYsigma}
 the definitions of 
\index{A}{Ssigma@$S(V)$}
$S(V)$ and 
\index{A}{SsigmaY@$S_Y(V)$}
$S_Y(V)$, 
 respectively.  

\begin{theorem}
[irreducibility criterion of $I_{\delta}(V,\lambda)$]
\label{thm:irrIV}
Let $G=O(n+1,1)$, $\delta \in \{\pm\}$, 
 $V \in \widehat{O(n)}$,
 and $\lambda \in {\mathbb{C}}$.  
\begin{enumerate}
\item[{\rm{(1)}}]
If $\lambda \in {\mathbb{C}} \setminus {\mathbb{Z}}$, 
 then the principal series representation $I_{\delta} (V,\lambda)$ of $G$ 
 is irreducible.  
\item[{\rm{(2)}}]
Suppose $\lambda \in {\mathbb{Z}}$.  
Then $I_{\delta}(V,\lambda)$ is irreducible if and only if
\begin{alignat*}{2}
&\text{$\lambda \in S(V)$}
&&\text{when $V$ is of type X}, 
\\
&\text{$\lambda \in S(V) \cup S_Y(V)$}
\quad
&&\text{when $V$ is of type Y}.  
\end{alignat*}
\end{enumerate}
Thus $\Reducible$
 $($see \eqref{eqn:reducible}$)$
 is given by 
\begin{equation}
\label{eqn:Red}
  \Reducible
 =\{(V, \lambda) \in \widehat{O(n)} \times {\mathbb{Z}}
  : \lambda \not \in S(V) \cup S_Y(V)\}
\end{equation}
with Convention \ref{conv:SYsigma}.  
\end{theorem}
The proof of Theorem \ref{thm:irrIV} will be given 
in Section \ref{subsec:pfirrIV} in Appendix II
 by inspecting the restriction of $I_{\delta}(V,\lambda)$
 of $G=O(n+1,1)$
 to its subgroups $\overline G=SO(n+1,1)$ and $G_0=SO_0(n+1,1)$.  

\begin{example}
\label{ex:irrIilmd}
Let $0 \le i \le n$.  
The exterior tensor representation
 on $\Exterior^i({\mathbb{C}}^n)$ is of type X
 if and only if $n \ne 2i$
 (see Example \ref{ex:2.1}).  
A simple computation shows
\begin{alignat*}{2}
S(\Exterior^{(i)}({\mathbb{C}}^n))=&{\mathbb{Z}} 
   \setminus (\{i,n-i\} \cup (-{\mathbb{N}}_+) \cup (n+{\mathbb{N}}_+))
\qquad
&&\text{for $0 \le i \le n$}, 
\\
S_Y(\Exterior^{(m)}({\mathbb{C}}^n))=& \emptyset
\qquad
&&\text{for $n=2m$}, 
\end{alignat*}
 see also Example \ref{ex:isig}.  
Hence $I_{\delta}(i,\lambda)$ is reducible 
 if and only if
\[
   \lambda \in \{i,n-i\} \cup (-{\mathbb{N}}_+) \cup (n+{\mathbb{N}}_+)
\]
by Theorem \ref{thm:irrIV}.  
See Theorem \ref{thm:LNM20}
 for the socle filtration of $I_{\delta}(i,\lambda)$
 for $\lambda =i$ or $n-i$.  
\end{example}



For later purpose,
 we decompose $\Reducible$
 into two disjoint subsets as follows:
\begin{definition}
\label{def:Red12}
 We recall from Definition \ref{def:OSO}
 that any $V \in \widehat{O(n)}$ is either
 of type X or of type Y for
 $\widehat{O(n)}$.  
We set
\index{A}{RzqSeducibleI@$\RedI$|textbf}
\index{A}{RzqSeducibleII@$\RedJ$|textbf}
\begin{align*}
\RedI
:=&\{(V, \lambda) \in \Reducible
:
\text{$V$ is of type X or $\lambda = \frac n 2$}
\}, 
\\
\RedJ
:=&\{(V, \lambda) \in \Reducible
:
\text{$V$ is of type Y and $\lambda \ne \frac n 2$}
\}.  
\end{align*}
Then we have a disjoint union 
\[
\Reducible=\RedI \amalg \RedJ.  
\]
\end{definition}

\begin{remark}
\label{rem:Red12}
If $n$ is odd,
 then 
\[
  \RedJ = \emptyset
  \quad
  \text{and}
  \quad
  \Reducible =\RedI.  
\]
\end{remark}

\subsection{Subquotients of $I_{\delta}(V,\lambda)$
\label{subsec:subrep}}
By Theorem \ref{thm:irrIV}, 
 the principal series representation $I_{\delta}(V,\lambda)$
 of $G=O(n+1,1)$ is reducible
 {\it{i.e.}}, $(V,\lambda) \in \Reducible$
 if and only if
\begin{equation*}
   \lambda \in {\mathbb{Z}} \setminus (S(V) \cup S_Y(V))
\end{equation*}
 with Convention \ref{conv:SYsigma}.  
In this section,
 we explain the socle filtration of $I_{\delta}(V,\lambda)$.  
A number of different characterizations of the subquotients 
 will be given in later sections,
 see Theorem \ref{thm:1808116}
 for summary.  
We divide the arguments into the following two cases:
\par\noindent
Case 1.\enspace
$\lambda \ne \frac n 2$, 
 see Section \ref{subsec:Isub1};
\par\noindent
Case 2.\enspace
$\lambda = \frac n 2$, 
 see Section \ref{subsec:Isub3}.  

\subsubsection{Subquotients of 
 $I_{\delta}(V,\lambda)$ for $V$ of type X
\label{subsec:Isub1}}
\index{B}{typeX@type X, representation of ${O(N)}$\quad}


We begin with the case
 where $\lambda \ne \frac n 2$.  
This means that we treat the following cases:
\begin{enumerate}
\item[$\bullet$]
$V$ is of type X, 
 and $\lambda \in {\mathbb{Z}} \setminus S(V)$;
\item[$\bullet$]
$V$ is of type Y, 
 and $\lambda \in {\mathbb{Z}} \setminus (S(V) \cup S_Y(V)\cup \{\frac n 2\})$. \end{enumerate}

\begin{proposition}
\label{prop:Xred}
Let $G=O(n+1,1)$, 
 $V \in \widehat{O(n)}$, 
$\delta \in \{\pm\}$, 
 and $\lambda \in {\mathbb{Z}} \setminus (S(V) \cup S_Y(V))$.  
Assume further that $\lambda \ne \frac n 2$.  
Then there exists a unique proper submodule
 of the principal series representation
 $I_{\delta}(V,\lambda)$, 
 to be denoted by 
\index{A}{IdeltaVf@$I_{\delta}(V, \lambda)^{\flat}$|textbf}
$I_{\delta}(V,\lambda)^{\flat}$.  
In particular, 
 the quotient $G$-module
\index{A}{IdeltaVs@$I_{\delta}(V, \lambda)^{\sharp}$|textbf}
\[
  I_{\delta}(V,\lambda)^{\sharp}
  :=
  I_{\delta}(V,\lambda)
  /I_{\delta}(V,\lambda)^{\flat}
\]
is irreducible.  
\end{proposition}

The proof of Proposition \ref{prop:Xred} will be given 
 in Section \ref{subsec:Isub2} of  Appendix II.  

\begin{remark}
The $K$-type formul{\ae}
 and the minimal $K$-types
 of the irreducible $G$-modules $I_{\delta}(V,\lambda)^{\flat}$
 and $I_{\delta}(V,\lambda)^{\sharp}$
 will be given
 in Proposition \ref{prop:KXred}
 and Proposition \ref{prop:minKXred}, 
 respectively.  
\end{remark}

\subsubsection
{Subrepresentations of $I_{\delta}(V,\frac n 2)$
 for $V$ of type Y}
\label{subsec:Isub3}
\index{B}{typeY@type Y, representation of ${O(N)}$\quad}

Next we discuss the case:
\begin{enumerate}
\item[$\bullet$]
$V$ is of type Y and $\lambda = \frac n 2$.  
\end{enumerate}


In this case $I_{\delta}(V,\lambda)$ is the smooth representation
 of a tempered unitary representation.  

\begin{proposition}
[reducible tempered principal series representation]
\label{prop:IVmtemp}
Let $G=O(n+1,1)$ with $n=2m$, 
 $V\in \widehat{O(n)}$ be of type Y, 
 and $\delta \in \{\pm\}$.  
Then the principal series representation
 $I_{\delta}(V,m)$ of $G$
 is decomposed
 into the direct sum
 of two irreducible representations of $G$,
 to be written as:
\[
  I_{\delta}(V,m)
  \simeq
  I_{\delta}(V,m)^{\flat} 
  \oplus
  I_{\delta}(V,m)^{\sharp}.  
\]
If we express $V=\Kirredrep {O(n)}{\sigma}_{\varepsilon}$
 by $\sigma=(\sigma_1, \cdots, \sigma_m) \in \Lambda^+(m)$
 with $\sigma_m >0$
 and $\varepsilon \in \{\pm\}$, 
 then the irreducible $G$-modules
 $I_{\delta}(V,m)^{\flat}$ and $I_{\delta}(V,m)^{\sharp}$
 are characterized by their minimal $K$-types
 given respectively by the following:
\begin{align*}
 & \Kirredrep{O(n+1)}{\sigma_1, \cdots, \sigma_m}_{\varepsilon} \boxtimes \delta, 
\\
 & \Kirredrep{O(n+1)}{\sigma_1, \cdots, \sigma_m}_{-\varepsilon} \boxtimes (-\delta).  
\end{align*}
\end{proposition}
\begin{proof}
This is proved in Proposition \ref{prop:180572} (2)
 except for the assertion on the $K$-types.  
The last assertion on the minimal $K$-types follow from 
 the $K$-type formula
 of $I_{\delta}(V,m)^{\flat}$ and $I_{\delta}(V,m)^{\sharp}$
 in Proposition \ref{prop:KXred} (2).  
\end{proof}

\subsubsection{Socle filtration of $I_{\delta}(V,\lambda)$}
By Theorem \ref{thm:irrIV}
 together with Propositions \ref{prop:Xred} and \ref{prop:IVmtemp}, 
 we obtain the following:
\begin{corollary}
\label{cor:length2}
Let $G=O(n+1,1)$ for $n \ge 2$.  
Then the principal series representation
 $I_{\delta}(V,\lambda)$
 $(\delta \in \{\pm\}$, $V \in \widehat{O(n)},$ $\lambda \in {\mathbb{C}})$
 of $G$ 
 is either irreducible or of composition 
 series of length two.  
\end{corollary}




\subsection{Definition of the height $i(V,\lambda)$}
\label{subesec:defiVlmd}

In this section we introduce the 
\index{B}{height@height, of $(V,\lambda)$|textbf}
\lq\lq{height}\rq\rq
\[
  i \colon \RInt \to \{0,1,\ldots,n\}, 
\quad
  (V, \lambda) \mapsto i(V, \lambda)
\]
which plays an important role
 in the study of the principal series representation
 $I_{\delta}(V,\lambda)$ of $G$.  
For instance, 
 we shall see in Section \ref{subsec:HasseV}
 that the $K$-type formula
 for subquotients
 of $I_{\delta}(V,\lambda)$ 
 is described by using the height $i(V,\lambda)$
 when $(V,\lambda) \in \Reducible$
 (Definition \ref{def:RIntRed}).  
Moreover,
 we shall prove in Theorem \ref{thm:1807113}
 that the $G$-module $I_{\delta}(V,\lambda)$
 is obtained by the translation functor
 applied to the principal series representation $I_{\pm}(i, i)$
 with the trivial infinitesimal character $\rho_G$
 without \lq\lq{crossing the wall}\rq\rq\
 if we take $i$ to be the height $i(V,\lambda)$, 
 see Theorem \ref{thm:1807113}.  
We note that the group $G=O(n+1,1)$
 is not in the Harish-Chandra class 
 when $n$ is even, 
 and will discuss carefully
 a {\it{translation functor}}
 in Appendix III
 (Chapter \ref{sec:Translation}).  



We recall from \eqref{eqn:IVZG}
 that 
\[
   r(V,\lambda)=(\sigma_1+\frac n 2-1, \sigma_2 + \frac n 2-2, \cdots, \sigma_m + \frac n 2-m, \lambda-\frac n 2), 
\]
where $m:=[\frac n 2]$.  
To specify the Weyl chamber for $W_{\mathfrak{g}}$
 that $r(V,\lambda)$ $\in (\frac 1 2 {\mathbb{Z}})^{m+1}$ belongs to, 
 we label the places
 where $\lambda-\frac n 2$ is located
 with respect to the following inequalities.  

\par\noindent
{\bf{\underline{Case 1.}}}\enspace
$n=2m$:
\[
   -\sigma_1-m+1< -\sigma_2-m+2
   < \cdots < -\sigma_m
   \le \sigma_m 
   < \cdots < \sigma_2 + m-2 < \sigma_1 + m-1;
\]
\par\noindent
{\bf{\underline{Case 2.}}}\enspace
$n=2m+1$:
\[
-\sigma_1-m+\frac 12< -\sigma_2-m+\frac 32
   < \cdots < -\sigma_{m}-\frac 1 2 < 0 
   < \sigma_{m}+ \frac 1 2 
   < \cdots < \sigma_1 + m-\frac 1 2.  
\]
Unifying these inequalities
 by adding $\frac n 2$
 to each term, 
 we may write as 
\[
  1-\sigma_1 < 2-\sigma_2 < \cdots< m - \sigma_m
  \le 
  \frac n 2
  \le \sigma_m + n-m
  < \cdots
  < \sigma_2 + n-2
  < \sigma_1 + n-1.  
\]
\begin{definition}
\label{def:RVi}
For $0 \le i \le n$, 
 we define the following subsets
\index{A}{RVi@$R(V;i)$|textbf}
 $R(V;i)$
 of ${\mathbb{Z}}$:
\begin{alignat*}{3}
&\{\lambda \in {\mathbb{Z}}: i - \sigma_{i} < \lambda < i+1-\sigma_{i+1}\}
\quad
&&\text{for $0 \le i < \frac {n-1} 2$}, 
&&
\\
&\{\lambda \in {\mathbb{Z}}: \frac{n-1}2 -\sigma_{\frac{n-1}2} < \lambda < \frac n 2\}
\quad
&&\text{for $i = \frac {n-1} 2$} \quad 
&&\text{($n$ odd)},  
\\
&\{\lambda \in {\mathbb{Z}}: \frac n 2-\sigma_{\frac n 2} < \lambda < \sigma_{\frac n 2}+\frac n 2\}
\quad
&&\text{for $i = \frac n 2$}
&&\text{($n$ even)}, 
\\
&\{\lambda \in {\mathbb{Z}}: \frac n 2 < \lambda < \sigma_{\frac {n-1} 2}+\frac {n+1} 2\}
\quad
&&\text{for $i = \frac {n+1} 2$}
&&\text{($n$ odd)}, 
\\
&\{\lambda \in {\mathbb{Z}}: \sigma_{n-i+1} +i-1< \lambda < \sigma_{n-i}+i\}
\quad
&&\text{for $\frac{n+1}2 < i \le n$}.  
&&
\end{alignat*}
\end{definition}
Here we regard $\sigma_0=\infty$.  
\begin{lemma}
\label{lem:ilsig}
Let $V \in \widehat{O(n)}$.  
We recall from \eqref{eqn:singint}
 that $S(V)$ is the set of singular integral parameters.  
\begin{enumerate}
\item[{\rm{(1)}}]
The set of regular integral parameters has the following disjoint decomposition:
\[
{\mathbb{Z}} \setminus S(V)
=
\coprod_{i=0}^n R(V;i). 
\]
In particular,
 there exists a map
\begin{equation}
\label{eqn:indexV}
   i(V, \cdot) \colon {\mathbb{Z}} \setminus S(V) \to \{0,1,\ldots,n\}
\end{equation}
such that $\lambda \in R(V;i(V, \lambda))$.  
\item[{\rm{(2)}}]
The set $S(V)$ is preserved
 by the transformations $\lambda \mapsto n-\lambda$
 and $V \mapsto V \otimes \det$, 
 and we have 
\begin{align*}
i(V,n-\lambda)\,=\,&n-i(V,\lambda)
\\
i(V \otimes \det ,\lambda)\,=\,&i(V,\lambda)
\end{align*}
for any $\lambda\in {\mathbb{Z}} \setminus S(V)$.    
\item[{\rm{(3)}}]
$R(V;\frac n 2) \ne \emptyset$
 if and only if $n$ is even and the irreducible $O(n)$-module $V$ 
 is of type Y.  
In this case,
 we have
\begin{equation}
\label{eqn:RSYV}
R(V;\frac n 2)=\{\frac n 2\} \cup S_Y(V)
\quad
\text{{\rm{(disjoint union).}}}
\end{equation}
\end{enumerate}
\end{lemma}

\begin{example}
\label{ex:isig}
Let $0 \le i \le n$.  
For the $i$-th exterior tensor representation
 $V=\Exterior^i({\mathbb{C}}^n)$ of $O(n)$, 
 we have
\[
   S(\Exterior^{(i)}({\mathbb{C}}^n))
   =
   {\mathbb{Z}} 
   \setminus 
   (\{i,n-i\}\cup (-{\mathbb{N}}_+) \cup (n+{\mathbb{N}}_+)).  
\]
Furthermore, 
 we see from Example \ref{ex:exteriorpm}
 that the set $R(V;j)$ is given as follows.  
\begin{enumerate}
\item[{\rm{(1)}}]
For $1 \le i \le n-1$, 
\[
  R(\Exterior^i({\mathbb{C}}^n);j)
  =
  \begin{cases}
  -{\mathbb{N}}_+
  \qquad
  &{\text{if $j=0$}}, 
\\
  \{j\}
  \qquad
  &{\text{if $j=i$ or $n-i$}}, 
\\
  n+{\mathbb{N}}_+
  \qquad
  &{\text{if $j=n$}}, 
\\
  \emptyset
  \qquad
  &{\text{otherwise}}.  
  \end{cases}
\]
\item[{\rm{(2)}}]
For $i =0$ or $n$, 
\[
  R(\Exterior^i({\mathbb{C}}^n);j)
  =
  \begin{cases}
  -{\mathbb{N}}
  \qquad
  &{\text{if $j=0$}}, 
\\
  n+{\mathbb{N}}
  \qquad
  &{\text{if $j=n$}}, 
\\
  \emptyset
  \qquad
  &{\text{otherwise}}.  
  \end{cases}
\]
\end{enumerate}

\end{example}



We recall from Definition \ref{def:RIntRed}
 that $\RInt$ is a subset of $\widehat {O(n)} \times {\mathbb{Z}}$.  
\begin{definition}
[height $i(V,\lambda)$]
\label{def:iVlmd}
By \eqref{eqn:indexV} in Lemma \ref{lem:ilsig}, 
 we define a map
\[
  i \colon \RInt \to \{0,1,\ldots,n\}, 
\]
see Lemma \ref{lem:regint}.  
We refer to 
\index{A}{iVlmd@$i(V,\lambda)$, height|textbf}
$i(V,\lambda)$ as the {\it{height}} of $(V,\lambda)$.  
We also refer it to as the 
\index{B}{heightIVlmd@height, of $I_{\delta}(V,\lambda)$|textbf}
height of the principal series representation
 $I_{\delta}(V,\lambda)$.  
\end{definition}

\begin{example}
\label{ex:iVlmdfig}
We illustrate the definition of the height
 $i(V,\lambda) \in \{0,1,\ldots,n\}$
 for $(V,\lambda) \in \RInt$
 by a graphic description 
 when $m(=[\frac n 2])=1$, 
 namely,
 when $n=2$ or $3$.  
In this case $G=O(n+1,1)$ is either $O(3,1)$ or $O(4,1)$, 
 and $V \in \widehat{O(n)}$ is given 
 by $V=\Kirredrep{O(n)}{\sigma_1}_{\varepsilon}$
 with $\sigma_1 \in {\mathbb{N}}$ and $\varepsilon \in \{\pm\}$.  
Then 
\begin{equation*}
\RInt \simeq
 \begin{cases}
\{(\sigma_1, \varepsilon,\lambda) \in {\mathbb{N}} \times \{\pm\}\times {\mathbb{Z}}:
\lambda-1 \ne \pm \sigma_1\}
\quad
 &\text{if $n=2$, }
\\
\{(\sigma_1, \varepsilon,\lambda) \in {\mathbb{N}} \times \{\pm\}\times {\mathbb{Z}}:
\lambda-2 \ne \pm \sigma_1, \lambda \ne 2\}
\quad
 &\text{if $n=3$.  }
\end{cases}
\end{equation*}
In the $(\sigma_1, \lambda)$-plane,
 the height $i(V,\lambda)$ is given as in Figure \ref{fig:iVlmd}.  

\renewcommand{\figurename}{Figure}
\begin{figure}[H]
\begin{center}
\includegraphics[scale=1]{figure-a.eps}
\end{center}
\caption{The height $i=i(V,\lambda)$
 for $(V,\lambda) \in \RInt$
 when $n=2,3$. }
\label{fig:iVlmd}
\end{figure}
\renewcommand{\figurename}{Diagram}

The red dots stand for 
 $(V,\lambda)=(\Exterior^j({\mathbb{C}}^n), j)$ 
 when $j=0,1,\ldots,n$. 
\end{example}


The case 
 where the height $i(V,\lambda)$ is equal to $\frac n 2$
 requires a special attention. 
\begin{lemma}
\label{lem:isig}
Let $m:=[\frac n 2]$.  
Suppose that $V =\Kirredrep {O(n)}{\sigma}_{\varepsilon}$ 
 with $\sigma \in \Lambda^+(m)$
 and $\varepsilon \in \{\pm\}$, 
 and $\lambda \in {\mathbb{Z}} \setminus S(V)$.  
\begin{enumerate}
\item[{\rm{(1)}}]
The height $i(V, \lambda)$ is equal to $\frac n 2$
 if and only if $n$ is even
 $(=2m)$
 and $\sigma_{m} > |\lambda -m|$.  
\item[{\rm{(2)}}]
If $\lambda \in S_Y(V)$
 (see \eqref{eqn:SYsigma}), 
 then $n$ is even $(=2m)$
 and $i(V, \lambda)=m$.  
\item[{\rm{(3)}}]
Suppose that $V$ is of type Y
 (Definition \ref{def:OSO}).  
Then,  
 for $(V, \lambda) \in \Reducible$, 
 the following two conditions are equivalent:
\begin{enumerate}
\item[{\rm{(i)}}]
$i(V,\lambda) = \frac n 2$;
\item[{\rm{(ii)}}]
$n$ is even 
 and $\lambda = \frac n 2$.  
\end{enumerate}
\end{enumerate}
\end{lemma}


\subsection{$K$-type formul{\ae} of irreducible $G$-modules
\label{subsec:Ktypes}}

In this section we provide explicit $K$-type formul{\ae}
 of irreducible representations of $G=O(n+1,1)$.  
The height $i(V,\lambda)$ plays a crucial role
 in describing the $K$-type formul{\ae}
 of irreducible subquotients
 of $I_{\delta}(V, \lambda)$, 
 see Proposition \ref{prop:KXred} (1).  

\subsubsection{$K$-type formula of $I_{\delta}(V, \lambda)$}
We begin with the $K$-type formula
 of the principal series representation
 $I_{\delta}(V, \lambda)$
 which generalizes Lemma \ref{lem:KtypeIi}
 for $I_{\delta}(i,i)$ 
 in the setting that $V=\Exterior^i({\mathbb{C}}^n)$.  

\begin{proposition}
[$K$-type formula of $I_{\delta}(V,\lambda)$]
\label{prop:KtypeIV}
\index{B}{Ktypeformula@$K$-type formula}
Let $G=O(n+1,1)$ and $m=[\frac n 2]$.  
Suppose that $V=\Kirredrep{O(n)}{\sigma}_{\varepsilon}$
 with $\sigma=(\sigma_1, \cdots,\sigma_m) \in \Lambda^+(m)$
 and $\varepsilon \in \{\pm\}$.  
\begin{enumerate}
\item[{\rm{(1)}}]
For $n=2m+1$, 
 the $K$-type formula 
 of the principal series representation 
 $I_{\delta}(V,\lambda)$ is given by
\[
  \bigoplus_{\mu} \Kirredrep{O(n+1)}{\mu_1, \cdots,\mu_{m+1}}_{\varepsilon}
  \boxtimes \delta(-1)^{\sum_{j=1}^{m+1} \mu_j - \sum_{j=1}^{m}\sigma_j},
\]
where $\mu=(\mu_1, \cdots,\mu_{m+1})$
 runs over $\Lambda^+(m+1)$ subject to 
\begin{equation}
\label{eqn:Kmuodd}
\mu_1 \ge \sigma_1 \ge \mu_2 \ge \sigma_2 \ge \cdots \ge \mu_m \ge \sigma_m 
\ge \mu_{m+1} \ge 0.  
\end{equation}

\item[{\rm{(2)}}]
For $n=2m$ and $V \in \widehat{O(n)}$ of type X
 (Definition \ref{def:OSO}), 
 the $K$-type formula 
 of $I_{\delta}(V,\lambda)$ is given by
\[
  \bigoplus_{\mu} \Kirredrep{O(n+1)}{\mu_1, \cdots,\mu_{m}}_{\varepsilon}
  \boxtimes \delta(-1)^{\sum_{j=1}^{m} \mu_j - \sum_{j=1}^{m}\sigma_j},
\]
where $\mu=(\mu_1, \cdots,\mu_{m})$
 runs over $\Lambda^+(m+1)$ subject to 
\begin{equation}
\label{eqn:Kmueven1}
\mu_1 \ge \sigma_1 \ge \cdots \ge \mu_m \ge \sigma_m 
(=0).  
\end{equation}

\item[{\rm{(3)}}]
For $n=2m$ and $V\in \widehat{O(n)}$ of type Y, 
 the $K$-type formula of $I_{\delta}(V,\lambda)$ is given by
\[
\bigoplus_{\kappa=\pm}
\bigoplus_{\mu} \Kirredrep{O(n+1)}{\mu_1, \cdots,\mu_{m}}_{\kappa\varepsilon}
  \boxtimes \kappa \delta(-1)^{\sum_{j=1}^{m} \mu_j - \sum_{j=1}^{m}\sigma_j},
\]
where $\mu=(\mu_1, \cdots,\mu_{m})$
 runs over $\Lambda^+(m)$ subject to 
\begin{equation}
\label{eqn:Kmueven2}
\mu_1 \ge \sigma_1 \ge \cdots \ge \mu_m \ge \sigma_m 
(>0).  
\end{equation}
\end{enumerate}
\end{proposition}

\begin{proof}
By the Frobenius reciprocity, 
Proposition \ref{prop:KtypeIV} follows from the classical
 branching rule for the restriction
 $O(n+1) \downarrow O(n)$, 
 see Fact \ref{fact:ONbranch}.  
\end{proof}

Since the principal series representation $I_{\delta}(V,\lambda)$
 of $G$ is multiplicity-free as a $K$-module,
 any subquotient of $I_{\delta}(V,\lambda)$
 can be characterized 
 by its $K$-types.  
In the next subsection, 
 we provide $K$-type formul{\ae}
 of subquotients of $I_{\delta}(V,\lambda)$
 based on Proposition \ref{prop:KtypeIV}.  


\subsubsection{$K$-types of subquotients
 $I_{\delta}(V,\lambda)^{\flat}$ and $I_{\delta}(V,\lambda)^{\sharp}$}

We recall from \eqref{eqn:reducible}
 and Theorem \ref{thm:irrIV}
 that the following two conditions on $(V,\lambda) \in \widehat{O(n)} \times {\mathbb{C}}$ 
 are equivalent.  
\begin{enumerate}
\item[(i)]
$(V,\lambda) \in \Reducible$, 
 {\it{i.e.}}, the $G$-module $I_{\delta}(V,\lambda)$ is reducible; 
\item[(ii)]
$\lambda \in {\mathbb{Z}} \setminus (S(V) \cup S_Y(V))$.  
\end{enumerate}

We note that $\lambda = \frac n 2$ belongs
 to ${\mathbb{Z}} \setminus (S(V) \cup S_Y(V))$
 when $n$ is even.  



In this section, 
  we describe the $K$-types of the subquotients
 $I_{\delta}(V,\lambda)^{\flat}$ and $I_{\delta}(V,\lambda)^{\sharp}$
 when the principal series representation
 $I_{\delta}(V,\lambda)$ is reducible, 
 {\it{i.e.}}, 
 when $(V,\lambda) \in \Reducible$, 
 see \eqref{eqn:Red}.  



We shall see that the description 
 depends on the height $i(V,\lambda)$ 
 (Definition \ref{def:iVlmd})
 when $\lambda = \frac n 2$.  
To be more precise,
 let $m=[\frac n 2]$ and $V \in \widehat{O(n)}$.  
Suppose $\lambda \in {\mathbb{Z}} \setminus(S(V) \cup S_Y(V))$
 and we define $i$ to be the height
 $i(V,\lambda) \in \{0,1,\ldots,n\}$.  
We write $V=\Kirredrep{O(n)}{\sigma}_{\varepsilon}$
 with $\sigma=(\sigma_1, \cdots, \sigma_m)\in \Lambda^+(m)$
 and $\varepsilon \in \{\pm\}$ as before.  
We observe the following:
\begin{enumerate}
\item[$\bullet$]
if $i< \frac{n-1}{2}$, 
 then $1 \le i+1 \le m$
 and the condition $i-\sigma_i < \lambda< i+1 -\sigma_{i+1}$
 (Definition \ref{def:RVi})
 amounts to 
\begin{equation}
\label{eqn:barK1}
 \sigma_{i+1} \le i-\lambda
\quad\text{and}\quad 
 i-\lambda+1 \le \sigma_i;
\end{equation}
\item[$\bullet$]
if $i=\frac{n-1}{2}$, 
 then $n$ is odd $(=2m+1)$ 
 and we have
\begin{equation}
\label{eqn:barK3}
0 \le m-\lambda
\quad\text{and}\quad
m-\lambda+1 \le \sigma_m;
\end{equation}

\item[$\bullet$]
if $i=\frac{n+1}{2}$, 
 then $n$ is odd $(=2m+1)$ 
 and we have
\begin{equation}
\label{eqn:barK4}
0 \le \lambda-m-1
\quad\text{and}\quad
\lambda-m \le \sigma_m; 
\end{equation}
\item[$\bullet$]
if $\frac{n+1}2 < i$, 
 then $ 1 \le n-i+1 \le m$
 and the condition $\sigma_{n-i+1}+ i-1 < \lambda< \sigma_{n-i} +i$
 amounts to  
\begin{equation}
\label{eqn:barK2}
\sigma_{n-i+1} \le \lambda-i
\quad\text{and}\quad 
\lambda-i+1 \le \sigma_{n-i}.  
\end{equation}
\end{enumerate}

We recall that the principal series representation
$I_{\delta}(V,\lambda)$
 of $G=O(n+1,1)$
 is $K$-multiplicity-free,
 and its $K$-type formula 
 is given explicitly in Proposition \ref{prop:KtypeIV}.  
To describe the $K$-type formul{\ae}
 of subquotients of $I_{\delta}(V,\lambda)$, 
 we use the inequalities
 \eqref{eqn:barK1}--\eqref{eqn:barK2}
 in Proposition \ref{prop:KXred} (1) below.  

\begin{proposition}
[$K$-type formul{\ae} of subquotients]
\label{prop:KXred}
Suppose that $(V,\lambda) \in \Reducible$, 
 or equivalently, 
 $V \in \widehat{O(n)}$
 and $\lambda \in {\mathbb{Z}} \setminus(S(V) \cup S_Y(V))$, 
 see Theorem \ref{thm:irrIV}.  
Let 
\index{A}{iVlmd@$i(V,\lambda)$, height}
$i:=i(V,\lambda) \in \{0,1,\ldots,n\}$
 be the height of $(V,\lambda)$
as in Definition \ref{def:iVlmd}.  
\begin{enumerate}
\item[{\rm{(1)}}]
Suppose $\lambda \ne \frac n2$.  
In this case $i \ne \frac n 2$. 
Then the $K$-types of the submodule $I_{\delta}(V,\lambda)^{\flat}$
 and the quotient $I_{\delta}(V,\lambda)^{\sharp}$
 of $I_{\delta}(V,\lambda)$, 
 see Proposition \ref{prop:Xred}, 
 are subsets of the $K$-types of $I_{\delta}(V,\lambda)$
 (Proposition \ref{prop:KtypeIV})
 characterized by the following additional inequalities:
\par\noindent
$\bullet$\enspace
for $i \le \frac{n-1}2$, 
the condition $\sigma_{i+1} \le \mu_{i+1} \le \sigma_i$
 in \eqref{eqn:Kmuodd}--\eqref{eqn:Kmueven2}
 is divided as follows:
\begin{alignat*}{5}
(\sigma_{i+1} 
&\le)\,\,
&&\mu_{i+1} 
&& 
&&\le i-\lambda
\qquad
&&\text{for $I_{\delta}(V,\lambda)^{\flat}$}, 
\\
 i-\lambda+1
&\le\,\,
&&\mu_{i+1} 
&& (
&&\le \sigma_i)
\qquad
&&\text{for $I_{\delta}(V,\lambda)^{\sharp}$};
\\
\intertext{$\bullet$\enspace for $\frac{n+1}{2}\le i$, 
the condition $\sigma_{n-i+1} \le \mu_{n-i+1} \le \sigma_{n-i}$
 in \eqref{eqn:Kmuodd}--\eqref{eqn:Kmueven2}
 is divided as follows:}
 \lambda-i+1 
&\le\,\,
&&\mu_{n-i+1} 
&& (
&&\le \sigma_{n-i})
\qquad
&&\text{for $I_{\delta}(V,\lambda)^{\flat}$}, 
\\
(\sigma_{n-i+1} 
&\le)\,\,
&&\mu_{n-i+1} 
&& 
&&\le \lambda-i
\qquad
&&\text{for $I_{\delta}(V,\lambda)^{\sharp}$}.  
\end{alignat*}
Here we regard $\sigma_{m+1}=0$
 (this happens when $i=\frac{n \pm 1}{2}$).  
\item[{\rm{(2)}}]
Suppose $\lambda =\frac n 2$.  
In this case $n$ is even $(=2m)$ and $i=m$.  
Then the $K$-types
 of the submodules $I_{\delta}(V,\lambda)^{\flat}$
 and $I_{\delta}(V,\lambda)^{\sharp}$
 of the (tempered) principal series representation
 $I_{\delta}(V,\lambda)$, 
 see Proposition \ref{prop:IVmtemp}, 
 are given by 
\begin{alignat*}{2}
  &\bigoplus_{\mu}
  \Kirredrep{O(n+1)}{\mu_1, \cdots,\mu_{m}}_{\varepsilon}
  \boxtimes 
   \delta(-1)^{\sum_{j=1}^{m} \mu_j - \sum_{j=1}^{m}\sigma_j}
   \quad
&&\text{for $I_{\delta}(V,\lambda)^{\flat}$}, 
\\
  &\bigoplus_{\mu}
  \Kirredrep{O(n+1)}{\mu_1, \cdots,\mu_{m}}_{-\varepsilon}
  \boxtimes 
   \delta(-1)^{\sum_{j=1}^{m} \mu_j - \sum_{j=1}^{m}\sigma_j-1}
   \quad
&&\text{for $I_{\delta}(V,\lambda)^{\sharp}$}
\end{alignat*}
where $\mu=(\mu_1, \cdots,\mu_{m})$
 runs over $\Lambda^+(m)$
 subject to \eqref{eqn:Kmueven2}.  
\end{enumerate}
\end{proposition}

\begin{proof}
The $K_0$-types for all irreducible subquotients
 of principal series representations
 of the connected Lie group $G_0=SO_0(n+1,1)$
 were obtained in Hirai \cite{Hirai62}, from
 which analogous results
 for the group $\overline G=S O(n+1,1)$
 are easily shown.  
Our concern is with the group $G=O(n+1,1)$.  
Then the first assertion follows from Proposition \ref{prop:KtypeIV}
 on the $K$-type formula 
 of $I_{\delta}(V,\lambda)$
 and from the branching rule for the restriction
 $G \downarrow \overline G$
 in Propositions \ref{prop:20180906}
 and \ref{prop:1808104} in Appendix II.  
The second assertion follows from the branching rule
 of $I_{\delta}(V,\frac n 2)$
 for the restriction $G\downarrow \overline G$
 in Proposition \ref{prop:180572}.  
\end{proof}



\subsection{$(\delta, V, \lambda)
\rightsquigarrow (\delta^{\uparrow},V^{\uparrow}, \lambda^{\uparrow})$
 and $(\delta^{\downarrow},V^{\downarrow}, \lambda^{\downarrow})$}
\label{subsec:HasseV}

In this section,
 we introduce a correspondence
\begin{alignat*}{4}
&\delta \in \{\pm\}, 
&&V \in \widehat{O(n)}, 
&&
\text{and }
&&\lambda \in {\mathbb{Z}} \setminus(S(V) \cup S_Y(V))
\\
&
&&
&&
&&
\rotatebox{270}{$\rightsquigarrow$}
\\
&\delta^{\uparrow} \in \{\pm\},\,\,
&&V^{\uparrow} \in \widehat{O(n)},\,\, 
&&
\text{and }\,\,
&&
\lambda^{\uparrow} \in {\mathbb{Z}} \setminus(S(V^{\uparrow}) \cup S_Y(V^{\uparrow}))
\end{alignat*}
satisfying the following two properties (Proposition \ref{prop:Vlmdup}):
\begin{align*}
  i(V^{\uparrow}, \lambda^{\uparrow}) =\,& i(V,\lambda)+1
\\
  I_{\delta^{\uparrow}}(V^{\uparrow}, \lambda^{\uparrow})^{\flat} \simeq\,
& I_{\delta}(V,\lambda)^{\sharp}.  
\end{align*}
We retain the notation
 that $G=O(n+1,1)$
 and $m = [\frac n 2]$.  

\begin{definition}
\label{def:180909}
Suppose that 
 $V=\Kirredrep{O(n)}{\sigma}_{\varepsilon}$
 with $\sigma \in \Lambda^+(m)$
 and $\varepsilon \in \{\pm\}$, 
 and $\lambda \in {\mathbb{Z}} 
 \setminus S(V)$.  
Let $i:=i(V,\lambda) \in \{0,1,\ldots,n\}$
 be the height of $(V,\lambda)$
 as in Lemma \ref{lem:ilsig}.  
\begin{enumerate}
\item[{\rm{(1)}}]
We assume $0 \le i \le n-1$, 
 or equivalently, 
 $\lambda \le \sigma_1 +n-1$.  
We define
\index{A}{0deltaVlmd1@$(\delta,V,\lambda)^{\uparrow}
 =(\delta^{\uparrow},V^{\uparrow},\lambda^{\uparrow})$|textbf}
\begin{equation}
\label{eqn:upVlmd}
 (\delta,V,\lambda)^{\uparrow} \equiv (\delta^{\uparrow},V^{\uparrow},\lambda^{\uparrow})
\in {\mathbb{Z}}/2{\mathbb{Z}} \times \widehat{O(n)} \times {\mathbb{Z}}
\end{equation}
with $V^{\uparrow} := \Kirredrep{O(n)}{\sigma^{\uparrow}}_{\varepsilon}$ 
 as follows:

\begin{enumerate}
\item[$\bullet$]
For $\lambda < \frac n 2$, 
 we have $0 \le i < \frac n2$
 and set
\begin{align*}
\delta^{\uparrow} := & \delta(-1)^{i+1-\sigma_{i+1}-\lambda}, 
\\
\sigma^{\uparrow} := & (\sigma_{1}, \cdots, \sigma_{i}, i+1-\lambda, 
 \sigma_{i+2}, \cdots, \sigma_{m}), 
\\
\lambda^{\uparrow} := & i+1-\sigma_{i+1}.  
\end{align*}
\item[$\bullet$]
For $\frac n 2 \le \lambda \le \sigma_1 +n-1$, 
 we have $\frac n2 \le i \le n-1$
 and set
\begin{align*}
\delta^{\uparrow} := & \delta(-1)^{\lambda-\sigma_{n-i}-i}, 
\\
\sigma^{\uparrow} := & (\sigma_{1}, \cdots, \sigma_{n-i-1}, \lambda-i, 
 \sigma_{n-i+1}, \cdots, \sigma_{m}), 
\\
\lambda^{\uparrow} := & \sigma_{n-i} +i.  
\end{align*}
\end{enumerate}


\item[{\rm{(2)}}]
Conversely,
 for $1 \le i \le n$, 
 namely, 
 for $1-\sigma_1 \le \lambda$, 
 we define
\index{A}{0deltaVlmd2@$(\delta,V,\lambda)^{\downarrow}
 =(\delta^{\downarrow},V^{\downarrow},\lambda^{\downarrow})$|textbf}
\begin{equation}
\label{eqn:downVlmd}
  (\delta,V,\lambda)^{\downarrow}
  \equiv
  (\delta^{\downarrow},V^{\downarrow},\lambda^{\downarrow})
\end{equation}
 as the inverse of the correspondence
\[
   (\delta,V,\lambda) \mapsto (\delta,V,\lambda)^{\uparrow}.  
\]
\end{enumerate}
\end{definition}

A prototype for Definition \ref{def:180909}
 appeared implicitly
 in Theorem \ref{thm:LNM20}
 for the principal series representations $I_{\delta}(i,i)$
 having the trivial ${\mathfrak{Z}}_G({\mathfrak{g}})$-infinitesimal character $\rho_G$.  
We now explain this explicitly
 as an example for $(V,\lambda)=(\Exterior^i({\mathbb{C}}^n), i)$
 $(1 \le i \le n)$:  
\begin{example}
\label{ex:up}
For the exterior representations 
 $\Exterior^i({\mathbb{C}}^n)$
 of $O(n)$, 
 we have
\begin{alignat*}{2}
 (\delta,\Exterior^i({\mathbb{C}}^n), i)^{\uparrow}
 =&
 (-\delta,\Exterior^{i+1}({\mathbb{C}}^n), i+1)
 \quad
&&\text{for $0 \le i \le n-1$, }
\\
(\delta,\Exterior^i({\mathbb{C}}^n), i)^{\downarrow}
 =&
 (-\delta,\Exterior^{i-1}({\mathbb{C}}^n), i-1)
 \quad
&&\text{for $1 \le i \le n$.  }
\end{alignat*}
The proof follows directly from 
 the definition, 
 see Example \ref{ex:exteriorpm}.  
\end{example}

Here are basic properties of the correspondence
\[
  (\delta,V,\lambda) \mapsto
(\delta^{\uparrow},V^{\uparrow},\lambda^{\uparrow})
\quad
\text{or}
\quad
(\delta^{\downarrow}, V^{\downarrow}, \lambda^{\downarrow}).  
\]

\begin{proposition}
\label{prop:Vlmdup}
Suppose that $(V,\lambda) \in \RInt$, 
 {\it{i.e.}}, 
 $V \in \widehat{O(n)}$
 and $\lambda \in {\mathbb{Z}} \setminus S(V)$.  
In what follows, 
 we assume the height $i(V,\lambda)$ is not equal to $n$
 when we consider $(V^{\uparrow}, \lambda^{\uparrow})$, 
 and is nonzero 
 when we consider $(V^{\downarrow}, \lambda^{\downarrow})$.  
\begin{enumerate}
\item[{\rm{(1)}}]
$r(V^{\uparrow},\lambda^{\uparrow}), 
r(V^{\downarrow},\lambda^{\downarrow})
\in W_G\, r(V,\lambda)$, 
 see \eqref{eqn:IVZG}.  
In particular,
 $(V^{\uparrow},\lambda^{\uparrow})$, 
 $(V^{\downarrow},\lambda^{\downarrow}) \in \RInt$.  
\item[{\rm{(2)}}]
$i(V^{\uparrow}, \lambda^{\uparrow})-1=i(V, \lambda)
 =i(V^{\downarrow}, \lambda^{\downarrow})+1$.  
\item[{\rm{(3)}}]
$\delta^{\uparrow} (-1)^{\lambda^{\uparrow}}
=\delta (-1)^{\lambda}
=\delta^{\downarrow} (-1)^{\lambda^{\downarrow}}$. 
\item[{\rm{(4)}}]
$(V^{\uparrow}, \lambda^{\uparrow}),  
(V^{\downarrow}, \lambda^{\downarrow})\in \Reducible$, 
 if $(V, \lambda) \in \Reducible$,
 see \eqref{eqn:reducible}.  
\item[{\rm{(5)}}]
Suppose that $(V,\lambda) \in \Reducible$
 and $\lambda \ne \frac n 2$.  
Then the unique submodule of $I_{-\delta}(V^{\uparrow}, \lambda^{\uparrow})$
 is isomorphic to the unique quotient
 of $I_{\delta}(V, \lambda)$, 
 that is, 
 we have the following $G$-isomorphisms
 with the notation 
 as in Proposition \ref{prop:Xred}:
\begin{align*}
  I_{\delta^{\uparrow}}(V^{\uparrow}, \lambda^{\uparrow})^{\flat}
  \simeq&
  I_{\delta}(V, \lambda)^{\sharp}, 
\\
  I_{\delta^{\downarrow}}(V^{\downarrow}, \lambda^{\downarrow})^{\sharp}
  \simeq&
  I_{\delta}(V, \lambda)^{\flat}.  
\end{align*}
\end{enumerate}
\end{proposition}

With these notations,
 we give the formul\ae\ for the minimal $K$-types
 of the irreducible subquotients 
 $I_{\delta}(V,\lambda)^{\flat}$ and $I_{\delta}(V,\lambda)^{\sharp}$
 in $I_{\delta}(V,\lambda)$
 in the setting of Proposition \ref{prop:Xred}.  

\begin{proposition}
\label{prop:minKXred}
Let $G=O(n+1,1)$ and $m=[\frac n2]$.  
Suppose $V=\Kirredrep{O(n)}{\sigma}_{\varepsilon}$
 with $\sigma=(\sigma_1, \cdots,\sigma_m) \in \Lambda^+(m)$
 and $\varepsilon \in \{\pm\}$.  
Let $\delta \in \{\pm\}$ 
 and $\lambda \in {\mathbb{Z}} \setminus (S(V) \cup S_Y(V) \cup \{\frac n 2\})$. 
\begin{enumerate}
\item[{\rm{(1)}}]
The minimal $K$-types of $I_{\delta}(V,\lambda)^{\flat}$ for $\lambda < \frac n 2$
 and of $I_{\delta}(V,\lambda)^{\sharp}$
 for $\lambda>\frac n 2$ are given by
\begin{alignat*}{2}
&\Kirredrep{O(n+1)}{\sigma}_{\varepsilon}\boxtimes \delta
&&\text{for $n=2m$ and $\sigma_m=0$, }
\\
&\Kirredrep{O(n+1)}{\sigma}_{\varepsilon}\boxtimes \delta,
\quad
\Kirredrep{O(n+1)}{\sigma}_{-\varepsilon}\boxtimes (-\delta)
\qquad
&&\text{for $n=2m$ and $\sigma_m>0$, }
\\
&\Kirredrep{O(n+1)}{\sigma,0}_{\varepsilon}\boxtimes \delta
\qquad
&&\text{for $n=2m+1$.}
\end{alignat*}
\item[{\rm{(2)}}]
The minimal $K$-types of $I_{\delta}(V,\lambda)^{\sharp}$ for $\lambda < \frac n 2$
 and of $I_{\delta}(V,\lambda)^{\flat}$
 for $\lambda > \frac n 2$ are given by
\begin{alignat*}{2}
&\Kirredrep{O(n+1)}{\sigma^{\uparrow}}_{\varepsilon}\boxtimes \delta^{\uparrow}
&&\text{for $n=2m$ and $\sigma_m=0$, }
\\
&\Kirredrep{O(n+1)}{\sigma^{\uparrow}}_{\varepsilon}\boxtimes \delta^{\uparrow},
\quad
\Kirredrep{O(n+1)}{\sigma^{\uparrow}}_{-\varepsilon}\boxtimes (-\delta^{\uparrow})
\qquad
&&\text{for $n=2m$ and $\sigma_m>0$, }
\\
&\Kirredrep{O(n+1)}{\sigma^{\uparrow},0}_{\varepsilon}\boxtimes \delta^{\uparrow}
\qquad
&&\text{for $n=2m+1$.}
\end{alignat*}
\end{enumerate}
\end{proposition}



\subsection{Classification of irreducible admissible representations
 of $G=O(n+1,1)$}

Irreducible admissible representations
 of the connected group $G_0=SO_0(n+1,1)$
 were classified infinitesimally
 ({\it{i.e.,}} on the level of $({\mathfrak{g}}, K_0)$-modules)
 by Hirai \cite{Hirai62}, 
 see also Borel--Wallach \cite{BW}
 and Collingwood \cite[Chap.~5]{C}.  
However,
 we could not find in the literature
 a classification of irreducible admissible representations
 of the indefinite orthogonal group $G=O(n+1,1)$, 
 which is not in the Harish-Chandra class
 when $n$ is even.  
For the sake of completeness,
 we give an infinitesimal classification 
 of irreducible admissible representations of $G$, 
 or equivalently,
 give a classification
 of irreducible $({\mathfrak{g}}, K)$-modules
 in this section.  
Moreover we give three characterizations
 of the irreducible representations of $G$
 when they are neither principal series representations
 nor tempered representations, 
 see Theorem \ref{thm:1808116}.  



\subsubsection{Characterizations of the irreducible subquotients
 $\Pi_{\delta}(V,\lambda)$}

We recall from Section \ref{subsec:subIii}
 the irreducible representations $\Pi_{\ell, \delta}$
 of $G$
 that have the trivial ${\mathfrak{Z}}_G({\mathfrak{g}})$-infinitesimal character $\rho_G$.  
Analogously to the notation $\Pi_{\ell, \delta}$
 in \eqref{eqn:Pild}
 for $\operatorname{Irr}(G)_{\rho}$, 
 we set
\index{A}{1PiideltaV@$\Pi_{\delta}(V,\lambda)$|textbf}
\begin{equation}
\label{eqn:PiVlmd}
  \Pi_{\delta}(V,\lambda):=I_{\delta}(V, \lambda)^{\flat}
\end{equation}
for $\delta \in \{\pm\}$
 and $(V,\lambda) \in \Reducible$.  
If $i(V,\lambda) \ne 0$, 
 then we have a $G$-isomorphism
\begin{equation}
\label{eqn:PiVlmddown}
  \Pi_{\delta}(V,\lambda) \simeq I_{\delta^{\downarrow}}(V^{\downarrow}, \lambda^{\downarrow})^{\sharp}, 
\end{equation}
where $(\delta^{\downarrow}, V^{\downarrow}, \lambda^{\downarrow})$
 is given in Definition \ref{def:180909}.  
We also have a $G$-isomorphism
\begin{equation}
\label{eqn:PiVlmddual}
  \Pi_{\delta}(V,\lambda) \simeq I_{\delta}(V, n-\lambda)^{\sharp}.  
\end{equation}
We have already discussed
 in Proposition \ref{prop:IVmtemp}
 irreducible subquotients
 of reducible tempered principal series representations
 $I_{\delta}(V, \lambda)$ 
under the assumption
 that $(V, \lambda)\in\Reducible$
 with $\lambda=\frac n 2$.  
This assumption implies that $n$ is even, 
 $V$ is of type Y
 and $\lambda = \frac n 2$.  
The next theorem discusses the remaining
 (and the important) case
 when the principal series representation $I_{\delta}(V, \lambda)$ 
 is reducible, 
 namely, 
 $(V,\lambda) \in \Reducible$
 with an additional condition $\lambda \ne \frac n 2$.  

\begin{theorem}
[characterizations of $\Pi_{\delta}(V, \lambda)$]
\label{thm:1808116}
Let $G=O(n+1,1)$, 
 and we set $m:=[\frac n 2]$.  
Assume that $(V, \lambda)\in\Reducible$.  
This means that $V \in \widehat{O(n)}$ and
\[
  \lambda \in 
\begin{cases}
  {\mathbb{Z}} \setminus S(V)
  \quad&\text{if $n=2m+1$, }
\\
{\mathbb{Z}} \setminus (S(V) \cup S_Y(V))
  \quad&\text{if $n=2m$, }
\end{cases}
\]
see Theorem \ref{thm:irrIV}.  
We further assume that $\lambda \ne \frac n 2$.  
\begin{enumerate}
\item[{\rm{(1)}}]
{\rm{(Langlands subrepresentation of principal series)}}\enspace
For $\delta \in \{\pm\}$, 
 $\Pi_{\delta}(V, \lambda)$ is the unique proper $G$-submodule
 of $I_{\delta}(V, \lambda)$.  

\item[{\rm{(2)}}]
{\rm{($\theta$-stable parameter)}}
\enspace
Let $i:=i(V,\lambda) \in \{0,1,\ldots,n\}$
 be the height of $(V,\lambda)$
 as in \eqref{eqn:indexV}.  
We write $V=\Kirredrep{O(n)}{\sigma}_{\varepsilon}$
 with $\sigma=(\sigma_1, \cdots,\sigma_m) \in \Lambda^+(m)$
 and $\varepsilon \in \{\pm\}$.   
Then the underlying $({\mathfrak {g}}, K)$-module of $\Pi_{\delta}(V, \lambda)$
 is given by means of $\theta$-stable parameter
 (see Section \ref{subsec:thetapara})
 as 
\[
\Pi_{\delta}(V, \lambda)_K
\simeq
\begin{cases}
\Rq{\sigma_1-1, \cdots, \sigma_i-1}{i-\lambda,\sigma_{i+1}, \cdots, \sigma_m}
{\varepsilon,\delta \varepsilon}
&\text{if $\lambda < \frac n 2$, }
\\
\Rq{\sigma_1-1, \cdots, \sigma_{n-i}-1, \lambda-i}{\sigma_{n-i+1}, \cdots, \sigma_m}
{\varepsilon,-\delta \varepsilon}
\,\,
&\text{if $\frac n 2< \lambda$.  }
\end{cases}
\]

\item[{\rm{(3)}}]
{\rm{(coherent family starting at $\Pi_{i,\delta} \in \operatorname{Irr}(G)_{\rho}$)}}\enspace
We set 
\begin{equation*}
   \text{$r(V, \lambda)\in {\mathbb{C}}^{m+1}$ $(\simeq {\mathfrak{h}}_{\mathbb{C}}^{\ast})$ 
         as in \eqref{eqn:IVZG}}.  
\end{equation*}
Denote by 
\index{A}{Pmu@$P_{\mu}$}
$P_{\mu}$ 
the projection to the primary component 
 with the generalized ${\mathfrak{Z}}_G({\mathfrak{g}})$-infinitesimal character
 $\mu \in {\mathfrak{h}}_{\mathbb{C}}^{\ast} \mod W_G$
 (see Section \ref{subsec:primary} in Appendix III).  
Let 
\index{A}{FVlmd@$F(V,\lambda)$}
$F(V,\lambda)$ be the irreducible finite-dimensional representation
 of $G=O(n+1,1)$, 
 which will be defined in Definition \ref{def:Fshift} 
 in Appendix III.  
Then there is a natural $G$-isomorphism:
\[
\Pi_{\delta}(V,\lambda)
\simeq
P_{r(V, \lambda)}(\Pi_{i,\delta} \otimes F(V,\lambda)). 
\]

\item[{\rm{(4)}}]
{\rm{(Hasse sequence and standard sequence starting at $F(V,\lambda)$)}}
\enspace
\index{B}{Hassesequence@Hasse sequence}
\index{B}{standardsequence@standard sequence}
Let 
\index{A}{1PiideltaF@$\Pi_i(F)$, standard sequence starting with $F$}
$\Pi_j(F)$
 $(j=0,1,\cdots,n)$
 be the standard sequence 
 starting with an irreducible finite-dimensional 
 representation $F$ of $G$
 (Definition \ref{def:Hasse}), 
 and $i=i(V,\lambda)$ the height of $(V,\lambda)$.  
Then there is a natural $G$-isomorphism:
\[
\Pi_{\delta}(V, \lambda)
\simeq
\Pi_{i}(F(V,\lambda)) \otimes \chi_{+\delta}. 
\]
\end{enumerate}
\end{theorem}

See Proposition \ref{prop:Xred} for (1), 
 Theorem \ref{thm:IndV} for (2), 
 Theorem \ref{thm:1807113} for (3)
 in Chapter \ref{sec:Translation}
 (Appendix III), 
 and Theorems \ref{thm:171471} and \ref{thm:171471b} for (4).  


\subsubsection{Classification of $\operatorname{Irr}(G)$}
\label{subsec:IrrG}
We give an infinitesimal classification 
 of irreducible admissible representations of $G=O(n+1,1)$.  
One may reduce the proof to the case 
 of connected groups, 
 by inspecting the restriction 
 to the subgroups $\overline G=SO(n+1,1)$
 or $G_0=SO_0(n+1,1)$, 
 see Chapter \ref{sec:SOrest} (Appendix II).  


\begin{theorem}
[classification of $\operatorname{Irr}(G)$]
\label{thm:irrG}
Irreducible admissible representations
 of moderate growth
 of $G=O(n+1,1)$ are listed as follows:
\index{A}{1PiideltaV@$\Pi_{\delta}(V,\lambda)$}
\begin{alignat*}{2}
&\bullet\enspace I_{\delta}(V, \lambda)
\qquad
&& \lambda \in ({\mathbb{C}} \setminus {\mathbb{Z}}) \cup S(V) \cup S_Y(V), 
\\
&\bullet\enspace 
\Pi_{\delta}(V, \lambda)
\qquad
&&\text{$\lambda \in {\mathbb{Z}} \setminus (S(V) \cup S_Y(V))$ and $\lambda \le \frac n 2$, }
\end{alignat*}
where $V \in \widehat{O(n)}$
 and $\delta \in \{\pm\}$.  
\end{theorem}
We note that there is an isomorphism
 of irreducible $G$-modules:
\[
  I_{\delta}(V, \lambda)\simeq I_{\delta}(V, n-\lambda)
\]
 when $\lambda \in ({\mathbb{C}} \setminus {\mathbb{Z}}) \cup S(V) \cup S_Y(V)$.  






%%%%%%%%%%%%%%%%%%%%%%%%%%%%%%%%%%%%%%%%%%%%
\subsection{$\theta$-stable parameters
 and cohomological parabolic induction}
\label{subsec:thetapara}
%%%%%%%%%%%%%%%%%%%%%%%%%%%%%%%%%%%%%%%%%%%%

In this section
 we give a parametrization of irreducible subquotients
 of the principal series representations
\[
    I_{\delta}(V,\lambda)={\operatorname{Ind}}_P^G(V \otimes \delta \otimes {\mathbb{C}}_{\lambda})
\]
 of the group $G=O(n+1,1)$ in terms of cohomological parabolic induction.  

\subsubsection{Cohomological parabolic induction $A_{\mathfrak{q}}(\lambda)={\mathcal{R}}_{\mathfrak{q}}^S({\mathbb{C}}_{\lambda+\rho({\mathfrak{u}})})$}
\label{subsec:Aqgeneral}
We fix some notation of
\index{B}{cohomologicalparabolicinduction@cohomological parabolic induction|textbf}
 cohomological parabolic induction.  
A basic reference is Vogan \cite{Vogan81} and Knapp--Vogan \cite{KV}.  
We begin with a {\it{connected}} real reductive Lie group $G$.  
Let $K$ be a maximal compact subgroup, 
 and $\theta$ the corresponding Cartan involution.  
Given an element $X \in {\mathfrak{k}}$, 
 the complexified Lie algebra
 ${\mathfrak{g}}_{\mathbb{C}}={\operatorname{Lie}}(G) \otimes_{\mathbb{R}}
{\mathbb{C}}$ is decomposed into the eigenspaces
 of $\sqrt{-1}{\operatorname{ad}}(X)$, 
 and we write 
\[
   {\mathfrak{g}}_{\mathbb{C}}
   ={\mathfrak{u}}_- + {\mathfrak{l}}_{\mathbb{C}} + {\mathfrak{u}}
\]
 for the sum of the eigenspaces 
 with negative, zero, 
 and positive eigenvalues.  
Then ${\mathfrak{q}}:={\mathfrak{l}}_{\mathbb{C}}+{\mathfrak{u}}$
 is a $\theta$-stable parabolic subalgebra 
 with Levi subgroup 
\begin{equation}
\label{eqn:LeviLq}
   L =\{g \in G: {\operatorname{Ad}}(g) {\mathfrak{q}}={\mathfrak{q}}\}.  
\end{equation}
The homogeneous space $G/L$ is endowed
 with a $G$-invariant complex manifold structure 
 with holomorphic cotangent bundle $G \times_L {\mathfrak{u}}$.  
As an algebraic analogue of Dolbeault cohomology groups
 for $G$-equivariant holomorphic vector bundle over $G/L$, 
 Zuckerman introduced a cohomological parabolic induction functor
 ${\mathcal{R}}_{\mathfrak{q}}^j(\cdot \otimes {\mathbb{C}}
_{\rho({\mathfrak{u}})})$ ($j \in {\mathbb{N}}$) from the category
 of $({\mathfrak{l}}, L \cap K)$-modules
 to the category of $({\mathfrak{g}}, K)$-modules. 
We adopt here the normalization of the cohomological parabolic induction
 ${\mathcal{R}}_{{\mathfrak {q}}}^{j}$ from a $\theta$-stable parabolic subalgebra
 ${\mathfrak {q}}={\mathfrak {l}}_{\mathbb{C}}+{\mathfrak {u}}$
 so that the ${\mathfrak{Z}}({\mathfrak {g}})$-infinitesimal character
 of the $({\mathfrak{g}},K)$-module 
 ${\mathcal{R}}_{{\mathfrak {q}}}^{j}(F)$ equals
\[
\text{
 the ${\mathfrak{Z}}({\mathfrak {l}})$-infinitesimal character
 of the ${\mathfrak {l}}$-module $F$
}
\]
 modulo the Weyl group 
 via the Harish-Chandra isomorphism.  

We note that if $F'$ is an $({\mathfrak{l}}, L \cap K)$-module
 then $F:=F' \otimes {\mathbb{C}}_{\rho({\mathfrak{u}})}$
 may not be defined 
 as an $({\mathfrak{l}}, L \cap K)$-module, 
 but can be defined as a module of the metaplectic covering group of $L$.  
When $F$ satisfies a positivity condition called 
\index{B}{goodrange@good range}
 \lq\lq{good range of parameters}\rq\rq, 
 the cohomology ${\mathcal{R}}_{\mathfrak{q}}^j(F)$ concentrates
 on the degree 
\[
   S:=\dim_{\mathbb{C}}({\mathfrak{u}} \cap {\mathfrak{k}}_{\mathbb{C}}).  
\]
For a one-dimensional representation $F$, 
 we also use another convention
 \lq\lq{$A_{\mathfrak{q}}(\lambda)$}\rq\rq.  
Following the normalization of Vogan--Zuckerman \cite{VZ}, 
 we set
\[
  A_{\mathfrak{q}}(\lambda):={\mathcal{R}}_{\mathfrak{q}}^S({\mathbb{C}}
_{\lambda+\rho({\mathfrak{u}})})
\]
for a one-dimensional representation ${\mathbb{C}}_{\lambda}$
 of $L$.  
In particular,
 we set
\[
   A_{\mathfrak{q}}:=A_{\mathfrak{q}}(0)
                   ={\mathcal{R}}_{\mathfrak{q}}^S({\mathbb{C}}_{\rho({\mathfrak{u}})}), 
\]
 which is an irreducible $({\mathfrak{g}},K)$-module
 with the same ${\mathfrak{Z}}({\mathfrak{g}})$-infinitesimal character $\rho$
 as that of the trivial one-dimensional representation ${\bf{1}}$ of $G$.  



Similar notation will be used for disconnected groups $G$.  
For a character $\chi$ of the component group $G/G_0$, 
 we have an isomorphism of $({\mathfrak{g}},K)$-modules:
\[
   (A_{\mathfrak{q}})_{\chi}:=A_{\mathfrak{q}} \otimes \chi
   \simeq {\mathcal{R}}_{\mathfrak{q}}^S(\chi\otimes {\mathbb{C}}_{\rho({\mathfrak{u}})}).  
\]



\subsubsection{$\theta$-stable parabolic subalgebra ${\mathfrak{q}}_i$
 for $G=O(n+1,1)$}
\label{subsec:qi}
We apply the general theory
 reviewed in Section \ref{subsec:Aqgeneral}
 to the group $G=O(n+1,1)$.  
For this, 
we set up some notation 
 for $\theta$-stable parabolic subalgebra ${\mathfrak{q}}_i$
 and ${\mathfrak{q}}_{\frac{n+1}{2}}^{\pm}$
 of ${\mathfrak{g}}_{\mathbb{C}} = {\operatorname{Lie}}(G) \otimes_{\mathbb{R}} {\mathbb{C}}
\simeq {\mathfrak{o}}(n+2,{\mathbb{C}})$
 as follows.  



We take a Cartan subalgebra ${\mathfrak{t}}^c$ of ${\mathfrak{k}}$, 
 and extend it to a fundamental Cartan subalgebra
 ${\mathfrak{h}}={\mathfrak{t}}^c + {\mathfrak{a}}^c$.  
If $n$ is odd then ${\mathfrak{a}}^c=\{0\}$.  
Choose the standard coordinates
 $\{f_k: 1 \le k \le [\frac n 2]+1\}$
 on ${\mathfrak{h}}_{\mathbb{C}}^{\ast}$
 such that the root system of ${\mathfrak{g}}$
 and ${\mathfrak{k}}$ are given by 
\begin{align*}
\Delta({\mathfrak{g}}_{\mathbb{C}}, {\mathfrak{h}}_{\mathbb{C}})
=&\{\pm(f_i \pm f_j) : 1 \le i < j \le [\frac n2]+1\}
\\
&\left(\cup \{\pm f_{\ell} : 1 \le \ell \le [\frac n2]+1\}
 \quad(\text{$n$: odd})\right), 
\\
\Delta({\mathfrak{k}}_{\mathbb{C}}, {\mathfrak{t}}_{\mathbb{C}})
=&\{\pm(f_i \pm f_j) : 1 \le i < j \le [\frac {n+1}2] \}
\\
&\left(\cup \{\pm f_{\ell} : 1 \le \ell \le [\frac {n+1}2]\}
 \quad(\text{$n$: even})\right).  
\end{align*}

For $1 \le i \le [\frac {n+1}2]$, 
 we define elements of ${\mathfrak{t}}_{\mathbb{C}}^{\ast}$
 by 
\begin{align*}
\mu_i:=&\sum_{k=1}^i (\frac n 2+1-k) f_k, 
\\
\mu_i^-:=& \mu_i- (n +2-2i) f_i. 
\end{align*}
It is convenient to set 
 $\mu_0=\mu_0^-=0$.  
(We shall use $\mu_i^-$
 only when we consider the identity component group 
 $G_0=SO_0(n+1,1)$
 with $n$ odd and when $n+1=2i$
 for later arguments.)
Let $\langle \, , \, \rangle$
 be the standard bilinear form
 on ${\mathfrak {h}}_{\mathbb{C}}^{\ast} \simeq {\mathbb{C}}^{[\frac n2 ]+1}$. 

\begin{definition}
[$\theta$-stable parabolic subalgebra ${\mathfrak{q}}_i$]
\label{def:qi}
For $0 \le i \le [\frac {n+1} 2]$, 
 we define $\theta$-stable parabolic subalgebras 
\index{A}{qi@${\mathfrak{q}}_i$, $\theta$-stable parabolic subalgebra|textbf}
$
{\mathfrak {q}}_i\equiv {\mathfrak {q}}_i^+
$
$
=({\mathfrak {l}}_i)_{\mathbb{C}} +{\mathfrak {u}}_i
$
and 
$
{\mathfrak {q}}_i^-
=({\mathfrak {l}}_i)_{\mathbb{C}} +{\mathfrak {u}}_i^-
$
 in ${\mathfrak {g}}_{\mathbb{C}}=\operatorname{Lie}(G) \otimes_{\mathbb{R}} {\mathbb{C}}$
by the condition
 that ${\mathfrak {q}}_i$ and ${\mathfrak {q}}_i^-$
 contain the fundamental Cartan subalgebra 
 ${\mathfrak {h}}$ 
 and that their nilradicals ${\mathfrak {u}}_i$ and ${\mathfrak {u}}_i^-$
 are given respectively
 by 
\begin{align*}
\Delta({\mathfrak {u}}_i, {\mathfrak {h}}_{\mathbb{C}})
=&\{\alpha \in \Delta({\mathfrak {g}}_{\mathbb{C}}, {\mathfrak {h}}_{\mathbb{C}}): \langle \alpha, \mu_i \rangle >0\}, 
\\
\Delta({\mathfrak {u}}_i^-, {\mathfrak {h}}_{\mathbb{C}})
=&\{\alpha \in \Delta({\mathfrak {g}}_{\mathbb{C}}, {\mathfrak {h}}_{\mathbb{C}}): \langle \alpha, \mu_i^- \rangle >0\}.  
\end{align*}
Then the Levi subgroup of 
 ${\mathfrak {q}}={\mathfrak {q}}_i$ and ${\mathfrak{q}}_i^-$
 is given by 
\index{A}{LSi@$L_i=SO(2)^i \times O(n-2i+1,1)$, Levi subgroup of ${\mathfrak{q}}_i$|textbf}
\begin{equation}
\label{eqn:Li}
  L_i := N_G({\mathfrak{q}})
      \equiv \{g \in G:{\operatorname{Ad}}(g){\mathfrak{q}}={\mathfrak{q}}\}
      \simeq SO(2)^i \times O(n-2i+1,1).  
\end{equation}
We note that $L_i$ is not in the Harish-Chandra class
 if $n$ is even, 
 as is the case $G=O(n+1,1)$.  
\end{definition}

If we write $\rho({\mathfrak{u}}_i)$ and $\rho({\mathfrak{u}}_i^-)$
 for half the sum of roots in ${\mathfrak{u}}_i$ and ${\mathfrak{u}}_i^-$, 
respectively, 
 then 
\[
  \rho({\mathfrak{u}}_i)=\mu_i
\quad
\text{and}
\quad
  \rho({\mathfrak{u}}_i^-)=\mu_i^-.  
\]
We suppress the superscript $+$ for ${\mathfrak{q}}_i^+$
 except for the case $n+1=2i$.  
For later purpose,
 we compare the following three groups with the same Lie algebras:
\begin{alignat}{3}
\label{eqn:GGG}
 G_0 =&SO_0(n+1,1) &&\hookrightarrow \overline G=SO(n+1,1) &&\hookrightarrow G=O(n+1,1)
\intertext{with maximal compact subgroups}
 K_0 =& SO(n+1)   &&\hookrightarrow \overline K=O(n+1) &&\hookrightarrow K=O(n+1)\times O(1).  
\nonumber
\end{alignat}
\begin{lemma}
\label{lem:conjqi}
\begin{enumerate}
\item[{\rm{(1)}}]
A complete system of the $K_0$-conjugacy classes
 of $\theta$-stable parabolic subalgebras 
 of ${\mathfrak{g}}_{\mathbb{C}}$
 with Levi subgroup $L_i$
 \eqref{eqn:Li}
is given by 
\index{A}{qiapm@${\mathfrak{q}}_i^{\pm}$|textbf}
\begin{alignat*}{2}
  &\{{\mathfrak{q}}_i\}
  &&\text{for $0 \le i < [\frac {n+1}2]$}, 
\\
&\{{\mathfrak{q}}_{\frac {n+1}{2}}^+, {\mathfrak{q}}_{\frac{n+1}2}^-\}
\quad
  &&\text{for $i = \frac {n+1}2$ $($$n$:odd$)$}.  
\end{alignat*}
\item[{\rm{(2)}}]
The $\theta$-stable parabolic subalgebra ${\mathfrak{q}}_i$
 with the property \eqref{eqn:Li} is unique up to conjugation
 by the disconnected group $\overline K$
 (and therefore, also by $K$)
 for all $i$ $(0 \le i \le [\frac {n+1}2])$.  
\end{enumerate}
\end{lemma}



We also make the following two observations:
\begin{lemma}
\label{lem:Litorus}
$L_i$ is compact
 if and only if $n$ is odd and $2i=n+1$.  
In this case, 
 $L_i \simeq SO(2)^{\frac {n+1}2} \times O(1)$.  
\end{lemma}
\begin{lemma}
\label{lem:GoverLi}
The inclusion maps \eqref{eqn:GGG} induce
 the following inclusion and bijection:
\[
   G_0/N_{G_0}({\mathfrak{q}}_i)
   \hookrightarrow
   \overline G/N_{\overline G}({\mathfrak{q}}_i)
   \overset \sim \rightarrow
   G/N_{G}({\mathfrak{q}}_i)
   =
   G/L_i
\]
for all $i$ $(0 \le i \le [\frac{n+1}2])$.  
The first inclusion is bijective
 if $n+1 \ne 2i$.  
\end{lemma}
The second bijection is reflected by the irreducibility of the $G$-module
 $\Pi_{\ell, \delta}$ when restricted to the subgroup 
 $\overline G=SO(n+1,1)$, 
 see Proposition \ref{prop:161648} (1)
 in Appendix II.  

Lemmas \ref{lem:conjqi} and \ref{lem:Litorus} yield the following 
 (well-known) representation theoretic results:
\begin{proposition}
\label{prop:disc}
\begin{enumerate}
\item[{\rm{(1)}}]
$G$ (or $\overline G$, $G_0$) admits a discrete series representation
 if and only if $n$ is odd.  
\item[{\rm{(2)}}]
Suppose $n$ is odd.  
Then there exists only one discrete series representation of $\overline G$
 for each regular integral infinitesimal character;
there exist exactly two discrete series representations of $G$
 (also of $G_0$)
 for each regular integral infinitesimal character.  
\end{enumerate}
\end{proposition}
For $n=2m-1$ in the second statement of Proposition \ref{prop:disc}, 
 we note the following properties
 for the three groups $G \supset \overline G \supset G_0$: 
\begin{enumerate}
\item[$\bullet$]
$L_m \simeq SO(2)^m \times O(1)$
 has two connected components;
\item[$\bullet$]
$L_m \cap \overline G=L_m \cap G_0$ are connected;
\item[$\bullet$]
${\mathfrak{q}}_m^+$ and ${\mathfrak{q}}_m^-$ are not conjugate
 by $G_0$;
 they are conjugate by $\overline G$ or $G$.  
\end{enumerate}
See \cite[Thm.~3 (0)]{KMemoirs92} 
 for results in a more general setting
 of the indefinite orthogonal group $O(p,q)$.  



For $\nu =(\nu_1, \cdots, \nu_i) \in {\mathbb{Z}}^i$, 
 $\mu \in \Lambda^+([\frac n 2]-i+1)$, 
 and $a, b \in \{\pm \}$, 
 we consider an irreducible finite-dimensional $L_i$-module
\[
  \Kirredrep {O(n-2i+1,1)}{\mu}_{a, b} \otimes {\mathbb{C}}_{\nu}
\]
and define an admissible smooth representation of $G$ of moderate growth, 
 to be denoted by 
\[
  \Rq {\nu_1, \cdots, \nu_i}{\mu_1, \cdots, \mu_{[\frac n 2]-i+1}}{a, b}, 
\]
 whose underlying $({\mathfrak{g}},K)$-module
 is given by the cohomological parabolic induction
\index{A}{RzqS@${\mathcal{R}}_{\mathfrak{q}}^S$, cohomological parabolic induction|textbf}
\begin{equation}
\label{eqn:Rqmunu}
  {\mathcal{R}}_{{\mathfrak {q}}_i}^{S_i}(\Kirredrep{O(n-2i+1,1)}{\mu}_{a, b} \otimes {\mathbb{C}}_{\nu+\rho({\mathfrak{u}}_i)})
\end{equation}
 of degree $S_i$,  
where we set
\begin{equation}
\label{eqn:cohSi}
   S_i:= \dim_{\mathbb{C}} ({\mathfrak {u}}_i \cap {\mathfrak {k}}_{\mathbb{C}})       =i(n-i).  
\end{equation}
We note
 that if $i=0$
 then $\Rq {}{\mu_1, \cdots, \mu_{[\frac n 2]+1}}{a, b}$ is
 finite-dimensional.  

\begin{definition}
[$\theta$-stable parameter]
\label{def:thetapara}
We call $\Rq{\nu_1, \cdots, \nu_i}{\mu_1, \cdots, \mu_{[\frac n 2]-i+1}}{a, b}$
 the 
\index{B}{1thetastableparameter@$\theta$-stable parameter|textbf}
{\it{$\theta$-stable parameter}}
 of the representation \eqref{eqn:Rqmunu}.  
\end{definition}



If the $\theta$-stable parameter 
 of a representation $\Pi$ of $G$ is given by 
\[
   \Rq{\nu_1, \cdots, \nu_i}{\mu_1, \cdots, \mu_{[\frac n 2]-i+1}}{a, b}, 
\]
 then that of $\Pi \otimes \chi_{c d}$ for $c, d \in \{\pm\}$ is given by
\begin{equation}
\label{eqn:thetatensor}
\Rq{\nu_1, \cdots, \nu_i}{\mu_1, \cdots, \mu_{[\frac n 2]-i+1}}{ac,bd}.
\end{equation}
The 
\index{B}{infinitesimalcharacter@infinitesimal character}
${\mathfrak{Z}}_G({\mathfrak {g}})$-infinitesimal character
 of $\Rq{\nu_1, \cdots, \nu_i}{\mu_1, \cdots, \mu_{[\frac n 2]-i+1}}{a, b}$
 is given by
\[
  (\nu_1, \cdots, \nu_i, \mu_1, \cdots, \mu_{[\frac n 2]-i+1})
+
  (\frac n 2, \frac n 2-1, \cdots, \frac n 2-[\frac n 2]).  
\]
In particular,
 the $G$-module 
\[
  (\underbrace{0, \cdots,0}_{i} || \underbrace{0, \cdots,0}_{[\frac n 2]-i+1})_{a, b}
\]
has the trivial infinitesimal character $\rho_G$.  
In this case
 we shall write 
\index{A}{Aqlmd@$A_{\mathfrak{q}}(\lambda)$|textbf}
\index{A}{Aqlmdab@$A_{{\mathfrak{q}}_i}$|textbf}
\index{A}{Aqlmdac@$(A_{{\mathfrak{q}}})_{\pm,\pm}$|textbf}
\begin{equation}
\label{eqn:AqRq}
   (A_{\mathfrak{q}_i})_{a, b}
   :={\mathcal{R}}_{\mathfrak{q}_i}^{S_i}(\chi_{ab}\otimes {\mathbb{C}}_{\rho({\mathfrak{u}}_i)})
\end{equation}
 for its underlying $({\mathfrak{g}},K)$-module, 
 see Proposition \ref{prop:161655} below.  



Sometimes we suppress the subscript $+,+$
 and write simply $A_{\mathfrak{q}_i}$
 to denote the $({\mathfrak{g}},K)$-module $(A_{\mathfrak{q}_i})_{+,+}$.  
\begin{remark}
\label{rem:goodrange}
\begin{enumerate}
\item[(1)]
(good range)\enspace
The irreducible finite-dimensional representation 
 $\Kirredrep{O(n-2i+1,1)}{\mu}_{a, b} \otimes {\mathbb{C}}_{\nu+\rho({\mathfrak{u}})}$ 
 of the metaplectic cover of $L_i$ is 
 in the 
\index{B}{goodrange@good range}
 {\it{good range}} 
 with respect to the $\theta$-stable parabolic subalgebra
 ${\mathfrak{q}_i}$
 (see \cite[Def.~0.49]{KV} for the definition)
 if and only if
\[ \nu_1 \ge \nu_2 \ge \cdots \ge \nu_i \ge \mu_1.\]
In this case, 
 the $({\mathfrak{g}},K)$-module \eqref{eqn:Rqmunu}
 is nonzero and irreducible,
 and therefore
 $\Rq{\nu_1, \cdots,\nu_i}{\mu_1, \cdots, \mu_{[\frac n 2]-i+1}}{a, b}$
 is a nonzero irreducible $G$-module.  
For the description
 of the Hasse sequence
 (Theorem \ref{thm:IndV} below), 
 we need only the parameter in the good range.  
\item[(2)]
(weakly fair range)\enspace
If $\mu=(0,\cdots,0)$, 
then the $({\mathfrak{g}},K)$-module \eqref{eqn:Rqmunu} reduces to
\[
  A_{\mathfrak{q}_i}(\nu)_{a, b}
   :={\mathcal{R}}_{\mathfrak{q}_i}^{S_i}
     (\chi_{ab}\otimes {\mathbb{C}}_{\nu+\rho({\mathfrak{u}}_i)})  
\]
cohomologically induced from the one-dimensional representation
 $\chi_{ab}\otimes {\mathbb{C}}_{\nu+\rho({\mathfrak{u}})}$.  
We note that $\chi_{ab}\otimes {\mathbb{C}}_{\nu+\rho({\mathfrak{u}})}$
 is in the 
\index{B}{weaklyfairrange@weakly fair range}
{\it{weakly fair range}} 
 with respect to $\mathfrak{q}_i$
 (see \cite[Def.~0.52]{KV} for the definition)
 if and only if 
\begin{equation}
\label{eqn:wfair}
\nu_{1} + \frac{n}{2} 
\ge 
\nu_{2} + \frac{n}{2} -1
\ge 
\cdots
\ge
\nu_{i} + \frac{n}{2} -i+1
\ge 
0.  
\end{equation}
In this case the $({\mathfrak{g}}, K)$-module
 $A_{\mathfrak{q}_i}(\nu)_{a, b}$
 may or may not vanish.  
See \cite[Thm.~3]{KMemoirs92}
 for the conditions on $\nu \in {\mathbb{Z}}^i$
 in the weakly fair range
 that assure the nonvanishing 
 and the irreducibility
 of $A_{\mathfrak{q}_i}^{S_i}({\mathbb{C}}_{\nu})_{a, b}$.  
We shall see in Section \ref{subsec:Aqsing} 
 that the underlying $({\mathfrak{g}}, K)$-modules
 of singular complementary series representations
 are isomorphic to these modules.  
\end{enumerate}
\end{remark}

\subsubsection{Irreducible representations $\Pi_{\ell, \delta}$ and $(A_{\mathfrak{q}_i})_{\pm,\pm}$}

In this subsection,
 we give a description
 of the underlying $({\mathfrak{g}}, K)$-modules
 of the subquotients $\Pi_{\ell,\delta}$
 of the principal series representation
 of the disconnected group $G=O(n+1,1)$
 in terms of the cohomologically parabolic induced modules
 $(A_{\mathfrak{q}_i})_{\pm,\pm}$.  

We recall from \eqref{eqn:Pild} 
 the definition of the irreducible representations 
\index{A}{1Piidelta@$\Pi_{i,\delta}$, irreducible representations of $G$}
 $\Pi_{\ell,\delta}$
 $(0 \le \ell \le n+1$, $\delta = \pm)$
 of $G=O(n+1,1)$.  
The set 
\[
\{\Pi_{\ell,\delta}: 0 \le \ell \le n+1, \delta=\pm\}
\]
 exhausts irreducible admissible representations
 of moderate growth 
 having ${\mathfrak{Z}}_G({\mathfrak{g}})$-infinitesimal
 character $\rho_G$, 
 see Theorem \ref{thm:LNM20} (2).  
Their underlying $({\mathfrak{g}},K)$-modules
 $(\Pi_{\ell,\delta})_K$
 can be given by cohomologically parabolic induced modules as follows.  
\begin{proposition}
\label{prop:161655}
For $0 \le i \le [\frac {n+1}2]$, 
 let 
\index{A}{qi@${\mathfrak{q}}_i$, $\theta$-stable parabolic subalgebra}
${\mathfrak{q}}_i$ be the $\theta$-stable parabolic subalgebras
 with the Levi subgroup $L_i \simeq SO(2)^i \times O(n-2i+1,1)$
 as in Definition \ref{def:qi}.  
\begin{enumerate}
\item[{\rm{(1)}}]
The underlying $({\mathfrak {g}},K)$-modules
 of the irreducible $G$-modules $\Pi_{\ell, \delta}$
 $(0 \le \ell \le n+1$, $\delta \in \{\pm\})$
 are given by the cohomological parabolic induction
 as follows:
\index{A}{Aqlmd@$A_{\mathfrak{q}}(\lambda)$}
\begin{alignat*}{2}
  (\Pi_{i,+})_K \simeq& (A_{\mathfrak {q}_i})_{+,+} 
&& \supset \Exterior^i({\mathbb{C}}^{n+1}) \boxtimes {\bf{1}}, 
\\
   (\Pi_{i,-})_K \simeq & (A_{\mathfrak {q}_i})_{+,-}
&& \supset \Exterior^i({\mathbb{C}}^{n+1}) \boxtimes {\operatorname{sgn}}, 
\\
   (\Pi_{n+1-i,+})_K \simeq & (A_{\mathfrak {q}_i})_{-,+}
&& \supset \Exterior^{n+1-i}({\mathbb{C}}^{n+1}) \boxtimes {\bf{1}}, 
\\
  (\Pi_{n+1-i,-})_K \simeq & (A_{\mathfrak {q}_i})_{-,-}
&& \supset \Exterior^{n+1-i}({\mathbb{C}}^{n+1}) 
\boxtimes {\operatorname{sgn}}.  
\end{alignat*}
For later purpose, 
 we also indicated their 
\index{B}{minimalKtype@minimal $K$-type}
 minimal $K$-types
 in the right column
 (see Theorem \ref{thm:LNM20} (3)).  
\item[{\rm{(2)}}]
If $n$ is even or if $2i\ne n+1$, 
 then the four $({\mathfrak{g}},K)$-modules
 $(A_{\mathfrak{q}_i})_{a,b}$ $(a,b \in \{\pm\})$ are not isomorphic to each other.  



If $2i=n+1$, 
 then there are isomorphisms
\[
   (A_{\mathfrak {q}_{\frac{n+1}{2}}})_{+,+}
 \simeq 
   (A_{\mathfrak {q}_{\frac{n+1}{2}}})_{-,+}
\quad
\text{and}
\quad 
   (A_{\mathfrak {q}_{\frac{n+1}{2}}})_{+,-}
\simeq
   (A_{\mathfrak {q}_{\frac{n+1}{2}}})_{-,-}
\]
 as $({\mathfrak {g}}, K)$-modules
 for the disconnected group $O(n+1,1)$.  
\end{enumerate}
\end{proposition}
Thus the left-hand sides of the formul\ae\ 
 in Proposition \ref{prop:161655} (1) have overlaps
 when $n$ is odd and $i=\frac{n+1}{2}$.  
In fact, 
the Levi part in this case is of the form 
 $L_{\frac{n+1}{2}} \simeq SO(2)^{\frac{n+1}{2}} \times O(0,1)$, 
 and $\chi_{-+} \simeq {\bf{1}}$
 and $\chi_{+-} \simeq \chi_{--}$
 as $O(0,1)$-modules.  

\subsubsection{Irreducible representations with nonzero $({\mathfrak{g}},K)$-cohomologies}
\label{subsec:gKnonzero}

In this section,
 we prove Theorem \ref{thm:LNM20} (9)
 on the classification 
 of irreducible unitary representations of $G=O(n+1,1)$
 with nonzero $({\mathfrak{g}},K)$-cohomologies.  
We have already seen in Lemma \ref{lem:172145}
 that $H^{\ast}({\mathfrak{g}},K;(\Pi_{\ell,\delta})_K) \ne \{0\}$
 for all $0 \le \ell \le n+1$
 and $\delta \in \{\pm\}$.  
Hence the proof of Theorem \ref{thm:LNM20} (9) will be completed
 by showing the following.  
\begin{proposition}
\label{prop:gKq}
Let $\Pi$ be an irreducible unitary representation
 of $G=O(n+1,1)$
 such that 
 $H^{\ast}({\mathfrak{g}},K;\Pi_K) \ne \{0\}$.  
Then the smooth representation $\Pi^{\infty}$ is isomorphic 
 to 
\index{A}{1Piidelta@$\Pi_{i,\delta}$, irreducible representations of $G$}
 $\Pi_{\ell,\delta}$
 (see \eqref{eqn:Pild})
 for some $0 \le \ell \le n+1$ and $\delta \in \{\pm\}$.  
\end{proposition}

\begin{proof}
We begin with representations of the identity component
 $G_0=SO_0(n+1,1)$.  
In this case,
 we write $A_{\mathfrak{q}}^0$
 by putting superscript 0
 to denote the $({\mathfrak{g}},K_0)$-module
 which is cohomologically induced from 
 the trivial one-dimensional representation
 of a $\theta$-stable parabolic subalgabra ${\mathfrak{q}}$.  



By a theorem of Vogan and Zuckerman \cite{VZ},
 any irreducible unitary representation $\Pi^0$
 of $G_0$ 
 with $H^{\ast}({\mathfrak{g}},K_0;(\Pi^0)_{K_0}) \ne \{0\}$
 is of the form
 $(\Pi^0)_{K_0} \simeq A_{\mathfrak{q}}^0$
 for some $\theta$-stable parabolic subalgebra 
 ${\mathfrak{q}}$ in ${\mathfrak {g}}_{\mathbb{C}}$.  
We recall from Definition \ref{def:qi}
 that ${\mathfrak{q}}_i$
 ($0 \le i < \frac{n+1}{2}$)
 and ${\mathfrak{q}}_i^{\pm}$
 ($i = \frac{n+1}{2}$)
 are $\theta$-stable parabolic subalgebras
 such that the Levi subgroup 
 $N_{G_0}({\mathfrak{q}}_i)$
 (or $N_{G_0}({\mathfrak{q}}_i^{\pm})$)
 are isomorphic to 
 $SO(2)^i \times SO_0(n-2i+1,1)$.  
They exhaust all $\theta$-stable parabolic subalgebras 
 up to inner automorphisms
 and up to cocompact Levi factors,
 namely,
 there exists $0 \le i \le [\frac{n+1}{2}]$
 such that 
\[
 {\mathfrak{q}}_i \subset {\mathfrak{q}}
\quad
\text{and}
\quad
  \text{$N_{G_0}({\mathfrak{q}})/N_{G_0}({\mathfrak{q}}_i)$ is compact}
\]
 if we take a conjugation
 of ${\mathfrak{q}}$
 by an element of $G_0$.  
(For $i = \frac{n+1}{2}$, 
 ${\mathfrak{q}}_i$ is considered as either ${\mathfrak{q}}_i^+$
 or ${\mathfrak{q}}_i^-$.)
Then we have a $({\mathfrak{g}},K_0)$-isomorphism
\begin{equation}
\label{eqn:ZVconn}
   (\Pi^0)_{K_0}
   \simeq
   A_{\mathfrak{q}}^0
   \simeq
\begin{cases}
 A_{\mathfrak{q}_i}^0 \quad 
&\text{if $2i < n+1$,}
\\
  A_{\mathfrak{q}_i^+}^0 \quad \text{or}\quad A_{{\mathfrak{q}}_i^-}^0
  \quad 
& \text{if $2i=n+1$.}
\end{cases}
\end{equation}
Now we consider an irreducible unitary representation $\Pi$
 of the {\it{disconnected}} group $G=O(n+1,1)$
 such that 
 $H^{\ast}({\mathfrak{g}},K;\Pi_K) \ne \{0\}$.  
The assumption implies $H^{\ast}({\mathfrak{g}},K_0;\Pi_K) \ne \{0\}$, 
 and therefore there exists a $G_0$-irreducible submodule
 $\Pi^0$ of the restriction $\Pi|_{G_0}$
 such that 
 $H^{\ast}({\mathfrak{g}},K_0;(\Pi^0)_{K_0}) \ne \{0\}$.  
By the reciprocity,
 the underlying $({\mathfrak{g}},K)$-module $\Pi_K$ must be an irreducible summand
 in the induced representation
\[
   {\operatorname{ind}}_{{\mathfrak{g}},K_0}^{{\mathfrak{g}},K}
   ((\Pi^0)_{K_0}).  
\]
It follows from \eqref{eqn:ZVconn}
 and from Proposition \ref{prop:161655} (2)
 that
\[
{\operatorname{ind}}_{{\mathfrak{g}},K_0}^{{\mathfrak{g}},K}
   ((\Pi^0)_{K_0})
\simeq
\begin{cases}
\underset {a, b \in \{\pm\}}{\bigoplus} (A_{\mathfrak{q}_i})_{a, b}
&\text{if $2i < n+1$,}
\\
(A_{{\mathfrak{q}}_{\frac{n+1}{2}}})_{+,+}
\oplus
(A_{\mathfrak{q}_{\frac{n+1}2}})_{-,-}
\quad 
&\text{if $2i=n+1$.}
\end{cases}
\]
Thus Proposition \ref{prop:gKq} follows from Proposition \ref{prop:161655} (1). \end{proof}




\subsubsection{Description of subquotients 
 in $I_{\delta}(V,\lambda)$}
We use the $\theta$-stable parameter
 for the description
 of irreducible subquotients
 of the principal series representations 
 $I_{\delta}(V,\lambda)$
 of $G=O(n+1,1)$ with regular integral infinitesimal character.  
\begin{theorem}
\label{thm:IndV}
Suppose $V \in \widehat{O(n)}$ and $\lambda \in {\mathbb{Z}} \setminus S(V)$.  
Let $i := i(\lambda, V)$ be the height as in Lemma \ref{lem:ilsig}.  
We write $V=\Kirredrep{O(n)}{\sigma}_{\varepsilon}$
 with $\sigma =(\sigma_1, \cdots, \sigma_{[\frac n2]})\in \Lambda^+([\frac n 2])$ and $\varepsilon \in \{\pm\}$.  
Let $\delta \in \{\pm\}$.  
\begin{enumerate}
\item[{\rm{(1)}}]
Suppose $\lambda \ge \frac n 2$.  
Then $\frac n 2 \le i \le n$.  

If $i \ne \frac n 2$, 
then we have the following nonsplit exact sequence
 of $G$-modules of moderate growth:
\begin{align}
  0 \to& \Rq{\sigma_1-1, \cdots, \sigma_{n-i}-1, \lambda-i}{\sigma_{n-i+1}, \cdots, \sigma_{[\frac n 2]}}{\varepsilon , -\delta \varepsilon}
\notag
\\
\to& I_{\delta}(V, \lambda)
\notag
\\
\to& 
\Rq{\sigma_1-1, \cdots, \sigma_{n-i}-1}{\lambda-i, \sigma_{n-i+1}, \cdots, \sigma_{[\frac n 2]}}{\varepsilon , \delta \varepsilon}
\to 0.  
\label{eqn:161688new}
\end{align}




\item[{\rm{(2)}}]
Suppose $\lambda \le \frac n 2$.   
Then $0 \le i \le \frac n 2$.  

If $i \ne \frac n 2$, 
then we have the following nonsplit exact sequence
 of $G$-modules
 of moderate growth:
\begin{align}
  0 \to& \Rq{\sigma_1-1, \cdots, \sigma_i-1}{i-\lambda, \sigma_{i+1}, \cdots, \sigma_{[\frac n 2]}}{\varepsilon , \delta \varepsilon}
\notag
\\
\to& I_{\delta}(V, \lambda)
\notag
\\
\to&
\Rq{\sigma_1-1, \cdots, \sigma_i-1, i-\lambda}{\sigma_{i+1}, \cdots, \sigma_{[\frac n 2]}}{\varepsilon , -\delta \varepsilon}
\to 0.  
\label{eqn:161616new}
\end{align}

\item[{\rm{(3)}}]
Suppose $i=\frac n 2$, 
 or equivalently, 
 suppose that $n$ is even and $\sigma_{\frac n 2} >|\lambda-\frac n 2|$.  



If $\lambda \ne \frac n 2$, 
then $\lambda \in S_Y(V)$
 (see \eqref{eqn:RSYV}).  
In this case,
 $I_{\delta}(V, \lambda)$ is irreducible
 and we have a $G$-isomorphism:
\[I_{\delta}(V, \lambda)
    \simeq 
\Rq{\sigma_1-1, \cdots, \sigma_{\frac n 2}-1}{|\lambda-\frac n2|}{a, b}
\]
whenever $a,b \in \{\pm\}$ satisfies $a b = \delta$.  



If $\lambda = \frac n 2$,
 then $I_{\delta}(V, \lambda)$ splits
 into the direct sum of two irreducible representations of $G$:
\begin{equation}
\label{eqn:IVYdeco}
I_{\delta}(V, \lambda)
    \simeq 
\bigoplus_{a,b \in \{\pm\}, ab =\delta}
\Rq{\sigma_1-1, \cdots, \sigma_{\frac n 2}-1}{0}{a, b}.  
\end{equation}
\end{enumerate}
\end{theorem}



\begin{remark}
\label{rem:IVlisom}
In Theorem \ref{thm:IndV} (3), 
 we have a $G$-isomorphism 
\[
   I_{\delta}({\Kirredrep{O(n)}{\sigma}}_+, \lambda)
   \simeq
   I_{\delta}({\Kirredrep{O(n)}{\sigma}}_-, \lambda)
\quad
\text{ for each $\delta= \pm$.  }
\]
In fact,
 by Lemma \ref{lem:isig}, 
 $i(\lambda,V)=\frac n 2$ implies 
 that $V$ is of type Y, 
 hence there is an $O(N)$-isomorphism
$\Kirredrep {O(N)} {\sigma}_+ \simeq \Kirredrep {O(N)} {\sigma}_-$
 by Lemma \ref{lem:typeY}.  



Moreover, 
 the restriction of each irreducible summand in \eqref{eqn:IVYdeco}
 to the special orthogonal group $SO(n+1,1)$ is irreducible
 (see Lemma \ref{lem:171523} (1) in Appendix II).  
\end{remark}


\subsubsection{Proof of Theorem \ref{thm:IndV}}

\begin{proof}
[Sketch of the proof of Theorem \ref{thm:IndV}]
If the ${\mathfrak{Z}}_G({\mathfrak{g}})$-infinitesimal character
 of the principal series representation
 $I_{\delta}(\Kirredrep {O(n)}{\sigma}_{\varepsilon}, \lambda)$ 
 is $\rho_G$, 
 then Theorem \ref{thm:IndV} is a reformulation of Theorem \ref{thm:LNM20}
 in terms of $\theta$-stable parameters.  
This is done in Proposition \ref{prop:IndV0} below.  



The general case is derived from the above case by the translation principle, 
 see Theorems \ref{thm:181104} and \ref{thm:20180904}, 
 and also the argument there
 ({\it{e.g.}}, Lemma \ref{lem:Rqtensor}) in Appendix III.  
\end{proof}
Suppose $V=\Exterior^i({\mathbb{C}}^n)$.  
By Example \ref{ex:exteriorpm}, 
the principal series representation 
 $I_{\delta}(i,\lambda)={\operatorname{Ind}}_P^G(\Exterior^i({\mathbb{C}}^n) \otimes \delta \otimes {\mathbb{C}}_{\lambda})$
 is expressed as follows.  
\begin{lemma}
\label{lem:171276}
There are natural $G$-isomorphisms:
\begin{alignat*}{2}
I_{\delta}(\ell,\ell) 
\simeq & 
I_{\delta}(\Kirredrep {O(n)}{1^\ell, 0^{[\frac n 2]-\ell}}_+,\ell)
&& \text{if $\ell \le \frac n 2$}, 
\\
I_{\delta}(\ell,\ell) 
\simeq & 
I_{\delta}(\Kirredrep {O(n)}{1^{n-\ell}, 0^{\ell-[\frac {n+1} 2]}}_-,\ell)
\qquad
&& \text{if $\ell \ge \frac n 2$}.  
\end{alignat*}
\end{lemma}





\begin{proposition}
\label{prop:IndV0}
Suppose $0 \le \ell \le \frac n 2$.  
Then Theorem \ref{thm:IndV} holds for $\lambda=\ell$ 
 and $\sigma = (1^\ell, 0^{[\frac n 2]-\ell}) \in \Lambda^+([\frac n 2])$.  
\end{proposition}
\begin{proof}
By Theorem \ref{thm:LNM20} (1), 
 we have an exact sequence of $G$-modules
\[
   0 \to \Pi_{\ell,\delta} \to I_{\delta}(\ell,\ell) \to \Pi_{\ell+1,-\delta} \to 0, 
\]
which does not split as far as $\ell \ne \frac n 2$.  
By Proposition \ref{prop:161655}, 
 this yields an exact sequence of $({\mathfrak {g}}, K)$-modules:
\[
  0 \to (A_{{\mathfrak {q}}_\ell})_{+,\delta}
    \to I_{\delta}(\ell,\ell)_K
    \to (A_{{\mathfrak {q}}_{\ell+1}})_{+,-\delta}
    \to 0.  
\]
By Lemma \ref{lem:171276} and the definition of $\theta$-stable parameters,
 this exact sequence can be written as
\[
  0 \to 
  \Rq{0^\ell}{0^{[\frac n 2]-\ell+1}}{+,\delta}
  \to 
  I_{\delta}(\Kirredrep{O(n)}{\sigma}_+,\ell)
  \to
  \Rq{0^{\ell+1}}{0^{[\frac n 2]-\ell}}{+,-\delta}
  \to 0.  
\]
Since the height of $(\Kirredrep{O(n)}{\sigma}_+,\ell)=(\Exterior^{\ell}({\mathbb{C}}^n),\ell)$
 is given by $i(\Exterior^{\ell}({\mathbb{C}}^n),\ell)=\ell$.  
$i(\ell,\sigma)=\ell$ by Example \ref{ex:isig}, 
 we get Proposition \ref{prop:IndV0} from
 Lemma \ref{lem:IVchi} and \eqref{eqn:thetatensor}.  
\end{proof}



\subsection{Hasse sequence in terms of $\theta$-stable parameters}
\label{subsec:HasseAq}

This section gives a description of the Hasse sequence 
 (Definition-Theorem \ref{def:UHasse})
 and the standard sequence (Definition \ref{def:Hasse})
 in terms of $\theta$-stable parameters.  



We set $m:=[\frac{n+1}{2}]$, 
 namely $n=2m-1$ or $2m$.  
Let $F$ be an irreducible finite-dimensional representation of $G=O(n+1,1)$,
 and $U_i \equiv U_i(F)$
 ($0 \le i \le [\frac{n+1}{2}]$) be the Hasse sequence with $U_0 \simeq F$.  
We write
\[
   F=\Kirredrep{O(n+1,1)}{s_0, \cdots, s_{[\frac n 2]}}_{a, b}
\]
as in Lemma \ref{lem:161612} (2).  
\begin{theorem}
\label{thm:171471}
Let $n=2m$
 and $0 \le i \le m$.  
\begin{enumerate}
\item[{\rm{(1)}}]
{\rm{(Hasse sequence)}}\enspace
$U_i(F) \simeq \Rq{s_0, \cdots, s_{i-1}}{s_i, \cdots, s_m}
{a, (-1)^{i-s_i} b}$.  
\item[{\rm{(2)}}]
{\rm{(standard sequence)}}\enspace
$U_i(F) \otimes \chi_{+-}^i \simeq 
\Rq{s_0, \cdots, s_{i-1}}{s_i, \cdots, s_m}{a, (-1)^{s_i}b}$. 
\end{enumerate}
\end{theorem}
\begin{proof}
(1)\enspace
We begin with the case $a=b=+$.  
Let $s:=(s_0,\cdots,s_m,0^{m+1}) \in \Lambda^+(2m+2)$.  
As in \eqref{eqn:sin=2m} of Section \ref{subsec:Hasseps}, 
 we define $s^{(\ell)} \in \Lambda^+(2m)$
 for $0 \le \ell \le m$.  
Then by Theorem \ref{thm:171426}, 
 there is an injective $G$-homomorphism
\[
  U_\ell(F) 
  \hookrightarrow 
  I_{(-1)^{\ell-s_{\ell}}}(\Kirredrep {O(n)}{s^{(\ell)}}, \ell-s_\ell).  
\]
The $O(n)$-module $\Kirredrep{O(n)}{s^{(\ell)}}$ is of type I
 (Definition \ref{def:type}), 
 and we have 
\[
  i(\Kirredrep{O(n)}{s^{(\ell)}}, \ell-s_\ell))=\ell
\]
 with the notation of Lemma \ref{lem:ilsig}.  

By Theorem \ref{thm:IndV}, 
 we get the theorem for $a=b=+$ case.  
The general case follows from the case
 $(a,b)=(+,+)$
 by the tensoring argument given in Proposition \ref{prop:HStensor}.  
\par\noindent
(2)\enspace
The second statement follows from Definition \ref{def:Hasse}
 and \eqref{eqn:thetatensor}.  
\end{proof}



The case $n$ odd is given similarly as follows.  
\begin{theorem}
\label{thm:171471b}
Let $n=2m-1$, 
 and $0 \le i \le m-1$.  
\begin{enumerate}
\item[{\rm{(1)}}]
{\rm{(Hasse sequence)}}\enspace
$U_i(F) \simeq \Rq{s_0, \cdots, s_{i-1}}{s_i, \cdots, s_{m-1}}
{a, (-1)^{i-s_i} b}$.  
\item[{\rm{(2)}}]
{\rm{(standard sequence)}}\enspace
$U_i(F) \otimes \chi_{+-}^i \simeq 
\Rq{s_0, \cdots, s_{i-1}}{s_i, \cdots, s_{m-1}}{a, (-1)^{s_i}b}$. 
\end{enumerate}
\end{theorem}
\begin{proof}
\begin{enumerate}
\item[(1)]
We begin with the case $a=b=+$.  
Let $s:=(s_0,\cdots,s_{m-1},0^{m+1}) \in \Lambda^+(2m+1)$.  
As in \eqref{eqn:sin=2m-1}, 
 we define $s^{(\ell)} \in \Lambda^+(2m-1)$
 for $0 \le \ell \le m-1$.  
Then by Theorem \ref{thm:171425}, 
\[
  U_\ell(F) 
  \subset 
  I_{(-1)^{\ell-s_{\ell}}}(\Kirredrep {O(n)}{s^{(\ell)}}, \ell-s_\ell).  
\]
The $O(n)$-module $\Kirredrep{O(n)}{s^{(\ell)}}$ is of type I, 
 and we obtain
\[
  i(\Kirredrep {O(n)}{s^{(\ell)}}, \ell-s_\ell)=\ell
\]
 with the notation of Lemma \ref{lem:ilsig}.  

By Theorem \ref{thm:IndV}, 
 we get the theorem for $a=b=+$ case.  
The general case follows from the case
 $(a,b)=(+,+)$
 by the tensoring argument given in Proposition \ref{prop:HStensor}.  
\item[(2)]
The second statement follows from Definition \ref{def:Hasse}
 and \eqref{eqn:thetatensor}.  
\end{enumerate}
\end{proof}

\subsection{Singular integral case}
\label{subsec:Aqsing}
\index{B}{singularintegralinfinitesimalcharacter@singular integral infinitesimal character}
We end this chapter 
 with cohomologically induced representations
 with {\it{singular}} parameter,
 and give a description 
 of complementary series representations
 with integral parameter
 (see Section \ref{subsec:singcomp})
 in terms of $\theta$-stable parameters.  



For $0 \le r \le [\frac{n+1}{2}]$,
 we define ${\mathfrak {q}}_r = ({\mathfrak {l}}_r)_{\mathbb{C}}
 +{\mathfrak {u}}_r$
 to be the $\theta$-stable parabolic subalgebra
 with Levi subgroups
$
  L_r \simeq SO(2)^r \times O(n+1-2r, 1)
$
 in $G=O(n+1,1)$ as in Definition \ref{def:qi}.  
We set $S_r = r(n-r)$.  


For $\nu=(\nu_1, \cdots, \nu_r) \in {\mathbb{Z}}^r \simeq (SO(2)^r)\hspace{1mm}
{\widehat{}}\hphantom{ii}$
 and $a, b \in \{\pm\}$, 
 we consider the underlying $({\mathfrak{g}},K)$-modules
 of the admissible smooth representations of $G$:
\[
  \Rq {\nu_1, \cdots, \nu_r}{\underbrace{0,\cdots,0}_{[\frac n 2]-r+1}}{a, b}, 
\]
 namely, 
the following $({\mathfrak{g}},K)$-modules
\[
  A_{\mathfrak{q}_r}(\nu)_{a, b}
   ={\mathcal{R}}_{\mathfrak{q}_r}^{S_r}
     (\chi_{ab} \otimes {\mathbb{C}}_{\nu+\rho({\mathfrak{u}})})
   \simeq 
   {\mathcal{R}}_{\mathfrak{q}_r}^{S_r}
   ({\mathbb{C}}_{\nu+\rho({\mathfrak{u}}_r)}) \otimes \chi_{ab}, 
\]
which are cohomologically induced from the one-dimensional representations
 ${\mathbb{C}}_{\nu} \boxtimes \chi_{a b}$
 of the Levi subgroup $L_r$, 
 see Remark \ref{rem:goodrange}
 for our normalization about \lq\lq{$\rho$-shift}\rq\rq.  



Sometimes we suppress the subscript $+,+$
 and write simply $A_{{\mathfrak{q}}_{r}}(\nu)$
 for $A_{{\mathfrak{q}}_{r}}(\nu)_{+,+}$.  



For a description of singular integral complementary series representations
 $I_{\delta}(i,s)$
 in terms of $\theta$-stable parameters,
 we need to treat the parameter $\nu$ outside the good range 
 (\cite[Def.~0.49]{KV})
 relative to the $\theta$-stable parabolic subalgebra ${\mathfrak {q}}_r$
 with $r = i+1$
 (see Theorem \ref{thm:compint} below), 
 for which the general theory 
 about the nonvanishing and irreducibility
 ({\it{e.g}}. \cite[Thm.~0.50]{KV})
 does not apply.  
For instance, 
the condition on the parameter $\nu$
 for which $A_{\mathfrak{q}_r}(\nu) \ne 0$
 is usually very complicated 
 when $\nu$ wanders outside the
\index{B}{goodrange@good range}
 good range.  
In our setting, 
 we use the following results from \cite{KMemoirs92}:
\begin{fact}
\label{fact:Memo92}
Let $0 \le r \le [\frac{n+1}{2}]$, 
 and $\mathfrak{q}_r$ be the $\theta$-stable parabolic subalgebra
 as defined in Definition \ref{def:qi}.  
Suppose that $\nu=(\nu_1, \cdots, \nu_r) \in {\mathbb{Z}}^r$
 satisfies the weakly fair condition \eqref{eqn:wfair}
 relative to ${\mathfrak{q}}_r$.  
Let $a, b \in \{\pm\}$.  
\begin{enumerate}
\item[{\rm{(1)}}]
The $G$-module $\Rq{\nu_1, \cdots, \nu_r}{0,\cdots,0}{a, b}$
 is nonzero
 if and only if
 $r=1$ or $\nu_{r-1} \ge -1$.  
\item[{\rm{(2)}}]
If the condition (1) is fulfilled, 
then $\Rq{\nu_1, \cdots, \nu_r}{0,\cdots,0}{a, b}$ is irreducible 
 and unitarizable.  
\end{enumerate}
\end{fact}
\begin{proof}
This is a special case
 of \cite[Thm.~3]{KMemoirs92}
 for the indefinite orthogonal group $O(p,q)$
 with $(p,q)=(n+1,1)$
 with the notation there.  
\end{proof}


Assume now $\nu_1 = \cdots = \nu_{r-1}=0$.  
Then the necessary and sufficient condition 
 for the parameter
 $\nu=(0,\cdots,0,\nu_r) \in {\mathbb{Z}}^r$ 
 to be in the weakly fair range 
 but outside the good range
 is given by 
\[
   \nu_r \in \{-1,-2, \cdots, r-1-[\frac n 2]\}.  
\]
In this case, 
 the $G$-module $\Rq{0,\cdots,0,\nu_r}{0,\cdots,0}{a, b}$ is nonzero,
 irreducible, 
 and unitarizable for $a,b \in \{\pm\}$
 as is seen in Fact \ref{fact:Memo92}. 
It turns out that these very parameters give rise to the complementary series representations
 with integral parameter stated in Section \ref{subsec:singcomp} as follows:
\begin{theorem}
\label{thm:compint}
Let $0 \le i \le [\frac n 2]-1$.  
Then the underlying $({\mathfrak{g}},K)$-modules
 of the
\index{B}{complementaryseries@complementary series representation}
 complementary series representations $I_\pm(i,s)$ and $I_\pm(n-i,s)$
 with integral parameter
 $s \in \{i+1,i+2, \cdots, [\frac n 2]\}$ are given by
\begin{align*}
  I_+(i,s)_K
  &\simeq
  A_{{\mathfrak{q}}_{i+1}}(0,\cdots,0,s-i)_{+,+};
\\
  I_-(i,s)_K
  &\simeq
  A_{{\mathfrak{q}}_{i+1}}(0,\cdots,0,s-i)_{+,-};
\\
  I_+(n-i,s)_K
  &\simeq
  A_{{\mathfrak{q}}_{i+1}}(0,\cdots,0,s-i)_{-,-};
\\
  I_-(n-i,s)_K
  &\simeq
  A_{{\mathfrak{q}}_{i+1}}(0,\cdots,0,s-i)_{-,+}.  
\end{align*}
Hence,
 their smooth globalizations are described
 by $\theta$-stable parameters as follows:
\begin{alignat*}{2}
   &I_+(i,s) 
   &&\simeq \Rq{\underbrace{0,\cdots,0,s-i}_{i+1}}{\underbrace{0,\cdots,0}_{[\frac n 2]-i}}{+,+};
\\
&I_-(i,s) 
   &&\simeq \Rq{0,\cdots,0,s-i}{0,\cdots,0}{+,-};
\\
&I_+(n-i,s) 
   &&\simeq \Rq{0,\cdots,0,s-i}{0,\cdots,0}{-,-};
\\
&I_-(n-i,s) 
   &&\simeq \Rq{0,\cdots,0,s-i}{0,\cdots,0}{-,+}.  
\end{alignat*} 
\end{theorem}

