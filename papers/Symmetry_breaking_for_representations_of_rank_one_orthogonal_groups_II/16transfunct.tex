\newpage
\section{Appendix III: A translation functor for $G=O(n+1,1)$}
\label{sec:Translation}

In this chapter,
 we discuss a translation functor 
 for the group $G=O(n+1,1)$, 
 which is not in the Harish-Chandra class
 if $n$ is even, 
 in the sense that $\operatorname{Ad}(G)$ is not contained 
 in the group $\operatorname{Int}({\mathfrak{g}}_{\mathbb{C}})$
 of inner automorphisms.  
Then the \lq\lq{Weyl group}\rq\rq\
 $W_G$ is larger
 than the group generated 
 by the reflections of simple roots.  
This causes some technical difficulties
 when we extend the idea of translation functor
 which is usually formulated 
 for reductive groups
 in the Harish-Chandra class or reductive Lie algebras, 
 see \cite{Jantzen, SpehVogan, Vogan81, Zuckerman} for instance.  

\subsection{Some features of translation functors
 for reductive groups
 that are not of Harish-Chandra class}
For $n$ even, 
say $n=2m$, 
 we write ${\mathfrak{h}}_{\mathbb{C}}$
 $(\simeq {\mathbb{C}}^{m+1})$
 for a Cartan subalgebra of ${\mathfrak{g}}_{\mathbb{C}}$.  
Then we recall from Section \ref{subsec:2.1.4}:
\begin{enumerate}
\item[$\bullet$]
\index{A}{Weylgroupg@$W_{\mathfrak{g}}$, Weyl group for ${\mathfrak{g}}_{\mathbb{C}}={\mathfrak{o}}(n+2,{\mathbb{C}})$}
the Weyl group $W_{\mathfrak{g}} \simeq {\mathfrak{S}}_{m+1} \ltimes ({\mathbb{Z}}/2{\mathbb{Z}})^{m}$ for the root system
 $\Delta({\mathfrak{g}}_{\mathbb{C}}, {\mathfrak{h}}_{\mathbb{C}})$
 is of index two in the Weyl group 
\index{A}{WeylgroupG@$W_G$, Weyl group for $G=O(n+1,1)$}
$W_G\simeq {\mathfrak{S}}_{m+1} \ltimes ({\mathbb{Z}}/2{\mathbb{Z}})^{m+1}$
 for the disconnected group $G$;
\item[$\bullet$]
the ${\mathfrak{Z}}_G({\mathfrak{g}})$-infinitesimal character
 for the irreducible admissible representation of $G$
 is parametrized by ${\mathfrak{h}}_{\mathbb{C}}^{\ast}/W_G$,
 but not by ${\mathfrak{h}}_{\mathbb{C}}^{\ast}/W_{\mathfrak{g}}$;
\item[$\bullet$]
$\rho_G=(m,\cdots,1,0)$ is not 
 \lq\lq{$W_G$-regular}\rq\rq, 
 although it is \lq\lq{$W_{\mathfrak{g}}$-regular}\rq\rq\
 (Definition \ref{def:intreg}).  
\end{enumerate}

We can still use the idea of a translation functor,
 but we need a careful treatment
 for disconnected groups $G$
 which are not in the Harish-Chandra class.  
In fact,
 differently from the usual setting for reductive Lie groups
 in the Harish-Chandra class, 
 we are faced with the following feature:
\begin{enumerate}
\item[$\bullet$]
translation from a
\index{B}{WGregular@$W_G$-regular}
 $W_G$-regular
 (in particular, 
\index{B}{Wgregular@$W_{\mathfrak{g}}$-regular}
$W_{\mathfrak{g}}$-regular) dominant parameter
 to the trivial infinitesimal character $\rho_G$ 
 does not necessarily preserve irreducibility, 
 see Theorem \ref{thm:1808101}. 
\end{enumerate}
This means that translation inside the same \lq\lq{$W_{\mathfrak{g}}$-regular Weyl chamber}\rq\rq\
 may involve a phenomenon 
 as if it were \lq\lq{translation from the wall 
 to regular parameter}\rq\rq, 
 {\it{cf.}},  \cite{SpehVogan}.  


In what follows,
 we retain the terminology
 \lq\lq{regular}\rq\rq\ for $W_{\mathfrak{g}}$
 but not for $W_G$
 as in Definition \ref{def:intreg}
 (in particular, $\rho_G$ is regular in our sense), 
 whereas we need to use $W_G$
 (not $W_{\mathfrak{g}}$)
 in describing ${\mathfrak{Z}}_G({\mathfrak{g}})$-infinitesimal characters
 of $G$-modules.  



\subsection{Translation functor for $G=O(n+1,1)$}

In this section 
we fix some notation
 for a 
\index{B}{translationfunctor@translation functor|textbf}
{\it{translation functor}}
 for the group $G=O(n+1,1)$.  
Usually, 
 a translation functor is defined in the category 
 of $({\mathfrak{g}},K)$-modules
 of finite length.  
However, 
 we also consider a translation functor
  in the category of {\bf{admissible representations
 of finite length of moderate growth}}.  



\subsubsection{Primary decomposition of admissible smooth representations}
\label{subsec:primary}

Let $\Pi$ be an admissible smooth representation of $G$ of finite length.  
For $\mu \in {\mathfrak{h}}_{\mathbb{C}}^{\ast}/W_G$, 
 we define the $\mu$-primary component $P_{\mu}(\Pi)$ of $\Pi$ by 
\[
    P_{\mu}(\Pi) := \bigcup_{N >0} \bigcap_{z \in {\mathfrak{Z}}_G({\mathfrak{g}})}
   \operatorname{Ker}(z-\chi_{\mu}(z))^N, 
\]
where we recall the Harish-Chandra isomorphism \eqref{eqn:HCpara}
\index{A}{ZGg@${\mathfrak{Z}}_G({\mathfrak{g}})$}
\[
   \operatorname{Hom}_{\mathbb{C}\text{-alg}}({\mathfrak{Z}}_G({\mathfrak{g}}), {\mathbb{C}})
   \simeq {\mathfrak{h}}_{\mathbb{C}}^{\ast}/W_G, 
\quad
   \chi_{\mu} \leftrightarrow \mu.  
\]
Then $P_{\mu}(\Pi)$ is a $G$-module with generalized ${\mathfrak{Z}}_G({\mathfrak{g}})$-infinitesimal character $\mu$, 
 and $\Pi$ is decomposed into a direct sum 
 of finitely many primary components:
\[
  \Pi = \bigoplus_{\mu}P_{\mu}(\Pi)
\qquad
\text{(finite direct sum)}.  
\]
By abuse of notation,
 we use the letter 
\index{A}{Pmu@$P_{\mu}$|textbf}
$P_{\mu}$ 
 to denote the $G$-equivariant projection $\Pi \to P_{\mu}(\Pi)$
 with respect to the direct sum decomposition.  



\subsubsection{Translation functor $\psi_{\mu}^{\mu+\tau}$
 for $G=O(n+1,1)$}
Let $G=O(n+1,1)$ and $m=[\frac n 2]$.  
We recall that $W_G \simeq {\mathfrak{S}}_{m+1} \ltimes ({\mathbb{Z}}/2{\mathbb{Z}})^{m+1}$
 acts on ${\mathfrak{h}}_{\mathbb{C}}^{\ast} \simeq {\mathbb{C}}^{m+1}$
 as a permutation group 
 and by switching the signatures of the standard coordinates.  
For $\tau \in {\mathbb{Z}}^{m+1}$, 
 we define 
\index{A}{1pataudom@$\tau_{\operatorname{dom}}$|textbf}
$\tau_{\operatorname{dom}}$ to be the unique element
 in $\Lambda^+(m+1)$
 (see \eqref{eqn:Lambda})
 in the $W_G$-orbit through $\tau$, 
{\it{i.e.}}, 
\begin{equation}
\label{eqn:Wdom}
  \tau_{\operatorname{dom}}= w \, \tau
  \qquad
  \text{for some $w \in W_G$.  }
\end{equation}
Let $\Kirredrep {O(n+1,1)}{\tau_{\operatorname{dom}}}_{+,+}$
 be the irreducible finite-dimensional representation of 
 $G=O(n+1,1)$ of type I 
 (Definition \ref{def:typeone})
 defined as in \eqref{eqn:ONCreal}.  



\begin{definition}
[translation functor $\psi_{\mu}^{\mu+\tau}$]
\label{def:transG}
For $\mu \in {\mathbb{C}}^{m+1}$ and $\tau \in {\mathbb{Z}}^{m+1}$, 
 we define translation functor $\psi_{\mu}^{\mu+\tau}$ by 
\index{A}{1psinlmutau@$\psi_{\mu}^{\mu+\tau}$|textbf}
\begin{equation}
\label{eqn:translation}
   \psi_{\mu}^{\mu+\tau}(\Pi)
   :=
   P_{\mu+\tau}(P_{\mu}(\Pi) \otimes \Kirredrep{O(n+1,1)}{\tau_{\operatorname{dom}}}_{+,+}).  
\end{equation}
\end{definition}
Then $\psi_{\mu}^{\mu+\tau}$ is a covariant functor
 in the category of admissible smooth representations of $G$
 of finite length, 
 and also  
 in the category
 of $({\mathfrak{g}},K)$-modules of finite length.  
Clearly,
 we have 
\begin{equation}
\label{eqn:V731}
  \psi_{\mu}^{\mu+\tau}=\psi_{w\mu}^{w\mu+w\tau}
\qquad
\text{for all $w \in W_G$.}
\end{equation}


In defining the translation functor $\psi_{\mu}^{\mu+\tau}$
 in \eqref{eqn:translation}, 
 we have used only finite-dimensional representations
\index{B}{type1indef@type I, representation of $O(N-1,1)$}
 of type I
 (Definition \ref{def:typeone})
 of the disconnected group $G=O(n+1,1)$.  
We do not lose any generality
 because taking the tensor product
 with the one-dimensional characters
 $\chi_{a b}$
 ($a, b \in \{\pm\}$)
 yields the following isomorphism as $G$-modules:
\begin{equation}
\label{eqn:chipsi}
   \psi_{\mu}^{\mu+\tau}(\Pi) \otimes \chi_{a b}
   \simeq 
   P_{\mu+\tau}
   (P_{\mu}(\Pi) \otimes 
    \Kirredrep{O(n+1,1)}{\tau_{\operatorname{dom}}}_{a,b}).  
\end{equation}
We shall use a finite-dimensional representation
 $F(V,\lambda)$
 (Definition \ref{def:Fshift})
 which is not necessarily of type I
 in Theorems \ref{thm:181104} and \ref{thm:181107}, 
 which are a reformulation of the properties
 (Theorems \ref{thm:1807113} and \ref{thm:1808101}, respectively)
 of the translation functor
 \eqref{eqn:translation}
 via \eqref{eqn:chipsi}.  



The translation functor $\psi_{\mu+\tau}^{\mu}$ is the adjoint functor
 of $\psi_{\mu}^{\mu+\tau}$.  
In our setting, 
 since $(-\tau)_{\operatorname{dom}}=\tau_{\operatorname{dom}}$, 
 the functor $\psi_{\mu+\tau}^{\mu}$
 takes the following form:
\[
  \psi_{\mu+\tau}^{\mu}(\Pi)
  =
  P_{\mu}
  (P_{\mu+\tau}(\Pi) \otimes 
   \Kirredrep{O(n+1,1)}{\tau_{\operatorname{dom}}}_{+,+}).  
\]



\subsubsection{The translation functor and the restriction $G \downarrow \overline G$}
We retain the notation of Appendix II, 
 and denote by $\overline G$ the subgroup 
 $SO(n+1,1)$ in $G=O(n+1,1)$.  
Then $\overline G=SO(n+1,1)$ is in the Harish-Chandra class
 for all $n$.  
For the group $\overline G$,
 we shall use the notation
 $\overline P_{\mu}$ and $\overline{\psi}_{\mu}^{\mu+\tau}$ 
instead of $P_{\mu}$ and $\psi_{\mu}^{\mu+\tau}$, 
 respectively.   
To be precise,
 for $\tau \in {\mathbb{Z}}^{m+1}$ 
 where $m=[\frac n 2]$, 
 we write  $\overline \tau_{\operatorname{dom}}$ for the unique element
 in the orbit $W_{\mathfrak{g}}\, \tau$
 which is dominant with respect to the positive system
 $\Delta^+({\mathfrak{g}}_{\mathbb{C}}, {\mathfrak{h}}_{\mathbb{C}})$.  
We denote by $\Kirredrep{SO(n+1,1)}{\overline \tau_{\operatorname{dom}}}_+$
 the irreducible representation of $\overline G=SO(n+1,1)$
 obtained by the restriction
 of the irreducible holomorphic representation of $SO(n+2,{\mathbb{C}})$
 having $\overline \tau_{\operatorname{dom}}$
 as its highest weight.  
For an admissible smooth representation $\overline \Pi$
 of $\overline G$ of finite length, 
 the translation functor $\overline \psi_{\mu}^{\mu+\tau}$
 is defined by
\begin{equation}
\label{eqn:transSO}
\overline \psi_{\mu}^{\mu+\tau}(\overline \Pi)
:=
\overline P_{\mu+\tau}
(\overline P_{\mu}(\overline \Pi) 
\otimes 
\Kirredrep{SO(n+1,1)}{\overline \tau_{\operatorname{dom}}}_+).  
\end{equation}
We collect some basic facts concerning the primary components
 for $G$-modules and $\overline G$-modules. 
The following lemma is readily shown by comparing \eqref{eqn:HCpara}
 of the Harish-Chandra isomorphisms
 for $G$ and $\overline G$.  
\begin{lemma}
\label{lem:primary}
Let $\Pi$ be an admissible smooth representation
of finite length of $G=O(n+1,1)$.  
We set $m:=[\frac n 2]$ as before.  
Suppose $\mu \in {\mathfrak{h}}_{\mathbb{C}}^{\ast} 
\simeq {\mathbb{C}}^{m+1}$.  
\begin{enumerate}
\item[{\rm{(1)}}]
If $n$ is odd or if $n$ is even
 and at least one of the entries $\mu_1$, $\cdots$, $\mu_{m+1}$
 is zero, 
then there is a natural isomorphism of $\overline G$-modules:
\[
   P_{\mu}(\Pi)|_{\overline G} \simeq \overline P_{\mu}(\Pi|_{\overline G}).  
\]
\item[{\rm{(2)}}]
If $n$ is even
 and all of $\mu_j$ are nonzero, 
 then we have a direct sum decomposition of a $\overline G$-module:
\[
   P_{\mu}(\Pi)|_{\overline G}=\overline P_{\mu}(\Pi|_{\overline G}) \oplus 
\overline P_{\mu'}(\Pi|_{\overline G}), 
\]
where we set $\mu':=(\mu_1, \cdots, \mu_m, -\mu_{m+1})$.  
\end{enumerate}
\end{lemma}

Now the following lemma is an immediate consequence
 of Lemma \ref{lem:primary}
 and of the definition of the translation functors
 $\psi_{\mu}^{\mu+\tau}$ and $\overline \psi_{\mu}^{\mu+\tau}$, 
 see \eqref{eqn:translation} and \eqref{eqn:transSO}. 
\begin{lemma}
\label{lem:Xtrans}
Let $G=O(n+1,1)$ and $\overline G=SO(n+1,1)$.  
Let $\Pi$ be an admissible smooth representation of $G$
 of finite length.  
\begin{enumerate}
\item[{\rm{(1)}}]
Suppose $n$ is odd.  
Then we have a canonical $\overline G$-isomorphism:
\begin{equation}
\label{eqn:restrans}
\psi_{\mu}^{\mu+\tau}(\Pi)|_{\overline G}
\simeq
\overline \psi_{\mu}^{\mu+\tau}(\Pi|_{\overline G}).  
\end{equation}
\item[{\rm{(2)}}]
Suppose that $n$ is even.  
If all of $\mu$, $\tau$ and $\mu+\tau$ contain 0 in their entries,
 then we have a canonical $\overline G$-isomorphism
 \eqref{eqn:restrans}.  
\end{enumerate}
\end{lemma}

\subsubsection{Some elementary properties
 of translation functor $\psi_{\mu}^{\mu+\tau}$}
Some of the properties of the translation functors remain true
 for the disconnected group $G=O(n+1,1)$.  
\begin{proposition}
\label{prop:transfunct}
Suppose $\mu \in {\mathfrak{h}}_{\mathbb{C}}^{\ast} (\simeq {\mathbb{C}}^{m+1})$
 and $\tau \in {\mathbb{Z}}^{m+1}$.  
\begin{enumerate}
\item[{\rm{(1)}}]
$\psi_{\mu}^{\mu+\tau}$ is a covariant exact functor.  
\item[{\rm{(2)}}]
Suppose $\mu$ and $\mu+\tau$ belong to the same Weyl chamber
 with respect to $W_{\mathfrak{g}}$.  
If $\mu+\tau$ is regular 
 (Definition \ref{def:intreg}), 
 then $\psi_{\mu}^{\mu+\tau}(\Pi)$ is nonzero 
 if $\Pi$ is nonzero.  
\end{enumerate}
\end{proposition}

\begin{proof}
(1)\enspace
The first statement follows directly from  the definition,
 see Zuckerman \cite{Zuckerman}.  
\par\noindent
(2)\enspace
By Lemma \ref{lem:primary} and the branching law from $G=O(n+1,1)$
 to the subgroup $\overline G=SO(n+1,1)$, 
 we have
\[
 \psi_{\mu}^{\mu+\tau}(\Pi)|_{\overline G}
  \supset 
  \overline \psi_{\mu}^{\mu+\tau}(\Pi|_{\overline G}).  
\]
Since ${\overline G}$ is in the Harish-Chandra class,
 $\overline \psi_{\mu}^{\mu+\tau}(\Pi|_{\overline G})$
 is nonzero under the assumption
 on $\mu$ and $\tau$.  
Hence $\psi_{\mu}^{\mu+\tau}(\Pi)$ is a nonzero $G$-module.  
\end{proof}

\begin{remark}
The regularity assumption 
 for $\mu + \tau$
 in Proposition \ref{prop:transfunct}
 is in the weaker sense
 ({\it{i.e.}}, $W_{\mathfrak{g}}$-regular), 
 and not in the stronger sense
 ({\it{i.e.}}, $W_G$-regular). 
\end{remark}

\subsection{Translation of principal series representation $I_{\delta}(V,\lambda)$}

We discuss how the translation functors affect 
 induced representations
 of $G=O(n+1,1)$.  
We recall that $G$ is not in the Harish-Chandra class when $n$ is even.  

\subsubsection{Main results: Translation of principal series representations}
\begin{theorem}
\label{thm:1807113}
Suppose $G=O(n+1,1)$
 and $(V,\lambda) \in \Reducible$, 
 see \eqref{eqn:reducible}, 
 or equivalently,
 $V \in \widehat {O(n)}$
 and $\lambda \in {\mathbb{Z}} \setminus (S(V) \cup S_Y(V))$, 
 see Theorem \ref{thm:irrIV}.  
Let $i:=i(V,\lambda) \in \{0,1,\ldots,n\}$ be the height of $(V,\lambda)$
 as in \eqref{eqn:indexV}, 
 and $r(V, \lambda)\in {\mathbb{Z}}^{m+1}$ as in \eqref{eqn:IVZG}.  
We write $V=\Kirredrep{O(n)}{\sigma}_{\varepsilon}$
 with $\sigma \in \Lambda^+(m)$ and $\varepsilon \in \{\pm\}$, 
 where $m:=[\frac n 2]$.  
We define a character $\chi$ of $G$ by
\index{A}{1chipmpmVlmd@$\chi(V,\lambda)$|textbf}
\begin{equation}
\label{eqn:sgnchi}
\chi \equiv \chi(V,\lambda):=
\begin{cases}
{\bf{1}}
&\text{if $\varepsilon(\frac n 2-i) \ge 0$,}
\\
\det
\qquad
&\text{if $\varepsilon(\frac n 2-i) < 0$.  }
\end{cases}
\end{equation}
Then there is a natural $G$-isomorphism:
\[
  \psi_{\rho^{(i)}}^{r(V,\lambda)}(I_{\delta}(i,i)) \otimes \chi
  \simeq 
  I_{(-1)^{\lambda-i}\delta}(V,\lambda).  
\]
\end{theorem}

\begin{remark}
\label{rem:FVchipm}
The conclusion of Theorem \ref{thm:1807113} does not change
 if we replace the definition \eqref{eqn:sgnchi}
 with $\chi=\det$
 when $i=\frac n 2$.  
In fact, 
$V$ is of type Y
 if the height $i(V,\lambda)$ equals $\frac n 2$, 
and thus $V \otimes \det \simeq V$
 as $O(n)$-modules
 (Lemma \ref{lem:typeY}).  
Then there is an isomorphism of $G$-modules
\[
   I_{\delta}(V,\lambda) \otimes \det \simeq I_{\delta}(V,\lambda)
\]
for any $\delta \in \{\pm\}$
 by Lemmas \ref{lem:IVchi} and \ref{lem:isig}.  
\end{remark}

The translation functor 
 $\psi_{r(V,\lambda)}^{\rho^{(i)}}$
 is the adjoint functor
 of $\psi_{\rho^{(i)}}^{r(V,\lambda)}$.  
Even when the infinitesimal character
 of $I_{\delta}(V,\lambda)$
 is $W_G$-regular
 (in particular,
 $W_{\mathfrak{g}}$-regular)
 (Definition \ref{def:intreg}), 
 the translation functor $\psi_{r(V,\lambda)}^{\rho^{(i)}}$
 does not always preserve {\it{irreducibility}}
 if $G$ is not of Harish-Chandra class as in the following theorem.  

\begin{theorem}
\label{thm:1808101}
Retain the setting and notation of Theorem \ref{thm:1807113}.  
In particular,
 we recall that 
\index{A}{RzqSeducible@$\Reducible$ $(\subset \widehat {O(n)} \times {\mathbb{Z}})$}
$(V,\lambda)\in \Reducible$, 
\index{A}{iVlmd@$i(V,\lambda)$, height}
 $i=i(V,\lambda)$ is the height of $(V,\lambda)$, 
 and $\chi\equiv \chi(V,\lambda)$, 
 see \eqref{eqn:sgnchi}.  
\begin{enumerate}
\item[{\rm{(1)}}]
If 
\index{A}{RzqSeducibleI@$\RedI$}
$(V,\lambda)\in \RedI$
 (Definition \ref{def:Red12}), 
 {\it{i.e.}}, 
 if $V$ is of type X
 (in particular, 
 if $n$ is odd)
 or if $\lambda=\frac n 2$, 
 then there is a natural $G$-isomorphism:
\[
  \psi_{r(V,\lambda)}^{\rho^{(i)}}(I_{(-1)^{\lambda-i}\delta}(V,\lambda))
  \otimes \chi
  \simeq 
  I_{\delta}(i,i).  
\]
\item[{\rm{(2)}}]
If 
\index{A}{RzqSeducibleII@$\RedJ$}
$(V,\lambda)\in \RedJ$, 
 {\it{i.e.}}, 
 if $V$ is of type Y 
 and $\lambda \ne \frac n 2$, 
 then $n$ is even, 
 $i \ne \frac n 2$
 and there is a natural $G$-isomorphism:
\[
  \psi_{r(V,\lambda)}^{\rho^{(i)}}(I_{(-1)^{\lambda-i}\delta}(V,\lambda))
  \otimes \chi
  \simeq 
  I_{\delta}(i,i) \oplus I_{\delta}(n-i,i).  
\]
\end{enumerate}
\end{theorem}
In Section \ref{sec:FVlmd}, 
 we introduce an irreducible finite-dimensional representation
 $F(V,\lambda)$
 by taking the tensor product
 of $\Kirredrep{O(n+1,1)}{\tau_{\operatorname{dom}}}_{+,+}$
 with an appropriate one-dimensional character of $G$, 
 see Definition \ref{def:Fshift}.  
Then, 
 by using $F(V,\lambda)$, 
 Theorems \ref{thm:1807113} and \ref{thm:1808101}
 can be reformulated in a simpler form about signatures, 
 see Theorems \ref{thm:181104} and \ref{thm:181107}.  



\subsubsection{Strategy of the proof
 for Theorems \ref{thm:1807113} and \ref{thm:1808101}}
\label{subsec:sketchtrans}
If $n$ is odd,
 then $G=\langle SO(n+1,1), -I_{2n+2} \rangle$ is in the Harish-Chandra class, 
 and therefore Theorems \ref{thm:1807113} and \ref{thm:1808101}
 are a special case of the general theory, 
 see \cite[Chap.~7]{Vogan81} for instance.  
Moreover,
 the translation functor
 behaves
 as we expect from the general theory 
 for reductive groups in the Harish-Chandra class
 when it is applied to the induced representation
 $I_{\delta}(V,\lambda)$
 if $(V,\lambda) \in \RedI$, 
 see Theorem \ref{thm:1808101} (1).  
We note that $\Reducible=\RedI$ and $\RedJ=\emptyset$
 if $n$ is odd
 (Remark \ref{rem:Red12}).  



On the other hand,
 its behavior is somewhat different
 if $(V,\lambda) \in \RedJ$, 
 see Theorem \ref{thm:1808101} (2)
 and Proposition \ref{prop:transrest2}
 for instance.  
Main technical complications arise from 
 the fact that we need the primary decomposition 
 for the generalized ${\mathfrak{Z}}_G({\mathfrak{g}})$-infinitesimal characters
 parametrized by ${\mathfrak{h}}_{\mathbb{C}}^{\ast}/W_G$
 where $W_G$ is larger than the group 
 generated by the reflections
 of simple roots
 if $n$ is even, 
 for which $G=O(n+1,1)$ is not in the Harish-Chandra class.  



Our strategy is to use partly 
 the relation 
 of translation functors for $G=O(n+1,1)$
 and the subgroup 
 $\overline G=SO(n+1,1)$
 which is in the Harish-Chandra class.  



Theorem \ref{thm:1807113} is proved in Section \ref{subsec:pftran1}
 as a consequence
 of the following two propositions.  

\begin{proposition}
\label{prop:IVsub}
Suppose that $(V,\lambda) \in \Reducible$.  
Retain the notation as in Theorem \ref{thm:1807113}.  
Then the $G$-module 
 $\psi_{\rho^{(i)}}^{r(V,\lambda)}(I_{\delta}(i,i))\otimes \chi$ contains
 $I_{(-1)^{\lambda-i}\delta}(V,\lambda)$ as a subquotient.   
Equivalently, the $G$-module $P_{r(V,\lambda)}(I_{\delta}(i,i) \otimes F(V,\lambda))$, 
 see Definition \ref{def:Fshift} below, 
 contains $I_{\delta}(V, \lambda)$ as a subquotient.  
\end{proposition}

We recall from \eqref{eqn:sgnchi}
 that the character 
\index{A}{1chipmpmVlmd@$\chi(V,\lambda)$}
$\chi \equiv \chi(V,\lambda)$
 is trivial 
 when restricted to the subgroup $\overline G=SO(n+1,1)$.  

\begin{proposition}
\label{prop:transrest}
Suppose that $(V,\lambda)\in \Reducible$.  
Retain the notation as in Theorem \ref{thm:1807113}.  
Then $\psi_{\rho^{(i)}}^{r(V,\lambda)}(I_{\delta}(i,i))|_{\overline G}$
 is isomorphic to 
 $I_{(-1)^{\lambda-i}\delta}(V,\lambda)|_{\overline G}$
 as a $\overline G$-module.  
\end{proposition}



Similarly, 
 Theorem \ref{thm:1808101} is proved 
in Section \ref{subsec:pf168}
 by using analogous results, 
 namely,
 Propositions \ref{prop:Iisub} and \ref{prop:PIFrest}
 in Section \ref{subsec:pf168}.  


\subsubsection{Basic lemmas for the translation functor}
\label{subsec:tensorcomb}
We use the following well-known lemma, 
 which holds 
 without the assumption that $G$ is of Harish-Chandra class.  
\begin{lemma}
\label{lem:Ftensor}
Let $F$ be a finite-dimensional representation
 of $G$, 
 $V \in \widehat{O(n)}$, 
 $\delta \in \{\pm\}$, 
 and $\lambda \in {\mathbb{C}}$.  
Then there is a $G$-stable filtration
\[
  \{0\} = I_0 \subset I_1 \subset \cdots \subset I_k 
 = I_{\delta}(V, \lambda) \otimes F
\]
such that
\[
 I_j /I_{j-1} \simeq \operatorname{Ind}_P^G(V_{\lambda,\delta} \otimes F^{(j)})
\quad
  (1 \le j \le k)
\]
where $F^{(j)}$ is a $P$-module
 such that the unipotent radical $N_+$ acts trivially
 and that $F^{(j)}|_{M A}$ is isomorphic to a subrepresentation
 of the restriction $F|_{M A}$ to the Levi subgroup $M A$.  
\end{lemma}
For the sake of completeness, 
we give a proof.  
\begin{proof}
Take a $P$-stable filtration
\[
   \{0\} = F_0 \subset F_1 \subset \cdots \subset F_k = F
\]
 such that the unipotent radical $N_+$ of $P$ acts trivially
 on 
\[
  F^{(j)}:=F_j/F_{j-1}
\quad (1 \le j \le k).  
\]


As in \eqref{eqn:Vlmddelta}, 
 we denote by 
\index{A}{Vln@$V_{\lambda,\delta}=V \otimes \delta \otimes {\mathbb{C}}_{\lambda}$, representation of $P$}
$V_{\lambda,\delta}$
 the irreducible $P$-module
 which is an extension of the $M A$-module
 $V \boxtimes \delta \boxtimes {\mathbb{C}}_{\lambda}$
 with trivial $N_+$ action.  
We define $G$-modules $I_j$ ($0 \le j \le k$)
 by 
\[
  I_j:=\operatorname{Ind}_P^G(V_{\lambda,\delta} \otimes F_j|_P).  
\]
Then there is a natural filtration of $G$-modules
\[
 0 =
 I_0
\subset
I_1
\subset
\cdots
\subset
I_k=\operatorname{Ind}_P^G(V_{\lambda,\delta} \otimes F|_P)
\]
such that 
\[
I_j/I_{j-1}
\simeq
\operatorname{Ind}_P^G(V_{\lambda,\delta} \otimes (F_j/F_{j-1}))
\]
as $G$-modules.  
Since the finite-dimensional representation $F$ of $G$
 is completely reducible
 when viewed as a representation of the Levi subgroup $M A$, 
 the $M A$-module 
$
   F^{(j)}=F_j/F_{j-1}
$
 is isomorphic
 to a subrepresentation of the restriction $F|_{M A}$.  
Now Lemma \ref{lem:Ftensor} follows from the following $G$-isomorphism:
\[
 \operatorname{Ind}_P^G(V_{\lambda,\delta} \otimes F|_P)
 \simeq
 \operatorname{Ind}_P^G(V_{\lambda,\delta}) \otimes F.  
\]
\end{proof}
Similarly to Lemma \ref{lem:Ftensor}, 
 we have the following lemma
 for cohomological parabolic induction.  
Retain the notation
 as in Section \ref{subsec:Aqgeneral}.  
\begin{lemma}
\label{lem:Rqtensor}
Suppose that ${\mathfrak{q}}={\mathfrak{l}}_{\mathbb{C}}+{\mathfrak{u}}$
 is a $\theta$-stable parabolic subalgebra
 of ${\mathfrak{g}}_{\mathbb{C}}$
 with Levi subgroup $L$, 
 see \eqref{eqn:LeviLq}, 
 and that $W$ a finite-dimensional $({\mathfrak{l}}, L \cap K)$-module.  
Let $F$ be a finite-dimensional representation of $G$, 
 and 
\[
  \{0\}= F_0 \subset F_1 \subset \cdots \subset F_k =F
\]
 a $({\mathfrak{q}},L)$-stable filtration 
 such that the nilpotent radical ${\mathfrak{u}}$
 acts trivially on $F^{(j)}:=F_j/F_{j-1}$.  
Then there is a natural spectral sequence
\index{A}{RzqS@${\mathcal{R}}_{\mathfrak{q}}^S$, cohomological parabolic induction}
\[
  {\mathcal{R}}_{\mathfrak{q}}^p(W \otimes F^{(j)} \otimes {\mathbb{C}}_{\rho({\mathfrak{u}})})
  \Rightarrow
  {\mathcal{R}}_{\mathfrak{q}}^p(W \otimes F \otimes {\mathbb{C}}_{\rho({\mathfrak{u}})})
  \simeq 
  {\mathcal{R}}_{\mathfrak{q}}^p(W \otimes {\mathbb{C}}_{\rho({\mathfrak{u}})})
  \otimes F
\]
 as $({\mathfrak{g}},K)$-modules.  
\end{lemma}


The proof is similar to the case 
 where $G$ is in the Harish-Chandra class, 
 see \cite[Lem.~7.23]{Vogan81}.  

\vskip 1pc
By the definition \eqref{eqn:translation}
 of the translation functor $\psi_{\mu}^{\mu+\tau}$, 
 we need to estimate
 possible ${\mathfrak{Z}}_G({\mathfrak{g}})$-infinitesimal characters
 of $\operatorname{Ind}_P^G(V_{\lambda,\delta} \otimes F^{(j)})$
 in Lemma \ref{lem:Ftensor}
 or that of ${\mathcal{R}}_{\mathfrak{q}}^p(W \otimes F^{(j)} \otimes {\mathbb{C}}_{\rho({\mathfrak{u}})})$
 in Lemma \ref{lem:Rqtensor}.  



In order to deal with reductive groups
that are not in the Harish-Chandra class, 
 we use the following lemma 
 which is formulated in a slightly stronger form
 than \cite[Lem.~7.2.18]{Vogan81}, 
but has the same proof.  

\begin{lemma}
\label{lem:V7218}
Let ${\mathfrak{h}}_{\mathbb{C}}$ be a Cartan subalgebra
 of a complex semisimple Lie algebra ${\mathfrak{g}}_{\mathbb{C}}$, 
 $W_{\mathfrak{g}}$ the Weyl group 
 of the root system $\Delta({\mathfrak{g}}_{\mathbb{C}}, {\mathfrak{h}}_{\mathbb{C}})$, 
 $\Delta^+({\mathfrak{g}}_{\mathbb{C}}, {\mathfrak{h}}_{\mathbb{C}})$ 
 a positive system,
 $\langle \, , \, \rangle$ 
 a $W_{\mathfrak{g}}$-invariant inner product 
 on ${\mathfrak{h}}_{\mathbb{R}}^{\ast}:=\operatorname{Span}_{\mathbb{R}} \Delta ({\mathfrak{g}}_{\mathbb{C}}, {\mathfrak{h}}_{\mathbb{C}})$, 
 and $||\,\cdot\,||$ its norm.  



Suppose that $\nu$ and $\tau \in {\mathfrak{h}}_{\mathbb{R}}^{\ast}$
 satisfy
\begin{alignat*}{2}
\langle \nu, \alpha^{\vee} \rangle &\in {\mathbb{N}}_+
\qquad
&&({}^{\forall} \alpha \in \Delta^+({\mathfrak{g}}_{\mathbb{C}}, {\mathfrak{h}}_{\mathbb{C}})), 
\\
\langle \nu+\tau, \alpha^{\vee} \rangle &\in {\mathbb{N}}
\qquad
&&({}^{\forall} \alpha \in \Delta^+({\mathfrak{g}}_{\mathbb{C}}, {\mathfrak{h}}_{\mathbb{C}})).  
\end{alignat*}

If $\gamma \in {\mathfrak{h}}_{\mathbb{C}}^{\ast}$
 satisfies the following two conditions:
\begin{alignat}{2}
\nu + \gamma =& w (\nu+\tau)
\qquad
&&\text{for some $w  \in W_{\mathfrak{g}}$}, 
\label{eqn:V81a}
\\
|| \gamma || \le& ||\tau||, 
\qquad
&&
\label{eqn:V81b}
\end{alignat}
then $\gamma = \tau$.  
\end{lemma}

\begin{remark}
In \cite[Lem.~7.2.18]{Vogan81}, 
$\gamma$ is assumed to be a weight
 occurring in the irreducible finite-dimensional representation
 of $G$ 
 (in the Harish-Chandra class)
 with extremal weight $\tau$
 instead of our assumption \eqref{eqn:V81b}.  
\end{remark}

\subsection{Definition of an irreducible finite-dimensional
\\
 representation $F(V,\lambda)$ of $G=O(n+1,1)$
\label{sec:FVlmd}}
For $(V,\lambda)\in \RInt$, 
 {\it{i.e.}}, 
  for $V \in \widehat {O(n)}$
 and $\lambda \in {\mathbb{Z}} \setminus S(V)$, 
 we defined in Chapter \ref{sec:aq}
\index{B}{ZGginfinitesimalcharacter@${\mathfrak{Z}}_G({\mathfrak{g}})$-infinitesimal character}
\begin{alignat*}{2}
& i(V,\lambda) \in \{0,1,\ldots,n\}, 
\quad
&&\text{height of $(V,\lambda)$
 (Definition \ref{def:iVlmd})}, 
\\
& r(V,\lambda) \in {\mathbb{C}}^{[\frac n 2]+1}, 
&&\text{giving the ${\mathfrak{Z}}_G({\mathfrak{g}})$-infinitesimal character 
 of $I_{\delta}(V,\lambda)$, }
\\
&
&&\text{see \eqref{eqn:IVZG}.}
\end{alignat*}
In this section 
 we introduce an irreducible finite-dimensional representation
 $F(V,\lambda)$ of $G=O(n+1,1)$
 which contains important information
 on signatures.  

\subsubsection{Definition of $\sigma^{(i)}(\lambda)$
 and $\widehat{\sigma^{(i)}}$}
We begin with some combinatorial notation.  
\begin{definition}
\label{def:sigmailmd}
Let $m:=[\frac n 2]$.  
For $1 \le i \le n$, 
 $\sigma=(\sigma_1, \cdots,\sigma_m) \in \Lambda^+(m)$, 
 and $\lambda \in {\mathbb{Z}}$, 
 we define $\sigma^{(i)}(\lambda) \in {\mathbb{Z}}^{m+1}$
 as follows.  
\par\noindent
{\bf{\underline{Case 1.}}}\enspace
$n=2m$
\begin{equation*}
\sigma^{(i)}(\lambda):=
\begin{cases}
(\sigma_{1}-1,\cdots,\sigma_{i}-1,i-\lambda,\sigma_{i+1},\cdots,\sigma_{m})
\quad
&\text{for $0 \le i \le m-1$,}
\\
(\sigma_{1}-1,\cdots,\sigma_{m}-1,|\lambda-m|)
\quad
&\text{for $i=m$,}
\\
(\sigma_{1}-1,\cdots,\sigma_{n-i}-1,\lambda-i,\sigma_{n-i+1},\cdots,\sigma_{m})
\quad
&\text{for $m+1 \le i \le n$.  }
\end{cases}
\end{equation*}

\par\noindent
{\bf{\underline{Case 2.}}}\enspace
$n=2m+1$
\begin{equation*}
\sigma^{(i)}(\lambda):=
\begin{cases}
(\sigma_{1}-1,\cdots,\sigma_{i}-1,i-\lambda,\sigma_{i+1},\cdots,\sigma_{m})
\quad
&\text{for $0 \le i \le m$,}
\\
(\sigma_{1}-1,\cdots,\sigma_{n-i}-1,\lambda-i,\sigma_{n-i+1}, \cdots,\sigma_m)
\quad
&\text{for $m+1 \le i \le n$.  }
\end{cases}
\end{equation*}
Moreover we define $\widehat{\sigma^{(i)}} \in {\mathbb{Z}}^m$
 to be the vector obtained by removing the 
 $\min(i+1, n-i+1)$-th component from $\sigma^{(i)}(\lambda)\in {\mathbb{Z}}^{m+1}$.  
\end{definition}

\par\noindent
{\bf{\underline{Case 1.}}}\enspace
$n=2m$
\begin{equation*}
\widehat{\sigma^{(i)}}:=
\begin{cases}
(\sigma_{1}-1,\cdots,\sigma_{i}-1,\sigma_{i+1},\cdots,\sigma_{m})
\quad
&\text{for $0 \le i \le m-1$,}
\\
(\sigma_{1}-1,\cdots,\sigma_{m}-1)
\quad
&\text{for $i=m$,}
\\
(\sigma_{1}-1,\cdots,\sigma_{n-i}-1,\sigma_{n-i+1},\cdots,\sigma_{m})
\quad
&\text{for $m+1 \le i \le n$.  }
\end{cases}
\end{equation*}

\par\noindent
{\bf{\underline{Case 2.}}}\enspace
$n=2m+1$
\begin{equation*}
\widehat{\sigma^{(i)}}:=
\begin{cases}
(\sigma_{1}-1,\cdots,\sigma_{i}-1,\sigma_{i+1},\cdots,\sigma_{m})
\quad
&\text{for $0 \le i \le m$,}
\\
(\sigma_{1}-1,\cdots,\sigma_{n-i}-1,\sigma_{n-i+1}, \cdots,\sigma_m)
\quad
&\text{for $m+1 \le i \le n$.  }
\end{cases}
\end{equation*}

\begin{definitionlemma}
\label{deflem:180694}
Let $m:=[\frac n 2]$.  
For $(V, \lambda) \in \RInt$, 
 {\it{i.e.}}, 
 for $V \in \widehat{O(n)}$
 and $\lambda \in {\mathbb{Z}} \setminus S(V)$, 
 we write $V=\Kirredrep{O(n)}{\sigma}_{\varepsilon}$
 with $\sigma \in \Lambda^+(m)$
 and $\varepsilon \in \{\pm\}$.  
We set
\begin{equation}
\label{eqn:sigmalmd}
\sigma(\lambda) := \sigma^{(i)}(\lambda), 
\end{equation}
where $i:=i(V,\lambda) \in \{0,1,\ldots,n\}$
 is the height of $(V,\lambda)$ as in \eqref{eqn:indexV}.  
Then we have
\[
    \sigma(\lambda) \in \Lambda^+(m+1).  
\]
\end{definitionlemma}
\begin{proof}
Suppose $n=2m$ (even).  
Let $\lambda \in {\mathbb{Z}}$.  
By the definition of $R(V;i)$ (Definition \ref{def:RVi}), 
 we have the following equivalences:
\vskip 1pc
\par\noindent
$\bullet$\enspace for $0 \le i \le m-1$, 
\begin{align*}
\lambda \in R(V;i)
\,\,\Leftrightarrow\,\, & \sigma_i-i> -\lambda > \sigma_{i+1}-i-1
\\
\,\,\Leftrightarrow\,\, & \sigma_i-1 \ge i -\lambda \ge  \sigma_{i+1};
\intertext{$\bullet$\enspace for $i=m$, }
\lambda \in R(V;m)
\,\,\Leftrightarrow\,\, & -\sigma_m <\lambda-m< \sigma_{m}
\\
\,\,\Leftrightarrow\,\, & \sigma_m-1 \ge |\lambda -m|;
\intertext{$\bullet$\enspace for $m+1 \le i \le n$, }
\lambda \in R(V;i)
\,\,\Leftrightarrow\,\, & \sigma_{n-i+1}-1 <\lambda-i< \sigma_{n-i}
\\
\,\,\Leftrightarrow\,\, & \sigma_{n-i}-1 \ge \lambda -i \ge \sigma_{n-i+1}.  
\end{align*}
Thus in all cases $\sigma^{(i)}(\lambda) \in \Lambda^+(m+1)$.  


The proof for $n$ odd is similar.  
\end{proof}

\subsubsection{Definition of a finite-dimensional representation $F(V,\lambda)$
 of $G$}
We are ready to define a finite-dimensional representation, 
 to be denoted by $F(V,\lambda)$, 
 for $(V,\lambda)\in \RInt$.  

\begin{definition}
[a finite-dimensional representation $F(V,\lambda)$]
\label{def:Fshift}
Suppose that  $(V,\lambda) \in \RInt$,
{\it{i.e.}},
 $V \in \widehat{O(n)}$ and $\lambda \in {\mathbb{Z}} \setminus S(V)
$.  
We write $V=\Kirredrep{O(n)}{\sigma}_{\varepsilon}$
 with $\sigma \in \Lambda^+(m)$
 and $\varepsilon \in \{\pm\}$
 where $m:=[\frac n 2]$.  
We set $i:=i(V,\lambda)$, 
 the height of $(V,\lambda)$
 as in \eqref{eqn:indexV}, 
 and 
\index{A}{1pasigmalmd@$\sigma(\lambda)$ $(=\sigma^{(i)}(\lambda))$|textbf}
$
\sigma(\lambda) \in \Lambda^+(m+1)
$
as in Definition-Lemma \ref{deflem:180694}.  



We define an irreducible finite-dimensional representation
 $F(V, \lambda)$ of $G=O(n+1,1)$
 as follows:
\begin{enumerate}
\item[$\bullet$]
for $V$ of type Y
 and $\lambda=\frac n 2 (=m)$, 
\begin{align*}
F(V,\lambda):=& \Kirredrep{O(n+1,1)}{\sigma(\lambda)}_{+,+}
\\
             =& \Kirredrep{O(n+1,1)}{\sigma_1-1, \cdots,\sigma_m-1,0}_{+,+};
\end{align*}
\item[$\bullet$]
for $V$ of type X
 or $\lambda \ne \frac n 2$, 
\index{A}{FVlmd@$F(V,\lambda)$|textbf}
\begin{equation}
\label{eqn:FVlmd}
   F(V,\lambda)
:=
\begin{cases}
\Kirredrep {O(n+1,1)}
           {\sigma(\lambda)}_{\varepsilon,(-1)^{\lambda-i}\varepsilon}
\quad
&\text{if $i \le \frac n 2$}, 
\\
\Kirredrep {O(n+1,1)}
           {\sigma(\lambda)}_{-\varepsilon,(-1)^{\lambda-i-1}\varepsilon}
\quad
&\text{if $i > \frac n 2$}, 
\end{cases}
\end{equation}
see \eqref{eqn:Fn1ab} for notation.  
\end{enumerate}



By using the character $\chi \equiv \chi(V,\lambda)$ of $G$
 as defined in \eqref{eqn:sgnchi}, 
 we obtain a unified expression
\begin{equation}
\label{eqn:FVchi}
F(V,\lambda) \simeq F(\sigma(\lambda))_{+,(-1)^{\lambda-i}} \otimes \chi.
\end{equation}
\end{definition}

\begin{remark}
\label{rem:FVlmd}
We note
 that \eqref{eqn:FVlmd} is well-defined.  
In fact, 
 if $V$ is of type Y (Definition \ref{def:OSO}), 
 then $\varepsilon$ is not uniquely determined 
 because there are two expressions for $V$:
\[
  V \simeq \Kirredrep{O(n)}{\sigma}_+ \simeq \Kirredrep{O(n)}{\sigma}_-, 
\]
 see Lemma \ref{lem:fdeq} (1).  
On the other hand, 
 the $(m+1)$-th component of $\sigma(\lambda)$ does not vanish
 except for the case $i=\lambda=m$
 by Definition \ref{def:sigmailmd}.  
Hence we obtain an isomorphism
 of $O(n+1,1)$-modules:
\[
   \Kirredrep {O(n+1,1)}{\sigma(\lambda)}_{a,b}
   \simeq
   \Kirredrep {O(n+1,1)}{\sigma(\lambda)}_{-a,-b}
\]
for any $a, b \in \{\pm\}$ by Lemma \ref{lem:fdeq} (2).  
\end{remark}

By Definition \ref{def:typeone}, 
 the following lemma is clear.  
\begin{lemma}
\label{lem:FVlmddet}
Suppose that $V$ is of type X 
 or $\lambda \ne \frac n 2$.  
Then there is a natural isomorphism
 of $O(n+1,1)$-modules:
\[
  F(V \otimes \det, \lambda)
\simeq
 F(V, \lambda) \otimes \det.  
\]
\end{lemma}
\begin{lemma}
\label{lem:FVlmdY}
The following two conditions on $(V,\lambda)\in \RInt$
 ({\it{i.e.}}, 
 $V \in \widehat{O(n)}$
 and $\lambda \in {\mathbb{Z}} \setminus S(V)$)
 are equivalent:
\begin{enumerate}
\item[{\rm{(i)}}]
$F(V, \lambda) \otimes \det \simeq F(V, \lambda)$
 as $G$-modules;
\item[{\rm{(ii)}}]
$V$ is of type Y (Definition \ref{def:OSO})
 and $\lambda \ne \frac n 2$.  
\end{enumerate}
In particular,
 for $(V,\lambda)\in \Reducible$, 
 (i) holds 
 if and only if $(V,\lambda)\in \RedJ$
 (Definition \ref{def:Red12}).  
\end{lemma}

\begin{proof}
Any of the conditions (i) or (ii) implies
 that $n$ is even, 
 say,  $n=2m$.  
Let us verify (ii) $\Rightarrow$ (i).  
If we write $V = \Kirredrep{O(n)}{\sigma}_{\varepsilon}$
 for some $\sigma=(\sigma_1, \cdots, \sigma_m) \in \Lambda^+(m)$
 and $\varepsilon\in \{\pm\}$, 
 then $\sigma_m \ne 0$
 because $V$ is of type Y.  
On the other hand,
 the height $i:=i(V,\lambda)$ is not equal to $m$
 because $\lambda \ne m$, 
 hence the $(m+1)$-th component of $\sigma^{(i)}(\lambda)$ equals 
 $\sigma_m (\ne 0)$
 by Definition \ref{def:sigmailmd}.  
Thus there is a natural $G$-isomorphism
 $F(V, \lambda) \otimes \det \simeq F(V, \lambda)$.  
The converse implication is similarly verified.  
\end{proof}
\begin{example}
\label{ex:FVexterior}
Let $(V,\lambda)=(\Exterior^\ell({\mathbb{C}}^n), \ell)$
 for $\ell=0,1,\cdots,n$.  
We set $m=[\frac n 2]$ as usual.  
Then
\[
\text{
 $i(V,\lambda)=\ell,\,\,
 \sigma(\lambda)=0$
 $(\in {\mathbb{Z}}^{m+1}),$ 
 and
 $\widehat{\sigma^{(i)}}=0 \in {\mathbb{Z}}^m$.
}  
\]
Moreover, 
we have an isomorphism of $G$-modules:
\[
   F(V,\lambda)\simeq {\bf{1}}
\quad
  \text{for $0 \le \ell \le n$}.
\]
\end{example}


\subsubsection{Reformulation of Theorems \ref{thm:1807113} and \ref{thm:1808101}}

By using the finite-dimensional representation $F(V,\lambda)$
 of $G=O(n+1,1)$
 (Definition \ref{def:Fshift}), 
 Theorems \ref{thm:1807113} and \ref{thm:1808101}
 may be reformulated in simpler forms, 
respectively, 
 as follows.  
\begin{theorem}
\label{thm:181104}
For $(V,\lambda) \in \Reducible$
 (Definition \ref{def:RIntRed}), 
 we set 
\index{A}{iVlmd@$i(V,\lambda)$, height}
$i:=i(V,\lambda)$, 
 the height of $(V,\lambda)$
 as in \eqref{eqn:indexV}.  
Then there is a natural $G$-isomorphism:
\index{A}{FVlmd@$F(V,\lambda)$}
\[
  P_{r(V,\lambda)}(I_{\delta}(i,i) \otimes F(V,\lambda))
  \simeq 
  I_{\delta}(V, \lambda).  
\]
\end{theorem}

\begin{theorem}
\label{thm:181107}
Suppose $(V,\lambda) \in \Reducible$.  
Retain the notation as in Theorem \ref{thm:181104}.  
\begin{enumerate}
\item[{\rm{(1)}}]
If $(V,\lambda) \in \RedI$
(Definition \ref{def:Red12}), 
 then there is a natural $G$-isomorphism:
\[
  P_{r(V,\lambda)}(I_{\delta}(V, \lambda) \otimes F(V,\lambda))
  \simeq 
  I_{\delta}(i, i).  
\]
\item[{\rm{(2)}}]
If $(V,\lambda) \in \RedJ$, 
 then there is a natural $G$-isomorphism:
\[
  P_{r(V,\lambda)}(I_{\delta}(V, \lambda) \otimes F(V,\lambda))
  \simeq 
  I_{\delta}(i, i) \oplus I_{\delta}(n-i, i).  
\]
\end{enumerate}
\end{theorem}

\subsubsection{Translation of irreducible representations
 $\Pi_{\ell,\delta}$}

We recall from \eqref{eqn:Pild}
 that $\Pi_{\ell,\delta}$ ($0 \le \ell \le n+1$, $\delta \in \{\pm\}$)
 are irreducible admissible smooth representations
 of $G$
 with trivial infinitesimal character $\rho_G$,
 and from \eqref{eqn:PiVlmd}
 that $\Pi_{\delta}(V,\lambda)$ is an irreducible admissible smooth representation of $G$
 with ${\mathfrak{Z}}_G({\mathfrak{g}})$-infinitesimal character
 $r(V,\lambda) \mod W_G$.  
We also recall that $\rho^{(i)} \equiv \rho_G \mod W_G$
for all $0 \le i \le n$.  
In this section,
 we determine the action 
 of translation functor 
 $\psi_{\rho^{(i)}}^{r(V,\lambda)}$ 
 on irreducible representations.  

\begin{theorem}
\label{thm:20180904}
Suppose that $(V,\lambda) \in \Reducible$.  
Let $i:=i(V,\lambda)$ be the height of $(V,\lambda)$, 
 and $F(V,\lambda)$ be the irreducible finite-dimensional representation
 of $G$
 (Definition \ref{def:Fshift}).  
Then there is a natural $G$-isomorphism:
\[
  P_{r(V,\lambda)} (\Pi_{i,\delta} \otimes F(V,\lambda))
  \simeq
  \Pi_{\delta}(V,\lambda).  
\]
\end{theorem}

\begin{proof}
Since the translation functor is a covariant exact functor
 (Proposition \ref{prop:transfunct} (1)), 
 the exact sequence of $G$-modules
\[
  0 \to \Pi_{i,\delta}\to I_{\delta}(i,i) \to \Pi_{i+1,-\delta} \to 0
\]
(Theorem \ref{thm:LNM20} (1)) yields 
 an exact sequence of $G$-modules
\[
  0 \to P_{r(V,\lambda)}(\Pi_{i,\delta} \otimes F(V,\lambda))
    \to I_{\delta}(V,\lambda)
    \to P_{r(V,\lambda)}(\Pi_{i+1,-\delta}\otimes F(V,\lambda))
    \to 0, 
\]
 where we have used Theorem \ref{thm:181104}
 for the middle term.  
Since the first and third terms do not vanish 
 by Proposition \ref{prop:transfunct} (2), 
 we conclude the following isomorphisms of $G$-modules:
\begin{align*}
I_{\delta}(V,\lambda)^{\flat}
&\simeq P_{r(V,\lambda)}(\Pi_{i,\delta} \otimes F(V,\lambda)), 
\\
I_{\delta}(V,\lambda)^{\sharp}
&\simeq P_{r(V,\lambda)}(\Pi_{i+1,-\delta} \otimes F(V,\lambda))
\end{align*}
because $I_{\delta}(V,\lambda)$ has composition series
 of length two
 (Corollary \ref{cor:length2}).  
Hence Theorem \ref{thm:20180904} follows from the definition 
 \eqref{eqn:PiVlmd} of $\Pi_{\delta}(V,\lambda)$.  
\end{proof}

\subsubsection{Proof of Theorems \ref{thm:181104} and \ref{thm:181107}}

In this subsection,
 we explain that Theorem \ref{thm:1807113}
 is equivalent to Theorem \ref{thm:181104};
 Theorem \ref{thm:1808101} is equivalent to Theorem \ref{thm:181107}.  

For this 
 we begin with the following lemma which clarifies
 some combinatorial meaning
 of the height $i(V,\lambda) \in \{0,1,\ldots,n\}$
 and the dominant integral weight $\sigma(\lambda) \in \Lambda^+(m+1)$
 in Definition \ref{def:Fshift}.  
Here we recall $m=[\frac n 2]$.  
\begin{lemma}
\label{lem:transvec}
Suppose $V=\Kirredrep{O(n)}{\sigma}_{\varepsilon}$
 with $\sigma \in \Lambda^+(m)$
 and $\varepsilon \in \{\pm\}$.  
For $0 \le i \le n$ and $\lambda \in {\mathbb{Z}}$, 
 we set
\index{A}{1patauiVlmd@$\tau^{(i)}(V,\lambda)$|textbf}
\begin{equation}
\label{eqn:tauiVlmd}
\tau^{(i)}(V,\lambda) :=r(V,\lambda)-\rho^{(i)} \in (\frac 1 2{\mathbb{Z}})^{m+1}, 
\end{equation}
see \eqref{eqn:IVZG} and Example \ref{ex:rhoi}
 for the notation.  
\begin{enumerate}
\item[{\rm{(1)}}]
Then $\tau^{(i)}(V,\lambda) \in {\mathbb{Z}}^{m+1}$
 is given by 
\[
\begin{cases}
(\sigma_{1}-1,\cdots,\sigma_{i}-1,\sigma_{i+1},\cdots,\sigma_{m}, \lambda-i)
\quad
&\text{for $0 \le i \le m$, }
\\
(\sigma_{1}-1,\cdots,\sigma_{n-i}-1,\sigma_{n-i+1}, \cdots,\sigma_m, \lambda-i)
\quad
&\text{for $m+1 \le i \le n$.  }
\end{cases}
\]
\item[{\rm{(2)}}]
Assume that $\lambda \in {\mathbb{Z}} \setminus S(V)$, 
 and we take $i$ to be the height $i(V, \lambda)$ of $(V,\lambda)$
 as in \eqref{eqn:indexV}.  
Then, 
\[
  {\textsl{$r(V,\lambda)$ and $\rho^{(i)}$ belong to the same Weyl chamber
 for $W_{\mathfrak{g}}$.   }}
\]
\item[{\rm{(3)}}]
Let $\sigma(\lambda)$ be as defined in Definition \ref{def:Fshift}.  
Then we have
\begin{equation}
\label{eqn:tauisigma}
   \tau^{(i)}(V,\lambda)_{\operatorname{dom}}
   =
  \sigma(\lambda).  
\end{equation}
\end{enumerate}
\end{lemma}

\begin{proof}
(1)\enspace
 Clear from the definition \eqref{eqn:IVZG}
 of $r(V,\lambda)$ and $\rho^{(i)}$.  
(2)\enspace
The assertion is verified by inspecting
 the definition \eqref{eqn:indexV} of the height $i(V,\lambda)$.  (3)\enspace
The statement follows from Definition-Lemma \ref{deflem:180694}.  
\end{proof}


Now we determine the action of the translation functor
 $\psi_{\rho^{(i)}}^{r(V,\lambda)}$.  
We recall that the principal series representation
 $I_{\delta}(i,i)$ $(0 \le i \le n)$
 has the trivial ${\mathfrak{Z}}_G({\mathfrak{g}})$-infinitesimal character,
 which is $W_{\mathfrak{g}}$-regular
 but not always $W_G$-regular.  
We apply the translation functor \eqref{eqn:translation} to $I_{\delta}(i,i)$
 for an appropriate choice of $i$.  

\begin{proposition}
\label{prop:transIii}
Let $m=[\frac n 2]$.  
Suppose $G=O(n+1,1)$, 
 $\delta \in \{\pm\}$, 
 $V=\Kirredrep{O(n)}{\sigma}_{\varepsilon}$
 with $\sigma \in \Lambda^+(m)$ and $\varepsilon \in \{\pm\}$, 
 and $\lambda \in {\mathbb{Z}} \setminus (S(V) \cup S_Y(V))$.  
Let $i:=i(V,\lambda) \in \{0,1,\ldots,n\}$ be as in \eqref{eqn:indexV}. 
We define $r(V, \lambda)\in {\mathbb{C}}^{m+1}$ as in \eqref{eqn:IVZG}
 and $\sigma(\lambda)\in \Lambda^+(m+1)$.  
Then we have 
\[
  \psi_{\rho^{(i)}}^{r(V,\lambda)}(I_{\delta}(i,i)) 
= P_{r(V,\lambda)}(I_{\delta}(i,i)  
  \otimes 
  \Kirredrep {O(n+1,1)}{\sigma(\lambda)}_{+,+}).  
\]
\end{proposition}
\begin{proof}
Since $I_{\delta}(i,i)$ has the trivial ${\mathfrak{Z}}_G({\mathfrak{g}})$-infinitesimal character,
 $P_{\rho^{(i)}}(I_{\delta}(i,i))=I_{\delta}(i,i)$
 by \eqref{eqn:rhoiW}.  
Since $r(V,\lambda)=\rho^{(i)} + \tau^{(i)}(V,\lambda)$
 by \eqref{eqn:tauiVlmd}, 
 and since $\sigma(\lambda)=\tau^{(i)}(V,\lambda)_{\operatorname{dom}}$ by
 \eqref{eqn:tauisigma}, 
 the definition of the translation functor shows 
\[
  \psi_{\rho^{(i)}}^{\rho^{(i)} + \tau^{(i)}(V,\lambda)}(I_{\delta}(i,i)) 
= P_{r(V,\lambda)}(I_{\delta}(i,i) \otimes \Kirredrep{O(n+1,1)}{\sigma(\lambda)}_{+,+}).
\]

Thus Proposition \ref{prop:transIii} is proved.  
\end{proof}

It follows from Proposition \ref{prop:transIii}
 and from the definition of $F(V,\lambda)$
(Definition \ref{def:Fshift})
 that Theorem \ref{thm:1807113}
 is equivalent to Theorem \ref{thm:181104}
 and Theorem \ref{thm:1808101} is equivalent to Theorem \ref{thm:181107}.  



\subsection{Proof of Proposition \ref{prop:IVsub}}
\label{subsec:pftrans}
In this section we complete the proof of Proposition \ref{prop:IVsub}.  
By Lemma \ref{lem:Ftensor}, 
 the proof reduces to some branching laws
 for the restriction
 of finite-dimensional representations
 of $G=O(n+1,1)$
 with respect to $M A \simeq O(n) \times SO(1,1)$
 and to the study of their tensor product representations, 
 see Proposition \ref{prop:180863}.  



\subsubsection{Irreducible summands
 for $O(n+2) \downarrow O(n) \times O(2)$
 and for tensor product representations}
Before working with Proposition \ref{prop:180863}
 in the noncompact setting, 
 we first discuss analogous branching rules 
 for the restriction with respect to a pair
 of compact groups $O(n+2) \supset O(n) \times O(2)$:
\begin{lemma}
[$O(n+2)\downarrow O(n) \times O(2)$]
\label{lem:1810122}
Let $\mu=(\mu_1, \cdots,\mu_{m+1}) \in \Lambda^+(m+1)$, 
 where $m:=[\frac n 2]$ as before.  
For $1\le k \le m+1$, 
 we set
\[
\mu_{(k)}':=(\mu_1, \cdots,\mu_{k-1}, \widehat{\mu_k}, \mu_{k+1}, \cdots,\mu_{m+1})
 \in \Lambda^+(m).  
\]
Then the $O(n+2)$-module $\Kirredrep{O(n+2)}{\mu}_+$
 (see \eqref{eqn:ONCreal}) contains the 
 $(O(n) \times O(2))$-module
\[
   \bigoplus_{k=1}^{m+1} \Kirredrep{O(n)}{\mu_{(k)}'}_+
   \boxtimes \Kirredrep{O(2)}{\mu_k}_+
\]
 when restricted to the subgroup $O(n) \times O(2)$.  
\end{lemma}

\begin{proof}
Take a Cartan subalgebra ${\mathfrak{h}}_{\mathbb{C}}$
 of ${\mathfrak{g l}}(n+2,{\mathbb{C}})$
 such that ${\mathfrak{h}}_{\mathbb{C}} \cap {\mathfrak{o}}(n+2,{\mathbb{C}})$
 is a Cartan subalgebra of ${\mathfrak{o}}(n+2,{\mathbb{C}})$.  
We identify ${\mathfrak{h}}_{\mathbb{C}}^{\ast}$
 with ${\mathbb{C}}^{n+1}$ 
 via the standard basis $\{f_j\}$
 as before,
 and choose a positive system $\Delta^+({\mathfrak{gl }}(n+2,{\mathbb{C}}), {\mathfrak{h}}_{\mathbb{C}})=\{f_i-f_j: 1 \le i < j \le n+2\}$.  
Then 
\[
\widetilde \mu:=(\mu_1, \cdots,\mu_{m+1}, 0^{n+1-m})\in \Lambda^+(n+2)
\]
 is a dominant integral 
 with respect to the positive system. 
Let $v_{\widetilde \mu}$ be a (nonzero) highest weight vector
 of the irreducible representation 
 $(\tau, \Kirredrep{U(n+2)}{\widetilde \mu})$
 of the unitary group $U(n+2)$.  
By definition,
 the $O(n+2)$-module
 $\Kirredrep{O(n+2)}{\mu}_+$, 
 see \eqref{eqn:ONCreal}, 
 is the unique irreducible $O(n+2)$-summand
 of $\Kirredrep{U(n+2)}{\widetilde \mu}$
 containing the highest weight vector $v_{\widetilde \mu}$.  
We now take a closer look at the $U(n+2)$-module
 $\Kirredrep{U(n+2)}{\widetilde \mu}$.  
Fix $1 \le k \le m+1$.  
Iterating the classical branching rule 
 for $U(N) \supset U(N-1) \times U(1)$
 for $N=n+2$, $n+1$,  
 we see that the restriction
 $\Kirredrep{U(n+2)}{\widetilde \mu}|_{U(n) \times U(2)}$
 contains
\[
   \Kirredrep{U(n)}{\widetilde {\mu_{(k)}'}} \boxtimes W
\]
 as an irreducible summand,
 where
\[
  \widetilde{\mu_{(k)}'}:=(\mu_1, \cdots,\widehat{\mu_k}, \cdots,\mu_{m+1}, 0^{n-m}) \in \Lambda^+(n)
\]
and $W$ is an irreducible representation of $U(2)$
 which has a weight $(\mu_k,0)$.  
Since all the weights of an irreducible finite-dimensional representation
 are contained in the convex hull
 of the Weyl group orbit
 through the highest weight, 
 we conclude 
 that $(\mu_k,0)$ is actually the highest weight of the $U(2)$-module $W$.  
Hence the $(U(n) \times U(2))$-module
\[
   \Kirredrep{U(n)}{\widetilde {\mu_{(k)}'}}
   \boxtimes
   \Kirredrep{U(2)}{\mu_k,0}
\]
occurs as an irreducible summand
 of the $U(n+2)$-module 
 $\Kirredrep{U(n+2)}{\widetilde \mu}$.  
We now consider the following diagram
 of subgroups of $U(n+2)$, 
 and investigate the restriction of the $U(n+2)$-module
 $\Kirredrep{U(n+2)}{\widetilde \mu}$.  
\begin{alignat*}{3}
 &U(n+2) &&\supset\,\, &&U(n) \times U(2)
\\
 &\hphantom{MM} \cup && &&\hphantom{MM}\cup
\\
 &O(n+2) &&\supset\,\, &&O(n) \times O(2)
\end{alignat*}
By our choice of the Cartan subalgebra
 ${\mathfrak{h}}_{\mathbb{C}}$, 
 we observe
 that there exists $w_k \in O(n+2)$
 such that $\operatorname{Ad}(w_k){\mathfrak{h}}_{\mathbb{C}}
={\mathfrak{h}}_{\mathbb{C}}$
 and 
\[
  w_k \widetilde \mu
=(\mu_1, \cdots,\widehat{\mu_k}, \cdots,\mu_{m+1}, 0^{n-m},\mu_k,0) \in 
{\mathbb{Z}}^{n+2}, 
\]
where we write $w_k \widetilde \mu$ simply
 for the contragredient action
 of $\operatorname{Ad}(w_k)$ on $\widetilde \mu \in {\mathfrak{h}}_{\mathbb{C}}^{\ast}$
 $(\simeq {\mathbb{C}}^{n+2})$.  
In particular,
 the $O(n+2)$-submodule
 $\Kirredrep{O(n+2)}{\mu}_+$
 of the restriction $\Kirredrep{U(n+2)}{\widetilde \mu}|_{O(n+2)}$
 contains the weight vector
 $v_{w_k \widetilde \mu}:=\tau(w_k) v_{\widetilde \mu}$
 for the weight $w_k \widetilde \mu$.  
Since $w_k \widetilde \mu$ is an extremal weight, 
 the weight vector in $\Kirredrep{U(n+2)}{\widetilde \mu}|_{O(n+2)}$
 is unique up to scalar multiplication.  
Hence $v_{w_k \widetilde \mu}$ is contained also in the submodule
 $\Kirredrep{U(n)}{\widetilde {\mu_{(k)}'}} \boxtimes \Kirredrep{U(2)}{\mu_{k},0}$.  
Thus we conclude that the irreducible $O(n+2)$-module
 $\Kirredrep{O(n+2)}{\mu}_+$ contains
\[
   \Kirredrep{O(n)}{\mu_{(k)}'} \boxtimes \Kirredrep{O(2)}{\mu_k}
\]
 as an $(O(n) \times O(2))$-summand 
 when restricted to the subgroup $O(n) \times O(2)$ of $O(n+2)$.  
\end{proof} 



Let $m=[\frac n 2]$
 as before.  
Let $V=\Kirredrep{O(n)}{\sigma}_{\varepsilon}$
 with $\sigma \in \Lambda^+(m)$
 and $\varepsilon \in \{\pm\}$.  
Suppose $\lambda \in {\mathbb{Z}} \setminus S(V)$.  
We recall from Definition-Lemma \ref{deflem:180694}
 that $\sigma(\lambda) \in \Lambda^+(m+1)$.  

\begin{lemma}
\label{lem:180862}
For $0 \le i \le n$, 
 the following $(O(n) \times O(2))$-module
\[
  (\Exterior^i({\mathbb{C}}^n) \boxtimes {\bf{1}})
  \otimes 
  \Kirredrep{O(n+2)}{\sigma(\lambda)}_{\varepsilon}|_{O(n) \times O(2)}
\]
contains
\begin{alignat*}{2}
  &V \boxtimes
  \Kirredrep{O(2)}{|\lambda-i|}_+
\qquad
  &&\text{if $i \le \frac n 2$}
\\
  &(V \otimes \det) \boxtimes \Kirredrep{O(2)}{|\lambda-i|}_+
\qquad
 &&\text{if $i \ge \frac n 2$}
\end{alignat*}
as an irreducible summand.  
\end{lemma}
We note that $V \simeq V \otimes \det$
 as $O(n)$-modules
 if $i=\frac n 2$
 by Lemmas \ref{lem:typeY} and \ref{lem:isig}.  
\begin{proof}
It suffices to prove Lemma \ref{lem:180862}
 for $\varepsilon =+$
 by using a similar argument
 to \eqref{eqn:314}
 for the pair $(O(n+2), O(n) \times O(2))$
 and for $\chi=\det$.  
Then Lemma \ref{lem:180862} is derived from the following two branching laws of compact Lie groups.  
\begin{enumerate}
\item[$\bullet$]
$O(n+2) \downarrow O(n) \times O(2)$:
\par
By Lemma \ref{lem:1810122}, 
 the $O(n+2)$-module $\Kirredrep{O(n+2)}{\sigma(\lambda)}_+$ contains
\[
  \Kirredrep{O(n)}{\widehat{\sigma^{(i)}}}_+
  \boxtimes
  \Kirredrep{O(2)}{|\lambda-i|}
\]
as an irreducible summand
when restricted to the subgroup $O(n) \times O(2)$, 
 see Definition \ref{def:sigmailmd} for the notation $\widehat{\sigma^{(i)}}$. 
\item[$\bullet$]
Tensor product for $O(n)$:
\par
The tensor product representation
\[
  \Exterior^i({\mathbb{C}}^n) 
  \otimes 
  \Kirredrep{O(n)}{\widehat{\sigma^{(i)}}}_+
\]
contains 
\begin{alignat*}{2}
V&\simeq  \Kirredrep {O(n)}{\sigma}_{+}
&&\text{if $i \le \frac n 2$}, 
\\
V\otimes \det &\simeq  \Kirredrep {O(n)}{\sigma}_{-}
\qquad
&&\text{if $i \ge \frac n 2$}
\end{alignat*}
 as an irreducible component.  
\end{enumerate}
\end{proof}

\subsubsection{Irreducible summand for the restriction
 $G \downarrow M A$
 and for tensor product representations}

We recall that the Levi subgroup $M A$
 of the parabolic subgroup $P$
 in $G=O(n+1,1)$ is expressed as
\[
   M A \simeq  O(n) \times S O(1,1)
       \simeq O(n) \times {\mathbb{Z}} / 2 {\mathbb{Z}} \times {\mathbb{R}}.
\]  
The goal of the subsection is to prove
 the following proposition.  
\begin{proposition}
[tensor product and the restriction $O(n+1,1) \downarrow M A$]
\label{prop:180863}
~~~\newline
Suppose that $(V,\lambda) \in \Reducible$, 
 {\it{i.e.}}, 
 $V \in \widehat{O(n)}$
 and $\lambda \in {\mathbb{Z}} \setminus (S(V) \cup S_Y(V))$.  
Let $i=i(V,\lambda)$ be the height of $(V,\lambda)$
 (see \eqref{eqn:indexV}), 
 and $F(V, \lambda)$ be the irreducible $O(n+1,1)$-module
 as in Definition \ref{def:Fshift}.  
Then the $M A$-module
\index{A}{FVlmd@$F(V,\lambda)$}
\begin{equation}
\label{eqn:itensorF}
  (\Exterior^i({\mathbb{C}}^n) \boxtimes \delta \boxtimes {\mathbb{C}}_i)
  \otimes
  F(V, \lambda)|_{M A}
\end{equation}
contains
\[
V \boxtimes \delta \boxtimes {\mathbb{C}}_{\lambda}
\]
 as an irreducible component.  
\end{proposition}
In what follows, 
 we use a mixture of notations
 in describing irreducible finite-dimensional representations
 (see Sections \ref{subsec:repON} and \ref{subsec:fdimrep}).  
To be precise, 
 we shall use:
\begin{enumerate}
\item[$\bullet$]
$\Lambda^+(O(n+2))$ $(\subset {\mathbb{Z}}^{n+2})$, 
 see \eqref{eqn:CWOn}, 
 to denote irreducible holomorphic finite-dimensional representations
 of the {\it{complex}} Lie group $O(n+1,{\mathbb{C}})$
 as in Section \ref{subsec:repON};
\item[$\bullet$]
$\Lambda^+(m+1)$ $(\subset {\mathbb{Z}}^{m+1})$
 and signatures to denote irreducible
 finite-dimensional representations
 of the {\it{real}} groups $O(n+2)$ and $O(n+1,1)$
 where $m:=[\frac n 2]$, 
 as in Section \ref{subsec:fdimrep}.  



See \eqref{eqn:ONCreal} 
 for the relationship 
 among these representations.  
\end{enumerate}

\begin{proof}
[Proof of Proposition \ref{prop:180863}]
We write $V=\Kirredrep {O(n)}{\sigma}_{\varepsilon}$ as before
 where $\sigma \in \Lambda^+(m)$, 
 $\varepsilon \in \{\pm\}$, 
 and $m=[\frac n 2]$.  
By Weyl's unitary trick for the disconnected group $O(n+1,1)$, 
 see \eqref{eqn:ONCreal}, 
 the restrictions of the holomorphic representation
 $\Kirredrep {O(n+2,{\mathbb{C}})}{\sigma(\lambda), 0^{n+1-m}}$
 to the subgroups $O(n+2)$ and $O(n+1,1)$
 are given respectively by
\begin{align*}
&\Kirredrep {O(n+2)}{\sigma(\lambda)}_{+}, 
\\
&\Kirredrep {O(n+1,1)}{\sigma(\lambda)}_{+,+}.  
\end{align*}
Then Lemma \ref{lem:180862} implies
 that the holomorphic $(O(n,{\mathbb{C}}) \times O(2,{\mathbb{C}}))$-representation
\[
 (\Exterior^i({\mathbb{C}}^n) \boxtimes {\bf{1}})
 \otimes
 \Kirredrep{O(n+2,{\mathbb{C}})}{\sigma(\lambda), 0^{n+1-m}}
 |_{O(n,{\mathbb{C}}) \times O(2,{\mathbb{C}})}
\]
contains
\begin{alignat*}{2}
   &\Kirredrep{O(n,{\mathbb{C}})}{\sigma,0^{n-m}}
   \boxtimes
   \Kirredrep{O(2,{\mathbb{C}})}{|\lambda-i|,0}
\quad
   &&\text{if $i \le \frac n 2$, }
\\
   &(\Kirredrep{O(n,{\mathbb{C}})}{\sigma,0^{n-m}} \otimes \det)
   \boxtimes
   \Kirredrep{O(2,{\mathbb{C}})}{|\lambda-i|,0}
\quad
   &&\text{if $i \ge \frac n 2$}
\end{alignat*}
as an irreducible summand.  
Because the restriction of the first factor
 to compact real from $O(n)$ is isomorphic
 to $\Kirredrep{O(n)}{\sigma}_+$ or $\Kirredrep{O(n)}{\sigma}_-$
 according to whether $i \le \frac n 2$ or $i \ge \frac n 2$.  
Taking the restriction to another real form 
 $O(n) \times O(1,1)$ of $O(n,{\mathbb{C}}) \times O(2,{\mathbb{C}})$, 
 we set that the $(O(n) \times O(1,1))$-module
\[
   (\Exterior^i({\mathbb{C}}^n) \boxtimes {\bf{1}})
   \otimes
   \Kirredrep {O(n+1,1)}{\sigma(\lambda)}_{+,+}
   |_{O(n) \times O(1,1)}
\]
contains
\begin{alignat*}{2}
   &\Kirredrep{O(n)}{\sigma}_+
   \boxtimes
   \Kirredrep{O(2,{\mathbb{C}})}{|\lambda-i|,0}|_{O(1,1)}
\quad
   &&\text{if $i \le \frac n 2$}, 
\\
   &\Kirredrep{O(n)}{\sigma}_- 
   \boxtimes
   \Kirredrep{O(2,{\mathbb{C}})}{|\lambda-i|,0}|_{O(1,1)}
\quad
   &&\text{if $i \ge \frac n 2$}
\end{alignat*}
as an irreducible summand.  



Since $V = \Kirredrep{O(n)}{\sigma}_{\varepsilon}$, 
 the definition of $F(V,\lambda)$
 (Definition \ref{def:Fshift})
 implies that the $M A$-module
\[
  (\Exterior^i({\mathbb{C}}^n) \boxtimes {\bf{1}}) \otimes F(V,\lambda)|_{M A}
\]
 contains 
\[
   V \boxtimes 
  (\Kirredrep {O(2,{\mathbb{C}})}{|\lambda-i|,0}|_{S O(1,1)} 
   \otimes 
   \chi_{\varepsilon,(-1)^{\lambda-i}\varepsilon}|_{S O(1,1)})
\]
as an $M A$-module.  
Here we have used that $M A \simeq O(n) \times S O (1,1)$
 and that $\chi_{a,b}|_{S O(1,1)} \simeq \chi_{-a,-b}|_{S O(1,1)}$.  
Hence Proposition \ref{prop:180863} is derived from the following lemma
 on the restriction $O(2,{\mathbb{C}}) \downarrow S O(1,1)$.  
\end{proof}


Let ${\mathbb{C}}_k$ denote the holomorphic character 
 of $S O(2,{\mathbb{C}})$ on ${\mathbb{C}} e^{i k \theta}$.  

\begin{lemma}
[$O(2,{\mathbb{C}}) \downarrow S O(1,1)$]
\label{lem:SOtworest}
\[
  \Kirredrep{O(2,{\mathbb{C}})}{k,0}|_{{\mathbb{Z}} / 2 {\mathbb{Z}} \times {\mathbb{R}}}
  \simeq
\begin{cases}
(-1)^k \boxtimes ({\mathbb{C}}_k \oplus {\mathbb{C}}_{-k}) 
\quad
&\text{for $k \in {\mathbb{N}}_+$}, 
\\
{\bf{1}} \boxtimes {\bf{1}}
\quad
&\text{for $k =0$, }
\end{cases}
\]
where we identify $S O(1,1) \simeq \{\pm I_2\} \times S O_0(1,1)$
 with ${\mathbb{Z}} / 2 {\mathbb{Z}} \times {\mathbb{R}}$.  
In particular, 
the $SO(1,1)$-module
\begin{align*}
&\Kirredrep {O(2,{\mathbb{C}})}{|\lambda-i|,0}|_{S O(1,1)} 
   \otimes 
   \chi_{\varepsilon,(-1)^{\lambda-i}\varepsilon}|_{S O(1,1)}
\\
\simeq
&\Kirredrep {O(2,{\mathbb{C}})}{|\lambda-i|,0}|_{S O(1,1)} 
   \otimes 
   \chi_{-\varepsilon,(-1)^{\lambda-i-1}\varepsilon}|_{S O(1,1)}
\end{align*}
contains
\[
  {\bf{1}} \boxtimes {\mathbb{C}}_{\lambda -i}
\]
 as an irreducible summand.  
\end{lemma}

\begin{proof}
For $k \in {\mathbb{N}}_+$, 
 the holomorphic representation $\Kirredrep{O(2,{\mathbb{C}})}{k,0}$
 is a two-dimensional representation 
 of $O(2,{\mathbb{C}})$, 
 which is isomorphic to $\operatorname{Ind}_{S O(2,{\mathbb{C}})}^{O(2,{\mathbb{C}})}({\mathbb{C}} e^{i k \theta})$.  
Its restriction to the connected subgroup $S O(2,{\mathbb{C}})$
 decomposes into a sum of two characters
 of $S O(2,{\mathbb{C}})$:  
\[
   \Kirredrep{O(2,{\mathbb{C}})}{k,0}|_{S O(2,{\mathbb{C}})}
  \simeq
  {\mathbb{C}} e^{i k \theta} \oplus {\mathbb{C}} e^{-i k \theta}, 
\]
on which the central element $-I_2$ acts
 as the scalar multiplication of $(-1)^k=(-1)^{-k}$.  
Since $S O(1,1)$ is generated by the central element
 $-I_2$ 
 and the identity component $S O_0(1,1)$, 
 Lemma \ref{lem:SOtworest} follows.  
\end{proof}

\subsubsection{Proof of Proposition \ref{prop:IVsub}}
\label{subsec:1653}
\begin{proof}
[Proof of Proposition \ref{prop:IVsub}]
Let $F(V,\lambda)$ be the finite-dimensional representation of $G$
 as in Definition \ref{def:Fshift}.  
Filter $F(V,\lambda)$ as in Lemma \ref{lem:Ftensor}.  
We may assume in addition 
 that each $F^{(j)} = F(V,\lambda)_j/F(V,\lambda)_{j-1}$ is irreducible
 as an $M A$-module.  
Then by Proposition \ref{prop:180863}, 
 $I_{\delta}(V,\lambda)$ occurs as a subquotient
 of the $G$-module
 $P_{r(V,\lambda)}(I_{\delta}(i,i) \otimes F(V,\lambda))$.  
Hence the second assertion 
 of Proposition \ref{prop:IVsub} is shown.  
By Proposition \ref{prop:transIii}, 
 the first assertion follows.  
\end{proof}

\subsection{Proof of Theorem \ref{thm:1807113}}
\label{subsec:pftran1}

In this section
 we complete the proof of Theorem \ref{thm:1807113}
 and also its reformulation Theorem \ref{thm:181104}.  
By Proposition \ref{prop:IVsub}, 
 it suffices to show Proposition \ref{prop:transrest}
 is an isomorphism
 in the level of $\overline G$-modules
 instead of the isomorphism
 in Theorem \ref{thm:1807113} as $G$-modules.  



We divide the argument 
according to the decomposition 
\[
 \Reducible=\RedI \amalg \RedJ, 
\]
where we recall from Definition \ref{def:Red12}:
\begin{enumerate}
\item[$\bullet$]
$(V,\lambda) \in \RedI$, 
 if $V$ is of type X or $\lambda = \frac n 2$;
\item[$\bullet$]
$(V,\lambda) \in \RedJ$, 
 if $V$ is of type Y and $\lambda =\frac n 2$.  
\end{enumerate}

As we shall show in the proof of Proposition \ref{prop:transrest} below, 
 the following assertion holds with the notation therein.  
\begin{proposition}
\label{prop:transrest2}
There  is a natural isomorphism, 
 as $\overline G$-modules
\begin{equation*}
  \psi_{\rho^{(i)}}^{r(V,\lambda)}(I_{\delta}(i,i))|_{\overline G}
\simeq
\begin{cases}
  \overline \psi_{\rho^{(i)}}^{r(V,\lambda)}
 (I_{\delta}(i,i)|_{\overline G})
\quad
&\text{if $(V,\lambda)\in \RedI$}, 
\\
\bigoplus_{\xi=\pm}
\overline \psi_{\rho^{(i)}}^{r(V^{(\xi)},\lambda)}
 (I_{\delta}(i,i)|_{\overline G})
\quad
&\text{if $(V,\lambda)\in \RedJ$}.  
\end{cases}
\end{equation*}
\end{proposition}

\subsubsection{Case: $(V,\lambda) \in \RedI$}
\index{A}{RzqSeducibleI@$\RedI$}
In this subsection,
 we discuss the case 
 where $V$ is of type X
 or $\lambda=\frac n 2$.  
\begin{proof}
[Proof of Proposition \ref{prop:transrest} for $(V,\lambda) \in \RedI$.]
If $n$ is odd, 
 then Proposition \ref{prop:transrest}
 follows from Lemma \ref{lem:Xtrans} (1).  


Hereafter we assume $n$ is even, 
 say $n=2m$.  
We claim 
 that Proposition \ref{prop:transrest} follows from
 Lemma \ref{lem:Xtrans} (2)
 if $V$ is of type X
 (Definition \ref{def:OSO})
 or $\lambda=m$.  
To see this, 
 it is enough 
 to verify 
 that all of $\rho^{(i)}$, 
 $r(V,\lambda)$, 
 and $\tau^{(i)}(V,\lambda)=r(V,\lambda)-\rho^{(i)}$, 
 see \eqref{eqn:tauiVlmd}, 
 contain 0
 in their entries.  
This is automatically true
 for $\rho^{(i)}$ as $\rho^{(i)} \in W_G \rho^G$
 (Example \ref{ex:rhoi} (3))
 and $n$ is even.  
For $r(V,\lambda)$, 
 one sees from \eqref{eqn:IVZG} 
 that the $m$-th component vanishes
 if $V$ is of type X
 and the $(m+1)$-th component vanishes
 if $\lambda=m$.  
For $\tau^{(i)}(V,\lambda)$, 
 one see from the formula 
 of $\tau^{(i)}(V,\lambda)$
in Lemma \ref{lem:transvec}
 that an analogous assertion holds
 because $\lambda=m$ $(=\frac n 2)$ implies
 that the height $i(V,\lambda)$ equals $m$
 by Definition \ref{def:iVlmd}.  
Hence Proposition \ref{prop:transrest}
 for $(V,\lambda) \in \RedI$ is shown.  
\end{proof}


\subsubsection{Case: $(V,\lambda) \in \RedJ$}
\index{A}{RzqSeducibleII@$\RedJ$}

In this subsection, 
 we discuss the case 
 where $V$ is of type Y and $\lambda \ne \frac n2$.  
In this case, 
 $n$ is even ($=2m$), 
 $i:=i(V,\lambda)=m$,  
 and the restriction of $V$ to $SO(n)$
 is a sum of two irreducible representations of $SO(n)$:
\[
  V=V^{(+)} \oplus V^{(-)}, 
\]
as in \eqref{eqn:Vpm}.  
We extend the definition \eqref{eqn:IVZG}
 of $r(V,\lambda)$ 
 to irreducible representations $V^{(\pm)}$ of $SO(n)$
 with $n=2m$ by 
\[
  r(V^{(\pm)},\lambda):=(\sigma_1+m-1,\cdots, \sigma_{m-1}+1, \pm \sigma_m, \lambda-m) \in {\mathbb{Z}}^{m+1}.  
\]
Then $r(V^{(\pm)}, \lambda)$ viewed as an element of ${\mathfrak{h}}_{\mathbb{C}}^{\ast}/W_{{\mathfrak{g}}}$ 
 is the ${\mathfrak{Z}}({\mathfrak{g}})$-infinitesimal character
 of the principal series representation
 $\overline I_{\delta}(V^{(\pm)},\lambda)$
 of $\overline G=SO(n+1,1)$.  



As in \eqref{eqn:tauiVlmd}, 
 we set
\[
   \tau^{(i)}(V^{(\pm)},\lambda):=r(V^{(\pm)},\lambda)-\rho^{(i)}.  
\]
Inspecting the definition \eqref{eqn:indexV}
 of the height $i:=i(V,\lambda)$, 
 we see that both $r(V^{(\pm)},\lambda)$ and $\rho^{(i)}$
 belong to the same Weyl chamber
 with respect to the Weyl group $W_{\mathfrak{g}}$
 ({\it{not}} $W_G$)
 as in Lemma \ref{lem:transvec}.  



By Lemma \ref{lem:FVlmdY}, 
 the irreducible finite-dimensional $G$-module
 $F(V, \lambda)$ decomposes into a direct sum
 of two irreducible $\overline G$-modules, 
 which we may write as
\[
  F(V, \lambda)|_{\overline G}=\overline F(V^{(+)}, \lambda)
  \oplus \overline F(V^{(-)}, \lambda).  
\]
To be precise, 
 we set $\sigma^{(+)} (\lambda):=\sigma(\lambda)$
 (Definition \ref{def:Fshift}), 
 and define $\sigma^{(-)}(\lambda)$
 by replacing the $(m+1)$-th component $\sigma_m$
 with $-\sigma_m$.  
For instance,
 if $\lambda<m$, 
 then the height $i=i(V,\lambda)$ is smaller than $m$
and
\begin{align*}
\sigma^{(+)}(\lambda) =&(\sigma_1-1,\cdots,\sigma_i-1,i-\lambda, \sigma_{i+1},\cdots, \sigma_{m-1}, \sigma_{m}), 
\\
\sigma^{(-)}(\lambda) =&(\sigma_1-1,\cdots,\sigma_i-1,i-\lambda, \sigma_{i+1},\cdots, \sigma_{m-1}, -\sigma_{m}).  
\end{align*}
Then $\overline F(V^{(\pm)}, \lambda)$ are
 the irreducible $\overline G$-modules
 such that 
\[
\overline F(V^{(\pm)}, \lambda) \otimes \chi_{+,(-1)^{\lambda-i}}|_{SO(n+1,1)}
\]
 extends to irreducible holomorphic finite-dimensional representations
 of the connected complex Lie group $SO(n+2,{\mathbb{C}})$
 with highest weights $\sigma^{(\pm)}(\lambda)$.  



\begin{proof}
[Proof of Proposition \ref{prop:transrest} for $(V,\lambda) \in \RedJ$]
By the definition \eqref{eqn:translation}
 of the translation functor
 and by Lemma \ref{lem:primary}, 
 there is a natural $\overline G$-isomorphism:
\begin{multline}
\label{eqn:Ytrans}
   \psi_{\rho^{(i)}}^{r(V,\lambda)}(I_{\delta}(i,i))|_{\overline G} 
\\
  \simeq
  (\overline P_{r(V^{(+)},\lambda)}
   + 
   \overline P_{r(V^{(-)},\lambda)})
  (I_{\delta}(i,i)|_{\overline G}
  \otimes 
  (\overline F(V^{(+)},\lambda) \oplus \overline F(V^{(-)},\lambda))).  
\end{multline}
We claim for $\xi, \eta \in \{\pm\}$: 
\begin{alignat}{2}
\label{eqn:paraterm}
   \overline P_{r(V^{(\xi)},\lambda)}
  (I_{\delta}(i,i)|_{\overline G}
  \otimes 
 \overline F(V^{(\eta)},\lambda))
  =&
  \overline \psi_{\rho^{(i)}}^{r(V^{(\xi)}, \lambda)}(I_{\delta}(i,i)|_{\overline G})
\quad
&&\text{if $\xi \eta =+$}, 
\\
\label{eqn:crosszero}
   \overline P_{r(V^{(\xi)},\lambda)}
  (I_{\delta}(i,i)|_{\overline G}
  \otimes 
  \overline F(V^{(\eta)},\lambda))=&0
\quad
&&\text{if $\xi \eta =-$}.  
\end{alignat}
The first claim \eqref{eqn:paraterm}
 holds by definition \eqref{eqn:transSO}.  
To see the vanishing \eqref{eqn:crosszero}
 of the cross terms in \eqref{eqn:crosszero},
 suppose that 
\[
   \rho^{(i)} + \gamma = w (\rho^{(i)}+ \tau^{(i)}(V^{(\xi)}, \lambda))
\]
for some weight $\gamma$ in $F(V^{(\eta)}, \lambda)$
 and for some $w \in W_{\mathfrak{g}}$.  
Then we have 
\[
   ||\gamma|| \le ||\tau^{(i)}(V^{(\eta)}, \lambda)||=||\tau^{(i)}(V^{(\xi)}, \lambda)||.  
\]
Hence we can apply Lemma \ref{lem:V7218}
 and conclude
\[
  \gamma = \tau^{(i)}(V^{(\xi)}, \lambda).  
\]
By the vanishing \eqref{eqn:crosszero}
 of the cross terms in \eqref{eqn:Ytrans}, 
 we obtain the following $\overline G$-isomorphisms:
\begin{align*}
  \psi_{\rho^{(i)}}^{r(V,\lambda)}(I_{\delta}(i,i))|_{\overline G} 
  \simeq&
  \bigoplus_{\xi \in \{\pm\}}
  \overline \psi_{\rho^{(i)}}^{r(V^{(\xi)},\lambda)}(I_{\delta}(i,i)|_{\overline G}) \\
  \simeq& 
  \bigoplus_{\xi \in \{\pm\}} \overline I_{(-1)^{\lambda-i} \delta}(V^{(\xi)}, \lambda), 
\end{align*}
which is isomorphic to the restriction
 of the principal series representation
 $I_{(-1)^{\lambda-i} \delta}(V, \lambda)$ of $G$
 to the subgroup $\overline G$
 by \eqref{eqn:IVSOpm}.  
\end{proof}

\subsection{Proof of Theorem \ref{thm:1808101}}
\label{subsec:pf168}
In this section,
 we show Theorem \ref{thm:1808101}, 
 or its reformulation,
 Theorem \ref{thm:181107}.  
The proof is similar 
 to that of Theorem \ref{thm:1807113}, 
 hence we give only a sketch
 of the proof with focus on necessary changes.  
A part of the proof is carried out
 separately
 according to the decomposition 
\[
\Reducible = \RedI \amalg \RedJ
\quad
\text{(Definition \ref{def:Red12})}.  
\]



The following lemma is a counterpart 
 of Proposition \ref{prop:180863}.  
\begin{lemma}
[tensor product and $G \downarrow MA$]
\label{lem:1808101}
Suppose $(V, \lambda) \in \Reducible$.  
Let $i:=i(V,\lambda)$ be the height 
 of $(V,\lambda)$, 
 see \eqref{eqn:indexV}, 
 and $F(V,\lambda)$ be the irreducible
 finite-dimensional representation 
 of $G=O(n+1,1)$
 as in Definition \ref{def:Fshift}.  
Then the $M A$-module
\[
  (V \boxtimes \delta \boxtimes {\mathbb{C}}_{\lambda})
 \otimes 
 F(V, \lambda)|_{M A}
\]
contains
\begin{alignat*}{2}
& \Exterior^i({\mathbb{C}}^n) \boxtimes \delta \boxtimes {\mathbb{C}}_i
&&
\text{if $(V,\lambda) \in \RedI$, }
\\
& (\Exterior^i({\mathbb{C}}^n) \boxtimes \delta \boxtimes {\mathbb{C}}_i)
 \oplus 
(\Exterior^{n-i}({\mathbb{C}}^n) \boxtimes \delta \boxtimes {\mathbb{C}}_i)
\qquad
&&
\text{if $(V,\lambda) \in \RedJ$, }
\end{alignat*}
 as an irreducible component.  
\end{lemma}
\begin{proof}
The proof is similar to that of Proposition \ref{prop:180863}
 except that there is a $G$-isomorphism
 $F(V, \lambda) \otimes \det \simeq F(V, \lambda)$
 by Lemma \ref{lem:FVlmdY}
 if $(V,\lambda) \in \RedJ$.  
In this case, 
 the height $i=i(V,\lambda)$ is not equal to $\frac n 2$
 by Lemma \ref{lem:isig} (3).  
Thus both the $O(n)$-modules 
$
 \Exterior^i({\mathbb{C}}^n)
$
 and 
$
   \Exterior^{n-i}({\mathbb{C}}^n) 
   \simeq 
   \Exterior^{i}({\mathbb{C}}^n) \otimes \det
$
 occur simultaneously 
 in $V \otimes F(V,\lambda)|_{O(n)}$.  
\end{proof}

Theorem \ref{thm:181107}, 
 or equivalently, 
 Theorem \ref{thm:1808101}
 is deduced from the following two propositions.  

\begin{proposition}
\label{prop:Iisub}
 Suppose $(V,\lambda) \in \Reducible$.  
(Definition \ref{def:RIntRed}), 
equivalently,
 $V \in \widehat{O(n)}$
 and $\lambda \in {\mathbb{Z}}\setminus(S(V) \cup S_Y(V))$.  
Then the $G$-module
 $P_{r(V,\lambda)}(I_{\delta}(V,\lambda)
 \otimes F(V,\lambda))$ contains
\begin{alignat*}{2}
& I_{\delta}(i,i)
\quad
&&\text{for $(V,\lambda) \in \RedI$}, 
\\
& \text{$I_{\delta}(i,i)$ and $I_{\delta}(n-i,i)$}
\quad
&&\text{for $(V,\lambda) \in \RedJ$}, 
\end{alignat*}
as subquotients.  
\end{proposition}

\begin{proof}
As in the proof of Proposition \ref{prop:IVsub}
 in Section \ref{subsec:1653}, 
 Proposition \ref{prop:Iisub} follows readily from 
Lemma \ref{lem:Ftensor}
 by using Lemma \ref{lem:1808101}.  
\end{proof}


\begin{proposition}
\label{prop:PIFrest}
Suppose $(V,\lambda) \in \Reducible$, namely,
 $V \in \widehat{O(n)}$
 and $\lambda \in {\mathbb{Z}} \setminus (S(V) \cup S_Y(V))$.  
Then there is a natural isomorphism
 of $\overline G$-modules:
\begin{equation*}
P_{r(V,\lambda)}(I_{\delta}(V,\lambda)
 \otimes F(V,\lambda))|_{\overline G}
 \simeq
\begin{cases}
I_{\delta}(i,i)|_{\overline G}
\quad
&
\text{for $(V,\lambda) \in \RedI$}, 
\\
I_{\delta}(i,i)|_{\overline G} \oplus I_{\delta}(n-i,i)|_{\overline G}
\quad
&
\text{for $(V,\lambda) \in \RedJ$}.  
\end{cases}
\end{equation*}
\end{proposition}

\begin{proof}
The proof is similar to that of Proposition \ref{prop:transrest}, 
 again by showing the vanishing of the cross terms
 as in \eqref{eqn:crosszero}.  
\end{proof}

