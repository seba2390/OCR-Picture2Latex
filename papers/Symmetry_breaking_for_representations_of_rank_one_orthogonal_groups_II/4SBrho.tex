\newpage
%%%%%%%%%%%%%%%%%%%%%%%%%%%%%%%%%%%%%%%%%%%%%%%%%%%%%%%%%%%%%%%%
\section
{Symmetry breaking for irreducible representations
 with infinitesimal character $\rho$.}
\label{sec:SBOrho}
\index{B}{infinitesimalcharacter@infinitesimal character}
%%%%%%%%%%%%%%%%%%%%%%%%%%%%%%%%%%%%%%%%%%%%%%%%%%%%%%%%%%%%%%%%%

In this chapter, 
 we focus on symmetry breaking operators from 
 {\it{irreducible}} representations $\Pi$ of $G=O(n+1,1)$
 with 
\index{B}{infinitesimalcharacter@infinitesimal character}
${\mathfrak{Z}}_G({\mathfrak{g}})$-infinitesimal character
\index{A}{1parho@$\rho_G$}
 $\rho_G$ 
 to {\it{irreducible}} representations $\pi$ of the subgroup $G'=O(n,1)$
 with ${\mathfrak{Z}}_{G'}({\mathfrak{g}}')$-infinitesimal character
 $\rho_{G'}$.  
The main results are Theorems \ref{thm:SBOvanish} and \ref{thm:SBOone}, 
 where we determine the multiplicity 
 $\dim_{\mathbb{C}}{\operatorname{Hom}}_{G'}(\Pi|_{G'}, \pi)$
 for all pairs $(\Pi,\pi)$.  
A diagrammatic formulation 
 of the main results is given in Theorem \ref{thm:SBOfg}.  



The proof uses basic properties
 of the normalized symmetry breaking operators
 for principal series representations
 of $G$ and $G'$, 
\begin{equation*}
\Atbb \lambda \nu {\delta \varepsilon}{i,j} \colon 
I_{\delta}(i,\lambda)  \rightarrow J_{\varepsilon}(j,\nu), 
\end{equation*}
in particular, 
 the $(K,K')$-spectrum 
 on basic $K$-types (Theorem \ref{thm:153315})
 and  their functional equations
 (Theorems \ref{thm:TAA} and \ref{thm:ATA}).



\subsection{Main Theorems} 
We recall from Theorem \ref{thm:LNM20}
 that irreducible admissible smooth representations of $G$ 
 with trivial
\index{B}{infinitesimalcharacter@infinitesimal character}
 ${\mathfrak {Z}}_G({\mathfrak {g}})$-infinitesimal character
 $\rho_G$
 are classified as
\index{A}{IrrGrho@${\operatorname{Irr}}(G)_{\rho}$,
 set of irreducible admissible smooth representation of $G$
 with trivial infinitesimal character $\rho$\quad}
\[
    {\operatorname{Irr}}(G)_{\rho}
   =
  \{
    \Pi_{i, \delta}
    :
    0 \le i \le n+1, \, \delta = \pm
\}.  
\]
Similarly,
 irreducible admissible smooth representations
 of the subgroup $G'=O(n,1)$
 with trivial ${\mathfrak {Z}}_{G'}({\mathfrak {g}}')$-infinitesimal character
 $\rho_{G'}$
 are classified as 
\[
  {\operatorname{Irr}}(G')_{\rho}
  =
  \{
  \pi_{j, \varepsilon}
    :
    0 \le j \le n, \, \varepsilon = \pm
\},   
\]
where we have used lowercase letters $\pi$
 for the subgroup $G'$ instead of $\Pi$.  
We also recall that the representation $\Pi_{i, \delta}$ of $G=O(n+1,1)$
 is 

\begin{enumerate}
\item[$\bullet$]
one-dimensional 
 if and only if $i=0$ or $n+1$;

\item[$\bullet$]
 the smooth representation of a discrete series representation 
 if $i=\frac{n+1}{2}$
 ($n$: odd);

\item[$\bullet$]
 that of a tempered representation
 if $i=\frac n 2$
 ($n$: even).  
\end{enumerate}


The following two theorems determine
 the dimension of
\[
  {\operatorname{Hom}}_{G'}(\Pi|_{G'},\pi)
\quad
\text{for $\Pi \in {\operatorname{Irr}}(G)_{\rho}$
      and $\pi \in {\operatorname{Irr}}(G')_{\rho}$.
}
\]

\begin{theorem}
[vanishing]
\label{thm:SBOvanish}
\index{B}{vanishingtheorem@vanishing theorem}
Suppose $0 \le i \le n+1$, $0 \le j \le n$, 
 $\delta, \varepsilon \in \{\pm\}$.  
\begin{enumerate}
\item[{\rm{(1)}}]
If $j \not = i, i-1$ then
$
  {\operatorname{Hom}}_{G'}(\Pi_{i,\delta}|_{G'}, \pi_{j,\varepsilon})=\{0\}. 
$

\item[{\rm{(2)}}] 
If $\delta \varepsilon =-$, 
 then 
$ 
{\operatorname{Hom}}_{G'}(\Pi_{i,\delta}|_{G'}, \pi_{j,\varepsilon}) =\{0\}.  
$ 
\end{enumerate}
\end{theorem}


\begin{theorem} 
[multiplicity-one]
\label{thm:SBOone}
\index{B}{multiplicityonetheorem@multiplicity-one theorem}
Suppose $0 \le i \le n+1$, $0 \le j \le n$
 and $\delta, \varepsilon \in \{\pm\}$.  
If $j=i-1$ or $i$
 and if $\delta \varepsilon =+$, 
 then 
\[
   \dim_{\mathbb{C}}
   {\operatorname{Hom}}_{G'}
   (\Pi_{i,\delta}|_{G'}, \pi_{j,\varepsilon}) =1.  
\]
\end{theorem}

The proof of Theorems \ref{thm:SBOvanish} and \ref{thm:SBOone}
 will be given in Chapter \ref{sec:pfSBrho}.  
The nonzero symmetry breaking operators from 
 $\Pi_{i,+}$ to $\pi_{j,+}$ ($j \in \{i-1,i\}$) will be applied 
 to construct
\index{B}{period@period}
 periods 
 in Chapter \ref{sec:period}
 (see Theorem \ref{thm:171517} for example).  



\subsection{Graphic description of the multiplicity
 for irreducible representations
 with infinitesimal character $\rho$}

Using the action of the Pontrjagin dual 
 of the component group 
 $(G/G_0)\hspace{1mm}{\widehat{}}\, \times (G'/G_0')\hspace{1MM}{\widehat{}}\,\,$
 on ${\operatorname{Hom}}_{G'}(\Pi_{i,\delta}|_{G'}, \pi_{j,\varepsilon})$,
 see Proposition \ref{prop:SBOdual}, 
 we see that Theorems \ref{thm:SBOvanish} and \ref{thm:SBOone}
 are equivalent to their special case
 where $i \le \frac{n+1}{2}$ and $\delta=+$.  
Furthermore, 
 taking the vanishing result (Theorem \ref{thm:SBOvanish}) into account, 
 we focus on the case $j \le \frac n 2$ and $\varepsilon=+$.  
We then describe Theorems \ref{thm:SBOvanish} and \ref{thm:SBOone}
 graphically in this setting.  

We suppress the subscript, 
 and write $\Pi_i$ for $\Pi_{i,+}$, 
 and $\pi_j$ for $\pi_{j,+}$.  
Then $\Pi_i$ ($0 \le i \le \frac{n+1}{2}$)
 and $\pi_j$ ($0 \le j \le \frac{n}{2}$) are the standard sequence of representations with infinitesimal character $\rho$ of $G$,
 respectively $G'$ starting with the trivial one-dimensional representation
(Definition \ref{def:Pii}). 
In the diagrams below, 
 the first row are representations of $G$,
 the second row are representations of the subgroup $G'$. 
Arrows mean
 that there exist nonzero symmetry breaking operators.  



\begin{theorem} 
\label{thm:SBOfg}
Symmetry breaking for the standard sequence
\index{B}{standardsequence@standard sequence}
 of irreducible representations
 starting at the trivial one-dimensional representations
 are represented graphically
 in Diagrams \ref{fig:Hasse1} and \ref{fig:Hasse2}. 

\medskip
\begin{figure}[htp]
\caption{Symmetry breaking for $O(2m+1,1)\downarrow O(2m,1)$}
\begin{center}
\begin{tabular}{c@{~}c@{~}c@{~}c@{~}c@{~}c@{~}c@{~}c@{~}c}
$\Pi_0$ & & $\Pi_1$ & &\dots & & $\Pi_{m-1}$ & & $\Pi_{m}$ 
\\
$\downarrow$ & $\swarrow$ & $\downarrow$ & $\swarrow$ & & $\swarrow$ & $ \downarrow $
&  $\swarrow $  &  $\downarrow$ 
\\
$\pi_0$ & & $\pi_1$ & &\dots & & $\pi_{m-1}$ & & $\pi_{m}$ 
\end{tabular}
\end{center}
\label{fig:Hasse1}
\end{figure}%


\medskip
\begin{figure}[htp]
\caption{Symmetry breaking for $O(2m+2,1) \downarrow O(2m+1,1)$ }
\begin{center}
\begin{tabular}{@{}c@{~}c@{~}c@{~}c@{~}c@{~}c@{~}c@{~}c@{~}c@{~}c@{~}c@{~}c@{}}
$\Pi_0$& &$\Pi_1$& &\dots & & $\Pi_{m-1} $& & $\Pi_{m}$ & &$\Pi_{m+1}$
\\
$\downarrow$ & $\swarrow$ & $\downarrow $ & $\swarrow$ & & $\swarrow$ & $ \downarrow $& $\swarrow $ & $\downarrow$ & $\swarrow$ &
\\
$\pi_0$& &$\pi_1$& &\dots & & $\pi_{m-1}$ & & $\pi_{m}$ &
\end{tabular}
\end{center}
\label{fig:Hasse2}
\end{figure}%
\end{theorem}

\newpage
