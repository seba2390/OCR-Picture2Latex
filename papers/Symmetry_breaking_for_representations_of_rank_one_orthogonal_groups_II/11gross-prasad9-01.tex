\newpage
%%%%%%%%%%%%%%%%%%%%%%%%%%%%%%%%%%%%%%%%%%%%%%%
\section{Application I: Some conjectures by B.~Gross and D.~Prasad:  Restrictions of tempered representations of $SO(n+1,1)$ to $SO(n,1)$}
\label{sec:Gross-Prasad}
%%%%%%%%%%%%%%%%%%%%%%%%%%%%%%%%%%%%%%%%%%%

Inspired by automorphic forms and $L$-functions,
 B.~Gross and D.~Prasad published in 1992 conjectured
 about the restriction of irreducible
\index{B}{temperedrep@tempered representation}
 tempered representations
 of special orthogonal groups $SO(p+1,q)$
 to a special orthogonal subgroup $SO(p,q)$, 
 see \cite{GP}.  
B.~Sun and C.-B.~Zhu \cite{SunZhu} proved
 that in this case the multiplicities are at most one,
 and B.~Gross and D.~Prasad conjectured that given a Vogan packet of tempered representations of $SO_{n+2} \times SO_{n+1}$ 
 there exist exactly one group $SO(p+1,q) \times SO(p,q)$
 with $p+q=n+1$
 and one (tempered) representation $\overline{U}_1 \boxtimes \overline{U}_2$
 of this group
 with $m( \overline{U}_1 \boxtimes \overline{U}_2,\mathbb C)=1$.
They also stated a conjectured algorithm
 to determine the  group
 and the  representation $\overline{U}_1 \boxtimes \overline{U}_2$
in the  Vogan packet with   $m( \overline{U}_1 \boxtimes \overline{U}_2,\mathbb C)=1$.

In this chapter
 we prove that the algorithm  of B.~Gross and D.~Prasad predicts the multiplicity correctly  for representations in  Vogan packets of  tempered principal series representations 
 of $SO(n+1,1) \times SO(n,1)$ 
 as well as for the 3 irreducible representations
 $\overline{\Pi}, \overline{\pi}, \overline{\varpi}$ of 
 $SO(2m+2,1)$, $SO(2m+1,1)$, $SO(2m,1)$
 with trivial infinitesimal character $\rho$.


\medskip
The Gross--Prasad conjectures are stated only for representations of special orthogonal groups in  \cite{GP}. 
Thus we are considering in this chapter symmetry breaking for tempered representations of 
\index{A}{GOindefinitespecial@$\overline{G}=SO(n+1,1)$}
$\overline{G}\times \overline{G'}=SO(n+1,1) \times SO(n,1)$ and not as in the previous chapters for $G \times G'=O(n+1,1) \times O(n,1)$. 
We refer to  Appendix II
 (Chapter \ref{sec:SOrest}) for notation and 
 for results about the restriction of representations from orthogonal groups to special orthogonal groups.
 
%%%%%%%%%%%%%%%%%%%%%%%%%%
\subsection{Vogan packets of tempered induced representations}
\label{subsec:Vpacket}
%%%%%%%%%%%%%%%%%%%%%%%%%%%

We use a bar
 over representations to distinguish
 between representations of the special orthogonal group
 and those of the orthogonal group.

\medskip
 Every tempered principal series representation of $SO(n+1,1)$ is of the form
\[
  \overline{I}_{\delta}(\overline{V},\lambda) 
  \equiv
  \mbox{Ind}_{\overline{P}}^{\overline{G}}(\overline{V} \boxtimes \delta ,\lambda ) 
\quad \text{for }\,\,(\overline{\sigma},\overline{V}) \in \widehat{SO(n)}, 
\,\, \delta \in \{\pm\}, 
\,\, \lambda \in \frac n 2 + \sqrt{-1}{\mathbb{R}}, 
\]
which is the smooth representation
 of a unitarily induced principal series representation from a finite-dimensional representation of the minimal parabolic subgroup 
\index{A}{PLanglandsdecompbar@$\overline P = \overline M A N_+$}
$\overline P$
 of $\overline G=SO(n+1,1)$.  



For $n$ even,
 we assume that the central element $-I_{n+2}$
 of the special orthogonal group $\overline G=SO(n+1,1)$ acts nontrivially
 on the principal series representation  
 ${\overline{I}}_{\delta}(\overline{V},\lambda)$, 
 and thus $\overline I_{\delta}(\overline V, \lambda)$ is a genuine representation
 of $\overline G$,
{\it{i.e.}},
 that $-I_{n+2}$ is not in the kernel of $\overline{V} \boxtimes \delta.$ 
For $n$ odd,
 $\overline G=SO(n+1,1)$ does not have
 a nontrivial center,
 and we do not need an assumption
 on the pair $(\overline{V},\delta)$.  



We observe if $n$ is odd,
 the Langlands parameter of the representations
 of $SO(n,0)$ factors through the identity component of its $L$-group,
 and it defines a representation of $SO(n-2p,2p)$ and not of $O(n-2p,2p)$, 
 see \cite{Arthur}.  



The Langlands parameter  of the induced representations $ \overline{I}_{\delta}(\overline{V},\lambda)$ factors through the Levi subgroup of a maximal parabolic subgroup of the Langlands dual group 
\index{A}{GOindefinitesprimelang@${}^L G$, Langlands dual group|textbf}
$^L G$ \cite{LLNM}. 
This parabolic subgroup corresponds to a maximal parabolic subgroup
 of $SO(n+1,1)$
 whose Levi subgroup $L$ is a real form of $SO(n,\bC) \times SO(2,\bC)$ and thus is isomorphic to $ SO(n,0) \times SO(1,1)\simeq SO(n) \times GL(1,{\mathbb{R}})$. 
Note that $SO(1,1)\simeq GL(1,{\mathbb{R}})$ is a disconnected group and so determines the character  $\delta $.  



The pure inner real forms of $SO(n,\bC)$
 with a compact Cartan subgroup are $SO(n-2p,2p)$, $0 \leq p \leq \frac n2$.  
For $n$ even, 
 we  assume that the center of $SO(n-2p,2p)$ is not contained
 in the kernel of the discrete series representation, 
 see Proposition \ref{prop:161648} (6).  



By \cite[p.~35]{A1},
 if $G$ is $SO(2m+2,1)$ or $SO(2m+1,1)$,
 then there are $2^m$ representations
 in the Vogan packet containing a tempered representation
 $\overline{I}_{\delta}(\overline{V},\lambda)$
 and they are parametrized by characters
 of a finite group ${\mathcal{A}}_1 \simeq (\bZ/2\bZ)^m$.  
We write
\index{A}{VP@$VP(\cdot)$, Vogan packet|textbf}
 $VP(\overline{I}_\delta(\overline{V},\lambda))$
for this Vogan packet.



The representations in the Vogan packet 
 $VP(\overline{I}_\delta(\overline{V},\lambda))$
 can be described as  follows: 
 we call a real form $SO(\ell,k)$ of $SO(\ell + k,\bC)$ 
\index{B}{pureinnerform@pure inner form|textbf}
{\it{pure}}
 if $\ell$ is even and thus admits discrete series representations.
We consider parabolic subgroups of $SO(n-2p+1,1+2p)$ with Levi subgroups $L$,
 which are pure inner forms of
 $ SO(n) \times GL(1,{\mathbb{R}})$.  
Hence they are isomorphic to 
\[
   L \simeq SO(n-2p,2p) \times GL(1,{\mathbb{R}}).  
\]
The Vogan packet $VP(\overline{I}_\delta(\overline{V},\lambda ))$
 contains the {\bf tempered principal series representations}
 of $SO(n-2p+1,1+2p)$
 which have the same infinitesimal character
 as $\overline{I}_\delta(\overline{V},\lambda)$, 
 and which are induced from the outer tensor product of a discrete series representation of $SO(n-2p,2p)$, with the same infinitesimal character
 as $\overline{V}$  and a one-dimensional representation $\chi_\lambda$ of $GL(1,\mathbb{R})$, 
 \cite{V}.  

We use the same conventions
 for a Vogan packet $VP(\overline{J}_\varepsilon(\overline{W},\nu))$
 of the tempered principal series representation
 $\overline{J}_\varepsilon(\overline{W},\nu)$ of $\overline{G'}$.


%%%%%%%%%%%%%%%%%%%%%%%%%
\subsection{Vogan packets of discrete series representations with integral infinitesimal character of $SO(2m,1)$} 
%%%%%%%%%%%%%%%%%%%%%%%%

We begin with the case
 $n=2m-1$.  
In this case $SO(n+1,1)=SO(2m,1)$
 has discrete series representations.  
We fix a set of positive roots 
$
   \Delta^+ \subset {\mathfrak{t}}_{\mathbb{C}}^{\ast}
$
 for the root system 
 $\Delta({\mathfrak{so}}(2m+1,\bC),{\mathfrak{t}}_{\mathbb{C}})$
 and denote by $\rho$ half the sum of positive roots as before. 
Let $\eta $ be an integral infinitesimal character, 
 which is dominant with respect to  $\Delta  ^+$.
For $\ell+k=2m+1$, 
 we call a real form $SO(\ell,k)$ {\it{pure}}
 if $\ell$ is even.
The Vogan packet containing the discrete series representation with infinitesimal character $\eta$ is the disjoint union of discrete series representations with infinitesimal character $\eta$ of the pure inner forms. 
The cardinality of this packet is 
\[
   2^m=\sum_{\substack{0 \le \ell \le 2m \\ \ell:\text{even}}} 
({m \atop{\frac \ell 2}}).  
\]
There exists a finite group 
$
  {\mathcal  A}_2 \simeq (\bZ/2\bZ)^m$  whose characters parametrize the representations in the Vogan packet. For the discrete series representation with parameter $\chi \in \widehat{{\mathcal{A}}_2}$
 we write $\overline{\pi}(\chi). $ For more details
 see \cite{GP} or \cite{V}. 
If $\overline{\pi}$ is a discrete series representation of $SO(2m,1)$ we write $VP(\overline{\pi})$ for the Vogan packet containing $\overline{\pi}$.
  

\begin{example}
Suppose that $\overline{\pi }$ is a discrete series representation
 of $SO(2m,1)$
 with trivial infinitesimal character $\rho$. 
{\rm{
\begin{itemize}
\item[{\rm{(1)}}] The trivial one-dimensional representation ${\bf{1}}$
 of the inner form $SO(0,2m+1)$ is in $VP(\overline{\pi })$.

\item[{\rm{(2)}}] We can define similarly a Vogan packet $VP(\overline{\pi })$ containing $(SO(1,2m),\overline{\pi })$.  
\end{itemize}
}}
\end{example}

%%%%%%%%%%%%%%%%%%%%%%%%%
\subsection{Embedding the group 
$\overline{G'}=SO(n-2p,2p+1)$ into the group 
 $\overline{G}=SO(n-2p+1,2p+1)$} 

To formulate the Gross--Prasad conjecture
 we have to fix an embedding of $\overline{G'}$ into $\overline{G}$. 

\medskip

We observe:
\begin{enumerate}
\item
[(1)] The quasisplit forms of the odd special orthogonal group are 
$SO(m,m+1)$ and $SO(m+1,m)$. 
The 
\index{B}{pureinnerform@pure inner form}
 pure inner forms in the same class
 as $SO(m,m+1)$
 are $SO(m-2p, m+2p+1)$ 
 and those in the same class as $SO(m+1,m)$ are  $SO(m+1-2p, m+2p)$.

\item
[(2)]  
The quasisplit forms 
 of the even special orthogonal group are $SO(m,m)$, 
 $SO(m-1,m+1)$, and $SO(m+1,m-1)$. 
The pure inner forms are $SO(n-2p, n+2p)$ and  $SO(m+1-2p, m-1 -2p)$,
respectively,
 with $p \leq \frac m 2$.
 \end{enumerate}

So
\begin{enumerate}
\item if $n=2m$, then
 the orthogonal group $SO(2m+1,1)$ is a pure inner form of $SO(m+1,m+1)$
 if $m$ is even and of $SO(m+2,m)$ if $m$ is odd; 
\item if $n= 2m-1$, then the orthogonal group $SO(2m,1)$ is a pure inner form of $SO(m+1,m)$ if $m$ is odd and of $SO(m,m+1)$ if $m$ is even.
\end{enumerate}

\medskip
We  consider an  indefinite quadric form 
\[ 
Q_{n-2p+1,2p+1}(x)
=x^2_1+ \dots + x^2_{n-2p+1} -x^2_{n-2p+2}-  \dots -x^2_{n+2}
\]
 of signature $(n-2p +1, 2p+1)$.
We assume that $n-2p +1> 0$ and identify
 $SO(n-2p,2p+1)$ with the subgroup of $SO(n-2p+1,2p+1)$ which stabilizes 
 the basis vector $e_{n-2p+1}$.  
This  allows us to  identify the Levi subgroup of the maximal parabolic subgroup of $SO(n-2p,2p+1)$ with the intersection of the corresponding maximal parabolic subgroup of $\overline{G}$. 
This embedding of $SO(n,1)$ into $SO(n+1,1)$  is conjugate to the one
 we consider in Section \ref{subsec:Xi}.  
We use this embedding in the  formulation of the Gross--Prasad conjectures.  



For tempered principal series representations we 
 consider symmetry breaking operators, 
 namely, 
 $SO(n-2p,2p+1)$-homomorphisms from representations
 in 
 $VP(\overline{I}_\delta(\overline{V},\lambda))$
 to representations
 in $VP(\overline{J}_\varepsilon(\overline{W},\nu))$, 
 see Section \ref{subsec:IV.4}.  



If the tempered representation
 of $\overline{G}$ or of $\overline{G'}$ is
 a discrete series representation, 
 we consider symmetry breaking from
 a Vogan packet of discrete series representations
 to a Vogan packet of tempered principal series representations,
 respectively from a Vogan packet
 of tempered principal series representations
 to a Vogan packet of discrete series representations
 (Section \ref{subsec:GPII}). 




%%%%%%%%%%%%%%%%%%%%%
\subsection{The Gross--Prasad conjecture I: Tempered principal series  representations}
\label{subsec:IV.4}
%%%%%%%%%%%%%%%%%%%%%

By Theorem \ref{thm:tempVW},
 there is a nontrivial  symmetry breaking operator
 between the {\bf{tempered}} principal representations ${I}_\delta(V,\lambda)$
 of $G=O(n+1,1)$
 and ${J}_\varepsilon(W,\nu)$ of $G'=O(n,1)$ 
 if and only if $(\sigma,V) \in \widehat{O(n)}$
 and $(\tau,W) \in \widehat {O(n-1)}$ satisfy 
\[
  [V:W]=\dim_{\mathbb{C}}{\operatorname{Hom}}_{O(n-1)}
                          (V|_{O(n-1)},W) \not = 0.  
\]
An analogous result holds 
 for a pair of the {\it{special}} orthogonal groups 
 $(\overline G, \overline {G'})=(SO(n+1,1),SO(n,1))$.  
We set
\[
   [\overline{V}: \overline{W}]
   \equiv
   [\overline{V}|_{SO(n-1)}: \overline{W}]
  :=
  \dim_{\mathbb{C}}
  {\operatorname{Hom}}_{SO(n-1)}
  (\overline{V}|_{SO(n-1)},\overline{W}).  
\]
In Theorem \ref{thm:tempSO} in Appendix II 
 we prove:
\begin{theorem}
There is a nontrivial  symmetry breaking operator
 between the {\bf{tempered}} principal series representations
 $\overline{I}_\delta(\overline{V,}\lambda)$
 of $\overline{G}=SO(n+1,1)$
 and $\overline{J}_\varepsilon(\overline{W},\nu)$ of $\overline{G'}=O(n,1)$ 
 if and only if $(\overline \sigma,\overline{V}) \in \widehat{SO(n)}$
 and $(\overline \tau,\overline{W}) \in \widehat {SO(n-1)}$ satisfy 
\[
   [\overline{V}|_{SO(n-1)}: \overline{W}] \not = 0.    
\]
\end{theorem}



In their article B.~Gross and D.~Prasad presented a conjectured algorithm
 to determine the pair of representations in the Vogan packets
 $VP(\overline{I}_\delta(\overline{V},\lambda))$
 and  $VP(\overline{J}_\varepsilon(\overline{W},\nu))$
 with a nontrivial $SO(n,1)$-symmetry breaking operator. We  prove next that  the algorithm in fact predicts :
\[ 
   [\overline{V}|_{SO(n-1)}: \overline{W}] \not = 0 
\quad
\text{ if and only if }
\quad
   \mbox{Hom}_{\overline{G'}} 
  (\overline{I}_\delta(\overline{V},\lambda)|_{\overline{G'}},
   \overline{J}_\varepsilon(\overline{W},\nu)) \not = \{0\}.
\]


\medskip
\begin{observation}
A Levi subgroup $L$ with
$[L,L] = SO(r,s)$ of the maximal parabolic subgroup determines the class
 of pure inner forms of $SO(r+1,s+1)$. 
So for any algorithm to determine the pair $(SO(r+1,s), SO(r,s))$ of the groups in the Gross--Prasad conjectures it is enough to  determine the pair  of the Levi subgroups and their corresponding discrete series representations.
\end{observation}
\medskip

\noindent
{\bf First case:}
\enspace 
{\it Suppose that } $(\overline{G},\overline{G'}) = (SO(2m+1,1), SO(2m,1))$.
\\
Let $T_\bC$ be a torus in $SO(2m+2,\bC)  \times SO(2m+1,\bC)$, 
 and $X^{\ast}(T_\bC)$ the character group. 
Fix a basis 
\[
   X^*(T_{\mathbb{C}})
   = 
   \bZ e_1\oplus \bZ e_2 \oplus \dots \oplus \bZ {{e_{m+1}}} \oplus \bZ f_1\oplus \bZ f_2 \oplus \dots \oplus \bZ f_m  
\]
such that the standard root basis $\Delta_0$ is given by
\[
e_1-e_2, e_2-e_3,\dots,{{e_{m}-e_{m+1},e_{m}+e_{m+1}}}, f_1-f_2 ,f_2-f_3, \dots, f_{m-1}-f_m,f_m
\]
{if $m \ge 1$}. 

We fix $\delta, \varepsilon \in \{\pm\}$ as in Section \ref{subsec:Vpacket}.

Recall that all representations in a Vogan packet have
 the same Langlands parameter. 
We identify the Langlands parameter of the representations
 in the same Vogan packet as
\[
   (SO(2m+1,1)\times SO(2m,1), 
   \overline{I}_\delta(\overline{V}, \lambda) \boxtimes \overline{J}_\varepsilon(\overline{W},\nu))
\]
 for a pair $(\overline{V},\overline{W})$ of irreducible finite-dimensional representations with infinitesimal character
\begin{eqnarray*}
 \lefteqn{ ( v_1+m-1)e_1+(v_2+m-2)e_2
+ \dots 
+(v_m)e_m - (\lambda -m) e_{m+1}}
\\
&  & +(u_1+m-\frac 3 2 )f_1+
 (u_2+m- \frac 5 2 ) f_2 +\dots + (u_{m-1}+\frac 1 2)f_{m-1}
-(\nu - m+\frac 1 2 ) f_m, 
\end{eqnarray*}
see \eqref{eqn:ZGinfI}.  
Here $(v_1,v_2, \dots, v_m)$ is the highest weight
 of the $SO(2m)$-module $\overline{V}$,
 $(u_1,u_2,\dots, u_{m-1}) $ is the highest weight
 of the $SO(2m-1)$-module $\overline{W}$
 and the continuous parameter $\lambda-m$ and $\nu-m+\frac 12$
 are purely imaginary, 
 and thus $\overline I_{\delta}(\overline V, \lambda)$
 and $\overline J_{\varepsilon}(\overline W, \nu)$
 are (smooth) tempered principal series representations
 of $\overline G$ and $\overline{G'}$, 
 respectively.  



As discussed before, 
 to determine the pair 
\[
     (SO(n-2p+1,2p+1),SO(n-2p,2p-1))
\]
 it suffices to solve this problem for the Levi subgroups. 
Hence it suffices to  consider the Langlands parameter
\begin{eqnarray*}
 \lefteqn{ ( v_1+m-2)e_1+(v_2+m-3)e_2+ \dots +(v_m)e_m }\\
&  & +(u_1+m-\frac 5 2 )f_1+
 (u_2+m- \frac 7 2 ) f_2 +\dots + (u_{m-1} + \frac 12) f_{m-1} .
\end{eqnarray*}



Let $\delta_i$ be the element which is $-1$ in the $i$-th factor
 of ${\mathcal{A}}_1$
 and equal to $1$ everywhere else, 
 and $\varepsilon_j$ the element which is $-1$
 in the $j$-th factor of ${\mathcal{A}}_2$ and $1$ everywhere else.  
Then the algorithm \cite[p.~993]{GP} determines
 $\chi_1 \in \widehat {{\mathcal{A}}_1}$
 and $\chi_2 \in \widehat {{\mathcal{A}}_2}$
 by 
\[
 \chi_1(\delta_i)= (-1)^{\#m-i+1>}
 \quad
 \mbox{ and }
 \quad 
 \chi_2(\varepsilon_j)=  (-1)^{\#m-j + \frac 1 2<}, 
\]
where $\#m-i+1>$ is the cardinality
 of the set 
\[
\{j: v_j+m-i> \text{the coefficients of $f_j$}\},
\]
 and $\# m-j+\frac 1 2 <$ is the cardinality
 of the set \[\{i:v_i +m-j-1+\frac 1 2 < \text{the coefficients of $e_i$}\}.  \]


If $\mbox{Hom}_{SO(n-1)} (\overline{V}|_{SO(n-1)},\overline{W}) \ne \{0\}$, 
 then $v_1 \leq u_1 \leq v_2 \leq \dots \leq u_{m-1} \leq |v_m|$.
Hence we deduce that both characters are  alternating characters
 if and only if 
$\mbox{Hom}_{SO(n-1)} (\overline{V}|_{SO(n-1)},  \overline{W}) \not = \{0\}$.  

\vskip 2pc
\noindent
{\bf Second case:}
\enspace 
{\it Suppose that } $(\overline{G},\overline{G'}) = (SO(2m,1), SO(2m-1,1))$.



We use the same arguments 
 for the pair 
\[
   (\overline G, \overline{G'})=(SO(2m,1), SO(2m-1,1)).
\]

\medskip
We normalize the quasisplit forms  by 
\begin{alignat*}{2}
 &SO(m+1,m) \times SO(m,m) \quad &&\mbox{if $m$ is even,} 
\\
 &SO(m,m+1) \times SO(m-1, m+1)  \quad &&\mbox{if $m$ is odd.} 
\end{alignat*}
Applying the formul{\ae} in \cite[(12.21)]{GP}, 
 we define the integers $p$ and $q$
 with $0 \leq p \leq m $ and $0 \leq q \leq m$
by
\[
 p = \# \{ i : \chi_1(    \delta_i  )  = (-1)^i  \}  \quad \mbox{ and }  
\quad  q  =  \# \{ j:  \chi_2(   \varepsilon_j     )  = (-1)^{m+j}  \}, 
\]
and we get the pure forms
\begin{alignat*}{2}
&SO(2m-2p+1, 2p) \times SO(2q ,2m-2q) \quad &&\mbox{if $m$ is even,}
\\
&SO(2p+1,2m-2p+1) \times SO(2m-2q, 2q+1) \quad &&\mbox{if $m$ is odd.}
\end{alignat*}
In our setting,  we get the pair of integers $(p,q)=(0,m)$ for $m$ even; $(p,q)=(m,0)$ for $m$ odd.
\medskip
Applying \cite[(12.22)]{GP} with correction by changing $n$ by $m$ 
 {\it{loc.~cit.}},
 we deduce
 that the alternating character $\chi$ defines
 the pure inner form 
\begin{equation*}
SO(2m+1,0) \times SO(2m,0)
\qquad
\mbox{for $m$ is even and odd.}
\end{equation*}  
Hence 
\[
   \overline G=SO(2m,1) \mbox{ and } \overline{G'}=SO(2m-1,1).   
\]
The only representation in $VP(\overline{I}_\delta(\overline{V,}\lambda))\times VP(\overline{J}_\varepsilon(\overline{W},\nu))$
 for this pair of pure inner forms  is 
\[
   \overline{I}_\delta(\overline{V},\lambda)
   \boxtimes 
   \overline{J}_\varepsilon(\overline{W},\nu). 
\]
If $\chi $ is not the alternating character,
 the calculation shows that  we obtain a different pair of groups.
Thus we can rephrase the conjecture by B.~Gross and D.~Prasad
 as follows:

\begin{conjecture}
[\bf Gross--Prasad conjecture I]
\label{conj:GPA}
Suppose that $\overline{I}_\delta(\overline{V},\lambda)\boxtimes \overline{J}_\varepsilon(\overline{W},\nu)$ are tempered principal series representations
 of $SO(n+1,1) \times SO(n,1)$.  
Then    
\[ 
    {\operatorname{Hom}}_{SO(n,1)}(\overline{I}_\delta(\overline{V},\lambda)
\boxtimes \overline{J}_\varepsilon(\overline{W},\nu),{\mathbb{C}})=\bC
\]
if and only if $\overline{V } \in \widehat{SO(n)}$ and $\overline{W} \in \widehat{SO(n-1)}$
 satisfies 
\[ 
   [\overline{V}|_{SO(n-1)}:\overline{W}] \ne 0.
\]
\end{conjecture}

\begin{theorem}
[see Theorem \ref{thm:tempSO}]
\label{theorem:GPtemp}
The Gross--Prasad conjecture I holds.  
\end{theorem}
We can deduce Theorem \ref{theorem:GPtemp} from the corresponding results
 (Theorem \ref{thm:tempVW}) for the orthogonal groups
 $O(n+1,1) \times O(n,1)$
 by using results
 about the reduction from $O(N,1)$
 to the special orthogonal group $SO(N,1)$.  
See the proof of Theorem \ref{thm:tempSO}
 in Section \ref{subsec:SOtemp}
 of Appendix II
 for details.  



\subsection{The Gross--Prasad conjecture II: 
 Tempered representations with trivial infinitesimal character
$\rho$}
\label{subsec:GPII}


For completeness,
 we include the discussion of the Gross--Prasad conjectures
 for tempered representations with trivial infinitesimal character $\rho$
 which we also discussed in detail in \cite{sbonGP}.  
\medskip

We modify here the notation from \cite{sbonGP} by denoting the restriction of a representation $\Pi$ of $O(n+1,1)$ to the subgroup $SO(n+1,1)$ by $\overline{\Pi}$.

The Gross--Prasad conjecture I in the previous section 
treated the case
 where both $\overline \Pi$ and $\overline \pi$ are tempered 
 {\it{principal series representations}}
 of the group $G=SO(n+1,1)$
 and $G'=SO(n,1)$, 
 respectively.  



Thus the remaining cases are
 when $\overline \Pi$ or $\overline \pi$ are 
 {\it{discrete series representations}}.  
We note 
 that both $\overline \Pi$ and $\overline \pi$ cannot be discrete series representations
 in our setting
 because $\overline G$ admit discrete series representations
 if and only if $n$ is odd and $\overline{G'}$ admit those 
 if and only if $n$ is even.  
Thus we discuss the Gross--Prasad conjecture
 in this case
 separately depending on the parity of $n$, 
 with the following notation.  



Consider symmetry breaking operators
 for tempered representations
 with trivial infinitesimal character $\rho$
 of the group $SO(n+1,1)$
 for $n=2m$, $2m-1$, 
 and $2m-2$.  
We denote the corresponding representations
 by $\Pi$, $\pi$, and $\varpi$, 
 respectively,
 using the subscripts
 defined in Section \ref{subsec:PiSO} in Appendix II.  
We thus consider symmetry breaking from $SO(2m+1,1)$ to $SO(2m,1)$
 and further to $SO(2m-1,1)$:



\begin{equation*} 
\begin{tabular}{ccccc}
{$\overline{\Pi}_{m,(-1)^{m+1}}$}& $ \rightarrow   $   & $\overline{ \pi}_m     $      &    $   \rightarrow      $         &   {$\overline{ \varpi }_{m-1,(-1)^m}$}. \\
 &   &   &  &
\end{tabular}
\end{equation*}

Here $\overline{\Pi}_{m,(-1)^{m+1}}$ and $\overline{\varpi}_{m-1,(-1)^m}$ are tempered principal series representations
 which are nontrivial on the center of $SO(2m+1,1)$,  respectively
 $SO(2m-1,1)$, 
 and thus are genuine representations
 of the special orthogonal groups,
 see Proposition \ref{prop:161648} (6).  
Since $\overline{\pi}_{m,+} \simeq \overline{\pi}_{m,-}$ 
 as $SO(2m,1)$-modules,
 we simply write $\overline{\pi}_{m}$
 for $\overline{\pi}_{m,\pm}$, 
 which is a discrete series representation of $SO(2m,1)$. All representations have the trivial infinitesimal  character $\rho$.

\subsubsection{The Gross--Prasad conjecture II: 
Symmetry breaking from 
 $\overline{\Pi}_{m,(-1)^{m+1}}$
 to the discrete series representation $\overline \pi_m$}

We consider first the Vogan packet of tempered representations which contains the pair $(SO(2m+1,1) \times SO(2m,1), \overline{\Pi }_{m,\delta} \boxtimes \overline{\pi}_m )$ 
 or the Vogan packet which contains the pair
$(SO(1,1+2m)\times SO(1,2m), \overline{ \Pi}_{m,\delta} \boxtimes \overline{ \pi}_m )$. 
The representations in these packets are parametrized by characters of
\[
     {\mathcal{A}}_1 \times {\mathcal{A}}_2 
     \simeq 
     (\bZ/2\bZ)^{\color{black}{m}} \times (\bZ/2\bZ)^m \simeq (\bZ/2\bZ)^{2m} . 
\] 
We recall the algorithm proposed  by B.~Gross and D.~Prasad 
 which determines a pair 
$
     (\chi_1, \chi_2)
 \in 
     \widehat{{\mathcal{A}}_1}\times \widehat{{\mathcal{A}}_2}
$, 
 hence representations 
\[
   (\overline{\Pi}(\chi_1),\overline{\pi}(\chi_2 )) \in VP(\overline{\Pi}_{m,\delta}) \times VP(\overline{\pi}_m)
\]
 so that 
\[ 
   \mbox{Hom}_{\overline{G}(\chi_2)}(\overline{\Pi}(\chi_1)|_{\overline G(\chi_2)},\overline{\pi}(\chi_2) ) \not = \{0\},
\]
where $\overline{G}(\chi_2) $ is the pure inner form determined by $\chi_2$.

\medskip
Let $T_\bC$ be a torus in $SO(2m+2,\bC)  \times SO(2m+1,\bC)$,
 and $X^{\ast}(T_\bC)$ the character group. 
As before the standard root basis $\Delta_0$ is given by
\[
e_1-e_2, e_2-e_3,\dots,{{e_{m}-e_{m+1},e_{m}+e_{m+1}}}, f_1-f_2 ,f_2-f_3, \dots, f_{m-1}-f_m,f_m
\]
{if $m \ge 1$}.

We fix $\delta=(-1)^{m+1}$
 so that $\overline{\Pi}_{m,\delta}$ is a genuine representation
 of $SO(2m+1,1)$.  
We can identify the Langlands parameter of the Vogan packet containing 
\[
(SO(2m+1,1)\times SO(2m,1),\overline{ \Pi}_{m, {{\delta}}} \boxtimes \overline{ \pi}_m)
\]
with 
 \[ me_1+(m-1)e_2+ \dots +e_m+0e_{m+1} +(m-\frac 1 2 )f_1+(m- \frac 3 2 ) f_2 +\dots + \frac 1 2 f_m.\]
 


Let $\delta_i$ be the character in $ \widehat{{\mathcal{A}}_1}$ which is $-1$ in the $i$-th factor
 of ${\mathcal{A}}_1$
 and equal to $1$ everywhere else, 
 and $\varepsilon_j$ be the character  which is $-1$
 in the $j$-th factor of ${\mathcal{A}}_2$ and $1$ everywhere else.  



Then the algorithm by B.~Gross and D.~Prasad 
\cite[p.~993]{GP} determines characters
 $\chi_1 \in \widehat {{\mathcal{A}}_1}$
 and $\chi_2 \in \widehat {{\mathcal{A}}_2}$
 by 
\[
 \chi_1(\delta_i)= (-1)^{\#m-i+1>}  \ \ \ \ 
 \mbox{and}  \ \ \ \  
 \chi_2(\varepsilon_j)=  (-1)^{\#m-j + \frac 1 2<}, 
\]
where $\#m-i+1>$ is the cardinality
 of the set 
\[
\{j: m-i+1> \text{the coefficients of $f_j$}\},
\]
 and $\# m-j+\frac 1 2 <$ is the cardinality
 of the set 
\[
\{i:m-j+\frac 1 2 < \text{the coefficients of $e_i$}\}.  
\]



As discussed before we normalize the quasisplit form by 
\begin{align*}
 SO(m+1,m+1) \times SO(m,m+1) \quad &\mbox{if $m$ is even,} 
\\
 SO(m+2,m) \times SO(m+1, m)  \quad &\mbox{if $m$ is odd.} 
\end{align*}
Applying the formul{\ae} in \cite[(12.21)]{GP}
 we define the integers $p$ and $q$
 with $0 \leq p \leq m $ and $0 \leq q \leq m$
by
\[
 p  =  \# \{ i : \chi_1(    \delta_i  )  = (-1)^i  \}  \quad \mbox{ and }  
\quad  q  =  \# \{ j:  \chi_2(   \varepsilon_j     )  = (-1)^{m+j}  \}
\]
and we get the pure forms
\begin{align}
SO(2m-2p+1, 2p+1) \times SO(2q ,2m-2q+1) \quad &\mbox{if $m$ is even,}
\\
SO(2p+1,2m-2p+1) \times SO(2m-2q, 2q+1) \quad &\mbox{if $m$ is odd.}
\end{align}
In our setting,  we get the pair of integers $(p,q)=(0,m)$ for $m$ even; $(p,q)=(m,0)$ for $m$ odd.
Applying \cite[(12.22)]{GP}
 with correction by changing $n$ by $m$ 
 {\it{loc.cit.}}, 
 we deduce that this character defines the pure inner form 
\[
 \text{$SO(2m+1,1) \times SO(2m,1)$ for $m$ even and odd.}
\]  

The only representation in $VP(\overline{\Pi}_{m,\delta} )\times VP(\overline{\pi}_m)$ for this pair of pure inner forms  is $\overline{\Pi}_{m,\delta} \boxtimes \overline{\pi}_m$.
Hence Theorem \ref{thm:170336} implies the Gross--Prasad conjecture
 in that case.



\subsubsection{The Gross--Prasad conjecture II: 
Symmetry breaking from the discrete series representation 
 $\pi_m$
 to $\varpi_{m-1,(-1)^m}$}



We now consider the Vogan packet of tempered representations containing the pair $(SO(2m,1) \times SO(2m-1,1), {\overline{\pi}}_{m} \boxtimes \overline{\varpi}_{m-1,(-1)^m} )$, 
 {\it{i.e.}}, 
 the Vogan packet 
\[
VP(\overline{\pi}_m\boxtimes \overline{\varpi}_{m-1,(-1)^m})\subset VP(\overline{\pi}_m) \times VP(\overline{\varpi}_{m-1,(-1)^m}).  
\] 
The packet $VP(\overline{\pi}_m) \times VP(\overline{\varpi }_{m,(-1)^m})$ is
 parametrized by  characters of the finite group  
\[
     {\mathcal{A}}_2 \times {\mathcal{A}}_3
 \simeq 
 (\bZ/2\bZ)^m \times (\bZ/2\bZ)^{m-1}\simeq (\bZ/2\bZ)^{2m-1}.
\]
Again 
 the algorithm by B.~Gross and D.~Prasad  determines a pair 
 $(\chi_2,\chi_3) \in \widehat{{\mathcal{A}}_2} \times \widehat{{\mathcal{A}}_3}$
 and hence representations 
\[
 (\overline{\pi}(\chi_2), \overline{\varpi}(\chi_3 )) \in VP(\overline{\pi}_{m})\times VP(\overline{\varpi}_{m-1,(-1)^m}  )\] so that 
\[ 
     \mbox{Hom}_{\overline{G}(\chi_3)}(\overline{\pi}(\chi_2)|_{\overline{G}(\chi_3)},\overline{\varpi}(\chi_3) ) \not = \{0\}, 
\]
where $\overline{G}(\chi_3) $ is the pure inner form determined by $\chi_3$.

\medskip

 
Let $T_\bC$ be a torus in $SO(2m+1,\bC)\times  SO(2m, \bC)$
 and $X^{\ast}(T_{\mathbb{C}})$ the character group.  
Fix a basis 
\[
  X^{\ast}(T_{\mathbb{C}})
  =
  {\mathbb{Z}}f_1 \oplus {\mathbb{Z}}f_2 \oplus \cdots
   \oplus {\mathbb{Z}}f_m
   \oplus {\mathbb{Z}}g_1 \oplus {\mathbb{Z}} g_2 \oplus
   \cdots
   \oplus {\mathbb{Z}} g_m 
\]
 such that the standard root basis
 $\Delta_0$ is given by 
\[
  f_1 - f_2, f_2 - f_3, \cdots, f_{m-1} - f_m, f_m,
  g_1-g_2, g_2-g_3, \cdots, g_{m-1}-g_m, g_{m-1}+g_m
\] 
for $m \ge 2$.  
Take $\varepsilon=(-1)^{m}$ as before.  



We identify the Langlands parameter of the Vogan packet 
\[
VP(\overline{ \pi}_{m} ) \times VP(\overline{ \varpi }_{m,(-1)^m})
\]
 with 
 \[ (m-\frac 1 2 )f_1+(m- \frac 3 2 ) f_2 +\dots + \frac 1 2 f_m +(m-1)g_1+(m-2)g_2+ \dots +g_{m-1}+0g_{m} .\]



Again applying \cite[Prop.~12.18]{GP}
 we define characters $\chi_2 \in \widehat{{\mathcal{A}}_2} $,
 $\chi_3 \in \widehat{{\mathcal{A}}_3}$ as follows: 
Let $\varepsilon_j \in {\mathcal{A}}_2 \simeq (\bZ/2\bZ)^m$ be
 the element which is $-1$ in the $j$-th factor
 and equal to $1$ everywhere else
 as in Section \ref{subsec:IV.4};
 $\gamma_k \in {\mathcal{A}}_3 \simeq (\bZ/2\bZ)^{m-1}$ the element which is $-1$ in the $k$-th factor and $1$ everywhere else.
Then $\chi_2 \in \widehat{{\mathcal{A}}_2}$ and 
 $\chi_3 \in \widehat{{\mathcal{A}}_3}$ are determined by 
\[ 
 \chi_2(\varepsilon_j)= (-1)^{\#m- j +1/2<}
  \ \ \ \ \mbox{and} \ \ \ \ 
 \chi_3(\gamma_k)=  (-1)^{\#m-k >}, 
\]
where $\#m-j+\frac 1 2 <$ is the cardinality
 of the set
  \[\{k:m-j+\frac 1 2 < \text{the coefficients of $g_k$}\}, 
\]
 and $\# m-k >$ is the cardinality of the set
 \[\{j:m-k > \text{the coefficients of $f_j$}\}.\]



As discussed  we normalize the quasisplit form by 
\begin{alignat*}{2}
 &SO(m+1, m) \times SO(m+1,m-1) \quad &&\text{if $m$ is even,} 
\\
 &SO(m,m+1) \times SO(m , m)    \quad &&\text{if $m$ is odd.} 
\end{alignat*}


 We define the integers $p$ and $q$
 with $0 \leq p \leq m$ and $0 \leq q \leq m-1$
by
\[
 p  =  \# \{ j : \chi_2(    \varepsilon_j  )  = (-1)^j  \}  
\quad \mbox{ and }  
\quad  q  =  \# \{ k :  \chi_3(   \gamma _k      )  = (-1)^{m+k}  \}, 
\]
and we get
\begin{alignat*}{2}
&SO(2m-2p+1, 2p) \times SO(2q+1 ,2m-2q-1) 
\quad 
&&\mbox{if $m$ is even,}
\\
&SO(2p+1,2m-2p) \times SO(2m-2q-1, 2q+1) \quad 
&&\mbox{if $m$ is odd.}
\end{alignat*}

In our setting, the pair of integers 
 $(p,q)$ is given by $(p,q)=(m,0)$ for $m$ even;
 $(p,q)=(0,m-1)$ for $m$ odd.
We  deduce that this character defines the pure inner form 
\[
\text{
$SO(1,2m) \times SO(1,2m-1)$ for $m$ even and odd.  
}
\]


The only representation in $VP(\overline{\pi}_{m}) \times VP(\overline{\varpi}_{m-1,(-1)^m})$ with this pair of pure inner forms 
 is $(\overline{\pi}_{m}, \overline{\varpi}_{m-1,(-1)^m})$.

\medskip

In Chapter \ref{sec:SBOrho}, 
 we have determined
\[
   {\operatorname{Hom}}_{G'}(\Pi \boxtimes \pi, {\mathbb{C}})
\quad
\text{for all $\Pi \in {\operatorname{Irr}}(G)_{\rho}$
 and $\pi \in {\operatorname{Irr}}(G')_{\rho}$}, 
\]
see Theorems \ref{thm:SBOvanish} and \ref{thm:SBOone}
 and also Theorem \ref{thm:SBOBF}
 for orthogonal groups
\[
   G \times G'=O(n+1,1) \times O(n,1), 
\]
{}from which we deduce analogous results about
\[
   {\operatorname{Hom}}_{\overline{G'}}(\overline \Pi \boxtimes \overline \pi, {\mathbb{C}})
\quad
\text{for all $\overline\Pi \in {\operatorname{Irr}}(\overline G)_{\rho}$
 and $\overline \pi \in {\operatorname{Irr}}(\overline{G'})_{\rho}$},
\]
for the special orthogonal groups
\[
   \overline G \times \overline{G'} = SO(n+1,1) \times SO(n,1), 
\]
in Theorem \ref{thm:170336}.  
By the aforementioned argument,
 Theorem \ref{thm:170336} implies the following.  

\begin{theorem}
\label{thm:GPdisctemp}
The conjectures by B.~Gross and D.~Prasad \cite{GP}
 for tempered representations
 of special orthogonal groups $SO(n+1,1) \times SO(n,1)$ with trivial infinitesimal character $\rho$ hold.  
\end{theorem}

\begin{remark}
The Gross--Prasad conjectures concern tempered representations
 with trivial infinitesimal character $\rho$, 
 but one may expect similar results for unitary representations
 of orthogonal groups with integral infinitesimal character. 
Considering \lq\lq{Arthur--Vogan packets}\rq\rq\
 instead of the Vogan packets will include 
 other unitary representations
 which are of interest to number theory
 for example to the representation 
\index{A}{Aqlmd@$A_{\mathfrak{q}}(\lambda)$}
$A_{\mathfrak{q}}(\lambda)$.  
Low dimensional examples  and our results suggest
 that there exists  pairs of groups
 $\overline G\times \overline{G'}=SO(p+1,q) \times SO(p,q)$
 and of representations $ U_1 \boxtimes U_2$
 in this \lq\lq{Arthur--Vogan packet}\rq\rq\ 
 so that 
$
   \mbox{Hom}_{\overline{G'}}(U_1|_{\overline{G'}} \boxtimes U_2, \mathbb{C}) \not = \{0\}. 
$
The examples also suggest an algorithm
 to determine  pairs of groups
 and the pairs of representations with nontrivial multiplicity.
\end{remark}

