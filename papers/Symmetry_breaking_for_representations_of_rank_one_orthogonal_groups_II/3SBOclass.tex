\newpage
\section{Symmetry breaking operators 
 for principal series representations---general theory}
\label{sec:general}

In this chapter we discuss important concepts
 and properties of symmetry breaking operators from 
 principal series representations
 $I_{\delta}(V,\lambda)$ 
 of the orthogonal group $G=O(n+1,1)$
 to $J_{\varepsilon}(W,\nu)$ of the subgroup $G'=O(n,1)$.  
In particular,
 we present a classification scheme
 (Theorem \ref{thm:VWSBO})
 of all symmetry breaking operators,
 which is built on the strategy
 of the classification 
 in the spherical case
 \cite{sbon}
 and also on a new phenomenon 
 for which we refer to as {\it{sporadic operators}}
 (Section \ref{subsec:SBOVW2}).  
The classification scheme is carried out
 in full details
 for symmetry breaking from principal series representations
 $I_{\delta}(V,\lambda)$ of $G$
 to $J_{\varepsilon}(W,\nu)$ of the subgroup $G'$, 
 which will play a crucial role 
 in understanding symmetry breaking 
 of all {\it{irreducible}} admissible representations
 of $G$ 
 having the trivial infinitesimal character
 (Chapters \ref{sec:SBOrho}, \ref{sec:Gross-Prasad}, and \ref{sec:period}).  
Various theorems stated in this chapter 
 will be proved in later chapters,
 in particular,
 in Chapter \ref{sec:section7}.  



\subsection{Generalities}
\label{subsec:SBOgen}
We refer to nontrivial homomorphisms
 in 
\[
  {\operatorname{Hom}}_{G'}
  (I_{\delta}(V,\lambda)|_{G'}, J_{\varepsilon}(W,\nu))
\]
as intertwining restriction operators
 or 
\index{B}{symmetrybreakingoperators@symmetry breaking operators|textbf}
{\it{symmetry breaking operators}}.  
Here $\delta, \varepsilon \in {\mathbb{Z}}/2 {\mathbb{Z}}$
 in our setting
 where $(G,G')=(O(n+1,1), O(n,1))$.  
For a detailed introduction to symmetry breaking operators
 we refer to \cite{xkvogan} and 
 \cite[Chaps.~1 and 3]{sbon}.  



\subsection{Summary of results}
\label{subsec:summarySBO}
We keep our setting
 where $(G,G')=(O(n+1,1), O(n,1))$.  



For $(\sigma, V) \in \widehat {O(n)}$, 
 $\delta \in \{\pm\}$, 
 and $\lambda \in {\mathbb{C}}$, 
 we write $I_{\delta}(V, \lambda)$
 for the principal series representation
 of $G$ as in Section \ref{subsec:ps}.  
Similarly, 
 let $(\tau, W)$ be an irreducible representation of $O(n-1)$, 
 $\varepsilon \in \{\pm \}$, 
 and $\nu \in{\mathbb{C}}$.  
We extend the outer tensor product representation
\index{A}{Wnuepsi@$W_{\nu,\varepsilon}=W \otimes \varepsilon \otimes {\mathbb{C}}_{\nu}$, representation of $P'$|textbf}
\[
 W_{\nu,\varepsilon}
:=W \boxtimes {\varepsilon} \boxtimes{\mathbb{C}}_{\nu}
\]
of the direct product group $M' A \simeq O(n-1) \times O(1) \times {\mathbb{R}}
$ to $P' = M' A N_+'$
 by letting $N_+'$ act trivially.  
We also write $W_{\nu,\varepsilon}=W \otimes \varepsilon \otimes {\mathbb{C}}_{\nu}$
 when we regard it as a $P'$-module.  
We form a $G'$-equivariant vector bundle
\index{A}{Wnuepsical@${\mathcal{W}}_{\nu,\varepsilon}$, homogeneous vector bundle over $G'/P'$|textbf}
$
  {\mathcal{W}}_{\nu,\varepsilon}
  :=
  G' \times_{P'} W_{\nu,\varepsilon}
$
 over the real flag manifold $G'/P'$.  
The principal series representation 
\index{A}{JWepsilon@$J_{\varepsilon}(W, \nu)$|textbf}
 $J_{\varepsilon}(W, \nu)$
 of $G'=O(n,1)$
 is defined to be the induced representation
 $\operatorname{Ind}_{P'}^{G'}(W_{\nu,\varepsilon})$
 on the space 
 $C^{\infty}(G'/P',{\mathcal{W}}_{\nu,\varepsilon})$
 of smooth sections for the vector bundle.  



For $(\sigma, V) \in \widehat {O(n)}$
 and $(\tau, W) \in \widehat {O(n-1)}$, 
 we set 
\index{A}{VWmult@$[V:W]$|textbf}
\begin{equation}
\label{eqn:multVW}
    [V:W]:= \dim_{\mathbb{C}} \operatorname{Hom}_{O(n-1)}(V|_{O(n-1)}, W).  
\end{equation}
If we want to emphasize the subgroup,
 we also write $[V|_{O(n-1)}:W]$ for $[V:W]$.  
We recall from Fact \ref{fact:ONbranch}
 on the classical branching rule for the restriction
 $O(N)\downarrow O(N-1)$
 that the multiplicity $[V:W]$ is either $0$ or $1$.  



\subsubsection{Symmetry breaking operators
 when $[V:W] \ne 0$}
\label{subsec:SBOVW1}
Suppose $[V:W] \ne 0$.  
In this case 
 we prove the existence of nonzero symmetry breaking operators
 for all $\lambda, \nu \in {\mathbb{C}}$
 and for all signatures $\delta$, $\varepsilon \in \{ \pm \}$:
\begin{theorem}
[existence of symmetry breaking operators, 
 see Theorem {\ref{thm:existSBO}}]
\label{thm:160150}
Suppose $(\sigma,V)\in \widehat{O(n)}$ and $(\tau,W)\in \widehat{O(n-1)}$.  
Assume $[V:W] \ne 0$.  
Then we have
\index{A}{IdeltaV@$I_{\delta}(V, \lambda)$}
\[
  \dim_{\mathbb{C}} \operatorname{Hom}_{G'}(I_{\delta}(V, \lambda)|_{G'}, J_{\varepsilon}(W, \nu))
  \ge 1
  \quad
  \text{for all }
  \delta, \varepsilon \in {\mathbb{Z}}/2 {\mathbb{Z}}, 
  \,
  \lambda, \nu \in {\mathbb{C}}.  
\]
\end{theorem}
%
Theorem \ref{thm:160150} is proved in Section \ref{subsec:existSBO}
 by constructing symmetry breaking operators:
 generic ones are nonlocal 
 ({\it{e.g.}} integral operators)
 see Theorem \ref{thm:regexist} below, 
 whereas a few are local operators
 ({\it{i.e.}} differential operators, 
 see Theorem \ref{thm:1716110}).  



\begin{definition}
\label{def:generic}
We say that the quadruple $(\lambda,\nu,\delta,\varepsilon)$
 is a 
\index{B}{genericparametercondition@generic parameter condition|textbf}
 {\it{generic parameter}} if
$(\lambda, \nu)\in {\mathbb{C}}^2$ and $\delta, \varepsilon 
 \in \{\pm\}$ satisfy
\begin{equation}
\label{eqn:nlgen}
\begin{cases}
\nu-\lambda \not \in 2{\mathbb{N}}
\quad
&
\text{when}
\quad
\delta \varepsilon =+;
\\
\nu-\lambda \not \in 2{\mathbb{N}}+1
\quad
&
\text{when}
\quad
\delta \varepsilon =-.  
\end{cases}
\end{equation}
\end{definition}

We recall from \eqref{eqn:singset}
 that the set of \lq\lq{special parameters}\rq\rq\ is given 
 as the complement of \lq\lq{generic parameters}\rq\rq, 
 namely,
\begin{alignat}{2}
  \Psising
  =
  \left\{(\lambda,\nu, \delta, \varepsilon) \in {\mathbb{C}}^2 \times \{\pm\}^2
   :\,\,\right.
&   \nu-\lambda \in 2 {\mathbb{N}} 
&&\text{when $\delta \varepsilon =+$}
\notag
\\
   \text{ or }\,\,
&  \nu-\lambda \in 2 {\mathbb{N}}+1
\quad 
&&
\text{when $\delta \varepsilon =-$}
\left.\right\}.  
\label{eqn:Psisp}
\end{alignat}


In the case $[V:W]\ne 0$, 
 we also prove the following 
\lq\lq{generic multiplicity-one theorem}\rq\rq, 
which extends \cite[Thm.\ 1.1]{sbon}
 in the scalar case 
($V=W={\mathbb{C}}$).  



\begin{theorem}
[generic multiplicity-one theorem]
\label{thm:unique}
Suppose $(\sigma, V) \in \widehat{O(n)}$, 
$(\tau,W) \in \widehat{O(n-1)}$
 with $[V:W] \ne 0$.  
If $(\lambda, \nu, \delta, \varepsilon) \in {\mathbb{C}}^2 \times \{\pm\}^2$
 satisfies the generic parameter condition, 
 namely,
 $(\lambda, \nu, \delta, \varepsilon) \not \in \Psising$, 
 then 
\index{A}{IdeltaV@$I_{\delta}(V, \lambda)$}
\index{A}{JWepsilon@$J_{\varepsilon}(W, \nu)$}
\[
\dim_{\mathbb{C}} \operatorname{Hom}_{G'}(I_{\delta}(V, \lambda)|_{G'}, J_{\varepsilon}(W, \nu))
  =1.  
\]
\end{theorem}



Theorem \ref{thm:unique} gives a stronger estimate 
 than what the existing general theory guarantees:
\begin{enumerate}
\item[$\bullet$]
the dimension $\le 1$
 if both $I_{\delta}(V, \lambda)$ and $J_{\varepsilon}(W, \nu)$ are 
irreducible \cite{SunZhu}, 
\item[$\bullet$]
the dimension is uniformly bounded 
 with respect to $\sigma$, $\tau$, 
 $\delta$, $\varepsilon$, 
 $\lambda$, $\nu$
 \cite{xKOfm}.  
\end{enumerate}
We note
 that $I_{\delta}(V,\lambda)$ or $J_{\varepsilon}(W,\nu)$
 can be reducible
 even if $(\lambda, \nu, \delta, \varepsilon) \in \Psising$.  
Theorem \ref{thm:unique} will be proved
 in a strengthened form by giving an explicit generator 
 (see Theorem \ref{thm:genbasis}
 in Section \ref{subsec:existSBO}).  

\subsubsection{Differential symmetry breaking operators
 when $[V:W] \ne 0$}
\label{subsec:DSBOVW}

We realize the principal series representations
 $I_{\delta}(V,\lambda)$ and $J_{\varepsilon}(W,\nu)$
 in the Fr{\'e}chet spaces
 $C^{\infty}(G/P, {\mathcal{V}}_{\lambda,\delta})$
 and 
 $C^{\infty}(G'/P', {\mathcal{W}}_{\nu,\varepsilon})$.  

\begin{definition}
[differential symmetry breaking operator]
\label{def:DSBO}
A linear map 
\[
   D \colon C^{\infty}(G/P, {\mathcal{V}}_{\lambda,\delta})
     \to 
            C^{\infty}(G'/P', {\mathcal{W}}_{\nu,\varepsilon})
\]
 is called a 
\index{B}{differentialsymmetrybreakingoperators@differential symmetry breaking operator|textbf}
{\it{differential symmetry breaking operator}}
 if $D$ is a differential operator
 with respect to the inclusion $G'/P' \hookrightarrow G/P$
 and $D$ intertwines the action of the subgroup $G'$.  
See Definition \ref{def:diff}
 in Chapter \ref{sec:DSVO}
 for the notion of differential operators
 between two different manifolds.  
We denote by 
\index{A}{DiffG@${\operatorname{Diff}}_{G'}(I_{\delta}(V,\lambda)\vert_{G'}, J_{\varepsilon}(W,\nu))$|textbf}
\[
{\operatorname{Diff}}_{G'}
   (I_{\delta}(V,\lambda)|_{G'}, J_{\varepsilon}(W,\nu))
\]
 the subspace of 
 ${\operatorname{Hom}}_{G'}
   (I_{\delta}(V,\lambda)|_{G'}, J_{\varepsilon}(W,\nu))$
 consisting of differential symmetry breaking operators.  
\end{definition}

We retain the assumption 
 that $[V:W] \ne 0$.  
We give a necessary and sufficient condition
 for the existence of nonzero differential symmetry breaking operators:
\begin{theorem}
[existence of differential symmetry breaking operators]
\label{thm:1716110}
Suppose $(\sigma, V) \in \widehat{O(n)}$
 and $(\tau,W) \in \widehat{O(n-1)}$
 satisfy $[V:W] \ne 0$. 
Then the following two conditions
 on the parameters $\lambda, \nu \in {\mathbb{C}}$
 and $\delta, \varepsilon \in \{\pm\}$
 are equivalent:
\begin{enumerate}
\item[{\rm{(i)}}]
The quadruple $(\lambda, \nu, \delta, \varepsilon)$ does not satisfy
 the generic parameter condition \eqref{eqn:nlgen}, 
 namely,
 $(\lambda, \nu, \delta, \varepsilon)\in \Psising$.  
\item[{\rm{(ii)}}]
${\operatorname{Diff}}_{G'}
   (I_{\delta}(V,\lambda)|_{G'}, J_{\varepsilon}(W,\nu)) \ne \{0\}.  
$
\end{enumerate}
\end{theorem}

We shall prove Theorem \ref{thm:1716110}
 in Chapter \ref{sec:DSVO},
 see Theorem \ref{thm:existDSBO}.  

\vskip 1pc
\subsubsection{Sporadic symmetry breaking operators
 when $[V:W]=0$}
\label{subsec:SBOVW2}
This section treats the case $[V:W]=0$.  
In the holomorphic setting,
 we found in \cite{KP1} a phenomenon 
 that all symmetry breaking operators
 are given by differential operators
 ({\it{localness theorem}}).  
This phenomenon does not occur 
 in the real setting
 if both $V$ and $W$ are the trivial one-dimensional representations
 \cite{sbon}.  
However,
 we shall see that this phenomenon may occur
 in the real setting for vector bundles.  
Indeed, 
 the following theorem shows that 
 there may exist 
\index{B}{sporadicsymmetrybreakingoperator@sporadic symmetry breaking operator|textbf}
{\it{sporadic}} symmetry breaking operators
 which are differential operators
 in the case $[V:W]=0$:
\begin{theorem}
[localness theorem]
\label{thm:152347}
\index{B}{localnesstheorem@localness theorem|textbf}
Assume $[V:W] = 0$.  
Then 
\[
{\operatorname{Hom}}_{G'}(I_{\delta}(V,\lambda)|_{G'}, J_{\varepsilon}(W, \nu))
  =
{\operatorname{Diff}}_{G'}(I_{\delta}(V,\lambda)|_{G'}, J_{\varepsilon}(W, \nu))
\]
for all $(\lambda,\nu,\delta,\varepsilon) \in {\mathbb{C}}^2 \times \{\pm\}^2$, 
 that is, 
any symmetry breaking operator
 (if exists)
\[
   C^{\infty}(G/P, {\mathcal{V}}_{\lambda,\delta})
   \to 
   C^{\infty}(G'/P', {\mathcal{W}}_{\nu,\varepsilon})
\]
is a differential operator.  
\end{theorem}



Theorem \ref{thm:152347} is proved
 in Section \ref{subsec:SDone}.  
We call such operators {\it{sporadic}}
 because there is no regular symmetry breaking operator
 if $[V:W]=0$, 
 see Theorem \ref{thm:regexist} below.  
Another localness theorem is formulated 
 in Theorem \ref{thm:VWSBO} (2-b)
 (see also Proposition \ref{prop:1532102} in Chapter \ref{sec:DSVO})
 under the assumption
 that the parameter 
$(\lambda,\nu)
 \in {\mathbb{C}}^2$
 satisfies $\nu-\lambda \in {\mathbb{N}}$.  
\begin{example}
\label{ex:3.3}
Suppose $(V,W)=(\Exterior^i({\mathbb{C}}^n), \Exterior^j({\mathbb{C}}^{n-1}))$.  
Then $[V:W] \ne 0$
 if and only if $j=i-1$ or $i$.  
Hence Theorem \ref{thm:152347} tells that 
 there exists a nonlocal symmetry breaking operators
$
I_{\delta}(i,\lambda)  \to J_{\varepsilon}(j, \nu) 
$
 only if $j \in \{i-1, i\}$.  
(In fact,
 this is also a sufficient condition,
 see Theorem \ref{thm:152271}.)
On the other hand,
 there exist nontrivial differential symmetry breaking operators
 for some $(\lambda, \nu) \in {\mathbb{C}}^2$
 if and only if $j \in \{i-2, i-1, i, i+1\}$, 
 as is seen from the complete classification
 of differential symmetry breaking operators 
 (Fact \ref{fact:2.7}).  
Thus there exist sporadic (differential) symmetry breaking operators
 when $j=i-2$ or $i+1$.  
\end{example}

\begin{remark}
\label{rem:MVW}
The assumption $[V:W]\ne 0$
 in Theorems \ref{thm:unique} and \ref{thm:1716110}
 is {\it{not}} an intertwining property
 for 
\index{A}{M1prime@$M'=O(n-1) \times O(1)$}
$M'=M \cap G' \simeq O(n-1) \times O(1)$
 but for the subgroup $O(n-1)$
 which is of index two in $M'$.  
We note that 
for 
\index{A}{Vlnd@$V_{\delta}:= V \boxtimes {\delta}$|textbf}
$V_{\delta}:= V \boxtimes {\delta} \in \widehat M$
 and $W_{\varepsilon}:= W \boxtimes {\varepsilon} \in \widehat {M'}$, 
\index{A}{VWmult@$[V:W]$}
\[
\text{
$
\operatorname{Hom}_{M'}(V_{\delta}|_{M'}, W_{\varepsilon})\ne\{0\}
$
 if and only if $[V:W]\ne 0$ and $\delta=\varepsilon$.  
}
\]
Indeed the condition
$
\delta = \varepsilon
$ 
 is not included in the assumption of Theorem \ref{thm:unique}
 on the construction of regular symmetry breaking operators.  
The reason is clarified
 in Theorem \ref{thm:regexist} in the next subsection.  
\end{remark}


\subsubsection{Existence condition for regular symmetry breaking operators}
\label{subsec:regSBOexist}
A {\it{regular symmetry breaking operator}} is 
 an \lq\lq{opposite}\rq\rq\ notion
 to a differential symmetry breaking operator
 in the sense
 that the support of its distribution kernel contains 
an interior point
 in the real flag manifold,
 see \cite[Def.~3.3]{sbon}.  
(See also Definition \ref{def:regSBO} in our special setting.)  
In \cite[Cor.~3.6]{sbon}
 we give a necessary condition 
 for the existence of regular symmetry breaking operators
 in the general setting.
This condition is also sufficient in our setting:
\begin{theorem}
[existence of regular symmetry breaking operators]
\label{thm:regexist}
Suppose $V \in \widehat {O(n)}$ and $W \in \widehat{O(n-1)}$.  
Then the following three conditions on the pair $(V,W)$ are equivalent:
\begin{enumerate}
\item[{\rm{(i)}}]
$[V:W] \ne 0$.  
\item[{\rm{(ii)}}]
There exists a nonzero regular symmetry breaking operator from 
 the $G$-module $I_{\delta}(V,\lambda)$
 to the $G'$-module $J_{\varepsilon}(W,\nu)$
 for some $(\lambda, \nu,\delta, \varepsilon) \in {\mathbb{C}}^2 \times \{\pm\}^2$.  
\item[{\rm{(iii)}}]
For any $(\delta, \varepsilon) \in \{\pm\}^2$, 
 there is an open dense subset $U$ in ${\mathbb{C}}^2$
 such that a nonzero regular symmetry breaking operator exists from $I_{\delta}(V,\lambda)$
 to $J_{\varepsilon}(W,\nu)$
 for all $(\lambda, \nu) \in U$.  
\end{enumerate}
\end{theorem}
The proof will be given in Section \ref{subsec:regexist}.  
The open dense subset $U$ is explicitly given in Proposition \ref{prop:172425}. 




\subsubsection{Integral operators, 
 analytic continuation, and normalization factors}
\label{subsec:VWSBO}

For an explicit construction of {\it{regular}} symmetry breaking operators,
 we use the reflection map 
\index{A}{1psin@$\psi_n$}
$\psi_n$ defined as follows: 

\index{A}{1psin@$\psi_n$|textbf}
\begin{equation}
\label{eqn:psim}
\psi_n \colon {\mathbb{R}}^n \setminus \{0\} \to O(n),
\quad
 x \mapsto I_n - \frac{2 x {}^{t\!} x}{|x|^2}.  
\end{equation}
Then $\psi_n(x)$ gives the reflection $\psi_n(x)$
 with respect to the hyperplane 
 $\{y \in {\mathbb{R}}^n:(x,y)=0\}$.  
Clearly,
 we have
\begin{equation}
\label{eqn:psix}
  \psi_n(x)=\psi_n(-x), 
\quad
\psi_n(x)^2=I_n, 
\quad
\text{ and }
\quad
\det \psi_n(x)=-1.  
\end{equation}



Suppose $(\sigma, V) \in \widehat{O(n)}$
 and $(\tau, W) \in \widehat{O(n-1)}$.  
For the construction of regular symmetry breaking operators,
 we need the condition $[V:W] \ne 0$, 
 see Theorem \ref{thm:regexist}.  
So let us assume 
$
[V:W] \ne 0.  
$
We fix a nonzero $O(n-1)$-homomorphism
\[
  \pr V W \colon V \to W, 
\]
which is unique up to scalar multiplication
 by Schur's lemma
 because $[V:W]=1$.  
We introduce a smooth map
\index{A}{RVW@$\Rij VW$|textbf}
\[
\Rij V W \colon {\mathbb{R}}^n \setminus \{ 0 \} 
            \to 
           \operatorname{Hom}_{\mathbb{C}}(V,W)
\]
by 
\begin{equation}
\label{eqn:RVW}
   \Rij VW := \pr VW  \circ \sigma \circ \psi_n.  
\end{equation} 



In what follows,
 we use the coordinates $(x,x_n) \in {\mathbb{R}}^n = {\mathbb{R}}^{n-1} \oplus {\mathbb{R}}$
 where $x=(x_1, \cdots, x_{n-1})$, 
 and the $n$-th coordinate $x_n$
 will play a special role.  

We set
\index{A}{Actg1@$\Atcal {\lambda}{\nu}{+}{V,W}$|textbf}
\index{A}{Actg2@$\Atcal {\lambda}{\nu}{-}{V,W}$|textbf}
\begin{align}
\label{eqn:KVWpt}
\Atcal {\lambda}{\nu}{+}{V,W}
 :=&
\frac{1}{\Gamma(\frac{\lambda + \nu -n +1}{2})
         \Gamma(\frac{\lambda-\nu}{2})}
         (|x|^2+ x_n^2)^{-\nu}
|x_n|^{\lambda+\nu-n}
\Rij V W (x,x_n), 
\\
\label{eqn:KVWmt}
\Atcal {\lambda}{\nu}{-}{V,W}
 :=&
\frac{1}{\Gamma(\frac{\lambda + \nu -n +2}{2})
         \Gamma(\frac{\lambda-\nu+1}{2})}
         (|x|^2 +x_n^2)^{-\nu}
|x_n|^{\lambda+\nu-n}
 {\operatorname{sgn}} x_n
\Rij VW (x,x_n).  
\end{align}

\begin{theorem}
[regular symmetry breaking operators]
\label{thm:152389}
\index {B}{regularsymmetrybreakingoperator@regular symmetry breaking operator}
Suppose $[V:W] \ne 0$
 and $\gamma \in \{ \pm \}$.  
Then the distributions 
 $\Atcal \lambda \nu {\gamma} {V,W}$, 
 initially defined
 as $\operatorname{Hom}_{\mathbb{C}}(V,W)$-valued
 locally integrable functions
 on ${\mathbb{R}}^n$
 for $\operatorname{Re} \lambda \gg |\operatorname{Re} \nu|$, 
 extends to $P'$-invariant elements in 
$
   {\mathcal{D}}'(G/P, {\mathcal{V}}_{\lambda,\delta}^{\ast}) \otimes W_{\nu,\varepsilon}
$
 for all $(\lambda, \nu) \in {\mathbb{C}}^2$
 and $\delta, \varepsilon \in \{ \pm \}$
with $\delta \varepsilon = \gamma$.  
Then the distributions $\Atcal \lambda \nu {\gamma} {V,W}$
 induce a family of symmetry breaking operators
\index{A}{Ahtg0@$\Atbb \lambda \nu {\pm} {V,W}$|textbf}
\begin{alignat*}{2}
\Atbb \lambda \nu {\gamma} {V,W}
\colon \,&C^{\infty}(G/P, {\mathcal{V}}_{\lambda,\delta})
   \to 
   C^{\infty}(G'/P', {\mathcal{W}}_{\nu,\varepsilon}), 
\end{alignat*}
which depends holomorphically on $(\lambda, \nu)$
 in the entire ${\mathbb{C}}^2$.  
\end{theorem}



\begin{remark}
The denominator in \eqref{eqn:KVWpt}
 is different from the product
 of the denominators
 of the two distributions 
$\frac{(|x|^2 + x_n^2)^{-\nu}}{\Gamma(\frac{n-\nu}{2})}$
and 
$\frac{|x_n|^{\lambda+\nu-n}}{\Gamma(\frac{\lambda+\nu-n+1}{2})}$
 on ${\mathbb{R}}^n$
 that depend holomorphically 
 on $(\lambda,\nu)$ in the entire ${\mathbb{C}}^2$.  
In fact the product
\begin{equation}
\label{eqn:bad}
\frac{(|x|^2 + x_n^2)^{-\nu}}{\Gamma(\frac{n-\nu}{2})}
\times
\frac{|x_n|^{\lambda+\nu-n}}{\Gamma(\frac{\lambda+\nu-n+1}{2})}
\end{equation}
 does not always make sense
 as distributions on ${\mathbb{R}}^n$.  
For instance, 
 if $(\lambda,\nu)=(-1,n)$, 
 then the multiplication \eqref{eqn:bad} means 
 the multiplication 
 (up to nonzero scalar multiplication)
 of the Dirac delta functions
 $\delta(x_1, \cdots,x_n)$ 
 by $\delta(x_n)$, 
 which is not well-defined in the usual sense.  
\end{remark}



Theorem \ref{thm:152389} will be proved in Section \ref{subsec:holoAVW}.  



We prove in Theorem \ref{thm:161243}
 that the normalization is optimal
 for $(V,W)=(\Exterior^i({\mathbb{C}}^n), \Exterior^j({\mathbb{C}}^{n-1}))$
 in the sense
 that the zeros of $\Atbb \lambda \nu \pm {V,W}$
 are of codimension $>1$
 in the parameter space
 of $(\lambda,\nu)$, 
 namely,
 discrete in ${\mathbb{C}}^2$
 in our setting.  
For the general $(V,W)$, 
 we shall give an upper and lower estimate
 of the null set of 
 the symmetry breaking operators 
 $\Atbb \lambda \nu {+} {V,W}$ 
 and $\Atbb \lambda \nu {-} {V,W}$
 in Theorem \ref{thm:1532113}.    


%%%%%%%%%%%%%%%%%%%%%%%%%%%%%%%%%%%%%%%%%%%%%%%%%%%%%%%%%%%%%%%%%%%
\subsection{
Classification scheme of symmetry breaking operators: general case}
\label{subsec:170228}
%%%%%%%%%%%%%%%%%%%%%%%%%%%%%%%%%%%%%%%%%%%%%%%%%%%%%%%%%%%%%%%%%%%
\index{B}{classificationscheme@classication scheme, symmetry breaking operators}

In this section,
we give a general scheme 
 for the classification of all symmetry breaking operators
 $I_{\delta}(V,\lambda)|_{G'} \to J_{\varepsilon}(W,\nu)$
 between the two principal series representations
 of $G$ and the subgroup $G'$
 in full generality
 where $(\sigma,V)\in \widehat{O(n)}$
 and $(\tau,W)\in \widehat{O(n-1)}$. 



We begin with conditions
on the parameter $(\lambda, \nu, \delta, \varepsilon)$
 for the existence of differential symmetry breaking operators.  
\begin{theorem}
[existence of differential symmetry breaking operators]
\index{B}{differentialsymmetrybreakingoperators@differential symmetry breaking operator}
\label{thm:vanDiff}
%%%
\qquad\qquad
%%%
\begin{enumerate}
\item[{\rm{(1)}}]
(Theorem {\ref{thm:vanishDiff}})
Suppose $\lambda, \nu \in {\mathbb{C}}$
 and $\delta, \varepsilon \in \{\pm\}$
 satisfy the generic parameter condition \eqref{eqn:nlgen}.  
Then,
\[
   {\operatorname{Diff}}_{G'}
   (I_{\delta}(V,\lambda)|_{G'}, J_{\varepsilon}(W,\nu))
   =\{0\}
\]
for any $(\sigma, V) \in \widehat{O(n)}$
 and $(\tau, W) \in \widehat{O(n-1)}$.  
\item[{\rm{(2)}}]
(Theorem {\ref{thm:existDSBO}})
Suppose $[V:W]\ne 0$.  
Then the converse statement holds,
 namely,
 if 
\index{A}{1psi@$\Psising$,
          special parameter in ${\mathbb{C}}^2 \times \{\pm\}^2$}
$(\lambda, \nu, \delta, \varepsilon) \in \Psising$
 (see \eqref{eqn:singset}), 
 then 
\[
   {\operatorname{Diff}}_{G'}
   (I_{\delta}(V,\lambda)|_{G'}, J_{\varepsilon}(W,\nu))
   \ne\{0\}.  
\]
\end{enumerate}
\end{theorem}
We give a proof for the first statement of Theorem \ref{thm:vanDiff}
 in Section \ref{subsec:vanDiff}, 
 and the second statement in Section \ref{subsec:extDiff}.  
Keeping Theorem \ref{thm:vanDiff} 
 on differential symmetry breaking operators in mind, 
 we state a general scheme for the classification
 of {\it{all}} symmetry breaking operators:

\begin{theorem}
[classification scheme
 of symmetry breaking operators]
\label{thm:VWSBO}
Let $n \ge 3$, 
 $(\sigma, V) \in \widehat{O(n)}$, 
 $(\tau,W) \in \widehat{O(n-1)}$, 
 $\lambda, \nu \in {\mathbb{C}}$
 and $\delta, \varepsilon \in \{\pm\}$.  
\begin{enumerate}
\item[{\rm{(1)}}]
Suppose 
\index{A}{VWmult@$[V:W]$}
$[V:W] =0$.  
Then 
\index{A}{DiffG@${\operatorname{Diff}}_{G'}(I_{\delta}(V,\lambda)\vert_{G'}, J_{\varepsilon}(W,\nu))$}
\[
  {\operatorname{Hom}}_{G'}(I_{\delta}(V,\lambda)|_{G'}, J_{\varepsilon}(W,\nu))  =
{\operatorname{Diff}}_{G'}(I_{\delta}(V,\lambda)|_{G'}, J_{\varepsilon}(W,\nu)).  
\]
\item[{\rm{(2)}}]
Suppose $[V:W]  \ne 0$.  
\begin{enumerate}
\item[{\rm{(2-a)}}]
{\rm{(generic case)}}\enspace
Suppose further 
 that $(\lambda, \nu, \delta, \varepsilon) \not\in \Psising$, 
 namely,
 it satisfies
 the generic parameter condition \eqref{eqn:nlgen}.  
Then 
\[
  {\operatorname{Hom}}_{G'}(I_{\delta}(V,\lambda)|_{G'}, J_{\varepsilon}(W,\nu))  =
{\mathbb{C}} \Atbb \lambda \nu {\delta\varepsilon}{V,W}.  
\]
In this case, 
 $\Atbb \lambda \nu {\delta\varepsilon}{V,W}$ is nonzero 
 and is not a differential operator.  
\item[{\rm{(2-b)}}]
{\rm{(special parameter case I, localness theorem)}}\enspace
\index{B}{localnesstheorem@localness theorem|textbf}
Suppose $\Atbb \lambda \nu {\delta\varepsilon}{V,W} \ne 0$
 and $(\lambda, \nu, \delta, \varepsilon)\in \Psising$
 ({\it{i.e.,}} does not satisfy the generic parameter condition \eqref{eqn:nlgen}).  
Then any symmetry breaking operator
 (in particular, 
 $\Atbb \lambda \nu {\delta\varepsilon}{V,W}$) is a differential operator and 
\[
  {\operatorname{Hom}}_{G'}(I_{\delta}(V,\lambda)|_{G'}, J_{\varepsilon}(W,\nu))  =
   {\operatorname{Diff}}_{G'}(I_{\delta}(V,\lambda)|_{G'}, J_{\varepsilon}(W,\nu))
\ni
\Atbb \lambda \nu {\delta\varepsilon}{V,W}.  
\]
\item[{\rm{(2-c)}}]
{\rm{(special parameter case II)}}\enspace
Suppose $\Atbb \lambda \nu {\delta\varepsilon}{V,W} = 0$.  
Then $(\lambda, \nu, \delta, \varepsilon)\in \Psising$, 
 and the 
 \index{B}{renormalized regular symmetry breaking operator@regular symmetry breaking operator, renormalized---}
renormalized operator 
\index{A}{Ahttgln0@$\Attbb {\lambda}{\nu}{\pm}{V,W}$}
$\Attbb \lambda \nu {\delta\varepsilon}{V,W}$
 (see Section \ref{subsec:AVW})
 gives a nonzero symmetry breaking operator
 which is not a differential operator.
We have
\[
  {\operatorname{Hom}}_{G'}(I_{\delta}(V,\lambda)|_{G'}, J_{\varepsilon}(W,\nu))  =
  {\mathbb{C}} \Attbb \lambda \nu {\delta\varepsilon}{V,W}
  \oplus
  {\operatorname{Diff}}_{G'}(I_{\delta}(V,\lambda)|_{G'}, 
 J_{\varepsilon}(W,\nu)).  
\]
In particular, 
\[
  \dim_{\mathbb{C}}
  {\operatorname{Hom}}_{G'}(I_{\delta}(V,\lambda)|_{G'}, J_{\varepsilon}(W,\nu))  \ge 2.  
\]
\end{enumerate}
\end{enumerate}
\end{theorem}




The first assertion of Theorem \ref{thm:VWSBO}
 is a restatement of Theorem \ref{thm:152347}.  
The case (2-a) is given in Theorem \ref{thm:genbasis}
 and the case (2-b) is in Proposition \ref{prop:1532102}. 
The first statement 
 for the case (2-c) is proved in Theorem \ref{thm:170340} (1).  
The direct sum decomposition is given in Corollary \ref{cor:Azero}.  
The last statement follows from the existence 
 of nonzero differential symmetry breaking operators
 for all special parameters
 (Theorem \ref{thm:vanDiff} (2)).  



Theorems \ref{thm:1716110} and \ref{thm:VWSBO} lead us to a vanishing result
 of symmetry breaking operators
 as follows:  

\begin{corollary}
[vanishing of symmetry breaking operators]
\label{cor:VWvanish}
Let $(\sigma, V) \in \widehat{O(n)}$, 
 $(\tau, W) \in \widehat{O(n-1)}$, 
 $\lambda, \nu \in {\mathbb{C}}$
 and $\delta, \varepsilon \in \{\pm\}$.  
If $[V:W]=0$ and 
 $(\lambda, \nu)$ satisfies 
 the generic parameter condition \eqref{eqn:nlgen}, 
 then 
\[
   {\operatorname{Hom}}_{G'}
   (I_{\delta}(V,\lambda)|_{G'}, J_{\varepsilon}(W,\nu))
   =\{0\}.  
\]
\end{corollary}

\begin{proof}
By Theorem \ref{thm:VWSBO} (1), 
 we have
\[
   {\operatorname{Hom}}_{G'}
   (I_{\delta}(V,\lambda)|_{G'}, J_{\varepsilon}(W,\nu))
   ={\operatorname{Diff}}_{G'}
    (I_{\delta}(V,\lambda)|_{G'}, J_{\varepsilon}(W,\nu))
\]
because $[V:W]=0$.  
In turn, 
 the right-hand side reduces to zero 
 by Theorem \ref{thm:1716110}
 because of the generic parameter condition \eqref{eqn:nlgen}.  
\end{proof}

Theorem \ref{thm:VWSBO} gives a classification 
 of symmetry breaking operators
 up to the following two problems:
\begin{enumerate}
\item[$\bullet$]
the location of zeros of the normalized
 regular symmetry breaking operator
 $\Atbb \lambda \nu \gamma {V,W}$;
\item[$\bullet$]
the classification of {\it{differential}} symmetry breaking operators.  
\end{enumerate}



For $(V,W)=(\Exterior^i({\mathbb{C}}^n), \Exterior^j({\mathbb{C}}^{n-1}))$, 
 these two problems are solved explicitly
 in Theorem \ref{thm:161243} and Fact \ref{fact:2.7}, 
 respectively, 
 and thus we accomplish the complete classification
 of symmetry breaking operators.  
This will be stated
 in Theorem \ref{thm:1.1} (multiplicity formula)
 and in Theorem \ref{thm:SBObasis} (explicit generators).  



\subsection{Summary: vanishing of regular 
 symmetry breaking operators $\Atbb \lambda \nu \pm {V,W}$}
\label{subsec:vanAVW}
As we have seen in the classification scheme
 (Theorem \ref{thm:VWSBO}) 
 for all symmetry breaking operators, 
 the parameter $(\lambda,\nu,\delta,\varepsilon)$ 
 for which the (generically) regular symmetry breaking operator
 $\Atbb \lambda \nu \pm {V,W}$
 vanishes plays a crucial role
 in the classification theory.  
For $(\lambda,\nu,\delta,\varepsilon) \in \Psising$, 
 we noted:
\begin{enumerate}
\item[$\bullet$]
when $\Atbb {\lambda_0} {\nu_0} \pm {V, W}=0$, 
 we can construct a nonzero symmetry breaking operator
 $\Attbb {\lambda_0} {\nu_0} \pm {V, W}$
 by \lq\lq{renormalization}\rq\rq\
 which is {\it{not}} a differential operator
 (Theorem \ref{thm:170340});
\item[$\bullet$]
when $\Atbb {\lambda_0} {\nu_0} \pm {V, W}\ne 0$, 
we prove a 
\index{B}{localnesstheorem@localness theorem}
 {\it{localness theorem}}
 asserting
 that all symmetry breaking operators are differential operators
 (Proposition \ref{prop:1532102}).  
\end{enumerate}



We obtain a condition 
 for the (non) vanishing of $\Atbb {\lambda} {\nu} \pm {V, W}$
 as follows.  
Using the same notation
 as in \cite[Chap.~1]{sbon}, 
 we define the following two subsets in ${\mathbb{Z}}^2$:
\index{A}{Leven@$L_{\operatorname{even}}$|textbf}
\index{A}{Lodd@$L_{\operatorname{odd}}$|textbf}
\begin{alignat}{2}
\label{eqn:Leven}
L_{\operatorname{even}}:=
&\left\{ \right.
 (-i,-j):
0 \le j\leq i \mbox{ and } i\equiv j 
&& \mod 2 \left. \right \},
\\
\label{eqn:Lodd}
L_{\operatorname{odd}}:=
&\left\{ \right.
  (-i,-j)
: 0 \le j\leq i \mbox{ and } i \equiv j +1 
&& \mod 2 \left. \right \}.
\end{alignat}

\begin{theorem}
\label{thm:1532113}
Let $(\sigma, V) \in \widehat {O(n)}$
 and $(\tau, W) \in \widehat {O(n-1)}$
 with $[V:W]\ne 0$.  
\begin{enumerate}
\item[{\rm{(1)}}]
There exists 
\index{A}{Nsigma@$N(\sigma)$}
$
N(\sigma) \in {\mathbb{N}}
$
such that
\begin{alignat*}{2}
\Atbb \lambda \nu {+} {V,W} =&0
\qquad
&&\text{if } (\lambda,\nu) \in L_{\operatorname{even}}
\text{ and } \nu \le -N(\sigma), 
\\
\Atbb \lambda \nu {-} {V,W} =&0
\qquad
&&\text{if } (\lambda,\nu) \in L_{\operatorname{odd}}
\text{ and } \nu \le -N(\sigma).  
\end{alignat*}
\item[{\rm{(2)}}]
If $\Atbb \lambda \nu {+} {V,W} =0$
 then 
 $\nu-\lambda \in 2{\mathbb{N}}$;
 if $\Atbb \lambda \nu {-} {V,W} =0$
then 
 $\nu - \lambda \in 2 {\mathbb{N}}+1$.    
\end{enumerate}
\end{theorem}


\begin{remark}
We shall show in Lemma \ref{lem:Nsigma}
 that $N(\sigma)$ can be taken
 to be $\ell(\sigma)$, 
 as defined in \eqref{eqn:lsigma}.  
\end{remark}

Theorem \ref{thm:1532113} (2) is a part
 of Theorem \ref{thm:VWSBO} (2), 
 and will be proved in Section \ref{subsec:1532113}.  


Combining Theorems \ref{thm:VWSBO} and \ref{thm:1532113}, 
 we see that there exist infinitely many 
 $(\lambda,\nu)\in {\mathbb{C}}^2$
 such that the multiplicity 
$
   m(I_{\delta}(V,\lambda), J_{\varepsilon}(W,\nu))>1
$
 as follows:
\begin{corollary}
\label{cor:170821}
Let $(\sigma,V) \in \widehat {O(n)}$ and $(\tau,W) \in \widehat{O(n-1)}$
 satisfy $[V:W] \ne 0$.  
If 
\[
  (\lambda, \nu) \in 
  \begin{cases}
  L_{\operatorname{even}} \cap \{\nu \le - N(\sigma)\}
  \quad
  &\text{for \,$\delta \varepsilon =+$}, 
\\
  L_{\operatorname{odd}} \cap \{\nu \le - N(\sigma)\}
  \quad
  &\text{for \,$\delta \varepsilon =-$}, 
  \end{cases}
\]
then we have
\[
  \dim_{\mathbb{C}}
  {\operatorname{Hom}}_{G'}
  (I_{\delta}(V,\lambda)|_{G'}, J_{\varepsilon}(W,\nu))>1.  
\]
\end{corollary}
By Theorem \ref{thm:1532113},
 we get readily the following corollary, 
 to which we shall return in Chapter \ref{sec:conjecture}
 (see Example \ref{ex:317}).  
\begin{corollary}
\label{cor:IV1}
Suppose that 
$\Atbb \lambda \nu \delta{V,W}=0$.  
Then $\Atbb {n-\lambda}{n-1-\nu} \delta{V,W} \ne 0$.  
\end{corollary}

Theorem \ref{thm:1532113} means that
\index{A}{Leven@$L_{\operatorname{even}}$}
\index{A}{Lodd@$L_{\operatorname{odd}}$}
\begin{align*}
L_{\operatorname{even}}
\cap 
\{\nu \le -N(\sigma) \}
\subset
&
\{(\lambda,\nu) \in {\mathbb{C}}^2
:
\Atbb \lambda \nu {+} {V,W} =0\}
\subset
\{(\lambda,\nu) \in {\mathbb{C}}^2
:
\nu-\lambda \in 2 {\mathbb{N}}\}, 
\\
L_{\operatorname{odd}}
\cap 
\{\nu \le -N(\sigma) \}
\subset
&
\{(\lambda,\nu) \in {\mathbb{C}}^2
:
\Atbb \lambda \nu {-} {V,W} =0\}
\subset
\{(\lambda,\nu) \in {\mathbb{C}}^2
:
\nu-\lambda \in 2 {\mathbb{N}}+1\}.  
\end{align*}


We shall determine in Theorem \ref{thm:161243}
 the set $\{(\lambda,\nu) \in {\mathbb{C}}^2:
\Atbb \lambda \nu \gamma {V,W} =0\}$
 for $\gamma = \pm$
 in the special case where $(V,W)=(\Exterior^i({\mathbb{C}}^n), \Exterior^j({\mathbb{C}}^{n-1}))$.  
If $\sigma$ is the $i$-th exterior representation $\sigma^{(i)}$
 on $\Exterior^i({\mathbb{C}}^n)$, 
 then we can take $N(\sigma)$ to be 0
 if $i=0$ or $n$;
 to be 1 if $1 \le i \le n-1$.  
In this case, 
 the left inclusion is almost a bijection.  
On the other hand, 
 concerning the right inclusions,
 we refer to Theorem \ref{thm:VWSBO} (2-b), 
 which will be proved
 in Section \ref{subsec:170213},
 see Proposition \ref{prop:1532102}.   



%%%%%%%%%%%%%%%%%%%%%%%%%%%%%%%%
\subsection{The classification of symmetry breaking operators
 for differential forms}
\label{subsec:exhaust}
%%%%%%%%%%%%%%%%%%%%%%%%%%%%%%%%

Let $(G,G')=(O(n+1,1),O(n,1))$ with $n \ge 3$
 as before.  
We consider the special case
\[
  (V,W)=(\Exterior^i({\mathbb{C}}^{n}), \Exterior^j({\mathbb{C}}^{n-1})).  
\]
Then the corresponding principal series representations
 $I_{\delta}(V,\lambda)$ of $G$
 and $J_{\varepsilon}(W,\nu)$ of the subgroup $G'$
 are denoted by $I_{\delta}(i,\lambda)$
 and $J_{\varepsilon}(j,\nu)$, 
 respectively.  
In this section 
 we summarize the complete classification
 of symmetry breaking operators from 
 the $G$-module $I_{\delta}(i,\lambda)$ 
 to the $G'$-module $J_{\varepsilon}(j,\nu)$.  
The main results are stated in Theorems \ref{thm:1.1} and \ref{thm:SBObasis}.  
Our results rely on the vanishing condition
 of the normalized regular symmetry breaking operators
 $\Atbb \lambda \nu \gamma {i,j}$
 (Theorem \ref{thm:161243}) 
 and the classification
 of differential symmetry breaking operators
 (Fact \ref{fact:2.7}).  



\subsubsection{Vanishing condition
 for the regular symmetry breaking operators 
 $\Atbb \lambda \nu \gamma {i,j}$}
\label{subsec:Aijvanish}

We apply the general construction
 of the (normalized) symmetry breaking operators
 $\Atbb \lambda \nu \gamma {V,W}$
 in Theorem \ref{thm:152389}
 to the pair
 of representations
$
  (V,W)=(\Exterior^i({\mathbb{C}}^{n}), \Exterior^j({\mathbb{C}}^{n-1})).  
$  
Then we obtain (normalized) symmetry breaking operators, 
 to be denoted by 
 $\Atbb \lambda \nu \gamma {i,j}$, 
that depend holomorphically on $(\lambda,\nu)$
 in the entire complex plane ${\mathbb{C}}^2$
 if $j \in \{i-1,i\}$
 and $\gamma \in \{\pm\}$, 
 see Theorem \ref{thm:1522a}.  



We determine the zero set of $\Atbb \lambda \nu {\gamma} {i,j}$
 explicitly as follows:
\begin{theorem}
[zeros of regular symmetry breaking operators
 $\Atbb \lambda \nu \pm {i,j}$]
\label{thm:161243}
~~~~~
\begin{enumerate}
\item[{\rm{(1)}}]
For $0 \le i \le n-1$, 
\begin{multline*}
\{(\lambda, \nu) \in {\mathbb{C}}^2
:
\Atbb \lambda \nu + {i,i} =0\}
\\
= 
\begin{cases}
L_{\operatorname{even}}
\quad
&
\text{if $i=0$}, 
\\
(L_{\operatorname{even}} \setminus \{\nu=0\}) \cup \{(i,i)\}
\quad
&
\text{if $1 \le i \le n-1$}.  
\end{cases}
\end{multline*}
\item[{\rm{(2)}}]
For $1 \le i \le n$, 
\begin{multline*}
\{(\lambda, \nu) \in {\mathbb{C}}^2
:
\Atbb \lambda \nu + {i,i-1} =0\}
\\
=
\begin{cases}
(L_{\operatorname{even}}\setminus \{\nu=0\}) \cup \{(n-i,n-i)\}
\quad
&
\text{if $1 \le i \le n-1$}, 
\\
L_{\operatorname{even}} 
\quad
&
\text{if $i=n$}.  
\end{cases}
\end{multline*}
\item[{\rm{(3)}}]
For $0 \le i \le n-1$, 
\begin{multline*}
\{(\lambda, \nu) \in {\mathbb{C}}^2
:
\Atbb \lambda \nu - {i,i} =0\}
\\
= 
\begin{cases}
L_{\operatorname{odd}}
\quad
&
\text{if $i=0$}, 
\\
L_{\operatorname{odd}}\setminus \{\nu=0\} 
\quad
&
\text{if $1 \le i \le n-1$}.  
\end{cases}
\end{multline*}
\item[{\rm{(4)}}]
For $1 \le i \le n$, 
\begin{multline*}
\{(\lambda, \nu) \in {\mathbb{C}}^2
:
\Atbb \lambda \nu - {i,i-1} =0\}
\\
=
\begin{cases}
L_{\operatorname{odd}}\setminus \{\nu=0\}
\quad
&
\text{if $1 \le i \le n-1$}, 
\\
L_{\operatorname{odd}} 
\quad
&
\text{if $i=n$}.  
\end{cases}
\end{multline*}
\end{enumerate}
\end{theorem}



Theorem \ref{thm:161243} will be proved in Section \ref{subsec:Aijzero}
 by using the residue formula
 of $\Atbb \lambda \nu{\pm}{i,j}$
 (\cite{xkresidue}).  



A special case of Theorem \ref{thm:161243} includes the following.  
\begin{example}
\label{ex:3.19}
\begin{enumerate}
\item[{\rm{(1)}}]
For $0 \leq i \leq n$, 
$\Atbb i i + {i,i}=0$
 and 
$\Atbb {n-i} {n-i-1} + {i,i} \ne 0$.  
\item[{\rm{(2)}}]
For $0 \leq i \leq n-1$, 
$\Atbb i i + {n-i,n-i-1}  = 0$
 and $\Atbb {n-i}{n-i-1} + {n-i,n-i-1} \ne 0$. 
\end{enumerate}
\end{example}

\begin{remark}
In the case $i=0$, 
 $\Atbb \lambda \nu + {i,i}$
 is the scalar-valued
 symmetry breaking operator
 induced from the scalar-valued distribution 
 $\Atcal \lambda \nu + {}$, 
 as we recall from \eqref{eqn:KAlnn+}.  
Thus the case $i=0$ in (1)
 was proved in \cite[Thm.~8.1]{sbon}.  
\end{remark}




\subsubsection{Differential symmetry breaking operators}

We review from \cite{KKP} the notation
 of conformally equivariant {\it{differential}} operators
 ${\mathcal{E}}^i(S^n) \to {\mathcal{E}}^j(S^{n-1})$, 
 namely,
 {\it{differential}} symmetry breaking operators 
 $I_{\delta}(V,\lambda)|_{G'} \to J_{\varepsilon}(W,\nu)$
 with $(V,W)=(\Exterior^i({\mathbb{C}}^n), \Exterior^j({\mathbb{C}}^{n-1}))$.  
The complete classification of {\it{differential}} symmetry breaking operators
 was recently accomplished in \cite[Thm.~2.8]{KKP}
 based on the F-method
 \cite{xkHelg85}.  

\begin{fact}
[classification of differential symmetry breaking operators]
\label{fact:2.7}
Let $n \ge 3$.  
Suppose $0 \le i \le n$, 
 $0 \le j \le n-1$, 
 $\lambda, \nu \in {\mathbb{C}}$, 
 and $\delta, \varepsilon \in \{ \pm \}$.  
Then the following three conditions on 6-tuple 
 $(i,j, \lambda,\nu,\delta, \varepsilon)$ are equivalent.  
\begin{enumerate}
\item[{\rm{(i)}}]
\index{A}{DiffG@${\operatorname{Diff}}_{G'}(I_{\delta}(V,\lambda)\vert_{G'}, J_{\varepsilon}(W,\nu))$}
\index{A}{Ideltai@$I_{\delta}(i, \lambda)$}
\index{A}{Jjepsilon@$J_{\varepsilon}(j, \nu)$}
$
\operatorname{Diff}_{G'}(I_{\delta}(i,\lambda)|_{G'}, J_{\varepsilon}(j,\nu))\ne \{0\}$.  
\item[{\rm{(ii)}}]
$\dim_{\mathbb{C}} \operatorname{Diff}_{G'}(I_{\delta}(i,\lambda)|_{G'}, J_{\varepsilon}(j,\nu))=1$.  
\item[{\rm{(iii)}}]
$\nu-\lambda \in {\mathbb{N}}$, 
 $(-1)^{\nu-\lambda}=\delta \varepsilon$, 
 and one of the following conditions holds:
\begin{enumerate}
\item[{\rm{(a)}}]
$j=i-2$, 
$2 \le i \le n-1$, 
$(\lambda,\nu)=(n-i,n-i+1)$;
\item[{\rm{(a$'$)}}]
$(i,j)=(n,n-2)$, 
 $-\lambda \in {\mathbb{N}}$, 
 $\nu=1$;
\item[{\rm{(b)}}]
$j=i-1$, 
$1 \le i \le n$;
\item[{\rm{(c)}}]
$j=i$, 
$0 \le i \le n-1$;
\item[{\rm{(d)}}]
$j=i+1$, 
$1 \le i \le n-2$, 
$(\lambda,\nu)=(i,i+1)$;
\item[{\rm{(d$'$)}}]
$(i,j)=(0,1)$, 
 $-\lambda \in {\mathbb{N}}$, 
 $\nu=1$.  
\end{enumerate}
\end{enumerate} 
\end{fact}



The generators are explicitly constructed
 in \cite[(2.24)--(2.32)]{KKP}
(see \cite{Juhl, KOSS, sbon} for the $i=0$ case), 
 which we review quickly.  
Let $\widetilde C_{\ell}^{\alpha}(z)$ 
be the 
\index{B}{Gegenbauerpolynomial@Gegenbauer polynomial|textbf}
Gegenbauer polynomial of degree $\ell$, 
 normalized by
\index{A}{CGegenbauernorm@$\widetilde C_l^{\alpha}(z)$, normalized Gegenbauer polynomial|textbf}
\begin{equation}
\label{eqn:Gegen}
\widetilde C_{\ell}^{\alpha}(z)
:=
\frac{1}{\Gamma(\alpha + [\frac{\ell+1}{2}])}
\sum_{k=0}^{[\frac \ell 2]}
(-1)^k
\frac{\Gamma(\ell-k+\alpha)}{k! (\ell-2k)!}
(2 z)^{\ell-2k}
\end{equation}
 as in \cite[(14.3)]{KKP}.  
Then $\widetilde C_{\ell}^{\alpha}(z) \not \equiv 0$
 for all $\alpha \in {\mathbb{C}}$
 and $\ell \in {\mathbb{N}}$.  



For $\ell \in {\mathbb{N}}$, 
 we inflate $\widetilde C_{\ell}^{\alpha} (z)$
 to a polynomial of two variables
 $x$ and $y$:
\begin{align}
  \widetilde C_{\ell}^{\alpha} (x,y)
  :=& x^{\frac \ell 2} \widetilde C_{\ell}^{\alpha} (\frac{y}{\sqrt x})
\notag
\\
  =& \sum_{k=0}^{[\frac \ell 2]} 
     \frac{(-1)^k \Gamma(\ell-k+\alpha)}
          {\Gamma(\alpha + [\frac{\ell +1}{2}])\Gamma(\ell-2k+1)k!}(2y)^{\ell -2k}x^k.  
\label{eqn:Gegentwo}
\end{align}
For instance, 
 $\widetilde C_{0}^{\alpha} (x,y)=1$, 
 $\widetilde C_{1}^{\alpha} (x,y)=2y$, 
 $\widetilde C_{2}^{\alpha} (x,y)=2(\alpha+1)y^2-x$, etc.  
Notice that $\widetilde C_{\ell}^{\alpha} (x^2,y)$ is a homogeneous
 polynomial of $x$ and $y$ of degree $\ell$.  



For $\nu-\lambda \in {\mathbb{N}}$, 
we set
 a scalar-valued differential operator
 $\Ctbb \lambda \nu {}:C^{\infty}({\mathbb{R}}^n)
 \to 
 C^{\infty}({\mathbb{R}}^{n-1})$
 by 
\index{A}{Chlmdn@$\Ctbb \lambda \nu {}$, Juhl's operator|textbf}
\begin{equation}
\label{eqn:Cln}
   \Ctbb \lambda \nu {}
   :=
   \operatorname{Rest}_{x_n=0} \circ
   \widetilde C_{\nu-\lambda}^{\lambda -\frac{n-1}{2}}
   (-\Delta_{{\mathbb{R}}^{n-1}}, \frac{\partial}{\partial x_n}).  
\end{equation}



For $\mu \in {\mathbb{C}}$
 and $a \in {\mathbb{N}}$, 
 we set 
\index{A}{0cgammamualpha@$\gamma(\mu,a)$|textbf}
\begin{equation}
\label{eqn:gamma}
\gamma(\mu,a)
:=
\begin{cases}
1
&\text{if $a$ is odd, }
\\
\mu + \frac a 2
\qquad
&\text{if $a$ is even.  }
\end{cases}
\end{equation}
We are ready 
 to define matrix-valued differential operators
\[
   \Ctbb \lambda \nu {i,j}\colon
 {\mathcal{E}}^i({\mathbb{R}}^n)
 \to 
 {\mathcal{E}}^j({\mathbb{R}}^{n-1})
\]
which were introduced in \cite[(2.24) and (2.26)]{KKP}
 by the following formul{\ae}:
\index{A}{Ciiln@$\Cbb \lambda \nu {i,j}$, matrix-valued differential operator|textbf}
\begin{equation}
\label{eqn:Ciiln}
\Cbb \lambda \nu {i, i}
:=
\Ctbb{\lambda+1}{\nu-1}{} d_{\mathbb{R}^n}d_{\mathbb{R}^n}^{\ast}
-
\gamma(\lambda - \frac n 2, \nu -\lambda) 
\Ctbb{\lambda}{\nu-1}{} d_{\mathbb{R}^n} \iota_{\frac{\partial}{\partial x_n}}
+\frac 1 2(\nu-i)
\Ctbb{\lambda}{\nu}{}, 
\end{equation}

\index{A}{Ciiln@$\Cbb \lambda \nu {i,j}$, matrix-valued differential operator|textbf}
\begin{equation}
\label{eqn:Cijln}
\Cbb \lambda \nu {i,i-1}
:=
- \Ctbb{\lambda+1}{\nu-1}{} d_{\mathbb{R}^n} d_{\mathbb{R}^n}^{\ast}
 \iota_{\frac{\partial}{\partial x_n}}
-
\gamma(\lambda - \frac {n-1} 2, \nu -\lambda) 
\Ctbb{\lambda+1}{\nu}{} d_{\mathbb{R}^n}^{\ast}
+
\frac 1 2
(\lambda+i-n)
\Ctbb{\lambda}{\nu}{}\iota_{\frac{\partial}{\partial x_n}}.  
\end{equation}
Here 
$
   \iota_{Z}\colon {\mathcal{E}}^i({\mathbb{R}^n}) \to {\mathcal{E}}^{i-1}({\mathbb{R}^n})
$
 stands
 for the interior product 
 which is defined to be the contraction with a vector field $Z$.  



We note that 
\begin{alignat*}{4}
&\Cbb \lambda \nu {0,0}
&&=\frac 1 2 \nu \Ctbb \lambda \nu {}, 
\qquad
&&\Cbb \nu \nu {i,i}
&&=\frac 1 2 (\nu-i){\operatorname{Rest}}_{x_n=0}, 
\\
&\Cbb \lambda \lambda {i,i-1}
&&=\frac 1 2 (\lambda+i-n){\operatorname{Rest}}_{x_n=0}
 \circ \iota_{\frac{\partial}{\partial x_n}}, 
\qquad
&&\Cbb \lambda \nu {n,n-1} 
&&= \frac 1 2 \nu \Ctbb \lambda \nu {}
    \circ \iota_{\frac{\partial}{\partial x_n}}.  
\end{alignat*}

The operators $\Cbb \lambda \nu {i,j}$ vanish
 for the following special values of $(\lambda, \nu)$:
\begin{alignat*}{3}
\Cbb \lambda \nu {i,i} = & 0
\qquad
&&\text{if and only if }
\lambda =  \nu = i\,\,
&&\text{ or }
\nu=i=0, 
\\
\Cbb \lambda \nu {i,i-1} = & 0
\qquad
&&\text{if and only if }
\lambda =  \nu =n-i\,\,
&&\text{ or }
\nu=n-i=0.  
\end{alignat*}



In order to provide {\it{nonzero}} operators, 
 following the notation as in \cite[(2.30)]{KKP}, 
 we renormalize $\Cbb \lambda \nu {i,j}$ as 
\index{A}{Ctiiln@$\Ctbb \lambda \nu {i,i}$|textbf}
\index{A}{Ctiim1ln@$\Ctbb \lambda \nu {i,i-1}$|textbf}
\begin{align}
\Ctbb \lambda \nu {i,i}
&:=
\begin{cases}
\operatorname{Rest}_{x_n=0}
\qquad
&\text{if $\lambda = \nu$}, 
\\
\Ctbb \lambda \nu {}
&\text{if $i=0$}, 
\\
\Cbb \lambda \nu {i,i}
&\text{otherwise}, 
\end{cases}
\label{eqn:Ciitilde}
\\
\Ctbb \lambda \nu {i,i-1}
&:=
\begin{cases}
\operatorname{Rest}_{x_n=0}\circ \iota_{\frac{\partial}{\partial x_n}}
\qquad
&\text{if $\lambda = \nu$}, 
\\
\Ctbb \lambda \nu {}\circ \iota_{\frac{\partial}{\partial x_n}}
&\text{if $i=n$}, 
\\
\Cbb \lambda \nu {i,i-1}
&\text{otherwise}.  
\end{cases}
\label{eqn:Cii-1tilde}
\end{align}

For $j=i-2$ or $i+1$, 
 we also set 
\index{A}{Ctii2ln@$\Ctbb \lambda {n-i+1} {i,i-2}$|textbf}
\index{A}{Ctii1ln@$\Ctbb \lambda {i+1} {i,i+1}$|textbf}
\begin{align*}
\Ctbb \lambda {n-i+1} {i,i-2}
:=&
\begin{cases}
- d_{\mathbb{R}^{n-1}}^{\ast} \circ \Ctbb \lambda 0 {n,n-1}
&\text{if $i=n$, $\lambda \in -{\mathbb{N}}$}, 
\\
\operatorname{Rest}_{x_n=0}
\circ 
\iota_{\frac{\partial}{\partial x_n}} 
d_{\mathbb{R}^{n}}^{\ast}
\qquad
&\text{if $2 \le i \le n-1$, $\lambda = n-i$}.   
\end{cases}
\\
\Ctbb \lambda {i+1} {i,i+1}
:=&
\begin{cases}
d_{{\mathbb{R}}^{n-1}}
\circ 
\Ctbb \lambda 0 {}
\qquad\hspace{6mm}
&\text{if $i=0$, $\lambda  \in -{\mathbb{N}}$}, 
\\
\operatorname{Rest}_{x_n=0}
\circ 
d_{\mathbb{R}^{n}}
&\text{if $1 \le i \le n-2$, $\lambda=i$}.  
\end{cases}
\end{align*}



With the notation as above,
 we can describe explicit generators
 of the space ${\operatorname{Diff}}_{G'}(I_{\delta}(i,\lambda)|_{G'}, J_{\varepsilon}(j,\varepsilon))$
 of differential symmetry breaking operators:
\begin{fact}
[basis, {\cite[Thm.~2.9]{KKP}}]
\label{fact:3.9}
\index{B}{differentialsymmetrybreakingoperators@differential symmetry breaking operator}
Suppose that 6-tuple $(i,j,\lambda,\nu, \delta, \varepsilon)$ is 
 one of the six cases
 in Fact \ref{fact:2.7} (iii).  
Then the differential symmetry breaking operators
 $I_{\delta}(i,\lambda) \to J_{\varepsilon}(j,\nu)$
 are proportional to 
\begin{alignat*}{2}
&j=i-2:\,\,&& \Ctbb {n-i}{n-i+1}{i,i-2} \, (2 \le i \le n-1);
              \quad
              \Ctbb {\lambda}{1}{n,n-2} \, (i=n), 
\\
&j=i-1:&&  \Ctbb \lambda \nu{i,i-1}, 
\\
&j=i:&& \Ctbb \lambda \nu {i,i}, 
\\
&j=i+1:&& \Ctbb {i}{i+1}{i,i+1} \, (1 \le i \le n-2);
          \quad
          \Ctbb {\lambda}{1}{0,1} \, (i=0).  
\end{alignat*}
\end{fact}
\begin{remark}
The scalar case ($i=j=0$)
 was classified
 in Juhl \cite{Juhl} for $n \ge 3$.  
See also \cite{KOSS}
 for a different approach 
 using the 
\index{B}{Fmethod@F-method}
F-method.  
The case $n=2$ (and $i=j=0$)
 is essentially equivalent to find 
 differential symmetry breaking operators from 
 the tensor product of two principal series representations
 to another principal series representation 
 for $SL(2,{\mathbb{R}})$.  
In this case,
 generic (but not all) operators are given
 by the 
\index{B}{RankinCohenbracket@Rankin--Cohen bracket}
Rankin--Cohen brackets, 
 and the complete classification
 was accomplished in \cite[Thms.~9.1 and 9.2]{KP2}.  
We note
 that the dimension of differential symmetry breaking operators
 may jump to two 
 at some singular parameters
 where $n=2$.  
\end{remark}



%%%%%%%%%%%%%%%%%%%%%%%%%%%%%%%%%%%%%%%%%%%%%%%%%%%%%%
\subsubsection{Formula
 of the dimension of 
$
   {\operatorname{Hom}}_{G'}
   (I_{\delta}(i,\lambda)|_{G'},J_{\varepsilon}(j,\nu))
$}
%%%%%%%%%%%%%%%%%%%%%%%%%%%%%%%%%%%%%%%%%%%%%%%%%%%%%%
For admissible smooth representations $\Pi$ of $G$
 and $\pi$ of the subgroup $G'$, 
 we set
\index{A}{mPipi@$m(\Pi, \pi)$, multiplicity|textbf}
\[
   m(\Pi, \pi) := \dim_{\mathbb{C}} {\operatorname{Hom}}_{G'}
                               (\Pi|_{G'}, \pi).  
\]
In this subsection
 we give a formula of the multiplicity $m(\Pi, \pi)$
 for $\Pi=I_{\delta}(i,\lambda)$
 and $\pi=J_{\varepsilon}(j,\nu)$.  



\begin{theorem}
[multiplicity formula]
\label{thm:1.1}
Let $(G,G')=(O(n+1,1),O(n,1))$
 with $n \ge 3$.  
Suppose $\Pi=I_{\delta}(i,\lambda)$
 and $\pi=J_{\varepsilon}(j,\nu)$
 for $0 \le i\le n$, $0 \le j \le n-1$,
 $\delta$, $\varepsilon \in \{\pm \}$, 
 and $\lambda,\nu \in {\mathbb{C}}$.  
Then we have the following.  
\begin{enumerate}
\item[{\rm{(1)}}]
\begin{alignat*}{2}
m(\Pi,\pi) \in &\{ 1,2 \} \qquad
&&\text{if $j=i-1$ or $i$}, 
\\
m(\Pi,\pi) \in &\{ 0,1 \} \qquad
&&\text{if $j=i-2$ or $i+1$}, 
\\
m(\Pi,\pi) =& 0 \qquad
&&\text{otherwise}.  
\end{alignat*}
\item[{\rm{(2)}}]
Suppose $j=i-1$ or $i$.  
Then $m(\Pi,\pi)=1$ 
 except for the countable set described as below.  
\begin{enumerate}
\item[{\rm{(a)}}]
Case $1 \le i \le n-1$.
Then $m(I_{\delta}(i,\lambda), J_{\varepsilon}(i,\nu))=2$
 if and only if 

\begin{alignat*}{3}
&j=i, 
&&\delta \varepsilon=+, 
&&(\lambda, \nu) \in L_{\operatorname{even}}\setminus \{\nu=0\}
                   \cup \{(i,i)\},
\\
&j=i, 
&&\delta \varepsilon=-, 
&&(\lambda, \nu) \in L_{\operatorname{odd}}\setminus \{\nu=0\}, 
\\
&j=i-1, \quad
&&\delta \varepsilon=+, \quad
&&(\lambda, \nu) \in L_{\operatorname{even}}\setminus \{\nu=0\} 
                   \cup \{(n-i,n-i)\}, 
\intertext{or}
&j=i-1, 
&&\delta \varepsilon=-, 
&&(\lambda, \nu) \in L_{\operatorname{odd}}\setminus \{\nu=0\}.   
\end{alignat*}
\item[{\rm{(b)}}]
Case $i=0$.  
Then 
$
   m(I_{\delta}(0,\lambda), J_{\varepsilon}(0,\nu))=2
$
if 
$
\delta \varepsilon=+, (\lambda, \nu) \in L_{\operatorname{even}}
$
 or 
$
\delta \varepsilon=-, (\lambda, \nu) \in L_{\operatorname{odd}}.
$   
\item[{\rm{(c)}}]
Case $i=n$.  
Then 
$
   m(I_{\delta}(n,\lambda), J_{\varepsilon}(n-1,\nu))=2
$
 if 

$\delta \varepsilon=+, (\lambda, \nu) \in L_{\operatorname{even}}
$
 or 
$
\delta \varepsilon=-, 
(\lambda, \nu) \in L_{\operatorname{odd}}.   
$
\end{enumerate}
\item[{\rm{(3)}}]
Suppose $j=i-2$ or $i+1$.  
Then $m(\Pi,\pi)=1$ 
 if one of the following conditions {\rm{(d)--(g)}}
 is satisfied, 
 and $m(\Pi,\pi)=0$ otherwise.  
\begin{enumerate}
\item[{\rm{(d)}}]
Case $j=i-2$, $2 \le i \le n-1$, $(\lambda, \nu)=(n-i,n-i+1)$, 
$\delta \varepsilon =-1$.  
\item[{\rm{(e)}}]
Case $(i,j)=(n,n-2)$, $-\lambda \in {\mathbb{N}}$, $\nu=1$, 
$\delta \varepsilon =(-1)^{\lambda+1}$.  
\item[{\rm{(f)}}]
Case $j=i+1$, $1 \le i \le n-2$, $(\lambda, \nu)=(i,i+1)$, 
$\delta \varepsilon =-1$.  
\item[{\rm{(g)}}]
Case $(i,j)=(0,1)$, $-\lambda \in {\mathbb{N}}$, $\nu=1$, 
$\delta \varepsilon =(-1)^{\lambda+1}$.  
\end{enumerate}
\end{enumerate}
\end{theorem}


The proof of Theorem \ref{thm:1.1} will be given right after Theorem \ref{thm:SBObasis}, 
by using Fact \ref{fact:2.7}
 and Theorems \ref{thm:VWSBO} and \ref{thm:161243}, 
 whose proofs are deferred at later chapters.  

%%%%%%%%%%%%%%%%%%%%%%%%%%%%%%%%%%%%%%%%%%%%%%%%%%%%%%%%%%%%%
\subsubsection{Classification of symmetry breaking operators
 $I_{\delta}(i,\lambda) \to J_{\varepsilon}(j,\nu)$}
\label{subsec:SBOthm}
%%%%%%%%%%%%%%%%%%%%%%%%%%%%%%%%%%%%%%%%%%%%%%%%%%%%%%%%%%%%%
In this subsection,
 we give explicit generators
 of
\[
   {\operatorname{Hom}}_{G'}
   (I_{\delta}(i,\lambda)|_{G'},J_{\varepsilon}(j,\nu)), 
\]
of which the dimension is determined in Theorem \ref{thm:1.1}.  
For most of the cases,
 the regular symmetry breaking operators
 $\Atbb {\lambda} {\nu} \pm {i,j}$
 and the differential symmetry breaking operators
 $\Ctbb \lambda \nu {i,j}$ give the generators.  
However,  for the exceptional discrete set classified in Theorem \ref{thm:161243}, 
 we need more operators
 which are defined as follows:
 for $(\lambda_0, \nu_0)\in {\mathbb{C}}^2$
 such that $\Atbb {\lambda_0} {\nu_0} \pm {i,j}=0$, 
 we renormalize the regular symmetry breaking operators
 $\Atbb \lambda \nu \pm {i,j}$
 as follows 
 (see Section \ref{subsec:161702}).  
For $j=i$ of $i-1$, 
 we set 
\index{A}{Ahttsln1@$\Attbb \lambda \nu {+} {i,j}$|textbf}
\index{A}{Ahttsln2@$\Attbb \lambda \nu {-} {i,j}$|textbf}
\begin{align}
\label{eqn:Aij+re}
\Attbb {\lambda_0} {\nu_0} {+} {i,j}
:=&
\lim_{\lambda \to \lambda_0}
\Gamma(\frac{\lambda-\nu_0}{2})
\Atbb \lambda {\nu_0} {+} {i,j},
\\
\label{eqn:Aij+re}
\Attbb {\lambda_0} {\nu_0} {-} {i, j}
:=&
\lim_{\lambda \to \lambda_0}
\Gamma(\frac{\lambda-\nu_0+1}{2})
\Atbb \lambda {\nu_0} {-} {i,j}.  
\end{align}
Then $\Attbb \lambda \nu {\pm} {i,j}$ are well-defined
 and nonzero symmetry breaking operators
(Theorem \ref{thm:170340}).  



For $j \in \{i-1,i\}$ and $\gamma \in \{\pm\}$, 
 the set
\[
  \{
(\lambda, \nu)\in {\mathbb{C}}^2
:
  \Atbb \lambda \nu \gamma {i,j}=0
\}
\]
 is classified in Theorem \ref{thm:161243}.  
Then we are ready to give an explicit basis
 of symmetry breaking operators:
\begin{theorem}
[generators]
\label{thm:SBObasis}
Suppose $j=i$ or $i-1$.  
\begin{enumerate}
\item[{\rm{(1)}}]
$m(I_{\delta}(i,\lambda),J_{\varepsilon}(j,\nu))=1$
 if and only if
 $\Atbb \lambda \nu {\delta\varepsilon} {i,j} \ne 0$.  
In this case
\[
   \operatorname{Hom}_{G'}
  (I_{\delta}(i,\lambda)|_{G'},J_{\varepsilon}(j,\nu))
  =
  {\mathbb{C}}\Atbb \lambda \nu {\delta\varepsilon} {i,j}.  
\]
\item[{\rm{(2)}}]
$m(I_{\delta}(i,\lambda),J_{\varepsilon}(j,\nu))=2$
 if and only if
 $\Atbb \lambda \nu {\delta\varepsilon} {i,j} = 0$.  
In this case
\[
   \operatorname{Hom}_{G'}
  (I_{\delta}(i,\lambda)|_{G'},J_{\varepsilon}(j,\nu))
  =
  {\mathbb{C}}\Attbb \lambda \nu {\delta\varepsilon} {i,j}
  \oplus
  {\mathbb{C}} \Ctbb \lambda \nu {i,j}.  
\]
\end{enumerate}
\end{theorem}

See Theorem \ref{thm:161243} for the necessary 
 and sufficient condition on $(i,j,\lambda,\nu,\gamma)$
 for $\Atbb \lambda \nu \gamma {i,j}$ to vanish.  

\begin{remark}
For $j=i+1$ or $i-2$, 
all symmetry breaking operators
 are differential operators
 by the localness theorem
\index{B}{localnesstheorem@localness theorem}
 (Theorem \ref{thm:152347}), 
 and the generators are given in Fact \ref{fact:3.9}. 
\end{remark}



\begin{proof}
[Proof of Theorems \ref{thm:1.1} and \ref{thm:SBObasis}]
We apply the general scheme
 of symmetry breaking operators
 (Theorem \ref{thm:VWSBO})
 to the special setting:
\[
   V=\Exterior^i({\mathbb{C}}^n)
\quad
  \text{and}
\quad
   W=\Exterior^j({\mathbb{C}}^{n-1}).  
\]
Then the theorems follow from the explicit description
 of the zero sets
 of the (normalized) regular symmetry breaking operators
 $\Atbb \lambda \nu \gamma {i,j}$
 (Theorem \ref{thm:161243})
 and the classification
 of {\it{differential}} symmetry breaking operators
 (Fact \ref{fact:2.7}).  
\end{proof}

\begin{remark}
The first statement ({\it{i.e.}}, $\delta \varepsilon=+$ case)
 of Theorem \ref{thm:1.1} (2) (b)
 was established in \cite[Thm.~1.1]{sbon},
 and the second statement ({\it{i.e.}}, $\delta \varepsilon=-$ case) of (b)
 can be proved similarly.  
In this article,
 we take another approach for the latter case:
we deduce results
 for all the matrix-valued cases
 (including the scalar-valued case with $\delta \varepsilon=-$) from 
 the scalar valued case with $\delta \varepsilon=+$.  
\end{remark}



%%%%%%%%%%%%%%%%%%%%%%%%%%%%%%%%%%%%%%
\subsection{Consequences of main theorems in Sections \ref{subsec:170228} and \ref{subsec:exhaust}}
\label{subsec:3.6}
%%%%%%%%%%%%%%%%%%%%%%%%%%%%%%%%%%%%%%%%%%
In this section
 we discuss symmetry breaking from principal series representations
 $\Pi=I_{\delta}(V,\lambda)$ of $G$
 to $\pi=J_{\varepsilon}(W,\nu)$ of the subgroup $G'$
 in the case
 where $\Pi$ and $\pi$ are {\it{unitarizable}}.  
Unitary principal series representations
 are treated in Section \ref{subsec:SBtemp}, 
 and complementary series representations
 are treated
 in Sections \ref{subsec:SBcomp} and \ref{subsec:singcomp}.  
We note that $\Pi$ and $\pi$ are irreducible
 in these cases.  
On the other hand,
 if $\lambda$ (resp. $\nu$) is integral, 
 then $\Pi$ (resp. $\pi$) may be reducible.  
We shall discuss symmetry breaking operators
 for the subquotients 
 in the next chapter
 in detail
 when they have the trivial infinitesimal character $\rho$.  


%%%%%%%%%%%%%%%%%%%%%%%%%%%%%%%%%%%%%%%%%%%%
\subsubsection{Tempered representations}
\label{subsec:SBtemp}
%%%%%%%%%%%%%%%%%%%%%%%%%%%%%%%%%%%%%%%%%%%%%%
We recall the concept of tempered unitary representations
 of locally compact groups.  
\begin{definition}
[tempered unitary representation]
\label{def:tempered}
A unitary representation of a unimodular group $G$ is called 
\index{B}{temperedrep@tempered representation|textbf}
{\it{tempered}}
 if it is weakly contained
 in the regular representations 
 on $L^2(G)$.  
By a little abuse of notation, 
 we also say the smooth representation 
 $\Pi^{\infty}$
 is {\it{tempered}}.  
\end{definition}
Returning to our setting 
 where $(G,G')=(O(n+1,1), O(n,1))$, 
 we see that the principal series representations $I_\delta (V,\lambda)$ and $J_\varepsilon (W,\nu)$ are tempered
 if and only if $\lambda \in \sqrt{-1}{\mathbb{R}}+\frac n 2$ and $\nu \in \sqrt{-1}{\mathbb{R}}+\frac 1 2 (n-1)$, 
 respectively.  
We refer to them as {\it{tempered principal series representations}}.  



We recall 
$
   [V:W]=\dim_{\mathbb{C}} {\operatorname{Hom}}_{O(n-1)}(V|_{O(n-1)}, W).
$
Then Theorem \ref{thm:VWSBO} implies the following:
\begin{theorem}
[tempered principal series representations]
\label{thm:tempVW}
Let $(\sigma, V) \in \widehat {O(n)}$, 
 $(\tau, W) \in \widehat {O(n-1)}$, 
 $\delta, \varepsilon \in \{\pm\}$, 
 and $\lambda \in \sqrt{-1}{\mathbb{R}}+\frac n 2$, 
 $\nu \in \sqrt{-1}{\mathbb{R}}+\frac 1 2(n-1)$
 so that $I_\delta(V,\lambda)$ and $J_\varepsilon(W,\nu)$ are tempered principal series representations. 
Then the following four conditions are equivalent:
\begin{enumerate}
\item
[{\rm{(i)}}]
\index{A}{VWmult@$[V:W]$}
$[V:W] \ne 0;$
\item
[{\rm{(i$'$)}}]
$[V:W] = 1;$
\item
[{\rm{(ii)}}]
$
 {\operatorname{Hom}}_{G'}
 (I_\delta(V,\lambda)|_{G'}, J_\varepsilon(W,\nu))\not = \{0\};
$
\item
[{\rm{(ii$'$)}}]
$
\dim_{\mathbb{C}} {\operatorname{Hom}}_{G'}(I_\delta(V,\lambda)|_{G'}, J_\varepsilon(W,\nu)) =1.  
$
\end{enumerate}
\end{theorem}



Applying Theorem \ref{thm:tempVW} to the exterior tensor representations
 $V=\Exterior^i({\mathbb{C}}^n)$ of $O(n)$
 and $W=\Exterior^j({\mathbb{C}}^{n-1})$ of $O(n-1)$, 
 we get:
\begin{corollary}
\label{cor:tempered}
Suppose $\lambda \in \sqrt{-1}{\mathbb{R}}+ \frac n 2$, 
 and $\nu \in \sqrt{-1}{\mathbb{R}}+\frac 1 2(n-1)$.  
Then  
\[
   \dim_{\mathbb{C}} {\operatorname{Hom}}_{G'}(I_\delta(i,\lambda)|_{G'}, J_\varepsilon (j,\nu))
=
\begin{cases}
1 \quad&\text{if $i=j$ or $j=i-1$}, 
\\
0      &\text{otherwise}.
\end{cases}
\]
\end{corollary}



%%%%%%%%%%%%%%%%%%%%%%%%%%%%%%%%%%%%%%%%%%%%%%%%%%%%%
\subsubsection{Complementary series representations}
\label{subsec:SBcomp}
%%%%%%%%%%%%%%%%%%%%%%%%%%%%%%%%%%%%%%%%%%%%%%%%%%%%%
We say that $I_\delta(V,\lambda)$ is a (smooth)
\index{B}{complementaryseries@complementary series representation|textbf}
 {\it{complementary series  representation}}
 if it has a Hilbert completion
 to a unitary complementary series representation.  
If the irreducible $O(n)$-module $(\sigma,V)$ is of
\index{B}{typeX@type X, representation of ${O(N)}$\quad}
 type X
 (see Definition \ref{def:OSO}), 
 {\it{i.e.,}} 
 the last digit of the highest weight of $V$ is not zero, 
 then the principal series representation 
 $I_\delta(V,\lambda)$ is irreducible at $\lambda = \frac n 2$, 
 and consequently, 
 there exist complementary series representations
 $I_\delta(V,\lambda)$
 for some interval $\lambda \in (\frac n 2-a, \frac n 2+a)$ with $a >0$.  
\begin{example}
\label{ex:Iicompl}
Suppose $(\sigma,V)$ is the $i$-th exterior tensor representation
 $\Exterior^i({\mathbb{C}}^n)$.  
We assume that this representation is of type X, 
 equivalently,
 $n \ne 2i$
 (see Example \ref{ex:2.1}).  
The first reduction point of the principal series representation
 of $I_{\delta}(i,\lambda)$ is given by $\lambda=i$ or $n-i$
 (see Proposition \ref{prop:redIilmd}).  
Therefore $I_\delta(i,\lambda) \equiv I_\delta(\Exterior^i({\mathbb{C}}^n),\lambda)$
 is a complementary series representation
 if 
\[
  {\operatorname{min}}(i,n-i) < \lambda < {\operatorname{max}}(i,n-i).  
\]
\end{example}


In the category of unitary representations, 
 the restriction of a tempered representation of $G$
 to a reductive subgroup $G'$
 decomposes into the direct integral 
 of irreducible unitary tempered representations
 of a reductive subgroup $G'$
 because it is weakly contained 
 in the regular representation.  
In particular, 
 complementary series representations of the subgroup $G'$ do not appear
 in the {\it{unitary}} branching law
 of the restriction of a unitary tempered principal series  representation 
 $I_\delta(V,\lambda)$, 
 whereas Theorem \ref{thm:VWSBO}
 in the category of admissible {\it{smooth}} representations shows that there are  nontrivial symmetry breaking operators 
\[ 
  \Atbb \lambda \nu {\delta\varepsilon} {V,W} : I_\delta(V,\lambda) \rightarrow J_\varepsilon(W,\nu)
\]
 to all complementary series representations $J_\delta(W,\nu)$
 of the subgroup $G'$ if $[V:W] \not = 0.$



Moreover,
 Theorem \ref{thm:VWSBO} (2) implies also 
 that there are nontrivial symmetry breaking operators from
 any (smooth) complementary series representation
 $I_{\delta}(V,\lambda)$
 of $G$ to all (smooth) tempered principal series representations
 $J_{\varepsilon}(W,\nu)$ of the subgroup $G'$
 as far as $[V:W] \ne 0$.



%%%%%%%%%%%%%%%%%%%%%%%%%%%%%%%%%%%%%%%%%%%%%%%%%%%%%%%%%%%%%%
\subsubsection{Singular complementary series representations}
\label{subsec:singcomp}
%%%%%%%%%%%%%%%%%%%%%%%%%%%%%%%%%%%%%%%%%%%%%%%%%%%%%%%%%%%%%%
We consider the complementary series representations $I_\delta(i,s)$ 
 for $i < s< \frac n 2$ 
 with an additional assumption
 that $s$ is an integer.  
These representations are irreducible
 and have {\it{singular}} integral infinitesimal characters. 
We may describe the underlying $({\mathfrak{g}},K)$-modules
 of these singular complementary series representations
 in terms of cohomological parabolic induction 
\index{A}{Aqlmd@$A_{\mathfrak{q}}(\lambda)$}
 $A_{\mathfrak{q}}(\lambda)$
 where the parameter $\lambda$ wanders outside
\index{B}{goodrange@good range}
 the good range
 relative to the $\theta$-stable parabolic subalgebra ${\mathfrak{q}}$
 (see \cite[Def.~0.49]{KV} for the definition).  


For $0 \le r \le \frac{n+1}{2}$, 
 we denote by
\index{A}{qi@${\mathfrak{q}}_i$, $\theta$-stable parabolic subalgebra}
 ${\mathfrak{q}}_r$ the $\theta$-stable parabolic subalgebra
 of ${\mathfrak{g}}_{\mathbb{C}}={\mathfrak{o}}(n+2,{\mathbb{C}})$
 with Levi factor $SO(2)^r \times O(n-2r+1,1)$
 in $G=O(n+1,1)$
 (see Definition \ref{def:qi}).  
\begin{lemma}
\label{lem:compAq}
Let $0 \le i \le [\frac n 2]-1$.  
For $s \in \{i+1,i+2,\cdots,[\frac n 2]\}$, 
 we have an isomorphism as $({\mathfrak{g}},K)$-modules:
\[
  I_+(i,s)_K \simeq A_{\mathfrak{q}_{i+1}}(0,\cdots,0,s-i).  
\]
\end{lemma}
See Remark \ref{rem:goodrange} in Appendix I
 for the normalization 
 of the $({\mathfrak{g}},K)$-module $A_{\mathfrak{q}}(\lambda)$
 and Theorem \ref{thm:compint} for more details
 about Lemma \ref{lem:compAq}.  
See also \cite[Thm.~3]{KMemoirs92} for some more general cases.  
The restriction of these representations
 to the special orthogonal group $SO(n+1,1)$ stays irreducible
 (see Lemma \ref{lem:170305} in Appendix II). 
Bergeron and Clozel proved
 that there are  automorphic square integrable representations,
 whose component at infinity is isomorphic
 to a representation $I_\delta(i,s)|_{SO(n+1,1)}$
 (see \cite{BC, BLS}). 

A special case of Theorem \ref{thm:1.1} includes:
\begin{proposition}
\label{prop:3.32}
Suppose $s \in {\mathbb{N}}$ and $i < s \le [\frac n 2]$.  
Let $\delta$, $\varepsilon \in \{\pm\}$.  
\newline\noindent
{\rm{(1)}}\enspace
For $i < r \le [\frac{n-1}2]$, 
\[ 
   {\operatorname{Hom}}_{G'}(I_{\delta}(i,s)|_{G'},J_{\varepsilon}(i,r))
 = {\mathbb{C}}.  
\]
\newline\noindent
{\rm{(2)}}\enspace
For $0 \le i-1 <r \le [\frac{n-1}{2}]$, 
\[  
   {\operatorname{Hom}}_{G'}(I_{\delta}(i,s)|_{G'},J_{\varepsilon}(i-1,r)) = {\mathbb{C}}.  
\]
\end{proposition}


\begin{remark}
Proposition \ref{prop:3.32} may be viewed as symmetry breaking operators from 
 the Casselman--Wallach globalization
 of the irreducible $({\mathfrak{g}},K)$-module $A_{\mathfrak{q}}(\lambda)$ to that of the irreducible $({\mathfrak{g}}',K')$-module $A_{\mathfrak{q}'}(\nu)$
 in some special cases 
 where both $\lambda$ and $\nu$ are 
 outside the good range of parameters
 relative to the $\theta$-stable parabolic subalgebras.  
\end{remark}

In the next chapter,
 we treat the case
 with trivial infinitesimal character $\rho$,
 and thus the parameters stay in the good range
 relative to the $\theta$-stable parabolic subalgebras.  
In particular,
 we shall determine a necessary and sufficient condition
 for a pair $({\mathfrak{q}}, {\mathfrak{q}}')$
 of $\theta$-stable parabolic subalgebras ${\mathfrak{q}}$ of ${\mathfrak{g}}_{\mathbb{C}}$
 and ${\mathfrak{q}}'$ of its subalgebra ${\mathfrak{g}}_{\mathbb{C}}'$
 such that 
\[
  {\operatorname{Hom}}_{G'}(\Pi|_{G'}, \pi) \ne \{0\}, 
\]
 when the underlying $({\mathfrak{g}},K)$-module $\Pi_K$
 of $\Pi \in {\operatorname{Irr}}(G)$
 is isomorphic to 
\index{A}{Aqlmdac@$(A_{{\mathfrak{q}}})_{\pm\pm}$}
 $(A_{\mathfrak{q}})_{\pm\pm}$ 
 and the underlying $({\mathfrak{g}}',K')$-module
 of $\pi \in {\operatorname{Irr}}(G')$
 is $(A_{\mathfrak{q}'})_{\pm \pm}$, 
 see Theorems \ref{thm:SBOvanish} and \ref{thm:SBOone}
 for the multiplicity-formula, 
 and Proposition \ref{prop:161655} in Appendix I
 for the description of $\Pi_K$ in terms of $(A_{\mathfrak{q}})_{\pm\pm}$.  
In contrast to the case of Proposition \ref{prop:3.32}, 
 the irreducible $G$-module $\Pi$
 and $G'$-module $\pi$ do not coincide with principal series representations,
 but appear as their subquotients in this case, 
 see Theorem \ref{thm:LNM20} (1).  



%%%%%%%%%%%%%%%%%%%%%%%%%%%%%%%%%%%%%%%%%%%%%%%%%%%
\subsection{Actions
 of $(G/G_0)
\hspace{1mm}
{\widehat{}}
\, 
\times (G'/G_0')
\hspace{1MM}
{\widehat{}}
\,$ on symmetry breaking operators}
\label{subsec:actPont}
%%%%%%%%%%%%%%%%%%%%%%%%%%%%%%%%%%%%%%%%%%%%%%%%%

In this section 
 we discuss the action of the character group of $G \times G'$
 on the set 
\[
  \{ {\operatorname{Hom}}_{G'}(\Pi|_{G'}, \pi) \}
\]
 of the spaces of symmetry breaking operators 
 where admissible smooth representations $\Pi$ of $G$
 and those $\pi$ of the subgroup $G'$ vary.  
Actual computations for the pair
 $(G,G')=(O(n+1,1),O(n,1))$ are carried out
 by using Lemma \ref{lem:IVchi}
 for principal series representations
 and Theorem \ref{thm:LNM20} (5) for their irreducible subquotients.  
%%%%%%%%%%%%%%%%%%%%%%%%%%%%%%%%%%%%%%%%%%%%%%%%%
\subsubsection{Generalities: The action of character group of 
 $G \times G'$ on $\{{\operatorname{Hom}}_{G'}(\Pi|_{G'}, \pi)\}$
 in the general case}
%%%%%%%%%%%%%%%%%%%%%%%%%%%%%%%%%%%%%%%%%%%%%%%%%%
Let $G \supset G'$ be a pair of real reductive Lie groups.  
Then the character group of $G \times G'$
 acts on the set of vector spaces
 $\{{\operatorname{Hom}}_{G'}(\Pi|_{G'}, \pi)\}$
 where $\Pi$ runs over admissible smooth representations of $G$, 
 and $\pi$ runs over those of the subgroup $G'$.  
Here the action is given by 
\[
   {\operatorname{Hom}}_{G'}(\Pi|_{G'}, \pi)
   \mapsto
   {\operatorname{Hom}}_{G'}((\Pi \otimes \chi^{-1})|_{G'}, \pi \otimes \chi')
\]
for a character $\chi$ of $G$
 and $\chi'$ of the subgroup $G'$.  



In what follows, 
 we regard a character of $G$ as a character of $G'$
 by restriction,
 and use the same letter
 to denote its restriction to the subgroup $G'$.  
Then for all characters $\chi$ and $\chi'$ of $G$, 
 we have the following isomorphisms:
\begin{alignat}{2} 
\mbox{Hom}_{G'} ((\Pi\otimes \chi)|_{G'}, \pi \otimes \chi' ) 
&\simeq\,\,
&&\mbox{Hom}_{G'}(\Pi|_{G'},\pi\otimes \chi^{-1}  \otimes \chi') 
\notag
\\
&\simeq 
&&\mbox{Hom}_{G'} ((\Pi\otimes (\chi')^{-1})|_{G'}, \pi \otimes \chi^{-1}) 
\notag
\\
&\simeq 
&&\mbox{Hom}_{G'} ((\Pi\otimes \chi \otimes (\chi')^{-1})|_{G'}, \pi).
\label{eqn:314}
\end{alignat}



The above isomorphisms define an equivalence relation
 on the set 
\[
   \{{\operatorname{Hom}}_{G'}(\Pi|_{G'}, \pi)\}
\]
 of the spaces of symmetry breaking operators
 where $\Pi$ and $\pi$ vary. 



%%%%%%%%%%%%%%%%%%%%%%%%%%%%%%%%%%%%%%%%%%%%%%%%%%%%%%%%%%%%%
\subsubsection{Actions of the character group
 of the component group
 on 
 $\{{\operatorname{Hom}}_{G'}
  (I_{\delta}(i,\lambda)|_{G'}, J_{\varepsilon}(j,\nu))\}$}
\label{subsec:chiIilmd}
%%%%%%%%%%%%%%%%%%%%%%%%%%%%%%%%%%%%%%%%%%%%%%%%%%%%%%%%%%%%%
We apply the above idea to our setting
\[
  (G,G') = (O(n+1,1), O(n,1)).  
\]
Then the 
\index{B}{componentgroup@component group $G/G_0$}
component groups of $G$ and $G'$ are a finite abelian group given by
\begin{equation}
\label{eqn:chiabrest}
   G'/G_0' \simeq G/G_0 
         \simeq {\mathbb{Z}}/2{\mathbb{Z}} \times {\mathbb{Z}}/2{\mathbb{Z}}.
\end{equation}



We recall from \eqref{eqn:chiab}
 that the set of their one-dimensional representations
 is parametrized by 
\[
(G'/G_0')\hspace{-1mm}{\widehat{\hphantom{m}}} 
\simeq (G/G_0)\hspace{-1mm}{\widehat{\hphantom{m}}} 
           = \{\chi_{a b}: a, b \in \{\pm\}\}.  
\]
By abuse of notation,
 we shall use the same letters $\chi_{a b}$
 to denote the corresponding one-dimensional representations
 of $G$, $G'$, $G/G_0$, and $G'/G_0'$.  



The action of the character group 
 ({\it{Pontrjagin dual}})
 $(G/G_0)\hspace{-1mm}{\widehat{\hphantom{m}}}$
 on the set of principal series representations
 can be computed
 by using Lemma \ref{lem:IVchi}.  
To describe the action of the Pontrjagin dual 
 $(G/G_0)\hspace{-1mm}{\widehat{\hphantom{m}}} \simeq
  (G'/G_0')\hspace{-1mm}{\widehat{\hphantom{m}}}$
 on the parameter set of the principal series representations $I_{\delta}(i,\lambda)$ of $G$
 and $J_{\varepsilon}(j,\nu)$ of the subgroup $G'$, 
 we define
\begin{alignat*}{3}
   &S:=\{0,1,\cdots,n\} \times {\mathbb{C}} \times {\mathbb{Z}}/2{\mathbb{Z}}, 
 \quad
   &&I(s):=I_{\delta}(i,\lambda)
\quad
  &&\text{for } s =(i,\lambda,\delta)\in S, 
\\
   &T:=\{0,1,\cdots,n-1\} \times {\mathbb{C}} \times {\mathbb{Z}}/2{\mathbb{Z}}, \quad
   &&J(t):=J_{\varepsilon}(j,\nu)
\quad
  &&\text{for } t =(j,\nu,\varepsilon)\in T.   
\end{alignat*}
We let the character group $(G/G_0)\hspace{-1mm}{\widehat{\hphantom{m}}}$ act on $S$
 by the following formula:
\begin{alignat*}{3}
\chi_{++} \cdot (i,\lambda,\delta):=&(i,\lambda,\delta),
\quad
&&\chi_{+-} \cdot (i,\lambda,\delta):=&&(i,\lambda,-\delta),
\\
\chi_{-+} \cdot (i,\lambda,\delta):=&(\tilde i,\lambda,-\delta),
\quad
&&\chi_{--} \cdot (i,\lambda,\delta):=&&(\tilde i,\lambda,\delta),
\end{alignat*}
where $\tilde i:=n-i$.  
The action of $(G'/G_0')\hspace{-1mm}{\widehat{\hphantom{m}}}$ on the set $T$
 is defined similarly,
 with obvious modification
\[
\tilde j:=n-1-j
\]
 when we discuss representations
 of the subgroup $G'=O(n,1)$.  
By Lemma \ref{lem:IVchi}
 and by the $O(n)$-isomorphism
 $\Exterior^i({\mathbb{C}}^n) \simeq \Exterior^{n-i}({\mathbb{C}}^n) \otimes \det$, 
 we obtain the following.  
\begin{lemma}
\label{lem:LNM27}
For all $\chi \in (G/G_0)\hspace{-1mm}{\widehat{\hphantom{m}}}\simeq (G'/G_0')\hspace{-1mm}{\widehat{\hphantom{m}}}$
 and for $s \in S$, $t \in T$, 
we have the following isomorphisms as $G$-modules
 and $G'$-modules, 
respectively:
\begin{align*}
   I(s) \otimes \chi &\simeq I(\chi \cdot s), 
\\
   J(t) \otimes \chi &\simeq J(\chi \cdot t).  
\end{align*}
\end{lemma}


Then the equivalence defined by the isomorphisms
 \eqref{eqn:314} implies
 that it suffices to consider symmetry breaking operators
 for $(\delta, \varepsilon) =(+,+)$
 and $(\delta, \varepsilon) =(+,-)$.
To be more precise,
 we obtain the following.  

\begin{proposition} 
Let $\lambda, \nu \in {\mathbb{C}}$.  
Then every symmetry breaking operator in 
\[
     \bigcup_{\delta,\varepsilon \in \{\pm\} }
     \bigcup_{0 \le i \le n} \bigcup_{0 \le j \le n-1}
     \operatorname{Hom}_{G'}(I_{\delta}(i,\lambda)|_{G'},J_{\varepsilon}(j,\nu))
\]
is equivalent to a symmetry breaking  operator in
\[ 
 \bigcup_{0 \le i \le [\frac n2]} \bigcup_{0 \le j \le n-1}
 \left(\operatorname{Hom}_{G'}(I_+(i,\lambda)|_{G'}, J_+(j,\nu)) 
 \cup
 \operatorname{Hom}_{G'}(I_+(i,\lambda)|_{G'}, J_-(j,\nu))
 \right).  
\]
\end{proposition}
\proof
We use a graph to prove this.
We set
\begin{eqnarray*} 
(\delta,\varepsilon) &:= & \mbox{Hom}_{G'}(I_{\delta} (i, \lambda)|_{G'}, J_{\varepsilon}(j,\nu)), 
\\
\begin{pmatrix} \delta \\ \varepsilon \end{pmatrix} 
&:=& \operatorname{Hom}_{G'}(I_{\delta} (n-i, \lambda)|_{G'}, J_{\varepsilon}(n-j-1,\nu)).  
\end{eqnarray*}
In the following graph the nodes are indexed by $(\delta,\varepsilon)$
 in first row and by
 $\begin{pmatrix} \delta \\ \varepsilon \end{pmatrix}$
 in the second row.  
The nodes are connected by a line if they are equivalent.  
By Lemma \ref{lem:LNM27}, 
 we obtain the graph by taking 
\index{A}{1chipm@$\chi_{+-}$}
$\chi=\chi'=\chi_{+-}$
 in \eqref{eqn:314}
 for horizontal equivalence,
 and 
\index{A}{1chipmm@$\chi_{--}=\det$}
$\chi=\chi'=\chi_{-+}$ in \eqref{eqn:314}
 for crossing equivalence
 (we omit here lines in the graph
 corresponding to 
 $\chi=\chi'=\chi_{--}$ in \eqref{eqn:314}
 for vertical equivalence):

\begin{center}
\unitlength.9cm
\begin{picture}(14, 8.4)
\put(0,6){\makebox(2,1){$(+,+)$}}
\put(3.7,6){\makebox(2,1){$(+,-)$}}
\put(8.3,6){\makebox(2,1){$(-,+)$}}
\put(12,6){\makebox(2,1){$(-,-)$}}

\put(7,7){\oval(12,1.8)[t]}
\put(7,7){\oval(4.6,1.2)[t]}

\put(0,1.9){\makebox(2,1){$\begin{pmatrix}+\\+ \end{pmatrix}$}}
\put(3.7,1.9){\makebox(2,1){$\begin{pmatrix}+\\- \end{pmatrix}$}}
\put(8.3,1.9){\makebox(2,1){$\begin{pmatrix}-\\+ \end{pmatrix}$}}
\put(12,1.9){\makebox(2,1){$\begin{pmatrix}-\\- \end{pmatrix}$}}

\put(7,1.6){\oval(12,1.8)[b]}
\put(7,1.6){\oval(4.6,1.2)[b]}

\color{magenta}
\put(1,3.2){\line(4,1){11.95}}
\put(1,6.15){\line(4,-1){11.95}}
\put(4.7,3.2){\line(3,2){4.5}}
\put(4.7,6.15){\line(3,-2){4.5}}
\end{picture}
\color{black}

\end{center}



We observe that there are exactly two connected components of the graph, 
  and that 
$\mbox{Hom}_{G'}(I_+(i,\lambda)|_{G'}, J_+(j,\nu))$ and
$\mbox{Hom}_{G'}(I_+(i,\lambda)|_{G'}, J_-(j,\nu))$ are
 in a different connected component.  
Moreover,
 we may choose $i$ or $n-i$
 in the same equivalence classes, 
 and thus we may take $0 \le i \le \frac n 2$
 as a representative.  
\qed



\begin{example}
\label{ex:tempdual}
\begin{enumerate}
\item[{\rm{(1)}}]
Suppose $n=2m$ and $i=m$.  
Applying the isomorphism \eqref{eqn:314} to $(\Pi,\pi)=(I_{\delta}(m,\lambda), J_{\varepsilon}(m,\nu))$
 with $\chi=\chi'=\chi_{--}$, 
 we obtain a natural bijection:
\[
   {\operatorname{Hom}}_{G'}
    (I_{\delta}(m,\lambda)|_{G'}, J_{\varepsilon}(m,\nu))
    \simeq
    {\operatorname{Hom}}_{G'}
    (I_{\delta}(m,\lambda)|_{G'}, J_{\varepsilon}(m-1,\nu)).  
\]
We note that the $G$-module $I_{\delta}(m,\lambda)$
 at $\lambda=m$ splits into the direct sum
 of two irreducible smooth tempered representations
 (Theorem \ref{thm:LNM20} (1) and (8)).  
\item[{\rm{(2)}}]
Suppose $n=2m+1$ and $i=m$.  
Similarly to the first statement,
 we have a natural bijection:
\[
   {\operatorname{Hom}}_{G'}
    (I_{\delta}(m,\lambda)|_{G'}, J_{\varepsilon}(m,\nu))
    \simeq
    {\operatorname{Hom}}_{G'}
    (I_{\delta}(m+1,\lambda)|_{G'}, J_{\varepsilon}(m,\nu)).  
\]
In this case,
 the $G'$-module $J_{\varepsilon}(m,\nu)$ at $\nu=m$
 splits into the direct sum
 of two irreducible smooth tempered representations.  
\end{enumerate}
\end{example}

\subsubsection{Actions of characters of the component group 
 on ${\operatorname{Hom}}_{G'}(\Pi_{i,\delta}|_{G'}, \pi_{j,\varepsilon})$}
In the next chapter,
 we discuss 
\[
 {\operatorname{Hom}}_{G'}(\Pi|_{G'}, \pi)
\]
 for $\Pi \in {\operatorname{Irr}}(G)_{\rho}$
 and $\pi \in {\operatorname{Irr}}(G')_{\rho}$.  
In this case,
 \eqref{eqn:314} implies the following:
\begin{proposition}
[duality for symmetry breaking operators]
\label{prop:SBOdual}
There are natural isomorphisms
\begin{align*}
   {\operatorname{Hom}}_{G'}(\Pi_{i,\delta}|_{G'}, \pi_{j,\varepsilon})
   \,\,\simeq\,\,&
   {\operatorname{Hom}}_{G'}(\Pi_{n+1-i,\delta}|_{G'}, \pi_{n-j,\varepsilon})
\\
   \,\,\simeq\,\,&
   {\operatorname{Hom}}_{G'}(\Pi_{i,-\delta}|_{G'}, \pi_{j,-\varepsilon})
\\
   \,\,\simeq\,\,&
{\operatorname{Hom}}_{G'}(\Pi_{n+1-i,-\delta}|_{G'}, \pi_{n-j,-\varepsilon}).  
\end{align*}
\end{proposition}
\begin{proof}
By Theorem \ref{thm:LNM20} (5),  
 we have a natural $G$-isomorphism
\index{A}{1chipmpm@$\chi_{\pm\pm}$, one-dimensional representation of $O(n+1,1)$}
$\Pi_{i,\delta} \otimes \chi_{-+} \simeq \Pi_{n+1-i,\delta}$
 and a $G'$-isomorphism
 $\pi_{j,\varepsilon} \otimes \chi_{-+} \simeq \pi_{n-j,\varepsilon}$.  
Hence the first isomorphism is derived from \eqref{eqn:314}.  
By taking the tensor product with $\chi_{+-}$, 
 we get the last two isomorphisms
 again by Theorem \ref{thm:LNM20} (5).  
\end{proof}


