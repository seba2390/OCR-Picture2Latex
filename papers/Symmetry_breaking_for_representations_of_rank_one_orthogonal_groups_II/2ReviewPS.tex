\newpage
\section{Review of principal series representations}
\label{sec:ps}
In this chapter
 we recall results about representations
 of the indefinite orthogonal group 
 $G = O(n+1,1)$.  

\subsection{Notation}
\label{subsec:Xi}

The object of our study is intertwining restriction operators
 ({\it{symmetry breaking operators}})
 between representations of $G=O(n+1,1)$
 and those of its subgroup $G'=O(n,1)$.  
Most of main results are stated 
 in a coordinate-free fashion, 
 whereas concrete description of symmetry breaking operators
 depends on coordinates.  
For the latter purpose,
 we choose subgroups of $G$ and $G'$
 in a compatible fashion.  
The notations here are basically taken from \cite{sbon}.  

\subsubsection{Subgroups of $G=O(n+1,1)$
 and $G'=O(n,1)$}
\label{subsec:subgpOn}
We define $G$ to be the indefinite orthogonal group 
 $O(n+1,1)$
 that preserves the quadratic form 
\begin{equation}
\label{eqn:Qn1}
Q_{n+1,1}(x)=x_0^2+ \cdots + x_{n}^2 - x_{n+1}^2
\end{equation}
of signature $(n+1,1)$.  
Let $G'$ be the stabilizer of the vector
 $e_n:= {}^{t\!}(0, \cdots,0,1,0)$.  
Then $G' \simeq O(n,1)$.  



We take maximal compact subgroups of $G$ and $G'$, 
 respectively, 
 as 
\index{A}{Kmaxcpt@$K=O(n+1) \times O(1)$, maximal compact subgroup of $G$|textbf}
\index{A}{Kmaxcptsub@$K'= K \cap G'$|textbf}
\begin{alignat*}{2}
K 
&:= O(n+2) \cap G
&&
\simeq O(n+1) \times O(1),
\\
K'
&:= K \cap G'
=\left\{
\begin{pmatrix}
                A & & \\ & 1 & \\ & & \varepsilon
\end{pmatrix}
:
 A \in O(n),\,\, \varepsilon =\pm 1
\right\}
&&
\simeq O(n) \times O(1).  
\end{alignat*}



Let ${\mathfrak{g}} = {\mathfrak{o}}(n+1,1)$
 and ${\mathfrak{g}}' = {\mathfrak{o}}(n,1)$
 be the Lie algebras
 of 
\index{A}{GOindefinitelarge@$G=O(n+1,1)$, Lorentz group|textbf}
$G=O(n+1,1)$
 and 
\index{A}{GOindefinitesprime@$G'=O(n,1)$|textbf}
$G'=O(n,1)$, 
 respectively.  
We take a hyperbolic element 
\index{A}{H@$H$|textbf}
\begin{equation}
\label{eqn:Hyp}
 H:= E_{0,n+1} + E_{n+1,0} \in {\mathfrak{g}}', 
\end{equation}
 and set 
\index{A}{0caLiealg@${\mathfrak{a}}$, maximally split abelian subspace|textbf}
\[
  {\mathfrak {a}}:={\mathbb{R}}H.  
\]
Then ${\mathfrak{a}}$ is a maximally split abelian subspace
 of ${\mathfrak{g}}'$, 
 as well as that of ${\mathfrak{g}}$.
The eigenvalues of 
 $\operatorname{ad} (H) \in \operatorname{End}({\mathfrak{g}})$
 are $\pm 1$ and $0$, 
 and the eigenspaces give rise to the following two maximal nilpotent subalgebras
 of ${\mathfrak {g}}$:
\begin{equation}
\label{eqn:ngen}
{\mathfrak {n}}_+=\operatorname{Ker} (\operatorname{ad}(H)-1)
                 =\sum_{j=1}^n {\mathbb{R}} N_j^+, 
\quad
{\mathfrak {n}}_-=\operatorname{Ker} (\operatorname{ad}(H)+1)
                 =\sum_{j=1}^n {\mathbb{R}} N_j^-, 
\end{equation}
where $N_j^+$ and $N_j^-$
 ($1 \le j \le n$) are nilpotent elements
 of ${\mathfrak{g}}$ defined by 
\index{A}{Nj1@$N_j^+$|textbf}
\index{A}{Nj2@$N_j^-$|textbf}
\begin{align*}
N_j^+=&-E_{0,j} + E_{j,0} - E_{j,n+1} -E_{n+1,j}, 
\\
N_j^-=&-E_{0,j} + E_{j,0} + E_{j,n+1} +E_{n+1,j}.  
\end{align*}



\medskip
For $b={}^t(b_1, \cdots, b_n) \in {\mathbb{R}}^n$, 
 we define unipotent matrices in $G$ by
\index{A}{n1@$n_+\colon {\mathbb{R}}^n \to N_+$|textbf}
\index{A}{n2@$n_-\colon {\mathbb{R}}^n \to N_-$|textbf}
\begin{align}
n_+(b):=& \exp(\sum_{j=1}^n b_j N_j^+)=I_{n+2} 
       + 
\begin{pmatrix} 
     - \frac 1 2 Q(b) &-{}^{t\!}b & \frac 1 2 Q(b) 
\\
       b & 0 & -b
\\
     - \frac 1 2 Q(b) &-{}^{t\!}b & \frac 1 2 Q(b) 
\end{pmatrix},
\label{eqn:nplus}
\\
n_-(b):=&\exp(\sum_{j=1}^n b_j N_j^-)= I_{n+2} 
       + 
\begin{pmatrix} 
     - \frac 1 2 Q(b) &-{}^{t\!}b & -\frac 1 2 Q(b) 
\\
       b & 0 & b
\\
     \frac 1 2 Q(b) &{}^{t\!}b & \frac 1 2 Q(b) 
\end{pmatrix}, 
\label{eqn:nbar}
\end{align}
where we set
\begin{equation}
\label{eqn:Qn}
Q(b)\equiv |b|^2 =\sum_{l=1}^n b_l^2.  
\end{equation}
Then 
\index{A}{n1@$n_+\colon {\mathbb{R}}^n \to N_+$|textbf}
$n_+$
 and 
\index{A}{n2@$n_-\colon {\mathbb{R}}^n \to N_-$|textbf}
$n_-$ give coordinates
 of the nilpotent groups 
\index{A}{N1@$N_+ = \exp ({\mathfrak{n}}_+)$|textbf}
 $N_+ := \exp ({\mathfrak{n}}_+)$ and 
\index{A}{N2@$N_- = \exp ({\mathfrak{n}}_-)$|textbf}
 $N_- := \exp ({\mathfrak{n}}_-)$, 
 respectively.  
Then $N_+$ stabilizes ${}^{t\!}(1,0,\cdots,0,1)$, 
 whereas $N_-$ stabilizes ${}^{t\!}(1,0,\cdots,0,-1)$.  




Since $H$ is contained in the Lie algebra ${\mathfrak {g}}'$, 
\[
  {\mathfrak{n}}_{\varepsilon}'
  :=
  {\mathfrak{n}}_{\varepsilon} \cap {\mathfrak{g}}'
  =
  \sum_{j=1}^{n-1} {\mathbb{R}} N_j^{\varepsilon}
\quad
  \text{for ${\varepsilon}=\pm$}
\]
are maximal nilpotent subalgebras 
 of ${\mathfrak{g}}'$.  
We set 
\index{A}{N1sub@$N_+' = N_+ \cap G'$|textbf}
 $N_+' := N_+ \cap G' = \exp ({\mathfrak{n}}_+')$
 and 
\index{A}{N2sub@$N_-' = N_- \cap G'$|textbf}
 $N_-' := N_- \cap G' = \exp ({\mathfrak{n}}_-')$.  



We define a split abelian subgroup $A$
 and its centralizers $M$ and $M'$ in $K$ and $K'$, 
 respectively, 
 as follows:
\index{A}{M1@$M=O(n) \times O(1)$, the centralizer of ${\mathfrak {a}}$ in $K$|textbf}
\index{A}{M1prime@$M'=O(n-1) \times O(1)$|textbf}
\begin{alignat*}{2}
A&:= \exp ({\mathfrak {a}}), 
&&
\\
M&:=\left\{ \begin{pmatrix} \varepsilon & & \\ &  B & \\ & & \varepsilon \end{pmatrix} : B \in O(n), \varepsilon = \pm 1\right\}
&& \simeq O(n) \times {\mathbb{Z}}/2 {\mathbb{Z}}, 
\\
M'
&:=\left\{ \begin{pmatrix} \varepsilon & & & \\ &  B & & \\ & & 1 & \\ & & & \varepsilon\end{pmatrix} : B \in O(n-1), \varepsilon = \pm 1\right\}
&& \simeq O(n-1) \times {\mathbb{Z}}/2 {\mathbb{Z}}.  
\end{alignat*}
Then 
\index{A}{PLanglandsdecomp@$P=MAN_+$, Langlands decomposition of 
a minimal parabolic subgroup of $G$|textbf}
$P= MAN_+$ is a Langlands decomposition
 of a minimal parabolic subgroup $P$ of $G$.  
Likewise,
\index{A}{PLanglandsdecompsubgp@$P'=M'AN_+'$|textbf}
 $P'= M' AN_+'$ is that
 of a minimal parabolic subgroup $P'$ of $G'$.  
We note
 that $A$ is a common maximally split abelian subgroup
 in $P'$ and $P$
 because we have chosen $H \in {\mathfrak {g}}'$.  
The Langlands decompositions
 of the Lie algebras of $P$ and $P'$
 are given in a compatible way 
 as 
\[
{\mathfrak {p}}={\mathfrak {m}}+{\mathfrak {a}}+{\mathfrak {n}}_+, 
\quad
 {\mathfrak {p}}'={\mathfrak {m}}'+{\mathfrak {a}}+{\mathfrak {n}}_+'
 =({\mathfrak {m}} \cap {\mathfrak {g}}') + ({\mathfrak {a}} \cap {\mathfrak {g}}')+({\mathfrak {n}}_+ \cap {\mathfrak {g}}').  
\]
We set 
\index{A}{m2@$m_-={\operatorname{diag}}(-1,1,\cdots,1,-1)$|textbf}
\begin{equation}\label{eqn:m-}
m_- \ := \ 
  \left(
   \begin{array}{cc|c}
      -1  &        &   \\
        & I_{n}&      \\
           \hline
        &        &      -1
    \end{array}
    \right)\in M'.  
   \end{equation}
We note
 that $m_-$ does not belong to the identity component
 of $G'$.  


\subsubsection{Isotropic cone $\Xi$}
\label{subsec:Xicone}
The isotropic cone
\index{A}{0nXi@$\Xi$, isotropic cone|textbf}
\[
     \Xi\equiv \Xi({\mathbb{R}}^{n+1,1})=
     \{(x_0, \cdots, x_{n+1}) \in {\mathbb{R}}^{n+2}:
        x_0^2+ \cdots + x_{n}^2 - x_{n+1}^2=0\}
     \setminus
     \{0\}    
\]
 is a homogeneous $G$-space 
 with the following fibration: 
\begin{alignat*}{6}
   &G/O(n) N_+ \,\,&& \simeq\,\,  && \Xi
   && \qquad\quad g O(n)N_+ && \mapsto && \ g p_+
\\
&  {\mathbb{R}}^{\times}\downarrow && && \downarrow {\mathbb{R}}^{\times}
   && \qquad\quad\qquad \rotatebox[origin=c]{-90}{$\mapsto$}  && && \  \rotatebox[origin=c]{-90}{$\mapsto$}
\\
     &\,G/P && \simeq \,\, && S^n,
     &&
     \qquad\quad\quad\,\,  gP &&\mapsto &&  [g p_+]
\end{alignat*}
where 
\index{A}{p1@$p_+={}^t(1,0,\cdots,0,1)$|textbf}
\begin{equation}
\label{eqn:p+}
p_+ :={} {}^{t\!} (1, 0, \cdots, 0, 1) \in \Xi. 
\end{equation}



The action of the subgroup $N_+$ on the isotropic cone
 $\Xi$ is given in the coordinates as 
\begin{equation}
\label{eqn:nbxi}
  n_+(b) \begin{pmatrix} \xi_0 \\ \xi \\ \xi_{n+1} \end{pmatrix}
 =
\begin{pmatrix}
 \xi_0 -(b,\xi)
\\
 \xi 
\\ 
 \xi_{n+1}-(b,\xi)
\end{pmatrix}
+
\frac{\xi_{n+1} -\xi_0}{2}
\begin{pmatrix} Q(b)\\ -2 b\\ Q(b)\end{pmatrix}, 
\end{equation}
where
 $b \in {\mathbb{R}}^n$, 
 $\xi \in {\mathbb{R}}^n$ and $\xi_0, \xi_{n+1}\in\mathbb{R}$.  



The intersections
 of the isotropic cone $\Xi$
 with the hyperplanes
 $\xi_0 + \xi_{n+1}=2$
 or $\xi_{n+1}=1$
 can be identified with ${\mathbb{R}}^n$ or $S^n$, 
 respectively.  
We write down the embeddings
 $\iota_N:{\mathbb{R}}^n \hookrightarrow \Xi$
 and $\iota_K:S^n \hookrightarrow \Xi$
 in the coordinates as follows:
\begin{align}
\label{eqn:NXi}
\iota_N:&
\mathbb{R}^n \hookrightarrow \Xi,
\
{}^t(x,x_n) \mapsto 
{n_-} (x,x_n) p_+
   = \left(\begin{array}{l}
           1 - |x|^2 - x_n^2  \\
           2x                     \\
           2x_n                  \\
           1 + |x|^2 + x_n^2
     \end{array}\right), 
\\
\label{eqn:KXi}
\iota_K\colon&
S^n \to \Xi, 
\
\eta \mapsto (\eta,1).  
\end{align}

The composition
 of $\iota_N$ and the projection
\begin{equation*}
   \Xi \to \Xi/\mathbb{R}^\times \overset{\sim}{\to} S^n,
\quad
\xi \mapsto \frac{1}{\xi_{n+1}} (\xi_0,\dots,\xi_n)
\end{equation*}
yields the conformal compactification of ${\mathbb{R}}^n$:
\begin{equation}
\label{eqn:XiSn}
\mathbb{R}^n \hookrightarrow S^n,
\quad
r \omega \mapsto \eta = (s, \sqrt{1-s^2} \, \omega ) 
= \Bigl( \frac{1-r^2}{1+r^2}, \frac{2r}{1+r^2} \omega \Bigr).  
\end{equation}
Here $\omega \in S^{n-1}$
 and the inverse map is given by 
$
r = \sqrt{\frac{1-s}{1+s}}
$
 for $s \ne -1$.  





\subsubsection{Characters $\chi_{\pm\pm}$ of the component group $G/G_0$}
\label{subsec:chcomp}

There are four connected components
 of the group $G=O(n+1,1)$.  
Let 
\index{A}{GOindefinitespecialidentity@$G_0=SO_0(n+1,1)$, the identity component
 of $O(n+1,1)$\quad|textbf}
$G_0$ denote the identity component of $G$.  
Then $G_0\simeq SO_0(n+1,1)$ and the quotient group $G/G_0$
\index{B}{componentgroup@component group $G/G_0$|textbf}
({\it{component group}})
 is isomorphic
 to ${\mathbb{Z}}/2 {\mathbb{Z}} \times {\mathbb{Z}}/2 {\mathbb{Z}}$.  
Accordingly, there are four one-dimensional representations of $G$, 
\index{A}{1chipmpm@$\chi_{\pm\pm}$, one-dimensional representation of $O(n+1,1)$|textbf}
\begin{equation}
\label{eqn:chiab}
\chi_{ab}\colon  G\to\{\pm1\}
\end{equation}
with $a,b\in\{\pm\}\equiv\{\pm1\}$ such that
\begin{equation*}
\chi_{ab}\left(\mathrm{diag}(-1,1,\cdots,1)\right)=a,\quad 
\chi_{ab}\left(\mathrm{diag}(1,\cdots,1,-1)\right)=b.
\end{equation*}
We note that $\chi_{--}$ is given 
 by the determinant, 
 $\mathrm{det}$, 
 of matrices in $O(n+1,1)$. 
Then the restriction of 
\index{A}{1chipmpm@$\chi_{\pm\pm}$, one-dimensional representation of $O(n+1,1)$|textbf}
$\chi_{--}$ to the subgroup $M\simeq O(n) \times O(1)$ is given by the outer tensor product representation:
\index{A}{1chipmm@$\chi_{--}=\det$}
\begin{equation}\label{eqn:chiM}
\chi_{--}\vert_{M}\simeq \det\boxtimes \,\mathbf{1},  
\end{equation}
where $\det$ in the right-hand side
 stands for the determinant
 for $n$ by $n$ matrices.  


\subsubsection{The center ${\mathfrak{Z}}_G({\mathfrak{g}})$ 
 and the Harish-Chandra isomorphism}
\label{subsec:2.1.4}

For a Lie algebra ${\mathfrak{g}}$ over ${\mathbb{R}}$, 
 we denote by 
\index{A}{Ug@$U({\mathfrak{g}})$, enveloping algebra|textbf}
$U({\mathfrak{g}})$
 the universal enveloping algebra
 of the complexified Lie algebra 
$
   {\mathfrak{g}}_{\mathbb{C}}={\mathfrak{g}} \otimes_{\mathbb{R}} {\mathbb{C}}
$, 
 and by
\index{A}{Zg@${\mathfrak{Z}}({\mathfrak{g}})$|textbf}
 ${\mathfrak{Z}}({\mathfrak{g}})$ its center.  
For a real reductive Lie group $G$ with Lie algebra ${\mathfrak{g}}$, 
 we define a subalgebra of ${\mathfrak{Z}}({\mathfrak{g}})$
 of finite index 
 by 
\index{A}{ZGg@${\mathfrak{Z}}_G({\mathfrak{g}})$|textbf}
\[
 {\mathfrak{Z}}_G({\mathfrak{g}})
  :=
  U({\mathfrak{g}})^G
  =\{z \in U({\mathfrak{g}}) : 
    {\operatorname{Ad}}(g)z=z
\quad
\text{for all } g \in G\}.  
\]
Schur's lemma implies 
 that the algebra ${\mathfrak{Z}}_G({\mathfrak{g}})$ acts
 on any irreducible admissible smooth representation of $G$
 by scalars, 
 which we refer to as 
\index{B}{ZGginfinitesimalcharacter@${\mathfrak{Z}}_G({\mathfrak{g}})$-infinitesimal character|textbf}
 the ${\mathfrak{Z}}_G({\mathfrak{g}})$-{\it{infinitesimal character}}.  
If the reductive group $G$ is of Harish-Chandra class, 
 then the adjoint group ${\operatorname{Ad}}(G)$
 is contained in the inner automorphism group
 ${\operatorname{Int}}({\mathfrak{g}}_{{\mathbb{C}}})$, 
 and consequently, 
 ${\mathfrak{Z}}_G({\mathfrak{g}}) ={\mathfrak{Z}}({\mathfrak{g}})$.  
However, 
 special attention is required 
 when $G$ is not of Harish-Chandra class, 
 as we shall see below.  



For the disconnected group $G=O(n+1,1)$, 
 ${\operatorname{Ad}}(G)$ is not contained in ${\operatorname{Int}}({\mathfrak{g}}_{{\mathbb{C}}})$
 and ${\mathfrak{Z}}_G({\mathfrak{g}})$ is of index two in 
 ${\mathfrak{Z}}({\mathfrak{g}})$
if $n$ is even, 
 whereas ${\operatorname{Ad}}(G) \subset {\operatorname{Int}}({\mathfrak{g}}_{{\mathbb{C}}})$ and ${\mathfrak{Z}}_G({\mathfrak{g}})={\mathfrak{Z}}({\mathfrak{g}})$
 if $n$ is odd.  
In both cases,
 via the standard coordinates
 of a Cartan subalgebra of ${\mathfrak{g}}_{{\mathbb{C}}} \simeq {\mathfrak{o}}
(n+2,{{\mathbb{C}}})$, 
 we have the following
\index{B}{HarishChandraisomorphism@Harish-Chandra isomorphism|textbf}
 Harish-Chandra isomorphisms
\begin{equation*}
\begin{array}{ccc}
   {\mathfrak{Z}}({\mathfrak{g}})
   & \simeq
   & S({{\mathbb{C}}}^{m+1})^{W_{\mathfrak{g}}}
\\
   \cup 
   & 
   & \cup
\\
  {\mathfrak{Z}}_G({\mathfrak{g}})
   & \simeq
   &\,\, S({{\mathbb{C}}}^{m+1})^{W_G}.  
\end{array}
\end{equation*}
Here  we identify a Cartan subalgebra ${\mathfrak{h}}_{{\mathbb{C}}}$
 of ${\mathfrak{g}}_{{\mathbb{C}}}\simeq {\mathfrak{o}}(n+2,{\mathbb{C}})$
 with ${{\mathbb{C}}}^{m+1}$
 where $m:=[\frac n 2]$, 
 and set 
\index{A}{Weylgroupg@$W_{\mathfrak{g}}$, Weyl group for ${\mathfrak{g}}_{\mathbb{C}}={\mathfrak{o}}(n+2,{\mathbb{C}})$|textbf}
\index{A}{WeylgroupG@$W_G$, Weyl group for $G=O(n+1,1)$|textbf}
\begin{align*}
W_{\mathfrak{g}}
:=& W(\Delta({\mathfrak{g}}_{\mathbb{C}},{\mathfrak{h}}_{\mathbb{C}}))
\simeq
\begin{cases}
{\mathfrak{S}}_{m+1} \ltimes ({\mathbb{Z}}/2 {\mathbb{Z}})^{m+1}
\quad
&\text{for $n=2m+1$}, 
\\
{\mathfrak{S}}_{m+1} \ltimes ({\mathbb{Z}}/2 {\mathbb{Z}})^{m}
\quad
&\text{for $n=2m$}, 
\end{cases}
\\
W_G:=& {\mathfrak{S}}_{m+1} \ltimes ({\mathbb{Z}}/2 {\mathbb{Z}})^{m+1}. 
\end{align*}

We shall describe
 the
\index{B}{infinitesimalcharacter@infinitesimal character|textbf}
 ${\mathfrak{Z}}_G({\mathfrak{g}})$-infinitesimal character
 by an element of ${{\mathbb{C}}}^N$ modulo $W_G$
 via the following isomorphism.  

\begin{equation}
\begin{array}{ccc}
    {\operatorname{Hom}}_{{\mathbb{C}}\operatorname{-alg}}({\mathfrak{Z}}({\mathfrak{g}}), {{\mathbb{C}}}) 
&\simeq 
&{\mathbb{C}}^N/W_{\mathfrak{g}}
\\
\rotatebox[origin=c]{-90}{$\twoheadrightarrow$}
&
&\rotatebox[origin=c]{-90}{$\twoheadrightarrow$}
\\
{\operatorname{Hom}}_{{\mathbb{C}}\operatorname{-alg}}({\mathfrak{Z}}_G({\mathfrak{g}}), {{\mathbb{C}}}) 
&\simeq 
&{\mathbb{C}}^N/W_G
\\
\label{eqn:HCpara}
\end{array}
\end{equation}

To define the notion of \lq\lq{regular}\rq\rq\
 or \lq\lq{singular}\rq\rq\
 about ${\mathfrak{Z}}_G({\mathfrak{g}})$-infinitesimal characters, 
 we use the action of the Weyl group $W_{\mathfrak{g}}$
 for the Lie algebra 
 ${\mathfrak{g}}_{\mathbb{C}}={\mathfrak{o}}(n+2,{\mathbb{C}})$
 rather than the Weyl group $W_G$
 for the disconnected group $G$ as below.  
 
\begin{definition}
\label{def:intreg}
Let $G=O(n+1,1)$ and $m:=[\frac n 2]$.  
Suppose $\chi\in {\operatorname{Hom}}_{{\mathbb{C}}\operatorname{-alg}}({\mathfrak{Z}}_G({\mathfrak{g}}), {{\mathbb{C}}})$
 is given by $\mu \in {\mathbb{C}}^{m+1} \mod W_G$ 
 via the Harish-Chandra isomorphism \eqref{eqn:HCpara}.  
We say $\chi$ is {\it{integral}}
 if 
\[
  \mu -\rho_G \in {\mathbb{Z}}^{m+1}, 
\]
see \eqref{eqn:rhoG} below
 for the definition of $\rho_G$, 
 or equivalently,
 if 
\begin{alignat*}{3}
\mu &\in {\mathbb{Z}}^{m+1}
\qquad
&&
\text{for $n=2m$}
\qquad
&&
\text{(even), }
\\
\mu &\in ({\mathbb{Z}}+\frac 1 2)^{m+1}
\qquad
&&
\text{for $n=2m+1$}
\qquad
&&
\text
{(odd).  }
\end{alignat*}
We note that this condition is stronger
 than the one which is usually referred
 to as \lq\lq{integral}\rq\rq:
\[
   \langle \mu, \alpha^{\vee} \rangle \in {\mathbb{Z}}
   \quad
  \text{for all $\alpha \in \Delta({\mathfrak{g}}_{\mathbb{C}}, {\mathfrak{h}}_{\mathbb{C}})$}
\]
where  $\alpha^{\vee}$ denotes the coroot of $\alpha$.  

For $\mu \in {\mathbb{C}}^{m+1}$, 
 we set 
\begin{align*}
  W_{\mu} \equiv (W_{\mathfrak{g}})_{\mu} 
          &:=\{w \in W_{\mathfrak{g}}: w \mu =\mu \}, 
\\
(W_G)_{\mu} 
          &:=\{w \in W_G: w \mu =\mu \}.  
\end{align*}
We say $\mu$ is 
\index{B}{Wgregular@$W_{\mathfrak{g}}$-regular|textbf}
{\it{$W_{\mathfrak{g}}$-regular}}
 (or simply, {\it{regular}})
 if $(W_{\mathfrak{g}})_{\mu}=\{e\}$, 
 and 
\index{B}{WGregular@$W_G$-regular|textbf}
{\it{$W_G$-regular}}
 if $(W_G)_{\mu}=\{e\}$.  
These definitions depend only
 on the $W_G$-orbit through $\mu$ 
 because $\# W_{\mu}= \# W_{\mu'}$ if $\mu' \in W_G \mu$.  
We say $\chi$ is \index{B}{regularintegralinfinitesimalcharacter@regular integral infinitesimal character|textbf}
 {\it{regular integral}}
 (respectively, 
\index{B}{singularintegralinfinitesimalcharacter@singular integral infinitesimal character|textbf}
{\it{singular integral}}) 
 infinitesimal character
 if $\chi$ is integral
 and $W_{\mu}=\{e\}$
 (respectively, $W_{\mu}\ne \{e\}$).  
In the coordinates of $\mu=(\mu_1, \cdots, \mu_{m+1})$, 
 $W_{\mu}=\{e\}$
 if and only if 
\begin{alignat*}{3}
\mu_i &\ne \pm \mu_j 
\quad
&&(1 \le \forall i < \forall j \le m+1)
&&\text{for $n$ even}, 
\\
\mu_i &\ne \pm \mu_j 
&&(1 \le \forall i < \forall j \le m+1), 
\,\,
\mu_k \ne 0
\quad
(1 \le \forall k \le m+1)
\quad
&&\text{for $n$ odd}.  
\end{alignat*}
\end{definition}

\begin{remark}
\label{rem:Fregint}
Suppose $G=O(n+1,1)$ with $n \ge 1$.  
Then the ${\mathfrak{Z}}_G({\mathfrak{g}})$-infinitesimal character
 of an irreducible finite-dimensional representation 
 of $G$ is regular integral, 
 and conversely,
 for any regular integral $\chi$, 
 there exists an irreducible finite-dimensional representation $F$
 of $G$ such that $\chi$ is the ${\mathfrak{Z}}_G({\mathfrak{g}})$-infinitesimal character of $F$.  
Here we remind from Definition \ref{def:intreg}
 above that by \lq\lq{regular}\rq\rq\
 we mean $W_{\mathfrak{g}}$-regular,
 and not $W_G$-regular.  
\end{remark}

The ${\mathfrak{Z}}_G({\mathfrak{g}})$-infinitesimal character
 of the trivial one-dimensional representation ${\bf{1}}$ of $G=O(n+1,1)$ 
 is given by 
\index{A}{1parho@$\rho_G$|textbf}
\begin{equation}
\label{eqn:rhoG}
     \rho 
\equiv 
     \rho_G =(\frac n 2, \frac n 2-1, \cdots, \frac n 2-[\frac n 2])
\in {\mathbb{C}}^{[\frac n 2]+1}/W_G.  
\end{equation}
The infinitesimal character $\rho_G$ will be also 
 referred to as the 
\index{B}{trivialinfinitesimalcharacter@trivial infinitesimal character|textbf}
 {\it{trivial infinitesimal character}}.  
\begin{definition}
\label{def:Irrrho}
We denote by 
\index{A}{IrrGrho@${\operatorname{Irr}}(G)_{\rho}$,
 set of irreducible admissible smooth representation of $G$
 with trivial infinitesimal character $\rho$\quad|textbf}
${\operatorname{Irr}}(G)_{\rho}$
 the set of equivalence classes
 of irreducible admissible smooth representations of $G$
 that have the trivial infinitesimal character $\rho$.  
\end{definition}
The finite set ${\operatorname{Irr}}(G)_{\rho}$ is classified 
 in Theorem \ref{thm:LNM20} (2)
 for $G=O(n+1,1)$
 and in Proposition \ref{prop:161648} (3)
 for the special orthogonal group $SO(n+1,1)$.  


\subsection{Representations of the orthogonal group $O(N)$}
\label{subsec:repON}
We recall that the orthogonal group $O(N)$ has two connected components.  
In this section,
 we review a parametrization 
 of irreducible finite-dimensional representations
 of the {\it{disconnected}} group $O(N)$
 following Weyl 
 \cite[Chap.~V, Sect.~7]{Weyl97}.  
For later reference we include classical branching theorems 
 for the restriction 
 of representations
 for the pairs $O(N) \supset O(N-1)$
 and $O(N) \supset SO(N)$.  
The results will be applied to the four compact subgroups $K$, 
\index{A}{Kmaxcptsub@$K'= K \cap G'$}
$K'$, 
\index{A}{M1@$M=O(n) \times O(1)$, the centralizer of ${\mathfrak {a}}$ in $K$}
$M$ and 
\index{A}{M1prime@$M'=O(n-1) \times O(1)$}
$M'=M \cap K'$ of $G$
 introduced in Section \ref{subsec:subgpOn}, 
 which satisfy the following obvious inclusive relations:
\[
\begin{pmatrix}
 K & \supset & K'
\\
\cup & & \cup
\\
 M & \supset & M'
\end{pmatrix}
=
\begin{pmatrix}
 O(n+1) \times O(1) & \supset & O(n) \times O(1)
\\
\cup & & \cup
\\
 O(n) \times {\operatorname{diag}}(O(1)) & \supset & O(n-1) \times {\operatorname{diag}}(O(1))
\end{pmatrix}.  
\]




\subsubsection{Notation for irreducible representations of $O(N)$}
\label{subsec:ONWeyl}
For finite-dimensional irreducible representations
 of orthogonal groups, 
 we use the following notation. 
We set
\index{A}{0tLambda@$\Lambda^+(N)$, dominant weight|textbf}
\begin{equation}
\label{eqn:Lambda}
\Lambda^+(N):=\{ \lambda = (\lambda_1, \ldots, \lambda_N) \in {\mathbb{Z}}^N:
\lambda_1 \geq \lambda_2 \geq \cdots \geq \lambda_N \geq 0\}.
\end{equation}
We write 
\index{A}{FUNl@$\Kirredrep{U(N)}{\lambda}$, irreducible representation of $U(N)$ with highest weight $\lambda$|textbf}
 $\Kirredrep {U(N)}{\lambda}$
 for the irreducible finite-dimensional representation 
of $U(N)$ (or equivalently, the irreducible polynomial representation
 of $GL(N,{\mathbb{C}})$)
with highest weight $\lambda \in \Lambda^+(N)$. 
If $\lambda$ is of the form 
\begin{equation*}
(\underbrace{c_1,\cdots,c_1}_{m_1},
\underbrace{c_2,\cdots,c_2}_{m_2},\cdots,\underbrace{c_\ell,\cdots,c_\ell}_{m_\ell},0,\cdots,0),
\end{equation*}
then we also write $\lambda=\left(c_1^{m_1},c_2^{m_2},\cdots,c_\ell^{m_\ell}\right)$ as usual. 



We define a subset of $\Lambda^+(N)$ by
\index{A}{0tExterior@$\Lambda^+(O(N))$|textbf}
\begin{equation*}
\Lambda^+(O(N)):= \{\lambda \in \Lambda^+(N):
\lambda_1' + \lambda_2' \leq N\},
\end{equation*}
where $\lambda_1':=\max\{i : \lambda_i \geq 1 \}$ and 
$\lambda_2': = \max\{i : \lambda_i \geq 2\}$ 
for $\lambda =(\lambda_1, \ldots, \lambda_N) \in \Lambda^+(N)$.
We note that $\lambda_1'$ equals the maximal column length
 in the corresponding Young diagram.  



It is readily seen that $\Lambda^+(O(N))$ consists of elements of 
the following two types:
\index{B}{typeON1@type I, for $\Lambda^+(O(N))$|textbf}
\index{B}{typeON2@type II, for $\Lambda^+(O(N))$|textbf}
\begin{align}
\label{eqn:TypeI}
&\text{Type I: $(\lambda_1,\cdots,\lambda_k,\underbrace{0,\cdots,0}_{N-k})$,}
\\
\label{eqn:TypeII}
&\text{Type II: $(\lambda_1,\cdots,\lambda_k,\underbrace{1,\cdots,1}_{N-2k},\underbrace{0,\cdots,0}_k)$,}
\end{align}
 with $\lambda_1\geq \lambda_2\geq\cdots\geq \lambda_k > 0$ and $0\leq k\leq\left[\frac N2\right]$.



For any $\lambda \in \Lambda^+(O(N))$, 
 there exists a unique $O(N)$-irreducible summand, 
 to be denoted by 
\index{A}{FONl@$\Kirredrep{O(N)}{\lambda}$|textbf}
$\Kirredrep {O(N)}{\lambda}$, 
 of the $U(N)$-module $\Kirredrep {U(N)}{\lambda}$
 which contains the highest weight vector.  
Following Weyl (\cite[Chap.~V, Sect.~7]{Weyl97}),
we parametrize the set $\widehat{O(N)}$ of equivalence classes
of irreducible representations of $O(N)$ by
\begin{equation} 
\label{eqn:CWOn}
\Lambda^+(O(N)) \stackrel{\sim}{\longrightarrow} \widehat{O(N)},
\quad
\lambda \mapsto \Kirredrep {O(N)}{\lambda}.  
\end{equation}
By the Weyl unitary trick, 
 we may identify
 $\Kirredrep {O(N)}{\lambda}$ 
 with a holomorphic irreducible representation
 of $O(N,{\mathbb{C}})$, 
 to be denoted by 
\index{A}{FONCl@$\Kirredrep{O(N,{\mathbb{C}})}{\lambda}$}
 $\Kirredrep {O(N,{\mathbb{C}})}{\lambda}$, 
 on the same representation space.  



\begin{definition}
\label{def:type}
We say $\Kirredrep{O(N)}{\lambda} \in \widehat{O(N)}$ is of type I
 (or type II), 
 if $\lambda \in \Lambda^+(O(N))$ is of 
\index{B}{type1@type I, representation of $O(N)$}
 type I (or type II), 
respectively.  
\end{definition}



We shall identify $\widehat{O(N)}$
 with $\Lambda^+(O(N))$ via \eqref{eqn:CWOn}, 
 and by abuse of notation, 
 we write $\sigma =(\sigma_1, \cdots, \sigma_N)\in \widehat{O(N)}$
 when $(\sigma_1, \cdots, \sigma_N)\in\Lambda^+(O(N))$.  

\begin{remark}
We shall also use the notation
\begin{align*}
&\Kirredrep{O(N)}
      {\sigma_1, \cdots, \sigma_k, \underbrace{0,\cdots,0}_{[\frac N2]-k}}_+
\text{ instead of }
 \Kirredrep{O(N)}
      {\sigma_1, \cdots, \sigma_k, \underbrace{0,\cdots,0}_{N-k}}, 
\\
&\Kirredrep{O(N)}
      {\sigma_1, \cdots, \sigma_k, \underbrace{0,\cdots,0}_{[\frac N2]-k}}_-
\text{ instead of }
 \Kirredrep{O(N)}
      {\sigma_1, \cdots, \sigma_k, \underbrace{1, \cdots,1}_{N-2k},\underbrace{0,\cdots,0}_{k}}, 
\end{align*}
 by putting the subscript $+$ or $-$
 for irreducible representations of type I
 or of type II,
respectively, 
 see Remark \ref{rem:FOn} in Appendix I.  
\end{remark}
We define a map by summing up 
the first $k$-entries 
 ($k \le [\frac N 2]$) of $\sigma$:
\index{A}{lengthrep@$\ell(\sigma)$|textbf}
\begin{equation}
\label{eqn:ONlength}
\ell \colon
\Lambda^+(O(N)) \to {\mathbb{N}}, 
\quad
\sigma
\mapsto 
\ell(\sigma):=\sum_{i=1}^{k}\sigma_i, 
\end{equation}
which induces a map
\[
  \ell \colon \widehat {O(N)} \to {\mathbb{N}}
\]
via the identification \eqref{eqn:CWOn}.  
By \eqref{eqn:type1to2}, 
 we have 
\begin{equation}
\label{eqn:lsigma}
\ell(\sigma)=\ell(\sigma \otimes \det).  
\end{equation}



\subsubsection{Branching laws for $O(N) \downarrow SO(N)$}
\label{subsec:OnSOn}
\begin{definition}
\label{def:OSO}
We say $\sigma \in \widehat{O(N)}$ is of 
\index{B}{typeX@type X, representation of ${O(N)}$\quad|textbf}
type X
 or 
\index{B}{typeY@type Y, representation of ${O(N)}$\quad|textbf}
type Y, 
 if the restriction $\sigma|_{SO(N)}$ to the special orthogonal group $SO(N)$ 
 is irreducible
 or reducible, 
 respectively.  
\end{definition}
With the convention as in Definition \ref{def:type},
 we recall a classical fact
 about the branching rule for the restriction
 $O(N) \downarrow SO(N)$. 
\begin{lemma}
[$O(N) \downarrow SO(N)$]
\label{lem:OSO}
Let $\sigma=(\sigma_1, \cdots, \sigma_N) \in \Lambda^+(O(N))$, 
 and $k$ $(\le [\frac N 2])$ be as in \eqref{eqn:TypeI} and \eqref{eqn:TypeII}.  
\begin{enumerate}
\item[{\rm{(1)}}]
{\rm{(type X)}}\enspace
The restriction of the irreducible $O(N)$-module 
 $\Kirredrep {O(N)}{\sigma}$
 to  $SO(N)$ is irreducible if and only if $N \ne2k$. 
In this case, 
the restricted $SO(N)$-module has highest weight
 $(\sigma_1, \cdots, \sigma_k, 0, \cdots, 0)$.  
\item[{\rm{(2)}}]
{\rm{(type Y)}}\enspace
If $N=2k$, 
 then the restriction $\Kirredrep {O(N)}{\lambda}|_{SO(N)}$ splits into two inequivalent 
 irreducible representations
 of $SO(N)$
 with highest weights
 $(\sigma_1, \cdots, \sigma_{k-1}, \sigma_k)$
 and $(\sigma_1, \cdots, \sigma_{k-1}, -\sigma_k)$.  
\end{enumerate}
\end{lemma}



\begin{example}
\label{ex:2.1}
The orthogonal group $O(N)$ acts irreducibly
 on the $\ell$-th exterior tensor space
\index{A}{0tExteriorCN@$\Exterior^{\ell}({\mathbb{C}}^N)$, exterior tensor|textbf}
 $\Exterior^{\ell}({\mathbb{C}}^N)$
 and on the space 
\index{A}{HSl@${\mathcal{H}}^s({\mathbb{C}}^N)$, spherical harmonics|textbf}
${\mathcal{H}}^s({\mathbb{C}}^N)$
of spherical harmonics of degree $s$.  
Via the parametrization \eqref{eqn:CWOn}, 
 these representations are described as follows:
\begin{alignat*}{2}
\Exterior^\ell ({\mathbb{C}}^N) &= \Kirredrep {O(N)}{1^{\ell}}
\qquad
&&(0\leq \ell \leq N), 
\\
\mathcal{H}^s({\mathbb{C}}^N)&=\Kirredrep {O(N)}{s,0,\cdots,0}
\qquad
&&(s \in {\mathbb{N}}).  
\end{alignat*}
The $O(N)$-module $\Exterior^{\ell}({\mathbb{C}}^N)$ is of type Y
 if and only if $N=2\ell$;
 the $O(N)$-module ${\mathcal{H}}^s({\mathbb{C}}^N)$ is of type Y
 if and only if $N=2$
 and $s\ne 0$.  
\end{example}



Irreducible $O(N)$-modules of types I and II are related by the following $O(N)$-isomorphism:
\begin{equation}
\label{eqn:type1to2}
   \Kirredrep {O(N)}{a_1,\cdots,a_k,1,\cdots,1,0,\cdots,0}
  =
   \det \otimes
   \Kirredrep {O(N)}{a_1,\cdots,a_k,0,\cdots,0}.
\end{equation}


Hence we obtain the following:
\begin{lemma}
\label{lem:typeY}
Let $\sigma \in \widehat{O(N)}$.  
Then $\sigma$ is of type Y
 if and only if $\sigma \otimes \det \simeq \sigma$.  
\end{lemma}

Then the following proposition is clear.  
\begin{proposition}
\label{prop:XYI}
Suppose $\sigma \in \widehat{O(n)}$.  
\begin{enumerate}
\item[{\rm{(1)}}]
If $\sigma$ is of type Y, 
 then $\sigma$ is of type I. 
\item[{\rm{(2)}}]
If $\sigma$ is of type II, 
 then $\sigma$ is of type X. 
\end{enumerate}
\end{proposition}

%%%%%%%%%%%%%%%%%%%%%%%%%%%%%%%%%%%%%%%%%%%%%%%%%%%%%%%%
\subsubsection{Branching laws $O(N) \downarrow O(N-1)$}
\label{subsec:ONbranch}
%%%%%%%%%%%%%%%%%%%%%%%%%%%%%%%%%%%%%%%%%%%%%%%%%%%%%%%%

Next we recall the classical branching laws for $O(N) \downarrow O(N-1)$.  
Let $\sigma=(\sigma_1, \cdots, \sigma_N) \in \Lambda^+(O(N))$
 and $\tau=(\tau_1, \cdots, \tau_{N-1}) \in \Lambda^+(O(N-1))$.  

\begin{definition}
\label{def:Young}
We denote by 
\index{A}{1pataorder@$\tau \prec \sigma$|textbf}
$\tau \prec \sigma$
 if  
\[
   \sigma_1 \ge \tau_1 \ge \sigma_2 \ge \tau_2 \ge \cdots \ge \tau_{N-1} \ge \sigma_N.  
\]
\end{definition}
Then the irreducible decomposition
 of representations of $O(N)$ with respect to the subgroup $O(N-1)$
 is given as follows:
\begin{fact}
[Branching rule for orthogonal groups]
\label{fact:ONbranch}
\index{B}{branching rule@branching rule, for $O(N) \downarrow O(N-1)$|textbf}
Let $(\sigma_1, \cdots,\sigma_N) \in \Lambda^+(O(N))$.  
Then the irreducible representation
 $\Kirredrep{O(N)}{\sigma_1, \cdots,\sigma_N}$ decomposes
 into a multiplicity-free sum
 of irreducible representations
 of $O(N-1)$ as follows:
\begin{equation}
\label{eqn:ONbranch}
   \Kirredrep{O(N)}{\sigma_1, \cdots, \sigma_N}|_{O(N-1)}
   =
  \bigoplus_{\tau \prec \sigma} 
   \Kirredrep{O(N-1)}{\tau_1, \cdots, \tau_{N-1}}.  
\end{equation}
\end{fact}



The commutant $O(1)$ of $O(N-1)$ in $O(N)$ acts
 on the irreducible summand
 $\Kirredrep {O(N-1)}{\tau_1, \cdots, \tau_{N-1}}$
 by 
$
   ({\operatorname{sgn}})^{\sum_{j=1}^N \sigma_j - \sum_{i=1}^{N-1} \tau_i}.
$  


The following lemma is derived from Lemma \ref{lem:typeY} and Fact \ref{lem:branchII}.  
\begin{lemma}
\label{lem:branchII}
Let $\sigma \in \widehat {O(n)}$
 be of type I (see Definition \ref{def:type}).  
Then the following four conditions are equivalent:
\begin{enumerate}
\item[{\rm{(i)}}]
$\sigma \otimes \det \simeq \sigma$; 
\item[{\rm{(ii)}}]
$[\sigma|_{O(n-1)}:\tau]=[\sigma|_{O(n-1)}:\tau \otimes \det]$
 for all $\tau \in \widehat {O(n-1)}$; 
\item[{\rm{(iii)}}]
$n$ is even and $\sigma =\Kirredrep{O(n)}{s_1, \cdots, s_{\frac n 2}, 0, \cdots, 0}$
 with $s_{\frac n 2} \ne 0$;  
\item[{\rm{(iv)}}]
$\sigma|_{S O(n)}$ is reducible, 
 {\it{i.e.,}} $\sigma$ is of type Y
 (Definition \ref{def:OSO}).  
\end{enumerate}
\end{lemma}



\subsection{Principal series representations $I_{\delta}(V,\lambda)$
 of the orthogonal group $G=O(n+1,1)$}
\label{subsec:ps}

We discuss here (nonspherical) principal series representations
\index{A}{IdeltaV@$I_{\delta}(V, \lambda)$}
 $I_{\delta}(V,\lambda)$ of $G=O(n+1,1)$.  
We shall use the symbol 
\index{A}{JWepsilon@$J_{\varepsilon}(W, \nu)$}
 $J_{\varepsilon}(W,\nu)$
 for the principal series representations of the subgroup 
 $G'=O(n,1)$.  



We recall the structure of  principal series representations for  rank one orthogonal groups.  
The main references are Borel--Wallach \cite{BW}
 and Collingwood \cite[Chap.~5, Sect.~2]{C}
 for the representations
 of the identity component $G_0=SO_0(n+1,1)$.  
We extend  here the results to the disconnected group.  
For representations of the disconnected group $G$, 
 see also \cite[Chap.~2]{sbon}
 for the spherical case
 ({\it{i.e.,}} $V={\bf{1}}$) and \cite[Chap.~2, Sect.~3]{KKP}
 for $V=\Exterior^i({\mathbb{C}}^n)$
 ($0 \le i \le n$).  



\subsubsection{$C^{\infty}$-induced representations $I_\delta(V,\lambda)$}  
\label{subsec:smoothI}

We recall from Section \ref{subsec:subgpOn}
 that the Levi subgroup $M A$ of the minimal parabolic subgroup $P$
 of $G$ is a direct product group 
 $(O(n) \times O(1)) \times {\mathbb{R}}$.  
Then any irreducible representation
 of $M A$
 is the outer tensor product
 of irreducible representations
 of the three groups $O(n)$, $O(1)$, and ${\mathbb{R}}$.  



One-dimensional representations $\delta$
 of
\index{A}{m2@$m_-={\operatorname{diag}}(-1,1,\cdots,1,-1)$}
 $O(1)=\{1,m_-\}$
 are labeled by $+$ or $-$, 
 where we write $\delta=+$ for the trivial representation ${\bf{1}}$,
 and $\delta=-$ for the nontrivial one
 given by $\delta(m_-)=-1$.  
Thus we identify $\widehat{O(1)}$ with the set $\{\pm\}$.  



For $\lambda \in {\mathbb{C}}$, 
 we denote by 
\index{A}{charA@${\mathbb{C}}_{\lambda}$, character of $A$|textbf}
${\mathbb{C}}_{\lambda}$
 the one-dimensional representation
 of the split group $A$
 normalized 
 by the generator $H \in {\mathfrak{a}}$
 (see \eqref{eqn:Hyp}) as 
\index{A}{H@$H$}
\[
   A \mapsto {\mathbb{C}}^{\times}, 
  \qquad
  \exp(t H) \mapsto e^{\lambda t}.  
\]



Let $(\sigma, V)$ be an irreducible representation of $O(n)$, 
 $\delta \in \{\pm\}$, 
 and $\lambda \in{\mathbb{C}}$.  
We extend the outer tensor product representation
\index{A}{Vln@$V_{\lambda,\delta}=V \otimes \delta \otimes {\mathbb{C}}_{\lambda}$, representation of $P$|textbf}
\begin{equation}
\label{eqn:Vlmddelta}
 V_{\lambda,\delta}
:=V \boxtimes \delta \boxtimes{\mathbb{C}}_{\lambda}
\end{equation}
of the direct product group $MA \simeq O(n) \times O(1) \times {\mathbb{R}}
$
 to a representation
 of the parabolic subgroup $P=MAN_+$
 by letting the unipotent subgroup $N_+$ act trivially.  
The resulting irreducible $P$-module will be written 
as 
$
   V_{\lambda,\delta}
   =V \otimes \delta \otimes{\mathbb{C}}_{\lambda}
$
 by a little abuse of notation.  
We define  the induced representation of $G$ by 
\[
   I_\delta({V},\lambda)
  \equiv I({V}\otimes \delta, \lambda)
  := {\operatorname{Ind}}_P^G(V_{\lambda,\delta}).
\]
We refer to $\delta$ as the 
\index{B}{signature of the induced representation}
{\it{signature}} of the induced representation. 
If $\delta=+$ (the trivial character ${\bf{1}}$),
 we sometimes suppress the subscript.  



If $(\sigma, V) \in \widehat{O(n)}$ is given 
 as $V=\Kirredrep{O(n)}{\sigma_1, \cdots,\sigma_n}$
with $(\sigma_1,\cdots,\sigma_n) \in \Lambda^+(O(n))$
 via \eqref{eqn:CWOn}, 
 then $I_{\delta}(V,\lambda)$ has 
a 
\index{B}{infinitesimalcharacter@infinitesimal character}
${\mathfrak {Z}}_G({\mathfrak{g}})$-infinitesimal character
\begin{equation}
\label{eqn:ZGinfI}
 (\sigma_1+\frac {n} 2-1, \sigma_2+\frac {n} 2-2, \cdots, 
  \sigma_k+\frac {n} 2-k, \frac {n} 2-k-1, \cdots, \frac {n} 2-[\frac n 2], 
  \lambda-\frac {n} 2)
\end{equation}
in the standard coordinates
 via the 
\index{B}{HarishChandraisomorphism@Harish-Chandra isomorphism}
 Harish-Chandra isomorphism, 
 see \eqref{eqn:HCpara}. 
We are using in this article
 unnormalized induction, 
{\it{i.e.}}, 
 the representation $I_{\delta}({V}, \frac n 2)$ is a unitarily induced principal series representation. 
Thus if $\lambda$ is purely imaginary, 
 the principal series representations
 $I_\delta(V,\lambda + \frac n 2)$ are 
\index{B}{temperedrep@tempered representation}
tempered.  
If $n$ is even, 
 then every irreducible tempered representation is isomorphic
 to a tempered principal series representation.  
If $n$ is odd, 
 then there is one family of discrete series representations
 parametrized by characters
 of the compact Cartan subgroup
 and every irreducible tempered representation
 is isomorphic to a tempered principal series representation
 or a discrete series representation. 



\medskip

We denote by
\index{A}{Vlnd@${\mathcal{V}}_{\lambda,\delta}$, homogeneous vector bundle over $G/P$|textbf}
\begin{equation}
\label{eqn:Vlmdbdle}
  {\mathcal{V}}_{\lambda,\delta}
  :=
  G \times_P V_{\lambda,\delta}
\end{equation}
 the $G$-equivariant vector bundle over the real flag manifold $G/P$
 associated to the representation 
 $V_{\lambda,\delta}$ of $P$.  
We assume from now on that the principal series representations
 $I_\delta({V},\lambda)$
 are realized  on the Fr{\'e}chet space
 $C^{\infty}(G/P, {\mathcal{V}}_{\lambda,\delta})$
 of smooth sections
 for the vector bundle 
 ${\mathcal{V}}_{\lambda,\delta} \to G/P$.  
Thus 
\index{A}{IdeltaV@$I_{\delta}(V, \lambda)$|textbf}
$I_{\delta}(V,\lambda)$
 is the induced representation 
 $C^{\infty}{\operatorname{-Ind}}_P^G(V_{\lambda, \delta})$
 which is of moderate growth, 
see \cite[Chap.~3, Sect.~4]{sbon}.  
As usual, 
 we denote the representation space and the representation by the same letter. 
We trivialize the vector bundle ${\mathcal{V}}_{\lambda,\delta}$
 over $G/P$
 on the open Bruhat cell via the following map
\[
   \iota_N \colon {\mathbb{R}}^n {\underset {n_-}{\overset \sim \to}} N_-
   \overset \sim \to N_- \cdot o \subset G/P.  
\]
Then $I_{\delta}(V,\lambda)$ is realized in a subspace
 of $C^{\infty}({\mathbb{R}}^n) \otimes V$ by 
\begin{equation}
\label{eqn:iotaN}
   \iota_N^{\ast} \colon 
   I_{\delta}(V,\lambda)
   \hookrightarrow
   C^{\infty}({\mathbb{R}}^n) \otimes V,
   \quad
   F \mapsto f(b):=F(n_-(b)), 
\end{equation}
and this model is referred to as
 the {\it{noncompact picture}}, 
 or the 
\index{B}{Npicture@$N$-picture|textbf}
{\it{$N$-picture}}, 
 see Section \ref{subsec:psKN}.  


\subsubsection{Tensoring with characters
 $\chi_{\pm\pm}$ of $G$}
The character group $(G/G_0)\hspace{-1mm}{\widehat{\hphantom{m}}}$
 of the component group
 $G/G_0 \simeq {\mathbb{Z}}/2{\mathbb{Z}} \times {\mathbb{Z}}/2{\mathbb{Z}}$
 acts on the set of admissible representations $\Pi$ of $G$, 
 by taking the tensor product
\begin{equation}
\label{eqn:Pichi}
   \Pi \mapsto \Pi \otimes \chi
\end{equation}
for $\chi \in (G/G_0)\hspace{-1mm}{\widehat{\hphantom{m}}}$.  
This action leaves the subsets 
\index{A}{Irredrep@${\operatorname{Irr}}(G)$}
 ${\operatorname{Irr}}(G)$ and ${\operatorname{Irr}}(G)_{\rho}$ 
(see Definition \ref{def:Irrrho})
invariant.  
We describe the action explicitly
 on principal series representations
 in Lemma \ref{lem:IVchi} below.  
The action on 
\index{A}{IrrGrho@${\operatorname{Irr}}(G)_{\rho}$,
 set of irreducible admissible smooth representation of $G$
 with trivial infinitesimal character $\rho$\quad}
${\operatorname{Irr}}(G)_{\rho}$
  will be given explicitly 
 in Theorem \ref{thm:LNM20} (5), 
 and on the space of symmetry breaking operators
 in Section \ref{subsec:actPont}.  

\begin{lemma}
\label{lem:IVchi}
Let $V \in \widehat {O(n)}$, 
 $\delta \in \{\pm\}$, 
 and $\lambda \in {\mathbb{C}}$.  
Let $\chi_{\pm\pm}$ be the one-dimensional representations
 of $G=O(n+1,1)$
 as defined in \eqref{eqn:chiab}.  
Then we have the following isomorphisms
 between representations of $G$:  
\begin{align*}
I_{\delta}(V,\lambda) \otimes \chi_{+-} \simeq & I_{-\delta}(V,\lambda), 
\\
I_{\delta}(V,\lambda) \otimes \chi_{-+} \simeq & I_{-\delta}(V \otimes \det,\lambda), 
\\
I_{\delta}(V,\lambda) \otimes \chi_{--} \simeq & I_{\delta}(V\otimes \det,\lambda).  
\end{align*}
\end{lemma}

\begin{proof}
For any $P$-module $U$
 and for any finite-dimensional $G$-module $F$, 
 there is an isomorphism of $G$-modules:
\[
  F \otimes {\operatorname{Ind}}_P^G(U)
  \simeq
  {\operatorname{Ind}}_P^G(F \otimes U).  
\]
Then Lemma \ref{lem:IVchi} follows from the restriction formula
 of the character $\chi$ of $G$ to the subgroup $M \simeq O(n) \times O(1)$ as below:
\[
   \chi_{+-}|_M \simeq {\bf{1}} \boxtimes {\operatorname{sgn}}, 
\quad   
   \chi_{-+}|_M \simeq \det \boxtimes {\operatorname{sgn}}, 
\quad
   \chi_{--}|_M \simeq \det \boxtimes {\bf{1}}.  
\]
\end{proof}

A special case of Lemma \ref{lem:IVchi} 
 for the exterior tensor representations $V=\Exterior^i({\mathbb{C}}^n)$
 will be stated in Lemma \ref{lem:LNM27}.  


%%%%%%%%%%%%%%%%%%%%%%%%%%%%%%%%%%%%%%%%%%%%%%%%%%%%%%%%%%%%%%%%%%%%
\subsubsection{$K$-structure of the principal series representation
 $I_{\delta}(V,\lambda)$}
\label{subsec:KstrI}
%%%%%%%%%%%%%%%%%%%%%%%%%%%%%%%%%%%%%%%%%%%%%%%%%%%%%%%%%%%%%%%%%%%%
Let $(\sigma,V) \in \widehat {O(n)}$
 and $\delta \in \{\pm\}$ as before.  
By the Frobenius reciprocity law, 
 $K$-types of the principal series representation $I_\delta({V},\lambda)$
 are the irreducible representations
 of $K=O(n+1) \times  O(1)$
 whose restriction to $M \simeq O(n)\times O(1)$ contains the representation
 $V\boxtimes \delta$ of $M$.  
The classical branching theorem (Fact \ref{fact:ONbranch}) is used
 to determine $K$-types of the $G$-module $I_{\delta}(V,\lambda)$.  
We shall give an explicit $K$-type formula
 in the next section
 when $V$ is the exterior tensor representation $\Exterior^i({\mathbb{C}}^n)$
 of $O(n)$.  
For the general representation $(\sigma,V) \in \widehat {O(n)}$, 
 we do not use an explicit $K$-type formula of $I_{\delta}(V,\lambda)$, 
 but just mention an immediate corollary of Fact \ref{fact:ONbranch}:
\begin{proposition}
The $K$-types of principal series representations $I_{\delta}(V,\lambda)$
 of $O(n+1,1)$ have multiplicity one.  
\end{proposition}







%%%%%%%%%%%%%%%%%%%%%%%%%%%%%%%%%%%%%%%%%%%%%%%%%%%%%%%%%%%%%%%%%%%%%%%%%%%%%%%
\subsection{Principal series representations $I_{\delta}(i, \lambda)$}
\label{subsec:irred}
%%%%%%%%%%%%%%%%%%%%%%%%%%%%%%%%%%%%%%%%%%%%%%%%%%%%%%%%%%%%%%%%%%%%%%%%%%%%%%%
For $0 \le i \le n$, 
 $\delta \in \{\pm\}$, 
 and $\lambda \in {\mathbb{C}}$, 
 we denote the principal series representation 
 $I_\delta (\Exterior^i({\mathbb{C}}^n),\lambda)
 =
 C^{\infty}{\text{-}}{\operatorname{Ind}}_P^G(\Exterior^i({\mathbb{C}}^n)\otimes \delta \otimes {\mathbb{C}}_{\lambda})$
 of $G=O(n+1,1)$ simply by 
\index{A}{Ideltai@$I_{\delta}(i, \lambda)$|textbf}
$I_\delta(i,\lambda)$.  
Similarly,
 we write 
\index{A}{Jjepsilon@$J_{\varepsilon}(j, \nu)$|textbf}
$J_{\varepsilon}(j,\nu)$
 for the induced representation
 $C^{\infty}{\operatorname{-Ind}}_{P'}^{G'}(\Exterior^j({\mathbb{C}}^{n-1}) \otimes {\varepsilon} \otimes {\mathbb{C}}_{\nu})$
 of $G'=O(n,1)$ 
 for $0 \le j \le n-1$, 
 $\varepsilon \in \{\pm\}$, 
 and $\nu \in {\mathbb{C}}$. 
In the major part of this monograph, 
 we focus our attention on special families of principal series representations
 $I_\delta(i,\lambda)$ of $G$
 and $J_{\varepsilon}(j,\nu)$ of the subgroup $G'$. 



In geometry, 
 $I_{\delta}(i,\lambda)$ is a family of representations
 of the conformal group $O(n+1,1)$ 
 of $S^n$ on the space ${\mathcal{E}}^i(S^n)$
 of differential forms
 ({\it{cf.}} \cite[Chap.~2, Sect.~2]{KKP})
 on one hand.  
In representation theory,
 any irreducible, unitarizable representations
 with nonzero $({\mathfrak{g}}, K)$-cohomologies arise
 as subquotients
 in $I_{\delta}(i,\lambda)$ with $\lambda=i$
 for some $0 \le i \le n$
 and $\delta=(-1)^i$, 
 see Theorem \ref{thm:LNM20} (9), 
 also Proposition \ref{prop:gKq} in Appendix I. 



In this section we collect some basic properties
 of the principal series representations
\[
  I_{\delta}(i,\lambda)
\qquad
  \text{for $\delta \in \{\pm\}$, $0 \le i \le n$, $\lambda \in {\mathbb{C}}$, }
\]
which will be used throughout the article.  

\subsubsection{${\mathfrak{Z}}_G({\mathfrak{g}})$-infinitesimal character of $I_{\delta}(i,\lambda)$}



As we have seen in \eqref{eqn:ZGinfI} in the general case, 
the 
\index{B}{infinitesimalcharacter@infinitesimal character}
 ${\mathfrak{Z}}_G({\mathfrak{g}})$-infinitesimal character
 of the principal series representation $I_{\delta}(i,\lambda)$ is given by 
\begin{alignat*}{2}
&(\underbrace{\frac n 2, \frac n 2-1, \cdots, \frac n 2-i+1}_i; 
  \underbrace{\frac n 2-i-1, \cdots, \frac n 2-[\frac n 2]}_{[\frac n 2]-i}; \lambda-\frac n 2)
\quad
&&\text{if }
0 \le i \le \frac n 2, 
\\
&(\underbrace{\frac n 2, \frac n 2-1, \cdots, -\frac n 2+i+1}_{n-i}; 
  \underbrace{-\frac n 2+i-1, \cdots, \frac n 2-[\frac n 2]}_{i-[\frac {n+1} 2]}; \lambda-\frac n 2)
\quad
&&\text{if }
\frac n 2 \le i \le n.  
\end{alignat*}



In particular,
 the $G$-module $I_{\delta}(i,\lambda)$ has the trivial infinitesimal character $\rho_G$
 if and only if $\lambda=i$ or $n-i$.  


\subsubsection{$K$-type formula of the principal series representations $I_\delta(i,\lambda)$}

By the Frobenius reciprocity,
 we can compute the $K$-type formula of $I_{\delta}(i,\lambda)$
 explicitly by using the classical branching law 
 (Fact \ref{fact:ONbranch})
 and Example \ref{ex:2.1} as follows:  
\begin{lemma}
[$K$-type formula of $I_{\delta}(i,\lambda)$]
\label{lem:KtypeIi}
Let $0 \le i \le n$ and $\delta \in\{\pm\}$.  
With the parametrization \eqref{eqn:CWOn}, 
 the 
\index{B}{Ktypeformula@$K$-type formula|textbf}
$K$-type formula
 of the principal series representation $I_{\delta}(i,\lambda)$
 of $G=O(n+1,1)$ is described 
 as below:
\begin{enumerate}
\item[{\rm{(1)}}] 
 for $i=0$, 
\[
\bigoplus_{a=0}^{\infty} \Kirredrep {O(n+1)}{a,0^n}
  \boxtimes
  (-1)^a \delta;
\]
\item[{\rm{(2)}}]
for $1 \le i \le n-1$, 
\[
  \bigoplus_{a=1}^{\infty} \Kirredrep {O(n+1)}{a,1^i,0^{n-i}}
  \boxtimes
  (-1)^a \delta
  \oplus
  \bigoplus_{a=1}^{\infty} \Kirredrep {O(n+1)}{a,1^{i-1},0^{n+1-i}}
  \boxtimes
  (-1)^{a+1} \delta;
\]
\item[{\rm{(3)}}] 
for $i=n$, 
\[
  \bigoplus_{a=1}^{\infty} (\det \otimes \Kirredrep {O(n+1)}{a,0^n})
  \boxtimes
  (-1)^{a+1} \delta.  
\]
\end{enumerate}
\end{lemma}
See Proposition \ref{prop:KtypeIV} for a more general $K$-type formula
 of the principal series representation $I_{\delta}(V,\lambda)$.  

\subsubsection{Basic $K$-types of $I_{\delta}(i,\lambda)$}
\label{subsec:KIilmd}
\index{B}{basicKtype@basic $K$-type|textbf}
Let $\delta \in \{ \pm \}$
 and $0 \le i\le n$.  
Following the notation \cite[Chap.~2, Sect.~3]{KKP}, 
 we define two irreducible representations
 of $K\simeq O(n+1) \times O(1)$ by:
\index{A}{0muflat@$\mub(i,\delta)$|textbf}
\index{A}{0musharp@$\mus(i,\delta)$|textbf}
\begin{align}
   \mub(i,\delta):=& \Exterior^i({\mathbb{C}}^{n+1}) \boxtimes \delta, 
\label{eqn:muflat}
\\
   \mus(i,\delta):=& \Exterior^{i+1}({\mathbb{C}}^{n+1}) \boxtimes (-\delta).  
\label{eqn:musharp}
\end{align}
This means:
\[
 \begin{cases}
\mub(i,+)&=\Exterior^i({\mathbb{C}}^{n+1}) \boxtimes {\bf{1}},
\\
\mub(i,-)
&=\Exterior^i({\mathbb{C}}^{n+1})\boxtimes {\operatorname{sgn}},
   \end{cases}
\quad
\begin{cases}
\mus(i,+)
&=\Exterior^{i+1}({\mathbb{C}}^{n+1}) \boxtimes {\operatorname{sgn}}, 
\\
\mus(i,-)
&=\Exterior^{i+1}({\mathbb{C}}^{n+1}) \boxtimes {\bf{1}}.  
\end{cases}
\]
The superscripts $\sharp$ and $\flat$ indicate
 that there are the following obvious $K$-isomorphisms
\begin{equation}
\label{eqn:flatsharp}
   \mus(i,\delta)=\mub(i+1,-\delta)
   \qquad
   (0 \le i \le n), 
\end{equation}
which will be useful
 in describing the standard sequence
 with trivial infinitesimal character $\rho_G$
 (Definition \ref{def:Pii} below), 
 see also Remark \ref{rem:flatsharp}.  



By the $K$-type formula of the principal series representation 
 $I_{\delta}(i,\lambda)$ in Lemma \ref{lem:KtypeIi}, 
 the $K$-types $\mub(i,\delta)$ and $\mus(i,\delta)$
 occur in $I_{\delta}(i,\lambda)$
 with multiplicity one for any $\lambda \in {\mathbb{C}}$.  

\begin{definition}
\label{def:basicK}
We say $\mub (i,\delta)$ and $\mus (i,\delta)$
 are {\it{basic $K$-types}}
 of the principal series representations $I_{\delta}(i,\lambda)$
 of $G=O(n+1,1)$.  
\end{definition}

\subsubsection{Reducibility of $I_{\delta}(i,\lambda)$}
\label{subsec:irredIilmd}
The principal series representation $I_{\delta}(i,\lambda)$
 is generically irreducible.  
More precisely,
 we have the following.
\begin{proposition}
\label{prop:redIilmd}
Let $G=O(n+1,1)$, 
 $0 \le i \le n$, 
 $\delta \in \{\pm\}$, 
 and $\lambda \in {\mathbb{C}}$.  
\begin{enumerate}
\item[{\rm{(1)}}]
The principal series representation $I_{\delta}(i,\lambda)$
 is reducible 
 if and only if 
\begin{equation}
\label{eqn:redIilmd}
 \lambda \in \{i,n-i\} \cup (-{\mathbb{N}}_+) \cup (n+{\mathbb{N}}_+).  
\end{equation}
\item[{\rm{(2)}}]
Suppose $(n,\lambda) \ne (2i,i)$.  
If $\lambda$ satisfies \eqref{eqn:redIilmd}, 
 then the $G$-module $I_{\delta}(i,\lambda)$ 
 has a unique irreducible proper submodule
 (say, A)
 and has a unique irreducible subquotient 
 (say, B)
 and there is a nonsplitting exact sequence
 of $G$-modules:
\[
   0 \to A \to I_{\delta}(i,\lambda) \to B \to 0.  
\]
\item[{\rm{(3)}}]
Suppose $(n,\lambda) = (2i,i)$.  
Then the $I_{\delta}(i,\lambda)$ decomposes
 into the direct sum 
 of two irreducible representations of $G$
 which are not isomorphic to each other.  
\end{enumerate}
\end{proposition}
When $n \ne 2 i$, 
 the \lq\lq{only if}\rq\rq\ part of the first statement and the second one
 in Proposition \ref{prop:redIilmd} follow readily from 
 the corresponding results 
 (\cite{BW,C,Hirai62})
 for the connected group $SO_0(n+1,1)$ 
 and from Lemma \ref{lem:GPconn} below
 because $\Exterior^i({\mathbb{C}}^n)$ is irreducible
 as an $SO(n)$-module.  
We need some argument for $n=2i$
 where $\Exterior^i({\mathbb{C}}^{n})$ is reducible
 as an $SO(n)$-module, 
 see Examples \ref{ex:irrIilmd} and \ref{ex:Imm}
 in Appendix II
 for the proof of Proposition \ref{prop:redIilmd} (1) and (3), 
 respectively.
In Section \ref{subsec:KerKnSt}, 
 we discuss the description of proper submodules
 of reducible $I_{\delta}(i,\lambda)$ by using the Knapp--Stein operator
 \eqref{eqn:KSii}
 and its normalized one \eqref{eqn:Ttilde}.  
The \lq\lq{if}\rq\rq\ part of the first statement is proved there, 
 see Lemma \ref{lem:reducibleI}.  


The composition series of $I_{\delta}(i,\lambda)$ 
with trivial infinitesimal character $\rho_G$
 ({\it{i.e.}}, for $\lambda=i$ or $n-i$)
 will be discussed
 in the next subsection 
 (see Theorem \ref{thm:LNM20}), 
 which will be extended in Theorem \ref{thm:1714107}
 to the case of
\index{B}{regularintegralinfinitesimalcharacter@regular integral infinitesimal character}
 regular integral infinitesimal characters.  

%%%%%%%%%%%%%%%%%%%%%%%%%%%%%%%%%%%%%%%%%%%%%%%%%%%%%%%%%%%%%%%%%%
\subsubsection{Irreducible subquotients of $I_{\delta}(i,i)$}
\label{subsec:subIii}
%%%%%%%%%%%%%%%%%%%%%%%%%%%%%%%%%%%%%%%%%%%%%%%%%%%%%%%%%%%%%%%%%%
Every irreducible representation of $G=O(n+1,1)$ with trivial infinitesimal character $\rho $ is equivalent to a subquotient of $I_{\delta}(i,i)$
 for some $0 \le i \le n$
 and $\delta\in \{\pm\}$, 
 or equivalently,
 of $I_+(i,i) \otimes \chi $ with $i \geq n/2$ and $\chi \in (G/G_0)\hspace{-1mm}{\widehat{\hphantom{m}}}$.
We recall now facts about the principal series representations
 $I_+(i,i)$, $I_-(i,i)$, $I_+(n-i,i)$ and $I_-(n-i,i)$
 of the orthogonal group $O(n+1,1)$
 and their composition factors.  



We denote by 
$
\index{A}{Ideltaiflat@$I_{\delta}(i)^{\flat}$, submodule of $I_{\delta}(i,i)$|textbf}
I_{\delta}(i)^{\flat}
$ 
and 
$
\index{A}{Ideltaisharp@$I_{\delta}(i)^{\sharp}$, quotient of $I_{\delta}(i,i)$|textbf}
I_{\delta}(i)^{\sharp}
$
 the unique irreducible subquotients
 of $I_{\delta}(i, i)$
 containing the basic $K$-types
 $\mu^{\flat}(i,\delta)$ and $\mu^{\sharp}(i,\delta)$, 
 respectively.  
Then we have $G$-isomorphisms:
\begin{equation}
\label{eqn:LNS218}
  I_{\delta}(i)^{\sharp} \simeq I_{-\delta}(i+1)^{\flat}
\quad
  \text{for $0 \le i \le n-1$ and $\delta \in \{\pm \}$}, 
\end{equation}
see Theorem \ref{thm:LNM20} (1) below.  
For $0 \le \ell \le n+1$ and $\delta \in \{\pm \}$, 
 we set 
\index{A}{1Piidelta@$\Pi_{i,\delta}$, irreducible representations of $G$|textbf}
\begin{equation}
\label{eqn:Pild}
  \Pi_{\ell,\delta}
  :=
  \begin{cases}
  I_{\delta}(\ell)^{\flat} \quad&(0 \le \ell \le n), 
  \\
  I_{-\delta}(\ell-1)^{\sharp} \quad&(1 \le \ell \le n+1).  
  \end{cases}
\end{equation}
In view of \eqref{eqn:LNS218}, 
 the irreducible representation $\Pi_{\ell,\delta}$ of $G$ is well-defined.  

\begin{remark}
\label{rem:flatsharp}
The point here is 
 that each irreducible representation
 $\Pi_{\ell,\delta}$ ($1 \le \ell \le n$, $\delta=\pm$)
 can be realized in two different principal series representations:
\begin{align*}
   I_{\delta}(\ell,\ell) =& {\operatorname{Ind}}_P^G (\Exterior^\ell({\mathbb{C}}^n) \otimes \delta \otimes {\mathbb{C}}_\ell), 
\\
I_{-\delta}(\ell-1,\ell-1) 
=& {\operatorname{Ind}}_P^G (\Exterior^{\ell-1}({\mathbb{C}}^n) \otimes (-\delta) \otimes {\mathbb{C}}_{\ell-1}).  
\end{align*}
\end{remark}

\begin{theorem}
\label{thm:LNM20}
Let $G=O(n+1,1)$ $(n \ge 1)$.  
\begin{enumerate}
\item[{\rm{(1)}}]
For $0 \le \ell \le n$ and $\delta \in \{ \pm \}$, 
 we have exact sequences of $G$-modules:
\begin{align*}
&0 \to \Pi_{\ell,\delta} \to I_{\delta}(\ell,\ell) \to \Pi_{\ell+1,-\delta}
\to 0, 
\\
&0 \to \Pi_{\ell+1,-\delta} \to I_{\delta}(\ell,n-\ell) \to \Pi_{\ell,\delta} \to 0.  
\end{align*}
These exact sequences split
 if and only if $n=2\ell$.  

\item[{\rm{(2)}}]
Irreducible admissible smooth representations of $G$ 
 with trivial
\index{B}{infinitesimalcharacter@infinitesimal character}
 ${\mathfrak {Z}}_G({\mathfrak {g}})$-infinitesimal character
\index{A}{1parho@$\rho_G$}
 $\rho_G$
 can be classified as
\index{A}{IrrGrho@${\operatorname{Irr}}(G)_{\rho}$,
 set of irreducible admissible smooth representation of $G$
 with trivial infinitesimal character $\rho$\quad}
\[
  {\operatorname{Irr}}(G)_{\rho}
  =\{
    \Pi_{\ell, \delta}
    :
    0 \le \ell \le n+1, \, \delta = \pm
\}.  
\]

\item[{\rm{(3)}}]
For any $0 \le \ell \le n+1$
 and $\delta \in \{\pm\}$, 
 the 
\index{B}{minimalKtype@minimal $K$-type}
minimal $K$-type of the irreducible $G$-module $\Pi_{\ell,\delta}$
 is given 
by $\mub(\ell,\delta)=\Exterior^\ell({\mathbb{C}}^{n+1}) \boxtimes \delta$.  

\item[{\rm{(4)}}]
There are four one-dimensional representations of $G$, 
 and they are given by 
\begin{equation*}
\{
  \Pi_{0,+} \simeq {\bf{1}}, \quad \Pi_{0,-} \simeq \chi_{+-}, 
\quad
   \Pi_{n+1,+} \simeq \chi_{-+}, \quad 
   \Pi_{n+1,-} \simeq \chi_{--} (=\det)
\}.  
\end{equation*}
The other representations $\Pi_{\ell,\delta}$
 $(1 \le \ell \le n, \delta \in \{\pm\})$
 are infinite-dimensional.  
\item[{\rm{(5)}}]
There are isomorphisms as $G$-modules
 for any $0 \le \ell \le n+1$ and $\delta = \pm$:
\index{A}{1chipmpm@$\chi_{\pm\pm}$, one-dimensional representation of $O(n+1,1)$}
\begin{align*}
   \Pi_{\ell,\delta} \otimes \chi_{+-} \simeq \, & \Pi_{\ell,-\delta}, 
\\
   \Pi_{\ell,\delta} \otimes \chi_{-+} \simeq \, & \Pi_{n+1-\ell,\delta}, 
\\   
   \Pi_{\ell,\delta} \otimes \chi_{--} \simeq \, & \Pi_{n+1-\ell,-\delta}.  
\end{align*}
\item[{\rm{(6)}}]
Every $\Pi_{\ell,\delta}$ $(0 \le \ell \le n+1, \delta = \pm)$
 is unitarizable and self-dual.  
\item[{\rm{(7)}}]
For $n$ odd, 
 there are exactly two inequivalent 
\index{B}{discreteseries@discrete series representation}
discrete series representations
 of $G=O(n+1,1)$ with infinitesimal character $\rho_G$.  
Their smooth representations
 are given by 
\[
  \{
   \Pi_{\frac {n+1}{2},\delta}
   :
    \delta = \pm
  \}.  
\]

\par
All the other representations in the list (2)
 are nontempered representations of $G$.  
\item[{\rm{(8)}}]
For $n$ even, 
 there are exactly four inequivalent irreducible
\index{B}{temperedrep@tempered representation}
 tempered representations of $G=O(n+1,1)$
 with infinitesimal character $\rho_G$.  
Their smooth representations are given by 
\[
  \{
   \Pi_{\frac {n}{2},\delta}, \Pi_{\frac {n}{2}+1,\delta}
   :
    \delta = \pm
  \}.  
\]
\item[{\rm{(9)}}]
Irreducible and unitarizable $({\mathfrak {g}}, K)$-modules
 with nonzero $({\mathfrak {g}}, K)$-cohomologies are
 exactly given as the set
 of the underlying $({\mathfrak {g}}, K)$-modules of $\Pi_{\ell, \delta}$
 $(0 \le \ell \le n+1, \delta = \pm)$.  
\end{enumerate}
\end{theorem}

The exact sequences in Theorem \ref{thm:LNM20} (1) leads us
 to a labeling of the finite set ${\operatorname{Irr}}(G)_{\rho}$
 as follows: 
\begin{definition}
[standard sequence]
\label{def:Pii}
Let $G=O(n+1,1)$
 and $n=2m$ or $2m-1$.  
We refer to the sequence 
\begin{center}
\begin{tabular}{ccccccccccc}
& &$\Pi_{0,+}$&,&$\Pi_{1,+} $ & ,  &\dots & , & $\Pi_{m-1,+} $ &, & $\Pi_{m,+}$ 
\end{tabular}
\end{center}
as the 
\index{B}{standardsequence@standard sequence|textbf}
{\it standard sequence starting
 with the trivial one-dimensional representation 
\index{A}{01one@${\bf{1}}$, trivial one-dimensional representation}
$\Pi_{0,+}={\bf{1}}$}.  
Likewise,
 we refer to the sequence
\begin{center}
\begin{tabular}{ccccccccccc}
& &$\Pi_{0,-}$&,&$\Pi_{1,-} $ & ,  &\dots & , & $\Pi_{m-1,-} $ &, & $\Pi_{m,-}$ 
\end{tabular}
\end{center}
as the standard sequence starting with the one-dimensional representation 
 $\Pi_{0,-}=\chi_{+-}$.  
Sometimes we suppress the subscript $+$
 and write 
\index{A}{1Pii@$\Pi_{i}=\Pi_{i,+}$|textbf}
 $\Pi_{i}$ for $\Pi_{i,+}$ for simplicity.  
\end{definition}



More generally,
 we shall define the standard sequence 
 starting with other irreducible finite-dimensional representations of $G$
 in Chapter \ref{sec:conjecture}, 
 see Definition \ref{def:Hasse} and Example \ref{ex:171455}. 
An analogous sequence, 
 which we refer to as the Hasse sequence,
 will be defined also in Chapter \ref{sec:conjecture}, 
 see Definition-Theorem \ref{def:UHasse}.   

\vskip 1pc
We give some remarks on the proof of Theorem \ref{thm:LNM20}.  
Basic references are \cite{BW, C, KKP}.  
Theorem \ref{thm:LNM20} (1) generalizes the results proved
 in Borel--Wallach 
 \cite[pp.~128--129 in the new edition; p.~192 in the old edition]{BW}
 for the identity component group 
 $G_0=SO_0(n+1,1)$.  
(Unfortunately and confusingly the restriction
 of our representations $I_+(i,i)$
 to the connected component $G_0$ are denoted there by $I_i$
 when $n \ne 2i$.) 
See also Collingwood \cite[Chap.~5, Sect.~2]{C} for 
 the identity component group $G_0$;
\cite[p.~20]{KKP}
 for the disconnected group $G=O(n+1,1)$. 



For the relationship
 between principal series representations of $G$
 and of its identity component group $G_0$, 
 we recall from \cite[Chap.~5]{sbon} the following.  
\begin{lemma}
\label{lem:GPconn}
For $G=O(n+1,1)$, 
 let $P_0:=P \cap G_0$.  
Then $P_0$ is connected, 
 and is a minimal parabolic subgroup of $G_0$.  
Then we have a natural bijection:
\[
G_0/P_0 \overset \sim \to G/P \ 
(\simeq S^n).  
\]
\end{lemma}
Then we can derive results
 for the disconnected group $G$ from those for the connected group $G_0$
 and {\it{vice versa}}
 by using the action of the Pontrjagin dual $(G/G_0)\hspace{-1mm}{\widehat{\hphantom{m}}}$
 of the component group $G/G_0$
 and the classical branching law
 $O(N) \downarrow SO(N)$
 (Section \ref{subsec:OnSOn}).  
In Appendix II (Chapter \ref{sec:SOrest}) 
 we discuss restrictions
 of representations of $O(n+1,1)$
 with respect to $SO(n+1,1)$
 in the same spirit.  



In Proposition \ref{prop:161655} of Appendix I, 
 we will give a description
 of the underlying $({\mathfrak{g}},K)$-modules $(\Pi_{i,\pm})_K$
 of the $G$-irreducible subquotients $\Pi_{i,\pm}$
 in terms of the so-called 
\index{A}{Aqlmd@$A_{\mathfrak{q}}(\lambda)$}
$A_{\mathfrak{q}}(\lambda)$-modules,
 {\it{i.e.,}} cohomologically induced representations from 
 one-dimensional representations
 of a $\theta$-stable parabolic subalgebra ${\mathfrak{q}}$.  



By using the description,
 Theorem \ref{thm:LNM20} (9) follows readily from 
 results of Vogan and Zuckerman \cite{VZ}, 
 see Proposition \ref{prop:gKq} in Appendix I.  
The unitarizability of the irreducible subquotients
 $\Pi_{i,\pm}$
 (Theorem \ref{thm:LNM20} (6)) traces
 back to T.~Hirai \cite{Hirai62}, 
 see also Howe and Tan \cite{HT93}.  
Alternatively,
 the unitarizability in Theorem \ref{thm:LNM20} (6)
 is deduced from the theory
 on $A_{\mathfrak{q}}(\lambda)$, 
 see \cite[Thm.~0.51]{KV}.  



\begin{remark}
Analogous results for the special orthogonal group $SO(n+1,1)$
 will be given in Proposition \ref{prop:161648}
 in Appendix II, 
 where we denote the group by $\overline G$.  
\end{remark}



