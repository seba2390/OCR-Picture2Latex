{}~~~~~{}
\newpage
%%%%%%%%%%%%%%%%%%%%%%%%%%%%%%%%%%%%%%%%%%%%%%
\section{Regular symmetry breaking operators}
\label{sec:section7}
%%%%%%%%%%%%%%%%%%%%%%%%%%%%%%%%%%%%%%%%%%%%%
\index {B}{regularsymmetrybreakingoperator@regular symmetry breaking operator}

Let 
\index{A}{IdeltaV@$I_{\delta}(V, \lambda)$}
$
I_{\delta}(V,\lambda)
$
 be a principal series representation
 of $G=O(n+1,1)$
 realized in the Fr{\'e}chet space 
 $C^{\infty}(G/P, {\mathcal{V}}_{\lambda,\delta})$, 
 and 
\index{A}{JWepsilon@$J_{\varepsilon}(W, \nu)$}
$
J_{\varepsilon}(W,\nu)
$
 that of $G'=O(n,1)$
 realized in $C^{\infty}(G'/P', {\mathcal{W}}_{\nu,\varepsilon})$
 as in Section \ref{subsec:smoothI}.  
In this chapter 
 we apply the general result in \cite[Chap.~3]{sbon} to construct a 
 \lq\lq{matrix-valued regular symmetry breaking operators}\rq\rq\
\index{A}{Ahtg0@$\Atbb \lambda \nu {\pm} {V,W}$}
$
   \Atbb \lambda \nu {\pm} {V,W}
   \colon
   I_{\delta}(V,\lambda) \to J_{\pm \delta}(W,\nu)
$
 that depend holomorphically on $(\lambda, \nu) \in {\mathbb{C}}^2$. 
We shall prove
 that the normalization \eqref{eqn:KVWpt} and \eqref{eqn:KVWmt}
 is optimal
 in the sense
 that the zeros of the operator
 $\Atbb \lambda \nu \pm {V,W}$
 are of codimension $>1$ in the parameter space of $(\lambda, \nu)$, 
 that is, 
 discrete in ${\mathbb{C}}^2$ in our setting.  
A key idea of the proof is a reduction to the scalar case.   

\subsection{Generalities}
We recall from the general theory \cite[Chap.~3]{sbon}
 on the distribution kernels 
 of symmetry breaking operators, 
 which will be the basic tool in this chapter.  
Furthermore,
 we discuss some subtle questions
 on the underlying topology 
 of representation spaces for symmetry breaking, 
 see Theorem \ref{thm:SBOBF}.  

\subsubsection{Distribution kernels of symmetry breaking operators}

Throughout this monograph,
 we shall regard distributions
 as the dual of compactly supported smooth densities
 rather than that of compactly supported smooth functions.
Thus we treat distributions as \lq\lq{generalized functions}\rq\rq, 
 and write their pairing 
 with test functions by using the integral symbol,  
 as if they were ordinary functions
 (with densities).  



Let $G \supset G'$ be a pair of real reductive Lie groups, 
 and $P$, $P'$ their parabolic subgroups.  
We do not require an inclusive relation $P \supset P'$ in this subsection.  
Let $(\widetilde \sigma,V)$ be a finite-dimensional representation of $P$, 
 and $(\widetilde \tau,W)$ that of the subgroup $P'$.  
We form homogeneous vector bundles
 over flag manifolds
 by 
\begin{alignat*}{2}
{\mathcal{V}}:=& G \times_P V &&\to G/P, 
\\
{\mathcal{W}}:=& G' \times_{P'} W &&\to G'/P'.   
\end{alignat*}


We write ${\operatorname{Ind}}_P^G(\widetilde \sigma)$
 for the admissible smooth representation 
 of $G$ on the Fr{\'e}chet space $C^{\infty}(G/P,{\mathcal{V}})$, 
 and ${\operatorname{Ind}}_{P'}^{G'}(\widetilde \tau)$
 for that of the subgroup of $G'$ on $C^{\infty}(G'/P',{\mathcal{W}})$.  



We denote by ${\mathcal{V}}^{\ast}$ 
 the dualizing bundle of ${\mathcal{V}}$, 
 which is a $G$-homogeneous vector bundle over $G/P$ 
 associated to the representation
\[
   V^{\ast}:=V^{\vee} \otimes |\det ({\operatorname{Ad}}_{{\mathfrak{g}}/{\mathfrak{p}}})|^{-1}
\]
 of the group $P$, 
where $V^{\vee}$ denotes the contragredient representation
 of $(\widetilde \sigma,V)$.  
Then the regular representation of $G$
 on the space ${\mathcal{D}}'(G/P, {\mathcal{V}}^{\ast})$
 of ${\mathcal{V}}^{\ast}$-valued distribution sections
 is the dual of the representation on $C^{\infty}(G/P, {\mathcal{V}})$.  



The Schwartz kernel theorem guarantees 
 that any symmetry breaking operator can be expressed
 by using a distribution kernel.  
Conversely,
 distributions that give rise to symmetry breaking operators
 are characterized as follows.  
\begin{fact}
[{\cite[Prop.~3.2]{sbon}}]
\label{fact:SBOdistr}
There are natural linear bijections:
\[
{\operatorname{Hom}}_{G'}
 (C^{\infty}(G/P, {\mathcal{V}})|_{G'}, 
  C^{\infty}(G'/P', {\mathcal{W}}))
\simeq\,  {\mathcal{D}}'(G/P \times G'/P', {\mathcal{V}}^{\ast} \boxtimes {\mathcal{W}}
)^{\Delta(G')}.  
\]
Here ${\mathcal{V}}^{\ast} \boxtimes {\mathcal{W}}$ denotes
 the outer tensor product bundle 
 over the direct product manifold $G/P \times G'/P'$.  
\end{fact}
We note that the multiplication map
\[
  m \colon G \times G' \to G, 
\quad
  (x,y) \mapsto y^{-1} x
\]
 induces a linear bijection
\[
{\mathcal{D}}'(G/P \times G'/P',
  {\mathcal{V}}^{\ast} \boxtimes {\mathcal{W}})
^{\Delta(G')}
\underset{m^{\ast}}{\overset \sim \leftarrow}
({\mathcal{D}}'(G/P, {\mathcal{V}}^{\ast}) \otimes W)^{\Delta(P')}, 
\]
where
 the right-hand side stands for the space
 of $P'$-invariant vectors
 under the diagonal action
 on the tensor product of the $G$-module
 ${\mathcal{D}}'(G/P, 
  {\mathcal{V}}^{\ast})$
 and the $P'$-module $W$.  



Thus Fact \ref{fact:SBOdistr} may be reformulated
 as the following linear bijection
\begin{equation}
\label{eqn:SBOdistr}
   {\operatorname{Hom}}_{G'}
   (C^{\infty}(G/P, {\mathcal{V}})|_{G'}, C^{\infty}(G'/P', {\mathcal{W}}))
   \simeq
   ({\mathcal{D}}'(G/P, {\mathcal{V}}^{\ast}) \otimes W)^{\Delta(P')}.  
\end{equation}


The point of Fact \ref{fact:SBOdistr} is 
 that the map
\[
   C^{\infty}(G/P, {\mathcal{V}}) 
  \to {\mathcal{D}}'(G'/P', {\mathcal{W}}), 
\qquad
   f \mapsto \int_X K(x,y) f(x)
\]
to the space ${\mathcal{D}}'(G'/P', {\mathcal{W}})$
 of {\it{distribution}} sections
 becomes automatically a continuous map
 to the space $C^{\infty}(G'/P', {\mathcal{W}})$
 of {\it{smooth}} sections 
 for any $K\in {\mathcal{D}}'(G/P \times G'/P', {\mathcal{V}}^{\ast} \boxtimes {\mathcal{W}})^{\Delta(G')}$.  
This observation leads us to the proof of the isomorphism 
 \eqref{eqn:SBOBF} in Theorem \ref{thm:SBOBF}.  



\subsubsection{Invariant bilinear forms
 on admissible smooth representations
 and symmetry breaking operators}
\label{subsec:BFSBO}
We retain the setting
 of the previous subsection,
 in particular,
 we suppose
 that $G \supset G'$
 are a pair of real reductive Lie groups.  



Let $(\Pi,U)$ and $(\pi,U')$ be admissible smooth representations
 of $G$ and $G'$, 
 respectively.  
We recall that the underlying topological vector space
 of any admissible smooth representation
 is a nuclear Fr{\'e}chet space.  
We define $\Pi \boxtimes \pi$ 
 to be the natural representation
 of the direct product group $G \times G'$
 on the space $U \widehat \otimes U'$.  
In this subsection,
 we study the space
 ${\operatorname{Hom}}_{G'}(\Pi \boxtimes \pi,{\mathbb{C}})$
 of continuous functionals
 that are invariant under the diagonal action
 of the subgroup $G'$.  



For an admissible smooth representation $(\Pi,U)$ of $G$, 
 we denote by $\Pi^{\vee}$ the contragredient representation 
 of $\Pi$
 in the category of admissible smooth representations, 
 namely,
 the Casselman--Wallach minimal globalization of $(\Pi^{\vee})_K$
 (\cite[Chap.~11]{W}).  
The topological dual $U^{\vee}$ of $U$ is the space
 of distribution vectors,
 on which we can define a continuous representation of $G$.  
This is the maximal globalization of $(\Pi^{\vee})_K$
 in the sense of Casselman--Wallach, 
 which we refer to $(\Pi^{\vee})^{-\infty}$.  
Thus we have
\[
   (\Pi^{\vee})_K \subset \Pi^{\vee} \subset (\Pi^{\vee})^{-\infty}.  
\]
We shall use these symbols
 for a representation $\pi$ of the subgroup $G'$ below.  

\begin{example}
\label{ex:psdual}
Let $\widetilde \tau$ be a finite-dimensional representation
 of a parabolic subgroup $P'$ of $G'$, 
 and $\pi:={\operatorname{Ind}}_{P'}^{G'}(\widetilde \tau)$
 the representation on $C^{\infty}(G'/P', {\mathcal{W}})$.  
The dualizing bundle ${\mathcal{W}}^{\ast}$ is given
 as the $G'$-homogeneous vector bundle 
 over $G'/P'$
 associated to $\tau^{\ast}:= \widetilde \tau^{\vee} \otimes |\det ({\operatorname{Ad}}|_{{\mathfrak{g}}'/{\mathfrak{p}}'})|^{-1}$, 
 where $\widetilde \tau^{\vee}$ is the contragredient representation of $\widetilde \tau$.  
Then the smooth admissible representation $\pi^{\vee}$ of $G'$
 is given as a representation 
$
{\operatorname{Ind}}_{P'}^{G'}(\tau^{\ast})
$ 
 on $C^{\infty}(G'/P', {\mathcal{W}}^{\ast})$, 
 whereas $(\pi^{\vee})^{-\infty}$
 is given as a representation
 on ${\mathcal{D}}'(G'/P', {\mathcal{W}}^{\ast})$.  
\end{example}
Any symmetry breaking operator $T \colon \Pi|_{G'} \to \pi^{\vee}$
 induces a continuous bilinear form
\[
  \Pi \boxtimes \pi \to {\mathbb{C}}, 
\qquad
   u \otimes v \mapsto \langle T u, v \rangle,
\]
and we have a natural embedding
\begin{equation}
\label{eqn:SBOtoBF}
   {\operatorname{Hom}}_{G'}(\Pi|_{G'}, \pi^{\vee})
   \hookrightarrow
   {\operatorname{Hom}}_{G'}(\Pi \boxtimes \pi,{\mathbb{C}})
   \simeq
   {\operatorname{Hom}}_{G'}(\Pi|_{G'}, (\pi^{\vee})^{-\infty}).  
\end{equation}
Here the second isomorphism follows from the natural bijections 
 for nuclear Fr{\'e}chet spaces
 (\cite[Prop.~50.7]{Treves}):
\[
   {\operatorname{Hom}}_{{\mathbb{C}}}(U \otimes U',{\mathbb{C}})
   \simeq
   {\operatorname{Hom}}_{{\mathbb{C}}}(U, (U')^{\vee}), 
\]
  where ${\operatorname{Hom}}_{{\mathbb{C}}}$ denotes the space of continuous linear maps.  



As an immediate consequence of Fact \ref{fact:SBOdistr}, 
 we have the following:
\begin{proposition}
\label{prop:BSBOps}
Suppose $\widetilde \sigma$ and $\widetilde \tau$ are finite-dimensional representations
 of parabolic subgroups $P$ and $P'$,
 respectively.  
Let $\Pi={\operatorname{Ind}}_P^G (\widetilde \sigma)$
 and $\pi={\operatorname{Ind}}_{P'}^{G'} (\widetilde \tau)$
 be admissible smooth representations
 of $G$ and $G'$, 
 respectively.  
Then the embedding in \eqref{eqn:SBOtoBF} is an isomorphism.  
\end{proposition}

\begin{proof}
We recall
 that ${\operatorname{Hom}}_{{\mathbb{C}}}(\cdot, {\mathbb{C}})$ denotes 
 the space of (continuous) functionals.  
Then ${\operatorname{Hom}}_{G'}(\Pi \boxtimes \pi,{\mathbb{C}})$
 is naturally isomorphic 
 to the spaces of $G'$-invariant elements
 of the following vector spaces
\[
   {\operatorname{Hom}}_{{\mathbb{C}}}
   (C^{\infty}(G/P \times G'/P', {\mathcal{V}} \boxtimes {\mathcal{W}}), {\mathbb{C}})
   \simeq
   {\mathcal{D}}'(G/P \times G'/P', {\mathcal{V}}^{\ast} \boxtimes {\mathcal{W}}^{\ast}), 
\]
 and so we have
\[
   {\operatorname{Hom}}_{G'}(\Pi \boxtimes \pi,{\mathbb{C}})
    \simeq
    {\mathcal{D}}'(G/P \times G'/P', {\mathcal{V}}^{\ast} \boxtimes {\mathcal{W}}^{\ast})^{\Delta(G')}.  
\]
Since $\tau^{\ast\ast} \simeq \tau$, 
 the right-hand side is canonically isomorphic to 
\[{\operatorname{Hom}}_{G'}
   (C^{\infty}(G/P, {\mathcal{V}})|_{G'}, C^{\infty}(G'/P',{\mathcal{W}}^{\ast}))
\simeq 
{\operatorname{Hom}}_{G'}(\Pi|_{G'}, \pi^{\vee})
\]
 by Fact \ref{fact:SBOdistr} and Example \ref{ex:psdual}.  
Hence Proposition \ref{prop:BSBOps} is proved.  
\end{proof}
More generally,
 we obtain the following.  
\begin{theorem}
\label{thm:SBOBF}
Let $G \supset G'$ be a pair of real reductive Lie groups.  
For any $\Pi \in {\operatorname{Irr}}(G)$
 and $\pi \in {\operatorname{Irr}}(G')$, 
 we have a canonical bijection:
\begin{equation}\label{eqn:SBOBF}
   {\operatorname{Hom}}_{G'}(\Pi|_{G'}, \pi^{\vee})
   \overset \sim \rightarrow
   {\operatorname{Hom}}_{G'}(\Pi \boxtimes \pi,{\mathbb{C}}).  
\end{equation}
\end{theorem}
By the second isomorphism \eqref{eqn:SBOtoBF}, 
Theorem \ref{thm:SBOBF} is deduced from the following proposition, 
 where we change the notation from 
 $\pi^{\vee}$ to $\pi$ for simplicity.  


\begin{proposition}
\label{prop:SBOtarget}
Suppose $\Pi \in {\operatorname{Irr}}(G)$
 and $\pi \in {\operatorname{Irr}}(G')$, 
Let $\pi^{-\infty}$ be the representation of $G'$
 on distribution vectors.  
Then the natural embedding
\[
   {\operatorname{Hom}}_{G'}(\Pi|_{G'}, \pi)
   \hookrightarrow
   {\operatorname{Hom}}_{G'}(\Pi|_{G'}, \pi^{-\infty})
\]
 is a bijection.  
\end{proposition}
\begin{proof}
[Proof of Proposition \ref{prop:SBOtarget}]
We take $P$ and $P'$ to be minimal parabolic subgroups of $G$ and $G'$, 
 respectively.  
By Casselman's subrepresentation theorem
 (or equivalently, \lq\lq{quotient theorem}\rq\rq), 
 see \cite[Chap.~3, Sect.~8]{W} for instance,
 for any $\Pi \in {\operatorname{Irr}}(G)$, 
 there exists an irreducible finite-dimensional representation
 $(\widetilde \sigma, V)$ of $P$
 such that
 $\Pi_K$ is obtained as a quotient
 of ${\operatorname{Ind}}_P^{G}(\widetilde \sigma)_K$, 
 and therefore,
 there is a surjective continuous $G$-homomorphism
$p \colon C^{\infty}(G/P, {\mathcal{V}}) \to \Pi$
 by the automatic continuity theorem
 \cite[Chap.~11, Sect.~4]{W}.  
Likewise, 
 for any $\pi \in {\operatorname{Irr}}(G')$, 
 there exists an irreducible finite-dimensional representation
 $(\widetilde \tau, W)$ of $P'$
 such that
 $\pi_{K'}$ is a subrepresentation of 
 ${\operatorname{Ind}}_{P'}^{G'}(\widetilde \tau)_{K'}$,
 and therefore, 
 there is an injective continuous $G'$-homomorphism
$\iota \colon \pi^{-\infty} \hookrightarrow {\mathcal{D}}'(G'/P', {\mathcal{W}})$
 by the dual of the automatic continuity theorem.  
If $T \colon \Pi \to\pi^{-\infty}$ is a continuous $G'$-homomorphism, 
 then $T$ induces a continuous $G'$-homomorphism
\[
  \iota\circ T \circ p \colon C^{\infty}(G/P, {\mathcal{V}}) \to {\mathcal{D}}'(G'/P', {\mathcal{W}}).  
\] 
By Proposition \ref{prop:BSBOps}, 
 $\iota\circ T \circ p$ is actually a continuous $G'$-homomorphism, 
\[
  C^{\infty}(G/P, {\mathcal{V}}) \to C^{\infty}(G'/P', {\mathcal{W}}).  
\] 
Hence the image of $T$ is contained in the admissible smooth representation $\pi$.  
Since the topology of the admissible smooth representation $\pi$ coincides
 with the relative topology of $C^{\infty}(G'/P', {\mathcal{W}})$, 
 $T$ is actually a $G'$-homomorphism $\Pi|_{G'} \to \pi$.  
\end{proof}

\begin{remark}
\begin{enumerate}
\item[{\rm{(1)}}]
In \cite[Lem.~A.0.8]{AG},
 the authors proved 
 the injectivity of the map \eqref{eqn:SBOtoBF}.  
\item[{\rm{(2)}}]
Theorem \ref{eqn:SBOBF} simplifies
 part of the proof
 of \cite[Thm.~4.1]{xkShintani}
 on twelve equivalence conditions
 including the finiteness
 criterion for the dimension
 of continuous invariant bilinear forms.  
\end{enumerate}
\end{remark}

\subsection{Distribution kernels of symmetry breaking operators
 for $G=O(n+1,1)$}
We analyze the distribution kernels
 of symmetry breaking operators
 in coordinates.  
For this,
 we set up some structural results
 for $G=O(n+1,1)$.  

%%%%%%%%%%%%%%%%%%%%%%%%%%%%%%%%%%%%%%%%%%%%%%%%%%
\subsubsection{Bruhat and Iwasawa decompositions for $G=O(n+1,1)$}
\label{subsec:KAN}
%%%%%%%%%%%%%%%%%%%%%%%%%%%%%%%%%%%%%%%%%%%%%%%%%%

We recall from \eqref{eqn:psim}
 that the map $\psi_n \colon {\mathbb{R}}^n \setminus \{0\}\to O(n)$, 
 $x \mapsto \psi_n(x)$
 is defined as the reflection
 with respect to the hyperplane orthogonal to $x$.  
By using $\psi_n(x)$, 
 we give an explicit formula
 of the Bruhat decomposition
 $G=N_+ w MAN_+
 \cup MA N_+$
 and the Iwasawa decomposition
 $G=KAN_+$
 for an element of $N_-$
 for $G=O(n+1,1)$.  
Here we set
\index{A}{winversion@$w$, inversion element|textbf}
\begin{equation}
\label{eqn:w}
w:= \operatorname{diag} (1,\cdots,1,-1) \in N_K({\mathfrak{a}}).  
\end{equation}


Retain the notation as in Section \ref{subsec:subgpOn}.  
In particular, 
 we recall from \eqref{eqn:nplus} and \eqref{eqn:nbar}
 the definition 
of the diffeomorphisms
\index{A}{n1@$n_+\colon {\mathbb{R}}^n \to N_+$}
$
n_+ \colon {\mathbb{R}}^n \overset \sim \to N_+
$
 and 
\index{A}{n2@$n_-\colon {\mathbb{R}}^n \to N_-$}
$
n_- \colon {\mathbb{R}}^n \overset \sim \to N_-
$, 
 respectively.  
\begin{lemma}
[Bruhat decomposition]
\label{lem:0.1}
\index{B}{Bruhatdecomposition@Bruhat decomposition}
For $b \in {\mathbb{R}}^n - \{0\}$, 
\[
n_-(b) = n_+(a)
\begin{pmatrix}
-1 & & 
\\
   & \psi_n(b) &
\\
 & & 1
\end{pmatrix}
e^{tH} n, 
\]
where $a \in {\mathbb{R}}^n$
 and $t \in {\mathbb{R}}$ are given uniquely by 
 $a =-\frac{b}{|b|^2}$ and $e^t =|b|^2$, 
 respectively, 
 and $n \in N_+$.  
\end{lemma}

\begin{proof}
Suppose that $a \in {\mathbb{R}}^n$, 
 $\varepsilon = \pm 1$, 
 $B \in O(n)$, 
 $t \in {\mathbb{R}}$
 and $n \in N_+$ satisfies
\begin{equation}
\label{eqn:nBruhat}
n_-(b)= n_+(a) w 
\begin{pmatrix} \varepsilon & & 
\\
                            & B & 
\\
                            & & \varepsilon
\end{pmatrix} e^{t H} n. 
\end{equation}
Applying \eqref{eqn:nBruhat}
 to the vector $p_+ ={}^{t\!}(1,0,\cdots,0,1)\in \Xi$
 (see \eqref{eqn:p+}),  
 we have
\[
\begin{pmatrix}
1-|b|^2
\\
2b
\\
1+ |b|^2
\end{pmatrix}
=
\varepsilon 
e^t
\begin{pmatrix}
1-|a|^2
\\
2a
\\
-1 - |a|^2
\end{pmatrix}.  
\]
Hence $\varepsilon =-1$, 
 $e^t=\frac{1}{|a|^2}$, 
 and $a = -|a|^2 b$.  
Thus $|a|\,|b|=1$.  
In turn, 
 \eqref{eqn:nBruhat} amounts to 
\[
 n_+(a)^{-1} n_-(b)
=
\begin{pmatrix} -1 & & 
\\
                            & B & 
\\
                            & & 1
\end{pmatrix} 
e^{t H} n, 
\]
whence 
$
  B = I_n + 2 a {}^{t\!}b
    = I_n - \frac{2 b {}^{t\!}b}{|b|^2}
    =\psi_n(b).  
$
\end{proof}
For $b \in {\mathbb{R}}^n$, 
 we define $k(b) \in SO(n+1)$ by 
\index{A}{kb@$k(b)$|textbf}
\begin{equation}
\label{eqn:kb}
   k(b):=I_{n+1} + \frac{1}{1+|b|^2}
        \begin{pmatrix}
         -2 |b|^2 & -2\, {}^{t\!}b
         \\
         2 b   & -2\,b\,{}^{t\!}b
        \end{pmatrix}
       = \psi_{n+1}(1,b) \begin{pmatrix} -1 & \\ & I_n \end{pmatrix}.  
\end{equation}
\begin{lemma}
[Iwasawa decomposition]
\label{lem:0.2}
\index{B}{Iwasawadecomposition@Iwasawa decomposition}
For any $b \in {\mathbb{R}}^n$, 
 we have 
\index{A}{n2@$n_-\colon {\mathbb{R}}^n \to N_-$}
\begin{equation}
\label{eqn:nKAN}
   n_-(b)=k (b) e^{t H} n_+(a) 
\in KAN_+, 
\end{equation}
where $a \in {\mathbb{R}}^n$ 
 and $t \in {\mathbb{R}}$ are given by 
$
   a= \frac{-b}{1+|b|^2}
$
 and 
$
   e^t=1+|b|^2. 
$
\end{lemma}
\begin{proof}
We shall prove 
 that $k(b)$ in \eqref{eqn:nKAN} is given
 by the formula \eqref{eqn:kb}.  
Since $n_-(b)$ is contained 
 in the connected component of $G$, 
$k(b)=(k(b)_{ij})_{0 \le i,j \le n}$
 in \eqref{eqn:nKAN}
 belongs to the connected group $SO(n+1)$.  
We write $k(b)=(k(b)_0, k'(b))$
 where $k(b)_0 \in {\mathbb{R}}^{n+1}$
 and $
  k'(b):=(k(b)_{ij})_{\substack {0 \le i \le n \\ 1 \le j \le n}}
  \in M(n+1,n;{\mathbb{R}})
$.  
Applying \eqref{eqn:nKAN} to the vector
\index{A}{p1@$p_+={}^t(1,0,\cdots,0,1)$}
 $p_+ ={}^{t\!} (1,0,\cdots,0,1)$, 
we have
\[
\begin{pmatrix}
1-|b|^2 \\ 2 b\\ 1+|b|^2
\end{pmatrix}
=
e^t
\begin{pmatrix}
k(b)_0 \\ 1
\end{pmatrix}.  
\]
The last component shows
 $e^t=1+|b|^2$.  
In turn, 
 we get the first column vector $k(b)_0$ of $k(b)$.  
On the other hand, 
 we observe 
\[
  k(b)_{i j}
  = 
  (n_-(b) n_+(a)^{-1} e^{-t H})_{ij}
  =
  (n_-(b) n_+(a)^{-1} )_{ij}
\]
for $0 \le i \le n+1$ and $1 \le j \le n$.  
Hence we get
\[
\begin{pmatrix}
k'(b)
\\
 0 \cdots 0
\end{pmatrix}
=
\begin{pmatrix}
(1-|b|^2) {}^{t\!}a-{}^{t\!}b
\\
I_n +2 b {}^{t\!}a
\\
(1+|b|^2) {}^{t\!}a + {}^{t\!}b
\end{pmatrix}, 
\]
which implies
\[
  a=-\frac{b}{1+|b|^2}
\quad
\text{ and }
\quad
  k'(b)
  =\begin{pmatrix}
   -\frac{1-|b|^2}{1+|b|^2} {}^{t\!}b-{}^{t\!}b
  \\
   I_n -\frac{2 b {}^{t\!}b}{1+|b|^2}
   \end{pmatrix}
  =\begin{pmatrix}
   \frac{-2}{1+|b|^2}{}^{t\!}b
   \\
   I_n-\frac{2 b\, {}^{t\!}b}{1+|b|^2}
   \end{pmatrix}.
\]
In particular,
  we have shown 
 that $k(b)$ in \eqref{eqn:nKAN} is given by the formula \eqref{eqn:kb}.  
\end{proof}

%%%%%%%%%%%%%%%%%%%%%%%%%%%%%%%%%%%%%%%%%%%%%%%%%%%%%%%%%%%%%%%%
\subsubsection{Distribution kernels for symmetry breaking operators}
%%%%%%%%%%%%%%%%%%%%%%%%%%%%%%%%%%%%%%%%%%%%%%%%%%%%%%%%%%%%%%%%
We apply Fact \ref{fact:SBOdistr} to the pair
 $(G,G')=(O(n+1,1),O(n,1))$
 and a pair of the minimal parabolic subgroups $P$ and $P'$.  
With the notation of Fact \ref{fact:SBOdistr}, 
 we shall take
\begin{alignat*}{2}
\widetilde \sigma =\, & V \otimes \delta \otimes {\mathbb{C}}_{\lambda}
\quad
&&\text{on $V_{\lambda,\delta}$}
\\
\widetilde \tau =\, & W \otimes \varepsilon \otimes {\mathbb{C}}_{\nu}
\quad
&&\text{on $W_{\varepsilon,\nu}$}
\end{alignat*}
as (irreducible) representations of $P$ and $P'$, 
 respectively,
 for $(\sigma, V) \in \widehat{O(n)}$, 
 $\delta \in \{\pm\}$, 
 and $\lambda \in {\mathbb{C}}$
 and $(\tau, W) \in \widehat{O(n-1)}$, 
$\varepsilon \in \{\pm\}$, 
 and $\nu \in {\mathbb{C}}$.  
We recall from \eqref{eqn:Vlmdbdle}
 that 
$
{\mathcal{V}}_{\lambda, \delta}
=G \times_P V_{\lambda,\delta}
$
 is a homogeneous vector bundle over the real flag variety $G/P$.  
The dualizing bundle 
\index{A}{Vlndast@${\mathcal{V}}_{\lambda,\delta}^{\ast}$, dualizing bundle|textbf}
\index{A}{Vlnd@${\mathcal{V}}_{\lambda,\delta}$, homogeneous vector bundle over $G/P$|textbf}
$
{\mathcal{V}}_{\lambda, \delta}^{\ast}
$
 of ${\mathcal{V}}_{\lambda, \delta}$, 
 is given by a $G$-homogeneous vector bundle over $G/P$
 associated to the representation
 of $P/N_+ \simeq MA \simeq O(n) \times {\mathbb{Z}}/2{\mathbb{Z}} \times {\mathbb{R}}$:
\[
 V_{\lambda, \delta}^{\ast}
:=(V_{\lambda,\delta})^{\vee} \otimes {\mathbb{C}}_{2\rho}
\simeq V^{\vee} \boxtimes {\delta} \boxtimes{\mathbb{C}}_{n-\lambda}, 
\]
where 
\index{A}{VWV@$V^{\vee}$, contragredient representation of $V$|textbf}
$V^{\vee}$ denotes the contragredient representation
 of $(\sigma,V)$.  
Then the regular representation
 of $G$ on the space
\index{A}{distribution@${\mathcal{D}}'$, distribution}
$
   {\mathcal{D}}'(G/P, {\mathcal{V}}_{\lambda, \delta}^{\ast})
$
 of ${\mathcal{V}}_{\lambda, \delta}^{\ast}$-valued distribution sections
 is the dual of the representation 
 $I_{\delta}(V,\lambda)$ of $G$
 on $C^{\infty}(G/P, {\mathcal{V}}_{\lambda, \delta})$
 as we discussed in Example \ref{ex:psdual}.  



In this special setting, 
 Fact \ref{fact:SBOdistr} amounts to the following.  
\begin{fact}
\label{fact:kernel}
There is a natural bijective map:
\begin{equation}
\label{eqn:ker}
\operatorname{Hom}_{G'}
  (
   I_{\delta}(V,\lambda)|_{G'}, 
   J_{\varepsilon}(W,\nu)
  )
\overset \sim \to 
   ({\mathcal{D}}'
    (G/P, {\mathcal{V}}_{\lambda, \delta}^{\ast})
    \otimes
    W_{\nu,\varepsilon}
    )
    ^{\Delta (P')}, 
\quad
   T \mapsto K_T.  
\end{equation}
\end{fact}



\vskip 0.8pc
In \cite[Def.~3.3]{sbon}, 
 we defined regular symmetry breaking operators
 in the general setting.  
In our special setting,
 there is only one open $P'$-orbit 
 in the real flag manifold $G/P$,
 and thus the definition is reduced to the following.  
\begin{definition}
[regular symmetry breaking operator]
\label{def:regSBO}
A symmetry breaking operator
 $T \colon I_{\delta}(V,\lambda) \to J_{\varepsilon}(W,\nu)$
 is 
\index{B}{regularsymmetrybreakingoperator@regular symmetry breaking operator|textbf}
{\it{regular}} 
 if the support of the distribution kernel $K_T$ is $G/P$.  
\end{definition}

%%%%%%%%%%%%%%%%%%%%%%%%%%%%%%%%%%%%%%%%%%%%%%%%%%%%%%%%%%%%%%%%
\subsubsection{Distribution sections for dualizing bundle
 ${\mathcal{V}}_{\lambda,\delta}^{\ast}$ over $G/P$}
\label{subsec:GPsec}
%%%%%%%%%%%%%%%%%%%%%%%%%%%%%%%%%%%%%%%%%%%%%%%%%%%%%%%%%%%%%%%%
This section provides a concrete description
 of the right-hand side of \eqref{eqn:ker}
 in the coordinates on the open 
\index{B}{Bruhatcell@Bruhat cell}
Bruhat cell.  




We begin with a description of 
 the $G$- and ${\mathfrak{g}}$-action 
 on ${\mathcal{D}}'
    (G/P, {\mathcal{V}}_{\lambda, \delta}^{\ast})$
 in the coordinates.  
We identify ${\mathcal{D}}'(G/P, {\mathcal{V}}_{\lambda, \delta}^{\ast})$
 with a subspace of $V^{\vee}$-valued distribution on $G$
 via the following map: 
\[
   {\mathcal{D}}'(G/P, {\mathcal{V}}_{\lambda, \delta}^{\ast})
   \simeq 
  ({\mathcal{D}}'(G) \otimes V_{\lambda, \delta}^{\ast})^{\Delta(P)}
   \subset 
   {\mathcal{D}}'(G) \otimes V^{\vee}.  
\]
We recall
 that the Bruhat decomposition of $G$ is given by 
 $G=N_+ w P \cup P$ 
 where 
\index{A}{winversion@$w$, inversion element}
$w= \operatorname{diag}(1,\cdots,1,-1) \in G$, 
 see \eqref{eqn:w}.  
Since the real flag manifold $G/P$ is covered by the two open subsets
 $N_+ w P/P$ and $N_- P/P$, 
 distribution sections on $G/P$ are determined uniquely
 by the restriction  
 to these two open sets:
\begin{equation}
\label{eqn:Dpm}
{\mathcal{D}}'(G/P, {\mathcal{V}}_{\lambda, \delta}^{\ast})
\hookrightarrow
{\mathcal{D}}'(N_+ w P/P, {\mathcal{V}}_{\lambda, \delta}^{\ast}|_{N_+ w P/P})
\oplus
{\mathcal{D}}'(N_- P/P,  {\mathcal{V}}_{\lambda, \delta}^{\ast}|_{N_- P/P}).  
\end{equation}
By a little abuse of notation,
 we use the letters $n_+$ and $n_-$
 to denote the induced diffeomorphisms
$
   {\mathbb{R}}^n \overset \sim \to 
   N_+ w P/P
$
 and 
$
   {\mathbb{R}}^n \overset \sim \to 
   N_- P/P
$, 
respectively.  
Via the following trivialization of the two restricted bundles:
\begin{alignat*}{9}
&  {\mathbb{R}}^n  \times V^{\vee}
\;\;
&& \overset \sim \to 
\;\;
&& {\mathcal{V}}_{\lambda,\delta}^{\ast}|_{N_+ w P/P}
\;\;
&& \subset 
\;\;
&& {\mathcal{V}}_{\lambda,\delta}^{\ast}
\;\;
&& \supset
\;\;
&& {\mathcal{V}}_{\lambda,\delta}^{\ast}|_{N_- P/P}
\;\;
&& \overset \sim \leftarrow
\;\;
&& {\mathbb{R}}^n \times V^{\vee}
\\
& \downarrow
&&
&& \downarrow
&& 
&& \downarrow
&&
&& \downarrow
&&
&&\downarrow
\\
&  {\mathbb{R}}^n  
\;\;
&& \underset{n_+} {\overset \sim \to} 
\;\;
&& N_+ w P / P
\;\;
&& \subset 
\;\;
&& G/P
\;\;
&& \supset
\;\;
&& N_- P / P
\;\;
&& \underset {n_-}{\overset \sim \leftarrow}
\;\;
&& {\mathbb{R}}^n, 
\end{alignat*}
the injection \eqref{eqn:Dpm} is restated as the following map:

\begin{equation}
\label{eqn:fFpm}
   {\mathcal{D}}'(G/P, {\mathcal{V}}_{\lambda, \delta}^{\ast})
   \hookrightarrow
   \left({\mathcal{D}}'({\mathbb{R}}^n) \otimes V^{\vee}\right)
   \oplus
   \left({\mathcal{D}}'({\mathbb{R}}^n) \otimes V^{\vee}\right), 
\qquad
   f \mapsto (F_{\infty}, F)
\end{equation}
where 
\index{A}{n1@$n_+\colon {\mathbb{R}}^n \to N_+$}
\index{A}{n2@$n_-\colon {\mathbb{R}}^n \to N_-$}
\begin{equation*}
F_{\infty}(a):= f(n_+(a) w), 
\qquad
F(b):= f(n_-(b)).  
\end{equation*}



\begin{lemma}
\label{lem:152341}
Let 
\index{A}{1psin@$\psi_n$}
$
\psi_n \colon {\mathbb{R}}^n\setminus \{0\} \to O(n)
$ be the map
 taking the reflection defined in \eqref{eqn:psim}.  

\begin{enumerate}
\item[{\rm{(1)}}]
The image of the injective map \eqref{eqn:fFpm}
 is characterized by the following identity
 in ${\mathcal{D}}'({\mathbb{R}}^n\setminus \{0\}) \otimes V^{\vee}$:
\begin{equation}
\label{eqn:F12}
   F(b)
   =
   \delta \sigma^{\vee}(\psi_n(b)^{-1})
   |b|^{2\lambda-2n}
   F_{\infty}(-\frac{b}{|b|^2})
\qquad
\text{on }
{\mathbb{R}}^n \setminus \{0\}.  
\end{equation}

\item[{\rm{(2)}}]
{\rm{(first projection)}}
$f \in {\mathcal{D}}'(G/P, {\mathcal{V}}_{\lambda, \delta}^{\ast})$
 is supported at the singleton 
\index{A}{p1@$p_+={}^t(1,0,\cdots,0,1)$}
$
\{[p_+]\} = \{ e P/P\}
$
 if and only if $F_{\infty}=0$.  
\item[{\rm{(3)}}]
{\rm{(second projection)}}
The second projection $f \mapsto F$ is injective.  
\end{enumerate}
\end{lemma}

\begin{proof}
(1) \enspace
The image of the map \eqref{eqn:Dpm} is characterized
 by the compatibility condition
 on the intersection $(N_+ w P \cap N_- P)/P$, 
 namely, 
 the pair $(F_{\infty}, F)$
 in \eqref{eqn:fFpm} should satisfy:
\[
  F (b)=\sigma_{\lambda,\delta}^{\ast}(p)^{-1} F_{\infty}(a)
\]
for all $(a,b,p) \in {\mathbb{R}}^n \times {\mathbb{R}}^n \times P$
 such that $n_+(a) w p =n_-(b)$.  
In this case, 
 $b \ne 0$
 because $N_+ w P \not \ni e$.  
By Lemma \ref{lem:0.1}, 
we have
\[
a = - \frac{b}{|b|^2}, 
\quad
p= \begin{pmatrix} -1 & & \\
                              & \psi_n(b) & \\
                              &  &  -1 \end{pmatrix} 
        e^{tH}, 
\]
where $e^t = |b|^2$.  
Then 
\begin{align*}
F(b) =& f(n_-(b))
\\
       =& \sigma_{\lambda, \delta}^{\ast}(p^{-1})
        f(n_+(a) w)
\\
       =& \delta|b|^{2\lambda-2n} 
          \sigma^{\vee} (\psi_n(b)) F_{\infty}(a).  
\end{align*}
\begin{enumerate}
\item[(2)]
Clear from $G \setminus N_+ w P=P$.  
\item[(3)]
Since $P' N_-P=G$
\cite[Cor.~5.5]{sbon}, 
the third statement follows from \cite[Thm.~3.16]{sbon}.  
\end{enumerate}
\end{proof}



The regular representation of $G$
 on ${\mathcal{D}}'(G/P, {\mathcal{V}}_{\lambda, \delta}^{\ast})$ induces
 an action
 on the pairs $(F_{\infty}, F)$
 of $V^{\vee}$-valued distributions
 through Lemma \ref{lem:152341} (1).  
We need an explicit formula
 of the action of the parabolic subgroup $P=MAN_+$
 or its Lie algebra
$
  {\mathfrak{p}}
 ={\mathfrak{m}}
 +{\mathfrak{a}}
 +{\mathfrak{n}}_+
$, 
 which is given in the following two elementary lemmas.  



We begin with the first projection $f \mapsto F_{\infty}$ in \eqref{eqn:fFpm}.  
Since the action of $P$ on $G/P$ leaves the open subset 
 $N_+ w P /P = P w P /P$ invariant, 
 we can define the geometric action
 of the group $P$ on ${\mathcal{D}}'(N_+ w P /P, {\mathcal{V}}_{\lambda,\delta}^{\ast})$
 as follows.  
We recall 
$
M=O(n) \times \{1, m_-\}
$
 (see \eqref{eqn:m-}).  
We collect
 some basic formul{\ae} for the coordinates 
 $n_{\varepsilon}\colon {\mathbb{R}}^n \overset \sim \to N_{\varepsilon}$:
 for $\varepsilon = \pm$
 (by abuse of notation, 
 we also write as $\varepsilon = \pm 1$), 
\index{A}{H@$H$|textbf}
\begin{align}
{n_{\varepsilon}} (Bb)
   =& \begin{pmatrix} 1 \\ & B \\ && 1 \end{pmatrix}
\,\,
n_{\varepsilon} (b)
      \begin{pmatrix} 1 \\ & B^{-1} \\ && 1 \end{pmatrix}
\quad
\text{for $B \in O(n)$, }
\label{eqn:nAb}
\\
 {n_{\varepsilon}} (-b)
   =& m_- {n_{\varepsilon}} (b) m_-^{-1},
\label{eqn:minvn}
\\
 {n_{\varepsilon}} (e^{{\varepsilon}t} b)
   =& e^{tH} {n_{\varepsilon}} (b) e^{-tH}.  
\label{eqn:invm}
\end{align}
\begin{lemma}
\label{lem:152343}
We let $P=M A N_+$ act on 
$
   {\mathcal{D}}'({\mathbb{R}}^n) \otimes V^{\vee}
$
by
\index{A}{m2@$m_-={\operatorname{diag}}(-1,1,\cdots,1,-1)$}
\begin{alignat}{3}
&\left(\pi \begin{pmatrix} 1 & & \\ & B & \\ & & 1 \end{pmatrix} F_{\infty}\right)(a)
&&=\sigma^{\vee}(B) F_{\infty}(B^{-1}a)
\quad
&&\text{ for $B \in O(n)$, }
\label{eqn:inftyM}
\\
&(\pi (m_-) F_{\infty})(a)
&&=\delta F_{\infty}(-a), 
&&
\label{eqn:inftym}
\\
&(\pi (e^{tH}) F_{\infty})(a)
&&= e^{(\lambda-n)t} F_{\infty}(e^{-t}a)
\quad
&&\text{ for all $t \in {\mathbb{R}}$, }
\label{eqn:inftyA}
\\
&(\pi (n_+(c)) F_{\infty})(a)
&&= F_{\infty}(a-c)
\quad
&&\text{ for all $c \in {\mathbb{R}}^{n}$. }
\label{eqn:inftyN}
\end{alignat}
Then the first projection $f \mapsto F_{\infty}$ in \eqref{eqn:fFpm}
 is a $P$-homomorphism.  
\end{lemma}
\begin{proof}
We give a proof for \eqref{eqn:inftyA} on the action
 of the split abelian group $A$.  
Let $t \in {\mathbb{R}}$. 
By \eqref{eqn:invm}
 and 
\index{A}{winversion@$w$, inversion element}
$
e^{-tH} w =w e^{tH}
$, 
 we have 
\[
f(e^{-tH} n_+(a) w)
=f(n_+(e^{-t}a)e^{-tH}w)
=e^{(\lambda-n)t} f(n_+(e^{-t}a)w)
=e^{(\lambda-n)t} F_{\infty}(e^{-t} a), 
\]
whence we get the desired formula.  
The proof for the actions of $M$ and $N_+$ is similar.  
\end{proof}



Next,
 we consider the second projection $f \mapsto F$
 in \eqref{eqn:fFpm}.  
In this case,
 the group $N_+$ does not preserve the open subset
 $N_-P/P$ in $G/P$, 
 and therefore we shall use the action 
 of the Lie algebra ${\mathfrak {n}}_+$ instead
 (see \eqref{eqn:On} below).  
We denote by 
\index{A}{Euler@$E$, Euler homogeneity operator\quad|textbf}
$
E
$ 
 the Euler homogeneity operator
 $\sum_{\ell=1}^n x_\ell \frac{\partial}{\partial x_\ell}$.  
\begin{lemma}
\label{lem:20161104}
We let the group $MA$ and the Lie algebra ${\mathfrak{n}}_+$
 act on
 ${\mathcal{D}}'({\mathbb{R}}^n) \otimes V^{\vee}$
 by 
\index{A}{Nj1@$N_j^+$}
\begin{alignat}{2}
& \left(\pi \begin{pmatrix} 1 & & \\ & B & \\ & & 1 \end{pmatrix}  F\right)(b)
&&=\sigma^{\vee}(B) F(B^{-1} b)
\quad
\text{for $B \in O(n)$}, 
\label{eqn:0M}
\\
& (\pi(m_-) F)(b)
&&=
\delta F(-b), 
\label{eqn:0m}
\\
& (\pi (e^{tH})F)(b)
&&=
e^{(n-\lambda) t}F(e^t b)
\quad
\text{for all $t \in {\mathbb{R}}$}, 
\label{eqn:0A}
\\
& d\pi (N_j^+)F(b)
&&=
\left((\lambda-n)b_j - b_j E 
 + \frac 1 2 |b|^2 \frac{\partial}{\partial b_j}\right) F
\quad
\text{for $1 \le j \le n$}.  
\label{eqn:On}
\end{alignat}
Here $b=(b_1, \cdots, b_n)$.  
Then the second projection $f \mapsto F$
 in \eqref{eqn:fFpm}
 is an $(M A, {\mathfrak {n}}_+)$-homomorphism.  
\end{lemma}

\begin{proof}
See \cite[Prop.~6.4]{sbon}
for \eqref{eqn:On}.  
The other formul\ae\ are easy, 
 and we omit the proof.  
\end{proof}



%%%%%%%%%%%%%%%%%%%%%%%%%%%%%%%%%%%%%%%%%%%%
\subsubsection{Pair of distribution kernels
 for symmetry breaking operators}
\label{subsec:Tpair}
%%%%%%%%%%%%%%%%%%%%%%%%%%%%%%%%%%%%%%%%%%%%

We extend Lemma \ref{lem:152341}
 to give a local expression
 of the distribution kernels
 of symmetry breaking operators
 via the isomorphism \eqref{eqn:ker}.  
Suppose $(\tau,W) \in \widehat{O(n-1)}$, 
 $\nu \in {\mathbb{C}}$, 
and $\varepsilon \in \{ \pm \}$.  
We define 
\begin{equation}
\label{eqn:DVWinfty}
   ({\mathcal{D}}'({\mathbb{R}}^n)
    \otimes 
    \operatorname{Hom}_{{\mathbb{C}}}(V,W))^{\Delta(P')}
   \equiv
({\mathcal{D}}'({\mathbb{R}}^n)
    \otimes 
    \operatorname{Hom}_{{\mathbb{C}}}(V_{\lambda, \delta},W_{\nu,\varepsilon})
   )^{\Delta(P')}
\end{equation}
 to be the space of $\operatorname{Hom}_{{\mathbb{C}}}(V,W)$-valued
 distributions ${\mathcal{T}}_{\infty}$ on ${\mathbb{R}}^n$
satisfying the following four conditions:
\begin{alignat}{2}
\label{eqn:152345a}
& \tau(B) \circ {\mathcal{T}}_{\infty}(B^{-1} y,y_n) \circ \sigma^{-1}(B)
  ={\mathcal{T}}_{\infty}(y,y_n)
\qquad
&&\text{for all } B \in O(n-1), 
\\
\label{eqn:152345b}
& {\mathcal{T}}_{\infty}(-y,-y_n)= \delta \varepsilon {\mathcal{T}}_{\infty}(y,y_n), 
&&
\\
\label{eqn:152345c}
& {\mathcal{T}}_{\infty}(e^t y,e^t y_n)= e^{(\lambda+\nu-n)t}{\mathcal{T}}_{\infty}(y,y_n)
\quad
&&\text{for all }t \in {\mathbb{R}}, 
\\
\label{eqn:152345d}
& {\mathcal{T}}_{\infty}(y-z,y_n)={\mathcal{T}}_{\infty}(y,y_n)
\quad
&&\text{for all }z \in {\mathbb{R}}^{n-1}.  
\end{alignat}
For the open Bruhat cell $N_- P \subset G$, 
 we consider the following.  
\begin{definition}
\label{def:solVW}
We define 
\index{A}{SolRVW@${\mathcal{S}}ol
({\mathbb{R}}^n;V_{\lambda, \delta}, W_{\nu, \varepsilon})$|textbf}
${\mathcal{S}}ol
({\mathbb{R}}^n;V_{\lambda, \delta}, W_{\nu, \varepsilon})
\subset
{\mathcal{D}}'({\mathbb{R}}^n)
    \otimes 
    \operatorname{Hom}_{{\mathbb{C}}}(V,W)
$
to be the space of $\operatorname{Hom}_{\mathbb{C}}(V,W)$-valued
 distributions 
${\mathcal{T}}$ on ${\mathbb{R}}^n$
 satisfying 
the following invariance 
under the action of the Lie algebras
 ${\mathfrak{a}}$, ${\mathfrak{n}}_+'$, 
 and the group $M' \simeq O(n-1) \times O(1)$:
\begin{alignat}{2}
&(E-(\lambda-\nu-n)){\mathcal{T}}=0, 
&&
\label{eqn:Fainv}
\\
&\left((\lambda-n)x_j - x_j E + \frac 1 2 (|x|^2 + x_n^2)\frac{\partial}{\partial x_j}\right){\mathcal{T}}=0
\quad
&&(1 \le j \le n-1), 
\label{eqn:Fninv}
\\
&\tau(m) \circ {\mathcal{T}}(m^{-1}b)\circ \sigma(m^{-1})={\mathcal{T}}(b)
\quad
&&\text{for all } m \in O(n-1), 
\label{eqn:FMinv}
\\
&{\mathcal{T}}(-b)= \delta \varepsilon {\mathcal{T}}(b).  
&&
\label{eqn:Fparity}
\end{alignat}
\end{definition}


Applying Lemma \ref{lem:152341}
 to the right-hand side of \eqref{eqn:ker}, 
 we have the following:
\begin{proposition}
\label{prop:Tpair}
Let $(\sigma, V) \in \widehat M$, 
$(\tau,W) \in \widehat {M'}$, 
 $\delta$, $\varepsilon \in \{ \pm \}$, 
 and $\lambda, \nu \in {\mathbb{C}}$.  
\begin{enumerate}
\item[{\rm{(1)}}]
There is a one-to-one correspondence between a symmetry breaking operator
\[
  {\mathbb{T}} 
  \in 
  \operatorname{Hom}_{G'}(I_{\delta}(V,\lambda)|_{G'},J_{\varepsilon}(W,\nu))
\]
 and a pair $({\mathcal{T}}_{\infty}, {\mathcal{T}})$
 of $\operatorname{Hom}_{{\mathbb{C}}}(V,W)$-valued distributions
  on ${\mathbb{R}}^n$
 subject to the following three conditions:
\begin{align}
{\mathcal{T}}_{\infty}
\in &
({\mathcal{D}}'({\mathbb{R}}^n) \otimes \operatorname{Hom}_{{\mathbb{C}}}(V_{\lambda,\delta},W_{\nu,\varepsilon}))^{\Delta(P')}, 
\label{eqn:Tinfty}
\\
{\mathcal{T}}
\in &
{\mathcal{S}}ol
({\mathbb{R}}^n;V_{\lambda, \delta}, W_{\nu, \varepsilon}), 
\label{eqn:Tzero}
\\
{\mathcal{T}}(b)
= &
\delta Q(b)^{\lambda-n}
{\mathcal{T}}_{\infty}
\left(-\frac{b}{|b|^2}\right)
\circ
\sigma(\psi_n(b))
\quad
\text{on }
{\mathbb{R}}^n \setminus \{0\}.  
\label{eqn:Tpatching}
\end{align}



\item[{\rm{(2)}}]
${\mathcal{T}}$ determines ${\mathbb{T}}$ uniquely.  
\item[{\rm{(3)}}]
Suppose that ${\mathbb{T}} \leftrightarrow 
({\mathcal{T}}_{\infty}, {\mathcal{T}})
\index{A}{TinftyT@$({\mathcal{T}}_{\infty}, {\mathcal{T}})$}
$
 is the correspondence in (1).  
Then the following three conditions are equivalent:
\begin{enumerate}
\item[{\rm{(i)}}]
${\mathcal{T}}_{\infty}=0$.  
\item[{\rm{(ii)}}]
$\operatorname{Supp} {\mathcal{T}} \subset \{0\}$.  
\item[{\rm{(iii)}}]
${\mathbb{T}}$ is a differential operator
 (see Definition \ref{def:diff}).  
\end{enumerate}
\end{enumerate}
\end{proposition}


\begin{proof}
The first statement follows from Fact \ref{fact:kernel}, 
 Lemmas \ref{lem:152341} (1), 
 \ref{lem:152343}
 and \ref{lem:20161104}.  
The second statement is immediate from 
 Lemma \ref{lem:152341} (3).  
The third one is proved in \cite{KP1}, 
 see Section \ref{subsec:diff}
 for more details about differential operators
 between two manifolds.  
\end{proof}



\begin{remark}
\label{rem:7.5}
The advantage of using ${\mathcal{T}}$ is
 that the second projection
\[
 \operatorname{Hom}_{G'}
 (I_{\delta}(V,\lambda)|_{G'}, J_{\varepsilon}(W, \nu))
\overset \sim \to 
{\mathcal{S}}ol
({\mathbb{R}}^n;\sigma_{\lambda, \delta}, \tau_{\nu, \varepsilon}),
\quad
 {\mathbb{T}} \mapsto {\mathcal{T}}
\]
 is bijective, 
 and therefore, 
it is sufficient
 to use ${\mathcal{T}}$ 
 in order to describe a symmetry breaking operator ${\mathbb{T}}$.  
This was the approach 
 that we took in \cite{sbon}.  
In this monograph,
 we shall use both ${\mathcal{T}}_{\infty}$ and ${\mathcal{T}}$.  
The advantage of using ${\mathcal{T}}_{\infty}$ is
 that the group $P'$ leaves $N_+ w P/P$ invariant,
 and consequently,
 we can easily determine ${\mathcal{T}}_{\infty}$
 (see Proposition \ref{prop:20150828-1231} below), 
 although the first projection
\[
 \operatorname{Hom}_{G'}
 (I_{\delta}(V,\lambda)|_{G'}, J_{\varepsilon}(W, \nu))
\to 
 ({\mathcal{D}}'({\mathbb{R}}^n) \otimes \operatorname{Hom}_{{\mathbb{C}}}(V,W))^{\Delta(P')}, 
\quad
 {\mathbb{T}} \mapsto {\mathcal{T}}_{\infty}
\]
 is neither injective nor surjective.  
We shall return to this point
 in Section \ref{subsec:holoAVW}.  
\end{remark}



\subsection{Distribution kernels near infinity}
\label{subsec:infty}
Let $({\mathcal{T}}_{\infty}, {\mathcal{T}})$ be
 as in Proposition \ref{prop:Tpair}.  
This section determines ${\mathcal{T}}_{\infty}$ 
 up to scalar multiplication.  
The main result is Proposition \ref{prop:20150828-1231}, 
 which also determines uniquely the restriction of ${\mathcal{T}}$
 to ${\mathbb{R}}^n \setminus \{0\}$
 up to scalar multiplication.  

\begin{example}
\label{ex:152341}
For $\sigma = {\bf{1}}$, 
$\tau={\bf{1}}$, $\delta=+1$, and 
\[
   {\mathcal{T}}_{\infty}(y,y_n)=|y_n|^{\lambda+\nu-n}, 
\]
 we have from \eqref{eqn:Tpatching}
\[
   {\mathcal{T}}(x,x_n) = (|x|^2+x_n^2)^{-\nu} |x_n|^{\lambda+\nu-n}.  
\]
\end{example}



We begin with the following classical result
 on homogeneous distributions
 of one variable:
\begin{lemma}
\label{lem:Riesz}
\begin{enumerate}
\item[{\rm{(1)}}]
Both $\frac{1}{\Gamma(\frac \mu 2)}|t|^{\mu-1}$
 and $\frac{1}{\Gamma(\frac {\mu+1} 2)}|t|^{\mu-1} \operatorname{sgn} t$
 are nonzero distributions
 on ${\mathbb{R}}$
 that depend holomorphically
 on $\mu$ in the entire complex plane ${\mathbb{C}}$.  
\item[{\rm{(2)}}]
Suppose $k \in {\mathbb{N}}$.  
Then 
\begin{alignat*}{2}
\frac{|t|^{\mu-1}}{\Gamma(\frac \mu 2)}
=&
\frac{(-1)^k}{2^k (2k-1)!!}
\delta^{(2k)}(t)
\qquad
&&\text{if }
\mu=-2k, 
\\
\frac{|t|^{\mu-1}\operatorname{sgn} t}{\Gamma(\frac {\mu+1} 2)}
=&
\frac{(-1)^k(k-1)!}{(2k-1)!}
\delta^{(2k-1)}(t)
\qquad
&&\text{if }
\mu=-2k-1.  
\end{alignat*}
\item[{\rm{(3)}}]
Suppose $\mu \in {\mathbb{C}}$ and $\gamma = \pm 1$.  
Then any distribution $g(t)$ on ${\mathbb{R}}$
 satisfying the homogeneity condition 
\[
  \text{$g(at)=a^{\mu-1}g (t)$
 for all 
 $a >0$, 
 and 
 $g(-t)= \gamma g(t)$
}
\]
 is a scalar multiple
 of $\frac{1}{\Gamma(\frac \mu 2)}|t|^{\mu-1}$ $(\gamma =1)$,
 or of $\frac{1}{\Gamma(\frac {\mu+1} 2)}|t|^{\mu-1} \operatorname{sgn} t$
 $(\gamma =-1)$.  
\end{enumerate}
\end{lemma}

For $(\sigma, V) \in \widehat{O(n)}$ and $(\tau, W) \in \widehat{O(n-1)}$,
 we recall that
$ [V:W] $
is the dimension of
$ 
{\operatorname{Hom}}_{O(n-1)}(V|_{O(n-1)},W)
$.  
Suppose $[V:W] \ne 0$, 
 or equivalently, 
 $[V:W] =1$.  
We fix a generator 
\[
 \pr V W \in \operatorname{Hom}_{O(n-1)}(V|_{O(n-1)},W)
\]
 which is unique
 up to nonzero scalar multiplication
 by Schur's lemma.  
In light of the $\Gamma$-factors in Lemma \ref{lem:Riesz}, 
we introduce $\operatorname{Hom}_{{\mathbb{C}}}(V,W)$-valued distributions 
\index{A}{Actt0inf@$(\Attcal \lambda \nu {\pm} {V,W})_{\infty}$|textbf}
$
(\Attcal \lambda \nu \pm {V,W})_{\infty}
$
 on ${\mathbb{R}}^n$ 
 that depend holomorphically on $(\lambda,\nu) \in {\mathbb{C}}^2$
 by 
\begin{align}
(\Attcal \lambda \nu + {V,W})_{\infty}(x,x_n)
:=&
\frac{1}{\Gamma(\frac{\lambda+ \nu-n+1}{2})}|x_n|^{\lambda+\nu-n} \pr V W, 
\label{eqn:AVW+u}
\\
(\Attcal \lambda \nu - {V,W})_{\infty}(x,x_n)
:=&
\frac{1}{\Gamma(\frac{\lambda+ \nu-n+2}{2})}|x_n|^{\lambda+\nu-n}
\operatorname{sgn} x_n \pr V W.  
\label{eqn:AVW-u}
\end{align}
We regard $\pr V W =0$
 if $[V:W] = 0$.  

\begin{remark}
The notation $(\Attcal \lambda \nu \gamma {V,W})_{\infty}$
 with double tildes
 is used here
 because it will be compatible
 with the 
\index{B}{renormalized regular symmetry breaking operator@regular symmetry breaking operator, renormalized---}
{\it{renormalization}}
 $\Attbb \lambda \nu \gamma {V,W}$
 of the normalized symmetry breaking operator
 $\Atbb \lambda \nu \gamma {V,W}$
 which we will introduce 
 in the next sections.  
\end{remark}



Let $\gamma=\delta \varepsilon$.  
If there exists 
$
   {\mathcal {T}}_{\gamma} 
   \in 
   {\mathcal{S}}ol({\mathbb{R}}^n;V_{\lambda,\delta}, W_{\nu,\varepsilon})
$
 such that the pair 
 $((\Attcal \lambda \nu \varepsilon {V,W})_{\infty}, {\mathcal{T}}_{\gamma})$
 satisfies the compatibility condition \eqref{eqn:Tpatching},  
 then the restriction ${\mathcal{T}}_{\gamma}|_{{\mathbb{R}}^n \setminus \{0\}}$
 must be of the form 
\index{A}{Actt0@$\Attcal \lambda \nu {\pm} {V,W}$|textbf}
$
(\Attcal \lambda \nu \gamma {V,W})' \in 
 {\mathcal{D}}'({\mathbb{R}}^n \setminus \{0\}) \otimes {\operatorname{Hom}}_{\mathbb{C}}(V,W)
$
where we set 
\begin{align}
(\Attcal {\lambda}{\nu}{+}{V,W})'
 :=&
\frac{1}{\Gamma(\frac{\lambda + \nu-n+1}{2})}
         (|x|^2+ x_n^2)^{-\nu}
|x_n|^{\lambda+\nu-n}
\Rij V W (x,x_n), 
\label{eqn:AVW+}
\\
(\Attcal {\lambda}{\nu}{-}{V,W})'
 :=&
\frac{1}{\Gamma(\frac{\lambda+\nu-n+2}{2})}
         (|x|^2 +x_n^2)^{-\nu}
|x_n|^{\lambda+\nu-n}
 {\operatorname{sgn}} x_n
\Rij VW (x,x_n), 
\label{eqn:AVW-}
\end{align}
with $\Rij V W = \pr V W \circ \sigma \circ \psi_n$
 (see \eqref{eqn:RVW}).  
We have used the notation
$
   (\Attcal {\lambda}{\nu}{\gamma}{V,W})'
$
 instead of 
$
   \Attcal {\lambda}{\nu}{\gamma}{V,W}
$
 because it is defined 
 only on ${\mathbb{R}}^n \setminus \{0\}$
 and may not extend to ${\mathbb{R}}^n$
 (see Proposition \ref{prop:20170213} below).  


Then we have:
\begin{proposition}
\label{prop:20150828-1231}
{\rm{(1)}}\enspace
For any $(\sigma,V) \in \widehat{O(n)}$, 
 $(\tau,W) \in \widehat{O(n-1)}$,
 $\delta, \varepsilon \in \{ \pm \}$, 
 and $\lambda, \nu \in {\mathbb{C}}$, 
 we have
\[
({\mathcal{D}}'({\mathbb{R}}^n)
   \otimes 
   \operatorname{Hom}_{\mathbb{C}} (V_{\lambda,\delta}, W_{\nu, \varepsilon}))
^{\Delta(P')}
=
 {\mathbb{C}}(\Attcal \lambda \nu {\delta\varepsilon}{V,W})_{\infty}.  
\]
\newline
{\rm{(2)}}\enspace
If $[V:W] \ne 0$
 then 
 $(\Attcal \lambda \nu \pm {V,W})' \ne 0$
 for all $\lambda, \nu \in {\mathbb{C}}$.  
\newline
{\rm{(3)}}\enspace
If ${\mathcal{T}} \in {\mathcal{S}}ol({\mathbb{R}}^n;V_{\lambda,\delta}, W_{\nu, \varepsilon})$, 
 then ${\mathcal{T}}|_{{\mathbb{R}}^n \setminus \{0\}}$ is a scalar multiple
 of $(\Attcal \lambda \nu {\delta \varepsilon}{V,W})'$.  
\end{proposition}

\begin{proof}
Suppose 
$
   F 
   \in 
   ({\mathcal{D}}'({\mathbb{R}}^n) \otimes {\operatorname{Hom}}_{\mathbb{C}}(V_{\lambda,\delta}, W_{\nu,\varepsilon}))^{\Delta(P')}$.  
\newline
(1)\enspace
Let $p_n \colon {\mathbb{R}}^n \to {\mathbb{R}}$ be
 the $n$-th projection, 
 and 
$
   p_n^{\ast} \colon {\mathcal{D}}'({\mathbb{R}}) \to {\mathcal{D}}'({\mathbb{R}}^n)
$
 the pull-back of distributions.  
By the $N_+'$-invariance
 \eqref{eqn:152345d}, 
 $F$ depends only on the last coordinate, 
 namely, 
 $F$ is of the form $p_n^{\ast}f$
 for some $f \in {\mathcal{D}}'({\mathbb{R}}) \otimes \operatorname{Hom}_{\mathbb{C}}(V,W)$.  
In turn,
 the $O(n-1)$-invariance \eqref{eqn:152345a} implies
\[
   f \in {\mathcal{D}}'({\mathbb{R}}) \otimes \operatorname{Hom}_{O(n-1)}(V|_{O(n-1)},W).  
\]
In particular, 
 $F=0$
 if $[V:W]=0$.  

{}From now,
 we assume $[V:W] \ne 0$.  
Then $f$ is of the form $h(y_n) \pr V W$
 for some $h(t) \in {\mathcal{D}}'({\mathbb{R}})$.  
By \eqref{eqn:152345b} and \eqref{eqn:152345c}, 
 $h$ is a homogeneous distribution
 of degree $\lambda+\nu-n$
 and of parity $\delta \varepsilon$.  
Then $h(t)$ is determined by Lemma \ref{lem:Riesz}, 
 and we get the desired result.  
\newline
(2)\enspace
The assertion follows from the nonvanishing statement
 for the distribution
 of one-variable
 (see Lemma \ref{lem:Riesz} (1)).  
\newline
(3)\enspace
The third statement follows from 
 the first assertion and Proposition \ref{prop:Tpair}.  
\end{proof}


%%%%%%%%%%%%%%%%%%%%%%%%%%%%%%%%%%%%%%%%%%%%%%%%%%
\subsection{Vanishing condition of differential symmetry breaking operators:
Proof of Theorem \ref{thm:vanDiff} (1)}
\label{subsec:vanDiff}
%%%%%%%%%%%%%%%%%%%%%%%%%%%%%%%%%%%%%%%%%%%%%%%%%%

In this section, we prove a necessary condition 
 for the existence of nonzero differential symmetry breaking operators
 as stated in Theorem \ref{thm:vanDiff} (1):
\begin{theorem}
[vanishing of differential symmetry breaking operators]
\label{thm:vanishDiff}
~~~\newline
Suppose that $V$ and $W$ are finite-dimensional representations
 of $O(n)$ and $O(n-1)$, 
respectively,
 $\delta$, $\varepsilon \in \{\pm\}$, 
 and $(\lambda,\nu) \in {\mathbb{C}}^2$.  
If $(\lambda, \nu, \delta, \varepsilon)$
 satisfies the 
\index{B}{genericparametercondition@generic parameter condition}
generic parameter condition
 \eqref{eqn:nlgen}, 
 namely, 
 $\nu - \lambda \not \in 2{\mathbb{N}}$
 for $\delta \varepsilon=+$, 
 or $\nu - \lambda \not \in 2{\mathbb{N}}+1$
 for $\delta \varepsilon=-$, 
then 
\[
  {\operatorname{Diff}}_{G'}
  (I_{\delta}(V,\lambda)|_{G'},J_{\varepsilon}(W,\nu))=\{0\}.  
\]
\end{theorem}

\begin{remark}
In the above theorem, 
 we do not impose any assumption on $V$ and $W$.  
In Chapter \ref{sec:DSVO}, 
 we give a converse implication
 under the assumption $[V:W] \ne 0$, 
 see Theorem \ref{thm:existDSBO}.  
\end{remark}

For the proof of Theorem \ref{thm:vanishDiff}, 
 we use the following properties of distributions supported at the origin:
\begin{lemma}
\label{lem:distpos}
Let $F$ be any $\operatorname{Hom}_{{\mathbb{C}}}
 (V,W)$-valued distribution on ${\mathbb{R}}^n$
 supported at the origin
 and satisfying the Euler homogeneity differential equation \eqref{eqn:Fainv}.  
\begin{enumerate}
\item[{\rm{(1)}}]
Assume $\nu - \lambda \not \in {\mathbb{N}}$.  
Then $F$ must be zero. 
\item[{\rm{(2)}}]
Assume $\nu - \lambda \in {\mathbb{N}}$.  
Then $F(-x)=(-1)^{\nu -\lambda}F(x)$.   
\end{enumerate}
\end{lemma}
\begin{proof}
Let $\delta(x) \equiv \delta(x_1, \cdots, x_n)$ be the Dirac delta function 
on ${\mathbb{R}}^n$.  
For a multi-index $\alpha=(\alpha_1, \cdots, \alpha_n) \in {\mathbb{N}}^n$, 
 we define another distribution by 
\[
\delta^{(\alpha)}(x_1, \cdots, x_n)
:=
\frac{\partial^{|\alpha|}}{\partial x_1^{\alpha_1} \cdots \partial x_n^{\alpha_n}}  
\delta(x_1, \cdots, x_n)
\]
where $|\alpha|=\alpha_1+ \cdots +\alpha_n$.   
By the structural theory 
 of distributions,
 $F$ must be of the following form 
\[
   F=\sum_{\alpha \in {\mathbb{N}}^n} a_{\alpha} 
   \delta^{(\alpha)}(x_1,\cdots,x_n)
\qquad
\text{(finite sum)}
\]
with some $a_{\alpha} \in \operatorname{Hom}_{{\mathbb{C}}}
 (V,W)$
 for $\alpha \in {\mathbb{N}}^n$.  
Since $\delta^{(\alpha)}(x_1,\cdots,x_n)$ is 
 a homogeneous distribution
 of degree $-n-|\alpha|$, 
 the Euler homogeneity operator 
\index{A}{Euler@$E$, Euler homogeneity operator\quad}
$E$ acts
 as the scalar multiplication
 by $-(n+|\alpha|)$, 
 and thus 
\[
  E F
  = 
  -\sum_{\alpha \in {\mathbb{N}}^n} 
  (n+|\alpha|)
   a_{\alpha} 
   \delta^{(\alpha)}(x_1,\cdots,x_n).  
\]
Since $\{ \delta^{(\alpha)}(x_1,\cdots,x_n) \}_{\alpha \in {\mathbb{N}}^n}$
 are linearly independent distributions,
 the differential equation \eqref{eqn:Fainv}, 
 namely, 
 $E F = (\lambda - \nu - n) F$
 implies that
\[
  a_{\alpha}=0
\qquad
\text{whenever }
  -n-|\alpha| \ne \lambda - \nu - n.  
\]
Thus we conclude:
\begin{enumerate}
\item[(1)]
If $\nu - \lambda \not \in {\mathbb{N}}$, 
 we get $a_{\alpha}=0$ 
 for all $\alpha \in {\mathbb{N}}^n$, 
 whence $F=0$.  
\item[(2)]
If $\nu - \lambda \in {\mathbb{N}}$, 
 then $a_{\alpha}$ can survive 
 only when $|\alpha|= \nu - \lambda$.  
Then $F(-x)=(-1)^{|\alpha|}F(x)=(-1)^{\nu - \lambda}F(x)$
 because $\delta(x)=\delta(-x)$.  
\end{enumerate}
Therefore Lemma \ref{lem:distpos} is proved.  
\end{proof}

\begin{proof}
[Proof of Theorem \ref{thm:vanishDiff}]
Immediate from the characterization 
 of differential symmetry breaking operators
 (Proposition \ref{prop:Tpair} (3)) and from Lemma \ref{lem:distpos}.  
\end{proof}

%%%%%%%%%%%%%%%%%%%%%%%%%%%%%%%%%%%%%%%%%%%%%%%%%%
\subsection{Upper estimate of the multiplicities}
\label{subsec:SDone}
%%%%%%%%%%%%%%%%%%%%%%%%%%%%%%%%%%%%%%%%%%%%%%%%%%
We recall from the general theory 
 \cite{xKOfm}
 that there exists a constant $C >0$
 such that
\begin{equation}
\label{eqn:IJdimC}
  \dim_{\mathbb{C}}{\operatorname{Hom}}_{G'}
  (I_{\delta}(V,\lambda)|_{G'}, J_{\varepsilon}(W,\nu))\le C
\end{equation}
for any $(\sigma,V) \in \widehat {O(n)}$, 
 $(\tau,W) \in \widehat {O(n-1)}$, 
 $\delta, \varepsilon \in \{\pm\}$, 
 and $(\lambda,\nu) \in {\mathbb{C}}^2$. 
Moreover,
 we also know
 that the left-hand side of \eqref{eqn:IJdimC} is either 0 or 1
 if both the $G$-module $I_{\delta}(V,\lambda)$
 and the $G'$-module $J_{\varepsilon}(W,\nu)$ are irreducible
 \cite{SunZhu}.  
In this section,
 we give a more precise upper estimate
 of the dimension of (continuous) symmetry breaking operators
 by that of {\it{differential}} symmetry breaking operators.  
Owing to the 
\index{B}{dualitytheorem@duality theorem, between differential symmetry breaking operators and Verma modules|textbf}
\lq\lq{duality theorem}\rq\rq\
 (see \cite[Thm.~2.9]{KP1}, 
 see also Fact \ref{fact:dualityDSV} in the next chapter), 
 the latter object can be studied algebraically
 as a branching problem for generalized Verma modules, 
 and is completely classified in \cite{KKP}
 when $(V,W)=(\Exterior^i({\mathbb{C}}^n), \Exterior^j({\mathbb{C}}^{n-1}))$.  
The proof for the upper estimate leads us 
 to complete the proof of a 
\index{B}{localnesstheorem@localness theorem}
 localness theorem
 (Theorem \ref{thm:152347}), 
 namely,
 a sufficient condition
 for all symmetry breaking operators
 to be differential operators. 

\begin{theorem}
[upper estimate of dimension]
\label{thm:SDone}
For any $V \in \widehat{O(n)}$,  
 $W \in \widehat{O(n-1)}$, 
 $\delta, \varepsilon \in \{ \pm \}$, 
 and $(\lambda, \nu)\in {\mathbb{C}}^2$, 
 we have
\begin{equation*}
   \dim_{\mathbb{C}}
   \operatorname{Hom}_{G'}
  (
   I_{\delta}(V,\lambda)|_{G'}, J_{\varepsilon}(W,\nu))
   \le 
   1+ 
   \dim_{\mathbb{C}}
  \operatorname{Diff}_{G'}
  (
   I_{\delta}(V,\lambda)|_{G'}, 
   J_{\varepsilon}(W,\nu)).  
\end{equation*}
\end{theorem}
\begin{proof}
Let $({\mathcal{T}}_{\infty},{\mathcal{T}})$ be the pair
 of distribution kernels 
 of a symmetry breaking operator ${\mathbb{T}}$ as in Proposition \ref{prop:Tpair}.  
Then the first projection
 ${\mathbb{T}} \mapsto {\mathcal{T}}_{\infty}$
 induces an exact sequence:
\begin{equation*}
0 \to
 \operatorname{Diff}_{G'}
  (
   I_{\delta}(V, \lambda)|_{G'}, 
   J_{\varepsilon}(W,\nu)
  )
  \to 
 \operatorname{Hom}_{G'}
  (
   I_{\delta}(V, \lambda)|_{G'}, 
   J_{\varepsilon}(W,\nu)
  )
\to 
{\mathbb{C}}(\Attcal \lambda \nu {\delta \varepsilon} {V,W})_{\infty}, 
\end{equation*}
 by Proposition \ref{prop:Tpair} (3)
 and Proposition \ref{prop:20150828-1231}.  
Thus Theorem \ref{thm:SDone} is proved.  
\end{proof}

We are ready to prove a localness theorem 
  stated in Theorem \ref{thm:152347}.  
\begin{proof}
[Proof of Theorem \ref{thm:152347}]
If $[V:W]=0$
 then 
$
   ({\mathcal{D}}'({\mathbb{R}}^n) 
    \otimes 
    \operatorname{Hom}_{\mathbb{C}}(V_{\lambda,\delta}, W_{\nu, \varepsilon}))^{P'}=\{0\}
$
 by Proposition \ref{prop:20150828-1231}
 because $\pr V W =0$.  
Hence we get Theorem \ref{thm:152347}
 by the exact sequence
 in the above proof.  
\end{proof}



We also prove a part of Theorem \ref{thm:unique}, 
 a generic uniqueness result.  
\begin{corollary}
\label{cor:160150upper}
Suppose $(\sigma, V) \in \widehat{O(n)}$, 
 $(\tau, W) \in \widehat{O(n-1)}$, 
 $\delta, \varepsilon \in \{\pm \}$, 
 and $(\lambda,\nu)\in {\mathbb{C}}^2$.  
If $(\lambda, \nu, \delta, \varepsilon)$
 satisfies the generic parameter condition
 \eqref{eqn:nlgen}, 
 namely, 
 if $\nu - \lambda \not \in 2{\mathbb{N}}$
 for $\delta \varepsilon=+$, 
 or $\nu - \lambda \not \in 2{\mathbb{N}}+1$
 for $\delta \varepsilon=-$, 
then 
\[
  \dim_{\mathbb{C}} 
  \operatorname{Hom}_{G'}
  (I_{\delta}(V,\lambda)|_{G'},J_{\varepsilon}(W,\nu))
  \le 1.  
\]
\end{corollary}

\begin{proof}
[Proof of Corollary \ref{cor:160150upper}]
Owing to Theorem \ref{thm:SDone}, 
 we obtain Corollary \ref{cor:160150upper} by 
 Theorem \ref{thm:vanishDiff}.  
\end{proof}

We shall see 
 that the inequality in Corollary \ref{cor:160150upper}
 is actually the equality 
 by showing the lower estimate
 of the multiplicities in Theorem \ref{thm:existSBO}
 below.  
%%%%%%%%%%%%%%%%%%%%%%%%%%%%%%%%%%%%%%
\subsection{Proof of Theorem \ref{thm:152389}: Analytic continuation of 
symmetry breaking operators
 $\Atbb \lambda \nu {\pm} {V, W}$}
\label{subsec:holoAVW}
%%%%%%%%%%%%%%%%%%%%%%%%%%%%%%%%%%%%%%
The goal of this section
 is to complete the proof of Theorem \ref{thm:152389}
 about the analytic continuation of $\Atbb \lambda \nu {\pm}{V,W}$.  
For $(\sigma,V) \in \widehat {O(n)}$ and 
 $(\tau,W) \in \widehat{O(n-1)}$ such that $[V:W]\ne 0$
 and for $\delta, \varepsilon \in \{\pm\}$, 
 we set $\gamma =\delta \varepsilon$
 and construct a family
 of matrix-valued symmetry breaking operators,
 to be denoted by
\[
  \Atbb \lambda \nu {\gamma} {V, W} \colon
  I_{\delta}(V,\lambda) \to J_{\varepsilon}(W,\nu), 
\]
 which are initially defined
 for $\operatorname{Re} \lambda \gg |\operatorname{Re} \nu|$
 in Lemma \ref{lem:Astep1}.  
We show that they have a holomorphic continuation 
 to the entire plane $(\lambda, \nu) \in {\mathbb{C}}^2$, 
 and thus complete the proof of Theorem \ref{thm:152389}.  


\vskip 0.8pc
Here is a strategy.  
\par\noindent
{\bf{Step 0.}}
(distribution kernel near infinity)\enspace

We define ${\operatorname{Hom}}_{\mathbb{C}}(V,W)$-valued distributions
 $(\Atcal \lambda \nu {\gamma} {V, W})_{\infty}$
 on ${\mathbb{R}}^n$ as a multiplication 
 of $(\Attcal \lambda \nu {\gamma} {V, W})_{\infty}$
 (see \eqref{eqn:AVW+u} and \eqref{eqn:AVW-u})
 by appropriate holomorphic functions
 of $\lambda$ and $\nu$
 (Section \ref{subsec:Ainfty}).  
The distributions $(\Atcal \lambda \nu \gamma{V,W})_{\infty}$ depend holomorphically
 on $(\lambda,\nu)$ in the entire plane ${\mathbb{C}}^2$
 (but may vanish at special $(\lambda,\nu)$).  

\vskip 0.8pc
\par\noindent
{\bf{Step 1.}}\enspace
(very regular case)
\enspace
For $\operatorname{Re} \lambda \gg |\operatorname{Re} \nu|$, 
 we define ${\operatorname{Hom}}_{\mathbb{C}}(V,W)$-valued,
 locally integrable functions
 $\Atcal \lambda \nu {\pm} {V,W}$ on ${\mathbb{R}}^n$
 such that the restriction $\Atcal \lambda \nu {\pm} {V,W}|_{\mathbb{R}^n \setminus \{0\}}$
 satisfies the compatibility condition \eqref{eqn:Tpatching}.  
We then prove that the pair
$
((\Atcal \lambda \nu {\gamma} {V,W})_{\infty}, 
  \Atcal \lambda \nu {\gamma} {V,W})
$
 belongs to 
$
   ({\mathcal{D}}'(G/P, {\mathcal{V}}_{\lambda,\delta}^{\ast}) \otimes W_{\nu,\varepsilon})^{\Delta(P')}
$
 for $\delta \varepsilon = \gamma$
 if ${\operatorname{Re}} \lambda \gg |{\operatorname{Re}} \nu|$.  



\vskip 0.8pc
\par\noindent
{\bf{Step 2.}}\enspace
(meromorphic continuation and possible poles of $\Atcal \lambda \nu \pm {V,W}$)\enspace
We find polynomials $p_{\gamma}^{V,W}(\lambda,\nu)$
 such that $p_{\gamma}^{V,W}(\lambda,\nu) \Atcal \lambda \nu \gamma {V,W}$
 is a family of distributions on ${\mathbb{R}}^n$
 that depend {\it{holomorphically}} on $(\lambda, \nu) \in {\mathbb{C}}^2$
 (see Proposition \ref{prop:20151209}).  



\vskip 0.8pc
\par\noindent
{\bf{Step 3.}}\enspace
(holomorphic continuation of $\Atcal \lambda \nu \pm {V,W}$)\enspace
We prove that there are actually no poles
 of the distributions $\Atcal \lambda \nu \gamma {V,W}$
 by inspecting the residue formula
 of the {\it{scalar-valued}} symmetry breaking operators
 and the zeros
 of the polynomials $p_{\gamma}^{V,W}(\lambda,\nu)$.  
Thus $\Atcal \lambda \nu \gamma {V,W}$ are
 distributions on ${\mathbb{R}}^n$
 that depend holomorphically on $(\lambda,\nu) \in {\mathbb{C}}^2$.  



Thus the pair $((\Atcal \lambda \nu {\gamma} {V,W})_{\infty}, \Atcal \lambda \nu {\gamma} {V,W})$
 gives an element
 of ${\mathcal{D}}'(G/P, {\mathcal{V}}_{\lambda,\delta}^{\ast}) \otimes W_{\nu,\varepsilon}$
 for $\delta \varepsilon =\gamma$
 which is invariant under the diagonal action of $P'$, 
 yielding a regular symmetry breaking operator $\Atbb \lambda \nu {\gamma} {V,W}$
 that depends holomorphically on $(\lambda,\nu) \in {\mathbb{C}}^2$
 by Proposition \ref{prop:Tpair}.  


\vskip 1pc
The key idea for Steps 1 and 2
 is a reduction 
 to {\it{scalar-valued}} symmetry breaking operators
 which will be discussed
 in Section \ref{subsec:shiftA}
 (Lemma \ref{lem:152392}).  


%%%%%%%%%%%%%%%%%%%%%%%%%%%%%%%%%%%%%%%%%%%%%%%%%%%%%%%
\subsubsection{Normalized distributions
 $(\Atcal \lambda \nu {\gamma} {V,W})_{\infty}$
 at infinity}
\label{subsec:Ainfty}
%%%%%%%%%%%%%%%%%%%%%%%%%%%%%%%%%%%%%%%%%%%%%%%%%%%%%%%%
This is for Step 0.  
We note that the map ${\mathbb{T}} \mapsto {\mathcal{T}}_{\infty}$
 in Proposition \ref{prop:Tpair}
 is neither injective nor surjective
 in general.  
In particular,
 the nonzero distribution $(\Attcal \lambda \nu \pm {V,W})_{\infty}$
 on ${\mathbb{R}}^n$
 (see \eqref{eqn:AVW+u} and \eqref{eqn:AVW-u})
 does not always extend
 to the compactification $G/P$ as an element 
in $({\mathcal{D}}'(G/P, {\mathcal{V}}_{\lambda,\delta}^{\ast}) \otimes W_{\nu,\varepsilon})^{\Delta(P')}$, 
 see Proposition \ref{prop:20170213}.  
However,
 we shall see in Section \ref{subsec:pfAholo}
 that the following {\it{renormalization}} extends to a distribution 
on the compact manifold $G/P$
 for any $\lambda$, $\nu \in {\mathbb{C}}$.  
\begin{alignat*}{2}
  (\Atcal \lambda \nu + {V,W})_{\infty}
  :=& \frac{1}{\Gamma(\frac{\lambda-\nu}{2})}(\Attcal \lambda \nu + {V,W})_{\infty}
  &&= \frac{1}{\Gamma(\frac{\lambda-\nu}{2}) \Gamma(\frac{\lambda+\nu-n+1}{2})}
       |x_n|^{\lambda + \nu -n} \pr V W, 
\\
  (\Atcal \lambda \nu - {V,W})_{\infty}
  :=& \frac{1}{\Gamma(\frac{\lambda-\nu+1}{2})}(\Attcal \lambda \nu - {V,W})_{\infty}
  &&= \frac{1}{\Gamma(\frac{\lambda-\nu+1}{2}) \Gamma(\frac{\lambda+\nu-n+2}{2})}
   |x_n|^{\lambda + \nu -n} \operatorname{sgn} x_n\pr V W.  
\end{alignat*}
By definition, 
 $(\Atcal \lambda \nu \pm {V,W})_{\infty}$ are distributions
 on ${\mathbb{R}}^n$ 
 that depend holomorphically on $(\lambda, \nu)$ in the entire ${\mathbb{C}}^2$.  
Inspecting the poles of $\Gamma(\frac{\lambda-\nu}{2})$ and 
$\Gamma(\frac{\lambda-\nu+1}{2})$, 
 we immediately have the following:


\begin{lemma}
\label{lem:ABzero}
Suppose $[V:W] \ne 0$.  
Then, 
$(\Atcal \lambda \nu + {V,W})_{\infty}=0$
 if and only if $\nu - \lambda \in 2 {\mathbb{N}}$;
 $(\Atcal \lambda \nu - {V,W})_{\infty}=0$
 if and only if $\nu - \lambda \in 2 {\mathbb{N}}+1$.  
\end{lemma}

%%%%%%%%%%%%%%%%%%%%%%%%%%%%%%%%%%%%%%%%%%%%%%%%%%%%%%%%%%%%
\subsubsection{Preliminary results in the scalar-valued case}
\label{subsec:shiftA}
%%%%%%%%%%%%%%%%%%%%%%%%%%%%%%%%%%%%%%%%%%%%%%%%%%%%%%%%%%%%

As we have seen 
 in Section \ref{subsec:Ainfty}, 
 the analytic continuation 
 of the distribution $(\Atcal \lambda \nu {\gamma} {V,W})_{\infty}$
 at infinity is easy.  
In order to deal with the nontrivial case,
 {\it{i.e.,}}
 the distribution kernel $\Atcal \lambda \nu {\gamma} {V,W}$
 near the origin, 
 we begin with some basic properties
 of the {\it{scalar-valued}} symmetry breaking operators.  
We recall from \cite[(7.8)]{sbon} 
 that the (scalar-valued) distribution kernels
 $\Atcal \lambda \nu \pm {} \in {\mathcal{D}}'({\mathbb{R}}^n)$
 are initially defined
 as locally integrable functions on ${\mathbb{R}}^n$
 by 
\index{A}{Act1@$\Atcal \lambda \nu + {}$}
\index{A}{Act2@$\Atcal \lambda \nu - {}$}
\begin{align}
\Atcal \lambda \nu + {} (x, x_n)
=& \frac{1}{\Gamma (\frac{\lambda+\nu-n+1}{2}) \Gamma (\frac{\lambda-\nu}{2})}
    (|x|^2+ x_n^2)^{-\nu} |x_n|^{\lambda + \nu -n},
\label{eqn:KAlnn+}
\\
\Atcal{\lambda}{\nu}{-}{} (x, x_n)
=& \frac{1}
   {\Gamma (\frac{\lambda+\nu-n+2}{2}) \Gamma (\frac{\lambda-\nu+1}{2})}
   (|x|^2+ x_n^2)^{-\nu} |x_n|^{\lambda+\nu-n}\operatorname{sgn} x_n,
\label{eqn:KAlnn-}
\end{align}
respectively
 for $\operatorname{Re}\lambda \gg |\operatorname{Re} \nu|$.  
(In \cite{sbon}, 
we used the notation $\widetilde K_{\lambda,\nu}^{\mathbb{A}}$
 for the scalar-valued distribution kernel $\Atcal{\lambda}{\nu}{+}{}$.)
More precisely,
 we have:
\begin{fact}
[{\cite[Chap.~7]{sbon}}]
\label{fact:sbonA}
$\Atcal{\lambda}{\nu}{\pm}{}$ are locally integrable 
 on ${\mathbb{R}}^n$
 if $\operatorname{Re}\left(\lambda - \nu\right)>0$
 and $\operatorname{Re}\left(\lambda + \nu\right)>n-1$, 
 and extend as distributions
 on ${\mathbb{R}}^n$
 that depend holomorphically on $\lambda$, $\nu$
 in the entire $(\lambda,\nu) \in {\mathbb{C}}^2$.  
\end{fact}
The distributions $\Atcal {\lambda}{\nu}{+}{}$ were
 thoroughly studied in \cite[Chap. 7]{sbon}, 
 and analogous results for $\Atcal \lambda \nu -{}$
 can be proved exactly in the same way. 



We introduce polynomials 
 $p_{\pm,N}(\lambda,\nu)$ of the two-variables $\lambda$ and $\nu$
 by 
\index{A}{p1N@$p_{+,N}(\lambda,\nu)$|textbf}
\index{A}{p2N@$p_{-,N}(\lambda,\nu)$|textbf}
\begin{alignat}{2}
p_{+,N}(\lambda,\nu):=& \prod_{j=1}^{N} (\lambda-\nu-2j)
\quad
&&\text{for $N \in {\mathbb{N}}_+$}, 
\label{eqn:p+N}
\\
p_{-,N}(\lambda,\nu):=& (\lambda+\nu-n)\prod_{j=0}^{N} (\lambda-\nu-1-2j)
\quad
&&\text{for $N \in {\mathbb{N}}$}.  
\label{eqn:p-N}
\end{alignat}
We use a trick to raise the regularity 
 of the distribution $\Atcal \lambda \nu +{}(x,x_n)$
 at the origin 
 by shifting the parameter.  
The resulting distributions are under control
 by the polynomials $p_{\pm,N}(\lambda,\nu)$ as follows:
\begin{lemma}
\label{lem:pAshift}
We have the following identities
 as distributions on ${\mathbb{R}}^n$:
\begin{align*}
p_{+,N}(\lambda, \nu) \Atcal \lambda \nu + {} (x,x_n)
=&
2^{N} (|x|^2+x_n^2)^N \Atcal {\lambda-N}{\nu+N}{+}{} (x,x_n), 
\\
p_{-,N}(\lambda, \nu) \Atcal \lambda \nu - {} (x,x_n)
=&
2^{N+2} (|x|^2+x_n^2)^N x_n \Atcal {\lambda-N-1}{\nu+N}{+}{} (x,x_n). 
\end{align*}
\end{lemma}

\begin{proof}
For $\operatorname{Re}\lambda \gg  |\operatorname{Re}\nu|$, 
 we have from the definition \eqref{eqn:KAlnn+}, 
\begin{align*}
  (|x|^2+ x_n^2)^N \Atcal {\lambda-N}{\nu+N}{+}{} (x,x_n)
=&
\frac{\Gamma(\frac{\lambda-\nu}{2})}{\Gamma(\frac{\lambda-\nu}{2}-N)}
\Atcal{\lambda}{\nu}{+}{} (x,x_n)
\\
=&
\frac{1}{2^N} p_{+,N}(\lambda,\nu) \Atcal \lambda \nu +{}(x,x_n).  
\end{align*}
Since both sides depend holomorphically on $(\lambda, \nu)\in {\mathbb{C}}^2$, 
 we get the first assertion.  
The proof of the second assertion goes similarly.  
\end{proof}

\begin{lemma}
\label{lem:1.16}
If $(\lambda,\nu) \in {\mathbb{C}}^2$ satisfies $p_{+,N}(\lambda, \nu)=0$, 
then 
\[
  h(x,x_n) \Atcal {\lambda-N}{\nu+N}{+}{}=0
\quad
 \text{in } {\mathcal{D}}'
   ({\mathbb{R}}^n), 
\]
for all homogeneous polynomials $h(x,x_n)$
 of degree $2N$.  
\end{lemma}

\begin{proof}
It follows from $p_{+, N}(\lambda, \nu)=0$
 that 
\[
   (\nu+N)-(\lambda-N) \in \{0,2,4,\cdots,2N-2\}.  
\]
By the residue formula
 of the scalar-valued symmetry breaking operator
 $\Atcal {\lambda'} {\nu'} {+} {}$ 
 (see \cite[Thm.~12.2 (2)]{sbon}), 
 we have 
\[
  \Atcal {\lambda-N}{\nu+N}{+}{} = q \, \Ctcal{\lambda-N}{\nu+N}{}
\]
 for some constant 
\index{A}{qAC@$q_C^A$}
$q\equiv q_C^A(\lambda-N,\nu+N)$
 depending on $\lambda-N$ and $\nu+N$.  
Since $\Ctcal{\lambda-N}{\nu+N}{}$ is a distribution
 of the form $D \delta(x_1, \cdots, x_n)$
 where 
\index{A}{CGegenbauernorm@$\widetilde C_l^{\alpha}(z)$, normalized Gegenbauer polynomial}
$
   D=\widetilde C_{2N-2j}^{\lambda - N - \frac{n-1}{2}}(-\Delta_{\mathbb{R}^{n-1}}, \frac{\partial}{\partial x_n})$ is a differential operator
 of homogeneous degree $2N-2j (< 2N)$, 
 see \eqref{eqn:Gegentwo}, 
 an iterated use of the Leibniz rule shows
\[
   \text{$h(x,x_n)D \delta(x_1, \cdots, x_n)=0$
 in ${\mathcal{D}}'
   ({\mathbb{R}}^n)$}
\]
 for any homogeneous polynomial $h(x,x_n)$
 of degree $2N$.  
\end{proof}

\begin{lemma}
\label{lem:152293-copy}
If $(\lambda,\nu) \in {\mathbb{C}}^2$ satisfies $p_{-,N}(\lambda,\nu)=0$, 
then 
\[
   x_n h(x,x_n) \Atcal {\lambda-N-1}{\nu+N}{+}{} (x,x_n) =0
\quad\text{ in }
{\mathcal{D}}'({\mathbb{R}}^n)
\]
for all homogeneous polynomial $h(x,x_n)$ of degree $2N$.  
\end{lemma}
\begin{proof}
It follows from $p_{-, N}(\lambda, \nu)=0$
 that $(\nu+N)-(\lambda-N-1) \in \{0,2,\cdots,2N\}$
 or $(\lambda-N-1)+(\nu+N) =n-1$.  
By using again the residue formula
 of the scalar-valued symmetry breaking operator
 $\Atcal {\lambda'}{\nu'}{+}{}$
 in \cite[Thm.~12.2]{sbon}, 
 we see that the distribution kernel $\Atcal {\lambda-N-1}{\nu+N}{+}{}(x,x_n)$ is
 a scalar multiple
 of the following distributions:
\begin{alignat*}{2}
& \delta(x_n)
\qquad
&&\text{if } \lambda + \nu=n, 
\\
& D \delta(x_1, \cdots,x_n)
\qquad
&&\text{if } \lambda - \nu =2j+1 \qquad (0 \le j \le N),   
\end{alignat*}
where 
$
    D 
    = 
    {\widetilde C}_{2N-2j}^{\lambda-N-1-\frac{n-1}2}(-\Delta_{\mathbb{R}^{n-1}}, \frac \partial{\partial x_n})
$
 is a differential operator
 of homogeneous degree $2N-2j$
($< 2N+1$).  
Then the multiplication by a homogeneous polynomial $x_n h(x,x_n)$
 of degree $2N+1$
 annihilates these distributions.  
Hence the lemma follows.  
\end{proof}


\subsubsection{Step 1. Very regular case}
\label{subsec:Astep1}


We recall from \eqref{eqn:RVW}
 that 
\index{A}{RVW@$\Rij VW$}
\index{A}{1psin@$\psi_n$}
$
\Rij VW = \pr VW \circ \sigma \circ \psi_n$
$\in C^{\infty}({\mathbb{R}}^n \setminus \{0\}) \otimes
 {\operatorname{Hom}}_{\mathbb{C}}(V,W)$.  
For $\operatorname{Re}\lambda \gg |\operatorname{Re}\nu|$, 
 we define
$\Atcal \lambda \nu \pm {V,W}
 \in C({\mathbb{R}}^n \setminus \{0\}) \otimes \operatorname{Hom}_{\mathbb{C}}(V,W)$
 by
\begin{alignat*}{2}
\Atcal \lambda \nu + {V,W}
:=&
\frac{1}{\Gamma(\frac{\lambda+\nu-n+1}{2})\Gamma(\frac{\lambda-\nu}{2})}
(|x|^2+x_n^2)^{-\nu} |x_n|^{\lambda+\nu-n}
\Rij VW (x,x_n), 
\\
\Atcal \lambda \nu - {V,W}
:=&
\frac{1}{\Gamma(\frac{\lambda+\nu-n+2}{2})\Gamma(\frac{\lambda-\nu+1}{2})}
(|x|^2+x_n^2)^{-\nu} |x_n|^{\lambda+\nu-n}
\operatorname{sgn} x_n
\Rij VW (x,x_n), 
\end{alignat*}
 (see \eqref{eqn:KVWpt} and \eqref{eqn:KVWmt}), 
 respectively.  
The goal of this section
 is to prove the following lemma
 in the matrix-valued case  
 for $\operatorname{Re} \lambda \gg |\operatorname{Re} \nu|$.  
\begin{lemma}
\label{lem:Astep1}
Let $(\sigma,V) \in \widehat {O(n)}$, 
 $(\tau,W) \in \widehat {O(n-1)}$
 and $\delta$, $\varepsilon \in \{ \pm \}$.  
Suppose $\operatorname{Re}\left(\lambda -\nu\right) >0$
 and $\operatorname{Re}\left(\lambda + \nu\right) > n-1$.  

\begin{enumerate}
\item[{\rm{(1)}}]
$\Atcal \lambda \nu \pm {V,W}$ are
 $\operatorname{Hom}_{\mathbb{C}}(V,W)$-valued
 locally integrable functions on ${\mathbb{R}}^n$.  

\item[{\rm{(2)}}]
The pair $((\Atcal \lambda \nu {\delta \varepsilon} {V,W})_{\infty}, \Atcal \lambda \nu {\delta \varepsilon} {V,W})$ defines an element
 of $({\mathcal{D}}'(G/P, {\mathcal{V}}_{\lambda,\delta}^{\ast}) \otimes W_{\nu,\varepsilon})^{\Delta(P')}$, 
 and thus yield a symmetry breaking operator
 $\Atbb \lambda \nu {\delta \varepsilon} {V,W}\colon I_{\delta}(V,\lambda) \to J_{\varepsilon}(W,\nu)$.  

\end{enumerate}
\end{lemma}



\begin{proof}
We fix inner products on $V$ and $W$
 that are invariant by $O(n)$ and $O(n-1)$, 
 respectively.  
\index{A}{opnor@$\mid\mid \cdot \mid\mid_{\operatorname{op}}$, operator norm}
Let $\| \cdot \|_{\operatorname{op}}$
 denote the operator norm
 for linear maps 
 between (finite-dimensional) Hilbert spaces.  
In view of the definition $\Rij V W = \pr V W \circ \sigma \circ \psi_n$
 (see \eqref{eqn:RVW}), 
 we have
\[
   \| \Rij V W(x,x_n)  \|_{\operatorname{op}}
  \le
  \| \sigma \circ \psi_n (x,x_n) \|_{\operatorname{op}}
  = 1
\quad
  \text{for all }
  (x,x_n) \in {\mathbb{R}}^n \setminus \{0\}.  
\]
Hence the first statement is reduced to the scalar case
 as stated in Fact \ref{fact:sbonA}.  



The compatibility condition \eqref{eqn:Tpatching}
 can be verified readily from the definition
 of $(\Atcal \lambda \nu \pm {V,W})_{\infty}$
 and $\Atcal \lambda \nu \pm {V,W}$.  
Hence the pair
$
   ((\Atcal \lambda \nu {\delta\varepsilon} {V,W})_{\infty}, 
   \Atcal \lambda \nu {\delta\varepsilon} {V,W})
$
 defines an element of 
 ${\mathcal{D}}'(G/P, {\mathcal{V}}_{\lambda,\delta}^{\ast}) \otimes W_{\nu, \varepsilon}$
 by Lemma \ref{lem:152341}.  
The invariance under the diagonal action of $P'$
 follows from Proposition \ref{prop:20150828-1231}
 for $(\Atcal \lambda \nu {\delta\varepsilon}{V,W})_{\infty}$
 and from a direct computation 
 for $\Atcal \lambda \nu {\delta\varepsilon}{V,W}$
 when $\operatorname{Re} \lambda \gg |\operatorname{Re}\nu|$
 because 
both $(\Atcal \lambda \nu {\delta\varepsilon}{V,W})_{\infty}$
 and $\Atcal \lambda \nu {\delta\varepsilon} {V,W} \in L_{\operatorname{loc}}^1({\mathbb{R}}^n)$.  
\end{proof}



\subsubsection{Step 2. Reduction to the scalar-valued case}
We shall prove:
\begin{proposition}
\label{prop:20151209}
Let $(\sigma,V) \in\widehat{O(n)}$
 and $(\tau,W) \in\widehat{O(n-1)}$.  
Then the distributions  
$\Atcal \lambda \nu {\pm} {V,W}$, 
 initially defined
 as an element of $L_{\operatorname{loc}}^1({\mathbb{R}}^n) \otimes \operatorname{Hom}_{\mathbb{C}}(V,W)$
 for $\operatorname{Re} \lambda \gg |\operatorname{Re} \nu|$
 in Lemma \ref{lem:Astep1}, 
 extend meromorphically 
 in the entire plane $(\lambda,\nu) \in {\mathbb{C}}^2$.  
\end{proposition}

In order to prove Proposition \ref{prop:20151209}, 
 we need to control the singularity
 of $\sigma \circ \psi_n \in C^{\infty}({\mathbb{R}}^n \setminus \{0\}) \otimes
 {\operatorname{End}}_{\mathbb{C}}(V)$
 at the origin.  
We formulate a necessary lemma:
\begin{lemma}
\label{lem:1.12}
For any irreducible representation $(\sigma, V)$ of $O(n)$, 
 there exists $N \in {\mathbb{N}}$ 
such that
\[
  g(x,x_n):= (|x|^2+x_n^2)^{N} \sigma(\psi_n(x,x_n))
\]
 is an $\operatorname{End}(V)$-valued homogeneous polynomial of degree $2N$.  
\end{lemma}

\begin{definition}
\label{def:Nsigma}
For $\sigma \in \widehat {O(n)}$, 
 we denote by 
\index{A}{Nsigma@$N(\sigma)$|textbf}
$N(\sigma)$
  the smallest integer $N$
 satisfying the conclusion of Lemma \ref{lem:1.12}.  
\end{definition}

We prove Lemma \ref{lem:1.12}
 by showing the following estimate of the integer $N(\sigma)$.  
Let $\ell(\sigma)$ be as defined in \eqref{eqn:ONlength}.  

\begin{lemma}
\label{lem:Nsigma}
\index{A}{lengthrep@$\ell(\sigma)$}
$N(\sigma) \le \ell(\sigma)$
 for all $\sigma \in \widehat {O(n)}$.  
\end{lemma}

\begin{proof}
[Proof of Lemma \ref{lem:Nsigma}]
Suppose 
\index{A}{0tExterior@$\Lambda^+(O(N))$}
$(\sigma_1, \cdots, \sigma_n) \in \Lambda^+(O(n))$, 
 and let $(\sigma,V)$ be the irreducible finite-dimensional 
 representation $\Kirredrep {O(n)}{\sigma_1, \cdots, \sigma_n}$
 of $O(n)$
 via the Cartan--Weyl isomorphism
 \eqref{eqn:CWOn}.  
It is convenient to set $\sigma_{n+1}=0$.  
Since the exterior representations
 of $GL(n,{\mathbb{C}})$ on $\Exterior^j({\mathbb{C}}^n)$
 have 
 highest weights $(1,\cdots,1,0,\cdots,0)$, 
 and since
\[
   \sum_{j=1}^{n}(\sigma_j-\sigma_{j+1})
   (\underbrace{1,\cdots,1}_j,0,\cdots,0)
  =(\sigma_1,\cdots,\sigma_n), 
\]
we can realize the irreducible representation of $GL(n,{\mathbb{C}})$
 with highest weight $(\sigma_1,\cdots,\sigma_n)$
 as a subrepresentation
 of the tensor product representation
\[
  \bigotimes_{j=1}^{n} (\Exterior^j({\mathbb{C}}^n))^{\sigma_j-\sigma_{j+1}}.  
\]
This is a polynomial representation of homogeneous degree
\[
   \sum_{j=1}^n j (\sigma_j-\sigma_{j+1})=\sum_{j=1}^n \sigma_j.  
\]
We set $N:=\sum_{j=1}^n \sigma_j$.  
Then the matrix coefficients of this $GL(n,{\mathbb{C}})$-module
 are given by homogeneous polynomials
 of degree $N$
 of $z_{ij}$
 ($1 \le i,j \le n$)
 where $z_{ij}$ are the coordinates
 of $GL(n,{\mathbb{C}})$.  
Since the representation $(\sigma, V)$ of $O(n)$
 arises as a subrepresentation
 of this $GL(n,{\mathbb{C}})$-module, 
 the formula \eqref{eqn:psim}
 of $\psi_n$ shows that the matrix coefficients of $\sigma(\psi_n(x,x_n))$
 is a polynomial of $x$ and $x_n$ 
 after multiplying $(|x|^2 + x_n^2)^N$.  



We note
 that $\det \psi_n(x,x_n)=-1$
 for all $(x,x_n) \in {\mathbb{R}}^n \setminus \{0\}$
 by \eqref{eqn:psix}.  
Therefore, 
 we may assume
 that $(\sigma,V)$ is of 
\index{B}{type1@type I, representation of $O(N)$}
type I
 by \eqref{eqn:type1to2}, 
 namely, 
 $\sigma_{k+1}=\cdots=\sigma_n$
 for some $k$
 with $2k \le n$.  
In this case $N=l(\sigma)$
 by the definition \eqref{eqn:ONlength}.  
By \eqref{eqn:lsigma}, 
 we have shown the lemma.  
\end{proof}


The estimate in Lemma \ref{lem:Nsigma} is not optimal.  



\begin{example}
\label{ex:Nsigma}
\begin{enumerate}
\item[{\rm{1)}}]
$N(\sigma)=0$ if $(\sigma, V)$ is a one-dimensional representation.  
\item[{\rm{2)}}]
$N(\sigma)=1$ if $\sigma$ is the exterior representation
 on $V= \Exterior^i({\mathbb{C}}^n)$
 ($1 \le i \le n-1$).  
See \eqref{eqn:exrep}
 and Lemma \ref{lem:psidet} (2)
 for the proof.  
\end{enumerate}
\end{example}



Let $N \equiv N(\sigma) \in {\mathbb{N}}$ 
 and $g \in {\operatorname{Pol}}[x_1, \cdots, x_n] \otimes {\operatorname{End}}_{\mathbb{C}}(V)$
 be as in Lemma \ref{lem:1.12}, 
 and $\pr V W \colon V \to W$ be a nonzero $O(n-1)$-homomorphism.  
We define 
$
   g^{V,W} 
   \in{\operatorname{Pol}}[x_1, \cdots, x_n] 
   \otimes 
   {\operatorname{Hom}}_{\mathbb{C}}(V,W)
$
 by
\index{A}{gVW@$g^{V,W}$|textbf}
\begin{equation}
   g^{V,W}
  :=
  \pr V W \circ g.  
\end{equation}
With notation of 
\index{A}{RVW@$\Rij VW$}
$
 \Rij VW
$ as in \eqref{eqn:RVW}, 
 we have 
\begin{align}
\label{eqn:gVW}
g^{V,W}(x,x_n)=&(|x|^2+x_n^2)^N \Rij VW (x,x_n)
\\
\notag
              =&(|x|^2+x_n^2)^N \pr VW \circ \sigma(\psi_n(x,x_n)).  
\end{align}
Then $g^{V,W}$ is a $\operatorname{Hom}_{\mathbb{C}}(V,W)$-valued
 polynomial of homogeneous degree $2N$.  



The following lemma will imply 
 that the singularity at the origin
 of the matrix-valued distributions
 $\Atcal \lambda \nu {\pm}{V,W}$
 is under control by the scalar-valued case:
\begin{lemma}
\label{lem:152392}
Suppose $\operatorname{Re} \lambda \gg |\operatorname{Re} \nu|$.  
Let $p_{\pm}(\lambda,\nu)$ be the polynomials of $\lambda$ and $\nu$
 defined in \eqref{eqn:p+N} and \eqref{eqn:p-N}.  
Then, 
\index{A}{p1N@$p_{+,N}(\lambda,\nu)$}
\index{A}{p2N@$p_{-,N}(\lambda,\nu)$}
\begin{align}
\label{eqn:152392a}
p_{+,N}(\lambda,\nu) \Atcal \lambda \nu {+} {V,W}(x,x_n)
=&
2^N \Atcal {\lambda-N} {\nu+N}{+}{}(x,x_n)g^{V,W}(x,x_n), 
\\
\label{eqn:152392b}
p_{-,N}(\lambda,\nu) \Atcal \lambda \nu {-} {V,W}(x,x_n)
=&
2^{N+2} x_n \Atcal {\lambda-N-1} {\nu+N}{+}{}(x,x_n) g^{V,W}(x,x_n). 
\end{align}  
\end{lemma}
\begin{proof}
For $\operatorname{Re} \lambda \gg |\operatorname{Re} \nu|$, 
 both $\Atcal \lambda \nu {\pm} {V,W}$
 and $\Atcal \lambda \nu {\pm} {}$
 are locally integrable in ${\mathbb{R}}^n$.  
By definition, 
 we have
\begin{equation*}
(|x|^2+ x_n^2)^N \Atcal {\lambda}{\nu}{\gamma}{V,W}
 =
\Atcal {\lambda}{\nu}{\gamma}{} (x,x_n) g^{V,W} (x,x_n)
\end{equation*}
for $\gamma=\pm$.  
By Lemma \ref{lem:pAshift}, 
 we have
\[
  (|x|^2+x_n^2)^N
  (p_{+,N}(\lambda,\nu) \Atcal \lambda \nu +{V,W}(x,x_n)
  -2^N \Atcal {\lambda-N} {\nu+N} +{}(x,x_n) g^{V,W}(x,x_n))
  =0.  
\]
Hence we get the equality \eqref{eqn:152392a}
 as ${\operatorname{Hom}}_{\mathbb{C}}(V,W)$-valued 
 locally integrable functions in ${\mathbb{R}}^n$.  
Similarly,
we obtain
\[
  (|x|^2+ x_n^2)^N \Atcal {\lambda-N-1}{\nu+N}{+}{} (x,x_n) x_n
=
\frac{1}{2^{N+2}}
p_{-,N}(\lambda, \nu)
\Atcal{\lambda}{\nu}{-}{} (x,x_n).  
\]
Thus the second statement follows.  
\end{proof}

We are ready to prove the main result
 of this section.  
\begin{proof}
[Proof of Proposition \ref{prop:20151209}]
Since $g^{V,W}(x,x_n)$ is a polynomial of $(x,x_n)=(x_1, \cdots, x_n)$, 
 the multiplication of any distributions
 on ${\mathbb{R}}^n$ 
 by $g^{V,W}$
 is well defined.  
Therefore, 
 the right-hand sides of \eqref{eqn:152392a} and \eqref{eqn:152392b}
 make sense as distributions on ${\mathbb{R}}^n$
 that depend holomorphically in $(\lambda, \nu) \in {\mathbb{C}}^2$.  

Taking their quotients
 by the polynomials $p_{\pm,N}(\lambda, \nu)$, 
 we set
\index{A}{Actg1@$\Atcal {\lambda}{\nu}{+}{V,W}$|textbf}
\index{A}{Actg2@$\Atcal {\lambda}{\nu}{-}{V,W}$|textbf}
\begin{align}
\label{eqn:pAVW+}
\Atcal \lambda \nu {+} {V,W}(x,x_n)
:=&
\frac{2^N}{p_{+,N}(\lambda,\nu)}
\Atcal {\lambda-N} {\nu+N}{+}{}(x,x_n)
g^{V,W}(x,x_n), 
\\
\label{eqn:pAVW-}
\Atcal \lambda \nu {-} {V,W}(x,x_n)
:=&
\frac{2^{N+2}}{p_{-,N}(\lambda,\nu)}
\Atcal {\lambda-N-1} {\nu+N}{+}{}(x,x_n)
x_n
g^{V,W}(x,x_n).  
\end{align}  
Then $\Atcal \lambda \nu {\pm} {V,W}$ are $\operatorname{Hom}_{\mathbb{C}}(V,W)$-valued distributions
 on ${\mathbb{R}}^n$
 which depend meromorphically on $(\lambda,\nu) \in {\mathbb{C}}^2$
 because $\Atcal {\lambda'} {\nu'}{+}{} (x,x_n)$ is a family of scalar-valued distributions 
 on ${\mathbb{R}}^n$ 
 that depend holomorphically on $(\lambda',\nu') \in {\mathbb{C}}^2$
(Fact \ref{fact:sbonA})
 and $g^{V,W}(x,x_n)$ is a polynomial.  
By Lemma \ref{lem:152392}, 
 they coincide locally integrable functions
 on ${\mathbb{R}}^n$
 that are defined in \eqref{eqn:KVWpt} and \eqref{eqn:KVWmt}, 
 respectively, 
 when $\operatorname{Re} \lambda \gg |\operatorname{Re} \nu|$.  
Thus Proposition \ref{prop:20151209} is proved.  
\end{proof}

%%%%%%%%%%%%%%%%%%%%%%%%%%%%%%%%
\subsubsection{Step 3. Proof of holomorphic continuation}
%%%%%%%%%%%%%%%%%%%%%%%%%%%%%%%

In this section,
 we show
 that there are no poles of $\Atcal \lambda \nu {\pm} {V,W}$.  
\begin{lemma}
\label{lem:Aholo}
$\Atcal \lambda \nu {\pm} {V,W}$ are distributions on ${\mathbb{R}}^n$
 that depend holomorphically on $(\lambda, \nu)\in {\mathbb{C}}^2$.  
\end{lemma}
\begin{proof}
By \eqref{eqn:pAVW+} and \eqref{eqn:pAVW-}, 
 the only possible places that the distribution
 $\Atcal \lambda \nu {\gamma} {V,W}$ may have poles 
are the zeros of the denominators,
 namely,
\begin{alignat*}{2}
p_{+,N}(\lambda,\nu)
&=\prod_{j=1}^N (\lambda - \nu -2j) \qquad 
&& \gamma =+, 
\\
p_{-,N}(\lambda,\nu)
&=
(\lambda + \nu -n) \prod_{j=0}^N (\lambda - \nu -1-2j) 
\qquad
&& \gamma =-, 
\end{alignat*} 
however,
 we have proved
 that they are not actually poles
 by Lemmas \ref{lem:1.16} and \ref{lem:152293-copy}, 
 respectively.  
Hence $\Atcal \lambda \nu {\gamma} {V,W}$ are distributions
 that depend holomorphically on $(\lambda,\nu) \in {\mathbb{C}}^2$.  
\end{proof}


%%%%%%%%%%%%%%
\subsubsection{Proof of Theorem \ref{thm:152389}}
\label{subsec:pfAholo}
%%%%%%%%%%%%%%

We are ready to prove
 that the matrix-valued symmetry breaking operator
 $\Atbb \lambda \nu {\pm} {V,W}$ has a holomorphic continuation
 in the entire plane $(\lambda,\nu) \in {\mathbb{C}}^2$.  
 
\begin{proof}
[Proof of Theorem \ref{thm:152389}]
Suppose $(\sigma,V) \in \widehat {O(n)}$.  
Let 
\index{A}{Nsigma@$N(\sigma)$}
$
N \equiv N(\sigma) \in {\mathbb{N}}
$ be as in Lemma \ref{lem:1.12}.  
We recall from \eqref{eqn:gVW}
 that the $\operatorname{Hom}_{\mathbb{C}} (V,W)$-valued function 
\index{A}{gVW@$g^{V,W}$}
\begin{equation*}
 g^{V,W}(x,x_n)= (|x|^2+x_n^2)^{N} \operatorname{pr}_{V\to W} \circ\sigma(\psi_n(x,x_n))
\end{equation*}
 is actually a ${\operatorname{Hom}}_{\mathbb{C}}(V,W)$-valued polynomial
 of homogeneous degree $2N$.  



We know 
 that the pair $((\Atcal \lambda \nu {\pm} {V,W})_{\infty},\Atcal \lambda \nu {\pm} {V,W})$
 satisfies the following properties:
\begin{enumerate}
\item[{\rm{(1)}}]
$(\Atcal \lambda \nu {\pm} {V,W})_{\infty}$ is
 a $\operatorname{Hom}_{\mathbb{C}}(V,W)$-valued distribution 
 on ${\mathbb{R}}^n$ satisfying \eqref{eqn:Tinfty}
 that depend holomorphically
 in $(\lambda,\nu) \in {\mathbb{C}}^2$.  
\item[{\rm{(2)}}]
$\Atcal \lambda \nu {\pm} {V,W}$
 is a $\operatorname{Hom}_{\mathbb{C}}(V,W)$-valued distribution 
 on ${\mathbb{R}}^n$
 that depend holomorphically
 on $(\lambda, \nu) \in {\mathbb{C}}^2$.  
\item[{\rm{(3)}}]
For $\delta$, $\varepsilon \in \{\pm\}$, 
 $\Atcal \lambda \nu {\delta\varepsilon} {V,W}
\in {\mathcal{S}}ol({\mathbb{R}}^n; V_{\lambda, \delta}, 
 W_{\nu, \varepsilon})$.  
Moreover,
 the conditions \eqref{eqn:Tzero} and \eqref{eqn:Tpatching} are satisfied
 when $\operatorname{Re}\lambda \gg |\operatorname{Re}\nu|$.  
\end{enumerate}



All the equations concerning 
$
{\mathcal{S}}ol({\mathbb{R}}^n; V_{\lambda, \delta}, 
 W_{\nu, \varepsilon})
$
 depend holomorphically
 on $(\lambda, \nu)$ in the entire ${\mathbb{C}}^2$.  
On the other hand,
 for $\gamma \in \{\pm \}$, 
 the properties (1) and (2) tell
 that the pair 
$
   ( (\Atcal \lambda \nu {\gamma} {V,W})_{\infty}, 
       \Atcal \lambda \nu {\gamma} {V,W}
   )
$
 depends holomorphically on $(\lambda, \nu)$ in the entire ${\mathbb{C}}^2$.  
Hence the property (3) holds 
 in the entire $(\lambda, \nu)\in {\mathbb{C}}^2$
 by analytic continuation.  
In turn, 
 Proposition \ref{prop:Tpair} implies
 that the pair
$
   ((\Atcal \lambda \nu {\gamma} {V,W})_{\infty}, 
\Atcal \lambda \nu {\gamma} {V,W})
$
 gives an element of $({\mathcal{D}}'(G/P, {\mathcal{V}}_{\lambda,\delta}^{\ast}) \otimes W_{\nu,\varepsilon})^{\Delta(P')}$
 for all $(\lambda, \nu)\in {\mathbb{C}}^2$, 
 and we have completed the proof of Theorem \ref{thm:152389}.  
\end{proof}

%%%%%%%%%%%%%%%%%%%%%%%%%%%%%%%%%%%%%%%%%%%%%%%%%%%%%%%%%%%
\subsection{Existence condition for regular symmetry breaking operators :
Proof of Theorem \ref{thm:regexist}}
\label{subsec:regexist}
%%%%%%%%%%%%%%%%%%%%%%%%%%%%%%%%%%%%%%%%%%%%%%%%%%%%%%%%%%%%
In Theorem \ref{thm:152389}, 
 we have assumed 
 $[V:W] \ne 0$ for the construction 
 of symmetry breaking operators.  
In this section 
 we complete the proof of Theorem \ref{thm:regexist}, 
 which asserts
 that the condition $[V:W] \ne 0$ is necessary
 and sufficient for the existence
 of regular symmetry breaking operators.  



Suppose $[V:W]\ne 0$.  
Let $\Atbb \lambda \nu {\delta\varepsilon}{V,W} \colon 
 I_{\delta}(V,\lambda) \to J_{\varepsilon}(W,\nu)$
 be the normalized symmetry breaking operator
 which is obtained by the analytic continuation
 of the integral operator
 in Section \ref{subsec:holoAVW}.  
We study the support
 of its distribution kernel 
 $\Atcal \lambda \nu {\delta\varepsilon}{V,W}$.  
We define subsets $U_{+}^{{\operatorname{reg}}}$ and $U_{-}^{{\operatorname{reg}}}$in ${\mathbb{C}}^2$ by
\begin{align}
\label{eqn:Ureg+}
U_{+}^{{\operatorname{reg}}}:=&
\{(\lambda,\nu) \in {\mathbb{C}}^2
:
 n-\lambda-\nu-1 \not\in 2 {\mathbb{N}}, \nu-\lambda \not\in 2 {\mathbb{N}}\}, 
\\
\label{eqn:Ureg-}
U_{-}^{{\operatorname{reg}}}:=&
\{(\lambda,\nu) \in {\mathbb{C}}^2
:
 n-\lambda-\nu-2 \not\in 2 {\mathbb{N}}, \nu-\lambda-1 \not\in 2 {\mathbb{N}}\}.  
\end{align}
\begin{proposition}
\label{prop:172425}
Suppose $V \in \widehat{O(n)}$ and $W \in \widehat{O(n-1)}$
 satisfy $[V:W]\ne 0$.  
Let $\delta, \varepsilon \in \{\pm\}$.  
Then $\Atbb \lambda \nu{\delta\varepsilon}{V,W}$
 is a nonzero 
\index{B}{regularsymmetrybreakingoperator@regular symmetry breaking operator}
{\it{regular symmetry breaking operator}}
 in the sense of Definition \ref{def:regSBO}
 for all $(\lambda,\nu) \in U_{\delta \varepsilon}^{{\operatorname{reg}}}$.  
\end{proposition}
\begin{proof}[Proof of Proposition \ref{prop:172425}]
As in Proposition \ref{prop:Tpair},
 the distribution kernel of the operator $\Atbb \lambda \nu {\delta\varepsilon}{V,W}$ can be expressed by 
 a pair $((\Atcal \lambda \nu{\delta\varepsilon}{V,W})_{\infty}, \Atcal \lambda \nu{\delta\varepsilon}{V,W})$
 of ${\operatorname{Hom}}_{\mathbb{C}}(V,W)$-valued distributions
 on ${\mathbb{R}}^n$
 corresponding to the open covering
 $G/P=N_+ w P/P \cup N_-P /P$.  
Then it suffices to show 
 ${\operatorname{Supp}}(\Atcal \lambda \nu{\delta\varepsilon}{V,W})_{\infty}
 ={\mathbb{R}}^n$
 for $(\lambda,\nu)\in U_{\delta\varepsilon}^{{\operatorname{reg}}}$.  
If $(\lambda,\nu) \in U_{\delta\varepsilon}^{{\operatorname{reg}}}$, 
then $(\lambda,\nu,\delta,\varepsilon)\not\in \Psising$, 
 and therefore $(\Atcal \lambda \nu {\delta\varepsilon}{V,W})_{\infty} \ne 0$ by 
 Lemma \ref{lem:ABzero}.  
Moreover,
 if $n-\lambda-\nu-1 \not\in 2{\mathbb{N}}$
 for $\delta \varepsilon =+$
 (or if $n-\lambda-\nu-2 \not\in 2{\mathbb{N}}$
 for $\delta \varepsilon =-$), 
 then we deduce ${\operatorname{Supp}}(\Atcal \lambda \nu {\delta\varepsilon}{V,W})_{\infty}={\mathbb{R}}^n$ from Lemma \ref{lem:Riesz}
 about the support of the Riesz distribution.  
Hence Proposition \ref{prop:172425} is proved.   
\end{proof}
\begin{definition}
[normalized regular symmetry breaking operator]
\label{def:AregSBO}
We shall say $\Atbb \lambda \nu {\delta\varepsilon}{V,W} \colon 
I_{\delta}(V,\lambda) \to J_{\varepsilon}(W,\nu)$ is a holomorphic family
 of the normalized
\index{B}{genericallyregularsymmetrybreakingoperator@generically regular symmetry breaking operator|textbf}
 {\it{(generically) regular symmetry breaking operators}}.  
For simplicity,
 we also call it a holomorphic family
 of the normalized 
\index{B}{regularsymmetrybreakingoperator@regular symmetry breaking operator|textbf}
 regular symmetry breaking operators
 by a little abuse of terminology.  
We are ready to complete the proof of Theorem \ref{thm:regexist}.  
\end{definition}
\begin{proof}[Proof of Theorem \ref{thm:regexist}]
The implication (i) $\Rightarrow$ (iii) follows from the explicit construction
 of the (normalized) regular symmetry breaking operators
 $\Atbb \lambda \nu {\pm} {V,W}$
 in Theorem \ref{thm:152389}, 
 and from Proposition \ref{prop:172425}.  



(iii) $\Rightarrow$ (ii) Clear.  



Let us prove the implication (ii) $\Rightarrow$ (i).  
We use the notation as in Section \ref{subsec:Xi}
 which is adopted from \cite[Chap.~5] {sbon}.  
Then there exists a unique open orbit of $P'$ on $G/P$, 
 and the isotropy subgroup
 at 
\index{A}{qplusvector@$q_+ ={}^{t}(0,\cdots,0,1,1)$}
$[q_+] =[{}^{t}\!(0,\cdots,0,1,1)]\in \Xi/{\mathbb{R}}^{\times} \simeq G/P$
 is given by 
\[
   \{\begin{pmatrix} 1 & & & \\ & B & & \\ & & 1 & \\ & & & 1\end{pmatrix}:
 B \in O(n-1)\}
 \simeq O(n-1).  
\]
Then the implication (ii) $\Rightarrow$ (i) follows from
 the necessary condition
 for the existence of regular symmetry breaking operators proved in 
 \cite[Prop.~3.5]{sbon}.  



Thus Theorem \ref{thm:regexist} is proved.  
\end{proof}


%%%%%%%%%%%%%%%%%%%%%%%%%%%%%%%%%%%%%%%%%%%%%%%%%%%%%%%%%%%
\subsection{Zeros of $\Atbb \lambda \nu {\pm} {V,W}$ : Proof of Theorem \ref{thm:1532113}}
\label{subsec:1532113}
%%%%%%%%%%%%%%%%%%%%%%%%%%%%%%%%%%%%%%%%%%%%%%%%%%%%%%%%%%%%
This section discusses the zeros
 of the analytic continuation of the symmetry breaking operator
$
   \Atbb \lambda \nu {\gamma} {V,W}
   \colon
   I_{\delta}(V,\lambda) \to J_{\varepsilon}(W,\nu)  
$
 with $\delta \varepsilon =\gamma$.  
\begin{proof}
[Proof of Theorem \ref{thm:1532113}]
(1)\enspace
Let 
\index{A}{Nsigma@$N(\sigma)$}
$
N:=N(\sigma)
$
 as in Definition \ref{def:Nsigma}.  
We first observe that 
\begin{align*}
(\lambda-N, \nu+N) \in L_{\operatorname{even}}\quad
&
\text{ if $(\lambda, \nu) \in L_{\operatorname{even}}$
 and $\nu \le -N$,}
\\
(\lambda-N-1, \nu+N) \in L_{\operatorname{even}}\quad
&
\text{ if $(\lambda, \nu) \in L_{\operatorname{odd}}$
 and $\nu \le -N$.}
\end{align*}
Then the scalar-valued distributions
 $\Atcal {\lambda-N}{\nu+N}{+}{}$
 and $\Atcal {\lambda-N-1}{\nu+N}{+}{}$
 vanish, 
 respectively
 by \cite[Thm.~8.1]{sbon}.  
By Lemma \ref{lem:152392}, 
 the $\operatorname{Hom}_{\mathbb{C}}(V,W)$-valued distributions
 $p_{+,N}(\lambda,\nu)\Atcal \lambda \nu {+} {V,W}$
 and $p_{-,N}(\lambda,\nu)\Atcal \lambda \nu {-} {V,W}$ vanish, 
respectively,
 because the multiplication of distributions
 by the polynomial $g^{V,W}(x,x_n)$
 is well-defined.  
Since $p_{+,N}(\lambda,\nu) \ne 0$
 for 
\index{A}{Leven@$L_{\operatorname{even}}$}
$(\lambda,\nu) \in L_{\operatorname{even}}$
 and $p_{-,N}(\lambda,\nu) \ne 0$
 for 
\index{A}{Lodd@$L_{\operatorname{odd}}$}
$(\lambda,\nu) \in L_{\operatorname{odd}}$, 
 the first assertion follows from 
 Proposition \ref{prop:Tpair} (2).  
\par\noindent
(2)\enspace
If the symmetry breaking operator $\Atbb \lambda \nu {\gamma} {V,W}$ vanishes, 
 then its distribution kernel is zero, 
 and in particular,
 $(\Atcal \lambda \nu {\gamma} {V,W})_{\infty}=0$
 (see Proposition \ref{prop:Tpair}).  
This implies $\nu-\lambda \in 2{\mathbb{N}}$
 for $\gamma=+$, 
 and $\nu-\lambda \in 2 {\mathbb{N}}+1$
 for $\gamma=-$, 
 owing to Lemma \ref{lem:ABzero}. 
Hence Theorem \ref{thm:1532113} is proved.   
\end{proof}

%%%%%%%%%%%%%%%%%%%%%%%%%%%%%%%%%%%%%%%%%%%%%%%%%%%%%%%%%%%%%%%%%%
\subsection{Generic multiplicity-one theorem:
 Proof of Theorem \ref{thm:unique}}
\label{subsec:generic}
\index{B}{genericmultiplicityonetheorem@generic multiplicity-one theorem}
%%%%%%%%%%%%%%%%%%%%%%%%%%%%%%%%%%%%%%%%%%%%%%%%%%%%%%%%%%%%%%%%%
We recall from \eqref{eqn:Psisp}
 the definition of \lq\lq{generic parameter}\rq\rq\
 \eqref{eqn:nlgen}
 that $(\lambda, \nu, \delta, \varepsilon) \not \in
 \Psising$
if and only if
\begin{equation*}
\text{
  $\nu - \lambda \not \in 2{\mathbb{N}}$
 for $\delta \varepsilon = +$;
\quad
  $\nu - \lambda \not \in 2{\mathbb{N}}+1$
 for $\delta \varepsilon = -$.
}
\end{equation*}
We are ready to classify symmetry breaking operators
 for generic parameters.  
The main result
 of this section 
 is Theorem \ref{thm:genbasis}, from which Theorem \ref{thm:unique}
 follows.  
\begin{theorem}
[generic multiplicity-one theorem]
\label{thm:genbasis}
Suppose $(\sigma,V) \in \widehat{O(n)}$, 
 $(\tau,W) \in \widehat{O(n-1)}$
 with 
\index{A}{VWmult@$[V:W]$}
$
[V:W]\ne 0.  
$
Assume $(\lambda, \nu)\in {\mathbb{C}}^2$
 and $\delta, \varepsilon \in \{\pm \}$
 satisfy the 
\index{B}{genericparametercondition@generic parameter condition}
 generic parameter condition, 
 namely, 
 $(\lambda, \nu, \delta, \varepsilon) \not \in
 \Psising$.  
Then the normalized operator $\Atbb \lambda \nu {\delta \varepsilon} {V,W}$ is nonzero
 and is not a differential operator.  
Furthermore we have
\[
    {\operatorname{Hom}}_{G'}
    (I_{\delta}(V,\lambda)|_{G'}, J_{\varepsilon}(W,\nu))
    =
   {\mathbb{C}} \Atbb \lambda \nu {\delta \varepsilon} {V,W}.  
\]
\end{theorem}

\begin{proof}
By Theorem \ref{thm:152389}, 
 $\Atbb \lambda \nu {\pm}{V,W}$ is a symmetry breaking operator
 for all $\lambda, \nu \in {\mathbb{C}}$.  
The generic assumption on $(\lambda, \nu, \delta, \varepsilon)$
 implies $\Atbb \lambda \nu {\delta \varepsilon} {V,W} \ne 0$
 by Theorem \ref{thm:1532113} (2).  
On the other hand,
 by Theorem \ref{thm:SDone} and Corollary \ref{cor:160150upper},
 we see that $\Atbb \lambda \nu {\delta \varepsilon} {V,W}$ is not 
 a differential operator
 and $\dim_{{\mathbb{C}}}{\operatorname{Hom}}_{G'}
    (I_{\delta}(V,\lambda)|_{G'}, J_{\varepsilon}(W,\nu)) \le 1$.  
Thus we have proved Theorem \ref{thm:genbasis}.  
\end{proof}

The generic multiplicity-one theorem 
 given in Theorem \ref{thm:unique} is the second statement
 of Theorem \ref{thm:genbasis}.  



\subsection{Lower estimate of the multiplicities}
\label{subsec:existSBO}

In this section
 we do not assume the generic parameter condition
 (Definition \ref{def:generic}), 
 and allow the case $(\lambda, \nu, \delta,\varepsilon) \in \Psising$.  
In this generality, 
 we give a lower estimate 
 of the dimension of the space of symmetry breaking operators.  
\begin{theorem}
\label{thm:existSBO}
Let $(\sigma, V) \in \widehat {O(n)}$
 and $(\tau,W) \in \widehat {O(n-1)}$
 satisfying 
$
[V:W]\ne 0.  
$
For any $\delta, \varepsilon \in \{ \pm \}$ 
 and $(\lambda, \nu)\in {\mathbb{C}}^2$, 
 we have
\[
   \dim_{\mathbb{C}}
   {\operatorname{Hom}}_{G'}(I_{\delta}(V,\lambda)|_{G'},J_{\varepsilon}(W,\nu))\ge 1.  
\]
\end{theorem}



We use a general technique from \cite[Lem.~11.10]{sbon}
 to prove that the multiplicity function
 is upper semicontinuous.  



As before, 
 we denote by 
$
   ((\Atcal \lambda \nu {\gamma} {V,W})_{\infty}, 
    \Atcal \lambda \nu {\gamma} {V,W})
$ 
 the pair of ${\operatorname{Hom}}_{\mathbb{C}}(V,W)$-valued distributions
 on ${\mathbb{R}}^n$
that represents the symmetry breaking operator $\Atbb \lambda \nu {\gamma} {V,W}$
 via Proposition \ref{prop:Tpair}.  



We fix $(\lambda_0, \nu_0)\in {\mathbb{C}}^2$,
 and define ${\operatorname{Hom}}_{\mathbb{C}}(V,W)$-valued
 distributions on ${\mathbb{R}}^n$
 for $k, \ell \in {\mathbb{N}}$ as follows:
\begin{align*}
F_{k \ell}:=& \left.\frac{\partial^{k+\ell}}{\partial \lambda^k \partial \nu^\ell}\right|_{\substack{\lambda=\lambda_0 \\ \nu=\nu_0}}
          \Atcal \lambda \nu {\gamma}{V,W}, 
\\
(F_{k \ell})_{\infty}:=& \left.\frac{\partial^{k+\ell}}{\partial \lambda^k \partial \nu^\ell}\right|_{\substack{\lambda=\lambda_0 \\ \nu=\nu_0}}
          (\Atcal \lambda \nu {\gamma}{V,W})_{\infty}.   
\end{align*}

\begin{lemma}
\label{lem:upsemi}
Let $\gamma \in \{ \pm \}$ and $m$ a positive integer
 such that 
\[
   ((F_{k \ell})_{\infty}, F_{k \ell})=(0,0)
\quad\text{for all $(k, \ell)\in {\mathbb{N}}^2$ with $k+\ell<m$.}
\]  
Then for any $(k,\ell)$ with $k+\ell=m$, 
 the pair $((F_{k\ell})_{\infty}, F_{k\ell})$ defines a symmetry breaking operator
 $I_{\delta}(V,\lambda) \to J_{\varepsilon}(W,\nu)$
 for $(\delta, \varepsilon)$ with $\delta \varepsilon=\gamma$.  
\end{lemma}

\begin{proof}
Since both the equations \eqref{eqn:Tinfty}--\eqref{eqn:Tpatching}
 and the pairs
$
   ((\Atcal \lambda \nu {\gamma}{V,W})_{\infty}, 
    \Atcal \lambda \nu {\gamma}{V,W})
$
 satisfying \eqref{eqn:Tinfty}--\eqref{eqn:Tpatching}
 depend holomorphically on $(\lambda,\nu)$
 in the entire ${\mathbb{C}}^2$, 
 we can apply \cite[Lem.~11.10]{sbon} 
 to conclude that the pair $((F_{k \ell})_{\infty},F_{k \ell})$ satisfies
 \eqref{eqn:Tinfty}--\eqref{eqn:Tpatching}
 at $(\lambda, \nu)=(\lambda_0, \nu_0)$ 
 for any $(k, \ell)\in {\mathbb{N}}^2$ with $k+\ell=m$.  
Then $((F_{k \ell})_{\infty},F_{k \ell})$ gives an element
 in 
$
   {\operatorname{Hom}}_{G'}
   (I_{\delta}(V,\lambda_0)|_{G'},J_{\varepsilon}(W,\nu_0))
$
by Proposition \ref{prop:Tpair}.  
\end{proof}

\begin{definition}
\label{def:AVWder}
Suppose we are in the setting of Lemma \ref{lem:upsemi}.  
For $(k,\ell)$ with $k+\ell=m$
 and $\delta, \varepsilon \in \{ \pm \}$ with $\delta \varepsilon = \gamma$, 
 we denote by 
\index{A}{Ahtgln2p@$\frac{\partial^{k+l}}{\partial \lambda^k \partial \nu^l}
\mid_{\substack {\lambda=\lambda_0 \\ \nu = \nu_0}} \Atbb{\lambda}{\nu}{\gamma}{V,W}$|textbf}
\[
 \left. \frac{\partial^{k+\ell}}{\partial \lambda^k \partial \nu^\ell}
 \right|_{\substack {\lambda=\lambda_0 \\ \nu = \nu_0}} 
 \Atbb \lambda \nu \gamma {V, W}
 \in 
 {\operatorname{Hom}}_{G'}(I_{\delta}(V,\lambda_0)|_{G'}, J_{\varepsilon}(W,\nu_0)), 
\]
the symmetry breaking operator
 associated to the pair $((F_{k \ell})_{\infty}, F_{k \ell})$.  
\end{definition}

\begin{proof}
[Proof of Theorem \ref{thm:existSBO}]
Set $\gamma:=\delta \varepsilon$.  
Then the pair 
$
   ((\Atcal \lambda \nu {\gamma} {V,W})_{\infty}, 
    \Atcal \lambda \nu {\gamma} {V,W})
$
 of ${\operatorname{Hom}}_{\mathbb{C}}(V,W)$-valued distributions
 depends holomorphically 
 on $(\lambda, \nu)$ in the entire ${\mathbb{C}}^2$
 and satisfies \eqref{eqn:Tinfty}--\eqref{eqn:Tpatching}
 for all $(\lambda, \nu)\in {\mathbb{C}}^2$.  
Moreover, 
the pair 
$
   ((\Atcal \lambda \nu {\gamma} {V,W})_{\infty}, 
    \Atcal \lambda \nu {\gamma} {V,W})
$
 is nonzero
 as far as $\nu-\lambda \not \in {\mathbb{N}}$
 by Lemma \ref{lem:ABzero}.  
This implies that, 
 given $(\lambda_0,\nu_0) \in {\mathbb{C}}^2$, 
 there exists $(k,\ell)\in {\mathbb{N}}^2$
 for which $((F_{k \ell})_{\infty},F_{k \ell})$ is nonzero. 
Take $(k, \ell) \in {\mathbb{N}}^2$
 such that $k+\ell$ attains the minimum
 among all $(k,\ell)$
 for which the pair $((F_{k \ell})_{\infty},F_{k \ell})$ is nonzero.  
By Lemma \ref{lem:upsemi}, 
$
 \left.\frac{\partial^{k+\ell}}{\partial \lambda^k \partial \nu^\ell}\right|_{\substack {\lambda=\lambda_0 \\ \nu = \nu_0}} 
 \Atbb {\lambda} {\nu} \gamma {V, W}
$
 is a symmetry breaking operator.  
\end{proof}



\subsection{Renormalization of symmetry breaking operators
 $\Atbb \lambda \nu \gamma {V,W}$}
\label{subsec:exAVW}

In this section we construct a nonzero symmetry breaking operator
 $\Attbb {\lambda_0} {\nu_0} \gamma {V, W}$
 by \lq\lq{renormalization}\rq\rq\
 when $\Atbb {\lambda_0} {\nu_0} \gamma {V, W}=0$.  
We shall also prove
 that the renormalized operator is {\it{not}} a differential operator.  
The main results are stated
 in Theorem \ref{thm:170340}.  

%%%%%%%%%%%%%%%%%%%%%%%%%%%%%%%%%%%%%%%%%%%%%%%%%%%%%%%%%%%%%%%%%%%
\subsubsection{Expansion of $\Atbb {\lambda} {\nu} \gamma {V, W}$
 along $\nu=\text{constant}$}
\label{subsec:expand}
%%%%%%%%%%%%%%%%%%%%%%%%%%%%%%%%%%%%%%%%%%%%%%%%%%%%%%%%%%%%%%%%%%%
We fix $\gamma \in \{\pm\}$ and $(\lambda_0, \nu_0)\in {\mathbb{C}}^2$
 such that
\[
  \nu_0 - \lambda_0
  =
  \begin{cases}
  2\ell \qquad &\text{for $\gamma=+$, }
\\
  2\ell+1 \qquad &\text{for $\gamma=-$, }
  \end{cases}
\]
with $\ell \in {\mathbb{N}}$.  
For every $(\sigma, V) \in \widehat{O(n)}$ and $(\tau, W) \in \widehat{O(n-1)}$, the distribution kernel $\Atcal \lambda \nu \gamma {V,W}$
 of the symmetry breaking operator
 $\Atbb \lambda \nu \gamma {V,W}$ is
 a ${\operatorname{Hom}}_{\mathbb{C}}(V,W)$-valued
 distribution on ${\mathbb{R}}^n$
 that depend holomorphically on $(\lambda,\nu) \in {\mathbb{C}}^2$
 by Theorem \ref{thm:152389}.  
We fix $\nu=\nu_0$ and expand $\Atcal \lambda {\nu_0} \gamma {V,W}$
 with respect to $\lambda$ near $\lambda=\lambda_0$
 as 
\begin{equation}
\label{eqn:Anear0}
   \Atcal \lambda {\nu_0} \gamma {V,W}
   =
   F_0 + (\lambda-\lambda_0) F_1 + (\lambda-\lambda_0)^2 F_2 + \cdots
\end{equation}
 with ${\operatorname{Hom}}_{\mathbb{C}}(V,W)$-valued
 distributions $F_0$, $F_1$, $F_2, \cdots$
 on ${\mathbb{R}}^n$.  
By definition, 
\[
\text{
$\Atcal {\lambda_0} {\nu_0} \gamma {V,W} \ne 0$
 if and only if 
$F_0 \ne 0$.
}
\]



For the next term $F_1$, 
 we have the following two equivalent expressions:
\begin{equation}
\label{eqn:F1}
F_1
=
\lim_{\lambda \to \lambda_0}\frac 1 {\lambda-\lambda_0}
(\Atcal \lambda {\nu_0} \gamma {V,W} - \Atcal {\lambda_0} {\nu_0} {\gamma} {V,W}), 
\end{equation}
 and 
\begin{equation}
\label{eqn:F2}
F_1
=
\left. \frac{\partial}{\partial \lambda}\right|_{\lambda=\lambda_0}
 \Atcal \lambda {\nu_0} \gamma {V,W}.  
\end{equation}



\subsubsection{
Renormalized regular symmetry breaking operator
 $\Attbb \lambda \nu \gamma {V,W}$}
\label{subsec:AVW}
 \index{B}{renormalized regular symmetry breaking operator@regular symmetry breaking operator, renormalized---|textbf}

We consider the following renormalized operators
\begin{alignat}{2}
\label{eqn:rAVW+}
  \Attbb \lambda \nu + {V,W}:=&\Gamma(\frac{\lambda-\nu}{2}) \Atbb \lambda \nu + {V,W}
  \qquad
  &&\text{for $\nu-\lambda \not \in 2 {\mathbb{N}}$, }
\\
\label{eqn:rAVW-}
  \Attbb \lambda \nu - {V,W}:=&\Gamma(\frac{\lambda-\nu+1}{2}) \Atbb \lambda \nu - {V,W}
  \qquad
  &&\text{for $\nu-\lambda \not \in 2 {\mathbb{N}}+1$.  }
\end{alignat}
Since $\Atbb \lambda \nu {\gamma} {V,W}$ depend holomorphically 
 on $(\lambda, \nu)$ in ${\mathbb{C}}^2$, 
\index{A}{Ahttgln0@$\Attbb {\lambda}{\nu}{\pm}{V,W}$|textbf}
 $\Attbb \lambda \nu \gamma {V,W}$ are obviously well-defined 
 as symmetry breaking operators
 $I_{\delta}(V,\lambda)\to J_{\varepsilon}(W,\nu)$
 if $\gamma = \delta \varepsilon$, 
 because the gamma factors do not have poles
 in the domain of definitions \eqref{eqn:rAVW+} and \eqref{eqn:rAVW-}.  



On the other hand, 
 Theorem \ref{thm:1532113} (2) implies 
 that the gamma factors in \eqref{eqn:rAVW+} or \eqref{eqn:rAVW-}
 have poles 
 if $\Atbb {\lambda_0} {\nu_0} \gamma {V,W}=0$.  
Nevertheless we shall see in Theorem \ref{thm:170340} below
 that the renormalization $\Attbb {\lambda_0} {\nu_0} \gamma {V,W}$
 still makes sense
 if $\Atbb {\lambda_0} {\nu_0} \gamma {V,W}=0$.  



\begin{theorem}
\label{thm:170340}
Suppose $[V:W]\ne 0$
 and let $({\lambda_0}, {\nu_0}) \in {\mathbb{C}}^2$
 such that $\Atbb {\lambda_0} {\nu_0} {\gamma} {V,W}=0$.  
\begin{enumerate}
\item[{\rm{(1)}}]
There exists $\ell \in {\mathbb{N}}$
 such that
\[
 \nu_0 - \lambda_0 =
 \begin{cases}
 2\ell \quad &\text{when $\gamma =+$, }
\\
 2\ell+1 \quad &\text{when $\gamma =-$. }
 \end{cases}
\]
\item[{\rm{(2)}}]
We set 
\begin{equation}
\label{eqn:Ader}
  \Attbb {\lambda_0} {\nu_0} {\gamma} {V,W}
  :=
  \frac{2(-1)^\ell}{\ell!}
  \left. \frac{\partial}{\partial \lambda}\right|_{\lambda=\lambda_0}
  \Atbb {\lambda} {\nu_0} {\gamma} {V,W}.  
\end{equation}
Then $\Attbb {\lambda_0} {\nu_0} {\gamma} {V,W}$
 gives a nonzero symmetry breaking operator from
 $I_{\delta}(V,\lambda)$ to $J_{\varepsilon}(W,\nu_0)$
 with $\delta \varepsilon = \gamma$.  
\item[{\rm{(3)}}]
We fix $\nu=\nu_0$.  
Then $\Attbb {\lambda} {\nu_0} {\gamma} {V,W}$ defined
 by \eqref{eqn:rAVW+} and \eqref{eqn:rAVW-}
 for $\lambda \ne \lambda_0$, 
 and by \eqref{eqn:Ader} for $\lambda = \lambda_0$, 
 is a family of symmetry breaking operators from 
 $I_{\delta}(V,\lambda)$ to $J_{\varepsilon}(W,\nu_0)$
 with $\delta \varepsilon =\gamma$ 
 that depend holomorphically
 on $\lambda$ 
 in the entire complex plane ${\mathbb{C}}$.  
In particular,
 we have 
\begin{equation}
\label{eqn:rSBOlim}
\Attbb {\lambda_0} {\nu_0} {\gamma} {V,W}
=
  \lim_{\lambda \to \lambda_0} \Attbb {\lambda} {\nu_0} {\gamma} {V,W}.  
\end{equation}
\item[{\rm{(4)}}]
$\Attbb {\lambda_0} {\nu_0} {\gamma} {V,W}$ is not a differential operator.  
\end{enumerate}
\end{theorem}

\begin{proof}
\begin{enumerate}
\item[(1)]
The assertion is already given in Theorem \ref{thm:1532113} (2).  
\item[(2)]
The assertion follows from Lemma \ref{lem:upsemi}.  
\item[(3)]
By the first statement,
 we see $(\lambda, \nu_0,\delta, \varepsilon)$ 
 with $\delta \varepsilon =\gamma$ satisfies the generic parameter condition 
\eqref{eqn:nlgen}
 if and only if $\lambda \ne \lambda_0$ and that
\[
  \Attbb {\lambda} {\nu_0} {\gamma} {V,W}
  =
  \Gamma(\frac {\lambda-\lambda_0}2 -\ell)
  \Atbb {\lambda} {\nu_0} {\gamma} {V,W}
\quad
  \text{if $\lambda \ne \lambda_0$.  }
\]



We expand the distribution 
$
   \Atcal \lambda {\nu_0} \gamma {V,W}
$
 as in \eqref{eqn:Anear0} near $\lambda=\lambda_0$.  
By the assumption 
 that $\Atbb {\lambda_0} {\nu_0} {\gamma} {V,W}=0$,
 it follows from the two expressions \eqref{eqn:F1} and \eqref{eqn:F2}
 of the second term $F_1$
 that  
\begin{align*}
  F_1 &= \lim_{\lambda \to \lambda_0} \frac 1 {\lambda-\lambda_0}
        \Atcal \lambda {\nu_0} \gamma{V,W}
      =\lim_{\lambda \to \lambda_0} \frac 1 {(\lambda-\lambda_0)
                                          \Gamma(\frac {\lambda-\lambda_0} 2-\ell)}
        \Attcal \lambda {\nu_0} \gamma{V,W}, 
\\
 F_1 &= \frac {(-1)^\ell \ell!}{2}
        \Attcal {\lambda_0} {\nu_0} \gamma{V,W}.  
\end{align*}
In light that $\lim_{\mu \to 0} \mu \Gamma (\frac {\mu}2- \ell)
 = \frac {2(-1)^\ell}{\ell !}$, 
 we obtain
\[
   \lim_{\lambda \to \lambda_0}\Attbb {\lambda} {\nu_0} {\gamma} {V,W}
   =
   \Attbb {\lambda_0} {\nu_0} {\gamma} {V,W}.  
\]
Since $\Attbb {\lambda} {\nu_0} {\gamma} {V,W}$ depends holomorphically
 on $\lambda$ in ${\mathbb{C}}\setminus \{\lambda_0\}$, 
 and since it is continuous at $\lambda=\lambda_0$, 
 $\Attbb {\lambda} {\nu_0} {\gamma} {V,W}$ is holomorphic in $\lambda$ 
 in the entire complex plane ${\mathbb{C}}$.  
\item[(4)]
Let $( (\Attcal {\lambda_0} {\nu_0} \gamma{V,W})_{\infty}, \Attcal {\lambda_0} {\nu_0} \gamma{V,W})$ be the pair
 of the distribution kernels 
 for $\Attbb {\lambda_0} {\nu_0} {\gamma} {V,W}$
 via Proposition \ref{prop:Tpair} (1).  
Then as in the above proof, 
 we have
\[
   (\Attcal {\lambda_0} {\nu_0} {\gamma} {V,W})_{\infty}
   =
    \lim_{\lambda \to \lambda_0}
    (\Attcal {\lambda} {\nu_0} {\gamma} {V,W})_{\infty}.  
\]
By Proposition \ref{prop:20150828-1231} (2), 
 the right-hand side is not zero.  
Hence $\Attbb {\lambda_0} {\nu_0} {\gamma} {V,W}$ is not 
 a differential operator
 by Proposition \ref{prop:Tpair} (3).  
\end{enumerate}
\end{proof}



We are ready to complete the proof
 of Theorem \ref{thm:VWSBO} (2-C).  
\begin{corollary}
\label{cor:Azero}
Let $\gamma \in \{\pm\}$.  
Suppose $\Atbb {\lambda} {\nu} {\gamma} {V,W}=0$.  
Then the following holds.  
\begin{equation}
\label{eqn:HAD}
  {\operatorname{Hom}}_{G'}(I_{\delta}(V,\lambda)|_{G'}, J_{\varepsilon}(W,\nu))  =
  {\mathbb{C}} \Attbb \lambda \nu {\delta\varepsilon}{V,W}
  \oplus
  {\operatorname{Diff}}_{G'}(I_{\delta}(V,\lambda)|_{G'}, 
 J_{\varepsilon}(W,\nu)).  
\end{equation}
\end{corollary}

\begin{proof}
[Proof of Corollary \ref{cor:Azero}]
By Theorem \ref{thm:170340}, 
 the renormalized operator $\Attbb \lambda \nu{\delta\varepsilon}{V,W}$
 is well-defined and nonzero.  
Moreover,
 the right-hand side of \eqref{eqn:HAD}
 is a direct sum, 
 and is contained in the left-hand side.  



Conversely,
 take any ${\mathbb{T}} \in  {\operatorname{Hom}}_{G'}(I_{\delta}(V,\lambda)|_{G'}, J_{\varepsilon}(W,\nu))$, 
 and write $({\mathcal{T}}_{\infty}, {\mathcal{T}})$
 for the corresponding pair
 of distribution kernels for ${\mathbb{T}}$ 
 via Proposition \ref{prop:Tpair}.  
Let $\gamma := \delta\varepsilon$.  
Then Proposition \ref{prop:20150828-1231} tells 
 that ${\mathcal{T}}_{\infty}$ must be proportional
 to $(\Attcal {\lambda} {\nu} {\gamma} {V,W})_{\infty}$, 
 namely, 
 ${\mathcal{T}}_{\infty}=C(\Attcal {\lambda} {\nu} {\gamma} {V,W})_{\infty}$
 for some $C \in {\mathbb{C}}$.  
This implies 
 that the distribution kernel 
 ${\mathcal{T}} - C \Attcal {\lambda} {\nu} {\gamma} {V,W}$
 of the symmetry breaking operator
 ${\mathbb{T}}-C \Attbb {\lambda} {\nu} {\gamma} {V,W}$
 is supported
 at the origin,
 and consequently
 ${\mathbb{T}}-C \Attbb {\lambda} {\nu} {\gamma} {V,W}$
 is a differential operator
 by Proposition \ref{prop:Tpair}.  
\end{proof}




