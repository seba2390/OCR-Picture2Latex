\newpage
%%%%%%%%%%%%%%%%%%%%%%%%%%%%%%%%%%%%%%%%%%%%%%%%%%%
\section{The Knapp--Stein intertwining operators revisited:
 Renormalization and $K$-spectrum}
\label{sec:psdetail}
%%%%%%%%%%%%%%%%%%%%%%%%%%%%%%%%%%%%%%%%%%%%%%%%%%%%

In this chapter,
 we discuss the classical Knapp--Stein operators,
 which may be viewed
 as a baby case of symmetry breaking operators
 ({\it{i.e.}},  $G=G'$ case).  
We determine the $(K,K)$-spectrum
 ($K$-spectrum, for short)
 of the matrix-valued Knapp--Stein operators
 $\Ttbb \lambda {n-\lambda} V \colon I_{\delta}(V,\lambda) \to I_{\delta}(V,n-\lambda)$, 
 see \eqref{eqn:KnappStein}
 in the case where $V=\Exterior^i({\mathbb{C}}^n)$.  
We also study the renormalization
of the operator $\Ttbb \lambda {n-\lambda} V$
 when it vanishes,
 see Section \ref{subsec:renormT}.  

%%%%%%%%%%%%%%%%%%%%%%%%%%%%%%%%%%%%%%%%%%%%%%%%%%%%%%%%%%%%%%%%
\subsection{Basic $K$-types in the compact picture}
\label{subsec:minKK}
%%%%%%%%%%%%%%%%%%%%%%%%%%%%%%%%%%%%%%%%%%%%%%%%%%%%%%%%%%%%%%%%
Let $(\mu,U)$ be an irreducible representation
 of a compact Lie group $K$, 
 and $(\sigma,V)$ that of a subgroup $M$.  
The classical Frobenius reciprocity tells
 that $\mu$ occurs in the induced representation 
 ${\operatorname{Ind}}_M^K \sigma$
 if and only if ${\operatorname{Hom}}_M(\mu|_M, \sigma) \ne \{0\}$.  
In this section we provide a concrete realization
 of $(\mu,U)$
 in the space $C^{\infty}(K/M, {\mathcal{V}})$
 of global sections 
 for the $K$-equivariant vector bundle 
 ${\mathcal{V}}=K \times_M V$
 which we will use later.  
\begin{lemma}
\label{lem:KM}
\begin{enumerate}
\item[{\rm{(1)}}]
Let $(\mu,U)$ be a finite-dimensional representation
 of a compact Lie group $K$.  
The left regular representation
 on $C^{\infty}(K,U)$ is defined by $f(\cdot) \mapsto f({\ell}^{-1} \cdot)$
 for $f \in C^{\infty}(K,U)$ and ${\ell} \in K$, 
 where we regard $U$ just as a vector space.  
By assigning to $u \in U$, 
 the function $f_u \colon K \to U$ is defined by 
 $f_u(k):=\mu(k)^{-1}u$.  
Then the $K$-module $U$ can be embedded 
 as a submodule of the left regular representation $C^{\infty}(K,U)$
 by
\[
  U \to C^{\infty}(K,U),
  \qquad
  u \mapsto f_u.  
\]
\item[{\rm{(2)}}]
Let $V$ be a vector space over ${\mathbb{C}}$, 
 and $\pr U V :U \to V$ a linear map.  
Then we have a $K$-homomorphism
\[
  U \to C^{\infty}(K,V),
  \qquad
  u \mapsto \pr U V \circ f_u.  
\]
\item[{\rm{(3)}}]
Suppose that $\sigma:M \to GL_{\mathbb{C}}(V)$ is
 a representation of a subgroup $M$ of $K$ 
 and that $\pr UV$ is an $M$-homomorphism.  
Then we have a well-defined $K$-homomorphism
\[
  U \to C^{\infty}(K/M,{\mathcal{V}}),
  \qquad
  u \mapsto \pr U V \circ f_u,  
\]
where we identify the space
 of smooth sections
 for ${\mathcal{V}}:=K \times_M V$
 over $K/M$ 
with the space of $M$-invariant elements
\[
  C^{\infty}(K,V)^M
  :=
  \{F \in C^{\infty}(K,V)
    :
    F(\cdot\, m)=\sigma(m)^{-1}F(\cdot)
   \quad
   \text{for all }
   m \in M\}.  
\]
\end{enumerate}
\end{lemma}
\begin{proof}
The detailed formulation of each statement gives a proof by itself.  
\end{proof}
Applying Lemma \ref{lem:KM}
 to differential forms
 on the sphere, 
 we obtain:
\begin{example}
\label{ex:KMi}
Let $K:=O(n+1)$, 
 and $\sigma$ be the $i$-th exterior tensor representation
 of the subgroup $M:=O(n)$ on $V:= \Exterior^i({\mathbb{C}}^{n})$.  
Then the vector bundle ${\mathcal{V}}=K \times_M V$ is identified
 with the $i$-th exterior tensor
 of the cotangent bundle
 of the $n$-sphere $S^n \simeq K/M$, 
 and we may identify $C^{\infty}(K,V)^M \simeq  C^{\infty}(K/M,{\mathcal{V}})$
 with the space ${\mathcal{E}}^i(S^n)$
 of differential $i$-forms on $S^n$.  
Suppose that $\mu$ is the $k$-th exterior tensor representation
 of $K=O(n+1)$ on $U:= \Exterior^k({\mathbb{C}}^{n+1})$.  
For $k=i$ or $i+1$, 
 the projection 
$
  \pr {k}i \colon \Exterior^{k}({\mathbb{C}}^{n+1}) \to \Exterior^i({\mathbb{C}}^n)
$, 
 see \eqref{eqn:Tii1} and \eqref{eqn:Tii2}, 
 is an $M$-homomorphism, 
 and therefore,
 Lemma \ref{eqn:fI} gives a concrete realization
 of the $K$-module $U=\Exterior^k({\mathbb{C}}^{n+1})$ in ${\mathcal{E}}^i(S^n)\simeq C^{\infty}(K,V)^M$ as below.  
Let $\{e_0, e_1, \cdots, e_n\}$ be the standard basis of ${\mathbb{C}}^{n+1}$, 
 and $\{e_{\mathcal{I}}: {\mathcal{I}} \in {\mathfrak {I}}_{n+1,k}\}$
 the standard basis of $\Exterior^k({\mathbb{C}}^{n+1})$.  



We treat the cases $k=i$ and $i+1$, 
 separately.  
In what follows,
 we use Convention \ref{conv:index}
 for the index set ${\mathfrak {I}}_{n+1,k}$.
See also Section \ref{subsec:psiN} 
 for minor determinant $(\det A)_{I J}$
 of $A \in M(N,{\mathbb{R}})$.  
\par\noindent
{\bf{Case 1.}}\enspace
Suppose $k=i$.  
Then 
\index{A}{01oneI@${\bf{1}}^{\mathcal{I}}$|textbf}
$
{\bf{1}}^{\mathcal{I}}:=\pr i i \circ f_{e_{\mathcal{I}}}
$ is a map given by 
\begin{equation}
\label{eqn:fI}
O(n+1) \to \Exterior^i({\mathbb{C}}^n), 
\qquad
  k \mapsto 
  {\bf{1}}^{\mathcal{I}}(k)=\sum_{J \in {\mathfrak {I}_{n,i}}}
  (\det k)_{{\mathcal{I}}J} e_J.  
\end{equation}
Thus ${\bf{1}}^{\mathcal{I}}$ is regarded as an element
 of 
$
   C^{\infty}(O(n+1),\Exterior^i({\mathbb{C}}^n))^{O(n)}\simeq {\mathcal{E}}^i(S^n).  
$
\end{example}

\par\noindent
{\bf{Case 2.}}\enspace
Suppose $k=i+1$.  
Then 
\index{A}{hI@$h^{\mathcal{I}}$|textbf}
$
h^{\mathcal{I}} := (-1)^i \pr {i+1} i \circ f_{e_{\mathcal{I}}}
$
 is a map given by 
\begin{equation}
\label{eqn:hJ}
O(n+1) \to \Exterior^i({\mathbb{C}}^n),
\quad
k \mapsto 
  h^{\mathcal{I}}(k) =\sum_{J \in {\mathfrak{I}}_{n,i}}
          (\det k)_{{\mathcal{I}},J \cup \{0\}}e_J, 
\end{equation}
which is again regarded as an element
 of ${\mathcal{E}}^i(S^n)$.  
We remark
 that the projection
\[
   \pr {i+1} j \colon
   \Exterior^{i+1}({\mathbb{C}}^{n+1}) \to \Exterior^i({\mathbb{C}}^n)
\]
is given by \lq\lq{removing}\rq\rq\
 $e_0$, 
 whereas the projection in \eqref{eqn:Tii2} was by \lq\lq{removing}\rq\rq\
 $e_n$.  
 

By Lemma \ref{lem:KM}, 
 we obtain injective $O(n+1)$-homomorphisms
\begin{alignat*}{3}
&\Exterior^{i}({\mathbb{C}}^{n+1})
&&\to 
{\mathcal{E}}^i(S^n), 
\qquad
&&e_{\mathcal{I}} \mapsto {\bf{1}}^{\mathcal{I}}, 
\\
&\Exterior^{i+1}({\mathbb{C}}^{n+1})
&&\to 
{\mathcal{E}}^i(S^n),  
\qquad
&&e_{\mathcal{I}} \mapsto h^{\mathcal{I}}.  
\end{alignat*}

%%%%%%%%%%%%%%%%%%%%%%%%%%%%%%%%%%%%%%%%%%%%%%%%%%%%%%%%%%%%%%%%
\subsection{$K$-picture and $N$-picture of principal series representations}
\label{subsec:psKN}
%%%%%%%%%%%%%%%%%%%%%%%%%%%%%%%%%%%%%%%%%%%%%%%%%%%%%%%%%%%%%%%%



Let $(\sigma, V) \in \widehat {O(n)}$, 
 $\delta \in \{\pm 1\}$, 
 and $\lambda \in {\mathbb{C}}$.  
We recall from Section \ref{subsec:smoothI}
 that the principal series representation
\[
 I_{\delta}(V,\lambda)
 =
 {\operatorname{Ind}}_P^G(V \otimes \delta \otimes {\mathbb{C}}_{\lambda})
\]
 of $G=O(n+1,1)$
 is realized on the Fr{\'e}chet space
 $C^{\infty}(G/P, {\mathcal{V}}_{\lambda,\delta})$
 of smooth sections for the homogeneous vector bundle
 $G \times_P V_{\lambda, \delta}$ 
over the real flag manifold $G/P$, 
 see \eqref{eqn:Vlmdbdle}. 

\subsubsection{Explicit $K$-finite vectors in the $N$-picture} 
In this subsection we review
 the 
\index{B}{Kpicture@$K$-picture}
$K$-picture and
\index{B}{Npicture@$N$-picture}
$N$-picture
 of the principal series representation $I_{\delta}(V,\lambda)$,
 and provide a concrete formula
 connecting the two pictures.  



As we saw in \eqref{eqn:iotaN}, 
 the noncompact picture ($N$-picture) of $I_{\delta}(V,\lambda)$
 is given by
\[
  \iota_N^{\ast} \colon 
  I_{\delta}(V,\lambda) \hookrightarrow C^{\infty}({\mathbb{R}}^n) \otimes V, 
\quad
  F \mapsto f(b):=F(n_-(b)), 
\]
 as the pull-back of sections
 via the coordinate map of the open Bruhat cell
$
  \iota_N \colon {\mathbb{R}}^n \hookrightarrow G/P, 
$
$b \mapsto n_-(b) \cdot o$, 
 where $n_- \colon {\mathbb{R}}^n \overset \sim \to N_-$
 is defined in \eqref{eqn:nbar}.  



Next,
 let $V_{\delta}$ denote the outer tensor product representation
 $V \boxtimes \delta$ of $M=O(n) \times O(1)$.  
Then the diffeomorphism
\index{A}{0iotaK@$\iota_K$|textbf}
 $\iota_{K} \colon K/M \overset \sim \to G/P$ induces an isomorphism
 $\iota_K^{\ast}({\mathcal{V}}_{\lambda,\delta}) \simeq K \times_M V_{\delta}$
 as $K$-equivariant vector bundles
 over $K/M$, 
 and hence $K$-isomorphisms
 between the space of sections:
\[
  \iota_K^{\ast} \colon
  I_{\delta}(V,\lambda)
  \overset \sim \to
  C^{\infty}(K/M, K \times_M V_{\delta})
  \simeq 
  (C^{\infty}(K) \otimes V_{\delta})^{M}, 
\]
which is referred to as the {\it{$K$-picture}}
 of $I_{\delta}(V,\lambda)$.  



The transform from the $K$-picture to the $N$-picture
 is given by 
\begin{equation}
\label{eqn:KtoN}
  \iota_{\lambda}^{\ast}:=\iota_{N}^{\ast} \circ (\iota_{K}^{\ast})^{-1}
  \colon 
  (C^{\infty}(K) \otimes V_{\delta})^M 
  \hookrightarrow 
  C^{\infty}({\mathbb{R}}^n) \otimes V.  
\end{equation}
Then the three realizations
 of the principal series representation
 $I_{\delta}(V,\lambda)$ of $G$ are summarized as below.  
\index{A}{0iotalmdd@$\iota_{\lambda}^{\ast}$}
$$
\xymatrix@R-=1.3pc@C+=0.9cm{
& C^{\infty}(G/P, {\mathcal{V}}_{\lambda,\delta})
\ar[lddd]
\ar[rddd]^{\;\; \iota_N^{\ast}}
&{} 
\\
\qquad\qquad\qquad\qquad\qquad{}_{\iota_K^{\ast}}
&
&
\\
&  
&{}
\\
\text{($K$-picture)}
\quad
(C^{\infty}(K) \otimes V_{\delta})^M 
\ar[rr]_{\hspace{25pt}\iota_{\lambda}^{\ast}}
&{}
&C^{\infty}({\mathbb{R}}^n) \otimes V
\quad
\text{($N$-picture)}
}
$$  



To compute $\iota_{\lambda}^{\ast}$, 
 we recall from Lemma \ref{lem:0.2}
 that the map
\[
     k \colon {\mathbb{R}}^n \to SO(n+1) \subset K =O(n+1) \times O(1), 
\]
 see \eqref{eqn:kb}, 
 induces the following commutative diagram:

\vskip 1pc
\[
\begin{xy}
(0,0) *{{\mathbb{R}}^n}="A",
(23,-1)*{\underset{n_-}{\overset \sim \longrightarrow} N_-
        \hookrightarrow G/P {\overset \sim \longleftarrow}}="D",
(45,0)*{K/M}="B",
(45,20)*{ K }="C",
\ar "C";"B"
\ar^k "A";"C" 
\end{xy}
\]



\begin{lemma}
\label{lem:fKN}
Suppose $F \in (C^{\infty}(K) \otimes V_{\delta})^M$.  
Then we have 
\begin{equation}
\label{eqn:fKN}
(\iota_{\lambda}^{\ast} F)(b)
=
(1+|b|^2)^{-\lambda}F(k(b))
\quad
\text{for all $b \in {\mathbb{R}}^n$}.  
\end{equation}
Here $k(b) \in SO(n+1)$ is viewed as an element of $K$
 on the right-hand side.  
\end{lemma}
\begin{proof}
We define $t \in {\mathbb{R}}$
 by $e^t=1+|b|^2$.  
It follows from Lemma \ref{lem:0.2}
 that 
\index{A}{n2@$n_-\colon {\mathbb{R}}^n \to N_-$}
\index{A}{kb@$k(b)$}
\begin{align*}
(\iota_{\lambda}^{\ast} F)(b)=({\iota_K^{\ast}}^{-1} F)(n_-(b))
=&({\iota_K^{\ast}}^{-1} F)
\left(
\begin{pmatrix}
k(b) & 0
\\
0& 1
\end{pmatrix}
e^{t H}
n_+(\frac{-b}{1+|b|^2})
\right)
\\
=&
(1+|b|^2)^{-\lambda}
F\left(
\begin{pmatrix}
k(b) & 0
\\
0& 1
\end{pmatrix}
\right). 
\end{align*}
Hence the lemma is verified.  
\end{proof}



%%%%%%%%%%%%%%%%%%%%%%%%%%%%%%%%%%%%%%%%%%%%%%%%%%%%%%%%%%%%%%%%
\subsubsection{Basic $K$-types in the $N$-picture}
\label{subsec:minKtype}
%%%%%%%%%%%%%%%%%%%%%%%%%%%%%%%%%%%%%%%%%%%%%%%%%%%%%%%%%%%%%%%%
We recall from the $K$-type formula
 (Lemma \ref{lem:KtypeIi})
 that the principal series representation
\index{A}{Ideltai@$I_{\delta}(i, \lambda)$}
$
I_{\delta}(i,\lambda)
$
 of $G=O(n+1,1)$ contains two 
\index{B}{basicKtype@basic $K$-type}
\lq\lq{basic $K$-types}\rq\rq\
 $\mub(i, \delta)=\Exterior^i({\mathbb{C}}^{n+1}) \boxtimes \delta$
 and $\mus(i,\delta)=\Exterior^{i+1}({\mathbb{C}}^{n+1}) \boxtimes (-\delta)$
 for $0 \le i \le n$.  


In this section,
 we write down explicit $K$-finite vectors
 belonging to $\mub(i,\delta)$ and $\mus(i,\delta)$
 in the noncompact picture.  



Let ${\bf{1}}^{\mathcal{I}}$ and $h^{\mathcal{I}}$ be the elements
 in ${\mathcal{E}}^i(S^n) \simeq C^{\infty}(O(n+1),V)^{O(n)}$
 constructed in Example \ref{ex:KMi}, 
 where we take $V$ to be $\Exterior^i({\mathbb{C}}^n)$.  
We note that the pair
\[
   (K,M) = (O(n+1) \times O(1), O(n) \times O(1))
\]
is not exactly the same with the pair
 $(O(n+1), O(n))$ 
 in Example \ref{ex:KMi}, 
however, 
 the diffeomorphism
 $O(n+1)/O(n) \overset \sim \to K/M$
 induces the following isomorphisms
\[
   {\mathcal{E}}^i(S^n)
  \simeq
   C^{\infty}(O(n+1) \otimes V)^{O(n)} \overset \sim \leftarrow
  C^{\infty}(K \otimes V_{\delta})^M.  
\]
Thus we may regard that 
 $\{{\bf{1}}^{\mathcal{I}}: {\mathcal{I}} \in {\mathfrak{I}}_{n+1,i}\}$
 is a basis of $\mub (i,\delta)$
 and 
 $\{h ^{\mathcal{I}}: {\mathcal{I}} \in {\mathfrak{I}}_{n+1,i+1}\}$
 is a basis of $\mus (i,\delta)$.  
Applying the map
 $\iota_{\lambda}^{\ast} \colon C^{\infty}(K \otimes V_{\delta})^M 
 \hookrightarrow 
 C^{\infty}({\mathbb{R}}^n) \otimes V$ 
 (see \eqref{eqn:fKN}), 
 we set
\begin{alignat*}{2}
{\bf{1}}_{\lambda}^{\mathcal{I}}:=& \iota_{\lambda}^{\ast} {\bf{1}}^{\mathcal{I}}
\qquad
&&\text{for ${\mathcal{I}} \in {\mathfrak{I}}_{n+1,i}$, }
\\
h_{\lambda}^{\mathcal{I}}:=& \iota_{\lambda}^{\ast} h^{\mathcal{I}}
\qquad
&&\text{for ${\mathcal{I}} \in {\mathfrak{I}}_{n+1,i+1}$.  }
\end{alignat*}
By Lemma \ref{lem:KM}
 and Example \ref{ex:KMi}, 
 we have shown the following.  


\begin{proposition}
[basic $K$-type $\mub$ and $\mus$]
\label{prop:minKN}
We define linear maps by 
\begin{alignat}{3}
\label{eqn:minlmd}
  \Exterior^i({\mathbb{C}}^{n+1})
  &\to 
  C^{\infty}({\mathbb{R}}^{n}, \Exterior^i({\mathbb{C}}^{n})), 
  \quad
  &&e_{\mathcal{I}} \mapsto {\bf{1}}_{\lambda}^{\mathcal{I}}
  \quad
  &&\text{for ${\mathcal{I}} \in {\mathfrak{I}}_{n+1,i}$}, 
\\
\Exterior^{i+1}({\mathbb{C}}^{n+1})
&\to 
C^{\infty}
({\mathbb{R}}^n,  
 \Exterior^{i}({\mathbb{C}}^{n}) 
), 
\quad
&&e_{\mathcal{I}} \mapsto 
h_{\lambda}^{\mathcal{I}}
\quad
  &&\text{for ${\mathcal{I}} \in {\mathfrak{I}}_{n+1,i+1}$}.  
\notag
\end{alignat}
Then, 
 for $\delta=\pm$, 
 the images give 
 the unique $K$-types 
$
   \mub(i,\delta)=\Exterior^i({\mathbb{C}}^{n+1}) \boxtimes \delta
$
 and 
$\mus(i,\delta)=\Exterior^{i+1}({\mathbb{C}}^{n+1}) \boxtimes (-\delta)$
 respectively, 
 of the principal series representation 
$I_{\delta}(i,\lambda)
=\operatorname{Ind}_P^G
  (\Exterior^i
  ({\mathbb{C}}^{n}) \otimes \delta \otimes {\mathbb{C}}_{\lambda})$
 of $G$ in the $N$-picture.  
\end{proposition}
An explicit formula 
 for ${\bf{1}}_{\lambda}^{\mathcal{I}}$
 and $h_{\lambda}^{\mathcal{I}}$
 is given as follows.  
\begin{lemma}
\label{lem:NKtype}
Let $S_{{\mathcal{I}} {\mathcal{J}}}(b)$
 be the quadratic polynomial 
 of $b=(b_1, \cdots, b_n)$ defined in \eqref{eqn:SIJ}.  
\begin{enumerate}
\item[{\rm{(1)}}]
Let $0 \le i \le n$.  
For ${\mathcal{I}} \in {\mathfrak{I}}_{n+1,i}$
 and $\lambda \in {\mathbb{C}}$, 
 we have 
\begin{align}
  {\bf{1}}_{\lambda}^{\mathcal{I}}(b)
  =& (1+|b|^2)^{-\lambda}
      \sum_{J \in {\mathfrak{I}}_{n,i}}
      \det \psi_{n+1}(1,b)_{{\mathcal{I}} J} e_J
\notag
\\
  =&-(1+|b|^2)^{-\lambda-1}
  \sum_{J \in {\mathfrak{I}}_{n,i}}
  S_{{\mathcal{I}}J}(1,b) e_J.  
\label{eqn:minI}
\end{align}

If $i=0$, 
 we regard ${\mathcal{I}}=\emptyset$
 and ${\mathbf{1}}_{\lambda}^{\emptyset} =(1+|b|^2)^{-\lambda}$
 (see \eqref{eqn:Sempty}).  
\item[{\rm{(2)}}]
Let $0 \le i \le n$.  
For ${\mathcal{I}} \in {\mathfrak{I}}_{n+1,i+1}$
 and $\lambda \in {\mathbb{C}}$, 
 we have 
\begin{align}
h_{\lambda}^{\mathcal{I}}(b)
=&
-(1+|b|^2)^{-\lambda}
      \sum_{J \in {\mathfrak{I}}_{n,i}}
      \det \psi_{n+1}(1,b)_{{\mathcal{I}}, J\cup\{0\}} e_J
\notag
\\
=&
(1+|b|^2)^{-\lambda-1}
\sum_{J \in {\mathfrak{I}}_{n,i}}
S_{{\mathcal{I}},J \cup \{0\}}(1,b)e_J.  
\label{eqn:hIlmd}
\end{align}
\end{enumerate}
\end{lemma}



We note that Lemma \ref{lem:NKtype} implies 
\begin{align}
\label{eqn:I10}
{\mathbf{1}}_{\lambda}^{\mathcal{I}}(0)
&=
 \begin{cases}
 e_{\mathcal{I}} \quad &\text{0 $\not\in {\mathcal{I}}$}, 
\\
 0 \quad &\text{0 $\in {\mathcal{I}}$}, 
 \end{cases} 
\\
\label{eqn:hI0}
h_{\lambda}^{\mathcal{I}}(0)
&=
\begin{cases}
e_{{\mathcal{I}} \setminus \{0\}}\qquad &0 \in {\mathcal{I}}, 
\\
0
\qquad &0 \notin {\mathcal{I}}.  
\end{cases}
\end{align}

\begin{proof}
[Proof of Lemma \ref{lem:NKtype}]
Suppose $b \in {\mathbb{R}}^n$, 
 and let $k(b) \in SO(n+1)$ be as defined 
 in \eqref{eqn:kb}.  
By \eqref{eqn:fI} and \eqref{eqn:hJ}, 
 respectively,
 the formula \eqref{eqn:fKN} of $\iota_{\lambda}^{\ast}$ tells
 that 
\index{A}{01oneIlmd@${\bf{1}}_{\lambda}^{\mathcal{I}}$|textbf}
\index{A}{hIlmd@$h_{\lambda}^{{\mathcal{I}}}$|textbf}

\begin{alignat*}{2}
{\bf{1}}_{\lambda}^{\mathcal{I}}(b)
&=(\iota_{\lambda}^{\ast} {\bf{1}}^{\mathcal{I}})(b)
&&=(1+|b|^2)^{-\lambda}{\bf{1}}^{\mathcal{I}}(k(b))
\\
&
&&=(1+|b|^2)^{-\lambda} \sum_{J \in {\mathfrak{I}}_{n,i}}
(\det k(b))_{{\mathcal{I}} J}e_J, 
\\
h_{\lambda}^{\mathcal{I}}(b)
&=(\iota_{\lambda}^{\ast} h^{\mathcal{I}})(b)
&&= (1+|b|^2)^{-\lambda} h^{\mathcal{I}}(k(b))
\\
&
&&= (1+|b|^2)^{-\lambda} \sum_{J \in {\mathfrak{I}}_{n,i}}
   (\det k(b))_{{\mathcal{I}}, J \cup \{0\}} e_J.  
\end{alignat*}



It follows from Lemma \ref{lem:psidet} (2)
 that, for ${\mathcal{I}},{\mathcal{J}} \subset \{0,1,\cdots,n\}$
 with 
 $\# {\mathcal{I}} = \# {\mathcal{J}}=i$, 
 the minor determinant
 of 
\index{A}{kb@$k(b)$}
$
k(b)
$ is given by 
\begin{equation}
\label{eqn:kbdet}
(\det k(b))_{{\mathcal{I}} {\mathcal{J}}}
=
-\varepsilon_{\mathcal{J}}(0) 
(\det \psi_{n+1}(1,b))_{{\mathcal{I}} {\mathcal{J}}} 
=
 \varepsilon_{\mathcal{J}} (0) \frac {S_{{\mathcal{I}} {\mathcal{J}}} (1,b)}{1 + |b|^2}, 
\end{equation}
where we set $\varepsilon_{\mathcal{J}} (0)=-1$
 for $0 \notin {\mathcal{J}}$
 and $\varepsilon_{\mathcal{J}} (0)=1$ for $0 \in {\mathcal{J}}$.  



Now the second formul{\ae} in Lemma \ref{lem:NKtype} 
 are also shown.  
\end{proof}

%%%%%%%%%%%%%%%%%%%%%%%%%%%%%%%%%%%%%%%%%%%%%%%%%
\subsection{Knapp--Stein intertwining operator}
\label{sec:KS}
%%%%%%%%%%%%%%%%%%%%%%%%%%%%%%%%%%%%%%%%%%%%%%%%%


In this section
 we summarize some basic results
 on the matrix-valued Knapp--Stein intertwining operators, 
 see \cite{KS, KS2}.  
In the general framework of symmetry breaking operators
 for the restriction $G \downarrow G'$, 
 this classical case may be thought of as a special case
 where $G=G'$, 
 and the proof is much easier than the general case 
 $G \supsetneqq G'$.  
Nevertheless, 
 we sketch a proof 
 of results which we need
 in other chapters.  


\subsubsection{Knapp--Stein intertwining operator}
\label{subsec:KerKS}

For $(\sigma,V)\in \widehat{O(n)}$, 
 $\delta, \varepsilon \in \{ \pm \}$
 and $\lambda,\nu \in {\mathbb{C}}$, 
 we consider intertwining operators between two principal series representations 
 $I_{\delta}(V,\lambda)$
 and $I_{\varepsilon}(V,\nu)$
 of $G=O(n+1,1)$.  
They are determined by distribution kernels,
 and Fact \ref{fact:kernel}
 (see \cite[Prop.~3.2]{sbon}) with $G=G'$ and $V=W$ 
 gives a linear isomorphism
\begin{equation}
\label{eqn:GGkernel}
  {\operatorname{Hom}}_G(I_{\delta}(V,\lambda), I_{\varepsilon}(V,\nu))
  \simeq
  ({\mathcal{D}}'(G/P, {\mathcal{V}}_{\lambda, \delta}^{\ast})
  \otimes 
  V_{\nu, \varepsilon})^{\Delta(P)}, 
\end{equation}
 where $P$ acts diagonally
 on the $(G \times P)$-module
 ${\mathcal{D}}'(G/P, {\mathcal{V}}_{\lambda,\delta}^{\ast})
  \otimes V_{\nu,\varepsilon}$.  
As in Proposition \ref{prop:Tpair} (2), 
 the restriction to the open Bruhat cell 
 determines invariant distributions 
 in the right-hand side, 
 and thus we have an injective homomorphism
\[
  ({\mathcal{D}}'(G/P, {\mathcal{V}}_{\lambda,\delta}^{\ast})
   \otimes 
   V_{\nu,\varepsilon})^{\Delta(P)}
  \hookrightarrow
  {\mathcal{D}}'({\mathbb{R}}^n)\otimes {\operatorname{End}}_{\mathbb{C}}(V), 
  \quad
  f \mapsto F(x):=f(n_-(x)), 
\]
where we have used the canonical isomorphism
 $V^{\vee} \otimes V \simeq {\operatorname{End}}_{\mathbb{C}}(V)$.  
Different from the case $G \supsetneqq G'$
 for symmetry breaking operators, 
 there are strong constraints
 on the parameter
 for the existence of nonzero elements
 in \eqref{eqn:GGkernel}.  
In fact,
 it follows readily from the $P$-invariance
 that $F|_{{\mathbb{R}}^n \setminus \{0\}}$ is nonzero
 only if $\nu=n-\lambda$, 
 and in this case it is proportional
 to $|x|^{2 \lambda-2n} \sigma(\psi_n(x))$, 
 where we recall from \eqref{eqn:psim} the definition
 of $\psi_n \colon {\mathbb{R}}^n \setminus \{0\} \to O(n)$.  
We normalize as 
\begin{equation}
\label{eqn:KT}
\Ttcal{\lambda}{n-\lambda}V(x)
:=
 \frac{1}{\Gamma(\lambda-\frac n 2)}
|x|^{2\lambda-2n} \sigma(\psi_n(x)).  
\end{equation}

\begin{remark}
The normalization of the Knapp--Stein operator
 is not unique,
 and different choices are useful
 for different purposes.  
See for example Knapp--Stein \cite{KS} or Langlands \cite{LLNM}.  
\end{remark}
With the normalization \eqref{eqn:KT},
 we now review the Knapp--Stein intertwining operators
 in this setting as follows.  



\begin{lemma}
[normalized Knapp--Stein operator]
\label{lem:nKS}
The distribution \eqref{eqn:KT} belongs
 to $L_{\operatorname{loc}}^1({\mathbb{R}}^n) \otimes {\operatorname{End}}_{\mathbb{C}}(V)$
 if $\operatorname{Re} \lambda \gg 0$, 
 and extends to an element of 
\index{A}{Vlndast@${\mathcal{V}}_{\lambda,\delta}^{\ast}$, dualizing bundle}
 $({\mathcal{D}}'(G/P, {\mathcal{V}}_{\lambda,\delta}^{\ast}) \otimes V_{n-\lambda, \delta})^{\Delta(P)}$.  
Furthermore,
 it has an analytic continuation
 to the entire $\lambda \in {\mathbb{C}}$.  
\end{lemma}
By definition, 
 the (normalized) 
\index{B}{KnappSteinoperator@Knapp--Stein operator|textbf}
Knapp--Stein intertwining operator
\index{A}{IdeltaV@$I_{\delta}(V, \lambda)$}
\index{A}{TVln@$\Ttbb \lambda{n-\lambda}{V}$, normalized Knapp--Stein intertwining operator|textbf}
\begin{equation}
\label{eqn:KnappStein}
  \Ttbb{\lambda}{n-\lambda} V
  :
  I_{\delta}(V,\lambda) \to I_{\delta}(V,n-\lambda) 
\end{equation}
is defined 
 in the $N$-picture
 of the principal series representation 
 by the formula
\[
(\Ttbb{\lambda}{n-\lambda} V f)(x)
 = \int_{{\mathbb{R}}^n} \Ttcal{\lambda}{n-\lambda} V (x-y) f(y) dy.  
\]



When $(\sigma,V)$ is the $i$-th exterior representation
 on $\Exterior^i({\mathbb{C}}^n)$,
 we write simply 
 $\Ttbb \lambda{n-\lambda}{i}$ and $\Ttcal \lambda{n-\lambda}{i}$
 for 
 the operator $\Ttbb \lambda{n-\lambda}{V}$
 and the distribution $\Ttcal \lambda{n-\lambda}{V}$, 
respectively.  



The Knapp--Stein operator \eqref{eqn:KnappStein} gives
 a continuous $G$-homomorphism 
 $I_{\delta}(i,\lambda) \to I_{\varepsilon}(j,\nu)$
  when $j=i$ 
 (and $\delta =\varepsilon$, $\nu=n-\lambda$).  
On the other hand,
 there exist $G$-intertwining operators
 $I_{\delta}(i,\lambda) \to I_{\varepsilon}(j,\nu)$
 also when $i \ne j$
 for special parameters.  
Like sporadic symmetry breaking operators
 ({\it{cf}}. Theorem \ref{thm:152347}),
 they are given by differential operators
 as follows.  
\begin{fact}
\label{fact:Di}
Suppose that $0 \le i \le n-1$.  
\begin{enumerate}
\item[{\rm{(1)}}]
We can identify $I_{(-1)^i}(i,i)$
 with the space ${\mathcal{E}}^i(S^n)$
 of differential $i$-forms
 endowed with the natural action of the conformal group 
 $G=O(n+1,1)$. 
\item[{\rm{(2)}}]
The exterior derivative
 $d \colon {\mathcal{E}}^i(S^n) \to {\mathcal{E}}^{i+1}(S^n)$
 induces a $G$-intertwining operator
\[
   D_i \colon I_{(-1)^i}(i,i) \rightarrow I_{(-1)^{i+1}}(i+1,i+1).  
\]
The kernel of $D_i$ is $\Pi_{i,(-1)^{i}}$, 
 and the image is $\Pi_{i+1,(-1)^{i+1}}$. 
\end{enumerate}
\end{fact}

This follows from \cite[Thm.~12.2]{KKP}.  
We note that the existence
 of such an intertwining operator is assured
 {\it{a priori}} by the composition series
 of the principal series representation
 (Theorem \ref{thm:LNM20}), 
 see also \cite{C}.  


\subsubsection{$K$-spectrum
 of the Knapp--Stein intertwining operator}
\label{subsec:KSspec}
\index{B}{Kspectrum@$K$-spectrum}

This section gives an explicit formula for the eigenvalues
 of the (normalized) Knapp--Stein intertwining operator
\index{A}{Tiln@$\Ttbb \lambda{n-\lambda}{i}$|textbf}
\begin{equation}
\label{eqn:KSii}
\Ttbb \lambda {n-\lambda}i :I_{\delta}(i,\lambda) \to I_{\delta}(i,n-\lambda)
\end{equation}
 on the basic $K$-types
 $\mub(i,\delta)$
 and $\mus(i,\delta)$
 (see \eqref{eqn:muflat} and \eqref{eqn:musharp}, 
respectively).  
For $0 \le i \le n$ and $\lambda \in {\mathbb{C}}$, 
 we set 
\begin{equation}
\label{eqn:specT}
  c^{\natural}(i,\lambda)
  =
  \frac{\pi^{\frac n2}}{\Gamma(\lambda+1)}
  \times
  \begin{cases}
  \lambda-i
  \qquad
  &\text{if $\natural = \flat$}, 
\\
  n-i-\lambda
  \qquad
  &\text{if $\natural = \sharp$}.  
  \end{cases}
\end{equation}

\begin{proposition}
\label{prop:TminK}
Suppose $0 \le i \le n$, 
 $\lambda \in {\mathbb{C}}$
 and $\delta \in \{ \pm \}$.  
Then the (normalized) Knapp--Stein intertwining operator
\index{A}{Tiln@$\Ttbb \lambda{n-\lambda}{i}$}
\index{A}{Ideltai@$I_{\delta}(i, \lambda)$}
\[
\Ttbb{\lambda}{n-\lambda}i
:
I_{\delta}(i,\lambda) \to I_{\delta}(i,n-\lambda)
\]
acts on the 
\index{B}{basicKtype@basic $K$-type}
basic $K$-types
\index{A}{0muflat@$\mub(i,\delta)$}
 $\mub (i,\delta)=\Exterior^i({\mathbb{C}}^{n+1}) \boxtimes \delta$
 and 
\index{A}{0musharp@$\mus(i,\delta)$}
 $\mus(i,\delta)=\Exterior^{i+1}({\mathbb{C}}^{n+1}) \boxtimes (-\delta)$
 as the scalar multiplication:
\[
  \Ttbb \lambda {n-\lambda}i  \circ \iota_{\lambda}^{\ast}
  =
  c^{\natural}(i,\lambda)\iota_{n-\lambda}^{\ast}
\quad
  \text{on $\mu^{\natural}(i,\delta)$
        for $\natural = \flat$ or $\sharp$}.  
\]
In other words, 
we have 
\index{A}{01oneIlmd@${\bf{1}}_{\lambda}^{\mathcal{I}}$}
\index{A}{hIlmd@$h_{\lambda}^{{\mathcal{I}}}$}
\begin{alignat*}{2}
\Ttbb \lambda {n-\lambda} i({\bf{1}}_{\lambda}^{\mathcal{I}})
=&
\frac{(\lambda-i)\pi^{\frac n 2}}{\Gamma(\lambda+1)}
{\bf{1}}_{n-\lambda}^{\mathcal{I}}  
\quad
&&\text{for all ${\mathcal{I}} \in {\mathfrak {I}}_{n+1,i}$}, 
\\
\Ttbb \lambda {n-\lambda}i
(h_{\lambda}^{\mathcal{I}})
=&
\frac{(n-i-\lambda) \pi^{\frac n 2}}{\Gamma(\lambda+1)}
h_{n-\lambda}^{\mathcal{I}}
\qquad
&&\text{for all ${\mathcal{I}} \in {\mathfrak{I}}_{n+1,i+1}$}.  
\end{alignat*}
\end{proposition}



\begin{remark}
Proposition \ref{prop:TminK}
 in the $i=0$ case for $\mub (i,\delta)$
 was proved in \cite[Prop.~4.6]{sbon}.  
\end{remark}
We will give a proof of Proposition \ref{prop:TminK}
 in Section \ref{subsec:Tspec}.  

We recall from Theorem \ref{thm:LNM20}
 that the composition series of $I_{\delta}(i,i)$ and $I_{\delta}(i,n-i)$
 are described by the following exact sequences of $G$-modules:
\begin{align*}
&0 \to \Pi_{i,\delta} \to I_{\delta}(i,i) \to \Pi_{i+1,-\delta} \to 0, 
\\
&0 \to \Pi_{i+1,-\delta} \to I_{\delta}(i,n-i) \to \Pi_{i,\delta} \to 0, 
\end{align*}
which do not split if $i \ne \frac n 2$.  
Thus Proposition \ref{prop:TminK} implies:
\begin{proposition}
\label{prop:Timage}
Suppose $G=O(n+1,1)$ and $i \ne \frac n 2$.  
Then the kernels and the images
 of the $G$-homomorphisms 
 $\Ttbb \lambda{n-\lambda}i \colon I_{\delta}(i,\lambda) \to I_{\delta}(i,n-\lambda)$
 for $\lambda=i$, 
 $n-i$ are given by  
\begin{alignat*}{2}
{\operatorname{Ker}}(\Ttbb i{n-i}i) 
&\simeq \Pi_{i,\delta}
&& \simeq {\operatorname{Image}}(\Ttbb {n-i}i i)
\\
{\operatorname{Image}}(\Ttbb i{n-i}i) 
&\simeq \Pi_{i+1,-\delta}
&&\simeq {\operatorname{Ker}}(\Ttbb {n-i}i i).  
\end{alignat*}
\end{proposition}

\subsubsection{Vanishing of the Knapp--Stein operator}

There are a few exceptional parameters
 $(i,\lambda)$ 
 for which $\Ttbb \lambda {n-\lambda} i$ vanishes:

\begin{proposition}
\label{prop:Tvanish}
Suppose $G=O(n+1,1)$, 
 $0 \le i \le n$, 
 and $\lambda \in {\mathbb{C}}$.  
Then the normalized Knapp--Stein intertwining operator 
$\Ttbb \lambda {n-\lambda}i$
 is zero
 if and only if 
 $\lambda=i=\frac n 2$.  
\end{proposition}



\begin{proof}
See \cite{xkresidue}.  
\end{proof}



A renormalization of the Knapp--Stein intertwining operator
 $\Ttbb \lambda {n-\lambda} i$
 for $n=2i$ will be discussed in Section \ref{subsec:renormT}.  

%%%%%%%%%%%%%%%%%%%%%%%%%%%%%%%%%%%%%%%%%%%%%%%%%%%%%%%%%%%%%%%%%
\subsubsection{Integration formula for the $(K,K)$-spectrum}
\label{subsec:Tspec}
%%%%%%%%%%%%%%%%%%%%%%%%%%%%%%%%%%%%%%%%%%%%%%%%%%%%%%%%%%%%%%%%%%%
In this subsection,
 we give a proof of Proposition \ref{prop:TminK}.  
Let $\natural = \flat$ or $\sharp$.  
Since the multiplicity
 of the $K$-type
 $\mu^{\natural}(i,\delta)$
 in the principal series representation $I_{\delta}(i,\lambda)$
 is one,
 there exists a constant $c^{\natural}(i,\lambda)$
 depending on $i$ and $\lambda$
 such that
\begin{equation}
\label{eqn:TcT}
\Ttbb \lambda{n-\lambda}i \circ \iota_{\lambda}^{\ast}
=
c^{\natural}(i,\lambda)
\iota_{n-\lambda}^{\ast}
\quad
\text{on $\mu^{\natural}(i,\delta)$}.  
\end{equation}
We shall show
 that the constants $c^{\natural}(i,\lambda)$
 in the equation \eqref{eqn:TcT}
 are given by the formul{\ae} \eqref{eqn:specT}.  
The first step is to give an integral formula 
 for the constants 
\index{A}{cyflat@$c^{\flat}(i,\lambda)$}
\index{A}{cyfsharp@$c^{\sharp}(i,\lambda)$}
$
c^{\natural}(i,\lambda)
$
for $\natural = \flat$ and $\sharp$:
\begin{lemma}
\label{lem:161783}
Suppose $0 \le i \le n$
 and $\lambda \in {\mathbb{C}}$
 with $\operatorname{Re} \lambda \gg 0$. 
Then we have  
\begin{align*}
  c^{\flat}(i,\lambda)
  &=
  \frac{1}{\Gamma(\lambda-\frac n 2)}
  \int_{{\mathbb{R}}^n} |b|^{2\lambda-2n}
                        (1+|b|^2)^{-\lambda} 
  \left(1-\frac{2}{|b|^2(1+|b|^2)}\sum_{k=1}^{i} b_k^2\right) d b,
\\
  c^{\flat}(i,\lambda) - c^{\sharp}(i,\lambda)
  &=
  \frac{2}{\Gamma(\lambda-\frac n 2)}
  \int_{{\mathbb{R}}^n} |b|^{2\lambda-2n+2}
                        (1+|b|^2)^{-\lambda-1} d b.  
\end{align*}
\end{lemma}

\begin{proof}
[Proof of Lemma \ref{lem:161783}]
We first consider \eqref{eqn:TcT} 
 for $\natural = \flat$.  
Then we have 
\[
  \Ttbb \lambda {n-\lambda} i ({\bf{1}}_{\lambda}^{\mathcal{I}})
= c^{\flat}(i,\lambda) {\bf{1}}_{n-\lambda}^{\mathcal{I}}
\qquad
\text{for all ${\mathcal{I}} \in {\mathfrak{I}}_{n+1,i}$.  }
\]
Take ${\mathcal{I}} \in {\mathfrak{I}}_{n+1,i}$
 such that $0 \not\in {\mathcal{I}}$.  
Then \eqref{eqn:I10} tells that 
\begin{equation}
\label{eqn:T1c}
(\Ttbb \lambda{n-\lambda}i {\bf{1}}_{\lambda}^{\mathcal{I}})
(0)
=
c^{\flat}(i,\lambda)e_{\mathcal{I}}.  
\end{equation}



Let us compute the left-hand side.  
In view of the distribution kernel \eqref{eqn:KT}
 of the normalized Knapp--Stein operator $\Ttbb \lambda {n-\lambda}i$, 
 for $\operatorname{Re}\lambda \gg 0$
 we have
\begin{equation*}
(\Ttbb \lambda {n-\lambda} i {\bf{1}}_{\lambda}^{\mathcal{I}})(0)
= \frac{1}{\Gamma(\lambda - \frac n 2)} \int_{\mathbb{R}^n}
   |-b|^{2 \lambda-2n} \sigma(\psi_n(-b)){\mathbf{1}}_{\lambda}^{\mathcal{I}}(b)db.  
\end{equation*}
By \eqref{eqn:exrep} and the formula \eqref{eqn:minI}
 of ${\bf{1}}_{\lambda}^{\mathcal{I}}(b)$, 
 the integrand amounts to 
\begin{equation*}
\sum_{J,J' \in {\mathfrak{I}}_{n,i}}
   |b|^{2 \lambda-2n} (1+|b|^2)^{-\lambda}
   (\det \psi_{n+1}(1,b))_{{\mathcal{I}} J}
   (\det \psi_n(b))_{J' J} e_{J'}.  
\end{equation*}
Comparing the coefficients
 of $e_{\mathcal{I}}$ in the both sides of \eqref{eqn:T1c}, 
 we get 
\[
   c^{\flat}(i,\lambda)
   =
   \frac{1}{\Gamma(\lambda-\frac n 2)}
   \int_{\mathbb{R}^n}
   |b|^{2 \lambda-2n} (1+|b|^2)^{-\lambda}
   g_{\mathcal{I}}(b)d b, 
\]
where we set 
\begin{equation}
   g_{\mathcal{I}}(b):=
   \sum_{J \in {\mathfrak{I}}_{n,i}}
   (\det \psi_{n+1} (1,b))_{{\mathcal{I}} J} 
   (\det \psi_n (b))_{{\mathcal{I}} J}
   =1-\frac{2 Q_{\mathcal{I}}(b)}{(1+|b|^2) |b|^2}.  
\label{eqn:Timinor}
\end{equation}
The second equality was proved as the minor summation formula
 in Proposition \ref{prop:msum} (4), 
 where we recall $Q_{\mathcal{I}}(b)=\sum_{l \in {\mathcal{I}}}b_l^2$.  
Therefore, 
 by taking ${\mathcal{I}}=\{1,2, \cdots, n\}$, 
 we get the first assertion
 of Lemma \ref{lem:161783}.  



Next,
 we consider \eqref{eqn:TcT} for $\natural =\sharp$.  
Then we have 
\[
  \Ttbb {\lambda}{n-\lambda}i (h_{\lambda}^{\mathcal{I}})
  =
  c^{\sharp}(i,\lambda) h_{n-\lambda}^{\mathcal{I}}
\quad
\text{for all ${\mathcal{I}} \in {\mathfrak {I}}_{n+1,i+1}$.}
\]  
Take $I \in {\mathfrak {I}}_{n,i}$, 
 and set ${\mathcal{I}}:=I \cup \{0\} \in {\mathfrak {I}}_{n+1,i+1}$.  
By \eqref{eqn:hI0}, 
 we have
\begin{equation}
\label{eqn:Thc}
\Ttbb {\lambda}{n-\lambda} i (h_{\lambda}^{\mathcal{I}})(0)
  =
  c^{\sharp}(i,\lambda) e_I.  
\end{equation}
By \eqref{eqn:KT}, 
 we have
\begin{equation*}
\Ttbb {\lambda}{n-\lambda}i (h_{\lambda}^{\mathcal{I}})(0)
  = \frac{1}{\Gamma(\lambda-\frac n 2)}
     \int_{\mathbb{R}^n} |-b|^{2 \lambda-2n} \sigma(\psi_n(-b)) h_{\lambda}^{\mathcal{I}} (b) d b.  
\end{equation*}
Comparing the coefficients of $e_{I}$ in the both sides of the equation \eqref{eqn:Thc}, 
 we get from \eqref{eqn:hIlmd} and \eqref{eqn:exrep}
\[
  c^{\sharp}(i,\lambda)
 =- \frac{1}{\Gamma(\lambda-\frac n 2)}
    \int_{\mathbb{R}^n} |b|^{2\lambda-2n} (1+|b|^2)^{-\lambda}
    g_I'(b)
        d b, 
\]
 where we set 
\[
g_{I}'(b):=\sum_{J \in {\mathfrak{I}}_{n,i}}
(\det \psi_{n+1}(1,b))_{I \cup \{0\}, J\cup \{0\}}
       (\det \psi_{n}(b))_{I J}.  
\]



We note 
 that 
\[
  \det \psi_{n+1}(1,b)_{I J}
  =
  \det \psi_{n}(b;\frac{2}{1+|b|^2})_{I J}
\]
if $I, J \in {\mathfrak{I}}_{n,i}$ is regarded as elements
 of ${\mathfrak{I}}_{n+1,i}$
 in the left-hand side.  
Then we have
\[
   g_I(b)+g_I'(b)=\frac{2|b|^2}{1+|b|^2}
\]
{}from Proposition \ref{prop:msum} (3), 
 and thus we get 
\[
 c^{\flat}(i,\lambda)-c^{\sharp}(i,\lambda)
 =\frac{2}{\Gamma(\lambda-\frac n 2)} 
   \int_{\mathbb{R}^n} |b|^{2\lambda+2-2n} (1+|b|^2)^{-\lambda-1} d b.  
\]
Now Lemma \ref{lem:161783} is proved.  
\end{proof}
The second step is to compute the integrals in Lemma \ref{lem:161783}.  
\begin{lemma}
\label{lem:cTcompute}
For $\operatorname{Re}\lambda \gg 0$, 
$c^{\flat}(i,\lambda)$ and $c^{\sharp}(i,\lambda)$ take 
the form \eqref{eqn:specT}.  
\end{lemma}
\begin{proof}
Let $B(\lambda,\nu)$ denote the Beta function.  
By the change of variables
 $r^2= \frac{x}{1-x}$, 
 we have
\begin{equation}
\label{eqn:Rabnew}
  \int_0^{\infty}
  r^a (1+r^2)^b d r 
  = 
  \frac 1 2
  \int_0^1 x^{\lambda-1} (1-x)^{\nu-1} d x
  = \frac 1 2 B(\lambda,\nu), 
\end{equation}
where $a = 2 \lambda-1$ and $b=-\lambda-\nu$.  
Then Lemma \ref{lem:161783}
 in the polar coordinates tells that 
\[
c^{\flat}(i,\lambda)
=
\frac{\operatorname{vol}(S^{n-1})}{2\Gamma(\lambda-\frac n2)}
(B(\lambda-\frac n2,\frac n 2) 
- \frac{2i}{n}B(\lambda- \frac n 2,\frac n 2+1))
\]
by \eqref{eqn:Rabnew} and by the following observation:
\[
   \int_{S^{n-1}} |\omega_i|^2 d \omega
   =
  \frac 1 n \operatorname{vol}(S^{n-1})
\quad
  (1 \le i \le n).  
\]
Since
$
\operatorname{vol}(S^{n-1})
=\frac{2 \pi^{\frac n 2}}{\Gamma(\frac n 2)}, 
$
we get the first statement.  



By the second formula of Lemma \ref{lem:161783}, 
 we have 
\begin{align*}
 c^{\flat}(i,\lambda)-c^{\sharp}(i,\lambda)
 =&\frac{1}{\Gamma(\lambda-\frac n 2)}
   {\operatorname{vol}}(S^{n-1}) B(\lambda-\frac n2, \frac n2 +1)
\\
=&\frac{(2\lambda-n) \pi^{\frac n 2}}
       {\Gamma(\lambda+1)}.  
\end{align*}
Thus the closed formula \eqref{eqn:specT}
 for $c^{\sharp}(i,\lambda)$ is also proved.  
\end{proof}

\begin{proof}
[Proof of Proposition \ref{prop:TminK}]
The assertion follows from Lemmas \ref{lem:161783} and \ref{lem:cTcompute}
 for $\operatorname{Re}\lambda\gg 0$.  
For general $\lambda \in {\mathbb{C}}$, 
Proposition holds by the analytic continuation.  
\end{proof}


%%%%%%%%%%%%%%%%%%%%%%%%%%%%%%%%%%%%%%%%%%%%%%%%%%%%%%%%%%%%%%%%%%%%%
\subsection{Renormalization of the Knapp--Stein intertwining operator}
\label{subsec:renormT}
%%%%%%%%%%%%%%%%%%%%%%%%%%%%%%%%%%%%%%%%%%%%%%%%%%%%%%%%%%%%%%%%%%%%%
Because of the vanishing of the normalized Knapp--Stein intertwining operators
 in the middle degree 
 when $n$ is even (Proposition \ref{prop:Tvanish}), 
 intertwining operators from $I_{\delta}(\frac n 2, \lambda)$ to $I_{\delta}(\frac n 2, n-\lambda)$ require special attention.  
In this case, 
we set $n=2m$ and renormalize the Knapp--Stein intertwining operator
 of $G=O(2m+1,1)$ at the middle degree by
\index{A}{TVttiln@$\Tttbb \lambda{n-\lambda}{\frac n 2}$, renormalized Knapp--Stein intertwining operator|textbf}
\begin{equation}
\label{eqn:Ttilde}
   \Tttbb \lambda {2m-\lambda}{m}
   :=
   \frac{1}{\lambda-m} \Ttbb \lambda {2m-\lambda}{m}.  
\end{equation}
Then $\Tttbb \lambda {2m-\lambda} {m} \colon I_{\delta}(m, \lambda) \to I_{\delta}(m, 2m-\lambda)$ depends
holomorphically in the entire $\lambda \in {\mathbb{C}}$,
 and is vanishing nowhere.  



If $\lambda=m$,
 then $\Tttbb \lambda {2m-\lambda} m$ acts
 as an endomorphism
 of $I_{\delta}(m,m)$.  
On the other hand,
 we know from Theorem \ref{thm:LNM20} (1)
 that the principal series representation $I_{\delta}(m, m)$
 decomposes into the direct sum
 of two irreducible 
\index{B}{temperedrep@tempered representation}
 tempered representations
 of $G$ as follows:
\[
   I_{\delta}(m, m)
   \simeq 
   I_{\delta}(m)^{\flat} \oplus I_{\delta}(m)^{\sharp}
   \equiv
   \Pi_{m, \delta} \oplus \Pi_{m+1, -\delta}.  
\]
\begin{lemma}
\label{lem:161745}
Let $n=2m$ and $G=O(2m+1,1)$.  
Then the 
\index{B}{KnappSteinoperatorrenorm@ Knapp--Stein operator, renormalized---|textbf}
 renormalized Knapp--Stein operator
 $\Tttbb {m}{m}{m}$
 acts on $I_{\delta}(m, m)\simeq \Pi_{m,\delta} \oplus \Pi_{m+1,-\delta}$
 as
\[
   \frac{\pi^{m}}{m!}
   ({\operatorname{id}}_{\Pi_{m, \delta}} 
    \oplus 
    (-{\operatorname{id}})_{\Pi_{m+1, -\delta}} ).  
\]
\end{lemma}
\begin{proof}
Since the irreducible $G$-module $\Pi_{m, \delta}$ is not isomorphic to 
the irreducible $G$-module $\Pi_{m+1, -\delta}$, 
 the renormalized Knapp--Stein intertwining operator $\Tttbb {m}{m}{m}$ acts on each irreducible summand
 by scalar multiplication.  
Therefore, 
 it is sufficient to find the scalars
 on specific $K$-types occurring 
 in each summand.  
By Proposition \ref{prop:TminK}, 
 the renormalized Knapp--Stein intertwining operator
 $\Tttbb {\lambda}{2m-\lambda}{m}$ acts on vectors
 that belong to the $K$-types
 $\mub (m, \delta)(\subset \Pi_{m,\delta})$
 and $\mus(m, \delta)(\subset \Pi_{m+1,-\delta})$
 by the scalars
\[
   \text{
   $\frac{1}{\lambda-m} \frac{(\lambda-m)\pi^{m}}{\Gamma(\lambda+1)}$
\quad
   and 
\quad
   $\frac{1}{\lambda-m} \frac{(2m-m- \lambda)\pi^m}{\Gamma(\lambda+1)}$, 
}
\]
respectively.  
Taking the limit as $\lambda$ tends to $m$, 
 we get the lemma.  
\end{proof}



\subsection{Kernel of the Knapp--Stein operator}
\label{subsec:KerKnSt}
In this section,
 we discuss the proper submodules
 of the principal series representation
 $I_{\delta}(i,\lambda)$ of $G=O(n+1,1)$
 at reducible points
 (see \eqref{eqn:redIilmd}).  

We consider the composition of the Knapp--Stein operators,
 $\Ttbb {n-\lambda} \lambda i \circ \Ttbb \lambda{n-\lambda} i
\in {\operatorname{End}}_{G} (I_{\delta}(i,\lambda))$.  
By Proposition \ref{prop:TminK}, 
 its $K$-spectrum on the basic $K$-type
 $\mub (i,\delta)$ is given as 
\[
   \Ttbb {n-\lambda} \lambda i \circ \Ttbb \lambda{n-\lambda} i
   ({\bf{1}}_{\lambda}^{\mathcal{I}})
   =
   \frac{(\lambda-i)(n-\lambda-i) \pi^n}
        {\Gamma(\lambda+1)\Gamma(n-\lambda+1)}
        ({\bf{1}}_{\lambda}^{\mathcal{I}})
\quad
\text{for all ${\mathcal{I}} \in {\mathfrak{I}}_{n+1,i}$}.  
\]
Since the principal series representation $I_{\delta}(i,\lambda)$
 is generically irreducible,
 we conclude
\begin{equation}
\label{eqn:TTc}
\Ttbb {n-\lambda} \lambda i \circ \Ttbb \lambda{n-\lambda} i
  = 
  \frac{(\lambda-i)(n-\lambda-i) \pi^n}
        {\Gamma(\lambda+1)\Gamma(n-\lambda+1)}
  {\operatorname{id}}
\quad
\text{on $I_{\delta}(i,\lambda)$}
\end{equation}
 for generic $\lambda$ by Schur's lemma, 
 and then for all $\lambda \in {\mathbb{C}}$
 by analytic continuation.  



\begin{lemma}
\label{lem:reducibleI}
Let $G=O(n+1,1)$, 
 $0 \le i \le n$, 
 and $\delta \in \{\pm\}$.  
Assume
\[
   \lambda \in \{i,n-i\} \cup (-{\mathbb{N}}_+) \cup (n + {\mathbb{N}}_+).  
\]
Then $I_{\delta}(i,\lambda)$ is reducible.  
\end{lemma}
\begin{proof}
If $(n,\lambda) = (2i,i)$, 
 we already know 
 that $I_{\delta}(i,\lambda)$ is reducible,
 see Lemma \ref{lem:161745}.  

Assume now $(n,\lambda) \ne (2i,i)$.  
Then Proposition \ref{prop:Tvanish} tells 
 that neither $\Ttbb {n-\lambda} \lambda i$
 nor $\Ttbb \lambda{n-\lambda} i$ vanishes.  
On the other hand, 
 by \eqref{eqn:TTc}, 
 the assumption on $\lambda$ implies 
\[
   \Ttbb {n-\lambda} \lambda i \circ \Ttbb \lambda{n-\lambda} i =0, 
\]
which shows that at least one of the $G$-modules $I_{\delta}(i,\lambda)$
 or $I_{\delta}(i,n-\lambda)$ is reducible.  
By Lemma \ref{lem:LNM27}, 
 we conclude 
 that both $I_{\delta}(i,\lambda)$
 and its contragredient representation $I_{\delta}(i,n-\lambda)$
 are reducible.  
\end{proof}
Lemma \ref{lem:reducibleI} gives an alternative proof 
 for the \lq\lq{if part}\rq\rq\
 of Proposition \ref{prop:redIilmd} (1).  


\begin{proposition}
\label{prop:KerKnSt}
Let $G=O(n+1,1)$, 
 $0 \le i \le n$, 
 $\delta \in \{\pm\}$, 
and $\lambda \in {\mathbb{C}}$.  
Assume further that $I_{\delta}(i,\lambda)$ is reducible,
 namely,
\[
  \lambda \in \{i,n-i\} \cup (-{\mathbb{N}}_+) \cup (n+{\mathbb{N}}_+).  
\]
\begin{enumerate}
\item[{\rm{(1)}}]
Suppose $(n,\lambda) \ne (2i,i)$.  
Then the unique proper submodule
 of $I_{\delta}(i,\lambda)$ is given 
 as the kernel of the Knapp--Stein operator
 $\Ttbb \lambda {n-\lambda} i \colon I_{\delta}(i,\lambda) \to I_{\delta}(i,n-\lambda)$.  
\item[{\rm{(2)}}]
Suppose $(n,\lambda) = (2i,i)$.  
Then $\Ttbb \lambda {n-i} i =0$, 
 and there are two proper submodules
 of $I_{\delta}(i,\lambda)$, 
 which are given 
 as the kernel of
 $\Tttbb i i i \pm \frac{\pi^i}{i!} {\operatorname{id}}
 \in {\operatorname{End}}_{G} (I_{\delta}(i,i))$
 where $\Tttbb i i i$ is the renormalized Knapp--Stein operator.  
\end{enumerate}
\end{proposition}
\begin{proof}
\begin{enumerate}
\item[(1)]
There is a unique irreducible submodule of $I_{\delta}(i,\lambda)$
 for the parameter $\lambda$
 under consideration.  
Hence ${\operatorname{Ker}} (\Ttbb \lambda{n-\lambda}i)$ 
 is the unique irreducible submodule
 by the proof 
 of Lemma \ref{lem:reducibleI}.  
\item[(2)]
This is already proved in Lemma \ref{lem:161745}.  
\end{enumerate}
\end{proof}

