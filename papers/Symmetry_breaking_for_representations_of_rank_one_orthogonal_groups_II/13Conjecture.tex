\newpage
%%%%%%%%%%%%%%%%%%%%%%%%%%%%%%%%%%%%%%%%%%%%%
\section{A conjecture: Symmetry breaking for irreducible representations
 with regular integral infinitesimal character}
\label{sec:conjecture}
%%%%%%%%%%%%%%%%%%%%%%%%%%%%%%%%%%%%%%%%%%%%%


We conjecture that Theorems \ref{thm:SBOvanish} and \ref{thm:SBOone}
 hold  in more generality. 
We  will formalize and explain  this conjecture in this chapter more precisely and provide some supporting evidence.

As before we assume
that $G=O(n+1,1)$ and $G'=O(n,1)$.

%%%%%%%%%%%%%%%%%%%%%%%%%

\subsection{Hasse sequences and standard sequences
 of irreducible representations with regular integral infinitesimal character
 and their Langlands parameters} 
\label{subsec:Hasseps}
\index{B}{regularintegralinfinitesimalcharacter@regular integral infinitesimal character}
%%%%%%%%%%%%%%%%%%%%%%%%%%


Before  stating the conjecture we define Hasse sequences and standard sequences  of irreducible representations, 
 and collect more information about the representations which occur in the Hasse and standard sequences.
In Chapter \ref{sec:aq} (Appendix I) we determine their $\theta$-stable parameters. 


\subsubsection{Definition of Hasse sequence
 and standard sequence}
\begin{definitiontheorem}
[Hasse sequence]
\label{def:UHasse}
Let $n=2m$ or $2m-1$.  
For every irreducible finite-dimensional representation $F$
 of the group $G=O(n+1,1)$,
 there exists uniquely
 a sequence 
\begin{center}
\begin{tabular}{ccccccccccc}
& &$U_0$&, &\dots   & , & $U_{m-1} $ &, & $U_{m}$ 
\end{tabular}
\end{center}
of irreducible admissible smooth representations 
\index{A}{UHasse@$U_i(F)$, Hasse sequence starting with $F$}
 $U_i\equiv U_i(F)$ of $G$ such that 
\begin{enumerate}
\item  $U_0 \simeq F$; 
\item  consecutive representations are composition factors of a principal series representation;
\item
$U_i$ $(0 \le i \le m)$ are pairwise inequivalent as $G$-modules.  
\end{enumerate}



We refer to the sequence 
\begin{center}
\begin{tabular}{ccccccccccc}
& &$U_0$&, &\dots   & , & $U_{m-1} $ &, & $U_{m}$ 
\end{tabular}
\end{center}
 as the 
\index{B}{Hassesequence@Hasse sequence|textbf}
{\it {Hasse sequence}}
 of irreducible representations
 starting with the finite-dimensional representation $U_0=F$.  
We shall write $U_j(F)$ for $U_j$
 if we emphasize the sequence $\{U_j(F)\}$
 starts with $U_0=F$.  
\end{definitiontheorem}

\begin{proof}
[Sketch of the proof]  
D.~Collingwood \cite[Chap.~6]{C} computed embeddings
 of irreducible Harish-Chandra modules
 into principal series representations
 for all connected simple groups of real rank one, 
 which allowed him to define a diagrammatic description 
 of irreducible representations
 with regular integral infinitesimal character 
 of the connected group
 $G_0=SO_0(n+1,1)$.
For the disconnected group $G=O(n+1,1)$,
 we can determine similarly the composition factors
 of principal series representations,
 as in Theorems \ref{thm:171426} and \ref{thm:171425} below
 (see Sections \ref{subsec:SO}--\ref{subsec:PiSO} in Appendix II
 for the relationship 
 between irreducible representations of the disconnected group $G=O(n+1,1)$
 and those of a normal subgroup of finite index).  
To show the existence and the uniqueness
 of the Hasse sequence, 
 we note that there exists uniquely a principal series representation
 that contains a given irreducible finite-dimensional representation 
$F$ 
 as a subrepresentation.  
Then there exists only one irreducible composition factor
 other than $F$, 
 which is defined to be $U_1$.  
Repeating this procedure, 
 we can find irreducible representations $U_2$, $U_3$, $\cdots$, 
 whence the existence and the uniqueness of the Hasse sequence is shown for the disconnected group $G=O(n+1,1)$.  
\end{proof}



As we have seen in Theorem \ref{thm:LNM20} (1)
 when $F$ is the trivial one-dimensional representation ${\bf{1}}$, 
 the representations $U_i$ and $U_{i+1}$
 in this sequence have different signatures.  
The standard sequence (Definition \ref{def:Pii})
 starting with ${\bf{1}}$ 
 has given an adjustment
 for the different signatures.  
Extending this definition for the sequence
 starting 
 with an {\it{arbitrary}} irreducible finite-dimensional representation $F$, 
  we define the standard sequence of irreducible representations
 starting with $F$ as follows:

\begin{definition}
[standard sequence]
\label{def:Hasse}
If 
\begin{center}
\begin{tabular}{ccccccccccc}
& &\ $U_0$&, &\dots   & , & $U_{m-1} $ &, & $U_{m}$ 
\end{tabular}
\end{center}
is the Hasse sequence
 starting with an irreducible finite-dimensional representation 
 $F$ of $G$, 
then we refer to 
\begin{center}
\begin{tabular}{ccccccccccc}
& ${\Pi}_0:=U_0$&, &\dots   & , 
& ${\Pi}_{m-1}:=U_{m-1}\otimes (\chi_{+-})^{m-1} $ &, 
& $\Pi_m:=U_{m}\otimes (\chi_{+-})^m$ 
\end{tabular}
\end{center}
as the
\index{B}{standardsequence@standard sequence|textbf}
 {\it{standard sequence}}
 of irreducible representations 
\index{A}{1PiideltaF@$\Pi_i(F)$, standard sequence starting with $F$|textbf}
${\Pi}_i={\Pi}_i(F)$
 starting with ${\Pi}_0=U_0=F$, 
 where $\chi_{+-}$  
 is the one-dimensional representation of $G$
 defined in \eqref{eqn:chiab}.  
\end{definition}

\begin{remark}
\label{rem:Hassereg}
Clearly,
 any $U_j(F)$ in the Hasse sequence
 (or any $\Pi_j(F)$ in the standard sequence)
 starting with an irreducible finite-dimensional representation $F$
 of $G$ has a regular integral ${\mathfrak{Z}}_G({\mathfrak{g}})$-infinitesimal character
 (Definition \ref{def:intreg}).  
\end{remark}



The next proposition follows readily from the definition.  
\begin{proposition}
[tensor product with characters]
\label{prop:HStensor}
Let $F$ be an irreducible finite-dimensional representation of $G$, 
 and $\chi$ a one-dimensional representation of $G$.  
Then the representations in the Hasse sequences 
 (and in the standard sequence)
 starting with $F$ and $F \otimes \chi$ have the following relations:
\begin{alignat*}{3}
&\text{\rm{(Hasse sequence)}}
\qquad
&&U_i(F) \otimes \chi &&\simeq U_i(F \otimes \chi), 
\\
&\text{\rm{(standard sequence)}}
\qquad
&&
\Pi_i(F) \otimes \chi &&\simeq \Pi_i(F \otimes \chi).  
\end{alignat*}
\end{proposition}

The Hasse sequences and the standard sequences starting 
 with one-dimensional representations of $G$
 are described as follows.  
\begin{example}
\label{ex:171455}
We recall from Theorem \ref{thm:LNM20}
 that 
\index{A}{1Piidelta@$\Pi_{i,\delta}$, irreducible representations of $G$}
$\Pi_{\ell, \delta}$
 ($0 \le \ell \le n+1$, $\delta \in \{\pm\}$)
 are irreducible representations
 of $G=O(n+1,1)$
 with ${\mathfrak{Z}}_G({\mathfrak{g}})$-infinitesimal character 
\index{A}{1parho@$\rho_G$}
 $\rho_G$.  
Then for each one-dimensional representation
 $F \simeq \chi_{\pm\pm}$ of $G$
 (see \eqref{eqn:chiab}), 
 the Hasse sequence
 $U_i(F)$ ($0 \le i \le [\frac{n+1}{2}]$) 
 that starts with $U_0(F) \simeq F$, 
 and the standard sequence $\Pi_i(F):=U_i(F) \otimes (\chi_{+-})^i$
 are given as follows.  
\index{A}{1Piidelta@$\Pi_{i,\delta}$, irreducible representations of $G$}
\begin{alignat*}{3}
U_i({\bf{1}})&=\Pi_{i,(-1)^i},
&&
\Pi_i({\bf{1}})&&=\Pi_{i,+},
\\
U_i(\chi_{+-})&=\Pi_{i,(-1)^{i+1}},
&&
\Pi_i(\chi_{+-})&&=\Pi_{i,-},
\\
U_i(\chi_{-+})&=\Pi_{n+1-i,(-1)^{i}},
&&
\Pi_i(\chi_{-+})&&=\Pi_{n+1-i,+},
\\
U_i(\chi_{--})&=\Pi_{n+1-i,(-1)^{i+1}}, 
\qquad
&&
\Pi_i(\chi_{--})&&=\Pi_{n+1-i,-}.  
\end{alignat*}
\end{example}

%%%%%%%%%%%%%%%%%%%%%%%%%%%%%%%%%%%%%%%%%%%%%%%

\subsubsection{Existence of Hasse sequence}
%%%%%%%%%%%%%%%%%%%%%%%%%%%%%%%%%%%%%%%%%%%%%%%

In Section \ref{subsec:13.2}, 
 we formalize a conjecture
 about when 
\[
  \operatorname{Hom}_{G'}(\Pi|_{G'}, \pi ) \ne \{0\}
\]
for $\Pi \in \operatorname{Irr} (G)$
 and $\pi \in \operatorname{Irr} (G')$
 that have regular integral infinitesimal characters
 by using the standard sequence
 (Definition \ref{def:Hasse}).  
The formulation is based on the following theorem
 which asserts 
 that the converse statement to Remark \ref{rem:Hassereg}
 is also true.  

\begin{theorem}
\label{thm:Hassereg}
Any irreducible admissible representation of $G$
 of moderate growth
 with regular integral ${\mathfrak{Z}}_G({\mathfrak{g}})$-infinitesimal character is of the form 
 $U_j(F)$
 in the Hasse sequence
 for some $j$ $(0 \le j \le [\frac{n+1}{2}])$
 and for some irreducible finite-dimensional representation $F$ of $G$.  



Similarly,
 any irreducible admissible representation of $G$
 of moderate growth 
 with regular ${\mathfrak{Z}}_G({\mathfrak{g}})$-infinitesimal character
 is of the form $\Pi_j(F')$
 in the standard sequence
 for some $j$ $(0 \le j \le [\frac{n+1}{2}])$
 and for some irreducible finite-dimensional representation $F'$ of $G$.  
\end{theorem}

The proof of Theorem \ref{thm:Hassereg} follows from the classification
 of $\operatorname{Irr}(G)$
 (Theorem \ref{thm:irrG} in Appendix I)
 and the Langlands parameter of the representations
 in the Hasse sequence below
 (see also Theorem \ref{thm:1808116}).  

\subsubsection{Langlands parameter of the representations
 in the Hasse sequence}
\label{subsec:LanHasse}
Let $F$  be an irreducible finite-dimensional representation of $G=O(n+1,1)$. 
 We now determine the Langlands parameter of the representations in the Hasse sequence  $\{ U_i(F) \}$ (and the standard sequence $\{ \Pi_i(F) \}$)  for $0 \le i \le [\frac {n+1}2]$  and their $K$-types.



We use the parametrization of the finite-dimensional representation
 of $O(n,1)$ introduced in Section \ref{subsec:fdimrep} in Appendix I.  



We begin with the case
 where $F$ is obtained from an irreducible representation 
 of $O(n+2)$ of 
\index{B}{type1@type I, representation of $O(N)$}
 type I (Definition \ref{def:type})
 via the unitary trick.
The description of $U_i(F)$ and $\Pi_i(F)$
 for more general $F$ can be derived from this case
 by taking the tensor product with one-dimensional representations 
 $\chi_{\pm\pm}$
 of $G$, 
 see Theorem \ref{thm:1714107} below.  

\vskip 1pc
\par\noindent
{\bf{\underline{Case 1.}}}\enspace
$n=2m$ and $G=O(2m+1,1)$.  

For $F \in \widehat{O(n+2)}$ of type I, 
 we define $\sigma^{(i)} \equiv \sigma^{(i)}(F) \in \widehat{O(n)}$
 of type I for $0 \le i \le m = \frac n 2$
 as follows.  
We write $F= \Kirredrep {O(n+2, {\mathbb{C}})}{s}$
 with 
\index{A}{0tLambda@$\Lambda^+(N)$, dominant weight}
\[
   s=(s_{0},\cdots,s_{m},0^{m+1}) \in \Lambda^+(n+2)\equiv \Lambda^+(2m+2)
\]
 as in \eqref{eqn:CWOn}, 
 and regard it as an irreducible finite-dimensional representation
 of $G=O(n+1,1)$.  
We set
\[
   \sigma^{(i)} : = \Kirredrep {O(n)}{s^{(i)}} \in \widehat{O(n)}
\quad
  \text{for }\,\, 0 \le i \le m, 
\]
where
 $s^{(i)}\in\Lambda^+(n)\equiv \Lambda^+(2m)$ is given 
 for $0 \le i \le m$ as follows:
\begin{equation}
\label{eqn:sin=2m}
  s^{(i)}
  :=
  (s_{0}+1,\cdots,s_{i-1}+1,\widehat {s_{i}},s_{i+1},\cdots,s_{m},0^m). 
\end{equation}
It is convenient to introduce the {\em extended Hasse sequence}
 $\{U_i \equiv U_i(F)\}$
 ($0 \le i \le 2m+1$) by defining 
\index{A}{1chipmm@$\chi_{--}=\det$}
\begin{equation}
\label{eqn:Uiext1}
   U_i(F):=U_{n+1-i}(F) \otimes \chi_{--}
\quad
\text{for  $m+1 \le i \le n+1 = 2m+1$}.  
\end{equation}


\begin{theorem}
[$n=2m$]
\label{thm:171426}
Given an irreducible finite-dimensional representation
 $\Kirredrep{O(n+2,{\mathbb{C}})}{s}$of $G=O(n+1,1)$
 with 
\[
s =(s_0, s_1, \cdots, s_m,0,\cdots,0)\in \Lambda^+(n+2)
(=\Lambda^+(2m+2)), 
\]
 there exists uniquely an extended
\index{B}{Hassesequence@Hasse sequence}
Hasse sequence $U_0$, $U_1$, $\cdots$, $U_{2m+1}$
 starting with the irreducible finite-dimensional representation
 $U_0=\Kirredrep{O(n+2,{\mathbb{C}})}{s}$.  
Moreover,
 the extended Hasse sequence $U_0$, $\cdots$, $U_{2m+1}$
 satisfies the following properties.  
\begin{enumerate}
\item[{\rm{(1)}}]
There exist exact sequences of $G$-modules:
\begin{alignat*}{2}
& 0\to U_i \to I_{(-1)^{i-s_i}}(\sigma^{(i)}, i-s_i) \to U_{i+1} \to 0
\quad
&&(0 \le i \le m), 
\\
& 0\to U_i \to I_{(-1)^{n-i-s_{n-i}}}(\sigma^{(n-i)} \otimes \det, i+s_{n-i}) \to U_{i+1} \to 0
\quad
&&(m \le i \le 2m).   
\end{alignat*}

\item[{\rm{(2)}}]
The 
\index{B}{Ktypeformula@$K$-type formula}
$K$-type formula of the irreducible $G$-module $U_i$
 $(0 \le i \le m)$ is given by
\[
  \bigoplus_b \Kirredrep{O(n+1)}{b} \boxtimes 
  (-1)^{\sum_{k=0}^m(b_k-s_k)}, 
\]
where $b=(b_0, b_1, \cdots, b_m,0,\cdots,0)$
 runs over $\Lambda^+(n+1)\equiv \Lambda^+(2m+1)$
 subject to 
\begin{align*}
& b_{0} \ge s_{0}+1 \ge b_{1} \ge s_{1}+1 \ge \cdots \ge b_{i-1} \ge s_{i-1}+1, 
\\
& s_{i}\ge b_{i} \ge s_{i+1} \ge b_{i+1} \ge \cdots \ge s_{m}\ge b_{m} \ge 0, 
\\
&b_{m} \in \{0,1\}.  
\end{align*}
In particular, 
 the 
\index{B}{minimalKtype@minimal $K$-type|textbf}
 minimal $K$-type(s) of the $G$-module $U_i$ $(0 \le i\le m)$
 are given as follows:

for $s_m=0$, 
\begin{multline*}
  \Kirredrep{O(n+1)}{s^{(i)},0}\boxtimes (-1)^{i-s_i}
\\
  =
  \Kirredrep{O(n+1)}{s_{0}+1, \cdots, s_{i-1}+1, \widehat{s_i}, s_{i+1}, \cdots,s_m,0^{m+1}}\boxtimes (-1)^{i-s_i};
\end{multline*}

for $s_m >0$, 
\begin{equation*}
\Kirredrep{O(n+1)}{s^{(i)},0} \boxtimes (-1)^{i-s_i}
\,\,
\text{ and }
\,\,
(\Kirredrep{O(n+1)}{s^{(i)},0} \otimes \det)\boxtimes (-1)^{i-s_i+1}.  
\end{equation*}
\end{enumerate}
\end{theorem}



\begin{proof}
[Sketch of the proof]
\begin{enumerate}
\item[(1)]
By the translation principle,
 the first exact sequence follows from Theorem \ref{thm:LNM20} (1)
 which corresponds to the case $F \simeq {\bf{1}}$.  
Taking its dual, 
 we obtain another exact sequence
\[
  0 \to U_{i+1} \to I_{(-1)^{i-s_i}}(\sigma^{(i)}, n-i+s_i)
    \to U_i \to 0
\quad
  \text{for $0 \le i \le m$, }
\]
because $U_i$ is self-dual.  
Taking the tensor product with the one-dimensional representation $\chi_{--}$ of $G$, 
 we obtain by \eqref{eqn:Uiext1} and by Lemma \ref{lem:IVchi}
 another exact sequence of $G$-modules:
\[
  0 \to U_{n-i} \to I_{(-1)^{i-s_i}}(\sigma^{(i)}\otimes \det, n-i+s_i)
    \to U_{n+1-i} \to 0.  
\]
Replacing $i$ ($0 \le i \le m$)
 by $n-i$ ($m \le n-i \le 2m$), 
 we have shown the second exact sequence.  
\item[(2)]
The $K$-type formula of the irreducible finite-dimensional representation 
 $U_0=\Kirredrep {O(n+1,1)}{s}$ of $G$ is known by the classical branching law
 (see Fact \ref{fact:ONbranch}).  
Since the $K$-type formula of the principal series representation
 is given by the Frobenius reciprocity
 which we can compute by using Fact \ref{fact:ONbranch} again,
 the $K$-type formula of $U_{i+1}$ follows inductively from that of $U_i$
 by the exact sequence 
 in the first statement.  
\end{enumerate}
\end{proof}
See also Theorem \ref{thm:171471} in Appendix I
 for another description of the irreducible representation $U_i(F)$
 in terms of $\theta$-{\it{stable parameters}}.  



\begin{remark}
\label{rem:Utemp}
When $i=m$ and $n=2m$, 
 $s^{(i)}$ is of the form
\[
  s^{(m)} =(s_0+1, \cdots, s_{m-1}+1, 0^m) \in \Lambda^+(2m), 
\]
and therefore
 the irreducible $O(n)$-module 
$\sigma^{(m)}=\Kirredrep{O(2m)}{s^{(m)}}$ is of type Y
 (Definition \ref{def:OSO}).  
Hence we have an isomorphism
\begin{equation}
\label{eqn:smdet}
    \sigma^{(m)} \simeq \sigma^{(m)} \otimes \det
\end{equation}
 as $O(2m)$-modules by Lemma \ref{lem:typeY}.  
We recall from Theorem \ref{thm:171426} (1)
 that there is an exact sequence of $G$-modules as follows:
\[
  0 \to U_m \to I_{(-1)^{m-s_m}}(\sigma^{(m)}, m-s_m) \to U_{m+1} \to 0.  
\]
Taking the tensor product with the character $\chi_{--} \simeq \det$, 
 we obtain from \eqref{eqn:Uiext1} and Lemma \ref{lem:IVchi}
 another exact sequence of $G$-modules:
\[
  0 \to U_{m+1} \to I_{(-1)^{m-s_m}}(\sigma^{(m)}\otimes \det, m-s_m)
 \to U_{m} \to 0.  
\]
By \eqref{eqn:smdet}, 
 the principal series representations 
\[
I_{(-1)^{m-s_m}}(\Kirredrep{O(2m)}{s^{(m)}}, m-s_m)
\simeq
I_{(-1)^{m-s_m}}(\Kirredrep{O(2m)}{s^{(m)}}\otimes \det, m-s_m)
\]
 split into a direct sum
 of two irreducible $G$-modules $U_m$ and $U_{m+1}$
 (see also Theorem \ref{thm:IndV} (3) in Appendix I).  
\end{remark}

\vskip 1pc
\noindent
{\bf{\underline{Case 2.}}}\enspace
$n=2m-1$ and $G=O(2m,1)$.  

For $F \in \widehat{O(n+2)}$ of type I, 
 we define $\sigma^{(i)} \equiv \sigma^{(i)}(F) \in \widehat{O(n)}$
 for $0 \le i \le m-1 = \frac 1 2 (n-1)$
 as follows.  
We write $F= \Kirredrep {O(n+2)}{s}$
 with 
\[
s=(s_{0},s_1,\cdots,s_{m-1},0^{m+1}) \in \Lambda^+(n+2)\equiv \Lambda^+(2m+1), 
\]
 as in \eqref{eqn:CWOn}.  
Then we define $s^{(i)}\in \Lambda^+(n)\equiv \Lambda^+(2m-1)$
 ($0 \le i \le m-1$) by
\begin{equation}
\label{eqn:sin=2m-1}
  s^{(i)}
  :=
  (s_{0}+1,\cdots,s_{i-1}+1,\widehat {s_{i}},s_{i+1},\cdots,s_{m-1},0^m), 
\end{equation}
and define irreducible finite-dimensional representations by 
\[
   \sigma^{(i)} : = \Kirredrep {O(n)}{s^{(i)}} \in \widehat{O(n)}
\quad
  \text{for }\,\, 0 \le i \le m-1.  
\]
For later purpose, we set
\[
s^{(m)}
  :=
  (s_{0}+1,\cdots,s_{m-2}+1,1,0^{m-1}) \in \Lambda^+(n). 
\]
Then there is an isomorphism as $O(n)$-modules: 
\[
  \sigma^{(m-1)} \otimes \det
  \simeq
  \Kirredrep{O(n)}{s^{(m)}}.  
\]
It is convenient to introduce the {\em extended Hasse sequence } $\{U_i \equiv U_i(F)\}$
 ($0 \le i \le 2m$) by defining for
 $m+1 \le i \le 2m$ 
\begin{equation}
\label{eqn:Uiext2}
  U_i(F):= U_{n+1-i}(F) \otimes \chi_{-+}.  
\end{equation}
Implicitly,
 the definition \eqref{eqn:Uiext2} includes a claim 
 that there is an isomorphism
 of discrete series representations
 (cf. Remark \ref{rem:Umdisc} below):
\index{A}{1chipmp@$\chi_{-+}$}
\begin{equation}
\label{eqn:UmF-+}
   U_m(F) \simeq U_m(F) \otimes \chi_{-+}
\end{equation}
when $G=O(n+1,1)$ with $n=2m-1$.  



We note that the one-dimensional representations
 $\chi_{--}$ and $\chi_{-+}$ 
 in \eqref{eqn:Uiext1} and \eqref{eqn:Uiext2}
 are chosen differently 
 according to the parity of $n$.  



The proof of the following theorem goes similarly to that of Theorem \ref{thm:171426}.  
\begin{theorem}
[$n=2m-1$]
\label{thm:171425}
Given an irreducible finite-dimensional representation
 $\Kirredrep {O(n+2,{\mathbb{C}})}{s}$
 of $G=O(n+1,1)$ with 
\[
s=(s_{0},s_1,\cdots,s_{m-1},0^{m+1}) \in \Lambda^+(n+2)
 (=\Lambda^+(2m+1)), 
\]
 there exists uniquely an extended 
\index{B}{Hassesequence@Hasse sequence}
Hasse sequence $U_0$, $U_1$, $\cdots$, $U_{2m}$
 of $G=O(2m,1)$
 starting with the irreducible finite-dimensional representation
 $U_0=\Kirredrep{O(n+2,{\mathbb{C}})}{s}$.  
Moreover, 
 the extended Hasse sequence $U_0$, $U_1$, $\cdots$, $U_{2m}$
 satisfies the following properties.  
\begin{enumerate}
\item[{\rm{(1)}}]
There exist exact sequences of $G$-modules:
\begin{alignat*}{2}
0 &\to U_i \to I_{(-1)^{i-s_i}}(\sigma^{(i)}, i-s_i)
   \to U_{i+1} \to 0
\qquad
&&(0 \le i \le m-1), 
\\
0 & \to U_i \to I_{(-1)^{n-i-s_{n-i}}}(\sigma^{(n-i)} \otimes \det, i+s_{n-i})
  \to U_{i+1} \to 0
\qquad
&&(m\le i \le 2m-1).  
\end{alignat*}

\item[{\rm{(2)}}]
The $K$-type formula of the irreducible $G$-module $U_i$
 $(0 \le i \le m)$ is given by
\[
  \bigoplus_b \Kirredrep{O(n+1)}{b} 
  \boxtimes 
  (-1)^{\sum_{k=0}^{m} b_k - \sum_{k=0}^{m-1}s_k}, 
\]
where $b=(b_0, b_1, \cdots, b_{m-1},0,\cdots,0)$
 runs over $\Lambda^+(n+1)\equiv \Lambda^+(2m)$
 subject to the following conditions:
\begin{align*}
& b_{0} \ge s_{0}+1 \ge b_{1} 
  \ge s_{1}+1 \ge \cdots \ge b_{i-1} \ge s_{i-1}+1, 
\\
& s_{i}\ge b_{i} \ge s_{i+1} \ge b_{i+1} \ge \cdots \ge s_{m-1}\ge b_{m-1}\ge 0.
\end{align*}
In particular,
 the minimal $K$-type of the $G$-module $U_i$ $(0 \le i \le m)$ is given by 
\begin{multline*}
  \Kirredrep{O(n+1)}{s^{(i)},0} \boxtimes (-1)^i
\\
  =
  \Kirredrep{O(n+1)}{s_0+1, \cdots, s_{i-1}+1, \widehat{s_i}, 
s_{i+1}, \cdots, s_{m-1}, 0^{m+1}} \boxtimes (-1)^{i-s_i}.  
\end{multline*}
\end{enumerate}
\end{theorem}

\begin{remark}
\label{rem:Umdisc}
$U_m$ is a discrete series representation
 of $G=O(2m,1)$.  
\end{remark}



See also Theorem \ref{thm:171471b} in Appendix I
 for another description of the irreducible representation
 $U_i(F)$ in terms of $\theta$-stable parameters.  



By applying Proposition \ref{prop:HStensor} and Lemma \ref{lem:IVchi},
 we may unify the first statement of Theorems \ref{thm:171426}
 and \ref{thm:171425}
 as follows.  
\begin{theorem}
\label{thm:1714107}
Let $F$ be an irreducible finite-dimensional representation of $G=O(n+1,1)$
 of type I
 (see Definition \ref{def:typeone} in Appendix I), 
 and $a,b \in \{\pm\}$.  
Then for $F_{a, b}:= F \otimes \chi_{a b}$, 
 there exists uniquely a Hasse sequence
 $U_i(F_{a,b})$  $(0 \le i \le [\frac{n+1}{2}])$ 
 starting with $U_0(F_{a, b})=F_{a, b}$.  
Moreover, 
 the irreducible $G$-modules $U_i(F_{a,b})$ occur 
 in the following exact sequence of $G$-modules
\[
   0 \to U_i(F_{a, b}) \to I_{a b (-1)^{i-s_i}}(\sigma_a^{(i)}, i-s_i)
      \to U_{i+1}(F_{a, b}) \to 0
\]
for $0 \le i \le [\frac{n-1}{2}]$. 
Here $\sigma_a^{(i)}=\sigma^{(i)}$ if $a=+;$
$\sigma^{(i)} \otimes \det$ if $a=-$.  
\end{theorem}

\medskip
\noindent
\begin{remark}
\label{remark:171478}
By \eqref{eqn:314}, 
 we have linear bijections for all $i$, $j$:
\[
{\operatorname{Hom}}_{G'}
 (U_i(F)|_{G'}, U_j'(F'))
\simeq
{\operatorname{Hom}}_{G'}
 (U_{n+1-i}(F)|_{G'}, U_{n-j}'(F') \otimes \chi_{+-}).  
\]
\end{remark}

\medskip
\begin{remark}
Using the definition of the extended Hasse sequence we also define an extended standard sequence.
\end{remark}
\medskip

By abuse of notation we will from now on not distinguish between Hasse sequences and extended Hasse sequences and refer to both as Hasse sequences. A similar convention applies to standard sequences.

\medskip
The following observation will be used 
 in Section \ref{subsec:Evidence4}
 for the proof of Evidence E.4
 of Conjecture \ref{conj:GPver1} below.  

\begin{proposition}
\label{prop:172009}
Suppose $F$ and $F'$ are irreducible finite-dimensional representations
 of $G=O(n+1,1)$ and $G'=O(n,1)$, 
respectively, 
 such that ${\operatorname{Hom}}_{G'}(F|_{G'}, F') \ne \{0\}$.  
Suppose the principal series representations
 $I_{\delta}(V,\lambda)$ of $G$
 and $J_{\varepsilon}(W,\nu)$ of $G'$
 contain $F$ and $F'$, 
 respectively,
 as subrepresentations. 
Then the following hold.  
\begin{enumerate}
\item[{\rm{(1)}}]
$[V:W]=1;$
\item[{\rm{(2)}}]
\index{A}{1psi@$\Psising$,
          special parameter in ${\mathbb{C}}^2 \times \{\pm\}^2$}
$(\lambda, \nu, \delta, \varepsilon) \in \Psising$
 (see \eqref{eqn:singset}), 
namely,
 the quadruple $(\lambda, \nu, \delta, \varepsilon)$ does not satisfy
 the generic parameter condition \eqref{eqn:nlgen}.  
\end{enumerate}
\end{proposition}
\begin{proof}
For the proof,
 we use a description 
 of irreducible finite-dimensional representations
 of the disconnected group $G=O(n+1,1)$
 in Section \ref{subsec:fdimrep}
 of Appendix I.  
In particular,
 using Lemma \ref{lem:161612}, 
we may write 
\[
  F = \Kirredrep {O(n+1,1)}{\lambda_0, \cdots, \lambda_{[\frac n 2]}}_{a, b}
\]
for some $(\lambda_0, \cdots, \lambda_{[\frac n 2]})\in \Lambda^+([\frac n 2]+1)$ and $a, b \in \{\pm\}$.  
By the branching rule 
 for $O(n+1,1) \downarrow O(n,1)$
 (see Theorem \ref{thm:ON1branch}), 
 an irreducible summand $F'$ of $F|_{O(n,1)}$ is of the form
\[
  F' = \Kirredrep {O(n,1)}{\nu_0, \cdots, \nu_{[\frac {n-1} 2]}}_{a, b}
\]
for some $(\nu_0, \cdots, \nu_{[\frac {n-1} 2]})\in \Lambda^+([\frac {n+1} 2])$ such that 
\begin{alignat*}{2}
&\lambda_{0} \ge \nu_{0} \ge \lambda_{1} \ge \cdots \ge \nu_{[\frac{n-1}2]} \ge 0\quad
&&\text{for $n$ odd}, 
\\
&\lambda_{0} \ge \nu_{0} \ge \lambda_{1} \ge \cdots \ge \nu_{[\frac{n-1}2]} \ge \lambda_{[\frac n 2]}
\quad
&&\text{for $n$ even}.   
\end{alignat*}
We recall 
 that for every irreducible finite-dimensional representation $F$
 of a real reductive Lie group
 there exists only one principal series representation
 that contains $F$ 
 as a subrepresentation.  
By Theorem \ref{thm:1714107} with $i=0$, 
 the unique parameter $(V, \delta, \lambda)$ is given
 by 
\[
\text{$V=\Kirredrep{O(n)}{\lambda_1, \cdots, \lambda_{[\frac n 2]}}$
 ($\otimes \det$ if $a=-$), 
 $\lambda=-\lambda_0$ and $\delta=a b (-1)^{-\lambda_0}$.}
\]  
Likewise, 
 the unique parameter $(W, \varepsilon, \nu)$ for $F'$
 is given by 
\[
 \text{$W=\Kirredrep{O(n-1)}{\nu_1, \cdots, \nu_{[\frac {n-1} 2]}}$
 ($\otimes \det$ if $a=-$), 
 $\nu=-\nu_0$, 
 and $\varepsilon = a b (-1)^{-\nu_0}$.  }
\]
Hence $[V:W]\ne 0$, 
 or equivalently,
 $[V:W]=1$ 
 by the branching rule 
 for $O(n) \downarrow O(n-1)$.  
Moreover, 
 $\delta \varepsilon = (-1)^{\nu_0 + \lambda_0}$
 and $\nu-\lambda= \lambda_0 -\nu_0 \in {\mathbb{N}}$.  
Hence the generic parameter condition \eqref{eqn:nlgen} fails, 
 or equivalently,
$(\lambda, \nu, \delta, \varepsilon) \in \Psising$. 
 
\end{proof}



%%%%%%%%%%%%%%%%%%%%%%%%%%%%%%%%%%%
\subsection{The Conjecture}
\label{subsec:13.2}
%%%%%%%%%%%%%%%%%%%%%%%%%%%%%%%%
We propose a conjecture about when 
\[
{\operatorname{Hom}}_{G'}(\Pi|_{G'},\pi)={\mathbb{C}}
\]
where $\Pi \in {\operatorname{Irr}}(G)$ and $\pi \in {\operatorname{Irr}}(G')$ have regular integral infinitesimal characters
 (Definition \ref{def:intreg}).  
We give two formulations of the conjecture, 
 see Conjectures \ref{conj:GPver1} and \ref{conj:V2} below.  
Supporting evidence is given in Section \ref{subsec:SuppConj}.  

\subsubsection{Conjecture: Version 1}
%%%%%%%%%%%%%%%%%%%%%%%%%%%%%%%%%%%%
We begin with a formulation of the conjecture
 in terms of a standard sequence
 (Definition-Theorem \ref{def:UHasse})
 of irreducible representations $\Pi_i$ of $G=O(n+1,1)$
 and that of irreducible representations $\pi_j$ of the subgroup $G'=O(n,1)$.  
We note that both $\Pi_i$ and $\pi_j$ have
 regular integral infinitesimal characters
 because both $F:=\Pi_0$ and $F':=\pi_0$ are
 irreducible {\it{finite-dimensional}} representations of $G$
 and $G'$, respectively. 

\begin{conjecture}
\label{conj:GPver1}
Let $F$ be an irreducible finite-dimensional representations
 of $G=O(n+1,1)$,
 and 
\index{A}{1PiideltaF@$\Pi_i(F)$, standard sequence starting with $F$}
$\{\Pi_i(F)\}$ be the standard sequence
 starting at $\Pi_0(F)=F$.  
Let $F'$ be an irreducible finite-dimensional representation of the subgroup $G'=O(n,1)$, 
 and $\{\pi_j(F')\}$ the standard sequence starting 
 at $\pi_0(F')=F'$.  
Assume that 
\[ 
   {\operatorname{Hom}}_{G'}(F|_{G'},F') \not = \{0\}.  
\]
Then the symmetry breaking for representations $\Pi_i(F)$, $\pi_j(F')$
 in the standard sequences  is represented graphically
 in Diagrams \ref{tab:SBodd} and \ref{tab:SBeven}. 
In the first row are representations of G, in the second row are representations of $G'$. 
Symmetry breaking operators are represented by arrows, 
 namely,
 there exist nonzero symmetry breaking operators
 if and only if there are arrows in the diagram.  
\end{conjecture}
\medskip

\begin{figure}[htp]
\color{black}
\caption{Symmetry breaking for $O(2m+1,1) \downarrow O(2m,1)$}
\begin{center}
\begin{tabular}{@{}c@{~}c@{~}c@{~}c@{~}c@{~}c@{~}c@{~}c@{~}c@{~}c@{}}
$\Pi_0(F)$
& 
&$\Pi_1(F)$
&  
&\dots
&
&$\Pi_{m-1}(F)$
& 
&$\Pi_{m}(F)$ 
\\
$\downarrow$ 
&$\swarrow$
& $\downarrow$
& $\swarrow$ 
& 
& $\swarrow$ 
& $ \downarrow $
&  $\swarrow $  
&  $\downarrow$ 
\\
$\pi_0(F')$& &$\pi_1(F')$
& 
&\dots 
&
& $\pi_{m-1}(F')$ 
& 
& $\pi_{m}(F')$ 
\end{tabular}
\end{center}
\label{tab:SBodd}
\end{figure}%

\medskip
\begin{figure}[htp]
\color{black}
\caption{Symmetry breaking for $O(2m+2,1) \downarrow O(2m+1,1)$ }
\begin{center}
\begin{tabular}{@{}c@{~}c@{~}c@{~}c@{~}c@{~}c@{~}c@{~}c@{~}c@{~}c@{~}c@{}}
$\Pi_0(F)$& &$\Pi_1(F)$ & & \dots &  & $\Pi_{m-1}(F)$& & $\Pi_{m}(F)$ & & $\Pi_{m+1}(F)$\\
$\downarrow$ &$\swarrow$& $\downarrow $& $\swarrow$ &  & $\swarrow$ & $ \downarrow $& $\swarrow $ & $\downarrow$ & $\swarrow$\\
$\pi_0(F')$& &$\pi_1(F')$& &\dots  & & $\pi_{m-1}(F')$ & & $\pi_{m}(F')$ 
\end{tabular}
\end{center}
\label{tab:SBeven}
\end{figure}%

\begin{remark}
Instead of using standard sequences to state the conjecture
 it may be also useful to rephrase it using extended  Hasse sequences.
\end{remark}
%%%%%%%%%%%%%%%%%%%%%%%%%%%%%%
%\newpage

\subsubsection{Conjecture: Version 2}
We rephrase the conjecture using $\theta$-stable parameters, 
 which will be introduced
 in Section \ref{subsec:thetapara}
 of Appendix I, 
 and restate Conjecture \ref{conj:GPver1} as an algorithm in this notation.



In Theorems \ref{thm:171471} and \ref{thm:171471b} of Appendix I, 
 we shall give the 
\index{B}{1thetastableparameter@$\theta$-stable parameter}
 $\theta $-stable parameters
 of the representations of the standard sequence starting
 with an irreducible finite-dimensional representation $F$
 summarized as follows.

\medskip
\noindent
\label{Hasse sequence} 
\begin{enumerate}
\item
Suppose that $n=2m$. 
Let 
\[
   F=
  \Kirredrep{O(2m+1,1)} {\mu}_{a,b}
  = \Kirredrep{O(2m+1,1)} {\mu} \otimes \chi_{a b}
\]
 for $\mu \in \Lambda^+(m+1)$ and $a,b \in \{\pm\}$
 be an irreducible finite-dimensional representation
 of $O(2m+1,1)$, 
 see Section \ref{subsec:fdimrep} in Appendix I.  
Its $\theta$-stable parameter is 
\[
(~||~\mu_1,\mu_2,\dots, \mu_{m},\mu_{m+1})_{a,b}
\]
 and we have the $\theta$-stable parameters of the representations in the  standard sequence (written in column).

\begin{eqnarray*}
\Pi_0(F)=&  (~||~\mu_1,\mu_2,\dots, \mu_{m},\mu_{m+1})_{a,b} & \\
\Pi_1(F)=& (\mu_1~|| ~\mu_2,\dots, \mu_{m}, \mu_{m+1})_{a,b} & \\
\vdots\qquad &  \vdots   & \\
\Pi_m(F)=&  (\mu_1,\mu_2,\dots, \mu_{m}~||~ \mu_{m+1})_{a,b}.  & 
\end{eqnarray*}
\item
Suppose that $n=2m+1$.  
Let 
\[
   F=\Kirredrep{O(2m+2,1)}{\mu}_{a,b} 
    = \Kirredrep{O(n+1,1)}{\mu} \otimes \chi_{ab} 
\]
   {for } $\mu \in \Lambda^+(m+1) \mbox{ and }
   a, b \in \{\pm\} $
 be an irreducible finite-dimensional representation of $O(2m+2,1)$.  
Its $\theta$-stable parameter is \[(||~\mu_1,\mu_2,\dots, \mu_{m},\mu_{m+1})_{a,b}\] 
and we have the $\theta$-stable parameters of the representations in standard sequence (written in column).

\begin{eqnarray*}
\Pi_0(F)&=  (||~\mu_1,\mu_2,\dots, \mu_{m+1})_{a,b} & 
\\
\Pi_1(F)&= (\mu_1~|| ~\mu_2,\dots, \mu_{m+1})_{a,b} & 
\\
\vdots\hphantom{mii} &  \vdots  &
\\
\Pi_{m+1}(F)&=  (\mu_1,\mu_2,\dots, \mu_{m+1}~|| )_{a,b}. & 
\end{eqnarray*}

\end{enumerate}

\medskip

We refer to the finite-dimensional representation
$\Pi_0(F)=F$
 as the {\it{starting representation}} of the standard sequence  
and to the tempered representation $\Pi_m(F)$
 (when $n=2m$)
 or the discrete series representation $\Pi_{m+1}(F)$
 (when $n=2m+1$)
 as the {\it{last representation}}
 of the standard sequence
 (see Remarks \ref{rem:Utemp} and \ref{rem:Umdisc}).

\begin{conjecture}
\label{conj:V2}
Let $\Kirredrep G \mu_{a,b}$ be an irreducible finite-dimensional representation of $G=O(n+1,1)$, 
 and $\Kirredrep {G'} \nu_{a,b}$ be an irreducible finite-dimensional representation of the subgroup $G'=O(n,1)$, 
 where $\mu \in \Lambda^+([\frac{n+2}{2}])$, 
 $\nu \in \Lambda^+([\frac{n+1}{2}])$, 
 and $a,b,c,d \in \{\pm\}$, 
 see \eqref{eqn:Fn1ab} and \eqref{eqn:ON1isom} in Appendix I.  
Assume that 
\begin{equation}
\label{eqn:fdabcd}
  {\operatorname{Hom}}_{G'} (\Kirredrep{G}{\mu}_{a,b}|_{G'}, \Kirredrep{G'}{\nu}_{c,d})\not = \{0\}.
\end{equation}


In (1) and (2) below, 
 nontrivial symmetry breaking operators are represented by arrows connecting the $\theta$-stable parameters of the representations.


\begin{enumerate}
\item
[{\rm{(1)}}]  
Suppose that $n=2m$.  
Then $\mu =(\mu_1, \cdots, \mu_{m+1})\in \Lambda^+(m+1)$
 and $\nu =(\nu_1, \cdots, \nu_{m})\in \Lambda^+(m)$.  
Then two  representations in the standard sequences have a nontrivial symmetry breaking operator if and only if the $\theta$-stable parameters of the representations satisfy one of the following conditions.

\begin{eqnarray*}
& (\mu_{1}, \dots , \mu_i ~|| ~\mu_{i+1} , \dots , \mu_{m+1})_{a,b}& \\
& \Downarrow &  \\
 &       (\nu_1 , \dots  ,\nu_i ~ ||~ \nu_{i+1}, \dots ,\nu_{m}  )_{c,d}
 \end{eqnarray*}
 \begin{center}
                      or
  \end{center}                    
\begin{eqnarray*}
& (\mu_{1}, \dots , \mu_i ~|| ~\mu_{i+1} , \dots , \mu_{m+1})_{a,b}&  
\\
& \Downarrow &  \\
&     (\nu_1, \dots  ,\nu_{i-1} ~ ||~ \nu_i, \nu_{i+1}, \dots ,\nu_{m})_{c,d} &
\end{eqnarray*}        

\item[{\rm{(2)}}]  
Suppose that $n=2m+1$.  
Then two infinite-dimensional representations in the standard sequences
 have a nontrivial symmetry breaking operator
 if and only if the $\theta$-stable parameters of the representations satisfy one of the following conditions:
\begin{eqnarray*} 
 & (\mu_{1}, \dots , \mu_i ~|| ~\mu_{i+1} , \dots , \mu_{m+1})_{a,b} &  \\  
 & \Downarrow &    \\
 &  (\nu_1, \dots  ,\nu_i ~ ||~ \nu_{i+1}, \dots ,\nu_{m+1}  )_{c,d} &
 \end{eqnarray*}
 \begin{center}
or 
\end{center}
\begin{eqnarray*}
& (\mu_{1}, \dots , \mu_i ~|| ~\mu_{i+1} , \dots , \mu_{m+1})_{a,b} &\\
& \Downarrow &  \\
 &  (\nu_1,  \dots  ,\nu_{i-1}  ~ ||~ \nu_{i}, \dots ,\nu_{m+1}  )_{c,d} &
 \end{eqnarray*}
\end{enumerate}  
\end{conjecture}

\begin{remark}
\label{rem:V2}
See Theorem \ref{thm:ON1branch} in Appendix I
 for the condition on the parameters $\mu$, $\nu$, and $a$, $b$, $c$, $d$
 such that \eqref{eqn:fdabcd} holds.  
In particular,
 \eqref{eqn:fdabcd} implies either 
 $(a,b)=(c,d)$ or $(a,b)=(-c,-d)$.  
See also Lemma \ref{lem:fdeq} (2) 
 for the description of overlaps in the expressions
 of irreducible finite-dimensional representations
 of $O(N-1,1)$ 
 when $N$ is even.  
\end{remark}
%%%%%%%%%%%%%%%%%%%%%%%
\subsection{Supporting evidence}
\label{subsec:SuppConj}

In this section,
 we provide some evidence 
 supporting our conjecture.  
\begin{description} 
\item[{\bf E.1}] \
If $F \in {\operatorname{Irr}}(G)_{\rho}$
 and $F' \in {\operatorname{Irr}}(G')_{\rho}$, 
 the Conjecture \ref{conj:GPver1} is true.  
(Equivalently,
 if $\Kirredrep{O(n+1,1)}\mu_{+,+}$ and $\Kirredrep{O(n,1)} {\nu}_{+,+}$ are both the trivial one-dimensional representations, 
Conjecture \ref{conj:V2} is true.)
\item[{\bf E.2}]  \ Some vanishing results  for symmetry breaking operators.  
\item[{\bf E.3}] \ Our conjecture is consistent
 with the Gross--Prasad conjecture for {\it{tempered}} representations of the special orthogonal group.
\item[{\bf E.4}]
  \  There exists a nontrivial symmetry breaking operator $\Pi_1 \rightarrow \pi_1$.  
\label{subsec:vanishtemp}
\end{description}
%%%%%%%%%%%%%%%%%%%%%%%%%%%%%%%%%%%%%%%%%%%%%%%%%%%


\bigskip
\noindent
\subsubsection{Evidence  E.1}
This was proved in Theorems \ref{thm:SBOvanish} and \ref{thm:SBOone}.  

\bigskip
\noindent
\subsubsection{Evidence E.2} 
Detailed proofs of the following propositions  will be published in a sequel to this monograph. 


\medskip
Recall from Definition-Theorem \ref{def:UHasse}
 that $U_i(F_{a,b})$ refers to the $i$-th term in the Hasse sequence
 starting with the finite-dimensional representation 
 $F_{a,b}=F \otimes \chi_{a b}$ of $G$ 
 and $U_j(F'_{c,d})$ to the $j$-th term in the Hasse sequence starting
 with the finite-dimensional representation 
$
   F_{c,d}'=F' \otimes \chi_{c d}
$ of $G'$.  
\begin{proposition}
\label{conj:SBOvanish}
Let $a,b,c,d \in \{\pm\}$, 
 $0 \le i \le [\frac{n+1}{2}]$
 and $0 \le j \le [\frac{n}{2}]$.  
Then 
\[
{\operatorname{Hom}}_{G'}
 (U_i(F_{a,b})|_{G'}, U_j(F_{c,d}'))=\{ 0 \}
\quad
\text{if $j \not =i-1$, $i$}.  
\] 
\end{proposition}



\medskip 
If one of the representations of $G=O(n+1,1)$ respectively of $G'=O(n,1)$
 is tempered then the following vanishing theorems hold.  
 
 
\vskip 1pc
\noindent
$\bullet$\enspace Assume first 
$(G,G')=(O(2m,1),O(2m-1,1))$.  


Let $s=(s_0, \cdots, s_{m-1},0^{m+1}) \in \Lambda^+(2m+1)$
 and $t=(t_0, \cdots, t_{m-1},0^m) \in \Lambda^+(2m)$
 satisfy $t \prec s$
 (see Definition \ref{def:Young} for the notation).  


\begin{proposition}
\label{prop:171427}
Let $U_0$, $\cdots$, $U_m$, $U_{m+1}$ be the Hasse sequence of $G=O(2m,1)$
 with $U_0= \Kirredrep{O(2m+1,{\mathbb{C}})}{s}$, 
 and $U_0', \cdots, U_{m-1}'$ be that of $G'=O(2m-1,1)$
 with $U_0'= \Kirredrep{O(2m,{\mathbb{C}})}{t}$.  
Then 
\[
{\operatorname{Hom}}_{G'}
(U_m|_{G'}, U_j')=\{0\}
\quad
\text{if $0 \le j \le m-2$.}
\]
\end{proposition}



\vskip 1pc
\noindent
$\bullet$\enspace Assume now
$(G,G')=(O(2m+1,1),O(2m,1))$.  


Let $s=(s_0, \cdots, s_{m},0^{m+1}) \in \Lambda^+(2m+2)$
 and $t=(t_0, \cdots, t_{m-1},0^{m+1}) \in \Lambda^+(2m+1)$
 satisfy $t \prec s$.  


\begin{proposition}
\label{prop:171429}
Let $U_0, \cdots, U_m$ be the Hasse sequence of $G=O(2m+1,1)$
 with $U_0= \Kirredrep{O(2m+2,{\mathbb{C}})}{s}$, 
 and $U_0', \cdots, U_{m}'$ be that of $G'=O(2m,1)$
 with $U_0'= \Kirredrep{O(2m+1,{\mathbb{C}})}{t}$.  
Then 
\[
{\operatorname{Hom}}_{G'}
(U_i|_{G'}, U_m')=\{0\}
\quad
\text{if \quad $0 \le i \le m-1$.}
\]
\end{proposition}

\medskip

\begin{remark}
These propositions prove only part of the vanishing statement
 of symmetry breaking operators
 formulated in Conjecture \ref{conj:V2}.
\end{remark}


\bigskip

%%%%%%%%%%%%%%%%%%%%%
\noindent 
\subsubsection{Evidence E.3} 


We use the notations and assumptions
 of the previous section, 
 and show that our conjecture is consistent
 with the original 
\index{B}{GrossPrasadconjecture@Gross--Prasad conjecture|textbf}
Gross--Prasad conjecture
 on {\it{tempered representations}}
 \cite{GP}.  
For simplicity,
 we treat here 
 only for $(G, G')=(O(n+1,1), O(n,1))$
 with $n=2m$. 
We shall see that a special case
 of Conjecture \ref{conj:V2}
 ({\it{i.e.}}, the conjecture for the last representation
 of the standard sequence)
 implies some results
 (see \eqref{eqn:GPm} below)
 that were predicted by the original conjecture of Gross and Prasad
 for tempered representations of special orthogonal groups.  



Assume that the irreducible finite-dimensional representations
 $\Pi_0$ of $G$ and $\pi_0$ of $G'$
 are of type I 
 (Definition \ref{def:typeone})
 and that 
 $(\mu_1,\dots, \mu_{m}, \mu_{m+1})$ and $(\nu_1,\dots, \nu_{m})$ are their highest weights. 



By the branching law 
 for {\it{finite-dimensional}} representations
 with respect to $G \supset G'$
 (see Theorem \ref{thm:ON1branch} in Appendix I), 
 the condition 
 \[\mbox{Hom}_{O(n,1)}(\Pi_{0}|_{G'},\pi_{0}) \not = \{0\}\]
is equivalent to 
\begin{equation}
\label{eqn:munuGP}
   \mu_1 \geq \nu_1 \geq \mu_2 \geq \dots \geq \nu_{m} \geq \mu_{m+1} \geq 0.  
\end{equation}
Let $U_m$ (resp.~ $\Pi_m=U_m \otimes (\chi_{+-})^m$)
 be the $m$-th term of the Hasse sequence 
 (resp. the standard sequence)
 starting with the irreducible finite-dimensional representation
 $\Pi_0 =U_0$
 (see Definitions \ref{def:UHasse} and \ref{def:Hasse}).  
Then we have a direct sum decomposition 
 of the principal series representation
\[
   I_{(-1)^{m-\mu_{m+1}}}(\Kirredrep{O(2m)}{\mu_1+1, \cdots,\mu_m+1,0^m}, m-\mu_{m+1})
 \simeq U_m \oplus (U_m \otimes \det)
\]
by Theorem \ref{thm:171426} (1) and Remark \ref{rem:Utemp}.  
Assume that $\Pi_m$ is tempered.  
Then $U_m$ is also tempered,
 and the continuous parameter of the principal series representation
 must lie on the unitary axis, 
 that is, 
 $m-\mu_{m+1} \in m + \sqrt{-1}{\mathbb{R}}$.  
Hence $\mu_{m+1}=0$.  



Since $\mu_{m+1}=0$, 
the $\theta$-stable parameters of the tempered representations
 $\Pi_m$, $\Pi_m \otimes \det$ are given by 
\[ 
   (\mu_1,\dots, \mu_m || 0  )_{+,+}, 
\quad
   \Rq{\mu_1, \cdots,\mu_m}{0}{-,-}, 
\]
 whereas the $\theta$-stable parameter 
 of the discrete series representation
 of $G'=O(2m,1)$ is given by 
\[
     (\nu_1, \dots , \nu_m ~|| )_{+,+}.  
\]
In view of the $K$-type formula in Theorem \ref{thm:171426} (2), 
 we see 
\[
   U_m \not \simeq U_m \otimes \det
\]
 as $G$-modules,
 and thus $\Pi_m \not \simeq \Pi_m \otimes \det$.  
Therefore,
 the restriction of the principal series representation $\Pi_m$
 of $G=O(2m+1,1)$ to the subgroup $\overline G=SO(2m+1,1)$ is irreducible
 by Lemma \ref{lem:171523} (1) in Appendix II.  
We set 
\[
  \overline \Pi_m := \Pi_m|_{\overline G}, 
\]
 which is an irreducible tempered representation of $\overline G$.  



We now consider representations
 of the subgroups $G'=O(2m,1)$
 and $\overline{G'}=SO(2m,1)$.  
We observe that there is at most one
 discrete series representation of $\overline{G'}=SO(n,1)$
 for each infinitesimal character 
(see Proposition \ref{prop:disc} in Appendix I). 
Therefore the restriction of the discrete series representation $\pi_m$
 of $G'=O(2m,1)$ to the subgroup $\overline{G'}=SO(2m,1)$
 is irreducible,
 which is denoted by $\overline{\pi_m}$.  



With these notations,
 Proposition \ref{prop:LNM26} in Appendix II yields
 a natural linear isomorphism:
\[
   {\operatorname{Hom}}_{G'}(\Pi_m|_{G'}, \pi_m)
   \oplus
   {\operatorname{Hom}}_{G'}((\Pi_m\otimes \det)|_{G'}, \pi_m)
   \simeq
   {\operatorname{Hom}}_{\overline{G'}}(\overline \Pi_m|_{\overline{G'}}, \overline{\pi_m}).  
\]
Conjecture \ref{conj:V2} for the pair $(G,G')=(O(n+1,1),O(n,1))$
 is applied to this specific situation;
 the first term in the left-hand side 
 equals
 ${\mathbb{C}}$
 and the second term vanishes.  
Thus Conjecture \ref{conj:V2} in this case implies the following statement
 for the pair $(\overline G,\overline{G'})=(SO(n+1,1),SO(n,1))$
 of special orthogonal groups:
\begin{equation}
\label{eqn:GPm}
   \text{
   ${\operatorname{Hom}}_{\overline{G'}}
   (\overline \Pi_m|_{\overline{G'}}, \overline{\pi_m})={\mathbb{C}}$
   \quad
   \text{if $\mu_{m+1}=0$ and \eqref{eqn:munuGP} is satisfied.}
}
\end{equation}



We now assume that the representation $\Pi_m$ is nontrivial on the center.
This determines the Langlands parameters of  the Vogan packets
 $VP(\overline{\Pi}_m)$ and $VP(\overline{\pi}_m)$ of $\overline G$
 respectively $\overline{G'}$, 
 and we follow exactly the steps of the algorithm by Gross and Prasad outlined
 in Chapter \ref{sec:Gross-Prasad}. 
We conclude again that  the Gross--Prasad conjecture predicts
 that $\{\overline \Pi_m, \overline \pi_m\}$ is the unique pair
 of representation in $VP(\overline \Pi_m)\times VP(\overline \pi_m)$
 with a nontrivial symmetry breaking operator.  




\bigskip
\noindent
\subsubsection{Evidence E.4}
\label{subsec:Evidence4}
We will prove the existence of a nontrivial symmetry breaking operator 
\[
   \Pi _1 \rightarrow \pi_1.  
\]

We first introduce graphs to encode information
 about the images and kernels of symmetry breaking operators
 between reducible principal series representations
 as well as information about the images of the subrepresentation
 under the symmetry breaking operators. 
 This will be helpful to visualize the composition of an symmetry breaking operator with a Knapp--Stein operator.
 
\medskip
\noindent {\bf Admissible graphs} \\
\index{B}{admissiblegraph@admissible graph|textbf} 
Consider the vertices of a square. 
We call the following set of directed  graphs {\it{admissible}}:


\medskip

\begin{center}
\begin{tabular}{c@{\kern2em}c@{\kern2em}c@{\kern2em}c}
	\begin{tabular}{c@{\kern.7em}c@{\kern.7em}c} 
		\blackO   & $\rightarrow$  & \blackO \\
		&  $\nearrow$ & \\
		\blackO & $\rightarrow$ & \blackO
	\end{tabular} &
	\begin{tabular}{c@{\kern.7em}c@{\kern.7em}c} 
		\blackO  & &  \blackO \\
		& $\nearrow$ & \\
		\blackO & $\rightarrow$  & \blackO
	\end{tabular} &
	\begin{tabular}{c@{\kern.7em}c@{\kern.7em}c} 
		\blackO & $\rightarrow$  & \blackO \\
		&  &  \\
		\blackO & $\rightarrow$  & \blackO
	\end{tabular} &
	\begin{tabular}{c@{\kern.7em}c@{\kern.7em}c} 
		\blackO & &  \blackO \\
		&  $\searrow$ & \\
		\blackO & $\rightarrow$  & \blackO
	\end{tabular}
\\[2em]
	\begin{tabular}{c@{\kern.7em}c@{\kern.7em}c} 
		\blackO & $\rightarrow$  & \blackO \\
		&  $\searrow$ & \\
		\blackO & &  \blackO
	\end{tabular} &
	\begin{tabular}{c@{\kern.7em}c@{\kern.7em}c} 
		\blackO &   & \blackO \\
		&  $\searrow$  & \\
		\blackO & & \blackO
	\end{tabular} &
	\begin{tabular}{c@{\kern.7em}c@{\kern.7em}c} 
		\blackO &   & \blackO \\
		&  & \\
		\blackO & $\rightarrow$ & \blackO
	\end{tabular}
\end{tabular}
\end{center}

\noindent
and the zero graph without arrows:

\begin{center}
	\begin{tabular}{c@{\kern.7em}c@{\kern.7em}c} 
		\blackO   &  & \blackO \\
             & \phantom{$\rightarrow$} & \\
            \blackO &  & \blackO
	\end{tabular}
\end{center}
Admissible graphs will encode
 information about the images
 and kernels of symmetry breaking operators.  
In the setting we shall use later,
 it is convenient to define the following equivalence relation 
among graphs,
 see Lemma \ref{lem:SBOadmgraph}.  
\begin{convention}
\label{conv:graph}
We identify two graphs ${\cal G}_1$ and ${\cal G}_2$
 if 
\[
   {\cal G}_1={\cal G}_2 \cup \{\ell\}
\]
 where $\ell$ is an arrow ending
 at the lower right vertex
 and ${\cal G}_2$ already contains an arrow 
 which starts from 
 the same vertex as $\ell$
 and which ends at the upper right vertex.  
\end{convention}
\begin{example}
\label{ex:samegraph}
The following graphs are pairwise equivalent.  
\begin{center}
\begin{tabular}{cccc}
	\begin{tabular}[m]{c@{\kern.7em}c@{\kern.7em}c}
		\blackO & $\to$ & \blackO \\
		 & $\searrow$ & \\ 
		\blackO &  & \blackO
	\end{tabular}& $\equiv$ & 
        \begin{tabular}[m]{c@{\kern.7em}c@{\kern.7em}c} 
		\blackO & $\to$ & \blackO \\
		~\\ 
		\blackO &  & \blackO
	\end{tabular}, & \\[2em]
\hline
\\
	\begin{tabular}[m]{c@{\kern.7em}c@{\kern.7em}c} 
		\blackO & $\to$ & \blackO \\
		~\\ 
		\blackO & $\to$ & \blackO
	\end{tabular} & $\equiv$ & 
	\begin{tabular}[m]{c@{\kern.7em}c@{\kern.7em}c}
		\blackO & $\to$ & \blackO \\
		 & $\searrow$ & \\ 
		\blackO & $\to$ & \blackO
	\end{tabular}, & \\[2em]
\hline
\\
	\begin{tabular}[m]{c@{\kern.7em}c@{\kern.7em}c}
		\blackO & $\to$ & \blackO \\
		 & $\nearrow$ & \\ 
		\blackO & $\to$ & \blackO
	\end{tabular}	& $\equiv$ & 
	\begin{tabular}[m]{c@{\kern.7em}c@{\kern.7em}c}
		\blackO & $\to$ & \blackO \\
		 & $\nearrow\kern-1em\searrow$ & \\ 
		\blackO & $\to$ & \blackO
	\end{tabular}
        & $\equiv$
          \begin{tabular}[m]{c@{\kern.7em}c@{\kern.7em}c}
		\blackO & $\to$ & \blackO \\
		 & $\nearrow$ & \\ 
		\blackO &  & \blackO
	\end{tabular},
\\[2em]
\hline
\\
        \begin{tabular}[m]{c@{\kern.7em}c@{\kern.7em}c}
		\blackO &  & \blackO \\
		 & $\nearrow$ & \\ 
		\blackO & $\to$ & \blackO
	\end{tabular}& $\equiv$ & 
        \begin{tabular}[m]{c@{\kern.7em}c@{\kern.7em}c}
		\blackO &  & \blackO \\
		 & $\nearrow$ & \\ 
		\blackO &  & \blackO
	\end{tabular}. &
\end{tabular}
\end{center}

\end{example}
We obtain a colored graph by coloring  the vertices
 of the graph by 4 different colors, each with a different color. We typically use the colors {\color{blue} blue } and {\color{red} red }
 for the vertices in the left column and {\color{green} green} and {\color{magenta} magenta } for the vertices in the right column.

\medskip
%%%%%%%%%%%%%%%%%%%%%%%%%%%%%%%%%%%%%%%%%%%%%%%
\noindent
{\bf Mutation of  admissible graphs} \\
%%%%%%%%%%%%%%%%%%%%%%%%%%%%%%%%%%%%%%%%%%%%%%%
We obtain a new colored graph ${\cal G}_2$  from a graph ${\cal G}_1$ 
 by 
\index{B}{mutation@mutation|textbf}
\lq\lq{mutation}\rq\rq.  
The rules of the mutation are given as follows.  
\begin{enumerate}
\item
[Rule 1.]  Consider the colored vertices on the right. 
Remove any arrow which ends at the lower right vertex. 
Interchange  the two colored  vertices on the right.  
The arrows which used to end at the upper right vertex now
 end at the lower right vertex.
\item
[Rule 2.] Consider the colored vertices on the left. 
Remove any arrow which starts at the upper left corner. 
Interchange the two colored vertices on the left. 
The arrows which used to start at the lower left vertex now start at the upper left vertex.
\item[Rule 3.] If the mutated graph ${\cal G}_2$ has no arrows, 
 {\it{i.e., }} ${\cal G}_2$ is the zero graph,  the mutation is not allowed.
\end{enumerate}

\medskip
We write {\bf R} for the mutation on the right column and {\bf L}  for the mutation on the left column.  
We sometimes refer to {\bf R} and {\bf L}
 as {\it{mutation rules}}.  

It is easy to see the following.  

\begin{lemma}
\begin{enumerate}
\item[{\rm{(1)}}] The mutated graph is again admissible.  
\item[{\rm{(2)}}] Mutation is well-defined for the equivalence relations
 given in Convention \ref{conv:graph}.  
\item[{\rm{(3)}}] Admissible graphs 
 for which no mutation is allowed
 do not have an arrow except for the one from the upper left vertex to the lower right vertex. 
\item[{\rm{(4)}}]  $\bf R \circ R $ and $\bf L \circ L$ are not allowed mutations. 
\item[{\rm{(5)}}]
 $\bf R \circ L= L \circ R$.
\end{enumerate}
\end{lemma}

\begin{definition}
[source and sink]
We call an admissible graph $\cal G$ a
\index{B}{source@source|textbf}
 {\it{source}} of a set of graphs
 if all other  graphs of the set are obtained through mutations of $\cal G$.
We call a graph $\cal G$ a
\index{B}{sink@sink|textbf}
 {\it{sink}} in a set of admissible graphs
 if neither $\bf R$ nor $\bf L$ is an allowed mutation of $\cal G$.
\end{definition}



Applying these rules,
 we obtain the following families of mutated graphs with one source. 
The source for the first,
 second, and third types is at the top right corner,
 applying {\bf R} changes the right column
 and applying {\bf L}  changes the left column.

\underline{First type} 

\def\Longdownarrow{\rotatebox[origin=c]{90}{$\Longleftarrow$}}
\def\Longuparrow{\rotatebox[origin=c]{90}{$\Longrightarrow$}}

\begin{center}
\begin{tabular}{ccc}
	\begin{tabular}{c@{\kern.7em}c@{\kern.7em}c} 
		\blueO &  & \magentaO \\
		 & $\searrow$  & \\
		 \redO & &  \greenO
	\end{tabular} & 	$\overset{\mbox{\textbf{L}}}{\Longleftarrow}$ &
	\begin{tabular}{c@{\kern.7em}c@{\kern.7em}c} 
		\redO  & $\rightarrow $ & \magentaO  \\
		 & & \\
		 \blueO & $\rightarrow$  & \greenO
	\end{tabular} \\[2em]
	& & \phantom{\textbf{R}}~\Longdownarrow~\textbf{R} \\[1em]
	& & 
	\begin{tabular}{c@{\kern.7em}c@{\kern.7em}c} 
		\redO &  & \greenO  \\
		& $\searrow$ & \\
		\blueO & & \magentaO
	\end{tabular}
\end{tabular}
\end{center}

\underline{Second type}

\begin{center}
\begin{tabular}{ccc}
	\begin{tabular}{c@{\kern.7em}c@{\kern.7em}c} 
		\blueO & $\rightarrow$ & \magentaO  \\
		& $\searrow$ & \\
		\redO & &  \greenO
	\end{tabular} 
	& $\overset{\mbox{\textbf{L}}}{\Longleftarrow}$ &
	\begin{tabular}{c@{\kern.7em}c@{\kern.7em}c} 
		\redO & $\rightarrow$ & \magentaO \\
		& $\nearrow$ & \\
		\blueO &  $\rightarrow$  & \greenO
	\end{tabular} \\[2em]
	\textbf{R}~\Longdownarrow~\phantom{\textbf{R}} & & \phantom{\textbf{R}}~\Longdownarrow~\textbf{R} \\[1em]
	\begin{tabular}{c@{\kern.7em}c@{\kern.7em}c} 
		\blueO & & \greenO \\
		&  $\searrow$  & \\
		\redO & & \magentaO
	\end{tabular} 
& $\underset{\mbox{\textbf{L}}}{\Longleftarrow}$ &
	\begin{tabular}{c@{\kern.7em}c@{\kern.7em}c} 
		\redO & & \greenO \\
		& $\searrow$ & \\
		\blueO & $\rightarrow$ & \magentaO
	\end{tabular} 
\end{tabular}
\end{center}


\underline{Third type}

\begin{center}
\begin{tabular}{ccc}
	\begin{tabular}{c@{\kern.7em}c@{\kern.7em}c} 
		\redO  & $\rightarrow$ & \magentaO  \\
		& $\searrow$ & \\
		\blueO & & \greenO
	\end{tabular}
	& $\overset{\mbox{\textbf{L}}}{\Longleftarrow}$ &
	\begin{tabular}{c@{\kern.7em}c@{\kern.7em}c} 
		\blueO  &  & \magentaO  \\
		& $\nearrow$ & \\
		\redO & $\rightarrow$ & \greenO 
	\end{tabular} 
\\[2em]
	\textbf{R}~\Longdownarrow~\phantom{\textbf{R}} & & \phantom{\textbf{R}}~\Longdownarrow~\textbf{R} \\[1em]	
\begin{tabular}{c@{\kern.7em}c@{\kern.7em}c} 
		\redO  &  & \greenO \\
		 &  $\searrow$  & \\
		 \blueO &  & \magentaO
	\end{tabular} 
& $\underset{\mbox{\textbf{L}}}{\Longleftarrow}$ &
	\begin{tabular}{c@{\kern.7em}c@{\kern.7em}c} 
		\blueO  &  &   \greenO \\
		&  & \\
		\redO & $\rightarrow$  & \magentaO
	\end{tabular}
\end{tabular}
\end{center}

\underline{Type A}

\begin{center}
\begin{tabular}{c}
	\begin{tabular}{c@{\kern.7em}c@{\kern.7em}c} 
		\blueO & $\rightarrow$ &  \magentaO  \\
		& $\searrow$ & \\
		\redO &  & \greenO
	\end{tabular}\\[2em]
 	\phantom{\textbf{R}}~\Longdownarrow~\textbf{R} \\[1em]
	\begin{tabular}{c@{\kern.7em}c@{\kern.7em}c} 
		\blueO  &  & \greenO \\
 		&  $\searrow$  & \\
 		\redO & &  \magentaO
 	\end{tabular} \\[2em]
\end{tabular}
\end{center}

\underline{Type B}

\begin{center}
\begin{tabular}{ccc}
	\begin{tabular}{c@{\kern.7em}c@{\kern.7em}c} 
		\blueO  &  &  \greenO  \\
		& $\searrow$  &    \\
		\redO &  & \magentaO
	\end{tabular}
& $\overset{\mbox{\textbf{L}}}{\Longleftarrow}$ &
	\begin{tabular}{c@{\kern.7em}c@{\kern.7em}c} 
		\redO  &  &\greenO  \\
		& $\searrow$ &  \\
		\blueO & $\rightarrow$ & \magentaO
	\end{tabular} 
\end{tabular}
\end{center}

\underline{Type C}

\begin{center}
\begin{tabular}{ccc}
	\begin{tabular}{c@{\kern.7em}c@{\kern.7em}c} 
		\blueO & & \greenO \\
		& $\searrow$ & \\
		\redO & & \magentaO     
	\end{tabular} 
& $\overset{\mbox{\textbf{L}}}{\Longleftarrow}$ &
	\begin{tabular}{c@{\kern.7em}c@{\kern.7em}c} 
		\redO &  & \greenO \\
		&   & \\
		\blueO & $\rightarrow$ & \magentaO
	\end{tabular}
\end{tabular}
\end{center}

\medskip

\noindent
This proves the following.  
 
\medskip
 
   
\begin{lemma}
\label{lem:graphs}
Let $\cal F$ be the family of admissible graphs
 that are obtained through mutations
 of a nonzero admissible graph.  
\begin{enumerate}
\item[{\rm{(1)}}] 
If $\cal F$ is not a singleton,
 it is one of the above six types.  
\item[{\rm{(2)}}] 
If $\cal F$ is a singleton,
 it is a coloring of the following graph. 

\begin{center}
	\begin{tabular}{c@{\kern.7em}c@{\kern.7em}c} 
		\blackO & & \blackO \\
		& $\searrow$ & \\
		\blackO & & \blackO
	\end{tabular}
\end{center}
\end{enumerate}
\end{lemma}

\medskip
{\bf From symmetry breaking operators to admissible graphs} \\

Assume that a principal series representation $I_{\delta}(V,\lambda)$ of $G$
 has exactly two composition factors $\Pi^1$ and $\Pi^2$, 
 which are not equivalent to each other.  
(The assumption is indeed satisfied for $G=O(n+1,1)$
 whenever $I_{\delta}(V,\lambda)$ is reducible.)
Thus there is an exact sequence of $G$-modules: 
\begin{equation}
  0  \to  {\color{red}\Pi^1}  \to  I_{\delta}(V,\lambda)
     \to {\color{blue}\Pi^2}  \to 0.  
\label{eqn:Inonsplit}
\end{equation}
Graphically,
 the irreducible inequivalent composition factors are represented by circles with different colors. The bottom circle represents the socle
 as follows.  

\begin{center}
	\begin{tabular}{c@{\kern2em}c@{\kern2em}c} 
		\blueO  & &\\
		& &\\
		\redO  & & 
	\end{tabular}
\end{center}
Later we shall assume in addition
 that the exact sequence \eqref{eqn:Inonsplit} does not split. 
(The assumption is satisfied
 if one of $\Pi^1$ or $\Pi^2$ is finite-dimensional.  
More generally,
 the assumption is indeed satisfied 
 for most of the pairs of the composition factors
 of the principal series representations of $G=O(n+1,1)$
 with regular integral infinitesimal characters, 
 see Theorem \ref{thm:LNM20} for example.)


\medskip
An analogous notation will be applied to principal series representations
 $J_{\varepsilon}(W,\nu)$
 of the subgroup $G'=O(n,1)$
 with two composition factors.  
Thus we represent the two composition factors
 of the reducible principal series representations
 $I_\delta(V, \lambda)$ and of $J_\varepsilon(W,\nu)$
 by four differently colored circles
 in a square; both the composition factors
 of a principal series representation are represented by circles vertically. 

We have the convention that the composition factors 
 of the representation $I_\delta(V,\lambda)$ of $G$
 are represented by the circles 
 on the left,
 those of $J_\varepsilon(W,\nu)$ of the subgroup on the right.  
Using this convention we get four squares with colored circles
 which  are obtained by changing the colors in each vertical column.

\bigskip



To a symmetry breaking operator 
\[{\mathbb B}^{V,W}_{\lambda,\nu}: I_\delta(V,\lambda) \rightarrow J_\varepsilon (W,\nu) \] 
we associate a graph
 which encodes information
 about the image and kernel of the symmetry breaking operator
 $\Bbb \lambda \nu {V,W}$
 as well as information about the image of the irreducible subrepresentation of the principal series representation
 $I_{\delta}(V,\lambda)$ of $G$
 under the symmetry breaking operator.  
We proceed as follows:
we obtain the arrows of the graph by considering the action of symmetry breaking operator ${\mathbb B}^{V,W}_{\lambda,\nu}$ on the composition factors. 
If no arrow starts at a circle, 
 then this means that the corresponding composition factor
 is in the kernel of the symmetry breaking operator. 
If no arrow ends at a circle,
 then this means
 that the $G'$-submodule
 of $J_{\varepsilon}(W,\nu)$ corresponding to the circle 
 is not in the image of the symmetry breaking operator. 
Then we have:
\begin{lemma}
\label{lem:SBOadmgraph}
Assume that both principal series representations
 $I_{\delta}(V,\lambda)$ and $J_{\varepsilon}(W,\nu)$
 have exactly two inequivalent composition factors
 with nontrivial extensions.  
Then with Convention \ref{conv:graph} the graph 
associated to our symmetry breaking operator
 $\Bbb \lambda \nu {V,W} \in {\operatorname{Hom}}_{G'}
(I_{\delta}(V,\lambda)|_{G'}, J_{\varepsilon}(W,\nu))$
 is an admissible graph.  
\end{lemma}
The proof of Lemma \ref{lem:SBOadmgraph} is straightforward.  
We illustrate it by examples as below.  
\begin{example}
[Graph of symmetry breaking operators]
\label{ex:SBOgraph}
{\rm{(1)}}\enspace
Suppose that the symmetry breaking operator is surjective and its restriction to the socle {\blueO} is also surjective. 
Then the associated graph is given by 
\begin{center}
\begin{tabular}{c@{\kern.7em}c@{\kern.7em}c} 
		\redO & $\rightarrow$ & \magentaO \\
		& $\nearrow\kern-1em\searrow$ & \\
		\blueO & $\rightarrow$ & \greenO
	\end{tabular}
\end{center}
by definition.  
With Convention \ref{conv:graph}, 
 we have 
\begin{center}
	\begin{tabular}{c@{\kern.7em}c@{\kern.7em}c} 
		\redO & $\rightarrow$ & \magentaO \\
		& $\nearrow$ & \\
		\blueO & $\rightarrow$ & \greenO
	\end{tabular}
$\equiv$
\begin{tabular}{c@{\kern.7em}c@{\kern.7em}c} 
		\redO & $\rightarrow$ & \magentaO \\
		& $\nearrow\kern-1em\searrow$ & \\
		\blueO & $\rightarrow$ & \greenO
	\end{tabular}, 
\end{center}
see Example \ref{ex:samegraph}.  
Then the graph in the left-hand side is admissible.  
\newline
{\rm{(2)}}\enspace
Suppose that the symmetry breaking operator is zero.  
Then it is depicted by the zero graph.  

\begin{center}
	\begin{tabular}{c@{\kern.7em}c@{\kern.7em}c} 
		\redO & & \magentaO \\
		& \phantom{$\rightarrow$} & \\
		\blueO & & \greenO
	\end{tabular}
\end{center}
\end{example}
To reduce the clutter in a digram representing a set of mutated graphs we often omit the zero graph, 
{\it{i.e.,}} the zero symmetry breaking operator. 

\medskip

We would like to encode information about a symmetry breaking operator
 and all its compositions with the Knapp--Stein operators at the same time. 
Composing symmetry breaking operators $\Bbb \lambda \nu {V,W}$
 with a Knapp--Stein intertwining operator 
\[
   \Ttbb \lambda {n-\lambda}V 
   \colon I_\delta(V,\lambda) \rightarrow I_\delta(V,n-\lambda)
\]
 for the group $G$ (see \eqref{eqn:KT}),  respectively 
\[
   \Ttbb \nu {n-1-\nu}W \colon J_\varepsilon(W,\nu) \rightarrow J_\varepsilon(W,n-1-\nu)
\]
 for the subgroup $G'$, 
 we obtain another symmetry breaking operator.  
If this new operator is not zero
 then it can be represented again by an admissible graph.~The graphs of these operators are arranged 
 compatible with our previous article \cite[Figs.~2.1--2.5]{sbon} where we draw $\nu$-value on the $x$-axis
 and the $\lambda$-value on the $y$-axis. 
We place the corresponding symmetry breaking operator in the corresponding quadrant.  
For example,
 if $\lambda \geq \frac{n}{2}$ and $\nu \geq \frac{n-1}{2}$, 
 then the parameters are arranged as 
\begin{alignat*}{2}
& ({n-1-\nu}, \lambda)
\qquad\qquad
&& (\nu,\lambda)
\\
&
&&
\\
& ({n-1-\nu},n-\lambda)
&& (\nu, {n-\lambda})
\end{alignat*}
 in the $(\nu,\lambda)$-plane, 
 and accordingly these symmetry breaking operators
 are arranged as follows.  

\begin{alignat*}{2}
& \Ttbb \nu {n-1-\nu}W \circ \Bbb \lambda \nu  {V,W}
&& \Bbb \lambda \nu {V,W}
\\
&
&&
\\
& \Ttbb \nu {n-1-\nu} W \circ \Bbb \lambda \nu {V,W} 
  \circ \Ttbb {n-\lambda} {\lambda}V
\qquad\qquad\qquad\quad\quad
&& \Bbb \lambda \nu {V,W} \circ \Ttbb {n-\lambda} {\lambda}V
\end{alignat*}


Accordingly,
 we shall consider four graphs
 of these four symmetry breaking operators.  



By the definition of the mutation rule,
 we obtain:
\begin{lemma}
Assume that a principal series representation $I_{\delta}(V,\lambda)$ has 
 two irreducible composition factors $\Pi^1$ and $\Pi^2$
 with nonsplitting exact sequence
 \eqref{eqn:Inonsplit} 
 and that the Knapp--Stein operator
 $\Ttbb \lambda {n-\lambda} V \colon I_{\delta}(V,\lambda) \to 
 I_{\delta}(V,n-\lambda)$ is nonzero 
 but vanishes on the subrepresentation $\Pi^1$.  
Then the graph associated to 
 a symmetry breaking operator composed
 with $\Ttbb {n-\lambda}\lambda V$
 for the group $G$
 is obtained by using the mutation rule ${\mathbf {L}}$ for graphs.  
Similarly,
 the graph associated to a symmetry breaking operator
 composed with a nonzero Knapp--Stein operator
 $\Ttbb \nu{n-1-\nu}W \colon J_{\varepsilon}(W, \nu) 
  \to J_{\varepsilon}(W,n-1-\nu)$
 for the subgroup $G'$
 (with an analogous assumption on $J_{\varepsilon}(W,\nu)$)
 is obtained by using the mutation rule ${\mathbf{R}}$
 for graphs.
\end{lemma}

\begin{example}
In the Memoirs article \cite{sbon}
 we considered the case of two spherical principal series representations $I(\lambda)$ and $J(\nu)$ for integral parameters $i$, $j$.  
If 
\index{A}{Leven@$L_{\operatorname{even}}$}
$(-i,-j)\in L_{\operatorname{even}}$, 
 namely,
 if $i \ge j \ge 0$
 and $i \equiv j \mod 2$, 
 then the normalized regular symmetry breaking operator $I(-i) \rightarrow J(-j)$ is zero \cite[Thm.~8.1]{sbon}. 
The other symmetry breaking operators
 for spherical principal series representations
 with the same infinitesimal character are nonzero
 and we have functional equations with nonvanishing coefficients
 \cite[Thm.~8.5]{sbon}. 
Thus the family of mutated graphs associated to the regular symmetry breaking operators  is 
given as follows.

\medskip

\begin{center}
\begin{tabular}{ccc}
	\begin{tabular}{c@{\kern.7em}c@{\kern.7em}c} 
		\blueO && \magentaO \\
		& $\searrow $ &\\
		\redO && \greenO
	\end{tabular} 
	& $\overset{\mbox{\textbf{L}}}{\Longleftarrow}$ & 
	\begin{tabular}{c@{\kern.7em}c@{\kern.7em}c}
		\redO  & $ \rightarrow $ &  \magentaO\\
		& \phantom{$\searrow$} & \\
		\blueO & $\rightarrow$ & \greenO
	\end{tabular} \\[2em]
	& & \phantom{\textbf{R}}~\Longdownarrow~\textbf{R}\\[1em]
	& & \begin{tabular}{c@{\kern.7em}c@{\kern.7em}c}
		\redO  &  & \greenO\\
		 & $\searrow$ & \\
		 \blueO &  & \magentaO
	\end{tabular}
\end{tabular}
\end{center}

We recall from \cite[Chap.~1]{sbon} 
 (or from Theorem \ref{thm:LNM20} in a more general setting)
 that both the $G$-module $I(-i)$ and the $G'$-module $J(-j)$ contain 
 irreducible finite-dimensional representations
 as their subrepresentations 
 ({\color{red}red} and {\color{magenta}magenta} circles)
 and irreducible infinite-dimensional representations
 $T(i)$ and $T(j)$
 ({\color{blue}blue} and {\color{green}green})
 as their quotients,
 respectively.  
The corresponding socle filtrations are given graphically
 as follows.
\[
  I(-i)=\begin{tabular}{c} 
		\blueO \\
		\\
		\redO
	\end{tabular} 
\qquad\qquad
  J(-j)=\begin{tabular}{c} 
		\greenO \\
		\\
		\magentaO
	\end{tabular}
\]
Note that, 
 under the assumption $i \ge j \ge 0$ and $i \equiv j \mod 2$, 
 we  have a nontrivial symmetry breaking operator
 between the two finite-dimensional representations 
 ({\color{red}red} and {\color{magenta}magenta} circles)
 and as well as between the nontrivial composition factors 
 $T(i) \rightarrow T(j)$ 
 ({\color{blue}blue} and {\color{green}green} circles), 
 see \cite[Thm~1.2 (1-a)]{sbon}. 
\end{example}



\begin{example}
\label{ex:317}
More generally in Corollary \ref{cor:IV1}
 we proved that 
\[
   \Atbb {\lambda_0}{\nu_0} {\gamma} {V,W}= 0 
\]
for negative integers $\lambda_0$, $\nu_0$ implies that 
\[
   \Atbb {n-\lambda_0}{n-1-\nu_0} {\gamma} {V,W} \not = 0.  
\]

Since $({n-1-\nu_0}, {n-\lambda_0}) \in {\mathbb{N}}^2$, 
 we may place the graph associated to the regular symmetry breaking operator 
$ 
   \Atbb {n-\lambda_0}{n-1-\nu_0} {\gamma} {V,W}
$
 in the NE corner
 according to the position 
 in the $(\nu,\lambda)$-plane
 as in \cite[Fig.~2.1, III.A or III.B]{sbon}.  

On the other hand,
 since $(\nu_0, \lambda_0) \in (-{\mathbb{N}})^2$, 
 we may place a zero graph associated to the zero operator
 $\Atbb {\lambda_0} {\nu_0} \gamma{V,W}$
 in the SW corner 
 according to the position
 in the $(\nu,\lambda)$-plane
 as in \cite[Fig.~2.1, I.A. or I.B.]{sbon}.  
\end{example}



\begin{example}
In the Memoirs article \cite[Thm.~11.1]{sbon}
 we prove that there is a differential symmetry breaking operator in the SW corner if the regular symmetry breaking operator is zero. 
To this operator and its composition with the Knapp--Stein operators the assigned graph is given as follows.  

\begin{center}
\begin{tabular}{ccc}
	\begin{tabular}{c@{\kern.7em}c@{\kern.7em}c} 
		\blueO && \magentaO \\
		& $\searrow $ &\\
		\redO && \greenO
	\end{tabular} 
	& $\overset{\mbox{\textbf{L}}}{\Longrightarrow}$ & 
	\begin{tabular}{c@{\kern.7em}c@{\kern.7em}c}
		\redO  &  &  \magentaO\\
		& \phantom{$\searrow$} & \\
		\blueO &  & \greenO
	\end{tabular} \\[2em]
	\textbf{R}~\Longuparrow~\phantom{\textbf{R}} & & \phantom{\textbf{R}}~\Longuparrow~\textbf{R}\\[1em]
	\begin{tabular}{c@{\kern.7em}c@{\kern.7em}c}
		\blueO  &  $\rightarrow$ & \greenO\\
		 &  & \\
		 \redO &  $\rightarrow$ & \magentaO
	\end{tabular} 
	& $\underset{\mbox{\textbf{L}}}{\Longrightarrow}$ &
	\begin{tabular}{c@{\kern.7em}c@{\kern.7em}c}
		\redO  &  & \greenO\\
		 & $\searrow$ & \\
		 \blueO &  & \magentaO
	\end{tabular}
\end{tabular}
\end{center}
Note that the differential operator gives a source 
 in the mutation graphs in the SW corner in this setting.  
\end{example}

\bigskip


%%%%%%%%%%%%
{\bf  Existence of a nontrivial symmetry breaking operators $\Pi_1 \rightarrow \pi_1$.} \\
%%%%%%%%%%%%%%%%%%%%%%%%%%%%%%%
\noindent
Recall that we assume that 
\[m(\Pi_0,\pi_0) =1\]
for the irreducible finite-dimensional representations
 $\Pi_0$ of $G$ and $\pi_0$ of the subgroup $G'$.
We consider now a pair of  reducible principal series representations  $I_\delta(V, \lambda)$ of $G$ and $J_\varepsilon (W,\nu)$ of $G'$
 with finite-dimensional composition factors $\Pi_0$, $\pi_0$, 
 respectively. 
\medskip
\begin{lemma}
 \label{lem:graphs}
Suppose that both \redO{} and \magentaO{} are representing irreducible finite-dimensional representations of $G$ and $G'$. 
We assume that \redO{} and \blueO{}
 respectively \magentaO{} and \greenO{} are representing the composition factors of a principal series representation of $G$, respectively $G'$. 
Then the following  graphs are  not  associated to a symmetry breaking operator.

\begin{center}
\begin{tabular}{c@{\kern2em}c@{\kern2em}c@{\kern2em}c}
	\begin{tabular}{c@{\kern.7em}c@{\kern.7em}c}
		\redO  &  & \magentaO \\
		& $\searrow$ & \\
		\blueO &  & \greenO
	\end{tabular} &
	\begin{tabular}{c@{\kern.7em}c@{\kern.7em}c}
		\redO &  & \magentaO \\
		& $\searrow$  & \\
		\blueO & $\rightarrow$  & \greenO
	\end{tabular} &
	\begin{tabular}{c@{\kern.7em}c@{\kern.7em}c}
		\blueO & & \greenO \\
		& $\nearrow$ & \\
		\redO & $\rightarrow$ & \magentaO
	\end{tabular} &
	\begin{tabular}{c@{\kern.7em}c@{\kern.7em}c}
		\blueO & $\rightarrow$  & \greenO \\
		& $\nearrow$ & \\
		\redO & $\rightarrow$ & \magentaO
	\end{tabular}
\end{tabular}
\end{center}

\end{lemma}

\proof The representations \magentaO{} and \redO{} are
 finite-dimensional. 
The image of a finite-dimensional representation by a symmetry breaking operator is finite-dimensional.  \qed



\medskip

\begin{lemma}
\label{lem:fdgraph}
We keep Convention \ref{conv:graph} and the assumptions of Lemma \ref{lem:graphs}.  
\begin{enumerate}
\item[{\rm{(1)}}]
Suppose that \blueO{} and \greenO{} stand for 
 both irreducible subrepresentations of the principal series representations
 of $G$ and $G'$, respectively. 
The graph associated to a nontrivial symmetry breaking operator
 is one of the following.  

\begin{center}
\begin{tabular}{c@{\kern2em}c@{\kern2em}c@{\kern2em}c}
	\begin{tabular}{c@{\kern.7em}c@{\kern.7em}c}
		\redO &  & \magentaO \\
		& $\nearrow$ & \\
		\blueO & $\rightarrow$ & \greenO
	\end{tabular} & 
	\begin{tabular}{c@{\kern.7em}c@{\kern.7em}c}
		\redO & $\rightarrow$ &\magentaO \\
		& $\nearrow$ & \\
		\blueO & $\rightarrow$ & \greenO
	\end{tabular} & 
	\begin{tabular}{c@{\kern.7em}c@{\kern.7em}c}
		\redO & $\rightarrow$ & \magentaO \\
		& &\\
		\blueO & $\rightarrow$ & \greenO
	\end{tabular} 
\end{tabular}
\end{center}

\item[{\rm{(2)}}] 
Suppose that \redO{} and \magentaO{} stand
 for both irreducible finite-dimensional subrepresentations
 of the principal series representations. The   graph associated to a nontrivial symmetry breaking operator is one of the following.  

\begin{center}
\begin{tabular}{@{}c@{\kern1.4em}c@{\kern1.4em}c@{\kern1.4em}c@{\kern1.4em}c@{}}
	\begin{tabular}{c@{\kern.7em}c@{\kern.7em}c}
		\blueO & $\rightarrow$ & \greenO \\
		&  & \\
		\redO & $\rightarrow$ & \magentaO
	\end{tabular} &
	\begin{tabular}{c@{\kern.7em}c@{\kern.7em}c}
		\blueO & $\rightarrow$ & \greenO \\
		& $\searrow$ & \\
		\redO & & \magentaO
	\end{tabular} &
	\begin{tabular}{c@{\kern.7em}c@{\kern.7em}c}
		\blueO & & \greenO \\
		& $\searrow$ & \\
		\redO & $\rightarrow$ & \magentaO
	\end{tabular} &
	\begin{tabular}{c@{\kern.7em}c@{\kern.7em}c}
		\blueO & & \greenO \\
		& $\searrow$ &\\
		\redO & & \magentaO
	\end{tabular}
\end{tabular}
\end{center}
\end{enumerate}
\end{lemma}


\medskip

\medskip

Using the composition with the Knapp--Stein operators
 we obtain an action
 of the (little) Weyl group of $O(n+1,1) \times O(n,1)$
 on the continuous parameters of the symmetry breaking operators,
 hence on the symmetry breaking operators
 and also on their associated admissible graphs through the mutation rules. 



\begin{example}
Let $\cal F$ be a family of mutated graphs
 such that the graph associated to the symmetry breaking operator 
 $\Atbb {n-\lambda_0}{n-1-\nu_0}{\gamma}{V,W}$
 is a source. 
If $\cal F$ is of first type,
 then the graph in the SE corner shows that there is a nontrivial symmetry breaking operator $\Pi_1 \rightarrow \pi_1$.
\end{example}
 
Using functional equations
 and the information about $(K,K')$-spectrum of regular symmetry breaking operators
 it is in some cases possible (see for example \cite{sbon})
 to show that the associated graph is of first type,
 but in general we do not have such explicit information about the regular symmetry breaking operators and so we have to proceed differently.

\bigskip

Suppose that $\Pi_0$ and $\pi_0$ are
 irreducible finite-dimensional subrepresentations
 of $I_\delta (V,\lambda)$ and $J_\varepsilon (W,\nu)$
 with ${\operatorname{Hom}}_{G'}(\Pi_0|_{G'}, \pi_0) \ne \{0\}$. 
By Proposition \ref{prop:172009}
 $[V:W] \not = 0$
 and
\index{A}{1psi@$\Psising$,
          special parameter in ${\mathbb{C}}^2 \times \{\pm\}^2$}
 $(\lambda, \nu, \delta, \varepsilon) \in \Psi_{\operatorname{sing}}$, 
 namely,
 the quadruple $(\lambda, \nu, \delta, \varepsilon)$
 does not satisfy the generic parameter condition  \eqref{eqn:nlgen}.
By Theorem \ref{thm:1716110} (see also Theorem \ref{thm:existDSBO} (1)), 
 there exists a nonzero differential symmetry breaking operator 
\[
   {\bf D}:I_\delta (V,\lambda) \rightarrow J_\varepsilon (W, \nu), 
\]
which we denote by $\bf D$. 
The image of $\bf D$ is infinite-dimensional
 by Theorem \ref{thm:imgDSBO}.  
Thus by Lemma \ref{lem:fdgraph} (2), 
 we obtain the following.  


\begin{lemma}
The graph associated to $\bf D$ is one of the following.

\begin{center}
\begin{tabular}{c@{\kern2em}c@{\kern2em}c@{\kern2em}c}
	\begin{tabular}{c@{\kern.7em}c@{\kern.7em}c}
		\blueO & $\rightarrow$ & \greenO \\
		& & \\
		\redO & $\rightarrow$ & \magentaO
	\end{tabular} &
	\begin{tabular}{c@{\kern.7em}c@{\kern.7em}c}
		\blueO & $\rightarrow$ & \greenO \\
		& $\searrow$ &\\
		\redO & &\magentaO
	\end{tabular} &  &
\end{tabular}
\end{center}
\end{lemma}

\medskip
Mutating the graph of $\bf D$ by $\bf R$ we get the following.  

\begin{center}
	\begin{tabular}{c@{\kern.7em}c@{\kern.7em}c}
		\blueO &  & \magentaO \\
		& $\searrow$ & \\
		\redO & & \greenO
	\end{tabular}
\end{center}


Thus composing the differential symmetry breaking operator
 with a Knapp--Stein operator
 on the right
 we obtain a nontrivial symmetry breaking operator with this diagram
and thus a symmetry breaking operator  $U_1(F) \rightarrow  U_1(F')$.
We are ready to prove the following theorem,
 which gives evidence of our conjecture.  

\medskip


\begin{theorem}
Suppose that $F$ and $F'$ are irreducible finite-dimensional representations
 of $G$ and $G'$, 
 respectively.  
Let $\Pi_i$, $\pi_j$ be the standard sequences starting at $F$, $F'$, 
respectively.  
Then there exists a nontrivial symmetry breaking operator 
\[
   \Pi_1 \rightarrow \pi_1  
\]
 if ${\operatorname{Hom}}_{G'}(F|_{G'}, F')\ne \{0\}$.  
\end{theorem}
\begin{proof}
Recall from Definition \ref{def:Hasse}
 that  $\Pi_0= F$,  $\pi_0= F'$
 and $\Pi_1 = U_1(F) \otimes \chi_{+-}$,
 $\pi_1 = U_1(F') \otimes \chi_{+-}$
 and so 
\[ \mbox{Hom}_{G'}( \Pi_1|_{G'} , \pi_1)
\simeq \mbox{Hom}_{G'}(U_1(F)|_{G'}, U_1(F')).
\]
\end{proof}



