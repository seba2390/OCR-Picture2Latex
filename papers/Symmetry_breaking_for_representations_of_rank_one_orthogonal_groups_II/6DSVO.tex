\newpage
\section{Differential symmetry breaking operators}
\label{sec:DSVO}

In this chapter,
 we analyze the space
\[
 {\operatorname{Diff}}_{G'}
  (I_{\delta}(V,\lambda)|_{G'},J_{\varepsilon}(W,\nu))
\]
of 
\index{B}{differentialsymmetrybreakingoperators@differential symmetry breaking operator}
 differential symmetry breaking operators
between principal series representations of $G=O(n+1,1)$ and $G'=O(n,1)$
 for arbitrary $V \in \widehat{O(n)}$ and $W \in \widehat{O(n-1)}$
 with 
\index{A}{VWmult@$[V:W]$}
$[V:W]\ne 0$.  



The goal of this chapter is to prove Theorem \ref{thm:existDSBO} below.  
We recall from \eqref{eqn:singset} that 
 the set of \lq\lq{special parameters}\rq\rq\ is denoted by 
\index{A}{1psi@$\Psising$,
          special parameter in ${\mathbb{C}}^2 \times \{\pm\}^2$}
\[
 \Psising
=\{(\lambda,\nu,\delta,\varepsilon) \in {\mathbb{C}}^2 \times \{\pm\}^2:
\text{$\nu-\lambda \in 2 {\mathbb{N}}$
 ($\delta\varepsilon = +$)
 or 
$\nu-\lambda \in 2 {\mathbb{N}}+1$
 ($\delta\varepsilon = -$)}
\}.  
\]
\begin{theorem}
\label{thm:existDSBO}
Let $(G, G')=(O(n+1,1), O(n,1))$.  
Suppose $(\sigma,V) \in \widehat{O(n)}$ and $(\tau,W) \in \widehat{O(n-1)}$
 satisfy $[V:W]\ne 0$.  
\begin{enumerate}
\item[{\rm{(1)}}]
The following two conditions on $\lambda, \nu \in {\mathbb{C}}$
 and $\delta, \varepsilon \in \{\pm\}$
 are equivalent:
\begin{enumerate}
\item[{\rm{(i)}}]
$(\lambda,\nu,\delta,\varepsilon) \in \Psising$.  
\item[{\rm{(ii)}}]
${\operatorname{Diff}}_{G'}
  (I_{\delta}(V,\lambda)|_{G'},J_{\varepsilon}(W,\nu)) \ne \{0\}.
$
\end{enumerate}
\item[{\rm{(2)}}]
If $2 \lambda \not \in {\mathbb{Z}}$ 
then {\rm{(i)}} (or equivalently, (ii) ) implies
\begin{enumerate}
\item[{\rm{(ii)$'$}}]
$
\dim_{\mathbb{C}} {\operatorname{Diff}}_{G'}
  (I_{\delta}(V,\lambda)|_{G'},J_{\varepsilon}(W,\nu)) =1.  
$
\end{enumerate}
\end{enumerate}
\end{theorem}



The implication (ii) $\Rightarrow$ (i) in Theorem \ref{thm:existDSBO}
 holds without the assumption $[V:W] \ne 0$
 as we have seen in Theorem \ref{thm:vanishDiff}.  
Thus the remaining part is to show the opposite implication (i) $\Rightarrow$ (ii)
 and the second statement, 
 which will be carried out
 in Sections \ref{subsec:extDiff} and \ref{subsec:mfDiff}, 
 respectively.  



\begin{remark}
In the setting
 where $(V,W)=(\Exterior^i({\mathbb{C}}^n), \Exterior^j({\mathbb{C}}^{n-1}))$, 
 an explicit construction
 and the complete classification
 of the space
$
  {\operatorname{Diff}}_{G'}
  (I_{\delta}(V,\lambda)|_{G'},J_{\varepsilon}(W,\nu))
$
 were carried out in \cite{KKP}
 without the assumption $[V:W] \ne 0$, 
 see Fact \ref{fact:3.9}.   
\end{remark}

%%%%%%%%%%%%%%%%%%%%%%%%%%%%%%%%%%%%%%%%%%%%%%%%%%%%%%%%
\subsection{Differential operators between two manifolds}
\label{subsec:diff}
%%%%%%%%%%%%%%%%%%%%%%%%%%%%%%%%%%%%%%%%%%%%%%%%%%%%%%%%%

To give a rigorous definition 
 of 
\index{B}{differentialsymmetrybreakingoperators@differential symmetry breaking operator|textbf}
{\it{differential}} symmetry breaking operators,
 we need the notion of differential operators
 between two manifolds, 
 which we now recall.  



For any smooth vector bundle ${\mathcal{V}}$
 over a smooth manifold $X$, 
 there exists the unique (up to isomorphism)
 vector bundle $J^k {\mathcal{V}}$
 over $X$
 (called the $k$-th 
\index{B}{jetprolongation@jet prolongation}
{\it{jet prolongation}} of ${\mathcal{V}}$)
 together with the canonical differential operator
\[
   J^k \colon C^{\infty}(X,{\mathcal{V}}) \to C^{\infty}(X,J^k{\mathcal{V}})
\]
of order $k$.  
We recall that a linear operator
$
   D \colon C^{\infty}(X,{\mathcal{V}}) \to C^{\infty}(X,{\mathcal{V}}')
$
between two smooth vector bundles ${\mathcal{V}}$ and ${\mathcal{V}}'$
 over $X$ is called
 a differential operator
 of order at most $k$, 
 if there is a bundle morphism
$
   Q \colon J^k{\mathcal{V}} \to {\mathcal{V}}'
$
 such that $D=Q_{\ast} \circ J^k$, 
 where $Q_{\ast} \colon C^{\infty}(X,J^k{\mathcal{V}}) \to C^{\infty}(X,{\mathcal{V}}')$ is the induced homomorphism.  
We need a generalization
of this classical definition to the case
 of linear operators
 acting between vector bundles
 over two {\it{different}} smooth manifolds.  
\begin{definition}
[differential operators between two manifolds
 \cite{KOSS, KP1}]
\label{def:diff}
~~~\newline
Suppose that $p \colon Y \to X$
 is a smooth map between two smooth manifolds $Y$ and $X$.  
Let ${\mathcal{V}} \to X$ and ${\mathcal{W}} \to Y$ 
 be  two smooth vector bundles.  
A linear map 
$
   D \colon  C^{\infty}(X,{\mathcal{V}}) \to C^{\infty}(Y,{\mathcal{W}})
$
 is said to be a {\it{differential operator}}
 of order at most $k$
 if there exists a bundle map 
 $Q \colon p^{\ast}(J^k{\mathcal{V}}) \to {\mathcal{W}}$ such that 
\[
   D= Q_{\ast} \circ p^{\ast} \circ J^k.  
\]
\end{definition}
Alternatively, 
one can give the following equivalent definitions
 of differential operators acting between vector bundles
 over two manifolds $Y$ and $X$ with morphism $p$:
\begin{enumerate}
\item[$\bullet$]
based on local properties that generalize Peetre's theorem
 \cite{Pe}
 in the $X=Y$ case
 (\cite[Def.~2.1]{KP1});
\item[$\bullet$]
based on the Schwartz kernel theorem
 (\cite[Lem.~2.3]{KP1});
\item[$\bullet$]
by local expression in coordinates
(\cite[Ex.~2.4]{KP1}).  
\end{enumerate}
Here is a local expression 
 in the case where $p$ is an immersion:
\begin{example}
[{\cite[Ex.~2.4 (2)]{KP1}}]
\label{ex:diffXY}
Suppose that $p \colon Y \hookrightarrow X$
 is an immersion.  
\newline
Choose an atlas
 of local coordinates
 $\{(y_i,z_j)\}$ on $X$
 such that $Y$ is given locally
 by $z_j=0$
 for all $j$.  
Then every differential operator
 $D \colon C^{\infty} (X, {\mathcal{V}}) \to C^{\infty} (Y, {\mathcal{W}})$
 is locally
 of the form
\[
   D=\sum_{\alpha,\beta} g_{\alpha \beta} (y)
     \left. 
     \frac{\partial^{|\alpha|+|\beta|}}{\partial y ^{\alpha}\partial z^{\beta}}
     \right|_{z=0}
\qquad
\text{(finite sum), }
\]
where $g_{\alpha \beta} (y)$ are ${\operatorname{Hom}}(V,W)$-valued smooth functions on $Y$.  
\end{example}

Let $X$ and $Y$ be two smooth manifolds
 acted by $G$ and its subgroup $G'$, 
respectively,
 with a $G'$-equivariant smooth map 
 $p \colon Y \to X$.  
When ${\mathcal{V}} \to X$ is a $G$-equivariant vector bundle
 and ${\mathcal{W}} \to Y$ is a $G'$-equivariant one,
 we denote by 
\[
   {\operatorname{Diff}}_{G'}
   (C^{\infty}(X,{\mathcal{V}})|_{G'}, C^{\infty}(Y,{\mathcal{W}}))   
\]
 the space of differential symmetry breaking operators, 
 namely,
 differential operators
 in the sense of Definition \ref{def:diff}
 that are also $G'$-homomorphisms.  

%%%%%%%%%%%%%%%%%%%%%%%%%%%%%%%%%%%%%%%%%%%%%%%%%%%%
\subsection{Duality for differential symmetry breaking operators}
\label{subsec:dualVerma}
%%%%%%%%%%%%%%%%%%%%%%%%%%%%%%%%%%%%%%%%%%%%%%%%%%%%

We review briefly the duality theorem
 between differential symmetry breaking operators
 and morphisms for branching of generalized Verma modules.  
See \cite[Sect.~2]{KP1} for details.  

Let $G$ be a (real) Lie group.  
We denote by 
\index{A}{envelopingalgebra@$U({\mathfrak{g}})$, enveloping algebra|textbf}
$U({\mathfrak{g}})$
 the universal enveloping algebra of the complexified Lie algebra
 ${\mathfrak{g}}_{\mathbb{C}}={\operatorname{Lie}}(G) \otimes_{{\mathbb{R}}}
 {\mathbb{C}}$.  
Analogous notations will be applied to other Lie groups.  


Let $H$ be a (possibly disconnected) closed subgroup of $G$.  
Given a finite-dimensional representation $F$ of $H$, 
we set 
\index{A}{indhgV@${\operatorname{ind}}_{\mathfrak{h}}^{\mathfrak{g}}(V)
=U({\mathfrak{g}}) \otimes_{U({\mathfrak{h}})}V$|textbf}
\begin{equation}
\label{eqn:VermaH}
{\operatorname{ind}}_{\mathfrak{h}}^{\mathfrak{g}}(F)
:=U({\mathfrak{g}}) \otimes_{U({\mathfrak{h}})}F.  
\end{equation}



The diagonal $H$-action 
 on the tensor product $U({\mathfrak{g}}) \otimes_{\mathbb{C}} F$
 induces an action of $H$
 on $U({\mathfrak{g}}) \otimes_{U({\mathfrak{h}})} F$, 
 and thus ${\operatorname{ind}}_{\mathfrak{h}}^{\mathfrak{g}}(F)$
 is endowed with a $({\mathfrak{g}},H)$-module structure.  



When $X$ and $Y$ are homogeneous spaces $G/H$ and $G'/H'$, 
respectively,
 with $G' \subset G$ and $H' \subset H \cap G'$, 
 we have a natural $G'$-equivariant smooth map
 $G'/H' \to G/H$ induced from 
the inclusion map $G' \hookrightarrow G$.  
In this case, 
 the following duality theorem
 (\cite[Thm.~2.9]{KP1}, 
 see also \cite[Thm.~2.4]{KOSS})
 is a generalization
 of the classical duality 
 in the case
 where $G=G'$ are complex reductive Lie groups
 and $H=H'$ are Borel subgroups:

\begin{fact}
[duality theorem]
\label{fact:dualityDSV}
Let $F$ and $F'$ be finite-dimensional representations
 of $H$ and $H'$, 
respectively,
 and we define equivariant vector bundles ${\mathcal{V}}= G \times_H F$
 and ${\mathcal{W}}= G' \times_{H'} F'$
 over $X$ and $Y$, 
 respectively.  
Then there is a canonical linear isomorphism:
\begin{equation}
\label{eqn:dualDSV}
   {\operatorname{Hom}}_{{\mathfrak{g}}',H'}
   ({\operatorname{ind}}_{\mathfrak{h}'}^{\mathfrak{g}'}
   (F'^{\vee}), 
    {\operatorname{ind}}_{\mathfrak{h}}^{\mathfrak{g}}
    (F^{\vee})|_{{\mathfrak{g}}',H'}
    )
\simeq
   {\operatorname{Diff}}_{G'}
   (C^{\infty}(X,{\mathcal{V}})|_{G'}, C^{\infty}(Y,{\mathcal{W}})).  
\end{equation}
\end{fact}
Applying Fact \ref{fact:dualityDSV}
 to our special setting, 
 we obtain the following:
\begin{proposition}
\label{prop:dualityDSV}
Let $(G, G')=(O(n+1,1), O(n,1))$, 
 $V \in \widehat{O(n)}$, $W \in \widehat{O(n-1)}$,
 $\lambda, \nu\in {\mathbb{C}}$, 
 and $\delta, \varepsilon \in \{\pm\}$.  
Let 
\index{A}{Vln@$V_{\lambda,\delta}=V \otimes \delta \otimes {\mathbb{C}}_{\lambda}$, representation of $P$}
$V_{\lambda,\delta}= V \otimes \delta \otimes {\mathbb{C}}_{\lambda}$
 be the irreducible representation of $P$
 with trivial $N_+$-action as before, 
 and $V_{\lambda,\delta}^{\vee}$ the contragredient representation.  
Similarly,
\index{A}{Wnuepsi@$W_{\nu,\varepsilon}=W \otimes \varepsilon \otimes {\mathbb{C}}_{\nu}$, representation of $P'$}
 $W_{\nu,\varepsilon}^{\vee}$ be the contragredient $P'$-module
 of $W_{\nu,\varepsilon}=W \otimes \varepsilon \otimes {\mathbb{C}}_{\nu}$.  
Then there is a canonical linear isomorphism:
\begin{equation}
\label{eqn:dualVW}
  {\operatorname{Hom}}_{{\mathfrak{g}}',P'}
  ({\operatorname{ind}}_{\mathfrak{p}'}^{\mathfrak{g}'}
   (W_{\nu,\varepsilon}^{\vee}), 
    {\operatorname{ind}}_{\mathfrak{p}}^{\mathfrak{g}}
    (V_{\lambda,\delta}^{\vee})|_{{\mathfrak{g}}',P'}
    )
\simeq
  {\operatorname{Diff}}_{G'}
   (I_{\delta}(V,\lambda)|_{G'},J_{\varepsilon}(W,\nu)).  
\end{equation}
\end{proposition}

%%%%%%%%%%%%%%%%%%%%%%%%%%%%%%%%%%%%%%%%%%
\subsection{Parabolic subgroup compatible 
 with a reductive subgroup}
\label{subsec:PGcompati}
%%%%%%%%%%%%%%%%%%%%%%%%%%%%%%%%%%%%%%%%%%%
In this section we treat the general setting 
 where $G$ is a real reductive Lie group
 and $G'$ is a reductive subgroup,
 and study basic properties
 of differential symmetry breaking operators
 between principal series representation $\Pi$
 of $G$
 and $\pi$ of the subgroup $G'$.  
We shall prove in Theorem \ref{thm:imgDSBO} below
 that the image of any nonzero differential symmetry breaking operator
 is infinite-dimensional 
 if $\Pi$ is induced from a parabolic subgroup $P$
 which is compatible with the subgroup $G'$
 (see Definition \ref{def:compatiP}).  



Let us give a basic setup.  
Suppose that $G$ is a real reductive Lie group
 with Lie algebra ${\mathfrak{g}}$.  
Take a hyperbolic element $H$ of ${\mathfrak{g}}$, 
 and we define the direct sum decomposition, 
 referred sometimes to as the Gelfand--Naimark decomposition
 (cf. \cite{GN}):
\[
   {\mathfrak{g}}={\mathfrak{n}}_- + {\mathfrak{l}} + {\mathfrak{n}}_+
\]
 where ${\mathfrak{n}}_-$, ${\mathfrak{l}}$, and ${\mathfrak{n}}_+$
 are the sum of eigenspaces
 of ${\operatorname{ad}}(H)$ with negative, zero and positive eigenvalues, 
respectively.  
We define a parabolic subgroup $P \equiv P(H)$ of $G$ by
\[
  P=L N_+
\qquad
 \text{(Levi decomposition)}, 
\]
 where $L = \{g \in G: {\operatorname{Ad}}(g)H=H\}$
 and $N_+=\exp ({\mathfrak{n}}_+)$.  
The following \lq\lq{compatibility}\rq\rq\ gives a sufficient condition
 for the \lq\lq{discrete decomposability}\rq\rq\
 of the generalized Verma module ${\operatorname{ind}}_{\mathfrak{p}}^{\mathfrak{g}}(V^{\vee})$
 when restricted to the subalgebra ${\mathfrak{g}}'$, 
 which concerns with the left-hand side
 of the duality \eqref{eqn:dualDSV}
 (see \cite[Thm.~4.1]{xktransgp12}):
\begin{definition}
[{\cite{xktransgp12}}]
\label{def:compatiP}
Suppose $G'$ is a reductive subgroup of $G$ with Lie algebra ${\mathfrak{g}}'$. A parabolic subgroup $P$ of $G$ is said to be $G'$-{\it{compatible}}
 if there exists a hyperbolic element $H$ in ${\mathfrak{g}}'$
 such that $P=P(H)$.  
\end{definition}

If $P$ is $G'$-compatible, 
 then $P' :=P \cap G'$
 is a parabolic subgroup of the reductive subgroup $G'$
 with Levi decomposition $P' = L' N_+'$
 where $L' :=L \cap G'$ and $N_+' :=N_+ \cap G'$.  

\begin{theorem}
\label{thm:imgDSBO}
Let $G$ be a real reductive Lie group,
 $P$ a parabolic subgroup 
 which is compatible with a reductive subgroup $G'$, 
 and $P' := P \cap G'$.  
Suppose that ${\mathcal{V}}$ is a $G$-equivariant vector bundle
 of finite rank over the real flag manifold $G/P$, 
and that ${\mathcal{W}}$ is a $G'$-equivariant one over $G'/P'$.  
Then for any nonzero differential operator
 $D \colon C^{\infty}(G/P,{\mathcal{V}}) \to C^{\infty}(G'/P',{\mathcal{W}})$, 
 we have 
\[
  \dim_{\mathbb{C}} {\operatorname{Image}} D= \infty.  
\]
\end{theorem}
As we shall see
 in the proof below,
Theorem \ref{thm:imgDSBO}
 follows from the definition
of differential operators (Definition \ref{def:diff})
 without the assumption
 that $D$ intertwines the $G'$-action.  

\begin{proof}
[Proof of Theorem \ref{thm:imgDSBO}]
We set $Y=G'/P'$ and $X=G/P$.  
Then $Y \subset X$ because $P' = P \cap G'$.  
There exist countably many disjoint open subsets $\{U_j\}$ of $X$
 such that $Y \cap U_j \ne \emptyset$.  
It suffices to show
 that for every $j$ there exists $\varphi_j \in C^{\infty}(X,{\mathcal{V}})$
 such that ${\operatorname{Supp}} (\varphi_j) \subset U_j$
 and $D \varphi_j \ne 0$
 because ${\operatorname{Supp}} (D \varphi_j) \subset U_j \cap Y$
 and because $\{U_j \cap Y\}$ is a set of disjoint open sets of $Y$.  
We fix $j$, 
 and write $U$ simply for $U_j$.  
By shrinking $U$ if necessary,
 we trivialize the bundles ${\mathcal{V}}|_U$ and ${\mathcal{W}}|_{U \cap Y}$.  Then we see from Example \ref{ex:diffXY}
 that $D$ can be written locally as the matrix-valued operators:
\[
   D=\sum_{\alpha,\beta} g_{\alpha \beta} (y)
     \left. 
     \frac{\partial^{|\alpha|+|\beta|}}{\partial y ^{\alpha}\partial z^{\beta}}
     \right|_{z=0}.  
\]
Take a multi-index $\beta$
 such that $g_{\alpha \beta} (0) \ne 0$
 on $U$ for some $\alpha$.  
We fix $\alpha$ such that $|\alpha|=\alpha_1+ \cdots +\alpha_{\dim Y}$
attains its maximum 
 among all multi-indeces $\alpha$ with $g_{\alpha \beta} (y) \not \equiv 0$.  
Take $v$ in the typical fiber ${\mathcal{V}}$ at $(y, z)=(0,0)$
 such that $g_{\alpha \beta}(0) v \ne 0$.  
By using a cut function,
 we can construct easily $\varphi \in C^{\infty}(X, {\mathcal{V}})$ 
 such that ${\operatorname{Supp}}(\varphi) \subset U$
 and that $\varphi(y,z) \equiv y^{\alpha} z^{\beta} v$
 in a neighbourhood of $(y,z)=(0,0)$.  
Then we have 
\[
  D \varphi \ne 0.  
\]
Thus Theorem \ref{thm:imgDSBO} is proved.  
\end{proof}

%%%%%%%%%%%%%%%%%%%%%%%%%%%%%%%%%%%%%%%%%%%%%%%%%%%%%%%%%%%%%%%%%%%%%%%%%%%%%%
\subsection{Character identity for branching in the parabolic BGG category}
\label{subsec:GroOp}
%%%%%%%%%%%%%%%%%%%%%%%%%%%%%%%%%%%%%%%%%%%%%%%%%%%%%%%%%%%%%%%%%%%%%%%%%%%%%%%

We retain the general setting 
 as in Section \ref{subsec:PGcompati}, 
 and discuss the duality theorem 
 in Section \ref{subsec:dualVerma}.  
To study the left-hand side of \eqref{eqn:dualVW}, 
 we use the results \cite{xktransgp12, KOSS} on the restriction
 of parabolic Verma modules
 ${\operatorname{ind}}_{\mathfrak{p}}^{\mathfrak{g}}(F)$
  with respect to a reductive subalgebra ${\mathfrak{g}}'$
 under the assumption
 that ${\mathfrak{p}}$ is compatible with ${\mathfrak{g}}'$.  
For later purpose, 
 we need to formulate the results
 in \cite{xktransgp12, KOSS}
 in a slightly more general form as below, 
 because a parabolic subgroup $P$ of a real reductive Lie group
 is not always connected.  




Suppose that $P =L N_+$ is a parabolic subgroup of $G$
 which is compatible with a reductive subgroup $G'$.  
We set ${\mathfrak{n}}_-':= {\mathfrak{n}}_- \cap {\mathfrak{g}}'$.  
Then the $L'$-module structure on the nilradical ${\mathfrak{n}}_-$
 descends to the quotient ${\mathfrak{n}}_-/{\mathfrak{n}}_-'$, 
 and extends to the (complex) symmetric tensor algebra
 $S(({\mathfrak{n}}_-/{\mathfrak{n}}_-') \otimes_{\mathbb{R}} {\mathbb{C}})$.  



For an irreducible $L$-module $F$
 and an irreducible $L'$-module $F'$, 
 we set 
\begin{equation}
\label{eqn:nFH}
  n(F,F') :=
  \dim_{\mathbb{C}}{\operatorname{Hom}}_{L'}
  (F',F|_L \otimes S(({\mathfrak{n}}_-/{\mathfrak{n}}_-') \otimes_{\mathbb{R}} {\mathbb{C}})).  
\end{equation}



Then we have the following branching rule
 in the Grothendieck group 
 of the parabolic BGG category
 of $({\mathfrak{g}}',P')$-modules
 ({{\cite[Prop.~5.2]{xktransgp12}},{\cite[Thm.~3.5]{KOSS}}}):

\begin{fact}
[character identity for branching to a reductive subalgebra]
\label{fact:Grobranch}
Suppose that $P=L N_+$ is a $G'$-compatible parabolic subgroup of $G$
 (Definition \ref{def:compatiP}).  
Let $F$ be an irreducible finite-dimensional $L$-module.  
\begin{enumerate}
\item[{\rm{(1)}}]
$n(F,F')<\infty$ for all irreducible finite-dimensional $L'$-modules $F'$.  
\item[{\rm{(2)}}]
We inflate $F$ to a $P$-module 
 by letting $N_+$ act trivially,
 and form a $({\mathfrak{g}},P)$-module
 ${\operatorname{ind}}_{\mathfrak{p}}^{\mathfrak{g}}(F)
  = U({\mathfrak{g}}) \otimes_{U(\mathfrak{p})} F$.  
Then we have the following identity
 in the Grothendieck group
 of the parabolic BGG category
 of $({\mathfrak{g}}',P')$-modules:
\[
{\operatorname{ind}}_{\mathfrak{p}}^{\mathfrak{g}}(F)|_{{\mathfrak{g}}',P'}
\simeq
\bigoplus_{F'}
n(F,F') 
{\operatorname{ind}}_{\mathfrak{p}'}^{\mathfrak{g}'}(F').  
\]
In the right-hand side, 
 $F'$ runs over all irreducible finite-dimensional $P'$-modules, 
 or equivalently, 
 all irreducible finite-dimensional $L'$-modules
 with trivial $N_+'$-actions.  
\end{enumerate}
\end{fact}

\begin{proof}
The argument is parallel to the one
 in \cite[Thm.~3.5]{KOSS}
 for $({\mathfrak{g}}',{\mathfrak{p}}')$-modules,
 which is proved by using \cite[Prop.~5.2]{xktransgp12}.  
\end{proof}
%%%%%%%%%%%%%%%%%%%%%%%%%%%%%%%%%%%%%%%%%%%%%%%%%%%%%%%%%%
\subsection{Branching laws for generalized Verma modules}
\label{subsec:branchVerma}
%%%%%%%%%%%%%%%%%%%%%%%%%%%%%%%%%%%%%%%%%%%%%%%%%%%%%%%%%%%

In this section we refine the character identity
 (identity in the Grothendieck group)
 in Section \ref{subsec:GroOp}
 to obtain actual branching laws.  
The idea works in the general setting
 (cf.~ \cite[Sect.~3]{KOSS}), 
however,
 we confine ourselves with the pair $(G,G')=(O(n+1,1),O(n,1))$
 for actual computations below.  
In particular,
 under the assumption $2 \lambda \not \in {\mathbb{Z}}$, 
 we give an explicit irreducible decomposition 
 of the $({\mathfrak{g}},P)$-module
$
   {\operatorname{ind}}_{\mathfrak{p}}^{\mathfrak{g}}(V_{\lambda,\delta}^{\vee})
$
 when we regard it 
 as a $({\mathfrak{g}}',P')$-module:

\begin{theorem}
[branching law for generalized Verma modules]
\label{thm:Vermadeco}
Let $V \in \widehat {O(n)}$, 
 $\lambda \in {\mathbb{C}}$, 
 and $\delta \in \{\pm\}$.  
Assume $2 \lambda \not \in {\mathbb{Z}}$.  
Then the $({\mathfrak{g}},P)$-module 
 ${\operatorname{ind}}_{\mathfrak{p}}^{\mathfrak{g}}
  (V_{\lambda,\delta}^{\vee})$ decomposes
 into the multiplicity-free direct sum of irreducible $({\mathfrak{g}}',P')$-modules
 as follows:
\begin{equation}
\label{eqn:Vermadeco}
 {\operatorname{ind}}_{\mathfrak{p}}^{\mathfrak{g}}
  (V_{\lambda,\delta}^{\vee})|_{{\mathfrak{g}}',P'}
  \simeq
  \bigoplus_{a=0}^{\infty}
  \bigoplus_{[V:W]\ne 0}
  {\operatorname{ind}}_{\mathfrak{p}'}^{\mathfrak{g}'}
   ((W_{\lambda+a,(-1)^a\delta})^{\vee}).  
\end{equation}
Here $W$ runs over all irreducible $O(n-1)$-modules
 such that $[V:W] \ne 0$.  
\end{theorem}

\begin{proof}
[Proof of Theorem \ref{thm:Vermadeco}]
The hyperbolic element $H$ defined in \eqref{eqn:Hyp}
 is contained in ${\mathfrak{g}}' = {\mathfrak{o}}(n,1)$, 
 and therefore, 
 the parabolic subgroup $P$
 is compatible with the reductive subgroup $G'=O(n,1)$
 in the sense of Definition \ref{def:compatiP}.  
We then apply Fact \ref{fact:Grobranch}
 to 
\[
  (F,{\mathfrak{n}}_-,{\mathfrak{n}}_-')
  =
  (V_{\lambda,\delta}^{\vee}, \sum_{j=1}^{n}{\mathbb{R}}N_j^-, \sum_{j=1}^{n-1}{\mathbb{R}}N_j^-).  
\]
Since ${\mathfrak{n}}_-/{\mathfrak{n}}_-' \simeq {\mathbb{R}}N_n^-$, 
 the $a$-th symmetric tensor space amounts to 
\[
  S^a(({\mathfrak{n}}_-/{\mathfrak{n}}_-') \otimes_{\mathbb{R}} {\mathbb{C}})
   \simeq
  {\bf{1}} \boxtimes (-1)^a \boxtimes {\mathbb{C}}_{-a}
\]
 as a module of $L' \simeq O(n-1) \times O(1) \times {\mathbb{R}}$.  
Therefore we have an $L'$-isomorphism:
\[
F|_{L'} \otimes S^a(({\mathfrak{n}}_-/{\mathfrak{n}}_-') \otimes_{\mathbb{R}} {\mathbb{C}})
\simeq
\bigoplus_{\substack {W \in \widehat{O(n-1)} \\ [V:W] \ne 0}} W^{\vee} \boxtimes (-1)^a\delta \boxtimes {\mathbb{C}}_{-\lambda-a}, 
\]
where we observe $[V^{\vee}:W^{\vee}] \ne 0$
 if and only if $[V:W] \ne 0$.  
Thus the identity \eqref{eqn:Vermadeco}
 in the level of the Grothendieck group
 of $({\mathfrak{g}}',P')$-modules
 is deduced from Fact \ref{fact:Grobranch}.  



In order to prove the identity \eqref{eqn:Vermadeco}
 as $({\mathfrak{g}}',P')$-modules, 
 we use the following two lemmas.  
\end{proof}

\begin{lemma}
\label{lem:indinf}
Assume $2 \lambda \not \in {\mathbb{Z}}$.  
Then any ${\mathfrak{Z}}({\mathfrak{g}}')$-infinitesimal characters
 of the summands in \eqref{eqn:Vermadeco}
 are all distinct.  
\end{lemma}

\begin{lemma}
\label{lem:indirred}
Assume $2 \lambda \not \in {\mathbb{Z}}$.  
Then any summand 
$
   {\operatorname{ind}}_{\mathfrak{p}'}^{\mathfrak{g}'}
   ((W_{\lambda+a,(-1)^a \delta})^{\vee})
$
 in \eqref{eqn:Vermadeco}
 is irreducible as a $({\mathfrak{g}}',P')$-module.  
\end{lemma}

\begin{proof}
[Proof of Lemma \ref{lem:indinf}]
Via the Cartan--Weyl bijection \eqref{eqn:CWOn}
 for the disconnected group $O(N)$
 ($N=n,n-1$), 
 we write $V=\Kirredrep{O(n)}{\mu}$
 and $W=\Kirredrep{O(n-1)}{\mu'}$
 for $\mu=(\mu_1, \cdots,\mu_n) \in \Lambda^{+}(O(n))$
 and $\mu'=(\mu_1', \cdots,\mu_{n-1}') \in \Lambda^{+}(O(n-1))$.  
By the classical 
\index{B}{branching rule@branching rule, for $O(N) \downarrow O(N-1)$}
branching law
 for the restriction $O(n) \downarrow O(n-1)$
 (Fact \ref{fact:ONbranch}), 
 $[V:W] \ne 0$
 if and only if
\begin{equation}
\label{eqn:mubranch}
  \mu_1 \ge \mu_1' \ge \mu_2 \ge \cdots \ge \mu_{n-1}' \ge \mu_{n}.   
\end{equation}
Since any irreducible $O(N)$-module
 is self-dual, 
 we have $W^{\vee} \simeq \Kirredrep{O(n-1)}{\mu'}$.  
Therefore, 
 the 
\index{B}{infinitesimalcharacter@infinitesimal character}
${\mathfrak{Z}}({\mathfrak{g}}')$-infinitesimal character
 of the ${\mathfrak{g}}'$-module 
${\operatorname{ind}}_{\mathfrak{p}'}^{\mathfrak{g}'}
 (W^{\vee} \otimes (-1)^a\delta \otimes {\mathbb{C}}_{-\lambda-a})
$ 
 is given by
\[
  (-\lambda-a+\frac{n-1}{2}, 
   \mu_1'+\frac{n-3}{2},
   \mu_2'+ \frac{n-5}{2}, 
   \cdots, 
   \mu_{[\frac{n-1}{2}]}'+\frac{n-1}{2}- [\frac{n-1}{2}])
\]
 modulo the Weyl group
 ${\mathfrak{S}}_m \ltimes ({\mathbb{Z}}/2 {\mathbb{Z}})^m$
 for the disconnected group $G'=O(n,1)$
 where $m=[\frac{n+1}{2}]$.  
Hence,
 if $2 \lambda \not \in {\mathbb{Z}}$, 
they are all distinct
 when $a$ runs over ${\mathbb{N}}$
 and $\mu'$ runs over $\Lambda^+(O(n-1))$
 subject to \eqref{eqn:mubranch}.  
Thus Lemma \ref{lem:indinf} is proved.  
\end{proof}
\begin{proof}
[Proof of Lemma \ref{lem:indirred}]
By the criterion of Conze-Berline and Duflo
 \cite{BeDu}, 
 the ${\mathfrak{g}}'$-module ${\operatorname{ind}}_{\mathfrak{p}'}^{\mathfrak{g}'}
 (\tau_{\nu} \otimes {\mathbb{C}}_{-\lambda-a})
$ 
 is irreducible
 if $\tau_{\nu}$ is an irreducible ${\mathfrak{so}}(n-1)$-module
 with highest weight $(\nu_1, \cdots,\nu_{[\frac{n-1}{2}]})$
 satisfying
\[
\langle
  -\lambda-a+\frac{n-1}{2}, 
   \nu_1+\frac{n-3}{2},
   \nu_2+ \frac{n-5}{2}, 
   \cdots, 
   \nu_{[\frac{n-1}{2}]}+\frac{n-1}{2}- [\frac{n-1}{2}], 
   \beta^{\vee}
\rangle
  \not \in{\mathbb{N}}_+, 
\]
 where $\beta^{\vee}$ is the coroot of $\beta$, 
 and $\beta$ runs over the set 
\[
\Delta^+({\mathfrak{g}}_{\mathbb{C}}) 
 \setminus \Delta^+({\mathfrak{l}}_{\mathbb{C}})
 =\{e_1 \pm e_j : 2 \le j \le [\frac{n+1}{2}]\}
 (\cup \{e_1\}, 
 \text{when $n$ is even}).  
\]
This condition is fulfilled
 if $2 \lambda \not \in {\mathbb{Z}}$
 because $\nu_1$, $\cdots$, $\nu_{[\frac{n-1}{2}]}$
 $\in \frac 1 2 {\mathbb{Z}}$
 and $a \in {\mathbb{N}}$.  
Hence ${\operatorname{ind}}_{\mathfrak{p}'}^{\mathfrak{g}'}
 ((W_{\lambda+a,(-1)^a\delta})^{\vee})
$
 is an irreducible ${\mathfrak{g}}'$-module
 if $W^{\vee}(\simeq W) \in O(n-1)$
 is of type X
 (Definition \ref{def:OSO}), 
 namely,
 if $W^{\vee}$ is irreducible 
 as an ${\mathfrak{so}}(n-1)$-module.  
On the other hand, 
if $W^{\vee} \in O(n-1)$ is of type Y, 
 then ${\operatorname{ind}}_{\mathfrak{p}'}^{\mathfrak{g}'}
 ((W_{\lambda+a,(-1)^a\delta})^{\vee})
$ splits into the direct sum
 of two irreducible ${\mathfrak{g}}'$-module
 according to the decomposition 
 of $W^{\vee}$
 into irreducible ${\mathfrak{so}}(n-1)$-modules.  
Since these two ${\mathfrak{g}}'$-submodules 
 are not stable
 by the $L'$-action,
 we conclude
 that ${\operatorname{ind}}_{\mathfrak{p}'}^{\mathfrak{g}'}
 ((W_{\lambda+a,(-1)^a\delta})^{\vee})
$
 is irreducible as a $({\mathfrak{g}'},L')$-module,
 in particular,
 as a $({\mathfrak{g}'},P')$-module.  
Thus Lemma \ref{lem:indirred} is proved.  
\end{proof}
%%%%%%%%%%%%%%%%%%%%%%%%%%%%%%%%%%%%%%%%%%%%%%%%%%%
\subsection{Multiplicity-one theorem for differential symmetry breaking operators: Proof of Theorem \ref{thm:existDSBO} (2)}
\label{subsec:mfDiff}
%%%%%%%%%%%%%%%%%%%%%%%%%%%%%%%%%%%%%%%%%%%%%%%%%%%
Combining Proposition \ref{prop:dualityDSV} (duality theorem)
 with the branching law
 for generalized Verma modules
 (Theorem \ref{thm:Vermadeco}), 
 we obtain a 
\index{B}{genericmultiplicityonetheoremforDSBO@
generic multiplicity-one theorem,
 for differential symmetry breaking operator}
generic multiplicity-one theorem
 for differential symmetry breaking operators
 as follows:
\begin{corollary}
\label{cor:DSVOgeneric}
Suppose $V \in \widehat {O(n)}$ and $W \in \widehat{O(n-1)}$
 satisfy $[V:W] \ne 0$.  
Suppose that $(\lambda, \nu, \delta, \varepsilon) \in \Psising$
 (see \eqref{eqn:singset}).  
Assume further $2 \lambda \not \in {\mathbb{Z}}$.  
Then 
\[
   \dim_{\mathbb{C}} 
   {\operatorname{Diff}}_{G'}
   (I_{\delta}(V,\lambda)|_{G'},J_{\varepsilon}(W,\nu))
   =1. 
\]
\end{corollary}
This gives a proof of the second statement
 of Theorem \ref{thm:existDSBO}.  

%%%%%%%%%%%%%%%%%%%%%%%%%%%%%%%%%%%%%%%%%%%%%%%%
\subsection{Existence of differential symmetry breaking operators:
Extension to special parameters}
\label{subsec:extDiff}
%%%%%%%%%%%%%%%%%%%%%%%%%%%%%%%%%%%%%%%%%%%%%%%%%
What remains to prove is the implication (i) $\Rightarrow$ (ii)
 in Theorem \ref{thm:existDSBO}
 for special parameters,
 namely,
 for $2 \lambda \in {\mathbb{Z}}$.  
We shall use the general idea given in \cite[Lem.~11.10]{sbon}
 and deduce the implication (i) $\Rightarrow$ (ii)
 for the special parameters from Corollary \ref{cor:DSVOgeneric}
 for the regular parameters,  
 and thus complete the proof of Theorem \ref{thm:existDSBO} (1).  



Let ${\operatorname{Diff}}^{\operatorname{const}}({\mathfrak{n}}_{-})$
 denote the ring of holomorphic differential operators
 on ${\mathfrak{n}}_{-}$
 with constant coefficients
 and $\langle \,\, , \,\, \rangle$ denote the natural pairing 
 ${\mathfrak{n}}_-=\sum_{j=1}^{n}{\mathbb{R}} N_j^-$
 and ${\mathfrak{n}}_+=\sum_{j=1}^{n}{\mathbb{R}} N_j^+$.  
Then the symbol map
\[
  {\operatorname{Symb}} \colon 
  {\operatorname{Diff}}^{\operatorname{const}}({\mathfrak{n}}_{-})
  \to 
  {\operatorname{Pol}}({\mathfrak{n}}_{+}), 
  \quad
  D_z \mapsto Q(\zeta)
\]
 given by the characterization
\[
   D_z e^{\langle z, \zeta \rangle}
   =
   Q(\zeta) e^{\langle z, \zeta \rangle}
\]
 is a ring isomorphism 
 between ${\operatorname{Diff}}^{\operatorname{const}}({\mathfrak{n}}_{-})$
 and the polynomial ring ${\operatorname{Pol}}({\mathfrak{n}}_{+})$.  



The 
\index{B}{Fmethod@F-method}
F-method (\cite[Thm.~4.1]{KP1})
 characterizes the \lq\lq{Fourier transform}\rq\rq\
 of differential symmetry breaking operators
 by certain systems
 of differential equations.  
It tells that 
 any element in ${\operatorname{Diff}}_{G'}
   (I_{\delta}(V,\lambda)|_{G'},J_{\varepsilon}(W,\lambda+a))$
 is given as a ${\operatorname{Hom}}_{\mathbb{C}}(V,W)$-valued
 differential operator $D$
 on the Bruhat cell $N_- \simeq {\mathbb{R}}^n$
 as 
\[
  D={\operatorname{Rest}}_{x_n=0} \circ ({\operatorname{Symb}}^{-1} \otimes {\operatorname{id}})(\psi), 
\]
 where $\psi(\zeta_1, \cdots, \zeta_n)$
 is a ${\operatorname{Hom}}_{\mathbb{C}}(V,W)$-valued homogeneous polynomial
 of degree $a$
 satisfying a system of linear (differential) equations
 (cf.~\cite[(4.3) and (4.4)]{KP1})
 that depend holomorphically on $\lambda \in {\mathbb{C}}$.  



If we write the solution $\psi(\zeta)$ as 
\[
  \psi(\zeta)=\sum_{\beta_1 + \cdots + \beta_n=a} 
              \varphi(\beta) \zeta_1^{\beta_1} \cdots \zeta_n^{\beta_n}, 
\]
then the system of differential equations
 for $\psi(\zeta)$ in the F-method
 amounts to a system 
 of linear (homogeneous) equations 
 for the coefficients $\{\varphi(\beta):|\beta|=a\}$.  
We regard $\varphi = (\varphi(\beta)) \in {\mathbb{C}}^k$
 where $k := \#\{\beta \in {\mathbb{N}}^n: |\beta|=\alpha\}$, 
 and use the following elementary lemma
 on the global basis of solutions:
\begin{lemma}
\label{lem:holoeq}
Let $Q_{\lambda}\varphi =0$ be a system
 of linear homogeneous equations
 of $\varphi \in {\mathbb{C}}^k$
 such that $Q_{\lambda}$ depends holomorphically on $\lambda \in {\mathbb{C}}$.  
Assume that there exists a nonempty open subset $U$ of ${\mathbb{C}}$
 such that the space of solutions to $Q_{\lambda} \varphi =0$
 is one-dimensional 
 for every $\lambda$ 
 in $U$.  
Then there exists $\varphi_{\lambda} \in {\mathbb{C}}^k$
 that depend holomorphically on $\lambda$
 in the entire ${\mathbb{C}}$
 such that $Q_{\lambda}\varphi_{\lambda} =0$
 for all $\lambda \in {\mathbb{C}}$.   
\end{lemma}

\begin{proof}
We may regard the equation $Q_{\lambda} \varphi =0$
 as a matrix equation 
 where $Q_{\lambda}$ is an $l$ by $k$ matrix
 ($l \ge k$) 
 whose entries are holomorphic functions of $\lambda \in {\mathbb{C}}$.  
By assumption,
 we have
\[
 {\operatorname{rank}} \, Q_{\lambda}=k-1
\qquad
 \text{for all }\,\, \lambda \in U.  
\]
We can choose a nonempty open subset $U'$ of $U$
 and $k$ row vectors in $Q_{\lambda}$
 such that the corresponding square submatrix $P_{\lambda}$
 is of rank $k-1$, 
 provided $\lambda$ belongs to $U'$.  
Then at least one of row vectors
 in the cofactor of $P_{\lambda}$ is nonzero, 
 which we choose and denote by $\varphi_{\lambda}$.  
Clearly, 
 $\varphi_{\lambda}$ depends holomorphically
 on the entire ${\lambda}\in {\mathbb{C}}$, 
 and $Q_{\lambda} \varphi_{\lambda}=0$
 for all $\lambda \in U'$.  



Since both $Q_{\lambda}$ and $\varphi_{\lambda}$ depend holomorphically
 on $\lambda$
 in the entire ${\mathbb{C}}$, 
 the equation $Q_{\lambda}\varphi_{\lambda}=0$ holds
 for all $\lambda \in {\mathbb{C}}$.  
\end{proof}



We note that the solution $\varphi_{\lambda}$
 in Lemma \ref{lem:holoeq} may vanish 
 for some $\lambda \in {\mathbb{C}}$.  
However,
 the following nonvanishing result holds for all $\lambda \in {\mathbb{C}}$.  
\begin{proposition}
\label{prop:holononvan}
Suppose we are in the setting of Lemma \ref{lem:holoeq}.  
Then 
\begin{equation}
\label{eqn:holononvan}
\dim_{\mathbb{C}}
\{\varphi\in {\mathbb{C}}^k:Q_{\lambda}\varphi=0\} \ge 1
\qquad
\text{for all $\lambda \in {\mathbb{C}}$.}
\end{equation}
\end{proposition}
\begin{proof}
Let $\varphi_{\lambda}$ be as in Lemma \ref{lem:holoeq}.  
Then it suffices to show
 \eqref{eqn:holononvan} for $\lambda$
 belonging to the discrete set $\{\lambda \in {\mathbb{C}}:\varphi_{\lambda}=0\}$.  
Take any $\lambda_0$ such that
 $\varphi_{\lambda_0}=0$.  
Let $k$ be the smallest positive integer 
 such that 
\[
 \psi_{\lambda_0}
 :=\left.\frac{\partial^k}{\partial\lambda^k}\right|_{\lambda=\lambda_0}\varphi_{\lambda} \ne 0
\quad
\text{and}
\quad
 \left.\frac{\partial^{j}}{\partial\lambda^{j}}\right|_{\lambda=\lambda_0}\varphi_{\lambda} = 0
\quad
\text{for $0 \le j \le k-1$}.  
\]
By the Leibniz rule,
 $\left.\frac{\partial^k}{\partial\lambda^k}\right|_{\lambda=\lambda_0}
  (Q_{\lambda} \varphi_{\lambda}) = 0
$
 yields $Q_{\lambda_0} \psi_{\lambda_0}=0$, 
 because $\left. \frac{\partial^{j}}{\partial\lambda^{j}}\right|_{\lambda=\lambda_0}
               \varphi_{\lambda} = 0$
 for all $0 \le j \le k-1$.  
Therefore $\psi_{\lambda_0}$
 is a nonzero solution to $Q_{\lambda_0} \varphi=0$, 
showing \eqref{eqn:holononvan}
 for ${\lambda}=\lambda_0$.  
Hence Proposition \ref{prop:holononvan} is proved.  
\end{proof}

As in the proof of Theorem \ref{thm:existSBO}, 
 the implication (i) $\Rightarrow$ (ii) in Theorem \ref{thm:existDSBO}
 follows from Corollary \ref{cor:DSVOgeneric}
 (generic parameters) and the extension result to special parameters
 (Proposition \ref{prop:holononvan}).  
Thus we have completed a proof
 of Theorem \ref{thm:existDSBO}, 
 and in particular,
 of Theorem \ref{thm:vanDiff} (2).  



%%%%%%%%%%%%%%%%%%%%%%%%%%%%%%%%%%%%%%%%%%%%%%%%%%%%%%%%%%
\subsection{Proof of Theorem \ref{thm:VWSBO} (2-b)}
\label{subsec:170213}
%%%%%%%%%%%%%%%%%%%%%%%%%%%%%%%%%%%%%%%%%%%%%%%%%%%%%%%%%%

In this section, 
 we give a proof of Theorem \ref{thm:VWSBO} (2-b), 
 namely,
 we prove the following proposition.  

\begin{proposition}
[localness theorem]
\label{prop:1532102}
\index{B}{localnesstheorem@localness theorem}
Suppose $[V:W]\ne 0$.  
Suppose that 
\index{A}{1psi@$\Psising$,
          special parameter in ${\mathbb{C}}^2 \times \{\pm\}^2$}
 $(\lambda,\nu,\delta,\varepsilon) \in \Psising$, 
 namely, 
 $(\lambda, \nu)\in {\mathbb{C}}^2$
 and $\delta, \varepsilon \in \{ \pm \}$
 satisfy
\[
  \text{
   $\nu-\lambda \in 2 {\mathbb{N}}$
   when $\delta \varepsilon =+;$
   $\nu-\lambda \in 2 {\mathbb{N}}+1$
   when $\delta \varepsilon =-$.  
  }
\]
Assume further that $\Atbb \lambda \nu {\delta\varepsilon}{V,W} \ne 0$.  
The we have 
\[
 {\operatorname{Hom}}_{G'}
 (I_{\delta}(V, \lambda)|_{G'}, 
  J_{\varepsilon}(W, \nu)
 )
 =
 {\operatorname{Diff}}_{G'}
 (I_{\delta}(V, \lambda)|_{G'}, 
  J_{\varepsilon}(W, \nu)
 ).  
\]
\end{proposition}


We need two lemmas from \cite{sbon}.  
\begin{lemma}
[{\cite[Lem.~11.10]{sbon}}]
\label{lem:sbon1110}
Suppose $D_\mu$ is a differential operator with holomorphic parameter $\mu$,
and $F_\mu$ is a distribution on $\mathbb{R}^n$
that depends holomorphically on $\mu$
having the following expansions:
\begin{align*}
&D_\mu = D_0 + \mu D_1 + \mu^2 D_2 + \dotsb ,
\\
&F_\mu = F_0 + \mu  F_1 + \mu^2 F_2 + \dotsb, 
\end{align*}
where $D_j$ are differential operators
 and $F_i$ are distributions on ${\mathbb{R}}^n$.  
Assume that there exists $\varepsilon>0$
 such that $D_\mu F_\mu = 0$ for any
complex number $\mu$ with $0 < |\mu| < \varepsilon$.  
Then the distributions $F_0$ and $F_{1}$ satisfy
the following differential equations:
\[
D_0 F_0 = 0  \quad\text{and}\quad 
D_0 F_1 + D_1 F_0 = 0.
\]
\end{lemma}
\begin{lemma}
[{\cite[Lem.~11.11]{sbon}}]
\label{lem:sbon1111}
Suppose $h \in \mathcal{D}' (\mathbb{R}^n)$
is supported at the origin.  
Let $E$ be the 
\index{B}{Eulerhomogeneityoperator@Euler homogeneity operator}
Euler homogeneity operator
 $\sum_{\ell=1}^n x_{\ell} \frac {\partial}{\partial x_{\ell}}$ as before.  
If
$(E + A)^2  h = 0$
for some $A \in \mathbb{Z}$ then
$(E + A) h = 0$.  
\end{lemma}

The argument below is partly similar
 to the one in Section \ref{subsec:AVW}, 
 however, 
 we note that the renormalization 
$
   \Attbb {\lambda_0} {\nu_0} {\gamma}{V,W}
$
 in Theorem \ref{thm:170340} is not defined 
 under our assumption
 that $\Atbb {\lambda_0} {\nu_0} {\gamma}{V,W} \ne 0$
 and $(\lambda_0, \nu_0, \delta, \varepsilon) \in \Psising$.  
Instead, 
we shall use the distribution
 $(\Attcal {\lambda} {\nu} {\gamma}{V,W})'$
 on ${\mathbb{R}}^n \setminus \{0\}$, 
 of which we recall \eqref{eqn:AVW+} and \eqref{eqn:AVW-}
 for the definition.   

\begin{proof}
[Proof of Proposition \ref{prop:1532102}]
Take any symmetry breaking operator
\[
 {\mathbb{T}} \in {\operatorname{Hom}}_{G'}(I_{\delta}(V,\lambda_0)|_{G'}, 
 J_{\varepsilon}(W,\nu_0)).  
\]
We write $({\mathcal{T}}_{\infty}, {\mathcal{T}})$
 for the pair of distribution kernels of ${\mathbb{T}}$
 as in Proposition \ref{prop:Tpair}.  
We set $\gamma:= \delta \varepsilon$.  


It follows from Proposition \ref{prop:20150828-1231} (3)
 that 
$
   {\mathcal{T}}|_{{\mathbb{R}}^n\setminus \{0\}}
   =
   c' (\Attcal {\lambda_0} {\nu_0} \gamma {V,W})'$
 for some $c' \in {\mathbb{C}}$.  

Suppose $\Atbb {\lambda_0} {\nu_0} {\gamma}{V,W} \ne 0$
 and $\nu_0 - \lambda_0 \in 2 {\mathbb{N}}$
 ($\gamma=+$)
 or $\in 2 {\mathbb{N}}+1$
 ($\gamma=-$).  
As in \eqref{eqn:Anear0}, 
 we expand $\Atcal {\lambda} {\nu_0} {\gamma}{V,W}$
 near $\lambda = \lambda_0$:
\[
  \Atcal {\lambda} {\nu_0} {\gamma}{V,W}
  =
  F_0 + (\lambda-\lambda_0) F_1 + (\lambda-\lambda_0)^2 F_2
  + \cdots,
\]
where $F_j \in {\mathcal{D}}'({\mathbb{R}}^n) \otimes {\operatorname{Hom}}_{\mathbb{C}}(V,W)$.  
We note that $F_0 \ne 0$
 because $\Atbb {\lambda_0} {\nu_0} {\gamma}{V,W} \ne 0$.  
We define a nonzero constant $c$
 by 
\begin{equation}
\label{eqn:1702110}
  c:= \lim_{\mu \to 0}
      \mu \Gamma(\frac \mu 2-l)
    = \frac{2(-1)^l}{l!}.  
\end{equation}
In view of the relation
\[
   \Atcal \lambda \nu + {V,W}|_{{\mathbb{R}}^n \setminus \{0\}}
   =
   \frac{1}{\Gamma(\frac{\lambda-\nu}{2})} (\Attcal \lambda \nu + {V,W})', 
\quad
  \Atcal \lambda \nu - {V,W}|_{{\mathbb{R}}^n \setminus \{0\}}
   =
   \frac{1}{\Gamma(\frac{\lambda-\nu+1}{2})} (\Attcal \lambda \nu - {V,W})',
\]
 we get
\[
  c F_1|_{{\mathbb{R}}^n \setminus \{0\}}
  =
  (\Attcal {\lambda_0} {\nu_0} \gamma {V,W})', 
\]
as in the proof of Theorem \ref{thm:170340} (3).  
We set 
\[
   D_0 := E-\lambda_0 + \nu_0 + n
       = \sum_{j=1}^n x_j \frac{\partial}{\partial x_j}
         -\lambda_0 + \nu_0 + n.  
\]
Applying Lemma \ref{lem:sbon1110}
 to the differential equation \eqref{eqn:Fainv}:
\[
  (E-\lambda + \nu_0 + n) \Atcal \lambda {\nu_0} \gamma {V,W}
  =
  (D_0-(\lambda-\lambda_0))\Atcal \lambda {\nu_0} \gamma {V,W}
  =0, 
\]
we get 
\begin{equation}
\label{eqn:DF01}
D_0 F_0 =0, 
\quad
D_0 F_1 - F_0 =0. 
\end{equation}



We set
\[
   h:={\mathcal{T}}- c c' F_1
  \in 
  {\mathcal{D}}'({\mathbb{R}}^n)
    \otimes 
    \operatorname{Hom}_{{\mathbb{C}}}(V,W). 
\]
Then ${\operatorname{Supp}}\, h \subset \{0\}$.  
Moreover, 
$D_0^2 h =0$
 by $D_0 {\mathcal{T}}=0$ and \eqref{eqn:DF01}.  



Applying Lemma \ref{lem:sbon1111}, 
 we get $D_0 h=0$.  
It turn, 
 $c c' F_0=0$
 again by $D_0 {\mathcal{T}}=0$
 and \eqref{eqn:DF01}.  
Therefore,
 if $\Atcal {\lambda_0} {\nu_0} \gamma {V,W} \ne 0$,
 or equivalently,
 if $\Atbb {\lambda_0} {\nu_0} \gamma {V,W} \ne 0$, 
 then we conclude $c'=0$
 because $F_0 \ne 0$.  
Thus ${\mathcal{T}}$ is supported at the origin, 
 and therefore ${\mathbb{T}}$ is a differential operator
 (see Proposition \ref{prop:Tpair} (3)).  

Hence Proposition \ref{prop:1532102} is proved.  
\end{proof}

The above proof implies 
 that the distribution $(\Attcal \lambda \nu \gamma {V,W})' \in {\mathcal{D}}'({\mathbb{R}}^n \setminus \{0\}) \otimes {\operatorname{Hom}}_{\mathbb{C}}(V,W)$
 in \eqref{eqn:AVW+} and \eqref{eqn:AVW-} does not always extend
 to an element of ${\mathcal{S}}{\it{ol}}({\mathbb{R}}^n; V_{\lambda, \delta}, 
                            W_{\nu, \varepsilon})$
 ($\gamma = \delta \varepsilon$):

\begin{proposition}
\label{prop:20170213}
Let $\gamma \in \{\pm\}$. 
Suppose $(\lambda, \nu)\in {\mathbb{C}}^2$ satisfies
\[
  \text{
  $\nu-\lambda \in 2{\mathbb{N}}$
  when $\gamma =+;$
  \,\,
  $\nu-\lambda \in 2{\mathbb{N}}+1$
  when $\gamma =-$.  
 }
\]
If $\Atbb \lambda \nu \gamma {V,W} \ne 0$, 
 then for $\delta, \varepsilon \in \{ \pm \}$
 with $\delta \varepsilon =\gamma$, 
 the restriction map
\[
   {\mathcal{S}}{\it{ol}}({\mathbb{R}}^n; V_{\lambda, \delta}, 
                            W_{\nu, \varepsilon})
   \to 
   {\mathcal{D}}'({\mathbb{R}}^n \setminus \{0\})
    \otimes 
    \operatorname{Hom}_{{\mathbb{C}}}(V,W)
\]
 is identically zero.  
\end{proposition}


