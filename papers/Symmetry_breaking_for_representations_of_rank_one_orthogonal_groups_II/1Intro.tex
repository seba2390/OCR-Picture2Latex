\newpage
\section{Introduction}
\label{sec:Intro}


\newcommand{\bZ}{{\mathbb Z}}
\newcommand{\bR}{{\mathbb R}}
\newcommand{\bC}{{\mathbb C}}
\newcommand{\bN}{{\mathbb N}}

A representation $\Pi$ of a group $G$
 defines a representation
 of a subgroup $G'$
 by restriction.  
In general irreducibility 
 is not preserved by the restriction.  
If $G$ is compact
 then the restriction $\Pi|_{G'}$ is isomorphic
 to a direct sum of irreducible finite-dimensional representations $\pi$ of $G'$ with multiplicities  $m(\Pi,\pi)$. 
These multiplicities are studied by using combinatorial techniques.  
We are interested in the case
 where $G$ and $G'$ are (noncompact) real reductive Lie groups.  
Then most irreducible representations $\Pi$ of $G$ are
 infinite-dimensional, 
 and generically the restriction $\Pi|_{G'}$
 is not a direct sum of irreducible representations
 \cite{KInvent98}.  
So we have to consider another notion of multiplicity.




For a continuous representation $\Pi$ of $G$
 on a complete, 
 locally convex topological vector space ${\mathcal{H}}$,
 the space ${\mathcal{H}}^\infty $ of $C^\infty$-vectors of ${\mathcal{H}}$
 is naturally endowed with a Fr{\'e}chet topology,
 and $(\Pi,{\mathcal{H}})$ induces a continuous representation $\Pi^{\infty}$ of $G$
 on ${\mathcal{H}}^\infty$.  
If $\Pi$ is an admissible representation
 of finite length on a Banach space ${\mathcal{H}}$, 
 then the Fr{\'e}chet representation 
 $(\Pi^{\infty}, {\mathcal{H}}^{\infty})$, 
 which we refer to as an 
\index{B}{admissiblesmoothrepresentation@admissible smooth representation}
{\it{admissible smooth representation}}, 
 depends only 
 on the underlying $({\mathfrak {g}}, K)$-module
 ${\mathcal{H}}_K$.  
In the context of asymptotic behaviour 
 of matrix coefficients, 
 these representations
 are also referred to as an admissible representations
 of moderate growth \cite[Chap.~11]{W}.  
We shall work with these representations
 and write simply $\Pi$ for $\Pi^{\infty}$.  
We denote by 
\index{A}{Irredrep@${\operatorname{Irr}}(G)$}
${\operatorname{Irr}}(G)$
 the set of equivalence classes
 of irreducible admissible smooth representations.  
We also sometimes call these representations
 \lq\lq{irreducible admissible representations}\rq\rq\
 for simplicity.  

Given another admissible smooth representation $\pi$
 of a reductive subgroup $G'$, 
 we consider the space of continuous $G'$-intertwining operators 
\index{B}{symmetrybreakingoperators@symmetry breaking operators}
({\it{symmetry breaking operators}})
\[ \operatorname{Hom}_{G'} ({\Pi}|_{G'}, {\pi}) .\] 
If $G=G'$ then these operators include the Knapp--Stein operators \cite{KS}
 and the differential intertwining operators 
 studied by B.~Kostant \cite{Kos}.
If $G\not = G'$ the dimension 
\index{A}{mPipi@$m(\Pi, \pi)$, multiplicity|textbf}
\[
   m(\Pi,\pi) :=\mbox{dim}_{\mathbb{C}} \operatorname{Hom}_{G'} ({\Pi}|_{G'}, {\pi}) 
\]
 yields important information of the restriction of $\Pi $ to $G'$
 and is called the
\index{B}{multiplicity@multiplicity|textbf}
 {\it{multiplicity}} of $\pi$
 occurring in the restriction $\Pi|_{G'}$.  
In general,
 $m(\Pi,\pi)$ may be infinite.  
The finiteness criterion in \cite{xKOfm} asserts that the
 multiplicity $m(\Pi,\pi)$ is finite
 for all $\Pi \in {\operatorname{Irr}}(G)$
 and for all $\pi \in {\operatorname{Irr}}(G')$
 if and only if a minimal parabolic subgroup $P'$
 of $G'$ has an open orbit 
 on the real flag variety $G/P$, 
 and that the multiplicity is uniformly bounded
 with respect to $\Pi$ and $\pi$
 if and only if a Borel subgroup of $G_{\mathbb{C}}'$
 has an open orbit
 on the complex flag variety of $G_{\mathbb{C}}$.  
 
\medskip
The latter condition depends only
 on the complexified pairs
 $({\mathfrak {g}}_{\mathbb{C}}, {\mathfrak {g}}_{\mathbb{C}}')$, 
 of which the classification
 was already known in 1970s
 by Kr{\"a}mer 
 \cite{Kr1} and Kostant.  
In particular,
 the multiplicity $m(\Pi,\pi)$ is uniformly bounded
 if the Lie algebras $({\mathfrak {g}}, {\mathfrak {g}}')$ of $(G,G')$ 
 are real forms
 of $({\mathfrak {sl}}(N+1,{\mathbb{C}}),
 {\mathfrak {gl}}(N,{\mathbb{C}}))$
 or $({\mathfrak {o}}(N+1,{\mathbb{C}}),
 {\mathfrak {o}}(N,{\mathbb{C}}))$.  
On the other hand, 
 the former condition depends on real forms
 $({\mathfrak {g}}, {\mathfrak {g}}')$, 
 and the classification
 of such symmetric pairs was recently accomplished
 in \cite{xKMt}.  
For instance,
 let $(G,G')=(O(n+1,1), O(n+1-k,1))$.  
Then the classification theory \cite{xKMt}
 and the finiteness criterion \cite{xKOfm}
 imply the following upper and lower estimates
 of the multiplicity
$
  m(\Pi,\pi)
$:
\begin{enumerate}
\item[(1)]
For $2 \le k \le n+1$, 
\begin{enumerate}
\item[]
$m(\Pi,\pi)<\infty$
 for every pair $(\Pi,\pi) \in {\operatorname{Irr}}(G)
 \times {\operatorname{Irr}}(G')$;
\item[]
$\underset{\Pi \in {\operatorname{Irr}}(G)}\sup
\,\,
 \underset{\pi \in {\operatorname{Irr}}(G')}\sup
 m(\Pi,\pi)=\infty.
$
\end{enumerate}
\item[(2)]
For $k=1$, 
there exists $C>0$ such that 
\begin{equation}
\label{eqn:mbdd}
m(\Pi,\pi) \le C
%\qquad
\text{ for all $\Pi \in {\operatorname{Irr}}(G)$
 and for all $\pi \in {\operatorname{Irr}}(G')$.}
\end{equation}
\end{enumerate}

B.~Sun  and C.-B.~Zhu \cite{SunZhu} showed
 that one can take $C$ to be one in \eqref{eqn:mbdd}, 
 namely,
 the  multiplicity
$
    m(\Pi,\pi) \in \{0,1\} 
$
 in this case.  
Thus one of the open problems is to determine
 when $m(\Pi, \pi) \ne 0$
 for irreducible representations $\Pi$ and $\pi$.  



In the previous publication \cite{sbon} we initiated a thorough study of symmetry breaking operators
 between {\it{spherical}} principal series representations
of
\begin{equation}
\label{eqn:GG}
     (G,G')=(O(n+1,1),O(n,1)).  
\end{equation}
In particular,
 we determined the multiplicities $m(\Pi, \pi)$
 when both $\Pi$ and $\pi$ are irreducible composition factors
 of the spherical principal series representations. 



In this article we will determine
 the multiplicities $m(\Pi,\pi)$ for all irreducible representations
 $\Pi$ and $\pi$
 with trivial infinitesimal character $\rho$ of $G=O(n+1,1)$ and $G'=O(n,1)$, 
 respectively, 
 and also for irreducible principal series representations.  



More than just determining the dimension $m(\Pi,\pi)$
 of the space of symmetry breaking operators, 
 we investigate these operators of their own 
 for {\it{general}} principal series representations of $G$ and the subgroup $G'$, 
{\it{i.e.}}, 
 for representations induced from irreducible finite-dimensional representations of a parabolic subgroup.  
We construct a holomorphic family
 of symmetry breaking operators, 
 and present a 
\index{B}{classificationscheme@classication scheme, symmetry breaking operators}
 classification scheme of {\it{all}} symmetry breaking operators $T$
 in Theorem \ref{thm:VWSBO}
 through an analysis of their distribution kernels $K_T$.  
In particular, 
 we prove that any symmetry breaking operators
 in this case is either a 
\index{B}{sporadicsymmetrybreakingoperator@sporadic symmetry breaking operator}
 sporadic differential symmetry breaking operator
 (cf. \cite{KKP})
 or the analytic continuation
 of integral symmetry breaking operators
 and their renormalization
 in Theorem \ref{thm:VWSBO}.  



The proof for the explicit formula
 of the multiplicity $m(\Pi, \pi)$ is built on 
 the 
\index{B}{functionalequation@functional equation}
 functional equations 
 (Theorems \ref{thm:TAA} and \ref{thm:ATA})
 satisfied by the regular symmetry breaking operators.  


\vskip 2pc
A principal series representation
 $I_\delta(V,\lambda)$ of $G=O(n+1,1)$
 is an (unnormalized) induced representation from an irreducible
 finite-dimensional representation
 $V\otimes \delta \otimes {\mathbb{C}}_\lambda$
 of a minimal parabolic subgroup $P=MAN_+$. 
In our setting, 
 $M \simeq O(n) \times  {\mathbb{Z}}/2{\mathbb{Z}}$ and $A \simeq \bR_+$. 
We assume that $V$ is a representation of $O(n+1)$, 
 $\delta \in \{\pm\}$,
 and $\lambda \in \bC$.
In what follows,
 we identify the representation space
 of $I_\delta(V, \lambda)$ 
 with the space
 of $C^{\infty}$-sections
 of the $G$-equivariant  bundle 
 $G \times_P {\mathcal V}_{\delta,\lambda} \to G/P$, 
so that 
 $I_\delta(V, \lambda)^{\infty} = I_\delta (V,\lambda)$
 is the Fr{\'e}chet globalization
 having moderate growth
 in the sense 
 of Casselman--Wallach \cite{W}.  
The parametrization is chosen
 so that the representation $ I_\delta (V,\frac n2)$ is a unitary tempered representation. 
The representations $I_\delta(V,\lambda)$ are either irreducible
 or of composition series of length 2, 
 see Corollary \ref{cor:length2} in Appendix I.



The group $P'= G' \cap P = M'AN_+'$ is a minimal parabolic subgroup
 of $G'=O(n,1)$.  
For an irreducible representation $(\tau, W)$ of $O(n-1)$, a character 
 $\varepsilon\in \{\pm\}$ of $O(1)$, 
 and $\nu \in{\mathbb{C}}$ we define the principal series representation $J_\varepsilon(W,\nu)$ of $G'$.  

\medskip
We set 
\[
   [V:W]:= \dim_{\mathbb{C}}\operatorname{Hom}_{O(n-1)} (W,V|_{O(n-1)})
         = \dim_{\mathbb{C}}\operatorname{Hom}_{O(n-1)} (V|_{O(n-1)},W).  
\]
For principal series representations $I_\delta(V,\lambda )$ of $G$
 and $J_\varepsilon(W,\nu)$ of the subgroup $G'$, 
 we consider the cases $[V:W] \ne 0 $ and $[V:W] = 0$ separately. In the first case we obtain a lower bound 
for the multiplicity.  



In what follows, 
 it is convenient to introduce the set of 
 \lq\lq{special parameters}\rq\rq:
\index{A}{1psi@$\Psising$,
          special parameter in ${\mathbb{C}}^2 \times \{\pm\}^2$|textbf}
\begin{alignat}{2}
  \Psising
  :=
  \left\{(\lambda,\nu, \delta, \varepsilon) \in {\mathbb{C}}^2 \times \{\pm\}^2
   :\,\, \right.
&   \nu-\lambda \in 2 {\mathbb{N}}
&&\text{when $\delta \varepsilon =+$}
\notag
\\
\label{eqn:singset}
   \text{ or }\quad
& \nu-\lambda \in 2 {\mathbb{N}}+1
 \qquad
&&\text{when $\delta \varepsilon =-$}
  \left. \right\}.  
\end{alignat}



\begin{theorem}
[see Theorem \ref{thm:VWSBO} (2) and Theorem \ref{thm:1532113}]
\label{thm:intro160150}
Suppose $(\sigma,V)\in \widehat{O(n)}$ and $(\tau,W)\in \widehat{O(n-1)}$.  
Assume $[V:W] \ne 0$.  
\begin{enumerate}
\item[{\rm{(1)}}] 
{\rm{(existence of symmetry breaking operators)}}\enspace
We have
\[
  \dim_{\mathbb{C}} \operatorname{Hom}_{G'}(I_{\delta}(V, \lambda)|_{G'}, J_{\varepsilon}(W, \nu))
  \ge 1
  \quad
  \text{for all 
  $\delta, \varepsilon \in \{\pm\}$, 
  and $\lambda, \nu \in {\mathbb{C}}$}.  
\]
\item[{\rm{(2)}}]
\index{B}{genericmultiplicityonetheorem@generic multiplicity-one theorem}
{\rm{(generic multiplicity-one)}}\enspace
\[
\dim_{\mathbb{C}} \operatorname{Hom}_{G'}(I_{\delta}(V, \lambda)|_{G'}, J_{\varepsilon}(W, \nu))
  =1
\]
for any $(\lambda,\nu, \delta, \varepsilon) \in ({\mathbb{C}}^2 \times \{\pm\}^2) \setminus \Psising$.  
\item[{\rm{(3)}}]
Let $\ell(\sigma)$ be the \lq\lq{norm}\rq\rq\
 of $\sigma$ defined by using its highest weight 
 (see \eqref{eqn:ONlength}).  
Then we have 
\[
  \dim_{\mathbb{C}} \operatorname{Hom}_{G'}
  (I_{\delta}(V, \lambda)|_{G'}, J_{\varepsilon}(W, \nu))>1
\]
 for any $(\lambda,\nu, \delta, \varepsilon) \in \Psising$
 such that $\nu \in {\mathbb{Z}}$ with $\nu \le -\ell(\sigma)$.  
\end{enumerate}
\end{theorem}

We prove Theorem  \ref{thm:intro160150}
 by constructing (generically) regular symmetry breaking operators
 $\Atbb \lambda \nu {\delta\varepsilon} {V,W}$:
 they are nonlocal operators 
 ({\it{e.g.}}, integral operators)
 for generic parameters, 
 whereas for some parameters they are local operators
 ({\it{i.e.}}, differential operators).  
See Theorem \ref{thm:152389}
 for the construction of the normalized operator $\Atbb \lambda \nu \pm {V,W}$;
Theorem \ref{thm:regexist}
 for \lq\lq{regularity}\rq\rq\
 (\cite[Def.~3.3]{sbon})
 of $\Atbb \lambda \nu \pm {V,W}$
 under a certain generic condition;
Theorem \ref{thm:170340}
 for a renormalization of $\Atbb \lambda \nu \pm {V,W}$
 when it vanishes;
Fact \ref{fact:153316}
 for the residue formula of $\Atbb \lambda \nu \pm {V,W}$
 when it reduces to a differential operator.  


\vskip 1pc
In the case $[V:W]=0$, 
 symmetry breaking operators are \lq\lq{rare}\rq\rq\ 
 but 
there may exist {\it{sporadic}} symmetry breaking operators: 
\begin{theorem}
\label{thm:VW0SBO}
Assume $[V:W] = 0$.  
\begin{enumerate}
\item[{\rm{(1)}}]
{\rm{(vanishing for generic parameters, 
 Corollary \ref{cor:VWvanish})}}
If $(\lambda, \nu, \delta, \varepsilon) \not \in \Psising$,
then 
\[
   {\operatorname{Hom}}_{G'}(I_{\delta}(V,\lambda)|_{G'}, J_{\varepsilon}(W,\nu))=\{0\}.  
\] 
\item[{\rm{(2)}}]
{\rm{(localness theorem, Theorem \ref{thm:152347})}}
\index{B}{localnesstheorem@localness theorem}
Any nontrivial symmetry breaking operator
\[
   C^{\infty}(G/P, {\mathcal{V}}_{\lambda,\delta})
   \to 
   C^{\infty}(G'/P', {\mathcal{W}}_{\nu,\varepsilon})
\]
is a differential operator.  
\end{enumerate}
\end{theorem}


Combining Theorem \ref{thm:intro160150} (2) and Theorem \ref{thm:VW0SBO} (1)
 together with the existence condition
 of differential symmetry breaking operators 
 (see Theorem \ref{thm:vanishDiff}), 
 we determine the following multiplicity formul{\ae}
 for {\it{generic parameters}}:
\begin{theorem}
Suppose that $(\lambda,\nu, \delta,\varepsilon) \not \in \Psising$.   
Then there are no differential symmetry breaking operators and
\[
     {\operatorname{dim}}_{\mathbb{C}} {\operatorname{Hom}}_{G'}(I_\delta(V,\lambda)|_{G'},J_\varepsilon (W,\nu))
=
\begin{cases}
 1 \qquad&\text{if $[V:W]\ne0$, }
\\
 0 \qquad&\text{if $[V:W]=0$.  }
\end{cases}
\]
\end{theorem}


\bigskip
It deserves to be mentioned
 that the parameter set $({\mathbb{C}}^2 \times \{\pm\}^2)\setminus \Psising$ contains parameters $(\lambda,\nu)$
 for which the $G$-module $I_{\delta}(V,\lambda)$
 or the $G'$-module $J_{\varepsilon}(W,\nu)$
 is {\it{not}} irreducible.  


\vskip 2pc
In the major part of this monograph,
 we focus our attention on the special case
\[
  (V,W)=(\Exterior^i({\mathbb{C}}^{n}), \Exterior^j({\mathbb{C}}^{n-1})).  
\]
The principal series representations
 of $G$ and the subgroup $G'$
 are written as $I_{\delta}(i,\lambda)$ for $I_{\delta}(\Exterior^i({\mathbb{C}}^{n}),\lambda)$
 and $J_{\varepsilon}(j,\nu)$ for $J_{\varepsilon}(\Exterior^j({\mathbb{C}}^{n-1}),\nu)$, 
 respectively.  
The representations
 $I_\delta(i,\lambda)$ of $G$
 and $J_{\varepsilon}(j,\nu)$ of $G'$ are of  interest in geometry as well as in automorphic forms and in the cohomology of arithmetic groups. 
In geometry, 
 given an arbitrary Riemannian manifold $X$, 
 one forms a natural family of representations
 of the conformal group $G$
 on the space ${\mathcal{E}}^i(X)$ of differential forms,
 to be denoted by ${\mathcal{E}}^i(X)_{\lambda',\delta'}$
 for $0 \le i \le \dim X$, $\lambda' \in {\mathbb{C}}$, 
 and $\delta' \in \{\pm\}$.  
Then the representations $I_\delta(i,\lambda)$
 are identified with such conformal representations
 in the case where $(G, X)=(O(n+1,1), S^n)$, 
 see {\it{e.g.}}, \cite[Chap.~2, Sect.~2]{KKP}
 for precise statement.  
In representation theory,
 all irreducible, unitarizable representations
 with nonzero $({\mathfrak{g}}, K)$-cohomology arise
 as subquotients of $I_{\delta}(i,\lambda)$ with $\lambda=i$
 for some $0 \le i \le n$
 and $\delta=(-1)^i$, 
 see Theorem \ref{thm:LNM20} (9).   



Our main results of this article include 
 a complete solution
 to the general problem
 of constructing and classifying the elements
 of ${\operatorname{Hom}}_{G'}(\Pi|_{G'}, \pi)$
 (see \cite[Prob.~7.3 (3) and (4)]{xkvogan})
 in the following special setting:
\begin{alignat*}{2}
(G,G')&=(O(n+1,1),O(n,1)) \qquad\text{with $n \ge 3$, }
\\
(\Pi,\pi)&=(I_{\delta}(i,\lambda),J_{\varepsilon}(j,\nu)), 
\end{alignat*}
where $0 \le i \le n$, $0 \le j \le n-1$, 
 $\delta, \varepsilon \in \{\pm\}$, 
 and $\lambda, \nu \in {\mathbb{C}}$.  
Thus our main results include a complete solution 
 to the following question in conformal geometry:
\begin{problem}
\label{prob:conf}
\begin{enumerate}
\item[{\rm{(1)}}]
Find a necessary and sufficient condition on 6-tuples
 $(i,j,\lambda,\nu,\delta,\varepsilon)$ for the existence
 of conformally covariant, 
 symmetry breaking operators
\[
  A \colon {\mathcal{E}}^i(X)_{\lambda,\delta} 
           \to {\mathcal{E}}^i(X)_{\nu,\varepsilon}
\]
in the model space $(X,Y)=(S^n, S^{n-1})$.  
\item[{\rm{(2)}}]
Construct those operators explicitly
 in the (flat) coordinates.  
\item[{\rm{(3)}}]
Classify all such symmetry breaking operators.  
\end{enumerate}
\end{problem}
Partial results were known earlier:
when the operator $A$ is given by a {\it{differential}} operator,
 Juhl \cite{Juhl} solved Problem \ref{prob:conf}
 in the case $(i,j)=(0,0)$,
 see also \cite{KOSS}, 
 which has been recently extended
 in Kobayashi--Kubo--Pevzner \cite{KKP}
 for the general $(i,j)$.  
Problem \ref{prob:conf} was solved for all 
 (possibly, {\it{nonlocal}}) operators
 in our previous paper \cite{sbon}
 in the case $(i,j)=(0,0)$.  
The complete classification of (continuous) symmetry breaking operators 
 for the general $(i,j)$ is given in Theorem \ref{thm:1.1} (multiplicity)
 and Theorem \ref{thm:SBObasis} (construction of explicit generators),  
 and we have thus settled Problem \ref{prob:conf}
 in this monograph.  
For this introduction,
 we explain only the
 \lq\lq{multiplicity}\rq\rq\
 (Theorem \ref{thm:1.1}).  
For this,
 using the same notation
 as in \cite[Chap.~1]{sbon}, 
 we define the following two subsets on ${\mathbb{Z}}^2$:
\begin{align*}
L_{\operatorname{even}}:=&\left \{ (-i,-j):
0 \le j\leq i \mbox{ and } i\equiv j \mod 2 \right \},
\\
L_{\operatorname{odd}}:=&\left \{ (-i,-j)
: 0 \le j\leq i \mbox{ and } i \equiv j +1 \mod 2 \right \}.
\end{align*}

\begin{theorem}
[multiplicity, Theorem \ref{thm:1.1}]
Suppose $\Pi=I_{\delta}(i,\lambda)$
 and $\pi=J_{\varepsilon}(j,\nu)$
 for $0 \le i\le n$, $0 \le j \le n-1$,
 $\delta, \varepsilon \in \{\pm \}$, 
 and $\lambda,\nu \in {\mathbb{C}}$.  
Then 
we have the following.  
\begin{enumerate}
\item[{\rm{(1)}}]
\begin{alignat*}{2}
m(\Pi,\pi) \in &\{ 1,2 \} \qquad
&&\text{if $j=i-1$ or $i$}, 
\\
m(\Pi,\pi) \in &\{ 0,1 \} \qquad
&&\text{if $j=i-2$ or $i+1$}, 
\\
m(\Pi,\pi) =& 0 \qquad
&&\text{otherwise}.  
\end{alignat*}
\item[{\rm{(2)}}]
Suppose $j=i-1$ or $i$.  
Then $m(\Pi,\pi)=1$ generically,
 and $=2$ when the parameter belongs to the following exceptional 
 countable set.  
Without loss of generality,
 we take $\delta$ to be $+$.  
\begin{enumerate}
\item[{\rm{(a)}}]
Case $1 \le i \le n-1$.
\begin{alignat*}{2}
m(I_+(i,\lambda), J_+(i,\nu))
=
&2
\qquad
&&\text{if }
(\lambda, \nu) \in L_{\operatorname{even}}\setminus \{\nu=0\}
                   \cup \{(i,i)\}.   
\\
m(I_+(i,\lambda), J_-(i,\nu))
=
&2
&&\text{if }
(\lambda, \nu) \in L_{\operatorname{odd}}\setminus \{\nu=0\}.   
\\
m(I_+(i,\lambda), J_+(i-1,\nu))
=
&2
&&
\text{if }
(\lambda, \nu) \in L_{\operatorname{even}}\setminus \{\nu=0\} 
                   \cup \{(n-i,n-i)\}.   
\\
m(I_+(i,\lambda), J_-(i-1,\nu))
=
&2
&&
\text{if }
(\lambda, \nu) \in L_{\operatorname{odd}}\setminus \{\nu=0\}.   
\end{alignat*}
\item[{\rm{(b)}}]
Case $i=0$.  
\begin{align*}
m(I_+(0,\lambda), J_+(0,\nu))
=&
2
\qquad
\text{if }
(\lambda, \nu) \in L_{\operatorname{even}}.   
\\
m(I_+(0,\lambda), J_-(0,\nu))
=&
2
\qquad
\text{if }
(\lambda, \nu) \in L_{\operatorname{odd}}.   
\end{align*}
\item[{\rm{(c)}}]
Case $i=n$.  
\begin{align*}
m(I_+(n,\lambda), J_+(n-1,\nu))
=&
2
\qquad
\text{if }
(\lambda, \nu) \in L_{\operatorname{even}}.   
\\
m(I_+(n,\lambda), J_-(n-1,\nu))
=&
2
\qquad
\text{if }
(\lambda, \nu) \in L_{\operatorname{odd}}.   
\end{align*}
\end{enumerate}
\item[{\rm{(3)}}]
Suppose $j=i-2$ or $i+1$.  
Then $m(\Pi,\pi)=1$ 
 if one of the following conditions {\rm{(d)--(g)}}
 is satisfied, 
 and $m(\Pi,\pi)=0$ otherwise.  
\begin{enumerate}
\item[{\rm{(d)}}]
Case $j=i-2$, $2 \le i \le n-1$, $(\lambda, \nu)=(n-i,n-i+1)$, 
$\delta \varepsilon =-1$.  
\item[{\rm{(e)}}]
Case $(i,j)=(n,n-2)$, $-\lambda \in {\mathbb{N}}$, $\nu=1$, 
$\delta \varepsilon =(-1)^{\lambda+1}$.  
\item[{\rm{(f)}}]
Case $j=i+1$, $1 \le i \le n-2$, $(\lambda, \nu)=(i,i+1)$, 
$\delta \varepsilon =-1$.  
\item[{\rm{(g)}}]
Case $(i,j)=(0,1)$, $-\lambda \in {\mathbb{N}}$, $\nu=1$, 
$\delta \varepsilon =(-1)^{\lambda+1}$.  
\end{enumerate}
\end{enumerate}
\end{theorem}

\medskip
More than just an abstract formula of multiplicities, 
 we also obtain explicit generators
 of ${\operatorname{Hom}}_{G'}(I_{\delta}(i,\lambda)|_{G'}, J_{\varepsilon}(j,\nu))$
 for $j \in \{i-1,i\}$
 in Theorem \ref{thm:SBObasis}.  
The generators for $j \in \{i-2,i+1\}$ are always differential operators
 ({\it{localness theorem}}, 
 see Theorem \ref{thm:VW0SBO} (2)), 
 and they were constructed and classified in \cite{KKP}
 (see Fact \ref{fact:3.9}).  



The principal series representations
 $I_{\delta}(i,\lambda)$ and $J_{\varepsilon}(j,\nu)$
 in the above theorem
 are not necessarily irreducible.  
For the study of symmetry breaking of the irreducible subquotients,
 we utilize the concrete generators of 
${\operatorname{Hom}}_{G'}(I_{\delta}(i,\lambda)|_{G'}, J_{\varepsilon}(j,\nu))$ and determine explicit formul{\ae}
 about 
\begin{enumerate}
\item[$\bullet$]
the $(K,K')$-spectrum
 of the normalized regular symmetry breaking operators
 $\Atbb \lambda \nu \pm {i,j}$
 on basic \lq\lq{$(K,K')$-types}\rq\rq\
 (Theorem \ref{thm:153315});
\item[$\bullet$]
the functional equations among symmetry breaking operators
 $\Atbb \lambda \nu \pm {i,j}$
(Theorems \ref{thm:TAA} and \ref{thm:ATA}).  
\end{enumerate}
Here, 
  the 
\index{B}{KspectrumKprime@$(K,K')$-spectrum}
$(K,K')$-spectrum is defined in Definition \ref{def:KKspec}.  
It resembles eigenvalues of symmetry breaking
 operators, 
 and serves as a clue to find the functional equations.  



\bigskip
We now highlight symmetry breaking
 of irreducible representations
 that have the same infinitesimal character $\rho$
 with the trivial one-dimensional representation ${\bf{1}}$.  
Denote by 
\index{A}{IrrGrho@${\operatorname{Irr}}(G)_{\rho}$,
 set of irreducible admissible smooth representation of $G$
 with trivial infinitesimal character $\rho$\quad}
${\operatorname{Irr}}(G)_{\rho}$
 the (finite) set
 of equivalence classes 
 of irreducible admissible representations
 of $G$
 with trivial infinitesimal character $\rho\equiv \rho_G$.  
The principal series representations $I_\delta(i,i)$ of $G=O(n+1,1)$
 are reducible, 
 and any element in ${\operatorname{Irr}}(G)_{\rho}$
 is a subquotient
 of the representations $I_\delta(i,i)$
 for some $0 \le i \le n$
 and $\delta \in \{\pm\}$.  
To be more precise,
 we have the following.  
\begin{theorem}
[see Theorem \ref{thm:LNM20}]
Let $G=O(n+1,1)$ $(n \ge 1)$.  
\begin{enumerate}
\item[{\rm{(1)}}]
For $0 \le \ell \le n$ and $\delta \in \{ \pm \}$, 
 there are exact sequences of $G$-modules:
\begin{align*}
&0 \to \Pi_{\ell,\delta} \to I_{\delta}(\ell,\ell) \to \Pi_{\ell+1,-\delta}
\to 0, 
\\
&0 \to \Pi_{\ell+1,-\delta} \to I_{\delta}(\ell,n-\ell) \to \Pi_{\ell,\delta} \to 0.  
\end{align*}
These exact sequences split
 if and only if $n=2\ell$.  


\item[{\rm{(2)}}]
Irreducible admissible representations of $G$ 
 with trivial infinitesimal character 
 can be classified as
\[
  {\operatorname{Irr}}(G)_{\rho}
  =\{
    \Pi_{\ell, \delta}
    :
    0 \le \ell \le n+1, \, \delta = \pm
\}.  
\]

\item[{\rm{(3)}}]
Every $\Pi_{\ell,\delta}$ $(0 \le \ell \le n+1, \delta = \pm)$
 is unitarizable.  
\end{enumerate}
\end{theorem}

\medskip
There are four one-dimensional representations of $G$, 
 and they are given by 
\begin{equation*}
\{
  \Pi_{0,+} \simeq {\bf{1}}, \quad \Pi_{0,-} \simeq \chi_{+-}, 
\quad
   \Pi_{n+1,+} \simeq \chi_{-+}, \quad 
   \Pi_{n+1,-} \simeq \chi_{--} (=\det)
\}.  
\end{equation*}
(See \eqref{eqn:chiab} for the definition of
\index{A}{1chipmpm@$\chi_{\pm\pm}$, one-dimensional representation of $O(n+1,1)$}
 $\chi_{\pm\pm}$.)
The other representations $\Pi_{\ell,\delta}$
 ($1 \le \ell \le n$, $\delta=\pm$)
 are infinite-dimensional representations.  



For the subgroup $G'=O(n,1)$,
 we use the letters $\pi_{j,\varepsilon}$ to 
 denote the irreducible representations
 in ${\operatorname{Irr}}(G')_{\rho}$, 
 similar to $\Pi_{i,\delta}$
 in ${\operatorname{Irr}}(G)_{\rho}$.  



With these notations,
 we determine
\[
  m(\Pi_{i,\delta},\pi_{j,\varepsilon}) = \dim_{\mathbb{C}} {\operatorname{Hom}}_{G'}
  (\Pi_{i,\delta}|_{G'},\pi_{j,\varepsilon})
\]
for all $\Pi_{i,\delta} \in {\operatorname{Irr}}(G)_{\rho}$
 and $\pi_{j,\varepsilon} \in {\operatorname{Irr}}(G')_{\rho}$
 as follows.  
\begin{theorem}
[vanishing, 
 see Theorem {\ref{thm:SBOvanish}}]
\label{thm:introSBOvanish}
\index{B}{vanishingtheorem@vanishing theorem}
Suppose $0 \le i \le n+1$, $0 \le j \le n$, 
 $\delta, \varepsilon \in \{\pm\}$.  
\begin{enumerate}
\item[{\rm{(1)}}]
If $j \not = i, i-1$ then
$
  {\operatorname{Hom}}_{G'}(\Pi_{i,\delta}|_{G'}, \pi_{j,\varepsilon})=\{0\}. 
$

\item[{\rm{(2)}}] 
If $\delta \varepsilon =-$, 
 then 
$ 
{\operatorname{Hom}}_{G'}(\Pi_{i,\delta}|_{G'}, \pi_{j,\varepsilon}) =\{0\}.  
$ 
\end{enumerate}
\end{theorem}


\medskip
\noindent

\begin{theorem} 
[multiplicity-one, see Theorem {\ref{thm:SBOone}}]
\label{thm:introSBOone}
\index{B}{multiplicityonetheorem@multiplicity-one theorem}
Suppose $0 \le i \le n+1$, $0 \le j \le n$
 and $\delta, \varepsilon \in \{\pm\}$.  
If $j=i-1$ or $i$
 and if $\delta \varepsilon =+$, 
 then 
\[
   \dim_{\mathbb{C}}
   {\operatorname{Hom}}_{G'}
   (\Pi_{i,\delta}|_{G'}, \pi_{j,\varepsilon}) =1.  
\]
\end{theorem}

\medskip

We can represent these results graphically as follows.  
We suppress the subscript, 
 and write $\Pi_i$ for $\Pi_{i,+}$, 
 and $\pi_j$ for $\pi_{j,+}$. 
 The first row are representations of $G$,
 the second row are representations of $G'$. 
The existence of nonzero symmetry breaking operators is represented by  arrows.


\begin{theorem}
[see Theorem \ref{thm:SBOfg}]
\label{thm:introSBOfg}
Symmetry breaking for  irreducible representations with infinitesimal character $\rho$  is represented graphically in the following form.  

{Symmetry breaking for $O(2m+1,1)\downarrow O(2m,1)$ }


\begin{center}
\begin{tabular}{c@{~}c@{~}c@{~}c@{~}c@{~}c@{~}c@{~}c@{~}c}
$\Pi_0$& &$\Pi_1$& &\dots & & $\Pi_{m-1} $& & $\Pi_{m}$ 
\\
$\downarrow$ &$\swarrow$& $\downarrow $& $\swarrow$ & &$\swarrow$ & $ \downarrow $&  $\swarrow $  &  $\downarrow$ 
\\
$\pi_0$& &$\pi_1$& &\dots & & $\pi_{m-1}$ & & $\pi_{m}$ 
\end{tabular}
\end{center}
\label{fig:introHasse1}


\bigskip


{Symmetry breaking for $O(2m+2,1) \downarrow O(2m+1,1)$ }
\begin{center}
\begin{tabular}{@{}c@{~}c@{~}c@{~}c@{~}c@{~}c@{~}c@{~}c@{~}c@{~}c@{~}c@{}}
$\Pi_0$& &$\Pi_1$& &\dots & & $\Pi_{m-1} $& & $\Pi_{m}$ & & $\Pi_{m+1}$
\\
$\downarrow$ & $\swarrow$ & $\downarrow $ & $\swarrow$ & & $\swarrow$ & $ \downarrow $& $\swarrow $ & $\downarrow$ & $\swarrow$ & \\
$\pi_0$& &$\pi_1$& &\dots & & $\pi_{m-1}$ & & $\pi_{m}$ 
\end{tabular}
\end{center}
\label{fig:introHasse2}
\end{theorem}



We believe that we are seeing in Theorem \ref{thm:SBOfg}
 only the \lq\lq{tip of the iceberg}\rq\rq, 
 and we present a conjecture that a similar statement holds
 in more generality, 
 see Conjecture \ref{conj:GPver1}.  
Suppose that $F$ and $F'$ are irreducible finite-dimensional 
 representations
 of $G$ and the subgroup $G'$, 
 respectively,  and 
 that
\[ \mbox{Hom}_{G'}(F|_{G'},F') \not = \{0\}.\] 
In Chapters \ref{sec:conjecture} and \ref{sec:aq}
 we describe  sequences of irreducible representations $\{\Pi_i\equiv \Pi_i(F) \}$
 and $\{\pi_j \equiv \pi_j(F') \}$ of $G$ and $G'$
 with the same infinitesimal characters with $F$ and $F'$,
 respectively.  
We refer to these  sequences as {\it{standard sequences}}
 that starting with $\Pi_0(F)=F$
 and $\pi_0(F')=F'$, 
 see Definition \ref{def:Hasse}. 
They generalize the standard sequence 
 with trivial infinitesimal character
 which we used in the formulation of Theorem \ref{thm:introSBOfg}.  
They are an analogue of a diagrammatic description
 of irreducible representations 
 with regular integral infinitesimal characters for the connected group $G_0=SO_0(n+1,1)$
 given in Collingwood
 \cite[p.~144, Fig.~6.3]{C}.  
In this generality,
 we conjecture
 that the results of symmetry breaking can be represented
 graphically exactly as in Theorem \ref{thm:introSBOfg} 
 for the representations
 with trivial infinitesimal character $\rho$.   
Again in the first row are representations of $G$, 
 and in the second row are representations of $G'$. 
Conjecture \ref{conj:GPver1} asserts 
 that symmetry breaking operators are represented by  arrows.


 \medskip
{Symmetry breaking for $O(2m+1,1)\downarrow O(2m,1)$ }

\begin{center}
\begin{tabular}{c@{~}c@{~}c@{~}c@{~}c@{~}c@{~}c@{~}c@{~}c@{~}c}
$\Pi_0(F)$ & &$\Pi_1(F)$ & &\dots & & $\Pi_{m-1}(F)$ & &$\Pi_{m}(F)$ 
\\
$\downarrow$ & $\swarrow$ & $\downarrow$ & $\swarrow$ &  & $\swarrow$ & $ \downarrow $ & $\swarrow $ &  $\downarrow$ 
\\
$\pi_0(F')$ & & $\pi_1(F')$ & &\dots & & $\pi_{m-1}(F')$ & & $\pi_{m}(F')$ 
\end{tabular}
\end{center}
\label{fig:introHa1}

 \medskip
{Symmetry breaking for $O(2m+2,1)\downarrow O(2m+1,1)$ }

\begin{center}
\begin{tabular}{@{}c@{~}c@{~}c@{~}c@{~}c@{~}c@{~}c@{~}c@{~}c@{~}c@{~}c@{~}c@{}}
$\Pi_0(F)$& &$\Pi_1(F)$ & & \dots & & $\Pi_{m-1}(F)$& & $\Pi_{m}(F)$ & & $\Pi_{m+1}(F)$
\\
$\downarrow$ &$\swarrow$& $\downarrow $& $\swarrow$ &  & $\swarrow$ & $ \downarrow $& $\swarrow $ & $\downarrow$ & $\swarrow$ & \\
$\pi_0(F')$& &$\pi_1(F')$& &\dots  & & $\pi_{m-1}(F')$ & & $\pi_{m}(F')$ 
\end{tabular}
\end{center}
\label{fig:introHa2}


\medskip \noindent
We present some supporting evidence for this conjecture
 in Chapter \ref{sec:conjecture}.

\bigskip

Applications of our formul{\ae} include
 some results about periods of representations.  
Suppose that $H$ is a subgroup of $G$. 
Following the  terminology used in automorphic forms and the relative trace formula,
 we say that a smooth representation $U $ of $G$ is
\index{B}{distinguishedH@distinguished, $H$-|textbf}
 {\it{$H$-distinguished}} if there is a nontrivial linear $H$-invariant linear functional 
\[F^{H}:U \rightarrow \bC. \]
If the $G$-module $U$ is $H$-distinguished,
 we say that $(F^{H}, H)$ is
 a 
\index{B}{period@period}
 {\it{period}} (or an $H$-period) of $U$.

Let $(G,H)=(O(n+1,1),O(m+1,1))$ with $m \le n$.  
For $0 \le i \le n+1$ and $0 \le j \le m+1$, 
 we denote by $\Pi_{i}$ and $\pi_j$
 the irreducible representations $\Pi_{i,+}$ of $G$
 and analogous ones of $H$ with trivial infinitesimal character $\rho $.  



\begin{theorem} 
[see Theorems \ref{thm:period1} and \ref{thm:171517}]
\
\begin{enumerate}
\item[{\rm{(1)}}]
The irreducible representation $\Pi_i $ is $H$-distinguished if $i \le n-m$.  
\item[{\rm{(2)}}]
The outer tensor product representation
\[ \Pi_i \boxtimes  \pi_j\]
has a nontrivial $H$-period if $0 \le i-j \le n-m$.  
\end{enumerate}
\end{theorem}

\medskip
The period is given by the composition 
 of the normalized regular symmetry breaking operators
 (see Chapter \ref{sec:section7})
 with respect to the chain
 of subgroups:  
\[
  G =O(n+1,1) \supset O(n,1) \supset O(n-1,1) \supset \cdots \supset O(m+1,1)
  =H.  
\]
Using the above chain of subgroups we also define a vector $v$ in the 
\index{B}{minimalKtype@minimal $K$-type}
minimal $K$-type of $\Pi_i$.  
We prove 
\begin{theorem}
[see Theorem \ref{thm:period2}]
\label{thm:intro-period2}
Suppose that $G= O(n+1,1)$ and $\Pi_i$ $(0 \le i \le n)$ is the irreducible representation
 with trivial infinitesimal character $\rho$
 defined as above.
Then the value of the $O(n+1-i,1)$-period on $v \in \Pi_i$ is
\[
  \frac{\pi^{\frac 1 4 i(2n-i-1)}}{((n-i)!)^{i-1}}
\times
\begin{cases}
\frac 1 {(n-2i)!}
\qquad
&\text{if $2i < n+1$, }
\\
(-1)^{n+1} (2i-n-1)!
\qquad
&\text{if $2i \ge n+1$.  }
\end{cases}
\]
\end{theorem} 

\medskip
We also prove in Chapter \ref{sec:period} a generalization of a theorem of Sun \cite{S}.  



\begin{theorem}
[see Theorem \ref{thm:gKOn}]
Let $(G,G')=(O(n+1,1),O(n,1))$, 
 $0 \le i \le n$, 
 and $\delta \in \{ \pm \}$.  
\begin{enumerate}
\item[{\rm{(1)}}]
The symmetry breaking operator
$
  T \colon
  \Pi_{i,\delta} \to \pi_{i,\delta}
$
 in Proposition \ref{prop:AiiAq}
 induces bilinear forms
\[
   B_T \colon 
   H^j({\mathfrak{g}}, K; \Pi_{i,\delta}) 
   \times 
   H^{n-j}({\mathfrak{g}}', K'; \pi_{n-i,(-1)^n \delta})
   \to {\mathbb{C}}
\]
 for all $j$.  
\item[{\rm{(2)}}]
The bilinear form $B_T$ is nonzero
 if and only if $j=i$ and $\delta=(-1)^i$.  

\end{enumerate}
\end{theorem}



\bigskip


Inspired by automorphic forms and number theory B.~Gross and D.~Prasad published  in  1992 conjectures about the multiplicities of irreducible tempered representations $(U,U')$ of $(SO(p,q), SO(p-1,q))$ \cite{GP}.  
Over time these conjectures have been modified and proved
 in some cases for automorphic forms
 and for $p$-adic orthogonal and unitary groups. See for example  Ast\'{e}risque volumes \cite{A1, A2}
 by W.~T.~Gan, B.~Gross, D.~Prasad, C.~M{\oe}glin and J.-L. Waldspurger 
 and the references therein as well as the work
 by R.~Beuzart-Plessis \cite{raphael} for the unitary groups.

We prove the multiplicity conjecture by B.~Gross and D.~Prasad for  tempered principal series representations of $(SO(n+1,1), SO(n,1))$  and also for 3 representations $\Pi, \pi, \varpi$
 of $SO(2m+2,1)$, $SO(2m+1,1)$ and $SO(2m,1)$ with infinitesimal character $\rho$.
More precisely we show:

\begin{theorem}
[Theorem \ref{thm:VWSBO}]  
Suppose that  
 $\Pi= I_\delta (V,\lambda)$, $\pi=J_\varepsilon(W,\nu )$ are (smooth) tempered principal series representations of $G=O(n+1,1)$ and $G'=O(n,1)$. 
Then
\[ {\operatorname{dim}}_{\mathbb{C}} {\operatorname{Hom}}_{G'}
   (\Pi|_{G'},\pi)  = 1  
\quad
\text{if and only if} 
\quad
  [V:W] \ne 0.
\] 
\end{theorem}
Restricting the principal series representations  to special orthogonal groups implies the conjecture of B.~Gross and D.~Prasad about multiplicities 
for tempered principal series representations
 (Theorem \ref{theorem:GPtemp}).



In 2000 B.~Gross and N.~Wallach \cite{GW} showed
 that the restriction of {\it{small}} discrete series representations
 of $G=SO(2p+1,2q)$ to $G'=SO(2p,2q)$ 
satisfies the Gross--Prasad conjectures
 \cite{GP}. 
In that case,
 both the groups $G$ and $G'$ admit discrete series representations.  
On the other hand,
 for the pair $(G,G')=(SO(n+1,1),SO(n,1))$, 
 only one of $G$ or $G'$ admits discrete series representations.  
Our results confirm the Gross--Prasad conjecture
 also for {\bf tempered} representations
 with trivial infinitesimal character $\rho$
 (Theorem \ref{thm:GPdisctemp}). 



\bigskip \noindent
The article is roughly divided in three parts and an appendix: 



In the first part, 
 Chapters \ref{sec:ps}--\ref{sec:SBOrho}, 
 we give an overview of the notation
 and the results about symmetry breaking operators.  
Notations and properties for principal series and irreducible representations of orthogonal groups are introduced in Chapter \ref{sec:ps}. 
Important concepts and properties of symmetry breaking operators are discussed 
 in Chapter \ref{sec:general}, 
 in particular,
 a classification scheme of all symmetry breaking operators
 is presented in Theorem \ref{thm:VWSBO}.  
This includes a number of theorems about the dimension
 of the space of symmetry operators for principal series representations
 which are stated and discussed also in Chapter \ref{sec:general}. 
The classification scheme is carried out
 in full details for symmetry breaking from principal series representations
 $I_\delta(i,\lambda)$ of $G$ to $J_\varepsilon(j,\nu)$
 of the subgroup $G'$, 
 and is used to obtain results on symmetry breaking of 
 irreducible representations with trivial infinitesimal character $\rho $
 in Chapter \ref{sec:SBOrho}. 



The second part,
 Chapters \ref{sec:section7}--\ref{sec:holo},
 contains the proofs of the results discussed in Part one. 
This is the technical heart of this monograph.  
In Chapter \ref{sec:section7} the estimates and results about regular symmetry breaking operators in Theorems \ref{thm:intro160150} and \ref{thm:VW0SBO}
 are proved. 
Chapter \ref{sec:DSVO} is devoted to differential symmetry breaking operators. 
In the remaining chapters of this part we concentrate on the symmetry breaking  $I_\delta(i,\lambda) \rightarrow J_\varepsilon(j,\nu )$. 
We collect some technical results
 in Chapters \ref{sec:section9} and \ref{sec:psdetail}.  
The analytic continuation of the regular symmetry breaking operators, 
 their $(K,K')$-spectrum, 
 and the functional equation are discussed in Chapter \ref{sec:holo}.
Many of the results and techniques developed here are of independent interest, 
 and would be applied to other problems.  



In the third part,
 Chapters \ref{sec:Gross-Prasad}--\ref{sec:conjecture}, 
 we use the results in Chapters \ref{sec:general} and \ref{sec:SBOrho}
 to prove some of the conjectures of Gross and Prasad about symmetry breaking for tempered representations of orthogonal groups in Chapter \ref{sec:Gross-Prasad}. 
We discuss periods of representations and a bilinear form
 on the ($\mathfrak{g},K$)-cohomology using symmetry breaking
 in Chapter \ref{sec:period}.  
It also includes a conjecture about symmetry breaking for a family of representations of irreducible representations
 with regular integral infinitesimal character
 in Chapter \ref{sec:conjecture},  
 which we plan to attack in a sequel to this monograph. 
A major portion of Part 3 can be read immediately after Part 1. 



The appendix contains technical results used in the monograph. 
We provide three characterizations of irreducible representations
 of the group $G=O(n+1,1)$:
Langlands quotients (or subrepresentations), 
 cohomological parabolic induction, 
 and translation from ${\operatorname{Irr}}(G)_{\rho}$.  
The first two are discussed in Appendix I
 (Chapter \ref{sec:aq})
 and the third one is in Appendix III
 (Chapter \ref{sec:Translation}).  
For the second description,
 we recall the description of the Harish-Chandra modules
 of the irreducible representations of $O(n,1)$
 as the cohomological induction from a $\theta$-stable Levi subgroup
 and introduce $\theta$-stable coordinates for irreducible representations
 with regular integral infinitesimal character. This notation is used in the formulation of the conjecture in Chapter \ref{sec:conjecture}.  
We discuss the restriction of representations of the orthogonal group $O(n,1)$ to the special orthogonal group $SO(n,1)$
 in  Appendix II (Chapter \ref{sec:SOrest}). 
The results are used in Chapter \ref{sec:Gross-Prasad}
 about the Gross--Prasad conjecture.  
In Appendix III, 
 we discuss translation functor of $G=O(n+1,1)$
 which is not in the Harish-Chandra class
 when $n$ is even. 



\bigskip
{\bf{Acknowlegements:}}\enspace
 Many of the results were obtained while the authors were supported by the Research in Pairs program   at the Mathematisches Forschungsinstitut MFO in Oberwolfach, Germany.
 
Research by the first author was partially supported by Grant-in-Aid for Scientific 
Research (A)
(25247006) and (18H03669), Japan Society for the Promotion of Science.  

Research by the second author was partially supported by  NSF grant DMS-1500644. 
Part of this research was conducted during a visit of the second author at  the Graduate School of Mathematics of The University of Tokyo, Komaba. 
She would like to thank it for its support and hospitality during her stay.

\vskip 3pc
{\bf{Notation:}}
\index{A}{N0natural@${\mathbb{N}}$}
\index{A}{R1real1@${\mathbb{R}}_+$}
\begin{align*}
& A \setminus B
\quad
&&\text{set theoretic complement of $B$ in $A$}
\\
&{\mathbb{N}}
&&\text{$\{\text{integers $\ge 0$}\}$}
\\
&{\mathbb{N}}_+
&&\text{$\{\text{positive integers}\}$}
\\
&{\mathbb{R}}_+
&&\text{$\{t \in {\mathbb{R}}:t>0\}$}
\\
& {\operatorname{Image}}\, (T)
&&\text{image of the operator $T$}
\\
& {\operatorname{Ker}}\, (T)
&&\text{kernel of the operator $T$}
\\
& E_{i j}
&&\text{the matrix unit}
\\
&[a]
&&\text{the largest integer that does not exceed $a$}
\\
&{\bf{1}}
&&\text{the trivial one-dimensional representation}
\\
& \pi^{\vee}
&&\text{the contragredient representation of $\pi$}
\\
& \pi_1 \boxtimes \pi_2
&&\text{the outer tensor product representation of a direct product group}
\\
& \pi_1 \otimes \pi_2
&&\text{the tensor product representation}
\\
& \rho (\equiv \rho_G)
&&\text{the infinitesimal character of the trivial representation ${\bf{1}}$}
\end{align*}


