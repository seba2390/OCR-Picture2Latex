\newpage
\section{Appendix II: Restriction to $\overline G=SO(n+1,1)$}
\label{sec:SOrest}

So far we have been working with symmetry breaking for a pair of the orthogonal groups
 $(O(n+1,1), O(n,1))$.  
On the other hand, 
 the Gross--Prasad conjectures
 (Chapters \ref{sec:Gross-Prasad} and \ref{sec:conjecture})
 are formulated for special orthogonal groups
 rather than orthogonal groups.  
In this chapter, 
 we explain how to translate the results 
 for $(G, G')=(O(n+1,1), O(n,1))$
 to those for the pair of special orthogonal groups
\index{A}{GOindefinitespecial@$\overline{G}=SO(n+1,1)$|textbf}
 $(\overline{G},\overline{G'})=(SO(n+1,1), SO(n,1))$.  
A part of the results here 
({\it{e.g.}}, Theorem \ref{thm:SOmult}) was announced in \cite{sbonGP}.  



In what follows, 
 we use a bar over representations of special orthogonal groups
 to distinguish them from those of orthogonal groups.  


\subsection{Restriction of representations of $G=O(n+1,1)$ to $\overline G= SO(n+1,1)$}
\label{subsec:SO}

It is well-known that any irreducible admissible representation $\Pi$
 of a real reductive group $G$ is decomposed into the direct sum 
 of finitely many irreducible admissible representations of $\overline G$
if $\overline G$ is an open normal subgroup of $G$
 (see \cite[Chap.~II, Lem.~5.5]{BW}).  
In order to understand
 how the restriction $\Pi|_{\overline G}$ decomposes,
 we use the action
 of the quotient group $G/\overline G$
 on the ring ${\operatorname{End}}_{\overline G}(\Pi|_{\overline G}) =
 {\operatorname{Hom}}_{\overline G}(\Pi|_{\overline G}, \Pi|_{\overline G})$.  



We apply this general observation to our setting 
where 
\[(G,\overline G)=(O(n+1,1), SO(n+1,1)).  
\]
In this case, 
 the quotient group $G/\overline G \simeq {\mathbb{Z}}/2 {\mathbb{Z}}$.  
With the notation \eqref{eqn:chiab} of the characters $\chi_{a b}$ of $G$, 
\index{A}{1chipmm@$\chi_{--}=\det$}
\[
   \{\chi_{++}, \chi_{--}\}=\{{\bf{1}}, \det\}
\]
 is the set
 of irreducible representations
 of $G=O(n+1,1)$
 which are trivial on $\overline G=S O(n+1,1)$.  
In other words,
 we have a direct sum decomposition as $G$-modules:
\[
   {\operatorname{Ind}}_{\overline G}^G \overline{\bf{1}}
  \simeq
   {\bf{1}} \oplus \det.  
\]
Then we have the following:
\begin{lemma}
\label{lem:LNM26}
Let $\Pi$ be a continuous representation of $G=O(n+1,1)$.  
Then there is a natural linear bijection:
\index{A}{1chipmpm@$\chi_{\pm\pm}$, one-dimensional representation of $O(n+1,1)$}
\[
  {\operatorname{End}}_{\overline{G}}(\Pi|_{\overline{G}})
\simeq
  \operatorname{Hom}_{G}(\Pi, \Pi)
  \oplus
  \operatorname{Hom}_{G}(\Pi, \Pi \otimes \det).  
\]
\end{lemma}

\begin{proof}
Clear from the following linear bijections:
\[
{\operatorname{End}}_{\overline G}(\Pi|_{\overline{G}})
\simeq 
{\operatorname{Hom}}_{G}(\Pi, {\operatorname{Ind}}_{\overline G}^{G}(\Pi|_{\overline{G}}))
\simeq 
{\operatorname{Hom}}_{G}(\Pi, \Pi \otimes {\operatorname{Ind}}_{\overline G}^{G}\overline{\bf{1}}).  
\]
\end{proof}

We examine the restriction
 of irreducible representations of $G$
 to the subgroup $\overline G$:

\begin{lemma}
\label{lem:171523}
Suppose that $\Pi$ is an irreducible admissible representation of $G=O(n+1,1)$.  
\begin{enumerate}
\item[{\rm{(1)}}]
If $\Pi \not \simeq \Pi \otimes \det$ as $G$-modules, 
 then the restriction $\Pi|_{\overline G}$ is irreducible.  
\item[{\rm{(2)}}]
If $\Pi \simeq \Pi \otimes \det$ as $G$-modules, 
then the restriction $\Pi|_{\overline G}$ is the direct sum 
of two irreducible admissible representations
of $\overline G$ that are not isomorphic to each other.  
\end{enumerate}
\end{lemma}
\begin{proof}
By Lemma \ref{lem:LNM26}, 
 we have 
\begin{align*}
\dim_{\mathbb{C}}{\operatorname{Hom}}_{\overline G}(\Pi|_{\overline G}, \Pi|_{\overline G})
=& \dim_{\mathbb{C}} {\operatorname{Hom}}_{G}(\Pi, \Pi) 
+ \dim_{\mathbb{C}}{\operatorname{Hom}}_{G}(\Pi, \Pi \otimes \det)
\\
=& 
\begin{cases}
1 \quad &\text{if $\Pi \not \simeq \Pi \otimes \det$, }
\\
2 \quad &\text{if $\Pi \simeq \Pi \otimes \det$.  }
\end{cases}
\end{align*}
Since the restriction $\Pi|_{\overline G}$ is the direct sum
 of irreducible admissible representations 
 of ${\overline G}$, 
 we may write the decomposition as 
\[
  \Pi|_{\overline G} \simeq \bigoplus_{j=1}^{N} m_j \overline{\Pi}_j,
\]
where $\overline{\Pi}_j$ are 
 (mutually inequivalent)
 irreducible admissible representations of $\overline G$
 and $m_j \in {\mathbb{N}}_+$
 denote the multiplicity of $\overline{\Pi}_j$ in $\Pi|_{\overline G}$
 for $1 \le j \le N$.  
By Schur's lemma, 
\[
  \dim_{\mathbb{C}} {\operatorname{End}}_{\overline{G}}
  (\Pi|_{\overline G}) = \sum_{j=1}^N m_j^2.  
\]
This is equal to 1 or 2 
 if and only if $N=m_1 =1$ or $N=2$ and $m_1 = m_2=1$, 
respectively.  
Hence we get  the conclusion.  
\end{proof}



\subsection{Restriction of principal series representation
 of $G=O(n+1,1)$ to $\overline G=SO(n+1,1)$ }
\label{subsec:psrest}
This section discusses the restriction of the principal series representation $I_{\delta}(V, \lambda)$
 of $G=O(n+1,1)$ to the normal subgroup $\overline G= SO(n+1,1)$ of index two.  
First of all, 
 we fix some notation 
 for principal series representations of $\overline G$.  
We set 
\index{A}{PLanglandsdecompbar@$\overline P = \overline M A N_+$|textbf}
$\overline P := P \cap \overline G$. 
Then $\overline P$ is a minimal parabolic subgroup of $\overline G$, 
 and its Langlands decomposition is given
 by $\overline P = \overline M A N_+$, 
 where 
\index{A}{M1primebar@$\overline M=SO(n) \times O(1)$}
\[
\overline M := M \cap \overline G
=
\{ 
\begin{pmatrix}
               \varepsilon & &
\\
                           & B & 
\\
                           & & \varepsilon
\end{pmatrix}
:
B \in SO(n), \varepsilon= \pm 1
\}
\simeq
SO(n) \times O(1)
\]
 is a subgroup of $M$ of index two.  
For an irreducible representation $(\overline\sigma, \overline V)$ of $SO(n)$, 
 $\delta \in \{\pm\}$, 
 and $\lambda \in {\mathbb{C}}$, 
 we denote by 
\index{A}{IdeltaVsbar@$\overline I_{\delta}(V, \lambda)$}
$\overline I_{\delta}(\overline V,\lambda)$
 the (unnormalized) induced representation
$
    \operatorname{Ind}_{\overline P}^{\overline G}
    (\overline V \otimes \delta \otimes {\mathbb{C}}_{\lambda})
$
 of $\overline G =SO(n+1,1)$.  



Let us compare principal series representations of $G$
 regarded as $\overline G$-modules
 by restriction with principal series representations of $\overline G$.  
For this, 
 we suppose $V$ is an irreducible representation of $O(n)$, 
 $\delta \in \{\pm \}$, and $\lambda \in {\mathbb{C}}$, 
 and form a principal series representation $I_{\delta}(V,\lambda)$
 of $G=O(n+1,1)$.  
Then its restriction to the subgroup $\overline G=SO(n+1,1)$ is isomorphic to 
$
   \operatorname{Ind}_{\overline P}^{\overline G}
   (V|_{SO(n)} \otimes \delta \otimes {\mathbb{C}}_{\lambda})
$
 as a $\overline G$-module, 
 because the inclusion $\overline G \hookrightarrow G$
 induces an isomorphism
 $\overline G/\overline P \overset \sim \to G/P$.  



Concerning the $SO(n)$-module $V|_{SO(n)}$, 
 we recall from Definition \ref{def:OSO} that $V \in \widehat {O(n)}$ is said to be 
\index{B}{typeX@type X, representation of ${O(N)}$\quad}
 of type X
or 
\index{B}{typeY@type Y, representation of ${O(N)}$\quad}
 of type Y
 according to whether $V$ is irreducible or reducible 
when restricted to $SO(n)$.  
In the latter case, 
$n$ is even (see Lemma \ref{lem:OSO}) 
 and $V$ is decomposed into the direct sum of two irreducible representations
 of $SO(n)$:
\begin{equation}
\label{eqn:Vpm}
   V= V^{(+)} \oplus V^{(-)}, 
\end{equation}
where $V^{(-)}$ is isomorphic to the contragredient representation 
 of $V^{(+)}$.  
Accordingly, 
 we have an isomorphism 
 as $\overline G$-modules:
\index{A}{IdeltaVsbarpm@$\overline I_{\delta}(V^{(\pm)}, \lambda)$|textbf}
\begin{equation}
\label{eqn:IVSOpm}
   I_{\delta}(V,\lambda)|_{\overline G}
   \simeq
   \begin{cases}
   \overline{I}_{\delta}(V,\lambda)
   & \text{if $V$ is of type X, }
\\
  \overline{I}_{\delta}(V^{(+)}, \lambda)
   \oplus
  \overline{I}_{\delta}(V^{(-)}, \lambda)
\quad
     & \text{if $V$ is of type Y.}  
   \end{cases}
\end{equation}
By using \eqref{eqn:IVSOpm}, 
 we obtain the structural results of the restriction 
 of the principal series representation ${I}_{\delta}(V,\lambda)$
 of $G=O(n+1,1)$
 to the subgroup $\overline G=SO(n+1,1)$
 and further to the identity component group 
\index{A}{GOindefinitespecialidentity@$G_0=SO_0(n+1,1)$, the identity component
 of $O(n+1,1)$\quad}
 $G_0=SO_0(n+1,1)$.  


\subsubsection{Restriction $I_{\delta}(V,\lambda)|_{\overline G}$
 when $I_{\delta}(V,\lambda)$ is irreducible}
We begin with the case
 where $I_{\delta}(V,\lambda)$ is irreducible as a $G$-module.  
\begin{lemma}
\label{lem:170305}
Let $(\sigma,V) \in \widehat{O(n)}$, 
$\delta \in \{\pm \}$
 and $\lambda \in {\mathbb{C}}$.  
Suppose $I_{\delta}(V,\lambda)$ is irreducible 
 as a module of $G=O(n+1,1)$.  
\par\noindent
{\rm{(1)}}\enspace
Suppose $V$ is of type X.  
Then the following three conditions on $(\delta,V,\lambda)$
 are equivalent:
\begin{enumerate}
\item[{\rm{(i)}}]
$I_{\delta}(V,\lambda)$ is irreducible
 as a $G$-module;
\item[{\rm{(ii)}}]
The restriction $I_{\delta}(V,\lambda)|_{\overline G}$ is irreducible
 as a $\overline G$-module;
\item[{\rm{(iii)}}]
The restriction $I_{\delta}(V,\lambda)|_{G_0}$ is irreducible
 as a $G_0$-module.  
\end{enumerate}
{\rm{(2)}}\enspace
Suppose $V$ is of type Y.  
If $I_{\delta}(V,\lambda)$ is irreducible as a $G$-module, 
 then $I_{\delta}(V,\lambda)|_{\overline G}$ splits
 into the direct sum of two irreducible $\overline G$-modules
 that are not isomorphic to each other.  
In this case, 
 $n$ is even
 and we may write the irreducible decomposition
 of $V|_{SO(n)}$
 as in \eqref{eqn:Vpm}.  
Then there is a natural isomorphism 
\[
I_{\delta}(V,\lambda)|_{\overline G}
\simeq
 \overline{I}_{\delta}(V^{(+)},\lambda) \oplus \overline{I}_{\delta}(V^{(-)},\lambda)
\]
 as $\overline G$-modules.  
Moreover,
 both $\overline I_{\delta}(V^{(+)}, \lambda)$
 and $\overline I_{\delta}(V^{(-)}, \lambda)$ stay irreducible
 when restricted to $G_0$, 
 and they are not isomorphic to each other also as $G_0$-modules.  
\end{lemma}

\begin{proof}
We observe
 that the first factor of $M$ is isomorphic to $O(n)$, 
 whereas that of $M \cap \overline G$
 $(=\overline M)$
 and of $M \cap G_0$ is isomorphic to $S O(n)$.  
Since the crucial part is the restriction from 
 the Levi subgroup $M A$ of $G$
 to that of $\overline G$ or of $G_0$, 
 we focus on the restriction $G \downarrow \overline G$, 
 which involves the restriction of $V$
 with respect to the inclusion $O(n) \supset S O(n)$.  
The restriction $G \downarrow G_0$ can be analyzed
 similarly by using the four characters
 $\chi_{a b}$ ($a,b \in \{\pm\}$)
 instead of $\chi_{--}=\det$
 as in \cite[Chap.~2, Sect.~5]{KKP}.



{}From now on,
 we consider the restriction $G \downarrow \overline G$.  
We recall from Lemma \ref{lem:IVchi}
 the following isomorphism of $G$-modules:
\index{A}{1chipmm@$\chi_{--}=\det$}
\[I_{\delta}(V, \lambda) \otimes \chi_{--}
 \simeq
 I_{\delta}(V \otimes \det, \lambda).  
\]
\par\noindent
(1)\enspace
If $V$ is of type X, 
then $V \not \simeq V \otimes \det$ as $O(n)$-modules.  
In turn, 
 the $G$-modules $I_{\delta}(V,\lambda)$ and $I_{\delta}(V\otimes \det,\lambda)$
 are not isomorphic to each other, 
 because their $K$-structures are different
 by the Frobenius reciprocity
 and the branching rule for $O(n) \downarrow O(n-1)$
 (Fact \ref{fact:ONbranch}).  
Therefore, 
 $I_{\delta}(V,\lambda)|_{\overline G}$ is irreducible
 by Lemma \ref{lem:171523} (1).  
\par\noindent
(2)\enspace
If $V$ is of type Y, 
 then $V \otimes \det \simeq V$
 by Lemma \ref{lem:typeY}, 
 and therefore Lemma \ref{lem:171523} (2) concludes the first assertion.  
The remaining assertions are now clear.  
\end{proof}

\subsubsection{Restriction $I_{\delta}(V,\lambda)|_{\overline G}$
 when $V$ is of type Y}
We take a closer look at the case
 where $V \in \widehat {O(n)}$ is of type Y
 (Definition \ref{def:OSO}).  
This means that $n$ is even, 
 say $n=2m$, 
 and the representation $V$ is of the form
\[
   V = \Kirredrep{O(2m)}{\sigma_1, \cdots,\sigma_m}_{\varepsilon}
\]
with $\sigma_1 \ge \cdots \ge \sigma_m \ge 1$
 and $\varepsilon \in \{\pm\}$, 
 see Section \ref{subsec:fdimrep} for the notation.  
Then the restriction $V|_{S O(n)}$ decomposes
 as 
\[
  V|_{S O(n)}=V^{(+)} \oplus V^{(-)} 
\]
as in \eqref{eqn:Vpm}, 
 where the highest weights of the irreducible $SO(2m)$-modules
  $V^{(\pm)}$ are given by $(\sigma_1, \cdots,\sigma_{m-1}, \pm \sigma_m)$.  
We recall from Definition \ref{def:SVSYV}
 for the subsets $S(V)$ and $S_Y(V)$ of ${\mathbb{Z}}$.  

\begin{proposition}
\label{prop:irrIVpm}
Suppose $G=O(n+1,1)$ with $n=2m$
 and $(\sigma,V) \in \widehat {O(n)}$ is of type Y.  
Let $\delta \in \{\pm\}$.  
\begin{enumerate}
\item[{\rm{(1)}}]
The following four conditions on $\lambda  \in {\mathbb{C}}$ 
 are equivalent.  
\begin{enumerate}
\item[{\rm{(i)}}]
$\overline I_{\delta}(V^{(+)},\lambda)$ is reducible
 as a representation 
 of $\overline G=SO(n+1,1);$
\item[{\rm{(ii)}}]
$\overline I_{\delta}(V^{(-)},\lambda)$ is reducible
 as a $\overline G$-module$;$
\item[{\rm{(iii)}}]
$\pm(\lambda-m)\in {\mathbb{Z}}
 \setminus 
 (\{\sigma_j+m-j:j=1,\cdots,m\} \cup \{0,1,2,\cdots,\sigma_m-1\});$
\item[{\rm{(iv)}}]
$\lambda \in {\mathbb{Z}} \setminus
(S(V) \cup S_Y(V) \cup \{m\})$.  
\end{enumerate}
\item[{\rm{(2)}}]
Suppose that $\lambda$ satisfies
 one of (therefore any of)
 the above equivalent conditions.  
Then, 
 for $\varepsilon \in \{\pm\}$, 
 the principal series representation
 $\overline I_{\delta}(V^{(\varepsilon)}, \lambda)$ of $\overline G$
 has a unique $\overline G$-submodule, 
 to be denoted by $\overline I_{\delta}(V^{(\varepsilon)}, \lambda)^{\flat}$, 
 such that the quotient $\overline G$-module
\index{A}{IdeltaVsbarpmflat@$\overline I_{\delta}(V^{(\pm)}, \lambda)^{\flat}$|textbf}
\index{A}{IdeltaVsbarpms@$\overline I_{\delta}(V^{(\pm)}, \lambda)^{\sharp}$|textbf}
\[
   \overline I_{\delta}(V^{(\varepsilon)}, \lambda)^{\sharp}
  :=\overline I_{\delta}(V^{(\varepsilon)}, \lambda)
   /\overline I_{\delta}(V^{(\varepsilon)}, \lambda)^{\flat}
\]
 is irreducible.  
Moreover we have
\begin{align*}
   \overline I_{\delta}(V^{(+)}, \lambda)^{\flat}
 &\not\simeq 
 \overline I_{\delta}(V^{(-)}, \lambda)^{\flat}, 
\\
  \overline I_{\delta}(V^{(+)}, \lambda)^{\sharp}
  &\not\simeq 
  \overline I_{\delta}(V^{(-)}, \lambda)^{\sharp}
\end{align*}
as $\overline G$-modules.  
\end{enumerate}
\end{proposition}

\begin{proof}
Since $\overline G=SO(2m+1,1)$ is generated 
 by the identity component 
 $G_0=SO_0(2m+1,1)$
 and a central element $-I_{2m+2}$, 
 any irreducible $\overline G$-module remains
 irreducible 
 when restricted to the connected subgroup $G_0$.  
Then the equivalence (i) $\Leftrightarrow$ (iii)
(also (ii) $\Leftrightarrow$ (iii))
 and the last assertion 
 in Proposition \ref{prop:irrIVpm} follows from
 Hirai \cite{Hirai62}.  
See also Collingwood \cite[Lem.~4.4.1 and Thm.~5.2.1]{C}
 for the computation of $\tau$-invariants
 of irreducible representations
 and a graphic description of the socle filtrations
 of principal series representations.  
Finally the equivalence (iii) $\Leftrightarrow$ (iv)
 is immediate from the definitions
 \eqref{eqn:singint} and \eqref{eqn:SYsigma}
 of $S(V)$ and $S_Y(V)$, 
 respectively.  



The last assertion about the $\overline G$-inequivalence follows from
 the Langlands theory \cite{La88}
 because $\operatorname{Re} \lambda \ne m$
 and $V^{(+)} \not \simeq V^{(-)}$
 as $SO(2m)$-modules.  
\end{proof}



In the following proposition,
 we treat the set
 of the parameters $\lambda$
 complementary to the one in Proposition \ref{prop:irrIVpm}.  
\begin{proposition}
\label{prop:180572}
Suppose $G=O(n+1,1)$ with $n=2m$
 and $V \in \widehat {O(n)}$ is of type Y.  
Let $\delta \in \{\pm\}$.  
Assume that $\overline I_{\delta}(V^{(\pm)},\lambda)$ are irreducible
 representations of $\overline G=SO(2m+1,1)$, 
 or equivalently, 
 assume that 
\[
   \lambda \in ({\mathbb{C}} \setminus {\mathbb{Z}}) \cup S(V) \cup S_Y(V) \cup \{m\}.  
\]
\begin{enumerate}
\item[{\rm{(1)}}]
The following two conditions on $\lambda \in {\mathbb{C}}$ 
 are equivalent:
\begin{enumerate}
\item[{\rm{(i)}}]
The two $\overline G$-modules $\overline I_{\delta}(V^{(+)},\lambda)$
 and $\overline I_{\delta}(V^{(-)},\lambda)$ are isomorphic
 to each other;
\item[{\rm{(ii)}}]
$\lambda =m$.  
\end{enumerate}

\item[{\rm{(2)}}]
If $\lambda=m$ then the principal series representation
 $I_{\delta}(V,\lambda)$ of $G$ is decomposed 
 into the direct sum of two irreducible representations
 of $G$.  
\item[{\rm{(3)}}]
If $\lambda \ne m$,  then $I_{\delta}(V,\lambda)$ is irreducible
 as a representation of $G$.  
\end{enumerate}
\end{proposition}

\begin{proof}
(1)\enspace
As in the proof of Proposition \ref{prop:irrIVpm} (2), 
 if $\operatorname{Re}\lambda \ne m$, 
 then the Langlands theory \cite{La88} implies
 $\overline I_{\delta}(V^{(+)},\lambda) \not \simeq \overline I_{\delta}(V^{(-)},\lambda)$
 because $V^{(+)} \not \simeq V^{(-)}$
 as $SO(2m)$-modules.  


If $\operatorname{Re}\lambda = m$, 
 then $\overline I_{\delta}(V^{(\pm)},\lambda)$ are (smooth)
 irreducible tempered representations,
 and the equivalence
 (i) $\Leftrightarrow$ (ii) follows from Hirai \cite{Hirai62}.  
This would follow also from the general theory 
 of the \lq\lq{$R$-group}\rq\rq\ (Knapp--Zuckerman \cite{KZ}).  
\par\noindent
(2)\enspace
Since $\operatorname{Re}\lambda = m$ is the unitary axis
 of the principal series representation
 $I_{\delta}(V,\lambda)$ 
 in our normalization
 (Section \ref{subsec:smoothI}), 
 the $G$-module $I_{\delta}(V,\lambda)$ decomposes into the direct sum
 of irreducible $G$-modules,
 say, 
 $\Pi^{(1)}$, $\dots$, $\Pi^{(k)}$, 
 and then decomposes further into irreducible $\overline G$-modules
 when restricted to the subgroup 
 $\overline G=SO(2m+1,1)$.  
Therefore the cardinality $k$ of irreducible $G$-summands satisfies
 either $k=1$
 ({\it{i.e.}}, $I_{\delta}(V,\lambda)$ is $G$-irreducible)
 or $k=2$ because the summands $\overline I_{\delta}(V^{(\pm)},\lambda)$ 
 in \eqref{eqn:IVSOpm} are irreducible as $\overline G$-modules
 by assumption.  
Since $\overline I_{\delta}(V^{(+)},m)
 \simeq \overline I_{\delta}(V^{(-)},m)$
 by the first statement, 
 we conclude $k \ne 1$ 
 by Lemma \ref{lem:170305} (2).  
Thus the second statement is proved.  
\par\noindent
(3)\enspace
We prove 
 that $I_{\delta}(V,\lambda)$ is irreducible
 by {\it{reductio ad absurdum}}.  
Suppose there were an irreducible proper submodule $\Pi$
 of $I_{\delta}(V,\lambda)$.  
Then $\Pi$ would remain irreducible 
 when restricted to the subgroup $\overline G=SO(2m+1,1)$
 because the restriction $\Pi|_{\overline G}$ must be isomorphic 
 to one of the $\overline G$-irreducible summands
 $\overline I_{\delta}(V^{(\pm)},\lambda)$ 
 in \eqref{eqn:IVSOpm}.  
Then $\Pi \not \simeq \Pi \otimes \det$
 as $G$-modules
 by Lemma \ref{lem:LNM26}.  
Therefore the direct sum $\Pi \oplus (\Pi \otimes \det)$ would be a $G$-submodule
 of $I_{\delta}(V,\lambda)$
 because $I_{\delta}(V,\lambda) \simeq I_{\delta}(V,\lambda) \otimes \det$
 when $V$ is of type Y.  
In turn, 
 its restriction to the subgroup $\overline G$ would yield 
 an isomorphism 
 $\overline I_{\delta}(V^{(+)},\lambda) \simeq \overline I_{\delta}(V^{(-)},\lambda)$ of $\overline G$-modules,
 contradicting the statement (1)
 of the proposition.  
Hence $I_{\delta}(V,\lambda)$ must be irreducible.  
\end{proof}

Applying Propositions \ref{prop:irrIVpm} and \ref{prop:180572}
 to the middle exterior representation
 $\Exterior^m({\mathbb{C}}^{n})$ of $O(n)$
 when $n=2m$, 
 we obtain the following.
\begin{example}
\label{ex:Imm}
Let $G=O(n+1,1)$ with $n=2m$, 
 and $\delta \in \{\pm\}$.  
As in \eqref{eqn:ISOneven}, 
 we write $\overline I_{\delta}^{(\pm)}(m,\lambda)$
 for the $\overline G$-modules
 $\overline I_{\delta}(V^{(\pm)},\lambda)$
 when $V=\Exterior^m({\mathbb{C}}^{2m})$.  
\begin{enumerate}
\item[{\rm{(1)}}]
The $\overline G$-modules
 $\overline I_{\delta}^{(\pm)}(m,\lambda)$ are reducible
 if and only if $\lambda \in (-{\mathbb{N}}_+) \cup (n+{\mathbb{N}}_+)$.  
\item[{\rm{(2)}}]
The $G$-module $I_{\delta}(m,m)$ decomposes
 into a direct sum of two irreducible $G$-modules
 (see also Theorem \ref{thm:LNM20} (1)).  
\item[{\rm{(3)}}]
$I_{\delta}(m,\lambda)$ is irreducible
 if $\lambda \in {\mathbb{Z}}$ 
 satisfies $0 \le \lambda \le n$ $(=2m)$
 and $\lambda \ne m$.  
\end{enumerate}
\end{example}

We refer to Theorem \ref{thm:LNM20}
 (also to Example \ref{ex:irrIilmd})
 for the irreducibility condition
 of $I_{\delta}(i,\lambda)$
 for general $i$ $(0 \le i \le n)$;
to Theorem \ref{thm:irrIV}
 for that of $I_{\delta}(V,\lambda)$, 
 which will be proved in the next section. 



\subsection{Proof of Theorem \ref{thm:irrIV}: 
 Irreducibility criterion of $I_{\delta}(V,\lambda)$}
\label{subsec:pfirrIV}

As an application of the results in the previous sections, 
 we give a proof of Theorem \ref{thm:irrIV}
 on the necessary and sufficient condition
 for the principal series representation 
 $I_{\delta}(V,\lambda)$
 of $G=O(n+1,1)$ to be irreducible.  
\begin{proof}
[Proof of Theorem \ref{thm:irrIV}]
Suppose first that $V$ is of type X
 (Definition \ref{def:OSO}). 
Then the restriction $V|_{SO(n)}$ is irreducible as an $SO(n)$-module,
 and $I_{\delta}(V,\lambda)$ is $G$-irreducible
 if and only if the restriction $I_{\delta}(V,\lambda)|_{G_0}$ is 
 $G_0$-irreducible 
 by Lemma \ref{lem:170305} (1).  
The latter condition was classified in Hirai \cite{Hirai62}, 
 which amounts to the condition that $\lambda \not \in {\mathbb{Z}}$
 or $\lambda \in S(V)$.  
Thus Theorem \ref{thm:irrIV} for $V$ of type X is proved.  



Next suppose $V$ is of type Y.  
As in \eqref{eqn:Vpm}, 
 we write $V|_{SO(n)} \simeq V^{(+)} \oplus V^{(-)}$
 for the irreducible decomposition as $SO(n)$-modules.  
If $I_{\delta}(V,\lambda)$ is $G$-irreducible, 
then $\overline I_{\delta}(V^{(\pm)},\lambda)$ are $\overline G$-irreducible
 by Lemma \ref{lem:170305} (2).  
Then the condition (iv) in Proposition \ref{prop:irrIVpm} (1) implies
 that
\begin{equation}
\label{eqn:lmdSVm}
  \text{$\lambda \not \in {\mathbb{Z}}$ or $\lambda \in S(V) \cup S_Y(V) \cup \{m\}$}.  
\end{equation}
Conversely,
 under the condition \eqref{eqn:lmdSVm}, 
 Proposition \ref{prop:180572} tells
 that $I_{\delta}(V,\lambda)$ is irreducible
 if and only if $\lambda \not \in {\mathbb{Z}}$
or $\lambda \in S(V) \cup S_Y(V)$.  
Thus Theorem \ref{thm:irrIV} is proved
 also for $V$ of type Y.  
\end{proof}



\subsection{Socle filtration of $I_{\delta}(V,\lambda)$:
 Proof of Proposition \ref{prop:Xred}}
\label{subsec:Isub2}

In this section,
 we complete the proof of Proposition \ref{prop:Xred}
 about the socle filtration
 of the principal series representation $I_{\delta}(V,\lambda)$
 of $G=O(n+1,1)$
 when it is reducible and $\lambda \ne \frac n 2$
 by using the restriction to the subgroups 
 $\overline G=SO(n+1,1)$ or $G_0=SO(n+1,1)$.  


We begin with the case 
 that $V \in \widehat{O(n)}$ is of type X
 (Definition \ref{def:OSO}).  

\begin{proof}[Proof of Proposition \ref{prop:Xred} when $V$ is of type X]
In this case,
 for any nonzero subquotient $\Pi$ 
 of the principal series representation $I_{\delta}(V,\lambda)$ of $G=O(n+1,1)$, 
 we have 
\[\Pi \not \simeq \Pi \otimes \det\]
as $G$-modules
 because their $K$-types are different
 by Proposition \ref{prop:KtypeIV}.  
In turn, 
 Lemma \ref{lem:171523} implies
 that $\Pi$ is irreducible as a $G$-module
 if and only if the restriction $\Pi|_{\overline G}$
 is irreducible.  



For $n$ even, 
 the restriction $\Pi|_{G_0}$
 further to the identity component $G_0=SO_0(n+1,1)$ 
 is still irreducible
 because $\overline G=SO(n+1,1)$ is generated
 by $G_0$ 
 and a central element $-I_{n+2}$.  
Thus the assertion follows from the socle filtration
 of the principal series representation of $G_0$
 in Hirai \cite{Hirai62}.  


For $n$ odd, 
 since the restriction $V|_{SO(n)}$ stays irreducible, 
 $I_{\delta}(V,\lambda)|_{G_0}$ is a principal series representation 
 of $G_0=SO_0(n+1,1)$.  
Therefore the restriction $\Pi|_{G_0}$ is a $G_0$-subquotient
 of a principal series representation
 of $G_0$, 
 of which the length of composition series is either 2 or 3
 by Hirai \cite{Hirai62}.  
Inspecting the $K$-structure of $I_{\delta}(V,\lambda)$ from 
 Proposition \ref{prop:KtypeIV} again 
 and the $K_0$-structure of subquotients
 of the principal series representation 
 of 
\index{A}{GOindefinitespecialidentity@$G_0=SO_0(n+1,1)$, the identity component
 of $O(n+1,1)$\quad}
$G_0=SO_0(n+1,1)$ in \cite{Hirai62}, 
 we see that the restriction $\Pi|_{G_0}$ is irreducible
 as a $G_0$-module
 if $\Pi$ is not (the smooth representation of)
 a discrete series representation, 
 whereas it is a sum 
 of two (holomorphic and anti-holomorphic) discrete series representations 
 of $G_0$
 if $\Pi$ is a discrete series representation.  
\end{proof}



Alternatively,
 one may reduce the proof for type X
 to the case $(V,\lambda)=(\Exterior^i({\mathbb{C}}^n), i)$
 by using the translation functor, 
 see Theorems \ref{thm:1807113} and \ref{thm:1808101} (1)
 in Appendix III.  



As the above proof shows, 
 we obtain the restriction formula
 of irreducible subquotients
\index{A}{IdeltaVf@$I_{\delta}(V, \lambda)^{\flat}$}
 $I_{\delta}(V,\lambda)^{\flat}$ and 
\index{A}{IdeltaVs@$I_{\delta}(V, \lambda)^{\sharp}$}
$I_{\delta}(V,\lambda)^{\sharp}$
 of the $G$-module
 $I_{\delta}(V,\lambda)$
 (Proposition \ref{prop:Xred})
 to the normal subgroup $\overline G=SO(n+1,1)$ as follows.  
\begin{proposition}
\label{prop:20180906}
Suppose $V \in \widehat{O(n)}$ is of type X
 and $\lambda \in {\mathbb{Z}} \setminus S(V)$.  
Let $\delta \in \{\pm\}$.  
\begin{enumerate}
\item[{\rm{(1)}}]
The principal series representation 
 $\overline I_{\delta}(V,\lambda)$ of $\overline G$
 has a unique proper submodule,
 to be denoted by
\index{A}{IdeltaVsbarpmflat@$\overline I_{\delta}(V^{(\pm)}, \lambda)^{\flat}$|textbf}
 $\overline I_{\delta}(V,\lambda)^{\flat}$.  
In particular,
 the quotient $\overline G$-module
\index{A}{IdeltaVsbarpms@$\overline I_{\delta}(V^{(\pm)}, \lambda)^{\sharp}$|textbf}
$
   \overline I_{\delta}(V,\lambda)^{\sharp}
 :=\overline I_{\delta}(V,\lambda)/\overline I_{\delta}(V,\lambda)^{\flat}
$
 is irreducible.  
\item[{\rm{(2)}}]
The restriction of the irreducible $G$-modules
 $I_{\delta}(V,\lambda)^{\flat}$ and $I_{\delta}(V,\lambda)^{\sharp}$ 
 to the normal subgroup $\overline G$ is given by
\begin{align*}
 I_{\delta}(V,\lambda)^{\flat}|_{\overline G}
 & \simeq \overline I_{\delta}(V,\lambda)^{\flat}, 
\\
 I_{\delta}(V,\lambda)^{\sharp}|_{\overline G}
 & \simeq \overline I_{\delta}(V,\lambda)^{\sharp}.   
\end{align*}
\end{enumerate}
\end{proposition}
We end this section 
 with the restriction of $I_{\delta}(V,\lambda)^{\flat}$
 and $I_{\delta}(V,\lambda)^{\sharp}$
 to the subgroup $\overline G$
 when $V$ is of type Y:

\begin{proposition}
\label{prop:1808104}
Suppose $G=O(n+1,1)$ with $n=2m$.  
Assume that $V \in \widehat {O(n)}$
 is of type Y, 
 $\delta \in \{\pm\}$, 
 and $\lambda \in {\mathbb{Z}} \setminus (S(V) \cup S_Y(V) \cup \{m\})$.  
Then the restriction of $I_{\delta}(V,\lambda)^{\flat}$
 and $I_{\delta}(V,\lambda)^{\sharp}$
 to the normal subgroup $\overline G =S O(n+1,1)$
 decomposes into the direct sum 
 of two irreducible $\overline G$-modules: 
\begin{align*}
  I_{\delta}(V,\lambda)^{\flat}|_{\overline G}
  &\simeq
  \overline I_{\delta}(V^{(+)},\lambda)^{\flat}
  \oplus
  \overline I_{\delta}(V^{(-)},\lambda)^{\flat}, 
\\
  I_{\delta}(V,\lambda)^{\sharp}|_{\overline G}
  &\simeq
  \overline I_{\delta}(V^{(+)},\lambda)^{\sharp}
  \oplus
  \overline I_{\delta}(V^{(-)},\lambda)^{\sharp}, 
\end{align*}
where we recall from Proposition \ref{prop:irrIVpm}
 for the definition of the irreducible $\overline G$-modules
 $\overline I_{\delta}(V^{(\pm)},\lambda)^{\flat}$
 and $\overline I_{\delta}(V^{(\pm)},\lambda)^{\sharp}$.  
\end{proposition}

\begin{proof}
[Proof of Proposition \ref{prop:Xred} for $V$ of type Y]
By Lemma \ref{lem:171523} and by the structural results
 on $\overline G$-modules $\overline I_{\delta}(V^{(\pm)},\lambda)$
 in Proposition \ref{prop:irrIVpm}, 
 the proof is reduced to the following lemma. \end{proof}
 
\begin{lemma}
\label{lem:1808103}
Under the assumption of Proposition \ref{prop:1808104}, 
 any $G$-submodule $\Pi$ of $I_{\delta}(V,\lambda)$ satisfies
\begin{equation}
\label{eqn:Pidet}
  \Pi \simeq \Pi \otimes \det
\end{equation}
as $G$-modules.  
\end{lemma}

\begin{proof}
Since $V$ is of type Y, 
 $V \simeq V \otimes \det$
 as $O(n)$-modules,
 hence we have natural $G$-isomorphisms
\[
   I_{\delta}(V,\lambda) \simeq
   I_{\delta}(V,\lambda) \otimes \det
\]
by Lemma \ref{lem:IVchi}.  
We prove \eqref{eqn:Pidet}
 by {\it{reductio ad absurdum}}.  
Suppose that the $G$-module $\Pi$ were not isomorphic
 to $\Pi \otimes \det$.  
Then the direct sum representation $\Pi \oplus (\Pi \otimes \det)$
 would be a $G$-submodule
 of $I_{\delta}(V,\lambda)$.  
In turn, 
 the $\overline G$-module $\Pi|_{\overline G}$ would occur in 
$
I_{\delta}(V,\lambda)|_{\overline G}
\simeq \overline I_{\delta}(V^{(+)},\lambda)
\oplus \overline I_{\delta}(V^{(-)},\lambda)
$
at least twice. 
But this is impossible by Proposition \ref{prop:irrIVpm}.  
Thus Lemma \ref{lem:1808103} is proved.  
\end{proof}







\subsection{Restriction of $\Pi_{\ell,\delta}$
 to $SO(n+1,1)$}
\label{subsec:PiSO}
\index{A}{1Piidelta@$\Pi_{i,\delta}$, irreducible representations of $G$}
In this section 
 we treat the case 
 where $I_{\delta}(V, \lambda)$ is not irreducible
 as a $G$-module.  
We discuss the restriction of $G$-irreducible subquotients
 of $I_{\delta}(V, \lambda)$
 to the subgroup $\overline G=SO(n+1,1)$.  



We focus on the case 
 when $(\sigma,V)$ is the exterior representation 
 on $V=\Exterior^i({\mathbb{C}}^{n})$.  
In particular,
 irreducible representations 
 that have
 the ${\mathfrak{Z}}_G({\mathfrak{g}})$-infinitesimal character $\rho$,  
 namely,
 the irreducible $G$-modules $\Pi_{\ell,\delta}$
 ($0 \le \ell \le n+1$, $\delta \in \{\pm\}$) arise as 
 $G$-irreducible subquotients
 of $I_{\delta}(V, \lambda)$.  
To be more precise,
 we recall from \eqref{eqn:Pild}
 that $\Pi_{\ell,\delta}$
 are the irreducible subrepresentations
 of $I_{\delta}(\ell,\ell)$
 for $0 \le \ell \le n$
 and coincidently those of $I_{-\delta}(\ell-1,\ell-1)$
 for $1 \le \ell \le n+1$.  
\begin{lemma}
\label{lem:Pilso}
For all $0 \le \ell \le n+1$
 and $\delta=\pm$, 
 the restriction of $\Pi_{\ell, \delta}$
 to the subgroup $\overline G = SO(n+1,1)$ stays irreducible.  
\end{lemma}
\begin{proof}
The restriction $\Pi_{\ell, \delta}|_{\overline G}$ is 
irreducible by the criterion in Lemma \ref{lem:171523} (1)
 because $\Pi_{\ell,\delta} \otimes \det \simeq \Pi_{n+1-\ell,-\delta}
\not \simeq \Pi_{\ell,\delta}$
 by Theorem \ref{thm:LNM20} (5).  
\end{proof}



We denote by $\overline{\Pi}_{\ell,\delta}$
 the restriction of the irreducible $G$-module ${\Pi}_{\ell,\delta}$
 ($0 \le \ell \le n+1$, $\delta=\pm$)
 to the subgroup ${\overline G} = SO(n+1,1)$.  
By a little abuse of notation,
 we write $\overline I _{\delta}(i, \lambda)$
 for the restriction of $I _{\delta}(i, \lambda)$,  
 to the subgroup $\overline G$.  
Then the $SO(n)$-isomorphism
 $\Exterior^i({\mathbb{C}}^n) \simeq \Exterior^{n-i}({\mathbb{C}}^n)$
 induces a $\overline G$-isomorphism
\[
   \overline I _{\delta}(i, \lambda)
   \simeq
   \overline I _{\delta}(n-i, \lambda).  
\]


Special attention is needed in the case
 when $n$ is even
 and $n=2i$.  
In this case,
 the $O(n)$-module $\Exterior^{i}({\mathbb{C}}^n)$ 
 is of type Y
 (see Example \ref{ex:2.1}), 
 and it splits into the direct sum
 of two irreducible $SO(n)$-modules:
\[
   \Exterior^{\frac n 2}({\mathbb{C}}^n) 
   \simeq 
   \Exterior^{\frac n 2}({\mathbb{C}}^n)^{(+)}
   \oplus
   \Exterior^{\frac n 2}({\mathbb{C}}^n)^{(-)}.  
\]
We set 
\[
   \overline I_{\delta}^{(\pm)}(\frac n 2, \lambda)
   :=
   {\operatorname{Ind}}_{\overline P}^{\overline G}
   (\Exterior^{\frac n 2}({\mathbb{C}}^n)^{(\pm)}
    \otimes \delta 
    \otimes {\mathbb{C}}_{\lambda}).  
\]
As in \eqref{eqn:IVSOpm}, 
 the restriction $I_{\delta}(\frac n 2, \lambda)|_{\overline G}$
 is the direct sum of two $\overline G$-modules:
\begin{equation}
\label{eqn:ISOneven}
   \overline I_{\delta}(\frac n 2, \lambda)
   \simeq
   \overline I_{\delta}^{(+)}(\frac n 2, \lambda)
   \oplus
   \overline{I}_{\delta}^{(-)}({\frac n 2}, \lambda)
\quad
\text{for all $\lambda \in {\mathbb{C}}$.}
\end{equation}



If $I_{\delta}(\frac n 2, \lambda)$ is $G$-irreducible, 
 then Lemma \ref{lem:170305} (2) tells
 that the representations $\overline {I}_{\delta}^{(\pm)}(\frac n 2, \lambda)$
 of the subgroup $\overline G$
 are irreducible,
 and that they are not isomorphic to each other.  



On the other hand, 
 if $\lambda=i$ $(=\frac n 2)$, 
 then the principal series representation $I_{\delta}(i,\lambda)$ is not irreducible as a $G$-module 
 but splits into the direct sum
 of two irreducible $G$-modules
 (see Theorem \ref{thm:LNM20} (1)):
\[
   I_{\delta}(\frac n 2,\frac n 2) \simeq \Pi_{\frac n 2+1,-\delta} \oplus \Pi_{\frac n 2,\delta}, 
\]
which are not isomorphic to each other.  
Moreover,
 the tensor product with $\chi_{--}$
 switches $\Pi_{\frac n 2+1,-\delta}$ and $\Pi_{\frac n 2,\delta}$
 (Theorem \ref{thm:LNM20} (5)).  
Hence we have a 
$
   \overline G$-isomorphism $\overline \Pi_{\frac n 2+1,-\delta}
   \simeq \overline\Pi_{\frac n 2,\delta}
$,
 which are $\overline G$-irreducible by Lemma \ref{lem:171523} (1).  
Therefore, 
 for $n$ even, 
 we have the following isomorphisms as $\overline G$-modules:
\begin{equation}
\label{eqn:161733}
\overline{\Pi}_{\frac n 2, \delta}
\simeq
\overline{\Pi}_{\frac n 2+1, -\delta}
\simeq
\overline{I}_{\delta}^{(+)}({\frac n 2}, \frac n2)
\simeq 
\overline{I}_{\delta}^{(-)}({\frac n 2}, \frac n2)
\quad
\text{for $\delta = \pm$}.  
\end{equation}
Similarly to Theorem \ref{thm:LNM20} 
 about the $O(n+1,1)$-modules $\Pi_{\ell,\delta}$, 
 we summarize the properties of the restriction $\overline{\Pi}_{\ell,\delta}=\Pi_{\ell,\delta}|_{\overline G}$ as follows.  
\begin{proposition}
\label{prop:161648}
Let ${\overline G}=SO(n+1,1)$
 with $n \ge 1$.  
\begin{enumerate}
\item[{\rm{(1)}}]
$\overline{\Pi}_{\ell,\delta}$ is irreducible as a $\overline G$-module
 for all $0 \le \ell \le n+1$ and $\delta = \pm $.  
\item[{\rm{(2)}}]
$\overline{\Pi}_{\ell,\delta} \simeq \overline{\Pi}_{n+1-\ell,-\delta}$
 as $\overline G$-modules
 for all $0 \le \ell \le n+1$ and $\delta = \pm $.  
\item[{\rm{(3)}}]
Irreducible representations of $\overline G$
 with ${\mathfrak {Z}}({\mathfrak {g}})$-infinitesimal character $\rho_{\overline G}$
 can be classified as 
\begin{alignat*}{2}
&
 \{ 
 \overline{\Pi}_{\ell,\delta}: 0 \le \ell \le \frac{n-1}{2}, \delta = \pm 
 \}
 \cup
 \{ 
 \overline{\Pi}_{\frac{n+1}{2},+}
 \}
 \qquad
&&
 \text{if $n$ is odd}, 
\\
&
 \{ 
 \overline{\Pi}_{\ell,\delta}: 0 \le \ell \le \frac{n}{2}, \delta = \pm 
 \}
 \qquad
&&
 \text{if $n$ is even}.  
\end{alignat*}
\item[{\rm{(4)}}]
Every $\overline{\Pi}_{\ell,\delta}$
 $(0 \le \ell \le n+1, \delta=\pm)$
 is unitarizable.  
\end{enumerate}
In the next statement,
 we use the same symbol $\overline{\Pi}_{\ell,\delta}$
 to denote the irreducible unitary representation 
obtained by the Hilbert completion 
of $\overline \Pi_{\ell,\delta}$ with respect to a $\overline G$-invariant inner product.  
\begin{enumerate}
\item[{\rm{(5)}}]
For $n$ odd, 
 $\overline{\Pi}_{\frac{n+1}{2},+}$ is a discrete series representation
 of $\overline G=SO(n+1,1)$.  
For $n$ even, 
 $\overline{\Pi}_{\frac{n}{2},\delta}$ $(\delta=\pm)$
 are 
\index{B}{temperedrep@tempered representation}
tempered representations.  
All the other representations in the list (2)
 are nontempered representations of $\overline G$.  
\item[{\rm{(6)}}]
For $n$ even, 
 the center of $\overline G=SO(n+1,1)$ acts nontrivially
 on $\overline{\Pi}_{\ell,\delta}$
 if and only if $\delta=(-1)^{\ell+1}$.  
For $n$ odd, 
 the center of $\overline G$ is trivial, 
 and thus acts trivially
 on $\overline{\Pi}_{\ell,\delta}$
 for any $\ell$ and $\delta$.  
\end{enumerate}
\end{proposition}

In Proposition \ref{prop:161655}, 
 we gave a description of the underlying $({\mathfrak{g}},{K})$-module
 of the irreducible $G$-module $\Pi_{\ell,\delta}$ 
 in terms of cohomological parabolic induction.  
We end this section 
 with analogous results
for the irreducible $\overline G$-module 
 $\overline \Pi_{\ell,\delta}= \Pi_{\ell,\delta}|_{\overline G}$
 (see Proposition \ref{prop:161648} (1)).  
\begin{proposition}
\label{prop:SO161655}
For $0 \le i \le [\frac {n+1}2]$, 
 let 
\index{A}{qi@${\mathfrak{q}}_i$, $\theta$-stable parabolic subalgebra}
${\mathfrak{q}}_i$ be the $\theta$-stable parabolic subalgebras
 with the Levi subgroup $\overline{L_i} \simeq SO(2)^i \times SO(n-2i+1,1)$
 as in Definition \ref{def:qi}
 and write $S_i=i(n-i)$, 
 see \eqref{eqn:cohSi}.  
Then the underlying $({\mathfrak {g}},\overline{K})$-modules
 of the irreducible $\overline G$-modules $\overline{\Pi}_{\ell, \delta}$
 $(0 \le \ell \le n+1$, $\delta \in \{\pm\})$
 are given by the cohomological parabolic induction
 as follows:
\index{A}{Aqlmd@$A_{\mathfrak{q}}(\lambda)$}
\begin{alignat*}{4}
  (\overline{\Pi}_{i,+})_{\overline K} 
&\simeq
 (\overline{\Pi}_{n+1-i,-})_{\overline K}
 \simeq
&& {\mathcal{R}}_{\mathfrak {q}_i}^{S_i}({\mathbb{C}}_{\rho({\mathfrak{u}})})
&& \simeq (A_{\mathfrak{q}_i})_{+,+}|_{({\mathfrak {g}},\overline{K})} 
&&  \simeq (A_{\mathfrak{q}_i})_{-,-}|_{({\mathfrak {g}},\overline{K})}, 
\\
   (\overline{\Pi}_{i,-})_{\overline K} 
&\simeq 
 (\overline {\Pi}_{n+1-i,+})_{\overline K}
   \simeq 
   && {\mathcal{R}}_{\mathfrak {q}_i}^{S_i}({\mathbb{C}}_{\rho({\mathfrak{u}})} \otimes \chi_{+-})
&& \simeq (A_{\mathfrak{q}_i})_{+,-}|_{({\mathfrak {g}},\overline{K})} 
&&\simeq (A_{\mathfrak{q}_i})_{-,+}|_{({\mathfrak {g}},\overline{K})}.  
\end{alignat*}
\end{proposition}
We notice that the four characters $\chi_{\pm \pm}$
 of $O(n+1,1)$ induce the following isomorphisms $\chi_{--} \simeq {\bf{1}}$
 and $\chi_{+-} \simeq \chi_{-+}$
 when restricted to the last factor
 $SO(n-2i+1,1)$ of the Levi subgroup $\overline{L}_i$, 
 whence Proposition \ref{prop:SO161655} gives an alternative proof
 for the isomorphism
\[
  \overline{\Pi}_{i,\delta} \simeq \overline{\Pi}_{n+1-i,-\delta}
\]
 as $\overline G$-modules
 for $0 \le i \le n+1$ and $\delta = \pm$.  



%%%%%%%%%%%%%%%%%%%%%%%%%%%%%%%%%%%%%%%%%%%%%%%%%%%%%%%%%%%
\subsection{Symmetry breaking for tempered principal series representations}
\label{subsec:SOtemp}
%%%%%%%%%%%%%%%%%%%%%%%%%%%%%%%%%%%%%%%%%%%%%%%%%%%%%%%%%%%

In this section, 
 we deduce a multiplicity-one theorem 
 for tempered principal series representations 
$
   \overline I_{\delta}(\overline V, \lambda)
$
 and 
$
   \overline J_{\varepsilon}(\overline W, \nu)
$
 of $\overline G =SO(n+1,1)$ and $\overline {G'} =SO(n,1)$, 
 respectively, from the corresponding result
 (see Theorem \ref{thm:tempVW})
 for the pair $(G, G') = (O(n+1,1), O(n,1))$.  



In \cite[Chap.~2, Sect.~5]{KKP}, 
 a trick analogous to Lemma \ref{lem:LNM26} was used 
 to deduce symmetry breaking 
 for the pair $(G_0, G_0')=(SO_0(n+1,1), SO_0(n,1))$ from that 
 for the pair $(G, G')$
 by using an observation
 that $G_0$ and $G_0'$ are normal subgroups
 of $G$ and $G'$, 
 respectively
 (cf.~\cite[page 26]{KKP}).  
This is formulated in our setting as follows:
\begin{proposition}
\label{prop:LNM26}
Let $\Pi$ and $\pi$ be continuous representations
 of $G=O(n+1,1)$ and $G'=O(n,1)$, 
respectively.  
Let $(\overline G, \overline {G'})=(SO(n+1,1), SO(n,1))$.  
Then we have natural isomorphisms:
\begin{align*}
  \operatorname{Hom}_{\overline{G'}}(\Pi|_{\overline{G'}}, \pi|_{\overline{G'}})&\simeq
  \operatorname{Hom}_{G'}(\Pi|_{G'}, \pi)
  \oplus
  \operatorname{Hom}_{G'}(\Pi|_{G'}, \chi_{--}\otimes \pi)
\\
&\simeq
  \operatorname{Hom}_{G'}(\Pi|_{G'}, \pi)
  \oplus
  \operatorname{Hom}_{G'}((\Pi \otimes \chi_{--})|_{G'}, \pi).  
\end{align*}
\end{proposition}

For $\overline V \in \widehat{SO(n)}$
 and $W \in \widehat{SO(n-1)}$, 
 we set
\[
[\overline V|_{SO(n-1)}:\overline W]
:=\dim_{\mathbb{C}} {\operatorname{Hom}}_{SO(n-1)}(\overline V|_{SO(n-1)},W).  
\]
The main result of this section
 is the following.  
\begin{theorem}
[tempered principal series representation]
\label{thm:tempSO}
Let $\overline V \in \widehat {SO(n)}$, 
 $\overline W \in \widehat {SO(n-1)}$, 
 $\delta, \varepsilon \in \{\pm\}$, 
 and $(\lambda,\nu) \in (\sqrt{-1}{\mathbb{R}}+ \frac n 2, 
 \sqrt{-1}{\mathbb{R}}+\frac 1 2(n-1))$
 so that $\overline {I}_{\delta}(\overline V, \lambda)$
 and $\overline {J}_{\varepsilon}(\overline W, \nu)$ are 
 irreducible tempered principal series representations
 of $\overline G=SO(n+1,1)$ and $\overline {G'}=SO(n,1)$, 
 respectively.  
Then the following conditions are equivalent:
\begin{enumerate}
\item[{\rm{(i)}}]
$[\overline V|_{SO(n-1)}: \overline W] 
\ne 0$.  

\item[{\rm{(ii)}}]
$\operatorname{Hom}_{SO(n,1)}
 (\overline {I}_{\delta}(\overline V, \lambda)|_{SO(n,1)}, 
  \overline {J}_{\varepsilon}(\overline W, \nu)) 
\ne \{0\}.$

\item[{\rm{(iii)}}]
$\dim_{\mathbb{C}}\operatorname{Hom}_{SO(n,1)}
 (\overline {I}_{\delta}(\overline V, \lambda)|_{SO(n,1)}, 
  \overline {J}_{\varepsilon}(\overline W, \nu)) =1.$  
\end{enumerate}
\end{theorem}
For the proof, 
we use the following elementary lemma
 on branching rules of finite-dimensional representations of $O(n)$.  
\begin{lemma}
\label{lem:Xbranch}
Suppose $\sigma \in \widehat{O(n)}$ and $\tau \in \widehat{O(n-1)}$
 are of both type $X$
 (Definition \ref{def:OSO}).  
If $[\sigma|_{O(n-1)}:\tau]\ne0$, 
 then $[\sigma|_{O(n-1)}:\tau \otimes \det]=0$.  
\end{lemma}

\begin{proof}
[Proof of Lemma \ref{lem:Xbranch}]
Easy from Fact \ref{fact:ONbranch} and from the characterization in Lemma \ref{lem:OSO}
 of representations of type X
 by means of the Cartan--Weyl bijection \eqref{eqn:CWOn}.  
\end{proof}
\begin{proof}
[Proof of Theorem \ref{thm:tempSO}]
There exist unique $V \in \widehat{O(n)}$ and $W \in \widehat{O(n-1)}$
such that $[V|_{SO(n)}:\overline V]\ne0$
 and $[W|_{SO(n-1)}:\overline W]\ne0$.  
We divide the argument into the following three cases:

Case XX: Both $V$ and $W$ are of type X. 

Case XY: $V$ is of type X and $W$ is of type Y.  

Case YX: $V$ is of type Y and $W$ is of type X.  


Then we have from \eqref{eqn:IVSOpm}
\[
{I}_{\delta}(V, \lambda)|_{\overline G}
\simeq
\begin{cases} 
  \overline {I}_{\delta}(\overline V, \lambda)
 &\text{if $V$ is of type X,}
\\
\overline{I}_{\delta}(\overline V, \lambda)
\oplus
\overline {I}_{\delta}(\overline V^{\vee}, \lambda)
\qquad
 &\text{if $V$ is of type Y,}
\end{cases}
\]
and similarly for the restriction $J_{\varepsilon}(W, \nu)|_{\overline {G'}}$.  


By Proposition \ref{prop:LNM26}, 
 we have
\begin{equation*}
  {\operatorname{Hom}}_{\overline{G'}} 
  ({I}_{\delta}(V, \lambda)|_{\overline {G'}}, 
   {J}_{\varepsilon}(W, \nu)|_{\overline {G'}})
   \simeq
   \bigoplus_{\chi \in \{{\bf{1}},\det\}}
   {\operatorname{Hom}}_{G'}({{I}_{\delta}(V, \lambda)|_{G'}},
   J_{\varepsilon}(W, \nu) \otimes \chi).  
\end{equation*}
Applying the 
\index{B}{multiplicityonetheorem@multiplicity-one theorem}
multiplicity-one theorem (Theorem \ref{thm:tempVW})
 for tempered representations 
 of the pair $(G,G')=(O(n+1,1),O(n,1))$
 to the right-hand side, 
 we get the following multiplicity formula:
\begin{multline}
\label{eqn:VWdetdim}
     \dim_{\mathbb{C}} {\operatorname{Hom}}_{\overline{G'}} 
  ({I}_{\delta}(V, \lambda)|_{\overline {G'}}, 
   {J}_{\varepsilon}(W, \nu)|_{\overline {G'}})
\\
=[V|_{O(n-1)}:W]+[V|_{O(n-1)}:W \otimes \det].  
\end{multline}
The right-hand side of \eqref{eqn:VWdetdim} does not vanish
 if and only if 
 $[\overline V|_{SO(n-1)}:\overline W] \ne 0$.  
In this case,
 we have 
\begin{equation*}
\text{\eqref{eqn:VWdetdim}}
=\begin{cases}
  1 \qquad &\text{Case XX, }
\\
  2 \qquad &\text{Case XY or Case YX, }
  \end{cases}
\end{equation*}
by Lemmas \ref{lem:branchII} and \ref{lem:Xbranch}.  
Thus the conclusion holds in Case XX.   



If $V$ is of type Y, 
 then the two $\overline G$-irreducible summands
 $\overline{I}_{\delta}(\overline V, \lambda)$
 and $\overline{I}_{\delta}(\overline V^{\vee}, \lambda)$
 in the restriction $I_{\delta}(V, \lambda)|_{G'}$ are switched 
 if we apply the outer automorphism of $\overline G$
 by an element $g_0:={\operatorname{diag}}(1,\cdots,1,-1,1) \in G=O(n+1,1)$.  
Since $g_0$ commutes with $\overline {G'}$, 
 we obtain an isomorphism
\[
  {\operatorname{Hom}}_{\overline{G'}} 
  (\overline{I}_{\delta}(\overline V, \lambda)|_{\overline {G'}}, 
   \overline{J}_{\varepsilon}(\overline W, \nu))
  \simeq
  {\operatorname{Hom}}_{\overline{G'}} 
  (\overline{I}_{\delta}(\overline V^{\vee}, \lambda)|_{\overline {G'}}, 
   \overline{J}_{\varepsilon}(\overline W, \nu)).
\]
Hence the conclusion holds
 for Case YX. 



Similar argument holds for Case XY
 where $W$ is of type Y.  
Therefore Theorem \ref{thm:tempSO} is proved.  
\end{proof}



\subsection{Symmetry breaking from
 $\overline I_{\delta}(i,\lambda)$ to $\overline J_{\varepsilon}(j,\nu)$}
\label{subsec:SBOpsSO}



In this section, 
 we give a closed formula
 of the multiplicity
 for the restriction $\overline G \downarrow \overline G'$
 when $(\sigma,V)$ is the exterior tensor $\Exterior^i({\mathbb{C}}^n)$.  
For the admissible smooth representations
 ${\overline{I}}_{\delta}(i,\lambda)$ of ${\overline{G}}=SO(n+1,1)$
 and ${\overline{J}}_{\varepsilon}(j,\nu)$ of ${\overline{G'}}=SO(n,1)$, 
 we set
\[
   m(i,j)
   \equiv m({\overline{I}}_{\delta}(i,\lambda), {\overline{J}}_{\varepsilon}(j,\nu))
   :=
   \dim_{\mathbb{C}} \operatorname{Hom}_{{\overline{G'}}}
    ({\overline{I}}_{\delta}(i,\lambda)|_{{\overline{G'}}}, {\overline{J}}_{\varepsilon}(j,\nu)).  
\]
%
%%%%%%%%%%%%%%%%%%%%%%%%%%%%%%%%%
%\subsection{Classification of symmetry breaking operators}
%\label{subsec:exhaust}
%%%%%%%%%%%%%%%%%%%%%%%%%%%%%%%%



In order to state a closed formula
 for the multiplicity $m(i,j)$
 as a function of $(\lambda, \nu,\delta, \varepsilon)$, 
 we introduce the following subsets of ${\mathbb{Z}}^2 \times \{\pm 1\}$:
\index{A}{Leven@$L_{\operatorname{even}}$}
\index{A}{Lodd@$L_{\operatorname{odd}}$}
\begin{align*}
  L:=&\{(-i,-j, (-1)^{i+j}): (i,j) \in {\mathbb{Z}}^2, 0 \le j \le i \}
    = L_{\operatorname{even}} \cup L_{\operatorname{odd}}, 
\\
  L':=&\{(\lambda,\nu,\gamma) \in L: \nu \ne 0 \}.  
\end{align*}

In the theorem below, we shall see
\begin{alignat*}{2}
m(i,j) \in &\{ 1,2,4 \} \qquad
&&\text{if $j=i-1$ or $i$}, 
\\
m(i,j) \in &\{ 0,1,2 \} \qquad
&&\text{if $j=i-2$ or $i+1$}, 
\\
m(i,j) =& 0 \qquad
&&\text{otherwise}.  
\end{alignat*}
By Proposition \ref{prop:LNM26} and Lemma \ref{lem:LNM27}, 
 the multiplicity formula
 for $(\overline G, \overline {G'})$ is derived from 
 the one for $(G,G')$ 
 by using Proposition \ref{prop:LNM26}, 
 which amounts to  
\begin{multline*}
\operatorname{Hom}_{\overline{G'}}(\overline{I}_{\delta}(i,\lambda)|_{\overline{G'}}, \overline{J}_{\varepsilon}(j,\nu))
\\
\simeq
 \operatorname{Hom}_{G'}({I}_{\delta}(i,\lambda)|_{G'}, {J}_{\varepsilon}(j,\nu))
  \oplus
  \operatorname{Hom}_{G'}({I}_{\delta}(n-i,\lambda)|_{G'}, {J}_{\varepsilon}(j,\nu)).  
\end{multline*}
The right-hand side was computed 
 in Theorem \ref{thm:1.1}.  
Hence we get an explicit formula of the multiplicity
 for the restriction of nonunitary principal series representations
in this setting:

\begin{theorem}
[multiplicity formula]
\label{thm:SOmult}
Suppose $n \ge 3$, $0 \le i\le [\frac n2]$, $0 \le j \le [\frac{n-1}2]$,
 $\delta$, $\varepsilon \in \{\pm \}\equiv \{ \pm 1\}$, 
 and $\lambda,\nu \in {\mathbb{C}}$.

Then the multiplicity 
$
   m(i,j)=
   \dim_{\mathbb{C}} \operatorname{Hom}_{{\overline{G'}}}
    ({\overline{I}}_{\delta}(i,\lambda)|_{{\overline{G'}}}, 
     {\overline{J}}_{\varepsilon}(j,\nu))
$
 is given as follows.  
\begin{enumerate}
\item[{\rm{(1)}}]
Suppose $j=i$.  
\begin{enumerate}
\item[{\rm{(a)}}]
Case $i=0$.
\begin{equation*}
m(0,0)
=
\begin{cases}
2
\qquad
&\text{if }
(\lambda, \nu, \delta\varepsilon) \in L, 
\\
1
&
\text{otherwise.}
\end{cases}
\end{equation*}
\item[{\rm{(b)}}]
Case $1 \le i < \frac  n2-1$.  
\begin{equation*}
m(i, i)
=
\begin{cases}
2
\qquad
&\text{if }
(\lambda, \nu, \delta \varepsilon) \in L' \cup \{(i,i,+)\},  
\\
1 
&
\text{otherwise}.  
\end{cases}
\end{equation*}
\item[{\rm{(c)}}]
Case $i= \frac n 2 -1$ $($$n$: even$)$.   
\begin{equation*}
m(\frac n 2 -1, \frac n 2 -1)
=
\begin{cases}
2
\qquad
&\text{if }
(\lambda, \nu, \delta\varepsilon) \in L' \cup \{(i,i,+)\}\cup \{(i,i+1,-)\},   
\\
1
&
\text{otherwise}.  
\end{cases}
\end{equation*}
\item[{\rm{(d)}}]
Case $i= \frac {n-1} 2$ $($$n$: odd$)$.   
\begin{equation*}
m(\frac {n-1} 2,\frac {n-1} 2)
=
\begin{cases}
4
\qquad
&\text{if }
(\lambda, \nu, \delta\varepsilon) \in L' \cup \{(i,i,+)\}, 
\\
2
&
\text{otherwise}.  
\end{cases}
\end{equation*}
\end{enumerate}

\item[{\rm{(2)}}]
Suppose $j=i-1$.  
\begin{enumerate}
\item[{\rm{(a)}}]
Case $1 \le i < \frac{n-1}{2}$.
\begin{equation*}
m(i,i-1)
=
\begin{cases}
2
\qquad
&\text{if }
(\lambda, \nu, \delta\varepsilon) \in L' \cup \{(n-i,n-i,+)\}, 
\\
1
&
\text{otherwise.}
\end{cases}
\end{equation*}
\item[{\rm{(b)}}]
Case $i = \frac{n-1}{2}$ $($$n$: odd$)$.  
\begin{equation*}
m(\frac{n-1}{2},\frac{n-3}{2})
=
\begin{cases}
2
\qquad
&\text{if }
(\lambda, \nu, \delta \varepsilon) \in L',  
\\
2
\qquad
&\text{if }
(\lambda, \nu, \delta \varepsilon) \in \{(n-i,n-i,+)\}
                                       \cup \{(i,i+1,-)\} ,  
\\
1 
&
\text{otherwise}.  
\end{cases}
\end{equation*}
\item[{\rm{(c)}}]
Case $i= \frac n 2$ $($$n$: even$)$.   
\begin{equation*}
m(\frac n 2,\frac n 2-1)
=
\begin{cases}
4
\qquad
&\text{if }
(\lambda, \nu, \delta\varepsilon) \in L' \cup \{(n-i,n-i,+)\},   
\\
2
&
\text{otherwise}.  
\end{cases}
\end{equation*}
\end{enumerate}

\item[{\rm{(3)}}]
Suppose $j=i-2$.  
\begin{enumerate}
\item[{\rm{(a)}}]
Case $2 \le i < \frac{n}{2}$.
\begin{equation*}
m(i,i-2)
=
\begin{cases}
1
\qquad
&\text{if }
(\lambda, \nu, \delta\varepsilon) = (n-i,n-i+1,-), 
\\
0
&
\text{otherwise.}
\end{cases}
\end{equation*}
\item[{\rm{(b)}}]
Case $i = \frac{n}{2}$ $($$n$: even$)$.  
\begin{equation*}
m(\frac{n}{2},\frac{n}{2}-2)
=
\begin{cases}
2
\qquad
&\text{if }
(\lambda, \nu, \delta \varepsilon) =(\frac{n}{2},\frac{n}{2}+1,-),  
\\
0 
&
\text{otherwise}.  
\end{cases}
\end{equation*}
\end{enumerate}

\item[{\rm{(4)}}]
Suppose $j=i+1$.  
\begin{enumerate}
\item[{\rm{(a)}}]
Case $i=0$ and $n>3$.
\begin{equation*}
m(0,1)
=
\begin{cases}
1
\qquad
&\text{if }
\lambda \in - {\mathbb{N}}, \nu=1, \text{ and }\delta\varepsilon = (-1)^{\lambda+1}, 
\\
0
&
\text{otherwise.}
\end{cases}
\end{equation*}
\item[{\rm{(b)}}]
Case $1 \le i < \frac{n-3}{2}$.  
\begin{equation*}
m(i,i+1)
=
\begin{cases}
1
\qquad
&\text{if }
(\lambda, \nu, \delta \varepsilon) =(i,i+1,-),  
\\
0 
&
\text{otherwise}.  
\end{cases}
\end{equation*}
\item[{\rm{(c)}}]
Case $i=\frac{n-3}{2}$ and $n>3$, odd.  
\begin{equation*}
m(\frac{n-3}{2}, \frac{n-1}{2})
=
\begin{cases}
2
\qquad
&\text{if }
(\lambda, \nu, \delta \varepsilon) =(\frac{n-3}{2},\frac{n-1}{2},-),  
\\
0 
&
\text{otherwise}.  
\end{cases}
\end{equation*}
\item[{\rm{(d)}}]
Case $i=0$ and $n=3$.  
\begin{equation*}
m(0, 1)
=
\begin{cases}
2
\qquad
&\text{if 
$\lambda \in -{\mathbb{N}}$, $\nu=1$, 
 and $\delta \varepsilon = (-1)^{\lambda+1}$},  
\\
0 
&
\text{otherwise}.  
\end{cases}
\end{equation*}
\end{enumerate}

\item[{\rm{(5)}}]
Suppose $j \not\in \{i-2, i-1, i, i+1\}$.
Then $m(i,j)=0$ for all $\lambda, \nu, \delta, \varepsilon$.
\end{enumerate}

\end{theorem}



\begin{remark}
[multiplicity-one property]
\label{rem:mult1}
In \cite{SunZhu}
 it is proved that 
$$
   \dim_{\mathbb{C}}\operatorname{Hom}_{{\overline{G'}}}(\Pi|_{{\overline{G'}}},\pi) \le 1
$$
 for any irreducible admissible smooth representations
 $\Pi$ and $\pi$ of ${\overline{G}}=SO(n+1,1)$ and ${\overline{G'}}=SO(n,1)$, 
respectively.  
Thus Theorem \ref{thm:1.1} fits well 
 with their multiplicity-free results 
 for 
  $\lambda, \nu \in {\mathbb{C}}\setminus {\mathbb{Z}}$, 
 where ${\overline{I}}_{\delta}(i,\lambda)$ and ${\overline{J}}_{\varepsilon}(j,\nu)$
 are irreducible admissible representations
 of ${\overline{G}}$ and ${\overline{G'}}$, 
 respectively,
 except for the cases $n=2 i$ or $n= 2j+1$.  
In the case $n=2i$ or $n=2j+1$, 
 the multiplicity is counted twice 
 as we saw in \eqref{eqn:ISOneven} and \eqref{eqn:161733}, 
 and thus the statements (1-d), (2-c), (3-b), 
 and (4-c)
 in Theorem \ref{thm:1.1} fit again with \cite{SunZhu}.  
\end{remark}
\vskip 1pc
\begin{remark}
[generic multiplicity-two phenomenon]
\label{rem:mult2}
In addition to the subgroup ${\overline{G'}}=SO(n,1)$, 
 the Lorentz group $O(n,1)$ contains two other subgroups
 of index two, 
 that is, 
 $O^+(n,1)$ (containing orthochronous reflections)
 and  $O^-(n,1)$ (containing anti-orthochronous reflections)
 with terminology
 in relativistic space-time for $n=3$.  
Our results yield also the multiplicity formula for such pairs
 by using an analogous result to Proposition \ref{prop:LNM26},
and it turns out that a 
\index{B}{genericmultiplicityonetheorem@generic multiplicity-one theorem}
generic multiplicity-one statement
 fails 
 if we replace $({\overline{G}},{\overline{G'}})=(SO(n+1,1),SO(n,1))$
 by $(O^-(n+1,1),O^-(n,1))$.  
In fact, 
 the multiplicity $m(\Pi, \pi)$ is generically equal to 2
 for irreducible representations $\Pi$ and $\pi$
 of $O^-(n+1,1)$ and $O^-(n,1)$, 
 respectively, 
 as is expected by the general theory \cite{xKOfm, sbon}
 because there are two open orbits
 in $P'^{-} \backslash G^-/P^-$ in this case.  
\end{remark} 


%%%%%%%%%%%%%%%%%%%%%%%%%%%%%%%%%%%%%%%%%%%%%%%%%%%%%%%%%%%%%%%%%%%%%%%%%%%
\subsection{Symmetry breaking
 between irreducible representations
 of $\overline G$ and $\overline{G'}$
 with trivial infinitesimal character $\rho$}
%%%%%%%%%%%%%%%%%%%%%%%%%%%%%%%%%%%%%%%%%%%%%%%%%%%%%%%%%%%%%%%%%%%%%%%%%%%
Similar to the notation $\overline{\Pi}_{i,\delta}$
 for the restriction of the irreducible representation
 ${\Pi}_{i,\delta}$ of $G=O(n+1,1)$ 
 to the special orthogonal group $\overline G=SO(n+1,1)$, 
 we denote by $\overline \pi_{j,\varepsilon}$
 the restriction of the irreducible representation $\pi_{j,\varepsilon}$
 ($0 \le j \le n$, $\varepsilon=\pm$)
 of $G'=O(n,1)$ to the special orthogonal group $\overline {G'}=SO(n,1)$.  
Then $\overline{\Pi}_{i,\delta}$ ($0 \le i \le n+1$, $\delta=\pm$)
 and $\overline{\pi}_{j,\varepsilon}$ ($0 \le j \le n$, $\varepsilon=\pm$)
 exhaust irreducible admissible smooth representations of $\overline G$
 and $\overline {G'}$ 
 having ${\mathfrak{Z}}({\mathfrak{g}})$-infinitesimal character 
 $\rho_{\overline G}$ 
 and ${\mathfrak{Z}}({\mathfrak{g}}')$-infinitesimal character 
 $\rho_{\overline {G'}}$
 respectively,
 by Lemma \ref{lem:Pilso}.  



In this section,
 we deduce the formula
 of the multiplicity
\[ 
   \dim_{\mathbb{C}} {\operatorname{Hom}}_{\overline {G'}}
   (\overline \Pi_{i,\delta}|_{\overline {G'}}, \overline \pi_{j,\varepsilon})
\]
for the symmetry breaking for 
 $(\overline G, \overline {G'})=(SO(n+1,1), SO(n,1))$ from the one
 for $(G, G')=(O(n+1,1), O(n,1))$.  
 
In view of the $\overline G$-isomorphism
 $\overline{\Pi}_{\frac {n+1}2,+}\simeq \overline{\Pi}_{\frac {n+1}2,-}$
 for $n$ even
 and the $\overline {G'}$-isomorphism
 $\overline{\pi}_{\frac {n}2,+}\simeq \overline{\pi}_{\frac {n}2,-}$
 for $n$ odd, 
 we shall use the following convention
\begin{equation}
\label{eqn:halfsgn}
  \text{$+ \equiv -$ for $\delta$ if $n+1=2i$;
\quad
       $+ \equiv -$ for $\varepsilon$ if $n=2j$
}
\end{equation}
when we deal with the representations
 $\overline{\Pi}_{i,\delta}$
 ($0 \le i \le [\frac{n+1}{2}]$)
 and $\overline{\pi}_{j,\varepsilon}$
 ($0 \le j \le [\frac{n}{2}]$).  



Owing to Proposition \ref{prop:LNM26}, 
 Theorem \ref{thm:LNM20} tells that 
\begin{equation*}
 \operatorname{Hom}_{\overline{G'}}(\overline{\Pi}_{i,\delta}|_{\overline{G'}}, \overline{\pi}_{j, \varepsilon})
\simeq
  \operatorname{Hom}_{G'}(\Pi_{i,\delta}|_{G'}, \pi_{j,\varepsilon})
  \oplus
  \operatorname{Hom}_{G'}(\Pi_{n+1-i,-\delta}|_{G'}, \pi_{j, \varepsilon}).
\end{equation*}
Applying Theorems \ref{thm:SBOvanish} and \ref{thm:SBOone}
 about symmetry breaking for the pair
 $(G,G')=(O(n+1,1), O(n,1))$
 to the right-hand side, 
 we determine the multiplicity
\[
   m(\overline \Pi, \overline \pi)
\quad
\text{for all $\overline \Pi \in {\operatorname{Irr}}(\overline G)_{\rho}$
      and $\overline \pi \in {\operatorname{Irr}}(\overline {G'})_{\rho}$}
\]
 for the pair $(\overline G, \overline{G'})=(SO(n+1,1), SO(n,1))$
 of special orthogonal groups as follows.  
\begin{theorem}
\label{thm:170336}
Suppose $0 \le i \le [\frac{n+1}{2}]$, 
 $0 \le j \le [\frac{n}{2}]$, 
 and $\delta, \varepsilon = \pm$
 with the convention \eqref{eqn:halfsgn}.  
then 
\[
\dim_{\mathbb{C}} {\operatorname{Hom}}_{\overline{G'}}
(\overline{\Pi}_{i,\delta}|_{\overline{G'}}, 
 \overline{\pi}_{j,\varepsilon})
=
\begin{cases}
1
\quad
&\text{if $\delta \equiv \varepsilon$ and $j \in \{i-1,i\}$},
\\
0
&\text{otherwise.}
\end{cases}
\]
\end{theorem}






