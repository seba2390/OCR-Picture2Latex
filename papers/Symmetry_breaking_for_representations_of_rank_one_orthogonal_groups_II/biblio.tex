\bigskip

\begin{thebibliography}{M}

\bibitem{ABV}
J.~Adams, D.~Barbasch and D.~A.~Vogan, Jr.,
The Langlands Classification and Irreducible Characters for Real Reductive Groups,
Progr. Math., {\bf{104}}, Birkh\"{a}user Boston, Boston, MA, 1992.

\bibitem{AG}
A.~Aizenbud and D.~Gourevitch,
Multiplicity one theorem for $(GL_{n+1}({\mathbb{R}}), GL_n({\mathbb{R}}))$,
Selecta Math. (N.S.), {\bf{15}} (2009), 271--294.

\bibitem{Arthur}
J.~Arthur,
Classifying automorphic representations,
In: Current Developments in Mathematics 2012,
Int. Press, Somerville, MA, 2013, pp. 1--58.

\bibitem{BaSiKn}
M.~W.~Baldoni Silva and A.~W.~Knapp,
Unitary representations induced from maximal parabolic subgroups,
J. Funct. Anal., {\bf{69}} (1986), 21--120.

\bibitem{BC}
N.~Bergeron and L.~Clozel,
Spectre automorphe des vari{\'e}t{\'e}s hyperboliques et applications topologiques,
Ast{\'e}risque, {\bf{303}}, Soc. Math. France, 2005, xx+218 pp.

\bibitem{BMM} 
N.~Bergeron, J.~Millson and C.~M{\oe}glin,
Hodge type theorems for arithmetic manifolds associated to orthogonal groups,
Int. Math. Res. Not. IMRN, {\bf 2017}, no.~{15}, 4495--4624.

\bibitem{BeDu}
N.~Conze-Berline and M.~Duflo,
Sur les repr{\'e}sentations induites des groupes semi-simples complexes,
Compositio~Math., {\bf{34}} (1977), 307--336.

\bibitem{raphael}
R.~Beuzart-Plessis,
A local trace formula for the Gan--Gross--Prasad conjecture for unitary groups: the archimedean case,
arXiv:1506.01452.

\bibitem{BW}
A.~Borel and N.~R.~Wallach,
Continuous Cohomology, Discrete Subgroups, and Representations of Reductive Groups.
Second ed., Math.~Surveys Monogr., {\bf{67}}, Amer. Math. Soc., Providence, RI,
2000, xviii+260 pp.;
First ed., Ann. of Math. Stud., {\bf{94}}, Princeton Univ. Press, Princeton, NJ,
1980, xvii+388 pp.

\bibitem{BLS}
M.~Burger, J.-S.~Li and P.~Sarnak,
Ramanujan duals and automorphic spectrum,
Bull. Amer. Math. Soc. (N.S.), {\bf{26}} (1992), 253--257.

\bibitem{C}
D. H.~Collingwood,
Representations of Rank One Lie Groups,
Res. Notes in Math., {\bf{137}}, 
Pitman (Advanced Publishing Program), Boston, MA, 1985, vii+244 pp.

\bibitem{EMOT}
A.~Erd\'elyi, W.~Magnus, F.~Oberhettinger and F.~G.~Tricomi,
Tables of integral transforms. Vol. II,
McGraw-Hill Book Company, Inc., New York--Toronto--London, 1954.

\bibitem{A1} 
W.~T.~Gan, B.~H.~Gross, D.~Prasad and J.-L.~Waldspurger,
Sur les conjectures de Gross et Prasad. I,
Ast{\'e}rique, {\bf{346}}, Soc. Math. France, 2012, xi+318 pp.

\bibitem{GN}
I.~ M.~Gel'fand and M.~A.~Na{\u{\i}}mark,
Unitary Representations of Classical Groups,
Trudy Mat. Inst. Steklov., {\bf{36}}, 
Izdat. Nauk SSSR, 1950, 288 pp.

\bibitem{GP} 
B.~H.~Gross and D.~Prasad,
On the decomposition of a representations of SO$_n$ when restricted to SO$_{n-1}$,
Canad. J. Math., {\bf{44}} (1992), 974--1002.

\bibitem{GW}
B.~H.~Gross and N.~Wallach,
Restriction of small discrete series representations to symmetric subgroups,
In: The Mathematical Legacy of Harish-Chandra, Baltimore, MD, 1998,
Proc. Sympos. Pure Math., {\bf{68}}, Amer. Math. Soc., Providence, RI, 2000,
pp. 255--272.

\bibitem{Hirai62}
T.~Hirai,
On irreducible representations of the Lorentz group of $n$-th order,
Proc. Japan Acad., {\textbf{38}} (1962), 
\href{https://projecteuclid.org/euclid.pja/1195523378}
{258--262}. 

\bibitem{HT93}
R.~E.~Howe and E.-C.~Tan,
Homogeneous functions on light cones: the infinitesimal structure of some
degenerate principal series representations,
Bull. Amer. Math. Soc. (N.S.), {\textbf{28}} (1993), 1--74.

\bibitem{Hu}
J.~E.~Humphreys,
Representations of Semisimple Lie Algebras in the BGG Category ${\mathcal{O}}$,
Grad. Stud. Math., {\bf{94}}, Amer. Math. Soc., Providence, RI, 2008.

\bibitem{II} 
A.~Ichino and T.~Ikeda,
On the periods of automorphic forms on special orthogonal groups and the Gross--Prasad conjecture,
Geom. Funct. Anal., {\bf{19}} (2010), 1378--1425.

\bibitem{IY} 
A.~Ichino and S.~Yamana,
Periods of automorphic forms: the case of $(\text{GL}_{n+1}\times \text{GL}_{n},\text{GL}_{n})$,
Compos. Math., {\bf{151}} (2015), 665--712.

\bibitem{Jantzen}
J.~C.~Jantzen, 
Moduln mit einem h{\"o}chsten Gewicht,
Lecture Notes in Math., 
{\bf 750}, 
Springer-Verlag, 
Berlin/Heidelberg/New York, 
1979.  

\bibitem{Juhl}
A.~Juhl,
Families of Conformally Covariant Differential Operators, $Q$-Curvature and Holography,
Progr. Math., {\bf{275}}, Birkh\"auser Verlag, Basel, 2009.

\bibitem{KS}
A.~W.~Knapp and E.~M.~Stein,
Intertwining operators for semisimple groups,
Ann. of Math. (2), {\bf{93}} (1971), 489--578.

\bibitem{KS2}
A.~W.~Knapp and E.~M.~Stein,
Intertwining operators for semisimple groups. II,
Invent. Math., {\bf{60}} (1980), 9--84.

\bibitem{KV}
A.~W.~Knapp and D. A. Vogan, Jr.,
Cohomological Induction and Unitary Representations,
Princeton Math. Ser., {\textbf{45}}. Princeton Univ. Press,
Princeton, NJ, 1995, xx+948 pp., ISBN: 978-0-691-03756-6.

\bibitem{KZ}
A.~W.~Knapp and G.~J.~Zuckerman,
Classification of irreducible tempered representations of semisimple groups,
Ann. of Math. (2), {\bf{116}} (1982), 389--455;
II, ibid, 457--501.

\bibitem{xkpro89}
T. Kobayashi, 
Proper action on a homogeneous space of reductive type,
Math.~Ann., {\bf{285}} (1989), 
\href{https://doi.org/10.1007/BF01443517}
{249--263}.  

\bibitem{KMemoirs92}
T.~Kobayashi,
Singular Unitary Representations and Discrete Series for Indefinite Stiefel
Manifolds $U(p,q;{\mathbb{F}})/U(p-m,q;{\mathbb{F}})$,
Mem. Amer. Math. Soc.,
\href{http://www.ams.org/books/memo/0462/}{{\bf{462}}},
Amer. Math. Soc., Providence, RI,
1992, 106 pp., ISBN: 978-0-8218-2524-2.

\bibitem{KInvent98}
T.~Kobayashi,
Discrete decomposability of the restriction of $A_{\mathfrak{q}}(\lambda)$
with respect to reductive subgroups. III.
Restriction of Harish-Chandra modules and associated varieties,
Invent. Math., {\bf{131}} (1998),
\href{http://dx.doi.org/10.1007/s002220050203}
{229--256}.

\bibitem{xktransgp12}
T.~Kobayashi,
Restrictions of generalized Verma modules to symmetric pairs,
Transform. Groups, {\bf{17}} (2012), 523--546.

\bibitem{xkHelg85}
T.~Kobayashi,
$F$-method for constructing equivariant differential operators,
In: Geometric Analysis and Integral Geometry,
Contemp. Math., {\bf{598}}, Amer. Math. Soc., Providence, RI,
2013, pp.
\href{http://dx.doi.org/10.1090/conm/598/11998}
{139--146}.

\bibitem{xkEastwood}
T.~Kobayashi,
F-method for symmetry breaking operators,
Differential Geom. Appl., {\bf{33}} (2014),
272--289, 
Special issue
``The Interaction of Geometry and Representation Theory.
Exploring New Frontiers"
(in honor of Michael Eastwood's 60th birthday).

\bibitem{xkShintani}
T.~Kobayashi,
Shintani functions, real spherical manifolds, and symmetry breaking operators,
In: Developments and Retrospectives in Lie Theory,
Dev. Math., {\bf{37}}, Springer, 2014, pp.
\href{http://dx.doi.org/10.1007/978-3-319-09934-7_5}
{127--159}.

\bibitem{xkvogan}
T.~Kobayashi,
A program for branching problems in the representation theory of real reductive groups,
In: Representations of Lie Groups. In Honor of David A. Vogan, Jr. on his 60th Birthday,
Progr. Math., {\bf{312}}, Birkh{\"a}user, 2015, pp.
\href{http://dx.doi.org/10.1007/978-3-319-23443-4_10}
{277--322}.

\bibitem{xkresidue}
T.~Kobayashi,
Residue formula for regular symmetry breaking operators,
Contemp. Math., {\bf{714}}, Amer. Math. Soc.,
Province, RI, 2018, pp.~
\href{http://doi.org/10.1090/com/714/14380}
{175--193};
available also at 
\href{http://arxiv.org/abs/1709.05035}
{arXiv:1709.05035}.  

\bibitem{KKP}
T.~Kobayashi, T.~Kubo and M.~Pevzner,
Conformal Symmetry Breaking Operators for Differential Forms on Spheres,
Lecture Notes in Math., 
\href{http://dx.doi.org/10.1007/978-981-10-2657-7}
{{\bf{2170}}}, Springer, 2016, iv+192 pp., ISBN: 978-981-10-2657-7.

\bibitem{xKMt}
T. Kobayashi and T. Matsuki,
Classification of finite-multiplicity symmetric pairs,
Transform. Groups, {\bf{19}} (2014), 
\href{http://dx.doi.org/10.1007/s00031-014-9265-x}
{457--493},
Special issue in honor of Dynkin for his 90th birthday.
%\href{http://dx.doi.org/10.1007/s00031-014-9265-x }
%{doi:10.1007/s00031-014-9265-x}
%(available at 
%%\href{http://arxiv.org/abs/1312.4246}
%{arXiv:1312.4246}).  

\bibitem{xkors3}
T.~Kobayashi and B.~{\O}rsted,
{Analysis on the minimal representation of
{${\rm O}(p,q)$}. {\rm{III}}.
Ultrahyperbolic equations on $\mathbb{R}^{p-1,q-1}$},
Adv. Math., \textbf{180} (2003), 
\href{http://dx.doi.org/10.1016/S0001-8708(03)00014-8}{551--595}.

\bibitem{KOSS}
T.~Kobayashi, B.~{\O}rsted, P.~Somberg and V.~Sou\v{c}ek,
Branching laws for Verma modules and applications in parabolic geometry. I,
Adv. Math., {\bf{285}} (2015),
\href{http://dx.doi.org/10.1016/j.aim.2015.08.020}
{1796--1852}.
%\href{http://arxiv.org/abs/1305.6040}
%{arXiv:1305.6040}.

\bibitem{xKOfm}
T.~Kobayashi and T.~Oshima,
Finite multiplicity theorems for induction and restriction,
Adv. Math., {\bf{248}} (2013),
\href{http://dx.doi.org/10.1016/j.aim.2013.07.015}
{921--944}.
%DOI 10.1016/j.aim.2013.07.015.  

\bibitem{KP1}
T.~Kobayashi and M.~Pevzner,
Differential symmetry breaking operators. I.
General theory and F-method,
Selecta Math. (N.S.), {\bf{22}} (2016),
\href{http://dx.doi.org/10.1007/s00029-015-0207-9}
{801--845}.

\bibitem{KP2}
T. Kobayashi and M. Pevzner,
Differential symmetry breaking operators. II.
Rankin--Cohen operators for symmetric pairs,
Selecta Math. (N.S.), {\bf{22}} (2016),
\href{http://dx.doi.org/10.1007/s00029-015-0208-8}
{847--911}.

\bibitem{sbon}
T. Kobayashi and B. Speh,
Symmetry Breaking for Representations of Rank One Orthogonal Groups,
Mem. Amer. Math. Soc., {\bf{238}}, Amer. Math. Soc., Providence, RI,
2015,
\href{http://dx.doi.org/10.1090/memo/1126}
{v+112} pp., ISBN: 978-1-4704-1922-6.
%ISBNs: 978-1-4704-1922-6 (print); 978-1-4704-2615-6 (online)

\bibitem{sbonGP}
T.~Kobayashi and B.~Speh,
Symmetry breaking for orthogonal groups and a conjecture by B.~Gross and D.~Prasad,
In: Geometric Aspects of the Trace Formula, Simons Symposia, 
W.~M{\"u}ller et al.(eds.), 
Springer, 2018, 
pp.
\href{https://doi.org/10.1007/978-3-319-94833-1_8}
{245--266};
available also at
\href{http://arxiv.org/abs/1702.00263}
{arXiv:1702.00263}.

\bibitem{Kos}
B.~Kostant,
Verma modules and the existence of quasi-invariant differential operators,
In: Non-commutative Harmonic Analysis, Marseille--Luminy, 1974,
Lecture Notes in Math., {\bf 466}, Springer, Berlin, 1975, pp. 101--128.

\bibitem{Kr1}
M.~Kr{\"a}mer,
Multiplicity free subgroups of compact connected Lie groups,
Arch. Math. (Basel), {\bf{27}} (1976), 28--36.

\bibitem{KM2}
S.~S.~Kudla and J.~J.~Millson,
Geodesic cyclics and the Weil representation. I.
Quotients of hyperbolic space and Siegel modular forms,
Compositio Math., {\bf{45}} (1982), 207--271.

\bibitem{KM-I}
S.~S.~Kudla and J.~J.~Milllson,
The theta correspondence and harmonic forms. I,
Math. Ann., {\bf{274}} (1986), 353--378.

\bibitem{KM-II}
S.~S.~Kudla and J.~J.~Millson,
The theta correspondence and harmonic forms. II,
Math. Ann., {\bf{277}} (1987), 267--314.

\bibitem{LLNM}
R.~P.~Langlands,
On the Functional Equation Satisfied by Eisenstein Series,
Lecture Notes in Math., {\bf{544}}, Springer-Verlag, New York, 1976.

\bibitem{La88}
R.~P.~Langlands,
On the classification of irreducible representations of real algebraic groups,
In: Representation Theory and Harmonic Analysis on Semisimple Lie Groups,
Math. Surveys Monogr., {\bf{31}}, Amer. Math. Soc., Providence, RI, 1989,
pp. 101--170.

\bibitem{Li92}
J.-S.~Li,
Nonvanishing theorems for the cohomology of certain arithmetic quotients,
J. Reine Angew. Math., {\bf{428}} (1992), 177--217.

\bibitem{A2}
C.~M{\oe}glin and J.-L.~Waldspurger,
Sur les conjectures de Gross et Prasad. II,
Ast\'{e}rique, {\bf{347}},
Soc. Math. France, 2012.

\bibitem{Pe}
J.~Peetre,
Une caract{\'e}risation abstraite des op{\'e}rateurs diff{\'e}rentiels,
Math. Scand., {\bf{7}} (1959), 211--218.
%\bibitem{RS}
%J.~Rohlfs and B.~Speh,  
%Pseudo Eisenstein forms and the cohomology of arithmetic groups III:
% residual cohomology classes. On certain L-functions, 501--523,
% Clay Math. Proc., {\bf{13}}, Amer. Math. Soc., Providence, RI, 2011. 

\bibitem{SpehVen}
B.~Speh and T.~N.~Venkataramana,
Discrete components of some complementary series,
Forum Math., {\bf{23}} (2011), 1159--1187.

\bibitem{SpehVogan}
B.~Speh and D.~A. Vogan, Jr.,
 Reducibility of generalized principal series 
representations,
Acta Math., {\bf{145}}
(1980), 227--299.  

\bibitem{S}
B.~Sun,
The nonvanishing hypothesis at infinity for Rankin--Selberg convolutions,
J. Amer. Math. Soc., {\bf{30}} (2017), 1--25.

\bibitem{SunZhu}
B. Sun and C.-B. Zhu,
Multiplicity one theorems: the Archimedean case,
Ann. of Math. (2), {\bf{175}} (2012),
\href{http://dx.doi.org/10.4007/annals.2012.175.1.2}
{23--44}.

\bibitem{TW}
Y.~L.~Tong and S.~P.~Wang,
Geometric realization of discrete series for semisimple symmetric spaces,
Invent. Math., {\bf{96}} (1989), 425--458.

\bibitem{Treves}
F.~Tr{\`e}ves,
Topological Vector Spaces, Distributions and Kernels,
Academic Press, New York--London, 1967, xvi+624 pp.

\bibitem{Vargas}
J.~A.~Vargas,
Restriction of some discrete series representations,
Algebras Groups Geom., {\bf{18}} (2001), 85--99.

\bibitem{Vogan81}
D.~A.~Vogan, Jr., 
Representations of Real Reductive Lie Groups, 
Progr.~Math., {\bf{15}}, 
Birkh{\"a}user, Boston, MA, 1981,
xvii+754 pp.

\bibitem{V}
D.~A.~Vogan, Jr.,
The local Langlands conjecture,
In: Representation Theory of Groups and Algebras,
Contemp. Math., {\bf{145}}, Amer. Math. Soc., Providence, RI,
1993, pp. 305--379.

\bibitem{VZ}
D.~A.~Vogan, Jr. and G.~J.~Zuckerman,
Unitary representations with nonzero cohomology,
Compositio Math., {\bf{53}} (1984), 51--90.

\bibitem{W}
N.~R.~Wallach,
Real Reductive Groups. I,
Pure Appl. Math., {\bf{132}}, Academic Press, Boston, MA, 1988,
xx+412 pp., ISBN: 0-12-732960-9;
II, ibid, {\bf{132}}-II, Academic Press, Boston, MA, 1992,
xiv+454 pp., ISBN: 978-0127329611.

\bibitem{Wa}
S.~P.~Wang,
Correspondence of modular forms to cycles associated to $O(p,q)$,
J. Differential Geom., {\bf{22}} (1985), 151--213.

\bibitem{Weyl97}
H.~Weyl,
The Classical Groups. Their Invariants and Representations,
Princeton Landmarks Math., Princeton Univ. Press, Princeton,
NJ, 1997.

\bibitem{Zuckerman}
G.~Zuckerman, 
Tensor products of finite and infinite dimensional representations
 of semisimple Lie groups, 
Ann.~of Math. (2),
{\bf{106}} (1977), 295--308.  

\end{thebibliography}
%\end{document}
