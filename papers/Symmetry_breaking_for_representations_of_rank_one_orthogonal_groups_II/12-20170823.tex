\newpage
\section{Application II:
 Periods, distinguished representations and $(\mathfrak{g},K)$-cohomologies}
\label{sec:period}
\index{B}{gKcohomology@$(\mathfrak{g},K)$-cohomology}

Let $H$ be a subgroup of $G$.  
Following the terminology used in automorphic forms and the relative trace formula,
 we say that a smooth representation $U$ of $G$ is $H$-distinguished if
there exists 
 a nontrivial $H$-invariant linear functional 
\[
   F^{H}:U \rightarrow \bC, 
\]
 {\it{i.e.}}, 
 if $U$ has a nontrivial 
\index{B}{period@period}
$H$-period $F^{H}$. 
We consider first irreducible representations of $G$
 with infinitesimal character $\rho$ which are $H$-distinguished
 for the  pair $(G,H)=(O(n+1,1),O(m+1,1))$ 
 or for the pair $(G,H)= (O(n,1) \times O(m,1), O(m,1))$  with $m \le n$.  
We then discuss  a bilinear form on the $(\mathfrak{g},K)$-cohomology
 of the representations of $(O(n+1,1) \times O(n,1))$
 with infinitesimal character $\rho$
 which is induced by a symmetry breaking operator.  



\subsection{Periods and $O(n,1)$-distinguished representations}
\subsubsection{Periods}
Let $\mathbb K$ be a number field, $\mathbb A$ its adels and 
 let $G_1\times G_2$ be a direct product
 of semisimple groups over a number field $\mathbb K$. 
We assume that $G_2 \subset G_1$. 
If the outer tensor product representation
 $\Pi_{\mathbb A} \boxtimes \pi_{\mathbb A}$ is an automorphic representation
 of the direct product group $G_1({\mathbb A}) \times G_2({\mathbb A})$, 
 then the $G_2$-period integral is defined as
\[
  \int_{G_2({\mathbb K} )\backslash G_2({\mathbb A} )}\Phi_1 (h) \phi_2(h) dh.
\]
Here $\Phi_1$ and $\phi_2$ are smooth vectors
 for the representation $\Pi_{\mathbb A} \boxtimes \pi_{\mathbb A}$. 
If $\Pi_{\mathbb A} \boxtimes \pi_{\mathbb A}$ is cuspidal, 
 then the integral converges and 
it defines a $G_2({\mathbb A})$-invariant linear functional
 on the smooth vectors of $\Pi_{\mathbb A} \boxtimes \pi_{\mathbb A}$.  
If this linear functional is not zero,
 then $\Pi_{\mathbb A}\boxtimes \pi_{\mathbb A}$
 is called $G_2$-{\it{distinguished}}. 
Conjecturally for certain pairs of groups
 the value of this integral is a multiple
 of the central value of an $L$-function, 
 see \cite{A1, II, IY}.


Often this period integral  factors into a product of local integrals. Following the global terminology
 we say that an admissible smooth representation $\Pi \boxtimes \pi$
 of the direct product group $G_1(\bR)\times G_2({\mathbb R})$ is $G_2({\mathbb R})$-{\it{distinguished}}
 if there is a nontrivial continuous linear functional 
\[F^{G_2({\mathbb R})}:\Pi \boxtimes \pi \rightarrow  {\mathbb C} \]
 which is invariant by $G_2({\mathbb{R}})$
 under the diagonal action.  
Here we recall Section \ref{subsec:BFSBO}
 for the topology on the tensor product.  
If $\Pi \boxtimes \pi$ is $G_2({\mathbb R})$-distinguished,
 we say that $F^{G_2({\mathbb R})}$ is a 
\index{B}{period@period|textbf}
{\it{period}} of $\Pi \boxtimes \pi$.  
We say that the period is nontrivial
 on a vector
 $\Phi \otimes \phi \in \Pi \boxtimes \pi$
 if $\Phi \otimes \phi$ is not in the kernel of $F^{G_2({\mathbb R})}$. 
If the period is nontrivial on a unit function $\Phi \otimes \phi$,
 we refer to its image as the value of the period on $\Phi \otimes \phi$. 



\begin{remark} 
The integral 
\[  \int_{G_2(\mathbb R )}
\Phi (h) \phi (h) dh
\]
converges for some  smooth vectors of 
\index{B}{discreteseries@discrete series representation}
discrete series representations
 $\Pi \boxtimes \pi$
 for some symmetric pairs $(G_1(\mathbb R),G_2(\mathbb R))$.  
This was used by J.~Vargas \cite{Vargas}
 to determine  some subrepresentations
 in the restriction of some discrete series representations $\Pi$ of $G_1(\mathbb R)$
 to the subgroup $G_2(\mathbb R)$.  
\end{remark}


\medskip
We recall from Theorem \ref{thm:SBOBF}
 that the space of symmetry breaking operators 
\[ 
     \mbox{Hom}_{G_2(\bR)}(\Pi|_{G_2(\bR)}, \pi^{\vee}) 
\]
 and the space of $G_2(\bR)$-invariant continuous linear functionals
\[\mbox{Hom}_{G_2(\bR)}(\Pi \boxtimes \pi,\bC)\]
 are naturally isomorphic to each other. 
Thus, 
 instead of considering a $G_2(\bR)$-equivariant continuous linear functional
 defined by an integral, 
 we may use symmetry breaking operators
 to construct $G_2(\bR)$-invariant continuous  linear functionals. 
This technique allows us to obtain
$G_2(\bR)$-invariant continuous linear functionals
 not only for discrete series representations
 but also for nontempered representations.  
Thus we can determine 
 for the pair $(G,G')= (O(n+1,1), O(n,1))$
 the dimension of the space
 ${\operatorname{Hom}}_{G'}(\Pi \boxtimes \pi, {\mathbb{C}})$
for all $\Pi \in {\operatorname{Irr}}(G)_{\rho}$
 and $\pi \in {\operatorname{Irr}}(G')_{\rho}$
 as follows.  



\begin{corollary}
\label{cor:bilin}
Suppose $0 \le i \le n+1$, 
 $0 \le j \le n$, 
 and $\delta$, $\varepsilon \in \{\pm\}$.  
Let $\Pi_{i,\delta}$ and $\pi_{j,\varepsilon}$ be
 irreducible admissible smooth representations of $G=O(n+1,1)$ and $G'=O(n,1)$,  respectively, 
 that have the trivial infinitesimal character $\rho$ as in \eqref{eqn:Pild}. 
Then the following three conditions 
 on $(i,j,\delta,\varepsilon)$ are equivalent:
\begin{enumerate}
\item[{\rm{(i)}}]
${\operatorname{Hom}}_{G'}( \Pi_{i,\delta} \boxtimes \pi_{j,\varepsilon}, \bC) \not = \{0\};$
\item[{\rm{(ii)}}]$\dim_{\mathbb{C}}{\operatorname{Hom}}_{G'}(\Pi_{i,\delta} \boxtimes \pi_{j,\varepsilon}, \bC) = 1;$
\item[{\rm{(iii)}}]$j \in \{i,i-1 \}$ and $\delta=\varepsilon$.  
\end{enumerate}
\end{corollary}
\begin{proof}
Owing to Theorem \ref{thm:SBOBF}, 
 this is a restatement
 of Theorems \ref{thm:SBOvanish} and \ref{thm:SBOone}.  
\end{proof}
%%%%%%%%%%%%%%%%%%%%
\subsubsection{Distinguished representations}
%%%%%%%%%%%%%%%%%%%%%%%%
Let $G$ be a reductive group, 
 and $H$ a reductive subgroup.  
We regard $H$ as a subgroup of the direct product group $G \times H$
 via the diagonal embedding $H \hookrightarrow G \times H$.  

\begin{definition}
Let $\psi$ be a one-dimensional representation of $H$. 
We say an admissible smooth representation  $\Pi$ of $G$ is
\index{B}{distinguishedHpsi@distinguished, $(H,\psi)$-|textbf}
$(H,\psi)$-{\it{distinguished}}
 if 
\[ \mbox{Hom}_{H}(\Pi \boxtimes \psi^{\vee}, {\mathbb{C}})
 \simeq \mbox{Hom}_{H}(\Pi|_H, \psi) \not = \{0\}.  
\]
If the character $\psi $ is trivial,
 we say $\Pi$ is 
\index{B}{distinguishedH@distinguished, $H$-|textbf}
$H$-{\it{distinguished}}.  
\end{definition}

In what follows, 
 we deal mainly with the pair
\[
   (G,H)=(O(n+1,1), O(m+1,1))
\quad
\text{for  $m \le n$.  }
\]


\begin{theorem}
\label{thm:period1}
Let $0 \le i \le n+1$.  
Then the representations $\Pi_{i,\delta}$ $(\delta \in \{\pm\})$
 of $G=O(n+1,1)$ are  $O(n+1-i,1)$-distinguished. 
\end{theorem}



The period is given by the composition 
 of the symmetry breaking operators
 that we constructed in Chapter \ref{sec:pfSBrho}
 with respect to the chain
 of subgroups
\begin{equation}
\label{eqn:seqOn}
  G=O(n+1,1) \supset O(n,1) \supset O(n-1,1) \supset \cdots \supset O(m+1,1)=H,
\end{equation}
 as we shall see in the proof
 in Section \ref{subsec:pfperiod}.  
Without loss of generality,
 we consider the case $\delta=+$, 
 and write simply $\Pi_i$ for $\Pi_{i,+}$.  
We recall from Theorem \ref{thm:LNM20} (3)
 that $\Pi_i \equiv \Pi_{i,+}$ has a 
\index{B}{minimalKtype@minimal $K$-type}
minimal $K$-type
 $\mub(i,+)=\Exterior^i({\mathbb{C}}^{n+1}) \boxtimes {\bf{1}}$. 



Let $v\in \Exterior^i({\mathbb{C}}^{n+1})$ be the image of $1 \in {\mathbb{C}}$
 via the following successive inclusions:
\[
   \Exterior^{i}({\mathbb{C}}^{n+1}) \supset 
   \Exterior^{i-1}({\mathbb{C}}^{n}) \supset 
   \cdots \supset
   \Exterior^{i-l}({\mathbb{C}}^{n+1-l}) \supset 
   \cdots \supset 
  \Exterior^{0}({\mathbb{C}}^{n+1-i}) \simeq {\mathbb{C}} \ni 1, 
\]
 and we regard $v$ as an element
 of the minimal $K$-type $\mub(i,+)$
 of $\Pi_i$.  

\begin{theorem}
\label{thm:period2}
Let $\Pi_i$ be the irreducible representation of $G=O(n+1,1)$, 
 and $v$ be the normalized element
 of its minimal $K$-type as above.  
For $0 \le i \le n$, 
 the value $F(v)$ of the $O(n+1-i,1)$-period $F$ 
 on $v \in \Pi_i$ is
\[
  \frac{\pi^{\frac 1 4 i(2n-i-1)}}{((n-i)!)^{i-1}}
  \times
\begin{cases}  
  \frac{1}{(n-2i)!}\quad &\text{if $2i < n+1$}, 
\\
  (-1)^{n+1} (2i-n-1)!\quad & \text{if $2i \ge n+1$}.  
\end{cases}
\]
\end{theorem} 

%%%%%%%%%%%%%%%%%%%%%%%%%%%%%%%%%%%%%%%%%%%%%%%%%%%%%%%%%%%%%%%%%%%%%%%%%%%%%%
\subsubsection{Symmetry breaking operators from $\Pi_{i,\delta}$
 to $\pi_{j,\delta}$
($j \in \{i-1,i\}$)}
\label{subsec:KKPipj}
%%%%%%%%%%%%%%%%%%%%%%%%%%%%%%%%%%%%%%%%%%%%%%%%%%%%%%%%%%%%%%%%%%%%%%%%%%%%%%
Let $(G,G')=(O(n+1,1),O(n,1))$.  
We recall from Theorem \ref{thm:LNM20} (2)
 that 
\begin{align*}
{\operatorname{Irr}}(G)_{\rho} = 
&\{\Pi_{i,\delta} : 0 \le i \le n+1, \delta=\pm\}, 
\\
{\operatorname{Irr}}(G')_{\rho} = 
&\{\pi_{j,\varepsilon} : 0 \le j \le n, \varepsilon=\pm\}.  
\end{align*}
In Chapter \ref{sec:pfSBrho},
 we constructed nontrivial symmetry breaking operators
\[
  A_{i,j} \colon \Pi_{i,\delta} \to \pi_{j,\varepsilon}
\]
for $j \in \{i-1,i\}$
 and $\delta = \varepsilon$, 
 and investigated their $(K,K')$-spectrum
 for minimal $K$- and $K'$-types, 
\[
  (\mu,\mu')=(\mub(i,\delta),\mub(j,\delta)'), 
\]
see Proposition \ref{prop:Aidown} in the case $j=i-1$
 and Proposition \ref{prop:AiiAq} in the case $j=i$.  



For the proofs of Theorems \ref{thm:period1} and \ref{thm:period2},
 we use these operators $A_{i,j}$ in the case $j=i-1$.  
For the study of the bilinear forms
 on $({\mathfrak{g}},K)$-cohomologies
 (see Section \ref{subsec:cohgkex} below), 
 we shall use them in the case $j=i$.  







%%%%%%%%%%%%%%%%%%%%%%%%%%%%%%%%%%%%%%%%%%%%%%%%%%%%%%%%%%%%%%%%%%%%%%
\subsection{Proofs of Theorems \ref{thm:period1} and \ref{thm:period2}}
\label{subsec:pfperiod}
%%%%%%%%%%%%%%%%%%%%%%%%%%%%%%%%%%%%%%%%%%%%%%%%%%%%%%%%%%%%%%%%%%%%%%%
We are ready to prove Theorems \ref{thm:period1} and \ref{thm:period2}
  by using Proposition \ref{prop:Aidown} successively.  
\begin{proof}
[Proof of Theorem \ref{thm:period1}]
Consider the chain \eqref{eqn:seqOn}
 of orthogonal subgroups 
 with $m=n-i$.  
For $1 \le \ell \le i$, 
 we denote by
\[
   A_{\ell, \ell-1}
   \colon
   \Pi_{i-\ell+1}^{O(n-\ell+2,1)} \to \Pi_{i-\ell}^{O(n-\ell+1,1)}
\]
 the symmetry breaking operator given in Proposition \ref{prop:Aidown}
 for the pair 
 $(O(n-\ell+2,1), O(n-\ell+1,1))$ of groups.  
Here \lq\lq{$\Pi_{i-\ell}^{O(n-\ell+1,1)}$}\rq\rq\ stands
 for the irreducible representation \lq\lq{$\Pi_{i-\ell,+}$}\rq\rq\
 of the group $O(n-\ell+1,1)$
 as given in Theorem \ref{thm:LNM20}, 
 by a little abuse of notation.  
Then the composition
\begin{equation}
\label{eqn:compfunct}
  F:= A_{1,0} \circ \cdots \circ A_{i-1,i-2} \circ A_{i,i-1}
\end{equation}
defines a nonzero $O(n+1-i,1)$-invariant functional
 on the irreducible representation $\Pi_i \equiv \Pi_{i,+}$
 of $G=O(n+1,1)$.  
\end{proof}

\begin{proof}
[Proof of Theorem \ref{thm:period2}]
The irreducible representation $\Pi_{i-\ell}^{O(n+1-\ell,1)}$, 
 namely, 
\lq\lq{$\Pi_{i-\ell,+}$}\rq\rq\
 of the group $O(n+1-\ell, 1)$
 has a minimal $K$-type
\[
   \mub(i-\ell,+)^{(\ell)}
   :=
   \Exterior^{i-\ell}({\mathbb{C}}^{n+1-\ell}) \boxtimes {\bf{1}}
   \in \widehat{O(n+1-\ell)}\times \widehat{O(1)}. 
\]
The $(K,K')$-spectrum of the symmetry breaking operator
 $A_{\ell, \ell-1}
   \colon
   \Pi_{i-\ell+1}^{O(n-\ell+2,1)} \to \Pi_{i-\ell}^{O(n-\ell+1,1)}
$
 for the minimal $K$-types
 $\mub (i-\ell+1,+)^{(\ell-1)} \hookleftarrow \mub (i-\ell,+)^{(\ell)}$
 is given by 
\[
\frac{\pi^{\frac {n-\ell}{2}}}{(n-i)!} \times 
\begin{cases}  
 n-2i+ \ell \qquad &\text{if $n \ne 2i -\ell$}, 
\\
  1 
  &\text{if $n = 2i -\ell$}, 
\end{cases}
\]
by Proposition \ref{prop:Aidown}.  
Applying this formula successively 
 to the sequence of minimal $K$-types:
\[
  \mub(i,+) \equiv \mub(i,+)^{(0)} \hookleftarrow \cdots \hookleftarrow
  \mub(i-\ell,+)^{(\ell)} \hookleftarrow \cdots 
  \hookleftarrow \mub(0,+)^{(i)}={\mathbb{C}}, 
\]
we get
\[
  F(v)=\prod_{\ell=1}^i \frac{\pi^{\frac{n-\ell}{2}} (n-2i+\ell)}{(n-i)!}
  =\frac{\pi^{\frac{1}{4}i(2n-i-1)}}{((n-i)!)^{i-1}(n-2i)!}
\]
 if $n > 2i-1$.  



On the other hand, 
 if $n < 2i-1< 2n-1$, 
 then 
\begin{align*}
  F(v)
  =\,&
  \left(\prod_{\ell=1}^{2i-n-1} \frac{\pi^{\frac{n-\ell}{2}} (n-2i+\ell)}{(n-i)!}\right)
  \cdot
  \frac{\pi^{n-i}}{(n-i)!}
  \cdot
 \left(\prod_{\ell=2i-n+1}^{i}\frac{\pi^{\frac{n-\ell}{2}} (n-2i+\ell)}{(n-i)!}\right)
\\
=\,&
\frac{\pi^{\frac 1 4 i(2n-i-1)}}{((n-i)!)^i}
(((-1)^{2i-n-1}(2i-n-1)!)\cdot 1 \cdot ((n-i)!))
\\
=\,&
\frac{(-1)^{n+1} \pi^{\frac 1 4 i(2n-i-1)}(2i-n-1)!}{((n-i)!)^{i-1}}.  
\end{align*}
The cases $i= \frac{n+1}{2}$ ($n$: odd) or $i=n$ are treated separately, 
 and it turns out that the formula of $F(v)$ coincides
 with the one for $i < 2i-1< 2n-1$.  
Thus we have completed the proof of Theorem \ref{thm:period2}.  
\end{proof}

In the next theorem,
 we consider the pair
\[
 \text{$(G,H)=(O(n+1,1),O(m+1,1))$
 with $m \le n$}.  
\]
We write $\Pi_i^G$ ($0 \le i \le n+1$)
 for the irreducible representation $\Pi_{i,+}$
 of $G$
 (see \eqref{eqn:Pild}),
 and write $\pi_j^H$ for the irreducible representation 
 \lq\lq{$\Pi_{j,+}$}\rq\rq\
 of the subgroup $H$ 
 for $0 \le j \le m+1$.  
Theorem \ref{thm:171517} below generalizes Theorem \ref{thm:period1}, 
 which corresponds to the case $j=0$.  

\begin{theorem}
\label{thm:171517}
Let $0 \le i \le n+1$ and $0 \le j \le m+1$.  
\begin{enumerate}
\item[{\rm{(1)}}]
The outer tensor product representation $\Pi_i^G \boxtimes \pi_j^H$ 
 of the direct product group $G \times H$ has an $H$-period if $0 \le i -j \le n-m$.  
\item[{\rm{(2)}}]
The period constructed 
 by the composition of the symmetry breaking operators 
 via the sequence \eqref{eqn:seqOn} 
 is nontrivial on the minimal $K$-type.  
\end{enumerate}
\end{theorem}



\begin{proof}
[Proof of Theorem \ref{thm:171517}]
The proof is essentially the same  
 with the one for Theorem \ref{thm:period1}
 except
 that we use not only the surjective symmetry breaking operator
 $A_{i,i-1} \colon \Pi_{i,+} \to \pi_{i-1,+}$
 for the pair $(G,G')=(O(n+1,1),O(n,1))$
but also the one 
\[
  A_{i,i} \colon \Pi_{i,+} \to \pi_{i,+}
\]
for which the $(K,K')$-spectrum
 on minimal $K$-types $\mub(i,+)\hookleftarrow \mub(i,+)'$
 is nonzero by Proposition \ref{prop:AiiAq}.  



Composing the symmetry breaking operators $A_{k,k-1}$ or $A_{k,k}$ successively
 to the sequence \eqref{eqn:seqOn} of orthogonal groups, 
 we get a nonzero continuous $H$-homomorphism
 $\Pi_i^G \to \pi_j^H$ 
 if $0 \le i-j \le n-m$.  
Then the first statement follows 
 because $\pi_j^H$ is self-dual.  
The second statement is clear by the construction
 and by the $(K,K')$-spectrum.   
\end{proof}


%%%%%%%%%%%%%%%%%%%%%%%%%%%%%%%%%%%%%%%%%%%%%%%%%%%%%%%%%%%%%%%%%%%%%%%%%%%%%
\subsection{Bilinear forms
 on $({\mathfrak{g}},K)$-cohomologies via symmetry breaking: General theory for nonvanishing}
%%%%%%%%%%%%%%%%%%%%%%%%%%%%%%%%%%%%%%%%%%%%%%%%%%%%%%%%%%%%%%%%%%%%%%%%%%%%%
For the rest of this chapter,
 we discuss $({\mathfrak{g}},K)$-cohomologies
 via symmetry breaking.  
In this section,
 we deal with a general setting
 where $G \supset G'$ is a pair of real reductive Lie groups.  
We shall define natural bilinear forms
 on $({\mathfrak{g}},K)$-cohomologies
 and $({\mathfrak{g}}',K')$-cohomologies
 via symmetry breaking $G \downarrow G'$, 
 and prove a nonvanishing result
 (Theorem \ref{thm:171556}) in the general setting generalizing a theorem of B.~Sun \cite{ S}.  
%%%%%%%%%%%%%%%%%%%%%%%%%%%%%%%%%
\subsubsection{Pull-back of $({\mathfrak{g}},K)$-cohomologies
 via symmetry breaking}
 %%%%%%%%%%%%%%%%%%%%%%%%%%%%
Let $G$ be a real reductive Lie group,
 and $K$ a maximal compact subgroup.  
We recall that the $({\mathfrak{g}},K)$-cohomology groups
 are the right derived functor of 
\[
   {\operatorname{Hom}}_{{\mathfrak{g}},K}({\mathbb{C}},\ast)
\]
{}from the category of $({\mathfrak{g}},K)$-modules.  
Suppose further 
 that $G'$ is a real reductive subgroup 
 such that $K':=K \cap G'$ is a maximal compact subgroup
 of $G'$.  
We write 
$
   {\mathfrak{g}}_{\mathbb{C}}
  ={\mathfrak{k}}_{\mathbb{C}}+{\mathfrak{p}}_{\mathbb{C}}
$ and 
$
   {\mathfrak{g}}_{\mathbb{C}}'
  ={\mathfrak{k}}_{\mathbb{C}}'+{\mathfrak{p}}_{\mathbb{C}}'
$
 for the complexifications of the corresponding Cartan decompositions.  
In what follows, 
we set 
\[
  d:=\dim G'/K'=\dim_{\mathbb{C}}{\mathfrak{p}}_{\mathbb{C}}'.  
\]
We shall use the Poincar{\'e} duality
 for the subgroup $G'$, 
 which may be disconnected.  
In order to deal with disconnected groups, 
 we consider the natural one-dimensional representation of $K'$
 defined by 
\begin{equation}
\label{eqn:Ksign}
   \chi \colon K' \to GL_{\mathbb{C}}
   (\Exterior^d {\mathfrak{p}}_{\mathbb{C}}')
   \simeq {\mathbb{C}}^{\times}.  
\end{equation}
The differential $d \chi$ is trivial 
 on the Lie algebra ${\mathfrak{k}}'$.  
We extend $\chi$ to a $({\mathfrak{g}}', K')$-module
 by letting ${\mathfrak{g}}'$ act trivially.  
Then we have 
\begin{equation}
\label{eqn:gKtop}
   H^d({\mathfrak{g}}',K';\chi) \simeq {\mathbb{C}}.  
\end{equation}

\begin{example}
\label{ex:chiOn}
For $G'=O(n,1)$, 
 the adjoint action of $K' \simeq O(n) \times O(1)$
 on ${\mathfrak{p}}_{\mathbb{C}}' \simeq {\mathbb{C}}^n$
 gives rise to the one-dimensional representation
\[
   \Exterior^n ({\mathfrak{p}}_{\mathbb{C}}')
   \simeq
   \Exterior^n ({\mathbb{C}}^n) \boxtimes (-1)^n.  
\]
Hence, 
 in terms of the one-dimensional character
\index{A}{1chipmpm@$\chi_{\pm\pm}$, one-dimensional representation of $O(n+1,1)$}
 $\chi_{a b}$ of $O(n,1)$
 defined in \eqref{eqn:chiab}, 
 the $({\mathfrak{g}}', K')$-module $\chi$
 defined in \eqref{eqn:Ksign} is isomorphic to $\chi_{-,(-1)^n}$.  
See also Example \ref{ex:gKchiOn} below.  
\end{example}

Now we recall the Poincar{\'e} duality
 for $({\mathfrak{g}}, K)$-cohomologies of $({\mathfrak{g}}, K)$-modules
 when $G$ is not necessarily connected:
\begin{lemma}
[Poincar{\'e} duality]
\label{lem:Poincare}
Let $\chi$ be the one-dimensional $({\mathfrak{g}}', K')$-module
 as in \eqref{eqn:Ksign}.  
Then for any irreducible $({\mathfrak{g}}', K')$-module $Y$, 
 there is a canonical perfect pairing
\begin{equation}
\label{eqn:Poincare}
   H^j({\mathfrak{g}}', K'; Y) \times H^{d-j}({\mathfrak{g}}', K'; Y^{\vee} \otimes \chi)
\to H^d({\mathfrak{g}}', K'; \chi) \simeq {\mathbb{C}}
\end{equation}
for all $j \in {\mathbb{N}}$.  
\end{lemma}

\begin{proof}
See  \cite[Cor.~3.6]{KV}
 (see also \cite[Chap.~I, Sect.~1]{BW}
 when $K$ is connected).  
\end{proof}

We use the terminology 
\index{B}{symmetrybreakingoperators@symmetry breaking operators|textbf}
 \lq\lq{symmetry breaking operator}\rq\rq\
 also in the category
 of $({\mathfrak{g}},K)$-modules, 
 when  we are given a pair $({\mathfrak{g}},K)$ and $({\mathfrak{g}}',K')$
 such that ${\mathfrak{g}} \supset {\mathfrak{g}}'$ and $K \supset K'$. 
We prove the following.  
\begin{proposition}
\label{prop:171543}
Let $X$ be a $({\mathfrak{g}}, K)$-module,
 $Y$ a $({\mathfrak{g}}', K')$-module,
 and $Y^{\vee}$ the contragredient $({\mathfrak{g}}', K')$-module
 of $Y$.  
Suppose $T\colon X \to Y$ is a $({\mathfrak{g}}', K')$-homomorphism, 
 where we regard the $({\mathfrak{g}}, K)$-module $X$
 as a $({\mathfrak{g}}', K')$ by restriction.  
Then the symmetry breaking operator $T$ induces a canonical homomorphism
\begin{equation}
\label{eqn:cohmap}
T_{\ast} \colon H^j({\mathfrak{g}}, K; X) \to H^j({\mathfrak{g}}', K'; Y)
\end{equation}
and a canonical bilinear form
\begin{equation}
\label{eqn:cohbi}
   B_T \colon 
   H^j({\mathfrak{g}}, K; X) \times H^{d-j}({\mathfrak{g}}', K'; Y^{\vee} \otimes \chi)
   \to 
  {\mathbb{C}}
\end{equation}
for all $j \in {\mathbb{N}}$.
\end{proposition}

\begin{proof}
The $({\mathfrak{g}}, K)$-module $X$ is viewed 
 as a $({\mathfrak{g}}', K')$-module
 by restriction.  
Then the map of pairs 
$
   ({\mathfrak{g}}', K') \hookrightarrow ({\mathfrak{g}}, K)
$
 induces natural homomorphisms
\[
   H^j({\mathfrak{g}}, K; X) \to H^j({\mathfrak{g}}', K'; X)
\quad
\text{for all $j \in {\mathbb{N}}$.}
\]



On the other hand,
 since $T \colon X \to Y$ is a $({\mathfrak{g}}', K')$-homomorphism,
 it induces natural homomorphisms
\[
   H^j({\mathfrak{g}}', K'; X) \to H^j({\mathfrak{g}}', K'; Y)
\quad
\text{for all $j \in {\mathbb{N}}$.}
\]
Composing these two maps,
 we get the homomorphisms
 \eqref{eqn:cohmap}.  



In turn, 
 combining the morphism \eqref{eqn:cohmap}
 with 
\index{B}{Poincareduality@Poincar{\'e} duality}
 the Poincar{\'e} duality in \eqref{eqn:Poincare}
 in Lemma \ref{lem:Poincare}, 
 we get the bilinear map $B_T$
 as desired.  
\end{proof}



%%%%%%%%%%%%%%%%%%%%%%%%%%%%%%%%%
\subsubsection{Nonvanishing of pull-back of $({\mathfrak{g}}, K)$-cohomologies
 of $A_{{\mathfrak{q}}}$ via symmetry breaking}
%%%%%%%%%%%%%%%%%%%%%%%%%%%%%

Retain the setting where $(G,G')$ is a pair of real reductive Lie groups.  
In this subsection, 
 we discuss a nonvanishing result 
 for morphisms between $({\mathfrak{g}}, K)$-cohomologies
 and $({\mathfrak{g}}', K')$-cohomologies
 under certain assumption 
 on the 
\index{B}{KspectrumKprime@$(K,K')$-spectrum}
$(K,K')$-spectrum
 of the symmetry breaking operator,
 see Theorem \ref{thm:171556} and Remark \ref{rem:cohKK} below.  



In order to formulate a nonvanishing theorem, 
 we begin with a setup 
 for finite-dimensional representations
 of compact Lie groups.  
Let $U$ be a $K$-module,
 $U'$ a $K'$-module,
 and $\varphi \colon U \to U'$ a $K'$-homomorphism.  
Via the inclusion map ${\mathfrak{p}}' \hookrightarrow {\mathfrak{p}}$, 
 the composition
 of the following two morphisms
\[
   {\operatorname{Hom}}_K(\Exterior^j {\mathfrak{p}}_{\mathbb{C}},U)
  \to 
   {\operatorname{Hom}}_{K'}(\Exterior^j {\mathfrak{p}}_{\mathbb{C}},U')
  \to
  {\operatorname{Hom}}_{K'}(\Exterior^j {\mathfrak{p}}_{\mathbb{C}}',U')
\]
 induces natural homomorphisms
\begin{equation}
\label{eqn:Upp}
  \varphi_{\ast} \colon 
  {\operatorname{Hom}}_K(\Exterior^j {\mathfrak{p}}_{\mathbb{C}},U)
  \to 
  {\operatorname{Hom}}_{K'}(\Exterior^j {\mathfrak{p}}_{\mathbb{C}}',U')
\end{equation}
for all $j \in {\mathbb{N}}$.  
\begin{definition}
\label{def:pnonvan}
A $K'$-homomorphism $\varphi$ is said to be ${\mathfrak{p}}$-{\it{nonvanishing
 at degree}} $j$
 if the induced morphism $\varphi_{\ast}$ in \eqref{eqn:Upp} is nonzero.  
\end{definition}



By a theorem of Vogan--Zuckerman \cite{VZ}
 every irreducible representation of $G$
 with nontrivial $({\mathfrak{g}},K)$-cohomology is equivalent
 to the representation, 
 to be denoted usually by $A_{\mathfrak{q}}$ 
 for some $\theta$-stable parabolic subalgebra ${\mathfrak {q}}$.  
Here $A_{\mathfrak{q}}$ is a $({\mathfrak {g}},K)$-module cohomologically
 induced from the trivial one-dimensional representation
 of the Levi subgroup $L=N_{G}({\mathfrak{q}}):=\{g \in G : {\operatorname{Ad}}(g){\mathfrak{q}}={\mathfrak{q}}\}$.  
Suppose ${\mathfrak{q}}={\mathfrak{l}}_{\mathbb{C}}+{\mathfrak{u}}$
 and ${\mathfrak{q}}'={\mathfrak{l}}_{\mathbb{C}}'+{\mathfrak{u}}'$
 be $\theta$-stable parabolic subalgebras
 of ${\mathfrak{g}}_{\mathbb{C}}$ and ${\mathfrak{g}}_{\mathbb{C}}'$, 
 respectively.  
In general, 
 we do not assume an inclusive relation
 of ${\mathfrak{q}}$ and ${\mathfrak{q}}'$.  
We shall work
 with a symmetry breaking operator $T \colon X \to Y$, 
 where $X$ is a $({\mathfrak{g}},K)$-module
 $A_{\mathfrak{q}}$ and $Y$ is a $({\mathfrak{g}}',K')$-module
 $A_{\mathfrak{q}'}$.  
We note that $Y$ contains a unique minimal $K'$-type,
 say $\mu'$.  
Let $Y'$ be the $K'$-submodule
 containing all the remaining $K'$-types in $Y$, 
 and 
\[
  {\operatorname{pr}} \colon Y \to \mu'
\] 
 be the first projection
 of the direct sum decomposition
 $Y= \mu' \oplus Y'$.  

\begin{theorem}
\label{thm:171556}
Let $T \colon X \to Y$ be a $({\mathfrak{g}}',K')$-homomorphism, 
 where $X$ is a $({\mathfrak{g}},K)$-module $A_{\mathfrak{q}}$
 and $Y$ is a $({\mathfrak{g}}',K')$-module $A_{\mathfrak{q}'}$.  
Let $U$ be the representation space
 of the minimal $K$-type $\mu$ in $X$, 
 and $U'$ that of the minimal $K'$-type $\mu'$ in $Y$.  
We define a $K'$-homomorphism by 
\begin{equation}
\label{eqn:phiT}
   \varphi_T := {\operatorname{pr}} \circ T|_{U}
   \colon U \to U'.  
\end{equation}
\begin{enumerate}
\item[{\rm{(1)}}]
If $\varphi_T$ is zero, 
 then the homomorphisms 
$
  T_{\ast} \colon 
  H^j({\mathfrak{g}},K;X) \to H^j({\mathfrak{g}}',K';Y)
$
 (see \eqref{eqn:cohmap})
 and the bilinear form $B_T$
 (see \eqref{eqn:cohbi})
 vanish for all degrees $j \in {\mathbb{N}}$.   
\item[{\rm{(2)}}]
If $\varphi_T$ is ${\mathfrak{p}}$-nonvanishing
 at degree $j$, 
 then $T_{\ast}$ and the bilinear forms $B_T$ are nonzero
  for this degree $j$.  
\end{enumerate}
\end{theorem}

\begin{proof}
[Proof of Theorem \ref{thm:171556}]
By Vogan--Zuckerman \cite[Cor.~3.7 and Prop.~3.2]{VZ}, 
 we have natural isomorphisms:
\begin{equation}
\label{eqn:gKp}
   {\operatorname{Hom}}_{K}(\Exterior^j {\mathfrak{p}}_{\mathbb{C}},U)
  \overset \sim \to
  {\operatorname{Hom}}_{K}(\Exterior^j {\mathfrak{p}}_{\mathbb{C}},X)
  \overset \sim \to 
  H^j ({\mathfrak{g}}, K; A_{\mathfrak{q}}).  
\end{equation}
By the definition \eqref{eqn:cohmap} of $T_{\ast}$
 in Proposition \ref{prop:171543}
 and $\varphi_{\ast}$ (see \eqref{eqn:Upp}), 
 the following diagram commutes:
\begin{alignat*}{5}
   &{\operatorname{Hom}}_{K}(\Exterior^j {\mathfrak{p}}_{\mathbb{C}},U)
   &&\overset \sim \to \,\,
   &&{\operatorname{Hom}}_{K}(\Exterior^j {\mathfrak{p}}_{\mathbb{C}},X)
   &&\overset \sim \to 
   &&H^j ({\mathfrak{g}}, K; X)
\\
   &\qquad (T|_U)_{\ast} \downarrow
   &&
   && \circlearrowright
   &&
   &&\qquad \downarrow T_{\ast}
\\
   &{\operatorname{Hom}}_{K'}(\Exterior^j {\mathfrak{p}}_{\mathbb{C}}',T(U))
  &&\subset
  &&{\operatorname{Hom}}_{K'}(\Exterior^j {\mathfrak{p}}_{\mathbb{C}}',Y)
  &&\overset \sim \to 
  &&H^j ({\mathfrak{g}}', K'; Y).  
\end{alignat*}



Since ${\operatorname{Hom}}_{K'}(\Exterior^j {\mathfrak{p}}_{\mathbb{C}}',Y') = \{0\}$
 for all $j$
 where $Y = \mu' \oplus Y'$ is the decomposition
 as a $K'$-module as before,
 we obtain the following commutative diagram
 by replacing $(T|_U)_{\ast}$ with $(\varphi_T)_{\ast}$:
\begin{alignat*}{5}
   &{\operatorname{Hom}}_{K}(\Exterior^j {\mathfrak{p}}_{\mathbb{C}},U)
   &&\overset \sim \to \,\,
   &&{\operatorname{Hom}}_{K}(\Exterior^j {\mathfrak{p}}_{\mathbb{C}},X)
   &&\overset\sim\to
   &&H^j ({\mathfrak{g}}, K; X)
\\
   &\qquad (\varphi_T)_{\ast} \downarrow
   &&
   && \circlearrowright
   &&
   &&\qquad \downarrow T_{\ast}
\\
   &{\operatorname{Hom}}_{K'}(\Exterior^j {\mathfrak{p}}_{\mathbb{C}}',U')
  &&\overset \sim \to
  &&{\operatorname{Hom}}_{K'}(\Exterior^j {\mathfrak{p}}_{\mathbb{C}}',Y)
  &&\overset\sim\to
  &&H^j ({\mathfrak{g}}', K'; Y).  
\end{alignat*}
Hence $T_{\ast}$ is a nonzero map
 if and only if $(\varphi_T)_{\ast}$ is nonzero.  
Since the bilinear map \eqref{eqn:Poincare} is a perfect pairing, 
 we conclude Theorem \ref{thm:171556}.  
\end{proof}

\begin{remark}
\label{rem:cohKK}
\begin{enumerate}
\item[{\rm{(1)}}]
The nonvanishing assumption of $\varphi_T$
 in the first statement of Theorem \ref{thm:171556}
 can be reformulated as the nonvanishing
 of the 
\index{B}{KspectrumKprime@$(K,K')$-spectrum}
 $(K,K')$-spectrum
 (see Section \ref{subsec:Kspec})
 of the symmetry breaking operator $T$
at $(\mu,\mu')$.  
\item[{\rm{(2)}}]
The verification of the ${\mathfrak{p}}$-vanishing assumption
 of $\varphi_T$ in the second statement of Theorem \ref{thm:171556} 
 reduces to a computation
 of finite-dimensional representations
 of compact Lie groups $K$ and $K'$.  
\item[{\rm{(3)}}]
If we set $R:=\dim_{\mathbb{C}}({\mathfrak{u}} \cap {\mathfrak{p}}_{\mathbb{C}})$
 and $R':=\dim_{\mathbb{C}}({\mathfrak{u}}'\cap {\mathfrak{p}}_{\mathbb{C}}')$,
 then the isomorphisms \cite[Cor.~3.7]{VZ} show
\begin{alignat*}{3}
&{\operatorname{Hom}}_{K}(\Exterior^j {\mathfrak{p}}_{\mathbb{C}},\mu)
   &&\simeq \,\,
   &&{\operatorname{Hom}}_{L \cap K}(\Exterior^{j-R} ({\mathfrak{l}}_{\mathbb{C}} \cap {\mathfrak{p}}_{\mathbb{C}}),{\mathbb{C}}), 
\\
  &{\operatorname{Hom}}_{K'}(\Exterior^j {\mathfrak{p}}_{\mathbb{C}}',\mu')
   &&\simeq
   &&{\operatorname{Hom}}_{L' \cap K'}(\Exterior^{j-R'}
   ({\mathfrak{l}}_{\mathbb{C}}' \cap {\mathfrak{p}}_{\mathbb{C}}'),{\mathbb{C}}).  
\end{alignat*}
\end{enumerate}
\end{remark}

%%%%%%%%%%%%%%%%%%%%%%%%
\subsection{Nonvanishing bilinear forms on $(\mathfrak{g}, K)$-cohomologies
 via symmetry breaking for $(G,G')=(O(n+1,1),O(n,1))$}
\label{subsec:cohgkex}
%%%%%%%%%%%%%%%%%%%%%%
%%%%%%%%%%%%%%%%%%%%%%%%%%%%%%%%%%%%%%%%%%%%%%%%%%%%%%%%%%%%%%%%%%%
\subsubsection{Nonvanishing theorem for $O(n+1,1) \downarrow O(n,1)$}
\label{subsec:cohbi}
%%%%%%%%%%%%%%%%%%%%%%%%%%%%%%%%%%%%%%%%%%%%%%%%%%%%%%%%%%%%%%%%%%%
In this section,
 we apply the general result
 (Theorem \ref{thm:171556})
 to the pair $(G,G')=(O(n+1,1),O(n,1))$.  



In Proposition \ref{prop:gKq} in Appendix I, 
 we shall see
 that if $\Pi$ is an irreducible unitary representation
 of $G=O(n+1,1)$
 with $H^{\ast}({\mathfrak{g}},K;\Pi_K) \ne \{0\}$, 
 then the smooth representation $\Pi^{\infty}$ must be isomorphic
 to $\Pi_{\ell, \delta}$ defined in \eqref{eqn:Pild}
 for some $0 \le \ell \le n+1$ and $\delta \in \{\pm\}$.  
Thus, 
 we shall apply Theorem \ref{thm:171556}
 to the representations $\Pi_{\ell, \delta}$ of $G$
 and similar representations $\pi_{m,\varepsilon}$ 
 of the subgroup $G'=O(n,1)$.  



In what follows,
 by abuse of notation,
 we denote an admissible  smooth representation and its underlying $({\mathfrak{g}},K)$-module by the same letter 
 when we discuss their $({\mathfrak{g}},K)$-cohomologies. 



.
\begin{theorem}
\label{thm:gKOn}
Let $(G,G')=(O(n+1,1),O(n,1))$, 
 $0 \le i \le n$, 
 and $\delta \in \{ \pm \}$.  
Let $T:=A_{i,i}$ be the symmetry breaking operator 
 $\Pi_{i,\delta} \to \pi_{i,\delta}$
 given  in Proposition \ref{prop:AiiAq}.  
\begin{enumerate}
\item[{\rm{(1)}}]
$T$ induces bilinear forms
\[
   B_T \colon 
   H^j({\mathfrak{g}}, K; \Pi_{i,\delta}) 
   \times 
   H^{n-j}({\mathfrak{g}}', K'; \pi_{n-i,(-1)^n \delta})
   \to {\mathbb{C}}
\quad
\text{for all $j$.}
\]
\item[{\rm{(2)}}]
The bilinear form $B_T$ is nonzero
 if and only if $j=i$ and $\delta=(-1)^i$.  
\end{enumerate}
\end{theorem}

\begin{remark}
A similar theorem was proved 
 by B.~Sun \cite{S}
 for the $(\mathfrak{g},K)$-cohomology with nontrivial coefficients
 of a tempered representation
 of the  pair $(GL(n,\mathbb R), GL(n-1,\mathbb R))$. 
\end{remark}


We begin with the computation of the $({\mathfrak{g}},K)$-cohomologies
 of the irreducible representation $\Pi_{\ell, \delta}$ of $G=O(n+1,1)$.  
\begin{lemma}
\label{lem:172145}
Suppose $0 \le \ell \le n+1$, 
 $j \in {\mathbb{N}}$, 
 and $\delta \in \{\pm\}$. 
Then 
\[
   H^j({\mathfrak{g}},K; \Pi_{\ell, \delta})
  =
\begin{cases}
{\mathbb{C}} \quad &\text{if $j=\ell$ and $\delta=(-1)^{\ell}$, }
\\
\{0\} &\text{otherwise.}
\end{cases}
\]
\end{lemma}

In view of Theorem \ref{thm:LNM20} (4), 
 we have:
\begin{example}
\label{ex:gKchiOn}
For $G'=O(n,1)$, 
 we have $\pi_{n,(-1)^n} \simeq \chi_{-,(-1)^n}$ from 
 Theorem \ref{thm:LNM20} (4).  
In turn,
 the assertion 
$  
   H^{n}({\mathfrak{g}}', K'; \chi_{-,(-1)^n})
   \simeq
  {\mathbb{C}} 
$ from Lemma \ref{lem:172145} 
 corresponds to the equation \eqref{eqn:gKtop}
 by Example \ref{ex:chiOn}.  
\end{example}

By Proposition \ref{prop:161655} in Appendix I, 
 Lemma \ref{lem:172145} may be reformulated
 in terms of the cohomologically induced representations
\[
  (A_{{\mathfrak{q}}_i})_{a b}
  =
  A_{{\mathfrak{q}}_i} \otimes \chi_{a b}
  \simeq
  {\mathcal{R}}_{{\mathfrak{q}}_i}^{S_i} (\chi_{a b} \otimes {\mathbb{C}}_{\rho({\mathfrak{u}}_i)})
\]
 (see Section \ref{subsec:Aqgeneral} for notation)
 as follows:
\begin{lemma}
\label{lem:1721456}
Suppose $0 \le i \le [\frac{n+1}{2}]$ and $j \in {\mathbb{N}}$.  
Then we have
\begin{alignat*}{5}
H^j({\mathfrak{g}}, K;(A_{\mathfrak{q}_i})_{++})
=& {\mathbb{C}}\quad
&&\text{if $j=i$} 
&&\in 2{\mathbb{N}};
&&=\{0\}\quad
\text{otherwise,}
\\
H^j({\mathfrak{g}}, K;(A_{\mathfrak{q}_i})_{+-})
=& {\mathbb{C}}\quad
&&\text{if $j=i$}
&&\in 2{\mathbb{N}}+1;
&&=\{0\}\quad
\text{otherwise,}
\\
H^j({\mathfrak{g}}, K;(A_{\mathfrak{q}_i})_{-+})
=& {\mathbb{C}}\quad
&&\text{if $j=n+1-i$} 
&&\in 2{\mathbb{N}};
&&=\{0\}\quad
\text{otherwise,}
\\
H^j({\mathfrak{g}}, K;(A_{\mathfrak{q}_i})_{--})
=& {\mathbb{C}}\quad
&&\text{if $j=n+1-i$} 
&&\in 2{\mathbb{N}}+1;
&&=\{0\}\quad
\text{otherwise.}
\end{alignat*}
\end{lemma}


\begin{proof}
[Proof of Lemma \ref{lem:1721456}]
We recall from Theorem \ref{thm:LNM20} (3)
 (see also Proposition \ref{prop:161655} in Appendix I)
 that the irreducible $G$-module
$\Pi_{i,\delta}$
 contains $\mub(i,\delta) \simeq \Exterior^i ({\mathbb{C}}^{n+1}) \boxtimes \delta$
 as its minimal $K$-type.  
By \cite{VZ}, 
 we have then a natural isomorphism
\[
   {\operatorname{Hom}}_K
   (\Exterior^j {\mathfrak{p}}_{\mathbb{C}}, 
    \mub (i,\delta))
   \simeq
   H^j({\mathfrak{g}}, K; \Pi_{i,\delta}).  
\]


On the other hand, 
 the adjoint action of $K=O(n+1) \times O(1)$
 on ${\mathfrak{p}}_{\mathbb{C}} \simeq {\mathbb{C}}^{n+1}$
 gives rise to the $j$-th exterior tensor representation
\[
 \Exterior^j({\mathfrak{p}}_{\mathbb{C}}) 
 \simeq 
  \Exterior^j({\mathbb{C}}^{n+1}) \boxtimes (-1)^j.  
\]
Now the lemma follows.
\end{proof}


\begin{lemma}
\label{lem:phiT}
Let $\varphi_T$ be the $K'$-homomorphism
 defined in \eqref{eqn:phiT}
 for the symmetry breaking operator
 $T \colon \Pi_{i,\delta} \to \pi_{i,\delta}$
 in Theorem \ref{thm:gKOn}.  
Then $\varphi_T$ is ${\mathfrak{p}}$-nonvanishing
 at degree $j$ (Definition \ref{def:pnonvan})
 if and only if $j=i$ and $\delta=(-1)^i$.  
\end{lemma}
\begin{proof}
Similarly to the $G$-module $\Pi_{i,\delta}$, 
  the $G'$-module
$\pi_{i,\delta}$ contains 
$
   \mub(i,\delta)' \simeq \Exterior^i ({\mathbb{C}}^n) \boxtimes \delta
$
 as its minimal $K$-type.  
Then $\varphi_T$ in Theorem \ref{thm:171556} 
 amounts to a nonzero multiple
 of the projection 
 (see \eqref{eqn:Tii1}), 
\[
  \pr ii \colon 
  \Exterior^i({\mathbb{C}}^{n+1}) \boxtimes \delta 
  \to 
  \Exterior^i({\mathbb{C}}^{n}) \boxtimes \delta.  
\]
Then $(\varphi_T)_{\ast}$ is a nonzero multiple
 of the natural map from 
\begin{align*}
& {\operatorname{Hom}}_{O(n+1) \times O(1)}
  (\Exterior^j({\mathbb{C}}^{n+1})\boxtimes (-1)^j, \Exterior^i({\mathbb{C}}^{n+1})\boxtimes \delta)
\\
\intertext{to}
& {\operatorname{Hom}}_{O(n) \times O(1)}
  (\Exterior^j({\mathbb{C}}^{n})\boxtimes (-1)^j, \Exterior^i({\mathbb{C}}^{n})\boxtimes \delta)
\end{align*}
induced by the projection $\pr i i$.  
Now the lemma is clear.  
\end{proof}
We are ready to apply the general result (Theorem \ref{thm:171556})
 to prove Theorem \ref{thm:gKOn}.  

\begin{proof}
[Proof of Theorem \ref{thm:gKOn}]
By Example \ref{ex:chiOn}, 
 we have an isomorphism
 $\chi \simeq \chi_{-, (-1)^n}$ as $({\mathfrak{g}}',K')$-modules.  
Then it follows from Theorem \ref{thm:LNM20} (5) and (6) 
 that there are natural $G'$-isomorphisms:
\[
  \pi_{i,\delta}^{\vee} \otimes \chi_{-,(-1)^n}
  \simeq
  \pi_{i,\delta} \otimes \chi_{-,(-1)^n} \simeq \pi_{n-i,(-1)^n \delta}.  
\]
Thus Theorem \ref{thm:gKOn} (1) follows from Proposition \ref{prop:171543}.  
It then follows from Lemma \ref{lem:phiT}
 that Theorem \ref{thm:gKOn} (2) holds
 as a special case of Theorem \ref{thm:171556}.  
\end{proof}



In Proposition \ref{prop:161655}, 
 we shall see that the underlying $({\mathfrak{g}}',K')$-module 
 of $\pi_{n-i, (-1)^n\delta}$ is isomorphic to 
 $(A_{{\mathfrak{q}}_i'})_{-,(-1)^n\delta}$
 if $0 \le i \le [\frac n 2]$. 
The symmetry breaking operator 
$
   A_{i,i} \colon \Pi_{i,\delta} \to \pi_{i,\delta}$
 given in Proposition \ref{prop:AiiAq}
 induces a $(\mathfrak{g}', K')$-homomorphism $(A_{{\mathfrak{q}}_i})_{+,\delta} \to (A_{{\mathfrak{q}}_i'})_{+,\delta}$.

\begin{corollary}
\label{cor:12.14}
If  $0 \le 2i \le n$, 
 then the symmetry breaking operator $A_{i,i}\colon \Pi_{i,\delta} \to \pi_{i,\delta}$
 induces bilinear forms
\[
   H^j({\mathfrak{g}}, K; (A_{{\mathfrak{q}}_i})_{+,\delta})
  \times   
   H^{n-j}({\mathfrak{g}}', K'; (A_{{\mathfrak{q}}_i'})_{-,(-1)^n\delta})
  \to {\mathbb{C}}
\]
and linear maps
\[
  H^j({\mathfrak{g}}, K; (A_{{\mathfrak{q}}_i})_{+,\delta})
  \to  
   H^{j}({\mathfrak{g}}', K'; (A_{{\mathfrak{q}}_i'})_{+,\delta})
\]
for all $j$. 
They are nontrivial if and only if $j=i$ and $\delta=(-1)^i$.
\end{corollary}


Composing the symmetry breaking operators we deduce the following.  

\begin{corollary}
\label{cor:12.14.com}
If $0 \le 2i \le n$ and $H=O(n+1-i,1)$, 
 then the composition of the symmetry  breaking operators
 induces a linear map
\[
  H^{j}({\mathfrak{g}}, K; (A_{{\mathfrak{q}}_i})_{+,\delta})
  \to  
   H^{j}({\mathfrak{h}}, K\cap H';  (A_{{\mathfrak{q}}_i \cap \mathfrak{h}})_{+,\delta})
 \quad \text{for all } j. 
\]
It is nontrivial if and only if $j=i$ and $\delta=(-1)^{n+1-i}$.  
\end{corollary}


\begin{remark}
\label{rem:TongWang}
Y.~Tong and S.~P.~Wang \cite{TW} considered representations of $SO(n+1,1)$
 with nontrivial $({\mathfrak{g}},K)$-cohomology which are  $SO(n-i) \times SO(n+1-i,1)$-distinguished. Independently 
  S.~Kudla and J.~Millson \cite{KM2}
 considered representations
 of $O(n+1,1)$ with nontrivial $({\mathfrak{g}},K)$-cohomology
 which are  $O(n-i) \times O(n+1-i,1)$-distinguished. 
Since $O(n-i)$ commutes with $O(n-i+1,1)$,
 we have an action of $O(n-i)$
 on ${\operatorname{Hom}}_{O(n-i+1,1)}(\Pi_{i,\delta},{\mathbb C})$
 and ${\operatorname{Hom}}_{O(n-i+1,1)}(\Pi_{i,\delta},{\mathbb C})^{O(n-i)}$
 is isomorphic to ${\operatorname{Hom}}_{O(n-i)\times O(n-i+1,1)}(\Pi_{i,\delta},{\mathbb C})$.
 By results in \cite{KM2} this induces a nontrivial linear map
 on the $({\mathfrak{g}},K)$-cohomology.  
\end{remark}


%%%%%%%%%%%%%%%%%%%%%%%
\subsubsection{Special Cycles}
%%%%%%%%%%%%%%%%%%%%%%
Geometric, topological and arithmetic properties of hyperbolic symmetric spaces $X_\Gamma=\Gamma\backslash O(n+1,1)/K$ for a discrete subgroup $\Gamma $ have been studied extensively using representation theoretic and geometric techniques. 
See for example \cite{BC, BMM} and references therein. 
If $X_\Gamma$ is compact, 
 then the Matsushima--Murakami formula 
 (\cite[Chap.~VII, Thm.~3.2]{BW})
shows
\[
   H^{\ast}(X_\Gamma, \bC) 
 \simeq \bigoplus_{\Pi \in \widehat{G}}  m(\Gamma, {\Pi})
   H^{\ast}({\mathfrak g},K; \Pi_K), 
\]
where
\index{A}{Ghat@$\widehat G$, unitary dual|textbf}
 $\widehat G$ is the set of equivalence classes 
 of irreducible unitary representations of $G$ 
({\it{i.e.}}, the {\it{unitary dual}} of $G$), 
 and we set for $\Pi \in \widehat G$
\[ 
  m(\Gamma,\Pi) := \dim_{\mathbb{C}} {\operatorname{Hom}}_G (\Pi,L^2(\Gamma\backslash G)). 
\]
By abuse of notation,
 we shall omit the subscript $K$ in the underlying $({\mathfrak g},K)$-module
 $\Pi_K$ of $\Pi$
 when we discuss its $({\mathfrak g},K)$-cohomologies.  



In Proposition \ref{prop:gKq} in Appendix I,
 we shall show that every irreducible unitary representations with nontrivial $(\mathfrak{g},K)$-cohomology is isomorphic to a representations ${\Pi}_{i,\delta}$
 for some $i$ and $\delta \in \{\pm\}$, 
 see also Theorem \ref{thm:LNM20} (9). 
Thus
\[
   H^{\ast}(X_\Gamma, \bC) 
   = \bigoplus_{i,\delta} m(\Gamma, {\Pi}_{i,\delta})
   H^{\ast}({\mathfrak{g}},K; {\Pi}_{i,\delta}).  
\]



To obtain arithmetic information about the cohomology and the homology of $X_\Gamma $,
special cycles, 
{\it{i.e.,}} orbits of subgroups $H \subset G$ on $X_\Gamma$, 
 and their homology classes are frequently used. 
Suppose $0 \le 2i \le n+1$.  
We let 
\[
 G_i=O(n+1-i,1), 
\quad
 K_i :=K \cap G_i \simeq O(n+1-i) \times O(1), 
\]
 and $X_i$ be the Riemannian symmetric space $G_i/K_i$. 
Let $b_i :=n+1-i$, 
 the dimension of $X_i$.  
We set $\delta= (-1)^{n+1-i}$.  
By Corollary \ref{cor:12.14.com}  there  exists a nontrivial linear map  
$A^{n+1-i,n+1-i}:$
\[  
    H^{n+1-i}({\mathfrak{g}}, K; (A_{{\mathfrak{q}_{n+1-i}}})_{+,\delta})
  \to  
   H^{n+1-i}({\mathfrak{g}}_i, K_i; (A_{{\mathfrak{q}_{n+1-i}} \cap (\mathfrak{g}_i)_{\mathbb{C}}})_{+,\delta}). 
\]
Note that 
$(A_{{\mathfrak{q}}_{n+1-i} \cap ({\mathfrak{g}}_i)_{\mathbb{C}}})_{+,\delta}$ is one-dimensional
 and the image of $A^{n+1-i,n+1-i}$ is isomorphic to
\[
 {\operatorname{Hom}}_{K_i}
 (\Exterior^{n+1-i}({\mathfrak{p}}_{\mathbb{C}}\cap({\mathfrak{g}}_i)_{\mathbb{C}}),\chi_{+,\delta})
 \simeq  
 {\operatorname{Hom}}_{K_i}
 (\Exterior^{n+1-i}({{\mathbb C}^{n+1-i})\boxtimes {\bf{1}},
 {\mathbf 1}}). \]



Since the nonzero element of 
\[
   {\operatorname{Hom}}_{K_i}
   (\Exterior^{n+1-i}({{\mathbb C}^{n+1-i})\boxtimes {\bf{1}}, {\mathbf 1}})
\]
 gives a volume form on the symmetric space $X_i=G_i/K_i$, 
 this suggests
 that the homology classes defined
 by the orbits of $O(n+1-i,1)$
 for $0 \le 2i \leq n+1$ on $X_\Gamma$ are related
 to the contribution of $H^{n+1-i}({\mathfrak{g}},K;\Pi_{i,\delta})$
 to the cohomology of $X_\Gamma$.  
The work of S.~Kudla and J.~Millson confirms this.
We sketch their results   following the exposition in \cite{KM2, KM-I, KM-II}.



We have an embedding 
\[ 
     \iota_{X_i }: X_i \hookrightarrow X=G/K.   
\]
            
We fix an orientation of $X$ and $X_i$
 which is invariant under the connected component of $G$ respectively $G_i$.    Let $\mathbb A$ be the adels of the real number field $\mathbb K.$  
Then
\[ 
     X_{\mathbb A} = X \otimes G(\mathbb A_f) 
\]  
is the adelic  symmetric space. 
We set $G^+:= \prod SO_0(p,q)$ where we take the product over all real places
 of $\mathbb K $ and 
 $G^+(\mathbb K) := G(\mathbb K) \cap G^+G(\mathbb A_f)$.  
Then
 \[ H^*(G(\mathbb K) \backslash G(\mathbb A_f);\bC) = H^*({\mathfrak
 g},K; C^\infty(G(\mathbb Q)\backslash G(\mathbb A) ))\]
and 
\[
   H^*(G^+(\mathbb K)\backslash G(\mathbb A) /KK_f;\bC)
  =H^*(X_\mathbb A; \mathbb C)^{K_f}.  
\]
The cohomology here is the de Rham cohomology if  $X_\mathbb A$ is compact,
 otherwise the cohomology with compact support.


Following the exposition and notation in \cite[Sect.~2]{KM-II}
 we have an inclusion
\[
 \iota_{X_i} : \ X_i \times G_i(\mathbb A_f) \rightarrow X \times G({\mathbb A}_f)
\]
which is equivariant under the right action of $G({\mathbb A}_f)$. 
For $g \in G({\mathbb A}_f)$ we obtain a special cycle   
\[
     X_{i,g} = X_i G_i(\mathbb A_f) / (gK_f g^{-1} \cap G_i({\mathbb A}_f)).
\]
Consider the subspace $SX_i(X_{\mathbb A})$  spanned by special cycles
 in the homology group $H_i(X_{\mathbb A})$. 

 
\medskip
We now assume that all but one factor of $G_\infty $ is compact and thus that $X_{\mathbb A}/K_f$ is compact.
Using the theta correspondence,
 S.~Kudla and J.~Millson show that there exist a subgroup $K_f$ and  nontrivial homomorphisms 
\[ 
\Psi \colon H^{b_i}({\mathfrak g},K; \Pi) 
            \rightarrow H^{b_i}(X_{\mathbb A}/K_f;\bC) \subset H^{b_i}(X_{\mathbb A})
\]
for some irreducible representation $\Pi$ of $G$.  

\medskip
Using integration, 
 S.~Kudla and J.~Millson \cite{KM2}, \cite[Thm.~7.1]{KM-II} prove the following:
\begin{theorem} 
\label{theorem:kudla+millson}
There exists a nontrivial pairing
\[ 
\Psi(H^{n+1-i}({\mathfrak g},K; \Pi) ) \times SX_i(X_{\mathbb A})  \rightarrow {\mathbb C}. 
\]
\end{theorem}


\begin{remark}
\begin{enumerate}
\item[{\rm{(1)}}]
As we see in Theorem \ref{thm:LNM20} (9), 
 Lemma \ref{lem:1721456}
 and Proposition \ref{prop:161655}, 
 the irreducible representation $\Pi$ of $G$
 with $H^{n+1-i}({\mathfrak{g}},K; \Pi) \ne \{0\}$
 must be of the form
\[
   \Pi \simeq \Pi_{n+1-i, (-1)^{n+1-i}}, 
\]
namely,
 $\Pi_K \simeq (A_{\mathfrak{q}_i})_{-,(-1)^{n+1-i}}$.  
\item[{\rm{(2)}}]
The nontrivial pairing in Theorem \ref{theorem:kudla+millson}
 defines an $O(n+1-i,1)$-invariant linear functional
 on the irreducible $G$-module $\Pi_{n+1-i, (-1)^{n+1-i}}$
 which is nontrivial on the minimal $K$-type. 
\end{enumerate}
\end{remark} 
