\section{Conclusions}
\label{sec:conclusions}
\begin{spacing}{0.9}
This paper introduces a new methodology to quantify the application resilience.
Different from the traditional random fault injection, 
our methodology employs a direct measurement of fault masking events inherent in applications. 
%Our methodology eliminates the randomness and 
%provides a repeatable and deterministic quantification for data objects in applications.
Based on our methodology, we introduce a new metric, a series of techniques, and a tool to analyze and identify error masking.
%the resilience of data objects in applications.
%We recognize the fundamental reasons why an application can tolerate faults by classifying fault masking events.
%The error masking is one of the fundamental reasons that account for the application resilience.
We hope that our methodology and tool can make the quantification of application resilience a common practice for evaluating applications in the future. 
\end{spacing}

\begin{comment}
Based on our methodology, we introduce a new metric, a series of techniques, and 
a tool to analyze and identify error masking.  
We apply our model to study fault tolerance of various scientific applications, and demonstrate
tangible benefits of using a model-driven approach to direct fault tolerance mechanisms
for large-scale high performance computing systems.
We hope that our methodology and tool can make the quantification of application resilience
a common practice for evaluating applications in the future. 
\end{comment}

%\vspace{-10pt}