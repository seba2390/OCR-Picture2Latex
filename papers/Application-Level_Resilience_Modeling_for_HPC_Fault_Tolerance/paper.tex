% This is "sig-alternate.tex" V2.1 April 2013
% This file should be compiled with V2.5 of "sig-alternate.cls" May 2012
%
% This example file demonstrates the use of the 'sig-alternate.cls'
% V2.5 LaTeX2e document class file. It is for those submitting
% articles to ACM Conference Proceedings WHO DO NOT WISH TO
% STRICTLY ADHERE TO THE SIGS (PUBS-BOARD-ENDORSED) STYLE.
% The 'sig-alternate.cls' file will produce a similar-looking,
% albeit, 'tighter' paper resulting in, invariably, fewer pages.
%
% ----------------------------------------------------------------------------------------------------------------
% This .tex file (and associated .cls V2.5) produces:
%       1) The Permission Statement
%       2) The Conference (location) Info information
%       3) The Copyright Line with ACM data
%       4) NO page numbers
%
% as against the acm_proc_article-sp.cls file which
% DOES NOT produce 1) thru' 3) above.
%
% Using 'sig-alternate.cls' you have control, however, from within
% the source .tex file, over both the CopyrightYear
% (defaulted to 200X) and the ACM Copyright Data
% (defaulted to X-XXXXX-XX-X/XX/XX).
% e.g.
% \CopyrightYear{2007} will cause 2007 to appear in the copyright line.
% \crdata{0-12345-67-8/90/12} will cause 0-12345-67-8/90/12 to appear in the copyright line.
%
% ---------------------------------------------------------------------------------------------------------------
% This .tex source is an example which *does* use
% the .bib file (from which the .bbl file % is produced).
% REMEMBER HOWEVER: After having produced the .bbl file,
% and prior to final submission, you *NEED* to 'insert'
% your .bbl file into your source .tex file so as to provide
% ONE 'self-contained' source file.
%
% ================= IF YOU HAVE QUESTIONS =======================
% Questions regarding the SIGS styles, SIGS policies and
% procedures, Conferences etc. should be sent to
% Adrienne Griscti (griscti@acm.org)
%
% Technical questions _only_ to
% Gerald Murray (murray@hq.acm.org)
% ===============================================================
%
% For tracking purposes - this is V2.0 - May 2012

%\documentclass[conference]{IEEEtran}
%\documentclass{sig-alternate-05-2015}
\documentclass[sigconf]{acmart}
%\documentclass[preprint,nocopyrightspace,numbers]{sigplanconf}

%\usepackage{xifthen}
%\usepackage{multirow}
%\usepackage{graphicx}
%\usepackage[cmex10]{amsmath}
\usepackage{verbatim}
%\usepackage{url}
\usepackage{courier}
\usepackage{amsmath}
\usepackage{listings}
\lstset{language=C++}
\lstset{breaklines}
\lstset{extendedchars=false}
\definecolor{mygreen}{rgb}{0,0.6,0}
\lstset{                        %Settings for listings package.
  language=[ANSI]{C},
  numbers=left,
  numberstyle=\small\color{gray},
  backgroundcolor=\color{white},
  basicstyle=\texttt{}\footnotesize,
  breakatwhitespace=false,
  breaklines=true,
  captionpos=b,
  commentstyle=\bfseries\upshape\color{mygreen},
  directivestyle=\color{blue},
  extendedchars=false,
  frame=tb
  framerule=0pt,
  keywordstyle=\color{blue}\bfseries,
  morekeywords={*,define,*,include...},
  numbersep=5pt,
  rulesepcolor=\color{red!20!green!20!blue!20},
  showspaces=false,
  showstringspaces=false,
  showtabs=false,
  stepnumber=1,
  stringstyle=\color{purple},
  tabsize=4,
  title=\lstname
}
%\usepackage[usenames, dvipsnames]{color}
%\makeatletter
%\patchcmd{\maketitle}{\@copyrightspace}{}{}{}
%\makeatother
%\usepackage[hidelinks]{hyperref} 
%\usepackage{etoolbox}
%\usepackage{array}% for fancier tabular
%\usepackage{mathtools}

%\apptocmd{\thebibliography}{\scriptsize}{}{}

%\makeatletter
%\patchcmd{\maketitle}{\@copyrightspace}{}{}{}
%\makeatother

\usepackage{subcaption}
%\PassOptionsToPackage{hyphens}{url}
%\usepackage[hyphens]{url}
%\usepackage{hyperref}
%\usepackage{booktabs}
%\usepackage{graphicx}
%\usepackage{caption}
\usepackage[font=small,labelfont=bf]{caption}
\usepackage{setspace}

\setcopyright{none}
%\acmConference[]{}{}{}
%\acmDOI{}
%\acmISBN{}
%\acmPrice{}
%\acmYear{}
\settopmatter{printacmref=false, printfolios=false}
\renewcommand\footnotetextcopyrightpermission[1]{} % removes footnote with conference information in first column
\pagestyle{plain} % removes running headers


\begin{document}

% Copyright
%\setcopyright{acmcopyright}
%\setcopyright{acmlicensed}
%\setcopyright{rightsretained}
%\setcopyright{usgov}
%\setcopyright{usgovmixed}
%\setcopyright{cagov}
%\setcopyright{cagovmixed}


% DOI
%\doi{10.475/123_4}

% ISBN
%\isbn{123-4567-24-567/08/06}

%Conference
%\conferenceinfo{PLDI '13}{June 16--19, 2013, Seattle, WA, USA}

%\acmPrice{\$15.00}

%
% --- Author Metadata here ---
%\conferenceinfo{WOODSTOCK}{'97 El Paso, Texas USA}
%\CopyrightYear{2007} % Allows default copyright year (20XX) to be over-ridden - IF NEED BE.
%\crdata{0-12345-67-8/90/01}  % Allows default copyright data (0-89791-88-6/97/05) to be over-ridden - IF NEED BE.
% --- End of Author Metadata ---

%\title{Application-Level Modeling to Analyze Application Resiliency to Transient Faults}
\title{Application-Level Resilience Modeling for HPC Fault Tolerance}
%\titlenote{A full version of this paper is available as
%\textit{Author's Guide to Preparing ACM SIG Proceedings Using
%\LaTeX$2_\epsilon$\ and BibTeX} at
%\texttt{www.acm.org/eaddress.htm}}}
%
\author{Luanzheng Guo}
\affiliation{
\institution{UC Merced}
}
\email{lguo4@ucmerced.edu}

\author{Hanlin He}
\affiliation{
\institution{UC Merced}
}
\email{hhe3@ucmerced.edu}

\author{Dong Li}
\affiliation{
\institution{UC Merced}
}
\email{dli35@ucmerced.edu}
%Luanzheng Guo
%\email{lguo4@ucmerced.edu}
%Hanlin He   Dong Li \\
%\texorpdfstring{lguo4, hhe3, dli35 @ucmerced.edu}

\begin{abstract}
Understanding the application resilience in the presence of faults is critical to address the HPC resilience challenge. Currently we largely rely on random fault injection (RFI) to quantify the application resilience.  However, RFI provides little information on how fault tolerance happens, and RFI results are often not deterministic due to its random nature.
In this paper, we introduce a new methodology to quantify the application resilience. Our methodology is based on the observation that at the application level, the application resilience to faults is due to the application-level fault masking. The application-level fault masking happens because of application-inherent semantics and program constructs. Based on this observation, we analyze application execution information and use a data-oriented approach to model the application resilience. 
We use our model to study how and why HPC applications can (or cannot) tolerate faults. We demonstrate tangible benefits of using the model to direct fault tolerance mechanisms.



%Understanding the application resilience in the presence of faults is critical to address the resilience challenge in high performance computing systems. 
%Currently we largely rely on random fault injection to quantify the application resilience.  
%However, because of the random nature of fault injection, 
%it is often difficult to bound the fault injection accuracy. 
%there is no quantitative guidance on how many fault injection tests should be performed and where they should happen;
%the random fault injection tests are also not deterministically repeatable.
%In this paper, we introduce a fundamentally new methodology to quantify the application resilience. Our methodology is based on the observation that 
%at the application level, the application resilience to faults is due to the application-level fault masking. The application-level 
%fault masking happens because of application-inherent semantics and program constructs. Based on this observation, we 
%classify common fault masking, and introduce a series of techniques to analyze application execution information and capture the application-level 
%fault masking to model the application resilience. 
%We apply our model to study the fault resilience of representative computation algorithms and two
%scientific applications. Our results reveal how and why these algorithms and applications can (or cannot) tolerate faults.
%Furthermore, we apply our model to optimize fault tolerance for a scientific application, and demonstrate tangible benefits of using the model to direct fault tolerance mechanisms for large-scale high performance computing systems.
\vspace{-10pt}
\end{abstract}

\maketitle

\begin{comment}
%
% The code below should be generated by the tool at
% http://dl.acm.org/ccs.cfm
% Please copy and paste the code instead of the example below. 
%
\begin{CCSXML}
<ccs2012>
 <concept>
  <concept_id>10010520.10010553.10010562</concept_id>
  <concept_desc>Computer systems organization~Embedded systems</concept_desc>
  <concept_significance>500</concept_significance>
 </concept>
 <concept>
  <concept_id>10010520.10010575.10010755</concept_id>
  <concept_desc>Computer systems organization~Redundancy</concept_desc>
  <concept_significance>300</concept_significance>
 </concept>
 <concept>
  <concept_id>10010520.10010553.10010554</concept_id>
  <concept_desc>Computer systems organization~Robotics</concept_desc>
  <concept_significance>100</concept_significance>
 </concept>
 <concept>
  <concept_id>10003033.10003083.10003095</concept_id>
  <concept_desc>Networks~Network reliability</concept_desc>
  <concept_significance>100</concept_significance>
 </concept>
</ccs2012>  
\end{CCSXML}

\ccsdesc[500]{Computer systems organization~Embedded systems}
\ccsdesc[300]{Computer systems organization~Redundancy}
\ccsdesc{Computer systems organization~Robotics}
\ccsdesc[100]{Networks~Network reliability}

%
% End generated code
%

%
%  Use this command to print the description
%
%\printccsdesc
\end{comment}


% We no longer use \terms command
%\terms{Theory}

%\keywords{ACM proceedings; \LaTeX; text tagging}

\input intro   %1page
\input background    %1page
\input modeling	 %3.5pages
\input impl	 %0.5page
\input evaluation   %3pages
\input case_study
\input related_work  %0.5page
\input conclusions   %0.5page


% The following two commands are all you need in the
% initial runs of your .tex file to
% produce the bibliography for the citations in your paper.
%\begin{spacing}{0.9}
\bibliographystyle{abbrv}
\bibliography{li,li_sc16_resilience_modeling}  % sigproc.bib is the name of the Bibliography in this case
% You must have a proper ".bib" file
%  and remember to run:
% latex bibtex latex latex
% to resolve all references
%\end{spacing}

\end{document}
