%%%%%%%%%%%%%%%%%%%%%%%%%%%%%%%%%%%%%%%%%%%%%%%%%%%%%%%%%%%%%%%%
\section{Results}
\label{s:results}
%%%%%%%%%%%%%%%%%%%%%%%%%%%%%%%%%%%%%%%%%%%%%%%%%%%%%%%%%%%%%%%%
We now discuss the main results obtained with our implementation in
\VTK{} of the previously described new hypertree grid filters.
%%%%%%%%%%%%%%%%%%%%%%%%%%%%%%%%
\subsection{Adaptive Surface Filter}
\label{s:adaptive-surface-filter-results}
%%%%%%%%%%%%%%%%%%%%%%%%%%%%%%%%
We begin with the adaptive geometry extraction filter introduced
in~\S\ref{s:adaptive-surface-filter} and implemented 
in~\VTK{}, exploring the $2$ and $3$-dimensional cases as well as
the two possible branch factor values.
%%%%%%%%%%%%%%%%
\begin{figure}[h!]
\centering
\begin{minipage}[t]{0.48\columnwidth}
\centering
\vspace{0pt}
\includegraphics[width=0.99\columnwidth]{HyperTreeGridBinary2DAdaptiveDataSetSurfaceFilter.png}\\
(a)
\end{minipage}
\hfil
\begin{minipage}[t]{0.48\columnwidth}
\centering
\vspace{0pt}
\includegraphics[width=0.99\columnwidth]{HyperTreeGridBinary2DAdaptiveDataSetSurfaceFilterMaterial.png}\\
(b)
\end{minipage}
\caption{Visualizations of a part of the data sets produced by the
application of the \texttt{AdaptiveDataSetSurface} filter to a
$2$-dimensional, binary hypertree grid with 6 root cells, without (a)
or with (b) a material mask; only cells that are visible based on
camera and object position are generated.}
\label{fig:HyperTreeGridBinary2DAdaptiveDataSetSurfaceFilter}
\end{figure}
%%%%%%%%%%%%%%%%

Figure~\ref{fig:HyperTreeGridBinary2DAdaptiveDataSetSurfaceFilter}
displays renderings obtained by applying the
\texttt{AdaptiveDataSetSurface} filter to a 2-dimensional binary hypertree grid,
with a $2\times3$ layout of root cells, to which is attached a single
attribute field filled with the cell depths, either without (a) or
with (b) a non-empty material mask attached to it.
We acknowledge that the adaptive surface filter did indeed
compute a reduced number of polygons, with respect to the non-adaptive
\texttt{Geometry} filter: specifically, where the entire
$2$-dimensional AMR mesh has $75$ leaf cells, the output of
\texttt{AdaptiveDataSetSurface} contains only $67$ quadrangles.
When a material mask is present, the corresponding
numbers are $62$ and $54$, respectively.
In the latter case, the gain is inferior to that of the former, as a
result of the fact that most culling is due to the masking rather than
to the camera position relative to object.

%%%%%%%%%%%%%%%%
\begin{figure}[ht!]
\centering
\begin{minipage}[t]{0.99\columnwidth}
\includegraphics[width=\columnwidth]{amrzoom.png}
\end{minipage}
\caption{Successive close-up renderings of a $2$-dimensional AMR
grid obtained with the \texttt{AdaptiveDataSetSurface} filter.}
\label{fig:AMRZoom}
\end{figure}
%%%%%%%%%%%%%%%%
The elimination of those polygons that are outside the rendering
window or, more precisely, their non-creation, is further illustrated
in Figure~\ref{fig:AMRZoom} with the example of a $2$-dimensional
hypertree grid data set resulting from a computational fluid dynamics
simulation.

%%%%%%%%%%%%%%%%
\begin{figure}[ht!]
\centering
\begin{minipage}[t]{0.99\columnwidth}
\includegraphics[width=\columnwidth]{levels.png}
\end{minipage}
\caption{Renderings of a $2$-dimensional AMR grid obtained with the
\texttt{AdaptiveDataSetSurface} filter, showing the effect of a
decreasing pixelization threshold.} 
\label{fig:levels}
\end{figure}
%%%%%%%%%%%%%%%%
The sub-pixel culling effect of the adaptive surface filter is
illustrated in Figure~\ref{fig:levels}, again with a $2$-dimensional
hypertree grid, where we can see the effect of adjusting the scale
factor~$s$ in~\eqref{eq:level-max}:
\begin{eqnarray*}
s_1 > s_2 & \Leftrightarrow & - \log{s_1} < - \log{s_2} \\
 & \Leftrightarrow & \delta_{\max}(w,z,s_1,f) <
\delta_{\max}(w,z,s_2,f)\ .
\end{eqnarray*}
In other words, increasing the scale factor results in decreasing the
maximum allowable depth in the DFS traversals of each of the
constituting hypertrees, meaning that more down-sampling will occur.
The same effect obviously occurs when instead of increasing $s$,
either the window size $w$ is decreased, or the view is zoomed out.
%%%%%%%%%%%%%%%%
\begin{figure}[h!]
\centering
\begin{minipage}[t]{0.48\columnwidth}
\centering
\vspace{0pt}
\includegraphics[width=0.99\columnwidth]{HyperTreeGridTernary3DAdaptiveDataSetSurfaceFilter.png}
\end{minipage}
\hfil
\begin{minipage}[t]{0.48\columnwidth}
\centering
\vspace{0pt}
\includegraphics[width=0.99\columnwidth]{HyperTreeGridTernary3DAdaptiveDataSetSurfaceFilterMaterial.png}
\end{minipage}
\caption{Renderings of the outside surface of $3$-dimensional, ternary
hypertree grid with 18 root cells, using the
\texttt{AdaptiveDataSetSurface}, without (left) or with (right) a non-empty
material mask.}
\label{fig:HyperTreeGridTernary3DAdaptiveDataSetSurfaceFilter}
\end{figure}
%%%%%%%%%%%%%%%%

A $3$-dimensional, ternary set of test cases is now used,
with a $3\times3\times2$ layout of roots to further illustrate our point.
We illustrate this in
Figure~\ref{fig:HyperTreeGridTernary3DAdaptiveDataSetSurfaceFilter},
without or with the presence of a non-empty material mask.
We note in particular that in this case, the visualizations obtained
using the novel \texttt{AdaptiveDataSetSurface} filter are indeed
identical to those obtained with the \texttt{Geometry} filter,
cf.~\cite{harel:17} for details; in addition, no culling in the
surface geometry occurred as expected.
%%%%%%%%%%%%%%%%%%%%%%%%%%%%%%%%
\subsection{Selection Extraction Filters}
%%%%%%%%%%%%%%%%%%%%%%%%%%%%%%%%
%%%%%%%%%%%%%%%%
\begin{figure}[h!]
\centering
\begin{minipage}[t]{0.48\columnwidth}
\centering
\vspace{0pt}
\includegraphics[width=0.99\columnwidth]{HyperTreeGridBinary2DExtractSelectedIds.png}\\
(a)
\includegraphics[width=0.99\columnwidth]{HyperTreeGridBinary2DExtractSelectedIdsMaterial.png}\\
(c)
\end{minipage}
\hfil
\begin{minipage}[t]{0.48\columnwidth}
\centering
\vspace{0pt}
\includegraphics[width=0.99\columnwidth]{HyperTreeGridBinary2DExtractSelectedLocations.png}\\
(b)
\includegraphics[width=0.99\columnwidth]{HyperTreeGridBinary2DExtractSelectedLocationsMaterial.png}\\
(d)
\end{minipage}
\caption{Visualizations of a part of the unstructured data sets
produced by the application of the \texttt{ExtractSelectedIds} (a\&c) and
\texttt{ExtractSelectedLocations} (b\&d) filters to a
$2$-dimensional, binary hypertree grid with 6 root cells, without
(a\&b) or with (c\&d) a material mask.}
\label{fig:HyperTreeGridBinary2DDataSetSurfaceFilter}
\end{figure}
%%%%%%%%%%%%%%%%

We continue with the selection extraction filter introduced
in~\S\ref{s:extract-selected-filters} and also implemented in~\VTK{},
exploring the $2$ and $3$-dimensional cases as well as the two
possible branch factor values.
Using the same 2-dimensional binary hypertree grid with a $2\times3$
layout of root cells, as in~\S\ref{s:adaptive-surface-filter-results}
Figure~\ref{fig:HyperTreeGridBinary2DDataSetSurfaceFilter} displays
the renderings obtained with the selection extraction filters applied
to this hypertree grid, either with or without a non-empty
material mask attached to it.

%%%%%%%%%%%%%%%%
\begin{figure}[h!]
\centering
\begin{minipage}[t]{0.48\columnwidth}
\centering
\vspace{0pt}
\includegraphics[width=0.99\columnwidth]{HyperTreeGridTernary3DExtractSelectedIds.png}\\
(a)\\[2mm]
\includegraphics[width=0.99\columnwidth]{HyperTreeGridTernary3DExtractSelectedIdsMaterial.png}\\
(c)
\end{minipage}
\hfil
\begin{minipage}[t]{0.48\columnwidth}
\centering
\vspace{0pt}
\includegraphics[width=0.99\columnwidth]{HyperTreeGridTernary3DExtractSelectedLocations.png}\\
(b)\\[2mm]
\includegraphics[width=0.99\columnwidth]{HyperTreeGridTernary3DExtractSelectedLocationsMaterial.png}\\
(d)
\end{minipage}
\caption{Visualizations of a part of the unstructured data sets
produced by the application of the \texttt{ExtractSelectedIds} (a\&c) and
\texttt{ExtractSelectedLocations} (b\&d) filters to a
$3$-dimensional, ternary hypertree grid with 18 root cells, without
(a\&b) or with (c\&d) a material mask.}
\label{fig:HyperTreeGridTernary3DExtractSelectedFilter}
\end{figure}
%%%%%%%%%%%%%%%%

Finally, the same $3$-dimensional, ternary hypertree grid with a
$3\times3\times2$ layout of root cells as
in~\S\ref{s:adaptive-surface-filter-results} is used to further
illustrate our point in
Figure~\ref{fig:HyperTreeGridTernary3DExtractSelectedFilter}.
Furthermore, these baseline comparisons are done with the presence of
a non-empty material mask, as well as without.

The interested reader is invited to draw parallels between
corresponding $2$ and $3$ dimensional sub-figures, and to inspect the
contents of the test harness we implemented for all existing hypertree
grid filters, across different dimensions and branching factors:
8 new individual tests are available and can be either executed
as they are, or modified and experimented with at will.
Of particular interest is the behavior of these
filters in the presence of a material mask, allowing for the
selection of masked out cells.
This can be seen in both $2$ and $3$-dimensional cases and is not a
bug, but a desired feature.
