%%%%%%%%%%%%%%%%%%%%%%%%%%%%%%%%%%%%%%%%%%%%%%%%%%%%%%%%%%%%%%%%
\section{Conclusion}
\label{s:conclusion}
%%%%%%%%%%%%%%%%%%%%%%%%%%%%%%%%%%%%%%%%%%%%%%%%%%%%%%%%%%%%%%%%
In this article we provided a description of the new native hypertree
grid filters which we developed, in order to further advance the
vision which we had outlined in~\cite{harel:17}.
One of these two filters implements an efficient version of a commonly
used visualization technique, whereas the two other ones already
belong to the field of data analysis: indeed, these offer the ability
to drill into simulation data sets, and extract from them those parts
that are considered relevant either topologically or geometrically.
These three new filters have been incorporated into the set of native
hypertree grid filters, and we are planning to release them as part of
the open-source \VTK{} library.

For future work, we are considering on one hand to extend the
\texttt{AdaptiveDataSetSurface} filter to support level-of-detailed
rendering for the $3$-dimensional case as well.
Note that we first focused on the $2$-dimensional implementation
first, because this is the case where performance gains are the
highest in relative terms.
This is a result of the fact that, by definition, no culling occurs
when showing the outside surface of planar data set, whereas in
dimension~$3$ only boundary cells can produce surface rectangles.
On the other hand, another selection extraction filter, using
attribute values instead of data set geometry or topology, would be a
useful addition the current data analysis capabilities of the tool set.
