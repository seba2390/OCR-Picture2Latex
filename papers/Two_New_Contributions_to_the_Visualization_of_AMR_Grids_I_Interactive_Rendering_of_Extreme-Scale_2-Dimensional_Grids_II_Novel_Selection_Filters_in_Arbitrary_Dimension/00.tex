%%%%%%%%%%%%%%%%%%%%%%%%%%%%%%%%%%%%%%%%%%%%%%%%%%%%%%%%%%%%%%%%
\section{Introduction}
\label{s:introduction}
%%%%%%%%%%%%%%%%%%%%%%%%%%%%%%%%%%%%%%%%%%%%%%%%%%%%%%%%%%%%%%%%
This short article is a sequel to~\cite{harel:17}, where we 
presented the first systematic treatment of the problems posed by the
visualization and analysis of large-scale, parallel, tree-based
adaptive mesh refinement (AMR) simulations on an Eulerian grid.
In that previous article, we proposed a novel data object for the
Visualization Toolkit (\VTK)~\cite{avila:10}, able to
support all conceivable types of rectilinear, tree-based AMR data sets
not only produced by today's simulation software, but also by what is
foreseeable of tomorrow's extreme-scale simulations.

%%%%%%%%%%%%%%%%
\begin{figure}[h!]
\centering
\includegraphics[width=0.9\columnwidth]{AMR_Explose2.png}
\caption{Visualization of the different levels of a $2$-dimensional
AMR grid, whose depth levels are raised by increasing elevations in
order to highlight the hierarchic nature of the object.}
\label{fig:AMR_Explose2}
\end{figure}
%%%%%%%%%%%%%%%%
The concrete result of this earlier work is a novel version
of the \texttt{vtkHyperTreeGrid} data object, differing in many key
aspects from the initial incarnation which we proposed
in~\cite{carrard:imr21}.
The novel version also includes a dozen of visualizations algorithms
(\emph{filters}) operating natively upon the new data object.
An example visualization obtained with this new object is shown in
Figure~\ref{fig:AMR_Explose2}.
Note that this set of filters includes, in particular, a conversion
algorithm to transform a tree-based AMR grid into a fully explicit,
unstructured mesh that can, albeit extremely 
inefficiently, be processed by other visualization techniques not
currently included in the native, hypertree grid filter.
One key aspect of this improved object is that it relies on a
hierarchy of templated traversal objects, called \emph{cursors} and
\emph{supercursors}, allowing for the easy creation of new filters
while retaining the intrinsic performance of our method, in terms
of both execution speed and memory footprint.

Meanwhile, we acknowledged that $2$-dimensional AMR visualization
could be especially challenging, as a result of its requirement that
all leaf cells be rendered.
This constraint makes the interactivity of the visualization process
decrease as input data object size increases.
It is important to note that this problem is further compounded by the
enhanced efficiency, in terms of memory footprint, of our hypertree
grid model.
This elicits a new situation where rendering has become
the bottleneck for the target platforms.

As a result, our next stated goal was to optimize rendering speed, in
order to maintain interactivity when visualizing the largest
possible tree grids that can be contained in memory.
We thus set out to address this urgent need, for which the lack of
existing solution was hindering the AMR visualization and analysis
workflow.
The results of this work are thus presented hereafter.

In addition, we also explain here how we expanded the set of hypertree
grid filters with two novel data selection and extraction filters: one
based on location (i.e., a geometric property), the other on element
Id (i.e., a topological property).
This further supports, in particular, our claim that the new hypertree
grid framework offers enough flexibility and convenience that
algorithms not initially envisioned can be readily added and deployed.

