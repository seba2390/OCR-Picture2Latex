We use two widely used ECG datasets collected at the Physikalisch-Technische Bundesanstalt (PTB-XL)~\cite{wag20} and Chapman University and Shaoxing People's Hospital (CUSPH)~\cite{zhe20} for our experiments. We selected 15,012 ECGs (9,527 normal, 5,485 samples with any myocardial infarction) from the PTB-XL dataset and 9,094 (7,314 normal, 1,780 atrial fibrillation) samples from the CUSPH dataset. Both datasets were split into stratified training, validation, and test sets with a ratio of 7:1:2. ECGT2T was trained using only training samples from the PTB-XL dataset. 
The two asynchronous input leads were sampled by taking a signal in Lead I and then selecting the delayed signal 0.5 seconds later in Lead II. 
% Table~\ref{tab:data} contains additional details regarding the counts and distribution of our data.


\subsection{Quality Assessment}


There are numerous ways to measure the discrepancy between two signals; however, common metrics such as root mean square error may not reflect the quality of generated electrocardiograms. If the original ECG waveform has artifacts such as power line interference, myokymia, and baseline wandering, a generated signal that accurately represents the cardiac condition will perform poorly under conventional metrics. Thus, we evaluated the quality of the ECGT2T outputs by comparing wave points between the original and generated signals.

An electrocardiogram has numerous points of clinical importance, and depending on a given patient's cardiac condition, some of these characteristics may not be visible in their electrocardiogram. However, the R-peak is both almost always present and is used for tasks such as  detecting heart rate and measuring stress level~\cite{sad12}. Therefore, we base our quality assessment with amplitude and positional missing errors for the R-peaks. We run our evaluation on V1 and V5 because of their electrode placement relative to Lead I: V1 is recorded orthogonally, while V5 is recorded in a similar position. We use Neurokit2~\cite{neu21} for R-peak detection on both the original and generated V1 and V5 leads.

Table~\ref{tab:gen_qual} shows the amplitude gaps and the position missing errors for the V1 and V5 leads generated by ECGT2T and ECGS2E for both datasets. For the PTB-XL dataset, the difference in amplitude is between 6.4$\sim$7.3\% and the positional missing error under 10 ms. For the CUSPH dataset the amplitude errors were higher than those in the PTB-XL dataset (7.6$\sim$11.4\% for amplitude and 3.8$\sim$14.4ms for position). We note that both models were only trained with samples from PTB-XL, which may explain this discrepancy.

Figure~\ref{fig:gen-ptb} shows 12-lead ECG samples over a two second window, where the black, red, and blue lines are the signals from the original electrocardiogram, ECGS2E, and ECGT2T respectively. Figure~\ref{fig:ptb-norm} and~\ref{fig:chap-norm} are normal samples from PTB-XL and CUSPH. The generated signals have less noise (seen in aVR and aVL in~\ref{fig:ptb-norm} and V4, V5, and V6 in~\ref{fig:chap-norm}) and baseline wandering (seen in V1, V3, and V6 in~\ref{fig:ptb-norm}). This artifact removal contributes to the amplitude gaps between the original and generated data seen in our quantitative evaluation.

% Baseline Wandering?
%  is a normal sample from the CUSPH dataset. While there are some amplitude errors in the signals generated by ECGS2E, both models show minimal positional errors and removed artifacts in V4, V5, and V6.

Figure~\ref{fig:ptb-case} is a sample with myocardial infarction. ST-segment elevation, T-wave inversion, and abnormal Q-waves are characteristic of the condition and are captured by the generated outputs at Lead II, Lead III, V2, and V3 in this sample with the exception of the ST-segment elevation at the first beat in V3 for both models and the T-Wave inversion in Lead II for ECGS2E. The discrepancy between the ECGT2T and ECGS2E Lead II results is because Lead I and Lead II typically have peaks that point toward the same direction in normal ECGs. However, as ECGS2E takes in only Lead I as input, it initially misses the inversion. In contrast, ECGT2T has two inputs and is therefore unaffected by the discrepancy between Lead I and Lead II.

Figure~\ref{fig:chap-case} is a sample with atrial fibrillation. The condition is characterized by the absence of the P-wave, irregular ventricular rate, and slightly aberrant QRS complexes. In this case, all leads in the original ECG are missing the P-wave, and both ECGT2T and ECGS2E outputs capture this absence across all generated leads. However, while the ECGS2E limb leads have good placement, the amplitude of peaks  stray from the original signal. 
% It seems that ECGS2E has a limitation for the elaborate calibration of the amplitude magnitude.
% %the CUSPH dataset은 부정맥과 관련된 것이기 때문에 비트의 모양보다는 
% % 대부분의 리드에서 pwave가 나타나지 않고 있다. 따라서 pwave를 그려내지 않는 것은 두 모델 모두 동일하게 잘 나타내고 있다. 그러나 ECGS2E는 리드1에 절대적인 영향을 받기 때문에, 리드1과 다른 리드간의 특정 관계를 벗어나게 되는 경우 전혀 다른 신호를 그려내는 단점을 보인다. 

% In this example with absence of the P-wave in all leads, both ECGT2T and ECGS2E output the signals without the P-wave in all leads.
% As the result of ECGS2E in Figure~\ref{fig:ptb-case}, the amplitude magnitude of peaks on ECGS2E strays from the original one though the arrangement of wave points are well matched. 
% Since arrhythmia is relevant to the abnormal wave points, the errors of the amplitude magnitude for the peaks in leads could not affect to find the arrhythmia cases.

% % Multiple signals in the ECGS2E stray from those in the original ECG. The signals of ECGT2T are very similar to the original signals, even though the CUSPH dataset is not used for training.
% In the conclusion for the ECG generation, we identified, in the normal examples, both models draw exquisitely other leads compared with original ones, besides V1 and V5 which are used in the quality assessment. They also have the effect of correction to the artifacts. 
% There are some limitations on the examples with heart diseases. For the purpose of correcting the outputs, the models transform the shape of wave points that can be important clues to determine classes. The error of the amplitude magnitude on case examples is larger than that on normal examples. 
% Finally, as we claimed, the generation performance of ECGT2T referring to two leads outperforms that of ECGS2E using a single lead.


\begin{table}
  \caption{Synthetic quality assessment for the generated R-peaks. \textit{Amp} denotes the amplitude error while \textit{Pos} is the position error.}
  \label{tab:gen_qual}
  \begin{tabular}{cccccc}
    \toprule
    
        {Dataset} & Lead & \multicolumn{2}{c}{V1} & \multicolumn{2}{c}{V5} \\
                            \cmidrule{2-6}
                            & Model & ECGT2T & ECGS2E & ECGT2T & ECGS2E \\ 
        \midrule
        {PTB-XL} & Amp    & 6.4\%      & 6.4\%     &         7.3\%             &      6.7\%                  \\
                                & Pos     & 8.3ms      & 8.2ms     &        2.0ms   &  2.2ms                    \\ 
        \midrule
        {CUSPH}  & Amp    & 11.4\%      & 11.2\%     &     7.6\%&   7.7\%                   \\
                                & Pos     & 12.9ms      & 14.4ms     & 3.8ms   &  4.1ms \\

  \bottomrule
\end{tabular}
\end{table}


\subsection{Classification Performance}

To assess how signals generated by ECGT2T perform in downstream classification tasks, we configure six ECG datasets with different lead combinations: \textit{Original} is the baseline 12-lead ECG, \textit{T2T} is original asynchronous Lead I \& Lead II and ten generated leads, \textit{S2E} is the original Lead I and 11 generated leads, \textit{Two Leads} is composed of asynchronous Lead I \& Lead II only, and \textit{Single Lead} is the set with only Lead I. We train and test six classifiers for the corresponding ECG sets on myocardial infarction and arrhythmia detection tasks. 

We run our classification experiments on a customized ResNet18~\cite{he16} model. The classifier uses residual blocks with one-dimensional convolutions and a single fully connected layer for the output layer, and is trained with focal loss~\cite{lin17} with $\alpha$ and $\gamma$ set to 0.5 and 2, and Adam optimization~\cite{kin14} with learning rate and weight decay set to $1e^{-4}$ and $1e^{-5}$ respectively.


Table~\ref{tab:clf-perform} shows the classification performance by dataset. For myocardial infarction on the PTB-XL dataset, the model using \textit{T2T} outperforms the other models with the exception of the classifier using the original 12-lead electrocardiogram. Additionally, the performance of the model trained on the original Lead I and 11 generated leads (\textit{S2E}) is worse than those of the models using the original asynchronous Lead I and Lead II with no generated leads (\textit{Two Leads}). For atrial fibrillation detection on the CUSPH dataset, the classifiers trained with generated data outperform the models trained with a single lead or two leads. However, the S2E model outperforms T2T classifier, and the spread of performance metric results is much narrower than for the myocardial infarction task. This can be explained in part due to the differences in the two medical conditions: myocardial infarction is detected by the shape change of the heartbeat (which requires multiple leads), while arrhythmia can be diagnosed if there are irregular rhythmic patterns in the waveform, a symptom that can be detected with a single lead.


\begin{table}
\caption{Classification performance. First column indicates the training data source: \textit{Original} is the standard 12-lead ECG, \textit{T2T} is the original asynchronous Lead I \& Lead II with 10 generated leads, \textit{S2E} is the original Lead I with 11 generated leads, \textit{Two Leads} is the original asynchronous Lead I and Lead II with no additional generated leads, and \textit{Single Lead} the original Lead I only.  Parenthesis denote 95\% confidence intervals.}
  \label{tab:clf-perform}
  \begin{tabular}{ccccc}
    \toprule



\multicolumn{1}{c}{} & \multicolumn{2}{c}{Myocardial infarction} & \multicolumn{2}{c}{Arrhythmia}     \\
\midrule
          & AUROC & AUPRC & AUROC & AUPRC \\
         \midrule
Original &
\begin{tabular}[c]{@{}c@{}}
0.960\\(0.95-0.97)\end{tabular} &
\begin{tabular}[c]{@{}c@{}}
0.957\\(0.95-0.96)\end{tabular} &
\begin{tabular}[c]{@{}c@{}}
0.994\\(0.98-0.99)\end{tabular} &
\begin{tabular}[c]{@{}c@{}}
0.972\\(0.96-0.98)\end{tabular} \\
\midrule

T2T &
\begin{tabular}[c]{@{}c@{}}
0.948\\(0.94-0.96)\end{tabular} &
\begin{tabular}[c]{@{}c@{}}
0.945\\(0.94-0.95)\end{tabular} &
\begin{tabular}[c]{@{}c@{}}
0.990\\(0.98-0.99)\end{tabular} &
\begin{tabular}[c]{@{}c@{}}
0.960\\(0.94-0.97)\end{tabular} \\
\midrule

S2E &
\begin{tabular}[c]{@{}c@{}}
0.929\\(0.92-0.94)\end{tabular} &
\begin{tabular}[c]{@{}c@{}}
0.931\\(0.92-0.94)\end{tabular} &
\begin{tabular}[c]{@{}c@{}}
0.993\\(0.99-1.0)\end{tabular} &
\begin{tabular}[c]{@{}c@{}}
0.965\\(0.95-0.98)\end{tabular} \\
\midrule

% SyncTwo &
% \begin{tabular}[c]{@{}c@{}}
% 0.937\\(0.93-0.95)\end{tabular} &
% \begin{tabular}[c]{@{}c@{}}
% 0.935\\(0.92-0.94)\end{tabular} &
% \begin{tabular}[c]{@{}c@{}}
% 0.987\\(0.98-0.99)\end{tabular} &
% \begin{tabular}[c]{@{}c@{}}
% 0.957\\(0.94-0.97)\end{tabular} \\
% \midrule

Two Leads &
\begin{tabular}[c]{@{}c@{}}
0.937\\(0.93-0.95)\end{tabular} &
\begin{tabular}[c]{@{}c@{}}
0.938\\(0.93-0.95)\end{tabular} &
\begin{tabular}[c]{@{}c@{}}
0.989\\(0.97-0.99)\end{tabular} &
\begin{tabular}[c]{@{}c@{}}
0.954\\(0.93-0.96)\end{tabular} \\
\midrule

Single Lead &
\begin{tabular}[c]{@{}c@{}}
0.883\\(0.87-0.89)\end{tabular} &
\begin{tabular}[c]{@{}c@{}}
0.891\\(0.88-0.90)\end{tabular} &
\begin{tabular}[c]{@{}c@{}}
0.989\\(0.98-0.99)\end{tabular} &
\begin{tabular}[c]{@{}c@{}}
0.945\\(0.92-0.96)\end{tabular} \\

  \bottomrule
\end{tabular}
\end{table}




% \begin{table}
%     \caption{Datasets used in this study. Table values are the number of samples used in each dataset. We selected samples with myocardial infarction and normal samples for the PTB-XL dataset and atrial fibrillation and normal samples for the CUSPH dataset.}
%     \label{tab:data}
%     % \vspace{-3mm}
%     \begin{tabular}{cccccc}
%     \toprule
%     {} & Total & Train & Validation & Test & \% Abnormal \\
%     \midrule
%     PTB-XL &15,012&  10,508       &        1,501      &   3,003    &    36.6\%      \\
%     \midrule
%     CUSPH  &9,094&     6,365    &        910      &    1,819    &    19.6\%  \\
%     \bottomrule
% \end{tabular}
% \end{table}






