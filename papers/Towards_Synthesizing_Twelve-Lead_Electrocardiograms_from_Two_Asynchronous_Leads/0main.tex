%%
%% This is file `sample-authordraft.tex',
%% generated with the docstrip utility.
%%
%% The original source files were:
%%
%% samples.dtx  (with options: `authordraft')
%% 
%% IMPORTANT NOTICE:
%% 
%% For the copyright see the source file.
%% 
%% Any modified versions of this file must be renamed
%% with new filenames distinct from sample-authordraft.tex.
%% 
%% For distribution of the original source see the terms
%% for copying and modification in the file samples.dtx.
%% 
%% This generated file may be distributed as long as the
%% original source files, as listed above, are part of the
%% same distribution. (The sources need not necessarily be
%% in the same archive or directory.)
%%
%% The first command in your LaTeX source must be the \documentclass command.
\documentclass[sigconf]{acmart}
%% NOTE that a single column version may required for 
%% submission and peer review. This can be done by changing
%% the \doucmentclass[...]{acmart} in this template to 
%% \documentclass[manuscript,screen]{acmart}
%% 
%% To ensure 100% compatibility, please check the white list of
%% approved LaTeX packages to be used with the Master Article Template at
%% https://www.acm.org/publications/taps/whitelist-of-latex-packages 
%% before creating your document. The white list page provides 
%% information on how to submit additional LaTeX packages for 
%% review and adoption.
%% Fonts used in the template cannot be substituted; margin 
%% adjustments are not allowed.

\usepackage{subcaption}
%%
%% \BibTeX command to typeset BibTeX logo in the docs
\AtBeginDocument{%
  \providecommand\BibTeX{{%
    \normalfont B\kern-0.5em{\scshape i\kern-0.25em b}\kern-0.8em\TeX}}}

%% Rights management information.  This information is sent to you
%% when you complete the rights form.  These commands have SAMPLE
%% values in them; it is your responsibility as an author to replace
%% the commands and values with those provided to you when you
%% complete the rights form.
\setcopyright{acmcopyright}
\copyrightyear{2022}
\acmYear{2022}
\acmDOI{XXXXXXX.XXXXXXX}

%% These commands are for a PROCEEDINGS abstract or paper.
\acmConference[Conference acronym 'XX]{Make sure to enter the correct
  conference title from your rights confirmation emai}{June 03--05,
  2018}{Woodstock, NY}
%
%  Uncomment \acmBooktitle if th title of the proceedings is different
%  from ``Proceedings of ...''!
%
%\acmBooktitle{Woodstock '18: ACM Symposium on Neural Gaze Detection,
%  June 03--05, 2018, Woodstock, NY} 
\acmPrice{15.00}
\acmISBN{978-1-4503-XXXX-X/18/06}


%%
%% Submission ID.
%% Use this when submitting an article to a sponsored event. You'll
%% receive a unique submission ID from the organizers
%% of the event, and this ID should be used as the parameter to this command.
%%\acmSubmissionID{123-A56-BU3}

%%
%% The majority of ACM publications use numbered citations and
%% references.  The command \citestyle{authoryear} switches to the
%% "author year" style.
%%
%% If you are preparing content for an event
%% sponsored by ACM SIGGRAPH, you must use the "author year" style of
%% citations and references.
%% Uncommenting
%% the next command will enable that style.
%%\citestyle{acmauthoryear}

%%
%% end of the preamble, start of the body of the document source.
\begin{document}

%%
%% The "title" command has an optional parameter,
%% allowing the author to define a "short title" to be used in page headers.
\title{ECGT2T: Towards Synthesizing Twelve-Lead Electrocardiograms from Two Asynchronous Leads}

%%
%% The "author" command and its associated commands are used to define
%% the authors and their affiliations.
%% Of note is the shared affiliation of the first two authors, and the
%% "authornote" and "authornotemark" commands
%% used to denote shared contribution to the research.
\author{Yong-Yeon Jo}
\email{yy.jo@medicalai.com}
\affiliation{\institution{Medical AI Inc.}
  \city{Seoul}
%   \state{}
  \country{South Korea}
%   \postcode{XXXX-XXXX}
}
\author{Young Sang Choi}
\email{ychoi@ncc.re.kr}
\affiliation{
  \institution{National Cancer Center}
  \city{Goyang-si}
  \state{Gyeonggi-do}
  \country{South Korea}
%   \postcode{XXXX-XXXX}
}


\author{ Jong-Hwan Jang}
\email{jangood1122@medicalai.com}
\affiliation{\institution{Medical AI Inc.}
  \city{Seoul}
%   \state{}
  \country{South Korea}
%   \postcode{XXXX-XXXX}
}

\author{Joon-myoung Kwon}
\email{cto@medicalai.com}
\affiliation{\institution{Medical AI Inc.}
  \city{Seoul}
%   \state{}
  \country{South Korea}
%   \postcode{XXXX-XXXX}
}

%%
%% By default, the full list of authors will be used in the page
%% headers. Often, this list is too long, and will overlap
%% other information printed in the page headers. This command allows
%% the author to define a more concise list
%% of authors' names for this purpose.
% \renewcommand{\shortauthors}{Trovato and Tobin, et al.}

%%
%% The abstract is a short summary of the work to be presented in the
%% article.
\begin{abstract}
An electrocardiogram (ECG) is a non-invasive measurement used to observe the condition of the heart and usually consists of 12 synchronous leads. Wearable devices can record ECGs, but typically provide only a single lead or a couple of asynchronous leads. Signals generated from these devices may be insufficient for accurately diagnosing more complex cardiac conditions.
To bridge this gap, we propose \textsf{ECGT2T}, a deep generative model that synthesizes ten leads from two input leads to simulate a 12-lead ECG. 
Synthesized ECGs had timing and amplitude errors under 15 milliseconds and 10\%, respectively. 
% Compared to the R-peaks of the original waveforms, synthesized ECGs had R-peaks with timing and amplitude errors under 15 milliseconds and 10\%, respectively. 
Experiments on two widely used ECG datasets show classifiers trained on simulated 12-lead ECGs generated with ECGT2T outperformed models trained on one- or two-lead ECGs in detecting myocardial infarction and arrhythmia.
\end{abstract}

%%
%% The code below is generated by the tool at http://dl.acm.org/ccs.cfm.
%% Please copy and paste the code instead of the example below.
%%

\ccsdesc[500]{Applied computing~Health informatics}
\ccsdesc[500]{Human-centered computing~Ubiquitous and mobile devices}


%%
%% Keywords. The author(s) should pick words that accurately describe
%% the work being presented. Separate the keywords with commas.
\keywords{Electrocardiogram; Healthcare; Wearables; Deep learning; 
Generative adversarial networks}

%% A "teaser" image appears between the author and affiliation
%% information and the body of the document, and typically spans the
%% page.
% \begin{teaserfigure}
%   \includegraphics[width=\textwidth]{sampleteaser}
%   \caption{Seattle Mariners at Spring Training, 2010.}
%   \Description{Enjoying the baseball game from the third-base
%   seats. Ichiro Suzuki preparing to bat.}
%   \label{fig:teaser}
% \end{teaserfigure}

%%
%% This command processes the author and affiliation and title
%% information and builds the first part of the formatted document.
\maketitle

\section{Introduction}
\section{Introduction}

Capturing the motion of the human body pose has great values in widespread applications, such as movement analysis, human-computer interaction, films making, digital avatar animation, and virtual reality. 
Traditional marker-based motion capture system can acquire accurate movement information of humans, but is only applicable to limited scenes due to the time-consuming fitting process and prohibitively expensive costs. 
In contrast, markerless motion capture based on RGB image and video processing algorithms is a promising alternative that has attracted numerous research in the fields of deep learning and computer vision. 
%encoded with prior knowledge of realistic human pose and shape deforming
Especially, thanks to the parameteric SMPL model~\citep{smpl:loper2015smpl} and various diverse datasets with 3D annotations~\citep{h36m:ionescu2013human3, mpii3d:mehta2017monocular, 3dpw:von2018recovering}, remarkable progress has been made on monocular 3D human pose and shape estimation and motion capture. 
%such as~\citep{hmr:kanazawa2018end, vibe:kocabas2020vibe,  maed:wan2021encoder, romp:sun2021monocular, spin:kolotouros2019learning, pare:kocabas2021pare}.




Existing regression-based human mesh recovery methods are actually implicitly based on an assumption that predicting 3d body joint rotations and human shape strongly depends on the given image features. 
The pose and shape parameters are directly estimated from the image feature using MLP regressors. 
Nevertheless, due to the inherent ambiguity, the mapping from the 2D image feature to 3D pose and shape is an ill-posed problem.
To achieve accurate pose and shape estimation, it is necessary to initialize the mean pose and shape parameters and use an \textit{iterative residual regression} manner to reduce error. 
Such an end-to-end learning and inference scheme~\citep{hmr:kanazawa2018end} has been proven to be effective in practice, but ignores the temporal information and produces implausible human motions and unsatisfactory pose jitters for video streaming data. 
Video-based methods such as~\citep{hmmr:kanazawa2019learning,vibe:kocabas2020vibe, tcmr:choi2021beyond, mps-net:wei2022capturing} may leverage large-scale motion capture data as priors and exploit temporal information among different frames to penalize implausible motions. 
 They usually enhance singe-frame feature using a temporal encoder and then still use a deep regressor to predict SMPL parameters based on the temporally enhanced image feature,  as shown in the left subfigure of Fig.\ref{fig:temporal_contrast}. 
This scheme, however, is unable to focus on joint-level rotational motion specific to each joint, failing to ensure the temporal consistency of local joints. 
To address these problems, we attempt to understand the human 3D reconstruction from a causal perspective. 
We argue that assuming a still background, the primary causes behind the image pixel changes and human body appearance changes are 1) the motions of 3D joint rotations in human skeletal dynamics and 2) the viewpoint changes of the observer (camera). 
In fact, a prior human body model exists independently of a given specific image. And the 3D relative rotations of all joints (relative to the parent joint) and body shape can be abstracted beyond image pixels and independent of the image contents and observer views. In other words, the joint rotations cannot be ``seen'' and they are image-independent and viewpoint-independent concepts.  %Additionally, the motion of the combined pose is huge and highly complicated, while the rotation motion of each separate joint is more tracable and periodic.

\begin{figure}
	\centering
	\includegraphics[width=0.95\linewidth]{figures/simple_contrast.pdf}
	\caption{\textbf{Left:} Mainstream temporal-based human mesh methods, e.g.~\citep{hmmr:kanazawa2019learning,vibe:kocabas2020vibe, tcmr:choi2021beyond}, adopt a temporal encoder to mix temporal information from past and future frames and then regress the SMPL parameters from the \textit{temporally enhanced feature} for each frame. \textbf{Right:} Our method first acquires tokens of each joint in the time dimension and then separately capture the motion of each joint using a shared temporal encoder.
	}
	\label{fig:temporal_contrast}\vspace{-0.2in}
\end{figure}

Based on the considerations above, we propose a novel 3D human pose and shape estimation model based on \textbf{in}dependent \textbf{t}okens (INT). 
The core idea of the model is to introduce three types of independent tokens that specifically encode the 3D rotation information of every joint, the shape of human body and the information about camera. 
These initialized tokens learn prior knowledge and mutual relationships from large-scale training data, requiring neither an iterative regressor to take mean shape and pose as initial estimate~\citep{hmr:kanazawa2018end, spin:kolotouros2019learning, vibe:kocabas2020vibe, tcmr:choi2021beyond}, nor a kinematic topology decoder defined by human prior knowledge~\citep{maed:wan2021encoder} . 
Given an image as a conditional observation, these tokens are repeatedly updated by interacting with 2D image evidence using a Transformer~\citep{transformer:vaswani2017attention}. 
Finally, they are transformed into the posterior estimates of pose, shape and camera parameters. 
As a consequence, this method of abstracting joint rotation tokens from image pixels can represent the motion state of each joint and establish correlations in time dimension.
 Benefiting from this, we can separately capture the temporal rotational motion of every joint by sending the tokens of each joint at different timestamps to a temporal model.
In comparison to capturing the overall temporal changes in image features and the whole pose, this modeling scheme focuses on capturing separate rotational motions of all joints, which is conducive to maintaining the temporal coherence and rationality of each joint rotation.

We evaluate our model on the challenging 3DPW~\citep{3dpw:von2018recovering} benchmark and Human3.6m~\citep{h36m:ionescu2013human3}. 
Using vanilla ResNet-50 and Transformer architectures, our model obtains 42.0 mm error in PA-MPJPE metric for 3DPW, outperforming all state-of-the-art counterparts with a large margin. The same model obtains 38.4 mm error in PA-MPJPE metric for Human3.6m, which is on par with the state-of-the-art methods. 
Also, the qualitative results show that our model produces accurate pixel-alignment human mesh reconstructions for indoor or in-the-wild images, and shows fewer motion jitters in local joints when processing video data. 
We strongly encourage the readers to see the video results in the supplementary materials for reference and comparison.

 %simple and straightforward, We first and then extend it to capture separate rotational motion of each joint.

% 可视化prior learned pose and shape



\section{Proposed Architecture}
\begin{figure*}[h]
\centering
\includegraphics[width=0.75\textwidth]{fig/0523_arch.pdf}
\caption{ECGT2T model architecture. The model is comprised of style, mapping, generative, and discriminative networks. Each network is built with residual blocks. $L_{adv}$ is the adversarial objective, $L_{rec}$ is the reconstruction objective, $L_{con}$ is the lead consistency objective, and $L_{sty}$ is the style consistency objective.}
\label{fig:arch}
\end{figure*}

Figure~\ref{fig:arch} illustrates the ECGT2T architecture. The model consists of \textit{style} ($S(\cdot)$), \textit{mapping} ($M(\cdot)$), \textit{generative} ($G(\cdot)$), and \textit{discriminative} ($D(\cdot)$) networks. The style network represents the cardiac styles for ten output leads based on the styles of the two input leads. The mapping network, as seen in previous image style transfer models~\cite{kar2019}, generates latent codes for a random variable. The generative network takes any single lead, generates its respective latent code, and reconstructs the leads with a cardiac style. To improve the efficiency of this network, we use an adaptive instance normalization layer~\cite{hua17}. The discriminative network distinguishes whether its inputs are real or not. Each network is built by stacking multiple residual blocks~\cite{he16}.



% % This encodes two cardiac styles for given leads $c_{x_{i}}\in C$ for a corresponding lead using two leads (namely, $c_{x_{i}}=R(x_{I,II})$). 
% It consists of three modules where two reference, an encoding, and multiple lead-specific modules. Each reference network extracts features from I \& II, respectively. 
% The encoding network captures the heart condition representation for all leads from an input concatenated with two features. Each lead-specific network encodes a representation of the heart condition $c_{x_{i}}$ corresponding to lead $x_{i}$.

% \textit{Mapping network $M(\cdot)$}
% It is devised in \cite{kar2019,choi20} for the efficient style translation. We adopt another network to represent the cardiac styles $\Tilde{c}_{x_{i}}\in \Tilde{C}$ for ten respective leads from a random variable $z\in Z$, instead of two given leads ($\Tilde{c}_{x_{i}}=M(z, i)$). It represents the latent code using the multi-layer perceptron, and then conditionally generate $\Tilde{c}_{x_{i}}$ on the corresponding style encoding block. This network makes style network $S(\cdot)$ to be robust by influencing generated styles.


% 

% \textit{Discriminative network $D(\cdot)$} It is the network typically used in GAN~\cite{mir14} which distinguishes whether its inputs are real or not ($ y = D(\Tilde{x_{i}},x_{i})$). It is composed to multiple discriminators as many as the number of leads to be generated.








\subsection{Objectives}

The full objective for ECGT2T is expressed as follows :
\begin{align*}
    \min_{E,H,G} \max_{D} \lambda_{adv} L_{adv} + \lambda_{rec} L_{rec}
    + \lambda_{con} L_{con} + \lambda_{sty} L_{sty} 
\end{align*}
where $\lambda$ and $L$ are the are the hyperparameters and losses for the respective objective.


The \textit{adversarial} objective is the mean of the log probability for original leads $D(x_{i})$ and the mean of the log of the inverted probabilities of the generated samples $D(G(x_{I},c_{i}))$:
\begin{align*}
    L_{adv} = \mathbb{E}_{x_{i}}[\log{D(x_{i})}] + \mathbb{E}_{x_{I},c_{i}}[\log{(1-D(G(x_{I},c_{i})))}]
\end{align*}
where $i$ is randomly selected lead that isn't Lead I or Lead II.

The \textit{reconstruction} objective compares an alternatively chosen Lead I or Lead II with a randomly selected lead $j$ via mean squared error:
\begin{align*}
    L_{rec} = \mathbb{E}_{x_{i},x_{j},c_{j}}[(x_{j} - G(x_{i},c_{j}))^2]
\end{align*}

The \textit{lead consistency} objective ensures that generated leads are consistent regardless of whether Lead I or Lead II was used as input.  For any randomly selected lead $i$ from Lead III to V6, the lead consistency objective is the mean squared error of the generated Lead I $G(x_{I},c_{i})$ and generated Lead II $G(x_{II},c_{i})$:
\begin{align*}
    L_{con} = \mathbb{E}_{x_{I},x_{II},c_{i}}[(G(x_{I},c_{i}) - G(x_{II},c_{i}))^2]
\end{align*}

To account for the diversity in  cardiac rhythm and beat shapes, we include the \textit{style consistency} objective as a form of regularization. This objective is the mean absolute error of the derived style $S(x_{I,II}, i)$ and mapping network output $M(z, i)$:
\begin{align*}
    L_{sty} = \mathbb{E}_{i,z,x_{I,II}}[||(M(z, i) - S(x_{I,II}, i))||_{1}]
\end{align*}
where $z$ is a random variable and $i$ is randomly selected from Lead III to V6.

\subsection{Training Procedure}
Adam optimization~\cite{kin14} with the learning rate of style, mapping, generative, and discriminative networks set to $3e^{-4}$, $1e^{-4}$, $3e^{-4}$, and $1e^{-4}$, respectively, and weight decay for all networks set to $1e^{-4}$. $\lambda_{rec}$ is set to 2, while the $\lambda$s for the other objectives are set to 1. All style latent vectors were set to size 512. The models used in this study are implemented with PyTorch and executed on a server equipped with Intel Xeon Silver 4210, 256 GB memory, and four NVIDIA RTX 3090 GPUs with 24 GB VRAM. We trained generative models (ECGT2T and ECGS2E) over seven days, and selected the model with the lowest loss.



\begin{figure*}[h]
\centering
\begin{subfigure}{.49\textwidth}
    \includegraphics[width=0.99\textwidth]{fig/10072Norm.pdf}
    
    \caption{Normal sample from the PTB-XL dataset}\label{fig:ptb-norm}
\end{subfigure} 
\begin{subfigure}{.49\textwidth}
    \includegraphics[width=0.99\textwidth]{fig/6842CASE.pdf}
    
    \caption{Myocardial infarction sample from the PTB-XL dataset}\label{fig:ptb-case}
\end{subfigure} \\
\begin{subfigure}{.49\textwidth}
    \includegraphics[width=0.99\textwidth]{fig/5443Norm.pdf}
    
    \caption{Normal sample from the CUSPH dataset}\label{fig:chap-norm}
\end{subfigure}
\begin{subfigure}{.49\textwidth}
    \includegraphics[width=0.99\textwidth]{fig/1258CASE.pdf}
    \caption{Atrial fibrillation sample from the CUSPH dataset}\label{fig:chap-case}
\end{subfigure}
\caption{12-lead electrocardiogram samples over a two second window. For each subplot, the black line denotes the original signal while the \textcolor{blue}{blue} and \textcolor{red}{red} lines represent the signals generated by \textcolor{blue}{ECGT2T}, and \textcolor{red}{ECGS2E} respectively. ECGS2E takes in only Lead I as input and outputs 11 leads, while ECGT2T takes asynchronous Lead I and Lead II inputs and generates signals for 10 leads. Although ECGT2T uses two asynchronous leads, all leads are visualized synchronously for convenience.
}
\label{fig:gen-ptb}
\end{figure*}

\section{Data and Experiments}
We use two widely used ECG datasets collected at the Physikalisch-Technische Bundesanstalt (PTB-XL)~\cite{wag20} and Chapman University and Shaoxing People's Hospital (CUSPH)~\cite{zhe20} for our experiments. We selected 15,012 ECGs (9,527 normal, 5,485 samples with any myocardial infarction) from the PTB-XL dataset and 9,094 (7,314 normal, 1,780 atrial fibrillation) samples from the CUSPH dataset. Both datasets were split into stratified training, validation, and test sets with a ratio of 7:1:2. ECGT2T was trained using only training samples from the PTB-XL dataset. 
The two asynchronous input leads were sampled by taking a signal in Lead I and then selecting the delayed signal 0.5 seconds later in Lead II. 
% Table~\ref{tab:data} contains additional details regarding the counts and distribution of our data.


\subsection{Quality Assessment}


There are numerous ways to measure the discrepancy between two signals; however, common metrics such as root mean square error may not reflect the quality of generated electrocardiograms. If the original ECG waveform has artifacts such as power line interference, myokymia, and baseline wandering, a generated signal that accurately represents the cardiac condition will perform poorly under conventional metrics. Thus, we evaluated the quality of the ECGT2T outputs by comparing wave points between the original and generated signals.

An electrocardiogram has numerous points of clinical importance, and depending on a given patient's cardiac condition, some of these characteristics may not be visible in their electrocardiogram. However, the R-peak is both almost always present and is used for tasks such as  detecting heart rate and measuring stress level~\cite{sad12}. Therefore, we base our quality assessment with amplitude and positional missing errors for the R-peaks. We run our evaluation on V1 and V5 because of their electrode placement relative to Lead I: V1 is recorded orthogonally, while V5 is recorded in a similar position. We use Neurokit2~\cite{neu21} for R-peak detection on both the original and generated V1 and V5 leads.

Table~\ref{tab:gen_qual} shows the amplitude gaps and the position missing errors for the V1 and V5 leads generated by ECGT2T and ECGS2E for both datasets. For the PTB-XL dataset, the difference in amplitude is between 6.4$\sim$7.3\% and the positional missing error under 10 ms. For the CUSPH dataset the amplitude errors were higher than those in the PTB-XL dataset (7.6$\sim$11.4\% for amplitude and 3.8$\sim$14.4ms for position). We note that both models were only trained with samples from PTB-XL, which may explain this discrepancy.

Figure~\ref{fig:gen-ptb} shows 12-lead ECG samples over a two second window, where the black, red, and blue lines are the signals from the original electrocardiogram, ECGS2E, and ECGT2T respectively. Figure~\ref{fig:ptb-norm} and~\ref{fig:chap-norm} are normal samples from PTB-XL and CUSPH. The generated signals have less noise (seen in aVR and aVL in~\ref{fig:ptb-norm} and V4, V5, and V6 in~\ref{fig:chap-norm}) and baseline wandering (seen in V1, V3, and V6 in~\ref{fig:ptb-norm}). This artifact removal contributes to the amplitude gaps between the original and generated data seen in our quantitative evaluation.

% Baseline Wandering?
%  is a normal sample from the CUSPH dataset. While there are some amplitude errors in the signals generated by ECGS2E, both models show minimal positional errors and removed artifacts in V4, V5, and V6.

Figure~\ref{fig:ptb-case} is a sample with myocardial infarction. ST-segment elevation, T-wave inversion, and abnormal Q-waves are characteristic of the condition and are captured by the generated outputs at Lead II, Lead III, V2, and V3 in this sample with the exception of the ST-segment elevation at the first beat in V3 for both models and the T-Wave inversion in Lead II for ECGS2E. The discrepancy between the ECGT2T and ECGS2E Lead II results is because Lead I and Lead II typically have peaks that point toward the same direction in normal ECGs. However, as ECGS2E takes in only Lead I as input, it initially misses the inversion. In contrast, ECGT2T has two inputs and is therefore unaffected by the discrepancy between Lead I and Lead II.

Figure~\ref{fig:chap-case} is a sample with atrial fibrillation. The condition is characterized by the absence of the P-wave, irregular ventricular rate, and slightly aberrant QRS complexes. In this case, all leads in the original ECG are missing the P-wave, and both ECGT2T and ECGS2E outputs capture this absence across all generated leads. However, while the ECGS2E limb leads have good placement, the amplitude of peaks  stray from the original signal. 
% It seems that ECGS2E has a limitation for the elaborate calibration of the amplitude magnitude.
% %the CUSPH dataset은 부정맥과 관련된 것이기 때문에 비트의 모양보다는 
% % 대부분의 리드에서 pwave가 나타나지 않고 있다. 따라서 pwave를 그려내지 않는 것은 두 모델 모두 동일하게 잘 나타내고 있다. 그러나 ECGS2E는 리드1에 절대적인 영향을 받기 때문에, 리드1과 다른 리드간의 특정 관계를 벗어나게 되는 경우 전혀 다른 신호를 그려내는 단점을 보인다. 

% In this example with absence of the P-wave in all leads, both ECGT2T and ECGS2E output the signals without the P-wave in all leads.
% As the result of ECGS2E in Figure~\ref{fig:ptb-case}, the amplitude magnitude of peaks on ECGS2E strays from the original one though the arrangement of wave points are well matched. 
% Since arrhythmia is relevant to the abnormal wave points, the errors of the amplitude magnitude for the peaks in leads could not affect to find the arrhythmia cases.

% % Multiple signals in the ECGS2E stray from those in the original ECG. The signals of ECGT2T are very similar to the original signals, even though the CUSPH dataset is not used for training.
% In the conclusion for the ECG generation, we identified, in the normal examples, both models draw exquisitely other leads compared with original ones, besides V1 and V5 which are used in the quality assessment. They also have the effect of correction to the artifacts. 
% There are some limitations on the examples with heart diseases. For the purpose of correcting the outputs, the models transform the shape of wave points that can be important clues to determine classes. The error of the amplitude magnitude on case examples is larger than that on normal examples. 
% Finally, as we claimed, the generation performance of ECGT2T referring to two leads outperforms that of ECGS2E using a single lead.


\begin{table}
  \caption{Synthetic quality assessment for the generated R-peaks. \textit{Amp} denotes the amplitude error while \textit{Pos} is the position error.}
  \label{tab:gen_qual}
  \begin{tabular}{cccccc}
    \toprule
    
        {Dataset} & Lead & \multicolumn{2}{c}{V1} & \multicolumn{2}{c}{V5} \\
                            \cmidrule{2-6}
                            & Model & ECGT2T & ECGS2E & ECGT2T & ECGS2E \\ 
        \midrule
        {PTB-XL} & Amp    & 6.4\%      & 6.4\%     &         7.3\%             &      6.7\%                  \\
                                & Pos     & 8.3ms      & 8.2ms     &        2.0ms   &  2.2ms                    \\ 
        \midrule
        {CUSPH}  & Amp    & 11.4\%      & 11.2\%     &     7.6\%&   7.7\%                   \\
                                & Pos     & 12.9ms      & 14.4ms     & 3.8ms   &  4.1ms \\

  \bottomrule
\end{tabular}
\end{table}


\subsection{Classification Performance}

To assess how signals generated by ECGT2T perform in downstream classification tasks, we configure six ECG datasets with different lead combinations: \textit{Original} is the baseline 12-lead ECG, \textit{T2T} is original asynchronous Lead I \& Lead II and ten generated leads, \textit{S2E} is the original Lead I and 11 generated leads, \textit{Two Leads} is composed of asynchronous Lead I \& Lead II only, and \textit{Single Lead} is the set with only Lead I. We train and test six classifiers for the corresponding ECG sets on myocardial infarction and arrhythmia detection tasks. 

We run our classification experiments on a customized ResNet18~\cite{he16} model. The classifier uses residual blocks with one-dimensional convolutions and a single fully connected layer for the output layer, and is trained with focal loss~\cite{lin17} with $\alpha$ and $\gamma$ set to 0.5 and 2, and Adam optimization~\cite{kin14} with learning rate and weight decay set to $1e^{-4}$ and $1e^{-5}$ respectively.


Table~\ref{tab:clf-perform} shows the classification performance by dataset. For myocardial infarction on the PTB-XL dataset, the model using \textit{T2T} outperforms the other models with the exception of the classifier using the original 12-lead electrocardiogram. Additionally, the performance of the model trained on the original Lead I and 11 generated leads (\textit{S2E}) is worse than those of the models using the original asynchronous Lead I and Lead II with no generated leads (\textit{Two Leads}). For atrial fibrillation detection on the CUSPH dataset, the classifiers trained with generated data outperform the models trained with a single lead or two leads. However, the S2E model outperforms T2T classifier, and the spread of performance metric results is much narrower than for the myocardial infarction task. This can be explained in part due to the differences in the two medical conditions: myocardial infarction is detected by the shape change of the heartbeat (which requires multiple leads), while arrhythmia can be diagnosed if there are irregular rhythmic patterns in the waveform, a symptom that can be detected with a single lead.


\begin{table}
\caption{Classification performance. First column indicates the training data source: \textit{Original} is the standard 12-lead ECG, \textit{T2T} is the original asynchronous Lead I \& Lead II with 10 generated leads, \textit{S2E} is the original Lead I with 11 generated leads, \textit{Two Leads} is the original asynchronous Lead I and Lead II with no additional generated leads, and \textit{Single Lead} the original Lead I only.  Parenthesis denote 95\% confidence intervals.}
  \label{tab:clf-perform}
  \begin{tabular}{ccccc}
    \toprule



\multicolumn{1}{c}{} & \multicolumn{2}{c}{Myocardial infarction} & \multicolumn{2}{c}{Arrhythmia}     \\
\midrule
          & AUROC & AUPRC & AUROC & AUPRC \\
         \midrule
Original &
\begin{tabular}[c]{@{}c@{}}
0.960\\(0.95-0.97)\end{tabular} &
\begin{tabular}[c]{@{}c@{}}
0.957\\(0.95-0.96)\end{tabular} &
\begin{tabular}[c]{@{}c@{}}
0.994\\(0.98-0.99)\end{tabular} &
\begin{tabular}[c]{@{}c@{}}
0.972\\(0.96-0.98)\end{tabular} \\
\midrule

T2T &
\begin{tabular}[c]{@{}c@{}}
0.948\\(0.94-0.96)\end{tabular} &
\begin{tabular}[c]{@{}c@{}}
0.945\\(0.94-0.95)\end{tabular} &
\begin{tabular}[c]{@{}c@{}}
0.990\\(0.98-0.99)\end{tabular} &
\begin{tabular}[c]{@{}c@{}}
0.960\\(0.94-0.97)\end{tabular} \\
\midrule

S2E &
\begin{tabular}[c]{@{}c@{}}
0.929\\(0.92-0.94)\end{tabular} &
\begin{tabular}[c]{@{}c@{}}
0.931\\(0.92-0.94)\end{tabular} &
\begin{tabular}[c]{@{}c@{}}
0.993\\(0.99-1.0)\end{tabular} &
\begin{tabular}[c]{@{}c@{}}
0.965\\(0.95-0.98)\end{tabular} \\
\midrule

% SyncTwo &
% \begin{tabular}[c]{@{}c@{}}
% 0.937\\(0.93-0.95)\end{tabular} &
% \begin{tabular}[c]{@{}c@{}}
% 0.935\\(0.92-0.94)\end{tabular} &
% \begin{tabular}[c]{@{}c@{}}
% 0.987\\(0.98-0.99)\end{tabular} &
% \begin{tabular}[c]{@{}c@{}}
% 0.957\\(0.94-0.97)\end{tabular} \\
% \midrule

Two Leads &
\begin{tabular}[c]{@{}c@{}}
0.937\\(0.93-0.95)\end{tabular} &
\begin{tabular}[c]{@{}c@{}}
0.938\\(0.93-0.95)\end{tabular} &
\begin{tabular}[c]{@{}c@{}}
0.989\\(0.97-0.99)\end{tabular} &
\begin{tabular}[c]{@{}c@{}}
0.954\\(0.93-0.96)\end{tabular} \\
\midrule

Single Lead &
\begin{tabular}[c]{@{}c@{}}
0.883\\(0.87-0.89)\end{tabular} &
\begin{tabular}[c]{@{}c@{}}
0.891\\(0.88-0.90)\end{tabular} &
\begin{tabular}[c]{@{}c@{}}
0.989\\(0.98-0.99)\end{tabular} &
\begin{tabular}[c]{@{}c@{}}
0.945\\(0.92-0.96)\end{tabular} \\

  \bottomrule
\end{tabular}
\end{table}




% \begin{table}
%     \caption{Datasets used in this study. Table values are the number of samples used in each dataset. We selected samples with myocardial infarction and normal samples for the PTB-XL dataset and atrial fibrillation and normal samples for the CUSPH dataset.}
%     \label{tab:data}
%     % \vspace{-3mm}
%     \begin{tabular}{cccccc}
%     \toprule
%     {} & Total & Train & Validation & Test & \% Abnormal \\
%     \midrule
%     PTB-XL &15,012&  10,508       &        1,501      &   3,003    &    36.6\%      \\
%     \midrule
%     CUSPH  &9,094&     6,365    &        910      &    1,819    &    19.6\%  \\
%     \bottomrule
% \end{tabular}
% \end{table}








\section{Conclusion}
Modern wearable devices can record electrocardiograms with a couple of leads and collect increasingly large volumes of physiological signal data. However, the reduced number of leads used in these devices can be less accurate in measuring heart health or insufficient in detecting more complex cardiovascular conditions compared to standard 12-lead ECGs. To address these limitations, we propose ECGT2T, a deep generative model that simulates a standard 12-lead ECG from an input of two asynchronous leads by generating ten leads. We quantitatively evaluated the quality of the generated signals by comparing the R-peaks with those from the original ECG. Qualitatively, ECGT2T removed artifacts such as noise and baseline wandering that can impact clinical interpretation. For classifier performance, the models trained on generated signals outperformed models trained on one- and two- signal models in detecting myocardial infarction and arrhythmia.


% Additionally, the different dimensions between nonstandard and 12-lead ECGs mean the wealth of open-source standard ECG datasets, prior research, and pre-trained models may not be immediately applicable to wearable-generated ECG data.



%%
%% The next two lines define the bibliography style to be used, and
%% the bibliography file.
\bibliographystyle{ACM-Reference-Format}
\bibliography{t2t}

%%
%% If your work has an appendix, this is the place to put it.
% \appendix

\end{document}
\endinput
%%
%% End of file `sample-authordraft.tex'.
