An electrocardiogram (ECG) is a common and non-invasive way to help diagnose many cardiovascular conditions. There is increasing interest in using deep learning-based methods for ECG analysis, with recent studies proposing deep learning-based models for detecting atrial fibrillation~\cite{han19}, myocardial infarction~\cite{bal19}, and hypertrophic cardiomyopathy~\cite{ko20}. Parallel to these computational advancements, consumer wearable devices are becoming increasingly ubiquitous and can continuously record ECGs outside of the hospital setting. Due to physical and energy constraints, these devices usually have only one or two lead sensors in contrast to the 12 used in standard ECGs. A nonstandard number of leads can be insufficient for diagnosing more complex cardiovascular conditions and degrade the performance of deep learning-based diagnostic support systems~\cite{cho20}.

% Additionally, the difference in dimensions between one- or two- and 12-lead ECGs may prevent the use of open-source datasets and pretrained models from being directly applicable to wearable-generated ECG data.


As a lead measures heart health along a particular axis, a standard 12-lead ECG can be generated by using projections from two asynchronous leads. To this end, we propose \textsf{ECGT2T}, a deep generative model for ECG synthesis that takes in two input leads and outputs ten leads to simulate a 12-lead electrocardiogram. We draw inspiration from generative adversarial network (GAN) based image-to-image translation models~\cite{hua17,kar2019}. Similar to how these networks generate images based off reference styles, ECGT2T learns lead styles during training, represents the cardiac condition from two input leads, and generates ten leads with a corresponding style. We quantitatively evaluate the generated signals by comparing their R-peaks, one of the important features used to diagnose heart disease, with the corresponding points in the original data. Additionally, we assess the quality of the ECGT2T outputs by visualizing the overlapping original and generated signals and confirm the applicability of the synthesized ECGs on downstream classification tasks. To demonstrate the importance of having a second asynchronous lead, we repeat the experiments with ECGS2E, a model with a similar architecture as ECGT2T with a single input lead and 11 generated lead outputs.

% ECGT2T consists of four networks: the \textbf{style}, \textbf{mapping}, \textbf{mapping}, and \textbf{discriminative} network. The style network represents a cardiac condition from two asynchronous leads and captures ten styles for other respective leads.
% The mapping network produces ten styles from a random variable, which help enhance the depiction capability for cardiac styles in the style network. The generative network generates sample leads from any single input lead with a cardiac style. The discriminative network classifies a lead as an original or generated signal.

% As the first assessment for the synthetic quality, 
% we compared the rhythm and amplitude between synthesized and original leads. The R peak is one of the most important wave points in ECG. We found the errors for the timing and amplitude were less than under 15ms and about 10\%, respectively. 
% In addition, we checked out the quality by visually comparing original and generative leads. 
% ECGT2T generates leads nearly equivalent to the original one, while it removes artifacts such as noise and baseline wandering that result in inaccurate and misleading clinical interpretation~\cite{bla08}. To verify the applicability, we used ECGT2T as a lead augmentation method to construct a standard 12-lead ECG from two asynchronous leads. 
% We constructed six evaluation sets with different lead combinations: single-lead, two synchronous-lead, two asynchronous-lead, 12-lead by ECGS2E, 12-lead by ECGT2T, and original 12-lead sets. 
% We also develop a classifier based on ResNet~\cite{he16} to classify heart diseases. 
% As a result, the classifier trained using the 12-lead ECG set augmented by ECGT2T outperforms others in all metrics, except for one using the original 12-lead ECG set. 
% We expect that this result implies that the wearable devices could be utilized to diagnose cardiac diseases, though they provide insufficient leads.



% GANs are seeing continued use in ECG applications: Anne Marie Delaney et al.~\cite{del19} utilized a Long Short Term Memory GAN (LSTM-GAN) and Deconvolutional GAN to generate realistic synthetic ECGs that offer a level of data privacy. Pratik Singh~\cite{sin20} proposed a GAN structure that can generate and denoise ECG at same time. The efficiency of learning process was improved by denoising the GAN, and the result of denoising GAN showed better performance than the denoising autoencoder. 
% Tomer Golany~\cite{gol19} generated synthetic training data by LSTM-GAN that generates various types of heart beats. As a result of the ECG classification model, which was trained by adding the synthesized training data, a model showed better performance than the model using only the existing data. In addition, by reflecting the PQRST complex that exist in ECG typically using simulator,the performance of its generator was enhanced. 
% Eoin Brophy~\cite{bro20} proposed the Multivariate GAN that generates ECG V data from ECG II data. This showed the possibility of generating another lead from a single lead.
% learns lead styles for the cardiac condition referring to two given leads and translates other leads on the basis of any given single lead and cardiac styles.







% Our contribution is as follows.
% \begin{itemize}
%     \item We proposed the electrocardiogram synthesis based on GAN from two asynchronous leads to ten leads to construct the standard 12-lead ECG, called ECGT2T.
%     \item We access the synthetic quality. We showed that both the rhythm and amplitude on leads generated by ECGT2T closely resemble those of the original ones. We also identified that ECGT2T removes artifacts such as noise and baseline fluctuations appearing in the original leads.
%     \item We validated the applicability of ECGT2T by using it as a data augmentation method. The classifier trained by the augmented ECG set outperforms those by the ECG sets composed with partial leads.
% \end{itemize}


























% ~\cite{alivecor,beurer,galaxy,pocket}.
% Though these devices provide ECGs with only a couple of leads~\cite{avi19},
% a missing of particular signals by insufficient leads could degrade diagnostic capabilities, compared with the standard 12-lead ECG taken using medical equipment. 
% The existing literature also shows that insufficient leads cause the performance of deep learning models in diagnosing diseases~\cite{cho20,kra17}.


% heart failure with preserved ejection fraction~\cite{kwon20},

% A standard ECG consists of 12 leads to view the heart from various angles: six limb leads and six precordial leads. 
% The limb leads (Lead I, Lead II, Lead III, aVR, aVL, and aVF) record signals of the heart at the arms and/or legs, while the precordial leads (V1$\sim$6) are placed on the chest, and record anteriorly directed electrical activity. 
% The limb and precordial leads are mutually orthogonal.
%연구들 많음
% Recently, several studies proposed deep learning-based ECG models for atrial fibrillation~\cite{att19,han19}, heart failure with preserved ejection fraction~\cite{kwon20}, myocardial infarction~\cite{ach17,bal19}, and hypertrophic cardiomyopathy~\cite{ko20}.







% **[유비쿼터스 심전도]** 최근 다양한 웨어러블 디바이스 (스마트 시계[애플워치,갤럭시워츠...], 휴대기기[얼라이브코어])를 통해 병원(clinic)이 아닌 어느곳에서나 쉽게 측정할 수 있다. 
% **[심전도 데이터]** 병원에서 측정하는 일반적인 심전도는  12리드이다. 양쪽 손과 한쪽발 그리고 가슴에 전극을 장착하여 측정한다.  반면, 스마트 기기는 손목 또는 양손과 다리로부터 리드를 추출함으로 측정되는 리드는 1~2개 최대 6개정도로 제한된다 [ref].
% **[데이터 획득의 어려움 및 이에 따른 문제]** 의료용 심전도 측정기가 아닌 스마트 기기등으로는 일상생활에서 12개 리드를 모두 측정하는 것은 매우 번거롭고 어렵다. MI, STEMI와 같은 일부 심장 질환 예측에 있어 12개 리드 모두 사용하는 것이 일부의 리드로 예측하는 것에 비해 우수하다는 것은 널리 알려진 사실이다 [ref MI,STEMI].

% 12 leads
% The standard ECG consists 12 leads considering different points of view for the heart. To obtain 12 leads, the position of electrode placements is defined in clinical practice~\cite{raj08}. Six of the leads are considered limb leads recorded from arms and/or legs of the individual (i.e., LeadI-III, avR, avL, and avF). The other six leads are considered precordial leads recorded from precordium (i.e., V1-6). The limb and precordial leads are orthogonal. 
%스마트 기기 장점및한계 및 문제정의 
% With the advancement of smart devices, various wearable ones have appeared and enabled measure ECGs immediately anywhere ~\cite{alivecor,beurer,galaxy,pocket}.
% Though these devices provide ECGs with only a couple of leads~\cite{avi19},
% a missing of particular signals by insufficient leads could degrade diagnostic capabilities, compared with the standard 12-lead ECG taken using medical equipment. 
% The existing literature also shows that insufficient leads cause the performance of deep learning models in diagnosing diseases~\cite{cho20,kra17}.
% However, there are limitation of mobile and wearable devices, compared to the medical equipment to record ECGs~\cite{apple}. Such potable devices only provide a couple of leads or multiple leads. They also have a poor quality of ECG by any unexpected interference from the measurement environment. Consequently, these limitations degrades the performance of diagnosis for the heart diseases~\cite{cho20, kra17} due to the absence of different points of view.%더 늘려볼까?

%%%%%%%%% 핵심 %%%%%%%%%%% 
% 조합으로 다른 리드 생성 가능
% A study reported that 
% a wearable device could measure more than a single lead with different postures, though measured leads are recorded asynchronously~\cite{spa20}.
% % With wearable devices, more than a single lead could be measured with different postures, though multiple leads could be not recorded concurrently~\cite{spa20}.
% Because a lead implies a heart condition projected for the vector representing the magnitude and direction of electrical signals at any particular axis,
% a 12-lead ECG is conjectured from given some leads.
% % some leads enable to conjecture other leads through their combination because a lead implies a heart condition projected for the vector representing the magnitude and direction of electrical signals at any particular axis. 
% With the assumption of the circumstance that given at least two asynchronous leads, we propose a deep generative model for the ECG synthesis from given leads to ten leads, and we named this model \textsf{ECGT2T}.

% % However, some leads enable to conjecture rest other leads through the combination of their vectors because a lead has the signal projected for the vector representing the magnitude and direction of blood flows at any particular axis. We suppose that the subject could not measure multiple leads by wearable devices for herself.
% % Thus, given two leads (e.g., I\&II) measured asynchronously, we propose a deep learning model for the ECG synthesis from two leads to ten leads (ECGT2T). 
% % This allow a classifier to be trained using given two leads and generated ten leads as standard 12 leads.


% % 우리 방법의 컨셉
% For ECGT2T, we adopt the concept of image-to-image translation models based on a generative adversarial network (GAN)~\cite{kim17,zhu17,choi20}, which translates a source image to another image depicting the style of a reference image. 
% ECGT2T first represents the cardiac condition referring to two given leads and learn the lead styles corresponding to their cardiac condition. It generates a lead referring to its corresponding style with any given single lead.
% % learns lead styles for the cardiac condition referring to two given leads and translates other leads on the basis of any given single lead and cardiac styles.

% The architecture consists of four networks. 
% The style network represents a cardiac condition from two asynchronous leads and captures ten styles for other respective leads.
% The mapping network only produces ten styles from a random variable, which helps to enhance the capability of the depiction for cardiac styles in the style network. 
% The generative network generates a lead from any single input lead with a cardiac style.
% The discriminative network distinguishes between original and generated leads as widely used typical GAN.

% To evaluate the quality for generated leads and the availability of ECGT2T,
% we obtained two ECG datasets collected at the Physikalisch Technische Bundesanstalt (PTB-XL)~\cite{wag20} and Chapman University and Shaoxing People's Hospital (CUSPH)~\cite{zhe20}, respectively. 
% We train ECGT2T using only training samples in the PTB-XL dataset. 
% To clarify the necessity for at least two leads in ECGT2T, we developed another model for the ECG synthesis from a single lead to the other 11 leads (ECGS2E). 

% As the first assessment for the synthetic quality, 
% we compared the rhythm and amplitude between synthesized and original leads. The R peak is one of the most important wave points in ECG. We found the errors for the timing and amplitude were less than under 15ms and about 10\%, respectively. 
% In addition, we checked out the quality by visually comparing original and generative leads. 
% ECGT2T generates leads nearly equivalent to the original one, while it removes artifacts such as noise and baseline wandering that result in inaccurate and misleading clinical interpretation~\cite{bla08}.

% % R peak 는 가장 중요한 ECG point 중 하나이다. R peak에 대한 position과 amplitude의 차이가 모두 3프로 미만이였다. 
% % 생성 결과, 합성된 데이터는 심장의 lead I과같은 리듬을 가지며, 노이즈를 지우고, baseline wandering도 제거.
% % ECGT2T provides ten good quality leads, removing artifacts such as noise and baseline wandering, which result in inaccurate and misleading clinical interpretation~\cite{bla08}. 
% % % original 신호의 리듬과 앰플리튜드를 비교하였다.
% % To evaluate the quality, we compared the rhythm and amplitude between synthesized and original leads. 
% % In addition, the synthesized ten leads retain the rhythm and amplitude of original leads.
% % 우리는 합성된 리드들에 신호가 유효한지 확인하기 위한 실험을 진행하였다.

% To verify the applicability, we used ECGT2T as a lead augmentation method to construct a standard 12-lead ECG from two asynchronous leads. 
% We constructed six evaluation sets with different lead combinations: single-lead, two synchronous-lead, two asynchronous-lead, 12-lead by ECGS2E, 12-lead by ECGT2T, and original 12-lead sets. 
% We also develop a classifier based on ResNet~\cite{he16} to classify heart diseases. 
% As a result, the classifier trained using the 12-lead ECG set augmented by ECGT2T outperforms others in all metrics, except for one using the original 12-lead ECG set. 
% We expect that this result implies that the wearable devices could be utilized to diagnose cardiac diseases, though they provide insufficient leads.


% Our contribution is as follows.
% \begin{itemize}
%     \item We proposed the electrocardiogram synthesis based on GAN from two asynchronous leads to ten leads to construct the standard 12-lead ECG, called ECGT2T.
%     \item We access the synthetic quality. We showed that both the rhythm and amplitude on leads generated by ECGT2T closely resemble those of the original ones. We also identified that ECGT2T removes artifacts such as noise and baseline fluctuations appearing in the original leads.
%     \item We validated the applicability of ECGT2T by using it as a data augmentation method. The classifier trained by the augmented ECG set outperforms those by the ECG sets composed with partial leads.
% \end{itemize}


% The organization of the paper is as follows. Section 2 introduces the related work. Section 3 briefly introduces the electrocardiogram and describes the problem definition and proposed methods. Section 4 analyzes the generation performance and validates the applicability as a lead augmentation method. Finally, Section 5 summarizes and concludes the paper.