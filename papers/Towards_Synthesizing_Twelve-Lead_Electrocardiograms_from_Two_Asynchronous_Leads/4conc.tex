Modern wearable devices can record electrocardiograms with a couple of leads and collect increasingly large volumes of physiological signal data. However, the reduced number of leads used in these devices can be less accurate in measuring heart health or insufficient in detecting more complex cardiovascular conditions compared to standard 12-lead ECGs. To address these limitations, we propose ECGT2T, a deep generative model that simulates a standard 12-lead ECG from an input of two asynchronous leads by generating ten leads. We quantitatively evaluated the quality of the generated signals by comparing the R-peaks with those from the original ECG. Qualitatively, ECGT2T removed artifacts such as noise and baseline wandering that can impact clinical interpretation. For classifier performance, the models trained on generated signals outperformed models trained on one- and two- signal models in detecting myocardial infarction and arrhythmia.


% Additionally, the different dimensions between nonstandard and 12-lead ECGs mean the wealth of open-source standard ECG datasets, prior research, and pre-trained models may not be immediately applicable to wearable-generated ECG data.

