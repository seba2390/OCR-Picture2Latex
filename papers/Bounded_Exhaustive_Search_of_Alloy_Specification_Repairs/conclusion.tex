\section{Conclusion}

Software specification and modeling are crucial activities of most software development methods. Getting a software specification right, i.e., capturing correctly a software design, the constraints and expected properties, etc., especially when the language to capture these is \emph{formal}, is very challenging. Thus, techniques and tools that help developers in correctly specifying software is highly relevant. In this paper, we have presented a technique that helps precisely in this task, in the context of formal specification using the Alloy language \cite{Jackson2006}. Our technique has a number of characteristics that distinguish it from related work \cite{Wang+2019}. Firstly, it does not require any particular form of the \emph{oracles}, i.e., the properties to be used for assessing fix candidates (as opposed to existing work which require such oracles to be expressed in terms of test cases). Secondly, it bounded exhaustively explores the state space of fix candidates, thus finding a specification fix, or guaranteeing that such a fix is impossible within the established bounds, for the identified faulty locations, and with the provided mutation (syntactic modification) operators. This is suitable in an Alloy context, where users are accustomed to bounded-exhaustive analyses. This bounded-exhaustive exploration of fix candidates demands then appropriate mechanisms to make the search more efficient. Our technique comes with two sound pruning strategies, that allow us to avoid visiting large parts of the state space for fix candidates, which are guaranteed not to contain valid fixes. We have assessed our technique on a large benchmark of Alloy specifications, and shown that the pruning strategies have an important impact in analysis. The technique has an efficiency comparable to that of the previous work  \cite{Wang+2019}, it complements the latter in terms of the fixes it is able to generate, and is less prone to overfitting, as it naturally supports stronger oracles based on assertion checking and property satisfiability, that usually accompany Alloy specifications.
