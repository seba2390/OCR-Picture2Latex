%%%%%%%%%%%%%%%%%%%%%%%%%%%%%%%%%%%%%%%%%%%%%%%%%%%%%%%%%%%%%%%%%%%%%%%%%%%%%%%%
%\section*{Appendix: Meson mass correction for pion DA}
%%%%%%%%%%%%%%%%%%%%%%%%%%%%%%%%%%%%%%%%%%%%%%%%%%%%%%%%%%%%%%%%%%%%%%%%%%%%%%%%



%%%%%%%%%%%%%%%%%%%%%%%%%%%%%%%%%%%%%%%%%%%%%%%%%%%%%%%%%%%%%%%%%%%%%%%%%%%%%%%

In this Appendix, we derive the meson-mass corrections to the quasi-DA of the pion. 
For the pion DA, we need to calculate the same series sum as for the unpolarized parton distribution in Ref.~\cite{Chen:2016utp}:
\beq\label{Kn}
 K_n=\frac{\langle (1-2x)^{n-1} \rangle_{\tilde\phi}}{ \langle (1-2x)^{n-1} \rangle_{\phi}}=\sum_{i=0}^{i_\text{max}} C_{n-i}^i c^i =\frac{\lambda_{(\mu_1}\cdots \lambda_{\mu_n)}
  P^{\mu_1}\cdots P^{\mu_n}}{\lambda_{\mu_1}\cdots \lambda_{\mu_n}
  P^{\mu_1}\cdots P^{\mu_n}},
\eeq
where $c=m_{\pi}^2/4 P_z^2$ and $(\ldots)$ means the indices enclosed are symmetric and traceless. The result for even $n$ ($=2k$) is
%
\begin{equation}\label{evensum}
\sum_{j=0}^k  C_{n-j}^j c^j =\frac{1}{\sqrt{1+4c}}\Big[\left(\frac{f_-}{2}\right)^{2k+1}+\left(\frac{f_+}{2}\right)^{2k+1}\Big],
\end{equation}
%
while for odd $n$ ($=2k+1$), it is
%
\begin{equation}\label{oddsum}
\sum_{j=0}^k C_{n-j}^j c^j =\frac{1}{\sqrt{1+4c}}\Big[-\left(\frac{f_-}{2}\right)^{2k+2}+\left(\frac{f_+}{2}\right)^{2k+2}\Big],
\end{equation}
%
where $f_{\pm}=\sqrt{1+4c}\pm 1$. 

With Eqs.~\ref{evensum} and \ref{oddsum}, we perform an inverse Mellin transform on the moment relation of Eq.~\ref{Kn}
%
\begin{equation}
\frac{1}{2\pi i}\int_{-i\infty}^{i\infty} dn \, s^{-n} \langle (1-2x)^{n-1} \rangle .
\end{equation}


To extract $\phi(x)$ from $\tilde\phi(x)$, let us rewrite Eq.~\ref{Kn} for an even $n=2k$ as
%
\begin{equation}\label{momentinverserel}
\langle (1-2x)^{2k-1}\rangle_\phi=\langle (1-2x)^{2k-1}\rangle_{\tilde \phi}\frac{\sqrt{1+4c}}{\left(\frac{f_-}{2}\right)^{2k+1}+\left(\frac{f_+}{2}\right)^{2k+1}}=\langle (1-2x)^{2k-1}\rangle_{\tilde q}\frac{\sqrt{1+4c}}{\left(\frac{f_+}{2}\right)^{2k+1}}\sum_{n=0}^\infty (-1)^n\left(\frac{f_-}{f_+}\right)^{(2k+1)n}.
\end{equation}
%
The inverse Mellin transform then leads to
%
\begin{align}
\phi(x)-\phi(1-x)=2\sqrt{1+4c}\sum_{n=0}^\infty \frac{(-f_-)^n}{f_+^{n+1}}\Big[\tilde\phi\Big(\frac{1}{2}-\frac{f_+^{n+1}(1-2x)}{4f_-^n}\Big)-\tilde\phi\Big(\frac{1}{2}+\frac{f_+^{n+1}(1-2x)}{4f_-^n}\Big)\Big].
\end{align}
Similarly, we have
%
\begin{equation}
\phi(x)+\phi(1-x)=2\sqrt{1+4c}\sum_{n=0}^\infty \frac{f_-^n}{f_+^{n+1}}\Big[\tilde\phi\Big(\frac{1}{2}-\frac{f_+^{n+1}(1-2x)}{4f_-^n}\Big)+\tilde\phi\Big(\frac{1}{2}+\frac{f_+^{n+1}(1-2x)}{4f_-^n}\Big)\Big].
\end{equation}
%
Therefore,
%
\begin{align}
\phi(x)&=\sqrt{1+4c}\sum_{n=0}^\infty \frac{f_-^n}{f_+^{n+1}}\Big[(1+(-1)^n)\tilde\phi\Big(\frac{1}{2}-\frac{f_+^{n+1}(1-2x)}{4f_-^n}\Big)+(1-(-1)^n)\tilde\phi\Big(\frac{1}{2}+\frac{f_+^{n+1}(1-2x)}{4f_-^n}\Big)\Big]\non\\
&=\sqrt{1+4c}\sum_{n=0}^\infty \frac{(4c)^n}{f_+^{2n+1}}\Big[(1+(-1)^n)\tilde\phi\Big(\frac{1}{2}-\frac{f_+^{2n+1}(1-2x)}{4(4c)^n}\Big)+(1-(-1)^n)\tilde\phi\Big(\frac{1}{2}+\frac{f_+^{2n+1}(1-2x)}{4(4c)^n}\Big)\Big],
\end{align}
where in the last line we have used $f_+ f_-=4c$. Since $f_+\gg f_-$ or $c$ and the quasi-DA $\tilde \phi(x)$ vanishes asymptotically for large $x$, the above sum is dominated by the first term with $n=0$. In practical calculations, we can reach reasonable accuracy by taking only the first few terms in the sum. In Refs.~\cite{Braun:2011zr,Braun:2011dg}, it was argued that for hadron-to-vacuum matrix elements, the mass corrections also receive contributions from higher-twist operators that can be reduced to total derivatives of twist-two ones. We do not explicitly consider such terms, since they will anyway be part of the higher-twist corrections that are parametrized with a specific form in the present work.


