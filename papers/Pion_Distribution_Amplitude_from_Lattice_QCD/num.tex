%%%%%%%%%%%%%%%%%%%%%%%%%%%%%%%%%%%%%%%%%%%%%%%%%%%%%%%%%%%%%%%%%%%%%%%%%%%%%%%%
% \section{Numerical Results}
%%%%%%%%%%%%%%%%%%%%%%%%%%%%%%%%%%%%%%%%%%%%%%%%%%%%%%%%%%%%%%%%%%%%%%%%%%%%%%%%

In this section, we report the first results of a lattice-QCD calculation of the $x$-dependence of the pion DA.
We use clover valence fermions on gauge ensembles with $2+1+1$ flavors (degenerate up/down, strange and charm degrees of freedom in the QCD vacuum) of highly improved staggered quarks (HISQ)~\cite{Follana:2006rc} generated by MILC Collaboration~\cite{Bazavov:2012xda}. The pion 
mass of this ensemble is $m_\pi \approx 310$~MeV with lattice spacing $a\approx 0.12$~fm and box size $L\approx 3$~fm, corresponding to $m_\pi L \approx 4.5$.
The HISQ ensembles are hypercubic (HYP)-smeared~\cite{Hasenfratz:2001hp} and the clover parameters are tuned to recover the lowest pion mass of the staggered quarks in the sea.\footnote{Other studies using the same setup are done in Refs.~\cite{Bhattacharya:2016zcn,Bhattacharya:2015wna,Bhattacharya:2015esa,Bhattacharya:2013ehc} and no exceptional-configuration behavior was observed.}   
HYP smearing has been shown to significantly improve the discretization effects on operators and shift their corresponding renormalizations toward their tree-level values (near 1 for quark bilinear operators). 
The results shown in this work are done using correlators calculated from 3 source locations on 986 configurations.  For each positive $z$-momentum $P_z$, the matrix elements are averaged with their corresponding $-P_z$ to improve the signal. 

\subsection{Pion Quasi-Distribution Amplitude}

\begin{figure}[tbp]
\includegraphics[width=0.6\textwidth]{pion_plot.pdf}
\caption{The pion quasi-distribution amplitude (at $\mu=2$~GeV) after one-loop and mass correction for $P_z=2$ (blue) and $3$ (green) (in units of $2\pi/L$).
The extrapolation to infinite momentum to remove the remaining higher-twist effects is shown in red. The Wilson-line renormalization that removes the power divergent contribution is not included in this plot, and will be implemented later in the results of improved pion quasi-DA. The purple dashed line is the asymptotic form $6x(1-x)$.
}
\label{fig:pionDA}
\end{figure}

We begin with the pion quasi-DA without the Wilson-line renormalization. Here, we follow similar steps to those listed in our previous work on nucleon parton distribution functions: First, we implement the one-loop and mass corrections whose formulae are detailed in the previous sections, and extrapolate to the infinite-momentum limit via {$\alpha(x)+\beta(x)/P_z^2$} (and thereby remove the higher-twist terms that come in at $O(\Lambda_\text{QCD}^2/P_z^2)$). The true light-cone pion DA should be recovered. 
Fig.~\ref{fig:pionDA} shows the results for the 
pion quasi-DA at $\mu=2$~GeV after including one-loop and mass corrections at different momenta $P_z=2, 3$ (in units of $2\pi/L$)\footnote{For this work, we initially calculate the pion quasi-DA for 3 momenta, $P_z= 1, 2, 3$ (in units of $2\pi/L$), but the corrections term for the smallest-momentum distribution is less well-behaved, as observed in the nucleon PDF case~\cite{Chen:2016utp}; thus, we drop it in the rest of this work.}. We then extrapolate using these 2 momenta to the infinite-momentum limit using the form {$\alpha(x)+\beta(x)/P_z^2$}, shown in red, where a linear divergence is present in the one-loop matching kernel (later, we will show improved results for the pion DA where the power divergence is removed by taking into account the Wilson-line normalization). The dashed line is the asymptotic form $6x(1-x)$. All our resulting curves are symmetric around $x=1/2$, as expected from the symmetry of the pion DA under the interchange $x\leftrightarrow 1-x$. The pion DA has often been expanded in terms of Gegenbauer polynomials in past studies, and the dashed curve here contains only the zeroth Gegenbauer polynomial. The other three curves are broader than the asymptotic form, indicating contribution from higher Gegenbauer polynomials.

We note several interesting features of this result.
First, the pion DA is expected to vanish outside the region $x\in [0,1]$ after taking the IMF limit. We see the $P_z=2$ pion quasi-DA is nonzero for $x\in [1,1.7]$, and this range shrinks to $x\in [1,1.4]$ for $P_z=3$. A similar pattern is observed for the region $x<0$. The distributions are moving in the right direction as the pion DA will vanish outside $[0,1]$ with $P_z \to \infty$. 
However, after taking the IMF limit via extrapolation formula $\alpha(x)+\beta(x)/P_z^2$, we find there is still residual distribution outside $x\in [0,1]$. 
This is likely due to using the approximation Eq.~\ref{cov}, where the cancellation among $\tilde\phi(x)$ outside the $x\in [0,1]$ region is between an all-order result and a perturbative expression, and is therefore incomplete\footnote{Although the difference here is formally of higher order, it might have a sizable numerical effect.}.
This can be improved by including the higher-order matching and going to larger momentum, which we will explore more extensively in future work.


Second, the results near $x=0$ and $x=1$ are not reliable. There are unphysical peaks and dips due to the linear divergence in the one-loop matching in these regions, which become smaller as $P_z$ becomes larger. The smallest-$x$ region is dominated by the smallest nonzero momentum fraction, which is proportional to $1/L$ (where $L$ is the lattice length along boosted-momentum direction), due to the finite box size. To improve results near these regions would require large momentum and large box size.

Third, the unphysical oscillatory behavior near $x=0$ and $x=1$ is largely due to the presence of a linear divergence in the one-loop matching for the bare long-link matrix element. In Refs.~\cite{Ishikawa:2016znu,Chen:2016fxx}, it has been shown that the power divergence (in the $a \to 0$ limit) in the long-link operator can be removed to all orders by a mass counterterm $\delta m$ (in the auxiliary $z$-field description of the Wilson line), which is the same as in the renormalization of an open Wilson line. After
the Wilson-line renormalization, the pion quasi-DA is improved such that it contains at most logarithmic
divergences. We will investigate this improved quasi-DA numerically in the rest of the paper. 

\subsection{Improved Pion Quasi-Distribution Amplitude}
The improved pion quasi-DA without power divergence can be defined as~\cite{Chen:2016fxx}
\beq\label{impDA}
{\tilde \phi}_{\text{imp}}(x, P_z)=\frac{i}{f_\pi}\int\frac{dz}{2\pi}e^{-i(x-1)P_z z-\delta m |z|}\langle \pi(P)|\bar\psi(0)\gamma^z\gamma_5 \Gamma(0,z)\psi(z)|0\rangle ,
\eeq
where $\delta m$ should be determined nonperturbatively through studying the Wilson-line renormalization. It is worthwhile to comment that since the mass counterterm $\delta m$ cancels all power divergence in the pion quasi-DA\footnote{At perturbative one-loop, it appears as a linear divergence, but more-divergent power divergences can appear at higher loops.}, when we do the perturbative matching between Eqs.~\ref{impDA} and \ref{LCDA}, we need to remove the linear divergence present in the one-loop matching kernel for consistency. Moreover, as shown in Ref.~\cite{Chen:2016fxx} and below, $\delta m$ is negative, the exponential factor $e^{-\delta m |z|}$ then increases the weight of matrix elements with relatively large
$z$, and thereby increases the contribution at relatively small momentum when Fourier
transforming to momentum space. It is therefore important to properly account for the higher-twist corrections.

We first explore the nonperturbative determination of $\delta m$ discussed in Ref.~\cite{Musch:2010ka} using the static-quark potential for the renormalization of Wilson loop. The Wilson loop $W(t,r)$ of width $r$ and length $t$ is long in the $t$-direction such that higher excitations are sufficiently suppressed. The quark potential is then obtained as
\beq
V(r)=-\frac{1}{a}\lim_{t\to\infty}\ln\frac{\langle \text{Tr}[W(t,r)]\rangle}{\langle \text{Tr}[W(t-a,r)]\rangle},
\eeq
where $a$ is the lattice spacing and the cusp anomalous dimensions from the four sharp corners of the Wilson loop are canceled between numerator and denominator. When $r$ is larger than the confinement scale but shorter than the string breaking scale\footnote{The onset of string breaking can be estimated by $V(r) > 2 m_B-m_{\Upsilon}=1.1$~GeV.}, the lattice data should be described by
the energy of the static quark pairs
\beq\label{V}
V(r)=\frac{c_1}{r}+c_2+c_3 r ,
\eeq
where the $c_1$ term is the Coulomb potential which dominates at short distance,  $c_3$ term is the confinement linear potential. The $c_2$ term is twice the rest mass of the heavy quark, and we expect $c_2=\tilde{c}/a+\mathcal{O}(\Lambda_{\text{QCD}})$. Thus, the $\delta m$ counterterm that cancels the linear divergence in the Wilson line is 
\beq
\delta m=-\frac{\tilde{c}}{2a}= -\frac{c_2}{2} +\mathcal{O}(\Lambda_{\text{QCD}}).
\eeq
This leads to
\beq\label{dm}
\delta m\simeq -260 \pm 200 \mbox{ MeV},
\eeq
where we have used the fitted value $\delta m=-0.16/a$ from Fig.~\ref{fig:V}, which is $0.38$ times of the one-loop value computed in Ref.~\cite{Chen:2016fxx}, 
and we estimate the error by the size of $\Lambda_{\text{QCD}} \sim 200$~MeV. The error can be reduced by performing the computation at different $a$ to extract the $1/a$-dependent term in $c_2$.


\begin{figure}[tbp]
\includegraphics[width=0.6\textwidth]{fit-potential.pdf}
\caption{The energy of the static-quark pairs fit to the functional form of Eq.~\ref{V}.
%with $c_1 = -0.42$, $c_2= 0.32$ and $c_3= 0.09$, all in units of the lattice spacing $a$. 
The point at $r=1$ is excluded from the fit to reduce discretization error. If we further exclude the $r=2$ point, then $c_2$ is increased by 15\%, still in the range of Eq.~\ref{dm}.}
\label{fig:V}
\end{figure}


As mentioned before, once the improved pion DA of Eq.~\ref{impDA} is used with $\delta m$ determined nonperturbatively, the linear divergence in the one-loop matching kernel will {be canceled} by the $\delta m$ counterterm as shown in Eq.~\ref{imprmat}.
In Ref.~\cite{Chen:2016fxx}, it was demonstrated that in the limit $\Lambda/P_z \to \infty$, only the Wilson-line self-energy diagram is divergent among the ``real diagrams'' (i.e. $Z_\phi^{(1)}(x, y)$ of Eq.~\ref{eq7}) 
in one loop and in the Feynman gauge. Therefore, in a lattice perturbation theory calculation, one only needs to calculate this diagram, which is linearly divergent ($\propto \Lambda/P_z$). Using the simplest version of gauge-field discretization, one finds the matching between the momentum and lattice cut-offs is $\Lambda =\pi/a +\mathcal{O}(a^2)$. This result holds not only for the non-singlet quasi-PDF operator used in Ref.~\cite{Chen:2016fxx}, but also for the pion quasi-DA in this work. The ``virtual diagrams'' (i.e. $C$ of Eq.~\ref{eq7}) will contain logarithmic divergence from the quark self-energy diagram, which can be removed by adding counterterms in the lattice action or treating the integration limits of $C$ carefully. In Eq.~\ref{imprmat}, the $\Lambda/P_z \to \infty$ limit is not taken, so $C$ is finite. We find that the difference between taking this limit and not is small, certainly within the error induced by the uncertainty of $\delta m$.


The resulting improved pion quasi-DA using Eq.~\ref{impDA} 
and the central value of $\delta m$ is shown in Fig.~\ref{fig:pionDApheno}. The unphysical oscillations near $x=0$ and $x=1$ are largely removed. There are still small kinks in the unphysical region, but they are expected to vanish when higher-order matching is taken into account and the $P_z \to \infty$ limit is approached. 

The final result that includes the lattice statistical uncertainties, finite-$P_z$ corrections and the uncertainty of $\delta m$ estimated in Eq.~\ref{dm} is presented in Fig.~\ref{fig:x}. 
Also shown in the same figure are the model calculation from the Dyson-Schwinger equation (DSE)~\cite{Chang:2013pq}, from the truncated Gegenbauer expansion fit to the Belle data for the $\gamma\gamma^*\to\pi^0$ form factor~\cite{Agaev:2012tm} and from parametrizations of the pion DA with the parameters fit to lowest-moment calculations from lattice QCD in~\cite{Braun:2015axa}. For the fit to the Belle data, we use the Gegenbauer polynomial expansion up to the eighth moment given in Ref.~\cite{Agaev:2012tm} and run to 2~GeV. For the fit to the lattice moment calculations, we have chosen two different parametrizations. One is simply a truncation of the Gegenbauer polynomial expansion of the pion DA to the second order $\phi(x)=6x(1-x)[1+a_2 C_2^{3/2}(2x-1)]$ (labeled ``Param 1") with the value of $a_2$ taken from~\cite{Braun:2015axa}. The other is $\phi(x)=A[x(1-x)]^B$ with $A$ and $B$ determined from the normalization condition and the second moment of the pion DA (labeled ``Param 2"). The second parametrization is close to the DSE result, but differs from the first parametrization. The difference between them can be viewed as a rough estimate of errors from the truncation, and reflects uncertainties in the parametrization, which are currently underestimated even though both bands have smaller errors than ours. A direct calculation of the $x$-dependence will help to resolve such uncertainties. Of course, this can be achieved only when the direct calculation reaches a sufficiently high accuracy, which is difficult at the current stage but might be improved in the foreseeable future. Nonetheless,
the results of our direct calculation at 310-MeV pion mass is in agreement within errors with DSE, Belle data fit result and the parametrized reconstruction of pion DAs in the region near $x=1/2$, although the two parametrized forms differ from each other.
The uncertainty of our distribution is dominated by the $\delta m$ uncertainty, which can be largely removed by performing calculations at different lattice spacing. 
As before, we still have residual distribution outside the $[0,1]$ region, which should vanish when larger momenta are reached and higher-order matching is taken into account in the future. 
Also, as is typical in an exploratory study, the pion mass in this work is still heavier than its physical value. However, the study of Ref.~\cite{Chen:2003fp} shows that the leading chiral correction for $\phi_{\pi}(x)$ is proportional to $m_{\pi}^2$ with the chiral logarithm $m_{\pi}^2 \ln m_{\pi}^2$ completely absorbed by $f_{\pi}$. This property will simplify the chiral extrapolation in future computations.
It is encouraging that our current result is qualitatively similar to other determinations using lattice-moment parametrization, models and fits to experimental data, and also favors a single-hump distribution in $\phi_{\pi}(x)$. 

\begin{figure}[tbp]
\includegraphics[width=0.6\textwidth]{pion_deltam38.pdf}
\caption{The improved pion distribution amplitude at $\mu=2$~GeV using $\delta m=0.38\delta m_{\text{1-loop}}$ in Eq.~\ref{impDA} for $P_z=2$ (blue) and $3$ (green) (in units of $2\pi/L$) 
and extrapolation to infinite-momentum limit (red), along with the asymptotic form $6x(1-x)$ (dashed line).
}
\label{fig:pionDApheno}
\end{figure}

\begin{figure}[tbp]
\includegraphics[width=0.65\textwidth]{pion_pheno_comp.pdf}
\caption{The improved pion distribution amplitude at $\mu=2$~GeV with $\delta m=(0.38 \pm 0.28)\delta m_{\text{1-loop}}$ (red band with the central value denoted by red dot-dashed) obtained in this work (labeled as ``LaMET''), along with that obtained from the Dyson-Schwinger equation (labeled ``DSE'') analysis of the pion (blue), a fit to the Belle data (labeled ``Belle", cyan), parametrized fits to the lattice moments (labeled ``Param 1" and ``Param 2'', respectively, gray and green) and the asymptotic form (labeled ``Asymp", purple).
}
\label{fig:x}
\end{figure}


\section{Summary and Outlook}\label{sec:sum}
In this work, we presented the first lattice-QCD calculation of the pion distribution amplitude using the large-momentum
effective field theory (LaMET) approach. 
We derived the mass-correction formulation needed for the pion quasi-distribution amplitude. 
We also implemented the Wilson-line renormalization in this work, which is important to remove the power divergences in LaMET approach; and found that it reduces the oscillation at the end points of the distribution amplitude.
Finally, our result at 310-MeV pion mass shows similar behavior as previous studies done using DSE, a fit to the Belle data and as parametrizations with latest lattice moment result, and favors a single-hump structure.

However, in the current study, we have not accounted for all possible systematic uncertainties, and there are multiple improvements that can be done in future studies. For example, in our work, it is clear that larger boosted momentum is needed for the pion distribution amplitude to make the result outside the physical region consistent with $0$ than for the unpolarized nucleon parton distribution function. Finer lattice spacing would help reduce the uncertainty in the counterterm determined by the Wilson-loop study. Larger lattice box and also higher-order matching would reduce the unphysical kinks near $x=1$ and $0$. Last but not least, we hope this work will encourage following works to extensively study the distribution amplitude of the pion and other hadrons. 

%%%%%%%%%%%%%%%%%%%%%%%%%%%%%%%%%%%%%%%%%%%%%%%%%%%%%%%%%%%%%%%%%%%%%%%%%%%%%%%% 
\section*{Acknowledgments}
%%%%%%%%%%%%%%%%%%%%%%%%%%%%%%%%%%%%%%%%%%%%%%%%%%%%%%%%%%%%%%%%%%%%%%%%%%%%%%%%

JHZ thanks G. Bali, V. Braun and M. G\"ockeler and A. Sch\"afer for valuable discussions and comments. The LQCD calculations were performed using the Chroma software
suite~\cite{Edwards:2004sx}. We thank MILC Collaboration for sharing the lattices used to perform this study. 
Computations for this work were carried out in part on facilities of the
USQCD Collaboration, which are funded by the Office of Science of the
U.S. Department of Energy, and on the National Energy Research Scientific Computing Center, a DOE Office of Science User Facility supported by the Office of Science of the U.S. Department of Energy under Contract No. DE-AC02-05CH11231.
This work was partially supported by the U.S. Department of Energy via grants DE-FG02-93ER-40762, a grant (No.~11DZ2260700) from the Office of Science and Technology in Shanghai Municipal Government, grants from National Science Foundation of China (No.~11175114, No.~11405104, No.~11655002), a DFG grant SCHA~458/20-1, the SFB/TRR-55 grant "Hadron Physics from Lattice QCD", the MIT MISTI program, the Ministry of Science and Technology, Taiwan under Grant Nos.~105-2112-M-002-017-MY3 and 105-2918-I-002 -003 and the CASTS of NTU. The work of JWC and XJ is supported in part by the U.S. Department of Energy, Office of Science, Office of Nuclear Physics, within the framework of the TMD Topical Collaboration.

