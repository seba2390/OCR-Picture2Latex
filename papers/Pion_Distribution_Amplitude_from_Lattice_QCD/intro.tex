%%%%%%%%%%%%%%%%%%%%%%%%%%%%%%%%%%%%%%%%%%%%%%%%%%%%%%%%%%%%%%%%%%%%%%%%%%%%%%%%
%\section{Introduction}\label{sec:intro}
%%%%%%%%%%%%%%%%%%%%%%%%%%%%%%%%%%%%%%%%%%%%%%%%%%%%%%%%%%%%%%%%%%%%%%%%%%%%%%%%

Hadronic lightcone distribution amplitudes (DAs) play an essential role in the description of hard exclusive processes involving large momentum transfer. They are crucial inputs for processes relevant to measuring fundamental parameters of the Standard Model and probing new physics~\cite{Stewart:2003gt}. The QCD factorization theorem and asymptotic freedom allow us to separate the short-distance physics incorporated in the hard quark and gluon subprocesses from the long-distance physics incorporated in the process-independent hadronic DAs. 
While the short-distance hard quark and gluon subprocesses are calculable perturbatively, the hadronic DAs are intrinsically nonperturbative. To determine them, 
we must resort to experimental measurements, lattice calculations or QCD models. 

The simplest and most extensively studied hadronic DA is the twist-2 DA of the pion. It represents the probability amplitude of finding the valence $q\bar q$ Fock state in the pion with the quark (antiquark) carrying a fraction $x$ ($1-x$) of the total pion momentum. 
The pion lightcone distribution amplitude (LCDA) is defined as 
\beq\label{LCDA}
\phi_{\pi}(x)={\frac{i}{f_{\pi}}}\int\frac{d\xi}{2\pi}e^{i(x-1)\xi\lambda\cdot P}\langle \pi(P)|\bar \psi(0)\lambda\cdot\gamma\gamma_5 \Gamma(0,\xi\lambda)\psi(\xi\lambda)|0\rangle
\eeq
with the normalization $\int_0^1 dx\, \phi_\pi(x)=1$, where the two quark fields are separated along the lightcone with $\lambda^\mu=(1,0,0,-1)/\sqrt 2$, and $x$ ($1-x$) denotes the momentum fraction of the quark (antiquark).
The twist-2 pion DA can be constrained from experimental measurements of e.g. the pion form factor~\cite{Farrar:1979aw}%
, and then as an input can be used to test QCD in, for example, $\gamma \gamma* \rightarrow \pi^0$ from BaBar and Belle~\cite{Aubert:2009mc,Uehara:2012ag}.
Some experiments proposed~\cite{Sawada:2016mao} at J-PARC might also be of use.
At large momentum transfer, the pion DA is well known to follow a universal asymptotic form~\cite{Lepage:1979zb}:  
$\phi_\pi(x, \mu \to \infty) \rightarrow 6 x(1-x)$.
However, 
there have been some debates over the shape of the pion DA at lower scales $\mu$.
For example, Ref.~\cite{Chernyak:1981zz} suggested a ``double-humped'' shape for the pion DA, which is very different from the asymptotic form, while other QCD models (for example, large-$N_c$ Regge model~\cite{RuizArriola:2006jge}, 
QCD sum rule calculations~\cite{Radyushkin:1994xv},
Nambu-Jona-Lasinio model~\cite{RuizArriola:2002bp},
Dyson-Schwinger equations~\cite{Chang:2013pq}, truncated Gegenbauer expansion~\cite{Agaev:2012tm}, just to name a few) 
do not suggest such a feature. Unfortunately, lattice calculations have traditionally only been able to extract the lowest few moments of the pion DA after using the operator product expansion (OPE). The highest moment ever calculated on the lattice is the second moment~\cite{Braun:2015axa,Arthur:2010xf,Braun:2006dg,Daniel:1990ah,Martinelli:1987si}, and most calculations struggled with the noise-to-signal ratio. Ref.~\cite{Cloet:2013tta} took the moment results from lattice-QCD calculations and reconstructed the pion DA using a specific parametrization; however, the errors propagating from the lattice calculations are relatively large, preventing them from discriminating between the QCD models. Calculating moments beyond the lowest two on the lattice is much more difficult due to the breaking of rotational symmetry by discretization, which induces divergent mixing coefficients to lower moments such that the noise-to-signal becomes a big problem. It was proposed to use a smeared source to reduce the discretization error~\cite{Davoudi:2012ya}, or to use another scale to replace the lattice cut-off in the mixing. For example, by using a heavy-light current in the OPE for the current-current correlator, the scale in the mixing parameters is replaced by the heavy-quark mass~\cite{Detmold:2005gg} or the gradient-flow scale in the proposal of Ref.~\cite{Monahan:2016bvm}.
Having an alternative approach to calculate the pion DA with better precision and quantifiable systematics is highly desirable so that it can be used to make predictions in other harder-to-calculate processes, such as $B\rightarrow \pi\pi$. 




%%%%%%%%%%%%%%%%%%%%%%%%%%%%%%%%%%%%%%%%%%%%%%%%%%%%%%%%%%%%

Recently, a new approach has been proposed to calculate the full 
$x$ dependence of parton quantities, such as parton distributions, distribution amplitudes, etc.~\cite{Ji:2013dva}. The method is based on the observation
that, while in the rest frame of the nucleon, parton physics corresponds to
lightcone correlations, the same physics can be obtained through
time-independent spatial correlations in the infinite-momentum frame (IMF) of the hadron after a matching procedure. For
finite but large momenta feasible in lattice simulations, a large-momentum
effective field theory (LaMET) can be used to relate Euclidean
quasi-observables to physical observables through a factorization theorem~\cite{Ji:2014gla} (there exist also other approaches to extract lightcone quantities from Euclidean ones, see e.g.~\cite{Braun:2007wv,Liu:1993cv,Liu:1998um,Liu:1999ak,Liu:2016djw}). 
Since then, there have been many follow-up studies on factorization~\cite{Ma:2014jla} and determinations of the
one-loop corrections needed
to connect finite-momentum quasi-distributions to lightcone distributions 
for nonsinglet leading-twist PDFs~\cite{Xiong:2013bka}, generalized parton distributions (GPDs)~\cite{Ji:2015qla}, transversity GPDs~\cite{Xiong:2015nua} and pion DA~\cite{Ji:2015qla} in the continuum. Reference~\cite{Ji:2015jwa} also explores the renormalization of quasi-distributions, and establishes that the quasi-distribution is multiplicatively renormalizable at two-loop order. 
There are also proposals to improve the quark correlators to remove linear divergences in the one-loop matching~\cite{Li:2016amo}, to improve the nucleon source to get higher nucleon momenta on the lattice~\cite{Bali:2016lva}, and to use the non-perturbative evolution of quasi-distributions as a guide for the extrapolation of lattice results at moderate momentum to infinite momentum~\cite{Radyushkin:2016hsy,Radyushkin:2017gjd}. In Refs.~\cite{Ishikawa:2016znu,Chen:2016fxx}, it was shown that the power divergence present in the long-link matrix elements can be removed by a mass renormalization in the auxiliary $z$-field formalism, in the same way as the renormalization of power divergence for an open Wilson line. After the Wilson-line renormalization, the long-link matrix elements are improved such that they contain at most logarithmic divergences. A nonperturbative determination of the mass counterterm can, for example, be done following the procedure based on the static-quark potential for the renormalization of Wilson loop in Ref.~\cite{Musch:2010ka}. 

The first attempts to apply the LaMET approach to compute parton observables were the direct lattice computations of the unpolarized, helicity and transversity isovector quark distributions~\cite{Lin:2014gaa,Lin:2014yra,Lin:2014zya,Alexandrou:2015rja,Chen:2016utp,Alexandrou:2016jqi}. 
Although the current lattice systematics are not yet fully accounted for, a sea-flavor asymmetry has been qualitatively seen in both the unpolarized and linearly polarized cases, part of which 
has been confirmed in the updated measurements by the STAR~\cite{Adamczyk:2014xyw} and PHENIX~\cite{Adare:2015gsd}
collaborations. The Drell-Yan experiments at FNAL (E1027+E1039) and future EIC data will be able to give more insight into the sea asymmetry in the transversely polarized nucleon. 

In this paper, we present the first direct lattice-QCD results for the Bjorken-$x$ dependence  of the pion DA 
using lattice gauge ensembles with $N_f=2+1+1$ highly improved staggered
quarks (HISQ)~\cite{Follana:2006rc} (generated by the MILC Collaboration~\cite{Bazavov:2012xda}) and clover valence fermions with pion mass $310$~MeV.
%
In the framework of LaMET, the pion LCDA $\phi(x)$ can be studied from the IMF limit of the following quasi-correlation 
\beq\label{quasiDA}
{\tilde \phi}(x, P_z)={\frac{i}{f_{\pi}}}\int\frac{dz}{2\pi}e^{-i(x-1)P_z z}\langle \pi(P)|\bar\psi(0)\gamma^z\gamma_5 \Gamma(0,z)\psi(z)|0\rangle
\eeq
with the two quark fields separated along the spatial $z$ direction. As shown in Ref.~\cite{Ji:2015qla}, the pion LCDA can be related to the quasi-DA by the following matching formula
\beq\label{pionDA1loopmatching}
{\tilde \phi}(x, \Lambda, P_z)=\int_0^1 dy\, Z_\phi(x, y, \Lambda, \mu, P_z)\phi(y, \mu)+\mathcal{O}\left(\frac{\Lambda^2_{QCD}}{P_z^2},\frac{m^2_{\pi}}{P_z^2}\right) ,
\eeq
where $\Lambda=\pi/a$ is the UV cutoff for the quasi-DA with $a$ the lattice spacing. $\mu$ denotes the $\overline{\text{MS}}$ renormalization scale of the pion LCDA. Using Eq.~\ref{pionDA1loopmatching}, we will be able to recover the pion LCDA.

The paper is organized as follows: We will 
start by discussing the finite-momentum corrections for the quasi-DA computed on the lattice in Sec.~\ref{sec:corrections}, and then present the lattice results in Sec.~\ref{sec:num}. We first show the results without Wilson-line renormalization to remove the power divergence, and then explore the impact of Wilson-line renormalization where the mass counterterm is determined by using the static-quark potential for the renormalization of Wilson loop discussed in Ref.~\cite{Musch:2010ka}. Finally we summarize in Sec.~\ref{sec:sum}. The details of the finite-momentum corrections are given in the Appendices.
