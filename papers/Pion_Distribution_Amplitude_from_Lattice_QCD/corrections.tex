%%%%%%%%%%%%%%%%%%%%%%%%%%%%%%%%%%%%%%%%%%%%%%%%%%%%%%%%%%%%%%%%%%%%%%%%%%%%%%%%
%\section{Finite $P_z$ Corrections for Pion DA}\label{sec:corrections}
%%%%%%%%%%%%%%%%%%%%%%%%%%%%%%%%%%%%%%%%%%%%%%%%%%%%%%%%%%%%%%%%%%%%%%%%%%%%%%%%


In this section, we present the finite-momentum corrections needed for the calculation of pion DA. In the limit $P_z \rightarrow \infty$, the matching becomes the most
important $P_z$ correction. The factor
$Z_\phi$
has been computed up to one loop in
Ref.~\cite{Ji:2015qla} using a momentum-cutoff regulator instead of a
lattice regulator. Therefore, this $Z$ factor is accurate up to the
leading logarithm but not for the numerical constant. Determining this
constant requires a calculation using lattice perturbation theory with the same lattice
action.

At tree level, the $Z_\phi$ factor is just a delta function. Up to one-loop level, we can write
\begin{equation}
Z_\phi(x, y) = \delta (x-y) + \frac{\alpha_s}{2\pi} \overline{Z}_\phi(x, y)
  + \mathcal{O}\left(\alpha_s^2 \right),
\end{equation}
such that
\begin{equation}
\tilde{\phi}(x) \simeq \phi(x)
  + \frac{\alpha_s}{2\pi} \int dy\,
    \overline{Z}_\phi\!\left(x, y\right)\phi(y).
\end{equation}
Since the difference between $\tilde{\phi}(x)$ and $\phi(x)$ starts at
the loop level, we can rewrite the above equation as
\begin{equation}
\phi(x) \simeq \tilde{\phi}(x)
  - \frac{\alpha_s}{2\pi} \int dy\,
    \overline{Z}_\phi\!\left(x, y\right)\tilde{\phi}(y)
\label{q}
\end{equation}
with an error of $\mathcal{O}\left(\alpha_s^2\right)$~\cite{Ma:2014jla}. As in the parton distribution, $\overline{Z}_\phi(x, y)$ can be written as
%
\begin{equation}\label{eq7}
\overline{Z}_\phi(x, y) = \left(Z_\phi^{(1)}(x, y) - C\delta(x-y)\right),
\end{equation}
with the first term coming from gluon emission and the second term from the quark
self-energy diagram, $C=\int_{-\infty}^{\infty} d x'\,Z_\phi^{(1)}(x', y)$. (This implies $\int dx \phi(x)=\int dx \tilde{\phi}(x)$ at one loop, which follows from the conservation of the non-singlet axial current when quark masses are neglected.) Using this, Eq.~\ref{q} becomes
\begin{equation}
\phi(x) \simeq \tilde{\phi}(x)
  - \frac{\alpha_s}{2\pi} \int_{-\infty}^{\infty}\!dy\,
  \left[ Z_\phi^{(1)}\!\left(x, y\right)
    \tilde{\phi}(y)
  - Z_\phi^{(1)}\!\left(y, x\right)
    \tilde{\phi}(x)\right] ,  \label{cov}
\end{equation}
where for simplicity we have extended the integration range of $y$ to infinity, which introduces an error at higher order. The expression for the matching factor $Z_\phi^{(1)}(x, y)$ is given in Appendix~A.


For a finite $P_z$, we need to take into account {the $\mathcal{O}\left(m^2_{\pi}/P_z^2\right)$
meson-mass and $\mathcal{O}\left(\Lambda^2_\text{QCD}/P_z^2\right)$} higher-twist corrections. Following a procedure similar to Ref.~\cite{Chen:2016utp}, we can derive the mass corrections to all orders in $m^2_{\pi}/P_z^2$, which leads to the following relation between the pion DAs (for details see Appendix~B).
\begin{align}\label{masscorr}
\phi(x)&=\sqrt{1+4c}\sum_{n=0}^\infty \frac{(4c)^n}{f_+^{2n+1}}\Big[(1+(-1)^n)\tilde\phi\Big(\frac{1}{2}-\frac{f_+^{2n+1}(1-2x)}{4(4c)^n}\Big)+(1-(-1)^n)\tilde\phi\Big(\frac{1}{2}+\frac{f_+^{2n+1}(1-2x)}{4(4c)^n}\Big)\Big],
\end{align}
where $c=m_\pi^2/4P_z^2$ and $ f_{+}=\sqrt{1+4c}+ 1$.


The $\mathcal{O}\left(\Lambda^2_\text{QCD}/P_z^2\right)$ correction can be derived in the same way as in Ref.~\cite{Chen:2016utp}, since the twist-4 operator involved is the same. The twist-4 effect can be implemented by adding a
$\tilde{\phi}_\text{twist-4}$ contribution to $\tilde{\phi}$, such that
\begin{equation}\label{highertwist}
\tilde{\phi}(x,\Lambda,P_z) \rightarrow
  \tilde{\phi}(x,\Lambda,P_z) + \tilde{\phi}_\text{twist-4}(x,\Lambda,P_z),
\end{equation}
where
\begin{equation}
\tilde{\phi}_\text{twist-4}(x,\Lambda,P_z) =
  \frac{1}{8\pi}\int_{-\infty}^\infty\!dz\,\Gamma_0 \left(-ixzP_z\right)
  \left\langle \pi(P)\left\vert \mathcal{O}_\text{tr}(z)\right\vert 0\right\rangle,
  \label{t4-1}
\end{equation}
$\Gamma_0$ is the incomplete Gamma function and
\begin{align}
\mathcal{O}_\text{tr}(z) &=
  \int_0^z\!dz_1\,\bar{\psi}(0) \Big[ \gamma^\nu\gamma_5 \Gamma\left(0,z_1\right)
    D_\nu\Gamma \left(z_1,z\right) \notag \\
&{} + \int_0^{z_1}\!dz_2\, \lambda \cdot \gamma\gamma_5
  \Gamma\left(0,z_2 \right) D^\nu \Gamma \left(z_2,z_1\right)
  D_\nu \Gamma\left(z_1,z\right) \Big] \psi (z\lambda)  \label{t4-2}
\end{align}
with $\lambda^\mu=(0,0,0,-1)$. Eqs.~\ref{cov}--\ref{highertwist} take into account the one-loop, mass and higher-twist corrections, respectively. We need to implement them step by step to achieve the final pion DA. For the higher-twist corrections, instead of computing them directly on the lattice, we only
parametrize and fit them as a $1/P_z^2$ correction after we have
removed other leading-$P_z$ corrections, as was done in Ref.~\cite{Chen:2016utp}.

%
