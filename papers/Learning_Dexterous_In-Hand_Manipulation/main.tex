\documentclass{article}
\pdfoutput=1

\PassOptionsToPackage{numbers}{natbib}
\usepackage{titletoc}
\usepackage{hyperref}
\usepackage[page,header]{appendix}
\usepackage[final]{nips_2017}
\usepackage[utf8]{inputenc} % allow utf-8 input
\usepackage[T1]{fontenc}    % use 8-bit T1 fonts
\usepackage{url}            % simple URL typesetting
\usepackage{booktabs}       % professional-quality tables
\usepackage{amsfonts}       % blackboard math symbols
\usepackage{nicefrac}       % compact symbols for 1/2, etc.
\usepackage{microtype}      % microtypography
\usepackage{xcolor}
\usepackage{hyperref}
\usepackage{graphicx}
\usepackage{algorithmicx}
\usepackage{algorithm}% http://ctan.org/pkg/algorithms
\usepackage{algpseudocode}% http://ctan.org/pkg/algorithmicx
\usepackage{mathrsfs}
\usepackage{amssymb}
\usepackage{graphicx,wrapfig,lipsum}
\usepackage{natbib}
\usepackage{caption}
\usepackage{float}
\usepackage{subcaption}
\usepackage{enumitem}
\usepackage{siunitx}
\usepackage{multirow}
\usepackage{tikz}
\usepackage{footnote}
\setlength{\bibsep}{0pt plus 0.3ex}
\usetikzlibrary{arrows,shapes,snakes,automata,backgrounds,petri,positioning,calc}
\sisetup{detect-all}
\usepackage{lipsum}
\usepackage[multiple]{footmisc}
\usepackage{placeins}
\newcommand{\figref}[1]{Fig.~\ref{#1}}
\newcommand{\tblref}[1]{Table~\ref{#1}}
\newcommand{\secref}[1]{Section~\ref{#1}}
\renewcommand{\eqref}[1]{Equation~(\ref{#1})}

\def\availableat{\url{url-published-on-acceptance}}

\newcommand{\todo}[1]{{\color{red} TODO: {#1}}}
\newcommand{\newstuff}[1]{{\color{red} CHECK: {#1}}}
%\newcommand{\todo}[1]{{}}

\newcommand{\ckp}[2]{$CK_{#1}P_{#2}$}
\newcommand{\cext}{\ckp{8}{16}$ext$}
\newcommand{\cfin}{$F_{CK_{X}P_{Y}}$}
\newcommand{\cray}{\ckp{8}{8}$ray$}
\newcommand{\csin}{$SK_{8}P_{8}$}
\newcommand{\casin}{$SK_{combined}$}
%\renewcommand{\cext}{$SK_{8}K_{8}P_{8}$}
\newcommand{\ckpnl}[2]{$CK_{#1}P_{#2}nl$}


\makeatletter
\newcommand{\Spvek}[2][r]{%
	\gdef\@VORNE{1}
	\left(\hskip-\arraycolsep%
	\begin{array}{#1}\vekSp@lten{#2}\end{array}%
	\hskip-\arraycolsep\right)}

\def\vekSp@lten#1{\xvekSp@lten#1;vekL@stLine;}
\def\vekL@stLine{vekL@stLine}
\def\xvekSp@lten#1;{\def\temp{#1}%
	\ifx\temp\vekL@stLine
	\else
	\ifnum\@VORNE=1\gdef\@VORNE{0}
	\else\@arraycr\fi%
	#1%
	\expandafter\xvekSp@lten
	\fi}
\makeatother

\newcommand\blfootnote[1]{%
  \begingroup
  \renewcommand\thefootnote{}\footnote{#1}%
  \addtocounter{footnote}{-1}%
  \endgroup
}

\renewcommand{\sectionautorefname}{Section}
\renewcommand{\subsectionautorefname}{Section}
\renewcommand{\subsubsectionautorefname}{Section}

\bibliographystyle{abbrv}

\hyphenation{Phase-Space}


\title{Learning Dexterous In-Hand Manipulation}
\newcommand{\asp}{~~~~}
\author{\\
\textbf{OpenAI}\thanks{Please use the following bibtex for citation: \url{https://openai.com/bibtex/openai2018learning.bib}},  Marcin~Andrychowicz, Bowen~Baker, Maciek~Chociej,\\
Rafał~Józefowicz, Bob~McGrew, Jakub~Pachocki, Arthur~Petron,\\
Matthias~Plappert, Glenn~Powell, Alex~Ray, Jonas~Schneider, Szymon~Sidor, \\
Josh~Tobin, Peter~Welinder, Lilian~Weng, Wojciech~Zaremba
}


\newcommand{\johncomment}[1]{[{\color{red} John: #1}]}

\begin{document}
\maketitle

\begin{figure}[h!]
\centering
\includegraphics[width=1.0\textwidth]{figures/hero} \\ \vspace{0.3cm}
\includegraphics[height=1cm]{figures/hero-legend} \\ \vspace{0.1cm}
\caption{A five-fingered humanoid hand trained with reinforcement learning manipulating a block from an initial configuration to a goal configuration using vision for sensing.}
\end{figure}



\begin{abstract}
We use reinforcement learning (RL) to learn dexterous in-hand manipulation
policies which can perform vision-based object reorientation on a physical
Shadow Dexterous Hand.
The training is performed in a simulated environment
in which we randomize many of the physical properties of the system like
friction coefficients and an object's appearance.
Our policies transfer to the physical robot
despite being trained entirely in simulation.
Our method does not rely on any human demonstrations, but many behaviors found in human manipulation emerge naturally, including finger gaiting, multi-finger coordination, and
the controlled use of gravity.
Our results were obtained using the same distributed RL system
that was used to train OpenAI~Five~\citep{five}.
We also include a video of our results: \url{https://youtu.be/jwSbzNHGflM}.
\end{abstract}

\startcontents[mainsections]


\section{Introduction}

\begin{figure}\centering
    \includegraphics[width=\textwidth]{figures/overview}
    \caption{
        System Overview. (a) We use a large distribution of simulations with randomized parameters and appearances to collect data for both the control policy and vision-based pose estimator. (b) The control policy receives observed robot states and rewards from the distributed simulations and learns to map observations to actions using a recurrent neural network and reinforcement learning. (c) The vision based pose estimator renders scenes collected from the distributed simulations and learns to predict the pose of the object from images using a convolutional neural network (CNN), trained separately from the control policy. (d) To transfer to the real world, we predict the object pose from 3 real camera feeds with the CNN, measure the robot fingertip locations using a 3D motion capture system, and give both of these to the control policy to produce an action for the robot.
    }
    \label{fig:overview}
\end{figure}


While dexterous manipulation of objects is a fundamental everyday task for humans,
it is still challenging for autonomous robots.
Modern-day robots are typically designed for specific tasks in constrained settings and are largely unable to utilize complex end-effectors.
In contrast, people are able to perform a wide range of dexterous manipulation tasks in a diverse set of environments, making the human hand a grounded source of inspiration for research into robotic manipulation.

The Shadow Dexterous Hand~\citep{shadow-robot} is an example of a robotic hand designed for human-level dexterity; it has five fingers with a total of \num{24} degrees of freedom.
The hand has been commercially available since 2005; however it still has not seen widespread adoption, which can be attributed to the daunting difficulty of controlling systems of such complexity.
The state-of-the-art in controlling five-fingered hands is severely limited.
Some prior methods have shown promising in-hand manipulation results in simulation but do not attempt to transfer to a real world robot \citep{DBLP:conf/icra/BaiL14, DBLP:conf/sca/MordatchPT12}.
Conversely, due to the difficulty in modeling such complex systems, there has also been work in approaches that only train on a physical robot \citep{falco2018policy, DBLP:conf/humanoids/HoofHN015, DBLP:journals/corr/KumarGTL16, DBLP:conf/icra/KumarTL16}.
However, because physical trials are so slow and costly to run, the learned behaviors are very limited.

In this work, we demonstrate methods to train control policies that perform in-hand manipulation % of a block and an octagonal prism,
and deploy them on a physical robot.
The resulting policy exhibits unprecedented levels of dexterity and naturally discovers grasp types found in humans, such as the tripod, prismatic, and tip pinch grasps, 
and displays contact-rich, dynamic behaviours such as finger gaiting, multi-finger coordination, the controlled use of gravity, and coordinated application of translational and torsional forces to the object.
Our policy can also use vision to sense an object's pose --- an important aspect for robots that should ultimately work outside of a controlled lab setting.

Despite training entirely in a simulator which substantially differs from the real world,
we obtain control policies which perform well on the physical robot.
We attribute our transfer results to (1) extensive randomizations and added effects in the simulated environment alongside calibration, (2) memory augmented control polices which admit the possibility to learn adaptive behaviour and implicit system identification on the fly, and (3) training at large scale with distributed reinforcement learning.
An overview of our approach is depicted in \autoref{fig:overview}.


The paper is structured as follows.
\autoref{sec:setup} gives a system overview, describes the proposed task in more detail, and shows the hardware setup. \autoref{sec:randomizations} describes observations for the control policy, environment randomizations, and additional effects added to the simulator that make transfer possible.
\autoref{sec:train-policy} outlines the control policy training procedure and the distributed RL system.
\autoref{sec:train-vision} describes the vision model architecture and training procedure.
Finally, \autoref{sec:results} describes both qualitative and quantitative results from deploying the control policy and vision model on a physical robot. %, achieving highly


\section{Task and System Overview}
\label{sec:setup}
\vspace{-0.1in}
\section{Neural Program Synthesis from Input-Output Examples}
\vspace{-0.1in}
In programming by example tasks, the program specification is a set of input-output examples~\cite{devlin2017robustfill,bunel2018leveraging}. Specifically, we provide the synthesizer with a set of $K$ input-output pairs $\{(I^{(k)}, O^{(k)})\}_{k=1}^K$ ($\{IO\}^K$ in short). These input-output pairs are annotated with a ground truth program $P^\star$, so that $P^\star(I^{(k)})=O^{(k)}$ for any $k \in \{1, 2, ..., K\}$. To measure the program correctness, we include another set of held-out test cases $\{IO\}_{test}^{K_{test}}$ that differs from $\{IO\}^K$. The goal of the program synthesizer is to predict a program $P$ from $\{IO\}^K$, so that $P(I)=P^\star(I)=O$ for any $(I, O) \in \{IO\}^K + \{IO\}_{test}^{K_{test}}$.

%\label{sec:c-data}
\textbf{C Program Synthesis}. In this work, we make the first attempt of synthesizing C code in a restricted domain from input-output examples only, and we focus on programs for list processing. List processing tasks have been studied in some prior works on input-output program synthesis, but they synthesize programs in restricted domain-specific languages instead of full-fledged popular programming languages~\cite{balog2016deepcoder,odena2020learning,odena2020bustle}. 

Our C code synthesis problem brings new challenges for programming by example. Compared to domain-specific languages, the syntax and semantics of C are much more complicated, which significantly enlarges the program search space. Meanwhile, learning good representations for partially decoded programs also becomes more difficult. In particular, prior neural program synthesizers that utilize per-line interpreters for the programming language to guide the synthesis and representation learning~\cite{chen2018execution,shin2018improving,nye2020representing,Ellis2019WriteEAExtendExecution,odena2020bustle} are not directly applicable to C. Although it is possible to dump some intermediate variable states during C code execution~\cite{campbell2012executable}, since partial C programs are not executable, we are able to obtain all the execution states only until a full C code is generated, which is too late to include them in the program decoding process. In particular, the intermediate execution state is not available when the partial program is syntactically invalid, and this happens more frequently for C due to its syntax design.
\begin{figure}
    \centering
    \includegraphics[width=\textwidth]{fig/c-program-synthesis-crop.pdf}
\caption{\small Illustration of the C program synthesis pipeline. For dataset construction, we develop a random program generator to sample random C programs, then execute the program over randomly generated inputs and obtain the outputs. The input-output pairs are fed into the neural program synthesizer to predict the programs. Note that the synthesized program can be more concise than the original random program.}
\label{fig:ex-c}
\end{figure}



\section{Transferable Simulations}
\label{sec:randomizations}
\subsection{Randomizations}
\label{app:randomizations}

A variety of randomizations are applied to the simulator, shrinking the reality gap between the simulated environment and the physical world in order to learn a policy that generalizes to reality.


\paragraph{Physical parameters.}
The physical parameters are sampled at the beginning of every episode and held fixed for the whole episode. The full set of randomized values are displayed in~\autoref{table:rand-physics}.


\paragraph{Observation noise.}
We use two types of noise ---
\textit{correlated} noise which is sampled once per episode and kept fixed,
and an \textit{uncorrelated} Gaussian one.
Apart from Gaussian correlated noise, we also add more structured noise
coming from inaccurate placement of the motion capture markers
by computing the observations using slightly misplaced
markers in the simulator.
The configuration of noise levels is described in~\autoref{table:obs-noise}.
The observation noise is only applied to the policy inputs and not to
the value network inputs as the value function is not used during the deployment on the physical system.







\begin{table}
    \footnotesize
    \centering
    \caption{Standard deviation of observation noise.}
    \renewcommand{\arraystretch}{1.3}
    \begin{tabular}{@{}lll@{}}
        \toprule
        \textbf{Measurement} & \textbf{Correlated noise} & \textbf{Uncorrelated noise} \\ \midrule
        fingertips positions & $1$mm & $2$mm \\
        object position & $5$mm & $1$mm \\
        object orientation & $0.1$rad & $0.1$rad \\ \hline
        fingertip marker positions & $3$mm & \\
        hand base marker position & $1$mm & \\
        \bottomrule
    \end{tabular}
\label{table:obs-noise}
\end{table}

\begin{table}
    \footnotesize
    \centering
    \caption{Standard deviation of action noise.}
    \renewcommand{\arraystretch}{1.3}
    \begin{tabular}{@{}ll@{}}
        \toprule
        \textbf{Noise type} & \textbf{Percentage of the action range} \\ \midrule
        uncorrelated additive & 5\% \\
        correlated additive & 1.5\% \\
        uncorrelated multiplicative & 1.5\% \\
        \bottomrule
    \end{tabular}
\label{table:action-noise}
\end{table}

\paragraph{PhaseSpace tracking errors.}
Noise aside, readings of the motion capture markers from the PhaseSpace system might be occasionally unavailable for a short period of time due to instability of the service. To model such error in the simulator, we mask the fingertip markers with a small probability (0.2 per second) for 1 second so that the policy has a chance to learn how to interact with the environment while the system temporarily loses track of some markers. 

Furthermore, the markers might be occluded while in motion, causing a brief delay of readings of some fingertip positions. In the simulator, a small weightless cuboid site\footnote{A site represents a location of interest relative to the body frame in MuJoCo. Also see \url{http://mujoco.org/book/modeling.html\#site}.} is attached to the back of each nail and we consider a marker occluded whenever a collision with the site is detected as another finger or object is getting too close. If a fingertip marker is deemed occluded, we use its last available position readings instead of the current one.



\paragraph{Action noise and delay.}
We add correlated and uncorrelated Gaussian noise to all actions to account for an imperfect actuation 
system.
The detailed noise levels can be found in~\autoref{table:action-noise}.
Moreover, the real system contains many potential sources of delays between the time that observations are sensed and actions are executed,
from network delay to the computation time of the neural network.
Therefore, we introduce a simple model of action delay to the simulator.
At the beginning of every episode we sample for every actuator whether
it is going to be delayed (with probability $0.5$) or not.
The actions corresponding to delayed actuator are delayed by one environment step, i.e. approximately $80$ms.


\paragraph{Timing randomization.}
We also randomize the timing of environment steps.
Every environment step is simulated as $10$ MuJoCo physics simulator steps
with $\Delta t=8\mbox{ms}+\mbox{Exp}(\lambda)$, where 
$\mbox{Exp}(\lambda)$ denotes the exponential distribution
and the coefficient $\lambda$ is once per episode sampled uniformly from the
range $[1250,10000]$.

\paragraph{Backlash model.}
The physical Shadow Dexterous Hand is tendon-actuated which causes a substantial
amount of backlash, while the MuJoCo model assumes direct actuation on the joints.
In order to account for it, we introduce a simple model of backlash which modifies
actions before they are sent to MuJoCo.
In particular, for every joint we have two parameters which
specify the amount of backlash in each direction, and are denoted $\delta_{-1}$ and $\delta_{+1}$,
as well as a time varying variable $s$ denoting the current state of slack.
We obtained the values of $\delta_{-1}, \delta_{+1}$ through
calibration.
At the beginning of every episode we sample the values of $\delta_{-1}, \delta_{+1}$
from the Gaussian distribution centered around the calibrated values with the standard deviation of $0.1$.
Let $a_{\text{in}} \in [-1,\,1]$ be an action specified by the policy.
Our backlash model works as follows: we compute the new value of the slack variable
$s'=[s+a_{\text{in}} \delta_{\textbf{sgn}(a_\text{in})} \Delta t]_{-1}^{+1}$,
compute the scaling factor
$\alpha=1-\left[\frac{|\textbf{sgn}(a_\text{in})-s|}{|s'-s|+\epsilon}\right]_0^1$,
where $\epsilon=10^{-12}$ is a constant used for numerical stability,
and finally multiply the action by $\alpha$:
$a_{\text{out}} = \alpha a_{\text{in}}$.
    
\paragraph{Random forces on the object.}
To represent unmodeled dynamics, we sometimes apply random forces on the object.
The probability $p$ that a random force is applied is sampled at the beginning of the episode from the loguniform distribution between $0.1\%$ and $10\%$.
Then, at every timestep, with probability $p$ we apply a random force
from the $3$-dimensional Gaussian distribution with the standard deviation equal to $1~m/s^2$ times the mass of the object on each coordinate and decay the force with the coefficient of $0.99$ per $80$ms.


\paragraph{Randomized vision appearance.} 
We randomize the visual appearance of the robot and object, as well as lighting and camera characteristics.
The materials and textures are randomized for every visible object in the scene.
We randomize the hue, saturation, and value for the object faces around calibrated values from real-world measurements.
The color of the robot is uniformly randomized. Material properties such as glossiness and shininess are randomized as well. Camera position and orientation are slightly randomized around values calibrated to real-world locations.
Lights are randomized individually, and intensities are scaled based on a randomly drawn total intensity. After rendering the scene to images from the three separate cameras, additional augmentation is applied.
The images are linearly normalized to have zero mean and unit variance.
Then the image contrast is randomized, and finally per-pixel Gaussian noise is added. Details are in~\autoref{table:vision-randomization}.

\begin{savenotes}
\begin{table}
    \footnotesize
    \centering
    \caption{Vision randomizations.}
    \renewcommand{\arraystretch}{1.3}
    \begin{tabular}{@{}ll@{}}
        \toprule
        \textbf{Randomization type} & \textbf{Range} \\ \midrule
        number of cameras & 3 \\
        camera position & $\pm$ 1.5 mm \\
        camera rotation & 0--3$^{\circ}$ around a random axis \\
        camera field of view & $\pm$ 1$^{\circ}$ \\
        \hline
        robot material colors & RGB \\
        robot material metallic level & 5\%--25\%\footnote{In units used by Unity. See \url{https://unity3d.com/learn/tutorials/s/graphics}.} \\
        robot material glossiness level & 0\%--100\%\footnotemark[\value{footnote}] \\
        \hline
        object material hue & calibrated hue $\pm$ 1\% \\
        object material saturation & calibrated saturation $\pm$ 15\% \\
        object material value & calibrated value $\pm$ 15\% \\
        object metallic level & 5\%--15\%\footnotemark[\value{footnote}] \\
        object glossiness level & 5\%--15\%\footnotemark[\value{footnote}] \\
        \hline
        number of lights & 4--6 \\
        light position & uniform over upper half-sphere \\
        light relative intensity & 1--5 \\
        total light intensity & 0--15\footnotemark[\value{footnote}] \\
        \hline
        image contrast adjustment & 50\%--150\% \\
        additive per-pixel Gaussian noise & $\pm$ 10\% \\
        \bottomrule
    \end{tabular}
\label{table:vision-randomization}
\end{table}
\end{savenotes}

    


\section{Learning Control Policies From State}
\label{sec:train-policy}



\subsection{Policy Architecture}


Many of the randomizations we employ persist across an episode, and thus it should be possible for a memory augmented policy to identify properties of the current environment and adapt its own behavior accordingly.
For instance, initial steps of interaction with the environment can reveal the weight of the object or how fast the index finger can move.
We therefore represent the policy as a recurrent neural network with memory, namely an LSTM~\citep{lstm}
with an additional hidden layer with ReLU~\citep{relu} activations inserted
between inputs and the LSTM.

The policy is trained with Proximal Policy Optimiztion (PPO)~\citep{ppo}.
We provide background on reinforcement learning and PPO in greater detail in~\autoref{sec:rl}.
PPO requires the training of two networks --- a policy network, which maps observations to actions, and a value network, which predicts the discounted sum of future rewards starting from a given state.
Both networks have the same architecture but have independent parameters.
Since the value network is only used during training, we use an Asymmetric Actor-Critic~\citep{pinto2017asymmetric} approach.
Asymmetric Actor-Critic exploits the fact that the value network can have access to information that is not available on the real robot system.\footnote{This includes noiseless observation and additional observations like joint angles and angular velocities, which we cannot sense reliably but which are readily available in simulation during training.}
This potentially simplifies the problem of learning good value estimates since less information needs to be inferred.
The list of inputs fed to both networks can be found in \autoref{table:policy-inputs}.



\makesavenoteenv{table}
\makesavenoteenv{tabular}
\begin{table}[h!]
    \footnotesize
    \centering
    \caption{Observations of the policy and value networks, respectively.}
    \renewcommand{\arraystretch}{1.3}
    \begin{tabular}{@{}llcc@{}}
        \toprule
        \textbf{Input} & \textbf{Dimensionality} & \textbf{Policy network} & \textbf{Value network} \\
        \midrule
        fingertip positions & 15D & \checkmark & \checkmark \\
        object position & 3D & \checkmark & \checkmark \\
        object orientation & 4D (quaternion) & $\times$\footnote{We accidentally did not include the current object orientation in the policy observations but found that it makes little difference since this information is indirectly available through the relative target orientation.} & \checkmark \\
        target orientation & 4D (quaternion) & $\times$ & \checkmark \\
        relative target orientation & 4D (quaternion) & \checkmark & \checkmark \\
        hand joints angles & 24D & $\times$ & \checkmark \\
        hand joints velocities & 24D & $\times$ & \checkmark \\
        object velocity & 3D & $\times$ & \checkmark \\
        object angular velocity & 4D (quaternion) & $\times$ & \checkmark \\
        \bottomrule
    \end{tabular}
\label{table:policy-inputs}
\end{table}


\newcommand{\link}[5]{
    \draw[->,flow,#3] ([xshift=-5pt]#1.south) -- node[left]{#4} ([xshift=-5pt]#2.north);
    \draw[->,flow,#3] ([xshift=5pt]#2.north) -- node[right]{#5} ([xshift=5pt]#1.south);
}

\newcommand{\optthread}[2]{
    \begin{scope}
      \node [thread,#2] (train#1) {optimizer};
      \node [physical,below=0.5cm of train#1] (gpu#1) {GPU};
      \node [thread,below=0.5cm of gpu#1] (stager#1) {stager};
      \node [physical,below=0.5cm of stager#1] (ram#1) {RAM};
      \node [thread,below=0.5cm of ram#1] (puller#1) {puller};
      \link{train#1}{gpu#1}{}{}{}
      \link{gpu#1}{stager#1}{}{}{}
      \link{stager#1}{ram#1}{}{}{}
      \link{ram#1}{puller#1}{}{}{}
    \end{scope}
}

\newcommand{\redis}[1]{
    \node[db,below=1.5cm of puller#1] (redis#1) {redis};
    \link{puller#1}{redis#1}{dashed}{parameters}{experience}
      
    \node[thread,below=1.5cm of redis#1] (w1) {worker};
    \node[thread,below right=0.1cm and 0.1cm of w1.north west] (w2) {worker};
    \node[thread,below right=0.1cm and 0.1cm of w2.north west] (w3) {worker};
    \node[thread,below right=0.1cm and 0.1cm of w3.north west] (w4) {worker};
    \node[thread,below right=0.1cm and 0.1cm of w4.north west] (w5) {worker};
    \link{redis#1}{w1}{dashed}{parameters}{experience}
      
}

\subsection{Actions and Rewards}

Policy actions correspond to desired joints angles relative to the current ones\footnote{The reason
for using \emph{relative} targets is that it is hard to precisely measure absolute joints angles
on the physical robot. See Appendix~\ref{app:hardware} for more details.}
(e.g. rotate this joint by $10$ degrees).
While PPO can handle both continuous and discrete action spaces, we noticed that
discrete action spaces work much better. This may be because a discrete probability distribution is more expressive than a multivariate Gaussian or because discretization of actions makes learning a good advantage function potentially simpler.
We discretize each action coordinate into $11$ bins.%\footnote{The action distribution

The reward given at timestep $t$ is $r_t=d_t-d_{t+1}$, where
$d_t$ and $d_{t+1}$ are the rotation angles between the desired 
and current object orientations before and after the transition, respectively.
We give an additional reward of $5$ whenever a goal is achieved and a reward of $-20$ (a penalty) whenever the object is dropped.
More information about the simulation environment
can be found in Appendix~\ref{app:sim}.


\subsection{Distributed Training with Rapid}
We use the same distributed implementation of PPO that was used to train OpenAI Five~\citep{five} without any modifications.
Overall, we found that PPO scales up easily and requires little hyperparameter tuning. %, and transferred successfully to a broad variety of environments.
The architecture of our distributed training system is depicted in~\autoref{fig:rapid}.

\begin{figure}[h]
    \centering
    \includegraphics[width=0.7\textwidth]{figures/policy2}
    \caption{Our distributed training infrastructure in Rapid. Individual threads are depicted as blue squares. Worker machines randomly connect to a Redis server from which they pull new policy parameters and to which they send new experience. The optimizer machine has one MPI process for each GPU, each of which gets a dedicated Redis server. Each process has a \emph{Puller} thread which pulls down new experience from Redis into a buffer. Each process also has a \emph{Stager} thread which samples minibatches from the buffer and stages them on the GPU. Finally, each \emph{Optimizer} thread uses a GPU to optimize over a minibatch after which gradients are accumulated across threads and new parameters are sent to the Redis servers.}
    \label{fig:rapid}
\end{figure}

In our implementation, a pool of $384$ worker machines, each with $16$ CPU cores, generate experience by rolling out the current version of the policy in a sample from distribution of randomized simulations.
Workers download the newest policy parameters from the optimizer at the beginning of every epoch,
generate training episodes, and send the generated episodes back to the optimizer.
The communication between the optimizer and workers is implemented using the
Redis in-memory data store.
We use multiple Redis instances for load-balancing, and workers are assigned to an instance randomly.
This setup allows us to generate about $2$ years of simulated experience per hour.

The optimization is performed on a single machine with $8$ GPUs.
The optimizer threads pull down generated experience from Redis
and then stage it to their respective GPU's memory for processing.
After computing gradients locally, they are averaged across all threads using MPI, which we then use to update
the network parameters.

The hyperparameters that we used can be found in Appendix~\ref{app:hyper-ppo}.


\section{State Estimation from Vision}
\label{sec:train-vision}
The policy that we describe in the previous section
takes the object's position as input and requires a motion capture system for tracking the object
on the physical robot.
This is undesirable because tracking objects with such a system is only feasible in a lab setting where markers can be placed on each object.
Since our ultimate goal is to build robots for the real world that can interact with arbitrary objects, sensing them using vision is an important building block.
In this work, we therefore wish to infer the object's pose from vision alone.
Similar to the policy, we train this estimator only on synthetic data coming from the simulator. %data and do not require data from the real robot.

\subsection{Model Architecture}
\label{sec:vision_model_arch}

To resolve ambiguities and to increase robustness, we use three RGB cameras mounted with differing viewpoints of the scene.
The recorded images are passed through a convolutional neural network, which is depicted in \autoref{fig:vision-architecture}.
The network predicts both the position and the orientation of the object.
During execution of the control policy on the physical robot, we feed the pose estimator's prediction into the policy,
which in turn produces the next action.

\begin{figure}[h]
    \begin{minipage}[c]{0.45\textwidth}
        \includegraphics[width=0.8\textwidth]{figures/policy3}
    \end{minipage}\hfill
    \begin{minipage}[c]{0.55\textwidth}
        \caption{Vision network architecture. Camera images are passed through a convolutional feature stack that consists of two convolutional layers, max-pooling, 4 ResNet blocks~\cite{He2016DeepRL}, and spatial softmax (SSM)~\cite{finn2015deep} layers with shared weights between the feature stacks for each camera.
        The resulting representations are flattened, concatenated, and fed to a fully connected network. %(128 neurons).
        All layers use ReLU \cite{relu} activation function.
        Linear outputs from the last layer form the estimates of the position and orientation of the object.
        }
        \label{fig:vision-architecture}
    \end{minipage}\hfill
\end{figure}

\subsection{Training}


We run the trained policy in the simulator until we gather one million states.
We then train the vision network by minimizing the mean squared error between the normalized prediction and the ground-truth
with minibatch gradient descent.
For each minibatch, 
we render the images with randomized appearance before feeding them to the network.
Moreover, we augment the data by modifying the object pose.
We use 2 GPUs for rendering and 1 GPU for running the network and training.




Additional training details are available in Appendix~\ref{app:vision_training} and randomization details are in Appendix~\ref{app:randomizations}.


\section{Results}
\label{sec:results}
In this section, we evaluate the proposed system.
We start by deploying the system on the physical robot,
and evaluating its performance on in-hand manipulation of a block and an octagonal prism.
We then focus on individual aspects of our system:
We conduct an ablation study of the importance of randomizations
and policies with memory capabilities in order to successfully transfer.
Next, we consider the sample complexity of our proposed method. % and show that we only simulation and scaled-up reinforcement learning can solve the problem in a reasonable amount of time.
Finally, we investigate the performance of the proposed vision pose estimator and show that using only synthetic images is sufficient to achieve good performance.

\subsection{Qualitative Results}

During deployment on the robot as well as in simulation, we notice that our policies naturally exhibit many of the grasps found in humans (see \autoref{fig:grasps}).
Furthermore, the policy also naturally discovers many strategies for dexterous in-hand manipulation described by the robotics community~\citep{DBLP:conf/icar/MaD11} such as finger pivoting, finger gaiting,
multi-finger coordination, the controlled use of gravity, and coordinated application of translational and torsional forces to the object.
It is important to note that we did not incentivize this directly: we do not use any human demonstrations and do not encode any prior into the reward function.

For precision grasps, our policy tends to use the little finger instead of the index or middle finger.
This may be because the little finger of the Shadow Dexterous Hand has an extra degree of freedom compared to the index, middle and ring fingers, making it more dexterous.
In humans the index and middle finger are typically more dexterous.
This means that our system can rediscover grasps found in humans, but adapt them to better fit the limitations and abilities of its own body.

\begin{figure}[h]
    \centering
    \includegraphics[width=0.32\textwidth]{figures/grasp-tip-pinch}
    \includegraphics[width=0.32\textwidth]{figures/grasp-palmar-pinch}
    \includegraphics[width=0.32\textwidth]{figures/grasp-tripod}\\ \vspace{0.05cm}
    \includegraphics[width=0.32\textwidth]{figures/grasp-quadpod}
    \includegraphics[width=0.32\textwidth]{figures/grasp-5-fingered-octo}
    \includegraphics[width=0.32\textwidth]{figures/grasp-default-octo}
    \caption{Different grasp types learned by our policy. From top left to bottom right: Tip Pinch grasp, Palmar Pinch grasp, Tripod grasp, Quadpod grasp, 5-Finger Precision grasp, and a Power grasp. Classified according to~\citep{grasp}.}
    \label{fig:grasps}
\end{figure}

We observe another interesting parallel between humans and our policy in finger pivoting, which is a strategy in which an object is held between two fingers and rotate around this axis.
It was found that young children have not yet fully developed their motor skills and therefore tend to rotate objects using the proximal or middle phalanges of a finger~\citep{pehoski1997hand}.
Only later in their lives do they switch to primarily using the distal phalanx, which is the dominant strategy found in adults.
It is interesting that our policy also typically relies on the distal phalanx for finger pivoting.

During experiments on the physical robot we noticed that the most common failure mode was dropping the object
while rotating the wrist pitch joint down.
Moreover, the vertical joint was the most common source of robot breakages, probably because
it handles the biggest load.
Given these difficulties,
we also trained a policy with the wrist pitch joint locked.\footnote{We had trouble training
in this environment from scratch, so we fine-tuned
a policy trained in the original environment instead.}
We noticed that not only does this policy transfer better to the physical robot but
it also seems to handle the object much more deliberately with many of the above grasps emerging frequently in this setting.
Other failure modes that we observed were dropping the object shortly after the start of a trial (which may be explained by incorrectly identifying some aspect of the environment) and getting stuck because the edge of an object got caught in a screw hole (which we do not model).



We encourage the reader to watch the accompanying video to get a better sense of the learned behaviors.\footnote{Real-time video of $50$ successful consecutive rotations: \url{https://youtu.be/DKe8FumoD4E}}




\subsection{Quantitative Results}
\label{sec:quant-results}
In this section we evaluate our results quantitatively.
To do so, we measure the number of \emph{consecutive} successful rotations until the object is either dropped, a goal has not been achieved within 80 seconds, or until $50$ rotations are achieved.
All results are available in~\autoref{table:perf}.




\begin{table}[h]
    \centering
    \renewcommand{\arraystretch}{1.1}
    \caption{The number of successful consecutive rotations in simulation and on the physical robot. All policies were trained on environments with all randomizations enabled. We performed 100 trials in simulation and 10 trails per policy on the physical robot. Each trial terminates when the object is dropped, 50 rotations are achieved or a timeout is reached. For physical trials, results were taken at different times on the physical robot.}
    \begin{tabular}{@{}llll@{}}
        \toprule
        \textbf{Simulated task} & \textbf{Mean} & \textbf{Median} & \textbf{Individual trials (sorted)} \\ 
        \midrule
        Block (state) & $43.4 \pm 13.8$ & $50$ & - \\
        Block (state, locked wrist) & $44.2 \pm 13.4$ & $50$ & - \\
        Block (vision) & $30.0 \pm 10.3$ & $33$ & - \\
        Octagonal prism (state) & $29.0 \pm 19.7$ & $30$ & - \\
        \midrule
        \textbf{Physical task} \\
        \midrule
        
        
        Block (state) & $18.8 \pm 17.1$  & $13$ & $50$, $41$, $29$, $27$, $14$, $12$, $6$, $4$, $4$, $1$ \\
        
        Block (state, locked wrist) & $26.4 \pm 13.4$ & $28.5$ & $50$, $43$, $32$, $29$, $29$, $28$, $19$, $13$, $12$, $9$ \\
        
        Block (vision) & $15.2 \pm 14.3$ & $11.5$ & $46$, $28$, $26$, $15$, $13$, $10$, $8$, $3$, $2$, $1$ \\
        Octagonal prism (state) & $7.8 \pm 7.8$ & $5$ & $27$, $15$, $8$, $8$, $5$, $5$, $4$, $3$, $2$, $1$ \\
        \bottomrule
    \end{tabular}
    \label{table:perf}
\end{table}

        
        
        
        


Our results allow us to directly compare the performance of each task in simulation and on the real robot.
For instance, manipulating a block in simulation achieves a median of $50$ successes while the median on the physical setup is $13$.
This is the overall trend that we observe: Even though randomizations and calibration narrow the reality gap, it still exists and performance on the real system is worse than in simulation. We discuss the importance of individual randomizations in greater detail in \autoref{sec:ablation-rand}.

When using vision for pose estimation, we achieve slightly worse results both in simulation and on the real robot. This is because even in simulation, our model has to perform transfer because it was only trained on images rendered with Unity but we use MuJoCo rendering for evaluation in simulation (thus making this a sim-to-sim transfer problem).
On the real robot, our vision model does slightly worse compared to pose estimation with PhaseSpace. However, the difference is surprisingly small, suggesting that training the vision model only in simulation is enough to achieve good performance on the real robot.
For vision pose estimation we found that it helps to use a white background and to wipe the object with a tack cloth between trials to remove detritus from the robot hand.

We also evaluate the performance on a second type of object, an octagonal prism.
To do so, we finetuned a trained block rotation control policy to the same randomized distribution of environments but with the octagonal prism as the target object instead of the block.
Even though our randomizations were all originally designed for the block, we were able to learn successful policies that transfer.
Compared to the block, however, there is still a performance gap both in simulation and on the real robot.
This suggests that further tuning is necessary and that the introduction of additional randomization could improve transfer to the physical system.

We also briefly experimented with a sphere but failed to achieve more than a few rotations in a row, perhaps
because we did not randomize any MuJoCo parameters related to rolling behavior or because rolling objects are much more sensitive to unmodeled imperfections in the hand such as screw holes.
It would also be interesting to train a unified policy that can handle multiple objects, but we leave this for future work.

Obtaining the results in \autoref{table:perf} proved to be challenging due to robot breakages during experiments.
Repairing the robot takes time and often changes some aspects of the system, which is why the results were obtained at different times.
In general, we found that problems with hardware breakage were one of the key challenges we had to overcome in this work.















\subsection{Ablation of Randomizations}
\label{sec:ablation-rand}

\begin{figure}[h]
    \begin{minipage}[c]{0.55\textwidth}
        \includegraphics[width=0.95\textwidth]{figures/randomization_ablations.pdf}
    \end{minipage}\hfill
    \begin{minipage}[c]{0.45\textwidth}
        \caption{Performance when training in environments with groups of randomizations held out. All runs show exponential moving averaged performance and 90\% confidence interval over a moving window of the RL agent in the environment it was trained. We see that training is faster in environments that are easier, e.g. \textit{no randomizations} and \textit{no unmodeled effects}. We only show one seed per experiment; however, in general we have noticed almost no instability in training.
        }
        \label{fig:rand-abl}
    \end{minipage}\hfill
\end{figure}

In \autoref{subsection:randomizations} we detail a list of parameters we randomize and effects we add that are not already modeled in the simulator. In this section we show that these additions to the simulator are vital for transfer. We train 5 separate RL policies in environments with various randomizations held out: \textit{all randomizations} (baseline), \textit{no observation noise}, \textit{no unmodeled effects}, \textit{no physics randomizations}, and \textit{no randomizations} (basic simulator, i.e. no domain randomization).

Adding randomizations or effects to the simulation does not come without cost; in \autoref{fig:rand-abl} we show the training performance in simulation for each environment plotted over wall-clock time. Policies trained in environments with a more difficult set of randomizations, e.g. \textit{all randomizations} and \textit{no observation noise}, converge much slower and therefore require more compute and simulated experience to train in. However, when deploying these policies on the real robot we find that training with randomizations is critical for transfer. \autoref{table:rand-abl} summarizes our results.
Specifically, we find that training with all randomizations leads to a median of $13$ consecutive goals achieved, while policies trained with \textit{no randomizations}, \textit{no physics randomizations}, and \textit{no unmodeled effects} achieve only median of $0$, $2$, and $2$ consecutive goals, respectively.

\begin{table}[h]
    \centering
    \caption{
    The number of successful consecutive rotations on the physical robot of 5 policies trained separately in environments with different randomizations held out.
    The first 5 rows use PhaseSpace for object pose estimation and were run on the same robot at the same time. Trials for each row were interleaved in case the state of the robot changed during the trials. The last two rows were measured at a different time from the first 5 and used the vision model to estimate the object pose.
    }
    \renewcommand{\arraystretch}{1.3}
    \begin{tabular}{@{}llll@{}}
        \toprule
        \textbf{Training environment} & \textbf{Mean} & \textbf{Median} & \textbf{Individual trials (sorted)} \\ 
        \midrule
        All randomizations (state) & $18.8 \pm 17.1$  & $13$ & $50$, $41$, $29$, $27$, $14$, $12$, $6$, $4$, $4$, $1$ \\
        No randomizations (state) & $1.1 \pm 1.9$ & $0$ & $6$, $2$, $2$, $1$, $0$, $0$, $0$, $0$, $0$, $0$ \\
        No observation noise (state) & $15.1 \pm 14.5$ & $8.5$ & $45$, $35$, $23$, $11$, $9$, $8$, $7$, $6$, $6$, $1$ \\
        No physics randomizations (state) & $3.5 \pm 2.5$ & $2$ & $7$, $7$, $7$, $3$, $2$, $2$, $2$, $2$, $2$, $1$ \\
        No unmodeled effects (state) & $3.5 \pm 4.8$ & $2$ & $16$, $7$, $3$, $3$, $2$, $2$, $1$, $1$, $0$, $0$ \\
        \midrule
        All randomizations (vision) & $15.2 \pm 14.3$ & $11.5$ & $46$, $28$, $26$, $15$, $13$, $10$, $8$, $3$, $2$, $1$ \\
        No observation noise (vision) & $5.9 \pm 6.6$ & $3.5$ & $20$, $12$, $11$, $6$, $5$, $2$, $2$, $1$, $0$, $0$ \\
    \bottomrule
    \end{tabular}
    \label{table:rand-abl}
\end{table}

When holding out \emph{observation noise} randomizations, the performance gap is less clear than for the other randomization groups.
We believe that is because our motion capture system has very little noise.
However, we still include this randomization because it is important when the vision and control policies are composed.
In this case, the pose estimate of the object is much more noisy, and, therefore, training with observation noise should be more important.
The results in \autoref{table:rand-abl} suggest that this is indeed the case, with a drop from median performance of $11.5$ to $3.5$ if the observation noise randomizations are withheld.

The vast majority of training time is spent making the policy robust to different physical dynamics. Learning to rotate an object in simulation without randomizations requires about 3 years of simulated experience, while achieving the same performance in a fully randomized simulation requires about 100 years of experience.
This corresponds to a wall-clock time of around 1.5 hours and 50 hours in our simulation setup, respectively.



\subsection{Effect of Memory in Policies}

We find that using memory is helpful to achieve good performance in the randomized simulation. In \autoref{fig:mem-abl} we show the simulation performance of three different RL architectures: the baseline which has a LSTM policy and value function, a feed forward (FF) policy and a LSTM value function, and both a  FF policy and FF value function. We include results for a FF policy with LSTM value function because it was plausible that having a more expressive value function would accelerate training, allowing the policy to act robustly without memory once it converged. However, we see that the baseline outperforms both variants, indicating that it is beneficial to have some amount of memory in the actual policy.

Moreover, we found out that LSTM state is predictive of the environment randomizations.
In particular, we discovered that the LSTM hidden state after 5 seconds of simulated interaction
with the block allows to predict whether the block is bigger or smaller than
average in $80\%$ of cases.

\begin{figure}[h]
    \begin{minipage}[c]{0.55\textwidth}
        \includegraphics[width=0.95\textwidth]{figures/effect_of_mem.pdf}
    \end{minipage}\hfill
    \begin{minipage}[c]{0.45\textwidth}
        \caption{Performance when comparing LSTM and feed forward (FF) policy and value function networks. We train on an environment with all randomizations enabled. All runs show exponential moving averaged performance and 90\% confidence interval over a moving window for a single seed. We find that using recurrence in both the policy and value function helps to achieve good performance in simulation.
        }
        \label{fig:mem-abl}
    \end{minipage}\hfill
\end{figure}


To investigate the importance of memory-augmented policies for transfer, we evaluate the same three network architectures as described above on the physical robot. \autoref{table:memory-abl} summarizes the results.
Our results show that having a policy with access to memory yields a higher median of successful rotations, suggesting that the policy may use memory to adapt to the current environment.\footnote{When training in an environment with no randomizations, the FF and LSTM policy converge to the same performance in the same amount of time. This shows that a FF policy has the capacity and observations to solve the non-randomized task but cannot solve it reliably with all randomizations, plausibly because it cannot adapt to the environment.}
Qualitatively we also find that FF policies often get stuck and then run out of time.

\begin{table}[h]
    \centering
    \caption{
    The number of successful consecutive rotations on the physical robot of 3 policies with different network architectures trained on an environment with all randomizations.
    Results for each row were collected at different times on the physical robot.
    }
    \renewcommand{\arraystretch}{1.3}
    \begin{tabular}{@{}llll@{}}
        \toprule
        \textbf{Network architecture} & \textbf{Mean} & \textbf{Median} & \textbf{Individual trials (sorted)} \\ 
        \midrule
        LSTM policy / LSTM value (state) & $18.8 \pm 17.1$  & $13$ & $50$, $41$, $29$, $27$, $14$, $12$, $6$, $4$, $4$, $1$ \\
        FF policy / LSTM value (state) & $4.7 \pm 4.1$ & $3.5$ & $15$, $7$, $6$, $5$, $4$, $3$, $3$, $2$, $2$, $0$ \\
        FF policy / FF value (state) & $4.6 \pm 4.3$ & $3$ & $15$, $8$, $6$, $5$, $3$, $3$, $2$, $2$, $2$, $0$ \\
    \bottomrule
    \end{tabular}
    \label{table:memory-abl}
\end{table}

\subsection{Sample Complexity \& Scale}



In \autoref{fig:scale} we show results when varying the number of CPU cores and GPUs used in training, where we keep the batch size per GPU fixed such that overall batch size is directly proportional to number of GPUs.
Because we could linearly slow down training by simply using less CPU machines and having the GPUs wait longer for data, it is more informative to vary the batch size.
We see that our default setup with an 8 GPU optimizer and 6144 rollout CPU cores
reaches 20 consecutive achieved goals approximately 5.5 times faster than a setup with a 1 GPU optimizer and 768 rollout cores. Furthermore, when using 16 GPUs we reach 40 consecutive achieved goals roughly 1.8 times faster than when using the default 8 GPU setup.
Scaling up further results in diminishing returns, but it seems that scaling up to 16 GPUs and 12288 CPU cores gives close to linear speedup.

\begin{figure}[h!]
    \centering
    \begin{subfigure}[t]{0.45\textwidth}
        \centering
        \includegraphics[width=\textwidth]{figures/effect_of_scale1.pdf}
    \end{subfigure}
    \begin{subfigure}[t]{0.45\textwidth}
        \centering
        \includegraphics[width=\textwidth]{figures/effect_of_scale2.pdf}
    \end{subfigure}
    \caption{We show performance in simulation when varying the amount of compute used during training versus wall clock training time (left) and years of experience consumed (right). Batch size used is proportional to the number of GPUs used, such that time per optimization step should remain constant apart from slow downs due to gradient syncing across optimizer machines.}
    \label{fig:scale}
\end{figure}




\subsection{Vision Performance}
\label{sec:result-vision}
In \autoref{table:perf} we show that we can combine a vision-based pose estimator and the control policy to successfully transfer to the real robot without embedding sensors in the target object.
To better understand why this is possible, we evaluate the precision of the pose estimator on both synthetic and real data.
Evaluating the system in simulation is easy because we can generate the necessary data and have access to the precise object's pose to compare against.
In contrast, real images had to be collected by running a state-based policy on our robot platform.
We use PhaseSpace to estimate the object's pose, which is therefore subject to errors.
The resulting collected test set consists of $992$ real samples.\footnote{A sample contains 3 images of the same scene. We removed a few samples that had no object in them after it being dropped.}
For simulation, we use test sets rendered using Unity and MuJoCo. The MuJoCo renderer was not used during training, thus the evaluation can be also considered as an instance of sim-to-sim transfer.
\autoref{table:vision} summarizes our results.

\begin{table}[h]
    \caption{Performance of a vision based pose estimator on synthetic and real data.}
    \centering
    \renewcommand{\arraystretch}{1.3}
    \begin{tabular}{@{}lll@{}}
        \toprule
        \textbf{Test set} & \textbf{Rotation error} & \textbf{Position error} \\ %\textbf{Rotation Error} & \textbf{Position Error} \\
        \midrule
        Rendered images (Unity) & $2.71^{\circ} \pm 1.62$ & $3.12\text{mm} \pm 1.52$ \\
        Rendered images (MuJoCo) & $3.23^{\circ} \pm 2.91$ & $3.71\text{mm} \pm 4.07$ \\
        Real images & $5.01^{\circ} \pm 2.47$ & $9.27\text{mm} \pm 4.02$ \\
    \bottomrule\end{tabular}
    \label{table:vision}
\end{table}

Our results show that the model achieves low error for both rotation and position prediction when tested on synthetic data.\footnote{For comparison, PhaseSpace is rated for a position accuracy of around $20$ $\mu$m but requires markers and a complex setup.}
On the images rendered with MuJoCo, there is only a slight increase in error, suggesting successful sim-to-sim transfer.
The error further increases on the real data, which is due to the gap between simulation and reality but also because the ground truth is more challenging to obtain due to noise, occlusions, imperfect marker placement, and delayed sensor readings.
Despite that the prediction error is bigger than the observation noise used during policy training (\autoref{table:obs-noise}), the vision-based policy performs well
on the physical robot (\autoref{table:perf}).










\section{Related Work}
\textbf{Related work}:
% Object detection related datasets/algo in non-medical domain
% Locally labeled CXR dataset
A few CXR datasets have localized abnormality annotations \cite{shih2019augmenting,filice2020crowdsourcing,jaeger2014two} that are curated manually. These are high quality gold standard ground truth datasets but tend to be smaller in scale (< 30,000 images) and have a narrow coverage, with typically only 1-2 labels. In addition, since most labeling efforts only have abnormality semantics attached, no direct relationships with the affected anatomical locations are available. 

%MEHDI: repeated concepts from above. I am removing the following: 

%The lack of anatomic semantics in the annotation is a limitation for complex multi-modal clinical reasoning work, e.g., differential diagnosis, since clinicians often integrate information along anatomical lines, and for downstream report generation tasks, which often requires describing not only the abnormality but also correctly communicate the location of the abnormalities (and medical devices) to the receiving clinicians. 

Two recent CXR datasets have labels for anatomies described in the reports. In \cite{datta2020dataset}, a small manually annotated dataset (2000 reports) included 10 abnormalities that are individually associated with 29 unique spatial locations (anatomies) at the report level. Another CXR dataset has automatically extracted abnormality and anatomy labels as disconnected concepts that are only correlated at the study level from  160,000 reports using a supervised NLP algorithm \cite{bustos2020padchest}. This was trained on a smaller set of manually annotated data. Neither datasets contain localized annotations for the associated CXR images, nor any comparison relation annotations between sequential exams, both of which are available in the Chest ImaGenome dataset. In Table \ref{tab:related}, we present a comparison of our Chest ImagGenome dataset with other datasets available in the literature.

% Table -- Kashyap

% MEdical imaging datasets to go here: Discussed that we will only focus on cxr datasets that are available for this paper. 
% \caption{\color{red} Kashyap, feel free to continue with the table. We should remove the questionmarks and add a line for our dataset (since all others are not graph). For longer text, using abbreviations and explaining them in the caption often works better. If fill in the values is not possible, it is better to remove the table altogether.}


\begin{table}[t!]
\caption{Summary of existing chest X-ray datasets}
\resizebox{\textwidth}{!}{%
\begin{tabular}{@{}lllllllll@{}}
\toprule
\textbf{Dataset} & \textbf{Annotation Level} & \textbf{Annotation Method} & \textbf{Num Labels} & \textbf{Anatomy Labeled} & \textbf{Graph} & \textbf{Dataset Size} & \textbf{Temporal Labels} & \textbf{Reports} \\ \midrule
SIIM-ACR Pneumothorax Segmentation \cite{filice2020crowdsourcing} & Segmentation & Manual + augmented & 1 & No & No & 12,047 & No & No \\
RSNA Pneumonia Detection Challenge   \cite{shih2019augmenting} & Bounding Boxes & Manual & 1 & No & No & 30,000 & No & No \\
Indiana University Chest X-ray collection \cite{demner2016preparing} & Global & Automated & 10 & No & No & 3,813 & No & Yes \\
NIH CXR dataset \cite{wang2017chestx} & Global & Automated & 14 & No & No & 112,120 & No & No \\
PLCO \cite{team2000prostate} & Global & Automated & 24 & Yes & No & 236,000 & Yes & No \\
Stanford CheXpert \cite{irvin2019chexpert} & Global & Automated & 14 & No & No & 224,316 & No & No \\
MIMIC-CXR \cite{johnson2019mimic} & Global & Automated & 14 & No & No & 377,110 & No & Yes \\
Dutta \cite{datta2020dataset} & Global & Manual & 10 & Yes & Yes & 2,000 & No & Yes \\
PadChest \cite{bustos2020padchest} & Global & Manual + automated & 297 & Yes & No & 160,868 & No & Yes \\
Montgomery County Chest X-ray   \cite{jaeger2014two} & Segmentation & Manual & 1 & Yes & No & 138 & No & No \\
Shenzen Hospital Chest X-ray   \cite{jaeger2014two} & Segmentation & Manual & 1 & Yes & No & 662 & No & No \\  \hline \hline
\textbf{Chest ImaGenome} & Bounding Boxes & Automated & 131 & Yes & Yes & 242,072 & Yes & Yes \\
\bottomrule
\end{tabular}%
}
\label{tab:related}
\vspace{-0.4cm}
\end{table}
% removed (Derived from MIMIC-CXR \cite{johnson2019mimic}) % makes table really small


\section{Conclusion}

\begin{comment}
\begin{figure}
\includegraphics[width=\linewidth]{figs/beyond_tss_lesion.pdf}
\caption[]{End-to-End runtime lesion study of the entire MNIST dataset and the FMA featurized music dataset. Each of DROP's contributions provides a runtime improvement.}
\label{fig:beyond_lesion}
\end{figure}
\end{comment}



\section{Conclusion}
\label{sec:conclusion}

Advanced data analytics techniques must scale to rising data volumes. 
DR techniques offer a powerful toolkit when processing these datasets, with PCA frequently outperforming popular techniques in exchange for high computational cost. 
In response, we propose DROP, a new dimensionality reduction optimizer. 
DROP combines progressive sampling, progress estimation, and online aggregation to identify high quality low dimensional bases via PCA without processing the entire dataset by balancing the runtime of downstream tasks and achieved dimensionality. 
Thus, DROP provides a first step in bridging the gap between quality and efficiency in end-to-end DR for downstream \red{analytics}. 

%We revisit canonical operators for time series dimensionality reduction and the measurement study of~\cite{keogh-study}, and show that PCA is more effective than popular alternatives in the data mining literature often by a margin of over $2\times$ on average on gold-standard time series benchmark data sets with respect to output data dimension. More surprisingly, we empirically demonstrate that a small number of samples are sufficient to accurately characterize directions of maximum variance and obtain a high-quality low-dimensional transformation.




\section{Acknowledgements}

Luca Herranz-Celotti was supported by the Natural Sciences and Engineering Research Council of Canada through the Discovery Grant from professor Jean Rouat, and by CHIST-ERA IGLU. We thank Compute Canada for the clusters used to perform the experiments and NVIDIA for the donation of two GPUs. We thank Wolfgang Maass for the opportunity to visit the Institute of Theoretical Computer Science, Guillaume Bellec, Darjan Salaj and Franz Scherr, for their invaluable insights on learning with surrogate gradients, and Maryam Hosseini, Ahmad El Ferdaoussi and Guillaume Bellec for their feedback on the article.

\medskip
{
\small
\bibliography{paper}
}


\newpage
\appendix
\appendixpage
\startcontents[appendices]
\printcontents[appendices]{l}{1}{\setcounter{tocdepth}{2}}
\newpage


% Panoptic segmentation

% 3D segmentation

% Multi-object tracking

% Online 3D panoptic:

% PanopticFusion: (IROS 2019)
% https://arxiv.org/pdf/1903.01177.pdf
%
% - most similar to ours
% - PSPNet + M-RCNN + 2D fusion
% - volumetric mapping, 
% - greedy matching with IoU -> optimal only with 0.5 threshold
% - voxel & class weighting
% - CRF regularisation
%
% - good:
%
% - bad:
%  - CRF post-processing step
%  - greedy data-association
%    - can't be tuned for lower overlap ratios -> has to have high framerate, large changes in viewpoint could break this
%    - IoU: sensitive to 2D labels projecting over object borders (CRF and voxel weighting seem to alleviate this)

% Voxblox++: (Robotics & automation letters 2019)
% https://arxiv.org/pdf/1903.00268.pdf
% https://github.com/ethz-asl/voxblox-plusplus
%
% - M-RCNN + geometric segmentation + fusion 
% - data association of geometric segments with 3D overlap (no. points inside volume), fixed threshold for min number of points
% - instance label is assigned to a segment based on highest overlap
% - only one detected segment per reference label, as in PanopticFusion and Ours
% - TSDF Integration 
%
% good: 
% - because of geometric segmentation objects with no associated semantic class can also be segmented
% bad:
% - two different object segment types -> confusing, overly complicated ?
% - quite inaccurate (fixed below)

% Reconstructing Interactive 3D Scenes by Panoptic Mapping and CAD Model Alignments (ICRA 2021)
% https://arxiv.org/pdf/2103.16095.pdf
% https://github.com/hmz-15/Interactive-Scene-Reconstruction
%
% - based heavily on Voxblox++, much more accurate
% - Scene-graph ("contact graph") for mapping object relations
% - Search & replace voxels with CAD models, with geometrical and physical constraints
% - Object 6D pose
% - Format for robot interaction
%
% - Segmentation: bilateral fusion of geomatric and semantic segments -> reduce segmentation noise compared to Voxblox++
% - Fusion: triplet count improves consistency over Voxblox++ pairwise count strategy (take semantic label into account in addition to instance and geometry)
% - Fusion: instance labels are also combined if there is enough overlap with common geometric label for long enough time
%   - this means multiple detections can match the same reference unlike ours, voxblox++ and PanopticFusion ?
%

% Panoptic-MOPE: (ROBOTICS AND AUTOMATION LETTERS 2020)
% https://ieeexplore.ieee.org/stamp/stamp.jsp?tp=&arnumber=8977356
% https://github.com/hoangcuongbk80/Object-RPE/tree/panoptic-mope
%
% - novel RGB-D semantic segmentation model + M-RCNN
% - camera tracking based on "addaptively weighted optimization of geometric, appearance, and semantic cues"
% - surfel map: 
%   - how does it scale ? authors satate they tested on room-sized environments, but could be applied in larger scale as well ...
%     - could maybe be applied as VO in a SLAM algorithm ...
%   - demo only on a small pallet + surroundings, might not be applicable in large-scale SLAM

% US VS THEM:
%
% - based heavily on PanopticFusion, with modifications:
%   - instead of greedy data-association (which seems to be the case in others as well), we solve LAP (JPDA?)
%     - overlap threshold can be tuned, which renders the algorithm more flexible
%     - could be extended to dynamic tracking ?
%   - multiple options for association likelihood
%   - outlier rejection (either clustering or probabilistic)
%   - test different options for decreasing processing time
%   - no post-processing
%
% - model-agnostic:
%   - completely separated from segmentation
%   - does not care how point clouds are obtained -> applicable for LIDAR segmentation (e.g. EfficientLPS) as well
%
% - also agnostic to localisation method
%   - could, however, be utilised to find landmark locations / poses

% More compact version of this paragraph to introduction to save space?
%Panoptic segmentation -- proposed in \cite{panoptic_segmentation} -- aims to solve the unified task of semantic- and instance segmentation. Semantic classes are separated to \textit{stuff} -- amorphous, unquantifiable regions like sky, road or floor -- and \textit{things} -- quantifiable objects. The distinction between the two can vary depending on the application, but a semantic class can only belong to one or another. The article also proposes a unified panoptic evaluation metric, coined \textbf{Panoptic Quality} (PQ). Many 2D approaches to panoptic segmentation -- \textit{e.g.} \cite{panopticfpn,seamless,panoptic_deeplab,efficientps} -- have since been proposed. Deep neural networks for performing semantic- or instance segmentation directly on the 3D reconstruction -- \textit{e.g.} on \cite{scannet,s3dis,paris_lille_3d} -- have also been proposed, but since they require the reconstructed 3D scene, they are mostly offline approaches and therefore out of scope for this work. Some recent works also apply panoptic segmentation to point clouds -- \textit{e.g.} methods in the SemanticKITTI panoptic segmentation competition \cite{semantic_kitti} -- mostly aimed at segmenting LiDAR output. They are suitable for online processing, but similar to RGB-D images require a method for tracking object instances persistent in both time and space. In fact, our proposed method, as well as some others mentioned in this work, could use segmented LiDAR point clouds as an input similarly to RGB-D images.

PanopticFusion \cite{panopticfusion} is the first work to propose online integration of panoptic image segmentations to a 3D reconstruction. They integrate point clouds generated from segmented images to a TSDF voxel volume \cite{tsdf,voxblox} by greedily matching detected segments with the reconstruction and regulating each voxel's corresponding instance with a weighting function. Semantic labels are inferred in a bayesian manner based on confidence scores provided by the segmentation model. They also apply a Conditional Random Field (CRF) to regularise the reconstruction, improving results significantly. Voxblox++ \cite{voxblox++} -- introduced later the same year -- is a similar approach that also integrates segmented RGB-D images into a TSDF volume. It leverages geometric segmentation of depth images to improve instance segmentation accuracy. Both geometric and semantic segments are used to compute a pair-wise weight, which is used to greedily match them with segments in the reconstruction. Because of the geometric segmentation, the method allows segmentation of objects with no known semantic class in addition to objects recognised by the instance segmentation model. 

Recently, \cite{interactive_3d_scenes} built upon the idea of Voxblox++. They apply Voxblox++ for 3D instance integration, with two small but effective modifications: the pair-wise weight is replaced by a triplet weight that also takes semantic labels into account in the fusion, and -- in addition to geometric segments -- instance segments are fused if they overlap by a significant amount. The article introduces a method for searching and aligning CAD models to reconstructed objects based on geometry and semantic class, as well as geometrical and physical rules. With the CAD models, a contact graph and interactive virtual scene are reconstructed to allow a robot to simulate its interaction with the environment. SceneGraphFusion \cite{scenegraphfusion} is another approach that forms a scene graph online from a stream of RGB-D images, but unlike the above-mentioned approach, it generates the graph with a deep neural network, after which the panoptic labels for geometrically segmented portions of the 3D reconstruction are produced a side product.

Panoptic-MOPE \cite{panoptic_mope} is another recent approach, which integrates sequences of RGB-D images into a surfel reconstruction. Unlike other mentioned approaches -- which assume the camera pose either known or estimated elsewhere -- it also tracks camera movements based on geometric-, appearance- and semantic cues. The method also applies a novel RGB-D panoptic segmentation model. Although it is only tested on room-sized environments, the authors claim it could be scaled to larger environments as well.
\section{Hardware Description}
\label{app:hardware}



\subsection{ShadowRobot Dexterous Hand}

We use the ShadowRobot Dexterous Hand. Concretely, we use the version with electric motor actuators, EDC hand (EtherCAT-Dual-CAN).

The hand has \num{24} degrees of freedom between its links and is actuated by \num{40} Spectra tendons controlled by \num{20} DC motors in the base of the hand, each actuating a pair of agonist--antagonist tendons connected via a spool.
\num{16} degrees of freedom can be controlled independently whereas the remaining \num{8} joints (which are the joints between the non-thumb finger proximal, middle, and distal segments) form $4$ pairs of coupled joints.

\subsection{PhaseSpace Visual Tracking}
We use a 3D tracking system to localize the tips of the fingers, to perform calibration procedures, and as ground truth for the RGB image-based tracking. The PhaseSpace Impulse X2E tracking system uses active LED markers that blink to transmit a unique ID code and linear detector arrays in the cameras to detect the positions and IDs. The system features capture speeds of up to 960 Hz and positional accuracies of below 20 $\mu$m. The data is exposed as a 3D point cloud together with labels associating the points with stable numerical IDs. Our setup uses 16 cameras distributed spherically around the hand and centered on the palm with a radius of approximiately $0.8$ meters.


\subsection{RGB Cameras} \label{app:cameras}
We also use RGB images to track the objects for manipulation. We perform object pose estimation using 3 Basler acA640-750uc RGB cameras with a resolution of 640x480 placed approximately 50 cm from the Shadow hand. We use 3 cameras to resolve pose ambiguities that may occur with monocular vision. We chose these cameras for their flexible parameterization and low latency. Figure \ref{fig:camera-setup} shows the placement of the cameras relative to the hand.

\begin{figure}[t]
    \centering
    \includegraphics[width=0.9\textwidth]{figures/setup_cage_top}
    \caption{Our 3-camera setup for vision-based state estimation.}
    \label{fig:camera-setup}
\end{figure}

\subsection{Control}

The high-level controller is implemented as a Python program
running a neural network policy implemented in Tensorflow~\citep{tensorflow}
on a GPU. Every 80ms it queries the Phasespace sensors and then runs inference with the neural network to obtain the action, which takes roughly 25ms. The policy outputs an action that specifies the change of position for each actuator, relative to the current position of the joints controlled by the actuator. It then sends the action to the low-level controller.

The low-level controller is implemented in C++ as a separate process on a different machine which is 
connected to the Shadow hand via an Ethernet cable. The controller is written as a real-time system 
--- it is pinned to a CPU core, has preallocated memory, and does not depend on any garbage collector to avoid non-deterministic delays.
The controller receives the relative action, converts it to 
an absolute joint angle and clips to the valid range, then sets each component of the action as the target for a PD controller. Every 5ms, the PD controller queries the Shadow hand joint angle sensors, then attempts to achieve the desired position.


Surprisingly, decreasing time between actions to 40ms increased training time but did not noticeably improve performance in the real world.
















\subsection{Joint Sensor Calibration} 
\label{app:sensor-calibration}
The hand contains 26 Hall effect sensors that sense magnetic field rotations along the joint axis.
To transform the raw magnetic measurements from the Hall sensors into joint angles, we use a piecewise linear function interpolated from 3-5 truth points per joint. To calibrate this function, we initialize to the factory default created using physical calibration jigs. For further accuracy, we attach PhaseSpace markers to the fingertips, and minimize error between the position reported by the PhaseSpace markers and the position estimated from the joint angles.
We estimate these linear functions by minimizing reprojection error with \texttt{scipy.minimize}.
\section{Simulated Environment}
\subsection{Deterministic Environment} \label{app:sim}



\paragraph{Simulation.} Our environment is based on the OpenAI Gym robotics environments described in~\citep{plappert2018multi}. We use MuJoCo for simulation~\citep{MuJoCo}.

\paragraph{States.}
The state of the system is \num{60}-dimensional and consists of angles and velocities of all robot joints as well as the position, rotation and velocities (linear and angular) of the object.
Initial states are sampled by placing the object on the palm in a random orientation
and applying random actions for $100$ steps (we discard the trial if the object is dropped in the meantime).

\paragraph{Goals.}
The goal is the desired orientation of the object represented as a quaternion.
A new goal is generated after the current one has been achieved within a tolerance of \SI{0.4}{\radian}.\footnote{I.e. we consider a goal as achieved if there exists a
rotation of the object around an arbitrary axis with an angle smaller than \SI{0.4}{\radian}
which transforms the current orientation into the desired one.}

\paragraph{Observations.}
Described in \autoref{table:policy-inputs}.



\paragraph{Rewards.} \label{sec:reward}
The reward given at timestep $t$ is $r_t=d_t-d_{t+1}$, where
$d_t$ and $d_{t+1}$ are the rotation angles between the desired 
and current object orientations before and after the transition, respectively.
We give an additional reward of $5$ whenever a goal is achieved with the tolerance of \SI{0.4}{\radian}
(i.e. $d_{t+1}<0.4$) and a reward of $-20$ (penalty) whenever the object is dropped.

\paragraph{Actions.}
Actions are \num{20}-dimensional and correspond to the desired angles of the hand joints.
We discretize each action coordinate into $11$ bins of equal size.
Due to the inaccuracy of joint angle sensors
on the physical hand (see Appendix~\ref{app:hardware}), actions are specified
relative to the current hand state.
In particular, the torque applied to the given joint in simulation is equal to $P*(s_t+a-s_{t'})$, where
$s_{t}$ is the joint angle at the time when the action was specified,
$a$ is the corresponding action coordinate, 
$s_{t'}$ is the current joint angle,
 and $P$ is the proportionality coefficient.
For the coupled joints, the desired and actual positions represent the sum of the two joint angles.
All actions are rescaled to the range $[-1,1]$.
To avoid abrupt changes to the action signal, which could harm a physical robot,
we smooth the actions using an exponential moving average\footnote{We use a coefficient of $0.3$ per 80ms.}
before applying them (both in simulation and during deployments on the physical robot).

\paragraph{Timing.}
Each environment step corresponds to \SI{80}{\ms} of real time and consists of \num{10} consecutive MuJoCo steps, each corresponding to \SI{8}{\ms}.
The episode ends when either the policy achieves \num{50} consecutive goals, the policy fails
to achieve the current goal within \num{8} seconds of simulated time, or the object is dropped.

\subsection{Randomizations}
\label{app:randomizations}

A variety of randomizations are applied to the simulator, shrinking the reality gap between the simulated environment and the physical world in order to learn a policy that generalizes to reality.


\paragraph{Physical parameters.}
The physical parameters are sampled at the beginning of every episode and held fixed for the whole episode. The full set of randomized values are displayed in~\autoref{table:rand-physics}.


\paragraph{Observation noise.}
We use two types of noise ---
\textit{correlated} noise which is sampled once per episode and kept fixed,
and an \textit{uncorrelated} Gaussian one.
Apart from Gaussian correlated noise, we also add more structured noise
coming from inaccurate placement of the motion capture markers
by computing the observations using slightly misplaced
markers in the simulator.
The configuration of noise levels is described in~\autoref{table:obs-noise}.
The observation noise is only applied to the policy inputs and not to
the value network inputs as the value function is not used during the deployment on the physical system.







\begin{table}
    \footnotesize
    \centering
    \caption{Standard deviation of observation noise.}
    \renewcommand{\arraystretch}{1.3}
    \begin{tabular}{@{}lll@{}}
        \toprule
        \textbf{Measurement} & \textbf{Correlated noise} & \textbf{Uncorrelated noise} \\ \midrule
        fingertips positions & $1$mm & $2$mm \\
        object position & $5$mm & $1$mm \\
        object orientation & $0.1$rad & $0.1$rad \\ \hline
        fingertip marker positions & $3$mm & \\
        hand base marker position & $1$mm & \\
        \bottomrule
    \end{tabular}
\label{table:obs-noise}
\end{table}

\begin{table}
    \footnotesize
    \centering
    \caption{Standard deviation of action noise.}
    \renewcommand{\arraystretch}{1.3}
    \begin{tabular}{@{}ll@{}}
        \toprule
        \textbf{Noise type} & \textbf{Percentage of the action range} \\ \midrule
        uncorrelated additive & 5\% \\
        correlated additive & 1.5\% \\
        uncorrelated multiplicative & 1.5\% \\
        \bottomrule
    \end{tabular}
\label{table:action-noise}
\end{table}

\paragraph{PhaseSpace tracking errors.}
Noise aside, readings of the motion capture markers from the PhaseSpace system might be occasionally unavailable for a short period of time due to instability of the service. To model such error in the simulator, we mask the fingertip markers with a small probability (0.2 per second) for 1 second so that the policy has a chance to learn how to interact with the environment while the system temporarily loses track of some markers. 

Furthermore, the markers might be occluded while in motion, causing a brief delay of readings of some fingertip positions. In the simulator, a small weightless cuboid site\footnote{A site represents a location of interest relative to the body frame in MuJoCo. Also see \url{http://mujoco.org/book/modeling.html\#site}.} is attached to the back of each nail and we consider a marker occluded whenever a collision with the site is detected as another finger or object is getting too close. If a fingertip marker is deemed occluded, we use its last available position readings instead of the current one.



\paragraph{Action noise and delay.}
We add correlated and uncorrelated Gaussian noise to all actions to account for an imperfect actuation 
system.
The detailed noise levels can be found in~\autoref{table:action-noise}.
Moreover, the real system contains many potential sources of delays between the time that observations are sensed and actions are executed,
from network delay to the computation time of the neural network.
Therefore, we introduce a simple model of action delay to the simulator.
At the beginning of every episode we sample for every actuator whether
it is going to be delayed (with probability $0.5$) or not.
The actions corresponding to delayed actuator are delayed by one environment step, i.e. approximately $80$ms.


\paragraph{Timing randomization.}
We also randomize the timing of environment steps.
Every environment step is simulated as $10$ MuJoCo physics simulator steps
with $\Delta t=8\mbox{ms}+\mbox{Exp}(\lambda)$, where 
$\mbox{Exp}(\lambda)$ denotes the exponential distribution
and the coefficient $\lambda$ is once per episode sampled uniformly from the
range $[1250,10000]$.

\paragraph{Backlash model.}
The physical Shadow Dexterous Hand is tendon-actuated which causes a substantial
amount of backlash, while the MuJoCo model assumes direct actuation on the joints.
In order to account for it, we introduce a simple model of backlash which modifies
actions before they are sent to MuJoCo.
In particular, for every joint we have two parameters which
specify the amount of backlash in each direction, and are denoted $\delta_{-1}$ and $\delta_{+1}$,
as well as a time varying variable $s$ denoting the current state of slack.
We obtained the values of $\delta_{-1}, \delta_{+1}$ through
calibration.
At the beginning of every episode we sample the values of $\delta_{-1}, \delta_{+1}$
from the Gaussian distribution centered around the calibrated values with the standard deviation of $0.1$.
Let $a_{\text{in}} \in [-1,\,1]$ be an action specified by the policy.
Our backlash model works as follows: we compute the new value of the slack variable
$s'=[s+a_{\text{in}} \delta_{\textbf{sgn}(a_\text{in})} \Delta t]_{-1}^{+1}$,
compute the scaling factor
$\alpha=1-\left[\frac{|\textbf{sgn}(a_\text{in})-s|}{|s'-s|+\epsilon}\right]_0^1$,
where $\epsilon=10^{-12}$ is a constant used for numerical stability,
and finally multiply the action by $\alpha$:
$a_{\text{out}} = \alpha a_{\text{in}}$.
    
\paragraph{Random forces on the object.}
To represent unmodeled dynamics, we sometimes apply random forces on the object.
The probability $p$ that a random force is applied is sampled at the beginning of the episode from the loguniform distribution between $0.1\%$ and $10\%$.
Then, at every timestep, with probability $p$ we apply a random force
from the $3$-dimensional Gaussian distribution with the standard deviation equal to $1~m/s^2$ times the mass of the object on each coordinate and decay the force with the coefficient of $0.99$ per $80$ms.


\paragraph{Randomized vision appearance.} 
We randomize the visual appearance of the robot and object, as well as lighting and camera characteristics.
The materials and textures are randomized for every visible object in the scene.
We randomize the hue, saturation, and value for the object faces around calibrated values from real-world measurements.
The color of the robot is uniformly randomized. Material properties such as glossiness and shininess are randomized as well. Camera position and orientation are slightly randomized around values calibrated to real-world locations.
Lights are randomized individually, and intensities are scaled based on a randomly drawn total intensity. After rendering the scene to images from the three separate cameras, additional augmentation is applied.
The images are linearly normalized to have zero mean and unit variance.
Then the image contrast is randomized, and finally per-pixel Gaussian noise is added. Details are in~\autoref{table:vision-randomization}.

\begin{savenotes}
\begin{table}
    \footnotesize
    \centering
    \caption{Vision randomizations.}
    \renewcommand{\arraystretch}{1.3}
    \begin{tabular}{@{}ll@{}}
        \toprule
        \textbf{Randomization type} & \textbf{Range} \\ \midrule
        number of cameras & 3 \\
        camera position & $\pm$ 1.5 mm \\
        camera rotation & 0--3$^{\circ}$ around a random axis \\
        camera field of view & $\pm$ 1$^{\circ}$ \\
        \hline
        robot material colors & RGB \\
        robot material metallic level & 5\%--25\%\footnote{In units used by Unity. See \url{https://unity3d.com/learn/tutorials/s/graphics}.} \\
        robot material glossiness level & 0\%--100\%\footnotemark[\value{footnote}] \\
        \hline
        object material hue & calibrated hue $\pm$ 1\% \\
        object material saturation & calibrated saturation $\pm$ 15\% \\
        object material value & calibrated value $\pm$ 15\% \\
        object metallic level & 5\%--15\%\footnotemark[\value{footnote}] \\
        object glossiness level & 5\%--15\%\footnotemark[\value{footnote}] \\
        \hline
        number of lights & 4--6 \\
        light position & uniform over upper half-sphere \\
        light relative intensity & 1--5 \\
        total light intensity & 0--15\footnotemark[\value{footnote}] \\
        \hline
        image contrast adjustment & 50\%--150\% \\
        additive per-pixel Gaussian noise & $\pm$ 10\% \\
        \bottomrule
    \end{tabular}
\label{table:vision-randomization}
\end{table}
\end{savenotes}

    
%Material
%\begin{itemize}
%\item P. Klimek et al, ATLAS Tile Calorimeter calibration with the Laser system during LHC Run-2, https://cds.cern.ch/record/2721936
%\end{itemize}

\subsection{Calibration procedure}

%\begin{itemize}
%\item Runs we take
%\item relative calibration
%\item constants update during data taking
%\item IOV concept
%\item reprocessing
%\end{itemize}

%..... The laser system is employed to perform the PMT response calibration relative to the previous global calibration of the TileCal detector with the caesium scan. Thus, to determine the laser calibration constants, a laser run taken close to the global calibration day is used to set the reference signals for each PMT. ......

As can be seen in Equation~\ref{eq:channelEnergy}, the reconstruction of the energy in TileCal depends on several constants, some of them being updated regularly. The main calibration of the TileCal energy scale is obtained using the caesium system~\cite{Blanchot:2020lyh}. However, since a caesium scan needs a pause in the $pp$ collisions of at least six hours, this calibration cannot be performed very often. Therefore, regular relative calibrations are accomplished between two caesium scans using the laser system. Moreover, during the LHC technical stop at the beginning of data taking period in 2016, few liquid traces coming from the caesium hydraulic system were found in the detector cavern. Since then until the end of Run~2, caesium scans were restricted to be taken only during the end of year technical stops, due to risk of the leak. In absence of the caesium calibration, the laser became the main calibration system, calibrating the PMTs and readout electronics. In order to address the fast drift of PMT response caused by the large instantaneous luminosity, the laser calibration constants were updated every 1--2 weeks, since July 2016. These constants were used in so-called prompt data processing, performed during the data taking period. 

Each year, the data recorded by the ATLAS detector is reprocessed. Data reprocessing consists of the update of the physics dataset (proton--proton and heavy ion collision runs) with updated conditions and calibration constants. Moreover, a reprocessing of the full Run~2 dataset was performed during LHC Long Shutdown 2 at the end of Run~2. This step is necessary to apply new reconstruction and calibration algorithms as well as the corrections that were impossible to be done or missed during prompt data processing. The IOVs are readjusted and chosen to coincide with the data taking periods. For laser calibration, they occur every 1--2 weeks in order to smoothly follow the evolution of PMTs response during the data taking period. 

The method to compute the laser constant $f_{\mathrm{Las}}$ introduced in Equation~\ref{eq:channelEnergy} is based on the analysis of specific laser calibration runs, taken daily during the data taking period, for which both the laser system photodiodes and the TileCal PMTs are read out. The laser calibration employs two types of successive laser runs:
\begin{itemize}
  \item Low Gain run (labeled as LG) consists of $\sim$10,000 pulses with a constant amplitude and the filter attenuation factor equal to 3,
  \item High Gain run (labeled as HG) consists of $\sim$20,000 pulses with a constant amplitude and the filter attenuation factor equal to 330.
\end{itemize}

The laser system is employed to perform the PMT response calibration relative to the previous global calibration of the TileCal detector with the caesium scan. Thus, to determine the laser calibration constants, a laser run taken close to the caesium scan is used to set the reference signals for each PMT. 
%The laser calibration is a relative calibration with respect to a laser reference run taken right after each caesium scan. 
By definition, if the response of a channel to a given laser intensity is stable (the response of the PMT and of the associated readout electronics are stable), the laser constant $f_{\mathrm{Las}}$ is 1. The references were set close to the start of each year's $pp$ collision runs. 
%The following references were used during LHC Run~2, for both prompt processing and reprocessing of data:
%\begin{itemize}
%	\item 2015 - LG: 271880, HG: 271882 (2015-07-17), IOV: 273795 (2015-07-27)
%	\item 2016 - LG: 294144, HG: 294145 (2016-04-01)
%	\item 2017 - LG: 317215, HG: 317216 (2017-03-06)
%	\item 2018 - 
%	\begin{itemize}
%		\item proton-proton runs: LG: 344221, HG: 344222 (2018-02-15)
%		\item heavy ion runs: LG: 364531, HG: 364533 (2018-10-27)
%	\end{itemize}
%\end{itemize}
%The references for the PMT gain $G{_i^{\mathrm{ref}}}$ from Equation~\ref{eq:gain_ratio}, used in the Combined method, significantly fluctuate run by run. The fluctuations are caused by the laser light instability and statistical uncertainty. Therefore, they are averaged over the laser runs taken within $\pm 10$ days with respect to the reference runs. The laser runs used for averaging were taken in the period before the physics collision started. 
The laser references and laser constants are stored in the conditions database. 
%in the following folders:
%\begin{itemize}
%	\item Laser references: \texttt{/TILE/OFL02/CALIB/CES}
%	\item Laser constants: \texttt{/TILE/OFL02/CALIB/LAS/LIN}
%\end{itemize}

The laser calibration procedure evolved during Run~2. Due to increasing instantaneous luminosity and response variation observed in all PMTs, the methods to derive laser constants were adapted. The applied methods are described in detail in Section~\ref{sec:determination_of_the_calibration_constants}. 
%In this section, the description of the procedure to produce the laser calibration constants $f_{\mathrm{Las}}$ using these methods for prompt processing and reprocessing of data 2015--2018 is presented. 


\subsection{Determination of the calibration constants}
\label{sec:determination_of_the_calibration_constants}

The laser runs are constituted by a set of laser pulses with corresponding signal readout from the individual PMTs, from which the pedestal is subtracted. 
%which is subtracted from pedestal at a pulse-by-pulse basis. 
For each pulse, the normalised response of a PMT channel, the ratio $R_{i,p}$, is defined as:

\begin{equation}
    R_{i,p} = \frac{A_{i,p}}{A_{\mathrm{D6},{p}} }
    \label{eq:Rip}
\end{equation}

where $p$ denotes the pulse, $A_i$ is the reconstructed signal amplitude of the PMT readout channel $i$ and $A_{\mathrm{D6},{p}}$ is the signal amplitude measured by the photodiode 6 (D6) in the laser box. The D6 measures the laser light after the beam expander and probes the beam close to the TileCal PMTs in the best dynamic range among available photodiodes D6--D9. The average of the ratio $R_{i,p}$ over all pulses of the laser run, denoted as $R_i\equiv \langle R_{i,p}\rangle$, is analysed for each PMT.

The laser calibration factors employed to reconstruct the cell energy, in Equation~\ref{eq:channelEnergy}, are simply the relative response of the channel:

\begin{equation}
    f_{\mathrm{Las}}^i = \frac{R_i}{R_i^{\mathrm{ref}}}
    \label{eq:fLaser}
\end{equation}

where $R_i^{\mathrm{ref}}$ is the normalised response of the PMT channel $i$ during the laser reference run. For monitoring purposes, these factors are usualy presented in percentage as a relative response variation:

\begin{equation}
    (f_{\mathrm{Las}}^i - 1)\times 100\;[\%]
\label{eq:PMTdriftCorrected}
\end{equation}

The measurement of $f_{\mathrm{Las}}^i$ may be influenced by instabilities with origin at the laser system itself, both at a global level, i.e. affecting equally all the detector PMTs, or at the fibre level, i.e. affecting the set of PMTs associated with each clear fibre. To take these effects into account, global and fibre corrections are determined, such that the corrected laser constant reads as:

\begin{equation}
    f_{\mathrm{Las}}^i \to f_{\mathrm{Las}}^i \times \frac{1}{\alpha_{\mathrm{G}} \times \alpha_{\mathrm{f}(i)}}
    \label{eq:fLaserOpticsCorrections}
\end{equation}

\begin{itemize}                                                                               
\item The global correction $\alpha_{\mathrm{G}}$ is associated with a coherent drift of all channels. The effect can be related to an instability of the reference diode, from the variation of light received by the TileCal PMTs or common ageing of the long fibres.
                                                                                       
\item The fibre correction $\alpha_{\mathrm{f}(i)}$, computed per fibre $\mathrm{f}(i)$, is associated with a time variation of the light transmission from fibre to fibre.
\end{itemize}                                                                                

During Run~2, two methods were used to evaluate these optics corrections: the so-called Direct and Combined methods. In the Direct method, the global and fibre corrections are simply determined from the average response variations of a set of stable PMTs reading outermost and least irradiated cells in the D layer used as references, and the sub-set of D-layer PMTs associated with the fibre, respectively. 
%PMTs reading outermost cells were used, since those cells were least irradiated and expected to be stable. 
This method was used to calibrate and monitor the detector response during 2015--2017 data taking but revealed to be inadequate for calibration when the response of the reference PMTs started to fluctuate due to larger integrated currents in the middle of the 2017 run. Then the Combined method was developed and employed in the 2018 TileCal calibration and also for the reprocessing of 2017 data. Instead of relying on the stability of a set of reference PMTs, the gain of the PMT is explicitly evaluated to determine the optics corrections by the Combined method.


\subsubsection*{Direct method}

In the Direct method, the global correction is evaluated from the relative response of all PMTs reading cells in the D-layer:

\begin{equation}
  \alpha_{\mathrm{G}} = \bigg\langle \frac{R_i}{R_i^{\mathrm{ref}}} \bigg\rangle^{\mathrm{D-cells}}
  \label{eq:globalCorrectionDM}
\end{equation}

The fibre corrections are evaluated using information from PMTs of the D layer for the fibres associated to the LB, and from PMTs reading the D, B13, B14 and B15 cells for the EB\footnote{These cells are less exposed to particle fluence, so their readout PMTs experience smaller integrated currents and a more stable response.}, corrected from global effects. This quantity is evaluated for each long clear fibre $\mathrm{f}(i)$ as 

\begin{equation}
  \alpha_{\mathrm{f}(i)} = \frac{1}{\alpha_{\mathrm{G}}} \bigg\langle \frac{R_i}{R_i^{\mathrm{ref}}} \bigg\rangle^{\mathrm{D,B-cells}}_{\mathrm{f}(i)}
  \label{eq:fibreCorrectionDM}
\end{equation}

In Equations~\ref{eq:globalCorrectionDM} and~\ref{eq:fibreCorrectionDM}, $\langle\;\rangle$ represents a geometric weighted average, where the weight is proportional to the number of laser pulses in the run, and the average RMS of the PMT signals.

Saturated channels, channels with bad status in the TileCal condition database, and channels for which the absolute difference between the applied and requested HV is above 
%HV$_{\mathrm{set}}$ and the actual PMT HV is above 
10~V ($\Delta \mathrm{HV}>$10~V) are excluded from the computation of the optics corrections. Moreover, an iterative procedure rejects outlier channels, more than $3\sigma$ apart from the $R_i/R_i^{\mathrm{ref}}$ distribution average.


\subsubsection*{Combined method}

In the Combined method, the actual PMT gain is measured based on the statistical nature of photoelectron production and multiplication inside the PMT. It assumes that the noise is negligible with respect to the laser-induced PMT signals and that the laser light is coherent. Under these conditions, the two main contributions to the PMT signal fluctuations to the laser scans are the poissonian fluctuations in the photoelectron emission spectrum and multiplication, and the variation of the intensity of the light source~\cite{Bures:74}. The PMT gain $G$ can be written as:

\begin{equation}
	G = \frac{1}{f \cdot e}\cdot \left( \frac{\mathrm{Var}[q]}{\langle q\rangle} - k \cdot \langle q\rangle \right)
	\label{eq:gain_statistical_method}
\end{equation}

where $e=1.6\times 10^{-19}$~C is the electron charge constant, $f$ stands for the excess noise factor extracted from the known gain of the individual PMT dynodes~\cite{Arisaka:2000id}.  For the eight dynode TileCal PMTs, $f=1.3$ at the nominal gain of $G=10^5$, $\langle q \rangle$ is the average value of PMT anode charge associated to each laser pulse, and $\mathrm{Var}[q]$ is the variance of the anode charge distribution. 
%$\langle q \rangle$ is the average anode charge and $\mathrm{Var}[q]$ the variance. 
The coherence factor $k = \frac{\mathrm{Var}[I]}{\langle I\rangle ^2}$ depends on the characteristics of the light source itself but not on the light intensity. The factor $k$ ranges from 0, for an ideal fully coherent light source, to 1, for a totally incoherent light source, and is determined with a set of PMTs measuring the same light source. For any PMT pair $i$ and $j$, $k$ is given by the average PMT measured charge $q_i$ and $q_j$ respectively, and the covariance $\mathrm{Cov}[q_i, q_j]$ of the charge measurements, as:

\begin{equation}
	k = \frac{\mathrm{Cov}[q_i, q_j]}{\langle q_i\rangle \langle q_j\rangle}
	\label{eq:PMT_excess_noise_factor}
\end{equation}

In order to decrease the dependence of the gain measurement on the $k$ factor determination, to which the sensitivity is more limited, the PMT gain is analysed in high gain laser calibration runs taken with filter wheel in position~8 with 31.6\% transmission (optical density of 2.5). For these runs the light intensity is lower leading also to a lower average PMT anode charge $\langle q \rangle$, thus the $k$ term in Equation~\ref{eq:gain_statistical_method} has a smaller effect on the gain measurement.

Moreover, since the PMT gain determination presents significant fluctuations, the average over a set of runs within $\pm$10 days around the laser reference run is taken to set the PMT reference gain, $G_i^{\mathrm{ref}}$. The PMT gain $G_i$ is used as an independent measure of the PMT signal and as the basis to evaluate the optics corrections by the Combined method. The global correction is determined from the average ratio between the PMT relative response and the PMT relative gain using PMTs reading the D-layer and the BC1, BC2, B13, B14, B15 cells:


\begin{equation}
    \alpha_{\mathrm{G}} = \bigg\langle \frac{R_i}{R_i^{\mathrm{ref}}} \Big/ \frac{G_i}{G_i^{\mathrm{ref}}} \bigg\rangle^{\mathrm{D,B-cells}}
    \label{eq:global_combined}
\end{equation}

The fibre corrections are determined in approximately the same way as the global correction except that the average runs over all the channels connected to a common long fibre $\mathrm{f}(i)$, with the global correction taken into account to avoid double correcting:

\begin{equation}
    \alpha_{\mathrm{f}(i)} = \frac{1}{\alpha_{\mathrm{G}}} \bigg\langle \frac{R_i}{R_i^{\mathrm{ref}}} \Big/ \frac{G_i}{G_i^{\mathrm{ref}}} \bigg\rangle_{\mathrm{f}(i)}
    \label{eq:fibre_combined}
\end{equation}

As for the Direct method, the PMTs having bad status or with $\Delta \mathrm{HV}>$10~V or saturated channels are discarded from analysis.


\subsection{Evolution of the optics corrections}

\begin{figure}[htbp]
\centering
\subfloat[\label{fig:optics_correction_2018_a}]{\includegraphics[width=0.5\linewidth]{figures/global_history}}
\subfloat[\label{fig:optics_correction_2018_b}]{\includegraphics[width=0.5\linewidth]{figures/LB10C_fiber_history}}
\caption{Evolution of the (a) global correction and (b) LB10C fibre correction associated with even/odd numbered PMTs in LBA10/LBC10 over time in 2018. The corrections are determined using laser high gain runs with the Combined method and are calculated as a weighted geometric mean. The corresponding errors are included in the data points.}
\label{fig:optics_correction_2018}
\end{figure}


Figure~\ref{fig:optics_correction_2018} shows the time evolution in 2018 of the global correction and the LB10C fibre correction (associated with the even/odd numbered PMTs in LBA10/LBC10), both shown in percentage and determined with the Combined method using laser runs taken in high gain. The global correction in 2018 is stable in time within 1\% and the correction is about the same order. During Run 2, the magnitude of this correction did not exceed 2.5\%. The fibre correction shown is generally representative of the 384 clear fibres in total. For all the years, the magnitude of the corrections did not exceed 1\% and was also found to be constant throughout the time.

The global correction dominates the scale of the PMT calibration. Its precision should match the global scale uncertainty on the PMT calibration assessed from laser and caesium comparisons presented in Section~\ref{sec:CsLas}, and thus be better than 0.4\%. The accuracy on the global correction was further assessed using two symmetric sets of PMTs, one composed of PMTs reading the TileCal A side and another with PMTs installed in the C side to derive independent corrections. The corrections obtained for the A and C sides matched well below the sub-percent level for all years in Run~2, attesting the robustness of the Combined method at disentangling the effects of fluctuations in the monitored light intensity common to all PMTs.


\subsection{Comparison with caesium calibration}
\label{sec:CsLas}

The response variation of PMTs measured with the laser system should match the full detector response variation obtained with the caesium system within short periods of time, where fluctuations from the scintillators and WLS fibers can be safely neglected. Thus, the comparison between the laser and caesium measurements constitute a procedure to validate the laser algorithm itself, employed to validate the Combined method.

During 2015 and 2016, three periods of low integrated luminosity were available within consecutive caesium scans. Figure~\ref{fig:CsLas2015Nov} shows the response variation between July 17 and November 3, 2015, obtained with the caesium system as a function of the response variation obtained with the laser. The results are displayed at channel level and separating channels per layer A, B/BC and D. The great majority of channels have the same response variation for laser and for caesium.

\begin{figure}[htbp]
\centering
\subfloat[\label{fig:CsLas2015Nov_a}]{\includegraphics[width=0.51\linewidth]{figures/y2015_iovII_CsLas_COLZ}}
\subfloat[\label{fig:CsLas2015Nov_b}]{\includegraphics[width=0.49\linewidth]{figures/y2015_iovII_CsLas_SCAT}}
\caption{Response variation (in \%) measured by caesium (y-axis) and by laser employing the Combined method (x-axis) between July 17 and November 3, 2015 for (a) all TileCal channels and (b) the channels in the A-, B/BC- and D-layers. Special channels not calibrated by the caesium system, such as the E-cells, are not included.
%  , i.e. all C10-, E-cells, and MBTS are not included (right). 
% The points visible on the left plot but missing on the right one correspond to the regular C10-cells.
%  (left) for each TileCal channel and (right) for each layer.
}
\label{fig:CsLas2015Nov}
\end{figure}

The corresponding distribution of the ratio between the caesium constants ($f_\mathrm{Cs}$) and the laser calibration constants ($f_\mathrm{Las}$), calculated to address the response variation of the PMTs during the same period of time, 
is shown in Figure~\ref{fig:CsLas2015Nov_1d} 
% for the TileCal channels 
separated by layer and Long/Extended barrel. 
%The calibration constants are specifically set to correct for the response evolution occurred exclusively during the period of time being evaluated. 
Each distribution is fitted with a Gaussian function to measure its average and standard deviation. The differences observed between the caesium and the laser systems are more evident in the extended barrel and on the A layer. These regions of the calorimeter are less shielded and thus the effects of radiation damage to scintillator and WLS fibre are faster. The average difference is well bellow 0.1\% and the standard deviation is 0.6\%.

For the three periods analysed, the maximum average difference observed was 0.4\%. This value is taken as the uncertainty on the scale of the PMT calibration with laser.

\begin{figure}[htbp]
\centering
\subfloat[\label{fig:CsLas2015Nov_1d_a}]{\includegraphics[width=0.5\linewidth]{figures/y2015_iovII_CsLas_Ratio_LB}}
\subfloat[\label{fig:CsLas2015Nov_1d_b}]{\includegraphics[width=0.5\linewidth]{figures/y2015_iovII_CsLas_Ratio_EB}}
\caption{Ratio between the caesium calibration constants ($f_\mathrm{Cs}$) and the laser calibration constants calculated with Combined method ($f_\mathrm{Las}$) for channels in Layer A, B/BC and D in the (a) Long Barrel and (b) Extended Barrel. Special channels not calibrated by the caesium system, such as the E-cells, are not included.}
\label{fig:CsLas2015Nov_1d}
\end{figure}

\subsection{Uncertainties on the PMT calibration}

Besides the systematic uncertainty on the PMT calibration scale, the uncertainty on the PMT relative inter-calibration, mostly sourced at the fibre correction procedure and at the channel-level readout, is evaluated. To do so, an indirect comparison between the responses to caesium source and laser, measured with left and right PMTs reading the same cell, is performed evaluating the following observable: 

\begin{equation}
\Delta f^{\mathrm{L-R}}_{\mathrm{Cs/Las}}=\left( \frac{f^{\mathrm{L}}_{\mathrm{Cs}}}{f^{\mathrm{L}}_{\mathrm{Las}}}- 
\frac{f^{\mathrm{R}}_{\mathrm{Cs}}}{f^{\mathrm{R}}_{\mathrm{Las}}} \right)
\label{eq:sys_Cs-Laser}
\end{equation}

where $f^{\mathrm{L(R)}}_{\mathrm{Las}}$ and $f^{\mathrm{L(R)}}_{\mathrm{Cs}}$ are the calibration constants corresponding to the cell relative response to laser and caesium source measured by the left (right) channel. With this quantity, the scintillator effects common to both left/right readouts are cancelled out. Assuming that the WLS fibre response from the left and right sides of the cell has a similar behaviour, the width of the distribution of $\Delta f^{\mathrm{L-R}}_{\mathrm{Cs/Las}}$ is driven by the uncertainties of the laser measurement and caesium measurements. The inter-calibration systematic uncertainty on the laser calibration was then determined by disentangling the contributions from the caesium uncertainty and constraining with measurements of $f^{\mathrm{L}}_{\mathrm{Cs}}-f^{\mathrm{R}}_{\mathrm{Cs}}$ and $f^{\mathrm{L}}_{\mathrm{Las}}-f^{\mathrm{R}}_{\mathrm{Las}}$. The results obtained with 2018 data are shown in Figure~\ref{fig:sigma_Las}. A dependence of the systematic uncertainty on the integrated luminosity, more pronounced for the extended barrel, is observed. The effect is due to a correlation between the integrated PMT charge and the response down-drift, with consequent increase in the spread of the response for a given PMT sample.

\begin{figure}[htbp]
\centering
\includegraphics[width=0.5\textwidth]{figures/Uncertainties_PMT_response}
\caption{Uncertainties on the PMT inter-calibration with the Laser~II system using the Combined method as a function of the integrated luminosity for the Long Barrel and for the Extended Barrel. The results are obtained using laser and caesium calibration data collected in 2018. The two points at 63.3~fb$^{-1}$ result from two successive caesium scans without LHC beam. The uncertainty is parametrised as a function of the luminosity by fitting the data points with a linear function. A global scale systematic resulting from direct comparison between laser and caesium data was found to be 0.4\%. This value should be summed in quadrature to obtain the total laser uncertainty. In 2018, three caesium scans were performed in LB (red points) and four in EB (blue points).}
\label{fig:sigma_Las}
\end{figure}

The total uncertainty on the laser calibration of a PMT, corresponding to the quadratic sum of the 0.4\% scale systematic and the luminosity-dependent inter-calibration uncertainty, in the Long Barrel ($\sigma_{\mathrm{Las,tot}}^{\mathrm{LB}}$) and in the Extended Barrel ($\sigma_{\mathrm{Las,tot}}^{\mathrm{EB}}$) yields:

\begin{equation}
\begin{aligned}
\sigma_{\mathrm{Las,tot}}^{LB} [\%] =0.4\oplus(0.3+0.0016\times L)\; [\%] \\
\sigma_{\mathrm{Las,tot}}^{EB} [\%] =0.4\oplus(0.3+0.0032\times L)\; [\%]
\end{aligned}
\end{equation}

\subsection{Overview of the PMT response}
The laser system is used to measure the evolution of the PMT response as a function of time. 
%In this Section, the average response variation of the PMTs as a function of time during the LHC Run~2 is presented. 
The Combined method, discussed in Section~\ref{sec:determination_of_the_calibration_constants}, is utilised to calculate the response variation with respect to a set of reference runs. In particular, the Equation~\ref{eq:PMTdriftCorrected} is used to obtain the response variation for each PMT. 
Channels marked with bad data quality status, unstable high voltage or flagged as problematic by any calibration system are discarded. For each cell type, the average response is obtained by a Gaussian fit to the distribution of PMT response variation. The $\chi ^2$ fit method is applied. The Gaussian approximation is used in order to obtain the average variation that is not affected by outliers. 

A sample of the mean response variation in the PMTs for each cell type averaged over $\phi$, measured with the laser system during the entire $pp$ collisions data-taking period in 2018, is shown in Figure~\ref{fig:map_2018}. The most affected cells are those located at the inner radius and in the gap and crack region with down-drift up to 4.5\% and 6\%, respectively. Those cells are the most irradiated and their readout PMTs experience the largest anode current. 

\begin{figure}[htbp]
\centering
    \includegraphics[width=1.0\textwidth]{figures/tile_laser_map364147}
    \caption{The mean response variation in the PMTs for each cell type, averaged over $\phi$, observed during the entire $pp$ collisions data-taking period in 2018 (between laser calibration runs taken on 18 April 2018 and 22 October 2018) calculated using the Combined method. For each cell type, the response variation is defined as the mean of a Gaussian fit to the response variations in the channels associated with given cell type. A total of 64 modules in $\phi$ were used for each cell type, with the exclusion of known pathological channels.
    %The average PMT response variation (in \%) per TileCal cell type as a function of $|\eta|$ and radius, observed during the entire high $\langle \mu \rangle$ proton-proton collisions data taking period in 2018 (between April 18 and October 22, 2018) calculated using the Combined method.
    }\label{fig:map_2018}
\end{figure}

Figure~\ref{fig:phimap_2018} shows the average response variation of the channels per layer and along the azimuthal angle $\phi$ for the same period in 2018. Each $\phi$ bin corresponds to one LB/EB module averaged over the A and C sides. Channels with low signal amplitude, bad data quality status or unstable high voltage are discarded in the average response calculation. It can be seen that PMTs reading the cells in layers closest to the beam axis, composed of A cells, are the most affected. Next layers, formed of the BC and D cells are significantly less affected. We observe larger uniformity across the modules in $\phi$ in layers with a larger number of channels (eg. 40 channels in the LB A layer, see Figure~\ref{fig:map_2018}), where the effect of discarding one bad quality channel has less impact. On the other hand, layers for which the spread in the response of the channels is larger (eg. EB layer A against LB layer A, see Figure~\ref{fig:map_2018}) are more affected in the $\phi$ uniformity with bad channel removal.
%These Figures provide useful insight about the uniformity of the PMT variation over $\phi$.

\begin{figure}[htbp]
\centering
    \subfloat[\label{fig:phimap_2018_a}]{\includegraphics[width=0.495\linewidth]{figures/364147_tile_laser_lb_map_phi_LasPaper}}\hfill
    \subfloat[\label{fig:phimap_2018_b}]{\includegraphics[width=0.495\linewidth]{figures/364147_tile_laser_eb_map_phi_LasPaper}}
    \caption{The mean response variation in the PMTs for each cell type, averaged over $\eta$, observed during the entire $pp$ collisions data-taking period in 2018 (between laser calibration runs taken on 18 April 2018 and 22 October 2018) in LB (a) and EB (b), calculated using the Combined method. For each cell type, the response variation is defined as the mean of a Gaussian fit to the response variations in the channels associated with given cell type. Known pathological channels were excluded.
   % The average PMT response variation (in \%) as a function of polar angle $\phi$ and layers for long barrel (left) and extended barrel (right), observed during the entire high $\langle \mu \rangle$ proton-proton collisions data taking period in 2017 (between April 18 and October 22, 2018). The plot is made using the Combined method.
    }\label{fig:phimap_2018}
\end{figure}

Figure~\ref{fig:LaserDrift_run2_a} shows the time evolution of the mean response variation in the PMTs for each layer observed during the entire Run~2. The PMT response variation strongly depends on the delivered luminosity by the LHC. Therefore, the delivered luminosity is also shown for comparison. The observed PMTs response variation is the result of three competing factors: i) the constant up-drift observed when PMTs are in rest; ii) the down-drift during high instantaneous luminosity period when PMTs are under stress; iii) the fast partial recovery after stress observed during technical stops. These effects result in 6\% accumulated mean response variation in the PMTs for the cells located at the inner layer at the end of Run~2. For the B/BC and D layers, the average PMT response degradation during $pp$ collisions was almost totally recovered in technical stops, resulting in $-1.5$\% accumulated PMT response variation at the end of Run~2 for the layer~B/BC and even in +0.5\% balance for the layer~D. 
%PMTs reading the layer~A exhibit larger response gradients and reach the end of the Run~2 with an average $-5.5$\% response drift. 

Figure~\ref{fig:LaserDrift_run2_b} shows the Gaussian width distribution as a function of time observed for each layer during the entire Run~2. The Gaussian width for all layers increases with time during high instantaneous luminosity period when PMTs are under stress. It is caused by the different behaviour of different PMTs over time which are at different $|\eta|$ positions. During technical stops, when PMTs are at rest, some inversion of this effect is observed resulting from the recovery of the most affected PMTs to the average response in a given layer or cell type.

\begin{figure}[htbp]
\centering
    \subfloat[\label{fig:LaserDrift_run2_a}]{\includegraphics[width=0.5\linewidth]{figures/Averagelayers_Run2}}
    \subfloat[\label{fig:LaserDrift_run2_b}]{\includegraphics[width=0.5\linewidth]{figures/layers_gausianWidth_Run2}}
    \caption{The mean response variation in the PMTs (a) and Gaussian width (b) for each layer, as a function of time, observed during the entire Run~2 (between stand-alone laser calibration runs taken on 17 July 2015 and 22 October 2018). For each layer, the response variation is defined as the mean of a Gaussian fit to the variations in the channels associated with given layer. Known pathological channels are excluded. The laser calibration runs were not taken during the ATLAS end-of-year technical stops. Moreover, the laser system was not operational due to technical problems in the period September 10--27, 2016. Thus, no laser data can be seen in the plots for these time intervals. The LHC delivered luminosity is shown for comparison in grey. The vertical dashed lines show the start of $pp$ collisions in respective years.
%    (a) Average response variation (in \%) and (b) Gaussian width per TileCal layer as a function of time in entire Run~2 calculated using the Combined method. For each layer, the average response is obtained by a Gaussian fit to the distribution of PMT response variation with respect to an average reference prior to the start of collisions (including all laser runs in the time period of $\pm 10$ days of July 17, 2015). The vertical dashed lines show the start of proton-proton collisions in respective years. The laser system was not operational due to technical problems in the period September 10--27, 2016. Thus, no Laser data can be seen in the time evolution plots for this time interval. 
    }\label{fig:LaserDrift_run2}
\end{figure}

\FloatBarrier

\FloatBarrier
\section{Optimization Details}\label{app:hyper}
\subsection{Control Policy}\label{app:hyper-ppo}

We normalize all observations given to the policy and value networks with running means and standard deviations. We then clip observations such that they are within 5 standard deviations of the mean. We normalize the advantage estimates within each minibatch. We also normalize targets for the value function with running statistics. The network architecture is depicted in \autoref{fig:ppo}. 


\begin{figure}[t]
    \begin{minipage}[c]{0.6\textwidth}
        \includegraphics[width=0.9\textwidth]{figures/policy1}
    \end{minipage}\hfill
    \begin{minipage}[c]{0.4\textwidth}
        \caption{Policy network (left) and value network (right). Each network consists
        of an input normalization, a single fully-connected hidden layer with ReLU activations \citep{relu}
        and a recurrent LSTM block \citep{lstm}. The normalization block
        subtracts the mean value of each coordinate (across all data gathered so far),
        divides by the standard deviation, and removes outliers by clipping.
        There is no weight sharing between the two networks.
        The goal provided to the policy is the noisy relative target orientation (see \autoref{table:policy-inputs} for details).}
        \label{fig:ppo}
    \end{minipage}
\end{figure}    


\begin{table}[h!]
    \footnotesize
    \centering
    \caption{Hyperparameters used for PPO.}
    \renewcommand{\arraystretch}{1.3}
    \begin{tabular}{@{}ll@{}}
        \toprule
        \textbf{Hyperparameter} & \textbf{Value} \\ \midrule
        hardware configuration & 8 NVIDIA V100 GPUs + 6144 CPU cores \\ 
        action distribution & categorical with $11$ bins for each action coordinate \\
        discount factor $\gamma$ & $0.998$ \\
        Generalized Advantage Estimation $\lambda$ & $0.95$ \\
        entropy regularization coefficient & $0.01$ \\
        PPO clipping parameter $\epsilon$ & $0.2$ \\
        optimizer & Adam~\citep{adam} \\
        learning rate & 3e-4 \\
        batch size (per GPU) & $80$k chunks x $10$ transitions = $800$k transitions \\
        minibatch size (per GPU) & $25.6$k transitions \\
        number of minibatches per step & $60$ \\
        network architecture & dense layer with ReLU + LSTM \\
        size of dense hidden layer & 1024 \\
        LSTM size & 512 \\
    \bottomrule\end{tabular}
    \label{tbl:ppo}
\end{table}

\FloatBarrier




\subsection{Vision Model}\label{app:vision-hyper}\label{app:vision_training}
Vision training hyperparameters are given in \autoref{tbl:vision-hyp} and the details of the model architecture are given in \autoref{tbl:vision-hyper-arch}.

We also apply data augmentation for training. More specifically, we leave the object pose as is with $20\%$ probability, rotate the object by $90^{\circ}$ around its main axes with $40\%$ probability, and ``jitter'' the object by adding Gaussian noise to both the position and rotation indepdently with $40\%$ probability.



\begin{table}[h!]
    \centering
    \footnotesize
    \caption{Hyperparameters used for the vision model training.}
    \renewcommand{\arraystretch}{1.3}
    \begin{tabular}{@{}ll@{}}
        \toprule
        \textbf{Hyperparameter} & \textbf{Value} \\ \midrule
        hardware configuration & 3 NVIDIA P40 GPUs\footnote{Two GPUs are used for rendering and one for the optimization.} + 32 CPU cores  \\
        optimizer & Adam~\citep{adam} \\
        learning rate & $0.0005$, halved every $20\,000$ batches \\
        minibatch size &  $64 \times 3 = 192$ RGB images \\
        image size & $200 \times 200$ pixels \\
        weight decay regularization & $0.001$ \\
        number of training batches & $400\,000$ \\
        network architecture & shown in \autoref{fig:vision-architecture} \\
    \bottomrule\end{tabular}
    \label{tbl:vision-hyp}
\end{table}




\begin{table}[h!]
    \centering
    \footnotesize
    \caption{Hyperparameters for the vision model architecture.}
    \begin{tabular}{@{}ll@{}}
        \toprule
        \textbf{Layer} & \textbf{Details} \\ \midrule
        Input RGB Image & $200\times200\times3$ \\
        Conv2D & 32 filters, $5\times5$, stride 1, no padding \\
        Conv2D & 32 filters, $3\times3$, stride 1, no padding \\
        Max Pooling & $3\times3$, stride 3 \\
        ResNet & 1 block, 16 filters,  $3\times3$, stride 3 \\
        ResNet & 2 blocks, 32 filters, $3\times3$, stride 3 \\
        ResNet & 2 blocks, 64 filters, $3\times3$, stride 3 \\
        ResNet & 2 blocks, 64 filters, $3\times3$, stride 3 \\
        Spatial Softmax & \\
        Flatten & \\
        Concatenate & all 3 image towers combined\\ \midrule
        Fully Connected & 128 units \\
        Fully Connected & output dimension ($3$ position + $4$ rotation) \\
    \bottomrule\end{tabular}
    \label{tbl:vision-hyper-arch}
\end{table}


 \FloatBarrier

\end{document}

