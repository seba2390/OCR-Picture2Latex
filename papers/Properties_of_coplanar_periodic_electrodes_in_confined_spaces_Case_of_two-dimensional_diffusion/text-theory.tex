% !TeX root = article
% !TeX encoding = utf8
% !TeX spellcheck = en_US


\section{Theory}
\label{theory}

\subsection{Definition of the periodic cell}
\label{coplanar:def:problema}

Consider an electrochemical cell with a coplanar configuration of electrodes
located at the bottom plane $z=0$,
and a roof (insulator layer) located at the top plane $z=H$.
The configuration of electrodes is periodic (with period $p_{x}$) along the $x$-axis,
and it is symmetric along the $y$-axis,
such that the concentration profiles don't depend on the variable $y$.

Inside the electrochemical cell there is an oxidated species $O$ and a reduced species $R$,
which react at the surface of the electrodes according to the reaction
\begin{equation}
	\label{coplanar:eqn:reaccion}
	O + n_{e}\, \mathrm{e}^{-} \flechadoble R
\end{equation}
where $n_{e}$ corresponds to the number of exchanged electrons.
Here it is assumed that the transport of the species $\sigma \in \{O,R\}$ is solely due to diffusion.

Under the stated conditions,
the concentration $c_{\sigma}(x,z,t)$ of the electrochemical species $\sigma$
can be modeled by the two-di\-men\-sio\-nal diffusion equation%
\footnote{
%	whenever a $\pm$ or a $\mp$ is found,
%	the upper sign corresponds to $\sigma=O$,
%	and the lower sign, to $\sigma=R$.
	whenever $\pm$ or $\mp$ are found,
	the upper and lower signs corresponds to 
	$\sigma=O$ and $\sigma=R$ respectively.
}
\begin{subequations}
	\label{coplanar:eqn:difusion:idae}
	\begin{align}
		\frac{1}{D_{\sigma}} \parderiv{c_{\sigma}}{t}(x,z,t)
		&= \parderiv{^2 c_{\sigma}}{x^2}(x,z,t)
		+ \parderiv{^2 c_{\sigma}}{z^2}(x,z,t)
		\\
		c_{\sigma}(x,z,0^{-}) &= c_{\sigma,i}(x,z)
		\\
		\mp D_{\sigma} \parderiv{c_{\sigma}}{z}(x,H,t) &= 0
		\\
		\mp D_{\sigma} \parderiv{c_{\sigma}}{z}(x,0,t) &= \frac{j(x,t)}{F n_{e}}
		\\
		c_{\sigma}(x, z, t) &= c_{\sigma}(x + p_{x}, z, t)
	\end{align}
\end{subequations}
where $c_{\sigma,i}(x,z)$ is the initial concentration profile,
which is assumed to come from a previous steady state,
$D_{\sigma}$ is the diffusion coefficient of the species $\sigma$,
$F$ is the Faraday's constant,
$j(x,t)$ is an \textbf{arbitrary} current density flowing at the bottom boundary $z=0$,
and $p_{x}$ is the period along the $x$-axis.


\subsection{Properties of the initial concentration in steady state}
\label{coplanar:properties:initial}

With the assumptions in \S\ref{coplanar:def:problema},
it is possible to derive some useful \emph{conservation} properties,
which araise as direct consequences of the theorem stated below

\begin{teorema}
	\label{coplanar:teo:ci}
	Consider the periodic cell described in \S\ref{coplanar:def:problema}.
	If the initial concentration $c_{\sigma,i}(x,z)$ of species $\sigma \in \{O, R\}$
	comes from a previous steady state,
	then the initial current density $j_{i}(x)$ satisfies Kirchhoff's current law
	within one period of the cell
	\begin{equation}
		\label{coplanar:eqn:kirchhoff:inicial}
		p_{x} \fourier j_{i}(0)
		= \int_{-p_{x}/2}^{+p_{x}/2} j_{i}(x) \ud{x} = 0
	\end{equation}
	and the Fourier coefficients of the initial concentration profile satisfy
	\begin{equation}
		\label{coplanar:eqn:ci}
		\fourier c_{\sigma,i}(n_{x},z) = \left\{
		\begin{array}{ll}
		\bar{c}_{\sigma,i}, & n_{x} = 0
		\\
		\displaystyle
		\pm G\!\left(H-z,\, n_{x}^{2} \frac{4\pi^{2}}{p_{x}^{2}}\right)
		\frac{\fourier j_{i}(n_{x})}{F n_{e} D_{\sigma}}, & n_{x} \neq 0
		\end{array}
		\right.
	\end{equation}
	where $\fourier j_{i}(n_{x})$ are the Fourier coefficients of the initial current density,
	$p_{x}$ is one period of the cell, $\bar{c}_{\sigma,i}$ is a real constant
	and $G(z,s)$ is given by
	\begin{equation}
		\label{coplanar:eqn:G}
		G(z,s) = \frac{\cosh(\sqrt{s}\, z)}{\sqrt{s} \sinh(\sqrt{s}\, H)}
	\end{equation}
\end{teorema}
See Supplementary Information \S\ref{transforms} for the definition used for the Fourier coefficients.

\begin{proof}
	Since the initial concentration comes from a previous steady state,
	it satifies the diffusion equation in Eqs. \eqref{coplanar:eqn:difusion:idae}
	with $\partial c_{\sigma}/ \partial t = 0$
	\begin{subequations}
		\label{coplanar:eqn:difusion:estacionaria}
		\begin{align}
			\parderiv{^2 c_{\sigma,i}}{x^2}(x,z) +
			\parderiv{^2 c_{\sigma,i}}{z^2}(x,z) &= 0
			\\
			\mp D_{\sigma} \parderiv{c_{\sigma,i}}{z}(x,H) &= 0
			\\
			\mp D_{\sigma} \parderiv{c_{\sigma,i}}{z}(x,0) &= \frac{j_{i}(x)}{F n_{e}}
			\\
			c_{\sigma,i}(x, z) &= c_{\sigma,i}(x + p_{x}, z)
		\end{align}
	\end{subequations}
	Taking the Fourier coefficients $\fourier c_{\sigma,i}(n_{x},z)$
	from the diffusion equation, one obtains
	\begin{subequations}
		\begin{align}
			- n_{x}^{2} \frac{4\pi^{2}}{p_{x}^{2}} \fourier c_{\sigma,i}(n_{x},z)
			+ \parderiv{^2 \fourier c_{\sigma,i}}{z^2}(n_{x},z) &= 0 \\
			\mp D_{\sigma} \parderiv{\fourier c_{\sigma,i}}{z}(n_{x},H) &= 0 \\
			\mp D_{\sigma} \parderiv{\fourier c_{\sigma,i}}{z}(n_{x},0)
			&= \frac{\fourier j_{i}(n_{x})}{F n_{e}}
		\end{align}
	\end{subequations}
	which corresponds to a linear ordinary differential equation (ODE),
	thus it can be solved using well known techniques.

	In case $n_{x} \neq 0$, solving the ODE in terms of the Fourier coefficients leads to
	\begin{subequations}
		\begin{equation}
			\fourier c_{\sigma,i}(n_{x}, z) =
			a(n_{x}) \cosh\!\left(n_{x} \frac{2\pi}{p_{x}} (H-z)\right) +
			b(n_{x}) \sinh\!\left(n_{x} \frac{2\pi}{p_{x}} (H-z)\right)
		\end{equation}
		Later, by applying the boundary conditions, the desired result is obtained
		\begin{equation}
			\fourier c_{\sigma,i}(n_{x}, z) =
			\pm G\!\left(H-z,\, n_{x}^{2} \frac{4\pi^{2}}{p_{x}^{2}}\right)
			\frac{\fourier j_{i}(n_{x})}{F n_{e} D_{\sigma}}
		\end{equation}
	\end{subequations}
	where $G(z,s)$ is defined in Eq. (\ref{coplanar:eqn:G}).

	In case $n_{x} = 0$, the solution of the ODE in terms of $\fourier c_{\sigma,i}(0, z)$
	leads to a real constant independent of $z$, name it $\bar{c}_{\sigma,i}$,
	and the Fourier coefficient of the current density equals zero due to Fick's law
	\begin{equation}
		\fourier c_{\sigma,i}(0,z) = \bar{c}_{\sigma,i},
		\quad
		\fourier j_{i}(0) = 0
	\end{equation}
\end{proof}

The first \emph{conservation} property that can be obtained from the previous theorem
holds for the horizontal average of $c_{\sigma,i}(x,z)$ at any $z$ in the cell

\begin{corolario}
	\label{coplanar:cor:bar_ci}
	Assume a two-dimensional periodic cell as in \S\ref{coplanar:def:problema},
	where the initial concentration $c_{\sigma,i}(x,z)$ of species $\sigma \in \{O, R\}$
	comes from a previous steady state.
	The average of the initial concentration, along any horizontal line,
	is independent of $z$ and equals $\bar{c}_{\sigma,i}$
	\begin{equation}
		\label{coplanar:eqn:bar_ci}
		\frac{1}{p_{x}} \int_{-p_{x}/2}^{+p_{x}/2} c_{\sigma,i}(x,z) \ud{x} =
		\fourier c_{\sigma,i}(0,z) = \bar{c}_{\sigma,i}
	\end{equation}
	where $p_{x}$ is one period of the cell.
\end{corolario}

The second \emph{conservation} property holds for the weighted sum of concentrations at any point in the cell,
which translates into the conservation of the total concentration at any point in the cell
when the diffusion coefficients of both electrochemical species are equal

\begin{corolario}
	\label{coplanar:cor:ci:total}
	Assume a two-dimensional periodic cell with period $p_{x}$ as in \S\ref{coplanar:def:problema},
	where the initial concentration $c_{\sigma,i}(x,z)$ comes from a previous steady state. The following weighted sum of the initial concentrations is independent of $(x,z)$ and equals
	\begin{equation}
		\label{coplanar:eqn:ci_total}
		D_{O} c_{O,i}(x,z) + D_{R} c_{R,i}(x,z) =
		D_{O} \bar{c}_{O,i} + D_{R} \bar{c}_{R,i}
	\end{equation}
\end{corolario}

\begin{proof}
	Take the weighted sum of both Fourier coefficients
	\begin{equation}
		D_{O} \fourier c_{O,i}(n_{x},z) + D_{R} \fourier c_{R,i}(n_{x},z) =
		\left\{
			\begin{array}{ll}
				D_{O} \bar{c}_{O,i} + D_{R} \bar{c}_{R,i}, & n_{x} = 0 \\
				0, & n_{x} \neq 0
			\end{array}
		\right.
	\end{equation}
	and later, take its Fourier series.
\end{proof}


\subsection{Properties of the concentration in transient state}

By using the Laplace transform and the Fourier coefficients
on the change in concentration $\Delta c_{\sigma}(x,z,t) = c_{\sigma}(x,z,t) - c_{\sigma,i}(x,z)$,
one can derive similar properties as in the previous section,
but now for the transient state.

\begin{teorema}
	\label{coplanar:teo:Dc}
	Consider the periodic cell described in \S\ref{coplanar:def:problema}.
	If the initial concentration $c_{\sigma,i}(x,z)$ of species $\sigma \in \{O, R\}$
	comes from a previous steady state,
	then the Laplace transform of the Fourier coefficients
	of $\Delta c_{\sigma}(x,z,t) = c_{\sigma}(x,z,t) - c_{\sigma,i}(x,z)$ is given by
	\begin{equation}
		\label{coplanar:eqn:Dc}
		\laplace \fourier \Delta c_{\sigma}(n_{x},z,s) =
		\pm G\!\left(H-z,\, \frac{s}{D_{\sigma}} + n_{x}^{2} \frac{4\pi^{2}}{p_{x}^{2}}\right)
		\frac{\laplace \fourier \Delta j(n_{x},s)}{F n_{e} D_{\sigma}}
	\end{equation}
	where $p_{x}$ is one period of the cell,
	$G(z,s)$ is defined in Eq. (\ref{coplanar:eqn:G}),
	and $\laplace\fourier \Delta j(n_{x},s)$ is the Laplace transform
	of the Fourier coefficients of $\Delta j(x,t) = j(x,t) - j_{i}(x)$.
\end{teorema}
See Supplementary Information \S\ref{transforms} for the definitions of the Fourier coefficients and the Laplace transform used in the previous theorem.

\begin{proof}
	First, substract Eqs. (\ref{coplanar:eqn:difusion:idae})
	and (\ref{coplanar:eqn:difusion:estacionaria})
	to obtain the following partial differential equation
	\begin{subequations}
		\begin{align}
			\frac{1}{D_{\sigma}} \parderiv{\Delta c_{\sigma}}{t}(x,z,t)
			&= \parderiv{^2 \Delta c_{\sigma}}{x^2}(x,z,t)
			+ \parderiv{^2 \Delta c_{\sigma}}{z^2}(x,z,t)
			\\
			\Delta c_{\sigma}(x,z,0^{-}) &= 0
			\\
			\mp D_{\sigma} \parderiv{\Delta c_{\sigma}}{z}(x,H,t) &= 0
			\\
			\mp D_{\sigma} \parderiv{\Delta c_{\sigma}}{z}(x,0,t)
			&= \frac{\Delta j(x,t)}{F n_{e}}
			\\
			\Delta c_{\sigma}(x, z, t) &= \Delta c_{\sigma}(x + p_{x}, z, t)
		\end{align}
	\end{subequations}
	which depends on the changes of concentration and current density
	with respect to the initial condition.

	By taking the Fourier coefficients in $x$ and the Laplace transform in $t$,
	one can convert this problem into an ordinary differential equation
	\begin{subequations}
		\begin{align}
			\left(
				\frac{s}{D_{\sigma}} + n_{x}^{2} \frac{4\pi^{2}}{p_{x}^{2}}
			\right)
			\laplace \fourier \Delta c_{\sigma}(n_{x},z,s)
			&= \parderiv{^2 \laplace \fourier \Delta c_{\sigma}}{z^2}(n_{x},z,s)
			\\
			\mp D_{\sigma} \parderiv{\laplace \fourier \Delta c_{\sigma}}{z}(n_{x},H,s)
			&= 0
			\\
			\mp D_{\sigma} \parderiv{\laplace \fourier \Delta c_{\sigma}}{z}(n_{x},0,s)
			&= \frac{\laplace \fourier \Delta j(n_{x},s)}{F n_{e}}
		\end{align}
	\end{subequations}
	of which its solution
	\begin{multline}
		\laplace \fourier \Delta c_{\sigma}(n_{x},z,s) =
		A(n_{x},s) \cosh\!\left(
			\sqrt{\frac{s}{D_{\sigma}} + n_{x}^{2} \frac{4\pi^{2}}{p_{x}^{2}}}
			(H - z)
		\right)
		\\
		+ B(n_{x},s) \sinh\!\left( 
			\sqrt{\frac{s}{D_{\sigma}} + n_{x}^{2} \frac{4\pi^{2}}{p_{x}^{2}}}
			(H - z)
		\right)
	\end{multline}
	after applying the boundary conditions, 
	is given by Eq. (\ref{coplanar:eqn:Dc}),
	where $G(z,s)$ is defined in Eq. (\ref{coplanar:eqn:G}).
\end{proof}

Before obtaining the properties for the concentration in transient state,
it is useful to obtain the time-domain counterparts of the frequency-domain function $G(z,s)$ in Eq. \eqref{coplanar:eqn:G}.
\begin{lema}
	\label{coplanar:lem:g-h}
	Consider the transfer function $G(z,s)$
	which is defined in Eq. (\ref{coplanar:eqn:G}).
	The inverse Laplace transform $g(z,t) = \laplace^{-1} G(z,s)$
	is given by
	\begin{equation}
		\label{coplanar:eqn:g}
		g(z,t)
		= \frac{1}{H}
		\left[
			1 + 2 \sum_{k=1}^{+\infty} (-1)^{k}
			\cos\!\left( k \frac{\pi}{H} z \right)
			\exp\!\left( -k^{2} \frac{\pi^{2}}{H^{2}} t \right)
		\right]
	\end{equation}
	where the argument of its exponential factors correspond to the poles of $G(z,s)$.
	Note that $g(z,t) = H^{-1} \theta_{4}(z\pi/2H|\bm{i}\pi t/H^{2})$ 
	\citex{Eq. (\dlmf[E]{20.10.}{5})}{dlmf} is related to the \emph{elliptic theta function}
	$\theta_{4}(z|\tau) = \theta_{4}(z,q)$ \citex{Eq. (\dlmf[E]{20.2.}{4})}{dlmf}
	where $q = \exp(\bm{i}\pi\tau)$ \cite[\S\dlmf{20.}{1}]{dlmf}.

	And the inverse Laplace transform $h(z,t) = \laplace^{-1} G(z,s)^{-1}$ is given by
	\begin{equation}
		\label{coplanar:eqn:h}
		h(z,t) =
		\frac{2}{z} \sum_{\ell=1}^{+\infty} (2\ell - 1)^{2} \frac{\pi^{2}}{4 z^{2}} (-1)^{\ell}
		\sin\!\left( (2\ell - 1) \frac{\pi H}{2z} \right)
		\exp\!\left( -(2\ell - 1)^{2} \frac{\pi^{2}}{4 z^{2}} t \right)
	\end{equation}
	where the argument of its exponential factors corresponds to the zeros of $G(z,s)$.
	Note that $h(z,t) = z^{-1} \dot{\theta}_{1}(H\pi/2z|\bm{i}\pi t/z^{2})$
	\citex{Eq. (\dlmf[E]{20.10.}{4})}{dlmf} 
	is related to time derivative of the \emph{elliptic theta function}
	$\theta_{1}(z|\tau) = \theta_{1}(z,q)$ \citex{Eq. (\dlmf[E]{20.2.}{1})}{dlmf}
	where $q = \exp(\bm{i}\pi\tau)$ \cite[\S\dlmf{20.}{1}]{dlmf}.
\end{lema}

\begin{proof}
	From \citex{Eq. (\dlmf[E]{20.10.}{5})}{dlmf}
	\begin{equation}
		\laplace \theta_{4}\!\left( \frac{z\pi}{2H} \left| \frac{\bm{i}\pi t}{H^{2}} \right. \right)
		= H G(z,s)
	\end{equation}
	Finally, we let $g(z,t) = H^{-1} \theta_{4}(z\pi/2H|\bm{i}\pi t/H^{2})$.
	
	From \citex{Eq. (\dlmf[E]{20.10.}{4})}{dlmf} and
	the property of the Laplace transform of the time derivative
	\begin{equation}
		\laplace \deriv{}{t} \theta_{1}\!\left( \frac{H\pi}{2z} \left| \frac{\bm{i}\pi t}{z^{2}} \right. \right)
		= s\, \laplace \theta_{1}\!\left( \frac{H\pi}{2z} \left| \frac{\bm{i}\pi t}{z^{2}} \right. \right)
		= z\, G(z,s)^{-1}
	\end{equation}
	Finally, we let $h(z,t) = z^{-1} \dot{\theta}_{1}(H\pi/2z|\bm{i}\pi t/z^{2})$.
\end{proof}

The first transient property that can be obtained from Theorem \ref{coplanar:teo:Dc}
holds for the horizontal average of $c_{\sigma}(x,z,t)$.
Note that unlike Corollary \ref{coplanar:cor:bar_ci},
the Corollary below shows that the horizontal average is not uniform along $z$
and also changes with time.

\begin{corolario}
	\label{coplanar:cor:delta_bar_c}
	Consider the periodic cell described in \S\ref{coplanar:def:problema}
	and assume that the initial concentration $c_{\sigma,i}(x,z)$ comes from a previous steady state.

	The average of the concentration, along any horizontal line, equals
	\begin{subequations}
		\label{coplanar:eqn:bar_c}
		\begin{gather}
			\frac{1}{p_{x}}
			\int_{-p_{x}/2}^{+p_{x}/2} c_{\sigma}(x,z,t) \ud{x}
			= \bar{c}_{\sigma,i} \pm \Delta \bar{c}_{\sigma}(z,t)
			\\
			\intertext{%
				where the change in average concentration $\Delta\bar{c}_{\sigma}(z,t)$
				depends on $z$, $t$, the electrochemical species $\sigma$,
				and the net current in a period of the cell
			}
			\label{coplanar:eqn:delta_bar_c}
			\Delta \bar{c}_{\sigma}(z,t)
			= g(H-z, D_{\sigma}t) * \frac{1}{p_{x}}
			\int_{-p_{x}/2}^{+p_{x}/2} \frac{j(x,t)}{F n_{e}} \ud{x}
		\end{gather}
	\end{subequations}
	and where $g(z,t) = \laplace^{-1} G(z,s)$ is given in Eq. (\ref{coplanar:eqn:g}).

	Conversely,
	the average current density (net current) in one period of the cell
	is dependent on $t$,
	and on the change in average concentration $\Delta\bar{c}_{\sigma}(0,t)$
	at the bottom of cell (where the electrodes are located)
	\begin{equation}
		\label{coplanar:eqn:i_net}
		\frac{1}{p_{x}} \int_{-p_{x}/2}^{+p_{x}/2} j(x,t) \ud{x}
		= F n_{e} D_{\sigma}^{2}\, h(H, D_{\sigma} t) * \Delta\bar{c}_{\sigma} (0,t)
	\end{equation}
	where $h(z,t) = \laplace^{-1} G(s,z)^{-1}$ is given by Eq. (\ref{coplanar:eqn:h}).

	In both cases, $*$ is the time convolution, $p_{x}$ is one period of the cell, and
	$\bar{c}_{\sigma,i}$ is the horizontal average of the initial concentration,
	see Eq. (\ref{coplanar:eqn:bar_ci}).
\end{corolario}

\begin{proof}
	Take the expression for $\laplace\fourier \Delta c_{\sigma}(0,z,s)$ from Eq. (\ref{coplanar:eqn:Dc}).
	\begin{equation}
		\frac{1}{p_{x}}
		\int_{-p_{x}/2}^{+p_{x}/2} \laplace \Delta c_{\sigma}(x,z,s) \ud{x} =
		\pm G\!\left( H-z, \frac{s}{D_{\sigma}} \right)
		\frac{1}{p_{x}}
		\int_{-p_{x}/2}^{+p_{x}/2} \frac{\laplace\Delta j(x,s)}{F n_{e} D_{\sigma}} \ud{x}
	\end{equation}
	Let $g(z,t) = \laplace^{-1} G(z,s)$,
	then the previous equation can be written in time domain
	by applying the inverse Laplace transform
	together with the \emph{time scaling} property
	\begin{equation}
		\frac{1}{p_{x}} \int_{-p_{x}/2}^{+p_{x}/2} \Delta c_{\sigma}(x,z,t) \ud{x} =
		 \pm D_{\sigma} g(H-z, D_{\sigma}t) * \frac{1}{p_{x}}
		\int_{-p_{x}/2}^{+p_{x}/2} \frac{\Delta j(x,t)}{F n_{e} D_{\sigma}} \ud{x}
	\end{equation}
	By adding Eq. (\ref{coplanar:eqn:bar_ci}) to the previous equation,
	and later, by applying Eq. (\ref{coplanar:eqn:kirchhoff:inicial}),
	leads to Eqs. (\ref{coplanar:eqn:bar_c}).

	By taking Eq. (\ref{coplanar:eqn:delta_bar_c}) in Laplace domain
	and later by isolating the average current density (net current) one obtains
	\begin{equation}
		\frac{1}{p_{x}} \int_{-p_{x}/2}^{+p_{x}/2} \frac{\laplace j(x,s)}{D_{\sigma} F n_{e}} \ud{x}
		= G\!\left( H - z, \frac{s}{D_{\sigma}} \right)^{-1} \laplace \Delta\bar{c}_{\sigma}(z,s)
	\end{equation}
	Let $h(z,t) = \laplace^{-1} G(z,s)^{-1}$,
	then the previous equation can be written in time domain
	by applying the inverse Laplace transform
	together with the \emph{time scaling} property
	\begin{equation}
		\frac{1}{p_{x}} \int_{-p_{x}/2}^{+p_{x}/2} \frac{j(x,t)}{D_{\sigma} F n_{e}} \ud{x}
		= D_{\sigma}\, h(H - z, D_{\sigma} t) * \Delta\bar{c}_{\sigma} (z,t)
	\end{equation}
	Since the average current density is independent of $z$,
	it suffices to take $z=0$, leading to Eq. (\ref{coplanar:eqn:i_net}).
\end{proof}

The second transient property is a \emph{conservation} property,
and holds for the total concentration at any point in the cell,
and any time $t \geq 0$.

\begin{corolario}
	\label{coplanar:cor:ct:total}
	Consider the periodic cell with period $p_{x}$,
	described in \S\ref{coplanar:def:problema},
	and assume that the initial concentration $c_{\sigma,i}(x,z)$ comes from a previous steady state.
	If the diffusion coefficients of both species are equal $D_{O} = D_{R}$,
	then the sum of the concentrations at any point in the cell is independent of $(x,z,t)$ and equals
	\begin{equation}
		\label{coplanar:eqn:ct:total}
		c_{O}(x,z,t) + c_{R}(x,z,t) = \bar{c}_{O,i} + \bar{c}_{R,i}
	\end{equation}
	where $\bar{c}_{\sigma,i}$ with $\sigma \in \{O,R\}$ is given in Eq. (\ref{coplanar:eqn:bar_ci}).
\end{corolario}

\begin{proof}
	If $D_{O} = D_{R}$, then the sum of Eq. (\ref{coplanar:eqn:Dc}) for both electrochemical species is
	\begin{equation}
		\laplace\fourier \Delta c_{O}(n_{x},z,s)
		+ \laplace\fourier \Delta c_{R}(n_{x},z,s) = 0
	\end{equation}
	Taking the inverse Laplace transform and later the Fourier series, one obtains
	\begin{equation}
		\Delta c_{O}(x,z,t) + \Delta c_{R}(x,z,t) = 0
	\end{equation}
	Finally, by adding Eq. (\ref{coplanar:eqn:ci_total}), Eq. (\ref{coplanar:eqn:ct:total}) is obtained.
\end{proof}


\subsection{Properties of the final concentration in steady state}

\emph{Conservation} properties similar to those in \S\ref{coplanar:properties:initial}
also hold for the final concentration in steady state,
which araise as direct consequences of the theorem stated below.

\begin{teorema}
	\label{coplanar:teo:cf}
	Consider the periodic cell described in \S\ref{coplanar:def:problema}.
	If the initial concentration $c_{\sigma,i}(x,z)$ of species $\sigma \in \{O, R\}$
	comes from a previous steady state and the following integral converges
	\begin{equation}
		\label{coplanar:eqn:i_neta_acumulada}
		\int_{0^{-}}^{+\infty} \int_{-p_{x}/2}^{+p_{x}/2} j(x,t) \ud{x} \ud{t}
	\end{equation}
	then the final current density $j_{f}(x) = \lim_{t \to +\infty} j(x,t)$
	satisfies Kichhoff's current law in one period of the cell
	\begin{equation}
		\label{coplanar:eqn:kirchhoff:final}
		\int_{-p_{x}/2}^{+p_{x}/2} j_{f}(x) \ud{x} = 0
	\end{equation}
	and the Fourier coefficients $\fourier c_{\sigma,f}(n_{x},z)$ of the final concentration 
	$c_{\sigma,f}(x,z) = \lim_{t \to +\infty} c_{\sigma}(x,z,t)$ are given by
	\begin{subequations}
		\begin{align}
			\fourier c_{\sigma,f}(n_{x},z) &=
			\left\{
				\begin{array}{ll}
					\displaystyle
					\bar{c}_{\sigma,f} =
					\bar{c}_{\sigma,i} \pm \Delta \bar{c}_{f}, & n_{x} = 0
					\\[1em]
					\displaystyle
					\pm G\!\left(H-z,\, n_{x}^{2} \frac{4\pi^{2}}{p_{x}^{2}}\right)
					\frac{\fourier j_{f}(n_{x})}{F n_{e} D_{\sigma}}, & n_{x} \neq 0
				\end{array}
			\right.
			\\[0.5em]
			\label{coplanar:eqn:delta_bar_cf}
			\Delta \bar{c}_{f} &= \frac{1}{H}
			\int_{0^{-}}^{+\infty} \frac{1}{p_{x}} \int_{-p_{x}/2}^{+p_{x}/2}
			\frac{j(x,t)}{F n_{e}} \ud{x} \ud{t}
		\end{align}
	\end{subequations}
	where $p_{x}$ is one period of the cell,
	$\bar{c}_{\sigma,i}$ is the horizontal average of the initial concentration
	defined in Eq. (\ref{coplanar:eqn:bar_ci}),
	and $G(z,s)$ is defined in Eq. (\ref{coplanar:eqn:G}).
\end{teorema}

Note from the theorem above that Eq. \eqref{coplanar:eqn:kirchhoff:final}
(that is, Kirchoff's current law be satisfied in steady state, or equivalently,
100\% collection efficiency in the final steady state)
is a \textbf{necessary condition}
for the convergence of the concentration profile in the final steady state.

\begin{proof}
	The final steady state can be obtained
	if one applies the \emph{final value theorem} of the Laplace transform
	to $\laplace \fourier \Delta c_{\sigma}(n_{x},z,s)$ in Eq. (\ref{coplanar:eqn:Dc})
	\begin{equation}
		\fourier \Delta c_{\sigma,f}(n_{x},z) =
		\lim_{t \to +\infty} \fourier \Delta c_{\sigma}(n_{x},z,t) =
		\lim_{s \to 0} s\, \laplace \fourier \Delta c_{\sigma}(n_{x},z,s)
	\end{equation}
	Separating the limits, according to the following equation,
	aids in the calculation of the Fourier coefficients in steady state
	\begin{multline}
		\fourier \Delta c_{\sigma,f}(n_{x},z) = \\
		\begin{cases}
			\displaystyle
			\pm \lim_{s \to 0} s\, G\!\left(H-z,\, \frac{s}{D_{\sigma}}\right)
			\cdot \lim_{s \to 0} s\, \frac{1}{s}
			\frac{\laplace \fourier \Delta j(0,s)}{F n_{e} D_{\sigma}},
			& n_{x} = 0
			\\[1em]
			\displaystyle
			\pm \lim_{s \to 0} G\!\left(H-z,\, \frac{s}{D_{\sigma}}
			+ n_{x}^{2} \frac{4\pi^{2}}{p_{x}^{2}}\right)
			\cdot \lim_{s \to 0} s\,
			\frac{\laplace \fourier \Delta j(n_{x},s)}{F n_{e} D_{\sigma}},
			& n_{x} \neq 0
		\end{cases}
	\end{multline}
	where the final value of $\fourier \Delta j(n_{x},t)$ is given by
	\begin{equation}
		\fourier \Delta j_{f}(n_{x}) =
		\lim_{t \to +\infty} \fourier \Delta j(n_{x},t) =
		\lim_{s \to 0} s\, \laplace \fourier \Delta j(n_{x},s)
	\end{equation}
	the final value of its time integral is given by
	\begin{equation}
		\int_{0^{-}}^{+\infty} \fourier \Delta j(n_{x},t) \ud{t} =
		\lim_{s \to 0} s\, \frac{1}{s} \laplace \fourier \Delta j(n_{x},s)
	\end{equation}
	and the following limit equals
	\begin{equation}
		\lim_{s \to 0} s\, G\!\left(H-z,\, \frac{s}{D_{\sigma}}\right) =
		\frac{D_{\sigma}}{H}
	\end{equation}
	These lead to the result in steady state
	\begin{equation}
		\fourier \Delta c_{\sigma,f}(n_{x},z) =
		\left\{
			\begin{array}{ll}
				\displaystyle
				\pm \frac{D_{\sigma}}{H} \cdot
				\int_{0^{-}}^{+\infty}
				\frac{\fourier \Delta j(0,t)}{F n_{e} D_{\sigma}}
				\ud{t}, & n_{x} = 0
				\\[1em]
				\displaystyle
				\pm G\!\left(H-z,\, n_{x}^{2} \frac{4\pi^{2}}{p_{x}^{2}}\right)
				\frac{\fourier \Delta j_{f}(n_{x})}{F n_{e} D_{\sigma}}, & n_{x} \neq 0
			\end{array}
		\right.
	\end{equation}

	Therefore, the Fourier coefficients of the full-scale concentrations are obtained by adding Eq. (\ref{coplanar:eqn:ci})
	\begin{equation}
		\fourier c_{\sigma,f}(n_{x},z) =
		\left\{
			\begin{array}{ll}
				\displaystyle
				\bar{c}_{\sigma,i} \pm \frac{1}{H}
				\int_{0^{-}}^{+\infty}
				\frac{\fourier j(0,t)}{F n_{e}}
				\ud{t}, & n_{x} = 0
				\\[1em]
				\displaystyle
				\pm G\!\left(H-z,\, n_{x}^{2} \frac{4\pi^{2}}{p_{x}^{2}}\right)
				\frac{\fourier j_{f}(n_{x})}{F n_{e} D_{\sigma}}, & n_{x} \neq 0
			\end{array}
		\right.
	\end{equation}
	where $\fourier j(0,t) = \fourier j(0,t) - \fourier j_{i}(0) = \fourier \Delta j(0,t)$
	due to Eq. (\ref{coplanar:eqn:kirchhoff:inicial}).
\end{proof}

Considering the previous result,
the first \emph{conservation} property holds for the horizontal average of $c_{\sigma.f}(x,z)$, at any $z$ of the cell,
which may deviate from its initial counterpart
due to unbalanced currents (Kirchhoff's law not satisfied) during the transient state.

\begin{corolario}
	\label{coplanar:cor:bar_cf}
	Assume that the initial concentration $c_{\sigma,i}(x,z)$ comes from a previous steady state
	and the time integral of the net current in Eq. (\ref{coplanar:eqn:i_neta_acumulada}) converges.
	The average of the final concentration, along any horizontal line,
	is independent of $z$ and equals 
	%$\bar{c}_{\sigma,f} = \bar{c}_{\sigma,i} \pm \Delta \bar{c}_{f}$
	\begin{equation}
		\label{coplanar:eqn:bar_cf}
		\frac{1}{p_{x}} \int_{-p_{x}/2}^{+p_{x}/2} c_{\sigma,f}(x,z) \ud{x} =
		\fourier c_{\sigma,f}(0,z) =
		\bar{c}_{\sigma,f} =
		\bar{c}_{\sigma,i} \pm \Delta \bar{c}_{f}
	\end{equation}
	where $p_{x}$ is one period of the cell
	and $\Delta \bar{c}_{f}$ is independent of the electrochemical species,
	but is proportional to the time integral of the net current,
	as shown in Eq. (\ref{coplanar:eqn:delta_bar_cf}).
\end{corolario}

The second \emph{conservation} property holds for the weighted sum of concentrations
at any point in the cell,
which translates into the total concentration at any point in the cell
when the diffusion coefficients of both species are equal.

\begin{corolario}
	\label{coplanar:cor:cf:total}
	Consider the periodic cell with period $p_{x}$ described in \S\ref{coplanar:def:problema},
	and assume that the initial concentration $c_{\sigma,i}(x,z)$ comes from a previous steady state
	and the time integral of the net current in Eq. (\ref{coplanar:eqn:i_neta_acumulada}) converges. 
	The following weighted sum of the final concentrations is independent of $(x,z)$ and equals
	\begin{equation}
		D_{O} c_{O,f}(x,z) + D_{R} c_{R,f}(x,z) =
		D_{O} \underbrace{(\bar{c}_{O,i} + \Delta \bar{c}_{f})}_{\bar{c}_{O,f}} +
		D_{R} \underbrace{(\bar{c}_{R,i} - \Delta \bar{c}_{f})}_{\bar{c}_{R,f}}
	\end{equation}
	where $\Delta \bar{c}_{f}$ is independent of the electrochemical species,
	but is proportional to the time integral of the net current,
	as shown in Eq. (\ref{coplanar:eqn:delta_bar_cf}).
\end{corolario}

\begin{proof}
	Take the weighted sum of both Fourier coefficients
	\begin{multline}
		D_{O} \fourier c_{O,f}(n_{x},z) + D_{R} \fourier c_{R,f}(n_{x},z)
		\\
		= \begin{cases}
			D_{O} (\bar{c}_{O,i} + \Delta \bar{c}_{f}) +
			D_{R} (\bar{c}_{R,i} - \Delta \bar{c}_{f}), & n_{x} = 0
			\\
			0, & n_{x} \neq 0
		\end{cases}
	\end{multline}
	and later, take its Fourier series.
\end{proof}
