% !TeX root = article
% !TeX encoding = utf8
% !TeX spellcheck = en_US


\section{Conclusions}

The properties of horizontal average and (weighted) sum of concentrations
show several implications in the behavior of
a periodic cell with finite height and two-dimensional symmetry.

In the initial steady state,
the net current is zero (100\% collection efficiency) and both,
the horizontal average and the (weighted) sum of concentrations of both species,
are uniform in the cell, see Corollaries \ref{coplanar:cor:bar_ci} and \ref{coplanar:cor:ci:total}.

During the transtient state,
a change in the average of concentration at the bottom of the cell
(where the electrodes are located) produces a non-zero net current
(collection efficiency less than 100\%),
see Eq. \eqref{coplanar:eqn:i_net} in Corollary \ref{coplanar:cor:delta_bar_c}.
Subsequently, this non-zero net current during the transient
produces accumulation (or depletion) of species in the cell,
see Eq. \eqref{coplanar:eqn:delta_bar_c}
in Corollary \ref{coplanar:cor:delta_bar_c}.
The duration of the transient is governed by the slowest exponential mode
in Eq. \eqref{coplanar:eqn:i_net}, and increases as the height $H$ increases.

In the final steady state, the net current must be zero 
(100\% collection efficiency)
despite any non-zero net current during the transient.
The horizontal average and (weighted) sum of concentrations
become again uniform in the cell, but may not equal their initial counterparts
due to accumulation (or depletion) of species.
See Corollaries \ref{coplanar:cor:bar_cf} and \ref{coplanar:cor:cf:total}.

Note that this accumulation (or depletion) of species
is not present during the transient and final steady states,
when the average of concentration at the bottom of the cell
remains at the same value as its initial counterpart.
In this case the net current always equals zero,
therefore the horizontal average and (weighted) sum of concentrations
maintain the same value in the initial and final steady states.

If the cell has semi-infinite geometry $H \to +\infty$,
the final steady state behaves differently.
When the average concentration at the bottom of the cell
is driven out from its initial counterpart,
the final net current becomes non-zero (collection efficiency less than 100\%) and the horizontal average of final concentration loses its uniformity along the $z$-axis.

These properties of horizontal averages and (weighted) sum of concentrations
are also useful for determining the concentration on a counter electrode
(which is not known a priori), and to determine non-linearities
caused by depletion of electrochemical species
at electrodes potentiostated at extreme voltages.
This can be seen for the case of IDAE
in Theorems \ref{idae:teo:cE-cE':intC} and \ref{idae:teo:cE-cE':extC},
and it is specially important to take into account when performing simulations.

More results have been found for IDAE using the previous properties.
Normally, the IDAE is operated in dual mode (voltages are applied at each array)
either with an external or internal counter electrode,
of which the former is most commonly found in the literature.
Comparing both modes, the results
in Lema \ref{idae:lem:if} and Theorems \ref{idae:teo:cE-cE':intC} and \ref{idae:teo:cE-cE':extC}
show that the maximum current range that can be spanned in steady state,
either when using external or internal counter electrode, is the same.
Also, it is shown that the time the current requires to reach steady state
in case of using external counter electrode is longer
than that required in case of using internal counter electrode,
see Eqs. \eqref{idae:eqn:tau:extC} and \eqref{idae:eqn:tau:intC} respectively.
This suggests that, despite of being more common in the literature,
an IDAE configuration with external counter electrode provides
no significant advantage compared with the case of internal counter electrode.
This is true, however, under the restrictions
of Theorems \ref{idae:teo:cE-cE':intC} and \ref{idae:teo:cE-cE':extC},
where the IDAE has bands of equal width
and the width of the counter electrode equals that of the whole IDAE.
