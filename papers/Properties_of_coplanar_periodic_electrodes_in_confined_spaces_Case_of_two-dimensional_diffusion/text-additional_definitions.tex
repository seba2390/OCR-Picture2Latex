% !TeX root = article
% !TeX encoding = utf8
% !TeX spellcheck = en_US


\section{Additional definitions}

\subsection{Fourier series and Laplace transform}
\label{transforms}

Fourier series and Laplace transform are extensively used in \S\ref{theory},
therefore they are defined briefly in this section.

\begin{definicion}
	The exponential version of the Fourier coefficients $\fourier f(n_{x})$
	and Fourier series of the periodic function $f(x)$, with period $p_{x}$,
	are defined by
	\begin{subequations}
		\label{transforms:eqn:fourier}
		\begin{align}
			\fourier f(n_{x}) &= \frac{1}{p_{x}}
			\int_{-p_{x}/2}^{+p_{x}/2} f(x)\, \e^{-\bm{i} x\, n_{x} 2\pi/p_{x}} \ud{x}
			\\
			f(x) &= \sum_{n_{x} = -\infty}^{+\infty} \fourier f(n_{x})\,
			\e^{\bm{i} x\, n_{x} 2\pi/p_{x}}
		\end{align}
	\end{subequations}
	where $\bm{i}$ is the imaginary unit.
\end{definicion}

\begin{definicion}
	The Laplace transform $\laplace f(s)$ of the time function $f(t)$,
	and its inverse, are defined respectively as
	\begin{subequations}
		\begin{align}
			\laplace f(s) &= \int_{0^{-}}^{+\infty} f(t)\, \e^{-st} \ud{t}
			\\
			f(t) &= \frac{1}{\bm{i} 2\pi} \oint_{\Gamma} \laplace f(s)\, \e^{st} \ud{s}
		\end{align}
	\end{subequations}
	where $\bm{i}$ is the imaginary unit,
	and $\Gamma$ is a closed path in the complex plane,
	surrounding the poles of $\laplace f(s)$.
\end{definicion}
