% !TeX root = article
% !TeX encoding = utf8
% !TeX spellcheck = en_US


\section{Introduction}

Microelectrodes have been used since early 1980 \cite{Dayton:1980:}
due to their many advantageous properties, such as reduced ohmic drops,
faster time constants, better signal-to-noise ratios
and steady-state signals \cite{Forster:2007:,Szunerits:2007:}.
These electrodes have also been arranged in a periodic fashion (arrays),
in order to produce higher currents, while still maintaining
the basic microelectrode properties \cite{Szunerits:2007:}.
From these periodic configurations, \emph{microband array electrodes} (MBAE, only anodes or cathodes)
and \emph{interdigitated array of electrodes} (IDAE, alternating anodes and cathodes)
are common examples found in the literature.

Theoretical results for periodic configurations were first obtained
considering unrestricted (semi-infinite) geometries.
Analytical results to predict steady-state currents and voltammograms were found
in case of MBAE \cite{Morf:1996:sep,Morf:2006:may} and
in case of IDAE \cite{Aoki:1988:dec,Aoki:1990:apr,Morf:2006:may}.
Numerical results through simulations have been also obtained
to estimate the time dependence of the current and voltammograms
in case of MBAE \cite{Bard:1986:sep,Streeter:2007:aug,Pebay:2013:dec} and
in case of IDAE \cite{Aoki:1989:jul,Jin:1996:aug:b,Yang:2007:oct}.
%These results are already mature and have been cited in several research works.
Besides these mature results, there are also novel semi-analytical results
predicting the chronoamperometry at microband electrodes \cite{Bieniasz:2015:oct}.

Currently, with the advent of microfluidic technology and flexible materials,
confined (finite) electrochemical cells have gained importance,
since electrochemical cells are placed inside shallow channels \cite{Han:2014:}
or meant to be used in narrow cavities of the body \cite{Kanno:2014:}.

In the literature, the behavior of such electrodes
in restricted or finite spaces has been predicted mostly through simulations,
which allows interpretation of electrochemical phenomena
in case of IDAE \cite{Strutwolf:2005:feb,Goluch:2009:may,Han:2014:,Kanno:2014:}
and in case of MBAE \cite{Bellagha-Chenchah:2016:jun}.
Analytical results to predict the behavior in confined spaces are few
\cite{GuajardoYevenes:2013:sep},
and commonly the results for semi-infinite counterparts are used instead \cite{Shim:2013:},
which are valid only when the cell is tall enough \cite{GuajardoYevenes:2013:sep}.

In this report, analytical properties which are common to any periodic cell
(or any cell that can be extended periodically) with finite height 
and two-dimensional symmetry are derived.
The base analytical result consists of
the concentration profile in stagnant solution,
expresed in terms of its Fourier coeficients,
and considers an arbitrary current density flowing in the cell.
From this result, analytical properties for average concentrations
and weighted sum of concentrations are derived.

These results are of importance since they can explain qualitative aspects
of collection efficiency and limiting currents.
Also they allow to determine the concentration of electrochemical species
on the counter electrode (commonly unknown \emph{a priori}),
which is particularly useful for defining boundary conditions
of simulations that include such electrode.
Explanation of non-linear effects on the concentration
caused by depletion of species at electrodes that are extremely polarized,
is also possible with the results.
Finally estimations of the time required by the current to reach steady state can be obtained.
These properties are illustrated analytically and numerically by simulations
for the particular case of interdigitated array of electrodes.
