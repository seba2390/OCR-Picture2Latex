% !TeX root = article
% !TeX encoding = utf8
% !TeX spellcheck = en_US


\section{Results and discussion}

The results of the theoretical part will be illustrated with a concrete case,
namely, the case of \emph{interdigitated array of electrodes} (IDAE).
For this configuration, it will be seen that the properties of
\emph{horizontal average} and \emph{weighted sum} of concentrations,
together with the physical constraint of \emph{non-negative} concentrations,
impose non-linearities that can affect the limiting current of the cell.
Besides, a rough prediction of the dynamic behavior of the current
and also a prediction of the change in the average concentration on the IDAE is done.
The last two results are contrasted against simulations.


\subsection{Average properties in case of interdigitated arrays}
\label{idae:promedios}

Consider the case of an IDAE configuration
in a cell of height $H$, total width $W_{T}$ and depth $L$,
as shown in Fig. \ref{idae:fig:cell}.
The cell is symmetric along the $y$-axis,
such that a two-dimensional representation $(x,z)$ suffices.
Inside this cell, there are two electrochemical species that react according to Eq. \eqref{coplanar:eqn:reaccion}.

The IDAE consists of two arrays of band electrodes, $A$ (black) and $B$ (gray),
of which two consecutive bands are separated by a center-to-center distance of $W$,
the width of the bands is $2w_{A}$ and $2w_{B}$,
and the number of bands is $N_{A} = N_{B}$ respectively.
The cell may have one of the arrays performing as counter electrode, Fig. \ref{idae:fig:cell:intC},
or have a counter electrode  of width $w_{C}$ external and coplanar to the IDAE, Fig. \ref{idae:fig:cell:extC}.

For the sake of simplicity,
it is assumed that the first and last bands of the IDAE have half width.
Therefore, the IDAE in Fig. \ref{idae:fig:cell:intC} can be represented exactly
as an assembly of units of symmetry of width $W$, height $H$ and half-band electrodes of $A$ and $B$.
Here, each unit of symmetry will be refered to as a \emph{unit cell},
and it is shown in Fig. \ref{idae:fig:cell:unit}.
Similarly, the IDAE in Fig. \ref{idae:fig:cell:extC} can be represented approximately as an assembly of unit cells,
provided that the number of electrode bands $N_{A} = N_{B}$ is sufficiently large,
so that the edge effects at the end of the IDAE are negligible.

\begin{figure}[t]
	\centering
	\subcaptionbox{
		\label{idae:fig:cell:intC}
		Internal counter electrode.}[60mm]{\includegraphics{fig-1a}}
	\subcaptionbox{
		\label{idae:fig:cell:extC}
		External counter electrode.}[60mm]{\includegraphics{fig-1b}}
	\caption{
		Sketch of interdigitated array of electrodes (IDAE) in a cell of finite height $H$,
		total width $W_{T}$ and depth $L$.
		Fig. (\subref{idae:fig:cell:intC}) can be regarded as an assembly of unit cells of width $W$.
		Fig. (\subref{idae:fig:cell:extC}) can be approximately regarded as an assembly of unit cells of width $W$,
		provided that the number of electrode bands is sufficiently large.
		Note that for both IDAEs, the first and the last bands have half width.
	}
	\label{idae:fig:cell}
\end{figure}

Due to the periodic nature of the IDAE configuration,
the average properties in Corollaries \ref{coplanar:cor:bar_ci}, \ref{coplanar:cor:delta_bar_c} and \ref{coplanar:cor:bar_cf}
must be satisfied at each unit cell of the IDAE ($p_{x} = 2W$)
\begin{subequations}
	\label{idae:eqn:bar_c:unit}
	\begin{align}
		\label{idae:eqn:bar_ci:unit}
		\frac{1}{W} \int_{0}^{W} c_{\sigma,i}(x,z) \ud{x}
		&= \bar{c}_{\sigma,i}\unit
		\\
		\label{idae:eqn:bar_ct:unit}
		\frac{1}{W} \int_{0}^{W} c_{\sigma}(x,z,t) \ud{x}
		&= \bar{c}_{\sigma,i}\unit \pm \Delta\bar{c}_{\sigma}\unit(z,t)
		\\
		\label{idae:eqn:bar_cf:unit}
		\frac{1}{W} \int_{0}^{W} c_{\sigma,f}(x,z) \ud{x}
		&= \underbrace{\bar{c}_{\sigma,i}\unit \pm \Delta\bar{c}_{f}\unit}_{\bar{c}_{\sigma,f}\unit}
	\end{align}
\end{subequations}
either when it fits exactly in the whole cell ($W_{T} = 2W N_{E}$, Fig. \ref{idae:fig:cell:intC}),
or when it doesn't ($W_{T} > 2W N_{E}$, Fig. \ref{idae:fig:cell:extC})
but considering a large number of bands $N_{E}$, with $E \in \{A,B\}$.
Note that the horizontal average in the final steady state holds after a sufficiently long time,
comparable with the time constant of the unit cell.

Besides, the same average properties
in Corollaries \ref{coplanar:cor:bar_ci}, \ref{coplanar:cor:delta_bar_c} and \ref{coplanar:cor:bar_cf}
should also hold for the whole cell ($p_{x} = 2W_{T}$)
\begin{subequations}
	\label{idae:eqn:bar_c:whole}
	\begin{align}
		\label{idae:eqn:bar_ci:whole}
		\frac{1}{W_{T}} \int_{0}^{W_{T}} c_{\sigma,i}(x,z) \ud{x}
		&= \bar{c}_{\sigma,i}\whole
		\\
		\label{idae:eqn:bar_ct:whole}
		\frac{1}{W_{T}} \int_{0}^{W_{T}} c_{\sigma}(x,z,t) \ud{x}
		&= \bar{c}_{\sigma,i}\whole 
		\pm \underbrace{\Delta\bar{c}_{\sigma}\whole(z,t)}_{=0}
		\\
		\label{idae:eqn:bar_cf:whole}
		\frac{1}{W_{T}} \int_{0}^{W_{T}} c_{\sigma,f}(x,z) \ud{x}
		&= \bar{c}_{\sigma,i}\whole 
		\pm \underbrace{\Delta\bar{c}_{f}\whole}_{=0}
	\end{align}
\end{subequations}
since it is surrounded by insulating walls,
and therefore it can be extended periodically along the $x$-axis.
Note that $\Delta\bar{c}_{\sigma}\whole(z,t) = \Delta\bar{c}_{f}\whole = 0$,
since the whole cell always contains its counter electrode,
therefore the its horizontal average remains constant for all $t$.
In particular, for the case of internal counter electrode (Fig. \ref{idae:fig:cell:intC}),
the horizontal average in the whole cell equals that in the unit cell
$\bar{c}_{\sigma,i}\whole = \bar{c}_{\sigma,i}\unit$
and remains unchanged for all $t$.

\subsection{Simulations}

Simulations\footnote{Scripts can be obtained from \url{\elektrodo}.}
were performed for the current in the unit cell using the finite volume PDE solver
\href{http://www.ctcms.nist.gov/fipy/}{FiPy} \cite{Guyer:2009:may}.
The numerical results are compared with their theoretical counterparts in the coming sections.

For the sake of simplicity, it is assumed that
the charge transfer on the electrodes follows reversible electrode reactions.
Also it is assumed that the species have equal diffusion coefficients $D_{O} = D_{R} = D$.
Both assumptions mean that the Nernst equation on both electrodes can be decoupled not only in steady state, but also during the transient, due to Corollary \ref{coplanar:cor:ct:total}.
This allows simulating the concentration of each electrochemical species independently.

The simulations consist of a normalized diffusion equation for the unit cell of Fig. \ref{idae:fig:cell:unit}
\begin{subequations}
	\label{idae:eqn:pde:idae:normalized}
	\begin{align}
		\pi^{2} \parderiv{\xi\simu}{t\simu}(x, z, t) 
		&= \parderiv{^2 \xi\simu}{x\simu^{2}}(x, z, t)
		+ \parderiv{^{2} \xi\simu}{z\simu^{2}}(x, z, t) \\
		\parderiv{\xi\simu}{x\simu}(0,z,t)
		&= \parderiv{\xi\simu}{x\simu}(W, z, t) = 0,
		\, \forall z \in [0,H]
	\end{align}
	\vspace{-1.5\baselineskip}
	\begin{gather}
		\parderiv{\xi\simu}{z\simu}(x, H, t) = 0,\, \forall x \in [0,W],
		\: \parderiv{\xi\simu}{z\simu}(x,0,t) = 0,
		\, \forall x \notin A \cup B
		\\
		\xi\simu(x,0,t) = 0,\, \forall x \in A,
		\quad \xi\simu(x,0,t) = 1,\, \forall x \in B
		\label{idae:eqn:pde:idae:normalized:onAB}
	\end{gather}
\end{subequations}
where
\begin{subequations}
	\label{idae:eqn:pde:normalization}
	\begin{gather}
		\xi\simu(x,z,t) = \frac{c_{\sigma}(x,z,t) - c_{\sigma,f}^{A}}{c_{\sigma,f}^{B} - c_{\sigma,f}^{A}}
		\\
		x\simu = \frac{x}{W}
		,\quad z\simu = \frac{z}{W}
		,\quad t\simu = \frac{\pi^{2} D t}{W^{2}}
	\end{gather}
\end{subequations}
which considers the transition  % of the simulated concentration,
from two possible initial states $\xi\simu(x,z,0^{-}) \in \{\numlist[list-pair-separator={,}]{0,25; 0,5}\}$ to its final state $\xi\simu(x,z,+\infty)$.

The width of each band electrode was taken equal to $2w_{A} = 2w_{B} = \num{0,5} W$ for all simulations
and three aspect ratios for the unit cell were considered
$H/W \in \{\numlist[list-final-separator={,}]{0,3; 0,5; 1,0}\}$.

\begin{figure}[t]
	\centering
	\subcaptionbox{
		Mesh: $n_{x} \times n_{z} = 124 \times 37$ and $\delta_{0} = \num{6,25e-5}$.
	}{\includegraphics{fig-2a}}
	\hspace{1em}
	\subcaptionbox{
		Mesh: $n_{x} \times n_{z} = 124 \times 34$ and $\delta_{0} = \num{6,25e-5}$.
	}{\includegraphics{fig-2b}}
	\subcaptionbox{
		Mesh: $n_{x} \times n_{z} = 120 \times 31$ and $\delta_{0} = \num{7,5e-5}$.
	}{\includegraphics{fig-2c}}
	\caption{
		Exponential meshes used in the simulations.
		The size of each mesh is $n_{x} \times n_{z}$
		and the dimensions of its smallest element are $\delta_{x} = \delta_{z} = \delta_{0}$.
	}
	\label{idae:fig:mesh}
\end{figure}

An exponential mesh was used to partition the unit cell \cite[\S7.2]{Britz:2016:},
in order to keep the memory usage low while maintaining good resolution near the electrode bands, see Fig. \ref{idae:fig:mesh}.
The number of elements of the mesh is $n_{x} \times n_{z}$,
of which the width and height of its smallest element are $\delta_{x} = \delta_{z} = \delta_{0}$.
The mesh was succesively refined until the absolute error of the current in steady state,
between two consecutive refinements, was less than \num{0,5e-4}
(which corresponds approximately to four decimal places of agreement between refinemts).
See Suplementary Information \S\ref{data:s32:mesh} for the output of the script of mesh refinement.

Fig. \ref{idae:fig:i_sim} shows the simulated current $i\simu^{E/2}(t)$
through a half-band electrode of $E \in \{A,B\}$
\begin{equation}
	i\simu^{E/2}(t)
	= \int_{E/2} -\parderiv{\xi\simu}{z\simu}(x,0,t) \ud{x\simu}
	= \int_{E/2} -\parderiv{\xi\simu}{z}(x,0,t) \ud{x}
\end{equation}
which was obtained by numerically solving Eqs. (\ref{idae:eqn:pde:idae:normalized})
subject to the initial condition $\xi\simu(x,z,0^{-}) \in \{\numlist[list-pair-separator={,}]{0,25; 0,5}\}$.

\begin{figure*}
	\centering
	\subcaptionbox{
		$H/W = \num{1}$.
		$|i\simu^{E/2}(+\infty)| \approx \num{0,496}$.
		$\Delta\bar{c}_{f}/[c_{\sigma,f}^{B} - c_{\sigma,f}^{A}]
		\approx \{\num{\pm 0,249},\, 0\}$.
	}{
		\includegraphics{fig-3a-xi025} \quad
		\includegraphics{fig-3a-xi05}
	}

	\subcaptionbox{
		$H/W = \num{0,5}$.
		$|i\simu^{E/2}(+\infty)| \approx \num{0,460}$.
		$\Delta\bar{c}_{f}/[c_{\sigma,f}^{B} - c_{\sigma,f}^{A}]
		\approx \{\num{\pm 0,250},\, 0\}$.
	}{
		\includegraphics{fig-3b-xi025} \quad
		\includegraphics{fig-3b-xi05}
	}

	\subcaptionbox{
		$H/W = \num{0,3}$.
		$|i\simu^{E/2}(+\infty)| \approx \num{0,377}$.
		$\Delta\bar{c}_{f}/[c_{\sigma,f}^{B} - c_{\sigma,f}^{A}]
		\approx \{\num{\pm 0,250},\, 0\}$.
	}{
		\includegraphics{fig-3c-xi025} \quad
		\includegraphics{fig-3c-xi05}
	}
	\caption{
		Simulated current $i\simu^{E/2}(t) = \int_{E/2} -\partial \xi\simu(x,0,t)/ \partial z\simu \ud{x\simu}$ through a half-band electrode of $E \in \{A,B\}$,
		as a function of time, for different aspect ratios of the unit cell $H/W$ and initial concentrations $\xi\simu(x,z,0^{-})$.
		The `$\times$' show the time required by the simulated current to reach \SI{0,67}{\percent} of its steady-state value.
		Left column:
		Average concentrations in initial and final steady states are different $\xi\simu(x,z,0^{-}) \neq 1/2$.
		Right column:
		Average concentrations in initial and final steady states are equal $\xi\simu(x,z,0^{-}) = 1/2$.
		The simulations were obtained by numerically solving Eqs. (\ref{idae:eqn:pde:idae:normalized}).
	}
	\label{idae:fig:i_sim}
\end{figure*}

\subsection{Effect of the counter electrode on the net current}

The fact of having an IDAE with internal or external counter electrode
influences the time that its current requires to reach steady state
and also its collection efficiency.
Both effects can be obtained as consecuence of the average properties of
Corollaries \ref{coplanar:cor:delta_bar_c} and \ref{coplanar:cor:bar_cf},
and will be discussed below.

\subsubsection{Time to reach steady state}

When using an external counter electrode (both arrays are potentiostated),
the average concentration at the bottom of the unit cell ($z = 0$)
is in general forced to a value different than its initial counterpart $\bar{c}_{\sigma,i}\unit$.
In case of the simulation in Eqs. \eqref{idae:eqn:pde:idae:normalized} and \eqref{idae:eqn:pde:normalization},
this equals the arithmetic average of the concentrations on both electrodes,
due to symmetry, since the electrode bands have equal width $2w_{A} = 2w_{B}$
\begin{equation}
	\label{idae:eqn:bar_x_sim}
	\frac{1}{W} \int_{0}^{W} \xi\simu(x,0,t) \ud{x} = \frac{1}{2}
	\Leftrightarrow
	\frac{1}{W} \int_{0}^{W} c_{\sigma}(x,0,t) \ud{x}
	= \frac{c_{\sigma,f}^{A} + c_{\sigma,f}^{B}}{2}
\end{equation}
which is different from its initial counterpart when $\xi\simu(x,z,0^{-}) = \num{0,25}$.
This produces a change of $\pm \Delta\bar{c}_{\sigma}\unit(0,t) \neq 0$
in the average concentration at the bottom of the unit cell,
see Eq. \eqref{idae:eqn:bar_ct:unit},
which subsequently generates a non-zero net current in the unit cell during transient state,
due to Eq. \eqref{coplanar:eqn:i_net} with $p_{x} = 2W$.

The simulations at the left column of Fig. \ref{idae:fig:i_sim} show
the generation of a net current in the unit cell when $\xi\simu(x,z,0^{-}) = \num{0,25}$.
Note that the net current in the unit cell,
as well as the current at each half-band electrode,
have similar dynamics and reach steady state nearly at the same time.
This time can be predicted from Eq. (\ref{coplanar:eqn:i_net}) in Corollary \ref{coplanar:cor:delta_bar_c},
since the net current in the unit cell has natural modes of the form
\begin{equation}
	\label{idae:eqn:tau:extC}
	\exp\!\left( -(2\ell - 1)^{2} \frac{W^{2}}{4H^{2}} \cdot \frac{\pi^{2}}{W^{2}} D_{\sigma} t \right),\quad \ell = 1,2,\ldots
\end{equation}
from its impulse response $F n_{e} D_{\sigma}^{2}\, h(H,D_{\sigma}t)$, which decay exponentially with time.
The slowest of these exponential modes, that is with $\ell=1$,
is the one that gives an idea of the time required to reach steady state.
This time is roughly approached when $\pi^{2} D_{\sigma} t/W^{2} = 5 \cdot (2H/W)^{2}$,
that is when the slowest exponential mode approximately vanishes $\exp(-5) \approx \SI{0,67}{\percent}$.
The left column of Fig. \ref{idae:fig:i_sim} shows with `$\times$'
the times needed for the simulated current to reach \SI{0,67}{\percent} of its steady-state value,
which correspond roughly to their theoretical counterparts: 20, 5 and \num{1,8}.

On the other hand, when one of the arrays performs as counter electrode (internal counter),
the net current in the unit cell must remain always zero.
This fact suggests that the average concentration at the bottom of the unit cell
is forced by the potentiostat to its initial counterpart $\bar{c}_{\sigma,i}\unit$.
In case of the simulation,
this average is forced to $\xi\simu(x,z,0^{-}) = \num{0,5}$.
This produces no change in average at the bottom of the unit cell
$\Delta \bar{c}_{\sigma}\unit(0,t) = 0$,
which is the cause of having zero net current during the transient.

The simulations at the right column of Fig. \ref{idae:fig:i_sim} show zero net current when $\xi\simu(x,z,0^{-}) = \num{0,5}$.
Despite the net current in the unit cell is zero,
the current at each array does evolve with time,
reaching its steady state in a shorter time than in the case of external counter
(compare with left column of Fig. \ref{idae:fig:i_sim}).
This behavior can be explained by looking at the Fourier series of the current density
\begin{equation}
	j(x,t) = \sum_{n_{x}=-\infty}^{+\infty} \fourier j(n_{x},t) \e^{\bm{i} x\, n_{x} \pi/W}
\end{equation}
where $\fourier j(n_{x},t)$ correspond to its Fourier coefficients.
Note that the Fourier coefficient $\fourier j(n_{x},t)$ with $n_{x}=0$
corresponds to the average component of the current density (net current) in the unit cell,
which equals zero when the counter electrode is internal to the IDAE.
Therefore, only the Fourier coefficients $\fourier j(n_{x},t)$ with $n_{x} \neq 0$ vary with time,
and they do so according to the impulse response
\begin{equation}
	\label{idae:eqn:h}
	% F n_{e} D_{\sigma}\,
	\laplace^{-1} G\!\left( H, \frac{s}{D_{\sigma}} + n_{x}^{2} \frac{\pi^{2}}{W^{2}} \right)^{-1}
	=
	% F n_{e} D_{\sigma}^{2}\,
	D_{\sigma}\,
	h(H,D_{\sigma} t) \exp\!\left( -n_{x}^{2} \frac{\pi^{2}}{W^{2}} D_{\sigma} t \right)
\end{equation}
from Eq. (\ref{coplanar:eqn:Dc}) in Theorem \ref{coplanar:teo:Dc}
and Eq. \eqref{coplanar:eqn:h} in Lemma \ref{coplanar:lem:g-h}.
Thus, the current density exhibits exponential modes that decay with time according to
\begin{equation}
	\label{idae:eqn:tau:intC}
	\exp\!\left( -\left[ n_{x}^{2} + (2\ell - 1)^{2} \frac{W^{2}}{4H^{2}} \right] \frac{\pi^{2}}{W^{2}} D_{\sigma} t \right),\quad 
	\begin{array}{rcl}
		n_{x} &=& \pm 1, \pm 2, \ldots \\
		\ell &=& 1, 2, \ldots
	\end{array}
\end{equation}
From all these exponential modes,
it is the slowest, that is with $n_{x} = \pm 1$ and $\ell=1$,
the one that gives an idea of the time required to reach steady state.
This time is roughly approached when $\pi^{2} D_{\sigma} t/W^{2} = 5 \cdot [1 + (\num{0,5} W/H)^{2}]^{-1}$,
that is when the slowest exponential mode approximately vanishes $\exp(-5) \approx \SI{0,67}{\percent}$.
At the right column of Fig. \ref{idae:fig:i_sim},
the times required by the simulated current
to reach \SI{0,67}{\percent} of its steady-state value are shown with `$\times$'
and correspond roughly to their theoretical counterparts: 4, \num{2,5} and \num{1,3}.

Finally, and independently of using internal or external coun\-ter electrode,
the time response of the current tends to speed up
as the height of the cell $H$ decreases.
This is justified by the shorter distances
that the electrochemical species must travel, due to lower roof of the cell.

\subsubsection{Collection efficiency in steady state}

For finite cell height $H$,
the steady-state current through a pair of electrode bands $A$ and $B$ is equal ($i_{f}^{A} = -i_{f}^{B}$).
This is confirmed by Eq. \eqref{coplanar:eqn:kirchhoff:final} with $p_{x} = 2W$ and Eq. \eqref{idae:eqn:if},
and it is shown in all plots of Fig. \ref{idae:fig:i_sim} after a sufficiently long time.
Therefore, 100\% collection efficiency must be obtained inside a unit cell,
independently of whether the counter electrode is internal or external.

But for cell heights approaching infinite $H \to +\infty$,
the collection efficiency is different for internal and external counter electrodes.
If the counter electrode is internal (one array performs as counter),
then the collection efficiency in the unit cell is automatically 100\%.
However, if the counter electrode is external,
then the collection efficiency is less than 100\%
when the average of the final concentration at the bottom of the unit cell
is forced to a different value than $\bar{c}_{\sigma,i}\unit$.

Collection efficiencies lower than 100\% in steady state,
for external counter electrode and very tall cells $H \to +\infty$,
can be explained by recalling the change in average concentration
at the bottom of a unit cell ($z=0$).
See Eqs. \eqref{coplanar:eqn:delta_bar_c} and \eqref{coplanar:eqn:delta_bar_cf}
% in the final steady state
\begin{equation}
	\label{idae:eqn:delta_bar_cf}
	\Delta\bar{c}_{\sigma}\unit(z, +\infty)
	= \Delta\bar{c}_{f}\unit
	= \frac{1}{H W L}
	\int_{0^{-}}^{+\infty}
		\frac{1}{F n_{e}}
		\underbrace{
			\int_{0}^{W} j(x,t)\, L \ud{x}
		}_{i^{\text{net}}(t) \text{ in unit cell}}
	\ud{t}
\end{equation}
Since fixing the average concentration at $z = 0$
to a value different than $\bar{c}_{\sigma,i}\unit$ means that 
$\Delta\bar{c}_{\sigma}\unit(0, +\infty) \neq 0$ is fixed to a finite value,
then the time integral $|\int_{0^{-}}^{+\infty} i^{\text{net}}(t) \ud{t}| \to +\infty$
is forced to diverge when $1/H \to 0^{+}$.
The infinite value of this integral is obtained when $i^{\text{net}}(+\infty) \neq 0$,
leading to a collection efficiency that is different from 100\% in steady state.

In this last case,
Corollary \ref{coplanar:cor:bar_cf} breaks due to $i^{\text{net}}(+\infty) \neq 0$,
producing a horizontal average of concentration, locally over the IDAE,
that is not uniform along the $z$-axis.
Therefore, a correction that takes into account
the effect of an external counter electrode (both arrays are potentiostated)
is needed to accurately predict the steady-state current through the IDAE.
This kind of correction was done the semi-empirically in \cite[Eq. (33)]{Aoki:1988:dec}
and later in \cite[Eqs. (13) and (20)]{Morf:2006:may}, both for the case of semi-infinite cells ($H \to +\infty$).


\subsection{Effect of net current on the average concentration}

The net current entering the unit cell
plays a determinant role on the horizontal average of concentration
for the entire unit cell at steady state.

As seen in the previous sections, a change of average concentration
at the bottom of the unit cell $\Delta\bar{c}_{\sigma}\unit(0,t)$
produces a non-zero net current due to Eq. \eqref{coplanar:eqn:i_net}.
Subsequently, this net current produces a change in horizontal average of concentration
at the entire unit cell $\Delta\bar{c}_{\sigma}\unit(z,t)$, due to Eq. \eqref{coplanar:eqn:delta_bar_c},
which reaches a steady state $\Delta\bar{c}_{f}\unit$
that is uniform $\forall z$ and independent of the electrochemical species $\sigma$, see Eq. \eqref{idae:eqn:delta_bar_cf}.

Therefore, the horizontal averages at the entire unit cell
for the final and initial steady sates are, in general, different
($\bar{c}_{\sigma,f}\unit \neq \bar{c}_{\sigma,i}\unit$) and this difference
($\bar{c}_{\sigma,f}\unit = \bar{c}_{\sigma,i}\unit \pm \Delta \bar{c}_{f}\unit$)
depends on the net current during the transition
from the initial towards the final state,
as seen in Eq. (\ref{idae:eqn:delta_bar_cf}).

If the net current is different from zero
during some finite time interval,
the currents through the generator and collector are different
and accumulation (or depletion) of species occurs inside the unit cell.
This generates the deviation of $\bar{c}_{\sigma,f}\unit$ with respect to $\bar{c}_{\sigma,i}\unit$.
Conversely, if the net current is zero for all $t$,
the currents at the generator and collector are equal
and no accumulation (or depletion) of species occurs.

In case of the simulation in Eq. \eqref{idae:eqn:pde:idae:normalized},
$\bar{c}_{\sigma,f}\unit$ must be given by 
the arithmetic average on both electrodes, since $2w_{A} = 2w_{B}$
\begin{equation}
	\frac{1}{W} \int_{0}^{W} c_{\sigma,f}(x,z) \ud{x}
	= \underbrace{\frac{c_{\sigma,f}^{A} + c_{\sigma,f}^{B}}{2} }_{ \bar{c}_{\sigma,f}\unit }
	\Leftrightarrow
	\frac{1}{W} \int_{0}^{W} \xi\simu(x,z,+\infty) \ud{x} = \frac{1}{2}
\end{equation}
The expression
$\bar{c}_{\sigma,f}\unit = \bar{c}_{\sigma,i}\unit \pm \Delta \bar{c}_{f}\unit$
has its simulated counterpart given by
\begin{equation}
	\bar{c}_{\sigma,f}\unit
	= \bar{c}_{\sigma,i}\unit \pm \Delta \bar{c}_{f}\unit
	\Leftrightarrow
	\frac{1}{2} 
	= \xi\simu(x,z,0^{-})
	\pm \frac{\Delta\bar{c}_{f}\unit}{[c_{\sigma,f}^{B} - c_{\sigma,f}^{A}]}
\end{equation}
meaning that the final average for the simulation must be $1/2$
independently of the initial concentration $\xi\simu(x,z,0^{-})$.
Also the change in horizontal average from Eq. \eqref{idae:eqn:delta_bar_cf}
was normalized to obtain
\begin{equation}
	\label{idae:eqn:delta_bar_cf_sim}
	\frac{\Delta\bar{c}_{f}\unit}{[c_{\sigma,f}^{B} - c_{\sigma,f}^{A}]}
	= \pm \frac{1}{\pi^{2}} \frac{W}{H}
	\int_{0^{-}}^{+\infty}
		\underbrace{
			\int_{0}^{1}
				-\parderiv{\xi\simu}{z\simu}(x,0,t)
			\ud{x\simu}
		}_{
			i\simu^{\text{net}}(t) 
			= i\simu^{A/2}(t) + i\simu^{B/2}(t)
		}
	\ud{t\simu}
\end{equation}
All normalizations were obtained by applying Eq. \eqref{idae:eqn:pde:normalization}

Fig \ref{idae:fig:i_sim} shows that, when $\xi\simu(x,z,0^{-}) = \num{0,25}$,
the horizontal average reaches $1/2$ in the final state, due to $i\simu^{\text{net}}(t) \geq 0$.
In this case, the change in horizontal average was obtained numerically
by computing the time integral of $i\simu^{\text{net}}(t)$ in Eq. \eqref{idae:eqn:delta_bar_cf_sim},
and approaches its theoretical value
$\pm \Delta\bar{c}_{f}\unit/[c_{\sigma,f}^{B} - c_{\sigma,f}^{A}] = \num{0,25}$
up to two decimal places for all simulated aspect ratios $H/W$
(see also Supplementary Information \S\ref{data:s32:simulation} for full numerical values).
On the other hand, when $\xi\simu(x,z,0^{-}) = \num{0,5}$,
the simulated net current $i\simu^{\text{net}}(t)$ equals zero for all $t$.
This produces no change in horizontal average
$\Delta\bar{c}_{f}\unit/[c_{\sigma,f}^{B} - c_{\sigma,f}^{A}] = 0$,
such that it can be maintained at $1/2$  until the final steady state ($\bar{c}_{\sigma,f}\unit = \bar{c}_{\sigma,i}\unit$).

Despite of not being mentioned explicitly in the literature,
earlier results showing change in average (bulk) concentration, due to non-zero net currents,
can also be found in the simulations of \cite[Fig. 5 and Eq. (4) with boundary conditions for coplanar electrodes]{Strutwolf:2005:feb}.


\subsection{Constraints on the limiting current}

\begin{figure}[t]
	\centering
	\subcaptionbox{
		\label{idae:fig:cell:unit}
		IDAE unit cell.}{\includegraphics{fig-4a}}
	\subcaptionbox{
		\label{idae:fig:cell:parallel}
		Transformed unit cell.}[15em]{\includegraphics{fig-4b}}  % [15em]
	\caption{
		Complex transformation $\bm{\rho} = T(\bm{r})$ of the unit cell
		from IDAE domain $\bm{r} = (x,z)$
		to parallel plates domain $\bm{\rho} = (\xi, \zeta)$.
		The boundary points $\bm{r}_{p}$ are bijectively mapped to $\bm{\rho}_{p}$,
		where $p \in \{a, \alpha, \beta, b, m, n\}$.
		This transformation can be obtained following a similar process to the one stated in \cite{Aoki:1988:dec},
		and its conformality is ensured by the conformality of Möbius functions \cite[\S5.7]{Ablowitz:2003:apr}
		and by the conformality of \emph{Schwarz-Christoffel} transformations \cite[Theorem 5.6.1]{Ablowitz:2003:apr}.
	}
	\label{idae:fig:cell:transformation}
\end{figure}

Before presenting the results on constraints for the limiting current,
it is convenient to show that, under certain conditions,
the current in steady state through the IDAE is proportional to the difference of concentration on both arrays.

\begin{lema}
	\label{idae:lem:if}
	Consider an IDAE electrochemical cell under the assumptions of \S\ref{idae:promedios}.
	If the final concentration of species $\sigma \in \{O,R\}$ at each electrode band $E \in \{A,B\}$ is uniform and equal to $c_{\sigma,f}^{E}$,
	then the current in the final steady state $i_{f}^{E}$
	through a band $E \in \{A,B\}$ is proportional to the difference $[c_{\sigma,f}^{E} - c_{\sigma,f}^{E'}]$
	\begin{equation}
		\label{idae:eqn:if}
		\pm \frac{
			i_{f}^{A}/L
		}{
			F n_{e} D_{\sigma} [c_{\sigma,f}^{A} - c_{\sigma,f}^{B}]
		}
		= \pm \frac{
			i_{f}^{B}/L
		}{
			F n_{e} D_{\sigma} [c_{\sigma,f}^{B} - c_{\sigma,f}^{A}]
		}
		= 2 \zeta(0,0)
	\end{equation}
	where $E'$ is the complementary band of $E$,
	and $\zeta(x,z)$ corresponds to the imaginary part of the conformal transformation $(\xi,\zeta) = T(x,z)$ shown in Fig.\ref{idae:fig:cell:transformation}.
\end{lema}

Note that this result is valid in steady state, that is,
after a sufficiently long time,
comparable with the time constant of the unit cell.
See Eqs. \eqref{idae:eqn:tau:extC} and \eqref{idae:eqn:tau:intC}
for external and internal counter electrode respectively.

\begin{proof}
	Consider the parallel-plates cell in Fig. \ref{idae:fig:cell:parallel}.
	It is known that the concentration profile $\gamma_{\sigma,f}(\xi,\zeta)$
	of species $\sigma \in \{O,R\}$ in the final steady state
	is given by a linear interpolation of the concentration at its electrodes
	\begin{subequations}
		\begin{align}
		\gamma_{\sigma,f}(\xi,\zeta)
		&= c_{\sigma,f}^{A} + [c_{\sigma,f}^{B} - c_{\sigma,f}^{A}] \xi
		\\
		\intertext{
			By returning to the IDAE domain
			$c_{\sigma,f}(x,z) = \gamma_{\sigma,f}(\xi,\zeta)$
			through the domain transformation $(\xi,\zeta) = T(x,z)$
			one obtains
		}
		c_{\sigma,f}(x,z)
		&= c_{\sigma,f}^{A} + [c_{\sigma,f}^{B} - c_{\sigma,f}^{A}]\xi(x,z)
		\end{align}
	\end{subequations}
	where $\xi(x,z)$ corresponds to the real part of the conformal transformation $(\xi,\zeta) = T(x,z)$.
	
	The current in final steady state $i_{f}^{E}$ can be obtained
	by integrating the flux through one electrode band $E \in \{A,B\}$
	\begin{equation}
		i_{f}^{E}
		= \mp \int_{E} F n_{e}\, D_{\sigma} \parderiv{c_{\sigma,f}}{z}(x,0) L \ud{x}
		= \mp \int_{E} F n_{e}\, D_{\sigma} [c_{\sigma,f}^{B} - c_{\sigma,f}^{A}] \parderiv{\xi}{z}(x,0) L \ud{x}
	\end{equation}
	Using the Cauchy-Riemann identities \cite[Theorem 3.2]{Olver:2017:} for $\bm{\rho} = T(\bm{r})$
	\begin{equation}
		\parderiv{\xi}{z} = -\parderiv{\zeta}{x} = -\Im \parderiv{\bm{\rho}}{x}
	\end{equation}
	the current can be further simplified
	\begin{equation}
		\label{idae:eqn:int_dxi7dz_dx}
		\mp \frac{
			i_{f}^{E}/L
		}{
			F n_{e} D_{\sigma} [c_{\sigma,f}^{B} - c_{\sigma,f}^{A}]
		}
		= \int_{E} \parderiv{\xi}{z}(x,0) \ud{x}
		= -\Im \int_{E} \parderiv{\bm{\rho}}{x}(x,0) \ud{x}
	\end{equation}
	Due to symmetry, this integral can be taken in half electrode band
	\begin{subequations}
		\begin{align}
			-\Im \int_{A} \partial\bm{\rho}(\bm{r})
			&= -2\,\Im\bm{\rho}(\bm{r})\Big|_{\bm{r}_{a}}^{\bm{r}_{\alpha}}
			= +2\, \Im \bm{\rho}(\bm{r}_{a})
			= +2\, \zeta(0,0)
			\\
			-\Im \int_{B} \partial\bm{\rho}(\bm{r})
			&= -2\,\Im\bm{\rho}(\bm{r})\Big|_{\bm{r}_{\beta}}^{\bm{r}_{b}}
			= -2\, \Im \bm{\rho}(\bm{r}_{b})
			= -2\, \zeta(0,0)
		\end{align}
	\end{subequations}
	where the imaginary parts $\Im\bm{\rho}_{\beta} = \Im\bm{\rho}_{\alpha} = 0$
	and $\Im\bm{\rho}_{b} = \Im\bm{\rho}_{a} = \zeta(0,0)$
	lead to the result in Eq. (\ref{idae:eqn:if}).
\end{proof}

Once it is clear that the current $i_{f}^{E}$ is proportional to
the difference of concentration between both arrays $[c_{\sigma,f}^{E} - c_{\sigma,f}^{E'}]$,
then it can be shown that non-negative concentrations,
together with the properties of \emph{horizontal average} and \emph{weighted sum} of concentrations,
restrict the maximum current that the cell can produce by directly limiting the difference $[c_{\sigma,f}^{E} - c_{\sigma,f}^{E'}]$.

\begin{teorema}
	\label{idae:teo:cE-cE':intC}
	Consider an IDAE electrochemical cell under the assumptions of \S\ref{idae:promedios}.
	Assume also that the array of bands $E \in \{A,B\}$ and its complementary array of bands $E'$
	perform as working and counter electrodes respectively,
	see Fig. \ref{idae:fig:cell:intC}.

	If the concentrations of species $\sigma \in \{O,R\}$ on both
	arrays are uniform and equal to $c_{\sigma}^{E}(t)$ and $c_{\sigma}^{E'}(t)$,
	and the bands have equal width $2w_{A} = 2w_{B}$, 
	then the concentrations on the working and counter electrodes
	and their difference are related for all $t$
	\begin{equation}
		\label{idae:eqn:cE':intC}
		2[c_{\sigma}^{E}(t) - \bar{c}_{\sigma,i}\whole]
		= -2[c_{\sigma}^{E'}(t) - \bar{c}_{\sigma,i}\whole]
		= [c_{\sigma}^{E}(t) - c_{\sigma}^{E'}(t)]
		% \quad \forall t \geq 0
	\end{equation}
	where $\bar{c}_{\sigma,i}\whole=\bar{c}_{\sigma,i}\unit$
	are the horizontal averages in the initial steady state,
	defined in Eqs. \eqref{idae:eqn:bar_ci:whole} and \eqref{idae:eqn:bar_ci:unit}.
	
	Moreover, the difference of concentrations in steady state
	is limited from above and below by
	\begin{equation}
		\label{idae:eqn:cotas:intC}
		-2 D_{\lambda} \bar{c}_{\lambda,i}\whole
		\leq D_{\sigma} [c_{\sigma,f}^{E} - c_{\sigma,f}^{E'}]
		\leq 2 D_{\lambda} \bar{c}_{\lambda,i}\whole
	\end{equation}
	where the determinant species $\lambda \in \{O,R\}$ is such that
	$D_{\lambda} \bar{c}_{\lambda,i}\whole = \min(D_{O} \bar{c}_{O,i}\whole, D_{R} \bar{c}_{R,i}\whole)$,
	$c_{\sigma,f}^{E} = c_{\sigma}^{E}(+\infty)$,
	and $c_{\sigma,f}^{E'} = c_{\sigma}^{E'}(+\infty)$.
	This last expression determines the limiting current of the cell.
\end{teorema}

It is important to note that Eq. \eqref{idae:eqn:cotas:intC}
is valid in steady state, that is, after a sufficiently long time,
comparable with the time constant of the unit cell,
see Eq. \eqref{idae:eqn:tau:intC} for internal counter electrode.
However, this expression can hold also $\forall t$
under the additional restriction $D_{O} = D_{R}$
(when applying Corollary \ref{coplanar:cor:ct:total}
instead of Corollary \ref{coplanar:cor:cf:total}
in Eqs. \eqref{idae:eqn:cf:unit:total}).

Note also that Eq. (\ref{idae:eqn:cotas:intC}) imply the necessity of
the simultaneous presence of $\bar{c}_{O,i}\whole$ and $\bar{c}_{R,i}\whole$
in order to achieve steady state currents.
This was also noted by \cite[before and after Eq. (17)]{Morf:2006:may}
and explained in \cite[\S2.3 and Fig. 2]{GuajardoYevenes:2013:sep},
both for the case when $D_{O} = D_{R}$.

\begin{proof}
	Consider the properties of horizontal averages in Eqs. \eqref{idae:eqn:bar_c:unit}.
	Since the counter electrode is internal to the IDAE,
	then the net current in the unit cell is zero for all $t$,
	therefore $\Delta \bar{c}_{\sigma,f}\unit(z,t) = \Delta \bar{c}_{f}\unit = 0$.
	Also since $\bar{c}_{\sigma,i}\unit = \bar{c}_{\sigma,i}\whole$ for Fig. \ref{idae:fig:cell:intC},
	then the average properties can be summarized for all $t$ as
	\begin{subequations}
		\begin{align}
			\int_{0}^{W} c_{\sigma}(x,z,t) - \bar{c}_{\sigma,i}\whole \ud{x} &= 0
			\\
			\intertext{%
				Due to bands of equal widths $2w_{A} = 2w_{B}$,
				the integral in the gap between consecutive bands
				$\int_{w_{A}}^{W - w_{B}} c_{\sigma}(x,z,t) - \bar{c}_{\sigma,i}\whole \ud{x}$
				 equals zero for all $t$.
				Therefore, the average property is reduced to
			}
			\label{idae:eqn:average:intC}
			[c_{\sigma}^{E}(t) - \bar{c}_{\sigma,i}\whole]
				+ [c_{\sigma}^{E'}(t) - \bar{c}_{\sigma,i}\whole] &= 0
		\end{align}
	\end{subequations}
	where $E'$ is the complementary band of $E \in \{A,B\}$.
	This agrees with \cite[Eq. (15)]{Morf:2006:may}.
	Finally, the difference of concentration between electrodes
	can be reduced to Eq. (\ref{idae:eqn:cE':intC}) since
	\begin{equation}
		c_{\sigma}^{E}(t) - c_{\sigma}^{E'}(t)
		= [c_{\sigma}^{E}(t) - \bar{c}_{\sigma,i}\whole]
		- [c_{\sigma}^{E'}(t) - \bar{c}_{\sigma,i}\whole]
	\end{equation}
	
	To obtain the limits for the difference of concentration in steady state,
	one considers Eq. \eqref{idae:eqn:average:intC} with non-negative concentrations on all electrodes
	\begin{subequations}
		\begin{align}
			-\bar{c}_{\sigma,i}\whole \leq [c_{\sigma,f}^{E} - \bar{c}_{\sigma,i}\whole]
			&= -[c_{\sigma,f}^{E'} - \bar{c}_{\sigma,i}\whole] \leq \bar{c}_{\sigma,i}\whole
			\\ {}
			-\bar{c}_{\sigma',i}\whole \leq [c_{\sigma',f}^{E} - \bar{c}_{\sigma',i}\whole]
			&= -[c_{\sigma',f}^{E'} - \bar{c}_{\sigma',i}\whole] \leq \bar{c}_{\sigma',i}\whole
		\end{align}
	\end{subequations}
	where $\sigma'$ is the complementary species of $\sigma \in \{O,R\}$.
	The limits for the concentration of species $\sigma'$
	may also affect the limits for the concentration of species $\sigma$.
	This can be seen by applying Corollary \ref{coplanar:cor:cf:total}
	(with $p_{x} = 2W$, $\Delta\bar{c}_{f}\unit = 0$
	and $\bar{c}_{\sigma,i}\unit = \bar{c}_{\sigma,i}\whole$)
	at the bands $E$ and $E'$
	\begin{subequations}
		\label{idae:eqn:cf:unit:total}
		\begin{align}
			D_{\sigma} [c_{\sigma,f}^{E} - \bar{c}_{\sigma,i}\whole]
			+ D_{\sigma'} [c_{\sigma',f}^{E} - \bar{c}_{\sigma',i}\whole] &= 0
			\\
			D_{\sigma} [c_{\sigma,f}^{E'} - \bar{c}_{\sigma,i}\whole]
			+ D_{\sigma'} [c_{\sigma',f}^{E'} - \bar{c}_{\sigma',i}\whole] &= 0
		\end{align}
	\end{subequations}
	and agrees with \cite[Eqs. (15) and (16)]{Morf:2006:may} when $D_{O} = D_{R}$.
	Combining the last two expressions leads to
	\begin{subequations}
		\begin{align}
			-D_{\sigma} \bar{c}_{\sigma,i}\whole
			\leq D_{\sigma} [c_{\sigma,f}^{E} - \bar{c}_{\sigma,i}\whole]
			&= -D_{\sigma} [c_{\sigma,f}^{E'} - \bar{c}_{\sigma,i}\whole]
			\leq D_{\sigma} \bar{c}_{\sigma,i}\whole
			\\ {}
			-D_{\sigma'} \bar{c}_{\sigma',i}\whole
			\leq D_{\sigma} [c_{\sigma,f}^{E} - \bar{c}_{\sigma,i}\whole]
			&= -D_{\sigma} [c_{\sigma,f}^{E'} - \bar{c}_{\sigma,i}\whole]
			\leq D_{\sigma'} \bar{c}_{\sigma',i}\whole
		\end{align}
	\end{subequations}
	Let $D_{\lambda} \bar{c}_{\lambda,i}\whole = \min(D_{O} \bar{c}_{O,i}\whole, D_{R} \bar{c}_{R,i}\whole)$,
	then the previous expressions can be summarized as
	\begin{equation}
		-D_{\lambda} \bar{c}_{\lambda,i}\whole
		\leq D_{\sigma} [c_{\sigma,f}^{E} - \bar{c}_{\sigma,i}\whole]
		= -D_{\sigma} [c_{\sigma,f}^{E'} - \bar{c}_{\sigma,i}\whole]
		\leq D_{\lambda} \bar{c}_{\lambda,i}\whole
	\end{equation}
	which leads to Eq. (\ref{idae:eqn:cotas:intC}).
\end{proof}

\begin{teorema}
	\label{idae:teo:cE-cE':extC}
	Consider an IDAE electrochemical cell under the assumptions of \S\ref{idae:promedios},
	but now with an external counter electrode as in Fig. \ref{idae:fig:cell:extC}.
	Assume also that the array of bands $E \in \{A,B\}$ is freely potentiostated,
	and its complementary array of bands $E'$ is also potentiostated,
	but fixed to a very extreme potential,
	such that the concentration of $\sigma \in \{O,R\}$ on its surface is $c_{\sigma}^{E'}(t) = 0$.
	
	If the IDAE has bands of equal width $2w_{A} = 2w_{B}$,
	the width of the counter electrode equals the width of the IDAE $w_{C} = 2W N_{E}$,
	and the integral $\int_{2W N_{E}}^{W_{T}-w_{C}} c_{\sigma}(x,0,t) - \bar{c}_{\sigma,i}\whole \ud{x} \approx 0$
	in the gap between the IDAE and the counter electrode,
	then the concentrations on the bands $c_{\sigma}^{E}(t)$
	and on the counter electrode $c_{\sigma}^{C}(t)$ are related for all $t$ by
	\begin{equation}
		\label{idae:eqn:cC:extC}
		\left[ \frac{c_{\sigma}^{E}(t)}{2} - \bar{c}_{\sigma,i}\whole \right]
		\approx -[c_{\sigma}^{C}(t) - \bar{c}_{\sigma,i}\whole]
	\end{equation}
	where $\bar{c}_{\sigma,i}\whole$ is the horizontal average
	in the initial steady state, defined in Eq. \eqref{idae:eqn:bar_ci:whole}.
	
	Moreover, when $\sigma = \lambda$, the difference of final concentrations
	is limited from above and below by
	\begin{subequations}
		\label{idae:eqn:cotas:extC}
		\begin{equation}
			0 
			\leq [c_{\lambda,f}^{E} - c_{\lambda,f}^{E'}] = c_{\lambda,f}^{E}
			\lesssim 4 \bar{c}_{\lambda,i}\whole
		\end{equation}
		and a similar situation occurs to its counter electrode,
		of which its concentration is limited from above and below by
		\begin{equation}
			0
			\leq c_{\lambda,f}^{C} 
			\lesssim 2 \bar{c}_{\lambda,i}\whole
		\end{equation}
	\end{subequations}
	where the determinant species $\lambda \in \{O,R\}$ is such that
	$D_{\lambda} \bar{c}_{\lambda,i}\whole = \min(D_{O} \bar{c}_{O,i}\whole, D_{R} \bar{c}_{R,i}\whole)$,
	$\lambda'$ is its complementary species,
	$c_{\lambda,f}^{E} = c_{\lambda}^{E}(+\infty)$,
	$c_{\lambda,f}^{E'} = c_{\lambda}^{E'}(+\infty)$
	and $c_{\lambda,f}^{C} = c_{\lambda}^{C}(+\infty)$.
	These last two expressions determine the limiting current of the cell
	when the electrochemical species satisfy
	\begin{equation}
		\label{idae:eqn:lambda':extC}
		D_{\lambda'} \bar{c}_{\lambda',i}\whole
		\geq 3 D_{\lambda} \bar{c}_{\lambda,i}\whole
	\end{equation}
\end{teorema}

Notice that Eqs. \eqref{idae:eqn:cotas:extC} are valid in steady state,
after a long time, comparable with the time constant of the whole cell.
See Eq. \eqref{idae:eqn:tau:intC} but replacing $W$ by $W_{T}$,
since the counter electrode is \emph{internal} to the whole cell.
Thus, for shallow cells $H \ll W_{T}$,
the dominant exponential mode is reduced to Eq. \eqref{idae:eqn:tau:extC}.
This agrees with the dominant mode of the unit cell with external counter electrode,
and only depends on the restricted height $H$.
Conversely, for very tall cells $H \gg W_{T}$,
Eq. \eqref{idae:eqn:tau:intC} depends only on the width of the whole cell $W_{T}$,
which produces much slower time responses.

Eqs. \eqref{idae:eqn:cotas:extC} can hold also $\forall t$
under the additional restriction $D_{O} = D_{R}$
(when applying Corollary \ref{coplanar:cor:ct:total}
instead of Corollary \ref{coplanar:cor:cf:total}
in Eqs. \eqref{idae:eqn:cf:whole:total}).

Finally, note that Eqs. (\ref{idae:eqn:cotas:extC}) imply the necessity of
the simultaneous presence of $\bar{c}_{O,i}\whole$ and $\bar{c}_{R,i}\whole$
in order to achieve steady state currents.

\begin{proof}
	Due to  Eqs. \eqref{idae:eqn:bar_c:whole},
	the average concentration at the bottom of the whole cell
	is given for all $t$ by
	\begin{subequations}
		\begin{align}
			\int_{0}^{W_{T}} c_{\sigma}(x,0,t) - \bar{c}_{\sigma,i}\whole \ud{x} &= 0
			\\
			\intertext{This integral can be splitted in three parts}
			\int_{0}^{2W N_{E}} +
			\underbrace{\int_{2W N_{E}}^{W_{T} - w_{C}}}_{\text{assumed}\approx 0} +
			\int_{W_{T} - w_{C}}^{W_{T}} &= 0
		\end{align}
		and when $2w_{A} = 2w_{B}$
		and the number of bands $N_{E}$ is large enough,
		it can be reduced to
		\begin{equation}
			\label{idae:eqn:pre-cC:extC}
			2W N_{E} \left[
				\frac{c_{\sigma}^{E}(t) + c_{\sigma}^{E'}(t)}{2} - \bar{c}_{\sigma,i}\whole
			\right]
			+ w_{C} [c_{\sigma}^{C}(t) - \bar{c}_{\sigma,i}\whole] \approx 0
		\end{equation}
	\end{subequations}
	since the average concentration on the IDAE satisfies
	\begin{equation}
		\frac{1}{2W N_{E}} \int_{0}^{2W N_{E}} c_{\sigma}(x,0,t) \ud{x}
		\approx \frac{c_{\sigma}^{E}(t) + c_{\sigma}^{E'}(t)}{2}
	\end{equation}
	Therefore, Eq. (\ref{idae:eqn:cC:extC}) is obtained from Eq. (\ref{idae:eqn:pre-cC:extC}),
	when $c_{\sigma}^{E'}(t) = 0$ and $2W N_{E} = w_{C}$.
	
	To obtain the limits for the difference of concentration in steady state,
	one considers Eq. \eqref{idae:eqn:cC:extC} with non-negative concentrations on all electrodes
	\begin{subequations}
		\begin{align}
		-\bar{c}_{\sigma,i}\whole
		\leq [c_{\sigma,f}^{E}/2 - \bar{c}_{\sigma,i}\whole]
		&\approx -[c_{\sigma,f}^{C} - \bar{c}_{\sigma,i}\whole]
		\leq \bar{c}_{\sigma,i}\whole
		\\
		-\bar{c}_{\sigma',i}\whole
		\leq [c_{\sigma',f}^{E}/2 - \bar{c}_{\sigma',i}\whole]
		&\approx -[c_{\sigma',f}^{C} - \bar{c}_{\sigma',i}\whole]
		\leq \bar{c}_{\sigma',i}\whole
		\end{align}
	\end{subequations}
	at the final steady state. Note that the limits for the concentration of species $\sigma'$ may also affect
	the limits for the concentration of species $\sigma$.
	This can be seen by applying the relations
	\begin{subequations}
		\label{idae:eqn:cf:whole:total}
		\begin{align}
			D_{\sigma} [c_{\sigma,f}^{E}/2 - \bar{c}_{\sigma,i}\whole]
			+ D_{\sigma'} [c_{\sigma',f}^{E}/2 - \bar{c}_{\sigma',i}\whole] &= 0
			\\
			D_{\sigma} [c_{\sigma,f}^{C} - \bar{c}_{\sigma,i}\whole]
			+ D_{\sigma'} [c_{\sigma',f}^{C} - \bar{c}_{\sigma',i}\whole] &= 0
		\end{align}
	\end{subequations}
	which are obtained from Corollary \ref{coplanar:cor:cf:total}
	(with $p_{x} = 2W_{T}$ and $\Delta\bar{c}_{f}\whole = 0$)
	at the middle of the gap between consecutive bands and on the surface of the counter electrode $C$.
	Then this leads to
	\begin{subequations}
		\begin{align}
		-D_{\sigma} \bar{c}_{\sigma,i}\whole
		\leq D_{\sigma} [c_{\sigma,f}^{E}/2 - \bar{c}_{\sigma,i}\whole]
		&\approx -D_{\sigma} [c_{\sigma,f}^{C} - \bar{c}_{\sigma,i}\whole]
		\leq D_{\sigma} \bar{c}_{\sigma,i}\whole
		\\
		-D_{\sigma'} \bar{c}_{\sigma',i}\whole
		\leq D_{\sigma} [c_{\sigma,f}^{E}/2 - \bar{c}_{\sigma,i}\whole]
		&\approx -D_{\sigma} [c_{\sigma,f}^{C} - \bar{c}_{\sigma,i}\whole]
		\leq D_{\sigma'} \bar{c}_{\sigma',i}\whole
		\end{align}
	\end{subequations}
	Let $D_{\lambda} \bar{c}_{\lambda,i}\whole = \min(D_{O} \bar{c}_{O,i}\whole, D_{R} \bar{c}_{R,i}\whole)$,
	then the previous expressions can be rewritten as
	\begin{equation}
		\label{idae:eqn:cotas:compacto}
		-D_{\lambda} \bar{c}_{\lambda,i}\whole
		\leq D_{\sigma} [c_{\sigma,f}^{E}/2 - \bar{c}_{\sigma,i}\whole]
		\approx -D_{\sigma} [c_{\sigma,f}^{C} - \bar{c}_{\sigma,i}\whole]
		\leq D_{\lambda} \bar{c}_{\lambda,i}\whole
	\end{equation}
	which finally leads to Eqs. (\ref{idae:eqn:cotas:extC}) when $\sigma = \lambda$.
	
	The reason to restrict Eqs. (\ref{idae:eqn:cotas:extC}) only to the determinant species $\lambda$
	is because only the species with least $D_{\sigma} \bar{c}_{\sigma,i}\whole$ 
	should be fixed to zero concentration $c_{\lambda,f}^{E'} = 0$ at the bands $E'$.
	Otherwise, non-negative concentrations are found in the equations, which are not possible physically.

	This can be seen by checking the minimum and maximum values for $D_{\sigma'} c_{\sigma',f}^{E}$.
	First, take the weighted sum of concentrations in Corollary \ref{coplanar:cor:cf:total}
	($p_{x} = 2W_{T}$ and $\Delta\bar{c}_{f}\whole = 0$) at the bands $E$
	\begin{subequations}
		\begin{equation}
			D_{\sigma} c_{\sigma,f}^{E} + D_{\sigma'} c_{\sigma',f}^{E}
			= D_{\lambda} \bar{c}_{\lambda,i}\whole
			+ D_{\lambda'} \bar{c}_{\lambda',i}\whole
		\end{equation}
		where $\lambda'$ is the complementary species of $\lambda$.
		At the same time that $D_{\sigma} c_{\sigma,f}^{E}$ reaches its extrema, given by Eq. \eqref{idae:eqn:cotas:compacto}
		\begin{equation}
			2 D_{\sigma} \bar{c}_{\sigma,i}\whole - 2 D_{\lambda} \bar{c}_{\lambda,i}\whole
			\leq D_{\sigma} c_{\sigma,f}^{E} \lesssim
			2 D_{\sigma} \bar{c}_{\sigma,i}\whole + 2 D_{\lambda} \bar{c}_{\lambda,i}\whole
		\end{equation}
		$D_{\sigma'} c_{\sigma',f}^{E}$ also reaches its extrema, which are given by
		\begin{equation}
			D_{\lambda'} \bar{c}_{\lambda',i}\whole
			- D_{\lambda} \bar{c}_{\lambda,i}\whole
			-2 D_{\sigma} \bar{c}_{\sigma,i}\whole
			% \\
			\lesssim D_{\sigma'} c_{\sigma',f}^{E}
			% \\
			\leq
			3 D_{\lambda} \bar{c}_{\lambda,i}\whole
			+ D_{\lambda'} \bar{c}_{\lambda',i}\whole
			- 2 D_{\sigma} \bar{c}_{\sigma,i}\whole
		\end{equation}
	\end{subequations}
	Since the minimun concentration should be non-negative,
	this leads to the fact that $\sigma$ must be only $\lambda$ (and not $\lambda'$),
	and also to the additional restriction in Eq. (\ref{idae:eqn:lambda':extC}).
\end{proof}

Theorems \ref{idae:teo:cE-cE':intC} and \ref{idae:teo:cE-cE':extC}
show the usefulness of the properties of horizontal averages in Corollaries
\ref{coplanar:cor:bar_ci}, \ref{coplanar:cor:delta_bar_c} and \ref{coplanar:cor:bar_cf},
since they provide a tool for determining the concentration on the counter electrode,
which is unknown \emph{a priori}, since it is controlled by the potentiostat.
This is useful for determining boundary conditions for the counter electrode
and it can be used in simulations that require its inclussion.

Interesting is the fact that having an internal (Theorem \ref{idae:teo:cE-cE':intC})
or external (Theorem \ref{idae:teo:cE-cE':extC}) counter electrode
produces bipolar or unipolar limiting currents respectively.

Also, under the conditions that were just analyzed: 
band electrodes of equal width $2w_{A} = 2w_{B}$ 
and external counter electrode of width equal to that of the IDAE $w_{C} = 2W N_{E}$,
it appears to be that there is no significant advantage
of having an external counter electrode (both electrode arrays potentiostated)
versus using one of the arrays as counter electrode (only one array potentiostated),
at least in terms of the current range.
According to Lemma \ref{idae:lem:if} and Theorem \ref{idae:teo:cE-cE':extC} for external counter electrode,
the current is $i_{f}^{E} \propto c_{\lambda,f}^{E} - c_{\lambda,f}^{E'}$
and the difference of concentration ranges from $0$ to $+4\bar{c}_{\lambda,i}\whole$.
On the other hand, according to Lemma \ref{idae:lem:if} and Theorem \ref{idae:teo:cE-cE':intC} for internal counter electrode,
the current is still $i_{f}^{E} \propto c_{\lambda,f}^{E} - c_{\lambda,f}^{E'}$
but the difference of concentration ranges from $-2\bar{c}_{\lambda,i}\whole$ to $+2\bar{c}_{\lambda,i}\whole$, that is, it also spans a range of $4\bar{c}_{\lambda,i}\whole$.

Nevertheless, taking a careful look to the average at the bottom of the whole cell in Eq. \eqref{idae:eqn:pre-cC:extC} % \eqref{idae:eqn:bar_cf:whole}
%\begin{equation}
%	\int_{0}^{W_{T}} c_{\lambda,f}(x,0) - \bar{c}_{\lambda,i}\whole \ud{x} = 0
%	\Rightarrow
%	2W N_{E} \left[ 
%		\frac{c_{\lambda,f}^{E}}{2} - \bar{c}_{\lambda,i}\whole
%	\right]
%	+ w_{C} [c_{\lambda,f}^{C} - \bar{c}_{\lambda,i}\whole]
%	\approx 0
%\end{equation} 
suggests that with an external counter electrode wider than the IDAE $w_{C} \geq 2W N_{E}$,
the current could span larger ranges.
This is because a slight decrease of the concentration on the counter electrode $c_{\lambda,f}^{C}$
below the average concentration $\bar{c}_{\lambda,i}\whole$
could cause a large increase of the concentration of the freely potentiostated array $c_{\lambda,f}^{E}$.
However, the analysis of the non-linearities
(related to the physical constraint of non-negative concentrations)
and the limits for the steady state current become more difficult.
