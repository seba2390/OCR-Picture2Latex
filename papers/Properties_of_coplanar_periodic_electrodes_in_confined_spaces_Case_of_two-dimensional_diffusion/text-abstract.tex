Periodic configurations of electrodes, in particular of microelectrodes,
have been of interest since the advent of microfabrication.
In this report, theory which is common to any periodic cell
(or any cell that can be extended periodically)
with finite height and two-dimensional symmentry was derived.
The diffusion equation in this cell was solved
and the concentration profile was obtained
in terms of its Fourier coefficients
and as a function of an arbitrary current density.
From this base result, a set of properties were derived
which are fairly general, since they don't assume restrictions
such as reversible electrode reactions
(Nernst equation valid when current circulates).
These properties involve:
horizontal averages and (weighted) sum of concentrations,
both with a close connection to the net current
and accumulation of species in the cell.
The derived properties allow:
to explain qualitative aspects of collection efficiency and limiting currents,
to predict the concentration on counter electrodes and non-linearities
caused by depletion of species at extremely polarized electrodes,
and to estimate the time required by the current to reach steady state
in potential controlled experiments.
The theoretical results are illustrated analytically
and numerically for the concrete case of interdigitated array of electrodes.
