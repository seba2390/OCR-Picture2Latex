\documentclass{article}

% if you need to pass options to natbib, use, e.g.:
%     \PassOptionsToPackage{numbers, compress}{natbib}
% before loading neurips_2021

% ready for submission
% \usepackage{neurips_2021}

% to compile a preprint version, e.g., for submission to arXiv, add add the
% [preprint] option:
    % \usepackage[preprint]{neurips_2021}

% to compile a camera-ready version, add the [final] option, e.g.:
    \usepackage[final]{neurips_2021}

% to avoid loading the natbib package, add option nonatbib:
%    \usepackage[nonatbib]{neurips_2021}

\usepackage[utf8]{inputenc} % allow utf-8 input
\usepackage[T1]{fontenc}    % use 8-bit T1 fonts
\usepackage{hyperref}       % hyperlinks
\usepackage{url}            % simple URL typesetting
\usepackage{booktabs}       % professional-quality tables
\usepackage{amsfonts}       % blackboard math symbols
\usepackage{nicefrac}       % compact symbols for 1/2, etc.
\usepackage{microtype}      % microtypography
\usepackage{paperpkg}

% \usepackage{xr}
% \externaldocument{sky_supp}


\usepackage[table,dvipsnames]{xcolor}
\renewcommand{\UrlFont}{\ttfamily\small}
\usepackage{url}


\renewcommand*{\ttdefault}{cmtt}
\usepackage{graphicx}               % Include graph
\usepackage{tabularx}               % Better table formatting
\newcolumntype{C}{>{\centering\arraybackslash}X}
\usepackage{multirow}               % Multi-row in tables
\usepackage{diagbox}                % Diagonal line in tables
\usepackage{hhline}                 % Draw double line in tables
\usepackage{color}                  % Text and background color
\usepackage{amsmath}                % Formula
\usepackage{amssymb}                % Formula
\usepackage{mathtools}              % Formula
\usepackage{enumitem}               % Better itemize environment
\usepackage{verbatim}               % comment env
\usepackage{multirow}
\newcommand{\bs}{\boldsymbol}
\newcommand{\ola}{\overleftarrow}
\newcommand{\ora}{\overrightarrow}
\newcommand{\ccgreen}{\cellcolor{Emerald!10}}
\newcommand{\ccdgreen}{\cellcolor{Emerald!20}}
\newcommand{\ccddgreen}{\cellcolor{Emerald!35}}
\newcommand{\x}{\checkmark}


\newtheorem{definition}{Definition} % definition in amsmath
\usepackage{ulem} % for delete line


\usepackage{xparse}
\NewDocumentCommand{\william}{ mO{} }{\textcolor{red}{\textsuperscript{\textit{William}}\textsf{\textbf{\small[#1]}}}}
\NewDocumentCommand{\vicki}{ mO{} }{\textcolor{blue}{\textsuperscript{\textit{Vicki}}\textsf{\textbf{\small[#1]}}}}
\NewDocumentCommand{\manling}{ mO{} }{\textcolor{Blue}{\textsuperscript{\textit{Manling}}\textsf{\textbf{\small[#1]}}}}
\NewDocumentCommand{\heng}{ mO{} }{\textcolor{cyan}{\textsuperscript{\textit{Heng}}\textsf{\textbf{\small[#1]}}}}

\NewDocumentCommand{\hh}{ mO{} }{\textcolor{green}{\textsuperscript{\textit{HH}}\textsf{\textbf{\small[#1]}}}}

\newcommand{\todo}[1]{\textcolor{red}{#1}}


\newcommand\topalign[1]{%
  \setbox0\hbox{#1}%
  \raisebox{\dimexpr-\ht0+\dp0\relax}{\usebox0}}




  
 

% \title{Sketched Attention for Fast Transformer}
% \title{Accelerate Attention with Nystr\"om Method\\ for Long Document}
% \title{Spectral Analysis and Fast Approximation of Self-attention for Long Documents}
\title{Skyformer: Remodel Self-Attention with Gaussian Kernel and Nystr\"om Method}

% The \author macro works with any number of authors. There are two commands
% used to separate the names and addresses of multiple authors: \And and \AND.
%
% Using \And between authors leaves it to LaTeX to determine where to break the
% lines. Using \AND forces a line break at that point. So, if LaTeX puts 3 of 4
% authors names on the first line, and the last on the second line, try using
% \AND instead of \And before the third author name.

\author{
Yifan Chen\thanks{\; Equal contribution. },~   
Qi Zeng\footnotemark[1],~ 
Heng Ji,~ 
Yun Yang \\ 
University of Illinois Urbana-Champaign \\
% \email{\texttt{\{yifanc10, qizeng2,  hengji, yy84\}@illinois.edu }}
\texttt{\{yifanc10, qizeng2,  hengji, yy84\}@illinois.edu }
}



\begin{document}

\maketitle


\begin{abstract}
With the increasing ability of large language models (LLMs), in-context learning (ICL) has become a new paradigm for natural language processing (NLP), where LLMs make predictions only based on contexts augmented with a few examples. It has been a new trend to explore ICL to evaluate and extrapolate the ability of LLMs.
In this paper, we aim to survey and summarize the progress and challenges of ICL. We first present a formal definition of ICL and clarify its correlation to related studies. Then, we organize and discuss advanced techniques, including training strategies, demonstration designing strategies, as well as related analysis.
Finally, we discuss the challenges of ICL and provide potential directions for further research. We hope that our work can encourage more research on uncovering how ICL works and improving ICL.
% \footnote{We sort out all the papers involved on \url{https://github.com/dqxiu/ICL_PaperList}.}

% and attempt to provide potential directions for further research
\end{abstract}


\section{Introduction}
\label{sec:intro}


The cost of language model training increases exponentially.
Among different models, Transformer-based language models ~\citep{DBLP:conf/nips/VaswaniSPUJGKP17, DBLP:journals/corr/abs-1810-04805, DBLP:journals/corr/abs-1907-11692, DBLP:conf/acl/LewisLGGMLSZ20} are shown to enjoy state-of-the-art (SOTA) performances on many Natural Language Processing (NLP) tasks despite their enormous training cost. %  that was unthinkable years ago
% For example, it may theoretically take around $355$ years to train GPT-3~\citep{DBLP:conf/nips/BrownMRSKDNSSAA20} on a Tesla V100.
One of the computation bottlenecks lies in the self-attention mechanism, which is known to be resource-intensive with quadratic time and space complexity ($O(n)$ where $n$ is the input sequence length).
Consequently, Transformers cannot support long sequence processing and large batch size with limited resources.

The challenge of improving computational efficiency of Transformers has motivated several recent studies on attention acceleration, using either sparse attention pattern \citep{DBLP:conf/emnlp/QiuMLYW020, DBLP:journals/corr/abs-1904-10509, DBLP:conf/nips/ZaheerGDAAOPRWY20, DBLP:journals/corr/abs-2004-05150, DBLP:conf/iclr/KitaevKL20} or low-rank approximation \citep{DBLP:journals/corr/abs-2009-14794, DBLP:journals/corr/abs-2006-04768}. 
However, there is usually a lack of theoretical analysis on the approximation error of these methods due to the complex softmax structure, which makes the theoretical comparison between the efficiency of each method infeasible.
It is also unclear in theory how to set the hyper-parameters of those methods to attain a desired level of approximation accuracy.


Another issue of Transformers is the training instability that small perturbations in parameter updates tend to be amplified, resulting in significant disturbances in the model output~\citep{DBLP:conf/emnlp/LiuLGCH20}.
Transformers on some NLP tasks have shown to be sensitive to hyper-parameters, learning schedulers, or even random seeds, which usually demands a time-costly grid search for the best configuration in real-world applications.
It has also been observed in our experiments that a slight change in the learning rate may cause the failure of convergence for some models.
We conjecture that the instability in Transformer training comes from the softmax structure,
as the un-normalized attention score matrices before softmax tend to have extremely large condition numbers due to its fast singular value decay.
% In addition to the high computational complexity, another potential issue of Transformer is its difficult training \citep{DBLP:conf/emnlp/LiuLGCH20}.
% -------------------------------------


To alleviate the instability issue, an extra factor of $1 / \sqrt{p}$ in the softmax kernel $\text{SM}$ is suggested by \citet{DBLP:conf/nips/VaswaniSPUJGKP17} to restrain the scale variation;
\citet{DBLP:conf/emnlp/LiuLGCH20} proposes a new scheme to control the magnitude of output change and stabilize the training in early stages.
In practice, we also need to consider the lower numerical precision of GPU implementation in model training, which further deteriorates the stability.
% This issue substantially influences the usage of \nystrom method, which requires to inverse an ill-conditioned matrix $\mtx{S}^T \mtx{A} \mtx{S}$.
% The large condition number of $\mtx{S}^T \mtx{A} \mtx{S}$ results in extremely large spectral norm of the pseudoinverse $(\mtx{S}^T \mtx{A} \mtx{S})^\dagger$,
% which causes large parameter updates and make the training process even more unstable.


Kernel methods may be the answer to both challenges.
As pointed out by \citet{DBLP:journals/corr/abs-2009-14794}, 
the softmax structure is closely related to Gaussian kernels up to diagonal matrix multiplications,
as the pairwise dot products naturally appear when expanding the squared $\ell_2$ distance.
We further notice some important connections between self-attention and Gaussian kernels.
First, the un-normalized attention score matrix can be formed via basic matrix operations on an empirical Gaussian kernel matrix.
Moreover, the form of Gaussian kernels has the natural interpretation of  assigning ``attention'' to different tokens.
Compared to the softmax function, Gaussian kernels automatically perform the normalization as softmax does (c.f. Section~\ref{sec:kernel_attn}).
These observations motivate us to replace the softmax structure with Gaussian kernels.
As we demonstrated in this paper, the new attention model, \textbf{Kernelized Attention}, empirically stabilizes the model training while being comparable to self-attention in model accuracy.
% and theoretically allows a more precise analysis considering the abundant studies on kernel methods. 

To further improve the efficiency, we propose \textbf{Skyformer} (\textbf{S}ymmetrization of \textbf{K}ernelized attention for N\textbf{Y}str\"om method) to accelerate kernelized attention.
Skyformer adapts the \nystrom method \citep{williams2001using, drineas2005nystrom} to the non-PSD empirical Gaussian kernel matrix
(as query matrices in general do not equal to key matrices), 
by instead lifting the kernelized attention score matrix into a large PSD matrix that contains the un-normalized attention score matrix as the off-diagonal block.
We further conduct theoretical analysis by showing that Skyformer has a small matrix approximation error on kernelized attention in the spectral norm.
Our experiments on the LRA benchmark show that Skyformer consistently uses less space and time while achieving better accuracy than other baseline methods.
% In addition, our method can be easily applied to some new variants of attention, such as Lambda Network~\citep{lambdanetwork}. 


In summary, our main contributions are:


(1) We revisit the intrinsic connection between self-attention and kernel methods, and explore a new kernel-based structure, kernelized attention, to stabilize the training of Transformers.


(2) We propose Skyformer, which approximates the kernelized attention via low dimensional randomized sketches by adapting the \nystrom method to a non-PSD matrix. 
We provide the theoretical guarantee that the matrix multiplication error is small in term of spectral norm. 


(3) Extensive experiments show that Skyformer achieves comparable performance to the original self-attention with fewer computational costs.\footnote{Our code is released at \url{https://github.com/pkuzengqi/Skyformer}}





%%%%%%%%%%%%%%%%%%%%%%%%%%%%
%%%%%%%%%%%%%%%%%%%%%%%%%%%%
%%%%%%%%%%%%%%%%%%%%%%%%%%%%






\begin{comment}

\iffalse
A potential issue in Transformer is its difficult training \citep{DBLP:conf/emnlp/LiuLGCH20},
which requires huge amounts of efforts to tune the hyper-parameters in training, including carefully choosing the optimizers and learning rate schedulers.
Specifically, the fluctuation caused by parameter updates will be amplified and thus destabilizes model training.
We conjecture this amplification effect comes from the softmax structure itself, as in applying softmax we need to enforce the row normalization to the matrix $\mtx{A}$, which greatly alters the scale of some rows.
\fi

This similarity motivates us to apply the most widely used computation method for large-scale kernel machines, the \nystrom method \citep{williams2001using, drineas2005nystrom}, to approximate the output of self-attention.
To realize the intuitive idea, we need to first address the thorny fact that the un-normalized attention score matrix is not positive semidefinite (PSD), 
as in general the query matrix unnecessarily equal the key matrix.
In this work, 

Furthermore, as we conjecture the instability in Transformer training comes from the softmax structure, we propose to replace the softmax structure with Gaussian kernel, which has a similar form to the original self-attention.

We conclude the two improvements with a new proposed model, Skyformer, 
Motivated by the computation methods for kernel machines that alleviates the high computational cost of pairwise dot products, we introduce Skyformer which replaces the softmax structure with Gaussian kernel to stabilize the model training and adapts the \nystrom method to a non-PSD matrix to accelerate the computation.





\textbf{general motivation}:
\begin{itemize}
\item Difficult to train the original Transformers with softmax structures.
\item resource-intensive: both memory consumption and training/inference time, party of labs rich in computational resource
\item lack of long document support: The dot-product self-attention operation is quadratic to seq length and fundamentally limits the maximum length of the input sequence. Truncation or sliding window may lose some local context and  long-distance dependency. 



\end{itemize}



\textbf{motivation}:
\begin{itemize}
\item resource-intensive: both memory consumption and training/inference time, party of labs rich in computational resource
\item lack of long document support: The dot-product self-attention operation is quadratic to seq length and fundamentally limits the maximum length of the input sequence. Truncation or sliding window may lose some local context and  long-distance dependency. 
\item Some previous methods coincide with the past literature to approximate gram matrix in kernel methods.
It is natural to introduce the powerful \nystrom method to self-attention approximation. \heng{explain why it's natural}
\item The current discussion on time complexity is groundless.
Some methods claim they have linear complexity solely because they keep using a fixed rank regardless of the sequence length in practice. 
Of course the original attention cannot be well approximated,
otherwise they theoretically outperform the SOTA kernel density estimation method.
To ease the efficiency comparison between different methods, we utilize the concepts from kernel methods to give a normative complexity criterion.

\end{itemize}

new claim: less dim_head, more num_head
original method cannot afford

do q and k dim must have the same dim as v?
does better approximation mean better performance?

\textbf{Some important discussion (to be deleted)}

Add some discussion on the metric (spectral norm loss), and remark on the complexity analysis.
\textbf{The first experiment} should be comparing the spectral norm loss of different methods.

why new setting can avoid rank collapse
\william{by setting $d=1$, maybe not very practical since the number of heads will blow up.}


\textbf{challenges in / advantages over existing methods}: 
\textbf{Not exactly mention the name of the following models.}
\begin{itemize}
\item  limitation of previous work (Performer):
% the projection dimension (number of random features) is required to be less than $d$, which is a concern when $d$ is tiny relative to $n$ and limits the highest possible accuracy.
Also, the efficiency of approximation could be further improved: it is known that random feature-based methods would require much more projection dimensions than needed.
% Specifically, for an empirical Gaussian kernel (equivalent to softmax kernel) matrix, the statistical dimension (the least rank needed to approximate the original matrix) should be much less than $d$ given usually $d = O(n)$ in applications.
Besides, random features are not easy to be extended to non-translation-invariant kernels, as Mercer's theorem only applies to invariant kernels.
In contrast, our method can be easily applied to some new variants of attention, such as lambda network~\cite{lambdanetwork}. 
\item lack of theoretical guarantee on approximation error in term of spectral norm, which is the most common metric in computation method,
theoretical guarantee that error is small in term of spectral norm
% \item evaluation of methods is based on the expensive experiments
% \item utilizing information from all tokens; separate blockwise or window-based attn will lose context
\item Nyströmformer is also based on \nystrom method, but they do not utilize a key property that \nystrom method should be applied to PSD matrix.
Thus their performance can still be greatly improved, and their theoretical results are loose.

\end{itemize}


\end{comment}

\section{Related Work}
\label{sec:related_work}


% \subsection{Relate efficient transformers to kernel methods}

Among all the transformer acceleration methods, including attention layer simplification by pruning redundant attention heads~\citep{DBLP:conf/acl/VoitaTMST19, DBLP:conf/nips/MichelLN19} and
model size reduction with knowledge distillation~\citep{DBLP:conf/emnlp/JiaoYSJCL0L20, DBLP:journals/corr/abs-1903-12136, DBLP:conf/acl/LiuZWZDJ20}, 
we focus on attention approximation models, which are closely related to kernel methods. 


% \textbf{Pattern-based Attention Approximation. }  
To reduce the time and space complexity by avoiding exhaustive computation over the attention metric, recent studies propose to apply sparse attention patterns to limit the numbers of elements participating in matrix multiplications~\citep{DBLP:conf/emnlp/QiuMLYW020, DBLP:journals/corr/abs-1904-10509, DBLP:conf/nips/ZaheerGDAAOPRWY20, DBLP:journals/corr/abs-2004-05150}. 
Beyond limiting the attention to fixed patterns, some approaches learn the patterns by determining
token assignments to relevant groups ~\citep{DBLP:conf/iclr/KitaevKL20, DBLP:journals/tacl/RoySVG21}. 
Those models utilize local and global information in the attention score matrix to perform approximation,
which coincides with the attempt to accelerate the computation in Gaussian processes \citep{snelson2007local}.


% \textbf{Low-Rank Attention Approximation. } 
The attention score matrix is known to exhibit a very fast rate of singular value decay \citep{bhojanapalli2020low, dong2021attention}, similar to that of an empirical kernel matrix \citep{yang2017randomized}.
This near singular property motivates many low-rank attention approximation methods to skillfully leverage the computation techniques in kernel methods.
Among them, Linformer~\citep{DBLP:journals/corr/abs-2006-04768} compresses the size of the key and value matrix with random projections based on the Johnson–Lindenstrauss transform, a common randomized sketching method in Gaussian processes \citep{yang2017randomized};
Reformer~\citep{DBLP:conf/iclr/KitaevKL20} applies locality-sensitive hashing (LSH) \citep{har2012approximate} to simplify the computation of the attention score matrix,
which is widely used in kernel density estimation \citep{charikar2017hashing, DBLP:conf/nips/BackursIW19};
Performer~\citep{DBLP:journals/corr/abs-2009-14794} projects both query and key matrix through random Fourier features~\citep{rahimi2007random},
heavily exploiting Bochner Theorem for stationary kernels. 

The most related papers to ours are linear attention~\citep{katharopoulos2020transformers}, Synthesizer~\citep{DBLP:journals/corr/abs-2005-00743}, and Nystr\"omformer~\citep{DBLP:journals/corr/abs-2102-03902}.
Linear attention takes the softmax structure in self-attention as a measure of similarity and replaces it with the dot product of separately activated query and key matrices;
Synthesizer aims to modify the original self-attention by replacing the dot product before softmax with Synthetic Attention, which generates the alignment matrix independent of token-token dependencies.
Their attempts indicate that the softmax structure in self-attention is not the only feasible choice, and justify our usage of kernelized attention.
Rather than remodeling self-attention, Nystr\"omformer applies the \nystrom method \citep{williams2001using, drineas2005nystrom}, a powerful and effective method for large-scale kernel machines acceleration, to approximate the attention score matrix.
However, Nystr\"omformer applies the \nystrom method to a non-PSD matrix, and thus fails to utilize the full potential of the \nystrom method.
This issue is resolved in our proposed Skyformer by instead lifting the kernelized attention score matrix into a large PSD matrix which contains the target non-PSD matrix as its off-diagonal block.
For more details on attention approximation methods,
we refer readers to a survey paper on efficient transformers~\citep{DBLP:journals/corr/abs-2009-06732}.



\iffalse
% Note that some models are not listed here because they do not aim to approximate the original self-attention, such as .

\subsection{\nystrom Method for Empirical Kernel Matrix Approximation}

% \william{Plan to combine the two subsections. Introduce a former paper, following a kernel method paper.}


Previous studies have shown the efficiency of \nystrom method~\citep{} in empirical kernel matrix approximation, such as such as Recursive Sampling for the Nyström Method~\citep{DBLP:conf/nips/MuscoM17} and ---more---.
% Our method mainly utilize the \nystrom method~\citep{}, which is very efficient for empirical kernel matrix approximation.
% There are a bundle of studies on its usage in kernel methods, such as Recursive Sampling for the Nyström Method~\citep{DBLP:conf/nips/MuscoM17}, and ---more---
% two paper from aistat\citep{chen2021fast, chen2021accumulations}.


% We will come back to the important property in the following sections.

\fi
\section{Preliminaries and notations}

\subsection{Revisiting self-attention}

For a given input sequence $\mtx{X} \in \mb R^{n \times d_0}$ of length $n$ and embedding dimension $d_0$,
The dot-product attention for a single head in Transformer~\cite{DBLP:conf/nips/VaswaniSPUJGKP17} is defined as
\begin{equation}
    \nonumber
    \text{Attention}(\mtx{Q},\mtx{K},\mtx{V}) = \text{softmax}\left(\frac{\mtx{Q} \mtx{K}^{T}}{\sqrt{p}}\right) \mtx{V}
\end{equation}
where $\mtx{Q} = \mtx{X} \mtx{W}_Q$, $\mtx{K} = \mtx{X} \mtx{W}_K$, and $\mtx{V} = \mtx{X} \mtx{W}_V$,
and $\mtx{W}_Q$, $\mtx{W}_K$ and $\mtx{W}_V$ are the query, key, and value weight metrics that linearly project the input $\mtx X$ of $d_0$ dimension to an output tensor of $p$ dimensions.

To simplify the future analysis, the left softmax term can be rewritten into $\mtx{D}^{-1} \mtx{A}$,
where $\mtx{A} \defeq \exp(\mtx{Q} \mtx{K}^{T} / \sqrt{p})$ is the un-normalized attention score matrix; 
$\mtx{D}$ is a diagonal matrix whose diagonal is $\exp(\mtx{Q} \mtx{K}^{T} / \sqrt{p}) \cdot \mtx{1}$ (by convention $\mtx{1}$ is a size-$n$ vector with all elements being $1$).
Following the notation in Performer \citep{DBLP:journals/corr/abs-2009-14794},
we define $\text{SM}(\mtx{q}, \mtx{k}) \defeq \exp(\mtx{q}^T \mtx{k} / \sqrt{p})$ as the softmax kernel function,
and represent $\mtx{A}$ by the notation $\text{SM}(\mtx{Q}, \mtx{K})$, which means the element $a_{ij}$ from the $i$-th row and $j$-th column in $\mtx{A}$ is equal to $\text{SM}(\mtx{q}_i, \mtx{k}_j)$.
Throughout this paper $\mtx{q}_i$ (resp. $\mtx{k}_j$) means the $i$-th (resp. $j$-th) row in $\mtx{Q}$ (resp. $\mtx{K}$).

We close this subsection with a short lemma to show $\text{SM}(\cdot, \cdot)$ is a positive semidefinite (PSD) kernel function \citep[Definition~12.6]{wainwright2019high} by relating it to Gaussian kernels.
\begin{lem}
$\text{SM}(\cdot, \cdot)$ is a PSD kernel function. 
Equivalently, for all integers $n \geq 1$ and elements $\{\mtx{q}_i\}_{i=1}^n \subseteq \mb R^{p}$,
the $n$-by-$n$ matrix $\mtx{C} = \text{SM}(\mtx{Q}, \mtx{Q})$ is PSD.
\end{lem}
\begin{proof}
We first state an important equation to connect the softmax kernel and Gaussian kernels as follows:
\begin{align*}
\text{SM}(\mtx{q}_i, \mtx{q}_j) = \exp\left(\frac{\mtx{q}_i^T \mtx{q}_j}{\sqrt{p}}\right) 
= \exp\left(\frac{\|\mtx{q}_i\|^2}{2 \sqrt{p}}\right) \exp\left(-\frac{\|\mtx{q}_i - \mtx{q}_j\|^2}{2 \sqrt{p}}\right) \exp\left(\frac{\|\mtx{q}_j\|^2}{2 \sqrt{p}}\right).
\end{align*}
The middle part $\exp\left(\frac{\|\mtx{q}_i - \mtx{q}_j\|^2}{2 \sqrt{p}}\right)$ is exactly a Gaussian kernel with bandwidth $p^{\frac14}$.
(\citet{DBLP:journals/corr/abs-2009-14794} have more discussion on the findings.)

Through this equation, we can rewrite $\mtx{C}$ as
\begin{align}
\label{eqn:sm_rbf}
\mtx{C} = \mtx{D}_Q^{1/2} \cdot \kappa\left(\frac{\mtx{Q}}{p^{1/4}}, \frac{\mtx{Q}}{p^{1/4}} \right) \cdot \mtx{D}_Q^{1/2},
\end{align}
where $\mtx{D}_Q$ is a diagonal matrix with elements $(\mtx{D}_Q)_{ii} = \exp\left(\frac{\|\mtx{q}_i\|^2}{\sqrt{p}}\right), \forall i \in [n]$,
and $\kappa(\mtx{q}_i, \mtx{q}_j) \defeq \exp\left(-{\|\mtx{q}_i - \mtx{q}_j\|^2}/{2}\right)$ is the standard Gaussian kernel function.

We prove the lemma by using the fact that $\kappa$ is a PSD kernel and $\kappa\left(\frac{\mtx{Q}}{p^{1/4}}, \frac{\mtx{Q}}{p^{1/4}} \right)$ is a PSD matrix.
\end{proof}


\subsection{\nystrom method}

Due to the intrinsic low-rankness of an empirical kernel matrix $\mtx{B}$, the so-called \nystrom method that replaces $\mtx{B}$ with its low-rank approximation $\tilde{\mtx{B}}$, has been applied to accelerate kernel methods \citep{gittens2016revisiting, kumar2009sampling, williams2001using}. 
% statistical dimension (eigenvalue decaying rate of empirical kernel matrix.)
Specifically, the \nystrom approximation of $\mtx{B}$ is the matrix $\tilde{\mtx{B}} = \mtx{B} \mtx{S}(\mtx{S}^T \mtx{B} \mtx{S})^\dagger \mtx{S}^T \mtx{B}$,
where $(\cdot)^\dagger$ denotes the Moore-Penrose pseudoinverse of a matrix, and $\mtx{S} \in \mb R^{n \times d}$ is a zero-one sub-sampling matrix whose columns are a subset of the columns in $\mtx{I}$, indicating which $d$ observations have been selected. 
The formal definition of the uniform sub-sampling matrix is given as follows:
\begin{definition}[Uniform sub-sampling matrix]
\label{def:subsampling}
% Consider a uniform discrete distribution which draws $i$ with probability $\frac1n$.
For a random matrix $\mtx{S} \in \mb R^{n \times d}$, if $\mtx{S}$ has i.i.d. columns and the $j$-th column $\mtx{S}^{(j)}$ can randomly be $\sqrt{\frac{1}{d}} \mtx{e}_i$ with probability $\frac1n$,
where $\mtx{e}_i$ is the $i$-th column of the $n$-by-$n$ identity matrix $\mtx{I}_n$, 
then $\mtx{S}$ is called a uniform sub-sampling.
\end{definition}

We close this subsection with a remark that it is not appropriate to directly extend the \nystrom method from kernel method to self-attention due to a core requirement that $\mtx{B}$ should be PSD with consideration of approximation performance improvement.
% , which is critical in improving the approximation performance.
We will show in the next section how to address this challenge and properly adapt \nystrom method to non-PSD matrices.

% \heng{maybe it would be good to add a table to summarize the pros and cons of these methods and your solution}

\subsection{Approximation evaluation}

Beyond the time and space complexity, attention acceleration methods have been mostly evaluated with empirical experiment results, such as the perplexity of pretrained language models and the fine-tuned performance on downstream natural language understanding tasks. 
Specifically, Long Range Arena benchmark~\citep{DBLP:journals/corr/abs-2011-04006} has been proposed to systematically evaluate the performance of efficient transformers with ten NLP tasks in long-context scenarios.
However, such empirical results are indirect for theoretical analysis.
% Many different methods have been proposed to approximate the self-attention.
% However, some of the existing methods solely evaluate the approximation performance by the downstream transformer model accuracy, 
% which is indirect and hard to perform theoretical analysis.
Therefore, we introduce a common criterion used in matrix approximation, spectral norm, to ease the future discussion on performance.
\begin{definition}[Spectral norm guarantee for matrix approximation (MA)]
\label{def:ma}
Given a matrix $\mtx{M} \in \mb R^{n_1 \times n_2}$, two constants $\varepsilon > 0, \delta < \frac12$, we say that its approximation matrix $\wt{\mtx{M}} \in \mb R^{n_1 \times n_2}$ satisfies $(\varepsilon, \delta)$-MA property for $\mtx{M}$, if
\begin{align}
\Prob{\|\mtx{M} - \wt{\mtx{M}}\| > \varepsilon \|\mtx{M}\|} < \delta.
\end{align}
\end{definition}

In previous works, the direct analysis of the approximation error to the entire output $\mtx{D}^{-1} \mtx{A} \mtx{V}$ in the $(\varepsilon, \delta)$-MA manner is usually spared due to the difficulty caused by the complex softmax structure.
In this paper, with the new kernelized attention, we are allowed to perform the analysis through the existing theoretical results in kernel methods.
Consequently, in Section~\ref{sec:norm} we are able to give a relatively precise error analysis on the approximation of Skyformer to the entire kernelized attention,
which eases the future comparison with other methods approximating kernelized attention.
% focus on the un-normalized attention score matrix $\mtx{A}$ study the $(\varepsilon, \delta)$-MA property for ,
% and from this perspective we are give precisely in Section~\ref{sec:related_work}.
% Concretely, in Section~\ref{sec:norm} we evaluate the approximation performance for the whole self-attention $\mtx{D}^{-1} \mtx{A} \mtx{V}$ from real BERT models.
\section{Method}
\label{Sec:Method}

\subsection{Kernelized Attention}
\label{sec:kernel_attn}

Kernelized Attention replaces the softmax structure in vanilla self-attention with a Gaussian kernel,
and the new attention model is stated as:
\begin{align}
\text{Kernelized-Attention}(\mtx{Q},\mtx{K},\mtx{V}) = \mtx{C} \mtx{V} \defeq \kappa\left(\frac{\mtx{Q}}{p^{1/4}}, \frac{\mtx{K}}{p^{1/4}} \right) \mtx{V},
\end{align}
where we define the $n$-by-$n$ matrix $\mtx{C}$ as the kernelized attention score matrix $\kappa(\mtx{Q} / p^{1/4}, \mtx{K} / p^{1/4})$.

The justification for using the kernelized attention model is as follows.
A significant advantage of softmax attention is that tokens are allowed to attend to a limited number of other important tokens in the sequence.
We observe that Gaussian kernel function can play a similar role.
The expression of a Gaussian kernel is $\kappa(\mtx{q}_i,  \mtx{k}_j) \defeq \exp \left(-\|\mtx{q}_i - \mtx{k}_j\|^2 / 2 \right)$.
Via this expression, for token $i$ in the query, Gaussian kernel assigns a large attention score to the token $j$ when $\mtx{k}_j$ is close to $\mtx{q}_i$.
The distance-based weight assignment is indeed considered as a major reason why kernel methods are powerful.
The form of kernelized attention also leads to an automatic normalization. 
Based on Equation~(\ref{eqn:sm_rbf}), the new attention model can be rewritten in terms of the un-normalized attention score matrix $\mtx{A}$ as
\begin{align*}
\text{Kernelized-Attention}(\mtx{Q},\mtx{K},\mtx{V}) = \left( \mtx{D}_Q^{-1/2} \cdot \mtx{A} \cdot \mtx{D}_K^{-1/2} \right) \mtx{V},
\end{align*}
where $\mtx{D}_Q$ (resp. $\mtx{D}_K$) is a diagonal matrix with elements $(\mtx{D}_Q)_{ii} = \exp\left( \frac{\|\mtx{q}_i\|^2}{\sqrt{p}} \right)$ 
(resp. $(\mtx{D}_K)_{ii} = \exp \left( \frac{\|\mtx{k}_i\|^2}{\sqrt{p}} \right)$), $\forall i \in [n]$.
We remark the kernelized attention model can thus be formally taken as a variant of the original self-attention, 
which instead normalizes the matrix $\mtx{A}$ in a form of $\mtx{D}^{-1} \mtx{A}$.
The intrinsic normalization allows kernelized attention to have a more reasonable condition number than self-attention, 
which benefits the stability of model training.
To demonstrate the improvement in stability, we additionally provide a toy experiment in Appendix~\ref{sec:exp_stability},
which shows the ``condition number" of kernelized attention is smaller than self-attention.
Moreover, empirical evaluation in Section~\ref{sec:exp} supports our claim that the new attention model can attain a comparable performance to the original attention model.


\subsection{Skyformer: a modified \nystrom method}
\label{sec:nystrom}


Before jumping into details of Skyformer, 
we first propose a method to apply \nystrom method to approximate an asymmetric (and thus non-PSD) empirical kernel matrix $\mtx{B}$ constructed with any PSD kernel $\phi(\cdot, \cdot)$.
Specifically, with two different $n$-by-$p$ design matrices $\mtx{Q}$ and $\mtx{K}$,
its element $b_{ij}$ from the $i$-th row and $j$-th column in $\mtx{B}$ is equal to $\phi(\mtx{q}_i, \mtx{k}_j)$,
where $\mtx{q}_i$ (resp. $\mtx{k}_j$) is the $i$-th (resp. $j$-th) row in $\mtx{Q}$ (resp. $\mtx{K}$).
We remark this type of empirical kernel matrices involves the un-normalized attention score matrix $\mtx{A} \defeq \text{SM}(\mtx{Q}, \mtx{K})$,
and the empirical Gaussian kernel matrix $\mtx{C} \defeq \kappa(\mtx{Q} / p^{1/4}, \mtx{K} / p^{1/4})$.
% (The proof to show $\kappa(\mtx{q}_i, \mtx{k}_j) = \exp(\mtx{q}_i^T \mtx{k}_j / \sqrt{p})$ is a PSD kernel function is deferred to Appendix~\ref{sec:facts}.)
Therefore this method leads to a low-rank approximation to the output of either self-attention $\mtx{D}^{-1} \mtx{A} \mtx{V}$ or Kernelized Attention $\mtx{C} \mtx{V}$.
($\mtx{D}$ in self-attention can be obtained by computing $\mtx{A} \cdot \mtx{1}$, and thus a low-rank approximation to $\mtx{A}$ also implies an approximation to $\mtx{D}$.)
% with a low dimensional randomized sketch of the complete symmetric matrix $\bar{\mtx{A}}$.

% Figure~\ref{fig:model} shows our proposed method. 
% mark: delete figure

Computational details are stated as follows.
To tackle the challenge of approximating a non-PSD matrix $\mtx{B}$, our first step is to complete the matrix into a PSD matrix $\bar{\mtx{B}}$:
\begin{align}
\label{eqn:concat}
    \bar{\mtx{B}} \defeq \phi \left(
        \begin{pmatrix}
        \mtx{Q}  \\
        \mtx{K} 
        \end{pmatrix},
        \begin{pmatrix}
        \mtx{Q}  \\
        \mtx{K} 
        \end{pmatrix} \right).
\end{align}
Then we approximate $\bar{\mtx{B}}$ with $\tilde{\bar{\mtx{B}}}$ through
\begin{align}
\label{eqn:tilde_bar}
\tilde{\bar{\mtx{B}}} = \bar{\mtx{B}} \mtx{S} (\mtx{S}^{T} \bar{\mtx{B}}\textbf{S})^{\dagger} \textbf{S}^{T}\bar{\mtx{B}},
\end{align}
where $\mtx{S}$ is a $2n$-by-$d$ uniform sub-sampling matrix as defined in Definition~\ref{def:subsampling}.
The final approximation will be given as 
\begin{align}
\label{eqn:approx}
\tilde{\mtx{B}} \defeq (\mtx{I}, \mtx{0}) \tilde{\bar{\mtx{B}}} (\mtx{0}, \mtx{I})^T.
\end{align}
The original matrix $\mtx{B}$ can be well-approximated by $\tilde{\mtx{B}}$ due to the following inequality
\begin{align*}
\|\mtx{B} - \tilde{\mtx{B}}\| = \|(\mtx{I}, \mtx{0}) (\bar{\mtx{B}} - \tilde{\bar{\mtx{B}}}) (\mtx{0}, \mtx{I})^T\| \leq \|\bar{\mtx{B}} - \tilde{\bar{\mtx{B}}}\|,
\end{align*}
and thus we show our task of approximating the non-PSD matrix $\mtx{B}$ boils down to well approximating the PSD matrix $\bar{\mtx{B}}$.

% \william{We empirically show the performance of the modified \nystrom method on approximation error in Figure~\ref{},}
% through comparing the method with other approximation methods used by existing efficient Transformers.
% The complete settings can be found in Appendix~\ref{sec:exp_norm}.

\textbf{Remark.} 
The reason why $\mtx{B}$ can be well approximated by a low-rank $\tilde{B}$ is that as an empirical kernel matrix the eigenvalues in $\bar{\mtx{B}}$ usually decay fast,
and thus there are many small eigenvalues in the long tail. 
In this case, theoretically a low-rank matrix (e.g. truncated singular value decomposition (SVD) of $\bar{\mtx{B}}$) has enough potential to well approximate the original matrix $\bar{\mtx{B}}$ (and $\mtx{B}$ accordingly) in terms of spectral norm.
% \section{The \MakeLowercase{i}W\MakeLowercase{inr}NFL model}
\label{sec:model}

In this section we are going to present the data we used to develop our in-game probability model as well as the design details of {\method}. 

{\bf Data: }In order to perform our analysis we utilize a dataset collected from NFL's Game Center for all the regular season games between the seasons 2009 and 2016. 
We access the data using the Python {\tt nflgame} API \cite{nflgame}. 
The dataset includes detailed play-by-play information for every game that took place during these seasons. 
This information is used to obtain the state of the game that will drive the design of {\method}. 
In total, we collected information for 2,048 regular season games and a total of 338,294 snaps/plays. 

{\bf Model: }
{\method} is based on a logistic regression model that calculates the probability of the home team winning given the current status of the game as: 

\begin{equation}
\Pr(H=1| \mathbf{x})= \frac{\exp(\mathbf{\weight}^T\cdot\mathbf{x})}{1+\exp(\mathbf{\weight}^T\cdot\mathbf{x})}
\label{eq:reg}
\end{equation}
where $H$ is the dependent random variable of our model representing whether the home team wins or not, $\mathbf{x}$ is the vector with the independent variables, while the coefficient vector $\mathbf{\weight}$ includes the weights for each independent variable and is estimated using the corresponding data.  
For a game of infinite duration a linear model could be a very good approximation.  
However, the boundary effects from the finite duration of a game create several non-linearities \cite{winston2012mathletics}.  
For this reason, we enhance our model - using the same set of features - with a Support Vector Machine classifier with radial kernel for the last three minutes of regulation.  
In order to obtain a probability output from the SVM classifier, we further use Platt's scaling \cite{platt1999probabilistic}: 

\begin{equation}
\Pr(H=1| \mathbf{x})= \frac{1}{1+\exp{(Af(x)+B)}}
\label{eq:platt}
\end{equation}
where $f(x)$ is the uncalibrated value produced by the SVM classifier: 

\begin{equation}
f(x) = \sum_{i} (\alpha_i y_i k(\mathbf{x}_i\cdot\mathbf{x}))+ b
\label{eq:svm}
\end{equation}
where $k(\mathbf{x},\mathbf{x}')$ is the kernel used for the SVM.   
Figure \ref{fig:iwinrNFL} depicts the simple flow chart of {\method}. 


\begin{figure}[t]
\begin{center}
\includegraphics[scale=0.35]{plots/iwinrNFL.pdf}%\vspacecap
 \caption{{\method} includes a linear and a non-linear component.}
 \label{fig:iwinrNFL}
\end{center}
\end{figure}

In order to describe the status of the game we use the following variables:

\begin{enumerate}
\item {\bf Ball Possession Team:} This binary feature captures whether the home or the visiting team has the ball possession
\item {\bf Score Differential:} This feature captures the current score differential (home - visiting)
\item {\bf Timeouts Remaining:} This feature is represented by two independent variables - one for the home and one for the away team - and they capture the number of timeouts remaining for each of the teams
%\item {\bf Quarter:} This feature captures the current quarter of the game
%\item {\bf Time Remaining:} This feature captures the time (in seconds) remaining for the current quarter to end
\item {\bf Time Elapsed: } This feature captures the time elapsed since the beginning of the game
\item {\bf Down:} This feature represents the down of the team in possession
\item {\bf Field Position:} This feature captures the distance covered by the team in possession from their own yard line
\item {\bf Yards-to-go:} This variables represents the number of yards needed for a first down
\item {\bf Ball Possession Time: } This variable captures the time that the offensive unit of the home team is on the field 
\item {\bf Ranking Differential: } This variable represents the difference of the win percentage for the two team (home - visiting)
\end{enumerate}

The last independent variable is representative of the power ranking difference between the two teams. 
Most of the existing models that include such a variable are using the Vegas line spread for each game.  
We choose not to do so for the following reason.  
The objective of the Vegas line is not to predict game outcomes but rather distribute money across the different bets.  
Exactly because of this objective the line is changing during the week before the game.  
While this line can change due to new information for the competing teams (e.g., injury updates), the line is mainly changing when a particular team has accumulated the majority of the bets. 
In this case it will also be hard to choose which line to use (e.g., the opening, the closing or some average of them).  
Therefore, we choose to use the win percentage differential of the two teams as an indicator of their strength (even though this has its own issues given the uneven schedule in NFL).  
However, note that if one would like to use the point spread as a variable this can be easily incorporated in the model. 
Table \ref{tab:iwinrnfl} presents the coefficients of the logistic regression model of {\method} with standardized independent variables for better comparisons. 


\begin{table}[ht]
\begin{center}
\def\sym#1{\ifmmode^{#1}\else\(^{#1}\)\fi}
\begin{tabular}{l*{1}{c}}
\toprule
                    &\multicolumn{1}{c}{(1)}\\
                    &\multicolumn{1}{c}{Winner}\\
\midrule
Possession Team (H)         &      0.41\sym{***}\\
                    &     (49.19)         \\
\addlinespace
Score Differential           &      3.59\sym{***}\\
                    &    (247.34)         \\
\addlinespace
Home Timeouts           &     0.12\sym{***}\\
                    &      (8.74)         \\
\addlinespace
Away Timeouts           &     -0.11\sym{***}\\
                    &    (-12.47)         \\
\addlinespace
Ball Possession Time  &     -0.05.\\
                    &    (-1.66)         \\
\addlinespace
Time Lapsed       &   -0.05.\\
                    &      (-1.66)         \\
\addlinespace
Down                &   -0.01         \\
                    &      (0.04)         \\
\addlinespace
Field Position            &   0.02\sym{**} \\
                    &      (2.71)         \\
\addlinespace
Yards-to-go                &  -0.01         \\
                    &      (0.23)         \\
\addlinespace
Rating differential         &       0.75\sym{***}\\
                    &     (80.47)         \\
\addlinespace
Intercept            &       0.57\sym{*}\\
                    &    (2.09)         \\
\midrule
Observations        &      338,294         \\
\bottomrule
\multicolumn{2}{l}{\footnotesize \textit{t} statistics in parentheses}\\
\multicolumn{2}{l}{\footnotesize \sym{$_.$} \(p<0.1\), \sym{*} \(p<0.05\), \sym{**} \(p<0.01\), \sym{***} \(p<0.001\)}\\
\end{tabular}
\end{center}
\caption{Standardized logisitic regression coefficients for {\method}.}
\label{tab:iwinrnfl}
\end{table}


As we can see, as one might have expected the current scoring differential exhibits the strongest correlation with the in-game win probability.  
The only factors that do not appear to be statistically significant predictors of the dependent variable are the down and the yards-to-go. 
Even though the corresponding coefficients are negative as one might have expected (e.g., being at an earlier down gives you more chances to advance the ball), they are not significant in estimating the win probability. 
On the contrary, all else being equal timeouts appear to be quiet important since they can help a team stop the clock, while teams with better win percentage appear to have an advantage as well, since this can be a sign of a better team. 
In the following section we provide a detailed evaluation of {\method}.


With the derivation above, we officially introduce our proposed Skyformer as an approximation to Kernelized Attention, which applies the modified \nystrom method to the kernelized attention score matrix $\mtx{C}$.
The next two subsections will continue our discussion on it, and respectively state the theoretical analysis of its approximation error and some details of its implementation in practice.


\subsection{Error analysis of Skyformer}


As mentioned, an implicit advantage of using Kernelized Attention is that we can leverage the existing conclusions for kernel methods to analyze the theoretical properties of the model.
In this subsection, we aim to provide some theoretical analysis of its approximation error.

We state a high probability bound on the size $d$ of the sub-sampling matrix used in Skyformer to attain $(\varepsilon, \delta)$-MA property for the kernelized attention score matrix $\mtx{C}$ by the following theorem.
We refer the readers to the proof in Appendix~\ref{sec:thm_error} to take a closer look at our claim that the matrix to be approximated should be PSD is a key to the theoretical guarantee of \nystrom method.
\begin{thm}[Adapted from Lemma~9 and Theorem~3 \citep{DBLP:conf/nips/MuscoM17}]
\label{thm:tilde_K}
Consider the query, key, and value matrix $\mtx{Q}, \mtx{K}, \mtx{V} \in \mb R^{n \times p}$ and two positive constants $\varepsilon < 1, \delta < \frac12$.
For the empirical Gaussian kernel matrix $\mtx{C} \defeq \kappa(\mtx{Q} / p^{1/4}, \mtx{K} / p^{1/4})$ defined above, 
we let $\lambda \defeq \varepsilon \|\mtx{C}\| < \|\bar{\mtx{C}}\|$, where $\bar{\mtx{C}}$ is the completion of $\mtx{C}$ (similar to $\bar{\mtx{B}}$, constructed as substituting the Gaussian kernel with bandwidth $p^{1/4}$ for the arbitrary kernel function $\phi$ in Equation~(\ref{eqn:concat})).
We comment $\lambda$ serves as the regularization coefficient as well as the approximation error bound. 
To ease the analysis, we specifically define the $i$-th diagonal element of $\bar{\mtx{C}} (\bar{\mtx{C}} + \lambda \mtx{I}_{2n})^{-1}$ as leverage score $\ell_i, \forall i = 1,\dots, 2n$, 
and define their sum $\Tr \big( \bar{\mtx{C}} (\bar{\mtx{C}} + \lambda \mtx{I})^{-1} \big)$ as the statistical dimension $d_{stat}$,
which increases with $1 / \varepsilon$ as $\lambda \propto \varepsilon$. 
Suppose $\mtx{S}$ is a uniform sub-sampling matrix, 
and assume there exists a constant $\beta \in (0, 1]$ such that $\beta \leq \frac{d_{stat}}{2n \ell_i}, \forall i = 1,\dots,2n$. 
For the approximation matrix $\tilde{\bar{\mtx{C}}}$ constructed with $\bar{\mtx{C}}, \mtx{S}$ as in Equation~(\ref{eqn:tilde_bar}), 
there exists a constant $C$ such that if
\begin{align*}
d \geq C \frac{d_{stat}}{\beta} \log \frac{n}{\delta}
\end{align*}
then $\tilde{\bar{\mtx{C}}} \psdle \bar{\mtx{C}} \psdle \tilde{\bar{\mtx{C}}} + \lambda \mtx{I}$ with probability $1-\delta$.
Here $\psdle$ denotes the Loewner ordering: $\mtx{B} \psdle \mtx{A}$ means $\mtx{A} - \mtx{B}$ is positive semidefinite.
Furthermore, for our approximation $\tilde{\mtx{C}}$ in Equation~(\ref{eqn:approx}) to the kernelized attention score $\mtx{C}$, we have 
\begin{align*}
\|\tilde{\mtx{C}} - \mtx{C}\| \leq \lambda = \varepsilon \|\mtx{C}\|.
\end{align*}
\end{thm}

This theorem implies the time and space complexity of our proposed approximation depends on the statistical dimension $d_{stat}$.
If we directly use the conclusion from Gaussian kernels, $d_{stat}$ should be $\wt{\m O}(1)$ (complexity modulo poly-log term) \citep{yang2017randomized} due to the exponential eigenvalue decay rate of Gaussian kernels,  
which is comparable to the complexity of most other efficient transformers.
However, different than the case in the classical kernel methods, the distribution of the query and key matrix $\mtx{Q}$ and $\mtx{K}$ changes during the training procedure,
which may invalidate the conclusion about $d_{stat}$.
We leave the exact non-asymptotic analysis of the computational complexity for future work.
% We also remark the complexity analysis is discussed under a high probability bound setting, 
% which is not comparable to the linear $\m O(n)$ complexity of some previous methods. 



% The training of the approximation model with \nystrom method also benefits from the reduced condition number of $\mtx{S}^T (\mtx{D}_Q^{-1/2} \mtx{A} \mtx{D}_K^{-1/2}) \mtx{S}$.
% Some empirical analysis are reported in Appendix~\ref{sec:nyst_kernel_attn} to further validate the statement above.


\subsection{Workaround in implementation}
\label{sec:method_implementation}

A potential limitation with the implementation of the proposed method lies in the tricky fact that the matrix inversion on GPU is much slower and numerically less stable than the same operation on CPU due to the different back-end libraries in the two platforms.
We attempt to circumvent the problem by adapting the strategy in Nystr\"{o}mformer \citep{DBLP:journals/corr/abs-2102-03902} to our setting.
Specifically, we use the matrix-product-based iterative method \citep{razavi2014new} for finding approximate inverses, instead of some division-based methods (such as the conjugate gradient method) which induces some instability in model training.

To apply the iterative method and inverse matrix $\mtx{M} = \mtx{S}^{T} \bar{\mtx{C}} \textbf{S}$, we need to satisfy its assumption \citep[Theorem~2]{razavi2014new} that $\|\mtx{I} - \mtx{M}\| < 1$.
In practice, we instead pass the matrix $\mtx{D}_M^{-1/2} (\mtx{M} + \gamma \mtx{I}) \mtx{D}_M^{-1/2}$ as an input to the iterative method,
where $\gamma>0$ is a small constant and the diagonal matrix $\mtx{D}_M$ is defined as $\text{diag}\left( (\mtx{M} + \gamma \mtx{I}) \mtx{1} \right)$.
We give the following lemma to justify our practical usage of the method. The proof is deferred to Appendix~\ref{sec:lem_iter}.
\begin{lem}
\label{lem:iterative}
Given a constant $\gamma > 0$, if matrices $\mtx{M}$ is constructed as $\mtx{S}^{T} \bar{\mtx{C}} \textbf{S}$, and $\mtx{D}_M$ are defined as above,
then all the singular values of $\mtx{D}_M^{-1/2} (\mtx{M} + \gamma \mtx{I}) \mtx{D}_M^{-1/2}$ are within $(0, 1)$,
which implies that $\|\mtx{I} - \mtx{D}_M^{-1/2} (\mtx{M} + \gamma \mtx{I}) \mtx{D}_M^{-1/2}\| < 1$.
\end{lem}
We further comment that numerically an implicit risk of the Schulz-type iterative method we use is the unintended consequence of ``zero fill-in".
If we use some sparse kernels (e.g. test functions with bounded support) other than Gaussian kernels,
the empirical kernel matrices are sparse while the approximate inverse will converge to a dense matrix,
which increases the computational cost.



\subsection{Empirical approximation evaluation}
\label{sec:norm}

Spectral norm, the maximum singular value of a matrix, is a computation-light indicator of matrix approximation performance. 
In this work, we compare the spectral norm of the difference between the outputs from attention functions and the output from vanilla self-attention with the same input.

We use the initialized and pretrained bert-base-cased models from Huggingface's implementation~\citep{DBLP:journals/corr/abs-1910-03771} .
The input vector $X$ is embedded from the tokenized raw text in Wikitext-2 dataset~\citep{DBLP:conf/iclr/MerityX0S17}.
The query, key and value weight matrices in initialized or pretrained models  transform input $X$ into $Q, K, V$ of different distributions.
We compare the results with different sequence lengths and different numbers of features used in attention approximation methods. 
We set the number of features in the range of $2^4$ to $2^8$. 
More features usually require more computation resources.
% Note that the number of features used in Big Bird is roughly ten times of \textit{block\_size} ($(r+b+g)*b$ in their notations and by default set $b$ to 64), which means it is much larger than the number of features used in other methods.

Figure~\ref{fig:norm} shows the performance of the modified Nystr\"om method on approximation error with regards to the number of features.
We conclude that for Skyformer the approximation is significantly better with the increased number of features, while for other methods the gain is not obvious.
The good performance of the modified \nystrom method also validates our previous claim that the \nystrom method is currently one of the most powerful methods in large-scale kernel machines acceleration.

\textbf{Remark.} 
Although in a single step the modified \nystrom method in Section~\ref{sec:nystrom} can give low approximation error, we do not recommend directly applying it to the original self-attention.
With some exploratory experiments on classification tasks, we find the variant suffers a more severe gradient explosion issue than usual transformers.
We speculate that it is because the matrix $\mtx{S}^{T} \bar{\mtx{A}}\textbf{S}$ (in the middle of Equation~(\ref{eqn:approx})) inherits the high condition number of the original attention score matrices $\mtx{A}$, 
while the derivative of matrix inverse ($\left( \mtx{A}^{-1} \right)' = - \mtx{A}^{-1} \mtx{A}' \mtx{A}^{-1}$) further amplifies the condition number during backpropagation.


\begin{figure}[htbp]
\centering
\includegraphics[width=1.0\linewidth]{figures/sky_norm8.png}
\caption{
Spectral norm results with different sequence lengths under different $W_Q, W_K, W_V$ settings, either from initialized or pretrained BERT models. 
All methods are approximating the original self-attention output.
Y axis: Lower spectral loss means better approximation. 
X axis: Higher $d$ (number of features) means visiting more elements in the original matrix and bringing more computation costs.
The label ``Skyformer" here means that we use the algorithm behind Skyformer, mainly Eq. (5), to approximate the raw attention score matrix $A$ in self-attention. In this experiment, ``Skyformer" also needs to first approximate $A$, and then approximate $D$, as Performer does. 
}
\label{fig:norm}
\end{figure}



\iffalse
% \william{
% Point out Nystromformer indeed use twice our landmarks}

% (c.f. Section~\ref{Sec:Method})
% In short, this method approximate the softmax kernel matrix by a low dimensional randomized sketch of the complete symmetric matrix, which is expected to have outstanding performance.

\textbf{Symmetrization}. We use $\mtx{A}$ to denote the term $\exp(\mtx{Q} \mtx{K}^{T} \sqrt{d})$ and try to approximate $\mtx{A}$ with $\wt{\mtx{A}}$.
We first consider $\textbf{A}$ as the result of a kernel function $\kappa$
\begin{equation}
    \nonumber
    \textbf{A} = \kappa (\textbf{Q},\textbf{K}),
\end{equation}
and symmetrize it by extending into
\begin{equation}
    \nonumber
    \bar{\textbf{A}} = \kappa (
        \begin{pmatrix}
        \textbf{Q}  \\
        \textbf{K} 
        \end{pmatrix},
        \begin{pmatrix}
        \textbf{Q}  \\
        \textbf{K} 
        \end{pmatrix}).
\end{equation} 
Note our target $A = (I, 0) \bar A (0, I)^T$, and the approximation term can be correspondingly attained as
\begin{equation}
    \nonumber
    \Tilde{\textbf{A}} = 
        \begin{pmatrix}
            \textbf{I}_{n \times n},
            \textbf{0}
        \end{pmatrix}
        \widetilde{\bar{\textbf{A}}}
        \begin{pmatrix}
            \textbf{0} \\
            \textbf{I}_{n \times n}  
        \end{pmatrix},
\end{equation}
where $\bar{\textbf{A}}$ can be approximated by
\begin{equation}
    \nonumber
    \Tilde{\bar{\textbf{A}}} = \bar{\textbf{A}} \textbf{S} (\textbf{S}^{T}\bar{\textbf{A}}\textbf{S})^{+}\textbf{S}^{T}\bar{\textbf{A}}.
\end{equation}
$\textbf{S}$ here is a rescaled uniform sub-sampling matrix.
\subsection{Empirical approximation evaluation}

norm experiment
\fi

\iffalse

\begin{align*}
\int_{\mb R^p} \kappa(\mtx{q}_i,  \mtx{k}) \dd \mtx{k} = 1,
\end{align*}
and taking the matrix multiplication as a numerical integration we can simply use $\frac{1}{nh} \kappa(\mtx{q}_i/h,  \mtx{k}_j/h)$, where $h$ is the bandwidth used in Gaussian kernel, similar to the factor $p^{1/4}$ in the original attention.


The new kernelized attention also allows the improvement on the singular value decay rate.
As motivated by the classical kernel density estimation, we can set the bandwith $h = \cdots$, 
so that according to Equation~(\ref{}) the decay rate becomes $\cdots$.
This change also reduces the condition number of the new attention score matrix $\frac{1}{nh} \kappa(\mtx{Q} / h, \mtx{K} / h)$,
which benefits our proposed approximation method involving matrix inversion.


\fi

\section{Experimental Results}
\label{sec:exp}

\begin{figure}[htbp]
\centering
\includegraphics[width=1.0\linewidth]{figures/acc37.png}
\caption{
Validation accuracy changes with respect to training time for $50k$ steps. X axis: training time (s). Y axis: classification accuracy.
}
\label{fig:devacc}
\end{figure}



\begin{table}
\caption{Classification accuracy (\%) on LRA benchmark.}
\label{table:lra_acc}
\centering
\begin{tabular}{lcccccc}
\toprule

% \multicolumn{5}{c}{Accuracy} &
% \multicolumn{5}{c}{Time} \\
Model & Text    & ListOps & Retrieval & Pathfinder &  Image &	AVG.  \\
\midrule
Self-Attention & 61.95& 	38.37& 	80.69& 	65.26& 	40.57 & 57.37\\
Kernelized Attention & 60.22 & 	38.78 & 	81.77 & 	70.73 & 41.29 & 58.56	\\
\cmidrule(r){2-6}
Nystromformer & 64.83 & 	38.51&	80.52 & 	69.48 & 41.30 & 58.93	 \\
Linformer  & 58.93 & 	37.45 &	78.19 & 	60.93 & 37.96 & 54.69	 \\
Informer& 62.64 & 	32.53&	77.57 & 	57.83 &  38.10 & 53.73	 \\
Performer  & 64.19 & 	38.02 &	80.04 & 	66.30 & 41.43 & 58.00	 \\
Reformer  & 62.93& 	37.68& 	78.99&	66.49& 	48.87 & 58.99\\
BigBird & 63.86 &	39.25&	80.28 &	68.72 & 43.16 & 59.05\\
\cmidrule(r){2-6}
\textbf{Skyformer}   & 64.70 & 	38.69&	82.06 & 	70.73 & 40.77 & \textbf{59.39}	 \\
\bottomrule
\end{tabular}
\end{table}


\begin{table}
\caption{Running time (hour) and peak memory usage (GB). \textbf{TC}: Text Classification. \textbf{LO}: ListOps. \textbf{RE}: Retrieval. \textbf{PF}: Pathfinder. \textbf{IC}: Image Classification. \textbf{KA}: Kernelized Attention.}
\label{table:lra_timespace}
\centering
\begin{tabular}{lcccccccccc}
\toprule
\multirow{2}{*}{Model} &
\multicolumn{5}{c}{Time (h)} &
\multicolumn{5}{c}{Memory (GB)} \\
 & TC    & LO & RE & PF & IC  & TC    & LO & RE & PF  & IC \\
\midrule
Self-Attention & 4.30 & 	2.24 & 	8.33 & 	2.57 & 	4.22 & 	10.37 & 		5.37 & 	10.77 & 	5.74 & 	11.47\\
KA & 3.91 & 	1.99 & 	7.46 & 	2.42 & 	4.05 & 	5.73 & 	5.94 & 	10.46 & 	6.38 & 	6.38\\
\cmidrule(r){2-6}
\cmidrule(r){7-11}
Nystromformer  &  0.71 &  	0.71 &  	1.29 &  	1.49 &  	2.70 &  	1.21 &  	1.37 &  	2.39 &  	3.35 &  	6.71\\
Linformer  & 0.65 & 	0.60 & 	1.13 & 	1.09 & 	2.19 & 	0.99 & 	0.99	 & 1.89 & 	1.97	 & 3.94 \\
Informer  & 1.60 & 	1.19 & 	2.91 & 	2.39 & 	3.90 & 	5.12 & 	4.85 & 	5.77 & 	4.75 & 	9.51 \\
Performer  &  0.77 &  	0.73 &  	1.41 &  	1.40 &  	2.55 &  	1.09 &  	1.09 &  	2.16 &  	2.20 &  	4.39\\
Reformer  &    0.94	 &  0.85 &  	1.73 &  	1.70 &  	3.08 &  	1.61 &  	1.61 &  	2.98 &  	3.21 &  	6.42\\
BigBird & 2.00 & 	1.88 & 	3.81 & 	3.39 & 	6.53 & 	2.83 & 	2.71 & 	4.97 & 	4.97 & 	9.95 \\
\cmidrule(r){2-6}
\cmidrule(r){7-11}
Skyformer  &   1.02 &  	1.29 &  	1.86 &  	2.03 &  	3.40 &  	1.59 &  	1.75 &  	3.15 &  	4.13 &  	8.26 \\
\bottomrule
\end{tabular}
\end{table}


% In this section our goal is to show whether the proposed method is sufficient in getting comparable performance to the full self-attention with much fewer computation resources. 

% We first compute the spectrum norm, a common evaluation metric and indicator of matrix multiplication approximation, to test the effectiveness of self-attention approximation comparing to other low-rank approximation methods as a sanity check.
% Next we consider two representative use cases of pretrained transformer-based language models, pre-training a Masked Language Model and fine-tuning a pretrained RoBERTa model on downstream tasks.
% , as the evidence of whether our proposed approximation method achieves comparable performance to the original self-attention with significant smaller computation cost.
% Next, we exam the performance on Long Range Arena (LRA) benchmark, which focuses on model quality under long-context scenarios.  

\textbf{Tasks and Datasets.}
We evaluate the proposed methods on five classification tasks on LRA benchmark~\citep{DBLP:journals/corr/abs-2011-04006}, which focuses on model quality under long-context scenarios: ListOps \citep{DBLP:conf/naacl/NangiaB18}, Text Classification on IMDb review dataset \citep{DBLP:conf/acl/MaasDPHNP11}, Document Retrieval on AAN dataset \citep{DBLP:journals/lre/RadevMQA13}, Pathfinder \citep{DBLP:conf/nips/LinsleyKVWS18}, and Image Classification on CIFAR-10 \citep{krizhevsky2009learning}.
The LRA benchmark covers diverse long-sequence tasks in sequence length, task difficulty, and inspected model abilities. 
For example, ListOps and Pathfinder evaluate the abilities to capture the long-range hierarchical dependency and spatial dependency, respectively, which poses challenges for sparse attention pattern based methods.
We report the classification accuracy on the test set, training time, and peak memory usage during training for each task.


\textbf{Baselines.}
Aside from the vanilla quadratic self-attention, we compare with Big Bird~\citep{DBLP:conf/nips/ZaheerGDAAOPRWY20}, Performer~\citep{DBLP:journals/corr/abs-2009-14794}, Linformer~\citep{DBLP:journals/corr/abs-2006-04768}, Nystr\"omformer \citep{DBLP:journals/corr/abs-2102-03902}, Informer~\citep{DBLP:journals/corr/abs-2012-07436},
and Reformer~\citep{DBLP:conf/iclr/KitaevKL20}.
Most methods are approximating the vanilla full attention for efficiency and thus are not expected to have better performance.
As it is not realistic to exhaustively fine-tune all models and search for the best performance under limited computation resources, we instead only replace the self-attention module with the various attention methods and keep other experimental settings the same for fair comparisons. 




% \heng{change 'almost impossible' to 'not realistic'}
% \heng{suggest to change 'aggressively' to exhaustively' because the former is too subjective about attitude }

% goals of exp:
% step num needed to converge, 
% less training time and space, 
% not necessarily better performance,
% longer seq may benefit some tasks

% hyper selection - number of features (two lengths)

% To show the approximation method really improves the efficiency, we need to train the models until convergence, and report the steps and time needed.
% Otherwise, it is possible that original transformer though needs more time in one step, but it needs less total steps, and finally instead requires less time.















\textbf{Implementation Details.}
We conduct each experiment on one Tesla V100 SXM2 16GB.
We use the LRA evaluation benchmark reimplemented in PyTorch by~\citet{DBLP:journals/corr/abs-2102-03902}.
We use a 2-layer transformer model with $64$ embedding dimension, $128$ hidden dimension, $2$ attention heads, and mean pooling for classification. 
Batch size is selected conditioned on the memory requirements of the standard self-attention method, which leads to $16$ for Text Classification, $32$ for ListOps, $16$ for Document Retrieval, $128$ for Pathfinder, and $256$ for Image Classification.
Learning rate is set to $1e-4$ for Text Classification, ListOps, and Image Classification, and $2e-4$ for Retrieval and Pathfinder.
Each model on each task is trained for $50k$ steps, during which the best checkpoint with the highest accuracy on the development set will be saved for evaluation.
For comparable computation complexity, we control the number of features to be $128$ used in all methods (except Big Bird), under which setting the models will visit $128\cdot n$ elements in the attention matrix.
For numerical consistency, all experiment results are averaged across three runs with different random seeds.


We do not follow all settings in \citep{DBLP:journals/corr/abs-2102-03902} due to the hardware limitation.
The compromises, such as approximation dimension and gradient accumulations steps, might bring performance differences comparing to results reported in \citep{DBLP:journals/corr/abs-2102-03902}.
The training instability problem also helps explain the performance gap.



\textbf{Results.}
The training process of the standard softmax-based method is unstable as observed in Figure~\ref{fig:devacc}: it takes more steps to reach the stationary distribution of its long-time limit, and it is more easily getting stuck in a local minimum. 
Runs with different random seeds may bring divergent performances, and probably leads to lower averaged scores.
We have also tried directly approximating the self-attention method with the \nystrom method and observed numerical instability during training.

Replacing the softmax structure with Gaussian kernel somehow alleviates this instability problem with boosted performance as shown in Table~\ref{table:lra_acc}. 
However, the time and space requirement of Kernelized Attention is not significantly improved compared to the original version, which serves as the motivation to approximate Kernelized Self-Attention with \nystrom method.

Though not necessarily the fastest, our proposed Skyformer can efficiently converge to the long-time limit with comparable general performance in classification accuracy (Table~\ref{table:lra_acc}) and resource consumption (Table~\ref{table:lra_timespace}).
The advantages over the standard self-attention are significant with consistently less training time and generally better performance.
For example, Skyformer brings nearly 4 times speed-up on text classification and document retrieval while with 2.75\% and 1.37\% accuracy improvement over the standard self-attention.


\textbf{Limitations.}
The applications of Skyformer might be limited to long sequence tasks because for small sequence length $n$ the statistical dimension $d_stat$ might be close to $n$.

To make the claim above clear, we first reiterate that the efficiency of Skyformer is related to $d_stat$.
As implied by Theorem~\ref{thm:tilde_K}, the intrinsic difficulty of approximating a raw attention score matrix is concluded as $d_{stat}$, which corresponds to the effective rank of matrix $\bar{C}$. 
The complexity of Skyformer depends on the sub-sample size $d$ (the size of the sub-sampling matrix $S$). A large $d_{stat}$ leads to a large $d$ , and an inefficient application of the \nystrom method. 

The classical theory for statistical dimension only guarantees that $d_{stat}$ is small (compared to $n$) when $n$ is large enough, and it is possible the statistical dimension associated with a short sequence might be even close to the sequence $n$. 
Therefore a large $n$ serves as a condition to make the method work. 
Figure \ref{fig:norm} empirically shows that our method performs better with larger $n$’s. 








% \begin{figure}[htbp]
\centering
\includegraphics[width=0.35\linewidth]{figures/LRASVD.png}
\caption{
Singular value distribution of attention output. }
\label{fig:svd}
\end{figure}
% Figure~\ref{fig:svd} shows the singular value distribution of attention output from the second layer of standard transformer. 
% Results are averaged across one batch from test set.
% We conclude that Document Retrieval and Pathfinder tasks are more difficult, on which tasks our models outperforming other models.
% \vicki{add conclusions from figure 4}













% \begin{comment}
% \subsection{Language Model}

% \input{tables/lm}

% goal: (1) comparable performance to the original self-attention with significant smaller computation cost. (2) whether this task can benefit from long seq compare 512 and 4096


% \textbf{Tasks and Datasets.}

% settings: continuing training from the roberta released checkpoint (12-layer), and from-scratch training a toy 2-layer model. correspond to the init and pretrain settings in exp1 (motivation: different distribution). 


% % MLM pretraining is expensive, we continue pretraining from the RoBERTa (Liu et al., 2019) released checkpoint, while only making the minimal changes necessary to support Longformer’s attention mechanism. 

% two data settings (seq 512 and 4096 but same dataset)


% BookCorpus\citep{DBLP:conf/iccv/ZhuKZSUTF15}

% Wikipedia, wikiextractor\citep{Wikiextractor2015}

% Realnews\citep{DBLP:conf/nips/ZellersHRBFRC19}




% \textbf{Implementation Details.}

% load and toy

% number of features


% \textbf{Results.}

% \subsection{GLUE Benchmark}

% 
\begin{table}[t]
\centering
\footnotesize
\renewcommand{\arraystretch}{1.3}
\begin{tabular}{l | c | c | c}
\toprule
\textbf{Model} & \bf SST & \bf CoLA & \bf Others (avg) \\
\midrule 
% \multicolumn{10}{l}{\textit{Transformer Models}}\\
% Base &  &  &  &  &  &  &  &  &  \\
% \quad + Masked &  &  &  &  &  &  &  &  &  \\
% \quad + MLM &  &  &  &  &  &  &  &  &  \\
% Large &  &  &  &  &  &  &  &  &  \\
% \quad + Masked &  &  &  &  &  &  &  &  &  \\
% \quad + MLM &  &  &  &  &  &  &  &  &  \\
% \midrule 
% 84.4 88.4 86.7 92.7
% $\text{BERT}$  & 84.3 & 88.4 &  &  & 92.7 & 86.7 &  &  &  \\
% $\text{BARNEY}_\text{BERT}$ &  &  &  &  &  &  &  &  &  \\
RoBERTa-base & 94.8 & 63.6 & 88.7 \\
\textsc{Autobot}-base & \textbf{95.0} & \textbf{66.0} & 88.7 \\
\midrule 
% $\text{BERT}$ & 86.6 & 92.3 & 91.3 & 70.4 & 93.2 & 88.0 & 60.6 & 90.0 &  -\\
% $\text{BARNEY}_\text{BERT}$ &  &  &  &  &  &  &  &  &  \\
RoBERTa-large & 96.4 & 68.0 & 91.1 \\
\textsc{Autobot}-large & \textbf{96.9} & \textbf{70.2} & 91.1 \\
\bottomrule
\end{tabular} % 87.6 92.8 91.9 78.7 94.8 90.2 63.6 91.2
\caption{%\nikos{can we fit the other tasks here too? if available (otherwise in appendix)} \ivan{I could add a col. for avg. of other tasks} \nikos{great idea! we should definitely do that} 
%\nikos{here we are not comparing with SRoberta?} \ivan{Hmmm, SRoBERTa was only intended for sentence representations and never evaluated for classification, so I didn't do that experiment}
Single-sentence GLUE classification dev.~results. Median accuracy is reported over over three random seeds. Our model improves performance on single-sentence classification tasks over both base and large RoBERTa models while maintaining their performance on the remaining multi-sentence tasks. % \ivan{Explain why we exclude other tasks}
% We exclude other GLUE task metrics as AUTOBOT achieves an identical performance as the underlying transformer in the encoder on dual sentence tasks.
% \ivan{Perhaps reduce this table to only show the single-sentence results? (MNLI, SST, CoLA) And note results unchanged for others} \nikos{good idea, let's keep the results for other non sentence similarity tasks to support the point that performance is largely maintained on the other tasks (unclear if this is the case with sentence bert).}
}


% Averages are obtained from the GLUE leaderboard.
\label{tab:glue}
\end{table}

% take the median, cite jesse paper



% \begin{table*}[ht]
% \centering
% \begin{tabular}{lcccccccccc}
% \toprule
% & \bf MNLI & \bf QNLI & \bf QQP & \bf RTE & \bf SST & \bf MRPC & \bf CoLA & \bf STS & \bf Avg \\
% \midrule 
% % \multicolumn{10}{l}{\textit{Transformer Models}}\\
% % Base &  &  &  &  &  &  &  &  &  \\
% % \quad + Masked &  &  &  &  &  &  &  &  &  \\
% % \quad + MLM &  &  &  &  &  &  &  &  &  \\
% % Large &  &  &  &  &  &  &  &  &  \\
% % \quad + Masked &  &  &  &  &  &  &  &  &  \\
% % \quad + MLM &  &  &  &  &  &  &  &  &  \\
% % \midrule 
% \multicolumn{10}{l}{\textit{Base Models}}\\
% % 84.4 88.4 86.7 92.7
% % $\text{BERT}$  & 84.3 & 88.4 &  &  & 92.7 & 86.7 &  &  &  \\
% % $\text{BARNEY}_\text{BERT}$ &  &  &  &  &  &  &  &  &  \\
% $\text{RoBERTa-base}$ & 87.6 & 92.8 & 91.9 & 78.7 & 94.8 & 90.2 & 63.6 & 91.2 & \\
% $\text{BARNEY}_{\text{RoBERTA-base}}$ & \textbf{88.0} &  & 91.9 &  & \textbf{95.0} & & \textbf{66.0} & &   \\
% \midrule 
% \multicolumn{10}{l}{\textit{Large Models}}\\
% % $\text{BERT}$ & 86.6 & 92.3 & 91.3 & 70.4 & 93.2 & 88.0 & 60.6 & 90.0 &  -\\
% % $\text{BARNEY}_\text{BERT}$ &  &  &  &  &  &  &  &  &  \\
% $\text{RoBERTa-large}$ & 90.2 & 94.7 & 92.2 & 86.6 & 96.4 & 90.9 & 68.0 & 92.4 \\
% $\text{BARNEY}_\text{RoBERTA-large}$ & 90.5 &  &  &  & 96.9 & & \textbf{70.2} &  &  \\
% \bottomrule
% \end{tabular} % 87.6 92.8 91.9 78.7 94.8 90.2 63.6 91.2
% \caption{
% Dev results on GLUE.
% % Results on GLUE. All results are based on a 24-layer architecture.
% % \bertlarge{} and \xlnetlarge{} results are from \newcite{devlin2018bert} and \newcite{yang2019xlnet}, respectively.
% % \ourmodel{} results on the development set are a median over five runs.
% % \ourmodel{} results on the test set are ensembles of \emph{single-task} models.
% \ivan{Perhaps reduce this table to only show the single-sentence results? (MNLI, SST, CoLA) And note results unchanged for others} \nikos{good idea, let's keep the results for other non sentence similarity tasks to support the point that performance is largely maintained on the other tasks (unclear if this is the case with sentence bert).}
% For RTE, STS and MRPC we finetune starting from the MNLI model instead of the baseline pretrained model.}


% % Averages are obtained from the GLUE leaderboard.
% \label{tab:roberta_glue}
% \end{table*}

% Following the pretrain-finetuning transfer learning settings

% motivation: following pretrain-finetuning transfer learning setting, check the performance of finetuning on downsteam nlu tasks. 
% goal see converge speed and results

% \textbf{Tasks and Datasets.}

% glue

% cola, mnli, mrpc, qnli, qqp, rte, sst2, stsb, wnli

% \textbf{Implementation Details.}

% similar to the continuing training setting in language modeling experiment, we load the Roberta checkpoint

% batch size is selected conditioned on the requirements of the vanilla method (bigbird may require more space )

% \textbf{Results.}
% \end{comment}
\section{Conclusions and future work}

Motivated by the connection between kernel methods and self-attention, 
we introduce Kernelized Attention, which replaces the softmax structure in self-attention with a Gaussian kernel.
We also propose Skyformer, which adapts the \nystrom method to Kernelized Attention to improve its efficiency.
We expect the new model can enjoy more stable training while inheriting the strong performance from self-attention.
Extensive experiments verify our intuitions and show that both Kernelized Attention and its \nystrom approximation variant have comparable accuracy to the original Transformer on the LRA benchmark.

Direct development of this work is the incorporation of further computation tricks in kernel methods, such as the local and global approximation for gram matrix \citep{snelson2007local} and the importance sampling in \nystrom methods \citep{DBLP:conf/nips/MuscoM17, chen2021fast, chen2021accumulations}.
Other related questions include the choice of the kernel other than the Gaussian kernel in our kernelized attention model.
It is expected that for different tasks there will be specific kernels more proper than the original self-attention.
The results in this work also shed new light on the design of the attention mechanism, which may benefit board downstream NLP tasks.





% A possible trick would be correcting the diagonal / band / block part of $\tilde A$ (from "local and global sparse Gaussian process approximation"):
% \begin{align*}
% \tilde A + \text{a certain part of}(A - \tilde{A}),
% \end{align*}
% which ensures the entries in the certain part are exact.

% Diagonal Correction on $\Tilde{\textbf{A}}$: $\Tilde{\textbf{A}} + \text{diag}(\textbf{A}-\Tilde{\textbf{A}})$.
% Here the correction can be of any low rank form, including diagonal, banded, or block-wise.


%  can be further combined with previous tricks in BigBird

%  optimize based on the properties of document (sentence-wise block)


\begin{ack}

This research is based upon work in part supported by the Office of the Director of National Intelligence (ODNI), Intelligence Advanced Research Projects Activity (IARPA), via contract No. FA8650-17-C-9116, and U.S. DARPA KAIROS Program No. FA8750-19-2-1004. This work is also in part supported by NSF grant DMS-1810831. The views and conclusions contained herein are those of the authors and should not be interpreted as necessarily representing the official policies, either expressed or implied, of DARPA, ODNI, IARPA, or the U.S. Government. The U.S. Government is authorized to reproduce and distribute reprints for governmental purposes notwithstanding any copyright annotation therein.


\end{ack}

\clearpage
\bibliographystyle{plainnat}
\bibliography{ref}
\clearpage
\appendix

\section{Validation loss}
\label{sec:dev_loss}
\begin{figure}[htbp]
\centering
\includegraphics[width=1.0\linewidth]{figures/loss37.png}
\caption{
Validation loss changes for $50k$ steps. X-axis: Training time (second). Y-axis: Cross Entropy Loss on validation set. 
}
\label{fig:devloss}
\end{figure}

Figure~\ref{fig:devloss} shows the validation loss changes with respect to training time for $50$k steps as supplementary results for the experiments in Section~\ref{sec:exp}.
In general, Skyformer converges faster and finishes $50$k steps earlier than vanilla Attention and Kernelized Attention over all tasks.
We further remark that on Text Classification, all models quickly fall into over-fitting, and thus the validation losses rise quickly.
On Pathfinder, due to the difficulty of training, in the trial shown in the figure vanilla Attention fails to reach the best long-time limit under a certain setting.



\section{Singular value decay rate}
\begin{figure}[htbp]
\centering
\includegraphics[width=0.35\linewidth]{figures/LRASVD.png}
\caption{
Singular value distribution of attention output. }
\label{fig:svd}
\end{figure}


% We propose to measure the task difficulty with the singular value decay rate in attention output. 


Figure~\ref{fig:svd} shows the singular value distribution of attention output from the second layer of a trained vanilla transformer.
Results are averaged across one random batch from the test set in each LRA task. 

The singular values decay fast and thus justify the low-rank approximation, as analyzed by \citet{DBLP:journals/corr/abs-2006-04768, dong2021attention}.
We propose to measure the task difficulty with the singular value decay rate in attention output, as higher intrinsic task difficulty forces the model to output a matrix with more large singular values.
Such matrices are considered more informative since they are harder to approximate, requiring more ranks even in the truncated SVD approximation.
With the observation in Figure~\ref{fig:svd}, we conclude that the singular values in Document Retrieval and Pathfinder tasks decay slower, and those two tasks are more difficult than Text Classification and ListOps.
% We suggest to focus on those two tasks as a metric to evaluate the performance of efficient transformers to capture long-distance dependency.




\section{Useful facts}
\label{sec:facts}

This section introduces some useful facts, which are key in the proof in the next section. 
To start with, we provide a matrix concentration inequality as follows.
% The following theorem is copied from the tutorial \citep[Theorem~1.4,~1.6]{tropp2012user} for the reader's convenience.
\begin{lemma}[Matrix Bernstein Inequality \citep{tropp2012user}]
\label{lem:intro-bernstein}
Consider a finite sequence $\{ \mtx{X}_k \}$ of independent, random, self-adjoint matrices with dimension $n$.  
Assume that each random matrix satisfies
$$
\Expect \mtx{X}_k = \mtx{0}
\quad\text{and}\quad
\| \mtx{X}_k \| \leq R
\quad\text{almost surely}.
$$
Then, for all $t \geq 0$,
$$
\Prob{ \| \sum\nolimits_k \mtx{X}_k \| \geq t }
	\leq 2n \cdot \exp\left( \frac{-t^2/2}{\sigma^2 + Rt/3} \right)
	\quad\text{where}\quad
	\sigma^2 \geq \norm{ \sum\nolimits_k \Expect \big(\mtx{X}_k^2 \big) }.
$$
\end{lemma}

For a certain $n$-by-$n$ orthogonal matrix $\mtx{H}$ ($\mtx{H} \mtx{H}^T$ is a diagonal matrix) 
and an $n$-by-$d$ uniform sub-sampling matrix $\mtx{S}$ (as defined in Definition~\ref{def:subsampling} in the main paper),
we denote the sketching matrix $\mtx{\Pi} \defeq \sqrt{n} \mtx{S}$.
We aim to show $\mtx{H} \mtx{\Pi} \mtx{\Pi}^T \mtx{H}^T$ can satisfy $(\frac12, \delta)$-MA property for $\mtx{H} \mtx{H}^T$ by the following lemma.
\begin{lemma}
\label{lem:frac12MA}
Denote the stable rank $s \defeq \frac{\|\mtx{H}\|_F^2}{\|\mtx{H}\|^2} \geq 1$, and a constant $\delta < 1/2$. 
Suppose there exists a constant $\beta \in (0, 1]$ such that $\beta \leq \frac{\|\mtx{H}\|_F^2}{n \|\mtx{H}^{(i)}\|^2}, \forall i = 1,\dots,n$,
where $\mtx{H}^{(i)}$ is the $i$-th column of $\mtx{H}$. 
There exists a constant $C_0$ that if 
\begin{align*}
d \geq C_0 \frac{s}{\beta} \log \frac{n}{\delta}, 
\end{align*}
then $\mtx{H} \mtx{\Pi} \mtx{\Pi}^T \mtx{H}^T$ satisfies $(\frac12, \delta)$-MA property for $\mtx{H} \mtx{H}^T$.
\end{lemma}

\begin{proof}
The main idea is to utilize Lemma~\ref{lem:intro-bernstein} by setting $t = \frac12 \|\mtx{H} \mtx{H}^T\| = \frac12 \|\mtx{H}\|^2$.
Specifically, we denote the matrices 
\begin{align*}
\mtx{X}_k &= \mtx{H} \mtx{\Pi}^{(i)} \left(\mtx{\Pi}^{(i)}\right)^T \mtx{H}^T - \frac1d \mtx{H} \mtx{H}^T, \quad \text{so that} \\
\sum\nolimits_k \mtx{X}_k &= \mtx{H} \mtx{\Pi} \mtx{\Pi}^T \mtx{H}^T - \mtx{H} \mtx{H}^T.
\end{align*}

% To prove the lower bound of $d$ in the lemma, 
We still need two steps to give control of $R$ and $\sigma^2$.
For $R$, we have
\begin{align*}
\|\mtx{X}_k\| &= \norm{\frac1d \sum_{i=1}^n (n z_{ki} -1) \mtx{H}^{(i)} \left(\mtx{H}^{(i)}\right)^T}
\leq \frac1d \max \left\{\max_i (n - 1) \|\mtx{H}^{(i)}\|^2, \|\mtx{H}\|^2 \right\} \\
&\leq \frac1d n \max_i \|\mtx{H}^{(i)}\|^2,
\end{align*}
where $\{z_{ki}\}_{i=1}^n$ are the indicators of whether the $i$-th column is chosen.
The first inequality of the preceding display holds due to the fact that $\mtx{H}$ is an orthogonal matrix.
Using the condition $n \leq \frac{\|\mtx{H}\|_F^2}{\beta \|\mtx{H}^{(i)}\|^2}, \forall i = 1,\dots,n$, we further have
\begin{align*}
\|\mtx{X}_k\| \leq \frac{\|\mtx{H}\|_F^2}{d \beta},
\end{align*}
and we thus set $R \defeq \frac{\|\mtx{H}\|_F^2}{d \beta}$.
On the other hand,
\begin{align*}
\Expect \mtx{X}_k^2 
= \frac{1}{d^2} \sum_{i=1}^n \Expect \left( (n z_{ki} -1)^2 \right) \|\mtx{H}^{(i)}\|^2 \mtx{H}^{(i)} \left(\mtx{H}^{(i)}\right)^T
= \frac{1}{d^2} \sum_{i=1}^n (n-1) \|\mtx{H}^{(i)}\|^2 \mtx{H}^{(i)} \left(\mtx{H}^{(i)}\right)^T.
\end{align*}
Again using the condition that $n \|\mtx{H}^{(i)}\|^2 \leq \frac{\|\mtx{H}\|_F^2}{\beta}, \forall i = 1,\dots,n$, we reach
\begin{align*}
\norm{\Expect \sum_{k=1}^d \mtx{X}_k^2} \leq \frac1d \frac{\|\mtx{H}\|_F^2}{\beta} \norm{\mtx{H} \mtx{H}^T} 
= \frac{\|\mtx{H}\|_F^2}{d \beta} \norm{\mtx{H}}^2,
\end{align*}
and set $\sigma^2 \defeq \frac{\|\mtx{H}\|_F^2}{d \beta} \norm{\mtx{H}}^2$.

Finally we plug $R$ and $\sigma^2$ into Lemma~\ref{lem:intro-bernstein} and obtain:
\begin{align*}
\Prob{ \norm{\mtx{H} \mtx{\Pi} \mtx{\Pi}^T \mtx{H}^T - \mtx{H} \mtx{H}^T} \geq \frac12 \|\mtx{H}\|^2}
	\leq 2n \cdot \exp\left( \frac{-\|\mtx{H}\|^4/8}{\frac{s \|\mtx{H}\|^4}{d \beta} + \frac{s \|\mtx{H}\|^4}{6 d \beta}} \right).
\end{align*}
To ensure the right-hand-side is smaller than $\delta$, we just need
\begin{align*}
d \geq \frac{28}{3} \frac{s}{\beta} \log \frac{2n}{\delta},
\end{align*}
which validates the lemma.
\end{proof}


\section{Proof of Theorem~\ref{thm:tilde_K} in the main paper}
\label{sec:thm_error}

\begin{proof}
The conclusion in the lemma can be divided into two parts, 
that $\tilde{\bar{\mtx{C}}} \psdle \bar{\mtx{C}}$ 
and $\bar{\mtx{C}} \psdle \tilde{\bar{\mtx{C}}} + \lambda \mtx{I}$.
To prove them we first introduce some notations and auxiliary results.
Since $\bar{\mtx{C}}$ is PSD, there exists a matrix $\mtx{B}$ satisfying $\mtx{B} \mtx{B}^T = \bar{\mtx{C}}$.
We further denote $\mtx{B}$'s SVD decomposition as $\mtx{B} = \mtx{U} \mtx{\Sigma}^{\frac12} \mtx{V}^T$ 
($\bar{\mtx{C}} = \mtx{U} \mtx{\Sigma} \mtx{U}^T$),
where both $\mtx{U}$ and $\mtx{V}$ are $2n$-by-$2n$ orthonormal matrices.
(In this section we slightly abuse the notation that $\mtx{V}$ represents the matrix of right-singular vectors, instead of the value matrix in self-attention.)
Define $\bar{\mtx{\Sigma}} \defeq \mtx{\Sigma} + \lambda \mtx{I}, \mtx{\Psi} \defeq \mtx{U} \mtx{\Sigma}^{\frac12} \bar{\mtx{\Sigma}}^{-\frac12}$, which implies $\bar{\mtx{C}} (\bar{\mtx{C}} + \lambda \mtx{I})^{-1} = \mtx{\Psi} \mtx{\Psi}^T$.
Also following the notations in the last section,
we define the $2n$-by-$d$ matrix $\mtx{\Pi} \defeq \sqrt{2n} \mtx{S}$
With those notations, $\tilde{\bar{\mtx{C}}}$ can be rewritten as $\mtx{B} \mtx{B}^T \mtx{\Pi} (\mtx{\Pi}^T \mtx{B} \mtx{B}^T \mtx{\Pi})^\dagger \mtx{\Pi} \mtx{B} \mtx{B}^T = \mtx{B} \mtx{P_\Pi} \mtx{B}^T$,
where $\mtx{P_\Pi}$ is the orthogonal projection matrix for the column space of $\mtx{B}^T \mtx{\Pi}$.
It is easy to check that $\bar{\mtx{C}} - \tilde{\bar{\mtx{C}}} = \mtx{B} (\mtx{I} - \mtx{P_\Pi}) \mtx{B}^T$.
Since $\mtx{I} - \mtx{P_\Pi}$ is an orthogonal projection matrix (which is PSD), 
we have $\bar{\mtx{C}} - \tilde{\bar{\mtx{C}}} \psdge \mtx{0}$,
which proves the first conclusion that $\tilde{\bar{\mtx{C}}} \psdle \bar{\mtx{C}}$.

For the second conclusion, we utilize the following important identity:
\begin{align*}
\mtx{B}^T \mtx{\Pi} \mtx{\Pi}^T \mtx{B} - \mtx{B}^T \mtx{B} 
&= \mtx{V} \bar{\mtx{\Sigma}}^{\frac12} \big( \bar{\mtx{\Sigma}}^{-\frac12} \mtx{V}^T (\mtx{B}^T \mtx{\Pi} \mtx{\Pi}^T \mtx{B} - \mtx{B}^T \mtx{B}) \mtx{V} \bar{\mtx{\Sigma}}^{-\frac12} \big) \bar{\mtx{\Sigma}}^{\frac12} \mtx{V}^T \\
&= \mtx{V} \bar{\mtx{\Sigma}}^{\frac12} \big( \mtx{\Psi}^T \mtx{\Pi} \mtx{\Pi}^T \mtx{\Psi} - \mtx{\Psi}^T \mtx{\Psi} \big) \bar{\mtx{\Sigma}}^{\frac12} \mtx{V}^T.
\end{align*}
For $\mtx{\Psi}$, we have that its squared Frobenius norm $\|\mtx{\Psi}\|_F^2 = d_{stat}$, and $\|\mtx{\Psi}\|^2 
= \|\bar{\mtx{C}} (\tilde{\bar{\mtx{C}}} + \lambda \mtx{I})^{-1}\| \geq 1/2$,
indicating that $\mtx{\Psi}$'s stable rank $s = \norm{\mtx{\Psi}}_F^2 / \norm{\mtx{\Psi}}^2$ is at most $2 d_{stat}$.

Taking $\varepsilon = \frac12$ and applying Lemma~\ref{lem:frac12MA}, we can conclude that with the conditions on $d$ in the theorem, 
$\mtx{\Psi}^T \mtx{\Pi} \mtx{\Pi}^T \mtx{\Psi}$ satisfies $(\frac12, \delta)$-MA property for $\mtx{\Psi}^T \mtx{\Psi}$.
Therefore it holds with probability $1-\delta$ that, 
\begin{align*}
\|\mtx{\Psi}^T \mtx{\Pi} \mtx{\Pi}^T \mtx{\Psi} - \mtx{\Psi}^T \mtx{\Psi}\| 
\leq \frac12 \|\mtx{\Psi}\|^2
\leq \frac12.
\end{align*}
From identity $\mtx{V} \bar{\mtx{\Sigma}}^{\frac12} \bar{\mtx{\Sigma}}^{\frac12} \mtx{V}^T = \mtx{B}^T \mtx{B} + \frac{\lambda}{2} \mtx{I}$, we obtain
\begin{align}
\frac12 \mtx{B}^T \mtx{B} - \frac{\lambda}{2} \mtx{I} \psdle \mtx{B}^T \mtx{\Pi} \mtx{\Pi}^T \mtx{B} 
\psdle \frac32 \mtx{B}^T \mtx{B} + \frac{\lambda}{2} \mtx{I},
\end{align}
which implies
\begin{align}
\label{eqn:BTB_bound}
\mtx{B}^T \mtx{B} \psdle 2 \mtx{B}^T \mtx{\Pi} \mtx{\Pi}^T \mtx{B} + \lambda \mtx{I}.
\end{align}

Finally, we multiply two sides of Eq.~(\ref{eqn:BTB_bound}) by $(\mtx{I} - \mtx{P_\Pi})$ to obtain
\begin{align*}
(\mtx{I} - \mtx{P_\Pi}) \mtx{B}^T \mtx{B} (\mtx{I} - \mtx{P_\Pi}) \psdle 2 \cdot \mtx{0} + \lambda (\mtx{I} - \mtx{P_\Pi}) \psdle \lambda \mtx{I},
\end{align*}
where the second inequality is due to the fact that $(\mtx{I} - \mtx{P_\Pi})$ is an orthogonal projection matrix.
The equation above implies $\|(\mtx{I} - \mtx{P_\Pi}) \mtx{B}^T \mtx{B} (\mtx{I} - \mtx{P_\Pi})\| = \|\mtx{B} (\mtx{I} - \mtx{P_\Pi}) \mtx{B}^T\| \leq \lambda$, 
which completes the proof for the second conclusion $\bar{\mtx{C}} \psdle \tilde{\bar{\mtx{C}}} + \lambda \mtx{I}$.

Based on the conclusion above, the last implication is direct:
\begin{align*}
\|\tilde{\mtx{C}} - \mtx{C}\| = \norm{(\mtx{I}, \mtx{0}) \left( \tilde{\bar{\mtx{C}}} - \bar{\mtx{C}} \right) (\mtx{0}, \mtx{I})^T}
\leq \norm{(\mtx{I}, \mtx{0})} \norm{\left( \tilde{\bar{\mtx{C}}} - \bar{\mtx{C}} \right)} \norm{(\mtx{0}, \mtx{I})^T}
\leq \lambda = \varepsilon \|\mtx{C}\|,
\end{align*}
which completes the proof.
\end{proof}


\section{Proof of Lemma~\ref{lem:iterative} in the main paper}
\label{sec:lem_iter}

\begin{proof}
As $\bar{\mtx{C}}$ is constructed based on a PSD kernel, $\bar{\mtx{C}}$ is also PSD.
Consequently $\mtx{M} = \mtx{S}^{T} \bar{\mtx{C}} \textbf{S}$ is PSD, and $\mtx{D}_M^{-1/2} (\mtx{M} + \gamma \mtx{I}) \mtx{D}_M^{-1/2}$ is positive definite, with all eigenvalues positive.
To prove the claim in the lemma we only need to show the eigenvalues of $\mtx{D}_M^{-1/2} (\mtx{M} + \gamma \mtx{I}) \mtx{D}_M^{-1/2}$ are bounded from above by $1$.
It is equivalent to prove that $\mtx{I} - \mtx{D}_M^{-1/2} (\mtx{M} + \gamma \mtx{I}) \mtx{D}_M^{-1/2}$ is PSD,
which can be induced by another statement that $\mtx{L} \defeq \mtx{D}_M - (\mtx{M} + \gamma \mtx{I})$ is PSD.

The proof of the statement above is similar to the proof of the well-known conclusion that graph Laplacian matrix is PSD.
For simplicity we denote $\mtx{W} \defeq \mtx{M} + \gamma \mtx{I}$, and given any vector $\mtx{x} \in \mb R^d$ we have
\begin{align*}
\mtx{x}^T \mtx{L} \mtx{x} &= \mtx{x}^T \mtx{D}_M \mtx{x} - \mtx{x}^T \mtx{W} \mtx{x}
= \sum_{i=1}^d (\mtx{D}_M)_{ii} \mtx{x}_i^2 - \sum_{i,j=1}^d \mtx{W}_{ij} \mtx{x}_i \mtx{x}_j \\
&= \frac12 \left( \sum_{i=1}^d (\mtx{D}_M)_{ii} \mtx{x}_i^2 - 2\sum_{i,j=1}^d \mtx{W}_{ij} \mtx{x}_i \mtx{x}_j + \sum_{j=1}^d (\mtx{D}_M)_{jj} \mtx{x}_j^2 \right) \\
&= \frac12 \sum_{i,j=1}^d \mtx{W}_{ij} (\mtx{x}_i - \mtx{x}_j)^2 \geq 0,
\end{align*}
where the last equation holds due to the fact that $(\mtx{D}_M)_{ii} = \sum_{j=1}^d \mtx{W}_{ij}$.

Combining the pieces above we can conclude that $\|\mtx{I} - \mtx{D}_M^{-1/2} (\mtx{M} + \gamma \mtx{I}) \mtx{D}_M^{-1/2}\| < 1$.
\end{proof}


\section{Additional discussions about the stability in model training}
\label{sec:exp_stability}

For our argument about stability, we mainly refer to the paper~\citep{DBLP:conf/emnlp/LiuLGCH20}, 
which identifies that the amplification of small parameter perturbations in the self-attention module is the root cause of training instability. 
We take kernelized attention as mitigation since it contains an automatic normalization. 
We have empirically used Figure~\ref{fig:devacc} and Figure~\ref{fig:devloss} in Appendix~\ref{sec:dev_loss} to support our claim. 

For further analysis we conduct a toy experiment adapting from Figure~4 in the aforementioned paper~\citep{DBLP:conf/emnlp/LiuLGCH20}.
We aim to show that in kernelized attention (and Skyformer) the output changes $f(x, W^*) - f(x, W)$ for parameter changes $W^* - W$ is smaller than in self-attention (and its approximation Nystr\"omformer). 
This concept involved is somewhat similar to condition number and below we will formalize it as ``instability score".

We show a table of the averaged ratios between the instability scores of kernelized attention (we also add Skyformer and Nystr\"omformer for reference) and self-attention to conclude our statement about stability. 
A ratio smaller than $1$ means higher stability compared to self-attention. 
We follow all the settings in Table~\ref{table:lra_acc} in the main paper except here we only update the model for 20 steps (we limit the number of steps as suggested by \citet{DBLP:conf/emnlp/LiuLGCH20} to make the results of the same step comparable among different models). 
In step $i$ for each model we compute the instability score $\tau_i = \frac{\|f(x_i, W_i) - f(x_i, W_{i-1})\|^2_F}{\|W_i - W_{i-1}\|^2_F}, i=1, \cdots, 20$, 
where $f()$ gives the embedding after two layers, $x_i$ is the $i$-th input sequence batch, $W_0$ represents the initial parameters, and $W_i$ represents the parameters after step $i$. 
In each step we compute the ratio of a certain method’s $\tau_i$ to the $\tau_i$ of self-attention, and finally average the $20$ ratios in Table~\ref{table:ratio_instability} in the appendix. 

As we can observe, both kernelized attention and Skyformer consistently have a lower instability score than self-attention, 
while the instability score of Nystr\"omformer, an approximation to self-attention, fluctuates around $1$ in all the tasks.
The results support our claim that the proposed kernelized attention can improve stability.

\begin{table}[t]
\caption{Ratios of instability score on LRA benchmark.}
\label{table:ratio_instability}
\centering
\begin{tabular}{lccccc}
\toprule
Model & Text    & ListOps & Retrieval & Pathfinder &  Image \\
\midrule
Nystr\"omformer         & 1.03 & 1.01 &	0.97 & 	0.99 & 1.02	\\
Kernelized Attention    & 0.83 & 0.77 & 0.64 & 	0.74 & 0.62	\\
Skyformer               & 0.81 & 0.79 &	0.64 & 	0.79 & 0.65 \\
\bottomrule
\end{tabular}
\end{table}


\end{document}

