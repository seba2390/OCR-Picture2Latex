              
                %% bare_jrnl.tex
%% V1.4b
%% 2015/08/26
%% by Michael Shell
%% see http://www.michaelshell.org/
%% for current contact information.
%%
%% This is a skeleton file demonstrating the use of IEEEtran.cls
%% (requires IEEEtran.cls version 1.8b or later) with an IEEE
%% journal paper.
%%
%% Support sites:
%% http://www.michaelshell.org/tex/ieeetran/
%% http://www.ctan.org/pkg/ieeetran
%% and
%% http://www.ieee.org/

%%*************************************************************************
%% Legal Notice:
%% This code is offered as-is without any warranty either expressed or
%% implied; without even the implied warranty of MERCHANTABILITY or
%% FITNESS FOR A PARTICULAR PURPOSE! 
%% User assumes all risk.
%% In no event shall the IEEE or any contributor to this code be liable for
%% any damages or losses, including, but not limited to, incidental,
%% consequential, or any other damages, resulting from the use or misuse
%% of any information contained here.
%%
%% All comments are the opinions of their respective authors and are not
%% necessarily endorsed by the IEEE.
%%
%% This work is distributed under the LaTeX Project Public License (LPPL)
%% ( http://www.latex-project.org/ ) version 1.3, and may be freely used,
%% distributed and modified. A copy of the LPPL, version 1.3, is included
%% in the base LaTeX documentation of all distributions of LaTeX released
%% 2003/12/01 or later.
%% Retain all contribution notices and credits.
%% ** Modified files should be clearly indicated as such, including  **
%% ** renaming them and changing author support contact information. **
%%*************************************************************************


% *** Authors should verify (and, if needed, correct) their LaTeX system  ***
% *** with the testflow diagnostic prior to trusting their LaTeX platform ***
% *** with production work. The IEEE's font choices and paper sizes can   ***
% *** trigger bugs that do not appear when using other class files.       ***                          ***
% The testflow support page is at:
% http://www.michaelshell.org/tex/testflow/


% Please refer to your journal's instructions for other
% options that should be set.
\documentclass[journal]{IEEEtran}
%\documentclass[journal,transmag]{IEEEtran}
%\documentclass[journal,onecolumn]{IEEEtran}
%\documentclass[journal,12pt,onecolumn,draftclsnofoot,]{IEEEtran}
%\documentclass[12pt,peerreview,onecolumn]{IEEEtran}
%
% If IEEEtran.cls has not been installed into the LaTeX system files,
% manually specify the path to it like:
% \documentclass[journal]{../sty/IEEEtran}





% Some very useful LaTeX packages include:
% (uncomment the ones you want to load)


% *** MISC UTILITY PACKAGES ***
%
%\usepackage{ifpdf}
% Heiko Oberdiek's ifpdf.sty is very useful if you need conditional
% compilation based on whether the output is pdf or dvi.
% usage:
% \ifpdf
%   % pdf code
% \else
%   % dvi code
% \fi
% The latest version of ifpdf.sty can be obtained from:
% http://www.ctan.org/pkg/ifpdf
% Also, note that IEEEtran.cls V1.7 and later provides a builtin
% \ifCLASSINFOpdf conditional that works the same way.
% When switching from latex to pdflatex and vice-versa, the compiler may
% have to be run twice to clear warning/error messages.






% *** CITATION PACKAGES ***
%
%\usepackage{cite}
% cite.sty was written by Donald Arseneau
% V1.6 and later of IEEEtran pre-defines the format of the cite.sty package
% \cite{} output to follow that of the IEEE. Loading the cite package will
% result in citation numbers being automatically sorted and properly
% "compressed/ranged". e.g., [1], [9], [2], [7], [5], [6] without using
% cite.sty will become [1], [2], [5]--[7], [9] using cite.sty. cite.sty's
% \cite will automatically add leading space, if needed. Use cite.sty's
% noadjust option (cite.sty V3.8 and later) if you want to turn this off
% such as if a citation ever needs to be enclosed in parenthesis.
% cite.sty is already installed on most LaTeX systems. Be sure and use
% version 5.0 (2009-03-20) and later if using hyperref.sty.
% The latest version can be obtained at:
% http://www.ctan.org/pkg/cite
% The documentation is contained in the cite.sty file itself.


\usepackage{cite}
\usepackage{amsmath,amssymb,amsfonts}
\usepackage{algorithmic}
\usepackage{graphicx}
\usepackage{textcomp}
\usepackage{xcolor}
%\usepackage{footnote}

\usepackage{float}
\usepackage{mathtools}
\usepackage{algorithm}
\usepackage{amsthm}
\usepackage{tablefootnote}
\usepackage{threeparttable}
\usepackage{enumitem}
\usepackage{subcaption}

\usepackage{multirow}

\newtheorem{theo}{Theorem}
\newtheorem{prop}{Proposition}
\newtheorem{property}{Property}
\newtheorem{claim}{Claim}
\newtheorem{defin}{Definition}
\newtheorem{lem}{Lemma}
\newcommand{\Dal}{\Delta\alpha}
\newcommand{\eqdef}{\stackrel{\triangle}{=}}
\newcommand{\be}{\begin{equation}}
\newcommand{\ee}{\end{equation}}
\newcommand{\ba}{\begin{array}}
	\newcommand{\ea}{\end{array}}
\newcommand{\nid}{\noindent}
\newcommand{\non}{\nonumber}
\newcommand{\psip}{\dot{\psi}}
\newcommand{\al}{\alpha}
\newcommand{\sg}{\sigma}
\newcommand{\Ld}{\Lambda}
\newcommand{\ld}{\lambda}
\newcommand{\Dt}{\Delta}
\newcommand{\bt}{\beta}
\newcommand{\dv}{\Delta \nu}
\newcommand{\Eb}{\mbox{E}}
\newcommand{\eb}{\mbox{e}}
\newcommand{\frat}[2]{\frac{\st#1}{\st#2}}

\newcommand{\fraz}[2]{\frac{\tx#1}{\tx#2}}
\newcommand{\esp}[1]{\exp\left(#1\right)}
\newcommand{\e}{{\rm e}}
\newcommand{\lf}{\left}
\newcommand{\rt}{\right}
\newcommand{\inty}{\int_{-\infty}^{\infty}}
\newcommand{\intt}{\int_0^t}
\newcommand{\inte}{2\pi\intt e(u) \, du}
\newcommand{\linf}{{\st n=\mid k\mid \atop n+k{\rm \ pari}}}
\newcommand{\p}{{\bf p}}
\newcommand{\br}{\mathbf{r}}
\newcommand{\bS}{\mathbf{S}}
\newcommand{\bG}{\mathbf{G}}

\newcommand{\sss}{\scriptscriptstyle}
\newcommand{\m}{\hspace{-.05cm}}
\renewcommand{\thesubsubsection}{\thesubsection.\arabic{subsubsection}}
\newtheorem{theorem}{Theorem}
\newtheorem{lemma}{Lemma}
\newtheorem{corollary}{Corollary}
\newtheorem{Proposition}{Proposition}
\theoremstyle{definition}
\newtheorem{definition}{Definition}[section]
%\useRomanappendicesfalse
%\flushbottom
%\textheight9.0in
%\textwidth6.8in
%\topmargin-0.6in
%\headheight0.25in
%\oddsidemargin-0.0in
%\parindent.3in
%\parskip.1in
\setcounter{secnumdepth}{3} \setcounter{tocdepth}{3}
\newcommand\floor[1]{\lfloor#1\rfloor}
\newcommand\ceil[1]{\lceil#1\rceil}
\newcommand{\argmin}{\operatornamewithlimits{argmin}}
\newcommand{\argmax}{\operatornamewithlimits{argmax}}
\newcommand\inlineeqno{\stepcounter{equation}\ (\theequation)}


%\IEEEoverridecommandlockouts
\DeclareMathOperator{\sgn}{sgn}

\renewcommand*{\thefootnote}{\fnsymbol{footnote}}


% *** GRAPHICS RELATED PACKAGES ***
%
\ifCLASSINFOpdf
  % \usepackage[pdftex]{graphicx}
  % declare the path(s) where your graphic files are
  % \graphicspath{{../pdf/}{../jpeg/}}
  % and their extensions so you won't have to specify these with
  % every instance of \includegraphics
  % \DeclareGraphicsExtensions{.pdf,.jpeg,.png}
\else
  % or other class option (dvipsone, dvipdf, if not using dvips). graphicx
  % will default to the driver specified in the system graphics.cfg if no
  % driver is specified.
  % \usepackage[dvips]{graphicx}
  % declare the path(s) where your graphic files are
  % \graphicspath{{../eps/}}
  % and their extensions so you won't have to specify these with
  % every instance of \includegraphics
  % \DeclareGraphicsExtensions{.eps}
\fi
% graphicx was written by David Carlisle and Sebastian Rahtz. It is
% required if you want graphics, photos, etc. graphicx.sty is already
% installed on most LaTeX systems. The latest version and documentation
% can be obtained at: 
% http://www.ctan.org/pkg/graphicx
% Another good source of documentation is "Using Imported Graphics in
% LaTeX2e" by Keith Reckdahl which can be found at:
% http://www.ctan.org/pkg/epslatex
%
% latex, and pdflatex in dvi mode, support graphics in encapsulated
% postscript (.eps) format. pdflatex in pdf mode supports graphics
% in .pdf, .jpeg, .png and .mps (metapost) formats. Users should ensure
% that all non-photo figures use a vector format (.eps, .pdf, .mps) and
% not a bitmapped formats (.jpeg, .png). The IEEE frowns on bitmapped formats
% which can result in "jaggedy"/blurry rendering of lines and letters as
% well as large increases in file sizes.
%
% You can find documentation about the pdfTeX application at:
% http://www.tug.org/applications/pdftex





% *** MATH PACKAGES ***
%
%\usepackage{amsmath}
% A popular package from the American Mathematical Society that provides
% many useful and powerful commands for dealing with mathematics.
%
% Note that the amsmath package sets \interdisplaylinepenalty to 10000
% thus preventing page breaks from occurring within multiline equations. Use:
%\interdisplaylinepenalty=2500
% after loading amsmath to restore such page breaks as IEEEtran.cls normally
% does. amsmath.sty is already installed on most LaTeX systems. The latest
% version and documentation can be obtained at:
% http://www.ctan.org/pkg/amsmath





% *** SPECIALIZED LIST PACKAGES ***
%
%\usepackage{algorithmic}
% algorithmic.sty was written by Peter Williams and Rogerio Brito.
% This package provides an algorithmic environment fo describing algorithms.
% You can use the algorithmic environment in-text or within a figure
% environment to provide for a floating algorithm. Do NOT use the algorithm
% floating environment provided by algorithm.sty (by the same authors) or
% algorithm2e.sty (by Christophe Fiorio) as the IEEE does not use dedicated
% algorithm float types and packages that provide these will not provide
% correct IEEE style captions. The latest version and documentation of
% algorithmic.sty can be obtained at:
% http://www.ctan.org/pkg/algorithms
% Also of interest may be the (relatively newer and more customizable)
% algorithmicx.sty package by Szasz Janos:
% http://www.ctan.org/pkg/algorithmicx




% *** ALIGNMENT PACKAGES ***
%
%\usepackage{array}
% Frank Mittelbach's and David Carlisle's array.sty patches and improves
% the standard LaTeX2e array and tabular environments to provide better
% appearance and additional user controls. As the default LaTeX2e table
% generation code is lacking to the point of almost being broken with
% respect to the quality of the end results, all users are strongly
% advised to use an enhanced (at the very least that provided by array.sty)
% set of table tools. array.sty is already installed on most systems. The
% latest version and documentation can be obtained at:
% http://www.ctan.org/pkg/array


% IEEEtran contains the IEEEeqnarray family of commands that can be used to
% generate multiline equations as well as matrices, tables, etc., of high
% quality.




% *** SUBFIGURE PACKAGES ***
%\ifCLASSOPTIONcompsoc
%  \usepackage[caption=false,font=normalsize,labelfont=sf,textfont=sf]{subfig}
%\else
%  \usepackage[caption=false,font=footnotesize]{subfig}
%\fi
% subfig.sty, written by Steven Douglas Cochran, is the modern replacement
% for subfigure.sty, the latter of which is no longer maintained and is
% incompatible with some LaTeX packages including fixltx2e. However,
% subfig.sty requires and automatically loads Axel Sommerfeldt's caption.sty
% which will override IEEEtran.cls' handling of captions and this will result
% in non-IEEE style figure/table captions. To prevent this problem, be sure
% and invoke subfig.sty's "caption=false" package option (available since
% subfig.sty version 1.3, 2005/06/28) as this is will preserve IEEEtran.cls
% handling of captions.
% Note that the Computer Society format requires a larger sans serif font
% than the serif footnote size font used in traditional IEEE formatting
% and thus the need to invoke different subfig.sty package options depending
% on whether compsoc mode has been enabled.
%
% The latest version and documentation of subfig.sty can be obtained at:
% http://www.ctan.org/pkg/subfig




% *** FLOAT PACKAGES ***
%
%\usepackage{fixltx2e}
% fixltx2e, the successor to the earlier fix2col.sty, was written by
% Frank Mittelbach and David Carlisle. This package corrects a few problems
% in the LaTeX2e kernel, the most notable of which is that in current
% LaTeX2e releases, the ordering of single and double column floats is not
% guaranteed to be preserved. Thus, an unpatched LaTeX2e can allow a
% single column figure to be placed prior to an earlier double column
% figure.
% Be aware that LaTeX2e kernels dated 2015 and later have fixltx2e.sty's
% corrections already built into the system in which case a warning will
% be issued if an attempt is made to load fixltx2e.sty as it is no longer
% needed.
% The latest version and documentation can be found at:
% http://www.ctan.org/pkg/fixltx2e


%\usepackage{stfloats}
% stfloats.sty was written by Sigitas Tolusis. This package gives LaTeX2e
% the ability to do double column floats at the bottom of the page as well
% as the top. (e.g., "\begin{figure*}[!b]" is not normally possible in
% LaTeX2e). It also provides a command:
%\fnbelowfloat
% to enable the placement of footnotes below bottom floats (the standard
% LaTeX2e kernel puts them above bottom floats). This is an invasive package
% which rewrites many portions of the LaTeX2e float routines. It may not work
% with other packages that modify the LaTeX2e float routines. The latest
% version and documentation can be obtained at:
% http://www.ctan.org/pkg/stfloats
% Do not use the stfloats baselinefloat ability as the IEEE does not allow
% \baselineskip to stretch. Authors submitting work to the IEEE should note
% that the IEEE rarely uses double column equations and that authors should try
% to avoid such use. Do not be tempted to use the cuted.sty or midfloat.sty
% packages (also by Sigitas Tolusis) as the IEEE does not format its papers in
% such ways.
% Do not attempt to use stfloats with fixltx2e as they are incompatible.
% Instead, use Morten Hogholm'a dblfloatfix which combines the features
% of both fixltx2e and stfloats:
%
% \usepackage{dblfloatfix}
% The latest version can be found at:
% http://www.ctan.org/pkg/dblfloatfix




%\ifCLASSOPTIONcaptionsoff
%  \usepackage[nomarkers]{endfloat}
% \let\MYoriglatexcaption\caption
% \renewcommand{\caption}[2][\relax]{\MYoriglatexcaption[#2]{#2}}
%\fi
% endfloat.sty was written by James Darrell McCauley, Jeff Goldberg and 
% Axel Sommerfeldt. This package may be useful when used in conjunction with 
% IEEEtran.cls'  captionsoff option. Some IEEE journals/societies require that
% submissions have lists of figures/tables at the end of the paper and that
% figures/tables without any captions are placed on a page by themselves at
% the end of the document. If needed, the draftcls IEEEtran class option or
% \CLASSINPUTbaselinestretch interface can be used to increase the line
% spacing as well. Be sure and use the nomarkers option of endfloat to
% prevent endfloat from "marking" where the figures would have been placed
% in the text. The two hack lines of code above are a slight modification of
% that suggested by in the endfloat docs (section 8.4.1) to ensure that
% the full captions always appear in the list of figures/tables - even if
% the user used the short optional argument of \caption[]{}.
% IEEE papers do not typically make use of \caption[]'s optional argument,
% so this should not be an issue. A similar trick can be used to disable
% captions of packages such as subfig.sty that lack options to turn off
% the subcaptions:
% For subfig.sty:
% \let\MYorigsubfloat\subfloat
% \renewcommand{\subfloat}[2][\relax]{\MYorigsubfloat[]{#2}}
% However, the above trick will not work if both optional arguments of
% the \subfloat command are used. Furthermore, there needs to be a
% description of each subfigure *somewhere* and endfloat does not add
% subfigure captions to its list of figures. Thus, the best approach is to
% avoid the use of subfigure captions (many IEEE journals avoid them anyway)
% and instead reference/explain all the subfigures within the main caption.
% The latest version of endfloat.sty and its documentation can obtained at:
% http://www.ctan.org/pkg/endfloat
%
% The IEEEtran \ifCLASSOPTIONcaptionsoff conditional can also be used
% later in the document, say, to conditionally put the References on a 
% page by themselves.




% *** PDF, URL AND HYPERLINK PACKAGES ***
%
%\usepackage{url}
% url.sty was written by Donald Arseneau. It provides better support for
% handling and breaking URLs. url.sty is already installed on most LaTeX
% systems. The latest version and documentation can be obtained at:
% http://www.ctan.org/pkg/url
% Basically, \url{my_url_here}.




% *** Do not adjust lengths that control margins, column widths, etc. ***
% *** Do not use packages that alter fonts (such as pslatex).         ***
% There should be no need to do such things with IEEEtran.cls V1.6 and later.
% (Unless specifically asked to do so by the journal or conference you plan
% to submit to, of course. )


% correct bad hyphenation here
\hyphenation{op-tical net-works semi-conduc-tor}


\begin{document}
	
	\title {NOMA Computation Over Multi-Access
		Channels for Multimodal Sensing}
	
	\author{Michel~Kulhandjian,~\IEEEmembership{ Senior Member,~IEEE,
} Gunes~Karabulut~Kurt,~\IEEEmembership{Senior Member,~IEEE,
} Hovannes~Kulhandjian,~\IEEEmembership{Senior Member,~IEEE,
} Halim~Yanikomeroglu,~\IEEEmembership{Fellow,~IEEE,
} and Claude~D'Amours,~\IEEEmembership{Member,~IEEE
}
      % <-this % stops a space
%\thanks{M. Shell was with the Department
%of Electrical and Computer Engineering, Georgia Institute of Technology, Atlanta,
%GA, 30332 USA e-mail: (see http://www.michaelshell.org/contact.html).}% <-this % stops a space
%\thanks{J. Doe and J. Doe are with Anonymous University.}% <-this % stops a space
%\thanks{Authors' affiliation}
\thanks{M. Kulhandjian and C. D'Amours are with the School of Electrical Engineering, \& Computer Science, University of Ottawa, Ottawa, Canada, e-mail: mkk6@buffalo.edu, cdamours@uottawa.ca.}% <-this % stops a space
\thanks{G. Karabulut Kurt is with the Department of Electrical Engineering,  Polytechnique Montr\'eal, Montr\'eal, Canada, e-mail: \mbox{gunes.kurt@polymtl.ca}.}
\thanks{H. Kulhandjian is with the Department of Electrical \& Computer Engineering, California State University, Fresno, U.S.A., e-mail: \mbox{hkulhandjian@csufresno.edu}.}
\thanks{H. Yanikomeroglu is with the Department of Systems \& Computer Engineering, Carleton University, Ottawa, Canada, e-mail: halim@sce.carleton.ca.}

}

	\maketitle
	
	\begin{abstract}
	An improved mean squared error (MSE) minimization solution based on eigenvector decomposition approach is conceived for wideband non-orthogonal multiple-access based computation over multi-access channel (NOMA-CoMAC) framework. This work aims at further developing NOMA-CoMAC for next-generation multimodal sensor networks, where a multimodal sensor monitors several environmental parameters such as temperature, pollution, humidity, or pressure. We demonstrate that our proposed scheme achieves an MSE value approximately $0.7$ lower at $E_b/N_o = 1$ dB in comparison to that for the average sum-channel based method. Moreover, the MSE performance gain of our proposed solution increases even more for larger values of subcarriers and sensor nodes due to the benefit of the diversity gain. This, in return, suggests that our proposed scheme is eminently suitable for multimodal sensor networks.
	\end{abstract}
	
	\begin{IEEEkeywords}
		Non-orthogonal multiple-access (NOMA), computation over multi-access channels (CoMAC).
	\end{IEEEkeywords}
	
	\section{{Introduction}}
	\renewcommand*{\thefootnote}{\arabic{footnote}}


%1.) IoT and Scalability
%2.) CoMP 
%3.) CoMP and MSE
%4.) NOMA-CoMP for  - wideband extension for improved rates- spectral efficiency and computation + sub-functions
%5.) Main Contributions of NOMA-COMP. The corresponding gap in terms of MSE 
%6.) Our contributions
%7.) Organization


Internet of Things (IoT) networks are evolving towards a wide range of applications, varying from e-health, autonomous transmission systems and smart factories with ever increasing data rate requirements and reduced latency \cite{ozgun}. Their energy efficiency also need to be high to extend the battery lifetimes as much as possible. To further complicate the design problem, the expanding number of applications introduce an ever-increasing number of devices that need to be serviced with the tight operational challenges. Unfortunately, the conventional multiple access techniques, such as time-division multiple access (TDMA), frequency  division  multiple  access (FDMA)/orthogonal FDMA (OFDMA) do not offer such scalability \cite{goldsmith}. 

A promising approach is to exploit the superposition property of the wireless multiple access channel to perform some of the functionalities associated with data collection from the sensor nodes of the IoT networks over the air, while transmitting simultaneously.  This approach, referred to as computation over multi-access channels (CoMAC), introduced in \cite{nazer}, can realize a desired function of the distributed data over the wireless channel. Its extension to practically relevant systems is later introduced in \cite{goldenbaum}, and further developed in \cite{goldenbaum2}. In CoMAC,  due to the simultaneous transmission of the sensor nodes, the transmission times are scalable. Yet due to the narrow transmission bandwidth of the classical CoMAC approach introduces limited performance improvement in terms of the  spectral efficiency. To improve the spectral efficiency and overall IoT network throughput, a wideband CoMAC is proposed in \cite{wu2020noma}, integrated with the power domain non-orthogonal multiple-access (NOMA) technique. By making use of NOMA-based computation over multi-access channel (NOMA-CoMAC) technique, the authors show that the computation rate can be improved. Yet, the optimization of NOMA-CoMAC is not considered in the literature in terms of obtaining the minimum mean square error (MMSE). It is widely known that the MMSE performance can be improved by optimizing the transmission, as shown for narrowband CoMAC in \cite{semiha}. However, a wideband-CoMAC design approach to obtain the MMSE solution has not been explored to the best of our knowledge.

To address this gap in the literature, in this work we introduce a mean squared error (MSE) minimization based optimization problem of NOMA-CoMAC. Our contributions are summarized as follows:
      \vspace{-0.0cm}
      \begin{enumerate}[label=(\arabic*)]
%\begin{itemize}
\item  We develop a MSE optimization criterion for the wideband NOMA-CoMAC framework.
\item We propose an eigenvector-based solution to the MSE optimization problem. 
\item We show through simulation studies that our proposed scheme achieves around $0.7$ lower at $E_b/N_o = 1$ dB in terms of MSE performance compared to average sum-channel based method.
%\end{itemize}
\end{enumerate}

	The rest of the paper is organized as follows. In Section \ref{SystemModel}, we discuss the NOMA-CoMAC framework, the formulation of MSE optimization problem and our proposed solution approach. After illustrating simulation results in Section \ref{simulations}, main conclusions are drawn in Section \ref{conclusion}.
	
	The following notations are used in this paper. All boldface lower
	case letters indicate column vectors and upper case letters indicate
	matrices, $()^T$ denotes the transpose operation, $()^H$ represents the conjugate transpose operation and $\mathbb{E} \{ \cdot \}$ denotes the expected value.
%	

%	The following notations are used in this paper. All boldface lower
%	case letters indicate column vectors and upper case letters indicate
%	matrices, $()^T$ denotes transpose operation, $\mathsf{sgn}$ denotes the sign function, $| . |$ is the scalar magnitude, $|| . ||$ is vector norm and $\mathbb{E} \{ \cdot \}$ denotes expected value.

	\begin{figure*}
		\centering
		\includegraphics[width=6.0 in]{SystemModel04.eps}
		\centering \caption{System model CoMAC via NOMA network.} \label{CoMACSystem01}
		\vspace{-0.2cm}
	\end{figure*}
	
	\section{System model}
	\label{SystemModel}
	 In this paper, we consider a fusion technique of the multimodal sensing that aims to model context from different modalities effectively by entailing the combination of the heterogeneous sensors. The proposed multimodal sensing results in achieving improved accuracy and more specific inferences than could be achieved by the use of a single sensor alone \cite{Castanedo2013}. 
	
	Based on the multimodal sensing benefits, we study a wireless sensor network consisting of $K$ multimodal sensors and a single access point (AP).	At each node multimodal sensors record the values of $P$ heterogeneous time-varying parameters of the environment, e.g., temperature, pollution, humidity, or pressure. The measurement vector of the $k$-th sensor node constitutes $P$ sample values and is denoted by $\mathbf{s}_k = [s_{k,1}, s_{k,2}, \dots, s_{k, P}]^T \in \mathbb{R}^{P \times 1}$, where $s_{k,p}$ is the measurement of the parameter $p$ at the $k$-th sensor. Rather than accumulating the multimodal data set, the AP aims at computing $P$ functions of $P$ corresponding measuring data types from $K$ sensor nodes, denoted by $\{h_p (s_{1,p}, s_{2,p}, \dots, s_{K, p}) \}_{p=1}^P$. A class of nomograpic functions of the distributed data can be carried out quite efficiently with the aid of CoMAC. 
	\begin{definition}
	The function $h_p (s_{1,p}, s_{2,p}, \dots, s_{K, p}) $ is defined as nomographic, if there exist $K$ preprocessing functions $g_{k,p}(\cdot)$ and a postprocessing function $f_p(\cdot)$ such that it can be represented in the form \vspace{-0.2cm}
	\begin{equation}
	    h_p (s_{1,p}, s_{2,p}, \dots, s_{K, p}) = f_p\left (\sum_{k=1}^K g_{k,p}(s_{k,p})\right ).
	    \label{nomographicF}
	\end{equation}
		\end{definition}
		\begin{center}
		\vspace{-0.2cm}

\end{center}
		By exploiting the fact that wireless sensor networks normally aim to obtain a function value of sensor readings (e.g., arithmetic mean, geometric mean, etc.) instead of requiring all readings from the sensors, CoMAC framework becomes suitable for such computations. Motivated by this fact, we propose a multimodal sensor network system for future IoT networks based on CoMAC scheme over the NOMA channels, which is portrayed in Fig. \ref{CoMACSystem01}. Explicitly, the readings at each sensor nodes are preprocessed by specified functions $g_k(\cdot) =\{ g_{k,p}(\cdot)\}$, where $g_{k,p}(\cdot)$ operates on $s_{k,p}$ and $g_k(\mathbf{s}_k) = [g_{k,1}(s_{k,1}),g_{k,1}(s_{k,1}),\dots, g_{k,P}(s_{k,P})]^T $. In practice, to be more resilient against noise, we encode the resultant preprocessed vectors $g_k(\mathbf{s}_k)$ of length $P$ by the nested lattice codes to obtain $\mathbf{x}'_k =[x'_{k,1}, x'_{k,2},\dots, x'_{k,n}]^T  \in \Lambda_n \subset \mathbb{R}^{n\times 1}$  \cite{Gaspar2011} and denote $\mathbf{x}'[p] = [x'_{1,p}, x'_{2,p},\dots, x'_{K,p}]^T$, as shown in Fig. \ref{CoMACSystem01}. We employ a filter $\mathbf{B}_k$ on $\mathbf{x}'_k$ at each sensor node with the objective of minimizing sum mean-squared error of computed functions. Hence, each sensor node $k$ transmits $\mathbf{x}_k$ through NOMA channel, as shown in Fig. \ref{CoMACSystem01}. The main objective of the AP characterizes in decoding the received vector $\mathbf{y} =[y_{1}, y_{2},\dots, y_{n}]^T$ into $P$ desired functions (\ref{nomographicF}). In order to discuss our proposed optimization technique based on MMSE criterion, we first present a transmitter model over the NOMA channel, as discussed below. 
		\vspace{-0.4cm}
		
			\begin{center}
		\begin{figure*}
			\centering
			\includegraphics[width=5.5 in]{WideBandNOMAC04.eps}
			\centering \caption{Framework of wideband CoMAC.} \label{CoMAC01}
			\vspace{-0.2cm}
		\end{figure*}
	\end{center}
	
	\vspace{-0.4cm}
	
\subsection{NOMA Scheme}
We consider a wideband NOMA scheme with $N$ subcarrier over $T_s$ OFDM symbols for transmitting $\mathbf{x}_k$ at each sensor nodes. Due to fact that NOMA spreading waveforms are sparse only part of the $K$ sensor nodes participate in the computation at each subcarrier. Thus, the desired functions $f_p(\cdot)$ in (\ref{nomographicF}) that is composed of all $K$ sensor nodes can be broken down into subfunctions, as detailed in \cite{wu2020noma}. The subfunction is only part of the desired function, which is computed by a subset of $K$ sensor nodes.
Each subfunctions considers only $M$ chosen sensor nodes as a distinct subset of all $K$ nodes. Therefore, the desired function is split into $B = \frac{K}{M}$ parts. In each subcarrier, $L$ subfunctions are chosen such that $L = \frac{B}{D}$, where $D\in \mathbb{N}$. The desired functions are reconstructed by these subfunctions at the AP. Then, the $m$-th received OFDM symbol at AP can be formulated as \cite{wu2020noma}
\iffalse As shown in Fig. \ref{CoMAC01}, the node $i$ draws data from the corresponding random source $S_i$ and obtains a data vector $\mathbf{s}_i$ with length of $T_d$. Then, the node $i$ encodes the data vector $\mathbf{s}_i$ into transmitted vector $\mathbf{x}_i$ with the pre-processing encoding function $g_i(\cdot)$. In this case we can refer encoding $g_i(\cdot)$ as a channel encoder e.g., $\mathbf{x}_i = g_i(\mathbf{s}_i)$ or in case of sensors the encoder $g_i(\cdot) =\{g_{i,t}(\cdot)\}$ for $1\leq t \leq T_d$, e.g., $\mathbf{x}_i = [g_{i,1}(s_{i,1}), \dots, g_{i,T_d}(s_{i,T_d})]^T$. \fi   
\begin{equation}
\label{mainSystem}
\mathbf{Y}[m] = \sum_{l=1}^L \sum_{k =1}^K \mathbf{V}_k^l[m] \mathbf{X}_k^l[m] \mathbf{H}_k[m] +\mathbf{W}[m],
\end{equation}
\noindent where $m \in [1:T_s]$, $T_s = \frac{n}{N}$, $N$ is the number of subcarriers and $T_s$ is the number of OFDM symbols. The power allocation matrix of the $k$-th sensor node is denoted as $\mathbf{V}_k^l[m] = \mathsf{diag}\{ v_{k,1}^l[m], \dots, v_{k,N}^l[m]\}$, whose diagonal element is the power allocated to compute the $l$-th function at each subcarrier, $\mathbf{X}_k^l[m] = \mathsf{diag}\{ x_{k,1}^l[m], \dots, x_{k,N}^l[m]\}$ is the transmitted diagonal matrix of the $k$-th sensor node to compute the $l$-th function, a diagonal matrix $\mathbf{H}_k[m] = \mathsf{diag}\{ h_{k,1}[m], \dots, h_{k,N}[m]\}$ is the channel matrix in which the diagonal elements are the channel response of each subcarrier for node $k$ and the diagonal element of $\mathbf{W}[m]$ is identically and independently distributed (i.i.d.) complex Gaussian random noise. Due to linearity and diagonal matrix structure, we re-write (\ref{mainSystem}) as 
\begin{equation}
\label{mainSystema}
\mathbf{Y}[m] = \sum_{k =1}^K \sum_{l=1}^L  \mathbf{V}_k^l[m] \mathbf{X}_k^l[m] \mathbf{H}_k[m] +\mathbf{W}[m].
\end{equation}
 We define the combined matrix as follows:
\begin{equation}
\label{Xi}
\mathbf{X}_k[m] \eqdef \sum_{l=1}^L  \mathbf{V}_k^l[m] \mathbf{X}_k^l[m],
\end{equation}
\noindent where $\mathbf{X}_k[m] \!=\! \mathsf{diag}\{ x_{k,j_1}[m]v_{k,j_1}[m], \dots, x_{k,j_T}[m]v_{k,j_T}[m]\}$, $j_t \in \{1, 2, \dots N\}$, $t \in \{1,2,\dots, T\}$, $j_{t_a}\neq j_{t_b} $, and $T = LM$. The $T$ chosen sensor nodes corresponds to the nodes with the largest channel gains, e.g., $|h_{j_1}|\geq |h_{j_2}|\geq \dots \geq |h_{j_T}|$. We substitute (\ref{Xi}) into (\ref{mainSystema}) to obtain
\begin{equation}
\label{mainSystemb}
\mathbf{Y}[m] = \sum_{k =1}^K  \mathbf{X}_k[m] \mathbf{H}_k[m] +\mathbf{W}[m].
\end{equation}
Since $\mathbf{X}_k[m]$ and $\mathbf{H}_k[m]$ are diagonal matrices for $1 \leq k \leq K$, (\ref{mainSystemb}) can be expressed equivalently as
\begin{equation}
\label{mainSystemc}
\mathbf{Y}[m] = \sum_{k =1}^K   \mathbf{H}_k[m] \mathbf{X}_k[m] +\mathbf{W}[m].
\end{equation}
For ease of transmission-power control and without loss of generality, the $m$-th OFDM symbols are assumed to be normalized to have unit variance, i.e., $\mathbb{E} \{\mathbf{X}_k[m] \mathbf{X}_k[m]^H\} =\mathbf{I}_N$ for $\forall k$ and $\forall m$, where $\mathbf{I}_N$ denotes the identity matrix of size $N$ by $N$. Ideally, we would like to receive $\mathbf{X}[m]$ as one-to-one mapping expressed as\vspace{-0.2cm}
\begin{equation}
\label{Xm}
\mathbf{X}[m] = \sum_{k =1}^K   \mathbf{X}_k[m].
\end{equation}\vspace{-0.0cm}
The proposed optimization formulation is discussed in the next section. 

\vspace{-0.1cm}
	\subsection{MMSE Filtering}
We consider the joint optimization of transmit and receive filtering under the MMSE criterion with the transmission power constraints. Let $\mathbf{A}[m] \in \mathbb{C}^{N \times N}$ denote the receiver MMSE filtering matrix for $m$-th OFDM symbol at the AP and $\mathbf{B}_k[m] \in \mathbb{C}^{N \times N}$ the transmit MMSE filtering matrix at sensor node $k$ for $m$-th OFDM symbol. Let the $m$-th OFDM symbol combined matrix notation in (\ref{Xi}) be  defined now as $\mathbf{X}'_k[m]$, since the $k$-th sensor node will apply MMSE filtering on $\mathbf{x}'_k$ instead of the $\mathbf{x}_k$, as in (\ref{mainSystem}). Hence, the $m$-th OFDM symbol received at AP can be expressed as
 \begin{equation}
 \label{ReceivedFusion}
 \mathbf{\hat{X}}[m] = \mathbf{A}[m]^H\sum_{k =1}^K   \mathbf{H}_k[m] \mathbf{B}_k[m] \mathbf{X}'_k[m] +\mathbf{A}[m]^H\mathbf{W}[m].
 \end{equation} \vspace{-0.1cm}
 
 The distortion error between estimated $\mathbf{\hat{X}}[m]$ and $\mathbf{{X}}[m]$, which quantifies the over-the-air computation performance can be measured by MSE defined as follows:
  \begin{equation}
 \label{MSE}
 \mathsf{MSE}(\mathbf{\hat{X}}[m], \mathbf{{X}}[m])\! =\! \mathbb{E}\left\{\mathsf{tr}(\mathbf{\hat{X}}[m]\!-\! \mathbf{{X}}[m])(\mathbf{\hat{X}}[m]\!-\! \mathbf{{X}}[m])^H\right\},
 \end{equation}
 \noindent where $\mathsf{tr}(\cdot)$ denotes the sum of elements on the main diagonal of the square matrix. For the sake of simplicity, we will drop the $m$ notation from our formulations but readers should bear in mind that the formulation contains $m$, which refers to $m$-th OFDM symbol. Substituting (\ref{ReceivedFusion}) and (\ref{Xm}) into (\ref{MSE}), the MSE can be explicitly written as a function of the transmitter and receiver MMSE filtering as follows: \vspace{-0.0cm}
   \begin{eqnarray}
 \label{MSE_A}
 \mathsf{MSE}(\mathbf{{A}}, \{\mathbf{B}_k\})\! \!&=&\! \!\sum_{k =1}^K \mathsf{tr}(\mathbf{A}^H \mathbf{H}_k \mathbf{B}_k - \mathbf{I}  )(\mathbf{A}^H \mathbf{H}_k \mathbf{B}_k - \mathbf{I} )^H \nonumber \\
 &+& \sigma_n^2 \mathsf{tr}(\mathbf{A}^H \mathbf{A} ),
 \end{eqnarray}\vspace{-0.0cm}
 due to the fact that $\mathbb{E} \{\mathbf{X}_k \mathbf{X}_k^H\} =\mathbf{I}_N$. Our main objective in (\ref{MSE_A}) is to find the set of matrices $\mathbf{{A}}, \{\mathbf{B}_k\}$ such that MSE is minimized. Based on the widely known approach, we constrain the matrix $\mathbf{A}$ to be orthonormal matrix. Furthermore, under the MMSE criterion, a positive scaling factor $\eta$, called denoising factor, is included in $\mathbf{A}$ for regulating the tradeoff between noise reduction and transmission-power control. Define $\mathbf{A} = \sqrt{\eta} \mathbf{F}$ with $\mathbf{F}$ being a tall unitary matrix and thus $\mathbf{F}^H\mathbf{F} = \mathbf{I}_N$. Then given the MSE in (\ref{MSE_A}) the MMSE filtering problem can be formulated as \vspace{-0.0cm}
\begin{equation}
\begin{aligned}
(\textbf{P1}) \min_{\eta, \mathbf{{A}}, \{\mathbf{B}_k\}} \quad & \mathsf{MSE}(\mathbf{{A}}, \{\mathbf{B}_k\})\\
\textrm{s.t.} \quad & ||\mathbf{B}_k ||^2 \leq P_0 \: \: \forall i\\
&\mathbf{A}^H \mathbf{A} = \eta \mathbf{I}_N    \\
\end{aligned}
\end{equation}\vspace{-0.0cm}
 The solution to (\textbf{P1}) can be shown to be $\mathbf{A}^* = \sqrt{\eta^*} \mathbf{F}^*$, $\mathbf{B}_k^* = \mathbf{A}_k^H(\mathbf{A}_k \mathbf{A}_k^H)^{-1} \: \: \: \forall k$ and $\eta^* = \mathsf{max}_k \frac{1}{P_0}\mathsf{tr}(\mathbf{F}_k \mathbf{F}_k^H)^{-1}$, where $\mathbf{F}_k = (\mathbf{F}^*)^H\mathbf{H}_k$, $\mathbf{A}_k = (\mathbf{A}^*)^H\mathbf{H}_k$, and $\mathbf{F}^* = \mathbf{V}_G$\iffalse \footnote{Note $[\mathbf{V}_G]_{:, 1:N}$ denotes the first $N$ columns of the matrix $\mathbf{V}$.}\fi\cite{Huang2018}. 
 
 Let the effective channel coefficients matrix defined by $\mathbf{G}$ as follows: \vspace{-0.1cm}
 \begin{equation} \vspace{-0.1cm}
 \mathbf{G} = \sum_{k =1}^K\lambda_{min}(\Sigma_k^2)\mathbf{U}_k\mathbf{U}_k^H,
 \label{Gmatrix}
 \end{equation} \vspace{-0.1cm}
 \noindent where $\mathbf{U}_k$ is the left matrix of singular value decomposition (SVD) of $\mathbf{H}_k$, namely $\mathbf{H}_k = \mathbf{U}_k \mathbf{\Sigma}_k \mathbf{V}_k^H$ and SVD of $\mathbf{G}$ is defined as $\mathbf{G}  = \mathbf{V}_G \mathbf{\Sigma}_G \mathbf{V}_G^H$.
 
 %\vspace{-0.2cm}

 \begin{figure*}[ht!]
\centering
	\begin{subfigure}{0.30\textwidth}
		\centering
		\includegraphics[height=4.6cm]{figs/N6K8_32_01.eps}
		\caption{ }
 
	\end{subfigure}
	\begin{subfigure}{0.30\textwidth}
		\centering
		\includegraphics[height=4.6cm]{figs/N12K8_32_01.eps}
		\caption{ }
	 
	\end{subfigure}
		\begin{subfigure}{0.30\textwidth}
		\centering
		\includegraphics[height=4.6cm]{figs/K3N8_12_32_04.eps}
		\caption{ }
		
	\end{subfigure}\vspace{-0.1cm}

	\caption{NOMA with  (a) $N = 6$ and $K = 2, 5, 8$, (b) $N = 12$ and $K = 8, 20, 32$, (c) $K = 3$ and $N = 8, 20, 32$}.
\label{3figs}
\end{figure*}

	
 \subsection{Eigenvector-Based Approach}
 \label{EigenVectorBased}
We consider an alternative solution of (\textbf{P1}) that is entirely based on the largest eigenvectors of the sum-channel matrix. Define the sum-channel matrix as \vspace{-0.1cm}
\begin{equation}
    \mathbf{H}_{s} = \sum_{k = 1}^K \mathbf{H}_k .
    \label{MatrixHs}
\end{equation}\vspace{-0.1cm}
The eigenvector-based solution can be achieved by $\mathbf{A}^* = \mathbf{Q}$, where $\mathbf{Q}$ is the eigenvector decomposition of $\mathbf{H}_s$ such that $\mathbf{H}_{s} = \mathbf{Q} \Lambda \mathbf{Q}^{-1}$. Note that $\mathbf{H}_s$ and $\mathbf{G}$ in (\ref{Gmatrix}) are diagonal matrices due to the fact that $\mathbf{H}_k$ is diagonal for $1\leq k \leq K$. Hence, it can be shown that the obtained solutions $\mathbf{A}^*$ are also diagonal indeed. More explicitly, it is an identity matrix, $\mathbf{I}_N$. We adopt the eigenvector-based approach as it is computationally less expensive to compute (\ref{MatrixHs}) than (\ref{Gmatrix}). 
\vspace{-0.1cm}
 \subsection{Channel Feedback Phase}
  The (\textbf{P1}) solution obtained requires perfect knowledge of global channel state coefficients $\{\mathbf{H}_k\}$ to be available at all $K$ sensor nodes. The proposed channel training and feedback mechanism, where we assume that the feedback observation at the AP can be noiseless is represented by  \vspace{-0.1cm}
 \begin{equation}
 \label{feedback}
 \mathbf{Z} =  \sum_{k =1}^K \mathbf{H}_k \mathbf{D}_k \: \: \in \mathbb{C}^{N \times N},
 \end{equation}\vspace{-0.1cm}
 \noindent where $\mathbf{D}_k\in \mathbb{C}^{N \times N}$ denotes the signal matrix transmitted by the sensor node $k$. Let $\mathbf{A}^*$ denoted the derived solution of (\textbf{P1}), $\tilde{f}(\cdot)$ and $\tilde{g}_k(\cdot)$ be the feedback counterparts of the pre- and post-processing operations of ${f}(\cdot)$ and ${g}_k(\cdot)$ for $1 \leq k \leq K$. One of the important design constraint is that the transmitted signal $\mathbf{D}_k$ in (\ref{feedback}) must be a function of $\mathbf{H}_k$ only, which we denote it as $\mathbf{D}_k = \tilde{g}_k(\mathbf{H}_k) $. Furthermore, the optimization problem can be formulated as\vspace{-0.1cm}
 \begin{equation} 
(\textbf{P2}) \: \: \:  \: \: \: \mathbf{A}^* = \tilde{f}(\sum_{k = 1}^K \mathbf{H}_k \tilde{g}_k(\mathbf{H}_k)),
 \end{equation}\vspace{-0.1cm}
 \noindent and the problem of feedback design reduces to the design of the functions $\tilde{f}(\cdot)$ and $\{\tilde{g}_i(\cdot)\}$. The solution is obtained when $\mathbf{Z} = \mathbf{G}$. Therefore, the feedback signal solution is obtained as $\mathbf{D}_k^* = \tilde{g}_k(\mathbf{H}_k) = \lambda_{min}(\mathbf{\Sigma}_k^2)\mathbf{V}_k \mathbf{\Sigma}_k^{-1} \mathbf{U}_k^H$ and feedback post-processing is $\mathbf{F}^* = \tilde{f}(\mathbf{Z}) = \mathbf{U}_Z$, where $\mathbf{U}_Z$ denotes the left eigenvectors of $\mathbf{Z}$. 
 \section{Simulation Results}
 \label{simulations}
 In this section, we evaluate the performance of proposed scheme via simulation studies. The MSE performance of (\ref{MSE}) are illustrated in Fig. \ref{3figs} for NOMA system. In Fig. \ref{3figs} (a), we set $N=6$ and vary $K=2$, $K=5$ and $K=8$, where $a1$, $a2$ and $a3$ denote average sum-channel based, eigenvector-based and effective channel based techniques, respectively. The MSE is a decreasing function of $E_b/N_o$. 
 

 In addition, we observed that when $K$ increases MSE performance decreases further for the eigenvector-based and effective channel based techniques but not for the average sum-channel as portrayed in Fig. \ref{3figs} (b). Similar results are obtained for the fixed $N=12$ and varying $K=8$, $K=20$ and $K=32$ as shown in Fig. \ref{3figs} (a). In Fig. \ref{3figs} (c), we set $K=3$ and vary $N=8$, $N=20$ and $N=32$, respectively. We note that increasing $N$ does not effect the MSE performance unlike the case for increasing $K$ the MSE performance decreased for all solution methods as shown in Fig. \ref{3figs} (c). 
 
 We further illustrate the effects on MSE performance by increasing the $K$ and $N$ values jointly in Figs. \ref{3dBarK2_4N2_06_01} and \ref{3dBarK8_32N8_18_01}, respectively. For each values of $K$ and $N$, we plot $6$ bar values where the front and back three values are for $a1$, $a2$ and $a3$ techniques evaluated at $E_b/N_o=1$ dB and $E_b/N_o=5$ dB, respectively. In all our simulation studies, we observe that the proposed eigenvector-based scheme outperforms the average sum-channel method, showing the effectiveness of new optimization based approach. Note the effective channel based and eigenvector-based methods have similar MSE performance, as discussed in Section \ref{EigenVectorBased}. Furthermore, the MSE performance gain of the proposed eigenvector-based scheme is more evident for larger values of $N$ and $K$, further confirming the effectiveness of the proposed solution for the multimodal sensing and dense networks. Our numerical results suggest that having a larger diversity gain is of great benefit, since it can provide a satisfactory MSE performance with reduced computational complexity.%, as  we will discuss in the next section.
    \vspace{-0.6cm}
\begin{center}
	\begin{figure}[tb]
		\centering
		\includegraphics[width=3.2 in]{figs/3dBarK2_4N2_06_01.eps}\vspace{-0.1cm}
		\centering \caption{NOMA with $K = 2, 5, 8$ and $N = 2,6,10$}. \label{3dBarK2_4N2_06_01}
		\vspace{-0.2cm} 
	\end{figure}
\end{center}
 
\vspace{-0.4cm}

 \begin{center}
	\begin{figure}[tb]
		\centering
		\includegraphics[width=3.2 in]{figs/3dBarK8_32N8_18_01.eps}\vspace{-0.1cm}
		\centering \caption{NOMA with $K = 8, 20, 32$ and $N = 8,12,18$.} \label{3dBarK8_32N8_18_01} \vspace{-0.2cm}
		%\vspace{-0.9cm}
	\end{figure}
\end{center}

\vspace{-0.5cm}
\section{Complexity of Solution Methods}
    In this section, we will be focusing on the computation complexity of construction of filter $\mathbf{A}^*$.    The computational complexity of the average sum-channel based method, $a1$, is $\mathcal{O}(K^3)$, which involves the addition of $K$ matrices having the size of $K \times K$. The proposed eigenvector-based method, $a2$, in addition to $a1$, involves eigenvector decomposition process with the complexity of $\mathcal{O}(K^3)$, hence overall complexity is $\mathcal{O}(K^3+K^3) = \mathcal{O}(K^3)$. On the other hand, the complexity of effective channel based technique, $a3$, involves SVD for each matrix $\mathbf{H}_k$ with the complexity of $\mathcal{O}(K^3)$, $K \times K$ matrix multiplication and $K$ matrix addition that results in an overall complexity of $\mathcal{O}((K^3 + K^3)K) = \mathcal{O}(K^4)$. Since our matrices are diagonal in our formulations, it is straightforward to show that the complexity of sum-channel based method, $a1$, eigenvector-based method, $a2$, and effective channel based technique, $a3$, are $\mathcal{O}(K^2)$, $\mathcal{O}(K^2)$, and $\mathcal{O}(K^3)$, respectively, as shown in Table \ref{table:complexity}. 
		\vspace{-0.1cm}
		\begin{table}[tb]
	\caption{Computational Complexity Comparison}
	\centering % centering table
	\begin{tabular}{l c c } % creating 10 columns
		\hline\hline \rule{0pt}{3ex}  % inserting double-line
		\bf{Algorithms} & \bf{Complexity} & \bf{Main procedures}  \\ [0.5ex]
		\hline \rule{0pt}{3ex}  % inserts single-line
		% Entering 1st row
		$a1$& $\mathcal{O}(K^2)$ & multiplication, addition  \\[0.2ex]
		% Entering 2nd row
		\:\:$a2$ &$\mathcal{O}(K^{2})$ & multiplication, addition   \\[0.2ex]
		\:\:$a3$ &$\mathcal{O}(K^3)$ & multiplication, addition   \\[0.2ex]
		\hline % inserts single-line
	\end{tabular}
	\label{table:complexity}
\end{table}

    In contrast to the $a3$ approach, the $a2$ approach has lower computation complexity although they both demonstrate similar MSE performance.
    \vspace{-0.0cm}
\section{Conclusion}
	\label{conclusion}
		In this paper, we developed a wideband non-orthogonal multiple-access based computation over multi-access channel (NOMA-CoMAC) framework. We formulated an optimization problem analytically in terms of the mean squared error (MSE), which is a prerequisite for the CoMAC, specifically in multimodal sensor networks.	We conceived an improved MSE minimization solution based on the eigenvector-based approach. 
		We demonstrated that our proposed eigenvector-based scheme achieves around $0.7$ lower at $E_b/N_o = 1$ dB in terms of MSE performance compared to average sum-channel based method. Moreover, MSE performance gain of our proposed solution increases for the larger values of $K$ by benefiting from the diversity gain. This, in return, suggests that our proposed scheme is eminently suitable for multimodal sensor networks.
		 In our future research, we will conceive NOMA-CoMAC for multimodal sensors for transmission over dispersive fading channels as well as possibility of incorporating network coding in the existing framework. 


 \vspace{-0.4cm}
 
	
		
%	\end{thebibliography}
%\bibliographystyle{IEEEtran}
%\bibliography{IEEEabrv,IEEEexample}
\bibliographystyle{IEEEtran}
\bibliography{CoMAC}


\end{document}
