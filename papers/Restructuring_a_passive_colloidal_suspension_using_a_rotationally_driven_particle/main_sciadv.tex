% Use only LaTeX2e, calling the article.cls class and 12-point type.

\documentclass[12pt]{article}

% Users of the {thebibliography} environment or BibTeX should use the
% scicite.sty package, downloadable from *Science* at
% http://www.sciencemag.org/authors/preparing-manuscripts-using-latex 
% This package should properly format in-text
% reference calls and reference-list numbers.

%%%%%%%% PACKAGES/FUNCTIONS

\usepackage{graphicx}% Include figure files
\usepackage{dcolumn}% Align table columns on decimal point
\usepackage{bm}% bold math
\usepackage{textcomp} %allow \textmu
\usepackage{siunitx} %dot in textmode for units.
\usepackage{amsmath}
\usepackage{hyperref}

\usepackage{navigator}

%\usepackage{hyperref} %causes problems with scicite package.
%\usepackage[all]{hypcap}    %for going to the top of an image when a figure reference is clicked
%following three to enable in text markers
\usepackage{color}
\usepackage{tikz}
\usepackage{ulem} %enable strike-through
\usetikzlibrary{shapes.geometric}

\makeatletter
\renewcommand{\fnum@figure}{\textbf{Fig. \thefigure}}
\makeatother

\newcommand{\Blaise}[1]{{\bf{\color{blue}#1}}} 
\newcommand*{\figref}[2][]{% set full link for figure references
  \hyperref[{fig:#2}]{%
    \ref*{fig:#2}%
    \ifx\\(#1)\\%
    \else
      \,(#1)%
    \fi
  }%
}
\newcommand{\Ernest}[1]{{\bf{\color{red}#1}}} 
\newcommand{\Michelle}[1]{{\bf[{\color{magenta}#1}]}} 

% Commands
\newcommand{\eqn}[1]{\begin{align}#1\end{align}}
\newcommand{\bs}[1]{\boldsymbol{#1}}
\newcommand{\ps}[1]{\partial_{#1}}
\newcommand{\pare}[1]{\left( #1 \right) }
\newcommand{\corchete}[1]{\left[ #1 \right]}
\newcommand{\fr}[2]{\frac{#1}{#2}}
\newcommand{\wtil}[1]{\widetilde{#1}}
\newcommand{\mc}[1]{\mathcal{#1}}
\newcommand{\avg}[1]{\langle #1 \rangle}
\newcommand{\tex}[1]{\mbox{\scriptsize{#1}}}
\newcommand{\crossout}[1]{{\color{red}\sout{#1}}}
\newcommand{\deleted}[1]{}
\newcommand{\textgreek}[1]{\begingroup\fontencoding{LGR}\selectfont#1\endgroup}
\newcommand{\what}[1]{\widehat{#1}}

% Markers in caption -> http://www.callumatkinsononline.com/adding-shape-and-symbol-to-figure-captions-in-latex/
\newcommand{\graycircle}{\raisebox{0.pt}{\tikz{\node[draw,scale=0.6,circle,white,fill=white!45!black](){};}}}
\newcommand{\blackcircle}{\raisebox{0.2pt}{\tikz{\node[draw,scale=0.6,circle,black,fill=black](){};}}}
\newcommand{\redcircle}{\raisebox{0.2pt}{\tikz{\node[draw,scale=0.6,circle,red,fill=red](){};}}}

\newcommand{\smallredcircle}{\raisebox{0.5pt}{\tikz{\node[draw,scale=0.3,circle,red,fill=red](){};}}}
\newcommand{\greentriangle}{\raisebox{0.8pt}{\tikz{\node[draw=black!60!green,scale=0.2,regular polygon, regular polygon sides=3,fill=black!60!green,rotate=0](){};}}}
\newcommand{\bluetriangle}{\raisebox{0.8pt}{\tikz{\node[draw=blue,scale=0.2,regular polygon, regular polygon sides=3,fill=blue,rotate=180](){};}}}

\newcommand{\opencircle}{\raisebox{0.2pt}{\tikz{\node[draw,scale=0.6,circle,fill=none](){};}}}
\newcommand{\opensquare}{\raisebox{0.2pt}{\tikz{\node[draw,scale=0.8,rectangle, fill=none](){};}}}
\newcommand{\opentriangle}{\raisebox{0.2pt}{\tikz{\node[draw,scale=0.4, regular polygon, regular polygon sides=3, color=blue](){};}}}
\newcommand{\opentriangleud}{\raisebox{0.2pt}{\tikz{\node[draw,scale=0.4, regular polygon, regular polygon sides=3, color=blue,rotate=180](){};}}}

\newcommand{\yellowline}{\raisebox{2pt}{\tikz{\draw[-,black!40!yellow,solid,line width = 1 pt](0,0) -- (3mm,0);}}}
\newcommand{\purpleline}{\raisebox{2pt}{\tikz{\draw[-,black!40!purple,solid,line width = 1 pt](0,0) -- (3mm,0);}}}
\newcommand{\greenline}{\raisebox{2pt}{\tikz{\draw[-,black!40!green,solid,line width = 1 pt](0,0) -- (3mm,0);}}}
\newcommand{\blueline}{\raisebox{2pt}{\tikz{\draw[-,black!40!blue,solid,line width = 1 pt](0,0) -- (3mm,0);}}}


\newcommand{\blackdashedline}{\raisebox{2pt}{\tikz{\draw[-,black!40!black,dashed,line width = 1 pt](0,0) -- (3mm,0);}}}

% Miscellaneous 
\def\dt{\Delta t}
\def\dd{\mathrm{d}}  
\def\kt{k_B T}
\def\bna{\bs{\nabla}}

% Bold symbols
\def\bC{\bs{C}}
\def\bD{\bs{D}}
\def\bX{\bs{X}}
\def\bx{\bs{x}}
\def\bXi{\bs{\Xi}}
\def\by{\bs{y}}
\def\bI{\bs{I}}
\def\bK{\bs{K}}
\def\bp{\bs{p}}
\def\bP{\bs{P}}
\def\bPi{\bs{\Pi}}
\def\bPsi{\bs{\Psi}}
\def\bM{\bs{M}}
\def\bR{\bs{R}}
\def\bomega{\bs{\omega}}
\def\bw{\bs{w}}
\def\bW{\bs{W}}
\def\bt{\bs{t}}
\def\bT{\bs{T}}
\def\btau{\bs \tau}
\def\bn{\bs{n}}
\def\bb{\bs{b}}
\def\bB{\bs{B}}
\def\be{\bs{e}}
\def\bu{\bs{u}}
\def\bU{\bs{U}}
\def\bv{\bs{v}}
\def\bV{\bs{V}}
\def\bl{\bs l}
\def\blambda{\bs{\lambda}}
\def\br{\bs{r}}
\def\bF{\bs{F}}
\def\bbf{\bs{f}}
\def\bphi{\bs{\phi}}
\def\bq{\bs{q}}
\def\bdq{\bs{\delta q}}
\def\bg{\bs{g}}
\def\bG{\bs{G}}
\def\ba{\bs{a}}
\def\bA{\bs{A}}
\def\bN{\bs{N}}
\def\bQ{\bs{Q}}

\def\bzero{\bs{0}}
\def\btheta{\bs \theta}
\def\bgamma{\bs \gamma}

%%%%%%%%


\usepackage{scicite}
\usepackage{times}

% The following parameters seem to provide a reasonable page setup.
\topmargin 0.0cm
\oddsidemargin 0.2cm
\textwidth 16cm 
\textheight 21cm
\footskip 1.0cm

%The next command sets up an environment for the abstract to your paper.
\newenvironment{sciabstract}{%
\begin{quote} \bf}
{\end{quote}}

% Include your paper's title here
\title{Restructuring a passive colloidal suspension using a rotationally driven particle} 

% Place the author information here.  Please hand-code the contact
% information and notecalls; do *not* use \footnote commands.  Let the
% author contact information appear immediately below the author names
% as shown.  We would also prefer that you don't change the type-size
% settings shown here.

\author
{Shih-Yuan Chen$^{1\dagger}$, \\
Hector Lopez-Rios$^{2\dagger}$,\\
Monica Olvera de la Cruz$^{1,2\ast}$,\\
Michelle M. Driscoll$^{1\ast}$\\
\\
\normalsize{$^{1}$Department of Physics \& Astronomy, Northwestern University,}\\
%\normalsize{Evanston, IL 60208, USA}\\
\normalsize{$^{2}$Department of Materials Science \& Engineering, Northwestern University,}\\
\normalsize{Evanston, IL 60208, USA}\\
\\
\normalsize{$\dagger$These authors contributed equally.}\\
\normalsize{$^\ast$To whom correspondence should be addressed; E-mail:  m-olvera@northwestern.edu,}\\
\normalsize{michelle.driscoll@northwestern.edu.}
}

% Include the date command, but leave its argument blank.
\date{}

%%%%%%%%%%%%%%%%% END OF PREAMBLE %%%%%%%%%%%%%%%%

%%% Comment %%%
\newcommand{\SYC}[1]{{\color{green} #1}}
\newcommand{\MD}[1]{{\color{red} #1}}
\newcommand{\HLR}[1]{{\color{blue} #1}}
%%%%%%%%%%%%%%%%% END OF Comment %%%%%%%%%%%%%%%%


\begin{document} 
% Double-space the manuscript.
\baselineskip24pt
% Make the title.
\maketitle 

% Place your abstract within the special {sciabstract} environment.

\begin{sciabstract} % word limit: 127 / 150

The interactions between passive and active/driven particles have been explored as a means to modify structures in solutions. Often times hydrodynamics plays a significant role in explaining emergent patterns in such mixtures. In this study, we demonstrate that strong advective flows generated by a single driven rotating particle near a surface can induce large-scale structural rearrangements in a passive suspension. The resulting emergent pattern exhibits an accumulation area in front of the driven particle and a wake along its trajectory. Notably, the center of the accumulation is found at a distance from the driven particle. Through experiments and Stokesian dynamics simulations, we show that the driven-passive interaction is solely determined by the heights of both particle types, which determines the shape and the size of the pattern. By modulating the height of the driven particle we can control the extension of the emergent pattern from 5 to 10 times its radius ($\sim 12 \, \mu \mathrm{m}$).

%The role of hydrodynamic interactions in colloidal suspensions of passive and active or driven particles has frequently been explored using a concentration of the advecting particles as a collective parameter of the system. Many questions within these systems revolved around the positional correlations of passive particles or objects as a function of the concentration of advecting particles. Here, we present the problem of whether one driven particle that rotates above a surface is sufficient to affect the positional order of passive particles within a suspension close to a solid surface. Using experiments and Stokesian dynamics simulations we find that the the range of hydrodynamic interaction of this single advecting particle is only a function of its height from the solid surface and can span a range about 5 times its diameter ($\sim 10 \, \mu \mathrm{m}$).

\end{sciabstract}

% In setting up this template for *Science* papers, we've used both
% the \section* command and the \paragraph* command for topical
% divisions.  Which you use will of course depend on the type of paper
% you're writing.  Review Articles tend to have displayed headings, for
% which \section* is more appropriate; Research Articles, when they have
% formal topical divisions at all, tend to signal them with bold text
% that runs into the paragraph, for which \paragraph* is the right
% choice.  Either way, use the asterisk (*) modifier, as shown, to
% suppress numbering.

\section*{Introduction}
%Material properties are intrinsically coupled to its microscopic behaviour and structure. 
Colloids have been extensively used to explore the relation between structure and function of materials. Their structure is easily observable with a simple optical microscope \cite{li_colloidal_2011,zhang_toward_2015,manoharan_colloidal_2015}, and colloidal particles can be actuated to roll \cite{driscoll_unstable_2017}, spin \cite{sabrina_shape-directed_2018}, and oscillate \cite{zhang_quincke_2021}.
Moreover, one can easily tune particle interactions in a colloidal system using straightforward modifications to particle size, shape, and surface chemistry, which allows for the design of exotic material properties such as tunable shear-jamming \cite{chen_leveraging_2023} and patterned wettability \cite{shao_superwettable_2019}.

Generally, self-assembly is driven by interparticle forces and entropic forces. External applied fields, however, unlock self-assembly pathways that are otherwise inaccessible including avalanches \cite{driscoll_leveraging_2019} and configurations that encode memory \cite{kaz_physical_2012,zhang_polar_2022}; indeed, the soft matter community has established that self-propelling particles can tune self-assembly \cite{massana-cid_active_2018,omar_swimming_2019}. Additional degrees of freedom, such as in mixtures of active and passive particles, enhance the phase space of microstructures \cite{mallory_active_2018,madden_hydrodynamically_2022}. The motion of self-propelling particles has been shown to enhance the diffusivity of passive particles \cite{mino_enhanced_2011,jepson_enhanced_2013}, create clusters of passive particles around self-propelling particles \cite{palacci_living_2013,katuri_inferring_2021}, or induce phase separation between particle types \cite{mccandlish_spontaneous_2012,cates_motility-induced_2015,stenhammar_activity-induced_2015,wysocki_propagating_2016,smrek_small_2017,dolai_phase_2018}. 
Moreover, as the self-propelling particles exert forces on their surroundings, they continue reconfiguring the local structure already built \cite{katuri_inferring_2021,singh_interaction_2022}.
%, and modify the self-assembly process of passive particles . 
When active particles are added to an equilibrium passive crystal structure, their activity accelerates the annealing process, leading to large-scale single crystals. \cite{ramananarivo_activity-controlled_2019}. 
%
%Driven particles, while their propulsion mechanism differs from that of self-propelling active particles, have shown similar capability to interact with their environments. For instance, when dragging a driven particle through a colloidal suspension, it redistributes passive particles around it and creates new structures \cite{burkholder_nonlinear_2020,knezevic_oscillatory_2021,rafael_active_2022}.
%When two driven particles rotate with their rotational axis perpendicular to the floor, they repel each other in a pure fluid but attract each other in a colloidal suspension \cite{aragones_elasticity-induced_2016}.
%
Through active-passive interactions, active particles via self-diffusiophoresis \cite{illien_fuelled_2017} and driven particles via external fields have the potential to tune the aggregation of passive particles \cite{massana-cid_active_2018,omar_swimming_2019}, and ship cargo passive particles to desired locations \cite{petit_selective_2012,zhang_targeted_2012}. 
These examples demonstrate the complexity of active-passive and driven-passive interactions. To understand how these particles reshape material structure, it is crucial to understand how they interact with their surroundings in a colloidal suspension. In many cases, hydrodynamics plays a significant role to explain the restructuring and the emergent patterns \cite{katuri_inferring_2021,marchetti_hydrodynamics_2013,boniface_hydrodynamics_2023}.

%Moreover, adding more passive particles lowers the velocity of a driven particle, which in turn serves as a probe to the local effective viscosity of colloidal suspensions \ref{}. Meanwhile, increasing activity in an active-passive micture generates local accumulations, which can alter the colloidal suspension from gelation to crystalization \ref{}. 

%For example, in a passive colloidal solution, Janus particles that consume chemical fuel cause a chemical gradient which creates flows that attract passive particles toward the non-coated side and push particles away from the coated side \cite{katuri_inferring_2021}. This example demonstrates the essential to study the flow fields of active/driven particles to understand how they restructure a passive matrix.
%. Depending on the facing side of the J
%In a dense suspension, this attraction creates a layer of jammed particles at its moving front \cite{singh_interaction_2022}. 
%%%%%
%This example demonstrates the complexity of active-passive interactions, and the essential to study the flow field of an active particle to understand how it restructures a passive matrix.

%Driven particles, while their propulsion mechanism differs from that of self-propelling active particles, have shown similar capability to interact with their environments.\cite{volpe_microswimmers_2011, tierno_recent_2014, yang_microwheels_2019, tsang_roads_2020, van_der_wee_simple_2023} 
%(Through the active-passive interaction, active and) driven particles have the potential to tune the aggregation of passive particles \cite{massana-cid_active_2018,omar_swimming_2019}, and ship cargo passive particles to desired locations \cite{petit_selective_2012,zhang_targeted_2012}. Moreover, adding more passive particles lowers the velocity of an driven particle, which in turn serves as a sensitive probe of local particle distribution \cite{zia_active_2018}. Meanwhile, increasing activity in an active-passive mixture can drive a phase transition from gelation to crystalization \cite{mallory_universal_2020}. 

In this work, we examine how a driven particle, called a microroller \cite{delmotte_hydrodynamic_2017}, immersed in a passive particle suspension can restructure its surroundings through experiments and Stokesian dynamics simulations. 
%
Here a microroller rearranges the passive particles through hydrodynamic interactions, creating an accumulation zone around itself and a depletion tail in its wake. We observe that a steady-state pattern emerges from these interactions, and this pattern is $10$ times larger than the microroller radius and is three dimensional in nature. We note that in striking contrast to restructuring created by actively dragging a particle through a suspension \cite{zia_active_2018}, the highest concentration of passive particles in the accumulation area is roughly $10$ microroller radii away from the microroller location. 
This underscores the mechanism of restructuring as the long-range flow field generated by the microroller. Due to the sensitive dependence of hydrodynamic interactions on the particle's height above the chamber floor, we find that the structure of the pattern is modified by changing the height of either the driven or passive particles.

%---------------------Old one ---------------------
\iffalse
Material properties are frequently linked to their microscopic behaviour and microstructure. It is often the change in a material's microstructure that determines its macroscopic response and alters its properties. This applies to many systems from liquids to solids.
For example, when worms stay in a water/ethanol solution, the viscosity decreases as the activity of the warms increases \cite{deblais_rheology_2020}. The manner how a perforated sheet breaks changes from brittle fracture to random breaking by varying the aspect ratio of the branches connecting the whole sheet \cite{driscoll_role_2016}. For crystalline materials, microstructures like dislocations affect the material's ductility and strength \cite{chang_experimental_1981, chen_dislocation-density_2023}. When a droplet is deposited on a solid substrate, the surface roughness changes the contact angle \cite{guo_superhydrophobic_2011, didarul_islam_template-free_2022}.
%
Therefore, studying microscture and its connection to the macroscopic behaviour helps the design of the material's properties.
%
Colloids, nano- to micron-size particles, provide a promising way to build the connection between microstructure to macroscopic properties. Moreover, by designing and controlling the colloids, one can create new structures and even tune the material properties \cite{mallory_active_2018}.
%
For instance, when a single driven particle moves fast enough to overcome the Brownian diffusivity, it modifies passive particles around it and creates a non-equilibrium structure \cite{burkholder_nonlinear_2020, knezevic_oscillatory_2021, rafael_active_2022}, which relates to the effective viscosity of colloidal suspension.
%
Similarly, a Janus particle accumulates passive particles around it by the chemical and hydrodynamic flows it generates near the boundary. When it propels, it accumulates particles in front of it and leaves a trajectory with no particles behind it \cite{katuri_inferring_2021, singh_interaction_2022}.
Doping many of such Janus particles into a passive colloidal suspension can form separate clusters or colloidal gel network depending on the area fraction of Janus particles and passive colloids \cite{massana-cid_active_2018}.

Importantly, the structure generated by colloids can be hundred times bigger than the colloid itself. In many cases, hydrodynamic interactions between the particles and the surroundings become important to explain such long-range phenomena. For instance, a hematite Janus particle, a particle of which one side interacts with the liquid and creates flow, aggregate passive particles around itself to form a raft. 
The self-propelling of the raft 
the previous Janus-passive cluster self-propels due to the osmotic flow generated from the chemical gradient near the substrate \cite{boniface_hydrodynamics_2023}.
When bacteria move in circular motion, they push and rearrange each other, modulating their spatial distribution from random to hyperuniform \cite{huang_circular_2021}. The interaction between bacteria is explained by the flow field created after averaging a cycle of the circular motion. 
These examples show that it is crucial to understand the flow field generated by a single colloid to explain the fundamental mechanism of the structure formation and its motion.
\fi
%-------------------------------------------------------------------------------
\iffalse
Swimming colloids, nano- to micron-size particles, can sense, interact, and modify their surroundings. When self-propelling colloids like Janus particles and bacteria are in a colloidal suspension, the activity can enhance passive colloids diffusivity,\cite{mino_enhanced_2011, jepson_enhanced_2013} induce phase separation between passive and active particles,\cite{mccandlish_spontaneous_2012, cates_motility-induced_2015, stenhammar_activity-induced_2015, wysocki_propagating_2016, smrek_small_2017, dolai_phase_2018}, and between active particles with different activities.\cite{weber_binary_2016, kumari_demixing_2017} Even with a single active particle, it can expel passive particles and create vacant holes or attract particles to create accumulation.\cite{katuri_inferring_2021, singh_interaction_2022}
%
Complex environments also change how active particles interact with each other.\cite{bechinger_active_2016} For example, when encountering obstacles in the fluid media, active matter change their speed to line up, squeeze through, or be trapped around the obstacles.\cite{di_leonardo_bacterial_2010, tanaka_hot_2017, bhattacharjee_bacterial_2019, kamdar_colloidal_2022} 

Colloids also swim with the help of external forces.\cite{zhang_characterizing_2009, driscoll_unstable_2017, delmotte_hydrodynamic_2017, alapan_multifunctional_2020} While these swimming colloids, or driven particles, move in a prescribed direction uniformly, which differs from their self-propelling counterparts, they have shown similar sensing and modifying capability when interacting with their surroundings.\cite{volpe_microswimmers_2011, tierno_recent_2014, yang_microwheels_2019, tsang_roads_2020, van_der_wee_simple_2023} 
For instance, when a single driven particle swims fast enough to overcome the Brownian diffusivity, it modifies passive particles around it and creates a non-equilibrium structure.\cite{burkholder_nonlinear_2020, knezevic_oscillatory_2021, rafael_active_2022}
%
Passive particles and obstacles also change the interaction between driven particles. 
For example, driven particles can be trapped around a pillar by diffusing into a stagnation region, where the Brownian motion dominates over the flow velocity.\cite{van_der_wee_simple_2023}
Two active rotating particles repel each other in a pure fluid but attract each other in a colloidal suspension.\cite{aragones_elasticity-induced_2016} 
  

Both driven and self-propelling particles can interact with the fluid media in a long distance, and create patterns larger than the particle size in many orders of magnitudes.\cite{huang_circular_2021, zhang_hyperuniform_2022} 
Hence, the study of swimming colloid interaction provides a bridge from single to collective behavior, and from microstructure to mesoscopic pattern to macroscopic physical observation. \SYC{(Any citation?)}
%
To understand how swimming colloids interact with each other and their surroundings, it is crucial to find out the flow generated by the swimming colloids to explain the fundamental mechanism of the motion.\cite{boniface_hydrodynamics_2023} Depending on the type of active particles, the flow field differs and changes the pattern mechanism and size. As an example, rotating particles with an angular momentum perpendicular to the plane break rotation symmetry and generate odd viscosity.\cite{fruchart_odd_2023} On the other hand, rotating particles with an angular momentum parallel to the floor translate along the floor and create fingering pattern and layer separation.\cite{driscoll_unstable_2017, sprinkle_driven_2020}

%For hydrodynamics at the micron size level, the flow generated by a swimming colloid can generally be decomposed into as a combination of a stresslet, a stokeslet, and/or a rotlet in a long range by means of Stokesian dynamics (cite brady here).\SYC{(need citation for those... not now...)}

To study how a driven particle influences its surroundings through long-range interaction, in this work, we examine a driven particle immersed in a passive colloidal suspension in a quasi-2D plane. The driven particle rotates above the floor and generates an asymmetric flow that drives the particle to move. The strong, long-range flow differs from a single stokeslet or a rotlet due to its spatial-asymmetric nature.\cite{delmotte_hydrodynamic_2017} When being in a colloidal suspension, it propels, accumulates, and orbits the particles around itself, creating a new steady 3D pattern that is ten times larger than the driven particle. The size and the shape of the pattern are undisturbed by the speed of the driven particle and the area fraction of the passive particles because the dominant mechanism is the flow field generated by the driven particle. Our simulation shows that the fundamental method to change the pattern is to change the height of the driven particle, which changes the asymmetric flow. Such a robust mechanism provides potential applications like cargo transportation,\cite{petit_selective_2012, zhang_targeted_2012} active microrheology,\cite{zia_active_2018} and colloidal gel manipulation.\cite{massana-cid_active_2018, omar_swimming_2019, mallory_universal_2020} 
\fi

%------------------------------------------------
\section*{Results}
%------------------------------------------------

\begin{figure*}%[tbhp]
\centering
\includegraphics[width=.9\linewidth]{fig/fig1_pattern.pdf}
\caption{
\textbf{Microrollers alter microstructure in passive suspensions.} We dope driven particles (microrollers) in a passive colloidal suspension, as demonstrated in the schematic (a) A small quantity of microrollers (driven particles) are added to a suspension of passive particles. (b) Microrollers contain a permenant magnetic moment, m, and are acuated by applying a uniform rotating magnetic field.  This actuation generates strong advective flows, which scale with $h_{\mathbf{roller}}$, the height of the particle above the surface; these flows are the driver for restructuring the passive suspension. (c) Microrollers restructure the passive suspension by modulating the mean local  density; this resructuring becomes more and more apparent as we average over longer and longer times. (d) Restructuring of the passive suspensions results in the emergence of a steady-state pattern. Left image is the experimental result, in which brighter areas indicate a higher local intensity, which is correlated to a higher concentration of passive particles.  Right image shows the result of Stokesian dynamics simulations, which reproduce the same pattern seen in the experiments.
}
\label{fig:roller}
\end{figure*}

%Using experiments and Stokesian dynamics simulations, 
We experimentally study suspensions of passive particles in water doped with magnetically driven particles, named the microrollers, see Fig.~\ref{fig:roller}(a). 
% 
Both the microrollers and passive particles are denser than water, so they sediment to the floor of the suspension's container, forming a quasi-2D layer of particles. Both types of the particles share the same size ($2 \ \mu$m in diameter). Due to the thermal fluctuation and electrostatic repulsion, both types of particles do not contact the floor but instead hover above the floor surface at an equilibrium height, $h_{\mathrm{roller}}$ and $h_{\mathrm{passive}}$ respectively, as shown in Fig.~\ref{fig:roller}(b). 
%
The microroller's translational velocity is determined by two parameters, the microroller's height $h_\mathrm{roller}$ and its rotational frequency $f$. 
%The moving mechanism can be found in the reference \cite{sprinkle_driven_2020}.
We select the rotational frequency to be between $5$~Hz to $13$~Hz in our experiments so that the speed of the microroller is linearly proportional to the rotating frequency (see SI.1). 

In the driven-passive colloidal mixture, passive particles are initially distributed randomly around the microroller in the $xy$ plane. 
When the external rotating magnetic field is on, the microroller is translating, and the flows generated by the microroller redistribute the passive particles. 
We observe that the passive particles are restructured as follows: there is an accumulation of passive particles in the region of the direction of motion of the microroller (in the $+\hat{x}$ direction), while there is a depleted region opposite to the direction of translation (in the $-\hat{x}$ direction). 
We note that the average microroller speed is constant (see SI.2); the system is in a non-equilibrium steady state.

To quantify the restructuring, we calculate the time averaged number density of passive particles $\left < \rho_{\mathrm{passive}}\left( \mathbf{r} \right) \right >_t$ in the microroller's frame of reference.
In the microroller's frame, the microroller is static, and it is the passive particles that move around the microroller.
A pattern in the passive particle distribution emerges around the microroller as we average more and more frames, as shown in Fig.~\ref{fig:roller}(c).
%
Brighter regions indicate a longer presence of passive particles while darker regions signal the opposite. The emergent pattern reveals a depletion of passive particles in the $-\hat{x}$ direction to the microroller, and a greater presence of passive particles in the vicinity of the microroller.
%
Surprisingly, the peak of the accumulation (the brightest location) in the pattern in the $+\hat{x}$ direction to the microroller is much larger than the particle itself, approximately $10$ times the radius of the microroller ($10 \, \mu \mathrm{m}$). 
%Moreover, the peak intensity, corresponding to the highest accumulation of passive particles, is several particles away from the microroller.

To complement our experimental work, we study this system computationally using Stokesian dynamics. Previous work has shown that a microroller rolls at constant angular velocity imposed by a rotating magnetic field rather than experiencing a constant torque \cite{sprinkle_driven_2020}. Thus, to resemble experiments we study a rotating particle with constant angular velocity in a region where we have initially fixed the area fraction of passive particles. For suspensions at finite temperature we stochastically evolve the system to solve for Brownian dynamics and correctly account for thermal fluctuations. Importantly, the height of the microroller ($h_{\mathrm{roller}}$) sets its velocity profile as a function of angular velocity. Obtaining $h_{\mathrm{roller}}$ from experiments is challenging, thus we use the velocity profile (the microroller's velocity as a function of rotational frequencies) to determine $ \left < h_{\mathrm{roller}} \right > $ to use in simulations. 

Moreover, we use Gaussian smoothing in all $\left < \rho_{\mathrm{passive}}\left( \mathbf{r} \right) \right >_t$ using a variance the size of the passive particle. This adds particle areal effects in $\left < \rho_{\mathrm{passive}}\left( \mathbf{r} \right) \right >_t$, and thus can be comparable to the experimental emergent pattern. In Fig.~\ref{fig:roller}(d), we see that the simulation reproduces the same pattern as observed in experiments, and the areas of high intensity in experiments correspond to areas of high particle density in simulations.

\begin{figure*}%[tbhp]
\centering
\includegraphics[width=.9\linewidth]{fig/fig2_robust.pdf}
\caption{\textbf{The length of the new structure is independent of the microroller's velocity and the area fraction of the passive particles; it is set only by the microroller's height above the surface.} 
(a-b) Pattern length at two different actuation frequencies: (a) 5 Hz and (b) 13 Hz.  We find that the pattern size, $L_{\mathrm{pattern}}$, is independent of microroller velocity.  (c-d) Similarly, altering the mean area fraction of passive particles has no effect on $L_{\mathrm{pattern}}$.  (c) $\phi$ = 0.02 (d) $\phi$ = 0.19.
(e) To quantify $L_{\mathrm{pattern}}$, we draw a box across the microroller along the $x$ axis of the pattern and measure the average intensity in the y axis, as illustrated in the inset; the pink dashed line indicates the center of the box. Then, we find the peak of the intensity (the blue open circle) in front of the roller (the orange circle), and measure the distance from the peak to the microroller (the blue double arrow). 
(f) $L_{\mathrm{pattern}}$ is independent of both microroller velocity (actuation frequency $f$) and $\phi$. We find the average $L_{\mathrm{pattern}}$ is $8.9 \pm 0.9 \ \mu$m (the black dashed line in (f)). We perform simulations at different microroller's heights $h_{\mathrm{roller}}$ at $T=0$K and $T= 293$K to investigate $L_{\mathrm{pattern}}$ as a function of $h_{\mathrm{roller}}$. We observe that $L_{\mathrm{pattern}}$ becomes smaller as $h_{\mathrm{roller}}$ decreases: (g) $h_{\mathrm{roller}}= 1.4 \ \mu$m, (h) $h_{\mathrm{roller}}= 1.05 \ \mu$m, and is identical at $T= 0$~K and $T = 293$~K. 
}
\label{fig:robust}
\end{figure*}

In order to determine what parameters control the features and the size of the emergent pattern, we carry out experiments by independently varying the rotating frequency $f$ from $5$~Hz to $13$~Hz, and the area fraction of passive particles $\phi_\mathrm{passive}$ from $2\%$ to $19 \%$. 
%
In no case do we observe that the pattern changes, see Fig.~\ref{fig:robust}(a-d).
%
To quantify our results, we compute a characteristic pattern length 
$L_{\mathrm{pattern}}$ by measuring the intensity across the pattern along the x axis and define $L_{\mathrm{pattern}}$ to be the distance between the microroller center to the intensity peak, see Fig.~\ref{fig:robust}(e). 
As expected from the experimental images, in Fig.~\ref{fig:robust}(f), $L_{\mathrm{pattern}}$ is invariant with respect to $\phi_\mathrm{passive}$ and $f$. 

Using simulations, we investigate the influence of the microroller height $h_{\mathrm{roller}}$. 
The simulations are carried out at $f= 10$~Hz and $\phi = 0.17$ at $T=0$K and $T=293$K. The results are shown in Fig.~\ref{fig:robust}(g-h). We measure the peak area fraction of the pattern to the microroller to extract $L_{\mathrm{pattern}}$. As shown in Fig.~\ref{fig:robust}(i), $L_{\mathrm{pattern}}$ is directly proportional to $h_{\mathrm{roller}}$, even in the the presence of thermal fluctuations. 

%However, we observe a shorter extension of the depletion region in the experiment with the slower microroller than compared to the microroller that is two times faster. This is due to the competition of time scales between the advection time scale $\tau_\mathrm{roll}$ that is associated to the average speed of the microroller $\left < v_\mathrm{roll}\right >$ and the diffusion time scale which is related to the diffusion of passive particles $D_\mathrm{passive}$. As the microroller is removing the passive particles in the $-\hat{x}$ direction of the microroller by streaming nearby passive particles over and in the direction of its motion, this now empty region will likely be occupied by diffusing particles that were not pushed over the microroller and into the accumulation region. (I could insert ideas about using the Peclet number for some analysis of the extension of the depletion streak $L = \left < v_\mathrm{roll}\right > \Delta t$; I should state that another Peclet number can be defined with $L = a$)

%------------------------------------------------
\section*{Discussion}
\label{sec:discussion}
%------------------------------------------------
%We begin our discussion by clarifying the importance of hydrodynamic interactions compared to thermal fluctuations by determining the suspension's P\'eclet number $\mathrm{Pe}$. Given the value of $\mathrm{Pe}$, we find that advective transport outweighs thermal fluctuations and we proceed to simulate suspensions at $0 \, \mathrm{K}$. From these simulations, we show which two regions of streamlines produced by the microroller generate the passive particle emergent pattern obtained from experiments. We then add thermal fluctuations to the simulated suspensions, and successfully recreate the same pattern shape and size as observed in experiments. Finally, we reveal the 3D nature of the emergent pattern by showing two ways to modify the emergent pattern. This is done by either changing the height of the passive particles relative to the microroller, or by changing the height of the microroller relative to the floor. Additionally, we show the pattern length is determined by the $xy$ stagnation point of the fluid flow generated by the microroller, thus, explaining the pattern's length robustness against passive particle concentration and microroller velocity.

%\subsection{The P\'eclet number \label{subsec:pe}}

To understand the origin of the emergent pattern from experiments we must determine the principle stresses at play in suspensions at finite temperature. Hydrodynamic forces between particles can be impacted by thermal fluctuations as they will disrupt particle's trajectories along streamlines. To estimate the relative influence of thermal fluctuations compared to the advective flows, we calculate the P\'eclet number, $\mathrm{Pe} = \frac{R_\mathrm{passive} u_\mathrm{fluid}}{D_\mathrm{passive}}$, where $R_\mathrm{passive}$ is the average passive particle radius, $u_\mathrm{fluid}$ is approximately the maximum velocity of the fluid due to a microroller, and $D_\mathrm{passive}$ is the passive particle diffusion coefficient. Using the values $R_\mathrm{passive} = 1 \ \mu \mathrm{m}$, $u_\mathrm{fluid} = 50 \ \mu\mathrm{m}/\mathrm{s}$, and $D_\mathrm{passive} = 0.15 \ \mu \mathrm{m}^2/\mathrm{s}$, we calculate that $\mathrm{Pe} = 333 \gg 1$; and thus reveals that passive particle transport is dominated by microroller-generated advective flows rather than from thermal fluctuations. Therefore, computationally less expensive simulations at $T=0 \mathrm{K}$ are sufficient to understand the origin of the pattern formation; Brownian motion does not alter the average size of the pattern. 

%\subsection{The fluid flow field generated by the roller}

\begin{figure*}%[tbhp]
\centering
\includegraphics{fig/fig_3_test.png}
\caption{\textbf{The emergent pattern results from hydrodynamic interactions around a microroller.} Fluid streamlines in the micoroller's frame produced from a microroller (orange circle) at $z = 3  h_\mathrm{r}$ (a). The fluid velocity is normalized by the microroller's translational speed. There are two characteristic sets of streamlines around the microroller, recirculating (blue) and bypassing (purple) streamlines. Additionally, we observe the presence of stagnation points (white x-crosses) in the front and back of the microroller. In a suspension of passive particles at $T = 0 \, \mathrm{K}$, we extract passive particles that have residence times $\tau$ larger than background particles $\tau_p$ (b) and plot their trajectories (c). These trajectories are confined to the recirculating and bypassing regions. Using the average passive particle velocity profile in (d), we determined that particles in the recirculating region have long residence because they are trapped around the microroller. Meanwhile, particles in the bypassing region persist around the microroller due to their curved trajectories around the microroller. Finally, using simulation data of suspensions at $\mathrm{T} = 293 \, \mathrm{K}$, we show the passive particle streamlines around the microroller and overlay them on the simulation  $\left< \rho_{\mathrm{passive}}\left( \mathbf{r} \right) \right>_t$  (top) and experimental emergent pattern (bottom)(e). This shows that the recirculating and bypassing streamlines span the emergent pattern.     
}
\label{fig:xy_mechanism}
\end{figure*}

As this is an advection-dominated system, we focus on the streamlines generated by the microroller to explain the formation of the emergent pattern. While these streamlines are three dimensional in nature we will show that it is sufficient to focus on streamlines in the $xy$ plane to understand the formation of the emergent pattern. Moreover, the experimental measurements only capture the pattern extent in the $xy$ plane.
%We later show that streamlines perpendicular to the floor will change $L_pattern$, and the affect the depletion region behind the microroller.   

We begin by calculating the flow velocity around the rotating microroller in the frame of the microroller, see Fig. \ref{fig:xy_mechanism}(a). In the microroller frame, the microroller is stationary while the passive particles are the mobile species. In the flow profile in the microroller frame, two different sets of streamlines are observed in the fluid in the vicinity of the microroller: (1) a set that surrounds and recirculates around a pair of axial symmetric vortices alongside the microroller (light blue streamlines in Fig.~\ref{fig:xy_mechanism}(a)), and (2) a set that bounds and bypasses the recirculating region (light purple streamlines in Fig.~\ref{fig:xy_mechanism}(a)). An additional feature in the flow field is the appearance of two axially symmetric stagnation points in the front and back of the microroller. These stagnation points are saddle points, that is the fluid flow is convergent along one direction and divergent in the orthogonal direction. In our system, the microroller is always driven in the $+ \hat{x}$ direction; this breaks the symmetry of the stagnation points. The front stagnation point focuses fluid along the $x$ axis and ejects fluid in the $+ \hat{y}$ and $ -\hat{y}$ directions. Meanwhile, the opposite is true for the stagnation point behind the microroller; fluid is focused through streamlines in the $y$ axis and expelled in the $+ \hat{x}$ and $ -\hat{x}$ directions. 

%Interestingly, the symmetry of the stagnation points and the nature of the bypassing streamlines recalls similar features of the streamlines around a static two dimensional cylinder. \cite{van_der_wee_simple_2023} 

To understand how a pattern emerges in the passive suspension, we first consider how a single passive particle interacts with the flow field generated by the microroller. We begin our analysis by considering a passive particle which remains in the plane of the $xy$ streamlines (above the microroller) as seen in Fig. \ref{fig:xy_mechanism}(a);
%at $z = 1.44 \times h_\mathrm{roller} $ above the floor.  
a single passive particle approaching the microroller from the right and located around $y=0$  will encounter the front stagnation point. Any slight perturbation will displace the particle in the $+ \hat{y}$ or $ -\hat{y}$ directions and direct the particle into the bypassing streamlines bounding the recirculating region. The particle would then be transported to the back stagnation point where once again a slight perturbation can either push the particle into the recirculating region or eject it in the $- \hat{x}$ direction. Sources of perturbations in the experimental system are thermal fluctuations and near field (lubrication) interactions from other particles. In dense passive particle suspension, near field interactions are prominent due to the proximity of particles in space. 
%Interestingly, the role of perturbations, specifically, thermal fluctuations as a mechanism for particles to jump streamlines and be caught at what would otherwise be inaccessible stagnation regions was reported in \cite{van_der_wee_simple_2023}.  
For equal sized spheres, this occurs when the distance between a pair of particles is equal to or less than two particle diameters. This interaction is largely associated to the squeezing of fluid out from between the narrowly separated particles. When two particles are close enough to each other in the stagnation region, where the fluid flow speed vanishes, near field interactions will enable these particles to travel across streamlines away from the stagnation point and enter either the recirculating or bypassing streamlines.

To explain the emergent pattern observed in experiments, we note that $\left< \rho_{\mathrm{passive}}\left( \mathbf{r} \right) \right>_t$ is set by the residence time of the passive particles in the vicinity of the microroller. The emergent pattern arises from the contrast of residence times between the background passive particles and particles that spend more time near the microroller; passive particles that comprise the emergent pattern are those that remain in the vicinity of the microroller for an amount of time $\tau$ greater than the background passive particles. We calculate the background residence time, $\tau_{p} $, which is the maximum time for background passive particles to spend within a square box that envelops the recirculating region of the fluid flow: 
$$\tau_{p} = \frac{L_x}{\left< v_\mathrm{roller} \right> - \sigma_{v_\mathrm{roller}}},$$ 
where $L_x$ is the box length, and $\sigma_{v_\mathrm{roller}}$ is the standard deviation of the microroller's speed in the $\hat{x}$ direction. Interestingly, even in a suspension of passive particles at $\mathrm{T} = 293 \, \mathrm{K}$ we observe deviations in the height of the microroller due to the near field interactions with passive particles. These height fluctuations result in fluctuations in the microroller's velocity. 
Here we choose $L_x = 20 \, \mu \mathrm{m}$ as this is larger than the recirculation zone at this value of the microroller height. We note that the choice of $L_x$ does not matter as long as the extension of the emergent pattern is contained within the box. This is because the variability of residence times only occurs in the regions of non-negligible hydrodynamic interactions near the microroller. 

There are two possible mechanisms for passive particles to have residence times greater than the background time ($\tau > \tau_{p}$): (i) particles traverse the length $L_x$ slower than the microroller, or (ii) particles travel a distance greater than $L_x$ within the square box with area $L_x^2$. Using the passive particle trajectories from our simulations at $\phi_\mathrm{passive} = 0.17$, $ f = 5 \, \mathrm{Hz}$ and, $\mathrm{T} = 0 \, \mathrm{K}$ we identify the set of passive particles where $\tau > \tau_p$. This set corresponds to the tail end of the distribution of residence times $P(\tau)$ as seen in Fig. \ref{fig:xy_mechanism}(b), where the mean of $P(\tau)$ approximately corresponds to $\frac{L_x}{\left< v_\mathrm{roller} \right> }$. In Fig. \ref{fig:xy_mechanism}(c) we plot the set of individual passive particle trajectories with $ \tau \geq \tau_p$ colored by their residence time normalized by $\tau_p$. Recalling that we are analyzing the particle trajectories in the microroller's frame, we observe that the trajectories with $\tau > \tau_p$ clearly replicate the microroller's streamlines as seen in Fig. \ref{fig:xy_mechanism}(a). This shows that near field interactions between passive particles do not qualitatively affect the trajectories expected from the flow field of the microroller. There are two visible regions with contrasting residence times which directly correlate to the two sets of streamlines from Fig. \ref{fig:xy_mechanism}(a), the recirculating and bypassing streamlines. Residence times within the recirculating region are on average 2.5 times greater than those that bypass it. This is further evidenced by Fig. \ref{fig:xy_mechanism}(d) where we have calculated the average spatial velocity profile of the passive particles and observe that passive particles travel at the background speed and sometimes faster than the microroller. There is an additional set of particles with $\tau > \tau_p$; these are the particles whose trajectories correspond to the bypass region near the recirculation zone. As these curved trajectories are longer than the straight trajectories of the background particles, $\tau > \tau_p$ even though these particles move at the same velocity as the background particles. These results suggest that these two regions are responsible for the emergent pattern in experiments. To test this hypothesis, we perform simulations at finite temperature to more accurately replicate experimental conditions. 

%\subsection{Emergent pattern of suspensions at finite temperature}

In order to produce simulation results that can be quantitatively compared to experiments, we carry out simulations at T $= 293$~K (see Fig.~\ref{fig:roller}). From these simulations we calculate a pattern length of $8 \ \mu$m, similar to the experimental length ($8.4 \pm 0.9 \mu \mathrm{m}$) and identical to the pattern length of suspensions at $\mathrm{T} = 0 \, \mathrm{K}$. This agreement is expected as thermal fluctuations are negligible in our system. To demonstrate the agreement between the experimental results and the Stokesian dynamics simulations we directly compare the pattern found in experiments with that found in the simulations; the top half of Fig. \ref{fig:xy_mechanism}(e) is the passive particle distribution from the simulations, while the lower half is that obtained from experiments. We observe that the passive particle streamlines associated to the recirculating and bypassing region in Fig. \ref{fig:xy_mechanism}(e) overlap with the emergent pattern. Moreover, these streamlines also reflect thermal fluctuations in the positions of the passive particles. Therefore, we have shown the emergent pattern from suspensions at finite temperature are well described by advective flows generated by the microroller. 

In summary, we have established that the emergent pattern reveals regions of non-negligible hydrodynamic interactions. We have characterized this by demonstrating passive particle residence times around the microroller are extended due to the recirculating and bypassing streamlines produced by a microroller. 
%
Thus far it has been sufficient to only use information from the $xy$ plane to explain the experimental results. This is due to the fact that the patterns obtained from the experiments are calculated from particle locations projected on the $xy$ plane within the depth of field of the microscope. Experimentally, we lose information away from the focal plane, but particle fluctuations from their average height seem to be negligible. Thus, it is sufficient to only use information from the $xy$ flow plane to trace passive particle trajectories and explain the origin of the emergent pattern.

%Since $f$ and $\phi$ do not affect the flow fields, the passive particle trajectories remain unaffected and so do the projected trajectories to the $xy$ plane.    
%Essentially, in order to change the pattern, we need to change the interaction between the flow field and the passive particles.
%We have already seen one way to achieve it, which is shown in Fig.~\ref{fig:robust}(g-h). In principle, lowering $h_{\mathrm{roller}}$ modifies the flow field, alters where the passive particles start interacting with the flow field, and changes the passive particle trajectories and the emergent pattern. 
%
%In the following two sections, we will provide two methods to modify the interaction between the flow fields and the passive particles. First, we will alter the passive particle height $h_{passive}$ while fixing $h_{roller}$. By doing so, we change where the passive particles start interacting with the flow field, while the flow field is not perturbed. Second, we will change $h_{roller}$ while fixing $h_{passive}$, which changes the flow field fundamentally and therefore changes the interaction.

%So far we have only analyzed the two dimensional flow in the $xy$ plane, ignoring completely the flow perpendicular to the floor. Using the microroller's $xy$ streamlines has been been sufficient to understand the pattern formation, however, we will show how to modify the emergent pattern by analyzing the perpendicular streamlines to the $xy$ plane. This will lead us to understand how different passive particle heights $h_\mathrm{passive}$ will probe different recirculating and bypassing streamlines across different $z$ levels as well as introducing the effects of $xz$ streamlines which we have so far neglected.

%\HLR{Mention comparisons between the papers concerning janus particles and how this is different plus the e coli enhancing diffusivity of the passive particles, i want to relate this to the fact that even though the region of interaction is not changing, we can modify the quantity of interaction/momentum transfer/forcing by the manner of driving the microroller}

%\subsection{Tuning pattern by passive particle height}

\begin{figure*}%[tbhp]
\centering
\includegraphics[width=.98\linewidth]{fig/fig4_test.png}
\caption{\textbf{The microroller's hydrodynamic interactions extend in three dimensions and can be probed by passive particles at different heights.} In the left panel, we plot the $xz$ microroller streamlines at the $(x,0,z)$ plane for a microroller height $ h_\mathrm{roller} = 1.34 \, \mu \mathrm{m}$, and show that not all streamlines (grey curves) intersect the $xy$ stagnation line or saddle line (dashed white curve). Additionally, the saddle line determines the $x$ axis extension of the $xy$ fluid recirculating streamlines at a given height (middle panel). Therefore, different average passive particle heights in suspensions will probe these different recirculating and bypassive streamlines and create different emergent patterns (right panel). We study suspensions with three different particle heights $1.01 \, \mu \mathrm{m}$,  $1.4 \, \mu \mathrm{m}$, and $2.6 \, \mu \mathrm{m}$ which we color code green, yellow and cyan, respectively. In the $xz$ streamlines we bound regions that correspond the the average height of the passive particles and by the $xz$ streamline far from the microroller at the average height of the passive particle. In the middle panel we plot the three different $xy$ streamlines whose spatial extensions are mirrored in suspensions' emergent pattern. We find that the depletion region is present only in the suspensions with particle heights whose bound region near the saddle line is closed.}
\label{fig:xz_mechanism}
\end{figure*}

%So far we have shown that the emergent pattern is the spatial extension of the recirculating and bypassing streamlines generated by a microroller. 

We now consider how the passive particle height, $h_\mathrm{passive}$, influences the emergent pattern. By modulating the passive particle height, we will sample a different region of the microroller flow field, and, consequently, the emergent pattern will change. To illustrate this, we need to analyze the flow field in three dimensions, and understand how the recirculating and bypassing regions change as a function of passive particle height. Additionally, we need to consider how $xz$ streamlines (Fig. \ref{fig:xz_mechanism} left panel) can lift particles above their average height. This will lead passive particles to interact with different sets of streamlines in the $xy$ plane and lead to emergent patterns with different spatial features (Fig. \ref{fig:xz_mechanism} right panel).  

%However, by focusing on the $xy$ plane, and therefore, $xy$ components of the microroller's fluid flow, we neglect entirely in our analysis the nature of the out-of-plane or $z$ component hydrodynamic interactions and their role in the formation of the emergent pattern. Most importantly, an understanding of the streamlines in the $z$ axis and their connection to the $xy$ streamlines is essential as passive particle trajectories at different heights will be impacted differently. This is due to the spatially varying nature of the microroller's hydrodynamic interactions with other particles. 
%whose complex spatial distribution is characteristic of hydrodynamic phenomena. 
%In this section we will expand on how $xz$ streamlines (Fig. \ref{fig:xz_mechanism} left panel) and $z$ levels connect to different recirculating and bypassing streamlines in the $xy$ plane (Fig. \ref{fig:xz_mechanism} middle panel). Whereby, different recirculating and bypassing regions will lead to the formation of different emergent patterns for suspensions of different passive particle heights (Fig. \ref{fig:xz_mechanism} right panel).  

Previously, we identified the existence of two stagnation points at the front and back of the microroller in the $xy$ plane, see Fig. \ref{fig:xy_mechanism}(a). The region bounded between both stagnation points is the recirculating region of the microroller streamlines in the $xy$ plane and gives rise to the emergent pattern. To understand how the emergent pattern changes with respect to the $z$ axis, it is sufficient to track the front stagnation point as a function of the height from the floor. In Fig. \ref{fig:xz_mechanism}, we show the microroller's $xz$ streamlines at $y = 0$ in the microroller's frame. We superimpose the calculated $xy$ velocity component stagnation line (dashed white line) which correlates to the extension of the recirculating region on the $xy$ plane. Note that the streamlines intersect the stagnation line, meaning the $z$ component velocity is not zero. This curve is not a true stagnation line as not not all velocity components are zero, therefore, it is a saddle line. 

The saddle line's $x$ axis extension varies as a function of $z$. This indicates that the spatial extension of the emergent pattern changes by tuning the height of the passive particles. This is because passive particles will sample different sections of the saddle line. We note that the saddle line is uniquely determined by $h_\mathrm{roller}$. Here we continue to solely focus on passive particle suspensions in water with $h_\mathrm{roller} =  1.34 \, \mu \mathrm{m}$. 
%At these conditions, the saddle line indicates the lack of recirculating and bypassing regions at heights above $8 \, \mu \mathrm{m}$. From this we can infer that an emergent pattern will not form if the average passive particle height is also above $8 \, \mu \mathrm{m}$.

In order to demonstrate the degree of tuning of the emergent pattern's extent with respect to the height of the passive particles, we perform sets of simulations at $\mathrm{T} = 0 \, \mathrm{K}$ with different average passive particle heights $\left < h_\mathrm{passive} \right > $ at constant $\left < h_\mathrm{roller} \right > $. In athermal suspensions, only hydrodynamic forces and the interplay between gravitational and charge repulsion from the floor will change particles' height. We tune $\left < h_\mathrm{passive} \right > $ by varying their excess mass $m_\mathrm{passive}$ with respect to water, and focus on three different $\left < h_\mathrm{passive} \right > $. In suspensions without the presence of a microroller, passive particles have an average height of $1.01 \, \mu \mathrm{m}$,  $1.4 \, \mu \mathrm{m}$, and $2.6 \, \mu \mathrm{m}$ for passive particle excess masses of $10 m_\mathrm{passive}$, $m_\mathrm{passive}$, and with $1/100 m_\mathrm{passive}$, respectively.  

As we have previously stated, suspensions with different $h_\mathrm{passive}$ will form distinct emergent patterns given the curvature of the saddle line. However, we must consider how the $xz$ streamlines in Fig. \ref{fig:xz_mechanism} will impact where passive particles intersect the saddle line. Near the saddle line, $xz$ streamlines can lift particles above their $\left < h_\mathrm{passive} \right >$ and lead to intersect at $z > h_\mathrm{passive}$. In Fig. \ref{fig:xz_mechanism} left panel, we color regions  green, yellow, and cyan, to indicate the range of heights where passive particles will most likely intersect the saddle line for $\left < h_\mathrm{passive} \right > =  \{1.01, 1.4,2.6\} \, \mu \mathrm{m}$, respectively. The lower bound of the colored regions is defined by the particle's $\left < h_\mathrm{passive} \right >$, and the upper bound corresponds to the $xz$ streamline far from the microroller at the respective $\left < h_\mathrm{passive} \right >$. At a distance far from the microroller, $\sim 30 \, \mu \mathrm{m} $, the microroller's hydrodynamic interactions have sufficiently decayed such that the $xz$ streamlines are parallel to one another. This simply represents the background flow which in this reference frame corresponds to the microroller's velocity. 

In general, all $xz$ streamlines curve upwards as approaching the saddle line. This leads to multiple $xz$ streamlines overlapping the different colored regions that passive particles will most likely intersect the saddle line for a given average height. Out of simplicity, we select to intersect the saddle line at the lower bound of this region, the average passive particle height, and plot the $xy$ streamlines around the microroller at this height, see Fig. \ref{fig:xz_mechanism} middle panel (where we show and color the borders of the $xy$ streamlines green, yellow, and cyan to indicate average passive particle heights $\left < h_\mathrm{passive} \right > =  \{1.01, 1.4,2.6\} \, \mu \mathrm{m}$, respectively). As expected, the spatial extension of the recirculating streamlines at different $z$ levels follow the saddle line $x$ values for a given height. In order to show that the emergent patterns obtained from athermal suspensions at different passive particle heights mirror their respective $xz$ streamlines, we plot their $\left < \rho_{\mathrm{passive}}\left( \mathbf{r} \right) \right >_t$ obtained from simulations at $\mathrm{T} = 0 \, \mathrm{K}$
%, where the green, yellow, and cyan borders indicate suspensions with average passive particle heights $1.01 \, \mu \mathrm{m}$,  $1.4 \, \mu \mathrm{m}$, and $2.6 \, \mu \mathrm{m}$, respectively
, see Fig. \ref{fig:xz_mechanism} right panel. For all cases, emergent patterns mirror the  $L_\mathrm{pattern}$ of their bypassing streamlines in the $xy$ plane. However, we find a wide variety in the geometry of the emergent pattern. Only suspensions of passive particles with $\left < h_\mathrm{passive} \right > = \{1.01,1.4 \} \, \mu \mathrm{m}$ have a depletion region behind the microroller. Meanwhile, suspensions composed of passive particles with $\left < h_\mathrm{passive} \right >= 2.6 \, \mu \mathrm{m}$ produce an emergent pattern without a depletion region. This difference is produced by the nature of the $xz$ streamlines and their ability to lift particles above their average height. Suspensions that produce emergent patterns with an inverted c-shape structure correspond to passive particle heights in which the majority of the particles will intersect with the saddle line. This can be observed in the colored bounded regions in Fig. \ref{fig:xz_mechanism} left panel, where all $xz$ streamlines in the green and yellow regions are also fully bounded by the saddle line. It is the cyan region which corresponds to passive particles with $\left < h_\mathrm{passive} \right >= 2.6 \, \mu \mathrm{m}$ that contains $xz$ streamlines that miss the saddle line. Thus, not all passive particles intersect with the saddle line and avoid interacting with the recirculating and bypassing streamlines. By not entering the recirculating region, the $xz$  streamlines will transport passive particles above and around the microroller and occupy the region where the depletion region is seen at lower $\left < h_\mathrm{passive} \right >$. 

Moreover, we hypothesize that the origin of the depletion region arises from passive particles interacting with one another and shielding streamlines that do not close at the rear of the microroller. In this manner, passive particles do not reach the back of the microroller and replenish the particles that have been pushed aside through hydrodynamic interactions at the front of the microroller. It is for this reason that the width of the depletion region is also a function of the extension of the recirculating region, that is larger recirculating regions will create a larger depletion region widths as seen in the emergent patterns in Fig. \ref{fig:xz_mechanism} right panel.  

Overall, no matter $\left < h_\mathrm{passive} \right > $, all emergent patterns mirror the extension of the recirculating and bypassing regions in the $xy$ plane at their respective height. We should thus expect that for $\left < h_\mathrm{passive} \right > $ above the saddle line no emergent pattern will appear as there would be no region of non-negligible hydrodynamic interactions for the passive particles to sample. Importantly, we have shown that the emergent pattern can be controlled by modifying the height of the passive particles. As we move up from the floor, both recirculating and bypassing regions should increase until they begin to decay as we probe heights further away from the microroller. 

%This is evidenced by the lack of the saddle line above $\sim 8 \, \mu \mathrm{m}$. 

%\subsection{Microroller height tunes of the pattern length}

As the saddle line correlates well with the extension of the emergent pattern, we propose $L_\mathrm{pattern}$ to be the distance between the microroller and the saddle line $L_\mathrm{saddle}$. As previously quantified in Fig. \ref{fig:robust}(f), $L_\mathrm{pattern}$ does not change when varying the suspension's passive particle area fraction nor the microroller's velocity, it only changes with the height of the microroller, as does the saddle line. However, to compare $L_\mathrm{saddle}$ and $L_\mathrm{pattern}$, one ambiguity persists, at which height to calculate $L_\mathrm{saddle}$? By analyzing the $xz$ streamlines we have identified a set of bounds for a given $\left < h_\mathrm{passive} \right >$ that delimits the heights passive particles will be driven to by the microroller's $xz$ streamlines. Using these bounds, and knowing both $\left < h_\mathrm{roller} \right >$ and $\left < h_\mathrm{passive} \right >$, we can provide an interval for $L_\mathrm{saddle}$ to compare with $L_\mathrm{pattern}$ obtained from simulations and experiments to show that these two quantities are equivalent. 

We perform simulations with a microroller at different $\left < h_\mathrm{roller} \right >$ and calculate $L_\mathrm{pattern}$. Furthermore, we compare and parameterize $L_\mathrm{pattern}$ by $L_\mathrm{saddle} \left( z ;\left <  h_\mathrm{roller} \right>)\right)$, where $z$ is the height above the floor. 
As previously discussed, the pattern length has a strong dependency on passive particle height for a given microroller height due to the curvature of the saddle line. Therefore, in simulations at $\mathrm{T} = 0 \mathrm{K}$ we keep the ratio between the passive particle and microroller height approximately constant, $ \left < h_\mathrm{passive} \right > / \left <  h_\mathrm{roller} \right> \approx 1.2 $, to only focus on the effects of $h_\mathrm{roller}$ on the emergent pattern. Meanwhile, the average height of the passive particles in finite temperature suspensions is $ \left < h_\mathrm{passive} \right >  = 2.5 \, \mu \mathrm{m}$. 
%and was fixed by the interplay between the temperature of the fluid and density of the passive particle, making any electrostatic repulsion with the floor negligible. The height distribution of passive particles in suspensions at finite temperature is exponential which is expected. 
%Near the region of the saddle line, there are multiple $xz$ streamlines that cross the suspension's average passive particle height which adds uncertainty as to at what $z$ level to cut the saddle line. 
%The lower bound is simply $L_\mathrm{saddle} \left( \left < h_\mathrm{passive} \right> ;\left <  h_\mathrm{roller} \right>)\right)$, where $\left < h_\mathrm{passive} \right>$ is the average passive particle height of the suspension. While the upper bound corresponds to $L_\mathrm{saddle}$ evaluated at the height at which the $xz$ streamline far from the microroller but at the average passive particle height intersects the saddle line. 
In Fig. \ref{fig:L}, we plot $L_\mathrm{pattern}$ and the $L_\mathrm{saddle}$ region, and show that the $L_\mathrm{saddle}$ region is a good descriptor for the emergent pattern obtained from suspensions at $\mathrm{T} = 0 \, \mathrm{K}$, and $\mathrm{T} = 293 \, \mathrm{K}$. Additionally, the extension of the pattern length and $L_\mathrm{saddle}$ decreases as the microroller approaches the no-slip boundary. This is expected, as moving closer to the surface effectively screens the hydrodynamic interactions, reducing their extent. The opposite is true for a microroller further away from the surface. However, as the microroller's height is increased the coupling between the microroller's rotation and translation diminishes until the translation velocity becomes infinitesimally small. In a suspension at finite temperature, thermal fluctuations would then disrupt the pattern created by advective flows. 

\begin{figure}%[tbhp]
\centering
\includegraphics[width=0.8\linewidth]{fig/fig_5_draft_4.png}
\caption{\textbf{Tuning the pattern length by changing the microroller height.} We show that the pattern length $L_\mathrm{pattern}$ directly correlates to the distance between the microroller and its front saddle line $L_\mathrm{saddle} \left( z ;\left <  h_\mathrm{roller} \right>)\right)$ (blue region). As previously stated, the saddle line is a function of height, and multiple streamlines in the $xz$ plane intersect a given height which complicates which height to choose to calculate $L_\mathrm{saddle}$. However, we bound the (blue) $L_\mathrm{saddle}$ region by using the bounds determined and shown in Fig. \ref{fig:xz_mechanism}.
\\%
}
\label{fig:L}
\end{figure}

%------------------------------------------------
%\subsection{Summary}
%------------------------------------------------
In conclusion, we have demonstrated the ability to use a driven particle to create a large-scale (10 times the particle radius), asymmetrical 3D pattern from a quasi-2D colloidal suspension. The pattern includes an accumulation region with its center being several particle sizes away from the microroller, and a depletion region along the microroller trajectory.
This pattern is created via hydrodynamic interactions, and is unmodified by thermal fluctuations, passive particle area fraction, or driving velocity. 
%
We show two main pathways to modify the pattern by altering the hydrodynamic interactions between the microroller and the passive particles. This can be done by tuning the height of the passive particles in the suspension with respect to the microroller, or by modifying the height of the microroller with respect to the floor. 
%
Our analysis of the microroller-driven advective flow that generates the pattern demonstrates that the extension of the emergent pattern is equivalent to the distance between the microroller and the flow's saddle line. Thus, modifying the average height of the microroller changes the pattern's size as it changes the features of its fluid velocity profile. Additionally, by modifying the height of passive particles in a suspension at constant microroller height, the particles are able to sample other planes of the non-negligible hydrodynamic regions, defying the pattern extent, and demonstrating the three dimensional nature of the microroller's streamlines. 

Our analysis reveals that the pattern scale is determined by equilibrium quantities: the microroller height and the height distribution of the passive particles. Thus, the size scale of the emergent pattern provides an alternative pathway to determine an approximate average passive particle height in a suspension if the $h_\mathrm{roller}$ is known. This analysis is straightforward, is not computationally demanding, and offers a new tool for studying fluid-mediated interactions of driven particles.
%
If there are weak or transient interaction between the passive particles (for example in a colloidal gel), this pattern formation could be exploited for material restructuring. %\cite{massana-cid_active_2018,omar_swimming_2019,saud_yield_2021}.
This system also offers an alternative way to do active microrheology \cite{zia_active_2018}. For example, we observe that a microroller moves slower when it is in a colloidal suspension than when it is in a pure fluid. One can thus calculate the effective viscosity of the colloidal suspension by measuring the change of the microroller speed as a function of colloidal volume fraction, and measure density fluctuations by measuring roller velocity fluctuations.
%However, our system differs from other active microrheolgoy techniques since the microroller provides not only translation but also rotation interaction, and the 3D structure plays a significant role in the system. 
Additionally, similar principles of microroller streamlines can perhaps be used to explain how a mixture of passive particles and biologically active swimmers lead to anomalous transport coefficients of the passive particles via hydrodynamic interactions \cite{mino_enhanced_2011, jepson_enhanced_2013}. 
Finally, we note that when the passive particles enter the recirculating steamlines, they are trapped and move together with the microrollers. Therefore, microrollers that generate these streamlines, or microvortices, have the potential to transport micron-size particles. %\cite{zhang_targeted_2012}.

%------------------------------------------------
\section*{Materials and Methods}
%------------------------------------------------
\subsection*{Experiments}
The passive particles are spherical and made of polystyrene (Bangs laboratory\copyright, FSPP005) with a density of $ 1.06$~g/cm$^3$ and a mean diameter of $= 2.07 \pm 0.15 \, \mu \mathrm{m}$. 
The microrollers in the experiment are described in detail in \cite{van_der_wee_simple_2023} and \cite{sprinkle_driven_2020}. 
The microroller has a mean diameter of $2.1 \pm 0.1 \, \mu \mathrm{m}$ and a permanent magnetic dipole as it is comprised of a hematite cube within a spherical polymer matrix, see Fig.~\ref{fig:roller}(b). The mean density of the microroller is $1.74$~g/cm$^3$.
%In short, the microroller's diameter is ($2.1 \pm 0.1 \, \mu \mathrm{m}$) but much higher density ($\approx 5$~g/cm$^3$) than the passive ones. The microroller has a permanent magnetic dipole as it is comprised of a hematite cube within a spherical polymer matrix, see Fig.~\ref{fig:roller}(b). 
%Detailed information can be found in \cite{van_der_wee_simple_2023}.
We clean the passive particles by replacing the solution with DI water for three times. Then we add a small amount of microroller solution to the passive particle solution, and mix the solution with a vortex followed by a sonicator. We withdraw the mixture solution with a capillary tube, and seal the tube entrance with glue. Then, we mount the sample on a microscope, and check the area fraction of passive particles after all particles sediment to the floor. Finally, we apply rotating magnetic fields and record the particle distribution with the fluorescent microscope. The microrollers and the passive particles have different fluorescent wavelength, giving us the ability to separate the two types of particles in two channels.

The system and the mechanism to drive a microroller is described in detail in \cite{sprinkle_driven_2020}. In short, we use two pairs of Helmholtz coils to generate an external rotating magnetic field ($100$~G). The permanent dipole of a driven particle experiences the torque from the external field, causing the driven particle to rotate synchronously with the field. As the microroller is near the floor, the flow generated by the rotating driven particle becomes asymmetrical due to the non-slip boundary of the floor, which causes the microroller to translate.
%At higher frequencies, the microroller ceases to follow the external field in a synchronous manner, and exhibits a decay in its velocity upon greater values of the field's frequency. The rotationally driven particle translates perpendicular to its axis of rotation, see SI.
We trace the location of the microrollers using Python and the package \href{https://soft-matter.github.io/trackpy/v0.6.1/}{Trackpy}, which we use to generate a sequence of images around the microroller. We then use the position of the microroller to shift the coordinates of all images to the microroller frame.
%As the window is moving with the microroller, we shift our observation from the lab frame to the microroller's frame.

\subsection*{Simulations}

As we have shown, the predominant interaction between a microroller and passive particles in suspension are hydrodynamic in nature. To correctly quantify these interactions we simulate these systems using lubricated corrected Brownian dynamics \cite{sprinkle_driven_2020}. In this method the position and orientation of a particle $\mathbf{q}_1 = \{ \mathbf{x}, \mathbf{\theta} \}$ are evolved by 
\begin{equation}
    \frac{\mathrm{d} \mathbf{Q}}{\mathrm{d} t} = \overline{\mathbf{M}} \mathbf{F} + k_\mathrm{B} \mathrm{T} \partial_\mathbf{Q} \cdot \overline{\mathbf{M}} + \sqrt{2 k_\mathrm{B} \mathrm{T} \overline{\mathbf{M}}} \, \mathcal{\mathbf{W}}(t)
\label{eq:BD}
\end{equation}
where $\mathbf{Q} = \left [ \mathbf{q}_1, \mathbf{q}_2, \dots, \mathbf{q}_n \right ]$ is the vector containing the the individual positions and orientations of all particles. Here, pairwise hydrodynamic interactions between particles are determined by their configuration in space and contained in the lubrication corrected mobility matrix $\overline{\mathbf{M}} \left ( \mathbf{Q} \right )$. The magnitude of these interactions are weighted by deterministic properties governed by external forces $\mathbf{f}$ and torques $\mathbf{\tau}$ acting on particles in solution, and stochastic properties arising from the presence of thermal fluctuations. The first term with respect to the right of Eq. \ref{eq:BD} details its deterministic character, here $\overline{\mathbf{M}}$ is weighted by the vector comprised of individual external forces and torques on impinging on all particles $\mathbf{F} = \left [ \mathbf{f}_1, \mathbf{\tau}_1, \dots, \mathbf{f}_n, \mathbf{\tau}_2, \right ]$. The second and third terms of the equation deal with the thermal drift, and random walk nature of the of the thermal fluctuations, respectively. Here, $k_\mathrm{B}$ denotes the Boltzmann constant, $\mathrm{T}$ indicates the solvent temperature, and $\mathcal{\mathbf{W}}(t)$ represents a Wiener process or a collection of independent white noise processes essential for the generation of Brownian velocities. Given that this is a stochastic differential equation, we temporally integrate this equation using a stochastic scheme, specifically the ‘Stochastic Trapezoidal Split’ (STS) scheme \cite{sprinkle_driven_2020}. In this paper we perform simulations of Eq. \ref{eq:BD} evolved by the STS scheme using a publicly accessible code found on github at \href{https://github.com/stochasticHydroTools/RigidMultiblobsWall}{RigidMultiblobsWall}. More details including the accuracy of this scheme, and pre-conditioners employed in the solution for $\overline{\mathbf{M}}^{1/2}$ can be found in \cite{sprinkle_driven_2020}.  

To simulate suspensions of passive particles with a microroller, we model passive particles and the microroller as spherical rigid particles with radii $R_\mathrm{passive} = 1.0 \, \mu \mathrm{m}$ , and $R_\mathrm{roller} = 1.0 \, \mu \mathrm{m}$, respectively. The particles are immersed in water at $T = 293.15 \, \mathrm{K}$, and they have a buoyant mass of $m_\mathrm{passive} = 2.5 \, \times \, 10^{-16} \, \mathrm{kg} $, and $m_\mathrm{roller} = 3.1 \, \times \, 10^{-15} \, \mathrm{kg}$, respectively. We perform three dimensional simulations with an initial condition set by fixing the area fraction of passive particles and randomly populating passive particles in a two dimensional strip of $250 \, \mu \mathrm{m}$ in length and $75 \, \mu \mathrm{m}$ width at $z = 1.5 \, \mu \mathrm{m}$. Finally, we place a non-rotating microroller to the left of the strip, and equilibrate particle positions by evolving the system for approximately $60 \, \mathrm{s}$, after which we rotate the microroller at constant angular velocity using the algorithm detailed in \cite{sprinkle_driven_2020}. For all instances of the simulations we use a time step of $\Delta t = \, 0.05 \, \mathrm{s}$. 

In this paper, all particles experience gravitational forces given their excess mass, and electrostatic repulsion with the lower surface. We model the electrostatic repulsion using the Yukawa type potential:
\begin{equation}
    U (h) = \begin{cases}
    \epsilon \exp{\left ( \frac{R - h}{\kappa}  \right)} & \text{if}\; h > R\\
    \epsilon \left ( 1 - \frac{R - h}{\kappa} \right ) & \text{if} \; h < R
    \end{cases}
    ,
    \label{eq:yukawa}
\end{equation}
where $h$ is the particle center to floor distance, and $R$ is the radius of the respective particle. For the microroller, we set the magnitude of the potential $\epsilon$, and its screening factor $\kappa$ that simultaneously matches the height that replicates its experimentally obtained velocity profile $\left ( h_\mathrm{roller} = 1.34 \, \mu \mathrm{m} \right )$, and its measured diffusion coefficient, $D_\mathrm{roller} = 0.15 \, \mu \mathrm{m}^2/\mathrm{s}$. For the passive particles, we assume the same $\kappa$ as the microroller, and instead fit $\epsilon$ to match its experimentally obtained diffusion coefficient, $D_\mathrm{roller} = 0.015 \, \mu \mathrm{m}^2/\mathrm{s}$. The list of parameters used in the simulations can be found tabulated in the SI. Additionally, as the suspension is located above a no-slip wall, hydrodynamic interactions are calculated using Blake's solution of the Green's function solution to the Stokes equation above a no-slip wall generalized for spherical rigid particles \cite{Blake1971,Swan_no_slip}. However, this Green's function only correctly describes far field hydrodynamic interactions between all pairs of surfaces, particle-particle and particle-wall. To include near field hydrodynamic interactions related to the squeezing of fluid between pairs of surfaces we use the previously mentioned lubrication corrected mobility matrix $\overline{\mathbf{M}}$ detailed in \cite{sprinkle_driven_2020}. Moreover, we use two different cutoffs that determines at which distance between surfaces at which to calculate either near field hydrodynamic interactions or far field hydrodynamic interactions. For particle-particle interactions the cutoff distance is $r \leq 5 \, \mu \mathrm{m}$ where for distances greater than $5 \, \mu \mathrm{m}$ we simply calculate interactions with the Green's function solution. Meanwhile, we calculate near field hydrodynamic interactions for particle-wall interactions for any distance above the wall. For more information about the implementation of the resistance scalars for near field interactions can be found in \cite{sprinkle_driven_2020}. We additionally complement near field hydrodynamic interactions with a short-ranged steric repulsion potential $U_\mathrm{cut} \left (r \right)$ with the Yukawa type potential of Eq. \ref{eq:yukawa} for particle-particle and particle-wall interactions. In the case for particle-particle interactions we substitute $R = 2R \,(1 - \delta_\mathrm{cut})$, while $R= R \, (1 - \delta_\mathrm{cut})$ for particle-wall interactions, where $\delta_\mathrm{cut} = 10^{-3} \, \mu \mathrm{m}$. The complete set of parameters used in $U_\mathrm{cut} \left (r \right)$ are also tabulated in the SI.

Additionally, we calculate $\left < \rho_\mathrm{passive} \right >_t$ averaging over at least 10 different simulation runs using a 30 $\mu \mathrm{m} \, \times \, 30  \mu \mathrm{m}$ mesh with bin width $\Delta L = 0.25 \mu \mathrm{m}$. We choose to average over frames $0.1 \, \mathrm{s}$ apart and where the microroller is $L_x \in [20,230] \, \mu \mathrm{m}$. Under these bounds the microroller is within the region of passive particles with a given area fraction $\phi$. We also use this implementation to avoid averaging over regions outside the bounds of the suspension which affects the formation and dimensional features of the emergent pattern. After calculating $\left < \rho_\mathrm{passive} \right >_t$ we use Gaussian smoothing with a variance the size of the passive particle radius to include particle areal size effects. This allows comparison between $\left < \rho_\mathrm{passive} \right >_t$ and experimental emergent patterns now that $\left < \rho_\mathrm{passive} \right >_t$ contains information of the passive particle size and loosens the constraints on the distributions obtained by using a homogeneous binning mesh of $0.25, \mu \mathrm{m}$. We calculate all velocity distributions of a microroller also using the Blake's solution to the Green's function of a stokeslet above a no-slip wall generalized for a spherical particle \cite{Swan_no_slip}.

%------------------------------------------------
\section*{Acknowledgments}
We thank Brennan Sprinkle for fruitful discussions concerning Brownian motion of low density particles, and Bhargav Rallabandi for discussions on the P\'eclet number and tracers. H.L.-R. acknowledges support from a MRSEC-funded Graduate Research Fellowship, (DMR-2011854). This work was primarily supported by the University of Chicago Materials Research Science and Engineering Center, which is funded by National Science Foundation under award number DMR-2011854.

\section*{Author Contributions}
%E.B.W. and M.M.D. conceived and designed research; 
%E.B.W., B.D. and M.M.D. wrote the manuscript.

\section*{Competing Interest}
The authors declare no competing interest.
\section*{Data and Materials Availability}
The data sets generated during and/or analysed during the current study are available from the corresponding authors upon reasonable request.
%Here you should list the contents of your Supplementary Materials -- below is an example. 
%You should include a list of Supplementary figures, Tables, and any references that appear only in the SM. 
%Note that the reference numbering continues from the main text to the SM.
% In the example below, Refs. 4-10 were cited only in the SM.     
\section*{Supplementary materials}
Supporting File Legends\\
Figs. S1 to S3\\
Tab. S1 to S5\\
%Vids. S1 to S5\\
%File S1\\
% Your references go at the end of the main text, and before the
% figures.  For this document we've used BibTeX, the .bib file
% scibib.bib, and the .bst file Science.bst.  The package scicite.sty
% was included to format the reference numbers according to *Science*
% style.

%BibTeX users: After compilation, comment out the following two lines and paste in
% the generated .bbl file. 
\bibliography{sciadvbib}
\bibliographystyle{ScienceAdvances}


\newpage


\end{document}




















