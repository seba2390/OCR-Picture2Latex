\section{Discussion}\label{discussion}
We propose a new  fully automated fast adaptive 
smoothing and thresholding algorithm suite called FAST with the
ability to detect activation in  low-signal settings. Three
variants -- ALL-FAST, AM-FAST and AR-FAST -- are proposed with AR-FAST
generally recommended because of its consistent good performance
across a range of low-CNR experiments and real datasets.
 AM-FAST's performance, while good, is more
variable, while   ALL-FAST appears to undersmooth but performs better
in two-sided activation detection scenarios. Our methodology realistically
accounts for both spatial correlation structure and is also developed
under more accurate extreme value theory. Our algorithm suite is
implemented in a R package {\sc RFASTfMRI} available at
\href{https://github.com/ialmodovar/RFASTfMRI}{https://github.com/ialmodovar/RFASTfMRI} and is fully automated with one threshold choice for which we provide 
easily-implemented guidance. This contrasts with AS and AWS that
require setting  maximum smoothing bandwidths related to the expected
diameter of activated regions~\cite{polzehletal10} -- a determination
that may require considerable dexterity and is ambivalent when different-sized
activation regions are expected.

A reviewer has pointed to the joint detection-estimation
literature~\citep{maknietal05,maknietal06} where estimation of the HRF
and activation detection occur jointly. The FAST, AS and AWS algorithms
can be placed  in a related framework, with the distinction that the
estimation step is of a more spatially consistent (smoothed)
SPM. We also agree with another reviewer on other ways of ensuring
spatial contiguity such as through Markov Random Field
priors~\citep{ngetal12} and on the need to incorporate
approaches also allowing for nonhomogeneous smoothing.
%We believe that the results using FAST, will hold since the analysis
%is done in the SPM rather than the dependence of HFR.
Our algorithms converge by construction and are guaranteed to
terminate. They are also seen to have good overall performance  but,
as observed by a  reviewer, establishing the 
optimality   properties  and conditions and
assumptions governing such properties may 
provide more solid theoretical grounding for FAST and improve its
understanding and widen its applicability. %Such development is beyond the scope
                                %of this paper. 
Developing FAST for more  sophisticated time series  and spatial
models, including in the 
context of complex-valued 
fMRI~\citep{adrianetal18} as well as increased use of diagnostics in
understanding activation and cognition are other important research areas
and directions that would benefit from further attention.



