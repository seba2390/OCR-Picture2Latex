\begin{abstract}
Functional Magnetic Resonance Imaging is a noninvasive tool for
studying cerebral function. Many factors challenge activation
detection, especially in low-signal scenarios
that arise in the performance of high-level cognitive tasks. We provide a fully
automated fast adaptive smoothing and thresholding (FAST) 
algorithm that uses smoothing and extreme value theory on correlated
statistical parametric maps for thresholding. Performance on experiments spanning a range of low-signal settings is very encouraging. The methodology also performs well in a study to identify the cerebral regions that perceive only-auditory-reliable or only-visual-reliable speech stimuli.
\begin{comment}


Assigning a general linear model to each voxel is a
powerful to tool to study this fMRI image. However, assuming the wrong
model can lead to severe misleading results. In here, we carried a
voxelwise analysis and using the classical Bayesian information
criterion ad we find the most reasonable model in a range of propose
ones. Using the summaries from these best models we obtained
statistical parametric maps and using the structural adaptive
segmentation approach we account for the spatial structure while
performing smoothing. We carried extensive simulation studies with
different proposed models and determine the proportion of activation
in each scenario. %Lastly, we re-analyzed the datasets of imagination
                  %from a single subject, single-run experiment and as
                  %a multi-task and multi-stimulus we analyze the
                  %\cite{moran2010social} false-belief experiments. 
\end{comment}
\end{abstract} 
