\section{Introduction}\label{sintro}

\IEEEPARstart{F}{unctional} Magnetic Resonance Imaging (fMRI)
\citep{belliveauetal91,kwongetal92,bandettinietal93,fristonetal95,howsemanandbowtell98,pennyetal06,lindquist08,lazar08,ashby11}
studies the spatial characteristics and extent of brain
function while at rest or, more commonly, while performing tasks or 
responding to external stimuli. The latter scenario is the
setting for this paper. Here, the imaging modality acquires voxel-wise Blood-Oxygen-Level-Dependent~(BOLD)  
measurements~\citep{ogawaetal90a,ogawaetal90b} at rest and during
stimulation or performance of a task. After pre-processing, a general
linear or other statistical model~\citep{fristonetal95,worsleyetal02} is fit to
the time course sequence against the expected BOLD 
response~\citep{fristonetal98,glover99,buxtonetal04}. Statistical
Parametric Mapping~(SPM)~\citep{fristonetal90} techniques provide
voxel-wise test statistics summarizing the association between the time
series response at each voxel and the expected BOLD  response
\citep{bandettinietal93}.  
The map of test statistics is then thresholded to identify  significantly 
activated voxels~\citep{friston1994statistical,worsley1995analysis,genoveseetal02}.   The analysis of fMRI datasets is
challenged~\citep{hajnaletal94,biswaletal96,woodetal98,gullapallietal05}
by factors such as scanner, inter- and intra-subject variability,
voluntary/involuntary or stimulus-correlated motion and also the
several-seconds delay in the BOLD response as the  neural
stimulus passes through the hemodynamic
filter~\citep{maitraetal02,gullapallietal05,maitra09b}. Pre-processing 
~\citep{woodetal98,saadetal09} mitigates some of these effects,
but additional challenges are presented by the fact that an fMRI study
is expected to have no more than 1-3\% activated voxels~\citep{chenandsmall07,lazar08}. Also, many activation 
studies involving  high-level cognitive processes have low
contrast-to-noise ratios (CNR), throwing up additional challenges as illustrated next.

\subsection{Activation Detection during Noisy Audiovisual Speech}
\label{intro:av} 
The most important %method of human  communication
form of human communication is
speech~\citep{kryter94,hauser96,dupontandluettin00}, 
which the brain is  adept at understanding even in noisy
surroundings. This ability may be due~\citep{nathandbeauchamp11}
to the brain's capacity  for  
multisensory integration of independently-acquired visual and auditory
input information which reduces noise and allows for more accurate
perception~\citep{sumbyandpollack54,steinandmeredith93}. Recently,
\citet{nathandbeauchamp11} studied the role of the superior 
temporal sulcus (STS) in  perceiving noisy speech, through fMRI
and behavioral experiments, and established increased connectivity
between the STS and the auditory or the visual cortex depending on
whichever modality was more reliable, that is,
less noisy.

\citet{nathandbeauchamp11} provided results on regions of interest
(ROIs) drawn on the STS and the 
auditory and visual cortices. However, the full benefit of  fMRI can
be realized only if we move beyond assessing
cerebral function at the ROI level to understanding it at the voxel
level.  Reliable voxel-wise activation detection in individual
subjects may increase the adoption 
of fMRI in a  clinical setting. All these are potential scenarios with
low CNRs where accurate activation detection methods are needed.
%We return to this application in Section~\ref{sec:Apps}.

\subsection{Background and Current Practices}
\label{background}
Many thresholding
methods~\citep{helleretal06,benjaminiandheller07,smithandfahrmeir07,smithandnichols09,wooetal14}
in fMRI address multiple testing issues in determining significance of 
test statistics but  ignore spatial resolution.
Acquired images are instead often spatially smoothed prior to
analysis, but such  non-adaptive smoothing reduces both the adaptive spatial 
resolution and the number of 
available independent tests for activation
detection~\citep{tabelowetal06}. There are also iterative adaptive 
smoothing and segmentation methods such as 
propagation-separation (PS)~\citep{tabelowetal06} and
adaptive-segmentation (AS)~\citep{polzehletal10} that essentially 
segment the SPM into  activated and inactivated voxels. PS
approximately yields a random $t$-field and uses Random Field Theory
for segmentation while  AS uses multi-scale
testing. \citep{polzehletal10} argued for AS because of its more general 
development and fewer model assumptions. AS also requires no heuristic
corrections for spatial correlation, provides decisions at 
prescribed significance levels and showed~\citep{polzehletal10} better
performance over PS in an auditory experiment. However, 
AS requires pre-specified bandwidth sequences and ignores correlation
within the SPM. So Section!\ref{sec:methodology} of this paper
develops theory and methodology for fully automated Fast Adaptive 
Smoothing and Thresholding (FAST) algorithms that account for correlation and
obviate the need for setting all but one parameters. Performance 
evaluations on real datasets and large-scale 
simulation experiments are in Section~\ref{sec:simulation}. %  on
                                %  real datasets and large-scale
                                %  simulation experiments.
Section~\ref{sec:Apps} revisits  the dataset of 
Section~\ref{intro:av}, while Section~\ref{discussion} provides
discussion. 
%an online supplement with additional illustrations and results
%of our experiments and data analysis:  sections, tables and figures in the
%supplement all have the prefix ``S-''.
A supplement with sections, figures and tables referenced
using  the prefix ``S'' is available. %http://dx.doi.org/10.1109/TMI.2019.xxxxxx.
