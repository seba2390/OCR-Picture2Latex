\section{Related Work}
\label{sec:related}

%{\color{blue} Status: Revised by Chuan/Olga}

\textbf{Complex Event Processing Systems} (CEP) have gained popularity in the recent years~\cite{esper,flink,streaminsight,oracle}. Some approaches use a Finite State Automaton (FSA) as an execution framework for pattern matching~\cite{ADGI08,DGPRSW07,WDR06,ZDI14}. Others employ tree-based models~\cite{MM09}. Some approaches study lazy match detection~\cite{KSS15}, compact event graph encoding~\cite{PLAR17}, and join plan generation~\cite{KS18join}. We refer to the recent survey~\cite{Giatrakos2020} for further details.
%
While these approaches support trend aggregation, they construct trends prior to their aggregation. Since the number of trends is exponential in the number of events per window~\cite{QCRR14, ZDI14}, such two-step approaches do not guarantee real-time response~\cite{PLRM18,PLRM19}. Worse yet, they do not leverage sharing opportunities in the workload. The re-computation overhead is substantial for workloads with thousands of queries.


\textbf{Online Event Trend Aggregation.} Similarly to single-event aggregation, event trend aggregation has been actively studied. A-Seq~\cite{QCRR14} introduces online aggregation of event sequences, i.e., sequence aggregation without sequence construction. \greta~\cite{PLRM18} extends A-Seq by Kleene closure. Cogra~\cite{PLRM19} further generalizes online trend aggregation by various event matching semantics. However, none of these approaches addresses the challenges of multi-query workloads, which is our focus.


\textbf{CEP Multi-query Optimization} follows the principles commonly used in relational database systems~\cite{Sellis:1988}, while focusing on pattern sharing techniques. RUMOR~\cite{hong2009rule} defines a set of rules for merging queries in NFA-based RDBMS and stream processing systems. E-Cube~\cite{LRGGWAM11} inserts sequence queries into a hierarchy based on concept and pattern refinement relations. SPASS~\cite{RLR16} estimates the benefit of sharing for event sequence construction using intra-query and inter-query event correlations. MOTTO~\cite{ZVDH17} applies merge, decomposition, and operator transformation techniques to re-write pattern matching queries. Kolchinsky et al.~\cite{KS19} combine sharing and pattern reordering optimizations for both NFA-based and tree-based query plans.
%
However, these approaches do not support online aggregation of event sequences, i.e., they construct all event sequences prior to their aggregation, which degrades query performance. To the best of our knowledge, \sharon~\cite{PRLRM18} and Muse~\cite{RPLR20} are the only solutions that support shared online aggregation. However, \sharon\ does not support Kleene closure. Worse yet, \sharon\ and Muse make static sharing decisions. In contrast, \app\ harnesses additional sharing benefit thanks to dynamic sharing decisions depending on the current stream properties. 


\textbf{Multi-query Processing over Data Streams.} Sharing query processing techniques are well-studied for streaming systems. NiagaraCQ~\cite{10.1145/342009.335432} is a large-scale system for processing multiple continuous queries over streams. TelegraphCQ~\cite{DBLP:conf/cidr/ChandrasekaranDFHHKMRRS03} introduces a tuple-based dynamic routing for inter-query sharing~\cite{10.1145/564691.564698}. 
AStream~\cite{KRM19} shares computation and resources among several queries executed in Flink~\cite{flink}.
Several approaches focus on sharing optimizations given different predicates, grouping, or window clauses~\cite{AW04,GSCL12,HFAE03,KWF06,LMTPT05,TKPP20,ZKOSZ10}. However, these approaches evaluate Select-Project-Join queries with windows and aggregate single events. They do not support CEP-specific operators such as event sequence and Kleene closure that treat the order of events as a first-class citizen. Typically, they require the construction of join results prior to their aggregation. In contrast, \app\ not only avoids the expensive event trend construction, but also exploits the sharing opportunities among trend aggregation queries with diverse Kleene patterns.

%%%%%%%%%%%%%%%%%%%%%%%%%%%%%%%%%%%%%%%%%%%%%%%%%%%%%%%%%%%%%%%%%%%
%\textbf{Relational Database Multi-Query Optimizations} include materialized views~\cite{MATViewICDE} and common sub-expression sharing~\cite{chakravarthy1986multiple,giannikis2010crescando}. These approaches do not tackle temporal relationships prevalent in CEP context. They neither focus on event trend construction nor their aggregation. They assume that the data is statically stored on disk and do not target in-memory execution nor real-time responsiveness essential in CEP world. 
