\subsection{Workload Analysis and Stream Partitioning}
\label{sec:tempalte}

Given that the workload may contain queries with different Kleene patterns, aggregation functions, windows, and groupby clauses, \app\ takes the following pre-processing steps:
(1)~it breaks the workload  into  sets of sharable queries at compile time;
(2)~it then constructs the \app\ query template for each sharable query set; and
% \ear{It is ambiguous here, if such
% partitioning below is done separately
% for each query set - likely it is; if so, maybe say:
% (3)~At runtime, it partitions the stream by window and groupby clauses 
% for each query template.}
% chuan - fixed
(3)~it partitions the stream by window and groupby clauses for each query template at runtime. 

\begin{definition}(\textbf{Shareable Kleene Sub-pattern})
%
Let $Q$ be a workload and $E$ be an event type. 
Assume that a Kleene sub-pattern $E+$ appears in queries $Q_E \subseteq Q$ and $|Q_E| > 1$. 
We say that $E+$ is shareable by queries $Q_E$.
%
\label{def:shareable-sub-pattern}
\end{definition}

However, sharable Kleene sub-patterns cannot always be shared due to other query clauses. For example, queries having $\mycount(*)$, $\mymin(E.$ $\mathit{attr})$ or $\mymax(E.\mathit{attr})$ can only be shared with queries that compute these same aggregates. 
In contrast,   
since $\myavg(E.\mathit{attr})$ is computed as $\mysum(E.\mathit{attr})$ divided by $\mycount(E)$, queries computing $\myavg(E.\mathit{attr})$ can be shared with queries that calculate $\mysum(E.\mathit{attr})$ or $\mycount(E)$.
We therefore define sharable queries below.

\begin{definition}(\textbf{Sharable Queries})
%
Two queries are \textit{sharable} if their patterns contain at least one sharable Kleene sub-pattern, their aggregation functions can be shared, their windows overlap, and their grouping attributes are the same.
%
\label{def:sharable_queries}
\end{definition}

%--------------------------------------------
To facilitate the shared runtime execution of each set of sharable queries, each pattern is converted into its Finite State Automaton-based representation~\cite{ADGI08, DGPRSW07, WDR06, ZDI14}, called \textbf{\textit{query template}}.
We adopt the state-of-the-art algorithm~\cite{PLRM18} to convert each pattern in in the workload $Q$ into its template.

Figure~\ref{fig:template} depicts the template of query $q_1$ with pattern $\seq(A,$ $B+)$.
States, shown as rectangles, represent event types in the pattern.
If a transition connects a type $E_1$ with a type $E_2$ in a template of a query $q$, then events of type $E_1$ precede events of type $E_2$ in a trend matched by $q$. $E_1$ is called a \textit{\textbf{predecessor type}} of $E_2$, denoted $E_1 \in \textit{pt}(E_2,q)$.
A state without ingoing edges is a \textit{\textbf{start type}}, and a state shown as a double rectangle is an \textit{\textbf{end type}} in a pattern. 

\begin{example}
In Figure~\ref{fig:template}, events of type $B$ can be preceded by events of types $A$ and $B$ in a trend matched by $q_1$, i.e., $\mathit{pt}(B,q_1)=\{A,B\}$. Events of type $A$ are not preceded by any events, $\mathit{pt}(A,q_1)$ $=\emptyset$. Events of type $A$ start trends and events of type $B$ end trends matched by $q_1$, i.e., $\mathit{start}(q_1)=\{A\}$ and $\mathit{end}(q_1)=\{B\}$.
\label{ex:template_one_query}
\end{example}

Our \app\ system processes the entire workload $Q$ instead of each query in isolation. To expose all sharing opportunities in $Q$, we convert the entire workload $Q$ into one \textit{\textbf{\app\ query template}}. It is constructed analogously to a query template with two additional rules. First, each event type is represented in the merged template only once. Second, each transition is labeled by the set of queries for which this transition holds.

\begin{example}
Figure~\ref{fig:merged_template} depicts the template for the workload $Q=\{q_1,q_2\}$ where query $q_1$ has pattern $\seq(A,B+)$ and query $q_2$ has pattern $\seq(C,B+)$.
The transition from $B$ to itself is labeled by two queries $q_1$ and $q_2$. This transition corresponds to the shareable Kleene sub-pattern $B+$ in these queries (highlighted in gray).
\label{ex:running_example}
\end{example}

%----------------------------------------------------------------
%\textbf{Stream Partitioning}.
%
The event stream is first partitioned by the grouping attributes.
To enable shared execution despite different windows of sharable queries, \app\ further partitions the stream into \textbf{\textit{panes}} that are sharable across  overlapping windows~\cite{AW04, GSCL12, KWF06, LMTPT05}.
The size of a pane is the greatest common divisor (gcd) of all window sizes and window slides. For example, for two windows $(\within\ 10\ min\ \slide$ $5\ min)$ and $(\within\ 15\ min$ $\slide\ 5$ $min)$, the gcd is 5 minutes. In this example, a pane contains all events per 5 minutes interval. For each set of sharable queries, we  apply the \app\ optimizer and executor within each pane. 








