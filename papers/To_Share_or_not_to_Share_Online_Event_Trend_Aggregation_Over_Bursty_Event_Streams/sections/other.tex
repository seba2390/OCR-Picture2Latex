\section{General Trend Aggregation Queries}
\label{sec:other}

While we focused on simpler queries so far, 
we now sketch how \app\ can be extended to support a broad class of trend aggregation queries as per Definition~\ref{def:query}.

\textbf{Disjunctive or Conjunctive Pattern}.
Let $P$ be a disjunctive or a conjunctive pattern and $P_1, P_2$ be its sub-patterns (Definition~\ref{def:pattern}). In contrast to event sequence and Kleene patterns,
$P$ does not impose a time order constraint upon trends matched by $P_1$ and $P_2$. 
%
Let $\mycount(P)$ denote the number of trends matched by $P$. $\mycount(P)$ can be computed based on $\mycount(P_1)$ and $\mycount(P_2)$ as defined below. The processing of $P_1$ and $P_2$ can be shared. 
%
Let $P_{1,2}$ be the pattern that detects trends matched by both $P_1$ and $P_2$. 
%
% $P_{1,2}$ can be obtained from its DFA representation that  corresponds to the intersection of the DFAs for $P_1$ and $P_2$~\cite{theoretical-info}.
%
Let 
$C_{1,2} = \mycount(P_{1,2})$,
$C_1 = \mycount(P_1) - C_{1,2}$, and
$C_2 = \mycount(P_2) - C_{1,2}$.
$C_{1,2}$ is subtracted to avoid counting trends matched by $P_{1,2}$ twice.

Disjunctive pattern $(P_1 \vee P_2)$  matches a trend that is a match of $P_1$ or $P_2$. 
$\mycount(P_1 \vee P_2) = C_1 + C_2 + C_{1,2}$.

Conjunctive pattern $(P_1 \wedge P_2)$  matches a pair of trends $tr_1$ and $tr_2$ where $tr_1$ is a match of $P_1$ and $tr_2$ is a match of $P_2$. 
$\mycount(P_1 \wedge P_2) = 
C_1 * C_2 + 
C_1 * C_{1,2} +
C_2 * C_{1,2} +
\binom{C_{1,2}}{2}$
since each trend detected only by $P_1$ (not by $P_2$) is combined with each trend detected only by $P_2$ (not by $P_1$). In addition, each trend detected by $P_{1,2}$ is combined with each other trend detected only by $P_1$, only by $P_2$, or by $P_{1,2}$.

%\textbf{Kleene Star} and \textbf{Optional Sub-patterns} are also supported since they are syntactic sugar operators. Indeed,
%$\seq(P_1*,P_2) = \seq(P_1+,$ $P_2) \vee P_2$ and 
%$\seq(P_1?,P_2) = \seq(P_1,P_2) \vee P_2$.

\textbf{Pattern with Negation} $\seq(P_1,\mynot\ N, P_2)$ is 
split into positive $\seq(P_1,P_2)$ and negative $N$ sub-patterns at compile time. At runtime, we maintain separate graphs for positive and negative sub-patterns. When a negative sub-pattern $N$ finds a match $e_n$, we disallow connections from matches of $P_1$ before $e_n$ to matches of $P_2$ after $e_n$. Aggregates are computed the same way~\cite{PLRM18}.

\textbf{Nested Kleene Pattern} $P=(\seq(P_1,P_2+))+$.
Loops exist at template level but not at graph level because previous events connect to new events in a graph but never the other way around due to temporal order constraints (compare Figures~\ref{fig:templates} and \ref{fig:snapshot}). 
The processing of $P$ and its sub-patters can be shared by several queries containing these patterns as illustrated by Example~\ref{ex:nested}. 

\begin{figure}[ht]
\centering
\includegraphics[width=0.25\columnwidth]{figures/merged-template-2.png}
\caption{\app\ query template for $Q$}
\label{fig:merged-template-2}
\end{figure}

\begin{example}
Consider query $q_1$ with pattern $(\seq(A,B+))+$ and query $q_2$ with pattern $(\seq(C,B+))+$. Figure~\ref{fig:merged-template-2} shows the merged template for the workload $Q=\{q_1,q_2\}$.
In contrast to the template in Figure~\ref{fig:merged_template}, there are two additional transitions (from $B$ to $A$ for $q_1$ and from $B$ to $C$ for $q_2$) forming two additional loops in the template.
Therefore, in addition to predecessor type relations in Example~\ref{ex:template_one_query}
($pt(B,q_1) = \{A,B\}$ and 
$pt(B,q_2) = \{C,B\}$),
two new predecessor type relations exist. 
Namely,
$pt(A,q_1) = \{B\}$ and 
$pt(C,q_2) = \{B\}$. 

Consider the stream in Figure~\ref{fig:snapshot}.
Similarly to Example~\ref{ex:three},
events in $A_1$--$C_2$ are predecessors of events in $B_3$, 
while events in $A_1$--$C_5$ are predecessors of events in $B_6$. 
Because of the additional predecessor type relations in the template in Figure~\ref{fig:merged-template-2}, 
events in $B_3$ are predecessors of events in $A_4$--$C_5$.
Due to these additional predecessor event relations,
more trends are now captured in the \app\ graph in Figure~\ref{fig:snapshot} and the intermediate and final trend counts have now higher values.
However, 
Definitions~\ref{def:graphlet}--\ref{def:dynamic-benefit} and Theorems~\ref{theo:pruning-1}--\ref{theo:optimality} hold and 
Algorithm~\ref{algo:snapshot-propagation} applies to share trend aggregation among queries in the workload $Q$.
\label{ex:nested}
\end{example}




%\textbf{Different Aggregation Functions in the Workload}.
%
% Let $E$ be an event type, $\mathit{attr}$ be an attribute of $E$, and $e$ be an event of type $E$.
%
% While $\mycount(*)$ returns the number of all trends per group, $\mycount(E)$ computes the number of all events $e$ in all trends per group. 
%
% $\mysum(E.\mathit{attr})$ ($\myavg(E.\mathit{attr})$) calculates the summation (average) of the value of $\mathit{attr}$ of all events $e$ in all trends per group.
%
% $\mymin(E.\mathit{attr})$ ($\mymax(E.\mathit{attr})$) computes the minimal (maximal) value of $\mathit{attr}$ for all events $e$ in all trends per group.

% Given that $\myavg(E.\mathit{attr})$ is computed as $\mysum(E.\mathit{attr})$ divided by $\mycount(E)$, queries computing $\myavg(E.\mathit{attr})$ can be shared with queries that calculate $\mysum(E.\mathit{attr})$ or $\mycount(E)$.
% In contrast, queries computing $\mycount(*)$, $\mymin(E.\mathit{attr})$ or $\mymax(E.\mathit{attr})$ can only be shared with queries that compute these aggregates.

%\textbf{Different Grouping and Window Clauses in the Workload}.
%
% Grouping partitions the stream into sub-streams by the values of grouping attributes~\cite{GSCL12, QCRR14}. Windows further partition these sub-streams into non-overlapping segments by time~\cite{AW04, GSCL12, KWF06, LMTPT05}. Thereafter, \app\ can then be applied within each segment.



