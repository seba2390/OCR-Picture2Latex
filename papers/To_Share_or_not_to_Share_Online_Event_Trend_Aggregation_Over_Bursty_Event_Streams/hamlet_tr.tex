\documentclass[twoside,10pt]{article} % twosided, draft copy
\usepackage{amsfonts}
\usepackage{amsmath}
\usepackage{amssymb}
\usepackage{epsfig}
\usepackage{graphicx}
\usepackage{latexsym}
\usepackage{subfigure}   
\usepackage{times}
\usepackage{hyperref}

\newcommand{\app}{\textsc{Hamlet}}
\newcommand{\greta}{\textsc{Greta}}
\newcommand{\sharon}{\textsc{Sharon}}
\newcommand{\mcep}{\textsc{MCEP}}

\newcommand{\anysem}{\textsf{\footnotesize ANY}}
\newcommand{\nextsem}{\textsf{\footnotesize NEXT}}
\newcommand{\contsem}{\textsf{\footnotesize CONT}}   

\newcommand{\return}{\textsf{\footnotesize RETURN}}
\newcommand{\pattern}{\textsf{\footnotesize PATTERN}}
\newcommand{\semantics}{\textsf{\footnotesize SEMANTICS}}
\newcommand{\where}{\textsf{\footnotesize WHERE}}
\newcommand{\group}{\textsf{\footnotesize GROUP-BY}}
\newcommand{\within}{\textsf{\footnotesize WITHIN}}
\newcommand{\slide}{\textsf{\footnotesize SLIDE}}
\newcommand{\as}{\textsf{\footnotesize AS}}
\newcommand{\seq}{\textsf{\footnotesize SEQ}}
\newcommand{\myand}{\textsf{\footnotesize AND}}
\newcommand{\next}{\textsf{\footnotesize NEXT}}
\newcommand{\mycount}{\textsf{\footnotesize COUNT}}
\newcommand{\mymin}{\textsf{\footnotesize MIN}}
\newcommand{\mymax}{\textsf{\footnotesize MAX}}
\newcommand{\mysum}{\textsf{\footnotesize SUM}}
\newcommand{\myavg}{\textsf{\footnotesize AVG}}
\newcommand{\mynot}{{\small \textsf{NOT}}}

\newcommand{\mystart}{\textsf{\footnotesize START}}
\newcommand{\mymid}{\textsf{\footnotesize MID}}
\newcommand{\myend}{\textsf{\footnotesize END}}

\makeatletter
\providecommand*{\cupdot}{%
  \mathbin{%
    \mathpalette\@cupdot{}%
  }%
}
\newcommand*{\@cupdot}[2]{%
  \ooalign{%
    $\m@th#1\cup$\cr
    \hidewidth$\m@th#1\cdot$\hidewidth
  }%
}

\usepackage[ruled]{algorithm}
\PassOptionsToPackage{noend}{algpseudocode}
\usepackage{algpseudocode}

\renewcommand{\algorithmicrequire}{\textbf{Input:}}
\renewcommand{\algorithmicensure}{\textbf{Output:}}
\renewcommand{\algorithmiccomment}[1]{\bgroup\hfill//~#1\egroup}
\renewcommand{\algorithmicforall}{\textbf{for each}}
\algnewcommand\algorithmicswitch{\textbf{switch}}
\algnewcommand\algorithmiccase{\textbf{case}}
\algnewcommand\algorithmicassert{\texttt{assert}}
\algnewcommand\Assert[1]{\State \algorithmicassert(#1)}%
% New "environments"
\algdef{SE}[SWITCH]{Switch}{EndSwitch}[1]{\algorithmicswitch\ #1\ \algorithmicdo}{\algorithmicend\ \algorithmicswitch}%
\algdef{SE}[CASE]{Case}{EndCase}[1]{\algorithmiccase\ #1}{\algorithmicend\ \algorithmiccase}%
\algtext*{EndSwitch}%
\algtext*{EndCase}%


\usepackage[font=small,labelfont=bf,tableposition=top]{caption}

\usepackage{listings}
\lstset{
	basicstyle=\small,
	numbers=none,
	numbersep=5pt,
	numberstyle=\tiny,
	stepnumber=2, 
	captionpos=b,
	keepspaces=true,
	mathescape=true,
    frame=bt,
	float=t,
}
\lstset{
	emph={DERIVE, FROM, WHERE, AND, ANTI, JOIN, USING, NOT, EXISTS, SELECT, if, then, else, end, PATTERN, SEQ, STOCK, DETECT, INSERT, INTO, VALUES, DELETE, IN, UPDATE, SET, OR, GROUP, BY, AS, VARIABLE, IF, END, FOR, EACH, Input, Output, DECLARE, WITHIN, CREATE, VIEW, STREAM, PRIORITY, high, NEW, RUN, ANTIJOIN, ON, LOOK, UP, WITH, SAME, EVENT, for, each, while, end, return, Input, Output, WHILE, THEN, ELSE, RETURN, SLIDE, do, and, continue, NEXT, FIRST, LAST},
	emphstyle={\textsf}%
}

% \usepackage{slashbox}
% \usepackage{cite}

%\renewcommand{\baselinestretch}{2} % turns on double spacing
\renewcommand{\algorithmicrequire}{\textbf{Input:}}
\renewcommand{\algorithmicensure}{\textbf{Output:}}
\renewcommand{\algorithmiccomment}[1]{/* #1 */}
\newcommand{\Stirling}[2]{\genfrac\{\}{0pt}{} {#1} {#2} }

\usepackage{amsthm}
\setcounter{secnumdepth}{5}

\usepackage{setspace,xspace}
\graphicspath{{figures/}}
\newcommand{\bigO}{\ensuremath{\mathcal{O}}} % big-O notation/symbol

\newtheorem{ndef}{Definition}
\newcommand{\nop}[1]{}

% ============================================================
%                      Margins and heights
% ============================================================

% margin paragraphs, for comments in the draft
\setlength{\marginparwidth}{1in}
% Also define a command ``\rem'' for remarks.
\newcommand{\rem}[1]{\marginpar{\flushleft{#1}}}
\renewcommand{\rem}[1]{} % disable remarks

% ------------------------------------------------------------

% Junk formatting. For Elke.
%
% set margins to 1 inch all around the page, no header or footer.
% use maximum space (no topmargin... header is one inch from paper border)
%\setlength{\textheight}{9in}
%\setlength{\textwidth}{6.2in}
%\setlength{\topmargin}{-0.125in}
%\setlength{\oddsidemargin}{-.2in}
%\setlength{\evensidemargin}{-.2in}
%\setlength{\headsep}{0in}

% ------------------------------------------------------------

% And now the pretty formatting for the production version
\renewcommand{\algorithmicrequire}{\textbf{Input:}}
\renewcommand{\algorithmicensure}{\textbf{Output:}}
\renewcommand{\algorithmiccomment}[1]{/* #1 */}
%\newcommand{\Stirling}[2]{\genfrac\{\}{0pt}{} {#1} {#2} }

\graphicspath{{figures/}}
%\providecommand{\OO}[1]{\operatorname{O}\left(#1\right)} \newtheorem{lemma}{Lemma}
\newtheorem{definition}{Definition}
\newtheorem{example}{Example}
\newtheorem{theorem}{Theorem}[section]

%==============================RIMMA_CUSTOM==================================
\usepackage{multirow}
\newcommand{\eat}[1] {}
\newcommand{\comment}[1]{ $\Rightarrow$ {\tiny #1} }
\newcommand{\srccomment}[1]{\textrm{// #1}}
\renewcommand{\thefootnote}{*}

\newenvironment{ALGORITHM}[2]
    {
        \begin{tabular}{l}
        \begin{minipage}{8.0cm} \begin{small} \begin{tt} \begin{tabbing}
        12 \= 12 \= 12 \= 12 \= 1 \= 1 \= 1 \kill
        {\bf #1} (#2) \\
    }
    {
        \end{tabbing} \end{tt} \end{small} \end{minipage}
        \\
        \hline
        \end{tabular}
    }
%==============================END_RIMMA_CUSTOM==================================

% ============================================================
% Headings. Use ``Fanch Headers''
\usepackage{fancyhdr}
\renewcommand{\sectionmark}[1]{\markright{\thesection.\ #1}}
\fancyhf{}
\fancyhead[LE,RO]{\bfseries\thepage}
\fancyhead[LO]{}
\fancyhead[RE]{}
\renewcommand{\headrulewidth}{0.5pt}
\renewcommand{\footrulewidth}{0pt}
\addtolength{\headheight}{1.5pt}
\fancypagestyle{plain}{%
  \fancyhead{}
  \renewcommand{\headrulewidth}{0pt}
}

% ============================================================

% WPI library conventions: 1.5in margin left, one inch
% elsewhere. (except that this is twosided if desired)

\setlength{\topmargin}{0pt}

 \newlength{\hoehe}
 \setlength{\hoehe}{\paperheight}
 \addtolength{\hoehe}{-2in} % leave one inch at top and bottom
 \addtolength{\hoehe}{-\topmargin}
 \addtolength{\hoehe}{-\headheight}
 \addtolength{\hoehe}{-\headsep}
 \setlength{\textheight}{\hoehe}

 \newlength{\breite}
 \setlength{\breite}{\paperwidth}
 \addtolength{\breite}{-2.5in}
 \setlength{\textwidth}{\breite}
 % for two-paged output

 \setlength{\oddsidemargin}{0in}
 \setlength{\evensidemargin}{0.5in}
 
% ============================================================
% Use doublespacing for the draft
\onehalfspacing
% \doublespacing
% ============================================================

% ============================================================
\setcounter{tocdepth}{2}
% ============================================================

% ============================================================
% A prettier font.
\renewcommand{\rmdefault}{ppl} % Use Palatino Font for Roman
\renewcommand{\sfdefault}{phv} % Use Helvetica Font for SansSerif
%\usepackage{zplppl}            % Use MathKit Palatino Mathfonts (Alan Hoenig)
% ============================================================

% ============================================================
% Titlepage
\title{\fontsize{15}{15}\selectfont To Share, or not to Share Online Event Trend Aggregation\\ Over Bursty Event Streams\\\vspace*{1cm}
\large Technical Report\\
\vspace*{1cm}}
\author{\large Olga Poppe,$^1$ Chuan Lei,$^2$ Lei Ma,$^3$ Allison Rozet,$^4$ and Elke A. Rundensteiner$^3$}
%\date{Draft. \today}
\date{\Large 
January, 2021\\
\vspace*{1cm}
\large $^1$Microsoft Gray Systems Lab, One Microsoft Way, Redmond, WA 98052\\
$^2$IBM Research, Almaden, 650 Harry Rd, San Jose, CA 95120\\
$^3$Worcester Polytechnic Institute, Worcester, MA 01609\\
$^4$MathWorks, 1 Apple Hill Dr, Natick, MA 01760\\
\vspace*{0.2cm}
olpoppe@microsoft.com, chuan.lei@ibm.com, lma5@wpi.edu, arozet@mathworks.com, rundenst@wpi.edu\\
\vspace*{7cm}
%\vfill
%\begin{minipage}{18cm}
%\textbf{Committee Members:}\\ \\
%Prof. Elke A. Rundensteiner, Worcester Polytechnic Institute, Advisor.\\
%Prof. Dan Dougherty, Worcester Polytechnic Institute. \\
%Prof. Mohammed Y. Eltabakh, Worcester Polytechnic Institute.\\
%Prof. David Maier, Portland State University.\\
%\end{minipage}
}

% ============================================================

% ============================================================
% Macro definitions
% ============================================================

\newcommand{\switch}{%
  \mathcode`+=\numexpr\mathcode`+ + "1000\relax % turn + into a relation
  \mathcode`*=\numexpr\mathcode`* + "1000\relax
}

\pagestyle{empty}

\begin{document}
\maketitle

\begin{spacing}{0.8}
{\footnotesize \noindent \textbf{Copyright} \copyright{} 2021 by
authors. Permission to make digital or hard copies of all or
part of this work for personal use is granted without fee provided
that copies bear this notice and the full citation on the first
page. To copy otherwise, to republish, to post on servers or to
redistribute to lists, requires prior specific permission. }
\end{spacing}

\clearpage
\pagestyle{fancy}

\clearpage
\tableofcontents

\pagenumbering{arabic}
\setcounter{page}{1}

% 
% Since this is a ``report'', the topmost level of hierarchy is
% ``Chapter'', not section as you may be used to. Chapters are
% enumerated starting from 1, so Sections are 1.1, Subsections are
% 1.1.1, subsubsections don't get numbers. (You can change that, if
% you want them to be called 1.1.1.1)
%

\newpage
  In this paper, we explore the connection between secret key agreement and secure omniscience within the setting of the multiterminal source model with a wiretapper who has side information. While the secret key agreement problem considers the generation of a maximum-rate secret key through public discussion, the secure omniscience problem is concerned with communication protocols for omniscience that minimize the rate of information leakage to the wiretapper. The starting point of our work is a lower bound on the minimum leakage rate for omniscience, $\rl$, in terms of the wiretap secret key capacity, $\wskc$. Our interest is in identifying broad classes of sources for which this lower bound is met with equality, in which case we say that there is a duality between secure omniscience and secret key agreement. We show that this duality holds in the case of certain finite linear source (FLS) models, such as two-terminal FLS models and pairwise independent network models on trees with a linear wiretapper. Duality also holds for any FLS model in which $\wskc$ is achieved by a perfect linear secret key agreement scheme. We conjecture that the duality in fact holds unconditionally for any FLS model. On the negative side, we give an example of a (non-FLS) source model for which duality does not hold if we limit ourselves to communication-for-omniscience protocols with at most two (interactive) communications.  We also address the secure function computation problem and explore the connection between the minimum leakage rate for computing a function and the wiretap secret key capacity.
  
%   Finally, we demonstrate the usefulness of our lower bound on $\rl$ by using it to derive equivalent conditions for the positivity of $\wskc$ in the multiterminal model. This extends a recent result of Gohari, G\"{u}nl\"{u} and Kramer (2020) obtained for the two-user setting.
  
   
%   In this paper, we study the problem of secret key generation through an omniscience achieving communication that minimizes the 
%   leakage rate to a wiretapper who has side information in the setting of multiterminal source model.  We explore this problem by deriving a lower bound on the wiretap secret key capacity $\wskc$ in terms of the minimum leakage rate for omniscience, $\rl$. 
%   %The former quantity is defined to be the maximum secret key rate achievable, and the latter one is defined as the minimum possible leakage rate about the source through an omniscience scheme to a wiretapper. 
%   The main focus of our work is the characterization of the sources for which the lower bound holds with equality \textemdash it is referred to as a duality between secure omniscience and wiretap secret key agreement. For general source models, we show that duality need not hold if we limit to the communication protocols with at most two (interactive) communications. In the case when there is no restriction on the number of communications, whether the duality holds or not is still unknown. However, we resolve this question affirmatively for two-user finite linear sources (FLS) and pairwise independent networks (PIN) defined on trees, a subclass of FLS. Moreover, for these sources, we give a single-letter expression for $\wskc$. Furthermore, in the direction of proving the conjecture that duality holds for all FLS, we show that if $\wskc$ is achieved by a \emph{perfect} secret key agreement scheme for FLS then the duality must hold. All these results mount up the evidence in favor of the conjecture on FLS. Moreover, we demonstrate the usefulness of our lower bound on $\wskc$ in terms of $\rl$ by deriving some equivalent conditions on the positivity of secret key capacity for multiterminal source model. Our result indeed extends the work of Gohari, G\"{u}nl\"{u} and Kramer in two-user case.
% \leavevmode
% \\
% \\
% \\
% \\
% \\
\section{Introduction}
\label{introduction}

AutoML is the process by which machine learning models are built automatically for a new dataset. Given a dataset, AutoML systems perform a search over valid data transformations and learners, along with hyper-parameter optimization for each learner~\cite{VolcanoML}. Choosing the transformations and learners over which to search is our focus.
A significant number of systems mine from prior runs of pipelines over a set of datasets to choose transformers and learners that are effective with different types of datasets (e.g. \cite{NEURIPS2018_b59a51a3}, \cite{10.14778/3415478.3415542}, \cite{autosklearn}). Thus, they build a database by actually running different pipelines with a diverse set of datasets to estimate the accuracy of potential pipelines. Hence, they can be used to effectively reduce the search space. A new dataset, based on a set of features (meta-features) is then matched to this database to find the most plausible candidates for both learner selection and hyper-parameter tuning. This process of choosing starting points in the search space is called meta-learning for the cold start problem.  

Other meta-learning approaches include mining existing data science code and their associated datasets to learn from human expertise. The AL~\cite{al} system mined existing Kaggle notebooks using dynamic analysis, i.e., actually running the scripts, and showed that such a system has promise.  However, this meta-learning approach does not scale because it is onerous to execute a large number of pipeline scripts on datasets, preprocessing datasets is never trivial, and older scripts cease to run at all as software evolves. It is not surprising that AL therefore performed dynamic analysis on just nine datasets.

Our system, {\sysname}, provides a scalable meta-learning approach to leverage human expertise, using static analysis to mine pipelines from large repositories of scripts. Static analysis has the advantage of scaling to thousands or millions of scripts \cite{graph4code} easily, but lacks the performance data gathered by dynamic analysis. The {\sysname} meta-learning approach guides the learning process by a scalable dataset similarity search, based on dataset embeddings, to find the most similar datasets and the semantics of ML pipelines applied on them.  Many existing systems, such as Auto-Sklearn \cite{autosklearn} and AL \cite{al}, compute a set of meta-features for each dataset. We developed a deep neural network model to generate embeddings at the granularity of a dataset, e.g., a table or CSV file, to capture similarity at the level of an entire dataset rather than relying on a set of meta-features.
 
Because we use static analysis to capture the semantics of the meta-learning process, we have no mechanism to choose the \textbf{best} pipeline from many seen pipelines, unlike the dynamic execution case where one can rely on runtime to choose the best performing pipeline.  Observing that pipelines are basically workflow graphs, we use graph generator neural models to succinctly capture the statically-observed pipelines for a single dataset. In {\sysname}, we formulate learner selection as a graph generation problem to predict optimized pipelines based on pipelines seen in actual notebooks.

%. This formulation enables {\sysname} for effective pruning of the AutoML search space to predict optimized pipelines based on pipelines seen in actual notebooks.}
%We note that increasingly, state-of-the-art performance in AutoML systems is being generated by more complex pipelines such as Directed Acyclic Graphs (DAGs) \cite{piper} rather than the linear pipelines used in earlier systems.  
 
{\sysname} does learner and transformation selection, and hence is a component of an AutoML systems. To evaluate this component, we integrated it into two existing AutoML systems, FLAML \cite{flaml} and Auto-Sklearn \cite{autosklearn}.  
% We evaluate each system with and without {\sysname}.  
We chose FLAML because it does not yet have any meta-learning component for the cold start problem and instead allows user selection of learners and transformers. The authors of FLAML explicitly pointed to the fact that FLAML might benefit from a meta-learning component and pointed to it as a possibility for future work. For FLAML, if mining historical pipelines provides an advantage, we should improve its performance. We also picked Auto-Sklearn as it does have a learner selection component based on meta-features, as described earlier~\cite{autosklearn2}. For Auto-Sklearn, we should at least match performance if our static mining of pipelines can match their extensive database. For context, we also compared {\sysname} with the recent VolcanoML~\cite{VolcanoML}, which provides an efficient decomposition and execution strategy for the AutoML search space. In contrast, {\sysname} prunes the search space using our meta-learning model to perform hyperparameter optimization only for the most promising candidates. 

The contributions of this paper are the following:
\begin{itemize}
    \item Section ~\ref{sec:mining} defines a scalable meta-learning approach based on representation learning of mined ML pipeline semantics and datasets for over 100 datasets and ~11K Python scripts.  
    \newline
    \item Sections~\ref{sec:kgpipGen} formulates AutoML pipeline generation as a graph generation problem. {\sysname} predicts efficiently an optimized ML pipeline for an unseen dataset based on our meta-learning model.  To the best of our knowledge, {\sysname} is the first approach to formulate  AutoML pipeline generation in such a way.
    \newline
    \item Section~\ref{sec:eval} presents a comprehensive evaluation using a large collection of 121 datasets from major AutoML benchmarks and Kaggle. Our experimental results show that {\sysname} outperforms all existing AutoML systems and achieves state-of-the-art results on the majority of these datasets. {\sysname} significantly improves the performance of both FLAML and Auto-Sklearn in classification and regression tasks. We also outperformed AL in 75 out of 77 datasets and VolcanoML in 75  out of 121 datasets, including 44 datasets used only by VolcanoML~\cite{VolcanoML}.  On average, {\sysname} achieves scores that are statistically better than the means of all other systems. 
\end{itemize}


%This approach does not need to apply cleaning or transformation methods to handle different variances among datasets. Moreover, we do not need to deal with complex analysis, such as dynamic code analysis. Thus, our approach proved to be scalable, as discussed in Sections~\ref{sec:mining}.
\section{Preliminaries}
\label{sec:basic}

\subsection{Basic Notions}
\label{sec:basic_notions}

Time is represented by a linearly ordered set of time points $(\mathbb{T},\leq)$, where $\mathbb{T} \subseteq \mathbb{Q^+}$ are the non-negative rational numbers. 
%
An \textbf{\textit{event}} $e$ is a data tuple describing an incident of interest to the application. An event $e$ has a time stamp $e.time \in \mathbb{T}$ assigned by the event source. 
An event $e$ belongs to a particular event type $E$, denoted \textit{e.type=E} and described by a schema that specifies the set of event attributes and the domains of their values. A specific attribute $\mathit{attr}$ of $E$ is referred to as $E.\mathit{attr}$.
Table~\ref{tab:notation} summarizes the notation.

Events are sent by event producers (e.g., vehicles and mobile devices) to an  \textbf{\textit{event stream}} $I$. We assume that events arrive in order by their time stamps. Existing approaches to handle out-of-order events can be applied~\cite{CGM10, LTSPJM08, LLGRC09, SW04}.

An event consumer (e.g., Uber stream analytics) continuously monitors the stream with \textbf{\textit{event queries}}. We adopt the commonly used query language and semantics from SASE~\cite{ADGI08, WDR06, ZDI14}.
The query workload in Figure~\ref{fig:queries} is expressed in this language. We assume that the workload is static. Adding or removing a query from a workload requires migration of the execution plan to a new workload which can be handled by existing approaches~\cite{KWF06, ZhuRH04}.


\begin{table}[!tb]
    \centering
    \begin{tabular}{|p{1.3cm}|p{11.3cm}|}
    \hline
        Notation 
        & Description \\\hline\hline
       $e.\mathit{time}$
       & Time stamp of event $e$ \\\hline
       $e.\mathit{type}$
       & Type of event $e$ \\\hline
       $E.\mathit{attr}$
       & Attribute $\mathit{attr}$ of event type $E$ \\\hline
       $\mathit{start}(q)$
       & Start types of the pattern of query $q$ \\\hline
       $\mathit{end}(q)$
       & End types of the pattern of query $q$ \\\hline
       $\mathit{pt}(E,q)$
       & Predecessor types of event type $E$ w.r.t query $q$ \\\hline
       $\mathit{pe}(e,q)$
       & Predecessor events of event $e$ w.r.t query $q$  \\\hline
       %$\mathit{count}(e,q)$
       %& Intermediate trend count of event $e$ w.r.t query $q$  \\\hline
       %$\mathit{value}(x,q)$ 
       %& Value of snapshot $x$ w.r.t query $q$ \\\hline
       %\multirow{2}{*}{$\mathit{sum}(A_1,q)$} 
       %& Sum of intermediate trend counts of all events in graphlet $A_1$ that are matched by query $q$ \\\hline
       $n$ & Number of events per window \\\hline
       $g$ & Number of events per graphlet \\\hline
       $b$ & Number of events per burst \\\hline
       $k$ & Number of queries in the workload $Q$ \\\hline
       $k_s$ & Number of queries that share the graphlet $G_E$ with other queries \\\hline
       $k_n$ & Number of queries that do not share the graphlet $G_E$ with other queries \\\hline
       $p$
       & Number of predecessor types per type per query \\\hline
       $s$ & Number of snapshots \\\hline
       $s_c$
       & Number of snapshots created from one burst of events \\\hline
       $s_p$
       & Number of snapshots propagated in one shared graphlet \\\hline
    \end{tabular}
    \caption{Table of notations}
    \label{tab:notation}
\end{table}



%{\color{blue} EAR:  i do not see where you enforce that at least one pi (e?) has a kleene, and that others may not have a kleene?}
% Olga: fixed

\begin{definition}(\textbf{Kleene Pattern})
%
A pattern $P$ can be in the form of 
$E$, $P_1+$, ($\mynot\ P_1$),
\seq$(P_1,$ $P_2)$, $(P_1 \vee P_2)$, or $(P_1 \wedge P_2)$, 
where 
$E$ is an event type,
$P_1,P_2$ are patterns, 
$+$ is a Kleene plus, 
$\mynot$ is a negation,
\seq\ is an event sequence,
$\vee$ is a disjunction, and
$\wedge$ is a conjunction.
%
$P_1$ and $P_2$ are called sub-patterns of $P$.
%
If a pattern $P$ contains a Kleene plus operator, $P$ is called a Kleene pattern.
% If an operator in a pattern $P$ is applied to the result of another operator, the pattern $P$ is called nested. Otherwise, the pattern $P$ is flat.
%
\label{def:pattern}
\end{definition}

%----------------- First and last event type in a pattern
%\begin{definition}(\textbf{Start and End Type of a Pattern})
%
%Let $P$ be a pattern.
%Let $start(P)$ denote the start and $end(P)$ denote the end event type of $P$.
%
%For an event type $E$, $start(E)=end(E)=E$.
%For a Kleene pattern $P+$, $start(P+)=start(P)$ and $end(P+)=end(P)$.
%For a sequence pattern $\seq(P_1,P_2)$, $start(\seq(P_1,P_2))=start(P_1)$ and $end(\seq(P_1,P_2))=end(P_2)$.
%
%\label{def:start-and-end-type-in-a-pattern}
%\end{definition}

% We focus on Kleene patterns that allow us to specify arbitrarily long event pattern matches, called event trends (Definition~\ref{def:query}). 
% To keep our discussion focused in Sections~\ref{sec:tempalte}--\ref{sec:runtime}, we consider Kleene patterns that neither contain other operators (e.g., conjunction, negation), nor are nested. We generalize \app\ in Section~\ref{sec:other}.


%{\color{blue} EAR: Below does not seem to enforce that te pattern has to be a kleene enriched pattern, thus this not necessarily a  "trend" query???}
% olga: fixed

%------------------ Query
\begin{definition}(\textbf{Event Trend Aggregation Query})
%
An event trend aggregation query $q$ consists of five clauses:

$\bullet$ Aggregation result specification (\textsf{RETURN} clause),

$\bullet$ Kleene pattern $P$ (\textsf{PATTERN} clause) as per Definition~\ref{def:pattern},

$\bullet$ Predicates $\theta$ (optional \textsf{WHERE} clause),

$\bullet$ Grouping $G$ (optional \textsf{GROUPBY} clause), and

$\bullet$ Window $w$ (\textsf{WITHIN/SLIDE} clause).
%
\label{def:query}
\end{definition}

%------------------ Trend
\begin{definition}(\textbf{Event Trend})
%
Let $q$ be a query per Definition~\ref{def:query}.
An event trend $tr = (e_1, \ldots, e_k)$ corresponds to a sequence of events that conform to the pattern $P$ of $q$. All events in a trend $tr$ satisfy predicates $\theta$, have the same values of grouping attributes $G$, and are within one window $w$ of $q$. 
%
\label{def:trend}
\end{definition}

%------------------------------
\textbf{Aggregation of Event Trends}.
Within each window specified by the query $q$, event trends are grouped by the values of grouping attributes $G$. Aggregates are then computed per group. 
% {\color{blue} Below is not true for Q1. so i have altered the stmt to say Q2 and Q3.  did you mean  Q1 on purpose be be a groupby over each driver only? even if drivers drive through multiple districts? will you explain the relationship of driver vs district somewhere???}  
% olga: i changed the example
%For example, queries $q_1$--$q_3$ in Figure~\ref{fig:queries} group trips per district and compute the number, total duration, and average speed of trips.
\app\ focuses on distributive (\mycount, \mymin, \mymax, \mysum) and algebraic aggregation functions (\myavg) since they can be computed incrementally~\cite{Gray97}. 
%
Let $E$ be an event type,
$\mathit{attr}$ be an attribute of $E$, and
$e$ be an event of type $E$.
%
While $\mycount(*)$ returns the number of all trends per group,
$\mycount(E)$ computes the number of all events $e$ in all trends per group. 
%
$\mysum(E.\mathit{attr})$ ($\myavg(E.\mathit{attr})$) calculates the summation (average) of the value of $\mathit{attr}$ of all events $e$ in all trends per group.
%
$\mymin(E.\mathit{attr})$ ($\mymax(E.\mathit{attr})$) computes the minimal (maximal) value of $\mathit{attr}$ for all events $e$ in all trends per group.


%\textbf{\textit{Count}}.
%\mycount$\mathit{(*)}$ returns the number of trends per group.
%Let \textit{tr.}\mycount(\textit{E}) be the number of events of type $E$ in a trend $tr$. \mycount(\textit{E}) corresponds to the sum of \textit{tr.}\mycount$(E)$ of all trends per group.


%\textbf{\textit{Minimum and Maximum}}.
%Let \textit{tr.}\mymin(\textit{E.attr}) be the minimal value of an attribute \textit{attr} of events of type $E$ in a trend $tr$.
%\mymin(\textit{E.attr}) returns the minimal value of \textit{tr.}\mymin$(E.attr)$ of all trends per group.
%\mymax(\textit{E.attr}) is defined analogously to \mymin$(E.\mathit{attr})$.

%\textbf{\textit{Summation and Average}}.
%Let \textit{tr.}\mysum(\textit{E.attr}) be the sum of values of an attribute \textit{attr} of events of type $E$ in a trend $tr$.
%\mysum(\textit{E.attr}) corresponds to the sum of \textit{tr.}\mysum(\textit{E.attr}) of all trends per group.
%\myavg(\textit{E.attr}) = \mysum(\textit{E.attr}) / \mycount(\textit{E}) per group.

%------------------------------
%\textbf{Predicates}.
%
% Predicates on single events and on pairs of adjacent events in a trend are commonly used to specify expressive CEP queries in diverse applications~\cite{ADGI08, DGPRSW07, MM09, PLRM18}. Thus, we focus on such predicates in this paper and leave predicates on non-adjacent events for future research.
%We differentiate between three types of predicates:

%\textit{Local predicates} filter events in the stream based on their attribute values. For example, query $q_2$ in Figure~\ref{fig:queries} matches only pool ride requests which is expressed by the local predicate $(R.\mathit{type} = \mathit{Pool})$.

%\textit{Equivalence predicates} partition the event stream so that all events in a trend have the same attribute value. The predicate [driver, rider] in Figure~\ref{fig:queries} specifies that all events in a trend must have the same driver and rider identifiers.

%\textit{Predicates on adjacent events} in a trend. For example, the predicate $(E.\mathit{price} < \textsf{NEXT}(E).\mathit{price})$ requires a trend of  events of type $E$ to strictly increase in price.

%------------------------------
% \textbf{Event Matching Semantics}.
%
% Our queries are evaluated under the skip-till-any-match semantics which is the most flexible semantics since it allows to detect the set of all matches of a query~\cite{ADGI08, WDR06, ZDI14}. An exponential number of matches in the number of input events can be produced in the worst case~\cite{ZDI14}. Trends conforming to more restrictive semantics can be extracted from the set of all matches~\cite{PLRM19}.

\begin{figure}
\centering

\def\picScale{0.08}    % define variable for scaling all pictures evenly
\def\colWidth{0.5\linewidth}

\begin{tikzpicture}
\matrix [row sep=0.25cm, column sep=0cm, style={align=center}] (my matrix) at (0,0) %(2,1)
{
\node[style={anchor=center}] (FREEhand) {\includegraphics[width=0.85\linewidth]{figures/FREEhand.pdf}}; %\fill[blue] (0,0) circle (2pt);
\\
\node[style={anchor=center}] (rigid_v_soft) {\includegraphics[width=0.75\linewidth]{figures/FREE_vs_rigid-v8.pdf}}; %\fill[blue] (0,0) circle (2pt);
\\
};
\node[above] (FREEhand) at ($ (FREEhand.south west)  !0.05! (FREEhand.south east) + (0, 0.1)$) {(a)};
\node[below] (a) at ($ (rigid_v_soft.south west) !0.20! (rigid_v_soft.south east) $) {(b)};
\node[below] (b) at ($ (rigid_v_soft.south west) !0.75! (rigid_v_soft.south east) $) {(c)};
\end{tikzpicture}


% \begin{tikzpicture} %[every node/.style={draw=black}]
% % \draw[help lines] (0,0) grid (4,2);
% \matrix [row sep=0cm, column sep=0cm, style={align=center}] (my matrix) at (0,0) %(2,1)
% {
% \node[style={anchor=center}] {\includegraphics[width=\colWidth]{figures/photos/labFREEs3.jpg}}; %\fill[blue] (0,0) circle (2pt)
% &
% \node[style={anchor=center}] {\includegraphics[width=\colWidth, height=160pt]{figures/stewartRender.png}}; %\fill[blue] (0,0) circle (2pt);
% \\
% };

% %\node[style={anchor=center}] at (0,-5) (FREEstate) {\includegraphics[width=0.7\linewidth]{figures/FREEstate_noLabels2.pdf}};

% \end{tikzpicture}

\caption{\revcomment{2.3}{(a) A fiber-reinforced elastomerc enclosure (FREE) is a soft fluid-driven actuator composed of an elastomer tube with fibers wound around it to impose specific deformations under an increase in volume, such as extension and torsion. (b) A linear actuator and motor combined in \emph{series} has the ability to generate 2 dimensional forces at the end effector (shown in red), but is constrained to motions only in the directions of these forces. (b) Three FREEs combined in \emph{parallel} can generate the same 2 dimensional forces at the end effector (shown in red), without imposing kinematic constraints that prohibit motion in other directions (shown in blue).}}

% \caption{A fiber-reinforced elastomeric enclosure (FREE) (top) is a soft fluid-driven actuator composed of an elastomer tube with fibers wound around it to impose deformation in specific directions upon pressurization, such as extension and torsion. \revcomment{2.3}{In this paper we explore the potential of combining multiple FREEs in parallel to generate fully controllable multi-dimensional spacial forces}, such as in a parallel arrangement around a flexible spine element (bottom-left), or a Stewart Platform arrangement (bottom-right).}

\label{fig:overview}
\end{figure}

\section{Core \app\ Execution Techniques}
\label{sec:executor}

\textbf{Assumptions}.
To keep the discussion focused on the core concepts, we make simplifying assumptions in Sections~\ref{sec:executor} and \ref{sec:runtime}. We drop them to extend \app\ to the broad class of trend aggregation queries (Definition~\ref{def:query}) in Section~\ref{sec:other}.
These assumptions include:
(1)~queries compute the number of trends per window $\mycount(*)$;
(2)~query patterns do not contain disjunction, conjunction nor negation; and 
(3)~Kleene plus operator is applied to an event type  and appears once per query.

% \ear{Did we have those assumptions before;
% or where they added through this revision?}
% chuan - we had them before

In Section~\ref{sec:tempalte}, we describe the workload and stream partitioning. We introduce strategies for processing  queries without sharing in Section~\ref{sec:non-shared-baseline} versus with \textit{shared} online trend aggregation in Section~\ref{sec:shared-approach}. In Section~\ref{sec:runtime}, we present the runtime optimizer that makes these sharing decisions.

% $Id: template.tex 11 2007-04-03 22:25:53Z jpeltier $
\documentclass{vgtc}                          % final (conference style)
%\documentclass[review]{vgtc}                 % review
%\documentclass[widereview]{vgtc}             % wide-spaced review
%\documentclass[preprint]{vgtc}               % preprint
%\documentclass[electronic]{vgtc}             % electronic version

%% Uncomment one of the lines above depending on where your paper is
%% in the conference process. ``review'' and ``widereview'' are for review
%% submission, ``preprint'' is for pre-publication, and the final version
%% doesn't use a specific qualifier. Further, ``electronic'' includes
%% hyperreferences for more convenient online viewing.

%% Please use one of the ``review'' options in combination with the
%% assigned online id (see below) ONLY if your paper uses a double blind
%% review process. Some conferences, like IEEE Vis and InfoVis, have NOT
%% in the past.

%% Figures should be in CMYK or Grey scale format, otherwise, colour 
%% shifting may occur during the printing process.

%% These few lines make a distinction between latex and pdflatex calls and they
%% bring in essential packages for graphics and font handling.
%% Note that due to the \DeclareGraphicsExtensions{} call it is no longer necessary
%% to provide the the path and extension of a graphics file:
%% \includegraphics{diamondrule} is completely sufficient.
%%
\ifpdf%                                % if we use pdflatex
  \pdfoutput=1\relax                   % create PDFs from pdfLaTeX
  \pdfcompresslevel=9                  % PDF Compression
  \pdfoptionpdfminorversion=7          % create PDF 1.7
  \ExecuteOptions{pdftex}
  \usepackage{graphicx}                % allow us to embed graphics files
  \DeclareGraphicsExtensions{.pdf,.png,.jpg,.jpeg} % for pdflatex we expect .pdf, .png, or .jpg files
\else%                                 % else we use pure latex
  \ExecuteOptions{dvips}
  \usepackage{graphicx}                % allow us to embed graphics files
  \DeclareGraphicsExtensions{.eps}     % for pure latex we expect eps files
\fi%

%% it is recomended to use ``\autoref{sec:bla}'' instead of ``Fig.~\ref{sec:bla}''
\graphicspath{{figures/}{pictures/}{images/}{./}} % where to search for the images
\usepackage{microtype}                 % use micro-typography (slightly more compact, better to read)
\PassOptionsToPackage{warn}{textcomp}  % to address font issues with \textrightarrow
\usepackage{textcomp}                  % use better special symbols
\usepackage{mathptmx}                  % use matching math font
\usepackage{times}                     % we use Times as the main font
\renewcommand*\ttdefault{txtt}         % a nicer typewriter font
\usepackage{cite}                      % needed to automatically sort the references
\usepackage{tabu}                      % only used for the table example
\usepackage{booktabs}   % only used for the table example
\usepackage{caption} 
\captionsetup[table]{skip=10pt}
\usepackage{color}


%%% 
%%% Meta notes
%%%
\newlength{\dummylen}
\newcommand{\NOTE}[1]{\setlength{\dummylen}{\fboxrule}\setlength{\fboxrule}{2pt}%
            \vspace{1ex}\noindent\hfill%
            \fbox{\begin{minipage}{.96\columnwidth}#1\end{minipage}}%
            \setlength{\fboxrule}{\dummylen}\hfill{}\vspace{1ex}}

%\renewcommand{\NOTE}[1]{\ignorespaces}



%% We encourage the use of mathptmx for consistent usage of times font
%% throughout the proceedings. However, if you encounter conflicts
%% with other math-related packages, you may want to disable it.


%% If you are submitting a paper to a conference for review with a double
%% blind reviewing process, please replace the value ``0'' below with your
%% OnlineID. Otherwise, you may safely leave it at ``0''.

\onlineid{0}

%% declare the category of your paper, only shown in review mode
\vgtccategory{Research}

%% allow for this line if you want the electronic option to work properly
\vgtcinsertpkg

%% In preprint mode you may define your own headline.
%\preprinttext{To appear in an IEEE VGTC sponsored conference.}

%% Paper title.

\title{Recognizing Handwritten Source Code}
%\title{Writing Source Code (literally!)
%% This is how authors are specified in the conference style

%% Author and Affiliation (single author).
%%\author{Roy G. Biv\thanks{e-mail: roy.g.biv@aol.com}}
%%\affiliation{\scriptsize Allied Widgets Research}

%Author and Affiliation (multiple authors with single affiliations).
\author{Qiyu Zhi\thanks{e-mail: qzhi@nd.edu} %
\and Ronald Metoyer\thanks{e-mail:rmetoyer@nd.edu}}
\affiliation{\scriptsize University of Notre Dame}

%% Author and Affiliation (multiple authors with multiple affiliations)
% \author{Roy G. Biv\thanks{e-mail: roy.g.biv@aol.com}\\ %
%         \scriptsize Starbucks Research %
% \and Ed Grimley\thanks{e-mail:ed.grimley@aol.com}\\ %
%      \scriptsize Grimley Widgets, Inc. %
% \and Martha Stewart\thanks{e-mail:martha.stewart@marthastewart.com}\\ %
%      \parbox{1.4in}{\scriptsize \centering Martha Stewart Enterprises \\ Microsoft Research}}

%% A teaser figure can be included as follows, but is not recommended since
%% the space is now taken up by a full width abstract.
% \teaser{
%  \includegraphics[width=1.5in]{sample.eps}
%  \caption{Lookit! Lookit!}
% }

%% Abstract section.
\abstract{
%Programming requires textual input and typically requires typing which is not always the %optimal input method for all users - especialy on touchscreen devices. 
Supporting programming on touchscreen devices requires effective text input and editing methods.  Unfortunately, the virtual keyboard can be inefficient and uses valuable screen space on already small devices.  Recent advances in stylus input make handwriting a potentially viable text input solution for programming on touchscreen devices.  
The primary barrier, however, is that handwriting
recognition systems are built to take advantage of the rules of natural language, not those of a programming language. In this paper, we explore this particular problem of handwriting recognition for source code.  
We collect and make publicly available a dataset of handwritten \textit{Python} code samples from 15 participants and we characterize the typical recognition errors for this handwritten \textit{Python} source code when using a state-of-the-art handwriting recognition tool.  We present an approach to improve the recognition accuracy by augmenting a handwriting recognizer with the programming language grammar rules. Our experiment on the collected dataset shows an 8.6\% word error rate and a 3.6\% character error rate which outperforms standard handwriting recognition systems and compares favorably to typing source code on virtual keyboards.
} % end of abstract

%% ACM Computing Classification System (CCS). 
%% See <http://www.acm.org/class/1998/> for details.
%% The ``\CCScat'' command takes four arguments.


\keywords{programming, handwriting recognition, touch screen, source code, python.}
\CCScatlist{ 
  \CCScat{H.1.2.}{User/Machine Systems}{Human information processing}{};
  \CCScat{H.5.2.}{User Interfaces}{Input devices and strategies (e.g., mouse, touchscreen)}{}
}


%% Copyright space is enabled by default as required by guidelines.
%% It is disabled by the 'review' option or via the following command:
% \nocopyrightspace

%%%%%%%%%%%%%%%%%%%%%%%%%%%%%%%%%%%%%%%%%%%%%%%%%%%%%%%%%%%%%%%%
%%%%%%%%%%%%%%%%%%%%%% START OF THE PAPER %%%%%%%%%%%%%%%%%%%%%%
%%%%%%%%%%%%%%%%%%%%%%%%%%%%%%%%%%%%%%%%%%%%%%%%%%%%%%%%%%%%%%%%%

\begin{document}

%% The ``\maketitle'' command must be the first command after the
%% ``\begin{document}'' command. It prepares and prints the title block.

%% the only exception to this rule is the \firstsection command
%\firstsection{Introduction}

\maketitle

\section{Introduction} %for journal use above \firstsection{..} instead
%Supporting programming on mobile devices is not a novel idea. 
With the rapid technology shift in current computing devices, high-quality low-cost mobile devices such as tablets and smartphones are being increasingly used in everyday activities. Many tasks that previously required a PC are now feasible on mobile devices. For example, tablets are typically equipped with powerful batteries, advanced graphic processors, high-resolution screens and fast processors, making writing and compiling code on them completely plausible. TouchDevelop, for example, is a novel programming environment, language and code editor for mobile devices  \cite{tillmann2011touchdevelop}. Furthermore, Tillmann et al. predict that programming on mobile devices will be widely used for teaching programming \cite{tillmann2012future}.  However, mobile devices are also inherently restricted by their limitations such as small screens and the clumsy virtual keyboard.  Entering and editing large amounts of text for programming tasks can quickly become difficult and time consuming with these virtual keyboards because they are notoriously difficult to use when compared to a physical keyboard and they consume valuable screen space\cite{raab2013refactorpad}.

%that is difficult to use and that takes up valuable screen space - posing challenges for a programmer who wishes to enter or edit source code.


% comparing handwriting and typing in different scenarios to motivate the work. 
While keyboards have been the primary input device for entering computer programs since the computer was invented\cite{gordon2013improving}, this predominant mechanism is not ideal for all programming situations. For example, software developers that suffer from repetitive strain injuries (RSI) and related disabilities may find typing on a keyboard difficult or impossible \cite{begelprogramming}.
%people suffering from repetitive strain injuries (RSI), carpal tunnel syndrome, or other motor impairments may experience tremendous difficulty using a keyboard and mouse \cite{arnold2000programming}.
Instead, handwriting with a stylus may be a preferred input mechanism for some of these users \cite{mankoff1998cirrin}.
%In addition, entering and editing large amounts of text for programming tasks can quickly become difficult and time consuming with
%\emph{virtual} keyboards, for any user, because they are notoriously difficult to use when compared to a physical keyboard and they consume valuable screen space\cite{raab2013refactorpad}.
%People with disabilities  (e.g. limb loss) or who do not have full mobility may also find typing on a keyboard  (physical or virtual) is difficult or impossible \cite{begelprogramming}. 
In addition, some physical configurations (e.g. seated on a plane) may simply be more suited to the writing posture than a typing posture for many users.

Handwriting has also been shown to have potential cognitive benefits\cite{alonso2015metacognition}.  In particular, Mueller and Oppenheimer found that students who took longhand notes performed better on conceptual questions than those that typed notes on a laptop \cite{mueller2014pen}.  Given these findings, and that fact that many programmers write pseudocode by hand before typing, it is reasonable to consider that handwriting may provide cognitive benefits for programming, especially on mobile devices. 
Furthermore, recent advances in pen-based input and handwriting recognition technology are quickly making handwriting a viable alternative to typing.
%Entering and editing large amounts of text for
%programming tasks can quickly become difficult and %time consuming
%with \emph{virtual} keyboards that are notoriously %difficult to use when compared to a physical %keyboard and consume valuable screen %space\cite{raab2013refactorpad}. 
%It is, therefore, important that we explore handwriting as an input option to support these users and their needs.
%keyboards are not conveniently available on many of today's touchscreen devices or come in the form of \emph{virtual} keyboards that are notoriously difficult to use and require the use of valuable screen space \cite{}.  


% One alternative to typing is to use speech (via dictation) to create the text of a program. Desilets et al. \cite{desilets2006voicecode} propose VoiceCode, which translates the pronounced syntax into native syntax in the current programming language to support programming by speech. Gordon \cite{gordon2013improving} employs language design and incorporates dynamic context for this purpose. Speech, however, is not always an appropriate option given social conventions and privacy issues.
% %While many alternatives have been explored, few have thoroughly examined the use of handwriting as a means for textual input, especially when considering source code. 
% Given the recent advances in pen-based input, however, handwriting is a potentially viable alternative to typing and speech.
%other input forms, such as speech, which are not always appropriate.

%Given recent advances in pen-based input, handwriting code is becoming a potentially viable alternative to typing.


In this paper, we explore the use of handwriting as a means for source code text input.  There are two ways to approach this problem.  One alternative is to develop or modify a handwriting recognition engine to take source code directly into account.  Given that source code often includes English language words, another alternative is to leverage the capabilities of an existing English language handwriting recognition engine. We explore this latter option.  First, we collect and present a publicly available dataset of handwritten \textit{Python} source code for use in handwriting recognition research.  Second, we explore the use of the state-of-the-art recognition system, MyScript \cite{myscript} for recognizing \textit{Python} source code.  We characterize the errors made by the MyScript engine and present a method for post-processing the engine's results to improve recognition performance on handwritten \textit{Python} source code.   

After presenting related work and necessary background information, we describe our data collection process and the resulting publicly available dataset. We then describe the performance of MyScript on recognizing the handwritten source code and present our algorithm for leveraging the MyScript engine to produce improved results.  We discuss those results in Section \ref{results} and conclude with avenues of future work.


%leveraging programming language grammar and lexicon information into a current handwriting recognition system, MyScript \cite{myscript}. Our primary contributions include a handwritten source code dataset, a characterization of errors made when using a general handwriting recognition system on the source code, and an algorithm to leverage general handwriting recognition systems to produce improved results for handwritten source code.


% overview of our work

%Another possible situation is when keyboards are not available, which is becoming a common scenario because touchscreen devices such as the advanced iPad and Android-based tablet is sufficient for daily use.

\section{Background and Related Work}
% Handwriting source code
% For each paper, what did they do  (few sentences), and what did they find, how are we going to be different.

% Handwriting recognition


Many alternatives to typed source code have been considered, typically in the context of making programming more accessible.   Most of those appoaches fall in the realm of speech-based programming \cite{desilets2006voicecode, gordon2013improving}, however speech is not always an acceptable solution, especially in quiet environments or with applications that require privacy.
%One approach is to use human speech (via dictation) to create the text of a program. Desilets et al. \cite{desilets2006voicecode} propose VoiceCode, which translates the pronounced syntax into native syntax in the current programming language to support programming by speech. Gordon \cite{gordon2013improving} employs language design and incorporates
%dynamic context for this purpose.   
Given the recent advances in pen-based input, handwriting is a potentially viable alternative.  The pendragon supports people who are unable to use
a keyboard and seeks to find new interaction techniques for input which may improve communication speed \cite{Pendragon}. Mankoff et al.also suggest that word prediction, sentence completion and the syntax of programming languages could be used for handwriting source code \cite{mankoff1998cirrin}. 
Most closely related to our work is a programming IDE integrated with a handwriting area in which the handwritten code is recognized by an enhanced handwriting recognition system \cite{frye2008pdp}.  This work, however, does not present an evaluation of the recognition engine or guidance for how to improve general handwriting recognition engines for application to source code recognition.
%However, they failed to present a clear evaluation for the handwritten source code recognition system.
In this paper, we focus on the recognition step of handwritten source code as the most important step for developing an effective handwriting interface for source code input and editing.

Research in handwriting recognition has a long history dating back to the 1960s~\cite{tappert1990state}. Hidden Markov Model (HMM) based handwriting recognition~\cite{hu1996hmm,kundu1988recognition,nag1986script} is one of the 
most widely used approaches while neural networks are gaining in popularity~\cite{jaeger2001online}. Some approaches also leverage additional constraints for recognizing handwriting in specific domains such as postal addresses \cite{srihari1993recognition, srihari1993interpretation} and banking checks \cite{gorski1999a2ia, agarwal1997bank}. These handwriting recognition systems are developed to take advantage of the English language \cite{van2003using}, which is intrinsically different from source code. For instance, variable names are often created from
concatenated words (e.g. camelCase or underscore naming), which poses a problem for the traditional handwriting
recognition system as it expects spaces to appear between words
contained within its dictionary.   We do not aim to contribute to the extensive literature in handwriting recognition, but rather, we intend to examine how we can leverage this existing work for application to handwritten source code recognition interfaces.  




\section{Data Collection}

Our first contribution in this paper is a data collection study designed to generate a sample set of handwritten source code for research purposes.  The first question to consider is what programming language to study.  We decided to collect handwritten \textit{Python} source code because of the current popularity of \textit{Python}\footnote{\url{http://www.tiobe.com/tiobe-index/}} and its projected growth rate 
%at some high-profile
%organizations, such as Google, Disney and the US
%Government 
\cite{radenski2006python}. 
We chose a ``copying task'', where three code samples are provided for every participant to copy on the tablet using the stylus. While we understand that a ``copying task'' may be cognitively quite different from other writing tasks that require synthesis, we sought to eliminate sources of cognitive load that could impact timing as well as writing quality for the purposes of this data collection task. The three shared code samples allow for comparison across participants. To broaden our dataset of unique handwritten source code samples, we also randomly selected a fourth source code sample function (per participant) that was unique to that participant. In this section, we describe our data collection process and the resulting database of handwritten \textit{Python} samples.


%its current popularity as a programming language and it's growth trajectory.  



%In order to test the recognition system, we collected a dataset of handwritten \textit{Python} code samples written by students from the [anonymized for review] campus. We chose \textit{Python} because of its current popularity as a programming language. 

\subsection{Participants}
We recruited 15 participants (9 females) from the University of Notre Dame for our study. Thirteen of the participants were computer science majors and all participants had at least two semesters of programming experience. Their ages ranged from 19 to 29 (mean = 22.3). Two participants were left-handed. Eight participants had used a pen/stylus for handwriting on a touchscreen device and only one participant had used a tablet for inputting source code (via the virtual keyboard). All participants were compensated \$5 for the study which took approximately 30 minutes each.

\subsection{Apparatus and Software}

\begin{figure}[tb]
 \centering % avoid the use of \begin{center}...\end{center} and use \centering instead (more compact)
 \includegraphics[width=\columnwidth]{datacollection}
 \caption{Screen shot of our data collection web application.  Participants entered their name and code sample number into the boxes in the upper left.  They then entered the code sample using the Apple Pencil and selected `Save' when finished.}
 \label{fig:webpage}
\end{figure}


We used a 12.9 inch iPad Pro with a 2732-by-2048 screen resolution at 264 pixels per inch (PPI) and fingerprint-resistant oleophobic coating. Participants used the Apple Pencil as the stylus device. We implemented a web application with a writing area to record user input. This application was responsible for converting touch points of the stylus into handwriting strokes and saving strokes to a JSON file. Each stroke consists of the coordinates of the sampled points and time-stamp information for each coordinate. The writing area in this application measured 795 * 805 pixels with subtle lines on the background to provide guides for the participants (See \autoref{fig:webpage}). We also implemented functions to undo or redo the previous stroke as well as clear the writing area of all strokes. 

\subsection{Representative Source Code Material}
Our goal was to create a database of representative samples of handwritten \textit{Python} source code for use in evaluating the performance of a handwriting recognition system. Because different \textit{Python} samples contain different language elements, there is no single representative corpus\cite{almusaly2015syntax}. Ideally, representative code samples should contain a variety of language constructs and not be restricted to a single project. 

Our process for choosing source code samples is based on that used by McMillan et al. \cite{mcmillan2012exemplar, rodeghero2014improving}. First, we selected six popular \textit{Python} projects on Github. \autoref{table:project} summarizes the details of these projects. We then extracted all functions from the project source code and eliminated comments in order to focus solely on the source code of the samples.  Next, to obtain functions that were sufficiently long to collect a substantial amount of handwriting, but not so long as to require multiple pages of handwriting, we filtered the functions to those with between 9 and 18 lines of source code and those with no lines greater than 60 characters (to eliminate long, wrapping lines).  We also manually filtered out highly repetitive functions, such as a function that includes only assignment statements for variables. The result was 1324 eligible functions. We randomly selected the three shared test code samples from this set for use by all participants and one additional unique code sample to be entered by each participant in the study. Although \textit{Python} syntax considers whitespace, we decided to ignore indentation for the purposes of focusing purely on handwriting recognition. 
%We are currently in the process of augmenting the collected samples to produce a second identical dataset with indentation inserted.


%We collected this unique code sample for each participant to broaden the coverage of code samples and characterize the common recognition errors from the common recognition system as well.


\subsection{Procedure}

For every participant, we began our data collection with an informed consent process.  Each participant then filled out a pre-study questionnaire about demographics and experience using touchscreen devices and a stylus. Participants were given a practice task to familiarize them with the process.  For each of the four tasks, participants were given a sheet of paper with the sample typed \textit{Python} source code.   Participants entered their name and code sample number into the web application and then entered the code sample using the Apple Pencil and selected `Save' when finished.  After completing all four input tasks, participants were compensated and the session ended.


\subsection{Data Collection Results}
The final dataset includes stroke data for four code samples for each of 15 participants resulting in a total of 60 handwritten source code samples. So, for each of 3 given source code input samples, we have 15 copies of handwritten source code (for a total of 45 handwritten source code samples). The remaining 15 are handwritten samples of unique input source code examples from each participant. 
The handwritten source code data can be downloaded at \newline \url{http://www.purl.org/recognizinghandwrittencode/data}.
 
\begin{table}
  \centering
  \begin{tabular}{l r r r}
    % \toprule
    {Project}
    & {Lines}
      & {Fuctions}
    & {Eligible Functions} \\
    \midrule
    AlphaGo & 1,963 & 151 & 1 \\
    Bittorrent & 7,164 &  570 & 39 \\
    Blender & 265,684  & 12,774 & 1,126 \\
    Instagram & 1,265 & 145 & 8 \\
    Requests & 14,009  & 862 & 84 \\
    Webpy & 10,199 & 1,029 & 66 \\
    % \bottomrule
  \end{tabular}
  \caption{\textit{Python} projects used for selecting code samples}~\label{table:project}
\end{table}



\section{Source Code Recognition Errors}

Current commercial handwriting recognition systems are built to take advantage of the rules of the English language as opposed to that of a programming language, therefore it is not surprising that these systems might perform poorly on source code recognition \cite{frye2008pdp}. There is, however, no previous research that evaluates how well existing state-of-the-art handwriting recognition systems perform on handwritten source code. Here we describe the state-of-the-art handwriting recognition system we employed and characterize the errors based on the dataset we collected.

\subsection{State-of-the-art: Myscript}
Automatic recognition of handwriting is now a mature discipline that has found many commercial uses\cite{plamondon2000online}. MyScript\cite{myscript} is an online handwriting recognition engine that supports more than 80 languages and achieved the best recognition rate in the International Conference on Document Analysis and Recognition competition\cite{el2011line}.
Here we use the MyScript engine as our baseline for comparison and study the typical recognition errors produced when applied to handwritten source code to better understand the complexities introduced by \textit{Python} source code and source code in general.


\subsection{Data Pre-Processing}

In order to use MyScript efficiently and to make a fair comparison between its performance and our algorithm, we apply two simple pre-processing steps to the data. First, we provide MyScript with \textit{Python} specific context through the Subset Knowledge (SK) facility and a custom lexicon. SK is a MyScript feature for telling the recognizer that we only want it to enable recognition of certain characters. For example, for a phone number field, we may want only digits to be recognized.   We created an SK resource in MyScript to allow only characters that can legally appear in \textit{Python} source code.  We also provide the legal \textit{Python} keywords through a user-defined lexicon.

%We used two main MyScript resources to process the data we collected: 
%The MyScript Cloud Development Kit (CDK) and Subset knowledge(SK).
The MyScript Cloud Development Kit (CDK) is an HTTP-based set of services that take handwritten strokes as input and produce potential recognition results as output. To use the CDK in experiments, we must send strokes to the recognizer at some level of granularity (e.g. single character, whole word, whole line, etc).
%The data processing step is to process the data for submission to the MyScript Cloud Development Kit (CDK). Pre-processing includes character filtration and sentence separation. 
%Character filtration is to restrict the recognition result to certain characters. We used the Subset knowledge(SK) feature by creating an SK resource in MyScript to allow only characters that can legally appear in \textit{Python} source code. In sentence separation, 
We chose to break the stroke data into lines assuming developers might write a statement at a time on a single line. To do so, we analyze stroke coordinates and create a new line each time the user moves to a new vertical position. Each line is then sent one at a time to the MyScript CDK.  This simulates a developer writing one programming statement (one line) at a time, pausing at the end of each line.  Alternative pre-processing is possible given the raw stroke data and timestamp information (e.g. sending incomplete lines when a participant pauses).

\subsection{Characterizing errors}
\label{sec:characterizing}
We processed all of the handwritten data as described above to collect baseline recognition results for all handwritten samples in our dataset.  We then set out to understand the types of recognition errors that were present in the final recognized text. We identified three major types of recognition errors: word errors, symbol errors, and space errors. 

Word errors occur when MyScript simply incorrectly recognizes a written word. This is typically due to poor writing and can occur for keywords as well as non-keywords.
For example, when the handwritten word `self' is recognized as `silt', we characterize this as a word error. 
Symbol errors represent incorrect recognition of symbols or non alpha-numeric characters.
For example, an `\_' (underscore) is often recognized as a `-' (dash). 
Finally, a space error results when the system inserts an unexpected space.
For example, when `ConflictError' is recognized as `Conflict Error', we characterize it as a space error. 

Most of the word errors and symbol errors can be attributed to poor writing or cursive writing (characters are written joined together in a flowing manner) which is inherently more difficult for MyScript to recognize than block writing (characters are written separately). Space errors, on the other hand, appear to depend on the language model of the recognizer, which most likely does not include training on CamelCase\footnote{\url{https://en.wikipedia.org/wiki/Camel\_case}} or proper English words separated by dot notation (e.g. student.name).  The result is that MyScript inserts space at these word and dot notation separators.

In summary, from the statistical results for each type of error presented in \autoref{fig:errors}, space errors, mainly caused by the internal mechanism of English handwriting recognition system, represent the most prevalent recognition error. In addition, poor writing and the tendency to return an English word for a non-English word in the source code lead to word errors, which also represents a significant portion of all errors. Symbol errors are also a prevalent error type. This makes sense given that MyScript is designed to recognize general words, however, symbols, dot notation, and combinations of symbols and words are typically not present in general text, especially in the way that they are used in source code. For example, the most problematic symbols includes underscore `\_', parentheses `( )' and equal `='.


\begin{figure}[t!]
\centering
\includegraphics[width=3.16in, height = 2.2in]{errors}
\caption{Average error numbers of all participants for each code sample from MyScript general handwriting recognition engine}
\label{fig:errors}
\end{figure}


\section{Handwritten Source Code Recognition Pipeline}



A programming language is governed by grammar rules, which stipulate the positions of keywords and symbols. For example, in \textit{Python}, a \textit{def} sentence must end with a `:'. However, handwritten symbols are often problematic.  For example, colons `:' are sometimes recognized as semicolons `;'. In addition to grammar rules, programming languages are highly repetitive with predictable properties\cite{hindle2012naturalness}. 
Function names and variable names are the most common repetitive words in a single source code project. If a function name appears more than once in the same handwritten code sample, however, it is impossible for users to hand write the \textit{exact} same strokes for this function name, which makes different recognition results of the same handwritten function name a possibility that we must account for.

In this section, we present an approach to improve the recognition rate for handwritten source code by addressing these issues as well as those common errors characterized in Section \ref{sec:characterizing}. We leverage what we know about the predictability and structure of source code to improve recognition results beyond that of the state-of-the-art recognizer. 


The general premise of our approach is that state-of-the-art engines can produce excellent results given good writing and the absence of symbols and programming practices like camelCase.  Our framework, illustrated in \autoref{fig:overview}, is therefore aimed at analyzing and post-processing the recognition results produced from MyScript to utilize its recognition capabilities but correct for those common errors. This framework can be divided into four parts: statement classification, statement parsing, token processing, and statement concatenation. 
%The various parts are presented in the following sub-sections.
The source code for this post-processing algorithm can be found at \url{http://www.purl.org/recognizinghandwrittencode/code}.

\begin{figure}[h!]
\centering
\includegraphics[width=0.5\textwidth]{overview}
\caption{Framework for augmenting MyScript to correct for common recognition errors in handwritten source code.}
\label{fig:overview}
\end{figure}

\begin{figure*}[h]
 \center
  \includegraphics[width=2.2in, height = 1.32in]{picture1}
  \includegraphics[width=2.2in, height = 1.32in]{picture2}
  \includegraphics[width=2.2in, height = 1.32in]{picture3}
  \caption{Average recognition error rate of MyScript and our augmented MyScript system for three test code samples}
  \label{result}
\end{figure*}



\subsection{Statement Classification}
As we mentioned before, we process the handwritten source code data considering each statement as a unit. According to the \textit{Python} grammar specification, we can restrict \textit{Python} source code statements into a limited number of classes, each of which has specified structure rules\footnote{\url{https://docs.python.org/2/reference/grammar.html}}. Here we use the first token in the statement as the symbol for classification. For example, a `def' statement starts with `def' and its structure is defined as `def' + `function name' + `(parameters0, parameters1 ...):'. We define 14 classes for \textit{Python} code statements, including an `assignment' statement, which means the first word in this statement is not a keyword but rather a variable name. In \autoref{fig:overview}, the recognition result is classified as an `if' statement. \autoref{table:statement} presents statistics for the various statement classes in the three code samples.



\begin{table}
  \centering
  \begin{tabular}{l r r r}
    % \toprule
    {Class}
    & {Frequency}
      & {Class}
    & {Frequency} \\
    \midrule
    def & 3 & except & 1  \\
    if & 7 &  while & 1 \\
    for & 3  & try & 1 \\
    raise & 2 & break & 1 \\
    return & 2  & else & 1 \\
    yield & 2 & assignment & 13 \\
    % \bottomrule
  \end{tabular}
  \caption{Frequency for each statement class in three test code samples}~\label{table:statement}
\end{table}



\subsection{Statement Parsing}
After classifying the statement, we need to break it down into independent parts according to the grammar rules.
Similar to a recursive-descent parser \cite{van1993recursive}, our system consists of a series of functions, each of which is responsible for one class of statement. Each function includes a set of mutually recursive procedures where each such procedure implements one of the productions of the grammar as a regular expression. We implement a top-down LL parser to parse the input from left to right and perform a leftmost derivation \cite{fernau1998regulated} of the statement. As a result, a statement is parsed into a list of single tokens and/or characters. For example, the statement in \autoref{fig:overview} is parsed into five individual tokens. Specifically, `if' is a keyword token; `Cookie. name' is a variable token; `==' is a symbol token; `naue' is a variable token; `;' is the last symbol token.



\subsection{Token Processing}
The previous stage results in a list of single tokens and/or characters that make up the statement.
%In this step, we process the list of the single words or characters received from the last step. 
We assume all non-keywords are properly recognized and add them to the lexicon assuming they are \emph{variable} names.
%We build a non-keyword lexicon to save all non-keywords in the first sentence. 
Then for all non-keywords in each statement that follows, we first compare the token to all the words in the non-keyword lexicon. If a `similar' token already exists in the lexicon, we replace it with the `similar' token in the lexicon. For example, in \autoref{fig:overview}, `naue' is very similar to `name', which is already in the lexicon, so we just replace the token `naue' with `name'. If there is no `similar' token in the lexicon, we accept this token as it is and add it to the lexicon. We calculate similarity using the Levenshtein distance \cite{levenshtein1966binary} with a threshold of 0.7, determined empirically.

\subsection{Statement concatenation}
After processing all tokens, we remove all extra spaces in any single token, then concatenate each token with a single space between them to reconstruct the final statement.
Additionally, we ensure that the last recognized character of a statement is a ':'.
For example, in \autoref{fig:overview}, we first remove the space in `cookie. name' and then replace the last character `;' with `:'.





\section{Evaluation}


%define WER and CER
To assess the performance of our system, we measure the Character Error Rate (CER) and Word Error Rate (WER). WER and CER are percentages obtained from the Levenshtein distance between the recognized sequence and the corresponding ground truth. They are calculated as
\[ \frac{D+I+S}{L} \times 100\% \]
where D is the number of deleted units, I is the number of inserted
units, S is the number of substituted units, and L is the total number of
units in the ground truth transcriptions. A unit is a word for WER or a
character for CER.


We evaluate our recognition approach by applying our framework to the 45 code samples in our database. In the following section, we compare the results of our enhanced recognizer to the results of using MyScript alone.

\section{Results} \label{results}

\begin{table}
  \centering
  \begin{tabular}{l r r}
    % \toprule
    {}
    & {t-test score ($t_{14}$)}
      & {P-value} \\
    \midrule
    WER on sample 1 & -9.02 & $P < 0.00001$ \\
    WER on sample 2 & -8.29 & $P < 0.00001$ \\
    WER on sample 3 & -6.57 &  $P < 0.00001$  \\
    CER on sample 1 & -3.88  & $P < 0.001$  \\
    CER on sample 2 & -5.45 & $P < 0.0001$ \\
    CER on sample 3 & -6.13 & $P < 0.0001$ \\
    % \bottomrule
  \end{tabular}
  \caption{Statistical evidence (T-test and P-value) for WER and CER on three code samples}~\label{table:significance}
\end{table}


%result
As shown in \autoref{result}, 
our augmented recognition approach results in an 8.6\% word error rate and 3.6\% character error rate, on average, over the three code samples, which outperforms the original MyScript recognizer with 31.31\%  and 9.24\% in word and character error rate respectively. We also find statistical evidence for an effect of our augmented recognition approach on both WER and CER (See \autoref{table:significance}). 

%on WER on sample \#1 ($t_{14} = -9.02, P < 0.00001$), on CER on sample #1 ($t_{14} = -3.88, P < 0.001$), WER on sample 2 ($t_{14} = -8.29, P < 0.00001$), CER on sample 2 ($t_{14} = -5.45, P < 0.0001$), WER on sample 3 ($t_{14} = -6.57, P < 0.00001$), and CER on sample3 ($t_{14} = -6.13, P < 0.0001$).


%compare to English handwriting recognition
%%Hi Ron, these four systems has lowest WER and CER, the 97% and 95% I said before is not English recognition, it's for arabic. 
\begin{table}
  \centering
  \begin{tabular}{l r r}
    % \toprule
    {System}
    & {WER(\%)}
      & {CER(\%)} \\
    \midrule
    \textcolor{red}{Augmented MyScript}  & \textcolor{red}{8.6} & \textcolor{red}{3.6} \\
    Kozielski et al. \cite{doetsch2013improvements} & 9.5 & 2.7 \\
    Keysers et al. \cite{keysers2016multi} & 10.4 &  4.3  \\
    Zamora et al. \cite{zamora2014neural} & 16.1  & 7.6  \\
    Poznanski et al. \cite{poznanski2016cnn} & 6.45 & 3.44 \\
    % \bottomrule
  \end{tabular}
  \caption{Performance of our system compared to handwritten English recognition systems on the IAM dataset}~\label{table:iam}
\end{table}

%To examine how our source code recognition rates compare
%to acceptable recognition rates in general handwriting recognizers,
%we present the recognition rate comparison between our augmented
%MyScript recognition system (on source code) to four other state-ofthe-art
%general handwritten English recognition systems

Since there is no existing handwriting source code recognizer for comparison, we compare the recognition rate of our our augmented MyScript recognition system (on source code) to that of four state-of-the-art general handwritten English recognition systems (on general text). 
%Since there is no existing handwriting source code recognizer for comparison, we compare our recognition rates on source code to acceptable recognition rates of general handwriting recognizers on general text.  Specifically, we compare the recognition rate of our our augmented MyScript recognition system (on source code) to that of four state-of-the-art general handwritten English recognition systems on general text. 
The IAM handwriting database \cite{marti2002iam} consists of 9,285 lines of general handwritten text
written by approximately 400 writers with no restrictions on style or writing tool. This database has been widely used to evaluate English handwriting recognition systems. The four systems in \autoref{table:iam} were tested based on this IAM handwriting database. \autoref{table:significance} shows that the WER and CER of our augmented source code recognition system are comparable with other state-of-the-art handwritten English recognition systems on general handwritten text.

%Finally, we acknowledge the importance of indentation in \textit{Python} source code.  We chose to ignore indentation for the purposes of this project to focus solely on recognition and would argue that recognition results will not be affected by indentation considerations.  We are in the process of introducing indentation into the collected data to create an identical set of source code samples with appropriate indentation.

%Because typing on a virtual keyboard is the standard input method on touchscreen devices, it's useful to examine how virtual keyboard typing error rates compare to those of handwritten source code recognition.
% %Regular typing is also compared with handwriting source code on a virtual keyboard. 
% For example, Almusaly et al. report a 7.81\% total error rate (TER) for typing java programs on a standard virtual keyboard as measured from 32 participants \cite{almusaly2015syntax}. TER, similar to CER, is a measure of the total number of errors (i.e., omissions, substitutions, and insertions) and corrections that are made in the resulting typed text. 


%explain how our system fix errors in characterization and existing errors

\section{Discussion}


%The improvement of recognition rate embodied in fixing all the three error types. 


Our approach achieved an 8.6\% word error rate and a 3.6\% character error rate on the collected dataset by taking the language grammar rules into account. Overall, improvement of our recognition pipeline over the baseline MyScript recognition engine can be attributed to addressing the three main error types identified in Section \ref{sec:characterizing}.  After statement concatenation, all unnecessary space errors in a single token are removed. Ensuring the last character of a statement eliminates 32\% of the symbol errors. Token processing fixes around 78\% of the word errors. 

Recognition results, however, are still not 100\% accurate. Initial inspection indicates that this is mainly due to the illegible or cursive handwriting of the participants and the incorrect recognition of symbols. Also, since one of our lexicons is dependent on the non-keywords already recognized in the code, incorrectly recognized words will also be added to the lexicon, thereby corrupting the lexicon and preventing it from enhancing the recognition of the following words. Additionally, it is difficult to identify incorrectly recognized symbols; for example, if `(' appearing in the middle of the text is recognized as `l', it becomes impossible to rectify it using our approach. 
%errors still existing, develop widget for errors, what will improve
%Our approach achieved an 8.6\% word error rate and a 3.6\% character error rate on the collected dataset by taking the language grammar rules into account. 
%It should be noted that there are still identified errors remaining to be corrected. 
Errors like unmatched `(' and `)' in a statement can be detected, but not reliably corrected. For example, `(name' can be recognized as `cname', but we have no evidence to correct `cname' to `(name'. Two methods can be employed to resolve remaining errors such as this. The first is to develop a widget in the handwriting interface to highlight all errors that are identified but can't be corrected and let users correct them manually. Another option is to train a language model to identify words that do not exist \cite{zamora2014neural}. 


Because typing on a virtual keyboard is the standard input method on touchscreen devices, it is useful to examine how virtual keyboard typing error rates compare to those of handwritten source code recognition.
% %Regular typing is also compared with handwriting source code on a virtual keyboard. 
Almusaly et al. report a 7.81\% total error rate (TER) for typing \textit{Java} programs on a standard virtual keyboard as measured from 32 participants \cite{almusaly2015syntax}. TER, similar to CER, is a measure of the total number of errors (i.e., omissions, substitutions, and insertions) and corrections that are made in the resulting typed text.  Our handwriting results are comparable.

%generalize to other languages
This approach can also be generalized to other programming languages with strict grammar rules. For instance, one can define statement classes for \textit{Java} according to the first word in the statement and then replace the regular expressions with productions of \textit{Java} grammar rules.  Algorithms for searching and replacing similar words can be kept unchanged. Other heuristic steps like concatenating tokens are also trivial to implement for new languages. 
%Due to the uncertainty of the program structure, however, our approach cannot easily be generalized to non-strict programming languages.

%\section{Discussion and Implication for HCI}
\section{Conclusion and Future Work}
%%mobile programming is widely used, and its limitation.
%Supporting programming on mobile devices is not a novel idea. With the rapid technology shift in current computing devices, high-quality low-cost mobile devices such as tablets and smartphones are being increasingly used in everyday activities. Many tasks that previously required a PC are now feasible on mobile devices. For example, tablets are typically equipped with powerful batteries, advanced graphic processors, high-resolution screens and fast processors, making writing and compiling code on them completely plausible. TouchDevelop \cite{tillmann2011touchdevelop}, for example, is a novel programming environment, language and code editor for mobile devices and Tilman et al. \cite{tillmann2012future} predict that programming on mobile devices will be widely used for teaching programming.  However, mobile devices are also inherently restricted by their limintations such as small screens and the clumsy virtual keyboard that is difficult to use and that takes up valuable screen space - posing challenges for a programmer who wishes to enter or edit source code.
%For this to happen, however, input must be made more intuitive.
%Alternatives to typing have been considered for a long time. One is to use speech (via dictation) to create the text of a program. Desilets et al. \cite{desilets2006voicecode} propose VoiceCode, which translates the pronounced syntax into native syntax in the current programming language to support programming by speech. Gordon \cite{gordon2013improving} employs language design and incorporates dynamic context for this purpose. Speech, however, is not always an appropriate option given social conventions and privacy issues. Given the advances of pen-based input technology, we chose to explore handwriting input in this paper.  

 
%conclusion
The keyboard is not an ideal input mechanism for every person and situation.
Alternatives to typing, such as speech, have been considered in the past \cite{desilets2006voicecode, gordon2013improving}. However, speech is not always an appropriate option given social conventions and privacy issues.
Given advances in pen-based technology that provides an opportunity for users to engage with devices in a potentially more `natural' way than that supported by a virtual keyboard, handwriting input is a viable alternative to virtual keyboard input. In this paper, we have explored handwriting recognition specifically for source code with the ultimate goal of supporting handwriting as a means for programming.
%Handwriting is a viable alternative to a keyboard given recent technological advances in pen-based technology.
%for programming can be used for people with disabilities and people suffering RSI. %In this paper, we focus on supporting handwriting recognition for the particular domain of source code text input. 
We collect and present a small database of publicly available handwritten source code samples and we propose an approach to recognize handwritten source code by leveraging a commercial handwriting recognition system. Experiments on the data collected from 15 participants show our framework has an average 8.6\% word error rate and 3.6\% character error rate which outperforms the baseline recognition system and produces rates comparable to the recognition of general handwritten English text.  We are encouraged by these initial results but believe there are several avenues of future work.


%Given the advances of pen-based input technology, we chose to explore handwriting input in this paper. %%writing is new viable, stylus technique
%Input via stylus is becoming more precise and the familiar writing action makes handwriting a viable alternative to typing code on mobile devices. 
%Given advances in pen-based technology that provides an opportunity for users to engage with devices in a potentially more `natural' way than that supported by a virtual keyboard, we have explored handwriting recognition for source code with the ultimate goal of supporting handwriting as a means for programming.
%Additionally, research suggests that handwriting can lead to cognitive, memory, and creativity enhancements \cite{alonso2015metacognition}. Alternative input mechanisms are also important for people with RSI or other motor impairments who find typing difficult, and in general, for those who may wish to carry out simple source code editing and entry tasks in mobile situations. 
%While handwriting without recognition produces `digital ink' that is appropriate for applications like annotation and graphic design, 
%For handwriting to be used in scenarios like programming, however, applications must be equipped with recognition technology to support translation to searchable and editable digital text. To effectively incorporate the handwriting experience for source code entry and editing, we must first address the source-code recognition problem.




%importance


  

% Summarize what we've done
%We have presented an initial study to collect data on handwritten source code and explored the use of a state-of-the-art recognition system for recognizing handwritten source code. 
%In this section, we talk about why it is important to HCI and how HCI community could benefit from our work.


%%writing is new viable, stylus technique
%Input via stylus is becoming more precise and the writing action is very similar to writing on paper. Handwriting is therefore a viable alternative to typing code on mobile devices. Moreover, handwriting is an acceptable input method for people with RSI or other motor impairments who find typing difficult. 
%Thus studying handwriting input method will be a benefit for a large group of users. 
%While handwriting without recognition produces `digital ink' that is appropriate for applications like annotation and graphic design, for handwriting to be used in scenarios like programming, it must be equipped with recognition technology to support translation to searchable and editable digital text.

%%implication and importance for HCI
From the view-point of human-computer interaction, usability and user satisfaction is critical. For handwriting text input, users expect recognition technology with a low error rate and responsive recognition speed. LaLomia et al. \cite{lalomia1994user} reported that users are willing to accept a recognition error rate of only 3\% (a 97\% recognition rate), although Frankish et al. \cite{frankish1995recognition} concluded that users will accept higher error rates depending on the text-editing task. It would not be surprising, therefore, if higher error rates were acceptable for source code entry and editing which is inherently difficult due primarily to the use of symbols. Input speed is another concern with respect to handwriting. Modest touch typing speeds on a virtual keyboard in the range of 20 to 40 words per minute (wpm) are achievable.
Handwriting speeds are commonly in the 15 to 25 wpm range \cite{card1983psychology,devoe1967alternatives,dunlop2009pickup}. We suspect that this decrease in speed, however, will be acceptable to the particular groups for whom handwriting is the most viable input option. Additionally, in professional programming, most of the code that developers
write involves reuse of existing example code and libraries \cite{bellon2007comparison}. This `reuse' typically amounts to editing existing code to suit a
new context or problem and generally provides benefits to developers in terms of time and error reduction \cite{ko2011state}.
%is able to save time and
%avoid the risk of writing erroneous new code \cite{ko2011state}.
For these reasons, we envision our system as being particularly useful in the code editing domain as opposed to writing extensive source code from scratch.  Studying how the algorithms perform in editing tasks is left as future work.

%a recognizer used primarily for code edits as opposed to being used to write an entire program from scratch.


% How does the HCI community benefit from this work
%% Dataset for testing

While databases exist for research in general handwritten text recognition \cite{marti2002iam, grosicki1rimes}, there is no such dataset for handwritten source code.  This paper represents the first such contribution of a handwritten source code dataset consisting of 555 lines of  \textit{Python} code written by 15 participants. While we recognize that using the same three code samples for all users and employing a ``copy task'' may lessen the generality of the dataset, we sought to eliminate all effects of cognitive complexity (e.g. actually solving programming problems) to focus solely on the handwritten source code quality.  Collecting data for other programming languages and for actual programming tasks is left as future work.

%A standard database is needed to facilitate research in handwritten source code recognition. For general handwritten text recognition, the IAM database \cite{marti2002iam} and the RIMES database \cite{grosicki1rimes} are widely used for research purposes. Based on Lancaster-Oslo/Bergen (LOB) corpus, the IAM database \cite{marti2002iam} consists of 9,285 lines of handwriting text from 400 writers. The RIMES Database comes from the ICDAR 2011 block-recognition competition and consists of 1,500 paragraphs of the handwritten French text.
%Unfortunately, there is no such dataset for handwritten source code - this paper represents the first such contribution.  We collected only a small handwritten source code dataset consisting of 555 lines written by 15 participants. While we recognize that using the same three code samples for all users and employing a ``copy task'' may lessen the generality of the dataset, we sought to eliminate all effects of cognitive complexity (e.g. actually solving programming problems) to focus solely on the handwritten source code quality.  Collecting data from actual programming tasks and for additional programming languages is left as future work.
%and affect the writing organization.   but cognitive load and possible errors are avoided. In addition, t
%It is expected that the database would be particularly useful for further handwritten source code recognition research using \textit{Python} as the language of choice.  More data on additional languages will be necessary to further investigate handwriting as a viable input mechanism for source code.



%% Approach based on state-of-the-art recognizer
%Our handwritten source code recognition framework is implemented by leveraging the programming language grammar information to augment an existing handwriting recognition system. By replacing the grammar rules in the framework, the system can be generalized to other programming languages. It should be noted, however, that using an existing handwriting recognition system designed for natural language is not tackling the problem at its source. Rebuilding the core of a recognition system based on properties of source code is thus an alternative approach that should be explored.  We leave this for future work.






%conclusion
%The keyboard is not an ideal input mechanism for everyone.
%Handwriting as an alternative to a keyboard for programming can be used for people with disabilities and people suffering RSI. In this paper, we focus on supporting the recognition aspect of handwriting for source code text input. We collect and present a small database of publicly available handwritten source code samples and we propose an approach to recognize handwritten source code by leveraging a commercial handwriting recognition system. Experiments on the data collected from 15 participants shows our framework has an average 8.6\% word error rate and 3.6\% character error rate which outperforms the baseline recognition system and produces rates comparable to recognition of general handwritten English text.

%future work

%%from scratch
%%IDE
%Clearly, the current work is limited in both scope and depth and 
%We are encouraged by these initial results but believe there are several avenue of future work. 

The next most obvious area of future work is to develop a handwritten source code recognition system from scratch instead of augmenting the results produced by an existing system.  We suspect this approach would lead to comparable and most likely improved recognition rates. Building a universal handwritten source code reading system could employ deep learning techniques such as Concurrent Neural Networks \cite{poznanski2016cnn} or neural network language models \cite{zamora2014neural} trained purely on the source code. 

Additionally, there are several opportunities to explore the integration of handwriting recognition into source code IDEs \cite{frye2008pdp}.  For example, how do we now integrate source code completion into a handwriting-based interaction?   Can we integrate elements such as syntax insertion and highlighting?  Exploring the affordances of handwriting in the context of an IDE is an exciting area of future work that is enabled by these initial findings.

%Similar to an IDE for handwriting C\# code \cite{frye2008pdp}, integrating handwriting source code to current programming IDE or building a programming IDE solely with handwriting as an input method is also valuable to explore.

%%speed, auto completion
%Furthermore, to address concerns pertain to handwriting speed, it may be possible to add auto-completion feature into handwriting source code interface. Next word or character suggestion is also helpful to facilitate inputting. In addition, combing voice input and handwriting input may also improve the inputting speed.

Multimodal methods present another area of future work.  Perhaps the combination of handwriting and speech input or handwriting and occasional keyboard input \cite{mueller2014pen} begin to produce interaction experiences that rival those of typed source code input.  

Finally, we will never reach a perfect recognition rate for handwritten text (general or source code).  How do we effectively support efficient editing of the recognized text so that users can quickly correct mistakes? Natural and effective text entry and editing is an interesting topic for future studies.

%%texue of future work.t editing
%, our research projeng avenct aims to develop a programming interface to support handwriting, editing and recognizing source code. We also intend to explore the potential of adding editing techniques such as selecting, deleting, copying, and pasting.






%% if specified like this the section will be committed in review mode
\acknowledgments{
The authors wish to thank all the study participants as well as Poorna Talkad Sukumar, Jason Liu, and Suwen Lin for their valuable discussions and input.}

%\bibliographystyle{abbrv}
\bibliographystyle{abbrv-doi}
%\bibliographystyle{abbrv-doi-narrow}
%\bibliographystyle{abbrv-doi-hyperref}
%\bibliographystyle{abbrv-doi-hyperref-narrow}

\bibliography{template}
\end{document}

\begin{figure*}[t]
\centering
%
\begin{minipage}{.15\textwidth}
\subfigure[Query $q_1$]{
  \includegraphics[width=1.\columnwidth]{figures/template.png}
\label{fig:template}
}
%
\subfigure[Workload $Q=\{q_1,q_2\}$]{
  \includegraphics[width=1.0\columnwidth]{figures/merged-template.png}
\label{fig:merged_template}
}
\vspace{-5pt}
\caption{Template}
\label{fig:templates}
\end{minipage}
\hspace{5mm}
%
\begin{minipage}{.8\textwidth}
\subfigure[Non-shared \greta\ graph]{
\includegraphics[width=0.5\columnwidth]{figures/not-shared-executor.png}
\label{fig:not-shared}
}
\hspace{3mm}
%
\subfigure[Shared \app\ graph]{
\includegraphics[width=0.4\columnwidth]{figures/snapshot.png}
\label{fig:snapshot}
}
\vspace{-5pt}
\caption{Non-shared vs shared execution}
\label{fig:greta_vs_hamlet}
\end{minipage}
\end{figure*}





%==================================================
\subsection{Non-Shared Online Trend Aggregation}
\label{sec:non-shared-baseline}

For the  non-shared execution, we describe below how the \app\ executor leverages state-of-the-art online trend aggregation approach
% \ear{I suggest we remove the name Greta here,
% and keep it a bit more generic.}
% chuan - fixed
~\cite{PLRM18} to compute trend aggregates for each query {\it independently from all other queries}. Given a query $q$, it encodes all trends matched by $q$ in a query graph.
% \ear{Why do we need to give
% this graph a name. can we simply call it query graph here.}
% chuan - fixed
The nodes in the graph are events matched by $q$. Two events $e'$ and $e$ are connected by an edge if $e'$ and $e$ are adjacent in a trend matched by $q$. The event $e'$ is called a \textbf{\textit{predecessor event}} of $e$. At runtime, trend aggregates are propagated along the edges. In this way, 
% \ear{Why can we not simply say ... In this way, we aggregate .... }
% chuan - fixed
we aggregate trends online, i.e., without actually constructing them.
%{\color{blue} Why say this here! because we do not want to admit that greta already solved the main complexity issue}
%Thus, it reduces the time complexity of trend aggregation from exponential to quadratic in the number of matched events compared to two-step approaches.

Assume a query $q$ computes the number of trends $\mycount(*)$. 
When an event $e$ is matched by $q$, $e$ is inserted in the graph for $q$ and the \textbf{\textit{intermediate trend count}} of $e$ (denoted $\mathit{count}(e,q)$) is computed. $\mathit{count}(e,q)$ corresponds to the number of trends that are matched by $q$ and end at $e$. 
%
If $e$ is of start type of $q$, $e$ starts a new trend. Thus, $\mathit{count}(e,q)$ is incremented by one (Equation~\ref{eq:start_event}).
%
In addition, $e$ extends all trends that were previously matched by $q$. Thus, $\mathit{count}(e,q)$ is incremented by the sum of the intermediate trend counts of the predecessor events of $e$ that were matched by $q$ (denoted $\textit{pe}(e,q)$) (Equation~\ref{eq:event_count}). 
%
The \textbf{\textit{final trend count}} of $q$ is the sum of intermediate trend counts of all matched events of end type of $q$ (Equation~\ref{eq:final_count}).
%
%\vspace*{-1mm}
\begin{align}
%
\mathit{start}(e,q) &=
    \begin{cases}
      1, & \text{if}\ \mathit{e.type} \in \mathit{start}(q) \\
      0, & \text{otherwise}
    \end{cases}
\label{eq:start_event}\\
%
\mathit{count}(e,q) &= 
\mathit{start}(e,q) + 
\sum_{e' \in \textit{pe}(e,q)}  \mathit{count}(e',q) 
\label{eq:event_count}\\
%
\mathit{fcount}(q) &= 
\sum_{\mathit{e.type} \in \textit{end}(q)}  \mathit{count}(e,q)
%
\label{eq:final_count}
\end{align}
\vspace{-5pt}


\begin{figure*}[t]
\centering
%
\subfigure[Snapshot $x$ at graphlet level]{
  \includegraphics[width=0.3\columnwidth]{figures/no-predicates.png}
\label{fig:no-predicates}
}
\hspace{2mm}
%
\subfigure[Snapshots $x$ and $y$ at graphlet level]{
  \includegraphics[width=0.3\columnwidth]{figures/snapshots.png}
\label{fig:snapshots}
}
\hspace{2mm}
%
\subfigure[Snapshot \textbf{\textit{z}} at event level]{
  \includegraphics[width=0.3\columnwidth]{figures/predicates.png}
\label{fig:predicates}
}
%
\vspace{-10pt}
\caption{Snapshots at graphlet and event levels}
\label{fig:snapshots_at_different_levels}
\end{figure*}


\begin{table*}[t]
    \begin{minipage}{.27\linewidth}
    \centering
    {\footnotesize\begin{tabular}{|c|p{2.5cm}|}
        \hline
        & \textbf{Trend count} 
        \\\hline\hline
        $b_3$ 
        & $x$ 
        \\\hline
        $b_4$ 
        & $x + count(b_3,Q) = 2x$ 
        \\\hline
        $b_5$ 
        & $x + count(b_3,Q) + count(b_4,Q) = 4x$ 
        \\\hline
        \multirow{2}{*}{$b_6$}
        & $x + count(b_3,Q) + count(b_4,Q) +$ 
        $count(b_5,Q) = 8x$ 
        \\\hline
    \end{tabular}}
    \caption{Shared propagation of \textbf{\textit{x}} within $\textbf{\textit{B}}_3$}
    \label{tab:snapshot}
    \end{minipage}
    %
    \begin{minipage}{.35\linewidth}
    \centering
    {\footnotesize\begin{tabular}{|c|p{1.8cm}|p{1.8cm}|}
        \hline
        & \textbf{Query} $q_1$
        & \textbf{Query} $q_2$ 
        \\\hline\hline
        $x$ 
        & $sum(A_1,q_1) = 2$
        & $sum(C_2,q_2) = 1$ 
        \\\hline
        \multirow{4}{*}{$y$}
        & $value(x,q_1) +$ 
        $sum(B_3,q_1) +$  
        $sum(A_4,q_1) =$ 
        $2 + 15*2 + 2 = 34$
        & $value(x,q_2) +$ 
        $sum(B_3,q_2) +$
        $sum(C_5,q_2) =$ 
        $1 + 15*1 + 3 = 19$ 
        \\\hline
    \end{tabular}}
    \caption{Values of snapshots \textbf{\textit{x}} and \textbf{\textit{y}} per query}
    \label{tab:snapshots}
    \end{minipage} 
    %
    \begin{minipage}{.35\linewidth}
    \centering
    {\footnotesize\begin{tabular}{|c|p{1.8cm}|p{1.8cm}|}
        \hline
        & \textbf{Query} $q_1$
        & \textbf{Query} $q_2$ 
        \\\hline\hline
        \multirow{3}{*}{$z$} 
        & $value(x,q_1) +$ 
        $count(b_3,q_1) +$ 
        $count(b_4,q_1) = 8$
        & $value(x,q_2) +$ 
        $count(b_3,q_2) = 2$ 
        \\\hline
        \multirow{3}{*}{$y$}
        & $value(x,q_1) +$ 
        $sum(B_3,q_1) +$ 
        $sum(A_4,q_1) = 34$
        & $value(x,q_2) +$ 
        $sum(B_3,q_2) +$ 
        $sum(C_5,q_2) = 15$ 
        \\\hline
    \end{tabular}}
    \caption{Values of snapshots \textbf{\textit{z}} and \textbf{\textit{y}} per query}
    \label{tab:snapshots2}
    \end{minipage} 
    %\caption{Snapshot maintenance}
    %\label{tab:snapshot_maintenance}
\end{table*}


%--------------------
\begin{example}
Continuing Example~\ref{ex:running_example}, a graph is maintained per each query in the workload $Q=\{q_1,q_2\}$ in Figure~\ref{fig:not-shared}. 
For readability, we sort all events by their types and timestamps. Events of types $A$, $B$, and $C$ are displayed as gray, white, and striped circles, respectively. We highlight the predecessor events of event $b_3$ by edges. All other edges are omitted for compactness. 
When $b_3$ arrives, two trends $(a_1,b_3)$ and $(a_2,b_3)$ are matched by $q_1$. Thus, $count$($b_3$,$q_1$) = $count(a_1,q_1) + count(a_2,q_1) = 2$. However, only one trend $(c_1,b_3)$ is matched by $q_2$. Thus, $count(b_3,q_2) = count(c_1,q_2) = 1$.
\end{example}

%\begin{table}[!htb]
%    \centering
%    \begin{tabular}{|c|p{2.7cm}|p{2.5cm}|}
%    \hline
%        Event $e$ 
%        & $count(e,q_1)$ 
%        & $count(e,q_2)$ \\\hline\hline
%        $a_1, a_2$ 
%        & 1 
%        & \\\hline
%        $c_1$ 
%        &  
%        & 1 \\\hline
%        $b_3$ 
%        & $count(a_1,q_1) +$ $count(a_2,q_1) = 2$ 
%        & $count(c_1,q_2) = 1$ \\\hline
%        $b_4$ 
%        & $count(a_1,q_1) +$ $count(a_2,q_1) +$ $count(b_3,q_1) = 4$ 
%        & $count(c_1,q_2) +$ $count(b_3,q_2) = 2$ \\\hline
%        $b_5$ 
%        & $count(a_1,q_1) +$ $count(a_2,q_1) +$ $count(b_3,q_1) +$ $count(b_4,q_1) = 8$ 
%        & $count(c_1,q_2) +$ $count(b_3,q_2) +$ $count(b_4,q_2) = 4$ \\\hline
%    \end{tabular}
%    \caption{Non-shared trend count propagation}
%    \label{tab:counts}
%\end{table}

%---------------
\textbf{Complexity Analysis}. 
Figure~\ref{fig:not-shared} illustrates that each event of type $B$ is stored and processed once for each query in the workload $Q$, introducing significant re-computation and replication overhead. 
Let $k$ denote the number of queries in the workload $Q$ and $n$  the number of events. Each query $q$ stores each matched event $e$ and computes the intermediate count of $e$ per Equation~\ref{eq:event_count}. All predecessor events of $e$ must be accessed, with $e$ having at most $n$ predecessor events.  
Thus, the time complexity of non-shared online trend aggregation is computed as follows:
\begin{equation}
\mathit{NonShared}(Q) = k \times n^2
\label{eq:nonshared-cost}
\end{equation}
\vspace{-3mm}

Events that are matched by $k$ queries are replicated $k$ times (Figure~\ref{fig:not-shared}). Each event stores its intermediate trend count. In addition, one final result is stored per query. Thus, the space complexity is $O(k \times n + k) = O(k \times n)$.

% \ear{Does it require connection edges from
% one event to other events; if this is not done
% well for certain semantics, could this not be
% quadratic in space to keep all pairwise edges?}
% chuan - i think here we are not storing these connection edges. only the raw input events ($n$) are stored.

% When an event $e$ is matched by a query $q \in Q$, \greta\ performs the following steps:

% (1)~To support expressive predicates of $q$, the events matched by $q$ are stored in a binary tree sorted by the attribute values that are accessed by the predicates of $q$~\cite{PLRM18}. Therefore, the time complexity of inserting $e$ into the tree is $O(\log_2(n))$.

% (2)~All predecessor events of $e$ are accessed to compute the intermediate aggregate of $e$ according to Equation~\ref{eq:event_count}. In the worst case, all events previously matched by query $q$ are predecessor events of $e$. Thus, the time complexity of accessing all predecessors of $e$ is $O(n)$.


%=======================
\subsection{Shared Online Trend Aggregation}
\label{sec:shared-approach}

In Equation~\ref{eq:nonshared-cost}, the overhead of processing each event once per query in the workload $Q$ is represented by the multiplicative factor $k$. 
Since the number of queries in a production workload may reach hundreds to thousands~\cite{ADLS, shared_clouds}, this re-computation overhead can be significant.
Thus, we design an efficient shared online trend aggregation strategy that encapsulates bursts of events of the same type in a graphlet such that the propagation of trend aggregates  within these graphlets can be shared among several queries. 

% In addition, graphlets enable sharing decisions at a finer granularity (Section~\ref{sec:runtime}). 

\begin{definition}(\textbf{Graphlet})
%
Let $q \in Q$ be a query and $T$ be a set of event types that appear in the pattern of $q$. 
%
A \textit{graphlet} $G_E$ is a graph of events of type $E$, if no events of type $E' \in T,\ E' \neq E,$ are matched by $q$ during the time interval $(e_\mathit{f}.time, e_l.time)$, where $e_\mathit{f}.time$ and $e_l.time$ are the time\-stamps of the first and the last events in $G_E$, respectively.
% Let $G_E$ be the graph of events of type $E$.
% Let $e_\mathit{f}$ be the first (i.e., an event with minimal timestamp) and $e_l$ be the last event (i.e., an event with maximal timestamp) in $G_E$.
% We call $G_E$ a graphlet if no events of type $E_1$ or $E_2$ are matched by the query $q$ during the time interval $(e_\mathit{f}.time, e_l.time)$.
%
If new events can be added to a graphlet $G_E$ without violating the constraints above, the graphlet $G_E$ is called \textit{active}. Otherwise, $G_E$ is called \textit{inactive}.
%
\label{def:graphlet}
\end{definition}

\begin{definition}(\textbf{Shared Graphlet, \app\ Graph})
%
Let $E+$ be a Kleene sub-pattern that is shareable by queries $Q_E \subseteq Q$ (Definition~\ref{def:shareable-sub-pattern}). 
We call a graphlet $G_E$ of events of type $E$ a shared graphlet. 
The set of all interconnected shared and non-shared graphlets for a workload $Q$ is called a \app\ graph.
%
\label{def:shared-graphlet}
\end{definition}

\begin{example}
In Figure~\ref{fig:snapshot}, matched events are partitioned into six graphlets $A_1$--$B_6$ by their types and timestamps. For example, graphlets $B_3$ and $B_6$ are of type $B$. They are shared by queries $q_1$ and $q_2$.
In contrast to the non-shared strategy in Figure~\ref{fig:not-shared}, each event is stored and processed once for the entire workload $Q$. 
Events in $A_1$--$C_2$ are predecessors of events in $B_3$, while events in $A_1$--$C_5$ are predecessors of events in $B_6$. 
For readability, only the predecessor events of $b_3$ are highlighted by edges in Figure~\ref{fig:snapshot}. All other edges are omitted. 
$a_1$ and $a_2$ are predecessors of $b_3$ only for $q_1$, while $c_1$ is a predecessor of $b_3$ only for $q_2$. 
\label{ex:three}
\end{example}

Example~\ref{ex:three} illustrates the following two challenges of online shared event trend aggregation.

\textit{\textbf{Challenge 1}}. 
Given that event $b_3$ has different predecessors for queries $q_1$ and $q_2$, the computation of the intermediate trend count of $b_3$ (and all other events in graphlets $B_3$ and $B_6$) cannot be directly shared by queries $q_1$ and $q_2$.

\textit{\textbf{Challenge 2}}. 
If queries $q_1$ or $q_2$ have predicates, then not all previously matched events are qualified to contribute to the trend count of a new event. Assume that the edge between events $b_4$ and $b_5$ holds for $q_1$ but not for $q_2$ due to predicates, and all other edges hold for both queries. Then $count(b_4,q_1)$ contributes to $count(b_5,q_1)$, but $count(b_4,q_2)$ does not contribute to $count(b_5,q_2)$.

We tackle these challenges by introducing \textbf{\textit{snapshots}}.
Intuitively, a snapshot is a variable that its value corresponds to an intermediate trend aggregate per query. 
In Figure~\ref{fig:snapshot}, the propagation of a snapshot $x$ within graphlet $B_3$ is shared by queries $q_1$ and $q_2$. 
We store the values of $x$ per query (e.g., $x=2$ for $q_1$ and $x=1$ for $q_2$).

\begin{definition}(\textbf{Snapshot at Graphlet Level})
%
Let $E'$ and $E$ be distinct event types.
Let $E+$ be a Kleene sub-pattern that is shared by queries $Q_E \subseteq Q$, $q \in Q_E$.
Let $E' \in \mathit{pt}(E,q)$ and $G_{E'}$ and $G_E$ be graphlets of events of types $E'$ and $E$, respectively.
Assume for any events $e' \in G_{E'}, e \in G_E$, $e'.time < e.time$ holds. 
A snapshot $x$ of the graphlet $G_{E'}$ is a variable whose value is computed per query $q$ and corresponds to the intermediate trend count of the query $q$ at the end of the graphlet $G_{E'}$. 
%
\begin{equation}
\mathit{value}(x,q) = \mathit{sum}(G_{E'},q) = \sum_{e' \in G_{E'}} \mathit{count}(e',q)
\label{eq:snapshot}
\end{equation}

The propagation of snapshot $x$ through the graphlet $G_E$ follows Equation~\ref{eq:event_count} and is shared by queries $Q_E$.
%
\label{def:snapshot}
\end{definition}



% Connections between Snapshots and Graphlets
\begin{example}
When graphlet $B_3$ starts, a snapshot $x$ is created. $x$ captures the intermediate trend count of query $q_1$ ($q_2$) based on the intermediate trend counts of all events in graphlet $A_1$ ($C_2$). $x$ is propagated through graphlet $B_3$ as shown in Figure~\ref{fig:no-predicates} and Table~\ref{tab:snapshot}. 

Analogously, when graphlet $B_6$ starts, a new snapshot $y$ is created. The value of $y$ is computed for queries $q_1$ ($q_2$) based on the value of $x$ for $q_1$ ($q_2$) and graphlets $B_3$ and $A_4$ ($C_5$). Figure~\ref{fig:snapshots} illustrates the connections between snapshots and graphlets. The edges from graphlets $A_1$ and $A_4$ ($C_2$ and $C_5$) hold only for query $q_1$ ($q_2$). Other edges hold for both queries $q_1$ and $q_2$.

% Values of Snapshots
Table~\ref{tab:snapshots} captures the values of snapshots $x$ and $y$ per query. For compactness, $sum(A_1,q_1)$ denotes the sum of intermediate trend counts of all events in $A_1$ that are matched by $q_1$ (Equation~\ref{eq:snapshot}). 
When the snapshot $y$ is created, the value of $x$ per query is plugged in to obtain the value of $y$ per query. The propagation of $y$ through $B_6$ is shared by $q_1$ and $q_2$. In this way, only one snapshot is propagated at a time to keep the overhead of snapshot maintenance low.
\end{example}

To enable shared trend aggregation despite expressive predicates, we now introduce snapshots at the event level.


\begin{definition}(\textbf{Snapshot at Event Level})
%
Let $G_E$ be a graphlet that is shared by queries $Q_E \subseteq Q$. Let $q_1,q_2 \in Q_E$ and $e_1, e_2 \in G_E$ be events such that the edge $(e_1,e_2)$ holds for $q_1$ but does not hold for $q_2$ due to predicates. 
A snapshot $z$ is the intermediate trend count of $e_2$ that is 
computed for $q_1$ and $q_2$ per Equation~\ref{eq:event_count} and propagated through the graphlet $G_E$ for all queries in $Q_E$.
%
\label{def:snapshot2}
\end{definition}
%
% For simplicity in Figures~\ref{fig:not-shared}--\ref{fig:snapshots} and Tables~\ref{tab:counts}--\ref{tab:snapshots}, we assumed that no predicates on adjacent events are specified in the queries. Thus, when an event $e$ of type $E$ arrives, all previously matched events of predecessor types of $E$ contribute to the intermediate trend count of $e$. 
% However, predicates may prevent some of the predecessor events of $e$ from passing their intermediate trend counts to $e$. To enable shared trend aggregation despite expressive predicates, we now introduce snapshots at the event level.

% chuan - what was before? it is not clear to me this `any` thing.
% olga - rephrased


\begin{example}
In Figure~\ref{fig:predicates}, assume that the edge between events $b_4$ and $b_5$ holds for query $q_1$ but not for query $q_2$ due to predicates. All other edges hold for both queries. Then, $count(b_4,q_1)$ contributes to $count(b_5,q_1)$, but $count(b_4,q_2)$ does not contribute to $count(b_5,q_2)$. To enable shared processing of graphlet $B_3$ despite predicates, we introduce a new snapshot $z$ as the intermediate trend count of $b_5$ and propagate both snapshots $x$ and $z$ within graphlet $B_3$. 
Table~\ref{tab:snapshots2} summarizes the values of $z$ and $y$ per query. 
\end{example}

%-----------------
\textbf{Shared Online Trend Aggregation Algorithm} computes the number of trends per query $q \in Q$ in the stream $I$. For simplicity, we assume that the stream $I$ contains events within one pane.
%
For each event $e \in I$ of type $E$, Algorithm~\ref{algo:snapshot-propagation} constructs the \app\ graph and computes the trend count as follows. 

\textbf{\textit{\app\ graph construction}} (Lines~4--14).
When an event $e$ of type $E$ is matched by a query $q \in Q$, $e$ is inserted into a graphlet $G_E$ that stores events of type $E$ (Line~14). 
%
if there is no active graphlet $G_E$ of events of type $E$, we create a new graphlet $G_E$, mark it as active and store it in the \app\ graph $G$ (Lines~7--8). If the graphlet $G_E$ is shared by queries $Q_E \subseteq Q$, then we create a snapshot $x$ at graphlet level (Line~9). $x$ captures the values of intermediate trend counts per query per Equation~\ref{eq:snapshot} at the end of graphlet $G_{E'}$ that stores events of type $E',\ E' \in pt(E,q)$. We save the value of $x$ per query in the table of snapshots $S$ (Lines~10--13).
%
Also, for each query $q \in Q$ with event types $T$, we mark all graphlets $G_{E'}$ of events of type $E' \in T,\ E' \neq E,$ as inactive (Lines~4--6). 

\textbf{\textit{Trend count computation}} (Lines~16--24).
If $G_E$ is shared by queries $Q_E \subseteq Q$ and the set of predecessor events of $e$ is identical for all queries $q \in Q_E$, then we compute $count(e,q)$ per Equation~\ref{eq:event_count} (Lines~16--18).
%
If $G_E$ is shared but the sets of predecessor events of $e$ differ among the different queries in $Q_E$ due to predicates, then we create a snapshot $y$ as the intermediate trend count of $e$ (Line~19). 
We compute the value of $y$ for each query $q \in Q_E$ per Equation~\ref{eq:event_count} and save it in the table of snapshots $S$ (Line~20).
%
If $G_E$ is not shared, the algorithm defaults to the non-shared trend count propagation per Equation~\ref{eq:event_count} (Line~21).
%
If $E$ is an end type for a query $q \in Q$, we increment the final trend count of $q$ in the table of results $R$ by the intermediate trend count of $e$ for $q$ per Equation~\ref{eq:final_count} (Lines~22--23).
%
Lastly, we return the table of results $R$ (Line~24).

%----------------
\begin{theorem}
Algorithm~\ref{algo:snapshot-propagation} returns correct event trend count for each query in the workload $Q$. 
\end{theorem}

\begin{proof}[Proof Sketch]
%
Correctness of the graph construction for a single query and the non-shared trend count propagation through the graph as defined in Equation~\ref{eq:event_count} are proven in~\cite{PLRM18}. 
Correctness of the snapshot computation per query as defined in Equation~\ref{eq:snapshot} follows from Equation~\ref{eq:event_count}. 
Algorithm~\ref{algo:snapshot-propagation} propagates snapshots through the \app\ graph analogously to trend count propagation through the \greta\ graph defined in~\cite{PLRM18}.
%
\end{proof}

%------------------
\textbf{Data Structures}.
%
Algorithm~\ref{algo:snapshot-propagation} utilizes the following physical data structures.

(1) \textbf{\textit{\app\ graph}} $G$ is a set of all graphlets. Each graphlet has two metadata flags \textit{active} and \textit{shared} (Definitions~\ref{def:graphlet} and \ref{def:shared-graphlet}). 

(2) \textbf{\textit{A hash table of snapshot coefficients}} per event $e$. The intermediate trend count of $e$ may be an expression composed of several snapshots. 
In Figure~\ref{fig:predicates}, $count(b_6,Q) = 4x + z$. 
Such composed expressions are stored in a hash table per event that maps a snapshot to its coefficient. In this example, $x \mapsto 4$ and $z \mapsto 1$ for $b_6$.

(3) \textbf{\textit{A hash table of snapshots}} $S$ is a mapping from a snapshot $x$ and a query $q$ to the value of $x$ for $q$ (Tables~\ref{tab:snapshots} and \ref{tab:snapshots2}). 

(4) \textbf{\textit{A hash table of trend count results}} $R$ is a mapping from a query $q$ to its corresponding trend count.

%% This declares a command \Comment
%% The argument will be surrounded by /* ... */
\SetKwComment{Comment}{/* }{ */}

\begin{algorithm}[t]
\caption{Training Scheduler}\label{alg:TS}
% \KwData{$n \geq 0$}
% \KwResult{$y = x^n$}
\LinesNumbered
\KwIn{Training data $\mathcal{D}_{train}=\{(q_i, a_i, p_i^+)\}_{i=1}^m$, \\
\qquad \quad Iteration number $L$.}
\KwOut{A set of optimal model parameters.}

\For{$l=1,\cdots, L$}{
    Sample a batch of questions $Q^{(l)}$\\
    \For{$q_i\in Q^{(l)}$}{
        $\mathcal{P}_{i}^{(l)} \gets \mathrm{arg\,max}_{p_{i,j}}(\mathrm{sim}(q_i^{en},p_{i,j}),K)$\\
        $\mathcal{P}_{Gi}^{(l)} \gets \mathcal{P}_{i}^{(l)}\cup\{p^+_i\}$\\
        Compute $\mathcal{L}^i_{retriever}$, $\mathcal{L}^i_{postranker}$, $\mathcal{L}^i_{reader}$\\ according to Eq.\ref{eq:retriever}, Eq.\ref{eq:rerank}, Eq.\ref{eq:reader}\\
    }
    % $\mathcal{L}^{(l)}_{retriever} \gets \frac{1}{|Q^{(l)}|}\sum_i\mathcal{L}^i_{retriever}$\\
    % $\mathcal{L}^{(l)}_{retriever} \gets \mathrm{Avg}(\mathcal{L}^i_{retriever})$,
    % $\mathcal{L}^{(l)}_{rerank} \gets \mathrm{Avg}(\mathcal{L}^i_{rerank})$,
    % $\mathcal{L}^{(l)}_{reader} \gets \mathrm{Avg}(\mathcal{L}^i_{reader})$\\
    % Compute $\mathcal{L}^{(l)}_{retriever}$, $\mathcal{L}^{(l)}_{rerank}$, and $\mathcal{L}^{(l)}_{reader}$ by averaging over $Q^{(l)}$\\
    $\mathcal{L}^{(l)} \gets \frac{1}{|Q^{(l)}|}\sum_i(\mathcal{L}^{i}_{retriever} + \mathcal{L}^{i}_{postranker}+ \mathcal{L}^{i}_{reader})$\\
    $\mathcal{P}^{(l)}_K\gets\{\mathcal{P}^{(l)}_i|q_i\in Q^{(l)}\}$,\quad $\mathcal{P}^{(l)}_{KG}\gets\{\mathcal{P}^{(l)}_{Gi}|q_i\in Q^{(l)}\}$\\
    Compute the coefficient $v^{(l)}$ according to Eq.~\ref{eq:v}\\
  \eIf{$ v^{(l)}=1$}{
    $\mathcal{L}^{(l)}_{final} \gets \mathcal{L}^{(l)}(\mathcal{P}_{KG}^{(l)})$\\
  }{
      $\mathcal{L}^{(l)}_{final} \gets \mathcal{L}^{(l)}(\mathcal{P}^{(l)}_{K}),$\\
    }
    Optimize $\mathcal{L}^{(l)}_{final}$
}
\end{algorithm}


%  \eIf{$ \mathcal{L}^{(l-1)}_{retriever}<\lambda$}{
%     $\mathcal{L}^{(l)}_{final} \gets \mathcal{L}^{(l)}(\mathcal{P}_K^{(l)})$\\
%   }{
%       $\mathcal{L}^{(l)}_{final} \gets \mathcal{L}^{(l)}(\mathcal{P}^{(l)}_{KG}),$\\
%     }

\textbf{Complexity Analysis}.
%
%{\color{blue}
We use the notations in Table~\ref{tab:notation} and Algorithm~\ref{algo:snapshot-propagation}.  
%
For each event $e$ that is matched by a query $q \in Q$, Algorithm~\ref{algo:snapshot-propagation} computes the intermediate trend count of $e$ in an online fashion. This requires access to all predecessor events of $e$. In the worst case, $n$ previously matched events are the predecessor events of $e$. Since the intermediate trend count of $e$ can be an expression that is composed of $s$ snapshots, the intermediate trend count of $e$ is stored in the hash table that maps snapshots to their coefficients. Thus, the time complexity of intermediate trend count computation is $O(n \times s)$. In addition, the final trend count is updated per query $q$ if $E$ is an end type of $q$ in $O(k \times s)$ time. In summary, the time complexity of trend count computation is $O(n \times (n \times s + k \times s)) = O(n^2 \times s)$ since $n \geq k$.

% \ear{why this one extra graphlet above? why do all events go into it, could some not be invalid due to  a predicate?}
% olga: rephrased

In addition, Algorithm~\ref{algo:snapshot-propagation} maintains snapshots to enable shared trend count computation.
To compute the values of $s$ snapshots for each query $q$ in the workload of $k$ queries, the algorithm accesses $g$ events in $t$ graphlets $G_{E'}$ of events of type $E' \in T,\ E' \neq E$. Thus, the time complexity of snapshot maintenance is $O(s \times k \times g \times t)$. 
%
In summary,  time complexity of Algorithm~\ref{algo:snapshot-propagation} is computed as follows:
%
\begin{equation}
\mathit{Shared}(Q) = n^2 \times s + s \times k \times g \times t
\label{eq:shared-cost}
\end{equation}
%The trade-off between the cost of snapshot maintenance in the shared strategy versus the overhead of re-computations in the non-shared strategy determines the benefit of sharing (Section~\ref{sec:runtime}).

Algorithm~\ref{algo:snapshot-propagation} stores each matched event in the \app\ graph once for the entire workload.
Each shared event stores a hash table of snapshot coefficients.
Each non-shared event stores its intermediate trend count.
In addition, the algorithm stores snapshot values per query. 
Lastly, the algorithm stores one final result per query.
%
Thus, the space complexity is $O(n + n \times s + s \times k + k) = O(n \times s + s \times k)$.

\section{Dynamic Sharing Optimizer}
\label{sec:runtime}

% \ear{Should we  call this section the Dynamic Sharing Optimizer?}
% olga: done


We first model the runtime benefit of sharing trend aggregation  (Section~\ref{sec:dynamic-benefit}). Based on this benefit model, our \app\ optimizer makes runtime sharing decisions for a given set of queries (Section~\ref{sec:decisions}).
% \ear{the sentence above and the one below sound the same --
% both choose that same set of queries?? what is the difference?
% if the above queries did not have this
% kleene subpattern, they could not share as per our definition and thus there would be no benefit calculated for that query set.}
% chuan - fixed.
Lastly, we describe how to choose a set of queries that share a Kleene sub-pattern (Section~\ref{sec:queries}).

% chuan - optimizer sounds static/offline. Shall we rename it as \app dynamic or online optimizer?
% olga - fixed

%%%%%%%%%%%%%%%%%%%%%%%%%%%%%%%%%%%%%%%%%%%%%%%%%%%%%%%%%%%
\subsection{Dynamic Sharing Benefit Model}
\label{sec:dynamic-benefit}

On the up side, shared trend aggregation avoids the re-computation overhead for each query in the workload. 
On the down side, it introduces overhead to maintain snapshots. 
Next, we quantify the trade-off between shared versus non-shared execution.

Equations~\ref{eq:nonshared-cost} and \ref{eq:shared-cost} determine the cost of non-shared and shared strategies of all events within the  window for the entire workload $Q$ based on stream statistics.
In contrast to these coarse-grained static decisions, the \app\ optimizer makes \textbf{\textit{fine-grained runtime decisions}} for each burst of events for a sub-set of queries $Q_E \subseteq Q$. 
%
Intuitively, a burst is a set of consecutive events of type $E$, the processing of which can be shared by  queries $Q_E$ that contain a $E+$ Kleene sub-pattern. The \app\ optimizer decides at runtime if sharing a burst is beneficial. In this way, beneficial sharing opportunities are harvested for each burst at runtime.

% More specifically, if $E+$ is a Kleene sub-pattern that is shareable by queries $Q_E$ and a burst of events of type $E$ arrives, our dynamic \app\ optimizer computes the benefit of sharing the graphlet $G_E$ of type $E$ by the queries $Q_E$. It then instructs the executor to either extend existing graphlets, 
%or split a shared graphlet $G_E$ into a set of non-shared graphlets $G_E^i$, 
%or merge non-shared graphlets $G_E^i$ into one shared graphlet $G_E$
%(Section~\ref{sec:decisions}).

% chuan - define burst
% chuan - the more i read this E+, the more i think we should make it as the GENERAL case in Sec. 2. and say E without Kleene is a special/easy case to handle.
% olga - added this clarification after Definition 3

\begin{definition}(\textbf{Burst of Events})
%
Let $E+$ be a sub-pattern that is sharable by queries $Q_E$.
Let $T$ be the set of event types that appear in the patterns of queries $Q_E$, $E \in T$. 
%
A set of events of type $E$ within a pane is called a \textit{burst} $B_E$, if no events of type $E' \in T,\ E' \neq E,$ are matched by the queries $Q_E$ during the time interval $(e_\mathit{f}.time, e_l.time)$, where $e_\mathit{f}.time$ and $e_l.time$ are the time\-stamps of the first and the last events in $B_E$, respectively.
%
If no events can be added to a burst $B_E$ without violating the above constraints, the burst $B_E$ is called \textit{complete}.
\label{def:burst}
\end{definition}

Within each pane, events that belong to the same burst are buffered until a burst is complete. The arrival of an event of type $E'$ or the end of the pane indicates that the burst is complete. In the following, we refer to complete bursts as bursts for compactness.
% Since we require that all events in a burst are consecutive, there can be only one burst at a time per pane.

\app\ restricts event types in a burst for the following reason. Assuming that a burst contained an event $e$ of type $E'$, the event $e$ could be matched by one query $q_1$ but not by another query $q_2$ in $Q_E$. Snapshots would have to be introduced to differentiate between the aggregates of $q_1$ and $q_2$ (Section~\ref{sec:shared-approach}). Maintenance of these snapshots may reduce the benefit of sharing. Thus, the previous sharing decision may have to be reconsidered as soon as the first event arrives that is matched  by some queries in $Q_E$. 


%\ear{Figure 6 caption - why is it so short and none-descript? try to make it more helpful.}
% olga: done

%--------------------------
\begin{definition}(\textbf{Dynamic Sharing Benefit})
%
Let $E+$ be a Kleene sub-pattern that is shareable by queries $Q_E$,
$B_E$ be a burst of events of type $E$,
$b$ be the number of events in $B_E$,
$s_c$ be the number of snapshots that are created from this burst $B_E$, and
$s_p$ be the number of snapshots that are propagated to compute the intermediate trend counts for the burst $B_E$. 
Let $G_E$ denote a shared graphlet and $G_E^i$ denote a set of non-shared graphlets (one graphlet per each query in $Q_E$).
Other notations are consistent with previous sections (Table~\ref{tab:notation}).

The \textit{benefit} of sharing a graphlet $G_E$ by the queries $Q_E$ is computed as the difference between the cost of the non-shared and shared execution of the burst $B_E$.

\vspace{-4mm}
\begin{align}
&\mathit{Shared}(G_E,Q_E) 
= b \times n \times s_p
+ s_c \times k \times g \times t
\nonumber\\
&\mathit{NonShared}(G_E^i,Q_E) 
= k \times b \times n
\nonumber\\
&\mathit{Benefit}(G_E,Q_E) 
= \mathit{NonShared}(G_E^i,Q_E)
- \mathit{Shared}(G_E,Q_E)
\label{eq:dynamic-benefit}
\end{align}

If $\mathit{Benefit}(G_E,Q_E)>0$, then it is beneficial to share trend aggregation within the graphlet $G_E$ by the queries $Q_E$.
%
\label{def:dynamic-benefit}
\end{definition}

Based on Definition~\ref{def:dynamic-benefit}, we conclude that the more queries $k$ share trend aggregation, the more events $g$ are in shared graphlets, and the fewer snapshots $s_c$ and $s_p$ are maintained at a time, the higher the benefit of sharing will be. Based on this conclusion, our dynamic \app\ optimizer decides to share or not to share online trend aggregation (Section~\ref{sec:decisions}).

% The formal proof is omitted here for compactness.

% chuan - this is not intuitive from Eq. 5. Do we want to prove it?
% olga - no space here, but Allison has a proof of this

% chuan - the distinction between Def.8 (Eq. 5) and Def.9 (Eq. 6) is not clear to me. the dynamic part is simply using the up-to-date statistics to compute the benefit.
% olga - correct. I merged these sections.

%--------------------------
\begin{definition}(\textbf{Dynamic Sharing Benefit})
%
Let $E+$ be a Kleene sub-pattern that is shareable by queries $Q_E$,
$b$ be the number of events of type $E$ in a burst,
$s_c$ be the number of snapshots that are created from this burst, and
$s_p$ be the number of snapshots that are propagated to compute the intermediate trend counts for the burst. 
Let $G_E$ denote a shared graphlet and $G_E^i$ denote a set of non-shared graphlets (one graphlet per each query in $Q_E$).
Other notations are consistent with previous sections (Table~\ref{tab:notation}).

The \textit{benefit} of sharing a graphlet $G_E$ by the queries $Q_E$ is computed as the difference between the cost of the non-shared and shared execution of the event burst.

\vspace{-3mm}
\begin{align}
\mathit{Shared}(G_E,Q_E) 
&= s_c \times k \times g \times p 
+ b \times (\log_2(g) + n \times s_p) \nonumber\\
\mathit{NonShared}(G_E^i,Q_E) 
&= k \times b \times (\log_2(g) + n) \nonumber\\
\mathit{Benefit}(G_E,Q_E) 
&= \mathit{NonShared}(G_E^i,Q_E)
- \mathit{Shared}(G_E,Q_E)
\label{eq:dynamic-benefit}
\end{align}

If $\mathit{Benefit}(G_E,Q_E)>0$, then it is beneficial to share trend aggregation within the graphlet $G_E$ by the queries $Q_E$.
%
\label{def:dynamic-benefit}
\end{definition}

Based on Definition~\ref{def:dynamic-benefit}, we conclude that the more queries $k$ share trend aggregation, the more events $g$ are in shared graphlets, and the fewer snapshots $s_c$ and $s_p$ are maintained at a time, the higher the benefit of sharing will be. Based on this conclusion, our dynamic \app\ optimizer decides to share or not to share online trend aggregation (Section~\ref{sec:decisions}).

% The formal proof is omitted here for compactness.

% chuan - this is not intuitive from Eq. 5. Do we want to prove it?
% olga - no space here, but Allison has a proof of this

% chuan - the distinction between Def.8 (Eq. 5) and Def.9 (Eq. 6) is not clear to me. the dynamic part is simply using the up-to-date statistics to compute the benefit.
% olga - correct. I merged these sections.

%%%%%%%%%%%%%%%%%%%%%%%%%%%%%%%%%%%%%%%%%%%%%%%%%%%%%%%%%%%
\begin{figure*}[t]
\centering
%
\subfigure[{\tiny Shared $B_3$}]{
  \includegraphics[width=0.13\columnwidth]{figures/decision_1.png}
\label{fig:benefit-monitoring-shared}
}
\subfigure[{\tiny Non-shared~$B_3$}]{
  \includegraphics[width=0.12\columnwidth]{figures/decision_2.png}
\label{fig:benefit-monitoring-non-shared}
}
%\hspace*{2mm}
%
\subfigure[{\tiny Shared $B_3$}]{
  \includegraphics[width=0.12\columnwidth]{figures/decision_3.png}
\label{fig:decision-not-to-share-shared}
}
\subfigure[{\tiny Non-shared~$B_4,B_5$}]{
  \includegraphics[width=0.16\columnwidth]{figures/decision_4.png}
\label{fig:decision-not-to-share-non-shared}
}
%\hspace*{2mm}
%
\subfigure[{\tiny Non-shared~$B_4,B_5$}]{
  \includegraphics[width=0.16\columnwidth]{figures/decision_5.png}
\label{fig:decision-to-share-non-shared}
}
\subfigure[{\tiny Shared $B_6$}]{
  \includegraphics[width=0.21\columnwidth]{figures/decision_6.png}
\label{fig:decision-to-share-shared}
}
%
\caption{Dynamic sharing decisions. 
Decision to merge $B_3$ in (a) and (b).
Decision to split $B_3$ in (c) and (d).
Decision to merge $B_6$ in (e) and (f).}
\label{fig:desicions}
\end{figure*}

\subsection{Decision to Split and Merge Graphlets}
\label{sec:decisions}
% chuan - i like this subsection, clear and convincing.

Our dynamic \app\ optimizer monitors the sharing benefit depending on changing stream conditions at runtime.
Let $B+$ be a sub-pattern sharable by queries $Q_B = \{q_1,q_2\}$.
In Figure~\ref{fig:desicions}, 
pane boundaries are depicted as dashed vertical lines
and bursts of newly arrived events of type $B$ are shown as bold empty circles. For each burst, the optimizer has a choice of sharing (Figure~\ref{fig:benefit-monitoring-shared}) versus not sharing (Figure~\ref{fig:benefit-monitoring-non-shared}). It concludes that it is beneficial to share based on calculations in Equation~\ref{eq:benefit-monitoring}.
 
\vspace{-4mm}
\begin{align}
&\mathit{Shared}(B_3,Q_B) 
= 4 \times 7 \times 1 
+ 1 \times 2 \times 4 \times 2
= 44
\nonumber\\
&\mathit{NonShared}(\{B_3,B_4\},Q_B) 
= 2 \times 4 \times 7 = 56
\nonumber\\
&\mathit{Benefit}(B_3,Q_B)
= 56 - 44  = 12 > 0
\label{eq:benefit-monitoring}
\end{align}

%--------------------------------------------------------------------------------
\textbf{Decision to Split}.
%
However, when the next burst of events of type $B$ arrives, a new snapshot $y$ has to be created due to predicates during the shared execution in Figure~\ref{fig:decision-not-to-share-shared}. 
In contrast, the non-shared strategy processes queries $q_1$ and $q_2$ independently from each other (Figure~\ref{fig:decision-not-to-share-non-shared}). 
Now the overhead of snapshot maintenance is no longer justified by the benefit of sharing (Equation~\ref{eq:decision-not-to-share}).

\vspace{-4mm}
\begin{align}
&\mathit{Shared}(B_3,Q_B) 
= 4 \times 11 \times 2 
+ 1 \times 2 \times 8 \times 2
= 120
\nonumber\\
&\mathit{NonShared}(\{B_4,B_5\},Q_B) 
= 2 \times 4 \times 11 = 88
\nonumber\\
&\mathit{Benefit}(B_3,Q_B)
= 88 - 120 = -32 < 0
\label{eq:decision-not-to-share}
\end{align}

Thus, the optimizer decides to split the shared graph\-let $B_3$ into two non-shared graphlets $B_4$ and $B_5$ for the queries $q_1$ and $q_2$ respectively in Figure~\ref{fig:decision-not-to-share-non-shared}. Newly arriving events of type $B$ then must be inserted into both graphlets $B_4$ and $B_5$. Their intermediate trend counts are computed separately for the queries $q_1$ and $q_2$. The snapshot $x$ is replaced by its value for the query $q_1$ ($q_2$) within the graphlet $B_4$ ($B_5$). The graphlets $A_1$ and $C_2$ are collapsed.

%--------------------------------------------------------------------------------
\textbf{Decision to Merge}.
%
When the next burst of events of type $B$ arrives, we could either continue the non-shared trend count propagation within $B_4$ and $B_5$ (Figure~\ref{fig:decision-to-share-non-shared}) or merge $B_4$ and $B_5$ into a new shared graphlet $B_6$ (Figure~\ref{fig:decision-to-share-shared}). The \app\ optimizer concludes that the latter option is more beneficial in Equation~\ref{eq:decision-to-share}. As a consequence, a new snapshot $z$ is created as input to $B_6$. $z$ consolidates the intermediate trend counts of the snapshot $x$ and the graphlets $B_3$--$B_5$ per query $q_1$ and $q_2$.

\vspace{-4mm}
\begin{align}
&\mathit{Shared}(B_6,Q_B) 
= 4 \times 15 \times 1 
+ 1 \times 2 \times 4 \times 2 
= 76
\nonumber\\
&\mathit{NonShared}(\{B_4,B_5\},Q_B) 
= 2 \times 4 \times 15 = 120 
\nonumber\\
&\mathit{Benefit}(B_6,Q_B)
= 120 - 76 = 44 > 0
\label{eq:decision-to-share}
\end{align}

% The algorithm that makes sharing decisions, splits and merges graphlets is straightforward. We omit it here due to space constraints.


%--------------------------------------------------------------------------------
\textbf{Complexity Analysis}.
%
The runtime sharing decision per burst has constant time complexity because it simply plugs in locally available stream statistics into Equation~\ref{eq:dynamic-benefit}.
%
A graphlet split comes for free since we simply continue graph construction per query (Figure~\ref{fig:decision-not-to-share-non-shared}).
%
Merging graphlets requires creation of one snapshot and calculation of its values per query (Figure~\ref{fig:decision-to-share-shared}). Thus, the time complexity of merging is $O(k \times g \times t)$ (Equation~\ref{eq:shared-cost}).
Since our workload is fixed (Section~\ref{sec:basic}), the number of queries $k$ and the number of types $t$ per query are constants. Thus, the time complexity of merge is linear in the number of events per graphlet $g$.
Merging graphlets  requires storing the value of one snapshot per query. Thus, its space complexity is $O(k)$.

%\lei{Doesn't duplicates of split graphlets introduce space overhead?}
% olga: no. I rephrased

%%%%%%%%%%%%%%%%%%%%%%%%%%%%%%%%%%%%%%%%%%%%%%%%%%%%%%%%%%%
%%%%%%%%%%%%%%%%%%%%%%%%%%%%%%%%%%%%%%%%%%%%%%%%%%%%%%%%%%%
\subsection{Choice of Query Set}
\label{sec:queries}

To relax the assumption from Section~\ref{sec:decisions} that a set of queries $Q_E$ that share a Kleene sub-pattern $E+$ is given, we now select a sub-set of queries $Q_E$ from the workload $Q$ for which sharing $E+$ is the most beneficial among all other sub-sets of $Q$.
%
% Search space
In general, the search space of all sub-sets of $Q$ is exponential in the number of queries in $Q$ since all combinations of shared and non-shared queries in $Q$ are considered. 
For example, if $Q$ contains four queries, Figure~\ref{fig:search-space} illustrates the search space of 12 possible execution plans of $Q$. 
Groups of queries in braces are shared. For example, the plan (134)(2) denotes that queries 1, 3, 4 share their execution, while query 2 is processed separately.
The search space ranges from maximally shared (top node) to non-shared (bottom node) plans. 
Each plan has its execution cost associated with it. For example, the cost of the plan (134)(2) is computed as the sum of $Shared(G_E,\{1,3,4\})$ and $NonShared(G_E^i,2)$ (Equation~\ref{eq:dynamic-benefit}). 
The goal of the dynamic \app\ optimizer is to find a plan with minimal execution cost.

%\vspace*{-1mm}
\begin{figure}[!htb]
\centering
\includegraphics[width=0.6\columnwidth]{figures/search-space.png}
\caption{Search space of sharing plans}
\label{fig:search-space}
\end{figure}
%\vspace*{-2mm}

% 2 pruning principles
Traversing the exponential search space for each Kleene sub-pattern and each burst of events would jeopardize real-time responsiveness of \app. Fortunately, most plans in this search space can be pruned without loosing optimality  (Theorems~\ref{theo:pruning-1} and \ref{theo:pruning-2}). 
Intuitively, Theorem~\ref{theo:pruning-1} states that it is always beneficial to share the execution of a query that introduces no new snapshots.

%-----------------------
\begin{theorem}
%
Let $E+$ be a Kleene sub-pattern that is shared by a set of queries $Q_E$ and not shared by a set of queries $Q_N$,
$Q_E \cap Q_N = \emptyset$,
$k_s = |Q_E|$, and
$k_n = |Q_N|$.
For a burst of events of type $E$,
let $q \in Q_E$ be a query that does not introduce new snapshots due to predicates for this burst of events (Definition~\ref{def:snapshot2}). 
Then the following follows:
%
\begin{align}
& \mathit{Shared}(Q_E) + \mathit{NonShared}(Q_N) \leq \nonumber\\
& \mathit{Shared}(Q_E \setminus \{q\}) + \mathit{NonShared}(Q_N \cup \{q\}) \nonumber
\end{align}
%
\label{theo:pruning-1}
\end{theorem}
\vspace{-5mm}

\begin{proof}
%
Equation~\ref{eq:one} summarizes the cost of sharing the execution of queries $Q_E$ where $q \in Q_E$. 
%
\begin{align}
&\mathit{Shared}(Q_E) + \mathit{NonShared}(Q_N) \nonumber\\
&= k_s \times \underline{s_c \times g \times p} \nonumber\\
&+ b \times (\log_2(g) + n \times s_p) \nonumber\\
&+ k_n \times b \times (\log_2(g) + n)
\label{eq:one}
\end{align}

Now assume the execution of $q$ is not shared with other queries in $Q_E$. That is, $q$ is removed from set $Q_E$ and added to set $Q_N$. Then, $k_s$ is decremented by one and $k_n$ is incremented by one in Equation~\ref{eq:two}. All other cost factors remain unchanged. In particular, the number of created $s_c$ and propagated $s_p$ snapshots do not change.

\begin{align}
&\mathit{Shared}(Q_E \setminus \{q\}) + \mathit{NonShared}(Q_N \cup \{q\}) \nonumber\\
&= (k_s-1) \times s_c \times g \times p \nonumber\\
&+ b \times (\log_2(g) + n \times s_p) \nonumber\\
&+ (k_n+1) \times \underline{b \times (\log_2(g) + n)} 
\label{eq:two}
\end{align}

Equations~\ref{eq:one} and \ref{eq:two} differ by one additive factor $(s_c \times g \times p)$ if $q$ is shared versus one additive factor $(b \times (\log_2(g) + n))$ if $q$ is not shared. These additive factors are underlined in Equations~\ref{eq:one} and \ref{eq:two}. Since $s_c \leq b$, $g \leq n$, and the number of predecessor types $p$ per type per query is negligible compared to other cost factors, we conclude that $(s_c \times g \times p) \leq (b \times (\log_2(g) + n))$, i.e., it is beneficial to share the execution of $q$ with other queries in $Q_E$.
%
\end{proof}

We formulate the following pruning principle per Theorem~\ref{theo:pruning-1}.

%%%%%%%%%%%%%%%%%%%%%
\textbf{\textit{Snapshot-Driven Pruning Principle}}. 
Plans at Level 2 of the search space that do not share queries that introduced no snapshots are pruned. All descendants of such plans are also pruned.

\begin{example}
%
In Figure~\ref{fig:search-space}, assume queries 1 and 3 introduced no snapshots, while queries 2 and 4 introduced snapshots.
Then, four plans are considered because they share queries 1 and 3 with other queries. These plans are highlighted by frames. The other eight plans are pruned since they are guaranteed to have higher execution costs.
%
\end{example}

Theorem~\ref{theo:pruning-2} below states that 
if it is beneficial to share the execution of a query $q$ with other queries $Q$, a plan that processes the query $q$ separately from other queries $Q_E \subseteq Q$ will have higher execution costs than a plan that shares $q$ with $Q_E$.
The reverse of the statement also holds. Namely, if it is not beneficial to share the execution of a query $q$ with other queries $Q$, a plan that shares the execution of $q$ with other queries $Q_E \subseteq Q$ will have higher execution costs than a plan that processes $q$ separately from $Q_E$.

% chuan - Theorem 2 is not clear to me, esp. Q_E = Q_E' \cup Q.
% olga - i changed the notation

%------------------------
\begin{theorem}
%
Let $E+$ be a Kleene sub-pattern that is shareable by a set of queries $Q$, 
$Q = Q_E \cup Q_N$, and 
$q \in Q_E$. Then:
%
\begin{align}
\text{If } 
& \mathit{Shared}(Q) \leq \mathit{Shared}(Q \setminus \{q\}) + \mathit{NonShared}(q) \text{,} 
\label{eq:three}\\
\text{then } 
& \mathit{Shared}(Q_E) + \mathit{NonShared}(Q_N) \leq
\mathit{Shared}(Q_E \setminus \{q\}) +
\mathit{NonShared}(Q_N \cup \{q\}) 
\label{eq:four}
\end{align}

This statement also holds if we replace all $\leq$ by $\geq$.
%
\label{theo:pruning-2}
\end{theorem}

\begin{proof}[Proof Sketch]
%
In Equation~\ref{eq:three}, if we do not share the execution of query $q$ with queries $Q$ and the execution costs increase, this means that the cost for re-computing $q$ is higher than the cost of maintenance of snapshots introduced by $q$ due to predicates.
Similarly in Equation~\ref{eq:four}, if we move the query $q$ from the set of queries $Q_E$ that share their execution to the set of queries $Q_N$ that are processed separately, the overhead of recomputing $q$ will dominate the overhead of snapshot maintenance due to $q$.
The reverse of Theorem~\ref{theo:pruning-2} can be proven analogously. 
%
\end{proof}

We formulate the following pruning principle per Theorem~\ref{theo:pruning-2}.

%%%%%%%%%%%%%%%%%%%%
\textbf{\textit{Benefit-Driven Pruning Principle}}.
Plans at Level 2 of the search space that do not share a query that is beneficial to share are pruned.
Plans at Level 2 of the search space that share a query that is not beneficial to share are pruned.
All descendants of such plans are also pruned.

\begin{example}
%
In Figure~\ref{fig:search-space}, if it is beneficial to share query 2, then we can safely prune all plans that process query 2 separately. That is, the plan (134)(2) and all its descendants are pruned.
Similarly, if it is not beneficial to share query 4, we can safely exclude all plans that share query 4. That is, all siblings of (123)(4) and their descendants are pruned.
The plan (123)(4) is chosen (highlighted by a bold frame).
%
\label{ex:search_space}
\end{example}

\textbf{\textit{Consequence of Pruning Principles}}.
Based on all plans at Levels 1 and 2 of the search space, the optimizer classifies each query in the workload as either shared or non-shared. Thus, it chooses the optimal plan without considering plans at levels below 2.

%------------------------
\textbf{Complexity Analysis}.
%
Given a burst of new events, let $m$ be the number of queries that introduce new snapshots to share the processing of this burst of events. The number of plans at Levels 1 and 2 of the search space is $m+1$. 
Thus both time and space complexity of sharing plan selection is $O(m)$.

% The sharing plan selection algorithm is straightforward. It is skipped here in the interest of space.

% chuan - i am not sure if you want to skip the algorithm here. also it is not clear to me why we want to stop at level 2? is it possible that the queries in Q should be executed separately from each other?
% olga - i changed the description

%\footnote{A stream is bursty if many of events of the same type arrive together and they are not interleaved by events of other types (Definition~\ref{def:burst}).}

%------------------------------------------------------------------
\begin{theorem}
%
Within one burst, \app\ has optimal time complexity.
%
\label{theo:optimality}
\end{theorem}

\begin{proof}
%
For a given burst of events, \app\ optimizer makes a decision to share or not to share depending on the sharing benefit in Section~\ref{sec:decisions}. 
If it is not beneficial to share, each query is processed separately and has optimal time complexity~\cite{PLRM18}.
If it is beneficial to share a pattern $E+$ by a set of queries $Q_E$, the time complexity is also optimal since it is optimal for one query $q \in Q_E$~\cite{PLRM18} and other queries in $Q_E$ are processed for free.
The set of queries $Q_E$ is chosen such that the benefit of sharing is maximal (Theorems~\ref{theo:pruning-1} and \ref{theo:pruning-2}).
%
\end{proof}


%--------------------------------------------------------------------------------
\textbf{Granularity of \app\ Sharing Decision}.
%
\app\ runtime sharing decisions are made per burst of events (Section~\ref{sec:decisions}). There can be several bursts per window (Definition~\ref{def:burst}). Within one burst, \app\ has optimal time complexity (Theorem~\ref{theo:optimality}).
According to the complexity analysis in Section~\ref{sec:decisions}, the choice of the query set has linear time complexity in the number of queries $m$ that introduce snapshots due to predicates. 
By Section~\ref{sec:queries}, the merge of graphlets has linear time complexity in the number of events $g$ per graphlet. 
\app\ would be optimal per window if it could make sharing decisions at the end of each window. However, waiting until all events per window arrive could introduce delays and jeopardise real-time responsiveness. Due to this low latency constraint, \app\ makes sharing decisions per burst, achieving significant performance gain over competitors (Section~\ref{sec:exp_results}).


%\footnote{A stream is bursty if many of events of the same type arrive together and they are not interleaved by events of other types (Definition~\ref{def:burst}).}



\section{Additional figures and tables}


\begin{table}[ht]
\caption{Atom-type-specific atomic partial charge comparisons for the two levels of theory (GFN2-xTB, $\omega$B97X-D/def2-SVP) for the QMugs database. Abbreviations: RMSE, root mean squared error; PCC, Pearson's correlation coefficient.}
\centering
\begin{tabular}{@{}llll@{}}
\toprule
\textbf{Atom type} & \textbf{Occurrence} & \textbf{RMSE}          & \textbf{PCC} \\ \midrule
Hydrogen           & 49.0M  & $1.00 \times 10^{-3}$   & 0.913        \\
Carbon         & 45.1M  & 0.0130                   & 0.575          \\
Nitrogen         & 7.28M  & 0.0264                  & 0.124        \\
Oxygen           & 6.08M  & 0.0188                  & 0.274        \\
Fluorine         & 1.14M  & $3.21 \times 10^{-4}$   & 0.868        \\
Sulfur           & 729k  & 0.043                   & 0.991        \\
Chlorine         & 513k  & $6.39 \times 10^{-3}$   & 0.862        \\
Bromine          & 82.5k & 0.0153                   & 0.831        \\
Phosphorus       & 32.5k  & 0.139                   & 0.872        \\
Iodine           & 13.2k  & 0.0353                   & 0.808        \\ \bottomrule
\end{tabular}%
\end{table}



\begin{table}[ht]
\caption{Wiberg bond order comparisons  for the two levels of theory (GFN2-xTB, $\omega$B97X-D/def2-SVP) and the 15 most frequent pair-wise atomic covalent bonds in QMugs. Abbreviations: RMSE, root mean squared error; PCC, Pearson's correlation coefficient.}
\centering
\begin{tabular}{@{}llll@{}}
\toprule
\textbf{Bond type} & \textbf{Occurrence} & \textbf{RMSE}     & \textbf{PCC} \\ \midrule
Carbon-Hydrogen      & 44.8M  & $1.30 \times 10^{-3}$    & 0.787         \\
Carbon-Carbon        & 40.6M  & $6.83\times 10^{-4}$ & 0.998        \\
Carbon-Nitrogen      & 14.4M  & 0.0137             & 0.995        \\
Nitrogen-Hydrogen    & 3.20M  & $2.86 \times 10^{-3}$    & 0.860        \\
Carbon-Fluorine      & 1.14M  & 0.108             & 0.153        \\
Carbon-Sulfur        & 1.13M  & 0.0223             & 0.977        \\
Oxygen-Hydrogen      & 985k  & 0.0520             & 0.904        \\
Carbon-Oxygen        & 683k  & 0.0972             & 0.998        \\
Sulfur-Oxygen        & 618k  & 0.0853             & 0.941         \\ 
Nitrogen-Nitrogen    & 538k  & 0.0278             & 0.997         \\
Carbon-Chlorine      & 513k  & 0.0684             & 0.846         \\
Nitrogen-Sulfur      & 249k  & 0.0139             & 0.964        \\
Nitrogen-Oxygen      & 234k  & 0.110             & 0.997        \\
Phosphorus-Oxygen    & 107k  & 0.0112             & 0.980         \\
Carbon-Bromine       & 82.4k  & 0.0384             & 0.840        \\\bottomrule
\end{tabular}%
\end{table}
\section{Evaluation}
\label{sec:evaluation}
\begin{table*}[!t]
\begin{center}
%\small
\caption {Benchmarks and applications for the study of the application-level resilience}
\vspace{-5pt}
\label{tab:benchmark}
\tiny
\begin{tabular}{|p{1.7cm}|p{7.5cm}|p{4cm}|p{2.5cm}|}
\hline
\textbf{Name} 	& \textbf{Benchmark description} 		& \textbf{Execution phase for evaluation}  			& \textbf{Target data objects}             \\ \hline \hline
CG (NPB)             & Conjugate Gradient, irregular memory access (input class S)   & The routine conj\_grad in the main computation loop  & The arrays $r$ and $colidx$     \\\hline
MG (NPB)    	       & Multi-Grid on a sequence of meshes (input class S)             & The routine mg3P in the main computation loop & The arrays $u$ and $r$ 	\\ \hline
FT (NPB)             & Discrete 3D fast Fourier Transform (input class S)            & The routine fftXYZ in the main computation loop  & The arrays $plane$ and $exp1$    \\ \hline
BT (NPB)             & Block Tri-diagonal solver (input class S)         		& The routine x\_solve in the main computation loop & The arrays $grid\_points$ and $u$	\\ \hline
SP (NPB)             & Scalar Penta-diagonal solver (input class S)         		& The routine x\_solve in the main computation loop & The arrays $rhoi$ and $grid\_points$  \\ \hline
LU (NPB)            & Lower-Upper Gauss-Seidel solver (input class S)        	& The routine ssor 	& The arrays $u$ and $rsd$ \\ \hline \hline
LULESH~\cite{IPDPS13:LULESH} & Unstructured Lagrangian explicit shock hydrodynamics (input 5x5x5) & 
The routine CalcMonotonicQRegionForElems 
& The arrays $m\_elemBC$ and $m\_delv\_zeta$ \\ \hline
AMG2013~\cite{anm02:amg} & An algebraic multigrid solver for linear systems arising from problems on unstructured grids (we use  GMRES(10) with AMG preconditioner). We use a compact version from LLNL with input matrix $aniso$. & The routine hypre\_GMRESSolve & The arrays $ipiv$ and $A$   \\ \hline
%$hierarchy.levels[0].R.V$ \\ \hline
\end{tabular}
\end{center}
\vspace{-5pt}
\end{table*}

%We evaluate the effectiveness of ARAT, and 
%We use ARAT to study the application-level resilience.
%The goal is to demonstrate 
%that aDVF can be a very useful metric to quantify the resilience of data objects
%at the application level. 
We study 12 data objects from six benchmarks of the NAS parallel benchmark (NPB) suite (we use SNU\_NPB-1.0.3) and 4 data objects from two scientific applications. 
%which is a c version of NPB 3.3, but ARAT can work for Fortran.
Those data objects are chosen to be representative: they have various data access patterns and participate in various execution phases.  
%For the benchmarks, we use CLASS S as the input problems and use the default compiler options of NPB.
For those benchmarks and applications, we use their default compiler options, and use gcc 4.7.3 and LLVM 3.4.2 for trace generation.
To count the algorithm-level fault masking, we use the default convergence thresholds (or the fault tolerance levels) for those benchmarks.
Table~\ref{tab:benchmark} gives 
%for->on by anzheng
detailed information on the benchmarks and applications.
The maximum fault propagation path for aDVF analysis is set to 10 by default.
%the value shadowing threshold is set as 0.01 (except for BT, we use $1 \times 10^{-6}$).
%These value shadowing thresholds are chosen such that any error corruption
%that results in the operand's value variance less than 1\% (for the threshold 0.01) or 0.0001\% (for the threshold $1 \times 10^{-6}$) during the 
%trace analysis does not impact the outcome correctness of six benchmarks.
%LU: check the newton-iteration residuals against the tolerance levels
%SP: check the newton-iteration residuals against the tolerance levels
%BT: check the newton-iteration residuals against the tolerance levels

\subsection{Resilience Modeling Results}
%We use ARAT to calculate aDVF values of 16 data objects. 
Figure~\ref{fig:aDVF_3tiers_profiling}
shows the aDVF results and breaks them down into the three levels 
(i.e., the operation-level, fault propagation level, and algorithm-level).
Figure~\ref{fig:aDVF_3classes_profiling} shows the 
%for->of by anzheng
results for the analyses at the levels of the operation and fault propagation,
and further breaks down the results into 
the three classes (i.e., the value overwriting, logical and comparison operations,
and value shadowing). %based on the reasons of the fault masking.
We have multiple interesting findings from the results.

\begin{figure*}
	\centering
        \includegraphics[width=0.8\textwidth]{three_tiers_gray.pdf}
% * <azguolu@gmail.com> 2017-03-23T03:20:28.808Z:
%
% ^.
        \vspace{-5pt}
        \caption{The breakdown of aDVF results based on the three level analysis. The $x$ axis is the data object name.}
        \vspace{-8pt}
        \label{fig:aDVF_3tiers_profiling}
\end{figure*}


\begin{figure*}
	\centering
	\includegraphics[width=0.8\textwidth]{three_types_gray.pdf}
	\vspace{-5pt}
	\caption{The breakdown of aDVF results based on the three classes of fault masking. The $x$ axis is the data object name. \textit{zeta} and \textit{elemBC} in LULESH are \textit{m\_delv\_zeta} and \textit{m\_elemBC} respectively.} % Anzheng
	\vspace{-5pt}
	\label{fig:aDVF_3classes_profiling}
    %\vspace{-5pt}
\end{figure*}

(1) Fault masking is common across benchmarks and applications.
Several data objects (e.g., $r$ in CG, and $exp1$ and $plane$ in FT)
have aDVF values close to 1 in Figure~\ref{fig:aDVF_3tiers_profiling}, 
which indicates that most of operations working on these data objects
have fault masking.
However, a couple of data objects have much less intensive fault masking.
For example, the aDVF value of $colidx$ in CG is 0.28 (Figure~\ref{fig:aDVF_3tiers_profiling}). 
Further study reveals that $colidx$ is an array to store column indexes of sparse matrices, and there is few operation-level or fault propagation-level fault masking  (Figure~\ref{fig:aDVF_3classes_profiling}).
The corruption of it can easily cause segmentation fault caught by the
algorithm-level analysis. 
$grid\_points$ in SP and BT also have a relatively small aDVF value (0.14 and 0.38 for SP and BT respectively in Figure~\ref{fig:aDVF_3tiers_profiling}).
Further study reveals that $grid\_points$ defines input problems for SP and BT. 
A small corruption of $grid\_points$ 
%change->changes by anzheng
can easily cause major changes in computation
caught by the fault propagation analysis. 

The data object $u$ in BT also has a relatively small aDVF value (0.82 in Figure~\ref{fig:aDVF_3tiers_profiling}).
Further study reveals that $u$ is read-only in our target code region
for matrix factorization and Jacobian, neither of which is friendly
for fault masking.
Furthermore, the major fault masking for $u$ comes from value shadowing,
and value shadowing only happens in a couple of the least significant bits 
of the operands that reference $u$, which further reduces the value of aDVF.
%also reduces fault masking.

(2) The data type is strongly correlated with fault masking.
Figure~\ref{fig:aDVF_3tiers_profiling} reveals that the integer data objects ($colidx$ in CG, $grid\_points$ in BT and SP, $m\_elemBC$ in LULESH) appear to be 
more sensitive to faults than the floating point data objects 
($u$ and $r$ in MG, $exp1$ and $plane$ in FT, $u$ and $rsd$ in LU, $m\_delv\_zeta$ in LULESH, and $rhoi$ in SP).
In HPC applications, the integer data objects are commonly employed to
define input problems and bound computation boundaries (e.g., $colidx$ in CG and $grid\_points$ in BT), 
or track computation status (e.g., $m\_elemBC$ in LULESH). Their corruption 
%these integer data objects
is very detrimental to the application correctness. 

(3) Operation-level fault masking is very common.
For many data objects, the operation-level fault masking contributes 
more than 70\% of the aDVF values. For $r$ in CG, $exp1$ in FT, and $rhoi$ in SP,
the contribution of the operation-level fault masking is close to 99\% (Figure~\ref{fig:aDVF_3tiers_profiling}).

Furthermore, the value shadowing is a very common operation level fault masking,
especially for floating point data objects (e.g., $u$ and $r$ in BT, $m\_delv\_zeta$ in LULESH, and $rhoi$ in SP in Figure~\ref{fig:aDVF_3classes_profiling}).
This finding has a very important indication for studying the application resilience.
In particular, the values of a data object can be different across different input problems. If the values of the data object are different, 
then the number of fault masking events due to the value shadowing will be different. 
Hence, we deduce that the application resilience
can be correlated with the input problems,
because of the correlation between the value shadowing and input problems. 
We must consider the input problems when studying the application resilience.
This conclusion is consistent with a very recent work~\cite{sc16:guo}.

(4) The contribution of the algorithm-level fault masking to the application resilience can be nontrivial.
For example, the algorithm-level fault masking contributes 19\% of the aDVF value for $u$ in MG and 27\% for $plane$ in FT (Figure~\ref{fig:aDVF_3tiers_profiling}).
The large contribution of algorithm-level fault masking in MG is consistent with
the results of existing work~\cite{mg_ics12}. 
For FT (particularly 3D FFT), the large contribution of algorithm-level fault masking in $plane$ (Figure~\ref{fig:aDVF_3tiers_profiling})
comes from frequent transpose and 1D FFT computations that average out 
or overwrite the data corruption.
CG, as an iterative solver, is known to have the algorithm-level fault masking
because of the iterative nature~\cite{2-shantharam2011characterizing}.
Interestingly, the algorithm-level fault masking in CG contributes most to the resilience of $colidx$ which is a vulnerable integer data object (Figure~\ref{fig:aDVF_3tiers_profiling}).

%Our study reveals the algorithm-level fault masking of CG from
%two perspectives. First, $a$ in CG, which is an array for intermediate results,
%has few algorithm-level fault masking (0.008\%);
%Second, $x$ in CG, which is a result vector, has 5.4\% of the aDVF value coming from the algorithm-level fault masking.
%This result indicates that the effects of the algorithm-level fault masking
%are not uniform across data objects. 

(5) Fault masking at the fault propagation level is small.
For all data objects, the contribution of the fault masking at the level of fault propagation is less than 5\% (Figure~\ref{fig:aDVF_3tiers_profiling}).
For 6 data objects ($r$ and $colidx$ in CG, $grid\_points$ and $u$ in BT, and 
$grid\_points$ and $rhoi$ in SP),  there is no fault masking at the level of fault propagation.
In combination with the finding 4, we conclude that once the fault
is propagated, it is difficult to mask it because of the contamination of
more data objects after fault propagation, and only the algorithm semantics can tolerate  propagated faults well. 
%This finding is consistent with our sensitivity analysis. 

(6) Fault masking by logical and comparison operations is small,
%For all data objects, the fault masking contributions due to logical and comparison operations are very small, 
comparing with the contributions of value shadowing and overwriting (Figure~\ref{fig:aDVF_3classes_profiling}). 
Among all data objects, 
the logical and comparison operations in $grid\_points$ in BT contribute the most (25\% contribution in Figure~\ref{fig:aDVF_fine_profiling}), 
because of intensive ICmp operations (integer comparison). %logical OR and SHL (left shifting).


(7) The resilience varies across data objects. %within the same application.
This fact is especially pronounced in two data objects $colidx$ and $r$ in CG (Figure~\ref{fig:aDVF_3tiers_profiling}).
 $colidx$ has aDVF much smaller than $r$, which means $colidx$ is much less resilient than $r$ (see finding 1 for a detailed analysis on $colidx$). 
Furthermore, $colidx$ and $r$ have different algorithm-level
fault masking (see finding 4 for a detailed analysis).

\begin{comment}
\textbf{Finding 7: The resilience of the same data objects varies across different applications.}
This fact is especially pronounced in BT and SP.
BT and SP address the same numerical problem but with different algorithms.
BT and SP have the same data objects, $qs$ and $rhoi$, but
$qs$ manifests different resilience in BT and SP.
This result is interesting, because it indicates that by using
different algorithms, we have opportunities to
improve the resilience of data objects.
\end{comment}

To further investigate the reasons for fault masking, 
we break down the aDVF results at the granularity of LLVM instructions,
based on the analyses at the levels of operation and fault propagation.
The results are shown in Figure~\ref{fig:aDVF_fine_profiling}.
%Because of the space limitation, 
%we only show one data object per benchmark, but each selected data object has the most diverse fault masking events within the corresponding benchmark.
%Based on Figure~\ref{fig:aDVF_fine_profiling}, we have another interesting finding.

(8) Arithmetic operations make a lot of contributions to fault masking.
%For $r$ in CG, $r$ in MG, $exp1$ in FT, $u$ in BT, $qs$ in SP, and $u$ in LU,
%the arithmetic operations, FMul (100\%), Add (16\%), FMul (85\%), 
%FMul (94\%), FMul (28\%), and FAdd (50\%)
For $r$ in CG, $u$ in BT, $plane$ and $exp1$ in FT, $m\_elemBC$ in LULESH, 
arithmetic operations (addition, multiplication, and division) contribute to almost 100\% of the fault masking (Figure~\ref{fig:aDVF_fine_profiling}).  
%(at the operation level and the fault propagation level).
%For $qs$ in SP and $u$ in LU, the store operation also makes
%important contributions as the arithmetic operations because of value overwriting.

\begin{figure*}
	\centering
	\includegraphics[width=0.77\textheight, height=0.23\textheight]{pie_chart.pdf}
	\vspace{-10pt}
	\caption{Breakdown of the aDVF results based on the analyses at the levels of operation and fault propagation}
    \vspace{-10pt}
	\label{fig:aDVF_fine_profiling}
\end{figure*}


\subsection{Sensitivity Study}
\label{sec:eval_sen}
%\textbf{change the fault propagation threshold and study the sensitivity of analysis to the threshold}
ARAT uses 10 as the default fault propagation analysis threshold. 
The fault propagation analysis will not go beyond 10 operations. Instead,
we will use deterministic fault injection after 10 operations. 
In this section, we study the impact of this threshold on the modeling accuracy. We use a range of threshold values and examine how the aDVF value varies and whether
the identification of fault masking varies. 
Figure~\ref{fig:sensitivity_error_propagation} shows the results for 
%add , after BT by anzheng
multiple data objects in CG, BT, and SP.
We perform the sensitivity study for all 16 data objects.
%in six benchmarks and two applications.
Due to the page space limitation, we only show the results for three data objects,
but we summarize the sensitivity study results for all data objects in this section.
%but other data objects in all benchmarks have the same trend.

Our results reveal that the identification of fault masking by tracking fault propagation is not significantly 
affected by the fault propagation analysis threshold. Even if we use a rather large threshold (50), 
the variation of aDVF values is 4.48\% on average among all data objects,
and the variation at each of the three levels of analysis (the operation level, fault propagation level,  and algorithm level) is less than 5.2\% on average. 
In fact, using a threshold value of 5 is sufficiently accurate in most of the cases (14 out of 16 data objects).
This result is consistent with our finding 5 (i.e., fault masking at the fault propagation level is small). %in most benchmarks).
However, we do find a data object ($m\_elementBC$ in LULESH) %and $exp1$ in FT) 
showing relatively high-sensitive (up to 15\% variation) to the threshold. For this uncommon data object, using 50 as the fault propagation path is sufficient. 

%In other words, even though using a larger threshold value can identify more error masking by tracking error 
%propagation, the implicit error masking induced by the error propagation is very limited.

\begin{figure}
		\begin{center}
		\includegraphics[width=0.48\textwidth,height=0.11\textheight]{sensi_study_gray.pdf}
		\vspace{-15pt}
		\caption{Sensitivity study for fault propagation threshold}
		\label{fig:sensitivity_error_propagation}
		\end{center}
\vspace{-15pt}
\end{figure}


\begin{comment}
\subsection{Comparison with the Traditional Random Fault Injection}
%\textbf{compare with the traditional fault injection to verify accuracy}
To show the effectiveness of our resilience modeling, we compare traditional random fault injection
and our analytical modeling. Figure~\ref{fig:comparison_fi} and Table~\ref{tab:comparison} show the results.
The figure shows the success rate of all random fault injection. The ``success'' means the application
outcome is verified successfully by the benchmarks and the execution does not have any segfault. The success rate is used as a metric
to evaluate the application resilience.

We use a data-oriented approach to perform random fault injection.
In particular, given a data object, for each fault injection test we trigger a bit flip
in an operand of a random instruction, and this operand must be a reference to the
target data object. We develop a tool based on PIN~\cite{pintool} to implement the above fault injection functionality.
For each data object, we conduct five sets of random fault injection tests, 
and each set has 200 tests (in total 1000 tests per data object). 
We show the results for CG and FT in this section, but we find that
the conclusions we draw from CG and FT are also valid for the other four benchmarks.


%\begin{table*}
%\label{tab:success_rate}
%\begin{centering}
%\renewcommand\arraystretch{1.1}
%\begin{tabular}{|c|c|c|c|c|c|c|}
%\hline 
%Success Rate (Difference) & Test set 1 & Test set 2 & Test set 3 & Test set 4 & Test set 5 & Average\tabularnewline
%\hline 
%\hline 
%CG-a & 66.1\% (11.7\%) & 68.5\% (15.7\%) & 56.7\% (4.21\%) & 61.3\% (3.57\%) & 43.3\% (26.8\%) & 59.2\%\tabularnewline
%\hline 
%CG-x & 99.2\% (2.2\%) & 98.6\% (1.5\%) & 96.5\% (0.63\%) & 97.8\% (0.64\%) & 93.6\% (3.7\%) & 97.1\%\tabularnewline
%\hline 
%CG-colidx & 36.8\% (12.7\%) & 49.6\% (17.8\%) & 40.2\% (4.6\%) & 52.6\% (24.9\%) & 31.4\% (25.4\%) & 42.1\%\tabularnewline
%\hline 
%FT-exp1 & 52.7\% (1.4\%) & 22.6\% (56.5\%) & 78.5\% (51.0\%) & 60.7\% (16.7\%) & 45.4\% (12.7\%) & 51.9\%\tabularnewline
%\hline 
%FT-plane & 82.1\% (2.5\%) & 79.3\% (5.6\%) & 99.5\% (18.2\%) & 93.2\% (10.7\%) & 66.8\% (20.6\%) & 84.2\%\tabularnewline
%\hline 
%\end{tabular}
%\par\end{centering}
%\caption{XXXXX}
%\end{table*}


\begin{table*}
\begin{centering}
\caption{\small The results for random fault injection. The numbers in parentheses for each set of tests (200 tests per set) are the success rate difference from the average success rate of 1000 fault injection tests.}
\label{tab:comparison}
\renewcommand\arraystretch{1.1}
\begin{tabular}{|c|p{2.2cm}|p{2.2cm}|p{2.2cm}|p{2.2cm}|p{2.2cm}|p{1.8cm}|}
\hline 
       %& Test set 1 & Test set 2 & Test set 3 & Test set 4 & Test set 5 & Average\tabularnewline
       & \hspace{13pt} Test set 1 \hspace{1pt}/  & \hspace{13pt} Test set 2 \hspace{1pt}/ & \hspace{13pt} Test set 3 \hspace{1pt}/ & \hspace{13pt} Test set 4 \hspace{1pt}/ & \hspace{13pt} Test set 5 \hspace{1pt}/ & Ave. of all test / \\
       & success rate (diff.) & success rate (diff.) & success rate (diff.) & success rate (diff.) & success rate (diff.) & \hspace{5pt} success rate \\
\hline 
\hline 
CG-a & 66.1\% (6.9\%) & 68.5\% (9.3\%) & 56.7\% (-2.5\%) & 61.3\% (2.1\%) & 43.3\% (-15.9\%) & 59.2\%\tabularnewline
\hline 
CG-x & 99.2\% (2.1\%) & 98.6\% (1.5\%) & 96.5\% (-0.6\%) & 97.8\% (0.7\%) & 93.6\% (-3.5\%) & 97.1\%\tabularnewline
\hline 
CG-colidx & 36.8\% (-5.3\%) & 49.6\% (7.5\%) & 40.2\% (-2.0\%) & 52.6\% (10.5\%) & 31.4\% (-10.7\%) & 42.1\%\tabularnewline
\hline 
FT-exp1 & 52.7\% (0.8\%) & 22.6\% (-29.3\%) & 78.5\% (26.6\%) & 60.7\% (8.8\%) & 45.4\% (-6.5\%) & 51.9\%\tabularnewline
\hline 
FT-plane & 82.1\% (-2.1\%) & 79.3\% (-4.9\%) & 99.5\% (15.3\%) & 93.2\% (9.0\%) & 66.8\% (-17.4\%) & 84.2\%\tabularnewline
\hline 
\end{tabular}
\par\end{centering}
\vspace{-0.4cm}
\end{table*}

\begin{figure}
	\begin{center}
		\includegraphics[width=0.48\textwidth,keepaspectratio]{verifi-study.png}
		\caption{The traditional random fault injection vs. ARAT}
		\label{fig:comparison_fi}
	\end{center}
\vspace{-0.7cm}
\end{figure}


We first notice from Table~\ref{tab:comparison} that 
%across 5 sets of random fault injection tests, there are big variances (up to 55.9\% in $exp1$ of FT) in terms of the success rate. 
the results of 5 test sets can be quite different from each other and from 1000 random fault inject tests (up to 29.3\%).
1000 fault injection tests provide better statistical significance than 200 fault injection tests.
We expect 1000 fault injection tests potentially provide higher accuracy to quantify the application resilience.
The above result difference is clearly an indication to the randomness of fault injection, and there
is no guarantee on the random fault injection accuracy.

%In Figure~\ref{fig:comparison_fi}, 
We compare the success rate of 1000 fault inject tests with the aDVF value (Fig.~\ref{fig:comparison_fi}). 
We find that the order of the success rate of the three data objects in CG (i.e., $colidx < a < x$) and the two data objects in FT 
(i.e., $exp1 < plane$) is the same as the order of the aDVF values of these data objects. 
%In fact, 1000 fault injection tests
%account for \textcolor{blue}{\textbf{xxx\%}} of total memory references to the data object,
%and provide better resilience quantification than 200 fault injection tests.
The same order (or the same resilience trend)
%between our approach and the random fault injection based on a large number of tests 
is a demonstration of the effectiveness of our approach.
Note that the values of the aDVF and success rate %for a data object
cannot be exactly the same (even if we have sufficiently large numbers of random fault injection), 
because aDVF and random fault injection quantify
the resilience based on different metrics.
Also, the random fault injection can miss some fault masking events that can be captured by our approach.

\end{comment}
\section{Related Work}
%\mz{We lack a comparison to this paper: https://arxiv.org/abs/2305.14877}
%\anirudh{refine to be more on-topic?}
\iffalse
\paragraph{In-Context Learning} As language models have scaled, the ability to learn in-context, without any weight updates, has emerged. \cite{brown2020language}. While other families of large language models have emerged, in-context learning remains ubiquitous \cite{llama, bloom, gptneo, opt}. Although such as HELM \cite{helm} have arisen for systematic evaluation of \emph{models}, there is no systematic framework to our knowledge for evaluating \emph{prompting methods}, and validating prompt engineering heuristics. The test-suite we propose will ensure that progress in the field of prompt-engineering is structured and objectively evaluated. 

\paragraph{Prompt Engineering Methods} Researchers are interested in the automatic design of high performing instructions for downstream tasks. Some focus on simple heuristics, such as selecting instructions that have the lowest perplexity \cite{lowperplexityprompts}. Other methods try to use large language models to induce an instruction when provided with a few input-output pairs \cite{ape}. Researchers have also used RL objectives to create discrete token sequences that can serve as instructions \cite{rlprompt}. Since the datasets and models used in these works have very little intersection, it is impossible to compare these methods objectively and glean insights. In our work, we evaluate these three methods on a diverse set of tasks and models, and analyze their relative performance. Additionally, we recognize that there are many other interesting angles of prompting that are not covered by instruction engineering \cite{weichain, react, selfconsistency}, but we leave these to future work.

\paragraph{Analysis of Prompting Methods} While most prompt engineering methods focus on accuracy, there are many other interesting dimensions of performance as well. For instance, researchers have found that for most tasks, the selection of demonstrations plays a large role in few-shot accuracy \cite{whatmakesgoodicexamples, selectionmachinetranslation, knnprompting}. Additionally, many researchers have found that even permuting the ordering of a fixed set of demonstrations has a significant effect on downstream accuracy \cite{fantasticallyorderedprompts}. Prompts that are sensitive to the permutation of demonstrations have been shown to also have lower accuracies \cite{relationsensitivityaccuracy}. Especially in low-resource domains, which includes the large public usage of in-context learning, these large swings in accuracy make prompting less dependable. In our test-suite we include sensitivity metrics that go beyond accuracy and allow us to find methods that are not only performant but reliable.

\paragraph{Existing Benchmarks} We recognize that other holistic in-context learning benchmarks exist. BigBench is a large benchmark of 204 tasks that are beyond the capabilities of current LLMs. BigBench seeks to evaluate the few-shot abilities of state of the art large language models, focusing on performance metrics such as accuracy \cite{bigbench}. Similarly, HELM is another benchmark for language model in-context learning ability. Rather than only focusing on performance, HELM branches out and considers many other metrics such as robustness and bias \cite{helm}. Both BigBench and HELM focus on ranking different language model, while fix a generic instruction and prompt format. We instead choose to evaluate instruction induction / selection methods over a fixed set of models. We are the first ever evaluation script that compares different prompt-engineering methods head to head. 
\fi

\paragraph{In-Context Learning and Existing Benchmarks} As language models have scaled, in-context learning has emerged as a popular paradigm and remains ubiquitous among several autoregressive LLM families \cite{brown2020language, llama, bloom, gptneo, opt}. Benchmarks like BigBench \cite{bigbench} and HELM \cite{helm} have been created for the holistic evaluation of these models. BigBench focuses on few-shot abilities of state-of-the-art large language models, while HELM extends to consider metrics like robustness and bias. However, these benchmarks focus on evaluating and ranking \emph{language models}, and do not address the systematic evaluation of \emph{prompting methods}. Although contemporary work by \citet{yang2023improving} also aims to perform a similar systematic analysis of prompting methods, they focus on simple probability-based prompt selection while we evaluate a broader range of methods including trivial instruction baselines, curated manually selected instructions, and sophisticated automated instruction selection.

\paragraph{Automated Prompt Engineering Methods} There has been interest in performing automated prompt-engineering for target downstream tasks within ICL. This has led to the exploration of various prompting methods, ranging from simple heuristics such as selecting instructions with the lowest perplexity \cite{lowperplexityprompts}, inducing instructions from large language models using a few annotated input-output pairs \cite{ape}, to utilizing RL objectives to create discrete token sequences as prompts \cite{rlprompt}. However, these works restrict their evaluation to small sets of models and tasks with little intersection, hindering their objective comparison. %\mz{For paragraphs that only have one work in the last line, try to shorten the paragraph to squeeze in context.}

\paragraph{Understanding in-context learning} There has been much recent work attempting to understand the mechanisms that drive in-context learning. Studies have found that the selection of demonstrations included in prompts significantly impacts few-shot accuracy across most tasks \cite{whatmakesgoodicexamples, selectionmachinetranslation, knnprompting}. Works like \cite{fantasticallyorderedprompts} also show that altering the ordering of a fixed set of demonstrations can affect downstream accuracy. Prompts sensitive to demonstration permutation often exhibit lower accuracies \cite{relationsensitivityaccuracy}, making them less reliable, particularly in low-resource domains.

Our work aims to bridge these gaps by systematically evaluating the efficacy of popular instruction selection approaches over a diverse set of tasks and models, facilitating objective comparison. We evaluate these methods not only on accuracy metrics, but also on sensitivity metrics to glean additional insights. We recognize that other facets of prompting not covered by instruction engineering exist \cite{weichain, react, selfconsistency}, and defer these explorations to future work. 
\section{Conclusions}
\label{sec:conclusions}

In this paper, we apply shared-workload techniques at the \sql level for
improving the throughput of \qaasl systems without incurring in additional
query execution costs. Our approach is based on query rewriting for grouping
multiple queries together into a single query to be executed in one go. This
results in a significant reduction of the aggregated data access done by the
shared execution compared to executing queries independently.

%execution times and costs of the shared scan operator when
%varying query selectivity and predicate evaluation. We observed that for
%\athena, although the cost only depends on the amount of data read, it is
%conditioned to its ability to use its statistics about the data. In some cases
%a wrong query execution plan leads to higher query execution costs, which the
%end-user has to pay. 

%\bigquery's minimum query execution cost is determined by
%the input size of a query.  However, the query cost can increase depending not
%just in the amount of computation it requires, but in the mix of resources the
%query requires.  

We presented a cost and runtime evaluation of the shared operator driving data access costs. 
Our experimental study using the TPC-H benchmark confirmed the benefits of our
query rewrite approach. Using a shared execution approach reduces significantly
the execution costs. For \athena, we are able to make it 107x cheaper and for
\bigquery, 16x cheaper taking into account Query 10 which we cannot execute,
but 128x if it is not taken into account. Moreover, when having queries that do
not share their entire execution plan, i.e., using a single global plan, we
demonstrated that it is possible to improve throughput and obtain a 10x cost
reduction in \bigquery.

%We followed the TPC systems pricing guideline for
%computing how expensive is to have a TPC-H workload working on the evaluated
%\qaasl systems. The result is that even though we are able to reduce overall
%costs a TPC-H workload in 15x for \bigquery (128x excluding query 10 which we
%could not optimize) and in 107x for \athena, the overall price is at least 10x
%more expensive than the cheapest system price published by the TPC.

There are multiple ways to extend our work. The first is
to implement a full \sql-to-\sql translation layer to encapsulate the proposed
per-operator rewrites.  Another one is to incorporate the initial work on
building a cost-based optimizer for shared execution
\cite{Giannikis:2014:SWO:2732279.2732280} as an external component for \qaasl
systems.  Moreover, incorporating different lines of work (e.g., adding
provenance computation \cite{GA09} capabilities) also based on query
rewriting is part of our future work to enhance our system.


\section*{Acknowledgments} 
This work was supported by 
NSF grants IIS-1815866, IIS-1018443, CRI-1305258, 
the U.S. Department of Agriculture grant 1023720,
and the U.S. Department of Education grant P200A150306.


\bibliographystyle{abbrv}
\bibliography{hamlet_tr}  

%\section*{Appendix}

%\begin{appendix}
%\chapter{Supplementary Material}
\label{appendix}

In this appendix, we present supplementary material for the techniques and
experiments presented in the main text.

\section{Baseline Results and Analysis for Informed Sampler}
\label{appendix:chap3}

Here, we give an in-depth
performance analysis of the various samplers and the effect of their
hyperparameters. We choose hyperparameters with the lowest PSRF value
after $10k$ iterations, for each sampler individually. If the
differences between PSRF are not significantly different among
multiple values, we choose the one that has the highest acceptance
rate.

\subsection{Experiment: Estimating Camera Extrinsics}
\label{appendix:chap3:room}

\subsubsection{Parameter Selection}
\paragraph{Metropolis Hastings (\MH)}

Figure~\ref{fig:exp1_MH} shows the median acceptance rates and PSRF
values corresponding to various proposal standard deviations of plain
\MH~sampling. Mixing gets better and the acceptance rate gets worse as
the standard deviation increases. The value $0.3$ is selected standard
deviation for this sampler.

\paragraph{Metropolis Hastings Within Gibbs (\MHWG)}

As mentioned in Section~\ref{sec:room}, the \MHWG~sampler with one-dimensional
updates did not converge for any value of proposal standard deviation.
This problem has high correlation of the camera parameters and is of
multi-modal nature, which this sampler has problems with.

\paragraph{Parallel Tempering (\PT)}

For \PT~sampling, we took the best performing \MH~sampler and used
different temperature chains to improve the mixing of the
sampler. Figure~\ref{fig:exp1_PT} shows the results corresponding to
different combination of temperature levels. The sampler with
temperature levels of $[1,3,27]$ performed best in terms of both
mixing and acceptance rate.

\paragraph{Effect of Mixture Coefficient in Informed Sampling (\MIXLMH)}

Figure~\ref{fig:exp1_alpha} shows the effect of mixture
coefficient ($\alpha$) on the informed sampling
\MIXLMH. Since there is no significant different in PSRF values for
$0 \le \alpha \le 0.7$, we chose $0.7$ due to its high acceptance
rate.


% \end{multicols}

\begin{figure}[h]
\centering
  \subfigure[MH]{%
    \includegraphics[width=.48\textwidth]{figures/supplementary/camPose_MH.pdf} \label{fig:exp1_MH}
  }
  \subfigure[PT]{%
    \includegraphics[width=.48\textwidth]{figures/supplementary/camPose_PT.pdf} \label{fig:exp1_PT}
  }
\\
  \subfigure[INF-MH]{%
    \includegraphics[width=.48\textwidth]{figures/supplementary/camPose_alpha.pdf} \label{fig:exp1_alpha}
  }
  \mycaption{Results of the `Estimating Camera Extrinsics' experiment}{PRSFs and Acceptance rates corresponding to (a) various standard deviations of \MH, (b) various temperature level combinations of \PT sampling and (c) various mixture coefficients of \MIXLMH sampling.}
\end{figure}



\begin{figure}[!t]
\centering
  \subfigure[\MH]{%
    \includegraphics[width=.48\textwidth]{figures/supplementary/occlusionExp_MH.pdf} \label{fig:exp2_MH}
  }
  \subfigure[\BMHWG]{%
    \includegraphics[width=.48\textwidth]{figures/supplementary/occlusionExp_BMHWG.pdf} \label{fig:exp2_BMHWG}
  }
\\
  \subfigure[\MHWG]{%
    \includegraphics[width=.48\textwidth]{figures/supplementary/occlusionExp_MHWG.pdf} \label{fig:exp2_MHWG}
  }
  \subfigure[\PT]{%
    \includegraphics[width=.48\textwidth]{figures/supplementary/occlusionExp_PT.pdf} \label{fig:exp2_PT}
  }
\\
  \subfigure[\INFBMHWG]{%
    \includegraphics[width=.5\textwidth]{figures/supplementary/occlusionExp_alpha.pdf} \label{fig:exp2_alpha}
  }
  \mycaption{Results of the `Occluding Tiles' experiment}{PRSF and
    Acceptance rates corresponding to various standard deviations of
    (a) \MH, (b) \BMHWG, (c) \MHWG, (d) various temperature level
    combinations of \PT~sampling and; (e) various mixture coefficients
    of our informed \INFBMHWG sampling.}
\end{figure}

%\onecolumn\newpage\twocolumn
\subsection{Experiment: Occluding Tiles}
\label{appendix:chap3:tiles}

\subsubsection{Parameter Selection}

\paragraph{Metropolis Hastings (\MH)}

Figure~\ref{fig:exp2_MH} shows the results of
\MH~sampling. Results show the poor convergence for all proposal
standard deviations and rapid decrease of AR with increasing standard
deviation. This is due to the high-dimensional nature of
the problem. We selected a standard deviation of $1.1$.

\paragraph{Blocked Metropolis Hastings Within Gibbs (\BMHWG)}

The results of \BMHWG are shown in Figure~\ref{fig:exp2_BMHWG}. In
this sampler we update only one block of tile variables (of dimension
four) in each sampling step. Results show much better performance
compared to plain \MH. The optimal proposal standard deviation for
this sampler is $0.7$.

\paragraph{Metropolis Hastings Within Gibbs (\MHWG)}

Figure~\ref{fig:exp2_MHWG} shows the result of \MHWG sampling. This
sampler is better than \BMHWG and converges much more quickly. Here
a standard deviation of $0.9$ is found to be best.

\paragraph{Parallel Tempering (\PT)}

Figure~\ref{fig:exp2_PT} shows the results of \PT sampling with various
temperature combinations. Results show no improvement in AR from plain
\MH sampling and again $[1,3,27]$ temperature levels are found to be optimal.

\paragraph{Effect of Mixture Coefficient in Informed Sampling (\INFBMHWG)}

Figure~\ref{fig:exp2_alpha} shows the effect of mixture
coefficient ($\alpha$) on the blocked informed sampling
\INFBMHWG. Since there is no significant different in PSRF values for
$0 \le \alpha \le 0.8$, we chose $0.8$ due to its high acceptance
rate.



\subsection{Experiment: Estimating Body Shape}
\label{appendix:chap3:body}

\subsubsection{Parameter Selection}
\paragraph{Metropolis Hastings (\MH)}

Figure~\ref{fig:exp3_MH} shows the result of \MH~sampling with various
proposal standard deviations. The value of $0.1$ is found to be
best.

\paragraph{Metropolis Hastings Within Gibbs (\MHWG)}

For \MHWG sampling we select $0.3$ proposal standard
deviation. Results are shown in Fig.~\ref{fig:exp3_MHWG}.


\paragraph{Parallel Tempering (\PT)}

As before, results in Fig.~\ref{fig:exp3_PT}, the temperature levels
were selected to be $[1,3,27]$ due its slightly higher AR.

\paragraph{Effect of Mixture Coefficient in Informed Sampling (\MIXLMH)}

Figure~\ref{fig:exp3_alpha} shows the effect of $\alpha$ on PSRF and
AR. Since there is no significant differences in PSRF values for $0 \le
\alpha \le 0.8$, we choose $0.8$.


\begin{figure}[t]
\centering
  \subfigure[\MH]{%
    \includegraphics[width=.48\textwidth]{figures/supplementary/bodyShape_MH.pdf} \label{fig:exp3_MH}
  }
  \subfigure[\MHWG]{%
    \includegraphics[width=.48\textwidth]{figures/supplementary/bodyShape_MHWG.pdf} \label{fig:exp3_MHWG}
  }
\\
  \subfigure[\PT]{%
    \includegraphics[width=.48\textwidth]{figures/supplementary/bodyShape_PT.pdf} \label{fig:exp3_PT}
  }
  \subfigure[\MIXLMH]{%
    \includegraphics[width=.48\textwidth]{figures/supplementary/bodyShape_alpha.pdf} \label{fig:exp3_alpha}
  }
\\
  \mycaption{Results of the `Body Shape Estimation' experiment}{PRSFs and
    Acceptance rates corresponding to various standard deviations of
    (a) \MH, (b) \MHWG; (c) various temperature level combinations
    of \PT sampling and; (d) various mixture coefficients of the
    informed \MIXLMH sampling.}
\end{figure}


\subsection{Results Overview}
Figure~\ref{fig:exp_summary} shows the summary results of the all the three
experimental studies related to informed sampler.
\begin{figure*}[h!]
\centering
  \subfigure[Results for: Estimating Camera Extrinsics]{%
    \includegraphics[width=0.9\textwidth]{figures/supplementary/camPose_ALL.pdf} \label{fig:exp1_all}
  }
  \subfigure[Results for: Occluding Tiles]{%
    \includegraphics[width=0.9\textwidth]{figures/supplementary/occlusionExp_ALL.pdf} \label{fig:exp2_all}
  }
  \subfigure[Results for: Estimating Body Shape]{%
    \includegraphics[width=0.9\textwidth]{figures/supplementary/bodyShape_ALL.pdf} \label{fig:exp3_all}
  }
  \label{fig:exp_summary}
  \mycaption{Summary of the statistics for the three experiments}{Shown are
    for several baseline methods and the informed samplers the
    acceptance rates (left), PSRFs (middle), and RMSE values
    (right). All results are median results over multiple test
    examples.}
\end{figure*}

\subsection{Additional Qualitative Results}

\subsubsection{Occluding Tiles}
In Figure~\ref{fig:exp2_visual_more} more qualitative results of the
occluding tiles experiment are shown. The informed sampling approach
(\INFBMHWG) is better than the best baseline (\MHWG). This still is a
very challenging problem since the parameters for occluded tiles are
flat over a large region. Some of the posterior variance of the
occluded tiles is already captured by the informed sampler.

\begin{figure*}[h!]
\begin{center}
\centerline{\includegraphics[width=0.95\textwidth]{figures/supplementary/occlusionExp_Visual.pdf}}
\mycaption{Additional qualitative results of the occluding tiles experiment}
  {From left to right: (a)
  Given image, (b) Ground truth tiles, (c) OpenCV heuristic and most probable estimates
  from 5000 samples obtained by (d) MHWG sampler (best baseline) and
  (e) our INF-BMHWG sampler. (f) Posterior expectation of the tiles
  boundaries obtained by INF-BMHWG sampling (First 2000 samples are
  discarded as burn-in).}
\label{fig:exp2_visual_more}
\end{center}
\end{figure*}

\subsubsection{Body Shape}
Figure~\ref{fig:exp3_bodyMeshes} shows some more results of 3D mesh
reconstruction using posterior samples obtained by our informed
sampling \MIXLMH.

\begin{figure*}[t]
\begin{center}
\centerline{\includegraphics[width=0.75\textwidth]{figures/supplementary/bodyMeshResults.pdf}}
\mycaption{Qualitative results for the body shape experiment}
  {Shown is the 3D mesh reconstruction results with first 1000 samples obtained
  using the \MIXLMH informed sampling method. (blue indicates small
  values and red indicates high values)}
\label{fig:exp3_bodyMeshes}
\end{center}
\end{figure*}

\clearpage



\section{Additional Results on the Face Problem with CMP}

Figure~\ref{fig:shading-qualitative-multiple-subjects-supp} shows inference results for reflectance maps, normal maps and lights for randomly chosen test images, and Fig.~\ref{fig:shading-qualitative-same-subject-supp} shows reflectance estimation results on multiple images of the same subject produced under different illumination conditions. CMP is able to produce estimates that are closer to the groundtruth across different subjects and illumination conditions.

\begin{figure*}[h]
  \begin{center}
  \centerline{\includegraphics[width=1.0\columnwidth]{figures/face_cmp_visual_results_supp.pdf}}
  \vspace{-1.2cm}
  \end{center}
	\mycaption{A visual comparison of inference results}{(a)~Observed images. (b)~Inferred reflectance maps. \textit{GT} is the photometric stereo groundtruth, \textit{BU} is the Biswas \etal (2009) reflectance estimate and \textit{Forest} is the consensus prediction. (c)~The variance of the inferred reflectance estimate produced by \MTD (normalized across rows).(d)~Visualization of inferred light directions. (e)~Inferred normal maps.}
	\label{fig:shading-qualitative-multiple-subjects-supp}
\end{figure*}


\begin{figure*}[h]
	\centering
	\setlength\fboxsep{0.2mm}
	\setlength\fboxrule{0pt}
	\begin{tikzpicture}

		\matrix at (0, 0) [matrix of nodes, nodes={anchor=east}, column sep=-0.05cm, row sep=-0.2cm]
		{
			\fbox{\includegraphics[width=1cm]{figures/sample_3_4_X.png}} &
			\fbox{\includegraphics[width=1cm]{figures/sample_3_4_GT.png}} &
			\fbox{\includegraphics[width=1cm]{figures/sample_3_4_BISWAS.png}}  &
			\fbox{\includegraphics[width=1cm]{figures/sample_3_4_VMP.png}}  &
			\fbox{\includegraphics[width=1cm]{figures/sample_3_4_FOREST.png}}  &
			\fbox{\includegraphics[width=1cm]{figures/sample_3_4_CMP.png}}  &
			\fbox{\includegraphics[width=1cm]{figures/sample_3_4_CMPVAR.png}}
			 \\

			\fbox{\includegraphics[width=1cm]{figures/sample_3_5_X.png}} &
			\fbox{\includegraphics[width=1cm]{figures/sample_3_5_GT.png}} &
			\fbox{\includegraphics[width=1cm]{figures/sample_3_5_BISWAS.png}}  &
			\fbox{\includegraphics[width=1cm]{figures/sample_3_5_VMP.png}}  &
			\fbox{\includegraphics[width=1cm]{figures/sample_3_5_FOREST.png}}  &
			\fbox{\includegraphics[width=1cm]{figures/sample_3_5_CMP.png}}  &
			\fbox{\includegraphics[width=1cm]{figures/sample_3_5_CMPVAR.png}}
			 \\

			\fbox{\includegraphics[width=1cm]{figures/sample_3_6_X.png}} &
			\fbox{\includegraphics[width=1cm]{figures/sample_3_6_GT.png}} &
			\fbox{\includegraphics[width=1cm]{figures/sample_3_6_BISWAS.png}}  &
			\fbox{\includegraphics[width=1cm]{figures/sample_3_6_VMP.png}}  &
			\fbox{\includegraphics[width=1cm]{figures/sample_3_6_FOREST.png}}  &
			\fbox{\includegraphics[width=1cm]{figures/sample_3_6_CMP.png}}  &
			\fbox{\includegraphics[width=1cm]{figures/sample_3_6_CMPVAR.png}}
			 \\
	     };

       \node at (-3.85, -2.0) {\small Observed};
       \node at (-2.55, -2.0) {\small `GT'};
       \node at (-1.27, -2.0) {\small BU};
       \node at (0.0, -2.0) {\small MP};
       \node at (1.27, -2.0) {\small Forest};
       \node at (2.55, -2.0) {\small \textbf{CMP}};
       \node at (3.85, -2.0) {\small Variance};

	\end{tikzpicture}
	\mycaption{Robustness to varying illumination}{Reflectance estimation on a subject images with varying illumination. Left to right: observed image, photometric stereo estimate (GT)
  which is used as a proxy for groundtruth, bottom-up estimate of \cite{Biswas2009}, VMP result, consensus forest estimate, CMP mean, and CMP variance.}
	\label{fig:shading-qualitative-same-subject-supp}
\end{figure*}

\clearpage

\section{Additional Material for Learning Sparse High Dimensional Filters}
\label{sec:appendix-bnn}

This part of supplementary material contains a more detailed overview of the permutohedral
lattice convolution in Section~\ref{sec:permconv}, more experiments in
Section~\ref{sec:addexps} and additional results with protocols for
the experiments presented in Chapter~\ref{chap:bnn} in Section~\ref{sec:addresults}.

\vspace{-0.2cm}
\subsection{General Permutohedral Convolutions}
\label{sec:permconv}

A core technical contribution of this work is the generalization of the Gaussian permutohedral lattice
convolution proposed in~\cite{adams2010fast} to the full non-separable case with the
ability to perform back-propagation. Although, conceptually, there are minor
differences between Gaussian and general parameterized filters, there are non-trivial practical
differences in terms of the algorithmic implementation. The Gauss filters belong to
the separable class and can thus be decomposed into multiple
sequential one dimensional convolutions. We are interested in the general filter
convolutions, which can not be decomposed. Thus, performing a general permutohedral
convolution at a lattice point requires the computation of the inner product with the
neighboring elements in all the directions in the high-dimensional space.

Here, we give more details of the implementation differences of separable
and non-separable filters. In the following, we will explain the scalar case first.
Recall, that the forward pass of general permutohedral convolution
involves 3 steps: \textit{splatting}, \textit{convolving} and \textit{slicing}.
We follow the same splatting and slicing strategies as in~\cite{adams2010fast}
since these operations do not depend on the filter kernel. The main difference
between our work and the existing implementation of~\cite{adams2010fast} is
the way that the convolution operation is executed. This proceeds by constructing
a \emph{blur neighbor} matrix $K$ that stores for every lattice point all
values of the lattice neighbors that are needed to compute the filter output.

\begin{figure}[t!]
  \centering
    \includegraphics[width=0.6\columnwidth]{figures/supplementary/lattice_construction}
  \mycaption{Illustration of 1D permutohedral lattice construction}
  {A $4\times 4$ $(x,y)$ grid lattice is projected onto the plane defined by the normal
  vector $(1,1)^{\top}$. This grid has $s+1=4$ and $d=2$ $(s+1)^{d}=4^2=16$ elements.
  In the projection, all points of the same color are projected onto the same points in the plane.
  The number of elements of the projected lattice is $t=(s+1)^d-s^d=4^2-3^2=7$, that is
  the $(4\times 4)$ grid minus the size of lattice that is $1$ smaller at each size, in this
  case a $(3\times 3)$ lattice (the upper right $(3\times 3)$ elements).
  }
\label{fig:latticeconstruction}
\end{figure}

The blur neighbor matrix is constructed by traversing through all the populated
lattice points and their neighboring elements.
% For efficiency, we do this matrix construction recursively with shared computations
% since $n^{th}$ neighbourhood elements are $1^{st}$ neighborhood elements of $n-1^{th}$ neighbourhood elements. \pg{do not understand}
This is done recursively to share computations. For any lattice point, the neighbors that are
$n$ hops away are the direct neighbors of the points that are $n-1$ hops away.
The size of a $d$ dimensional spatial filter with width $s+1$ is $(s+1)^{d}$ (\eg, a
$3\times 3$ filter, $s=2$ in $d=2$ has $3^2=9$ elements) and this size grows
exponentially in the number of dimensions $d$. The permutohedral lattice is constructed by
projecting a regular grid onto the plane spanned by the $d$ dimensional normal vector ${(1,\ldots,1)}^{\top}$. See
Fig.~\ref{fig:latticeconstruction} for an illustration of the 1D lattice construction.
Many corners of a grid filter are projected onto the same point, in total $t = {(s+1)}^{d} -
s^{d}$ elements remain in the permutohedral filter with $s$ neighborhood in $d-1$ dimensions.
If the lattice has $m$ populated elements, the
matrix $K$ has size $t\times m$. Note that, since the input signal is typically
sparse, only a few lattice corners are being populated in the \textit{slicing} step.
We use a hash-table to keep track of these points and traverse only through
the populated lattice points for this neighborhood matrix construction.

Once the blur neighbor matrix $K$ is constructed, we can perform the convolution
by the matrix vector multiplication
\begin{equation}
\ell' = BK,
\label{eq:conv}
\end{equation}
where $B$ is the $1 \times t$ filter kernel (whose values we will learn) and $\ell'\in\mathbb{R}^{1\times m}$
is the result of the filtering at the $m$ lattice points. In practice, we found that the
matrix $K$ is sometimes too large to fit into GPU memory and we divided the matrix $K$
into smaller pieces to compute Eq.~\ref{eq:conv} sequentially.

In the general multi-dimensional case, the signal $\ell$ is of $c$ dimensions. Then
the kernel $B$ is of size $c \times t$ and $K$ stores the $c$ dimensional vectors
accordingly. When the input and output points are different, we slice only the
input points and splat only at the output points.


\subsection{Additional Experiments}
\label{sec:addexps}
In this section, we discuss more use-cases for the learned bilateral filters, one
use-case of BNNs and two single filter applications for image and 3D mesh denoising.

\subsubsection{Recognition of subsampled MNIST}\label{sec:app_mnist}

One of the strengths of the proposed filter convolution is that it does not
require the input to lie on a regular grid. The only requirement is to define a distance
between features of the input signal.
We highlight this feature with the following experiment using the
classical MNIST ten class classification problem~\cite{lecun1998mnist}. We sample a
sparse set of $N$ points $(x,y)\in [0,1]\times [0,1]$
uniformly at random in the input image, use their interpolated values
as signal and the \emph{continuous} $(x,y)$ positions as features. This mimics
sub-sampling of a high-dimensional signal. To compare against a spatial convolution,
we interpolate the sparse set of values at the grid positions.

We take a reference implementation of LeNet~\cite{lecun1998gradient} that
is part of the Caffe project~\cite{jia2014caffe} and compare it
against the same architecture but replacing the first convolutional
layer with a bilateral convolution layer (BCL). The filter size
and numbers are adjusted to get a comparable number of parameters
($5\times 5$ for LeNet, $2$-neighborhood for BCL).

The results are shown in Table~\ref{tab:all-results}. We see that training
on the original MNIST data (column Original, LeNet vs. BNN) leads to a slight
decrease in performance of the BNN (99.03\%) compared to LeNet
(99.19\%). The BNN can be trained and evaluated on sparse
signals, and we resample the image as described above for $N=$ 100\%, 60\% and
20\% of the total number of pixels. The methods are also evaluated
on test images that are subsampled in the same way. Note that we can
train and test with different subsampling rates. We introduce an additional
bilinear interpolation layer for the LeNet architecture to train on the same
data. In essence, both models perform a spatial interpolation and thus we
expect them to yield a similar classification accuracy. Once the data is of
higher dimensions, the permutohedral convolution will be faster due to hashing
the sparse input points, as well as less memory demanding in comparison to
naive application of a spatial convolution with interpolated values.

\begin{table}[t]
  \begin{center}
    \footnotesize
    \centering
    \begin{tabular}[t]{lllll}
      \toprule
              &     & \multicolumn{3}{c}{Test Subsampling} \\
       Method  & Original & 100\% & 60\% & 20\%\\
      \midrule
       LeNet &  \textbf{0.9919} & 0.9660 & 0.9348 & \textbf{0.6434} \\
       BNN &  0.9903 & \textbf{0.9844} & \textbf{0.9534} & 0.5767 \\
      \hline
       LeNet 100\% & 0.9856 & 0.9809 & 0.9678 & \textbf{0.7386} \\
       BNN 100\% & \textbf{0.9900} & \textbf{0.9863} & \textbf{0.9699} & 0.6910 \\
      \hline
       LeNet 60\% & 0.9848 & 0.9821 & 0.9740 & 0.8151 \\
       BNN 60\% & \textbf{0.9885} & \textbf{0.9864} & \textbf{0.9771} & \textbf{0.8214}\\
      \hline
       LeNet 20\% & \textbf{0.9763} & \textbf{0.9754} & 0.9695 & 0.8928 \\
       BNN 20\% & 0.9728 & 0.9735 & \textbf{0.9701} & \textbf{0.9042}\\
      \bottomrule
    \end{tabular}
  \end{center}
\vspace{-.2cm}
\caption{Classification accuracy on MNIST. We compare the
    LeNet~\cite{lecun1998gradient} implementation that is part of
    Caffe~\cite{jia2014caffe} to the network with the first layer
    replaced by a bilateral convolution layer (BCL). Both are trained
    on the original image resolution (first two rows). Three more BNN
    and CNN models are trained with randomly subsampled images (100\%,
    60\% and 20\% of the pixels). An additional bilinear interpolation
    layer samples the input signal on a spatial grid for the CNN model.
  }
  \label{tab:all-results}
\vspace{-.5cm}
\end{table}

\subsubsection{Image Denoising}

The main application that inspired the development of the bilateral
filtering operation is image denoising~\cite{aurich1995non}, there
using a single Gaussian kernel. Our development allows to learn this
kernel function from data and we explore how to improve using a \emph{single}
but more general bilateral filter.

We use the Berkeley segmentation dataset
(BSDS500)~\cite{arbelaezi2011bsds500} as a test bed. The color
images in the dataset are converted to gray-scale,
and corrupted with Gaussian noise with a standard deviation of
$\frac {25} {255}$.

We compare the performance of four different filter models on a
denoising task.
The first baseline model (`Spatial' in Table \ref{tab:denoising}, $25$
weights) uses a single spatial filter with a kernel size of
$5$ and predicts the scalar gray-scale value at the center pixel. The next model
(`Gauss Bilateral') applies a bilateral \emph{Gaussian}
filter to the noisy input, using position and intensity features $\f=(x,y,v)^\top$.
The third setup (`Learned Bilateral', $65$ weights)
takes a Gauss kernel as initialization and
fits all filter weights on the train set to minimize the
mean squared error with respect to the clean images.
We run a combination
of spatial and permutohedral convolutions on spatial and bilateral
features (`Spatial + Bilateral (Learned)') to check for a complementary
performance of the two convolutions.

\label{sec:exp:denoising}
\begin{table}[!h]
\begin{center}
  \footnotesize
  \begin{tabular}[t]{lr}
    \toprule
    Method & PSNR \\
    \midrule
    Noisy Input & $20.17$ \\
    Spatial & $26.27$ \\
    Gauss Bilateral & $26.51$ \\
    Learned Bilateral & $26.58$ \\
    Spatial + Bilateral (Learned) & \textbf{$26.65$} \\
    \bottomrule
  \end{tabular}
\end{center}
\vspace{-0.5em}
\caption{PSNR results of a denoising task using the BSDS500
  dataset~\cite{arbelaezi2011bsds500}}
\vspace{-0.5em}
\label{tab:denoising}
\end{table}
\vspace{-0.2em}

The PSNR scores evaluated on full images of the test set are
shown in Table \ref{tab:denoising}. We find that an untrained bilateral
filter already performs better than a trained spatial convolution
($26.27$ to $26.51$). A learned convolution further improve the
performance slightly. We chose this simple one-kernel setup to
validate an advantage of the generalized bilateral filter. A competitive
denoising system would employ RGB color information and also
needs to be properly adjusted in network size. Multi-layer perceptrons
have obtained state-of-the-art denoising results~\cite{burger12cvpr}
and the permutohedral lattice layer can readily be used in such an
architecture, which is intended future work.

\subsection{Additional results}
\label{sec:addresults}

This section contains more qualitative results for the experiments presented in Chapter~\ref{chap:bnn}.

\begin{figure*}[th!]
  \centering
    \includegraphics[width=\columnwidth,trim={5cm 2.5cm 5cm 4.5cm},clip]{figures/supplementary/lattice_viz.pdf}
    \vspace{-0.7cm}
  \mycaption{Visualization of the Permutohedral Lattice}
  {Sample lattice visualizations for different feature spaces. All pixels falling in the same simplex cell are shown with
  the same color. $(x,y)$ features correspond to image pixel positions, and $(r,g,b) \in [0,255]$ correspond
  to the red, green and blue color values.}
\label{fig:latticeviz}
\end{figure*}

\subsubsection{Lattice Visualization}

Figure~\ref{fig:latticeviz} shows sample lattice visualizations for different feature spaces.

\newcolumntype{L}[1]{>{\raggedright\let\newline\\\arraybackslash\hspace{0pt}}b{#1}}
\newcolumntype{C}[1]{>{\centering\let\newline\\\arraybackslash\hspace{0pt}}b{#1}}
\newcolumntype{R}[1]{>{\raggedleft\let\newline\\\arraybackslash\hspace{0pt}}b{#1}}

\subsubsection{Color Upsampling}\label{sec:color_upsampling}
\label{sec:col_upsample_extra}

Some images of the upsampling for the Pascal
VOC12 dataset are shown in Fig.~\ref{fig:Colour_upsample_visuals}. It is
especially the low level image details that are better preserved with
a learned bilateral filter compared to the Gaussian case.

\begin{figure*}[t!]
  \centering
    \subfigure{%
   \raisebox{2.0em}{
    \includegraphics[width=.06\columnwidth]{figures/supplementary/2007_004969.jpg}
   }
  }
  \subfigure{%
    \includegraphics[width=.17\columnwidth]{figures/supplementary/2007_004969_gray.pdf}
  }
  \subfigure{%
    \includegraphics[width=.17\columnwidth]{figures/supplementary/2007_004969_gt.pdf}
  }
  \subfigure{%
    \includegraphics[width=.17\columnwidth]{figures/supplementary/2007_004969_bicubic.pdf}
  }
  \subfigure{%
    \includegraphics[width=.17\columnwidth]{figures/supplementary/2007_004969_gauss.pdf}
  }
  \subfigure{%
    \includegraphics[width=.17\columnwidth]{figures/supplementary/2007_004969_learnt.pdf}
  }\\
    \subfigure{%
   \raisebox{2.0em}{
    \includegraphics[width=.06\columnwidth]{figures/supplementary/2007_003106.jpg}
   }
  }
  \subfigure{%
    \includegraphics[width=.17\columnwidth]{figures/supplementary/2007_003106_gray.pdf}
  }
  \subfigure{%
    \includegraphics[width=.17\columnwidth]{figures/supplementary/2007_003106_gt.pdf}
  }
  \subfigure{%
    \includegraphics[width=.17\columnwidth]{figures/supplementary/2007_003106_bicubic.pdf}
  }
  \subfigure{%
    \includegraphics[width=.17\columnwidth]{figures/supplementary/2007_003106_gauss.pdf}
  }
  \subfigure{%
    \includegraphics[width=.17\columnwidth]{figures/supplementary/2007_003106_learnt.pdf}
  }\\
  \setcounter{subfigure}{0}
  \small{
  \subfigure[Inp.]{%
  \raisebox{2.0em}{
    \includegraphics[width=.06\columnwidth]{figures/supplementary/2007_006837.jpg}
   }
  }
  \subfigure[Guidance]{%
    \includegraphics[width=.17\columnwidth]{figures/supplementary/2007_006837_gray.pdf}
  }
   \subfigure[GT]{%
    \includegraphics[width=.17\columnwidth]{figures/supplementary/2007_006837_gt.pdf}
  }
  \subfigure[Bicubic]{%
    \includegraphics[width=.17\columnwidth]{figures/supplementary/2007_006837_bicubic.pdf}
  }
  \subfigure[Gauss-BF]{%
    \includegraphics[width=.17\columnwidth]{figures/supplementary/2007_006837_gauss.pdf}
  }
  \subfigure[Learned-BF]{%
    \includegraphics[width=.17\columnwidth]{figures/supplementary/2007_006837_learnt.pdf}
  }
  }
  \vspace{-0.5cm}
  \mycaption{Color Upsampling}{Color $8\times$ upsampling results
  using different methods, from left to right, (a)~Low-resolution input color image (Inp.),
  (b)~Gray scale guidance image, (c)~Ground-truth color image; Upsampled color images with
  (d)~Bicubic interpolation, (e) Gauss bilateral upsampling and, (f)~Learned bilateral
  updampgling (best viewed on screen).}

\label{fig:Colour_upsample_visuals}
\end{figure*}

\subsubsection{Depth Upsampling}
\label{sec:depth_upsample_extra}

Figure~\ref{fig:depth_upsample_visuals} presents some more qualitative results comparing bicubic interpolation, Gauss
bilateral and learned bilateral upsampling on NYU depth dataset image~\cite{silberman2012indoor}.

\subsubsection{Character Recognition}\label{sec:app_character}

 Figure~\ref{fig:nnrecognition} shows the schematic of different layers
 of the network architecture for LeNet-7~\cite{lecun1998mnist}
 and DeepCNet(5, 50)~\cite{ciresan2012multi,graham2014spatially}. For the BNN variants, the first layer filters are replaced
 with learned bilateral filters and are learned end-to-end.

\subsubsection{Semantic Segmentation}\label{sec:app_semantic_segmentation}
\label{sec:semantic_bnn_extra}

Some more visual results for semantic segmentation are shown in Figure~\ref{fig:semantic_visuals}.
These include the underlying DeepLab CNN\cite{chen2014semantic} result (DeepLab),
the 2 step mean-field result with Gaussian edge potentials (+2stepMF-GaussCRF)
and also corresponding results with learned edge potentials (+2stepMF-LearnedCRF).
In general, we observe that mean-field in learned CRF leads to slightly dilated
classification regions in comparison to using Gaussian CRF thereby filling-in the
false negative pixels and also correcting some mis-classified regions.

\begin{figure*}[t!]
  \centering
    \subfigure{%
   \raisebox{2.0em}{
    \includegraphics[width=.06\columnwidth]{figures/supplementary/2bicubic}
   }
  }
  \subfigure{%
    \includegraphics[width=.17\columnwidth]{figures/supplementary/2given_image}
  }
  \subfigure{%
    \includegraphics[width=.17\columnwidth]{figures/supplementary/2ground_truth}
  }
  \subfigure{%
    \includegraphics[width=.17\columnwidth]{figures/supplementary/2bicubic}
  }
  \subfigure{%
    \includegraphics[width=.17\columnwidth]{figures/supplementary/2gauss}
  }
  \subfigure{%
    \includegraphics[width=.17\columnwidth]{figures/supplementary/2learnt}
  }\\
    \subfigure{%
   \raisebox{2.0em}{
    \includegraphics[width=.06\columnwidth]{figures/supplementary/32bicubic}
   }
  }
  \subfigure{%
    \includegraphics[width=.17\columnwidth]{figures/supplementary/32given_image}
  }
  \subfigure{%
    \includegraphics[width=.17\columnwidth]{figures/supplementary/32ground_truth}
  }
  \subfigure{%
    \includegraphics[width=.17\columnwidth]{figures/supplementary/32bicubic}
  }
  \subfigure{%
    \includegraphics[width=.17\columnwidth]{figures/supplementary/32gauss}
  }
  \subfigure{%
    \includegraphics[width=.17\columnwidth]{figures/supplementary/32learnt}
  }\\
  \setcounter{subfigure}{0}
  \small{
  \subfigure[Inp.]{%
  \raisebox{2.0em}{
    \includegraphics[width=.06\columnwidth]{figures/supplementary/41bicubic}
   }
  }
  \subfigure[Guidance]{%
    \includegraphics[width=.17\columnwidth]{figures/supplementary/41given_image}
  }
   \subfigure[GT]{%
    \includegraphics[width=.17\columnwidth]{figures/supplementary/41ground_truth}
  }
  \subfigure[Bicubic]{%
    \includegraphics[width=.17\columnwidth]{figures/supplementary/41bicubic}
  }
  \subfigure[Gauss-BF]{%
    \includegraphics[width=.17\columnwidth]{figures/supplementary/41gauss}
  }
  \subfigure[Learned-BF]{%
    \includegraphics[width=.17\columnwidth]{figures/supplementary/41learnt}
  }
  }
  \mycaption{Depth Upsampling}{Depth $8\times$ upsampling results
  using different upsampling strategies, from left to right,
  (a)~Low-resolution input depth image (Inp.),
  (b)~High-resolution guidance image, (c)~Ground-truth depth; Upsampled depth images with
  (d)~Bicubic interpolation, (e) Gauss bilateral upsampling and, (f)~Learned bilateral
  updampgling (best viewed on screen).}

\label{fig:depth_upsample_visuals}
\end{figure*}

\subsubsection{Material Segmentation}\label{sec:app_material_segmentation}
\label{sec:material_bnn_extra}

In Fig.~\ref{fig:material_visuals-app2}, we present visual results comparing 2 step
mean-field inference with Gaussian and learned pairwise CRF potentials. In
general, we observe that the pixels belonging to dominant classes in the
training data are being more accurately classified with learned CRF. This leads to
a significant improvements in overall pixel accuracy. This also results
in a slight decrease of the accuracy from less frequent class pixels thereby
slightly reducing the average class accuracy with learning. We attribute this
to the type of annotation that is available for this dataset, which is not
for the entire image but for some segments in the image. We have very few
images of the infrequent classes to combat this behaviour during training.

\subsubsection{Experiment Protocols}
\label{sec:protocols}

Table~\ref{tbl:parameters} shows experiment protocols of different experiments.

 \begin{figure*}[t!]
  \centering
  \subfigure[LeNet-7]{
    \includegraphics[width=0.7\columnwidth]{figures/supplementary/lenet_cnn_network}
    }\\
    \subfigure[DeepCNet]{
    \includegraphics[width=\columnwidth]{figures/supplementary/deepcnet_cnn_network}
    }
  \mycaption{CNNs for Character Recognition}
  {Schematic of (top) LeNet-7~\cite{lecun1998mnist} and (bottom) DeepCNet(5,50)~\cite{ciresan2012multi,graham2014spatially} architectures used in Assamese
  character recognition experiments.}
\label{fig:nnrecognition}
\end{figure*}

\definecolor{voc_1}{RGB}{0, 0, 0}
\definecolor{voc_2}{RGB}{128, 0, 0}
\definecolor{voc_3}{RGB}{0, 128, 0}
\definecolor{voc_4}{RGB}{128, 128, 0}
\definecolor{voc_5}{RGB}{0, 0, 128}
\definecolor{voc_6}{RGB}{128, 0, 128}
\definecolor{voc_7}{RGB}{0, 128, 128}
\definecolor{voc_8}{RGB}{128, 128, 128}
\definecolor{voc_9}{RGB}{64, 0, 0}
\definecolor{voc_10}{RGB}{192, 0, 0}
\definecolor{voc_11}{RGB}{64, 128, 0}
\definecolor{voc_12}{RGB}{192, 128, 0}
\definecolor{voc_13}{RGB}{64, 0, 128}
\definecolor{voc_14}{RGB}{192, 0, 128}
\definecolor{voc_15}{RGB}{64, 128, 128}
\definecolor{voc_16}{RGB}{192, 128, 128}
\definecolor{voc_17}{RGB}{0, 64, 0}
\definecolor{voc_18}{RGB}{128, 64, 0}
\definecolor{voc_19}{RGB}{0, 192, 0}
\definecolor{voc_20}{RGB}{128, 192, 0}
\definecolor{voc_21}{RGB}{0, 64, 128}
\definecolor{voc_22}{RGB}{128, 64, 128}

\begin{figure*}[t]
  \centering
  \small{
  \fcolorbox{white}{voc_1}{\rule{0pt}{6pt}\rule{6pt}{0pt}} Background~~
  \fcolorbox{white}{voc_2}{\rule{0pt}{6pt}\rule{6pt}{0pt}} Aeroplane~~
  \fcolorbox{white}{voc_3}{\rule{0pt}{6pt}\rule{6pt}{0pt}} Bicycle~~
  \fcolorbox{white}{voc_4}{\rule{0pt}{6pt}\rule{6pt}{0pt}} Bird~~
  \fcolorbox{white}{voc_5}{\rule{0pt}{6pt}\rule{6pt}{0pt}} Boat~~
  \fcolorbox{white}{voc_6}{\rule{0pt}{6pt}\rule{6pt}{0pt}} Bottle~~
  \fcolorbox{white}{voc_7}{\rule{0pt}{6pt}\rule{6pt}{0pt}} Bus~~
  \fcolorbox{white}{voc_8}{\rule{0pt}{6pt}\rule{6pt}{0pt}} Car~~ \\
  \fcolorbox{white}{voc_9}{\rule{0pt}{6pt}\rule{6pt}{0pt}} Cat~~
  \fcolorbox{white}{voc_10}{\rule{0pt}{6pt}\rule{6pt}{0pt}} Chair~~
  \fcolorbox{white}{voc_11}{\rule{0pt}{6pt}\rule{6pt}{0pt}} Cow~~
  \fcolorbox{white}{voc_12}{\rule{0pt}{6pt}\rule{6pt}{0pt}} Dining Table~~
  \fcolorbox{white}{voc_13}{\rule{0pt}{6pt}\rule{6pt}{0pt}} Dog~~
  \fcolorbox{white}{voc_14}{\rule{0pt}{6pt}\rule{6pt}{0pt}} Horse~~
  \fcolorbox{white}{voc_15}{\rule{0pt}{6pt}\rule{6pt}{0pt}} Motorbike~~
  \fcolorbox{white}{voc_16}{\rule{0pt}{6pt}\rule{6pt}{0pt}} Person~~ \\
  \fcolorbox{white}{voc_17}{\rule{0pt}{6pt}\rule{6pt}{0pt}} Potted Plant~~
  \fcolorbox{white}{voc_18}{\rule{0pt}{6pt}\rule{6pt}{0pt}} Sheep~~
  \fcolorbox{white}{voc_19}{\rule{0pt}{6pt}\rule{6pt}{0pt}} Sofa~~
  \fcolorbox{white}{voc_20}{\rule{0pt}{6pt}\rule{6pt}{0pt}} Train~~
  \fcolorbox{white}{voc_21}{\rule{0pt}{6pt}\rule{6pt}{0pt}} TV monitor~~ \\
  }
  \subfigure{%
    \includegraphics[width=.18\columnwidth]{figures/supplementary/2007_001423_given.jpg}
  }
  \subfigure{%
    \includegraphics[width=.18\columnwidth]{figures/supplementary/2007_001423_gt.png}
  }
  \subfigure{%
    \includegraphics[width=.18\columnwidth]{figures/supplementary/2007_001423_cnn.png}
  }
  \subfigure{%
    \includegraphics[width=.18\columnwidth]{figures/supplementary/2007_001423_gauss.png}
  }
  \subfigure{%
    \includegraphics[width=.18\columnwidth]{figures/supplementary/2007_001423_learnt.png}
  }\\
  \subfigure{%
    \includegraphics[width=.18\columnwidth]{figures/supplementary/2007_001430_given.jpg}
  }
  \subfigure{%
    \includegraphics[width=.18\columnwidth]{figures/supplementary/2007_001430_gt.png}
  }
  \subfigure{%
    \includegraphics[width=.18\columnwidth]{figures/supplementary/2007_001430_cnn.png}
  }
  \subfigure{%
    \includegraphics[width=.18\columnwidth]{figures/supplementary/2007_001430_gauss.png}
  }
  \subfigure{%
    \includegraphics[width=.18\columnwidth]{figures/supplementary/2007_001430_learnt.png}
  }\\
    \subfigure{%
    \includegraphics[width=.18\columnwidth]{figures/supplementary/2007_007996_given.jpg}
  }
  \subfigure{%
    \includegraphics[width=.18\columnwidth]{figures/supplementary/2007_007996_gt.png}
  }
  \subfigure{%
    \includegraphics[width=.18\columnwidth]{figures/supplementary/2007_007996_cnn.png}
  }
  \subfigure{%
    \includegraphics[width=.18\columnwidth]{figures/supplementary/2007_007996_gauss.png}
  }
  \subfigure{%
    \includegraphics[width=.18\columnwidth]{figures/supplementary/2007_007996_learnt.png}
  }\\
   \subfigure{%
    \includegraphics[width=.18\columnwidth]{figures/supplementary/2010_002682_given.jpg}
  }
  \subfigure{%
    \includegraphics[width=.18\columnwidth]{figures/supplementary/2010_002682_gt.png}
  }
  \subfigure{%
    \includegraphics[width=.18\columnwidth]{figures/supplementary/2010_002682_cnn.png}
  }
  \subfigure{%
    \includegraphics[width=.18\columnwidth]{figures/supplementary/2010_002682_gauss.png}
  }
  \subfigure{%
    \includegraphics[width=.18\columnwidth]{figures/supplementary/2010_002682_learnt.png}
  }\\
     \subfigure{%
    \includegraphics[width=.18\columnwidth]{figures/supplementary/2010_004789_given.jpg}
  }
  \subfigure{%
    \includegraphics[width=.18\columnwidth]{figures/supplementary/2010_004789_gt.png}
  }
  \subfigure{%
    \includegraphics[width=.18\columnwidth]{figures/supplementary/2010_004789_cnn.png}
  }
  \subfigure{%
    \includegraphics[width=.18\columnwidth]{figures/supplementary/2010_004789_gauss.png}
  }
  \subfigure{%
    \includegraphics[width=.18\columnwidth]{figures/supplementary/2010_004789_learnt.png}
  }\\
       \subfigure{%
    \includegraphics[width=.18\columnwidth]{figures/supplementary/2007_001311_given.jpg}
  }
  \subfigure{%
    \includegraphics[width=.18\columnwidth]{figures/supplementary/2007_001311_gt.png}
  }
  \subfigure{%
    \includegraphics[width=.18\columnwidth]{figures/supplementary/2007_001311_cnn.png}
  }
  \subfigure{%
    \includegraphics[width=.18\columnwidth]{figures/supplementary/2007_001311_gauss.png}
  }
  \subfigure{%
    \includegraphics[width=.18\columnwidth]{figures/supplementary/2007_001311_learnt.png}
  }\\
  \setcounter{subfigure}{0}
  \subfigure[Input]{%
    \includegraphics[width=.18\columnwidth]{figures/supplementary/2010_003531_given.jpg}
  }
  \subfigure[Ground Truth]{%
    \includegraphics[width=.18\columnwidth]{figures/supplementary/2010_003531_gt.png}
  }
  \subfigure[DeepLab]{%
    \includegraphics[width=.18\columnwidth]{figures/supplementary/2010_003531_cnn.png}
  }
  \subfigure[+GaussCRF]{%
    \includegraphics[width=.18\columnwidth]{figures/supplementary/2010_003531_gauss.png}
  }
  \subfigure[+LearnedCRF]{%
    \includegraphics[width=.18\columnwidth]{figures/supplementary/2010_003531_learnt.png}
  }
  \vspace{-0.3cm}
  \mycaption{Semantic Segmentation}{Example results of semantic segmentation.
  (c)~depicts the unary results before application of MF, (d)~after two steps of MF with Gaussian edge CRF potentials, (e)~after
  two steps of MF with learned edge CRF potentials.}
    \label{fig:semantic_visuals}
\end{figure*}


\definecolor{minc_1}{HTML}{771111}
\definecolor{minc_2}{HTML}{CAC690}
\definecolor{minc_3}{HTML}{EEEEEE}
\definecolor{minc_4}{HTML}{7C8FA6}
\definecolor{minc_5}{HTML}{597D31}
\definecolor{minc_6}{HTML}{104410}
\definecolor{minc_7}{HTML}{BB819C}
\definecolor{minc_8}{HTML}{D0CE48}
\definecolor{minc_9}{HTML}{622745}
\definecolor{minc_10}{HTML}{666666}
\definecolor{minc_11}{HTML}{D54A31}
\definecolor{minc_12}{HTML}{101044}
\definecolor{minc_13}{HTML}{444126}
\definecolor{minc_14}{HTML}{75D646}
\definecolor{minc_15}{HTML}{DD4348}
\definecolor{minc_16}{HTML}{5C8577}
\definecolor{minc_17}{HTML}{C78472}
\definecolor{minc_18}{HTML}{75D6D0}
\definecolor{minc_19}{HTML}{5B4586}
\definecolor{minc_20}{HTML}{C04393}
\definecolor{minc_21}{HTML}{D69948}
\definecolor{minc_22}{HTML}{7370D8}
\definecolor{minc_23}{HTML}{7A3622}
\definecolor{minc_24}{HTML}{000000}

\begin{figure*}[t]
  \centering
  \small{
  \fcolorbox{white}{minc_1}{\rule{0pt}{6pt}\rule{6pt}{0pt}} Brick~~
  \fcolorbox{white}{minc_2}{\rule{0pt}{6pt}\rule{6pt}{0pt}} Carpet~~
  \fcolorbox{white}{minc_3}{\rule{0pt}{6pt}\rule{6pt}{0pt}} Ceramic~~
  \fcolorbox{white}{minc_4}{\rule{0pt}{6pt}\rule{6pt}{0pt}} Fabric~~
  \fcolorbox{white}{minc_5}{\rule{0pt}{6pt}\rule{6pt}{0pt}} Foliage~~
  \fcolorbox{white}{minc_6}{\rule{0pt}{6pt}\rule{6pt}{0pt}} Food~~
  \fcolorbox{white}{minc_7}{\rule{0pt}{6pt}\rule{6pt}{0pt}} Glass~~
  \fcolorbox{white}{minc_8}{\rule{0pt}{6pt}\rule{6pt}{0pt}} Hair~~ \\
  \fcolorbox{white}{minc_9}{\rule{0pt}{6pt}\rule{6pt}{0pt}} Leather~~
  \fcolorbox{white}{minc_10}{\rule{0pt}{6pt}\rule{6pt}{0pt}} Metal~~
  \fcolorbox{white}{minc_11}{\rule{0pt}{6pt}\rule{6pt}{0pt}} Mirror~~
  \fcolorbox{white}{minc_12}{\rule{0pt}{6pt}\rule{6pt}{0pt}} Other~~
  \fcolorbox{white}{minc_13}{\rule{0pt}{6pt}\rule{6pt}{0pt}} Painted~~
  \fcolorbox{white}{minc_14}{\rule{0pt}{6pt}\rule{6pt}{0pt}} Paper~~
  \fcolorbox{white}{minc_15}{\rule{0pt}{6pt}\rule{6pt}{0pt}} Plastic~~\\
  \fcolorbox{white}{minc_16}{\rule{0pt}{6pt}\rule{6pt}{0pt}} Polished Stone~~
  \fcolorbox{white}{minc_17}{\rule{0pt}{6pt}\rule{6pt}{0pt}} Skin~~
  \fcolorbox{white}{minc_18}{\rule{0pt}{6pt}\rule{6pt}{0pt}} Sky~~
  \fcolorbox{white}{minc_19}{\rule{0pt}{6pt}\rule{6pt}{0pt}} Stone~~
  \fcolorbox{white}{minc_20}{\rule{0pt}{6pt}\rule{6pt}{0pt}} Tile~~
  \fcolorbox{white}{minc_21}{\rule{0pt}{6pt}\rule{6pt}{0pt}} Wallpaper~~
  \fcolorbox{white}{minc_22}{\rule{0pt}{6pt}\rule{6pt}{0pt}} Water~~
  \fcolorbox{white}{minc_23}{\rule{0pt}{6pt}\rule{6pt}{0pt}} Wood~~ \\
  }
  \subfigure{%
    \includegraphics[width=.18\columnwidth]{figures/supplementary/000010868_given.jpg}
  }
  \subfigure{%
    \includegraphics[width=.18\columnwidth]{figures/supplementary/000010868_gt.png}
  }
  \subfigure{%
    \includegraphics[width=.18\columnwidth]{figures/supplementary/000010868_cnn.png}
  }
  \subfigure{%
    \includegraphics[width=.18\columnwidth]{figures/supplementary/000010868_gauss.png}
  }
  \subfigure{%
    \includegraphics[width=.18\columnwidth]{figures/supplementary/000010868_learnt.png}
  }\\[-2ex]
  \subfigure{%
    \includegraphics[width=.18\columnwidth]{figures/supplementary/000006011_given.jpg}
  }
  \subfigure{%
    \includegraphics[width=.18\columnwidth]{figures/supplementary/000006011_gt.png}
  }
  \subfigure{%
    \includegraphics[width=.18\columnwidth]{figures/supplementary/000006011_cnn.png}
  }
  \subfigure{%
    \includegraphics[width=.18\columnwidth]{figures/supplementary/000006011_gauss.png}
  }
  \subfigure{%
    \includegraphics[width=.18\columnwidth]{figures/supplementary/000006011_learnt.png}
  }\\[-2ex]
    \subfigure{%
    \includegraphics[width=.18\columnwidth]{figures/supplementary/000008553_given.jpg}
  }
  \subfigure{%
    \includegraphics[width=.18\columnwidth]{figures/supplementary/000008553_gt.png}
  }
  \subfigure{%
    \includegraphics[width=.18\columnwidth]{figures/supplementary/000008553_cnn.png}
  }
  \subfigure{%
    \includegraphics[width=.18\columnwidth]{figures/supplementary/000008553_gauss.png}
  }
  \subfigure{%
    \includegraphics[width=.18\columnwidth]{figures/supplementary/000008553_learnt.png}
  }\\[-2ex]
   \subfigure{%
    \includegraphics[width=.18\columnwidth]{figures/supplementary/000009188_given.jpg}
  }
  \subfigure{%
    \includegraphics[width=.18\columnwidth]{figures/supplementary/000009188_gt.png}
  }
  \subfigure{%
    \includegraphics[width=.18\columnwidth]{figures/supplementary/000009188_cnn.png}
  }
  \subfigure{%
    \includegraphics[width=.18\columnwidth]{figures/supplementary/000009188_gauss.png}
  }
  \subfigure{%
    \includegraphics[width=.18\columnwidth]{figures/supplementary/000009188_learnt.png}
  }\\[-2ex]
  \setcounter{subfigure}{0}
  \subfigure[Input]{%
    \includegraphics[width=.18\columnwidth]{figures/supplementary/000023570_given.jpg}
  }
  \subfigure[Ground Truth]{%
    \includegraphics[width=.18\columnwidth]{figures/supplementary/000023570_gt.png}
  }
  \subfigure[DeepLab]{%
    \includegraphics[width=.18\columnwidth]{figures/supplementary/000023570_cnn.png}
  }
  \subfigure[+GaussCRF]{%
    \includegraphics[width=.18\columnwidth]{figures/supplementary/000023570_gauss.png}
  }
  \subfigure[+LearnedCRF]{%
    \includegraphics[width=.18\columnwidth]{figures/supplementary/000023570_learnt.png}
  }
  \mycaption{Material Segmentation}{Example results of material segmentation.
  (c)~depicts the unary results before application of MF, (d)~after two steps of MF with Gaussian edge CRF potentials, (e)~after two steps of MF with learned edge CRF potentials.}
    \label{fig:material_visuals-app2}
\end{figure*}


\begin{table*}[h]
\tiny
  \centering
    \begin{tabular}{L{2.3cm} L{2.25cm} C{1.5cm} C{0.7cm} C{0.6cm} C{0.7cm} C{0.7cm} C{0.7cm} C{1.6cm} C{0.6cm} C{0.6cm} C{0.6cm}}
      \toprule
& & & & & \multicolumn{3}{c}{\textbf{Data Statistics}} & \multicolumn{4}{c}{\textbf{Training Protocol}} \\

\textbf{Experiment} & \textbf{Feature Types} & \textbf{Feature Scales} & \textbf{Filter Size} & \textbf{Filter Nbr.} & \textbf{Train}  & \textbf{Val.} & \textbf{Test} & \textbf{Loss Type} & \textbf{LR} & \textbf{Batch} & \textbf{Epochs} \\
      \midrule
      \multicolumn{2}{c}{\textbf{Single Bilateral Filter Applications}} & & & & & & & & & \\
      \textbf{2$\times$ Color Upsampling} & Position$_{1}$, Intensity (3D) & 0.13, 0.17 & 65 & 2 & 10581 & 1449 & 1456 & MSE & 1e-06 & 200 & 94.5\\
      \textbf{4$\times$ Color Upsampling} & Position$_{1}$, Intensity (3D) & 0.06, 0.17 & 65 & 2 & 10581 & 1449 & 1456 & MSE & 1e-06 & 200 & 94.5\\
      \textbf{8$\times$ Color Upsampling} & Position$_{1}$, Intensity (3D) & 0.03, 0.17 & 65 & 2 & 10581 & 1449 & 1456 & MSE & 1e-06 & 200 & 94.5\\
      \textbf{16$\times$ Color Upsampling} & Position$_{1}$, Intensity (3D) & 0.02, 0.17 & 65 & 2 & 10581 & 1449 & 1456 & MSE & 1e-06 & 200 & 94.5\\
      \textbf{Depth Upsampling} & Position$_{1}$, Color (5D) & 0.05, 0.02 & 665 & 2 & 795 & 100 & 654 & MSE & 1e-07 & 50 & 251.6\\
      \textbf{Mesh Denoising} & Isomap (4D) & 46.00 & 63 & 2 & 1000 & 200 & 500 & MSE & 100 & 10 & 100.0 \\
      \midrule
      \multicolumn{2}{c}{\textbf{DenseCRF Applications}} & & & & & & & & &\\
      \multicolumn{2}{l}{\textbf{Semantic Segmentation}} & & & & & & & & &\\
      \textbf{- 1step MF} & Position$_{1}$, Color (5D); Position$_{1}$ (2D) & 0.01, 0.34; 0.34  & 665; 19  & 2; 2 & 10581 & 1449 & 1456 & Logistic & 0.1 & 5 & 1.4 \\
      \textbf{- 2step MF} & Position$_{1}$, Color (5D); Position$_{1}$ (2D) & 0.01, 0.34; 0.34 & 665; 19 & 2; 2 & 10581 & 1449 & 1456 & Logistic & 0.1 & 5 & 1.4 \\
      \textbf{- \textit{loose} 2step MF} & Position$_{1}$, Color (5D); Position$_{1}$ (2D) & 0.01, 0.34; 0.34 & 665; 19 & 2; 2 &10581 & 1449 & 1456 & Logistic & 0.1 & 5 & +1.9  \\ \\
      \multicolumn{2}{l}{\textbf{Material Segmentation}} & & & & & & & & &\\
      \textbf{- 1step MF} & Position$_{2}$, Lab-Color (5D) & 5.00, 0.05, 0.30  & 665 & 2 & 928 & 150 & 1798 & Weighted Logistic & 1e-04 & 24 & 2.6 \\
      \textbf{- 2step MF} & Position$_{2}$, Lab-Color (5D) & 5.00, 0.05, 0.30 & 665 & 2 & 928 & 150 & 1798 & Weighted Logistic & 1e-04 & 12 & +0.7 \\
      \textbf{- \textit{loose} 2step MF} & Position$_{2}$, Lab-Color (5D) & 5.00, 0.05, 0.30 & 665 & 2 & 928 & 150 & 1798 & Weighted Logistic & 1e-04 & 12 & +0.2\\
      \midrule
      \multicolumn{2}{c}{\textbf{Neural Network Applications}} & & & & & & & & &\\
      \textbf{Tiles: CNN-9$\times$9} & - & - & 81 & 4 & 10000 & 1000 & 1000 & Logistic & 0.01 & 100 & 500.0 \\
      \textbf{Tiles: CNN-13$\times$13} & - & - & 169 & 6 & 10000 & 1000 & 1000 & Logistic & 0.01 & 100 & 500.0 \\
      \textbf{Tiles: CNN-17$\times$17} & - & - & 289 & 8 & 10000 & 1000 & 1000 & Logistic & 0.01 & 100 & 500.0 \\
      \textbf{Tiles: CNN-21$\times$21} & - & - & 441 & 10 & 10000 & 1000 & 1000 & Logistic & 0.01 & 100 & 500.0 \\
      \textbf{Tiles: BNN} & Position$_{1}$, Color (5D) & 0.05, 0.04 & 63 & 1 & 10000 & 1000 & 1000 & Logistic & 0.01 & 100 & 30.0 \\
      \textbf{LeNet} & - & - & 25 & 2 & 5490 & 1098 & 1647 & Logistic & 0.1 & 100 & 182.2 \\
      \textbf{Crop-LeNet} & - & - & 25 & 2 & 5490 & 1098 & 1647 & Logistic & 0.1 & 100 & 182.2 \\
      \textbf{BNN-LeNet} & Position$_{2}$ (2D) & 20.00 & 7 & 1 & 5490 & 1098 & 1647 & Logistic & 0.1 & 100 & 182.2 \\
      \textbf{DeepCNet} & - & - & 9 & 1 & 5490 & 1098 & 1647 & Logistic & 0.1 & 100 & 182.2 \\
      \textbf{Crop-DeepCNet} & - & - & 9 & 1 & 5490 & 1098 & 1647 & Logistic & 0.1 & 100 & 182.2 \\
      \textbf{BNN-DeepCNet} & Position$_{2}$ (2D) & 40.00  & 7 & 1 & 5490 & 1098 & 1647 & Logistic & 0.1 & 100 & 182.2 \\
      \bottomrule
      \\
    \end{tabular}
    \mycaption{Experiment Protocols} {Experiment protocols for the different experiments presented in this work. \textbf{Feature Types}:
    Feature spaces used for the bilateral convolutions. Position$_1$ corresponds to un-normalized pixel positions whereas Position$_2$ corresponds
    to pixel positions normalized to $[0,1]$ with respect to the given image. \textbf{Feature Scales}: Cross-validated scales for the features used.
     \textbf{Filter Size}: Number of elements in the filter that is being learned. \textbf{Filter Nbr.}: Half-width of the filter. \textbf{Train},
     \textbf{Val.} and \textbf{Test} corresponds to the number of train, validation and test images used in the experiment. \textbf{Loss Type}: Type
     of loss used for back-propagation. ``MSE'' corresponds to Euclidean mean squared error loss and ``Logistic'' corresponds to multinomial logistic
     loss. ``Weighted Logistic'' is the class-weighted multinomial logistic loss. We weighted the loss with inverse class probability for material
     segmentation task due to the small availability of training data with class imbalance. \textbf{LR}: Fixed learning rate used in stochastic gradient
     descent. \textbf{Batch}: Number of images used in one parameter update step. \textbf{Epochs}: Number of training epochs. In all the experiments,
     we used fixed momentum of 0.9 and weight decay of 0.0005 for stochastic gradient descent. ```Color Upsampling'' experiments in this Table corresponds
     to those performed on Pascal VOC12 dataset images. For all experiments using Pascal VOC12 images, we use extended
     training segmentation dataset available from~\cite{hariharan2011moredata}, and used standard validation and test splits
     from the main dataset~\cite{voc2012segmentation}.}
  \label{tbl:parameters}
\end{table*}

\clearpage

\section{Parameters and Additional Results for Video Propagation Networks}

In this Section, we present experiment protocols and additional qualitative results for experiments
on video object segmentation, semantic video segmentation and video color
propagation. Table~\ref{tbl:parameters_supp} shows the feature scales and other parameters used in different experiments.
Figures~\ref{fig:video_seg_pos_supp} show some qualitative results on video object segmentation
with some failure cases in Fig.~\ref{fig:video_seg_neg_supp}.
Figure~\ref{fig:semantic_visuals_supp} shows some qualitative results on semantic video segmentation and
Fig.~\ref{fig:color_visuals_supp} shows results on video color propagation.

\newcolumntype{L}[1]{>{\raggedright\let\newline\\\arraybackslash\hspace{0pt}}b{#1}}
\newcolumntype{C}[1]{>{\centering\let\newline\\\arraybackslash\hspace{0pt}}b{#1}}
\newcolumntype{R}[1]{>{\raggedleft\let\newline\\\arraybackslash\hspace{0pt}}b{#1}}

\begin{table*}[h]
\tiny
  \centering
    \begin{tabular}{L{3.0cm} L{2.4cm} L{2.8cm} L{2.8cm} C{0.5cm} C{1.0cm} L{1.2cm}}
      \toprule
\textbf{Experiment} & \textbf{Feature Type} & \textbf{Feature Scale-1, $\Lambda_a$} & \textbf{Feature Scale-2, $\Lambda_b$} & \textbf{$\alpha$} & \textbf{Input Frames} & \textbf{Loss Type} \\
      \midrule
      \textbf{Video Object Segmentation} & ($x,y,Y,Cb,Cr,t$) & (0.02,0.02,0.07,0.4,0.4,0.01) & (0.03,0.03,0.09,0.5,0.5,0.2) & 0.5 & 9 & Logistic\\
      \midrule
      \textbf{Semantic Video Segmentation} & & & & & \\
      \textbf{with CNN1~\cite{yu2015multi}-NoFlow} & ($x,y,R,G,B,t$) & (0.08,0.08,0.2,0.2,0.2,0.04) & (0.11,0.11,0.2,0.2,0.2,0.04) & 0.5 & 3 & Logistic \\
      \textbf{with CNN1~\cite{yu2015multi}-Flow} & ($x+u_x,y+u_y,R,G,B,t$) & (0.11,0.11,0.14,0.14,0.14,0.03) & (0.08,0.08,0.12,0.12,0.12,0.01) & 0.65 & 3 & Logistic\\
      \textbf{with CNN2~\cite{richter2016playing}-Flow} & ($x+u_x,y+u_y,R,G,B,t$) & (0.08,0.08,0.2,0.2,0.2,0.04) & (0.09,0.09,0.25,0.25,0.25,0.03) & 0.5 & 4 & Logistic\\
      \midrule
      \textbf{Video Color Propagation} & ($x,y,I,t$)  & (0.04,0.04,0.2,0.04) & No second kernel & 1 & 4 & MSE\\
      \bottomrule
      \\
    \end{tabular}
    \mycaption{Experiment Protocols} {Experiment protocols for the different experiments presented in this work. \textbf{Feature Types}:
    Feature spaces used for the bilateral convolutions, with position ($x,y$) and color
    ($R,G,B$ or $Y,Cb,Cr$) features $\in [0,255]$. $u_x$, $u_y$ denotes optical flow with respect
    to the present frame and $I$ denotes grayscale intensity.
    \textbf{Feature Scales ($\Lambda_a, \Lambda_b$)}: Cross-validated scales for the features used.
    \textbf{$\alpha$}: Exponential time decay for the input frames.
    \textbf{Input Frames}: Number of input frames for VPN.
    \textbf{Loss Type}: Type
     of loss used for back-propagation. ``MSE'' corresponds to Euclidean mean squared error loss and ``Logistic'' corresponds to multinomial logistic loss.}
  \label{tbl:parameters_supp}
\end{table*}

% \begin{figure}[th!]
% \begin{center}
%   \centerline{\includegraphics[width=\textwidth]{figures/video_seg_visuals_supp_small.pdf}}
%     \mycaption{Video Object Segmentation}
%     {Shown are the different frames in example videos with the corresponding
%     ground truth (GT) masks, predictions from BVS~\cite{marki2016bilateral},
%     OFL~\cite{tsaivideo}, VPN (VPN-Stage2) and VPN-DLab (VPN-DeepLab) models.}
%     \label{fig:video_seg_small_supp}
% \end{center}
% \vspace{-1.0cm}
% \end{figure}

\begin{figure}[th!]
\begin{center}
  \centerline{\includegraphics[width=0.7\textwidth]{figures/video_seg_visuals_supp_positive.pdf}}
    \mycaption{Video Object Segmentation}
    {Shown are the different frames in example videos with the corresponding
    ground truth (GT) masks, predictions from BVS~\cite{marki2016bilateral},
    OFL~\cite{tsaivideo}, VPN (VPN-Stage2) and VPN-DLab (VPN-DeepLab) models.}
    \label{fig:video_seg_pos_supp}
\end{center}
\vspace{-1.0cm}
\end{figure}

\begin{figure}[th!]
\begin{center}
  \centerline{\includegraphics[width=0.7\textwidth]{figures/video_seg_visuals_supp_negative.pdf}}
    \mycaption{Failure Cases for Video Object Segmentation}
    {Shown are the different frames in example videos with the corresponding
    ground truth (GT) masks, predictions from BVS~\cite{marki2016bilateral},
    OFL~\cite{tsaivideo}, VPN (VPN-Stage2) and VPN-DLab (VPN-DeepLab) models.}
    \label{fig:video_seg_neg_supp}
\end{center}
\vspace{-1.0cm}
\end{figure}

\begin{figure}[th!]
\begin{center}
  \centerline{\includegraphics[width=0.9\textwidth]{figures/supp_semantic_visual.pdf}}
    \mycaption{Semantic Video Segmentation}
    {Input video frames and the corresponding ground truth (GT)
    segmentation together with the predictions of CNN~\cite{yu2015multi} and with
    VPN-Flow.}
    \label{fig:semantic_visuals_supp}
\end{center}
\vspace{-0.7cm}
\end{figure}

\begin{figure}[th!]
\begin{center}
  \centerline{\includegraphics[width=\textwidth]{figures/colorization_visuals_supp.pdf}}
  \mycaption{Video Color Propagation}
  {Input grayscale video frames and corresponding ground-truth (GT) color images
  together with color predictions of Levin et al.~\cite{levin2004colorization} and VPN-Stage1 models.}
  \label{fig:color_visuals_supp}
\end{center}
\vspace{-0.7cm}
\end{figure}

\clearpage

\section{Additional Material for Bilateral Inception Networks}
\label{sec:binception-app}

In this section of the Appendix, we first discuss the use of approximate bilateral
filtering in BI modules (Sec.~\ref{sec:lattice}).
Later, we present some qualitative results using different models for the approach presented in
Chapter~\ref{chap:binception} (Sec.~\ref{sec:qualitative-app}).

\subsection{Approximate Bilateral Filtering}
\label{sec:lattice}

The bilateral inception module presented in Chapter~\ref{chap:binception} computes a matrix-vector
product between a Gaussian filter $K$ and a vector of activations $\bz_c$.
Bilateral filtering is an important operation and many algorithmic techniques have been
proposed to speed-up this operation~\cite{paris2006fast,adams2010fast,gastal2011domain}.
In the main paper we opted to implement what can be considered the
brute-force variant of explicitly constructing $K$ and then using BLAS to compute the
matrix-vector product. This resulted in a few millisecond operation.
The explicit way to compute is possible due to the
reduction to super-pixels, e.g., it would not work for DenseCRF variants
that operate on the full image resolution.

Here, we present experiments where we use the fast approximate bilateral filtering
algorithm of~\cite{adams2010fast}, which is also used in Chapter~\ref{chap:bnn}
for learning sparse high dimensional filters. This
choice allows for larger dimensions of matrix-vector multiplication. The reason for choosing
the explicit multiplication in Chapter~\ref{chap:binception} was that it was computationally faster.
For the small sizes of the involved matrices and vectors, the explicit computation is sufficient and we had no
GPU implementation of an approximate technique that matched this runtime. Also it
is conceptually easier and the gradient to the feature transformations ($\Lambda \mathbf{f}$) is
obtained using standard matrix calculus.

\subsubsection{Experiments}

We modified the existing segmentation architectures analogous to those in Chapter~\ref{chap:binception}.
The main difference is that, here, the inception modules use the lattice
approximation~\cite{adams2010fast} to compute the bilateral filtering.
Using the lattice approximation did not allow us to back-propagate through feature transformations ($\Lambda$)
and thus we used hand-specified feature scales as will be explained later.
Specifically, we take CNN architectures from the works
of~\cite{chen2014semantic,zheng2015conditional,bell2015minc} and insert the BI modules between
the spatial FC layers.
We use superpixels from~\cite{DollarICCV13edges}
for all the experiments with the lattice approximation. Experiments are
performed using Caffe neural network framework~\cite{jia2014caffe}.

\begin{table}
  \small
  \centering
  \begin{tabular}{p{5.5cm}>{\raggedright\arraybackslash}p{1.4cm}>{\centering\arraybackslash}p{2.2cm}}
    \toprule
		\textbf{Model} & \emph{IoU} & \emph{Runtime}(ms) \\
    \midrule

    %%%%%%%%%%%% Scores computed by us)%%%%%%%%%%%%
		\deeplablargefov & 68.9 & 145ms\\
    \midrule
    \bi{7}{2}-\bi{8}{10}& \textbf{73.8} & +600 \\
    \midrule
    \deeplablargefovcrf~\cite{chen2014semantic} & 72.7 & +830\\
    \deeplabmsclargefovcrf~\cite{chen2014semantic} & \textbf{73.6} & +880\\
    DeepLab-EdgeNet~\cite{chen2015semantic} & 71.7 & +30\\
    DeepLab-EdgeNet-CRF~\cite{chen2015semantic} & \textbf{73.6} & +860\\
  \bottomrule \\
  \end{tabular}
  \mycaption{Semantic Segmentation using the DeepLab model}
  {IoU scores on the Pascal VOC12 segmentation test dataset
  with different models and our modified inception model.
  Also shown are the corresponding runtimes in milliseconds. Runtimes
  also include superpixel computations (300 ms with Dollar superpixels~\cite{DollarICCV13edges})}
  \label{tab:largefovresults}
\end{table}

\paragraph{Semantic Segmentation}
The experiments in this section use the Pascal VOC12 segmentation dataset~\cite{voc2012segmentation} with 21 object classes and the images have a maximum resolution of 0.25 megapixels.
For all experiments on VOC12, we train using the extended training set of
10581 images collected by~\cite{hariharan2011moredata}.
We modified the \deeplab~network architecture of~\cite{chen2014semantic} and
the CRFasRNN architecture from~\cite{zheng2015conditional} which uses a CNN with
deconvolution layers followed by DenseCRF trained end-to-end.

\paragraph{DeepLab Model}\label{sec:deeplabmodel}
We experimented with the \bi{7}{2}-\bi{8}{10} inception model.
Results using the~\deeplab~model are summarized in Tab.~\ref{tab:largefovresults}.
Although we get similar improvements with inception modules as with the
explicit kernel computation, using lattice approximation is slower.

\begin{table}
  \small
  \centering
  \begin{tabular}{p{6.4cm}>{\raggedright\arraybackslash}p{1.8cm}>{\raggedright\arraybackslash}p{1.8cm}}
    \toprule
    \textbf{Model} & \emph{IoU (Val)} & \emph{IoU (Test)}\\
    \midrule
    %%%%%%%%%%%% Scores computed by us)%%%%%%%%%%%%
    CNN &  67.5 & - \\
    \deconv (CNN+Deconvolutions) & 69.8 & 72.0 \\
    \midrule
    \bi{3}{6}-\bi{4}{6}-\bi{7}{2}-\bi{8}{6}& 71.9 & - \\
    \bi{3}{6}-\bi{4}{6}-\bi{7}{2}-\bi{8}{6}-\gi{6}& 73.6 &  \href{http://host.robots.ox.ac.uk:8080/anonymous/VOTV5E.html}{\textbf{75.2}}\\
    \midrule
    \deconvcrf (CRF-RNN)~\cite{zheng2015conditional} & 73.0 & 74.7\\
    Context-CRF-RNN~\cite{yu2015multi} & ~~ - ~ & \textbf{75.3} \\
    \bottomrule \\
  \end{tabular}
  \mycaption{Semantic Segmentation using the CRFasRNN model}{IoU score corresponding to different models
  on Pascal VOC12 reduced validation / test segmentation dataset. The reduced validation set consists of 346 images
  as used in~\cite{zheng2015conditional} where we adapted the model from.}
  \label{tab:deconvresults-app}
\end{table}

\paragraph{CRFasRNN Model}\label{sec:deepinception}
We add BI modules after score-pool3, score-pool4, \fc{7} and \fc{8} $1\times1$ convolution layers
resulting in the \bi{3}{6}-\bi{4}{6}-\bi{7}{2}-\bi{8}{6}
model and also experimented with another variant where $BI_8$ is followed by another inception
module, G$(6)$, with 6 Gaussian kernels.
Note that here also we discarded both deconvolution and DenseCRF parts of the original model~\cite{zheng2015conditional}
and inserted the BI modules in the base CNN and found similar improvements compared to the inception modules with explicit
kernel computaion. See Tab.~\ref{tab:deconvresults-app} for results on the CRFasRNN model.

\paragraph{Material Segmentation}
Table~\ref{tab:mincresults-app} shows the results on the MINC dataset~\cite{bell2015minc}
obtained by modifying the AlexNet architecture with our inception modules. We observe
similar improvements as with explicit kernel construction.
For this model, we do not provide any learned setup due to very limited segment training
data. The weights to combine outputs in the bilateral inception layer are
found by validation on the validation set.

\begin{table}[t]
  \small
  \centering
  \begin{tabular}{p{3.5cm}>{\centering\arraybackslash}p{4.0cm}}
    \toprule
    \textbf{Model} & Class / Total accuracy\\
    \midrule

    %%%%%%%%%%%% Scores computed by us)%%%%%%%%%%%%
    AlexNet CNN & 55.3 / 58.9 \\
    \midrule
    \bi{7}{2}-\bi{8}{6}& 68.5 / 71.8 \\
    \bi{7}{2}-\bi{8}{6}-G$(6)$& 67.6 / 73.1 \\
    \midrule
    AlexNet-CRF & 65.5 / 71.0 \\
    \bottomrule \\
  \end{tabular}
  \mycaption{Material Segmentation using AlexNet}{Pixel accuracy of different models on
  the MINC material segmentation test dataset~\cite{bell2015minc}.}
  \label{tab:mincresults-app}
\end{table}

\paragraph{Scales of Bilateral Inception Modules}
\label{sec:scales}

Unlike the explicit kernel technique presented in the main text (Chapter~\ref{chap:binception}),
we didn't back-propagate through feature transformation ($\Lambda$)
using the approximate bilateral filter technique.
So, the feature scales are hand-specified and validated, which are as follows.
The optimal scale values for the \bi{7}{2}-\bi{8}{2} model are found by validation for the best performance which are
$\sigma_{xy}$ = (0.1, 0.1) for the spatial (XY) kernel and $\sigma_{rgbxy}$ = (0.1, 0.1, 0.1, 0.01, 0.01) for color and position (RGBXY)  kernel.
Next, as more kernels are added to \bi{8}{2}, we set scales to be $\alpha$*($\sigma_{xy}$, $\sigma_{rgbxy}$).
The value of $\alpha$ is chosen as  1, 0.5, 0.1, 0.05, 0.1, at uniform interval, for the \bi{8}{10} bilateral inception module.


\subsection{Qualitative Results}
\label{sec:qualitative-app}

In this section, we present more qualitative results obtained using the BI module with explicit
kernel computation technique presented in Chapter~\ref{chap:binception}. Results on the Pascal VOC12
dataset~\cite{voc2012segmentation} using the DeepLab-LargeFOV model are shown in Fig.~\ref{fig:semantic_visuals-app},
followed by the results on MINC dataset~\cite{bell2015minc}
in Fig.~\ref{fig:material_visuals-app} and on
Cityscapes dataset~\cite{Cordts2015Cvprw} in Fig.~\ref{fig:street_visuals-app}.


\definecolor{voc_1}{RGB}{0, 0, 0}
\definecolor{voc_2}{RGB}{128, 0, 0}
\definecolor{voc_3}{RGB}{0, 128, 0}
\definecolor{voc_4}{RGB}{128, 128, 0}
\definecolor{voc_5}{RGB}{0, 0, 128}
\definecolor{voc_6}{RGB}{128, 0, 128}
\definecolor{voc_7}{RGB}{0, 128, 128}
\definecolor{voc_8}{RGB}{128, 128, 128}
\definecolor{voc_9}{RGB}{64, 0, 0}
\definecolor{voc_10}{RGB}{192, 0, 0}
\definecolor{voc_11}{RGB}{64, 128, 0}
\definecolor{voc_12}{RGB}{192, 128, 0}
\definecolor{voc_13}{RGB}{64, 0, 128}
\definecolor{voc_14}{RGB}{192, 0, 128}
\definecolor{voc_15}{RGB}{64, 128, 128}
\definecolor{voc_16}{RGB}{192, 128, 128}
\definecolor{voc_17}{RGB}{0, 64, 0}
\definecolor{voc_18}{RGB}{128, 64, 0}
\definecolor{voc_19}{RGB}{0, 192, 0}
\definecolor{voc_20}{RGB}{128, 192, 0}
\definecolor{voc_21}{RGB}{0, 64, 128}
\definecolor{voc_22}{RGB}{128, 64, 128}

\begin{figure*}[!ht]
  \small
  \centering
  \fcolorbox{white}{voc_1}{\rule{0pt}{4pt}\rule{4pt}{0pt}} Background~~
  \fcolorbox{white}{voc_2}{\rule{0pt}{4pt}\rule{4pt}{0pt}} Aeroplane~~
  \fcolorbox{white}{voc_3}{\rule{0pt}{4pt}\rule{4pt}{0pt}} Bicycle~~
  \fcolorbox{white}{voc_4}{\rule{0pt}{4pt}\rule{4pt}{0pt}} Bird~~
  \fcolorbox{white}{voc_5}{\rule{0pt}{4pt}\rule{4pt}{0pt}} Boat~~
  \fcolorbox{white}{voc_6}{\rule{0pt}{4pt}\rule{4pt}{0pt}} Bottle~~
  \fcolorbox{white}{voc_7}{\rule{0pt}{4pt}\rule{4pt}{0pt}} Bus~~
  \fcolorbox{white}{voc_8}{\rule{0pt}{4pt}\rule{4pt}{0pt}} Car~~\\
  \fcolorbox{white}{voc_9}{\rule{0pt}{4pt}\rule{4pt}{0pt}} Cat~~
  \fcolorbox{white}{voc_10}{\rule{0pt}{4pt}\rule{4pt}{0pt}} Chair~~
  \fcolorbox{white}{voc_11}{\rule{0pt}{4pt}\rule{4pt}{0pt}} Cow~~
  \fcolorbox{white}{voc_12}{\rule{0pt}{4pt}\rule{4pt}{0pt}} Dining Table~~
  \fcolorbox{white}{voc_13}{\rule{0pt}{4pt}\rule{4pt}{0pt}} Dog~~
  \fcolorbox{white}{voc_14}{\rule{0pt}{4pt}\rule{4pt}{0pt}} Horse~~
  \fcolorbox{white}{voc_15}{\rule{0pt}{4pt}\rule{4pt}{0pt}} Motorbike~~
  \fcolorbox{white}{voc_16}{\rule{0pt}{4pt}\rule{4pt}{0pt}} Person~~\\
  \fcolorbox{white}{voc_17}{\rule{0pt}{4pt}\rule{4pt}{0pt}} Potted Plant~~
  \fcolorbox{white}{voc_18}{\rule{0pt}{4pt}\rule{4pt}{0pt}} Sheep~~
  \fcolorbox{white}{voc_19}{\rule{0pt}{4pt}\rule{4pt}{0pt}} Sofa~~
  \fcolorbox{white}{voc_20}{\rule{0pt}{4pt}\rule{4pt}{0pt}} Train~~
  \fcolorbox{white}{voc_21}{\rule{0pt}{4pt}\rule{4pt}{0pt}} TV monitor~~\\


  \subfigure{%
    \includegraphics[width=.15\columnwidth]{figures/supplementary/2008_001308_given.png}
  }
  \subfigure{%
    \includegraphics[width=.15\columnwidth]{figures/supplementary/2008_001308_sp.png}
  }
  \subfigure{%
    \includegraphics[width=.15\columnwidth]{figures/supplementary/2008_001308_gt.png}
  }
  \subfigure{%
    \includegraphics[width=.15\columnwidth]{figures/supplementary/2008_001308_cnn.png}
  }
  \subfigure{%
    \includegraphics[width=.15\columnwidth]{figures/supplementary/2008_001308_crf.png}
  }
  \subfigure{%
    \includegraphics[width=.15\columnwidth]{figures/supplementary/2008_001308_ours.png}
  }\\[-2ex]


  \subfigure{%
    \includegraphics[width=.15\columnwidth]{figures/supplementary/2008_001821_given.png}
  }
  \subfigure{%
    \includegraphics[width=.15\columnwidth]{figures/supplementary/2008_001821_sp.png}
  }
  \subfigure{%
    \includegraphics[width=.15\columnwidth]{figures/supplementary/2008_001821_gt.png}
  }
  \subfigure{%
    \includegraphics[width=.15\columnwidth]{figures/supplementary/2008_001821_cnn.png}
  }
  \subfigure{%
    \includegraphics[width=.15\columnwidth]{figures/supplementary/2008_001821_crf.png}
  }
  \subfigure{%
    \includegraphics[width=.15\columnwidth]{figures/supplementary/2008_001821_ours.png}
  }\\[-2ex]



  \subfigure{%
    \includegraphics[width=.15\columnwidth]{figures/supplementary/2008_004612_given.png}
  }
  \subfigure{%
    \includegraphics[width=.15\columnwidth]{figures/supplementary/2008_004612_sp.png}
  }
  \subfigure{%
    \includegraphics[width=.15\columnwidth]{figures/supplementary/2008_004612_gt.png}
  }
  \subfigure{%
    \includegraphics[width=.15\columnwidth]{figures/supplementary/2008_004612_cnn.png}
  }
  \subfigure{%
    \includegraphics[width=.15\columnwidth]{figures/supplementary/2008_004612_crf.png}
  }
  \subfigure{%
    \includegraphics[width=.15\columnwidth]{figures/supplementary/2008_004612_ours.png}
  }\\[-2ex]


  \subfigure{%
    \includegraphics[width=.15\columnwidth]{figures/supplementary/2009_001008_given.png}
  }
  \subfigure{%
    \includegraphics[width=.15\columnwidth]{figures/supplementary/2009_001008_sp.png}
  }
  \subfigure{%
    \includegraphics[width=.15\columnwidth]{figures/supplementary/2009_001008_gt.png}
  }
  \subfigure{%
    \includegraphics[width=.15\columnwidth]{figures/supplementary/2009_001008_cnn.png}
  }
  \subfigure{%
    \includegraphics[width=.15\columnwidth]{figures/supplementary/2009_001008_crf.png}
  }
  \subfigure{%
    \includegraphics[width=.15\columnwidth]{figures/supplementary/2009_001008_ours.png}
  }\\[-2ex]




  \subfigure{%
    \includegraphics[width=.15\columnwidth]{figures/supplementary/2009_004497_given.png}
  }
  \subfigure{%
    \includegraphics[width=.15\columnwidth]{figures/supplementary/2009_004497_sp.png}
  }
  \subfigure{%
    \includegraphics[width=.15\columnwidth]{figures/supplementary/2009_004497_gt.png}
  }
  \subfigure{%
    \includegraphics[width=.15\columnwidth]{figures/supplementary/2009_004497_cnn.png}
  }
  \subfigure{%
    \includegraphics[width=.15\columnwidth]{figures/supplementary/2009_004497_crf.png}
  }
  \subfigure{%
    \includegraphics[width=.15\columnwidth]{figures/supplementary/2009_004497_ours.png}
  }\\[-2ex]



  \setcounter{subfigure}{0}
  \subfigure[\scriptsize Input]{%
    \includegraphics[width=.15\columnwidth]{figures/supplementary/2010_001327_given.png}
  }
  \subfigure[\scriptsize Superpixels]{%
    \includegraphics[width=.15\columnwidth]{figures/supplementary/2010_001327_sp.png}
  }
  \subfigure[\scriptsize GT]{%
    \includegraphics[width=.15\columnwidth]{figures/supplementary/2010_001327_gt.png}
  }
  \subfigure[\scriptsize Deeplab]{%
    \includegraphics[width=.15\columnwidth]{figures/supplementary/2010_001327_cnn.png}
  }
  \subfigure[\scriptsize +DenseCRF]{%
    \includegraphics[width=.15\columnwidth]{figures/supplementary/2010_001327_crf.png}
  }
  \subfigure[\scriptsize Using BI]{%
    \includegraphics[width=.15\columnwidth]{figures/supplementary/2010_001327_ours.png}
  }
  \mycaption{Semantic Segmentation}{Example results of semantic segmentation
  on the Pascal VOC12 dataset.
  (d)~depicts the DeepLab CNN result, (e)~CNN + 10 steps of mean-field inference,
  (f~result obtained with bilateral inception (BI) modules (\bi{6}{2}+\bi{7}{6}) between \fc~layers.}
  \label{fig:semantic_visuals-app}
\end{figure*}


\definecolor{minc_1}{HTML}{771111}
\definecolor{minc_2}{HTML}{CAC690}
\definecolor{minc_3}{HTML}{EEEEEE}
\definecolor{minc_4}{HTML}{7C8FA6}
\definecolor{minc_5}{HTML}{597D31}
\definecolor{minc_6}{HTML}{104410}
\definecolor{minc_7}{HTML}{BB819C}
\definecolor{minc_8}{HTML}{D0CE48}
\definecolor{minc_9}{HTML}{622745}
\definecolor{minc_10}{HTML}{666666}
\definecolor{minc_11}{HTML}{D54A31}
\definecolor{minc_12}{HTML}{101044}
\definecolor{minc_13}{HTML}{444126}
\definecolor{minc_14}{HTML}{75D646}
\definecolor{minc_15}{HTML}{DD4348}
\definecolor{minc_16}{HTML}{5C8577}
\definecolor{minc_17}{HTML}{C78472}
\definecolor{minc_18}{HTML}{75D6D0}
\definecolor{minc_19}{HTML}{5B4586}
\definecolor{minc_20}{HTML}{C04393}
\definecolor{minc_21}{HTML}{D69948}
\definecolor{minc_22}{HTML}{7370D8}
\definecolor{minc_23}{HTML}{7A3622}
\definecolor{minc_24}{HTML}{000000}

\begin{figure*}[!ht]
  \small % scriptsize
  \centering
  \fcolorbox{white}{minc_1}{\rule{0pt}{4pt}\rule{4pt}{0pt}} Brick~~
  \fcolorbox{white}{minc_2}{\rule{0pt}{4pt}\rule{4pt}{0pt}} Carpet~~
  \fcolorbox{white}{minc_3}{\rule{0pt}{4pt}\rule{4pt}{0pt}} Ceramic~~
  \fcolorbox{white}{minc_4}{\rule{0pt}{4pt}\rule{4pt}{0pt}} Fabric~~
  \fcolorbox{white}{minc_5}{\rule{0pt}{4pt}\rule{4pt}{0pt}} Foliage~~
  \fcolorbox{white}{minc_6}{\rule{0pt}{4pt}\rule{4pt}{0pt}} Food~~
  \fcolorbox{white}{minc_7}{\rule{0pt}{4pt}\rule{4pt}{0pt}} Glass~~
  \fcolorbox{white}{minc_8}{\rule{0pt}{4pt}\rule{4pt}{0pt}} Hair~~\\
  \fcolorbox{white}{minc_9}{\rule{0pt}{4pt}\rule{4pt}{0pt}} Leather~~
  \fcolorbox{white}{minc_10}{\rule{0pt}{4pt}\rule{4pt}{0pt}} Metal~~
  \fcolorbox{white}{minc_11}{\rule{0pt}{4pt}\rule{4pt}{0pt}} Mirror~~
  \fcolorbox{white}{minc_12}{\rule{0pt}{4pt}\rule{4pt}{0pt}} Other~~
  \fcolorbox{white}{minc_13}{\rule{0pt}{4pt}\rule{4pt}{0pt}} Painted~~
  \fcolorbox{white}{minc_14}{\rule{0pt}{4pt}\rule{4pt}{0pt}} Paper~~
  \fcolorbox{white}{minc_15}{\rule{0pt}{4pt}\rule{4pt}{0pt}} Plastic~~\\
  \fcolorbox{white}{minc_16}{\rule{0pt}{4pt}\rule{4pt}{0pt}} Polished Stone~~
  \fcolorbox{white}{minc_17}{\rule{0pt}{4pt}\rule{4pt}{0pt}} Skin~~
  \fcolorbox{white}{minc_18}{\rule{0pt}{4pt}\rule{4pt}{0pt}} Sky~~
  \fcolorbox{white}{minc_19}{\rule{0pt}{4pt}\rule{4pt}{0pt}} Stone~~
  \fcolorbox{white}{minc_20}{\rule{0pt}{4pt}\rule{4pt}{0pt}} Tile~~
  \fcolorbox{white}{minc_21}{\rule{0pt}{4pt}\rule{4pt}{0pt}} Wallpaper~~
  \fcolorbox{white}{minc_22}{\rule{0pt}{4pt}\rule{4pt}{0pt}} Water~~
  \fcolorbox{white}{minc_23}{\rule{0pt}{4pt}\rule{4pt}{0pt}} Wood~~\\
  \subfigure{%
    \includegraphics[width=.15\columnwidth]{figures/supplementary/000008468_given.png}
  }
  \subfigure{%
    \includegraphics[width=.15\columnwidth]{figures/supplementary/000008468_sp.png}
  }
  \subfigure{%
    \includegraphics[width=.15\columnwidth]{figures/supplementary/000008468_gt.png}
  }
  \subfigure{%
    \includegraphics[width=.15\columnwidth]{figures/supplementary/000008468_cnn.png}
  }
  \subfigure{%
    \includegraphics[width=.15\columnwidth]{figures/supplementary/000008468_crf.png}
  }
  \subfigure{%
    \includegraphics[width=.15\columnwidth]{figures/supplementary/000008468_ours.png}
  }\\[-2ex]

  \subfigure{%
    \includegraphics[width=.15\columnwidth]{figures/supplementary/000009053_given.png}
  }
  \subfigure{%
    \includegraphics[width=.15\columnwidth]{figures/supplementary/000009053_sp.png}
  }
  \subfigure{%
    \includegraphics[width=.15\columnwidth]{figures/supplementary/000009053_gt.png}
  }
  \subfigure{%
    \includegraphics[width=.15\columnwidth]{figures/supplementary/000009053_cnn.png}
  }
  \subfigure{%
    \includegraphics[width=.15\columnwidth]{figures/supplementary/000009053_crf.png}
  }
  \subfigure{%
    \includegraphics[width=.15\columnwidth]{figures/supplementary/000009053_ours.png}
  }\\[-2ex]




  \subfigure{%
    \includegraphics[width=.15\columnwidth]{figures/supplementary/000014977_given.png}
  }
  \subfigure{%
    \includegraphics[width=.15\columnwidth]{figures/supplementary/000014977_sp.png}
  }
  \subfigure{%
    \includegraphics[width=.15\columnwidth]{figures/supplementary/000014977_gt.png}
  }
  \subfigure{%
    \includegraphics[width=.15\columnwidth]{figures/supplementary/000014977_cnn.png}
  }
  \subfigure{%
    \includegraphics[width=.15\columnwidth]{figures/supplementary/000014977_crf.png}
  }
  \subfigure{%
    \includegraphics[width=.15\columnwidth]{figures/supplementary/000014977_ours.png}
  }\\[-2ex]


  \subfigure{%
    \includegraphics[width=.15\columnwidth]{figures/supplementary/000022922_given.png}
  }
  \subfigure{%
    \includegraphics[width=.15\columnwidth]{figures/supplementary/000022922_sp.png}
  }
  \subfigure{%
    \includegraphics[width=.15\columnwidth]{figures/supplementary/000022922_gt.png}
  }
  \subfigure{%
    \includegraphics[width=.15\columnwidth]{figures/supplementary/000022922_cnn.png}
  }
  \subfigure{%
    \includegraphics[width=.15\columnwidth]{figures/supplementary/000022922_crf.png}
  }
  \subfigure{%
    \includegraphics[width=.15\columnwidth]{figures/supplementary/000022922_ours.png}
  }\\[-2ex]


  \subfigure{%
    \includegraphics[width=.15\columnwidth]{figures/supplementary/000025711_given.png}
  }
  \subfigure{%
    \includegraphics[width=.15\columnwidth]{figures/supplementary/000025711_sp.png}
  }
  \subfigure{%
    \includegraphics[width=.15\columnwidth]{figures/supplementary/000025711_gt.png}
  }
  \subfigure{%
    \includegraphics[width=.15\columnwidth]{figures/supplementary/000025711_cnn.png}
  }
  \subfigure{%
    \includegraphics[width=.15\columnwidth]{figures/supplementary/000025711_crf.png}
  }
  \subfigure{%
    \includegraphics[width=.15\columnwidth]{figures/supplementary/000025711_ours.png}
  }\\[-2ex]


  \subfigure{%
    \includegraphics[width=.15\columnwidth]{figures/supplementary/000034473_given.png}
  }
  \subfigure{%
    \includegraphics[width=.15\columnwidth]{figures/supplementary/000034473_sp.png}
  }
  \subfigure{%
    \includegraphics[width=.15\columnwidth]{figures/supplementary/000034473_gt.png}
  }
  \subfigure{%
    \includegraphics[width=.15\columnwidth]{figures/supplementary/000034473_cnn.png}
  }
  \subfigure{%
    \includegraphics[width=.15\columnwidth]{figures/supplementary/000034473_crf.png}
  }
  \subfigure{%
    \includegraphics[width=.15\columnwidth]{figures/supplementary/000034473_ours.png}
  }\\[-2ex]


  \subfigure{%
    \includegraphics[width=.15\columnwidth]{figures/supplementary/000035463_given.png}
  }
  \subfigure{%
    \includegraphics[width=.15\columnwidth]{figures/supplementary/000035463_sp.png}
  }
  \subfigure{%
    \includegraphics[width=.15\columnwidth]{figures/supplementary/000035463_gt.png}
  }
  \subfigure{%
    \includegraphics[width=.15\columnwidth]{figures/supplementary/000035463_cnn.png}
  }
  \subfigure{%
    \includegraphics[width=.15\columnwidth]{figures/supplementary/000035463_crf.png}
  }
  \subfigure{%
    \includegraphics[width=.15\columnwidth]{figures/supplementary/000035463_ours.png}
  }\\[-2ex]


  \setcounter{subfigure}{0}
  \subfigure[\scriptsize Input]{%
    \includegraphics[width=.15\columnwidth]{figures/supplementary/000035993_given.png}
  }
  \subfigure[\scriptsize Superpixels]{%
    \includegraphics[width=.15\columnwidth]{figures/supplementary/000035993_sp.png}
  }
  \subfigure[\scriptsize GT]{%
    \includegraphics[width=.15\columnwidth]{figures/supplementary/000035993_gt.png}
  }
  \subfigure[\scriptsize AlexNet]{%
    \includegraphics[width=.15\columnwidth]{figures/supplementary/000035993_cnn.png}
  }
  \subfigure[\scriptsize +DenseCRF]{%
    \includegraphics[width=.15\columnwidth]{figures/supplementary/000035993_crf.png}
  }
  \subfigure[\scriptsize Using BI]{%
    \includegraphics[width=.15\columnwidth]{figures/supplementary/000035993_ours.png}
  }
  \mycaption{Material Segmentation}{Example results of material segmentation.
  (d)~depicts the AlexNet CNN result, (e)~CNN + 10 steps of mean-field inference,
  (f)~result obtained with bilateral inception (BI) modules (\bi{7}{2}+\bi{8}{6}) between
  \fc~layers.}
\label{fig:material_visuals-app}
\end{figure*}


\definecolor{city_1}{RGB}{128, 64, 128}
\definecolor{city_2}{RGB}{244, 35, 232}
\definecolor{city_3}{RGB}{70, 70, 70}
\definecolor{city_4}{RGB}{102, 102, 156}
\definecolor{city_5}{RGB}{190, 153, 153}
\definecolor{city_6}{RGB}{153, 153, 153}
\definecolor{city_7}{RGB}{250, 170, 30}
\definecolor{city_8}{RGB}{220, 220, 0}
\definecolor{city_9}{RGB}{107, 142, 35}
\definecolor{city_10}{RGB}{152, 251, 152}
\definecolor{city_11}{RGB}{70, 130, 180}
\definecolor{city_12}{RGB}{220, 20, 60}
\definecolor{city_13}{RGB}{255, 0, 0}
\definecolor{city_14}{RGB}{0, 0, 142}
\definecolor{city_15}{RGB}{0, 0, 70}
\definecolor{city_16}{RGB}{0, 60, 100}
\definecolor{city_17}{RGB}{0, 80, 100}
\definecolor{city_18}{RGB}{0, 0, 230}
\definecolor{city_19}{RGB}{119, 11, 32}
\begin{figure*}[!ht]
  \small % scriptsize
  \centering


  \subfigure{%
    \includegraphics[width=.18\columnwidth]{figures/supplementary/frankfurt00000_016005_given.png}
  }
  \subfigure{%
    \includegraphics[width=.18\columnwidth]{figures/supplementary/frankfurt00000_016005_sp.png}
  }
  \subfigure{%
    \includegraphics[width=.18\columnwidth]{figures/supplementary/frankfurt00000_016005_gt.png}
  }
  \subfigure{%
    \includegraphics[width=.18\columnwidth]{figures/supplementary/frankfurt00000_016005_cnn.png}
  }
  \subfigure{%
    \includegraphics[width=.18\columnwidth]{figures/supplementary/frankfurt00000_016005_ours.png}
  }\\[-2ex]

  \subfigure{%
    \includegraphics[width=.18\columnwidth]{figures/supplementary/frankfurt00000_004617_given.png}
  }
  \subfigure{%
    \includegraphics[width=.18\columnwidth]{figures/supplementary/frankfurt00000_004617_sp.png}
  }
  \subfigure{%
    \includegraphics[width=.18\columnwidth]{figures/supplementary/frankfurt00000_004617_gt.png}
  }
  \subfigure{%
    \includegraphics[width=.18\columnwidth]{figures/supplementary/frankfurt00000_004617_cnn.png}
  }
  \subfigure{%
    \includegraphics[width=.18\columnwidth]{figures/supplementary/frankfurt00000_004617_ours.png}
  }\\[-2ex]

  \subfigure{%
    \includegraphics[width=.18\columnwidth]{figures/supplementary/frankfurt00000_020880_given.png}
  }
  \subfigure{%
    \includegraphics[width=.18\columnwidth]{figures/supplementary/frankfurt00000_020880_sp.png}
  }
  \subfigure{%
    \includegraphics[width=.18\columnwidth]{figures/supplementary/frankfurt00000_020880_gt.png}
  }
  \subfigure{%
    \includegraphics[width=.18\columnwidth]{figures/supplementary/frankfurt00000_020880_cnn.png}
  }
  \subfigure{%
    \includegraphics[width=.18\columnwidth]{figures/supplementary/frankfurt00000_020880_ours.png}
  }\\[-2ex]



  \subfigure{%
    \includegraphics[width=.18\columnwidth]{figures/supplementary/frankfurt00001_007285_given.png}
  }
  \subfigure{%
    \includegraphics[width=.18\columnwidth]{figures/supplementary/frankfurt00001_007285_sp.png}
  }
  \subfigure{%
    \includegraphics[width=.18\columnwidth]{figures/supplementary/frankfurt00001_007285_gt.png}
  }
  \subfigure{%
    \includegraphics[width=.18\columnwidth]{figures/supplementary/frankfurt00001_007285_cnn.png}
  }
  \subfigure{%
    \includegraphics[width=.18\columnwidth]{figures/supplementary/frankfurt00001_007285_ours.png}
  }\\[-2ex]


  \subfigure{%
    \includegraphics[width=.18\columnwidth]{figures/supplementary/frankfurt00001_059789_given.png}
  }
  \subfigure{%
    \includegraphics[width=.18\columnwidth]{figures/supplementary/frankfurt00001_059789_sp.png}
  }
  \subfigure{%
    \includegraphics[width=.18\columnwidth]{figures/supplementary/frankfurt00001_059789_gt.png}
  }
  \subfigure{%
    \includegraphics[width=.18\columnwidth]{figures/supplementary/frankfurt00001_059789_cnn.png}
  }
  \subfigure{%
    \includegraphics[width=.18\columnwidth]{figures/supplementary/frankfurt00001_059789_ours.png}
  }\\[-2ex]


  \subfigure{%
    \includegraphics[width=.18\columnwidth]{figures/supplementary/frankfurt00001_068208_given.png}
  }
  \subfigure{%
    \includegraphics[width=.18\columnwidth]{figures/supplementary/frankfurt00001_068208_sp.png}
  }
  \subfigure{%
    \includegraphics[width=.18\columnwidth]{figures/supplementary/frankfurt00001_068208_gt.png}
  }
  \subfigure{%
    \includegraphics[width=.18\columnwidth]{figures/supplementary/frankfurt00001_068208_cnn.png}
  }
  \subfigure{%
    \includegraphics[width=.18\columnwidth]{figures/supplementary/frankfurt00001_068208_ours.png}
  }\\[-2ex]

  \subfigure{%
    \includegraphics[width=.18\columnwidth]{figures/supplementary/frankfurt00001_082466_given.png}
  }
  \subfigure{%
    \includegraphics[width=.18\columnwidth]{figures/supplementary/frankfurt00001_082466_sp.png}
  }
  \subfigure{%
    \includegraphics[width=.18\columnwidth]{figures/supplementary/frankfurt00001_082466_gt.png}
  }
  \subfigure{%
    \includegraphics[width=.18\columnwidth]{figures/supplementary/frankfurt00001_082466_cnn.png}
  }
  \subfigure{%
    \includegraphics[width=.18\columnwidth]{figures/supplementary/frankfurt00001_082466_ours.png}
  }\\[-2ex]

  \subfigure{%
    \includegraphics[width=.18\columnwidth]{figures/supplementary/lindau00033_000019_given.png}
  }
  \subfigure{%
    \includegraphics[width=.18\columnwidth]{figures/supplementary/lindau00033_000019_sp.png}
  }
  \subfigure{%
    \includegraphics[width=.18\columnwidth]{figures/supplementary/lindau00033_000019_gt.png}
  }
  \subfigure{%
    \includegraphics[width=.18\columnwidth]{figures/supplementary/lindau00033_000019_cnn.png}
  }
  \subfigure{%
    \includegraphics[width=.18\columnwidth]{figures/supplementary/lindau00033_000019_ours.png}
  }\\[-2ex]

  \subfigure{%
    \includegraphics[width=.18\columnwidth]{figures/supplementary/lindau00052_000019_given.png}
  }
  \subfigure{%
    \includegraphics[width=.18\columnwidth]{figures/supplementary/lindau00052_000019_sp.png}
  }
  \subfigure{%
    \includegraphics[width=.18\columnwidth]{figures/supplementary/lindau00052_000019_gt.png}
  }
  \subfigure{%
    \includegraphics[width=.18\columnwidth]{figures/supplementary/lindau00052_000019_cnn.png}
  }
  \subfigure{%
    \includegraphics[width=.18\columnwidth]{figures/supplementary/lindau00052_000019_ours.png}
  }\\[-2ex]




  \subfigure{%
    \includegraphics[width=.18\columnwidth]{figures/supplementary/lindau00027_000019_given.png}
  }
  \subfigure{%
    \includegraphics[width=.18\columnwidth]{figures/supplementary/lindau00027_000019_sp.png}
  }
  \subfigure{%
    \includegraphics[width=.18\columnwidth]{figures/supplementary/lindau00027_000019_gt.png}
  }
  \subfigure{%
    \includegraphics[width=.18\columnwidth]{figures/supplementary/lindau00027_000019_cnn.png}
  }
  \subfigure{%
    \includegraphics[width=.18\columnwidth]{figures/supplementary/lindau00027_000019_ours.png}
  }\\[-2ex]



  \setcounter{subfigure}{0}
  \subfigure[\scriptsize Input]{%
    \includegraphics[width=.18\columnwidth]{figures/supplementary/lindau00029_000019_given.png}
  }
  \subfigure[\scriptsize Superpixels]{%
    \includegraphics[width=.18\columnwidth]{figures/supplementary/lindau00029_000019_sp.png}
  }
  \subfigure[\scriptsize GT]{%
    \includegraphics[width=.18\columnwidth]{figures/supplementary/lindau00029_000019_gt.png}
  }
  \subfigure[\scriptsize Deeplab]{%
    \includegraphics[width=.18\columnwidth]{figures/supplementary/lindau00029_000019_cnn.png}
  }
  \subfigure[\scriptsize Using BI]{%
    \includegraphics[width=.18\columnwidth]{figures/supplementary/lindau00029_000019_ours.png}
  }%\\[-2ex]

  \mycaption{Street Scene Segmentation}{Example results of street scene segmentation.
  (d)~depicts the DeepLab results, (e)~result obtained by adding bilateral inception (BI) modules (\bi{6}{2}+\bi{7}{6}) between \fc~layers.}
\label{fig:street_visuals-app}
\end{figure*}

%\end{appendix}

\end{document}