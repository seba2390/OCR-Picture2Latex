\subsection{Numerical Simulation of ShadowRobot Hand}\label{sec:sr_hand}

The grippers used during previous tests have all been developed specifically for the above simulations. They were designed to test the approach's performance under a number of varying circumstances, and compare the results to those obtained using alternate grasping simulation methods. To illustrate efficacy of our approach using a real-world, complex gripper, we have applied our algorithm to a grasping scenario considering the ShadowRobot Hand (see Figure \ref{fig:srhand}). This anthropomorphic gripper features 24 joints and 20 DoFs. As done in previous examples, we compared performance of the proposed approach to that of a constraint-based method. As was the case with simpler grippers, the constraint-based method displays a singular behavior after some simulation time, whereas the proposed penalty-based algorithm behaves as desired. See \fig{srhandcontactforces}, which shows the constraint-based approach failing at about time instant $t=5.8$s.
%
% As demonstrated here, our approach can be utilized with more complex grasping scenarii, including configurations which mimic the complexities of human hand grasping.
% [ym]CHECK yeah they saw.
%
\begin{figure*}
	\centering
	\scalebox{1}[0.95]{\includegraphics[width=\textwidth,trim={0 3cm 0 7cm},clip]{images/sr_bunny_hold_4}}
	\caption{ShadowRobot Hand grasping the Stanford bunny model (\cite{levoy2005scanning}).}
	\label{fig:srhand}
\end{figure*}
%
\begin{figure}
	\centering
	\scalebox{1}[0.90]{\includegraphics[width=\linewidth]{images/sr_hand_contact_forces}}
	\caption{Contact force comparison for ShadowRobot Hand.}
	\label{fig:srhandcontactforces}
\end{figure}
%\begin{figure}
%	\centering
%	\includegraphics[width=\columnwidth]{images/sr_bunny_hold_4}
%	\caption{ShadowRobot Hand grasping the Stanford bunny model (\cite{levoy2005scanning}).}
%	\label{fig:srhand}
%\end{figure} 