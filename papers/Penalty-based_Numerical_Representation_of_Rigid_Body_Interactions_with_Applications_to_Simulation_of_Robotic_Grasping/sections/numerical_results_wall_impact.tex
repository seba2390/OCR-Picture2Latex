\subsection{Wall Impact} \label{sec:experimental_tuning}
%
A collision with a wall is a very simple scenario. We can manually compute the expected contact forces by hand and thus adjust the parametric gains $k_P$ and $k_D$ of our PD controller. Afterwards, we can use these values and see how they fare in more complex experiments.

%\begin{figure}
%	\centering
\begin{figure}%{0.45\columnwidth}
	\centering
	%\def\svgwidth{\columnwidth}
	\includegraphics[width=0.7\columnwidth]{images/wall_collision.png}
	\caption{Simple wall collision trajectory}
	\label{fig:wall_collision}
\end{figure}%
\begin{figure}%{0.45\columnwidth}
	\centering
	%\def\svgwidth{\columnwidth}
	%\includesvg[width=.9\columnwidth]{images/wall_collision_pos.svg}
	\includegraphics[width=.8\columnwidth]{images/wall_collision_pos.pdf}
	\caption{Comparison of Wall Collision trajectory between simulation and mathematically computed result}
	\label{fig:wall_collision_graph}
\end{figure}
%\end{figure}

The simulation itself comprises a rectangular object, a wall and floor (see figure \ref{fig:wall_collision}). Normal gravity of $9.81 \text{m s}^{-2}$ is acting on the object. The horizontal object starting velocity is set to $1\text{m s}^{-1}$.
To properly calibrate the $k_P$ and $k_D$ parameters, we compare the simulated trajectory with the correct trajectory that would result from a stiff collision with the wall. The latter can be trivially computed mathematically, thus being a good basis for tuning. Upon completion of the experiment, we determine that the parameters $k_P = 3967\;\text{kg}\;\text{s}^{-2}$ and $k_D = -97\;\text{kg}\;\text{s}^{-1}$
approximate the real-life trajectory rather well. Both simulated and expected trajectories are described in figures \ref{fig:wall_collision} and \ref{fig:wall_collision_graph}. The following experiments will all be using the computed parameters. 