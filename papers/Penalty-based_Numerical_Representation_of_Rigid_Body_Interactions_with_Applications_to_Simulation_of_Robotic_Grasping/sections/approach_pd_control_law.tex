\subsection{Reaction Efforts} \label{sec:reaction_efforts}
%
%  The value $\|s(t)\|$, $t\geqslant0$, is descriptive of the extent of overlap occurring.
% [ym]CHECK who cares???
Hereafter, we define reaction effort magnitude in such a manner that overlap in-between objects is reduced. In particular, the effort is conceived of as representative of a virtual mechanical impedance, the stiffness of which we choose proportional to the overlap volume $v(t)$, $t\geqslant0$, and add a damping term which would ideally be proportional to this volume's rate of change. However, note that there exists no direct way to compute the time derivative of $v(t)$. Using time-step differences (i.e. subtracting successive values, divided by time-step) is undesirable as it commonly proves numerically unstable (\cite{kagiwada2013numerical}). Instead, we rely on a simple second order low-pass to obtain a filtered derivative estimate. In particular, consider
%
\begin{eqnarray}
\ddot v_{\rm f}(t) &=& -2 \zeta \omega_{\rm n} \dot v_{\rm f}(t) + \omega_{\rm n}^2 (v(t) -v_{\rm f}(t)), \quad \dot v_{\rm f}(0)=0,\nonumber\\
                   & & \hspace{1.05in}  v_{\rm f}(0)=v(0), \quad  t\geqslant0, \label{eq:lowpass}
\end{eqnarray}
%
\noindent where $\zeta$, $\omega_{\rm n}\in \mathds R^{+*}$, and $\dot v_{\rm f}(t)$ provides a low-pass filtered estimate of $\dot v(t)$.

To better reflect the nonlinear (impulsive) behavior of body collision on first contact, we include a transient temporal-scale correction multiplier upon first contact. Results of numerical simulations show that use of this gain results in a substantially faster alignment of relative velocities at point of contact, and a reduced overall overlap. This gain is computed as
%
\begin{equation}
	\delta_{\textrm{i}} (t) \triangleq \frac{i_\textrm{v}(t)}{\sqrt{\pi}d_\textrm{v}(t)} \exp \left(-\frac{(t-t_\textrm{v}(t))^2}{d_\textrm{v}^2(t)}\right),\quad
t \geqslant 0,
\end{equation}
%
\noindent where
%
\begin{gather}
	i_\textrm{v}(t) \triangleq g_\textrm{i} (m_\textrm{A} + m_\textrm{B}) \Delta v_\textrm{c}(t), \nonumber\\
	d_\textrm{v}(t) \triangleq \frac{\textrm{d}_\textrm{t}}{\Delta v_\textrm{c}(t)}, \qquad t_\textrm{v}(t) \triangleq g_\textrm{t} d_\textrm{v}(t) - t_\textrm{c}, \nonumber
\end{gather}
%
\noindent $\Delta v_\textrm{c}(t) \in \mathds{R}$ is the relative speed (in m/s) of object $\mathcal{B}$ with respect to object $\mathcal{A}$ at $c(t)$ along the vector $s_{\rm d}(t)$, $\textrm{d}_\textrm{t} \in \mathds{R}^+$ is a target depth (in meters), $t_\textrm{c} \in \mathds{R}^+$ is the time instant of first contact (in $\textrm{s}$), $g_\textrm{i},$ $g_\textrm{t} \in \mathds{R}^+$ are parameters defining the magnitude and duration of applicability of gain $\delta_{\textrm{i}}(t)$, respectively. We then define the reaction efforts as follows,
%
\begin{align}
f_{\rm r}(t) &\triangleq \delta_{\textrm{i}} (t) \left( g_1 v(t) + g_2 \dot v_{\rm f}(t) \right) s_{\rm n}(t),\quad t \geqslant 0,\label{eq:fr}\\
 \tau_{\rm r}(t) &\triangleq (c(t)-p_{\rm o}) \times f_{\rm r}(t), \label{eq:tr}
\end{align}
%
where $g_1$, $g_2 \in \mathds R^{+*}$ are stiffness and damping gains (in kg/s and kg/s$^2$, respectively), $p_{\rm o}\in \mathds{R}^3$ represents the position of the considered object's center of mass, $s_{\rm n}(t)\triangleq s_{\rm d}(t)/\|s_{\rm d}(t)\|$, and $\|\cdot \|$ denotes the Euclidian norm. Note that the efforts provided by \eqn{fr}--\eqn{tr} is comparable to a linear Kelvin-Voigt contact model, descriptive of a compliant contact reaction between objects $\mathcal A$ and $\mathcal B$ (see \cite{diolaiti2005}, as well as the discussion in \cite{jain2011controlling} on modeling contact impedance). As previously alluded to, by allowing object meshes to intersect with one another, we mirror the deformation behavior expected of real-life objects at points of contact. For the purpose of numerical simulations, we built upon the models defined in \cite{diolaiti2005} and \cite{erickson2003contact} to define appropriate parameter values.
% Such models who have both based their models on values acquired through real-life experimentation. Their values are used in other setups, such as the simulator designed by \cite{flores2010contact}.
% [ym]CHECK whattever
Algorithm \ref{alg:reaction_efforts} describes the manner in which we perform the computation of \eqn{fr}--\eqn{tr} when an overlap is detected.
%
%\begin{figure}
%	\centering
%	\input{images/overlap_forces_01.pdf_tex}
%	\caption{Forces and torques resulting from a collision and the force application point ${c}$.}
%	\label{fig:overlap_forces}
%\end{figure}
%
\RestyleAlgo{boxed}
\LinesNumbered
\SetAlFnt{\footnotesize}
\IncMargin{0.4em}
\begin{algorithm}[t]
	\DontPrintSemicolon	
	\SetKwProg{Fn}{Function}{}{}
	\Fn{ComputeOverlap($\texttt{shape}_{\texttt{A}}$, $\texttt{shape}_{\texttt{B}}$)}
	{
		$\texttt{mesh}_\texttt{ov} = \texttt{bool\_intersect} \left( \texttt{mesh}_{\texttt{A}}, \texttt{mesh}_{\texttt{B}} \right)$\\
		\BlankLine\BlankLine
		\tcc{Sort triangles according to corresponding object}
		$\texttt{tri}_\texttt{A} = \texttt{extract} \left( \texttt{mesh}_{\texttt{ov}}, \texttt{mesh}_{\texttt{A}} \right)$\\
		$\texttt{tri}_\texttt{B} = \texttt{extract} \left( \texttt{mesh}_{\texttt{ov}}, \texttt{mesh}_{\texttt{B}} \right)$\\
		\tcc{Determine depth weight of individual triangles}
		$s_{\text d}(t) \leftarrow 0$\\
		\ForEach{$\texttt{v}_\texttt{A} \in \texttt{tri}_\texttt{A}$}
		{
			$\texttt{depth}_{\texttt{v}_\texttt{A}} = \| \texttt{center}\left( \texttt{tri}_\texttt{A} \right) - c(t) \|$\\
			$\omega_{\texttt{v}_\texttt{A}} \leftarrow \frac{1}{3} \texttt{surf} \left( \texttt{v}_\texttt{A} \right) \times \texttt{depth}_{\texttt{v}_\texttt{A}}$\\
			$\omega_\texttt{A} \leftarrow \left\{\omega_\texttt{A}; \omega_{\texttt{v}_\texttt{A}} \right\}$\\
			$s_{\text d} \leftarrow s_{\text d} + \omega_{\texttt{v}_\texttt{A}} n_{\texttt{v}_\texttt{A}}$\\
		}
	
		\ForEach{$\texttt{v}_\texttt{B} \in \texttt{tri}_\texttt{B}$}
		{
			$\texttt{depth}_{\texttt{v}_\texttt{B}} = \| \texttt{center}\left( \texttt{tri}_\texttt{B} \right) - c(t) \|$\\
			$\omega_{\texttt{v}_\texttt{B}} \leftarrow \frac{1}{3} \texttt{surf} \left( \texttt{v}_\texttt{B} \right) \times \texttt{depth}_{\texttt{v}_\texttt{B}}$\\
			$\omega_\texttt{B} \leftarrow \left\{\omega_\texttt{B}; \omega_{\texttt{v}_\texttt{B}} \right\}$\\
			$s_{\text d} \leftarrow s_{\text d} - \omega_{\texttt{v}_\texttt{A}} n_{\texttt{v}_\texttt{A}}$\\
		}
		
%		\BlankLine\BlankLine
%		$s \leftarrow \frac{s_{\text d}}{\|s_{\text d}\|} \times \texttt{volume}\left(\texttt{mesh}_\texttt{ov}\right)$
		
		\BlankLine\BlankLine
%		\tcc{Apply PD Control (see \sec{reaction_efforts})}
        \tcc{Compute virtual impedance (see \sec{reaction_efforts})}
		$f_{\rm r} \leftarrow \frac{\texttt{IMP}_{\texttt{col}}.\texttt{correction}\left( \|{s_\text{d}}\| \right)}{\|{s_\text{d}}\|} s_\text{d}$
		
		\BlankLine\BlankLine
		\tcc{Return forces and torques acting on objects $\mathcal A$ and $\mathcal B$}
		\Return $\left[ \begin{matrix}
					{f}_{\rm r}\\\left({c} - {p}_{\mathcal{A}}\right) \times {f}_{\rm r}
				\end{matrix}\right]$,
				$\left[ \begin{matrix}
					-{f}_{\rm r}\\\left({c} - {p}_{\mathcal{B}}\right) \times -{f}_{\rm r}
				\end{matrix} \right]$
	}
	
	\BlankLine\BlankLine
	\KwResult{
			$\left[ \begin{matrix}
				{f}_{\mathcal{A}}\\{\tau}_{\mathcal{A}}
			\end{matrix}\right]$,
			$\left[ \begin{matrix}
				{f}_{\mathcal{B}}\\{\tau}_{\mathcal{B}}
			\end{matrix} \right]$}
	
	\caption{Computation of reaction efforts magnitude and direction. }
	\label{alg:reaction_efforts}
\end{algorithm}
% 