\section{Numerical Simulation}\label{sec:simulation}
%
We use \texttt{Bullet} as a basis for our numerical rigid body simulation. Its modular design has allowed us to quickly replace relevant portions of their solver (pertaining to contact detection and contact efforts) with the proposed approach. In particular, we have replaced the constraint solver used by \texttt{Bullet} with our penalty-based solver (as described in Algorithm \ref{alg:reaction_efforts}). In addition, we replaced the Euler integration with an Ordinary Differential Equation (ODE) solver based on the explicit predictive-corrective Runge-Kutta(4,5) method (\cite{dormand1980family}), \texttt{ode45}, used to integrate \eqn{dyn_trans}--\eqn{dyn_rot}, where efforts $f(t)$, $\tau(t)$, $t\geqslant0$, are composed of reaction efforts provided by \eqn{fr}--\eqn{tr}, gravity's acceleration, and friction forces. To compute the latter, we rely on a Coulomb Friction model adapted from \cite{liu2015experimental}. To circumvent possible issues stemming from the usual discontinuity, we substitute a hyperbolic tangent in the place of the signum function, as described hereafter,
%
\begin{align}
	f_{\rm f}(t) = -\Big( \mu \|f_{\rm n}(t)\| \tanh \left(\gamma\|\vvR\| \right) + \beta \|v_{\rm r}(t)\|&\Big) v_{\rm rn}(t), \nonumber \\
	t\geqslant0&,
\end{align}
%
\noindent where $f_{\rm f}(t) \in \mathds{R^3}$ is the friction force (in N), $\mu$, $\beta \in \mathds{R^+}$ represent the Coulomb friction coefficient and the viscous friction coefficient, respectively, $\gamma \in \mathds R^+$ is a scaling factor, $f_{\rm n}(t)\in \mathds{R}^3$ is the reaction force, normal to the surface of contact, $v_{\rm r}(t) \in \mathds{R}^3$ is the relative velocity at contact, and $v_{\rm rn}(t)\triangleq v_{\rm r}(t) /\|v_{\rm r}(t) \|$. The scaling factor $\gamma$ is used to adjust the slope of the zero crossing. In the limit that $\gamma\rightarrow+\infty$, one recovers the usual (discontinuous) model. As in \cite{keller1994}, we define the normal contact force acting on an object as $f_{\rm n}(t) \triangleq f_{\rm r}(t)$, i.e. the previously computed repulsive force.

%
%\begin{figure}%{0.45\columnwidth}
%	\centering
%	\includegraphics[width=1\columnwidth]{images/wall_collision.png}
%	\caption{Object colliding with a wall. \todo{TODO: Check if this section will be kept. If so, take images in Gazebo}}
%	\label{fig:wall_collision}
%\end{figure}%
%%
%\begin{figure}%{0.45\columnwidth}
%	\centering
%	\includesvg[width=1\columnwidth]{images/wall_collision_pos.svg}
%	\caption{Comparison of numerically simulated trajectory and analytically exact trajectory. \todo{TODO: Check if this section will be kept. If so, take images in Gazebo. Else replace with distance of object and gripper along test trajectory, show that little wobble occurs, if any}}
%	\label{fig:wall_collision_graph}
%\end{figure}
%%
%
%\subsection{Wall Impact} \label{sec:wall}
%%
%As a first verification of the proposed concept, we simulate a situation in which a rigid body in a gravity field collides with a vertical wall, then comes to rest on a horizontal ground (see \fig{wall_collision}). The object chosen is a rectangular rigid body of a mass of $0.5$ kg, gravity's acceleration is set to be $9.81$ m/s$^{2}$, friction parameters are chosen to be $\mu=1.25$ s/m, $\beta=1.4$ s/m, $\gamma=100$, the impedance gains in \eqn{fr} are set to $g_1 = 1654$ kg/s$^{2}$, $g_2 = 21$ kg/s, the low-pass filter parameters in \eqn{lowpass} are set to $\zeta=1/\sqrt2$ and $\omega_{\rm n}=10$s$^{-1}$. The transient temporal gain parameters are set to $g_\textrm{i} = 743\frac{1}{\textrm{s}}$ and $g_\textrm{i} = 3.98$ Finally, the object is set with an initial horizontal velocity of $1$ m/s in the direction of the wall. The trajectory obtained using the proposed approach is shown in \fig{wall_collision} and in blue in \fig{wall_collision_graph}. It is compared to the exact trajectory in \fig{wall_collision_graph} (exact trajectory shown in orange). This comparison illustrates that the proposed approach provides a fair reflection of the expected behavior. \todo{TODO: Create image}

\subsection{Simulated Grasps}\label{sec:grasp}
%
To evaluate our approach in various grasping scenarii, we have implemented three grippers, representative of a range of structural complexity. The first is a simple 2-fingered prismatic system, the second a 2-fingered revolute gripper, and the final one is a 3-fingered, 3-dimensional revolute gripper. For testing purposes, we have computed the Lagrangian equations of motion for each model, and will be using these instead of the standard constraint-based approach of \texttt{Bullet} to compute the joints' dynamics. Figure \ref{fig:grippersobjects} shows the grippers and objects used for testing.
%
\begin{figure}
	\centering
	\scalebox{1}[0.9]{\includegraphics[width=1.0\linewidth]{images/grippers_objects}}
	\caption{Grippers (top) and objects (bottom) used for performance evaluation.}
	\label{fig:grippersobjects}
\end{figure}
%
Three test operations are performed for each gripper individually, with each operation aiming to grasp a particular object. The target objects are a cube, a sphere, and a Stanford Bunny model (\cite{levoy2005scanning}). The mass of each object is set to $0.2$kg. We compare the results of our approach to those obtained with \texttt{Bullet}'s default constraint-based method. Simulation length is chosen to be $10$s, and the following parameters are used, $g_1 = 740 \frac{\text{N}}{\text{cm}}$, $g_2 = 50 \frac{\text{N}}{\text{cms}}$, $g_\text{i} = 250$, $d_\text{t} = 2 \text{mm}$, $d_\text{t} = 2 \text{mm}$, $\mu = \beta = 0.2$, $\gamma = 1$.

The grippers start in an open configuration, the object placed in such a manner that it is ready to be grasped. Once the simulation begins, the fingers enclose the object and exert a constant force from each side for the duration of the simulation. In our experience, using constraint-based approaches has consistently led to failure in finite time for grippers including revolute joints, as shown in \fig{grasp_explosion} where the constraint based approach fails at around simulation time $t=7.7$s (large contact force spike, leading to simulation failure). Here, a 2-finger revolute gripper grasps a ball.
%The constraint-based method fails to keep the object at the expected position for both rotational grippers. In particular, after a finite simulation time, a sudden and substantial increase in volume overlap occurs, and the object is accelerated away with high velocities.
% [ym]DISCUSS Is this a repeatable behavior? If so mention it? Does it happen every time? If not, would have needed some stats (breaks 80% of the time with this configuration). That only happens for the rotational grippers, not prismatic one, correct? That should be stated explicitly, it's ambiguous right now. Also, one figure showing efforts or acceleration or energy spiking up is needed.

This behavior does not occur with the proposed approach, which results in the object stably remaining in the gripper's center, between the enclosing fingers. See \fig{grasp_throw} for an illustration of measured contact forces. To better illustrate the shape of the efforts produced upon impact (transient) in comparison to the steady-state value corresponding to contact efforts, we selected a particular time-window.
%
\begin{figure}
	\centering
	%\def\svgwidth{\columnwidth}
	\scalebox{1}[1]{\includegraphics[width=\columnwidth]{svg-inkscape/contact_forces_4.pdf}}
	\vspace{-0.7cm}
	\caption{Comparison of contact forces generated by \texttt{Bullet}'s Constraint-based LCP solver and our approach. Dashed lines denote results from the constraint-based simulator, solid lines from the penalty-based method.}
	\label{fig:grasp_throw}
\end{figure}
%
\begin{figure}
	\centering
	%\def\svgwidth{\columnwidth}
	\scalebox{1}[0.9]{\includegraphics[width=\columnwidth]{svg-inkscape/contact_force_explosion.pdf}}
	%\vspace{-0.7cm}
	\caption{Grasp failure of constraint-based approach.}
	\label{fig:grasp_explosion}
\end{figure}
%
As shown in \fig{grasp_throw}, the transient efforts upon contact differ between penalty- and constraint-based methods. However, both approaches converge to an identical contact force magnitude once the initial impact has subsided and bodies have reached their steady-state configurations.
% Not shown here are the instances later on where the constraint-based approach explodes.
% [ym]DISCUSS (reference sr hand example) yeah no, you need to show it, you can't expect them to take your work for it. Even one single example.
% To prevent this scenario, our approach decreases mechanical contact stiffness. Mirroring the deformation of real-life objects, a certain amount of object overlap is tolerated, as long as contact forces attempt to minimize said intersection.
% [ym]CHECK we've said that already, and they probably don't care.

% Our aim is a simulator that mirrors reality as much as possible, while not jeopardizing the possibility of execution. The following analyses the limits of our approach. 