Our approach allows objects to first overlap on contact, then apply corrective forces that return them to a physically correct state. In real-life scenarios, two bodies never occupy the same space at the same time. In contrast to this, penalty-based methods must allow overlaps to occur to compute reaction efforts, which inhibits their precision in modeling rigid bodies. However, no real-life objects possess an infinite stiffness. Upon contact, objects will bend ever so slightly before returning to their original shape, thus moving their center of mass closer together. This can be mimicked by mechanical impedance, as demonstrated in \cite{jain2011controlling}. Our process mimics the brief bending and restitution in a similar manner, allowing objects to move closer together than a rigid model would allow (bending), then pushing them back to their intended distance (restitution). This effect is noticeable in grasping scenarios, where objects are bent slightly out of shape, exhibiting spring forces similar to mechanical impedance at the contact points.