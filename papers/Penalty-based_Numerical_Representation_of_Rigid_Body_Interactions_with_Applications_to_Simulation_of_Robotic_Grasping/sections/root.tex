\documentclass[letterpaper, 10pt, conference]{ieeeconf}
\IEEEoverridecommandlockouts

\usepackage[noend, boxruled, linesnumbered] {algorithm2e}
\SetAlCapSkip{1em}
\makeatletter
\renewcommand{\algorithmcfname}{Alg.}
\renewcommand{\fnum@algocf}{\AlCapSty{\AlCapFnt\algorithmcfname\nobreakspace\thealgocf}}
\makeatother

\usepackage{svg}
\usepackage{float}
\usepackage{amsmath, array}
\usepackage{bm}
\usepackage{subcaption}
\usepackage{dsfont}
\usepackage{amssymb}
\usepackage[labelfont=bf]{caption}
\usepackage[noadjust]{cite}
\usepackage{filecontents}
\usepackage{url}

\newlength{\bibitemsep}\setlength{\bibitemsep}{0\baselineskip plus .01\baselineskip minus .05\baselineskip}
\newlength{\bibparskip}\setlength{\bibparskip}{0pt}
\let\oldthebibliography\thebibliography
\renewcommand\thebibliography[1]{%
	\oldthebibliography{#1}%
	\setlength{\parskip}{\bibitemsep}%
	\setlength{\itemsep}{\bibparskip}%
}

%\usepackage{flushend}

\renewcommand{\citedash}{--}

\newcommand{\simplicity}[1]{\textcolor{orange}{#1}}
\newcommand{\impedance}[1]{\textcolor{blue}{#1}}
\newcommand{\grasping}[1]{\textcolor{cyan}{#1}}
\newcommand{\todo}[1]{\textcolor{red}{#1}}

\newcommand{\vdeltaL}{\delta_{\text{l}}}

\newcommand{\vf}{f(t)}
\newcommand{\vfEst}{d(t)}
\newcommand{\vdfEst}{\dot{d}_{\text{p}}(t)}
\newcommand{\vfP}{d_{\text{p}}(t)}
\newcommand{\vfF}{f_F (t)}
\newcommand{\vdn}{d_n (t)}

\newcommand{\vt}{\tau(t)}

\newcommand{\vvR}{v_{\text{r}}(t)}
\newcommand{\vvRn}{v_{\text{r}_n}(t)}
\newcommand{\vkV}{k_{\text{v}}(t)}

\newcommand{\vc}{c(t)}
\newcommand{\vcP}{c_{\text{p}}(t)}

\newcommand{\veP}{e_{\text{p}}(t)}
\newcommand{\veD}{e_{\text{d}}(t)}
\newcommand{\vdeD}{\dot{e}_{\text{d}}(t)}

\newcommand{\vkP}{k_{\text{p}}}
\newcommand{\vkD}{k_{\text{d}}}

\newcommand{\vx}{x(t)}
\newcommand{\vddx}{\ddot{x}(t)}

\newcommand{\vw}{\omega(t)}
\newcommand{\vdw}{\dot{\omega}(t)}

\newcommand{\vp}{p}
\newcommand{\vpO}{p_{\text{O}}(t)}

\newcommand{\vA}{\text{A}}
\newcommand{\vB}{\text{B}}

\newcommand{\vlA}{l_{\text{A}}}
\newcommand{\vlB}{l_{\text{B}}}
\newcommand{\vnlA}{n_{l_{\text{A}}}}

\newcommand{\vthetaO}{\theta_1 (t)}
\newcommand{\vthetaT}{\theta_2 (t)}
\newcommand{\vdthetaO}{\dot{\theta}_1 (t)}
\newcommand{\vdthetaT}{\dot{\theta}_2 (t)}
\newcommand{\vddthetaO}{\ddot{\theta}_1 (t)}
\newcommand{\vddthetaT}{\ddot{\theta}_2 (t)}
\newcommand{\vdeltaTheta}{\Delta\theta(t)}

\def\sec#1{Section \ref{sec:#1}}
\def\fig#1{Figure \ref{fig:#1}}
\def\alg#1{Alg. \ref{alg:#1}}
\def\eqn#1{(\ref{eq:#1})}

\title{\LARGE \bf Penalty-based Numerical Representation of Rigid Body Interactions with Applications to Simulation of Robotic Grasping}


\author{Michael Zechmair, Yannick Morel%
	\thanks{Michael and Yannick are with the Faculty of Psychology and Neuroscience,
		Bonnefantenstraat 2, 6211 KL Maastricht, Netherlands,
		{\tt\scriptsize \{m.zechmair, y.morel\}@maastrichtuniversity.nl}.}%
	\thanks{This research has received funding from the European Union's Horizon 2020 Framework Programme for Research and Innovation under the Specific Grant Agreement No. 785907 (Human Brain Project SGA3).}%
}

\setlength{\textfloatsep}{10pt}
\setlength{\belowdisplayskip}{3pt}
\setlength{\abovedisplayskip}{3pt}
\setlength{\belowcaptionskip}{-5pt}

%
\begin{document}
	\maketitle
	\thispagestyle{empty}
	\pagestyle{empty}
	
	%
	\noindent\begin{abstract}
		This paper presents a novel approach to numerically describe the interactions between rigid bodies, with a special focus on robotic grasping. Some of the more common approaches used to address such issues rely on satisfaction of a set of strict constraints, descriptive of the expected physical reality of such interactions in practice. However, application of constraint-based methods in a numerical setting may lead to problematic configurations in which, for instance, volumes occupied by distinct bodies may overlap. Such situations lying beyond the range of admissible configurations for constraint-based methods, their occurrence typically results in non-meaningful simulation outcomes. We propose a method which acknowledges the possibility of such occurrences while demoting them. This is pursued through the use of a penalty-based approach, and draws on notions of mechanical impedance to infer apposite reaction forces. Results of numerical simulations illustrate efficacy of the proposed approach.
	\end{abstract}
	%
	% \leavevmode
% \\
% \\
% \\
% \\
% \\
\section{Introduction}
\label{introduction}

AutoML is the process by which machine learning models are built automatically for a new dataset. Given a dataset, AutoML systems perform a search over valid data transformations and learners, along with hyper-parameter optimization for each learner~\cite{VolcanoML}. Choosing the transformations and learners over which to search is our focus.
A significant number of systems mine from prior runs of pipelines over a set of datasets to choose transformers and learners that are effective with different types of datasets (e.g. \cite{NEURIPS2018_b59a51a3}, \cite{10.14778/3415478.3415542}, \cite{autosklearn}). Thus, they build a database by actually running different pipelines with a diverse set of datasets to estimate the accuracy of potential pipelines. Hence, they can be used to effectively reduce the search space. A new dataset, based on a set of features (meta-features) is then matched to this database to find the most plausible candidates for both learner selection and hyper-parameter tuning. This process of choosing starting points in the search space is called meta-learning for the cold start problem.  

Other meta-learning approaches include mining existing data science code and their associated datasets to learn from human expertise. The AL~\cite{al} system mined existing Kaggle notebooks using dynamic analysis, i.e., actually running the scripts, and showed that such a system has promise.  However, this meta-learning approach does not scale because it is onerous to execute a large number of pipeline scripts on datasets, preprocessing datasets is never trivial, and older scripts cease to run at all as software evolves. It is not surprising that AL therefore performed dynamic analysis on just nine datasets.

Our system, {\sysname}, provides a scalable meta-learning approach to leverage human expertise, using static analysis to mine pipelines from large repositories of scripts. Static analysis has the advantage of scaling to thousands or millions of scripts \cite{graph4code} easily, but lacks the performance data gathered by dynamic analysis. The {\sysname} meta-learning approach guides the learning process by a scalable dataset similarity search, based on dataset embeddings, to find the most similar datasets and the semantics of ML pipelines applied on them.  Many existing systems, such as Auto-Sklearn \cite{autosklearn} and AL \cite{al}, compute a set of meta-features for each dataset. We developed a deep neural network model to generate embeddings at the granularity of a dataset, e.g., a table or CSV file, to capture similarity at the level of an entire dataset rather than relying on a set of meta-features.
 
Because we use static analysis to capture the semantics of the meta-learning process, we have no mechanism to choose the \textbf{best} pipeline from many seen pipelines, unlike the dynamic execution case where one can rely on runtime to choose the best performing pipeline.  Observing that pipelines are basically workflow graphs, we use graph generator neural models to succinctly capture the statically-observed pipelines for a single dataset. In {\sysname}, we formulate learner selection as a graph generation problem to predict optimized pipelines based on pipelines seen in actual notebooks.

%. This formulation enables {\sysname} for effective pruning of the AutoML search space to predict optimized pipelines based on pipelines seen in actual notebooks.}
%We note that increasingly, state-of-the-art performance in AutoML systems is being generated by more complex pipelines such as Directed Acyclic Graphs (DAGs) \cite{piper} rather than the linear pipelines used in earlier systems.  
 
{\sysname} does learner and transformation selection, and hence is a component of an AutoML systems. To evaluate this component, we integrated it into two existing AutoML systems, FLAML \cite{flaml} and Auto-Sklearn \cite{autosklearn}.  
% We evaluate each system with and without {\sysname}.  
We chose FLAML because it does not yet have any meta-learning component for the cold start problem and instead allows user selection of learners and transformers. The authors of FLAML explicitly pointed to the fact that FLAML might benefit from a meta-learning component and pointed to it as a possibility for future work. For FLAML, if mining historical pipelines provides an advantage, we should improve its performance. We also picked Auto-Sklearn as it does have a learner selection component based on meta-features, as described earlier~\cite{autosklearn2}. For Auto-Sklearn, we should at least match performance if our static mining of pipelines can match their extensive database. For context, we also compared {\sysname} with the recent VolcanoML~\cite{VolcanoML}, which provides an efficient decomposition and execution strategy for the AutoML search space. In contrast, {\sysname} prunes the search space using our meta-learning model to perform hyperparameter optimization only for the most promising candidates. 

The contributions of this paper are the following:
\begin{itemize}
    \item Section ~\ref{sec:mining} defines a scalable meta-learning approach based on representation learning of mined ML pipeline semantics and datasets for over 100 datasets and ~11K Python scripts.  
    \newline
    \item Sections~\ref{sec:kgpipGen} formulates AutoML pipeline generation as a graph generation problem. {\sysname} predicts efficiently an optimized ML pipeline for an unseen dataset based on our meta-learning model.  To the best of our knowledge, {\sysname} is the first approach to formulate  AutoML pipeline generation in such a way.
    \newline
    \item Section~\ref{sec:eval} presents a comprehensive evaluation using a large collection of 121 datasets from major AutoML benchmarks and Kaggle. Our experimental results show that {\sysname} outperforms all existing AutoML systems and achieves state-of-the-art results on the majority of these datasets. {\sysname} significantly improves the performance of both FLAML and Auto-Sklearn in classification and regression tasks. We also outperformed AL in 75 out of 77 datasets and VolcanoML in 75  out of 121 datasets, including 44 datasets used only by VolcanoML~\cite{VolcanoML}.  On average, {\sysname} achieves scores that are statistically better than the means of all other systems. 
\end{itemize}


%This approach does not need to apply cleaning or transformation methods to handle different variances among datasets. Moreover, we do not need to deal with complex analysis, such as dynamic code analysis. Thus, our approach proved to be scalable, as discussed in Sections~\ref{sec:mining}.
	\section{Proposed Approach}\label{sec:approach}
%
\baselineskip12pt
We consider the physical interaction of two or more rigid bodies. Each body is defined by its inertia properties (including mass $m\in \mathds R^{+*}$, assuming for simplicity homogeneous mass density distribution, and rotational inertia $I\in \mathds R^{3\times 3}$), a mesh describing the object's geometry as a set of vertices, and relevant friction parameters (as further discussed in \sec{simulation}). Each object's equations of motion are of the following form (\cite{tong2004lectures}),
%
\begin{eqnarray}
m \ddot x(t) &=& f(t),  \quad t\geqslant 0, \label{eq:dyn_trans}\\
I \dot \omega(t) &=& \tau(t) - \omega(t) \times I \omega(t), \label{eq:dyn_rot}
\end{eqnarray}
%
\noindent where $f(t) \in \mathds{R}^3$ is the force (in N) acting on the considered object, $\tau(t) \in \mathds{R}^3$ is the torque (in Nm), $m$ is expressed in kg, $I$ in kg$\cdot\text{m}^2$, $x(t) \in \mathds{R}^3$ represents the object's position (in m), and $\omega(t) \in \mathds{R}^3$ the object's angular velocity in $\text{s}^{-1}$. In the following, we discuss the manner in which we detect the emergence of contacts (and incidentally of overlapping object configurations), then infer appropriate reaction forces such that the overlap is mitigated.

\subsection{Overview}
%
\begin{figure}
	\centering
	\scalebox{1}[0.9]{\includegraphics[width=1\columnwidth]{images/overlap_vectors.pdf}}
	\caption{Schematic representation of interaction force computation, the sphere is labeled $\mathcal{A}$, the cube $\mathcal{B}$. Upon detection of overlap, the mesh describing the overlap volume is identified (dark blue mesh, left). Evaluation of the depth of overlap is done in a discrete manner, breaking down a sample surface in a set of triangles (cyan and orange, right). The contact effort direction of forces acting from object $\mathcal{A}$ on object $\mathcal{B}$ is computed using the normals $n_\mathcal{A}$ and $n_\mathcal{B}$, $c(t)$ is the contact force application point.}
	\label{fig:overlap}
\end{figure}
%
Appropriate simulation of interacting rigid bodies necessarily requires that aspects related to collision detection and reaction efforts be accounted for. If relying on a penalty-based approach, such as that discussed in the following, this is typically achieved by monitoring and detecting situations in which objects overlap; that is, situations in which parts of two or more objects occupy the same space. As such situations do not arise in practice, their occurrence in simulation is undesirable, and penalty-based approaches introduce mitigating measures to overcome them. Such measures revolve around carefully designed interaction forces, developed such that overlap events are limited to transient occurrences.

The method pursued hereafter involves four successive steps. First, object collisions and volume overlaps are determined by monitoring the relative position of bodies whose dynamics are simulated. The individual meshes are intersected, and possible overlaps registered. As a direct comparison of vertices would prove exceedingly computationally expensive, this step is broken down into two stages. \texttt{\textbf 1} Object Mesh pairs in close proximity to each other are identified. \texttt{\textbf 2} Detection of a possible overlap is then performed by use of a standard Axis-Aligned Bounding Box (AABB) collision detection scheme (such as that in \cite{jimenez20013d}). \texttt{\textbf 3} Once an overlap is detected, we compute the corresponding reaction forces. Broadly speaking, the more severe the overlap, the greater the required repulsive displacement effort. Accordingly, the reaction forces we design will be made (in some measure) proportional to this volume (as is commonly the case for penalty-based techniques, see \cite{sagardia_penalty}). \texttt{\textbf 4} Direction of the resulting force vector is defined in such a manner that objects' movements decrease the overlap, and the point of application for this effort is selected. The proposed methodology is summarized in Algoritm \ref{alg:reaction_efforts}. The methodology is described in more detail hereafter.

% Current simulators, such as ODE and Bullet, use the maximum intersection depth of two colliding objects to determine contact points. Our approach instead computes the total intersection volume to compute reaction efforts as well as application points and directions.

\subsection{Characterizing the overlap}
%
We use the \texttt{libigl} library \cite{jacobson2016libigl} for mesh processing, which is instrumental to the proposed approach. First, we examine the meshes from selected potentially overlapping bodies and perform a standard boolean intersection operation. The result produced by \texttt{libigl} consists of a triangle mesh describing the shape of the overlap. Should the operation result in an empty set, we infer that the considered pair of bodies does not intersect and no further computations are necessary.

Should the boolean intersection provide a non-empty shape, we proceed to identifying three reaction effort parameters, the application point $c(t) \in \mathds{R}^3$, $t\geqslant 0$, the reaction effort's direction $s_{\text d}(t)\in \mathds{R}^3 $, and its magnitude $s_{\rm m}(t)\in \mathds{R}$. The point of application is assigned to the geometric center of the previously computed intersection mesh. A simple method to assign $s_{\rm m}(t)$ could involve selecting a value proportional to the scalar overlap volume measurement $v(t) \in \mathds{R}$, as is commonly done in penalty-based approaches. Such a choice leads to a proportional relationship between reaction effort and overlap volume. We propose a different approach, building upon this proportional term, in such a manner that issues commonly encountered in the use of penalty-based approaches are circumvented, as discussed in section \ref{sec:reaction_efforts}.

Assessing an appropriate $s_{\text d}(t)$, $t\geqslant 0$, requires a measure of special attention. In particular, it is necessary that collision forces be designed in such a manner that intersecting objects tend to separate from one another.
%Further, contact points with deeper intersections should result in higher repulsive efforts than shallow ones. [YM]DISCUSS That's related to the magnitude, not the direction?
% We have introduced the following approach for determining collision force direction to satisfy these requirements.
To ensure that this is the case, we approach the problem by considering the previously computed intersection mesh. More specifically, we define two sub-meshes, distinguishing, within the overlap mesh, which triangles originated from which body. For ease of exposition, we consider a case in which only two rigid bodies are overlapping, denoting them as object $\mathcal A$ and $\mathcal B$ (see \fig{overlap}). One sub-mesh is designed to include all triangles overlapping with the mesh of object $\mathcal A$, the other all triangles overlapping with the mesh of object $\mathcal B$ (as illustrated in \fig{overlap}). Then, we iterate over all triangles in subset $\mathcal A$, estimating a weight factors reflecting a triangle's contact depth and the overlap volume it represents. In greater detail, for each triangle $i$ we determine the distance vector $n_{\mathcal{A}_i}(t) \in \mathds{R}^3$ from its center point to the center of the intersection mesh $c(t)$. We then compute a weight $\omega_{\mathcal{A}_i}(t) \in \mathds{R}^+$ reflecting the scalar volume of the pyramid defined by the triangle surface as base and the point $c(t)$ as additional vertex. Once this has been performed for all triangles in the $\mathcal A$ sub-mesh, the same operation is conducted for triangles in subset $\mathcal B$. The direction of application is then computed as the following weighted sum,
%
%\begin{eqnarray}
%s_{\text d}(t)\!\!\!\! &\triangleq& \!\!\! \sum_{i=1}^{m_\mathcal{A}} \omega_{\mathcal{A}_i}(t)n_{\mathcal{A}_i}(t) - \sum_{i=1}^{m_\mathcal{B}} \omega_{\mathcal{B}_i}(t)n_{\mathcal{B}_i}(t), \quad t\geqslant0,\hspace{0.1in}
%\end{eqnarray}
\begin{equation}
	s_{\text d}(t) \triangleq \sum_{i=1}^{m_\mathcal{A}} \omega_{\mathcal{A}_i}(t)n_{\mathcal{A}_i}(t) - \sum_{i=1}^{m_\mathcal{B}} \omega_{\mathcal{B}_i}(t)n_{\mathcal{B}_i}(t), \quad t\geqslant0,\hspace{0.1in}
\end{equation}
%
\noindent where $m_\mathcal{A}$, $m_\mathcal{B} \in \mathds N$ represent the number vertices in subset $\mathcal{A}$ and $\mathcal{B}$, respectively. The value of $s_{\text d}(t)$ describes the reaction force direction vector. Efficacy of this design is illustrated in section \sec{simulation}. In the following, we discuss the design of the interaction effort magnitude, $s_{\rm m}(t)$.

% [YM]CHECK: You don't use the following, so why present it?
%
%With $v(t)$, we arrive at the force estimate as
%
%\begin{eqnarray}
%	s(t) &\triangleq& v(t) \frac{s_{\text d}(t)}{\|s_{\text d}(t)\|} , \quad t\geqslant0. \\
%\end{eqnarray}
%
%In the following, we discuss the manner in which we may rely on this quantity, as well as $c(t)$ interaction efforts between objects $\mathcal A$ and $\mathcal B$.
%\noindent In the following, we discuss the design of the interaction effort magnitude, $s_{\rm m}(t)$.


	\subsection{Reaction Efforts} \label{sec:reaction_efforts}
%
%  The value $\|s(t)\|$, $t\geqslant0$, is descriptive of the extent of overlap occurring.
% [ym]CHECK who cares???
Hereafter, we define reaction effort magnitude in such a manner that overlap in-between objects is reduced. In particular, the effort is conceived of as representative of a virtual mechanical impedance, the stiffness of which we choose proportional to the overlap volume $v(t)$, $t\geqslant0$, and add a damping term which would ideally be proportional to this volume's rate of change. However, note that there exists no direct way to compute the time derivative of $v(t)$. Using time-step differences (i.e. subtracting successive values, divided by time-step) is undesirable as it commonly proves numerically unstable (\cite{kagiwada2013numerical}). Instead, we rely on a simple second order low-pass to obtain a filtered derivative estimate. In particular, consider
%
\begin{eqnarray}
\ddot v_{\rm f}(t) &=& -2 \zeta \omega_{\rm n} \dot v_{\rm f}(t) + \omega_{\rm n}^2 (v(t) -v_{\rm f}(t)), \quad \dot v_{\rm f}(0)=0,\nonumber\\
                   & & \hspace{1.05in}  v_{\rm f}(0)=v(0), \quad  t\geqslant0, \label{eq:lowpass}
\end{eqnarray}
%
\noindent where $\zeta$, $\omega_{\rm n}\in \mathds R^{+*}$, and $\dot v_{\rm f}(t)$ provides a low-pass filtered estimate of $\dot v(t)$.

To better reflect the nonlinear (impulsive) behavior of body collision on first contact, we include a transient temporal-scale correction multiplier upon first contact. Results of numerical simulations show that use of this gain results in a substantially faster alignment of relative velocities at point of contact, and a reduced overall overlap. This gain is computed as
%
\begin{equation}
	\delta_{\textrm{i}} (t) \triangleq \frac{i_\textrm{v}(t)}{\sqrt{\pi}d_\textrm{v}(t)} \exp \left(-\frac{(t-t_\textrm{v}(t))^2}{d_\textrm{v}^2(t)}\right),\quad
t \geqslant 0,
\end{equation}
%
\noindent where
%
\begin{gather}
	i_\textrm{v}(t) \triangleq g_\textrm{i} (m_\textrm{A} + m_\textrm{B}) \Delta v_\textrm{c}(t), \nonumber\\
	d_\textrm{v}(t) \triangleq \frac{\textrm{d}_\textrm{t}}{\Delta v_\textrm{c}(t)}, \qquad t_\textrm{v}(t) \triangleq g_\textrm{t} d_\textrm{v}(t) - t_\textrm{c}, \nonumber
\end{gather}
%
\noindent $\Delta v_\textrm{c}(t) \in \mathds{R}$ is the relative speed (in m/s) of object $\mathcal{B}$ with respect to object $\mathcal{A}$ at $c(t)$ along the vector $s_{\rm d}(t)$, $\textrm{d}_\textrm{t} \in \mathds{R}^+$ is a target depth (in meters), $t_\textrm{c} \in \mathds{R}^+$ is the time instant of first contact (in $\textrm{s}$), $g_\textrm{i},$ $g_\textrm{t} \in \mathds{R}^+$ are parameters defining the magnitude and duration of applicability of gain $\delta_{\textrm{i}}(t)$, respectively. We then define the reaction efforts as follows,
%
\begin{align}
f_{\rm r}(t) &\triangleq \delta_{\textrm{i}} (t) \left( g_1 v(t) + g_2 \dot v_{\rm f}(t) \right) s_{\rm n}(t),\quad t \geqslant 0,\label{eq:fr}\\
 \tau_{\rm r}(t) &\triangleq (c(t)-p_{\rm o}) \times f_{\rm r}(t), \label{eq:tr}
\end{align}
%
where $g_1$, $g_2 \in \mathds R^{+*}$ are stiffness and damping gains (in kg/s and kg/s$^2$, respectively), $p_{\rm o}\in \mathds{R}^3$ represents the position of the considered object's center of mass, $s_{\rm n}(t)\triangleq s_{\rm d}(t)/\|s_{\rm d}(t)\|$, and $\|\cdot \|$ denotes the Euclidian norm. Note that the efforts provided by \eqn{fr}--\eqn{tr} is comparable to a linear Kelvin-Voigt contact model, descriptive of a compliant contact reaction between objects $\mathcal A$ and $\mathcal B$ (see \cite{diolaiti2005}, as well as the discussion in \cite{jain2011controlling} on modeling contact impedance). As previously alluded to, by allowing object meshes to intersect with one another, we mirror the deformation behavior expected of real-life objects at points of contact. For the purpose of numerical simulations, we built upon the models defined in \cite{diolaiti2005} and \cite{erickson2003contact} to define appropriate parameter values.
% Such models who have both based their models on values acquired through real-life experimentation. Their values are used in other setups, such as the simulator designed by \cite{flores2010contact}.
% [ym]CHECK whattever
Algorithm \ref{alg:reaction_efforts} describes the manner in which we perform the computation of \eqn{fr}--\eqn{tr} when an overlap is detected.
%
%\begin{figure}
%	\centering
%	\input{images/overlap_forces_01.pdf_tex}
%	\caption{Forces and torques resulting from a collision and the force application point ${c}$.}
%	\label{fig:overlap_forces}
%\end{figure}
%
\RestyleAlgo{boxed}
\LinesNumbered
\SetAlFnt{\footnotesize}
\IncMargin{0.4em}
\begin{algorithm}[t]
	\DontPrintSemicolon	
	\SetKwProg{Fn}{Function}{}{}
	\Fn{ComputeOverlap($\texttt{shape}_{\texttt{A}}$, $\texttt{shape}_{\texttt{B}}$)}
	{
		$\texttt{mesh}_\texttt{ov} = \texttt{bool\_intersect} \left( \texttt{mesh}_{\texttt{A}}, \texttt{mesh}_{\texttt{B}} \right)$\\
		\BlankLine\BlankLine
		\tcc{Sort triangles according to corresponding object}
		$\texttt{tri}_\texttt{A} = \texttt{extract} \left( \texttt{mesh}_{\texttt{ov}}, \texttt{mesh}_{\texttt{A}} \right)$\\
		$\texttt{tri}_\texttt{B} = \texttt{extract} \left( \texttt{mesh}_{\texttt{ov}}, \texttt{mesh}_{\texttt{B}} \right)$\\
		\tcc{Determine depth weight of individual triangles}
		$s_{\text d}(t) \leftarrow 0$\\
		\ForEach{$\texttt{v}_\texttt{A} \in \texttt{tri}_\texttt{A}$}
		{
			$\texttt{depth}_{\texttt{v}_\texttt{A}} = \| \texttt{center}\left( \texttt{tri}_\texttt{A} \right) - c(t) \|$\\
			$\omega_{\texttt{v}_\texttt{A}} \leftarrow \frac{1}{3} \texttt{surf} \left( \texttt{v}_\texttt{A} \right) \times \texttt{depth}_{\texttt{v}_\texttt{A}}$\\
			$\omega_\texttt{A} \leftarrow \left\{\omega_\texttt{A}; \omega_{\texttt{v}_\texttt{A}} \right\}$\\
			$s_{\text d} \leftarrow s_{\text d} + \omega_{\texttt{v}_\texttt{A}} n_{\texttt{v}_\texttt{A}}$\\
		}
	
		\ForEach{$\texttt{v}_\texttt{B} \in \texttt{tri}_\texttt{B}$}
		{
			$\texttt{depth}_{\texttt{v}_\texttt{B}} = \| \texttt{center}\left( \texttt{tri}_\texttt{B} \right) - c(t) \|$\\
			$\omega_{\texttt{v}_\texttt{B}} \leftarrow \frac{1}{3} \texttt{surf} \left( \texttt{v}_\texttt{B} \right) \times \texttt{depth}_{\texttt{v}_\texttt{B}}$\\
			$\omega_\texttt{B} \leftarrow \left\{\omega_\texttt{B}; \omega_{\texttt{v}_\texttt{B}} \right\}$\\
			$s_{\text d} \leftarrow s_{\text d} - \omega_{\texttt{v}_\texttt{A}} n_{\texttt{v}_\texttt{A}}$\\
		}
		
%		\BlankLine\BlankLine
%		$s \leftarrow \frac{s_{\text d}}{\|s_{\text d}\|} \times \texttt{volume}\left(\texttt{mesh}_\texttt{ov}\right)$
		
		\BlankLine\BlankLine
%		\tcc{Apply PD Control (see \sec{reaction_efforts})}
        \tcc{Compute virtual impedance (see \sec{reaction_efforts})}
		$f_{\rm r} \leftarrow \frac{\texttt{IMP}_{\texttt{col}}.\texttt{correction}\left( \|{s_\text{d}}\| \right)}{\|{s_\text{d}}\|} s_\text{d}$
		
		\BlankLine\BlankLine
		\tcc{Return forces and torques acting on objects $\mathcal A$ and $\mathcal B$}
		\Return $\left[ \begin{matrix}
					{f}_{\rm r}\\\left({c} - {p}_{\mathcal{A}}\right) \times {f}_{\rm r}
				\end{matrix}\right]$,
				$\left[ \begin{matrix}
					-{f}_{\rm r}\\\left({c} - {p}_{\mathcal{B}}\right) \times -{f}_{\rm r}
				\end{matrix} \right]$
	}
	
	\BlankLine\BlankLine
	\KwResult{
			$\left[ \begin{matrix}
				{f}_{\mathcal{A}}\\{\tau}_{\mathcal{A}}
			\end{matrix}\right]$,
			$\left[ \begin{matrix}
				{f}_{\mathcal{B}}\\{\tau}_{\mathcal{B}}
			\end{matrix} \right]$}
	
	\caption{Computation of reaction efforts magnitude and direction. }
	\label{alg:reaction_efforts}
\end{algorithm}
% 
	\section{Numerical Simulation}\label{sec:simulation}
%
We use \texttt{Bullet} as a basis for our numerical rigid body simulation. Its modular design has allowed us to quickly replace relevant portions of their solver (pertaining to contact detection and contact efforts) with the proposed approach. In particular, we have replaced the constraint solver used by \texttt{Bullet} with our penalty-based solver (as described in Algorithm \ref{alg:reaction_efforts}). In addition, we replaced the Euler integration with an Ordinary Differential Equation (ODE) solver based on the explicit predictive-corrective Runge-Kutta(4,5) method (\cite{dormand1980family}), \texttt{ode45}, used to integrate \eqn{dyn_trans}--\eqn{dyn_rot}, where efforts $f(t)$, $\tau(t)$, $t\geqslant0$, are composed of reaction efforts provided by \eqn{fr}--\eqn{tr}, gravity's acceleration, and friction forces. To compute the latter, we rely on a Coulomb Friction model adapted from \cite{liu2015experimental}. To circumvent possible issues stemming from the usual discontinuity, we substitute a hyperbolic tangent in the place of the signum function, as described hereafter,
%
\begin{align}
	f_{\rm f}(t) = -\Big( \mu \|f_{\rm n}(t)\| \tanh \left(\gamma\|\vvR\| \right) + \beta \|v_{\rm r}(t)\|&\Big) v_{\rm rn}(t), \nonumber \\
	t\geqslant0&,
\end{align}
%
\noindent where $f_{\rm f}(t) \in \mathds{R^3}$ is the friction force (in N), $\mu$, $\beta \in \mathds{R^+}$ represent the Coulomb friction coefficient and the viscous friction coefficient, respectively, $\gamma \in \mathds R^+$ is a scaling factor, $f_{\rm n}(t)\in \mathds{R}^3$ is the reaction force, normal to the surface of contact, $v_{\rm r}(t) \in \mathds{R}^3$ is the relative velocity at contact, and $v_{\rm rn}(t)\triangleq v_{\rm r}(t) /\|v_{\rm r}(t) \|$. The scaling factor $\gamma$ is used to adjust the slope of the zero crossing. In the limit that $\gamma\rightarrow+\infty$, one recovers the usual (discontinuous) model. As in \cite{keller1994}, we define the normal contact force acting on an object as $f_{\rm n}(t) \triangleq f_{\rm r}(t)$, i.e. the previously computed repulsive force.

%
%\begin{figure}%{0.45\columnwidth}
%	\centering
%	\includegraphics[width=1\columnwidth]{images/wall_collision.png}
%	\caption{Object colliding with a wall. \todo{TODO: Check if this section will be kept. If so, take images in Gazebo}}
%	\label{fig:wall_collision}
%\end{figure}%
%%
%\begin{figure}%{0.45\columnwidth}
%	\centering
%	\includesvg[width=1\columnwidth]{images/wall_collision_pos.svg}
%	\caption{Comparison of numerically simulated trajectory and analytically exact trajectory. \todo{TODO: Check if this section will be kept. If so, take images in Gazebo. Else replace with distance of object and gripper along test trajectory, show that little wobble occurs, if any}}
%	\label{fig:wall_collision_graph}
%\end{figure}
%%
%
%\subsection{Wall Impact} \label{sec:wall}
%%
%As a first verification of the proposed concept, we simulate a situation in which a rigid body in a gravity field collides with a vertical wall, then comes to rest on a horizontal ground (see \fig{wall_collision}). The object chosen is a rectangular rigid body of a mass of $0.5$ kg, gravity's acceleration is set to be $9.81$ m/s$^{2}$, friction parameters are chosen to be $\mu=1.25$ s/m, $\beta=1.4$ s/m, $\gamma=100$, the impedance gains in \eqn{fr} are set to $g_1 = 1654$ kg/s$^{2}$, $g_2 = 21$ kg/s, the low-pass filter parameters in \eqn{lowpass} are set to $\zeta=1/\sqrt2$ and $\omega_{\rm n}=10$s$^{-1}$. The transient temporal gain parameters are set to $g_\textrm{i} = 743\frac{1}{\textrm{s}}$ and $g_\textrm{i} = 3.98$ Finally, the object is set with an initial horizontal velocity of $1$ m/s in the direction of the wall. The trajectory obtained using the proposed approach is shown in \fig{wall_collision} and in blue in \fig{wall_collision_graph}. It is compared to the exact trajectory in \fig{wall_collision_graph} (exact trajectory shown in orange). This comparison illustrates that the proposed approach provides a fair reflection of the expected behavior. \todo{TODO: Create image}

\subsection{Simulated Grasps}\label{sec:grasp}
%
To evaluate our approach in various grasping scenarii, we have implemented three grippers, representative of a range of structural complexity. The first is a simple 2-fingered prismatic system, the second a 2-fingered revolute gripper, and the final one is a 3-fingered, 3-dimensional revolute gripper. For testing purposes, we have computed the Lagrangian equations of motion for each model, and will be using these instead of the standard constraint-based approach of \texttt{Bullet} to compute the joints' dynamics. Figure \ref{fig:grippersobjects} shows the grippers and objects used for testing.
%
\begin{figure}
	\centering
	\scalebox{1}[0.9]{\includegraphics[width=1.0\linewidth]{images/grippers_objects}}
	\caption{Grippers (top) and objects (bottom) used for performance evaluation.}
	\label{fig:grippersobjects}
\end{figure}
%
Three test operations are performed for each gripper individually, with each operation aiming to grasp a particular object. The target objects are a cube, a sphere, and a Stanford Bunny model (\cite{levoy2005scanning}). The mass of each object is set to $0.2$kg. We compare the results of our approach to those obtained with \texttt{Bullet}'s default constraint-based method. Simulation length is chosen to be $10$s, and the following parameters are used, $g_1 = 740 \frac{\text{N}}{\text{cm}}$, $g_2 = 50 \frac{\text{N}}{\text{cms}}$, $g_\text{i} = 250$, $d_\text{t} = 2 \text{mm}$, $d_\text{t} = 2 \text{mm}$, $\mu = \beta = 0.2$, $\gamma = 1$.

The grippers start in an open configuration, the object placed in such a manner that it is ready to be grasped. Once the simulation begins, the fingers enclose the object and exert a constant force from each side for the duration of the simulation. In our experience, using constraint-based approaches has consistently led to failure in finite time for grippers including revolute joints, as shown in \fig{grasp_explosion} where the constraint based approach fails at around simulation time $t=7.7$s (large contact force spike, leading to simulation failure). Here, a 2-finger revolute gripper grasps a ball.
%The constraint-based method fails to keep the object at the expected position for both rotational grippers. In particular, after a finite simulation time, a sudden and substantial increase in volume overlap occurs, and the object is accelerated away with high velocities.
% [ym]DISCUSS Is this a repeatable behavior? If so mention it? Does it happen every time? If not, would have needed some stats (breaks 80% of the time with this configuration). That only happens for the rotational grippers, not prismatic one, correct? That should be stated explicitly, it's ambiguous right now. Also, one figure showing efforts or acceleration or energy spiking up is needed.

This behavior does not occur with the proposed approach, which results in the object stably remaining in the gripper's center, between the enclosing fingers. See \fig{grasp_throw} for an illustration of measured contact forces. To better illustrate the shape of the efforts produced upon impact (transient) in comparison to the steady-state value corresponding to contact efforts, we selected a particular time-window.
%
\begin{figure}
	\centering
	%\def\svgwidth{\columnwidth}
	\scalebox{1}[1]{\includegraphics[width=\columnwidth]{svg-inkscape/contact_forces_4.pdf}}
	\vspace{-0.7cm}
	\caption{Comparison of contact forces generated by \texttt{Bullet}'s Constraint-based LCP solver and our approach. Dashed lines denote results from the constraint-based simulator, solid lines from the penalty-based method.}
	\label{fig:grasp_throw}
\end{figure}
%
\begin{figure}
	\centering
	%\def\svgwidth{\columnwidth}
	\scalebox{1}[0.9]{\includegraphics[width=\columnwidth]{svg-inkscape/contact_force_explosion.pdf}}
	%\vspace{-0.7cm}
	\caption{Grasp failure of constraint-based approach.}
	\label{fig:grasp_explosion}
\end{figure}
%
As shown in \fig{grasp_throw}, the transient efforts upon contact differ between penalty- and constraint-based methods. However, both approaches converge to an identical contact force magnitude once the initial impact has subsided and bodies have reached their steady-state configurations.
% Not shown here are the instances later on where the constraint-based approach explodes.
% [ym]DISCUSS (reference sr hand example) yeah no, you need to show it, you can't expect them to take your work for it. Even one single example.
% To prevent this scenario, our approach decreases mechanical contact stiffness. Mirroring the deformation of real-life objects, a certain amount of object overlap is tolerated, as long as contact forces attempt to minimize said intersection.
% [ym]CHECK we've said that already, and they probably don't care.

% Our aim is a simulator that mirrors reality as much as possible, while not jeopardizing the possibility of execution. The following analyses the limits of our approach. 
	\subsection{Parameter-space exploration}
\label{sec:comparison}
% [ym]CHECK no, this title is wrong, you have comparison to constraint in previous subsection too, and here you explore parameter space
%
To better assess merit of the proposed methodology, we explore grasp behaviors achieved in different areas of the method's parameter space, focusing on parameters which have shown to have the greatest impact on this behavior. For this purpose,
% A series of tests have been run to determine the viable execution parameters of our approach. For this,
we use the 3-pronged rotational gripper shown in \fig{grippersobjects}, and apply to it a pre-defined sinusoidal pattern. Over the course of this simulation, we measure the kinetic energy of the system, as computed from the objects' simulated linear and angular speeds, and compare it to the expected kinetic energy, as computed from the prescribed movement velocity and assuming no relative movements between object and gripper. In an ideal case, the object would remain at a constant relative position from the gripper, and the ratio of considered kinetic energies would be 1. In situations in which the grasp is less-than-perfect, parasite movements emerge (\emph{wobble creep}), leading to additional (and undesirable) kinetic energy. In the extreme, abrupt large efforts may emerge, disrupting simulation (as is the case when using the constraint method in \fig{grasp_explosion}).
% Below we have listed several graphs comparing various parameters to each other, with the ratio between energy inserted via motion and energy measured as a quality metric. The closer the ratio is to 1, the better the simulation reflects reality.
% [ym]CHECK check my language, whether I lie or not. Your mention of "quality metric" is obscure.

%We have explored a meaningful range of the methods' parameter space, relevant results are discussed hereafter. 
As shown in \fig{impcoeffstiffnesscomp}, high contact stiffness $g_\text{1}$ and gain multiplier values $g_\text{i}$ may lead to kinetic energy creation, which implies colored simulation results. Setting these parameters to particularly high values can be verified to result in a jittery behavior for the grasped object, with computed repulsive forces not only reducing contact overlap, but also accelerating objects away from each other.
% At high gains, this behavior is observed even when initializing the simulation with two objects merely touching each other.
\fig{depthstiffnesscomp} illustrates the impact on behavior of the depth parameter $d_\text{t}$ and of gain multiplier $g_\text{i}$. Behaviors similar to those observed in the previous example can be seen. Excessively small values of the depth parameter result in a clear increase in kinetic energy over time. It can be observed that decreasing admissible depth leads to greater mechanical stiffness, which in turn leads to a jittery behavior of the grasped object.
%
\begin{figure}
	\centering
	\scalebox{0.9}{\includegraphics[width=1.0\columnwidth]{svg-inkscape/impcoeff_stiffness_comp_dot}}
	\caption{Kinetic energy ratio comparison for differing values of gain multiplier $g_\text{i}$ and stiffness $g_1$. The green dot indicates the parameters used in \fig{grasp_throw}.}
	\label{fig:impcoeffstiffnesscomp}
\end{figure}
%
\begin{figure}
	\centering
	\scalebox{0.9}{\includegraphics[width=1.0\columnwidth]{svg-inkscape/depth_stiffness_comp_dot}}
	\caption{Kinetic energy ratio comparison for differing values of gain multiplier $g_\text{i}$ and admissible depth $d_\text{t}$. The green dot indicates the parameters used in \fig{grasp_throw}.}
	\label{fig:depthstiffnesscomp}
\end{figure}
%\begin{figure}
%	\begin{subfigure}{0.5\columnwidth}
%		\centering
%		\includegraphics[width=1.0\columnwidth]{svg-inkscape/impcoeff_stiffness_comp_dot}
%	\end{subfigure}%
%	\begin{subfigure}{0.5\columnwidth}
%		\centering
%		\includegraphics[width=1.0\columnwidth]{svg-inkscape/depth_stiffness_comp_dot}
%	\end{subfigure}
%\end{figure}
%

	\subsection{Numerical Simulation of ShadowRobot Hand}\label{sec:sr_hand}

The grippers used during previous tests have all been developed specifically for the above simulations. They were designed to test the approach's performance under a number of varying circumstances, and compare the results to those obtained using alternate grasping simulation methods. To illustrate efficacy of our approach using a real-world, complex gripper, we have applied our algorithm to a grasping scenario considering the ShadowRobot Hand (see Figure \ref{fig:srhand}). This anthropomorphic gripper features 24 joints and 20 DoFs. As done in previous examples, we compared performance of the proposed approach to that of a constraint-based method. As was the case with simpler grippers, the constraint-based method displays a singular behavior after some simulation time, whereas the proposed penalty-based algorithm behaves as desired. See \fig{srhandcontactforces}, which shows the constraint-based approach failing at about time instant $t=5.8$s.
%
% As demonstrated here, our approach can be utilized with more complex grasping scenarii, including configurations which mimic the complexities of human hand grasping.
% [ym]CHECK yeah they saw.
%
\begin{figure*}
	\centering
	\scalebox{1}[0.95]{\includegraphics[width=\textwidth,trim={0 3cm 0 7cm},clip]{images/sr_bunny_hold_4}}
	\caption{ShadowRobot Hand grasping the Stanford bunny model (\cite{levoy2005scanning}).}
	\label{fig:srhand}
\end{figure*}
%
\begin{figure}
	\centering
	\scalebox{1}[0.90]{\includegraphics[width=\linewidth]{images/sr_hand_contact_forces}}
	\caption{Contact force comparison for ShadowRobot Hand.}
	\label{fig:srhandcontactforces}
\end{figure}
%\begin{figure}
%	\centering
%	\includegraphics[width=\columnwidth]{images/sr_bunny_hold_4}
%	\caption{ShadowRobot Hand grasping the Stanford bunny model (\cite{levoy2005scanning}).}
%	\label{fig:srhand}
%\end{figure} 
	% \vspace{-0.5em}
\section{Conclusion}
% \vspace{-0.5em}
Recent advances in multimodal single-cell technology have enabled the simultaneous profiling of the transcriptome alongside other cellular modalities, leading to an increase in the availability of multimodal single-cell data. In this paper, we present \method{}, a multimodal transformer model for single-cell surface protein abundance from gene expression measurements. We combined the data with prior biological interaction knowledge from the STRING database into a richly connected heterogeneous graph and leveraged the transformer architectures to learn an accurate mapping between gene expression and surface protein abundance. Remarkably, \method{} achieves superior and more stable performance than other baselines on both 2021 and 2022 NeurIPS single-cell datasets.

\noindent\textbf{Future Work.}
% Our work is an extension of the model we implemented in the NeurIPS 2022 competition. 
Our framework of multimodal transformers with the cross-modality heterogeneous graph goes far beyond the specific downstream task of modality prediction, and there are lots of potentials to be further explored. Our graph contains three types of nodes. While the cell embeddings are used for predictions, the remaining protein embeddings and gene embeddings may be further interpreted for other tasks. The similarities between proteins may show data-specific protein-protein relationships, while the attention matrix of the gene transformer may help to identify marker genes of each cell type. Additionally, we may achieve gene interaction prediction using the attention mechanism.
% under adequate regulations. 
% We expect \method{} to be capable of much more than just modality prediction. Note that currently, we fuse information from different transformers with message-passing GNNs. 
To extend more on transformers, a potential next step is implementing cross-attention cross-modalities. Ideally, all three types of nodes, namely genes, proteins, and cells, would be jointly modeled using a large transformer that includes specific regulations for each modality. 

% insight of protein and gene embedding (diff task)

% all in one transformer

% \noindent\textbf{Limitations and future work}
% Despite the noticeable performance improvement by utilizing transformers with the cross-modality heterogeneous graph, there are still bottlenecks in the current settings. To begin with, we noticed that the performance variations of all methods are consistently higher in the ``CITE'' dataset compared to the ``GEX2ADT'' dataset. We hypothesized that the increased variability in ``CITE'' was due to both less number of training samples (43k vs. 66k cells) and a significantly more number of testing samples used (28k vs. 1k cells). One straightforward solution to alleviate the high variation issue is to include more training samples, which is not always possible given the training data availability. Nevertheless, publicly available single-cell datasets have been accumulated over the past decades and are still being collected on an ever-increasing scale. Taking advantage of these large-scale atlases is the key to a more stable and well-performing model, as some of the intra-cell variations could be common across different datasets. For example, reference-based methods are commonly used to identify the cell identity of a single cell, or cell-type compositions of a mixture of cells. (other examples for pretrained, e.g., scbert)


%\noindent\textbf{Future work.}
% Our work is an extension of the model we implemented in the NeurIPS 2022 competition. Now our framework of multimodal transformers with the cross-modality heterogeneous graph goes far beyond the specific downstream task of modality prediction, and there are lots of potentials to be further explored. Our graph contains three types of nodes. while the cell embeddings are used for predictions, the remaining protein embeddings and gene embeddings may be further interpreted for other tasks. The similarities between proteins may show data-specific protein-protein relationships, while the attention matrix of the gene transformer may help to identify marker genes of each cell type. Additionally, we may achieve gene interaction prediction using the attention mechanism under adequate regulations. We expect \method{} to be capable of much more than just modality prediction. Note that currently, we fuse information from different transformers with message-passing GNNs. To extend more on transformers, a potential next step is implementing cross-attention cross-modalities. Ideally, all three types of nodes, namely genes, proteins, and cells, would be jointly modeled using a large transformer that includes specific regulations for each modality. The self-attention within each modality would reconstruct the prior interaction network, while the cross-attention between modalities would be supervised by the data observations. Then, The attention matrix will provide insights into all the internal interactions and cross-relationships. With the linearized transformer, this idea would be both practical and versatile.

% \begin{acks}
% This research is supported by the National Science Foundation (NSF) and Johnson \& Johnson.
% \end{acks}
	
	\section{REFERENCES}
	\bibliographystyle{adaptive}
	\begingroup
	\renewcommand{\section}[2]{}%
	\bibliography{IEEEabrv,IEEEexample}
	\endgroup
	
\end{document}
% 