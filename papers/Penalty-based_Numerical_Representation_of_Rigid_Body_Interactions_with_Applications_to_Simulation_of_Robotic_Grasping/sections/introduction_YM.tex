\section{Introduction}\label{sec:introduction}
%
The manufacturing industry has been undergoing a radical paradigm shift over the past decade (referred to as fourth industrial revolution in \cite{schuh2017}), brought about by the development of Information and Communications Technology (ICT). The developments involved have allowed production lines to become more agile, making it possible for instance to address specialised small product batches in a cost-effective manner (ideally, at mass production cost). A key operation involved in a meaningful range of such production line is that of assembly. Its automation by robotic means implies the necessity for the involved robot to locate, grasp, then place specific parts. The grasping operation is, in general, a non-trivial one. It typically involves interactions between a robotic arm with a meaningful number of Degrees of Freedom (DoFs), an object being grasped (whose physical properties of inertia may be either partially uncertain, or not straightforward to represent in a concise manner), and an assembly onto which the grasped object must be placed or fixed. The grasping itself involves friction forces, which are notoriously challenging to accurately model (\cite{haessig1991}). The final step involves exchange of effort between the robotic arm, the object grasped, and the assembly. Both the arm and the assembly being typically rigidly mounted, the operation involves a closed mechanical chain, which further complicates the process.

These different aspects contribute to rendering the reconfiguration process of a production line involving such assembly tasks rather complex. While it is possible to explore this design in a numerical setting, some of the dimensions involved remain challenging to faithfully, accurately model in numerical simulations. Grasping constitutes one of the more challenging aspects. A number of applications have been developed to address this problem, including \texttt{GraspIt!} and \texttt{OpenRAVE} (\cite{graspit, openrave}). They provide a number of functions, a significant number of them accounting for static quality of considered grasps, reflecting how well a given gripper (characterised by geometry, configuration, actuation) is able to grasp a considered object. However, such static considerations fail to address the dynamic nature of usual manipulation scenarii encountered in practice, where objects are picked, moved, placed, dynamically. A number of dynamic performance metrics have been defined (\cite{roa_grasp_metrics}), their assessment requires consideration of the dynamic nature of the considered system. Existing grasping applications commonly rely on numerical physics engines such as the Open Dynamics Engine (\texttt{ODE} \cite{smith2005}) and \texttt{Bullet} (\cite{bullet_physics_library}). The \texttt{GraspIt!} toolbox relies on Bullet, whereas \texttt{OpenRAVE} has an interface allowing the use of either ODE and Bullet.

To simulate dynamic rigid body interactions, these engines rely on constraint-based methods (\cite{baraff1993non}). In particular, for each discrete time-step of the simulation, a set of constraints is defined, reflecting properties and situation of the considered gripper, the object being grasped, and the environment. Each constraint is intended to be descriptive of the interactions between two given rigid bodies, defining reaction forces descriptive of contacts between such bodies (including impacts, support, or reflecting exchanges of efforts in joints). From the considered constraint is inferred an appropriate effort being applied on the involved objects. For instance, hinge joint constraints define reactive forces that ensure the two considered bodies are kept in contact at the hinge position; they are left free to rotate around the specified axis, but other relative movements are entirely compensated by the constraint effort. Of particular relevance to grasping are contact constraints, which prevent objects from physically overlapping. They are implemented in manner that is similar to the aforementioned hinge constraint. In particular, upon detection of the emergence of one such contact, a constraint enforcing non-overlap is added to the previously considered ones (see the discussion in \cite{baraff1993non}). The constraint prevents body movement in the direction of the contact point, preventing overlap. Different approaches may be used to infer appropriate efforts from the defined set of constraints, one of the most commonly encountered method formulates a Linear Complementary Problem (LCP \cite{cottle2009}), which may be solved using a number of different techniques (see \cite{anitescu_time_stepping_lcp_solver,lemke_lcp_solver}). Once evaluated, constraint forces are applied to concerned rigid bodies, thereby affecting their trajectory in the desired manner (see discussion in \cite{extending_ode_robots}).

Constraint-based approaches perform appropriately for a certain range of scenarii. However, a number of situations may give rise to substantial issues. For instance, rapid relative movement of objects, if simulation time steps are not carefully adjusted, may result in actual overlap between the bodies; this overlap occurring before the contact is detected and appropriate constraint forces are generated to avoid this situation. A straightforward mitigating measure to such issues consists in carefully monitoring for such abhorrences, and adjusting (that is, reducing) the simulation time-step to a degree that is commensurate with relative velocity of the concerned objects. However, such methods add considerable complexity and offer no guarantee of success (\cite{engine_simulation}). Another situation in which such overlaps commonly occur (when working in discrete time numerical simulation) is that involving closed mechanical chains. Grasping, when a gripper applies efforts on an object from different, opposing directions falls into that category. The situation is particularly prevalent when considering force-closure grasp (\cite{nguyen1988constructing}). Successfully enforcing non-overlap constraints in such situations requires particularly small simulation time steps. In the event that an overlap ends up occurring, LCP based methods typically fail, as such situations are not foreseen (\cite{baraff_analytical}).

A number of results can be found in  the literature to address such issues. A common approach involves accounting for the possibility of overlaps (which is essentially inherent to discrete-time simulation), and compute reaction forces in such a manner that, when such situation do occur, they only exist in short transient state, rapidly replaced with a steady-state situation in which contacts and interactions faithfully their real-world, physical reality. The approach in \cite{yamane2008}, for instance, defines a set of criteria to overcome inter-body penetrations. These result in the application of a repulsive force to overlapping objects, proportional to their relative velocity, at the point(s) of contact. The approach implies a number of additional steps to the simulation which may prove exceedingly computationally intensive, to the extent that real-time computations are in practice difficult to achieve. In addition, the defined repulsive forces can in some configurations lead to the emergence of a resonance effect, destabilizing force-closed grasps. Numerical simulation of grasping using such an approach typically requires the addition of artificial, large damping efforts diffusing excess energy (see \cite{engine_simulation}). Another approach involves the replacement of constraints with impulses to describe interactions, as in \cite{impulse_based}. The approach is able to account for overlap of rigid bodies. However, the reliance on impulses may lead to inaccuracies (see discussion in \cite{impulse_based}). A manner to directly address and account for overlaps consists in treating such instances not as an anomaly, but rather as an error to be reduced (or regulated). This can be pursued by defining repulsion forces in light of these errors, in a manner that concerned objects are pushed apart, as described in \cite{drumwright_penalty_based}. Penalty-based methods offer a rapid and robust method to address simulation scenarii in which objects intersect with each other, solution allowing to consider thousands of objects in real-time have been demonstrated (\cite{fast_penalty_based,drumwright_fast_penalty}). However, the penalty parameters must be selected for specific objects and collision scenarios. Should a scenario change too drastically, improperly tuned parameters could lead to oscillating behavior, as discussed in \cite{mirtich1998}. As grasping presents a situation where objects should remain in contact at specific positions, these limitations are not of consequence.
In the following, we build upon this penalty-based perspective, to develop an algorithm supporting reliable numerical simulation of robotic grasping. In particular, we ground the definition of the penalty efforts, computed from measured overlaps, in considerations of mechanical impedance (including stiffness and damping), designed to reflect the expected behavior of the considered objects in their interactions.

This paper is structured as follows. \sec{approach} provides an overview of the proposed approach, and results of illustrative numerical simulations are provided in \sec{simulations}. \sec{conclusion} concludes this paper. 