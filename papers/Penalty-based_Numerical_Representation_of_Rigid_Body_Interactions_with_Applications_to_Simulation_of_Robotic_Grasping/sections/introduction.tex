\section{Introduction}\label{sec:introduction}
%
%\baselineskip10pt
Over the past decade, the manufacturing industry has been undergoing a radical paradigm shift (referred to as fourth industrial revolution, \cite{schuh2017}), brought about by the development of Information and Communications Technology (ICT). The developments involved have allowed production lines to become more agile, making it possible for instance to address specialized, small product batches in a cost-effective manner (ideally, at mass production cost). A key operation involved in such production lines typically is that of assembly. Its automation by robotic means implies the necessity for the involved systems to locate, grasp, and then assemble parts in a specific configuration. The grasping operation constitutes, from a dynamical modeling perspective, a non-trivial task. It involves interactions between a robotic arm, featuring a meaningful number of Degrees of Freedom (DoFs), an object being manipulated (whose physical properties of inertia might be either partially uncertain or not straightforward to concisely represent), and an assembly onto which the grasped object must be placed or affixed. The final step involves an exchange of effort between the robotic arm, the grasped object, and the target assembly. This generally involves friction forces, which are notoriously challenging to model accurately (\cite{haessig1991}). With both the arm and the assembly usually being rigidly mounted, the operation involves a closed mechanical chain, which further complicates the process.

The interplay between these different aspects is such that the reconfiguration process of a production line involving assembly tasks oftentimes constitutes a challenging, effort-intensive proposition. While it is possible to explore this design in a numerical setting, some of the dimensions involved remain challenging to faithfully and accurately model in simulation. Grasping constitutes one of the more challenging aspects of such simulations. A number of applications have been developed to address this problem, including \texttt{GraspIt!} (\cite{graspit}) and \texttt{OpenRAVE} (\cite{openrave}). They provide a number of functions, most of them accounting for static quality of considered grasps. The involved quality metrics provide a reflection of how well a given gripper, characterised by its geometry, configuration, and actuation, is able to grasp a considered object. However, such static considerations come short of addressing the inherently dynamic nature of manipulation scenarii encountered in practice, where objects are picked, moved, and placed dynamically. Assessing quality of a grasp in such a situation requires consideration of this dynamic dimension. To that end, existing grasping applications commonly rely on numerical physics engines, such as the Open Dynamics Engine (\texttt{ODE}, \cite{smith2005}) or \texttt{Bullet} (\cite{bullet_physics_library}). The \texttt{GraspIt!} toolbox relies on \texttt{Bullet}, whereas \texttt{OpenRAVE} offers an interface allowing the use of either \texttt{ODE} or \texttt{Bullet}.

To simulate dynamic rigid body interactions, these engines rely on constraint-based methods (\cite{baraff1993non}). For each discrete time-step of the simulation, a set of constraints is defined to reflect the respective situations of the considered gripper, of the object being grasped, and of the environment. Each constraint is descriptive of interactions between two given rigid bodies, allowing to compute reaction forces emerging from contacts between them (including impacts, support, or reflecting exchange of efforts in joints). From the considered constraint is inferred an appropriate effort to be applied to the involved objects. For instance, hinge joint constraints define reactive forces that ensure the two considered bodies are kept in contact at the hinge position; they are left free to rotate around the specified axis, but other relative movements are impeded by the constraint effort. Of particular relevance to grasping are \emph{contact constraints}, which prevent objects from physically overlapping. They are implemented in a manner that is similar to the aforementioned hinge constraint. More specifically, upon detection of the emergence of one such contact, a constraint which enforces non-overlap is added to the previously considered ones (see the discussion in \cite{baraff1993non}). The constraint prevents body movement in the direction of the contact point, thus preventing overlap. Different approaches may be used to infer appropriate efforts from the defined set of constraints. One of the most commonly encountered method formulates a Linear Complementary Problem (LCP, \cite{cottle2009}), which may be solved using a number of different techniques (see \cite{anitescu_time_stepping_lcp_solver,lemke_lcp_solver}). Once evaluated, constraint forces are applied to concerned rigid bodies, thereby affecting their trajectory in the desired manner (the interested reader is referred to \cite{extending_ode_robots} for additional details).

Constraint-based approaches perform appropriately for a certain range of scenarii. However, a number of specific situations may give rise to substantial issues. For instance, rapid relative movement of objects, if simulation time steps are not carefully adjusted, may result in an effective overlap between the bodies; this overlap occurring before contact is detected and appropriate constraint forces generated to prevent the situation. A straightforward mitigating measure to such issues consists in carefully monitoring for such problems, and adjusting (that is, reducing) the simulation time-step to a degree that is commensurate with the relative velocity of concerned objects. However, such methods add considerable complexity and offer no guarantee of success (\cite{engine_simulation}). Another set of situations in which such overlaps may occur (within the context of discrete-time numerical simulation) is those involving closed mechanical chains. Grasping, when a gripper applies efforts on an object from different, opposing directions, gives rise to such situations. The issue is particularly prevalent when considering force-closure grasps (\cite{nguyen1988constructing}), for which successfully enforcing non-overlap constraints requires exceedingly small simulation time steps. In the event that an overlap ends up occurring, LCP based methods typically fail (\cite{baraff_analytical}). A detailed analysis of the performance of constraint-based rigid body simulators can be found in \cite{taylor2016analysis}, which provides an overview of most common factors leading to numerical instability in modern rigid body simulation.
\\[5pt]
\indent A number of results can be found in the literature to overcome such problems. A common approach involves accounting for the possibility of overlaps (which is largely inherent to discrete-time simulation) and computing reaction forces in such a manner that, when such situations do emerge, they only occur as short transient events, rapidly replaced with a steady-state situation in which contacts faithfully reflect the expected real-world behavior. The approach in \cite{yamane2008}, for instance, defines a set of criteria to overcome inter-body penetrations. These result in the application of a repulsive force to overlapping objects, proportional to their relative velocity at the point(s) of contact. Unfortunately, the approach implies a number of additional steps to the simulation which may prove computationally intensive, to the extent that real-time computations are in practice difficult to achieve. In addition, the defined repulsive forces can in some configurations lead to the emergence of a resonance effect, destabilizing force-closed grasps. Numerical simulation of grasping using such an approach typically requires the addition of artificial, large damping efforts diffusing excess energy (see \cite{engine_simulation}). Another existing approach involves the replacement of constraints with impulses to describe interactions, as in \cite{impulse_based}. The approach is able to account for overlap of rigid bodies. However, the reliance on impulses may give rise to a number of issues (including loss of accuracy, as discussed in \cite{impulse_based}). A manner to directly address and account for overlaps consists in treating such instances not as anomalies, but rather as errors to be reduced (or regulated). This can be pursued by defining repulsion forces in light of such errors, in a manner that concerned objects are pushed apart, as described in \cite{drumwright_penalty_based}. Penalty-based methods offer a lightweight, computationally effective approach to address simulation scenarii in which objects may find themselves intersecting with each other. Their performance has been demonstrated with implementations that allow to consider thousands of objects in real-time (\cite{sagardia_penalty}). However, in most instances, the penalty parameters (which the approach relies on) must be carefully selected for specific objects and collision scenarii, implying the need for expert knowledge and a meaningful effort overhead to achieve desired results. Furthermore, penalty-based approaches have been shown to lead to undesirable oscillating behaviors in a range of situations, as discussed in \cite{mirtich1998}. In the following, we build upon such a penalty-based perspective, proposing measures to circumvent limitations encountered with existing solutions.
\\[5pt]
\indent More specifically, when accounting for contacts in-between objects, we draw from physical considerations to infer appropriate efforts. In particular, though the objects simulated (e.g. in robotic grasping scenarii) are commonly treated as perfectly rigid, this constitutes an approximation of a more complex reality. In practice, when subjected to external efforts, objects undergo a measure of deformation (\cite{stoianovici1996critical}). In instances in which such deformations remain limited, one oftentimes overlooks their presence to facilitate proceedings. In the following instead, in situations in which bodies come in contact with one-another, we treat emerging volume overlaps between objects as zones of mechanical deformation (as may be caused by contact forces). Though not intended to provide a quantitatively faithful depiction of actual deformation phenomena, the approach affords a number of key benefits. Using models descriptive of compliant contacts to inform computation of reaction efforts, the proposed approach grounds its treatment of rigid body interactions (including unnatural configurations emerging from discrete-time simulation) in physical considerations. Building upon compliant contact models (such as those in \cite{diolaiti2005}), we propose reaction effort laws that allow to describe simulation overlaps as compliant, virtual body deformations. Thus imparting contact regions with a measure of mechanical impedance, we promote the emergence of restoring forces and moments that counteract overlaps. Further, the natural inclusion of damping within such impedance addresses the emergence of undesirable oscillatory behaviors.
\\[5pt]
\indent The proposed approach is presented in \sec{approach}. Results of illustrative numerical simulations are provided in \sec{simulation}. \sec{conclusion} concludes this paper. 