\section{Proposed Approach}\label{sec:approach}
%
\baselineskip12pt
We consider the physical interaction of two or more rigid bodies. Each body is defined by its inertia properties (including mass $m\in \mathds R^{+*}$, assuming for simplicity homogeneous mass density distribution, and rotational inertia $I\in \mathds R^{3\times 3}$), a mesh describing the object's geometry as a set of vertices, and relevant friction parameters (as further discussed in \sec{simulation}). Each object's equations of motion are of the following form (\cite{tong2004lectures}),
%
\begin{eqnarray}
m \ddot x(t) &=& f(t),  \quad t\geqslant 0, \label{eq:dyn_trans}\\
I \dot \omega(t) &=& \tau(t) - \omega(t) \times I \omega(t), \label{eq:dyn_rot}
\end{eqnarray}
%
\noindent where $f(t) \in \mathds{R}^3$ is the force (in N) acting on the considered object, $\tau(t) \in \mathds{R}^3$ is the torque (in Nm), $m$ is expressed in kg, $I$ in kg$\cdot\text{m}^2$, $x(t) \in \mathds{R}^3$ represents the object's position (in m), and $\omega(t) \in \mathds{R}^3$ the object's angular velocity in $\text{s}^{-1}$. In the following, we discuss the manner in which we detect the emergence of contacts (and incidentally of overlapping object configurations), then infer appropriate reaction forces such that the overlap is mitigated.

\subsection{Overview}
%
\begin{figure}
	\centering
	\scalebox{1}[0.9]{\includegraphics[width=1\columnwidth]{images/overlap_vectors.pdf}}
	\caption{Schematic representation of interaction force computation, the sphere is labeled $\mathcal{A}$, the cube $\mathcal{B}$. Upon detection of overlap, the mesh describing the overlap volume is identified (dark blue mesh, left). Evaluation of the depth of overlap is done in a discrete manner, breaking down a sample surface in a set of triangles (cyan and orange, right). The contact effort direction of forces acting from object $\mathcal{A}$ on object $\mathcal{B}$ is computed using the normals $n_\mathcal{A}$ and $n_\mathcal{B}$, $c(t)$ is the contact force application point.}
	\label{fig:overlap}
\end{figure}
%
Appropriate simulation of interacting rigid bodies necessarily requires that aspects related to collision detection and reaction efforts be accounted for. If relying on a penalty-based approach, such as that discussed in the following, this is typically achieved by monitoring and detecting situations in which objects overlap; that is, situations in which parts of two or more objects occupy the same space. As such situations do not arise in practice, their occurrence in simulation is undesirable, and penalty-based approaches introduce mitigating measures to overcome them. Such measures revolve around carefully designed interaction forces, developed such that overlap events are limited to transient occurrences.

The method pursued hereafter involves four successive steps. First, object collisions and volume overlaps are determined by monitoring the relative position of bodies whose dynamics are simulated. The individual meshes are intersected, and possible overlaps registered. As a direct comparison of vertices would prove exceedingly computationally expensive, this step is broken down into two stages. \texttt{\textbf 1} Object Mesh pairs in close proximity to each other are identified. \texttt{\textbf 2} Detection of a possible overlap is then performed by use of a standard Axis-Aligned Bounding Box (AABB) collision detection scheme (such as that in \cite{jimenez20013d}). \texttt{\textbf 3} Once an overlap is detected, we compute the corresponding reaction forces. Broadly speaking, the more severe the overlap, the greater the required repulsive displacement effort. Accordingly, the reaction forces we design will be made (in some measure) proportional to this volume (as is commonly the case for penalty-based techniques, see \cite{sagardia_penalty}). \texttt{\textbf 4} Direction of the resulting force vector is defined in such a manner that objects' movements decrease the overlap, and the point of application for this effort is selected. The proposed methodology is summarized in Algoritm \ref{alg:reaction_efforts}. The methodology is described in more detail hereafter.

% Current simulators, such as ODE and Bullet, use the maximum intersection depth of two colliding objects to determine contact points. Our approach instead computes the total intersection volume to compute reaction efforts as well as application points and directions.

\subsection{Characterizing the overlap}
%
We use the \texttt{libigl} library \cite{jacobson2016libigl} for mesh processing, which is instrumental to the proposed approach. First, we examine the meshes from selected potentially overlapping bodies and perform a standard boolean intersection operation. The result produced by \texttt{libigl} consists of a triangle mesh describing the shape of the overlap. Should the operation result in an empty set, we infer that the considered pair of bodies does not intersect and no further computations are necessary.

Should the boolean intersection provide a non-empty shape, we proceed to identifying three reaction effort parameters, the application point $c(t) \in \mathds{R}^3$, $t\geqslant 0$, the reaction effort's direction $s_{\text d}(t)\in \mathds{R}^3 $, and its magnitude $s_{\rm m}(t)\in \mathds{R}$. The point of application is assigned to the geometric center of the previously computed intersection mesh. A simple method to assign $s_{\rm m}(t)$ could involve selecting a value proportional to the scalar overlap volume measurement $v(t) \in \mathds{R}$, as is commonly done in penalty-based approaches. Such a choice leads to a proportional relationship between reaction effort and overlap volume. We propose a different approach, building upon this proportional term, in such a manner that issues commonly encountered in the use of penalty-based approaches are circumvented, as discussed in section \ref{sec:reaction_efforts}.

Assessing an appropriate $s_{\text d}(t)$, $t\geqslant 0$, requires a measure of special attention. In particular, it is necessary that collision forces be designed in such a manner that intersecting objects tend to separate from one another.
%Further, contact points with deeper intersections should result in higher repulsive efforts than shallow ones. [YM]DISCUSS That's related to the magnitude, not the direction?
% We have introduced the following approach for determining collision force direction to satisfy these requirements.
To ensure that this is the case, we approach the problem by considering the previously computed intersection mesh. More specifically, we define two sub-meshes, distinguishing, within the overlap mesh, which triangles originated from which body. For ease of exposition, we consider a case in which only two rigid bodies are overlapping, denoting them as object $\mathcal A$ and $\mathcal B$ (see \fig{overlap}). One sub-mesh is designed to include all triangles overlapping with the mesh of object $\mathcal A$, the other all triangles overlapping with the mesh of object $\mathcal B$ (as illustrated in \fig{overlap}). Then, we iterate over all triangles in subset $\mathcal A$, estimating a weight factors reflecting a triangle's contact depth and the overlap volume it represents. In greater detail, for each triangle $i$ we determine the distance vector $n_{\mathcal{A}_i}(t) \in \mathds{R}^3$ from its center point to the center of the intersection mesh $c(t)$. We then compute a weight $\omega_{\mathcal{A}_i}(t) \in \mathds{R}^+$ reflecting the scalar volume of the pyramid defined by the triangle surface as base and the point $c(t)$ as additional vertex. Once this has been performed for all triangles in the $\mathcal A$ sub-mesh, the same operation is conducted for triangles in subset $\mathcal B$. The direction of application is then computed as the following weighted sum,
%
%\begin{eqnarray}
%s_{\text d}(t)\!\!\!\! &\triangleq& \!\!\! \sum_{i=1}^{m_\mathcal{A}} \omega_{\mathcal{A}_i}(t)n_{\mathcal{A}_i}(t) - \sum_{i=1}^{m_\mathcal{B}} \omega_{\mathcal{B}_i}(t)n_{\mathcal{B}_i}(t), \quad t\geqslant0,\hspace{0.1in}
%\end{eqnarray}
\begin{equation}
	s_{\text d}(t) \triangleq \sum_{i=1}^{m_\mathcal{A}} \omega_{\mathcal{A}_i}(t)n_{\mathcal{A}_i}(t) - \sum_{i=1}^{m_\mathcal{B}} \omega_{\mathcal{B}_i}(t)n_{\mathcal{B}_i}(t), \quad t\geqslant0,\hspace{0.1in}
\end{equation}
%
\noindent where $m_\mathcal{A}$, $m_\mathcal{B} \in \mathds N$ represent the number vertices in subset $\mathcal{A}$ and $\mathcal{B}$, respectively. The value of $s_{\text d}(t)$ describes the reaction force direction vector. Efficacy of this design is illustrated in section \sec{simulation}. In the following, we discuss the design of the interaction effort magnitude, $s_{\rm m}(t)$.

% [YM]CHECK: You don't use the following, so why present it?
%
%With $v(t)$, we arrive at the force estimate as
%
%\begin{eqnarray}
%	s(t) &\triangleq& v(t) \frac{s_{\text d}(t)}{\|s_{\text d}(t)\|} , \quad t\geqslant0. \\
%\end{eqnarray}
%
%In the following, we discuss the manner in which we may rely on this quantity, as well as $c(t)$ interaction efforts between objects $\mathcal A$ and $\mathcal B$.
%\noindent In the following, we discuss the design of the interaction effort magnitude, $s_{\rm m}(t)$.

