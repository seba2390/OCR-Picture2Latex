\begin{figure}
	\centering
	%\def\svgwidth{\columnwidth}
	\includesvg[width=0.9\columnwidth]{images/simple_collision_all_2.svg}
	\caption{Overlap Repulsive Force Computation(a: Sort lines to corresponding object; b: Iterate over blue line, get distance vector and force apply position, c: Calculate total force and apply position)}
	\label{fig:overlap_computation}
\end{figure}

A major aspect of physics simulation is collision detection and reaction effort computation. For penalty-based approaches such as the one we are using, this requires some form of overlap detection to determine where two or more objects are occupying the same space. As bodies do not overlap in real-life scenarios, such occurrences are penalized by applying a corrective force which push the bodies apart.

In our case, we developed a new approach to determining body overlap. First, we determine the intersecting area of two polyshapes. Should the bodies be far enough apart for no overlap to occur, the function returns an empty shape.

Once an overlap is detected, penalizing forces must be computed. The larger the area of overlap, the greater the repulsive forces that act on both objects. However, we also require a directional component that indicates the direction in which either objects should be moved, namely away from the area of overlap. 
Thirdly, once the efforts have been determined, we must also identify the force application position.

For all of these components, we require more information from the intersecting shape than just the area. Our approach iterates over the shape's outline and determines how deep the bodies penetrate at each position. The repulsive force is adjusted proportionally. The following describes the process in greater detail.
 
First, we analyze the outer lines of the overlapping shape. Each line is sorted to its corresponding object, either $\vA$ or $\vB$ (see figure \ref{fig:overlap_computation}).
Next, we iterate over the lines assigned to body $\vA$. For each line $\vlA$, we compute the normal $\vnlA$. Then, we iterate over all positions on $\vlA$ with an increment of $\vdeltaL$. At each of these positions $\vp$ along $\vlA$, we store the distance vector $\vfP$ to the closest position on all lines in $\vB$ along the normal $\vnlA$. This distance vector will be used as a component of the total repulsive force. Additionally the center $\vcP$ between $\vp$ and the aforementioned minimum distance point is also stored. $\vcP$ will be used to determine the force application position. If no line $\vlB$ can be found, the position is ignored.
%The distance vector is used as an indicator of the repulsive force that should be applied at position $\bm{c}_p$ to push both objects apart.

Once all lines of $\vA$ have been analyzed as described above, we sum up all distance vectors and center positions to
\begin{align}
	\vfEst &\triangleq \sum_{p \in \vlA}^{} \vdeltaL \vfP, \quad                             &t \geqslant 0,\\
	\vc    &\triangleq \sum_{p \in \vlA}^{} \vdeltaL \frac{||\vfP||}{||\vfEst||} \vcP, \quad &t \geqslant 0,
\end{align}
where $||\cdot||$ is the usual Euclidian norm.
$\vfEst$ will be used to estimate the repulsive forces between objects $\vA$ and $\vB$ using a PD control law (described below), and $\vc$ is the position where the force will be applied.