\subsection{Parameter-space exploration}
\label{sec:comparison}
% [ym]CHECK no, this title is wrong, you have comparison to constraint in previous subsection too, and here you explore parameter space
%
To better assess merit of the proposed methodology, we explore grasp behaviors achieved in different areas of the method's parameter space, focusing on parameters which have shown to have the greatest impact on this behavior. For this purpose,
% A series of tests have been run to determine the viable execution parameters of our approach. For this,
we use the 3-pronged rotational gripper shown in \fig{grippersobjects}, and apply to it a pre-defined sinusoidal pattern. Over the course of this simulation, we measure the kinetic energy of the system, as computed from the objects' simulated linear and angular speeds, and compare it to the expected kinetic energy, as computed from the prescribed movement velocity and assuming no relative movements between object and gripper. In an ideal case, the object would remain at a constant relative position from the gripper, and the ratio of considered kinetic energies would be 1. In situations in which the grasp is less-than-perfect, parasite movements emerge (\emph{wobble creep}), leading to additional (and undesirable) kinetic energy. In the extreme, abrupt large efforts may emerge, disrupting simulation (as is the case when using the constraint method in \fig{grasp_explosion}).
% Below we have listed several graphs comparing various parameters to each other, with the ratio between energy inserted via motion and energy measured as a quality metric. The closer the ratio is to 1, the better the simulation reflects reality.
% [ym]CHECK check my language, whether I lie or not. Your mention of "quality metric" is obscure.

%We have explored a meaningful range of the methods' parameter space, relevant results are discussed hereafter. 
As shown in \fig{impcoeffstiffnesscomp}, high contact stiffness $g_\text{1}$ and gain multiplier values $g_\text{i}$ may lead to kinetic energy creation, which implies colored simulation results. Setting these parameters to particularly high values can be verified to result in a jittery behavior for the grasped object, with computed repulsive forces not only reducing contact overlap, but also accelerating objects away from each other.
% At high gains, this behavior is observed even when initializing the simulation with two objects merely touching each other.
\fig{depthstiffnesscomp} illustrates the impact on behavior of the depth parameter $d_\text{t}$ and of gain multiplier $g_\text{i}$. Behaviors similar to those observed in the previous example can be seen. Excessively small values of the depth parameter result in a clear increase in kinetic energy over time. It can be observed that decreasing admissible depth leads to greater mechanical stiffness, which in turn leads to a jittery behavior of the grasped object.
%
\begin{figure}
	\centering
	\scalebox{0.9}{\includegraphics[width=1.0\columnwidth]{svg-inkscape/impcoeff_stiffness_comp_dot}}
	\caption{Kinetic energy ratio comparison for differing values of gain multiplier $g_\text{i}$ and stiffness $g_1$. The green dot indicates the parameters used in \fig{grasp_throw}.}
	\label{fig:impcoeffstiffnesscomp}
\end{figure}
%
\begin{figure}
	\centering
	\scalebox{0.9}{\includegraphics[width=1.0\columnwidth]{svg-inkscape/depth_stiffness_comp_dot}}
	\caption{Kinetic energy ratio comparison for differing values of gain multiplier $g_\text{i}$ and admissible depth $d_\text{t}$. The green dot indicates the parameters used in \fig{grasp_throw}.}
	\label{fig:depthstiffnesscomp}
\end{figure}
%\begin{figure}
%	\begin{subfigure}{0.5\columnwidth}
%		\centering
%		\includegraphics[width=1.0\columnwidth]{svg-inkscape/impcoeff_stiffness_comp_dot}
%	\end{subfigure}%
%	\begin{subfigure}{0.5\columnwidth}
%		\centering
%		\includegraphics[width=1.0\columnwidth]{svg-inkscape/depth_stiffness_comp_dot}
%	\end{subfigure}
%\end{figure}
%
