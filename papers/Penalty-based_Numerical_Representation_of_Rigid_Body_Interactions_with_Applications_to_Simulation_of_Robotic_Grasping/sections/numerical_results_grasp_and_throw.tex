%
\begin{figure*}
	\centering
	%\def\svgwidth{\columnwidth}
	\includegraphics[width=\textwidth]{images/grasp_throw_7.eps}
	\caption{Grasping an object and throwing it into box}
	\label{fig:grasp_throw}
\end{figure*}

Once appropriate parameters have been determined, we can simulate more complex scenarios. Here, a pick-and-throw operation is modeled. An hourglass-shaped object is placed on the floor next to a box. The gripper moves itself over the object, in a position from which it can easily grasp the object. The fingers close, locking the object in place. In contrast with other simulators that often encounter problems with this scenario, our physics simulation remains steady. This is due to the fact that our approach takes overlaps into account, and reduces knockback through the simulation of mechanical impedance, which prevents objects from bouncing around in a completed grasp.
%TODO: Explain numerical instability of other solvers better

The following describes the outcomes of individual steps in greater detail.
Once the object is grasped, the gripper lifts it up. The finger contacts transfer forces to the object, countering the simulation's gravity. Next, the object is thrown towards the box. The throw demonstrates the simulations capability of modeling impulse transfer from one object to another, in this case from the finger to the object. At launch, the previously grasped body has attained enough velocity to continue traveling away from the finger.
After following the throw's trajectory, the object lands inside the target box. Here, friction forces act upon the body, and it comes to rest once more.
The entire process is illustrated in figure \ref{fig:grasp_throw}. 