\documentclass[sigconf, screen]{acmart}

%\settopmatter{printacmref=false}


%%%%% NEW MATH DEFINITIONS %%%%%

\usepackage{amsmath,amsfonts,bm}

% Mark sections of captions for referring to divisions of figures
\newcommand{\figleft}{{\em (Left)}}
\newcommand{\figcenter}{{\em (Center)}}
\newcommand{\figright}{{\em (Right)}}
\newcommand{\figtop}{{\em (Top)}}
\newcommand{\figbottom}{{\em (Bottom)}}
\newcommand{\captiona}{{\em (a)}}
\newcommand{\captionb}{{\em (b)}}
\newcommand{\captionc}{{\em (c)}}
\newcommand{\captiond}{{\em (d)}}

% Highlight a newly defined term
\newcommand{\newterm}[1]{{\bf #1}}


% Figure reference, lower-case.
\def\figref#1{figure~\ref{#1}}
% Figure reference, capital. For start of sentence
\def\Figref#1{Figure~\ref{#1}}
\def\twofigref#1#2{figures \ref{#1} and \ref{#2}}
\def\quadfigref#1#2#3#4{figures \ref{#1}, \ref{#2}, \ref{#3} and \ref{#4}}
% Section reference, lower-case.
\def\secref#1{section~\ref{#1}}
% Section reference, capital.
\def\Secref#1{Section~\ref{#1}}
% Reference to two sections.
\def\twosecrefs#1#2{sections \ref{#1} and \ref{#2}}
% Reference to three sections.
\def\secrefs#1#2#3{sections \ref{#1}, \ref{#2} and \ref{#3}}
% Reference to an equation, lower-case.
% \def\eqref#1{equation~\ref{#1}}
 \def\eqref#1{(\ref{#1})}
% Reference to an equation, upper case
\def\Eqref#1{Equation~\ref{#1}}
% A raw reference to an equation---avoid using if possible
\def\plaineqref#1{\ref{#1}}
% Reference to a chapter, lower-case.
\def\chapref#1{chapter~\ref{#1}}
% Reference to an equation, upper case.
\def\Chapref#1{Chapter~\ref{#1}}
% Reference to a range of chapters
\def\rangechapref#1#2{chapters\ref{#1}--\ref{#2}}
% Reference to an algorithm, lower-case.
\def\algref#1{algorithm~\ref{#1}}
% Reference to an algorithm, upper case.
\def\Algref#1{Algorithm~\ref{#1}}
\def\twoalgref#1#2{algorithms \ref{#1} and \ref{#2}}
\def\Twoalgref#1#2{Algorithms \ref{#1} and \ref{#2}}
% Reference to a part, lower case
\def\partref#1{part~\ref{#1}}
% Reference to a part, upper case
\def\Partref#1{Part~\ref{#1}}
\def\twopartref#1#2{parts \ref{#1} and \ref{#2}}

\def\ceil#1{\lceil #1 \rceil}
\def\floor#1{\lfloor #1 \rfloor}
\def\1{\bm{1}}
\newcommand{\train}{\mathcal{D}}
\newcommand{\valid}{\mathcal{D_{\mathrm{valid}}}}
\newcommand{\test}{\mathcal{D_{\mathrm{test}}}}

\def\eps{{\epsilon}}


% Random variables
\def\reta{{\textnormal{$\eta$}}}
\def\ra{{\textnormal{a}}}
\def\rb{{\textnormal{b}}}
\def\rc{{\textnormal{c}}}
\def\rd{{\textnormal{d}}}
\def\re{{\textnormal{e}}}
\def\rf{{\textnormal{f}}}
\def\rg{{\textnormal{g}}}
\def\rh{{\textnormal{h}}}
\def\ri{{\textnormal{i}}}
\def\rj{{\textnormal{j}}}
\def\rk{{\textnormal{k}}}
\def\rl{{\textnormal{l}}}
% rm is already a command, just don't name any random variables m
\def\rn{{\textnormal{n}}}
\def\ro{{\textnormal{o}}}
\def\rp{{\textnormal{p}}}
\def\rq{{\textnormal{q}}}
\def\rr{{\textnormal{r}}}
\def\rs{{\textnormal{s}}}
\def\rt{{\textnormal{t}}}
\def\ru{{\textnormal{u}}}
\def\rv{{\textnormal{v}}}
\def\rw{{\textnormal{w}}}
\def\rx{{\textnormal{x}}}
\def\ry{{\textnormal{y}}}
\def\rz{{\textnormal{z}}}

% Random vectors
\def\rvepsilon{{\mathbf{\epsilon}}}
\def\rvtheta{{\mathbf{\theta}}}
\def\rva{{\mathbf{a}}}
\def\rvb{{\mathbf{b}}}
\def\rvc{{\mathbf{c}}}
\def\rvd{{\mathbf{d}}}
\def\rve{{\mathbf{e}}}
\def\rvf{{\mathbf{f}}}
\def\rvg{{\mathbf{g}}}
\def\rvh{{\mathbf{h}}}
\def\rvu{{\mathbf{i}}}
\def\rvj{{\mathbf{j}}}
\def\rvk{{\mathbf{k}}}
\def\rvl{{\mathbf{l}}}
\def\rvm{{\mathbf{m}}}
\def\rvn{{\mathbf{n}}}
\def\rvo{{\mathbf{o}}}
\def\rvp{{\mathbf{p}}}
\def\rvq{{\mathbf{q}}}
\def\rvr{{\mathbf{r}}}
\def\rvs{{\mathbf{s}}}
\def\rvt{{\mathbf{t}}}
\def\rvu{{\mathbf{u}}}
\def\rvv{{\mathbf{v}}}
\def\rvw{{\mathbf{w}}}
\def\rvx{{\mathbf{x}}}
\def\rvy{{\mathbf{y}}}
\def\rvz{{\mathbf{z}}}

% Elements of random vectors
\def\erva{{\textnormal{a}}}
\def\ervb{{\textnormal{b}}}
\def\ervc{{\textnormal{c}}}
\def\ervd{{\textnormal{d}}}
\def\erve{{\textnormal{e}}}
\def\ervf{{\textnormal{f}}}
\def\ervg{{\textnormal{g}}}
\def\ervh{{\textnormal{h}}}
\def\ervi{{\textnormal{i}}}
\def\ervj{{\textnormal{j}}}
\def\ervk{{\textnormal{k}}}
\def\ervl{{\textnormal{l}}}
\def\ervm{{\textnormal{m}}}
\def\ervn{{\textnormal{n}}}
\def\ervo{{\textnormal{o}}}
\def\ervp{{\textnormal{p}}}
\def\ervq{{\textnormal{q}}}
\def\ervr{{\textnormal{r}}}
\def\ervs{{\textnormal{s}}}
\def\ervt{{\textnormal{t}}}
\def\ervu{{\textnormal{u}}}
\def\ervv{{\textnormal{v}}}
\def\ervw{{\textnormal{w}}}
\def\ervx{{\textnormal{x}}}
\def\ervy{{\textnormal{y}}}
\def\ervz{{\textnormal{z}}}

% Random matrices
\def\rmA{{\mathbf{A}}}
\def\rmB{{\mathbf{B}}}
\def\rmC{{\mathbf{C}}}
\def\rmD{{\mathbf{D}}}
\def\rmE{{\mathbf{E}}}
\def\rmF{{\mathbf{F}}}
\def\rmG{{\mathbf{G}}}
\def\rmH{{\mathbf{H}}}
\def\rmI{{\mathbf{I}}}
\def\rmJ{{\mathbf{J}}}
\def\rmK{{\mathbf{K}}}
\def\rmL{{\mathbf{L}}}
\def\rmM{{\mathbf{M}}}
\def\rmN{{\mathbf{N}}}
\def\rmO{{\mathbf{O}}}
\def\rmP{{\mathbf{P}}}
\def\rmQ{{\mathbf{Q}}}
\def\rmR{{\mathbf{R}}}
\def\rmS{{\mathbf{S}}}
\def\rmT{{\mathbf{T}}}
\def\rmU{{\mathbf{U}}}
\def\rmV{{\mathbf{V}}}
\def\rmW{{\mathbf{W}}}
\def\rmX{{\mathbf{X}}}
\def\rmY{{\mathbf{Y}}}
\def\rmZ{{\mathbf{Z}}}

% Elements of random matrices
\def\ermA{{\textnormal{A}}}
\def\ermB{{\textnormal{B}}}
\def\ermC{{\textnormal{C}}}
\def\ermD{{\textnormal{D}}}
\def\ermE{{\textnormal{E}}}
\def\ermF{{\textnormal{F}}}
\def\ermG{{\textnormal{G}}}
\def\ermH{{\textnormal{H}}}
\def\ermI{{\textnormal{I}}}
\def\ermJ{{\textnormal{J}}}
\def\ermK{{\textnormal{K}}}
\def\ermL{{\textnormal{L}}}
\def\ermM{{\textnormal{M}}}
\def\ermN{{\textnormal{N}}}
\def\ermO{{\textnormal{O}}}
\def\ermP{{\textnormal{P}}}
\def\ermQ{{\textnormal{Q}}}
\def\ermR{{\textnormal{R}}}
\def\ermS{{\textnormal{S}}}
\def\ermT{{\textnormal{T}}}
\def\ermU{{\textnormal{U}}}
\def\ermV{{\textnormal{V}}}
\def\ermW{{\textnormal{W}}}
\def\ermX{{\textnormal{X}}}
\def\ermY{{\textnormal{Y}}}
\def\ermZ{{\textnormal{Z}}}

% Vectors
\def\vzero{{\bm{0}}}
\def\vone{{\bm{1}}}
\def\vmu{{\bm{\mu}}}
\def\vtheta{{\bm{\theta}}}
\def\va{{\bm{a}}}
\def\vb{{\bm{b}}}
\def\vc{{\bm{c}}}
\def\vd{{\bm{d}}}
\def\ve{{\bm{e}}}
\def\vf{{\bm{f}}}
\def\vg{{\bm{g}}}
\def\vh{{\bm{h}}}
\def\vi{{\bm{i}}}
\def\vj{{\bm{j}}}
\def\vk{{\bm{k}}}
\def\vl{{\bm{l}}}
\def\vm{{\bm{m}}}
\def\vn{{\bm{n}}}
\def\vo{{\bm{o}}}
\def\vp{{\bm{p}}}
\def\vq{{\bm{q}}}
\def\vr{{\bm{r}}}
\def\vs{{\bm{s}}}
\def\vt{{\bm{t}}}
\def\vu{{\bm{u}}}
\def\vv{{\bm{v}}}
\def\vw{{\bm{w}}}
\def\vx{{\bm{x}}}
\def\vy{{\bm{y}}}
\def\vz{{\bm{z}}}

% Elements of vectors
\def\evalpha{{\alpha}}
\def\evbeta{{\beta}}
\def\evepsilon{{\epsilon}}
\def\evlambda{{\lambda}}
\def\evomega{{\omega}}
\def\evmu{{\mu}}
\def\evpsi{{\psi}}
\def\evsigma{{\sigma}}
\def\evtheta{{\theta}}
\def\eva{{a}}
\def\evb{{b}}
\def\evc{{c}}
\def\evd{{d}}
\def\eve{{e}}
\def\evf{{f}}
\def\evg{{g}}
\def\evh{{h}}
\def\evi{{i}}
\def\evj{{j}}
\def\evk{{k}}
\def\evl{{l}}
\def\evm{{m}}
\def\evn{{n}}
\def\evo{{o}}
\def\evp{{p}}
\def\evq{{q}}
\def\evr{{r}}
\def\evs{{s}}
\def\evt{{t}}
\def\evu{{u}}
\def\evv{{v}}
\def\evw{{w}}
\def\evx{{x}}
\def\evy{{y}}
\def\evz{{z}}

% Matrix
\def\mA{{\bm{A}}}
\def\mB{{\bm{B}}}
\def\mC{{\bm{C}}}
\def\mD{{\bm{D}}}
\def\mE{{\bm{E}}}
\def\mF{{\bm{F}}}
\def\mG{{\bm{G}}}
\def\mH{{\bm{H}}}
\def\mI{{\bm{I}}}
\def\mJ{{\bm{J}}}
\def\mK{{\bm{K}}}
\def\mL{{\bm{L}}}
\def\mM{{\bm{M}}}
\def\mN{{\bm{N}}}
\def\mO{{\bm{O}}}
\def\mP{{\bm{P}}}
\def\mQ{{\bm{Q}}}
\def\mR{{\bm{R}}}
\def\mS{{\bm{S}}}
\def\mT{{\bm{T}}}
\def\mU{{\bm{U}}}
\def\mV{{\bm{V}}}
\def\mW{{\bm{W}}}
\def\mX{{\bm{X}}}
\def\mY{{\bm{Y}}}
\def\mZ{{\bm{Z}}}
\def\mBeta{{\bm{\beta}}}
\def\mPhi{{\bm{\Phi}}}
\def\mLambda{{\bm{\Lambda}}}
\def\mSigma{{\bm{\Sigma}}}

% Tensor
\DeclareMathAlphabet{\mathsfit}{\encodingdefault}{\sfdefault}{m}{sl}
\SetMathAlphabet{\mathsfit}{bold}{\encodingdefault}{\sfdefault}{bx}{n}
\newcommand{\tens}[1]{\bm{\mathsfit{#1}}}
\def\tA{{\tens{A}}}
\def\tB{{\tens{B}}}
\def\tC{{\tens{C}}}
\def\tD{{\tens{D}}}
\def\tE{{\tens{E}}}
\def\tF{{\tens{F}}}
\def\tG{{\tens{G}}}
\def\tH{{\tens{H}}}
\def\tI{{\tens{I}}}
\def\tJ{{\tens{J}}}
\def\tK{{\tens{K}}}
\def\tL{{\tens{L}}}
\def\tM{{\tens{M}}}
\def\tN{{\tens{N}}}
\def\tO{{\tens{O}}}
\def\tP{{\tens{P}}}
\def\tQ{{\tens{Q}}}
\def\tR{{\tens{R}}}
\def\tS{{\tens{S}}}
\def\tT{{\tens{T}}}
\def\tU{{\tens{U}}}
\def\tV{{\tens{V}}}
\def\tW{{\tens{W}}}
\def\tX{{\tens{X}}}
\def\tY{{\tens{Y}}}
\def\tZ{{\tens{Z}}}


% Graph
\def\gA{{\mathcal{A}}}
\def\gB{{\mathcal{B}}}
\def\gC{{\mathcal{C}}}
\def\gD{{\mathcal{D}}}
\def\gE{{\mathcal{E}}}
\def\gF{{\mathcal{F}}}
\def\gG{{\mathcal{G}}}
\def\gH{{\mathcal{H}}}
\def\gI{{\mathcal{I}}}
\def\gJ{{\mathcal{J}}}
\def\gK{{\mathcal{K}}}
\def\gL{{\mathcal{L}}}
\def\gM{{\mathcal{M}}}
\def\gN{{\mathcal{N}}}
\def\gO{{\mathcal{O}}}
\def\gP{{\mathcal{P}}}
\def\gQ{{\mathcal{Q}}}
\def\gR{{\mathcal{R}}}
\def\gS{{\mathcal{S}}}
\def\gT{{\mathcal{T}}}
\def\gU{{\mathcal{U}}}
\def\gV{{\mathcal{V}}}
\def\gW{{\mathcal{W}}}
\def\gX{{\mathcal{X}}}
\def\gY{{\mathcal{Y}}}
\def\gZ{{\mathcal{Z}}}

% Sets
\def\sA{{\mathbb{A}}}
\def\sB{{\mathbb{B}}}
\def\sC{{\mathbb{C}}}
\def\sD{{\mathbb{D}}}
% Don't use a set called E, because this would be the same as our symbol
% for expectation.
\def\sF{{\mathbb{F}}}
\def\sG{{\mathbb{G}}}
\def\sH{{\mathbb{H}}}
\def\sI{{\mathbb{I}}}
\def\sJ{{\mathbb{J}}}
\def\sK{{\mathbb{K}}}
\def\sL{{\mathbb{L}}}
\def\sM{{\mathbb{M}}}
\def\sN{{\mathbb{N}}}
\def\sO{{\mathbb{O}}}
\def\sP{{\mathbb{P}}}
\def\sQ{{\mathbb{Q}}}
\def\sR{{\mathbb{R}}}
\def\sS{{\mathbb{S}}}
\def\sT{{\mathbb{T}}}
\def\sU{{\mathbb{U}}}
\def\sV{{\mathbb{V}}}
\def\sW{{\mathbb{W}}}
\def\sX{{\mathbb{X}}}
\def\sY{{\mathbb{Y}}}
\def\sZ{{\mathbb{Z}}}

% Entries of a matrix
\def\emLambda{{\Lambda}}
\def\emA{{A}}
\def\emB{{B}}
\def\emC{{C}}
\def\emD{{D}}
\def\emE{{E}}
\def\emF{{F}}
\def\emG{{G}}
\def\emH{{H}}
\def\emI{{I}}
\def\emJ{{J}}
\def\emK{{K}}
\def\emL{{L}}
\def\emM{{M}}
\def\emN{{N}}
\def\emO{{O}}
\def\emP{{P}}
\def\emQ{{Q}}
\def\emR{{R}}
\def\emS{{S}}
\def\emT{{T}}
\def\emU{{U}}
\def\emV{{V}}
\def\emW{{W}}
\def\emX{{X}}
\def\emY{{Y}}
\def\emZ{{Z}}
\def\emSigma{{\Sigma}}

% entries of a tensor
% Same font as tensor, without \bm wrapper
\newcommand{\etens}[1]{\mathsfit{#1}}
\def\etLambda{{\etens{\Lambda}}}
\def\etA{{\etens{A}}}
\def\etB{{\etens{B}}}
\def\etC{{\etens{C}}}
\def\etD{{\etens{D}}}
\def\etE{{\etens{E}}}
\def\etF{{\etens{F}}}
\def\etG{{\etens{G}}}
\def\etH{{\etens{H}}}
\def\etI{{\etens{I}}}
\def\etJ{{\etens{J}}}
\def\etK{{\etens{K}}}
\def\etL{{\etens{L}}}
\def\etM{{\etens{M}}}
\def\etN{{\etens{N}}}
\def\etO{{\etens{O}}}
\def\etP{{\etens{P}}}
\def\etQ{{\etens{Q}}}
\def\etR{{\etens{R}}}
\def\etS{{\etens{S}}}
\def\etT{{\etens{T}}}
\def\etU{{\etens{U}}}
\def\etV{{\etens{V}}}
\def\etW{{\etens{W}}}
\def\etX{{\etens{X}}}
\def\etY{{\etens{Y}}}
\def\etZ{{\etens{Z}}}

% The true underlying data generating distribution
\newcommand{\pdata}{p_{\rm{data}}}
% The empirical distribution defined by the training set
\newcommand{\ptrain}{\hat{p}_{\rm{data}}}
\newcommand{\Ptrain}{\hat{P}_{\rm{data}}}
% The model distribution
\newcommand{\pmodel}{p_{\rm{model}}}
\newcommand{\Pmodel}{P_{\rm{model}}}
\newcommand{\ptildemodel}{\tilde{p}_{\rm{model}}}
% Stochastic autoencoder distributions
\newcommand{\pencode}{p_{\rm{encoder}}}
\newcommand{\pdecode}{p_{\rm{decoder}}}
\newcommand{\precons}{p_{\rm{reconstruct}}}

\newcommand{\laplace}{\mathrm{Laplace}} % Laplace distribution

\newcommand{\E}{\mathbb{E}}
\newcommand{\Ls}{\mathcal{L}}
\newcommand{\R}{\mathbb{R}}
\newcommand{\emp}{\tilde{p}}
\newcommand{\lr}{\alpha}
\newcommand{\reg}{\lambda}
\newcommand{\rect}{\mathrm{rectifier}}
\newcommand{\softmax}{\mathrm{softmax}}
\newcommand{\sigmoid}{\sigma}
\newcommand{\softplus}{\zeta}
\newcommand{\KL}{D_{\mathrm{KL}}}
\newcommand{\Var}{\mathrm{Var}}
\newcommand{\standarderror}{\mathrm{SE}}
\newcommand{\Cov}{\mathrm{Cov}}
% Wolfram Mathworld says $L^2$ is for function spaces and $\ell^2$ is for vectors
% But then they seem to use $L^2$ for vectors throughout the site, and so does
% wikipedia.
\newcommand{\normlzero}{L^0}
\newcommand{\normlone}{L^1}
\newcommand{\normltwo}{L^2}
\newcommand{\normlp}{L^p}
\newcommand{\normmax}{L^\infty}

\newcommand{\parents}{Pa} % See usage in notation.tex. Chosen to match Daphne's book.

\DeclareMathOperator*{\argmax}{arg\,max}
\DeclareMathOperator*{\argmin}{arg\,min}

\DeclareMathOperator{\sign}{sign}
\DeclareMathOperator{\Tr}{Tr}
\let\ab\allowbreak

\newcommand{\norm}[2]{\left\| #1 \right\|_{#2}}

\newcommand{\zz}[1]{\textcolor{blue}{ [{\em Zhihui:} #1]}}
\newcommand{\jz}[1]{\textcolor{red}{ [{\em JZ:} #1]}}
% \newcommand{\td}[1]{\textcolor{blue}{ [{\em TD:} #1]}}
\newcommand{\jj}[1]{\textcolor{pink}{ [{\em JJ:} #1]}}

\usepackage{hyperref}
\usepackage{url}
\usepackage{times}
\usepackage{latexsym}
\usepackage{verbatim}
\usepackage[T1]{fontenc}
\usepackage[utf8]{inputenc}
\usepackage{microtype}
\usepackage{algorithm} 
\usepackage{algpseudocode}
\usepackage{balance}
\usepackage{breqn}
\usepackage{lipsum,graphicx}
\usepackage{booktabs}
\usepackage[export]{adjustbox}
\usepackage{layouts}
\renewcommand{\UrlFont}{\ttfamily\small}
\usepackage{caption}
\usepackage{subcaption}
\usepackage{multirow}
\usepackage{makecell}
\usepackage{pifont}% http://ctan.org/pkg/pifont
\usepackage{geometry} 
\usepackage{tikz} 
\usepackage{lipsum}

%%%%%%%%%%%%% Defined Commands %%%%%%%%%%%%%%%%%%%
\newcommand{\myboxx}[4]{
    \begin{figure}[h]
        \centering
    \begin{tikzpicture}
        \node[anchor=text,text width=\columnwidth-1cm, draw, rounded corners, line width=1pt, fill=#3, inner sep=mm] (big) {\\#4};
        \node[draw, rounded corners, line width=.5pt, fill=#2, anchor=west, xshift=5mm] (small) at (big.north west) {#1};
    \end{tikzpicture}
    \vspace{-6pt}
    \end{figure}
    }
    
\newcommand{\mybox}[4]{
\vspace{4pt}
\noindent{}\fbox{\parbox{0.963\columnwidth}{\textbf{#1:} 
#4
}
}
\vspace{4pt}


}
    
\newcommand{\cmark}{\ding{51}}%
\newcommand{\xmark}{\ding{55}}%
\newcommand\BibTeX{B\textsc{ib}\TeX}
\newcommand\jdime{\textsc{JDime}}
\newcommand\jsfstmerge{\textsc{jsFSTMerge}}
\newcommand\fstmerge{\textsc{FSTMerge}}
\newcommand\thistool{MergeBERT}
\newcommand{\ic}[1]{\begin{small}\texttt{#1}\end{small}}

%%%%%%%%%%%%% Research Questions %%%%%%%%%%%%%%%%%%%
\newcommand{\rqOne}{How effective is \thistool{} in producing merge conflict resolutions?}
\newcommand{\rqTwo}{How well does \thistool{} perform across different languages?}
\newcommand{\rqThree}{How do different choices of context encoding impact performance of \thistool{}?}
\newcommand{\rqFour}{How do users perceive \thistool{} resolutions?}



\title{Program Merge Conflict Resolution via Neural Transformers}


\author{Alexey Svyatkovskiy}
\affiliation{%
  \institution{Microsoft}
  \city{Redmond}
  \state{WA}
  \country{USA}
}
%\email{alsvyatk@microsoft.com}

\author{Sarah Fakhoury}
\affiliation{%
  \institution{Washington State University}
  \city{Pullman}
  \state{WA}
  \country{USA}
}

\author{Negar Ghorbani}
\affiliation{%
  \institution{UC Irvine}
  \city{Irvine}
  \state{CA}
  \country{USA}
}

\author{Todd Mytkowicz}
\affiliation{%
  \institution{Microsoft Research}
  \city{Redmond}
  \state{WA}
  \country{USA}
}

\author{Elizabeth Dinella}
\affiliation{%
  \institution{University of Pennsylvania}
  \city{Philadelphia}
  \state{PA}
  \country{USA}
}

\author{Christian Bird}
\affiliation{%
  \institution{Microsoft Research}
  \city{Redmond}
  \state{WA}
  \country{USA}
}

\author{Jinu Jang}
\affiliation{%
  \institution{Microsoft}
  \city{Redmond}
  \state{WA}
  \country{USA}
}

\author{Neel Sundaresan}
\affiliation{%
  \institution{Microsoft}
  \city{Redmond}
  \state{WA}
  \country{USA}
}

\author{Shuvendu K. Lahiri}
\affiliation{%
  \institution{Microsoft Research}
  \city{Redmond}
  \state{WA}
  \country{USA}
}

\newcommand{\fix}{\marginpar{FIX}}
\newcommand{\new}{\marginpar{NEW}}

\renewcommand{\shortauthors}{Svyatkovskiy, Fakhoury, Ghorbani, Mytkowicz, Dinella, Bird, Jang, Sundaresan, Lahiri}


\begin{document}

\begin{abstract}
Collaborative software development is an integral part of the modern software development life cycle, essential to the success of large-scale software projects. When multiple developers make concurrent changes around the same lines of code, a merge conflict may occur. Such conflicts stall pull requests and continuous integration pipelines for hours to several days, seriously hurting developer productivity. To address this problem, we introduce \thistool{}, a novel neural program merge framework based on token-level three-way differencing and a transformer encoder model. By exploiting the restricted nature of merge conflict resolutions, we reformulate the task of generating the resolution sequence as a classification task over a set of primitive merge patterns extracted from real-world merge commit data. Our model achieves 63--68\% accuracy for merge resolution synthesis, yielding nearly a 3$\times$ performance improvement over existing semi-structured, and 2$\times$ improvement over neural program merge tools. Finally, we demonstrate that \thistool{} is sufficiently flexible to work with source code files in Java, JavaScript, TypeScript, and C\# programming languages.
To measure the practical use of \thistool{}, we conduct a user study to evaluate \thistool{} suggestions with 25 developers from large OSS projects on 122 real-world conflicts they encountered. Results suggest that in practice, \thistool{} resolutions would be accepted at a higher rate than estimated by automatic metrics for precision and accuracy. Additionally, we use participant feedback to identify future avenues for improvement of \thistool{}.

\end{abstract}


\begin{CCSXML}
<ccs2012>
   <concept>
       <concept_id>10011007.10011074.10011111.10011695</concept_id>
       <concept_desc>Software and its engineering~Software version control</concept_desc>
       <concept_significance>500</concept_significance>
       </concept>
   <concept>
       <concept_id>10011007.10011074.10011092.10011782</concept_id>
       <concept_desc>Software and its engineering~Automatic programming</concept_desc>
       <concept_significance>500</concept_significance>
       </concept>
 </ccs2012>
\end{CCSXML}

\ccsdesc[500]{Software and its engineering~Software version control}
\ccsdesc[500]{Software and its engineering~Automatic programming}

\keywords{Software evolution, program merge, ml4code}

%%% The following is specific to ESEC/FSE '22 and the paper
%%% 'Program Merge Conflict Resolution via Neural Transformers'
%%% by Alexey Svyatkovskiy, Sarah Fakhoury, Negar Ghorbani, Todd Mytkowicz, Elizabeth Dinella, Christian Bird, Jinu Jang, Neel Sundaresan, and Shuvendu K. Lahiri.
%%%
\setcopyright{acmcopyright}
\acmPrice{15.00}
\acmDOI{10.1145/3540250.3549163} 
\acmYear{2022}
\copyrightyear{2022}
\acmSubmissionID{fse22main-p1294-p}
\acmISBN{978-1-4503-9413-0/22/11}
\acmConference[ESEC/FSE '22]{Proceedings of the 30th ACM Joint European Software Engineering Conference and Symposium on the Foundations of Software Engineering}{November 14--18, 2022}{Singapore, Singapore}
\acmBooktitle{Proceedings of the 30th ACM Joint European Software Engineering Conference and Symposium on the Foundations of Software Engineering (ESEC/FSE '22), November 14--18, 2022, Singapore, Singapore}


\maketitle

% \leavevmode
% \\
% \\
% \\
% \\
% \\
\section{Introduction}
\label{introduction}

AutoML is the process by which machine learning models are built automatically for a new dataset. Given a dataset, AutoML systems perform a search over valid data transformations and learners, along with hyper-parameter optimization for each learner~\cite{VolcanoML}. Choosing the transformations and learners over which to search is our focus.
A significant number of systems mine from prior runs of pipelines over a set of datasets to choose transformers and learners that are effective with different types of datasets (e.g. \cite{NEURIPS2018_b59a51a3}, \cite{10.14778/3415478.3415542}, \cite{autosklearn}). Thus, they build a database by actually running different pipelines with a diverse set of datasets to estimate the accuracy of potential pipelines. Hence, they can be used to effectively reduce the search space. A new dataset, based on a set of features (meta-features) is then matched to this database to find the most plausible candidates for both learner selection and hyper-parameter tuning. This process of choosing starting points in the search space is called meta-learning for the cold start problem.  

Other meta-learning approaches include mining existing data science code and their associated datasets to learn from human expertise. The AL~\cite{al} system mined existing Kaggle notebooks using dynamic analysis, i.e., actually running the scripts, and showed that such a system has promise.  However, this meta-learning approach does not scale because it is onerous to execute a large number of pipeline scripts on datasets, preprocessing datasets is never trivial, and older scripts cease to run at all as software evolves. It is not surprising that AL therefore performed dynamic analysis on just nine datasets.

Our system, {\sysname}, provides a scalable meta-learning approach to leverage human expertise, using static analysis to mine pipelines from large repositories of scripts. Static analysis has the advantage of scaling to thousands or millions of scripts \cite{graph4code} easily, but lacks the performance data gathered by dynamic analysis. The {\sysname} meta-learning approach guides the learning process by a scalable dataset similarity search, based on dataset embeddings, to find the most similar datasets and the semantics of ML pipelines applied on them.  Many existing systems, such as Auto-Sklearn \cite{autosklearn} and AL \cite{al}, compute a set of meta-features for each dataset. We developed a deep neural network model to generate embeddings at the granularity of a dataset, e.g., a table or CSV file, to capture similarity at the level of an entire dataset rather than relying on a set of meta-features.
 
Because we use static analysis to capture the semantics of the meta-learning process, we have no mechanism to choose the \textbf{best} pipeline from many seen pipelines, unlike the dynamic execution case where one can rely on runtime to choose the best performing pipeline.  Observing that pipelines are basically workflow graphs, we use graph generator neural models to succinctly capture the statically-observed pipelines for a single dataset. In {\sysname}, we formulate learner selection as a graph generation problem to predict optimized pipelines based on pipelines seen in actual notebooks.

%. This formulation enables {\sysname} for effective pruning of the AutoML search space to predict optimized pipelines based on pipelines seen in actual notebooks.}
%We note that increasingly, state-of-the-art performance in AutoML systems is being generated by more complex pipelines such as Directed Acyclic Graphs (DAGs) \cite{piper} rather than the linear pipelines used in earlier systems.  
 
{\sysname} does learner and transformation selection, and hence is a component of an AutoML systems. To evaluate this component, we integrated it into two existing AutoML systems, FLAML \cite{flaml} and Auto-Sklearn \cite{autosklearn}.  
% We evaluate each system with and without {\sysname}.  
We chose FLAML because it does not yet have any meta-learning component for the cold start problem and instead allows user selection of learners and transformers. The authors of FLAML explicitly pointed to the fact that FLAML might benefit from a meta-learning component and pointed to it as a possibility for future work. For FLAML, if mining historical pipelines provides an advantage, we should improve its performance. We also picked Auto-Sklearn as it does have a learner selection component based on meta-features, as described earlier~\cite{autosklearn2}. For Auto-Sklearn, we should at least match performance if our static mining of pipelines can match their extensive database. For context, we also compared {\sysname} with the recent VolcanoML~\cite{VolcanoML}, which provides an efficient decomposition and execution strategy for the AutoML search space. In contrast, {\sysname} prunes the search space using our meta-learning model to perform hyperparameter optimization only for the most promising candidates. 

The contributions of this paper are the following:
\begin{itemize}
    \item Section ~\ref{sec:mining} defines a scalable meta-learning approach based on representation learning of mined ML pipeline semantics and datasets for over 100 datasets and ~11K Python scripts.  
    \newline
    \item Sections~\ref{sec:kgpipGen} formulates AutoML pipeline generation as a graph generation problem. {\sysname} predicts efficiently an optimized ML pipeline for an unseen dataset based on our meta-learning model.  To the best of our knowledge, {\sysname} is the first approach to formulate  AutoML pipeline generation in such a way.
    \newline
    \item Section~\ref{sec:eval} presents a comprehensive evaluation using a large collection of 121 datasets from major AutoML benchmarks and Kaggle. Our experimental results show that {\sysname} outperforms all existing AutoML systems and achieves state-of-the-art results on the majority of these datasets. {\sysname} significantly improves the performance of both FLAML and Auto-Sklearn in classification and regression tasks. We also outperformed AL in 75 out of 77 datasets and VolcanoML in 75  out of 121 datasets, including 44 datasets used only by VolcanoML~\cite{VolcanoML}.  On average, {\sysname} achieves scores that are statistically better than the means of all other systems. 
\end{itemize}


%This approach does not need to apply cleaning or transformation methods to handle different variances among datasets. Moreover, we do not need to deal with complex analysis, such as dynamic code analysis. Thus, our approach proved to be scalable, as discussed in Sections~\ref{sec:mining}.

%auto-ignore
\begin{figure}[t!]
\centering
\includegraphics[width=1.0\linewidth]{figures/wireishard.pdf}
% \includegraphics[width=1.0\linewidth]{figures/wiresarehard2.pdf}
\caption{\textbf{Challenges of wire segmentation.} Wires have a diverse set of appearances. Challenges include but are not limited to (a) structural complexity, (b) visibility and thickness, (c) partial occlusion by other objects, (d) camera aberration artifacts, and variations in (e) object attachment, (f) color, (g) width and (h) shape.
% \zwei{this needs to be correspondent to the attributes you mentioned}
}
\vspace{-5.5mm}
\label{fig:motivation}
\end{figure}

\section{Background: Data-driven Merge}
\label{sec:background}
\citet{Dinella2021} introduced the {\it data-driven program merge} problem as a supervised machine learning problem. 
A program merge consists of a 4-tuple of programs $(\mathcal{A}, \mathcal{B}, \mathcal{O}, \mathcal{M})$, where 
\begin{enumerate} 
\item The base program $\mathcal{O}$ is the lowest common ancestor in the version history for programs $\mathcal{A}$ and $\mathcal{B}$, 
\item \texttt{diff3} produces an unstructured line-level conflict when applied to $(\mathcal{A}, \mathcal{B}, \mathcal{O})$, and 
\item $\mathcal{M}$ is the merged program with the developer resolution, incorporating changes made in  $\mathcal{A}$ and $\mathcal{B}$. 
\end{enumerate}
A merge may have multiple unstructured conflicts, we define a {\it merge tuple} $(A, B, O, M)$, where $A, B, O$ correspond to the conflicting regions in $(\mathcal{A}, \mathcal{B}$, and $\mathcal{O})$, respectively, and $M$ denotes the resolution region.

Given a set of merge tuples $(A_i, B_i, O_i, M_i)$, i = 0...N, the goal of a data-driven merge algorithm is to learn a function, $\texttt{merge}$, that maximizes $\sum_{i=0}^{N}\texttt{merge}(A_i, B_i, O_i) = M_i$.
Throughout the text, we will use notations $(a, b, o, m)$ to refer to the token-level merge tuples. 

\citet{Dinella2021} also provide an algorithm for extracting the exact resolution regions for each merge tuple and define a dataset that corresponds to {\it non-trivial} resolutions; resolutions where the developer does not drop the changes from one side of the merge.  
Further, they provide a sequence-to-sequence encoder-decoder based architecture, where a bi-directional gated recurrent unit (GRU) is used for encoding the merge inputs comprising of $(A, B, O)$ segments of a merge tuple, and a {\it pointer mechanism} is used to restrict the output to only choose from line segments present in the input. 
Their paper suffers from two limitations.
First, given the restriction on copying only lines from inputs, their dataset  did not consider merges where the resolution required token-level interleaving, such as the conflict in Figure~\ref{fig:word1}. 
Second, their dataset consists of merge conflicts in a single language, namely JavaScript. 
Our approach addresses both of these limitations.


\begin{figure*}
\begin{center}
    \includegraphics[width=.85\textwidth]{images/mergebert2.pdf}
\caption{An overview of the \thistool{} architecture. From left to right: given conflicting programs $\mathcal{A}$, $\mathcal{B}$ and $\mathcal{O}$ token-level differencing is performed first, next, programs are tokenized and the corresponding sequences are aligned ($a|_o$ and $o|_a$, $b|_o$, and $o|_b$). We extract edit steps for each pair of token sequences ($\Delta_{ao}$ and $\Delta_{bo}$). Four aligned token sequences are fed to the multi-input encoder neural network, while edit sequences are consumed as edit type embeddings. Finally, encoded token sequences are aggregated into a hidden state which serves as input to classification layer.}
\label{fig:mergebert}
\end{center}
\vspace{-8pt}
\end{figure*}


\section{Merge Conflict Resolution as a Classification Task}
\label{formulation}

In this work, we demonstrate how to exploit the restricted nature of merge conflict resolutions -- compared to an arbitrary program repair -- to leverage discriminative models to synthesize the merge resolution sequence.
We have empirically observed that the application of \texttt{diff3} at token granularity enjoys two useful properties over its line-level counterpart: (i) it helps localize the merge conflicts to small program segments, effectively reducing the size of conflicting regions, and (ii) most resolutions of merge conflicts produced by token \texttt{diff3} consist entirely of changes from $a$ or $b$ or $o$ or a sequential composition of $a$ followed by $b$ or vice versa. Here, and throughout the paper we will use lower case notations to refer to attributes of token-level differencing (e.g. $a$, $b$, and $o$ are conflict regions produced by \texttt{diff3} at token granularity).
On the flip side, a token-level merge can introduce many small conflicts. 
To balance the trade-off, we start with the line-level conflicts as produced by the standard \texttt{diff3} and perform a token-level merge of only the segments present in the line-level conflict.
There are several potential outcomes for such a two-level merge at the line-level: 
\begin{itemize}
\item {\it A conflict-free token-level merge}: For example, the edit from $A$ about \texttt{let} is merged since $B$ does not edit that slot as shown in Fig.~\ref{fig:word1}(b).  
\item {\it A single localized token-level merge conflict}: For example, the edit from both $A$ and $B$ for the arguments of \texttt{max} yields a single conflict as shown in Fig.~\ref{fig:word1}(b).
\item {\it Multiple token-level conflicts}: Such a case (not illustrated above) can result in several token-level conflicts. %which forms 59\%, 36\%, and 5\% of our dataset, respectively.
\end{itemize}

Token-level diff3 applied to a 4-tuple of programs $(\mathcal{A}, \mathcal{B}, \mathcal{O}, \mathcal{M})$, would usually result in a set of localized merge tuples $\langle a_j, b_j, o_j, m_j\rangle$. 
We empirically observe that 74\% of such resolutions $m_j$ are comprised of ($i$) exactly the tokens in $a_j$ or ($ii$) exactly the tokens in $b_j$.  Another 0.4\% of the resolutions are ($iii$) just the tokens in $o_j$. In addition, 23\% of the resolutions are the result of concatenating ($iv$) $a_j$ and $b_j$ or ($v$) $b_j$ and $a_j$.  Finally, 1.8\% comprise another four variants, obtained by taking $i$, $ii$, $iv$ and $v$ above and removing the tokens that also occur in the base, $o_j$. In total, this provides \textit{nine} primitive merge resolution patterns (see online Appendix~\cite{FSE22Appendix} for more details about the primitive merge patterns). 

We, therefore, treat the problem of constructing merge conflict resolutions $m_j$ as a classification task to predict between these possibilities. It is important to note that although we are predicting simple resolution strategies at the token-level, they may translate to complex resolutions at the line-level. In addition, not all conflicts are resolved by breaking that conflict into tokens and applying these patterns---some resolutions such as those introducing new tokens or reordering tokens are not expressible as a choice at the token-level.  

% Removing for now...
%One of the practical advantages of formulating merge conflict resolution as a classification task is a significant reduction in total FLOPS \Shuvendu{describe} required to decode a resolution region, as compared to generative models, making this approach an appealing candidate for deployment in IDEs (see section~\ref{sec:inference}). \sarah{appendix}


% Overview of basic mergeBERT
\section{\thistool{}: Neural Program Merge Framework}
\label{sec:main_model}

\thistool{} is a textual program merge model based on the bidirectional transformer encoder (BERT) model~\cite{bert}.
We refer the reader to CodeBERT~\cite{feng-etal-2020-codebert} for a discussion on applying transformers to code. A transformer, like
a recurrent neural network, maps a sequence of text into a high
dimensional representation, which can later be decoded to solve
downstream tasks. While not originally designed for code, transformers have found many applications in software engineering~\cite{clement2020pymt5,kanade2020learning,svyatkovskiy2020intellicode}

\thistool{} approaches merge conflict resolution as a sequence classification task given conflicting regions extracted with token-level differencing and surrounding code as context. 
%By focusing on token-level merge conflicts, we are able to resolve real-world merges. 
The key technical innovation in \thistool{} lies in how it breaks program text into an input representation amenable to learning with a transformer encoder and how it aggregates various input encodings for classification. 

In the standard sequence learning setting there is a single input and single output sequence. In the merge conflict resolution task, there are multiple conflicting input programs and one resolution. To facilitate learning in this setting, we construct \thistool{} as a multi-input encoder neural network, which first encodes token sequences of conflicting programs, then aggregates them into a single hidden summarization state. 

An overview of the \thistool{} model architecture is shown in Fig.~\ref{fig:mergebert}. Given conflicting programs $\mathcal{A}$, $\mathcal{B}$ and $\mathcal{O}$ we first perform tokenization and then repeat the three-way differencing at token granularity. If a conflict still exists in this token-level three-way differencing, we collect the token sequences corresponding to conflicting regions $a$, $b$, and $o$, and compute pair-wise alignments of $a$ and $b$ with respect to the base $o$. Finally, for each pair of aligned token sequences we extract an ``edit sequence'' that represents how to turn the second sequence into the first. The resulting aligned token sequences are fed to the multi-input encoder neural network, while the corresponding edit sequences are consumed as type embeddings. Finally, the encoded token sequences are summarized into a hidden state which serves as input to the classification layer. 

Given a 4-tuple of programs $(\mathcal{A}, \mathcal{B}, \mathcal{O}, \mathcal{M})$ which contains token-level merge tuples $(a_{j}, b_{j}, o_{j}, m_{j})$, j=0...N, \thistool{} models the following conditional probability distribution:
\begin{equation}
    p(m_{j} | a_{j}, b_{j}, o_{j}),
\end{equation}
and consequently, for entire programs:
\begin{equation}
    p(\mathcal{M} | \mathcal{A}, \mathcal{B}, \mathcal{O}) = \prod_{j=1}^{N} p(m_{j} | a_{j}, b_{j}, o_{j})
\end{equation}
Independence of token-level conflicts is a simplifying assumption. However, we observe that in our data set only 5\% of merge conflicts result in more than 1 token-level conflict per line-level conflict. 




\subsection{Context Encoding}

For a merge tuple $(a, b, o, m)$ \thistool{} calculates two pair-wise alignments between the sequences of tokens of conflicting regions $a$ (respectively $b$) with respect to that of the original program $o$: $a|_o$, $o|_a$, $b|_o$, and $o|_b$. For each pair of aligned token sequences we compute an edit sequence. These edit sequences -- $\Delta_{ao}$ and $\Delta_{bo}$ -- are comprised of the following editing actions (kinds of edits): $\textbf{=}$ represents equivalent tokens, $\textbf{+}$ represents insertions, $\textbf{-}$ represents deletions,
$\boldsymbol{\leftrightarrow}$ represents a replacement, and
$\boldsymbol{\emptyset}$ is used as a padding token. Overall, this produces four token sequences and two edit sequences: ($a|_{o}$,
$o|_{a}$, and $\Delta_{ao}$) and ($b|_{o}$, $o|_{b}$, and $\Delta_{bo}$). Fig.~\ref{fig:embedding} provides an example of an edit sequence. Each token sequence covers the corresponding conflicting region and, potentially, surrounding code tokens. We make use of Byte-Pair Encoding (BPE) unsupervised tokenization to avoid a blowup in the vocabulary size given the sparse nature of code identifiers~\cite{10.1145/3377811.3380342}.
To help the model learn to recognize editing steps we introduce an edit type embedding. We combine it with the standard token and position embeddings utilized in BERT model architecture via addition. 
%: $\mathcal{S} = \mathcal{S_{T}} + \mathcal{S_{P}} +\mathcal{S_{E}}$. 
%\alexey{Chris: I agree with the comment. We can comment out the formula since it is not used anywhere else. Or we can add one more layer of details: to explain how embeddings work in general. Since we use standard implementation here, I think it is not necessary.}
%\chris{@Alexey, this last paragraph probably needs a bit more explanation.  $\mathcal{S}$ is never defined and never appears anywhere else in the paper.  We should explain where it fits in fig 2 and how it connects to the rest of the model.  How do we arrive at the position and edit type embeddings?}
\begin{figure}
\begin{center}
    \includegraphics[width=.48\textwidth]{images/Embedding.pdf}
\caption{An example edit sequence extracted between a pair of token sequences.  Top row is $o|_b$, bottom is $b|_o$, and middle is $\Delta_{bo}$. Padding symbols \texttt{[PAD]} are introduced for alignment. In this case, the target token sequence is obtained by swapping a token and inserting two tokens.}
\label{fig:embedding}
\end{center}
\vspace{-12pt}
\end{figure}


\subsection{Merge Tuple Aggregation}
%\label{sec:mergesumm}

We utilize transformer encoder model $\mathcal{E}$ to independently encode each of the four token sequences of token-level conflicting regions $a|_{o}$, $o|_{a}$, $b|_{o}$, and $o|_{b}$, passing corresponding edit sequences $\Delta_{ao}$ and $\Delta_{bo}$ as type embeddings. Finally, \thistool{} aggregates the resulting encodings into a single hidden summarization state $h$:
\begin{dmath}
h = \sum_{x \in (a|_{o}, o|_{a}, b|_{o}, o|_{b})} \theta_{x} \cdot \mathcal{E} (x, \Delta_x)
\end{dmath}
where $\mathcal{E} (x, \Delta_x)$ are the encoded tensors for each of the sequences $x \in (a|_{o}, o|_{a}, b|_{o}, o|_{b})$, and $\theta_{x}$ are learnable weights. After aggregation a linear classification layer with \texttt{softmax} is applied:
\begin{equation}
      p(m_{j} | a_{j}, b_{j}, o_{j}) = \mathrm{softmax}(W\cdot h + b)
\end{equation}

The resulting line-level resolution region is obtained by concatenating the prefix, predicted token-level resolution $m_{j}$, and the suffix. Finally, in the case of a one-to-many correspondence between the original line-level and the token-level conflicts (see Appendix for more details and a pseudocode), \thistool{} uses a standard beam-search to decode the most promising predictions. 

%\subsection{Model Training}

%We exploit the traditional two-step pretraining and finetuning training procedure. First, we pretrain a transformer encoder $\mathcal{E}$ on a multilingual source code corpus with unsupervised masked language modeling (MLM) pretraining objective. 
%We transfer the weights of the pretrained transformer encoder~\cite{feng-etal-2020-codebert} into the \thistool{} multi-input neural network, and attach a randomly initialized linear layer with softmax. We then finetune the resulting neural network in a supervised setting for the sequence classification task. 
%See section~\ref{sec:implement} in the Appendix for more details about the implementation. 


\subsection{Implementation Details}
\label{sec:implement}

We utilize a pretrained CodeBERT\footnote{\url{https://huggingface.co/huggingface/CodeBERTa-small-v1}} model with 6 encoder layers, 12 attention heads, and a hidden state size of 768. The vocabulary is constructed using byte-pair encoding \citep{sennrich2015neural} and the vocabulary size is 50000. We transfer the weights of the pretrained transformer encoder into the \thistool{} multi-input neural network, and attach a randomly initialized linear layer with softmax. We then finetune the resulting neural network in a supervised setting for the sequence classification task. Input sequences for finetuning training cover conflicting regions and surrounding code (i.e., fragments of prefix and suffix of a conflicting region) up to a maximum length of 512 BPE tokens. The backbone of our implementation is HuggingFace's~\footnote{\url{https://github.com/huggingface/transformers}} \texttt{RobertaModel} and \\
\texttt{RobertaForSequenceClassification} classes in PyTorch, which are modified to turn the model into a multi-input architecture shown in Fig.~\ref{fig:mergebert}. 
We finetune \thistool{} with Adam stochastic optimizer with weight decay fix using a
learning rate of 5e-5, 512 batch size and 8 backward passes per \texttt{allreduce}. 
The finetuning training was performed on 4 NVIDIA Tesla V100 GPUs with 16GB memory for 6 hours. 

In the inference phase, the model prediction for each line-level conflict consists of one or more token-level predictions. Given the token-level predictions and the contents of the merged file, \thistool{} generates the code corresponding to the resolution region. The contents of the merged file include the conflict in question and its surrounding regions. Afterward, \thistool{} checks the syntax of the generated code with a tree-sitter\footnote{\url{https://tree-sitter.github.io/tree-sitter}} parser and outputs it as the candidate merge conflict resolution only if it is syntactically correct.


\section{Research Questions}

We pose the following research questions to evaluate the effectiveness of utility of \thistool{}.


\noindent \textbf{RQ\scriptsize{1}: }\textbf{\rqOne}
We evaluate \thistool{}'s performance of producting resolutions in terms of precision and accuracy of matching the actual user resolution extracted from real-world merge resolutions. We also provide a comparison \thistool{} to baseline approaches (at both the line and token level) and state of the art merge resolution approaches.

\noindent \textbf{RQ\scriptsize{2}: }\textbf{\rqTwo}
One of our primary goals is to be able to work on multiple languages with minimal effort.  
The core approach of \thistool{} is fundamentally language agnostic (though a parser and tokenizer is required for each additional language).  
We evaluate performance of \thistool{} across four languages and also compare the results of using four language-specific models (each trained on just one language) to using one multi-lingual model trained on the data from all four languages.

\noindent \textbf{RQ\scriptsize{3}: }\textbf{\rqThree}
We conduct an ablation study of the edit type embedding to understand and evaluate the impact of our novel edit-aware encoding on model performance.

\noindent \textbf{RQ\scriptsize{4}: }\textbf{\rqFour}
We conduct a user study involving a survey of real-world conflicts recently encountered by developers from large OSS projects. To understand how developers would use \thistool{} in practice, we provide them with an interface to explore \thistool{}'s conflict resolution suggestions in relation to their original conflicting code ask them evaluate suggestions and explain why they do or do not correctly resolve the merge conflict. 


\section{Dataset}
\label{sec:dataset}
%\sarah{add statistics about distribution of merge patterns}
%\alexey{I added some numbers in the section 4 (around line 270). Detailed numbers are in Appendix. We can move it up here if needed...}
%To create a dataset for self-supervised pretraining, we clone all non-fork repositories with more than 20 stars in GitHub that have C, C++, C\#, Python, Java, JavaScript, TypeScript, PHP, Go, and Ruby as their top language. The resulting dataset comprises over 64 million source code files. 
%\chris{why do we list languages here that we don't ever evaluate on?  A reviewer will find this confusing and ask about it.  We found that language specific models work better than multi-lingual models, right?}

The finetuning dataset is mined from over 100,000 open source software repositories in multiple programming languages with merge conflicts. It contains commits from git histories with exactly two parents, which resulted in a merge conflict.  We replay \texttt{git merge} on the two parents to see if it generates any conflicts. Otherwise, we ignore the merge from our dataset. We use the approach introduced by~\citet{Dinella2021} to extract resolution regions---however, we do not restrict ourselves to conflicts with less than 30 lines only.  Lastly, we extract token-level conflicts and conflict resolution classification labels (introduced in Section \ref{formulation}) from line-level conflicts and resolutions. Tab.~\ref{tab:fintuning_dataset} provides a summary of the finetuning dataset.

\begin{table}[htb]
\centering
\caption{Number of merge conflicts in the dataset.}
\begin{tabular}{llllllllllll} \toprule
\textbf{Programming language} & \textbf{Development set}  & \textbf{Test set} \\ \midrule
C\# & 27874 & 6969 \\ 
JavaScript & 66573 & 16644\\ 
TypeScript & 22422 & 5606\\ 
Java & 103065 & 25767 \\ 
\bottomrule
\end{tabular}
\label{tab:fintuning_dataset}
\end{table}
The finetuning dataset is split into development and test sets in the proportion 80/20 at random at the file-level. The development set is further split into training and validation sets in 80/20 proportion at the merge conflict level.    


\section{Evaluation}

\subsection{Evaluation Metrics}

We evaluate \thistool{}'s performance of resolution synthesis in terms of precision and accuracy of string match (modulo whitespaces or indentation) to the user resolution extracted from real-world historical merge resolutions. This approach is rather restrictive as a suggested resolution might differ from the actual user resolution by, for instance, only the order of statements, being semantically equivalent otherwise. As such, this evaluation approach gives a lower bound of performance.

We evaluate \thistool{} and compare it to baselines and existing approaches using two metrics, precision at top-k and accuracy at top-k.  
Since \thistool{} is a neural approach, it may provide more than one suggestion, which we rank according to the associated prediction probabilities.
In addition, because we filter out resolution suggestions that are not syntactically valid, it may provide no suggestions in rare cases.  
Accuracy at top-1 indicates the percentage of total conflicts for which \thistool{} produces the correct resolution as its top suggestion. Precision at top-1 indicates how often (as a percentage) the top suggestion is correct when the \thistool{} provides any suggestions at all.  As a concrete example, if a tool produces a resolution suggestion for 50 out of 100 conflicts and 40 of the suggestions matched the actual historical user resolution, then the precision would be 80\% (40/50), but the accuracy would be 40\% (40/100).  Precision at top-k indicates how often the correct resolution is found in the top-k suggestions and Accuracy at top-k is analogous. When ``top-k'' is omitted from the metric name (e.g. just "Precision") then k is 1.

%In addition to the precision and accuracy, we also report the fraction of syntactically correct (or parseable) source code suggestions to filter out merge resolutions with syntax errors. 

%\chris{we only report syntactic correctness in table 3 and no others.  I suggest we remove it, as reviewers will ask why it is so rarely present and it's not critical to the perf. evaluation.}

\subsection{Baseline Models}
\label{sec:baselines}

\subsubsection{Language Model Baseline}

Neural language models (LMs) have shown great performance in natural language generation~\citep{gpt2, sellam-etal-2020-bleurt}, and have been successfully applied to the domain of source code~\citep{10.5555/2337223.2337322, gptc, feng-etal-2020-codebert}. We consider the generative pretrained transformer language model for code (GPT-C) and appeal to the naturalness of software~\citep{naturalness} to construct our baseline approach for the merge resolution synthesis task. We establish the following baseline:
given an unstructured line-level conflict produced by \texttt{diff3}, we take the common source code prefix acting as user intent for program merge. We attempt to generate an entire resolution region token-by-token using beam search. As an ablation experiment, we repeat this for the conflicts produced with the token-level differencing algorithm (Fig.~\ref{fig:word1} shows details about prefix and conflicting regions).

\subsubsection{DeepMerge: Neural Model for Interleavings}

Next, we consider \textsc{DeepMerge}~\citep{Dinella2021}: a sequence-to-sequence model based on the bidirectional GRU summarized in section~\ref{sec:background}. It learns to generate a resolution region by choosing from line segments present in the input (line interleavings) with a pointer mechanism. We retrain the \textsc{DeepMerge} model on our TypeScript dataset.

\subsubsection{JDIME}
Looking for a stronger baseline, we consider \textsc{JDime}, a Java-specific merge tool that automatically tunes the merging process by switching between structured and unstructured merge algorithms \citep{apel2012structured}. Structured merge is abstract syntax tree (AST) aware and leverages syntactic information to improve matching precision of conflicting nodes.  We use the publicly available implementation~\citep{jdime}, and run JDime in semi-structured mode. 

\subsubsection{jsFSTMerge}
\citet{tavares2019javascript} implemented \jsfstmerge{} by adapting an off-the-shelf grammar for JavaScript to address shortcomings of \fstmerge{}~\cite{apel2012fstmerge} and modify its algorithm.
\jsfstmerge{} allows for certain types of nodes to maintain their relative order (\emph{e.g.}, statements) while others may be order independent (\emph{e.g.}, function declarations) even when sharing the same parent node.
For cases where \jsfstmerge{} produces a resolution not matching the user resolution, we manually inspect the output for semantic equivalence (e.g., reordered import statements).

\subsection{Results}
\label{sec:eval}


\noindent \textbf{RQ\scriptsize{1}: }\textbf{\rqOne}

To evaluate \thistool{} We first compare it to other neural approaches and to \texttt{diff3}. 
To be comprehensive, we evaluate at both the token level and the line level.  
We then compare \thistool{} to existing state of the art structured and semi-structured merge language-specific merge approaches.

\begin{table}[htb]
\small
\caption{Evaluation results for \thistool{} and various neural baselines calculated for merge conflicts in TypeScript programming language test set. The table shows top-1 precision and accuracy metrics.}
\centering
\begin{tabular}{lllllllllll} \toprule
\textbf{Approach}  & \textbf{Granularity} & {\textbf{Precision}} & {\textbf{Accuracy}} \\ \midrule
LM   & Line  &3.6 & 3.1 \\      % line 
DeepMerge & Line  & 55.0 & 35.1  \\ % Tyepscript aligned linearized
\midrule
\texttt{diff3} & Token & 82.4 & 36.1  \\
\midrule
LM & Token  & 49.7  & 48.1    \\      % token
DeepMerge & Token  & 64.5 & 42.7 \\ % Tyepscript aligned linearized
\thistool{} & Token  & \textbf{69.1} & \textbf{68.2}  \\  
\bottomrule   
\end{tabular}
\label{tab:baselines_left}
\end{table}




As seen in Tab.~\ref{tab:baselines_left}, language model baselines' performance on merge resolution synthesis is relatively low, suggesting that the naturalness hypothesis is insufficient to capture the developer intent when merging programs. This is perhaps not surprising given the notion of precision that does not tolerate even a single token mismatch. 

\thistool{} is based on two core components: token-level \texttt{diff3} and a multi-input neural transformer model. The token-level differencing algorithm alone gives a high top-1 precision of 82.4\%, with a relatively low accuracy of only 36.1\% (i.e., it doesn't always generate a resolution suggestion, but when it does, it is very often correct). Combined with the neural transformer model, the accuracy is increased to a total of 68.2\%. Note, as a deterministic algorithm token-level \texttt{diff3} can only provide a single suggestion. 

DeepMerge precision of merge resolution synthesis is quite admirable, showing 55.0\% top-1 precision. However, it fails to generate predictions for merge conflicts which are not representable as a line interleaving. This type of merge conflict comprises only roughly one third of the test set, resulting in an accuracy of only 35.1\% which is significantly lower than \thistool{}.

% \negar{should we mention both the token and the line-lvl precision of DeepMerge here?}.
% \alexey{Separated this as an ablation experiment below...}
% \negar{This number (63.8\%) is not in the table, and hence, a bit confusing. We can only mention the low accuracy and its reason, i.e., line interleaving.} 
% \alexey{Negar: I edited the paragraph above per your suggestion. Please take a look and make changes if needed}

As an experiment, we also evaluate the DeepMerge model in combination with the token-level \texttt{diff3}. This enables DeepMerge to overcome the limitation of providing only resolutions comprised of interleavings of lines from the conflict region by interleaving tokens instead. As seen in Tab.~\ref{tab:baselines_left} (DeepMerge with Token granularity) overall accuracy improves from 35.1\% to 42.7\%. However this still falls short of \thistool{} with precision that is 5\% less (64.5\% vs. 69.1\%) and accuracy that is 25\% less (42.7\% vs 68.2\%). 

%Second, note that DeepMerge can only produce resolution suggestions for merge conflicts that are representable as an interleaving of lines from the conflict region. As an experiment, we tried evaluating performance when restricting the test set to this type of merge conflicts only. Using this smaller test set of resolutions that DeepMerge could provide correct resolutions for, \thistool{} has 70.6\%  precision (not shown in Tab.~\ref{tab:baselines_left}) and Deepmerge has 55.0\%.


\begin{table}[htb]
\small
\caption{Comparison of \thistool{} to \jdime{} and \jsfstmerge{} semi-structured merge tools. The table shows the percentage of conflicts in which the tool produces a resolution, the top-1 precision of produced resolutions, and the overall top-1 accuracy of merge resolution synthesis. \jdime{} evaluation is on a Java data set and \jsfstmerge{} is on a JavaScript data set.}
\vspace{-4pt}
\centering
\resizebox{0.98\columnwidth}{!}{%
\begin{tabular}{lllllll} \toprule
\textbf{Approach} & \textbf{Language} & \textbf{\% conf. w/ res.} & \textbf{Precision} & \textbf{Accuracy} \\ %& \textbf{Syn (\%)} \\ 
\midrule
\jdime{} & Java & 82.1 & 26.3 & 21.6 \\ %&  90.9 \\
\thistool{} & Java & \textbf{98.9} & \textbf{63.9} & \textbf{63.2} \\ \midrule % & \textbf{98.3} \\ \midrule
\jsfstmerge & JavaScript & 22.8 & 15.8 & 3.6 \\ %& 94.4 \\
\thistool{} & JavaScript & \textbf{98.1} & \textbf{66.9} & \textbf{65.6} \\ %& \textbf{97.4} \\ % JavaScript
\bottomrule
\end{tabular}%
}
\label{tab:baselines_right}
\vspace{-6pt}
\end{table}

We also compared \thistool{} to state of the art structured and semi-structured merge tools.  Since both \jdime{} and \jsfstmerge{} are language-specific, to compare against \thistool{}, we use our dataset's corresponding language-specific subset of conflicts (leading to slightly different results for \thistool{} on Java and JavaScript).

As can be seen from Tab.~\ref{tab:baselines_right}, \jsfstmerge{} only produces a resolution for 22.8\% of conflicts and when a resolution is produced by \jsfstmerge{}, it is only correct 15.8\% of the time, yielding a total accuracy of 3.6\%. 
This is in line with the conclusions of the creators of \jsfstmerge{} that semi-structured merge approaches may not be as advantageous for dynamic scripting languages~\citep{tavares2019javascript}. Because \jsfstmerge{} may produce reformatted code, we manually examined cases where a resolution was produced but did not match the user resolution (our oracle).  If the produced resolution was semantically equivalent to the user resolution, we classified it as correct.

To compare the accuracy of \textsc{JDime} to that of \thistool{}, we use the Java Test data set introduced previously and complete the following evaluation steps: \textsc{JDime} does not merge all conflicts and generates a resolution for 82.1\% of conflicts. This is in line with related work reporting that as much as 21\% of files cannot be merged~\cite{apel2012structured}. Therefore, first, we identify the set of merge conflict scenarios where \texttt{diff3} reports a conflict and \textsc{JDime} produces a non-conflicted merge. 
% \Shuvendu{Why not cases where JDIME does not resolve the conflict?}, \sarah{This is because JDime only resolves conflicts 35\% of the time. If we were to consider all conflicts, accuracy would be lower than 15\% as reported in the ICSE paper. I believe the numbers here are from ICLR submission, so this decision is persisting from what we decided earlier. In the ICSE paper we did this differently. We reported the percentage of conflicts with predicted resolutions, precision as the number of correctly resolved conflicts over the total number of conflicts for which an approach attempts a resolution. Accuracy is the number of correctly resolved conflicts over the total number of conflicts. } 
When comparing the \textsc{JDime} output to the actual historical user-performed merge conflict resolution, we do not use a simple syntactic match.  As a result of its AST matching approach, code generated by \jdime{} is reformatted, and the original order of statements and other constructs are not always preserved. 
%In addition, source code comments that are part of conflicting code chunks are not merged. This makes a simple syntactic comparison is too restrictive, and \jdime{} merge output can still be semantically correct. 
In an effort to accurately and fairly identify semantically equivalent merges, we use GumTree \cite{FalleriMBMM14}, an AST differencing tool, to identify and ignore semantically equivalent differences between \textsc{JDime} output and the user resolution, such as reordered method declarations. When \textsc{JDime} produces a resolution, it generates a semantically equivalent match 26.3\% of the time, with an accuracy of 21.6\%. 
%we have a more accurate baseline comparison between the number of semantically equivalent merges generated by \jdime{} and \thistool{}.

\noindent \textbf{RQ\scriptsize{2}: }\textbf{\rqTwo}
One goal of our approach is to be able to handle multiple languages with minimal effort.  For \thistool{} to be able to provide merge resolution suggestions for conflicts in a particular language, it needs three things.  First, a tokenizer in that language, which allows us to split the source text into tokens for processing.  Second, a parser in that language, which allows us to filter out syntactically incorrect merge resolution suggestions. Third, a data set of merge conflicts and their user-resolutions to train \thistool{}.  Fortunately, tokenizers and parsers for nearly any language are readily available (e.g., we use GitHub's tree-sitter for this) and repositories that use a particular language can be easily identified (e.g. on GitHub) and mined for conflicts and resolutions.

We incorporated tokenizers and parsers into \thistool{} for JavaScript, TypeScript, Java, and C\# and gathered merge conflict data for these languages as described previously.  
Note that both comments and strings in these languages are represented as single tokens and can be quite long.  
Therefore we further split these tokens on whitespace.
Tab.~\ref{tab:mergebert_summary} shows the detailed evaluation results of \thistool{} broken down by language. The top section of results shows performance when \thistool{} is trained on data for that specific language.  
The bottom section shows performance for each language when \thistool{} is trained on a data set comprising data for all languages (we term this the \emph{multilingual} model).
Note that for the language specific models, performance is fairly consistent across all four languages with Top-1 precision ranging from 63.9\% to 69.1\% and Top-1 Accuracy ranging from 63.2\% to 68.2\%. We also find that over 97\% of \thistool{} suggestions are syntactically correct across all programming languages. 

We had no a priori expectations of the performance of the multilingual model, as it is trained on more data, which could lead to improvement, but it is not language specific, which could lead to poorer results.
%As can be seen, the multilingual variant of \thistool{} yields $63.6-68.5\%$ top-1 and $75.2-76.8\%$ top-3 precision of verbatim match and relatively high recall values.
Overall, the multilingual variant of the model generates results that are just slightly below the monolingual versions.
Thus performance on one language isn't improved by adding more data in other languages.
Thus, from a pragmatic perspective, if one chooses to simplify their use of \thistool{} by training just one model instead of one model per language, then the performance takes only a negligible hit.



\begin{table}
\small
\caption{Detailed evaluation results for (top) monolingual JavaScript, TypeScript, Java, and C\# models, and (bottom) multilingual \thistool{} model trained on all four programming languages. The table shows precision and accuracy of merge resolution synthesis.}
\vspace{-4pt}
\centering
\begin{tabular}{llllllllllll} \toprule
\textbf{Test (Train) Languages} & \multicolumn{2}{c}{\textbf{Precision}} &  \multicolumn{2}{c}{\textbf{Accuracy}}  \\ \cmidrule{2-3} \cmidrule{4-5} 
& Top-1 & Top-3 & Top-1 & Top-3 \\ 
\midrule
JavaScript (JS)  & 66.9 &75.4 & 65.6& 73.9 \\ %& 97.4 \\    % monolingual
TypeScript (TS)  & 69.1 &76.6 & 68.2& 75.6 \\ %& 97.0 \\   % monolingual
Java (Java)  & 63.9 &76.1 & 63.2 &75.2 \\ %& 98.3 \\  % monolingual
C\# (C\#)  & 68.7 &76.4 & 67.3& 74.8 \\ %& 98.3 \\   % monolingual
\midrule
JavaScript (JS, TS, C\#, Java)  & 66.6& 75.2 & 65.3 &73.8 \\ %& 97.4 \\   
TypeScript (JS, TS, C\#, Java)  & 68.5 &76.8 & 67.6 &75.8 \\ %& 96.9 \\   
Java (JS, TS, C\#, Java)  &  63.6 &76.0 & 62.9& 75.1 \\ %& 98.2 \\  
C\# (JS, TS, C\#, Java)  & 66.3 &76.2 & 65.1 &74.8 \\ %& 98.3\\  
\bottomrule
\end{tabular}
\label{tab:mergebert_summary}
\vspace{-4pt}
\end{table}

\noindent \textbf{RQ\scriptsize{3}: }\textbf{\rqThree}

We conduct an ablation study on the edit type embedding to understand the impact of edit-awareness of encoding on the model performance. As shown in Tab.~\ref{tab:edit_ablation}, use of the edit type embedding improves  \thistool{} from 63\% to 68\%.
\begin{table}[htb]
\small
\caption{Evaluation results for \thistool{} and the model variant without edit-type embedding for merge conflicts in TypeScript programming language test set. The table shows top-1 precision and accuracy metrics.}
\centering
\begin{tabular}{lllllll} \toprule
\textbf{Approach} & \textbf{Precision} & \textbf{Accuracy}   \\ 
\midrule
w/o edit type embeddings  & 65.2 & 63.1  \\
\thistool{} w/ edit type embeddings & \textbf{69.1} & \textbf{68.2}  \\ 
\bottomrule
\end{tabular}
\label{tab:edit_ablation}
\end{table}

%\negar{I have a general suggestion: We can move Tab. 4 up to the beginning of the results section. This way, for RQ1, which is about efficiency, we can mention and briefly discuss Tab. 4 as the detailed evaluation results of MergeBERT and continue with the comparison of MergeBERT and other techniques using Tab. 2 and 3. Then for RQ2, we can mention Tab. 4 again and further discuss the monolingual and multilingual results.}

\section{User Evaluation}
\begin{figure}[h] %!th
   \includegraphics[width=0.46\textwidth, angle=0]{images/method.png}
  \caption{Methodology to identify candidate conflicts for the user study.}
  \label{fig:methodology}
  \vspace{-12pt}
\end{figure}

\begin{table}[t!]
\small
\centering
\caption{Summary of projects in user study, total number of conflicts per project, number of conflicts evaluated in the study, and the survey participants.}
\label{tab:projects}
\begin{tabular}{llcccl} \toprule
Language & Project & Conflicts & Survey & Participants \\
 &  &  &  Conflicts &  \\\midrule
\multirow{3}{*}{Java} & Azure-Cosmosdb  & 341 & 6 & P1 \\
 & Azure-SDK  & 997 & 14 & P2-4 \\
 & ApplicationInsights&  313 & 10 & P5-6 \\ \midrule
\multirow{2}{*}{TS} & MakeCode & 106 & 12 & P7-8 \\
 & VSCode  & 2256 & 48 & \begin{tabular}[c]{@{}l@{}}P9-17\end{tabular} \\ \midrule
\multirow{3}{*}{C\#} & AspNetCore  & 567 & 11 & P18-19 \\
 & EFCore  & 397 & 7 & P20-21 \\
 & Roslyn  & 1894 & 14 & P22-25 \\ \midrule
 Total & 8 projects  & 6871 & 122 & 25 \\ \bottomrule
\end{tabular}
\end{table} 

\subsection{User Study Design}

To better understand how \thistool{} performs in practice, we ask developers about conflicts that \thistool{} is unable to correctly resolve. Since \thistool{}'s resolution suggestions are evaluated against user resolutions using a verbatim string match (modulo whitespace), asking study participants to confirm identical resolutions predicted by \thistool{} is not informative. Therefore, we extract conflicts where \thistool{} suggestions are not a direct match to the user resolution to determine what the limitations of the suggestions are, and how they might be perceived in practice.

To build an oracle of merge conflicts and resolutions we identify 8 open source projects hosted on GitHub. The selected projects are active, with multiple contributors, and contain a large number of conflict scenarios in one of the languages supported by \thistool{}.
Tab.~\ref{tab:projects} contains a list of projects chosen.
% EFCore\footnote{https://github.com/dotnet/efcore}
% Roslyn\footnote{https://github.com/dotnet/roslyn}
% AspNetCore\footnote{https://github.com/dotnet/aspnetcore}  
% VSCode\footnote{https://github.com/microsoft/vscode}
% MakeCode\footnote{https://github.com/microsoft/pxt} 
% Azure-SDK-Java\footnote{https://github.com/Azure/azure-sdk-for-java} 
% Azure-Cosmosdb-Java\footnote{https://github.com/Azure/azure-cosmosdb-java}
For each project, we follow the same steps outlined in Section~\ref{sec:dataset} to extract candidate conflicts and user resolutions to use in the survey.



Fig.~\ref{fig:methodology} explains the methodology used to identify candidate merge conflicts. We identify the set of conflicts \thistool{} is unable to correctly merge (within the top-3 suggestions). From this set of conflicts, we identify candidate conflicts to use as part of the user study. We filter candidate files with the following criteria:
\begin{enumerate}
  \item Conflicts should have been recently resolved i.e., at most within the past 12 months. Participants may not retain the context needed to evaluate suggestions for older conflicts. 
    \item Files must have at most 4 conflicts. Participants evaluate up to 3 suggestions per conflict. More conflicts may be too complex to evaluate within the interview time slot. 
    \item Conflicts should be non-trivial.  Trivial conflicts, such as those that only involve formatting changes or renames, are manually excluded. The determination of if a conflict was non-trivial was manual and subjective, informed by our belief that more substantive conflicts would lead to more insights in the user study.
\end{enumerate}

%\sarah{this might be the place to add discussion of subsumptive merge conflicts?}
For each candidate conflict identified, we use the GitHub API to identify authors for each of the conflicting branches and the resolved file. Authors with at least 3 candidate merge conflicts are identified as potential survey participants. Our final pool of candidate participants consists of 52 unique authors. We recruit participants via email, using contact information on GitHub. Out of the 52 contacted developers, 25 agreed to participate in the study. All participants were professional software developers with at least 2-8 years of experience working at large technology companies. We asked participants to evaluate \thistool{} resolution suggestions for their own merge conflicts.  Tab.~\ref{tab:projects} contains the final number of participants and conflicts evaluated in our study. 122 conflicts were evaluated: 32 C\# conflicts, 30 Java, and 60 Typescript. 

\subsubsection{\thistool{} Interface}
We designed an online interface where participants can view their own conflicts and explore \thistool{}'s resolution suggestions. Participants are asked to evaluate their own recently resolved merge conflicts, and the corresponding generated resolution suggestions by \thistool{}. The interface is customized based on the signed-in participant and displays a list of their recently encountered merge conflicts. Participants can click through different resolution suggestions to evaluate if they are acceptable ways to resolve the merge conflict. They can view their original resolution on the same page, and if needed, participants can navigate to the conflicting commit on GitHub using a link if they need additional context. They can also view a diff between the conflict file and any of the selected options (resolution suggestion or user resolution). Participants use this interface to select one or more of the suggested resolutions, indicate if the suggested resolution is acceptable, and explain the reasons why or why not.  Our online data package~\cite{ICSE22Replication} and appendix~\cite{FSE22Appendix} contain the questions, images of the interface, and participant responses.  

% \begin{figure}[h!] %!th
%  \includegraphics[width=0.5\textwidth, angle=0]{images/survey.png}
%  \caption{Interface used by participants to interact with \thistool{} resolution suggestions and answer survey questions.}
%  \label{fig:survey}
% \end{figure}


\subsubsection{Protocol}
The user study was conducted as 30 minute interviews remotely over Microsoft Teams using the interface we built. First, participants watched a video explaining \thistool{} and how to navigate conflicts and evaluate resolution suggestions using the interface. Then, the participants evaluated a set of conflicts and submitted their responses. One of the authors was on the teams call to help participants navigate the interface and ask any clarifying questions based on their evaluation of the \thistool{} resolution suggestions.
% Then participants were asked a list of questions on the following topics: (i) Their existing process to resolve merge conflicts, and obstacles faced, (ii) trust of existing merging algorithms and proposed approaches, and (iii) utility of suggestion-based merge conflict resolution tools.
Questions were iteratively developed based on two pilot interviews. Each interview was recorded for transcription and analysis. 
%Direct quotes used in the results were manually validated by the authors.

% \subsubsection{Analysis}
% \sarah{update to only describe survey result analysis}
% We analyzed the submitted survey responses, video recordings and generated transcripts of the semi-structured interviews. We divided the transcripts into different sections reflecting the semi-structured interview questions and then used open coding for each topic, looking for similarities and differences between the interviewees’ responses. 




\subsection{User Study Results}


\noindent \textbf{RQ\scriptsize{4}: }\textbf{\rqFour}


Using the interface participants evaluate the conflict resolution suggestions generated by \thistool{} and indicate if any of the suggestions were acceptable, and explain why or why not. There were no noticeable differences in the participants' responses across different languages or projects so we do not break down our results by those dimensions.
Participant's evaluations of the merge suggestions generally fall into three categories: 1) the merge suggestion is correct and would be used to resolve the conflict 2) the merge is incorrect but the correct resolution would require an understanding of external context and 3) the merge is incorrect and no external context is needed.  

\subsubsection{Acceptable Merge Suggestions}

Surprisingly, of the 122 conflicts included in the study, participants indicated that at least one of the 3 suggestions generated by \thistool{} was correct for 54\% (66/122) of the examples. By design, the suggestions presented in the study are not syntactically equivalent to the participant's original resolution, however, they still indicated that the suggestion was a correct merge. Using participant responses, we identify a few reasons why merge suggestions may be acceptable to a developer, even if it is not syntactically equivalent to their original resolution:

\vspace{6pt}
\noindent{}\textbf{Semantically Equivalent Resolution} (54 of 122 conflicts) \\
    Semantically equivalent resolutions include scenarios where the statements are re-ordered, equivalent changes made to naming or documentation, and unneeded import statements or commented out code is  preserved or removed. 
   
      One example in the study of conflicting changes that are both equally acceptable, and one is arbitrarily accepted by the resolving author is when authors of conflicting branches renamed the same variable with a slight variation:\\ \ic{SPAN\_TARGET\_ATTRIBUTE\_NAME} and \\ \ic{SPAN\_TARGET\_APP\_ID\_ATTRIBUTE\_NAME}. In these cases, either version selected by the merging algorithm might still be acceptable to the developer. \textsc{MergeBert} generated a suggestion to keep the variable name \ic{SPAN\_TARGET\_ATTRIBUTE\_NAME} whereas the user resolution originally kept the other. Participant P5 marked this resolution as acceptable and semantically equivalent, explaining that in this scenario they had `no preference as to which one is better'.
%\sarah{add more examples if we have room}      

\mybox{Takeaway 1}{grey!20}{grey!7}{Evaluating the performance of \thistool{} using strict syntactic approaches estimates a lower bound of performance. Survey results show  almost 45\% of \thistool{} suggestions are acceptable merges that are semantically equivalent to the participant's original resolution. \thistool{}'s performance could be improved by considering semantic information, for example, to identify how changes related to naming or documentation should be merged.}
\noindent{}\textbf{Tangled Code Changes in Oracle} (10/122) \\
Resolutions for 10 of the conflicts contained additional ``tangled'' changes~\cite{Herzig:msr13:ImpactOfTangledCodeChanges,Kirinuki:icpc14:TangledChanges} that were unrelated to the resolution. Examples include renaming a method and adding a variable in the conflict region that is then used later outside the conflict region.
In all 10 instances, \textsc{MergeBert} generates a suggestion that does not include the additional tangled code, but is acceptable to the participant as a resolution of the conflict. 
Participants indicated that if they had access to the \textsc{MergeBert} suggestions, they would select the correct resolution and then manually add the additional code. 

\mybox{Takeaway 2}{grey!20}{grey!7}{When committing merged code, developers may introduce changes unrelated to the conflict which are inadvertently included in conflict resolution oracles. These changes can negatively impact model performance estimated with automatic metrics.}




\subsubsection{Merge Requires External Context}
 \textsc{MergeBert} did not generate an acceptable suggestion for 46\% (56/122) of examples shown to survey participants. Participants were asked to indicate whether they resolved these examples using external context that cannot be inferred from the conflicting code regions and to explain what the external context was.  
Results indicate that 16\% (20/122) of conflicts in the survey sample require external information not found in either conflicting file, in order to be correctly resolved. 
One example of external context is knowledge of linter rules enabled at a project level. Projects often require linter checks before code can be committed to the repository, as a step towards improving the quality and maintainability of the source code. One example is a merge conflict from Roslyn where the correct resolution was to remove a null check from the code. Participant P23 explained the decision to remove the check: \emph{"The previousResults parameter is non-nullable because C\# nullability checking is now enabled at the project level. The null check is unnecessary"}. In this scenario, without specific knowledge of linter checks, an automatic approach is unable to predict an accurate merge. 
 
 Another example of external context is updates to languages rules that have cascading effects on existing code. Participant P22 from the Roslyn project explained one such conflict:  "Changes were due to updates in '\ic{using}' rules for the C\# language".  Language updates in C\# version 8.0 introduced an alternative syntax for the \ic{using} statement and P22's team had made to adopt this syntax.  P22 therefore updated this code (involved in the conflict) during the merge. Other examples of external context identified through the survey include: removal of global dependencies from non-conflicting files within a project, rolling back features that shouldn't be included in a release branch, and project-level decisions to remove \ic{'final'} modifiers for variables. 

\mybox{Takeaway 3}{grey!20}{grey!7}{The local view of a conflict is sufficient to merge a majority of conflicts. Around 16\% of the conflicts  require external information to correctly resolve. One direction to improve \thistool{} is to consider external context as an additional information source for resolving conflicts.}

\subsubsection{Unacceptable Merge Suggestions}
Survey results show that \textsc{MergeBert} suggestions were incorrect for 29\% (36/122) of the conflicts. Participants indicated that none of the 36 conflicts required external context to be resolved. We manually analyze the conflicts looking to identify patterns that may explain the incorrect merges, for example, affected language construct~\cite{pan2021ProgramSynthesis} and type of conflict~\cite{shen2021automatic}, but do not identify any consistent patterns. 
% \Shuvendu{Is this really true? Wasn't extraneous and subsumptive merges came here?} \sarah{subsumptive conflicts are filtered out before the survey since they have obvious resolutions. These unacceptable suggestions are for other conflicts where MergeBert is suggesting something that really didn't make sense.}
In summary, existing automatic evaluation strategies estimate a lower bound of approach performance: \thistool{} suggestions are correct for 54\% of conflicts included in our sample, despite not being syntactically equivalent to the user resolution.  Further, suggestions from \thistool{} helped two participants find bugs in their own recent merge conflict resolutions!  This is in addition to those resolutions where \thistool{} does provide an exact match.  This finding suggests that automatic evaluation techniques that rely on a strict syntactic comparison between the user resolution and merge suggestion might be estimating a much lower bound of performance. This highlights a discrepancy between how approaches are typically automatically evaluated, and how developers may evaluate an approach in practice.  Researchers should consider conducting user studies to more accurately evaluate approaches when feasible. 
 Tools like \thistool{} can reduce effort and bug proneness involved in manually merging conflicts. Future studies should investigate these potential benefits. 
%\chris{definitely include that we found bugs in two developer resolutions.  In addition, we should reiterate that the 54\% of correct merges resolutions is out of the percentage that \thistool{} didn't already match exactly correctly.  We don't want the reader to think that we learned that \thistool{} is correct only 54\% of the time.}
% \sarah{subsumptive conflicts are one way MergeBert can significantly improve. We need to find a place to introduce them, explain how they are filtered out of the survey questions, their proportion.}
%%%%%%%%%%%%%%%%%%%%%%%%%%%%%%%%%%%%%%%%%%%%%%%%%%%%%%%%

\section{Related Work}
%\mz{We lack a comparison to this paper: https://arxiv.org/abs/2305.14877}
%\anirudh{refine to be more on-topic?}
\iffalse
\paragraph{In-Context Learning} As language models have scaled, the ability to learn in-context, without any weight updates, has emerged. \cite{brown2020language}. While other families of large language models have emerged, in-context learning remains ubiquitous \cite{llama, bloom, gptneo, opt}. Although such as HELM \cite{helm} have arisen for systematic evaluation of \emph{models}, there is no systematic framework to our knowledge for evaluating \emph{prompting methods}, and validating prompt engineering heuristics. The test-suite we propose will ensure that progress in the field of prompt-engineering is structured and objectively evaluated. 

\paragraph{Prompt Engineering Methods} Researchers are interested in the automatic design of high performing instructions for downstream tasks. Some focus on simple heuristics, such as selecting instructions that have the lowest perplexity \cite{lowperplexityprompts}. Other methods try to use large language models to induce an instruction when provided with a few input-output pairs \cite{ape}. Researchers have also used RL objectives to create discrete token sequences that can serve as instructions \cite{rlprompt}. Since the datasets and models used in these works have very little intersection, it is impossible to compare these methods objectively and glean insights. In our work, we evaluate these three methods on a diverse set of tasks and models, and analyze their relative performance. Additionally, we recognize that there are many other interesting angles of prompting that are not covered by instruction engineering \cite{weichain, react, selfconsistency}, but we leave these to future work.

\paragraph{Analysis of Prompting Methods} While most prompt engineering methods focus on accuracy, there are many other interesting dimensions of performance as well. For instance, researchers have found that for most tasks, the selection of demonstrations plays a large role in few-shot accuracy \cite{whatmakesgoodicexamples, selectionmachinetranslation, knnprompting}. Additionally, many researchers have found that even permuting the ordering of a fixed set of demonstrations has a significant effect on downstream accuracy \cite{fantasticallyorderedprompts}. Prompts that are sensitive to the permutation of demonstrations have been shown to also have lower accuracies \cite{relationsensitivityaccuracy}. Especially in low-resource domains, which includes the large public usage of in-context learning, these large swings in accuracy make prompting less dependable. In our test-suite we include sensitivity metrics that go beyond accuracy and allow us to find methods that are not only performant but reliable.

\paragraph{Existing Benchmarks} We recognize that other holistic in-context learning benchmarks exist. BigBench is a large benchmark of 204 tasks that are beyond the capabilities of current LLMs. BigBench seeks to evaluate the few-shot abilities of state of the art large language models, focusing on performance metrics such as accuracy \cite{bigbench}. Similarly, HELM is another benchmark for language model in-context learning ability. Rather than only focusing on performance, HELM branches out and considers many other metrics such as robustness and bias \cite{helm}. Both BigBench and HELM focus on ranking different language model, while fix a generic instruction and prompt format. We instead choose to evaluate instruction induction / selection methods over a fixed set of models. We are the first ever evaluation script that compares different prompt-engineering methods head to head. 
\fi

\paragraph{In-Context Learning and Existing Benchmarks} As language models have scaled, in-context learning has emerged as a popular paradigm and remains ubiquitous among several autoregressive LLM families \cite{brown2020language, llama, bloom, gptneo, opt}. Benchmarks like BigBench \cite{bigbench} and HELM \cite{helm} have been created for the holistic evaluation of these models. BigBench focuses on few-shot abilities of state-of-the-art large language models, while HELM extends to consider metrics like robustness and bias. However, these benchmarks focus on evaluating and ranking \emph{language models}, and do not address the systematic evaluation of \emph{prompting methods}. Although contemporary work by \citet{yang2023improving} also aims to perform a similar systematic analysis of prompting methods, they focus on simple probability-based prompt selection while we evaluate a broader range of methods including trivial instruction baselines, curated manually selected instructions, and sophisticated automated instruction selection.

\paragraph{Automated Prompt Engineering Methods} There has been interest in performing automated prompt-engineering for target downstream tasks within ICL. This has led to the exploration of various prompting methods, ranging from simple heuristics such as selecting instructions with the lowest perplexity \cite{lowperplexityprompts}, inducing instructions from large language models using a few annotated input-output pairs \cite{ape}, to utilizing RL objectives to create discrete token sequences as prompts \cite{rlprompt}. However, these works restrict their evaluation to small sets of models and tasks with little intersection, hindering their objective comparison. %\mz{For paragraphs that only have one work in the last line, try to shorten the paragraph to squeeze in context.}

\paragraph{Understanding in-context learning} There has been much recent work attempting to understand the mechanisms that drive in-context learning. Studies have found that the selection of demonstrations included in prompts significantly impacts few-shot accuracy across most tasks \cite{whatmakesgoodicexamples, selectionmachinetranslation, knnprompting}. Works like \cite{fantasticallyorderedprompts} also show that altering the ordering of a fixed set of demonstrations can affect downstream accuracy. Prompts sensitive to demonstration permutation often exhibit lower accuracies \cite{relationsensitivityaccuracy}, making them less reliable, particularly in low-resource domains.

Our work aims to bridge these gaps by systematically evaluating the efficacy of popular instruction selection approaches over a diverse set of tasks and models, facilitating objective comparison. We evaluate these methods not only on accuracy metrics, but also on sensitivity metrics to glean additional insights. We recognize that other facets of prompting not covered by instruction engineering exist \cite{weichain, react, selfconsistency}, and defer these explorations to future work. 

%!TEX root = ../paper.tex

\section{Threats to Validity}
\label{sec:threats_to_validity}

%In this section, we discuss possible threats to the validity of our methodology and experiments. 

\InlineSec{Construct Validity}
Any detector's performance is dependent on its configuration. 
%it is possible that the detectors we compared against can perform better on our benchmark, given a different configuration. 
Due to the high effort of reviewing findings, we could not try different configurations for each detector.
However, to give each detector a fair chance, we used the optimal configurations reported in the respective publications.

Our study focuses on static misuse detectors.
Approaches based on dynamic analyses may perform differently and have unique strengths and weaknesses.
To enable dynamic analyses of the project versions in \MUBench, we would have to ensure that the respective code is executable (which requires a sufficient run-time environment, in addition to compile-time dependencies) and to provide example inputs for the execution.
It is unclear how to do this such that it results in a fair comparison of both static and dynamic techniques, without resorting to comparing apples to oranges.
In this work, we focused only on static approaches.

Our experiments focus on detectors that detect misuses in Java code.
Therefore, the results may not generalize to detectors for other languages.
We decided to focus on this subset of detectors, because the majority of approaches we identified in our survey targets Java.
To include detectors that target other languages, we would have to either migrate them to Java or build up additional datasets for the respective languages, both of which is outside the scope of this work.

\InlineSec{Internal Validity}
Reviewing the detectors' findings was done by three of the authors and was not blind (i.e., we knew the detectors we were reviewing findings for).
We could not do blind reviewing, because each approach has a distinct representation of usages and violations that cannot be anonymized.
Moreover, two of the authors of this work are among the original authors of \GROUMiner.
We did our best to review objectively.
To avoid bias, every finding was independently reviewed by two authors and for all findings of \GROUMiner, at least one review was done by an author who was not involved in the original work.

% \sa{Can we roughly quantify our review effort?}~\sn{perhaps using an average of 2min per review is fair? a lot of them took less but some took a bit and if we include discussion then 2min is very fair}~\sa{Taking the number of potential hits ($122 + 230 + 42$) times 2 reviews times 2min/review means $26.3h$ of review work in total. The efforts per detector are 4h for \Jadet, 7.8h for \GROUMiner, 5h for \Tikanga, and 9h for \DMMC. Should we put this as an argument for not involving the original authors?}~\sn{yeah i guess we can. I had a feeling that it took more time than this, but maybe I'm mistaken :D}~\sa{It took more time, because we rereviewed (parts of) the results several times after doing changes, but this is not workload somebody else would have to assess our final results. We also reviewed MuDetect, which is not in here anymore.}
%
While we did ask the original authors to confirm our assessment of the conceptual capabilities of their tools, we did not ask them to confirm the empirical results of our experiments.
We estimate that, including discussions to resolve disagreements, it required each reviewer on average \checkNum{2 minutes to verify whether a detector identified one of the known misuses in Experiments RUB and R} and \checkNum{5 minutes to verify whether a detector's finding identifies an actual misuse in \nameref{e2}}, where we needed to understand the respective code, check documentation, and sometimes also look into transitively called methods.
This amounts to \checkNum{24.8 hours of review effort} per reviewer, \checkNum{4 hours for \Jadet}, \checkNum{7.2 hours for \GROUMiner}, \checkNum{4.7 hours for \Tikanga}, and \checkNum{8.9 hours for \DMMC}.
We decided it is unreasonable to expect the original authors to invest this amount of time in verifying our assessments.
We do, however, publish all our review data~\cite{artifact-page} to allow them and others to revisit our decisions.

% For \Jadet, we measure a similar precision as in the original evaluation (10% vs 9-12%)
% For \Tikanga, we measure a better precision than in the original evaluation (11% vs 6%)
% For \DMMC, we measure a much worse precision than in the original evaluation (9% vs. 85%)

\InlineSec{External Validity}
There may be violation categories we miss in \MUC.
The \MUBench dataset may also not have enough examples of all violations.
This may impact the detectors' comparisons.
However, the existing \MUBench dataset is based on over 1,200 reports from state-of-the-art bug datasets as well as developer input~\cite{ANNN+16} and the results of two empirical studies on API usage directives.
Our survey of existing detectors' capabilities also includes \checkNum{12 detectors}.
This makes it unlikely that we miss a prevalent violation category.

Our dataset may not be representative of all possible real-world API misuses, especially, because we could only compile \checkNum{29 (52\%)} of the \checkNum{55 project versions} and had to exclude the misuses in the other versions from our experiments.
Compiling arbitrary versions of projects from the source control history of project is a challenging task.
We invested two full weeks work of one of the authors and additional 3 months work of a student, to include as many project versions as possible.
Still, loosing the examples for which we could not compile the respective project versions may bias the results of our experiments.

Ideally, our experiments would include thousands of misuses from a large number of projects and in each individual project version, to give us greater confidence in the generalizability of our results.
However, currently, there is no such dataset.
We invested several months of effort to collect and prepare \MUBench in its current state, to make a first step towards a large benchmark.
Now that the we have the infrastructure in place, it is straightforward to extend \MUBench with misuse examples from different sources.

We publish \MUPipe and \MUBench~\cite{mubench} and encourage others to extend the dataset and repeat our experiments, also with other detectors and detector configurations.


% \vspace{-0.5em}
\section{Conclusion}
% \vspace{-0.5em}
Recent advances in multimodal single-cell technology have enabled the simultaneous profiling of the transcriptome alongside other cellular modalities, leading to an increase in the availability of multimodal single-cell data. In this paper, we present \method{}, a multimodal transformer model for single-cell surface protein abundance from gene expression measurements. We combined the data with prior biological interaction knowledge from the STRING database into a richly connected heterogeneous graph and leveraged the transformer architectures to learn an accurate mapping between gene expression and surface protein abundance. Remarkably, \method{} achieves superior and more stable performance than other baselines on both 2021 and 2022 NeurIPS single-cell datasets.

\noindent\textbf{Future Work.}
% Our work is an extension of the model we implemented in the NeurIPS 2022 competition. 
Our framework of multimodal transformers with the cross-modality heterogeneous graph goes far beyond the specific downstream task of modality prediction, and there are lots of potentials to be further explored. Our graph contains three types of nodes. While the cell embeddings are used for predictions, the remaining protein embeddings and gene embeddings may be further interpreted for other tasks. The similarities between proteins may show data-specific protein-protein relationships, while the attention matrix of the gene transformer may help to identify marker genes of each cell type. Additionally, we may achieve gene interaction prediction using the attention mechanism.
% under adequate regulations. 
% We expect \method{} to be capable of much more than just modality prediction. Note that currently, we fuse information from different transformers with message-passing GNNs. 
To extend more on transformers, a potential next step is implementing cross-attention cross-modalities. Ideally, all three types of nodes, namely genes, proteins, and cells, would be jointly modeled using a large transformer that includes specific regulations for each modality. 

% insight of protein and gene embedding (diff task)

% all in one transformer

% \noindent\textbf{Limitations and future work}
% Despite the noticeable performance improvement by utilizing transformers with the cross-modality heterogeneous graph, there are still bottlenecks in the current settings. To begin with, we noticed that the performance variations of all methods are consistently higher in the ``CITE'' dataset compared to the ``GEX2ADT'' dataset. We hypothesized that the increased variability in ``CITE'' was due to both less number of training samples (43k vs. 66k cells) and a significantly more number of testing samples used (28k vs. 1k cells). One straightforward solution to alleviate the high variation issue is to include more training samples, which is not always possible given the training data availability. Nevertheless, publicly available single-cell datasets have been accumulated over the past decades and are still being collected on an ever-increasing scale. Taking advantage of these large-scale atlases is the key to a more stable and well-performing model, as some of the intra-cell variations could be common across different datasets. For example, reference-based methods are commonly used to identify the cell identity of a single cell, or cell-type compositions of a mixture of cells. (other examples for pretrained, e.g., scbert)


%\noindent\textbf{Future work.}
% Our work is an extension of the model we implemented in the NeurIPS 2022 competition. Now our framework of multimodal transformers with the cross-modality heterogeneous graph goes far beyond the specific downstream task of modality prediction, and there are lots of potentials to be further explored. Our graph contains three types of nodes. while the cell embeddings are used for predictions, the remaining protein embeddings and gene embeddings may be further interpreted for other tasks. The similarities between proteins may show data-specific protein-protein relationships, while the attention matrix of the gene transformer may help to identify marker genes of each cell type. Additionally, we may achieve gene interaction prediction using the attention mechanism under adequate regulations. We expect \method{} to be capable of much more than just modality prediction. Note that currently, we fuse information from different transformers with message-passing GNNs. To extend more on transformers, a potential next step is implementing cross-attention cross-modalities. Ideally, all three types of nodes, namely genes, proteins, and cells, would be jointly modeled using a large transformer that includes specific regulations for each modality. The self-attention within each modality would reconstruct the prior interaction network, while the cross-attention between modalities would be supervised by the data observations. Then, The attention matrix will provide insights into all the internal interactions and cross-relationships. With the linearized transformer, this idea would be both practical and versatile.

% \begin{acks}
% This research is supported by the National Science Foundation (NSF) and Johnson \& Johnson.
% \end{acks}


\bibliography{refs}
\bibliographystyle{ACM-Reference-Format}

\end{document}