\documentclass[sigconf, screen]{acmart}

%\settopmatter{printacmref=false}


%%%%% NEW MATH DEFINITIONS %%%%%

\usepackage{amsmath,amsfonts,bm}
\usepackage{xifthen}

% Highlight a newly defined term
\newcommand{\newterm}[1]{{\bf #1}}

\def\eps{{\epsilon}}


% Utility for ticks 
\newcommand{\cmark}{\ding{51}}%
\newcommand{\xmark}{\ding{55}}%

% Theorem styles 
\theoremstyle{definition}
\newtheorem{theorem}{Theorem}[section]
\newtheorem{definition}{Definition}[section]
% \newtheorem{remark}{Remark}[theorem] %numbered remark
\newtheorem*{remark}{Remark} %unnumbered remark
\newtheorem{lemma}{Lemma}[section]
\newtheorem{prop}{Proposition}[section]
\newtheorem{corollary}{Corollary}[theorem]
\newtheorem{conjecture}{Conjecture}[section]
\newtheorem{assumption}{Assumption}[section]

\newtheorem{manualtheoreminner}{Theorem}
\newenvironment{manualtheorem}[1]{%
  \renewcommand\themanualtheoreminner{#1}%
  \manualtheoreminner
}{\endmanualtheoreminner}


% Math helper - standard function
\DeclareMathOperator*{\argmax}{arg\,max}
\DeclareMathOperator*{\argmin}{arg\,min}
\DeclareMathOperator{\support}{support}
\DeclareMathOperator{\MAX}{MAX}
\DeclareMathOperator{\term}{\texttt{term}}
\DeclareMathOperator*{\logsumexp}{log-sum-exp}
\DeclareMathOperator*{\TV}{TV}
\newcommand{\norm}[1]{\left\lVert#1\right\rVert}
\DeclarePairedDelimiter\set\{\}
\DeclarePairedDelimiter\abs{\lvert}{\rvert}%
\newcommand*{\mytop}{\mathrel{\scalebox{0.5}{$\top$}}}
\newcommand*{\mybot}{\mathrel{\scalebox{0.5}{$\bot$}}}
\newcommand*{\mydiese}{\mathrel{\scalebox{0.5}{$\#$}}}
\newcommand*{\myplus}{\mathrel{\scalebox{0.5}{$+$}}}
\newcommand*{\myminus}{\mathrel{\scalebox{0.5}{$-$}}}
\newcommand*{\bmg}{\bm{\gamma}}
\newcommand*{\bml}{\bm{\lambda}}

% MDP notation
\renewcommand{\S}{\mathcal{S}}
\newcommand{\X}{\mathcal{X}}
\newcommand{\A}{\mathcal{A}}
\newcommand{\T}{\mathcal{T}}
\newcommand{\M}{\mathcal{M}}
\newcommand{\B}{\mathcal{B}}
\newcommand{\Bset}{\mathfrak{B}}
\newcommand{\Dist}{\mathscr{P}}
\newcommand{\D}{\mathcal{D}}
\newcommand{\Real}{\mathbb{R}}
\renewcommand{\P}{\mathcal{P}}
\newcommand{\E}{\mathop{\mathbb{E}}}
\renewcommand{\H}{\mathcal{H}}
% \newcommand{\R}{\mathcal{R}}
% \newcommand{\C}{\mathcal{C}}

% Extended MDP notation
\newcommand{\Pstar}{p^{\star}}
\newcommand{\Rstar}{\bm{r}^{\star}}
\newcommand{\Cstar}{C^{\star}}
% \newcommand{\rmax}{\textsc{Rmax}}
\newcommand{\rmax}{r_{\mytop}}
\newcommand{\cmax}{\textsc{Cmax}}

\newcommand{\mstar}{m^{\star}}
\newcommand{\mhat}{\hat{m}}
\newcommand{\mopt}{m^{\star}}

\newcommand{\Phat}{\hat{p}}
\newcommand{\Rhat}{\hat{\bm{r}}}
\newcommand{\Chat}{\hat{C}}

% Math helper - custom function
\newcommand{\expwrtpi}[1]{\E_{\pi} [\sum_{t=0}^{\infty} \gamma^t #1(s_t, a_t)]}
\newcommand{\expangle}[1]{\langle #1  \rangle}

% helper function for return and constraints

% for value function, takes arguments:
% #1: policy 
% #2: the function of interest, R or C_i
% #3 (optional): the MDP for which this is estimated
\newcommand{\V}[3]{ %
    \ifthenelse{\isempty{#3}}%
    {V^{#1}(#2)}% #3 is empty 
    {V^{#1}_{#3}(#2)}%
}

\newcommand{\Q}[3]{
    \ifthenelse{\isempty{#3}}
    {Q^{#1}(#2)}% #3 is empty 
    {Q^{#1}_{#3}(#2)}%
}


\newcommand{\Adv}[3]{
    \ifthenelse{\isempty{#3}}
    {A^{#1}(#2)}% #3 is empty 
    {A^{#1}_{#3}(#2)}%
}

% careful diff notation
% 1: pi
% 2: R/C
% 3: M
\newcommand{\J}[3]{
    \ifthenelse{\isempty{#3}}
    {\mathcal{J}^{#1}_{#2}}% #3 is empty -> eg V^{\pi}(x ; R)
    {\mathcal{J}^{#1}_{#3,#2}}% -? eg V^{\pi}_{M}(x ; C)
    % {J_{#2}(#1)}% #3 is empty 
    % {J_{#2}(#1, #3)} %
}



\newcommand{\MRkern}{%
  \mkern-6.5mu
  \mathchoice{}{}{\mkern0.2mu}{\mkern0.5mu}%
}

% for value function, takes arguments:
% #1: policy 
% #2: the function of interest, R or C_i
% #3 (optional): the MDP for which this is estimated
% #4: variables to be given input (x) or (x,a)
\newcommand{\val}[4]{ %
    \ifthenelse{\isempty{#3}}%
    {v^{#1}_{#2}(#4)}% #3 is empty -> eg V^{\pi}(x ; R)
    {v^{#1}_{#3,#2}(#4)}% -? eg V^{\pi}_{M}(x ; C)
    % {V^{#1}_{#3}(#4 ;#2)}% -? eg V^{\pi}_{M}(x ; C)
    % {V_{#2}(#4 ; #1)}% #3 is empty -> eg V_R(x ; \pi)
    % {V_{#2}(#4 ;#1, #3)}% -? eg V_C(x ; \pi, M)
    % {#2 \MRkern V^{#1}_{#3}(#4)}% -? eg V^{\pi}_{M}(x ; C) # combines the letter V and R together
}

\newcommand{\qval}[4]{
    \ifthenelse{\isempty{#3}}
    {q^{#1}_{#2}(#4)}% #3 is empty -> eg V^{\pi}(x ; R)
    {q^{#1}_{#3,#2}(#4)}% -? eg V^{\pi}_{M}(x ; C)
    % {Q^{#1}(#4 ; #2)}% #3 is empty -> eg Q^{\pi}(x,a ; R)
    % {Q^{#1}_{#3}(#4 ;#2)}% -? eg Q^{\pi}_{M}(x,a ; C)
    % {Q_{#2}(#4 ; #1)}% #3 is empty -> eg Q_R(x,a ; \pi)
    % {Q_{#2}(#4 ;#1, #3)}% -? eg Q_C(x,a ; \pi, M)
}
\DeclareMathOperator*{\advantage}{Adv}

\newcommand{\adv}[4]{
    \ifthenelse{\isempty{#3}}
    {\advantage^{#1}_{#2}(#4)}% #3 is empty -> eg V^{\pi}(x ; R)
    {\advantage^{#1}_{#3,#2}(#4)}% -? eg V^{\pi}_{M}(x ; C)
    % {A^{#1}(#4 ; #2)}% #3 is empty -> eg Q^{\pi}(x,a ; R)
    % {A^{#1}_{#3}(#4 ;#2)}% -? eg Q^{\pi}_{M}(x,a ; C)
    % {A_{#2}(#4 ; #1)}% #3 is empty -> eg A_R(x,a ; \pi)
    % {A_{#2}(#4 ;#1, #3)}% -? eg A_C(x,a ; \pi, M)
}




\newcommand{\ci}{C}

\newcommand{\pib}{\pi_{b}}
\newcommand{\piopt}{\pi^{*}}
\newcommand{\pie}{\pi_{t}}

\newcommand{\lR}{\lambda_{R}}
\newcommand{\lC}{\lambda_{C}}
\newcommand{\ephi}{e_{\phi}}

\newcommand{\pr}{\text{Pr}}
\newcommand{\IS}{\text{IS}}
\newcommand{\CI}{\text{CI}}


% SPIBB symbols 
\newcommand{\EpsPib}{(\pi_b, e, \epsilon)}

\usepackage{hyperref}
\usepackage{url}
\usepackage{times}
\usepackage{latexsym}
\usepackage{verbatim}
\usepackage[T1]{fontenc}
\usepackage[utf8]{inputenc}
\usepackage{microtype}
\usepackage{algorithm} 
\usepackage{algpseudocode}
\usepackage{balance}
\usepackage{breqn}
\usepackage{lipsum,graphicx}
\usepackage{booktabs}
\usepackage[export]{adjustbox}
\usepackage{layouts}
\renewcommand{\UrlFont}{\ttfamily\small}
\usepackage{caption}
\usepackage{subcaption}
\usepackage{multirow}
\usepackage{makecell}
\usepackage{pifont}% http://ctan.org/pkg/pifont
\usepackage{geometry} 
\usepackage{tikz} 
\usepackage{lipsum}

%%%%%%%%%%%%% Defined Commands %%%%%%%%%%%%%%%%%%%
\newcommand{\myboxx}[4]{
    \begin{figure}[h]
        \centering
    \begin{tikzpicture}
        \node[anchor=text,text width=\columnwidth-1cm, draw, rounded corners, line width=1pt, fill=#3, inner sep=mm] (big) {\\#4};
        \node[draw, rounded corners, line width=.5pt, fill=#2, anchor=west, xshift=5mm] (small) at (big.north west) {#1};
    \end{tikzpicture}
    \vspace{-6pt}
    \end{figure}
    }
    
\newcommand{\mybox}[4]{
\vspace{4pt}
\noindent{}\fbox{\parbox{0.963\columnwidth}{\textbf{#1:} 
#4
}
}
\vspace{4pt}


}
    
\newcommand{\cmark}{\ding{51}}%
\newcommand{\xmark}{\ding{55}}%
\newcommand\BibTeX{B\textsc{ib}\TeX}
\newcommand\jdime{\textsc{JDime}}
\newcommand\jsfstmerge{\textsc{jsFSTMerge}}
\newcommand\fstmerge{\textsc{FSTMerge}}
\newcommand\thistool{MergeBERT}
\newcommand{\ic}[1]{\begin{small}\texttt{#1}\end{small}}

%%%%%%%%%%%%% Research Questions %%%%%%%%%%%%%%%%%%%
\newcommand{\rqOne}{How effective is \thistool{} in producing merge conflict resolutions?}
\newcommand{\rqTwo}{How well does \thistool{} perform across different languages?}
\newcommand{\rqThree}{How do different choices of context encoding impact performance of \thistool{}?}
\newcommand{\rqFour}{How do users perceive \thistool{} resolutions?}



\title{Program Merge Conflict Resolution via Neural Transformers}


\author{Alexey Svyatkovskiy}
\affiliation{%
  \institution{Microsoft}
  \city{Redmond}
  \state{WA}
  \country{USA}
}
%\email{alsvyatk@microsoft.com}

\author{Sarah Fakhoury}
\affiliation{%
  \institution{Washington State University}
  \city{Pullman}
  \state{WA}
  \country{USA}
}

\author{Negar Ghorbani}
\affiliation{%
  \institution{UC Irvine}
  \city{Irvine}
  \state{CA}
  \country{USA}
}

\author{Todd Mytkowicz}
\affiliation{%
  \institution{Microsoft Research}
  \city{Redmond}
  \state{WA}
  \country{USA}
}

\author{Elizabeth Dinella}
\affiliation{%
  \institution{University of Pennsylvania}
  \city{Philadelphia}
  \state{PA}
  \country{USA}
}

\author{Christian Bird}
\affiliation{%
  \institution{Microsoft Research}
  \city{Redmond}
  \state{WA}
  \country{USA}
}

\author{Jinu Jang}
\affiliation{%
  \institution{Microsoft}
  \city{Redmond}
  \state{WA}
  \country{USA}
}

\author{Neel Sundaresan}
\affiliation{%
  \institution{Microsoft}
  \city{Redmond}
  \state{WA}
  \country{USA}
}

\author{Shuvendu K. Lahiri}
\affiliation{%
  \institution{Microsoft Research}
  \city{Redmond}
  \state{WA}
  \country{USA}
}

\newcommand{\fix}{\marginpar{FIX}}
\newcommand{\new}{\marginpar{NEW}}

\renewcommand{\shortauthors}{Svyatkovskiy, Fakhoury, Ghorbani, Mytkowicz, Dinella, Bird, Jang, Sundaresan, Lahiri}


\begin{document}

\begin{abstract}
Collaborative software development is an integral part of the modern software development life cycle, essential to the success of large-scale software projects. When multiple developers make concurrent changes around the same lines of code, a merge conflict may occur. Such conflicts stall pull requests and continuous integration pipelines for hours to several days, seriously hurting developer productivity. To address this problem, we introduce \thistool{}, a novel neural program merge framework based on token-level three-way differencing and a transformer encoder model. By exploiting the restricted nature of merge conflict resolutions, we reformulate the task of generating the resolution sequence as a classification task over a set of primitive merge patterns extracted from real-world merge commit data. Our model achieves 63--68\% accuracy for merge resolution synthesis, yielding nearly a 3$\times$ performance improvement over existing semi-structured, and 2$\times$ improvement over neural program merge tools. Finally, we demonstrate that \thistool{} is sufficiently flexible to work with source code files in Java, JavaScript, TypeScript, and C\# programming languages.
To measure the practical use of \thistool{}, we conduct a user study to evaluate \thistool{} suggestions with 25 developers from large OSS projects on 122 real-world conflicts they encountered. Results suggest that in practice, \thistool{} resolutions would be accepted at a higher rate than estimated by automatic metrics for precision and accuracy. Additionally, we use participant feedback to identify future avenues for improvement of \thistool{}.

\end{abstract}


\begin{CCSXML}
<ccs2012>
   <concept>
       <concept_id>10011007.10011074.10011111.10011695</concept_id>
       <concept_desc>Software and its engineering~Software version control</concept_desc>
       <concept_significance>500</concept_significance>
       </concept>
   <concept>
       <concept_id>10011007.10011074.10011092.10011782</concept_id>
       <concept_desc>Software and its engineering~Automatic programming</concept_desc>
       <concept_significance>500</concept_significance>
       </concept>
 </ccs2012>
\end{CCSXML}

\ccsdesc[500]{Software and its engineering~Software version control}
\ccsdesc[500]{Software and its engineering~Automatic programming}

\keywords{Software evolution, program merge, ml4code}

%%% The following is specific to ESEC/FSE '22 and the paper
%%% 'Program Merge Conflict Resolution via Neural Transformers'
%%% by Alexey Svyatkovskiy, Sarah Fakhoury, Negar Ghorbani, Todd Mytkowicz, Elizabeth Dinella, Christian Bird, Jinu Jang, Neel Sundaresan, and Shuvendu K. Lahiri.
%%%
\setcopyright{acmcopyright}
\acmPrice{15.00}
\acmDOI{10.1145/3540250.3549163} 
\acmYear{2022}
\copyrightyear{2022}
\acmSubmissionID{fse22main-p1294-p}
\acmISBN{978-1-4503-9413-0/22/11}
\acmConference[ESEC/FSE '22]{Proceedings of the 30th ACM Joint European Software Engineering Conference and Symposium on the Foundations of Software Engineering}{November 14--18, 2022}{Singapore, Singapore}
\acmBooktitle{Proceedings of the 30th ACM Joint European Software Engineering Conference and Symposium on the Foundations of Software Engineering (ESEC/FSE '22), November 14--18, 2022, Singapore, Singapore}


\maketitle

\section{Introduction}  \label{sec:introduction}

\newcommand\inexpIntro[3]{#1?(#2,#3).}
\newcommand\rinexpIntro[3]{*#1?(#2,#3).}
\newcommand\outexpIntro[3]{#1!(#2,#3).}
\newcommand\outatomIntro[3]{#1!(#2,#3)}

We propose a fully automated method for proving termination of \(\pi\)-calculus processes.
Although there have been a lot of studies on termination analysis for the \(\pi\)-calculus
and related calculi~\cite{Deng06IC,Demangeon07,SangiorgiTermination,KobayashiHybrid,Yoshida04IC,DBLP:journals/jlp/DemangeonHS10,Venet98SAS}, most of them have been rather theoretical,
and there have been surprisingly little efforts in developing  fully automated termination
verification methods and tools based on them. To our knowledge,
Kobayashi's \typical{}~\cite{TyPiCal,KobayashiHybrid} is the only exception that
can prove termination of \(\pi\)-calculus processes (extended with natural numbers)
fully automatically, but its termination analysis is quite limited (see Section~\ref{sec:relatedwork}).

Our method is based on a reduction to termination analysis for sequential programs:
we translate a \(\pi\)-calculus process \(P\) to a sequential program \(S_P\), so that
if \(S_P\) is terminating, so is \(P\). The reduction allows us to use
powerful, mature methods and tools
for termination analysis of sequential programs~\cite{heizmann2016ultimate,freqterm,DBLP:conf/lics/PodelskiR04,Kuwahara2014Termination,DBLP:journals/cacm/CookPR11}.

The idea of the translation is to convert a chain of communications on replicated input
channels to a chain of recursive function calls of the target sequential program.
Let us consider the following Fibonacci process:
\begin{align*}
    & \rinexpIntro{\fib}{n}{r}
        \ifexp{n<2}{ \soutatom{r}{1} \\ &\quad}
                   { \nuexp{s_1} \nuexp{s_2} (\outatomIntro{\fib}{n-1}{s_1} \PAR \outatomIntro{\fib}{n-2}{s_2} \PAR \sinexp{s_1}{x}\sinexp{s_2}{y}\soutatom{r}{x+y}) \\}
    & \PAR \outatomIntro{\fib}{m}{r}
\end{align*}
Here, the process
$\rinexpIntro{\fib}{n}{r} \ldots$ is a function server that computes the \(n\)-th Fibonacci number
in parallel and returns the result to \(r\),
and $\outatom{\fib}{m}{r}$ sends a request for computing the \(m\)-th Fibonacci number;
those who are not familiar with the syntax of the \(\pi\)-calculus may wish to consult
Section~\ref{sec:targetlanguage} first.
To prove that the process above is terminating for any integer \(m\),
it suffices to show that there is no infinite chain of communications on $\fib$:
\[
    \fib(m,r) \to \fib(m_1,r_1) \to \fib(m_2,r_2) \to \cdots.
\]
We convert the process above to the following program:\footnote{The actual translation
  given later is a little more complex.}
\begin{verbatim}
 let rec fib(n) = if n<2 then () else (fib(n-1) [] fib(n-2)) in
 fib(m)
\end{verbatim}
Here, \texttt{[]} represents the non-deterministic choice.
Note that, although the calculation of Fibonacci numbers is not preserved,
for each chain of communications on \texttt{fib}, there is a corresponding
sequence of recursive calls:
\[
\mathtt{fib}(m) \to \mathtt{fib}(m_1) \to \mathtt{fib}(m_2) \to \cdots.
\]
Thus, the termination of the sequential program above implies the termination of
the original process.
As shown in the example above, (i) each communication on a replicated input channel
is converted to a function call, (ii) each communication on a non-replicated input
channel is just removed (or, in the actual translation, replaced by a call of
a trivial function defined by \(f(\seq{x})=(\,)\)), and (iii) parallel composition
is replaced by a non-deterministic choice.
We formalize the translation outlined above and prove its correctness.

The basic translation sketched above sometimes loses too much information.
For example, consider the following process:
\begin{align*}
    & \rinexpIntro{\pre}{n}{r} \soutatom{r}{n-1} \\
    & \PAR \rinexpIntro{f}{n}{r} \ifexp{n<0}{ \soutatom{r}{1} }
                                       { \nuexp{s} (\outatomIntro{\pre}{n}{s} \PAR \sinexp{s}{x}\outatomIntro{f}{x}{r}) } \\
    & \PAR \outatomIntro{f}{m}{r}
\end{align*}
The translation sketched above would yield:
\begin{verbatim}
  let pred(n) = n-1 in
  let rec f(n) = if n<0 then () else (pred(n) [] f(*)) in
  f(m)
\end{verbatim}
Here, \texttt{*} represents a non-deterministic integer: since we have removed
the input $\sinatom{s}{x}$, we do not have information about the value of \( x \).
As a result, the sequential program above is non-terminating, although the original
process is terminating.
To remedy this problem, we also refine the basic translation above by using a refinement
type system for the \(\pi\)-calculus. Using the refinement type system,
we can infer that the value of \(x\) in the original process is less than \(n\),
so that we can refine the definition of \texttt{f} to:
\begin{verbatim}
 let rec f(n) = ... else (pred(n) [] let x=* in assume(x<n);f(x))
\end{verbatim}
The target program is now terminating, from which
we can deduce that the original process is also terminating.
We have implemented an automated tool based on the refined translation above.

The contributions of this paper are summarized as follows.
\begin{itemize}
\item The formalization of the basic translation from the \(\pi\)-calculus
  (extended with integers) to sequential programs, and a proof of its correctness.
\item The formalization of a refined translation based on a refinement type system.
\item An implementation of the refined translation, including automated refinement type
  inference based on CHC solving, and experiments to evaluate the effectiveness of
  our method.
\end{itemize}

The rest of this paper is structured as follows.
Section~\ref{sec:targetlanguage} introduces the source and target languages
of our translation.
Section~\ref{sec:approach} 
formalizes the basic translation, and proves its correctness.
Section~\ref{sec:refinement} refines the basic translation by using a refinement type system.
Section~\ref{sec:implementation} reports an implementation and experiments.
Section~\ref{sec:relatedwork} discusses related work,
and Section~\ref{sec:conclusion} concludes the paper.


\section{Motivations for Empirical Study}
\label{sec:motivations}
The key question that we try to answer is when and why we should use standard
iteration space tiling over cache oblivious tiling.  The two approaches
perform similar partitioning of the iteration space, but the schedules given
to the partitions are different.  Theoretically, cache oblivious code seems to
have advantages over iteration space tiling.  However, many factors complicate
the actual performance, which made our initial experiments difficult to
interpret.  In this section, we describe the obstacles between the theory and
practice we have identified.

We use Single-Level Tiling (SLT) for iteration space tiling, and Cache
Oblivious Tiling (COT) for cache oblivious techniques in this
paper, which are further described in Section~\ref{sec:background}.

\paragraph{Recursion Overhead} This is a well-known overhead of
COT~\cite{yotov2007experimental}.  The recursion introduces overheads, such as
function call overhead, and increased register pressure.  Furthemore, the
functions force inter-procedural analysis/optimization, known to be more
difficult for compilers well.  Thus, the leaf tiles must be ``sufficiently
large'' to avoid excessive overhead due to the recursion.

 \paragraph{Recursive Split Constraints the Tile Sizes} In typical cache
 oblivious algorithms, the problem is recursively split into halves in each
 dimension. This is in fact a rather coarse-grained exploration of the
 hierarchical partitioning of the iteration space. For instance, if the
 current problem size is $B^3$, then the next sub-problem would be
 $(\frac{B}{2})^3$.  If the best problem size for utilizing a level of cache
 is $(B-x)^3$ where $x\ll \frac{B}{2}$ then the subproblems due to
 divide-and-conquer will not match the best.  This is another factor that
 necessitates fine tuning of leaf tile sizes even for COT, since the utilization
 rate of L1 cache has strong impact on performance.  

%\paragraph{COT Leads to Imbalanced Tiles} Current COT tools recursively split
%the problem into halves in each dimension.  If the original bounds are not
%powers of two, every power-of-two leaf will be paired with a non-power-of-two
%leaf.  Since leaf tile sizes are often carefully tuned, thismeans that half
%the leaves will be suboptimal.  Our code generator incorporates a simple
%optimization that ensures that such suboptimal leaf nodes only occur at the
%boundaries of the iteration space.

\paragraph{COT has more Conflict Misses} The divide-and-conquer execution
order may negatively affect cache interference, especially with high
dimensional data.  This happens when the memory is allocated such that the
accesses are contiguous along some direction in the iteration space (typically
along innermost canonical axis).  With lexicographic order of execution, this
contiguity is largely preserved in the tiled execution.  However,
divide-and-conquer executes neighboring tiles in all dimensions, and many of
those tiles access some distant location in memory.  In contrast to accessing
contiguous regions of memory, accessing various segments of the memory
increases the chances of conflicts.

\paragraph{Hardware Prefetching}  Modern architectures are equipped with
hardware prefetchers that can bring data to the L1 cache. When
having sufficient locality at L2 or LLC makes the program compute-bound, then
the latency to L2/LLC can be hidden by the prefetcher. For such programs, it is
unnecessary to tile for the fastest cache, and larger tiles targeting slower
caches improve performance by maximizing prefetcher
effectiveness~\cite{mehta2016turbotiling}. When the primary objective is speed,
the leaf tiles for COT should also be large, which negates the benefit of
divide-and-conquer, as the leafs are already targeting slower caches.
Prefetching have little impact on parallel executions, since prefetching is
bandwidth limited. When multiple cores try to prefetch at the same time,
the bandwidth limit is quickly reached, and the latency hiding effect is
lost. Furthermore, smaller tile sizes are better for parallel execution for
load balancing  reasons.


These factors limit the effectiveness of COT in various ways and are also
closely tied to the characteristics of the computation. Our empirical study
illustrate the impact of these factors on polyhedral computations.

% Local Variables: ***
% TeX-master: "TACO2017.tex" ***
% fill-column: 78 ***
% End: ***


\section{Background: Data-driven Merge}
\label{sec:background}
\citet{Dinella2021} introduced the {\it data-driven program merge} problem as a supervised machine learning problem. 
A program merge consists of a 4-tuple of programs $(\mathcal{A}, \mathcal{B}, \mathcal{O}, \mathcal{M})$, where 
\begin{enumerate} 
\item The base program $\mathcal{O}$ is the lowest common ancestor in the version history for programs $\mathcal{A}$ and $\mathcal{B}$, 
\item \texttt{diff3} produces an unstructured line-level conflict when applied to $(\mathcal{A}, \mathcal{B}, \mathcal{O})$, and 
\item $\mathcal{M}$ is the merged program with the developer resolution, incorporating changes made in  $\mathcal{A}$ and $\mathcal{B}$. 
\end{enumerate}
A merge may have multiple unstructured conflicts, we define a {\it merge tuple} $(A, B, O, M)$, where $A, B, O$ correspond to the conflicting regions in $(\mathcal{A}, \mathcal{B}$, and $\mathcal{O})$, respectively, and $M$ denotes the resolution region.

Given a set of merge tuples $(A_i, B_i, O_i, M_i)$, i = 0...N, the goal of a data-driven merge algorithm is to learn a function, $\texttt{merge}$, that maximizes $\sum_{i=0}^{N}\texttt{merge}(A_i, B_i, O_i) = M_i$.
Throughout the text, we will use notations $(a, b, o, m)$ to refer to the token-level merge tuples. 

\citet{Dinella2021} also provide an algorithm for extracting the exact resolution regions for each merge tuple and define a dataset that corresponds to {\it non-trivial} resolutions; resolutions where the developer does not drop the changes from one side of the merge.  
Further, they provide a sequence-to-sequence encoder-decoder based architecture, where a bi-directional gated recurrent unit (GRU) is used for encoding the merge inputs comprising of $(A, B, O)$ segments of a merge tuple, and a {\it pointer mechanism} is used to restrict the output to only choose from line segments present in the input. 
Their paper suffers from two limitations.
First, given the restriction on copying only lines from inputs, their dataset  did not consider merges where the resolution required token-level interleaving, such as the conflict in Figure~\ref{fig:word1}. 
Second, their dataset consists of merge conflicts in a single language, namely JavaScript. 
Our approach addresses both of these limitations.


\begin{figure*}
\begin{center}
    \includegraphics[width=.85\textwidth]{images/mergebert2.pdf}
\caption{An overview of the \thistool{} architecture. From left to right: given conflicting programs $\mathcal{A}$, $\mathcal{B}$ and $\mathcal{O}$ token-level differencing is performed first, next, programs are tokenized and the corresponding sequences are aligned ($a|_o$ and $o|_a$, $b|_o$, and $o|_b$). We extract edit steps for each pair of token sequences ($\Delta_{ao}$ and $\Delta_{bo}$). Four aligned token sequences are fed to the multi-input encoder neural network, while edit sequences are consumed as edit type embeddings. Finally, encoded token sequences are aggregated into a hidden state which serves as input to classification layer.}
\label{fig:mergebert}
\end{center}
\vspace{-8pt}
\end{figure*}


\section{Merge Conflict Resolution as a Classification Task}
\label{formulation}

In this work, we demonstrate how to exploit the restricted nature of merge conflict resolutions -- compared to an arbitrary program repair -- to leverage discriminative models to synthesize the merge resolution sequence.
We have empirically observed that the application of \texttt{diff3} at token granularity enjoys two useful properties over its line-level counterpart: (i) it helps localize the merge conflicts to small program segments, effectively reducing the size of conflicting regions, and (ii) most resolutions of merge conflicts produced by token \texttt{diff3} consist entirely of changes from $a$ or $b$ or $o$ or a sequential composition of $a$ followed by $b$ or vice versa. Here, and throughout the paper we will use lower case notations to refer to attributes of token-level differencing (e.g. $a$, $b$, and $o$ are conflict regions produced by \texttt{diff3} at token granularity).
On the flip side, a token-level merge can introduce many small conflicts. 
To balance the trade-off, we start with the line-level conflicts as produced by the standard \texttt{diff3} and perform a token-level merge of only the segments present in the line-level conflict.
There are several potential outcomes for such a two-level merge at the line-level: 
\begin{itemize}
\item {\it A conflict-free token-level merge}: For example, the edit from $A$ about \texttt{let} is merged since $B$ does not edit that slot as shown in Fig.~\ref{fig:word1}(b).  
\item {\it A single localized token-level merge conflict}: For example, the edit from both $A$ and $B$ for the arguments of \texttt{max} yields a single conflict as shown in Fig.~\ref{fig:word1}(b).
\item {\it Multiple token-level conflicts}: Such a case (not illustrated above) can result in several token-level conflicts. %which forms 59\%, 36\%, and 5\% of our dataset, respectively.
\end{itemize}

Token-level diff3 applied to a 4-tuple of programs $(\mathcal{A}, \mathcal{B}, \mathcal{O}, \mathcal{M})$, would usually result in a set of localized merge tuples $\langle a_j, b_j, o_j, m_j\rangle$. 
We empirically observe that 74\% of such resolutions $m_j$ are comprised of ($i$) exactly the tokens in $a_j$ or ($ii$) exactly the tokens in $b_j$.  Another 0.4\% of the resolutions are ($iii$) just the tokens in $o_j$. In addition, 23\% of the resolutions are the result of concatenating ($iv$) $a_j$ and $b_j$ or ($v$) $b_j$ and $a_j$.  Finally, 1.8\% comprise another four variants, obtained by taking $i$, $ii$, $iv$ and $v$ above and removing the tokens that also occur in the base, $o_j$. In total, this provides \textit{nine} primitive merge resolution patterns (see online Appendix~\cite{FSE22Appendix} for more details about the primitive merge patterns). 

We, therefore, treat the problem of constructing merge conflict resolutions $m_j$ as a classification task to predict between these possibilities. It is important to note that although we are predicting simple resolution strategies at the token-level, they may translate to complex resolutions at the line-level. In addition, not all conflicts are resolved by breaking that conflict into tokens and applying these patterns---some resolutions such as those introducing new tokens or reordering tokens are not expressible as a choice at the token-level.  

% Removing for now...
%One of the practical advantages of formulating merge conflict resolution as a classification task is a significant reduction in total FLOPS \Shuvendu{describe} required to decode a resolution region, as compared to generative models, making this approach an appealing candidate for deployment in IDEs (see section~\ref{sec:inference}). \sarah{appendix}


% Overview of basic mergeBERT
\section{\thistool{}: Neural Program Merge Framework}
\label{sec:main_model}

\thistool{} is a textual program merge model based on the bidirectional transformer encoder (BERT) model~\cite{bert}.
We refer the reader to CodeBERT~\cite{feng-etal-2020-codebert} for a discussion on applying transformers to code. A transformer, like
a recurrent neural network, maps a sequence of text into a high
dimensional representation, which can later be decoded to solve
downstream tasks. While not originally designed for code, transformers have found many applications in software engineering~\cite{clement2020pymt5,kanade2020learning,svyatkovskiy2020intellicode}

\thistool{} approaches merge conflict resolution as a sequence classification task given conflicting regions extracted with token-level differencing and surrounding code as context. 
%By focusing on token-level merge conflicts, we are able to resolve real-world merges. 
The key technical innovation in \thistool{} lies in how it breaks program text into an input representation amenable to learning with a transformer encoder and how it aggregates various input encodings for classification. 

In the standard sequence learning setting there is a single input and single output sequence. In the merge conflict resolution task, there are multiple conflicting input programs and one resolution. To facilitate learning in this setting, we construct \thistool{} as a multi-input encoder neural network, which first encodes token sequences of conflicting programs, then aggregates them into a single hidden summarization state. 

An overview of the \thistool{} model architecture is shown in Fig.~\ref{fig:mergebert}. Given conflicting programs $\mathcal{A}$, $\mathcal{B}$ and $\mathcal{O}$ we first perform tokenization and then repeat the three-way differencing at token granularity. If a conflict still exists in this token-level three-way differencing, we collect the token sequences corresponding to conflicting regions $a$, $b$, and $o$, and compute pair-wise alignments of $a$ and $b$ with respect to the base $o$. Finally, for each pair of aligned token sequences we extract an ``edit sequence'' that represents how to turn the second sequence into the first. The resulting aligned token sequences are fed to the multi-input encoder neural network, while the corresponding edit sequences are consumed as type embeddings. Finally, the encoded token sequences are summarized into a hidden state which serves as input to the classification layer. 

Given a 4-tuple of programs $(\mathcal{A}, \mathcal{B}, \mathcal{O}, \mathcal{M})$ which contains token-level merge tuples $(a_{j}, b_{j}, o_{j}, m_{j})$, j=0...N, \thistool{} models the following conditional probability distribution:
\begin{equation}
    p(m_{j} | a_{j}, b_{j}, o_{j}),
\end{equation}
and consequently, for entire programs:
\begin{equation}
    p(\mathcal{M} | \mathcal{A}, \mathcal{B}, \mathcal{O}) = \prod_{j=1}^{N} p(m_{j} | a_{j}, b_{j}, o_{j})
\end{equation}
Independence of token-level conflicts is a simplifying assumption. However, we observe that in our data set only 5\% of merge conflicts result in more than 1 token-level conflict per line-level conflict. 




\subsection{Context Encoding}

For a merge tuple $(a, b, o, m)$ \thistool{} calculates two pair-wise alignments between the sequences of tokens of conflicting regions $a$ (respectively $b$) with respect to that of the original program $o$: $a|_o$, $o|_a$, $b|_o$, and $o|_b$. For each pair of aligned token sequences we compute an edit sequence. These edit sequences -- $\Delta_{ao}$ and $\Delta_{bo}$ -- are comprised of the following editing actions (kinds of edits): $\textbf{=}$ represents equivalent tokens, $\textbf{+}$ represents insertions, $\textbf{-}$ represents deletions,
$\boldsymbol{\leftrightarrow}$ represents a replacement, and
$\boldsymbol{\emptyset}$ is used as a padding token. Overall, this produces four token sequences and two edit sequences: ($a|_{o}$,
$o|_{a}$, and $\Delta_{ao}$) and ($b|_{o}$, $o|_{b}$, and $\Delta_{bo}$). Fig.~\ref{fig:embedding} provides an example of an edit sequence. Each token sequence covers the corresponding conflicting region and, potentially, surrounding code tokens. We make use of Byte-Pair Encoding (BPE) unsupervised tokenization to avoid a blowup in the vocabulary size given the sparse nature of code identifiers~\cite{10.1145/3377811.3380342}.
To help the model learn to recognize editing steps we introduce an edit type embedding. We combine it with the standard token and position embeddings utilized in BERT model architecture via addition. 
%: $\mathcal{S} = \mathcal{S_{T}} + \mathcal{S_{P}} +\mathcal{S_{E}}$. 
%\alexey{Chris: I agree with the comment. We can comment out the formula since it is not used anywhere else. Or we can add one more layer of details: to explain how embeddings work in general. Since we use standard implementation here, I think it is not necessary.}
%\chris{@Alexey, this last paragraph probably needs a bit more explanation.  $\mathcal{S}$ is never defined and never appears anywhere else in the paper.  We should explain where it fits in fig 2 and how it connects to the rest of the model.  How do we arrive at the position and edit type embeddings?}
\begin{figure}
\begin{center}
    \includegraphics[width=.48\textwidth]{images/Embedding.pdf}
\caption{An example edit sequence extracted between a pair of token sequences.  Top row is $o|_b$, bottom is $b|_o$, and middle is $\Delta_{bo}$. Padding symbols \texttt{[PAD]} are introduced for alignment. In this case, the target token sequence is obtained by swapping a token and inserting two tokens.}
\label{fig:embedding}
\end{center}
\vspace{-12pt}
\end{figure}


\subsection{Merge Tuple Aggregation}
%\label{sec:mergesumm}

We utilize transformer encoder model $\mathcal{E}$ to independently encode each of the four token sequences of token-level conflicting regions $a|_{o}$, $o|_{a}$, $b|_{o}$, and $o|_{b}$, passing corresponding edit sequences $\Delta_{ao}$ and $\Delta_{bo}$ as type embeddings. Finally, \thistool{} aggregates the resulting encodings into a single hidden summarization state $h$:
\begin{dmath}
h = \sum_{x \in (a|_{o}, o|_{a}, b|_{o}, o|_{b})} \theta_{x} \cdot \mathcal{E} (x, \Delta_x)
\end{dmath}
where $\mathcal{E} (x, \Delta_x)$ are the encoded tensors for each of the sequences $x \in (a|_{o}, o|_{a}, b|_{o}, o|_{b})$, and $\theta_{x}$ are learnable weights. After aggregation a linear classification layer with \texttt{softmax} is applied:
\begin{equation}
      p(m_{j} | a_{j}, b_{j}, o_{j}) = \mathrm{softmax}(W\cdot h + b)
\end{equation}

The resulting line-level resolution region is obtained by concatenating the prefix, predicted token-level resolution $m_{j}$, and the suffix. Finally, in the case of a one-to-many correspondence between the original line-level and the token-level conflicts (see Appendix for more details and a pseudocode), \thistool{} uses a standard beam-search to decode the most promising predictions. 

%\subsection{Model Training}

%We exploit the traditional two-step pretraining and finetuning training procedure. First, we pretrain a transformer encoder $\mathcal{E}$ on a multilingual source code corpus with unsupervised masked language modeling (MLM) pretraining objective. 
%We transfer the weights of the pretrained transformer encoder~\cite{feng-etal-2020-codebert} into the \thistool{} multi-input neural network, and attach a randomly initialized linear layer with softmax. We then finetune the resulting neural network in a supervised setting for the sequence classification task. 
%See section~\ref{sec:implement} in the Appendix for more details about the implementation. 


\subsection{Implementation Details}
\label{sec:implement}

We utilize a pretrained CodeBERT\footnote{\url{https://huggingface.co/huggingface/CodeBERTa-small-v1}} model with 6 encoder layers, 12 attention heads, and a hidden state size of 768. The vocabulary is constructed using byte-pair encoding \citep{sennrich2015neural} and the vocabulary size is 50000. We transfer the weights of the pretrained transformer encoder into the \thistool{} multi-input neural network, and attach a randomly initialized linear layer with softmax. We then finetune the resulting neural network in a supervised setting for the sequence classification task. Input sequences for finetuning training cover conflicting regions and surrounding code (i.e., fragments of prefix and suffix of a conflicting region) up to a maximum length of 512 BPE tokens. The backbone of our implementation is HuggingFace's~\footnote{\url{https://github.com/huggingface/transformers}} \texttt{RobertaModel} and \\
\texttt{RobertaForSequenceClassification} classes in PyTorch, which are modified to turn the model into a multi-input architecture shown in Fig.~\ref{fig:mergebert}. 
We finetune \thistool{} with Adam stochastic optimizer with weight decay fix using a
learning rate of 5e-5, 512 batch size and 8 backward passes per \texttt{allreduce}. 
The finetuning training was performed on 4 NVIDIA Tesla V100 GPUs with 16GB memory for 6 hours. 

In the inference phase, the model prediction for each line-level conflict consists of one or more token-level predictions. Given the token-level predictions and the contents of the merged file, \thistool{} generates the code corresponding to the resolution region. The contents of the merged file include the conflict in question and its surrounding regions. Afterward, \thistool{} checks the syntax of the generated code with a tree-sitter\footnote{\url{https://tree-sitter.github.io/tree-sitter}} parser and outputs it as the candidate merge conflict resolution only if it is syntactically correct.


\section{Research Questions}

We pose the following research questions to evaluate the effectiveness of utility of \thistool{}.


\noindent \textbf{RQ\scriptsize{1}: }\textbf{\rqOne}
We evaluate \thistool{}'s performance of producting resolutions in terms of precision and accuracy of matching the actual user resolution extracted from real-world merge resolutions. We also provide a comparison \thistool{} to baseline approaches (at both the line and token level) and state of the art merge resolution approaches.

\noindent \textbf{RQ\scriptsize{2}: }\textbf{\rqTwo}
One of our primary goals is to be able to work on multiple languages with minimal effort.  
The core approach of \thistool{} is fundamentally language agnostic (though a parser and tokenizer is required for each additional language).  
We evaluate performance of \thistool{} across four languages and also compare the results of using four language-specific models (each trained on just one language) to using one multi-lingual model trained on the data from all four languages.

\noindent \textbf{RQ\scriptsize{3}: }\textbf{\rqThree}
We conduct an ablation study of the edit type embedding to understand and evaluate the impact of our novel edit-aware encoding on model performance.

\noindent \textbf{RQ\scriptsize{4}: }\textbf{\rqFour}
We conduct a user study involving a survey of real-world conflicts recently encountered by developers from large OSS projects. To understand how developers would use \thistool{} in practice, we provide them with an interface to explore \thistool{}'s conflict resolution suggestions in relation to their original conflicting code ask them evaluate suggestions and explain why they do or do not correctly resolve the merge conflict. 


\section{The Semantic Urban Mesh Dataset}\label{sec:framework}
\subsection{Dataset Specification}

We have used Helsinki's 3D texture meshes as input and annotated them as a benchmark dataset of semantic urban meshes. 
The Helsinki's raw dataset covers about 12 $ km^2 $, and it was generated in 2017 from oblique aerial images that have about a 7.5 $cm$  ground sampling distance (GSD) using an off-the-shelf commercial software namely ContextCapture~\citep{contextcap}.
The source images have three colour channels (i.e., red, green, and blue) and are collected from an airplane with five cameras that have $80\%$ length coverage and $60\%$ side coverage.
To recover the 3D water bodies that do not fulfil the Lambertian hypothesis, 2D vector maps and ortho-photos are used when performing the surface reconstruction.
Furthermore, processing like aerial triangulation, dense image matching, and mesh surface reconstruction were all performed with ContextCapture.
It should be noticed that the entire region of Helsinki is split into tiles, and each of them covers about 250 $ m^2 $~\citep{kalasatamaReport}.
As shown in Figure \ref{fig:overview},  we have selected the central region of Helsinki as the study area, which includes 64 tiles and covers about 4 $km^2$ map area (8 $km^2$ surface area) in total.   

\subsection{Object Classes}
We define the semantic categories for urban meshes by the most common objects in the urban environment with unambiguous geometry and texture appearance.
Moreover, each triangle face is assigned to a label of one of the six semantic classes. 
Ambiguous regions (which account for about 2.6\% of the total mesh surface area), such as shadowed regions or distorted surfaces, are labelled as unclassified (see Figure \ref{fig:ambigious}).
The object classes we consider in the benchmark dataset are: 
\begin{itemize}
	\item \textbf{terrain}: roads, bridges, grass fields, and impervious surfaces;
	\item \textbf{building}: houses,high-rises, monuments, and security booths;
	\item \textbf{high vegetation}: trees, shrubs, and bushes;
	\item \textbf{water}: rivers, sea, and pools;
	\item \textbf{vehicle}: cars, buses, and lorries;  
	\item \textbf{boat}: boats, ships, freighters, and sailboats;
	\item \textbf{unclassified}: incomplete objects like buses and trains, distorted surfaces like tables, tents and facades, construction sites, underground walls.
\end{itemize}

\begin{figure}[!tb]
	\includegraphics[height=0.48\textwidth]{figures/overview_grids/yaxis.png}
	\begin{subfigure}[t]{0.48\textwidth}
		\includegraphics[width=\linewidth]{figures/overview_grids/texture_global_birdsview00.png}
		\includegraphics[width=\linewidth]{figures/overview_grids/xaxis.png}
		\label{fig:textop}
	\end{subfigure}
	\hspace*{\fill}
	\begin{subfigure}[t]{0.48\textwidth}		
		\includegraphics[width=\linewidth]{figures/overview_grids/semantic_global_birdsview00.png}
		\vspace*{-0.78cm}
		\begin{center}
		\includegraphics[width=0.8\linewidth]{figures/semantic_results/semantic_legend2.png}
		\end{center}
		\label{fig:semtop}
	\end{subfigure}
	\vspace*{-0.7cm}
	\caption{Overview of the semantic urban mesh benchmark.
	Left: the texture meshes covering about 4 $km^2$ map area. Right: the ground truth meshes.
	More views of the same scene (with different visualization styles) are shown in Figures \ref{fig:texside} and \ref{fig:semside}.}
	\label{fig:overview}
\end{figure}

\begin{figure}[!tb]
	\centering
	\begin{subfigure}[t]{0.48\textwidth}
		\includegraphics[width=\linewidth]{figures/ambigious/shadow_tex_zoom.png}
		\caption{}
	\end{subfigure}
	\hspace*{\fill}
	\begin{subfigure}[t]{0.48\textwidth}
		\includegraphics[width=\linewidth]{figures/ambigious/shadow_fc_zoom.png}
		\caption{}
	\end{subfigure}
	\begin{subfigure}[t]{0.48\textwidth}
		\includegraphics[width=\linewidth]{figures/ambigious/distort_tex_zoom.png}
		\caption{}
	\end{subfigure}
	\hspace*{\fill}
	\begin{subfigure}[t]{0.48\textwidth}
		\includegraphics[width=\linewidth]{figures/ambigious/distort_fc_zoom.png}
		\caption{}
	\end{subfigure}
	\caption{Ambiguous regions are labelled as unclassified (in black). 
		(a) Shadow region with texture.
		(b) Shadow region with semantic colour.
		(c) Distorted region with texture.
		(d) Distorted region with semantic colour.} 
	\label{fig:ambigious}
\end{figure}


\subsection{Semi-automatic Mesh Annotation}  \label{sec:mesh_annota}
Rather than manually labelling each triangle face of the raw meshes, we design a semi-automatic mesh labelling framework to accelerate the labelling process. Figure~\ref{fig:pipeline} shows the overall pipeline of our labelling workflow.

Given the fact that urban environments consist of a large number of planar regions in the data, we opt to label the data at the segment level instead of individual triangle faces. 
Specifically, we over-segment the input meshes into a set of planar segments. 
These segments can enrich local contextual information for feature extraction and serve as the basic annotation unit to improve annotation efficiency.

\begin{figure}[!tb]
	\centering
	\includegraphics[width=\textwidth]{figures/pipeline/pipeline_L1.png}
	\caption{The pipeline of the labelling workflow.}
	\label{fig:pipeline}
\end{figure}

Instead of randomly choosing a mesh tile as input for annotation and refinement, which is insufficient for manual annotation progress, we favour picking a mesh tile that is more difficult to classify.
Similar to active learning, we first compute the feature diversity (see Equation \ref{eq:fea_div}) to optimally select a mesh tile containing a variety of classes and objects at different scales and complexity.
The feature diversity $F_{m}$ of tile $m$ is computed as
\begin{equation}\label{eq:fea_div}
	F_{m}=\frac{\sum_{i=1}^{N_{f}}\left ( f_i - \bar{f} \right )^{2}}{N_{f}}
\end{equation}
where $f_i$ represents each handcrafted feature which describe in Section \ref{sec:initial_seg}, and $\bar{f}$ is mean value of a $N_{f}$ dimensional feature vector.
To acquire the first ground truth data, we manually annotate the mesh (with segments) that is selected with the highest feature diversity.
Then, we add the first labelled mesh into the training dataset for the supervised classification.
Specifically, we use the segment-based features as input for the classifier, and the output is a pre-labelled mesh dataset.
Next, we use the mesh annotation tool to manually refine the pre-labelled mesh according to the feature diversity.
Finally, the new refined mesh will be added to the training dataset to improve the automatic classification accuracy incrementally.


\subsubsection{Initial Segmentation}\label{sec:initial_seg}

To avoid redundant computations of numerous triangles, we first apply mesh over-segmentation (i.e., linear least-squares fitting of planes) based on region growing on the input data to group triangle faces into homogeneous regions~\citep{lafarge2012creating}.
Such grouped regions are beneficial for computing local contextual features.
We then extract both geometric and radiometric features from those mesh segments as follows: 
\begin{itemize}
	\item[$\bullet$] \textit{Eigen-based features} are computed from the covariance matrix of the triangle vertices with respect to the average centre within each segment, which is beneficial for identifying urban objects with various surface distributions.
	The linearity $= (\lambda_{1} - \lambda_{2}) / \lambda_{1}$, sphericity $= \lambda_{3}/ \lambda_{1}$ and change of curvature $= \lambda_{3} / (\lambda_{1} + \lambda_{2} + \lambda_{3})$ are computed based on the three eigenvalues $\lambda_{1} \geq \lambda_{2} \geq \lambda_{3}\geq 0$.
	The local eigenvectors $\mathbf{n}_{i} $ and the unit normal vector $\mathbf{n}_{z} $ along Z-axis are used to compute the verticality $=1-\left | \mathbf{n}_{i}\cdot \mathbf{n}_{z} \right | $~\citep{hackel2016fast}.
	Note that many eigen-based features have been studied in literature~\citep{hackel2016fast,west2004context,weinmann2013feature}, and some of them were designed for and tested on LiDAR point clouds. 
	\textcolor{ao}{
	These eigen-based features are mostly computed per point based on its spherical neighbourhood, which often contains noise and does not form a surface. 
	Our chosen eigen-based features are defined on a segment representing the surface of a mesh, and thus they can capture non-local geometric properties of an object.
	}
	Additionally, in this work, we have tested all eigen-based features from the literature~\citep{hackel2016fast}, and we only present the ones that are effective for texture meshes.
	\item[$\bullet$] \textit{Elevation} is divided into absolute elevation $z_{a}$, relative elevation $z_{r}$ and multiscale elevations $z_{m}$.
	Where $z_{a}$ is the average elevation of the segment;
	the relative elevation is computed as $z_{r} = z_{a}-z_{r_{min}}$;
	the multiscale elevation~\citep{Verdie2015,Rouhani2017} $z_{m} = \sqrt{\frac{z_{a} - z_{min}}{z_{max} - z_{min}}}$.
	And $z_{r_{min}}$ denotes the lowest elevation of the local largest ground segment computed within a cylindrical neighbourhood with 30 meters radius around the segment centre.
	$z_{min}$ and $z_{max}$ represent the local minimum and maximum elevation values of a cylindrical neighbourhood within the scale of 10 meters, 20 meters, and 40 meters.
	Such large cylindrical neighbourhoods allow to find the local ground considering the resilience to hilly environments, \textcolor{ao}{and the square root ensures that small relative height values (i.e., values smaller than 1 $ m $) get a larger elevation attribute to enlarge elevation differences between small objects and the local ground (e.g., cars against the ground, boats against the water surfaces).}
	More importantly, due to the influence of terrain fluctuations and various scales of urban objects, the elevation of these three categories can complement each other.
	\item[$\bullet$] \textit{Segment area} is computed as $area(S_k) = \sum_{i = 1}^{N} area(f_i) $, where $f_i$ denotes a triangle of the segment $S_k$, and $N$ denotes the total number of triangles in $S_k$.
	\item[$\bullet$] \textit{Triangle density} is defined as $density(S_k) = \frac{N}{area(S_k)} $,  which reveals the object complexity, especially for adaptive urban meshes.
	\item[$\bullet$] \textit{Interior radius of 3D medial axis transform (InMAT)}~\citep{ma20123d,peters2016robust} of a segment $S_k$ is formulated as $r_k = \frac{\sum_{i=1}^{M} r_i}{M}$, where $M$ denotes the total number of triangle vertices of $S_k$, and $r_i$ denotes the interior radius of the shrinking ball that touches the vertex $v_i$ within the segment $S_k$. 
	It is designed to distinguish objects with different scales. 
	\item[$\bullet$] \textit{HSV colour-based features} are derived from the RGB channel of the entire texture map.
	We use the HSV colour space since it can better differentiate different objects than RGB.
	We compute the average colour, the variance of the colour distribution of all pixels within each segment, and we further discretize it into a histogram that consists of 15 bins of the hue channel, five bins of the saturation channel, and five bins of the value channel.
	\item[$\bullet$] \textit{Greenness} $a_{g}$ is used to classify objects that are similar to green vegetation.
	Specifically, it is computed according to the averaged RGB colour of each segment via $a_{g}=G-0.39\cdot R-0.61\cdot B$~\citep{mckinnon2017comparing}. 
\end{itemize}
	All the above features are concatenated into a 44-dimensional feature vector used by our random forest (RF) classifier in the initial segmentation. 

\subsubsection{Annotation Tool for Refinement}

Because of the under-segmentation errors and the imperfect results of the semantic mesh segmentation process, we design a mesh annotation tool (see Figure \ref{fig:annotator}) to manually correct the labelling errors.
Our mesh annotation tool is developed based on the labelling tool of CGAL~\citep{cgal:eb-20b}.

\begin{figure}[!tb]
	\centering
	\includegraphics[width=\textwidth]{figures/annotator/annotator.png}
	\caption{The interface of our annotation tool for 3D texture meshes. }
	\label{fig:annotator}
\end{figure}

As shown in Table \ref{tab:annotation_operation}, it consists of three operation categories: view, selection, and annotation.
The	view operations provide essential functions for the user to manipulate the scene camera, such as translate, rotate, zoom, or set the new pivot for the scene.
In addition, to use textures as a reference for labelling, we map texture and face colour with a certain degree of transparency, and we visualize the segment border to differentiate each segment. 

\begin{table}[!tb]
	\centering
	\noindent\adjustbox{max width=0.8\textwidth}
	{
		\begin{threeparttable}
			\centering
			\begin{tabular}{ccc}
				\toprule
				Categories & Operations & Objects \\
				\midrule
				\multirow{4}[2]{*}{View} & Translate & Camera \\
				& Rotate & Camera \\
				& Zoom in / out & Camera \\
				& Set pivot & Camera \\
				\midrule
				\multirow{6}[2]{*}{Selection} & Multi-selection / Lasso & Triangles / Segments \\
				& Expand / Reduce & Triangles / Segments \\
				& Semantic selection & Segments \\
				& Split region & Segments \\
				& Planar region extraction & Triangles \\
				& Split mesh & Triangles \\
				\midrule
				\multirow{3}[2]{*}{Annotation} & Probability slider & Segments \\
				& Segment area slider & Segments \\
				& Progress bar & Triangles \\
				& Switch semantic view & Triangles \\ 
				& Labelling & Triangles / Segments \\
				\bottomrule
			\end{tabular}%
		\end{threeparttable}
	}
	\caption{Basic operations in our annotation tool.} 
	\label{tab:annotation_operation}%
\end{table}%


The	selection operations allow the user to select or deselect either triangle faces (see Figure \ref{fig:tri_sel}) or segments (see Figure \ref{fig:seg_sel}) freely via a brush or a lasso.
Specifically, the face selection operation is used to fix the under-segmentation errors and generate new segments, and the segment selection operation is to fix incorrect segment labels.

\begin{figure}[!tb]
	\centering
	\begin{subfigure}[t]{0.32\textwidth}
		\includegraphics[width=\linewidth]{figures/pipeline/tri_select_a.png}
		\caption{}
	\end{subfigure}
	\hspace*{\fill}
	\begin{subfigure}[t]{0.32\textwidth}
		\includegraphics[width=\linewidth]{figures/pipeline/tri_select_b.png}
		\caption{}
	\end{subfigure}
	\hspace*{\fill}
	\begin{subfigure}[t]{0.32\textwidth}
		\includegraphics[width=\linewidth]{figures/pipeline/tri_select_c.png}
		\caption{}
	\end{subfigure}
	\caption{An example of labelling by selecting triangles using the lasso tool (blue edges: segment boundaries). 
		(a) Before selection.
		(b) Lasso selection result (in red).
		(c) The correct label has been assigned to the selected region. 
		In this example, the label of the selected region has been changed from `ground' to `vehicle'.
	} 
	\label{fig:tri_sel}
\end{figure}


\begin{figure}[!tb]
	\centering
	\begin{subfigure}[t]{0.32\textwidth}
		\includegraphics[width=\linewidth]{figures/pipeline/seg_select_a.png}
		\caption{}
	\end{subfigure}
	\hspace*{\fill}
	\begin{subfigure}[t]{0.32\textwidth}
		\includegraphics[width=\linewidth]{figures/pipeline/seg_select_b.png}
		\caption{}
	\end{subfigure}
	\hspace*{\fill}
	\begin{subfigure}[t]{0.32\textwidth}
		\includegraphics[width=\linewidth]{figures/pipeline/seg_select_c.png}
		\caption{}
	\end{subfigure}
	\caption{An example of segment labelling. 
		(a) Part of a wall of the building was previously labelled as `high vegetation' (in green).
		(b) Segment selection result (in red).
		(c) The label of the selected segment has been corrected with the new label `building'.
	}
	\label{fig:seg_sel}
\end{figure}

We also allow the user to edit the selection of each individual segment with splitting functions (see Figure \ref{fig:pnp_func}) and automatic extraction of the most planar region (see Figure \ref{fig:seg_func}). 
As for splitting, we first detect the potential planar and non-planar segments marked by user strokes, and then the non-planar one is split according to the vertex-to-plane distance.
It allows generating candidate non-planar regions (with respect to the detected planar segment) for the user to edit, and
it is useful to split a segment that covers large non-planar regions or contains more than one dominant planar area.
To extract the most planar region, we apply the region growing algorithm~\citep{lafarge2012creating} within the selected segment to automatically generate the candidate triangle faces with user-defined thresholds (i.e., the maximum distance to the plane, the maximum accepted angle, and the minimum region size).
Such an operation allows the user to filter out some small bumpy regions of the selected segment.

\begin{figure}[!tb]
	\centering
	\begin{subfigure}[t]{0.48\textwidth}
		\includegraphics[width=\linewidth]{figures/annotator/pnp_pipeline1.png}
		\caption{}
	\end{subfigure}
	\hspace*{\fill}
	\begin{subfigure}[t]{0.48\textwidth}
		\includegraphics[width=\linewidth]{figures/annotator/pnp_pipeline2.png}
		\caption{}
	\end{subfigure}
	\caption{An example splitting planar and non-planar regions. 
		(a) The user draws a stroke (in red) across the border of the non-planar segment and the planar segment. 
		(b) The detected non-planar segment has been split into two parts (i.e., a non-planar region shown in red and a planar segment shown in green).
	} 
	\label{fig:pnp_func}
\end{figure}

\begin{figure}[!tb]
	\centering
	\begin{subfigure}[t]{0.48\textwidth}
		\includegraphics[width=\linewidth]{figures/annotator/planar_split_pipeline1.png}
		\caption{}
	\end{subfigure}
	\hspace*{\fill}
	\begin{subfigure}[t]{0.48\textwidth}
		\includegraphics[width=\linewidth]{figures/annotator/planar_split_pipeline3.png}
		\caption{}
	\end{subfigure}
	\caption{Editing an individual segment. 
		(a) A segment is selected (highlighted in green) for splitting. 
		(b) Automatic extraction of the most planar region (shown in red) within the selected segment according to user-defined thresholds.} 
	\label{fig:seg_func}
\end{figure}

Besides, probability and area-based sliders and a progress bar are provided in the annotation panel to improve annotation efficiency and experience, respectively. 
Specifically, the probability slider is introduced for the user to visually inspect the segments that are most likely misclassified.
Moreover, the user can further use it to inspect a specific class by switching the view to highlight a specific semantic class.
The segment area slider is used to identify isolated tiny segments, which commonly appear as errors.
The progress bar is used to indicate the estimated labelling progress during the annotation.
After performing the selection, the user can easily assign the corresponding label to the selected area.


\section{Evaluation}

\subsection{Evaluation Metrics}

We evaluate \thistool{}'s performance of resolution synthesis in terms of precision and accuracy of string match (modulo whitespaces or indentation) to the user resolution extracted from real-world historical merge resolutions. This approach is rather restrictive as a suggested resolution might differ from the actual user resolution by, for instance, only the order of statements, being semantically equivalent otherwise. As such, this evaluation approach gives a lower bound of performance.

We evaluate \thistool{} and compare it to baselines and existing approaches using two metrics, precision at top-k and accuracy at top-k.  
Since \thistool{} is a neural approach, it may provide more than one suggestion, which we rank according to the associated prediction probabilities.
In addition, because we filter out resolution suggestions that are not syntactically valid, it may provide no suggestions in rare cases.  
Accuracy at top-1 indicates the percentage of total conflicts for which \thistool{} produces the correct resolution as its top suggestion. Precision at top-1 indicates how often (as a percentage) the top suggestion is correct when the \thistool{} provides any suggestions at all.  As a concrete example, if a tool produces a resolution suggestion for 50 out of 100 conflicts and 40 of the suggestions matched the actual historical user resolution, then the precision would be 80\% (40/50), but the accuracy would be 40\% (40/100).  Precision at top-k indicates how often the correct resolution is found in the top-k suggestions and Accuracy at top-k is analogous. When ``top-k'' is omitted from the metric name (e.g. just "Precision") then k is 1.

%In addition to the precision and accuracy, we also report the fraction of syntactically correct (or parseable) source code suggestions to filter out merge resolutions with syntax errors. 

%\chris{we only report syntactic correctness in table 3 and no others.  I suggest we remove it, as reviewers will ask why it is so rarely present and it's not critical to the perf. evaluation.}

\subsection{Baseline Models}
\label{sec:baselines}

\subsubsection{Language Model Baseline}

Neural language models (LMs) have shown great performance in natural language generation~\citep{gpt2, sellam-etal-2020-bleurt}, and have been successfully applied to the domain of source code~\citep{10.5555/2337223.2337322, gptc, feng-etal-2020-codebert}. We consider the generative pretrained transformer language model for code (GPT-C) and appeal to the naturalness of software~\citep{naturalness} to construct our baseline approach for the merge resolution synthesis task. We establish the following baseline:
given an unstructured line-level conflict produced by \texttt{diff3}, we take the common source code prefix acting as user intent for program merge. We attempt to generate an entire resolution region token-by-token using beam search. As an ablation experiment, we repeat this for the conflicts produced with the token-level differencing algorithm (Fig.~\ref{fig:word1} shows details about prefix and conflicting regions).

\subsubsection{DeepMerge: Neural Model for Interleavings}

Next, we consider \textsc{DeepMerge}~\citep{Dinella2021}: a sequence-to-sequence model based on the bidirectional GRU summarized in section~\ref{sec:background}. It learns to generate a resolution region by choosing from line segments present in the input (line interleavings) with a pointer mechanism. We retrain the \textsc{DeepMerge} model on our TypeScript dataset.

\subsubsection{JDIME}
Looking for a stronger baseline, we consider \textsc{JDime}, a Java-specific merge tool that automatically tunes the merging process by switching between structured and unstructured merge algorithms \citep{apel2012structured}. Structured merge is abstract syntax tree (AST) aware and leverages syntactic information to improve matching precision of conflicting nodes.  We use the publicly available implementation~\citep{jdime}, and run JDime in semi-structured mode. 

\subsubsection{jsFSTMerge}
\citet{tavares2019javascript} implemented \jsfstmerge{} by adapting an off-the-shelf grammar for JavaScript to address shortcomings of \fstmerge{}~\cite{apel2012fstmerge} and modify its algorithm.
\jsfstmerge{} allows for certain types of nodes to maintain their relative order (\emph{e.g.}, statements) while others may be order independent (\emph{e.g.}, function declarations) even when sharing the same parent node.
For cases where \jsfstmerge{} produces a resolution not matching the user resolution, we manually inspect the output for semantic equivalence (e.g., reordered import statements).

\subsection{Results}
\label{sec:eval}


\noindent \textbf{RQ\scriptsize{1}: }\textbf{\rqOne}

To evaluate \thistool{} We first compare it to other neural approaches and to \texttt{diff3}. 
To be comprehensive, we evaluate at both the token level and the line level.  
We then compare \thistool{} to existing state of the art structured and semi-structured merge language-specific merge approaches.

\begin{table}[htb]
\small
\caption{Evaluation results for \thistool{} and various neural baselines calculated for merge conflicts in TypeScript programming language test set. The table shows top-1 precision and accuracy metrics.}
\centering
\begin{tabular}{lllllllllll} \toprule
\textbf{Approach}  & \textbf{Granularity} & {\textbf{Precision}} & {\textbf{Accuracy}} \\ \midrule
LM   & Line  &3.6 & 3.1 \\      % line 
DeepMerge & Line  & 55.0 & 35.1  \\ % Tyepscript aligned linearized
\midrule
\texttt{diff3} & Token & 82.4 & 36.1  \\
\midrule
LM & Token  & 49.7  & 48.1    \\      % token
DeepMerge & Token  & 64.5 & 42.7 \\ % Tyepscript aligned linearized
\thistool{} & Token  & \textbf{69.1} & \textbf{68.2}  \\  
\bottomrule   
\end{tabular}
\label{tab:baselines_left}
\end{table}




As seen in Tab.~\ref{tab:baselines_left}, language model baselines' performance on merge resolution synthesis is relatively low, suggesting that the naturalness hypothesis is insufficient to capture the developer intent when merging programs. This is perhaps not surprising given the notion of precision that does not tolerate even a single token mismatch. 

\thistool{} is based on two core components: token-level \texttt{diff3} and a multi-input neural transformer model. The token-level differencing algorithm alone gives a high top-1 precision of 82.4\%, with a relatively low accuracy of only 36.1\% (i.e., it doesn't always generate a resolution suggestion, but when it does, it is very often correct). Combined with the neural transformer model, the accuracy is increased to a total of 68.2\%. Note, as a deterministic algorithm token-level \texttt{diff3} can only provide a single suggestion. 

DeepMerge precision of merge resolution synthesis is quite admirable, showing 55.0\% top-1 precision. However, it fails to generate predictions for merge conflicts which are not representable as a line interleaving. This type of merge conflict comprises only roughly one third of the test set, resulting in an accuracy of only 35.1\% which is significantly lower than \thistool{}.

% \negar{should we mention both the token and the line-lvl precision of DeepMerge here?}.
% \alexey{Separated this as an ablation experiment below...}
% \negar{This number (63.8\%) is not in the table, and hence, a bit confusing. We can only mention the low accuracy and its reason, i.e., line interleaving.} 
% \alexey{Negar: I edited the paragraph above per your suggestion. Please take a look and make changes if needed}

As an experiment, we also evaluate the DeepMerge model in combination with the token-level \texttt{diff3}. This enables DeepMerge to overcome the limitation of providing only resolutions comprised of interleavings of lines from the conflict region by interleaving tokens instead. As seen in Tab.~\ref{tab:baselines_left} (DeepMerge with Token granularity) overall accuracy improves from 35.1\% to 42.7\%. However this still falls short of \thistool{} with precision that is 5\% less (64.5\% vs. 69.1\%) and accuracy that is 25\% less (42.7\% vs 68.2\%). 

%Second, note that DeepMerge can only produce resolution suggestions for merge conflicts that are representable as an interleaving of lines from the conflict region. As an experiment, we tried evaluating performance when restricting the test set to this type of merge conflicts only. Using this smaller test set of resolutions that DeepMerge could provide correct resolutions for, \thistool{} has 70.6\%  precision (not shown in Tab.~\ref{tab:baselines_left}) and Deepmerge has 55.0\%.


\begin{table}[htb]
\small
\caption{Comparison of \thistool{} to \jdime{} and \jsfstmerge{} semi-structured merge tools. The table shows the percentage of conflicts in which the tool produces a resolution, the top-1 precision of produced resolutions, and the overall top-1 accuracy of merge resolution synthesis. \jdime{} evaluation is on a Java data set and \jsfstmerge{} is on a JavaScript data set.}
\vspace{-4pt}
\centering
\resizebox{0.98\columnwidth}{!}{%
\begin{tabular}{lllllll} \toprule
\textbf{Approach} & \textbf{Language} & \textbf{\% conf. w/ res.} & \textbf{Precision} & \textbf{Accuracy} \\ %& \textbf{Syn (\%)} \\ 
\midrule
\jdime{} & Java & 82.1 & 26.3 & 21.6 \\ %&  90.9 \\
\thistool{} & Java & \textbf{98.9} & \textbf{63.9} & \textbf{63.2} \\ \midrule % & \textbf{98.3} \\ \midrule
\jsfstmerge & JavaScript & 22.8 & 15.8 & 3.6 \\ %& 94.4 \\
\thistool{} & JavaScript & \textbf{98.1} & \textbf{66.9} & \textbf{65.6} \\ %& \textbf{97.4} \\ % JavaScript
\bottomrule
\end{tabular}%
}
\label{tab:baselines_right}
\vspace{-6pt}
\end{table}

We also compared \thistool{} to state of the art structured and semi-structured merge tools.  Since both \jdime{} and \jsfstmerge{} are language-specific, to compare against \thistool{}, we use our dataset's corresponding language-specific subset of conflicts (leading to slightly different results for \thistool{} on Java and JavaScript).

As can be seen from Tab.~\ref{tab:baselines_right}, \jsfstmerge{} only produces a resolution for 22.8\% of conflicts and when a resolution is produced by \jsfstmerge{}, it is only correct 15.8\% of the time, yielding a total accuracy of 3.6\%. 
This is in line with the conclusions of the creators of \jsfstmerge{} that semi-structured merge approaches may not be as advantageous for dynamic scripting languages~\citep{tavares2019javascript}. Because \jsfstmerge{} may produce reformatted code, we manually examined cases where a resolution was produced but did not match the user resolution (our oracle).  If the produced resolution was semantically equivalent to the user resolution, we classified it as correct.

To compare the accuracy of \textsc{JDime} to that of \thistool{}, we use the Java Test data set introduced previously and complete the following evaluation steps: \textsc{JDime} does not merge all conflicts and generates a resolution for 82.1\% of conflicts. This is in line with related work reporting that as much as 21\% of files cannot be merged~\cite{apel2012structured}. Therefore, first, we identify the set of merge conflict scenarios where \texttt{diff3} reports a conflict and \textsc{JDime} produces a non-conflicted merge. 
% \Shuvendu{Why not cases where JDIME does not resolve the conflict?}, \sarah{This is because JDime only resolves conflicts 35\% of the time. If we were to consider all conflicts, accuracy would be lower than 15\% as reported in the ICSE paper. I believe the numbers here are from ICLR submission, so this decision is persisting from what we decided earlier. In the ICSE paper we did this differently. We reported the percentage of conflicts with predicted resolutions, precision as the number of correctly resolved conflicts over the total number of conflicts for which an approach attempts a resolution. Accuracy is the number of correctly resolved conflicts over the total number of conflicts. } 
When comparing the \textsc{JDime} output to the actual historical user-performed merge conflict resolution, we do not use a simple syntactic match.  As a result of its AST matching approach, code generated by \jdime{} is reformatted, and the original order of statements and other constructs are not always preserved. 
%In addition, source code comments that are part of conflicting code chunks are not merged. This makes a simple syntactic comparison is too restrictive, and \jdime{} merge output can still be semantically correct. 
In an effort to accurately and fairly identify semantically equivalent merges, we use GumTree \cite{FalleriMBMM14}, an AST differencing tool, to identify and ignore semantically equivalent differences between \textsc{JDime} output and the user resolution, such as reordered method declarations. When \textsc{JDime} produces a resolution, it generates a semantically equivalent match 26.3\% of the time, with an accuracy of 21.6\%. 
%we have a more accurate baseline comparison between the number of semantically equivalent merges generated by \jdime{} and \thistool{}.

\noindent \textbf{RQ\scriptsize{2}: }\textbf{\rqTwo}
One goal of our approach is to be able to handle multiple languages with minimal effort.  For \thistool{} to be able to provide merge resolution suggestions for conflicts in a particular language, it needs three things.  First, a tokenizer in that language, which allows us to split the source text into tokens for processing.  Second, a parser in that language, which allows us to filter out syntactically incorrect merge resolution suggestions. Third, a data set of merge conflicts and their user-resolutions to train \thistool{}.  Fortunately, tokenizers and parsers for nearly any language are readily available (e.g., we use GitHub's tree-sitter for this) and repositories that use a particular language can be easily identified (e.g. on GitHub) and mined for conflicts and resolutions.

We incorporated tokenizers and parsers into \thistool{} for JavaScript, TypeScript, Java, and C\# and gathered merge conflict data for these languages as described previously.  
Note that both comments and strings in these languages are represented as single tokens and can be quite long.  
Therefore we further split these tokens on whitespace.
Tab.~\ref{tab:mergebert_summary} shows the detailed evaluation results of \thistool{} broken down by language. The top section of results shows performance when \thistool{} is trained on data for that specific language.  
The bottom section shows performance for each language when \thistool{} is trained on a data set comprising data for all languages (we term this the \emph{multilingual} model).
Note that for the language specific models, performance is fairly consistent across all four languages with Top-1 precision ranging from 63.9\% to 69.1\% and Top-1 Accuracy ranging from 63.2\% to 68.2\%. We also find that over 97\% of \thistool{} suggestions are syntactically correct across all programming languages. 

We had no a priori expectations of the performance of the multilingual model, as it is trained on more data, which could lead to improvement, but it is not language specific, which could lead to poorer results.
%As can be seen, the multilingual variant of \thistool{} yields $63.6-68.5\%$ top-1 and $75.2-76.8\%$ top-3 precision of verbatim match and relatively high recall values.
Overall, the multilingual variant of the model generates results that are just slightly below the monolingual versions.
Thus performance on one language isn't improved by adding more data in other languages.
Thus, from a pragmatic perspective, if one chooses to simplify their use of \thistool{} by training just one model instead of one model per language, then the performance takes only a negligible hit.



\begin{table}
\small
\caption{Detailed evaluation results for (top) monolingual JavaScript, TypeScript, Java, and C\# models, and (bottom) multilingual \thistool{} model trained on all four programming languages. The table shows precision and accuracy of merge resolution synthesis.}
\vspace{-4pt}
\centering
\begin{tabular}{llllllllllll} \toprule
\textbf{Test (Train) Languages} & \multicolumn{2}{c}{\textbf{Precision}} &  \multicolumn{2}{c}{\textbf{Accuracy}}  \\ \cmidrule{2-3} \cmidrule{4-5} 
& Top-1 & Top-3 & Top-1 & Top-3 \\ 
\midrule
JavaScript (JS)  & 66.9 &75.4 & 65.6& 73.9 \\ %& 97.4 \\    % monolingual
TypeScript (TS)  & 69.1 &76.6 & 68.2& 75.6 \\ %& 97.0 \\   % monolingual
Java (Java)  & 63.9 &76.1 & 63.2 &75.2 \\ %& 98.3 \\  % monolingual
C\# (C\#)  & 68.7 &76.4 & 67.3& 74.8 \\ %& 98.3 \\   % monolingual
\midrule
JavaScript (JS, TS, C\#, Java)  & 66.6& 75.2 & 65.3 &73.8 \\ %& 97.4 \\   
TypeScript (JS, TS, C\#, Java)  & 68.5 &76.8 & 67.6 &75.8 \\ %& 96.9 \\   
Java (JS, TS, C\#, Java)  &  63.6 &76.0 & 62.9& 75.1 \\ %& 98.2 \\  
C\# (JS, TS, C\#, Java)  & 66.3 &76.2 & 65.1 &74.8 \\ %& 98.3\\  
\bottomrule
\end{tabular}
\label{tab:mergebert_summary}
\vspace{-4pt}
\end{table}

\noindent \textbf{RQ\scriptsize{3}: }\textbf{\rqThree}

We conduct an ablation study on the edit type embedding to understand the impact of edit-awareness of encoding on the model performance. As shown in Tab.~\ref{tab:edit_ablation}, use of the edit type embedding improves  \thistool{} from 63\% to 68\%.
\begin{table}[htb]
\small
\caption{Evaluation results for \thistool{} and the model variant without edit-type embedding for merge conflicts in TypeScript programming language test set. The table shows top-1 precision and accuracy metrics.}
\centering
\begin{tabular}{lllllll} \toprule
\textbf{Approach} & \textbf{Precision} & \textbf{Accuracy}   \\ 
\midrule
w/o edit type embeddings  & 65.2 & 63.1  \\
\thistool{} w/ edit type embeddings & \textbf{69.1} & \textbf{68.2}  \\ 
\bottomrule
\end{tabular}
\label{tab:edit_ablation}
\end{table}

%\negar{I have a general suggestion: We can move Tab. 4 up to the beginning of the results section. This way, for RQ1, which is about efficiency, we can mention and briefly discuss Tab. 4 as the detailed evaluation results of MergeBERT and continue with the comparison of MergeBERT and other techniques using Tab. 2 and 3. Then for RQ2, we can mention Tab. 4 again and further discuss the monolingual and multilingual results.}

\section{User Evaluation}
\begin{figure}[h] %!th
   \includegraphics[width=0.46\textwidth, angle=0]{images/method.png}
  \caption{Methodology to identify candidate conflicts for the user study.}
  \label{fig:methodology}
  \vspace{-12pt}
\end{figure}

\begin{table}[t!]
\small
\centering
\caption{Summary of projects in user study, total number of conflicts per project, number of conflicts evaluated in the study, and the survey participants.}
\label{tab:projects}
\begin{tabular}{llcccl} \toprule
Language & Project & Conflicts & Survey & Participants \\
 &  &  &  Conflicts &  \\\midrule
\multirow{3}{*}{Java} & Azure-Cosmosdb  & 341 & 6 & P1 \\
 & Azure-SDK  & 997 & 14 & P2-4 \\
 & ApplicationInsights&  313 & 10 & P5-6 \\ \midrule
\multirow{2}{*}{TS} & MakeCode & 106 & 12 & P7-8 \\
 & VSCode  & 2256 & 48 & \begin{tabular}[c]{@{}l@{}}P9-17\end{tabular} \\ \midrule
\multirow{3}{*}{C\#} & AspNetCore  & 567 & 11 & P18-19 \\
 & EFCore  & 397 & 7 & P20-21 \\
 & Roslyn  & 1894 & 14 & P22-25 \\ \midrule
 Total & 8 projects  & 6871 & 122 & 25 \\ \bottomrule
\end{tabular}
\end{table} 

\subsection{User Study Design}

To better understand how \thistool{} performs in practice, we ask developers about conflicts that \thistool{} is unable to correctly resolve. Since \thistool{}'s resolution suggestions are evaluated against user resolutions using a verbatim string match (modulo whitespace), asking study participants to confirm identical resolutions predicted by \thistool{} is not informative. Therefore, we extract conflicts where \thistool{} suggestions are not a direct match to the user resolution to determine what the limitations of the suggestions are, and how they might be perceived in practice.

To build an oracle of merge conflicts and resolutions we identify 8 open source projects hosted on GitHub. The selected projects are active, with multiple contributors, and contain a large number of conflict scenarios in one of the languages supported by \thistool{}.
Tab.~\ref{tab:projects} contains a list of projects chosen.
% EFCore\footnote{https://github.com/dotnet/efcore}
% Roslyn\footnote{https://github.com/dotnet/roslyn}
% AspNetCore\footnote{https://github.com/dotnet/aspnetcore}  
% VSCode\footnote{https://github.com/microsoft/vscode}
% MakeCode\footnote{https://github.com/microsoft/pxt} 
% Azure-SDK-Java\footnote{https://github.com/Azure/azure-sdk-for-java} 
% Azure-Cosmosdb-Java\footnote{https://github.com/Azure/azure-cosmosdb-java}
For each project, we follow the same steps outlined in Section~\ref{sec:dataset} to extract candidate conflicts and user resolutions to use in the survey.



Fig.~\ref{fig:methodology} explains the methodology used to identify candidate merge conflicts. We identify the set of conflicts \thistool{} is unable to correctly merge (within the top-3 suggestions). From this set of conflicts, we identify candidate conflicts to use as part of the user study. We filter candidate files with the following criteria:
\begin{enumerate}
  \item Conflicts should have been recently resolved i.e., at most within the past 12 months. Participants may not retain the context needed to evaluate suggestions for older conflicts. 
    \item Files must have at most 4 conflicts. Participants evaluate up to 3 suggestions per conflict. More conflicts may be too complex to evaluate within the interview time slot. 
    \item Conflicts should be non-trivial.  Trivial conflicts, such as those that only involve formatting changes or renames, are manually excluded. The determination of if a conflict was non-trivial was manual and subjective, informed by our belief that more substantive conflicts would lead to more insights in the user study.
\end{enumerate}

%\sarah{this might be the place to add discussion of subsumptive merge conflicts?}
For each candidate conflict identified, we use the GitHub API to identify authors for each of the conflicting branches and the resolved file. Authors with at least 3 candidate merge conflicts are identified as potential survey participants. Our final pool of candidate participants consists of 52 unique authors. We recruit participants via email, using contact information on GitHub. Out of the 52 contacted developers, 25 agreed to participate in the study. All participants were professional software developers with at least 2-8 years of experience working at large technology companies. We asked participants to evaluate \thistool{} resolution suggestions for their own merge conflicts.  Tab.~\ref{tab:projects} contains the final number of participants and conflicts evaluated in our study. 122 conflicts were evaluated: 32 C\# conflicts, 30 Java, and 60 Typescript. 

\subsubsection{\thistool{} Interface}
We designed an online interface where participants can view their own conflicts and explore \thistool{}'s resolution suggestions. Participants are asked to evaluate their own recently resolved merge conflicts, and the corresponding generated resolution suggestions by \thistool{}. The interface is customized based on the signed-in participant and displays a list of their recently encountered merge conflicts. Participants can click through different resolution suggestions to evaluate if they are acceptable ways to resolve the merge conflict. They can view their original resolution on the same page, and if needed, participants can navigate to the conflicting commit on GitHub using a link if they need additional context. They can also view a diff between the conflict file and any of the selected options (resolution suggestion or user resolution). Participants use this interface to select one or more of the suggested resolutions, indicate if the suggested resolution is acceptable, and explain the reasons why or why not.  Our online data package~\cite{ICSE22Replication} and appendix~\cite{FSE22Appendix} contain the questions, images of the interface, and participant responses.  

% \begin{figure}[h!] %!th
%  \includegraphics[width=0.5\textwidth, angle=0]{images/survey.png}
%  \caption{Interface used by participants to interact with \thistool{} resolution suggestions and answer survey questions.}
%  \label{fig:survey}
% \end{figure}


\subsubsection{Protocol}
The user study was conducted as 30 minute interviews remotely over Microsoft Teams using the interface we built. First, participants watched a video explaining \thistool{} and how to navigate conflicts and evaluate resolution suggestions using the interface. Then, the participants evaluated a set of conflicts and submitted their responses. One of the authors was on the teams call to help participants navigate the interface and ask any clarifying questions based on their evaluation of the \thistool{} resolution suggestions.
% Then participants were asked a list of questions on the following topics: (i) Their existing process to resolve merge conflicts, and obstacles faced, (ii) trust of existing merging algorithms and proposed approaches, and (iii) utility of suggestion-based merge conflict resolution tools.
Questions were iteratively developed based on two pilot interviews. Each interview was recorded for transcription and analysis. 
%Direct quotes used in the results were manually validated by the authors.

% \subsubsection{Analysis}
% \sarah{update to only describe survey result analysis}
% We analyzed the submitted survey responses, video recordings and generated transcripts of the semi-structured interviews. We divided the transcripts into different sections reflecting the semi-structured interview questions and then used open coding for each topic, looking for similarities and differences between the interviewees’ responses. 




\subsection{User Study Results}


\noindent \textbf{RQ\scriptsize{4}: }\textbf{\rqFour}


Using the interface participants evaluate the conflict resolution suggestions generated by \thistool{} and indicate if any of the suggestions were acceptable, and explain why or why not. There were no noticeable differences in the participants' responses across different languages or projects so we do not break down our results by those dimensions.
Participant's evaluations of the merge suggestions generally fall into three categories: 1) the merge suggestion is correct and would be used to resolve the conflict 2) the merge is incorrect but the correct resolution would require an understanding of external context and 3) the merge is incorrect and no external context is needed.  

\subsubsection{Acceptable Merge Suggestions}

Surprisingly, of the 122 conflicts included in the study, participants indicated that at least one of the 3 suggestions generated by \thistool{} was correct for 54\% (66/122) of the examples. By design, the suggestions presented in the study are not syntactically equivalent to the participant's original resolution, however, they still indicated that the suggestion was a correct merge. Using participant responses, we identify a few reasons why merge suggestions may be acceptable to a developer, even if it is not syntactically equivalent to their original resolution:

\vspace{6pt}
\noindent{}\textbf{Semantically Equivalent Resolution} (54 of 122 conflicts) \\
    Semantically equivalent resolutions include scenarios where the statements are re-ordered, equivalent changes made to naming or documentation, and unneeded import statements or commented out code is  preserved or removed. 
   
      One example in the study of conflicting changes that are both equally acceptable, and one is arbitrarily accepted by the resolving author is when authors of conflicting branches renamed the same variable with a slight variation:\\ \ic{SPAN\_TARGET\_ATTRIBUTE\_NAME} and \\ \ic{SPAN\_TARGET\_APP\_ID\_ATTRIBUTE\_NAME}. In these cases, either version selected by the merging algorithm might still be acceptable to the developer. \textsc{MergeBert} generated a suggestion to keep the variable name \ic{SPAN\_TARGET\_ATTRIBUTE\_NAME} whereas the user resolution originally kept the other. Participant P5 marked this resolution as acceptable and semantically equivalent, explaining that in this scenario they had `no preference as to which one is better'.
%\sarah{add more examples if we have room}      

\mybox{Takeaway 1}{grey!20}{grey!7}{Evaluating the performance of \thistool{} using strict syntactic approaches estimates a lower bound of performance. Survey results show  almost 45\% of \thistool{} suggestions are acceptable merges that are semantically equivalent to the participant's original resolution. \thistool{}'s performance could be improved by considering semantic information, for example, to identify how changes related to naming or documentation should be merged.}
\noindent{}\textbf{Tangled Code Changes in Oracle} (10/122) \\
Resolutions for 10 of the conflicts contained additional ``tangled'' changes~\cite{Herzig:msr13:ImpactOfTangledCodeChanges,Kirinuki:icpc14:TangledChanges} that were unrelated to the resolution. Examples include renaming a method and adding a variable in the conflict region that is then used later outside the conflict region.
In all 10 instances, \textsc{MergeBert} generates a suggestion that does not include the additional tangled code, but is acceptable to the participant as a resolution of the conflict. 
Participants indicated that if they had access to the \textsc{MergeBert} suggestions, they would select the correct resolution and then manually add the additional code. 

\mybox{Takeaway 2}{grey!20}{grey!7}{When committing merged code, developers may introduce changes unrelated to the conflict which are inadvertently included in conflict resolution oracles. These changes can negatively impact model performance estimated with automatic metrics.}




\subsubsection{Merge Requires External Context}
 \textsc{MergeBert} did not generate an acceptable suggestion for 46\% (56/122) of examples shown to survey participants. Participants were asked to indicate whether they resolved these examples using external context that cannot be inferred from the conflicting code regions and to explain what the external context was.  
Results indicate that 16\% (20/122) of conflicts in the survey sample require external information not found in either conflicting file, in order to be correctly resolved. 
One example of external context is knowledge of linter rules enabled at a project level. Projects often require linter checks before code can be committed to the repository, as a step towards improving the quality and maintainability of the source code. One example is a merge conflict from Roslyn where the correct resolution was to remove a null check from the code. Participant P23 explained the decision to remove the check: \emph{"The previousResults parameter is non-nullable because C\# nullability checking is now enabled at the project level. The null check is unnecessary"}. In this scenario, without specific knowledge of linter checks, an automatic approach is unable to predict an accurate merge. 
 
 Another example of external context is updates to languages rules that have cascading effects on existing code. Participant P22 from the Roslyn project explained one such conflict:  "Changes were due to updates in '\ic{using}' rules for the C\# language".  Language updates in C\# version 8.0 introduced an alternative syntax for the \ic{using} statement and P22's team had made to adopt this syntax.  P22 therefore updated this code (involved in the conflict) during the merge. Other examples of external context identified through the survey include: removal of global dependencies from non-conflicting files within a project, rolling back features that shouldn't be included in a release branch, and project-level decisions to remove \ic{'final'} modifiers for variables. 

\mybox{Takeaway 3}{grey!20}{grey!7}{The local view of a conflict is sufficient to merge a majority of conflicts. Around 16\% of the conflicts  require external information to correctly resolve. One direction to improve \thistool{} is to consider external context as an additional information source for resolving conflicts.}

\subsubsection{Unacceptable Merge Suggestions}
Survey results show that \textsc{MergeBert} suggestions were incorrect for 29\% (36/122) of the conflicts. Participants indicated that none of the 36 conflicts required external context to be resolved. We manually analyze the conflicts looking to identify patterns that may explain the incorrect merges, for example, affected language construct~\cite{pan2021ProgramSynthesis} and type of conflict~\cite{shen2021automatic}, but do not identify any consistent patterns. 
% \Shuvendu{Is this really true? Wasn't extraneous and subsumptive merges came here?} \sarah{subsumptive conflicts are filtered out before the survey since they have obvious resolutions. These unacceptable suggestions are for other conflicts where MergeBert is suggesting something that really didn't make sense.}
In summary, existing automatic evaluation strategies estimate a lower bound of approach performance: \thistool{} suggestions are correct for 54\% of conflicts included in our sample, despite not being syntactically equivalent to the user resolution.  Further, suggestions from \thistool{} helped two participants find bugs in their own recent merge conflict resolutions!  This is in addition to those resolutions where \thistool{} does provide an exact match.  This finding suggests that automatic evaluation techniques that rely on a strict syntactic comparison between the user resolution and merge suggestion might be estimating a much lower bound of performance. This highlights a discrepancy between how approaches are typically automatically evaluated, and how developers may evaluate an approach in practice.  Researchers should consider conducting user studies to more accurately evaluate approaches when feasible. 
 Tools like \thistool{} can reduce effort and bug proneness involved in manually merging conflicts. Future studies should investigate these potential benefits. 
%\chris{definitely include that we found bugs in two developer resolutions.  In addition, we should reiterate that the 54\% of correct merges resolutions is out of the percentage that \thistool{} didn't already match exactly correctly.  We don't want the reader to think that we learned that \thistool{} is correct only 54\% of the time.}
% \sarah{subsumptive conflicts are one way MergeBert can significantly improve. We need to find a place to introduce them, explain how they are filtered out of the survey questions, their proportion.}
%%%%%%%%%%%%%%%%%%%%%%%%%%%%%%%%%%%%%%%%%%%%%%%%%%%%%%%%

The industry standard for pose edition is to create rigs, a collection of pieces of software designed to manipulate a character's skeleton. The rig describes the skeleton's bones, how they relate to each other, are constrained in their possible motion and are deformed. These rules are loosely specified and creating a good rig requires a detailed understanding of physics and anatomy, as well as technical and artistic skills. Rigging is thus a time consuming task even for experienced animators, and even more so in large scale productions which often require a different in-depth rig for each character in the cast.
Previous work has helped alleviate this difficulty by providing efficient tools to speed up/and or ease the rigging process, relying on inverse kinematics or data-driven methods.
\subsection{Character pose design}
\subsubsection{Inverse Kinematics (IK)}
IK solvers are a family of methods commonly used in robotics, engineering and computer graphics, in which the parameterization of a kinematic chain is determined from the position of its end effector.
They are a staple tool in pose design software, ensuring the respect of elementary constraints during pose edition. Their de-facto role is to guarantee the length of the limbs, and in some cases to enforce the orientation angle range of a joint.
Many IK solutions have been studied over the years \cite{aristidou_inverse_2018}; usually revolving around approximated linearizations or heuristics. 

Numerical methods require a set of iterations to achieve a satisfactory solution formulated by a cost function to be minimized.
IK solutions can generally be divided into three sub-categories: Jacobian \cite{Siciliano_Handbook_Robot_2007}, Newtonians \cite{cohen_ik_1996} and Heuristics. Most software implement heuristic methods such as Cyclic Coordinate Descent (CCD) \cite{wang_ccd_1991} or 
Forward-Backward Reaching IK (FABRIK) \cite{aristidou_fabrik:_2011} due to their simplicity and extensibility. 

The main drawback of 
these solvers is that they manipulate kinematic chains without taking into account many morphological aspects that make a pose more or less plausible. They offer a first level of help to users but are not sufficient to guarantee a realistic pose. Many joints constraints are dependent on each other and require subjective, human-made approximations.

\subsubsection{Data-driven pose edition}
Data-driven methods offer promising opportunities to solve these approximations. Using real-life data can help in modelling the complex inter-dependencies of skeletons and providing users with smarter edition tools.
While it is still an early field of research, some solutions have been studied. Wu \etal \cite{wu_posing_2009} propose a method for natural character posing from a large motion database. It employs adaptive KD-clustering to select a representative frame from a database and sparse approximations to accelerate training and posing. 
Huang \etal in \cite{Huang_IK_MGDM_2017} present a method based on the formulation of multi-variate Gaussian distribution models (MGDMs), which learn the joint constraints of a kinematic skeleton from motion capture data. 

Some work has also been dedicated to finding new editing interfaces. \modify{}{Instead of the usual setup manipulating joints directly, Guay \etal \cite{guay_line_2013} articulate a framework based on the conceptual "line of action" which describes the overall pose dynamics. They provide a mathematical definition of the line of action, and a interface in which the software modifies the pose to follow a user-provided line. In the same line of though} Garcia \etal \cite{garcia_sketching_2019} propose \modify{a method transforming doodle of trajectories (position and orientation over time) }{a virtual reality-based interface where the user's hands motion (position and orientation over time) are transformed} into sequences of actions and then into detailed character animations using a dataset of parametrized motion clips automatically fitted to the trajectory. 

% ==> DL et Latent Space. 
\subsection{Neural modelling of human motion}
Neural networks have received a great amount of attention over the last decade and shown impressive result in modelling complex data. Human motion has not been spared and deep learning methods have proven their capability of generating realistic motion in a number of difficult cases. 

The literature in neural-based animation include example in user-controlled character navigation \cite{Holden2017} and interactions with the environment \cite{starke_neural_2019}. 
Holden \etal \cite{Holden2020} also show that neural networks can be used to replace parts of existing data-driven methods, improving their scalability potential.
More recently, some work has also focused on improving smaller parts of the animation pipeline rather than replacing it completely. Berson et al. \cite{berson_intuitive_2020} leverage neural networks to provide an interactive system to edit facial animation. 

% Wrap up
Data-driven IK and pose editing can relieve animators from time-consuming, back-and-forth pose adjustments by applying constraints extracted from real-world data. Recently, neural-network-based approaches have demonstrated their ability to model the intricacies of human motion while scaling to large amount of data and retaining a fast inference time. In this paper we seek to take advantage of these properties to create an efficient posing tool, intuitively usable even by a inexperienced user.

While this initial study provides promising evidence that prompt engineering can enhance LLMs for software traceability tasks, several threats could limit the validity of our findings. First, we evaluated only three open-source projects and only provide a detailed analysis of one, limiting the generalization of our findings. However, we selected projects that spanned multiple domains, artifact types, and sizes to improve generalizability. We also constructed trace queries that were representative of their parent distribution. Second, existing traceability datasets are typically incomplete, as truly considering every candidate link in a project grows $\mathcal{O}(n^2)$ with the number of artifacts. The LLMs identified potential missing traces, but we could not fully validate their accuracy without a project expert. Third, our study used a limited set of LLMs which may not represent the full space of the current state-of-the-art. However, we chose the leading LLMs from our initial explorations with publicly available commercial models. Clearly, there are many extension to this study considering more datasets, different LLMs, and other prompt engineering methods. We leave the full exploration of the problem space to future work and focus on showing the potential these models have towards advancing automated software traceability.


\begin{comment}
\begin{figure}
\includegraphics[width=\linewidth]{figs/beyond_tss_lesion.pdf}
\caption[]{End-to-End runtime lesion study of the entire MNIST dataset and the FMA featurized music dataset. Each of DROP's contributions provides a runtime improvement.}
\label{fig:beyond_lesion}
\end{figure}
\end{comment}



\section{Conclusion}
\label{sec:conclusion}

Advanced data analytics techniques must scale to rising data volumes. 
DR techniques offer a powerful toolkit when processing these datasets, with PCA frequently outperforming popular techniques in exchange for high computational cost. 
In response, we propose DROP, a new dimensionality reduction optimizer. 
DROP combines progressive sampling, progress estimation, and online aggregation to identify high quality low dimensional bases via PCA without processing the entire dataset by balancing the runtime of downstream tasks and achieved dimensionality. 
Thus, DROP provides a first step in bridging the gap between quality and efficiency in end-to-end DR for downstream \red{analytics}. 

%We revisit canonical operators for time series dimensionality reduction and the measurement study of~\cite{keogh-study}, and show that PCA is more effective than popular alternatives in the data mining literature often by a margin of over $2\times$ on average on gold-standard time series benchmark data sets with respect to output data dimension. More surprisingly, we empirically demonstrate that a small number of samples are sufficient to accurately characterize directions of maximum variance and obtain a high-quality low-dimensional transformation.





\bibliography{refs}
\bibliographystyle{ACM-Reference-Format}

\end{document}