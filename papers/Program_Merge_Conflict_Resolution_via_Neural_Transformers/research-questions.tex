\section{Research Questions}

We pose the following research questions to evaluate the effectiveness of utility of \thistool{}.


\noindent \textbf{RQ\scriptsize{1}: }\textbf{\rqOne}
We evaluate \thistool{}'s performance of producting resolutions in terms of precision and accuracy of matching the actual user resolution extracted from real-world merge resolutions. We also provide a comparison \thistool{} to baseline approaches (at both the line and token level) and state of the art merge resolution approaches.

\noindent \textbf{RQ\scriptsize{2}: }\textbf{\rqTwo}
One of our primary goals is to be able to work on multiple languages with minimal effort.  
The core approach of \thistool{} is fundamentally language agnostic (though a parser and tokenizer is required for each additional language).  
We evaluate performance of \thistool{} across four languages and also compare the results of using four language-specific models (each trained on just one language) to using one multi-lingual model trained on the data from all four languages.

\noindent \textbf{RQ\scriptsize{3}: }\textbf{\rqThree}
We conduct an ablation study of the edit type embedding to understand and evaluate the impact of our novel edit-aware encoding on model performance.

\noindent \textbf{RQ\scriptsize{4}: }\textbf{\rqFour}
We conduct a user study involving a survey of real-world conflicts recently encountered by developers from large OSS projects. To understand how developers would use \thistool{} in practice, we provide them with an interface to explore \thistool{}'s conflict resolution suggestions in relation to their original conflicting code ask them evaluate suggestions and explain why they do or do not correctly resolve the merge conflict. 
