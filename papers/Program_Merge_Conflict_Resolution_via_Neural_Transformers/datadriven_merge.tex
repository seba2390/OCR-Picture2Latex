\section{Background: Data-driven Merge}
\label{sec:background}
\citet{Dinella2021} introduced the {\it data-driven program merge} problem as a supervised machine learning problem. 
A program merge consists of a 4-tuple of programs $(\mathcal{A}, \mathcal{B}, \mathcal{O}, \mathcal{M})$, where 
\begin{enumerate} 
\item The base program $\mathcal{O}$ is the lowest common ancestor in the version history for programs $\mathcal{A}$ and $\mathcal{B}$, 
\item \texttt{diff3} produces an unstructured line-level conflict when applied to $(\mathcal{A}, \mathcal{B}, \mathcal{O})$, and 
\item $\mathcal{M}$ is the merged program with the developer resolution, incorporating changes made in  $\mathcal{A}$ and $\mathcal{B}$. 
\end{enumerate}
A merge may have multiple unstructured conflicts, we define a {\it merge tuple} $(A, B, O, M)$, where $A, B, O$ correspond to the conflicting regions in $(\mathcal{A}, \mathcal{B}$, and $\mathcal{O})$, respectively, and $M$ denotes the resolution region.

Given a set of merge tuples $(A_i, B_i, O_i, M_i)$, i = 0...N, the goal of a data-driven merge algorithm is to learn a function, $\texttt{merge}$, that maximizes $\sum_{i=0}^{N}\texttt{merge}(A_i, B_i, O_i) = M_i$.
Throughout the text, we will use notations $(a, b, o, m)$ to refer to the token-level merge tuples. 

\citet{Dinella2021} also provide an algorithm for extracting the exact resolution regions for each merge tuple and define a dataset that corresponds to {\it non-trivial} resolutions; resolutions where the developer does not drop the changes from one side of the merge.  
Further, they provide a sequence-to-sequence encoder-decoder based architecture, where a bi-directional gated recurrent unit (GRU) is used for encoding the merge inputs comprising of $(A, B, O)$ segments of a merge tuple, and a {\it pointer mechanism} is used to restrict the output to only choose from line segments present in the input. 
Their paper suffers from two limitations.
First, given the restriction on copying only lines from inputs, their dataset  did not consider merges where the resolution required token-level interleaving, such as the conflict in Figure~\ref{fig:word1}. 
Second, their dataset consists of merge conflicts in a single language, namely JavaScript. 
Our approach addresses both of these limitations.
