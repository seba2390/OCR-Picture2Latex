% ---------- Code Example ----------------
\newcommand{\rulesep}{\unskip\ \vrule\ }

% ---------- End Code Example ----------------
%\thistool{} can deal with non-trivial real-world merges composed of multiple conflicting chunks. We include a complete example of such a merge conflict in the Appendix\footnote{add a clickable link like \url{http://www.google.com} to the appendix as a footnote. We should put the appendix as it's own pdf in the anonymized data sharing repo.}.

\begin{figure*}[t]
    \centering
    \begin{subfigure}[t]{0.3\textwidth}
        \includegraphics[width=\textwidth]{images/line-level-conflicts-with-prefix.pdf}
        \caption{Line-level conflict}
        \label{fig:line-level-conflict-b}
    \end{subfigure}
    \begin{subfigure}[t]{0.3\textwidth}
        \includegraphics[width=\textwidth]{images/token-level-conflicts-with-prefix.pdf}
        \caption{Token-level conflict}
        \label{fig:token-level-conflict-b}
    \end{subfigure}
    \begin{subfigure}[t]{0.3\textwidth}
        \includegraphics[width=\textwidth]{images/suggested-resolution-with-prefix.pdf}
        \caption{Resolved merge}
        \label{fig:suggested-merge-res-a}
    \end{subfigure}
    \caption{Example merge conflict represented through standard \texttt{diff3} (left) and token-level \texttt{diff3} (center), and the user resolution (right). The merge conflict resolution takes the token-level edit $b$.}
    \label{fig:word1}
\end{figure*}
%\vspace{-8pt}