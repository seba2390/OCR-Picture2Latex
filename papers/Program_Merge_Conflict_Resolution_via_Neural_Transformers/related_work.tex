\subsection{Related Work}

\newcommand{\etal}{et al. }
There have been multiple attempts to improve merge algorithms by restricting them to a particular programming language or a specific type of applications~\citep{mens2002state}. Typically, such attempts result in algorithms that do not scale well or have low coverage. Syntactic merge algorithms improve upon \texttt{diff3} by verifying the syntactic correctness of the merged programs. Several syntactic program merge techniques have been proposed \citep{westfechtel1991structure,Asklund1999IdentifyingCD} which are based on parse trees or ASTs and graphs. 

Apel~\etal noted that structured and unstructured merge each has strengths and weaknesses. 
They developed \textsc{FSTMerge}, a semi-structured merge, that alternates between approaches~\cite{apel2010semistructured}. 
Tavares~\etal implemented \textsc{jsFSTMerge} by adapting an off-the-shelf grammar for JavaScript to address shortcomings of \textsc{FSTMerge} and also modifying the \textsc{FSTMerge} algorithm itself~\cite{tavares2019semistructured}.
 Cavalcanti~\etal performed a large scale retrospective evaluation of semi-structure merge on over 30,000 merges and found that it can still suffer from false negatives, cases where there is actually a semantic conflict but the merge approach produces a (incorrect) resolution~\cite{cavalcanti2017evaluating}. They improve \textsc{FSTMerge} by adding ``handlers'' that check for common false negative cases (\emph{e.g.} renames, added references to modified elements) that remove these cases completely.
Le{\ss}enich noted that using AST representations works well for merge, but differencing is NP-hard due to renamings and shifted code. They propose an approach to improve performance of the \textsc{JDime} algorithm at minimal cost~\cite{lessenich2017renaming}.
%Sousa~\etal introduced \textsc{SafeMerge}, which verifies that the resolution preserves semantic conflict freedom~\cite{sousa2018verified}.
Dinella~\etal take a data driven approach to the merge conflict resolution problem and introduce \textsc{DeepMerge}, a deep neural network that uses a pointer network architecture to construct the resolution from lines in the different input versions of the code~\cite{Dinella2021}.
%Pan \etal~\cite{pan2021ProgramSynthesis} explore the use of program synthesis for learning repeated resolutions in a large project. 

Finally, \citet{Sousa18} explore the use of program verification to certify that a merge obeys a semantic correctness criteria, but does not help synthesize resolutions. 
On the other hand, \citet{pan-synthesis-2021} explore using program synthesis to learn repeated merge resolutions within a project. 
However, the approach is limited to a single C++ project, and only deals with restricted cases of import statements. 

\subsubsection{Empirical Studies}

Several empirical studies have investigated merge conflicts and challenges faced by developers in merge resolution. McKee \etal \cite{mckee2017software} and Nelson~\etal \cite{nelson2019life} interviewed developers and performed a follow-up survey with 162 developers to build a detailed understanding of developer perceptions regarding merge conflicts in general. They found, among other things, that complexity of the conflicting lines of code and file as a whole, the number of LOC in the conflict, and developers’ familiarity with the conflicting lines of code impact how difficult developers find a conflict to resolve.  %They also identified needs in merge tools that are currently unmet such as visualization of information and surfacing of meta-data.
Brindescu \etal investigated the impact of merge conflicts and their resolutions on software quality~\cite{brindescu2020lifting,brindescu2020empirical}.   They found that 20\% of code changes resulted in a merge conflict and the code in these conflicts were twice as likely to contain bugs as other changes. Further, if the changes included semantically interacting changes, the likelihood of a defect is 26 times that of non-conflicting changes.

Costa \etal presented TIPMerge, an approach for identifying and recommending developers to participate in merge sessions when resolving conflicts~\cite{costa2016tipmerge}. They  evaluated it on 2,040 merges across 25 open source projects and found that TIPMerge can improve joint knowledge coverage by an average of 49\% in merge scenarios~\cite{de2019recommending}.

Vale et al. \citep{vale2021challenges} performed an empirical study to understand what makes merge challenging for developers.  Through a large scale automated analysis and a survey of 140 developers, they identified factors that contribute to merge conflict resolution difficulty (e.g., number of chunks in the conflict and number of developers involved in the merge scenario).
Seibt et al. \citep{seibt2021leveraging} explore and evaluate merge algorithms on a suite of ten software repositories, paying attention to the amount of resolutions produced, size of conflict, runtime cost, and correctness. Interestingly, they use the test suites of each project as an oracle to assess correctness of code after the merge.

None of the existing studies evaluate automatic merge resolution tools with software developers on their own real-world conflicts. The participants in our survey have expertise to understand when \thistool{} resolution suggestions would be acceptable on their own real-world conflicts, providing rich explanations about when external context is required, or when tangled code changes are made.

%\sarah{maybe highlight correctness of a merge can be evaluated using test cases or comparison to actual user merge. Latest TSE paper by Seibt et al uses test cases.  }
%\Shuvendu{If we highlight use of tests, you will have to say that it substantially reduces our dataset due to the need to execute, and also a test may happily pass if we revert both changes to O.}



% \subsection{Merge Conflict Resolution Tools}


% There have been many attempts to improve merge algorithms by taking language specific analyses into account (see the work of Mens for a broad survey~\cite{mens2002state}).
% Westfechtel \etal use the structure of the source code to reduce merge conflicts~\cite{westfechtel1991structure}. 
% Apel~\etal noted that structured and unstructured merge each has strengths and weaknesses. 
% They developed FSTMerge, a semi-structured merge, that alternates between approaches~\cite{apel2010semistructured}. 
% They later introduced JDIME, an approach that automatically tunes a mixture of structured and unstructured merge based conflict locations~\cite{apel2012structured}. 
% Tavares~\etal implemented jsFSTMerge by adapting an off-the-shelf grammar for JavaScript to address shortcomings of FSTMerge and also modifying the FSTMerge algorithm itself~\cite{tavares2019semistructured}.
% Le{\ss}enich noted that using AST representations works well for merge, but differencing is NP-hard due to renamings and shifted code.
% They implemented an approach of looking ahead in the tree for common patterns of renaming and shifting and were able to improve performance of JDIME the algorithm at minimal cost~\cite{lessenich2017renaming}.

% Sousa~\etal introduced a verification approach, SAFEMERGE that examines the base program, both changed programs, and the merge resolution to verify that the resolution preserves semantic conflict freedom~\cite{sousa2018verified}.
% Dinella~\etal take a data driven approach to the merge conflict resolution problem and introduce DeepMerge, a deep neural network trained on nearly nine thousand JavaScript conflicts and resolutions that uses a pointer network architecture to construct the resolution from lines in the different input versions of the code~\cite{Dinella2021}.
% \chris{Add MergeBert when we get the URL from Alexey.}
% Pan \etal~\cite{pan2021ProgramSynthesis} explore the use of program synthesis for learning repeated resolutions in a large project. 
% The approach requires the design of a domain specific languages inspired by a small class of resolutions (around imports and macros in C++).

% Cavalcanti~\etal performed a large scale retrospective evaluation of semi-structure merge on over 30,000 merges and found that it can still suffer from false negatives, cases where there is actually a semantic conflict but the merge approach produces a (incorrect) resolution~\cite{cavalcanti2017evaluating}. They improve FSTMerge by adding ``handlers'' that check for common false negative cases (\emph{e.g.} renames, added references to modified elements) that remove these cases completely.

% Zhu and He present an approach for interactive merge resolution that uses Version Space Algebra to explore the space of possible resolved programs in a way that minimizes the number of interactions needed with a developer~\cite{zhu2018conflict}.




%In contrast to both these approaches, DEEPMERGE only requires a corpus of merge resolutions in the target language, and can apply to all merge conflicts. However, we believe that both these approaches are complementary and can be incorporated into DEEPMERGE.


% \subsection{Empirical Studies}




% Menezes et al. used number of conflicting chunks to determine patterns that occur in merge conflicts. \cite{menezes2020causes}\\

% Shen~\etal used JDIME, git-merge, and Crystal in a retro-spective style study on 204 merge conflicts in Java from 15 OSS projects~\cite{shen2021automatic}.  Similar to our findings, they found that the developers select the changes from just one side of the merge 69\% of the time.  They presented a taxonomy of merge conflict introduction and noted that existing tools are poor at detecting conflicts that manifest at run-time or during compilation.

% To assess different approaches to merge resolution, Cavalcanti~\etal evaluated structured and semi-structured merge approaches on over 40,000 merges across more than 500 projects and found that semi-structured merge suffer from false positives (where the tool provides a bad resolution in the presence of a true semantic conflict) and structured merges have more false negatives (merges that can actually be resolved without conflict, but the tool is unable to)~\cite{cavalcanti2019impact}.

% Le{\ss}enich \etal surveyed 41 developers to identify indicators of merge conflicts and evaluted these on 163 OSS projects~\cite{lessenich2018indicators}.  
% Surprisingly, \emph{none} of the seven indicators that emerged had any predictive power on the number of conflicts!

% McKee \etal \cite{mckee2017software} and Nelson~\etal \cite{nelson2019life} interviewed developers and then performed a follow-up survey with 162 developers to build a detailed understanding of developer perceptions regarding merge conflicts.
% They found, among other things, that complexity of the conflicting lines of code and file as a whole, number of LOC involved in the conflict, and developers’ familiarity with the lines of code in conflict all impact how difficult developers find a conflict to resolve.
% They also identified needs in merge tools that are currently unmet such as visualization of information and surfacing of meta-data.

% Brindescu \etal investigated the impact of merge conflicts and their resolutions on software quality~\cite{brindescu2020lifting,brindescu2020empirical}.  
% They presented a taxonomy of merge resolution strategies: ``TAKE ONE'', which is akin to ``Take A'' and ``Take B'', ``INTERLEAVING'', which is the union of our ``Concatenation'' and ``Combination'' strategies, and ``ADAPTED'', which is the same as our ``Additional Code'' strategy. 
% They found that nearly 20\% of code changes resulted in a merge conflict and the code in these conflicts were twice as likely to contain bugs as other changes.  Further, if the changes included semantically interacting changes, the likelihood of introducing defect is 26 times that of non-conflicting changes!

% Costa \etal presented TIPMerge, an approach for identifying and recommending developers to participate in merge sessions when resolving conflicts~\cite{costa2016tipmerge}.  They recently evaluated it on 2,040 merges across 25 open source projects and found that TIPMerge can improve joint knowledge coverage by an average of 49\% in merge scenarios~\cite{de2019recommending}.

% In an empirical study of over 3,000 merge conflicts, Mahmoudi \etal found that over 20\% were a result of refactorings~\cite{mahmoudi2019refactorings}.