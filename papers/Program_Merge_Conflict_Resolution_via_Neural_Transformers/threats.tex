\section{Threats to Validity}

The choice of hyper-parameters in our model (Section ~\ref{sec:implement}) is based on prior work of others and generally accepted norms~\cite{bert}.  It's possible that exploring the hyper-parameter space could yield different results.
The sample of conflicts and projects used in the study may pose a threat to the external validity of our work. We only considered public open-source projects hosted on GitHub, therefore, results may not generalize to closed source projects or repositories hosted on other platforms. To mitigate this threat, we select a diverse set of projects varying in size and language. Similarly, survey participants evaluate their own recently-merged conflicts and the set of conflicts used in the survey to answer RQ4 may not be a representative sample, as it was dependent on participant availability. We filtered out merge conflicts from the user study that we considered to be ``trivial'' conflicts.  This was a subjective judgement, but we did aim to select substantive conflicts in the hopes that they would elicit more valuable and informative feedback from participants.
The survey interface replicates the VSCode diff3 view. Participants not familiar with this view may have a harder time navigating the conflict view and answering survey questions, to mitigate this threat, we create an instructional video for participants to watch. 
 
%Threats to construct validity include...
 
%\sarah{adding threats section which is usually important for FSE. }
%\christodo{identify additional threats and how we attempt to mitigate them.}