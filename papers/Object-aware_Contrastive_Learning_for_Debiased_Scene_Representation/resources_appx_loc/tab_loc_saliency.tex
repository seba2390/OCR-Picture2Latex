\begin{figure}[h]
\centering\small

\begin{subfigure}{0.24\textwidth}
\includegraphics[width=\textwidth]{figures/saliency_original.png}
\caption{Original}
\end{subfigure}
\begin{subfigure}{0.24\textwidth}
\includegraphics[width=\textwidth]{figures/saliency_intgrad.png}
\caption{IntGrad \citep{sundararajan2017axiomatic}}
\end{subfigure}
\begin{subfigure}{0.24\textwidth}
\includegraphics[width=\textwidth]{figures/saliency_smoothgrad.png}
\caption{SmoothGrad \citep{smilkov2017smoothgrad}}
\end{subfigure}
\begin{subfigure}{0.24\textwidth}
\includegraphics[width=\textwidth]{figures/saliency_cam.png}
\caption{ContraCAM (ours)}
\end{subfigure}
\caption{
Visualization of various saliency methods using the contrastive score Eq.~\eqref{eq:con-score}.
}\label{fig:loc-saliency}

\captionof{table}{
MaxBoxAccV2 of various saliency methods using the contrastive score Eq.~\eqref{eq:con-score}. We compute the saliencies from the MoCov2 trained on the ImageNet dataset under the ResNet-50 architecture.
}\label{tab:loc-saliency}
\begin{tabular}{lccccc}
\toprule
Method & ImageNet & CUB & Flowers & VOC & OpenImages \\
\midrule
IntGrad \citep{sundararajan2017axiomatic} & 48.40 & 35.44 & 70.73 & 48.52 & 49.48 \\
SmoothGrad \citep{smilkov2017smoothgrad}  & 51.70 & 51.50 & 72.83 & 57.26 & 48.67 \\
ContraCAM (ours) & \textbf{55.88} & \textbf{64.07} & \textbf{75.64} & \textbf{59.40} & \textbf{49.89} \\
\bottomrule
\end{tabular}
\end{figure}