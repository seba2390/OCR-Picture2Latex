\begin{figure}[t]
\centering
\begin{subfigure}{0.46\textwidth}
\includegraphics[width=\textwidth]{figures/intro_crop.pdf}
\caption{Scene bias in random crop}\label{fig:intro-crop}
\end{subfigure}~
\begin{subfigure}{0.33\textwidth}
\includegraphics[width=\textwidth]{figures/intro_multi.pdf}
\caption{Performance drop}\label{fig:intro-multi}
\end{subfigure}~
\begin{subfigure}{0.19\textwidth}
\includegraphics[width=\textwidth]{figures/intro_bg.pdf}
\caption{Biased prediction}\label{fig:intro-bg}
\end{subfigure}
\caption{
Scene bias issue (a) and its negative effects on contrastive learning. (b) Linear evaluation \citep{kolesnikov2019revisiting} of the original MoCov2 \citep{he2020momentum} and our debiased method, trained and evaluated on the COCO \citep{lin2014microsoft} and Flowers \citep{nilsback2006visual} datasets, respectively, using the ResNet-50 architecture \cite{he2016deep}. The vanilla MoCov2 often loses its discriminative power as training goes as it entangles different objects, while the debiased model stably improves the classification performance. (c) Prediction of MoCov2 on an image from the Background Challenge \citep{xiao2021noise}. The vanilla MoCov2 makes decisions from the background instead of the object, leading to biased prediction on background-shifted images.
}\label{fig:intro}
\end{figure}

