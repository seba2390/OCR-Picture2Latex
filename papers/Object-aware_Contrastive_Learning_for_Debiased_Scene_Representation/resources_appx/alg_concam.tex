\begin{algorithm}[h]
   \caption{PyTorch-style pseudo-code for the Iterative ContraCAM.}
   \label{alg:iconcam}
    \definecolor{codeblue}{rgb}{0.25,0.5,0.5}
    \newcommand{\algofontsize}{9.5pt}
    \lstset{
      backgroundcolor=\color{white},
      basicstyle=\fontsize{\algofontsize}{\algofontsize}\ttfamily\selectfont,
      columns=fullflexible,
      breaklines=true,
      captionpos=b,
      commentstyle=\fontsize{\algofontsize}{\algofontsize}\color{codeblue},
      keywordstyle=\fontsize{\algofontsize}{\algofontsize}\color{black},
    }
\begin{lstlisting}[language=python]
# input: image (b, c, h, w)

masked_image = image  # initial: original image

queues = []
for i in n_iters:
    feature = get_features(masked_image)  # spatial features
    feature.requires_grad = True
    output = get_projection(feature)  # projection outputs

    if i == 0:
        key = output.detach()  # original images
    queues.append(output.detach())  # masked images

    score = contrastive_score(output, key, queues)  # See Algorithm 2
    cam = compute_cam(feature, score, size=(h, w))  # See Algorithm 3

    mask = max(mask, cam) if i > 0 else cam  # union over iterations
    masked_image = image * (1 - mask) + mask_color * mask  # soft mask

return mask
\end{lstlisting}
\end{algorithm}


\begin{algorithm}[h]
   \caption{PyTorch-style pseudo-code for the contrastive score.}
   \label{alg:con-score}
    \definecolor{codeblue}{rgb}{0.25,0.5,0.5}
    \newcommand{\algofontsize}{9.5pt}
    \lstset{
      backgroundcolor=\color{white},
      basicstyle=\fontsize{\algofontsize}{\algofontsize}\ttfamily\selectfont,
      columns=fullflexible,
      breaklines=true,
      captionpos=b,
      commentstyle=\fontsize{\algofontsize}{\algofontsize}\color{codeblue},
      keywordstyle=\fontsize{\algofontsize}{\algofontsize}\color{black},
    }
\begin{lstlisting}[language=python,upquote=true]
# input: query (b,d), key (b,d), queues List[(b,d)]

pos = einsum('nc,nc->n', [query, key])
neg = cat([einsum('nc,kc->nk', [query, q]) * (1 - query.size(0))
            for q in queues], dim=1)

score = (pos.exp().sum(dim=1) / neg.exp().sum(dim=1)).log()
return score
\end{lstlisting}
\end{algorithm}


\begin{algorithm}[h]
   \caption{PyTorch-style pseudo-code for the Class Activation Map (CAM).}
   \label{alg:cam}
    \definecolor{codeblue}{rgb}{0.25,0.5,0.5}
    \newcommand{\algofontsize}{9.5pt}
    \lstset{
      backgroundcolor=\color{white},
      basicstyle=\fontsize{\algofontsize}{\algofontsize}\ttfamily\selectfont,
      columns=fullflexible,
      breaklines=true,
      captionpos=b,
      commentstyle=\fontsize{\algofontsize}{\algofontsize}\color{codeblue},
      keywordstyle=\fontsize{\algofontsize}{\algofontsize}\color{black},
    }
\begin{lstlisting}[language=python]
# input: feature (b,c,h,w), score (b), size=(H,W)

grad = autograd.grad(score.sum(), feature)[0]

weight = adaptive_avg_pool2d(grad, output_size=(1, 1))
weight = weight.clamp_min(0)  # clamp negative weights

cam = sum(weight * feature, dim=1, keepdim=True).detach()  # weighted sum
cam = interpolate(cam, size=(H,W))  # scale-up to image size
cam = normalize(relu(cam))  # normalize to [0,1]
return cam
\end{lstlisting}
\end{algorithm}