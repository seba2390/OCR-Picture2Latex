We now introduce the planar GAF, and explain why its zeros are a
good approximation to those of the symmetric planar GAF. In other words, we
justify why the spectrogram of the real white Gaussian noise can be approximated
by that of the complex white Gaussian noise. We conclude by
considering the analytic white noise. 

\subsection{Definitions}
\label{s:complexWhiteNoise}
Consider the \emph{two-dimensional white noise} of \cite[Section
2.1.2]{HOUZ10}, that is, the space $\cS'\times
\cS'$, with the Borel $\sigma$-algebra associated to the product weak star topology, and measure
$\mu_1\times\mu_1$. A draw $\bom=(\fomega_1,\fomega_2)\sim\mu_1\times\mu_1$
consists of two independent white noises. Letting
$\boldsymbol{\phi}=(\phi_1,\phi_2)$ in
$\cS\times \cS$, we \emph{define} the smoothed complex
white noise as in \cite[Exercise 2.26]{HOUZ10} through 
$$ 
w(\boldsymbol{\phi},\fomega) =  \langle\fomega_1, \phi_1\rangle + i \langle\fomega_2, \phi_2\rangle, 
$$
where $\bom\sim\mu_1\times\mu_1 $. It is called ``smoothed'' because we define
it using a pair of test functions $\boldsymbol{\phi}$, which will be enough for
our purpose. Note also that in signal processing, this is typically called a
\emph{proper} or \emph{circular} Gaussian white noise \citep{PiBo97}.

Now, if we let both test functions be $t\mapsto M_v T_u g$, we recover what can
reasonably be called the STFT of complex white noise 
\beq
u,v\mapsto \langle\fomega_1, M_v T_ug\rangle + i \langle\fomega_2, M_v T_u
g\rangle.
\label{e:complexSTFT}
\eeq
\subsection{Characterizing the zeros}
\label{e:zerosPlanarGAF}
The same arguments as in the proofs of Propositions~\ref{p:series} and
\ref{p:symmetricPlanarGAF} lead to
\begin{prop}
\label{p:planarGAF}
With $\mu_1\times\mu_1$ probability $1$, the zeros of the STFT
\eqref{e:complexSTFT} are those of the entire function
\beq
\label{e:planarGAF}
\frac{1}{\sqrt{2}} \sum_{k=0}^\infty\left( \langle \fomega_1,h_k \rangle + i\langle \fomega_2, h_k
  \rangle \right) \frac{\pi^{k/2}z^{k}}{\sqrt{k!}},
\eeq
where $z=u+iv$.
\end{prop}
We note that under $\mu_1\times\mu_1$, the random variables $2^{-1/2}(\langle \fomega_1,h_k \rangle + i\langle \fomega_2, h_k
  \rangle)$ are i.i.d. unit complex Gaussians, and the entire function
  \eqref{e:planarGAF} is called the planar Gaussian analytic function in the
  literature. In particular, the planar
  GAF is one of the three fundamental GAFs in the monograph of \cite{HKPV09}, and more is known about its zeros than for the symmetric planar GAF in
  Proposition~\ref{p:symmetricPlanarGAF}. We group some known results in
  Proposition~\ref{p:propertiesOfPlanarGAF}, selecting results that could be of statistical use in signal processing.
\begin{prop}[\cite{HKPV09,Nis10}]
The planar GAF satisfies the following properties:
\begin{enumerate} 
\item The distribution of its zeros is invariant to rotations
  and translations in the complex plane \cite[Proposition 2.3.7]{HKPV09}. In
  particular, it is a stationary point process.
\item Its correlation functions are known \cite[Corollary 3.4.2]{HKPV09}. In
  particular, the intensity in constant equal to $1$, and with the notation of Section~\ref{s:spatial}, for $z,z'\in\mathbb{C}$ such that $\vert z-z'\vert=r$, the pair correlation function reads
\beq
\label{e:pairCorrelation}
\rho^{(2)}(z,z') = g_0(r) = 
\frac{\left[\sinh^2\left(\frac{\pi r^2}{2}\right) + \frac{\pi^2r^4}{4}\right]\cosh\left(\frac{\pi r^2}{2}\right) - \pi r^2\sinh(\frac{\pi r^2}{2})}{\sinh^3\left(\frac{\pi r^2}{2}\right)}.
\eeq
\item The hole probability 
$$
p_r = \mathbb{P}(\text{no points in the disk
    centered at $0$ and with radius $r$})
$$
scales as 
\beq
r^{-4}\log p_r \rightarrow -3e^{2}/4
\label{p:holeProba}
\eeq
as $r\rightarrow+\infty$ \citep{Nis10}.
\end{enumerate}
\label{p:propertiesOfPlanarGAF}
\end{prop}

\begin{figure}
\subfigure[Pair correlation functions $g_0$]{
\includegraphics[width=\twofig]{\figdir/complexWGNstats_rho.pdf}
\label{f:rho}
}
\subfigure[$r\mapsto L(r)-r$ functional statistics]{
\includegraphics[width=\twofig]{\figdir/complexWGNstats_L.pdf}
\label{f:L}
}
\caption{Comparison of the Ginibre point process, the zeros of the planar GAF,
  and a realization of the zeros of the spectrogram of complex white noise, using
  (a) pair correlation functions $g_0$, and (b) the $L$ functional statistic, see Section~\ref{s:spatial} for definitions.}
\label{f:functionalStatistics}
\end{figure}

Figure~\ref{f:functionalStatistics} illustrates
Proposition~\ref{p:propertiesOfPlanarGAF}. We plot the pair correlation function
\eqref{e:pairCorrelation} of the planar GAF, along with the pair correlation
functions of the Poisson and Ginibre point processes introduced in
Section~\ref{s:spatial}. We also superimpose an estimate of $g_0$
obtained from the spectrogram of a realization of a complex white noise, see
Section~\ref{s:stats} for computational procedures. Finally, we also plot the
$L$ functional statistic for the same point processes, as introduced in Section~\ref{s:spatial}.

Both the planar GAF and Ginibre are repulsive at small scales, but the planar
GAF alone has a small ring of attractivity around $r=1$, well visible in
Figure~\ref{f:rho}. This implies that the zeros of the planar GAF cannot be a
DPP with Hermitian kernel, as introduced in
Section~\ref{e:ginibre}, unlike what we and \cite{Fla17} may have intuited. DPPs
were indeed a good candidate for the zeros, as they are repulsive point
processes and naturally relate to reproducing kernel Hilbert spaces, such as
those behind the STFT \cite[Theorem 3.4.2]{Gro01}. But the zeros of the planar
GAF show no repulsion at large scales, and more importantly the pair correlation
function \eqref{e:pairCorrelation} is larger than $1$ around $r=1$, while the
pair correlation of a DPP with Hermitian kernel cannot exceed 1 by definition
\eqref{e:dppCorrelationFunctions}. Note that strictly speaking, it is still
possible that the zeros of the planar GAF are a DPP with a non-Hermitian kernel.

Even if they are not a DPP with Hermitian kernel, the zeros of the planar GAF
are often compared to the Ginibre ensemble, which is a DPP and is also invariant
to isometries of the plane \cite[Section 4.3.7]{HKPV09}. In particular, the decay of the log hole probability \eqref{p:holeProba} is also
in $r^{4}$ for the Ginibre process \cite[Proposition 7.2.1]{HKPV09}. This is to
be compared to the slower decay in $r^2$ of a Poisson process with constant
rate. This is an indication that locally, the zeros of the planar GAF and the
Ginibre ensemble are similarly rigid or regularly spread, and that both are more
rigid than Poisson. There are other intriguing similarities between the two
point processes, see \citep{KrVi14}, where Ginibre is shown to be the zeros of a
GAF with a randomized kernel.

\subsection{The zeros of the planar GAF approximate those of the symmetric
  planar GAF}
  \label{s:GAFapproxSYM}
  
To sum up, the spectrogram of real white noise is described by the symmetric
planar GAF, but the zeros of the planar GAF are more amenable to further
statistical processing. In this section, we survey results by \cite{Fel13} and
\cite{Pro96} that support approximating the zeros of the symmetric planar GAF by
those of the planar GAF.

The zeros of the symmetric planar GAF \eqref{e:symmetricPlanarGAF} have the same
distribution as the zeros of
\beq
f_{\sym}(z) = e^{-\frac{\pi}{2}z^2}\sum_{k=0}^\infty
\frac{a_k}{\sqrt{k!}}\pi^{k/2}z^k,
\label{e:scaledSymmetricGAF}
\eeq
where $a_k$ are i.i.d. unit real Gaussians. Note that the covariance kernel of $f_\sym$ is
\begin{eqnarray*}
K_{\sym}(z,w) &\triangleq& \mathbb{E} f_\sym(z)\overline{f_\sym(w)}\\
&=& e^{-\frac{\pi}{2}z^2} e^{-\frac{\pi}{2}\bar{w}^2} e^{\pi z\bar{w}}\\
&=& e^{-\frac{\pi}{2}(z-\bar{w})^2}.
\end{eqnarray*}
This hints some invariance of $f_\sym$ to translations along the real axis. By a limiting argument, see e.g. \cite[Lemma
  2.3.3]{HKPV09}, \eqref{e:scaledSymmetricGAF} is indeed a stationary symmetric
GAF in the sense of \cite{Fel13}. Namely, for any $n$, any
$z_1,\dots,z_n$, and any $t\in\mathbb{R}$,
$\left(f_\sym(z_1+t),\dots,f_\sym(z_n+t)\right)$ has the same distribution as
$\left( f_\sym(z_1),\dots,f_\sym(z_n) \right)$.

\cite{Fel13} derives the intensity of the zeros of general stationary
symmetric GAFs. More precisely, let $n_{\sym}(B)$ be the random number of zeros
of $f_\sym$ in a Borel set $B\subset\mathbb{C}$, she says that there exists a
so-called \emph{horizontal counting measure} $\nu_\sym$ s.t., almost surely, we have the weak convergence of measures
$$
\nu_\sym(A) = \lim_{T\rightarrow\infty}\frac{n_\sym([0,T]\times A)}{T},
$$
where $A$ is a Borel set on the vertical axis. In other
words, $\nu_\sym$ characterizes the density of zeros averaged across the
horizontal axis. For our symmetric planar GAF \eqref{e:scaledSymmetricGAF},
\cite[Theorem 1]{Fel13} yields
% \beq
% \nu_\sym(y)dy = dS(y) +
% \frac{1}{4\pi}\sqrt{\frac{\psi''(0)}{\psi(0)}}\delta_0,
% \eeq
% where 
% $$
% \psi(y)\triangleq K_{\sym}(x+iy, x+iy) = e^{-\frac{\pi}{2}(2iy)^2} = e^{2\pi
%   y^2} 
% $$
\begin{eqnarray}
\nu_\sym(A) &=& \int_{A} \left[ dS(y) + \delta_0\right],
\label{e:countingMeasureSym}
\end{eqnarray}
where
$$
S(y)  = \frac{y}{\sqrt{1-e^{-4\pi y^2}}}.
$$
%= \left[\frac{1}{4\pi}\frac{\psi'(y)}{\sqrt{\psi(y)^2-\psi(0)^2}}\right]
Equation \eqref{e:countingMeasureSym} is the sum of a continuous component and a
Dirac mass at $0$. The Dirac mass relates to the accumulation of zeros on the
real axis discussed in Section~\ref{s:real}. The numerator of the continuous
part $S$ is the unnormalized cumulative density of a uniform distribution, and the
denominator quickly converges to $1$ as $y$ grows. 

Now compare \eqref{e:countingMeasureSym} to the horizontal
counting measure of the zeros of the planar GAF, which is simply the uniform
$dy$, without any atom, see e.g. \cite[Theorem 1]{Fel13} again. We observe that the two counting
measures are quickly approximately equal, as one goes away from the real axis.
More precisely, for $A\subset [1,+\infty)$, the ratio of $S(A)$ by the Lebesgue
measure of $A$ is within $2\cdot 10^{-6}$ of $1$. For Gaussian windows of
arbitrary width, the change of variables \eqref{e:complexTiling} yields that
the approximation is tight for $\text{Im}(z)\geq a$. This is no obstacle in
signal processing practice, as spectrograms are never considered close to the
real axis, where 'close' is defined by the spread of the observation window in
frequency, which is of order $a$, see Section~\ref{s:analytic}. We also plot the
densities of the continuous part of both measures in Figure~\ref{f:GAFs}. The
Dirac mass of the symmetric planar GAF corresponds to the subset of zeros on the
real axis. 

A natural question is whether the approximation is also accurate for
higher-order interactions in the two point processes. This question can be
addressed by comparing $k$-point correlation functions. The case of the planar
GAF was derived by \cite{Han98}, and closed-form formulas are derived for the
symmetric planar GAF in \cite[Equation (12)]{Pro96}. The latter are not easy to
interpret as they involve nonstandard combinatorial combinations of matrix
coefficients. Still, \cite[Equation 25]{Pro96} shows that when $\text{Im}(z)\gg 0$, the
$k$-point correlation functions of the zeros of the symmetric planar GAF are
well approximated by those of the zeros of the planar GAF. 

To conclude, the distribution of the zeros of the STFT of real white Gaussian
noise is well approximated by that of complex white Gaussian noise, as long as
the observation window is sufficiently far from the time axis. 
% Rather than attempt a technical estimation of the difference between
% the $k$-point correlations of the two sets of zeros, we replicate in
% Figure~\ref{f:} a figure in

% \citep{Pro96} that shows the pair correlation function of the symmetric
% planar GAF on a horizontal strip $\mathbb{R}\times {iy}$ for several values of
% $y$. \textcolor{red}{Comment on how fast the approximation is accurate.}

\subsection{On the analytic white noise}
\label{s:analytic}
A real-valued function $f\in L^2$ has an Hermitian Fourier transform. In signal
processing, it is thus common to cancel out the negative frequencies of a
real-valued signal $f \in L^2$ by defining a complex-valued associated function called its \emph{analytic signal},
\begin{equation}
f^+(x) = 2 \cF^{-1}(\IND_{\mathbb{R}_+}\cF f)(x), \forall x\in\mathbb{R}.
\end{equation}
where $\cF$ is the usual Fourier transform. The term ``analytic'' is related to
the alternative definition of $f^+$ as the boundary function of a particular holomorphic function on the lower half of the
complex plane, see e.g. \cite[Section 2.1]{Pug82} for a concise and rigorous treatment. In signal processing practice, beyond removing redundant
frequencies, the modulus and argument of $f^+$ have meaningful interpretations
for elementary signals
 \citep{Pic97}. Since our initial goal is to understand the behaviour of the zeros of a
real white noise, it is thus tempting to define and consider an
analytic white noise to represent this real white noise. If this approach led to
a simple statistical characterization of zeros, then we would avoid the
approximation by the complex white noise of Section~\ref{s:complexWhiteNoise}.

While folklore has it that the analytic white noise is the circular white noise of Section~\ref{s:complexWhiteNoise}, this is not the case for the most natural
  definition of the analytic signal of a distribution. Following
  \citep[Section 3.3]{Pug82}, we define in this paper the analytic white noise by its action on $L^2$: letting $\fomega\sim\mu_1$ be a real white noise\footnote{As a side
    note, \cite[Section 3]{Pug82} investigates the random field that would be the formal equivalent to the holomorphic continuation of the classical analytic
signal of a function in $L^2$. But this time, the limit on the real axis is
rather ill-behaved.}, we take
\begin{equation}
\langle \fomega^+,f \rangle \triangleq 2 \langle \fomega,
\cF^{-1}(\IND_{\mathbb{R}_+}\cF f)\rangle, \qquad\forall f\in L^2.
\label{e:analyticWhiteNoise}
\end{equation}
For our purpose, it is enough to consider $\fomega^+$
through its action \eqref{e:analyticWhiteNoise}. In particular, if we want to
follow the lines of Sections~\ref{s:real} and \ref{s:complex} and identify the general term of
a random series corresponding to the STFT of
$\fomega^+$, we need an orthonormal basis $(\zeta_k)$ of $L^2$ and a window $g$ such that 
\begin{equation}
\langle \zeta_k, \cF^{-1}(\IND_{\mathbb{R}_+}\cF M_vT_u g)\rangle 
\label{e:startingPoint}
\end{equation}
is known in closed-form and simple enough. Hermite functions and the Gaussian
window definitely do not satisfy our criteria anymore, and we leave this existence as an
open question. Still, we have the following heuristic argument: when $g$ is the
unit-norm Gaussian, \eqref{e:startingPoint} becomes
\begin{equation}
\langle \zeta_k, \cF^{-1}(\IND_{\mathbb{R}_+}T_vM_{-u} g)\rangle,
\label{e:desiderata}
\end{equation}
so that when $v$ is
large enough, say a few times the width of the window $g$, $T_v M_{-u} g$ puts
almost all its mass on $\mathbb{R}_+$, and the indicator in
\eqref{e:desiderata} can be dropped. The Hermite basis then satisfies our
requirements, giving the planar GAF of Section~\ref{s:complex}. Intuitively, far
from the real axis, the spectrogram of the analytic white noise will look
like that of proper complex white noise. This heuristic is to relate to standard
time-frequency practice, where one leaves out of the spectrogram a band that is
within the width of the window of the lower half plane. This is meant to avoid
taking into account both positive and negative frequencies of the signal
simultaneously.

