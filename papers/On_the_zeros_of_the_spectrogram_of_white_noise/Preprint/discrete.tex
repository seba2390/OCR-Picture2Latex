In Section~\ref{s:implementation}, we discuss how to relate the continuous
complex plane $\mathbb{C}$ with the practical discrete implementation of the
Fourier transform. In Section~\ref{s:detection}, we investigate simple
hypothesis tests for signal detection, as in \citep{Fla15}. 

\subsection{Going discrete}
\label{s:implementation}
To fully bridge the gap with numerical signal processing practice, there is an
additional level of approximation that needs to be discussed: Continuous
integrals are replaced by discrete Fourier transforms, so that the fast Fourier
transform can be used. We first describe an experimental setting to study the zeros of the spectrogram of Gaussian white noise. In particular, we explain how to reach an asymptotic regime where the noise occupies an infinite range both in time and frequency and the spectrogram is infinitely well resolved.
% When a signal is present
Second, we investigate practical issues related to detecting a signal in white
noise by using its influence on the distribution of zeros of the spectrogram. 

%\note[RB]{Je crois qu'a ce stade, mieux vaut eviter de partler de discretization steps et de ``specific scales'', car on ne comprend pas encore de quoi il s'agit.}
%the consequences of the presence of signalwhen a signal is present, some specific scales enter in the game. Then one cannot change discretization steps at will anymore.


%%%%
\subsubsection{Zeros of noise only}
\label{s:noiseonly}
Let $F_s$ the sampling frequency, $\Delta t=1/F_s$ the time sampling step size
and $T$ the duration of the observation window. The number of samples is then $N +1$ with 
$N = T/\Delta t$. 

% Approximating $\langle \chi_n,M_v T_u g \rangle$ by $M_vT_ug (n\Delta t)$, it comes
% $$\langle  \fomega, M_{v}T_u g\rangle \approx \lim_{N\rightarrow \infty}
% \sum_{n=1}^N\langle \fomega,\chi_n \rangle e^{-2i\pi v n\Delta t} g(n\Delta t-u),$$
% and we thus expect the discrete spectrogram of a sequence of i.i.d Gaussians
% with large $N$ to be a good approximation to the STFT of white noise.
% }

Let $K$ be
the length of the discretized Gaussian analysis window, i.e. its duration is
$K\Delta t$; therefore $\Delta \nu = F_s/K=1/K\Delta t$ is the frequency
sampling step. In practice, the spectrogram obtained from a discrete STFT is
then an array of size $(N+1, K/2+1)$. Then we consider the time-frequency domain
$[0,T]\times[0,F_s/2]$ only; it corresponds to the analytic signal. This is due
to the Hermitian symmetry of the Fourier transform of real signals: negative
frequencies do not add any information to that carried by positive frequencies,
see also Section~\ref{s:analytic}. This Hermitian symmetry can also be seen on
the zeros of the symmetric GAF in Figure~\ref{f:symmetricPlanarGAF}, where
signal processing practice would have us only consider the upper half-plane ($\nu\geq 0$).
 From \cite{Fel13}'s results, see (\ref{e:countingMeasureSym}), we know that the expected number of zeros of the continuous spectrogram is close to $TF_s/2$ if we neglect the (asymptotically negligible) region $|\nu|\leq a$ close to the time axis, see Section~\ref{s:GAFapproxSYM}. Assuming that we are able to extract every zero, the expected number of zeros in the discrete spectrogram is then $TF_s/2=N/2$ in very good approximation.

Let $\sigma_t=1/(a\sqrt{2\pi})$ and $\sigma_\nu= 1/(2\pi\sigma_t)$
denote the spreads of the Gaussian analysis window $g_a$ in time and frequency, respectively. Note that the scale $a$ serves as a fixed reference for scales in the sequel. We would like to retain the stationary properties of the planar GAF in our discrete
STFTs. We thus require that, in the discrete setting, the resolution --~in number of points~-- should be the same in time and frequency, that is
\begin{equation}
\label{eq_isotropy}
	\frac{\sigma_t}{\Delta t} = \frac{\sigma_\nu}{\Delta \nu} \Longleftrightarrow \sigma_t \cdot F_s = \sigma_\nu \cdot K\Delta t
\end{equation}
This leads to
\begin{equation}
	\left(\frac{\sigma_t}{\Delta t}\right)^2 = \frac{K}{2\pi} \Leftrightarrow \sigma_t = \sqrt{\frac{K}{2\pi}}\Delta t.
\end{equation}
If we want to study the spectrogram of continuous white noise over an infinite time-frequency domain, numerical simulations must obey two necessary conditions:
\begin{equation}
	\begin{cases}
		\text{infinite duration } \Leftrightarrow \text{fine frequency resolution} & : T/\sigma_t = 2\pi\sigma_\nu/\Delta\nu \rightarrow + \infty\\
		\text{infinite frequency range }  \Leftrightarrow \text{fine time resolution} & : F_s/\sigma_\nu=2\pi\sigma_t/\Delta t \rightarrow + \infty\\
		%\text{infinitely fine resolution } & \sigma_t/T \rightarrow 0 \text{ and } \sigma_\nu/F_s \rightarrow 0\\
	\end{cases}
\end{equation}
In terms of samples, these two conditions imply that $N, K \rightarrow \infty$. More precisely, 
\begin{eqnarray}
	\frac{\sigma_t}{T} & = & \frac{1}{N}\sqrt{\frac{K}{2\pi}} \rightarrow 0 \text{ as } N, K \rightarrow \infty\\
	\frac{\sigma_\nu}{F_s} & = & \frac{1}{\sqrt{2\pi K}}\to 0 \text{ as } N, K \rightarrow \infty.
\end{eqnarray}

%% Bilan
These conditions are directly satisfied for $K \propto N$, where $\propto$ means
``proportional to''. Note that in practice because of border effects one chooses $N = 2K$ and keeps the $N$ samples whose time index $n$ is such that $K/2 \leq n \leq N - K/2$. Then, $\sigma_\nu/F_s=1/\sqrt{2\pi K}\propto 1/\sqrt{N}$, $\sigma_t/T\propto 1/\sqrt{N}$; note that $\Delta t/\sigma_t=\Delta \nu/\sigma_\nu\propto 1/\sqrt{N}$ as well. As a result, simulations can asymptotically well approximate the continuous spectrogram of Gaussian white noise.
%% Figure 3 and 4

\begin{figure}
\begin{center}
	\includegraphics[height=60mm]{\figdir/discretization_spectro.png}
	\caption{\label{fig_discrete} Illustration of the discrete time-frequency plane $\{(n\Delta t, k\Delta \nu),\; 0\leq n\leq N-1,\; 0\leq k\leq K/2\}$. The resolution of the spectrogram is controlled by the analysis window's Gabor parameters $(\sigma_t, \sigma_\nu)$.}
\end{center}
\end{figure}
\begin{figure}
	\centering
	\includegraphics[height=80mm]{\figdir/STFT_illustration.png}
\caption{\label{fig_stft} Illustration of the STFT: the noisy signal is convolved with a Gaussian
that is translated in time and frequency. The colour code corresponds to
Figure~\ref{fig_discrete} for ease of reference.}
\end{figure}{}

Figure~\ref{fig_discrete} illustrates the relative scales of the duration $T =
N\Delta t$, the frequency range $K/2\Delta t$ (for $\nu\geq 0$), the time and
frequency resolutions $\Delta t$ and $\Delta \nu$, as well as the resolution of
the time-frequency kernel corresponding to the window $g(t)$ with Gabor spread
$(\sigma_t, \sigma_\nu)$. For the sake of completeness and the reader new to
time-frequency, we include in Figure~\ref{fig_discrete} an illustration of the
STFT of a noisy signal.

%% discussion normalization+correspondence discrete.
Now we detail how to relate the discrete coordinates of a discrete spectrogram
with the continuous complex plane. For a given value of $a$, one has
$\sigma_t=1/(a\sqrt{2\pi})$ and thus making the correspondence between samples
and time-frequency units implies setting $\Delta t = \sqrt{2\pi/K}\sigma_t$. For
$a = 1$ one has $\Delta t = \sqrt{1/K}$ so that $u = n/\sqrt{K}$ and $v =
k/\sqrt{K}$ are the coordinates of the time-frequency plane corresponding to
time sample $n$ and frequency sample $k$, respectively.
Figure~\ref{fig:edge_effects} depicts the whole numerical simulation procedure.
It represents the simulated spectrogram and the corresponding extracted area,
taking border effects in consideration. The bound $\ell$ fixes how many samples
close to the zero-frequency axis should be removed. For $a=1$, we have chosen
$\ell = \sqrt{K}$, at it corresponds to $y = 1$ in (\ref{e:countingMeasureSym}).
Note also that border effects alone would actually allow us to extend the shaded
square in Figure~\ref{fig:edge_effects} on its left and right to include $K$
samples. Instead, we chose to reduce it to $K/2-\ell$ mostly for esthetical concerns:
since the point process we observe is almost stationary when only noise is
present, we favoured a square window rather than a rectangle.

% Figure 5
\begin{figure}
	\centering 
	\includegraphics[width=\textwidth]{\figdir/tikz/edge_discrete.pdf}\caption{Numerical
    simulation procedure. Black ticks indicate the number of samples, while blue
    ticks show time-frequency units for a choice of $\Delta t = 1/\sqrt{K}$ (see
    text for details). In other words, blue ticks are the coordinates in the
    complex plane that are implicit in the mathematical results of
    Sections~\ref{s:real} and \ref{s:complex}. The dashed region corresponds to the area used in subsequent simulations.}\label{fig:edge_effects}
\end{figure}

% discrete spectro
When the conditions above are satisfied, several phenomena occur in the limit of
infinite oversampling $N\to\infty$, which is equivalent to letting both the
duration $T$ and the sampling
frequency $F_s$ grow to infinity. In a dual manner, the resolution $(\Delta t, \Delta \nu)$ of the
discrete spectrogram tends to zero. The time-frequency extent $(\sigma_t, \sigma_\nu)$ of
the analysis window remains constant but is described by a number of samples that grows as $\sigma_t/\Delta t \propto\sqrt{N}$ while $\sigma_t/T\propto 1/\sqrt{N}\to 0$. The analysis window is thus more and more finely resolved, and
we become close to a continuous description.
% zeros
In parallel, the expected number of zeros in the spectrogram of the white noise is $F_sT/2$ and
tends to $\infty$ as $N$ grows. Therefore, assuming perfect zero detection, statistics such as Ripley's $K$ function
or the variance-stabilized $L$ functional statistic of Section~\ref{s:LAndK} can
be asymptotically perfectly well estimated. 

In practice, we defined a numerical
zero as a local minimum among its eight neighbouring bins, and found that the
number of zeros was consistent with what we expected from
Proposition~\ref{p:propertiesOfPlanarGAF}, even if we did not impose a threshold
on the value of the spectrogram at the local minimum. 

We leave this section on a mathematical note. In this section, we implicitly assumed that in the limit on an infinite
observation window and an infinite sampling rate, the discrete Fourier
transforms involved in the computation of the discrete spectrogram converge to
their continuous counterpart. For the sake of completeness, we mathematically
justify in what sense this convergence can be expected. With the notation of Section~\ref{s:real},
subdivide again $[0,T]$ into $N$ equal intervals and denote by $\chi_{n}$ the indicator of the $n$th
interval $[(n-1)\Delta t, n\Delta t]$. Let $P_{N,T}:\cS\rightarrow L^2$ attach
to a Schwartz function $f$ the ``sampled'' simple function $\sum_{n=1}^N
f(n)\chi_n$. Then $P_{N,T}f\rightarrow f$ in $L^2$ as $T$ and $N$ go to infinity
and $T/\sqrt{N}\rightarrow \alpha>0$, which is the setting described above in
this section. On the other hand, 
\begin{equation}
\label{e:dft}
\langle \fomega, P_{N,T}M_vT_u g\rangle = \sum_{n=1}^N\langle \fomega,\chi_n
\rangle e^{-2i\pi v n\Delta t} g(n\Delta t-u)
\end{equation}
is what we call the discrete STFT at $(u,v)$ of a realization of white
noise. Note that in distribution, $(\langle \fomega,\chi_n \rangle)_n$ is a
sequence of i.i.d. Gaussians with variance $\Delta t$. To see how \eqref{e:dft} is a good
approximation to our initial continuous STFT, we note that for all $u,v$,
\begin{eqnarray*}
\mathbb{E}_{\mu_1}\vert\langle \fomega,  M_{v}T_u g\rangle - \langle \fomega,
  P_{N,T} M_{v}T_u g\rangle\vert^2 &=& \mathbb{E}_{\mu_1}\vert\langle \fomega,
                                       M_{v}T_u g
  - P_{N,T} M_{v}T_u g\rangle\vert^2\\
&=& \Vert  M_{v}T_u g
  - P_{N,T} M_{v}T_u g \Vert^2_{L_2} \rightarrow 0.
\end{eqnarray*}


%% When a signal is present  %% ICI
\subsubsection{Zeros of signal plus noise}
\label{s:signalplusnoise}

When a signal is present, its specific scales destroy the scale invariance
property of Gaussian white noise and deprives us from any asymptotic regime in
our numerical simulations. Let $A_S$ denote the typical time and frequency area
occupied by the considered signal. The presence of this signal creates a region
of the spectrogram of size $A_S$ where a decrease in the number of zeros is
expected due to the positive amount of energy corresponding to the signal. This
decrease is clearly visible in the spectrograms of Figure~\ref{f:testPower} for
linear chirps with various $A_S$ and various signal-to-noise ratios (SNR).
% Basis for future tests
The approach proposed here to build statistical detection tests is based on this
intuition. To this purpose one needs to quantify how far the presence of a
signal can influence the statistics used in our tests so that we can maximize
this influence and the efficiency of the proposed test.

% F_s, T...number of zeros and \eta_t, \eta_\nu...
Given a sampling rate $F_s$ and a duration of observation $T$, the unit
intensity in Proposition~\ref{p:propertiesOfPlanarGAF} yields that the expected
number of zeros in the spectrogram of a real white noise is $F_s\cdot T/2 = N/2$, neglecting what happens at small frequencies close to the time axis. Note that this is independent of the width $(\sigma_t,\sigma_\nu)$ of the Gaussian
analysis window $g$. % since it is proportional to $1/\sigma_t\sigma_\nu=2\pi$. 
If one wants to increase the number of zeros in the spectrogram to get
better statistics, it is enough to increase either $F_s$ or $T$. However, the
expected decrease in the number of zeros due to the presence of a signal is of
the order of the area $A_S$, the finite time-frequency area $A_S$ corresponding
to the spectrogram of the signal alone. As a consequence, an excessive increase
in either $F_s$ and/or $T$ would result in an asymptotically complete dilution
of the influence of the signal on the considered statistics. Thus, our purpose
is to build statistics over one or more patches $P$ of the spectrogram of maximal area
$A_P=\eta_t\eta_\nu$ such that $A_S/A_P\simeq 1$. On one hand, a maximal area
$A_P$ is necessary to ensure that the estimate of the chosen statistic be as
accurate as possible (in particular in the presence of noise only, to take into
account as many zeros as possible and minimize the false positive detection
rate); on the other hand, this statistic will be more sensitive to the presence
of a signal if it mostly depends on the influence of the signal on the
distribution of zeros in the spectrogram (in particular, in the presence of signal, we maximize
the true positive detection rate). In practice, note that one can hope to detect only signals
such that $A_S\gg \sigma_t\sigma_\nu=1/2\pi$, which means signals with a
time-frequency support that affects more than $\sigma_t/\Delta t\cdot \sigma_\nu/\Delta \nu = K/2\pi$ samples of the spectrogram.

%The proposed approach cannot be efficient if tests are built on statistics that mostly depend on the noise and will be more efficient if these statistics are significantly 

%characterized by $(\sigma_t,\sigma_\nu = 1/2\pi\sigma_t)$. To ensure that the footprint of $g$ in the spectrogram is isotropic, we have imposed~(\ref{eq_isotropy}) which is equivalent to $F_s/\sigma_\nu = T/\sigma_t$.

%



% \subsection{Sampling GAFs}
% \label{s:sampling}
% The convergence of the truncated GAFs $p_N(z) = \sum_{k=0}^N
% a_kz^k$ in \eqref{e:symmetricPlanarGAF} and \eqref{e:planarGAF} is uniform on
% compact sets. By Hurwitz' theorem \cite{}, the zeros of the truncated GAF are a
% good approximation to those of the GAF \textcolor{red}{I will make this precise later.} 

% Simulating the zeros of truncated GAFs $p_N(z) = \sum_{k=0}^N
% a_kz^k$ can be done through various methods. We have found it more stable to
% diagonalize companion matrices, noting that the roots of the polynomial
% $p_N(z)$ are the eigenvalues of \textcolor{red}{XXX, also in practice, this approach
%   has some limit. FFT on discrete white noise seems to be the most scalable in
%   the end. Not sure this section is necessary anymore, actually.}.
