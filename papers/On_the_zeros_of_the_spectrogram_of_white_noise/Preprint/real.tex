In this section, we define real white noise, and examine the zeros of its
spectrogram. 
\subsection{Definitions}

To define white noise, we closely follow \cite[Chapter 2.1]{HOUZ10} through a
classical approach that does not require defining Brownian motion first. We denote by $\cS=\cS(\mathbb{R})$
the Schwartz space of rapidly decaying smooth complex-valued functions of a real
variable. The dual $\cS'=\cS'(\mathbb{R})$, equipped with the weak-star
topology, is the space of \emph{tempered distributions}. The topology yields the
Borel sigma-algebra $\mathcal{B}(\cS')$ on $\cS'$. Now, the Bochner-Minlos
theorem \citep[Theorem 2.1.1]{HOUZ10} states that there exists a unique probability
measure $\mu_1$ on $(\cS',\mathcal{B}(\cS'))$ such that 
\beq
\forall \phi\in \cS, \quad \mathbb{E}_{\mu_1}e^{i\langle \cdot,\phi \rangle} =
e^{-\frac{1}{2}\Vert \phi\Vert_2^2}.
\label{e:bochner}
\eeq
We call this measure white noise, and $(\cS',B(\cS'),\mu_1)$ the white noise
probability space. In particular, \eqref{e:bochner}
implies that for a random variable\footnote{We use the term \emph{random variable}, but it is also customary to call $\fomega$ a \emph{generalized random process} in the literature.} with distribution
$\mu_1$ and a set of real-valued orthonormal functions $\varphi_1,\dots,\varphi_p$ in $\cS$, the vector
$(\langle \fomega_1,\varphi_1 \rangle,\dots,\langle \fomega_p,\varphi_p \rangle)$
follows a real multivariate Gaussian, with mean zero and identity covariance matrix,
see \cite[Lemma 2.1.2]{HOUZ10}. This is in accordance with the usual heuristic of white noise having a Dirac delta covariance function.

Let $\fomega$ be a random variable with distribution $\mu_1$.
If $g\in\cS$, then $(u,v)\mapsto M_v T_u g$ is in $\cS$, so that we can define
the STFT of $\fomega$ as the random function
$$ u,v\mapsto \langle \fomega,  M_v T_u g \rangle.$$
From now on, we restrict ourselves to the Gaussian window $g(x) =
2^{1/4}e^{-\pi x^2}$, normalized so that $\Vert g\Vert_2 = 1$. We are interested
in defining and studying the zeros of the spectrogram
\beq
\label{e:spectrogram}
S:u,v\mapsto \vert \langle \fomega,  M_v T_u g \rangle\vert^2.
\eeq

\subsection{Characterizing the zeros}
\label{e:zerosSymmetricGAF}
We work in two steps: in Proposition~\ref{p:series}, we identify each value
$S(u,v)$ in \eqref{e:spectrogram} as a limit in $L^2(\mu_1)$, and we then
show in Proposition~\ref{p:symmetricPlanarGAF} that the resulting random field
defines an entire function, the zeros of which are known.
\begin{prop}
Let $u,v\in\mathbb{R}^2$, and write $z=u+iv\in\mathbb{C}$. Then
\beq
\langle \fomega,  M_v T_u g \rangle = \sqrt{\pi} e^{i\pi uv}e^{-\frac{\pi}{2}\vert z\vert^2} \sum_{k=0}^{\infty}\langle \fomega,h_k
\rangle \frac{\pi^{k/2}z^k}{\sqrt{k!}} 
\label{e:series}
\eeq
where $(h_k)$ denote the orthonormal Hermite functions \cite[Section 2.2.1]{HOUZ10}, and convergence is in $L^2(\mu_1)$.
\label{p:series}
\end{prop}
\begin{rk}
\label{r:zeros}
Note that in Proposition~\ref{p:series}, $u$ and $v$ are fixed, and the equality
is a limit in $L^2(\mu_1)$. It is still too early to identify the zeros of
the left-hand side to the zeros of the right-hand side.
\end{rk}
\begin{rk}
Note that our choice of the window $g(x) = 2^{1/4}e^{-\pi x^2}$ is made to
simplify expressions. The proof of Proposition~\ref{p:series}, along with
Sections~\ref{s:hermite} and \ref{s:bargmann}, immediately yield that for a non-unit Gaussian window $g_a(x) \propto \exp(-\pi
a^2 x^2)$, Proposition~\ref{p:series} is unchanged, provided that $z$ is defined
as $z = au + iv/a$ and a constant is prepended to the RHS of
\eqref{e:series}. In other words, given a particular value of $a$, it is always possible to
dilate/squeeze the time-frequency axes to obtain the results detailed
here for $a = 1$. 
\end{rk}
\begin{proof}
Let $u,v\in\mathbb{R}^2$. Decomposing $M_v T_u g$ in the Hermite basis $(h_k)$ of
$L^2(\mathbb{R})$, it comes
\begin{eqnarray}
\langle \fomega, M_v T_u g \rangle &=& \sum_{k=0}^\infty \langle \fomega,h_k
  \rangle \langle  M_v T_u g,h_k\rangle\nonumber\\
&=&  \sum_{k=0}^\infty \langle \fomega,h_k
  \rangle \overline{V_g(h_k)(u,v)}
\label{e:decomp}
\end{eqnarray}
where the limits are in $L^2(\mu_1)$. The STFT of Hermite functions is
well-known, see e.g. the proof of \cite[Proposition 3.4.4]{Gro01} or our Section~\ref{s:bargmann}, and it reads
\beq 
V_g(h_k)(u,v) = e^{-i\pi
  uv}e^{-\frac{\pi}{2}(u^2+v^2)}\frac{\pi^{k/2}}{\sqrt{k!}}(u-iv)^k.
\label{e:HermiteSTFT}
\eeq
Plugging \eqref{e:HermiteSTFT} into \eqref{e:decomp} yields the result.
\end{proof}

Now we focus on the regularity of the right-hand side of \eqref{e:series}.
\begin{prop}
\label{p:symmetricPlanarGAF}
The random series
\beq
\label{e:symmetricPlanarGAF}
\sum_{k=0}^{\infty}\langle \fomega,h_k
\rangle \frac{\pi^{k/2}z^k}{\sqrt{k!}}
\eeq
$\mu_1$-almost surely defines an entire function. 
\end{prop}
\begin{proof}
By \cite[Lemma 2.1.2]{HOUZ10}, ($\langle \fomega,h_k
\rangle)_{k\geq 0}$ are i.i.d. unit real Gaussians. We then apply the first part
of \cite[Lemma 2.2.3]{HKPV09}.
\end{proof}
Since both $L^2$ and almost sure convergence imply convergence in probability,
$L^2$ and almost sure limits have to be the same. In particular,
Propositions~\ref{p:series} and \ref{p:symmetricPlanarGAF} together yield that the
distribution of the zeros of the spectrogram $S$ in \eqref{e:spectrogram} is
the same as the distribution of the zeros of the random entire function
\eqref{e:symmetricPlanarGAF}. This answers Remark~\ref{r:zeros}. In particular,
we now know that the zeros of $S$ are isolated.

The entire function in \eqref{e:symmetricPlanarGAF} is called the
\emph{symmetric planar Gaussian analytic function} (GAF), and a few of its properties are known
\citep{Fel13}. However, its zeros do not define a stationary point process. In
particular, a portion of the zeros concentrate on the real axis, see
Figure~\ref{f:symmetricPlanarGAF}. Intuitively, one can approximate the zeros of
\eqref{e:symmetricPlanarGAF} by the zeros of the random polynomial obtained from
truncating the series. The resulting polynomial has real coefficients, and it is
thus expected to have real zeros as well as pairs of conjugate complex zeros. As a side note, the number of real zeros is a
topic of study on its own, see e.g. \citep{ScMa08}. 

\begin{figure}
\subfigure[Real white noise/symmetric GAF]{
\includegraphics[width=\twofig]{\figdir/realWGNdist.pdf}
\label{f:symmetricPlanarGAF}
}
\subfigure[Complex white noise/planar GAF]{
\includegraphics[width=\twofig]{\figdir/complexWGN/complexWGNdist.pdf}
\label{f:complexPlanarGAF}
}
\caption{The spectrogram of (a) a realization of real white noise, and (b) a
  realization of complex white noise. The right and top plots on each panel show
marginal histograms, superimposed with the theoretical marginal density, see
text for details.}
\label{f:GAFs}
\end{figure}

Coming back to our problem of detecting signals, this non-stationarity
makes it uneasy to approach via traditional spatial statistics techniques, which
often assume some degree of stationarity. However, there is a stationary point
process that is a good approximation for the zeros of the symmetric planar GAF,
and that has been studied in depth. This point process is the zeros of the
\emph{planar GAF}, the entire function corresponding to the STFT of complex
white noise. 
