\documentclass[a4paper,11pt]{article}
\usepackage[english]{babel}
\usepackage{natbib}
\usepackage{anysize}
\marginsize{3cm}{3cm}{3cm}{3cm}
\usepackage{amsfonts,amsmath,amssymb,amsthm,bbm}
\usepackage{graphicx} 
%\usepackage{bmpsize}
\usepackage{color}
\usepackage{subfigure}

\usepackage[colorlinks=True, citecolor=blue]{hyperref}
%\usepackage{amssymb,amsthm,amsmath,bbm}
%\usepackage{mathrsfs}

\theoremstyle{plain}
\newtheorem{prop}{Proposition}
\theoremstyle{remark}
\newtheorem{rk}{Remark}

\def\beq{\begin{equation}}
\def\eeq{\end{equation}}
\def\fomega{\xi}
\def\cS{\mathcal{S}}
\def\cF{\mathcal{F}}
\newcommand{\IND}{\ensuremath{\mathbbm{1}}}
\def\figdir{figures}
\def\twofig{.48\textwidth}
\def\threefig{.32\textwidth}
\def\bom{{\boldsymbol{\fomega}}}
\def\sym{\text{sym}}
\def\XXX{\textcolor{red}{XXX}}
\def\th{\text{th}}
\def\textdots{\text{...}}
\title{On the zeros of the spectrogram of
  white noise}
\newcommand{\AFFicrr}{\affiliation{Kamioka Observatory, Institute for Cosmic Ray Research, University of Tokyo, Kamioka, Gifu 506-1205, Japan}}
\newcommand{\AFFkashiwa}{\affiliation{Research Center for Cosmic Neutrinos, Institute for Cosmic Ray Research, University of Tokyo, Kashiwa, Chiba 277-8582, Japan}}
\newcommand{\AFFicrronly}{\affiliation{Institute for Cosmic Ray Research, University of Tokyo, Kashiwa, Chiba 277-8582, Japan}}
\newcommand{\AFFipmu}{\affiliation{Kavli Institute for the Physics and
Mathematics of the Universe (WPI), The University of Tokyo Institutes for Advanced Study,
University of Tokyo, Kashiwa, Chiba 277-8583, Japan }}
\newcommand{\AFFmad}{\affiliation{Department of Theoretical Physics, University Autonoma Madrid, 28049 Madrid, Spain}}
\newcommand{\AFFubc}{\affiliation{Department of Physics and Astronomy, University of British Columbia, Vancouver, BC, V6T1Z4, Canada}}
\newcommand{\AFFbu}{\affiliation{Department of Physics, Boston University, Boston, MA 02215, USA}}
\newcommand{\AFFuci}{\affiliation{Department of Physics and Astronomy, University of California, Irvine, Irvine, CA 92697-4575, USA }}
\newcommand{\AFFcsu}{\affiliation{Department of Physics, California State University, Dominguez Hills, Carson, CA 90747, USA}}
\newcommand{\AFFcnm}{\affiliation{Institute for Universe and Elementary Particles, Chonnam National University, Gwangju 61186, Korea}}
\newcommand{\AFFduke}{\affiliation{Department of Physics, Duke University, Durham NC 27708, USA}}
\newcommand{\AFFfukuoka}{\affiliation{Junior College, Fukuoka Institute of Technology, Fukuoka, Fukuoka 811-0295, Japan}}
\newcommand{\AFFgifu}{\affiliation{Department of Physics, Gifu University, Gifu, Gifu 501-1193, Japan}}
\newcommand{\AFFgist}{\affiliation{GIST College, Gwangju Institute of Science and Technology, Gwangju 500-712, Korea}}
\newcommand{\AFFuh}{\affiliation{Department of Physics and Astronomy, University of Hawaii, Honolulu, HI 96822, USA}}
\newcommand{\AFFicl}{\affiliation{Department of Physics, Imperial College London , London, SW7 2AZ, United Kingdom }}
\newcommand{\AFFkek}{\affiliation{High Energy Accelerator Research Organization (KEK), Tsukuba, Ibaraki 305-0801, Japan }}
\newcommand{\AFFkobe}{\affiliation{Department of Physics, Kobe University, Kobe, Hyogo 657-8501, Japan}}
\newcommand{\AFFkyoto}{\affiliation{Department of Physics, Kyoto University, Kyoto, Kyoto 606-8502, Japan}}
\newcommand{\AFFliv}{\affiliation{Department of Physics, University of Liverpool, Liverpool, L69 7ZE, United Kingdom}}
\newcommand{\AFFmiyagi}{\affiliation{Department of Physics, Miyagi University of Education, Sendai, Miyagi 980-0845, Japan}}
\newcommand{\AFFnagoya}{\affiliation{Institute for Space-Earth Environmental Research, Nagoya University, Nagoya, Aichi 464-8602, Japan}}
\newcommand{\AFFkmi}{\affiliation{Kobayashi-Maskawa Institute for the Origin of Particles and the Universe, Nagoya University, Nagoya, Aichi 464-8602, Japan}}
\newcommand{\AFFpol}{\affiliation{National Centre For Nuclear Research, 02-093 Warsaw, Poland}}
\newcommand{\AFFsuny}{\affiliation{Department of Physics and Astronomy, State University of New York at Stony Brook, NY 11794-3800, USA}}
\newcommand{\AFFokayama}{\affiliation{Department of Physics, Okayama University, Okayama, Okayama 700-8530, Japan }}
\newcommand{\AFFosaka}{\affiliation{Department of Physics, Osaka University, Toyonaka, Osaka 560-0043, Japan}}
\newcommand{\AFFox}{\affiliation{Department of Physics, Oxford University, Oxford, OX1 3PU, United Kingdom}}
\newcommand{\AFFqmul}{\affiliation{School of Physics and Astronomy, Queen Mary University of London, London, E1 4NS, United Kingdom}}
\newcommand{\AFFregina}{\affiliation{Department of Physics, University of Regina, 3737 Wascana Parkway, Regina, SK, S4SOA2, Canada}}
\newcommand{\AFFseoul}{\affiliation{Department of Physics, Seoul National University, Seoul 151-742, Korea}}
\newcommand{\AFFsheff}{\affiliation{Department of Physics and Astronomy, University of Sheffield, S3 7RH, Sheffield, United Kingdom}}
\newcommand{\AFFshizuokasc}{\affiliation{Department of Informatics in
Social Welfare, Shizuoka University of Welfare, Yaizu, Shizuoka, 425-8611, Japan}}
\newcommand{\AFFstfc}{\affiliation{STFC, Rutherford Appleton Laboratory, Harwell Oxford, and Daresbury Laboratory, Warrington, OX11 0QX, United Kingdom}}
\newcommand{\AFFskk}{\affiliation{Department of Physics, Sungkyunkwan University, Suwon 440-746, Korea}}
\newcommand{\AFFtokyo}{\affiliation{The University of Tokyo, Bunkyo, Tokyo 113-0033, Japan }}
\newcommand{\AFFtodai}{\affiliation{Department of Physics, University of Tokyo, Bunkyo, Tokyo 113-0033, Japan }}
\newcommand{\AFFtit}{\affiliation{Department of Physics,Tokyo Institute of Technology, Meguro, Tokyo 152-8551, Japan }}
\newcommand{\AFFtus}{\affiliation{Department of Physics, Faculty of Science and Technology, Tokyo University of Science, Noda, Chiba 278-8510, Japan }}
\newcommand{\AFFtoronto}{\affiliation{Department of Physics, University of Toronto, ON, M5S 1A7, Canada }}
\newcommand{\AFFtriumf}{\affiliation{TRIUMF, 4004 Wesbrook Mall, Vancouver, BC, V6T2A3, Canada }}
\newcommand{\AFFtokai}{\affiliation{Department of Physics, Tokai University, Hiratsuka, Kanagawa 259-1292, Japan}}
\newcommand{\AFFtsinghua}{\affiliation{Department of Engineering Physics, Tsinghua University, Beijing, 100084, China}}
\newcommand{\AFFynu}{\affiliation{Department of Physics, Yokohama National University, Yokohama, Kanagawa, 240-8501, Japan}}
\newcommand{\AFFllr}{\affiliation{Ecole Polytechnique, IN2P3-CNRS, Laboratoire Leprince-Ringuet, F-91120 Palaiseau, France }}
\newcommand{\AFFbari}{\affiliation{ Dipartimento Interuniversitario di Fisica, INFN Sezione di Bari and Universit\`a e Politecnico di Bari, I-70125, Bari, Italy}}
\newcommand{\AFFnapoli}{\affiliation{Dipartimento di Fisica, INFN Sezione di Napoli and Universit\`a di Napoli, I-80126, Napoli, Italy}}
\newcommand{\AFFroma}{\affiliation{INFN Sezione di Roma and Universit\`a di Roma ``La Sapienza'', I-00185, Roma, Italy}}
\newcommand{\AFFpadova}{\affiliation{Dipartimento di Fisica, INFN Sezione di Padova and Universit\`a di Padova, I-35131, Padova, Italy}}
\newcommand{\AFFkeio}{\affiliation{Department of Physics, Keio University, Yokohama, Kanagawa, 223-8522, Japan}}
\newcommand{\AFFwinnipeg}{\affiliation{Department of Physics, University of Winnipeg, MB R3J 3L8, Canada }}
\newcommand{\AFFkcl}{\affiliation{Department of Physics, King's College London, London, WC2R 2LS, UK }}
\newcommand{\AFFwarwick}{\affiliation{Department of Physics, University of Warwick, Coventry, CV4 7AL, UK }}
\newcommand{\AFFral}{\affiliation{Rutherford Appleton Laboratory, Harwell, Oxford, OX11 0QX, UK }}
\newcommand{\AFFwu}{\affiliation{Faculty of Physics, University of Warsaw, Warsaw, 02-093, Poland }}
\newcommand{\AFFbcit}{\affiliation{Department of Physics, British Columbia Institute of Technology, Burnaby, BC, V5G 3H2, Canada }}
\newcommand{\AFFtohoku}{\affiliation{Department of Physics, Faculty of Science, Tohoku University, Sendai, Miyagi, 980-8578, Japan }}


\AFFicrr
\AFFkashiwa
\AFFicrronly
\AFFmad
\AFFbu
\AFFbcit
\AFFuci
\AFFcsu
\AFFcnm
\AFFduke
\AFFllr
\AFFfukuoka
\AFFgifu
\AFFgist
\AFFuh
\AFFicl
\AFFbari
\AFFnapoli
\AFFpadova
\AFFroma
\AFFkeio
\AFFkek
\AFFkcl
\AFFkobe
\AFFkyoto
\AFFliv
\AFFmiyagi
\AFFnagoya
\AFFkmi
\AFFpol
\AFFsuny
\AFFokayama
\AFFox
\AFFral
\AFFseoul
\AFFsheff
\AFFshizuokasc
\AFFstfc
\AFFskk
\AFFtokai
\AFFtokyo
\AFFtodai
\AFFipmu
\AFFtit
\AFFtus
\AFFtoronto
\AFFtriumf
\AFFtsinghua
\AFFwu
\AFFwarwick
\AFFwinnipeg
\AFFynu

% First authors
\author{S.~Locke} 
\AFFuci
\author{A.~Coffani}
\AFFllr
%%%%%%%%%%%%%%%%%%%%%%%%%%%%%%%%%%%%%%%%%%%%%%%%%%%%%%%%%%%%%%%%%%%%
%ICRR
\author{K.~Abe}
\AFFicrr
\AFFipmu
\author{C.~Bronner}
\AFFicrr
\author{Y.~Hayato}
\AFFicrr
\AFFipmu
\author{M.~Ikeda}
\author{S.~Imaizumi}
\AFFicrr
\author{H.~Ito}
\AFFicrr 
\author{J.~Kameda}
\AFFicrr
\AFFipmu
\author{Y.~Kataoka}
\AFFicrr
\author{M.~Miura} 
\author{S.~Moriyama} 
\AFFicrr
\AFFipmu
\author{Y.~Nagao} 
\AFFicrr
\author{M.~Nakahata}
\AFFicrr
\AFFipmu
\author{Y.~Nakajima}
\AFFicrr
\AFFipmu
\author{S.~Nakayama}
\AFFicrr
\AFFipmu
\author{T.~Okada}
\author{K.~Okamoto}
\author{A.~Orii}
\author{G.~Pronost}
\AFFicrr
\author{H.~Sekiya} 
\author{M.~Shiozawa}
\AFFicrr
\AFFipmu 
\author{Y.~Sonoda}
\author{Y.~Suzuki} 
\AFFicrr
\author{A.~Takeda}
\AFFicrr
\AFFipmu
\author{Y.~Takemoto}
\author{A.~Takenaka}
\AFFicrr 
\author{H.~Tanaka}
\AFFicrr 
\author{T.~Yano}
\AFFicrr 
\author{K.~Hirade}
\author{Y.~Kanemura}
\author{S.~Miki}
\author{S.~Watabe}
\AFFicrr
%%%%%%%%%%%%%%%%%%%%%%%%%%%%%%%%%%%%%%%%%%%%%%%%%%%%%%%%%%%%%%%%%%%%%
%%Kashiwa
\author{S.~Han} 
\AFFkashiwa
\author{T.~Kajita} 
\AFFkashiwa
\AFFipmu
\author{K.~Okumura}
\AFFkashiwa
\AFFipmu
\author{T.~Tashiro}
\author{J.~Xia}
\author{X.~Wang}
\AFFkashiwa

%%%%%%%%%%%%%%%%%%%%%%%%%%%%%%%%%%%%%%%%%%%%%%%%%%%%%%%%%%%%%%%%%%%%%
%%Kashiwa2
\author{G.~D.~Megias}
\AFFicrronly
%%%%%%%%%%%%%%%%%%%%%%%%%%%%%%%%%%%%%%%%%%%%%%%%%%%%%%%%%%%%%%%%%%%%%
%% Madrid
\author{D.~Bravo-Bergu\~{n}o}
\author{L.~Labarga}
\author{Ll.~Marti}
\author{B.~Zaldivar}
\AFFmad
%%%%%%%%%%%%%%%%%%%%%%%%%%%%%%%%%%%%%%%%%%%%%%%%%%%%%%%%%%%%%%%%%%%%%%
%% BCIT
\author{B.~W.~Pointon}
\AFFbcit
\AFFtriumf

%%%%%%%%%%%%%%%%%%%%%%%%%%%%%%%%%%%%%%%%%%%%%%%%%%%%%%%%%%%%%%%%%%%%%
%%Boston U
\author{F.~d.~M.~Blaszczyk}
\AFFbu
\author{E.~Kearns}
\AFFbu
\AFFipmu
\author{J.~L.~Raaf}
\AFFbu
\author{J.~L.~Stone}
\AFFbu
\AFFipmu
\author{L.~Wan}
\AFFbu
\author{T.~Wester}
\AFFbu
%%%%%%%%%%%%%%%%%%%%%%%%%%%%%%%%%%%%%%%%%%%%%%%%%%%%%%%%%%%%%%%%%%%%%
%%%%%%%%%%%%%%%%%%%%%%%%%%%%%%%%%%%%%%%%%%%%%%%%%%%%%%%%%%%%%%%%%%%%%
%%Irvine
\author{J.~Bian}
\author{N.~J.~Griskevich}
\author{W.~R.~Kropp}
\altaffiliation{Deceased.}
\author{S.~Mine} 
\author{A.~Yankelevic}
\AFFuci
\author{M.~B.~Smy}
\author{H.~W.~Sobel} 
\AFFuci
\AFFipmu
\author{V.~Takhistov}
\AFFuci
\AFFipmu

%%%%%%%%%%%%%%%%%%%%%%%%%%%%%%%%%%%%%%%%%%%%%%%%%%%%%%%%%%%%%%%%%%%%%
%%CSU
\author{J.~Hill}
\AFFcsu

%%%%%%%%%%%%%%%%%%%%%%%%%%%%%%%%%%%%%%%%%%%%%%%%%%%%%%%%%%%%%%%%%%%%%
%%Chonnam
\author{J.~Y.~Kim}
\author{I.~T.~Lim}
\author{R.~G.~Park}
\AFFcnm

%%%%%%%%%%%%%%%%%%%%%%%%%%%%%%%%%%%%%%%%%%%%%%%%%%%%%%%%%%%%%%%%%%%%%
%%Duke
\author{B.~Bodur}
\AFFduke
\author{K.~Scholberg}
\author{C.~W.~Walter}
\AFFduke
\AFFipmu

%%%%%%%%%%%%%%%%%%%%%%%%%%%%%%%%%%%%%%%%%%%%%%%%%%%%%%%%%%%%%%%%%%%%%
%%LLR
\author{L.~Bernard}
\author{O.~Drapier}
\author{S.~El Hedri}
\author{A.~Giampaolo}
\author{M.~Gonin}
\author{Th.~A.~Mueller}
\author{P.~Paganini}
\author{B.~Quilain}
\author{A.~D.~Santos}
\AFFllr

%%%%%%%%%%%%%%%%%%%%%%%%%%%%%%%%%%%%%%%%%%%%%%%%%%%%%%%%%%%%%%%%%%%%%
%%Fukuoka
\author{T.~Ishizuka}
\AFFfukuoka

%%%%%%%%%%%%%%%%%%%%%%%%%%%%%%%%%%%%%%%%%%%%%%%%%%%%%%%%%%%%%%%%%%%%%
%%Gifu U
\author{T.~Nakamura}
\AFFgifu

%%%%%%%%%%%%%%%%%%%%%%%%%%%%%%%%%%%%%%%%%%%%%%%%%%%%%%%%%%%%%%%%%%%%%
%%Gwangju
\author{J.~S.~Jang}
\AFFgist

%%%%%%%%%%%%%%%%%%%%%%%%%%%%%%%%%%%%%%%%%%%%%%%%%%%%%%%%%%%%%%%%%%%%%
%%Hawaii U
\author{J.~G.~Learned} 
\AFFuh

%%%%%%%%%%%%%%%%%%%%%%%%%%%%%%%%%%%%%%%%%%%%%%%%%%%%%%%%%%%%%%%%%%%%%
%%ICL
\author{L.~H.~V.~Anthony}
\author{A.~A.~Sztuc} 
\author{Y.~Uchida}
\author{D.~~Martin}
\author{M.~Scott}
\AFFicl

%%%%%%%%%%%%%%%%%%%%%%%%%%%%%%%%%%%%%%%%%%%%%%%%%%%%%%%%%%%%%%%%%%%%%
%%BARI
\author{V.~Berardi}
\author{M.~G.~Catanesi}
\author{E.~Radicioni}
\AFFbari

%%%%%%%%%%%%%%%%%%%%%%%%%%%%%%%%%%%%%%%%%%%%%%%%%%%%%%%%%%%%%%%%%%%%%
%%NAPOLI
\author{N.~F.~Calabria}
\author{L.~N.~Machado}
\author{G.~De Rosa}
\AFFnapoli

%%%%%%%%%%%%%%%%%%%%%%%%%%%%%%%%%%%%%%%%%%%%%%%%%%%%%%%%%%%%%%%%%%%%%
%%PADOVA
\author{G.~Collazuol}
\author{F.~Iacob}
\author{M.~Lamoureux}
\author{N.~Ospina}
\author{M.~Mattiazzi}
\AFFpadova

%%%%%%%%%%%%%%%%%%%%%%%%%%%%%%%%%%%%%%%%%%%%%%%%%%%%%%%%%%%%%%%%%%%%%
%%Roma
\author{L.\,Ludovici}
\AFFroma

%%%%%%%%%%%%%%%%%%%%%%%%%%%%%%%%%%%%%%%%%%%%%%%%%%%%%%%%%%%%%%%%%%%%%
%%Keio
\author{Y.~Nishimura}
\author{Y.~Maewaka}
\AFFkeio

 
%%%%%%%%%%%%%%%%%%%%%%%%%%%%%%%%%%%%%%%%%%%%%%%%%%%%%%%%%%%%%%%%%%%%%
%%KEK
\author{S.~Cao}
\author{M.~Friend}
\author{T.~Hasegawa} 
\author{T.~Ishida} 
\author{T.~Kobayashi} 
\author{M.~Jakkapu}
\author{T.~Matsubara}
\author{T.~Nakadaira} 
\AFFkek 
\author{K.~Nakamura}
\AFFkek 
\AFFipmu
\author{Y.~Oyama} 
\author{K.~Sakashita} 
\author{T.~Sekiguchi} 
\author{T.~Tsukamoto}
\AFFkek 

%%%%%%%%%%%%%%%%%%%%%%%%%%%%%%%%%%%%%%%%%%%%%%%%%%%%%%%%%%%%%%%%%%%%%
%%Kobe U
\author{Y.~Nakano}
\author{T.~Shiozawa}
\AFFkobe
\author{A.~T.~Suzuki}
\AFFkobe
\author{Y.~Takeuchi}
\AFFkobe
\AFFipmu
\author{S.~Yamamoto}
\AFFkobe
\author{Y.~Kotsar}
\author{H.~Ozaki}
\AFFkobe

%%%%%%%%%%%%%%%%%%%%%%%%%%%%%%%%%%%%%%%%%%%%%%%%%%%%%%%%%%%%%%%%%%%%%
%%Kyoto
\author{A.~Ali}
\author{Y.~Ashida}
\author{J.~Feng}
\author{S.~Hirota}
\author{A.~K.~Ichikawa}
\author{T.~Kikawa}
\author{M.~Mori}
\AFFkyoto
\author{T.~Nakaya}
\AFFkyoto
\AFFipmu
\author{R.~A.~Wendell}
\AFFkyoto
\AFFipmu
\author{K.~Yasutome}
\AFFkyoto

%%%%%%%%%%%%%%%%%%%%%%%%%%%%%%%%%%%%%%%%%%%%%%%%%%%%%%%%%%%%%%%%%%%%%
%%Liverpool
\author{P.~Fernandez}
\author{N.~McCauley}
\author{P.~Mehta}
\author{K.~M.~Tsui}
\AFFliv

%%%%%%%%%%%%%%%%%%%%%%%%%%%%%%%%%%%%%%%%%%%%%%%%%%%%%%%%%%%%%%%%%%%%%
%%Miyagi
\author{Y.~Fukuda}
\AFFmiyagi

%%%%%%%%%%%%%%%%%%%%%%%%%%%%%%%%%%%%%%%%%%%%%%%%%%%%%%%%%%%%%%%%%%%%%
%%Nagoya
\author{Y.~Itow}
\AFFnagoya
\AFFkmi
\author{H.~Menjo}
\author{T.~Niwa}
\author{K.~Sato}
\AFFnagoya
\author{M.~Tsukada}
\AFFnagoya

%%%%%%%%%%%%%%%%%%%%%%%%%%%%%%%%%%%%%%%%%%%%%%%%%%%%%%%%%%%%%%%%%%%%%
%% POLAND
\author{P.~Mijakowski}
\author{J.~Lagoda}
\author{S.~M.~Lakshmi}
\author{J.~Zalipska}
\AFFpol

%%%%%%%%%%%%%%%%%%%%%%%%%%%%%%%%%%%%%%%%%%%%%%%%%%%%%%%%%%%%%%%%%%%%%
%%SUNY
\author{C.~K.~Jung}
\author{C.~Vilela}
\author{M.~J.~Wilking}
\author{C.~Yanagisawa}
\altaffiliation{also at BMCC/CUNY, Science Department, New York, New York, 1007, USA.}
\author{J.~Jiang}
\AFFsuny

%%%%%%%%%%%%%%%%%%%%%%%%%%%%%%%%%%%%%%%%%%%%%%%%%%%%%%%%%%%%%%%%%%%%%
%%Okayama U
\author{K.~Hagiwara}
\author{M.~Harada}
\author{T.~Horai}
\author{H.~Ishino}
\author{S.~Ito}
\AFFokayama
\author{Y.~Koshio}
\AFFokayama
\AFFipmu
\author{W.~Ma}
\author{N.~Piplani}
\author{S.~Sakai}
\author{H.~Kitagawa}
\AFFokayama

%%%%%%%%%%%%%%%%%%%%%%%%%%%%%%%%%%%%%%%%%%%%%%%%%%%%%%%%%%%%%%%%%%%%%
%%Oxford
\author{G.~Barr}
\author{D.~Barrow}
\AFFox
\author{L.~Cook}
\AFFox
\AFFipmu
\author{A.~Goldsack}
\AFFox
\AFFipmu
\author{S.~Samani}
\AFFox
\author{D.~Wark}
\AFFox
\AFFstfc

%%%%%%%%%%%%%%%%%%%%%%%%%%%%%%%%%%%%%%%%%%%%%%%%%%%%%%%%%%%%%%%%%%%%%
%%RAL
\author{F.~Nova}
\AFFral

%%%%%%%%%%%%%%%%%%%%%%%%%%%%%%%%%%%%%%%%%%%%%%%%%%%%%%%%%%%%%%%%%%%%%
%%KCL
\author{T.~Boschi}
\author{F.~Di Lodovico}
\author{M.~Taani}
\author{S.~Zsoldos}
\author{J.~Gao}
\author{J.~Migenda}
\AFFkcl


%%%%%%%%%%%%%%%%%%%%%%%%%%%%%%%%%%%%%%%%%%%%%%%%%%%%%%%%%%%%%%%%%%%%%
%%Seoul
\author{J.~Y.~Yang}
\AFFseoul

%%%%%%%%%%%%%%%%%%%%%%%%%%%%%%%%%%%%%%%%%%%%%%%%%%%%%%%%%%%%%%%%%%%%%
%%Sheffield
\author{S.~J.~Jenkins}
\author{M.~Malek}
\author{J.~M.~McElwee}
\author{O.~Stone}
\author{M.~D.~Thiesse}
\author{L.~F.~Thompson}
\AFFsheff

%%%%%%%%%%%%%%%%%%%%%%%%%%%%%%%%%%%%%%%%%%%%%%%%%%%%%%%%%%%%%%%%%%%%%
%%Shizuoka Seika College
\author{H.~Okazawa}
\AFFshizuokasc

%%%%%%%%%%%%%%%%%%%%%%%%%%%%%%%%%%%%%%%%%%%%%%%%%%%%%%%%%%%%%%%%%%%%%
%%Tohoku
\author{K.~Nakamura}
\AFFtohoku
%%%%%%%%%%%%%%%%%%%%%%%%%%%%%%%%%%%%%%%%%%%%%%%%%%%%%%%%%%%%%%%%%%%%%
%%SungKyunKwan
\author{S.~B.~Kim}
\author{I.~Yu}
\author{J.~W.~Seo}
\AFFskk

%%%%%%%%%%%%%%%%%%%%%%%%%%%%%%%%%%%%%%%%%%%%%%%%%%%%%%%%%%%%%%%%%%%%%
%%Tokai U
\author{K.~Nishijima}
\AFFtokai

%%%%%%%%%%%%%%%%%%%%%%%%%%%%%%%%%%%%%%%%%%%%%%%%%%%%%%%%%%%%%%%%%%%%%
%%Tokyo
\author{M.~Koshiba}
\altaffiliation{Deceased.}
\AFFtokyo

%%%%%%%%%%%%%%%%%%%%%%%%%%%%%%%%%%%%%%%%%%%%%%%%%%%%%%%%%%%%%%%%%%%%%
%%Tokyo, Department of Physics
\author{K.~Iwamoto}
\author{N.~Ogawa}
\AFFtodai
\author{M.~Yokoyama}
\AFFtodai
\AFFipmu


%%%%%%%%%%%%%%%%%%%%%%%%%%%%%%%%%%%%%%%%%%%%%%%%%%%%%%%%%%%%%%%%%%%%%
%%IPMU

\author{K.~Martens}
\AFFipmu
\author{M.~R.~Vagins}
\AFFipmu
\AFFuci
\author{K.~Nakagiri}
\AFFipmu
\AFFtokyo
%%%%%%%%%%%%%%%%%%%%%%%%%%%%%%%%%%%%%%%%%%%%%%%%%%%%%%%%%%%%%%%%%%%%%
%%TIT
\author{M.~Kuze}
\author{S.~Izumiyama}
\author{T.~Yoshida}
\AFFtit

%%%%%%%%%%%%%%%%%%%%%%%%%%%%%%%%%%%%%%%%%%%%%%%%%%%%%%%%%%%%%%%%%%%%%
%%TUS
\author{M.~Inomoto}
\author{M.~Ishitsuka}
\author{R.~Matsumoto}
\author{K.~Ohta}
\author{M.~Shinoki}
\author{T.~Suganuma}
\author{T.~Kinoshita}
\AFFtus

%%%%%%%%%%%%%%%%%%%%%%%%%%%%%%%%%%%%%%%%%%%%%%%%%%%%%%%%%%%%%%%%%%%%%
%%Toronto
\author{J.~F.~Martin}
\author{H.~A.~Tanaka}
\author{T.~Towstego}
\AFFtoronto

%%%%%%%%%%%%%%%%%%%%%%%%%%%%%%%%%%%%%%%%%%%%%%%%%%%%%%%%%%%%%%%%%%%%%
%%Triumf
\author{R.~Akutsu}
\author{M.~Hartz}
\author{A.~Konaka}
\author{P.~de Perio}
\author{N.~W.~Prouse}
\AFFtriumf

%%%%%%%%%%%%%%%%%%%%%%%%%%%%%%%%%%%%%%%%%%%%%%%%%%%%%%%%%%%%%%%%%%%%%
%%Tshinghua U
\author{S.~Chen}
\author{B.~D.~Xu}
\author{Y.~Zhang}
\AFFtsinghua

%%%%%%%%%%%%%%%%%%%%%%%%%%%%%%%%%%%%%%%%%%%%%%%%%%%%%%%%%%%%%%%%%%%%%
%%Warsaw
\author{M.~Posiadala-Zezula}
\AFFwu

%%%%%%%%%%%%%%%%%%%%%%%%%%%%%%%%%%%%%%%%%%%%%%%%%%%%%%%%%%%%%%%%%%%%%
%%Warwick
\author{B.~Richards}
\AFFwarwick

%%%%%%%%%%%%%%%%%%%%%%%%%%%%%%%%%%%%%%%%%%%%%%%%%%%%%%%%%%%%%%%%%%%%%
%%Winnipeg
\author{B.~Jamieson}
\author{J.~Walker}
\AFFwinnipeg

%%%%%%%%%%%%%%%%%%%%%%%%%%%%%%%%%%%%%%%%%%%%%%%%%%%%%%%%%%%%%%%%%%%%%
%%Yokohama
\author{A.~Minamino}
\author{K.~Okamoto}
\author{G.~Pintaudi}
\author{R.~Sasaki}
\AFFynu
\date{}

\begin{document}
\maketitle
\begin{abstract}
In a recent paper, \cite{Fla15} has proposed filtering based on the zeros of a
spectrogram, using the short-time Fourier transform and a Gaussian window. His
results are based on empirical observations on the distribution of the zeros of
the spectrogram of white Gaussian noise. These zeros tend to be uniformly spread over the
time-frequency plane, and not to clutter. Our contributions are threefold:
we rigorously define the zeros of the spectrogram of continuous white Gaussian noise, we
explicitly characterize their statistical distribution, and we investigate the
computational and statistical underpinnings of the practical implementation of
signal detection based on the statistics of spectrogram zeros. In
particular, we stress that the zeros of spectrograms of white Gaussian noise
correspond to zeros of Gaussian analytic functions, a topic of recent
independent mathematical interest \citep{HKPV09}.
\end{abstract}

\section{Introduction}
\label{s:intro}
Spectrograms are a cornerstone of time-frequency analysis \citep{Fla98}. They are quadratic time-frequency representations of a
signal \cite[Chapter 4]{Gro01}, associating to each time and frequency a real
number that measures the energy content of a signal at that time and frequency, unlike global-in-time tools such as the Fourier
transform. Since it is natural to expect that there is more energy where there
is more information or signal, most methodologies have focused on detecting and
processing the local maxima of the spectrogram \citep{Coh95, Fla98, Gro01}. Usual techniques include \emph{ridge extraction},
e.g., to identify chirps, or \emph{reassignment} and \emph{synchrosqueezing}, to
better localize the maxima of the spectrogram before further quantitative
analysis. 

In contrast, \cite{Fla15} has recently observed that the locations of the zeros of a
spectrogram in the time-frequency plane almost completely characterize the
spectrogram, and he proposed to use the point pattern formed by the zeros in
filtering and reconstruction of signals in noise. This proposition stems from
the empirical observation that the zeros of the short-time Fourier transform of
white noise are uniformly spread over the time-frequency plane, and tend not to
clutter, as if they repelled each other. In the presence of a signal, zeros are
absent in the time-frequency support of the signal, thus creating large holes
that appear to be very rare when observing pure white noise. This leads to testing
the presence of signal by looking at statistics of the point pattern of zeros,
and trying to identify holes. In this paper, we attempt a formalization of the
approach of \cite{Fla15}. To this purpose, we put together notions of signal
processing, complex analysis, probability, and spatial statistics.

Our contributions are threefold: we rigorously define the zeros of the
spectrogram of continuous white noise, we explicitely characterize their
statistical distribution, and we investigate the computational and statistical
underpinnings of the practical implementation of signal detection. In
particular, we stress that zeros of spectrograms of white noise correspond to
zeros of Gaussian analytic functions, a topic of recent independent mathematical interest \citep{HKPV09}.

In short, our approach starts from the usual definition of white noise as a random
tempered distribution. Using a classical equivalence between the short-time
Fourier transform and the Bargmann transform, we show that the short-time
Fourier transform of white noise can be identified with a random analytic
function, so that we can give a precise meaning to the zeros of the spectrogram of white
noise. It turns out that real and complex Gaussian white noises lead to
recently studied random analytic functions, with completely characterized zeros. We
then investigate how to leverage probabilistic information on these zeros to design
statistical detection procedures. This includes linking probability and
complex analysis results to the discrete implementation of the Fourier
transform.

The rest of the paper is organized as follows. In Section~\ref{s:preliminary},
we introduce the relevant notions of complex analysis, probability, and spatial
statistics. In Section~\ref{s:real}, we characterize the zeros of the short-time
Fourier transform of real white noise, while the complex and the analytical case
are treated in Section~\ref{s:complex}. In Section~\ref{s:stats}, we investigate
the relation between the previous sections and the usual discrete implementation
of the Fourier transform, and we demonstrate a detection task using the spectrogram zeros.


\section{Spectrograms, complex analysis, and point processes}
\label{s:preliminary}
\section{Preliminaries}

\subsection{Notation}

Let $\mX \subset \R^{I_1 \times \cdots \times I_K}$ be the space of
order-$K$ tensors, where $I_k$ denotes the dimensionality of the $k$-th
mode for $k=1,\dots,K$.  For brevity, we define
$I_{<k} := \prod_{k'<k}I_{k'}$; similarly, $I_{\leq k}, I_{k<}$ and
$I_{k \leq}$ are defined.  For a vector $Y \in \R^d$, $[Y]_i$ denotes
the $i$-th element of $Y$.  Similarly, $[X]_{i_1,\ldots,i_K}$ denotes
the $(i_1,\ldots,i_K)$ elements of a tensor $X\in\mX$. Let
$[X]_{i_1,\ldots,i_{k-1},:,i_{k+1},\ldots,i_K}$ denote an
$I_k$-dimensional vector
$(X_{i_1,\ldots,i_{k-1},j,i_{k+1},\ldots,i_K})_{j=1}^{I_k}$ called the
mode-$k$ fiber.  For a vector $Y \in \R^d$, $\|Y\| = (Y^T Y)^{1/2}$
denotes the $\ell_2$-norm and $\|Y\|_{\infty} = \max_i|[Y]_i|$ denotes
the max norm.  For tensors $X,X' \in \mX$, an inner product is defined
as
$\langle X,X' \rangle := \sum_{i_1,\ldots,i_K =1}^{I_1 \dots I_K}
X(i_1,\ldots,i_K)X'(i_1,\ldots,i_K)$
and $\|X\|_{F} = \langle X,X \rangle^{1/2}$ denotes the Frobenius
norm.  For a matrix $Z$, $\|Z\|_s := \sum_{j} \sigma_{j}(Z)$ denotes
the Schatten-1 norm, where $\sigma_j(\cdot)$ is a $j$-th singular value
of $Z$.

\subsection{Tensor Train Decomposition}

%\textit{Tensor train (TT) decomposition} is a tensor factorization
%method with a matrix product representation
%\cite{oseledets2010tt,oseledets2011tensor}.  
Let us define a tuple of positive integers $(R_1, \ldots, R_{K-1})$
and an order-$3$ tensor $G_k \in \R^{I_k \times R_{k-1} \times R_k}$
for each $k = 1,\ldots,K$.  Here, we set $R_0 = R_K = 1$.  Then, TT
decomposition represents each element of $X$ as follows:
\begin{align}
	X_{i_1,\ldots,i_K} = [G_1]_{i_1,:,:} [G_2]_{i_2,:,:} \cdots [G_K]_{i_K,:,:}. \label{eq:tt}
\end{align}
Note that $[G_k]_{i_k,:,:}$ is an $R_{k-1} \times R_k$ matrix.  We
define $\mG := \{G_k\}_{k=1}^K$ as a set of the tensors, and let $X(\mG)$
be a tensor whose elements are represented by $\mG$ as
\eqref{eq:tt}.  The tuple $(R_1, \ldots, R_{K-1})$ controls
the complexity of TT decomposition, and it is called a \textit{Tensor
  Train (TT) rank}.  Note that TT decomposition is universal, i.e.,
any tensor can be represented by TT decomposition with sufficiently
large TT rank~\cite{oseledets2010tt}.


When we evaluate the computational complexity, we assume the shape of
$\mG$ is roughly symmetric. That is, we assume there exist
$I,R\in\mathbb{N}$ such that $I_k=O(I)$ for $k=1,\dots,K$ and
$R_k=O(R)$ for $k=1,\dots,K-1$.


\subsection{Tensor Completion Problem}

Suppose there exists a true tensor $X^* \in \mX$ that is unknown, and
a part of the elements of $X^*$ is observed with some noise.  Let
$S \subset \{(j_1,j_2,
\ldots,j_K)\}_{j_1,\ldots,j_K=1}^{I_1,\ldots,I_K}$
be a set of indexes of the observed elements and
$n := |S| \leq \prod_{k=1}^K I_k$ be the number of observations.  Let
$j(i)$ be an $i$-th element of $S$ for $i=1,\ldots,n$, and $y_i$
denote $i$-th observation from $X^*$ with noise.  We consider the
following observation model:
\begin{align}
	y_i = [X^*]_{j(i)} + \epsilon_i, \label{model:obs}
\end{align}
where $\epsilon_i$ is i.i.d. noise with zero mean and variance
$\sigma^2$.  For simplicity, we introduce  observation vector
$Y := (y_1, \ldots, y_n)$, noise vector
$\mE := (\epsilon_1, \ldots , \epsilon_n)$, and rearranging operator
$\mathfrak{X} : \mX \to \mathbb{R}^n$ that randomly picks the elements of $X$.
%  $[\mathfrak{X}(X)]_i = [X]_{j(i)}$.
Then, the model \eqref{model:obs} is rewritten as follows:
\begin{align*}
	Y = \mathfrak{X}(X^*) + \mE.
\end{align*}

%%%
The goal of tensor completion is to estimate the true tensor $X^*$
from the observation vector $Y$.  Because the estimation problem is
ill-posed, we need to restrict the degree of freedom of $X^*$, such as
rank. Because the direct optimization of rank is difficult, its convex
surrogation is alternatively
used~\cite{candes2012exact,candes2010matrix, krishnamurthy2013low,
  zhang2016exact, phien2016efficient}.  For tensor
completion, the convex surrogation yields the following optimization
problem
\cite{gandy2011tensor,liu2013tensor,signoretto2011tensor,tomioka2010estimation}:
\begin{align}
	\min_{X \in \Theta} \left[ \frac{1}{2n} \|Y - \mathfrak{X}(X)\|^2 + \lambda_n \|X\|_{s^*} \right], \label{opt:general}
\end{align}
where $\Theta \subset \mX$ is a convex subset of $\mX$, 
%and
%$\Omega : \Theta \to \R_+$ is a regularization for tensors, 
$\lambda_n\geq 0$ is a regularization coefficient, and
$ \|\cdot\|_{s^*}$ is the overlapped Schatten norm defined as
$ \|X\|_{s^*} := \frac{1}{K} \sum_{k=1}^K \|\tilde{X}_{(k)}\|_s$.
Here, $\tilde{X}_{(k)}$ is the $k$-unfolding matrix defined by
concatenating the mode-$k$ fibers of $X$.  The overlapped Schatten
norm regularizes the rank of $X$ in terms of Tucker
decomposition~\cite{negahban2011estimation, tomioka2011statistical}.
Although the Tucker rank of $X^*$ is unknown in general, the convex
optimization adjusts the rank depending on $\lambda_n$.

To solve the convex problem~\eqref{opt:general}, the ADMM algorithm is often
employed~\cite{boyd2011distributed,tomioka2010estimation,
  tomioka2011statistical}.  Since the overlapped Schatten norm is not
differentiable, the ADMM algorithm avoids the differentiation of the
regularization term by alternatively minimizing the augmented
Lagrangian function iteratively.


%%% Local Variables:
%%% mode: latex
%%% TeX-master: "TTcomp_NIPS2017.tex"
%%% End:


\section{The spectrogram of real white noise}
\label{s:real}
In this section, we define real white noise, and examine the zeros of its
spectrogram. 
\subsection{Definitions}

To define white noise, we closely follow \cite[Chapter 2.1]{HOUZ10} through a
classical approach that does not require defining Brownian motion first. We denote by $\cS=\cS(\mathbb{R})$
the Schwartz space of rapidly decaying smooth complex-valued functions of a real
variable. The dual $\cS'=\cS'(\mathbb{R})$, equipped with the weak-star
topology, is the space of \emph{tempered distributions}. The topology yields the
Borel sigma-algebra $\mathcal{B}(\cS')$ on $\cS'$. Now, the Bochner-Minlos
theorem \citep[Theorem 2.1.1]{HOUZ10} states that there exists a unique probability
measure $\mu_1$ on $(\cS',\mathcal{B}(\cS'))$ such that 
\beq
\forall \phi\in \cS, \quad \mathbb{E}_{\mu_1}e^{i\langle \cdot,\phi \rangle} =
e^{-\frac{1}{2}\Vert \phi\Vert_2^2}.
\label{e:bochner}
\eeq
We call this measure white noise, and $(\cS',B(\cS'),\mu_1)$ the white noise
probability space. In particular, \eqref{e:bochner}
implies that for a random variable\footnote{We use the term \emph{random variable}, but it is also customary to call $\fomega$ a \emph{generalized random process} in the literature.} with distribution
$\mu_1$ and a set of real-valued orthonormal functions $\varphi_1,\dots,\varphi_p$ in $\cS$, the vector
$(\langle \fomega_1,\varphi_1 \rangle,\dots,\langle \fomega_p,\varphi_p \rangle)$
follows a real multivariate Gaussian, with mean zero and identity covariance matrix,
see \cite[Lemma 2.1.2]{HOUZ10}. This is in accordance with the usual heuristic of white noise having a Dirac delta covariance function.

Let $\fomega$ be a random variable with distribution $\mu_1$.
If $g\in\cS$, then $(u,v)\mapsto M_v T_u g$ is in $\cS$, so that we can define
the STFT of $\fomega$ as the random function
$$ u,v\mapsto \langle \fomega,  M_v T_u g \rangle.$$
From now on, we restrict ourselves to the Gaussian window $g(x) =
2^{1/4}e^{-\pi x^2}$, normalized so that $\Vert g\Vert_2 = 1$. We are interested
in defining and studying the zeros of the spectrogram
\beq
\label{e:spectrogram}
S:u,v\mapsto \vert \langle \fomega,  M_v T_u g \rangle\vert^2.
\eeq

\subsection{Characterizing the zeros}
\label{e:zerosSymmetricGAF}
We work in two steps: in Proposition~\ref{p:series}, we identify each value
$S(u,v)$ in \eqref{e:spectrogram} as a limit in $L^2(\mu_1)$, and we then
show in Proposition~\ref{p:symmetricPlanarGAF} that the resulting random field
defines an entire function, the zeros of which are known.
\begin{prop}
Let $u,v\in\mathbb{R}^2$, and write $z=u+iv\in\mathbb{C}$. Then
\beq
\langle \fomega,  M_v T_u g \rangle = \sqrt{\pi} e^{i\pi uv}e^{-\frac{\pi}{2}\vert z\vert^2} \sum_{k=0}^{\infty}\langle \fomega,h_k
\rangle \frac{\pi^{k/2}z^k}{\sqrt{k!}} 
\label{e:series}
\eeq
where $(h_k)$ denote the orthonormal Hermite functions \cite[Section 2.2.1]{HOUZ10}, and convergence is in $L^2(\mu_1)$.
\label{p:series}
\end{prop}
\begin{rk}
\label{r:zeros}
Note that in Proposition~\ref{p:series}, $u$ and $v$ are fixed, and the equality
is a limit in $L^2(\mu_1)$. It is still too early to identify the zeros of
the left-hand side to the zeros of the right-hand side.
\end{rk}
\begin{rk}
Note that our choice of the window $g(x) = 2^{1/4}e^{-\pi x^2}$ is made to
simplify expressions. The proof of Proposition~\ref{p:series}, along with
Sections~\ref{s:hermite} and \ref{s:bargmann}, immediately yield that for a non-unit Gaussian window $g_a(x) \propto \exp(-\pi
a^2 x^2)$, Proposition~\ref{p:series} is unchanged, provided that $z$ is defined
as $z = au + iv/a$ and a constant is prepended to the RHS of
\eqref{e:series}. In other words, given a particular value of $a$, it is always possible to
dilate/squeeze the time-frequency axes to obtain the results detailed
here for $a = 1$. 
\end{rk}
\begin{proof}
Let $u,v\in\mathbb{R}^2$. Decomposing $M_v T_u g$ in the Hermite basis $(h_k)$ of
$L^2(\mathbb{R})$, it comes
\begin{eqnarray}
\langle \fomega, M_v T_u g \rangle &=& \sum_{k=0}^\infty \langle \fomega,h_k
  \rangle \langle  M_v T_u g,h_k\rangle\nonumber\\
&=&  \sum_{k=0}^\infty \langle \fomega,h_k
  \rangle \overline{V_g(h_k)(u,v)}
\label{e:decomp}
\end{eqnarray}
where the limits are in $L^2(\mu_1)$. The STFT of Hermite functions is
well-known, see e.g. the proof of \cite[Proposition 3.4.4]{Gro01} or our Section~\ref{s:bargmann}, and it reads
\beq 
V_g(h_k)(u,v) = e^{-i\pi
  uv}e^{-\frac{\pi}{2}(u^2+v^2)}\frac{\pi^{k/2}}{\sqrt{k!}}(u-iv)^k.
\label{e:HermiteSTFT}
\eeq
Plugging \eqref{e:HermiteSTFT} into \eqref{e:decomp} yields the result.
\end{proof}

Now we focus on the regularity of the right-hand side of \eqref{e:series}.
\begin{prop}
\label{p:symmetricPlanarGAF}
The random series
\beq
\label{e:symmetricPlanarGAF}
\sum_{k=0}^{\infty}\langle \fomega,h_k
\rangle \frac{\pi^{k/2}z^k}{\sqrt{k!}}
\eeq
$\mu_1$-almost surely defines an entire function. 
\end{prop}
\begin{proof}
By \cite[Lemma 2.1.2]{HOUZ10}, ($\langle \fomega,h_k
\rangle)_{k\geq 0}$ are i.i.d. unit real Gaussians. We then apply the first part
of \cite[Lemma 2.2.3]{HKPV09}.
\end{proof}
Since both $L^2$ and almost sure convergence imply convergence in probability,
$L^2$ and almost sure limits have to be the same. In particular,
Propositions~\ref{p:series} and \ref{p:symmetricPlanarGAF} together yield that the
distribution of the zeros of the spectrogram $S$ in \eqref{e:spectrogram} is
the same as the distribution of the zeros of the random entire function
\eqref{e:symmetricPlanarGAF}. This answers Remark~\ref{r:zeros}. In particular,
we now know that the zeros of $S$ are isolated.

The entire function in \eqref{e:symmetricPlanarGAF} is called the
\emph{symmetric planar Gaussian analytic function} (GAF), and a few of its properties are known
\citep{Fel13}. However, its zeros do not define a stationary point process. In
particular, a portion of the zeros concentrate on the real axis, see
Figure~\ref{f:symmetricPlanarGAF}. Intuitively, one can approximate the zeros of
\eqref{e:symmetricPlanarGAF} by the zeros of the random polynomial obtained from
truncating the series. The resulting polynomial has real coefficients, and it is
thus expected to have real zeros as well as pairs of conjugate complex zeros. As a side note, the number of real zeros is a
topic of study on its own, see e.g. \citep{ScMa08}. 

\begin{figure}
\subfigure[Real white noise/symmetric GAF]{
\includegraphics[width=\twofig]{\figdir/realWGNdist.pdf}
\label{f:symmetricPlanarGAF}
}
\subfigure[Complex white noise/planar GAF]{
\includegraphics[width=\twofig]{\figdir/complexWGN/complexWGNdist.pdf}
\label{f:complexPlanarGAF}
}
\caption{The spectrogram of (a) a realization of real white noise, and (b) a
  realization of complex white noise. The right and top plots on each panel show
marginal histograms, superimposed with the theoretical marginal density, see
text for details.}
\label{f:GAFs}
\end{figure}

Coming back to our problem of detecting signals, this non-stationarity
makes it uneasy to approach via traditional spatial statistics techniques, which
often assume some degree of stationarity. However, there is a stationary point
process that is a good approximation for the zeros of the symmetric planar GAF,
and that has been studied in depth. This point process is the zeros of the
\emph{planar GAF}, the entire function corresponding to the STFT of complex
white noise. 


\section{The case of complex white noise}
\label{s:complex}
\section{Existence and Complexity}\label{sec:complex}
In this section we discuss the computational issues surrounding the three types of $\theta$ fair flows. The existence of Pure Nash equilibrium in nonatomic routing games guarantees the existence of any $\theta$ fair flow for $\theta\geq 1$. The next question would be whether we can compute $\theta$ fair flows with good social cost.  In particular, we consider the following problems:
\begin{enumerate}
	\item[(P1)] Find a $\theta$-EF path flow with the minimal social cost.
	\item[(P2)] Find a $\theta$-UNE path flow with the minimal social cost.
	\item[(P3)] Find a $\theta$-PNE edge flow with the minimal social cost.
\end{enumerate}
We show that for large $\theta$, the socially optimal flow is guaranteed to be contained in those $\theta$-flows, and hence the optimal $\theta$-flows be computed efficiently.  However, for small $\theta$, we will show that solving Problem~(P1) and Problem~(P2) is NP-hard, while it remains open whether Problem~(P3) can be computed efficiently.  
More precisely, for a latency class $\mc{L}$, this particular threshold is $\gamma(\mc{L})= \min\{\gamma: \ell^{*}(x)\leq \gamma \ell(x), \forall \ell\in \mc{L}, \forall x\geq 0\}$, where $\ell^{*}(x)= \ell(x)+x\ell'(x)$.  The main result of this section is given as follows:
\begin{theorem}
	%\begin{enumerate}
	%\item \fin{
	For any multi commodity instance $\mc{G}$ 
	with latency functions in any class $\mc{L}$, there are polynomial time algorithms\footnotemark $\, $  for solving Problem~(P1)-(P3) 
	for $\theta \ge \gamma(\mc{L})$. %, for any class $\mc{L}$.
	%
	On the other hand, it is NP-hard to solve Problem~(P1) for $\theta \in [1, \gamma(\mc{L}))$ and Problem~(P2) for $\theta \in (1, \gamma(\mc{L}))$, for  arbitrary single commodity instances %$\mc{G}$ 
	with latency functions in an arbitrary class $\mc{L}$.
	%\end{enumerate}
	\label{thm:main_hardness}
\end{theorem} 

\footnotetext{The existence of polynomial time algorithms for our problem depends on the assumption that we can minimize separable convex functions with linear constraints in polynomial time; numerical issues for convex optimization are discussed in \cite{hochbaum1990convex,  nemirovski2004interior} and are beyond the scope of our work.}
%on the existence of polynomial time algorithms for minimizing separable convex functions with linear constraints upto arbitrary precision. See~\cite{hochbaum1990convex,  nemirovski2004interior}   for details, which are quite technical for the scope of this paper.}}





In the following sections, we first prove the first part of Theorem \ref{thm:main_hardness}. Right after we show that, for any $\theta$, from any $\theta$ fair flow we may get another $\theta$ fair flow, which uses only polynomially many paths. This, on the one hand, serves as a clarification that the difficulty of problems~(P1)-(P3) does not lie in the size of their solutions. On the other hand, it helps in showing that the decision version of these problems lies in NP, since for a YES instance, a non-deterministic machine will (non-deterministically) choose a path flow of polynomial size and in polynomial time check that it satisfies the conditions needed. Finally, the second part of Theorem \ref{thm:main_hardness} follows from an NP-hardness proof for a stronger version of the decision versions of problems~(P1) and (P2) (Theorem \ref{thm:12Hard}).


\subsection{When the Social Optimum is Guaranteed to be the Solution}\label{sec:complexity_so}
First, we show that Problems~(P1)-(P3) are easy for $\theta \ge \gamma(\mc{L})$ because the social optimum is the solution.
%For a latency class $\mc{L}$, define $\gamma(\mc{L})= \min\{\gamma: \ell^{*}(x)\leq \gamma \ell(x), \forall \ell\in \mc{L}, \forall x\geq 0\}$. Here $\ell^{*}(x)= \ell(x)+x\ell'(x)$ is the marginal latency.  
The following lemma, which is a direct extension of Theorem~4.2 in Correa et al~\cite{correa2007fast}, shows that any path decomposition of the socially optimal flow is a $\gamma(\mc{L})$ fair flow, provided that the latency functions are in class $\mc{L}$.  While in the proof of Theorem~4.2 in Correa et al~\cite{correa2007fast} they only conclude that the set of socially optimal path flows is $\gamma(\mc{L})$-EF, it is easy to see that the same argument holds for $\gamma(\mc{L})$-PNE.
\begin{lemma}(\cite{correa2007fast})
For a network $\mc{G}$ with latency functions  in class $\mc{L}$, any socially optimal path decomposition $o \in SO_p$ is $\gamma(\mc{L})$-PNE.
\end{lemma}
 
Since the social optimum can be computed using convex programming~\cite{roughgarden2002selfish}, it follows that %With this result, we are able to give polynomial time algorithms for 
Problems~(P1)-(P3) can be solved in polynomial time\footnotemark[6] for $\theta\ge\gamma(\mc{L})$.
\begin{proof}[Proof, first part of Theorem~\ref{thm:main_hardness}]
Note that given a path flow, which is $\gamma(\mc{L})$-PNE, it is $\gamma(\mc{L})$-UNE and $\gamma(\mc{L})$-EF as well.  This means that for all $\theta \geq \gamma(\mc{L})$, we can simply compute the socially optimal flow, and give any path decomposition as the $\theta$ fair flow. The socially optimal edge flow can be computed in time polynomial in the size of the network.  Further, a greedy path decomposition suffices. In the greedy algorithm, at every step we pick the current minimum path (among all commodities) and assign the maximum possible flow, under the social optimum, through this path. This can be computed in time $O(|\mc{K}|\times|E|)$.  Also, the output path flow can be represented with a sparse vector with $O(|\mc{K}|\times|E|)$ entries.
%linear in number of edges times the number of commodities ).
\end{proof}



\subsection{Existence of Polynomial-size Path Flow Solutions}
An observation to Problem~(P1) and (P2) is that the outputs of these two problems are path flow vectors, which are potentially of exponential size relative to the problem instances. 
In Section~\ref{sec:complexity_so} we showed a way to compute a path flow vector with polynomial support under the social optimum.  Here we ask whether we can do this for any edge flow.  In particular, we are interested in whether we can always find an answer to either Problem~(P1) or (P2) using only polynomially many paths.  If not, then there is no hope for us to find an efficient algorithm for these problems.  In this subsection, we show that the answer to this question is \emph{yes}.  To see this, we make a more general argument than Lemma~3.1 in Correa et al.~\cite{correa2007fast}, showing that given any path flow vector, we can always find another path flow assignment of polynomial support that preserves four important properties.
\begin{proposition}\label{lemma:correa1}
	Let $\bm{f}$ be a feasible flow for a multicommodity flow network with load-dependent edge latencies. Then, there
	exists another feasible flow $\bm{f}'$ such that 
	\begin{enumerate}
		\item $\bm{f}$ and $\bm{f}'$ have the same edge flow.
		\item The longest used path for commodity $k$ satisfies $\max_{\pi \in \mc{P}_u^k(\bm{f'})} l_{\pi}(\bm{f'}) \le \max_{\pi \in \mc{P}_u^k(\bm{f})} l_{\pi}(\bm{f})$.
		%$L_{max}(\bm{f}')\leq L_{max}(\bm{f}')$.
		\item The shortest used path for commodity $k$ satisfies $\min_{\pi \in \mc{P}_u^k(\bm{f})} l_{\pi}(\bm{f}) \le \min_{\pi \in \mc{P}_u^k(\bm{f'})} l_{\pi}(\bm{f'})$.
		\item The flow $\bm{f}'$ uses at most $|E|$ paths for each source-sink pair.
	\end{enumerate}
\end{proposition}
\begin{remark}
The proof of this proposition directly follows the proof of Lemma~3.1 in Correa et al.~\cite{correa2007fast}, although our lemma statement is more general. (Their Lemma only states part 2 of our Lemma statement.)
\end{remark}

%\begin{proof}
%	The proof basically follows the proof of Lemma~3.1 in Correa et al.~\cite{}.  They introduce a process that iteratively create a new path flow $\bm{f}'$ that has the same edge flow as $\bm{f}$ by moving the flow along one particular path to other used paths.  In their lemma, while they only conclude that the length of the longest path will not increase, similar argument will also hold for that the length of the shortest path will not decrease. 
	%Fix commodity $k$, we index the used paths in $\mc{P}_u^k(\bm{f})$ by $\pi_1, \pi_2, \dots, \pi_r$, where $r$ is the number of used paths in commodity $k$ under $\bm{f}$.  For each $\pi_i$, we define an edge incident vector $\bm{p}_i \in \{0,1\}^{|E|}$, where $p_{ie}=1$ if and only if $e \in \pi_i$.  If $r>|E|$, then the edge incident vectors are linearly dependent.  Then, we can reduce the number of used path with the following process:  First let $\lambda_1, \lambda_2, \dots, \lambda_r$ be such that $\sum_{i=1}^{r} \lambda_i \bm{p}_i = 0$.
%\end{proof}
With this proposition, we can make the following argument that given an edge flow $\bm{x}$, if there is at least one $\theta$-EF or $\theta$-UNE path flow decomposition, then we can always find one with %only linear
polynomial support:
\begin{lemma}\label{thm:npproblems}
	Given a $\theta$-EF path flow $\bm{f}_1$, there exists a $\theta$-EF path flow $\bm{f}'_1$ that uses at most $|E|$ paths for each source-sink pair and has the same edge flow as $\bm{f}_1$.  Similarly, given a $\theta$-UNE path flow $\bm{f}_2$, there exists a $\theta$-UNE path flow $\bm{f}'_2$ that uses at most $|E|$ paths for each source-sink pair and has the same edge flow as $\bm{f}_2$.
\end{lemma}
\begin{proof}
	For a $\theta$-EF path flow $\bm{f}_1$, by Proposition~\ref{lemma:correa1}, there exists a flow $\bm{f}'_1$ that has the same edge flow as $\bm{f}_1$ and the ratio of the longest used path to the shortest used path is bounded by
	$$
	\frac{\max_{\pi \in \mc{P}_u^k(\bm{f'}_1)} l_{\pi}(\bm{f'}_1)}{\min_{\pi \in \mc{P}_u^k(\bm{f'}_1)} l_{\pi}(\bm{f'}_1)} \le \frac{\max_{\pi \in \mc{P}_u^k(\bm{f}_1)} l_{\pi}(\bm{f}_1)}{\min_{\pi \in \mc{P}_u^k(\bm{f}_1)} l_{\pi}(\bm{f}_1)} \le \theta
	$$
	which indicates that $\bm{f}'$ is a $\theta$-EF path flow.  Similarly, given a $\theta$-UNE path flow $\bm{f}_2$, we can find a path flow $\bm{f}_2'$ that has the same edge flow as $\bm{f}_2$ and 
	$$
	\frac{\max_{\pi \in \mc{P}_u^k(\bm{f'}_2)} l_{\pi}(\bm{f'}_2)}{\min_{\pi \in \mc{P}^k} l_{\pi}(\bm{f'}_2)} \le \frac{\max_{\pi \in \mc{P}_u^k(\bm{f}_2)} l_{\pi}(\bm{f}_2)}{\min_{\pi \in \mc{P}^k} l_{\pi}(\bm{f}_2)} \le \theta
	$$
	from which we can conclude that $\bm{f}'_2$ is a $\theta$-UNE as well. 
\end{proof}
Now suppose $\bm{f}_1^*$ is the optimal solution to Problem~(P1).  According to Lemma~\ref{thm:npproblems}, we can see that there is an alternative path flow $\bm{f}_2^*$ that is also $\theta$-EF and has the same edge flow as $\bm{f}_1^*$.  Since the social cost only depends on the amount of the edge flow, $\bm{f}_1^*$ and $\bm{f}_2^*$ have the same social cost, from which we can conclude that $\bm{f}_2^*$ is an optimal solution to Problem~(P1) that uses only polynomially many paths.  A similar argument can be made for Problem~(P2) as well.

\subsection{Hardness Results}
In this section, we prove the second part of Theorem~\ref{thm:main_hardness} that it is NP-hard to solve Problem~(P1) and (P2) for small values of $\theta$.  More precisely, we consider the class of polynomial functions of degree at most $p$, which we denote as $\mc{L}_p$. We note that $\gamma(\mathcal{L}_p)=p+1$. We show that when the latency functions are in $\mc{L}_p$, then the related decision problems we state in Theorem~\ref{thm:12Hard} have polynomial-time reductions from the NP-complete problem PARTITION.  We state this result in the following theorem:

%Suppose we are given an instance of a single commodity flow network $\mc{G}$ 
%with latency functions in class $\mc{L}_p$ for $p \ge 1$.

\begin{theorem}
For an arbitrary single commodity instace $\mc{G}$ 
with latency functions in class $\mc{L}_p$ for $p \ge 1$, it is NP-hard to
\begin{enumerate}
\item decide whether a socially optimal flow has a $\theta$-UNE path flow decomposition for $\theta \in (1, p+1)$.
\item decide whether a socially optimal flow has a $\theta'$-EF path flow decomposition for $\theta' \in [1, p+1)$.
\end{enumerate}
\label{thm:12Hard}
\end{theorem}

We state the following corollary that readily follows from Theorem~\ref{thm:12Hard}.
\begin{corollary}
For  any finite $\theta> 1$, it is NP-hard to find the optimal $\theta$-UNE or $\theta$-EF flow of an arbitrary instance $\mathcal{G}$.
\end{corollary}
\begin{proof}
For a given $\theta$ pick any $p\in \mathbb{N}:\theta<p+1$. Since $p+1=\gamma(\mathcal{L}_p)$, we may use  Theorem~\ref{thm:12Hard} to get the result.
\end{proof}

The proof of Theorem~\ref{thm:12Hard} is composed of two parts.  For the first part, we show the NP-hardness for $1.5$-UNE and $1$-EF path flow decompositions under the social optimum  in Lemma~\ref{lemm:corehardness}, based on the construction in Theorem~3.3 in Correa et al.~\cite{correa2007fast}.  Then, in the second part, we propose a novel way to generalize the construction to the entire range of $\theta$ and $\theta'$ specified in Theorem~\ref{thm:12Hard}.

\begin{lemma}\label{lemm:corehardness}
For single commodity instances with linear latency functions it is NP-hard to decide whether a social optimum flow has a $1.5$-UNE flow decomposition or a $1$-EF flow decomposition.
\end{lemma}
\begin{proof}
We consider the PARTITION problem, where we are given a set of $n$ positive integer numbers $q_1,\ldots, q_n$, and we need to decide  \emph{is there a subset $I \subset \{1,\ldots,n\}$ such that $\sum_{i\in I} q_i = \sum_{i \notin I}q_i$?}
%-----------------------------------------------------------------------------------------
 \begin{figure}[!htb]
 \centering
 \includegraphics[width=0.6\linewidth]{reduction}
 \caption{An instance of congestion game constructed from a given instance of PARTITION}
 \label{Fig:instForProg4}
 \end{figure}
%-----------------------------------------------------------------------------------------

Consider the two link parallel network with the top link $e_{u}$ having latency $\ell_u(x)=q$ and the bottom link $e_b$ having latency $\ell_b(x)=qx$. The demand between the source and the destination is $1$. 
%In the equilibrium flow  the bottom link carries $\frac{1}{2}(1+k)^{\frac{1}{k}}$ flow and the Nash equilibrium length $L_{NE,1}=\frac{k+1}{2^k}$. 
The unique socially optimal flow splits the flow equally through the top and bottom link. Call this instance $G(q)$.

 
Given an instance of the PARTITION problem, $q_1,\ldots, q_n$, $\sum_{i=1}^{n} q_i=2B$, we now construct a single commodity network as the two link $n$ stage network $G$, as shown in Figure \ref{Fig:instForProg4}. In stage $i$ we connect $G(q_{i-1})$ to $G(q_{i})$ to the right for $i=2$ to $n$. A unit demand has to be routed from the source in $G(q_1)$ to the destination in $G(q_n)$.  For the graph $G$, the socially optimal flow $o$ routes $1/2$ flow through all top links and the remaining $1/2$ flow through each bottom link. We first observe that there is a one-to-one correspondence between the subsets $I\subseteq [n]$ and paths $p$ in $G$. Specifically, we can define the path corresponding to $I$ as $P_I=\left\{e_{u,i}: i\in I\right\} \cup \left\{e_{b,i}: i\notin I\right\}$. Further, the latency of the path is given by $\ell_I = \frac{1}{2}(\sum_{i\in [n]} q_i+\sum_{i\in I} q_i)$. 

In one direction, we observe that if the answer to the PARTITION problem is YES then there exists a subset $I^*$ such that $\sum_{i\in I^*} q_i = \sum_{i \notin I^*}q_i = B$. Consider the path flow under socially optimal flow $o$, with path $P_{I^*}$ carrying flow $1/2$ and path $P_{[n]\setminus I^*}$ carrying flow $1/2$.  The lengths of paths $P_{I^*}$ and $P_{[n]\setminus I^*}$ are both  equal to $\frac{3}{4}\sum_{i=1}^{n} q_i = 3B/2$. Whereas, the shortest path in the network is  $P_{\emptyset}$  with length $\frac{1}{2}\sum_{i=1}^{n} q_i = B$. Therefore, the socially optimal flow $o$ is a $3/2$-UNE flow and a $1$-EF flow, if $G$ comes from a YES instance of PARTITION. 



In the other direction, we first observe that if a path $P_{I}$ under edge flow $o$ has length $3B/2 = \frac{3}{4}\sum_{i=1}^{n} q_i$, then $\sum_{i\in I} q_i = \frac{1}{2}\sum_{i\in [n]} q_i$. This implies the given answer to the PARTITION problem is YES. Now assuming $o$ is a $3/2$-UNE, there exists a path flow with the maximum used path of length less or equal to $\frac{3}{4}\sum_{i=1}^{n} q_i$. But the average length of any used path under $o$ is equal to $\frac{3}{4}\sum_{i=1}^{n} q_i$. This implies that all the paths in the path flow must have length $\frac{3}{4}\sum_{i=1}^{n} q_i$.  Next we assume that $o$ is a $1$-EF flow. This implies that there exists a path flow for which all the used paths have equal length. But then any used path under this decomposition has length $\frac{3}{4}\sum_{i=1}^{n} q_i$. Therefore, if $o$ is a $3/2$-UNE or a $1$-EF then the PARTITION instance corresponding to $G$ is a YES instance.   
\end{proof}
 
\begin{proof}[Proof of Theorem~\ref{thm:12Hard}]
Consider $\theta \in (1,p+1)$ for a UNE flow and $\theta' \in [1,p+1)$ for an EF flow. Given a PARTITION instance, let $G'$ be a two link parallel network with latency of the top link $\ell_{u,(n+1)}(x) = ax^p+b$ and bottom link latency $\ell_{d,(n+1)}(x) = cx^p$. We set $a=\frac{\alpha B}{(1-3/8B)^p}$, $b=\beta B(p+1)$, and $c=\frac{(\alpha+\beta)B}{(3/8B)^p}$, where $\alpha,\beta>0$ are some parameters to be determined later.

Using the fact that the social optimum is an equilibrium of the instance with latencies modified to $(\ell(x) + x\ell'_e(x))$, we  get that the socially optimal flow in network $G'$ is $\frac{3}{8B}$ through the bottom link and $(1-\frac{3}{8B})$ through the top link. We also get that at the social optimum the latency function satisfies the following condition:
$$
c\bigg(\frac{3}{8B}\bigg)^p = a\bigg(1-\frac{3}{8B}\bigg)^p + \frac{b}{p+1} < a\bigg(1-\frac{3}{8B}\bigg)^p + b
$$
From the latter, we can see that the top link has larger cost than the bottom link.  We then combine in series the network $G$ of Lemma \ref{lemm:corehardness} with the network $G'$ to obtain network $H$. The unique socially optimal flow in  network $H$ is the union of the two unique socially optimal flows in $G$ and $G'$. Recall the notation from Lemma \ref{lemm:corehardness}.

Assume the PARTITION problem admits a solution $I$. Consider the path decomposition in $H$:
\begin{enumerate}
	\item Path $p = P_{I}-e_{u, (n+1)}$ carries $1/2$ flow (note that $3/8B < 1/2$).
	\item Path $q = P_{I^c}-e_{u, (n+1)}$ carries $(1/2 - 3/8B)$ flow.
	\item Path $r = P_{I^c}-e_{d, (n+1)}$ carries $3/8B$ flow.
\end{enumerate}
We can see that the path $s = P_{\emptyset}-e_{d, (n+1)}$ is the shortest path, with latency $\ell_{s} = B + c(3/8B)^p = (\alpha+\beta+1)B$.  The longest used path $q$ has latency $\ell_{q} = 3B/2 + a(1-3/8B)^p + b = (\alpha+\beta+\beta p+3/2)B$.  Letting $c_1 = \frac{\ell_{q}}{ \ell_{s}}=\bigg(\frac{\alpha+\beta+\beta p+3/2}{\alpha+\beta+1}\bigg)$, the social optimum flow in $H$ is a $c_1$-UNE flow.

We next consider a different path flow for the EF setting. In this path flow:
\begin{enumerate}
	\item Path $s' = P_{[n]}-e_{d, (n+1)}$ carries $\frac{3}{8B}$ flow.
	\item Path $p = P_{I}-e_{u, (n+1)}$ carries $(1/2 - 3/8B)$ flow.
	\item Path $q = P_{I^c}-e_{u, (n+1)}$ carries $(1/2 - 3/8B)$ flow.
	\item Path $r'= P_{\emptyset}-e_{u, (n+1)}$ carries $\frac{3}{8B}$ flow.
\end{enumerate}
We claim that path $s'$ is the shortest path if $\beta p>1$ as
$$
\ell_{s'} = 2B + c(3/8B)^p = (\alpha+\beta+2)B <
(\alpha + \beta + \beta p + 1)B = B + a\bigg(1-\frac{3}{8B}\bigg)^p + b = \ell_{r'}<\ell_p=\ell_q.
$$
In this setting, the minimum ratio of longest `used' path and shortest `used' path is  $c_2 = \frac{\ell_{q}}{ \ell_{s'}}=\bigg(\frac{\alpha+\beta+\beta p+3/2}{\alpha+\beta+2}\bigg)$ and the socially optimal flow is a $c_2$-EF flow.

Next, we need to show that if the answer to PARTITION is NO then the socially optimal flow is neither a $c_1$-UNE flow nor a $c_2$-EF flow. For this we need to ensure that for all possible path flows under the social optimum, there exists at least one used path which is obtained by concatenating a `long' positive subpath in $G$ with the upper edge in $G'$. The following claim lower bounds the flow through the longest path in $G$ for any valid path decomposition. 

\begin{claim}\label{lemm:flowlower}
If the answer to PARTITION is NO then in the sub-network $G$ any path decomposition of the socially optimal flow $o$ routes at least $\frac{1}{2B}$ amount of flow through paths of length strictly greater than $\frac{3}{2}B$. 
\end{claim}
\begin{proof}
Recall that if the given instance for the PARTITION problem is a NO instance then there is no path under $o$ which has length exactly $3B/2$.
Fix any path decomposition for $o$ and let $\delta$ be the flow passing through the paths of length strictly greater than $\frac{3}{2}B$.   Also let $\ell$ be the maximum length among the set of paths strictly smaller than $\frac{3}{2}B$. As $q_i$'s are integers  and the given instance of PARTITION is a NO instance, it is easy to observe that $\ell \leq \frac{3}{2}B-\frac{1}{2}$. Also $\ell\geq B$. Moreover, if we route $(1-\delta)$ flow through a path of length $\ell$ and  $\delta$ flow through the path of maximum length $2B$, then the cost of this routing is greater or equal to the socially optimal cost. This implies,
%\begin{align*}
$$\ell(1-\delta)+2\delta B \geq \frac{3}{2}B  \implies
\delta \geq \frac{3B/2-\ell}{2B-\ell} \geq \frac{3B/2-3B/2+1/2}{2B-B} \geq \frac{1}{2B}.$$
%\end{align*}  
\end{proof}   

From the above claim we see that the longest used path $q$ has length strictly greater than $\ell_{q}$ as the bottom link under $o$ has flow $3/8B < 1/2B$. The shortest path in the network has length $\ell_{s}$ as in the YES case. If the PARTITION instance is a NO instance, the optimal flow $o$ is not a $c_1$-UNE. Moreover, for the EF flow the best path flow again contains the path $s'$ as the shortest path but now the longest path is strictly greater than $\ell_{q}$. So it is not a $c_2$-EF flow. 

All that is left to show is that there are appropriate values of $\alpha$ and $\beta$ which make $c_1 = \theta$ or $c_2 = \theta'$, for any $\theta\in (1, p+1)$, and for any $\theta'\in (1, p+1)$.  This can be shown by observing that:
\begin{align*}
	c_1=\frac{\alpha+\beta+\beta p + \frac{3}{2}}{1+\alpha+\beta}&=1+\frac{\frac{1}{2}+\beta p}{1+\alpha+\beta} & c_2=\frac{\alpha+\beta+\beta p + \frac{3}{2}}{2+\alpha+\beta}&=1+\frac{-\frac{1}{2}+\beta p}{2+\alpha+\beta}
\end{align*}   
Combining this with what we have shown in Lemma~\ref{lemm:corehardness} for $1$-EF flows completes the proof.
\end{proof}
\begin{proof}[Proof, second part of Theorem~\ref{thm:main_hardness}]
The proof follows by constructing a reduction from the decision problems specified in Theorem~\ref{thm:12Hard} and recalling that $\gamma(\mathcal{L}_p)=p+1$.  The answer to each of the decision problem in Theorem~\ref{thm:12Hard} is YES if and only if the solution to Problem~(P1) or (P2) is a social optimum, the cost of which is known in advance, by construction.
\end{proof}




%\begin{remark}
%	The proof is inspired from the proof of the NP hardness of length bounded flow problem (Theorem~3.3) in Correa et al.~\cite{}. But the conclusions apply to the completely new setting of UNE and EF flow and we generalize it through novel constructions. 
%\end{remark}

For Problem~(P3), the proof technique in Theorem~\ref{thm:main_hardness} does not go through.  In fact, we show that the relevant decision problem related to Problem~(P3) is in P: % We consider the following problem:
\begin{enumerate}
\item[(P3')] Is there a socially optimal flow which is a $\theta$-PNE?
\end{enumerate}

To show that (P3') is in P, we first define an edge flow $\bm{x}$ to be \emph{acyclic} if for each commodity $k$, the subgraph $G_k$, induced by the edges $E_k(\bm{x}) = \{e: e\in E, x_e^k>0\} $ is a directed acyclic graph (DAG).
  
\begin{claim}
Given an instance of a multicommodity flow network $\mc{G}$ with standard latency functions, we can decide whether an `acyclic' edge flow $\bm{x}$ is in $\theta$-PNE in polynomial time.
\label{clm:Acyclic}
\end{claim}
\begin{proof}
We present the polynomial time algorithm which decides whether an `acylic' edge flow $\bm{x}$ is a $\theta$-PNE or not for some given $\theta$. For each commodity $k$ in $\mc{G}$, we construct the DAG induced by $E_k(\bm{x})$. Next, under the edge weights $w_e = \ell_e(x_e)$, we compute the costs of the shortest $(s_k, t_k)$ path in $G$ (call it $\ell_1$) and the longest $(s_k,t_k)$ path in $G_k$ (call it $\ell_2$).  Recall that shortest path computation and longest path computation in a DAG can both be done in polynomial time. Finally, we accept if $\ell_2 \leq \theta\ell_1$ and reject otherwise. 
\end{proof}

\begin{lemma}
Problem~(P3') can be solved in polynomial time.
\label{lemma:3Easy}
\end{lemma}
\begin{proof} 
We first claim that for any $k$, the set of edges that carry flow for commodity $k$ at the social optimum, $E_k(\bm{x}^*)$, has no positive loops.  This can be shown by contradiction.  Assume there is a positive loop in $E_k(\bm{x}^*)$, then, we can construct a new flow $\bm{x}'$ by removing some $\epsilon>0$ flow on the loop.  The flow $\bm{x}'$ can be kept feasible, and it has strictly smaller social cost due to the monotonicity and non-negativity of the latency functions, which contradicts the fact that $\bm{x}$ is the socially optimal flow.  Also, if there is a zero cost loop in $E_k(\bm{x}^*)$, we can safely remove the flow on that loop without changing the social cost. Therefore, the procedure in Lemma~\ref{clm:Acyclic} completes the proof. 
\end{proof}

%\note[RB]{Describe L function, give closed form if available}
%\note[RB]{Make a quantitive comment on the difference of cdfs for symmetric and
%  planar}

\section{Practical spatial statistics using the zeros of the STFT}
\label{s:stats}
In Section~\ref{s:implementation}, we discuss how to relate the continuous
complex plane $\mathbb{C}$ with the practical discrete implementation of the
Fourier transform. In Section~\ref{s:detection}, we investigate simple
hypothesis tests for signal detection, as in \citep{Fla15}. 

\subsection{Going discrete}
\label{s:implementation}
To fully bridge the gap with numerical signal processing practice, there is an
additional level of approximation that needs to be discussed: Continuous
integrals are replaced by discrete Fourier transforms, so that the fast Fourier
transform can be used. We first describe an experimental setting to study the zeros of the spectrogram of Gaussian white noise. In particular, we explain how to reach an asymptotic regime where the noise occupies an infinite range both in time and frequency and the spectrogram is infinitely well resolved.
% When a signal is present
Second, we investigate practical issues related to detecting a signal in white
noise by using its influence on the distribution of zeros of the spectrogram. 

%\note[RB]{Je crois qu'a ce stade, mieux vaut eviter de partler de discretization steps et de ``specific scales'', car on ne comprend pas encore de quoi il s'agit.}
%the consequences of the presence of signalwhen a signal is present, some specific scales enter in the game. Then one cannot change discretization steps at will anymore.


%%%%
\subsubsection{Zeros of noise only}
\label{s:noiseonly}
Let $F_s$ the sampling frequency, $\Delta t=1/F_s$ the time sampling step size
and $T$ the duration of the observation window. The number of samples is then $N +1$ with 
$N = T/\Delta t$. 

% Approximating $\langle \chi_n,M_v T_u g \rangle$ by $M_vT_ug (n\Delta t)$, it comes
% $$\langle  \fomega, M_{v}T_u g\rangle \approx \lim_{N\rightarrow \infty}
% \sum_{n=1}^N\langle \fomega,\chi_n \rangle e^{-2i\pi v n\Delta t} g(n\Delta t-u),$$
% and we thus expect the discrete spectrogram of a sequence of i.i.d Gaussians
% with large $N$ to be a good approximation to the STFT of white noise.
% }

Let $K$ be
the length of the discretized Gaussian analysis window, i.e. its duration is
$K\Delta t$; therefore $\Delta \nu = F_s/K=1/K\Delta t$ is the frequency
sampling step. In practice, the spectrogram obtained from a discrete STFT is
then an array of size $(N+1, K/2+1)$. Then we consider the time-frequency domain
$[0,T]\times[0,F_s/2]$ only; it corresponds to the analytic signal. This is due
to the Hermitian symmetry of the Fourier transform of real signals: negative
frequencies do not add any information to that carried by positive frequencies,
see also Section~\ref{s:analytic}. This Hermitian symmetry can also be seen on
the zeros of the symmetric GAF in Figure~\ref{f:symmetricPlanarGAF}, where
signal processing practice would have us only consider the upper half-plane ($\nu\geq 0$).
 From \cite{Fel13}'s results, see (\ref{e:countingMeasureSym}), we know that the expected number of zeros of the continuous spectrogram is close to $TF_s/2$ if we neglect the (asymptotically negligible) region $|\nu|\leq a$ close to the time axis, see Section~\ref{s:GAFapproxSYM}. Assuming that we are able to extract every zero, the expected number of zeros in the discrete spectrogram is then $TF_s/2=N/2$ in very good approximation.

Let $\sigma_t=1/(a\sqrt{2\pi})$ and $\sigma_\nu= 1/(2\pi\sigma_t)$
denote the spreads of the Gaussian analysis window $g_a$ in time and frequency, respectively. Note that the scale $a$ serves as a fixed reference for scales in the sequel. We would like to retain the stationary properties of the planar GAF in our discrete
STFTs. We thus require that, in the discrete setting, the resolution --~in number of points~-- should be the same in time and frequency, that is
\begin{equation}
\label{eq_isotropy}
	\frac{\sigma_t}{\Delta t} = \frac{\sigma_\nu}{\Delta \nu} \Longleftrightarrow \sigma_t \cdot F_s = \sigma_\nu \cdot K\Delta t
\end{equation}
This leads to
\begin{equation}
	\left(\frac{\sigma_t}{\Delta t}\right)^2 = \frac{K}{2\pi} \Leftrightarrow \sigma_t = \sqrt{\frac{K}{2\pi}}\Delta t.
\end{equation}
If we want to study the spectrogram of continuous white noise over an infinite time-frequency domain, numerical simulations must obey two necessary conditions:
\begin{equation}
	\begin{cases}
		\text{infinite duration } \Leftrightarrow \text{fine frequency resolution} & : T/\sigma_t = 2\pi\sigma_\nu/\Delta\nu \rightarrow + \infty\\
		\text{infinite frequency range }  \Leftrightarrow \text{fine time resolution} & : F_s/\sigma_\nu=2\pi\sigma_t/\Delta t \rightarrow + \infty\\
		%\text{infinitely fine resolution } & \sigma_t/T \rightarrow 0 \text{ and } \sigma_\nu/F_s \rightarrow 0\\
	\end{cases}
\end{equation}
In terms of samples, these two conditions imply that $N, K \rightarrow \infty$. More precisely, 
\begin{eqnarray}
	\frac{\sigma_t}{T} & = & \frac{1}{N}\sqrt{\frac{K}{2\pi}} \rightarrow 0 \text{ as } N, K \rightarrow \infty\\
	\frac{\sigma_\nu}{F_s} & = & \frac{1}{\sqrt{2\pi K}}\to 0 \text{ as } N, K \rightarrow \infty.
\end{eqnarray}

%% Bilan
These conditions are directly satisfied for $K \propto N$, where $\propto$ means
``proportional to''. Note that in practice because of border effects one chooses $N = 2K$ and keeps the $N$ samples whose time index $n$ is such that $K/2 \leq n \leq N - K/2$. Then, $\sigma_\nu/F_s=1/\sqrt{2\pi K}\propto 1/\sqrt{N}$, $\sigma_t/T\propto 1/\sqrt{N}$; note that $\Delta t/\sigma_t=\Delta \nu/\sigma_\nu\propto 1/\sqrt{N}$ as well. As a result, simulations can asymptotically well approximate the continuous spectrogram of Gaussian white noise.
%% Figure 3 and 4

\begin{figure}
\begin{center}
	\includegraphics[height=60mm]{\figdir/discretization_spectro.png}
	\caption{\label{fig_discrete} Illustration of the discrete time-frequency plane $\{(n\Delta t, k\Delta \nu),\; 0\leq n\leq N-1,\; 0\leq k\leq K/2\}$. The resolution of the spectrogram is controlled by the analysis window's Gabor parameters $(\sigma_t, \sigma_\nu)$.}
\end{center}
\end{figure}
\begin{figure}
	\centering
	\includegraphics[height=80mm]{\figdir/STFT_illustration.png}
\caption{\label{fig_stft} Illustration of the STFT: the noisy signal is convolved with a Gaussian
that is translated in time and frequency. The colour code corresponds to
Figure~\ref{fig_discrete} for ease of reference.}
\end{figure}{}

Figure~\ref{fig_discrete} illustrates the relative scales of the duration $T =
N\Delta t$, the frequency range $K/2\Delta t$ (for $\nu\geq 0$), the time and
frequency resolutions $\Delta t$ and $\Delta \nu$, as well as the resolution of
the time-frequency kernel corresponding to the window $g(t)$ with Gabor spread
$(\sigma_t, \sigma_\nu)$. For the sake of completeness and the reader new to
time-frequency, we include in Figure~\ref{fig_discrete} an illustration of the
STFT of a noisy signal.

%% discussion normalization+correspondence discrete.
Now we detail how to relate the discrete coordinates of a discrete spectrogram
with the continuous complex plane. For a given value of $a$, one has
$\sigma_t=1/(a\sqrt{2\pi})$ and thus making the correspondence between samples
and time-frequency units implies setting $\Delta t = \sqrt{2\pi/K}\sigma_t$. For
$a = 1$ one has $\Delta t = \sqrt{1/K}$ so that $u = n/\sqrt{K}$ and $v =
k/\sqrt{K}$ are the coordinates of the time-frequency plane corresponding to
time sample $n$ and frequency sample $k$, respectively.
Figure~\ref{fig:edge_effects} depicts the whole numerical simulation procedure.
It represents the simulated spectrogram and the corresponding extracted area,
taking border effects in consideration. The bound $\ell$ fixes how many samples
close to the zero-frequency axis should be removed. For $a=1$, we have chosen
$\ell = \sqrt{K}$, at it corresponds to $y = 1$ in (\ref{e:countingMeasureSym}).
Note also that border effects alone would actually allow us to extend the shaded
square in Figure~\ref{fig:edge_effects} on its left and right to include $K$
samples. Instead, we chose to reduce it to $K/2-\ell$ mostly for esthetical concerns:
since the point process we observe is almost stationary when only noise is
present, we favoured a square window rather than a rectangle.

% Figure 5
\begin{figure}
	\centering 
	\includegraphics[width=\textwidth]{edge_discrete.pdf}
\caption{Numerical
    simulation procedure. Black ticks indicate the number of samples, while blue
    ticks show time-frequency units for a choice of $\Delta t = 1/\sqrt{K}$ (see
    text for details). In other words, blue ticks are the coordinates in the
    complex plane that are implicit in the mathematical results of
    Sections~\ref{s:real} and \ref{s:complex}. The dashed region corresponds to the area used in subsequent simulations.}\label{fig:edge_effects}
\end{figure}

% discrete spectro
When the conditions above are satisfied, several phenomena occur in the limit of
infinite oversampling $N\to\infty$, which is equivalent to letting both the
duration $T$ and the sampling
frequency $F_s$ grow to infinity. In a dual manner, the resolution $(\Delta t, \Delta \nu)$ of the
discrete spectrogram tends to zero. The time-frequency extent $(\sigma_t, \sigma_\nu)$ of
the analysis window remains constant but is described by a number of samples that grows as $\sigma_t/\Delta t \propto\sqrt{N}$ while $\sigma_t/T\propto 1/\sqrt{N}\to 0$. The analysis window is thus more and more finely resolved, and
we become close to a continuous description.
% zeros
In parallel, the expected number of zeros in the spectrogram of the white noise is $F_sT/2$ and
tends to $\infty$ as $N$ grows. Therefore, assuming perfect zero detection, statistics such as Ripley's $K$ function
or the variance-stabilized $L$ functional statistic of Section~\ref{s:LAndK} can
be asymptotically perfectly well estimated. 

In practice, we defined a numerical
zero as a local minimum among its eight neighbouring bins, and found that the
number of zeros was consistent with what we expected from
Proposition~\ref{p:propertiesOfPlanarGAF}, even if we did not impose a threshold
on the value of the spectrogram at the local minimum. 

We leave this section on a mathematical note. In this section, we implicitly assumed that in the limit on an infinite
observation window and an infinite sampling rate, the discrete Fourier
transforms involved in the computation of the discrete spectrogram converge to
their continuous counterpart. For the sake of completeness, we mathematically
justify in what sense this convergence can be expected. With the notation of Section~\ref{s:real},
subdivide again $[0,T]$ into $N$ equal intervals and denote by $\chi_{n}$ the indicator of the $n$th
interval $[(n-1)\Delta t, n\Delta t]$. Let $P_{N,T}:\cS\rightarrow L^2$ attach
to a Schwartz function $f$ the ``sampled'' simple function $\sum_{n=1}^N
f(n)\chi_n$. Then $P_{N,T}f\rightarrow f$ in $L^2$ as $T$ and $N$ go to infinity
and $T/\sqrt{N}\rightarrow \alpha>0$, which is the setting described above in
this section. On the other hand, 
\begin{equation}
\label{e:dft}
\langle \fomega, P_{N,T}M_vT_u g\rangle = \sum_{n=1}^N\langle \fomega,\chi_n
\rangle e^{-2i\pi v n\Delta t} g(n\Delta t-u)
\end{equation}
is what we call the discrete STFT at $(u,v)$ of a realization of white
noise. Note that in distribution, $(\langle \fomega,\chi_n \rangle)_n$ is a
sequence of i.i.d. Gaussians with variance $\Delta t$. To see how \eqref{e:dft} is a good
approximation to our initial continuous STFT, we note that for all $u,v$,
\begin{eqnarray*}
\mathbb{E}_{\mu_1}\vert\langle \fomega,  M_{v}T_u g\rangle - \langle \fomega,
  P_{N,T} M_{v}T_u g\rangle\vert^2 &=& \mathbb{E}_{\mu_1}\vert\langle \fomega,
                                       M_{v}T_u g
  - P_{N,T} M_{v}T_u g\rangle\vert^2\\
&=& \Vert  M_{v}T_u g
  - P_{N,T} M_{v}T_u g \Vert^2_{L_2} \rightarrow 0.
\end{eqnarray*}


%% When a signal is present  %% ICI
\subsubsection{Zeros of signal plus noise}
\label{s:signalplusnoise}

When a signal is present, its specific scales destroy the scale invariance
property of Gaussian white noise and deprives us from any asymptotic regime in
our numerical simulations. Let $A_S$ denote the typical time and frequency area
occupied by the considered signal. The presence of this signal creates a region
of the spectrogram of size $A_S$ where a decrease in the number of zeros is
expected due to the positive amount of energy corresponding to the signal. This
decrease is clearly visible in the spectrograms of Figure~\ref{f:testPower} for
linear chirps with various $A_S$ and various signal-to-noise ratios (SNR).
% Basis for future tests
The approach proposed here to build statistical detection tests is based on this
intuition. To this purpose one needs to quantify how far the presence of a
signal can influence the statistics used in our tests so that we can maximize
this influence and the efficiency of the proposed test.

% F_s, T...number of zeros and \eta_t, \eta_\nu...
Given a sampling rate $F_s$ and a duration of observation $T$, the unit
intensity in Proposition~\ref{p:propertiesOfPlanarGAF} yields that the expected
number of zeros in the spectrogram of a real white noise is $F_s\cdot T/2 = N/2$, neglecting what happens at small frequencies close to the time axis. Note that this is independent of the width $(\sigma_t,\sigma_\nu)$ of the Gaussian
analysis window $g$. % since it is proportional to $1/\sigma_t\sigma_\nu=2\pi$. 
If one wants to increase the number of zeros in the spectrogram to get
better statistics, it is enough to increase either $F_s$ or $T$. However, the
expected decrease in the number of zeros due to the presence of a signal is of
the order of the area $A_S$, the finite time-frequency area $A_S$ corresponding
to the spectrogram of the signal alone. As a consequence, an excessive increase
in either $F_s$ and/or $T$ would result in an asymptotically complete dilution
of the influence of the signal on the considered statistics. Thus, our purpose
is to build statistics over one or more patches $P$ of the spectrogram of maximal area
$A_P=\eta_t\eta_\nu$ such that $A_S/A_P\simeq 1$. On one hand, a maximal area
$A_P$ is necessary to ensure that the estimate of the chosen statistic be as
accurate as possible (in particular in the presence of noise only, to take into
account as many zeros as possible and minimize the false positive detection
rate); on the other hand, this statistic will be more sensitive to the presence
of a signal if it mostly depends on the influence of the signal on the
distribution of zeros in the spectrogram (in particular, in the presence of signal, we maximize
the true positive detection rate). In practice, note that one can hope to detect only signals
such that $A_S\gg \sigma_t\sigma_\nu=1/2\pi$, which means signals with a
time-frequency support that affects more than $\sigma_t/\Delta t\cdot \sigma_\nu/\Delta \nu = K/2\pi$ samples of the spectrogram.

%The proposed approach cannot be efficient if tests are built on statistics that mostly depend on the noise and will be more efficient if these statistics are significantly 

%characterized by $(\sigma_t,\sigma_\nu = 1/2\pi\sigma_t)$. To ensure that the footprint of $g$ in the spectrogram is isotropic, we have imposed~(\ref{eq_isotropy}) which is equivalent to $F_s/\sigma_\nu = T/\sigma_t$.

%



% \subsection{Sampling GAFs}
% \label{s:sampling}
% The convergence of the truncated GAFs $p_N(z) = \sum_{k=0}^N
% a_kz^k$ in \eqref{e:symmetricPlanarGAF} and \eqref{e:planarGAF} is uniform on
% compact sets. By Hurwitz' theorem \cite{}, the zeros of the truncated GAF are a
% good approximation to those of the GAF \textcolor{red}{I will make this precise later.} 

% Simulating the zeros of truncated GAFs $p_N(z) = \sum_{k=0}^N
% a_kz^k$ can be done through various methods. We have found it more stable to
% diagonalize companion matrices, noting that the roots of the polynomial
% $p_N(z)$ are the eigenvalues of \textcolor{red}{XXX, also in practice, this approach
%   has some limit. FFT on discrete white noise seems to be the most scalable in
%   the end. Not sure this section is necessary anymore, actually.}.

\documentclass[letterpaper, 10pt, conference, english]{ieeeconf}   % Comment this line out
                                                          % if you need a4paper
%\documentclass[a4paper, 10pt, conference]{ieeeconf}      % Use this line for a4
                                                          % paper

%\IEEEoverridecommandlockouts                              % This command is only
                                                          % needed if you want to
                                                          % use the \thanks command
\overrideIEEEmargins
% See the \addtolength command later in the file to balance the column lengths
% on the last page of the document

%\documentclass[conference]{IEEEtran}
%\usepackage[latin9]{inputenc}
%\usepackage[letterpaper]{geometry}
%\geometry{verbose,tmargin=1in,bmargin=1in,lmargin=1in,rmargin=1in}
%\usepackage[cmex10]{amsmath}
%\usepackage{amssymb}
\usepackage{graphicx}
%%\usepackage[dvipdfmx]{graphicx}
%\usepackage{float}
%\usepackage{array}
%\usepackage{tikz}
%\usepackage{amssymb}
\usepackage{epstopdf}
%%\usepackage{ifpdf}
%\usepackage{cite}
%\usepackage{algorithm}
%\usepackage{algorithmic}
%\usepackage{amsthm}
%
%\DeclareGraphicsExtensions{.eps,.pdf,.jpg,.png}
%\DeclareGraphicsRule{.jpg}{eps}{.bb}{}
%%%%%%\newtheorem{lemma}{Lemma}
%%%%%%\newcommand{\ind}[1]{\mathbf{1}\left(#1\right)}
%%%%%%\newcommand{\bx}{\mathbf{x}}
%%%%%%\newcommand{\E}{\mathbf{E}}
%%%%%%\newtheorem{theorem}{Theorem}%[section]
%%%%%%\newenvironment{definition}[1][Definition]{\begin{trivlist}
%%%%%%\item[\hskip \labelsep {\bfseries #1}]}{\end{trivlist}}
%\hyphenation{op-tical net-works semi-conduc-tor}

\usepackage{amsmath,amsfonts, amssymb}
\newtheorem{problem}{Problem}
\newtheorem{theorem}{Theorem}
\newtheorem{corollary}{Corollary}
\newtheorem{lemma}{Lemma}
\newtheorem{remark}{Remark}
\newtheorem{definition}{Definition}
\newtheorem{example}{Example}
\newtheorem{algorithm}{Algorithm}
%\newtheorem{proof}{Proof}

% The following packages can be found on http:\\www.ctan.org
%\usepackage{graphics} % for pdf, bitmapped graphics files
\usepackage{epsfig} % for postscript graphics files
\usepackage{psfrag}
%\usepackage{mathptmx} % assumes new font selection scheme installed
%\usepackage{times} % assumes new font selection scheme installed
%\usepackage{amsmath} % assumes amsmath package installed
%\usepackage{amssymb}  % assumes amsmath package installed

\def\endtheorem{\hspace*{\fill}~\QEDopen\par\endtrivlist\unskip}
\def\endlemma{\hspace*{\fill}~\QEDopen\par\endtrivlist\unskip}
\def\endcorollary{\hspace*{\fill}~\QEDopen\par\endtrivlist\unskip}
\def\endexample{\hspace*{\fill}~\QEDopen\par\endtrivlist\unskip}
\def\endremark{\hspace*{\fill}~\QEDopen\par\endtrivlist\unskip}
\def\enddefinition{\hspace*{\fill}~\QEDopen\par\endtrivlist\unskip}

\usepackage{times}
\usepackage[tight,footnotesize]{subfigure}
\usepackage{bibspacing}
%\setlength{\bibspacing}{\baselineskip}

\newcommand{\PA}[1]{$\spadesuit$\footnote{PAJA: #1}}
\newcommand{\FM}[1]{$\clubsuit$\footnote{FEI: #1}}
\newcommand\fixme[1]{$\star$ \emph{\small #1} $\star$}

\newcommand{\tableref}[1]{Table~\ref{tab:#1}}
\newcommand{\figref}[1]{Figure~\ref{fig:#1}}


%%%%%%%%%%%%%%%%%%%%%%%
\usepackage{verbatim}
\usepackage{amsmath}
\usepackage{babel}
\usepackage{listings}
%%%%%%%%%%%%%%%%%%%%\usepackage[]{graphicx}
\usepackage{comment}
\usepackage{multirow}
\usepackage{algorithmic}
\usepackage{algorithm}

%%%% Math Packages %%%%%%%%%%%%
%%\usepackage{amsmath}
%%\usepackage{amsthm}
%%\usepackage{amsfonts}

\usepackage[tight]{subfigure}
%%%%%%\usepackage{wrapfig}


%% Line Spacing %%%%%%%%%%%%%%%%%%%%%%%%%%%%%%%%%%%%%%%%%%%%%
%\usepackage{setspa ce}
%\singlespacing        %% 1-spacing (default)
%\onehalfspacing       %% 1,5-spacing
%\doublespacing        %% 2-spacing
%\usepackage[belowskip=-15pt,aboveskip=0pt]{caption}
%\setlength{\intextsep}{10pt plus 2pt minus 2pt}
%\allowdisplaybreaks
\begin{document}
\title{\LARGE \bf Stochastic Game Approach for Attack Detection}
\author{Fei Miao \and Quanyan Zhu%\and Rahul Mangharam\and George J. Pappas% <-this % stops a space
%\thanks{This research has been partially supported by the NSF-CPS 0931239 grant.}% <-this % stops a space
%\thanks{F. Miao, M. Pajic, R. Mangharam and G. J. Pappas are with the Department of Electrical and Systems Engineering, University of Pennsylvania, Philadelphia, PA, USA 19014. Email: \{{\tt miaofei,pajic,rahulm,pappasg\}@seas.upenn.edu}.}%
}
\maketitle

\input{model_replay}

\section{A Hybrid Stochastic Game Model}
\label{sec:game_form}
To obtain a switching policy that minimizes the expected real-time worst case payoff for the given subsystems, 
we formulate a zero-sum, hybrid stochastic game between the system and the attacker. System dynamics knowledge are combined with the game definition, and the quantitative process for the game parameters will be introduced in this section. We assume that one game stage $k$ is also one time step of the physical system. The total stage number is $K$. The hybrid game state space $(X_{[k-T,k]}\times S)$ contains information about both the system dynamics $\mathbf{x}_k$ and the discrete modes $\delta_l, l=1,2,3$. Here, $T$ is the window size of system dynamics needed to keep the state transition between stages $k$ and $(k+1)$ Markov. The joint state includes information we need to compute the game strategy at the current stage. This is the main difference compared with the previous work~(\cite{cdc_replay}), while the latter is not Markov since it needs to consider all the possible histories of strategies for deciding the physical dynamics and getting a strategy. At each stage $k \in \{T,\cdots, K+T\}$, parameters include the action space for the attacker (system) $A_{t}$ ($A_{s}$), the state transition probability matrix $\mathbb{P}_{k}$, and the immediate payoff matrix $r_{k}$. The solution set of the game is mixed strategies $\mathbf{F}_{k}$ for the attacker, and $\mathbf{G}_{k}$ for the system. Formally, the game is defined as a sequence of tuples:
 $\{(X_{[k-T,k]} \times S),A_{t},A_{s}, \mathbf{F}_{k},\mathbf{G}_{k}, P, r\}$.

\iffalse
\begin{table*}
\centering
%\caption{Parameters of the hybrid stochastic game between the system and the attacker}
\begin{tabular}{|c|c|}
  \hline
   $s_{kl}=(x_{[k-T,k]}, \delta_l)$& Joint game state: sequence of physical dynamics, and piecewise constant mode \\ \hline
   $A_{t}$ & Attacker's action space \\ \hline
   $A_{s}$ & System's action space \\ \hline  
   $\mathbf{f}_k(s_{kl})$ & Strategy of the attacker at stage $k$, 
                                           state $s_{kl}, l=1,2,3$  \\ \hline     
      $\mathbf{g}_k(s_{kl})$ & Strategy of the system at stage $k$, 
                                            state $s_{kl}, l=1,2,3$  \\ \hline     
    ${P}(s_{(k+1)h}|s_{kl})$ & Probability that system transits from state $s_{kl}$ 
                                              at stage $k$ to state $s_{(k+1)l}$ at stage $k+1$\\ \hline
    ${r}(s_{kl})$ & Immediate payoff matrix at stage $k$ \\ 
   \hline
\end{tabular}
\centering
%\captionsetup{justification=centerlast}
\caption{Parameters of the hybrid stochastic game between the system and the attacker}
\label{game_parameter}
\end{table*} 
\fi

\textbf{Game State Space}: The joint state of the system at stage $k$ is described by the pair $s_{kl}=(x_{[k-T,k]}, \delta_l)$, where
\centerline{$
x_{[k-T,k]}=(x_{k-T}, x_{k-T+1}, \cdots, x_k ) \in X_{[k-T,k]}$} is the discrete-time dynamics of the physical process provided to the system--the state estimations $\hat{x}_{k-T},\cdots, \hat{x}_k$, $\delta_l \in S=\{\delta_1,\delta_2, \delta_3\}$ denote the cyber state of the system. We assume that once the game reach $\delta_1$, the system wins and will not enter other modes till next game, i.e., $\delta_1$ is an absorbing state. The moving-horizon transition of the joint states on stage axis is shown as Figure~\ref{sg}. The window size of system dynamics $T$ keeps the state transition between time $k$ and $k+1$ Markov. For instance, if the detector of the system requires system dynamics $\hat{x}_{[k-T_1,k]}$, and we consider sensor data injection attacks and replay attacks with replay windows less than $T_2$ steps, then $T=max\{T_1, T_2\}$. 
%With the system dynamics $\hat{x}_{[k-T, k]}$ denoted as $x_{[k-T,k]}$ for game stage $k$,  information needed to define the following action space of two players, the payoff and state transition probability is included. 

%i.e., the detector needs information for T steps to decide whether the alarm should be triggered. 
%\FM{In replay attack, we can say: once alarm is triggered, the system can stop the execution and check whether attack occurred, is this true for other attacks? Can the system distinguish between successfully detection and false alarm trigger?}
\begin{figure}[t!]
%\vspace{-5pt}
\centering
\includegraphics [width=0.32\textwidth]{xk.pdf}
\vspace{-8pt}
\caption{Joint state transition of the hybrid stochastic game when moving the horizon of game state one step ahead. When the state transits from stage $k$ to $k+1$, we slice the window of the sequence of physical dynamics one step ahead, add $x_{k+1}$ and remove $x_{k-T}$,  thus $x_{[k-T,k]} \to x_{[k-T+1,k+1]}$. The piecewise constant modes $\delta_l$, $\delta_h$ describe the cyber states provided by the detector at stage $k$, respectively.}
\label{sg}
\vspace{-5pt}
\end{figure}

%For simplicity, we omit the subscript k and just write state at every time k as $s_i,  i=1,2,3$.
\textbf{Attacker's Action Space}: We assume that the system is vulnerable to different attack models described by the action space $A_{t}$, where 
\\\centerline{$
A_{t}=\{a_{1}(x_{[k-T,k]}), a_{2}(x_{[k-T,k]}), \cdots, a_{M}(x_{[k-T,k]})\}
$}
is the attacker's action space at stage $k$, and $a_{1}$ means no attack. Here we only consider discretized action space of the attacker for computational efficiency. For the LTI system dynamics considered in this work, the distance of a continuous point to its nearest discrete point in action space is bounded. With bounded error of the dynamics by discretized continuous action space, the quality of game solutions under different conditions is analyzed by work~\cite{disaction}. 

The actions can describe both multiple types of attacks and the same type attack with different values. For instance, when considering only sensor data injection attacks with different norms of injection value, we will denote $a_i (x_{[k-T,k]}), i=2, 3,\dots$ as changing the sensor value from $\mathbf{y}_k=\mathbf{Cx}_k+\mathbf{v}_k$ to $\mathbf{y}'_k= \mathbf{y}_k+\mathbf{y}_{k,i}^a$, where any injection $\mathbf{y}_k^a$ is classified as $a_i (x_{[k-T,k]}), i=inf\{i:\ \|\mathbf{y}_k^a-\mathbf{y}_{k,i}^a\|_2\}$ in attacker's action space.
Similarly, for replay attack only, the action space is discretized as changing sensor values from $\mathbf{y}_k=\mathbf{Cx}_k+\mathbf{v}_k$ to $\mathbf{y}'_k= \mathbf{y}_{k-T_i}$ for action index $a_i (x_{[k-T,k]})$, where any replay time length $T_a$ is classified as $a_i (x_{[k-T,k]}), i=inf\{i:|T_a - T_i|\}$. Considering multiple types of attacks, we assume that the system is valnerable under $m_a$ types of attacks, and attack type $A_i$ is corresponding to $M_{a,i}$ discretized actions in the action space, then there are $\sum_{i=1}^{m_a} M_{a,i}+1$ actions in total within the attacker's action space $A_t$.
 

% with discretized, bounded norm $\|\mathbf{y}_{k,i}^a\|_2 \leqslant b$, since the attacker has limited energy for every data injection. This means any injection data that satisfies $\|\mathbf{y}_{k}^a\|_2 \leqslant b$ is considered as $inf\{i:\ \|\mathbf{y}_k^a-\mathbf{y}_{k,i}^a\|_2\}$ in attacker's action space. 

%For example, 
%when considering replay attacks and false data injection attacks, we take $a_{2k}$ as~\eqref{replay_y} for a given replay window size, and $a_{3k}$ as~\eqref{attackmodel} for a given data injection range. 

\iffalse by any given controller/estimator/detector combination of the system \fi
%; here, $y_{k}$ is the real sensor value and $\mathbf{y}_{k-t_{i}}$ denotes any replay sensor value in the strategy set. %\FM{here we assume during time $k \in \{1,...,K\}$ the replay window size $T$ does not change for simulation.}
%\item
%\FM{rewrite the action space definition}

\textbf{System's Action Space}: The system's action space at stage $k$ is defined as
\\\centerline{$
A_{s}=\{u_{1}(x_{[k-T,k]}), u_{2}(x_{[k-T,k]}),\cdots, u_{N}(x_{[k-T,k]})\},
$}  %= \{\mathbf{u}^{*}_{k}, \mathbf{u}^{*}_{k} + \Delta \mathbf{u}_{k}\}$
where $u_{j}$ is the index for the $j$th subsystem. We assume that the $N$ subsystems (a model for each component in Figure~\ref{system}) are determined priorly. For example, a subsystem can be the plant with a given optimal LQG controller, a Kalman filter and a $\chi^2$ detector. A subsystem can also be the plant with an optimal LQG controller, a resilient state estimator~\cite{res_estimator} and its corresponding estimation residual checking component. We assume that the attacker's action space is defined, with corresponding system's action or a subsystem that the detection rate is greater than $0$. A switched system does not ensure performance under the attack outside the action space of the game.

\textbf{Mixed Strategy}: Let $f^{i}_{k}(s_{kl})$ ($g^{j}_{k}(s_{kl})$) be the probability that the attacker (system) chooses action $a_{i}(x_{[k-T,k}) \in A_{t}$ ($u_{j}(x_{[k-T,k}) \in A_{s}$) at state $s_{kl}\in (X_{[k-T,k]}\times S)$. Define $\mathbf{F}_{k}$ and $\mathbf{G}_{k}$ as the mixed strategy sets of the attacker and the system for stage $k$:
$\mathbf{F}_{k} :=\{\mathbf{f}_{k}= [\mathbf{f}_{k}(s_{k1}), \mathbf{f}_{k}(s_{k2}), \mathbf{f}_{k}(s_{k3})]
|f_{k}^{i}(s_{kl})\geq 0, \mathbf{f}_k \in [0,1]^{M\times 3},
\sum \limits_{a_{ik} \in A_{tk}}f_{k}^{i}(s_{kl}) = 1,\mathbf{f}_{k}(s_{kl})\in \mathbb{R}^{M}, \forall s_{kl} \in(X_{[k-T,k]}\times S)\},$
$\mathbf{G}_{k}:=\{\mathbf{g}_{k}= [\mathbf{g}_{k}(s_{k1}), \mathbf{g}_{k}(s_{k2}), \mathbf{g}_{k}(s_{k3})]|$
$g_{k}^{j}(s_{kl})\geq 0, \mathbf{g}_k \in [0,1]^{N \times 3}, %&\forall u_{jk} \in A_{sk},%k \in \{1,...,K\},\\
\sum \limits_{u_{jk} \in A_{sk}}g_{k}^{j}(s_{kl}) = 1, \mathbf{g}_{k}(s_{kl}) \in \mathbb{R}^{N}, \forall s_{kl} \in (X_{[k-T,k]}\times S)\}. $ Note that $\mathbf{x}_{[k-T,k]}$ provides exogenous information for the strategy $\mathbf{f}_k (\mathbf{g}_k)$, since for every $l$, $\mathbf{f}_{k}(s_{kl}) (\mathbf{g}_{k}(s_{kl}))$ is the strategy at mode $\delta_l$ for the same $\mathbf{x}_{[k-T,k]}$ at stage $k$. Hence, $\mathbf{g}_k$ and $\mathbf{f}_k$ are finite dimensional vectors, that the stationary strategy chosen by each player at stage $k$ depends on the cyber state. %Mixed strategy set $\mathbf{F}_k$ also include the case that the attacker only implement one specific type of attack in the action space at time instance $k$, since we do not have know the strategy of the attacker, we are able to consider all possible combinations of attacks by exploiting mixed strategies. 

%\textbf{}:
%\label{dynamicgame}
%With all the above definition, %for any strategy history $h_{k}$ (a sequence of switching policy), 


\textbf{System and Subsystem Dynamics under game framework}: Given the subsystem and attack models in Section~\ref{sec:replay1} and the game definition,  
%we can transform it to a new system dynamic model decided by both players' actions. 
we show the dynamics at stage $k$ given an action pair $(a_{i}(x_{[k-T,k}),u_{j}(x_{[k-T,k}))$ (assume initial $\mathbf{\hat{x}}_{1|0}=\bar{\mathbf{x}}_{0}$, $\mathbf{x}_{1}=\mathbf{x}_{0}$). Each action pair $(a_{i}(x_{[k-T,k]}),u_{j}(x_{[k-T,k]}))$ defines the corresponding system dynamics at $k$. For instance, when we focus on sensor attacks (like replay or false data injection), let $\mathbf{\gamma}_{k}(a_{i}(x_{[k-T,k]}), u_{j}(x_{[k-T,k]}))$ be the control input with $(a_{i}(x_{[k-T,k]}),u_{j}(x_{[k-T,k]}))$, a subsystem $u_{j}(x_{[k-T,k]})$ with a Kalman filter, an optimal LQG controller has the following dynamics (we denote $(a_{i}(x_{[k-T,k]}),u_{j}(x_{[k-T,k]}))$ as $(a_{ik}, u_{jk})$ for convenience): 
\begin{align}
\begin{split}
&\mathbf{x}_{k}=\mathbf{Ax}_{k-1}+ \mathbf{Bu}_{k-1}+\mathbf{w}_{k-1},\\
& \mathbf{y}_{k}=\begin{cases}a_{1k} = \mathbf{Cx}_k+\mathbf{v}_{k},\ \text{without attack}\\
a_{ik}, i=2,\cdots, M, \ \ \text{with attack,} \end{cases}\\
&\hat{\mathbf{x}}_{k|k-1}= \mathbf{A\hat{x}}_{k-1|k-1}+\mathbf{Bu}_{k-1},\\
%&\mathbf{z}_{k+1}(h_{k},a_{ik},u_{jk})=a_{ik}(h_{k}) - \mathbf{C\hat{x}}_{k+1|k}(h_{k},a_{ik},u_{jk}),\\
&\hat{\mathbf{x}}_{k|k}(a_{ik}) =\hat{\mathbf{x}}_{k|k-1}+ \mathbf{K}(a_{ik} - \mathbf{C\hat{x}}_{k|k-1}),\\
%\end{split}
&\mathbf{\hat{x}}_{k+1|k}(a_{ik},u_{jk})=\mathbf{A\hat{x}}_{k|k}(a_{ik})+\mathbf{B\gamma}_{k}(a_{ik},u_{jk}),\\
& \mathbf{\gamma}_{k}(a_{ik}, u_{jk}) = \mathbf{L\hat{x}}_{k|k} (a_{ik}),\\%+\Delta \mathbf{u}_{k},\\
&\mathbf{z}_{k+1}(a_{ik},u_{jk})=a_{ik} - \mathbf{C\hat{x}}_{k+1|k}(a_{ik},u_{jk}).
\label{dynamicgame}
\end{split}
\end{align}
\textbf{State Transition Probability}: Given a set of subsystem models, define the state transition probability $P$ as a function of the state of the game and both players' actions $P:\ (X_{[k-T,k]}\times S) \times A_{t} \times A_{s}\to [0, 1],$
where
\\\centerline{$
P(s_{(k+1)h}|s_{kl},a_{ik}, u_{jk}), h=1,2,3
$}
%\end{align*}
is the probability that system transits from state $s_{kl}$ to state $s_{(k+1)h}$ at stage $k+1$, given both players' action $(a_{ik},u_{jk})$ at stage $k$. Given the current game state $s_{kl}=(x_{[k-T,k]}, \delta_l)$ and an action pair $(a_{ik},u_{jk})$, the dynamics of the system at stage $k+1$ is described as $x_{[k-T+1,k+1]}$ for all possible cyber modes $\delta_h \in S$, hence the dimension of state transition probability $P(s_{(k+1)h}|s_{kl},a_{ik}, u_{jk})$ is determined by the number of cyber modes of the game. We denote $P(s_{(k+1)h}|s_{kl}, a_{ik}, u_{jk})$ as $P^{ij}(s_{(k+1)h}|s_{kl})$ for short.
%and $\tilde{P}^{ij}(s_{(k+1)h}|s_{kl})$ is the entry at the $i$-th row and $j$-the column  of the state transition matrix $\tilde{P}(s_{(k+1)h}|s_{kl})$ of the game at hybrid state $s_{kl}$.
 As a state transition probability, this function should also satisfy
%\begin{align*}
\\\centerline{$\sum_{\delta_h \in S} {P}^{ij}(s_{(k+1)h}|s_{kl}) = 1,\quad \forall (a_{ik},u_{jk}) \in A_{t} \times A_{s},$}
\\\centerline{$s_{(k+1)h} \in (X_{[k-T+1,k+1]}\times S), s_{kl} \in(X_{[k-T,k]}\times S).$}
%\end{align*}
The transition probability is provided by intrusion detectors of the subsystem. 
%For computational efficiency, we assume that every element of the state transition matrix is a convex function of the system dynamics $x_{[k-T,k]}$ or can be convexified with bounded error. 
%For example, if a $\chi^{2}$ detector is the detector component of subsystem $u_{j}$, we apply~\eqref{alarm} to decide the state transition probability.

\textbf{Immediate Payoff Function}: The immediate payoff matrix at stage $k$ is a $\mathbb{R}^{M\times N}$ matrix for given game state and every action pair $(a_{ik}, u_{jk})$. We define the immediate payoff function as a continuous, convex function of the hybrid game state and the actions of both players
\\\centerline{$r: (X_{[k-T,k]}\times S) \times A_{t} \times A_{s} \to \mathbb{R}^{M \times N},$}
where $r(s_{kl}, a_{ik}, u_{jk}) \geqslant 0$ is the payoff at joint state $s_{kl}$ given action pair $(a_{ik}, u_{jk})$. For definition convenience, we denote ${r}(s_{kl}, a_{ik}, u_{jk})$ as ${r}^{ij}(s_{kl})$ for short, since it is the element on the $i$-th row and $j$-th column of the payoff matrix ${r}(s_{kl})$. It is a zero-sum game between the system and the attacker, and we assume the system is the minimizer and the attacker is the maximizer, hence the payoff function for the attacker and the system is defined as
\centerline{$
{r}^{ij}(s_{kl})={r}_t^{ij}(s_{kl})=-{r}_s^{ij}(s_{kl}).
$}
For instance, when the linear quadratic cost is a metric of system performance, let $\gamma_{k}(a_{ik}, u_{jk})$ be the control input given action pair $(a_{ik}, u_{jk})$, then the payoff function is defined as
\begin{align}
\begin{split}
{r}^{ij} (s_{k1}) =&\mathbb{E}[\mathbf{\hat{x}}^{T}_{k}]\mathbf{W}\mathbb{E}[\mathbf{\hat{x}}_{k}]+\mathbb{E}[\mathbf{\gamma}^{T}_{k}(a_{1k},u_{jk})]\mathbf{U}\mathbb{E}[\mathbf{\gamma}_{k}(a_{1k},u_{jk})],\\
{r}^{ij} (s_{k2}) =&\mathbb{E}[\mathbf{\hat{x}}^{T}_{k}]\mathbf{W}\mathbb{E}[\mathbf{\hat{x}}_{k}]+\mathbb{E}[\mathbf{\gamma}^{T}_{k}(a_{ik},u_{jk})]\mathbf{U}\mathbb{E}[\mathbf{\gamma}_{k}(a_{ik}, u_{jk})],\\
{r}^{ij} (s_{k3}) =& p_f,
\end{split}
\label{payoff}
\end{align}
where $p_f$ is the false alarm trigger penalty, the cost that the system needs to stop execution, check the reason of an alarm, and restart later; $\mathbf{x}_{k}$ is the physical state under the game framework. At mode $\delta_{1}$ the system wins, so the payoff is a normal system payoff with correct sensor data. The larger $p_f$ is, the less probable it is for the system to choose a strategy to transit to state $s_{k3}$.

\textbf{System dynamics update with strategies at stage k}:
 Let $p(s_{kl})$ be the probability system is at state $s_{kl}$ at stage $k$. The initial state distribution $p(s_{1l})$ is given. With  a strategy $\mathbf{f}_{k},\mathbf{g}_{k}$, the attacker and the system randomly sample an action pair $(a_{ik}, u_{jk})$ according to the probability distribution. Then, the control input and sensor value for calculating expectation cost are: 
%\begin{align*}
\centerline{$
\mathbf{u}_{k}=\sum\limits_{j=1}^{N}\sum\limits_{i=1}^{M} \sum\limits_{l=1}^{3}p(s_{kl})f_{k}^{i}(s_{kl})g^{j}_{k}(s_{kl})\mathbf{\gamma}_{k}(a_{ik},u_{jk}),
$}
$\text{ }\quad\quad\mathbf{y}_{k} =\sum\limits_{i=1}^{M}\sum\limits_{l=1}^{3} p(s_{kl})f_{k}^{i}(s_{kl}) a_{ik}.$
%\end{align*}
\\The probability that system is at state $s_{(k+1)h}$ for $k+1$ is:
\\\centerline{$
%\begin{align*}
p(s_{(k+1)h})= \sum\limits_{l=1}^{3}p(s_{kl})[\mathbf{f}_{k}(s_{kl})]^{T}{P}_{k}(s_{(k+1)h}|s_{kl})\mathbf{g}_{k}(s_{kl}). 
%\end{align*}
$}

\iffalse
\begin{align*}
\mathbf{F}_{k} :=\{&\mathbf{f}_{k}= [\mathbf{f}_{k}(s_{k1}), \mathbf{f}_{k}(s_{k2}), \mathbf{f}_{k}(s_{k3})]
|f_{k}^{i}(s_{kl})\geq 0,\\& \mathbf{f}_k \in [0,1]^{M\times 3} %&\forall a_{ik} \in A_{tk},  %k\in \{1,...,K\},\\
\sum \limits_{a_{ik} \in A_{tk}}f_{k}^{i}(s_{kl}) = 1,\mathbf{f}_{k}(s_{kl})\in \mathbb{R}^{M},\\&\forall s_{kl} \in(X_{[k-T,k]}\times S)\},\\
\mathbf{G}_{k}:=\{&\mathbf{g}_{k}= [\mathbf{g}_{k}(s_{k1}), \mathbf{g}_{k}(s_{k2}), \mathbf{g}_{k}(s_{k3})]|
g_{k}^{j}(s_{kl})\geq 0,\\& \mathbf{g}_k \in [0,1]^{N \times 3}, %&\forall u_{jk} \in A_{sk},%k \in \{1,...,K\},\\
\sum \limits_{u_{jk} \in A_{sk}}g_{k}^{j}(s_{kl}) = 1, \mathbf{g}_{k}(s_{kl}) \in \mathbb{R}^{N},\\ &\forall s_{kl} \in (X_{[k-T,k]}\times S)\}. 
\end{align*} 
\fi

\begin{algorithm}[!ht]
\begin{algorithmic}[1]
\Require Query workload $Q$, event stream $I$, \app\ graph $G$, hash table of snapshots $S$
\Ensure Hash table of results $R$ 
\State $G \leftarrow \emptyset$, $S, R \leftarrow$ empty hash tables
\ForAll {event $e \in I$ with $e.type=E$} 
    \State $//$ \textbf{\app\ graph construction}
    \ForAll {$q \in Q$ \text{ with event types }T}
        \ForAll {$E' \in T,\ E' \neq E$}
            \State $G_{E'} \leftarrow \mathit{getGraphlet}(G,E')$,
            $G_{E'}.\mathit{active} \leftarrow \mathit{false}$
        \EndFor
    \EndFor
    \If {\textbf{not} $G_E.\mathit{active}$}
        \State $G_E \leftarrow \mathit{createGraphlet()}$, $G_{E}.\mathit{active} \leftarrow \mathit{true}$,
        $G \leftarrow G \cup G_E$
        \If {$G_E.\mathit{shared}$ by $Q_E \subseteq Q$}
            $x \leftarrow \mathit{createSnapshot()}$ 
            \ForAll {$q \in Q_E$}
                \ForAll{$E' \in \mathit{pt}(E,q), E' \neq E$}
                    \State $G_{E'} \leftarrow \mathit{getGraphlet}(G,E')$
                    \State $S(x,q) \leftarrow S(x,q) + sum(G_{E'},q)$ \hspace{0.5cm}$//$ Eq.~5
                \EndFor
            \EndFor
        \EndIf    
    \EndIf
    \State insert $e$ into $G_E$
    \State $//$ \textbf{Trend count computation}
    \If {$G_E.\mathit{shared}$ by $Q_E \subseteq Q$}
        \If {$\forall q \in Q_E\ pe(e,q)$ are identical}
            \State $count(e,Q_E) \leftarrow count(e,q)$ \hspace{2.3cm}$//$ Eq.~2
        \Else\ $y \leftarrow \mathit{createSnapshot()}$, $count(e,Q_E) = y$
            \ForAll {$q \in Q_E$}
                $S(y,q) \leftarrow count(e,q)$ \hspace{0.2cm}$//$ Eq.~2
            \EndFor
          \EndIf
    \Else\ $count(e,q)$ \hspace{5.2cm}$//$ Eq.~2
    \EndIf
    \ForAll{$q \in Q$}
  	    \If {$E \in \mathit{end}(q)$} 
  		    $R(q) \leftarrow R(q) + count(e,q)$ $//$ Eq.~3
        \EndIf
    \EndFor
\EndFor
\State \Return $R$
\end{algorithmic}
\caption{\app\ shared online trend aggregation}
\label{algo:snapshot-propagation}
\end{algorithm}


\section{Simulation}

\subsection{motivation} %
The most efficient way to figure out the answers to the questions we posed in the introduction is to deploy the proposed framework on a real-world platform and analyze how users adopt different and complex privacy policies to optimize their rewards.
However, direct deployment of these strategies and investments is currently impractical due to the following reasons.


Firstly, the most important reason is that such an online experiment may lead to the decline of the recommendation performances as well as the user experience, which harms the platform's revenue.
In the real world, nearly all the companies determine their platform mechanism driven by interest, and the revenues of the platforms are highly correlated with the recommendation performances. 
Therefore, it's nearly impossible to persuade any platform to directly deploy proposed strategies and mechanism online without other benefits.  



Secondly, the experiments are \czq{built} upon several simplifications, mentioned in \cref{assumptions}, which poses challenges towards recommendation model training process.
For example, we assume when a user \czq{adjusts} his data disclosure policy, the recommendation system will forget his un-disclosed data. 
To facilitate such challenges, model unlearning or other privacy-preserving technologies are imposed.
However, in real-world applications, very few the e-commercial platforms have deployed these privacy-preserving technologies during the deep recommendation model training and evaluation processing.
As a result, we may still fail to guarantee the assumptions and simulation methodology becomes a substitution.










In summary, inspired by the success of simulation study on dynamic interactive problems in real-world applications~\cite{Ie:arxiv19:RecSim,krauth2020offline,lucherini2021t,yao2021measuring},  we employ the simulation to study the effects of the proposed framework and the possible game between users and the platform.















\subsection{Simplified Assumptions}
\label{assumptions}

To simplify the simulation process for easier analysis, we make some necessary assumptions to simplify the problem.

\begin{assumption}[Static Assumption] User $i$ optimizes her/his policy on the fixed data $\di{}$ which is not affected by user policy $\pi_i$.
\label{assumption:static}
\end{assumption}

Here static means the user data $\di{}$ is fixed during the simulation, but the disclosed data $\si{}$ produced by different user policies is dynamic. 
It is also the most common setting for recommendation task in research papers~\cite{Rendle:www10:Factorizing,Hidasi:ICLR2016:gru4rec,NCF,kang2018self,Sun:cikm19:BERT4Rec}.
In the simulation, we train the recommendation system $\texttt{M}_{{\scriptscriptstyle \mathcal{S}}}$ on the collected dynamic data $\mathcal{S}$ and validate the recommendation efficiency on a fixed test set. 
In real-world applications, the data $\di{}$, which contains the behavior data from the interaction with the recommender $\texttt{M}_{{\scriptscriptstyle \mathcal{S}}}$, is also dynamically changing with the user's policy $\pi_i$.
It is beyond the scope of this paper and we leave it as the future work. 


\begin{assumption}[Immediate Assumption] The recommendation model $\texttt{M}_{{\scriptscriptstyle \mathcal{S}}}$ can only use the data $\si{}$ currently disclosed by each user $i$.
\label{assumption:forget}
\end{assumption}
The motivation of this assumption is that an untrusted platform can leverage user $i$' all data $\di{}$ if it can use the data disclosed in previous actions.
Without this constraint, the privacy right discussed in this paper is meaningless.
To achieve this, the platform can retrain the model from scratch with new data $\si{'}$ or quick unlearn the data in $\si{}$ then finetune with data $\si{'}$~\cite{cao2015towards,bourtoule2021machine,chen2022recommendation}.


However, the \cref{assumption:forget} also raises a new challenge that the asynchronous changes of user policy bring intractable computation costs for the platform since each time the user changes the disclosed data, the platform needs to update the model.
Here, we make an assumption for simplifying the simulation, assuming all users realize that the platform will cyclically (e.g., once a day) collect their privacy decisions and update recommender systems.
\begin{assumption}
[%
Cyclical Assumption]
Platform cyclically collects user privacy choices, and then the platform updates the model using all newly disclosed data. 
\label{assumption:synchronization}
\end{assumption}



In summary, for easy analysis in simulations, we introduce these assumptions to ignore the time and dynamic effects in this feedback system, just like the traditional recommendation task formulation.
























\subsection{Platform Mechanism Simulation}
\label{sec:plat_mech}
In order to validate the effect of the platform mechanism, we adopt several mechanisms during simulation. 
For easy comparison, we utilize one mechanism at each experiment. %


\subsubsection{\textbf{Data Split Rule}}

\czq{In our simulation, we do not split the profile attribution and the user can determine whether to share all of their attributes.}
For behavior data, we apply ``percentage split'' as $\delta_b$ with different split granularity $p$ (e.g., 1/3) to split the behavior sequence into $1/p$ parts. 
One obvious advantage of ``percentage split'' is that it can normalize the size of user action space and decrease the inconvenience of the interaction between the user and the platform.

\subsubsection{\textbf{Data Disclosure Strategy}}
\label{sec:data_disclose_choice}

As the platforms have certain flexibility to implement different data disclosure strategies, we discuss three representative disclosure strategies used in our study for behavior data in this subsection.
These strategies determine the data disclosure action space $\Pi$ the user can choose.
For profile attributes, we found that all users tend not to disclose them in the experiments since these features are negligible for improving recommendation utility in the presence of behavior data.
Similar result that user profile features contribute very marginal to the recommendation results in the case of strong user behavior modeling on public benchmark datasets has also been reported in other works~\cite{kang2018self,Sun:cikm19:BERT4Rec}.
Thus, in the following study, we mainly focus on modeling only  behavior data.

The ``\textit{separate}'' rule gives the users the control to freely disclose any split personal data.
For this rule, the size of user $i$'s the action space is exponentially expended on the size of the spilt data set $|\delta_b ({\scriptstyle \mathcal{D}_{i,b}})|$, denoted as $2^{|\delta_b ({\scriptstyle \mathcal{D}_{i,b}})|}$. 
However, too many choices might make it difficult for users to make better data disclosure decisions.

Another data disclosure strategy named ``\textit{oldest continuous}'' provides users the choices to disclose continuous behavior data from the beginning time, such as selecting ``the oldest 33\% data''.
In this strategy, to disclose newer behavior data ${\scriptstyle \mathcal{S}_{i,bj}}$, users must also disclose all behavior data before it.
Take an already split data $\delta_b ({\scriptstyle \mathcal{D}_{i,b}}) = \{\scriptstyle \mathcal{S}_{i,b1}, \scriptstyle \mathcal{S}_{i,b2}, \scriptstyle \mathcal{S}_{i,b3}\}$ as an example, the action space provided by oldest continuous strategy is $\Pi = \{[0,0,0], [1,0,0], [1,1,0], [1,1,1]\}$, and its corresponding disclosed data is $\{\varnothing ,
    \{\! {\scriptstyle \mathcal{S}_{i,b1}} \!\},
    \{\! {\scriptstyle \mathcal{S}_{i,b1}} , {\scriptstyle \mathcal{S}_{i,b2}}\! \},$
    $\{ {\scriptstyle \mathcal{S}_{i,b1}}, {\!\scriptstyle \mathcal{S}_{i,b2}},$ ${\scriptstyle \mathcal{S}_{i,b3}}\} \}$.
``\textit{Latest continuous}'' mechanism is similar to ``oldest continuous'', with the only difference in the opposite direction.
The size of these two mechanisms' action spaces is $|\delta_b ({\scriptstyle \mathcal{D}_{i,b}})|$.









\subsection{User Policy Simulation}
\label{sec:user}


In this subsection, we introduce the simulation of user policy in our proposed framework.
As defined in \cref{eq:S_i}, the disclosed data $\si{}$ is result of the platform mechanism $\mathrm{G}$ and user's disclosure policy $\pi_i$. 
Meanwhile, in \cref{eq:updated_rec}, the recommendation utility $\texttt{U}_i( \si{})=\texttt{U}'(\si{},\, \texttt{M}_{{\scriptscriptstyle \mathcal{S}}})$ is also determined by the recommendation model $\texttt{M}_{{\scriptscriptstyle \mathcal{S}}}$, which is \czq{built} upon the all users' disclosed data $\mathcal{S}$. 
The reward of user $i$ may be varied even when $i$ keeps the disclosed data $\si{}$ unchanged since other users might change their disclosed data and the recommender system is changed.
Thus, the expectation rewards are considered rather than immediate value defined in \cref{eq:framework} and we assume all the users are rational and seek for the optimal privacy disclosure action  $\alpha_i^*$ to the optimal expected reward $E[ \texttt{R}_i | \alpha_i ] $ as his policy, i.e., %
\begin{equation}
    \begin{aligned}
    \alpha_i^{*} &= \argmax_{\alpha_i \in \Pi} E[ \texttt{R}_i | \alpha_i ]=  \argmax_{\si{\in [ \Pi \otimes \mathrm{\delta}(\di{}) ] }} E[ \texttt{R}_i(\si{)} ] \\
& =\argmax_{\alpha_i \in \Pi } E\Bigl[ -\lambda_i \texttt{C}_i\bigl( \alpha_i \otimes \mathrm{\delta}(\di{}) \bigr) + \texttt{U}_i\bigl( \alpha_i \otimes \mathrm{\delta}(\di{}) )\bigr) \Bigr].
    \end{aligned}
\label{eq:opt_pi}
\end{equation}

As mentioned before, recommendation utility $\texttt{U}_i$ has been discussed in \cref{sec:platform_obj}.
To study this objective, we need to define the privacy cost function $\texttt{C}_i$ and sensitive weight $\lambda_i$.



\subsubsection{\textbf{Privacy Cost Function}}
\label{sec:privacy_cost}
We simulate every user with the same cost function $\texttt{C}$ and leave the diversity of user privacy sensitivity to the parameter $\lambda_i$. 
Following current experiment specifications in the economics literature~\cite{lin2019valuing,tang2019value}, we model the privacy cost function as a linear summation\footnote{See the Eq. 2 in \cite{lin2019valuing} and the dis-utility from disclosure in the econometric specification session in \cite{tang2019value}.} of disclosed personal data.


\czq{According to the comprehensive survey on privacy value definition \cite{MKT-053}, people will measure the value of their privacy into the intrinsic value of privacy and the instrumental value of privacy.}
\czq{
The intrinsic loss indicates the sake of protecting their intrinsic private data, which measures the valuation on the intrinsic properties such as the education or the income levels. }
\czq{In this work, we denote the intrinsic loss towards the privacy cost on amount of the sharing user profile attributes.}
\czq{The instrumental value of privacy indicates how the transaction efficiency would be affected by sharing user data, especially the data generated in the applications. 
In this work we denote the privacy cost towards the percentage of shard user historical behavior data. 
Therefore, the privacy cost function is described below,}
\begin{equation}
    \texttt{C}_i(\si{})= \beta_i * | {\scriptstyle \mathcal{S}_{i,a}} | + \frac{| {\scriptstyle \mathcal{S}_{i,b}} |}{ | {\scriptstyle \mathcal{D}_{i,b}} |}
    \label{eq:cost_function0}
\end{equation}
\czq{where the first term indicates the intrinsic loss and the second term indicates the instrumental loss.} 
\czq{If user does not tend to disclose profile attribute, such privacy cost function can be simplified to the following format with the instrumental value alone.}
As mentioned in \cref{sec:plat_mech}, user tends not to disclose profile attributes $\scriptstyle \mathcal{D}_{i,a}$ due to no gains in our experiments, so we only consider behavior data here, i.e.,
\begin{equation}
    \texttt{C}_i(\si{})=\texttt{C}(\si{}) =  \frac{| {\scriptstyle \mathcal{S}_{i,b}} |}{ | {\scriptstyle \mathcal{D}_{i,b}} |}
    , %
    \label{eq:cost_function}
\end{equation}
where the $|x|$ is the number of elements in $x$.
Here, the percentage based measurement regards different amount of users' data equally. %


This reduced form specification is not unrealistic as it captures the substitution effect among personal data and incorporates the idea of constant marginal privacy cost. 
One might argue for a higher order functional to capture richer implications. 
However, there is little experimental evidence that the higher order form for privacy cost exists and how the functional form looks like.






\subsubsection{\textbf{Privacy Sensitive Weight}}
\label{sec:user_type}
For user $i$ who disclosed all her/his data (i.e., $\si{} = \di{}$), her/his privacy cost compared to not sharing any data (i.e., $\si{} = \varnothing $) is 
\begin{equation}
    \texttt{C}(\di{}) - \texttt{C}(\varnothing).
\label{eq:privacy_diff}
\end{equation}
Meanwhile, her/his anticipated recommendation utility compared to not sharing any data is:
\begin{equation}
    \texttt{U}(\di{}) - \texttt{U}(\varnothing).
\label{eq:utility_diff}
\end{equation}
We assume all users have accessed to the recommendation utility $\texttt{U}(\di{})=\texttt{U}'(\di{},\texttt{M}_{{\scriptscriptstyle \mathcal{D}}})$ computed on all the data $\di{}$ and the recommendation utility without their data $\texttt{U}(\varnothing)$ before they can take data disclosing actions, which can be regard as a prior knowledge, like the experiences before the platform adopted our framework.
With \cref{eq:privacy_diff} and \cref{eq:utility_diff}, we define the privacy sensitive weight $\lambda_i$ as: 
\begin{equation}
    \lambda_i = w_i  * \frac{\texttt{U}(\di{})  - \texttt{U}(\varnothing) } {  \texttt{C}(\di{}) - \texttt{C}(\varnothing) },
    \label{marginal_define}
\end{equation}
where $w_i$ indicates the diversity of user types towards privacy sensitivity.
The users with $w_i > 1$ is privacy sensitive users, as they will not be willing to disclose the corresponding data $\di{}$ if they only get $\texttt{U}(\di{})$ as before.
While users  with $w_i < 1$ are just the opposite. %
Therefore, the user's privacy sensitive weight is pre-computed, and the $\texttt{U}(\di{})$ can be regarded as the benchmark expectation of the platform.
The formulation of the privacy sensitive weight $\lambda_i$ also meets the idea from \cite{lin2019valuing}, where the heterogeneity from users' social demographic variety should also be explicitly characterized. %



\subsubsection{\textbf{Simulation Algorithm}}

As users behave rationally to find the optimal strategy with a trade-off of exploration and exploitation, it just meets the idea of the reinforcement learning algorithm. 
Therefore, we model each user as a unique agent and apply a multi-agent reinforcement learning method to simulate user possible policy adaptation. 
The recommender system is regarded as the environment to provide feedback, which is built upon the disclosed user data.
All agents' policies are optimized simultaneously by determining their actions, i.e., the disclosed data $\scriptstyle \mathcal{S}^t$ at simulation epoch $t$, which is used to train the recommendation model $\texttt{M}_{{\scriptscriptstyle \mathcal{S}}^{t}}$.
As mentioned before, users tend to find an optimal action over possible action space $\Pi$ to maximize his expected reward, which is determined by all agents in this dynamic MARL environment. 


We assume each user (agent) realizes this situation that the immediate reward is the result of all agents, but no communication or observation among agents is permitted. 
Then, each agent is concerned about her/his own utility and regards the environment as a dynamic system that is partially correlated to herself/himself. 
Now, it is simplified to a Multi-Armed Bandit problem~\cite{katehakis1987multi}.


However, the challenge of the exploration and explication problem also exists in our simulation. 
To address it, we adopt a simple but efficient method, Epsilon Greedy~\cite{sutton2018reinforcement} algorithm, to simulate user's policy $\pi_i$ as following, %
\begin{equation}
     \alpha_i^{t+1} = \left\{
\begin{array}{l l }
\alpha_i \sim \texttt{P}^t_i,
& \text{with possibility } \epsilon \\
\argmax_{\alpha_i} Q_i^t(\alpha_i), & \text{with possibility }  1-\epsilon \\
\end{array} \right. 
    \label{epsilon_greedy}
\end{equation}
where $Q_i^t(\alpha)$ is the user $i$'s estimation value at simulation epoch $t$ on action $\alpha$, and $\texttt{P}^t_i$ denotes a random sample policy.
To conduct an efficient policy exploration, we sample a less explored action with a higher possibility as following,
\begin{equation}
    \texttt{P}^t_i(\alpha) = \frac{ 1/ (N^{t-1}_i(\alpha) +1) } { \sum_{x \in \Pi} 1/(N^{t-1}_i(x) +1) },
    \label{random_rule}
\end{equation}
where $N^{t-1}_i(\alpha)$ represents the total number of action $\alpha$ was taken by user $i$ from start to the last simulation epoch $t{-}1$.
In convenience, we adopt the approximated expected estimation results and 
update it with the residual between the estimation $Q_i^{t-1}(\alpha_i^{t-1})$ and immediate reward  $\texttt{R}_i^{t-1}$ when she/he takes action $\alpha_i^{t-1}$ as following.
\begin{equation*}
       Q_i^t(\alpha) {=} \left\{\!\!\!
\begin{array}{l l}
Q_i^{t-1}(\alpha), &  \text{if } \alpha_i^{t-1} {\neq} \alpha \\
Q_i^{t{-}1}(\alpha) {+} \frac{1}{N^t_i(\alpha)} \bigl(\texttt{R}_i^{t-1}(\! \alpha {\otimes} \mathrm{\delta}(\di{})  ) {-} Q_i^{t{-}1}(\alpha) \bigr), &  \text{if }  \alpha_i^{t{-}1} {=} \alpha \\
\end{array} \right. 
\end{equation*}
where $\texttt{R}_i^{t-1}$ is user $i$-th immediate objective at simulation epoch $t{-}1$, computed by \cref{eq:framework}. 
$Q_i^0(\alpha) $ is the user $i$'s initial expected reward if she/he takes action $\alpha$. 
which is initialized to $0$ as users have no prior about their behaviors on the new dynamic environment.


In our simulation, we set initial $\epsilon=0.5$ for all agents and decay a half during the MARL training processing. The detailed decay epoch is co-related to the size of possible action space $\Pi$.
Here, we define it as $\epsilon = 0.5^{ t /(3 * |\Pi |) }  $, 
where $t$ is the epoch during the reinforcement learning training processing. 

\subsection{\textbf{Discussion}}
To figure out how the platform mechanism affects users' behavior, we turn to the simulation built upon several simplified assumptions. 
One fundamental assumption is the hypothesis of rational man, where users will seek their optimal policies to maximize their objectives. 
However, in the real-world scenarios, human behaviors are also affected by psychological factors, which should also be modeled in future work.
One detailed example is that some users may feel exhausted digging out all the potential privacy choices with the provided platform mechanism.
In our simulation, we assume there remains no mental cost when a user adjusts his policy. 
However, in the reality, some users may refuse to change their policy frequently, especially in complex user interaction applications.
For such situation, a convenient user interface (UI) could be a potential solution to mitigate users' fatigues. 
\czq{
Another important factor is that users may adjust their trust level towards the platform during their exploration. One detailed example is that if the platform or even the recommender system \cite{zhang2022pipattack} is easy to be attacked or the platform will abuse their disclosed data to other applications, they may re-consider their privacy sensitivity. 
Though some works have discussed the utilization of trusted platform or the privacy-preserved recommendation model, the possible effects on user psychological factors might be tackled by a dynamic modeling on the user privacy sensitive weights, which is out of the scope of this work.
}
We simplify the influences of the psychological factors in this work and leave the exploration of psychological effects in mechanism designs and UI designs for future works. 






\bibliographystyle{IEEEtran}
{  \small
\bibliography{Greplay2}
}
\end{document}\end{document} 

\section{Discussion}
\label{s:discussion}
In this paper, 2D and 3D CNN models were used to generate pelvic sCTs from T1-weighted MR images. Our sCT generation methods were fully automated, requiring no deformable registration or manual segmentation of bone tissues. As shown in Figure~\ref{fig3}, the 2D and 3D CNN models generated high quality sCTs. MAE curves shown in Figure~\ref{fig4} indicated that both models could precisely estimate soft-tissue HU values but had difficulty in reproducing air and high-density bone tissues. 

The MAEs within the body contour across all patients were 40.5 $\pm$ 5.4 HU and 37.6 $\pm$ 5.1 HU for the 2D and 3D models, respectively. The time required for generating a pelvic sCT using our CNN models was about 5.5 s. Our MAE results are comparable to previous studies. Kim $et \ al.$\cite{RN41} presented a voxel-based weighted summation method that produced an MAE of 74.3 $\pm$ 3.9 HU. However, manual contouring of bone tissues required for this method can be tedious and time-consuming. An MAE of 40.5 $\pm$ 8.2 HU was achieved by Dowling $et \ al.$\cite{RN11} using an average MRI-CT atlas from 38 patients. Andreasen $et \ al.$\cite{RN42} reported an MAE of 54 $\pm$ 8 HU using an atlas-based method with pattern recognition, and its prediction time was about 20.8 min. Another random forest model proposed by Andreasen $et \ al.$\cite{RN43} generated sCTs with an MAE of 58 $pm$ 9 HU. A hybrid method suggested by Siversson $et \ al.$ \cite{RN45} obtained an MAE of 36.5 $\pm$ 4.1 HU when ignoring errors introduced by gas cavities. This hybrid method was implemented in the cloud-based commercial software MriPlanner (Spectronic Medical AB, Helsingborg, Sweden), which required 50 to 80 min to generate a sCT.\cite{RN45} The patch-based 3D context-aware generative adversarial network presented by Nie $et \ al.$\cite{RN26} achieved an MAE of 39.0 $\pm$ 4.6 HU. 

Our CNN models reproduced low-density bone as shown in Figure ~\ref{fig4}. The bone-region DSCs were 0.81 $\pm$ 0.04 and 0.82 $\pm$ 0.04 from the 2D and 3D models, respectively. These results are comparable to reported DSC results of 0.79 $\pm$ 0.12\cite{RN10} and 0.91$\pm$0.03{\cite{RN11}}, where the authors compared bone contours manually drawn on the sCT and CT.

It was feasible to train the proposed 3D model with 16 image volumes from scratch. Results of the Wilcoxon signed-rank tests shown in Table~\ref{tab1} demonstrated a statistically significant improvement in overall MAE, bone DSC, and bone precision of the 3D model compared to the 2D model. However, as shown in Figure~\ref{fig4}, the 2D model seemed to perform better in estimating the high-density bone HU values. It should be noted that smaller overall MAEs do not guarantee improved sCT dose calculation and patient positioning performance. While the models performed well, we will continue to acquire more patient data to potentially improve model accuracy and further test model differences.

As this was a retrospective study, the MR image voxel sizes were not matched, resulting in different voxel intensities between images. This may have affected the sCT generation accuracy although we applied intensity normalization. A potential study could examine how voxel size variations affects sCT estimation. 

The proposed 3D model can be implemented on a 12 GB GPU to process volumetric images with dimensions of 256 $\times$ 256 $\times$ 30. More GPU memory would be required to process higher resolution 3D images. Considering the limited access to multi-GPU systems, a 3D architecture with fewer convolutional layers could be considered to deal with higher resolutions. However, the performance could be affected by the reduced parameters and smaller receptive fields of the less complex model. Another approach would be to extract 30-slice sub-volumes from CT and MR images for training the 3D model. The sCT could then be generated by averaging 30-slice sCT sub-volumes produced by the model. 

A number of techniques could be investigated for improving model performance.  Nie $et \ al.$\cite{RN26} showed that introducing an additional adversarial discriminator improved overall sCT quality. The same approach could be adapted in our proposed 2D and 3D CNN models.  Non-rigid deformation\cite{RN44} could also be applied to both CT and MR images in the process of the on-the-fly data augmentation to produce more training pairs. Multiple MR images acquired with different sequences could be fed into models to provide more information for distinguishing different tissues. Multi-GPU systems with more memory would enable the exploration of larger batch sizes for training CNN models, which could reduce variances in gradient estimation and accelerate the training. 



\subsection*{Acknowledgments}
We thank Patrick Flandrin, Adrien Hardy, and Fred Lavancier for fruitful
discussions on various aspects of this paper. RB acknowledges support from ANR
\textsc{BoB} (ANR-16-CE23-0003), and all authors acknowledge support from
ANR \textsc{BNPSI} (ANR-13-BS03-0006).
\bibliography{stft,stats}
\bibliographystyle{plainnat}

\end{document}
