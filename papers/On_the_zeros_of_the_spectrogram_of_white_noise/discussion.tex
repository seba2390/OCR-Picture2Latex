We showed how to give a mathematical meaning to the zeros of the spectrogram of
white noise, and investigated their statistical distribution for real, complex,
and --~to a lesser extent~-- analytical white noise. We have related these zeros
to the zeros of Gaussian analytic functions, a topic of booming interest in
probability. More pragmatically, we investigated the computational issues raised
by implementing tests based on spectrogram zeros. 

The connection with GAFs puts signal processing algorithms based on
spectrogram zeros on firm ground, and further progress on GAFs is bound to be
fruitful for signal processing. Perhaps less obviously, we believe
signal processing tools can also bring insight into probabilistic questions on
GAFs. For starters, the Bargmann transform, spectrogram zeros and the fast Fourier transform give a novel way to approximately simulate the zeros of the planar GAF, or even the zeros of random polynomials.

As for the detection of signals using spectrogram
zeros, we have investigated the application of standard frequentist testing
tools. They showed good power for high SNR, but the performance decreases for
low SNR and small signal support compared to the observation window. 
There are various leads to improve on these two points. First, we could
transform our global test into several local tests, trying to adapt the tested
patch to the support of the signal. Second, models for signals could be fed to
Bayesian techniques, allowing to explore all signals compatible with a
given pattern of zeros.

