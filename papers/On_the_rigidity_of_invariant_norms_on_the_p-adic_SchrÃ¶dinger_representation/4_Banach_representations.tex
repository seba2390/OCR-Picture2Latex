%%%%%%%%%%%%%%%%%%%%%%%%%%%%%%%%%%%%---------------------------
\section{Banach representations}
%%%%%%%%%%%%%%%%%%%%%%%%%%%%%%%%%%%%---------------------------
The goal of this section is twofold.
We introduce the terminology about norms and Banach representations that will be used throughout this paper, and we prove some fundamental properties of Banach representations that we will later need.
We address two issues.
The first is a notion of minimality of norms that we call weak minimality.
The second consists of several characterizations of quotients of universal unitary completions of (algebraically) cyclic representations.
Let $G$ be an abstract group and $V$ a representation of $G$ over $\C_p$.
Under the assumptions that $V$ is cyclic, $V$ has a universal unitary completion in the sense of \cite{emerton2005p} that we denote by $\Ucomp{V}$.
The quotients of $\Ucomp{V}$ by closed sub-representations will play an important role in this paper, especially quotients by maximal sub-representations.
We give two intrinsic characterizations of these quotients: in terms of a special type of norms which we call \textit{locally maximal} and in terms of the existence of a special type of vectors which we call \textit{strongly cyclic}.

%%%%%%%%%%%%%%---------------
\subsection{General terminology and notation}
%%%%%%%%%%%%%%---------------
Let $V$ be vector space over $\C_p$.
A norm on $V$ is a map $\norm{}:V\map \R_{\geq 0}$ such that 
 \begin{enumerate}
    \item $\norm{v}=0$ if and only if $v=0$.
    \item $\norm{a\cdot v}=\abs{a}_p\cdot \norm{v}$ for any $v\in V$ and $a\in\C_p$.
    \item $\norm{v_1+v_2}\leq \max(\norm{v_1},\norm{v_2})$.
    \end{enumerate}
If $\norm{}$ satisfies only $2$ and $3$ we say that it is a seminorm.

Let $\norm{}_1,\norm{}_2$ be two norms on $V$.
We write $\norm{}_1\leq \norm{}_2$ if $\norm{v}_1\leq \norm{v}_2$ for any $v\in V$.
We say that $\norm{}_1$ is dominated by $\norm{}_2$, and denote it by $\norm{}_1\dominated \norm{}_2$ if there exists a constant $D>0$ such that $\norm{}_1\leq D\cdot \norm{}_2$.
We say that $\norm{}_1$ and $\norm{}_2$ are equivalent if each dominates the other: $\norm{}_1\dominated \norm{}_2$ and $\norm{}_2\dominated \norm{}_1$.
These two norms are called homothetic if there exists $c>0$ such that $\norm{v}_1=c\cdot \norm{v}_2$ for any $v\in V$. 

If $v\in V$ is a non-zero vector, we say that $\norm{}$ is normalized at $v$ if $\norm{v}=1$.
In any homothety class of norms there is exactly one norm that is normalized at $v$.

Given a norm $\norm{}$ on $V$, we denote the completion of $V$ with respect to $\norm{}$ by $\completion{V}{\norm{}}$.

Assume that a group $G$ acts on $V$.    
A norm $\norm{}$ on $V$ is said to be $G$-invariant if $\norm{gv}=\norm{v}$ for any $v\in V$ and $g\in G$.    
When there is no ambiguity about the group $G$, we will simply say that $\norm{}$ is an invariant norm.
We denote the set of norms on $V$ by $\Norms(V)$ and by $\Norms(V)^G$ its subset of $G$-invariant norms.

In this paper the term \textit{Banach representation} means the following.
\begin{defn}\label{def_Banach_rep}
A Banach representation (over $\C_p$) of $G$ is a pair $(B,\norm{})$ of a $G$-representation $B$ and a $G$-invariant norm $\norm{}$ such that $B$ is complete with respect to $\norm{}$.
\end{defn} 

A morphism of Banach representations of $G$ is a continuous $G$-equivariant map, but it need not be an isometry.
In particular, isomorphic Banach representations of $G$ are not necessarily isometric.

%%%%%%%%%%%%%%---------------
\subsection{Weakly minimal norms}
%%%%%%%%%%%%%%---------------
\begin{defn}
Let $(B,\norm{})$ be a Banach representation of the group $G$ and $v\in B$ a non-zero vector.
We say that $\norm{}$ is weakly minimal at $v$ if the following holds.
    \begin{itemize}
    \item For any $G$-invariant norm $\norm{}'$ on $B$ such that $\norm{}'\leq \norm{}$ and $\norm{v}'=\norm{v}$, we have $\norm{}'=\norm{}$.
    \end{itemize}
\end{defn} 

\begin{lem}
Let $(B,\norm{})$ be a Banach representation of $G$.
Assume that the values of $\norm{}$ are the same as the values of $\abs{}_p$ on $\C_p$.
Assume that $v\in B$ is a non-zero vector such that $\norm{v}=1$ and such that its image $\overline{v}$ in the quotient 
\[\overline{B}(\norm{}):=\quot{\braces{v\in B\ |\ \norm{v}\leq 1}}{\braces{v\in B\ |\ \norm{v}< 1}}\]
is contained in any non-zero sub-representation of $\overline{B}(\norm{})$.
Then $\norm{}$ is weakly minimal at $v$.
\end{lem} 
	\begin{proof}
    Let $\norm{}'\in \Norms(B)^G$ be a $G$-invariant norm such that $\norm{v}'=1$ and $\norm{}'\leq \norm{}$.
    The identity map $Id:B\map B$ induces a map $T:\overline{B}(\norm{})\map \overline{B}(\norm{}')$.
    The kernel of $T$ is a sub-representation of $\overline{B}(\norm{})$ that does not contain $v$, hence by assumption, this kernel is trivial.
    It follows that $T$ is injective.
    Therefore, $\norm{w}'=1$ for any $w$ with $\norm{w}=1$.
    Since the values of $\norm{}$ are the same as the values of $\abs{}_p$, $\norm{}'=\norm{}$.
    \end{proof} 
    
\begin{prop}\label{prop_weak_minimality}
Let $G$ be a pro-$p$ group.
Let $C(G)$ denote the space of continuous functions on $G$ with values in $\C_p$ and let $\supnorm{}$ be the sup norm on $C(G)$. 
Consider the action of $G$ on $C(G)$ by right translations.
The sup norm is weakly minimal at $\textbf{1}$, where $\textbf{1}$ denotes the constant function $\textbf{1}(x)=1$.
\end{prop} 
    \begin{proof}
    Identify the quotient
    \[\quot{\braces{f\in C(G)\ |\ \supnorm{f}\leq 1}}{\braces{f\in C(G)\ |\ \supnorm{f}< 1}}\]
    with the space $\schw(G,\overline{\fld{F}}_p)$ of locally constant functions on $G$ with values in an algebraic closure $\overline{\fld{F}}_p$ of $\fld{F}_p$.
    By the previous lemma, it is enough to show that any non-zero sub-representation of $\schw(G,\overline{\fld{F}}_p)$ contains the constant function $\textbf{1}$.
    Let $f\in \schw(G,\overline{\fld{F}}_p)$ be non-zero and denote by $V$ the sub-representation generated by $f$.
    As $f$ is fixed by some open normal subgroup $N\subset G$, $V$ is a cyclic representation of the finite group $G/N$.
    In particular, $V$ is a finite dimensional representation of the finite $p$-group $G/N$ over $\overline{\fld{F}}_p$.
    Thus, $V$ contains a non-zero $G$-invariant vector $\phi$.
    This $\phi$ is a non-zero constant function.
    \end{proof} 
%%%%%%%%%%%%%%---------------
\subsection{The universal unitary completion of a cyclic representation}
%%%%%%%%%%%%%%---------------
Let $V$ be a representation of $G$ and assume that $v\in V$ is a cyclic vector.
In addition, assume that $\Norms(V)^G$ is non-empty, i.e. there exists a $G$-invariant norm on $V$.

If $\norm{}\in \Norms(V)^G$, its closed unit ball $\braces{w\in V\ |\ \norm{w}\leq 1}$ is an $\mathcal{O}_{\C_p}[G]$-module that contains a non-zero multiple of any vector in $V$, but contains no $\C_p$-lines.
Such an $\mathcal{O}_{\C_p}$-module is called an integral structure.
Conversely, any integral structure $L$ defines a $G$-invariant norm, called the gauge of $L$, by
\[\norm{v}_L=\inf\braces{\abs{a}_p\ |\ v\in a\cdot L}.\]
We stress the fact that in general $L$ might not be equal to the closed unit ball nor to the open unit ball of $\norm{}_L$, but lies strictly between them.
 For future use we record the following formulas for the closed and open unit balls of $\norm{}_L$,
    \begin{equation}\label{equation_close_open_unit_balls}
    \braces{v\in V\ |\ \norm{v}_L\leq 1}=\bigcap_{\substack{\lambda\in \C_p\\ \pabs{\lambda}>1}}\lambda L,\ \ \ 
    \braces{v\in V\ |\ \norm{v}_L< 1}=\bigcup_{\substack{\lambda\in \C_p\\ \pabs{\lambda}<1}}\lambda L.  
    \end{equation} 

Since $\C_p$ is not discretely valued, two different invariant norms give rise to different integral structures, but two different integral structures might define the same norm.
Nevertheless, the correspondence between invariant norms and integral structures inverts order.


The set $L_{v}:=\mathcal{O}_{\C_p}[G]\cdot v$ is an integral structure.
Indeed, it contains a multiple of any vector since $v$ is cyclic, and it contains no lines because of the existence of an invariant norm.
As $L_v$ is the smallest integral structure that contains $v$,
its corresponding norm, which we denote by $\norm{}_v$, is normalized at $v$ and is the maximal invariant norm normalized at $v$.
This means that if $\norm{}\in\Norms(V)^G$ is normalized at $v$, then $\norm{}\leq \norm{}_v$.
We call the norm $\norm{}_{v}$ \textit{the maximal invariant norm at $v$} or \textit{the maximal norm at $v$} for short.

If $v_1,v_2\in V$ are two cyclic vectors of $V$, the norms $\norm{}_{v_1}$ and $\norm{}_{v_2}$ are equivalent.
In particular the completion of $V$ with respect to $\norm{}_v$, where $v$ is a cyclic vector, is independent of $v$ as a topological vector space.
We denote this completion by $\Ucomp{V}$ and call it the universal unitary completion of $V$, or the universal completion for short. 
Note that this is a particular case of Example A in \cite{emerton2005p}.
The universal completion of $V$ has the following universal property: if $(B,\norm{})$ is a Banach representation of $G$ and $T:V\map (B,\norm{})$ is  $G$-equivariant, then $T$ factors continuously through $\Ucomp{V}$.

    \begin{remark}
    The assumption that $\Norms(V)^G\neq \phi$ is superfluous, is made for simplification and because this is the case that will appear later.
    If $V$ does not have a $G$-invariant norm, $\mathcal{O}_{\C_p}[G]\cdot v$ contains $\C_p$-lines.
    The union of these lines is a sub-representation $W\subset V$ and the quotient $V'=V/W$ is cyclic and has an invariant norm.
    The universal completion of $V$ is $\Ucomp{V'}$.    
    \end{remark} 

Any element of $\Ucomp{V}$ can be written as 
\[\sum_{g\in G}\lambda_g\cdot g(v)\]
where $(\lambda_g)_{g\in G}\subset \C_p$ is \textit{summable}, meaning that for any $\eps>0$, at most finitely many of the $\lambda_g$ satisfy $\abs{\lambda_g}_p\geq \eps$.
Conversely, any element of this form is in $\Ucomp{V}$.

\begin{defn}\label{def_local_maximality}
Let $W$ be a representation of $G$ and $\norm{}\in\Norms(W)^G$.
Let $w\in W$ a non-zero vector such that $\norm{}$ is normalized at $w$.
We say that $\norm{}$ is \textit{locally maximal at $w$} if the following property holds.
    \begin{itemize}
    \item For any $\norm{}'\in \Norms(W)^G$ that is normalized at $w$ and is dominated by $\norm{}$ we have $\norm{}'\leq \norm{}$.
    \end{itemize}
\end{defn} 
For example, the norm $\norm{}_v$ on $V$ is a locally maximal norm at $v$.
Were it also locally maximal at another cyclic vector $u$, then the two norms $\norm{}_v$ and $\norm{}_u$, being equivalent, would be homothetic.
Easy examples show that this need not be the case.
Thus a norm which is locally maximal at one cyclic vector is in general not locally maximal at another one.
For another example, consider the space $C_0(\Q_p^d)$ of $\C_p$-valued continuous functions on $\Q_p^d$ that go to zero at infinity, and the action of the Heisenberg group $\heis$ on it by the formula given in the previous section.
We will later show that the sup norm $\supnorm{}$ is a locally maximal norm at $\textbf{1}_{\Z_p^d}(x)$ on $C_0(\Q_p^d)$, where $\textbf{1}_{\Z_p^d}$ is the characteristic function of $\Z_p^d$.

\begin{defn}\label{def_srongly_cyclic_vector}
Let $(B,\norm{})$ be a Banach representation of $G$ and $v\in B$ a non-zero vector.
We say that $v$ is \textit{topologically cyclic} if $v$ generates (algebraically) a dense representation in $B$.
We say that $v$ is \textit{strongly cyclic} if any $w\in V$ can be written as
\[w=\sum_{g\in G}\lambda_g\cdot g(v),\]
where the $(\lambda_g)_{g\in G}$ is summable.
\end{defn} 
For example, if $v$ is a cyclic vector in $V$, $v$ is strongly cyclic in $\Ucomp{V}$.
In the end of this section we give an example of a topologically cyclic vector which is not a strongly cyclic vector.

We make the following two observations.
Let $T:B'\map B$ be a map of Banach representations of $G$.
    \begin{itemize}
    \item Assume that the image of $T$ contains a strongly cyclic vector. 
    Then $T$ is surjective.
    \item Assume that $v'\in B'$ is strongly cyclic and that $T$ is surjective.
    Then $v=T(v')$ is strongly cyclic in $B$.
    \end{itemize}

We begin with two lemmas.
The first says that a quotient norm of a locally maximal one is locally maximal.
The second says that strongly cyclic vectors give rise to locally maximal norms.

\begin{lem}\label{lem_quotient_locally_maximal_norms}
Let $W$ be a representation of $G$, $\norm{}\in \Norms(W)^G$ and $K\subset W$ a closed (with respect to $\norm{}$) sub-representation.
Assume that $\norm{}$ is normalized and locally maximal at $w\in W$.
Then the quotient norm on $W/K$ is normalized and locally maximal at the image of $w$.
\end{lem} 
    \begin{proof}
    Let $\norm{}'$ denote the quotient norm on $W/K$.
    Let $\norm{}_2$ be a $G$-invariant norm on $W/K$ that is normalized at the image of $w$ and dominated by $\norm{}'$.
    Using the quotient map $W\map W/K$ we view $\norm{}_2$ also as a semi-norm on $W$.
    Then $\max(\norm{}_2,\norm{})$ is a $G$-invariant norm on $W$ that is normalized at $w$ and dominated by $\norm{}$.
    Thus, $\norm{}_2\leq \norm{}$, as semi-norms on $W$.
    Taking the quotient by $K$, we obtain $\norm{}_2\leq \norm{}'$, as norms on $W/K$.    
    \end{proof} 


\begin{lem}\label{lem_locally_maximal_from_strongly_cyclic}
Assume that $(B,\norm{})$ is a Banach representation of $G$ and that $0\neq v\in B$ is a strongly cyclic vector.
Then there exists a unique norm, which we denote by $\norm{}_{v,B}$, which is normalized and locally maximal at $v$ and is equivalent to $\norm{}$.
In addition, if $w\in B$ with $\norm{w}_{v,B}=r$, then for any $\eps>0$ there exists a summable sequence $(\lambda_g)_{g\in G}$ such that 
\[w=\sum_{g\in G}\lambda_g\cdot g(v)\]
and $\max_{g\in G}\pabs{\lambda_g}<(1+\eps)\cdot r$.
\end{lem} 

    \begin{proof}
    The uniqueness of a normalized and locally maximal norm at $v$ is clear.
    In the rest of the proof we construct the norm $\norm{}_{v,B}$ using an integral structure and show the additional property.
    
    Let $\overline{L}$ be the closure in $B$ of 
    \[L=\braces*{\sum_{g\in G}\lambda_g\cdot g(v)\ |\ (\lambda_g)_{g\in G} \text{ is summable and } \abs{\lambda_g}_p\leq 1 \text{ for all }g\in G}.\]
    Assume, for convenience, that $\norm{v}=1$. 
    We first show that $\overline{L}$ is an open integral structure.
    It is straightforward that $\overline{L}$ is an integral structure, the only non-obvious part is that $\overline{L}$ contains no $\C_p$-lines.
    This is true since $L$, and therefore $\overline{L}$, is contained in the closed unit ball of $\norm{}$.
    We now show that $\overline{L}$ is open.
    Since $v$ is strongly cyclic, $B=\bigcup_{n=0}^\infty p^{-n}\cdot \overline{L}$.
    Since $B$ is a complete metric space, it follows from Baire's category theorem that $\overline{L}$ has a non-empty interior, and since it is a topological subgroup, it must be open.
    Let $\norm{}_{v,B}$ be the norm that corresponds to $\overline{L}$.
    Since $\overline{L}$ is an open integral structure, $\norm{}_{v,B}$ is a $G$-invariant norm and equivalent to $\norm{}$.
    
    Now we show that $\norm{}_{v,B}$ is normalized at $v$ and locally maximal at $v$.
    Since $v\in \overline{L}$, it follows that $\norm{v}_{v,B}\leq 1$.
    Let $\norm{}'$ be a $G$-invariant norm dominated by $\norm{}_{v,B}$ and normalized at $v$.
    The closed unit ball of $\norm{}'$ contains $L$, and since
    \[\norm{}'\dominated\norm{}_{v,B}\dominated \norm{}\]
    its unit ball is closed in $B$.
    Thus, the unit ball of $\norm{}'$ contains $\overline{L}$.
    Therefore, $\norm{}'\leq \norm{}_{v,B}$.
    Substituting $v$, we see that $\norm{v}_{v,B}\geq 1$, so $\norm{}_{v,B}$ is normalized at $v$ and locally maximal at $v$.
    
    Finally, we prove the additional property.
    Let $w\in B$ and let $\eps>0$.
    We may assume that $(1+\eps)^{-1}<\norm{w}_{v,B}<1$.
    By \ref{equation_close_open_unit_balls} (formula for the open unit ball), $w\in\overline{L}$, so there exists $w_0\in L$ such that $\norm{w-w_0}_{v,B}<p^{-1}$.
    Similarly, there exists $w_1\in pL$ such that $\norm{w-w_0-p\cdot w_1}_{v,B}<p^{-2}$.
    Continuing in this manner we obtain a sequence $(w_n)_{n=0}^\infty$, where $w_n\in p^n\cdot L$ for all $n$, and $w=\sum_{n=0}^\infty w_n$.
    Therefore, $w\in L$, so it can be written as $w=\sum_{g\in G}\lambda_g\cdot g(v)$ and $\max_{g\in G}\pabs{\lambda_g}<1<(1+\eps)\cdot \norm{w}_{v,B}$.
    \end{proof} 

    \begin{remark}
    Note that the last step in the proof can be modified slightly to show that $L=\overline{L}$.
    However, the closed unit ball of $\norm{}_{v,B}$ might be strictly larger than $L$.
    \end{remark} 


\begin{comment}---------------------------------------------------------------------
\begin{lem}\label{lem_locally_maximal_from_strongly_cyclic}
Assume that $(B,\norm{})$ is a Banach representation of $G$ and that $0\neq v\in B$ is a strongly cyclic vector.
Let $\overline{L}$ be the closure in $B$ of 
\[L=\braces*{\sum_{g\in G}\lambda_g\cdot g(v)\ |\ (\lambda_g)_{g\in G} \text{ is summable and } \abs{\lambda_g}_p\leq 1 \text{ for all }g\in G}.\]
Then $\overline{L}$ is an integral structure in $B$ and the norm that corresponds to it, which we denote by $\norm{}_{v,B}$, is equivalent to $\norm{}$.
In addition, $\norm{}_{v,B}$ is normalized at $v$ and locally maximal at $v$.
\end{lem} 
    \begin{proof}
    We assume for convenience that $\norm{v}=1$. 
    We first show that $\overline{L}$ is an integral structure.
    The only non-obvious part is that $\overline{L}$ contains no $\C_p$-lines.
    This is true since $L$, and therefore $\overline{L}$, is contained in the closed unit ball of $\norm{}$.
    Now we show that $\norm{}_{v,B}$ is equivalent to $\norm{}$.
    Because $\overline{L}$ is contained in the closed unit ball of $\norm{}$, we have $\norm{}\leq \norm{}_{v,B}$.
    Showing that $\norm{}_{v,B}$ is dominated by $\norm{}$ is equivalent to showing that the unit ball of $\norm{}_{v,B}$ is open in $B$.
    Since $v$ is strongly cyclic, $B=\bigcup_{n=0}^\infty p^{-n}\cdot \overline{L}$.
    Since $B$ is a complete metric space, it follows from Baire's category theorem that $\overline{L}$ has a non-empty interior, and since it is a topological subgroup, it must be open.    
    
    Now we show that $\norm{}_{v,B}$ is normalized at $v$ and locally maximal at $v$.
    Since $v\in \overline{L}$ it follows that $\norm{v}_{v,B}\leq 1$.
    Let $\norm{}'$ be a $G$-invariant norm dominated by $\norm{}_{v,B}$ and normalized at $v$.
    The closed unit ball of $\norm{}'$ contains $L$, and since
    \[\norm{}'\dominated\norm{}_{v,B}\dominated \norm{}\]
    its unit ball is closed in $B$.
    Thus, the unit ball of $\norm{}'$ contains $\overline{L}$.
    Therefore, $\norm{}'\leq \norm{}_{v,B}$.
    Substituting $v$, we see that $\norm{v}_{v,B}\geq 1$, so $\norm{}_{v,B}$ is normalized at $v$.
    \end{proof} 
    -----------------------------------------------------------------\end{comment}

\begin{thm}\label{thm_strongly_cyclic_spaces}
Let $(B,\norm{})$ be a Banach representation of $G$ and $v\in B$ a non-zero vector.
The following are equivalent.
    \begin{enumerate}
    \item $v$ is a strongly cyclic vector of $B$ and $\norm{}=\norm{}_{v,B}$.
    \item Let $V$ be the (algebraic) sub-representation generated by $v$.
    Then the map $I:\Ucomp{V}\map B$ is surjective and if $K$ is its kernel, the induced map $\Ucomp{V}/K\map B$ is an isometry when we equip $\Ucomp{V}$ with the norm $\norm{}_v$.
    \item $\norm{}$ is normalized at $v$ and locally maximal at $v$.
    \end{enumerate}
\end{thm} 
    \begin{proof}
    We will show $(1)\Rightarrow (2)\Rightarrow (3)\Rightarrow (1)$.
    Assume $(1)$. 
    Since $v$ is strongly cyclic, the map $I:\Ucomp{V}\map B$ is surjective. 
    Equip $\Ucomp{V}$ with the norm $\norm{}_v$ and let $\norm{}'$ denote the quotient norm on $\Ucomp{V}/K$.
    Via $I$, we view $\norm{}'$ as a norm on $B$.
    By the open mapping theorem, $\norm{}'$ and $\norm{}=\norm{}_{v,B}$ are equivalent.
    By Lemmas \ref{lem_quotient_locally_maximal_norms} and \ref{lem_locally_maximal_from_strongly_cyclic}, these two norms are normalized and locally maximal at $v$.
    Thus, they are equal.
    
    Assume $(2)$.
    $(3)$ Follows from Lemma \ref{lem_quotient_locally_maximal_norms}.
    
    Assume $(3)$.
    We will prove $(1)$.
    We assume that $\norm{}$ is normalized and locally maximal at $v$.
    Let $w\in B$; we want to show that $w$ is of the form $\sum_{g\in G}\lambda_g\cdot g(v)$, where $(\lambda_g)_{g\in G}$ is summable.
    We may assume that $\norm{w}\leq 1$.
    Let $L=\mathcal{O}_{\C_p}[G]\cdot v$ and let $D$ be the closed unit ball of $\norm{}$.
    Then $L+p^2\cdot D$ is an open integral structure in $B$ that contains $v$.
    Its corresponding norm, that we denote by $\norm{}'$, is dominated by $\norm{}$ and satisfies $\norm{v}'\leq 1$.
    Therefore, by $(3)$, $\norm{}'\leq \norm{}$.
    By \ref{equation_close_open_unit_balls} (formula for the closed unit ball) it follows that 
    \[w\in D\subset\bigcap_{\substack{\lambda\in\C_p}{\pabs{\lambda}>1}}\lambda(L+p^2\cdot D)\subset p^{-1}(L+p^2\cdot D)=p^{-1}L+pD.\]
    Thus, there exist $x_1\in L$ and $d_1\in D$ such that $w=p^{-1}x_1+p\cdot d_1$.
    Repeating this process with $d_1$ instead of $w$, there exist $x_2\in L$ and $d_2\in D$ such that $d_1=p^{-1}x_2+pd_2$.
    Thus, $w=p^{-1}x_1+p(p^{-1}x_2+pd_2)=p^{-1}x_1+x_2+p^2d_2$.
    Repeating this process, we obtain sequences $(x_n)_{n=1}^\infty\subset L$ and $(d_n)_{n=1}^\infty \subset D$ such that $d_n=p^{-1}x_{n+1}+pd_{n+1}$ for any $n\geq 0$.
    Thus, $w=\sum_{n=0}^\infty p^{n-1}\cdot x_n$ which is of the desired form.    
    \end{proof} 

In particular, if $(B,\norm{})$ is a Banach representation of $G$, the following are equivalent.
    \begin{enumerate}
    \item $B$ is isomorphic to a quotient of a universal completion of a cyclic representation.
    \item $B$ has a strongly cyclic vector.
    \item $\norm{}$ is equivalent to a locally maximal norm with respect to some vector $v$.
    \item There exists $v\in B$ such that any map $T:B'\map B$ of Banach representations of $G$ such that $v$ lies in its image is surjective.
    \end{enumerate}




%%%%%%%%%%%%%%---------------
\subsection{Strongly irreducible Banach representations}
%%%%%%%%%%%%%%---------------

\begin{defn}\label{def_strong_irreducible}
Let $(B,\norm{})$ be a Banach representation of $G$.
We say that $B$ is strongly irreducible if any non-zero vector in $B$ is strongly cyclic. 
\end{defn}
Clearly, a strongly irreducible Banach representation is topologically irreducible.
The converse is not true (see the example at the end of this section).

\begin{prop}\label{prop_open_unit_ball_of_strongly_cyclic}
Let $(B,\norm{})$ be a Banach representation of $G$ and $v\in B$ a strongly cyclic vector.
Assume that $\norm{}$ is normalized and locally maximal at $v$.
Then, any $w\in B$ with $\norm{v-w}<1$ is also strongly cyclic.
\end{prop} 
    \begin{proof}
    First, note that $\norm{}$ is equal to $\norm{}_{v,B}$ from Lemma \ref{lem_locally_maximal_from_strongly_cyclic}.
    Let $w\in B$ such that $\norm{v-w}<1$.
    By Theorem \ref{thm_strongly_cyclic_spaces}, it is enough to show that $\norm{}$ is normalized and locally maximal at $w$.
    That $\norm{}$ is normalized at $w$ follows from the strong triangle inequality.
    To show that $\norm{}$ is locally maximal at $w$, let $\norm{}'\in\Norms(V)^G$ a norm that is dominated by $\norm{}$ and normalized at $w$.
    By Lemma \ref{lem_locally_maximal_from_strongly_cyclic} we can write $v-w=\sum_{g\in G}\lambda_g\cdot g(v)$, where $(\lambda_g)_{g\in G}$ is summable and $\max_{g\in G}(\lambda_g)<1$.
    Therefore, if $\norm{w}'=1$, then also $\norm{v}'=1$.
    Since $\norm{}$ is locally maximal at $v$, $\norm{}'\leq \norm{}$.
    This prove that $\norm{}$ is also locally maximal at $w$.
    \end{proof} 

\begin{comment}-------------------------------------------------------------------------
\begin{prop}
Let $V$ be a cyclic representation of $V$, $v\in V$ a cyclic vector.
Assume that $\Norms(V)^G$ is not empty and let $\Ucomp{V}$ be the universal completion of $V$.
Then any vector $w\in \Ucomp{V}$ with $\norm{w-v}_v<1$ is strongly cyclic in $\Ucomp{V}$.
\end{prop} 
        \begin{proof}
        By Theorem \ref{thm_strongly_cyclic_spaces} it is enough to show that $\norm{}_v$ is locally maximal at $w$.
        Let $\norm{}$ be a $G$-invariant norm on $\Ucomp{V}$ that is normalized at $w$ and dominated by $\norm{}_v$.
        By assumption, $w$ can be written as
        \[w=v+\sum_{e\neq g\in G}\lambda_g\cdot g(v)\]
        where $(\lambda_g)_{g\in G}$ is summable and $\abs{\lambda_g}_p<1$ for any $g$.
        By the strong triangle inequality, $\norm{w}=\norm{v}$, so $\norm{v}=1$.
        The same argument shows that $\norm{}_v$ is normalized at $w$.
        Since $\norm{}_v$ is locally maximal at $v$, $\norm{}\leq \norm{}_v$.
        This proves that $\norm{}_v$ is also locally maximal at $w$.
        \end{proof} 
-------------------------------------------------------------------------------------\end{comment}

\begin{prop}
Let $(B,\norm{})$ be a Banach representation of $G$.
The set of all strongly cyclic vectors in $B$ is open (possibly empty).
\end{prop} 
    \begin{proof}
    Assume that the set of strongly cyclic vectors in $B$ is not empty.
    Let $v$ be a strongly cyclic vector in $B$.
    By the previous proposition, all the vectors in the open unit ball around $v$ with respect to $\norm{}_{v,B}$ are strongly cyclic.
    By Lemma \ref{lem_locally_maximal_from_strongly_cyclic}, the open unit ball of $\norm{}_{v,B}$ is open in $B$.
    \end{proof} 
    
\begin{cor}\label{cor_proper_subspaces}
Let $(B,\norm{})$ be a Banach representation of $G$ and assume that $B$ contains a strongly cyclic vector.
If $B$ is not strongly irreducible, $B$ contains a non-zero proper closed sub-representation.
\end{cor} 
    \begin{proof}
    Let $U$ be the open subset of strongly cyclic vectors in $B$.
    By assumption, $U$ is not empty.
    Assume that $B$ is not strongly irreducible and let $0\neq w\in B$ a non strongly cyclic vectors.
    Denote by $W$ the algebraic representation generated by $w$ in $B$, then $W\subset B\backslash U$.
    Thus, the closure of $W$ is a proper non-zero closed sub-representation of $B$.
    \end{proof} 


\begin{thm}\label{Thm_strong_irreducibility}
Let $(B,\norm{})$ be a Banach representation of $G$.
The following are equivalent.
    \begin{enumerate}
    \item $B$ is strongly irreducible.
    \item $B$ is topologically irreducible and there exists a strongly cyclic vector in $B$.
    \item $B$ is topologically irreducible and there exists a locally maximal norm at some vector $0\neq v\in B$ on $B$, equivalent to $\norm{}$.
    \item Any non-zero $G$-equivariant bounded map $B'\map B$, where $(B',\norm{}')$ is a Banach representation of $G$, is surjective.
    \item $(B,\norm{})$ is isomorphic to a quotient of a universal completion of a cyclic representation of $G$ by a maximal sub-representation.
    \end{enumerate}
\end{thm} 
    \begin{proof}
    The implication $(1)\Rightarrow (2)$ is trivial and the implication $(2)\Rightarrow(1)$ follows from Corollary \ref{cor_proper_subspaces}.
    The equivalence $(2)\iff (3)$ follows from Theorem \ref{thm_strongly_cyclic_spaces}.
    Next we show $(1)\iff(4)$.
    Assume $(1)$. 
    Let $(B',\norm{}')$ be a Banach representation of $G$ and $T:B'\map B$ a non-zero $G$-equivariant bounded map.
    Let $0\neq v\in Im(T)$.
    Then $v$ is strongly cyclic and by a previous observation, $T$ is surjective.
    Assume $(4)$.
    Let $v\in B$ non-zero.
    Let $V$ be the algebraic representation generated by $v$ in $B$.
    The map $I:\Ucomp{V}\map B$ is a non-zero $G$-equivariant bounded map, so by assumption $I$ is surjective.
    Thus, by a previous observation, $v$ is strongly cyclic in $B$.
    Finally, we prove $(2)\iff (5)$.
    Assume $(2)$.
    Let $v\in B$ be a strongly cyclic vector, let $V$ be the algebraic representation generated by $v$ and $I:\Ucomp{V}\map B$.
    Since $v$ is strongly cyclic, $I$ is surjective.
    If $W\subset\Ucomp{V}$ denotes the kernel of $I$, then $\Ucomp{V}/W$ is isomorphic to $B$.
    Since $B$ is topologically irreducible, $W$ is a maximal closed sub-representation.
    Assume $(5)$.
    As a quotient of a universal completion of a cyclic representation by a maximal sub-representation, $B$ is topologically irreducible and contains a strongly cyclic vector. 
    By Corollary \ref{cor_proper_subspaces}, $B$ is strongly cyclic.
    \end{proof} 

\begin{prop}\label{prop_small_representations}
Let $(B_i,\norm{}_i)$, for $i=1,...,n$, be pairwise non-isomorphic strongly irreducible Banach representations of $G$. 
Let $B$ be the Banach representation $B=\bigoplus_{i=1}^nB_i$, equipped with the norm $\max\sog{\norm{}_1,...,\norm{}_n}$.
Then any $x=(x_1,...,x_n)\in B$ such that $x_i\neq 0$ for all $i$ is strongly cyclic in $B$.
\end{prop} 
    \begin{proof}
    By induction on $n$.
    The case $n=1$ is trivial.
    Assume that $n>1$ and that the claim holds for any $1\leq k<n$.
    By Theorem \ref{thm_strongly_cyclic_spaces}, it is enough to show that for any Banach representation $(B',\norm{}')$ of $G$ and any continuous map of representations $T:B'\map B$, if $x$ lies in the image of $T$ then $T$ is surjective.
    Let $T:B'\map B$ be such a map.
    Denote by $P_1:B\map B_1$ and $P_2:B\map \bigoplus_{i=2}^nB_i$ the projections.
    Since $x_1$ lies in the image of $P_1\circ T$ and $(x_2,...,x_n)$ lies in the image of $P_2\circ T$, it follows from the induction hypothesis that both $P_1\circ T$ and $P_2\circ T$ are surjective.
    Let $K_1,K_2$ be the kernels of $P_1\circ T$ and $P_2\circ T$ respectively.
    Then $K_2$ is not contained in $K_1$, for otherwise we would have a non-zero map $\bigoplus_{i=2}^nB_i\map B_1$.
    Such a map would give a non-zero map between one of the $B_i$, for $i\geq 2$, and $B_1$.
    Since both $B_1$ and $B_i$ are strongly irreducible, such a map must be an isomorphism, contradicting the hypothesis.
    Therefore, the restriction of $P_1\circ T$ to $K_2$ is a non-zero map, hence surjective since $B_1$ is strongly irreducible.
    It follows that $B_1$ is contained in the image of $T$.
    Similarly, for any $1\leq i\leq n$, $B_i$ is contained in the image of $T$.
    Thus, $T$ is surjective.
    \end{proof} 
    
We end this section with an example of a topologically irreducible Banach representation which is not strongly irreducible.
    \begin{exmp}
    Let $C(\Z_p)$ be the space of continuous functions on $\Z_p$ with values in $\C_p$, equipped with the sup norm $\supnorm{}$.
    We choose $\zeta\in \C_p$, not a root of unity, such that $\abs{\zeta-1}_p<1$ and denote by $G$ the group generated by translations and by multiplications by $\zeta^{nx}$, $n\in\Z$.
    The sup norm is invariant under the action of $G$, so $C(\Z_p)$ is a Banach representation of $G$.
    
    As a representation of $G$, $C(\Z_p)$ is topologically irreducible, as we now show.
    Let $A$ be the linear span of the functions $\zeta^{nx}$, $n\in\Z$.
    Then $A$ is an algebra in $C(\Z_p)$ that separates points and contains the constant functions.
    By the $p$-adic Stone-Weierstrass theorem (\cite{kaplansky1995weierstrass}), $A$ is dense in $C(\Z_p)$.
    Let $0\neq f\in C(\Z_p)$ and $W$ be the closed sub-representation generated by $f$.
    We will show that $W=C(\Z_p)$.
    Applying translations, $W$ contains a nowhere vanishing functions $g(x)$.
    Then $A\cdot g\subset W$ is dense in $C(\Z_p)$, so $W=C(\Z_p)$.
    
    We show that the constant function $\textbf{1}(x)$ is not a strongly cyclic vector, thus $C(\Z_p)$ is not strongly irreducible.
    Let $f(x)\in C(\Z_p)$ and assume that it can be written as
    \[f(x)=\sum_{n\in\Z}\lambda_n\cdot \zeta^{nx},\]
    where $\limit{n}\lambda_n=0$.
    The Mahler expansion of $f(x)$ is 
        \begin{align*}
        f(x)=
        \sum_{n\in\Z}\lambda_n\cdot (\zeta^{n})^x
        =\sum_{n\in\Z}\lambda_n\cdot \sum_{k=0}^\infty (\zeta^n-1)^k\cdot\binom{x}{k}
        =\sum_{k=0}^\infty \sog{\sum_{n\in\Z}\lambda_n\cdot (\zeta^n-1)^k}\cdot\binom{x}{k}
        =\sum_{k=0}^\infty b_k\cdot\binom{x}{k}.
        \end{align*} 
    There exists $0<\eps<1$ such that $\pabs{\zeta^n-1}<\eps$ for all $n\in\Z$.
    Let $m=\max_{n\in\Z}\pabs{\lambda_n}$.
    Then the coefficients $(b_k)_{k=1}^\infty$ obey the asymptotic formula
    \[\pabs{b_k}\leq m\cdot \eps^k.\]
    
    In particular, the function $f\in C(\Z_p)$ with Mahler expansion $f(x)=\sum_{k=0}^\infty p^k\cdot\binom{x}{p^k}$ is not of the form $\sum_{n\in\Z}\lambda_n\cdot\zeta^{nx}$.
      
    \end{exmp} 