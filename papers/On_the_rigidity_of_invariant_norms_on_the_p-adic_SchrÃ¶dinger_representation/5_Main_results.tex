%%%%%%%%%%%%%%%%%%%%%%%%%%%%%%%%%%%%---------------------------
\section{The main results}
%%%%%%%%%%%%%%%%%%%%%%%%%%%%%%%%%%%%---------------------------
In this section and for the rest of this paper all the representations are assumed to be over $\C_p$.
Fix a non-trivial smooth character $\psi:(\Q_p,+)\map \C_p^\times$, and let $(\rho_\psi,\schw)$ be the Schrödinger representation of the Heisenberg group $\heis=\heis_{2d+1}(\Q_p)$.
In particular, the functions in $\schw=\schw(\Q_p^d)$ are valued in $\C_p$.
The action of $\heis$ on $\schw$ is generated by translations and multiplication by smooth characters.
An $\heis$-invariant norm on $\schw$ is therefore a norm $\norm{}$ on $\schw$ such that
\[\norm{f(x+a)}=\norm{f(x)},\ \ \ \norm{\chi(x)\cdot f(x)}=\norm{f(x)}\]
 for any $f\in\schw$, any $a\in\Q_p$ and any smooth character $\chi$ of $\Q_p^d$.
 %\textcolor{blue}{I should remark something about the that $\psi$ does not appear in the notations. Also about the fact that the invariant norms are independent of $\psi$.}

Our main results concern a family of $\heis$-invariant norms on $\schw$ with a surprising rigidity.
This family is the orbit of the sup norm by intertwining operators.
In the first sub-section we define these norms and show that they are parameterized by a Grassmannian.
In the second sub-section we state the main results.
The proofs are given in the next sections.

%%%%%%%%%%%%%%---------------
\subsection{A special family of $\mathcal{H}$-invariant norms parameterized by a Grassmannian}
%%%%%%%%%%%%%%---------------
Let $g=\twomat{a}{b}{c}{d}$ be a matrix in the symplectic group $\mathrm{Sp}_{2d}(\Q_p)$ and choose $T_g$, a corresponding intertwining operator. 
If $\norm{}$ is an $\heis$-invariant norm on $\schw$, the norm $f\mapsto \norm{T_g(f)}$ is also $\heis$-invariant.
Indeed, 
\[\norm{T_g([w,t]f)}=\norm{[wg,t]T_g(f)}=\norm{T_g(f)}.\]
As the $T_g$ are determined up to a constant, this defines a right action of $\mathrm{Sp}_{2d}(\Q_p)$ on the space $\invHomothety$ of homothety classes of $\heis$-invariant norm on $\schw$.
If $x\in\invHomothety$ denotes the homothety class of the norm $\norm{}$, then we denote by $xg$ the homothety class of the norm $\norm{T_g(\cdot )}$.
Then $(xg_1)g_2=x(g_1g_2)$ are both equal to the homothety class of the norm $\norm{T_{g_1}(T_{g_2}(\cdot))}$.

An important example of an $\heis$-invariant norm on $\schw$ is the sup norm:
\[\supnorm{f}=\sup_{x\in \Q_p^d}\ \abs{f(x)}_p.\]
In the following proposition we determine the stabilizer in $\mathrm{Sp}_{2d}(\Q_p)$ of the homothety class of the sup norm.

\begin{prop}\label{prop_norms_Grassmannian}
Let $g=\twomat{a}{b}{c}{d}$ be a matrix in the symplectic group $\mathrm{Sp}_{2d}(\Q_p)$ and $T_g$ a corresponding intertwining operator.
The norms $\supnorm{}$ and $\supnorm{T_g(\cdot)}$ are homothetic if and only if they are equivalent, if and only if $c=0$.
\end{prop} 
    \begin{proof}
    If $c=0$, Proposition \ref{prop_intertwining_formula} says that there exists $\lambda\in\C_p^\times$ such that
    \[T_g(f)(x)=\lambda\cdot \psi\sog{\frac{1}{2}(xa)\dotprod(xb)}\cdot f(xa).\]
    As $a$ must be invertible, $\supnorm{f(xa)}=\abs{\lambda}_p\cdot\supnorm{f(x)}$.
    Thus, $\supnorm{}$ and $\supnorm{T_g(\cdot)}$ are homothetic and therefore equivalent.
    
    Assume that $c\neq 0$ and let $k\geq 1$ be the dimension of $Im(c)$.
    Recall that $c$ acts on $\Q_p^d$ by $v\mapsto v\cdot c$.
    Choose a basis $v_1,..,v_k$ of $Im(c)$ and complete it to a basis $v_1,...,v_k,v_{k+1},...,v_d$ of $\Q_p^d$.
    Let $U_n$ and $V_n$ be the compact open sets in $\Q_p^d$ and in $Im(c)$ respectively, given by
    \[U_n=\braces{\sum_{i=1}^d\lambda_iv_i\ |\ \lambda_1,...,\lambda_d\in p^n\Z_p},\ \ \ 
    V_n=\braces{\sum_{i=1}^k\lambda_iv_i\ |\ \lambda_1,...,\lambda_k\in p^n\Z_p}.\]
    Denote by $f_n(x)$ the characteristic function of $U_n$.
    Note that $f_n(x)\in\schw$.
    By Proposition \ref{prop_intertwining_formula}, there exists a Haar distribution $d\mu$ on $Im(c)$ such that
    \[T_g(f_n)(x)=\intop_{Im(c)}\psi\sog{\frac{1}{2}(xa)\dotprod(xb)-(xb)\dotprod y+\frac{1}{2}y\dotprod(yd)}\cdot f_n(xa+y)\ d\mu(y).\]
    We may assume that $\mu(V_0)=1$.
    Substituting $x=0$, we obtain
        \begin{align*}
        T_g(f_n)(0)
        =\intop_{Im(c)}\psi\sog{\frac{1}{2}y\dotprod(yd)}\cdot f_n(y)\ d\mu(y)
        =\intop_{V_n}\psi\sog{\frac{1}{2}y\dotprod(yd)}\ d\mu(y).
        \end{align*} 
    When $n$ is sufficiently large, $\frac{1}{2}y\dotprod (yd)\in \ker(\psi)$ for any $y\in V_n$, so
    \[T_g(f_n)(0)=\intop_{V_n}1\ d\mu(y)=p^{-nk}.\]
    Thus, $\limit{n}\supnorm{T_g(f_n)}=\infty$, whereas $\supnorm{f_n}=1$ for any $n$.
    Then $\norm{}_g$ and $\supnorm{}$ are not equivalent and therefore not homothetic.
    \end{proof} 

Let $P$ be the Siegel parabolic subgroup 
\[P=\braces*{\twomat{a}{b}{0}{d}\in \mathrm{Sp}_{2d}(\Q_p)},\]
and denote $\Grassmannian=P\backslash \mathrm{Sp}_{2d}(\Q_p)$.
Then $\Grassmannian$ is the Grassmannian of maximal isotropic subspaces of $(W,\omega)$.
\begin{defn}
We denote the point that corresponds to $P$ in $\Grassmannian$ by $\infty$.
For any $\alpha=Pg\in \Grassmannian$ we denote by $\norm{}_\alpha$ the unique $\heis$-invariant norm in the homothety class of $\supnorm{T_g(\cdot)}$ that is normalize at $\textbf{1}_{\Z_p^d}(x)$.
\end{defn} 

%%%%%%%%%%%%%%---------------
\subsection{The main results}
%%%%%%%%%%%%%%---------------
Our deepest results are Theorems \ref{thm_strong_minimality} and \ref{thm_strong_minimality_Z_p} below.
\begin{thm}[Rigidity]\label{thm_strong_minimality}
Let $\alpha\in\Grassmannian$.
If $\norm{}\in \invNorms$ is an $\heis$-invariant norm on $\schw$ that is dominated by $\norm{}_\alpha$, then $\norm{}=r\cdot \norm{}_\alpha$ for some constant $r>0$.    
\end{thm} 
In particular, each of the norms $\norm{}_\alpha$ is locally maximal at every non-zero vector in the completion of $\schw$ in it. 

The following proposition gives some basic properties of the completions of $\schw$ by a norm $\norm{}_\alpha$.
The first property follows from Theorem \ref{thm_strong_minimality} in conjunction with Theorem \ref{thm_strongly_cyclic_spaces} while the other are simpler.
\begin{prop}\label{topologically_irreducible}
Let $\alpha\in\Grassmannian$ and $\norm{}_\alpha$ the corresponding norm.
    \begin{enumerate}
    \item The completion $\completion{\schw}{\norm{}_\alpha}$ is a strongly irreducible Banach representation of $\heis$.
    \item The smooth part of $\completion{\schw}{\norm{}_\alpha}$ is precisely $\schw$.
    \item Let $\beta\in\Grassmannian$.
    The space of continuous $\heis$-equivariant maps from $\completion{\schw}{\norm{}_\alpha}$ to $\completion{\schw}{\norm{}_\beta}$ is given by
    \[Hom_{\heis}\sog{\completion{\schw}{\norm{}_\alpha},\completion{\schw}{\norm{}_\beta}}\simeq
    \begin{cases}
    \C_p & \alpha=\beta\\
    0    & \alpha\neq \beta
    \end{cases}.\]
    \end{enumerate}
\end{prop} 
Theorem \ref{thm_strong_minimality} and Proposition \ref{topologically_irreducible} form a $p$-adic analog of a classical result about unitary representations:
\begin{thm}[Classical known result]
Let $\schw^{\C}$ denote the space of $\C$-valued Schwartz functions on $\Q_p^d$, and $\rho_\psi^{\C}$ the complex Schrödinger representation.
Then, up to a positive scalar, there exists a unique $\heis$-invariant unitary structure on $\schw^{\C}$.
The completion with respect to the associated norm is topologically irreducible and its smooth part is the space $\schw^{\C}$.
\end{thm} 

Using Theorem \ref{Thm_strong_irreducibility} we will derive the following rigidity property.

\begin{thm}\label{thm_Rigidity}
Let $\alpha\in\Grassmannian$.
Let $(B,\norm{})$ be a topologically irreducible Banach representation of $\heis$ (see Definition \ref{def_Banach_rep}).
Assume that we are in one of the two following cases.
    \begin{enumerate}
    \item $F:B\map \completion{\schw}{\norm{}_\alpha}$ is a non-zero continuous map of representations.
    \item $F:\completion{\schw}{\norm{}_\alpha}\map B$ is a non-zero continuous map of representations.
    \end{enumerate}
Then $F$ is an isomorphism.
Moreover, there exists $r>0$ such that by replacing $\norm{}$ with $r\cdot \norm{}$, $F$ becomes an isometric isomorphism.
\end{thm} 

In order to prove Theorem \ref{thm_strong_minimality} we will prove its $\Z_p$-analog.
Let $\schw(\Z_p^d)$ denote the space of locally constant, $\C_p$-valued functions on $\Z_p^d$.
The sup norm on $\schw(\Z_p^d)$ is invariant under translations and under multiplication by the smooth characters of $\Z_p^d$.
Here, as before, a smooth character of $\Z_p^d$ is a homomorphism $\chi:(\Z_p^d,+)\map \C_p^\times$ with an open kernel.

\begin{thm}\label{thm_strong_minimality_Z_p}
Let $\norm{}$ be a norm on $\schw(\Z_p^d)$ that is dominated by the sup norm and invariant under translations and multiplication by smooth characters.
Then $\norm{}=r\cdot \supnorm{}$ for some $r>0$.
\end{thm} 

%%%%%%%%%%%%%%---------------
\subsection{A new proof of the main results of Fresnel and de Mathan}\label{subsection_new_proof_Fresnel_de_Mathan}
%%%%%%%%%%%%%%---------------
Our method gives a new proof of the main results in \cite{demathan}.
In that paper, Fresnel and de Mathan studied the Fourier transform 
\[\mathcal{F}:C_0(\Q_p/\Z_p)\map C(\Z_p)\]
attached to a smooth character $\psi:(\Q_p,+)\map \C_p^\times$ with $\ker(\psi)=\Z_p$.
Here, $C_0(\Q_p/\Z_p)$ denotes the space of $\C_p$-valued functions on $\Q_p/\Z_p$ which go to zero at infinity, and is equipped with the sup norm.
The main results in \cite{demathan} are stated in the following theorem.
\begin{thm}
The Fourier transform $\mathcal{F}$ is surjective and is not injective.
Moreover, if $K$ denotes its kernel, the induced map 
\[C_0(\Q_p/\Z_p)/K\map C(\Z_p)\]
is a surjective isometry.
\end{thm} 

    \begin{proof}
    Let $\mathcal{H}(\Z_p)$ be the following subgroup of the Heisenberg group $\mathcal{H}_3(\Q_p)$,
    \[\heis(\Z_p)=\braces{[a,b,t]\ |\ a\in\Z_p}.\]
    Then $\heis(\Z_p)$ acts on $C(\Z_p)$ by the usual rule
    \[([a,b,t]f)(x)=\psi(t+bx)\cdot f(x+a),\]
    namely, by translations and multiplication by smooth characters.
    The group $\heis(\Z_p)$ also acts on $C(\Q_p/\Z_p)_0$ by the rule
    \[([a,b,t]g)(x)=\psi(t-ab+ax)\cdot g(x-b).\]
    It is easy to verify that \[\mathcal{F}:C_0(\Q_p/\Z_p)\map C(\Z_p)\]
    is a continuous homomorphism of Banach representations of $\heis(\Z_p)$.
    By Theorem \ref{thm_strong_minimality_Z_p}, the sup norm on $C(\Z_p)$ is locally maximal with respect to any $f\in C(\Z_p)$ with $\supnorm{f}=1$.
    It follows from Theorem \ref{thm_strongly_cyclic_spaces} that any non-zero $f\in C(\Z_p)$ is strongly cyclic, hence that $C(\Z_p)$ is a strongly irreducible representation.
    By theorem \ref{Thm_strong_irreducibility}, $\mathcal{F}$ is surjective.
    If $\mathcal{F}$ were also injective, it would be, by the open mapping theorem, an isomorphism of Banach spaces.
    To see that this is not true, consider the characteristic functions $\phi_n(x):=\textbf{1}_{p^{-n}\Z_p}(x)\in C_0(\Q_p/\Z_p)$.
    Then $\supnorm{\phi_n}=1$, while $\supnorm{\mathcal{F}(\phi_n)}=\supnorm{p^n\cdot\textbf{1}_{p^n\Z_p}}=p^{-n}$.
    Therefore, $\mathcal{F}$ is not injective.
    Finally, denoting the kernel of $\mathcal{F}$ by $K$, we have the induced isomorphism of Banach representations
    \[C(\Q_p/\Z_p)_0/K\map C(\Z_p).\]
    By \ref{thm_strong_minimality_Z_p}, there exists a real number $r>0$ such that by taking the norm $r\cdot \supnorm{}$ on $C(\Z_p)$, the above isomorphism is an isometry.
    To show that $r=1$, it is enough to show that the image of the characteristic function $\phi_0$ in the quotient $C_0(\Q_p/\Z_p)/K$ has norm $1$.
    Note that $\phi_0$ is a strongly cyclic vector in $C_0(\Q_p/\Z_p)$, and that $\supnorm{}$ is a normalized and locally maximal at $\phi_0$.
    Thus, by Proposition \ref{prop_open_unit_ball_of_strongly_cyclic}, the open unit ball around $\phi_0$ in $C_0(\Q_p/\Z_p)$ consists of strongly cyclic vectors.
    In particular, all elements of $K$ are at distance at least one from $\phi_0$.
    It follows that the image of $\phi_0$ in the quotient has norm $1$.
    Therefore, $r=1$.
    \end{proof}
    
    \begin{remark}
    In \cite{demathan}, Fresnel and de Mathan first show that $\mathcal{F}$ is not injective by constructing non-zero elements in the kernel of $\mathcal{F}$.
    These elements have some special properties which then enable them to show that $\mathcal{F}$ is surjective.
    \end{remark} 

