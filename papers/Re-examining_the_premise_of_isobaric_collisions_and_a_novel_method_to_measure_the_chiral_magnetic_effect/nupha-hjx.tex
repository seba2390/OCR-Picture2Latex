% Template for Elsevier CRC journal article
% version 1.2 dated 09 May 2011

% This file (c) 2009-2011 Elsevier Ltd.  Modifications may be freely made,
% provided the edited file is saved under a different name

% This file contains modifications for Nuclear Physics B Proceedings Supplement

% Changes since version 1.1
% - added "procedia" option compliant with ecrc.sty version 1.2a
%   (makes the layout approximately the same as the Word CRC template)
% - added example for generating copyright line in abstract

%-----------------------------------------------------------------------------------

%% This template uses the elsarticle.cls document class and the extension package ecrc.sty
%% For full documentation on usage of elsarticle.cls, consult the documentation "elsdoc.pdf"
%% Further resources available at http://www.elsevier.com/latex

%-----------------------------------------------------------------------------------

%%%%%%%%%%%%%%%%%%%%%%%%%%%%%%%%%%%%%%%%%%%%%%%%%%%%%%%%%%%%%%
%%%%%%%%%%%%%%%%%%%%%%%%%%%%%%%%%%%%%%%%%%%%%%%%%%%%%%%%%%%%%%
%%                                                          %%
%% Important note on usage                                  %%
%% -----------------------                                  %%
%% This file should normally be compiled with PDFLaTeX      %%
%% Using standard LaTeX should work but may produce clashes %%
%%                                                          %%
%%%%%%%%%%%%%%%%%%%%%%%%%%%%%%%%%%%%%%%%%%%%%%%%%%%%%%%%%%%%%%
%%%%%%%%%%%%%%%%%%%%%%%%%%%%%%%%%%%%%%%%%%%%%%%%%%%%%%%%%%%%%%

\documentclass[3p,times,procedia]{elsarticle}
\usepackage{nupha_ecrc}

%% The ecrc package defines commands needed for running heads and logos.
%% For running heads, you can set the journal name, the volume, the starting page and the authors

%% set the volume if you know. Otherwise `00'
\volume{00}

%% set the starting page if not 1
\firstpage{1}

%% Give the name of the journal
\journalname{Nuclear Physics A}

%% Give the author list to appear in the running head
%% Example \runauth{C.V. Radhakrishnan et al.}
\runauth{}

%% The choice of journal logo is determined by the \jid and \jnltitlelogo commands.
%% A user-supplied logo with the name <\jid>logo.pdf will be inserted if present.
%% e.g. if \jid{yspmi} the system will look for a file yspmilogo.pdf
%% Otherwise the content of \jnltitlelogo will be set between horizontal lines as a default logo

%% Give the abbreviation of the Journal.
\jid{nupha}

%% Give a short journal name for the dummy logo (if needed)
\jnltitlelogo{Nuclear Physics A}

%% Hereafter the template follows `elsarticle'.
%% For more details see the existing template files elsarticle-template-harv.tex and elsarticle-template-num.tex.

%% Elsevier CRC generally uses a numbered reference style
%% For this, the conventions of elsarticle-template-num.tex should be followed (included below)
%% If using BibTeX, use the style file elsarticle-num.bst

%% End of ecrc-specific commands
%%%%%%%%%%%%%%%%%%%%%%%%%%%%%%%%%%%%%%%%%%%%%%%%%%%%%%%%%%%%%%%%%%%%%%%%%%

%% The amssymb package provides various useful mathematical symbols
\usepackage{amssymb}
%% The amsthm package provides extended theorem environments
%% \usepackage{amsthm}

%% The lineno packages adds line numbers. Start line numbering with
%% \begin{linenumbers}, end it with \end{linenumbers}. Or switch it on
%% for the whole article with \linenumbers after \end{frontmatter}.
%% \usepackage{lineno}

%% natbib.sty is loaded by default. However, natbib options can be
%% provided with \biboptions{...} command. Following options are
%% valid:

%%   round  -  round parentheses are used (default)
%%   square -  square brackets are used   [option]
%%   curly  -  curly braces are used      {option}
%%   angle  -  angle brackets are used    <option>
%%   semicolon  -  multiple citations separated by semi-colon
%%   colon  - same as semicolon, an earlier confusion
%%   comma  -  separated by comma
%%   numbers-  selects numerical citations
%%   super  -  numerical citations as superscripts
%%   sort   -  sorts multiple citations according to order in ref. list
%%   sort&compress   -  like sort, but also compresses numerical citations
%%   compress - compresses without sorting
%%
%% \biboptions{comma,round}

% \biboptions{}

% if you have landscape tables
\usepackage[figuresright]{rotating}

% put your own definitions here:
%   \newcommand{\cZ}{\cal{Z}}
%   \newtheorem{def}{Definition}[section]
%   ...

% add words to TeX's hyphenation exception list
%\hyphenation{author another created financial paper re-commend-ed Post-Script}

% declarations for front matter

\newcommand {\snn}	{\sqrt{s_{_{\rm NN}}}}
\newcommand {\gevc}	{GeV/$c$}
\newcommand {\gevcc}	{GeV/$c^2$}
\newcommand {\mevcc}	{MeV/$c^2$}
\newcommand {\Nch}	{N_{\rm ch}}
\newcommand {\Npart}	{N_{\rm part}}

\newcommand {\hic}	{{\sc hic}}
\newcommand {\rhic}	{{\sc rhic}}
\newcommand {\lhc}	{{\sc lhc}}
\newcommand {\dft}	{{\sc dft}}
\newcommand {\ampt}	{{\sc ampt}}
\newcommand {\hijing}	{{\sc hijing}}
\newcommand {\mcg}	{{\sc mcg}}
\newcommand {\ws}	{{\sc ws}}
\newcommand {\cme}	{{\mbox{\sc cme}}}
\newcommand {\cs}	{{\sc cs}}

\newcommand {\RP}	{{\sc rp}}
\newcommand {\PP}	{{\sc pp}}
\newcommand {\EP}	{{\sc ep}}


%\newcommand {\psir}	{\psi_2^{(r)}}
\newcommand {\phir}	{\phi_{_{r}}}
\newcommand {\rt}	{r_{\perp}}
%\newcommand {\pt}	{p_{T}}

\newcommand {\rhows}	{\rho_{_{\rm WS}}}
\newcommand {\psiRP}	{\psi_{_{\rm RP}}}
\newcommand {\psiPP}	{\psi_{_{\rm PP}}}
\newcommand {\psiEP}	{\psi_{_{\rm EP}}}

\newcommand {\etwo}	{\epsilon_2}
\newcommand {\epsi}	{\etwo\{\psi\}}
\newcommand {\eRP}	{\etwo\{\psiRP\}}
\newcommand {\ePP}	{\etwo\{\psiPP\}}
\newcommand {\vv}	{v_2}
\newcommand {\vRP}	{\vv\{\psiRP\}}
\newcommand {\vEP}	{\vv\{\psiEP\}}

\newcommand {\rbf}	{{\bf r}}
%\newcommand {\Bcos}	{\mean{(eB({\bf 0},0)/m_\pi^2)^2\cos2(\psiB-\psi)}}


%\newcommand {\note}[1]	{#1}

\newcommand {\vpsi}	{\vv\{\psi\}}
\newcommand {\Au}	{$^{197}_{\;\,79}$Au}
\newcommand {\Cu}	{$^{62}_{29}$Cu}
\newcommand {\Pb}	{$^{207}_{\;\,82}$Pb}
\newcommand {\Ru}	{$^{96}_{44}$Ru}
\newcommand {\Zr}	{$^{96}_{40}$Zr}
\newcommand {\CuCu}	{CuCu}
\newcommand {\PbPb}	{PbPb}
\newcommand {\RuRu}	{RuRu}
\newcommand {\ZrZr}	{ZrZr}
\newcommand {\AuAu}	{AuAu}
\newcommand {\bkg}	{{\mbox{\sc bkg}}}
\newcommand {\Qw}	{Q_{w}}
\newcommand {\rhoWS}	{\rho_{_{\rm WS}}}
\newcommand {\aEM}	{\alpha_{_{\rm EM}}}
\newcommand {\dmin}	{d_{\rm min}}
\newcommand {\tpc}	{{\sc tpc}}
\newcommand {\ftpc}	{{\sc ftpc}}
\newcommand {\zdc}	{{\sc zdc}}
%\newcommand {\rcme}	{r_{_{\cme/\bkg}}}
\newcommand {\rcme}	{r}
%\newcommand {\fcme}	{f_{_{\cme}}}
\newcommand {\fcmeRP}	{f^{\rm RP}_{_{\cme}}}
\newcommand {\fcmeEP}	{f^{\rm EP}_{_{\cme}}}

%\newcommand {\psir}	{\psi_{_{2}}^{(r)}}
\newcommand {\rpr}	{r^{\prime}}
\newcommand {\pt}	{p_{T}}
\newcommand {\vc}	{v_{_{2,c}}}
\newcommand {\phia}	{\phi_{\alpha}}
\newcommand {\phib}	{\phi_{\beta}}
\newcommand {\vEPobs}	{\vv\{\psiEP^{\rm rec}\}}
\newcommand {\psiEPobs}	{\psiEP^{\rm rec}}
\newcommand {\psiPPEP}	{\psi_{_{\rm PP(EP)}}}
\newcommand {\psione}	{\psi_{_{1}}}
\newcommand {\psiB}	{\psi_{_{\rm B}}}
\newcommand {\dpsi}	{\mean{\cos2(\psiPP-\psiRP)}}
\newcommand {\dpsiEP}	{\mean{\cos2(\psiEP-\psiRP)}/\res}
\newcommand {\res}	{\mathcal{R}_{\rm EP}}
\newcommand {\vTT}	{\vv\{2\}}
\newcommand {\vTF}	{\vv\{4\}}
\newcommand {\vtpc}	{\vv^{\mbox\tpc}}
\newcommand {\vftpc}	{\vv^{\mbox\ftpc}}
\newcommand {\vzdc}	{\vv^{\mbox\zdc}}
\newcommand {\vave}	{\frac{\vTT+\vTF}{2}}

\newcommand {\aee}	{a_{\etwo}^{\rm PP}}
\newcommand {\av}	{a_{\vv}^{\rm EP}}
\newcommand {\aB}	{a_{_{\Bsq}}}
\newcommand {\aPP}	{a^{\rm PP}}
\newcommand {\aEP}	{a^{\rm EP}}
\newcommand {\aBPP}	{\aB^{\rm PP}}
\newcommand {\aBEP}	{\aB^{\rm EP}}
\newcommand {\RPP}	{R^{\rm PP}}
\newcommand {\REP}	{R^{\rm EP}}
\newcommand {\Rexp}	{R^{\rm exp}}
\newcommand {\rPP}	{R_{\rm PP}}
\newcommand {\rEP}	{R_{\rm EP}}
\newcommand {\rPPEP}	{R_{\rm PP(EP)}}
\newcommand {\RPPEP}	{R^{\rm PP(EP)}}
\newcommand {\Bcos}	{\mean{(eB(\mathbf{0},0)/m_\pi^2)^2\cos2(\psiB-\psi)}}
%\newcommand {\Bcos}	{\mean{(eB/m_\pi^2)^2\cos2(\psiB-\psi)}}
\newcommand {\BcosPP}	{\mean{(eB(\mathbf{0},0)/m_\pi^2)^2\cos2(\psiB-\psiPP)}}
\newcommand {\BcosRP}	{\mean{(eB(\mathbf{0},0)/m_\pi^2)^2\cos2(\psiB-\psiRP)}}
\newcommand {\BcosEP}	{\mean{(eB(\mathbf{0},0)/m_\pi^2)^2\cos2(\psiB-\psiEP)}/\res}
\newcommand {\Bbf}	{\mathbf{B}}
\newcommand {\Bsq}	{B_{\rm sq}}
\newcommand {\Bpsi}	{\Bsq\{\psi\}}
\newcommand {\BRP}	{\Bsq\{\psiRP\}}
\newcommand {\BPP}	{\Bsq\{\psiPP\}}
\newcommand {\BEP}	{\Bsq\{\psiEP\}}
\newcommand {\psitwo}	{\psi_{_{2}}}
\newcommand {\gOS}	{\gamma_{_{\rm OS}}}
\newcommand {\gSS}	{\gamma_{_{\rm SS}}}
\newcommand {\dg}	{\Delta\gamma}
\newcommand {\dgpsi}	{\dg\{\psi\}}
\newcommand {\dgRP}	{\dg\{\psiRP\}}
\newcommand {\dgPP}	{\dg\{\psiPP\}}
\newcommand {\dgEP}	{\dg\{\psiEP\}}
\newcommand {\dgone}	{\dg\{\psione\}}
\newcommand {\dgtwo}	{\dg\{\psitwo\}}

\newcommand {\mean}[1]	{\langle #1\rangle}
\newcommand {\note}[1]	{{\color{red} #1}}



\begin{document}

\begin{frontmatter}

%% Title, authors and addresses

%% use the tnoteref command within \title for footnotes;
%% use the tnotetext command for the associated footnote;
%% use the fnref command within \author or \address for footnotes;
%% use the fntext command for the associated footnote;
%% use the corref command within \author for corresponding author footnotes;
%% use the cortext command for the associated footnote;
%% use the ead command for the email address,
%% and the form \ead[url] for the home page:
%%
%% \title{Title\tnoteref{label1}}
%% \tnotetext[label1]{}
%% \author{Name\corref{cor1}\fnref{label2}}
%% \ead{email address}
%% \ead[url]{home page}
%% \fntext[label2]{}
%% \cortext[cor1]{}
%% \address{Address\fnref{label3}}
%% \fntext[label3]{}

%% Instructions from Editor: Please use the following \dochead only in the preprint version (e-print arXiv etc.); 
%% use empty \dochead{} when submitting to Nuclear Physics A!
\dochead{XXVIIth International Conference on Ultrarelativistic Nucleus-Nucleus Collisions\\ (Quark Matter 2018)}
%\dochead{}
%% Use \dochead if there is an article header, e.g. \dochead{Short communication}
%% \dochead can also be used to include a conference title, if directed by the editors
%% e.g. \dochead{17th International Conference on Dynamical Processes in Excited States of Solids}

\title{Re-examining the premise of isobaric collisions and a novel method to measure the chiral magnetic effect}

%% use optional labels to link authors explicitly to addresses:
\author[label1]{Hao-jie Xu}
\author[label2]{Jie Zhao}
\author[label1]{Xiaobao Wang}
\author[label3]{Hanlin Li}
\author[label4,label5]{Zi-Wei Lin}
\author[label1]{Caiwan Shen}
\author[label1,label2]{Fuqiang Wang}
\address[label1]{School of Science, Huzhou University, Huzhou, Zhejiang 313000, China}
\address[label2]{Department of Physics and Astronomy, Purdue University, West Lafayette, Indiana 47907, USA}
\address[label3]{College of Science, Wuhan University of Science and Technology, Wuhan, Hubei 430065, China}
\address[label4]{Department of Physics, East Carolina University, Greenville, North Carolina 27858, USA}
\address[label5]{Key Laboratory of Quarks and Lepton Physics (MOE) and Institute of Particle Physics, Central China Normal University, Wuhan, Hubei 430079, China}

\begin{abstract}
In this proceeding we will show that the expectations of the isobaric $^{96}_{44}\mathrm{Ru}+^{96}_{44}\mathrm{Ru}$ and $^{96}_{40}\mathrm{Zr}+^{96}_{40}\mathrm{Zr}$ collisions on chiral magnetic effect (CME) search may not hold as originally anticipated due to large uncertainties in the isobaric nuclear structures. We demonstrate this using Woods-Saxon densities and the proton and neutron densities calculated by the density functional theory. Furthermore, a novel method is proposed to gauge background and possible CME contributions in the same system, intrinsically better than the isobaric collisions of two different systems.  We illustrate the method with Monte Carlo Glauber and AMPT (A Multi-Phase Transport) simulations.


%% Text of abstract
\end{abstract}

\begin{keyword}
	chiral magnetic effect \sep isobaric collisions  \sep density functional theory 
%% keywords here, in the form: keyword \sep keyword

%% MSC codes here, in the form: \MSC code \sep code
%% or \MSC[2008] code \sep code (2000 is the default)

\end{keyword}

\end{frontmatter}

%%
%% Start line numbering here if you want
%%
% \linenumbers

%% main text
\section{Introduction}
\label{sec:introduction}


In quantum chromodynamics (QCD), the interactions of quarks with topological gluon fields can induce chirality imbalance and parity violation in local domain under the approximate chiral symmetry restoration~\cite{Kharzeev:1998kz}. A chirality imbalance could lead to an electric current, or charge separation (\cs) %~\cite{Kharzeev:2004ey,Kharzeev:2007tn} 
in the direction of a strong magnetic field ($\Bbf$). %~\cite{Bzdak:2011yy,Deng:2012pc,Bloczynski:2012en,Kharzeev:2015znc}. 
This phenomenon is called the chiral magnetic effect (\cme)~\cite{Kharzeev:1998kz}. %,Muller:2010jd,Liu:2011ys
Searching for the \cme\ is one of the most active research in heavy ion collisions (\hic)~\cite{Kharzeev:2015znc,Zhao:2018ixy}. In \hic\ the \cs\ is commonly measured by the three-point correlator~\cite{Voloshin:2004vk}, $\gamma\equiv\cos(\phi_\alpha+\phi_\beta-2\psiRP)$, where $\phi_\alpha$ and $\phi_\beta$ are the azimuthal angles of two charged particles, and $\psiRP$ is that of the reaction plane (\RP, spanned by the impact parameter and beam directions) to which the $\Bbf$ produced by the incoming protons is perpendicular on average. %,Kharzeev:2007tn,Kharzeev:2015znc
Positive $\dg\equiv\gOS-\gSS$ ({\sc os}:opposite-sign, {\sc ss}:same -sign) signals, consistent with the \cme-induced \cs\ perpendicular to the \RP, have been observed~\cite{Adamczyk:2014mzf}. 
The signals are, however, inconclusive because of a large charge-dependent background arising from particle correlations (e.g.~resonance decays) coupled with the elliptic flow anisotropy ($\vv$)~\cite{Wang:2009kd}. 
%This background arises from the coupling of this elliptical anisotropy and the intrinsic decay correlation (nonflow). 
%$\dg\propto\mean{\cos(\alpha+\beta-2\phi_{\rho})\cos2(\phi_{\rho}-\psi)}\approx\mean{\cos(\alpha+\beta-2\phi_{\rho})}v_{2,\rho}$

To better control the background, isobaric collisions of \Ru+\Ru\ (\RuRu) and \Zr+\Zr\ (\ZrZr) have been proposed~\cite{Voloshin:2010ut}. One expects their backgrounds to be almost equal because of the same mass number, while the atomic numbers, hence $\Bbf$, differ by 10\%. 
This is verified by {\em Monte Carlo} Glauber (\mcg) calculations using the Woods-Saxon (\ws) density profile~\cite{Deng:2016knn}.
As a net result, the \cme\ signal-to-background ratio would be improved by over a factor of 7 in comparative measurements between RuRu and ZrZr collisions than in each of them individually~\cite{Deng:2016knn}.  
The isobaric collisions are planned for 2018 at \rhic; they would yield a \cme\ signal of $5\sigma$ significance with the projected data volume, %of $400\times10^6$ minimum bias (MB) events for each collision type. 
if one assumes that the \cme\ contributes 1/3 of the current $\dg$ measurement in \AuAu\ collisions.

However, there can be non-negligible deviations of the Ru and Zr nuclear densities from \ws. In this proceeding, we will show their effects on the sensitivity of isobaric collisions for the \cme\ search~\cite{Xu:2017zcn}, and a novel method will be proposed to avoid those uncertainties~\cite{Xu:2017qfs}.  

\section{Re-examining the premise of isobaric collisions}
Because of the different numbers of protons--which suffer from Coulomb repulsion--and neutrons, the structures of the \Ru\ and \Zr\ nuclei must not be identical.
By using density functional theory (\dft) , we calculate the Ru and Zr proton and neutron distributions using the well-known SLy4 mean field including pairing correlations (Hartree-Fock-Bogoliubov, HFB approach)~\cite{Wang:2016rqh}.
Those density distributions are shown in Fig.~\ref{fig:rho}.
Protons in Zr are more concentrated in the core, while protons in Ru, 10\% more than in Zr, are pushed more toward outer regions. The neutrons in Zr, four more than in Ru, are more concentrated in the core but also more populated on the nuclear skin. Theoretical uncertainties are estimated by using different sets of density functionals, SLy5 and SkM* for the mean field, with and without pairing (HFB/HF), and found to be small. %Their effects on the \RuRu\ vs.~\ZrZr\ $\etwo$ and $\Bbf$ differences, the main subject of this work, are negligible.
%

The $\etwo$ of the transverse overlap geometry in \RuRu\ and \ZrZr\ collisions is calculated event-by-event with \mcg~\cite{Xu:2014ada}, using the \dft\ nucleon densities in Fig.~\ref{fig:rho}.
$\Bbf(\rbf,t=0)$ is also calculated %$(\rbf,t)=0$ 
for \RuRu\ and \ZrZr\ collisions using the \dft\ proton densities. The calculations follow Ref.~\cite{Deng:2012pc}, with a finite proton radius (0.88~fm~ is used but the numeric value is not critical) to avoid the singularity at zero relative distance.  %as well as \ws\ of Eq.~(\ref{eq:ws}) with $\beta_2=0$. %0.8775~fm
%The $\Bbf$ direction, due to fluctuations, is not always perpendicular to the \RP\ or \PP~\cite{Bzdak:2011yy,Deng:2012pc,Bloczynski:2012en}. 
Two reference planes are used for each collision system: reaction plane ($\psiRP$) and participant plane ($\psiPP$).
For the \cme\ search with isobaric collisions, the relative differences in $\etwo$ and $\Bsq$ are of importance. Figure~\ref{fig:R} shows the relative differences $R(\ePP)$, $R(\eRP)$, $R(\BPP)$, and $R(\BRP)$; $R(X)$ is defined as
\begin{equation}
R(X)\equiv2(X_{\rm\RuRu}-X_{\rm\ZrZr})/(X_{\rm\RuRu}+X_{\rm\ZrZr})\;
\end{equation}
where $X_{\rm\RuRu}$ and $X_{\rm\ZrZr}$ are the $X$ values in \RuRu\ and \ZrZr\ collisions, respectively. The thick solid curves are the default results with the \dft\ densities in Fig.~\ref{fig:rho} . %The thin solid curves correspond to the \dft\ density case where Ru is deformed with $\beta_2=0.158$ and Zr is spherical. The dashed thin curves correspond to the \dft\ density case where Ru is spherical and Zr is deformed with $\beta_2=0.217$. These two cases and the \dft-\shm\ difference likely represent the maximal theoretical uncertainties on $R$ from the structure functions of these nuclei. 
The shaded areas correspond to theoretical uncertainties bracketed by the two \dft\ density cases where Ru is deformed with $\beta_2=0.158$ and Zr is spherical and where Ru is spherical and Zr is deformed with $\beta_2=0.217$. 
The hatched areas represent our results using \ws\ densities with the above two cases of nuclear deformities.
%The dotted lines indicate the expected trivial values using \ws\ (with identical shape for Ru and Zr): 0.19 for $R(\Bpsi)$ and of course zero for $R(\epsi)$.

\begin{figure}[hbt]
%    \hspace*{-0.02\textwidth}
    %\includegraphics[width=0.5\textwidth]{fig_dec12/AveragedB.pdf}
    %\includegraphics[width=0.5\textwidth]{fig_dec12/Fig3.pdf}\\
	  \begin{minipage}[t]{0.35\linewidth}
  \begin{center}
   	 \includegraphics[width=1.0\textwidth]{Fig1.pdf}
  \end{center}
  \vspace{-0.2in}
  \caption{(Color online) Proton and neutron density distributions of the \Ru\ and \Zr\ nuclei, assumed spherical, calculated by the \dft\ method.}
  \label{fig:rho}
	  \end{minipage}
	  \hspace*{0.08\textwidth}
	  \begin{minipage}[t]{0.55\linewidth}
  \begin{center}
    \includegraphics[width=1.0\textwidth]{Fig3.pdf}
  \end{center}
  \vspace{-0.2in}
  \caption{(Color online) Relative differences between \RuRu\ and \ZrZr\ collisions in $\epsi$ and $\Bpsi$ with respect to (a) $\psi=\psiRP$ and (b) $\psi=\psiPP$, using the \dft\ densities. The shaded areas correspond to \dft\ density uncertainties from Ru and Zr deformities; the hatched areas show the corresponding results using \ws\ density distributions.
%The dotted lines at $R(\Bsq)=0.19$ and $R(\etwo)=0$ indicate results using the \ws\ density profiles. 
}
  \label{fig:R}
	  \end{minipage}
\end{figure}

%$R(\BRP)$ is significantly smaller for our density profiles than for \ws\ except at large $b$. This is because the protons in Zr are relatively concentrated in the core, each contributing to a larger $\Bbf$ at $\rbf=0$ because of the shorter distances. $R(\BPP)$ is larger than $R(\BRP)$, and it is found to arise from a better alignment of $\psiPP$ with $\psiRP$ in \RuRu, by about 10\%, than in \ZrZr. This is because the Ru mass density outweights the Zr's in the outer region while Zr is more concentrated at the core, making the $\psiPP$ better determined in \RuRu\ than in \ZrZr. $R(\BPP)$ with \ws\ and with our calculated densities happen to be similar.

%%%%%%%%%%%%%%%%%%%%%%%%%%%%%%%%%%%%%%%%%%%%%%%%%%%%%%%%%%%%%%%%%%%%%%%%%%%%%%%%
We also investigate whether our density profiles would, in a dynamical model, lead to a final-state $\vv$ difference between \RuRu\ and \ZrZr\ collisions and whether the $\Bsq$ difference preserves with respect to the event plane (\EP) reconstructed from the final-state particle momenta. %, which constitutes the major background to the \cme\ observable. 
We employ A Multi-Phase Transport (\ampt) model with ``string melting''~\cite{Lin:2004en}, %(v2.26t5, available online at~\cite{AMPTcode}). 
which can reasonably reproduce heavy ion bulk data at \rhic\ and the \lhc~. We found that the general trends are similar to those in Fig.~\ref{fig:R}~\cite{Xu:2017zcn}.

From the \mcg\ and \ampt\ simulations, we find that the \dft\ nuclear densities, together with the Woods-Saxon (\ws) densities, yield wide ranges of differences in $\Bsq$ with respect to the participant plane (\PP) and the reaction plane (\RP). %On the other hand, those nuclear densities introduce, in contrast to \ws, a significant difference in $\etwo$ and $\vv$, as large as $\sim3$\%. %The situation becomes worse if the reaction plane (\RP) is used, a 10\% difference in $\etwo$ and $\vv$, and only a 10-15\% difference in $\Bsq$. 
It is further found that those nuclear densities introduce, in contrast to \ws, comparable differences in $\eRP$ ($\vRP$) and $\BRP$ with respect to the reaction plane (\RP), deminishing the premise of isobaric collisions to help identify the \cme. With respect to the participant plane (\PP), the $\ePP$ ($\vEP$) difference can still be sizable, as large as $\sim3$\%, possibly weakening the power of isobaric collisions for the \cme\ search~\cite{Xu:2017zcn}. 
%

\section{A movel method to measure the \cme}
Based on the above study~\cite{Xu:2017zcn}, we found that with respect to $\psiPP$, $\vv$ is stronger than that with respect to $\psiRP$. This is because elliptic flow ($\vv$) develops in relativistic heavy ion collisions from the anisotropic overlap geometry of the participant nucleons.
The magnetic field ($\Bbf$) is, on the other hand, produced mainly by spectator protons and its direction fluctuates nominally about $\psiRP$, not $\psiPP$. Therefore, $\Bbf$ with respect to $\psiPP$ is weaker than $\Bbf$ with respect to $\psiRP$. 
Our new method is based on the opposite behaviors in the fluctuations of the magnetic field and $v_2$ in a single nucleus-nucleus collision, thus bears minimal theoretical and experimental uncertainties.
It is convenient to define a relative difference~\cite{Xu:2017qfs},
\begin{equation}
\RPPEP(X)\equiv2\cdot\frac{X\{\psiRP\}-X\{\psiPPEP\}}{X\{\psiRP\}+X\{\psiPPEP\}}\;,
\end{equation}
where $X\{\psiRP\}$ and $X\{\psiPPEP\}$ are the measurements of quantity $X$ with respect to $\psiRP$ and $\psiPP$ (or $\psiEP$ described below), respectively. 
The upper panels of Fig.~\ref{fig} show $\RPP(\etwo)$ and $\RPP(\Bsq)$ calculated by a \mcg\ model for \Au+\Au\ (\AuAu), \Cu+\Cu\ (\CuCu), \RuRu,
\ZrZr\ collisions at \rhic\ and \Pb+\Pb\ (\PbPb) collisions at the \lhc. 
%The nucleon-nucleon cross sections are set to be 42 and 64~mb for \rhic\ and \lhc, respectively. 
Although a theoretical concept, the \RP\ may be assessed by Zero-Degree Calorimeters (\zdc) measuring sidewards-kicked spectator neutrons (directed flow $v_1$)~.
The lower panels of Fig.~\ref{fig} show \ampt\ simulation results of $\REP(\vv)$ and $\REP(\Bsq)$, compared to $\pm\rEP$ ($\rEP \equiv 2(1 - \mean{\cos2(\psiEP-\psiRP)}/\mathcal{R}_{EP})/(1 + \mean{\cos2(\psiEP-\psiRP)}/\mathcal{R}_{EP})$, where $\mathcal{R}_{EP}$ is the \EP\ resolution). 

\begin{figure*}
  \begin{center}
    \includegraphics[width=0.83\textwidth]{Fig1v2.pdf}
  \vspace{-0.2in}
  \caption{\label{fig}(Color online) Relative differences $\RPP(\etwo)$, $\RPP(\Bsq)$, $\RPP$ from \mcg\ (upper panel) and $\REP(\vv)$, $\REP(\Bsq)$, $\rEP$ from \ampt\ (lower panel) for (a,f) \AuAu, (b,g) \CuCu, (c,h) \RuRu, and (d,i) \ZrZr\ at \rhic, and (e,j) \PbPb\ at the \lhc. Both the \ws\ and \dft-calculated densities are shown for the \mcg\ results, while the used density profiles are noted for the \ampt\ results. Errors, mostly smaller than the symbol size, are statistical.}
  \end{center}
\end{figure*}

The commonly used $\dg$ variable contains, in addition to the \cme\ it is designed for, $\vv$-induced background, % contributions which likely dominate over \cme: %arising from genuine particle correlations, such as $\rho$ particle decay, coupled with $\vv$:
$\dgpsi=\cme(\Bpsi)+\bkg(\vpsi)$.
$\dgpsi$ can be measured with respect to $\psi=\psiRP$ (using the 1st order event plane $\psione$ by the \zdc) and $\psi=\psiEP$ (2nd order event plane $\psitwo$ via final-state particles). 
%The nominal direction for $\Bsq$ is $\psiRP$ and the nominal direction for $\vv$ is $\psiPP$. With respect to $\psiRP$ compared to $\psiPP$, $\Bsq$ is stronger and $\vv$ is weaker, and vice verse. Therefore, a $\dgRP$ measurement contains relatively more \cme\ contribution and a $\dgPP$ measurement contains relatively more background contribution. Comparative measurements of $\dgRP$ vs $\dgPP$ in the same collision system, if of sufficient statistical and systematic precision, should unveil the \cme. 
If $\bkg(\vv)$ is proportional to $\vv$~ and $\cme(\Bsq)$ to $\Bsq$~, then 
\begin{equation}
\REP(\dg)=2\frac{r(1-\aBEP)-(1-\av)}{r(1+\aBEP)+(1+\av)}\approx\frac{1-r}{1+r}\REP(\vv)\;.
\end{equation}
Here $\rcme\equiv\cme(\BRP)/\bkg(\vEP)$ can be considered as 
the relative \cme\ signal to background contribution.
If the experimental measurement $\REP(\dg)$ equals to $\REP(\vv)$ (i.e.~$\dg$ scales like $\vv$), then \cme\ contribution is zero; if $\REP(\dg)\approx-\REP(\vv)$ (i.e.~$\dg$ scales like $\Bsq$), then background is close to zero and all would be \cme; and if $R(\dg)=0$, then background and \cme\ contributions are of similar magnitudes.
Recently, our new method has been  applied to experimental data by the STAR collaboration, see Ref.~\cite{Zhao:2018en} for more details.

\section{Summary} 
To reduce background effects in \cme\ search, isobaric \Ru+\Ru\ and \Zr+\Zr\ collisions have been proposed where the $\vv$-induced backgrounds are expected to be similar while the \cme-induced signals to be different. In our  study, the proton and neutron density distributions of \Ru\ and \Zr\ are calculated using the energy density functional theory (\dft). They are then implemented in the {\em Monte Carlo} Glauber (\mcg) model to calculate the eccentricities ($\etwo$) and magnetic fields ($\Bbf$), and in A Multi-Phase Transport (\ampt) model to simulate the $\vv$. 
It is found that those nuclear densities, together with the Woods-Saxon (\ws) densities, yield wide ranges of differences in $\Bsq$ with respect to the participant plane (\PP) and the reaction plane (\RP). %On the other hand, those nuclear densities introduce, in contrast to \ws, a significant difference in $\etwo$ and $\vv$, as large as $\sim3$\%. %The situation becomes worse if the reaction plane (\RP) is used, a 10\% difference in $\etwo$ and $\vv$, and only a 10-15\% difference in $\Bsq$. 

We thus propose a novel method with comparative measurements of $\dg$ with respect to $\psiRP$ and $\psiPP$ in the same collision system. Our method is superior to isobaric collisions where large systematics persist. The novel method has been applied to experimental data by the STAR collaboration.
With improved statistics, the novel method we report here should be able to decisively answer the question of the CME in quantum chromodynamics.

This work was supported in part by the National Natural Science Foundation of China under Grants No.~11647306, 11747312, U1732138, 11505056, 11605054, and 11628508, and US~Department of Energy under Grant No.~DE-SC0012910.
%% The Appendices part is started with the command \appendix;
%% appendix sections are then done as normal sections
%% \appendix

%% \section{}
%% \label{}

%% References
%%
%% Following citation commands can be used in the body text:
%% Usage of \cite is as follows:
%%   \cite{key}         ==>>  [#]
%%   \cite[chap. 2]{key} ==>> [#, chap. 2]
%%

%% References with BibTeX database:

%\bibliographystyle{elsarticle-num}
%\bibliography{ref.bib}


\begin{thebibliography}{10}
\expandafter\ifx\csname url\endcsname\relax
  \def\url#1{\texttt{#1}}\fi
\expandafter\ifx\csname urlprefix\endcsname\relax\def\urlprefix{URL }\fi
\expandafter\ifx\csname href\endcsname\relax
  \def\href#1#2{#2} \def\path#1{#1}\fi

\bibitem{Kharzeev:1998kz}
D.~Kharzeev, R.~D. Pisarski, M.~H.~G. Tytgat, Phys. Rev. Lett. 81 (1998) 512--515.

\bibitem{Kharzeev:2015znc}
D.~E.~Kharzeev, J.~Liao, S.~A.~Voloshin and G.~Wang, Prog. Part. Nucl. Phys.  88 (2016) 1.

\bibitem{Zhao:2018ixy}
J.~Zhao, Int. J. Mod. Phys. A33~(13) (2018) 1830010.

\bibitem{Voloshin:2004vk}
S.~A. Voloshin, Phys. Rev. C70
  (2004) 057901.

\bibitem{Adamczyk:2014mzf}
L.~Adamczyk, et~al., Phys. Rev. Lett. 113 (2014)
  052302.

\bibitem{Wang:2009kd}
F.~Wang, Phys. Rev. C81 (2010) 064902.
S.~Pratt, S.~Schlichting and S.~Gavin, Phys. Rev. C84 (2011) 024909.
A.~Bzdak, V.~Koch and J.~Liao, Phys. Rev. C83 (2011) 014905.
F.~Wang and J.~Zhao, Phys. Rev. C95 (2017) 051901.

\bibitem{Voloshin:2010ut}
S.~A. Voloshin, Phys. Rev. Lett. 105 (2010) 172301.

\bibitem{Deng:2016knn}
W.-T. Deng, X.-G. Huang, G.-L. Ma, G.~Wang, Phys. Rev. C94 (2016) 041901.

\bibitem{Xu:2017zcn}
H.-j. Xu, X.~Wang, H.~Li, J.~Zhao, Z.-W. Lin, C.~Shen, F.~Wang, Phys.
  Rev. Lett. 121 (2018) 022301.

\bibitem{Xu:2017qfs}
H.-j. Xu, J.~Zhao, X.~Wang, H.~Li, Z.-W. Lin, C.~Shen, F.~Wang, Chin. Phys. C42 (2018) 084103.

\bibitem{Wang:2016rqh}
X.~B. Wang, J.~L. Friar, A.~C. Hayes, Phys. Rev. C94~(3) (2016) 034314.

\bibitem{Xu:2014ada}
H.-j. Xu, L.~Pang, Q.~Wang, Phys. Rev. C89~(6) (2014) 064902.

\bibitem{Deng:2012pc}
W.-T. Deng, X.-G. Huang, Phys. Rev. C85 (2012) 044907.

\bibitem{Lin:2004en}
Z.-W. Lin, C.~M. Ko, B.-A. Li, B.~Zhang, S.~Pal, Phys. Rev. C72 (2005) 064901.

\bibitem{Zhao:2018en}
J.~Zhao, {Quark Matter 2018 proceedings}, arXiv:1807.09925.

\end{thebibliography}


%% Authors are advised to use a BibTeX database file for their reference list.
%% The provided style file elsarticle-num.bst formats references in the required Procedia style

%% For references without a BibTeX database:

% \begin{thebibliography}{00}

%% \bibitem must have the following form:
%%   \bibitem{key}...
%%

% \bibitem{}

% \end{thebibliography}

\end{document}

%%
%% End of file `nupha-template.tex'.
