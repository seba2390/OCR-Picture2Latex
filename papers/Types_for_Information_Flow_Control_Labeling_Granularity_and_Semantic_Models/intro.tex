\section{Introduction}
\label{sec:intro}

Information flow control (IFC) is the problem of tracking flows of
information within a computer system and controlling or prohibiting
flows that contravene the policy in effect. In a language-based
setting, IFC requires tracking \emph{dependencies} between a program's
inputs, intermediate values and outputs. This can be done dynamically
with runtime monitoring~\cite{plas09-austinFlanagan,
  plas10-austinFlanagan} or statically using some form of abstraction
interpretation such as a type system. Our focus in this paper is the
second of these methods---IFC enforced through type systems.
%
In fact, literature has proposed several type systems for IFC, e.g.,
\cite{toplas03-flowcaml,jcs96-volpanoSmith,tosem00-jif,aplas13-paragon,popl98-SLAM,DBLP:journals/jcs/MatosB09,DBLP:conf/Boudol08,icfp15-HLIO,popl99-DCC}. All
these type systems have one aspect in common: \update{They all
  introduce \emph{security labels} or \emph{levels}, elements of a
  security lattice, that abstract program values. These labels are
  used to track dependencies between program values.}

\update{A significant design consideration for an IFC type system is
  the granularity (or extent) of the label abstraction, and the effect
  of this granularity on the expressiveness of the type system. By
  expressiveness here we mean the ability of a type system to type as
  many semantically secure programs as possible.\footnote{No sound
    type system can type all semantically secure programs, since
    freedom from bad flows (specifically, the standard information
    flow security property called noninterference~\cite{goguen82:ni})
    is undecidable.}  More specifically, we call a type system $T$
  more expressive than a type system $T'$ if there is a compositional,
  semantics-preserving transformation of programs typed under $T'$ to
  programs typed under $T$.\footnote{This notion of expressiveness is
    closely related to what Felleisen calls macro
    expressibility~\cite{DBLP:journals/scp/Felleisen91}.}}

\update{
The question of granularity has at least two aspects.
%
First, one may vary the granularity of the \emph{labels}
themselves. For example, a fine-grained label on a value may precisely
specify the program variables or inputs that the value depends on. On
the other hand, a coarse-grained label may specify only an upper-bound
on the confidentiality level (e.g., ``secret'', ``top-secret'', etc.)
of all inputs on which the labeled value depends. The effect of
varying this notion of granularity (of the labels) on the
expressiveness of the type system has been studied in prior
work~\cite{DBLP:conf/popl/HuntS06}.}

\update{A different kind of granularity, whose expressiveness is the
  focus of this paper, is the granularity of \emph{labeling} (not
  \emph{labels}).%
  %
  \footnote{In the rest of the paper, granularity refers to the
    granularity of labeling, not the granularity of labels.}
  %
  Here,} a \emph{fine-grained type
  system} is one that labels every program value individually. For
instance, Flow Caml, a type system for IFC on
ML~\cite{toplas03-flowcaml}, adds a label on every type constructor
and, hence, on every value, top-level and nested. As an example, the
type $(A^H \times B^L)^L$ might ascribe low (public) pairs, whose
first projection is high (private) and whose second projection is
low. ($H$ and $L$ are standard labels for high and low confidentiality
data, respectively.) Since fine-grained type systems label individual
values, they also track dependencies at the granularity of individual
values. For example, combining a high value with a low value using a
primitive operator in the language results in a high value. Many other
type systems are similarly
fine-grained~\cite{jcs96-volpanoSmith,tosem00-jif,aplas13-paragon,popl98-SLAM}.

In contrast, a \emph{coarse-grained type system} labels an entire
sub-computation using a single label. All values produced within the
scope of the sub-computation implicitly have that label. Hence, it not
necessary to label individual values.
%
As an example, the SLIO and HLIO systems~\cite{icfp15-HLIO} introduce
a monad for heap I/O (similar to Haskell's IO monad), but refine the
monadic type to include two labels, $\llabel_i$ and $\llabel_o$, as in
$(SLIO~\llabel_i~\llabel_o~\tau)$. This type represents stateful
computations of type $\tau$ that start from the confidentiality label
$\llabel_i$ and end with the confidentiality label
$\llabel_o$. $\llabel_i$ is an upper-bound on the confidentiality of
all prior computations that the current computation depends on;
accordingly, the current computation can have write effects at levels
above $\llabel_i$ only. $\llabel_o$ is an upper-bound on the
confidentiality of the current computation; accordingly, the current
computation can have read effects only at levels below
$\llabel_o$. Importantly, there is no need to label individual values
or nested types. Instead, every value produced by the current
computation implicitly inherits the label $\llabel_o$, and labels are
tracked via monadic sequencing (bind) at the granularity of
computations. Other type
systems~\cite{DBLP:journals/jcs/MatosB09,DBLP:conf/Boudol08,popl99-DCC}
are also coarse-grained,
although~\cite{DBLP:journals/jcs/MatosB09,DBLP:conf/Boudol08} do not
use monads to confine effects.

Given these vastly contrasting labeling granularities for the same
end-goal---information flow control, a natural question is one of
their relative expressiveness~\cite{siglog17-ifcComp}.
%% By expressiveness here we mean the ability of a type system to type
%% as many semantically secure programs as possible.\footnote{No sound
%% type system can type all semantically secure programs, since
%% freedom from bad flows (specifically, the standard security
%% property called noninterference) is undecidable.}
In general, it seems that fine-grained type systems should be at least
as expressive as coarse-grained type systems, since the former track
flows at finer granularity \update{(individual values as opposed to
  entire sub-computations)} and, hence, should abstract flows less
than the latter. In the other direction, the situation is less
clear. Upfront, it seems that coarse-grained type systems \emph{may}
be less expressive than fine-grained type systems, but then one
wonders whether by structuring programs as extremely small
computations in a coarse-grained type system, one may recover the
expressiveness of a fine-grained type system.

In this paper, we show constructively that both these intuitions are,
in fact, correct. We do this using specific instances of the two kinds
of type systems in the setting of a higher-order language with state
(similar to ML). For the fine-grained type system we use a system very
close to SLam~\cite{popl98-SLAM} and the exception-free fragment of
Flow Caml~\cite{toplas03-flowcaml}. For the coarse-grained type system
we use a variant of the static fragment of
HLIO~\cite{icfp15-HLIO}. This calculus has a specific construct to
limit the scope of a computation's label in a safe way. We then show
that well-typed programs in each type system can be translated to the
other, preserving typability and meaning. This establishes that the
type systems are equally expressive.

We believe this settles an open question about the relative
expressiveness of setting up IFC type systems \update{with different
  labeling granularities}. Our result also has an immediate practical
consequence: Since coarse-grained IFC type systems usually burden a
programmer less with annotations (since not every value has to be
labeled) and we have shown now that they are as expressive as
fine-grained IFC type systems, there seems to be some merit to
preferring coarse-grained IFC type systems over fine-grained ones in
general.

As a second contribution of independent interest, we show how to set
up semantic, logical relations models of IFC types in both the
fine-grained and the coarse-grained settings, over calculi with
higher-order state. While models of IFC types have been considered
before~\cite{popl98-SLAM,popl99-DCC,esop99-PER-IFC,DBLP:conf/csfw/MantelSS11},
we do not know of any development that covers higher-order state.  In
fact, models of types in the presence of higher-order state are
notoriously difficult. Here, we have the added complication of
information flow labels. Fortunately, enough development has occurred
in the programming languages community in the past decade to give us a
good starting point. Specifically, our models are based on
step-indexed Kripke logical
relations~\cite{DBLP:conf/popl/AhmedDR09}. Like earlier work, our
models are relational, i.e., they relate two runs of a program to each
other. This is essential since we are interested in proving
noninterference~\cite{goguen82:ni}, the standard security property
which says that public outputs of a program are not influenced by
private inputs (i.e., there are no bad flows). This property is
naturally defined using two runs. Using our models, we derive proofs
of the soundness of both the fine-grained and the coarse-grained type
systems.

%% Prior proofs of the soundness of similar systems are either fully
%% syntactic (based directly on the operational semantics) and far
%% more complex~\cite{toplas03-flowcaml,icfp15-HLIO} or ignore the
%% stateful part~\cite{popl98-SLAM}.

We also use our logical relations to show that our translations are
meaningful. Specifically, we set up \emph{cross-language logical
  relations} to prove that our translations preserve program
semantics, and from this, we derive a crucial result for each
translation: Using the noninterference theorem of the target language
as a lemma, we are able to re-prove the noninterference theorem for
the source language directly. These results imply that our
translations preserve label annotations
meaningfully~\cite{DBLP:journals/cl/BartheRB07}. Like all logical
relations models, we expect that our models can be used for other
purposes as well.

To summarize, the two contributions of this work are:
\begin{itemize}
\item Typability- and meaning-preserving translations between a
  fine-grained and a coarse-grained IFC type system, showing that
  these type systems are equally expressive.
\item Logical relations models of both type systems, covering both
  higher-order functions and higher-order state.
\end{itemize}


Due to lack of space, many technical details and proofs are omitted
from this paper. These are provided in an appendix available online
from the authors' homepages.


\medskip
\noindent \textbf{Note.} Readers interested only in our translations
but not the details of our semantic models can skip sections
pertaining to the latter (e.g., Section~\ref{sec:fg-sem}). This will
not affect the readability of the rest of the paper.


% \medskip
% \noindent \textbf{Note for CSF reviewers.}
% %
% Some preliminary ideas on which this paper is based were presented in
% an earlier non-refereed magazine article authored by
% us~\cite{siglog17-ifcComp}. We cannot cite this article in the clear
% due to the light double-blind review, but we have provided it to the
% CSF PC Chairs. That article was based on a different version of the
% coarse-grained type system (standard HLIO, as opposed to our modified
% HLIO here). Hence, all the work here is technically
% different. Nonetheless, the translation from the coarse-grained to the
% fine-grained type system in this paper is similar to the one in the
% earlier article. Everything else in this paper is completely new,
% including the translation from the fine-grained to the coarse-grained
% type system (the earlier article attempted a very different,
% incomplete translation), the logical relations, and the statements and
% proofs of semantic correctness.


\if 0
Information Flow Control (IFC) is a form of dependence analysis which aims at prohibiting secret
(high) inputs to flow into public (low) outputs of the program. The secrets can flow into public
outputs either directly via explicit assignments, file outputs, etc. or they can flow indirectly via
the choice of conditional branches and various subtle side channels. A sound IFC analysis must
prohibit leaks via all channels that can be observed by an attacker. Such analysis can be performed,
for instance, either dynamically ~\cite{plas09-austinFlanagan, plas10-austinFlanagan} by monitoring
the program's execution or statically~\cite{toplas03-flowcaml,popl98-SLAM} by analyzing its source
code. The soundness (no covert flows have been missed) of such dependence analysis is often stated
as a variant of a well-known criterion called \NI, which basically states that a secure program must
always produce public outputs which remain unaffected by the choice of its secret inputs.

Irrespective of the stage at which IFC is performed (statically or dynamically), IFC analysis (in
general) requires a) association of secrecy label (like confidential, top secret, public) with the
sources (inputs) and sinks (outputs) of the program and b) tracking these labels through the
program's execution to make sure that the public outputs remain independent of the secret
inputs. While labeling of sources and sinks (requirement a) remains invariant across the different
kinds of IFC analysis, tracking of these labels (requirement b) can vary based on the level of
granularity of dependencies being tracked. A fine-grained approach tracks label on every individual
value of the program while a coarse-grained approach tracks label on the entire process (or
context). This leads to a trade-off between the two. On one hand, coarse-grained approaches are
observed to be much easier (as also more efficient) to implement than fine-grained approaches
because the complexity (and the overhead) of tracking just the context label is far less than the
complexity (and overhead) of tracking labels on all individual values. But on the other hand,
coarse-grained approaches often suffer from an \textit{over-tainting} (also referred to as the
\textit{label creep}) problem -- an unnecessarily high context label which leads to rejection of
several benign programs, while the fine-grained approaches don't. Consequently, it is believed that
fine-grained approaches are more expressive (accept more secure programs) than coarse-grained
approaches. For example, a program that reads a secret file but always outputs a fixed constant will
be rejected by a coarse-grained approach (despite being completely secure). But such a benign
program will be accepted by a fine-grained approach because the constant will be given a public
label and thus sending it on a public output is, rightfully, not considered a leak. This trade-off
has lead to an interest in the development of separate enforcement methods (both static and dynamic)
based on fine-grained and coarse-grained tracking. In this paper, we will only focus on the static
methods (type systems in particular) for both granularities of tracking.

IFC using fine-grained tracking was historically envisioned and incorporated by Denning into a form
of static analysis for program certification~\cite{purdue75-denning}. Volpano, Irvine and Smith's
type system~\cite{jcs96-volpanoSmith} was the first satisfactory embodiment of Denning's analysis
into a type system for a while language. Later Pottier and Simonet extended this approach to a type
system~\cite{toplas03-flowcaml} for a higher-order, ML like, language. % Similar fine-grained
% tracking has also been used to provide static enforcement of DLM (Decentralized Label
% Model)~\cite{sosp97-dlm} polices in the JIF~\cite{tosem00-jif}
% language. Paragon~\cite{aplas13-paragon} is another type-based enforcement for a policy model
% known as Paralocks~\cite{popl10-paralocks}, which
Various other type systems for fine-grained IFC (like JIF~\cite{tosem00-jif},
Paragon~\cite{aplas13-paragon}) have been studied in the past, we refer
to~\cite{jsac06-surveySabelfeld} for a comprehensive review. Despite the versatility in the
landscape of fine-grained IFC type systems, a common denominator amongst all of them is the presence
of security labels on \textit{all} types (unlike coarse-grained type systems which only add labels
to some of the types, and use the context label to infer the rest). A label on a type represents an
upper-bound on the confidentiality of the secrets that have influenced the computation of an
expression of this type. The type rules of the language track and use these labels to control the
flow of information through the program, making sure that no secret information flows to public
outputs. These type systems also use a concept of the context label (like coarse-grained type
systems) but are often more precise in dealing with false leaks, mainly because context label
doesn't have to be an implicit label on all the values computed in that context (thanks to
fine-grained labeling). % Consequently, it is believed that \textit{label creep} is a less of a
% problem in fine-grained type systems as compared to their coarse-grained counter parts.

IFC using coarse-grained tracking, on the other hand, was historically used mainly at the operating
system level~\cite{sosp05-asbestosOS,osdi06-histarOS,sosp07-flumeOS}. However, of-late we are seeing
language-based methods also making use of coarse-grained tracking. For instance, IFC using a labeled
IO (LIO for short) monad~\cite{haskell11-LIO,icfp15-HLIO} has been used in the context of Haskell
programming language to enforce IFC using the coarse-grained approach. A purely dynamic
enforcement~\cite{haskell11-LIO} of this concept uses the IO monad for handling input and output
effects (as usual) but additionally it also tracks the context label (referred to as the current
label) to prevent information leaks via such effects. And a hybrid (static + dynamic)
enforcement~\cite{icfp15-HLIO} of this concept uses monadic types to achieve the same statically
whenever possible and falls back to the dynamic version~\cite{haskell11-LIO} otherwise. The context
label of an LIO computation (doesn't matter how it is tracked - dynamically or statically) is an
upper-bound on the level of secrets it has read and the monadic $\ebind$ makes sure that the
continuation is always executed with a context label that is at least equally high. Along with that,
LIO has a construct ($\elabel {\llabel}$) to add discretionary labels (labels are not always
required to be added by the user, a term inherits the label from its context) to terms, and a dual
construct ($\eunlabel$) to remove label from such labeled terms, by pushing it into the context
(this is in sync with the coarse-grained tracking philosophy -- reading a secret should make the
context secret too). All these constructs described so far implements the kind of coarse-grained
tracking which suffers from an \textit{over-tainting} problem. However, LIO also has a scoping
construct ($\etolabeled$) which can be used to execute a computation without raising its context
label, by instead pushing that extra label into the value returned by the computation, and thus can
potentially avoid the \textit{label creep}. For example, without the use of $\etolabeled$, a
computation that adds two labeled values would raise the context label by the label on the
individual values being added (as the computation has to unlabel them before the addition can be
performed). But with the use of $\etolabeled$, we would get the same context label before and after
the computation, and the returned result would get the extra label (the join of the individual
labels on the values being added) instead.

This modern view of coarse-grained tracking with the $\etolabeled$ construct raises a question about
the relative expressiveness of the two type systems -- does the addition of $\etolabeled$ reconcile
the differences in expressivity of the two granularities ? In other words, can this modern view of
coarse-grained tracking scale up to the expressiveness of the fine-grained tracking with diligent
use of $\etolabeled$ ? This question has been investigated partially in~\cite{siglog17-ifcComp} by
studying the relative expressiveness of {\fg} (a variant of FlowCaml~\cite{toplas03-flowcaml}, as a
representative of fine-grained IFC enforcement) and {\cg} (a static variant of
HLIO~\cite{icfp15-HLIO}, as a representative of coarse-grained IFC enforcement). It was shown that
the whole of {\cg} can be embedded into {\fg} but only a subset of {\fg} (referred to as {\rfg}) can
be embedded in {\cg}. This was shown by presenting type-preserving translations from one to the
other. The embedding from {\rfg} to {\cg} makes clever use of label polymorphism and constraint
types to achieve type-preservation. And it leaves a translation from the whole of {\fg} to {\cg} as
an open problem. We believe that~\cite{siglog17-ifcComp} is an important step towards relating the
expressiveness of {\fg} and {\cg} but has the following gaps a) the non-standard addition of label
polymorphism and constraint types to both FlowCaml~\cite{toplas03-flowcaml} and
HLIO~\cite{icfp15-HLIO} raises concerns about the semantic foundations of the two type systems
especially when the proof theories of both FlowCaml~\cite{toplas03-flowcaml} and
HLIO~\cite{icfp15-HLIO} (on which {\fg} and {\cg} type systems are based) haven't been proven sound
and b) type-preservation should be seen as a necessary but not a sufficient step when it comes to
embedding {\fg} to {\cg} and vice versa. We must also show that the embeddings (which ever exist)
preserve semantics and security of the source program.

In this paper we extend the efforts of \cite{siglog17-ifcComp} by coming up with a type-preserving
translation from the whole of {\fg} to {\cg}. We fill the gaps highlighted in the previous paragraph
by a) giving semantic models to both the type systems (for {\fg} and {\cg}). Our models use a
standard technique of Kripke step-indexed logical relations~\cite{esop06-SILR}, and are based on
basic set-theory and higher-order logic. We use these models to give foundational
proofs~\cite{popl00-semanticModelPCC} of {\NI} for both {\fg} and {\cg}; and b) by building
cross-language models (again based on Kripke step-indexed logical relations) for both the
translations ({\cg} to {\fg} and {\fg} to {\cg}) to show the semantics and security preservation of
the source.

\noindent{\textbf{Outline:}}

In Section~\ref{sec:ts} we begin by describing the type systems we work with (both {\fg} and
{\cg}). Before jumping to the translation, we first prove the soundness of the proof-theory for both
{\fg} and {\cg} in Section~\ref{sec:sm-fg-cg}. In Section~\ref{sec:trans} we first quickly recap the
translation from {\rfg} to {\cg} from \cite{siglog17-ifcComp} which uses label polymorphism and
constraint types, then we describe a translation from {\fg} to {\rfg} as a proof of the existence of
the translation from the whole of {\fg} to {\cg}. This turns out to very insightful and the insights
learned from {\fg} to {\rfg} help us in obtaining a direct translation from the whole of {\fg} to
{\cg}. We describe the soundness of this translation by showing semantics and security preservation
of the source. In Section~\ref{sec:related} we describe some related work. And finally in
Section~\ref{sec:conclusion} we conclude and describe some future directions.
\fi

%%% Local Variables:
%%% mode: latex
%%% TeX-master: "main"
%%% End:
