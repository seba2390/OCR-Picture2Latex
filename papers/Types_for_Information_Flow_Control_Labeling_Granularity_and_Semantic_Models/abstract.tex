\begin{abstract}
Language-based information flow control (IFC) tracks dependencies
within a program using sensitivity labels and prohibits public outputs
from depending on secret inputs. In particular, literature has
proposed several type systems for tracking these dependencies. On one
extreme, there are fine-grained type systems (like Flow Caml) that
label all values individually and track dependence at the level of
individual values. On the other extreme are coarse-grained type
systems (like HLIO) that track dependence coarsely, by associating a
single label with an entire computation context and not labeling all
values individually.

In this paper, we show that, despite their glaring differences, both
these styles are, in fact, equally expressive. To do this, we show a
semantics- and type-preserving translation from a coarse-grained type
system to a fine-grained one and vice-versa. The forward translation
isn't surprising, but the backward translation is: It requires a
construct to arbitrarily limit the scope of a context label in the
coarse-grained type system (e.g., HLIO's ``toLabeled'' construct). As
a separate contribution, we show how to extend work on logical
relation models of IFC types to higher-order state. We build such
logical relations for both the fine-grained type system and the
coarse-grained type system. We use these relations to prove the two
type systems and our translations between them sound.

  %% Information Flow Control (IFC) is a form of dependence analysis that tracks and prohibits
  %% dependence of secret inputs on public outputs. Such dependence analysis is often carried out at
  %% the level of type systems. These IFC type systems can track dependence (via confidentiality
  %% labels) at varying levels of granularity. On one extreme, there are fine-grained type systems
  %% (like FlowCaml) that track dependence at the level of individual values by using labels on their
  %% types. As a result all types in the language are annotated with a label. On the other extreme,
  %% there are coarse-grained type systems (like HLIO) that only track dependence at the level of
  %% entire computation (computation's labels is a lower bound on the level of all the secrets it has
  %% read), represented at the type level using a monad. Since, tracking of labels only happen inside a
  %% monad, there is no need to attach labels on all types.

  %% There has been a growing confusion around the relative expressiveness of these two classes of type
  %% systems. Prior work has made partial progress on this problem by showing a type-preserving
  %% translation from a variant of HLIO to a variant of FlowCaml but not vice versa. In this paper we
  %% come up with a type-preserving translation in the other direction too, prove that both the
  %% translations preserve the semantics and the security of the source program. This resolves the
  %% confusion around their relative expressiveness and shows that the two are equivalent.
\end{abstract}

%%% Local Variables:
%%% mode: latex
%%% TeX-master: "main"
%%% End:
