\section{Conclusion}
\label{sec:conclusion}

\update{This paper has examined the question of whether information
  flow type systems that label at fine granularity and those that
  label at coarse granularity are equally expressive.} We answer this
question in the affirmative, assuming that the coarse-grained type
system has a construct to limit the scope of the taint label. A more
foundational contribution of our work is a better understanding of
semantic models of information flow types. To this end, we have
presented logical relations models of types in both the fine-grained
and the coarse-grained settings, for calculi with mutable higher-order
state.

\medskip \noindent \textbf{\update{Acknowledgments.}}  \update{This
  work was partly supported by the German Science Foundation (DFG)
  through the project ``Information Flow Control for Browser Clients
  -- IFC4BC'' in the priority program ``Reliably Secure Software
  Systems -- RS$^3$'', and also through the Collaborative Research
  Center ``Methods and Tools for Understanding and Controlling
  Privacy'' (SFB 1223). We thank our anonymous reviewers and anonymous
  shepherd for their helpful feedback.}

%% In this paper we unified the two granularities of information flow tracking namely fine-grained and
%% coarse-grained. We extended a prior work \cite{siglog17-ifcComp} on this problem by coming up with a
%% type-preserving translation from {\fg} (a variant of FlowCaml) to {\cg} (a variant of HLIO), the
%% missing translation from \cite{siglog17-ifcComp}. We showed semantic soundness of the two type
%% systems involved and also proved that the type-preserving translations between them (in both
%% directions) preserve the program semantics and security of the source program. We are planning to
%% scale up this translation to other forms of dependence analysis like provenance tracking.

%%% Local Variables:
%%% mode: latex
%%% TeX-master: "main"
%%% End:
