\subsection{Translating {\cg} to {\fg}}
\label{sec:cg2fg}

\begin{figure*}[!htbp]
\begin{mathpar}
  \inferrule
    {
       \Gamma \vdash e : \tau \leadsto e_F
    }
    {
       \Gamma \vdash \elabel{\llabel}(e)
      : \tlabeled{\llabel}{\tau}
      \leadsto
      \einl(e_F)
    }\rname{label}

    \inferrule
    {
      \Gamma \vdash e : \tlabeled{\llabel}{\tau} \leadsto e_F
    }
    {
      \Gamma \vdash \eunlabel(e) : \tslio{\top}{\llabel}{\tau}
      \leadsto
      \lambda \_. e_F
    }\rname{unlabel}

    \inferrule
    {
      \Gamma \vdash e : \tslio{\llabel_1}{\llabel_2}{\tau} \leadsto e_F
    }
    {
      \Gamma \vdash \etolabeled(e)
      : \tslio{\llabel_1}{\bot}{(\tlabeled{\llabel_2}{\tau})}
      \leadsto
      \lambda \_. \einl(e_F ~ ())
    }\rname{toLabeled}

    \inferrule
    {
      \Gamma \vdash e : \tau \leadsto e_F
    }
    {
      \Gamma \vdash \eret(e) : \tslio{\top}{\bot}{\tau}
      \leadsto \lambda \_. \einl(e_F)
    }\rname{ret}

    \inferrule
    {
      \Gamma \vdash e_1 : \tslio{\llabel_1}{\llabel_2}{\tau} \leadsto e_{F1}
      \\ \Gamma, x : \tau \vdash e_2 : \tslio{\llabel_3}{\llabel_4}{\tau'} \leadsto e_{F2}
     \\ \llabel \lbelow \llabel_1
      \\ \llabel \lbelow \llabel_3
      \\ \llabel_2 \lbelow \llabel_3
      \\ \llabel_2 \lbelow \llabel_4
     % \\ \llabel_4 \lbelow \llabel_4
    }
    {
      \Gamma \vdash \ebind(e_1, x.e_2) : \tslio{\llabel}{\llabel_4}{\tau'}
      \leadsto \lambda \_. \ecase(e_{F1} (), x. e_{F2} (), y.\einr())
    }\rname{bind}
  \end{mathpar}
    \caption{Expression translation {\cg} to {\fg} (selected rules only)}
  \label{fig:cg-to-fg-term}
\end{figure*}


This section describes the translation in the other direction---from
{\cg} to {\fg}. The overall structure (but not the details!) of this
translation are similar to that of the earlier {\fg} to {\cg}
translation, so we skip some boilerplate material here. The
superscript or subscript $s$ (source) now marks elements of {\cg} and
$t$ (target) marks elements of {\fg}.

The key idea of the translation is to map a source ({\cg}) expression
$e_s$ satisfying $\vdash e_s: \tau$ to a target ({\fg}) expression
$e_t$ satisfying $\vdash_\top e_t: \trans{\tau}$. The type translation
$\trans{\tau}$ is defined below. The $\pc$ for the translated
expression is $\top$ because, in {\cg}, all effects are confined to a
monad, so at the top-level, there are no effects. In particular, there
are no write effects, so we can pick any $\pc$; we pick the most
informative $\pc$, $\top$.

The type translation, $\trans{\tau}$, is defined by induction on
$\tau$.
\[
    \begin{array}{l@{~}c@{~}l}
      \trans{\tbase} 						& = & \tbase^\bot \\
      \trans{\tau_1 \to \tau_2} 			& = &
                                                              (\trans{\tau_1} \overset{\top}{\to} \trans{\tau_2})^\bot \\
      \trans{\tau_1 \times \tau_2} 		& = & (\trans{\tau_1} \times \trans{\tau_2})^\bot \\
      \trans{\tau_1 + \tau_2} 			& = & (\trans{\tau_1} + \trans{\tau_2})^\bot \\
  \trans{\tref~ \llabel~ \tau} 		& = & (\tref~ (\trans{\tau} + \tunit)^\llabel)^\bot \\
  \trans{\tslio{\llabel_1}{\llabel_2}{\tau}}
                                  & = &
                                        (\tunit \overset{\llabel_1}{\to} (\trans{\tau} + \tunit)^{\llabel_2})^\bot\\
  \trans{\tlabeled{\llabel}{\tau}} & = & (\trans{\tau} + \tunit)^\llabel
    \end{array}
\]

The most interesting case of the translation is that for
$\tslio{\llabel_1}{\llabel_2}{\tau}$. Since a {\cg} value of this type
is a suspended computation, we map this type to a \emph{thunk}---a
suspended computation implemented as a function whose argument has
type $\tunit$. The pc-label on the function matches the pc-label
$\llabel_1$ of the source type. The taint label $\llabel_2$ is placed
on the output type $\trans{\tau}$ using a coding trick: $(\trans{\tau}
+ \tunit)^{\llabel_2}$. The expression translation of monadic
expressions only ever produces values labeled $\einl$, so the right
type of the sum, $\tunit$, is never reached during the execution of a
translated expression. The same coding trick is used to translate
labeled and ref types. We could also have used a different coding in
place of $(\trans{\tau} + \tunit)^{\llabel_2}$. For example,
$(\trans{\tau} \times \tunit)^{\llabel_2}$ works equally well.

The expression translation is directed by source typing derivations
and is defined by the judgment $\Gamma \vdash e_s: \tau \leadsto e_t$,
some of whose rules are shown in Figure~\ref{fig:cg-to-fg-term}. The
translation is fairly straightforward (given the type
translation). The only noteworthy aspect is the use of the injection
$\einl$ wherever an expression of the type form $(\trans{\tau} +
\tunit)^{\llabel}$ needs to be constructed.

\medskip \noindent \textbf{Properties.}  The translation preserves
typing by construction, as formalized in the following theorem. The
context translation $\trans{\Gamma}$ is defined pointwise on all types
in $\Gamma$.
%
\begin{thm}[Typing preservation]\label{thm:cg-2-fg-typePres}
  If $\Gamma \vdash e_s: \tau$ in {\cg}, then there is a unique $e_t$
  such that $\Gamma \vdash e_s: \tau \leadsto e_t$ and that $e_t$
  satisfies $\trans{\Gamma} \vdash_\top e_t: \trans{\tau}$ in {\fg}.
\end{thm}
Again, a corollary of this theorem is that well-typed source programs
translate to noninterfering target programs.

We further prove that the translation preserves the semantics of
programs. Our approach is the same as that for the {\fg} to {\cg}
translation---we set up a cross-language logical relation, this time
indexed by {\cg} types, and show the fundamental theorem. From this,
we derive that the translation preserves the meanings of
programs. Additionally, we derive the noninterference theorem for
{\cg} using the binary fundamental theorem of {\fg}, thus gaining
confidence that our translation maps security labels properly. Since
this development mirrors that for our earlier translation, we defer
the details to the appendix.

%% In this section we describe a type-directed
%% translation from {\cg} to {\fg}. Fig.~\ref{fig:typeTransCg2Fg}
%% describe the translation function from {\cg} types to {\fg} types. We
%% describe some key elements of the type translation here.

%% \medskip

%% Not all types in {\cg} have labels attached to them. As a result, the plain types (the ones without
%% a label attached) can be mapped to a corresponding {\fg} type with a $\bot$ label. This can be seen
%% in the translation of, for instance, the base type ($\tbase$). When the {\cg} type does carry a
%% label i.e have the form $\tlabeled {\llabel} {\tau}$ then one possible translation to a
%% corresponding {\fg} type would be ${\utype}^{\llabel' \ljoin \llabel}$ where $\utype^{\llabel'}$ is
%% the translation of $\tau$. However, by flattening (joining) the label in this fashion we won't be
%% able to simulate an unlabel of the outer label ($\llabel$) on the {\fg} side. This is because, there
%% is no way to separate the label $\llabel$ from the label $\llabel' \ljoin \llabel$ once it has been
%% joined. So, instead we choose to translate $\tlabeled {\llabel} {\tau}$ to
%% $(\trans{\tau} + \tunit)^\llabel$ thereby keeping the label in $\tau$ separate from $\llabel$, this
%% would permit us to simulate the selective unlabeling of the outer label. And the corresponding term
%% translation always uses $\einl$ (as described later). Coming to the case of a monadic type (i.e type
%% for effectful computation), $\tslio {\llabel_i} {\llabel_o} {\tau}$. The monadic type represents a
%% suspended computations in {\cg} which we translate a thunk in {\fg}. At the type level, we translate
%% $\tslio {\llabel_i} {\llabel_o} {\tau}$ into
%% $(\tunit \overset{\llabel_i}\to (\tau + \tunit)^{\llabel_o})^\bot$. The effect label (on the arrow)
%% is chosen to be $\llabel_i$ as {\cg} also treats $\llabel_i$ as a lower-bound on the write effects
%% that the term of monadic type can perform when forced. And the outer label of the return type is
%% chosen as $\llabel_o$ in accordance with {\cg}'s interpretation of $\llabel_o$ as an upper-bound on
%% the read effects of the computation (when forced).

%% \begin{figure*}[!htbp]
%% \begin{minipage}{0.45\textwidth}
%% \[
%%     \begin{array}{r@{~}c@{~}l}
%%       \trans{\tbase} 						& = & \tbase^\bot \\
%%       \trans{\tau_1 \to \tau_2} 			& = &
%%                                                               (\trans{\tau_1} \overset{\top}{\to} \trans{\tau_2})^\bot \\
%%       \trans{\tau_1 \times \tau_2} 		& = & (\trans{\tau_1} \times \trans{\tau_2})^\bot \\
%%       \trans{\tau_1 + \tau_2} 			& = & (\trans{\tau_1} + \trans{\tau_2})^\bot \\
%%     \end{array}
%% \]
%% \end{minipage}
%% %
%% \begin{minipage}{0.45\textwidth}
%% \[
%% \begin{array}{r@{~}c@{~}l}
%%   \trans{\tref~ \llabel~ \tau} 		& = & (\tref~ (\trans{\tau} + \tunit)^\llabel)^\bot \\
%%   \trans{\tslio{\llabel_1}{\llabel_2}{\tau}}
%%                                   & = &
%%                                         (\tunit \overset{\llabel_1}{\to} (\trans{\tau} + \tunit)^{\llabel_2})^\bot\\
%%   \trans{\tlabeled{\llabel}{\tau}} & = & (\trans{\tau} + \tunit)^\llabel
%% \end{array}
%% \]
%% \end{minipage}
%% \caption{Type translation from {\cg} to {\fg}}
%% \label{fig:typeTransCg2Fg}
%% \end{figure*}

%% \medskip The translation $\trans{\cdot}$ is lifted pointwise to contexts:
%% $\trans{\Gamma} \triangleq \{ x: \trans{\tau} \mid x: \tau \in \Gamma \}$. The translation of terms
%% is directed by the translation of the types which is denoted by the judgment
%% $\Gamma \vdash e : \tau \leadsto e'$. Selected rules of the translation are
%% shown in Fig.~\ref{fig:cg-to-fg-term}. They should be unsurprising given the type translation.


%% The following theorem shows that this translation preserves types, in the sense that $\leadsto$
%% always maps a valid {\cg} typing derivation to a valid {\fg} typing derivation.

%% \begin{thm}[Type soundness, {\cg} $\leadsto$ {\fg}]
%%   If $\Gamma \vdash e : \tau$ has a valid {\cg} typing derivation, then there
%%   exists an $e'$ such that $ \Gamma \vdash e : \tau \leadsto e'$ and
%%   $\trans{\Gamma} \vdash_\top e' : \trans{\tau}$ has a valid {\fg} typing
%%   derivation.
%% \end{thm}

%% \subsection{Soundness of {\cg} to {\fg}}
%% \label{sec:cg2fg-soundness}

%% \begin{figure*}[!htbp]
%%   \centering
%%   \begin{displaymath}
%%     \begin{array}{lll}
%%       \cfUlrv {\tbase} & \triangleq & \{(\uWorlds, m,  \sval, \tval ) ~|~ \sval \in \llbracket \tbase \rrbracket \wedge
%%                                       \tval \in \llbracket \tbase \rrbracket \wedge \sval = \tval
%%                                       \} \\

%%       \cfUlrv {\tunit} & \triangleq & \{(\uWorlds, m,  \sval, \tval) ~|~ \sval \in \llbracket \tunit \rrbracket
%%                                       \wedge \tval \in \llbracket \tunit \rrbracket \} \\

%%       \cfUlrv {\tau_1 \times \tau_2} & \triangleq & \{(\uWorlds, m,  (\sval_1, \sval_2), (\tval_1, \tval_2)) ~|~ \\
%%                      & & (\uWorlds,  m,  \sval_1, \tval_1) \in \cfUlrv {\tau_1} \wedge
%%                          (\uWorlds,  m,  \sval_2, \tval_2) \in \cfUlrv {\tau_2}
%%                          \}\\

%%       \cfUlrv {\tau_1 + \tau_2} & \triangleq & \{(\uWorlds,  m,  \einl ~  \sval, \einl ~  \tval) ~|~
%%                                                (\uWorlds,  m,  \sval, \tval) \in \cfUlrv {\tau_1} \} ~ \union \\
%%                      & & \{(\uWorlds,  m,  \einr ~  \sval, \einr ~  \tval) ~|~
%%                          (\uWorlds,  m,  \sval, \tval) \in \cfUlrv {\tau_2} \} \\

%%       \cfUlrv {\tau_1 \to \tau_2} & \triangleq & \{(\uWorlds,  m,  \lambda x. e_s, \lambda x. e_t) ~|~
%%                                                  \forall \uWorlds' \wExtendsR \uWorlds,  \sval, \tval, j<m, \pb \wExtends \pb'.
%%                                                  (\uWorlds', j, \sval, \tval) \in \cflrvu {\tau_1} {\pb'}  \\
%%                      & & \implies (\uWorlds', j, e_s[\sval/x], e_t[\tval/x]) \in \cflreu {\tau_2} {\pb'}  \} \\

%%       \cfUlrv {\tref ~{\llabel}~ \tau} & \triangleq & \{(\uWorlds, m, {^s}\loc, {^t}\loc) ~|~
%%                                                       \uWorlds ({^s}\loc) = \tlabeled  {\llabel} {\tau} \wedge
%%                                                       ({^s}\loc, {^t}\loc) \in \pb    \} \\

%%       \cfUlrv {\tlabeled {\llabel} {\tau}} & \triangleq & \{(\uWorlds,m,  \sval, \tval) ~|~ \\
%%                      & & \exists \sval', \tval'. \sval = \elabel {} (\sval') \wedge \tval = \einl ~ \tval' \wedge
%%                          (\uWorlds, m, \sval', \tval') \in \cfUlrv {\tau} \}\\

%%       \cfUlrv {\tslio {\llabel_1} {\llabel_2} {\tau}} & \triangleq & \{(\uWorlds, m, \sval, \tval) ~|~
%%                                                                      \forall \uWorlds_e \wExtendsR \uWorlds, \heap_s, \heap_t, i,
%%                                                                      \sval', k \leq  m, \pb \wExtends \pb'. \\
%%                      & & (k,\heap_s,\heap_t) \wDefined{\pb'} (\uWorlds_e) \wedge
%%                          (\heap_s, \sval) \ReducesToF{i} (\heap_s', \sval') \wedge i<k \implies \\
%%                      &  &  \exists \heap_t', \tval'. (\heap_t, \tval ()) \ReducesTo{} (\heap_t', \tval')  \wedge
%%                           \exists  \uWorlds' \wExtendsR \uWorlds_e, \pb' \wExtends \pb''. (k-i,\heap_s', \heap_t') \wDefined{\pb''}
%%                           \uWorlds' \wedge \\
%%                      & & \exists \tval''. \tval' = \einl ~ \tval'' \wedge (\uWorlds', k-i, \sval', \tval'') \in \cflrvu {\tau} {\pb''}
%%                          \}
%%   \end{array}
%% \end{displaymath}

%% \begin{displaymath}
%%   \begin{array}{lll}
%%     \cfUlre {\tau} & \triangleq & \{(\uWorlds, n,  e_s, e_t) ~|~ \\
%%                    & & \forall \heap_s, \heap_t.
%%                        (n, \heap_s, \heap_t) \wDefined{\pb} \uWorlds \wedge \forall i<n, \sval.
%%                        e_s \ReducesTo{i} \sval \implies \\
%%                    & & \exists \heap_t', \tval. (\heap_t, e_t) \ReducesTo{} (\heap_t',\tval) \wedge
%%                        (\uWorlds, n-i, \sval, \tval) \in \cflrvu {\tau} {\pb} \wedge
%%                        (n-i, \heap_s, \heap_t') \wDefined{\pb} \uWorlds
%%                        \}
%%   \end{array}
%% \end{displaymath}
%% \caption{Cross language value and expression relation for {\cg} to {\fg}}
%% \label{fig:crossLangfg2cg}
%% \end{figure*}

%%% Local Variables:
%%% mode: latex
%%% TeX-master: "main"
%%% End:
