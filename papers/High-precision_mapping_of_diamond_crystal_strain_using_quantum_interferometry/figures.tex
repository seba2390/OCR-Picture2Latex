\newcommand{\w}{3.5in}

\newcommand{\introfigure}{
\begin{figure}[htbp!]
\begin{center}
\includegraphics*[width=\columnwidth]{fig1_introfiguresmall_smf.eps}
%introfigure_v4.eps}
\caption{\textbf{(a)} Simplified NV center level structure.  
%The electronic spin is initialized and read out via application of green light and collection of spin-dependent red fluorescence.  
Detail shows the ground-state electronic spin energy levels, as well as microwave (MW) field frequencies and detunings used in strain spectroscopy.  (Additional details in text.)   \textbf{(b)} and \textbf{(c)} Schematics of the confocal microscope and widefield quantum diamond microscope (QDM) used in this work - see Appendices \ref{sec:confocalapparatus} and \ref{sec:QDMapparatus} for additional details.}
\label{fig:introfigure}
\end{center}
\end{figure}
}

\newcommand{\sequencefig}{
\begin{figure}[htbp!]
\begin{center}
\includegraphics*[width=\columnwidth]{fig2_sequencefig.eps}
\caption{\textbf{(a)} Timing diagram of microwave (MW) pulses applied during a strain-CPMG measurement. \textbf{(b)} strain-CPMG measurement of NV ensemble visibility $\nu$ (defined in equation \eqref{eqn:suppvis}) as a function of evolution time, for MW detuning $\delta_{\rm{cm}}\approx$ 0.1 MHz (see text for details).  Red dashed line gives the decay envelope, with amplitude $A e^{-\tau/T_D}$ (see text).  The strain-CPMG dephasing time $T_D$ is 21 ${\rm \mu s}$, compared to the canonical inhomgeneous dephasing time $T_2^\star = 7.5 ~{\rm \mu s}$ measured using a Ramsey sequence (see Appendix \ref{sec:dephasingtimes}).  \textbf{(c)} Calibration curves for fixed evolution time $\tau_1 = 21~{\rm \mu s}$, showing $\nu$ as a function of applied detunings $\delta_{\rm cm}$ (blue, calibrating strain- or temperature-induced shifts in $D$) and $\delta_{\rm diff}$ (red, calibrating magnetic field variation).}
\label{fig:sequencefig}
\end{center}
\end{figure}
}

\newcommand{\confocalfig}{
\begin{figure}[htbp!]
\begin{center}
\includegraphics*[width=\columnwidth]{fig3_confocalfig.eps}
\caption{Three-dimensional NV-diamond strain measurements with a scanning confocal microscope. \textbf{(a)} Strain structure measured at the diamond surface. Position of slice in panel (c) is marked with a red dashed line. \textbf{(b)} Allan deviation for confocal strain measurements.   Blue points are for single-position measurements, while red are for the ``gradiometry'' configuration (see text).  Blue dotted line represents the calculated limit from shot noise in the avalanche photodiode (APD) used to measure NV fluorescence (see Appendix \ref{sec:noiselimit} for details). \textbf{(c)} Orthogonal slice across the strain feature, scanning beneath the diamond surface (marked by a red dashed line).  Spatial scales are unequal due to the high index of refraction of diamond.}
\label{fig:confocalfig}
\end{center}
\end{figure}
}




\newcommand{\imageronefig}{
\begin{figure*}[htbp!]
\begin{center}
\includegraphics*[width=\textwidth]{fig4_imageronefig.eps}
\caption{Widefield strain images taken with a quantum diamond microscope (QDM).  \textbf{(a)} Wide survey of strain features on diamond section B.  Individual fields-of-view, each representing about 150 $\times$ 150 $\mu$m$^2$ and one second of data acquisition, were manually stitched together by aligning overlapping features to form a composite image; minor compositing artifacts result from temperature drift between image acquisitions.  \textbf{(b)} and \textbf{(c)} Single fields-of-view, as measured in one second of data acquisition each. Panel \textbf{(b)} features extremely homogeneous strain, characteristic of this diamond material. Panel \textbf{(c)} showcases two strain features, likely induced by crystallographic defects incorporated during CVD growth.  \textbf{(d)} The denominator of the XY-normalized visibility indicates areas where intra-pixel strain gradients degrade the dephasing time $T_D$ (see text).  \textbf{(e)} Histogram of Allan deviations for 1 second of data acquisition for each pixel in panel (b), giving an estimate of the strain measurement uncertainty.}
\label{fig:imageronefig}
\end{center}
\end{figure*}
}

\newcommand{\smfive}{
\begin{figure*}[hbtp!]
\begin{center}
\includegraphics*[width=\textwidth]{SM5_new.eps}
\caption{\textbf{(a)} Detailed schematics of the strain-CPMG interferometric sensing protocol. Top: strain-CPMG pulse sequence. Bottom: NV spin population in each of the $\ket{{\rm m_s = 0,\pm 1}}$ states as a function of time. The swap operation exchanges the population between $\ket{+1}$ and $\ket{-1}$ states. Free evolution intervals are highlighted in red. Insets schematically illustrate evolution of the NV spin-state populations during swap operations. The first laser pulse initializes the NVs into $\ket{0}$, while the final laser pulse reads out the state-dependent fluorescence. \textbf{(b,c)} Time evolution of the state populations under the swap operation, using Eq. \eqref{eqn:swap_evolution}. Gray and blue shaded areas show evolution under resonant MW fields addressing the $\ket{0} \leftrightarrow \ket{-1}$ and $\ket{0} \leftrightarrow \ket{+1}$ transitions, respectively.  \textbf{(b)} Analytical calculation of the time-course of NV spin-state populations undergoing the strain-CPMG protocol, for an initial condition chosen such that the NV spin-state populations are equally separated between $\ket{0}$ and $\ket{-1}$, similar to an ideal case for the first swap pulse in the experiment. \textbf{(c)} Analytical calculation for an arbitrary initial condition with  70\% of the NV population in $\ket{-1}$ and 30\% in $\ket{+1}$, such that the non-equal populations at each step better illustrates the dynamics under the swap pulses.}
\label{fig:SMfig1}
\end{center}
\end{figure*}
}

\newcommand{\smsix}{
\begin{figure}[htbp!]
\begin{center}
\includegraphics*[width=\columnwidth]{SM6_new.eps}
\caption{Comparison of strain-CPMG sequences with different numbers of swap operations: time traces of the visibility $\nu$ (see equation \eqref{eqn:suppvis} below), with both MW tones detuned from resonance with a common-mode detuning $\delta_{\rm cm} = 0.35~{\rm MHz}$, for {\bf (a)} 2-swap, {\bf (b)} 4-swap, and {\bf (c)} 6-swap strain-CPMG sequences. Blue data points are measured values, while the red curves are fits to an exponentially decaying sinusoid. The two resulting fit parameters that determine sensitivity, i.e., the amplitude $A$ and decay rate $T_D$, are quoted for each sequence. The initial lag for the first data point arises from the minimum initial delay between swap operations.}
\label{fig:SMfig2}
\end{center}
\end{figure}
}

\newcommand{\smseven}{
\begin{figure}[htbp!]
\begin{center}
\includegraphics*[width=\columnwidth]{SM7_new.eps}
\caption{Comparison of free induction decays for different measurement protocols. {\bf (a)} Visibility ($\nu$) as a function of evolution time for the strain-CPMG sequence with common MW detuning $\delta_{\rm cm} = 0.35 ~{\rm MHz}$, yielding a determination of $T_D$. {\bf (b)} SQ measurement, addressing the $\ket{0} \leftrightarrow \ket{+1}$ transition with 3 MHz detuning, providing a determination of $T_2^\star$. The visibility oscillates as a superposition of three frequencies, each of which are detunings from hyperfine splittings.  The SQ measurement result is shown for the first 16 ${\rm \mu s}$, addressing the same period as that shown in the gray shaded region in (a) for the strain-CPMG measurement result. %{\bf (c)} DQ measurement with 3 MHz detuning from each transition frequency. $T_2^*\{\rm DQ\}$ is the shortest dephasing time, indicating a large magnetic field inhomogeneity throughout the interrogated NV ensemble.
}
\label{fig:SMfig3}
\end{center}
\end{figure}
}

\newcommand{\smeight}{
\begin{figure*}[hbtp!]
\begin{center}
\includegraphics*[width=0.9\textwidth]{SM8.eps}
\caption{\textbf{(a)} Four NV orientation classes, pointing along four crystallographic directions in the diamond crystal lattice.  \textbf{(b)} Widefield strain map of diamond section B, acquired using methods and QDM apparatus described in \cite{StrainPaper}.  The black rectangle designates the region shown in Fig.~\ref{fig:imageronefig}.  {\bf (c)} Example optically detected magnetic resonance (ODMR) spectrum from an NV ensemble; changes in fluorescence intensity reveal spin transition frequencies. Each NV resonance, corresponding to one or more of the four NV orientation classes, is split into three lines due to hyperfine interactions with the spin-1 $^{14}$N nucleus.  \textbf{(d)} Map of $\chi^2$ goodness-of-fit statistic for per-pixel fits yielding the strain map of panel (b).  Large intra-pixel strain gradients degrade the fits via deviation from Lorentzian lineshapes.}
\label{fig:SMfig_odmr}
\end{center}
\end{figure*}
}