\documentclass[12pt]{article}
\usepackage{indentfirst,latexsym,bm,amsmath,amssymb,amsfonts,lscape,rawfonts}
\usepackage{algorithm,algorithmic}
\usepackage{graphicx}
\usepackage{cite,epic,eepic,euscript,verbatim,amsmath,amssymb,amsthm,afterpage,float,bm,paralist, amscd}
\usepackage{mathtools}
\usepackage{epsfig}
\usepackage{caption}
\usepackage{subcaption}
\usepackage{geometry}
\usepackage{MnSymbol}
\geometry{margin=1in}


\newtheorem{theorem}{Theorem}
\newtheorem{lemma}{Lemma}[section]
\newtheorem{proposition}{Proposition}[section]
\newtheorem{corollary}{Corollary}[section]
\newtheorem{remark}{Remark}
\newtheorem{definition}{Definition}
\newtheorem{assumption}{Assumption}

\numberwithin{figure}{section}


\newcommand{\Real}{{\mathbb R}}
\newcommand{\R}{{\mathbb R}}
\newcommand{\Integer}{{\mathbb Z}}
\newcommand{\Vector}{{\mathbb R}^3}
\newcommand{\Matrix}{{\mathbb R}^{3 \times 3}}
%\newcommand{\supp}{\text{\rm supp\,}}
\newcommand{\Ah}{{\mathcal A}_h}
\newcommand{\B}{{\mathcal B}}
\newcommand{\E}{{\mathcal E}}
\newcommand{\F}{{\mathcal F}}
\newcommand{\J}{{\mathcal J}}
\newcommand{\N}{{\mathcal N}}
\newcommand{\calS}{{\mathcal S}}
\newcommand{\T}{{\mathcal T}}
\newcommand{\V}{{\mathcal V}}
\newcommand{\U}{{\mathcal U}}
\newcommand{\bigO}{{\cal O}}
\newcommand{\vsnoi}{\vspace{4mm}\noindent}

\newcommand{\bth}{\begin{theorem}}
\newcommand{\eth}{\end{theorem}}
\newcommand{\bpr}{\begin{proposition}}
\newcommand{\epr}{\end{proposition}}
\newcommand{\bde}{\begin{definition}}
\newcommand{\ede}{\end{definition}}
\newcommand{\blem}{\begin{lemma}}
\newcommand{\elem}{\end{lemma}}
\newcommand{\bco}{\begin{corollary}}
\newcommand{\eco}{\end{corollary}}
\newcommand{\prove}{\begin{proof}}
\newcommand{\done}{\end{proof}}
\newcommand{\brem}{\begin{remark}}
\newcommand{\erem}{\end{remark}}

\newcommand{\va}{{\bf a}}
\newcommand{\vb}{{\bf b}}
\newcommand{\vc}{{\bf c}}
\newcommand{\vd}{{\bf d}}
\newcommand{\ve}{{\bf e}}
\newcommand{\vf}{{\bf f}}
\newcommand{\vg}{{\bf g}}
\newcommand{\vj}{{\bf j}}
\newcommand{\vm}{{\bf m}}
\newcommand{\vn}{{\bf n}}
\newcommand{\vp}{{\bf p}}
\newcommand{\vq}{{\bf q}}
\newcommand{\vr}{{\bf r}}
\newcommand{\vt}{{\bf t}}
\newcommand{\vu}{{\bf u}}
\newcommand{\vv}{{\bf v}}
\newcommand{\vw}{{\bf w}}
\newcommand{\vx}{{\bf x}}
\newcommand{\vy}{{\bf y}}
\newcommand{\vz}{{\bf z}}
\DeclareMathOperator\Dom{Dom}

\numberwithin{equation}{section}

\newcommand{\beq}{\begin{equation}}
\newcommand{\eeq}{\end{equation}}
\newcommand{\lb}{\label}
\newcommand{\eps}{\varepsilon}

\newcommand{\bbN}{{\mathbb{N}}}
\newcommand{\bbR}{{\mathbb{R}}}
\newcommand{\bbD}{{\mathbb{D}}}
\newcommand{\bbP}{{\mathbb{P}}}
\newcommand{\bbE}{{\mathbb{E}}}
\newcommand{\bbZ}{{\mathbb{Z}}}
\newcommand{\bbC}{{\mathbb{C}}}
\newcommand{\bbQ}{{\mathbb{Q}}}
\newcommand{\bbT}{{\mathbb{T}}}
\newcommand{\bbS}{{\mathbb{S}}}
\newcommand{\calH}{{\mathcal{H}}}
\newcommand{\calV}{{\mathcal{V}}}

%\usepackage{showkeys}


\begin{document}

\title{Numerical Evidence of Exponential Mixing by Alternating Shear Flows}

\author{Li-Tien Cheng
\and
Frederick Rajasekaran
\and
Kin Yau James Wong
\and
Andrej Zlato\v s
\date{\today}
}

\newcommand{\Addresses}{{% additional braces for segregating \footnotesize
  \bigskip
  \footnotesize

  \textsc{Department of Mathematics, UC San Diego, La Jolla, CA 92093, USA}\par\nopagebreak
  \textit{E-mail:} \texttt{l3cheng@ucsd.edu, zlatos@ucsd.edu}

  \medskip

  \textsc{UC San Diego, La Jolla, CA 92093, USA}\par\nopagebreak
  \textit{E-mail:} \texttt{frajasek@ucsd.edu, k1wong@ucsd.edu}
}}

%\address{\noindent Department of Mathematics \\ UC San Diego \\ La Jolla, CA 92093, USA \newline Email:
%l3cheng@ucsd.edu, zlatos@ucsd.edu}
%
%\address{\noindent UC San Diego \\ La Jolla, CA 92093, USA \newline Email:
%frajasek@ucsd.edu, k1wong@ucsd.edu}


\maketitle

\begin{abstract}
We performed a numerical study of the efficiency of mixing by alternating horizontal and vertical shear ``wedge'' flows on the two-dimensional torus.  Our results suggest that except in cases where each individual flow is applied for only a short time, these flows produce exponentially fast mixing.  The observed mixing rates are higher when the individual flow times are shorter (but not too short), and randomizing either the flow times or phase shifts of the flows does not appear to enhance mixing (again when the flow times are not too short).  In fact, the latter surprisingly seems to inhibit it slightly.
\end{abstract}



\section{Introduction and Motivation}

The study of mixing of substances by incompressible flows has practical applications
% (mixing concrete and understanding weather trends \cite{concrete, ORourke2010OptimalMO}???) 
as well as connections to multiple branches of  mathematics and science.  The simplest mathematical model of the process of mixing in the absence of diffusion is the transport PDE
\beq \lb{1.1}
\rho_t+u\cdot\nabla\rho=0,
\eeq
where $\rho$ represents concentration of the mixed substance with some initial value $\rho(\cdot,0)=\rho_0$, and $u$ is the (prescribed and divergence-free) velocity of the mixing flow.  We will consider here $\rho\in L^\infty(\bbT^2\times[0,\infty))$, so the physical domain will be the two-dimensional torus.  Moreover,  since addition of a constant does not affect the dynamic of \eqref{1.1} and the spatial average $\int_{\bbT^2} \rho(x,t)dx= \int_{\bbT^2} \rho_0(x)dx$ of solutions is conserved, we can restrict our analysis to mean-zero solutions without loss.  We also stress that our interest is in mixing  by {\it divergence-free fluid flows} (due to practical considerations preferably time-periodic ones, possibly up to some simple transformations), acting on $\bbT^2$ continuously in time, as opposed to general {\it measure-preserving maps} $T:\bbT^2\to\bbT^2$ (which represent the discrete-time version of the problem but may not result from real-world advective stirring).

Two important questions about \eqref{1.1} concern optimal mixing rates of solutions, given certain natural constraints on the mixing flows $u$ (see the Section \ref{S2} for related definitions and further details), and which flows achieve these rates.   Addressing the first question, Crippa and De Lellis essentially showed in \cite{CL} that one cannot achieve faster than exponential-in-time mixing, thus also proving a modified version of Bressan's rearrangement cost  conjecture \cite{Bressan, B2}.  That this exponential rate is indeed achievable was shown by Yao and Zlato\v s \cite{YaoZla}, who found flows exponentially mixing any given initial data $\rho_0$ (including on domains with boundaries), as well as by Alberti, Crippa, and Mazzucato \cite{ACM2}, whose results only apply to a special class of initial data on $\bbT^2$ but also to a larger set of flow constraints.  These results therefore established optimality of exponential mixing and also found flows that achieve it.

Unfortunately, the flows constructed in \cite{YaoZla, ACM2} are quite complicated, far from time periodic, and heavily depend on the initial data.  All these facts have obvious practical limitations.  These issues were remedied by Elgindi and Zlato\v s \cite{ElgZla}, who constructed much simpler and time-periodic {\it almost universal exponential mixers} --- $\rho_0$-independent flows that mix exponentially all initial data that have at least some degree of regularity (the construction also extends to domains with boundaries and, unlike \cite{YaoZla, ACM2}, to all spatial dimensions).  These flows even mix all initial data asymptotically as $t\to\infty$, so they are {\it universal mixers}, but it was shown in \cite{ElgZla} that no universal mixer can have a  rate (exponential or otherwise) that is uniform in all bounded mean-zero $\rho_0$.

The construction in \cite{ElgZla} nevertheless still has one limitation.  While the constructed flows are H\" older continuous in space, they are not Lipschitz and their flow maps are discontinuous.  (The flows in \cite{YaoZla} share this limitation;  those in \cite{ACM2} apply to solutions taking only two values, so they can be modified arbitrarily on each of the two level sets without changing the solution dynamics, which allows one to avoid potential singularities in their construction.)  Also, Bedrossian, Blumenthal, and Punshon-Smith \cite{BBP} showed that generic solutions $u$ to the 2D Navier-Stokes equations with certain stochastic forcings are almost universal exponential mixers, which is of obvious practical interest.  On the other hand, these flows  are again quite complicated and not time-periodic, as well as not deterministic, and they only satisfy the required constraints on average in time rather than pointwise.  It is therefore still an open question whether (time-periodic) smooth or at least Lipschitz continuous almost universal exponential mixers (or even just universal mixers) exist on $\bbT^2$ or other domains.


One candidate for such flows on $\bbT^2$ was proposed by Pierrehumbert \cite{Pie,Pie2},
% and also studied in \cite{???}, 
and this suggestion is quite simple although not time-periodic.  It is almost every realization of the random vector field taking values $(\sin(2\pi x_2+\omega_n),0)$ and $(0,b\sin(2\pi x_1+\omega_n))$ (with $b\in\bbR$ some constant) on time intervals $(n-1,n-\frac 12]$ and $(n-\frac 12, n]$ (for $n\in\bbN)$, respectively.  Here  $\omega_n$ are independent random variables uniformly distributed over $\bbT$; note also that while these flows are not continuous in time, this is easily remedied by a simple reparametrization described in \cite{YaoZla}.

These flows are a representative of a wider class of alternatively horizontal and vertical shear flows. Heuristically, they appear to have very good mixing properties in many cases, but we are not aware of any rigorous proofs.
The goal of the present work is to demonstrate numerically that such flows can indeed yield exponential mixing of passive scalars advected by them.  We will consider random flows, with randomness in phase and/or flow time (the latter will replace the amplitude $b$ above), as well as deterministic time-periodic ones.

One difficulty with a computational approach to \eqref{1.1} is that when mixing is fast, solutions quickly become very rough, which poses a challenge from the numerical standpoint.  This may be further amplified when one considers non-smooth initial data, which we will do here because
\beq \lb{1.2}
\rho_0:= \chi_{[0,1/2)\times[0,1)} - \chi_{[1/2,1)\times[0,1)}
\eeq  
(considered also in, e.g., \cite{Bressan, ACM2}) is a natural choice of a ``minimally premixed''  initial datum; of course, $\rho_0$ is smooth away from the line $x_1=\frac 12$.

This problem can be somewhat remedied by adding a smoothing diffusion term to \eqref{1.1}, 
%which was done for instance in \cite{???}.  
but we do not wish to take this route and will instead address the issue by using a setup in which we can minimize the resulting complications.  We will use the approach from the advection step of Pierrehumbert's lattice method \cite{Pie2} (but with no diffusion step), where the solution is approximated by a linear combination of characteristic functions of $2^{2N}$ squares of size $2^{-N}\times 2^{-N}$ from a fine grid into which $\bbT^2$ is split (with $N\in\bbN$).  These squares are then moved according to the prescribed shear flow, with each shift rounded to be an integer multiple of the grid scale $2^{-N}$.  This results in a specific permutation of these squares in each advective step.  Of course, one can equivalently represent each square by its ``lower left'' vertex, and these vertices are then the grid points from
\beq \lb{1.5}
G_N:= \left\{ 0, \frac 1{2^N},\dots,\frac {2^N-1}{2^N} \right\}^2 \subseteq\bbT^2,
\eeq
 whose coordinates are integer multiples of $2^{-N}$.   We will do so and thus have $\rho(\cdot, t)\in L^\infty(G_N)$.

Moreover, we will avoid having to round the shifts by considering the
% (non-smooth but Lipschitz) 
horizontal and vertical  ``wedge'' flows
\beq \lb{1.3}
    u^H_{\omega}\left(    x_1,x_2\right) := \left( d_{\mathbb{T}}(x_2, \omega), 0 \right) \qquad\text{and}\qquad
    u^V_{\omega}\left(    x_1,x_2\right) := \left( 0, d_{\mathbb{T}}(x_1, \omega) \right)
 \eeq
instead of the sine flows from \cite{Pie}, with 
\[
d_{\mathbb{T}}(x,y) := \min \left\{ |x - y|, 1 - |x - y| \right\}
\]
 the distance on $\bbT$ (so $d_{\mathbb{T}}(x,y)\in [0,\frac 12 ]$) and $\omega\in\bbT$ some {\it phase shift}.  If $\omega$ is an integer multiple of $2^{-N}$ and either of these flows is applied for an integer length of time $\tau\in\bbZ$, then the resulting time-$\tau$ flow maps
\beq \lb{1.4}
    H_{\omega}^{\tau}\left(    x_1,x_2\right) := \left(x_1+ \tau d_{\mathbb{T}}(x_2, \omega), x_2\right) \qquad\text{and}\qquad
    V_{\omega}^{\tau}\left(x_1,x_2\right) :=\left(x_1, x_2+ \tau d_{\mathbb{T}}(x_1, \omega) \right)
\eeq
acting on $\bbT^2$ keep $G_N$  invariant. 
These then transform functions $f\in L^\infty(\bbT^2)$ via
\beq \lb{1.6}
    \calH_{\omega}^{\tau}[f] \left(    x_1,x_2\right) := f\left(x_1- \tau d_{\mathbb{T}}(x_2, \omega), x_2\right) \qquad\text{and}\qquad
    \calV_{\omega}^{\tau}[f] \left(x_1,x_2\right) := f\left(x_1, x_2- \tau d_{\mathbb{T}}(x_1, \omega) \right).
\eeq
Since we want our flows to satisfy time-uniform constraints,  $\tau\in\bbZ$ will represent the length of time  during which the particular shear flow is acting rather than the flow amplitude, and we will refer to it as \textit{flow time}.  For instance, the solution to \eqref{1.1} with $u=u_\omega^H$ is given by $\rho(\cdot,\tau)=\calH_{\omega}^{\tau}[\rho(\cdot,0)]$ (see Figure~\ref{flow_examples}).  We note that while these flows are only Lipschitz, which nevertheless still guarantees continuity of their flow maps, this comes with an additional advantage over the sine flows from \cite{Pie} and smooth flows in general.  Since they do not have ``flat'' spots (such as at the maxima and minima of the sines where their derivatives vanish), where less mixing is happening due to much less advective stretching in those regions, they should be much better candidates for efficient mixers.





\begin{figure}[ht]
    \centering
    \begin{subfigure}[t]{0.3\textwidth}
        \centering
        \includegraphics[width=\textwidth]{domain_1_0.png}
        \caption{$\rho_0$}
    \end{subfigure}
        \hfill
    \begin{subfigure}[t]{0.3\textwidth}
        \centering
        \includegraphics[width=\textwidth]{domain_1_1.png}
        \caption{$\calH_{1/4}^{1}[\rho_0]$}
    \end{subfigure}
    \hfill
    \begin{subfigure}[t]{0.3\textwidth}
        \centering
        \includegraphics[width=\textwidth]{domain_3_1.png}
        \caption{$\calH_{0}^{3}[\rho_0]$}
    \end{subfigure}

    
    \caption{Action of horizontal wedge flows on $\rho_0$ from \eqref{1.2} (values 1 and $-1$ are represented by blue and yellow colors, respectively).}
    \label{flow_examples}
\end{figure}
%Figure \ref{flow_examples} gives an example of three iterations of our flow with time switching 1 and fixed shift at $\omega=0.5$. In this figure, consider the purple and yellow as two different substances, say substance A and substance B, respectively. The initial set up is a $2^8 \times 2^8$ grid of points, with half of the points labeled as  substance A and half substance B. After the first iteration, the domain is sheared horizontally, then the second iteration is a vertical shear, and finally the third iteration reverts back to a horizontal shear.



The above setup means that we will consider here functions $\rho_0,\rho_1,\dots\in L^\infty(G_N)$, with $\rho_{k+1}$  being either $\calH_\omega^1[\rho_k]$ or $\calV_\omega^1[\rho_k]$, where $\omega\in\{0,\frac 1{2^{N}},\cdots,\frac{2^N-1}{2^{N}}\}$ is the phase shift of the either horizontal or vertical wedge flow that acts during the integer-length time interval that contains $[k,k+1)$.
We will study four cases here, with both $\omega$ and $\tau$ fixed as well as random.  When both are fixed, the resulting  flow is $2\tau$-periodic in time (one could fix separate values of $\tau$ for the horizontal and vertical flows but we will not do this here).  In the random case, we will randomly choose new $\omega$ and/or $\tau$ each time we switch the direction of the flow between horizontal and vertical (see Section \ref{S2} below for details).  In Figure~\ref{mixing_example}  we demonstrate the action of alternating wedge flows, with both phase shift and flow time fixed,  on the initial datum $\rho_0$ from \eqref{1.2}. 
In  the rest of this paper, we restrict $\rho$ and $\rho_0$ to $G_N\subseteq \bbT^2$ with $N=15$, where each point from $G_N$ represents its adjacent $2^{-N}\times 2^{-N}$ grid square.


\begin{figure}[htbp]
    \centering
    \begin{subfigure}[t]{0.3\textwidth}
        \centering
        \includegraphics[width=\textwidth]{domain_2_4.png}
        \caption{$(\calV_{0}^{2} \circ \calH_{0}^{2})^2[\rho_0]$}
    \end{subfigure}
    \hfill
    \begin{subfigure}[t]{0.3\textwidth}
        \centering
        \includegraphics[width=\textwidth]{domain_2_8.png}      
        \caption{$(\calV_{0}^{2} \circ \calH_{0}^{2})^{4}[\rho_0]$}
    \end{subfigure}
    \hfill
    \begin{subfigure}[t]{0.3\textwidth}
        \centering
        \includegraphics[width=\textwidth]{domain_2_16.png}
        \caption{$(\calV_{0}^{2} \circ \calH_{0}^{2})^{8}[\rho_0]$}
    \end{subfigure}
    
    \caption{Action of alternating wedge flows with fixed $\tau$ and $\omega$ (the monochromatic  lines in the top left and bottom right of the pictures are due to $\tau=2$, see Subsection \ref{S3.2}).}
    \label{mixing_example}
\end{figure}

Our results support the conjecture that all the flows studied here are indeed exponential mixers for $\rho_0$,
% (and likely also almost universal exponential mixers), 
except possibly some of those with $|\tau|\le 2$.  Although we only consider the single initial datum \eqref{1.2}, the fact that it is ``minimally premixed'' and that it appears to be  mixed exponentially quickly regardless of the choice of the phase shifts and flow times, suggest that  results for other initial data would be similar, and therefore these flows could in fact be almost universal exponential mixers.

Let us now turn to the specifics of our work.  We discuss the details of our setup and the definition of mixing scales that we will use here in Section \ref{S2}.  We then present our results in Section \ref{S3}, the main one being numerical evidence of exponential mixing by alternating wedge flows, summarized in the table in Figure \ref{main_data}.
\medskip

{\bf Acknowledgements}.  AZ thanks Gautam Iyer and Jean-Luc Thiffeault for illuminating discussions.   LTC acknowledges partial support by  NSF grant DMS-1913144.  FR and KYJW were supported in part   by Division of Physical Sciences Undergraduate Summer Research Awards and 
%Triton Research \& Experiential Learning Scholars 
TRELS Awards from UC San Diego.
% Division of Physical Sciences.  
AZ acknowledges partial support by  NSF grant DMS-1900943 and by a Simons Fellowship.

\section{Definition of Mixing Scales and the Modeled Flows} \lb{S2}

\subsection{Mixing Scales of Bounded Functions}

In order to be able to study mixing efficiency of flows, we need to adapt a relevant definition of mixing scales of solutions $\rho$ to \eqref{1.1} with initial data $\rho_0$.  Following \cite{Bressan,YaoZla}, one option is to say that $\rho(\cdot, t)$ is $\kappa$-mixed to scale $\eps>0$  (for some $\kappa\in(0,1)$) when
\beq \lb{2.1}
\left| \strokedint_{B_{\varepsilon}(y)} \rho(x,t)\,dx \right| \leq \kappa ||\rho_0||_{\infty}
\eeq
holds for each $y\in\bbT^2$ (note that $||\rho||_{\infty}=||\rho_0||_{\infty}$).  The  mixing scale of $\rho(\cdot,t)$ is then the infimum of all $\eps> 0$ such that $\rho(\cdot,t)$ is mixed to scale $\eps$.  This is also called the {\it geometric mixing scale} in \cite{YaoZla}, and was recently used in various other works including \cite{ACM, ACM2,CL, ElgZla, CLS}.  Other alternatives are the {\it functional mixing scale} $\|\rho(\cdot,t)\|_{\dot H^{-s}}\|\rho(\cdot,t)\|_{\infty}^{-1}$ for some $s>0$ (particularly $s\in\{\frac 12,1\}$), used for instance in  \cite{ACM2, BBP, IKX, LTD, LLNMD, S,CLS, ElgZla} and closely related to the geometric mixing scale (see \cite{YaoZla}), as well as the Wasserstein distance of the positive and negative parts of $\rho(\cdot,t)$ \cite{BOS, OSS, S, S2}.
We note that there is also a large literature on the interaction between mixing and diffusion, as well as alternative definitions of mixing in the diffusive setting, but we will not attempt to provide an overview here and instead refer the reader to the review \cite{T} (which also concerns the diffusion-less case \eqref{1.1}) and references therein.

We will consider here a version of the geometric mixing scale above, which is most suited to our setting, but with further adjustments.  In the discrete setting of the grid $G_N$ that we will consider here, averaging the solution over discs may be problematic as the number of points from $G_N$ inside the disc $B_\eps(y)$ is not a constant multiple of $\eps^2$.  Moreover, finding averages over all discs of a particular radius centered at points from $G_N$ is unnecessarily computationally intensive, while restricting this to only points from some sub-grid would mean that not all points from $G_N$ are equally represented in the mixing scale computation.  

It is therefore both more reasonable and better suited to our setup to consider only $\eps=2^{-n}$ for $n=0,1,\dots$ and replace the discs $B_\eps(y)$ in \eqref{2.1} by all $2^{2n}$ squares
\[
S_n^{i,j} := \left[\frac{i}{2^n},\frac{i+1}{2^n}\right)\times\left[\frac{j}{2^n},\frac{j+1}{2^n}\right)
\]  
with $i,j\in \{0,1,\dots,2^{n}-1\}$.
One can easily show (see the proof of Lemma 2.7 in \cite{ElgZla}) that the resulting definition of the mixing scale (which will always be a power of $\frac 12$) is essentially equivalent to the above definition of geometric mixing scale when it comes to the study of asymptotic mixing rates (ratio of one with any $\kappa\in(0,1)$ and the other with any $\kappa'<\kappa$ is bounded above by a constant depending only on $\kappa-\kappa'$).  As mentioned in the introduction, we will also have $\rho_0,\rho_1,\dots\in L^\infty(G_N)$ instead of $\rho\in L^\infty(\bbT^2\times[0,\infty))$, so the averages over these squares will be just the averages over the points from $G_N$ contained in them.  Of course, this means that  the minimal possible mixing scale will be $2^{1-N}$ (unless $f\equiv 0$).
The choice of $\kappa\in(0,1)$ does not affect the exponential mixing rates as $t\to\infty$ (in the continuous space setting as well as in the $N\gg t$ regime in the discrete setting) but will have some effect on finite time intervals.  We choose $\kappa:=\frac 13$ as in \cite{Bressan}, which finally yields the following definition.

\bde \lb{D.2.1}
We say that a mean-zero function $f:G_N\to\bbR$ is {\it mixed to scale} $2^{-n}$ for some $n\in\{0,1,\dots,N\}$ if for each pair $i,j\in \{0,1,\dots,2^{n}-1\}$ we have
\[
\left| {2^{-2(N-n)}}\sum_{x\in\mathcal{S}_n^{i,j}\cap G_N} f(x)\right| \leq \frac{\| f\|_\infty}{3}.
\]
The {\it mixing scale} of $f$ is the smallest such $2^{-n}$.
\ede

Note that in our setting we will always have $\|\rho_k\|_\infty=\|\rho_0\|_\infty = 1$ for all $k\in \bbN_0$.

\subsection{The Modeled Flow Types}
\label{S2.2}


As discussed in the introduction, we will consider here four basic flow types, all alternating wedge flows.  Two will have fixed phase shifts (chosen randomly at the start) and two will have their phase shifts chosen randomly each time we switch the flow direction.  Two will have fixed flow times and two will have their  flow times chosen randomly each time we switch the flow direction. We will use the discretized framework described in the introduction with $N=15$ (so grid scale will be $2^{-15}$), modeling the flow dynamic via the mappings from \eqref{1.6} applied to the initial data \eqref{1.2} on the domain \eqref{1.5}.  The specific details are as follows.


\medskip\noindent
\textbf{Fixed Shift Fixed Time (FSFT):} 
We choose randomly phase shifts $\omega,\omega'\in\{0,\frac 1{2^{N}},\cdots,\frac{2^N-1}{2^{N}}\}$ (with uniform joint distribution), then fix these and some flow time $\tau\in\{2,\dots,10\}$, and  let 
\[
\rho_{k+1}:=
\begin{cases}
\calH_{\omega}[\rho_k] & k\in [2j\tau,(2j+1)\tau) \text{ for some $j\in\bbN_0$}, 
\\ \calV_{\omega'}[\rho_k] & k\in [(2j+1)\tau,(2j+2)\tau) \text{ for some $j\in\bbN_0$}, 
\end{cases}
\]
for $k=0,1,\dots$.
 Since one should expect different behavior for different  $\tau$ (which we do confirm), we model these cases  separately.  The phase shifts $\omega,\omega'$ are not expected to have a significant effect on the mixing rates (which we also confirm), although they will have some effect on the computed mixing scales at individual times $k$.  This is clear for the horizontal shift $\omega'$ of the vertical  flow $u^V_{\omega'}$ because different shifts align differently with $\rho_0$.  The vertical shift $\omega$ would have no effect on the mixing scales if we were to  include in Definition~\ref{D.2.1} all the $2^{-n}\times 2^{-n}$ squares with vertices in $G_N$ rather than just the squares $S_n^{i,j}$ (which have vertices in $G_n$).  We do not do this in order to shorten the required computing time, and our simulations show that the effect on the obtained mixing rates would also be negligible.  We run the simulation 100 times (i.e., with 100 random choices of $(\omega,\omega')$) for each $\tau$.
 
\medskip\noindent
\textbf{Random Shift Fixed Time (RSFT):} 
Here the phase shifts are i.i.d.~random variables $\omega_0,\omega_0', \omega_1,\omega_1',\dots \in\{0,\frac 1{2^{N}},\cdots,\frac{2^N-1}{2^{N}}\}$ (with uniform distribution) and the flow time $\tau\in\{2,\dots,10\}$ is again fixed, so we have
\[
\rho_{k+1}:=
\begin{cases}
\calH_{\omega_j}[\rho_k] & k\in [2j\tau,(2j+1)\tau) \text{ for some $j\in\bbN_0$}, 
\\ \calV_{\omega_j'}[\rho_k] & k\in [(2j+1)\tau,(2j+2)\tau) \text{ for some $j\in\bbN_0$}, 
\end{cases}
\]
for $k=0,1,\dots$.  Contrasting the results in this case with those for FSFT allows one to see whether in the latter case the mappings $\calV_{\omega'}\circ\calH_{\omega}$ (which generate the FSFT dynamic, and coincide for all $(\omega,\omega')$ up to translation) may involve structures that slow down or accelerate mixing, since such structures would not persist in the RSFT case. We perform 100 simulations for each $\tau$.
 
\medskip\noindent
\textbf{Fixed Shift Random Time (FSRT):} 
Here the phase shifts are randomly chosen at the start and then kept fixed as in FSFT, but the flow time  is chosen randomly at each direction switch to see whether this can improve mixing. Since our FSFT and RSFT results show that the mixing rate depends nontrivially on the flow time when the latter is fixed, decreasing as $\tau$ increases from 3 or 4 to higher values, it makes sense to limit the randomness in the flow time to small intervals.  We therefore let the flow times be i.i.d.~random variables $\tau_0,\tau_0',\tau_1,\tau_1',\dots\in\{\tau-1,\tau,\tau+1\}$ (with uniform distribution) for some fixed $\tau\in\{2,\dots,10\}$, so
\[
\rho_{k+1}:=
\begin{cases}
\calH_{\omega}[\rho_k] & k\in [t_j,t_j+\tau_j) \text{ for some $j\in\bbN_0$}, 
\\ \calV_{\omega'}[\rho_k] & k\in [t_j+\tau_j,t_{j+1}) \text{ for some $j\in\bbN_0$}, 
\end{cases}
\]
for $k=0,1,\dots$, with $t_j:=\sum_{l=0}^{j-1}(\tau_l+\tau_l')$.  We  perform 100 simulations for each $\tau$.

\medskip\noindent
\textbf{Random Shift Random Time (RSRT):} 
Here the phase shifts $\omega_0,\omega_0', \omega_1,\omega_1',\dots$ are chosen as in RSFT and flow times $\tau_0,\tau_0',\tau_1,\tau_1',\dots$ as in FSRT, so now
\[
\rho_{k+1}:=
\begin{cases}
\calH_{\omega_j}[\rho_k] & k\in [t_j,t_j+\tau_j) \text{ for some $j\in\bbN_0$}, 
\\ \calV_{\omega_j'}[\rho_k] & k\in [t_j+\tau_j,t_{j+1}) \text{ for some $j\in\bbN_0$} ,
\end{cases}
\]
for $k=0,1,\dots$, with $t_j:=\sum_{l=0}^{j-1}(\tau_l+\tau_l')$.  We  perform 100 simulations for each $\tau\in\{2,\dots,10\}$.

\medskip
Of course, the formula for $\rho_{k+1}$ in the RSRT case also applies in the other cases, but with $(\omega_j,\omega_j'):=(\omega,\omega')$ for all $j\in\bbN_0$ and/or  $(\tau_j,\tau_j'):=(\tau,\tau)$ for all $j\in\bbN_0$.

%-------------------
%
%
%\subsection{Types of Flows}
%
%
%
%Given that we run mixing simulations on $Q_{2,N}$ for $N\in\mathbb{Z}^{\geq0}$, we need $H_{\omega}^{\tau}$ and $V_{\omega}^\tau$ to be bijections on $Q_{2,N}$. Hence, we restrict $\omega$ and $\tau$ to values in $D_N$ and $\mathbb{Z}^+$ respectively. The use of rounding functions could give approximate results for noninteger $\tau$ values and non-dyadic rational $\omega$ values as done by Pierrehumbert in his lattice model \cite{Pierrehumbert1999}. However, by restricting $\tau$ to integer values, we also get the convenient property that $H_{\omega}^{\tau} = (H_{\omega}^1)^{\tau}$ and $V_{\omega}^{\tau} = (V_{\omega}^1)^{\tau}$ for $\tau \in \mathbb{Z}^+$. It is useful to think of $H_{\omega}^{\tau}$ flow as $\tau$ consecutive $H_{\omega}^1$ flows for the purposes of measuring the mixing scale at each unit of time (we shall consider each $H_{\omega}^1$ and $V_{\omega}^1$ to occupy exactly one unit of time). For example, a $H_{\omega}^{8}$ flow can be decomposed into eight $H_{\omega}^1$ flows, and the mixing scale can be evaluated between each. This technique is beneficial as it not only gives information on the mixed-ness of the domain after the flow, but also during. Breaking larger flows down into smaller flows is also useful when analytically calculating bounds on the mixing rate as in Section 3.2.
%
%
%
%The specific parameters we vary to construct the flows are as follows: 
%
%\begin{enumerate}
%    \item Randomization of the shift at each alternating flow by picking $\omega$ uniformly at random in $D_N$, where $Q_{2,N}$ is the domain as done in \cite{PIERREHUMBERT19941091, Pierrehumbert1999}.
%    
%    \item Variation of the switching time for integer values between 1 and 10.
%    
%    \item Randomization of the switching time by picking $\tau$ uniformly at random in a range of values.
%\end{enumerate}
%If the shift is not randomized, then we let $\omega = \frac{1}{2}$ for the entire simulation.


\subsection{Computation of Exponential Mixing Rates} \lb{S2.3}

In each of the four  cases above and for each $\tau\in\{2,\dots,10\}$, we performed 100 simulations.  In every simulation we found the mixing scale $2^{-n_k}$ of the solution $\rho_k$ at each time $k\in\bbN_0$ via Definition \ref{D.2.1}.
% (to optimize efficiency, at time $k+1$ we started by finding the averages for squares $S_{n_k+1}^{i,j}$ at scale $2^{-(n_k+1)}$ because the mixing scale decreases with time on average and it is faster to move from a smaller scale to a larger one than the other way around).  
We then used these mixing scales to find an exponential mixing rate of the flow in each individual run via linear regression over a relevant time interval (see below) and finally averaged these rates over the 100 runs.
%Then, for each flow case and each $\tau$, we computed the mean and standard deviation of the binary logarithms of the mixing scales at each time $k$  (i.e., of the 100 values of $-n_k$).  Finally, we then used these mean mixing scales to find the exponential mixing rates of these flows via linear regression over a relevant time interval.

%The mixing scale computed via Definition \ref{D.2.1}
The computed mixing scales can never reach the grid scale $2^{-N}$, and they typically plateaued around $2^{4-N}$ in our simulations  with $N=15$ (see Figure \ref{plot_example}) as well as for other values of $N$ (since the square $S_{N-3}^{i,j}$ has 64 points from $G_N$, reaching mixing scale $2^{3-N}$ requires it to contain between 22 and 42 points of either color {\it for each} $(i,j)$).  In order to suppress this grid scale effect, we chose the end of the time interval for computation of the mixing rate to be the first time when the mixing scale reached $2^{5-N}$.  For $N=15$ this is $2^{-10}$, and we denote this time $T_{10}$ below.
%(i.e., 10 for $$ (i.e., where for each $m\in\bbN_0$ we let $T_m\in\bbN_0$ be  the first time such that the mean of the 100 values of $-n_{T_{m}}$ drops below $-m$ (for a fixed flow case and fixed $\tau$).  

Our simulations also showed that there is an initial time interval where the mixing scale may display somewhat irregular behavior.  An example of this is  in Figure \ref{plot_example}, which contains means and standard deviation error bars of the binary logarithms $-n_k$ of the mixing scales at different times $k$  for the 100 simulations  in two flow cases, the RSFT case with $\tau=3$ and the FSRT case with $\tau=7$.  One can observe near-plateaus of the averaged mixing scales in the time interval $[3,7]$ (roughly while these scales are between $2^{-1}$ and $2^{-2}$), likely due to an interaction of the initial data with different phase shifts, before they start an almost constant rate descent (until they plateau around $2^{-11}$).  A similar pattern appears before time $8$ in most of the other cases of flows with $\tau\ge 3$ (which are the ones providing efficient mixing, see Section \ref{S3} below), more so for smaller values of $\tau$.  None is more pronounced than on the left of Figure \ref{plot_example}, and therefore their effect on the computed mixing rates would be very small.  Nevertheless, in the interest of obtaining the most accurate approximations of the actual asymptotic mixing rates, we suppressed it by choosing the start of the time interval for computation of the mixing rates to be 8.  (We note that in almost all cases with $\tau\ge 3$, the first time when the averaged mixing scales dropped below $2^{-2}$ was either 7 or 8.  The exceptions were FSRT and RSRT with $\tau=3$, when this time was 9; these are however also somewhat exceptional, see Subection \ref{S3.1} below.)

%This is likely due to an interaction of the initial data with different phase shifts, and  we suppressed it by choosing the start of the time interval for computation of the mixing rate to be  $T_2$.  
%That is, we computed the exponential mixing rates over the time intervals during which the mean mixing scales dropped $N-8$ binary scales.
%%, from $2^{-2}$ to $2^{5-N}$. 
%Nevertheless, this  adjustment still only affects the mixing scales minimally --- of all the cases of flows with $\tau\ge 3$ (which are the ones providing efficient mixing, see Section \ref{S3} below), none had a more irregular behavior of its mean mixing scales at small times than the RSFT case with $\tau=3$ in Figure \ref{plot_example}.  Also, the value of $T_2$ varied little among the cases with $\tau\ge 3$ (see Figure \ref{Tvalues}).

\begin{figure}[htbp]
    \centering
        \includegraphics[scale = 0.55]{RSFT03_mean.png} %\hskip 10mm
        \includegraphics[scale = 0.55]{FSRT07_mean.png}        
    \caption{Means and standard deviation error bars of the binary logarithms of the mixing scales for the 100 simulations 
    %plotted against time.  This is 
   with $N = 15$ in the RSFT case with $\tau=3$ (left) and  in the FSRT case with $\tau=7$ (right).  The (mean) mixing scales have a near-uniform exponential decay before eventually plateauing around $2^{-11}$.}
    %, and the  (averaged) base-2  exponential mixing rate computed via linear regression on the  time interval $[T_2,T_{10}]=[8,30]$  is $0.3792$ (with R-squared being $0.9958$).}
    \label{plot_example}
\end{figure}

So with $N=15$, we considered the time interval $[8,T_{10}]$ for each individual simulation (with $T_{10}$ being simulation-dependent).  We then found the corresponding exponential mixing rate 
%by performing linear regression for
as the {\it negative} of the slope of the line that best (in terms of least squares) fits
 the computed  binary logarithms $-n_k$ of mixing scales at all the integer times within this time interval (so this is then a base-2 exponential rate).  
We also computed the R-squared value of the fitting line (its proximity to 1 indicates a good fit and hence a near-uniform exponential decay of the mixing scales), as well as averages and standard deviations of the mixing rates and the R-values for the 100 runs for each flow type and each $\tau$.
 %One should of course expect this slope to be roughly $\frac{8}{T_2-T_{10}}$ if the exponential decay of the (averaged) mixing scales is mostly uniform on $[T_2,T_{10}]$, with the degree of uniformity being high if the R-squared value of the fitting line is close to 1. For instance, Figure \ref{plot_example} contains the corresponding plot and values in the RSFT case with $\tau=3$.



%For our work, the maximum N we tested was 15, The tests were run on the BOOM Cluster in the UCSD Mathematics Department. A simulation with with N = 15, 60 flows, and 100 iterations (to account for the randomness) takes around 25 hours to run (each
% iteration takes approximately 15 minutes).






%\section{Our Approach}
%
%\subsection{Summary} %rename
%%compare method to naive mixing tests
%As previously mentioned, all mixing simulations are run on $Q_{2,N}$, and each simulation can be decomposed into a product of time-switching 1 flows. Thus, the geometric mixing scale of the concentration $\rho$ is computed at each $t\in\mathbb{Z}^{\geq0}$, with the unit of time being either a $H_\omega^1$ flow or a $V_\omega^1$ flow. The geometric mixing scale at time $t$ is calculated by testing the mixed-ness of $\rho(\cdot,t)$ with neighboring scales of a `reasonable' mixing scale estimate. If $\rho(\cdot,t)$ is indeed mixed to the estimated scale, then the mixed-ness is tested for consecutively smaller scales. On the other hand, if $\rho(\cdot,t)$ is not mixed to the estimated scale, then the mixed-ness is tested for consecutively larger scales. We always need one extra test for mixed-ness to confirm the geometric mixing scale. One should also note that computing mixed-ness down scales is slower than computing mixed-ness up scales. This is because one can simply add the averages calculated from the smaller scales to evaluate mixed-ness at larger scales. Computing mixed-ness at smaller scales essentially requires a recount of averages.
%
%A reasonable mixing scale estimate is a value that reduces the computation needed to find the actual geometric mixing rate. If the mixing scale of $\rho(\cdot,t)$ is $2^{-n}$ at some time $t$, then a reasonable estimate for the mixing scale at time $t+1$ would be $2^{-n-1}$. This is because $\rho$ is likely to be non-increasing with respect to $t$, and the mixer is unlikely to exceed being exponential in time. Hence, taking estimates to be a scale smaller than previous geometric mixing scales reduces the probability of computing mixed-ness down scales (which, as noted earlier, is slower that computing mixed-ness up scales).
%
%%need clarification on mixing rate
%The mixing rate $\lambda:\mathbb{Z}^{\geq0}\to\mathbb{R}$ is a function that assigns the geometric mixing scale of $\rho(\cdot,t)$ at time $t$ to each $t\in\mathbb{Z}^{\geq0}$. If the mixing rate $\lambda(t)$ is exponential, i.e., $\lambda(t)\leq2^{-\gamma t}$ for some $\gamma > 0$, we call $\gamma$ the exponential mixing constant. So with our mixers, once we show evidence of exponential mixing, we can calculate the exponential mixing constant by finding the slope of the best-fit line of time versus the log of the mixing rate.
%
%\begin{figure}[H]
%    \centering
%        \includegraphics[scale = 0.75]{Images/sampleslopecalcgraph.eps}
%    \caption{An example of the log of the mixing rate data plotted against time in order to calculate the exponential mixing constant. This mixing data is from a sequence of maps with switching time four and random shift. This simulation was run with $N = 15$, i.e., $2^{15} \times 2^{15} $ points. Due to resolutions issues, the mixing scale plateaus at $2^{-11}$.}
%    \label{plot_example}
%\end{figure}
%
%Figure \ref{plot_example} gives an example of the plot of mixing data. Since this simulation was done with a random shift, we simulated the mixing a hundred times, and then averaged the results (hence the error bars). From this, one can see that the exponential mixing constant for this mixer is 0.333363.


%\subsection{Constraints and Technical Details}
%As noted in Figure \ref{plot_example}, resolution issues prevent the mixing scale from going lower than $2^3$ or $2^4$ times the number of grid points along one edge of the domain, i.e., $2^N$ for some $Q_{2,N}$. In other words, when there are less than $256 = 2^4 \times 2^4$ points within a grid square, the mixing scale generally does not go lower. However, this proved to be a nonissue as trends in the mixing scale were consistent regardless of what $N$ is. The mixing rate and the trends as shown in Figure $\ref{plot_example}$ (where $N = 15$) were consistent when tested with $N =10$ and $N=13$.
%
%For our work, the maximum $N$ we tested was 15, The tests were run on the BOOM Cluster in the UCSD Mathematics Department. A simulation with with $N = 15$, 60 flows, and 100 iterations (to account for the randomness) takes around 25 hours to run (each iteration takes approximately 15 minutes). 
%
%Finally, due to resolution issues, the flows we were able to test were highly constricted. In order to prevent these issues, all flows had to be bijections from $Q_{2,N}$ to $Q_{2,N}$. This restricted our switching time values to positive integers, shift values to $D_N$, and prevented us from testing flows besides wedge shear flows, e.g. we could not test smooth flows such as sine functions. 


\section{Main Results and Discussion} \label{S3}

In this section, we first present the results of the simulations described in Section \ref{S2}.  These constitute numerical evidence that the alternating wedge flows from Subection \ref{S2.2} indeed exhibit exponential mixing, except in some cases with small $\tau$, and we also identify the apparently most efficient mixers among these flows.  We then discuss the cases with small $\tau$, when mixing rates are lower and mixing can even be algebraic, as well as the related existence of structures fixed by the mixing dynamic.

\subsection{Evidence of Exponential Mixing} \label{S3.1}

The table in Figure \ref{main_data} contains averages of the  base-2 exponential mixing rates  for all four flow types and all $\tau\in\{2,\dots,10\}$, computed as described in Subsection \ref{S2.3} (with $N=15$ and 100 runs in each case).  It suggests that the most efficient mixing by the alternating wedge flows considered here happens when each individual wedge flow acts for the same time, which is either 3 or 4, and the phase shifts of the flows are not varied during each run.

It also shows that flows with $\tau=2$ are much worse mixers than those with $\tau\ge 3$; this is even more pronounced for $\tau=1$, which is why we did not include that case in our simulations.  We discuss these issues in Subsection \ref{S3.2} below, so will now concentrate on the cases with $\tau\ge 3$.
We note that  in all 32 of them, the means of the R-squared values were at least $0.9660$ and their
%(and at least $0.9958$ when $\tau=3,4$), 
%Figure~\ref{plot_example} is an example of this, with 
%%$[T_2,T_{10}]=[8,30]$ and 
%R-squared being $0.9958$.  
 standard deviations were no more than 0.0109, which is why we do not report these here (we note that even the 3200 individual R-squared values were all no less than 0.9290).  These numbers indicate an excellent fit and near-uniform exponential decay in all cases. 

% \begin{figure}[H]
%     \centering
%         \begin{tabular}{|c|c|c|}
%     \hline
%          Flow Type & Slope \\
%          \hline
%          \hline
%          Switching Time Range 1-3, Fixed Shift  & -0.122094 \\
%          \hline
%          Switching Time Range 2-4, Fixed Shift  & -0.307231 \\
%          \hline
%          Switching Time Range 3-5, Fixed Shift  & -0.330145 \\
%          \hline
%          Switching Time Range 4-6, Fixed Shift  & -0.314095 \\
%          \
%          Switching Time Range 5-7, Fixed Shift  & -0.306429\\
%          \hlinehline
%          Switching Time Range 1-3, Random Shift  & -0.087282 \\
%          \hline
%          Switching Time Range 2-4, Random Shift  & -0.295070 \\
%          \hline
%          Switching Time Range 3-5, Random Shift  & -0.309476 \\
%          \hline
%          Switching Time Range 4-6, Random Shift  & -0.303556 \\ 
%          \hline
%          Switching Time Range 5-7, Random Shift & -0.293136 \\
%          \hline
%          Fixed Switching Time 1, Random Shift  & -0.039867\\
%          \hline
%          Fixed Switching Time 2, Random Shift  & -0.180384 \\
%          \hline
%          Fixed Switching Time 3, Random Shift  & -0.320945 \\
%          \hline
%          Fixed Switching Time 4, Random Shift & -0.325099 \\
%          \hline
%          Fixed Switching Time 5, Random Shift  & -0.305369 \\
%          \hline
%          Fixed Switching Time 6, Random Shift & -0.301724 \\
%          \hline
%          Fixed Switching Time 7, Random Shift & -0.279211 \\
%          \hline
%          Fixed Switching Time 8, Random Shift  & -0.269351 \\
%          \hline
%          Fixed Switching Time 3, Fixed Shift  & -0.339362 \\
%          \hline
%          Fixed Switching Time 4, Fixed Shift & -0.346675 \\
%          \hline
%          Fixed Switching Time 5, Fixed Shift  & -0.314347 \\
%          \hline
%          Fixed Switching Time 6, Fixed Shift  & -0.310530 \\
%          \hline
%          Fixed Switching Time 7, Fixed Shift  & -0.287711 \\
%          \hline
%          Fixed Switching Time 8, Fixed Shift  & -0.280727 \\
%          \hline
%     \end{tabular}
%     \caption{The mixing rates for all of the different flow types.}
%     \label{fig:my_label}
% \end{figure}


%    \begin{subfigure}[t]{0.24\textwidth}
%        \centering
%         \begin{tabular}{|c|c|}
%    \hline
%         $\tau$ & Slope \\
%         \hline
%         \hline
%         2 &  0 \\
%         \hline
%         3  & -0.3365 \\
%         \hline
%         4 & -0.3186 \\
%         \hline
%         5  & -0.2956 \\
%         \hline
%         6 & -0.2864 \\
%         \hline
%         7  & -0.2633 \\
%         \hline
%         8 & -0.2573 \\
%         \hline
%         9 & -0.2438 \\
%         \hline
%         10 & -0.2252 \\ 
%         \hline
%    \end{tabular}
%        \caption{Fixed Shift}
%    \label{data_fixed}
%    \end{subfigure}
%    %\centering
%    \begin{subfigure}[t]{0.24\linewidth}
%        \centering
%         \begin{tabular}{|c|c|}
%    \hline
%         $\tau$ & Slope \\
%         \hline
%         \hline
%         2  & -0.2168 \\
%         \hline
%         3  & -0.3792 \\
%         \hline
%         4 & -0.3665\\
%         \hline
%         5  & -0.3376 \\
%         \hline
%         6 & -0.3235 \\
%         \hline
%         7 & -0.2999\\
%         \hline
%         8  & -0.2757 \\
%         \hline
%         9 & -0.2675 \\
%         \hline
%         10 & -0.2547 \\
%         \hline
%    \end{tabular}
%        \caption{Random Shift}
%    \label{data_random}
%    \end{subfigure}
%    %\centering
%     \begin{subfigure}[t]{0.24\textwidth}
%        \centering
%         \begin{tabular}{|c|c|}
%    \hline
%         $\tau$ & Slope \\
%         \hline
%         \hline
%         1-3  & -0.1181 \\
%         \hline
%         2-4  & -0.3081 \\
%         \hline
%         3-5 & -0.3184 \\
%         \hline
%         4-6 & -0.3042 \\
%         \hline
%         5-7 & -0.2839 \\
%         \hline
%         6-8 & -0.2705 \\
%         \hline
%         7-9 & -0.2560 \\
%         \hline
%         8-10 & -0.2438 \\
%         \hline
%         9-11 & -0.2268 \\
%         \hline
%    \end{tabular}
%        \caption{Fixed Shift}
%    \label{data_fixed}
%    \end{subfigure}
%    %\centering
%         \begin{subfigure}[t]{0.24\textwidth}
%        \centering
%         \begin{tabular}{|c|c|}
%    \hline
%        $\tau$ & Slope \\
%         \hline
%         \hline
%         1-3  & -0.1185 \\
%         \hline
%         2-4  & -0.3520 \\
%         \hline
%         3-5  & -0.3615 \\
%         \hline
%         4-6  & -0.3436 \\ 
%         \hline
%         5-7 & -0.3185 \\
%         \hline
%         6-8 & -0.3035 \\
%         \hline
%         7-9 & -0.2832 \\
%         \hline
%         8-10 & -0.2682 \\
%         \hline
%         9-11 & -0.2523 \\
%         \hline
%    \end{tabular}
%        \caption{Random Shift}
%    \label{data_fixed}
%    \end{subfigure}
  
    %\centering

\begin{figure}[htbp]
\centering    
\begin{tabular}{|c|c|c|c|c|}
	\hline
	$\tau$ & FSFT & RSFT & FSRT & RSRT \\\hline\hline
%	1 & & & & \\\hline
	2 & --- & 0.2137 & 0.1279 & 0.1220 \\\hline
	3 & 0.3954 & 0.3781 & 0.3560 & 0.3480 \\\hline
	4 & 0.3955 & 0.3726 & 0.3723 & 0.3595 \\\hline
	5 & 0.3542 & 0.3342 & 0.3560 & 0.3429 \\\hline
	6 & 0.3409 & 0.3258 & 0.3293 & 0.3195 \\\hline
	7 & 0.3121 & 0.3046 & 0.3125 & 0.3070 \\\hline
	8 & 0.2959 & 0.2891 & 0.2946 & 0.2861 \\\hline
	9 & 0.2789 & 0.2723 & 0.2788 & 0.2714 \\\hline
	10 & 0.2631 & 0.2595 & 0.2620 & 0.2556 \\\hline
\end{tabular}
    \caption{Mean  (base-2) exponential mixing rates.}
    % for each flow type and each $\tau$ (in the FSRT and RSRT cases, the individual flow times are i.i.d.~random variables with uniform distribution on $\{\tau-1,\tau,\tau+1\}$).}
    \label{main_data}
\end{figure}





For flow types FSFT and RSFT, the largest mixing rates were obtained for $\tau=3$ and $\tau=4$; those for $\tau\ge 5$ are noticeably lower, and steadily decrease as $\tau$ grows.  This phenomenon can be explained by noticing that mixing scales for time-independent shear flows decrease no faster than $O(t^{-1})$, and exponential mixing therefore results from alternation of horizontal and vertical flows.  This alternation cannot be too fast (as the cases $\tau=1,2$ show), but once each individual flow acts for a long enough time (which our simulations suggest to be 3 or 4), switching the flow direction becomes more beneficial for fast mixing than keeping it.  The reason for smaller mixing rates in the cases FSRT and RSRT with $\tau=3$ vs.~$\tau=4$ is the fact that the set  $\{\tau-1,\tau,\tau+1\}$, from which we randomly chose the flow times, contains the less-conducive-to-mixing value 2.  
Moreover, the best mixing was obtained by FSFT flows with $\tau=3,4$, which are time-periodic and hence most convenient in potential applications.


Randomizing the phase shift each time the direction of the flow switches seems to have a (slight but consistent) {\it negative effect} on the mixing rates for all $\tau\ge 3$, so mixing in these cases appears to be solely a result of stretching by the individual wedge flows.  Note that this effect is smaller for larger $\tau$, but in those cases there were also fewer additional random choices made during each run (besides the initial randomly chosen vertical and horizontal phase shifts).  While it need not be surprising that this randomness does not increase mixing, the strictly lower mixing rates in the RSXT cases vs.~their FSXT counterparts are unexpected.  We do not know what is the underlying reason, and further study of this phenomenon could be of interest.

We note that this could also suggest that the phase shift randomization proposed by Pierrehumbert in \cite{Pie,Pie2} might have a positive effect on mixing only when the flow direction switching is too frequent.  Moreover, in that case it might be more beneficial to increase flow times rather than randomize the phases, which could also be more easily implemented in real world situations.  Nevertheless, the sine flows in \cite{Pie,Pie2} have a different geometry from our wedge flows due to decreased stretching near the lines where their velocities are extremal (and hence their derivatives vanish), so simulations with these flows will be needed to determine whether the above conclusions also apply in this case.

The effect of randomizing the flow times is more difficult to discern.  Comparing the corresponding XSRT and XSFT cases suggests that it is negligible for $\tau\ge 7$, but this may not be unexpected since the variation in the flow times is small relative to the mean flow time $\tau$.  For $\tau=4,6$, one can observe some decrease of mixing rates when the flow times are randomized, but this reverses for $\tau=5$ 
(the XSRT cases with $\tau=3$ are again special because $2\in \{\tau-1,\tau,\tau+1\}$).  However, in the cases $\tau=4,5$ these differences can be explained at least in part by noticeably slower mixing when the (fixed) flow time is $\tau=5$ vs.~$\tau=4$, meaning that randomizing the flow time should slow down mixing for $\tau=4$ but speed it up for $\tau=5$.  In fact, in order to exclude  effects of both values 2 and 5, we also made 100 runs of each of the XSRT cases with flow times uniformly distributed in $\{3,4\}$; the obtained average mixing rates were much closer to the corresponding XSFT cases with $\tau=3,4$, namely $0.3905$ in the FSRT case and $0.3778$ in the RSRT case. Hence no clear pattern seems to emerge here.


%With all this in mind, the above data suggest that the most efficient mixing by the alternating wedge flows considered here is obtained when each individual wedge flow acts for time 3 or 4 (with possibly 3 being the optimal value), while phase shifts are kept constant.
%%marginally better).  






\begin{figure}[htbp]
\centering
\begin{tabular}{|c|c|c|c|c|}
	\hline
	$\tau$ & FSFT & RSFT & FSRT & RSRT \\\hline\hline
%	1 & & & & \\\hline
	2 & --- & 0.0234 & 0.0569 & 0.0508 \\\hline
	3 & 0.0115 & 0.0168 & 0.0301 & 0.0250 \\\hline
	4 & 0.0136 & 0.0170 & 0.0144 & 0.0175 \\\hline
	5 & 0.0096 & 0.0161 & 0.0189 & 0.0193 \\\hline
	6 & 0.0105 & 0.0135 & 0.0146 & 0.0155 \\\hline
	7 & 0.0105 & 0.0113 & 0.0135 & 0.0168 \\\hline
	8 & 0.0102 & 0.0101 & 0.0126 & 0.0149 \\\hline
	9 & 0.0123 & 0.0139 & 0.0124 & 0.0154 \\\hline
	10 & 0.0125 & 0.0129 & 0.0139 & 0.0146 \\\hline
\end{tabular}
\caption{Standard deviations of (base-2) exponential mixing rates.}
% of the 100 runs for each  flow type and each $\tau$ (in the FSRT and RSRT cases, the individual flow times are i.i.d.~random variables with uniform distribution on $\{\tau-1,\tau,\tau+1\}$).}
\label{stdev}
\end{figure}

Let us also discuss variations in the data that yielded the average rates in Figure \ref{main_data}, since one may  wonder whether individual runs of our simulation exhibited mixing rates that are close enough to these averages.  
%It is immediate from the formula for the slope of the least squares line that one would obtain the same numbers if for each flow type and each $\tau$, one were to first find the slope of the least squares line for each of the 100 runs (with $T_2$ and $T_{10}$ being the same ``averaged'' times from Subsection \ref{S2.3} for all the 100 runs) and then compute the mean of these slopes.  
Figure \ref{stdev} shows the standard deviations of the 100 individual base-2 mixing rates in each case, demonstrating that the averages in Figure \ref{main_data} are also good approximations of the mixing rates of most individual runs.
Indeed, the standard deviations were no more than $0.0193$ in all cases with $\tau\ge 3$ except for the cases FSRT and RSRT with $\tau=3$, where the variation of mixing rates was  magnified due to  $2\in \{\tau-1,\tau,\tau+1\}$.  For instance, the largest difference between the mixing rate for an individual run with $\tau\ge 3$ and the  average mixing rate in the corresponding flow case  was $0.1020$ in the FSRT case with $\tau=3$; that run included a number of consecutive flow times 2 (and fixed phase shifts), a setup that leads to much slower mixing (see the next subsection).  
%The computed mixing rate for this particular run was $0.2555$, while the average value in this case was  $0.3575$.
%, with the next largest difference from average being $???$.

%We also note that smallness of the standard deviations of mixing rates and closeness of the R-squared values imply that most individual runs also agreed with each other quite well.  In fact,  the largest difference of binary logarithms of mixing scales for two distinct runs of the same type (same flow case and $\tau$) and at the same time was at most 3 in all cases with $\tau\ge 3$ except for the cases FSRT and RSRT with $\tau=3$, where it was 4 (and it was 1 or 2 in all the XSFT cases with $\tau\ge 3$).

Finally, for completeness, we list the means and standard deviations of $T_{10}$ in Figure \ref{Tvalues}.
%, and note that the R-squared values of the 3200 individual least squares lines in the 32 cases with $\tau\ge 3$ were all at least $0.9301$ (the one achieving this value was from the exceptional run in the FSRT case with $\tau=3$, mentioned above).


%    \begin{subfigure}[t]{0.24\textwidth}
%        \centering
%         \begin{tabular}{|c|c|}
%    \hline
%         $\tau$ & Slope \\
%         \hline
%         \hline
%         2 &  0 \\
%         \hline
%         3  & -0.3365 \\
%         \hline
%         4 & -0.3186 \\
%         \hline
%         5  & -0.2956 \\
%         \hline
%         6 & -0.2864 \\
%         \hline
%         7  & -0.2633 \\
%         \hline
%         8 & -0.2573 \\
%         \hline
%         9 & -0.2438 \\
%         \hline
%         10 & -0.2252 \\ 
%         \hline
%    \end{tabular}
%        \caption{Fixed Shift}
%    \label{data_fixed}
%    \end{subfigure}



\begin{figure}[htbp]
\centering
\begin{subfigure}{0.5\textwidth}
\centering
\begin{tabular}{|c|c|c|c|c|}
	\hline
	$\tau$ & FSFT & RSFT & FSRT & RSRT \\\hline\hline
%	1 & & & & \\\hline
	2 & --- & 48.13 & 72.73 & 75.86 \\\hline
	3 & 27.58 & 28.88 & 30.09 & 30.80 \\\hline
	4 & 26.73 & 28.00 & 28.26 & 28.77 \\\hline
	5 & 28.22 & 29.48 & 28.56 & 29.46 \\\hline
	6 & 29.41 & 30.46 & 30.52 & 31.07 \\\hline
	7 & 31.28 & 32.00 & 31.79 & 31.86 \\\hline
	8 & 33.89 & 34.23 & 34.01 & 34.30 \\\hline
	9 & 34.90 & 35.93 & 34.92 & 35.92 \\\hline
	10 & 35.73 & 36.49 & 36.12 & 36.88 \\\hline
\end{tabular}
\end{subfigure}%
\begin{subfigure}{0.5\textwidth}
\centering
\begin{tabular}{|c|c|c|c|c|}
	\hline
	$\tau$ & FSFT & RSFT & FSRT & RSRT \\\hline\hline
%	1 & & & & \\\hline
	2 & --- & 3.67 & 18.95 & 17.66 \\\hline
	3 & 0.64 & 0.83 & 2.17 & 1.87 \\\hline
	4 & 0.49 & 1.11 & 0.84 & 1.05 \\\hline
	5 & 1.05 & 1.08 & 1.13 & 1.08 \\\hline
	6 & 1.54 & 1.18 & 1.37 & 1.30 \\\hline
	7 & 1.19 & 1.03 & 1.11 & 1.21 \\\hline
	8 & 0.65 & 0.45 & 1.04 & 1.21 \\\hline
	9 & 1.47 & 1.57 & 1.19 & 1.55 \\\hline
	10 & 1.32 & 1.66 & 1.49 & 1.68 \\\hline
\end{tabular}
\end{subfigure}

\caption{Means (left) and standard deviations (right) of $T_{10}$.}
% for each flow type and each $\tau$.}
\label{Tvalues}
\end{figure}


%We use our algorithm to extract the exponential mixing constants of four classes of
%flows, FSFT, RSFT, FSRT, RSRT, previously described (see Section \ref{S2.2}).
%Comparisons of these results can then be made to determine which parameters lead to
%the highest mixing rates.
%
%The first two classes of flows we consider involve fixed and random shifts under
%integer switching times, FSFT and RSFT.  For each random shift case for a switching time
%of interest, we ran $100$ different mixing simulations, each using a shift chosen
%uniformly randomly from $D_N$, then computed the mixing scales of each at each time
%step, then averaged them
%among all the simulations at each time step, and finally computed the mixing rate as
%the slope of the least squares line that best fits the log of this data against time,
%just as illustrated in Figure \ref{plot_example}.  

%The application of our approach to the 
%cases FSFT and RSFT
%%two classes 
%produced the tables of Figure \ref{main_data}(a) and \ref{main_data}(b) and
%two of the curves of \ref{mixing_rate_plot}.
%R-Squared values are not included in the tables, as they were always above $0.98$,
%and in fact even closer to $1$ for larger switching times, indicating an excellent
%fitting least squares line and accurate computed mixing rates.  
%Note, though
%theoretically integer switching times can start from $1$, results were often
%inconsistent for switching times of $1$ and $2$.  We believe this is in part due to
%structures that persist throughout the flow mainly for these small switching times
%that are a result of the geometry of the flow, preventing mixing (see Section
%\ref{section3.2} for discussion).  Thus we may not include these spurious results in
%the tables shown.

%The last two classes of flows we consider involve fixed and random shifts, as before,
%but with random integer switching times, instead of fixed chosen integers, in ranges
%of size $2$, namely FSRT and RSRT.  Thus, for example, a switching time of $1-3$
%means the switching time was chosen uniformly randomly
%among $\{1,2,3\}$.  Again with $100$ simulations, results are as before processed to
%extract mixing rates and presented in Figure \ref{main_data}(c) and \ref{main_data}(d)
%as well as the remaining two curves of \ref{mixing_rate_plot}.  R-Squared values also
%remain as close to $1$ as before, indicating accurate mixing rate calculations.

%In analyzing these tables and plots, we note that optimal mixing rates for each are
%typically between switching times of $3$ and $4$, with discretely $3$ for FSFT, $4$
%for RSFT, and $3-5$ for both FSRT and RSRT.  In comparison between fixed and
%random shifts, both the cases involving random shifts have improved, slightly higher
%mixing rates over almost all switching times.  In comparison between fixed and random
%switching times, the results are mostly similar, so the effect of random switching
%times neither improved nor impaired mixing rates.  Note, in comparison with fixed shift
%and fixed switching time of $2$ with random switching time of $1-3$, the randomness
%improved results, we believe, by destroying the geometric structures that remain
%preserved in the flow in the fixed switching time case.

%Thus, we see in our experiments on different parameters of shifts and switching times,
%exponential mixing occurred in all cases when switching times were not too small.
%Furthermore, our results allow for comparison of mixing rates, where switching times
%for optimal mixing rates can be identified and the effects of random shifts and
%switching times studied.


%
%\begin{figure}
%    \centering
%    \includegraphics[scale = 0.75]{Images/NewMixingRateGraph.eps}
%    \caption{A plot of mixing rate versus switching time. For the random $\tau$ data, we plot the average $\tau$ value. For example, random time switching 2-4 corresponds to $\tau = 3$ on the x-axis.}
%    \label{mixing_rate_plot}
%\end{figure}

\subsection{Fixed Structures and Grid Scale Effects}
\label{S3.2}

Let us now turn to flow times $\tau=1,2$, starting with $\tau=2$.  The tables above are all missing data in the case FSFT with $\tau=2$, which is because one can easily show that these flows do not produce exponential mixing (hence we did not run our simulations in this case).  The segments $\{(s,s+\frac 34)\,|\,s\in (0,\frac 14)\}$ and $\{(s,s+\frac 14)\,|\,s\in (\frac 14,\frac 12)\}$ are fixed by $(\calV_{0}^{2} \circ \calH_{0}^{2})^2$, and mapped onto each other by $\calV_{0}^{2} \circ \calH_{0}^{2}$.  The same is true about the segments $\{(s,\frac 34-s)\,|\,s\in (\frac 12,\frac 34)\}$ and $\{(s,\frac 54-s)\,|\,s\in (\frac 34,1)\}$.  Obviously, $(\omega',\omega)$-shifts of these segments have the same relationship to $\calV_{\omega'}^{2} \circ \calH_{\omega}^{2}$.  Moreover, locally at any point on these segments, $(\calV_{\omega'}^{2} \circ \calH_{\omega}^{2})^2$ is represented by the matrix 
\[
\begin{bmatrix}
            -3 & 4 \\
            -4 & 5 \\
\end{bmatrix},
\]
which is similar to the $2\times 2$ Jordan block with diagonal elements $1$ (the eigenspace for this eigenvalue is generated by $(1,1)$).  It is easy to see from this that one can  at best hope for algebraic-in-time mixing in this case.  Figure \ref{mixing_example} in fact shows how mixing is inhibited near the above segments, as well as that it is faster elsewhere in the domain.

These structures of course do not survive the randomization in the RSFT, FSRT, and RSRT cases with $\tau=2$, but they show that flow time $\tau=2$ does not provide enough stretching and layering for the most efficient mixing.  This is compounded by flow times 1 occurring in the FSRT and RSRT cases, as data in Figure \ref{main_data} demonstrates (while the RSFT entry does show a decent mixing rate in that case, it is still well below the cases with $\tau\ge 3$).

The cases with $\tau=1$ were even slower mixers and we therefore did not perform their full simulations.  In the FSFT case, one can again easily identify a fixed structure of the mapping $\calV_{0}^{1} \circ \calH_{0}^{1}$.  Indeed the segments $\{(s,s+\frac 12)\,|\,s\in (0,\frac 12)\}$, $\{(\frac 12,s)\,|\,s\in (0,\frac 12)\}$, and $\{(s,\frac 12)\,|\,s\in (\frac 12,1)\}$ form a 3-cycle for this mapping and each is fixed by $(\calV_{0}^{1} \circ \calH_{0}^{1})^3$.  The latter mapping is locally represented by matrices
\[
\begin{bmatrix}
            3 & -2 \\
            2 & -1 \\
\end{bmatrix}
\qquad \text{and} \qquad 
\begin{bmatrix}
            -1 & 2 \\
            -2 & 3 \\
\end{bmatrix}
\]
on the two sides of the segment $\{(s,s+\frac 12)\,|\,s\in (0,\frac 12)\}$, and both are again similar to the $2\times 2$ Jordan block with diagonal elements $1$ (the eigenspace in both cases is again generated by $(1,1)$).  Thus there is no exponential mixing in this case either.

%%%When the shift and switching time were held constant, certain structures were preserved by the flow. Two main related structures were observed. The first of these are lines that are periodic under the flow, i.e., after a certain number of iterations, the line gets mapped back into itself). The second is an "unmixing" of the domain. However, this unmixing does not affect the mixing scale calculation, though is still interesting to examine. 
%%%
%%%The first kind of structure is most notable for switching times 1 and 2, since near these fixed lines little mixing occurs. However, these structures still appear at larger time switching values. As an example, consider the sequence of flows $\{H_{1/2}^1,V_{1/2}^1,H_{1/2}^1,V_{1/2}^1,\dots\}$ and the half-line at $x_2 = \frac{1}{2}$ from $x_1 = \frac{1}{2}$ to $x_1 = 1$. After six mappings, this line gets mapped back into itself, as shown in Figure \ref{SwitchingTime1Structures}.
%%%
%%%\begin{figure}[H]
%%%    \centering
%%%    \begin{subfigure}[t]{0.12\textwidth}
%%%        \centering
%%%        \includegraphics[width=\textwidth]{Images/fixline6.jpg}
%%%        \caption{t = 0}
%%%    \end{subfigure}
%%%    \hfill
%%%    \begin{subfigure}[t]{0.12\textwidth}
%%%        \centering
%%%        \includegraphics[width=\textwidth]{Images/fixline1.jpg}  
%%%        \caption{t = 1}
%%%    \end{subfigure}
%%%    \hfill
%%%    \begin{subfigure}[t]{0.12\textwidth}
%%%        \centering
%%%        \includegraphics[width=\textwidth]{Images/fixline2.jpg}
%%%        \caption{t = 2}
%%%    \end{subfigure}
%%%    \hfill
%%%    \begin{subfigure}[t]{0.12\textwidth}
%%%        \centering
%%%        \includegraphics[width=\textwidth]{Images/fixline3.jpg}
%%%        \caption{t = 3}
%%%    \end{subfigure}
%%%    \hfill
%%%    \begin{subfigure}[t]{0.12\textwidth}
%%%        \centering
%%%        \includegraphics[width=\textwidth]{Images/fixline4.jpg}
%%%        \caption{t = 4}
%%%    \end{subfigure}
%%%    \hfill
%%%    \begin{subfigure}[t]{0.12\textwidth}
%%%        \centering
%%%        \includegraphics[width=\textwidth]{Images/fixline5.jpg}
%%%        \caption{t = 5}
%%%    \end{subfigure}
%%%    \hfill
%%%    \begin{subfigure}[t]{0.12\textwidth}
%%%        \centering
%%%        \includegraphics[width=\textwidth]{Images/fixline6.jpg}
%%%        \caption{t = 6}
%%%    \end{subfigure}
%%%    
%%%    \caption{An example of a line that is preserved over six mappings by the fixed shift $\omega = \frac{1}{2}$ and fixed switching time $\tau = 1$ flow.}
%%%    \label{SwitchingTime1Structures}
%%%\end{figure}

\begin{figure}[htbp]
    \centering
%    \begin{subfigure}[t]{0.45\textwidth}
%        \centering
        \includegraphics[scale = 0.55]{FSFT08_mean_x.png}
%       \caption{N = 13, Random Shift, Fixed $\tau = 8$}
%    \end{subfigure}
%    \hfill
%    \begin{subfigure}[t]{0.45\textwidth}
%        \centering
        \includegraphics[scale = 0.55]{RSFT08_mean_x.png}      
 %       \caption{N = 13, Fixed Shift, Fixed $\tau = 8$}
%    \end{subfigure}
%    \hfill
    \caption{Means and standard deviation error bars of the binary logarithms of the mixing scales for the 100 simulations 
    %plotted against time.  This is 
   with $N = 15$ in the FSFT  (left) and RSFT   (right) cases with $\tau=8$, over an extended time interval.}
    \label{kinks}
\end{figure}

Finally, we mention here a curious phenomenon that we observed for the FSFT and RSFT cases with $\tau=8$.  In both cases, after the mixing scale $\sim 2^{-11}$ was reached and the observed exponential decrease stopped, the computed scale rebounded to $\sim 2^{-9}$ for some time (as if some {\it unmixing} were happening there).  This was then followed by a time interval (fairly long one, particularly in the FSFT case) where the mixing scale equaled $2^{-10}$ for {\it each} of the 100 simulations, before it stopped having this surprisingly uniform behavior and settled into a slightly more varied dynamic with values near $2^{-10}$ and $2^{-11}$.  
Figure \ref{kinks} contains means and standard deviation error bars of the binary logarithms $-n_k$ of the mixing scales   in these two flow cases.
We ran our simulations with a larger grid scale as well, and this behavior persisted in that case, albeit clearly at larger mixing scales.



These observations are of course completely irrelevant to the mixing theory, since they involve behavior on time intervals where the simulation cannot anymore capture the mixing dynamic due to the mixing scales being too close to the grid scale $2^{-15}$.  Nevertheless, we did not observe it or something similar for other values of $\tau$ (as well as in the XSRT cases where $\tau$ varies over time), so it may point to some special feature of the discrete grid dynamic for flow time $\tau=8$.  At the same time, the fact that this behavior is observed in the RSFT case as well, where the randomness in phase shifts would destroy any potential special structures present in the FSFT dynamic, is quite curious.  We do not currently have a candidate for the possible reason behind this phenomenon, and do not know why $\tau=8$ is the only flow time out of those we studied for which it occurs.




%%%Discretizing the domain also leads to certain interesting effects that otherwise would not appear in an analytical examination of the mixing scale. Certain flows, both fixed and random shift flows, led to some unmixing in the mixing scale. Note that this is different than the unmixing described above, as this unmixing actually appears in the mixing scale. 
%%%
%%%
%%%
%%%
%%%%In Figure \ref{unmixing_example}(a), at around 40 seconds (5 iterations with $\tau = 8$), the domain begins to unmix itself, and does not generally stabilize until around 150 seconds. 
%%%%\begin{figure}[H]
%%%%    \centering
%%%%        \centering
%%%%        \includegraphics[width=.8\textwidth]{Images/scale15switching8randombump.eps}
%%%%        \caption{N = 15, Random Shift, Fixed $\tau = 8$}
%%%%    \label{unmixing_example_scale15}
%%%%\end{figure}
%%%
%%%However, this unmixing that appears in the mixing scale is only an effect of discretizing the domain. When we increase $N$ from 13 to 15, the same pattern occurs, except at a lower scale, as shown in Figure \ref{unmixing_example_scale15}. Thus, in an analytical calculation of the mixing scale, this phenomenon would not occur.
%%%
%%%
%%%
%%%
%%%
%%%A similar phenomenon happens with switching time $\tau = 8$, where certain lines get mapped back into themselves after a fixed number of mappings, causing the domain to appear "unmixed". For example, after 8 mappings of time switching $\tau = 8$ flow with shift $\omega = \frac{1}{2}$, the vertical line at odd multiples of $\frac{1}{16}$ and the horizontal lines at multiples of $\frac{1}{16}$ are preserved. After 16 mappings, all vertical lines and horizontal lines at multiples of $\frac{1}{16}$ are preserved. See Figure \ref{ts8_fixed_structures} for a visual representation.
%%%
%%%%\begin{figure}[H]
%%%%    \centering
%%%%    \begin{subfigure}[t]{0.4\textwidth}
%%%%        \centering
%%%%        \includegraphics[width=\textwidth]{Images/ts8_8iter.jpg}
%%%%        \caption{After 8 mappings (t = 64)}
%%%%    \end{subfigure}
%%%%    \hfill
%%%%    \begin{subfigure}[t]{0.4\textwidth}
%%%%        \centering
%%%%        \includegraphics[width=\textwidth]{Images/ts8_16iter.jpg}  
%%%%        \caption{After 16 mappings (t = 128)}
%%%%    \end{subfigure}
%%%%    \caption{The switching time 8 flow preserves horizontal and vertical lines at multiples of $\frac{1}{16}$. This is the "unmixing" of the domain. However, this unmixing becomes irrelevant at finer grid scales, and evidence for this will be provided below.}
%%%%    \label{ts8_fixed_structures}
%%%%\end{figure}
%%%
%%%However, we claim that fixed structures for switching time $\tau$ = 8 most likely do not affect the mixing rate. The idea behind the argument is that locally, at points away from the flow's vertex, the map is linear. Thus, locally around a certain point $(x_1, x_2)$, each horizontal and vertical shear map can be written as a $2 \times 2$ triangular matrix. Certain points are mapped back into themselves after a finite number of mappings, and thus we can define a finite sequence of matrices that locally model the flow at that point. Then from the singular value decomposition, the image of a disk under a linear transformation $M$ is an ellipse with major and minor axes of length $\sqrt{\lambda_i}$, where the $\lambda_i$ are the eigenvalues of $MM^T$. Thus, for small $\varepsilon > 0$, $B_{\varepsilon}((x_1, x_2))$ will be mapped into an ellipse of known size. As a result, using the analytical definition of mixing, we can get an upper bound on the mixing rates, since if $B_{\varepsilon}((x_1, x_2))$ is all substance A, the resulting ellipse will be as well, and if $\delta$ is the smaller singular value, then $B_{\delta}((x_1, x_2))$ will be entirely unmixed. The argument uses the analytical definition, but it's not hard to see using a bit of Euclidean geometry that the mixing scale from Definition 2 cannot be much smaller than $\delta$.
%%%
%%%First, we perform the computations for $\tau = 8$. From equations (1.1) and (1.2) with $\tau$ = 1, one can write the following:
%%%\begin{align}
%%%    H_{1/2}^1(x_1, x_2) &= 
%%%        \begin{cases}\begin{bmatrix}
%%%            1 & 1 \\
%%%            0 & 1 \\
%%%        \end{bmatrix}\begin{bmatrix}
%%%        x_1 \\
%%%        x_2 \\
%%%        \end{bmatrix}, &  x_2 \leq \frac{1}{2} \\
%%%        \begin{bmatrix}
%%%            1 & -1 \\
%%%            0 & 1 \\
%%%        \end{bmatrix}\begin{bmatrix}
%%%        x_1 \\
%%%        x_2 \\
%%%        \end{bmatrix}, &  x_2 > \frac{1}{2} \\
%%%        \end{cases}
%%%        \\
%%%    V_{1/2}^1(x_1, x_2) &= 
%%%        \begin{cases}\begin{bmatrix}
%%%            1 & 0 \\
%%%            1 & 1 \\
%%%        \end{bmatrix}\begin{bmatrix}
%%%        x_1 \\
%%%        x_2 \\
%%%        \end{bmatrix}, &  x_1 \leq \frac{1}{2} \\
%%%        \begin{bmatrix}
%%%            1 & 0 \\
%%%            -1 & 1 \\
%%%        \end{bmatrix}\begin{bmatrix}
%%%        x_1 \\
%%%        x_2 \\
%%%        \end{bmatrix}, &  x_1 > \frac{1}{2} \\
%%%        \end{cases}
%%%\end{align}
%%%In order to calculate the corresponding matrix for $H_{1/2}^8$ (or $V_{1/2}^8$), simply take the matrices in (3.1) (or (3.2)) to their eighth powers. Now consider the point $\left(\frac{1}{16}, \frac{1}{4}\right)$. Under fixed shift switching time 8 flows $\{H_{1/2}^81,V_{1/2}^8,H_{1/2}^8,V_{1/2}^8,\dots\}$, it gets mapped into itself after 4 mappings, with the trajectory $\left(\frac{1}{16}, \frac{1}{4}\right) \rightarrow \left(\frac{1}{16}, \frac{1}{4}\right) \rightarrow \left(\frac{1}{16}, \frac{3}{4}\right) \rightarrow \left(\frac{1}{16}, \frac{3}{4}\right) \rightarrow \left(\frac{1}{16}, \frac{1}{4}\right)$. Thus, around $\left(\frac{1}{16}, \frac{1}{4}\right)$, this sequence of transformations is given by 
%%%\begin{equation}
%%%\begin{bmatrix}
%%%            1 & 8 \\
%%%            0 & 1 \\
%%%\end{bmatrix}
%%%\begin{bmatrix}
%%%            1 & 0 \\
%%%            8 & 1 \\
%%%\end{bmatrix}
%%%\begin{bmatrix}
%%%            1 & -8 \\
%%%            0 & 1 \\
%%%\end{bmatrix}
%%%\begin{bmatrix}
%%%            1 & 0 \\
%%%            8 & 1 \\
%%%\end{bmatrix} = 
%%%\begin{bmatrix}
%%%            -63 & -512 \\
%%%            -496 & -4031 \\
%%%\end{bmatrix} = A
%%%\end{equation}
%%%Then $AA^{T} = 
%%%\begin{bmatrix}
%%%            266113 & 2095120 \\
%%%            2095120 & 16494977 \\
%%%\end{bmatrix} $
%%%with eigenvalues of $\lambda_1 = \frac{16761090+\sqrt{16761090^2-4}}{2}$ and $\lambda_2 = \frac{16761090-\sqrt{16761090^2-4}}{2}$. The length of the minor axis is then $\delta =  \left(\frac{16761090-\sqrt{16761090^2-4}}{2}\right)^{1/2}$, so in 32 seconds, the smallest the analytical mixing scale can be is $\delta$. If we assume the mixing to be exponential, i.e., $\lambda(t) = 2^{-\gamma t}$, then setting $\lambda = \delta$ and $t = 32$ gives an exponential mixing constant of $\gamma \approx 0.375$. However as in Figure \ref{data_fixed}, the numerical analysis found the exponential mixing constant for switching time $\tau$ = 8, fixed shift was approximately 0.2807, much less than the calculated upper bound $\gamma$. Hence, it is reasonable to conclude that these fixed points are not bottle-necking the mixing rates of the switching time $\tau = 8$ flows.
%%%
%%%Now, similar calculations can be performed for the $\tau = 1$ fixed shift flows. 
%%%
%%%From \ref{SwitchingTime1Structures}, one can check that locally above the line in \ref{SwitchingTime1Structures}
%%%(a), the sequence of 6 transformations is locally given by the Jordan block matrix 
%%%\begin{equation}
%%%\begin{bmatrix}
%%%            1 & -2 \\
%%%            0 & 1 \\
%%%\end{bmatrix}
%%%\end{equation}
%%% and similarly below the line the Jordan block
%%%\begin{equation}
%%%\begin{bmatrix}
%%%            1 & 2 \\
%%%            0 & 1 \\
%%%\end{bmatrix}    
%%%\end{equation}
%%%locally defines the transformation. Thus, locally, the domain is sheared parallel to the x-axis along the line at $y = 0.5$ from $x = 0.5$ to $x = 1$. This fact shows that these structures actually force a linear mixing rate. To see this, consider a small square of side length $\varepsilon$ at a distance of $\varepsilon$ just above the line, and not at the two endpoints.  This square is then sheared to the left, with it's top edge moving by $4\varepsilon$ and its bottom edge by $2\varepsilon$. However, there are identical squares to the right which are also sheared leftward, and hence the original square is preserved. In order for this chain of preservations to be broken, these squares must be sheared all the way across the fixed line. The bottoms move at a rate of $2\varepsilon$ per unit time, and must travel a distance of $\frac{1}{2}$, resulting in a total time of $\frac{1}{4\varepsilon}$. So in order to roughly decrease the mixing scale by $\varepsilon$, it takes $\frac{1}{4\varepsilon}$ time units, so locally around these fixed lines, mixing is only linear. 

% Consider the point $\left(\frac{3}{4}, \frac{1}{2}\right)$ which follows the trajectory $\left(\frac{3}{4}, \frac{1}{2}\right) \rightarrow \left(\frac{1}{4}, \frac{1}{2}\right) \rightarrow \left(\frac{1}{4}, \frac{3}{4}\right) \rightarrow \left(\frac{1}{2}, \frac{3}{4}\right) \rightarrow \left(\frac{1}{2}, \frac{1}{4}\right) \rightarrow \left(\frac{3}{4}, \frac{1}{4}\right) \rightarrow \left(\frac{3}{4}, \frac{1}{2}\right)$, and thus locally the sequence of transformations is modelled by 
% \begin{equation}
% \begin{bmatrix}
%             1 & 0 \\
%             -1 & 1 \\
% \end{bmatrix}
% \begin{bmatrix}
%             1 & 1 \\
%             0 & 1 \\
% \end{bmatrix}
% \begin{bmatrix}
%             1 & 0 \\
%             1 & 1 \\
% \end{bmatrix}
% \begin{bmatrix}
%             1 & -1 \\
%             0 & 1 \\
% \end{bmatrix}
% \begin{bmatrix}
%             1 & 0 \\
%             1 & 1 \\
% \end{bmatrix}
% \begin{bmatrix}
%             1 & 1 \\
%             0 & 1 \\
% \end{bmatrix}  = 
% \begin{bmatrix}
%             1 & 0 \\
%             0 & 1 \\
% \end{bmatrix}
% \end{equation}
% Thus, locally the domain is preserved around the point $\left(\frac{3}{4}, \frac{1}{2}\right) $ after 6 mappings of the flow. This provides a very strong upper bound on the mixing scale and thus is most likely the reason that the fixed shift time switching 1 flows mixed poorly. 

% \brem
% The point $\left(\frac{3}{4}, \frac{1}{2}\right)$ lies on the vertex of the flow twice during it's trajectory; once at the beginning and then again at the point $\left(\frac{1}{2}, \frac{3}{4}\right)$. The flow is not linear at these locations, but the flow is continuous, so locally it can be approximated well by either of the matrices in (3.1) at $\left(\frac{3}{4}, \frac{1}{2}\right)$, and either of those in (3.2)  at $\left(\frac{1}{2}, \frac{3}{4}\right)$. Other choices of said matrices result in a minor axis length of 0.414, which over 6 seconds bounds the mixing rate at 0.2119. However, if one examines a point $(x_1, x_2)$, for some small $\varepsilon > 0$, in $\left(\frac{3}{4} -\varepsilon, \frac{3}{4}\right) \times \left(\frac{1}{2}, \frac{1}{2} + \varepsilon \right)$, the sequence in (3.4) holds, with the flow being linear locally around $(x_1, x_2)$. The calculations above remain mostly accurate, although the point will not be mapped back into itself. 
% \erem


%
%\section{Conclusion and Future Directions}
%
%To conclude, we found substantial evidence that most flows we tested mixed exponentially in time. Certain flows have fixed structures that result from their inherent geometry. For larger switching time values, these structures are irrelevant, and exponential mixing in time occurs. However, for smaller switching times such as for $\tau = 1$, the structures actually induce linear mixing in time. Furthermore, we found that the optimal mixers occur with a switching time $\tau$ between 3 and 4, and randomizing the shift $\omega$ and the switching time resulted in slightly worse exponential mixing constants than fixing $\omega$ and $\tau$. 

%Statement on Lagrangian vs Eulerian methods
%Parallel Computing
%Mixing at the walls, i.e. not on a Torus
%Smooth flows
%Finer Scales
%Analytical Results
%

%\section{Acknowledgements}



%\bibliographystyle{ieeetr}
%\bibliography{ReferenceLibrary}

%%%%%%%%%%%%%%%%%%%%%%%%%%%%%%%%%%%%%%%%%%%%%%
\begin{thebibliography}{99}
%%%%%%%%%%%%%%%%%%%%%%%%%%%%%%%%%%%%%%%%%%%%%%
\bibitem{ACM} 
G. Alberti, G. Crippa, and A.L. Mazzucato, 
{\it Exponential self-similar mixing and loss of regularity for continuity equations},
C. R. Acad. Sci. Paris, Ser. I {\bf 352} (2014), 901--906.

\bibitem{ACM2} 
G. Alberti, G. Crippa, and A.L. Mazzucato,
{\it Exponential self-similar mixing by incompressible flows}, 
J. Amer. Math. Soc.  {\bf 32} (2019), 445--490.

%%\bibitem{ACM3} G. Alberti, G. Crippa, and A.L. Mazzucato. 
%%Loss of regularity for the continuity equation with non-Lipschitz velocity field. \emph{Preprint,} arXiv:1802.02081.

%\bibitem{Anosov} D. V. Anosov. Geodesic flows on closed Riemannian manifolds with negative curvature. 
%\emph{Proc. Steklov Inst.},   90, 1967.



\bibitem{BBP} J. Bedrossian, A. Blumenthal, and S. Punshon-Smith, 
%Lagrangian chaos and scalar advection in stochastic fluid mechanics. \emph{Preprint,} arXiv:1809.06484.
{\it Almost-sure exponential mixing of passive scalars by the stochastic Navier-Stokes equations},
Ann. Probab., to appear.

\bibitem{BOS} 
Y. Brenier, F. Otto, and C. Seis, 
{\it Upper bounds on the coarsening rates in demixing binary viscous fluids},
SIAM J. Math. Anal. {\bf 43} (2011), 114--134.

\bibitem{Bressan} 
A. Bressan,
{\it A lemma and a conjecture on the cost of rearrangements},
Rend. Sem. Mat. Univ. Padova {\bf 110} (2003), 97--102. 

\bibitem{B2} A. Bressan. Prize offered for the solution of a problem on mixing flows. http://www.math.psu.edu/bressan/PSPDF/prize1.pdf, 2006.

%\bibitem{CKRZ} P. Constantin, A. Kiselev, L. Ryzhik, and A. Zlato\v{s}. Diffusion and mixing in fluid flow. \emph{Ann.
%of Math. (2)}, 168(2):643--674, 2008.

%\bibitem{CFS} I.P. Cornfeld, S.V. Fomin, and Y.G. Sinai. \emph{Ergodic Theory}. {Springer-Verlag,} New York, 1982.

%\bibitem{CZDE} M. Coti-Zelati, M.G. Delgadino, and T.M. Elgindi. On the relation between enhanced dissipation time-scales and mixing rates. \emph{Comm. Pure Appl. Math.}, to appear, arXiv:1806.03258.

\bibitem{CL} 
G. Crippa and C. De Lellis, 
{\it Estimates and regularity results for the DiPerna-Lions flow},
J. Reine Angew. Math. {\bf 616} (2008), 15--46.

\bibitem{CLS} 
G. Crippa, R. Luc\'{a}, and C. Schulze,
{\it  Polynomial mixing under a certain stationary Euler flow},
Phys. D {\bf 394} (2019), 44--55.

%%
%%\bibitem{CS} G. Crippa and C. Schulze. Cellular mixing with bounded palenstrophy. \emph{Preprint,} arXiv:1707.01352.
%%
%%\bibitem{DePauw} 
%%N. Depauw.
%%Non unicit\' e des solutions born\' ees pour un champ de vecteurs BV en dehors d'un hyperplan.  \emph{C. R. Math. Acad. Sci. Paris}, 337(4):249--252, 2003.

%\bibitem{DT} C. R. Doering and J.-L. Thiffeault. Multiscale mixing efficiencies for steady sources. \emph{Phys. Rev. E}, 74 (2), 025301(R), August 2006. 

%\bibitem{Dolgopyat} D. Dolgopyat. On decay of correlations in Anosov flows. 
%\emph{Ann. of Math.}  {\bf 147} (1998), 357-390.
%
%\bibitem{Dolgopyat2} D. Dolgopyat. Personal communication. 
%
%\bibitem{DKK} D. Dolgopyat, V. Kaloshin, and L. Koralov. Sample path properties of the stochastic flows. \emph{Ann. of Prob.} {\bf 32} (2004), 1-27. 
%
%\bibitem{IF} Y. Feng and G. Iyer. Dissipation Enhancement by Mixing. \emph{Preprint,} arXiv:1806.03699.

\bibitem{ElgZla} 
T.M. Elgindi and A. Zlato\v s, 
{\it Universal mixers in all dimensions}, 
Adv. Math. {\bf 356} (2019), 106807, 33 pp. 

\bibitem{IKX} 
G. Iyer, A. Kiselev, and X. Xu, 
{\it Lower bounds on the mix norm of passive scalars advected by incompressible enstrophy-constrained flows},
Nonlinearity {\bf 27} (2014), 973--985.

%\bibitem{K} A. Katok. Bernoulli diffeomorphisms on surfaces. \emph{Ann. of Math.} {\bf 110} (1979), 529-547. 

\bibitem{LTD} 
Z. Lin, J. L. Thiffeault, and C. R. Doering,
{\it Optimal stirring strategies for passive scalar mixing},
J. Fluid Mech. {\bf 675}, 465--476.

%\bibitem{Liverani} C. Liverani. On contact Anosov flows. 
%\emph{Ann. of Math.}  {\bf 159} (2004), 1275-1312.

\bibitem{LLNMD} 
E. Lunasin, Z. Lin, A. Novikov, A. Mazzucato, and C. R. Doering, 
{\it Optimal mixing and optimal stirring for fixed energy, fixed power, or fixed palenstrophy flows},
J. Math. Phys. {\bf 53} (2012), 115611, 15pp.

%\bibitem{MMP} G. Mathew, I. Mezi\'c, and L. Petzold. A multiscale measure for mixing. \emph{Physica D}, 211(1-2):23--46, 2005.

%\bibitem{M} V. Maz'ya, \emph{Sobolev Spaces. With Applications to Elliptic Partial Differential Equations}, 2nd, revised and augmented ed., Springer, Berlin, 2011.
%
%\bibitem{NPV} E. Di Nezza, G. Palatucci, and E. Valdinoci. Hitchhiker's guide to the fractional Sobolev spaces, \emph{Bull. Sci. math.}, 136(5):521--573, 2012.

\bibitem{OSS} 
F. Otto, C. Seis, and D. Slep\v{c}ev,
{\it Crossover of the coarsening rates in demixing of binary viscous liquids},
Commun. Math. Sci. {\bf 11} (2013), 441--464.

\bibitem{Pie} 
R. Pierrehumbert, 
\it Tracer microstructure in the large-eddy dominated regime, 
\rm Chaos, Solitons \& Fractals {\bf 4} (1994), 1091--1110.

\bibitem{Pie2} 
R. Pierrehumbert, 
\it Lattice models of advection-diffusion, 
\rm Chaos {\bf 10} (2000), 61--74.

\bibitem{S} 
C. Seis,
{\it  Maximal mixing by incompressible fluid flows},
Nonlinearity {\bf 26} (2013), 3279--3289.

\bibitem{S2} 
D. Slep\v{c}ev,
{\it Coarsening in nonlocal interfacial systems}, 
SIAM J. Math. Anal. {\bf 40} (2008), 1029--1048.

%\bibitem{STD} T. A. Shaw, J.-L. Thiffeault, and C. R. Doering. Stirring up trouble: multi-scale mixing measures for steady scalar sources. \emph{Phys. D}, 231(2):143--164, 2007.

%\bibitem{SpringhamThesis} J. Springham. Ergodic properties of linked-twist maps. \emph{PhD Thesis,} University of Bristol, 2008.

%\bibitem{SS} E. M. Stein and R. Shakarchi. Real Analysis. \emph{Princeton University Press}, 2005.

\bibitem{T} 
J.-L. Thiffeault,
{\it Using multiscale norms to quantify mixing and transport},
Nonlinearity {\bf 25} (2012), 1--44.

%\bibitem{TDG} J.-L. Thiffeault, C. R. Doering, and J. D. Gibbon. A bound on mixing efficiency for the advection-diffusion equation. \emph{J. Fluid Mech.}, 521:105--114, 2004.

\bibitem{YaoZla} 
Y. Yao and A. Zlato\v{s},
{\it  Mixing and un-mixing by incompressible flows},
J. Eur. Math. Soc. {\bf 19} (2017), 1911--1948. 

%\bibitem{Zill} C. Zillinger. On geometric and analytic mixing scales: comparability and convergence rates for transport problems. \emph{Preprint,} arXiv:1804.11299.
%%%%%%%%%%%%%%%%%%%%%%%%%%%%%%%%%%%%%%%%%%%%%%

\end{thebibliography}
%%%%%%%%%%%%%%%%%%%%%%%%%%%%%%%%%%%%%%%%%%%%%%

\Addresses

\end{document}
