%!TEX root = Zjets_Angular.tex

\section{Numerical predictions} 
\label{sec3}

In this section, we provide a comparison of the predictions for a set of angular coefficients
to the available ATLAS and CMS data in $\Pp\Pp$ collisions at $\sqrt{s} = 8~\TeV$. 
While the LHCb collaboration has not yet performed a measurement of the
angular coefficients, previous measurements of the $\PZ$-boson $p_\rT$ 
spectrum~\cite{Aaij:2015gna,Aaij:2015zlq,Aaij:2016mgv}
and forward--backward asymmetry~\cite{Aaij:2015lka} indicate that there is potential for such
a measurement in the forward region. We therefore also provide predictions
in the LHCb fiducial region at $\sqrt{s} = 8$~TeV. In all cases (ATLAS, CMS, LHCb), the angular coefficients 
are defined in the Collins--Soper reference frame~\cite{Collins:1977iv}.
%
In addition to the angular coefficients, we also provide absolute predictions for the unpolarised $\ptz$ distributions.%
\footnote{As compared to the results shown in Ref.~\cite{Ridder:2016nkl}, the kinematic setup differs slightly 
both for the ATLAS and CMS measurements and the theory uncertainty includes a seven-point scale variation.}
Special attention is also given to the difference $(A_0-A_2)$, 
where the quality of theoretical description with respect to the observed distributions is 
quantified by means of a $\chi^2$ test.
%
Furthermore, we also present a comparison to data for the new observable $\Delta^{\rm LT}$.

The measurements of the angular coefficients are performed differentially in \ptz and for various rapidity 
intervals, where in all cases an invariant-mass window for the lepton-pair final state is imposed around 
the $\PZ$-boson resonance. 
%
For non-vanishing values of \ptz, the LO prediction for this distribution can be obtained
from the $\cO(\alphas)$ tree-level $\PZ+\jet$ process, where the transverse 
momentum of the $\PZ$ boson is balanced with that of a single final-state QCD parton. 
%
The NLO QCD and EW corrections to this process have been computed in Refs.~\cite{Giele:1993dj,Denner:2011vu}, and
more recently the NNLO QCD corrections to this process have been completed~\cite{Ridder:2015dxa,Boughezal:2015ded}.
%
In this work, we employ the calculation of Ref.~\cite{Ridder:2015dxa} based on the antenna
subtraction formalism~\cite{GehrmannDeRidder:2005cm,GehrmannDeRidder:2005aw,GehrmannDeRidder:2005hi,Daleo:2006xa,Daleo:2009yj,Boughezal:2010mc,Gehrmann:2011wi,GehrmannDeRidder:2012ja,Currie:2013vh}
to provide NNLO-accurate QCD predictions for the \ptz distributions of the angular coefficients.
This process is implemented in the flexible parton-level Monte Carlo generator \textsc{NNLOjet}.

The predictions are provided in the $G_\mu$-scheme, where we take the following
set of numerical inputs: $M_{\PZ}^\mathrm{os} = 91.1876~\GeV$, $\Gamma_{\PZ}^\mathrm{os} = 2.4952~\GeV$, 
$M_{\PW}^\mathrm{os} = 80.385~\GeV$,  $\Gamma_{\PW}^\mathrm{os} = 2.085~\GeV,$ 
and $G_\mu = 1.16638 \cdot 10^{-5}~\GeV^{-2}$. In the extraction of the corresponding numerical values 
for $\alpha$ and $\sw^2$, 
we additionally include the dominant one- and two-loop universal corrections to the $\rho$-parameter~\cite{Fleischer:1993ub}
which relate $M_{\PW}-M_{\PZ}$ interdependence present beyond tree-level. Including these 
contributions leads to the effective values of $\alpha_\mathrm{eff.} = 0.007779$, $s_{\mathrm{w,eff.}}^2 = 0.2293$.

As a baseline PDF set, we use the central member of \verb|PDF4LHC15_nnlo_30|~\cite{Butterworth:2015oua,
Dulat:2015mca,Harland-Lang:2014zoa,Ball:2014uwa,Gao:2013bia,Carrazza:2015aoa}, and extract $\alphas$ from the grid provided with the PDF set---corresponding to $\alphas(Q=M_{\PZ}^\mathrm{os}) = 0.118$.

As discussed in Section~\ref{sec2}, the theoretical predictions for the coefficients $A_i$ can be obtained by computing the normalised expectation values of the spherical harmonics according to Eq.~\eqref{eq:proj}. To assess the theoretical uncertainty in the extraction of the coefficients through this method, various solutions are possible. In this paper, for comparison with LHC data, we choose to perform an independent variation of factorisation ($\muf$) and renormalization ($\mur$) scales in both the numerator and denominator of this expression.
%
The scales $\mur^{\rm{num.}}, \muf^{\rm{num.}}, \mur^{\rm{den.}},$ and $\muf^{\rm{den.}}$ are each independently varied by factors of $\tfrac{1}{2}$ and $2$ about the transverse energy of the lepton pair,
\begin{equation}
  \mu_0 \equiv E_{\rT,\PZ} = \sqrt{m_{\Pl\Pl}^2 + p_{\rT,\Pl\Pl}^2} \,
\end{equation}
with the constraint that all pairs of these \textbf{uncorrelated} scales satisfy $\tfrac{1}{2} \leq \mu^{i}_{a}/\mu^{j}_{b} \leq 2$.
In total this corresponds to 31 possible combinations and the associated uncertainty is obtained as 
the envelope around the central scale $\mur^{\rm{num.}} = \muf^{\rm{num.}} = \mur^{\rm{den.}} = \muf^{\rm{den.}}  = \mu_0$.

An alternative approach is to \textbf{correlate} the scale uncertainties between numerator and denominator. However, this
treatment can lead to an underestimation of the uncertainty due to missing higher-order effects. For example, 
at LO the renormalisation scale dependence is fully encapsulated in the strong coupling $\alpha_s(\mu)$ which entirely cancels
if the scales between numerator and denominator are \textbf{correlated}.

To further demonstrate this point, we show the impact of these two different approaches in Fig.~\ref{fig:A2_Scales} where the \ptz distribution for $A_2$ is evaluated at NLO and NNLO. The distributions are obtained with an invariant-mass cut of $80 < m_{\Pl\Pl} < 100~\GeV$ on the lepton-pair final state and inclusively with respect to rapidity of the lepton pair.
%------------------------------------------------                                                                                  
\begin{figure}
\centering
\includegraphics[width=.49\linewidth]{Figures/A2_NLO_ScaleCorrelation_Ratio.pdf} \hfill
\includegraphics[width=.49\linewidth]{Figures/A2_NNLO_ScaleCorrelation_Ratio.pdf} 
\caption{The \ptz distribution for the angular coefficient $A_2$ in $\Pp\Pp$ collisions at $\sqrt{s} = 8~\TeV$.
The uncertainty obtained when choosing to (un)correlate the scale choices in the extraction of $A_2$ is shown at 
NLO (left) and NNLO (right). In the lower panel, each distribution is shown normalised with respect
to the central NLO prediction.
}
\label{fig:A2_Scales}
\end{figure}
%------------------------------------------------
It is clearly seen that at NLO, these two prescriptions result in substantial differences in the scale uncertainty bands, with the correlated approach yielding considerably smaller uncertainty bands for \ptz above $20 ~\GeV$. At NNLO however, the scale uncertainty estimate of the $A_i$ coefficients obtained with either choice gives similar results in the low-pt region ($\ptz < 80$~GeV) where the NNLO effects are largest. Throughout this work, all distributions are obtained with the \textbf{uncorrelated} prescription discussed above.

It is worth commenting that the angular coefficients are evaluated differentially in \ptz
and in multiple kinematic regions, corresponding to the various experimental setups. 
In addition, each of these coefficients are computed through the projectors in Eq.~\eqref{eq:Ai_proj}, 
which are highly oscillating functions with respect to the leptonic kinematics, and consequently 
their stable numerical evaluation is rather challenging. This is particularly true for 
the difference $(A_0-A_2)$ for which large non-local cancellations
occur at the level of the angular coefficient as opposed to the integrand.

\subsection{Comparison to ATLAS data}
The ATLAS measurements have been performed with an invariant-mass cut of $80 < m_{\Pl\Pl} < 100~\GeV$
on the lepton-pair final state, and distributions for the angular coefficients in \ptz were extracted
for a range of different accessible rapidity regions. The data has also been presented integrated in \yz, 
obtained after performing an extrapolation to the full phase-space region. The comparison of the 
theoretical predictions is performed with respect to this data, referred to as \yz inclusive, for which 
the measurements are most precise.
%
The predictions of the various \ptz distributions are provided for the kinematic range of 
$\ptz \in [11.4,\,600]~\GeV$, and compared to the available data in this region. Furthermore, in the 
region of \ptz $< 85.4~\GeV$ we provide our predictions with a coarser choice of \ptz bins
with respect to the data, which are obtained by pair-wise combining neighbouring bins.
%
Before continuing, it is also important to highlight that we perform the comparison to the 
ATLAS data which is obtained prior to the regularisation procedure outlined as part of the experimental 
analysis---more detail can be found in Appendix~C of~\cite{Aad:2016izn}.
Our motivations for doing so are as follows. Firstly, the regularisation procedure introduces large
bin-to-bin correlations for the distributions of $A_i$ coefficients meaning that a visual comparison
to the regularised data can be misleading as large correlations are hidden from view. Secondly,
we wish to quantify the agreement between theory and data by performing a $\chi^2$ test, which
requires knowledge of the bin-to-bin correlations between the different $A_i$ coefficients in \ptz (this is 
particularly important if these correlations are large, which is the case for the regularised data). However, to our knowledge, 
a well-defined covariance matrix for the regularised version of the $A_i$ coefficients is not available.

%------------------------------------------------                                                                                  
\begin{figure}
\centering
\includegraphics[width=.49\linewidth]{Figures/ZJ_ATLAS_ptz_A0_yincl_Ratio_Uncorr.pdf} \hfill
\includegraphics[width=.49\linewidth]{Figures/ZJ_ATLAS_ptz_A2_yincl_Ratio_Uncorr.pdf} \\
\includegraphics[width=.49\linewidth]{Figures/ZJ_ATLAS_ptz_A1_yincl_Ratio_Uncorr.pdf} \hfill
\includegraphics[width=.49\linewidth]{Figures/ZJ_ATLAS_ptz_dsig_yincl_Ratio.pdf} 
\caption{The \ptz distribution for the angular coefficients $A_0$ (upper left), $A_2$ (upper right), 
 $A_1$ (lower left), and the unpolarised cross section (lower right)  in $\Pp\Pp$ collisions at $\sqrt{s} = 8~\TeV$. 
 The ATLAS data (black points) are compared to the LO (blue fill), NLO (green fill), and NNLO (red fill)
 theoretical predictions. In the lower panel, each distribution is shown normalised 
 with respect to the central NLO prediction.
}
\label{fig:ATLAS_1}
\end{figure}
%------------------------------------------------
\begin{figure}
\centering
\includegraphics[width=.49\linewidth]{Figures/ZJ_ATLAS_ptz_A3_yincl_Ratio_Uncorr.pdf} \hfill
\includegraphics[width=.49\linewidth]{Figures/ZJ_ATLAS_ptz_A4_yincl_Ratio_Uncorr.pdf}
\caption{
The \ptz distribution for the angular coefficients $A_3$ (left) and $A_4$ (right) 
in $\Pp\Pp$ collisions at $\sqrt{s} = 8~\TeV$. The ATLAS data (black points) are 
compared to the LO (blue fill), NLO (green fill), and NNLO (red fill) 
theoretical predictions. In the lower panel, each distribution is shown normalised 
with respect to the central NLO prediction.
}
\label{fig:ATLAS_2}
\end{figure}
%------------------------------------------------

The \ptz distributions for the angular coefficients $A_0$ (upper left), $A_1$ (lower left), 
$A_2$ (upper right), and the unpolarised cross section (lower right) are shown in Fig.~\ref{fig:ATLAS_1}. 
The ATLAS data is represented by black points, and is compared to theoretical predictions at LO (blue), 
NLO (green), and NNLO (red). In the lower panel of each plot, the same distributions are shown normalised
with respect to the central NLO prediction.

The NNLO corrections are observed to have an important impact on each distribution, substantially reducing the scale uncertainties in all cases. With respect to the central value at NLO, the NNLO corrections to $A_0$ are negative and typically below $5\%$ in magnitude. In the case of the $A_2$ distribution, the corrections are also negative and most sizeable in the region of \ptz $\in[10,\,50]$~GeV. The description of the observed $A_2$ distribution is visibly improved at NNLO, while the NLO predictions systematically overestimate the data.
%
It should be noted that the $y$-axis ranges for both $A_0$ and $A_2$ distributions are fixed to the same values to
allow a straightforward visual comparison of the relative impact of the NNLO corrections in each case
(this is also the reason why they are placed in neighbouring positions within the Figures).
%A1
In the case of the $A_1$ distribution, the corrections are positive at low $p_\rT$ and 
change sign to become negative at large $p_\rT$, resulting in a modified shape of the distribution.
The size of the corrections across the whole $\ptz$ range vary between $+10\%$ at low values of $\ptz$ and $-5\%$ 
in the last $\ptz$ bin shown.

%A3 and A4
In Figure~\ref{fig:ATLAS_2}, the same comparison is performed for the 
parity-violating angular coefficients $A_3$ (left) and $A_4$ (right).
%
The NNLO corrections reduce the scale uncertainty of the prediction,
while having little impact on the central value. With respect to the experimental precision, the NNLO corrections to these distributions (which are well described by the central NLO prediction) are phenomenologically unimportant for a comparison to data.
%
As discussed in Section~\ref{sec2}, these particular coefficients are sensitive to the product of vector- and axial-vector-couplings 
of the $\PZ$-boson to the initial-state quarks. The corresponding predictions for these distributions are 
therefore sensitive to a combination of the input value of $\sw^2$ as well as to the relative 
contribution of up- and down-type quark initiated processes.
%
To the accuracy of the experimental distributions for these coefficients, our
choice of input parameter scheme (including universal corrections via the $\rho$-parameter) 
provides a consistent description of the data. However, if the precision of future measurements of these 
coefficients improves, it would be important to revisit this comparison while including possibly 
also the effect of electroweak corrections and to assess the impact of PDF uncertainties on these distributions.
%
We note that while a measurement of these coefficients is sensitive to the weak mixing angle, 
a more direct extraction of this parameter is possible through the measurement
of the forward--backward asymmetry in lepton-pair production. Indeed, such a measurement 
has already been performed by the ATLAS collaboration~\cite{Aad:2015uau}.

\subsection{Comparison to CMS data}
A similar measurement of the angular coefficients has also been 
presented by the CMS collaboration~\cite{Khachatryan:2015paa}.
In this case, the angular coefficients have been measured differentially within rapidity 
bins of $\lvert\yz\rvert \in [0.0,\,1.0]$ and $\lvert\yz\rvert \in [1.0,\,2.1]$, and with an invariant-mass 
window of $80 < m_{\Pl\Pl} < 100~\GeV$ on the lepton-pair final state.
% 
In the following, we perform a comparison to this CMS data for the measured $A_{0,..,4}$ coefficients.
For both rapidity selections, this comparison is performed for six bins within the range \ptz $\in [10,\,200]~\GeV$ 
as well as an overflow bin for $\ptz > 200~\GeV$.

%------------------------------------------------                                                                                     
\begin{figure}
\centering
\includegraphics[width=.49\linewidth]{Figures/ZJ_CMS_ptz_A0_0_0_1_0_Ratio_Uncorr.pdf} \hfill
\includegraphics[width=.49\linewidth]{Figures/ZJ_CMS_ptz_A2_0_0_1_0_Ratio_Uncorr.pdf} \\
\includegraphics[width=.49\linewidth]{Figures/ZJ_CMS_ptz_A1_0_0_1_0_Ratio_Uncorr.pdf} \hfill
\includegraphics[width=.49\linewidth]{Figures/ZJ_CMS_ptz_dsig_0_0_1_0_Ratio.pdf}
\caption{
The \ptz distribution for the angular coefficients $A_0$ (upper left), $A_2$ (upper right),
$A_1$ (lower left), as well as the unpolarised cross section (lower right) in $\Pp\Pp$ collisions at $\sqrt{s} = 8~\TeV$
where a kinematic cut of $\lvert\yz\rvert \in [0.0,\,1.0]$ is required for all distributions.
The CMS data (black points) are compared to the LO (blue fill), NLO (green fill), and NNLO (red fill) 
theoretical predictions. In the lower panel, each distribution is shown normalised 
with respect to the central NLO prediction.
}
\label{fig:CMS_1}
\end{figure}
%------------------------------------------------
\begin{figure}
\centering
\includegraphics[width=.49\linewidth]{Figures/ZJ_CMS_ptz_A0_1_0_2_1_Ratio_Uncorr.pdf} \hfill
\includegraphics[width=.49\linewidth]{Figures/ZJ_CMS_ptz_A2_1_0_2_1_Ratio_Uncorr.pdf} \\
\includegraphics[width=.49\linewidth]{Figures/ZJ_CMS_ptz_A1_1_0_2_1_Ratio_Uncorr.pdf} \hfill
\includegraphics[width=.49\linewidth]{Figures/ZJ_CMS_ptz_dsig_1_0_2_1_Ratio.pdf}
\caption{
The \ptz distribution for the angular coefficients $A_0$ (upper left), $A_2$ (upper right),
$A_1$ (lower left), as well as the unpolarised cross section (lower right) in $\Pp\Pp$ collisions at $\sqrt{s} = 8~\TeV$
where a kinematic cut of $\lvert\yz\rvert \in [1.0,\,2.1]$ is required for all distributions.
The CMS data (black points) are compared to the LO (blue fill), NLO (green fill), and NNLO (red fill) 
theoretical predictions. In the lower panel, each distribution is shown normalised 
with respect to the central NLO prediction.
}
\label{fig:CMS_2}
\end{figure}
%------------------------------------------------

The distributions for $A_0$ (upper left), $A_1$ (lower left), $A_2$ (upper right), and the unpolarised cross section (lower right) are shown in Figs.~\ref{fig:CMS_1} and~\ref{fig:CMS_2}, where Fig.~\ref{fig:CMS_1} corresponds to the rapidity bin of $\lvert\yz\rvert \in [0.0,\,1.0]$, and Fig.~\ref{fig:CMS_2} to $\lvert\yz\rvert \in [1.0,\,2.1]$. The CMS data is represented by black points
and is compared to LO (blue), NLO (green), and NNLO (red) predictions. As before, 
the distributions are shown normalised to the central NLO prediction in the lower panel of each plot.
%
The NNLO corrections exhibit similar behaviour in both rapidity bins as was the case for
the rapidity-integrated distributions shown in Fig.~\ref{fig:ATLAS_1} for ATLAS. Namely, large negative
corrections (reaching $-15\%$) to $A_2$ are found within the range of \ptz $\in[20,\,100]~\GeV$, 
and positive (negative) corrections are observed in the $A_1$ distribution at low (large) \ptz.
The description of the data is visibly improved by the precise NNLO predictions, which
have relative scale uncertainties of order $5\%$.

For the parity-violating angular coefficients $A_3$ and $A_4$, we see that the NNLO corrections do not alter the shapes of these distributions for the CMS kinematical setup (as was the case for the \yz-inclusive distributions for ATLAS shown in Fig. 5). Fig.~\ref{fig:CMS_3} shows that these distributions are well approximated by the central NLO predictions. The NNLO corrections to these distributions, as compared to the accuracy of the data, are phenomenologically unimportant.

%------------------------------------------------
\begin{figure}
\centering
\includegraphics[width=.49\linewidth]{Figures/ZJ_CMS_ptz_A3_0_0_1_0_Ratio_Uncorr.pdf} \hfill
\includegraphics[width=.49\linewidth]{Figures/ZJ_CMS_ptz_A4_0_0_1_0_Ratio_Uncorr.pdf} \\
\includegraphics[width=.49\linewidth]{Figures/ZJ_CMS_ptz_A3_1_0_2_1_Ratio_Uncorr.pdf} \hfill
\includegraphics[width=.49\linewidth]{Figures/ZJ_CMS_ptz_A4_1_0_2_1_Ratio_Uncorr.pdf}
\caption{
The \ptz distribution for the angular coefficients $A_3$ (left) and $A_4$ (right) 
in $\Pp\Pp$ collisions at $\sqrt{s} = 8~\TeV$, where kinematic cuts of $\lvert\yz\rvert \in [0.0,\,1.0]$ (left)
and $\lvert\yz\rvert \in [1.0,\,2.1]$ have been required for the shown distributions.
The CMS data (black points) are compared to the LO (blue fill), NLO (green fill), and NNLO (red fill) 
theoretical predictions. In the lower panel, each distribution is shown normalised 
with respect to the central NLO prediction. 
}
\label{fig:CMS_3}
\end{figure}
%------------------------------------------------


\subsection{Predictions for LHCb}
The LHCb collaboration has not yet performed a measurement of the angular
coefficients in $\PZ$-boson production. Such an analysis would however be of interest to provide a
probe of the $\PZ$-boson production mechanisms which may be enhanced at forward rapidities, and is also an
important stepping-stone towards performing an extraction of $M_{\PW}$ within the LHCb acceptance~\cite{Bozzi:2015zja}. 
%------------------------------------------------                                                                                                         
\begin{figure}
\centering
\includegraphics[width=.49\linewidth]{Figures/ZJ_LHCb_ptz_A0_2_0_4_5_Ratio_Uncorr.pdf} \hfill
\includegraphics[width=.49\linewidth]{Figures/ZJ_LHCb_ptz_A2_2_0_4_5_Ratio_Uncorr.pdf} \\
\includegraphics[width=.49\linewidth]{Figures/ZJ_LHCb_ptz_A1_2_0_4_5_Ratio_Uncorr.pdf} \hfill
\includegraphics[width=.49\linewidth]{Figures/ZJ_LHCb_ptz_dsig_2_0_4_5_Ratio.pdf}
\caption{\small 
The \ptz distribution for the angular coefficients $A_0$ (upper left), $A_2$ (upper right),
$A_1$ (lower left), as well as the unpolarised cross section (lower right) in $\Pp\Pp$ collisions at $\sqrt{s} = 8~\TeV$.
A kinematic cut of $\yz \in [2.0,\,4.5]$ (corresponding to the LHCb fiducial region) is required for all distributions.
Theoretical predictions are provided at LO (blue fill), NLO (green fill), and NNLO (red fill) 
accuracy. In the lower panel, each distribution is shown normalised 
with respect to the central NLO prediction.
}
\label{fig:LHCb_1}
\end{figure}
%------------------------------------------------
We therefore provide predictions for the $\ptz$ distributions for the LHCb fiducial region of 
\yz $\in [2.0,\,4.5]$, placing an invariant mass selection of $80 < m_{\Pl\Pl} < 100$~GeV on the lepton-pair final state.
No other cuts are placed on the lepton-pair final state as it is assumed that this will be corrected for in 
the experimental analysis.
%
The predictions for $A_0$ (upper left), $A_1$ (lower left), $A_2$ (upper right), as well as
the unpolarised cross section (lower right) are provided in Fig.~\ref{fig:LHCb_1}. Each \ptz distribution 
is provided in the region $\ptz \in [10.5,\,270]$~GeV, guided by the choice of binning taken 
in the recent LHCb measurement of $\PZ$-boson production at 13~TeV~\cite{Aaij:2016mgv}.

The predicted shapes of the distributions within the LHCb acceptance are similar
to what is observed at more central rapidities. 
In addition, the NNLO corrections 
for each of the angular coefficients are also observed to be of similar size to those
at more central rapidities. Although not shown here, the NNLO corrections to $A_3$ and $A_4$
were also computed for this kinematic setup and found to be negligibly small.
%
We can therefore conclude that the NNLO corrections to the \ptz spectrum 
for $A_0$, $A_1$, and $A_2$ should be included when performing a comparison to data, while
the central NLO prediction for the corresponding $A_3$ and $A_4$ distributions are likely sufficient.
%
If an experimental determination of the $A_3$ and $A_4$ distributions is achievable at LHCb
with smaller relative uncertainties as compared to ATLAS and CMS, it would be important to include 
the effects of electroweak corrections and to also assess the impact of PDF uncertainties 
on these distributions.

\subsection{Assessing the violation of the Lam--Tung relation} 
\label{LamTung}
As highlighted in Section~\ref{sec2}, the Lam--Tung relation is expected to be violated, i.e.\ $(A_0-A_2)\ne0$,
starting at order $\cO(\alphas^2)$ in the framework of pQCD.
%
Measurements of this violation therefore provide an important test of the
$\PZ$-boson production dynamics. As discussed in Section~\ref{sec2}, 
the extent of the breaking of the Lam--Tung relation can be assessed by measuring 
the \ptz distribution for the difference of the angular coefficients $A_0$ and $A_2$, or equivalently through
the normalised observable $\Delta^{\rm LT}$. The latter has the benefit of better exposing 
the violation of the Lam--Tung relation in the lower \ptz range, where the angular 
coefficients $A_0$ and $A_2$ are individually relatively small.
%
In the following, we discuss the corrections to $(A_0-A_2)$ and quantify the consistency of the 
data and the predictions by performing a $\chi^2$ test. We then present results for $\Delta^{\rm LT}$ 
and perform a comparison to data, where the data points for the latter are obtained by 
re-expressing $\Delta^{\rm LT}$ in terms of the measured angular coefficients.

Before comparing to data, it is important to comment on the expected accuracy of our theoretical predictions for 
these two observables. As for the individual angular coefficients, our theoretical predictions for $(A_0-A_2)$ 
are obtained from the computation of the production process for $\PZ+\jet$ at $\cO(\alphas^3)$. 
While the $\cO(\alphas^3)$ contributions comprise genuine NNLO corrections to the individual $A_i$ coefficients 
as demonstrated throughout this section, the prediction degrades to an NLO-accurate description for the 
difference $(A_0-A_2)$ and $\Delta^{\rm LT}$.\footnote{In the sense that the first non-trivial prediction for these observables begins at $\cO(\alphas^2)$.}
For consistency with the rest of the paper, we will continue to refer to the corrections of order $\cO(\alphas^3)$ as 
``NNLO corrections'' and similarly label the figures in this section as NNLO predictions.

%------------------------------------------------                                                                                                         
\begin{figure}
\centering
\includegraphics[width=.49\linewidth]{Figures/ZJ_ATLAS_ptz_A0mA2_yincl_Ratio_Uncorr.pdf}
\caption{\small 
The \ptz distribution for the difference of angular coefficients $(A_0-A_2)$ in $\Pp\Pp$ collisions at $\sqrt{s} = 8~\TeV$. 
The ATLAS data (black points) are compared to the LO (blue fill), NLO (green fill), and NNLO (red fill) 
theoretical predictions. In addition, the regularised ATLAS data is also included (grey fill). In the lower panel, 
each distribution is shown normalised with respect to the central NLO prediction.
}
\label{fig:ATLAS_A0mA2}
%------------------------------------------------
\bigskip
%------------------------------------------------
\includegraphics[width=.49\linewidth]{Figures/ZJ_CMS_ptz_A0mA2_0_0_1_0_Ratio_Uncorr.pdf} \hfill
\includegraphics[width=.49\linewidth]{Figures/ZJ_CMS_ptz_A0mA2_1_0_2_1_Ratio_Uncorr.pdf}
\caption{\small
The \ptz distribution for the difference of angular coefficients $(A_0-A_2)$ in $\Pp\Pp$ collisions at $\sqrt{s} = 8~\TeV$,
with the kinematic cuts of $\lvert\yz\rvert \in [0.0,\,1.0]$ (left) and $\lvert\yz\rvert \in [1.0,\,2.1]$ (right). 
The CMS data (black points) are compared to the LO (blue fill), NLO (green fill), and NNLO (red fill) 
theoretical predictions. In the lower panel, each distribution is shown normalised with respect to the central NLO prediction.}
\label{fig:CMS_A0mA2}
\end{figure}
%-----------------------------------------------

Figure~\ref{fig:ATLAS_A0mA2} shows the \ptz distribution for $(A_0-A_2)$,
where the ATLAS data is represented by black points, and is compared to LO (blue), 
NLO (green), and NNLO (red) theoretical predictions.
%
The NNLO corrections are observed to be large and positive, amounting to $+40\%$ at moderate $\ptz$ values,
and provide an improved description of the ATLAS data. It is worth noting that while a reduction of the absolute 
scale uncertainties is already observed at NNLO with respect to NLO, the relative uncertainty is in fact reduced 
by almost a factor of two across the shown $\ptz$ range.
%
This is a reflection of the fact that the computation of the $\PZ+\jet$-production process at
$\cO(\alphas^3)$ at finite $\ptz$ used to predict the difference $(A_0-A_2)$ is only NLO accurate  
and therefore yields corrections and remaining scale uncertainties which are typical for NLO
effects.
%
For most of the \ptz range, and for both experimental setups, it is found that the NLO and NNLO predictions 
for $(A_0-A_2)$ are consistent within uncertainties.
%
In Fig.~\ref{fig:ATLAS_A0mA2}, we have also chosen to include the regularised ATLAS data (indicated
by the grey fill). As discussed towards the start of this section, large bin-to-bin correlations are introduced
in the regularisation procedure of the ATLAS data (see for example Fig.~24 of Ref.~\cite{Aad:2016izn}).
We believe that this demonstrates how a visual comparison of the theory prediction with respect to the
regularised data (in this case at least) can lead one to overestimate the disagreement between theory and data.
%

As an alternative to a visual comparison, the quality of the theoretical description of the data can be
quantified by performing a $\chi^2$ test according to
%
\begin{align}
\chi^2 = \sum_{i,j}^{\rm N_{data}} ( O_{\rm exp}^i - O_{\rm th.}^i ) \sigma_{ij}^{-1} ( O_{\rm exp}^j - O_{\rm th.}^j ) ,
\end{align}
%
where $O_{\rm exp}^i$ and $O_{\rm th.}^i$ are respectively the central value of the experimental and theoretical predictions
for data point $i$, and $\sigma_{ij}^{-1}$ is the inverse covariance matrix.
%
In this case, we consider the unregularised ATLAS data for the angular coefficients $A_0$ and $A_2$ (and their correlations, as provided through the covariance matrix) within the range of $\ptz \in [11.4,\,600]~\GeV$, corresponding to a total of 38 data points. To perform this comparison, the central theory predictions for these angular coefficients are evaluated with the same binning choice as the data.
%
The results are
% 
\begin{align} %\label{eq:chi2ATLAS} 
  \text{NLO  (ATLAS):}&\qquad \chi^2/{\rm N_{data}} = 185.8/38 = 4.89 \,, \nonumber \\
  \text{NNLO (ATLAS):}&\qquad \chi^2/{\rm N_{data}} = 68.3/38  = 1.80 \,. \nonumber
\end{align}
%
This test indeed demonstrates that the NLO predictions give a poor description of the
data in the considered $\ptz$ range, a point that was also highlighted in the experimental analysis~\cite{Aad:2016izn}. 
This tension is largely reduced with the inclusion of the NNLO corrections, and from closer inspection 
of Fig.~\ref{fig:ATLAS_1}, can be mainly attributed to the large negative corrections to the $A_2$ distribution.

The corresponding \ptz distributions for the CMS measurement are shown in Fig.~\ref{fig:CMS_A0mA2}
for the rapidity bins $\lvert\yz\rvert \in [0.0,\,1.0]$ (left) and $\lvert\yz\rvert \in [1.0,\,2.1]$ (right). 
%
The NNLO corrections to $(A_0-A_2)$ exhibit a similar behaviour for the CMS kinematic selections,
and again improve the description of data. This agreement can also be quantified by performing 
a $\chi^2$ test, where in this case the test is performed directly on the $(A_0-A_2)$ distribution 
as no covariance matrix for these $A_i$ coefficients is publicly available. In total 14 data points are considered, corresponding
to seven \ptz bins for each rapidity selection. The results, assuming uncorrelated bins, are
%
\begin{align} %\label{eq:chi2CMS} 
  \text{NLO  (CMS):}&\qquad \chi^2/{\rm N_{data}} = 24.5/14 = 1.75\,, \nonumber \\
  \text{NNLO (CMS):}&\qquad \chi^2/{\rm N_{data}} = 14.2/14 = 1.01 \,. \nonumber
\end{align}
%
Similar to the findings for the ATLAS data, the description of the CMS data is substantially improved at NNLO.

%------------------------------------------------
\begin{figure}[ht!]
\centering
\includegraphics[width=.49\linewidth]{Figures/ZJ_ATLAS_ptz_LamTung_yincl_Uncorr.pdf}
%
\caption{\small The extent of the Lam--Tung violation as expressed through $\Delta^{\rm LT}$
for the ATLAS data in $\Pp\Pp$ collisions at $\sqrt{s} = 8~\TeV$. The data is compared to the corresponding 
NLO (green fill) and NNLO (red fill) predictions.
}
\label{fig:LamTungATLAS}
%------------------------------------------------
\bigskip
%------------------------------------------------
\includegraphics[width=.49\linewidth]{Figures/ZJ_CMS_ptz_LamTung_0_0_1_0_Uncorr.pdf} \hfill
\includegraphics[width=.49\linewidth]{Figures/ZJ_CMS_ptz_LamTung_1_0_2_1_Uncorr.pdf}
%
\caption{\small The extent of the Lam--Tung violation as expressed through $\Delta^{\rm LT}$
for the CMS data in $\Pp\Pp$ collisions at $\sqrt{s} = 8~\TeV$, with the kinematic cuts of 
$\lvert\yz\rvert \in [1.0,\,2.1]$ (left) and $\lvert\yz\rvert \in [1.0,\,2.1]$ (right).
The data is compared to the corresponding NLO (green fill) and NNLO (red fill) predictions.
}
\label{fig:LamTungCMS}
\end{figure}
%------------------------------------------------

As discussed previously, it is also informative to express the data in terms of the new obserable $\Delta^{\rm LT}$ as 
defined in Eq.~\eqref{eq:dLT}. This comparison is performed in Figs.~\ref{fig:LamTungATLAS} and \ref{fig:LamTungCMS}
for the ATLAS and CMS measurements, respectively, where the data has been re-expressed in terms of this quantity.%
\footnote{
	We omit the lower panels with the $K$-factors in these figures, as they are almost identical to the case of $(A_0-A_2)$ shown in Figs.~\ref{fig:ATLAS_A0mA2} and \ref{fig:CMS_A0mA2} due to the small corrections to the $A_0$ coefficient.
}
%
It is found that the extent of the Lam--Tung violation observed in data is consistently described
by the NNLO predictions. While there is some tendency for the data to prefer a stronger Lam--Tung 
violation for $\ptz > 40~\GeV$, more precise data is required to confirm this behaviour.
