%!TEX root = Zjets_Angular.tex

\section{Introduction} 
\label{sec1}

The production of $\PZ$ bosons followed by subsequent leptonic decay is a benchmark process at hadron colliders.
%
The production rate for this process is extremely
large, and, combined with the fact that the final state is
clean experimentally, it has allowed precise
(multi-) differential $\PZ$-boson cross section measurements 
to be performed both at the Tevatron~\cite{Aaltonen:2010zza,Abazov:2007jy} 
and the LHC~\cite{Aad:2014xaa,Khachatryan:2015oaa,Chatrchyan:2011wt,Aad:2015auj,Aaij:2015gna,Aaij:2015zlq,Aaij:2016mgv}.
Typically, these measurements are performed inclusively
with respect to the kinematic information of the gauge boson decay 
and have a wide range of phenomenological applications, including PDF and luminosity determinations.

Additional tests of the QCD dynamics for $\PZ$-boson production can also be performed by explicitly studying the angular distribution of the final-state leptons~\cite{Collins:1977iv,Lam:1978pu,Lam:1978zr,Lam:1980uc,Mirkes:1992hu,Mirkes:1994eb,Mirkes:1994dp}.
%
A prime example being the measurement of the forward--backward asymmetry
in lepton-pair production, differential in the lepton polar angle,
which provides important information on the coupling structure of the $\PZ$ boson to fermions~\cite{Chatrchyan:2011ya,
Aaltonen:2014loa,Abazov:2014jti,Aad:2015uau,Aaij:2015lka}.
%
However, an even richer structure is accessible by retaining the full 
differential information of the lepton kinematics. Under the assumption that the 
lepton pair is produced through the exchange of a gauge boson, the reconstructed 
lepton kinematics provide a direct probe of the polarisation of the intermediate gauge boson, 
which in turn exposes the underlying QCD production mechanism.
%
The QCD dynamics of this process can be expressed in terms of a set of eight frame dependent
angular coefficients $A_{i=0,\ldots,7}$, which depend on the 
invariant mass, transverse momentum, and rapidity of the lepton pair
and describe the production of the
intermediate gauge boson. 

The angular coefficients $A_0$ and $A_2$ further satisfy an important relation known 
as the Lam--Tung relation~\cite{Lam:1978pu,Lam:1978zr,Lam:1980uc}, $A_0-A_2=0$.
%
In the framework of perturbative QCD (pQCD), this relation can be shown to hold up to $\cO(\alphas)$ 
and is violated only at $\cO(\alphas^2)$ and higher.
%
At leading order, $A_0=A_2$ as a direct consequence of the spin-$\tfrac{1}{2}$ nature of the quarks,
and is further preserved at $\cO(\alphas)$ due to the vector-coupling of the spin-1 gluon to quarks.

Distributions for the angular coefficients can be extracted experimentally through 
fits to the measured final-state lepton kinematics, which can then be compared 
to the corresponding predictions obtained in pQCD.
%
The measurement of these angular coefficients is therefore interesting in its own right 
and much effort has been devoted to their precise determination.
%
Moreover, such a measurement also plays an important role 
in the determination of the $\PW$-boson mass $M_\PW$ at hadron colliders.
Indeed, a precise extraction of $M_\PW$ requires control of the Monte Carlo samples used to 
describe the kinematic distribution of leptons resulting from $\PW$-boson decay. The approach 
to generating these samples (and/or reweighting them) can in part be validated by using 
the  $\PZ$-boson production process as a case study, where the predicted values for all relevant angular coefficients 
can be directly compared to data.

On the theoretical side, the angular coefficients have been computed in pQCD up to $\cO(\alphas)$~\cite {Collins:1977iv,Lam:1978pu,Lam:1978zr,Lam:1980uc}
and $\cO(\alphas^2)$~\cite{Mirkes:1992hu,Mirkes:1994eb,Mirkes:1994dp} for non-vanishing transverse momenta $\ptz$ of the $\PZ$ boson.
For the inclusive Drell--Yan process, the $\cO(\alphas^2)$ corrections are available 
in the parton-level generators DYNNLO~\cite{Catani:2009sm} and FEWZ~\cite{Gavin:2010az},
which retain the full kinematical information of the final state and allow for a direct comparison to data in the fiducial region. 
These fixed-order predictions have been further matched to parton showers at NNLO in Ref.~\cite{Karlberg:2014qua}, 
where a comparison to the angular coefficients has also been performed.
Using the results obtained with DYNNLO and FEWZ, a detailed comparison 
to all available hadron collider and fixed target data has been carried out in Ref.~\cite{Lambertsen:2016wgj}. 
%
Studies of the Lam--Tung relation in the context of the intrinsic transverse momentum of the parton have 
also recently been considered in Ref.~\cite{Motyka:2016lta}.

Experimentally, a number of the angular coefficients were determined
in fixed target experiments by the NA10~\cite{Guanziroli:1987rp}, E615~\cite{Conway:1989fs},
and FNAL E866/NuSea~\cite{ Zhu:2006gx, Zhu:2008sj} collaborations 
using a variety of beams (pions, protons) and targets (tungsten, deuterium).
%
It is worth noting that the kinematical range probed in these fixed-target experiments was restricted to small invariant masses of the lepton pairs, 
typically from a few \GeV up to $\sim15~\GeV$.
In this regime, photon-exchange in the Drell--Yan process is by far the dominant contribution and
only the parity-even angular coefficients could be determined.

At high-energy colliders such as the Tevatron and the LHC, on the other hand, lepton-pair invariant masses around the $\PZ$-boson mass are considered, which are dominated by $\PZ$-boson exchange and also allow for the study of the parity-odd angular coefficients.
The measurement of angular coefficients at hadron colliders were performed 
by the CDF~\cite{Aaltonen:2011nr} collaboration in $\Pp\Pap$ collisions at a centre-of-mass (CoM)
energy of $\sqrt{s} = 1.96~\TeV$, and more recently by the CMS~\cite{Khachatryan:2015paa} and 
ATLAS~\cite{Aad:2016izn} collaborations in $\Pp\Pp$ 
collisions at $\sqrt{s} = 8~\TeV$.
%
Each of these analyses were performed in an invariant-mass window around the $\PZ$-boson resonance and in the Collins--Soper reference frame~\cite{Collins:1977iv}. 
%
Most notably, both ATLAS and CMS observe for the first time clear evidence for 
the violation of the Lam--Tung relation in $\PZ$-boson production.%
\footnote{
Note that this effect had been already observed  
by NA10~\cite{Guanziroli:1987rp} and E615~\cite{Conway:1989fs} for low-mass lepton pairs, whereas 
both FNAL E866/NuSea~\cite{Zhu:2006gx, Zhu:2008sj} and CDF~\cite{Aaltonen:2011nr} found results 
consistent with the difference $(A_0-A_2)$ being zero.
}
%
The new results from ATLAS and CMS are therefore particularly interesting as they 
demonstrate the violation of the Lam--Tung relation at energies never probed before.  

However, compared to the fixed-order $\cO(\alphas^2)$ prediction for 
lepton-pair production using the fixed order parton level code DYNNLO~\cite{Catani:2009sm}, a ``significant deviation'' is reported by the ATLAS collaboration~\cite{Aad:2016izn} for the difference $(A_0-A_2)$ in the region with $\ptz >20~\GeV$. 
Although less significant, a similar trend is also observed in the CMS data~\cite{Khachatryan:2015paa} where the $\cO(\alphas^2)$ prediction for the Drell--Yan pair production reaction is obtained 
using the parton-level generator FEWZ~\cite{Gavin:2010az}. 
Both experiments observe that the data exceeds the corresponding theory prediction for this observable. 
A tension is also observed in the \ptz spectrum for the angular coefficient $A_2$, where the data tends to undershoot the theory prediction.   
%
It is worth noting that although both FEWZ and DYNNLO yield predictions which are accurate at
next-to-next-to-leading order (NNLO) for the inclusive $\PZ$-boson production cross section, in analogy 
to the case for the \ptz or \phistar distributions studied in Refs.~\cite{Ridder:2016nkl,Gehrmann-DeRidder:2016jns}, 
these codes provide NLO-accurate predictions for the angular coefficients $A_i$ and LO-accurate 
predictions for the difference $(A_0 -A_2)$ (since $A_0 = A_2$ 
up to $\cO(\alphas)$ by virtue of the Lam--Tung relation).

The purpose of this work is to reassess the compatibility of the LHC data
to theory by providing predictions for the phenomenologically most important angular coefficients 
in high-mass lepton pair production at $\cO(\alphas^3)$, while focussing on the kinematic region with $\ptz >10~\GeV$, 
where many of the angular coefficients start to acquire non-vanishing values. 
This accuracy is achieved through the calculation of the $\PZ+\jet$ process at 
$\cO(\alphas^3)$~\cite{Ridder:2015dxa} at finite \ptz without requiring any resolved jets in the final state.

The layout of the paper is as follows. In Section~\ref{sec2}, the theoretical formalism for 
decomposing $\PZ$-boson production in terms of angular coefficients and spherical
harmonics is discussed. 
We further propose a new observable $\Delta^\mathrm{LT}$, which is particularly suited to 
study the violation of the Lam--Tung relation in the low-$\ptz$ regime.
Numerical predictions for these angular coefficients are provided
in Section~\ref{sec3}, along with a detailed comparison to the available LHC data. In addition, 
predictions for the LHCb experiment are provided, for which no measurement is available at present.
%
A summary of our findings and concluding remarks are presented in Section~\ref{sec:4}.