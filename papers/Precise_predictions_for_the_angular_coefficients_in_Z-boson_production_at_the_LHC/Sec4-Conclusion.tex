%!TEX root = Zjets_Angular.tex

\section{Conclusions and outlook} 
\label{sec:4}

Using our calculation of the $\PZ+\jet$ process at NNLO~\cite{Ridder:2015dxa}, we have computed the $\ptz$ distributions 
for the angular coefficients in $\PZ$-boson production to $\cO(\alphas^3)$.
%
We have focussed on the phenomenologically most relevant angular coefficients $A_{i=0,\ldots,4}$ for 
$\Pp\Pp$ collisions at $\sqrt{s} = 8~\TeV$ and have compared them with available LHC data.
%
With the theory uncertainties estimated by the uncorrelated variation of the factorisation and renormalisation scales (as described at the beginning of Section~\ref{sec3}), we find that these coefficients display a good perturbative convergence. In particular, a reduction of scale uncertainties is observed at each successive order and the residual scale uncertainties at NNLO are typically at the level of 5\%.
%
The NNLO corrections are observed to have an important impact on the predicted shapes of the distributions for $A_0$, $A_1$, and $A_2$. Of particular note is that the corrections to the $A_2$ distribution are both large and negative (up to $-20\%$) in the direction of data for both ATLAS and CMS measurements. 
%
It is found that the impact of the NNLO corrections to $A_3$ and $A_4$ distributions is small, and that these distributions are well described by the central NLO prediction.
%
Besides comparing our predictions to the available LHC data for these coefficients, we have also provided predictions for the $\ptz$ distributions for $A_{i=0,1,2}$ within the LHCb fiducial region, which would allow to probe the $\PZ$-boson production mechanism at forward rapidities. 
We find that the corrections to these distributions exhibit a similar behaviour to that at central rapidities both in size and shape.

Particular emphasis has been placed also on testing the consistency between the Lam--Tung violation 
observed in CMS and ATLAS data with respect to the theory predictions. 
To this end, we have studied the $\ptz$ distributions for the observable $(A_0-A_2)$ directly, where
the quality of the data--theory comparison has been assessed through a $\chi^2$ test.
%
Here, the inclusion of NNLO corrections leads to a significant improvement in the $\chi^2/N_\mathrm{dat.}$ values:
In the case of ATLAS, the $\chi^2/N_\mathrm{dat.}$ value reduces from 4.89 at NLO to 1.80 at NNLO; whereas for CMS, 
a reduction from 1.75 at NLO to 1.01 at NNLO is observed. With respect to the NNLO predictions,
no significant deviation is observed for the ATLAS data, and the CMS data is found to be fully consistent.

We further introduced a new observable  $\Delta^{\rm LT}$ defined in Eq.~\eqref{eq:dLT}, 
which is designed to better expose the violation of the Lam--Tung relation in the lower $\ptz$ regime. 
Expressed through this quantity, it becomes clear that the extent of Lam--Tung violation observed within the
range of $\ptz \in [10,40]~\GeV$, where this effect is the strongest, is consistent with the NNLO predictions. 
There however still remains some tendency for the data to systematically exceed the corresponding predictions 
at larger \ptz values. More precise data is required to clarify this situation.

Throughout this work, we have shown how the NNLO QCD predictions obtained 
via the calculation of the $\PZ+\jet$ process at $\cO(\alphas^3)$ are essential
to provide an adequate description of the \ptz distributions of several angular coefficients 
present in $\PZ$-boson production. 
It is therefore likely that a similar statement will also apply to
the case of $\PW$-boson production.
%
At present, a precise extraction of $M_{\PW}$ at the LHC relies on an accurate modelling of the corresponding 
angular coefficients in $\PW$-boson production~\cite{Aaboud:2017svj} based
on the fixed-order $\cO(\alphas^2)$ prediction. 
Our studies indicate that this level of theoretical accuracy is inadequate. 
The $\cO(\alphas^3)$ corrections to the decay lepton distributions in vector-boson production computed here are providing 
an important step towards improving the theoretical description of reference quantities necessary for the precise measurement of $M_{\PW}$. 

