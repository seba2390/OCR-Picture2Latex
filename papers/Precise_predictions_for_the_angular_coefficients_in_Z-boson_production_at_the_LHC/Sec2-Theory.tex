%!TEX root = Zjets_Angular.tex

\section{Theoretical preliminaries}
\label{sec2}

We consider the inclusive production of lepton pairs through the decay of an intermediate gauge boson, $\Pp(p_1) + \Pp(p_2) \to V(q) + X \to \Pl(k_1) + \Pal(k_2) + X$ as depicted in Fig.~\ref{fig:pp_VX}.
The cross section for this process can be written as the contraction of a lepton tensor ($L^{\mu\nu}$) describing the final-state decay with a hadronic tensor ($H_{\mu\nu}$) that describes the production sub-process, namely $L^{\mu\nu} \; H_{\mu\nu}$.
The lepton tensor in this context takes the role of an analyser, providing a probe of the structure of $H_{\mu\nu}$.
Note that the definition of the hadronic tensor includes the convolution with the PDFs as well as the integral over 
any degrees of freedom associated with the hadronic recoil ``$+X$''. 
As a result, $H_{\mu\nu}$ only depends on the four-momenta $p_1$, $p_2$, and $q$. 
Based on Lorentz- and gauge-invariance, the general decomposition of the hadronic tensor into form factors therefore reads%
\footnote{Owing to $H_{\mu\nu}^* = H_{\nu\mu}$, the symmetric and anti-symmetric parts of the hadronic tensor are purely real and imaginary, respectively.}
\begin{align}
  H_{\mu\nu} &=
  %\phantom{+} 
  H_1 \; \tg_{\mu\nu} %\nonumber \\ &\quad 
  + H_2 \; \tp_{1,\mu} \, \tp_{1,\nu} %\nonumber \\ &\quad 
  + H_3 \; \tp_{2,\mu} \, \tp_{2,\nu} %\nonumber \\ &\quad 
  + H_4 \; ( \tp_{1,\mu} \, \tp_{2,\nu} + \tp_{2,\mu} \, \tp_{1,\nu} ) \nonumber \\ &\quad 
  + \ri\, H_5 \; ( \tp_{1,\mu} \, \tp_{2,\nu} - \tp_{2,\mu} \, \tp_{1,\nu} ) %\nonumber \\ &\quad 
  + \ri\, H_6 \; \epsilon(\mu, \nu, p_1, q) %\nonumber \\ &\quad 
  + \ri\, H_7 \; \epsilon(\mu, \nu, p_2, q) \nonumber \\ &\quad 
  + H_8 \; \bigl(\, \tp_{1,\mu} \, \epsilon(\nu, p_1, p_2, q) + \mu \leftrightarrow \nu \,\bigr) %\nonumber \\ &\quad 
  + H_9 \; \bigl(\, \tp_{2,\mu} \, \epsilon(\nu, p_1, p_2, q) + \mu \leftrightarrow \nu \,\bigr) ,
  \label{eq:Hi}
\end{align}
with $\tg_{\mu\nu} = g_{\mu\nu} - \frac{q_\mu q_\nu}{q^2}$ and $\tp_\mu = \tg_{\mu\nu} p^\nu$.
The decomposition~\eqref{eq:Hi} further incorporates discrete symmetries such that $H_{1,\ldots,5}$ ($H_{6,\ldots,9}$) and $H_{5,8,9}$ ($H_{1,\ldots,4,6,7}$) are respectively even (odd) under parity and time-reversal.

\begin{figure}
  \begin{minipage}[t]{.48\linewidth}
  	\centering
  	\includegraphics[scale=1.2]{Figures/pp_VX.pdf}
  	\caption{Schematic diagram illustrating the kinematic configuration for the process.}
  	\label{fig:pp_VX}
  \end{minipage}
  %
  \hfill
  %
  \begin{minipage}[t]{.48\linewidth}
  	\centering
  	\includegraphics[width=\linewidth]{Figures/CS_frame.pdf}
  	\caption{The definition of the Collins--Soper~\cite{Collins:1977iv} angles in the di-lepton rest frame.}
  	\label{fig:CS_frame}
  \end{minipage}
\end{figure}

It is interesting to note that lepton-pair production satisfies an analogous relation to the Callan--Gross relation in deep-inelastic scattering~(DIS) known as the Lam--Tung relation~\cite{Lam:1978pu,Lam:1978zr,Lam:1980uc}, 
\begin{align}
  % H^{\mu}{}_{\mu} & = 2\, H_1 .
  H_1 &= \frac{1}{2} \; H^{\mu}{}_{\mu} \,.
  \label{eq:LT_cov}
\end{align}
This relation, formulated in a covariant manner, is frame independent and characteristic of the spin-$\tfrac{1}{2}$ nature of the quark. 
It has been further shown~\cite{Lam:1980uc} that Eq.~\eqref{eq:LT_cov} is not affected by $\cO(\alphas)$ QCD corrections,%
\footnote{
In the DIS process, the Born kinematics are highly constrained and 
are necessarily part of the Callan--Gross relation. In the presence of
real-emission corrections, these constraints are lifted leading to
a violation of the Callan--Gross relation at $\cO(\alphas)$.
}
which follows as a direct consequence of the vector-coupling of the spin-1 gluon to quarks~\cite{ArteagaRomero:1983ji}.
However, this relation has been shown to be violated at $\order{\alphas^2}$~\cite{Mirkes:1994dp}.
As such, the Lam--Tung relation offers a unique opportunity to study the pQCD predictions of the underlying dynamics encoded in $H_{\mu\nu}$ in more detail than through rate measurements alone.

To further elucidate the Lam--Tung relation, let us consider the kinematics of this process in
the lepton-pair rest frame where the final-state lepton momenta can be expressed in terms of the angles $\theta$ and $\phi$:
\begin{align}
  k^\mu_{1,2} &= \frac{Q}{2} \; (1, \pm\sin\theta \cos\phi, \pm\sin\theta \sin\phi, \pm\cos\theta)^\rT , &
  Q &= \sqrt{q^2} ,
\end{align}
where so far the orientation of the coordinate axes remains unspecified.
The only non-vanishing entries of the hadronic tensor~\eqref{eq:Hi} are the space--space components $H_{ij}$ ($H_{\mu 0} = H_{0 \nu} = 0$) ,
\begin{equation}
  H_{\mu\nu} \;\xrightarrow{\vec{q}=0}\; 
  \begin{pmatrix}
  0 & 0 & 0 & 0 \\
  0 & H_{11} & H_{12} & H_{13} \\
  0 & H_{21} & H_{22} & H_{23} \\
  0 & H_{31} & H_{32} & H_{33} 
  \end{pmatrix} ,
\end{equation}
in this reference frame.
After contracting the hadronic tensor with the lepton tensor of the $\PZ\to\Plm\Plp$ decay, the cross section can be decomposed in terms of spherical harmonics of up to degree two according to
\begin{align} 
  \frac{ \rd\sigma}{\rd^4q~\rd\cos\theta~\rd\phi} &= 
  \frac{3}{16 \pi} \; \frac{\rd\sigma^\mathrm{unpol.}}{\rd^4q}  
  \; \bigg\{ 
  (1 + \cos^2\theta) 
  + \frac{1}{2}\ A_0 \ (1 - 3 \cos^2\theta)  \nonumber \\ & \quad
  + A_1 \ \sin (2 \theta) \cos\phi  
  + \frac{1}{2}\ A_2 \ \sin^2\theta\ \cos (2 \phi)  \nonumber \\ & \quad
  + A_3 \ \sin \theta\ \cos \phi 
  + A_4 \ \cos\theta  
  + A_5 \ \sin^2\theta \ \sin (2 \phi)  \nonumber \\ & \quad
  + A_6 \ \sin (2 \theta)\ \sin \phi 
  + A_7 \ \sin \theta \ \sin \phi 
  \bigg\} ,
  \label{eq:Ai}
\end{align}
where $\rd\sigma^\mathrm{unpol.}$ is the unpolarised cross section. 
We note that the first term inside the parenthesis equal to $(1 + \cos^2\theta)$ is not accompanied by a separate angular coefficient, as its normalisation is described by $\rd\sigma^\mathrm{unpol.}$ that has been extracted as a global pre-factor in Eq.~\eqref{eq:Ai}.
The unpolarised cross section is given by the trace of the hadronic tensor and for $\PZ$ exchange explicitly reads
\begin{align}
  \frac{\rd\sigma^\mathrm{unpol.}}{\rd^4q}
  &= \frac{32\pi^2}{3} \; \alpha \;\bigl[ (g_\Pl^+)^2+(g_\Pl^-)^2 \bigr]
  \; Q^2 \;
  (H_{11}+H_{22}+H_{33}) ,
  \label{eq:unpol}
\end{align}
with $\alpha$ denoting the fine-structure constant and $g_\Pl^{\pm}$ the chiral $\PZ$-boson couplings to charged leptons.%
\footnote{
  In general, when both $\PZ$- and $\Pgg$-exchange are considered, the total cross section in this formalism can be expressed as a sum of three terms.
  Each comprises independent contractions between lepton and hadronic tensors associated to the photon-exchange, $\PZ$-exchange, and the $\PZ$--$\Pgg$-interference.
}
These couplings are defined according to
\begin{align}
  g_\Pl^+ &= \frac{\sw}{\cw} , &
  g_\Pl^- &= \frac{\sw^2-\frac{1}{2}}{\sw\cw} , &
  \sw &\equiv \sin\theta_\rw, \quad
  \cw \equiv \cos\theta_\rw ,
\end{align}
where $\theta_\rw$ is the weak mixing angle.

The form factors $H_{1,\ldots,9}$ in Eq.~\eqref{eq:Hi}, or equivalently the nine non-vanishing components $H_{ij}$ of the hadronic tensor, are directly related to the eight angular coefficients $A_{0,\ldots,7}$ and the unpolarised cross section.
Explicitly, the $A_i$ are given by
\begin{align}
  A_0 &= 2 \; H_{33}            \,\;c_{+}, &
  A_1 &= - (H_{13}+H_{31})      \,\;c_{+}, & %\nonumber \\ 
  A_2 &= 2 (H_{22}-H_{11})      \,\;c_{+}, \nonumber \\ 
  A_3 &= 2 \ri (H_{23}-H_{32})  \,\;c_{-}, & %\nonumber \\ 
  A_4 &= 2 \ri (H_{12}-H_{21})  \,\;c_{-}, & %\nonumber \\ 
  A_5 &= - (H_{12}+H_{21})      \,\;c_{+}, \nonumber \\ 
  A_6 &= - (H_{23}+H_{32})      \,\;c_{+}, & %\nonumber \\ 
  A_7 &= 2 \ri (H_{31}-H_{13})  \,\;c_{-},
  \label{eq:Ai_Hij}
\end{align}
where the proportionality factors $c_{\pm}$ arise from the fact that $\rd\sigma^\mathrm{unpol.}$ has been removed as a prefactor in the definition~\eqref{eq:Ai} and are given by
\begin{align}
  c_+ &= (H_{11}+H_{22}+H_{33})^{-1} , &
  c_- &= \frac{(g_\Pl^+)^2-(g_\Pl^-)^2}{(g_\Pl^+)^2+(g_\Pl^-)^2} \; (H_{11}+H_{22}+H_{33})^{-1} .
\end{align}
The unpolarised cross section~\eqref{eq:unpol} completes Eq.~\eqref{eq:Ai_Hij} as the ninth linearly independent combination of the $H_{ij}$.

Let us now choose a specific reference frame by defining the direction of the axes in the lepton-pair rest frame. 
To this end, we consider the Collins--Soper frame~\cite{Collins:1977iv} shown in Fig.~\ref{fig:CS_frame}:
The $z$-axis is chosen as the external bisector of the incoming beam directions, $\hat{e}_{z}^\mathrm{CS} \sim \pm (\vec{p}_1 - \vec{p}_2)$, where the positive $z$-direction is aligned with the $z$-direction of the lepton pair in the laboratory frame.
The $x$-axis lies in the hadron plane orthogonal to the $z$-axis and points in the direction of $\hat{e}_{x}^\mathrm{CS} \sim -(\vec{p}_1+\vec{p}_2)$. 
Lastly, the $y$-axis is chosen to complete a right-handed Cartesian coordinate system and is orthogonal to the hadronic event plane.
The four-momenta of the incoming hadrons in this reference frame are given by%
\footnote{
Note that we have suppressed the additional sign ambiguity in the $z$-component of $p_{1,2}^\mu$ 
due to the alignment of the $z$-axis w.r.t.\ the $\PZ$-boson direction in the laboratory frame.
} 

\begin{align}
  p_{1,2}^\mu &= E_{1,2} \; (1, -\sin\gamma, 0, \pm\cos\gamma)^\rT , & 
  E_{1,2} &= \frac{(q \cdot p_{1,2})}{Q} , &
  \cos\gamma &= \frac{Q}{\sqrt{Q^2 + q_\rT^2}} .
\end{align}

Returning to the Lam--Tung relation~\eqref{eq:LT_cov}, one can derive the corresponding relation in terms of the angular coefficients $A_i$ in the Collins--Soper frame
\begin{align}
  0 &\equiv
  2 H_1 - H^{\mu}{}_{\mu}  \nonumber \\
  &= 2 H_1 - H_1\, \tg^{\mu}{}_{\mu} - H_2\, \tp_1^2 - H_3\, \tp_2^2 -H_4\, 2 (\tp_1\cdot\tp_2)  \nonumber \\
  % &= - H_1 +(E_1)^2 H_2 + (E_2)^2 H_3 - \left( s - 2 E_1 E_2 \right) H_4  \nonumber \\
  &= - H_1 +(E_1)^2 H_2 + (E_2)^2 H_3 + 2 E_1 E_2 \left( \sin^2\gamma - \cos^2\gamma \right) H_4  \nonumber \\
  &= H_{33} - H_{22} + H_{11}  \nonumber \\
  &\propto A_0 - A_2  ,
  \label{eq:LT_Ai}
\end{align}
where we have used
\begin{align}
  H_{11} &= -H_1 + \left[ (E_1)^2 H_2 + (E_2)^2 H_3 + 2 E_1 E_2 \,H_4 \right] \; \sin^2\gamma , \nonumber \\
  H_{22} &= -H_1 , \nonumber \\
  H_{33} &= -H_1 + \left[ (E_1)^2 H_2 + (E_2)^2 H_3 - 2 E_1 E_2 \,H_4 \right] \; \cos^2\gamma , 
  \label{eq:Hii}
\end{align}
for the non-vanishing diagonal components of the hadronic tensor.
We observe that the Lam--Tung relation is equivalent to $A_0 - A_2 = 0$. 
Note that the result of Eq.~\eqref{eq:LT_Ai} is not frame independent but only holds if both the $z$- and $x$-axis 
in the lepton-pair rest frame lie in the hadronic event plane.
This condition enters in the step where the form factors $H_i$ are expressed in terms of the diagonal $H_{ij}$ components using Eq.~\eqref{eq:Hii} and can be understood by inspecting the covariant formulation of Eq.~\eqref{eq:LT_cov} in the lepton-pair rest frame:
The only form factors that contribute to the trace of the hadronic tensor on the r.h.s.\ of Eq.~\eqref{eq:LT_cov} are $H_{1,\ldots,4}$.
The tensor structures multiplying $H_{2,3,4}$ only involve momenta lying inside the hadronic plane and it is solely the tensor $\tg_{\mu\nu}$ multiplying the form factor $H_1$ which has a non-vanishing component orthogonal to it.
The Lam--Tung relation therefore distinguishes the direction perpendicular to the hadronic plane and can be interpreted as a statement about the current--current correlation of the hadronic tensor in this direction.%
\footnote{For hypothetical spin-$0$ partons, the current correlator would be completely confined within the hadronic event plane, which then yields for Eq.~\eqref{eq:LT_cov}: $H_1=0$.}

Making use of the completeness of the spherical harmonics, the angular coefficients appearing in the decomposition provided
in Eq.~\eqref{eq:Ai} can be extracted through the projectors
\begin{align}
  A_0 &= 4-10 \, \avg{\cos^2\theta} , &
  A_1 &= 5 \, \avg{\sin(2\theta) \, \cos\phi} , &
  A_2 &= 10 \, \avg{\sin^2\theta \, \cos(2\phi)} , \nonumber \\
  A_3 &= 4 \, \avg{\sin\theta \, \cos\phi} , &
  A_4 &= 4 \, \avg{\cos\theta} , &
  A_5 &= 5 \, \avg{\sin^2\theta \, \sin(2\phi)} , \nonumber \\
  A_6 &= 5 \, \avg{\sin(2\theta) \, \sin\phi} , &
  A_7 &= 4 \, \avg{\sin\theta \, \sin\phi} , 
  \label{eq:Ai_proj}
\end{align}
where $\avg{\ldots}$ denotes taking the (normalised) weighted average over the angular variables $\theta$, $\phi$ and is defined as
\begin{align}
  \avg{f(\theta, \phi)} &\equiv 
  \frac{
    \int_{-1}^{1} \rd\cos\theta \int_0^{2\pi} \rd\phi \; 
    \rd\sigma(\theta,\phi) \;
    f(\theta, \phi)
  }{
    \int_{-1}^{1} \rd\cos\theta \int_0^{2\pi} \rd\phi \; 
    \rd\sigma(\theta,\phi)
  } .
  \label{eq:proj}
\end{align}

The dominant angular coefficients are $A_{0,\ldots,4}$, while $A_{5,6,7}$ vanish at $\order{\alphas}$ and only receive small $\order{\alphas^2}$ corrections from the absorptive parts of the one-loop amplitudes in $\PZ+\jet$ production.
We therefore will not discuss the coefficients $A_{5,6,7}$ in the following.
In the case of pure $\Pgg^*$ exchange, the relevant coefficients are the parity-conserving coefficients $A_{0,1,2}$.
$A_3$ and $A_4$, on the other hand, are odd under parity and proportional to the product of vector- and axial-vector-couplings of the gauge boson to the fermions.
As such, they are sensitive to the relative rate of incoming down- and up-type quark fluxes as well as the weak mixing angle $\sw$.
All the coefficients $A_i$ vanish in the limit $\ptz\to0$ with the exception of $A_4$, which is finite in this limit and directly related to the forward--backward asymmetry.

One of the goals of this work is to assess the compatibility of the observed extent of the Lam--Tung violation
with that expected in predictions based on pQCD. This can be done by directly studying the \ptz distribution for 
the difference of the angular coefficients $A_0$ and $A_2$. Here, we propose a new observable
%
\begin{align} \label{eq:dLT}
  \Delta^{\rm LT} \equiv 1 - \frac{A_2}{A_0} ,
\end{align}
%
which has the benefit that the strong suppression of the individual angular coefficients
in the low-$\ptz$ region is absent.
In addition, the dependence on the unpolarised cross section appearing in the denominator of Eq.~\eqref{eq:proj} cancels in the ratio between the two coefficients.
Consequently, this observable may help to expose the extent of the Lam--Tung
violation in this region.
% 
In Section~\ref{sec3}, we shall compare our predictions to the available ATLAS and CMS data for the
$\ptz$ distributions of both $(A_0-A_2)$ and $\Delta^\mathrm{LT}$. In the latter case, 
the data will be re-expressed in terms of $\Delta^\mathrm{LT}$.
