\chapter{Abstract}
\label{chap:abstract}

% Main aim of this thesis
Computer vision can be understood as the ability to perform \emph{inference} on
image data. Breakthroughs in computer vision technology are often marked by advances
in inference techniques, as even the model design is often dictated by the
complexity of inference in them. This thesis proposes learning based inference schemes
and demonstrates applications in computer vision. We propose techniques for inference
in both generative and discriminative computer vision models.

% Improving inference in Generative models
Despite their intuitive appeal, the use of generative models in vision is hampered
by the difficulty of posterior inference, which is often too complex or too slow to
be practical. We propose techniques for improving inference in two widely used
techniques: Markov Chain Monte Carlo (MCMC) sampling and message-passing inference.
Our inference strategy is to learn separate discriminative models that assist Bayesian
inference in a generative model. Experiments on a range of generative vision models show
that the proposed techniques accelerate the inference process and/or converge to better solutions.

% Improving inference in Discriminative Models
A main complication in the design of discriminative models is the inclusion of prior
knowledge in a principled way. For better inference in discriminative models, we propose
techniques that modify the original model itself, as inference is simple evaluation of
the model. We concentrate on convolutional neural network (CNN) models and propose a
generalization of standard spatial convolutions, which are the basic building blocks of CNN
architectures, to bilateral convolutions. First, we generalize the existing use of
bilateral filters and then propose new neural network architectures with learnable
bilateral filters, which we call `Bilateral Neural Networks'. We show how the bilateral
filtering modules can be used for modifying existing CNN architectures for better image
segmentation and propose a neural network approach for temporal information propagation
in videos. Experiments demonstrate the potential of the proposed bilateral networks on
a wide range of vision tasks and datasets.

% Evaluation and Summary
In summary, we propose learning based techniques for better inference in several
computer vision models ranging from inverse graphics to freely
parameterized neural networks. In generative vision models, our inference techniques
alleviate some of the crucial hurdles in Bayesian posterior inference, paving new ways
for the use of model based machine learning in vision. In discriminative CNN models,
the proposed filter generalizations aid in the design of new neural network architectures
that can handle sparse high-dimensional data as well as provide a way for incorporating
prior knowledge into CNNs.
