\chapter{Symbols and Notation}
\label{chap:symbols}

\newcommand{\obs}{\mathbf{x}}
\newcommand{\target}{\mathbf{y}}
\newcommand{\targetProp}{\bar{\mathbf{y}}}
\newcommand{\params}{\mathbf{\theta}}
\newcommand{\f}{\mathbf{f}}
\newcommand{\func}{\mathcal{F}}

\newcommand{\eg}{\textit{e.g.}\@\xspace}
\newcommand{\ie}{\textit{i.e.}\@\xspace}
\newcommand{\ala}{\textit{\'{a}~la}\@\xspace}
\newcommand{\etal}{\textit{et~al.}\@\xspace}

Unless otherwise mentioned, we use the following notation and symbols in this
thesis. Here, we only list those symbols which are used across multiple chapters.
Those symbols that are specific to particular sections or chapters are not
listed here.

\begin{longtable}[l]{p{50pt} p{300pt}}
\toprule
\textbf{Symbol}	& \textbf{Description} \\
\midrule
$\obs$	 	& Observation variables (in vectorized form)\\
$\target$	 	& Target variables (in vectorized form)\\
$\targetProp$	 	& Intermediate/Proposed target variables\\
$K_t,m_t,\cdots$	 	& Random variables at time step $t$\\
$\params$	 	& Set of all model parameters \\
$\alpha, \beta, \mu, \gamma$	 & Model or training parameters \\
$\f$	 	& Pixel or superpixel features such as $(x,y,r,g,b)$ \\
$P(\cdot|\cdot)$	&  Probability distribution or density \\
$\mathcal{N}(\cdot|\cdot)$	& Gaussian distribution \\
$\psi_u$ & Unary potential at each pixel/superpixel \\
$\psi_p$ & Pairwise potential between two pixels/superpixels \\
$L(\cdot), E(\cdot)$ & Loss/Objective/Energy function \\
$\func(\cdot)$ & Generic function relating input to output variables \\
$\Lambda$ & Diagonal matrix for scaling image features say $(x,y,r,g,b)$ \\
\bottomrule
\end{longtable}
