\chapter{Zusammenfassung}

Maschinelles Sehen kann als die F\"ahigkeit verstanden werden Bilddaten zu interpretieren. Durchbr\"uche in diesem Feld gehen oft einher mit Fortschritten in Inferenztechniken, da die Komplexit\"at der Inferenz die Komplexit\"at der verwendeten Modelle bestimmt. Diese Arbeit beschreibt lernbasierte Inferenzmechanismen und zeigt Anwendungen im maschinellen Sehen auf, wobei auf Techniken f\"ur Inferenz in sowohl generativen als auch diskriminativen Modellen eingegangen wird.


Obwohl naheliegend und intuitiv verst\"andlich, sind generative Modelle im maschinellen Sehen h\"aufig nur eingeschr\"ankt nutzbar, da die Berechnung der A-Posteriori-Wahrscheinlichkeiten oft zu komplex oder zu langsam ist, um praktikabel zu sein. Wir beschreiben Techniken zur Verbesserung der Inferenz in zwei weit verbreiteten Inferenzverfahren: `Markov Chain Monte Carlo Sampling' (MCMC) und `Message-Passing'. Die vorgeschlagene Verbesserung besteht darin, mehrere diskriminative Modelle zu lernen, die die Grundlage f\"ur Bayes'sche Inferenz \"uber einem generativen Modell bilden. Wir demonstrieren anhand einer Reihe von generativen Modellen, dass die beschriebenen Techniken den Inferenzprozess beschleunigen und/oder zu besseren L\"osungen konvergieren.


Eine der gr\"o{\ss}ten Schwierigkeiten bei der Verwendung von diskriminativen Modellen ist die systematische Ber\"ucksichtigung von Vorkenntnissen. Zur Verbesserung der Inferenz in diskriminativen Modellen schlagen wir Techniken vor die das ursprüngliche Modell selbst verändern, da Inferenz in diesen die schlichte Auswertung des Modells ist. Wir konzentrieren uns auf `Convolutional Neural Networks' (CNN) und schlagen eine Generalisierung der Faltungsoperation vor, die den Kern jeder CNN-Architektur bildet. Dazu verallgemeinern wir bilaterale Filter und pr\"asentieren eine neue Netzarchitektur mit trainierbaren bilateralen Filtern, die wir `Bilaterale Neuronale Netze' nennen. Wir zeigen, wie die bilateralen Filtermodule verwendet werden können, um existierende Netzwerkarchitekturen für Bildsegmentierung zu verbessern und entwickeln ein auf Bilateralen Netzen basierendes Modell zur zeitlichen Integration von Information für Videoanalyse. Experimente mit einer breiten Palette von Anwendungen und Datens\"atzen zeigen das Potenzial der vorgeschlagenen bilateralen Netzwerke.


Zusammenfassend schlagen wir Lernmethoden f\"ur bessere Inferenz in einer Reihe von Modellen des maschinellen Sehens vor, von inversen Renderern bis zu trainierbaren neuronalen Netzwerken. Unsere Inferenz-Techniken helfen bei der Berechnung der A-Posteriori-Wahrscheinlichkeiten in generativen Modellen und erm\"oglichen so neue Ans\"atze des modellbasierten machinellen Lernens im Bereich des maschinellen Sehens. In diskriminativen Modellen wie CNNs helfen die vorgeschlagenen verallgemeinerten Filter beim Entwurf neuer Netzarchitekturen, die sowohl hochdimensionale Daten verarbeiten k\"onnen als auch Vorkenntnisse in die Inferenz einbeziehen.
