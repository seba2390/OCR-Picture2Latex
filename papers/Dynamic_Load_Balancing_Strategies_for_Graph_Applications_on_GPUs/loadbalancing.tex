\documentclass[conference,10pt]{IEEEtran}
%\documentclass[conference]{/home/satish/papers/style/hipc/IEEEtran}

\IEEEoverridecommandlockouts

\usepackage{setspace}
\usepackage{enumerate}
\usepackage{cite}
\usepackage{times}
\usepackage{url}
\usepackage{graphicx}
\usepackage{subfigure}
\usepackage{amsmath}
\usepackage{color}
\usepackage[ruled,figure,linesnumbered]{algorithm2e}
\usepackage{multirow}
\usepackage{array}
\usepackage{wrapfig}
%\usepackage[ruled,figure,linesnumbered]{/home/satish/papers/style/algorithm2e/tex/algorithm2e}
% \usepackage{algpseudocode}

\newcommand{\todo}[1]{\textrm{\color{blue} #1}}
\newcommand{\REM}[1]{}

\begin{document}
\title{Dynamic Load Balancing Strategies for Graph Applications on GPUs}
\author{}
%\REM {
\author{
\IEEEauthorblockN{
$^1$Ananya Raval,
$^2$Rupesh Nasre,
$^1$Vivek Kumar,
$^1$Vasudevan R,
$^1$Sathish Vadhiyar,
$^3$,$^4$Keshav Pingali
}
\IEEEauthorblockA{
$^1$Department of Computational and Data Sciences, Indian Institute of Science, Bangalore, India \\
$^2$Department of Computer Science and Engineering, Indian Institute of Technology, Madras, India \\
$^3$Institute for Computational Engineering and Sciences, University of Texas at Austin, USA \\
$^4$Department of Computer Science, University of Texas at Austin, USA \\
ananya.raval@gmail.com, rupesh@cse.iitm.ac.in, vivekkumar1987@gmail.com, \\vasudevan@ssl.serc.iisc.in, vss@serc.iisc.in, pingali@cs.utexas.edu
}
}
%}
\maketitle

\begin{abstract}
Acceleration of graph applications on GPUs has found large interest due to the ubiquitous use of graph processing in various domains. The inherent \textit{irregularity} in graph applications leads to several challenges for parallelization. A key challenge, which we address in this paper, is that of load-imbalance. If the work-assignment to threads uses node-based graph partitioning, it can result in skewed task-distribution, leading to poor load-balance. In contrast, if the work-assignment uses edge-based graph partitioning, the load-balancing is better, but the memory requirement is relatively higher. This makes it unsuitable for large graphs. In this work, we propose three techniques for improved load-balancing of graph applications on GPUs. Each technique brings in unique advantages, and a user may have to employ a specific technique based on the requirement. Using Breadth First Search and Single Source Shortest Paths as our processing kernels, we illustrate the effectiveness of each of the proposed techniques in comparison to the existing node-based and edge-based mechanisms.
\end{abstract}

Reinforcement learning has achieved great success in areas such as Game-playing \citep{silver2018general,vinyals2019grandmaster}, robotics \cite{kober2013reinforcement}, large language models \citep{ouyang2022training}, etc.
However, due to safety concerns or physical limitations, in some real-world reinforcement learning problems, we must consider additional constraints that may influence the optimal policy and the learning process \citep{garcia2015comprehensive}.
% For example, a robotic arm must not take actions that may cause harm to itself or the environments.
A standard framework to handle such cases is the constrained Markov Decision Process (CMDP) \citep{altman1999constrained}.
Within the CMDP framework, the agent has to maximize
the expected cumulative reward while
obeying a finite number of constraints, which are usually in the form of expected cumulative cost criteria.

However, we are sometimes concerned with the problem with a continuum of constraints.
For example,
the constraints we meet might be time-evolving or subject to uncertain parameters, which
cannot be formulated as an ordinary CMDP
(see Examples \ref{Example_Time_Evolving} and  \ref{Example_Uncertain}).
In this paper we would study a generalized CMDP  
to address the above problem.  Because the constraints are not only infinite-number but also lie
in a continuous set,
the generalization is not trivial. Fortunately, we find that we can borrow the idea behind semi-infinite programming (SIP) \citep{remez1934determination, hettich1993semi} to deal with the semi-infinite constraints.
Accordingly, we propose \emph{semi-infinitely constrained Markov decision processes} (SICMDPs)
as a novel complement to the ordinary CMDP framework.
%More specifically,  an SICMDP model %, we consider 
%contains a continuum of constraints whereas an ordinary CMDP contains a finite number of constraints. 

%This generalization is natural but not trivial. However, we can brows the idea  
%The idea is quite natural and can be backtracked
%to the practice of extending linear programming to linear semi-infinite programming (LSIP) %\cite{remez1934determination, GobernaLSIO1998}.
%In addition, 
%As a complementary approach to the ordinary CMDP framework, 
%SICMDP can be used to model these problems  which cannot be described by a finite number of constraints
%that are not covered by .
%For example,
%the restrictions we consider can be time-evolving or subject to uncertain parameters
%, thus
%cannot be described by a finite number of constraints but a continuum of constraints 
%(see Examples \ref{Example_Time_Evolving} and  \ref{Example_Uncertain}).

We also present two reinforcement learning algorithms to solve SICMDPs called SI-CRL and SI-CPO, respectively.
SI-CRL is a model-based reinforcement learning algorithm designed for tabular cases, and SI-CPO is a policy optimization algorithm for non-tabular cases.
% and analyze its performance both theoretically and empirically.
The main challenge is that we need to deal with a continuum of constraints, thus reinforcement learning algorithms for ordinary CMDPs do not work anymore.
In SI-CRL, we tackle this difficulty by first transforming the reinforcement learning problem to an equivalent LSIP problem, which can then be solved using methods in the LSIP literature like the dual exchange methods \citep{Hu1990,reemtsen1998numerical}.
In SI-CPO, we resort to the idea of cooperative stochastic approximation developed in \cite{lan2020algorithms, wei2020comirror}.
As far as we know, we are the first to introduce tools from semi-infinitely programming (SIP) into the reinforcement learning community for solving constrained reinforcement learning problems.

% To the best of our knowledge, we are the first to apply tools from semi-infinitely programming (SIP) to solve reinforcement learning problems.
Furthermore, we give theoretical analysis for both SI-CRL and SI-CPO.
We decompose the error of SI-CRL into two parts: the statistical error from approximating the true SICMDP with an offline dataset and the optimization error due to the fact that the solution of the LSIP problem obtained by the dual exchange method is inexact.
On the optimization side, we show that the iteration complexity of SI-CRL is $O\left(\left\{\mathrm{diam}(Y)L\sqrt{|\gS|^2|\gA|m}/\left[(1-\gamma)\epsilon\right]\right\}^m\right)$.
On the statistical side, we show that the sample complexity of SI-CRL is $\widetilde O\left(\frac{|S|^2|A|^2}{\epsilon^2(1-\gamma)^3}\right)$ if the offline dataset is generated by a generative model, and $\widetilde O\left(\frac{|S||A|}{\nu_{\min} \epsilon^2(1-\gamma)^3}\right)$ if the dataset is generated by a probability measure $\nu$ as considered in \cite{chen2019information}.
Here $\widetilde O$ means that all logarithm terms are discarded.
For SI-CPO, things become a little more complicated because other than the statistical error and the optimization error, we also need to consider the function approximation error, which comes from imperfect policy parametrizations.
It is shown if the function approximation error can be controlled to $O(\epsilon)$ order, the iteration complexity of SI-CPO is $\widetilde{O}\left(\frac{1}{\epsilon^2(1-\gamma)^6}\right)$ and the sample complexity of SI-CPO is $\widetilde{O}(\frac{1}{\epsilon^4(1-\gamma)^{10}})$.
Here our iteration complexity bound is equivalent to a typical $\widetilde O(1/\sqrt{T})$ global convergence rate.

We perform a set of numerical experiments to illustrate the SICMDP model and validate our proposed algorithms.
Specifically, we examine two numerical examples, namely the discharge of sewage and ship route planning.
Through the discharge of sewage example, we show the advantage of the SICMDP framework over the CMDP baseline obtained by naive discretization in modeling realistic sequential decision-making problems.
Moreover, we demonstrate the effectiveness of the SI-CRL and SI-CPO algorithms in such tabular environments. 
In the ship route planning example, we illustrate the benefits of the SICMDP framework and the ability of the SI-CPO algorithm to address complex continuous control tasks involving continuous state spaces with modern deep reinforcement learning techniques.

% In summary, our contributions are listed as follows.
% First, we present the SICMDP model, which can be viewed as a generalization of the ordinary CMDP model.
% Second, we propose an algorithm to perform reinforcement learning for SICMDPs, which is called SI-CRL, and we believe that we are the first to apply tools from SIP
% to solve reinforcement learning problems.
% Third, we give a theoretical analysis of SI-CRL and identify both its sample complexity and iteration complexity.
% In addition, we perform numerical experiments to illustrate the SICMDP model and validate the SI-CRL algorithm.
% \{This paragraph can be removed!!! \}





\section{Motivations for Empirical Study}
\label{sec:motivations}
The key question that we try to answer is when and why we should use standard
iteration space tiling over cache oblivious tiling.  The two approaches
perform similar partitioning of the iteration space, but the schedules given
to the partitions are different.  Theoretically, cache oblivious code seems to
have advantages over iteration space tiling.  However, many factors complicate
the actual performance, which made our initial experiments difficult to
interpret.  In this section, we describe the obstacles between the theory and
practice we have identified.

We use Single-Level Tiling (SLT) for iteration space tiling, and Cache
Oblivious Tiling (COT) for cache oblivious techniques in this
paper, which are further described in Section~\ref{sec:background}.

\paragraph{Recursion Overhead} This is a well-known overhead of
COT~\cite{yotov2007experimental}.  The recursion introduces overheads, such as
function call overhead, and increased register pressure.  Furthemore, the
functions force inter-procedural analysis/optimization, known to be more
difficult for compilers well.  Thus, the leaf tiles must be ``sufficiently
large'' to avoid excessive overhead due to the recursion.

 \paragraph{Recursive Split Constraints the Tile Sizes} In typical cache
 oblivious algorithms, the problem is recursively split into halves in each
 dimension. This is in fact a rather coarse-grained exploration of the
 hierarchical partitioning of the iteration space. For instance, if the
 current problem size is $B^3$, then the next sub-problem would be
 $(\frac{B}{2})^3$.  If the best problem size for utilizing a level of cache
 is $(B-x)^3$ where $x\ll \frac{B}{2}$ then the subproblems due to
 divide-and-conquer will not match the best.  This is another factor that
 necessitates fine tuning of leaf tile sizes even for COT, since the utilization
 rate of L1 cache has strong impact on performance.  

%\paragraph{COT Leads to Imbalanced Tiles} Current COT tools recursively split
%the problem into halves in each dimension.  If the original bounds are not
%powers of two, every power-of-two leaf will be paired with a non-power-of-two
%leaf.  Since leaf tile sizes are often carefully tuned, thismeans that half
%the leaves will be suboptimal.  Our code generator incorporates a simple
%optimization that ensures that such suboptimal leaf nodes only occur at the
%boundaries of the iteration space.

\paragraph{COT has more Conflict Misses} The divide-and-conquer execution
order may negatively affect cache interference, especially with high
dimensional data.  This happens when the memory is allocated such that the
accesses are contiguous along some direction in the iteration space (typically
along innermost canonical axis).  With lexicographic order of execution, this
contiguity is largely preserved in the tiled execution.  However,
divide-and-conquer executes neighboring tiles in all dimensions, and many of
those tiles access some distant location in memory.  In contrast to accessing
contiguous regions of memory, accessing various segments of the memory
increases the chances of conflicts.

\paragraph{Hardware Prefetching}  Modern architectures are equipped with
hardware prefetchers that can bring data to the L1 cache. When
having sufficient locality at L2 or LLC makes the program compute-bound, then
the latency to L2/LLC can be hidden by the prefetcher. For such programs, it is
unnecessary to tile for the fastest cache, and larger tiles targeting slower
caches improve performance by maximizing prefetcher
effectiveness~\cite{mehta2016turbotiling}. When the primary objective is speed,
the leaf tiles for COT should also be large, which negates the benefit of
divide-and-conquer, as the leafs are already targeting slower caches.
Prefetching have little impact on parallel executions, since prefetching is
bandwidth limited. When multiple cores try to prefetch at the same time,
the bandwidth limit is quickly reached, and the latency hiding effect is
lost. Furthermore, smaller tile sizes are better for parallel execution for
load balancing  reasons.


These factors limit the effectiveness of COT in various ways and are also
closely tied to the characteristics of the computation. Our empirical study
illustrate the impact of these factors on polyhedral computations.

% Local Variables: ***
% TeX-master: "TACO2017.tex" ***
% fill-column: 78 ***
% End: ***


\section{Proposed Load Balancing Strategies}
\label{strategies}
We propose three load balancing strategies: \textit{workload decomposition}, \textit{node splitting} and \textit{hierarchical processing} which we discuss in the following subsections.  
All our strategies implement data-driven GPU executions~\cite{nasre-datavstoplogy-ipdps2013} in which only the \textit{active} elements are processed using a worklist.
%In contrast to a \textit{topology-driven} execution in which all the elements are processed in each iteration, a data-driven execution has been shown to be work-efficient as well as performance-efficient~\cite{merrill-scalablegputraversal-ppopp2012,davidson-sssp-ipdps2014}.

\subsection{Workload Decomposition}\label{workload-decomposition}
%In this work, we employ workload decomposition, a form of space decomposition, for the first time, to our knowledge, for graph applications.
Workload decomposition combines node-based and edge-based task-distribution.
It can be viewed as a form of space decomposition.
In this approach, the processing elements in the worklist continue to be the nodes, but the workload of the nodes, namely, the edges, are decomposed across threads using a block distribution.
$E$ number of graph edges are partitioned across $T$ threads such that each thread receives a contiguous chunk of $E/T$ edges for processing.
%rewrite: Given $E$ edges and $T$ threads, the first $E/T$ edges are assigned to thread 0, the second $E/T$ edges are assigned to thread 1, the third $E/T$ to threads 2, and so on.
Thus, a given thread processes a subset of edges corresponding to a subset of nodes and all the edges outgoing from a node may not be processed by the same thread.
Figure~\ref{wd-illustration} illustrates the workload decomposition strategy with two nodes. 
In this figure, four threads process three edges each from two nodes with the first node having five outgoing edges and the second node having seven edges. 
We find that Thread 2 processes two edges from node 1 and one edge from node 8.

\begin{figure}
\centering
\includegraphics[scale=0.5]{images/rn-wd.pdf}
%\includegraphics[scale=0.5]{images/wd-illustration.pdf}
\caption{Workload Decomposition}
\label{wd-illustration}
\end{figure}

Figure~\ref{WD-cpu-pseudocode} presents the pseudocode for the strategy. 
Each GPU thread processes \textit{edgesPerThread} number of edges starting from a particular outgoing edge of a particular node. 
Each worklist maintains the nodes to be processed and each node's outdegree as two associative arrays.
Both the arrays are populated while updating the worklist, but only the second array (node outdegrees) is used in the prefix sum computation to determine \textit{edgePerThread} and to populate the \textit{offset} array of structure.
%These particular nodes and edges are stored in the \textit{offset} array of structure. 
These offsets are calculated in the GPU in a separate kernel function \textbf{\textit{find\_offsets}} (Lines~\ref{getoffset-line-1} and \ref{getoffset-line-2}). 
A GPU thread in this kernel uses the prefix sums of the outdegrees of the nodes and the \textit{edgesPerThread} to find the particular node and edge that it has to start with for processing. 
The prefix sums of the outdegrees are also calculated on the GPU using a separate kernel. 
We use the NVIDIA's Thrust API for inclusive scan for this purpose (Line~\ref{prefixsum-line}). 
The \textit{while} loop in the kernel (Line~\ref{while-line}) handles the situation when a thread, after processing a subset of edges for a node, moves to processing the next set of edges from the next node in the worklist.

\begin{algorithm}[t]
\begin{small}
\SetKwInOut{Input}{input}
\SetKwInOut{Output}{output}
\SetKwProg{Fn}{}{ \{}{\}}

\Input{a graph \textit{graph(N,E)} with \textit{N} nodes and \textit{E} edges}
%\Input{For SSSP: edgeswt[E] contains the weight of all edges.}
%\Input{prefixsum[N] - stores prefix sum of outdegrees of nodes in the worklist}
\label{offset-line} \Input{\textit{offset[numThreads]} - array of structure containing two fields: (i) \textit{NodeOffset} is the starting offset of a node in the input worklist to be processed by a thread, (ii) \textit{EdgeOffset} is the starting edge of the node to be processed by the thread}

\BlankLine
%worklist structure - contains 2 arrays, 1. Nodenumbers - indices of the nodes, 2. outgoing - outdegree of the nodes \;
\textit{graph}.init()\\
%Create two worklists, $inputWl$ and $outputWl$ \;
%Create distance vector, $dist[N]$ \;
\textit{$\forall$ n: dist[n] = $\infty$} \\
\textit{dist[source] = 0} \\
\textit{inputWl}.push(\textit{source}) \\
%Add $source$ to $inputWl$; Initialize $sizeWl$ with the size of the $inputWl$\;

%\BlankLine
\textit{offsets} = \textbf{\textit{find\_offsets}}(\textit{graph, inputWl, edgesPerThread, numThreads}) \label{getoffset-line-1} \\
%CUDA\_GETOFFSETS ($graph,inputWl,edgePerThr,nthreads,size)$ \; \label{getoffset-line-1} 

\BlankLine
\While{inputWl.size() $>$ 0} {
   \textbf{\textit{sssp\_kernel}}(\textit{graph, dist, inputWl, outputWl, offsets}) \\
   %copy \textit{outputWl} to CPU \\
   \textit{inputWl = outputWl} \\ 
   \textit{outputWl}.clear() \\
   \BlankLine

   \textit{prefixsum = scan(\textit{graph})}	\label{prefixsum-line}	\tcc{Use Thrust API}
   %Do Inclusive Scan of $outgoing$ array and store in $prefixsum$ ; \label{prefixsum-line}
   \textit{edgesPerThread = ceil(prefixsum.size() / outputWl.size())} \\
   %$psum$ = Last value of $prefixsum$ array \;
   %$nthreads =$ size of $outputWl$ \;
   %$nblocks = ceil(nthreads/ blocksize)$ ; \tcc{blockSize = 1024}
   %$edgePerThread = ceil(psum / nthreads)$ \;
   \textit{offsets} = \textbf{\textit{find\_offsets}}(\textit{graph, inputWl, edgesPerThread, numThreads}) \label{getoffset-line-2} \\
   %CUDA\_GETOFFSETS ($graph,inputWl,edgePerThr,nthreads,size)$ \; \label{getoffset-line-2}
}

\BlankLine
\Fn(){\textbf{sssp\_kernel}(graph, dist, inputWl, outputWl, offsets)}{

%Intialize $tid, start, end$ \;
%$size$ = $numThreads$ = size of inputWl \;
%$edgePerThr = ceil(prefixsum[size-1] / numThreads)$ \;
%$source = $ node number located at $offsets[tid].NodeOffset$\;
%$currentindex =$ startindex of $source$ in edge array of CSR $+ offsets[tid].Edgeoffset$ \;
%$lastIndex =$ startIndex of $source$ in edge array of CSR $+$ outdegree of $source$ \;
\textit{source = offsets[myid].NodeOffset} \\
\textit{ecurrent = offsets[myid].EdgeOffset} \\
\For{(\textit{edge = 0; edge $<$ edgesPerThread; ++edge})}{
  \While{ecurrent does not belong to source} {	\label{while-line} \tcc{check next node's edge}
	++\textit{offsets[myid].NodeOffset} \\
     	\textit{source = offsets[myid].NodeOffset} \\
     	\textit{offsets[myid].EdgeOffset = 0} \\
	\textit{ecurrent = offsets[myid].EdgeOffset + first edge index of source} \\
  }
% \While{($currentindex >= lastIndex$)}{
%
%    \tcc{if $currentIndex$ points to an edge of the next node}
%     $offsets[tid].NodeOffset++$ \;
%     $source = $ node number located at $offsets[tid].NodeOffset$\;
%     $offsets[tid].EdgeOffset = 0$ \;
%     $currentindex =$ startindex of $source$ in edge array of CSR $+ offsets[tid].Edgeoffset$ \;
%   }
   %Get $destination$ at $currentindex$ \;
   %Initialize $weight$; 
   \textit{altdist = dist[ecurrent.source] + ecurrent.weight} \\
   \If{dist[ecurrent.destination] $>$ altdist}{
	\textit{dist[ecurrent.destination] = altdist} \\
      %Atomically update $dist[ecurrent.destination]$ with $altDist$ \;
	\textit{outputWl}.push(nodes with updated distance values) \\
      %Atomically add node numbers of the updated nodes to $outputWl.Nodenumbers$ \;
      %Add the outdegree of node to $outputWl.outgoing$ \;                   
   }
   \textit{offsets[myid].EdgeOffset++} \\
   \textit{ecurrent++} \\
}
}
\caption{SSSP Pseudocode illustrating Workload Decomposition}
\label{WD-cpu-pseudocode}
\end{small}
\end{algorithm}

\REM {
\begin{algorithm}
\begin{small}
\SetKwProg{Fn}{}{ \{}{\}}

\Fn(){CUDAKERNEL($dist, graph, inputWl, outputWl$)}{

Intialize $tid, start, end$ \;
$size$ = $numThreads$ = size of inputWl \;
$edgePerThr = ceil(prefixsum[size-1] / numThreads)$ \;
$source = $ node number located at $offsets[tid].NodeOffset$\;
$currentindex =$ startindex of $source$ in edge array of CSR $+ offsets[tid].Edgeoffset$ \;
$lastIndex =$ startIndex of $source$ in edge array of CSR $+$ outdegree of $source$ \;

\For{($edge=0; edge<edgePerThr; edge++$)}{
  \While{($currentindex >= lastIndex$)}{

    \tcc{if $currentIndex$ points to an edge of the next node}
     $offsets[tid].NodeOffset++$ \;
     $source = $ node number located at $offsets[tid].NodeOffset$\;
     $offsets[tid].EdgeOffset = 0$ \;
     $currentindex =$ startindex of $source$ in edge array of CSR $+ offsets[tid].Edgeoffset$ \;
   }
   Get $destination$ at $currentindex$ \;
   Initialize $weight$; $altDist = dist[source] + weight$ \;
   \If{$dist[destination] > altDist$}{
      Atomically update $dist[destination]$ with $altDist$ \;
      Atomically add node numbers of the updated nodes to $outputWl.Nodenumbers$ \;
      Add the outdegree of node to $outputWl.outgoing$ \;                   
   }
   $offsets[tid].EdgeOffset++$ \;
   $currentindex++$ \;
}
}
\caption{GPU Pseudocode for Workload Decomposition}
\label{WD-gpu-pseudocode}
\end{small}
\end{algorithm}
}

An advantage of workload decomposition is that it works with the CSR format and therefore, has a lower space complexity.
%The advantage of this approach is that the load balancing strategy using decomposition of edges in space is natural and easy to implement. 
%Unlike the edge-based assignment strategy, since the processing elements are nodes, larger graphs can be accommodated.  
Assuming a conservative estimate of the number of nodes equal to half the number of edges, graphs of at least 350 million edges can be accommodated with the CSR format on the GPU.  
Also, since it distributes edges of a node across threads, it has a better load-balancing compared to a node-based distribution.
%The drawback of this approach is the need for extra atomic instructions since multiple threads can process the outgoing edges of a single node. 
A drawback of the workload decomposition is that it can lead to uncoalesced accesses since a node's edges may get separated.
The method also incurs some overhead for the prefix sum operations, extra kernel-calls to obtain node offsets, and due to the atomic instructions required as a node may be operated upon by multiple threads.
Despite saving space compared to a COO format, workload decomposition requires extra space to store the node and edge offsets for each thread.

In our experiments, we observe that the limitations of workload decomposition affect its performance for large-diameter graphs (such as the road networks) 
but the method performs very well for scale-free graphs such as the social networks (Section~\ref{exp_res}).



\subsection{Node Splitting}\label{node-splitting}
The second approach we propose is based on changing the graph structure itself to balance the load.
Since the root cause of the load-imbalance is skewed outdegree distribution across graph nodes, node splitting preprocesses the graph to split each high-degree node into multiple low-degree child-nodes.
This approach is implemented as follows.
We define an input parameter called maximum-out-degree-threshold (MDT). 
If a node's outdegree is more than MDT, then the node is split into $\lceil\frac{outdegree}{MDT}\rceil$ nodes, with the outgoing edges of the node distributed evenly among the original (parent) and the split (child) nodes. 
For example, Figure~\ref{fig:ns} depicts our node splitting approach, where a node 8 is split into two nodes, 8 (parent) and 8' (child) which share the outgoing edges.
Multiple child nodes can get formed from a parent node. 
Note that the graph now does not contain the original high-degree node.
By repeating this splitting procedure for each of the high-degree graph nodes, we can obtain another graph with maximum degree bounded by MDT.

\begin{figure}
\centering
\includegraphics[scale=0.48]{images/rn-ns.pdf}
%\includegraphics[scale=0.5]{images/ns.pdf}
\caption{Node Splitting (maximum outdegree threshold MDT = 4)}
\label{fig:ns}
\end{figure}

As evident, node-splitting approach has the advantage that it can work with the space-efficient CSR representation.
Since node-splitting creates duplicates of a node, incoming edges to the original node need special consideration.
To address this issue without increasing any existing node degrees, we maintain the incoming edges of a node only for the parent node.
The parent node, in turn, keeps track of its children.
The algorithm is modified to reflect the attributes (distance in case SSSP) of a parent node onto its children.
This strategy ensures that no new edges get added to the graph (parent-child relationship does not interfere with the normal graph edges).
%To achieve this, every node maintains a count of its children by which it can update the distance of the child nodes. 
%The distance updates to the child nodes are represented as dashed edges (pseudo edges) in Figure~\ref{fig:ns}. 
%Thus, while the original graph representation is augmented with new nodes, no new edges are added to the graph data structure.

%Figure~\ref{NS-cpu-pseudocode} presents the pseudocode for the node-splitting strategy. 
%To identify the nodes to be split, we use a histogram-based method.
%This method determines MDT, the maximum outdegrees of the child nodes after node splitting (Line~\ref{histo-line} of Algorithm~\ref{NS-cpu-pseudocode}). 
%The distances of both the parent and the child nodes are updated in Line~\ref{child-update-line}.
%%We update the distances of both the parent and the child nodes in our GPU function (Line~\ref{child-update-line} of the GPU function).
%Note that similar to that of the parent, the child nodes' distances also need to be updated atomically.

\REM{
\begin{algorithm}[t]
\begin{small}
\SetKwInOut{Input}{input}
\SetKwInOut{Output}{output}
\SetKwProg{Fn}{}{ \{}{\}}

\Input{a graph $graph(N,E)$ with $N$ Nodes and $E$ edges}
\Input{$HistoBinCount$ indicating number of bins}
%\Input{edgesdst[E] - contains the destinations of all edges}
%\Input{For SSSP: edgeswt[E] contains the weight of all edges}

$graph$.init() \\
\BlankLine
%Take command line argument, $HistogramBinCount$ \;
$MDT$ = getMDT($HistoBinCount$) \label{histo-line} \\
%Call getMaxAllowedOutDegree(HistogramBinCount) \; \label{histo-line} 
%Expand graph structure with new nodes, distributing the out-going edges to new nodes, and without adding in-coming edges \;
\For{each node $n$ with degree $>$ maxDegree} {
	$graph$.split($n$) \\
}

%Create two worklists, $inputWl$ and $outputWl$ \;
%Create distance vector, $dist[N]$ \;
$\forall\ n: dist[n] = \infty$ \\
$dist[source] = 0$ \\
$inputWl$.push($source$) \\
\For{each child $c$ of $source$} {
	$dist[c] = 0$ \\
	$inputWl$.push($c$) \\
}
%Add $source$ to $inputWl$; Initialize $sizeWl$ with the size of the $inputWl$\;
%Initialise $dist = \infty$; Initialize $dist[source]$ and  $dist$ of source's children to $0$ \;
%Add $source$ and its children to $inputWl$; Initialize $sizeWl$ with the size of the $inputWl$\;

\BlankLine
\While{$inputWl.size() > 0$}{
   \textbf{\textit{sssp\_kernel}}($graph, dist, inputWl, outputWl$) \\
   %CUDAKERNEL($dist, graph, inputWl, outputWl$) \;
   %copy $outputWl$ to CPU \\
   $inputWl = outputWl$ \\ 
   $outputWl$.clear() \\
}

\BlankLine
\Fn(){\textbf{sssp\_kernel}($graph, dist, inputWl, outputWl$)}{
\For{each node $n$ assigned to me} {	\tcc{uses round-robin node assignment}
  \For{each neighbor $neigh$ of $n$}{
    %Get $destination$ for neighbor \;
    %Initialize $weight$; 
    $altdist = dist[n] + weight(n, neigh)$ \\
    \If{$dist[neigh] > altdist$}{    
      %Get $ChildCount$ of $destination$ \;
	$dist[neigh] = altdist$ \\
	$outputWl.push(neigh)$ \\
	\For{each child $cneigh$ of $neigh$} {
		$dist[cneigh] = altdist$ \label{child-update-line}	\\
		$outputWl.push(cneigh)$ \\
	}
      %Atomically update $dist[destination]$ and its children with $altDist$ \;  \label{child-update-line} 
       %Atomically add node numbers to $outputWl$ \;
    }
  }
   %Increment $tid$ by $gridsize \times blocksize$ ; \tcc{round-robin assignment}
}

}
\caption{SSSP Pseudocode illustrating Node-Splitting}
\label{NS-cpu-pseudocode}
\label{NS-gpu-pseudocode}
\end{small}
\end{algorithm}
}

\REM {
\begin{algorithm}
\begin{small}
\SetKwProg{Fn}{}{ \{}{\}}

\Fn(){CUDAKERNEL($dist, graph, inputWl, outputWl$)}{
Initialize $tid, start, end$ \;
$nodeNumber$ = $start+tid$ \;

\BlankLine
\While{$tid < end$}{
  \For{all neighbors of $nodeNumber$}{
    Get $destination$ for neighbor \;
    Initialize $weight$; $altDist = dist[source] + weight$ \;
    \If{$dist[destination] > altDist$}{    
      Get $ChildCount$ of $destination$ \;
      Atomically update $dist[destination]$ and its children with $altDist$ \;  \label{child-update-line} 
       Atomically add node numbers to $outputWl$ \;
    }
  }
   Increment $tid$ by $gridsize \times blocksize$ ; \tcc{round-robin assignment}
}

}

\caption{GPU Pseudocode for Node-Splitting}
\label{NS-gpu-pseudocode}
\end{small}
\end{algorithm}
}

\begin{figure*}
\centering
\includegraphics[scale=0.4]{images/rn-hp.pdf}
%\includegraphics[scale=0.5]{images/hp-illustration.pdf}
\caption{Hierarchical Processing (MDT = 3, i.e., three edges per node are processed in each iteration)}
\label{hp-illustration}
\end{figure*}

 
\vspace{0.1in}
\noindent
{\bf Automatic Determination of Node Splitting Threshold:}
%The discussion in the previous section explains \textit{how} to perform node-splitting, but it does not tell us \textit{when} to do it.
A salient feature of our node splitting strategy is to automatically determine the threshold MDT for node splitting.
%Let's understand why this automation is non-trivial.
Obvious methods based on a threshold or max-degree etc. do not work in general.
For instance, we cannot fix the value of MDT to a constant, since it is unsuitable across graphs and degree distributions.
We cannot also fix MDT to the maximum degree in the graph, since there could be a big skew in the degree distribution with a few very high degree nodes and a large number of medium degree nodes.
A better alternative is to use the difference between the average degree and the maximum degree in the graph; but such a function would be influenced by the graph size.
Another constraint is that the number of nodes to be split should be minimum possible, to reduce the splitting (and processing) cost.
To account for these issues, 
we use a histogram based method in which we use \textit{HistogramBinCount} number of bins representing the ranges of outdegrees of the nodes in the original graph. 
The number of bins is given as an input parameter to our algorithm. 
We then find the distributions of the outdegrees across the bins. 
We find the bin or range with the maximum height, i.e., the range of outdegrees for which the graph has the maximum number of nodes. 
Let \textit{binIndex} be the index of this bin. 
We then find the maximum degree threshold \textit{MDT} for the outdegrees as \textit{(binIndex/HistogramBinCount)$\times$ maxDegree}, where \textit{maxDegree} is the maximum outdegree in the graph. 
Our node-splitting algorithm then finds the nodes in the graph with outdegrees greater than \textit{MDT} and splits them into child nodes such that each parent and the child nodes will have a maximum of \textit{MDT} outdegrees. 
Our histogram approach of finding \textit{MDT} attempts to maximize the number of nodes (parent and child) with \textit{MDT} outdegrees. 
By choosing the bin with the maximum height in which the nodes already have their outdegrees closer to \textit{MDT}, our algorithm minimizes the amount of splitting. 
Since the histogram-based method considers degree distributions of a graph to achieve load balancing, it can be applied to different graphs with different kinds of distributions including highly skewed distributions.

An advantage of the node splitting approach is that it continues to work with the space-efficient CSR format.
Another advantage over workload-decomposition (Section~\ref{workload-decomposition}) is that all the edges of a node are processed by the same thread, reducing bookkeeping and improving the scope for memory coalescing.
%The advantages of the node-splitting approach is that unlike the workload-decomposition strategy, each thread processes all edges of only one node (parent or child node), thereby simplifying book-keeping. 
The primary disadvantage of node splitting is that it results in extra atomic operations to update the child nodes whenever the parent node gets updated.
A secondary disadvantage is the overhead of computing the histogram to find the MDT.
One may wonder that the strategy has a space overhead due to explicit splitting, but we found in our experiments that less than 5\% of the nodes undergo split, resulting in negligible space overhead.
%the additional space and time complexity for creating the new nodes. If for a graph with $N$ nodes and $E$ edges, $N'$ child nodes are created, the space complexity for the new graph is $N+N'+E$. $N'$ depends on the degree distribution of the graph. In our experiments, we found that in the worst case, $N'=N/2$, while in the average case, $N'$ is less than $5\%$ of $N$. Thus the increase in space complexity is mostly negligible for our experiments. Additional overhead is also incurred to determine the MDT using the histogram-based approach. Node-splitting also results in extra atomic operations to update the child nodes along with the parent nodes.

As we observe in our experiments (Section~\ref{exp_res}), node-splitting provides considerably better load-balancing.
In addition, it provides comparable performance for large diameter graphs (such as road networks); but it has a high overhead for power-law degree distribution graphs.



\subsection{Hierarchical Processing}\label{hierarchical-processing}
Hiearchical processing performs a time-decomposition of the workload.
It achieves this by partitioning the main (super) worklist into several sub-worklists. 
If the sub-worklist is large, it can be further partitioned into sub-sub-worklists, and so on. 
This builds a hierarchy of worklists. 
The depth of this hierarchy is tunable, and we utilize the histogram-based approach in node-splitting (Section~\ref{node-splitting}) for finding the maximum degree threshold (MDT) which determines when to split a worklist into sub-worklists.
%Similar to node-splitting approach, this strategy also uses a maximum degree threshold (MDT) for nodes in a node worklist. 
%The histogram strategy used in the node-splitting approach is also used for finding the maximum degree threshold. 
%However, instead of splitting or forming new nodes, this method processes a worklist in a hierarchical manner forming different sub worklists. 

An iteration for processing a node worklist is composed of sub-iterations. 
In each sub-iteration, a sublist consisting of nodes remaining to be processed from the super-worklist is formed, and a GPU kernel is invoked to process this sublist. 
Each GPU thread considers a set of nodes in the sublist, processing only up to MDT unprocessed outgoing edges of these nodes. 
Thus, all the threads corresponding to the kernel invocation of the sub-iteration are load-balanced within this threshold. 
The nodes in the sublist with the number of unprocessed outgoing edges less than or equal to MDT will be processed. % and removed from the next sublist. 
The next sublist with a reduced set of nodes will be processed in the next sub-iteration. 
This continues until all the nodes in the super list are processed before processing the next super list in the next iteration. 

Figure~\ref{hp-illustration} illustrates the hierarchical processing of sublists within an iteration for an input graph. 
The super worklist in the figure contains two nodes 1 and 8 with five and seven outgoing edges respectively. 
In each kernel for a sub-iteration, two threads are spawned for processing the two nodes.
Let MDT be $3$. Then each thread in a sub-iteration processes maximum three edges.
%The thick edges represent the unprocessed edges while the dashed lines represent the processed edges.
%The shaded portion represents the subgraph processed in each sub-iteration.

%Figure~\ref{HP-cpu-pseudocode} presents the pseudocode for the hierarchical processing strategy. 
%Lines~\ref{subiter-begin-line}~--~\ref{subiter-end-line} show the inner loop corresponding to the sub-iterations. 
%In the kernel code, the creation of the next sub-list for the next sub-iteration is shown in Line~\ref{sublist-creation-line}. 

The sizes of the sublists decrease over the sub-iterations due to the removal of the processed nodes. 
Since the GPU kernel is spawned using a node-parallel approach in which the number of GPU threads is proportional to the size of the sublist, the reduction of the sublist size can result in a situation where the GPU kernel is invoked with a small number of threads to process a small number of nodes in the sublist. 
This will result in very low GPU occupancy. 
For example, consider a situation where, if after a few sub iterations, only one node remains to be processed and this node has $100$ edges remaining to be relaxed. 
If the MDT is $5$, twenty more sub-iterations will invoke twenty more GPU kernels successively, each spawning one GPU thread to process $5$ edges of the node. 
To avoid this situation, our strategy switches to the workload-decomposition technique when the sublist size becomes smaller than a threshold.
% (Line~\ref{switching-line}). 
A natural threshold is governed by the GPU kernel block size which, in our experiments, is set to $1024$. 
We also switch to workload decomposition for processing the super worklist at the beginning of the top-level iterations when the size of the super list becomes smaller than the block size.

The fundamental advantage of the hierarchical processing strategy over node-splitting is that it avoids the space and time complexity needed for creation of new nodes. 
By following a hybrid method of using workload-decomposition for small number of nodes and using the technique of sub-iterations for larger number of nodes, it combines the advantages of multiple approaches. 
However, the hierarchical processing method incurs extra overhead due to additional kernel invocations corresponding to the sub-iterations. 
The method also incurs increased space complexity and atomic operations for the sub worklists.

In our experiments (Section~\ref{exp_res}) we found that despite its overheads, hierarchical processing is a scalable mechanism.
For larger graphs in our experimental suite where other proposed strategies fail to execute due to insufficient memory requirment, hierarchical processing
successfully completes execution offering good benefits in terms of load-balancing.

\REM{
\begin{algorithm}[t]
\begin{small}
\SetKwInOut{Input}{input}
\SetKwInOut{Output}{output}
\SetKwProg{Fn}{}{ \{}{\}}

\Input{A graph $graph(N,E)$ with $N$ Nodes and $E$ edges}
\Input{$HistoBinCount$ indicating number of bins}
%\Input{edgesdst[E] - contains the destinations of all edges}
%\Input{For SSSP: edgeswt[E] contains the weight of all edges}

$graph$.init() \\
%Take command line argument, $HistogramBinCount$ \;
%Create three worklists, $inputWl$, $outputWl$, $subWl$ \;
%Create distance vector, $dist[N]$ \;
%Initialise $dist = \infty$; Initialize $dist[source]$ to $0$ \;
%Add $source$ to $inputWl$; Initialize $sizeWl$ with the size of the $inputWl$\;
%$MDT = getMaxAllowedOutDegree(HistogramBinCount)$ \;
$maxDegree$ = getMaxDegree($HistoBinCount$) \\
$\forall\ n: dist[n] = \infty$ \\
$dist[source] = 0$ \\
$inputWl$.push($source$) \\

\BlankLine
\While{$inputWl.size() > 0$}{
  $subWl$.clear() \\
  $subIteration = 0$ \\
  $continue$ = TRUE \\
  \While{continue == TRUE}{ \label{subiter-begin-line} 
    \If{$subWl.size() < Threshold$}{ \label{switching-line} 
      Call Workload-Decomposition() \tcc{Algorithm~\ref{WD-cpu-pseudocode}}
    }
    $continue$ = FALSE \\
    copy $continue, subIteration$ to GPU \\
    \textbf{\textit{sssp\_kernel}}(\textit{graph, dist, inputWl, outputWl, subWl, continue}) \\
    $inputWl = subWl$ \\
  } \label{subiter-end-line} 
  %copy $outputWl$ to CPU \\
  $inputWl = outputWl$ \\ 
  clear $outputWl$ \\
}
\Fn(){\textbf{sssp\_kernel}(graph, dist, inputWl, outputWl, subWl, continue)}{
\For{each edge $e$ assigned to me} {
	$altdist = dist[e.source] + e.weight$ \\
	\If {$dist[e.destination] > altdist$} {
		$dist[e.destination] = altdist$ \\
		$outputWl.push(e.destination)$ \\
	}
}
\If{the assigned node $n$ has more edges} {
	$continue$ = TRUE \\
	$subWl$.push($n$) \label{sublist-creation-line}
}
}
\caption{SSSP Pseudocode illustrating Hierarchical Processing}
\label{HP-cpu-pseudocode}
\label{HP-gpu-pseudocode}
\end{small}
\end{algorithm}

\REM {
\begin{algorithm}
\begin{small}
\SetKwProg{Fn}{}{ \{}{\}}

\Fn(){CUDAKERNEL($dist, graph, inputWl, outputWl, subWl,$ \\$subIteration, continue, MDT$)}{
Initialize $tid, start, end$ \;
$loffset$ = $MDT$ \;

\BlankLine
\If{$tid < end$}{
  $NodeNumber = start+tid$ \;
  $NodeIndex =$ get index of $NodeNumber$ from CSR \;
  $currentindex = NodeIndex + (subIteration * loffset)$ \;
  $lastindex = NodeIndex + outdegree$ \;
  \For{($E=currentindex; E<(currentindex+loffset), E<lastindex; E++$)}{
    Get $destination$ for $E$ \;
    Initialize $weight$; $altDist = dist[source] + weight$ \;
    \If{$dist[destination] > altDist$}{
       Atomically update $dist[destination]$ with $altDist$ \;
       Atomically add $destination$ to $outputWl$ \;
    }
  }
  \If{$currindex + loffset < lastindex$}{
    $continue = TRUE$ \;
    Atomically add $NodeNumber$ to $subWl$ ; \label{sublist-creation-line} \tcc{More edges to be processed for this node. Add node to the next sublist}
  }
}
}

\caption{GPU Pseudocode for Hierarchical-Processing}
\label{HP-gpu-pseudocode}
\end{small}
\end{algorithm}
}
}

Table \ref{strategies-summary} summarizes the advantages and disadvantages of the different load balancing strategies.

\begin{table*}
 \centering
 \footnotesize
 \caption{Advantages and Disadvantages of the Load Balancing Strategies}
 \begin{tabular}{|c|p{1.45in}|p{2.45in}|p{2.55in}|}
  \hline\hline
  & {\em Strategy} & {\em Advantages} & {\em Disadvantages} \\
  \hline\hline
  \parbox[t]{2mm}{\multirow{2}{*}{\rotatebox{90}{Existing}}} &
  Node-based Distribution (BS) &
   \parbox{2.5in} {
	\begin{itemize}
	\item Simple to implement (static)
	\item  Works with CSR graph format
   	\end{itemize} 
   } &
   \parbox{2.5in} {
   	\begin{itemize}
	\item High load-imbalance
   	\end{itemize} 
   } \\\cline{2-4}
   &
  Edge-based Distribution (EP) & 
   \parbox{2.5in} {
   	\begin{itemize}
	\item Implicit load balancing 
	\item Simple to implement (static)
   	\end{itemize} 
   } &
  \parbox{2.5in}{
  \begin{itemize}
   \item Large space complexity for COO representation
   \item Explosion in worklist size, worklist condensing overhead, large memory consumption
   \item Requires the kernel operation to be distributive
  \end{itemize}
  } \\ \hline
  \parbox[t]{2mm}{\multirow{3}{*}{\rotatebox{90}{Proposed}}} &
  Workload Decomposition (WD) & 
  \parbox{2.5in}{
  \begin{itemize}
   \item Larger graphs can be processed
   \item Space decomposition is easy to implement
  \end{itemize}
  } & 
  \parbox{2.5in}{
  \begin{itemize}
   %\item A thread can process multiple nodes
   \item Atomic operations for updating same out-going edges by multiple threads
   \item Overheads for prefix sum and offset computations, extra space for offsets
   \item  Uncoalesced data access on the GPU
  \end{itemize}
  } \\ \cline{2-4}
  &
  Node Splitting (NS) & 
  \parbox{2.5in}{
   \begin{itemize}
	\item Graph algorithm does not require modification
   \end{itemize} 
  } &
  \parbox{2.5in}{
  \begin{itemize} 
   \item Additional space and time complexity for new nodes
   \item Extra atomic operations for updating child nodes
   \item  Overhead for MDT finding
  \end{itemize}
  } \\ \cline{2-4}
  &
  Hierarchical Processing (HP) & 
  \parbox{2.5in}{
  \begin{itemize}
   \item Performs well for large graphs
   \item  A thread processes only one node without forming child nodes
   \item  Hybrid method for switching to workload decomposition strategy for small super and sub worklists
  \end{itemize}
  } & 
  \parbox{2.5in}{
  \begin{itemize}
   \item Sub lists result in additional space and atomics
   \item Multiple kernel calls
  \end{itemize}
  } \\ \hline\hline
 \end{tabular}
 \label{strategies-summary}
\end{table*}





\section{Experiments and Results}
\label{exp_res}

To assess the effectiveness of our proposed techniques, we embed them in the implementation of two graph algorithms: breadth-first search computation (BFS) and single-source shortest paths computation (SSSP).
Both these algorithms are fundamental to several domains and form the building block for several interesting applications.
We compare our implementations against the LonestarGPU benchmark implementations~\cite{lonestargpu-web},
which use a node-based task-distribution.
%and integrated our load balancing strategies into the baseline implementations. 
We used LonestarGPU-1.02 version, an older version available at the time of our work. 
For our experiments, we use both synthetically generated graphs as well as the real-world graphs.
The synthetic ones are the RMAT graphs based on the recursive matrix model~\cite{gtgraph} and random graphs based on the Erd\H{o}s-R\'enyi model (denoted as ER*). 
Both are generated using GTgraph~\cite{gtgraph}.
For real-world graphs, we use the USA road networks (for West, Florida and overall USA).
To assess scalability, we include three relatively larger graphs obtained using the graph generation tool available in the Graph500 benchmark~\cite{graph500-web}. 
%For all three graphs, the number of nodes is 16.78 million and the number of edges is 335 million, with the average degree of the nodes being 20.  
The tool accepts three parameters: number of nodes, number of edges and a seed value for random number generation, and generates a corresponding graph.
Depending upon the seed value, the graph connectivity differs.
%For all the graphs, the minimum degree is 0, maximum degree is about 924000, and standard deviation of degrees is about 20900.
The properties of all the graphs are given in Table~\ref{graph_properties}. 
The last column of the table represents the amount of load imbalance in the form of the skewness in the outdegree distribution of the nodes.
This is indicated in terms of the maximum, average and standard deviations ($\sigma$) of their outdegrees.

We observe in Table~\ref{graph_properties} that Graph500 and RMAT graphs have a high maximum degree as well as a lot of variance in the number of outdegrees.
RMAT graphs are also characterized by \textit{small-world property} due to their low diameter.
In contrast, the road networks have very small maximum degree and little variation in the outdegree distribution. 
They have large diameters (not shown) in comparison to the RMAT and ER graphs.
ER graphs have a random distribution of edges in the graph and have a higher max-degree as well as the standard deviation than road networks.
However, they do not have large diameters as in the case of road networks, nor do they exhibit the small-world property.
It should be noted that despite this variance, the average degrees of all the graphs remain comparable.
Together these graphs test various aspects of our strategies.
%Each graph has the minimum outdegree as $0$ or $1$.

\begin{table}
\centering
\caption{Graphs Used in our Experiments}
\footnotesize
\begin{tabular}{|r|r|r|rrr|}
%\begin{tabular}{|p{0.9in}|p{0.45in}|p{0.45in}|p{0.9in}|}
\hline%\hline
{\em Graph} 	& {\em Nodes} 		& {\em Edges} 		& \multicolumn{3}{c|}{\em Outdegrees}	\\
 		& {\em (Million)}	& {\em (Million)}	& {\em Max}	& {\em Avg}	& $\sigma$ \\
\hline\hline
rmat20 & 1.05 & 8.26 & 1,181 & 8 & 177.40 \\
\hline
road-FLA & 1.07 & 2.71 & 8 & 3 & 2.45 \\
road-W & 6.26 & 15.12 & 9 & 4 & 2.74 \\
road-USA & 23.95 & 57.71 & 9 & 3 & 2.74 \\
\hline
ER20 & 1.05 & 4.19 & 15 & 4 & 4.47 \\
ER23 & 8.39 & 33.55 & 10 & 3 & 4.46 \\
\hline
Graph500 & 16.78 & 335.00 & 924,000 & 20 & 20,900 \\
(three graphs) &  &  &  &  &  \\
\hline%\hline
\end{tabular}
\label{graph_properties}
\end{table}

We implemented all our proposed strategies using CUDA.
Our experiments were performed on a Kepler-based GPU system. 
The GPU has a Tesla K20c architecture with 13 SMXs each having 192 CUDA cores (2,496 CUDA cores totally) with 4.66 GB of main memory, 1 MB of L2 cache and 64 KB of registers per SM.  
It has a configurable 64 KB of fast memory per SMX that is split between the L1 data cache and shared memory. 
The programs have been compiled using \textit{nvcc} version 5.0 with \textit{-O3 -arch=sm\_35} flags. 
The CPU is a hex-core Intel Xeon E5-2620 2.0 GHz workstation with CentOS 6.4, 16 GB RAM and 500 GB hard disk.

\newcommand{\figurewidth}{0.25}

\begin {figure*}
\centering
\subfigure[Low Diameter Graphs]{
  \includegraphics[width=\figurewidth\linewidth]{images/sssp-synthetic}
  %\includegraphics[scale=0.2]{images/AllRGraphs_sssp.pdf}
  \label{rgraphs_sssp}
}
\subfigure[Road Networks]{
  \includegraphics[width=\figurewidth\linewidth]{images/sssp-road}
  \label{USA-road-d_FLA_sssp}
  \label{USA-road-d_W_sssp}
  \label{USA-road-d_USA_sssp}
}
\subfigure[Graph500 Graphs]{
  \includegraphics[width=\figurewidth\linewidth]{images/sssp-graph500}
  \label{very_large_sssp}
}
\caption{Comparison of Load Balancing Strategies for \textbf{SSSP}} %\todo{explanation of baseline kernel time being high}}
\label{overall-comparisons-sssp}
\end{figure*}

\begin {figure*}
\centering
\subfigure[Low Diameter Graphs]{
  \includegraphics[width=\figurewidth\linewidth]{images/bfs-synthetic}
  \label{rgraphs_bfs}
}
\subfigure[Road Networks]{
  \includegraphics[width=\figurewidth\linewidth]{images/bfs-road}
  \label{USA-road-d_FLA_bfs}
  \label{USA-road-d_W_bfs}
  \label{USA-road-d_USA_bfs}
}
\subfigure[Graph500 Graphs]{
  \includegraphics[width=\figurewidth\linewidth]{images/bfs-graph500}
  \label{very_large_bfs}
}
\caption{Comparison of Load Balancing Strategies for \textbf{BFS}}
\label{overall-comparisons-bfs}
\label{very_large_graph_results}
\end{figure*}

\REM {
\begin {figure}
\centering
\subfigure[Synthetic Graphs]{
  \includegraphics[scale=0.2]{images/AllRGraphs_bfs.pdf}
  \label{rgraphs_bfs}
}
\subfigure[road-FLA]{
  \includegraphics[scale=0.2]{images/USA-road-d_FLA_bfs.pdf}
  \label{USA-road-d_FLA_bfs}
}
\subfigure[road-W]{
  \includegraphics[scale=0.2]{images/USA-road-d_W_bfs.pdf}
  \label{USA-road-d_W_bfs}
}
\subfigure[road-USA]{
  \includegraphics[scale=0.2]{images/USA-road-d_USA_bfs.pdf}
  \label{USA-road-d_USA_bfs}
}
\caption{Comparison of Load Balancing Strategies - BFS}
\label{overall-comparisons-bfs}
\end{figure}
}

\subsection{Performance Comparison of Strategies}\label{expt:exectime}

In this section, we compare the various strategies in terms of performance or execution times. The strategies are denoted as BS for LoneStar GPU baseline version that implements node-based task-distribution, EP for edge-based distribution, WD for workload decomposition, NS for node-splitting, and HP for hierarchical processing. We split the overall execution time into useful kernel time and the overhead associated with implementing a strategy. The overheads encompass all the corresponding initializations, extra kernel invocations and bookkeeping. Note that BS also has an overhead component.

Figure \ref{overall-comparisons-sssp} shows the comparison results for SSSP. We find that all our strategies perform significantly better than the baseline (BS) method in almost all cases, for graphs with small as well as large diameters. This is because in SSSP, especially for the graphs with large diameters, the kernel times dominate the overheads (unlike in BFS, discussed below).  This shows that our load balancing strategies in particular, and load balancing in general, are fruitful for applications that perform even a reasonable amount of computations. We believe the techniques would be more useful for computation-intensive irregular applications. The edge-based parallelism (EP) method performs the best, giving 60--80\% smaller execution times than the baseline. Unfortunately, due to its high storage requirement, EP is unable to run on larger graphs such as Graph500. Among the node-based strategies, workload decomposition (WD) method performs the best for graphs with highly skewed or random degree distribution. For such graphs (RMAT and ER), the node splitting (NS) performs the worst since its node creation overhead coupled with highly skewed degree distribution dominates the kernel times. However, when the deviation in the size of the neighborhood is less, the NS method performs the best among the node-based strategies since its node creation overhead is a one-time cost and is amortized by relatively large total kernel execution times.

Hierarchical processing (HP) performs in between the WD and NS methods for smaller graphs. However, the main advantage of HP is seen in dealing with larger graphs such as Graph500. At the time of writing this paper, we were able to execute only the HP strategy of the three load balancing strategies (WD,NS and HP) for these large graphs.
% Apart from BS, only HP is able to complete execution on Graph500 graphs without running out of memory.
As mentioned, the edge-based parallelism (EP) has a high storage complexity related to storing the edges and hence cannot be executed for these large graphs.
% The workload decomposition (WD) strategy could not be executed since its additional memory requirements related to node and edge offsets, and prefix sum arrays could not be accommodated for these graphs. The node splitting (NS) strategy's additional memory requirements for storing the additional child nodes could not be accommodated as well.
We find that the HP method gives large improvements resulting in 48-75\% reduction in execution times for these large graphs. Thus the HP method will have larger importance as we explore real-world BigData graphs.


\REM {
\begin {figure}
\centering
\subfigure[BFS]{
  \includegraphics[scale=0.2]{images/very_large_bfs.pdf}
  \label{very_large_bfs}
}
\subfigure[SSSP]{
  \includegraphics[scale=0.2]{images/very_large_sssp.pdf}
  \label{very_large_sssp}
}
\caption{Results for Very Large Graphs}
\label{very_large_graph_results}
\end{figure}
}

\REM {
Following are the overheads reported for the different strategies: \\
1. Baseline (BS): initialization of worklist, and total time for the main loop excluding GPU kernel executions. \\
2. Edge-based parallelism (EP): initialization of worklist, and total time for the main loop excluding GPU kernel executions. \\
3. Workload Decomposition (WD): initialization of worklist, CUDA kernel for obtaining the offsets before the main loop, and total time for the main loop excluding GPU kernel executions. \\
4. Node Splitting (NS): finding node-splitting level using histogram approach and 10 bins, initialization of worklist, and total time for the main loop excluding GPU kernel executions. \\
5. Hierarchical processing (HP): finding node-splitting level using histogram approach and 10 bins, creating space for extra worklists, initialization of worklist, creating GPU memory for allocating worklists and total time for the main loop excluding GPU kernel executions.
}

Figure \ref{overall-comparisons-bfs} shows the comparison results for BFS. It is noteworthy that BFS is a memory-bound kernel, and it performs only a little computation. Therefore, we observe the associated overheads are large in general, unlike in SSSP where the overheads were lesser than the computations. Only when the graphs get sufficiently bigger, do the overheads amortize. The EP method, similar to SSSP, consistently delivers better performance than BS. However, its high storage requirements could not be accommodated for the large-sized Graph500 graphs. For the graphs with small diameters, namely the RMAT and ER graphs, the execution time with EP is 48--68\% lesser than that of BS (0.17 MTEPS (BS) vs. 0.54 MTEPS (EP) for RMAT20). For the graphs with large diameters, namely, the road networks, the maximum performance gain with EP over BS is about 10\%.  

Similar to SSSP results, the WD method performs the best among the node-based approaches for graphs with small diameters for the BFS application. The NS method involves considerable overhead for these graphs. For graphs with large diameters, the NS method performs the best since it incurs lesser one-time overheads. In case of relatively larger graphs such as Graph500, HP performs considerably ($>$2$\times$) better than BS, while the EP method fails
% other approaches (EP, NS and WD) fail
to complete execution due to insufficient memory.

% Though the EP and WD methods both provide equal load to all threads, the WD method requires many more global memory accesses that are uncoalesced due to which the kernel execution times are larger with the WD method. The overheads associated with WD method, namely, inclusive scan and finding offsets are not present in the EP method. The kernel execution times of the HP method are consistently more (about 5--27\% more) than in the NS method due to extra kernel invocations which leads to more global memory accesses and atomic operations to maintain the sub iteration worklist. However, for graphs with smaller diameters, the large overheads due to the formation of extra nodes in the NS method results in the HP method giving significantly smaller (about 65\% smaller) total execution times when compared to the NS method. For graphs with large diameters, the overheads in the HP method are also greater than in the NS method due to the prominent number of memory copies in the HP method due to many kernel invocations while the NS method incurs only a one-time overhead for the formation of the nodes. Hence the NS methods gives 30--38\% smaller total execution times than the HP method for these graphs. For road networks, though the kernel execution times in HP are 7--12\% smaller than in WD, the total execution times are 22--31\% greater due to larger overhead of HP.

\subsection{Performance, Space Complexity and Implementation Tradeoffs}

\begin{figure}
\centering
\includegraphics[width=0.5\linewidth]{images/all-by-dimension-tight.pdf}
\caption{Overall comparison of strategies. Each axis represents a ranked order. A strategy closer to the origin (at the center) is ranked higher.}
\label{expt:overall}
\end{figure}

While the previous section focused on the performance aspects of the strategies, in this section we compare the strategies in terms of three axes of comparison: (i) execution time, (ii) memory requirement, and (iii) implementation complexity. The first two are quantitative, while the last one is a qualitative assessment out of our experience. Figure~\ref{expt:overall} shows the relative rankings of the strategies in terms of the three aspects. In each axis, a strategy that is closer to the origin is superior in terms of the corresponding factor.

Overall, it is clear that no one technique fares in all aspects. This suggests that we may have to use different load balancing strategies depending upon the performance requirements, amount of GPU memory and personnel expertise available. Despite the lack of a clear winner, Edge-based Processing (EP) ranks better on two axes (Execution Time and Implementation Complexity). This makes it a more desirable option when the amount of memory is not an issue. 
%We believe our study explains why recent implementations focus on EP~\cite{sariyuce-bc-gpgpu2013}.
Node-based processing (which we call as baseline BS) is also easy to implement, and has a low memory requirement (due to CSR representation), but in our experience, performs the worst. Another useful choice could be Hierarchical Processing (HP). It incurs lesser memory penalty and has a moderate implementation complexity. HP does not fare well in terms of performance for small graphs but performs well for large graphs.
% (while the other methods could not be executed for these large graphs due to high memory complexity).
Workload Decomposition (WD) and Node Splitting (NS) could be the methods of choice when performance is more important but memory is insufficient to execute EP. NS (implemented as a static phase) is likely to perform better than WD despite graph modification, but incurs larger memory overhead.

\subsection{Degree Distribution due to Node Splitting}
%The NS method computes the threshold for node-splitting (i.e., maximum degree threshold MDT) using histogram-based approach.
%In our node-splitting strategy, our histogram-based method automatically determines the maximum degree threshold (MDT) or correct level of node splitting. 
The NS method modifies the graph by creating child-nodes to distribute the outdegrees.
Therefore, the degree distribution in the modified graph differs from the original.
Figure~\ref{ns-degreedist} shows the distribution of out-degrees of the nodes before (red curve) and after (green curve) node splitting for two synthetic graphs.
The maximum degree thresholds (MDTs) determined using the histogram approach are also shown. 
It is evident from the figure that NS achieves a better load balancing by confining all the nodes to outdegrees within a small range (represented by green curves). We obtained similar results for the other graphs.
%We find that our histogram-based approach effectively leads to load balancing using node splitting by confining all the nodes to degree distribution within a small range (represented by the green curves).
It should also be noted that by exploiting histogramming, the MDT does not get biased to a range based on the graph size. %or degree distribution.
For instance, for road networks and random graphs, MDT is 2--4 whereas for RMAT graph, it gets rightly computed as 118.

\begin {figure}
\centering
\subfigure[rmat20. MDT=118]{
  \includegraphics[scale=0.2]{images/rmat20_degNS.pdf}
  \label{rmat20_degNS}
}
\subfigure[ER23. MDT=3]{
  \includegraphics[scale=0.2]{images/r4-2e23_degNS.pdf}
  \label{r42e23_degNS}
}
\caption{Degree Distributions of Graphs Before and After Node Splitting}
\label{ns-degreedist}
\end{figure}

%\subsection{Very Large Graphs}
%\todo{merge with earlier subsections}
%Although the edge-based parallelism (EP) method gave the best results in all cases, as mentioned earlier, one of the disadvantages of this method is that it cannot accommodate very large-size graphs since it follows the COO representation for edge processing. We experimented with three such large graphs. These graphs were obtained using the graph generation tool available in the Graph500 benchmark. For all three graphs, the number of nodes is 16.78 million and the number of edges is 335 million, with the average degree of the nodes being 20.  The graphs are generated using a random number generator provided in the Graph500 Benchmark.  A seed is provided to the generator that generates a graph with the mentioned number of nodes and edges with different connectivity. For all the graphs, the minimum degree is 0, maximum degree is about 924000, and standard deviation of degrees is about 20900. 

\subsection{Work Chunking Optimization for Edge-based Parallelism}
While using atomic operations to add edges to the worklist in the GPU kernel, we use work chunking in which we collect all edges of a node and add them together using a single atomic operation. 
We compare this strategy with the default strategy of using an atomic operation for adding every edge. 
Figure~\ref{work-chunk-table} shows the speedups obtained due to work chunking EP method over the the default EP method.
We find that work chunking results in 1.11--3.125, with an average of 1.82, speedups over the default method.

\REM {
\begin{table}
\small
\centering
\caption{Benefits of Work Chunking for Reducing Atomic Operations}
\begin{tabular}{|>{\hfill}p{0.55in}|>{\hfill}p{0.45in}|>{\hfill}p{0.5in}|>{\hfill}p{0.55in}|>{\hfill}p{0.55in}|}
\hline %\hline
{\em Graph} & \multicolumn{2}{|c|}{\em BFS} & \multicolumn{2}{|c|}{\em SSSP} \\
\cline{2-5}
& EP (msecs) & EP$_{\mbox{chunk}}$  (msecs) &  EP (msecs) & EP$_{\mbox{chunk}}$ (msecs) \\
\hline \hline
rmat20 & 36.10 & 14.84 & 297.75 & 95.23 \\
\hline
road-FLA & 592.08 & 597.24 & 3535.33 & 2479.11 \\
road-W & 993.44 & 990.82 & 18421.72 & 11790.62 \\
road-USA & 2116.23 & 1865.22 & 228329.00 & 125358.55 \\
\hline
ER20 & 20.47 & 12.62 & 61.94 & 37.19 \\
ER23 & 162.57 & 84.93 & 674.17 & 292.9 \\
%\hline
%\todo{Graph500} &  &  & & \\
\hline %\hline
\end{tabular}
\label{work-chunk-table}
\end{table}
}
\begin{figure}
\centering
  \includegraphics[width=0.70\linewidth]{images/sssp-bfs-chunking}
  \caption{Benefits of Work Chunking in Edge-based Processing}
  %\caption{Benefits of Work Chunking due to Reduced Atomic Operations}
  \label{work-chunk-table}
\end{figure}

\textbf{Related work}:
% Object detection related datasets/algo in non-medical domain
% Locally labeled CXR dataset
A few CXR datasets have localized abnormality annotations \cite{shih2019augmenting,filice2020crowdsourcing,jaeger2014two} that are curated manually. These are high quality gold standard ground truth datasets but tend to be smaller in scale (< 30,000 images) and have a narrow coverage, with typically only 1-2 labels. In addition, since most labeling efforts only have abnormality semantics attached, no direct relationships with the affected anatomical locations are available. 

%MEHDI: repeated concepts from above. I am removing the following: 

%The lack of anatomic semantics in the annotation is a limitation for complex multi-modal clinical reasoning work, e.g., differential diagnosis, since clinicians often integrate information along anatomical lines, and for downstream report generation tasks, which often requires describing not only the abnormality but also correctly communicate the location of the abnormalities (and medical devices) to the receiving clinicians. 

Two recent CXR datasets have labels for anatomies described in the reports. In \cite{datta2020dataset}, a small manually annotated dataset (2000 reports) included 10 abnormalities that are individually associated with 29 unique spatial locations (anatomies) at the report level. Another CXR dataset has automatically extracted abnormality and anatomy labels as disconnected concepts that are only correlated at the study level from  160,000 reports using a supervised NLP algorithm \cite{bustos2020padchest}. This was trained on a smaller set of manually annotated data. Neither datasets contain localized annotations for the associated CXR images, nor any comparison relation annotations between sequential exams, both of which are available in the Chest ImaGenome dataset. In Table \ref{tab:related}, we present a comparison of our Chest ImagGenome dataset with other datasets available in the literature.

% Table -- Kashyap

% MEdical imaging datasets to go here: Discussed that we will only focus on cxr datasets that are available for this paper. 
% \caption{\color{red} Kashyap, feel free to continue with the table. We should remove the questionmarks and add a line for our dataset (since all others are not graph). For longer text, using abbreviations and explaining them in the caption often works better. If fill in the values is not possible, it is better to remove the table altogether.}


\begin{table}[t!]
\caption{Summary of existing chest X-ray datasets}
\resizebox{\textwidth}{!}{%
\begin{tabular}{@{}lllllllll@{}}
\toprule
\textbf{Dataset} & \textbf{Annotation Level} & \textbf{Annotation Method} & \textbf{Num Labels} & \textbf{Anatomy Labeled} & \textbf{Graph} & \textbf{Dataset Size} & \textbf{Temporal Labels} & \textbf{Reports} \\ \midrule
SIIM-ACR Pneumothorax Segmentation \cite{filice2020crowdsourcing} & Segmentation & Manual + augmented & 1 & No & No & 12,047 & No & No \\
RSNA Pneumonia Detection Challenge   \cite{shih2019augmenting} & Bounding Boxes & Manual & 1 & No & No & 30,000 & No & No \\
Indiana University Chest X-ray collection \cite{demner2016preparing} & Global & Automated & 10 & No & No & 3,813 & No & Yes \\
NIH CXR dataset \cite{wang2017chestx} & Global & Automated & 14 & No & No & 112,120 & No & No \\
PLCO \cite{team2000prostate} & Global & Automated & 24 & Yes & No & 236,000 & Yes & No \\
Stanford CheXpert \cite{irvin2019chexpert} & Global & Automated & 14 & No & No & 224,316 & No & No \\
MIMIC-CXR \cite{johnson2019mimic} & Global & Automated & 14 & No & No & 377,110 & No & Yes \\
Dutta \cite{datta2020dataset} & Global & Manual & 10 & Yes & Yes & 2,000 & No & Yes \\
PadChest \cite{bustos2020padchest} & Global & Manual + automated & 297 & Yes & No & 160,868 & No & Yes \\
Montgomery County Chest X-ray   \cite{jaeger2014two} & Segmentation & Manual & 1 & Yes & No & 138 & No & No \\
Shenzen Hospital Chest X-ray   \cite{jaeger2014two} & Segmentation & Manual & 1 & Yes & No & 662 & No & No \\  \hline \hline
\textbf{Chest ImaGenome} & Bounding Boxes & Automated & 131 & Yes & Yes & 242,072 & Yes & Yes \\
\bottomrule
\end{tabular}%
}
\label{tab:related}
\vspace{-0.4cm}
\end{table}
% removed (Derived from MIMIC-CXR \cite{johnson2019mimic}) % makes table really small


\section{Conclusions and Future Work}
\label{con_fut}

In this paper, we had evaluated four load balancing strategies for BFS and SSSP applications for different graphs. We found that the edge-based processing method performs the best giving about 10\% better performance than the baseline for BFS, and about 60-80\% better performance than the baseline for SSSP. Among the node-based strategies, the workload decomposition method performs the best for graphs with small diameters while the node splitting method performs the best for graphs with large diameters. While the node-based strategies gave worse performance than the baseline in BFS, all our load balancing strategies gave significantly better results (at least 20\% better) than the baseline for SSSP. This shows that load balancing becomes very essential for computationally-intensive graph applications especially for large graphs.
For very large graphs in which some of our load balancing strategies cannot be executed due to memory constraints, our novel hierarchical processing method proposed in this work gives 48-75\% reduction in execution time compared to the baseline.
In future, we plan to explore our strategies for other graph applications including minimum spanning tree and betweenness centrality applications. We also plan to explore dynamic parallelism offered by modern GPU architectures for load balancing graphs. Finally, we plan to build data reorganization strategies for improved coalescing.

\bibliographystyle{IEEEtran}
\bibliography{loadbalancing,others}

\end{document}
