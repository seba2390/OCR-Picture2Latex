\subsection{Node Splitting}\label{node-splitting}
The second approach we propose is based on changing the graph structure itself to balance the load.
Since the root cause of the load-imbalance is skewed outdegree distribution across graph nodes, node splitting preprocesses the graph to split each high-degree node into multiple low-degree child-nodes.
This approach is implemented as follows.
We define an input parameter called maximum-out-degree-threshold (MDT). 
If a node's outdegree is more than MDT, then the node is split into $\lceil\frac{outdegree}{MDT}\rceil$ nodes, with the outgoing edges of the node distributed evenly among the original (parent) and the split (child) nodes. 
For example, Figure~\ref{fig:ns} depicts our node splitting approach, where a node 8 is split into two nodes, 8 (parent) and 8' (child) which share the outgoing edges.
Multiple child nodes can get formed from a parent node. 
Note that the graph now does not contain the original high-degree node.
By repeating this splitting procedure for each of the high-degree graph nodes, we can obtain another graph with maximum degree bounded by MDT.

\begin{figure}
\centering
\includegraphics[scale=0.48]{images/rn-ns.pdf}
%\includegraphics[scale=0.5]{images/ns.pdf}
\caption{Node Splitting (maximum outdegree threshold MDT = 4)}
\label{fig:ns}
\end{figure}

As evident, node-splitting approach has the advantage that it can work with the space-efficient CSR representation.
Since node-splitting creates duplicates of a node, incoming edges to the original node need special consideration.
To address this issue without increasing any existing node degrees, we maintain the incoming edges of a node only for the parent node.
The parent node, in turn, keeps track of its children.
The algorithm is modified to reflect the attributes (distance in case SSSP) of a parent node onto its children.
This strategy ensures that no new edges get added to the graph (parent-child relationship does not interfere with the normal graph edges).
%To achieve this, every node maintains a count of its children by which it can update the distance of the child nodes. 
%The distance updates to the child nodes are represented as dashed edges (pseudo edges) in Figure~\ref{fig:ns}. 
%Thus, while the original graph representation is augmented with new nodes, no new edges are added to the graph data structure.

%Figure~\ref{NS-cpu-pseudocode} presents the pseudocode for the node-splitting strategy. 
%To identify the nodes to be split, we use a histogram-based method.
%This method determines MDT, the maximum outdegrees of the child nodes after node splitting (Line~\ref{histo-line} of Algorithm~\ref{NS-cpu-pseudocode}). 
%The distances of both the parent and the child nodes are updated in Line~\ref{child-update-line}.
%%We update the distances of both the parent and the child nodes in our GPU function (Line~\ref{child-update-line} of the GPU function).
%Note that similar to that of the parent, the child nodes' distances also need to be updated atomically.

\REM{
\begin{algorithm}[t]
\begin{small}
\SetKwInOut{Input}{input}
\SetKwInOut{Output}{output}
\SetKwProg{Fn}{}{ \{}{\}}

\Input{a graph $graph(N,E)$ with $N$ Nodes and $E$ edges}
\Input{$HistoBinCount$ indicating number of bins}
%\Input{edgesdst[E] - contains the destinations of all edges}
%\Input{For SSSP: edgeswt[E] contains the weight of all edges}

$graph$.init() \\
\BlankLine
%Take command line argument, $HistogramBinCount$ \;
$MDT$ = getMDT($HistoBinCount$) \label{histo-line} \\
%Call getMaxAllowedOutDegree(HistogramBinCount) \; \label{histo-line} 
%Expand graph structure with new nodes, distributing the out-going edges to new nodes, and without adding in-coming edges \;
\For{each node $n$ with degree $>$ maxDegree} {
	$graph$.split($n$) \\
}

%Create two worklists, $inputWl$ and $outputWl$ \;
%Create distance vector, $dist[N]$ \;
$\forall\ n: dist[n] = \infty$ \\
$dist[source] = 0$ \\
$inputWl$.push($source$) \\
\For{each child $c$ of $source$} {
	$dist[c] = 0$ \\
	$inputWl$.push($c$) \\
}
%Add $source$ to $inputWl$; Initialize $sizeWl$ with the size of the $inputWl$\;
%Initialise $dist = \infty$; Initialize $dist[source]$ and  $dist$ of source's children to $0$ \;
%Add $source$ and its children to $inputWl$; Initialize $sizeWl$ with the size of the $inputWl$\;

\BlankLine
\While{$inputWl.size() > 0$}{
   \textbf{\textit{sssp\_kernel}}($graph, dist, inputWl, outputWl$) \\
   %CUDAKERNEL($dist, graph, inputWl, outputWl$) \;
   %copy $outputWl$ to CPU \\
   $inputWl = outputWl$ \\ 
   $outputWl$.clear() \\
}

\BlankLine
\Fn(){\textbf{sssp\_kernel}($graph, dist, inputWl, outputWl$)}{
\For{each node $n$ assigned to me} {	\tcc{uses round-robin node assignment}
  \For{each neighbor $neigh$ of $n$}{
    %Get $destination$ for neighbor \;
    %Initialize $weight$; 
    $altdist = dist[n] + weight(n, neigh)$ \\
    \If{$dist[neigh] > altdist$}{    
      %Get $ChildCount$ of $destination$ \;
	$dist[neigh] = altdist$ \\
	$outputWl.push(neigh)$ \\
	\For{each child $cneigh$ of $neigh$} {
		$dist[cneigh] = altdist$ \label{child-update-line}	\\
		$outputWl.push(cneigh)$ \\
	}
      %Atomically update $dist[destination]$ and its children with $altDist$ \;  \label{child-update-line} 
       %Atomically add node numbers to $outputWl$ \;
    }
  }
   %Increment $tid$ by $gridsize \times blocksize$ ; \tcc{round-robin assignment}
}

}
\caption{SSSP Pseudocode illustrating Node-Splitting}
\label{NS-cpu-pseudocode}
\label{NS-gpu-pseudocode}
\end{small}
\end{algorithm}
}

\REM {
\begin{algorithm}
\begin{small}
\SetKwProg{Fn}{}{ \{}{\}}

\Fn(){CUDAKERNEL($dist, graph, inputWl, outputWl$)}{
Initialize $tid, start, end$ \;
$nodeNumber$ = $start+tid$ \;

\BlankLine
\While{$tid < end$}{
  \For{all neighbors of $nodeNumber$}{
    Get $destination$ for neighbor \;
    Initialize $weight$; $altDist = dist[source] + weight$ \;
    \If{$dist[destination] > altDist$}{    
      Get $ChildCount$ of $destination$ \;
      Atomically update $dist[destination]$ and its children with $altDist$ \;  \label{child-update-line} 
       Atomically add node numbers to $outputWl$ \;
    }
  }
   Increment $tid$ by $gridsize \times blocksize$ ; \tcc{round-robin assignment}
}

}

\caption{GPU Pseudocode for Node-Splitting}
\label{NS-gpu-pseudocode}
\end{small}
\end{algorithm}
}

\begin{figure*}
\centering
\includegraphics[scale=0.4]{images/rn-hp.pdf}
%\includegraphics[scale=0.5]{images/hp-illustration.pdf}
\caption{Hierarchical Processing (MDT = 3, i.e., three edges per node are processed in each iteration)}
\label{hp-illustration}
\end{figure*}

 
\vspace{0.1in}
\noindent
{\bf Automatic Determination of Node Splitting Threshold:}
%The discussion in the previous section explains \textit{how} to perform node-splitting, but it does not tell us \textit{when} to do it.
A salient feature of our node splitting strategy is to automatically determine the threshold MDT for node splitting.
%Let's understand why this automation is non-trivial.
Obvious methods based on a threshold or max-degree etc. do not work in general.
For instance, we cannot fix the value of MDT to a constant, since it is unsuitable across graphs and degree distributions.
We cannot also fix MDT to the maximum degree in the graph, since there could be a big skew in the degree distribution with a few very high degree nodes and a large number of medium degree nodes.
A better alternative is to use the difference between the average degree and the maximum degree in the graph; but such a function would be influenced by the graph size.
Another constraint is that the number of nodes to be split should be minimum possible, to reduce the splitting (and processing) cost.
To account for these issues, 
we use a histogram based method in which we use \textit{HistogramBinCount} number of bins representing the ranges of outdegrees of the nodes in the original graph. 
The number of bins is given as an input parameter to our algorithm. 
We then find the distributions of the outdegrees across the bins. 
We find the bin or range with the maximum height, i.e., the range of outdegrees for which the graph has the maximum number of nodes. 
Let \textit{binIndex} be the index of this bin. 
We then find the maximum degree threshold \textit{MDT} for the outdegrees as \textit{(binIndex/HistogramBinCount)$\times$ maxDegree}, where \textit{maxDegree} is the maximum outdegree in the graph. 
Our node-splitting algorithm then finds the nodes in the graph with outdegrees greater than \textit{MDT} and splits them into child nodes such that each parent and the child nodes will have a maximum of \textit{MDT} outdegrees. 
Our histogram approach of finding \textit{MDT} attempts to maximize the number of nodes (parent and child) with \textit{MDT} outdegrees. 
By choosing the bin with the maximum height in which the nodes already have their outdegrees closer to \textit{MDT}, our algorithm minimizes the amount of splitting. 
Since the histogram-based method considers degree distributions of a graph to achieve load balancing, it can be applied to different graphs with different kinds of distributions including highly skewed distributions.

An advantage of the node splitting approach is that it continues to work with the space-efficient CSR format.
Another advantage over workload-decomposition (Section~\ref{workload-decomposition}) is that all the edges of a node are processed by the same thread, reducing bookkeeping and improving the scope for memory coalescing.
%The advantages of the node-splitting approach is that unlike the workload-decomposition strategy, each thread processes all edges of only one node (parent or child node), thereby simplifying book-keeping. 
The primary disadvantage of node splitting is that it results in extra atomic operations to update the child nodes whenever the parent node gets updated.
A secondary disadvantage is the overhead of computing the histogram to find the MDT.
One may wonder that the strategy has a space overhead due to explicit splitting, but we found in our experiments that less than 5\% of the nodes undergo split, resulting in negligible space overhead.
%the additional space and time complexity for creating the new nodes. If for a graph with $N$ nodes and $E$ edges, $N'$ child nodes are created, the space complexity for the new graph is $N+N'+E$. $N'$ depends on the degree distribution of the graph. In our experiments, we found that in the worst case, $N'=N/2$, while in the average case, $N'$ is less than $5\%$ of $N$. Thus the increase in space complexity is mostly negligible for our experiments. Additional overhead is also incurred to determine the MDT using the histogram-based approach. Node-splitting also results in extra atomic operations to update the child nodes along with the parent nodes.

As we observe in our experiments (Section~\ref{exp_res}), node-splitting provides considerably better load-balancing.
In addition, it provides comparable performance for large diameter graphs (such as road networks); but it has a high overhead for power-law degree distribution graphs.


