\begin{abstract}
Spiking Neuromorphic Computing uses binary activity to improve Artificial Intelligence energy efficiency. However, the non-smoothness of binary activity requires approximate gradients, known as Surrogate Gradients (SG), to close the performance gap with Deep Learning. Several SG have been proposed in the literature, but it remains unclear how to determine the best SG for a given task and network.
Good performance can be achieved with most SG shapes, after a costly search of hyper-parameters. Thus, we aim at experimentally and theoretically define the best SG across different stress tests, to reduce future need of grid search. 
To understand the gap for this line of work, we show that more complex tasks and networks need more careful choice of SG, even if overall the derivative of the fast sigmoid outperforms other SG across tasks and networks, for a wide range of learning rates.
We therefore design a stability based theoretical method to choose initialization and SG shape before training on the most common spiking architecture, the Leaky Integrate and Fire (LIF). Since our stability method suggests the use of high firing rates at initialization, which is non-standard in the neuromorphic literature, we show that high initial firing rates, combined with a sparsity encouraging loss term introduced gradually, can lead to better generalization, depending on the SG shape.
Our stability based theoretical solution, finds a SG and initialization that experimentally result in improved accuracy. We show how it can be used to reduce the need of extensive grid-search of dampening, sharpness and tail-fatness of the SG.
\end{abstract}
