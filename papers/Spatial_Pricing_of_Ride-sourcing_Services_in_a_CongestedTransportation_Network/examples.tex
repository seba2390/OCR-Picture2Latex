% jd 2015-05-22 23:30

\section{Numerical Examples}

\subsection{Computational performance}\label{sec:comp_perf}

We use Sioux Fall network, a widely used benchmark network as shown in Figure \ref{fig:siou_fall}, to test the numerical performance of our solution method. Even though the proposed model is capable of handling heterogeneous trip types, in the case study, we only implemented non-commuting trips demanding fast charging at the destinations (typically those fast charging facilities co-located with other commercial/recreational activities).   

The green nodes in Figure \ref{fig:siou_fall} represent the set of origins, destinations and candidate investment location for all the firms, i.e. $R = S = S_i, \forall i$.  The number on each link is the link index.  The input data related to the physical topology of Sioux Fall network is given in Table \ref{tab:data_netw} in \ref{Appendix_1}.  The total travel demand originated from each origin node is given in Table \ref{tab:trav_dema} in \ref{Appendix_1}.  The parameters in the travelers' utility function are set to be   $\beta_0=0$, $\beta_1=1$, $\beta_2=0.4$, $\beta_3=0.1$, $e=8.25$,    $inc=1$.  The investors' cost functions are set to be quadratic: $\phi_c(c) = 0.01c^2 + 0.0044c$ and $\phi_g(g) = 0.01g^2 + 0.013g$, where $c$ and $g$ are in unit of KW.  The link travel costs follow a $4^{th}$-order BPR function.


\begin{figure}[htbp]
\begin{center}
    \includegraphics[width=0.5\textwidth]{Sioux_Falls.pdf}
\caption{Sioux Falls Test Network (green dots represent candidate investment locations)}
\label{fig:siou_fall}
\end{center}
\end{figure}

In this example, the total number of variables is 11148 and the number of constraints is 3504.  The convergence patterns of $\rho$ and the supply surplus are shown in Figure \ref{fig:conv_pattern}. We can see that the price and excess supply converge within 15 iterations for all nodes, with a total computing time of 8782 seconds\footnote{We were not able to solve this example directly using PATH solver.}.  

\begin{figure*}[!t]
\centering
\begin{subfigure}{.35\linewidth}
       \includegraphics[width=2.8in]{conv_prices.pdf}
        \caption{}
        \label{fig:conv_prices}
\end{subfigure}
\hskip8em
\begin{subfigure}{.35\linewidth}
       \includegraphics[width=2.8in]{conv_exsu.pdf}%
    \caption{}
    \label{fig:conv_exsu}
\end{subfigure}
\caption{Convergence Patterns: (a) Market Clearing Prices; (b) Excess Supply}
\label{fig:conv_pattern}
\end{figure*}

The result at the obtained equilibrium state, including the charging infrastructure locations, capacities, and the traffic flows over the network, is plotted in Figure \ref{fig:result_map}.  There are only two locations with ``significant'' positive locational charging price: node 11 and node 13. Correspondingly, these two locations have large investments, which are 97.9 MW and 28.2 MW, respectively.  The detailed results are provided in \ref{Appendix_2} for the convenience of future researchers.  The numerical values of the results do not carry real meaning as the model input data are manufactured and lack empirical grounds.  We shall also point out that the problem studied here may have multiple equilibrium solutions, like many general equilibrium problems.  The solution procedure proposed here aims to identify just one equilibrium solution.  An interesting observation that we have made in our numerical experiments is that node 11 is always chosen no matter which equilibrium solution the algorithm reports.  This coincides with the high centrality measure of node 11 - node 11 is most centralized (in terms of travel time of shortest paths from all other nodes to it) than other nodes in the network.   At this point, we are unable to make any generalization without engaging more numerical testing in realistic settings. 

\begin{figure}[htbp]
\begin{center}
    \includegraphics[width=0.5\textwidth]{result_map.pdf}
\caption{A result map showing the charging infrastructure locations, capacities, and corresponding traffic network flows at the obtained equilibrium state.}
\label{fig:result_map}
\end{center}
\end{figure}
\subsection{Sensitivity Analysis} \label{sec:sens_anal}
In order to investigate impact of different parameters on the equilibrium outcome and converge patter, \zg{to be added...}
\begin{figure*}[!ht]
\centering
\makebox[1.3\linewidth][c]{\hspace{-12em}
\begin{subfigure}{.4\linewidth}
       \centering
       \caption{$\beta_2 = 0.4$, $\beta_3 = 0.1$}
       \includegraphics[width=.99\linewidth]{./new_figures/p_beta_2_0_4_beta_3_0_1.pdf} 
        \label{fig:sens1_p}
\end{subfigure}
\begin{subfigure}{.4\linewidth}
	\centering
       \caption{$\beta_2 = 0$, $\beta_3 = 0.1$}
       \includegraphics[width=.99\linewidth]{./new_figures/p_beta_2_0_beta_3_0_1.pdf} 
        \label{fig:sens2_p}
\end{subfigure}
\begin{subfigure}{.4\linewidth}
       \centering
       \caption{$\beta_2 = 0$, $\beta_3 = 0.01$}
       \includegraphics[width=.99\linewidth]{./new_figures/p_beta_2_0_beta_3_0_01.pdf} 
        \label{fig:sens3_p}
\end{subfigure}}\par\medskip
\vspace{-2em}
\makebox[1.3\linewidth][c]{\hspace{-12em}
\begin{subfigure}{.4\linewidth}
       \centering
       %\caption{$\beta_2 = 0.4$, $\beta_3 = 0.1$}
       \includegraphics[width=.99\linewidth]{./new_figures/es_beta_2_0_4_beta_3_0_1.pdf} 
        \label{fig:sens1_es}
\end{subfigure}
\begin{subfigure}{.4\linewidth}
	\centering
       %\caption{$\beta_2 = 0$, $\beta_3 = 0.1$}
       \includegraphics[width=.99\linewidth]{./new_figures/es_beta_2_0_beta_3_0_1.pdf} 
        \label{fig:sens2_es}
\end{subfigure}
\begin{subfigure}{.4\linewidth}
       \centering
       %\caption{$\beta_2 = 0$, $\beta_3 = 0.01$}
       \includegraphics[width=.99\linewidth]{./new_figures/es_beta_2_0_beta_3_0_01.pdf} 
        \label{fig:sens3_es}
\end{subfigure}}\par\medskip
\vspace{-2em}
\makebox[1.2\linewidth][c]{\hspace{-8em}
\begin{subfigure}{1.2\linewidth}
       \centering
%       \caption{$\beta_2 = 0$, $\beta_3 = 0.01$}
       \includegraphics[width=1\linewidth]{./new_figures/legend.pdf} 
        \label{fig:sens_lege}
\end{subfigure}}
\vspace{-2em}
\caption{Sensitivity Analysis}
\label{fig:sens_anal}
\end{figure*}

\subsection{Impact of business-driven infrastructure investment decisions in a competitive market} \label{sec:cp_cm}
In this section, we explore the impact of business-driven charging facility investment behaviors in a competitive market.  The case study in \citep{He_et_al_13} is used as a benchmark, where the power generators and the charging infrastructure investors are all controlled by a central planner to maximize the total social welfare.  While in our model, we allow the charging infrastructure investors to make their individual decisions in a competitive market.  To keep consistency, the parameter setting\footnote{see \ref{Appendix_1}.} and cost functions are kept the same as those adopted in \citep{He_et_al_13}.  Note that the locational marginal electricity prices, which are model output in \citep{He_et_al_13}, are taken as exogenous parameters in our model.     

Table \ref{tab:comp_cent_equi} compares the results in terms of the total allocation of charging capacity at each candidate node.  There are two equilibrium cases: the first case allows investment on all candidate nodes; the second case assumes that nodes 11-21 already have existing charging facilities and are excluded from the candidate location set, which is consistent with the setting in \citep{He_et_al_13}.  We can see that the total investment,  which is 126.17 MW, is identical across all three cases.  This is straightforward because the setting of this example assumes inelastic total charging demand. However, the charging capacities allocated to different nodes are very different: the result in\citep{He_et_al_13} tends to concentrate more investment on nodes 4, 5 and 10, while our model leads to a more diffused investment outcome in both equilibrium cases. The difference is not surprising.  The objective of \citep{He_et_al_13} is to maximize the social welfare, which includes surplus for consumers, investors and generators, while in our model each investor makes decision  to maximize individual profits.  Even though the value of locational marginal price $\rho^s$ sends signal of the market charging demand to the investors, it cannot capture the externality of traffic and charging congestion. Therefore, a business-driven competitive market generally cannot yield a social optimal solution. 
 
% Table generated by Excel2LaTeX from sheet 'results'
\begin{table}[htbp]
  \centering
  \caption{Comparison between central planner allocation and market equilibrium outcome}
    \begin{tabular}{cccc}
    \toprule
    Node  & Central Planner & Equilibrium &  Equilibrium (fixing results at Nodes 11-21) \\
    \midrule
    1     & 1.77  & 10.22 & 13.52 \\
    2     & 1.52  & 10.06 & 11.85 \\
    4     & 34.01 & 10.86 & 18.85 \\
    5     & 24.63 & 10.58 & 16.51 \\
    10    & 37.53 & 10.55 & 38.74 \\
    11    & 3.51  & 11.27 & 3.51 \\
    13    & 2.04  & 9.02  & 2.04 \\
    14    & 3.08  & 10.65 & 3.08 \\
    15    & 5.99  & 12.10 & 5.99 \\
    19    & 4.60  & 11.43 & 4.60 \\
    20    & 3.94  & 9.66  & 3.94 \\
    21    & 3.55  & 9.78  & 3.55 \\
    Total & 126.17 & 126.18 & 126.18 \\
    \bottomrule
    \end{tabular}%
  \label{tab:comp_cent_equi}%
\end{table}%
%\subsection{Spatial localization pattern}
%
%Next, we explore the possible spatial investment patterns derived by competitive market. The spatial localization of firms has been widely discussed in economic literature since the seminal study of Hotelling \citep{Hotelling_29}. However, both the theoretical and empirical studies do not reach an agreement about whether firms prefer closed to or distant   from each other. The results are known to be sensitive to assumptions about consumer's utility function, travel cost, and competition intensity \citep{Bernardo_et_al_15}.In this study, we do not aim to provide a general answer to this question. Instead, we only focus on the consumer's preference on charging services. More specifically, we are looking at the investment pattern when consumers put different weight on charging availability ($\beta_2$) and charging price ($\beta_3$). 
%
%We begin with a small example with three nodes and two links. The network structure is shown in Figure \ref{fig:thre_node}. Node 1 is origin node, with fixed demand 10. Node 2 and 3 are symmetric destination nodes for travelers. Travel time is assumed to be independent of travel flow. Furthermore, we assume there are two symmetric investors, both of who can invest in both destinations. The investment cost function and generation cost function are $c^2 + 20c$ and $g^2 + 13g$. With this setting, we create two cases: case 1 is travelers care a lot more about charging price $\beta_2 = 0, \beta_3 = 100$, and Case 2 is travelers care a lot more about charging availability $\beta_2 = 100, \beta_3 = 0$. The results are shown in Figure \ref{fig:resu_thre_node}. In case 1, both firms invest the same quantities in both locations and travelers have no preference about both destinations; in case 2, the investment of both firms and the travelers all cluster to node 2. However, notice that these results only show one possible equilibrium and the uniqueness of equilibrium is not guaranteed in this study. 
%
%\begin{figure}[htbp]
%\begin{center}
%    \includegraphics[width=0.3\textwidth]{three_nodes_example.pdf}
%\caption{Three Nodes Test Network}
%\label{fig:thre_node}
%\end{center}
%\end{figure}
%
%\begin{figure}[htbp]
%\begin{center}
%    \includegraphics[width=1\textwidth]{three_nodes_results.pdf}
%\caption{Results of Three Nodes Test Network}
%\label{fig:resu_thre_node}
%\end{center}
%\end{figure}
%
%Following this observation, we implement our algorithm using our Sioux Falls Test Network. 
%
%\zg{Results from Julio}
%
%\zg {Prof. Fan, as far as I know, we don't have a Sioux Falls results to illustrate our hypothesis (Julio, please confirm). So we may want to leave it just as a hypothesis in this paper? }


