%\documentclass[aps,prl,groupedaddress]{revtex4}
%\usepackage[dvipdfmx]{graphicx,color,hyperref}
%\hypersetup{pdfpagemode=UseNone,pdfstartview=FitH,colorlinks=true,linkcolor=blue,citecolor=blue,urlcolor=blue}
%\usepackage{amsmath,amssymb, booktabs, braket}

%\begin{document}

%\newcommand{\etal}      {{\it et~al.}}

%\title{Supplemental Material for\\ $\;$\\ $\;$\\Measurement-Free Ultrafast Quantum Error Correction by Using Multi-Controlled Gates in Higher-Dimensional State Space}


%\maketitle

Qutrit-based Toffoli gate decomposition was first proposed in Ref.~\cite{gokhale2019} by using a $\ket{2}$-controlled $X_{01}$ gate sandwiched by a pair of $\ket{1}$-controlled $X_{+}$ and $X_{-}$ gates, as shown in D1 of Fig.~\ref{fig5}.
Note that similar decompositions are possible, for example, by swapping the two rotational gates, yielding a NOT operation on the target when the input state is $\ket{{\rm Q_1 Q_2}}=\ket{10}$.
% and replacing the central CX gate with a $\ket{0}$-controlled one.
%\textcolor{red}{$\ket{10}$}
Since the rotational gates are decomposed into subspace gates as
\begin{eqnarray}
X_{+} &=& X_{i\,j} X_{j\,k}\ \ \ \ {\rm with}\ \,j=i-1\ {\rm mod}\ 3,\ \,k=i-2\ {\rm mod}\ 3   \\
X_{-} &=& X_{i\,j'} X_{j'\,k'}\ \ {\rm with}\ j'=i+1\ {\rm mod}\ 3,\ k'=i+2\ {\rm mod}\ 3,
\end{eqnarray}
for $i=0,1,2$ (and with $X_{il} = X_{li}$), D1 can be transformed into D2 in Fig.~\ref{fig5}, where the modulus is abbreviated for simplicity.
As D2 contains five two-qutrit gates, this naive decomposition costs unreasonably high in the actual implementation.

\begin{figure}[b!]
\includegraphics[width=120mm]{fig5.eps}
\caption{
D1:~Original Toffoli gate decomposition with two controlled rotational gates (X$_+$ and X$_-$)~\cite{gokhale2019}.
Similar decompositions are possible, for example, by swapping the two rotational gates, yielding a NOT operation on the target when the input state is $\ket{{\rm Q_1 Q_2}}=\ket{10}$.
D2:~Each controlled rotational gate in D1 is naively decomposed into two controlled subspace gates, producing five two-qutrit gates in total.
Note that ``mod~3'' is abbreviated in the subscripts of the Q$_1$-Q$_2$ gates for simplicity.
D3:~More elaborate decomposition with three two-qutrit gates by using non-trivial single- and two-qutrit gates (see the main text for details).
}
\label{fig5}
\end{figure}

A more elaborate implementation of D1 is possible by directly approximating the rotational gates to a controlled SUM or controlled MINUS gate~\cite{blok2021}, as shown in D3 of Fig.~\ref{fig5}.
They work on $\ket{{\rm Q_1Q_2}}$ as
\begin{eqnarray}
{\rm CSUM}\ket{m, n} &=& \ket{m, n+m \ {\rm mod} \ 3} \\
{\rm CMIN}\ket{m, n} &=& \ket{m, n-m \ {\rm mod} \ 3},
\end{eqnarray}
which are composed of the controlled phase gate $U_{{\rm C}\phi} = \sum_{n=0}^{2} \ket{n} \bra{n} \otimes Z^{n}$ with qutrit $Z = \sum_{j=0}^{2} e^{i 2\pi/3 j} \ket{j}\bra{j}$,
sandwiched by a pair of qutrit Hadamard gates
\begin{equation}
H = \frac{1}{\sqrt{3}}
\begin{pmatrix}
1&1&1\\
1&e^{i 2\pi/3}&e^{-i 2\pi/3}\\
1&e^{-i 2\pi/3}&e^{i 2\pi/3}\\
\end{pmatrix},
\end{equation}
implemented with a four-pulse sequence~\cite{morvan2021} or simultaneous driving of all three transitions~\cite{yurtalan2020}.
Although the decomposition D3 consists of three two-qutrit gates, the costs for those non-trivial gates are also high, supporting the efficiency of decompositions into controlled subspace gates described in the main text.

%\bibliography{cite_sup.bib}

%\end{document}
