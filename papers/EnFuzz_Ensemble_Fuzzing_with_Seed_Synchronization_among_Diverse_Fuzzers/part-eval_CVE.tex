We apply \toolname to fuzz more real-world applications from GitHub and commercial products from Cisco, some of which are well-fuzzed projects such as the image processing library libpng and libjepg, the video processing library libwav, the IoT device communication protocol libiec61850 used in hundreds of thousands of cameras, etc. 
\toolname also performs well. Within 24 hours, besides the coverage improvements, \toolname finds \bugnum more unknown real bugs including \cvenum successfully registered as CVEs, as shown in Table \ref{tab:CVE}. All of these new bugs and security
vulnerabilities are detected in a 64-bit machine with 36 cores (Intel(R) Xeon(R)  CPU E5-2630 v3@2.40GHz), 128GB of main memory, and Ubuntu 16.04 as the host OS. 


\newcolumntype{B}{>{\arraybackslash}p{1.0cm}}
\newcolumntype{C}{>{\arraybackslash}p{0.75cm}}
\NewEnviron{mytable_cve}[2]{
  \begin{table}[!htbp]
    \caption{#1}
    \scalebox{0.8}[0.9]{%
      \label{tab:#2}
      \begin{tabular}{l|CCCBCC}
        \toprule
        {\mysize Project}
        & {\mysize AFL}
        & {\mysize AFLFast}
        & {\mysize FairFuzz}
        & {\mysize LibFuzzer}
        & {\mysize QSYM}
        & {\mysize \toolThree}\\
        \midrule
        \BODY
        \bottomrule
      \end{tabular}
    }
  \end{table}%
}


\begin{mytable_cve}{Unique previously unknown bugs detected by each tool within 24 hours on some real-world applications.}{real_bug}
Bento4\_mp4com &          5  &          4  &          5  &          5  &          4  & \textbf{6}  \\
Bento4\_mp4tag     &          5  &          4  &          4  &          5  &          4  & \textbf{7}  \\
bitmap             &          1  &          1  &          1  &          0  &          1  & \textbf{2}  \\
cmft               &          1  &          1  &          0  &          1  &          0  & \textbf{2}  \\
ffjpeg             &          1  &          1  &          1  &          0  &          1  & \textbf{2}  \\
flif               &          1  &          1  &          1  &          2  &          1  & \textbf{3}  \\
imageworsener      & \textbf{1}  &          0  &          0  &          0  & \textbf{1}  & \textbf{1}  \\
libjpeg-05-2018    &          3  &          3  &          3  &          4  &          3  & \textbf{5}  \\
libiec61850        &          3  &          2  &          2  &          1  &          2  & \textbf{4}  \\
libpng-1.6.34      &          2  &          1  &          1  &          1  &          2  & \textbf{3}  \\
libwav\_wavgain    &          3  &          2  &          3  &          0  &          2  & \textbf{5}  \\
libwav\_wavinfo    &          2  &          1  &          2  &          4  &          2  & \textbf{5}  \\
LuPng              &          1  &          1  &          1  &          3  &          1  & \textbf{4}  \\
pbc                &          5  &          5  &          6  &          7  &          6  & \textbf{9}  \\
pngwriter          &          1  &          1  &          1  &          1  & \textbf{2}  & \textbf{2}  \\
\midrule  
total              & 35 & 28 & 31 & 34 & 32 & \textbf{60} \\
\end{mytable_cve}

As a comparison, we also run each tool on those real-world applications to detect unknown vulnerabilities. The results are presented in table \ref{tab:real_bug}. \toolThree ~found all 60 unique bugs, while other tools only found a portion of these bugs. Compared with AFL, AFLFast, FairFuzz, LibFuzzer and QSYM, \toolThree ~detected 71.4\%, 114\%, 93.5\%, 76.4\%, 87.5\% more unique bugs respectively.
The results demonstrate the effectiveness of \toolThree ~in detecting real vulnerabilities in more general projects. For example, in the well-fuzzed projects libwav and libpng, we can still detect 13 more real bugs, 7 of which are assigned as CVEs. We give an analysis of the project libpng for a more detailed illustration. 
libpng is a widely used C library for reading and writing PNG image files. It has been fuzzed many times and is one of the projects in Google's OSS-Fuzz, which means it has been continually fuzzed by multiple fuzzers many times. But with \toolThree, we detect three vulnerabilities, including one segmentation fault, one stack-buffer-overflow and one memory leak. The first two vulnerabilities were assigned as CVEs (CVE-2018-14047, CVE-2018-14550). 
 


In particular, CVE-2018-14047 allows remote attackers to cause a segmentation fault via a crafted input. We analyze the vulnerability with AddressSanitizer and find it is a typical memory access violation. 
%As Listing \ref{gdb} shows, 
The problem is that in function \texttt{png\_free\_data} in line 564 of png.c, the info\_ptr attempts to access an invalid area of memory.
The error occurs in \texttt{png\_free\_data} during the free of text-related data with specifically crafted files, and causes reading of invalid or unknown memory, as show in Listing \ref{libpng}.
The new vulnerabilities and CVEs in the IoT device communication protocol libiec6185 can also crash the service and have already been confirmed and  repaired.

\begin{figure}[!htbp]
\begin{lstlisting}[language = C, numbers = none,
        commentstyle=\color{red!50!green!50!blue!50},frame=shadowbox,
        rulesepcolor=\color{red!20!green!20!blue!20},basicstyle=\ttfamily,breaklines=true,
        extendedchars=true,caption={The error code of libpng for CVE-2018-14047 },label={libpng}]
#ifdef PNG_TEXT_SUPPORTED
/* Free text item num or (if num == -1) all text items */
   	if (info_ptr->text != NULL &&
       	((mask & PNG_FREE_TEXT) & info_ptr->free_me) != 0)
\end{lstlisting} 
\end{figure}


We also apply each base fuzzer (AFL, AFLFast, FairFuzz, libFuzzer and QSYM) to fuzz libpng separately, the above vulnerability is not detected.
%, as is consistent with previous studies.
To trigger this bug, 6 function calls and 11 compares (2 for integer, 1 for boolean and 8 for pointer) are required. It is difficult for other fuzzers to detect bugs in such deep paths without the seeds synchronization of EnFuzz.
The performances of these fuzzers over time in libpng are presented in Figure \ref{fig:performance in libpng}. The results demonstrate that generalization and scalability limitations exist in these base fuzzers -- the two optimized fuzzers AFLFast and FairFuzz perform worse than the original AFL for libpng, while \toolThree ~performs the best. Furthermore, except for those evaluations on benchmarks and real projects, \EnFuzz~ had already been deployed in industry practice, and more new CVEs were being continuously reported.


\begin{figure}[!htbp]
\centering
\begin{minipage}[!htbp]{0.40\textwidth}
     \centering
     \includegraphics[width=1.0\textwidth]{img/libpng_path.pdf}
     \small{(a) Number of paths over time}
\end{minipage}
\begin{minipage}[!htbp]{0.40\textwidth}
     \centering
     \includegraphics[width=1.0\textwidth]{img/libpng_branch.pdf}     
     \small{(b) Number of branches over time}
\end{minipage}
\caption{Performance of each fuzzer over time in libpng. Each fuzzer runs in four CPU cores for 24 hours.}
\label{fig:performance in libpng}
\end{figure}
  



\newcolumntype{C}{>{\arraybackslash}p{0.7 cm}}
\NewEnviron{CVETable}[2]{
  \begin{table}[!htp]
    \caption{#1}
    \scalebox{0.95}[0.9]{%
      \label{tab:#2}
      \begin{tabular}{l|C|p{4.5 cm}}
        \toprule
        {\mysize Project}
        & {\mysize Count}
        & {\mysize CVE-2018-Number}\\
        \midrule
        \BODY
        \bottomrule
      \end{tabular}
    }
  \end{table}%
}

\begin{CVETable}{The \cvenum CVEs detected by \toolname in 24 hours.}{CVE}
Bento4\_mp4com   & 6		&14584, 14585, 14586, 14587, 14588, 14589                                        \\
Bento4\_mp4tag   & 6		&13846, 13847, 13848, 14590, 14531, 14532                                        \\
bitmap           & 1        &17073	                                                                            \\
cmft		     & 1		&13833	                                                                            \\           
ffjpeg           &1         &16781	                                                                            \\
flif	         & 1	    &12109	                                                                            \\
imageworsener    &1         &16782	                                                                            \\
libjpeg-05-2018  & 4        &11212, 11213, 11214, 11813                                                      \\
libiec61850      & 3        &18834, 18937, 19093                                                             \\
libpng-1.6.34    & 2		&14048, 14550	                                                                \\
libwav\_wavgain	 & 2		&14052, 14549	                                                                  \\
libwav\_wavinfo	 & 3		&14049, 14050, 14051	                                                              \\
LuPng	         & 3 	    &18581, 18582, 18583	                                                              \\
pbc			     & 9		&14736, 14737, 14738, 14739, 14740, 14741, 14742, 14743, 14744                    \\
pngwriter	     & 1		&14047\\
\end{CVETable}

