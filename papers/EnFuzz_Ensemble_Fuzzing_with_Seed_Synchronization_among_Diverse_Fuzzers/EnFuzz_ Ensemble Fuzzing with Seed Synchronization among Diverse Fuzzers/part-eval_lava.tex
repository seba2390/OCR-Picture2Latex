We first evaluate ensemble fuzzing on LAVA-M, which has been used for testing other systems such as Angora, T-Fuzz and QSYM, and QSYM shows the best performance. 
We run \toolFive ~(which ensembles AFL, AFLFast, FairFuzz and QSYM) on the LAVA-M dataset. %for five hours, which is the test duration set by the original LAVA work \cite{dolan2016lava}. 
To demonstrate its effectiveness, we also run each base fuzzer using the same resources --- four instances of AFL in parallel mode, four instances of AFLFast in parallel mode, four instances of FairFuzz in parallel mode, QSYM with four CPU cores used in parallel mode (two instances of concolic execution engine and two instances of AFL). To identify unique bugs, we used built-in bug identifiers provided by the LAVA project.
The results are presented in Table \ref{tab:lava_path},  \ref{tab:lava_branch} and  \ref{tab:lava_bug}, which show the number of paths executed, branches covered and unique bugs detected by AFL, AFLFast, FairFuzz, QSYM, \toolFive. 

From Tables \ref{tab:lava_path},  \ref{tab:lava_branch} and  \ref{tab:lava_bug}, we found that AFL, AFLFast and FairFuzz perform worse due to the complexity of their branches.
The practical concolic execution engine helps QSYM solve complex branches and find significantly more bugs. The base code of the four applications in LAVA-M are small (2K-4K LOCs) and concolic execution could work well on them. However, real projects have code bases that easily reach 10k LOCs. Concolic execution might perform worse or even get hanged, as presented in the latter subsections. 
Furthermore, when we ensemble AFL, AFLFast, FairFuzz and QSYM together with the GALS based seed synchronization mechanism -- \toolFive ~ always performs the best in both coverage and bug detection.
In total, compared with AFL, AFLFast, FairFuzz and QYSM, \toolFive ~ executes 44\%, 45\%, 43\% and 7.7\% more paths, covers 195\%, 215\%, 194\% and 5.8\% more branches, and detectes 8314\%, 19533\%, 12989\% and 0.68\% more unique bugs respectively. 
From these preliminary statistics, we can determine that the performance of fuzzers can be improved by our ensemble approach.

\newcolumntype{C}{R{1.1 cm}}
\NewEnviron{mytable_lava1}[2]{
  \begin{table}[!htbp]
    \caption{#1}
    \scalebox{0.9}[0.9]{%
      \label{tab:#2}
      \begin{tabular}{l|CCCCC}
        \toprule
        {\mysize Project}
        & {\mysize AFL}
        & {\mysize AFLFast}
        & {\mysize FairFuzz}
        & {\mysize QSYM}
        & {\mysize \toolFive}\\
        \midrule
        \BODY
        \bottomrule
      \end{tabular}
    }
  \end{table}%
}

%\vspace{-0.4cm}
\begin{mytable_lava1}{Number of paths covered by AFL, AFLFast, FairFuzz, QSYM and \toolFive ~on LAVA-M.}{lava_path}
base64 & 1078 & 1065 & 1080 & 1643 & \textbf{1794} \\
md5sum & 589  & 589  & 601  & 1062 & \textbf{1198} \\
who    & 4599 & 4585 & 4593 & 5621 & \textbf{5986} \\
uniq   & 476  & 453  & 471  & 693  & \textbf{731 } \\
\midrule 
total  & 6742 & 6692 & 6745 & 9019 & \textbf{9709} \\
\end{mytable_lava1}
%\vspace{-0.4cm}

%\vspace{-0.4cm}
\begin{mytable_lava1}{Number of branches covered by AFL, AFLFast, FairFuzz, QSYM and \toolFive ~on LAVA-M.}{lava_branch}
base64 & 388  & 358  & 389  & 960  & \textbf{993 } \\
md5sum & 230  & 208  & 241  & 2591 & \textbf{2786} \\
who    & 813  & 791  & 811  & 1776 & \textbf{1869} \\
uniq   & 1085 & 992  & 1079 & 1673 & \textbf{1761} \\
\midrule  
total  & 2516 & 2349 & 2520 & 7000 & \textbf{7409} \\
\end{mytable_lava1}
%\vspace{-0.4cm}

%\vspace{-0.4cm}
\begin{mytable_lava1}{Number of bugs found by AFL, AFLFast, FairFuzz, QSYM and \toolFive ~on LAVA-M.}{lava_bug}
base64 & 1  & 1 & 0 & 41            & \textbf{42  } \\
md5sum & 0  & 0 & 1 & \textbf{57}   & \textbf{57  } \\
who    & 2  & 0 & 1 & 1047          & \textbf{1053} \\
uniq   & 11 & 5 & 7 & 25            & \textbf{26  } \\
\midrule
total  & 14 & 6 & 9 & 1170         & \textbf{1178} \\
\end{mytable_lava1}





