To investigate the effectiveness of ensemble fuzzing, we use a simple example in Figure \ref{fig:motivating example}
which takes two strings as input, and crashes when one of the two strings is ``Magic Str'' and the other string is ``Magic Num''.



\begin{figure}[!htb]
 \centering
 \includegraphics[width=0.4\textwidth]{img/example.pdf}
 \caption{Motivating example of ensemble fuzzing with seed synchronization.}
 \label{fig:motivating example}
\end{figure}


%\begin{figure}[!htbp]
%\centering
%\begin{minipage}[!htbp]{0.4\textwidth}
%     \centering
%     \begin{minted}{c}
%void crash(char* A, char* B){
%    if (A == "Magic Str"){
%        if (B == "Magic Num") {
%            crash();
%        }else{
%            normal();
%        }
%    }else if (A == "Magic Num"){
%        if (B == "Magic Str"){
%            crash();
%        }else{
%            normal();
%        }
%    }
%}
%\end{minted}   
%     \small{(a)Code of Motivating example}
%\end{minipage}
%\begin{minipage}[!htbp]{0.40\textwidth}
%     \centering
%     \includegraphics[width=1.0\textwidth]{img/branch_example.pdf}     
%     \small{(b)Branches of Motivating example}
%\end{minipage}
%\caption{Motivating Example of Ensemble Fuzzing.}
%\label{fig:motivating example}
%\end{figure}



Many existing fuzzing strategies tend to be designed with certain preferences. %, and the coverage of these preferences varies greatly on the same application.
Suppose that we have two different fuzzers \(fuzzer_1\) and \(fuzzer_2\): \(fuzzer_1\) is good at solving the "Magic Str" problem, so it is better for reaching targets T1 and T3, but fails to reach targets T2 and T4. \(fuzzer_2\) is good at solving the "Magic Num" problem so it is better for reaching targets T2 and T6, but fails to reach targets T1 and T5.
If we use these two fuzzers separately, we can only cover one path and two branches.
At the same time, if we use them simultaneously and ensemble their final fuzzing results without seed synchronization, we can cover two paths and four branches.
However, if we ensemble these two fuzzers with some synchronization mechanisms throughout the fuzzing process, then, once \(fuzzer_1\) reaches T1, it synchronizes the seed that can cover T1 to \(fuzzer_2\). As a result, then, with the help of this synchronized seed, \(fuzzer_2\) can also reach T1, and because of its ability to  solve the "Magic Num" problem, \(fuzzer_2\) can further reach T4. Similarly, with the help of the seed input synchronized by \(fuzzer_2\), \(fuzzer_1\) can also further reach T2 and T5.
Accordingly, all four paths and all six branches can be covered through this ensemble approach.


\newcommand{\mysize}{\small}
\newcolumntype{C}{>{\arraybackslash}p{0.8cm}}
\NewEnviron{mytable_motivate}[2]{
  \begin{table}[!htbp]
  	\renewcommand\arraystretch{1.5}
    \caption{#1}
    \scalebox{0.9}[0.9]{%
      \label{tab:#2}
      \begin{tabular}{p{4.2cm}|CCCp{0.6cm}}
        \toprule
        {\mysize Tool}
        & {\mysize T1-T3}
        & {\mysize T1-T4}
        & {\mysize T2-T5}
        & {\mysize T2-T6}\\
        \midrule
        \BODY
        \bottomrule
      \end{tabular}
    }
  \end{table}%
}

\begin{mytable_motivate}{covered paths of each fuzzing option}{motivating_list} 
fuzzer1  &\checkmark \par   &   &   &   \\  
fuzzer2 &     &     &     &\checkmark \par    \\  
ensemble fuzzer1 and fuzzer2 without seed synchronization   & \checkmark \par  &    &    & \checkmark \par  \\  
ensemble fuzzer1 and fuzzer2 with seed synchronization       & \checkmark \par   & \checkmark \par   & \checkmark \par   & \checkmark \par   \\  
\end{mytable_motivate}


%过渡段,以上示例能够有效工作主要基于以下两点假设: 1,两个fuzzer是不同,(diverisity 比较大) 2. 在多个base fuzzer地fuzzing过程中存在高效地消息同步机制,能实时同步关键种子。 因此,ensmeble fuzzing 有两个重要地点: diversity地定义和挑选base fuzzer很重要 ensemble method (这段话非常关键!!!)
The ensemble approach in this motivating example works based on the following two hypotheses: (1) \(fuzzer_1\) and \(fuzzer_2\) expert in different domains; (2) the interesting seeds can be synchronized to all base fuzzers in a timely way.
To satisfy the above hypotheses as much as possible, successful ensemble fuzzers rely on two key points: (1) 
the first is to select base fuzzers with great diversity (as yet to be well-defined); 
(2) the second is a concrete synchronization mechanism to enhance effective cooperation among those base fuzzers.

