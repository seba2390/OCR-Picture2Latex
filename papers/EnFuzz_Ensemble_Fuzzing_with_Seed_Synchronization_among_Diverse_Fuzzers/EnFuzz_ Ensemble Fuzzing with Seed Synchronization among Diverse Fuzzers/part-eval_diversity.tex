To help select base fuzzers with larger diversity, we need to estimate the diversity between each base fuzzer. In general, the more differently they perform on different
applications, the more diversity among these base fuzzers.
Accordingly, we first run each base fuzzer in single mode, with one CPU core on Google's fuzzer-test-suite for 24 hours. 
Table \ref{tab:single_path} and Table \ref{tab:single_branch} show the number of paths and branches covered by AFL, AFLFast, FairFuzz, libFuzzer, Radamsa and QSYM. Table \ref{tab:single_crash} shows the corresponding number of unique bugs. Below we present the performance effects of the three diversity heuristics proposed in Section 4.1 in detail. 

%Then, we choose AFL as baseline, and calculate diversity of each base fuzzers between AFL, as described in formula \ref{eq:mean} and formula \ref{eq:diversity}.
%the diversity between AFL and  AFLFast is 0.040; 
%the diversity between AFL and  FairFuzz is 0.062; 
%the diversity between AFL and combining Radamsa with AFL is 0.197; 
%the diversity of between AFL and QSYM is 0.271; 
%the diversity of between AFL and libFuzzer is 11.929; 

\textit{1) Effects of seed mutation and seed selection strategy -- what kind of mutation and selection strategy you use, what kind of path and branch you would cover}
The first three columns of Table \ref{tab:single_path} show the performance of the AFL family tools. Their differences are the seed mutation and seed selection strategies. The original AFL performs the best on 5 applications, but  performs the worst on other 10 applications. 
%However, according to literature reviews, those optimizations should outperform AFL. 
%More specifically, 
AFLFast performs the best on 13 applications, and only performs the worst on 4 applications. FairFuzz also performs the best on 8 applications, but the worst on the other 9 applications.
Although the total number of paths covered improves slightly, the performance variation on each application is huge, ranging from -57\% to 38\% in single cases.
%, as demonstrated in Figure \ref{fig:paths executed compared with AFL}. 

From the first three columns in Table \ref{tab:single_branch} and Table \ref{tab:single_crash}, we get the same observation that the performance of these optimized fuzzers varies significantly on different applications. Although the total number of covered branches and unique crashes improves slightly, the deviation of each application is huge.
AFLFast selects seeds that exercise low-frequency paths to mutate more times. Take project lcms for example, this seed selection strategy exercises more new paths by avoiding covering ``hot paths'' too many times, but on project libarchive, its ``hot path'' may be the key to further paths.
FairFuzz mutates seeds to hit rare branches. Take project libxml2 for example, the rare branch fuzzing strategy guides FairFuzz into deeper areas and covers more branches. However, on libarchive, this strategy fails. FairFuzz spends much time in deep paths and branches, ignoring breadth search. Unlike libxml2, the breadth first search strategy of other fuzzers is more effective on libarchive. In general, the mutation and selection strategy decides the depth and breath of the covered branch and path.

\textit{2) Effects of coverage information granularity--what kind of guided information you use, what kind of coverage metric you improve.}
%4. 与AFL相比,libfuzzer的path上比它好, branch却不如afl,因为libfuzzer是path-coverage guided的
The diversity between AFL and libFuzzer is their coverage information granularity.
According to the fourth column of Table \ref{tab:single_path}, we find that compared with AFL, libFuzzer performs better on 17 applications, and covers 30.3\% more paths in total. However, according to the fourth column of the Table \ref{tab:single_branch}, compared with AFL, libFuzzer only performs better on 11 applications, which means on 6 applications, libFuzzer covers more paths but less branches. For total branch count, AFL covers 7.3\% more than libFuzzer.
The reason is that AFL mutates seed by tracking edge hit counts while libFuzzer utilizes the SanitizerCoverage instrumentation to track block hit counts. AFL prefers to cover more branches while libFuzzer is better at executing more paths. In general, edge-guided means more branches covered, and block-guided means more paths covered.

\vspace{-0.5cm}
\newcolumntype{C}{R{0.9cm}}
\NewEnviron{mytable_single}[2]{
  \begin{table}[!htbp]
    \caption{#1}
    \scalebox{0.83}[0.83]{%
      \label{tab:#2}
      \begin{tabular}{l|CCCCCC}
        \toprule
        {\mysize Project}
        & {\mysize AFL}
        & {\mysize AFLFast}
        & {\mysize FairFuzz}
        & {\mysize libFuzzer}
        & {\mysize Radamsa}
        & {\mysize QSYM}\\
        \midrule
        \BODY
        \bottomrule
      \end{tabular}
    }
  \end{table}%
}

\begin{mytable_single}{Average number of paths for single mode. }{single_path} 
boringssl     &          1334  &          1674  &          1760  & \textbf{3528}   &          1682  &          1207  \\
c-ares        &          80    &          84    &          88    & \textbf{123}    &          78    &          72    \\
guetzli       &          1382  &          1090  &          1030  & \textbf{1773}   &          1562  &          1268  \\
lcms          &          656   & \textbf{864}   &          434   &          338    &          550   &          605   \\
libarchive    &          3756  &          2834  &          1630  & \textbf{10124}  &          4570  &          3505  \\
libssh        &          64    &          68    &          62    & \textbf{201}    &          63    &          87    \\
libxml2       &          5762  &          7956  &          8028  & \textbf{19663}  &          9392  &          5098  \\
openssl-1.0.1 & \textbf{2397}  &          2103  &          2285  &          1709   &          2303  &          2330  \\
openssl-1.0.2 &          2456  & \textbf{2482}  &          2040  &          1881   &          2108  &          1947  \\
openssl-1.1.0 &          2439  &          2380  & \textbf{2501}  &          1897   &          2311  &          2416  \\
pcre2         &          32310 &          35288 &          36176 &          20981  & \textbf{37850} &          24501 \\
proj4         &          220   &          218   &          218   & \textbf{334}    &          182   &          208   \\
re2           &          5860  &          6014  &          5016  & \textbf{6327}   &          5418  &          5084  \\
woff2         &          14    &          10    &          12    & \textbf{224}    &          10    &          15    \\
freetype2     &          7748  &          10939 &          10714 & \textbf{16360}  &          9825  &          7188  \\
harfbuzz      &          6793  &          8068  &          8668  & \textbf{10800}  &          5688  &          6881  \\
json          &          466   &          412   &          408   &          499    & \textbf{564}   &          504   \\
libjpeg       &          704   & \textbf{979}   &          722   &          448    &          634   &          638   \\
libpng        &          170   &          159   &          76    &          263    &          493   & \textbf{577}   \\
llvm          &          4830  & \textbf{5760}  &          5360  &          5646   &          4593  &          4096  \\
openthread    &          104   &          123   &          127   & \textbf{976}    &          144   &          141   \\
sqlite        &          179   &          193   &          172   & \textbf{431}    &          256   &          180   \\
vorbis        &          891   & \textbf{1122}  &          821   &          848    &          875   &          898   \\
wpantund      &          2959  &          3048  & \textbf{3513}  &          3510   &          3146  &          2975  \\
\midrule  
Total         & 83575 & 93867 & 91862 & \textbf{108884} & 94296 & 72422 \\  
\end{mytable_single}

%\vspace{-0.5cm}
\begin{mytable_single}{Average number of branches for single mode.}{single_branch}
boringssl     &         2645   &          3054   &          3115   &          3608   &  \textbf{3641}  &          2539   \\
c-ares        & \textbf{126}   &          122    &  \textbf{126}   &          100    &  \textbf{126}   &  \textbf{126}   \\
guetzli       &         1913   &          1491   &          1428   &  \textbf{2774}  &          2118   &          1906   \\
lcms          &         2216   & \textbf{2755}   &          935    &          2661   &          1661   &          2075   \\
libarchive    &         4906   &          3961   &          2387   &          3561   &  \textbf{5263}  &          4366   \\
libssh        &         604    &          604    &          604    &          518    &          604    & \textbf{626}    \\
libxml2       &         10082  &          12407  &          12655  &          13037  &  \textbf{14287} &          9779   \\
openssl-1.0.1 &         3809   &          3879   & \textbf{3901}   &          2591   &          2993   &          3829   \\
openssl-1.0.2 &         3978   &          4015   &          3883   &          2308   &  \textbf{4068}  &          3796   \\
openssl-1.1.0 &         8091   &          8132   &          8212   &          7810   &  \textbf{8292}  &          8032   \\
pcre2         &         27308  &          29324  &          28404  &          13463  &  \textbf{30615} &          19557  \\
proj4         &         264    &          260    &          260    & \textbf{683}    &          264    &          258    \\
re2           &         15892  &          15970  &          15073  &          11369  &  \textbf{16485} &         14477  \\
woff2         &         114    &          112    &          114    & \textbf{1003}   &          114    &          115    \\
freetype2     &         36798  &          44028  &          45319  &          45541  &  \textbf{49468} &          33492  \\
harfbuzz      &         16872  &          16051  & \textbf{19045}  &          18659  &          16782  &          16886  \\
json          &         4462   &          3626   & \textbf{4846}   &          4547   &          4821   &          4538   \\
libjpeg       &         6865   &          8495   &          4028   & \textbf{8828}   &          6982   &          6377   \\
libpng        &         1917   &          1878   &          1135   &          1651   &          2126   & \textbf{2294}   \\
llvm          &         54107  &          55697  & \textbf{57356}  &          51548  &          53427  &          47226  \\
openthread    &         2062   &          2473   &          2646   & \textbf{5295}   &          2231   &          2410   \\
sqlite        &         2706   & \textbf{2784}   &          2771   &          2178   &          2190   &          2709   \\
vorbis        &         11836  & \textbf{13561}  &          12605  &          5902   &          11217  &          12531  \\
wpantund      &         36059  &          36620  & \textbf{37269}  &          28694  &          37075  &          35960  \\
\midrule  
Total         & 255631 & 271299 & 268116 & 238329 & \textbf{276850} & 235903 \\  
\end{mytable_single}


\textit{3) Effects of Input generation strategy--what kind of generation strategy you use, what kind of corresponding application you fuzz better.}
%5. 与AFL相比,加入radamsa能够在某些特定的项目上得到有效的提升,以libxml为例,radamsa具备一些领域知识,能够有效的生成大量类似于xml一样的结构化数据,对于complex format input类型,generation-based + mutation-based 能够比较有效。 但在一些非这种场景下,radamsa生成的大量无用input可能会影响afl的效率
The diversity between AFL and Radamsa is the input generation strategy.
From the fifth columns of Table \ref{tab:single_path} and Table \ref{tab:single_branch}, compared with AFL, the plenty of inputs generated by Radamsa have some side effects on most target applications (14 applications). Too many extra inputs will slow down the execution speed of the fuzzer. However, for some applications, the inputs generated by Radamsa will improve the performance effectively. Take libxml2 for example, Radamsa has some domain knowledge that prefers to generate some structured data and specific complex format data. These domain knowledge are not available in most mutation-based fuzzers, and this is a critical disadvantage of AFL. But with the help of generation-based fuzzers, the performance of AFL can be improved greatly. %\textbf{In general, generation-based fuzzers can be better at applications with complex structured inputs with some domain information.}


\begin{mytable_single}{Average number of bugs for single mode.}{single_crash}
boringssl     &          0 &          0 &          0 & \textbf{1}  &          0  &          0 \\
c-ares        &          1 & \textbf{2} & \textbf{2} &          1  & \textbf{2}  &          1 \\
guetzli       &          0 &          0 &          0 &          0  &          0  &          0 \\
lcms          &          0 &          0 &          0 &          0  &          0  &          0 \\
libarchive    &          0 &          0 &          0 &          0  &          0  &          0 \\
libssh        &          0 &          0 &          0 & \textbf{1}  &          0  &          0 \\
libxml2       &          0 & \textbf{1} &          0 & \textbf{1}  & \textbf{1}  &          0 \\
openssl-1.0.1 &          0 &          0 &          0 &          0  &          0  &          0 \\
openssl-1.0.2 & \textbf{2} &          1 &          0 &          1  &          1  & \textbf{2} \\
openssl-1.1.0 &          0 &          0 &          0 &          0  &          0  &          0 \\
pcre2         & \textbf{2} &          1 &          1 &          1  & \textbf{2}  &          1 \\
proj4         &          0 &          0 &          0 & \textbf{1}  &          0  &          0 \\
re2           &          0 &          0 &          0 & \textbf{1}  &          0  &          0 \\
woff2         &          0 &          0 &          0 & \textbf{1}  &          0  &          0 \\
freetype2     &          0 &          0 &          0 &          0  &          0  &          0 \\
harfbuzz      &          0 &          0 &          0 & \textbf{1}  &          0  &          0 \\
json          & \textbf{1} & \textbf{1} &          0 &          0  & \textbf{1}  &          0 \\
libjpeg       &          0 &          0 &          0 &          0  &          0  &          0 \\
libpng        &          0 & \textbf{1} & \textbf{1} & \textbf{1}  & \textbf{1}  & \textbf{1} \\
llvm          &          0 &          0 & \textbf{1} & \textbf{1}  &          0  & \textbf{1} \\
openthread    &          0 &          0 &          0 & \textbf{1}  &          0  &          0 \\
sqlite        &          0 &          0 &          0 & \textbf{1}  & \textbf{1}  & \textbf{1} \\
vorbis        &          1 &          1 & \textbf{2} &          1  &          1  & \textbf{2} \\
wpantund      &          0 &          0 &          0 &          0  &          0  &          0 \\  
\midrule  
Total         & 7 & 8 & 7 & \textbf{15} & 10 & 9 \\  
\end{mytable_single}

\vspace{0.2cm}
%It is reasonable to draw the conclusion that
\noindent\textbf{In conclusion: }Different base fuzzers perform variously on distinct target applications, showing the diversity for the base fuzzers. The more diversity of these base fuzzers, the more differently they perform on different applications. Furthermore, the above three types of effects should be considered and could be incorporated into the fuzzing evaluation guideline \cite{klees2018evaluating} to avoid biased test cases or metrics selection when evaluating different types of fuzzing optimization. 


