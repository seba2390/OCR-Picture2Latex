%% 详细介绍base fuzzers 的多样性

The first step in ensemble fuzzing is to select a set of base fuzzers. These fuzzers can be generation-based fuzzers, e.g. Peach and Radamsa, or mutation-based fuzzers, e.g. libFuzzer and AFL. We can randomly choose some base fuzzers, but selecting base fuzzers with well-defined diversity improves the performance of an ensemble fuzzer. 

\begin{comment}
Let us take the coverage information for example. Suppose that we have a set of base fuzzers \(F = \{ fuzzer_1, fuzzer_2, \ldots , fuzzer_k \} \). The code coverage of target program \(P\) by \(fuzzer_j \) is \(C_j\). If we use all these \(k\) base fuzzers together to fuzz the target program \(P\), the total code coverage of \(P\) will be their union, that is \( C = C_1 \cup C_2 \cup C_3 \ldots \cup C_k \). The following two points are critical: 
%This is the core idea of the ensemble fuzzing.
%Furthermore, there are two key points we need to know:
\vspace{0.03in}
\begin{itemize}

	 \item [1.] If \( C_1 \cap C_2 \cap C_3 \ldots \cap C_k  \neq \emptyset \), then \( |C| \>> |C_j|\) where \(j\in 1, \ldots , k\), it means that the ensemble fuzzer always performs better than that of any base fuzzer.  

	 \item [2.] The smaller \( |C_i \cap C_j| \) is, the bigger \( |C_i \cup C_j|\) is, and the bigger \(|C|\) is, where \(0 \leq i, j \leq k, i \neq j \). 
	 
\end{itemize}
\vspace{0.03in} 
Therefore, the diversity of base fuzzers is the difference between their fuzzing strategies. The greater the difference of these base fuzzers, the more diversity they have, the better coverage ensemble fuzzers perform. 
%Here, for the mutation-based fuzzers, 
%We propose some simple solutions to these question based on the following three intuitions:
\end{comment}

We classify the diversity of base fuzzers according to three heuristics: seed mutation and selection strategy diversity, coverage information granularity diversity, inputs generation strategy diversity. The diversity heuristics are as follows: 

\begin{itemize}

\item  [1.] Seed mutation and selection strategy based heuristic: the diversity of base fuzzers can be determined by the variability of seed mutation strategies and seed selection strategies. %These two strategies are critical to mutation-based fuzzers.
%The main difference of most fuzzers are these fuzzing strategies. 
For example, AFLFast selects seeds that exercise low-frequency paths and mutates them more times, FairFuzz strives to ensure that the mutant seeds hit the rarest branches.

\item  [2.] Coverage information granularity based heuristic: many base fuzzers determine interesting inputs by tracking different coverage information. Hence, the coverage information is critical, and different kinds of coverage granularity tracked by fuzzers enhances diversity. For example, libFuzzer guides seed mutation by tracking block coverage while AFL tracks edge coverage.

\item  [3.] Input generation strategy based heuristic: fuzzers with different input generation strategies are suitable for different tasks. For example, generation-based fuzzers use the specification of input format to generate test cases, while the mutation-based fuzzers mutate initial seeds by tracking code coverage. So the generation-based fuzzers such as Radamsa perform better on complex format inputs and the mutation-based fuzzers such as AFL prefer complex logic processing.

\end{itemize}

Based on these three basic heuristics, we should be able to select a diverse set of base fuzzers with large diversity. It is our intuition that the diversity between the fuzzers following in two different heuristics is usually larger than the fuzzers that follows in the same heuristic. So, the diversity among the AFL family tools should be the least, while the diversity between Radamsa and AFL, between Libfuzzer and AFL, and between QSYM and AFL is should be greater. In this paper, we select base fuzzers manually based on the above heuristics. the base fuzzers will be dynamically selected according to the real-time coverage information.

\begin{comment}
Furthermore, we quantify the initial diversity value among different fuzzers for more accurate selection. 
As defined in \cite{benjamin2014probability}, the variance or diversity is a measure of the distance of the data in relation to the average. The average standard deviation of a data set is a percentage that indicates how much, on average, each measurement differs from the other. 
%Take path coverage for example, as shown in Table \ref{tab:single_path}, 
To evaluate the diversity of different base fuzzers, we currently choose the most widely used AFL and its path coverage as a baseline and then calculate relative deviation of each tool from this baseline on the Google fuzzing-test-suite. Then we calculate the relative deviation of these values as the initial measure of diversity for each base fuzzer, as presented in formula (\ref{eq:diversity}) and (\ref{eq:mean}),  
where \(n\) means the number of applications fuzzed by these base fuzzers, \(p_i\) means the number of paths covered by the current fuzzer of the target application \(i\) and \(p_{A_i}\) means the number of paths covered by AFL of the application \(i\).

\begin{equation}\label{eq:mean}
mean = \frac{1}{n} \sum_{i=1}^{n}{ \frac{p_i - p_{A_i}}{p_{A_i}} }
\end{equation}

\begin{equation}\label{eq:diversity}
diversity = \frac{1}{n} \sum_{i=1}^{n}{ {( \frac{p_i - p_{A_i}}{p_{A_i}} - mean )} ^{ 2 }  }
\end{equation}
\end{comment}

\begin{comment}
Take the diversity of AFLFast, FairFuzz, Radamsa, QSYM, and libFuzzer for example, as shown in the statistics presented in Table \ref{tab:single_path} of the appendix, compared with AFL on different applications, 
the diversity of AFLFast is 0.040; 
the diversity of FairFuzz is 0.062; 
the diversity of Radamsa is 0.197; 
the diversity of QSYM is 0.271; 
the diversity of libFuzzer is 11.929. 
\end{comment}

%We have one intuition that the more differently they perform on different applications in terms of path coverage, the more diversity that exists among these widely used base fuzzers. 
\begin{comment}
Based on this hypothesis and the initial quantification of diversity, we can select base fuzzers. In the future, the base fuzzers will be dynamically selected according to the real-time number of paths/branches/bugs found individually by each fuzzer.
\end{comment}


%From Table \ref{tab:single_branch} and Table \ref{tab:single_crash}, we can get similar results.  
%
%
%\subsubsection{In conclusion}
%In conclusion, diversity certainly exists among these base fuzzers, the performance of different base fuzzers varies on different applications. The more diversity among these base fuzzers, the more variously they perform.
%Because of the diversity of fuzzers and complexity of real-world applications, we usually don't know how to fuzz them and find crashes effectively.