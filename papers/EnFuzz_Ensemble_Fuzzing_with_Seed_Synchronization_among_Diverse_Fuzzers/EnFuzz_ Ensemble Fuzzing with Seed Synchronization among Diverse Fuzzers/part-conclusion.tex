In this paper, we systematically investigate the practical ensemble fuzzing strategies and the effectiveness of ensemble fuzzing of various fuzzers. %Ensembling multiple diverse fuzzing strategies with the global asynchronous and local synchronous based seed sharing mechanism solves the issue of performance variation in different base fuzzers, and obtains better performance than that of any constituent base fuzzer alone. 
%
Applying the idea of ensemble fuzzing, we bridge two gaps.
First, we come up with a method for defining the diversity of base fuzzers and propose a way of selecting  a diverse set of base fuzzers. 
Then, inspired by AFL in parallel mode, we implement a concrete ensemble architecture with one effective ensemble mechanism, a seed synchronization mechanism.
%For evaluation, we implement three prototypes of ensemble fuzzing. 
%We evaluate them with base fuzzers on a third-party benchmark, Google's fuzzer-test-suite, which consists of real-world applications.
\EnFuzz ~always outperforms other popular base fuzzers in terms of unique bugs, path and branch coverage with the same resource usage. 
%We also find that some existing optimizations for fuzzing strategies work well in single mode, but fail in parallel mode.  
%due to lack of information synchronization among threads.  
\EnFuzz ~has found \bugnum new bugs in several well-fuzzed projects and \cvenum new CVEs were assigned.
Our ensemble architecture can be easily utilized to integrate other base fuzzers for industrial practice. 

Our future work will focus on three directions: the first is to try some other heuristics and more accurate accumulated quantification of diversity in base fuzzers; the second is to improve the ensemble architecture with more advanced en- semble mechanism and synchronize more fine-grained information; the last is to improve the ensemble architecture with intelligent resource allocation such as  dynamically adjusting the synchronization period for each base fuzzer, and allocating more CPU cores to the base fuzzer that shares more interesting seeds. %and release the source code for academic study and industry practice in vulnerability detection. 


