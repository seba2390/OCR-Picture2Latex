
\begin{figure*}[!htb]
\centering
\begin{minipage}[!htbp]{0.19\textwidth}
     \centering
     \includegraphics[width=1.0\textwidth]{img/aflfast.pdf}
     \small{(a) performance of AFLFast in single thread}
\end{minipage}
\begin{minipage}[!htbp]{0.19\textwidth}
     \centering
     \includegraphics[width=1.0\textwidth]{img/fairfuzz.pdf}     
     \small{(b) performance of FairFuzz in single thread}
\end{minipage}
\begin{minipage}[!htbp]{0.19\textwidth}
     \centering
     \includegraphics[width=1.0\textwidth]{img/libfuzzer.pdf}     
     \small{(c) performance of libFuzzer  in single thread}
\end{minipage}
\begin{minipage}[!htbp]{0.19\textwidth}
     \centering
     \includegraphics[width=1.0\textwidth]{img/radamsa.pdf}     
     \small{(d) performance of Radamsa  in single thread}
\end{minipage}
\begin{minipage}[!htbp]{0.19\textwidth}
     \centering
     \includegraphics[width=1.0\textwidth]{img/qsym.pdf}     
     \small{(d) performance of QSYM  in single thread}
\end{minipage}
\caption{Paths covered by base fuzzers compared with AFL in single mode on a single core.} %, blue means improvement and yellow means worse
\label{fig:paths executed compared with AFL}
\end{figure*}

%\begin{figure*}[!htbp]
%\centering
%\begin{minipage}[!htbp]{0.24\textwidth}
%     \centering
%     \includegraphics[width=1.0\textwidth]{img/branch_AFLFast.pdf}
%     \small{(a) performance of AFLFast in single thread}
%\end{minipage}
%\begin{minipage}[!htbp]{0.24\textwidth}
%     \centering
%     \includegraphics[width=1.0\textwidth]{img/branch_FairFuzz.pdf}     
%     \small{(b) performance of FairFuzz in single thread}
%\end{minipage}
%\begin{minipage}[!htbp]{0.24\textwidth}
%     \centering
%     \includegraphics[width=1.0\textwidth]{img/branch_LibFuzzer.pdf}     
%     \small{(c) performance of libFuzzer  in single thread}
%\end{minipage}
%\begin{minipage}[!htbp]{0.24\textwidth}
%     \centering
%     \includegraphics[width=1.0\textwidth]{img/branch_Radamsa.pdf}     
%     \small{(d) performance of Radamsa  in single thread}
%\end{minipage}
%\caption{Branches covered by base fuzzers compared with AFL in single mode on a single core, blue means improvement and yellow means worse.}
%\label{fig:paths executed compared with AFL}
%\end{figure*}


\begin{figure*}[!htb]
\centering
\begin{minipage}[!htbp]{0.19\textwidth}
     \centering
     \includegraphics[width=1.0\textwidth]{img/aflfast4.pdf}
     \small{(a) performance of AFLFast in four threads}
\end{minipage}
\begin{minipage}[!htbp]{0.19\textwidth}
     \centering
     \includegraphics[width=1.0\textwidth]{img/fairfuzz4.pdf}     
     \small{(b) performance of FairFuzz in four threads}
\end{minipage}
\begin{minipage}[!htbp]{0.19\textwidth}
     \centering
     \includegraphics[width=1.0\textwidth]{img/libfuzzer4.pdf}     
     \small{(c) performance of libFuzzer in four threads}
\end{minipage}
\begin{minipage}[!htbp]{0.19\textwidth}
     \centering
     \includegraphics[width=1.0\textwidth]{img/radamsa4.pdf}     
     \small{(d) performance of Radamsa in four threads}
\end{minipage}
\begin{minipage}[!htbp]{0.19\textwidth}
     \centering
     \includegraphics[width=1.0\textwidth]{img/qsym4.pdf}     
     \small{(d) performance of QSYM in four threads}
\end{minipage}
\caption{Paths covered by base fuzzers compared with AFL in parallel mode with four threads on four cores.}
\label{fig:paths-AFL4}
\end{figure*}



\begin{figure*}[!htb]
\centering
\begin{minipage}[!htbp]{0.19\textwidth}
     \centering
     \includegraphics[width=1.0\textwidth]{img/enfuzz-.pdf}     
     \small{(a)  performance of \toolFour ~in four threads}
\end{minipage}
\begin{minipage}[!htbp]{0.19\textwidth}
     \centering
     \includegraphics[width=1.0\textwidth]{img/enfuzz-A.pdf}     
     \small{(a)  performance of \toolOne ~in four threads}
\end{minipage}
\begin{minipage}[!htbp]{0.19\textwidth}
     \centering
     \includegraphics[width=1.0\textwidth]{img/enfuzz-Q.pdf}     
     \small{(a)  performance of \toolFive ~in four threads}
\end{minipage}
\begin{minipage}[!htbp]{0.19\textwidth}
     \centering
     \includegraphics[width=1.0\textwidth]{img/enfuzz-L.pdf}     
     \small{(b) performance of \toolTwo ~in four threads}
\end{minipage}
\begin{minipage}[!htbp]{0.19\textwidth}
     \centering
     \includegraphics[width=1.0\textwidth]{img/enfuzz.pdf}     
     \small{(c) performance of \toolThree ~in four threads}
\end{minipage}
\caption{Paths covered by \EnFuzz ~with four threads on four cores compared with AFL in parallel mode with four threads on four cores. \toolFour ~without the proposed seed synchronization performs the worst, and \EnFuzz ~performs the best.}
\label{fig:paths executed by EnFuzz}
\end{figure*}


%\begin{figure*}[!htbp]
%\centering
%\begin{minipage}[!htbp]{0.3\textwidth}
%     \centering
%     \includegraphics[width=1.0\textwidth]{img/path_enfuzz1.pdf}     
%     \small{(a)  performance of Enfuzz1 in four threads}
%\end{minipage}
%\begin{minipage}[!htbp]{0.3\textwidth}
%     \centering
%     \includegraphics[width=1.0\textwidth]{img/path_enfuzz2.pdf}     
%     \small{(b) performance of Enfuzz2 in four threads}
%\end{minipage}
%\begin{minipage}[!htbp]{0.3\textwidth}
%     \centering
%     \includegraphics[width=1.0\textwidth]{img/path_enfuzz3.pdf}     
%     \small{(b) performance of Enfuzz3 in four threads}
%\end{minipage}
%\caption{Branches covered by \EnFuzz ~with four threads on four cores compared with AFL in parallel mode with four threads on four cores.}
%\label{fig:paths executed by EnFuzz}
%\end{figure*}

%During this evaluation, we get many interesting results that are missed or not consistent with many previous literature studies. 
We choose AFL as the baseline, and compare other tools with AFL on path coverage to demonstrate the performance variation. 
Figure \ref{fig:paths executed compared with AFL} shows the average number of paths executed on Google's fuzzer-test-suite by each base fuzzer compared with AFL in single mode.
%Considering that \toolFour, \toolOne, \toolTwo ~and \toolThree ~use four times of computing resources compared with any base fuzzer running in single mode, for fairness,
We also collect the result of each base fuzzer running in parallel mode with four threads, 
and the result is presented in Figure \ref{fig:paths-AFL4}.
Figure \ref{fig:paths executed by EnFuzz} shows the average number of paths executed by \EnFuzz ~compared with AFL in parallel mode with four CPU cores.
From these results, we get the following conclusions:
\begin{itemize}

%1. 不同base fuzzer 在不同项目上的fuzzing效果差异性很大,不稳定 (在单线程和多线程场景下都有这问题), 说明它们的generalization ability 不好
\item From the results of Figure \ref{fig:paths executed compared with AFL} and Figure \ref{fig:paths-AFL4}, we find that compared with AFL, the two optimized fuzzers AFLFast and FairFuzz, block coverage guided fuzzer libFuzzer, generation-based fuzzer Radamsa and hybrid fuzzer QSYM perform variously on different applications both in single mode and in parallel mode. It demonstrates that the performance of these base fuzzers is challenged by the diversity of the diverse real applications. %at least not as good as the descriptions in their corresponding literature, which shows that their performance is almost always better than that of AFL. 
The performance of their fuzzing strategies cannot constantly perform better than AFL. The performance variation exists in these state-of-the-art fuzzers. %, while in previous literature studies, fuzzers such as AFLFast and FairFuzz are evaluated to be always better than AFL within 24 hours performance. 

\item Comparing the result of Figure \ref{fig:paths executed compared with AFL} and Figure \ref{fig:paths-AFL4}, we find that the performance of these base fuzzers in parallel mode are quite different from those in single mode, especially for AFLFast and FairFuzz. In single mode, the other two optimized base fuzzers perform better than AFL in many applications. But in parallel mode, the result is completely opposite that the original AFL performs better on almost all applications. %, which is missed by evalutions of many previous literature studies.% The detail will be discussed later.

%2. 与base fuzzer相比,enfuzz 在所有的项目上都是最好的,说明通过ensemble的方法,能有效提升generalization ability
\item From the result of Figure \ref{fig:paths executed by EnFuzz}, it reveals that \toolOne, \toolTwo ~and \toolThree ~always perform better than AFL on the target applications.
For the same computing resources usage where AFL running in parallel mode with four CPU cores, \toolOne ~covers 11.26\% more paths than AFL, ranging from 4\% to 38\% in single cases, \toolFive ~covers 12.48\% more paths than AFL, ranging from 5\% to 177\% in single cases, \toolTwo ~covers 37.50\% more paths than AFL, ranging from 13\% to 455\% in single cases.
\toolThree ~covers 42.39\% more paths than AFL, ranging from 14\% to 462\% in single cases.
Through ensemble fuzzing, the performance variation can be reduced. % and the performance metrics can be improved. %, even with little diversity among base fuzzers.

%3. 通过多个enfuzz之间的比较,发现 集成的base fuzzers diversity越大,最终集成的效果越好
\item From the result of Figure \ref{fig:paths executed by EnFuzz}, it reveals that \toolFour ~without seed synchronization performs worse than AFL parallel mode under the same resource constraint. Compared with \toolOne, \toolFive ~covers 
1.09\% more paths, \toolTwo ~covers 23.58\% more paths. 
For \toolThree, it covers 27.97\% more paths than \toolOne, 26.59\% more paths than \toolFive,  3.6\% more paths than \toolTwo, and always performs the best on all applications. 
The more diversity among those integrated base fuzzers, the better performance of ensemble fuzzing, and the seed synchronization contributes more to the improvements.

\end{itemize}

\noindent\textbf{In conclusion: }the performance of the state-of-the-art fuzzers is greatly challenged by the diversity of those real-world applications, and it can be improved through the ensemble fuzzing approach. Furthermore, those optimized strategies work in single mode can not be directly scaled to parallel mode which is widely used in industrial practice. The ensemble fuzzing approach is a critical enhancement to the single and parallel mode of those optimized strategies. 



