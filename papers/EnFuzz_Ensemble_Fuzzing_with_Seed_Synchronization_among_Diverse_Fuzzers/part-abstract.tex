\subsection*{Abstract}
Fuzzing is widely used for vulnerability detection.
There are various kinds of fuzzers with different fuzzing strategies, and most of them perform well on their targets. 
However, in industrial practice, it is found that the performance of those well-designed fuzzing strategies is challenged by the complexity and diversity of real-world applications. 
%
In this paper, 
we systematically study an ensemble fuzzing approach.
%inspired by the idea of ensemble learning, we propose an ensemble fuzzing approach which we refer as \EnFuzz.
First, we define the diversity of base fuzzers in three heuristics: diversity of coverage information granularity, diversity of input generation strategy and diversity of seed selection and mutation strategy. Based on those heuristics, we choose several of the most recent base fuzzers that are as diverse as possible, and propose a globally asynchronous and locally synchronous (GALS) based seed synchronization mechanism to seamlessly ensemble those base fuzzers and obtain better performance. 
%
For evaluation, we implement \EnFuzz ~based on several widely used fuzzers such as QSYM and FairFuzz, and then test them on LAVA-M and Google's fuzzing-test-suite, which consists of 24 widely used real-world applications. This experiment indicates that, under the same constraints for resources, these base fuzzers perform differently on different applications, while EnFuzz always outperforms others in terms of path coverage, branch coverage and bug discovery. Furthermore, \EnFuzz ~found \bugnum new vulnerabilities in several well-fuzzed projects such as libpng and libjpeg, and \cvenum new CVEs were assigned.
%For evaluation, we implement \EnFuzz ~based on several widely used fuzzers, 
%(including AFL, AFLFast, FairFuzz, libFuzzer, Radamsa and QSYM),
%and then test them on two benchmarks and several real projects under the same resources constraint. Specially, on Google's fuzzing-test-suite consisting of widely used real-world application with code base 80K-220K LOCs, the experiment indicates that, these base fuzzers perform variously on different applications, while \EnFuzz ~always outperforms others in terms of path coverage, branch coverage and bug discovery.
%Compared with AFL, AFLFast, FairFuzz, QSYM, libFuzzer and Radamsa. \EnFuzz ~discovers 76.4\%, 140\%, 100\%, 81.8\%, 66.7\% and 93.5\% more bugs, executes 42.4\%, 61.2\%, 45.8\%, 66.4\%, 29.5\% and 44.2\% more paths and covers 15.5\%, 17.8\%, 12.9\%, 26.1\%, 19.9\% and 14.8\% more branches respectively. 
%For the result on LAVA-M with code base 2K-4K LOCs, it outperforms each base fuzzers as well.
%Furthermore, \EnFuzz ~found \bugnum new vulnerabilities in several  well-fuzzed projects such as libpng and libjpeg, while other base fuzzers only detect 35 new vulnerabilities at most, and \cvenum new CVEs were assigned.
%for image processing, and libiec61850 for device communication
%, as presented in the appendix.



\keywords{Ensemble Fuzzing, Seed Synchronization}