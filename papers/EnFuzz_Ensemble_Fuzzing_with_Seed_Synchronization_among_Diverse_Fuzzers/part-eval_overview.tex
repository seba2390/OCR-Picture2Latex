To present the effectiveness of ensemble fuzzing, we first implement several prototypes of ensemble fuzzer based on the state-of-the-art fuzzers. Then, we refer to some kernel descriptions of evaluating fuzzing guideline \cite{klees2018evaluating}. We conduct thorough evaluations repeatedly   on LAVA-M and Google's fuzzer-test-suite, several well-fuzzed open-source projects from GitHub, and several commercial products from companies. Finally, according to the results, we answer the following three questions:
(1) Can ensemble fuzzer perform better?
(2) How do different base fuzzers affect Enfuzz?
(3) How does Enfuzz perform on real-world applications
%\textbf{(1) Can ensemble fuzzer perform better?
%(2) How do different base fuzzers affect Enfuzz?
%(3) How does Enfuzz perform on real-world applications?}

%\begin{itemize}

%	 \item [1.] Does Generalization Limitation exists in these base %fuzzers?

%	 \item [2.] How does diversity affect the performance of base fuzzers?

%	 \item [3.] Can ensemble fuzzing perform better with more diversity?
	 
%\end{itemize}
