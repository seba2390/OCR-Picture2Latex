%% 稍微较为详细的介绍一下集成学习,引出集成fuzz以及两个重要的点,多样性定义和集成的方式

%The main idea of 
For an ensemble fuzzing, we need to construct a set of base fuzzers and seamlessly combine them to test the same target application together. 
The overview of this approach is presented in Figure \ref{fig:framework_of_ensemble_fuzzing}. 
When a target application is prepared for fuzzing, we first choose several existing fuzzers as base fuzzers. 
%In fuzzing, the generalization refers to the ability of an fuzzing strategies to be effective across a range of target applications. 
%The generalization ability of a fuzzer can be described as the ability to perform well, no matter what the target application is.
The existing fuzzing strategies of any single fuzzer %have generalization limitations, because most of these strategies 
are usually designed with preferences. In real practice, these preferences vary greatly across different applications. They can be helpful in some applications, but may be less effective on other applications. Therefore, choosing base fuzzers with more diversity can lead to better ensemble performance. %We provide some initial quantification for their diversity. 
%The existing fuzzing strategies of any single fuzzer are usually designed with some preferences, and the performance of these preferences varies greatly on different applications. They can be helpful in some applications, but not certainly effective on other applications. 
After the base fuzzer selection, we integrate fuzzers with the globally asynchronous and locally synchronous based seed synchronization mechanism so as to monitor the fuzzing status of these base fuzzers and share interesting seeds among them. 
Finally, we collect crash and coverage information and feed this information into the fuzzing report.

\begin{figure}[!htb]
 \centering
 \includegraphics[width=0.47\textwidth]{img/framework_of_ensemble_fuzzing.pdf}
 \vspace{-0.5 cm}
 \caption{The overview of ensemble fuzzing consists of base fuzzer selection and ensemble architecture design. The base fuzzer selection contains the diversity heuristic definition, and the architecture design includes the seed synchronization mechanism as well as final fuzzing report.}
 \vspace{-0.5 cm}
 \label{fig:framework_of_ensemble_fuzzing}
\end{figure}