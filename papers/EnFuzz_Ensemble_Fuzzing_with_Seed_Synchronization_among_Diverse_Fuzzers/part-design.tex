

Figure \ref{fig:interaction} presents an overview of the proposed approach. 
%The seed circular queue is a bridge to communicate all the technologies.
Seed generation takes the C/C++ source code as input and utilizes symbolic execution to generate qualified initial seeds to form the original circular queue.
During fuzzing process, seeds are taken out from the queue one after another. Seed selection determines whether to skip the seed and compute the mutations times for each selected seeds.
Seed mutation is responsible for mutating seeds in the restrict or random manner based on whether the seed can touch hard-to-reach areas. 
The bug report is produced by the seeds which crash the program.
Or if the mutated seeds are interesting, they will be placed in the seed queue for further mutations. Otherwise, the worthless mutated seeds will be thrown into the trash. 

% of DeepFuzz's main components. As mentioned in section 1, we focus on three key obstacles: poor initial seeds quality, non-directed and inefficient mutation strategy, complexity of the environment. 
% Three components, namely the symbolic executor, the directed fuzzing engine and the toolchain are designed to cross these barriers. The duty of the symbolic executor is to generate high-quality initial seeds. With these seeds, the fuzzing engine runs the directed mutation strategy efficiently and effectively. The bitcode for symbolic execution and the hardened binary for fuzzing are built by the toolchain component. To enhance the adaptation of DeepFuzz for different environments, the toolchain component can recognize different build systems and use relatively compile options. 


\begin{figure}[!htbp]
 \centering
 \includegraphics[width=0.4\textwidth]{img/structure.pdf}
 \vspace{-0.3cm}
 \caption{The interaction of DeepFuzzer algorithms}
 \label{fig:interaction}
\end{figure}