\section{BC Analysis Extras}

\subsection{RegularNet's BC Graph}\label{section:appendix_regular_net_bc}

Figure \ref{fig:regularnet_bc} shows the BC graph of RegularNet (see ablation) for the HalfCheetah-v2 environment. This serves as an example estimator which fails in RL despite a good BC graph (its failure in RL is illustrated in Table \ref{table:ablation_regnet}). As described in Section \ref{sec:our_approach}, a BC graph's utility is mainly one way. That is, when poor it is indicative of failure in RL, but the reverse is not necessarily the case. Beyond that however, it also provides insight into what is captured by an estimator as well as being valuable as a quick sanity test of an estimator’s quality (can be generated quickly on saved BC trajectories). 

\begin{figure}[h!]
\vskip 0.1in
\centering
\includegraphics[width=0.35\textwidth]{figures/bc_analysis_regularnet.png} % Reduce the figure size so that it is slightly narrower than the column. Don't use precise values for figure width.This setup will avoid overfull boxes.
\caption{The BC graph of RegularNet (see ablation) for the HalfCheetah-v2 environment.}
\label{fig:regularnet_bc}
\vskip -0.1in
\end{figure} 

\subsection{Two-Dimensional BC Analysis}
What we have termed the BC graph is the diagonal slice of a broader two-dimensional picture. Recall the generation process described in Section \ref{sec:our_approach}: BC graphs show the $i$'th estimator's evaluation of the $i$'th BC trajectory against the true environment reward of that same trajectory. However, one may take interest in evaluating all $N$ estimators on all $N$ trajectories. To that end, out of the $N=375$ estimator updates, we show every $25$'th (along with the first), evaluated on all the BC trajectories, scattered against their true environment rewards. This is illustrated in Figures \ref{fig:bc_2d_coupled}, \ref{fig:bc_2d_uncoupled} and \ref{fig:bc_2d_regularnet} corresponding to coupled flows, uncoupled (independent) flows and RegularNet, respectively. The figures provide insight into what individual estimators captured as well as how they morphed over time. For example, for the coupled flows, we can see how the later estimators (synthetic rewards) incentivize the agent not to go beyond expert level, while the early ones do not. This is reasonable, since estimator $i$ is only encountered by policy $\pi_i$ (hence why the BC graph is only the diagonal cut). For this reason, evaluations of early estimators on much later trajectories aren't particularly insightful with regards to what an RL agent with the analogously constructed reward will face. Still, they're interesting in their own right, showing what has been captured by the estimator at that time. 



\begin{figure}[h]
\vskip 0.1in
\centering
\includegraphics[width=0.7\textwidth]{figures/bc_2d_analysis_coupled.png} % Reduce the figure size so that it is slightly narrower than the column. Don't use precise values for figure width.This setup will avoid overfull boxes.
\caption{Two-dimensional BC analysis of coupled flows for the HalfCheetah-v2 environment. In contrast to the BC graph which is the $i$'th estimator's evaluation of the $i$'th BC trajectory vs the true environment reward of that same trajectory, here we see scatters for individual estimators along the way. That is, their evaluations of all BC trajectories scattered against the true environment rewards of the trajectories.}
\label{fig:bc_2d_coupled}
\vskip -0.1in
\end{figure} 

\begin{figure}[h]
\vskip 0.1in
\centering
\includegraphics[width=0.7\textwidth]{figures/bc_2d_analysis_uncoupled.png} % Reduce the figure size so that it is slightly narrower than the column. Don't use precise values for figure width.This setup will avoid overfull boxes.
\caption{Two-dimensional BC analysis of uncoupled flows for the HalfCheetah-v2 environment. See Figure \ref{fig:bc_2d_coupled}'s caption.}
\label{fig:bc_2d_uncoupled}
\vskip -0.1in
\end{figure} 

\begin{figure}[h]
\vskip 0.1in
\centering
\includegraphics[width=0.7\textwidth]{figures/bc_2d_analysis_regularnet.png} % Reduce the figure size so that it is slightly narrower than the column. Don't use precise values for figure width.This setup will avoid overfull boxes.
\caption{Two-dimensional BC analysis of RegularNet for the HalfCheetah-v2 environment. See Figure \ref{fig:bc_2d_coupled}'s caption.}
\label{fig:bc_2d_regularnet}
\vskip -0.1in
\end{figure} 

