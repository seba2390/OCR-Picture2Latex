\section{Introduction}

The persisting discrepancy between the Standard Model evaluation and the experimental determination~\cite{BNL} of the muon anomalous magnetic moment $a_\mu$ is one of the outstanding open problems in particle physics and is traditionally considered a harbinger of New Physics. Moreover, the forthcoming results from the Fermilab E989 experiment, which aim to improve the present accuracy by a factor of 4 to reach an uncertainty of about $16 \times 10^{-11}$ ({\it i.e.} 0.14 ppm)~\cite{FermilabTDR}, make it even more crucial and timely to further scrutinize and improve control over theory predictions.

Together with the hadronic vacuum polarization contribution, the hadronic light-by-light (HLbL) is the major source of theoretical uncertainty in the Standard Model~\cite{Prades:2009tw,Jegerlehner:2009ry,Jegerlehner:2008zza}. In the last years, significant efforts have been devoted to improve the determination of $a_\mu^{\text{HLbL}}$ and reduce model dependence by using analytic approaches based on dispersion relations~\cite{Hoferichter:2013ama,BTT1,Colangelo:2014pva,BTT2,Colangelo:2017qdm,BTT3,Hoferichter:2014vra,PionTFFshort,PionTFF,Pauk:2014rfa,Danilkin:2016hnh,Hagelstein:2017obr} as well as lattice QCD~\cite{Blum:2014oka,Blum:2015gfa,Blum:2016lnc,Blum:2017cer,Blum:2019ugy,Green:2015sra,Asmussen:2019act}. In particular, the dispersive framework for the HLbL tensor in Refs.~\cite{Hoferichter:2013ama,BTT1,Colangelo:2014pva,BTT2,Colangelo:2017qdm,BTT3} has enabled accurate data-driven determinations with controlled error estimates of the contributions from one- and two-pion intermediate states.

In this framework $a_\mu^{\text{HLbL}}$ is evaluated via a two-loop integral of dispersively reconstructed scalar functions against analytically known kernels. At sufficiently small space-like photon virtualities, contributions from low-mass states accessible to a dispersive treatment are enhanced. At higher virtualities such an enhancement does not occur leading to important effects from higher intermediate states, which are constrained by operator product expansions (OPEs) and perturbative QCD (pQCD).

More specifically, there are two relevant kinematic regimes concerning short-distance constraints (SDCs) on $a_\mu^{\text{HLbL}}$ for asymptotic values of (subsets of) the photon virtualities.
Since one of the photons corresponds to the static electromagnetic source in the definition of $g-2$, one asymptotic regime is realized when the remaining three space-like photon virtualities are comparable and much larger than $\LQCD^2$, and the other when two space-like photon virtualities are much larger than both the third and $\LQCD^2$. The latter SDC was first derived by Melnikov and Vainshtein (MV)~\cite{MV} using an OPE that leads to relations involving longitudinal and transversal amplitudes of the correlator of two vector and one axial current (VVA) in the chiral limit. The former SDC was also discussed in Ref.~\cite{MV} based on the quark-loop calculation at leading order in pQCD and its derivation was recently put on a firmer theoretical ground by means of an OPE in an external electromagnetic background field~\cite{Bijnens}.

Tree-level resonance exchanges cannot make $a_\mu^{\text{HLbL}}$ comply with all SDCs unless an infinite number of states is included. This is due to the fact that the transition form factors (TFFs) describing the resonance couplings to off-shell photons are subject themselves to asymptotic QCD constraints~\cite{Lepage:1980fj,Brodsky:1981rp,Hoferichter:2020lap},
which make the full HLbL four-point function decay too fast at high virtualities.\footnote{See {\it e.g.}\ Ref.~\cite{Bijnens:2003rc} where the analogous case of a three-point function is treated explicitly.} MV proposed a model to satisfy the longitudinal and transversal OPE SDCs through a modification of the pion pole contribution~\cite{MV, MVNew}, which affects also the low-energy region. Recently, alternative model-dependent solutions have been investigated to fulfill both OPE and pQCD SDCs by instead adding degrees of freedom to the ground-state pseudoscalars. In this context, Refs.~\cite{BernSDCShort,BernSDCLong} proposed the inclusion of infinite towers of excited pseudoscalar poles in large-$N_c$-inspired Regge models to satisfy longitudinal SDCs away from the chiral limit,\footnote{For other calculations based on large-$N_c$ arguments to satisfy long- and short-distance constraints on QCD correlators using finite or infinite sets of narrow resonances, see Refs. \cite{Peris:1998nj, Knecht:1998sp, Bijnens:2003rc, Golterman:2001nk, Knecht:2001xc, Golterman:2001pj, DAmbrosio:2019xph}.} while the effect of summing over axial-vector contributions in holographic QCD was the subject of   Refs.~\cite{HolographyVienna, HolographyItaly}.\footnote{See also Ref.~\cite{Masjuan:2020jsf} for a discussion of the role of axial-vector mesons in the saturation of the SDCs.} Through the explicit summation of intermediate states, these models provide specific interpolations between the low-energy region and the asymptotic regimes for the scalar functions that determine $a_\mu^{\text{HLbL}}$.

The goal of this paper is complementary to these studies. We introduce an approach based on more general interpolating scalar functions, independent of the physical mechanism that is ultimately responsible for their actual form outside the low-energy region. The multi-parameter families of functions studied here satisfy all constraints rigorously derived from general principles: unitarity, analyticity and crossing in the low-energy domain, OPE and pQCD constraints in the mixed and high-energy regions. Here we focus on longitudinal SDCs since these are tightly related to the pseudoscalar poles for which accurate low-energy input is available and 
their implementation does not involve any mixing of OPE constraints among different scalar functions~\cite{BernSDCLong}. Error estimates as well as the role played by the various parameters and assumptions, can be easily and transparently addressed in our approach and are investigated in detail in our numerical study. 

Crucial input for our analysis is provided by an accurate low-energy representation of the scalar functions. In the following we will mostly assume that this is given by the ground-state pseudoscalar poles. In this context, an important role is played by the lightest state ($\pi^0$), whose contribution is under firm theoretical control thanks to a dispersive evaluation~\cite{Hoferichter:2014vra,PionTFFshort,PionTFF}. Improved determinations of the effects of SDCs can be obtained in a straightforward way within our approach once similarly precise, model-independent information about further relevant intermediate states in the energy region up to \SIrange[range-units=single, range-phrase=--]{1}{2}{\GeV} become available. In order to illustrate this aspect and compare against a different way to estimate the contribution from SDCs, we have applied our method also to the case where the lightest axial-vector meson is included in the low-energy region using input from holographic QCD~\cite{HolographyVienna,HolographyItaly} and neglecting issues related to intrinsic model dependence.

The paper is structured as follows. In Sec.~\ref{Sec:SDC} we review the relevant constraints on HLbL and the assumptions made in their derivations. Sec.~\ref{Sec:HEInt} describes our interpolation between the OPE and pQCD asymptotic constraints while in Sec.~\ref{Sec:Int} we discuss its smooth connection with the low-energy region. In sec.~\ref{Sec:Num} we present our numerical analysis with particular emphasis on the error estimation. Conclusions are drawn in Sec.~\ref{Sec:Con}. \ref{App:Convergence} is devoted to the analysis of the convergence properties of our interpolants.