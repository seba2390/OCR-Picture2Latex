\section{Interpolating between low and high energies}
\label{Sec:Int}
The next step is to smoothly connect our representation of $\bar{\Pi}_1$ for $\Sigma \gg \{\LQCD^2,M_\text{PS}^2,\dots\}$ given by Eq.~(\ref{Eq:AsymPi}) to an accurate low-energy description. We achieve this by adding suitable terms to $\PiPS[\text{asymp}]$ that are sub-leading at large $\Sigma$. For each choice of $r$ and $\phi$, the coefficients of these terms are then matched onto an input low-energy representation of $\bar{\Pi}_1$ at a suitably defined surface $\Sigma^\text{match}(r,\phi)$. In Sec.~\ref{Sec:Sigmamatch} we will discuss how $\Sigma^\text{match}$ is related to the mass scale at which intermediate states beyond the ones explicitly considered start to affect $\bar{\Pi}_1$. 


\subsection{Interpolation functions and matching procedure}
For $\Sigma > \Sigma^\text{match}$ we consider the following two interpolation functions
\begin{eqnarray}
\PiPS[\text{int 1}](\Sigma, r, \phi) & = & \PiPS[\text{asymp}](\Sigma, r, \phi) \nonumber\\
&&\times \left(1 + \sum_{i = 1}^{N}b_i(r, \phi) \Sigma^{-i}\right)\,, \nonumber \\
\PiPS[\text{int 2}](\Sigma, r, \phi) & = & \PiPS[\text{asymp}](\Sigma, r, \phi) \nonumber\\
&&\times \left(1 + \sum_{i = 1}^{N}b_i(r, \phi) \Sigma^{-i}\right)^{-1}
\label{Eq:interpolant12}
\end{eqnarray}
whose leading terms at asymptotic $\Sigma$ are given in Eq.~(\ref{Eq:AsymPi}), whereas below the matching surface we set $\PiPS[\text{int 1,2}] = \bar{\Pi}_1^\text{PS-pole}$. In \ref{App:Convergence} we will show that these functions converge to the true $\PiPS$ in the limit $N\to\infty$ when matched to exact low-energy input using the convergence property of a Taylor series. The two different forms given in Eq.~(\ref{Eq:interpolant12}) will be used to estimate the sensitivity of our numerical results on the specific choice of interpolation between low and high energies.\footnote{We also considered multiplying the asymptotic expression in Eq.~(\ref{Eq:AsymPi}) by Pad\'e approximants in $\Sigma^{-1}$. Using up to 3 free parameters and fixing them in the way discussed below, however, leads to poles within the $a_\mu$ integration domain, where $\bar{\Pi}_1$ is known to be analytic.}

The coefficients $b_i(r,\phi)$ are fixed from the requirement that the $\PiPS[\text{int i}]$ have the same value and the same $N-1$ $\Sigma$-derivatives as the low-energy representation if evaluated at $\Sigma = \Sigma^\text{match}(r,\phi)$ for each $(r,\phi)$. No expansion is performed in $r$ and $\phi$, which is crucial to obtain a smooth transition to the low-energy regime.

Determining the optimal value of $N$ is a non-trivial issue. On the one hand, larger values of $N$ seem to be preferable since the true $\bar{\Pi}_1$ is analytic for space-like momenta and thus all derivatives are continuous. On the other hand, matching many derivatives leads to a function that is almost saturated by the low-energy input contribution up to considerably higher energies than $\Sigma^\text{match}(r,\phi)$. Since it is desirable to match at least one derivative in order to have $\bar{\Pi}_1$ differentiable at the matching point, we will use $N \in \{2,3\}$ in order to estimate the dependence on $N$.

Interpolation functions with a logarithmic dependence on $\Sigma$ are not forbidden.
This can stem, for example, from non-perturbative corrections leading to terms like $\ln{(Q_i^2/M^2)}$, where $M$ is some non-perturbative mass scale. In fact, the Regge model considered in Refs.~\cite{BernSDCShort,BernSDCLong} leads to interpolants containing terms like $Q^{-4}\ln{(Q^2/\sigma^2)}$ for $Q_i^2 = Q^2 \to \infty$, where $\sigma^2$ could {\it e.g.}\ be the Regge slope of the excited pseudoscalar masses. In order to allow for such a logarithmic approach of the asymptotic expression, we additionally consider the alternative interpolant
\begin{eqnarray}
\PiPS[\text{int 3}](\Sigma, r, \phi) & = & \PiPS[\text{asymp}](\Sigma, r, \phi) \nonumber\\
&&\times\left(1 + b_1(r,\phi) \Sigma^{-1} \ln\bigg(\frac{\Sigma}{\LQCD^2}\right) + \nonumber\\
&&\quad\sum_{i = 1}^{N-1}b_{i+1}(r, \phi) \Sigma^{-i}\bigg)
\label{Eq:interpolant3}
\end{eqnarray}
and use again $N \in \{2,3\}$.

%%%%%%%%%%%%%%%%%%%%%%%%%%%%%%%%%%%%%%%%%%%%%%%%%%%%%%%

\subsection{The matching surface \texorpdfstring{$\Sigma^\text{match}$}{Sigma match}}
\label{Sec:Sigmamatch}
The remaining crucial ingredient in our procedure is the function $\Sigma^\text{match}(r,\phi)$, which determines the value of $\Sigma$ at which the matching is performed for given $r$ and $\phi$. Choosing it too low leads to important modifications of $\PiPS$ at low energies with consequent overestimation of $\aPS$.\footnote{We denote by $\aPS$ the result of the integral in Eq.~(\ref{Eq:MasterFPi1}) for $\bar{\Pi}_1 \equiv \PiPS$.} Conversely, choosing $\Sigma^\text{match}$ too high assumes the low-energy input to dominate beyond what is expected according to mass and phase-space considerations and thus leads to underestimate $\aPS$.

For small values of $Q_3^2$, the $\pi^0$, $\eta$, $\eta'$ poles are assumed to dominate independently of $Q_{1,2}^2$, due to the pole at $Q_3^2 = -m_\text{PS}^2$ (see Secs.~\ref{Sec:SDCBTT} and~\ref{Sec:OPE}). This implies that no matching is needed in this regime, {\it i.e.}\ $\Sigma^\text{match}(1,\pi) = \infty$. The most general function that is analytic for all $(r,\phi)$ except for a (first-order) pole at $(r,\phi) = (1,\pi)$ can be written as
\begin{equation}
\Sigma^\text{match}(r,\phi) = \frac{3m^2}{1+r \cos \phi}\left(1 + P(r \cos \phi, r \sin \phi)\right)\, ,
\label{Eq:MatchSc}
\end{equation}
where $m^2$ determines the matching scale at $r = 0$ and $P$ is a polynomial with two arguments and no constant term. The transformation property of $\bar{\Pi}_1$ under crossing specified in Eq.~(\ref{Eq:crossing}) restricts $P$ to contain only even powers of $r \sin \phi$.

The parameter $m^2$ sets the absolute mass scale of $\Sigma^\text{match}$ and should thus be related to the masses of the states affecting $\bar{\Pi}_1$ beyond the ones explicitly included, namely $\pi^0$, $\eta$, $\eta'$ here.
In the following, we will assume that contributions to $\bar{\Pi}_1$ in the $g-2$ kinematics stemming from multi-particle intermediate states are dominated by narrow resonances while non-resonant effects lead to negligible corrections to the matching procedure and can be simply added to our final results.\footnote{We have checked the effects of the inclusion of the pion-loop contribution to $\bar{\Pi}_1$~\cite{BTT3} in the low-energy representation. Since the two-pion state contains a five-dimensional representation of the isospin group, a full decomposition into $\PiPS$ is not possible. However, even if its complete contribution is added to the isovector channel, we find that at the current level of accuracy it is irrelevant whether the pion loop is included in the matching procedure or not.} This is realized for example in the large-$N_c$ limit of pure QCD: since the short-distance expressions for $\bar{\Pi}_1$ in both symmetric and asymmetric limits scale like $N_c$, these can indeed be saturated by single-meson exchanges (see Ref.~\cite{deRafael}). Non-resonant contributions from multi-hadron intermediate states (like $2\pi$, $2 K$, $\pi\eta$, $3\pi$, \dots) are sub-leading for large $N_c$ and thus cannot contribute to the SDCs. Since scalar mesons have no impact on $\bar{\Pi}_1$, the lightest states beyond the ground-state pseudoscalars that are the most relevant at small $Q_3^2$ (see Sec.~\ref{Sec:SDCBTT}) are the axial mesons like $a_1(1260)$ and the tensor mesons like $f_2(1270)$, with masses in the \SIrange[range-units=single, range-phrase=--]{1}{2}{\GeV} region, whose effects on $a_\mu^\text{HLbL}$ can presently be estimated only using hadronic models.

For $P(x,y) = 0$, $\Sigma=\Sigma^\text{match}$ corresponds to $Q_3^2 = m^2$. Since a state of mass $M$ ceases to be suppressed by the denominator $(Q_3^2 + M^2)$ compared to lighter states when $Q_3^2$ approaches $M^2$, $m^2$ should be chosen well below $M^2$. At the same time, it should not be taken too small, because we do not expect any large contribution to $\bar{\Pi}_1$ at $Q_3^2 \ll M^2$. We thus regard $m^2 = \SI{0.5}{\,\GeV^2}$ as a good starting point for our analysis. In Sec.~\ref{Sec:NumMatch} we will discuss a range of choices for this parameter as well as the effects of the polynomial
\begin{equation}
P(x,y) = \sum_{i=0}^{M} \sum_{j=0}^{M} \frac{1}{i!\ j!} p_{i,j}\ x^i y^j\,,
\label{Eq:PolynomialSigmaMatch}
\end{equation}
which we have estimated by means of a Monte Carlo sampling over the coefficients $p_{i,j}$.