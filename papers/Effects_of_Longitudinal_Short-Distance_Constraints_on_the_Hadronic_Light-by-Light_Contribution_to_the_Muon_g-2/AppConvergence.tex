\section{Convergence of the interpolants}
\label{App:Convergence}
In this appendix we discuss under which conditions our interpolants in Eqs.~(\ref{Eq:interpolant12}) and (\ref{Eq:interpolant3}) converge to the true $\PiPS$ as $N\to \infty$. To this end, we assume that $\PiPS$ is known exactly in the region below the matching surface, {\it i.e.}\ for $\Sigma < \Sigma^\text{match}(r,\phi)$, and that $\PiPS \to \PiPS[\text{asymp}]$ for asymptotic $\Sigma$.

The BTT scalar function $\PiPS$ is free of kinematic singularities and analytic except for poles and branch cuts for configurations where the real part of at least one $Q_i^2$ is negative. For fixed $(r,\phi)$ with $0 \le r < 1$ and $0 \le \phi < 2\pi$, $\PiPS$ is an analytic function of $\Sigma$ except for poles and branch cuts for $\Re(\Sigma) < 0$. $\PiPS[\text{asymp}]$ for fixed $(r,\phi)$ is also an analytic function except for isolated poles at $\Sigma \leq 0$ (see Eq.~(\ref{Eq:AsymPi})). The ratio $\PiPS / \PiPS[\text{asymp}]$ therefore has the same singularities as $\PiPS$ and has a pole at the zero of $\PiPS[\text{asymp}]$, which we assume to be at $\Sigma^\text{pole} < \Sigma^\text{match}$.
We can thus write the ratio as a Taylor series in $\Sigma^{-1}$ at $(\Sigma^\text{match})^{-1}$,
\begin{equation}
\frac{\PiPS(\Sigma)}{\PiPS[\text{asymp}](\Sigma)} = \sum_{i=0}^{\infty} a_i \left(\frac{1}{\Sigma} - \frac{1}{\Sigma^\text{match}}\right)^i\,.
\end{equation}
This series converges for $\Sigma^{-1} \in (2(\Sigma^\text{match})^{-1} - (\Sigma^\text{pole})^{-1},\allowbreak (\Sigma^\text{pole})^{-1})$ or equivalently for $\Sigma \in (\Sigma^\text{pole}, \infty)$ if the relation $\Sigma^\text{pole} < \Sigma^\text{match}/2$ holds, which will be checked below. Since $\PiPS / \PiPS[\text{asymp}] \to 1$ as $\Sigma \to \infty$, we also know that
\begin{equation}
\sum_{i=0}^\infty a_i \left(-\Sigma^\text{match}\right)^{-i} = 1\,.
\label{Eq:AppAsymp}
\end{equation}
We can thus write
\begin{equation}
\begin{aligned}
\PiPS(\Sigma) &= \PiPS[\text{asymp}](\Sigma) \sum_{i=0}^{\infty} a_i \left(\frac{1}{\Sigma} - \frac{1}{\Sigma^\text{match}}\right)^i \\
&= \PiPS[\text{asymp}](\Sigma) \sum_{i=0}^{\infty} b_i \Sigma^{-i}\,,
\end{aligned}
\label{Eq:AppAsympSeries}
\end{equation}
where the $b_i$ are linear combinations of the $a_i$ with coefficients depending on $\Sigma^\text{match}$. In particular, $b_0 = 1$
due to Eq.~(\ref{Eq:AppAsymp}). Eq.~(\ref{Eq:AppAsympSeries}) shows that $\PiPS[\text{int 1}]$ converges to the true $\PiPS$ for $N \to \infty$ if $\Sigma^\text{pole} < \Sigma^\text{match}/2$ for all $(r,\phi)$ in the HLbL integration domain and $\Sigma > \Sigma^\text{pole}$. In the applications of our method, $N$ is limited to rather low values since $\PiPS$ and its derivatives at the matching surface are determined only from the $\pi^0$, $\eta$, $\eta'$-poles (and additionally from the lightest axial in Sec.~\ref{Sec:NumAxials}).

Let us now examine under which conditions the zero in $\PiPS[\text{asymp}]$ occurs for $\Sigma^\text{pole} < \Sigma^\text{match}/2$. This relation is independent of the low-energy input but depends on the pseudoscalar mass in $\PiPS[\text{asymp}]$. In our reference interpolant we set $P(x, y) = 0$ in $\Sigma^\text{match}$ and $m^2 = \SI{0.5}{\GeV^2}$. For these choices the zero in $\Pipion[\text{asymp}]$ is at sufficiently low $\Sigma$ for all $(r, \phi)$ and $m^2$ can be reduced down to \SI{0.0019}{\GeV^2} without violating the requirement $\Sigma^\text{pole} < \Sigma^\text{match}/2$. For the iso-singlet cases there is no zero for positive $\Sigma$ in $\Pietaetap[\text{asymp}]$ allowing all values for $m$. This does not place a serious limitation on the values for $m$ we consider in the uncertainty estimation in Sec.~\ref{Sec:NumMatch}.

Since $\PiPS[\text{asymp}]$ for fixed $(r,\phi)$ in the integration domain has poles only for $\Sigma \le 0$ and $\PiPS$ has no kinematic zero, also the ratio $\PiPS[\text{asymp}] / \PiPS$ can be expanded in $\Sigma^{-1}$ at positive $(\Sigma^\text{match})^{-1}$. The same line of arguments thus proves the convergence of interpolant 2 in Eq.~(\ref{Eq:interpolant12}) for $N\to\infty$ and the requirement $\Sigma^\text{pole} < \Sigma^\text{match}/2$ is trivially fulfilled due to $\Sigma^\text{pole} = 0$. The convergence of interpolant 3 given in Eq.~(\ref{Eq:interpolant3}) also easily follows from that of interpolant 1, because the logarithmic term can be written as a Taylor series at finite $(\Sigma^\text{match})^{-1}$ so that the parameter $b_1$ in Eq.~(\ref{Eq:interpolant3}) is redundant as $N\to\infty$. 
