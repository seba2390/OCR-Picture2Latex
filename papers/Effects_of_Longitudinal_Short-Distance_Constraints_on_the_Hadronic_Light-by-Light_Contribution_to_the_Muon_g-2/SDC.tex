\section{Longitudinal short-distance constraints on HLbL}
\label{Sec:SDC}
\subsection{Master formula for \texorpdfstring{$a_\mu^\text{HLbL}$}{a mu HLbL} and pseudoscalar pole contributions}
\label{Sec:SDCBTT}
In order to set up the notation, we start by summarizing the relevant definitions and results from Refs.~\cite{BTT2,BTT3}. The HLbL contribution to $a_\mu$ is governed by the fourth-rank vacuum polarization tensor for fully off-shell photon-photon scattering in pure QCD, 
\begin{eqnarray}
\Pi^{\mu\nu\lambda\sigma}(q_1,q_2,q_3) & = & -\imag \int \dif^4 x\, \dif^4 y \,\dif^4 z \, e^{-\imag(q_1\cdot x + q_2 \cdot y + q_3 \cdot z)} \nonumber \\
&&\times \bra{0}\T\{j^\mu (x) j^\nu (y) j^\lambda (z) j^\sigma (0) \} \ket{0}
\label{Eq:HLbLTensor}
\end{eqnarray}
with  momenta assigned as $q_1+q_2+q_3 = q_4$. In this expression, the electromagnetic current for the light quark triplet is given by
\begin{eqnarray}
j^\mu (x) & = & \bar{\psi}(x) {\Q} \gamma^\mu \psi(x)\,, \nonumber \\
\psi & = & (u,d,s)^T\,, \nonumber \\
\Q & = & \frac{1}{3}\operatorname{diag}(2,-1,-1)\, .
\label{Eq:EM}
\end{eqnarray}

By generalizing the procedure introduced by Bardeen and Tung~\cite{BardeenTung} and Tarrach~\cite{Tarrach} in their studies of doubly-virtual Compton scattering, it is possible to derive a generating redundant \enquote{BTT} set of \num{54} Lorentz structures,
\begin{equation}
\Pi^{\mu\nu\lambda\sigma} = \sum_{i=1}^{54} T_i^{\mu\nu\lambda\sigma} \Pi_i\,,
\label{Eq:BTT}
\end{equation}
which is manifestly gauge invariant, closed with respect to crossing relations and such that the scalar functions $\Pi_i$ are free of kinematic singularities.

The HLbL contribution to $a_\mu$ can be derived from the tensor $\Pi^{\mu\nu\lambda\sigma}$ by using projection operator methods~\cite{Aldins:1970id,Barbieri:1974nc,Jegerlehner:2008zza} in the static limit $q_4 \to 0$. After performing a Wick rotation to Euclidean momenta, angular averages~\cite{Rosner:1967zz, Levine:1974xh} lead to the master formula~\cite{BTT3}
\begin{eqnarray}
a_\mu^\text{HLbL} & = & \frac{2\alpha^3}{3\pi^2}\int_{0}^{\infty} \dif Q_1 \int_{0}^{\infty} \dif Q_2 \int_{-1}^{1} \dif \tau \sqrt{1-\tau^2}\,Q_1^3\, Q_2^3\,\nonumber\\
&& \times \sum_{i=1}^{12} T_i(Q_1,Q_2,\tau) \,\bar{\Pi}_i(Q_1,Q_2,\tau)\,,
\label{Eq:MasterFQ}
\end{eqnarray}
where $Q_{1,2}$ denote the magnitudes of the Euclidean loop four-momenta, $Q_{1,2} = \sqrt{-q_{1,2}^2}$, and $\tau$ is the cosine of the angle between these vectors. The scalar functions $\bar{\Pi}_i$ are linear combinations of the previous $\Pi_i$ for $q_4 \to 0$. The analytic expressions of the integration kernels $T_i$ are given in Ref.~\cite{BTT3}.

Parameterizing the three-dimensional integration domain by the coordinates \cite{EichmannParametrization}
\begin{equation}
\Sigma \in [0,\infty)\,,\quad r \in [0,1]\,,\quad \phi \in [0,2\pi)\, ,
\end{equation}
which are related to the non-vanishing photon virtualities by
\begin{eqnarray}
Q_1^2 & = & \frac{\Sigma}{3}\left(1 - \frac{r}{2} \cos\phi - \frac{r}{2} \sqrt{3}\sin \phi\right)\,, \nonumber \\
Q_2^2 & = & \frac{\Sigma}{3}\left(1 - \frac{r}{2} \cos\phi + \frac{r}{2} \sqrt{3}\sin \phi\right)\,, \nonumber \\
Q_3^2 & = & Q_1^2 + 2 Q_1 Q_2 \tau + Q_2^2 = \frac{\Sigma}{3}\left(1 + r \cos\phi\right)\,,
\end{eqnarray}
will prove very useful in the following discussion about asymptotic constraints on HLbL. The master formula in Eq.~(\ref{Eq:MasterFQ}) then takes the form
\begin{eqnarray}
a_\mu^\text{HLbL} & = & \frac{\alpha^3}{432\pi^2}\int_{0}^{\infty} \dif \Sigma \,\Sigma^3 \int_{0}^{1} \dif r\, r\sqrt{1-r^2} \int_{0}^{2\pi} \dif \phi \nonumber\\
&& \times\sum_{i=1}^{12} T_i(\Sigma,r,\phi) \bar{\Pi}_i(\Sigma,r,\phi)\, .
\end{eqnarray}

In terms of the $Q_i^2$ coordinates, the integration domain amounts to a cone with tip at the origin. In terms of $(\Sigma, r, \phi)$, a given point in this cone is specified by the distance $\Sigma$ to the tip of the point's projection on the symmetry axis ($\Sigma = Q_1^2+Q_2^2+Q_3^2$), and by the polar coordinates $r$ and $\phi$ on the plane containing the point and orthogonal to the symmetry axis, normalized such that $r = 1$ corresponds to the surface of the cone.

In the master formula, a special role is played by the scalar functions $\bar{\Pi}_{1,2}$, which fulfill
\begin{equation}
\bar{\Pi}_2 = \mathcal{C}_{2,3}[\bar{\Pi}_1] \text{ and}\quad \mathcal{C}_{1,2}[\bar{\Pi}_1] = \bar{\Pi}_1\, ,
\label{Eq:crossing}
\end{equation}
where the crossing operator $\mathcal{C}_{i,j}$ exchanges momenta and Lorentz indices of the photons $i$ and $j$. These functions are the only ones describing the effects of pseudoscalar tree-level exchanges. For small values of $\Sigma$, the pion pole dominates yielding the largest contribution to $a_\mu^{\text{HLbL}}$ and also $\eta$/$\eta'$ poles yield sizable effects. Furthermore, distinctively, the OPE SDCs on $\bar{\Pi}_{1,2}$ do not involve other scalar functions~\cite{BernSDCLong}.

The functional form of $\bar{\Pi}_{1,2}$ in specific kinematic regimes is constrained according to analytic QCD results, which we will fully exploit to estimate the impact of (longitudinal) SDCs on
\begin{eqnarray}
\aLon & \equiv & \frac{\alpha^3}{432\pi^2}\int_{0}^{\infty} \dif \Sigma  \int_{0}^{1} \dif r   \int_{0}^{2\pi} \dif \phi \,\Sigma^3\, r\sqrt{1-r^2}
\nonumber \\
&& \times \left[T_1(\Sigma, r, \phi) + T_2\left(\Sigma, r, \phi + \frac{2\pi}{3}\right)\right]\bar{\Pi}_1(\Sigma, r, \phi) \,,
\label{Eq:MasterFPi1}
\end{eqnarray}
where the shift in the variable $\phi$ in $T_2$ corresponds to the crossing operation on $\bar{\Pi}_1$. Thus for our analysis, we only need to study one BTT scalar function in the $g-2$ kinematics.

For the purpose of later discussion, we stress here that a pole term in $\bar{\Pi}_1$ due to a single-particle intermediate state of mass $M$ yielding the denominator $Q_3^2 + M^2$ leads to a hierarchy among contributions in the space-like momentum region relevant for $a_\mu^\text{HLbL}$. For small values of $Q_3^2$ larger masses get suppressed, while for $Q_3^2$ comparable to the squared mass of the heavier state or larger, no suppression is expected.\footnote{This argument obviously also holds if the denominator of the heavier state gets replaced by $M^2$ as for the axial-meson longitudinal contribution in Eq.~(\ref{Eq:Axial}) below (see also Ref.~\cite{Masjuan:2020jsf}).} This effect is of course modified by the numerator in $\bar{\Pi}_1$, which encodes information on the strength of the coupling to two (off-shell) photons, but it still helps us identify which states \emph{can} be relevant at specific energy scales and which not, independent of the values of $Q_{1,2}^2$.

The lightest state contributing to $\bar{\Pi}_1$ is $\pi^0$. The unitarity relation for a single pseudoscalar intermediate state yields
\begin{equation}
\bar{\Pi}_1^\text{PS-pole} = -\frac{F_{\text{PS} \gamma^* \gamma^*}(-Q_1^2, -Q_2^2) F_{\text{PS} \gamma^* \gamma^*}(-Q_3^2, 0)}{Q_3^2 + m_\text{PS}^2}\, ,
\label{Eq:Pi1PSpole}
\end{equation}
where the numerator is given by the product of a doubly-virtual and a singly-virtual transition form factor (TFFs) for an on-shell pseudoscalar meson (PS), which is defined by the matrix element
\begin{eqnarray}
&&\imag \int \dif^4 x\, e^{\,\imag q_1\cdot x} \bra{0} \T\{j_\mu(x) j_\nu(0)\}\ket{\text{PS}(q_1+q_2)} \nonumber\\
&&\quad= \epsilon_{\mu\nu\alpha\beta} q_1^\alpha q_2^\beta \,F_{\text{PS}\gamma^*\gamma^*}(q_1^2,q_2^2)
\label{Eq:TFF}
\end{eqnarray}
with $\epsilon^{0 1 2 3}=+1$. If we set $\bar{\Pi}_1 = \bar{\Pi}_1^\text{PS-pole}$, then $\aLon$ amounts to the pseudoscalar pole contribution $a_\mu^{\text{PS-pole}}$. In the $\pi^0$ case, this has been evaluated within a few percent accuracy via a data-driven dispersive approach~\cite{Hoferichter:2014vra,PionTFFshort,PionTFF},
\begin{equation}
a_{\mu,\text{disp}}^{\pi^0\text{-pole}} = 62.6^{+3.0}_{-2.5}\times 10^{-11}\, .
\label{Eq:aPionPole}
\end{equation}
This result agrees with other recent determinations based on lattice QCD~\cite{Gerardin:2019vio}, Canterbury approximants~\cite{Canterbury}, Dyson-Schwinger equations~\cite{DSE,Raya:2019dnh} and AdS/QCD models~\cite{Leutgeb:2019zpq}. While a dispersive analysis of the doubly-virtual $\eta$/$\eta'$ TFFs has not been completed yet,\footnote{The dispersive formalism for the singly-virtual $\eta$/$\eta'$ TFF has been established~\cite{Hanhart:2013vba} and progress has been made towards the determination of the doubly-virtual isovector contribution~\cite{Xiao:2015uva,Kubis:2018bej}.} the method of Canterbury approximants in Ref.~\cite{Canterbury} provides data-driven determinations 
and associated uncertainties also for the $\eta$/$\eta '$ TFFs.  In our numerical analysis of SDCs, we have used as input the dispersive $\pi^0$ TFF from Refs.~\cite{PionTFFshort,PionTFF} and compared our final results against those with form factors from Canterbury and Dyson-Schwinger approaches, while for $\eta/\eta'$ we have used the TFFs in Ref.~\cite{Canterbury} and compared against Ref.~\cite{DSE}.

The asymptotic constraints on $a_\mu^\text{HLbL}$~\cite{MV,Bijnens}  (see also Refs.~\cite{Knecht:2020xyr,Masjuan:2020jsf}) that we are going to discuss in the next sections have been translated into the BTT framework in Refs.~\cite{Bijnens,BernSDCShort,BernSDCLong}.
In this context, there are two distinct relevant kinematic regimes. The first (asymmetric) one is realized when one of the photon virtualities is much smaller than the other two, which are large and comparable, {\it e.g.}\ $Q_1^2 \sim Q_2^2 \gg Q_3^2$. The second (symmetric) limit occurs when all the Euclidean non-vanishing photon virtualities are large and comparable in size ($Q_1^2 \sim Q_2^2 \sim Q_3^2 \gg \LQCD^2 $). Both asymptotic limits correspond to $\Sigma \to \infty$ but for different values of $r$ and $\phi$: the asymmetric limit $Q_1^2 \sim Q_2^2 \gg Q_3^2$ corresponds to $r=1$ and $\phi = \pi$ while the symmetric configuration occurs in a neighborhood of $r=0$ (see Fig.~\ref{Fig:patches}). 

In the following we will review the relevant constraints on $\bar{\Pi}_1$ at large $\Sigma$ and describe in detail our method to provide general families of interpolants for $\bar{\Pi}_1(\Sigma, r, \phi)$ between low- and high-energy regions in the $g-2$ integral.

%%%%%%%%%%
\begin{figure}
	\centering
	\includegraphics{figs/AsymPlane}
	
	\caption{The circle represents the boundary of the $g-2$ integration domain for a fixed value of $\Sigma$. The angles $\phi = \pi/3$, $\phi = \pi$ and $\phi=5 \pi/3$ correspond to $Q_2^2= Q_3^2$, $Q_1^2 = Q_2^2$ and $Q_1^2=Q_3^2$, respectively. The colored regions denote where SDCs on $\bar{\Pi}_1$ hold at large $\Sigma$. The blue domains yield contributions to $\bar{\Pi}_1$ from the OPE expansion of the VVA correlator that are sub-leading compared to the green one, while the orange region corresponds to the pQCD constraint.}
	\label{Fig:patches}
\end{figure}
%%%%%%%%%%

%%%%%%%%%%%%%%%%%%%%%%%%%%%%%%%%%%%%%%%%%%%%%%%%%%%%%%%

\subsection{The asymmetric asymptotic region: OPE constraints}
\label{Sec:OPE}
For large Euclidean values of $\hat{q} \equiv (q_1 - q_2)/2$, one can expand the time-ordered product of two electromagnetic currents, which defines the tensor
\begin{equation}
\Pi^{\mu\nu}(q_1, q_2) = \imag \int \dif^4 x\, \dif^4 y \,e^{-\imag(q_1\cdot x + q_2 \cdot y)}\T\{j^\mu (x) j^\nu (y)\}\,,
\label{Eq:Pimunu}
\end{equation}
into a series of local operators. At leading order in $\alpha_s$, by matching single-quark matrix elements and omitting the unit operator, which does not contribute to the connected HLbL tensor in Eq.~(\ref{Eq:HLbLTensor}), one obtains~\cite{Bjorken:1966jh}
\begin{eqnarray}
&&\Pi^{\mu\nu}(q_1, q_2)\nonumber\\
&&\quad = \int \dif^4 z\, e^{-\imag(q_1 + q_2) \cdot z} \left(-\frac{2\imag}{\hat{q}^2}\epsilon^{\mu\nu\alpha\beta} \hat{q}_\alpha j_{5\beta}(z)\right) + \dots\, ,
\label{Eq:OPE}
\end{eqnarray}
where the axial current $j_5^\mu$ is defined by $j_5^\mu(x)=\bar{\psi}(x)\Q^2 \gamma^\mu \gamma_5 \psi(x)$ with charge matrix given in Eq.~(\ref{Eq:EM}). The ellipsis denotes sub-leading terms suppressed by powers of $\{|q_1+q_2|/|\hat{q}|,\,\allowbreak\Lambda_{\rm QCD}/|\hat{q}|\}$. This result implies that, at leading order in the OPE and at leading order in $\alpha_s$, the HLbL tensor can be expressed in terms of the correlator of two vector currents with an axial current,
\begin{eqnarray}
\Pi_{\mu\nu\lambda\sigma}(q_1,q_2,q_3) & = & \frac{2 \imag}{\hat{q}^2} \epsilon_{\mu\nu\alpha\beta} \hat{q}^\alpha \int \dif^4 x\dif^4 y\, e^{-\imag q_3\cdot x} e^{\imag q_4\cdot y} \nonumber \\
&& \times\bra{0} \T \{j_\lambda(x) j_\sigma(y) j_5^\beta(0)\}\ket{0} + \dots \nonumber \\
\end{eqnarray}
for $Q_1^2 \sim Q_2^2 \gg \{Q_3^2,Q_4^2,\LQCD^2\}$. This three-point function, which also appears in the calculation of fermion loop electroweak contributions to $a_\mu$~\cite{Knecht:2002hr,VVAold}, can be decomposed into Lorentz structures that are longitudinal and transversal with respect to the Lorentz index of the axial current (see {\it e.g.}\ Ref.~\cite{VVA}). The corresponding longitudinal scalar function determines the asymptotic behavior of $\bar{\Pi}_1$ in the asymmetric region and is fixed by the axial Adler--Bell--Jackiw anomaly up to chiral corrections and the gluon anomaly. Neglecting these effects, which will be discussed below, this translates into the following constraint~\cite{BernSDCLong} for the singlet and octet flavor components of $\bar{\Pi}_1(Q_1^2, Q_2^2, Q_3^2)$, defined by the decomposition of the axial current,
\begin{equation}
\bar{\Pi}_1^{(a), {\rm OPE}}(Q^2, Q^2, Q_3^2) = -\frac{2 N_c \,C_a^2}{\pi^2 Q^2 Q_3^2}\quad {\rm for} \quad  a=\{3,8,0\}\,,
\label{Eq:OPEConstraint}
\end{equation}
where $C_a = \tr(\Q^2 \lambda_a)/2$ in terms of the charge matrix $\Q$  and Gell-Mann matrices $\lambda_a$. In particular,
\begin{equation}
C_3 = \frac{1}{6}\, ,\quad C_8 = \frac{1}{6\sqrt{3}}\, ,\quad C_0 = \frac{2}{3\sqrt{6}}\,.
\label{Eq:Ca}
\end{equation}
Since Eq.~(\ref{Eq:OPEConstraint}) relies on a perturbative calculation of the VVA correlator, it holds in the kinematic limit $Q_1^2 \sim Q_2^2 \equiv Q^2 \gg Q_3^2 \gg \LQCD^2$. For the non-singlet components ($a=3,8$), since perturbative~\cite{Adler:1969er} as well as non-perturbative~\cite{tHooft:1979rat, Witten:1983tw} corrections are absent in the chiral limit, the hierarchy between $\LQCD^2$ and $Q_3^2$ can be dropped. In contrast, the singlet channel ($a=0$) is affected by the gluon anomaly, even in the chiral limit. This does not modify Eq.~(\ref{Eq:OPEConstraint}) for $Q_3^2 \gg \LQCD^2$~\cite{BernSDCLong}, but the extrapolation to small $Q_3^2$ is only valid if in addition the large-$N_c$ limit is considered, where the anomaly vanishes. Furthermore, in the crossed kinematics ($Q_2^2 \sim Q_3^2 \gg Q_1^2$ and $Q_1^2 \sim Q_3^2 \gg Q_2^2$), the leading-order OPE contributions to $\bar{\Pi}_1$ vanish.

Let us now compare $\PiPS[\text{OPE}]$ against the pseudoscalar pole contributions, focusing on the pion pole first. In the chiral limit and using the fact that $\lim\limits_{Q^2\to\infty} Q^2 F_{\pi\gamma^*\gamma^*}(-Q^2, -Q^2) = 4 C_3 F_\pi$ at leading order in $\alpha_s$~\cite{Novikov:1983jt,Manohar:1990hu}, one finds
\begin{minipage}{\columnwidth}
\begin{equation}
\lim\limits_{Q^2 \to \infty} Q^2 \bar{\Pi}_1^\text{$\pi^0$-pole}(Q^2, Q^2, Q_3^2) = -4 C_3 F_\pi \frac{F_{\pi\gamma^*\gamma^*}(-Q_3^2, 0)}{Q_3^2}\, .
\end{equation}
\end{minipage}
This expression has a pole at $Q_3^2 = 0$ since $F_{\pi\gamma^*\gamma^*}(0, 0) = 3 C_3 / (2\pi^2 F_\pi)$. The location of this pole as well as its residue agree with $\Pipion[\text{OPE}]$ in Eq.~(\ref{Eq:OPEConstraint}) (cf.~Ref.~\cite{MV}), which is consistent with the pion being the only massless isovector state in the chiral limit.

For finite quark masses, the pole in $\bar{\Pi}_1^\text{$\pi^0$-pole}$ is shifted from $Q_3^2 = 0$ to $Q_3^2 = -\mpi^2$, which lies outside the integration domain for $a_\mu^{\text{HLbL}}$. The closest point in the integration region for fixed asymptotic $\Sigma$ (see Fig.~\ref{Fig:patches}) is at $Q_3^2 = 0$, where
\begin{eqnarray}
\lim\limits_{Q^2 \to \infty} Q^2 \bar{\Pi}_1^\text{$\pi^0$-pole}(Q^2, Q^2, 0) & = & -4 C_3 F_\pi \frac{F_{\pi\gamma^*\gamma^*}(0, 0)}{\mpi^2} \nonumber \\ & = & - \frac{6 C_3^2}{\pi^2 \mpi^2}\, .
\label{Eq:PiPoleVanQ3}
\end{eqnarray}
Here few percent chiral corrections to $F_{\pi\gamma^*\gamma^*}(0, 0)$~\cite{Moussallam:1994xp, Goity:2002nn, Ananthanarayan:2002kj, Ioffe:2007eg, Kampf:2009tk} have been neglected. This is still close to the actual pole, which leads to the enhancement by $\mpi^{-2}$. Since no other contribution receives the same enhancement, the last expression is expected to provide an excellent approximation to the true $\Pipion$ in the specified limit.\footnote{At variance with the MV model of Ref.~\cite{MV}, we do not neglect the momentum dependence of the singly-virtual TFF and we allow for the contribution from other states besides the pion at finite $Q_3^2$.} We observe that the OPE result, which is derived in the chiral limit, reproduces Eq.~(\ref{Eq:PiPoleVanQ3}) if the pole position is shifted by the pion mass as dictated by the pion pole contribution
\begin{equation}
\lim\limits_{Q^2\to\infty} Q^2 \Pipion(Q^2, Q^2, Q_3^2) = -\frac{6 C_3^2}{\pi^2 (Q_3^2 + \mpi^2)}\, .
\label{Eq:OPEPionMass}
\end{equation}
This is also consistent with the OPE result in Eq.~(\ref{Eq:OPEConstraint}) for $Q^2 \gg Q_3^2 \gg \LQCD^2$, where chiral corrections are sub-leading. Thus, Eq.~(\ref{Eq:OPEPionMass}) is exact for $Q_3^2 \gg \LQCD^2$, relies on the assumption of pion dominance at $Q_3^2 \ll \LQCD^2$ and has the correct chiral limit Eq.~(\ref{Eq:OPEConstraint}) for all $Q_3^2$.  We extend it to the $\eta$/$\eta'$ channels and write
\begin{equation}
\lim\limits_{Q^2\to\infty} Q^2 \PiPS(Q^2, Q^2, Q_3^2) = -\frac{6 C_\text{PS}^2}{\pi^2 (Q_3^2 + m_\text{PS}^2)}\, .
\label{Eq:OPEConstraintMasses}
\end{equation}
Here $C_\pi = C_3$ but $C_\etaetap$ cannot be directly identified with $C_{0/8}$ due to $\eta$-$\eta'$-mixing. In analogy to the pion channel, we assume that ground-state single-pseudoscalar exchanges dominate $\Pietaetap(Q^2, Q^2, 0)$, despite the fact that the $\eta$/$\eta'$ poles are further away from $Q_3^2 = 0$. This assumption implies that $C_\etaetap$ can be read off from the pole contributions
\begin{equation}
\lim\limits_{Q^2\to\infty} Q^2\bar{\Pi}_1^{\text{$\etaetap$-pole}}(Q^2, Q^2, 0) = -\frac{6 C_{\etaetap}^2}{\pi^2 m_{\etaetap}^2}\,.
\label{Eq:defCetaetap}
\end{equation}

One can show that in the chiral limit and neglecting the gluon anomaly~\cite{BernSDCLong}
\begin{eqnarray}
&&\lim\limits_{Q_3^2\to 0} \, \lim\limits_{Q^2\to\infty} Q^2 Q_3^2 \nonumber \\ &&\times\left(\bar{\Pi}_1^{\eta\text{-pole}}(Q^2, Q^2, Q_3^2) + \bar{\Pi}_1^{\eta'\text{-pole}}(Q^2, Q^2, Q_3^2)\right) \nonumber \\ &&\quad= -\frac{6 (C_8^2 + C_0^2)}{\pi^2}\,.
\label{Eq:mixingOPE}
\end{eqnarray}
At this point we note that, besides the $\alpha_s$ corrections to the TFFs and OPE coefficient discussed in Sec.~\ref{Sec:MVpert}, which affect all ground-state pseudoscalars in the same way, the gluon anomaly induces a running of the flavor singlet decay constant~\cite{Leutwyler:1997yr,Kaiser:1998ds,Kaiser:2000gs}. This running leads to an incomplete cancellation between the decay constants in the symmetric asymptotic and the real photon limits, which has a sizable impact due to the large scale separation~\cite{Agaev:2014wna,PabloPhD}. 

Since $\bar{\Pi}_1^{\text{$\etaetap$-pole}}$ can be expressed in terms of TFFs according to Eq.~(\ref{Eq:Pi1PSpole}), assuming that corrections due to non-vanishing meson masses are negligible both in the real photon limit and in the symmetric asymptotic limit of the $\etaetap$ TFFs, Eqs.~(\ref{Eq:OPEConstraintMasses}--\ref{Eq:mixingOPE}) together imply
\begin{equation}
C_\eta^2 + C_{\eta'}^2= C_8^2 + C_0^2
\label{Eq:CPhysFlavor}
\end{equation}
up to the above-mentioned anomaly-induced scale-dependence, which leads to a violation of this equality (cf.\ Sec.~\ref{Sec:NumIS}).

For $Q_3^2 \gg \LQCD^2$, the additional $Q_3^2$-suppression of the singly-virtual TFF leads to a mismatch between the pseudoscalar pole contributions and the OPE constraint. In Ref.~\cite{MV} MV proposed to solve this issue by setting the singly-virtual TFF equal to a constant. This prescription is not compatible with the dispersive definition of the pole contributions in the framework summarized in Sec.~\ref{Sec:SDCBTT}, according to which, instead, an infinite tower of heavier intermediate states is needed to saturate the constraint (see {\it e.g.}\ Ref.~\cite{BernSDCLong}). For this purpose, summations of series of contributions from excited pseudoscalars~\cite{BernSDCShort,BernSDCLong} and axials~\cite{HolographyVienna,HolographyItaly} have been recently performed in the context of hadronic models. In Secs.~\ref{Sec:NumResults} and~\ref{Sec:NumAxials}, we will compare the outcome of our analysis against these estimates of the effects of longitudinal SDCs.

%%%%%%%%%%%%%%%%%%%%%%%%%%%%%%%%%%%%%%%%%%%%%%%%%%%%%%%

\subsection{\texorpdfstring{$\alpha_s$}{alpha\_s} corrections to the OPE}
\label{Sec:MVpert}

The derivation of Eq.~(\ref{Eq:OPE}) has been performed at leading order in $\alpha_s$. Since no other operator of dimension 3 can appear in that OPE, $\alpha_s$ corrections only affect the OPE coefficient of the axial-vector current. In Refs.~\cite{Kodaira:1978sh, Kodaira:1979ib, Kodaira:1979pa}, this coefficient has been calculated to next-to-leading order (NLO). Including this contribution in Eq.~(\ref{Eq:OPE}) leads to
\begin{eqnarray}
\Pi^{\mu\nu}(q_1, q_2) & = & \int \dif^4 z \,e^{-\imag(q_1 + q_2) \cdot z} \nonumber \\
&&\times\bigg(-\frac{2\imag}{\hat{q}^2} \left(1 - \frac{\alpha_s}{\pi}\right) \epsilon^{\mu\nu\alpha\beta} \hat{q}_\alpha j_{5\beta}(z) \nonumber\\
&&\qquad+ \Order\left(\hat{q}^{-2}\right)\bigg)\,.
\label{Eq:OPENLO}
\end{eqnarray}
It follows that the NLO version of Eq.~(\ref{Eq:OPEConstraintMasses}) reads
\begin{equation}
\lim\limits_{Q^2 \to \infty} Q^2 \PiPS(Q^2, Q^2, Q_3^2) = -\frac{6 C_\text{PS}^2 }{\pi^2 (Q_3^2 + m_\text{PS}^2)}\left(1-\frac{\alpha_s}{\pi}\right)\, .
\label{Eq:OPEConstraintNLO}
\end{equation}

The two-current operator product not only enters the HLbL tensor, but also the pion TFF (see Eq.~(\ref{Eq:TFF})). Thus, any perturbative correction to the OPE Wilson coefficient automatically implies the same perturbative correction to the symmetric limit of the pion TFF and vice versa. In fact, the symmetric asymptotic pion TFF has been calculated to NLO in Refs.~\cite{Braaten,PionTFF},\footnote{In Ref.~\cite{Braaten} the hard scattering kernel has been computed to NLO. In the limit $Q_1^2 = Q_2^2$ this is independent of the momentum fraction carried by the interacting quark, which makes the result independent of the pion distribution amplitude.}
\begin{equation}
F_{\pi\gamma^*\gamma^*}(-Q^2,-Q^2) = \left(1 - \frac{\alpha_s}{\pi} + \Order\left(\alpha_s^2\right)\right)\frac{2 F_\pi}{3Q^2} + \Order\left(Q^{-4}\right)\,,
\end{equation}
which is consistent with Eq.~(\ref{Eq:OPENLO}). The fact that the $\alpha_s$ corrections agree between the HLbL tensor in the asymmetric asymptotic limit and the symmetric asymptotic pion TFF guarantees that the pion pole saturates $\Pipion$ at $Q_3^2 = 0$ in the chiral limit also beyond leading order in $\alpha_s$.

A comment on the renormalization scale dependence of the terms in Eq.(\ref{Eq:OPENLO}) is in order here. The non-singlet components of the axial current are conserved (up to quark mass corrections) and thus their anomalous dimensions vanish. This is not true for the singlet component due to the gluon anomaly~\cite{Espriu:1982bw, Bos:1992nd}, but we neglect this effect here because it starts at $\Order(\alpha_s^2)$. Therefore, since the perturbatively expanded dimensionless part $d$ of the Wilson coefficient is scale ($\mu$) independent,
\begin{equation}
d\left(-\frac{\hat{q}^2}{\mu^2}, \alpha_s(\mu^2)\right) = d(1, \alpha_s(-\hat{q}^2)) 
\end{equation}
and the terms $\alpha_s^n \ln^{n-1} (-\hat{q}^2/\mu^2)$ ($n\ge 1$) can be resummed using as input the $\beta$-function and the one-loop result with $\alpha_s$ evaluated at the scale $-\hat{q}^2$ (see also~\cite{Narison:1992fd}).

%%%%%%%%%%%%%%%%%%%%%%%%%%%%%%%%%%%%%%%%%%%%%%%%%%%%%%%

\subsection{The symmetric asymptotic limit: perturbative QCD constraints}
\label{Sec:pQCD}
In Ref.~\cite{Bijnens} it has been shown that the pQCD quark loop is the leading term of an OPE in the kinematic limit $Q_1^2 \sim Q_2^2 \sim Q_3^2 \gg \LQCD^2$, where the fourth (external) photon has vanishing momentum in $(g-2)$-kinematics. At leading order in this OPE and at leading order in $\alpha_s$~\cite{BernSDCLong}, 
\begin{eqnarray}
\bar{\Pi}_1^\text{pQCD} (q_1^2, q_2^2, q_3^2) & = & \frac{N_c \tr \Q^4}{16\pi^2} \int_{0}^{1} \dif x \int_{0}^{1-x} \dif y \, I_1(x,y) \nonumber \\
& = & \frac{1}{24\pi^2} \int_{0}^{1} \dif x \int_{0}^{1-x} \dif y\,  I_1(x,y)\,, \nonumber \allowdisplaybreaks\\
I_1(x,y) & = & -\frac{16 x(1-x-y)}{\Delta_{132}^2} \nonumber\\
&&- \frac{16 x y(1-2x)(1-2y)}{\Delta_{132} \Delta_{32}}\,, \nonumber \\
\Delta_{ijk} & = & m_q^2-xy q_i^2 -x(1-x-y)q_j^2 \nonumber \\
&&- y(1-x-y) q_k^2\,, \nonumber \\
\Delta_{ij} & = & m_q^2-x(1-x)q_i^2 - y(1-y) q_j^2\,.
\label{Eq:pQCDIntegral}
\end{eqnarray}

In the symmetric limit, neglecting terms that are suppressed by powers of $m_q^2/Q^2$,
\begin{eqnarray}
\bar{\Pi}_1^\text{pQCD} (Q^2, Q^2, Q^2) & = & \sum_{a={3,8,0}} \PiPS[\text{pQCD}](Q^2, Q^2, Q^2) \nonumber \\
& = & \sum_{a={3,8,0}} -  \frac{4 N_c C_a^2}{3 \pi^2 Q^4}\, ,
\label{Eq:pQCDConstraint}
\end{eqnarray}
where we have chosen to adopt the same flavor decomposition as for the asymmetric OPE case, Eq.~(\ref{Eq:OPEConstraint}). If higher-order perturbative corrections are small, the leading-order result above is expected to be a good approximation also away from the fully symmetric configuration as long as large logarithms of ratios of momenta are absent.

Since $\bar{\Pi}_1^\text{PS-pole}$ decays like $Q^{-6}$, (towers of) ha\-dro\-nic contributions beyond ground-state pseudoscalar poles have to be responsible for the behavior shown by Eq.~(\ref{Eq:pQCDConstraint}). Following the MV prescription in Ref.~\cite{MV}, the parametric dependence on $Q$ can be reproduced but with an incorrect coefficient.

In order to saturate the pQCD result in the isosinglet channels, we need coefficients $C_\etaetap^\text{pQCD}$ satisfying 
\begin{equation}
C_8^2 + C_0^2 = \left(C^\text{pQCD}_{\eta}\right)^2 + \left(C^\text{pQCD}_{{\eta'}}\right)^2\,.
\label{Eq:CPhysFlavorQCD}
\end{equation}
Since Eq.~(\ref{Eq:CPhysFlavor}) is violated, we define
\begin{equation}
\left(C_\etaetap^\text{pQCD}\right)^2 = (1 + \delta_0) C_\etaetap^2\, ,
\label{Eq:delta0}
\end{equation}
where the parameter $\delta_0$ is chosen such that Eq.~(\ref{Eq:CPhysFlavorQCD}) holds.