\section{Numerical results and error analysis}
\label{Sec:Num}
\subsection{The isovector channel}
\label{Sec:NumIV}
The isovector channel is best suited to our method since it is characterized by a large contribution from the low-energy region dominated by the well-known pion pole, which does not mix (strongly) into the other flavor channels. The lightest one-particle intermediate state beyond the $\pi^0$ in this channel is the $a_1(1260)$, whose effect at low energies is suppressed by the large mass gap. The numerical dominance of this channel at low energies, however, does not imply that the same holds true at intermediate and high energies. In fact, the values of $C_a$ in Eq.~(\ref{Eq:Ca}) make the flavor singlet channel the numerically most important one in the asymptotic region where meson masses can be neglected, {\it i.e.}\ for $Q_i^2 \gg \LQCD^2$. In Sec.~\ref{Sec:NumIS} we will discuss the inclusion of $\eta/\eta'$ and in Sec.~\ref{Sec:NumAxials} also the case of the isovector ground-state axial, which is however affected by a larger degree of model dependence.

We start by selecting a \enquote{reference} set of assumptions and input parameters. The impact of their modifications will be assessed in the next sections and will define the range of our predictions in the form of an uncertainty band. This procedure allows us also to examine how the estimate of the effects of SDCs would be improved by more precise information on the pion pole, the contributions from states with masses around \SI{1}{\GeV} and the asymptotic regime.

As low-energy reference input, we took the leading-order dispersive $\pi^0$ singly- and doubly-virtual TFFs~\cite{PionTFFshort,PionTFF}, while the corresponding $\Order{(\alpha_s)}$ correction is included in the uncertainty. As reference interpolating function, we used $\Pipion[\text{int 1}]$ with $N=3$ (see Eq.~(\ref{Eq:interpolant12})), which turned out to yield results that are central in the range spanned by the interpolants 1, 2 and 3 and $N\in \{2,3\}$ (cf.\ Eq.~(\ref{Eq:NumInterpolant}) below). For the asymptotic function, we included information from pQCD away from $r=0$ in the way explained in Sec.~\ref{Sec:HEInt}, while $\alpha_s$ corrections contribute to the uncertainty. For the matching surface we used Eq.~(\ref{Eq:MatchSc}) with $P(x,y) = 0$ and $m_\text{ref}^2 = \SI{0.5}{\GeV^2}$. The resulting function, which we call $\Pipion[\text{ref}]$, is shown in Fig.~\ref{Fig:matchPionCenter} for $r=0$ together with the uncertainty band for the interpolants that we are going to discuss in the next sections. Our reference outcome for the contribution to $\aLon$ due to the longitudinal SDCs in the isovector channel is
\begin{equation}
\Delta \aPion[\text{ref}] = \aPion[\text{ref}] - a_{\mu,\text{disp}}^{\pi^0\text{-pole}} = \num{2.56e-11}\,,
\label{Eq:refRes}
\end{equation}
where $\aPion[\text{ref}]$ comes from using $\Pipion[\text{ref}]$ in the master formula Eq.~(\ref{Eq:MasterFPi1}), and 
$a_{\mu,\text{disp}}^{\pi^0\text{-pole}}$ is given in Eq.~(\ref{Eq:aPionPole}) according to Refs.~\cite{PionTFFshort,PionTFF}.
In Sec.~\ref{Sec:ResultsIV} we will argue that our final result does not strongly depend on the choice of the reference set of parameters.

%%%%%%%%%%
\begin{figure*}
	\centering
	\includegraphics{figs/IntPiCenter}

	\caption{The pion pole contribution and associated uncertainty from Refs.~\cite{PionTFFshort,PionTFF} vs. the reference interpolant and its error band which includes all sources of uncertainty considered in the present analysis (see discussion in Secs.~\ref{Sec:NumTFF} -- \ref{Sec:NumMatch} below).}
	\label{Fig:matchPionCenter}
\end{figure*}
%%%%%%%%%%

%%%%%%%%%%%%%%%%%%%%%%%%%%%%%%%%%%%%%%%%%%%%%%%%%%%%%%%

\subsubsection{Pion TFF uncertainties}
\label{Sec:NumTFF}

We shall now describe the effects of modifying the different ingredients of the reference configuration, one by one, starting from the pion TFF.
By propagating the errors quoted in Refs.~\cite{PionTFFshort,PionTFF} for the dispersive determination of the pion TFF and by summing the different contributions in quadrature, taking as well into account that a modification of the TFF affects both terms in Eq.~(\ref{Eq:refRes}), we obtained an asymmetric error band around the reference result with boundary values
\begin{equation}
\delta^+_\text{TFF} \Delta \aPion = \num{0.06e-11}\,, \quad \delta^-_\text{TFF} \Delta \aPion = \num{0.13e-11}\,,
\end{equation}
which correspond to the asymmetric error for the dispersive $\pi^0$ TFF. Given the smallness of these uncertainties, the (negative) correlation between them and the uncertainties of $a_{\mu,\text{disp}}^{\pi^0\text{-pole}}$ can be safely neglected.

In order to study the impact of different pion TFF parameterizations, we compared the previous results against the ones obtained using, both for the construction of the interpolant and the evaluation of $a_{\mu}^{\pi^0\text{-pole}}$, the $C^1_2$ Canterbury approximant with $a_{\pi; 1,1} = 2 b_\pi^2$ of Ref.~\cite{Canterbury}  and the Dyson-Schwinger TFF from Ref.~\cite{DSE}. We obtained
\begin{eqnarray}
\Delta \aPion[\text{Can}] &=& \num{2.60e-11}\,, \nonumber\\ \Delta \aPion[\text{DSE}] &=& \num{2.52e-11}\,, 
\end{eqnarray}
which are both compatible with the reference result within the range given above. We conclude that the outcome of our analysis is very robust against changes in the TFF input and that the present knowledge of the pion TFF is sufficient for our purposes.
%%%%%%%%%%%%%%%%%%%%%%%%%%%%%%%%%%%%%%%%%%%%%%%%%%%%%%%

\subsubsection{Asymptotic uncertainties}
\label{Sec:AsymUnc}
Here we focus on the uncertainties in $\Pipion[\text{asymp}]$ (see Eq.~(\ref{Eq:AsymPi})), which are related to
\begin{itemize}
	\item the choices made in the fit to the quark-loop result that lead to Eq.~(\ref{Eq:AsymPi_a_pion}), namely the degree $M$ of the polynomial and the radius $r_\text{max}$ of the fitting domain;
	\item $\alpha_s$ corrections to the OPE constraint as given by Eq.~(\ref{Eq:OPEConstraintNLO});
	\item $\alpha_s$ corrections to the quark loop. 
\end{itemize} 
We start by discussing the fit to the quark-loop result. In Sec.~\ref{Sec:HEInt}, we chose $M=2$, which leads to a strongly improved fit quality compared to $M=1$. Considering a larger value of $M$ gives an estimate of the errors made by approximating the pQCD result by a polynomial at $r < r_\text{max}$ at fixed asymptotic $\Sigma$ and by extrapolating to the regime $r>r_\text{max}$, which is unknown except for the OPE constraint. Choosing $M=3$ shifts the result for $\Delta \aPion$ by only \num{0.02e-11} indicating that the truncation at $M=2$ is sufficient. We also studied the effects of a substantial reduction of the radius, namely from $r_\text{max} = 0.9$ down to $r_\text{max} = 0.5$, where no logarithm of ratios of squared momenta is larger than 1. We found that this leads to a small shift (\num{0.07e-11}). We did not consider $r_\text{max}>0.9$ since fixed-order pQCD is not expected to converge for $r$ close to 1 due to large logarithms. Combining linearly the uncertainties from the choice of $M$ and the fitting radius gives 
\begin{equation}
\delta_\text{pQCD fit} \Delta \aPion = \num{0.09e-11}\,,
\end{equation}
with respect to the reference contribution of longitudinal SDCs to the pion pole input in Eq.~(\ref{Eq:refRes}).


Let us now focus on the estimate of the separate perturbative corrections to either the OPE or the pQCD result. Since those concerning the OPE should not be extrapolated into the domain of validity of pQCD, for asymptotic $\Sigma$ we write (cf.\ Eq.~(\ref{Eq:OPEConstraintNLO}))
\begin{eqnarray}
\Pipion[\text{asymp, $\delta$OPE}] & = & \Pipion[\text{asymp}]\bigg[1-\frac{\alpha_s(\mu^2=Q_1^2 + Q_2^2)}{\pi} \nonumber \\
&&\qquad\times \operatorname{\theta}\!\left(A-\frac{Q_3^2}{Q_1^2 + Q_2^2}\right)\bigg]\,.
\label{Eq:Pi1NLOMVOPE}
\end{eqnarray}
Here the Heaviside step function $\theta$ ensures that the perturbative correction only affects a region around $Q_3^2 = 0$, whose size can be varied via the free parameter $A$. By setting $A=1/29$, this region does not extend into the $r<0.9$ domain. The choice of the renormalization scale $\mu^2$ is the same as in Ref.~\cite{PionTFF} and agrees with our discussion in Sec.~\ref{Sec:MVpert} up to a factor of 2 since $-\hat{q}^2 = (Q_1^2 + Q_2^2)/2 - Q_3^2/4 \approx (Q_1^2 + Q_2^2)/2$ in the relevant regime. In our numerical analysis, for the running of $\alpha_s$ we used the three-flavor one-loop beta function and matched to $\alpha_s(\mu^2 = M_\tau^2) = 0.35$.

According to the discussion in Sec.~\ref{Sec:SDC}, the OPE constraint in the chiral limit is saturated by the pion pole at $Q_3^2 = 0$ to all orders in perturbation theory. For this reason in a consistent analysis the OPE coefficient and the pion TFF in the symmetric limit should be taken at the same perturbative accuracy. Hence we replaced in Eq.~(\ref{Eq:interpolant12}) $\Pipion[\text{asymp}]$ by $\Pipion[\text{asymp, $\delta$OPE}]$ and matched the correspondingly modified $\Pipion[\text{int 1}]$ to the pion-pole contribution with TFFs including $\Order(\alpha_s)$ effects~\cite{Braaten, PionTFF}.\footnote{We thank Bai-Long Hoid for kindly providing us with a numerical representation of the dispersive pion TFF with $\Order(\alpha_s)$ corrections.} Using this interpolant and this pion-pole result, our outcome for $\Delta \aPion$ is larger than the reference result Eq.~(\ref{Eq:refRes}) by
\begin{equation}
\delta_\text{NLO OPE}^+ \Delta \aPion = \num{0.01e-11}\,.
\label{Eq:uncNLOMV}
\end{equation}
For $A=1/3$, the domain where the correction applies extends down to $r=0.25$, but nevertheless the shift of $\Delta \aPion$ turns out to be \num{-0.05e-11} and thus still negligible. The smallness of these shifts can be understood from the large values of $\Sigma^\text{match}$ in the region where these perturbative corrections apply. For this reason, the effect is almost completely included in the pion pole contribution, where it also has a small impact~\cite{PionTFF}.

Since the NLO calculation of the quark loop has not been performed yet, we can only provide a rough estimate of the uncertainty related to unknown ${\cal O}(\alpha_s)$ corrections. We assumed in analogy with Eq.~(\ref{Eq:Pi1NLOMVOPE}),
\begin{eqnarray}
\Pipion[\text{asymp, $\delta$pQCD}] & = & \Pipion[\text{asymp}] \bigg[1-\frac{\alpha_s(\mu^2=\Sigma)}{\pi} \nonumber\\
&&\qquad\times\theta\left(r_\text{max} - r\right)\bigg]\,,
\label{Eq:Pi1NLOpQCD}
\end{eqnarray}
and as in the leading-order quark loop fit, we set $r_\text{max}=0.9$. Using this expression in Eq.~(\ref{Eq:interpolant12}) for the matching to the pion-pole with leading-order dispersive TFF, we obtained a shift of \num{-0.18e-11} compared to the reference result. Even when inflating this uncertainty by a factor of 2,
\begin{equation}
\delta_\text{NLO pQCD} \Delta \aPion = \num{0.36e-11}
\end{equation}
this effect is still sufficiently small compared to the current precision goal. We stress that once NLO calculations become available, $\Pipion[\text{asymp}]$ should be constructed to \emph{analytically} interpolate between the NLO expressions for the OPE and the quark loop. The discontinuous functions employed here only serve to provide a ballpark estimate of NLO effects.

%%%%%%%%%%%%%%%%%%%%%%%%%%%%%%%%%%%%%%%%%%%%%%%%%%%%%%%

\subsubsection{Choice of interpolation functions}
\label{Sec:NumInt}
In Eqs.~(\ref{Eq:interpolant12}) and (\ref{Eq:interpolant3}) we have introduced three different interpolation functions, characterized by two or three free parameters to be matched to the low-energy representation. The corresponding results for the contribution from longitudinal SCDs are
\begin{eqnarray}
\Delta \aPion[\text{int 1},\ N=2] & = & \num{3.18e-11}\,, \nonumber \\
\Delta \aPion[\text{int 1},\ N=3] & = & \num{2.56e-11}\,, \nonumber \\
\Delta \aPion[\text{int 2},\ N=2] & = & \num{2.75e-11}\,, \nonumber \\
\Delta \aPion[\text{int 2},\ N=3] & = & \num{2.16e-11}\,, \nonumber \\
\Delta \aPion[\text{int 3},\ N=2] & = & \num{2.69e-11}\,, \nonumber \\
\Delta \aPion[\text{int 3},\ N=3] & = & \num{1.94e-11}\,,
\label{Eq:NumInterpolant}
\end{eqnarray}
where $\Delta \aPion[\text{int 1},\ N=3] = \Delta \aPion[\text{ref}]$ given by Eq.~(\ref{Eq:refRes}) has been included for completeness.

We observe that the slower logarithmic approach to the asymptotic limits in the interpolant 3 leads to smaller results, especially when compared to the similar interpolant 1. Setting
\begin{equation}
\delta_\text{int} \Delta \aPion = \num{0.62e-11}\,,
\end{equation}
all values listed above are within the range $\Delta \aPion[\text{ref}] \pm \delta_\text{int} \Delta \aPion$.

%%%%%%%%%%%%%%%%%%%%%%%%%%%%%%%%%%%%%%%%%%%%%%%%%%%%%%%

\subsubsection{Choice of \texorpdfstring{$\Sigma^\text{match}(r,\phi)$}{sigma match}}
\label{Sec:NumMatch}
The function $\Sigma^\text{match}(r,\phi)$ in Eq.~(\ref{Eq:MatchSc}) contains the mass parameter $m$ 
and the polynomial $P(x,y)$, which has been set equal to zero so far. 
We have argued in Sec.~\ref{Sec:Sigmamatch} that $m$ should be chosen considerably smaller than the mass $M$ of the lightest resonances contributing to $\bar{\Pi}_1$ in addition to the ground-state pseudoscalar mesons. For this reason, for the reference interpolant we set $m^2=\SI{0.5}{\GeV^2}$. Here we discuss the effects of alternative choices for this parameter within a range between $m_\text{min}$ and $m_\text{max}$. 

Since according to Sec.~\ref{Sec:Sigmamatch} a conservative choice for the upper end of the range is $m_\text{max} \simeq M$, we set $m_\text{max}^2 = \SI{1}{\GeV^2}$. In order to determine an appropriate value for $m_\text{min}$, one has to estimate isovector contributions beyond the $\pi^0$-pole. Following our argument in Sec.~\ref{Sec:Sigmamatch}, it is sufficient to restrict ourselves to single-particle intermediate states and focus on the one giving the largest effect at energies around $m$. We assumed this to be given by the pseudoscalar $\pi(1300)$ for the following reasons. Models for tensor mesons around \SI{1}{\GeV} give similar or smaller contributions to $a_\mu^\text{HLbL}$ \cite{Pauk:2014rta, Danilkin:2016hnh, Danilkin:2019mhd}.
For ground-state axials, recent studies based on different approximations and hadronic models yield quite different numerical results, see {\it e.g.}\ Refs.~\cite{Pauk:2014rta,Roig:2019reh,HolographyVienna,HolographyItaly}, leading to large uncertainties. If future model-independent analyses show that axial-meson exchanges are responsible for significant effects in $\Pipion$ also at relatively small momenta, then these contributions should be added to the pion pole before the matching is performed since our procedure relies on a sufficiently precise knowledge of $\Pipion$ below $\Sigma^\text{match}(r,\phi)$. Neglecting issues related to model dependence, in Sec.~\ref{Sec:NumAxials} we will discuss the inclusion in our procedure of information from holographic QCD on the lightest axial meson.

%%%%%%%%%%
\begin{figure*}
	\centering
	\includegraphics{figs/MMin}

	\caption{The figure displays the dispersive pion pole contribution, the reference interpolant and the (orange) band corresponding to the various choices of the parameter $m$. The blue line indicates the value of the matching surface for $m^2 = m_\text{min}^2$. The green band shows the sum of the $\pi^0$- and $\pi(1300)$-pole contributions, where the latter has been calculated using input from R$\chi$T and phenomenology, including errors.}
	\label{Fig:mFromRChiT}
\end{figure*}
%%%%%%%%%%

As for the light pseudoscalars, the interaction of $\pi(1300)$ with two photons can be described by a TFF, which determines the contribution to $\bar{\Pi}_1$ as in Eq.~(\ref{Eq:Pi1PSpole}).
In our analysis we used as input the $\pi(1300)$ TFF derived in Ref.~\cite{KampfRChiT} in the framework of Resonance Chiral Theory (R$\chi$T)~\cite{RChiT}. We fixed the free parameters in Eq.~(69) of Ref.~\cite{KampfRChiT} by requiring that (i) the $\pi^0$-TFF satisfies the Brodsky-Lepage condition, {\it i.e.}\ $F_{\pi^0\gamma^*\gamma^*}(-Q^2,0) = 2 F_\pi/Q^2 +\Order(Q^{-4})$ and (ii) the two-real-photon limit of the excited pion TFF is in the range $F_{\pi(1300)\gamma^*\gamma^*}(0,0)\in[0,\num{0.0544}]\si{\GeV^{-1}}$, argued for in Ref.~\cite{BernSDCLong} based on experimental results~\cite{Acciarri:1997rb,Salvini:2004gz}. The upper boundary of this interval leads to the most conservative error estimate in our analysis, and is used in the following. 

Our procedure to determine $m_\text{min}$ can be illustrated by means of Fig.~\ref{Fig:mFromRChiT}. Given a value of $m_\text{min}$, at large enough $\Sigma$, the range of interpolants (orange band) spanned by $m\in [m_\text{min},m_\text{max}]$ safely includes the green band representing the sum of the contributions from $\pi^0$ and $\pi(1300)$, including errors on the latter due to the range for $F_{\pi(1300)\gamma^*\gamma^*}(0,0)$. Since this is not the case at small $\Sigma$, the contribution to $a_\mu^\text{long}$ from this region is underestimated in our approach. To gauge this effect, we calculated the integral in Eq.~(\ref{Eq:MasterFPi1}) with
\begin{equation}
\bar{\Pi}_1 = \bar{\Pi}_1^\text{$\pi^0$-pole} + \bar{\Pi}_1^\text{$\pi(1300)$-pole} - \Pipion[\text{int}](m=m_\text{min})
\label{Eq:Pi_miss}
\end{equation}
using the maximal $\pi(1300)$ contribution and restricting the $\Sigma$-domain to the region below the point where the bands start to fully overlap (as a function of $r$ and $\phi$). This integral gives the missed contribution $\aPion[\text{missed}]$ at a fixed $m_\text{min}$. Repeating this calculation for different values of $m_\text{min}$ yields the function $\aPion[\text{missed}](m_\text{min})$ and by inverting this, we determined $m_\text{min}$ by fixing $\aPion[\text{missed}]$ to values well below the accuracy goal set by forthcoming experimental results.
For $\aPion[\text{missed}] = \num{0.5e-11}$ we obtained $m_\text{min, 1}^2 = \SI{0.35}{\GeV^2}$ and for $\aPion[\text{missed}] = \num{0.2e-11}$, $m_\text{min, 2}^2 = \SI{0.13}{\GeV^2}$.

Numerically, $m_\text{max} = \SI{1}{\GeV}$ leads to the shift $\delta_{m}^- \Delta \aPion = \num{1.20e-11}$ and the two values $m_\text{min, 1}$ and $m_\text{min, 2}$ yield $\delta_{m, 1}^{+\prime} \Delta \aPion = \num{0.81e-11}$ and $\delta_{m, 2}^{+\prime} \Delta \aPion = \num{3.66e-11}$, respectively. If we add $\aPion[\text{missed}](m_\text{min, \{1, 2\}})$ to the latter numbers, we obtain the conservative estimates
\begin{eqnarray}
\delta_{m}^- \Delta \aPion & = & \num{1.20e-11}\,, \nonumber \\
\delta_{m, 1}^+ \Delta \aPion & = & \num{1.31e-11}\,, \  \delta_{m, 2}^+ \Delta \aPion = \num{3.86e-11}\,.
\label{Eq:NumMatching}
\end{eqnarray}
In the following we will use  $\delta_{m, 1}^+ \Delta \aPion$ for the main results and keep $\delta_{m, 2}^+ \Delta \aPion$ as an alternative, even more conservative uncertainty. 

We also considered a different parameterization for the $\pi(1300)$ TFF, namely the one given by the Regge model in Refs.~\cite{BernSDCShort,BernSDCLong}. Using the empirical $m_{\pi(1300)} = \SI{1.30}{\GeV}$ instead of the Regge-model value of \SI{1.36}{\GeV} used in these references and following the same procedure discussed above, we obtained $m_\text{min, 1}^2 = \SI{0.53}{\GeV^2}$ and $m_\text{min, 2}^2 = \SI{0.20}{\GeV^2}$. This leads to
\begin{equation}
\delta_{m, 1}^+ \Delta \aPion = \num{0.36e-11}\, ,\quad \delta_{m, 2}^+ \Delta \aPion = \num{2.63e-11}\,,
\end{equation}
where we again added $a_\mu^\text{missed}$ to the uncertainties in the upward direction. For our final result we use the more conservative uncertainty estimates given in Eq.~(\ref{Eq:NumMatching}).

In order to study the effects of the polynomial in $\Sigma^\text{match}(r,\phi)$, Eq.~(\ref{Eq:PolynomialSigmaMatch}), we set $M=2$ and sampled the free parameters according to a standard normal distribution. Since the pion pole gives an excellent approximation of $\bar{\Pi}_1$ for very small $\Sigma$ at any $(r,\phi)$, we only allowed for parameters giving $\Sigma^\text{match}(r,\phi)>\Sigma_t$ for all $(r,\phi)$, where $\Sigma_t$ is defined as the smallest value of $\Sigma$ such that
\begin{equation}
\frac{\bar{\Pi}_1^\text{$\pi(1300)$-pole}(\Sigma,r,\phi)}{\bar{\Pi}_1^\text{$\pi^0$-pole}(\Sigma,r,\phi)} = 0.02
\label{Eq:MCRatio}
\end{equation}
holds for some $(r,\phi)$. With $\bar{\Pi}_1^\text{$\pi(1300)$-pole}$ calculated using R$\chi$T, we obtained $\Sigma_t = \SI{0.57}{\GeV^2}$. This condition ensures that there are no large contributions from our interpolation at points where R$\chi$T predicts a very small excited pion contribution. From this we calculated a distribution of results for $\Delta \aPion$, which features a Gaussian-like peak close to the reference result and asymmetric tails, and read off the \SI{16}{\percent} quantiles from both sides corresponding to the $1\sigma$ errors for a Gaussian. This gives
\begin{minipage}{\columnwidth}
\begin{equation}
\delta_{P(x,y)}^+ \Delta \aPion = \num{0.39 e-11}\ ,\quad \delta_{P(x,y)}^- \Delta \aPion = \num{0.32 e-11}\,.
\end{equation}
\end{minipage}
We have checked that this result is stable against the inclusion of terms of order 3 in the polynomial and moderate changes in the value of the ratio in Eq.~(\ref{Eq:MCRatio}).

%%%%%%%%%%%%%%%%%%%%%%%%%%%%%%%%%%%%%%%%%%%%%%%%%%%%%%%

\subsubsection{Estimate of the effects of longitudinal SDCs in the isovector channel}
\label{Sec:ResultsIV}

%%%%%%%%%%
\begin{table}
	\caption{The effects on $a_\mu^{\text{HLbL}}$ of longitudinal SDCs assuming that the low-energy region is dominated by ground-state pseudoscalar poles, whose contributions are taken as input. In each flavor channel the results are presented as the shifts $\Delta a_{\mu,\text{ref}}$ with respect to the pole contributions for a specific reference set of parameters and a list of uncertainties corresponding to different choices for each of these parameters. In the last two rows, these uncertainties are added in quadrature and the final range is symmetrized. See main text for details.}
	\label{Tab:results}
	{\def\arraystretch{1.5}\tabcolsep=10pt
		\begin{tabular*}{\columnwidth}{@{\extracolsep{\fill}}lrrr@{}}
			\hline
			& $\pi^0$ & $\eta$ & $\eta'$ \\ \hline
			$\Delta a_{\mu,\text{ref}} \times 10^{11}$ & $2.56$ & $2.58$ & $3.91$ \\ \hline
			$\delta_\text{TFF} \Delta a_{\mu} \times 10^{11}$ & $~^{+0.06}_{-0.13}$ & $0.47$ & $0.30$ \\
			$\delta_\text{pQCD fit} \Delta a_\mu \times 10^{11}$ & $0.09$ & $0.08$ & $0.14$ \\
			$\delta_\text{NLO OPE} \Delta a_{\mu} \times 10^{11}$ & $~^{+0.01}_{-0.00}$ & $~^{+0.01}_{-0.00}$ & $~^{+0.02}_{-0.00}$ \\
			$\delta_\text{NLO pQCD} \Delta a_{\mu} \times 10^{11}$ & $0.36$ & $0.36$ & $0.55$ \\
			$\delta_\text{int} \Delta a_{\mu} \times 10^{11}$ & $0.62$ & $~^{+0.61}_{-0.65}$ & $~^{+0.74}_{-0.84}$ \\
			$\delta_{m, 1} \Delta a_{\mu} \times 10^{11}$ & $~^{+1.31}_{-1.20}$ & $~^{+1.27}_{-1.17}$ & $~^{+1.68}_{-1.60}$ \\
			$\delta_{P(x,y)} \Delta a_{\mu} \times 10^{11}$ & $~^{+0.39}_{-0.32}$ & $~^{+0.31}_{-0.33}$ & $~^{+0.32}_{-0.43}$ \\ \hline
			$\delta_\text{tot} \Delta a_{\mu} \times 10^{11}$ & $~^{+1.55}_{-1.44}$ & $~^{+1.56}_{-1.50}$ & $1.97$ \\
			$\Delta a_{\mu} \times 10^{11}$ & $2.6\pm 1.5$ & $2.6\pm 1.5$ & $3.9\pm 2.0$ \\ \hline
		\end{tabular*}
	}
\end{table}

%%%%%%%%%%

\begin{figure}
	\centering
	\includegraphics{figs/Errors}
	
	\caption{Relative contributions to the total uncertainty in the isovector channel. For asymmetric errors the mean of the squared errors is used.}
	\label{Fig:PieChartErrors}
\end{figure}

%%%%%%%%%%

The $\pi^0$-column of Tab.~\ref{Tab:results} collects all uncertainties in our estimate of the effects of longitudinal SDCs in the isovector channel, as described in the previous subsections. By combining them in quadrature we get
\begin{equation}
\delta_\text{tot}^+ \Delta \aPion = \num{1.55e-11}\,, \quad \delta_\text{tot}^- \Delta \aPion = \num{1.44e-11}\,.
\end{equation}
Since we do not regard the reference parameterization as the central value, we symmetrized the uncertainty to finally obtain the range
\begin{equation}
\Delta \aPion = \num[separate-uncertainty]{2.6\pm 1.5 e-11}\, .
\end{equation}
Using instead $\delta_{m, 2}^+ \Delta \aPion$ in Eq.~(\ref{Eq:NumMatching}), the final result would be \num[separate-uncertainty]{3.8 \pm 2.7e-11}. Notice that, despite the fact that it likely overestimates the range of longitudinal short-distance effects, this interval is still definitely compatible with the current precision goal.

Fig.~\ref{Fig:PieChartErrors} shows the contributions to the quadratic error from the different sources discussed above. The vastly dominant effect stems from the interpolation between low and high energies, with an especially crucial role played by the choice of $m$, the scale at which the matching between the low-energy representation of $\bar{\Pi}_1$ and the interpolant is performed. The uncertainties $\delta_\text{int}$, $\delta_m$ and $\delta_{P(x,y)}$ could be reduced by additional low-energy input concerning further intermediate states and higher-order terms in the symmetric and asymmetric OPEs, which would help constrain the coefficients $b_i(r,\phi)$ in the interpolants in Eqs.~(\ref{Eq:interpolant12}) and (\ref{Eq:interpolant3}). The uncertainties related to the perturbative corrections are considerably smaller. While we do not expect that calculations of $\alpha_s$ corrections will crucially improve the final estimate, these perturbative results will definitely be important to better assess the regime of validity of the asymptotic constraints and thereby verify and sharpen some of our assumptions. 

We have also checked that our results are robust against the choice of different reference sets of parameters. For example, if we set $m_\text{ref}$ equal to the previous boundary values for the uncertainty in the reference configuration, namely $m_\text{ref}^2 = \SI{0.35}{\GeV^2}$ and $m_\text{ref}^2 = \SI{1}{\GeV^2}$ and choose $m^2\in [0.35, 1]\si{GeV^2}$ as the range for the error estimation as before, we get
\begin{eqnarray}
\Delta \aPion[m_\text{ref}^2 = \SI{0.35}{\GeV^2}] & = & \num[separate-uncertainty]{2.8\pm 1.7e-11}\,, \nonumber \\
\Delta \aPion[m_\text{ref}^2 = \SI{1.00}{\GeV^2}] & = & \num[separate-uncertainty]{2.4\pm 1.5e-11}\,,
\end{eqnarray}
where all other sources of uncertainty are included. We obtained similar results by selecting as reference different interpolants or different values of the number of free parameters $N$ contained therein.

%%%%%%%%%%%%%%%%%%%%%%%%%%%%%%%%%%%%%%%%%%%%%%%%%%%%%%%

\subsection{The isoscalar contributions}
\label{Sec:NumIS}
In this section the procedure presented above for the isovector case is applied to the isoscalar channels with $\etaetap$-poles as low-energy input. In our analysis, we employed the Canterbury TFFs from Ref.~\cite{Canterbury} in the reference solution.\footnote{In the conventions of  Ref.~\cite{Canterbury}, we used the $C^1_2$ approximant with $a_{\etaetap;1,1} = 2 b_\etaetap^2$ as for the pion.} We determined the parameters encoding $\eta-\eta'$-mixing as explained in Secs.~\ref{Sec:OPE} and~\ref{Sec:pQCD} and obtained
\begin{equation}
C_\eta = \num{0.164}\,, \quad C_{\eta'} = \num{0.219}\,, \quad \delta_0 = \num{0.110}\,,
\label{Eq:CCan}
\end{equation}
which shows that $\delta_0$ is indeed sizable.

Following the same procedure for the construction of the reference interpolant as in Sec.~\ref{Sec:NumIV}, we found
\begin{equation}
\Delta a_{\mu, \text{ref}}^\eta = \num{2.58e-11}\,, \quad \Delta a_{\mu, \text{ref}}^{\eta'} = \num{3.91e-11}\,.
\end{equation}

The uncertainty estimation proceeds in the same way as in the isovector channel up to minor modifications. Since error bands for the doubly-virtual TFFs in all kinematic configurations are not available in the literature, we estimated uncertainties by considering another TFF representation, namely the one based on Dyson-Schwinger equations (DSE)~\cite{DSE}. This yields $\Delta a_{\mu, \text{ref}}^\eta = \num{2.11e-11}$ and $\Delta a_{\mu, \text{ref}}^{\eta'} = \num{4.20e-11}$. The fact that individual results for $\eta$ and $\eta'$ channels differ by \num{18} and \SI{8}{\percent}, but the sum only by \SI{3}{\percent} can be understood by comparing the mixing parameters
\begin{equation}
C_\eta^\text{DSE} = \num{0.148}\,, \quad C_{\eta'}^\text{DSE} = \num{0.228}\,, \quad \delta_0^\text{DSE} = \num{0.127}\,,
\label{Eq:CDSE}
\end{equation}
against those obtained from the Canterbury parameterization. These coefficients enter $\Pietaetap[\text{asymp}]$ quadratically, which leads to a reshuffling between $\aEta$ and $\aEtap$. Due to Eq.~(\ref{Eq:CPhysFlavor}) this effect drops out in the sum up to the anomaly correction affecting the OPE regime. As TFF contribution to the uncertainty on $\Delta \aEtaEtap$ we took the absolute value of the differences between the results from the Canterbury and DSE TFFs.

Since NLO results are not available for the $\etaetap$ TFFs, we estimated the NLO OPE uncertainty by simply rescaling the one in the pion channel by the ratio of the reference outcomes. Due to the smallness of this uncertainty, this is expected to be sufficiently accurate.

For the range $[m_\text{min}, m_\text{max}]$ and the minimal allowed value $\Sigma_t$ for $\Sigma^\text{match}$ in the Monte Carlo simulation for $P(x,y)$, we took the results from the isovector channel. We rescaled the $\pi(1300)$ term below the matching surface by the ratio of reference results when adding this contribution to $\delta_m \aEtaEtap$. This is justified by the fact that the first excited pseudoscalars in the three flavor channels have similar masses, despite the large mass difference of the pseudo-Goldstone bosons.

All results are collected in Tab.~\ref{Tab:results} and our final estimate for the longitudinal short-distance effects in $\aEtaEtap$ reads
\begin{eqnarray}
\Delta \aEta & = & \num[separate-uncertainty]{2.6 \pm 1.5 e-11}\,, \nonumber \\
\Delta \aEtap & = & \num[separate-uncertainty]{3.9 \pm 2.0 e-11}\, .
\end{eqnarray}
The relative contributions to the uncertainties are similar to the pion case illustrated in Fig.~\ref{Fig:PieChartErrors}. A more precise description of $\eta-\eta'$-mixing would of course help better separate the two isoscalar channels but would not play an important role in their sum leading to negligible shifts in the total contribution from longitudinal SDCs.

%%%%%%%%%%%%%%%%%%%%%%%%%%%%%%%%%%%%%%%%%%%%%%%%%%%%%%%

\subsection{Sum over the flavor channels and comparison with literature}
\label{Sec:NumResults}
Combining the results from Secs.~\ref{Sec:NumIV} and~\ref{Sec:NumIS}, obtained under the assumption that the ground-state pseudoscalar mesons dominate the low-energy region, our estimate for the total effect of the longitudinal SDCs on HLbL amounts to
\begin{eqnarray}
\Delta \aLon & = &
\Delta \aPion + \Delta \aEta + \Delta \aEtap \nonumber \\
& = & \num[separate-uncertainty]{9.1\pm 5.0 e-11}\, ,
\label{Eq:resultSum}
\end{eqnarray}
where we have combined the three uncertainties linearly since they originate from the same sources in all three channels. 

This result is remarkably close to what is expected based on flavor symmetry considerations. If the $U(3)$ symmetry emerging in the combined chiral and large-$N_c$ limit is assumed, then $\Delta \aEta + \Delta \aEtap = 3 \Delta \aPion$. Using our isovector uncertainty and adding linearly a standard \SI{30}{\percent} $U(3)$ breaking effect, we obtain
\begin{equation}
\Delta \aLon = \num[separate-uncertainty]{10.4 \pm 8.3e-11}\,.
\end{equation}
For this reason we do not expect that a more refined analysis of the subtler isosinglet contributions is going to change substantially our final results.

Refs.~\cite{BernSDCShort,BernSDCLong} have recently studied the possibility of saturating SDCs away from the chiral limit by including a tower of excited pseudoscalar states in the context of a Regge model matched to the pQCD quark loop. Their outcome is $\Delta \aLon = \num{13 \pm 6 e-11}$, which is well compatible with ours within errors. For the $\eta'$-channel, the Regge model yields $\Delta \aEtap = \num[separate-uncertainty]{6.5\pm 2.0 e-11}$, which is somewhat larger than our result but still compatible within errors.\footnote{The quoted result does not include the matching to the pQCD quark loop, which has only been performed for the sum of all channels in Refs.~\cite{BernSDCShort,BernSDCLong}.} This can partly be explained by the different value for $C_{\eta'}$ used in Refs.~\cite{BernSDCShort,BernSDCLong}, namely $C_{\eta'} = 0.239$, which results from imposing that Eq.~(\ref{Eq:CPhysFlavor}) holds exactly. 

%%%%%%%%%%
\begin{figure*}
    \centering
    \includegraphics{figs/CompBern}
    \caption{$\Delta \aPion$ as a function of a lower limit on $Q_3^2$ in Eq.~(\ref{Eq:MasterFPi1}): our reference result and corresponding error band against the tower of excited pseudoscalars in the large-$N_c$ Regge model 1 of Refs.~\cite{BernSDCShort,BernSDCLong} and the curve from the MV model~\cite{MV} evaluated using the up-to-date dispersive pion TFF. At small non-vanishing $Q_{3,\text{min}}^2$, our reference curve is constant due to the finite $\Sigma^\text{match}$, which for $P(x,y)=0$ corresponds to $m^2 = Q_3^2 = \text{const.}$, whereas the Regge model has a slope due to the absence of such a cutoff. The upper end of our error band shows a slope because of the inclusion of the $\pi(1300)$ contribution in that region.}
    \label{Fig:Q32MinPion}
\end{figure*}
%%%%%%%%%%

Fig.~\ref{Fig:Q32MinPion} shows $\Delta \aPion$ as a function of a lower cutoff on $Q_3^2$ in our approach as well as the large-$N_c$ Regge model 1 of Refs.~\cite{BernSDCShort,BernSDCLong}. In order to obtain this plot, we calculated the integral in Eq.~(\ref{Eq:MasterFPi1}) as a function of a lower limit on $Q_3^2$ (which depends on $\Sigma$, $r$ and $\phi$) for both the full $\aPion$ as well as the pion pole contribution (cf.\ Eq.~(\ref{Eq:refRes})). The Regge model result lies within our error band for all $Q_{3,\text{min}}^2$.

Our estimate of longitudinal short-distance effects as well as the one in Refs.~\cite{BernSDCShort,BernSDCLong} are smaller than the shift obtained in Ref.~\cite{MV}, $\Delta \aLon = \num{23.5 e-11}$, which even increases to about \num{38e-11} if up-to-date TFF input is used~\cite{BernSDCLong}. These large values are due to two features of the MV model: the fact that the singly-virtual TFF is set to a constant over the whole integration region and not only in the OPE regime, and the fact that in the symmetric asymptotic limit the parametric momentum dependence is correct but its coefficient is too large. Both of these features can be clearly seen in Fig.~\ref{Fig:Q32MinPion} and are responsible for the discrepancies in the slope at small $Q_{3,\text{min}}^2$ and the values at large $Q_{3,\text{min}}^2$, respectively.

Refs.~\cite{HolographyVienna, HolographyItaly} have studied how the inclusion of an infinite tower of axial-vector mesons could help satisfy the OPE SDCs, focusing for this purpose on the relevant TFFs in the context of holographic QCD models. According to Ref.~\cite{HolographyVienna}, the tower of axial-vector mesons contributes \SIrange[range-phrase = --,scientific-notation = fixed, range-units = brackets, fixed-exponent = -11]{29e-11}{41e-11}{\noop} to $a_\mu^\text{HLbL}$ of which \SIrange[range-phrase = --, range-units = brackets]{57}{58}{\percent} are attributed to $\aLon$. Using instead holographic QCD input only for the momentum dependence of the TFF and fixing its normalization from experiment reduces the estimate of the contribution to $a_\mu^\text{HLbL}$ from the tower of axials to \num{22\pm5e-11}. Ref.~\cite{HolographyItaly} finds \num{14e-11} for the effect of axials on $\aLon$. Thus, the results of these studies appear to be at the high end of our range in Eq.~(\ref{Eq:resultSum}). However, we stress that comparing these numbers with our result is not properly justified. Indeed, while in these models the parametric $\Sigma$-dependences implied by pQCD and the OPE in the respective limits are correctly reproduced, the coefficients thereof are typically too small. In addition, the lightest multiplet of axials significantly alters $\bar{\Pi}_1$ at small photon virtualities, which implies that in our approach its contribution should be included in the low-energy representation. This aspect will be discussed in the next section, also to show how information on additional states in the \SI{1}{\GeV} region can be incorporated in our analysis.

%%%%%%%%%%%%%%%%%%%%%%%%%%%%%%%%%%%%%%%%%%%%%%%%%%%%%%%

\subsection{Including ground-state axial mesons at low energies}
\label{Sec:NumAxials}
Here we adopt a model-dependent approach to illustrate the application of our procedure to the case of the inclusion in the low-energy region of the lightest of the axial-vector mesons, for which no dispersive treatment in the BTT formalism is available yet. According to the holographic QCD models in Refs.~\cite{HolographyVienna, HolographyItaly} and using the notation of Ref.~\cite{HolographyVienna}, the contribution to $\bar{\Pi}_1$ of an axial meson of mass $M_A$ in the flavor channel $a$ can be written as
\begin{eqnarray}
\PiPS[\text{axial}] & = & -\frac{9 \,C_a^2}{16\pi^4 M_A^2}\left[Q_1^2 \,A(Q_1^2, Q_2^2) + Q_2^2 \,A(Q_2^2, Q_1^2)\right]\nonumber\\
&&\times A(Q_3^2,0)\,,
\label{Eq:Axial}
\end{eqnarray}
where $A(Q_1^2, Q_2^2)$ is the axial TFF.

Among the various scenarios analyzed in Ref.~\cite{HolographyVienna}, the hard-wall model by Hirn and Sanz (HW2)~\cite{Hirn:2005nr}, which was also studied in Ref.~\cite{HolographyItaly} with different parameters, reproduces best the measured mass, the measured equivalent two-photon decay width and the singly virtual momentum dependence measured by L3 for the lightest multiplet~\cite{Achard:2001uu, Achard:2007hm}. Furthermore, it yields asymptotic axial TFFs whose momentum dependence is consistent with the behavior derived in Ref.~\cite{Hoferichter:2020lap}. The infinite tower of axials has the correct momentum scaling in the asymmetric asymptotic regime dictated by the OPE constraint, but the coefficient is \SI{38}{\percent} too small~\cite{HolographyVienna}. 

We focused on the isovector channel, which is sufficient for our illustrative purposes, and thus on the inclusion of the $a_1$ meson. Based on the HW2 model, we obtained $a_{\mu,\text{HW2}}^{a_1} = \num{3.3e-11}$ for the $a_1$ contribution to $\aLon$. The rest of the tower of isovector axial mesons in this model yields $\Delta \aPion[A,\ \text{HW2}] = \num{0.8e-11}$, implying that in this framework about \SI{80}{\percent} of the total effect comes from the lightest state.\footnote{We thank Josef Leutgeb for checking these numbers and the ones for HW2(UV-fit) below.}

By matching the interpolant in Eq.~(\ref{Eq:interpolant12}) to the contributions from the pion and the holographic $a_1$ with the reference set of assumptions in Sec.~\ref{Sec:NumIV}, we obtained\footnote{We had to shift $C_\pi^2$ by \SI{1.7}{\percent} in order to account for the $a_1$ at small $Q_3^2$ and asymptotic $Q_1^2\sim Q_2^2$.}
\begin{equation}
\Delta \aPion[A] = \aPion[A] - a_{\mu,\text{disp}}^\text{$\pi^0$-pole} - a_{\mu,\text{HW2}}^{a_1} =\num{1.9e-11}\,.
\label{Eq:resAxialHW2}
\end{equation}
This result is more than twice as large as the resummed tower in HW2, $\Delta \aPion[A,\ \text{HW2}]$, but the significance of this discrepancy could only be assessed by a more sophisticated analysis including uncertainties, which is beyond the scope of this work. However, in the holographic model the infinite tower of axials does not fully saturate the pQCD nor the OPE constraints, which suggests that additional degrees of freedom besides axials should be included in a more realistic model.

We then considered the choice of parameters made in Ref.~\cite{HolographyItaly} and referred to as HW2(UV-fit) in Ref.~\cite{HolographyVienna}. This model is constructed to obey the OPE constraint exactly, but fails to describe low-energy physics like the $\rho$-meson mass, the pion TFF and the axial TFFs measured by L3. The longitudinal contribution from $a_1$ in this case amounts to $a_{\mu,\ \text{HW2(UV-fit)}}^{a_1} = \num{3.4e-11}$ and the tower of states increases the value by $\Delta \aPion[A, \text{HW2(UV-fit)}] = \num{0.8e-11}$. Our reference interpolant leads to $\Delta \aPion[A] = \num{1.4e-11}$, which is again larger than the model result. However, also in HW2(UV-fit) the pQCD constraint is not fully fulfilled by the tower of axials.

Neglecting issues related to intrinsic model dependence in the low-energy input, our method based on interpolants that by construction satisfy all constraints indicates that the effects of longitudinal SDCs are relatively small compared to the dominant low-energy contributions, and what is crucial in order to achieve higher precision is to gain control over the latter. We stress that a reliable prediction with a robust uncertainty estimate of the effects of axial meson exchanges would require model-independent input information.