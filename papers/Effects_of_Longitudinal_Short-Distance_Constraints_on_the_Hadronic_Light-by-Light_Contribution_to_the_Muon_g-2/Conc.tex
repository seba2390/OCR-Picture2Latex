\section{Conclusions}
\label{Sec:Con}
In this paper we have introduced a novel approach to incorporate longitudinal SDCs into the calculation of the HLbL contribution to the muon $g-2$. At variance with the previous estimates based on hadronic models, we have constructed general functions interpolating between low-, mixed- and high-energy regions, without resorting to specify which and how hadronic intermediate states are responsible for saturating the constraints. Furthermore, our method allows us also to study in detail the role played by parameters and assumptions in a transparent and numerically efficient way.

Our main premise is that an accurate low-energy representation of the longitudinal function $\bar{\Pi}_1(Q_1^2, Q_2^2, Q_3^2)$ entering the HLbL integral can be obtained by taking into account only intermediate states that are under good theoretical and numerical control. For the $\pi^0$, due to the location of its pole, the form of this low-energy representation can be straightforwardly extended even to large $Q_1^2$ and $Q_2^2$ as long as $Q_3^2$ stays small. Using available input for the $\pi^0$-pole term, we find that the shift due to longitudinal SDCs on the isovector part of $a_\mu^\text{HLbL}$ is in the range \num[separate-uncertainty]{2.6\pm1.5e-11}. By including in the analysis also the isoscalar components, which the $\eta$- and $\eta'$-poles are assumed to dominate at low energies, we obtained that longitudinal SDCs increase $a_\mu^\text{HLbL}$ by \num[separate-uncertainty]{9.1\pm5.0e-11} in total. 
The quoted ranges encompass uncertainties in the low-energy input, perturbative corrections and fitting errors at asymptotic momenta, parametric variations of the functional form of the interpolants and of the matching surface, at which these functions are matched to the low-energy input, with the latter dominating the total uncertainty. Thus, according to our analysis, infinite towers of states heavier than \SI{1}{\GeV}, albeit crucial for the saturation of SDCs, give a relatively small contribution to $a_\mu^\text{HLbL}$ and this effect can be estimated with sufficient precision using our method. Conversely, states with masses around \SI{1}{\GeV} contributing significantly to the low-energy region play a decisive role also in a precision determination of short-distance effects.

Our result for the effects of longitudinal SDCs on $a_\mu^\text{HLbL}$ agrees with recent model estimates~\cite{BernSDCShort,BernSDCLong}, fulfills the accuracy goal set by the forthcoming experimental results and is significantly smaller than the earlier model result of Ref.~\cite{MV}, especially when up-to-date TFF input is used. Furthermore, neglecting issues concerning intrinsic model dependence and the fact that holographic QCD calculations in Refs.~\cite{HolographyVienna, HolographyItaly} do not completely saturate the SDCs, we find in agreement with these studies that the infinite tower of axials has a relatively small impact on the longitudinal part of $a_\mu^\text{HLbL}$ if the lightest multiplet is treated explicitly as a low-energy contribution.

It will be straightforward to incorporate in our approach model-independent information on further intermediate states as well as higher-order corrections to asymptotic expressions once these become available. Furthermore, our method can be generalized to the case of transversal SDCs. Therefore, it paves the way for a combination of all available low- and high-energy information on HLbL into one model-independent, accurate numerical estimate of this contribution to the muon $g-2$.