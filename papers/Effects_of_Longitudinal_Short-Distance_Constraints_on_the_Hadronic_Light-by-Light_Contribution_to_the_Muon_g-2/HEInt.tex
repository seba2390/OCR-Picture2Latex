\section{Interpolating between asymptotic constraints}
\label{Sec:HEInt}

We approximate the true $\bar{\Pi}_1(\Sigma, r, \phi)$ following a two-step procedure. We first select functional forms that are valid for asymptotic $\Sigma$ and are compatible with the constraints discussed in the previous section. We then interpolate between this set of functions and various representations of $\bar{\Pi}_1$ at small $\Sigma$ determined by single-particle intermediate states. Here we work at leading order in $\alpha_s$. Perturbative corrections will be discussed in our numerical analysis in Sec.~\ref{Sec:Num}.

The relevant constraints on $\bar{\Pi}_1$ at large $\Sigma$ are given by Eq.~(\ref{Eq:OPEConstraintMasses}) for $Q_1^2 = Q_2^2 \gg Q_3^2$ and Eq.~(\ref{Eq:pQCDConstraint}) for $Q_1^2 = Q_2^2 = Q_3^2$. Both expressions as well as the vanishing result of the leading-order OPE contribution in the crossed kinematics are compatible with
\begin{eqnarray}
\PiPS[\text{asymp}'] & = & -\frac{4 N_c C_\text{PS}^2}{\pi^2(Q_3^2 + m_\text{PS}^2)(Q_1^2 + Q_2^2 + Q_3^2)} \nonumber \\
& = & -\frac{12 N_c C_\text{PS}^2}{\pi^2 \Sigma (3m_\text{PS}^2 + \Sigma + \Sigma r \cos\phi)}
\label{Eq:AsymPiPrime}
\end{eqnarray}
if $ C_\text{PS} = C_\text{PS}^\text{pQCD} $. Thus, Eq.~(\ref{Eq:AsymPiPrime}) interpolates between symmetric and asymmetric asymptotic limits. According to Sec.~\ref{Sec:pQCD}, $\delta_0$ parameterizes the anomaly corrections to the singlet VVA correlator and the resulting shift in $C_\text{PS}^\text{pQCD}$ with respect to $C_\text{PS}$. Since a term proportional to $\Sigma^{-2}$ and independent of $(r, \phi)$ does not change the leading behavior at $Q_3^2 = 0$ and thus does not spoil compatibility with the OPE constraint, we subtract $36 \delta_0 C_\etaetap^2/(\pi^2 \Sigma^2)$ from Eq.~(\ref{Eq:AsymPiPrime}) in the case of $\eta$/$\eta'$.

Obviously, the choice made in Eq.~(\ref{Eq:AsymPiPrime}) and the exact form of the singlet correction are not unique and we are free to add a generic function such that the interpolant still satisfies the constraints. In order to have a non-negligible effect at asymptotic values of $\Sigma$, this additional function should also scale as $\Sigma^{-2}$ and we demand it to be finite and analytic for all $r\le 1$.\footnote{$\bar{\Pi}_1$ cannot decay more slowly than $\Sigma^{-2}$ for any $(r,\phi)$ region in order for the $a_\mu$ integral in Eq.~(\ref{Eq:MasterFPi1}) to be finite.} Therefore it can be approximated by a Taylor series in $r\cos\phi$ and $r \sin\phi$ truncated after order $M$,
\begin{eqnarray}
\Pipion[\text{asymp}] & = & \Pipion[\text{asymp}'] +\frac{12 N_c C_\pi^2}{\pi^2 \Sigma^2} \sum_{i=0}^{M} \sum_{j=0}^{M} \frac{1}{i!\ j!} \nonumber\\
&&\times a_{i,j} (r\cos\phi)^i (r\sin\phi)^j\,, \nonumber \\
\Pietaetap[\text{asymp}] & = & \Pietaetap[\text{asymp}'] - \frac{36 \delta_0 C_\etaetap^2}{\pi^2 \Sigma^2} \nonumber \\
&& + \frac{12 N_c C_\etaetap^2}{\pi^2 \Sigma^2} \sum_{i=0}^{M} \sum_{j=0}^{M} \frac{1}{i!\ j!} \nonumber \\
&&\times a_{i,j} (r\cos\phi)^i (r\sin\phi)^j\, ,
\label{Eq:AsymPi}
\end{eqnarray}
where
\begin{equation}
a_{0,0} = 0\,, \quad a_{i,2j+1} = 0
\label{Eq:Conaij}
\end{equation}
for integer $j$, due to the pQCD constraint and crossing symmetry. 

Up to now, we have applied the quark-loop result only at $r = 0$. However, the fact that Eq.~(\ref{Eq:pQCDIntegral}) holds also in a neighborhood of this point can be used to fix the coefficients $a_{i,j}$. To this end, we fitted Eq.~(\ref{Eq:AsymPi}) at fixed asymptotic $\Sigma$ with $M = 2$ to Eq.~(\ref{Eq:pQCDIntegral}) for $r<0.9$.\footnote{Since the maximal ratio of two squared momenta for $r=0.9$ is \num{14.7} and $\ln(14.7) \approx 2.7$, large logarithms do not occur in this region.} We chose a grid of equally separated points in this fitting region and minimized the sum of the relative squared differences between our interpolant and the leading-order quark-loop expression. The resulting 5 dimensionless fit parameters in the pion channel read
\begin{eqnarray}
a_{1,0} & = & \num{-0.170} \,,\quad a_{2,0} = \num{0.094} \,,\quad a_{0,2} = \num{-0.554}\,,\nonumber\\
\quad a_{1,2} & = & \num{-0.169} \,,\quad a_{2,2} = \num{-0.756}
\label{Eq:AsymPi_a_pion}
\end{eqnarray}
and are all at most $\Order(1)$, as expected since in Eq.~(\ref{Eq:AsymPi}) they parameterize relative corrections. This holds true also for the $\etaetap$ channels, where the numerical values are different. In Sec.~\ref{Sec:AsymUnc} we will discuss uncertainties due to the chosen fitting range, the number of parameters in the fit and $\alpha_s$ corrections to the asymptotic constraints.