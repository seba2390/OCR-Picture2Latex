% ****** Start of file apssamp.tex ******
%
%   This file is part of the APS files in the REVTeX 4.2 distribution.
%   Version 4.2a of REVTeX, December 2014
%
%   Copyright (c) 2014 The American Physical Society.
%
%   See the REVTeX 4 README file for restrictions and more information.
%
% TeX'ing this file requires that you have AMS-LaTeX 2.0 installed
% as well as the rest of the prerequisites for REVTeX 4.2
%
% See the REVTeX 4 README file
% It also requires running BibTeX. The commands are as follows:
%
%  1)  latex apssamp.tex
%  2)  bibtex apssamp
%  3)  latex apssamp.tex
%  4)  latex apssamp.tex
%
\documentclass[%
 reprint,
superscriptaddress,
%groupedaddress,
%unsortedaddress,
%runinaddress,
%frontmatterverbose, 
%preprint,
%preprintnumbers,
%nofootinbib,
%nobibnotes,
%bibnotes,
 amsmath,amssymb,
 aps,
%prl,
pra,
%prb,
%rmp,
%prstab,
%prstper,
%floatfix,
longbibliography
]{revtex4-2}
\usepackage{xcolor}
\usepackage[normalem]{ulem}
\usepackage{graphicx}% Include figure files
\usepackage{dcolumn}% Align table columns on decimal point
\usepackage{bm}% bold math
\usepackage{hyperref}% add hypertext capabilities
\hypersetup{
    colorlinks=true
}
%\renewcommand{\theequation}{S\arabic{equation}} 
%\renewcommand{\thefigure}{S\arabic{figure}} 

%\usepackage[mathlines]{lineno}% Enable numbering of text and display math
%\linenumbers\relax % Commence numbering lines

%\usepackage[showframe,%Uncomment any one of the following lines to test 
%%scale=0.7, marginratio={1:1, 2:3}, ignoreall,% default settings
%%text={7in,10in},centering,
%%margin=1.5in,
%%total={6.5in,8.75in}, top=1.2in, left=0.9in, includefoot,
%%height=10in,a5paper,hmargin={3cm,0.8in},
%]{geometry}

\begin{document}

\preprint{APS/123-QED}

\title{Phonon condensation and cooling
%Condensate of vibration energy
%in multimode mechanical systems 
via nonlinear feedback}% Force line breaks with \\

\author{Xu Zheng}
\email{Xu.Zheng@Colorado.Edu}
\affiliation{Department of Physics, University of Colorado, Boulder, CO, 80309, USA}
\author{Baowen Li}
\email{Baowen.Li@Colorado.Edu}
\affiliation{Paul M. Rady Department of Mechanical Engineering, University of Colorado, Boulder, CO, 80309, USA}
\affiliation{Department of Physics, University of Colorado, Boulder, CO, 80309, USA}
%\homepage{https://www.colorado.edu/faculty/li-baowen/}
\date{\today}% It is always \today, today,
             %  but any date may be explicitly specified

\begin{abstract}
%The scheme we considered is the same as the standard feedback cooling except the feedback we applied %is not proportional to the velocity of the system but a nonlinear functional of the measured position %and velocity. 
%Analytical and numerical results illustrate 
We show that in multimode mechanical systems, the amplification of the lowest mode and the damping of all the other modes can be realized simultaneously via nonlinear feedback. The feedback-induced dynamics of the multimode system is related to the formation of phonon condensation. The phonon statistics of the lowest mode are similar to those of
%between the feedback-induced state and 
a phonon laser. Finally, we show the coherence of the lowest mode can be improved by an order of magnitude.
\end{abstract}
\pacs{}

\maketitle


%\tableofcontents

\emph{Introduction}. -- 
Manipulation and control of both coherent and incoherent phonons and/or vibration energy is of great interest for both engineering applications and fundamental research. On the one hand, the coherent phonons and/or vibrations shows great potential in applications ranging from conventional nondestructive testing \cite{dekorsy2000coherent,ruello2015physical}, high resolution imaging and sensing \cite{poyser2015coherent}, to quantum information transfer and storage, etc \cite{ruskov2012coherent,gustafsson2014propagating,bienfait2019phonon}.  On the other hand, the control of incoherent phonons and/or vibrations is of primary importance in noise reduction \cite{liu2020review}, thermoelectric energy conversion \cite{takabatake2014phonon}, heat management such as heat dissipation and heat insulation \cite{li2012colloquium,li2021transforming}. 

%Recent years have witnessed an increasing attention on the manipulation of phonons in micromechanical and nanomechanical resonators which have been widely used 
In applications like highly sensitive and ultra-fast sensing \cite{rugar2004single,yang2006zeptogram,burg2007weighing}, acoustic actuation \cite{masmanidis2007multifunctional,feng2008self,wen2020coherent}, information processing \cite{mahboob2008bit,unterreithmeier2009universal,tadokoro2018driven}, biological imaging \cite{tamayo2001high,shekhawat2005nanoscale,tetard2008imaging},
%In these applications, 
amplifying vibration amplitude and narrowing phonon linewidth are critical for good performance. Active linear feedback control, namely, feedback force is proportional to the measured mechanical displacement or velocity but with a phase difference,  is a well-known technique for achieving these two goals. The control consists of the real-time monitoring of mechanical motion and feedback loop.
Depending on the phase difference, either positive or negative feedback can be realized. This technique works well when the resonator can be considered as a single mode system. 

In general, however, mechanical resonators have a series of normal modes. A single linear feedback loop results in the simultaneous amplification or damping of multiple modes \cite{ohta2017feedback,sommer2019partial}. In the cases like energy transfer/harvesting and phonon lasing, where only one selected mode is interested, it is then quite straight forward to ask if amplifying only one mode while cooling all the other modes can be achieved via a single feedback loop. 

In this Letter,  we propose to use a single nonlinear feedback loop
%whose scheme is the same as the active linear feedback loop except that the feedback force is a nonlinear function of mechanical motion, %to Our main result is
to realize the amplification of the lowest normal mode and the damping of all the other modes simultaneously in multimode mechanical systems. In the feedback-induced steady state, the phonon statistics indicate strong amplitude coherence in the lowest mode. The phase coherence of the lowest mode is largely improved as well.

The amplification of the lowest modes and damping of all the other modes is closely related to the well-known Fr\"{o}hlich condensate \cite{frohlich1968bose,frohlich1968long,frohlich1970long,wu1977bose,wu1978cooperative,wu1981frohlich,reimers2009weak,preto2017semi,zhang2019quantum,zheng2021froh}, where the vibration energy of a
collection of oscillators would condensate in the lowest mode once the external energy supply exceeds a threshold. In Fr\"{o}hlich condensate, the essential process is the energy redistribution among vibration modes induced by nonlinear couplings, whereas in our case it is through the nonlinear feedback.
\begin{figure}[b]
    \centering
    \includegraphics[width=0.7\linewidth]{scheme.pdf}
    \caption{Sketch of the multimode system considered. The reflected optical field provides information on the collective displacement of the resonator. Based on the detected signal, the feedback loop determines the drive applied onto the resonator. The feedback force can be realized using optomechanical, photothermal or electromechanical effect.}
    \label{fig:config}
\end{figure}

\emph{Model}. -- We consider a mechanical resonator with several modes of oscillations.  The configuration is shown in Fig. (\ref{fig:config}). The reflected and reference optical field from probe laser are collected by an interferometer to measure the collective displacement of the mechanical resonator. The measured displacement is then fed through a feedback loop to determine the drive applied onto the resonator. The feedback force can be provided by another laser via the optomechanical or photothermal effect, or by an electrical signal via the electromechanical effect. 

%For simplicity, the mechanical resonator is simplified as a mechanical beam by using continuum elasticity theory. The set of normal modes for the flexural vibration is characterized by the frequencies
%\begin{align}
%    \omega_j=\frac{c_j}{L^2}\sqrt{\frac{EI_z}{\rho S}},
%    \label{eq:normalfrequency}
%\end{align}
%where $L$ is the length, $E$ is the Young's modulus, $\rho$ is the mass density, $S$ is the cross section area, and $c_j$'s are obtained by solving the equation $\cos{\sqrt{c_j}}\cosh{\sqrt{c_j}}=1$. 

The equations of motion (EOM) of the resonator are given by
\begin{align}
    \dot{Q}_j&=\omega_jP_j,\nonumber\\
    \dot{P}_j&=-\omega_jQ_j-\gamma_jP_j+\xi_j+H_{\text{fb}}^{(j)},
    \label{eq:eoms}
\end{align}
where $\omega_j$ is the frequency, $Q_j=\sqrt{\frac{m}{k_BT}}\omega_jq_j$ and $P_j=\frac{p_j}{\sqrt{mk_BT}}$ are the dimensionless displacement and momentum of the $j$th normal mode so that the dimensionless vibration energy $\frac{1}{2}(Q_j^2+P_j^2)$ at thermal equilibrium is equal to one, $\gamma_j$ is the damping rate, $\xi_j$ is the thermal noise and $H_{\text{fb}}^{(j)}$ is the feedback force acted on the $j$th mode. At high temperature limit $k_BT\gg\hbar\omega_j$, the thermal noise satisfies $\langle\xi_j(t)\xi_j(t^{\prime})\rangle=2\gamma_j\delta(t-t^{\prime})$. The feedback force $H_{\text{fb}}^{(j)}$ is determined by the measured collective displacement $Q=\sum_jQ_j$ of the resonator. To realize the condensation of phonons at the lowest mode, we design the feedback loop as follows:
\begin{align}
    F_I&=\int_{0}^tQ(s)ds,\nonumber\\
    F_D&=\dot{Q},\nonumber\\
    H_{\text{fb}}^{(j)}&=-g_j\omega_{\text{fb}}\tanh{[\omega_{\text{fb}}(F_I^2F_D+3Q^2F_I)]},
    \label{eq:feedbackloop}
\end{align}
where $g_j$ is the feedback gain term and $\omega_{\text{fb}}$ is a reference frequency to make $H_{\text{fb}}^{(j)}$ have correct unit. The terms $F_I$ and $F_D$ are like the integral and derivative terms in the proportional–integral–derivative (PID) controller, which can be realized by the integrator and differentiator circuits, respectively. A hyperbolic tangent function is introduced to $H_{\text{fb}}^{(j)}$ to limit the strength of the feedback force between $\pm g_j\omega_{\text{fb}}$. To understand how the feedback works, we first consider a simpler $H_{\text{fb},0}^{(j)}$ where the hyperbolic tangent function is replaced by the identity function, i.e., $H_{\text{fb},0}^{(j)}=-g_j\omega^2_{\text{fb}}(F_I^2F_D+3Q^2F_I)$. After introducing the complex amplitude
\begin{align}
    Q_j&=\frac{1}{2}\left[a_j(t)e^{-i\omega_jt}+a^{\ast}_j(t)e^{i\omega_jt}\right]
    \label{eq:complexamp}
\end{align}
with $a_j(t)$ being slowly varying amplitudes ($\dot{a}_j\ll\omega_ja_j$), we can simplify the amplitude equations as
\begin{align}
    \dot{a}_{j}=-\frac{\gamma_j}{2}a_j+\sum_i\frac{g_j\omega_{\text{fb}}^2}{4\omega_i^2\omega_j}(\omega_i^2-\omega_j^2)|a_i|^2a_j+\Xi_j.
    \label{eq:amplitudeequations}
\end{align}
In the derivation we have assumed $\omega_j\gg\gamma_j$, ignored off-resonant terms and averaged the thermal noise $\xi_j(t)$ over the fast dynamics,
\begin{align}
    \Xi_j(t)=\frac{\omega_j}{2\pi}\int_{t-\pi/\omega_j}^{t+\pi/\omega_j}ds\xi_j(s)e^{i\omega_js}.
\end{align}
The slowly varying noise $\Xi_j(t)$ satisfies
\begin{align}
    \langle\Xi_j(t)\Xi^{\ast}_j(t^{\prime})\rangle=2\gamma_j\delta(t-t^{\prime}).
    \label{eq:correlation}
\end{align}

\emph{Phonon condensation in the lowest mode}. -- From the amplitude equations (\ref{eq:amplitudeequations}), we can define the effective damping rate 
\begin{align}
    \tilde{\gamma}_j=\gamma_j+\sum_i\frac{g_j\omega_{\text{fb}}^2}{2\omega_i^2\omega_j}(\omega_j^2-\omega_i^2)|a_i|^2,
    \label{eq:effectiverate}
\end{align}
where the second term is induced by the nonlinear feedback and its sign depends on the frequencies of the modes. To gain an insight into the dynamics of the system, we first consider the case of $N=2$ modes. In this case, the second term is negative for the first mode [sgn$(\omega_1^2-\omega_2^2)$] and is positive for the second mode [sgn$(\omega_2^2-\omega_1^2)$], which means the damping rate of the first mode decreases ($\tilde{\gamma}_1<\gamma_1$) and that of the second mode increases ($\tilde{\gamma}_2>\gamma_2$). According to the approximate solution of the steady-state energy $E_{j,ss}=\langle|a_j|^2\rangle_{ss}/2\approx\gamma_j/\tilde{\gamma}_j$, we can expect the steady-state energy of the first (second) mode to be greater (less) than that of the thermal equilibrium case, i.e. $E_{1,ss}>1>E_{2,ss}$. For the case of $N>2$ modes, it is straightforward to see $\tilde{\gamma}_1<\gamma_1$ since the feedback-induced term is always negative ($\omega_1^2-\omega_{i>1}^2<0$). Although a complete analysis of the other modes cannot be done, we can formulate a self-consistent statement by assuming the steady-state energy of the lowest mode is much larger than those of other modes, i.e. the phonon condensation can be achieved. In this limit, the dominant term introduced by the feedback force for the $j$th mode is the term proportional to $(\omega_j^2-\omega_1^2)|a_1|^2$, which means the system with $N>2$ modes can be considered as a collection of systems with $N=2$ modes whose frequencies is $\omega_1$ and $\omega_j$. Following the analysis of $N=2$ modes, we then obtain $E_{1,ss}>1>E_{j,ss}$, which is consistent with the assumption above.

While the analysis above is based on the feedback $H_{\text{fb},0}^{(j)}$, we find the feedback $H_{\text{fb}}^{(j)}$ induces similar results, as shown by Fig. (\ref{fig:energy}). The numerical results are obtained from the integration of Eq. (\ref{eq:eoms}) using SciML package \cite{rackauckas2017differentialequations}. Figure (\ref{fig:energy}a) shows the time evolution of the vibration energy induced by the feedback $H_{\text{fb}}^{(j)}$ in a system with $N=4$ modes. It is clearly seen that the vibration energy in the lowest mode is dominant at large times. The lowest mode is amplified by a factor of 10, while the other three modes are cooled to 0.57, 0.5 and 0.48, respectively. The ratio of vibration energy in the lowest mode to the total vibration energy is greater than 85\% at steady state, as shown in Fig. (\ref{fig:energy}b).

\begin{figure}[t]
    \centering
    \includegraphics[width=0.8\linewidth]{energynew.pdf}
    \caption{(color online). Phonon condensation in a system with $N=4$ modes. (a) Vibration energy as a function of time. The dimensionless vibration energy at thermal equilibrium ($t=0$) is one. The inset is zoom-in. The lowest mode is amplified by a
factor of 10, while the other three modes are cooled to
0.57, 0.5 and 0.48, respectively. (b) The ratio of vibration energy in the lowest mode to the total vibration energy of the system as a function of time. In our simulation, the frequencies of each mode are chosen according to the continuum elasticity theory $\omega_j=kc_j$, where $c_j$'s are obtained by solving the equation $\cos{\sqrt{c_j}}\cosh{\sqrt{c_j}}=1$ and $k$ is a constant depending on the geometry and material of the resonator \cite{landau1986course}. The first four modes satisfy $\omega_2/\omega_1=2.75$, $\omega_3/\omega_1=5.13$, $\omega_4/\omega_1=8.75$. The other parameters are $\gamma_j/\omega_j=10^{-2}$, $g_j=g=0.2$, $\omega_{\text{fb}}/\omega_1=1$.}
    \label{fig:energy}
\end{figure}

\emph{Phonon statistics and coherence of the lowest mode}. -- So far, we have only discussed the vibration energy in each mode and shown the feedback can give rise to the phonon condensation in the lowest mode. To gain more information about the feedback-induced steady state, we investigate the phonon statistics of the lowest mode. Figure (\ref{fig:phononstatistics}a)-(\ref{fig:phononstatistics}b) shows the phase portrait of the lowest mode without and with feedback. Without feedback, the phase portrait is peaked at the origin, which is consistent with the result of a thermal Brownian motion. With feedback, however, the phase portrait has a ring shape, which is similar to the phase portrait of a phonon laser \cite{pettit2019optical,wen2020coherent} and indicates the existence of amplitude coherence. 

To give a further comparison of the phonon statistics without and with feedback, we show the phonon-number (energy) distribution of the lowest mode in Fig. (\ref{fig:phononstatistics}c)-(\ref{fig:phononstatistics}d). Without feedback, the phonon distribution follows the exponential distribution of thermal Boltzmann statistics. With feedback, the phonon distribution shifts from Boltzmann statistics to super-Poissonian statistics, where the most probable phonon number (energy) is non-zero. For an ideal coherent state, the variance of the phonon distribution is equal to the mean. The variance ($\sim19$) we obtained from Fig. (\ref{fig:phononstatistics}d) is greater than the mean ($\sim10$), but it is much smaller than the variance ($\sim100$) of a thermal state with the same mean phonon number (energy), indicating the existence of strong thermal noise squeezing. The second order correlation function $g^{(2)}(0)$ obtained from the phonon distribution is $g^{(2)}(0)=\langle E_1^2\rangle/\langle E_1\rangle^2=1.17$, which is close to the $g^{(2)}(0)$ value of the coherent state.


\begin{figure}[t]
    \centering
    \includegraphics[width=\linewidth]{phononstatistics.pdf}
    \caption{(color online). Phonon statistics of the lowest mode at steady state. (a) Phase portrait of the lowest mode without feedback. (b) Phase portrait with feedback. (c) Phonon-number (energy) distribution without feedback. (d) Phonon-number (energy) distribution with feedback. The parameters used are the same as those in Fig. (\ref{fig:energy}).
    The variance ($\sim19$) in (d) is much smaller than that  ($\sim100$) in (c) of a thermal state with the same mean phonon number (energy).
    }
    \label{fig:phononstatistics}
\end{figure}


By plotting the phase portrait, we show the amplitude coherence of the lowest mode in the feedback-induced steady state. We are also interested in the phase coherence, which can be determined by the linewidth of the noise power spectral density $S_{Q_1Q_1}(\omega)$. The spectral density is obtained by the Fourier transform of autocorrelation function, i.e.,
\begin{align}
    S_{Q_1Q_1}(\omega)=\int_{-\infty}^{+\infty}d\tau\langle \overline{Q_1(t)Q_1(t+\tau)}\rangle e^{i\omega\tau},
\end{align}
where the overline denotes time average over $t$ and the angle brackets denote ensemble average. Fig. (\ref{fig:correlation}) shows the spectral density $S_{Q_1Q_1}(\omega)$ without and with feedback. Without feedback, the intrinsic linewidth is $\gamma_1/\omega_1=1\times10^{-2}$, and the coherence time is given by $\omega_1\tau_{coh}=2\omega_1/\gamma_1=200$. With feedback, the linewidth is $\tilde{\gamma}_1/\omega_1=7\times10^{-4}$. The corresponding coherence time is $\omega_1\tilde{\tau}_{coh}=2857$, which is an order of magnitude longer than the intrinsic coherence time.


\begin{figure}[t]
    \centering
    \includegraphics[width=0.8\linewidth]{spectral.pdf}
        \caption{(color online). Noise power spectral density of the lowest mode. For clear comparison, the spectral density has been rescaled so that the maximum value is 1. The blue solid line is the spectral density with feedback, and the red dashed line is the spectral density without feedback. With feedback, the linewidth is $\tilde{\gamma}_1/\omega_1=7\times10^{-4}$. Without feedback, the intrinsic linewidth is $\gamma_1/\omega_1=1\times10^{-2}$. The parameters used are the same as those in Fig. (\ref{fig:energy}).}
    \label{fig:correlation}
\end{figure}

\emph{Discussions and conclusions}. -- The energy evolution in Fig. (\ref{fig:energy}) is similar to the phonon number evolution in the formation of Fr\"{o}hlich condensate \cite{preto2017semi,zheng2021froh}, illustrating the close connection between our model and Fr\"{o}hlich condensate. This connection can be further seen from the similar formulas of the amplitude equations (\ref{eq:amplitudeequations}) and the rate equations of phonon numbers in Fr\"{o}hlich's model (see Appendix \ref{appendix:comparison}). In Fr\"{o}hlich's model, there are third order terms in the Hamiltonian that couples the environment or auxiliary optical field with pairs of vibration modes \cite{wu1977bose,wu1978cooperative,wu1981frohlich,zheng2021froh}, inducing the energy redistribution among these modes. In our model, these interaction terms are replaced by the nonlinear feedback loop where a nonlinear functional of the collective motion $Q=\sum_jQ_j$ induces the interactions between different vibration modes. From this perspective, we provide a method to realize Fr\"{o}hlich-like 
phonon condensation even in linear systems. 

In summary, we have analyzed the prospects for using a feedback loop to realize the condensation of phonon or vibration energy in multimode mechanical systems. We have shown the proposed feedback decreases the effective damping rate of the lowest mode while increases the effective damping rate of other modes, which gives rise to the amplification of the lowest mode and the damping of all the other modes.  
For the phonon statistics and coherence of the lowest mode, the ring shape in the phase portrait, super-Poissonian phonon distribution and longer coherence time reveal intriguing similarities between the feedback-induced state and phonon laser. These features suggest the nonlinear feedback loop could be used in sensitive sensing and the design of novel monochromatic phonon laser where no two-level gain mediums are needed.

While we have used the continuum elasticity theory to model the mechanical resonator, the proposed feedback loop is applicable to general mechanical systems with incommensurable modes. The potential systems include nanoelectromechanical systems (NEMS), levitated nanoparticles in the optical tweezers, collective motions of cold atoms or ions in potential traps, etc. Further development from current model could replace the harmonic oscillators with more realistic nonlinear oscillators or self-sustained oscillators (e.g., Van der Pol oscillator). There are many interesting phenomena in coupled self-sustained oscillators, such as synchronization \cite{pikovsky2003synchronization} and mode competition \cite{jonsson2008self,kemiktarak2014mode,zhang2018mode}. Our proposed feedback would be used in the control of these phenomena. 

In our system, the essential part is the detailed form of the feedback loop, which determines the capability and efficiency of achieving phonon or energy condensation. While the proposed 
one works well, other feedback strategies might provide similar or even better results. Looking for better feedback strategies, especially with the help of the fast-developing machine learning models, is a promising direction \cite{sommer2020prospects}. 
Besides, the effect of a time delay and phase difference in the feedback loop is an interesting topic and deserves further study.
%is not discussed in this paper and will be the direction of future works.

\appendix

\section{Comparison of our model and Fr\"{o}hlich's model}
\label{appendix:comparison}
The rate equations of phonon numbers in Fr\"{o}hlich's model are given by
\begin{align}
    \dot{n}_j=&s-\gamma_j(n_j-\bar{n}_{j,\text{th}})\nonumber\\
    &+\chi\sum_i[(n_j+1)n_i-n_j(1+n_i)e^{\hbar(\omega_j-\omega_i)/k_BT}], \label{eq:frohlichoriginal}
\end{align}
where $s$ is the external pumping, $\chi$ is the coupling strength of two-phonon process, $\bar{n}_{j,\text{th}}$ is the thermal phonon number. In the limit of large phonon number $n_j\gg1$, the equations are simplified as
\begin{align}
    \dot{n}_j=&s-\gamma_j(n_j-\bar{n}_{j,\text{th}})+\chi\sum_i[1-e^{\hbar(\omega_j-\omega_i)/k_BT}]n_in_j. \label{eq:frohlichrate}
\end{align}
Recently, there is a proposal to realize Fr\"{o}hlich condensate in optomechanical systems \cite{zheng2021froh}. The modified rate equations of phonon numbers are given by
\begin{align}
    \dot{n}_j=&-\gamma_j(n_j-\bar{n}_{j,\text{th}})\nonumber\\
    &+\sum_{i\neq j}4U_{i,j}^2\left[\Gamma(\omega_i-\omega_j)-\Gamma(\omega_j-\omega_i)\right]n_in_j,
    \label{eq:optorate}
\end{align}
where $U_{i,j}$ is coefficient and $\Gamma(\omega)$ is a function of frequency. To compare the amplitude equations (\ref{eq:amplitudeequations}) with Eq. (\ref{eq:frohlichrate}) and (\ref{eq:optorate}), we need to convert Eq. (\ref{eq:amplitudeequations}) to the rate equations of $\langle |a_j(t)|^2\rangle$. The formal solution of Eq. (\ref{eq:amplitudeequations}) is given by
\begin{align}
    &a_j(t)=\nonumber\\
    &\int_{-\infty}^{t}dse^{-\frac{\gamma_j}{2}(t-s)+\sum_i\frac{g_j\omega_{\text{fb}}^2}{4\omega_i^2\omega_j}(\omega_i^2-\omega_j^2)\int_{s}^tdt^{\prime}|a_i(t^{\prime})|^2}\Xi_j(s)
    \label{eq:formalaj}
\end{align}
From Eq. (\ref{eq:formalaj}) we can get the formal solution of $\langle |a_j(t)|^2\rangle$,
\begin{align}
    &\langle |a_j(t)|^2\rangle\nonumber\\
    &\approx2\gamma_j\int_{-\infty}^{t}ds\langle e^{-\gamma_j(t-s)+\sum_i\frac{g_j\omega_{\text{fb}}^2}{4\omega_i^2\omega_j}(\omega_i^2-\omega_j^2)\int_{s}^tdt^{\prime}|a_i(t^{\prime})|^2}\rangle,
    \label{eq:formalaj2}
\end{align}
where we have used Eq. (\ref{eq:correlation}) and decorrelation approximation. Hence, the rate equations of $\langle |a_j(t)|^2\rangle$ are given by
\begin{align}
    \frac{d\langle |a_j|^2\rangle}{dt}\approx&-\gamma_j(\langle|a_j|^2\rangle-2)\nonumber\\
    &+\sum_i\frac{g_j\omega_{\text{fb}}^2}{4\omega_i^2\omega_j}(\omega_i^2-\omega_j^2)\langle|a_i|^2\rangle\langle|a_j|^2\rangle
    \label{eq:aj2equations}
\end{align}
under decorrelation approximation. In the high temperature limit, the phonon numbers are determined by $n_j=\langle |a_j|^2\rangle k_BT/(2\hbar\omega_j)$. From Eq. (\ref{eq:aj2equations}), the rate equations of phonon numbers in our model are then given by
\begin{align}
    \dot{n}_j\approx&-\gamma_j(n_j-\bar{n}_{j,\text{th}})+\sum_i\frac{\hbar g_j\omega_{\text{fb}}^2}{2k_BT\omega_i\omega_j}(\omega_i^2-\omega_j^2)n_i n_j
    \label{eq:rateequationourmodel}
\end{align}
Eq. (\ref{eq:rateequationourmodel}) has the same form as Eq. (\ref{eq:frohlichrate}) and (\ref{eq:optorate}) except the coupling function before $n_in_j$ is different and there is no external pumping when compared with Eq. (\ref{eq:frohlichrate}).

% The \nocite command causes all entries in a bibliography to be printed out
% whether or not they are actually referenced in the text. This is appropriate
% for the sample file to show the different styles of references, but authors
% most likely will not want to use it.
%\nocite{*}

%\bibliographystyle{apssamp}% Produces the bibliography via BibTeX.
%\bibliographystyle{apsrev4-2}
%apsrev4-2.bst 2019-01-14 (MD) hand-edited version of apsrev4-1.bst
%Control: key (0)
%Control: author (8) initials jnrlst
%Control: editor formatted (1) identically to author
%Control: production of article title (0) allowed
%Control: page (0) single
%Control: year (1) truncated
%Control: production of eprint (0) enabled
%\bibliographystyle{apsrev4-2}
%\bibliographystyle{abbrv}
%\bibliography{reference}
%apsrev4-2.bst 2019-01-14 (MD) hand-edited version of apsrev4-1.bst
%Control: key (0)
%Control: author (8) initials jnrlst
%Control: editor formatted (1) identically to author
%Control: production of article title (0) allowed
%Control: page (0) single
%Control: year (1) truncated
%Control: production of eprint (0) enabled
%apsrev4-2.bst 2019-01-14 (MD) hand-edited version of apsrev4-1.bst
%Control: key (0)
%Control: author (8) initials jnrlst
%Control: editor formatted (1) identically to author
%Control: production of article title (0) allowed
%Control: page (0) single
%Control: year (1) truncated
%Control: production of eprint (0) enabled
%apsrev4-2.bst 2019-01-14 (MD) hand-edited version of apsrev4-1.bst
%Control: key (0)
%Control: author (8) initials jnrlst
%Control: editor formatted (1) identically to author
%Control: production of article title (0) allowed
%Control: page (0) single
%Control: year (1) truncated
%Control: production of eprint (0) enabled
\begin{thebibliography}{42}%
\makeatletter
\providecommand \@ifxundefined [1]{%
 \@ifx{#1\undefined}
}%
\providecommand \@ifnum [1]{%
 \ifnum #1\expandafter \@firstoftwo
 \else \expandafter \@secondoftwo
 \fi
}%
\providecommand \@ifx [1]{%
 \ifx #1\expandafter \@firstoftwo
 \else \expandafter \@secondoftwo
 \fi
}%
\providecommand \natexlab [1]{#1}%
\providecommand \enquote  [1]{``#1''}%
\providecommand \bibnamefont  [1]{#1}%
\providecommand \bibfnamefont [1]{#1}%
\providecommand \citenamefont [1]{#1}%
\providecommand \href@noop [0]{\@secondoftwo}%
\providecommand \href [0]{\begingroup \@sanitize@url \@href}%
\providecommand \@href[1]{\@@startlink{#1}\@@href}%
\providecommand \@@href[1]{\endgroup#1\@@endlink}%
\providecommand \@sanitize@url [0]{\catcode `\\12\catcode `\$12\catcode
  `\&12\catcode `\#12\catcode `\^12\catcode `\_12\catcode `\%12\relax}%
\providecommand \@@startlink[1]{}%
\providecommand \@@endlink[0]{}%
\providecommand \url  [0]{\begingroup\@sanitize@url \@url }%
\providecommand \@url [1]{\endgroup\@href {#1}{\urlprefix }}%
\providecommand \urlprefix  [0]{URL }%
\providecommand \Eprint [0]{\href }%
\providecommand \doibase [0]{https://doi.org/}%
\providecommand \selectlanguage [0]{\@gobble}%
\providecommand \bibinfo  [0]{\@secondoftwo}%
\providecommand \bibfield  [0]{\@secondoftwo}%
\providecommand \translation [1]{[#1]}%
\providecommand \BibitemOpen [0]{}%
\providecommand \bibitemStop [0]{}%
\providecommand \bibitemNoStop [0]{.\EOS\space}%
\providecommand \EOS [0]{\spacefactor3000\relax}%
\providecommand \BibitemShut  [1]{\csname bibitem#1\endcsname}%
\let\auto@bib@innerbib\@empty
%</preamble>
\bibitem [{\citenamefont {Dekorsy}\ \emph {et~al.}(2000)\citenamefont
  {Dekorsy}, \citenamefont {Cho},\ and\ \citenamefont
  {Kurz}}]{dekorsy2000coherent}%
  \BibitemOpen
  \bibfield  {author} {\bibinfo {author} {\bibfnamefont {T.}~\bibnamefont
  {Dekorsy}}, \bibinfo {author} {\bibfnamefont {G.~C.}\ \bibnamefont {Cho}},\
  and\ \bibinfo {author} {\bibfnamefont {H.}~\bibnamefont {Kurz}},\ }\bibfield
  {title} {\bibinfo {title} {Coherent phonons in condensed media},\ }\href@noop
  {} {\bibfield  {journal} {\bibinfo  {journal} {Light scattering in solids
  VIII}\ }\textbf {\bibinfo {volume} {76}},\ \bibinfo {pages} {169} (\bibinfo
  {year} {2000})}\BibitemShut {NoStop}%
\bibitem [{\citenamefont {Ruello}\ and\ \citenamefont
  {Gusev}(2015)}]{ruello2015physical}%
  \BibitemOpen
  \bibfield  {author} {\bibinfo {author} {\bibfnamefont {P.}~\bibnamefont
  {Ruello}}\ and\ \bibinfo {author} {\bibfnamefont {V.~E.}\ \bibnamefont
  {Gusev}},\ }\bibfield  {title} {\bibinfo {title} {Physical mechanisms of
  coherent acoustic phonons generation by ultrafast laser action},\ }\href@noop
  {} {\bibfield  {journal} {\bibinfo  {journal} {Ultrasonics}\ }\textbf
  {\bibinfo {volume} {56}},\ \bibinfo {pages} {21} (\bibinfo {year}
  {2015})}\BibitemShut {NoStop}%
\bibitem [{\citenamefont {Poyser}\ \emph {et~al.}(2015)\citenamefont {Poyser},
  \citenamefont {Akimov}, \citenamefont {Campion},\ and\ \citenamefont
  {Kent}}]{poyser2015coherent}%
  \BibitemOpen
  \bibfield  {author} {\bibinfo {author} {\bibfnamefont {C.~L.}\ \bibnamefont
  {Poyser}}, \bibinfo {author} {\bibfnamefont {A.~V.}\ \bibnamefont {Akimov}},
  \bibinfo {author} {\bibfnamefont {R.~P.}\ \bibnamefont {Campion}},\ and\
  \bibinfo {author} {\bibfnamefont {A.~J.}\ \bibnamefont {Kent}},\ }\bibfield
  {title} {\bibinfo {title} {Coherent phonon optics in a chip with an
  electrically controlled active device},\ }\href@noop {} {\bibfield  {journal}
  {\bibinfo  {journal} {Sci. Rep.}\ }\textbf {\bibinfo {volume} {5}},\ \bibinfo
  {pages} {1} (\bibinfo {year} {2015})}\BibitemShut {NoStop}%
\bibitem [{\citenamefont {Ruskov}\ and\ \citenamefont
  {Tahan}(2012)}]{ruskov2012coherent}%
  \BibitemOpen
  \bibfield  {author} {\bibinfo {author} {\bibfnamefont {R.}~\bibnamefont
  {Ruskov}}\ and\ \bibinfo {author} {\bibfnamefont {C.}~\bibnamefont {Tahan}},\
  }\bibfield  {title} {\bibinfo {title} {Coherent phonons as a new element of
  quantum computing and devices},\ }in\ \href@noop {} {\emph {\bibinfo
  {booktitle} {J. Phys. Conf. Ser.}}},\ Vol.\ \bibinfo {volume} {398}\
  (\bibinfo {organization} {IOP Publishing},\ \bibinfo {year} {2012})\ p.\
  \bibinfo {pages} {012011}\BibitemShut {NoStop}%
\bibitem [{\citenamefont {Gustafsson}\ \emph {et~al.}(2014)\citenamefont
  {Gustafsson}, \citenamefont {Aref}, \citenamefont {Kockum}, \citenamefont
  {Ekstr{\"o}m}, \citenamefont {Johansson},\ and\ \citenamefont
  {Delsing}}]{gustafsson2014propagating}%
  \BibitemOpen
  \bibfield  {author} {\bibinfo {author} {\bibfnamefont {M.~V.}\ \bibnamefont
  {Gustafsson}}, \bibinfo {author} {\bibfnamefont {T.}~\bibnamefont {Aref}},
  \bibinfo {author} {\bibfnamefont {A.~F.}\ \bibnamefont {Kockum}}, \bibinfo
  {author} {\bibfnamefont {M.~K.}\ \bibnamefont {Ekstr{\"o}m}}, \bibinfo
  {author} {\bibfnamefont {G.}~\bibnamefont {Johansson}},\ and\ \bibinfo
  {author} {\bibfnamefont {P.}~\bibnamefont {Delsing}},\ }\bibfield  {title}
  {\bibinfo {title} {Propagating phonons coupled to an artificial atom},\
  }\href@noop {} {\bibfield  {journal} {\bibinfo  {journal} {Science}\ }\textbf
  {\bibinfo {volume} {346}},\ \bibinfo {pages} {207} (\bibinfo {year}
  {2014})}\BibitemShut {NoStop}%
\bibitem [{\citenamefont {Bienfait}\ \emph {et~al.}(2019)\citenamefont
  {Bienfait}, \citenamefont {Satzinger}, \citenamefont {Zhong}, \citenamefont
  {Chang}, \citenamefont {Chou}, \citenamefont {Conner}, \citenamefont {Dumur},
  \citenamefont {Grebel}, \citenamefont {Peairs}, \citenamefont {Povey} \emph
  {et~al.}}]{bienfait2019phonon}%
  \BibitemOpen
  \bibfield  {author} {\bibinfo {author} {\bibfnamefont {A.}~\bibnamefont
  {Bienfait}}, \bibinfo {author} {\bibfnamefont {K.~J.}\ \bibnamefont
  {Satzinger}}, \bibinfo {author} {\bibfnamefont {Y.}~\bibnamefont {Zhong}},
  \bibinfo {author} {\bibfnamefont {H.-S.}\ \bibnamefont {Chang}}, \bibinfo
  {author} {\bibfnamefont {M.-H.}\ \bibnamefont {Chou}}, \bibinfo {author}
  {\bibfnamefont {C.~R.}\ \bibnamefont {Conner}}, \bibinfo {author}
  {\bibfnamefont {{\'E}.}~\bibnamefont {Dumur}}, \bibinfo {author}
  {\bibfnamefont {J.}~\bibnamefont {Grebel}}, \bibinfo {author} {\bibfnamefont
  {G.~A.}\ \bibnamefont {Peairs}}, \bibinfo {author} {\bibfnamefont {R.~G.}\
  \bibnamefont {Povey}}, \emph {et~al.},\ }\bibfield  {title} {\bibinfo {title}
  {Phonon-mediated quantum state transfer and remote qubit entanglement},\
  }\href@noop {} {\bibfield  {journal} {\bibinfo  {journal} {Science}\ }\textbf
  {\bibinfo {volume} {364}},\ \bibinfo {pages} {368} (\bibinfo {year}
  {2019})}\BibitemShut {NoStop}%
\bibitem [{\citenamefont {Liu}\ \emph {et~al.}(2020)\citenamefont {Liu},
  \citenamefont {Guo},\ and\ \citenamefont {Wang}}]{liu2020review}%
  \BibitemOpen
  \bibfield  {author} {\bibinfo {author} {\bibfnamefont {J.}~\bibnamefont
  {Liu}}, \bibinfo {author} {\bibfnamefont {H.}~\bibnamefont {Guo}},\ and\
  \bibinfo {author} {\bibfnamefont {T.}~\bibnamefont {Wang}},\ }\bibfield
  {title} {\bibinfo {title} {A review of acoustic metamaterials and phononic
  crystals},\ }\href@noop {} {\bibfield  {journal} {\bibinfo  {journal}
  {Crystals}\ }\textbf {\bibinfo {volume} {10}},\ \bibinfo {pages} {305}
  (\bibinfo {year} {2020})}\BibitemShut {NoStop}%
\bibitem [{\citenamefont {Takabatake}\ \emph {et~al.}(2014)\citenamefont
  {Takabatake}, \citenamefont {Suekuni}, \citenamefont {Nakayama},\ and\
  \citenamefont {Kaneshita}}]{takabatake2014phonon}%
  \BibitemOpen
  \bibfield  {author} {\bibinfo {author} {\bibfnamefont {T.}~\bibnamefont
  {Takabatake}}, \bibinfo {author} {\bibfnamefont {K.}~\bibnamefont {Suekuni}},
  \bibinfo {author} {\bibfnamefont {T.}~\bibnamefont {Nakayama}},\ and\
  \bibinfo {author} {\bibfnamefont {E.}~\bibnamefont {Kaneshita}},\ }\bibfield
  {title} {\bibinfo {title} {Phonon-glass electron-crystal thermoelectric
  clathrates: Experiments and theory},\ }\href@noop {} {\bibfield  {journal}
  {\bibinfo  {journal} {Rev. Mod. Phys.}\ }\textbf {\bibinfo {volume} {86}},\
  \bibinfo {pages} {669} (\bibinfo {year} {2014})}\BibitemShut {NoStop}%
\bibitem [{\citenamefont {Li}\ \emph {et~al.}(2012)\citenamefont {Li},
  \citenamefont {Ren}, \citenamefont {Wang}, \citenamefont {Zhang},
  \citenamefont {H{\"a}nggi},\ and\ \citenamefont {Li}}]{li2012colloquium}%
  \BibitemOpen
  \bibfield  {author} {\bibinfo {author} {\bibfnamefont {N.}~\bibnamefont
  {Li}}, \bibinfo {author} {\bibfnamefont {J.}~\bibnamefont {Ren}}, \bibinfo
  {author} {\bibfnamefont {L.}~\bibnamefont {Wang}}, \bibinfo {author}
  {\bibfnamefont {G.}~\bibnamefont {Zhang}}, \bibinfo {author} {\bibfnamefont
  {P.}~\bibnamefont {H{\"a}nggi}},\ and\ \bibinfo {author} {\bibfnamefont
  {B.}~\bibnamefont {Li}},\ }\bibfield  {title} {\bibinfo {title} {Colloquium:
  Phononics: Manipulating heat flow with electronic analogs and beyond},\
  }\href@noop {} {\bibfield  {journal} {\bibinfo  {journal} {Rev. Mod. Phys.}\
  }\textbf {\bibinfo {volume} {84}},\ \bibinfo {pages} {1045} (\bibinfo {year}
  {2012})}\BibitemShut {NoStop}%
\bibitem [{\citenamefont {Li}\ \emph {et~al.}(2021)\citenamefont {Li},
  \citenamefont {Li}, \citenamefont {Han}, \citenamefont {Zheng}, \citenamefont
  {Li}, \citenamefont {Li}, \citenamefont {Fan},\ and\ \citenamefont
  {Qiu}}]{li2021transforming}%
  \BibitemOpen
  \bibfield  {author} {\bibinfo {author} {\bibfnamefont {Y.}~\bibnamefont
  {Li}}, \bibinfo {author} {\bibfnamefont {W.}~\bibnamefont {Li}}, \bibinfo
  {author} {\bibfnamefont {T.}~\bibnamefont {Han}}, \bibinfo {author}
  {\bibfnamefont {X.}~\bibnamefont {Zheng}}, \bibinfo {author} {\bibfnamefont
  {J.}~\bibnamefont {Li}}, \bibinfo {author} {\bibfnamefont {B.}~\bibnamefont
  {Li}}, \bibinfo {author} {\bibfnamefont {S.}~\bibnamefont {Fan}},\ and\
  \bibinfo {author} {\bibfnamefont {C.-W.}\ \bibnamefont {Qiu}},\ }\bibfield
  {title} {\bibinfo {title} {Transforming heat transfer with thermal
  metamaterials and devices},\ }\href@noop {} {\bibfield  {journal} {\bibinfo
  {journal} {Nat. Rev. Mater.}\ }\textbf {\bibinfo {volume} {6}},\ \bibinfo
  {pages} {488} (\bibinfo {year} {2021})}\BibitemShut {NoStop}%
\bibitem [{\citenamefont {Rugar}\ \emph {et~al.}(2004)\citenamefont {Rugar},
  \citenamefont {Budakian}, \citenamefont {Mamin},\ and\ \citenamefont
  {Chui}}]{rugar2004single}%
  \BibitemOpen
  \bibfield  {author} {\bibinfo {author} {\bibfnamefont {D.}~\bibnamefont
  {Rugar}}, \bibinfo {author} {\bibfnamefont {R.}~\bibnamefont {Budakian}},
  \bibinfo {author} {\bibfnamefont {H.}~\bibnamefont {Mamin}},\ and\ \bibinfo
  {author} {\bibfnamefont {B.}~\bibnamefont {Chui}},\ }\bibfield  {title}
  {\bibinfo {title} {Single spin detection by magnetic resonance force
  microscopy},\ }\href@noop {} {\bibfield  {journal} {\bibinfo  {journal}
  {Nature}\ }\textbf {\bibinfo {volume} {430}},\ \bibinfo {pages} {329}
  (\bibinfo {year} {2004})}\BibitemShut {NoStop}%
\bibitem [{\citenamefont {Yang}\ \emph {et~al.}(2006)\citenamefont {Yang},
  \citenamefont {Callegari}, \citenamefont {Feng}, \citenamefont {Ekinci},\
  and\ \citenamefont {Roukes}}]{yang2006zeptogram}%
  \BibitemOpen
  \bibfield  {author} {\bibinfo {author} {\bibfnamefont {Y.-T.}\ \bibnamefont
  {Yang}}, \bibinfo {author} {\bibfnamefont {C.}~\bibnamefont {Callegari}},
  \bibinfo {author} {\bibfnamefont {X.}~\bibnamefont {Feng}}, \bibinfo {author}
  {\bibfnamefont {K.~L.}\ \bibnamefont {Ekinci}},\ and\ \bibinfo {author}
  {\bibfnamefont {M.~L.}\ \bibnamefont {Roukes}},\ }\bibfield  {title}
  {\bibinfo {title} {Zeptogram-scale nanomechanical mass sensing},\ }\href@noop
  {} {\bibfield  {journal} {\bibinfo  {journal} {Nano Lett.}\ }\textbf
  {\bibinfo {volume} {6}},\ \bibinfo {pages} {583} (\bibinfo {year}
  {2006})}\BibitemShut {NoStop}%
\bibitem [{\citenamefont {Burg}\ \emph {et~al.}(2007)\citenamefont {Burg},
  \citenamefont {Godin}, \citenamefont {Knudsen}, \citenamefont {Shen},
  \citenamefont {Carlson}, \citenamefont {Foster}, \citenamefont {Babcock},\
  and\ \citenamefont {Manalis}}]{burg2007weighing}%
  \BibitemOpen
  \bibfield  {author} {\bibinfo {author} {\bibfnamefont {T.~P.}\ \bibnamefont
  {Burg}}, \bibinfo {author} {\bibfnamefont {M.}~\bibnamefont {Godin}},
  \bibinfo {author} {\bibfnamefont {S.~M.}\ \bibnamefont {Knudsen}}, \bibinfo
  {author} {\bibfnamefont {W.}~\bibnamefont {Shen}}, \bibinfo {author}
  {\bibfnamefont {G.}~\bibnamefont {Carlson}}, \bibinfo {author} {\bibfnamefont
  {J.~S.}\ \bibnamefont {Foster}}, \bibinfo {author} {\bibfnamefont
  {K.}~\bibnamefont {Babcock}},\ and\ \bibinfo {author} {\bibfnamefont {S.~R.}\
  \bibnamefont {Manalis}},\ }\bibfield  {title} {\bibinfo {title} {Weighing of
  biomolecules, single cells and single nanoparticles in fluid},\ }\href@noop
  {} {\bibfield  {journal} {\bibinfo  {journal} {Nature}\ }\textbf {\bibinfo
  {volume} {446}},\ \bibinfo {pages} {1066} (\bibinfo {year}
  {2007})}\BibitemShut {NoStop}%
\bibitem [{\citenamefont {Masmanidis}\ \emph {et~al.}(2007)\citenamefont
  {Masmanidis}, \citenamefont {Karabalin}, \citenamefont {De~Vlaminck},
  \citenamefont {Borghs}, \citenamefont {Freeman},\ and\ \citenamefont
  {Roukes}}]{masmanidis2007multifunctional}%
  \BibitemOpen
  \bibfield  {author} {\bibinfo {author} {\bibfnamefont {S.~C.}\ \bibnamefont
  {Masmanidis}}, \bibinfo {author} {\bibfnamefont {R.~B.}\ \bibnamefont
  {Karabalin}}, \bibinfo {author} {\bibfnamefont {I.}~\bibnamefont
  {De~Vlaminck}}, \bibinfo {author} {\bibfnamefont {G.}~\bibnamefont {Borghs}},
  \bibinfo {author} {\bibfnamefont {M.~R.}\ \bibnamefont {Freeman}},\ and\
  \bibinfo {author} {\bibfnamefont {M.~L.}\ \bibnamefont {Roukes}},\ }\bibfield
   {title} {\bibinfo {title} {Multifunctional nanomechanical systems via
  tunably coupled piezoelectric actuation},\ }\href@noop {} {\bibfield
  {journal} {\bibinfo  {journal} {Science}\ }\textbf {\bibinfo {volume}
  {317}},\ \bibinfo {pages} {780} (\bibinfo {year} {2007})}\BibitemShut
  {NoStop}%
\bibitem [{\citenamefont {Feng}\ \emph {et~al.}(2008)\citenamefont {Feng},
  \citenamefont {White}, \citenamefont {Hajimiri},\ and\ \citenamefont
  {Roukes}}]{feng2008self}%
  \BibitemOpen
  \bibfield  {author} {\bibinfo {author} {\bibfnamefont {X.}~\bibnamefont
  {Feng}}, \bibinfo {author} {\bibfnamefont {C.}~\bibnamefont {White}},
  \bibinfo {author} {\bibfnamefont {A.}~\bibnamefont {Hajimiri}},\ and\
  \bibinfo {author} {\bibfnamefont {M.~L.}\ \bibnamefont {Roukes}},\ }\bibfield
   {title} {\bibinfo {title} {A self-sustaining ultrahigh-frequency
  nanoelectromechanical oscillator},\ }\href@noop {} {\bibfield  {journal}
  {\bibinfo  {journal} {Nat. Nanotechnology}\ }\textbf {\bibinfo {volume}
  {3}},\ \bibinfo {pages} {342} (\bibinfo {year} {2008})}\BibitemShut {NoStop}%
\bibitem [{\citenamefont {Wen}\ \emph {et~al.}(2020)\citenamefont {Wen},
  \citenamefont {Ares}, \citenamefont {Schupp}, \citenamefont {Pei},
  \citenamefont {Briggs},\ and\ \citenamefont {Laird}}]{wen2020coherent}%
  \BibitemOpen
  \bibfield  {author} {\bibinfo {author} {\bibfnamefont {Y.}~\bibnamefont
  {Wen}}, \bibinfo {author} {\bibfnamefont {N.}~\bibnamefont {Ares}}, \bibinfo
  {author} {\bibfnamefont {F.}~\bibnamefont {Schupp}}, \bibinfo {author}
  {\bibfnamefont {T.}~\bibnamefont {Pei}}, \bibinfo {author} {\bibfnamefont
  {G.}~\bibnamefont {Briggs}},\ and\ \bibinfo {author} {\bibfnamefont
  {E.}~\bibnamefont {Laird}},\ }\bibfield  {title} {\bibinfo {title} {A
  coherent nanomechanical oscillator driven by single-electron tunnelling},\
  }\href@noop {} {\bibfield  {journal} {\bibinfo  {journal} {Nat. Physics}\
  }\textbf {\bibinfo {volume} {16}},\ \bibinfo {pages} {75} (\bibinfo {year}
  {2020})}\BibitemShut {NoStop}%
\bibitem [{\citenamefont {Mahboob}\ and\ \citenamefont
  {Yamaguchi}(2008)}]{mahboob2008bit}%
  \BibitemOpen
  \bibfield  {author} {\bibinfo {author} {\bibfnamefont {I.}~\bibnamefont
  {Mahboob}}\ and\ \bibinfo {author} {\bibfnamefont {H.}~\bibnamefont
  {Yamaguchi}},\ }\bibfield  {title} {\bibinfo {title} {Bit storage and bit
  flip operations in an electromechanical oscillator},\ }\href@noop {}
  {\bibfield  {journal} {\bibinfo  {journal} {Nat. Nanotechnology}\ }\textbf
  {\bibinfo {volume} {3}},\ \bibinfo {pages} {275} (\bibinfo {year}
  {2008})}\BibitemShut {NoStop}%
\bibitem [{\citenamefont {Unterreithmeier}\ \emph {et~al.}(2009)\citenamefont
  {Unterreithmeier}, \citenamefont {Weig},\ and\ \citenamefont
  {Kotthaus}}]{unterreithmeier2009universal}%
  \BibitemOpen
  \bibfield  {author} {\bibinfo {author} {\bibfnamefont {Q.~P.}\ \bibnamefont
  {Unterreithmeier}}, \bibinfo {author} {\bibfnamefont {E.~M.}\ \bibnamefont
  {Weig}},\ and\ \bibinfo {author} {\bibfnamefont {J.~P.}\ \bibnamefont
  {Kotthaus}},\ }\bibfield  {title} {\bibinfo {title} {Universal transduction
  scheme for nanomechanical systems based on dielectric forces},\ }\href@noop
  {} {\bibfield  {journal} {\bibinfo  {journal} {Nature}\ }\textbf {\bibinfo
  {volume} {458}},\ \bibinfo {pages} {1001} (\bibinfo {year}
  {2009})}\BibitemShut {NoStop}%
\bibitem [{\citenamefont {Tadokoro}\ \emph {et~al.}(2018)\citenamefont
  {Tadokoro}, \citenamefont {Tanaka},\ and\ \citenamefont
  {Dykman}}]{tadokoro2018driven}%
  \BibitemOpen
  \bibfield  {author} {\bibinfo {author} {\bibfnamefont {Y.}~\bibnamefont
  {Tadokoro}}, \bibinfo {author} {\bibfnamefont {H.}~\bibnamefont {Tanaka}},\
  and\ \bibinfo {author} {\bibfnamefont {M.}~\bibnamefont {Dykman}},\
  }\bibfield  {title} {\bibinfo {title} {Driven nonlinear nanomechanical
  resonators as digital signal detectors},\ }\href@noop {} {\bibfield
  {journal} {\bibinfo  {journal} {Sci. Rep.}\ }\textbf {\bibinfo {volume}
  {8}},\ \bibinfo {pages} {1} (\bibinfo {year} {2018})}\BibitemShut {NoStop}%
\bibitem [{\citenamefont {Tamayo}\ \emph {et~al.}(2001)\citenamefont {Tamayo},
  \citenamefont {Humphris}, \citenamefont {Owen},\ and\ \citenamefont
  {Miles}}]{tamayo2001high}%
  \BibitemOpen
  \bibfield  {author} {\bibinfo {author} {\bibfnamefont {J.}~\bibnamefont
  {Tamayo}}, \bibinfo {author} {\bibfnamefont {A.}~\bibnamefont {Humphris}},
  \bibinfo {author} {\bibfnamefont {R.}~\bibnamefont {Owen}},\ and\ \bibinfo
  {author} {\bibfnamefont {M.}~\bibnamefont {Miles}},\ }\bibfield  {title}
  {\bibinfo {title} {High-{Q} dynamic force microscopy in liquid and its
  application to living cells},\ }\href@noop {} {\bibfield  {journal} {\bibinfo
   {journal} {Biophys. J.}\ }\textbf {\bibinfo {volume} {81}},\ \bibinfo
  {pages} {526} (\bibinfo {year} {2001})}\BibitemShut {NoStop}%
\bibitem [{\citenamefont {Shekhawat}\ and\ \citenamefont
  {Dravid}(2005)}]{shekhawat2005nanoscale}%
  \BibitemOpen
  \bibfield  {author} {\bibinfo {author} {\bibfnamefont {G.~S.}\ \bibnamefont
  {Shekhawat}}\ and\ \bibinfo {author} {\bibfnamefont {V.~P.}\ \bibnamefont
  {Dravid}},\ }\bibfield  {title} {\bibinfo {title} {Nanoscale imaging of
  buried structures via scanning near-field ultrasound holography},\
  }\href@noop {} {\bibfield  {journal} {\bibinfo  {journal} {Science}\ }\textbf
  {\bibinfo {volume} {310}},\ \bibinfo {pages} {89} (\bibinfo {year}
  {2005})}\BibitemShut {NoStop}%
\bibitem [{\citenamefont {Tetard}\ \emph {et~al.}(2008)\citenamefont {Tetard},
  \citenamefont {Passian}, \citenamefont {Venmar}, \citenamefont {Lynch},
  \citenamefont {Voy}, \citenamefont {Shekhawat}, \citenamefont {Dravid},\ and\
  \citenamefont {Thundat}}]{tetard2008imaging}%
  \BibitemOpen
  \bibfield  {author} {\bibinfo {author} {\bibfnamefont {L.}~\bibnamefont
  {Tetard}}, \bibinfo {author} {\bibfnamefont {A.}~\bibnamefont {Passian}},
  \bibinfo {author} {\bibfnamefont {K.~T.}\ \bibnamefont {Venmar}}, \bibinfo
  {author} {\bibfnamefont {R.~M.}\ \bibnamefont {Lynch}}, \bibinfo {author}
  {\bibfnamefont {B.~H.}\ \bibnamefont {Voy}}, \bibinfo {author} {\bibfnamefont
  {G.}~\bibnamefont {Shekhawat}}, \bibinfo {author} {\bibfnamefont {V.~P.}\
  \bibnamefont {Dravid}},\ and\ \bibinfo {author} {\bibfnamefont
  {T.}~\bibnamefont {Thundat}},\ }\bibfield  {title} {\bibinfo {title} {Imaging
  nanoparticles in cells by nanomechanical holography},\ }\href@noop {}
  {\bibfield  {journal} {\bibinfo  {journal} {Nat. Nanotechnology}\ }\textbf
  {\bibinfo {volume} {3}},\ \bibinfo {pages} {501} (\bibinfo {year}
  {2008})}\BibitemShut {NoStop}%
\bibitem [{\citenamefont {Ohta}\ \emph {et~al.}(2017)\citenamefont {Ohta},
  \citenamefont {Okamoto},\ and\ \citenamefont {Yamaguchi}}]{ohta2017feedback}%
  \BibitemOpen
  \bibfield  {author} {\bibinfo {author} {\bibfnamefont {R.}~\bibnamefont
  {Ohta}}, \bibinfo {author} {\bibfnamefont {H.}~\bibnamefont {Okamoto}},\ and\
  \bibinfo {author} {\bibfnamefont {H.}~\bibnamefont {Yamaguchi}},\ }\bibfield
  {title} {\bibinfo {title} {Feedback control of multiple mechanical modes in
  coupled micromechanical resonators},\ }\href@noop {} {\bibfield  {journal}
  {\bibinfo  {journal} {Appl. Phys. Lett.}\ }\textbf {\bibinfo {volume}
  {110}},\ \bibinfo {pages} {053106} (\bibinfo {year} {2017})}\BibitemShut
  {NoStop}%
\bibitem [{\citenamefont {Sommer}\ and\ \citenamefont
  {Genes}(2019)}]{sommer2019partial}%
  \BibitemOpen
  \bibfield  {author} {\bibinfo {author} {\bibfnamefont {C.}~\bibnamefont
  {Sommer}}\ and\ \bibinfo {author} {\bibfnamefont {C.}~\bibnamefont {Genes}},\
  }\bibfield  {title} {\bibinfo {title} {Partial optomechanical refrigeration
  via multimode cold-damping feedback},\ }\href@noop {} {\bibfield  {journal}
  {\bibinfo  {journal} {Phys. Rev. Lett.}\ }\textbf {\bibinfo {volume} {123}},\
  \bibinfo {pages} {203605} (\bibinfo {year} {2019})}\BibitemShut {NoStop}%
\bibitem [{\citenamefont
  {{F}r{\"o}hlich}(1968{\natexlab{a}})}]{frohlich1968bose}%
  \BibitemOpen
  \bibfield  {author} {\bibinfo {author} {\bibfnamefont {H.}~\bibnamefont
  {{F}r{\"o}hlich}},\ }\bibfield  {title} {\bibinfo {title} {Bose condensation
  of strongly excited longitudinal electric modes},\ }\href@noop {} {\bibfield
  {journal} {\bibinfo  {journal} {Phys. Lett. A}\ }\textbf {\bibinfo {volume}
  {26}},\ \bibinfo {pages} {402} (\bibinfo {year}
  {1968}{\natexlab{a}})}\BibitemShut {NoStop}%
\bibitem [{\citenamefont
  {{F}r{\"o}hlich}(1968{\natexlab{b}})}]{frohlich1968long}%
  \BibitemOpen
  \bibfield  {author} {\bibinfo {author} {\bibfnamefont {H.}~\bibnamefont
  {{F}r{\"o}hlich}},\ }\bibfield  {title} {\bibinfo {title} {Long-range
  coherence and energy storage in biological systems},\ }\href@noop {}
  {\bibfield  {journal} {\bibinfo  {journal} {Int. J. Quantum Chem.}\ }\textbf
  {\bibinfo {volume} {2}},\ \bibinfo {pages} {641} (\bibinfo {year}
  {1968}{\natexlab{b}})}\BibitemShut {NoStop}%
\bibitem [{\citenamefont {{F}r{\"o}hlich}(1970)}]{frohlich1970long}%
  \BibitemOpen
  \bibfield  {author} {\bibinfo {author} {\bibfnamefont {H.}~\bibnamefont
  {{F}r{\"o}hlich}},\ }\bibfield  {title} {\bibinfo {title} {Long range
  coherence and the action of enzymes},\ }\href@noop {} {\bibfield  {journal}
  {\bibinfo  {journal} {Nature}\ }\textbf {\bibinfo {volume} {228}},\ \bibinfo
  {pages} {1093} (\bibinfo {year} {1970})}\BibitemShut {NoStop}%
\bibitem [{\citenamefont {Wu}\ and\ \citenamefont {Austin}(1977)}]{wu1977bose}%
  \BibitemOpen
  \bibfield  {author} {\bibinfo {author} {\bibfnamefont {T.}~\bibnamefont
  {Wu}}\ and\ \bibinfo {author} {\bibfnamefont {S.}~\bibnamefont {Austin}},\
  }\bibfield  {title} {\bibinfo {title} {Bose condensation in biosystems},\
  }\href@noop {} {\bibfield  {journal} {\bibinfo  {journal} {Phys. Lett. A}\
  }\textbf {\bibinfo {volume} {64}},\ \bibinfo {pages} {151} (\bibinfo {year}
  {1977})}\BibitemShut {NoStop}%
\bibitem [{\citenamefont {Wu}\ and\ \citenamefont
  {Austin}(1978)}]{wu1978cooperative}%
  \BibitemOpen
  \bibfield  {author} {\bibinfo {author} {\bibfnamefont {T.}~\bibnamefont
  {Wu}}\ and\ \bibinfo {author} {\bibfnamefont {S.}~\bibnamefont {Austin}},\
  }\bibfield  {title} {\bibinfo {title} {Cooperative behavior in biological
  systems},\ }\href@noop {} {\bibfield  {journal} {\bibinfo  {journal} {Phys.
  Lett. A}\ }\textbf {\bibinfo {volume} {65}},\ \bibinfo {pages} {74} (\bibinfo
  {year} {1978})}\BibitemShut {NoStop}%
\bibitem [{\citenamefont {Wu}\ and\ \citenamefont
  {Austin}(1981)}]{wu1981frohlich}%
  \BibitemOpen
  \bibfield  {author} {\bibinfo {author} {\bibfnamefont {T.}~\bibnamefont
  {Wu}}\ and\ \bibinfo {author} {\bibfnamefont {S.~J.}\ \bibnamefont
  {Austin}},\ }\bibfield  {title} {\bibinfo {title} {{F}r{\"o}hlich's model of
  {B}ose condensation in biological systems},\ }\href@noop {} {\bibfield
  {journal} {\bibinfo  {journal} {J. Biol. Phys.}\ }\textbf {\bibinfo {volume}
  {9}},\ \bibinfo {pages} {97} (\bibinfo {year} {1981})}\BibitemShut {NoStop}%
\bibitem [{\citenamefont {Reimers}\ \emph {et~al.}(2009)\citenamefont
  {Reimers}, \citenamefont {McKemmish}, \citenamefont {McKenzie}, \citenamefont
  {Mark},\ and\ \citenamefont {Hush}}]{reimers2009weak}%
  \BibitemOpen
  \bibfield  {author} {\bibinfo {author} {\bibfnamefont {J.~R.}\ \bibnamefont
  {Reimers}}, \bibinfo {author} {\bibfnamefont {L.~K.}\ \bibnamefont
  {McKemmish}}, \bibinfo {author} {\bibfnamefont {R.~H.}\ \bibnamefont
  {McKenzie}}, \bibinfo {author} {\bibfnamefont {A.~E.}\ \bibnamefont {Mark}},\
  and\ \bibinfo {author} {\bibfnamefont {N.~S.}\ \bibnamefont {Hush}},\
  }\bibfield  {title} {\bibinfo {title} {Weak, strong, and coherent regimes of
  {F}r{\"o}hlich condensation and their applications to terahertz medicine and
  quantum consciousness},\ }\href@noop {} {\bibfield  {journal} {\bibinfo
  {journal} {Proc. Nat. Acad. Sci.}\ }\textbf {\bibinfo {volume} {106}},\
  \bibinfo {pages} {4219} (\bibinfo {year} {2009})}\BibitemShut {NoStop}%
\bibitem [{\citenamefont {Preto}(2017)}]{preto2017semi}%
  \BibitemOpen
  \bibfield  {author} {\bibinfo {author} {\bibfnamefont {J.}~\bibnamefont
  {Preto}},\ }\bibfield  {title} {\bibinfo {title} {Semi-classical statistical
  description of {F}r{\"o}hlich condensation},\ }\href@noop {} {\bibfield
  {journal} {\bibinfo  {journal} {J. Biol. Phys.}\ }\textbf {\bibinfo {volume}
  {43}},\ \bibinfo {pages} {167} (\bibinfo {year} {2017})}\BibitemShut
  {NoStop}%
\bibitem [{\citenamefont {Zhang}\ \emph {et~al.}(2019)\citenamefont {Zhang},
  \citenamefont {Agarwal},\ and\ \citenamefont {Scully}}]{zhang2019quantum}%
  \BibitemOpen
  \bibfield  {author} {\bibinfo {author} {\bibfnamefont {Z.}~\bibnamefont
  {Zhang}}, \bibinfo {author} {\bibfnamefont {G.~S.}\ \bibnamefont {Agarwal}},\
  and\ \bibinfo {author} {\bibfnamefont {M.~O.}\ \bibnamefont {Scully}},\
  }\bibfield  {title} {\bibinfo {title} {Quantum fluctuations in the
  {F}r{\"o}hlich condensate of molecular vibrations driven far from
  equilibrium},\ }\href@noop {} {\bibfield  {journal} {\bibinfo  {journal}
  {Phys. Rev. Lett.}\ }\textbf {\bibinfo {volume} {122}},\ \bibinfo {pages}
  {158101} (\bibinfo {year} {2019})}\BibitemShut {NoStop}%
\bibitem [{\citenamefont {Zheng}\ and\ \citenamefont
  {Li}(2021)}]{zheng2021froh}%
  \BibitemOpen
  \bibfield  {author} {\bibinfo {author} {\bibfnamefont {X.}~\bibnamefont
  {Zheng}}\ and\ \bibinfo {author} {\bibfnamefont {B.}~\bibnamefont {Li}},\
  }\bibfield  {title} {\bibinfo {title} {Fr\"ohlich condensate of phonons in
  optomechanical systems},\ }\href@noop {} {\bibfield  {journal} {\bibinfo
  {journal} {Phys. Rev. A}\ }\textbf {\bibinfo {volume} {104}},\ \bibinfo
  {pages} {043512} (\bibinfo {year} {2021})}\BibitemShut {NoStop}%
\bibitem [{\citenamefont {Rackauckas}\ and\ \citenamefont
  {Nie}(2017)}]{rackauckas2017differentialequations}%
  \BibitemOpen
  \bibfield  {author} {\bibinfo {author} {\bibfnamefont {C.}~\bibnamefont
  {Rackauckas}}\ and\ \bibinfo {author} {\bibfnamefont {Q.}~\bibnamefont
  {Nie}},\ }\bibfield  {title} {\bibinfo {title} {Differentialequations.jl--a
  performant and feature-rich ecosystem for solving differential equations in
  julia},\ }\href@noop {} {\bibfield  {journal} {\bibinfo  {journal} {J. Open
  Res. Softw.}\ }\textbf {\bibinfo {volume} {5}} (\bibinfo {year}
  {2017})}\BibitemShut {NoStop}%
\bibitem [{\citenamefont {Landau}\ \emph {et~al.}(1986)\citenamefont {Landau},
  \citenamefont {Pitaevskii}, \citenamefont {Kosevich},\ and\ \citenamefont
  {Lifshitzch}}]{landau1986course}%
  \BibitemOpen
  \bibfield  {author} {\bibinfo {author} {\bibfnamefont {L.~D.}\ \bibnamefont
  {Landau}}, \bibinfo {author} {\bibfnamefont {L.~P.}\ \bibnamefont
  {Pitaevskii}}, \bibinfo {author} {\bibfnamefont {A.~M.}\ \bibnamefont
  {Kosevich}},\ and\ \bibinfo {author} {\bibfnamefont {E.~M.}\ \bibnamefont
  {Lifshitzch}},\ }\href@noop {} {\emph {\bibinfo {title} {Course of Theorical
  Physics: Theory of Elasticity}}}\ (\bibinfo  {publisher} {Pergamon press},\
  \bibinfo {year} {1986})\BibitemShut {NoStop}%
\bibitem [{\citenamefont {Pettit}\ \emph {et~al.}(2019)\citenamefont {Pettit},
  \citenamefont {Ge}, \citenamefont {Kumar}, \citenamefont {Luntz-Martin},
  \citenamefont {Schultz}, \citenamefont {Neukirch}, \citenamefont
  {Bhattacharya},\ and\ \citenamefont {Vamivakas}}]{pettit2019optical}%
  \BibitemOpen
  \bibfield  {author} {\bibinfo {author} {\bibfnamefont {R.~M.}\ \bibnamefont
  {Pettit}}, \bibinfo {author} {\bibfnamefont {W.}~\bibnamefont {Ge}}, \bibinfo
  {author} {\bibfnamefont {P.}~\bibnamefont {Kumar}}, \bibinfo {author}
  {\bibfnamefont {D.~R.}\ \bibnamefont {Luntz-Martin}}, \bibinfo {author}
  {\bibfnamefont {J.~T.}\ \bibnamefont {Schultz}}, \bibinfo {author}
  {\bibfnamefont {L.~P.}\ \bibnamefont {Neukirch}}, \bibinfo {author}
  {\bibfnamefont {M.}~\bibnamefont {Bhattacharya}},\ and\ \bibinfo {author}
  {\bibfnamefont {A.~N.}\ \bibnamefont {Vamivakas}},\ }\bibfield  {title}
  {\bibinfo {title} {An optical tweezer phonon laser},\ }\href@noop {}
  {\bibfield  {journal} {\bibinfo  {journal} {Nat. Photonics}\ }\textbf
  {\bibinfo {volume} {13}},\ \bibinfo {pages} {402} (\bibinfo {year}
  {2019})}\BibitemShut {NoStop}%
\bibitem [{\citenamefont {Pikovsky}\ \emph {et~al.}(2003)\citenamefont
  {Pikovsky}, \citenamefont {Rosenblum},\ and\ \citenamefont
  {Kurths}}]{pikovsky2003synchronization}%
  \BibitemOpen
  \bibfield  {author} {\bibinfo {author} {\bibfnamefont {A.}~\bibnamefont
  {Pikovsky}}, \bibinfo {author} {\bibfnamefont {M.}~\bibnamefont
  {Rosenblum}},\ and\ \bibinfo {author} {\bibfnamefont {J.}~\bibnamefont
  {Kurths}},\ }\href@noop {} {\emph {\bibinfo {title} {Synchronization: a
  universal concept in nonlinear sciences}}},\ Vol.~\bibinfo {volume} {12}\
  (\bibinfo  {publisher} {Cambridge University Press},\ \bibinfo {year}
  {2003})\BibitemShut {NoStop}%
\bibitem [{\citenamefont {Jonsson}\ \emph {et~al.}(2008)\citenamefont
  {Jonsson}, \citenamefont {Santandrea}, \citenamefont {Gorelik}, \citenamefont
  {Shekhter},\ and\ \citenamefont {Jonson}}]{jonsson2008self}%
  \BibitemOpen
  \bibfield  {author} {\bibinfo {author} {\bibfnamefont {L.~M.}\ \bibnamefont
  {Jonsson}}, \bibinfo {author} {\bibfnamefont {F.}~\bibnamefont {Santandrea}},
  \bibinfo {author} {\bibfnamefont {L.~Y.}\ \bibnamefont {Gorelik}}, \bibinfo
  {author} {\bibfnamefont {R.~I.}\ \bibnamefont {Shekhter}},\ and\ \bibinfo
  {author} {\bibfnamefont {M.}~\bibnamefont {Jonson}},\ }\bibfield  {title}
  {\bibinfo {title} {Self-organization of irregular nanoelectromechanical
  vibrations in multimode shuttle structures},\ }\href@noop {} {\bibfield
  {journal} {\bibinfo  {journal} {Phys. Rev. Lett.}\ }\textbf {\bibinfo
  {volume} {100}},\ \bibinfo {pages} {186802} (\bibinfo {year}
  {2008})}\BibitemShut {NoStop}%
\bibitem [{\citenamefont {Kemiktarak}\ \emph {et~al.}(2014)\citenamefont
  {Kemiktarak}, \citenamefont {Durand}, \citenamefont {Metcalfe},\ and\
  \citenamefont {Lawall}}]{kemiktarak2014mode}%
  \BibitemOpen
  \bibfield  {author} {\bibinfo {author} {\bibfnamefont {U.}~\bibnamefont
  {Kemiktarak}}, \bibinfo {author} {\bibfnamefont {M.}~\bibnamefont {Durand}},
  \bibinfo {author} {\bibfnamefont {M.}~\bibnamefont {Metcalfe}},\ and\
  \bibinfo {author} {\bibfnamefont {J.}~\bibnamefont {Lawall}},\ }\bibfield
  {title} {\bibinfo {title} {Mode competition and anomalous cooling in a
  multimode phonon laser},\ }\href@noop {} {\bibfield  {journal} {\bibinfo
  {journal} {Phys. Rev. Lett.}\ }\textbf {\bibinfo {volume} {113}},\ \bibinfo
  {pages} {030802} (\bibinfo {year} {2014})}\BibitemShut {NoStop}%
\bibitem [{\citenamefont {Zhang}\ \emph {et~al.}(2018)\citenamefont {Zhang},
  \citenamefont {Lin}, \citenamefont {Tian}, \citenamefont {Du}, \citenamefont
  {Zou}, \citenamefont {Chau},\ and\ \citenamefont {Zhou}}]{zhang2018mode}%
  \BibitemOpen
  \bibfield  {author} {\bibinfo {author} {\bibfnamefont {X.}~\bibnamefont
  {Zhang}}, \bibinfo {author} {\bibfnamefont {T.}~\bibnamefont {Lin}}, \bibinfo
  {author} {\bibfnamefont {F.}~\bibnamefont {Tian}}, \bibinfo {author}
  {\bibfnamefont {H.}~\bibnamefont {Du}}, \bibinfo {author} {\bibfnamefont
  {Y.}~\bibnamefont {Zou}}, \bibinfo {author} {\bibfnamefont {F.~S.}\
  \bibnamefont {Chau}},\ and\ \bibinfo {author} {\bibfnamefont
  {G.}~\bibnamefont {Zhou}},\ }\bibfield  {title} {\bibinfo {title} {Mode
  competition and hopping in optomechanical nano-oscillators},\ }\href@noop {}
  {\bibfield  {journal} {\bibinfo  {journal} {Appl. Phys. Lett.}\ }\textbf
  {\bibinfo {volume} {112}},\ \bibinfo {pages} {153502} (\bibinfo {year}
  {2018})}\BibitemShut {NoStop}%
\bibitem [{\citenamefont {Sommer}\ \emph {et~al.}(2020)\citenamefont {Sommer},
  \citenamefont {Asjad},\ and\ \citenamefont {Genes}}]{sommer2020prospects}%
  \BibitemOpen
  \bibfield  {author} {\bibinfo {author} {\bibfnamefont {C.}~\bibnamefont
  {Sommer}}, \bibinfo {author} {\bibfnamefont {M.}~\bibnamefont {Asjad}},\ and\
  \bibinfo {author} {\bibfnamefont {C.}~\bibnamefont {Genes}},\ }\bibfield
  {title} {\bibinfo {title} {Prospects of reinforcement learning for the
  simultaneous damping of many mechanical modes},\ }\href@noop {} {\bibfield
  {journal} {\bibinfo  {journal} {Sci. Rep.}\ }\textbf {\bibinfo {volume}
  {10}},\ \bibinfo {pages} {1} (\bibinfo {year} {2020})}\BibitemShut {NoStop}%
\end{thebibliography}%



\end{document}
%
% ****** End of file apssamp.tex ******
