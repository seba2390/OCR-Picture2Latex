  Mixtures provide some technical tools, which we will use in
  Section~\ref{s:tropical}. The input data for the mixture operation
  is a family of $\Gbf$-con\-fi\-gu\-ra\-tions, parametrized by a
  probability space. As result one obtains another
  $\Gbf$-con\-fi\-gu\-ra\-tion with the pre-specified
  conditionals. One particular instance of a mixture is when one mixes
  two configurations $\Xcal$ and $\set{\bullet}^{\Gbf}$, the latter
    being a constant $\Gbf$-configuration of one-point probability spaces. This
  operation will be used as a substitute for taking radicals
  ``$\Xcal^{\otimes(1/n)}$'' in Section~\ref{s:tropical} below.

\subsection{Definition and elementary properties}
  Let $\Gbf$ be a complete diagram category and $\Theta$ be a
  probability space. Let $\set{\Xcal_{\theta}}_{\theta\in\un\Theta}$
  be a family of $\Gbf$-configurations parametrized by $\Theta$. The
  \term{mixture} of the family $\set{\Xcal_{\theta}}$ is the reduction
  \[      
  \mix\set{\Xcal_{\theta}}=
  \left(
    \Ycal
    \too
    \Theta^{\Gbf}
  \right)
  \]      
  such that
  \[\tageq{mixture}
  \Ycal\rel\theta\cong\Xcal_{\theta}
  \]

  The mixture exists and is uniquely defined by
  property~(\ref{eq:mixture}) up to an isomorphism which is identity
  on $\Theta^{\Gbf}$.

  We denote the top configuration of the mixture
  \[
  \Ycal=\bigoplus_{\theta\in\Theta}\Xcal_{\theta}
  \]
  and also call it the mixture of the family $\set{\Xcal_{\theta}}$.

  When 
  \[
  \Theta=\Lambda_{\alpha}
  =
  \big(\set{\square,\blacksquare};
  p(\blacksquare)=\alpha
  \big) 
  \]
  is a binary space we write simply
  \[
  \Xcal_{\blacksquare}\oplus_{\Lambda_{\alpha}}\Xcal_{\square}
  \]
  for the mixture. The configuration subindexed by the $\blacksquare$
  will always be the first summand.

  The entropy of the mixture can be evaluated by the following formula
  \[
  \ent_{*}\left(\bigoplus_{\theta\in\Theta}\Xcal_{\theta}\right)
  =
  \int_{\Theta}\ent_{*}(\Xcal_{\theta})\d
  p(\theta) + \ent_{*}(\Theta^{\Gbf})
  \]
  Mixtures satisfy the distributive law with respect to the tensor
  product
  \begin{align*}
    \mix(\set{\Xcal_{\theta}}_{\theta\in\Theta})
    \otimes
    \mix(\set{\Ycal_{\theta'}}_{\theta'\in\Theta'})
    &\cong
    \mix(\set{\Xcal_{\theta}\otimes\Ycal_{\theta'}}
    _{(\theta,\theta')\in\Theta\otimes\Theta'})
    \\
    \left(\bigoplus_{\theta\in\Theta}\Xcal_{\theta}\right)
    \otimes
    \left(\bigoplus_{\theta'\in\Theta'}\Ycal_{\theta'}\right)
    &\cong
    \bigoplus_{(\theta,\theta')\in\Theta\otimes\Theta'}(\Xcal_{\theta}\otimes\Ycal_{\theta'})
  \end{align*}
  
\subsection{The distance estimates}
  Recall that for a diagram category $\Gbf$ we denote by
  $\set{\bullet}=\set{\bullet}^{\Gbf}$ the constant
  $\Gbf$-configuration of one-point spaces. 
  
  The mixture of a $\Gbf$-configuration with $\set{\bullet}^{\Gbf}$
  may serve as an ersatz of taking radicals of the configuration.  The
  following lemma provides a justification of this by some distance
  estimates related to mixtures and will be used in
  Section~\ref{s:tropical}.

  \begin{lemma}{p:mixtures-dist1}
    Let $\Gbf$ be a complete diagram category and
    $\Xcal,\Ycal\in\prob\<\Gbf>$. Then
    \begin{enumerate}
    \item
      $\displaystyle
      \aikd(\Xcal,\Xcal^{\otimes n}\oplus_{\Lambda_{1/n}}\set{\bullet})
      \leq
      \ent(\Lambda_{1/n})
      $
    \item
      $\displaystyle
      \aikd\big(\Xcal,(\Xcal\oplus_{\Lambda_{1/n}}\set{\bullet})^{\otimes n}\big)
      \leq
      n\cdot\ent(\Lambda_{1/n})
      $
    \item
      $\displaystyle
      \aikd\big(
      (\Xcal\otimes\Ycal)\oplus_{\Lambda_{1/n}}\set{\bullet},
      (\Xcal\oplus_{\Lambda_{1/n}}\set{\bullet})
      \otimes
      (\Ycal\oplus_{\Lambda_{1/n}}\set{\bullet})
      \big)
      \leq
      3\ent(\Lambda_{1/n})
      $
    \item
      $\displaystyle
      \aikd\big((\Xcal\oplus_{\Lambda_{1/n}}\set{\bullet}),
      (\Ycal\oplus_{\Lambda_{1/n}}\set{\bullet})\big)
      \leq
      \frac1n\aikd(\Xcal,\Ycal)
      $
    \end{enumerate}
  \end{lemma}
  
  The proof can be found on page \pageref{p:mixtures-dist1.rep}.  Note,
  that the distance estimates in the lemma above are with respect to
  the asymptotic Kolmogorov distance. This is essential, since from
  the perspective of the intrinsic Kolmogorov distance mixtures are
  very badly behaved.




