\let\thesubsectionstandard=\thesubsection
This section contains some proofs that did not make it into the
  main text.

%%%%%%%%%%%%%%%%%%%%%%%%%%%%%%%%%%%%%%%%%%%%%%%%%%%%%%%%%%%%%%%%%%%%%%%%%%%%%%%%
\def\thesubsection{\thesection.\ref{s:config}}
\subsection{Statements from the section ``Configurations''}
\repeatclaim{p:minimalfansconfig}

\begin{Proof}
  We will need the following lemma
  \begin{tlemma}{p:minimising-reduction}
    Suppose we are given two two-fans of probability spaces
    \begin{align*}
      \Fcal
      &=
      (X\oot[\alpha]Z\too[\beta]Y)\\
      \Fcal''
      &=
      (X''\oot[\alpha''] Z''\too[\beta''] Y'')
    \end{align*}
    such that $\Fcal''$ is minimal. Let
    \[
    \Fcal\,\too[\mu]\,\Fcal'\!\!=\!(X\!\oot[\alpha']\! Z'\!\too[\beta']\! Y)
    \]
    be a minimal reduction of
    $\Fcal$. Then for any reduction $\rho:\Fcal\to\Fcal''$, there exists
    a reduction $\rho':\Fcal'\to\Fcal''$ such that $\rho=\rho'\circ \mu$
  \end{tlemma}
  \begin{Proof}
  	We define $\rho'$ on the terminal spaces of $\Fcal'$ to coincide with $\rho$.
   
    To prove the lemma we just need to provide a dashed arrow that makes the
    following diagram commutative
    \[
    \begin{tikzcd}[row sep=normal, column sep=small, ampersand replacement=\&]
      \mbox{}
      \&
      Z
      \arrow{d}[description]{\mu}
      \arrow{ddl}[description]{\alpha}
      \arrow{ddr}[description, near start]{\beta}
      \arrow{drrr}{\rho}
      \\
%       \mbox{}\&\&
%       \\
      \&
      Z'
      \arrow{dl}[description]{\alpha'}
      \arrow{dr}[description]{\beta'}
      \arrow[dashrightarrow, crossing over]{rrr}{\rho'}
      \&
      \&
      \&
      Z''
      \arrow{dl}[description]{\alpha''}
      \arrow{dr}[description]{\beta''}
      \\
      X
      \&
      \&
      Y
      \arrow[bend right, crossing over]{rrr}{\rho=\rho'}
      \&
      X''
      \arrow[bend left, leftarrow, crossing over]{lll}[swap]{\rho=\rho'}
      \&
      \&
      Y''
    \end{tikzcd}
    \]
    The reduction $\rho'$ is constructed by simple diagram chasing and by
    using the minimality of $\Fcal''$.  
    Suppose $z'\in \un {Z'}$ and $z_{1},z_{2}\in\un Z$ are such that
    $z' = \mu(z_{1})=\mu(z_{2})$. By commutativity of the solid arrows in the
    diagram above, we have 
    \[ 
    \alpha''\circ\rho(z_{1})
    =
    \rho\circ\alpha'\circ\mu(z_{1})
    =
    \rho\circ\alpha'\circ\mu(z_{2})
    =
    \alpha''\circ\rho(z_{2})
    \]
    Similarly
    \[
    \beta''\circ\rho(z_{1})
    =
    \beta''\circ\rho(z_{2})
    \]
    Thus by minimality of $\Fcal''$ it follows that
    $\rho(z_{1})=\rho(z_{2})$. Hence, $\rho'$ can be constructed by setting $\rho'(z') = \rho(z_1)$.
  \end{Proof}
  Now we proceed to prove claim~(\ref{p:minimalfansconfig1}) of
  Lemma~\ref{p:minimalfansconfig}. Let $\Gbf=\set{O_{i};m_{ij}}$ be a
  diagram category, $\Xcal,\Ycal,\Zcal\in\Prob\<\Gbf>$ be three
  $\Gbf$-configurations and  
  $\Fcal=(\Xcal\ot\Zcal\to\Ycal)$ be a two-fan.
  Recall that it can also be considered as a $\Gbf$-configuration of
  two-fans 
  \[
  \Fcal=\set{\Fcal_{i};f_{ij}}
  \]
  Any minimizing reduction
  \[
  \Fcal\!=\!(\Xcal\!\ot\!\Zcal\!\to\!\Ycal)
  \too
  \Fcal'\!=\!(\Xcal\!\ot\!\Zcal'\!\to\!\Ycal)
  \]
  induces reductions 
  \[
  \Fcal_{i}\!=\!(X_{i}\!\ot\! Z_{i}\!\to\! Y_{i})
  \too
  \Fcal_{i}\!=\!(X_{i}\!\ot\! Z_{i}'\!\to\! Y_{i})
  \]
  for all $i$ in the index set $I$.
  It follows that if all $\Fcal_{i}$'s are minimal, then so is
  $\Fcal$. 

  Now we prove the implication in the other direction. Suppose $\Fcal$ is
  minimal. We have to show that all $\Fcal_{i}$ are minimal as well.
  Suppose there exist a non-minimal fan among $\Fcal_{i}$'s.  For an
  index $i \in I$ let
  \begin{align*}
    \check J(i)
    &:=
    \set{j\in I \st\Hom_{\Gbf}(O_{j},O_{i})\neq\emptyset}
    \\
    \hat J(i)
    &:=
    \set{j\in I \st\Hom_{\Gbf}(O_{i},O_{j})\neq\emptyset}
  \end{align*}
  
 	  
  Choose an index $i_{0}$ such that 
  \begin{enumerate}
  	\item  $\Fcal_{i_{0}}$ is not minimal
  	\item for any $j\in \hat J(i_{0}) \backslash \{ i_0 \}$ the two-fan $\Fcal_{j}$ is
  minimal. 
  \end{enumerate}
  Consider now the minimal reduction
  $\mu:\Fcal_{i_{0}}\to\Fcal'_{i_{0}}$ and construct a two-fan
  $\Gcal=\set{\Gcal_{i};g_{ij}}$ of $\Gbf$-configurations by setting
  \[
  \Gcal_{i}
  :=
  \begin{cases}
    \Fcal'_{i}
    &
    \text{if $i=i_{0}$}
    \\
    \Fcal_{i}
    &
    \text{otherwise}
  \end{cases}
  \]
  and 
  \[
  g_{ij}
  :=
  \begin{cases}
    \mu\circ f_{ij}
    &
    \text{if $j=i_{0}$ and $i\in\check J(i_{0})$}
    \\
    f'_{ij}
    &
    \text{if $i=i_{0}$ and $j\in\hat J(i_{0})$}
    \\
    f_{ij}
    &
    \text{otherwise}
  \end{cases}
  \]
  where $f'_{i_{0}j}$ is the reduction provided by the
  Lemma~\ref{p:minimising-reduction} applied to the diagram
  \[
  \begin{tikzcd}[ampersand replacement=\&]
    \Fcal_{i_{0}}
    \arrow{d}{\mu}
    \arrow{dr}{f_{i_{0}j}}
    \\
    \Fcal'_{i_{0}}
    \arrow[dashrightarrow]{r}{f'_{i_{0}j}}
    \&
    \Fcal_{j}
  \end{tikzcd}
  \]
  We thus constructed a non-trivial reduction $\Fcal\to\Gcal$ which is
  identity on the terminal  $\Gbf$-configurations $\Xcal$ and
  $\Ycal$. This contradicts the minimality of $\Fcal$.
  
  To address the second assertion of the
  Lemma~\ref{p:minimalfansconfig} observe that the argument above
  gives an algorithm for the construction of a minimal reduction of any
  two-fan of $\Gbf$-configurations.
\end{Proof}

%%%%%%%%%%%%%%%%%%%%%%%%%%%%%%%%%%%%%%%%%%%%%%%%%%%%%%%%%%%%%%%%%%%%%%%%%%%%%%%%


%%%%%%%%%%%%%%%%%%%%%%%%%%%%%%%%%%%%%%%%%%%%%%%%%%%%%%%%%%%%%%%%%%%%%%%%%%%%%%%%


%%%%%%%%%%%%%%%%%%%%%%%%%%%%%%%%%%%%%%%%%%%%%%%%%%%%%%%%%%%%%%%%%%%%%%%%%%%%%%%%
\def\thesubsection{\thesection.\ref{s:kolmogorov}}
\subsection{Statements from the section ``Kolmogorov distance''}
\repeatclaim{p:kolmogorovisdistance}

\begin{Proof}
  The symmetry of $\ikd$ is immediate. The non-negativity of $\ikd$
  follows from the fact that entropy of the target space of a reduction
  is not greater then the entropy of the domain, which is a
  particular instance of the Shannon inequality~(\ref{eq:shannonineq}).

  We proceed to prove the triangle inequality. We will make use of the
  following lemma
  \begin{tlemma}{p:kdtriangle}
    For a minimal full configuration of probability spaces
    \[
    \<X,Y,Z>=
    \left(
    \begin{tikzcd}[row sep=small,column sep=tiny,ampersand replacement=\&]
      \mbox{}
      \&
      XYZ
      \arrow{dl}{}
      \arrow{d}{}
      \arrow{dr}{}
      \&
      \\
      XY
      \arrow{d}{}
      \&
      XZ
      \arrow{dr}{}
      \arrow{dl}{}
      \&
      YZ
      \arrow{d}{}
      \\
      X
      \&
      Y
      \arrow[leftarrow,crossing over]{ur}{}
      \arrow[leftarrow,crossing over]{ul}{}
      \&
      Z
    \end{tikzcd}
    \right)
    \] 
    holds
    \[
    \kd(X\ot XZ\to Z)
    \leq
    \kd(X\ot XY\to Y)
    +
    \kd(Y\ot YZ\to Z)
    \]
  \end{tlemma}
  
  \begin{Proof}
    By Shannon inequality~(\ref{eq:shannonineq}) on page
    \pageref{eq:shannonineq} we have
    \[ 
    \ent( X \rel Z ) 
    \leq 
    \ent(XY \rel Z) 
    \leq 
    \ent( X \rel Y ) + \ent (Y \rel Z) 
    \] 
    Similarly, 
    \[ 
    \ent(Z \rel X ) 
    \leq 
    \ent(Z \rel Y) + \ent (Y \rel X) 
    \] 
    and therefore 
    \[ 
    \kd(X\ot XZ\to Z) 
    \leq 
    \kd(X\ot XY\to Y) 
    + 
    \kd(Y\ot YZ\to Z)  
    \]
  \end{Proof}

  Now we continue with the proof of
  Proposition~\ref{p:kolmogorovisdistance}.

  Let $\Gbf$ be an arbitrary complete reduction category.
  Suppose $\Xcal = \set{X_i ; f_{ij}}$, $\Ycal = \set{Y_i; g_{ij}}$
  and $\Zcal = \set{Z_i; h_{ij}}$ are $\Gbf$-configurations, with
  initial spaces being $X_{0}$, $Y_{0}$ and $Z_{0}$, respectively. Let
  \begin{align*}
    \hat\Fcal
    &=
    (\Xcal\ot\Fcal\to\Ycal)
    \\
    \hat\Gcal
    &=
    (\Ycal\ot\Gcal\to\Zcal)
  \end{align*}
  be two optimal minimal two-fans
  satisfying
  \begin{align*} 
    \ikd(\Xcal, \Ycal)
    &= 
    \kd(\hat\Fcal) 
    \\ 
    \ikd(\Ycal,\Zcal)
    &= 
    \kd(\hat\Gcal) 
  \end{align*}
   
  Recall that each two-fan of $\Gbf$-configurations is a
  $\Gbf$-configuration of two-fans between the individual spaces, that
  is 
  \[ 
  \begin{split} 
    \Fcal 
    &= 
    \set{\rule{0em}{2ex} \Fcal_i = (X_i \ot F_i \to Y_i)} 
    \\ 
    \Gcal 
    &= 
    \set{ \rule{0em}{2ex} \Gcal_i = (Y_i \ot G_i \to Z_i)} 
  \end{split} 
  \] 

  We construct a coupling $\Hcal$ between $\Xcal$ and $\Zcal$ in the following
  manner. Starting with the two-tents configuration between the
  initial spaces, we use adhesion to extend it to a full
  configuration, thus constructing a coupling between $X_{0}$ and
  $Z_{0}$. This full configuration could then be ``pushed down'' and
  provides full extensions of two-tents on all lower levels. Thus we
  could ``compose'' couplings $\Fcal$ and $\Gcal$ and use a Shannon
  inequality to establish the triangle inequality for the Kolmogorov
  distance. Details are as follows. 
  
  Consider a two-tents configuration
  \[
  X_{0}\ot F_{0}\to Y_{0}\ot G_{0}\to Z_{0}
  \]
  and extend it by adhesion, as described in Section
  \ref{s:config-adhesion} to a $\Lambdabf_{3}$-configuration
  \[
  \begin{tikzcd}[row sep=small, column sep=small, ampersand replacement=\&]
    \mbox{}
    \&
    A_{0}
    \arrow{dl}{}
    \arrow{d}{}
    \arrow{dr}{}
    \\
    F_{0}
    \arrow{d}
    \&
    H_{0}
    \arrow{dl}
    \arrow{dr}  
    \&
    G_{0}
    \arrow{d}
    \\
    X_{0}
    \&
    Y_{0}
    \arrow[leftarrow, crossing over]{ul}
    \arrow[leftarrow, crossing over]{ur}
    \&
    Z_{0}
  \end{tikzcd}
  \]
  Together with the reductions 
  \begin{align*}
  (X_{0})^{\Gbf}
    &\to 
    \Xcal
    \\
  (Y_{0})^{\Gbf}
    &\to 
    \Ycal
    \\
  (Z_{0})^{\Gbf}
    &\to 
    \Zcal
  \end{align*}
  it gives rise to a $\Lambdabf_{3}$-configuration of
  $\Gbf$-configurations
  \[\tageq{bigfull}
  \begin{tikzcd}[row sep=small, column sep=small, ampersand replacement=\&]
    \mbox{}
    \&
    (A_{0})^{\Gbf}
    \arrow{dl}{}
    \arrow{d}{}
    \arrow{dr}{}
    \\
    (F_{0})^{\Gbf}
    \arrow{d}
    \&
    (H_{0})^{\Gbf}
    \arrow{dl}
    \arrow{dr}  
    \&
    (G_{0})^{\Gbf}
    \arrow{d}
    \\
    \Xcal
    \&
    \Ycal
    \arrow[leftarrow, crossing over]{ul}
    \arrow[leftarrow, crossing over]{ur}
    \&
    \Zcal
  \end{tikzcd}
  \]

  Note that the minimal reductions of the two-fan subconfigurations of
  (\ref{eq:bigfull})
  \begin{align*}
    \Xcal\ot(F_{0})^{\Gbf}\to\Ycal\\
    \Ycal\ot(G_{0})^{\Gbf}\to\Zcal
  \end{align*}
  are the two-fans $\hat\Fcal$ and $\hat\Gcal$, respectively, by
  Lemma~\ref{p:minimalfansconfig}.

  Now consider the ``minimization'' of the above configuration
  \[\tageq{minfull}
  \begin{tikzcd}[row sep=small,column sep=small, ampersand replacement=\&]
    \mbox{}
    \&
    \Acal
    \arrow{dl}{}
    \arrow{d}{}
    \arrow{dr}{}
    \\
    \Fcal
    \arrow{d}
    \&
    \Hcal
    \arrow{dl}
    \arrow{dr}  
    \&
    \Gcal
    \arrow{d}
    \\
    \Xcal
    \&
    \Ycal
    \arrow[leftarrow, crossing over]{ul}
    \arrow[leftarrow, crossing over]{ur}
    \&
    \Zcal
  \end{tikzcd}
  \]
  It could also be viewed as a $\Gbf$-configuration of $\Lambdabf_{3}$
  configurations, 
  \[
  \begin{tikzcd}[row sep=small, column sep=small, ampersand replacement=\&]
    \mbox{}
    \&
    A_{i}
    \arrow{dl}{}
    \arrow{d}{}
    \arrow{dr}{}
    \\
    F_{i}
    \arrow{d}
    \&
    H_{i}
    \arrow{dl}
    \arrow{dr}  
    \&
    G_{i}
    \arrow{d}
    \\
    X_{i}
    \&
    Y_{i}
    \arrow[leftarrow, crossing over]{ul}
    \arrow[leftarrow, crossing over]{ur}
    \&
    Z_{i}
  \end{tikzcd}  
  \]
  each of which is minimal by Corollary \ref{p:minimalfullconfig}.
  
  Now we can apply Lemma~\ref{p:kdtriangle}
  to each level to conclude that
  \begin{align*}
    \ikd(\Xcal,\Zcal)
    &\leq
    \kd(\Xcal\ot\Hcal\to\Zcal)
    \\
    &\leq
    \kd(\Xcal\ot\Fcal\to\Ycal)
    +
    \kd(\Xcal\ot\Gcal\to\Ycal)
    \\
    &=
    \ikd(\Xcal,\Ycal)
    +
    \ikd(\Ycal,\Zcal)
  \end{align*}

  Finally, if $k(\Xcal, \Ycal) = 0$, then there is a two-fan $\Fcal$
  of $\Gbf$-configurations between $\Xcal$ and $\Ycal$ with
  $\kd(\Fcal)= 0$, from which it follows that $\Xcal$ and $\Ycal$ are
  isomorphic.
\end{Proof}


%%%%%%%%%%%%%%%%%%%%%%%%%%%%%%%%%%%%%%%%%%%%%%%%%%%%%%%%%%%%%%%%%%%%%%%%%%%%%%%%
\repeatclaim{p:tensor1lip}
\begin{Proof}
  The claim follows easily from the additivity of
  entropy in equation (\ref{eq:entropyadditive}). Suppose that
  $\Xcal=\set{X_{i};f_{ij}}$, $\Ycal=\set{Y_{i};g_{ij}}$ and
  $\Ycal'=\set{Y'_{i};g'_{ij}}$ are three $\Gbf$-configurations and
  \[
  \Fcal
  =
  (\Ycal\ot \Zcal\to \Ycal')
  \]
  is an optimal fan, so that
  \[
  \ikd(\Ycal,\Ycal')
  =
  \sum_{i} \big[2\ent(Z_{i})-\ent(Y_{i})-\ent(Y'_{i})\big]
  \]
  Consider the fan
  \[
  \Gcal
  =
  (\Xcal\otimes \Ycal\ot \Xcal\otimes \Zcal\to \Xcal\otimes \Ycal')
  \]
  Then, by additivity of entropy, in equation (\ref{eq:entropyadditive}), we have
  \begin{align*}
    \kd(\Fcal)
    &=
    \sum_{i}\big[2\ent(X_{i}\otimes Z_{i})-\ent(X_{i}\otimes Y_{i}) -
      \ent(X_{i}\otimes Y'_{i})\big]
    \\
    &=
    \sum_{i}\big[2\ent(Z_{i})-\ent(Y_{i}) -
                 \ent(Y'_{i})\big]
    \\
    &=
    \kd(\Gcal)
  \end{align*}
  and, therefore,
  \begin{align*}
    \ikd(\Xcal\otimes \Ycal,\Xcal\otimes \Ycal')
    &\leq
    \kd(\Gcal)=\kd(\Fcal)=\ikd(\Ycal,\Ycal')
  \end{align*}
  Thus, the tensor product of probability spaces is 1-Lipschitz with
  respect to each argument.
\end{Proof}
  
%%%%%%%%%%%%%%%%%%%%%%%%%%%%%%%%%%%%%%%%%%%%%%%%%%%%%%%%%%%%%%%%%%%%%%%%%%%%%%%%
\repeatclaim{p:entropy1lip}
\begin{Proof}

  Let $\Xcal,\Ycal\in\prob\<\Gbf>$ and let
  \[
  \Gcal=(\Xcal\ot\Zcal\to\Ycal)
  \]
  be an optimal fan with components
  \[
  \Gcal_{i}=(X_{i}\ot Z_{i}\to Y_{i})
  \]
  
  For a fixed index $i$ we can estimate
  the difference of entropies
  \[
    \ent(X_{i})-\ent(Y_{i})
    =
    2\big(\ent(X_{i})-\ent(Z_{i})\big) + \kd(\Gcal_{i})
    \leq
    \kd(\Gcal_{i})
  \]
  By symmetry we then have
  \[
  |\ent(X_{i})-\ent(Y_{i})|
  \leq
  \kd(\Gcal_{i})    
  \]

  Adding above inequalities for all $i$ we have
  \[
  |\ent_{*}(\Xcal)-\ent_{*}(\Ycal)|_{1}
  \leq
  \kd(\Gcal)=\ikd(\Xcal,\Ycal)
  \]
  By the additivity of entropy we also obtain the $1$-Lipschitz
  property of the entropy function with respect to the asymptotic
  Kolmogorov distance $\aikd$.
\end{Proof}






%%%%%%%%%%%%%%%%%%%%%%%%%%%%%%%%%%%%%%%%%%%%%%%%%%%%%%%%%%%%%%%%%%%%%%%%%%%%%%%%
\repeatclaim{p:restriction1lip}
\begin{Proof}
  The claim  follows from the functoriality of the restriction
  operator. We argue as follows.

  Suppose that $R:\Gbf'\to\Gbf$ is a functor and $R^{*}$ is the
  corresponding restriction operator. For
  $\Xcal_{1},\Xcal_{2}\in\prob\<\Gbf>$ let
  \[
  \Fcal=(\Xcal_{1}\ot\Ycal\to\Xcal_{2})
  \]
  be an optimal fan. Then
  \[
  \Fcal':=(R^{*}\Xcal_{1}\ot R^{*}\Ycal\to R^{*}\Xcal_{2})
  \]
  is a fan with the terminal vertices being the restrictions of
  $\Xcal_{1}$ and $\Xcal_{2}$. It can be considered as a
  $\Gbf'$-configuration of two-fans over individual spaces in
  $R^{*}\Xcal_{1}$ and $R^{*}\Xcal_{2}$ each of which also appears as
  a fan in $\Fcal$. 
  
  Thus, we obtain the rough estimate
  \[
  \ikd(R^{*}\Xcal_{1},R^{*}\Xcal_{2})
  \leq
  \size{\Gbf'}\cdot\ikd(\Xcal_{1},\Xcal_{2})
  \]
  Since the restriction operator commutes with tensor powers, the same
  estimate also holds for the asymptotic Kolmogorov distance $\aikd$.
\end{Proof}

%%%%%%%%%%%%%%%%%%%%%%%%%%%%%%%%%%%%%%%%%%%%%%%%%%%%%%%%%%%%%%%%%%%%%%%%%%%%%%%%

\repeatclaim{p:slicing}
\begin{Proof}
  Since the two-fan $(U\ot W\to V)$ is minimal the probability space
  $W$ could be considered having underlying set to be a subset of the
  Cartesian product of the underlying sets of $U$ and $V$. For any
  pair $(u,v)\in \un W$ with a positive weight consider an optimal
  two-fan
  \[\tageq{opt4slices}
  \Gcal_{uv}
  =
  (\Xcal\rel u
  \stackrel{\pi_{\Xcal}}{\oot}
  \Zcal_{uv}
  \stackrel{\pi_{Y}}{\too}
  \Ycal\rel v)
  \]
  where $\Zcal_{uv}=\set{Z_{uv,i};\rho_{ij}}$.  Let $p_{uv,i}$ be the
  probability distributions on $Z_{uv,i}$ -- the individual spaces in
  the configuration $\Zcal_{uv}$.  The next step is to take a convex
  combination of distributions $p_{uv,i}$ weighted by $p_{W}$ to
  construct a coupling $\Xcal\ot\Zcal\to\Ycal$.
  
  First we extend the 7-vertex configuration to a full
  $\Lambdabf_{4}$-configuration of $\Gbf$-configurations, such
  that the top vertex has the distribution
  \[\tageq{definitiondbinslicing}
  p_{i}(x,y,u,v):=p_{uv,i}(x,y)p_{W}(u,v)
  \]
  as described in the Section~\ref{s:config-examples-full}. 
  
  If we integrate over $y$, we obtain
  \[
  \sum_{y} p_{i}(x,u,v,y) = ((\pi_{\Xcal,i})_{*}p_{uv,i})(x) p_{W}(u,v)
  \]
  Then we use that by (\ref{eq:opt4slices}) it holds that $
  (\pi_{\Xcal,i})_{*}p_{uv,i}=p_{X_i}(\,\cdot\,\rel u)$ and therefore
  \[
  \sum_y p_i(x,y,u,v) = p_{X_i}(x \rel u) p_{W}(u,v).
  \]
  In the same way,
  \[
  \sum_x p_i(x,y,u,v) = p_{Y_i}(y \rel v) p_{W}(u,v).
  \]
  Note that this exactly corresponds to adhesion, as described in
  Section \ref{s:config-adhesion}.  It follows that
  \[\tageq{reluvisrelu}
  \Xcal\rel uv = \Xcal \rel u 
  \quad \text{and} \quad 
  \Ycal\rel uv = \Ycal \rel v
  \]
  and
  \[\tageq{slicingentad}
  \ent(X_{i}\rel UV)=\ent(X_{i}\rel U)
  \quad \text{and} \quad 
  \ent(Y_{i}\rel UV)=\ent(Y_{i}\rel V)
  \]
  
  The extended configuration contains a two-fan of configurations 
  $\Fcal=(\Xcal\ot\Zcal\to\Ycal)$ with
  terminal vertices $\Xcal$ and $\Ycal$. We call its initial vertex
  $\Zcal=\set{XY_{i},f_{ij}}$.
  
  The following estimates conclude the proof the the Slicing Lemma.
  First we use the definitions of intrinsic Kolmogorov distance $\ikd$
  and of $\kd(\Fcal)$ to estimate
  \begin{align*}
    \ikd(\Xcal,\Ycal)
    &\leq
    \kd(\Fcal)\\
    &=
    \sum_{i}\kd(\Fcal_{i})\\
    &=
    \sum_{i}\big[
      2\ent(XY_{i}) - \ent(X_{i}) - \ent(Y_{i})
      \big]
  \end{align*}
  Next, we apply the definition of the conditional entropy to rewrite
  the right-hand side
  \begin{align*}
    \ikd(\Xcal, \Ycal)
    &\leq  
    \sum_{i}\big[
      2\ent(XY_{i}\rel UV)+2\ent(UV)-2\ent(UV\rel XY_{i})\\
      &\quad\quad\;
      -\ent(X_{i}\rel U)-\ent(U)+\ent(U\rel X_{i})\\
      &\quad\quad\;
      -\ent(Y_{i}\rel V)-\ent(V)+\ent(V\rel Y_{i})\big]
  \end{align*}
  We now use (\ref{eq:slicingentad}) and rearrange terms to obtain
  \begin{align*}
    \ikd(\Xcal, \Ycal)
    &\leq
    \sum_{i}\big[
      2\ent(XY_{i}\rel UV)-\ent(X_{i}\rel UV)-\ent(Y_{i}\rel UV)\\ 
      &\quad\quad\;
      + 2\ent(UV) - \ent(U) - \ent(V)\\
      &\quad\quad\;
      - 2\ent(UV\rel XY_{i})
      +\ent(U\rel X_{i})
      +\ent(V\rel Y_{i})\big]
  \end{align*}
  By the integral formula for conditional
  entropy (\ref{eq:relentinteg}) applied to the first
  three terms we get
  \begin{align*}
    \sum_i 
    \big[
      2\ent(XY_{i}\rel UV)&-\ent(X_{i}\rel UV)-\ent(Y_{i}\rel UV)
    \big]
    \\
    &=
    \int_{UV}\ikd(\Xcal\rel{uv},\Ycal\rel uv)\d p_{W}(u,v)
  \end{align*}
  However, because of (\ref{eq:reluvisrelu}) this simplifies to
\[
\int_{UV}\ikd(\Xcal\rel{uv},\Ycal\rel uv)\d p_{W}(u,v)
= \int_{UV}\ikd(\Xcal\rel{u},\Ycal\rel v)\d p_{W}(u,v)
\]
Therefore,
  \begin{align*}
  \ikd(\Xcal, \Ycal)
  &\leq \int_{UV}\ikd(\Xcal\rel{u},\Ycal\rel v)\d p_{W}(u,v)+
  \size{\Gbf}\cdot\kd(U \ot W \to V) \\
  &\quad+ 
  \sum_{i}\big[\ent(U\rel X_{i})+\ent(V\rel Y_{i})\big]
  \end{align*}
  
\end{Proof}
  

%%%%%%%%%%%%%%%%%%%%%%%%%%%%%%%%%%%%%%%%%%%%%%%%%%%%%%%%%%%%%%%%%%%%%%%%%%%%%%%%

\repeatclaim{p:kolmogorovlocal}
\begin{Proof}
  We will need the following obvious rough estimate of the Kolmogorov
  distance that holds for any $p,q\in\Delta\Scal$:
  \[\tageq{roughlocalestimate}
  2 \cdot \ikd(\Xcal,\Ycal)\leq 2\size{\Gbf}\cdot\ln|S_{0}|
  \]
  It can be obtained by taking a tensor product for the coupling
  between $\Xcal$ and $\Ycal$.

  Our goal now is to write $p$ and $q$ as the convex combination of
  three other distributions $\hat p$, $p^{+}$ and $q^{+}$ as in
  \begin{align*}
    p
    &=
    (1-\alpha)\cdot\hat p + \alpha\cdot p^{+}
    \\
    q
    &=
    (1-\alpha)\cdot\hat p + \alpha\cdot q^{+}
  \end{align*}
  with the smallest possible $\alpha\in[0,1]$.

  We could do it the following way.  Let
  $\alpha:=\frac12|p_{0}-q_{0}|$. If $\alpha=1$ then the proposition
  follows from the rough estimate~(\ref{eq:roughlocalestimate}), so
  from now on we assume that $\alpha<1$.  Define three probability
  distributions $\hat p_{0}$, $p_{0}^{+}$ and $q_{0}^{+}$ on $S_{0}$
  by setting for every $x\in S_{0}$
  \begin{align*}
    \hat p_{0}(x) 
    &:= \frac1{1-\alpha}
    \min\set{p_{0}(x),q_{0}(x)}
    \\ 
    p_{0}^{+} 
    &:=
    \frac{1}{\alpha}\big(p_{0}-(1-\alpha)\hat p_{0}\big)
    \\ 
    q_{0}^{+} 
    &:= 
    \frac{1}{\alpha}\big(q_{0}-(1-\alpha)\hat p_{0}\big)
  \end{align*}
  
  Denote by $\hat p,p^{+},q^{+}\in\Delta\Scal$ the distributions
  corresponding to $\hat p_{0},p_{0}^{+},q_{0}^{+}\in\Delta S_{0}$
  under isomorphism~(\ref{eq:distribonconfig}). Thus we have
  \begin{align*}
    p&=(1-\alpha)\hat p+\alpha\cdot p^{+}\\
    q&=(1-\alpha)\hat p+\alpha\cdot q^{+}
  \end{align*}

  Now we construct a ``two-tents'' configuration of
  $\Gbf$-configurations 
  \[\tageq{2tentsforlocal}
  \Xcal\ot\tilde\Xcal\to\Lambda_{\alpha}\ot\tilde\Ycal\to\Ycal
  \]
  by setting 
  \begin{align*}
    \tilde X_{i}
    &:=
    \Big(S_{i}\times\un\Lambda_{\alpha};\;
         \tilde p_{i}(s,\square)=(1-\alpha)\hat p_{i}(s),\,
         \tilde p_{i}(s,\blacksquare)=\alpha\cdot p^{+}_{i}(s)
    \Big)
    \\
    \tilde Y_{i}
    &:=
    \Big(S_{i}\times\un\Lambda_{\alpha};\;
         \tilde q_{i}(s,\square)=(1-\alpha)\hat p_{i}(s),\,
         \tilde q_{i}(s,\blacksquare)=\alpha\cdot q^{+}_{i}(s)
    \Big)
    \\
  \end{align*}
  and
  \begin{align*}
    \tilde\Xcal
    &:=
    \set{\tilde X_{i};\,f_{ij}\times\Id}\\
    \tilde\Ycal
    &:=
    \set{\tilde Y_{i};\,f_{ij}\times\Id}\\
  \end{align*}
  The reductions in the ``two-tents'' sub-configurations
  of~(\ref{eq:2tentsforlocal}) are given by coordinate projections.  Note
  that the following isomorphisms hold
  \begin{align*}
    \Xcal\rel\square
    &\cong
    (\Scal,\hat p)
    \\
    \Xcal\rel\blacksquare
    &\cong
    (\Scal, p^{+})
    \\
    \Ycal\rel\square
    &\cong
    (\Scal, \hat p)
    \cong
    \Xcal\rel\square
    \\
    \Ycal\rel\blacksquare
    &\cong
    (\Scal,q^{+})
  \end{align*}

  Now we apply part (\ref{i:slicing2tents}) of Corollary
  \ref{p:slicingcorollary} to obtain the desired inequality
  \begin{align*}
    \ikd(\Xcal,\Ycal)
    &\leq
    (1-\alpha)\ikd(\Xcal\rel\square,\Ycal\rel\square)
    +
    \alpha\cdot\ikd(\Xcal\rel\blacksquare,\Ycal\rel\blacksquare)
    \\
    &\quad
    +\sum_{i}\big[\ent(\Lambda_{\alpha}\rel X_{i})
      +\ent(\Lambda_{\alpha}\rel Y_{i})\big]
    \\
    &\leq
    2\cdot\size{\Gbf}\cdot
    \big(\alpha\cdot\ln|S_{0}|+\ent(\Lambda_{\alpha})\big)
  \end{align*}
\end{Proof}


%%%%%%%%%%%%%%%%%%%%%%%%%%%%%%%%%%%%%%%%%%%%%%%%%%%%%%%%%%%%%%%%%%%%%%%%%%%%%%%%
\def\thesubsection{\thesection.\ref{s:extensions}}
\subsection{Statements from the section ``Extensions''}
\repeatclaim{p:extensionlemma}
\begin{Proof}
  Denote by $X_{0}$, $X_{0}'$ the initial spaces in the
  configurations $\Xcal$, $\Xcal'$, respectively.
  Let $Y_{0}$ be the initial space of the full sub-configuration of
  $\Ycal$ generated by $Y_{k+1},\ldots,Y_{l}$. Let $K_{0}$ be the
  initial space in the optimal coupling between $\Xcal$ and $\Xcal'$
  \[
  \Fcal=(\Xcal'\ot\Kcal\to\Xcal)
  \]
  Recall that $X_{0}$ could be considered as the Cartesian product of
  the underlying sets of spaces generating $\Xcal$ with some
  distribution on it. A similar view holds for $X_{0}'$, $Y_{0}$ and
  $K_{0}$.
  Thus we have in particular $\un K_{0}=\un X_{0}'\times\un X_{0}$
  
  Define a full minimal configuration
  $\Zcal\in\prob\<\Lambdabf_{2k+l}>$ by providing a distribution on
  \[
  \un X_{0}'\times\un X_{0}\times\un Y_{0}=\un K_{0}\times\un Y_{0}
  \]
  as explained in the Section \ref{s:config-examples-full}.
  The distribution will be defined by
  \[
  p(\xbf',\xbf,\ybf)
  :=
  p_{\Fcal}(\xbf',\xbf)\cdot p_{\Ycal}(\xbf,\ybf)/p_{\Xcal'}(\xbf)
  \]
  
  It is clear that $\Zcal$ contains both the coupling $\Fcal$ and
  configuration $\Ycal$ as restrictions. It also contains the minimal full
  configuration 
  \[
  \Ycal'=\<X_{1}',\ldots,X_{k}',Y_{k+1},\ldots,Y_{l}>
  \]
  and a coupling $\Gcal$ between $\Ycal$ and $\Ycal'$. 
  
  For a pair of spaces $A$ and $B$ in $\Zcal$ we denote by $AB$
  the initial space of a minimal fan in $\Zcal$ with the terminal
  spaces $A$ and $B$. 
  The two-fan of $\Lambdabf_{k+l}$-con\-fi\-gu\-rations $\Gcal$ can be considered as a $\Lambdabf_{k+l}$-configuration
  of two-fans
  \[
  \Gcal_{IJ}:=(X_{I}Y_{J}\oot G_{IJ}\too X'_{I}Y_{J})
  \]
  Using this notation we estimate for $I\subset\set{1,\ldots,k}$ and
  	$J\subset\set{k+1,\ldots,l}$
  \begin{align*}
    \ikd(\Ycal,\Ycal')
    &\leq
    \kd{\Gcal}
    =
    \sum_{I,J}\kd(\Gcal_{IJ})
    \\
    &=
    \sum_{I,J}
      \big[
        2\ent(G_{IJ})-\ent(X_{I}Y_{J})-\ent(X_{I}'Y_{J})
      \big]
    \\
    &\leq
    \sum_{I,J}
    \Big[
      2\ent(X_{I}X'_{I})-\ent(X_{I})-\ent(X_{I}')+\\
      &\quad\quad+
      \big(
        2\ent(Y_{J}\rel X_{I}X_{I}')-\ent(Y_{J}\rel
        X_{I})-\ent(Y_{J}\rel X_{I}')
      \big)
    \Big]
    \\
    &\leq
    2^{k-l}
    \sum_{I}\kd(\Fcal_{I})
    \\
    &\leq
    2^{k-l}\kd(\Fcal)
    =
    2^{k-l}\ikd(\Xcal,\Xcal')
  \end{align*}
\end{Proof}
  

%%%%%%%%%%%%%%%%%%%%%%%%%%%%%%%%%%%%%%%%%%%%%%%%%%%%%%%%%%%%%%%%%%%%%%%%%%%%%%%%
\repeatclaim{p:stabilized-rel-ent-set-Lipschitz}
\begin{Proof}
  Note that by Corollary \ref{p:unstabilized-rel-ent-set-Lipschitz},
  or more directly by Proposition \ref{p:extensionlemma}, for $n \in
  \Nbb$
  \[
  \frac1n \d_H 
  \left(
    \res_l\big((\Xcal)^{\otimes n}\big), 
    \res_l \big((\Xcal')^{\otimes n}\big)
  \right)
  \leq 
  2^{l-k}\frac1n \ikd \big( (\Xcal)^{\otimes n}, (\Xcal')^{\otimes n} \big)
  \]	
  Hence, by the scaling properties of the Hausdorff distance
  \[
  \d_H 
  \left(
    \frac{1}{n}\res_l\big((\Xcal)^{\otimes n}\big), 
    \frac{1}{n}\res_l \big((\Xcal')^{\otimes n}\big)
  \right)
  \leq 
  2^{l-k} \frac1n \ikd \big( (\Xcal)^{\otimes n}, (\Xcal')^{\otimes n} \big)
  \]
  For convenience, we introduce the notation
  \begin{align*}
    K_n 
    &= 
    \closure\left( \frac{1}{n}\res_l\big((\Xcal)^{\otimes n}\big) \right) 
    & 
    K
    &= 
    \res_l(\Xcal) 
    \\
    K_n' 
    &= 
    \closure
    \left( 
       \frac{1}{n}\res_l\big((\Xcal')^{\otimes n}\big) 
    \right) 
    &
    K'
    &= 
    \sres_l(\Xcal')
  \end{align*}
  Recall that by definition,
  \[
  K 
  = 
  \closure \left(\bigcup_{n \in \Nbb} K_n\right) 
  \qquad 
  K'  
  = 
  \closure \left( \bigcup_{n \in \Nbb} K_n'\right)
  \]

  Note that by the superadditivity property of the unstabilized
  relative entropic sets (see inclusion
  (\ref{eq:superadditivity-res})) the sequences $n \mapsto K_{n!}$ and
  $n \mapsto K_{n!}'$ are monotonically increasing sequences of sets,
  and
  \[
  \bigcup_{i =1}^n K_i \subset K_{n!} 
  \qquad 
  \bigcup_{i =1}^n K'_i \subset K'_{n!}
  \]

  Now select a large radius $R > 0$. Let $B_R(0)$ denote the ball of
  radius $R$ around the origin in $\Rbb^{2^{\{1,\ldots,l\}}}$.  By
  compactness and the definition of the stabilized relative entropic
  set
  \begin{align*}
    \d_H( K_{n!} \cap B_R(0) , K \cap B_R(0) ) 
    &\to 
    0
    \\
    \d_H( K_{n!}' \cap B_R(0), K' \cap B_R(0) ) 
    & \to 
    0
  \end{align*}
  as $n \to \infty$. Therefore also
  \[
  \d_H
  \big( 
    K \cap B_R(0) , 
    K' \cap B_R(0) 
  \big) 
  \leq 
  2^{l-k} \aikd\big( \Xcal, \Xcal' \big)
  \]
  Because this inequality holds for every $R>0$, the estimate in the
  lemma follows.
\end{Proof}


%%%%%%%%%%%%%%%%%%%%%%%%%%%%%%%%%%%%%%%%%%%%%%%%%%%%%%%%%%%%%%%%%%%%%%%%%%%%%%%%
\def\thesubsection{\thesection.\ref{s:mixtures}}
\subsection{Statements from the section ``Mixtures''}
\repeatclaim{p:mixtures-dist1}
\begin{Proof}
  Recall that for the empirical reduction
  \[
  \emp:\Lambda_{1/n}^{\otimes N}\to\Delta\Lambda_{1/n}
  \]
  the quantity $N\cdot\emp(\lambda)(\blacksquare)$ counts the number of
  black squares in the sequence $\lambda$. It is a binomially
  distributed random variable with the mean $N/n$ and variance
  $\frac{N}{n}(1-\frac1n)$.
  
  The first claim is then proven by the following calculation
  \begin{align*}
    \aikd(\Xcal&,\Xcal^{\otimes n}\oplus_{\Lambda_{1/n}}\set{\bullet})
    \\
    &=
    \lim_{N\to\infty}
      \frac1N
        \ikd\left(
          \Xcal^{\otimes N},
          (\Xcal^{\otimes n}\oplus_{\Lambda_{1/n}}\set{\bullet})^{\otimes N}
        \right)
    \\
    &=
    \lim_{N\to\infty}
      \frac1N
        \ikd\left(
          \Xcal^{\otimes N},
          \bigoplus_{\lambda\in\Lambda_{1/n}^{\otimes N}}
              \Xcal^{\otimes n\cdot N\cdot \emp(\lambda)(\blacksquare)}
        \right)
    \\
    &\leq
    \ent(\Lambda_{1/n}) +
    \lim_{N\to\infty}
      \frac1N
      \int_{\lambda\in\Lambda^{\otimes n}_{1/n}}
        \ikd(\Xcal^{\otimes N},
             \Xcal^{\otimes(N\cdot n\cdot \emp(\lambda)(\blacksquare))})
      \d p(\lambda)
    \\
    &\leq
    \ent(\Lambda_{1/n}) +
    |\ent_{*}(\Xcal)|_{1}\cdot
    \lim_{N\to\infty}
      \frac{n}{N}
      \cdot
      \int_{\lambda\in\Lambda_{1/n}^{\otimes N}}
         \big|N/n- N\cdot \emp(\lambda)(\blacksquare)\big|
         \d p(\lambda)
    \\
    &\leq
    \ent(\Lambda_{1/n}) +
    |\ent_{*}(\Xcal)|_{1}\cdot
    \lim_{N\to\infty}
      \frac{n}{N}\cdot\sqrt{N\cdot\frac1n(1-\frac1n)}
    \\
    &=
    \ent(\Lambda_{1/n})
  \end{align*}

  The second claim is proven similarly and the third follows from the
  second and the $1$-Lipschitz property of the tensor
  product. Finally, the fourth follows from
  Corollary~\ref{p:slicingcorollary}(\ref{p:slicingcofan}), by slicing
  both arguments along $\Lambda_{1/n}$.
\end{Proof}

%%%%%%%%%%%%%%%%%%%%%%%%%%%%%%%%%%%%%%%%%%%%%%%%%%%%%%%%%%%%%%%%%%%%%%%%%%%%%%%%
\def\thesubsection{\thesection.\ref{s:tropical}}
\subsection{Statements from the section ``Tropical Probability''}
\begin{tlemma}{p:subadditive}
  Suppose the sequence $\set{a(i)}_{i\in\Nbb_{0}}$ of  real numbers is
  bounded from below and is
  quasi-subadditive, that is there is a constant $C\in\Rbb$ such that
  for any $i,j\in\Nbb_{0}$ holds
  \[
  a(i+j)\leq a(i) + a(j) + C
  \]
  Then the limit
  \[
  \lim_{i\to\infty}\frac1i a(i)
  \]
  exists and is finite.
\end{tlemma}

\begin{Proof}
  The lemma is standard and is sometimes refered to as Fekete's subadditive lemma. We include a proof for the convenience of the reader.
  Assume first that $C=0$.  Then the sequence satisfies $a(k\cdot
  i)\leq k\cdot a(i)$ and in particular $a(i)\leq i\cdot a(1)$.  Let
  $l:=\liminf\frac1i a(i)\in[0,\infty)$.  Choose $\epsilon>0$. Then we can
  find $k\in\Nbb$ such that $\frac1k a(k)\leq l+\epsilon$.  For
  $n\in\Nbb$ let $q,r$ be the quotient and the reminder of the integer
  division of $n$ by $k$, that is
  \[
  n=q\cdot k+r,\quad0\leq r<k
  \]
  Then
  \begin{align*}
    \frac1n a(n)
    &\leq
    \frac1n (q\cdot a(k)+a(r))
    \leq
    \frac1{q\cdot k+r}(q\cdot a(k))+
    \frac1n a(r)
    \leq
    l+\epsilon+\epsilon=l+2\epsilon
  \end{align*}
  The last inequality holds once $n$ is sufficiently large, specifically when
  \[
  n\geq\frac1\epsilon\max_{0\leq i\leq k}\;a(i)
  \]
  Therefore 
  \[
  \lim_{i \to \infty}\frac1i a(i)=l
  \]
  
  Now if $C>0$ then the sequence $b(i):=a(i)+C$ is subadditive and $\frac1i b(i)$
  converges by the previous argument. Thus we have
  \[
  \lim_{i \to \infty} \frac1i b(i)=\lim_{i \to \infty}\frac1i (a(i)+C)=\lim_{i \to \infty}\frac1i a(i)
  \]
\end{Proof}


%%%%%%%%%%%%%%%%%%%%%%%%%%%%%%%%%%%%%%%%%%%%%%%%%%%%%%%%%%%%%%%%%%%%%%%%%%%%%%%%
\repeatclaim{p:adistonql} 
\begin{Proof} 
  Suppose $\bar\gamma_{1}$ and $\bar\gamma_{2}$ are two quasi-linear
  sequences of elements of $\Gamma$, then for any $i,j\in\Nbb_{0}$

  \begin{align*}
    \dist\big(\gamma_{1}(i+j),&\gamma_{2}(i+j)\big)
    \\
    &\leq
    \dist\big(\gamma_{1}(i+j),\gamma_{1}(i)\otimes\gamma_{1}(j)\big) +
    \dist\big(\gamma_{2}(i+j),\gamma_{2}(i)\otimes\gamma_{2}(j)\big) \\
    &\;\;\;+
    \dist\big(\gamma_{1}(i)\otimes\gamma_{1}(j),
            \gamma_{2}(i)\otimes\gamma_{2}(j)\big)\\
            &\leq
            \defect_{\dist}(\bar\gamma_{1}) +
            \defect_{\dist}(\bar\gamma_{2}) +
            \dist\big(\gamma_{1}(i),\gamma_{2}(i)\big) +
            \dist\big(\gamma_{1}(j),\gamma_{2}(j)\big)
  \end{align*}
  
  Thus the sequence $\dist(\gamma_{1}(i),\gamma_{2}(i))$ is
  quasi-subadditive and by Lemma~\ref{p:subadditive} the limit 
  \[
  \lim_{i\to\infty} \frac1i \dist\big(\gamma_{1}(i),\gamma_{2}(i)\big)
  \]
  exists and is finite.
\end{Proof}

%%%%%%%%%%%%%%%%%%%%%%%%%%%%%%%%%%%%%%%%%%%%%%%%%%%%%%%%%%%%%%%%%%%%%%%%%%%%%%%%
\repeatclaim{p:boundeddefect} 
\begin{Proof}
  Given a Cauchy sequence $\set{\bar\gamma_{i}}$ of elements in
  $(\qlin_{\dist}(\Gamma),\dista)$ we need to find a limiting element
  $\bar\phi\in\qlin_{\dist}(\Gamma)$.  We will do that by a version of
  the diagonal process, that is we define $\phi(n)$ to
  have value $\gamma_{i}(n)$ for $i$ sufficiently large depending on
  $n$. The quasi-linearity of $\bar\phi$ would follow from the fact
  that for a fixed $n$ and all sufficiently large $i$ the set
  $\set{\gamma_{i}(n)}$ is uniformly bounded.

  Now we give the detailed argument.  First we replace each element
  of the sequence $\set{\bar\gamma_{i}}$ by an asymptotically
  equivalent element with defect bounded by the constant $C$
  according to assumption~(\ref{pi:boundeddefect}) of the lemma. We
  will still call the new sequence $\set{\bar\gamma_{i}}$. The Cauchy
  sequence $\set{\bar\gamma_{i}}$ satisfies
  \[
  \sup_{i,j\geq\ibf}\dista(\bar\gamma_{i},\bar\gamma_{j})\to0
  \quad\text{as}\quad
  \ibf\to\infty
  \]
    
  By assumption~(\ref{pi:boundeddefect}) of the lemma for any
  $n,k\in\Nbb_{0}$ holds
  \begin{align*}
    k \cdot \dist\big(\gamma_{i}(n),\gamma_{j}(n)\big)
    &=
    \dist\big(\gamma_{i}(n)^{\otimes k},\gamma_{j}(n)^{\otimes k}\big)\\
    &\leq
    \dist\big(\gamma_{i}(kn),\gamma_{j}(kn)\big) + 2 k \cdot C
  \end{align*}
  Dividing by $k$ we obtain
  \[
  \dist(\gamma_{i}(n),\gamma_{j}(n)) \leq \frac1k\dist(\gamma_{i}(kn),\gamma_{j}(kn)) + 2C
  \]
  Now we pass to the limit sending $k$ to infinity, while keeping $n$
  fixed:
  \[
  \dist(\gamma_{i}(n),\gamma_{j}(n))
  \leq
  n\cdot\dista(\bar\gamma_{i},\bar\gamma_{j}) + 2C
  \]
  Given $n$ let $\ibf(n)$ be a number such that for any $i,j\geq\ibf(n)$
  holds
  \[
  \dista(\bar\gamma_{i},\bar\gamma_{j})\leq \frac1n
  \]
  
  We may assume that $\ibf(n)$ is nondecreasing as a function of $n$.
  Then for any $i,j,n\in\Nbb$ with $i,j\geq \ibf(n)$ we have
  the following bound
  \[\tageq{boundedmembers}
    \dist\big(\gamma_{i}(n),\gamma_{j}(n)\big)
  \leq
  2C+1
  \]
  
  Now we are ready to define the limiting sequence $\bar\phi$ by
  setting
  \[
  \phi(n):=\gamma_{\ibf(n)}(n)
  \]
  First we verify that $\bar\phi$ is quasi-linear
  \begin{align*}
    \dist
    &\big(
      \phi(n+m),
      \phi(n)\otimes\phi(m)
    \big)
    =
    \dist
    \big(
      \gamma_{\ibf(n+m)}(n+m),
      \gamma_{\ibf(n)}(n)\otimes\gamma_{\ibf(m)}(m)
    \big)
    \\
    &\leq
    \dist
    \big(
      \gamma_{\ibf(n+m)}(n+m),
      \gamma_{\ibf(n+m)}(n)\otimes\gamma_{\ibf(n+m)}(m)
    \big)
    \\
    &\quad+
    \dist
    \big(
      \gamma_{\ibf(n+m)}(n)\otimes\gamma_{\ibf(n+m)}(m),
      \gamma_{\ibf(n)}(n)\otimes\gamma_{\ibf(m)}(m)
    \big)
    \\
    &\leq  
    C+
    \dist
    \big(
      \gamma_{\ibf(n+m)}(n),
      \gamma_{\ibf(n)}(n)
    \big)
    +
    \dist
    \big(
      \gamma_{\ibf(n+m)}(m),
      \gamma_{\ibf(m)}(m)
    \big)
    \\
    &\leq
    C+2(2C+1)=5C+2=:C'
  \end{align*}
  
  The convergence of $\bar\gamma_{i}$ to $\bar\phi$ is shown as
  follows. For $n,k\in\Nbb$ let $q,r\in\Nbb_{0}$ be the quotient and
  the remainder of the division of $n$ by $k$, that is $n=q\cdot k+r$
  and $0\leq r<k$.  Fix $k\in\Nbb$ and let $i\geq\ibf(k)$, then
  \begin{align*}
    \dista(\bar\gamma_{i},\bar\phi)
    &=
    \lim_{n\to\infty}\frac1n\dist\big(\gamma_{i}(n),\phi(n)\big)\\
    &=
    \lim_{n\to\infty}
    \frac1n
    \dist\big(\gamma_{i}(q\cdot k+r),\gamma_{\ibf(n)}(q\cdot k+r)\big)\\
    &\leq
    \lim_{n\to\infty}
    \frac1n
    \left(\rule{0mm}{5mm}
    q\cdot\dist\big(\gamma_{i}(k),\gamma_{\ibf(n)}(k)\big)+
    \dist\big(\gamma_{i}(r),\gamma_{\ibf(n)}(r)\big) +
    2qC' + 2C'
    \right)\\
    &\leq
    \lim_{n\to\infty}
    \frac1n
    \big((3q+3)\cdot C'\big)\\
    &=
    \frac{3C'}{k}      
  \end{align*}
  Since $k\in\Nbb$ is arbitrary we have 
  \[
  \lim_{i\to\infty}\dista(\bar\gamma_{i},\bar\phi)=0
  \]
\end{Proof}




%%%%%%%%%%%%%%%%%%%%%%%%%%%%%%%%%%%%%%%%%%%%%%%%%%%%%%%%%%%%%%%%%%%%%%%%%%%%%%%%
\repeatclaim{p:eps-linear-dense}
\begin{Proof}
	Let $\bar\gamma=\set{\gamma(n)}$ be a $\dist$-quasi-linear
	sequence. We need to approximate it with linear sequences. For
	$i\in\Nbb$, let
	$\bar\gamma_{i}$ be a sequence asymptotically equivalent to $\gamma$
	and satisfying
	\[
	\defect_{\dist}\bar\gamma_{i}\leq 1/i
	\]
	as provided by the $1/i$-uniformly bounded defect property.
	
	Define a $\dist$-linear sequence $\bar\eta_{i}$ by
	\[
	\eta_{i}(n):=\gamma_{i}(1)^{\otimes n}
	\]
	Then
	\begin{align*}
	\dista(\bar\gamma,\bar\eta_{i})
	&=
	\dista(\bar\gamma_{i},\bar\eta_{i})
	\\
	&=
	\lim_{n\to\infty}
	\frac1n
	\dist(\gamma_{i}(n),\eta_{i}(n))
	\\
	&=
	\lim_{n\to\infty}
	\frac1n
	\dist(\gamma_{i}(n),\gamma_{i}(1)^{\otimes n})
	\\
	&\leq
	\lim_{n\to\infty}
	\frac1n
	\cdot n\cdot\defect_{\dist}(\bar\gamma_{i})
	\\
	&\leq
	\frac1i    
	\end{align*}
	Thus $\lim\bar\eta_{i}=\gamma$.
\end{Proof}



%%%%%%%%%%%%%%%%%%%%%%%%%%%%%%%%%%%%%%%%%%%%%%%%%%%%%%%%%%%%%%%%%%%%%%%%%%%%%%%%

\repeatclaim{p:dist-dista-isometry}


\begin{Proof}
  Let $\bar\gamma_{1},\bar\gamma_{2}\in\qlin_{\dist}(\Gamma)$ be two
  sequences of $\dist$-quasi-linear sequences.  We have to show that the
  two numbers
  \[
  \dista(\bar\gamma_{1},\bar\gamma_{2})
  =
  \lim_{n\to\infty}\frac1n \dist\big(\gamma_{1}(n),\gamma_{2}(n)\big)
  \]
  and
  \[
  \distaa(\bar\gamma_{1},\bar\gamma_{2})
  =
  \lim_{n\to\infty}\frac1n \dista\big(\gamma_{1}(n),\gamma_{2}(n)\big)
  \]
  are equal.  Since shifts are non-expanding maps, we have
  $\dista\leq\dist$ and it follows immediately that
  \[
  \distaa(\bar\gamma_{1},\bar\gamma_{2})
  \leq
  \dista(\bar\gamma_{1},\bar\gamma_{2})
  \]
  and we are left to show the opposite inequality.
  We will do it as follows. Fix $n>0$, then
  \begin{align*}
    \dista(\bar\gamma_{1},\bar\gamma_{2})
    &=
    \lim_{k\to\infty}\frac{1}{kn}\dist\big(\gamma_1(kn),\gamma_{2}(kn)\big)\\
    &\leq
    \lim_{k\to\infty}\frac{1}{kn}
    \bigg(
    \dist\big(\gamma_1(n)^{\otimes k},\gamma_{2}(n)^{\otimes k}\big)
    +
    k\cdot\big(
    \defect_{\dist}(\bar\gamma_{1})
    +
    \defect_{\dist}(\bar\gamma_{2})
    \big)
    \bigg)\\
    &\leq
    \frac1n\dista\big(\gamma_{1}(n),\gamma_{2}(n)\big)
    +
    \frac1n\big(\defect_{\dist}(\bar\gamma_{1})
    +
    \defect_{\dist}(\bar\gamma_2)\big)
  \end{align*}
  Passing to the limit with respect to $n$ gives required inequality
  \[
  \dista(\bar\gamma_{1},\bar\gamma_{2})
  \leq
  \distaa(\bar\gamma_{1},\bar\gamma_{2})
  \]
\end{Proof}



%%%%%%%%%%%%%%%%%%%%%%%%%%%%%%%%%%%%%%%%%%%%%%%%%%%%%%%%%%%%%%%%%%%%%%%%%%%%%%%%

\repeatclaim{p:dist-dista-dense}
\begin{Proof}
	Given an element $\bar\gamma=\set{\gamma(n)}$ in
	$\qlin_{\dista}(\Gamma)$ we have to find a $\distaa$-approxi\-ma\-ting
	sequence $\bar\gamma_{i}=\set{\gamma_{i}(n)}$ in
	$\qlin_{\dist}(\Gamma)$.
	Define 
	\[
	\gamma_{i}(n):=\gamma(i)^{\otimes\lfloor\frac{n}{i}\rfloor}
	\]
	We have to show that each $\bar\gamma_{i}$ is $\dist$-quasi-linear and
	that $\distaa(\bar\gamma_{i},\bar\gamma)\too[i\to\infty]0$.
	These follow from 
	\begin{align*}
	\dista\big(\gamma_i(m+n), \gamma_i(m) \otimes \gamma_i(n) \big)
	&=\dista\left( \gamma(i)^{\otimes \lfloor \frac{m+n}{i}\rfloor} , 
	\gamma(i)^{\otimes \lfloor \frac{m}{i}\rfloor } \otimes \gamma(i)^{\otimes \lfloor \frac{n}{i} \rfloor}\right)\\
	&\leq \dista \left( \gamma(i), \bm 1 \right)
	\end{align*} 
	and
	\begin{align*}
	\distaa(\bar\gamma_i,\bar\gamma)
	&= \lim_{n \to \infty} \frac1n\dista\left( \gamma_i(n), \gamma(n) \right) \\
	&= \lim_{n \to \infty} \frac1n \dista \left( \gamma(i)^{\otimes \lfloor\frac{n}{i}\rfloor}, \gamma(n) \right)\\
	&\leq \lim_{n \to \infty} \left[\frac1n \dista \left( \gamma\left(i\lfloor\tfrac{n}{i}\rfloor\right), \gamma(n) \right) + \frac1n \lfloor \tfrac{n}{i} \rfloor \defect_{\dista}(\bar\gamma) \right]\\
	&\leq \lim_{n \to \infty}\left[ \frac1n \max_{k=0,\ldots,i-1} \dista\left(\bm 1, \gamma(k) \right)
	 + \frac{i}{n} \defect_{\dista} (\bar\gamma) \right]
	 + \frac{1}{i} \defect_{\dista}(\bar\gamma)\\
	 &\leq \frac{1}{i} \defect_{\dista}(\bar \gamma)
	\end{align*}
  It is worth noting, that the defect of $\bar\gamma_{i}$ need not to
	be uniformly bounded with respect to $i$.
\end{Proof}






%%%%%%%%%%%%%%%%%%%%%%%%%%%%%%%%%%%%%%%%%%%%%%%%%%%%%%%%%%%%%%%%%%%%%%%%%%%%%%%%
\repeatclaim{p:unifsmalldefectaikd}
\begin{Proof}
  Let $\bar\Xcal=\set{\Xcal(i)}$ be a quasi-linear sequence and let $\epsilon > 0$. We will
  find an asymptotically equivalent sequence with defect less than $\epsilon$.
  
  Define a
  new sequence $\bar\Ycal=\set{\Ycal(i)}$ by
  \[
  \Ycal(i)
  :=
  \big[\Xcal(k\cdot i)\big]
  \oplus_{\Lambda_{1/k}}
  \set{\bullet}
  \]
  where the number $k\in\Nbb$ will be chosen later.  First we verify
  that the sequences $\bar\Xcal$ and $\bar\Ycal$ are asymptotically
  equivalent, that is 
  \begin{align*}
    \hat\aikd(\bar\Xcal,\bar\Ycal)
    &:=
    \lim_{i\to\infty}
    \frac{1}{i}
    \aikd\left(\Xcal(i), 
               \Ycal(i)
         \right)
     =
     0
  \end{align*}
  We estimate the asymptotic distance between individual members of
  sequences $\bar\Xcal$ and $\bar\Ycal$ using
  Lemma~\ref{p:mixtures-dist1} as
  follows
  \begin{align*}
    \aikd(&\Xcal(i), 
               \Ycal(i)
         )
    =
    \aikd\big(\Xcal(i), 
               \Xcal(k\cdot i)
               \oplus_{\Lambda_{1/k}}\set{\bullet}
         \big)    
    \\
    &\leq
    \aikd\left(\Xcal(i), 
               \Xcal(i)^{\otimes k}
               \oplus_{\Lambda_{1/k}}\set{\bullet}
         \right)
    +    
    \aikd\left(\Xcal(i)^{\otimes k}
               \oplus_{\Lambda_{1/k}}\set{\bullet}, 
               \Xcal(k\cdot i)
               \oplus_{\Lambda_{1/k}}\set{\bullet}
         \right)    
    \\
    &\leq
    \ent(\Lambda_{1/k})
    +    
    \defect_{\aikd}(\bar\Xcal)   
  \end{align*}
  Thus $\hat\aikd(\bar\Xcal,\bar\Ycal)=0$ and the two sequences are
  asymptotically equivalent.
  
  Next we show that the sequence $\bar\Ycal$ is $\aikd$-quasi-linear and evaluate
  its defect using Lemma~\ref{p:mixtures-dist1}. Let $i,j\in\Nbb$, then
  \begin{align*}
    \aikd&\big(\Ycal(i+j),\Ycal(i)\otimes\Ycal(j)\big)
    \\
    &=
    \aikd\Big(\Xcal(k\cdot i+k\cdot j)
                \oplus_{\Lambda_{1/k}}\set{\bullet},
              \big[
                \Xcal(k\cdot i)
                \oplus_{\Lambda_{1/k}}\set{\bullet}
              \big]
              \otimes
              \big[
                \Xcal(k\cdot j)
                \oplus_{\Lambda_{1/k}}\set{\bullet}
              \big]
         \Big)    
    \\
    &\leq
    \aikd\Big(\big[\Xcal(k\cdot i)\!\otimes\!\Xcal(k\cdot j)\big]
                   \oplus_{\Lambda_{1/k}}\!\set{\bullet},     
              \big[
                \Xcal(k\cdot i)
                \oplus_{\Lambda_{1/k}}\!\set{\bullet}
              \big]
              \otimes
              \big[
                \Xcal(k\cdot j)
                \oplus_{\Lambda_{1/k}}\!\set{\bullet}
              \big]
         \Big)    
     \\
     &\quad+
     \frac1k \defect(\bar\Xcal)
     \\
     &\leq
     3\ent(\Lambda_{1/k})+\frac1k \defect_{\aikd}(\bar\Xcal)
  \end{align*}
  Thus, by choosing $k$ to be a solution to the inequality
  \[
  3\ent(\Lambda_{1/k})+\frac1k \defect_{\aikd}(\bar\Xcal)\leq\epsilon
  \]
  we can make sure that 
  \[
  \defect_{\aikd}(\bar\Ycal)\leq\epsilon
  \]
\end{Proof}





%%%%%%%%%%%%%%%%%%%%%%%%%%%%%%%%%%%%%%%%%%%%%%%%%%%%%%%%%%%%%%%%%%%%%%%%%%%%%%%%
\repeatclaim{p:ikduniform}
\begin{Proof} 
  Consider a two-fan $U_{n}\stackrel{f}{\ot} U_{nm}\stackrel{g}{\to}
  U_{m}$. To construct specific reductions $f$ and $g$ we identify
  $U_{nm}$, $U_{n}$ and $U_{m}$ with the cyclic groups of the
  corresponding order
\begin{align*}
  U_{nm}&\leftrightarrow \Zbb_{nm}\\
  U_{n}&\leftrightarrow\Zbb_{n}\\
  U_{m}&\leftrightarrow\Zbb_{m}
\end{align*}

Consider the short exact sequences
\begin{align*}
  \set{0} \too \Zbb_{n}
  &\stackrel{\times m}{\too}
  \Zbb_{nm}
  \stackrel{\mod m}{\too}
  \Zbb_{m} \too \set{0}\\
  \set{0} \too \Zbb_{m}
  & \stackrel{\times n}{\too}
  \Zbb_{nm}
  \stackrel{\mod n}{\too}
  \Zbb_{n}\too \set{0}\\  
\end{align*}
Choose for $f$ the left splitting in the first exact sequence, and for $g$ the left splitting in the second exact sequence.

Now that we constructed a two-fan $U_{n}\stackrel{f}{\ot} U_{nm}\stackrel{g}{\to}
U_{m}$, let $U_{n}\ot Z\to U_{m}$ be its minimal reduction. Now we estimate
$|Z|\leq n+m$, which implies that
\begin{align*}
  \ikd(U_{n},U_{m})
  &\leq
  2\Ent(Z)-\ent(U_{n})-\ent(U_{m})\\
  &\leq
  2\ln(n+m)-\ln n-\ln m\\
  &\leq
  2\ln2+2\ln\max\set{n,m}-\ln n-\ln m\\
  &=2\ln2+\left|\ln\frac{n}{m}\right|
\end{align*}

To prove the second assertion note that entropy is a
$\ikd$-1-Lipschitz function. Therefore we have
\[
|\ent(U_{n})-\ent(U_{m})|\leq\ikd(U_{n},U_{m})
\leq
|\ent(U_{n})-\ent(U_{m})|+2\ln2
\]
Substituting in the definition of asymptotic Kolmogorov distance we
obtain the required equality.
\end{Proof}

%%%%%%%%%%%%%%%%%%%%%%%%%%%%%%%%%%%%%%%%%%%%%%%%%%%%%%%%%%%%%%%%%%%%%%%%%%%%%%%%


\let\thesubsection=\thesubsectionstandard
