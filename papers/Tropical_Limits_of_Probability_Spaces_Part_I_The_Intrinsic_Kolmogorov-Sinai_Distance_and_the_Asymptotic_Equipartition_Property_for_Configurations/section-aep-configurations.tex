  Below we prove that any Bernoulli sequence can be approximated by a
  sequence of homogeneous configurations.  This is essentially the
  \term{Asymptotic Equipartition Theorem for configurations}.
 
  \begin{theorem}{p:aep-complete}
    Suppose $\Xcal\in\prob\<\Gbf>$ is a complete configuration of
    probab ility spaces.  Then there exists a sequence
    $\bar\Hcal=(\Hcal_{n})_{n=0}^{\infty}$ of homogeneous
    configurations of the same c ombinatorial type as $\Xcal$ such
    that
    \[
      \frac{1}{n}\ikd (\Xcal^{\otimes n},\Hcal_{n}) 
      = 
      \O\left( \sqrt{\frac{\ln^3 n}{n}} \right)
    \]  
  
  
  More precisely, the sequence $\bar\Hcal$ may be chosen such that for
  all $n \geq |X_0|$
  \[\tageq{quantaep} 
  \frac{1}{n} 
  \ikd (\Xcal^{\otimes n},\Hcal_{n}) 
  \leq 
  C(|X_0|,\size{\Gbf}) \cdot \sqrt{\frac{\ln^3 n}{n}} 
  \]
  where $C(|X_0|, \size{\Gbf})$ is a constant only depending on
  $|X_0|$ and $\size{\Gbf}$.
\end{theorem}

\begin{Proof}
  Denote by $\Scal = \un \Xcal$ the underlying configuration of sets
  and by $p_\Xcal$ the true distribution on $\Scal$, such that
  \[
  \Xcal = (\Scal, p_X)
  \]

  We will construct the approximating homogeneous sequence by taking
  types over rational approximations of $p_\Xcal$ in $\Delta\Scal$, that
  converge sufficiently fast to the true distribution $p_{\Xcal}$.

  More specifically, we select rational distributions $p_n \in \Deltan
  \Scal$ such that
  \[
  |p_n - p_\Xcal|_{1} \leq \frac{|S_{0}|}{n}
  \]

  As homogeneous spaces $\Hcal_n$ we set $\Hcal_n = T_{p_{n}}^{\c
    n}\Scal$.  We will show that the Kolmogorov distance between
  $\Hcal_n$ and $\Xcal^{\otimes n}$ satisfies the required estimate
  (\ref{eq:quantaep}).

  First we apply slicing along the empirical two-fan 
  \[
  \Rcal_{n}(\Xcal)
  =
  \left(
    \Xcal^{\otimes n}
    \ot
    \tilde\Xcal^{(n)}
    \to
    (\Delta\Scal,\tau_{n})^{\Gbf} 
  \right)
  \]
  defined in Section~\ref{s:disttypes-types-config},
  Equation~(\ref{eq:empirical-2fan}) on
  page~\pageref{eq:empirical-2fan}.

  For the estimate below we use the fact that
  \[
    \ent(\Delta\Scal,\tau_{n})
    \leq
    \ln|\Deltan\Scal|
    \leq |S_0|\cdot\ln (n+1)
  \]

  By slicing (see
  Corollary~\ref{p:slicingcorollary}(\ref{p:slicingtwofan})) along the
  empirical two-fan we have
  \begin{align*}
    \ikd(T^{(n)}_{p_{n}}\Scal,\Xcal^{\otimes n})
    &\leq
    2 \cdot \size{\Gbf}\cdot\ent(\Delta \Scal,\tau_{n})
    +
    \int_{\Delta \Scal}
    \ikd(T^{(n)}_{p_{n}}\Scal,T^{(n)}_{\pi}\Scal)
    \d\tau_{n}(\pi) 
    \\
    &\leq
    2 \cdot \size{\Gbf} \cdot|S_0|\cdot \ln (n + 1)
    +
    \int_{\Delta \Scal}
    \ikd(T^{(n)}_{p_{n}}\Scal,T^{(n)}_{\pi}\Scal)
    \d\tau_{n}(\pi) 
  \end{align*}
  To estimate the integral we split the domain into a small divergence
  ball $B_{\epsilon_n} = B_{\epsilon_n}(p_\Xcal)$ around the ``true''
  distribution and its complement
  \begin{align*}
    \int_{\Delta \Scal}
    \ikd(T^{(n)}_{p_{n}}\Scal,T^{(n)}_{\pi}\Scal)
    \d\tau_{n}(\pi) 
    &=
    \int_{\Delta \Scal\setminus B_{\epsilon_{n}}}
    \ikd(T^{(n)}_{p_{n}}\Scal,T^{(n)}_{\pi}\Scal)
    \d\tau_{n}(\pi)\\
    &\quad +\tageq{aepinball}
    \int_{B_{\epsilon_{n}}}
    \ikd(T^{(n)}_{p_{n}}\Scal,T^{(n)}_{\pi}\Scal)
    \d\tau_{n}(\pi)
  \end{align*}
  and we set the radius $\epsilon_n$ equal to
  \[
  \epsilon_n 
  := 
  (|S_0| + 1) \frac{\ln (n+1)}{n}
  \]
  
  To estimate the first integral on the right-hand side of equality (\ref{eq:aepinball}) note that the distance between two
  types over the same configuration of sets can always be crudely estimated
  by 
  \[
  2 \cdot \ln |S_0| \cdot \size{\Gbf} \cdot n
  \]
  Moreover, by Sanov's theorem, Theorem \ref{p:entropydivergence}, we can estimate the empirical measure of the complement of the divergence ball
  \[
  \tau_{n}(\Delta\Scal\setminus
  B_{\epsilon_{n}}) \leq \ebf^{-n\cdot\epsilon_{n}+|S_0| \cdot \ln (n+1)} \leq \frac{1}{n}
  \] 
  where we used the definition of $\epsilon_n$ to conclude the last inequality.
  Therefore we obtain
  \begin{align*}
    \int_{\Delta \Scal\setminus B_{\epsilon_{n}}}
    \ikd(T^{(n)}_{p_{n}}\Scal,T^{(n)}_{\pi}\Scal) \d\tau_{n}(\pi) 
    &\leq
    2 \cdot\ln |S_0| \cdot \size{\Gbf} \cdot n \cdot
    \tau_{n}(\Delta\Scal\setminus
    B_{\epsilon_{n}})
    \\
    &\leq 
    2 \cdot\ln |S_0| \cdot \size{\Gbf}
  \end{align*}
  
  Define
  \[
  \alpha_n 
  = 
  \frac{|S_0|}{n} + \sqrt{2 \epsilon_n }
  \]
  if the right-hand side is smaller than $1$ and set $\alpha_n=1$
  otherwise.  Then every $\pi \in B_{\epsilon_n}(p_\Xcal)$ satisfies
  $|p_n - \pi| \leq 2 \alpha_n$ by Pinsker's inequality (Lemma
  \ref{p:entropydivergence}, (\ref{p:entropydivergence1})), and the
  triangle inequality.  Consequently, by the estimate on the distance
  between types in Proposition \ref{p:dist-types-complete}
  
  \begin{align*}
  \int_{B_{\epsilon_{n}}}
  &\ikd(T^{(n)}_{p_{n}}\Scal,T^{(n)}_{\pi}\Scal)
  \d\tau_{n}(\pi)
  \\
  &\leq
  2n \cdot
  \size{\Gbf}
  \cdot
  \left( \alpha_n \ln |S_0| + 2 \ent(\Lambda_{\alpha_n}) \right)
    + 4\cdot \size{\Gbf} \cdot |S_{0}| \cdot \ln(n+1)
  \end{align*}
  
  Using the definition of $\alpha_n$ and $\epsilon_n$ we find that 
  \[
    \int_{B_{\epsilon_{n}}}
    \ikd(T^{(n)}_{p_{n}}\Scal,T^{(n)}_{\pi}\Scal)
    \d\tau_{n}(\pi) 
    = 
    \O\left(\sqrt{n \cdot \ln^{3} n} \right)
  \]
  and hence combining the above estimates
  \[
    \frac{1}{n}\ikd(T^{(n)}_{p_{n}}\Scal,\Xcal^{\otimes n}) 
    = 
    \O\left( \sqrt{\frac{\ln^3 n }{n}} \right).
  \]
  A more precise check shows that for $n \geq |S_0|$, the constants
  appearing in $\O$ only depend on $|S_0|$ and $\size{\Gbf}$.
\end{Proof}

\bigskip


  It is worth noting that each type considered as a subspace of the
  tensor power takes up only small probability. In fact its
  probability converges to zero with growing $n$. But as the
  calculation above shows, most (in terms of probability) of the
  configuration $\Xcal^{\otimes n}$ consists of \emph{polynomially
  many} types that are $\ikd$-similar to each other. Relative to the
  exponential growth of sizes of all the parts, ``polynomially many''
  is as good as one.  This is the difference with the setup used in
  Gromov's \cite{Gromov-Search-2012}

