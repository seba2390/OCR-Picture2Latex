  As explained in Section~\ref{s:disttypes-types-config}, given a
  complete $\Gbf$-configuration $\Scal$ of sets and a rational
  distribution $\pi\in\Delta\Scal$ we construct a homogeneous
  configuration $T^{(n)}_{\pi}\Scal$, which is called the type of
  $\Scal$ over $\pi$. Our goal in this section is to estimate the
  Kolmogorov distance between two types over two different
  distributions $\pi_{1},\pi_{2}\in\Deltan\Scal$ in terms of the total
  variation distance $|\pi_{1}-\pi_{2}|_1$.

  For this purpose we use a ``lagging'' technique which is explained
  below.
 
\subsection{The lagging trick}\label{s:lagging-lagging}
  Let $\Lambda_\alpha$ be a binary probability space, 
  \[ 
  \Lambda_\alpha
  = 
  \big( \set{\square,\blacksquare};
        p_{\Lambda_\alpha}(\blacksquare)=\alpha 
  \big) 
  \] 
  and let $\Xcal=\set{(\un X_{i},p_{i});f_{ij}}$, $\Zcal=\set{(\un
    Z_{i},q_{i});g_{ij}}$ be two configurations modeled on a complete
  diagram category $\Gbf$ and included in a minimal two-fan
  \[ 
  \Lambda_\alpha \oot[\lambda] \Zcal \too[\rho] \Xcal 
  \]
  Recall that the left terminal vertex in this two-fan should be
  interpreted as a constant $\Gbf$-configuration
  $\Lambda_\alpha^{\Gbf}$.
  
  Assume further that the distribution $q$ on $\Zcal$ is rational with
  denominator $n\in\Nbb$, that is $q \in\Deltan\un\Zcal$.  It follows that
  $p$ and $p_{\Lambda_\alpha}$ are also rational with the same
  denominator $n$.

  We construct a \term{lagging two-fan} 
  \[\tageq{laggingtwofan}
    \Lcal
    :=
    \big(
      T^{\c{(1-\alpha) n}}(\Xcal\rel\square)
      \oot[l]
      T^{\c{n}}\Zcal
      \too[T\rho]
      T^{\c{n}}\Xcal 
    \big)
  \]
  as follows. 
  The right leg $T\rho$ of $\Lcal$ is induced by the right leg
  $\rho$ of the original two-fan.  The left leg
  \[ 
  l:T^{\c{n}}\Zcal
  \to 
  T^{\c{(1-\alpha) n}}(\Xcal\rel\square)
  \]
  is obtained by erasing symbols that reduce to $\blacksquare$ and
  applying $\rho$ to the remaining symbols.  The target space for the
  reduction $l$ is the true type of $\Xcal \rel \square$ which is
  ``lagging'' behind $T^{(n)} \Zcal$ by a factor of $(1-\alpha)$.
  More specifically, the reduction $l$ is constructed as follows.  Let
  $\lambda_{j}:Z_{j}\to \Lambda_\alpha$ be the components of the
  reduction $\lambda:\Zcal\to\Lambda_\alpha$.  

  Given $\bar z=(z_{i})_{i=1}^{n}\in T^{\c{n}}Z_{j} $ define the
  subset of indexes
  \[
    I_{\bar z}
    :=
    \set{i\st \lambda_{j}(z_{i})=\square }
  \]
  and define the $j^{\text{th}}$ component of $l$ by 
  \[
    l_{j}\big((z_{i})_{i=1}^{n}\big)
    :=
    (\rho(z_{i}))_{i\in I_{\bar z}}
  \]

  By equivariance each $l_{j}$ is a reduction of homogeneous spaces,
  since the inverse image of any point has the same
  cardinality. Moreover the reductions $l_{j}$ commute with the reductions in
  $T^{\c{n}}\Zcal$ as explained in Section \ref{s:disttypes-types-config}
  and therefore $l$ is a reduction of configurations.

  The next lemma uses the lagging two-fan to estimate the
  Kolmogorov distance between its terminal configurations.
 
  \begin{lemma}{p:lagging-kd}
    Let $\Xcal,\Zcal\in\prob\<\Gbf>$ be
    two configurations modeled on a complete diagram category $\Gbf$
    and included in a minimal two-fan
    \[
      \Lambda_\alpha
      \oot[\lambda]
      \Zcal
      \too[\rho]
      \Xcal
    \]
    where distribution on $\Zcal$ is rational with denominator
    $n\in\Nbb$.  Then
    \begin{align*}
      \ikd\Big(T^{\c{(1-\alpha) n}}&(\Xcal\rel\square)\,,\,
               T^{\c{n}}\Xcal\Big)
      \\
	&
	\leq
	n\cdot\size{G}\cdot
	\left[
	2\ent(\Lambda_\alpha)+
	\alpha\cdot\ln|X_{0}| 
	\right]
	+
	2 \cdot \size{\Gbf}\cdot|X_{0}|\cdot
	\ln(n+1)
      \\
      &= 
      n\cdot\size{G}\cdot
      \big[
        2\ent(\Lambda_\alpha)+
        \alpha \cdot\ln|X_{0}|
      \big]
      + \O\left(|X_{0}|\cdot\ln n \right)
    \end{align*}
  \end{lemma}

  The Lemma (and Proposition \ref{p:dist-types-complete} below) are
  closely related to the local estimate,
  Proposition \ref{p:kolmogorovlocal}. It is an immediate consequence
  of the Slicing Lemma, in particular
  Corollary \ref{p:slicingcorollary} part (\ref{p:slicingtwofan}) that
  \[          
  \ikd\Big(\Xcal\rel\square\,,\,
  \Xcal\Big)
  \leq
  \size{G}\cdot
  \left[
    2\ent(\Lambda_\alpha)+
    \alpha\cdot\ln|X_{0}| 
    \right]
  \]
  This is a tacit ingredient in the proof of the local estimate. 
  By the subadditivity of the Kolmogorov distance, 
  \[
  \ikd\Big((\Xcal\rel\square)^{\otimes n}\,,\,
  \Xcal^{\otimes n}\Big)
  \leq
  n\cdot
  \size{G}\cdot
  \left[
    2\ent(\Lambda_\alpha)+
    \alpha\cdot\ln|X_{0}| 
    \right]
  \]

  This bound is almost the estimate in Lemma \ref{p:lagging-kd},
  except Lemma \ref{p:lagging-kd} estimates the distance between types
  rather than tensor powers. We will soon see that tensor powers and
  types are very close in the Kolmogorov distance. However, for the
  purpose of the proof of Lemma \ref{p:lagging-kd}, it suffices to
  know that their entropies are close, an estimate that is provided by
  Corollary \ref{p:entropy-type}.

\medskip
\noindent{\it Proof} (of Lemma~\ref{p:lagging-kd}): 
  We will use the lagging two-fan constructed in
  Equation~(\ref{eq:laggingtwofan}), namely
  \[
    \Lcal
    :=
    \big(
    T^{\c{(1-\alpha) n}}(\Xcal\rel\square)
    \oot[l]
    T^{\c{n}}\Zcal
    \too[T\rho]
    T^{\c{n}}\Xcal 
    \big)
  \]
  as a coupling to estimate the Kolmogorov distance
  \[
    \ikd\big(T^{\c{(1-\alpha) n}}(\Xcal\rel\square)\,,\,
    T^{\c{n}}\Xcal\big)
    \leq
    \kd(\Lcal)
  \]
	
  Recall that by Corollary \ref{p:entropy-type} for a probability
  space $X$ with a rational distribution we have
  \[  
    n \cdot\ent(X) - |X| \cdot \ln (n+1) 
    \leq 
    \ent(T^{\c n} X) \leq n \cdot\ent(X)
  \]  
  
  Thus we can estimate $\kd(\Lcal)$ as follows
  \begin{align*}
    \kd(\Lcal)
    &=
    \sum_{i}
    \Big[
      \big(
      \ent(T^{\c n}Z_{i}) 
      - 
      \ent(T^{\c{n}} X_{i})
      \big)
      \\
      &\qquad \qquad +  
      \big(
      \ent(T^{\c n}Z_{i}) 
      - 
      \ent(T^{((1-\alpha)n)} (X_{i}\rel\square))
      \big)
    \Big]
    \\
    &\leq
    n\cdot
    \sum_{i}
    \Big[
      \big(
      \ent(Z_{i}) 
      - 
      \ent(X_{i})
      \big)
      +
      \big(
      \ent(Z_{i}) 
      - 
      (1-\alpha)\ent(X_{i}\rel\square)
      \big)
    \Big] 
    \\
    &\quad + 
    2 \cdot \size{\Gbf}\cdot |X_0|\cdot\ln(n+1) 
  \end{align*}
     
  By minimality of the original two-fan and Shannon inequality
  (\ref{eq:shannonineq}) we have a bound
  \[
    \ent(Z_{i}) - \ent(X_{i}) 
    \leq
    \ent(\Lambda_\alpha)
  \]
	
  The second part in the sum can be estimated using relation
  (\ref{eq:relentinteg}) as follows
  \begin{align*}
    \ent(Z_{i}) - (1-\alpha)\ent(X_{i}\rel\square)
    &=
    \ent(\Lambda_\alpha) + 
    \ent(X_{i}\rel\Lambda_\alpha) - 
    (1-\alpha)\ent(X_{i}\rel\square)
    \\
    &=
    \ent(\Lambda_\alpha) + 
    (1-\alpha)\ent(X_{i}\rel\square) + 
    \alpha\Ent(X_{i}\rel\blacksquare) -
    \\
    &\quad-
    (1-\alpha)\ent(X_{i}\rel\square)
    \\
    &\leq
    \ent(\Lambda_\alpha)
    +
    \alpha\cdot\ln|X_{i}|
  \end{align*}
  Combining all of the above we obtain the estimate in the conclusion
  of the lemma.  
\eproof

\subsection{Distance between types}
  In this section we use the lagging trick as described above to
  estimate the distance between types over two different distributions
  in $\Delta \Scal$ where $\Scal$ is a complete configuration of sets.

  \begin{proposition}{p:dist-types-complete}
    Suppose $\Scal$ is a complete $\Gbf$-configuration of sets with initial
    set $S_{0}$. 
    Suppose $p, q \in\Deltan \Scal$ and let
    $\alpha=\frac12|p_0-q_0|_1$. Then
    \begin{align*}  
      \ikd(T^{\c n}_{p}\Scal,T^{\c n}_{q}\Scal)
      &\leq 
      2n\cdot\size{\Gbf}\cdot
      \left[
        \alpha\cdot\ln |S_{0}|
        + 
        2\ent(\Lambda_{\alpha})
      \right]
      + 
      4 \size{\Gbf}\cdot|X_{0}|\cdot\ln(n+1)
      \\
      &=
      2n\cdot\size{\Gbf}\cdot
      \left[
        \alpha\cdot\ln |S_{0}|
        + 
        2\ent(\Lambda_{\alpha})
        \right]
      + 
      \O(|X_{0}|\cdot\ln n)
    \end{align*} 
  \end{proposition}
  

  As in the local estimate, the idea of the proof is to write $p$ and
  $q$ as a convex combination of a common distribution $\hat p$ and
  ``small amounts'' of $p^{+}$ and $q^{+}$, respectively. Then we use
  the lagging trick to estimate distances between types over $p$ and
  $\hat p$, as well as between types over $q$ and $\hat p$.
  We now present details of the proof.
  
\medskip
\noindent\textit{Proof} (of Proposition~\ref{p:dist-types-complete}):
  	Recall that for a complete configuration $\Scal$ with initial set
  	$S_{0}$ we have 
  	\[\tageq{distr-complete-again}
  	\Delta\Scal\cong\Delta S_{0}
  	\]
  	
  	
  	Our goal now is to write $p$ and $q$ as the convex combination of
  	three other distributions $\hat p$, $p^{+}$ and $q^{+}$ as in
  	\begin{align*}
  	p
  	&=
  	(1-\alpha)\cdot\hat p + \alpha\cdot p^{+}
  	\\
  	q
  	&=
  	(1-\alpha)\cdot\hat p + \alpha\cdot q^{+}
  	\end{align*}
  	
  	We could do it the following way.  Let
  	$\alpha:=\frac12|p_{0}-q_{0}|_1$. If $\alpha=1$ then the proposition
  	follows trivially by constructing a tensor-product fan, so
  	from now on we assume that $\alpha<1$.  Define three probability
  	distributions $\hat p_{0}$, $p_{0}^{+}$ and $q_{0}^{+}$ on $S_{0}$
  	by setting for every $x\in S_{0}$
  	\begin{align*}
  	\hat p_{0}(x) 
  	&:= \frac1{1-\alpha}
  	\min\set{p_{0}(x),q_{0}(x)}
  	\\ 
  	p_{0}^{+} 
  	&:=
  	\frac{1}{\alpha}\big(p_{0}-(1-\alpha)\hat p_{0}\big)
  	\\ 
  	q_{0}^{+} 
  	&:= 
  	\frac{1}{\alpha}\big(q_{0}-(1-\alpha)\hat p_{0}\big)
  	\end{align*}
  	
  	Denote by $\hat p,p^{+},q^{+}\in\Delta\Scal$ the distributions
        corresponding to $\hat p_{0},p_{0}^{+},q_{0}^{+}\in\Delta
        S_{0}$ under the affine
        isomorphism~(\ref{eq:distr-complete-again}). Thus we have
  	\begin{align*}
  	p&=(1-\alpha)\hat p+\alpha\cdot p^{+}\\
  	q&=(1-\alpha)\hat p+\alpha\cdot q^{+}
  	\end{align*}
  	
  	Now we construct a pair of two-fans of
  	$\Gbf$-configurations 
  	\begin{align*}
  	\tageq{twofanfortypes}
  	\Lambda_{\alpha}\ot\tilde\Xcal\to\Xcal\\
  	\Lambda_{\alpha}\ot\tilde\Ycal\to\Ycal
  	\end{align*}
  	by setting 
  	\begin{align*}
  	\Xcal
  	&:=
  	(\Scal,p)
  	\\
  	\Ycal
  	&:=
  	(\Scal,q)
  	\\
  	\tilde X_{i}
  	&:=
  	\Big(S_{i}\times\un\Lambda_{\alpha};\;
  	\tilde p_{i}(s,\square)=(1-\alpha)\hat p_{i}(s),\,
  	\tilde p_{i}(s,\blacksquare)=\alpha\cdot p^{+}_{i}(s)
  	\Big)
  	\\
  	\tilde Y_{i}
  	&:=
  	\Big(S_{i}\times\un\Lambda_{\alpha};\;
  	\tilde q_{i}(s,\square)=(1-\alpha)\hat p_{i}(s),\,
  	\tilde q_{i}(s,\blacksquare)=\alpha\cdot q^{+}_{i}(s)
  	\Big)
  	\\
  	\end{align*}
  	and
  	\begin{align*}
  	\tilde\Xcal
  	&:=
  	\set{\tilde X_{i};\,f_{ij}\times\Id}\\
  	\tilde\Ycal
  	&:=
  	\set{\tilde Y_{i};\,f_{ij}\times\Id}\\
  	\end{align*}
  	
  	The reductions in~(\ref{eq:twofanfortypes}) are given by coordinate
  	projections. We have the following isomorphisms
  	\begin{align*}
  	\Xcal\rel\square
  	&\cong
  	\Ycal\rel\square
  	\cong
  	(\Scal,\hat p)
  	\end{align*}
  	
  	To estimate the distance between types we now apply
  	Lemma~\ref{p:lagging-kd} to the fans
        in~(\ref{eq:twofanfortypes}) 
  	\begin{align*}
  	\ikd(T^{(n)}_{p}\Scal,T^{(n)}_{q}\Scal)
  	&=
  	\ikd(T^{(n)}\Xcal,T^{(n)}\Ycal)
  	\\
  	&\leq
  	\ikd\big(T^{(n)}\Xcal,T^{((1-\alpha)n)}(\Xcal\rel\square)\big)  
  	+
  	\ikd\big(T^{((1-\alpha)n)}(\Ycal\rel\square),T^{(n)}\Ycal\big)  
  	\\
        &\leq
        2n\cdot\size{\Gbf}\cdot
        \left[
          \alpha\cdot\ln |S_{0}|
          + 
          2\ent(\Lambda_{\alpha})
        \right]
        + 
        4 \size{\Gbf}\cdot|X_{0}|\cdot\ln(n+1)
  	\end{align*}	
\eproof
  

  The reason for the similarity between the local estimate and the
  distance estimate between types will become clear in the next
  section, when we establish the asymptotic equivalence between the
  Bernoulli sequence of probability spaces and sequence of types over
  rational distributions approximating the true distribution.
