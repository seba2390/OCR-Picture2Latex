  In this section we will look at configurations in more detail. We
  start by considering some important examples.

\subsection[Examples]{Examples of configurations}\label{s:config-examples}

\subsubsection{Singleton}\label{s:config-examples-single}
  We denote by $\bullet$ a diagram category with a single object.
  Clearly configurations modeled on $\bullet$ are just probability
  spaces and we have $\prob\equiv\prob\<\bullet>$.

\subsubsection{Chains}\label{s:config-examples-chain}
  The chain $\Cbf_{n}$ of length $n\in\Nbb$ is a category with $n$ objects
  $\set{O_{i}}_{i=1}^{n}$ and morphisms from $O_{i}$ to $O_{j}$
  whenever $i\leq j$. A configuration $\Xcal\in\prob\<\Cbf_{n}>$ is a
  chain of reductions
  \[
    \Xcal=(X_{1}\to X_{2}\to\cdots\to X_{n})
  \]

\subsubsection{Two-fan}\label{s:config-examples-fan}
  The two-fan $\Lambdabf$ is a category with three objects
  $\set{O_{1},O_{12},O_{2}}$ and two non-identity morphisms $O_{12}\to
  O_{1}$ and $O_{12}\to O_{2}$.  See also
  Section~\ref{s:category-config-2fan}.  Two-fans are the simplest
  configurations for which asymptotic equivalence classes contain more
  information than just entropies of the entries.
  
  Recall that a fan $(X\ot Z\to Y)$ is called minimal, if for any pair
  of points $x\in X$ and $y\in Y$ with positive weights there exists at
  most one $z\in Z$, that reduces to $x$ and to $y$. Equivalently, for
  any super-configuration
  \[
  \begin{tikzcd}[column sep=small,row sep=small]
    \mbox{}
    &
    Z
    \arrow{ddl}{}
    \arrow{ddr}{}
    \arrow{d}{}
    &
    \mbox{}
    \\
    \mbox{}
    &
    Z'
    \arrow{dl}{}
    \arrow{dr}{}
    &
    \mbox{}
    \\
    X
    &
    &
    Y
  \end{tikzcd}
  \]
  the reduction $Z\to Z'$ must be an isomorphism.

\subsubsection{Full configuration}\label{s:config-examples-full}
  The full category $\Lambdabf_{n}$ on $n$ objects is a category with
  objects $\set{O_{I}}_{I\in
    2^{\set{1,\ldots,n}}\setminus\set{\emptyset}}$ indexed by all
  non-empty subsets $I\in 2^{\set{1,\ldots,n}}$ and a morphism from
  $O_{I}$ to $O_{J}$, whenever $J\subseteq I$.

  For a collection of random variables $X_{1},\ldots,X_{n}$ one may
  construct a minimal full configuration $\Xcal\in\prob\<\Lambdabf_{n}>$ by
  considering all joint distributions and ``marginalization''
  reductions. We denote such a configuration by
  $\<X_{1},\ldots,X_{n}>$. 
  On the other hand, the terminal
  vertices of a full configuration can be viewed as random variables
  on the domain of definition given by the (unique) initial space.
  
  Suppose $\Xcal\in\prob\<\Lambdabf_{n}>$ is a minimal full
  configuration with terminal vertices $X_{1},\ldots,X_{n}$. It is
  convenient to view $\Xcal$ as a distribution on the Cartesian
  product of the underlying sets of the terminal vertices:
  \[
    p_{\Xcal}\in\Delta(\un X_{1}\times\cdots\times\un X_{n})
  \]
  Once the underlying sets of the terminal spaces are fixed, there is
  a one-to-one correspondence between the full minimal configurations
  and distributions as above.

\subsubsection{``Two-tents'' configuration}
\label{s:config-examples-twotents}
  The ``two-tents'' category $\Mbf_{2}$ consists of five objects, of
  which two are initial and three are terminal, and morphisms are as
  follows
  \[
  \Mbf_{2}=
  \left(
  \begin{tikzcd}[column sep=small,ampersand replacement=\&]
  O_{12}
  \arrow[xshift=-0.3em]{d}{}
  \arrow{dr}{} 
  \&
  \&
  O_{23} 
  \arrow[xshift=-0.3em]{dl}{}
  \arrow{d}{}
  \\
  O_1 
  \&
  O_2
  \&
  O_3
  \end{tikzcd}
  \right)
  \]
  
  Thus, a typical two-tents configuration consists of five probability
  spaces and reduction as in
  \[
    \Xcal=(X\ot U\to Y\ot V\to Z)
  \]
  The probability spaces $U$ and $V$ are initial and $X$, $Y$ and $Z$
  are terminal.
 
\subsubsection{``Many-tents'' configuration}
\label{s:config-examples-manytents} 
  The previous example could be generalized to a ``many-tents'' category
  \[
    \Mbf_{n}
    =
    (O_{1}\ot O_{12}\to O_{2}\ot\cdots\to O_{n-1}\ot O_{n-1,n}\to O_{n})
  \]

\subsubsection{``Fence'' configuration}
\label{s:config-examples-fence} 
  The ``fence'' category $\Wbf_{3}$ consists of six objects and the
  morphisms are 
  \[
    \Wbf_{3}
    =
    \left(
    \begin{tikzcd}[ampersand replacement=\&,row sep=small, column sep=small]
      O_{12}
      \arrow{d}
      \&
      O_{13}
      \arrow{dr}
      \arrow{dl}
      \&
      O_{23}
      \arrow{d}
      \\
      O_{1}
      \&
      O_{2}
      \arrow[leftarrow, crossing over]{ur}
      \arrow[leftarrow, crossing over]{ul}
      \&
      O_{3}      
    \end{tikzcd}
    \right)
  \]

\subsubsection{Co-fan}\label{s:config-examples-cofan}
A co-fan $\Vbf$ is a category with three objects and morphisms as in the diagram
\[
\Vbf =
\left(  \begin{tikzcd}[row sep=small, column sep=small]
\mbox{}
O_{1}
\arrow{dr}{}
& 
&
O_{2}
\arrow{dl}{}
\\
\mbox{}
&
O_{\bullet}
&
\end{tikzcd} \right)
\]

\subsubsection{``Diamond'' configurations}
\label{s:config-examples-diamond} 
  A ``diamond'' configuration $\dia$ is modeled on a diamond category
  that consists of a fan and a co-fan
  \[
  \dia =
  \left(  
  \begin{tikzcd}[ampersand replacement=\&,row sep=small, column sep=small]
    \mbox{}
    \& 
    O_{12}
    \arrow{dl}{}
    \arrow{dr}{}
    \&
    \\
    O_{1}
    \arrow{dr}{}
    \& 
    \&
    O_{2}
    \arrow{dl}{}
    \\
    \mbox{}
    \&
    O_{\bullet}
    \&
  \end{tikzcd} \right)
  \]
  See also Section~\ref{s:category-config-diamond}.

  \bigskip

  Examples
  \ref{s:config-examples-single}, 
  \ref{s:config-examples-chain}, 
  \ref{s:config-examples-fan},
  \ref{s:config-examples-full} and 
  \ref{s:config-examples-diamond}
  are complete. Examples \ref{s:config-examples-single},
  \ref{s:config-examples-chain} and \ref{s:config-examples-cofan} do
  not contain a two-fan. Tropical limits of such configurations are
  very simple. Essentially, such tropical limits correspond to the
  tuple of numbers corresponding to the entropies of the constituent
  spaces. Therefore, we call configurations not containing a two-fan
  \term[simple configuration]{simple}.
  
\subsection{Constant configurations}
\label{s:config-constantconfig}
  Suppose $X$ is a probability space and $\Gbf$ is a diagram category.
  One may form a \term[constant configuration]{constant
    $\Gbf$-configuration} by considering a functor that maps all
  objects in $\Gbf$ to $X$ and all the morphisms to the identity
  morphism $X\to X$. We denote such a constant configuration by
  $X^{\Gbf}$ or simply by $X$, when $\Gbf$ is clear from the context.
  Any constant configuration is automatically minimal.

  If $\Ycal=\set{Y_{i};f_{ij}}$ is another $\Gbf$-configuration, then
  a reduction $\rho:\Ycal\to X^{\Gbf}$ (which we write sometimes
  simply as $\rho:\Ycal\to X$) is a collection of reductions
  $\rho_{i}:Y_{i}\to X$, such that
  \[
    f_{ij}\circ\rho_{i}=\rho_{j}
  \]
  
\subsection{Configurations of configurations}
\label{s:config-config}
  Of course, the operation of ``configuration'' could be iterated, so
  given a pair $\Gbf_{1}$, $\Gbf_{2}$ of diagram categories we could
  form a $\Gbf_{2}$-configuration of $\Gbf_{1}$-configurations, so we
  could speak, for example, about a two-fan of configurations of the same
  type.
  \[
    \Prob\<\Gbf_{1},\Gbf_{2}>
    :=
    \Prob\<\Gbf_{1}>\<\Gbf_2>
    =
    \prob\<\Gbf_{1}\square\Gbf_2>
  \]
  where $\Gbf_{1}\square\Gbf_2$ is the ``Cartesian product of
  graphs'' (as every diagram category could be considered as a
  transitively closed directed graph).  This operation is commutative,
  thus, for example, a two-fan of $\Gbf$-configurations is a
  $\Gbf$-configuration of two-fans.
  
  We will rarely need anything beyond a two-fan of configurations.

  A two-fan $\Fcal=(\Xcal\ot\Zcal\to\Ycal)$ of
  $\Gbf$-configurations is called minimal if in any extension of
  $\Fcal$ of the form
  \[
  \begin{tikzcd}[ row sep=small, column sep=small,ampersand replacement=\&]
    \mbox{}
    \&
    \Zcal
    \arrow{dd}[description]{f}
    \arrow{dddl}{}
    \arrow{dddr}{}
    \&
    \mbox{}
    \\
    \mbox{}
    \&
    \&
    \\
    \mbox{}
    \&
    \Zcal'
    \arrow{dr}{}
    \arrow{dl}{}
    \&
    \mbox{}
    \\
    \Xcal
    \&
    \&
    \Ycal
  \end{tikzcd}
  \]
  the reduction $f:\Zcal\to\Zcal'$ must be an isomorphism of
  $\Gbf$-configurations.

  Recall that a two-fan of $\Gbf$-configurations could also be viewed
  as a $\Gbf$-configuration of two-fans of probability spaces. In the
  following lemma we show that in order to verify the minimality of a
  two-fan of configurations it is sufficient to check the minimality
  of all the constituent two-fans.
  \begin{lemma}{p:minimalfansconfig}
    Let $\Gbf$ be a diagram category. Then
    \begin{enumerate}
    \item\label{p:minimalfansconfig1} A two-fan
      $\Fcal=(\Xcal\ot\Zcal\to\Ycal)$ of $\Gbf$-configurations is
      minimal, if and only if the constituent two-fans of probability
      spaces $\Fcal_{i}=(X_{i}\ot Z_{i}\to Y_{i})$ are all minimal.
    \item \label{p:minimalfansconfig2}
      For any two-fan $\Fcal=(\Xcal\ot\Zcal\to\Ycal)$ of
      $\Gbf$-configurations its minimal reduction exists,
      that is, there exists a minimal two-fan
      $\Fcal'=(\Xcal\ot\Zcal'\to\Ycal)$ included in the following
      diagram
      \[
      \begin{tikzcd}[row sep=small, column sep=small, ampersand replacement=\&]
        \mbox{}
        \&
        \Zcal
        \arrow{d}{}
        \arrow{ddl}{}
        \arrow{ddr}{}
        \&
        \mbox{}
        \\
        \mbox{}
        \&
        \Zcal'
        \arrow{dr}{}
        \arrow{dl}{}
        \&
        \mbox{}
        \\
        \Xcal
        \&
        \&
        \Ycal
      \end{tikzcd}
      \]      
    \end{enumerate}
  \end{lemma}

  Even though this lemma is rather elementary, there are many similar
  statements that are not true. Thus we are compelled to provide a
  proof, which can be found in Section~\ref{s:technical} on
  page~\pageref{p:minimalfansconfig.rep}.

  Similarly, a full configuration $\Fcal$ of $\Gbf$-configurations is
  called minimal if for every two-fan of $\Gbf$-configurations in
  $\Fcal$ there is a minimal two-fan in $\Fcal$ of
  $\Gbf$-configurations with the same terminal configurations.

  Lemma \ref{p:minimalfansconfig} has the following corollary and
  counterpart for full configurations of $\Gbf$-configurations.
  \begin{corollary}{p:minimalfullconfig}
    Let $\Gbf$ be a diagram category. Then
    \begin{enumerate}
    \item\label{p:minimalfullconfig1} 
      A full configuration $\Fcal$ of $\Gbf$-configurations is minimal,
      if and only if the constituent full configurations of probability
      spaces $\Fcal_{i}$ are all minimal.
    \item \label{p:minimalfullconfig2}   
      For any full configuration $\Fcal$ of $\Gbf$-configurations its
      minimal reduction exists.
    \end{enumerate}
  \end{corollary}
  
\subsection{Restrictions and extensions}
  \label{s:config-restrictions}
  Suppose $R:\Gbf_{1}\to \Gbf_{2}$ is a functor between two diagram
  categories. For a configuration $\Xcal:\Gbf_{2}\to \prob$, the pull-back
  configuration $\Ycal=R^{*}\Xcal\in\prob\<\Gbf_{1}>$ defined as the
  composition
  \[
    \Ycal
    :=
    \Xcal\circ R
  \]
  is called an \term[restriction functor]{$R$-restriction} of $\Xcal$
  to $\Gbf_1$ and $\Xcal$ is the \term[extension of a
    configuration]{extension} of $\Ycal$. If the functor $R$ is
  injective then we call $\Ycal=R^{*}\Xcal$ a \term{sub-configuration}
  of $\Xcal$ and write $\Ycal\subset\Xcal$, likewise $\Xcal$ will be
  called a \term{super-configuration} of $\Ycal$.

  The restriction operation is functorial in the sense that given
  two configurations 
  $\Xcal,\Xcal'\in\prob\<\Gbf_{1}>$ and a reduction
  $f:\Xcal\to\Xcal'$, there is a canonical reduction
  $R^{*}f:R^{*}\Xcal\to R^{*}\Xcal'$. Thus $R^{*}$ can be considered
  as a functor
  \[
    R^{*}:\prob\<\Gbf_{2}>\to\prob\<\Gbf_{1}>
  \]
  
  Some important examples of restrictions and extensions are below.

\subsubsection{Restriction of a full configuration to a smaller full
  configuration} 
\label{s:config-restrictions-fullfull}
  Recall that, as explained in Section
  \ref{s:config-examples-full}, the terminal vertices of a full
  configuration could be considered as random variables and any
  collection of random variables ``generates'' a full configuration.

  For a full configuration $\Xcal=\langle X_{i}\rangle_{i=1}^{n}$ and
  a subset $I\subset\set{1,\ldots,n}$ we denote by $R_{I}^*\Xcal=\langle
  X_{i}\rangle_{i\in I}$ the restriction of $\Xcal$ to a full
  configuration generated by $X_{i}$, $i\in I$.  We will also make use
  of the notation $R_{k,l}^*$ for the restriction operator $R_{\set{1,\ldots,k}}^*: \prob\<\Lambdabf_l> \to \prob\<\Lambdabf_k>$.

\subsubsection{Restriction of a $\Lambdabf_{3}$-configuration to
  an $\Mbf_{2}$-configuration}
\label{s:config-restrictions-full2tents}

  Given a full configuration $\Xcal\in\prob\<\Lambdabf_{3}>$ we may
  ``forget'' part of the data. If we, for example, forget the top
  space and the relation between a pair out of three terminal spaces
  we end up with the two-tents configuration. This operation
  corresponds to the inclusion functor
  \[
  M:
  \left(
  \begin{tikzcd}[column sep=small,row sep=normal,ampersand replacement=\&]
    O_{12}
    \arrow{d}{}
    \arrow{dr}{}
    \&
    \&
    O_{23}
    \arrow{d}{}
    \arrow{dl}{}
    \\
    O_{1}
    \&
    O_{2}
    \&
    O_{3}
  \end{tikzcd}
  \right)
  \longrightarrow
  \left(
  \begin{tikzcd}[column sep=small, row sep=normal,ampersand replacement=\&]
    \mbox{}
    \&
    Q_{123}
    \arrow{dl}{}
    \arrow{d}{}
    \arrow{dr}{}
    \\
    Q_{12}
    \arrow{d}{}
    \&
    Q_{31}
    \arrow{dl}{}
    \arrow{dr}{} 
    \&
    Q_{23}
    \arrow{d}
    \\
    Q_{1}
    \&
    Q_{2}
    \arrow[leftarrow, crossing over]{ul}
    \arrow[leftarrow, crossing over]{ur}
    \&
    Q_{3}
  \end{tikzcd}
  \right)
  \]
  that preserves the sub-indices.

We show in Section \ref{s:config-adhesion} below that the corresponding 
restriction operator
\[
M^{*}:\prob\<\Lambdabf_{3}>\to\prob\<\Mbf_{3}> 
\]
is surjective, both on objects and all morphisms. Thus, as a map of
collections of objects it has a right inverse. However, no natural right
inverse exists. 

\subsubsection{Restriction of a $\Lambdabf_{3}$-configuration to
  a $\Wbf_{3}$-configuration} 
\label{s:config-restrictions-fullfence}

Starting with a full configuration we might choose to forget the
initial space (and reductions, for which it was the domain). The
remaining configuration has the combinatorial type of a fence. This
operation corresponds to the functor
\[
W:
\left(
\begin{tikzcd}[column sep=small, row sep=normal,ampersand replacement=\&]
  O_{12}
  \arrow{d}{}
  \&
  O_{31}
  \arrow{dl}{}
  \arrow{dr}{} 
  \&
  O_{23}
  \arrow{d}{}
  \\
  O_{1}
  \&
  O_{2}
  \arrow[leftarrow, crossing over]{ul}
  \arrow[leftarrow, crossing over]{ur}
  \&
  O_{3}
\end{tikzcd}
\right)
\longrightarrow
\left(
\begin{tikzcd}[column sep=small, row sep=normal, ampersand replacement=\&]
  \&
  Q_{123}
  \arrow{dl}{}
  \arrow{d}{}
  \arrow{dr}{}
  \\
  Q_{12}
  \arrow{d}{}
  \&
  Q_{31}
  \arrow{dl}{}
  \arrow{dr}{} 
  \&
  Q_{23}
  \arrow{d}
  \\
  Q_{1}
  \&
  Q_{2}
  \arrow[leftarrow, crossing over]{ul}
  \arrow[leftarrow, crossing over]{ur}
  \&
  Q_{3}
\end{tikzcd}
\right)
\]
The corresponding operator 
\[
W^{*}:\prob\<\Lambdabf_{3}>\to\prob\<\Wbf_{3}>
\]
is not surjective. To find out when a $\Wbf_{3}$-configuration is extendable
to a $\Lambdabf_{3}$-configuration is an interesting problem, see for
example, \cite{Abramsky-Contextuality-2015} and references
therein.
It is our hope that the methods developed in this article might be
useful to address these questions.

\subsubsection{Doubling}
\label{s:config-restrictions-doubling}
 This will be the first example of an interesting
    functor between diagram categories, which is not injective. In
    this situation the term ``restriction'' does not really reflect
    the operation of pull-back well, however we did not come up with a
    better terminology.
 
  The doubling operation is the restriction of a two-fan to a two-tents
  configuration.
  Consider the two-fan category $\Lambdabf=(O_{1}\ot
  O_{12}\to O_{2})$ and a two-tents category $\Mbf_{2}=(Q_{1}\ot
  Q_{12}\to Q_{2}\ot Q_{23}\to Q_{3})$. Define the functor
  $D:\Mbf_{2}\to\Lambdabf$ by setting
  \[
  D:
  \begin{cases}
    Q_{1}\mapsto O_{1}\\
    Q_{2}\mapsto O_{2}\\
    Q_{3}\mapsto O_{1}\\
    Q_{12}\mapsto O_{12}\\
    Q_{23}\mapsto O_{12}\\
  \end{cases}
  \]

  Note that $D$ extends uniquely to the spaces of morphisms, since each
  morphism space is either empty or a one-point set.
  
  Thus $D^{*}(X\ot Z\to Y)=(X\ot Z\to Y\ot Z\to X)$, where the ``left''
  and ``right'' two-fans are isomorphic.
  
  This operation along with a particular Information-Optimization
    problem is related to the so-called copy operation, that was
  used to find many non-Shannon information inequalities, as
  described, for example, in \cite{Dougherty-Non-Shannon-2011}.

\subsection{Adhesion}\label{s:config-adhesion}
  Given a minimal two-tents configuration $\Xcal=(X\ot U\to Y\ot V\to
  Z)\in\prob\<\Mbf_{2}>_{\mbf}$ one could always construct an extension of
  $\Xcal$ to a full configuration $\ad(\Xcal)\in\prob\<\Lambdabf_{3}>_{\mbf}$ in the
  following way: As explained in Section \ref{s:config-examples-full},
  to construct a minimal full configuration with terminal vertices
  $X$, $Y$ and $Z$ it is sufficient to provide a distribution on $\un
  X\times\un Y\times\un Z$ with the correct marginals. We do this
  by setting
  \[
  p(x,y,z):=\frac{p_{U}(x,y)\cdot p_{V}(y,z)}{p_{Y}(y)}
  \]
  It is straightforward to check that the appropriate restriction of
  the full configuration defined in the above manner is indeed the
  original two-tents configuration.  Essentially, to extend we need to
  provide a relationship (coupling) between spaces $X$ and $Z$ and we
  do it by declaring $X$ and $Z$ independent relative to $Y$.  This is
  an instance of operation called \term{adhesion}, see
  \cite{Matus-Infinitely-2007}.

  If we call the top vertex in the full configuration $W$, the
  entropies achieve equality in the Shannon inequality, that is
  \[
    \Ent( U ) + \Ent( V ) - \Ent( W ) - \Ent(Y) = 0.
  \]
 
  Adhesion provides a right inverse $\ad$ to the restriction
  functor $M^{*}$ described in
  Section~\ref{s:config-restrictions-full2tents}
  \[
  \begin{tikzcd}[ampersand replacement=\&]
    \prob\<\Lambdabf_{3}>_{\mbf}
    \arrow{r}{M^{*}}
    \&
    \prob\<\Mbf_{2}>_{\mbf}
    \arrow[dashed,bend left]{l}{\ad}
  \end{tikzcd}
  \]

  It is important to note though, that the map $\ad$ is not functorial
  and, in fact, no functorial inverse of $M^{*}$ exists.

\subsection{Homogeneous configurations}\label{s:config-homo}
  A configuration $\Xcal\in\prob\<\Gbf>$ modeled on some diagram
  category $\Gbf$ is called \term[homogeneous
    configuration]{homogeneous} if its automorphism group
  $\Aut(\Xcal)$ acts transitively on every probability space in
  $\Xcal$.  Three examples of homogeneous configurations were given in
  the introduction. Other examples of a homogeneous configurations (of
  combinatorial type $\Lambdabf_{3}$) are shown in
  Figure~\ref{f:MonstersDusk}.  The
  subcategory of all homogeneous configurations modeled on $\Gbf$ will
  be denoted $\Prob\<\Gbf>_{\hbf}$.
  
  \begin{figure}
	\begin{lpic}{MonstersDusk(140mm,)}
	\end{lpic}
	\caption{Examples of homogeneous configurations}\label{f:MonstersDusk}
  \end{figure}

  In fact, for $\Xcal$ to be homogeneous it is sufficient that the
  $Aut(\Xcal)$ acts transitively on every initial space in $\Xcal$.
  Thus, if $\Xcal$ is complete with initial space $X_{0}$, to
  check homogeneity it is sufficient to check the transitivity of the
  action of the symmetries of $\Xcal$ on $X_{0}$.

  By functoriality of the restriction operator, any restriction of a
  homogeneous configuration is also homogeneous. In other words, if
  $R:\Gbf\to\Gbf'$ is a functor and
  \[
  R^{*}:\prob\<\Gbf'>\to\prob\<\Gbf>
  \] 
  is the associated restriction operator, then
  \[
  R^{*}\big(\prob\<\Gbf'>_{\hbf}\big)\subset\prob\<\Gbf>_{\hbf}
  \]

  In particular, all the individual spaces of a homogeneous
  configuration are homogeneous 
  \[
    \prob\<\Gbf>_{\hbf}\subset\probhom\<\Gbf>
  \]
  However homogeneity of the whole of the configuration is a stronger
  property than homogeneity of the individual spaces in the
  configuration, thus in general
  \[
    \prob\<\Gbf>_{\hbf}
    \subsetneq
    \probhom\<\Gbf>
  \]

  A single probability space is homogeneous if and only if there is a
  representative in its isomorphism class with uniform measure and the
  same holds true for chain configurations, for the co-fan or any
  other configuration that does not contain a two-fan.  However, for
  more complex configurations, for example for two-fans, no such
  simple description is available.
 
 \subsubsection{Universal construction of homogeneous configurations} 
  Examples of homogeneous configurations could be constructed in the
  following manner.  Suppose $\Gamma$ is a finite group and
  $\set{H_{i}}$ is a collection of subgroups. Consider a collection of
  sets $\un X_{i}:=\Gamma/H_{i}$ and consider a natural surjection
  $f_{ij}:\un X_{i}\to \un X_{j}$ whenever $H_{i}$ is a subgroup of
  $H_{j}$. Equipping each $\un X_{i}$ with the uniform distribution
  one can turn the configuration of sets $\set{\un X_{i};f_{ij}}$ into
  a homogeneous configuration. It will be complete if there is a
  smallest subgroup (under inclusion) among $H_{i}$'s.  
  
  Such a configuration will be complete and minimal, if
  together with any pair of groups $H_{i}$ and $H_{j}$ in the
  collection, their intersection $H_{i}\cap H_{j}$ also belongs to the
  collection $\set{H_{i}}$.

  In fact, any homogeneous configuration arises this way.  Suppose
  configuration $\Xcal = \set{ X_i ; f_{ij} }$ is homogeneous, then
  we set $\Gamma = \Aut(\Xcal)$ and choose a collection of points $x_i
  \in X_i$ such that $f_{ij} (x_i) = x_j$ and denote by $H_i :=
  \stab(x_i) \subset \Gamma$.  Then, if one applies the construction
  of the previous paragraph to $\Gamma$, with the collection of
  subgroups $\set{H_i}$, one recovers the original configuration
  $\Xcal$.

\subsection{Conditioning}\label{s:config-conditioning}
  Suppose a configuration $\Xcal$ contains a fan 
  \[
  \Fcal = \left(X \oot[f] Z \too[g] Y\right)
  \] 
  Given a point $x\in X$ with a non-zero weight one may consider
  \term{conditional probability distributions} $p_Z(\;\cdot\;\rel x)$
  on $\un Z$, and $p_Y( \; \cdot \;\rel x)$ on $\un Y$.  The
  distribution $p_Z(\; \cdot \; \rel x)$ is supported on $f^{-1}(x)$
  and is given by
  \[
  p_Z( z \rel x) 
  = 
  \frac{p_Z(z)}{p_X(x)}
  \]
  The distribution $p_Y(\;\cdot\;\rel x)$ is the pushforward of $p_Z(
  \; \cdot \; \rel x)$ under $g$
  \[
   p_Y(\; \cdot \; \rel x) 
   = 
   g_* p_Z(\; \cdot\;\rel x)
  \]
  Recall that if $\Fcal$ is minimal, the underlying set of $Z$ can be
  assumed to be the product $\un X \times \un Y$. In that case
  \[
  p_Y( y \rel x ) = \frac{p_Z(x,y)}{p_X(x)}
  \]
  
  We denote the corresponding space $Y\rel x:=\big(\un
  Y,p_Y(\;\cdot\;\rel x)\big)$, as discussed at the end of
  Section~\ref{s:category-config-2fan}.

  Under some assumptions it is possible to condition a whole
  sub-configuration of $\Xcal$.  More specifically, if a configuration
  $\Xcal$ contains a sub-configuration $\Ycal$ and a probability space
  $X$ satisfying the condition that
  \begin{quote} 
    for every $Y$ in $\Ycal$ there is a fan in $\Xcal$ with terminal
    vertices $X$ and $Y$,
  \end{quote}
  then we may condition the whole of $\Ycal$ on $x \in X$ given that
  $p_X(x)>0$.
  
  For $x\in X$ with positive weight we denote by $\Ycal\rel x$ the
  configuration of spaces in $\Ycal$ conditioned on $x\in X$. The
  configuration $\Ycal\rel x$ has the same combinatorial type as
  $\Ycal$ and will be called the \term{slice} of $\Ycal$ over
  $x\in X$.  Note that the space $X$ itself may or may not belong to
  $\Ycal$. The conditioning $\Ycal\rel x$ may depend on the choice of
  a fan between $\Ycal$ and $X$, however when $\Xcal$ is complete the
  conditioning $\Ycal\rel x$ is well-defined and is independent of the
  choice of fans.
  
  Suppose now that there are two subconfiguration $\Ycal$ and
  $\Zcal$ in $\Xcal$ and in addition $\Zcal$ is a constant
  configuration, $\Zcal=Z^{\Gbf'}$ for some diagram category
  $\Gbf'$. Let $z\in\un Z$, then $\Ycal\rel z$ is well defined and is
  independent of the choice of the space in $\Zcal$, the element of
  which $z$ is to be considered.

  If $\Xcal$ is homogeneous, then $\Ycal\rel x$ is also homogeneous and
  its isomorphism class does not depend on the choice of $x\in\un X$.

\subsection{Entropy}
\label{s:config-entropy}
  For a $\Gbf$-configuration $\Xcal=\set{X_{i},f_{ij}}$ define the
  entropy function
  \[
    \ent_{*}:\prob\<\Gbf>\to\Rbb^{\size{G}},
    \quad
    \ent_{*}:\Xcal=\set{X_{i},f_{ij}}\mapsto 
    \big(\ent(X_{i})\big)\in\Rbb^{\size{G}}
  \]

  It will be convenient for us to equip the target
  $\Rbb^{\size{\Gbf}}$ with the $\ell^{1}$-norm. Thus
  \[
    |\Ent_{*}(\Xcal)|_{1}=\sum_{i=1}^{\size{\Gbf}}\ent(X_{i})
  \]

  If $\Xcal$ is a complete $\Gbf$-configuration with initial space
  $X_{0}$, then by Shannon inequality (\ref{eq:shannonineq}) there is an
  obvious estimate
  \[
    \ent(X_{0})
    \leq
    |\Ent_{*}(\Xcal)|_{1}
    \leq
    \size{\Xcal}\cdot\ent(X_{0})
  \]

