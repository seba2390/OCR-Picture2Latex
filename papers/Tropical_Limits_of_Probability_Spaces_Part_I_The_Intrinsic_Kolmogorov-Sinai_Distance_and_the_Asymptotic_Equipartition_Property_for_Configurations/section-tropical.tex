  In this section we introduce the notion of \term[tropical
  probability space]{tropical probability spaces} and
  their \term[configuration of tropical probability
  spaces]{configurations}. Configurations of tropical
  probability spaces are points in the asymptotic cone of the space
  $\prob\<\Gbf>$, that is they are ``limits'' of certain divergent
  sequences of ``normal'' configurations. We will first give
  the construction of an asymptotic cone in an abstract context. Next,
  we will apply the construction to the particular case of
  configurations of probability spaces. For some background on asymptotic cones, see for instance \cite{Burago-Course-2001}.
     
\subsection{Asymptotic cones of metric spaces.}
\label{s:tropical-ac}
  The \term[asymptotic cone]{asymptotic cone} captures large-scale
  geometry of a metric space.  Abstractly, the asymptotic cone of a
  pointed metric space is the pointed Gromov-Hausdorff limit of the
  sequence of spaces obtained from the given one by scaling down the
  metric. Of course, convergence is in general by no means
  assured. Sometimes a weaker type of convergence (using ultrafilters)
  is considered.  Since, in our case, the asymptotic cone can be
  evaluated relatively explicitly we do not give the definition of
  Gromov-Hausdorff convergence or convergence with respect to an
  ultrafilter here, but instead give a construction.
  
  We would like to understand asymptotic cones of the space of
  configurations of probability spaces, considered as a metric space
  with the pseudo-metric $\ikd$ or $\aikd$. For a fixed complete
  diagram category $\Gbf$ the space $\prob\<\Gbf>$ is a monoid with
  operation $\otimes$.  It has the additional property that shifts are
  non-expanding maps.  This simplifies the construction and analysis
  of its asymptotic cone.  In fact, as we will see later the metric
  $\aikd$ is already \emph{asymptotic} relative to $\ikd$. The
  application of the asymptotic cone construction to the metric
  $\aikd$ allows us to obtain a complete metric space with a simple
  description of points in it.
  
  Note that even though the monoid
  $(\prob\<\Gbf>,\otimes)$ is not Abelian it has the property that for
  any $\Xcal_{1},\Xcal_{2}\in\prob\<\Gbf>$ one
  has \[ \ikd(\Xcal_{1}\otimes\Xcal_{2},\Xcal_{2}\otimes\Xcal_{1})=0 \]
  Thus, from a metric perspective it is as good as being Abelian.

\subsubsection{Metrics versus pseudo-metrics}
\label{s:tropical-ac-metrics} 
  A pseudo-metric $\dist$ on a set $X$ is a bivariate function
  satisfying all the axioms of a distance function except it is
  non-negative definite rather than positive definite. That is, the
  pseudo-distance function is allowed to vanish on pairs of
  non-identical points. A set equipped with a pseudo-metric will be
  called a pseudo-metric space. An isometry of such spaces is a
  distance-preserving map, such that for any point in the target space
  there is a point in the image, which is distance zero from it.
  Given such an pseudo-metric space $(X,\dist)$ one could always
  construct an isometric metric space $(X/_{\dist=0}\,,\dist)$ by
  identifying all pairs of points that are distance zero apart.

  Any property formulated in terms of the pseudo-metric holds
  simultaneously for a pseudo-metric space and its metric quotient.
  It will be convenient for us to construct pseudo-metrics on spaces
  instead of passing to the quotient spaces.
 
\subsubsection{Asymptotic cone of a metric Abelian monoid}
 Let $(\Gamma,\otimes,\dist)$ be a monoid with a pseudo-metric
 $\dist$, which satisfies the following properties
 \begin{enumerate}
 \item 
   The shifts
   \[
   \cdot \otimes \gamma' : 
   \Gamma \to \Gamma, \quad 
   \gamma \mapsto \gamma \otimes \gamma'
   \]
   are non-expanding for any $\gamma' \in \Gamma$
 \item
   For any $\gamma,\gamma'\in\Gamma$ holds
  \[
  \dist(\gamma \otimes \gamma' , \gamma' \otimes \gamma) = 0
  \]   
 \end{enumerate}
  We will call a monoid with pseudo-metric that satisfies these
  conditions a \term{metric Abelian monoid}.  It follows from the
  shift-invariance property that for any
  $\gamma_{1},\gamma_{2},\gamma_{3}\in\Gamma$ holds
  \[\tageq{shifts}
    \dist(\gamma_{1}\otimes\gamma_{3},\gamma_{2}\otimes\gamma_{3})
    \leq
    \dist(\gamma_{1},\gamma_{2})
  \]
  and for any quadruple
  $\gamma_{1},\gamma_{2},\gamma_{3}, \gamma_{4} \in\Gamma$ holds
  \[\tageq{subadditive}
    \dist(\gamma_{1}\otimes\gamma_{2},\gamma_{3}\otimes\gamma_{4})
    \leq
    \dist(\gamma_{1},\gamma_{3}) + \dist(\gamma_{2},\gamma_{4})
  \]
  and, in particular, the monoid operation is 1-Lipschitz with
  respect to each argument.



  As a direct consequence, for every $n \in \Nbb$, and $\gamma_1,
  \gamma_2 \in \Gamma$ also holds
  \[\tageq{contractiondist}
    \dist(\gamma_1^{\otimes n}, \gamma_2^{\otimes n} ) 
    \leq 
    n\dist(\gamma_1, \gamma_2)
  \]

  For a sequence $\bar\gamma=\set{\gamma(i)}\in\Gamma^{\Nbb_{0}}$
  define its \term{defect} with respect to the distance function
  $\dist$ by
  \[
    \defect_{\dist}(\bar\gamma)
    =
    \sup_{i,j\in\Nbb_{0}}\dist\big(\gamma(i+j),\gamma(i)\otimes\gamma(j)\big)
  \] 
 
  The sequence $\bar\gamma$ will be called \term[linear
  sequence]{$\dist$-linear} if $\defect_{\dist}(\bar\gamma)=0$,
  and \term[quasi-linear sequence]{$\dist$-quasi-linear} if
  $\defect_{\dist}(\bar\gamma)<\infty$.  Denote by
  $\lin_{\dist}(\Gamma)$ and $\qlin_{\dist}(\Gamma)$ the sets of all
  linear and, respectively, quasi-linear sequences in $\Gamma$ with
  respect to the distance $\dist$.

  For two elements $\bar\gamma_1,\bar\gamma_{2}\in\qlin_{d}(\Gamma)$, define an
  asymptotic distance between them by
  \[
    \dista(\bar\gamma_{1},\bar\gamma_{2})
    :=
    \lim_{n\to\infty}\frac1n\dist\big(\gamma_{1}(n),\gamma_{2}(n)\big)
  \]

  \begin{lemma}{p:adistonql}
    For a pair $\bar\gamma_{1},\bar\gamma_{2}\in\qlin_{\dist}(\Gamma)$
    the limit
    \[
      \lim_{n\to\infty}\frac1n\dist\big(\gamma_{1}(n),\gamma_{2}(n)\big)
    \]
    exists and is finite.
  \end{lemma}
  We provide the proof in Section~\ref{s:technical} on
  page~\pageref{p:adistonql.rep}. 

  The bivariate function $\dista$ is a pseudo-distance on the set
  $\qlin_{\dist}(\Gamma)$. We call two sequences
  $\bar\gamma_{1},\bar\gamma_{2}\in\qlin_{\dist}(\Gamma)$
  \term[asymptotic equivalence]{asymptotically equivalent} if
  $\dista(\bar\gamma_{1},\bar\gamma_{2})=0$ and write 
  \[
  \bar \gamma_{1} \aseq{\dista}  \bar \gamma_2
  \]
  



  We will call a sequence $\bar\gamma$ \term[weakly quasi-linear
  sequence]{weakly quasi-linear}, if it is asymptotically equivalent
  to a quasi-linear sequence. Note that the space of all weakly
  quasi-linear sequences can also be endowed with the asymptotic
  distance and it is isometric to the space of quasi-linear
  sequences. As we will see later all the natural operations we consider are
  $\dista$-Lipschitz and therefore coincide for the asymptotically
  equivalent sequences. Thus given a weakly quasi-linear sequence we
  could always replace it by an equivalent quasi-linear sequence
  without any visible effect. Thus, we take the liberty to omit the
  adverb ``weakly''. Whenever we say quasi-linear sequence, we mean
  a weakly quasi-linear sequence, that is silently replaced
  by an asymptotically equivalent genuine quasi-linear sequence, if
  necessary.

  The validity of the following constructions is very easy to verify,
  so we omit the proofs.

  The set $\qlin_{\dist}(\Gamma)$ admits an action of the
  multiplicative semigroup $(\Rbb_{\geq0},\,\cdot\,)$
  defined in the following way. Let $\lambda\in\Rbb_{\geq 0}$ and
  $\bar\gamma=\set{\gamma(n)}\in\qlin_{\dist}(\Gamma)$. Then define
  the action of $\lambda$ on $\bar\gamma$
  by 
  \[\tageq{R-action} 
  {\bar\gamma}^\lambda
  := 
  \set{\gamma(\lfloor\lambda\cdot n\rfloor)}_{n\in\Nbb_{0}} 
  \]
  This is only an action up to asymptotic equivalence. 
  Similarly, in the constructions that follow we are tacitly assuming they are valid up to asymptotic equivalence.

  The action 
  \[
  \cdot:\Rbb_{\geq 0}\times(\qlin_{\dist}(\Gamma),\dista)
  \to
  (\qlin_{\dist}(\Gamma),\dista)
  \]
  is continuous with respect to $\dista$ and, moreover it is
  a homothety (dilation), that is
  \[
  \dista(\bar\gamma_{1}^\lambda,\bar\gamma_{2}^\lambda)
  =
  \lambda\cdot\dista(\bar\gamma_{1},\bar\gamma_{2})
  \]

  The group operation $\otimes$ on $\Gamma$ induces a
  $\dista$-continuous (in fact, 1-Lipschitz) group
  operation on $\qlin_{\dist}(\Gamma)$ by multiplying sequences
  element-wise.  The semigroup structure on $\qlin_{\dist}(\Gamma)$
  is distributive with respect to the $\Rbb_{\geq 0}$-action
  \begin{align*}
    (\bar\gamma_{1}\otimes\bar\gamma_{2})^\lambda
    &=
    \bar\gamma_{1}^{\lambda}\otimes\bar\gamma_{2}^{\lambda}\\
   \bar\gamma^{ \lambda_{1}+\lambda_{2}}
    &\aseq{\dista}
    \bar\gamma^{\lambda_1}\otimes \bar\gamma^{\lambda_2}
  \end{align*}

  In particular for $n \in \mathbb{N}$
  \[
  \bar{\gamma}^n \aseq{\dista} \bar\gamma^{\otimes n}
  \]
  

  The path
  \[
  [0,1] \ni \lambda \mapsto \bar\gamma_1^{1-\lambda} \otimes \bar\gamma_2^{\lambda}
  \]
  will be called a convex interpolation and is a constant-speed
  $\dista$-geodesic between $\bar\gamma_1$ and $\bar\gamma_2$, that is
  for $\lambda \in [0,1]$,
  \begin{align*}
  \dista(\bar\gamma_1^{(1-\lambda)} \otimes \bar\gamma_2^{\lambda} , 
         \bar\gamma_1) 
  &= 
  \lambda \dista(\bar\gamma_1,\bar\gamma_2)
  \\
    \dista(\bar\gamma_1^{(1-\lambda)} \otimes \bar\gamma_2^{\lambda}, 
           \bar\gamma_2) 
    &= 
    (1-\lambda) \dista(\bar\gamma_1,\bar\gamma_2)
  \end{align*}

\subsubsection{Conditions for completeness}
  We would like to call
  \[
  \Gamma^{(\infty)}_{\dist}
  :=
  (\qlin_{\dist}(\Gamma),\otimes,\cdot,\dista)
  \]    
  the asymptotic cone of $(\Gamma,\otimes,\dist)$. However it is not
  clear in general, whether $\Gamma^{(\infty)}_{\dist}$ is a complete
  space.
 
  We can simply consider the metric completion, and call it the asymptotic
  cone of $(\Gamma,\otimes,\dist)$. We feel, however, that it adds
  just another level of obscurity as to what the points of
  $\Gamma^{(\infty)}_{\dist}$ are.

  Under some circumstances, however, the completeness of the space of
  quasi-linear sequences comes for free. This is the subject of the
  proposition below.

  Suppose the metric Abelian monoid $(\Gamma,\otimes,\dist)$ has an
  additional property: There exists a constant $C>0$, such that for
  any quasi-linear sequence $\bar\gamma\in\qlin_{\dist}(\Gamma)$,
  there exists an asymptotically equivalent quasi-linear sequence
  $\bar\gamma'$ with defect bounded by $C$.  If this is the case, we
  say that the metric monoid $(\Gamma,\otimes,\dist)$ has the
  ($C$-)\term{uniformly bounded defect property}.
    

  \begin{proposition}{p:boundeddefect}
    Suppose $(\Gamma,\otimes,\dist)$ is a metric Abelian monoid such that 
    \begin{enumerate}
    \item
      \label{pi:homogeneity} 
      the distance function $\dist$ is homogeneous, that is
      for any $\gamma_{1},\gamma_{2}\in\Gamma$ and $n\in\Nbb_{0}$ 
      \[
      \dist(\gamma_{1}^{\otimes n},\gamma_{2}^{\otimes n})
      =
      n\cdot\dist(\gamma_{1},\gamma_{2})
      \]
    \item
      \label{pi:boundeddefect} 
      $(\Gamma,\otimes,\dist)$ has the uniformly bounded defect property.
    \end{enumerate}
    Then the space $(\qlin_{\dist}(\Gamma),\dista)$ is complete.
  \end{proposition}
  The proof of the proposition can be found on
  page~\pageref{p:boundeddefect.rep}.

\subsubsection{On the density of linear sequences}

  In Section~\ref{s:ac-aep-conf} we have shown that Bernoulli sequences of
    configurations can be approximated by sequences of homogeneous
    configurations. The proposition below will allow us to extend this
    statement to a wider class of sequences.  It gives a sufficient condition under which
  the linear sequences are dense in the quasi-linear sequences.

  \begin{proposition}{p:eps-linear-dense}
    Suppose $(\Gamma, \otimes, \dist)$ has the $\epsilon$-uniformly
    bounded defect property for every $\epsilon > 0$. Then
    $\lin_{\dist}(\Gamma)$ is dense in $\qlin_{\dist}(\Gamma)$
  \end{proposition}
  See page~\pageref{p:eps-linear-dense.rep} for the proof.

\subsubsection{Asymptotic metric on original semigroup}
\label{suse:metric-original-group}
  Starting with an element $\gamma\in\Gamma$ one can construct a
  linear sequence $\linseq\gamma=\set{\gamma^{\otimes
      i}}_{i\in\Nbb_{0}}$. In view of inequality
  (\ref{eq:contractiondist}), this map is a contraction
  \[\tageq{inclusions}
  \big(\Gamma, \dist \big) \to \big(\lin_{\dist}(\Gamma), \dista\big)
  \]
 
  By the inclusions in (\ref{eq:inclusions}) we have an induced metric
  $\dista$ on $\Gamma$, satisfying for any
  $\gamma_{1},\gamma_{2}\in\Gamma$
  \[\tageq{deltalessd}
  \dista(\gamma_{1},\gamma_{2})\leq\dist(\gamma_{1},\gamma_{2})
  \]
  and the following scale-invariance condition is gained
  \[\tageq{deltalin}
  \dista(\gamma_{1}^{\otimes n},\gamma_{2}^{\otimes n})
  =
  n\cdot\dista(\gamma_{1},\gamma_{2})
  \]
   for all $n \in \Nbb_0$.

  Note moreover that if $\dist$ was scale-invariant to begin with,
  then $\dista$ coincides with $\dist$ on $\Gamma$.


\subsubsection{Iteration of construction}
\label{suse:iterate-construction}
  We may now iterate the constructions above, that is, we may apply
  them to $(\Gamma, \dista)$ instead of $(\Gamma, \dist)$.  One may
  wonder what is the purpose.  However, we have already observed that
  $\dista$ satisfies the scale-invariance condition
  (\ref{eq:deltalin}), which is one of the conditions going into a
  proof of completeness in Proposition \ref{p:boundeddefect}.
  Moreover, when we will later apply the theory in this section to the
  particular case of $\Gamma = \prob\<\Gbf>$, we will see that
  \[
  (\Gamma, \dista) = ( \prob\<\Gbf> , \aikd )
  \]
  and we will show that the latter space has the $\epsilon$-uniformly
  bounded defect property for every $\epsilon > 0$.
 
  By virtue of the bound $\dista \leq \dist$, sequences that are
  quasi-linear with respect to $\dista$, are also quasi-linear with
  respect to $\dist$.  Since $\dista$ is scale-invariant, the
  associated asymptotic distance $\distaa$ coincides with $\dista$ on
  $\Gamma$. We will show (in Lemma \ref{p:dist-dista-isometry} below)
  that $\distaa$ also corresponds to $\dista$ on $\dist$-quasi-linear
  sequences.
 
  In order to organize all these statements, and to be more precise,
  let us include the spaces in the following commutative diagram.

  \[\tageq{tropical-diagram-0}
  \begin{tikzcd}[ampersand replacement=\&,row sep=tiny]
    \mbox{}
    \&
    \big(\lin_{\dist}(\Gamma), \dista)
    \arrow[hookrightarrow]{dd}{\i_{1}}
    \arrow[hookrightarrow]{r}{\j_{1}}
    \&
    \big(\qlin_{\dist}(\Gamma), \dista )
    \arrow[hookrightarrow]{dd}{\i_{2}}
    \\
    (\Gamma,\dista)
    \arrow{ru}{f}
    \arrow{rd}{\phi}
    \\
    \mbox{}
    \&
    \big(\lin_{\dista}(\Gamma),\distaa)
    \arrow[hookrightarrow]{r}{\j_{2}}
    \&
    \big(\qlin_{\dista}(\Gamma),\distaa)
  \end{tikzcd}
  \]

  The maps $f, \phi$ and $\i_1$ are isometries.  The maps $\j_1$ and
  $\j_2$ are isometric embeddings.  The next lemmas show that $\i_2$
  is also an isometric embedding, and it has dense image.

  \begin{lemma}{p:dist-dista-isometry}
    The natural inclusion 
    \[
    \i_{2}:(\qlin_{\dist}(\Gamma),\dista)
    \into
    (\qlin_{\dista}(\Gamma),\distaa)
    \]
    is an isometric embedding.
  \end{lemma}
  
  \begin{lemma}{p:dist-dista-dense}
    The image of the isometric embedding
    \[
    \i_{2}:(\qlin_{\dist}(\Gamma),\dista)
    \into
    (\qlin_{\dista}(\Gamma),\distaa)
    \]
    is dense in $(\qlin_{\dista}(\Gamma),\distaa)$
  \end{lemma}
  The proofs of the two lemmas above are to be found on
  page~\pageref{p:dist-dista-isometry.rep}.

\subsection{Tropical probability spaces and configurations}
  Now we apply the above construction to the space of complete
  configurations with fixed combinatorial type $\Gbf$.

  Fix a complete diagram category $\Gbf$ and consider the space $\prob\<\Gbf>$
  of configurations modeled on $\Gbf$. 
  It carries the following structures:
  \begin{enumerate}
  \item A pseudo-metric $\ikd$ or $\aikd$.
  \item A 1-Lipschitz tensor product $\otimes$.
  \item A 1-Lipschitz entropy function
    $\ent_{*}:\prob\<\Gbf>\to\Rbb^{\size{\Gbf}}$.
  \end{enumerate}

  The tensor product of configurations is commutative from a metric
  perspective.  Recall that in Corollary~\ref{p:subadditivity} the
  subadditivity of both $\ikd$ and $\aikd$ was established, namely for
  any $\Xcal, \Ycal, \Ucal, \Vcal \in \prob\<\Gbf>$ holds
  \[
  \ikd(\Xcal \otimes \Ucal, \Ycal \otimes \Vcal)
  \leq 
  \ikd(\Xcal, \Ycal) + \ikd( \Ucal, \Vcal ).
  \]	
  and 
  \[
  \aikd(\Xcal \otimes \Ucal, \Ycal \otimes \Vcal)
  \leq 
  \aikd(\Xcal, \Ycal) + \aikd( \Ucal, \Vcal ).
  \]

  The space $(\prob\<\Gbf>, \otimes, \ikd)$ is a metric Abelian
  monoid.  Note also that $\hat{\ikd} = \aikd$ on $\prob\<\Gbf>$,
  along the lines of Section \ref{suse:metric-original-group}.
  
  However, the metric $\ikd$ is not scale-invariant. Moreover, it is
  unclear whether the metric semigroup $(\prob\<\Gbf>, \otimes ,\ikd)$
  has the uniformly bounded defect property. This is why we iterate
  the construction, as announced in Section
  \ref{suse:iterate-construction}, and consider the space of
  $\aikd$-quasi-linear sequences instead.
  
  \begin{lemma}{p:unifsmalldefectaikd}
    For a complete diagram category, and for every $\epsilon > 0$, the
    space $(\prob\<\Gbf>,\otimes,\aikd)$ has the $\epsilon$-uniformly
    bounded defect property, that is for any $\aikd$-quasi-linear
    sequence $\bar\Xcal\in\qlin_{\aikd}(\prob\<\Gbf>)$ there exists an
    asymptotically equivalent sequence $\bar\Ycal$ with defect not
    exceeding $\epsilon$.
  \end{lemma}

  By applying the general setup in the previous section to the metric
  semigroups $(\prob\<\Gbf>, \otimes, \ikd)$ and $(\prob\<\Gbf>,
  \otimes, \aikd)$ and as a corollary to Lemma
  \ref{p:unifsmalldefectaikd} we obtain the following theorem.

  \begin{theorem}{p:corollary-acone-ikd-aikd}
    Consider the commutative diagram
    \[\tageq{tropical-diagram}
    \begin{tikzcd}[ampersand replacement=\&,row sep=tiny]
      \mbox{}
      \&
      \big(\lin_{\ikd}(\prob\<\Gbf>) , \hat\ikd \big)
      \arrow[hookrightarrow]{dd}{\i_{1}}
      \arrow[hookrightarrow]{r}{\j_{1}}
      \&
      \big(\qlin_{\ikd}(\prob\<\Gbf>) , \hat\ikd\big)
      \arrow[hookrightarrow]{dd}{\i_{2}}
      \\
      (\prob\<\Gbf>,\aikd)
      \arrow{ru}{f}
      \arrow{rd}{\phi}
      \\
      \mbox{}
      \&
      \big(\lin_{\aikd}(\prob\<\Gbf>), \hat\aikd \big)
      \arrow[hookrightarrow]{r}{\j_{2}}
      \&
      \big(\qlin_{\aikd}(\prob\<\Gbf>), \hat\aikd \big)
    \end{tikzcd}
    \]
    Then the following statements hold:
    \begin{enumerate}
    \item 
      The maps $f, \phi, \i_1$ are isometries.
    \item 
      The maps $\i_2, \j_1,\j_2$ are isometric embeddings and each map
      has a dense image in the corresponding target space.
    \item 
      The space in the lower-right corner,
      $\big(\qlin_{\aikd}(\prob\<\Gbf>), \hat\aikd \big)$, is
      complete.
    \end{enumerate}
  \end{theorem} 
  
  We may finally define the space of \term[tropical
    configuration]{tropical $\Gbf$-configurations}, as the space in
  the lower-right corner of the diagram
  \[
  \prob\<\Gbf>^{(\infty)}
  :=
  \big(\qlin_{\aikd}(\prob\<\Gbf>),\otimes,\cdot,\hat\aikd\big)
  \]
  By the Theorem~\ref{p:corollary-acone-ikd-aikd} above, this space is
  complete.
  
  The entropy function $\ent_{*}:\prob\<\Gbf>\to\Rbb^{\size{\Gbf}}$
  extends to a linear functional
  \[
  \ent_{*}:\prob\<\Gbf>^{(\infty)}\to(\Rbb^{\size{\Gbf}},|\,\cdot\,|_{1})
  \]
  of norm one, defined by
  \[
  \ent_{*}(\bar\Xcal) = \lim_{n\to \infty } \frac1n \ent_{*} \big(\Xcal(n)\big)
  \]
  
  \subsubsection{Sequences of homogeneous configurations are dense.}
  Let $\tilde\lin_{\ikd}(\prob\<\Gbf>_{\hbf})$ stand for the weakly linear
  sequences of homogeneous configurations, that is those sequences,
  that are asymptotically equivalent to a linear sequence (not
  necessarily of homogeneous spaces).
  
  For a sequence of homogeneous spaces
  $\bar\Hcal \in \tilde\lin_{\ikd}(\prob\<\Gbf>_{\hbf})$ define
  $\aep( \bar\Hcal )$ to be a $\ikd$-linear sequence asymptotically
  equivalent to $\bar\Hcal$.
  
  Now we can extend the commutative
  diagram~(\ref{eq:tropical-diagram}) as follows
  \[\tageq{tropical-diagram-hom}
  \begin{tikzcd}[ampersand replacement=\&,row sep=tiny]
  \tilde\lin_{\ikd}(\prob\<\Gbf>_{\hbf})
  \arrow{r}{\aep}
  \&
  \lin_{\ikd}(\prob\<\Gbf>)
  \arrow[hookrightarrow]{dd}{\i_{1}}
  \arrow[hookrightarrow]{r}{\j_{1}}
  \&
  \qlin_{\ikd}(\prob\<\Gbf>)
  \arrow[hookrightarrow]{dd}{\i_{2}}
  \\
  (\prob\<\Gbf>,\aikd)
  \arrow{ru}{f}
  \arrow{rd}{\phi}
  \\
  \mbox{}
  \&
  \lin_{\aikd}(\prob\<\Gbf>)
  \arrow[hookrightarrow]{r}{\j_{2}}
  \&
  \prob\<\Gbf>^{(\infty)}    
  \end{tikzcd}
  \]
  
  
  By the Asymptotic Equipartition Property for configurations,
  Theorem~\ref{p:aep-complete}, the map $\aep$ is an isometry, hence
  we have the following theorem.
    \begin{theorem}{p:homo-dense}
    The map
    \[
    \j_2\circ\i_1\circ\aep:
    \tilde\lin_{\ikd}(\prob\<\Gbf>_{\hbf})\to\prob\<\Gbf>^{(\infty)}
    \]
    is an isometric embedding with dense image.
  \end{theorem}

  Let $\prob\<\Gbf>_{\hbf}^{(\infty)} \subset \prob\<\Gbf>^{(\infty)}$
  denote the space of weakly quasi-linear sequences of configurations
  $\bar{\Hcal} \in \prob\<\Gbf>^{(\infty)}$, such that for every
  $n \in \Nbb_0$, the configuration $\Hcal(n)$ is homogeneous.  We
  will refer to $\prob\<\Gbf>_{\hbf}^{(\infty)}$ as the space of
  homogeneous tropical configurations.

  Denote by $\aep$ the embedding 
  \[
    \aep: 
    \prob\<\Gbf>_{\hbf}^{(\infty)} 
    \hookrightarrow 
    \prob\<\Gbf>^{(\infty)}
  \]  
  
  \begin{theorem}{p:aep-tropical-config}{\rm(Asymptotic Equipartition Theorem for tropical configurations)}
  Let $\Gbf$ be a complete diagram category. Then the map 
  \[
  \aep: \prob\<\Gbf>_{\hbf}^{(\infty)} \hookrightarrow \prob\<\Gbf>^{(\infty)}
  \]
  is an isometry.
  \end{theorem}
  
  \begin{Proof}
  	 We need to show that for every tropical configuration $\bar\Xcal \in \prob\<\Gbf>^{(\infty)}$, there 
  	exists a homogeneous tropical configuration $\bar \Hcal \in \prob\<\Gbf>_\hbf^{(\infty)}$ such that
  	\[
  	\hat{\aikd}(\bar\Hcal, \bar\Xcal) = 0
  	\]
  By Lemma \ref{p:unifsmalldefectaikd} and Proposition \ref{p:eps-linear-dense}, for every $j \in \Nbb$ there exists a sequence $\bar\Ycal_j \in \lin_{\aikd}(\prob\<\Gbf>) \cong \lin_{\ikd}(\prob\<\Gbf>)$	such that 
  \[
  \hat{\aikd}(\bar{\Ycal}_j, \bar\Xcal) \leq \frac{1}{j}
  \]
  By the Asymptotic Equipartition Property for configurations, Theorem \ref{p:aep-complete},
  there are sequences of homogeneous configurations $\bar\Hcal_j$ such that 
  \[
  \hat{\aikd}(\bar\Ycal_j, \bar\Hcal_j) = 0
  \]
  Define $\mathbf{i}(j)$ such that for all $k \geq \mathbf{i}(j)$
  \[
  \frac{1}{k} \aikd(\Ycal_j(k), \Hcal_j(k)) \leq \frac{1}{j}
  \]
  and  moreover
  \[
  \frac{1}{k} \aikd(\Ycal_j(k), \Xcal(k)) \leq \frac{2}{j}
  \]
  The function $\mathbf{i}$ can be chosen monotonically increasing. 
  For every $i \in \Nbb_0$ there is a unique $\jbf(i) \in \Nbb_0$ such that 
  \[
  \mathbf{i}\big(\jbf(i)\big) \leq i < \mathbf{i}\big(\jbf(i)+1\big)
  \]
  Define then
  \[
  \Hcal(k) = \Hcal_{\jbf(k)}(k)
  \]
  It follows that for $k > \mathbf{i}(1)$,
  \[
  \frac{1}{k}\aikd(\Hcal(k), \Xcal(k)) \leq \frac{3}{\jbf(k)}
  \]
  Since $\jbf$ is a non-decreasing, divergent sequence, the theorem follows.
  \end{Proof}  

  \bigskip
  
  Thus we have shown that all arrows in
  diagram~(\ref{eq:tropical-diagram-hom}) are isometric embeddings
  with dense images.  We would like to conjecture, that, in fact, they
  are all isometries.  In any case, the difference between metrics
  $\hat\aikd$ and $\aikd$ is so small ($\aikd$ is defined on the dense
  subset of the domain of definition of $\hat\aikd$ and they coincide
  whenever both are defined), that we will not write the hat anymore
  and just use notation $\aikd$ for the metric on
  $\prob\<\Gbf>^{(\infty)}$.
  


\subsection{Tropical probability spaces and tropical chains}
\label{se:tropprobtropchains}
  In this section we evaluate the spaces $\prob^{(\infty)}$ and
  $\prob\<\Cbf_{n}>^{(\infty)}$, where $\Cbf_n$ is a chain, which is
  the diagram category introduced in \ref{s:config-examples-chain} on
  page \pageref{s:config-examples-chain}.

  Recall that a finite probability space $U$ is homogeneous if
  $\Aut(U)$ acts transitively on the support of the measure. The property
  of being homogeneous is invariant under isomorphism and every
  homogeneous space is isomorphic to a probability space with the
  uniform distribution. 

  Homogeneous chains also have a very simple description. A chain of
  reductions is homogeneous, if and only if all the individual spaces
  are homogeneous.

  This simple description allows us to evaluate explicitly the
  Kolmogorov distance on the spaces of weakly linear sequences of
  homogeneous chains and consequently the space of tropical chains.

  \begin{theorem}{p:tropical-simple}\rule{0mm}{1mm}\\
    \begin{enumerate}
    \item 
      $\displaystyle
      \prob^{(\infty)}
      \cong
      (\Rbb_{\geq0},|\cdot - \cdot|,+,\cdot)
      $
      \vspace{3mm}
    \item 
      $\displaystyle
      \prob\<\Cbf_{n}>^{(\infty)}
      \cong
      \set{
        \left(
        \begin{matrix}
          x_{1}\\\vdots\\x_{n}
        \end{matrix}
        \right) \in\Rbb^{n} \st 0\leq x_{n}\leq \cdots\leq x_{1}} 
      $
      \vspace{3mm}
      \\ 
      where the right-hand side is a cone in $(\Rbb^{n},|\,\cdot\,|_{1})$.
    \end{enumerate}
  \end{theorem}

  To prove the Theorem~\ref{p:tropical-simple} we evaluate first
  the isometry class of the space of weakly linear sequences of
  homogeneous spaces (or chains). We will only present an argument for
  single spaces, since the argument for chains is very similar.

  \begin{lemma}{p:tropical-homo-simple}
    \[
    \tilde\lin_{\ikd}(\prob_{\hbf})
    \cong
    (\Rbb_{\geq0},|\cdot - \cdot|,+,\cdot)
    \]
  \end{lemma}

  Note that the right-hand side is a complete metric space, thus the
  Asymptotic Equipartition Theorem for tropical configurations,
  Theorem ~\ref{p:homo-dense}, together with
  Lemma~\ref{p:tropical-homo-simple}, imply
  Theorem~\ref{p:tropical-simple}.
 
  To prove Lemma~\ref{p:tropical-homo-simple} we need to evaluate the
  Kolmogorov distance between two homogeneous spaces, or chains of
  homogeneous spaces. This is the
  subject of the next lemma, from which
  Lemma~\ref{p:tropical-homo-simple} follows immediately.


\begin{lemma}{p:ikduniform}
  Denote by $U_{n}$ a finite uniform probability space of cardinality
  $n$, then
  \begin{enumerate}
  \item 
    \[
    \ikd(U_{n},U_{m})\leq 2 \ln 2 + \left|\ln\frac nm\right|
    \]
  \item
    \[
    \aikd(U_{n},U_{m})=|\ent(U_{n})-\ent(U_{m})|
    \]
  \end{enumerate}
\end{lemma}


\subsection{Stochastic processes}
  Often, stochastic processes naturally give rise to
  $\aikd$-quasi-linear sequences.  We include this last subsection as
  an indication that our statements, together with the construction of
  the tropical cone, have a much larger reach than sequences of
  independent random variables.  We will be brief, and come back to
  the topic in a subsequent article.

  For a minimal diamond configuration
  \[
  \begin{tikzcd}[row sep=small, column sep=small]
    \mbox{}
    & 
    C
    \arrow{dl}{}
    \arrow{dr}{}
    \arrow{dd}{}
    &
    \\
    A
    \arrow{dr}{}
    & 
    &
    B
    \arrow{dl}{}
    \\
    \mbox{}
    &
    D
    &
  \end{tikzcd}
  \]
  we define the conditional mutual information between $A$ and $B$ given $D$ by
  \[
  \MI(A; B \, \rel \, D) := \ent(A) + \ent(B) - \ent(C) - \ent(D)
  \]
  Shannon's inequality (\ref{eq:shannonineq}) says that the
  conditional mutual information is always non-negative.  Any minimal
  two-fan $A \ot C \to B$ can be completed to a diamond with the
  one-point probability space $\set{\bullet}$ as the terminal vertex,
  and the mutual information between $A$ and $B$ is defined as
  \[
  \MI(A; B) = \MI(A; B \, \rel \, \set{ \bullet }) = \ent(A) + \ent(B) - \ent(C)
  \]
  
  Let 
  \[
  \ldots, X_{-1}, X_0, X_1, \ldots
  \]
  be a stationary stochastic process with finite state space $\un
  X$. Thus, for any $I=\set{k,k+1,\ldots,l}\subset\Zbb$ we have
  jointly distributed random variables $X_{k},\ldots,X_{l}$, that
  generate a full configuration
  \[
  \Xcal_{I}
  =
  \langle X_{k},\ldots,X_{l}\rangle
  =
  \set{X_{J}}_{J\subset I}
  \]
  as explained in Section~\ref{s:config-examples-full}.
  This collection of full configurations is consistent in the sense that
  for $k\leq k'\leq l'\leq l$ there are canonical isomorphisms  
  \[
  \Xcal_{I'}
  \cong 
  R^{*}_{I',I}\Xcal_{I}
  \]
  where $I:=\set{k,\ldots,l}$, $I':=\set{k',\ldots,l'}$ and
  $R^*_{I',I}$ is the restriction operator introduced in
  Section~\ref{s:config-restrictions-fullfull}.
  
  The property of being
  stationary means that there are canonical isomorphisms for any finite
  subset $I\subset\Zbb$ and $l\in\Zbb$
  \[
  \Xcal_{I}\too[\cong]\Xcal_{I+l}
  \]
  
  For $I=\set{k,k+1,\ldots,l}$ we call the initial space $X_{I}$ of the
  configuration $\Xcal_{I}$ the
  space of trajectories of the process over $I$ and denote it $X_{k}^{l}$.
  
  Note that by stationarity, for every $m \in \Zbb$, $k\in
  \Nbb$ and $l \in \Nbb_0$,
  \[
  \MI(X_m^{m+k-1}; X_{m+k}^{m+k + l-1}) = \MI(X_{-k+1}^{0}; X_{1}^l )
  \]
  Moreover the right-hand side is an increasing function of both $k$
  and $l$.  We make the following important observation.  The defect
  of the sequence $n \mapsto X_1^n$ is equal to
  \begin{align*}
    \defect_{\aikd}\left( \{ X_1^n\} \right) 
    &= 
    \sup_{m, n \in \Nbb_0} 
    \aikd\left( X_1^{m+n}, X_1^m \otimes X_1^n\right) 
    \\
    &= 
    \sup_{m, n \in \Nbb_0} 
    \left| \ent(X_1^m \otimes X_1^n) - \ent(X_1^{m+n}) \right| 
    \\
    &= 
    \sup_{m, n \in \Nbb_0} 
    \left| \ent(X_{-m+1}^0 \otimes X_{1}^{n}) - \ent(X_1^{m+n})
    \right| 
    \\  
    &= 
    \sup_{m, n \in \Nbb_0} \MI(X_{-m+1}^0, X_1^n)
  \end{align*}
  Therefore, the sequence $n \mapsto X_1^n$ is $\aikd$-quasi-linear if
  and only if
  \[\tageq{stochastic-ql}
    \lim_{k,l\to \infty}\MI(X_{-k+1}^{0}, X_1^l) < \infty
  \]
  
  Once condition~(\ref{eq:stochastic-ql}) is satisfied for a stochastic
  process, it defines a tropical probability space $\bar X\in\prob$. 
  
  Note that condition~(\ref{eq:stochastic-ql}) is satisfied for
  any stationary, finite-state Markov chains.









