  In this section we recall some elementary inequalities for
  (relative) entropies and the total variation distance for
  distributions on finite sets. Furthermore, we generalize the notion
  of a probability distribution on a set to a distribution on a
  configuration of sets. Finally, we give a perspective on the theory
  of types, and also introduce types in the context of complete
  configurations.

\subsection{Distributions}
\label{s:disttypes-distributions}

\subsubsection{Single probability spaces}
\label{s:disttypes-distributions-single}
  For a finite set $S$ we denote by $\Delta S$ the collection of all
  probability distributions on $S$.  It is a unit simplex in the real
  vector space $\Rbb^S$.  We often use the fact that it is a compact,
  convex set, whose interior points correspond to fully supported
  probability measures on $S$.

  For $\pi_{1},\pi_{2}\in\Delta S$ denote by $|\pi_{1}-\pi_{2}|_1$ the total
  variation of the signed measure $(\pi_{1}-\pi_{2})$ and 
  define the entropy of the distribution $\pi_{1}$ by
  \[\tageq{entropyformula}
    h(\pi_{1})
    :=
    -\sum_{x\in X}\pi_{1}(x)\ln \pi_{1}(x)
  \]
  If, in addition, $\pi_{2}$ lies in the interior of
  $\Delta S$ define the relative entropy by
  \[\tageq{relativeentropy}
    D(\pi_{1}\sep \pi_{2})
    :=
    \sum_{x\in X}\pi_{1}(x)\ln \frac{\pi_{1}(x)}{\pi_{2}(x)}
  \]
  The entropy of a probability space is often defined through
  formula~(\ref{eq:entropyformula}).  It is a standard fact, and can
  be verified with the help of Lemma~\ref{p:typesestimates} below,
  that for $\pi\in\Delta S$ holds
  \[\tageq{entropiesequal}
    h(\pi)
    =
    \ent(S,\pi)
  \]
  which justifies the name ``entropy'' for the function $h:\Delta S\to\Rbb$.

  Define a \term{divergence ball} of radius $\epsilon>0$ centered at
  $\pi\in\Interior\Delta S$ as
  \[\tageq{divergenceball}
    B_{\epsilon}(\pi)
    :=
    \set{\pi'\in\Delta S\st D(\pi'\sep\pi)\leq \epsilon}
  \]
  For a fixed $\pi$ and $\epsilon<\!<1$ the ball $B_{\epsilon}(\pi)$
  also lies in the interior of $\Delta S$.
  \begin{lemma}{p:entropydivergence} 
  	Let $S$ be a finite set, then
    \begin{enumerate}
    \item
      \label{p:entropydivergence1}
      For any $\pi_{1},\pi_2\in\Delta S$, Pinsker's inequality holds
      \[
      |\pi_{1}-\pi_{2}|_1\leq\sqrt{2D(\pi_{1}\sep\pi_{2})}
      \]    
    \item
      \label{p:entropydivergence2} For any
      $\pi_{2}\in\Interior\Delta S$ there exists a positive constant
      $C = C_{\pi_2}$ such that for any $\pi_{1} \in \Delta S$, holds
      \[
      |\pi_{1}-\pi_{2}|_1
      \geq 
      C\sqrt{D(\pi_{1}\sep\pi_{2})}
      \]
    \item
      \label{p:entropydivergence3}  
      Suppose $\pi$ is a point in the interior of $\Delta S$ and $r>0$
      is such that $B_{r}(\pi)$ also lies in the interior of $\Delta
      S$. There exist a constant $C=C_{\pi,r}$ such that for any
      $\epsilon\leq r$ holds
      \[
      \max\set{|h(\pi_{1})-h(\pi_{2})|
        \st 
        \pi_{1},\pi_{2}\in B_{\epsilon}(\pi)}\leq C\sqrt{\epsilon}
      \]
    \end{enumerate}
  \end{lemma}
  
The first claim of the Lemma, Pinsker's inequality, is a well-known
inequality in for instance information theory, and a proof can be
found in \cite{Cover-Elements-1991}.

The second claim follows from the fact that for the fixed
$\pi_{2}\in\Interior\Delta S$ the relative entropy as the function
of the first argument is bounded, smooth on
the interior of the simplex and has a minimum at $\pi_{2}$.

To prove the last claim, note that the entropy function $h$ is
smooth in the interior of the simplex. Then this last claim follows
from the first claim.  
  
\subsubsection{Distributions on configurations}
\label{s:disttypes-distributions-config}
  A map $f:S\to S'$ between two finite sets
   induces an affine map $f_{*}:\Delta S\to\Delta S'$.

  
  For a configuration of sets $\Scal=\set{S_{i};f_{ij}}$ we define the
  \term[distribution on configuration of sets]{space of distributions
    on the configuration} $\Scal$ by
  \[
    \Delta\Scal
    :=
    \set{(\pi_{i})\in\prod_i\Delta S_{i}\st (f_{ij})_{*}\pi_{i}=\pi_{j}}
  \]
  Essentially, an element of $\Delta \Scal$ is a collection of
  distributions on the sets $S_i$ in $\Scal$ that is consistent with
  respect to the maps $f_{ij}$.  The consistency conditions
  $(f_{ij})_* \pi_i = \pi_j$ form a collection of linear equations
  with integer coefficients with respect to the standard convex
  coordinates in $\prod \Delta S_i$. Thus, $\Delta \Scal$ is a
  rational affine subspace in the product of simplices. In particular,
  $\Delta \Scal$ has a convex structure.
  
  If $\Scal$ is complete with initial set $S_{0}$, then specifying a
  distribution $\pi_{0}\in\Delta S_{0}$ uniquely determines
  distributions on all of the $S_{i}$'s by setting
  $\pi_{i}:=(f_{0i})_{*}\pi_{0}$. In such a situation we have
  \[\tageq{distribonconfig}
    \Delta\Scal\cong\Delta S_{0}
  \]
  If $\Scal$ is not complete and $S_{0},\ldots,S_{k}$ is a collection
  of its initial sets, then $\Delta\Scal$ is isomorphic to an affine
  subspace of the product $\Delta S_{0}\times\dots\times\Delta S_{k}$
  cut out by linear equations with integer coefficients corresponding
  to co-fans in $\Scal$ with initial sets among $S_{0},\ldots,S_{k}$.
  
  To simplify notation, for a probability space $X$ or a
  configuration $\Xcal$ we will write
  \begin{align*}
    \Delta X
    &:=
    \Delta\un X
    \\
    \Delta\Xcal
    &:=
    \Delta\un\Xcal
  \end{align*}
  
  \bigskip
  We now discuss briefly the theory of types. Types are special
  subspaces of tensor powers that consist of seqences with the same
  ``empirical distribution'' as explained in details below. For a
  more detailed discussion the reader is referred to
  \cite{Cover-Elements-1991} and \cite{Csiszar-Method-1998}. We generalize the
  theory of types to complete configurations of sets and complete
  configurations of probability spaces.

  The theory of types for configurations, that are not complete, is
  more complex and will be addressed in a
    subsequent article.


\subsection{Types for single probability spaces}
\label{s:disttypes-types-single}
  Let $S$ be a finite set. For $n\in\Nbb$ denote also
  \[
  \Deltan S
  :=
  \Delta S\cap \frac1n\Zbb^{S}
  \]
  a collection of rational points in $\Delta S$ with denominator $n$.
  (We say that a rational number $r\in\Qbb$ has denominator $n\in\Nbb$
  if $r\cdot n\in\Zbb$)

  Define the \term{empirical distribution map} $\emp:S^{n}\to\Delta S$,
  that sends $(s_{i})_{i=1}^{n}=\sbf\in S^{n}$ to the empirical
  distribution $\emp(\sbf)\in\Delta S$ given by 
  \[
  \emp(\sbf)(a)
  =
  \frac1n\cdot\big|\set{i\st s_{i}=a}\big|
  \quad
  \text{for any $a\in S$}
  \]
  Clearly the image of $\emp$ lies in $\Deltan S$.
  
  For $\pi\in\Deltan S$, the space $T^{\c n}_{\pi}S:=\emp^{-1}(\pi)$
  equipped with the uniform measure is called a \term{type} over $\pi$.
  The symmetric group $\Sbb_{n}$ acts on $S^{\otimes n}$ by permuting the
  coordinates. This action leaves the empirical distribution
  invariant and therefore could be restricted to each type, where it
  acts transitively. Thus, for $\pi\in\Deltan S$ the
  probability space $(T^{\c n}_{\pi}S,u)$ with $u$ being a uniform
  ($\Sbb_{n}$-invariant)
  distribution, is a homogeneous space.
  
  Suppose $X=(\un X,p)$ is a probability space.  Let $\tau_{n}$ be the
  pushforward of $p^{\otimes n}$ under the empirical distribution map
  $\emp:\un X^{n}\to\Delta X$ . Clearly $\supp\tau_{n}\subset\Deltan
  X$, thus $(\Delta X,\tau_{n})$ is a finite probability
  space.  Therefore we have a reduction
  \[
  \emp:X^{\otimes n}\to(\Delta X,\tau_{n})
  \]
  which we call the \term{empirical reduction}.
  If $\pi\in\Deltan X$ is such that $\tau_{n}(\pi)>0$, then 
  \[\tageq{typesiso}
  T^{\c n}_{\pi}\un X = X^{\otimes n}\rel\pi
  \]
  In particular, it follows that the right-hand side does not depend on
  the probability $p$ on $X$ as long as $\pi$ is ``compatible'' to it.
  
  The following lemma records some standard facts about types, which
  can be checked by elementary combinatorics and found in
  \cite{Cover-Elements-1991}.
  \newpage
  \begin{lemma}{p:typesestimates}
    Let $X$ be a probability space and $\xbf\in X^{\otimes n}$, then
    \begin{enumerate}
    \item 
      \[
      |\Deltan X|
      =
      \choose{n+|X|\\|X|}\leq \ebf^{|X| \cdot\ln (n+1)}
      \]
    \item
      \[
      p^{\otimes n}(\xbf)
      =
      \ebf^{-n\big[h\big(\emp(\xbf)\big)+D\big(\emp(\xbf)\sep p\big)\big]}
      \]
    \item
      \[
      \ebf^{n \cdot h(\pi) - |X| \cdot\ln(n+1)}\leq
      |T^{\c n}_{\pi} \un X|
      \leq
      \ebf^{n\cdot h(\pi)}
      \]
     \item
      \[
      \ebf^{-n\cdot D(\pi\sep p)-|X| \cdot\ln (n+1)}
      \leq
      \tau_n(\pi)=p^{\otimes n}(T^{\c n}_{\pi} \un X)
      \leq
      \ebf^{-n\cdot D(\pi\sep p)}
      \]
    \end{enumerate}
  \end{lemma}
  
  
  If $X=(\un X,p_{X})$ is a probability space with rational probability
  distribution with denominator $n$, then the type over $p_{X}$ will be
  called the true type of $X$
  \[
  T^{\c n}X
  :=
  T^{\c n}_{p_{X}}\un X
  \]
  
  As a corollary to Lemma \ref{p:typesestimates} and
  equation~(\ref{eq:entropiesequal}) we obtain the following.
  
  \begin{corollary}{p:entropy-type}
    For a finite set $S$ and $\pi\in\Deltan S$ holds
    \[
      n\cdot h(\pi)- |S| \cdot \ln (n+1)
      \leq
      \ent(T^{(n)}_{\pi}S)
      \leq
      n\cdot h(\pi)
    \]

    In particular, for a finite probability space $X=(S,p)$ with a rational
    distribution $p$ with denominator $n$ holds

    \[
      n\cdot\ent(X)- |S| \cdot \ln (n+1)
      \leq
      \ent(T^{(n)}X)
      \leq
      n\cdot \ent(X)
    \]
    
  \end{corollary}
  
  The following important theorem is known as Sanov's theorem. 
  It can be derived from Lemma
  \ref{p:typesestimates} and found in \cite{Cover-Elements-1991}.
  
  \begin{theorem}{p:sanov}{\rm (Sanov's Theorem)}
    Let $X$ be a finite probability space and let $\emp: X^{\otimes n}
    \to (\Delta X, \tau_n)$ be the empirical reduction.  Then for
    every $r>0$,
    \[
      \tau_n(\Delta X \backslash B_r(p)) 
      \leq 
      \ebf^{- n \cdot r + |X| \cdot \ln( n + 1 )}
    \]
    where $B_r(p)$ is the divergence ball (relative entropy ball)
    defined in (\ref{eq:divergenceball}).
  \end{theorem}

\subsection{Types for complete configurations}
\label{s:disttypes-types-config}
  In this subsection we generalize the theory of types for
  configurations modeled on a complete category. The theory for a
  non-complete configurations is more complex and will be addressed in
  our future work.  We will give three equivalent definitions of a
  type for a complete configuration, each of which will be useful in
  its own way. Before we describe the three approaches we need some
  preparatory material.
  \begin{lemma}{p:specialdiamond}
    Given a diamond configuration of probability spaces
    \[
    \Dcal=
    \left(
    \begin{tikzcd}[ampersand replacement=\&,row sep=small,column sep=small]
      X
      \arrow{d}{\rho_{1}}
      \arrow{r}{f}
      \&
      Y
      \arrow{d}{\rho_{2}}
      \\
      A
      \arrow{r}{g}
      \&
      B
    \end{tikzcd}
    \right)
    \]
    the following two conditions are equivalent
    \begin{enumerate}
    \item
      \label{p:specialdiamond-indep} 
      The minimal reduction of the diamond $\Dcal$ is isomorphic to the
      adhesion of its co-fan or equivalently the following independence
      condition holds
      \[
      A\indep Y\rel B
      \]
    \item
      \label{p:adhesion-cond}
      For any $a,a'\in \un A$ such that $f(a)=f(a')=b\in B$ holds
      \[
      Y\rel a=Y\rel a'=Y\rel b
      \]
    \end{enumerate}
  \end{lemma}

  Suppose $\Dcal$ is the diamond as in the
  Lemma~\ref{p:specialdiamond}.  The top row $X\to Y$ is a two-chain
  subconfiguration of $\Dcal$ and we can consider a conditioning by an
  element $a\in\un A$
  \[
  X\rel a\to Y\rel a
  \]
  If $\Dcal$ satisfies any of two conditions in
  Lemma~\ref{p:specialdiamond}, then $Y\rel a=Y\rel b$ for
  $b=g(a)$. Thus, we constructed a reduction 
  \[
  X\rel a\to Y\rel b  
  \]
  Suppose we have a reduction $f:X\to Y$ between a pair of probability
  spaces. Then for any $n\in\Nbb$ there is an induced reduction
  $f_{*}:(\Delta X,\tau_{n})\to(\Delta Y,\tau_{n})$ that can be
  included in the following diamond configuration
  \[
  \begin{tikzcd}[ampersand replacement=\&]
    X^{\otimes n} 
    \arrow{r}{f^{\otimes n}}
    \arrow{d}{\emp}
    \& 
    Y^{\otimes n} 
    \arrow{d}{\emp}
    \\
    (\Delta X,\tau_n)
    \arrow{r}{f_{*}}
    \& 
    (\Delta Y,\tau_{n})
  \end{tikzcd}
  \]
  that satisfies conditions in Lemma~\ref{p:specialdiamond}.
  It means that there is a reduction
  \[
  Tf:T^{(n)}_{\pi} X\to T^{(n)}_{\pi'}Y
  \]
  for $\pi\in\Deltan X$ and $\pi'=f_{*}\pi\in\Deltan Y$.

  Now we are ready to give the definitions of types.  Let
  $\Xcal\in\prob\<\Gbf>$ be a complete configuration,
  $\Xcal=\set{X_{i};f_{ij}}$ with initial space $X_{0}$ and let 
  $\pi\in\Deltan \Xcal$.
 

\subsubsection{Type of a configuration as the configuration of types.}
\label{s:type-config-analytic}
  Define the type $T^{(n)}_{\pi}\un\Xcal$ as the $\Gbf$-configuration,
  whose individual spaces are types of the individual spaces of
  $\Xcal$ over the corresponding push-forwards of $\pi$
  \[
  T^{(n)}_{\pi}\un\Xcal
  :=
  \set{T^{(n)}_{\pi_{i}}\un X_{i};T f_{ij}}
  \]
\subsubsection{Types as $\Sbb_{n}$-orbits in the tensor power}
\label{s:type-config-symmetric}
  By a \term{section} in $\Xcal$ we mean a consistent collection of points
  \[
  \xbf=(x_{0},\ldots,x_{\size{\Xcal}-1})\in \prod_{i=0}^{\size{\Xcal}-1}\un X_{i}
  \]
  such that $f_{ij}x_{i}=x_{j}$, whenever $f_{ij}$ is defined.  For
  any $j$ define the projection
  $\rho_{j}:\prod\un X_{i}\to\un X_{j}$, so that
  $\rho_{j}(\xbf)=x_{j}$.

  The symmetric group $\Sbb_{n}$ acts on the collection of sections in
  the tensor power $\Xcal^{\otimes n}$, by permuting the coordinates.
  Let $\Tcal\subset\prod\un X_{i}$ be an orbit of the action such that
  $\rho_{0}(\Tcal)=T^{(n)}_{\pi}\un X_{0}$.  Suppose
  that the pair $(i,j)$ is such, that $f_{ij}$ is defined. Since
  $f_{ij}^{\otimes n}:X_{i}^{\otimes n}\to X_{j}^{\otimes n}$ is
  $\Sbb_{n}$-equivariant, we have a map
  \[
  Tf:\rho_{i}(\Tcal)\to\rho_{j}(\Tcal)
  \]
  We can turn $\Tcal$ into a $\Gbf$-configuration, which we will call
  a type of $\Xcal$ 
  \[
  \hat T^{(n)}_{\pi}\un\Xcal:=\set{\rho_{i}(\Tcal),Tf_{ij}}
  \]
  where $\pi$ is the value of the impirical distribution on
  $\rho_{0}(\Tcal)$. 
  
  Since the initial space in $\hat T^{(n)}_{\pi}\un\Xcal$ coincides
  with the initial space in $T^{(n)}_{\pi}\un\Xcal$ and all the
  reductions coincide, we conclude that 
  \[
  \hat T^{(n)}_{\pi}\un\Xcal
  =
  T^{(n)}_{\pi}\un\Xcal
  \]

\subsubsection{Type as conditionining of the tensor power}
\label{s:type-config-synthetic}
  We can extend $\Xcal^{\otimes n}$ to a configuration $\hat{\Xcal}$ by
  adding $(\Delta \Xcal, \tau_n)$ and the empirical reduction
  $X_0^{\otimes n}\to (\Delta \Xcal, \tau_n)$.
  
  Let $\pi \in \Delta \Xcal$ with $\tau_n(\pi)>0$ and recall $\Delta
  X_0 \cong \Delta \Xcal$.  We may now define $\Xcal^{\otimes n} \rel
  \pi$ as in Section \ref{s:config-conditioning}.  
  Define a type of $\Xcal$ over $\pi\in\Deltan\Xcal$ by
  \[
  \check T^{(n)}_{\pi}\un\Xcal:=\Xcal^{\otimes n}\rel\pi
  \]
  
  By definition, it holds that
  \[
  X_0^{\otimes n} \rel \pi 
  = 
  T_\pi^{\c n} X_0
  \]
  Let $\pi_{i}=(f_{0i})_{*}\pi$.
  Using Lemma~\ref{p:specialdiamond} and discussion thereafter we
  conclude that 
  \[
  X_{i}^{\otimes n}\rel\pi=X_{i}^{\otimes n}\rel\pi_{i}
  \] 
  and therefore
  \[
  \check T^{(n)}_{\pi}\un\Xcal
  =
  T^{(n)}_{\pi}\un\Xcal
  \]
  
\subsubsection{The empirical two-fan}
\label{s:types-empirical-twofan}
  We construct a two-fan of $\Gbf$-configurations with terminal
  vertices $\Xcal^{\otimes n}$ and $(\Delta \Xcal,\tau_{n})^{\Gbf}$
  (the constant $\Gbf$-configuration in which every probability space
  is $(\Delta \Xcal, \tau_n)$, and every reduction is an identity)
  \[\tageq{empirical-2fan}
  \Rcal_{n}(\Xcal)
  =
  \left(
  \begin{tikzcd}[column sep=tiny, ampersand replacement=\&]
  \mbox{}
  \&
  \tilde\Xcal^{(n)}
  \arrow{ld}{}
  \arrow{rd}{}
  \&
  \mbox{}
  \\
  \Xcal^{\otimes n}
  \&
  \&
  (\Deltan \Xcal,\tau_{n})^{\Gbf}
  \end{tikzcd}
  \right)
  \]
  With the help of Lemma \ref{p:minimalfansconfig}, we construct
  $\Rcal_n(\Xcal)$ as the minimal reduction of the two-fan of
  $\Gbf$-configurations
  \[
  \begin{tikzcd}[column sep=tiny,ampersand replacement=\&]
  \mbox{}
  \&
  (X_{0}^{\otimes n})^{\Gbf}
  \arrow{dl}[swap]{f_{0*}^{n}}
  \arrow{dr}{\emp^{\Gbf}}
  \&
  \mbox{}
  \\
  \Xcal^{\otimes n}
  \&
  \&
  (\Delta\Xcal,\tau_{n})^{\Gbf}
  \end{tikzcd}
  \]

  Let $\pi \in (\Delta \Xcal, \tau_n)$ with $\tau_n(\pi) > 0$. 
  Then within $\Rcal_{n}(\Xcal)$ holds
  \[
  \Xcal^{\otimes n}\rel\pi=T^{(n)}_{\pi}\Xcal
  \]

\bigskip

  For every $n\in\Nbb$ and $\pi\in\Deltan\un X_{0}$ the type $T^{\c
    n}_{\pi}\Xcal$ is a homogeneous configuration. Suppose that a complete configuration
  $\Xcal$ is such that the probability distribution $p_{0}$ on the
  initial set is rational with the denominator $n$, then we call
  $T^{\c n}_{p}\un\Xcal$ the \term{true type} of $\Xcal$ and
  denote
  \[
  T^{\c n}\Xcal:=T^{n}_{p_{0}}\un\Xcal
  \]
  
