  This section is devoted to the basic setup used throughout the
  present article.  We introduce a category of probability spaces and
  reductions, similar to categories introduced in \cite{Baez-Characterization-2011} and
  \cite{Gromov-Search-2012}, and define configurations of probability
  spaces and the corresponding category.  The last subsection recalls
  the notion of entropy and its elementary properties.

\subsection{Probability spaces and reductions}\label{s:category-prob}
  Below we will consider probability spaces such that the support of
  the probability measure is finite. Any such space contains a
  full-measure subspace isomorphic to a finite space, thus we call
  such objects \term[finite probability space]{finite probability
  spaces}.  For a probability space $X=(S,p_{X})$ denote by $\un
  X=\supp p_{X}$ the support of the measure and by $|X|$ its
  cardinality. Slightly abusing the language, we call this quantity
  the \term[cardinality of probability space]{cardinality} of $X$.

  For a pair of probability spaces a \term{reduction} $X\to Y$ is a
  class of measure-preserving maps, with two maps being equivalent if
  they coincide on a set of full measure. The composition of two
  reductions is itself a reduction.  Two probability spaces are
  isomorphic if there is a measure-preserving bijection between
  the supports of the probability measures. Such a bijection defines an
  invertible reduction from one space into another. Clearly the
  cardinality $|X|$ is an isomorphism invariant.
  The automorphism group $\Aut(X)$ is the group of all
  self-isomorphisms of $X$.

  A probability space $X$ is called \term[homogeneous probability
  space]{homogeneous} if the automorphism group $\Aut(X)$ acts
  transitively on the support $\un X$ of the measure.  The property of
  being homogeneous is an isomorphism invariant.  In the isomorphism
  class of a homogeneous space there is a representative with uniform
  measure.

  The finite probability spaces and reductions form a category, that
  we denote by $\prob$. The subcategory of homogeneous spaces will be
  denoted by $\probhom$.  The isomorphism in the category coincides
  with the notion of isomorphism above.

  The category $\prob$ is not a small category. However it has a small
  full subcategory, that contains an object for every isomorphism class in
  $\prob$ and for every pair of objects in it, it contains all the
  available morphisms between them. From now on we imagine that such a
  subcategory was chosen and fixed and replaces $\prob$ in all
  considerations below.

  There is a product in $\prob$ given by the Cartesian product of
  probability spaces, that we will denote by $X\otimes Y:=(\un
  X\times\un Y,p_{X}\otimes p_{Y})$. There are canonical reductions
  $X\otimes Y\to X$ and $X\otimes Y\to Y$ given by projections to
  factors. For a pair of reductions $f_{i}:X_{i}\to Y_{i}$, $i=1,2$
  their tensor product is the reduction $f_{1}\otimes
  f_{2}:X_{1}\otimes X_{2}\to Y_{1}\otimes Y_{2}$, which is equal to
  the class of the Cartesian product of maps representing $f_{i}$'s.
   The tensor product is not however a categorical product. The product
  leaves the subcategory of homogeneous spaces invariant.

  The probability measure on $X$ will be usually denoted by $p_{X}$ or
  simply $p$, when the risk of confusion is low.

\subsection{Configurations of probability spaces}
\label{s:category-config}
  Essentially, a configuration $\Xcal=\set{X_{i};f_{ij}}$ is a
  commutative diagram consisting of a finite number of probability spaces and
  reductions between some of them, that is transitively closed, while
  a morphism $\rho:\Xcal\to\Ycal$ between two configurations
  $\Xcal=\set{X_{i};f_{ij}}$ and $\Ycal=\set{Y_{i};g_{ij}}$ of the
  same combinatorial type is a collection of reductions between
  corresponding individual objects $\rho_{i}:X_{i}\to Y_{i}$, that
  commute with the reductions within each configuration,
  $\rho_{j}\circ f_{ij}=g_{ij}\circ\rho_{i}$.
  
  We need to keep track of the combinatorial structure of the
  collection of reductions within a configuration. There are several
  possibilities for doing so: 
  \begin{itemize}
  \item 
    the reductions form a directed transitively closed graph without
    loops;
  \item 
    the spaces in the configuration form a poset; 
  \item
    the underlying combinatorial structure could be recorded as a finite
    category. 
  \end{itemize}

  The last option seems to be most convenient since it has many
  operations necessary for our analysis already built-in.

  A \term{diagram category} $\Gbf$ is a finite category such that
  for each pair of objects $O_{1}$, $O_{2}$ in $\Gbf$ the morphism
  space between them
  \[
    \Hom_{\Gbf}(O_{1},O_{2})\cup\Hom_{\Gbf}(O_{2},O_{1})
  \] 
  contains at most one element.

  For a diagram category $\Gbf$ a \term{configuration of probability
    spaces} modeled on $\Gbf$ is a functor $\Xcal: \Gbf\to \prob$.
  The collection of all configurations of probability spaces modeled
  on a fixed diagram category $\Gbf$ forms the category of functors
  $\prob\<\Gbf>:=[\Gbf,\prob]$. The objects of $\prob\<\Gbf>$ are
  configurations, that is functors from $\Gbf$ to $\prob$, while
  morphisms in $\prob\<\Gbf>$ are natural transformations between
  them.  For a configuration $\Xcal\in\prob\<\Gbf>$, the diagram
  category $\Gbf$ will be called the \term[combinatorial type of a
    configuration]{combinatorial type} of $\Xcal$.

  For a diagram category $\Gbf$ or a configuration
  $\Xcal\in\prob\<\Gbf>$ we denote by $\size{\Gbf}=\size{\Xcal}$ the
  number of objects in the category $\Gbf$.

  An object $O$ in a diagram category $\Gbf$ will be called
  \term[initial object/space]{initial}, if it is not a target of any
  morphism except for the identity. Likewise a \term[terminal
    object/space]{terminal} object is not a source of any morphism,
  except for the identity morphism. Note that this terminology is
  somewhat unconventional from the point of view of category theory.
  
   A diagram category is called \term[complete diagram
     category]{complete} if it has a unique initial object. Thus a
   configuration modeled on a complete category includes a space that
   reduces to all other spaces in the configuration.

  The above terminology transfers to configurations modeled on
  $\Gbf$: An initial space in $\Xcal\in\prob\<\Gbf>$ is one that
  is not a target space of any reduction within the configuration, a
  terminal space is not a source of any non-trivial reduction and
  $\Xcal$ is complete if $\Gbf$ is, that is there is a unique initial
  space.

  The tensor product of probability spaces extends to a tensor product
  of configurations. For $\Xcal,\Ycal\in\prob\<\Gbf>$, such that
  $\Xcal=\set{X_{i};f_{ij}}$ and $\Ycal=\set{Y_{i};g_{ij}}$ define
  \[
    \Xcal\otimes\Ycal
    :=
    \set{X_{i}\otimes Y_{i};f_{ij}\otimes g_{ij}}
  \] 

  Occasionally we will also talk about configuration of sets. Denote by
  $\Set$ the category of finite sets and surjective maps. Then all of the
  above constructions could be repeated for sets instead of probability
  spaces. Thus we could talk about the category of configurations of sets
  $\Set\<\Gbf>$.

  Given a reduction $f:X\to Y$ between two probability spaces, the
  restriction $\un f:\un X\to\un Y$ is a well-defined surjective map.
  Given a configuration $\Xcal=\set{X_{i};f_{ij}}$ of probability
  spaces, there is an underlying configuration of sets, obtained by
  taking the supports of measures on each level and restricting
  reductions on these supports. We will denote it by
  $\un\Xcal=\set{\un X_{i};f_{ij}}$, where $\un X_{i}:=\supp
  p_{X_{i}}$. Thus we have a forgetful functor
  \[
  \underline{\mkern5mu\cdot\mkern5mu}:\prob\<\Gbf>\to\Set\<\Gbf>
  \]
  
  For now we will consider two important examples of diagram
  categories and configurations modeled on them. We give further
  examples in Section \ref{s:config-examples}.

\subsubsection{Two-fans:}\label{s:category-config-2fan}
  A two-fan is a configuration modeled on the category $\Lambdabf$
  with three objects, one initial and two terminal.
  \[
    \Lambdabf=(O_{1}\ot O_{12}\to O_{2})
  \]
  
  There is a special significance to two-fans, since these are the
  simplest non-trivial configurations, as we will see later.

  Essentially, a two-fan $X\ot Z\to Y$ is a triple of probability spaces and a pair of
  reductions  between them. 

  A reduction of a two-fan $X\ot Z\to Y$ to another two-fan $X'\ot Z'\to Y'$
  is a triple of reductions $Z\to Z'$, $Y\to Y'$ and $X\to X'$ that
  commute with the reductions within each fan, that is, the following diagram is commutative
  \[
  \begin{tikzcd}
  X 
  \arrow{d}
  & Z
  \arrow{l}
  \arrow{d}
  \arrow{r}
  & Y 
  \arrow{d}\\
  X' 
  & 
  Z' 
  \arrow{l}
  \arrow{r}
  & Y'  
  \end{tikzcd}
  \] 
  
  Isomorphisms and the automorphism group $\Aut(\cdot)$
  are defined accordingly.  Note that terminal spaces in a two-fan are
  labeled and reductions preserve the labeling.
  
  A two-fan $X\ot Z\to Y$ is called \term[minimal 2-fan]{minimal} if
  for a.e.~$x\in X$ and $y\in Y$ there is a unique $z\in Z$, that
  reduces to $y$ and to $x$.  Given a two-fan $X\ot Z\to Y$, there is
  always a reduction to a minimal two-fan $X\ot Z'\to Y$. Such minimal
  reduction is unique up to isomorphism. Explicitly, take $Z':=\un
  X\times \un Y$ as a set and consider a probability distribution on
  $Z'$ induced by a map $Z\to Z'$ which is the Cartesian product of
  the reductions $X\ot Z\to Y$ in the original two-fan.

  The notion of being minimal is in fact a categorical notion. It could
  be equivalently defined by saying that a two-fan $\Xcal=(X_{1}\ot X_{12}\to
  X_{2})$ is minimal if for any reduction $\lambda:\Xcal\to\Xcal'$
  holds: if both $\lambda_{1}$ and $\lambda_{2}$ are isomorphisms, then
  $\lambda_{12}$ is also an isomorphism. Consequently, if one specifies
  reductions from the terminal spaces of a minimal two-fan to another
  two-fan, then there exists at most one extension to the reduction of the
  whole fan.

  The inclusion of a pair of probability
  spaces $X$ and $Y$ as terminal vertices in a minimal two-fan is
  equivalent to specifying a joint distribution on $\un X\times \un Y$.

  An arbitrary configuration $\Xcal$ will be called \term[minimal
    configuration]{minimal} if with every two-fan, it also contains a
  minimal two-fan with the same terminal spaces.
  We will denote the space of minimal configurations modelled on a
    diagram category $\Gbf$ by $\prob\<\Gbf>_{\mbf}$.

  Given a two-fan 
  \[
    \Fcal=(X\ot Z\to Y)
  \]
  with terminal spaces $X$ and $Y$, and a point $x \in X$ with $p_X(x) > 0$, one may construct a \term{conditional probability distribution}
  $p_Y(\;\cdot\;\rel x)$ on $\un Y$. We denote the corresponding space
  $Y\rel x:=\big(\un Y,p_Y(\;\cdot\;\rel x)\big)$. The usual bar
  ``$|$'', that is normally used for conditioning, interferes with our
  notations for cardinality of spaces. 
  We will give more details in Section \ref{s:config-conditioning}.

\subsubsection{A diamond configuration}
\label{s:category-config-diamond}
  A ``diamond'' configuration is modeled on a diamond category $\dia$,
  that consists of a two-fan and a ``co-fan'':
   \[
   \dia =
  \left(  \begin{tikzcd}[row sep=small, column sep=small]
  \mbox{}
  & 
  O_{12}
  \arrow{dl}{}
  \arrow{dr}{}
  &
  \\
  O_{1}
  \arrow{dr}{}
  & 
  &
  O_{2}
  \arrow{dl}{}
  \\
  \mbox{}
  &
  O_{\bullet}
  &
  \end{tikzcd} \right)
  \]
  
   Of course, there is also a morphism $O_{12}\to O_{\bullet}$, which
  lies in the transitive closure of the given four morphisms. As a
  rule, we will skip writing morphisms, that are implied by the
  transitive closure.

  A diamond configuration is minimal if the top two-fan in it is
  minimal.

\subsection{Entropy}\label{s:category-entropy}
  Our working definition of \term{entropy} will be based on the
  following version of the asymptotic equipartition theorem for
  Bernoulli process, see \cite{Cover-Elements-1991}.
  \begin{theorem}{p:aepbernoullisingle}
    Suppose $X$ is a finite probability space, then for any $\epsilon>0$ and
    any $n>\!>0$ there exists a subset $A^{\c n}_{\epsilon}\subset
    X^{\otimes n}$ such that
    \begin{enumerate}
    \item 
      $p(A^{\c n}_{\epsilon})\geq1-\epsilon$
    \item
      For any $a,a'\in A^{\c n}_{\epsilon}$ holds
      \[
        \left|\frac{\ln p(a)}{n}-\frac{\ln p(a')}{n}\right|\leq\epsilon
      \]
    \end{enumerate}
    Moreover, if $A^{\c n}_{\epsilon}$ and $B^{\c n}_{\epsilon}$ are
    two subsets of $X^{\otimes n}$ satisfying two conditions above, then their
    cardinalities satisfy
    \[\tageq{cardinalityrange}
      \left|
        \frac{\ln |A^{\c n}_{\epsilon}|}{n}
        -
        \frac{\ln |B^{\c n}_{\epsilon}|}{n}
      \right|
      \leq 2\epsilon
    \]
  \end{theorem}
 
  Then we define
  \[
    \ent(X):=\lim_{\epsilon\to0}\lim_{n\to\infty}\frac1n\ln|A^{\c n}_{\epsilon}|
  \]

  Clearly, in view of the property (\ref{eq:cardinalityrange}) in the
  Theorem \ref{p:aepbernoullisingle} the limit above is well-defined
  and is independent of the choice of the typical subsets $A_\epsilon^{(n)}$.

  Entropy satisfies the so-called Shannon inequality,
  see for example \cite{Cover-Elements-1991}, namely for any minimal
  diamond configuration
  \[
  \begin{tikzcd}[row sep=small, column sep=small]
    \mbox{}
    & 
    X_{12}
    \arrow{dl}{}
    \arrow{dr}{}
    \arrow{dd}{}
    &
    \\
    X_{1}
    \arrow{dr}{}
    & 
    &
    X_{2}
    \arrow{dl}{}
    \\
    \mbox{}
    &
    X_{\bullet}
    &
  \end{tikzcd}
  \]
  the following inequality holds, 
  \[\tageq{shannonineq}
    \ent(X_{1})+\ent(X_{2})
    \geq
    \ent(X_{12})+\ent(X_{\bullet})
  \]

  Furthermore, entropy is additive with respect to the tensor product,
  that is, for a pair of probability spaces $X,Y\in\prob$ holds
  \[\tageq{entropyadditive}
  \ent(X\otimes Y)=\ent(X)+\ent(Y)
  \]

  Further, for a pair $X$, $Y$ of probability spaces included in a
  minimal two-fan $(X\ot Z\to Y)$ we define the conditional entropy
  \[
    \ent(X\rel Y):=\ent(Z)-\ent(X)
  \]

  The above quantity is always non-negative in view of Shannon
  inequality~(\ref{eq:shannonineq}).  Moreover, the following
  identity holds, see \cite{Cover-Elements-1991}
  \[\tageq{relentinteg}
    \ent(X\rel Y)
    =
    \int_{Y}\ent(X\rel y)\d p_{Y}(y)
  \]
