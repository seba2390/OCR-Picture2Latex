% SIAM Shared Information Template
% This is information that is shared between the main document and any
% supplement. If no supplement is required, then this information can
% be included directly in the main document.


% Packages and macros go here
\usepackage{lipsum}
\usepackage{amsfonts}
\usepackage{graphicx}
\usepackage{epstopdf}
\usepackage{algorithmic}
\ifpdf
  \DeclareGraphicsExtensions{.eps,.pdf,.png,.jpg}
\else
  \DeclareGraphicsExtensions{.eps}
\fi

% Add a serial/Oxford comma by default.
\newcommand{\creflastconjunction}{, and~}

% Used for creating new theorem and remark environments
\newsiamremark{remark}{Remark}
\newsiamremark{hypothesis}{Hypothesis}
\crefname{hypothesis}{Hypothesis}{Hypotheses}
\newsiamthm{claim}{Claim}

% Sets running headers as well as PDF title and authors
\headers{Finite population effects on optimal communication}{Hyunjoong Kim, Yoichiro Mori, and Joshua Plotkin}

% Title. If the supplement option is on, then "Supplementary Material"
% is automatically inserted before the title.
\title{Finite population effects on optimal communication for social foragers\thanks{Submitted to the editors December xx, 2022}}% \thanks{Submitted to the editors DATE. \funding{This work was funded by the Fog Research Institute under contract no.~FRI-454.}}

% Authors: full names plus addresses.
\author{Hyunjoong Kim\thanks{Center for Mathematical Biology \& Department of Mathematics, University of Pennsylvania, Philadelphia, PA 19104 (\email{h6kim@sas.upenn.edu})} 
\and Yoichiro Mori\thanks{Center for Mathematical Biology \& Department of Mathematics \& Department of Biology, University of Pennsylvania, Philadelphia, PA 19104 (\email{y1mori@sas.upenn.edu})}
\and Joshua B. Plotkin\thanks{Center for Mathematical Biology \& Department of Mathematics \& Department of Biology, University of Pennsylvania, Philadelphia, PA 19104 (\email{jplotkin@sas.upenn.edu})}}

\usepackage{amsopn}
\DeclareMathOperator{\diag}{diag}


%% Added on Overleaf: enabling xr
\makeatletter
\newcommand*{\addFileDependency}[1]{% argument=file name and extension
  \typeout{(#1)}% latexmk will find this if $recorder=0 (however, in that case, it will ignore #1 if it is a .aux or .pdf file etc and it exists! if it doesn't exist, it will appear in the list of dependents regardless)
  \@addtofilelist{#1}% if you want it to appear in \listfiles, not really necessary and latexmk doesn't use this
  \IfFileExists{#1}{}{\typeout{No file #1.}}% latexmk will find this message if #1 doesn't exist (yet)
}
\makeatother

\newcommand*{\myexternaldocument}[1]{%
    \externaldocument{#1}%
    \addFileDependency{#1.tex}%
    \addFileDependency{#1.aux}%
}

\newcommand{\paren}[1]{\left(#1\right)}
\newcommand{\jump}[1]{\left[#1\right]}
\newcommand{\D}[2]{\frac{d#1}{d#2}}
\newcommand{\PD}[2]{\frac{\partial#1}{\partial#2}}
\newcommand{\PDD}[3]{\frac{\partial^{#1}{#2}}{\partial{#3}^{#1}}}
\newcommand{\lap}[1]{\Delta #1}
\newcommand{\at}[2]{\left. #1 \right|_{#2}}
\newcommand{\grad}[1]{\nabla #1}
%\newcommand{\mb}[1]{\mathbf{#1}}
\newcommand{\mc}[1]{\mathcal{#1}}
\newcommand{\bm}[1]{\boldsymbol{#1}}
\newcommand{\abs}[1]{\left\lvert #1 \right\rvert}
\newcommand{\norm}[1]{\left\lVert #1 \right\rVert}
\newcommand{\dual}[2]{\left\langle #1,#2 \right\rangle}


%%% END HELPER CODE
%%% Local Variables: 
%%% mode:latex
%%% TeX-master: "ex_article"
%%% End: 
