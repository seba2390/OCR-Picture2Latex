\documentclass[english]{article}
\usepackage{preamble}

\title{Chromatic Cardinalities via Redshift}

\begin{document}

% \date{\today}.
\maketitle

\begin{abstract}
    Using higher descent for chromatically localized algebraic $K$-theory, we show that the higher semiadditive cardinality of a $\pi$-finite $p$-space $A$ at the Lubin--Tate spectrum $E_n$ is equal to the higher semiadditive cardinality of the free loop space $LA$ at $E_{n-1}$.
    By induction, it is thus equal to the homotopy cardinality of the $n$-fold free loop space $L^n A$.
\end{abstract}

\begin{figure}[ht!]
    \centering
    \includegraphics[width=120mm]{cardinal.jpg}
    \caption*{
        ``\href{https://www.flickr.com/photos/92252798@N07/26805851281/}{Cardinalis sinuatus -- the Pyrrhuloxia}''
        by \href{https://www.flickr.com/photos/92252798@N07/}{Dick Culbert}
        licensed under \href{https://creativecommons.org/licenses/by/2.0/}{CC BY 2.0}.
    }
\end{figure}

\newpage
% \tableofcontents{}

\section{Introduction}

Natural numbers appear, naturally, as cardinalities of finite sets. Their fundamental role in algebra can be explained from a categorical perspective via the notion of \textit{semiadditivity}. 
A semiadditive category $\cC$ is one in which finite products and finite coproducts canonically coincide (e.g.\ abelian groups). 
When this property holds, every finite set $A$ with cardinality $|A| \in \NN$ induces a natural operation of multiplication-by-$|A|$ on objects $X \in \cC$ via the composition
\[
    X \too[\Delta] \invlim_A X \simeq \colim_A X \too[\nabla] X.
\]
When $\cC$ is monoidal (such that the tensor product commutes with finite (co)products in each variable) and $R$ is an algebra object in $\cC$, the map $|A|_R\colon R \to R$ can be identified with an element of the semiring $\hom(\one,R)$. The resulting semiring homomorphism $\NN \to \hom(\one,R)$ realizes every abstract natural number $n$ as a specific element $n_R$ of $R$. 

In \cite{AmbiKn}, it was observed that the property of semiadditivity sits in a natural hierarchy of \textit{higher semiadditivity} properties, which are most natural to consider in the setting of $\infty$-categories, to which from now on we shall refer simply as \textit{categories}.
In particular, a category $\cC$ is $0$-semiadditive if it is semiadditive in the ordinary sense, and it is $\infty$-semiadditive if, roughly, limits and colimits over $\pi$-finite spaces (i.e.\ those with finitely many connected components, each of which has finitely many non-vanishing homotopy groups all of which are finite)
canonically coincide. As above, this provides a natural multiplication-by-$|A|$ map $|A|_X\colon X \to X$ for every $\pi$-finite space $A$. 
If $\cC$ is moreover monoidal (such that the tensor product commutes with (co)limits over $\pi$-finite spaces in each variable) and $R$ is an algebra object in $\cC$, we obtain an element $|A|_R$ in the semiring $\pi_0\Map(\one, R)$. However, the determination of this element in concrete examples is in general a non-trivial task. 

The most prominent examples of $\infty$-semiadditive categories come from chromatic homotopy theory. Hopkins and Lurie show in \cite{AmbiKn} that for every $n \ge 0$, the Bousfield localization $\SpKn$ of the category of spectra with respect to the height $n$ and (implicit) prime $p$ Morava $K$-theory ring spectrum $K(n)$ is $\infty$-semiadditive. In \cite{TeleAmbi}, it was further shown that the larger telescopic localizations $\SpTn$ are $\infty$-semiadditive as well (and are in fact the maximal ones in a certain precise sense).
In the height $n=0$ case, we have $K(0) = \QQ$ and the corresponding cardinality of every $\pi$-finite space coincides with its Baez--Dolan \textit{homotopy cardinality}:

\begin{example}[{\cite[Proposition 2.3.4]{AmbiHeight}}]\label{htpy-card}
    Working in $\Sp_\QQ$, for every connected $\pi$-finite space $A$ we have
    \[
        |A|_{\Sph_\QQ} = \frac{|\pi_2(A)| |\pi_4(A)| |\pi_6(A)| \cdots}{|\pi_1(A)| |\pi_3(A)| |\pi_5(A)| \cdots}
        \qin \pi_0(\Sph_\QQ) \simeq \QQ.
    \]
\end{example}

In higher chromatic heights the situation is considerably more involved. Note that the prime $p$ is no longer invertible so the above formula does not even make sense. In fact, Yuan and the second author showed in \cite{CarmeliYuan} that already in height $n=1$ the cardinality of a $\pi$-finite space may be \textit{non-rational}:

\begin{example}[{\cite[Theorem A]{CarmeliYuan}}]\label{CY}
    Working in $\Sp_{T(1)} = \Sp_{K(1)}$ and $p = 2$, we have\footnote{In fact, they have completely determined $|A|$ for all $\pi$-finite $p$-spaces $A$. This computation was further extended to all $\pi$-finite spaces by Yifan Li.}
    \[
        |BC_2|_{\Sph_{K(1)}} = 1 + \varepsilon
        \qin \pi_0(\Sph_{K(1)}) \simeq \ZZ_2[\varepsilon]/(\varepsilon^2, 2\varepsilon).
    \]
\end{example}

In higher heights, not much is known about cardinalities of $\pi$-finite spaces for $\Sph_{K(n)}$, as even the knowledge of the ring $\pi_0(\Sph_{K(n)})$ itself is limited, and for $\Sph_{T(n)}$ the situation is even worse. However, the situation improves drastically if one passes to the algebraic closure of $\Sph_{\Kn}$, namely, the Lubin--Tate ring spectrum $E_n$. The image of the map 
\[
    \pi_0(\Sph_{\Kn}) \too 
    \pi_0(E_n) \simeq
    W(\cl{\FF}_{p})[[u_1,\dots,u_n]]
\]
is the subring $\ZZ_p = W(\FF_p)$, and since $|A|_{E_n}$ is the image of $|A|_{\Sph_{\Kn}}$ under this map, we can consider it simply as a $p$-adic integer. Furthermore, the kernel of the above map is the nil-radical of $\pi_0(\Sph_{\Kn})$ (see, e.g., \cite[Proposition 2.2.6]{AmbiHeight}), hence the number $|A|_{E_n}$ retains precisely the information in $|A|_{\Sph_{\Kn}}$ modulo nilpotents. Some simple cases can be worked out explicitly:

\begin{example}[{\cite[Lemma 5.3.3]{TeleAmbi}}]\label{Ex_Card_EM}
    Working in $\SpKn$ (or $\SpTn$), for every $d \ge 0$, we have
    \[
        |B^dC_p|_{E_n} = p^{\binom{n-1}{d}}
        \qin \ZZ_p \sseq \pi_0(E_n).
    \]
\end{example}

As mentioned in \cite[Example 2.2.4]{AmbiHeight}, it is possible to show that the cardinalities at the Lubin--Tate spectrum are in fact \textit{natural numbers} and deduce a fairly explicit formula for them.
The idea is to use the technology of tempered ambidexterity developed in \cite{Lurie_Ell3}, extending \cite{hopkins2000generalized, stapleton2013transchromatic}, to express the height $n+1$ cardinality of a space via the height $n$ cardinality of a closely related space.
Applying this inductively, this reduces computations to height $0$.
In more detail, let us denote by $L_p A := \Map(B\ZZ_p, A)$ the $p$-adic free loop space of $A$. 
There is a certain $K(n)$-local $E_{n+1}$-algebra $C_n$ and a natural transchromatic character map
\[
    \chi^A \colon E_{n+1}^A \too C_n^{L_pA}.
\]
The compatibility of these maps with transfers in the variable $A$ implies that $\pi_0(E_{n+1}) \to \pi_0(C_n)$ takes the element $|A|_{E_{n+1}}$ to the element $|L_pA|_{C_n}$ for every $\pi$-finite space $A$. 
One can further show that the cardinalities at $C_n$ and at $E_n$ agree, thus showing that $|A|_{E_{n+1}} = |L_pA|_{E_n}$. Intuitively, the $p$-adic free loop space accounts for the decrease in height.

In the special case where $A$ is further assumed to be a $p$-space (i.e., all homotopy groups are $p$-groups), the $p$-adic free loop space coincides with the ordinary free loop space $LA = \Map(B\ZZ,A)$, in which case $|A|_{E_{n+1}} = |LA|_{E_n}$. In this paper we provide a \textit{different} proof of this special case:

\begin{restatable}{theorem}{cardblue}\label{card-blue}
    Let $A$ be a $\pi$-finite $p$-space. The cardinality $|A|_{E_{n+1}}$ is a natural number and
    \[
        |A|_{E_{n+1}} = |LA|_{E_n} \qin \NN.
    \]
\end{restatable}

Applying this inductively and combining with \cref{htpy-card}, we get:

\begin{cor}\label{card-to-0}
    $|A|_{E_n}$ equals the Baez--Dolan homotopy cardinality of $L^nA$ from \Cref{htpy-card}.
\end{cor}

\begin{rem}
    Is is easy to see that the homotopy cardinality of $LA$ is just the size of the set $\pi_0(A)$ (see \cite[Proposition 2.15]{yanovski2023homotopy}). Thus, we can reformulate \Cref{card-to-0} as
    \[
        |A|_{E_n} = 
        |\pi_0 \Map((S^1)^{n-1},A)| \qin 
        \NN.
    \]
\end{rem}

In contrast with the transchromatic proof, which proceeds by analyzing the behaviour of the Quillen $p$-divisible group under transchromatic base change, our proof employs the \textit{redshift philosophy} to transport the computation to an analogous, but simpler, categorical setting. Roughly, the transchromatic proof relates cardinalities at $E_{n+1}$ and at $E_n$ by approximating the latter with $C_n$, whereas our proof proceeds instead by approximating the former with $K(E_n)$.

To begin with, the category $\cMod_{E_n}$ of $\Kn$-local $E_n$-modules is an object of the category $\Catpfin$ of categories admitting $\pi$-finite colimits and functors that preserve them. 
The latter is $p$-typically $\infty$-semiadditive, allowing us to speak of the cardinality of $A$ at the object $\cMod_{E_n} \in \Catpfin$. This turns out to be simply the colimit of the constant $A$-shaped diagram on $E_n$, that is
\[
    |A|_{\cMod_{E_n}} \simeq A \otimes E_n
    \qin \cMod_{E_n}.
\]
Furthermore, the object $A\otimes E_n$ is dualizable, and its symmetric monoidal dimension is given by
\[
    \dim(A \otimes E_n) = |LA|_{E_n}
    \qin \pi_0(E_n).
\]

Note that while $\Catpfin$ is not stable, so we can not measure the chromatic height of the object $\cMod_{E_n}$, we can measure its \textit{semiadditive height}, which by the semiadditive redshift theorem is $n+1$ (see \cite[Theorem B]{AmbiHeight}).
Thus, the combination of the two displayed formulas above bears a close resemblance to \Cref{card-blue}. 
The bridge between the categorical story and the chromatic one is the algebraic $K$-theory functor, which produces a spectrum from a (stable) category. The key ingredient in our proof is the higher descent property of chromatically localized algebraic $K$-theory \cite[Theorem A]{cycloredshift}. It implies that the map 
\[
    \pi_0(\cMod_{E_n}^{\dbl}) \too 
    \pi_0(L_{T(n+1)}K(E_n))
\]
sending a dualizable module $M$ to its class $[M]$ in ($T(n+1)$-localized) $K$-theory preserves cardinalities of $\pi$-finite $p$-spaces. To conclude, we observe that by chromatic redshift, $L_{T(n+1)}K(E_n)$ is a non-zero commutative algebra and hence by the chromatic nullstellensatz admits a map of $T(n+1)$-local commutative algebras to (a mild extension of) $E_{n+1}$. Putting everything together we get
\[
    |A|_{E_{n+1}} = 
    |A|_{L_{T(n+1)}K(E_n)} = 
    \dim(A \otimes E_n) =
    |LA|_{E_n}.
\]

\begin{rem}
    To complete the picture and tighten the analogy, the role of the transchromatic character map is played by the ordinary character map
    \[
        \chi^A \colon 
        (\cMod_{E_n}^{\dbl})^A \too 
        E_n^{LA}
    \]
    taking a local system $V$ of dualizable $\Kn$-local $E_n$-modules on $A$ to its character $\chi^A_V \colon LA \to E_n$ whose value on a loop $\gamma \in LA$ is given by $\tr(\gamma \mid V)$. 
    Though not used in the present paper, it is true that the computation $\dim(A \otimes E_n) = |LA|_{E_n}$ is a special case of the compatibility of $\chi^A$ with higher semiadditive transfers. This is part of an upcoming work of Cnossen, Ramzi and the second and fourth authors.
\end{rem}

The symmetric monoidal dimension $\dim(A\otimes E_n)$ identifies with the perhaps more familiar \textit{Morava--Euler characteristic} (see \Cref{dim_dim})
\[
    \chi_{n}(A) :=
    \dim_{\FF_p}(K(n)^{0}(A)) - 
    \dim_{\FF_p}(K(n)^{1}(A)).
\]
Thus, using the relation $\dim(A \otimes E_n) = |LA|_{E_n}$ between cardinality and symmetric monoidal dimension, \Cref{card-blue} implies the following corollary which is stated in more concrete terms:

\begin{cor}[{cf.\ \cite[Corollary 4.8.6]{Lurie_Ell3}}]
    Let $A$ be a $\pi$-finite $p$-space. We have,
    \[
        \chi_n(A) = 
        |\pi_0 \Map((S^1)^n,A)|
        \qin \NN.
    \]
\end{cor}

\begin{rem}
    In \cite{yanovski2023homotopy}, the fourth author used the above corollary to show that the sequence $|A|_{E_n}$, as a function of $n$, is $\ell$-adically continuous for every prime $\ell \mid p-1$. Or equivalently, that the sequence of natural numbers $\chi_n(A)$ is $\ell$-adically continuous and extrapolates to the homotopy cardinality of $A$ at $n = -1$.
\end{rem}

Given a $\pi$-finite $p$-space $A$, computing the homotopy cardinality of the iterated free loop space $L^nA$ might be quite difficult. There is however a family of examples, generalizing  \Cref{Ex_Card_EM}, in which it can be expressed in elementary terms. If $A$ happens to be a loop space, we have a decomposition $LA \simeq A \times \Omega A$. Since the homotopy groups of $\Omega A$ are the same as those of $A$ shifted by 1, one can easily deduce by induction the following:

\begin{example}[{cf.\ \cite[Example 3.1]{yanovski2023homotopy}}]\label{Ex_Card_Loop}
    Working in $\SpKn$ (or $\SpTn$), for every $\pi$-finite $p$-space $A$ which is a loop space we have
    \[
        |A|_{E_n} = 
        \prod_{n \ge 0} |\pi_n(A)|^{\binom{n-1}{d}}.
    \]
\end{example} 

We conclude with the following observation. 
The proof of Hopkins--Lurie of the $\infty$-semiadditivity of $\SpKn$ relies heavily on a careful analysis of the seminal computation of Ravenel--Wilson of $\Kn_*(B^dC_p)$ as a Hopf algebra \cite{ravenelwilson1980}, and an integral lift of this computation to $E_n$. In contrast, the proof of the $\infty$-semiadditivity of $\SpTn$ (hence in particular of $\SpKn$) in \cite{TeleAmbi} avoids such an elaborate analysis, and only relies on the computation of the Morava--Euler characteristics of the spaces $B^dC_p$ (in fact, only that they are rational and non-zero). While at the time it was not known whether one can prove this without invoking the full power of the Ravenel--Wilson computation, we can now close the circle by providing an alternative route. 
Indeed, the $m$-semiadditivity of $\Sp_{T(n)}$ requires only the knowledge of the Morava--Euler characteristic of $B^dC_p$ for $1\le d \le m-1$. 
Thus, by preforming all of the arguments under an inductive hypothesis on $m$, we can compute the Morava--Euler characteristic of $B^dC_p$ in the required range from \Cref{card-blue}, thus, in particular, eliminating the reliance on the Ravenel--Wilson computation.



\section{The Proof}

We now carry out in the detail the proof of \cref{card-blue} sketched in the introduction.
Given a small (symmetric monoidal) stable idempotent complete category $\cC$, we can consider its algebraic $K$-theory (commutative ring) spectrum $K(\cC)$. 
For a (commutative) ring spectrum $R$, one usually defines $K(R)$ as $K(\cC)$ for $\cC$ the (symmetric monoidal) category $\Mod_R^\omega$ of perfect $R$-module spectra. 
However, when $R$ is $T(n)$-local, we can also consider the algebraic $K$-theory of the category $\cMod_R^\dbl$ of dualizable $T(n)$-local $R$-modules. 
By \cite[Proposition 4.15]{clausen2020descent}, these two algebraic $K$-theory spectra identify after $T(n+1)$-localization. 
We shall henceforth adopt the definition in terms of dualizable $T(n)$-local modules and use the notations
\begin{gather*} 
    \KTnp(\cC) := \LTnp(K(\cC)), \\
    \KTnp(R) := \LTnp(K(R)) \simeq \LTnp(K(\cMod_R^\dbl)).
\end{gather*}
By construction, every $M\in \cMod_R^\dbl$ defines a class 
\[
    [M] \qin \pi_0(\KTnp(R)).
\]

We briefly recall the setup of the higher descent result for $\KTnp$ from \cite{cycloredshift}. Let $\CatLnf$ be the category of small stable $\Lnf$-local idempotent complete categories and exact functors. In \cite[Theorem A]{cycloredshift} we show that the functor
\[
    \KTnp \colon \CatLnf \too \SpTnp
\]
preserves limits and colimits of $\pi$-finite $p$-space shape. 
We let $\Catpfin \subset \Cat$ be the subcategory on those categories which admit $\pi$-finite $p$-space colimits and functors preserving them, and set
\[
    \CatLnfpfin := \CatLnf \cap \Catpfin.
\]

\begin{prop}\label{KTnp_Sadd}
    The category $\CatLnfpfin$ is $p$-typically $\infty$-semiadditive and the restricted functor
    \[
        \KTnp \colon \CatLnfpfin \too \SpTnp
    \]
    is $p$-typically $\infty$-semiadditive (i.e., preserves $\pi$-finite $p$-space limits and colimits).
\end{prop}

\begin{proof}
    The category $\Catpfin$ is $p$-typically $\infty$-semiadditive by \cite[Proposition 2.2.7]{AmbiHeight} (which immediately follows from \cite[Proposition 5.26]{harpaz2020ambidexterity}).
    The inclusion $\CatLnfpfin \hookrightarrow \Catpfin$ preserves limits, thus $\CatLnfpfin$ is also $p$-typically $\infty$-semiadditive by \cite[Proposition 2.1.4(3)]{AmbiHeight}. Similarly, the second inclusion $\CatLnfpfin \hookrightarrow \CatLnf$ preserves limits, hence the composition
    \[
        \CatLnfpfin \too
        \CatLnf \oto{\KTnp}
        \SpTnp
    \]
    preserves $\pi$-finite $p$-space limits hence is $p$-typically $\infty$-semiadditive.
\end{proof}

We deduce the following fundamental identity:

\begin{prop}\label{card-K}
    Let $R \in \CAlg(\SpTn)$ and let $A$ be a $\pi$-finite $p$-space, then
    \[
        |A|_{\KTnp(R)} = [A\otimes R]
        \qin \pi_0(\KTnp(R)).
    \]
\end{prop}

\begin{proof}
    A $p$-typically $\infty$-semiadditive functor preserves cardinalities of $\pi$-finite $p$-spaces by \cite[Corollary 3.2.7]{TeleAmbi}.
    By \cite[Proposition 2.54 and Proposition 4.15]{moshe2021higher} the category $\cMod_R^\dbl$ is stable and has $\pi$-finite $p$-space indexed colimits (in fact, it is $p$-typically $\infty$-semiadditive), and is clearly $\Lnf$-local, i.e., it is in $\CatLnfpfin$. 
    Since the inclusion $\CatLnfpfin \into \Catpfin$ 
    is $p$-typically $\infty$-semiadditive, cardinalities in the source are computed as in the target, so in particular, by \cite[Proposition 7.6]{moshe2021higher}, we have
    \[
        |A|_{\cMod_R^\dbl} = A\otimes R \qin 
        \pi_0(\cMod_R^\dbl).
    \]
    Combining this with \Cref{KTnp_Sadd}, which implies that $\KTnp$ preserves cardinalities of $\pi$-finite $p$-spaces, we get
    \[
        |A|_{\KTnp(R)} = [A\otimes R] \qin 
        \pi_0(\KTnp(R)).
    \]
\end{proof}

We now specialize to the case of the Lubin--Tate spectrum $R = E_n$. Since $E_n$ is even-periodic with $\pi_0(E_n)$ a complete regular local ring, we have by \cite[Proposition 10.11]{AkhilGalois} that $\cMod_R^\dbl = \Mod_R^\omega$. Thus, the above considerations regarding algebraic $K$-theory of perfect vs.\ dualizable modules apply even before $T(n+1)$-localization. Moreover, in this case, the equality of \Cref{card-K} is essentially of integers. 

\begin{prop}\label{K0_Z}
    For every $n$, we have an isomorphism of rings
    \(
       \pi_0(K(E_n)) \simeq \ZZ.
    \)
\end{prop}

\begin{proof}
    We observe that in \cite[Proposition 10.11]{AkhilGalois}, the proof of the ``local and dualizable implies perfect'' direction does not use retracts. Thus, a perfect $E_n$-module is just an iterated cofiber of finite free $E_n$-modules, hence its class in $\pi_0(K(E_n))$ is an integer multiple of the unit $[E_n]$.
    Furthermore, the classes $k[E_n]$ are distinct from one another since they are distinguished by the dimension map $\pi_0(K(E_n)) \to \pi_0(E_n)$.
\end{proof}

\begin{rem}\label{dim_dim}
    \Cref{K0_Z} implies the well known fact that
    \[
        \dim_{E_n}(M) = 
        \dim_{\FF_p}(\pi_0(K(n)\otimes_{E_n}M)) - 
        \dim_{\FF_p}(\pi_1(K(n)\otimes_{E_n}M))
    \]
    for every dualizable $K(n)$-local $E_n$-module $M$. Indeed, both sides of the equation are additive invariants, hence factor through $\pi_0(K(E_n)) \simeq \ZZ$, and agree on the unit $[E_n]$. 
\end{rem}

We are now in position to prove our main theorem.

\begin{proof}[Proof of \cref{card-blue}]
    By \cref{K0_Z}, a class $[M] \in \pi_0(K(E_n))$ is just an integer, which can be further identified with $\dim(M) \in \pi_0(E_n)$. Thus, combining \cref{card-K} and \cite[Corollary 3.3.10]{TeleAmbi}, we have an equality of integers
    \[
        |A|_{\KTnp(E_n)} = [A \otimes E_n] = \dim(A \otimes E_n) = |LA|_{E_n}.
    \]
    It remains to observe that since $\KTnp(E_n) \neq 0$ by the seminal redshift result of \cite[Theorem A]{yuan2021examples}, we can apply the chromatic nullstellensatz in the form of \cite[Theorem D]{Null}, to get a map of commutative algebras 
    \[
        \KTnp(E_n) \too E_{n+1}(\kappa)
    \]
    for some algebraically closed field $\kappa$. Finally, both this map and the map $E_{n+1} \to E_{n+1}(\kappa)$ induced by $\cl{\FF}_p \into \kappa$ preserve cardinalities, so we get 
    \[
        |A|_{E_{n+1}} = |A|_{\KTnp(E_n)} = |LA|_{E_n}.
    \]
\end{proof}



\section{Acknowledgements}

The second author is partially supported by the Danish National Research Foundation through the Copenhagen Centre for Geometry and Topology (DNRF151).
The third author was supported by ISF1588/18, BSF 2018389 and the ERC under the European Union's Horizon 2020 research and innovation program (grant agreement No. 101125896). The fourth author was supported by ISF1848/23.


\bibliographystyle{alpha}
\phantomsection\addcontentsline{toc}{section}{\refname}
\bibliography{cyclored}


\end{document}

