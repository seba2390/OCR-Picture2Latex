%\documentclass[10pt]{article}
%\usepackage[usenames]{color} %used for font color
%\usepackage[utf8]{inputenc} %useful to type directly diacritic characters
%\usepackage[pdftex]{graphicx}
%\usepackage[top=2cm, bottom=2cm, left=2.2cm, right=2.1cm]{geometry}
%\usepackage[top=1.9cm, bottom=1.9cm, left=1.9cm, right=1.9cm]{geometry}
%\usepackage{setspace}
%\usepackage{floatflt}
%\usepackage{amsfonts}
%\usepackage{times}
%\usepackage[toc]{appendix}
\usepackage{subfigure}
%\usepackage{epsfig,epstopdf}
%\usepackage{url}
%\usepackage{array}
\usepackage{amsmath}
\usepackage{amssymb}
\usepackage{amsthm}
%\usepackage{latexsym}
%\usepackage{pstricks,pst-node,pst-grad,pst-xkey}
%\usepackage{float}
%\usepackage{multirow}
%\usepackage{cite}
%\usepackage{algorithm,algorithmic}
%\usepackage{psfrag}
%\usepackage{xy} 
%\usepackage{fancyhdr}
%\usepackage{setspace} 
%\usepackage{rotating} 
%\usepackage{citesort}
%\usepackage{cite} 
%\usepackage{verbatim} 
%\usepackage{multicol} 
%\usepackage{longtable}
%\usepackage{booktabs}
%\usepackage{pdfsync} 
%\usepackage{pstricks,pst-node,pst-grad,pst-xkey}
%\usepackage{epigraph}
\usepackage{tikz}
%\usepackage{gantt}
%\usepackage[lined,boxed,commentsnumbered]{algorithm2e}
%\usepackage{algorithm2e}
%\usepackage{color}
%\usepackage{algorithmicx}
%\usepackage[noend]{algpseudocode}


\newcommand{\stevecom}[1]{\textcolor{red} {\emph{Steve says: #1}}}
\newcommand{\jcom}[1]{\textcolor{cyan} {\emph{Jan says: #1}}}
\newcommand{\mikecom}[1]{\textcolor{yellow} {\emph{Mike says: #1}}}
% THEOREMS -------------------------------------------------------
\newtheorem{thm}{Theorem}[section]
\newtheorem{cor}[thm]{Corollary}
\newtheorem{lem}[thm]{Lemma}
\newtheorem{prop}[thm]{Proposition}
\newtheorem{alg}[thm]{Algorithm}
\newtheorem{conj}[thm]{Conjecture}
\theoremstyle{definition}
\newtheorem{defn}[thm]{Definition}
\theoremstyle{remark}
\newtheorem{rem}[thm]{Remark}
\newtheorem{remark}[thm]{Remark}
\newtheorem{ques}[thm]{\textbf{QUESTION}}
\newtheorem{ex}[thm]{\textbf{Example}}
\newtheorem{ass}[thm]{\textbf{Assumption}}
%\newenvironment{proof}{\textbf{Proof:}}{\hfill$\square$}
% MATH -----------------------------------------------------------
\newcommand{\matnorm}[1]{{\left\vert\kern-0.25ex\left\vert\kern-0.25ex\left\vert #1 
    \right\vert\kern-0.25ex\right\vert\kern-0.25ex\right\vert}}
\newcommand{\opnorm}[1]{{\left\vert\kern-0.25ex\left\vert\kern-0.25ex\left\vert #1 
    \right\vert\kern-0.25ex\right\vert\kern-0.25ex\right\vert}}		
\newcommand{\specnorm}[1]{\matnorm{#1}_2}
\newcommand{\norm}[1]{\left\Vert#1\right\Vert}
\newcommand{\normtop}[1]{\left\Vert#1\right\Vert^p}
\newcommand{\Ltwonorm}[1]{\left\Vert#1\right\Vert_{\Ltwo}}
\newcommand{\abs}[1]{\left\vert#1\right\vert}
\newcommand{\abstop}[1]{\left\vert#1\right\vert^p}
\newcommand{\Set}[1]{\left\{#1\right\}}
\newcommand{\Real}{\mathbb R}
\newcommand{\Complex}{\mathbb C}
\newcommand{\integer}{\mathbb Z}
\newcommand{\nat}{\mathbb N}
\newcommand{\eps}{\varepsilon}
\newcommand{\To}{\longrightarrow}
\newcommand{\BX}{\mathbf{B}(X)}
\newcommand{\A}{\mathcal{A}}
\newcommand{\aff}[1]{\text{aff } #1}
\newcommand{\dom}[1]{\text{dom } #1}
\newcommand{\epi}[1]{\text{epi } #1}
\newcommand{\ri}[1]{\text{ri } #1}
\newcommand{\cl}[1]{\text{cl } #1}

\newcommand{\QM}[1]{\ensuremath{\mathcal{QM}(#1)} }

\newcommand{\proj}{\Pi_{\text{proj} \,}}
%-----------------------------------------
\newcommand{\argmin}{\text{argmin}}
\newcommand{\argmax}{\text{argmax}}
\newcommand{\arginf}{\text{arginf}}
\newcommand{\argsup}{\text{argsup}}

\newcommand{\sgn}{\ensuremath{\text{sgn}}}
% fuer skalare:
%\newcommand{\s}[1]{\mathfrak{#1}}
\newcommand{\s}[1]{{#1}}
% fuer vectoren:
%\newcommand{\vc}[1]{\mathbf{#1}}
\newcommand{\vc}[1]{#1}
% scalar product:
\newcommand{\Sp}[1]{\ensuremath{\mathbf{\langle} #1 \mathbf{\rangle}}}
\newcommand{\SP}[2]{\ensuremath{\mathbf{\langle} \vc{#1} \mathbf{,} \vc{#2} \mathbf{\rangle}}}
\newcommand{\SPL}[2]{\ensuremath{\mathbf{\langle} \vc{#1}\mathbf{,} \vc{#2} \mathbf{\rangle}_{L_2}}}
\newcommand{\SPLintI}[2]{\ensuremath{\int_I {#1(t)}\, {\overline{#2}(t)} \, {d\mu(t)}}}
\newcommand{\SPLtwo}[2]{\ensuremath{\mathbf{\langle} \vc{#1}\mathbf{,} \vc{#2} \mathbf{\rangle}_{\Ltwo}}}

\newcommand{\normLtwo}[1]{\ensuremath{\norm{ #1}_{\Ltwo}}}

\newcommand{\ball}[2]{\ensuremath{\mathfrak B_{\vc{#1}} \bigl( #2\bigr) }}

\newcommand{\pp}[1]{\langle #1 \rangle}

%HS:
\newcommand{\hs}[1]{\ensuremath{\mathcal{#1}}}
%operators:
\newcommand{\op}[1]{\ensuremath{ {#1}}}

\newcommand{\postmean}{\mathfrak m}

\newcommand{\dif}[2]{{\operatorname{d}\over\operatorname{d}#2}#1}
\newcommand{\pdif}[2]{{\operatorname{\partial}\over\operatorname{\partial}#2}#1}
\newcommand{\ppdif}[3]{{\operatorname{\partial^2}\over\operatorname{\partial}#2 \partial#3}#1}
%\renewcommand{\d}[1]{\operatorname{d}#1}
\renewcommand{\d}[1]{\text{ d}#1}

\newcommand{\floor}[1]{ \left\lfloor #1 \right\rfloor }
\newcommand{\ceil}[1]{ \left\lceil #1 \right\rceil }
\newcommand{\fexp}[1]{\ensuremath{\exp\bigl( #1 \bigr)}}

\newcommand{\errinp}{\varepsilon_{\text{inp}}}

\newcommand{\bra}[1]{\langle #1 \vert}
\newcommand{\ket}[1]{\vert #1 \rangle}

%Fourier Operator:
\newcommand{\FT}[1]{\mathcal F #1}

\newcommand{\spec}{\sigma }
\newcommand{\specrad}{\rho}

\newcommand{\radfct}{\ensuremath{\mathfrak r}}
%Koerper:
\newcommand{\field}{\mathbb F}
%nullspace:
\newcommand{\ns}{\mathcal{NS}}
\newcommand{\GP}{\mathcal {GP} }
\newcommand{\Gauss}{\mathcal {N} }
\newcommand{\gauss}{p_{\mathcal {N}} }
\newcommand{\expect}[1]{\ensuremath{ \langle#1\rangle } }
\newcommand{\cexpect}[2]{\ensuremath{ \langle#1 \, \vert \, #2\rangle } }
\newcommand{\cov}[2]{\ensuremath{ \text{cov}(#1,#2) } }
\newcommand{\covmat}[2]{\ensuremath{ \text{Cov}(#1,#2) } }
\newcommand{\var}[1]{\ensuremath{ \text{var}[#1]} }
\newcommand{\varmat}[1]{\ensuremath{ \text{Var}[#1]} }
\newcommand{\data}{\ensuremath{ \mathcal D} }

\newcommand{\varmatnormbnd}{\ensuremath{ V}}

\newcommand{\rank}{\ensuremath{ \text{ rank }}}
%parameter:
\newcommand{\pardop}{\ensuremath{\omega}}%parameter for lin. Dif OP


\newcommand{\param}{\ensuremath{\Xi}}%parameter for lin. Dif OP


\newcommand{\pardopspace}{\ensuremath{\Omega}}%parameter space for lin. OP

\newcommand{\pardi}{\ensuremath{\mathfrak p}}%parameter for %inhomogeneous ODE-part / driving force

\newcommand{\pardispace}{\ensuremath{\mathfrak P}}%parameter for %inhomogeneous ODE-part / driving force

\newcommand{\state}{\ensuremath{ \vc x}} % state of the system... integral curve of the ODE/ signal.
\newcommand{\isv}{\ensuremath{ \vc s}} % interaction space vector

\newcommand{\statespace}{\ensuremath{ \mathcal X}}
\newcommand{\configspace}{\ensuremath{ \mathcal Q}}
\newcommand{\inspace}{\ensuremath{ \mathcal X}}
\newcommand{\outspace}{\ensuremath{ \mathcal Y}}
\newcommand{\taskspace}{\ensuremath{ \mathcal }}
\newcommand{\ctrlspace}{\ensuremath{ \mathcal U}}
\newcommand{\fctspace}{\ensuremath{ \mathcal F}}
\newcommand{\hilspace}{\ensuremath{ \mathcal H}}
\newcommand{\HS}{\ensuremath{ \mathcal H}}
\newcommand{\samplespace}{\ensuremath{\Omega}}
\newcommand{\freespace}{\ensuremath{ \mathfrak F}}
\newcommand{\iaspace}{\ensuremath{ \mathcal S}} %interaction space
\newcommand{\policyspace}{\ensuremath{ \mathcal P_u}} %policy space


\newcommand{\grid}{\ensuremath{  G}}

\newcommand{\tube}{\ensuremath{ \mathbb T}}

\newcommand{\Ltwo}{\ensuremath{\mathcal L_2 \,}}

\newcommand{\dW}{\ensuremath{\, \text{d}W}}

\newcommand{\linopspace}{\ensuremath{ \mathcal L}}

\newcommand{\rvc}{\ensuremath{ \vc{X}}}
\newcommand{\rv}{\ensuremath{ {X}}}

\newcommand{\indset}{\ensuremath{ \mathcal I}}
\newcommand{\indsett}{\ensuremath{ {\mathcal I_{t}}}}

\newcommand{\di}{\ensuremath{ u}} % driving force / control input

\newcommand{\agi}{\ensuremath{ \mathfrak a}} %an agent index
\newcommand{\agii}{\ensuremath{ \mathfrak r}} %index if another agent
\newcommand{\agiii}{\ensuremath{ \mathfrak q}} %index if another agent

\newcommand{\agdiam}{\ensuremath{ {\Lambda^{\agi,\agii}}} }
\newcommand{\nagi}{\ensuremath{ {\neg \agi}} }
\newcommand{\nagii}{\ensuremath{ {\neg \agii}} }
\newcommand{\nagiii}{\ensuremath{ {\neg \agiii}}} 

\newcommand{\collcritfct}{\gamma^{\agi,\agii}}

\newcommand{\nuref}{\nu_{\text{ref}}}
\newcommand{\nufb}{\nu_{\text{fb}}}

\newcommand{\xref}{x_{\text{ref}}}
\newcommand{\dxref}{\dot x_{\text{ref}}}
\newcommand{\coupleset}{\mathcal C}
\newcommand{\confl}{\text{confl}}
\newcommand{\tailrad}{\ensuremath{ {\varrho}} }

\newcommand{\maxerr}{\bar{\mathfrak v}} %max error in Hfe
\newcommand{\maxerrn}{\bar{\mathfrak N}} %maximum error norm

\newcommand{\metric}{\, \mathfrak{d}} % distance metric
\newcommand{\metricp}{\,  \tilde {d}} % distance metric
\newcommand{\Metric}[2]{\metric\bigl(#1,#2\bigr) }
\newcommand{\Metrici}[2]{\metric\bigl(#1,#2\bigr) }
\newcommand{\Metrico}[2]{\metric_\outspace\bigl(#1,#2\bigr) }
\newcommand{\Metrictop}[2]{\metric^p\bigl(#1,#2\bigr) }
\newcommand{\Metrictopi}[2]{\metric^p_\inspace\bigl(#1,#2\bigr) }
\newcommand{\Metrictopo}[2]{\metric^p_\outspace\bigl(#1,#2\bigr) }
\newcommand{\dist}{\ensuremath{ \text{dist}}} % driving force / control input

\newcommand{\predf}{\, \mathfrak{  \hat f}} % hypothesis
\newcommand{\predfn}{\, \mathfrak{  \hat f_n}} % hypothesis
\newcommand{\predfnj}{\, \mathfrak{  \hat f_{n,j}}} % hypothesis
\newcommand{\prederr}{\, \mathfrak{\hat v}} % error around hyp
\newcommand{\prederrn}{\, \mathfrak{\hat v_n}} % error around hyp
\newcommand{\prederrnj}{\, \mathfrak{\hat v_{n,j}}} % error around hyp
\newcommand{\prederrbox}{\, {\hat H}} % error around hyp

\newcommand{\noise}{{\nu}} % error around hyp


\newcommand{\hexp}{{ \alpha }}%hoelder exponent
\newcommand{\hconst}{\ensuremath{ L }}%hoelder const
\newcommand{\hestthresh}{\ensuremath{ \lambda}}
\newcommand{\ubf}{\, {\bar B}} % upper bound function
\newcommand{\lbf}{\, {\underline{B}}} % lower bound function
\newcommand{\bfset}{\mathcal B}

\newcommand{\lipset}[2]{\text{Lip}_{#2}(#1)}
\newcommand{\hoelset}[3]{\mathfrak{H}_{#2}(#1,#3)}
\newcommand{\Hoelset}{\mathfrak{H}}
%\newcommand{\coeff1}{ \underline{c}} % 
%\newcommand{\coeff2}{ \overline{c}} %

\newcommand{\errhyprec}{\mathbf  E} % upper bound function
\newcommand{\obserrhyprec}{\mathbf  O} % upper bound function

\newcommand{\obserrpar}{e} % parameter of presupposed obserr
\newcommand{\obserr}{\mathfrak e} % upper bound function on observational noise
\newcommand{\obserrbnd}{\bar{\mathfrak e}}

\newcommand{\agset}{\ensuremath{ \mathfrak A}} %set of Agents
\newcommand{\agsubset}{\ensuremath{ \mathfrak S}} %set of Agents

\newcommand{\indicator}[1]{\ensuremath{\textbf{1}_{#1}}} % indicator function

\newcommand{\collevent}{\ensuremath{\mathfrak C}}
\newcommand{\colleventtwoA}{\ensuremath{\mathfrak C^{\agi,\agii}}}
%\newcommand{\hull}{\ensuremath{\mathfrak C}}

\newcommand{\decke}{\ensuremath{\mathfrak u}}
\newcommand{\boden}{\ensuremath{\mathfrak l}}
\newcommand{\einschluss}{\ensuremath{\mathcal E^{\decke_{n}}_{\boden_{n}}}}
\newcommand{\opteinschluss}{\ensuremath{\mathcal E^*_n}}
\newcommand{\Kprior}{\mathcal K_{\text{prior}} }
\newcommand{\Kdat}{\mathcal K(D_n)}

\newcommand{\menge}[2]{\ensuremath{\{#1 \, \vert \, #2 \}}}

\newcommand{\seq}[2]{\ensuremath{\bigl(#1\bigr)_{#2}}}

%\newcommand{\diag}{\ensuremath{ \text{diag}} %set of

%\newcommand{\e}[1]{\ensuremath{\exp \Bigl(#1\Bigr) }}
\newcommand{\card}[1]{\ensuremath{\left\vert #1 \right\vert }}

\renewcommand{\Pr}{\mathrm{Pr}}
\newcommand{\sqmats}{\mathcal{M}} %set of square matrices


\renewcommand{\d}{\ensuremath{\text{ d}}}
\newcommand{\KLD}[2]{\text{KLD}(#1||#2) }

\newcommand{\erw}[2][]{
%    \ensuremath{\pmb{\mathcal{M}}^{\operatorfont{#1}}}%
     \ensuremath{\mathbb{E}_{#1} \! \left\lbrace #2 \right\rbrace}%
}


%----------- probability stuff------------------------
\newcommand{\pmeas}{\ensuremath{\mathbb P}} % probability measure


\newcommand{\embedding}{\ensuremath{\tilde \mu}} % probability measure
\newcommand{\ewx}[1]{\ensuremath{\mathbb E_x[ #1 ] }} % expected value
\newcommand{\ew}{\ensuremath{\mathbb E }} % expected value
\newcommand{\is}{\ensuremath{\mathfrak X }} % input space / index set
\newcommand{\Vol}{\ensuremath{\text{Vol}}}
\newcommand{\mslim}{\ensuremath{\text{m.s.-lim}}}
\newcommand{\msto}{\ensuremath{\stackrel{ \text{2}}{\longrightarrow}}}

% --------------------environments--------------------------------------------
\newcommand{\beq}{\begin{equation}}
\newcommand{\eeq}{\end{equation}}
\newcommand{\eq}[1]{\begin{equation} #1 \end{equation}}
\newcommand{\eqn}[2]{\begin{equation} #2 \label{#1}\end{equation}}
\newcommand{\dt}{\ensuremath{ \, dt \,}}
%\newcommand{\dW}{\ensuremath{ \, dW \,}}
\newcommand{\tinc}{\ensuremath{ \Delta}}

\newcommand{\tr}{\ensuremath{ \text{ tr}}}

\newcommand{\rlem}[1]{Lem. \ref {#1}} % indicator function
\newcommand{\rthm}[1]{Thm. \ref {#1}} % indicator function
%\newcommand{\rch}[1]{Ch. \ref {#1}} % indicator function
% ----------------------------------------------------------------
\newcommand{\convton}{\stackrel{n \to \infty}\longrightarrow}
\newcommand{\convto}{\longrightarrow}
\newcommand{\setconvto}{\longrightarrow}
\newcommand{\setconvtounif}{\stackrel{\text{unif}}{\longrightarrow}}
\newcommand{\bd}{\leadsto} %becomes dense relative to 
\newcommand{\bdu}{\twoheadrightarrow} %becomes dense relative to uniformly

\newcommand{\errmetric}{\mathcal E} %becomes dense relative to uniformly
\newcommand{\queryset}{I} %becomes dense relative to uniformly

%-------------- Meinzeugs Ende-----------------------------------------