  In this paper, we explore the connection between secret key agreement and secure omniscience within the setting of the multiterminal source model with a wiretapper who has side information. While the secret key agreement problem considers the generation of a maximum-rate secret key through public discussion, the secure omniscience problem is concerned with communication protocols for omniscience that minimize the rate of information leakage to the wiretapper. The starting point of our work is a lower bound on the minimum leakage rate for omniscience, $\rl$, in terms of the wiretap secret key capacity, $\wskc$. Our interest is in identifying broad classes of sources for which this lower bound is met with equality, in which case we say that there is a duality between secure omniscience and secret key agreement. We show that this duality holds in the case of certain finite linear source (FLS) models, such as two-terminal FLS models and pairwise independent network models on trees with a linear wiretapper. Duality also holds for any FLS model in which $\wskc$ is achieved by a perfect linear secret key agreement scheme. We conjecture that the duality in fact holds unconditionally for any FLS model. On the negative side, we give an example of a (non-FLS) source model for which duality does not hold if we limit ourselves to communication-for-omniscience protocols with at most two (interactive) communications.  We also address the secure function computation problem and explore the connection between the minimum leakage rate for computing a function and the wiretap secret key capacity.
  
%   Finally, we demonstrate the usefulness of our lower bound on $\rl$ by using it to derive equivalent conditions for the positivity of $\wskc$ in the multiterminal model. This extends a recent result of Gohari, G\"{u}nl\"{u} and Kramer (2020) obtained for the two-user setting.
  
   
%   In this paper, we study the problem of secret key generation through an omniscience achieving communication that minimizes the 
%   leakage rate to a wiretapper who has side information in the setting of multiterminal source model.  We explore this problem by deriving a lower bound on the wiretap secret key capacity $\wskc$ in terms of the minimum leakage rate for omniscience, $\rl$. 
%   %The former quantity is defined to be the maximum secret key rate achievable, and the latter one is defined as the minimum possible leakage rate about the source through an omniscience scheme to a wiretapper. 
%   The main focus of our work is the characterization of the sources for which the lower bound holds with equality \textemdash it is referred to as a duality between secure omniscience and wiretap secret key agreement. For general source models, we show that duality need not hold if we limit to the communication protocols with at most two (interactive) communications. In the case when there is no restriction on the number of communications, whether the duality holds or not is still unknown. However, we resolve this question affirmatively for two-user finite linear sources (FLS) and pairwise independent networks (PIN) defined on trees, a subclass of FLS. Moreover, for these sources, we give a single-letter expression for $\wskc$. Furthermore, in the direction of proving the conjecture that duality holds for all FLS, we show that if $\wskc$ is achieved by a \emph{perfect} secret key agreement scheme for FLS then the duality must hold. All these results mount up the evidence in favor of the conjecture on FLS. Moreover, we demonstrate the usefulness of our lower bound on $\wskc$ in terms of $\rl$ by deriving some equivalent conditions on the positivity of secret key capacity for multiterminal source model. Our result indeed extends the work of Gohari, G\"{u}nl\"{u} and Kramer in two-user case.


\section{Introduction}

Typically, a controller is designed on the basis of a dynamical model of the system. When little is known about these dynamics a priori or, if the dynamics may be subject to unexpected change, machine learning methods can be employed to learn such a model online on the basis of measurements. 

Supervised machine learning methods are algorithms for inductive inference. On the basis of a sample, they construct (learn) a computable model of a data generating process that facilitates inference over the underlying ground truth function and aims to predict its function values at unobserved inputs.
%
Among supervised learning methods, nonparametric algorithms tend to offer greater flexibility to learn rich 
function classes (e.g. rich classes of nonlinear dynamics). 
 
Perhaps the most popular nonparametric machine learning method is Bayesian inference with \textit{Gaussian processes (GPs)} \cite{GPbook:2006}. GPs offer a flexible and principled probabilistic method for nonparametric regression and have evolved into one of the primary work-horses for learning dynamic systems \cite{Deisenroth2009,Deisenroth2011,Deisenroth2015,Tuongmodellearningsurvey2011,deisenrothsurvey2013,McHutchonthesis2014} in the research communities related to artificial intelligence. 
However, they suffer from several limitations, including scalability to large data sets, a lack of understanding of how to bound the closed-loop dynamics resulting from controlling on the basis of a GP state-space model and the question of how to choose a good prior in a principled yet computable manner. To alleviate the last problem, it is common practice to tune hyper-parameters of a chosen (typically universal) kernel to explain the data via the marginal log-likelihood. While often successful on many data sets, the result can be highly sensitive to the choice of optimiser, initialisations, data sets and computational budget. Unfortunately, little theoretical understanding of the important interplay between these components in the resulting inference mechanism seems to exist.
%
%Apart from the confinement to Gaussian a priori knowledge, the GP approach suffers from a few additional limitations. Firstly, the computational complexity for learning grows cubically and for computing the prediction variances quadratically with the training data size. This either renders the approach intractable in data rich control applications or necessitates approximations. These turn the Bayesian nonparametric method into a parametric method with somewhat limited flexibility and whose uncertainty quantifications may be hard to compute. 
%Secondly, it seems to be an open problem of how to conservatively quantify the uncertainty of the solution trajectory of the resulting uncertain dynamic system. Therefore, current methods rely on approximations whose effects on the uncertainty are not well understood (see \cite{McHutchonthesis2014} for a discussion of some of the most recent approaches). The latter deficiency is a serious shortcoming in decision making in the presence of state constraints. For instance, if it is hard to quantify or bound the probability of a trajectory being within an obstacle-free region then it will be hard to compute controls that guarantee collision-free state trajectories. 
%Finally, the uncertainties and prediction performance depend on parameter choices of the assumed prior. Since in practice, it is often hard to conceive a good prior a priori, most authors working with GPs seem to adopt a pragmatic approach and choose a prior with a maximally flexible covariance function whose parameters are (a posteriori) computed maximise the marginal log-likelihood of the data (cf. e.g. \cite{GPbook:2006,Deisenroth2015}). Apart from being a departure from Bayesian orthodoxy, this approach has a number of downsides. Firstly, the performance is sensitive to the choice of optimisation algorithm and relies on finding a good (local) maximum. Secondly, the optimisation approach introduces substantial computational overhead that currently can be prohibitive in online learning scenarios where the data needs to be updated incrementally under real-time constraints.
%Furthermore, we are not aware of any theoretical guarantees on the learning success of the inference algorithm resulting from combining the GP with a hyper-parameter optimiser. 

In contrast to such Bayesian methods this work considers an extension of ideas existing in applied mathematics  (e.g. \cite{Shubert:72,Zabinsky2003,Cooper1995,Cooper2006,Baran2008,Beliakov2006}) as well as in control \cite{Milanese2004,Canale2014} that harnesses Lipschitz regularity of the target function to provide bounds on the predictions of the target function at unobserved inputs. Applied to machine learning, the basic idea is that Lipschitz continuity constrains the set of possible function values of a target function at a query input as a function of the distance between the query and the previously observed training examples. A prediction is then made by choosing a function value in the middle of the set of possible function values. This idea, which at least goes back to \cite{Sukharev1978}, has been leveraged in different fields under different headlines including \emph{Lipschitz Interpolation} \cite{Zabinsky2003,Beliakov2006} and \emph{Nonlinear Set Membership (NSM)}  methods \cite{Milanese2004}. If the Lipschitz constant is known, Lipschitz interpolation provides uncertainties around the predictions of function values at unobserved inputs. The uncertainties are maximally tight if no other knowledge is known other than the presupposed Lipschitz regularity \cite{Sukharev1978,calliess2014_thesis}.
The presupposed Lipschitz constant is a crucial parameter of the inference rule, quite similar to the choice of a prior (e.g. via kernel and hyperparameters)  in Bayesian inference.

We would argue that one of the advantages of these methods is their computational simplicity. That is, they are numerically robust and only involve basic computational steps that could be more efficiently computed even on a simple embedded RISC micro-controller.
A practical concern is that the predictions and bounds of these methods hinge on the a priori knowledge of the presupposed Lipschitz parameter of the underlying target function. Some previous works remark that, in the absence of such knowledge the constants might in practice be estimated from the data (e.g. via estimators discussed in  \cite{Strongin1973,Wood1996,Beliakovsmoothing2007,Milanese2004,Beliakov2006}). In fact, \cite{Milanese2004} suggest fitting a parametric regressor to the data first and utilise the Lipschitz constant of the fitted model for the NSM approach. Unfortunately, there is no theoretical analysis given anywhere what the impact of that estimate is to the predictive performance of the regression method. (And, consequently control methods that rely on the resulting predictions would be unable to assert closed-loop stability or robustness guarantees).

Our work addresses this deficiency by proposing an that allows us to provide learning and stability guarantees even in the absence of a priori knowledge of this constant.

 As a first step, we rehearse \emph{Kinky Inference (KI)} \cite{calliess2014_thesis} which generalises the Lipschitz interpolation and NSM frameworks in several ways.
%
%In its basic setup, \cite{calliess2014_thesis}, our generalisations yield 
%a nonparametric machine learning method that facilitates the conversion of assumptions about H\"older continuity (and possibly boundedness) into a set-based a posteriori inference rule over multi-output functions on pseudo-metric spaces. Its prediction time grows at most linearly with the training data size. Learning merely requires the computational effort for storing the training data in memory. Furthermore, it allows us to provide worst-case bounds on the norm of the error around the inferred vector-valued predictions. Since the prediction surface of the inference rule will generally exhibit kinks, we refer to it as \textit{kinky inference (KI)}.
We then combine the KI machine learning approach with a simple online parameter estimator that allows us to incrementally compute an estimate of the H\"older constant on the basis of incrementally arriving (noisy) data. This merger yields a new inference rule we refer to as \emph{Lazily Adapted Constant Kinky Inference (LACKI)}. We prove that this rule is sample-consistent (up to the level observational error in the data) and that the LACKI predictors themselves are H\"older continuous. This allows us to establish strong universal approximation properties: That is, in the limit of infinite data, the LACKI rule is capable of approximating not only any H\"older continuous target function but also any non-H\"older continuous function with arbitrarily low error (up to a bound dependent on the level of observational error in the training data). Since our LACKI rule can be seen as an extension of Lipschitz interpolation with empirical Lipschitz constant estimation as considered in \cite{Beliakov2006}, our results also provide new theoretical guarantees for this previously proposed interpolation rule where the Lipschitz parameters are estimated online from the noisy training data with our proposed modified constant estimation method.

In addition to these learning-theoretic considerations, we apply our LACKI approach to online-learning based model-reference adaptive control.

 As a testbed, we replicate simulations of  the roll dynamics of an F-4 fighter aircraft under uncertain wing rock previously considered by other authors in model-reference adaptive control \cite{Chowdhary2013,Monahemi1996,chowdharyacc2013} and compare our controller against their methods. Here our LACKI-based controller outperforms its competitors across a range of metrics including computational speed, prediction accuracy and tracking success. 

For discrete-time feedback-linearisable systems with uncertain nonlinear drift, we provide theoretical guarantees on the tracking success of our LACKI- model-reference adaptive controller both in the batch and in the online learning setting. 

In contrast to much of the standard literature of probabilistic nonparametric regression (e.g. \cite{Gyoerfi2002,Tsybakov2009}), our analysis focusses on the derivation of deterministic  worst-case error bounds. While possibly being more conservative, we would argue that this type of analysis has the benefit of being more meaningful in a control setting where the learner receives training examples and queries that will typically violate statistical assumptions typically presupposed  in the probabilistic literature.

The remainder of the paper is structured as follows: 

Sec. \ref{sec:LACKI_all} contains the core of the LACKI regression methods. Following a rehearsal of the kinky inference (KI) framework for nonparametric learning, Sec. \ref{sec:lacki} describes the our LACKI approach for setting the KI parameters. Sec. \ref{sec:properties_lacki} is dedicated the derivation of several properties of the resulting LACKI approach, including our consistency guarantees.

Sec. \ref{sec:MRAC_lacki} contains the control part of the paper. We first introduce the framework of model-reference adaptive control in which we propose a controller based on our LACKI learning method.  For illustration purposes, we closely follow the setting of wing-rock control considered in \cite{Chowdhary2013,ChowdharyCDC2013} and compare our LACKI-based controller to other MRAC controllers consdered and proposed by previous work. The section concludes by giving stability guarantees for LACKI-MRAC in discrete-time systems.

The paper concludes with Sec. \ref{sec:concl}, summarising our findings and containing various suggestions for future work. The appendix contains a variety of background material on 
H\"older continuity and various supplementary derivations referred to at various points of the main body of the paper. 

%\jcom{Can we now combine the remaining dynamics with an MPC controller on top that also gets rid of the observational error?}


%In Sec. \ref{sec:KI_core} we will present the inference rule and establish theoretical guarantees. In Sec. \ref{sec:KIcollavoiddiscretetime} we illustrate how the KI approach can be employed in the context of multiagent collision avoidance under uncertain agent dynamics. Linking to work in robust and stochastic predictive control \cite{Kuwata2005,Kuwata2006,Trodden2006,LyonsACC2012}, we state the planning problem as optimisation with constraint tightening to enforce collision avoidance. In the presented example, an initially unmodelled nonlinear drift causes a collision among the agents during plan execution. Introduction of our KI method yields robustness bounds that constrain the planning problem sufficiently to prevent the collision. And, the conservatism of the joint plans can be reduced with increasing training data. We believe the combination of KI learning and multi-agent control is the first work that achieves\textit{ guaranteed }collision avoidance by taking the epistemic uncertainty of a nonparametric machine learning method into account. 
 
 
 %The paper concludes with a summary and hints at some of the manifold future directions this work facilitates. 

%
%
%Typically there are infinitely many explanations that could explain any finite data set. Therefore, one needs to impose assumptions a priori in order to be able to generalise beyond the observed data and to establish the desired uncertainty quantifications. 
%
%These assumptions, sometimes referred to as inductive bias \cite{mitchellbook:97}, take different forms and are an essential distinguishing factor between different machine learning algorithms. 
%The inductive bias implicitly or explicitly restricts the hypothesis space. That is, the space of target functions that can be learned.
%
%
%
%In this work, we devise a learning method for ground truth functions that are assumed to be contained in the a priori hypothesis space $\Kprior$ of H\"older continuous functions. Moreover, these may be known to be contained within certain predefined bounds. 
%
%In the case in which the parameters of the H\"older class the function is contained in are known (or at least an upper bound on the H\"older constant and bounds) we can give robust error estimates that give worst-case guarantees on the predictions. In the event the presumed parameters turn out to be wrong, Sec. \ref{sec:uncertainconstants} contains proposals on how to update the belief over the constants in the light of new data.


\section{Kinky Inference with lazily adapted constants}


\label{sec:LACKI_all}

\subsection{Kinky Inference}
\label{sec:KI_core}
In this section, we will introduce the class of learning rules we refer to as \emph{Kinky Inference}. They encompass a host of other methods such as Lipschitz Interpolation and Nonlinear Set Interpolation.  
%
%We will then establish some theoretical guarantees for certain parameter settings, in particular of the sequence of Lipschitz parameters $L(n)$. In subsequent parts of this section we will then consider a method for adapting these parameters to the data yielding a rule we will refer to as \emph{Lazily Adapted Constant Kinky Inference (LACKI)}. The section culminates in establishing universal approximation guarantees for our LACKI rule.

The rules possess a parameter $L(n)$ that needs to be specified by any KI algorithm. In this paper, we are most concerned with studying LACKI, a KI rule algorithm where $L(n)$ coincides with a noise-robust and multi-variate generalisation of Strongin's estimate \cite{Strongin1973} of a H\"older constant computed from the data set $\data_n$ available at time step $n$. 

%We will consider the special case $\inspace = \Real^d, \outspace \subseteq \Real^m$ with pseudo-metric $\metric$ on the input space and output-space metric defined by $\metric_\outspace(x,x') := \norm{x-x'}_\infty , \forall x,x' \in \inspace$. 

%\subsubsection{Conservative learning and inference in a nutshell}
%
%The general approach we will consider is as follows. We desire to learn a function $f$ based prior knowledge of the form that $f \in \mathcal K_{prior}$ and some (noisy) function sample $\data$. 
%The task of conservative inference over functions (i.e. learning) then is the step of forming a posterior belief of the form $f \in \mathcal K_{post}= \mathcal K_{prior} \cap \mathcal K(\data)$. Here,
%$\mathcal K (\data)$ is the set of all functions that could have generated the observed data. 
%The conservatism lies in the fact that we have not ruled out any candidate functions that are consistent with our information. 
%To perform inference over a function value at input $x$ in a conservative manner, we merely infer that $f \in \prederrbox(x) =[\boden(x), \decke(x)]$ 
%where \textit{floor function} $\boden(x) =\inf_{\phi \in \mathcal K_{post}} \phi(x)$ and  \textit{ceiling function } $\decke(x) = \sup_{\phi \in \mathcal K_{post}} \phi(x)$ delimit the values that any function in the posterior class $\mathcal K_{post}$ could give rise to. If a point prediction of a function value is required, we could choose any function value between the values given by the floor and ceiling which also give rise to the bounds on the point prediction. 
%Of course, the type of inference we get from this general procedure depends on the chosen prior class $\mathcal K_{prior}$ and the specifics of the data. We will develop \textit{kinky inference} which is a method for considering prior classes of multi-output functions over metric input spaces that are optionally constrained by boundedness and by knowledge of H\"older regularity as well as knowledge on uncertainty bounds on the observations and function inputs. For this class of inference mechanisms we prove conservatism and convergence guarantees to the learning target. What is more, we will consider classes where the imposed regularity assumptions are uncertain or adapted lazily to adjust for overly restrictive prior assumptions that do not fit the observations.
%
%
%
%This capability of imposing boundedness assumptions, accommodating for noise and being able to adjust the regularity assumptions will prove of great importance in the 
%control applications of our method.
%
%\subsection{The framework of conservative inference -- definitions, setting and desiderata }
%\label{sec:prob_def}

%As a basic example, consider the ordinary differential equation $\frac{\d x}{\d t} = (\xi - x)$ where $\xi: I \subset \Real \to \Real $ is an absolutely continuous function superimposed by a countable number of steps. Suppose, the ODE describes the trajectory $x: I \to \Real$ of a controlled particle in one-dimensional state space. Based on $N$ time-state observations $D_N=(t_1,x_1 ),..., (t_N,x_N )$ we desire to estimate the total distance travelled $S$ in time interval $I$, $S = \int_I x(t) \d t$. Solving


%%\textbf{Setting.}
%Let $(\inspace, \metric)$, $(\outspace ,\metric_\outspace)$ be two metric spaces and $f: \inspace \to \outspace$ be a function. Spaces $\inspace, \outspace$ will be referred to as \textit{input space} and \textit{output space}, respectively.
%Assume, we have access to a \textit{sample set} $\data_n:= \{\bigl( s_i, \tilde f_i, \obserr(s_i) \bigr) \vert i=1,\ldots, N_n \} $ containing $N_n \in \nat$ sample vectors $\tilde f_i \in \outspace$ of function $f$ at sample input $s_i \in \inspace$. To aide our discussion, we define $\grid_n =\{s_i | i =1,\ldots,N_n\}$ to be the \textit{grid} of sample inputs contained in $\data_n$.
%
%
%The sampled function values are allowed to have interval-bounded \textit{observational error} given by $\obserr : \inspace \to \Real^m_{\geq 0}$. That is, all we know is that $f(s_i) \in \obserrhyprec_i := [\underline f_1(s_i), \overline f_1(s_i) ] \times \ldots \times [\underline f_d(s_i), \overline f_d(s_i) ]$ where $\underline f_j(s_i) := \tilde f_{i,j} - \obserr_j(s_i)$, $\overline f(s_i) := \tilde f_{i,j} + \obserr_j(s_i)$ and $\tilde f_{i,j}$ denotes the $j$th component of vector $\tilde f_{i}$.
%
%The interpretation of these errors depends on the given application. For instance, in the context of system identification, the sample might be based on noisy measurements of velocities. The noise may be due to sensor noise or may represent numerical approximation errors. As we will see below, $\obserr$ may also quantify \textit{input} uncertainty (that is when predicting $f(x)$, $x$ is uncertain). This is an important case to consider, especially in the context of control applications where the ground truth might represent a vector field or an observer model.
%
%
%It is our aim to learn function $f$ in the sense that, folding in knowledge about $\data_n$, we infer \textit{predictions} $\predf(x)$ of $f(x)$ at unobserved \textit{query inputs} $x \notin \grid_n$. Being the target of learning, we will refer to $f$ as the \emph{target} or \emph{ground-truth} function. 
%
%In addition to the predictions themselves, we are also interested in \textit{conservative bounds} on the error of the inferred predictions. Depending on the hypothesis space under consideration these bounds can be of a probabilistic or deterministic nature. In this chapter, for the time being, we are interested in the latter.
%
%That is, we desire to provide a computable \textit{prediction uncertainty} function $\prederr: \inspace \to \outspace$ such that we can guarantee that for any query $x$, our prediction $\predf(x)$ is within the cube of error given by $\prederr(x)$ as follows 
%
%
%Being interested in conservative worst-case guarantees, we are mostly interested 
%
%
%
%In order to be able to deduce the error bounds around the predictions $\predf$, it is important to impose a priori restrictions on the class of possible targets (the hypothesis space). 

%\textbf{Definitions.}
%
\textbf{Setting.}  Let $\inspace$, $\outspace$ be two spaces endowed with (pseudo-) metrics $\metric: \inspace^2 \to \Real_{\geq 0}, \metric_\outspace:\outspace^2 \to \Real_{\geq 0}$, respectively. Spaces $\inspace, \outspace$ will be referred to as \textit{input space} and \textit{output space}, respectively. 
It will be convenient to restrict our attention to input and output spaces that are additive abelian groups and which are \emph{translation-invariant} with respect to their (pseudo-) metrics. That is, for the input space $\inspace$, we assume $ \metric(x+x',x''+x') = \metric(x,x''),\forall x,x',x'' \in \inspace$. 

For simplicity, throughout the remainder of this work, we will assume the output space is the canonical Hilbert space $\outspace = \Real^m$ $(m \in \nat)$ endowed with the $\metric_{\outspace}(y,y') = \norm{y-y'}_\infty, \forall y,y' \in \outspace$. 

Let $f: \inspace \to \outspace$ be a \emph{target} or \emph{ground-truth} function we desire to learn. For our purposes, learning means regression. That is, we utilise the data to construct a computable function that allows us predict values of the target function at any given input.

%\footnote{Extensions to more general output spaces seem very possible, albeit, instead of learning a function, we would then learn a discrepancy from a nominal reference in output space. We will leave this to future work.}
%In contrast, we do not impose restrictions on the input space (pseudo-) metric. 
%While our simulations will consider the deterministic case, it would be entirely possible to consider input spaces that are index sets of stochastic processes endowed with suitable norm metrics. Since the output metric is defined via a norm, we will often drop the subscript of the input metric. That is we may on occasion write $\metric$ instead of $\metric$.





Assume that, at time step $n$, we have access to a \textit{sample} or \textit{data set} $\data_n:= \{\bigl( s_i, \tilde f_i \bigr) \, \vert \, i=1,\ldots, N_n \} $ containing $N_n \in \nat$ sample vectors $\tilde f_i = \tilde f(s_i) \in \outspace$ of an \emph{observable}   function $\tilde f$ at sample input $s_i \in \inspace$. Here, the observable $\tilde f : \inspace \to \outspace$ is a ``noise-corrupted'' version of the true \emph{target function} $f: \inspace \to \outspace$ we would like to make inferences about on the basis of the available sample. In this work, we will typically assume that the observable $\tilde f$ coincides with the target $f$ up to a level of interval-bounded observational noise: $\forall x \in \inspace: \metric_\outspace(\tilde f(x),f(x) ) \leq \obserr(x) $ where $\obserr : \inspace \to \Real_{\geq 0}$ is the error bound function whose values we assume to be bounded by some (known or unknown) bound $\obserrbnd \in \Real_{\geq 0}$. We can model the situation by the presence of a bounded additive observational error (or ``noise'') function $\noise:\inspace \to \outspace$ with $\tilde f = f + \noise$.
%That is, all we know is that $\tilde f_i = \tilde f(s_i) \in \obserrhyprec_i := [\underline f_1(s_i), \overline f_1(s_i) ] \times \ldots \times [\underline f_d(s_i), \overline f_d(s_i) ]$ where $\underline f_j(s_i) := \tilde f_{i,j} - \obserr_j(s_i)$, $\overline f(s_i) := \tilde f_{i,j} + \obserr_j(s_i)$ and $\tilde f_{i,j}$ denotes the $j$th component of vector $\tilde f_{i}$. 
The interpretation of these errors depends on the given application and this ``noise'' may be deterministic or stochastic. For instance, in the context of system identification, the sample might be based on noisy measurements of velocities and it may be due to sensor noise or, the noise might model systematic error such as numerical approximation errors. 

Furthermore, $\obserr$ may also accommodate \textit{input} uncertainty (that is when predicting $f(x)$, $x$ is uncertain) (for details refer to \cite{calliess2014_thesis}). In the course of our theoretical considerations below the error will also serve to absorb the discrepancy between a H\"older and a non-H\"older function.
%
%This is an important case to consider, especially in the context of control applications where the ground truth might represent a vector field or an observer model. Finally, the observational error can also model error around the computation of the metric. This might be of interest if the metric has to be approximated numerically or estimated with statistical methods that provide confidence intervals (in the latter case our predictions also only hold within pertaining confidence bounds). 

\textbf{Learning.}
It is our aim to learn target function $f$ in the sense that, combining prior knowledge about $f$ with the observed data $\data_n$, we infer \textit{predictions} $\predfn(x)$ of $f(x)$ at unobserved \textit{query inputs} $x \notin \grid_n$. Here, $\grid_n =\{s_i | i =1\ldots,N_n\} \subset \inspace$ refers to the (not necessarily regular) \emph{grid} of sample inputs. The entire function $\predfn$ that is learned to facilitate predictions is referred to as the \textit{predictor}. Since the computation of the predictor is based on the available data and utilised to make inferences over unobserved inputs, we can view the learning process as an instance of (inductive) inference. Therefore, the formula to compute the predictor $\predfn$ will also be referred to as an inference rule.
%
%For a discussion of the competing definitions of non-deductive inference in a philosophical context, the reader is referred to \cite{Flach2000}. 
%
In our context, we will understand a \textit{machine learning algorithm} to implement a such an inference rule. That is, it is a computable function that maps a data set $\data_n $ to a 
 predictor $\predf_n$ (and possibly an uncertainty estimate function $\prederr_n$). 
A typical \textit{desideratum} of a good predictor is that it is \textit{efficiently computable}. Its learning performance is measured in terms of the degree and rapidity it  \textit{converges to the target} (up to the observational error given by $\obserr$) in the limit of increasingly dense data. Of course there are many different metrics with respect to which one can assess convergence. Perhaps, the most convenient one is mean-square convergence. However, inspired by our control applications, we desire to investigate worst-case convergence rates which hold independently from distributional assumptions and will yield performance guarantees even in zero-measure events.
%Typically,
%the predictor lies in the space that coincides with (or is at least dense in) the a priori hypothesis space $\mathcal K_{prior}$ of conceivable target functions. In this case the predictor can be referred to as a \textit{candidate function} or \textit{hypothesis} \cite{mitchellbook:97}.

%In addition to the predictions themselves, we are also interested in \textit{uncertainty bounds} on the error of the inferred predictions. 
%%Depending on the hypothesis space under consideration, these bounds can be of a probabilistic or deterministic nature. In this chapter, we  restrict ourselves to the latter.
%
%That is, we desire to provide a computable \textit{prediction uncertainty function} $\prederr: \inspace \to \outspace$ such that we \textit{believe} that for any query $x \in \inspace$, the ground truth value $f(x)$ lies somewhere within the \textit{uncertainty hyperrectangle }
%\begin{equation}
%	\prederrbox_n(x) := \Bigl\{ y \in \outspace \, | \, \forall j \in \{1,...,\dim{\outspace}\} : y_j \in \prederrbox_{n,j}(x)\Bigr\}
%\end{equation} around the prediction $\predfn (x)$ where 
%\begin{equation}\label{eq:prederrint}
%	\prederrbox_{n,j}(x) : = [ \predfnj(x) - \prederrnj(x), \predfnj(x) + \prederrnj(x)] 
%\end{equation}
%is referred to as the $j$th \textit{prediction uncertainty} interval.
%Here, $\predfnj(x)$, $\prederrnj(x)$ denote the $j$th components of vectors $\predfn(x), \prederrn(x)$, respectively.

 %As a word of caution, we remark that using the last statement as a definition still seems very broad. 

%\textbf{Desiderata.}


%The resulting learning or inference rule is called \textit{conservative} if the uncertainty quantifications never underestimate the error. That is, based on the an a priori assumption $f \in \mathcal K_{prior}$, we can guarantee $f(x) \in \prederrbox_n(x), \forall x$, regardless of the observed data set $\data_n$.
\textbf{The Kinky Inference Learning rules.}
In this work we will expand on the basis of the following class of predictors to perform learning as inference over unobserved function values:

\begin{defn}[Kinky Inference (KI) rule ] \label{def:KIL}
Let  $\Real_\infty := \Real \cup\{- \infty, \infty\}$ and $\inspace$ be some space endowed with a pseudo-metric $\metric$. Let $\lbf,\ubf: \inspace \to \outspace \subseteq \Real_\infty^m$ denote \textit{lower- and upper bound functions}, that can be specified in advance and assume $\lbf(x) \leq \ubf(x), \forall x \in I \subset \inspace$ component-wise. 
Furthermore, let $\obserrpar$ denote a parameter that specifies a deterministic belief about the true observational error bound $\obserr$.
Given sample set $\data_n$, we define the predictive functions $\predfn: \inspace \to \outspace, \prederrn: \inspace \to \Real^m_{\geq 0}$ to perform inference over function values.
For $j=1,\ldots,m$, their $j$th output components are given by:
%Furthermore, define
% $\coeff1,\coeff2 \in \Real_{\geq0}$ such that $\coeff1+\coeff2 =1$ (unless explicitly stated otherwise, we assume $\coeff1=\coeff2 =\frac 1 2)$.
	\begin{align*}
   \predfnj(x; \param(n)) :=& \frac{1}{2} \min\{ \ubf_j(x), \decke_{n,j}(x;\param(n)\bigr)\} \\
    &+ \frac{1}{2} \max\{ \lbf_j(x), \boden_{n,j}(x;\param(n)\bigr) \},\\
	\prederrnj(x; \param(n)) :=& \frac{1}{2} \min\{ \ubf_j(x), \decke_{n,j}\bigl(x;\param(n)\bigr)\} \\
	&- \frac{1}{2} \max\{ \lbf_j, \boden_{n,j}\bigl(x;\param(n)\bigr) \}.
	\end{align*}
	Here, $\decke_n\bigl(\cdot;\param(n)\bigr), \boden_n\bigl(\cdot;\param(n)\bigr): \inspace \to \Real^m$ are called ceiling and floor functions, respectively. Their $j$th component functions are given by
	\[\decke_{n,j}\bigl(x; \param(n)\bigr) := \min_{i=1,\ldots,N_n}   \tilde f_{i,j} +  \tilde \metric(x,s_i;\param(n)) \] and 
	\[\boden_{n,j}\bigl(x; \param(n)\bigr) := \max_{i=1,\ldots,N_n}   \tilde f_{i,j} -  \tilde  \metric(x,s_i;\param(n)),\] respectively.
  Here, $\tilde \metric (\cdot,\cdot;\param(n))$ is a mapping parameterised by $ \param(n)$. While there are many conceivable parameterisations, we restrict our attention to the case where, for some pseudo-metric $\metric$ on the input space $\inspace$, we have $\param(n) = (L(n), \hexp,\obserrpar)$ with $$\tilde \metric(\cdot,\cdot; \param(n)) = L(n) \metric^\hexp(\cdot,\cdot) + \obserrpar(x).$$
  As will be seen below,  parameter $L(n)$ has the interpretation of a H\"older constant of the predictor relative to pseudo-metric $\metric$ while $\hexp \in (0,1]$ can be interpreted as a H\"older exponent (cf. Thm. \ref{lem:LACKIpredHoelder}). That is, we will show that $\predfn(\cdot;\param(n))$ belongs to the class $$\Hoelset \bigl(L(n),\hexp\bigr) = \{ \phi: \inspace \to \outspace | \forall x,x' \in \inspace : \metric_\outspace(\phi(x),\phi(x') ) \leq L(n) \metric(x,x')^\hexp\}$$ of $L(n)-\hexp$- H\"older continuous functions. Note, we could alternatively re-express this H\"older class as a generalised class of Lipschitz functions $\text{Lip}(L(n))= \{ \phi: \forall x,x' \in \inspace: \Metrico{\phi(x)}{\phi(x')} \leq L(n) \tilde \metric (x,x')\} $ where, for any $\hexp \in (0,1]$, $\tilde \metric = \metric^\alpha$ is a pseudo-metric provided  $\metric$ is (refer to Lem. \ref{lem:hoeldererror_metric}). However, as it is often customary to define Lipschitz and H\"older continuity in a more restricted sense relative to standard norm-induced metrics (in which case the H\"older class is strictly more general than the Lipschitz class) we chose to refer to H\"older continuity rather than Lipschitz continuity to highlight that such H\"older functions can be learned as well.
  
  As insinuated by our notation, we consider parameter $L(n)$ to be adaptive, i.e.  data-dependent while the other parameters are assumed to be set in advance.
 Function $\obserrpar$ can be utilised to accommodate observational noise. 
That is, if the noise level in the data is assumed to be contained in $[-\obserrbnd, \obserrbnd]$ then one would choose $\obserrpar(x) = \obserrbnd,\forall x$.
 % 
In addition,  functions $\lbf,\ubf:  \inspace \to \Real_\infty^m$ are parameters that have to be specified in advance and can impose a priori knowledge of bounds on the target function. For example, if we know the target function to exclusively map to nonnegative values, then one can set $\lbf(x) = 0,\forall x$.
To disable restrictions of boundedness, it is allowed to specify the upper and lower bound functions to constants $\infty$ or $-\infty$, respectively.	

  
Function $\predfn$ is the predictor that is to be utilised for predicting/inferring function values at unseen inputs. Function $\prederrn(x;\param(n))$ is meant to quantify the uncertainty of prediction $\predfn(x;\param(n))$. 
For ease of notation, we shall often omit explicit mention of the parameter, e.g. we may write $\predfn(x)$ instead of $\predfn(x;\param(n))$.
	%Note, if the data set is fixed (i.e. $n$ does not change), we may sometimes drop the subscript $n$ and write $\predf$ instead of $\predf_n$.
\end{defn}


To provide an intuition of the inference rule, consider the following special case where we have access to a noise-free sample $\data_n$ and suppose 
the target $f$ is a real-valued $L^*-\hexp$ H\"older continuous function. Observing the noise-free sample point $(s_i,f_i)$ constrains the set of function values $f(x)$ to the set $\mathbb S_i(x) =\{ \phi \in \outspace | \metric_\outspace(\phi, f_i) \leq L^* \metric(s_i,x) \}$. Considering a set of sample points $\data_n$, target value $f(x)$ is constrained to lie in the intersection $\mathbb S(x)=\cap_{i=1}^{N_n} \mathbb S_i(x)$. It is easy to see that the floor and ceiling functions are tight lower and upper bounds of $\mathbb S(x)$ with $\mathbb S(x) := \{ \phi \in \outspace | \boden_n(x; L^*) \leq \phi \leq \decke_n(x;L^*)\}$. In other words, setting parameter $L(n)$ to the best H\"older constant $L^*$ and bounds $\lbf =-\infty,\ubf=+\infty$ yields a predictor $\predfn(x)$ that for every query $x$ chooses the mid-point of the set $\mathbb S(x)$ of those function values that can possibly be assumed by a H\"older continuous function that interpolates the observed sample. Prediction error $\prederrn(x)$ simply is the radius of the set.

For the case of $\hexp=1$, this approach is known as \emph{Lipschitz interpolation} \cite{Beliakov2006,Zabinsky2003}. Since a set is utilised for interpolation, the approach is also known as \emph{Nonlinear Set Interpolation} \cite{Milanese2004,Canale2014} in control. Specification of $\ubf,\lbf$ allows us to incorporate additional knowledge and constrain our set $\mathbb S(x)$ further. 
For instance, when estimating densities we might incorporate the knowledge of dealing with nonnegative functions. In this case, it makes sense to set $\lbf$ to a constant value of zero yielding $\mathbb S(x) = \{\phi | \phi \geq 0 \} \cap_{i=1}^{N_n} \mathbb S_i(x) $.

%\begin{figure*}
        %\centering
				  %\subfigure[Initial prediction $\predf_1$.]{
    %%\includegraphics[width = 3.7cm, height = 3cm]{content/figures/graph1_klein.eps}
    %\includegraphics[width = 5cm]
								%%{content/Ch_kinkyinf/figs/resultswingrock_555trials}
								%{content/Ch_kinkyinf/figs/akt2learneddrift}
    %\label{fig:KIcollavoidopenlooppredmodels1}
  %} 	 
					  %\subfigure[Improved prediction $\predf_2$.]{
    %%\includegraphics[width = 3.7cm, height = 3cm]{content/figures/graph1_klein.eps}
    %\includegraphics[width = 5cm]
								%%{content/Ch_kinkyinf/figs/resultswingrock_555trials}
								%{content/Ch_kinkyinf/figs/akt3learneddrift}
    %\label{fig:KIcollavoidopenlooppredmodels2}
  %} 	
							  %\subfigure[Social cost after each experiment.]{
    %%\includegraphics[width = 3.7cm, height = 3cm]{content/figures/graph1_klein.eps}
    %\includegraphics[width = 5cm, clip, trim = 3.1cm 9.5cm 4.5cm 9cm]
								%%{content/Ch_kinkyinf/figs/resultswingrock_555trials}
								%{content/Ch_kinkyinf/figs/soccost}
    %\label{fig:KIsoccost}
  %} 
	%\label{fig:KIcollavoidopenlooppredmodels}
   %\caption{Belief over drift models based on $\data_1$ and $\data_2 \supset \data_1$ with $\abs{\data_1} = 2, \abs{\data_2} = 100$. Here, the top figures show the predictions of $\predf_{n,3}(s)$ and the bottom plots depict the predictions of $\predf_{n,4}(s)$ for $n =1$ and $n=2$, respectively. The ground-truth drift model was $f(x) = \bigl(0,0, - \sin(0.5 \, x_1), - \sin( 0.5 \, x_2) \bigr)^\top$. Rightmost plot: Social cost after each experiment. Note how the reduced uncertainty translated to reduced social cost of the last experiment (bar plot 3) vs the cost in the second (bar plot 2). While the first experiment (bar plot 1) also accumulated low social cost, a collision occurred.}
%\end{figure*}	 
%
%\begin{figure}
%        \centering
%				  \subfigure[KI prediction with two examples.]{
%    %\includegraphics[width = 3.7cm, height = 3cm]{content/figures/graph1_klein.eps}
%    \includegraphics[width = 4.1cm]
%								%{content/Ch_kinkyinf/figs/resultswingrock_555trials}
%								{content/Ch_kinkyinf/figs/ex1hfe1.pdf}
%    \label{fig:hfe1}
%  } 	
%	 \subfigure[Refined prediction.]{
%    %\includegraphics[width = 3.7cm, height = 3cm]{content/figures/graph1_klein.eps}
%    \includegraphics[width = 4.1cm]
%								%{content/Ch_kinkyinf/figs/resultswingrock_555trials}
%								{content/Ch_kinkyinf/figs/ex1hfe2.pdf}
%    \label{fig:hfe2}
%  } 	%\hspace{2cm}
%	  \subfigure[Using a periodic pseudo-metric.]{
%    %\includegraphics[width = 3.7cm, height = 3cm]{content/figures/graph1_klein.eps}
%    \includegraphics[width = 4.1cm]
%								%{content/Ch_kinkyinf/figs/resultswingrock_555trials}
%								{content/Ch_kinkyinf/figs/ex1hfeperiodic.pdf}
%    \label{fig:hfe3}
%  } 	
%   \caption{Examples of different target functions and full kinky inference predictions. \textbf{Fig. \ref{fig:hfe1} and  Fig. \ref{fig:hfe2}} show kinky inference performed for two and ten noisy sample inputs of target function $x\mapsto \sqrt{\abs{\sin(x)}}$ with observational noise level $\varepsilon =0.3$. The floor function folds in knowledge of the non-negativity of the target function ($\lbf = 0$). The predictions were made based on the assumption of $p=\frac 1 2$ and $L = 1$ and employed the standard metric $\metric(x,x') = \abs{x-x'}$. 
%	As can be seen from the plots, the function is well interpolated between sample points with small perturbations. Furthermore, the target is within the bounds predicted by the ceiling and floor functions.
%		\textbf{Fig. \ref{fig:hfe3}} depicts a prediction folding in knowledge about the periodicity of the target function by choosing a pseudo-metric $\metric = \Bigl|\sin(\pi \abs{x-x'}\Bigr|$. The prediction performs well even in unexplored parts of input space taking advantage of the periodicity of the target. This time, there was no observational error and the target function was $x \mapsto \abs{\sin(\pi x)} +1$.   }
%			\label{fig:exHfe1}
%\end{figure}	  

%
%\begin{figure}
%        \centering
%							  %\subfigure[Wingrock dynamics.]{
%    %%\includegraphics[width = 3.7cm, height = 3cm]{content/figures/graph1_klein.eps}
%    %\includegraphics[width = 5.5cm]
%								%%{content/Ch_kinkyinf/figs/resultswingrock_555trials}
%								%{content/Ch_kinkyinf/figs/hfewingrock_dyn_groundtruth}
%    %\label{fig:KIvs2NN}
%  %} 	
%				  \subfigure[Wingrock dynamics.]{
%    %\includegraphics[width = 3.7cm, height = 3cm]{content/figures/graph1_klein.eps}
%    \includegraphics[width = 5.5cm]
%								%{content/Ch_kinkyinf/figs/resultswingrock_555trials}
%								{content/Ch_kinkyinf/figs/hfewingrock_dyn_groundtruth}
%    \label{fig:hfe5}
%  } 	
%	 \subfigure[Inferred model.]{
%    %\includegraphics[width = 3.7cm, height = 3cm]{content/figures/graph1_klein.eps}
%    \includegraphics[width = 5.5cm]
%								%{content/Ch_kinkyinf/figs/resultswingrock_555trials}
%								{content/Ch_kinkyinf/figs/hfewingrock_dyn_learned}
%    \label{fig:hfe4}
%  } 	%\hspace{2cm}
%	 
%%
%   \caption{Example of a prediction of the KI rule on two-dimensional input space. \textit{Left plot:} the target function being a patch of the wingrock dynamics we will study in Sec. \ref{sec:KIMRAC}. \textit{Right plot:} the predictions inferred on the basis of 100 sample points using metric $\metric(x,x') = \norm{x-x'}_\infty$ and $p=1, L=2.5$.}
%		\label{fig:exHfe2}
%\end{figure}	 

%Note the inference rule is nonparametric in that, that its complexity grows with the sample size. However, as any inference method one has to a priori specify prior knowledge, some of which is encoded in hyper-parmeters. In the KI rule, these parameters include the choice of pseudo-metric (or its parameters), parameter $\hexp$, the sequence of constants $\seq{L(n)}{n \in \nat}$ as well as the presupposed level of observational error $\obserr$.

%To develop a first feel for this kinky inference rule, we plotted some examples of target functions and pertaining predictions in Fig. \ref{fig:exHfe1}. %and Fig. \ref{fig:exHfe2}.
%Firstly, we can see that the predictions involve kinks. This is due to the minimisations and maximisations that occur in the computation of the ceiling and floor functions which introduce kinks into the prediction signal. This property provided the motivation behind the term ``kinky inference'' \cite{calliess2014_thesis}.

When choosing $L(n)$ to coincide with the best H\"older constant, one can give strong guarantees of convergence to the target as on the tightness of the prediction bounds  \cite{Sukharev1978,calliess2014_thesis} showing that bounds are as tight as possible without imposing additional assumptions and that the predictor minimises the worst-case risk.

Unfortunately, this requires us to know at least an upper bound of $L^*$ and therefore, several authors have proposed different approaches of how to estimate the constant from the data (e.g. \cite{Strongin1973,Wood1996}). However, it appears to be largely unknown how to do so in the presence of bounded observational noise $\obserr >0$ in a principled manner. Furthermore, when we replace $L(n)$ by the empirical estimates, nothing seems to be known about the convergence properties of the resulting kinky inference rule that is based on such  estimates.  

In the remainder of the paper, we shall address this gap. Firstly, we propose an estimator to be utilised in place of $L(n)$ that can be set to be robust to noise (i.e. does not grow unbounded). Referring to the resulting KI rule as LACKI, we then prove universal approximation properties of the LACKI rule before considering its performance in a control application.


\subsection{Lazily Adapted Constant Kinky Inference (LACKI)}
\label{sec:lacki}
Above we explained the benefits of choosing parameter $L(n)$ to coincide with a H\"older constant of the target. 
However, if such a constant is unavailable a priori, we desire to compute $L(n)$ as a data-dependent estimate of the H\"older constant. Our proposal for such an estimator will be introduced next. 

For notational convenience, for two sets $S,S' \subset \inspace$ of inputs we define  $$U(S,S') := \{(s,s') \in S \times S' | \metric(s,s') >0\} \text{ and  let } U_n := U(\grid_n,\grid_n) $$ be the set of all pairs grid inputs deemed disparate under the pseudo-metric $\metric$.

The \emph{best} H\"older constant of a function $f$ is the smallest nonnegative number $L^*$ such that $f$ is contained in the set 
$\hoelset {L^*}{}{p} = \{\phi: \inspace \to \outspace \, \vert \, \forall x,x' \in \inspace : \metric_\outspace \bigl(\phi(x),\phi(x')\bigr) \leq L^* \, \bigl( \metric (x,x') \bigr)^\hexp\} $ of $L^*-\hexp$- H\"older continuous functions. So, this best H\"older constant is given by  
%
$$L^* = \sup_{(x,x') \in U(\inspace,\inspace)} \frac{\metric_\outspace \bigl(f(x) - f(x')\bigr)}{\metric^\hexp(x,x') }.$$




Given the noisy data $\data_n = \{(s_i,\tilde f_i) | i=1,\ldots,N_n\}$ a natural estimate of the best H\"older constant might be to compute $\hat \ell_n^* := \max_{(s,s') \in U_n} \frac{\metric_\outspace(\tilde f(s),\tilde f(s')) }{\metric^\hexp(s,s')}$ \cite{Strongin1973}. In the absence of \emph{noise} (may it be stochastic or deterministic), that is, if $\tilde f_i = f(s_i),\forall i$, $\hat \ell_n^*$ never overestimates the true best H\"older constant. That is, $\hat \ell_n^* \leq L^*$. However, in the presence of noise $\noise: \inspace \to \outspace$ (such that $\tilde f = f+ \noise$) this boundedness assumption of the estimates no longer holds true. In particular if the noise is not H\"older continuous, we expect $\hat \ell_n^*$ to grow unbounded with increasingly dense data.
For practical reasons and for the sake of our theoretical arguments presented below, we desire the parameters $L(n)$ to remain bounded. Thus, without further modifications $\hat \ell_n^*$ is not a suitable candidate for $L(n)$.

To ensure bounded estimates even in the presence of noise, we propose the following estimator : %
\begin{equation}\label{eq:lazyconstupdaterule_batch_main}
\ell(\data_n;\hestthresh,\underline L)  := 
 \max \Bigl\{ \underline L, \max_{(s,s') \in U_n} \frac{\metric_\outspace(\tilde f(s),\tilde f(s')) - \hestthresh}{\metric^\hexp(s,s')} \Bigr\}.
\end{equation}
\begin{figure}
        \centering
    \includegraphics[width = 8.8cm,height = 4cm]
								{content/figs/LACKIhestthreshvar}
   \caption{Two LACKI inferences over function values of the target $f: x \mapsto \abs{\cos(2\pi x)}+x$ (dashed line) on the basis of a noisy sample (plotted as dots). The predictors are plotted in grey, the noisy observational function $\tilde f (\cdot) = f(x) + \nu_x$ is plotted in cyan. Here the $\nu_x$ were drawn i.i.d. at random from a uniform distribution on the interval $[-0.5,0.5]$. In both cases we chose the parameters $\ubf \equiv \infty,\lbf \equiv -\infty,\underline L = 0$ and $\hexp=1$.
   The left plot shows the LACKI predictor $\predfn$ (grey line) for parameter choice $\hestthresh =0$, falsely assuming absence of observational noise. As a result, the prediction overfitted to the noise. The right plot depicts the prediction $\predfn$ (grey curve) for correct parameter choice $\hestthresh = 2 \obserrbnd =1$, causing the noise to be smoothed out and resulting in more accurate prediction of the underlying ground truth $f$.  }
			\label{fig:LACKInoise}
\end{figure}

%
Here $\underline L$ is a parameter that can be used to specify a priori knowledge of a lower bound on the best Lipschitz constant. In the absence of particular domain-specific knowledge, one can of course always set $\underline L =0$.
%
\begin{rem} \label{rem:bndedlipconstestimates}
By setting parameter $\hestthresh$ at least as large as twice the maximum level of observational noise, i.e. $\hestthresh = 2 \obserrbnd + q$ for some $q\geq0$, it is easy to see that the $\ell(\data_n;\hestthresh,0)$ are bounded from above by $\bar L  =\sup_{x,x', \metric(x,x') >0} \frac{\metric_\outspace(f(x), f(x'))   -q}{\metric^\hexp(x,x')} \leq L^* $ ( and,  $L^* < \infty$ if the target is H\"older continuous). 
\end{rem}
%
Next, consider an online learning situation where the available data increases over time. 
That is, $\data_n \subseteq \data_{n+1}$ for all time steps $n \in \nat$. 
For time step $n \in \nat$, let $S_{n+1} := G_{n+1} \backslash \grid_n$ be the set of new sample inputs.
We can define an incremental update rule recursively as follows: 
\begin{align} \label{eq:Hoelconstlazyupdateincr_main}
\ell_{n+1} := \max\Bigl\{ & \ell_n, \max_{(s,s') \in U(\grid_n, S_{n+1})} \frac{\metric_\outspace\bigl(\tilde f(s),\tilde f(s')\bigr) - \hestthresh}{\metric^\hexp(s,s')},\\
&\max_{(s,s') \in U(S_{n+1}, S_{n+1})} \frac{\metric_\outspace\bigl(\tilde f(s),\tilde f(s')\bigr) - \hestthresh}{\metric^\hexp(s,s')} \Bigr\},
\end{align} for $n \in \nat$ 
and where 
$\ell_0 := \underline L$. 
The effort of computing $\ell_{n+1}$ is in the order of $\mathcal O\bigl(M (\abs{S_{n+1}} N_n+ \abs{S_{n+1}}^2)\bigr)$ where $M$ denotes the effort for evaluating the pseudo-metrics.
By construction, we have $\ell_n =\ell(\mathcal \data_n;\hestthresh,\underline L), \forall n \in \nat.$ 

\begin{rem}\label{rem:convlipconstestimates}
Remember that $\seq{\ell_n}{n\in \nat}$ is bounded. Since it is also growing monotonically we can appeal to the monotone convergence theorem to show that the sequence is convergent to some number $\bar L \leq \max\{L^*,\underline L\}$.
\end{rem}

So far, we have defined a rule of how to update noise-robust and convergent estimates $\ell_n$ of the H\"older constant. Using these data-dependent estimates in place of $L(n)$ in our kinky inference framework as per Def. \ref{def:KIL} yields an inference rule that shall henceforth be referred to as \textit{Lazily Adapted Constant Kinky Inference (LACKI)}.

\begin{defn}[LACKI rule] \label{def:LACKI} For each output component $j \in \{1,\ldots,\text{dim} \, \outspace \}$ 
define $\predfn(\cdot)_{j}$  as per Def. \ref{def:KIL} but assume we choose the parameters $L(n) := \ell(\data_n;\hestthresh,\underline L)$ (according to Eq. \ref{eq:lazyconstupdaterule_batch_main}). We refer to the resulting predictor $\predfn\bigl(\cdot \bigr)$ as a \emph{Lazily Adapted Constant Kinky Inference (LACKI)} rule. Here, the free parameters are $\hexp \in (0,1], \hestthresh \in \Real_{\geq 0}$ and $\underline L \in \Real_{\geq 0}$.  
\end{defn}

To develop a first feel for our inference rule, refer to Fig. \ref{fig:LACKInoise}. Here, we depicted the predictors for an underlying ground-truth function on the basis of a sample but with different parameter choices $\hestthresh$. When setting this parameter to two times the observational noise level, the predictor accurately smoothes out the noise. In contrast, when the parameter is set to zero, the resulting predictor will perfectly interpolate through the noisy observations, thereby limiting the approximation quality to the level of observational noise. 

Furthermore, we notice that the predictors are H\"older continuous but non-differentiable. Informally speaking, the inference exhibits ``kinks'', motivating the term ``kinky inference''.

Finally note, the estimator $\ell_n$ determining  parameter $L(n)$ is ``lazy'' in the sense that it only increases the estimate of the H\"older constant just enough to be consistent with the observed data. That is, it chooses $L(n)$ to coincide with the smallest H\"older constant of a conceivable target function $f$ that could have generated the data under the given noise assumption. Below, we will see that the predictor $\predfn$ has H\"older constant $L(n)$. Therefore, the ``laziness'' of the estimator of $L(n)$ implements \emph{Occam's razor}: it \emph{regularises} the hypothesis space of continuous functions to prefer simple explanations of the data (i.e. functions with low H\"older constants) over complex ones (i.e. functions with higher H\"older constants). Here, $\hestthresh$ can serve as a parameter that can be utilised to regularise the predictor further in order to compensate for (bounded) noise in the data. 



\begin{figure}
        \centering
    \includegraphics[width = 8.8cm,height = 4cm]
								{content/figs/LACKIhestthreshvar_with_bounds}
   \caption{Repetition of the experiment but with $p=0.5$. This time, we also plotted floor and ceiling functions (grey dotted and dashed-dotted curves) delimiting the uncertainty bounds. Note, when setting $\lambda =0$ the estimate $\ell(\mathcal \data_n;\hestthresh,\underline L)$ was found to be 128 resulting in extremely conservative uncertainty estimates (left figure). In contrast, choosing $\lambda =2 \obserrbnd $ gave a parameter estimate $\ell(\mathcal \data_n;\hestthresh,\underline L) = 2.4$ yielding sensible uncertainty bounds (right figure). }
			\label{fig:LACKInoise2}
\end{figure}
%
%
%\begin{figure}
%        \centering
%				  \subfigure[ ]{
%    %\includegraphics[width = 3.7cm, height = 3cm]{content/figures/graph1_klein.eps}
%    \includegraphics[width = 4.8cm]
%								%{content/Ch_kinkyinf/figs/resultswingrock_555trials}
%								{content/figs/LACKInoise_e_alpha_trueLinit_7tex.pdf}
%    \label{fig:LACKInoise1}
%  } 	
%	 \subfigure[ ]{
%    %\includegraphics[width = 3.7cm, height = 3cm]{content/figures/graph1_klein.eps}
%    \includegraphics[width = 4.8cm]
%								%{content/Ch_kinkyinf/figs/resultswingrock_555trials}
%								{content/figs/LACKInoise_noe_noalpha_smallLinit_7tex.pdf}
%    \label{fig:LACKInoise2}
%  } 	
%  %
%  %\hspace{2cm}
%	  \subfigure[ ]{
%    %\includegraphics[width = 3.7cm, height = 3cm]{content/figures/graph1_klein.eps}
%    \includegraphics[width = 4.8cm]
%								%{content/Ch_kinkyinf/figs/resultswingrock_555trials}
%								{content/figs/LACKInoise_e_alpha_smallLinit_7tex.pdf}
%    \label{fig:LACKInoise3}
%  } 
%  				  \subfigure[ ]{
%    %\includegraphics[width = 3.7cm, height = 3cm]{content/figures/graph1_klein.eps}
%    \includegraphics[width = 4.8cm]
%								%{content/Ch_kinkyinf/figs/resultswingrock_555trials}
%								{content/figs/LACKInoise_e_alpha_trueLinit_77tex.pdf}
%    \label{fig:LACKInoise4}
%  } 	
%	 \subfigure[ ]{
%    %\includegraphics[width = 3.7cm, height = 3cm]{content/figures/graph1_klein.eps}
%    \includegraphics[width = 4.8cm]
%								%{content/Ch_kinkyinf/figs/resultswingrock_555trials}
%								{content/figs/LACKInoise_noe_noalpha_smallLinit_77tex.pdf}
%    \label{fig:LACKInoise5}
%  } 	
%  %
%  %\hspace{2cm}
%	  \subfigure[ ]{
%    %\includegraphics[width = 3.7cm, height = 3cm]{content/figures/graph1_klein.eps}
%    \includegraphics[width = 4.8cm]
%								%{content/Ch_kinkyinf/figs/resultswingrock_555trials}
%								{content/figs/LACKInoise_e_alpha_smallLinit_77tex.pdf}
%    \label{fig:LACKInoise6}
%  } 	
%   \caption{LACKI inference on the basis of two noisy data sets 
%   $\data_1,\data_2$ with varying parameter settings. Target (blue curve) was $f:x \mapsto 5 \sqrt{\abs x}$ and observational model (magenta curve) $\tilde f: x \mapsto f(x) + \phi_x$ where $\phi_x$ is a uniformly distributed disturbance with $\phi_x \stackrel{i.i.d.}{\sim} \mathcal U(-0.5,0.5),\forall x$. 
%  %
%   \textbf{1st column:} Initial constant L(0) = 5; assumed obs. error bound $\obserrpar= 0.5$; $\hestthresh = 2 \obserr=1$. 
%   %
%   \textbf{2nd column:} Initial constant L(0) = 0.01; assumed obs. error bound $\obserrpar = 0$; $\hestthresh = 2 \obserr =0$.
%   %
%   \textbf{3rd column:} Initial constant L(0) = 0.01; assumed obs. error bound $\obserrpar = 0.5$; $\hestthresh = 2 \obserr =1$.
%   %
%   \textbf{Top row:} Data set $\data_1$ of size 7. 
%   \textbf{Bottom row:} Data set $\data_2$ of size 77.
%   In all cases the exponent parameter was set to $p = 0.5$.}
%			\label{fig:LACKInoise}
%\end{figure}
%
%The figure depicts the inference $\predfn$ (black curve)	 over function values of the target $f$ (blue curve) based on noisy data sets $\data_1$ and $\data_2$. The true observational noise was bounded by level $\obserrbnd = 0.5$ yielding a data-generating function $\tilde f(x) = f(x) +\nu(x)$ (magenta curve) where $\nu(x) \in [-0.5,0.5],\forall x$.
%The first row shows the predictions of our LACKI rule applied to a small sample $\data_1$ with varying parameter settings.
%The second row depicts the pertaining predictions resulting applying the corresponding LACKI rules to an enlarged data set $\data_2 \supset \data_1$.
%
%In plots in the first column and third columns depict situations where LACKI was applied with parameter $\hestthresh=2 \obserrbnd$.  By contrast, the second column depicts a situation where LACKI was parameterised with a parameter $\hestthresh = 0$ and a small initial guess of the H\"older constant of $\underline L = L(0) = 0.01$. As a result, in this case, the LACKI predictor perfectly follows the noisy data. Furthermore, we observe a large uncertainty bound $\prederrn$ with increasing sample size. In the third column the LACKI rule was also initialised with $\underline L =0.01$. However, the larger $\hestthresh = 2 \obserrbnd$ prevents 
%the H\"older constant estimate $L(2)$ to exceed the true constant $L^*=5$. As a result, the prediction ``smoothes out the noise''. The degree of smoothing and approximation error in the limit of infinite data will be theoretically investigated below. 
%At this point we will already mention that all behaviours depicted in the plots seem consistent with our theoretical guarantees we will give in what is to follow.


\subsection{Properties}
\label{sec:properties_lacki}
We will now establish several properties of the LACKI rules including boundedness of the predictors, sample-consistency and H\"older continuity. Most importantly however, we will show that the LACKI rules are universal approximators, in the sense that they can be set to learn any continuous function with arbitrary worst-case error.

The core idea behind this can be sketched as follows: 
First, we establish H\"older continuity and sample-consistency. This allows us to prove that LACKI can learn any H\"older function.
Note, some universal approximators, such as RBFNs with Gaussian kernels, are provably Lipschitz. Therefore, learning any continuous function can be interpreted as learning some Gaussian RBFN with an observational error level that absorbs the discrepancy between the RBFN and the ground truth. Since a finite RBFN with smooth, bounded-derivative kernel is provably H\"older and since we can learn any H\"older function with LACKI up to the level of observational error, we can learn the continuous ground-truth up to the approximation error of the RBFN. 

Following this outline, we will now proceed to establish the desired properties formally.

\begin{lem}[Boundedness of the predictor]
Irrespective of the boundedness of input space $\inspace$ and assuming finite sample size  $N_n=\abs{\data_n} < \infty$, the predictor $\predfn:\inspace \to \outspace$ is bounded. In particular,
for $\hexp =1$, we have  
$\forall x \in \inspace: \norm{\predfn(x)}_\infty \leq \max_{i=1,...,N_n} \norm{\tilde f_i}_\infty + \frac{L(n)}{2}  \max_{i,j = 1,...,N_n} \metric^\alpha(s_i,s_j) <\infty$.
\begin{proof}
Let $D = \max_{i,j = 1,...,N_n} \metric^\hexp(s_i,s_j)$ and for the $k$th output dimension let  $F_k= \max_{i=1,...,N_n} \abs{\tilde f_{i,k}}$. As shown in Sec. \ref{sec:Hoelder_brief}, $\metric^\hexp$ is a pseudo-metric too and hence, adheres to the triangle inequality. Utilising the definition of the predictor and the triangle inequality we see that, for any  $x \in \inspace$ and any output dimension $k$, there are some $i,j \in \{1,...,N_n\}$ such that we have: 
$\predf_{n,k}(x) =\frac{\tilde f_{j,k} + \tilde f_{i,k}}{2} + \frac{L(n)}{2} \bigl( \metric^\hexp(x,s_i) - \metric^\hexp(x,s_j) \bigr) \leq \frac{\tilde f_{j,k} + \tilde f_{i,k}}{2} +  \frac{L(n)}{2} \metric^\hexp(s_j,s_i) \leq F_k + \frac{L(n)}{2}  D < \infty$.
% We can represent $\predfn$ as a function of the form 
%$g(h_1,...,h_{N_n})$ with $h_i(x) = \metric(x,s_i)$. Now, assume $\hexp \neq 1$. Defining $p: x \mapsto = x^\alpha$. Since 
\end{proof}
\label{lem:LACKIpredbounded}
\end{lem}

As promised, we establish that the predictors of the LACKI inference rule are H\"older continuous:
\begin{lem}[H\"older regularity of LACKI]
With definitions as before, let $(\outspace,\metric_\outspace) = (\Real^m,(x,y) \mapsto \norm{x-y}_\infty)$.
Provided that the bounding functions $\lbf,\ubf$ are H\"older continuous (or set to $-\infty,\infty$, respectively),
the predictors $\predfn$ are H\"older continuous $(n \in \nat)$ with constant $L(n)$ and exponent $\hexp$. That is, $\forall n \in \nat: \predfn \in \Hoelset \bigl(L(n),\hexp\bigr)$.

\begin{proof}
It is easy to show that the one-dimensional mappings of the form $x \mapsto \ell \Metrici{x}{x'}^\hexp$ are $\ell-\hexp-$ H\"older continuous for any choices of $\ell,\hexp$ and inputs $x'$. Furthermore, taking point-wise $\max$, $\min$ as well as averages of H\"older continuous functions is known to not change their H\"older properties (e.g. cf. \cite{calliess2014_thesis}). Therefore, the output-component predictors  $\predf_{n,j}$ $(j=1,...,m)$ are $L(n)$-$\hexp$- H\"older. 
Hence, $\forall x,x': \Metrico{\predfn(x)}{\predfn(x')} = \norm{\predfn(x) - \predfn(x')}_\infty = \max_{j=1,...,m} \abs{\predf_{n,j}(x) - \predf_{n,j}(x')} \leq \max_{j=1,...,m} L(n) \metric(x,x')^\hexp = L(n) \metric(x,x')^\hexp$.
\end{proof} 
\label{lem:LACKIpredHoelder}
\end{lem} 

We now establish how well our LACKI rule can interpolate the training data as function of the noise bound and regularisation parameter $\hestthresh$: 
\begin{lem}[Sample-consistency of LACKI] \label{lem:LACKIsampleconsistency} If for each output dimension $j \in \{1,...,d\}$ and some $\hestthresh \geq 0$ we have $L(n)  \geq  \max_{(s,s') \in U_n} \frac{\abs{\tilde f_j(s)- \tilde f_j(s')} -\hestthresh }{\metric^\hexp(s,s')}
$ then the LACKI rule is sample-consistent (up to $\frac \hestthresh 2$). 
That is,  \[\forall q \in \{1,\ldots,N_n \}:  \predfn(s_q) \in \ball{\frac \hestthresh 2}{\tilde f_q} \] where $\ball{\frac \hestthresh 2}{\tilde f_q} =\{ x \in \outspace |  \norm{x-\tilde f_q}_\infty \leq \frac \hestthresh 2\}$ denotes the $\frac \hestthresh 2$-ball around the observation $\tilde f_q$.\\
Thus, we also have $\norm{ f(s_q) - \predfn(s_q)}_\infty \leq \frac \hestthresh 2 + \norm{\obserr(s_q)}_\infty \leq \frac \hestthresh 2 + \obserrbnd$.
\begin{proof}
Remember, our output-space metric is given by $\Metrico{y}{y'} = \norm{y-y'}_\infty$.
For ease of notation, we will confine our proof to the case of one-dimensional outputs ($d=1$). The multi-dimensional case follows trivially from the one-dimensional result by applying it to each output component function. 
Let $n \in \nat  $ be fixed and, for ease of notation, write $L := L(n)$. Let $j,k \in   \{1,\ldots,N_n \} $ such that $j \in \argmin_i \tilde f_i + L \metric^\hexp(s_i,s_q) $ and 
$k \in \argmax_i \tilde f_i - L \metric^\hexp(s_i,s_q) $. 
By definition of $\predfn$ we have:
\begin{align}
\predfn(s_q) 
%&= \frac 1 2 \bigl( \tilde f_j + L \metric^\hexp(s_j,s_q)  +\obserrpar(s_q) \bigr) + \frac 1 2 \bigl( \tilde  f_k - L \metric^\hexp(s_k,s_q) - \obserrpar(s_q) \bigr)\\
&= \frac 1 2 \bigl(\underbrace{ \tilde f_j + L \metric^\hexp(s_j,s_q)}_{:=B}   \bigr) + \frac 1 2 \bigl( \underbrace{\tilde  f_k - L \metric^\hexp(s_k,s_q) }_{=:A}\bigr). \label{eq:hru582jsokbbbn}
\end{align}
%
%
(i) Firstly, we show  \underline{$ A \in [\tilde f_q, \tilde f_q +\hestthresh]$}:
If $k = q$, this holds trivially true since then $A= \tilde f_q$. 
So, assume $k \neq q$. 
We have $\tilde f_k \geq \tilde f_k - L \metric^\hexp(s_k,s_q)  \geq \tilde f_q - L \metric^\hexp(s_q,s_q)  = \tilde f_q$ where the second inequality holds due to $k \in \argmax_i \tilde f_i - L \metric^\hexp(s_i,s_q) $. That is,
%\begin{equation} 
%{\tilde f_k - \tilde f_q}  \geq L \metric(s_k,s_q). \label{eq:9p4t37ru8ewihk3hjtu}
%\end{equation} 
\begin{equation} 
A=\tilde f_k - L \metric^\hexp(s_k,s_q) \geq \tilde f_q. \label{eq:9p4t37ru8ewihk3hjtu}
\end{equation} 
On the other hand, since $L  \geq  \max_{(s,s') \in U_n} \frac{\abs{\tilde f(s)-\tilde f(s')} - \hestthresh}{\metric^\hexp(s,s')}$  we have in particular: $L  \geq  \frac{ \abs{\tilde f_k- \tilde f_q}-\hestthresh }{\metric^\hexp(s_k,s_q)}$. Thus,  
 $L \metric^\hexp(s_k,s_q) + \hestthresh \geq \abs{\tilde f_k-\tilde f_q} = \tilde f_k - \tilde f_q$. Hence, $\tilde f_q + \hestthresh \geq \tilde f_k - L \metric^\hexp(s_k,s_q) = A$. 
 Together with (\ref{eq:9p4t37ru8ewihk3hjtu}) we have shown $A \in [\tilde f_q , \tilde f_q +\hestthresh].$
%  In conjunction with (\ref{eq:9p4t37ru8ewihk3hjtu}) this means that $L \metric(s_k,s_q)  = \abs{\tilde f_k-\tilde f_q} $. Substituting this into the left hand side of Eq. \ref{eq:irndhfkkdnsww} yields the expression 
% $ \tilde f_k - \abs{\tilde  f_k -\tilde  f_q } \stackrel{!}{=}   \tilde f_q $. But since $f_k \geq f_q$, we also have $f_k - \abs{ f_k - f_q }  = f_q$. Therefore, Eq. \ref{eq:irndhfkkdnsww} is equivalent to $f_q = f_q$ which obviously holds. 
% 
%Analogously to the proof of  $f_k - L \metric(s_k,s_q) = f_q$ we can also show $f_j + L \metric(s_j,s_q) = f_q$. Leveraging both equations in Eq. \ref{eq:hru582jsokbbbn} concludes the proof. 

(ii) The proof of \underline{$B \in [ \tilde f_q - \hestthresh, \tilde f_q]$} is completely analogous to that of (i) and hence, is omitted.
%and secondly, that 
%\begin{equation}B \in [ \tilde f_q - \hestthresh, \tilde f_q]. \label{eq:irndhfkkdnswwB}
%\end{equation}

(iii) Together, the statements in (i) and (ii)  entail  $\predfn(s_q) = \frac 1 2 A + \frac 1 2 B  \in [\tilde f_q - \frac \hestthresh 2, \tilde f_q + \frac \hestthresh 2]$.

Hence, $\Metrico{\predfn(s_q)}{\tilde f(s_q)} \leq \frac \hestthresh 2$.

Moreover, for any sample input $s_q$ we have $\predfn(s_q) = f(s_q) + \phi_q + \psi_q$ with $\metric_{\outspace}(0,\psi_q) \leq \frac \hestthresh 2,  \metric_{\outspace}(0,\phi_q) \leq \Metrico 0 {\obserr(s_q)} \leq \obserrbnd$. 
Our output-space metric is translation-invariant and hence, $\Metrico{f(s_q)}{\predfn(s_q)} = \Metrico{0}{\predfn(s_q)- f(s_q)} = \Metrico{0}{  \phi_q + \psi_q} \leq \frac \hestthresh 2 +\Metrico 0 {\obserr(s_q)} \leq \frac \hestthresh 2 + \obserrbnd$.
\end{proof}
\end{lem}
%
%\begin{lem}\label{lem:groups_sampleconsandobserr} As always, let $\obserr(\cdot) $ denote the bound on the observational error.
%If $\predf_n$ is sample-consistent up to error $E$ with respect to set data $\data_n$ then, for any sample input $s \in \grid_n$, we have $\metric_\outspace\bigl(f(s), \predfn(s) \bigr) \leq E(s) + \metric_\outspace(0,\obserr(s))$.  Given an upper bound  $\obserrbnd \in \Real$ on the observational error  such that $\obserrbnd \geq \obserr(x),\forall x$ then sample-consistency up to error $E$ entails that true function is contained in the $E(s) + \obserrbnd$-balls around the predictions of observed inputs. That is, for all $s \in \grid_n$ we have:
%\begin{equation}
% f(s) \in \ball{ E(s) + \obserrbnd}{\predfn(s)} = \{y \in \outspace | \metric_\outspace(\predfn(s),y ) \leq E(s) + \obserrbnd \}.
%\end{equation}
%\end{lem}
%\begin{proof}
%Sample consistency up to $E$ implies that for each sample input $s \in \grid_n$ $\Metrico{\predfn}{\tilde f(s)} \leq E(s)$.
%On the other hand, bounded observational errors mean that for $s$ 
%there is $\nu_s \in \outspace, \Metrico 0 {\nu_s} \leq \obserrbnd$ such that we have $\Metrico{f(s)}{ \tilde f(s) } = \Metrico{f(s)}{ f(s)+\nu_s } \leq \Metrico{0}{\nu_s } \leq \obserrbnd$.
%Since the output pseudo-metric $\Metrico \cdot \cdot$ adheres to the triangle inequality, we have 
%$\Metrico{f(s)}{\predfn(s)} \leq \Metrico{f(s)}{ \tilde f(s) } + \Metrico{\predfn(s)}{ \tilde f(s) } \leq E(s) + \obserrbnd $.
%\end{proof}




\subsubsection{Prediction error bounds and consistency}

To asses our learning rule, we might be interested the discrepancy $\metric_\fctspace(\predfn,f)$ between the predictor $\predfn$ and the target function $f$ relative to some metric $\metric_{\fctspace}$ between functions in the space $\fctspace$ of continuous functions. In statistics, a typical choice is the mean-square error metric assessed with respect to some distribution over inputs, the function space and the noise. However, in many safety-critical applications, often arising in control, worst-case error considerations are of greater value, leading to a worst-case metric 
$\metric_\fctspace(f,g) = \sup_{x \in I} \Metrico{ f(x)}{g(x)} $ for some subset $I \subseteq \inspace$ of queries on finds interesting to take into consideration. 

Therefore, we will now establish worst-case consistency guarantees of our LACKI inference rules. That is, we shall study the worst-case error  
sequence $\errmetric^\infty :=\seq{\errmetric^\infty_n}{n \in \nat}$, 
\begin{equation}
\errmetric^\infty_n := \sup_{x \in \queryset } \metric_\outspace\bigl(\predfn(x), f(x)\bigr)\end{equation}
for data $\data_n$ that becomes increasingly dense over time relative to a set of query inputs $\queryset \subseteq \inspace$. 
To clarify the latter concept, 
consider the sequence of grids $\seq{\grid_n}{n \in \nat}$. 
We say this sequence converges to a set that \emph{becomes dense relative to a set $I$ in the limit of large $n$ } if we can use points in the sequence to approximate any points in $I$ with increasing accuracy. That is, if $\forall \epsilon >0,x \in I \exists n_0 \forall n \geq n_0 \exists g \in \grid_n: \metric(x,g) < \epsilon$. If the rate at which this happens is independent of $x$ then we say that the grid sequence becomes dense \emph{uniformly}. This is the case iff $\forall \epsilon >0 \exists n_0 \forall n \geq n_0, x \in I \exists g \in \grid_n: \metric(x,g) < \epsilon$.

To make the rates explicit in our notation, we list the following general definitions: 

%\begin{defn} [Becoming dense]
%Let $\inspace$ be a space endowed with a pseudo-metric $\metric$. 
%\begin{itemize}
%\item 
%We say the sequence of sets $(S_n)$, with $S_n \subset \inspace$, \emph{becomes dense} to $S' \subset \inspace$ \emph{pointwise} (written as $(S_n) \convto S'$), iff $$ \forall s \in S' \exists r_s: \nat \to \Real : \bigl(r_s(n) \stackrel{n \to \infty}{\to} 0 \wedge \forall n \in \nat: \inf_{s_n \in S_n} \metric(s_n,s) \leq r_s(n)\bigr).$$
%\item The sequence becomes dense to $S$ \emph{uniformly} (written as (S_n) \to S') if there is a vanishing rate that is independent of the point $s \in S'$ with which $S_n$ becomes dense to $S'$ at all points. That is, if there exists a uniformly applicable rate $r: \nat \to \Real$ with  $r(n)  \stackrel{n \to \infty}{\to} 0$ and  
%$  \forall n \in \nat : \sup_{s \in S} \inf_{s_n \in S_n} \metric(s_n,s) \leq r(n) . $
%%\item Finally, a sequence of points $\seq{s_n}{n \in \nat}$ in $ \inspace$ is said to converge to the set $S \subset \inspace$ iff $\inf_{s \in S} \metric(s_n,s) \to 0$ as $n \to \infty$.
%\end{itemize}
%\end{defn}

\begin{defn} [Becoming dense, rates, $\stackrel{r}{\convto}, \stackrel{r}{\bd}, \stackrel{r}{\bdu}$]
Let $\inspace$ be a space endowed with a pseudo-metric $\metric$. 
Let $r:\nat \to \Real$ be a ``rate'' function. 
that vanishes, that is, with $\lim_{n \to \infty} r(n) = 0$ (i.e. $r \in o(1)$).
\begin{itemize}
\item The sequence $s =\seq{s_n}{n \in \nat}$ of points in $\inspace$ is said to converge to a point $x \in \inspace$ \emph{with rate} $r$ (denoted by $s \stackrel{r} \convto x$) iff  $\forall n \in \nat : \metric(x,s_n) \leq r(n)$ and $r(n) \convton 0$. 
\begin{itemize}
\item The sequence $s $ is said to converge to a set $\mathbb S \subset \inspace$  with rate $r:\nat \to \Real$ (denoted by $s \stackrel{r} \setconvto \mathbb S$) iff $\forall n\in \nat : \inf_{x \in \mathbb S}\metric(x,s_n) \leq r(n)$ and $r(n) \convton 0$. 
\end{itemize}
\item A sequence of sets $S = \seq{S_n}{n \in \nat}$ is said to \emph{become dense} relative to $x \in \inspace$  \emph{with rate} $r$ (denoted by $S \stackrel{r}{\bd} x$) iff $S$ contains a point sequence that converges to $x$ with that rate. That is, iff $\exists s= \seq{s_n}{n\in \nat}: s \stackrel{r}\convto x \wedge \forall n: s_n \in S_n$. 
\begin{itemize}
\item Similarly, the sequence of sets $S$ is said to become dense relative to a \emph{set}  of points $\mathbb S \subset \inspace $ (denoted by $S \bd \mathbb S$) iff it becomes dense relative to all points of $\mathbb S$, i.e. iff $\forall x \in \mathbb S : S \stackrel{r_x}{\bd} x$ for some vanishing rate $r_x : \nat \to \Real$. 
\item The sequence is becoming dense relative to $\mathbb S$ \emph{uniformly} (denoted by $S \bdu \mathbb S$) iff there is a single vanishing rate for all $x \in \mathbb S$. That is, if $\exists r:\nat \to \Real:$ $\lim_{n\to \infty} r(n) =0 \wedge \sup_{x \in \mathbb S} \inf{s_n \in S_n} \metric(s_n,x) \leq r(n), \forall n$. Function $r$ is referred to as the convergence rate and we write $S \stackrel{r}{\bdu}\mathbb S$ to denote that $S$ becomes dense relative to $\mathbb S$ with uniform rate $r$.
\end{itemize}
\end{itemize}
\end{defn}

\begin{thm}[LACKI can learn any H\"older function] \label{thm:convergenceifboundedconstandsamplecons_LACKI}
Assume the following holds true:
\begin{enumerate}
%\item The grid sequence $\seq{\grid_n}{n \in \nat}$ becomes dense in the input domain $\inspace$. 
\item The observational errors given by $\obserr$ are bounded from above by $\obserrbnd= \sup_x \metric_\outspace\bigl(0,\obserr(x)\bigr) \in \Real_{\geq 0}$.
\item  The target $f: \inspace \to \outspace$ is H\"older continuous, i.e. $\exists L^* \in \Real: f \in \hoelset {L^*}{}  p$.
\end{enumerate}
%
%and let the predictors be sample-consistent up to error $\bar E$ and H\"older continuous, 
%$\predfn \in \hoelset {L(n)}{} p$ with $L(n) \leq \bar L \in \Real_+\, (n \in \nat)$ for some $\bar L \geq 0$.

Under these assumptions we can give the following guarantees: 
%\begin{itemize} \item 

\textbf{(A)} If the grid becomes dense (pointwise), the point-wise worst-case error vanishes up to $\frac \hestthresh 2 + \obserr$:
$$
\text{If }\forall x \in I \subset \inspace \exists r_x \in o(1) : L(\cdot) r_x^\hexp(\cdot) \in o(1) \wedge \seq{G_n}{n \in \nat} \stackrel{r_x}{\bd} x  \text{ then }\forall x \in I : \seq{\Metrico{\predfn(x)}{f(x)}}{n \in \nat} \stackrel{\varrho_x} \convto [0,\frac \hestthresh 2 + \obserr]$$
where for the error convergence rate $\varrho_x$ we have $\varrho_x(n) \leq (L(n)+ L^*) r_x^\hexp(n) , \forall n \in \nat$. 

\textbf{(B)} If the grid becomes dense in $I \subset \inspace$ uniformly, then the worst-case prediction error vanishes uniformly (up to $\frac \lambda 2 + \obserrbnd$):
$$\text{If } \exists r \in o(1) :  L(\cdot) r_x^\hexp(\cdot) \in o(1) \wedge (G_n) \stackrel{r}{\bdu} I  \text{ then }  \errmetric^\infty \stackrel{\varrho}{\convto} [0,\frac \hestthresh 2 + \obserrbnd]$$
where for the uniform error convergence rate $\varrho$ we have $\varrho(n) \leq (L(n)+ L^*) r^\hexp(n) , \forall n \in \nat$. 

%
%
%(A) If, the grid becomes dense in $\inspace$ with rates $r_x$ (for $x \in \inspace$) and if the $L(n)$ satisfy $L(\cdot) r_x(\cdot) \in o(1)$  then the sequence of predictors $\seq \predfn {n \in \nat}$ converges to the ground truth $f$ pointwise up to error $\frac \hestthresh 2 +\obserrbnd$. That is, $\forall \epsilon > 0,  x \in \inspace \exists n_0 \in \nat \forall n \geq n_0: \metric_{\outspace}\bigl(\predfn(x), f(x) \bigr) \in \bigl[0, \epsilon + \frac \hestthresh 2 + \obserrbnd\bigr]$.
%
%In particular,  $\sup_{} \mathcal O\bigl(\,  (L(n)+L^*) r_x(n)^\hexp \, \bigr)$. 


%\textbf{(B)} If the grid sequence $\seq{\grid_n}{n \in \nat }$ becomes dense in $\inspace$ uniformly with rate of at most $n \mapsto r(n) \in o(1)$ and if we have $L(\cdot) r(\cdot) \in o(1)$ then $\seq \predfn {n \in \nat}$ converges to the target $f$ uniformly up to error $ \frac \hestthresh 2 +\obserrbnd$. 
%That is, $\forall \epsilon > 0 \exists n_0 \in \nat \forall n \geq n_0, x \in \inspace: \metric_{\outspace}\bigl(\predfn(x), f(x) \bigr) \in \bigl[0, \epsilon + \frac \hestthresh 2 + \obserrbnd\bigr]$. Convergence occurs with a rate of at most $\mathcal O\bigl( \, (L(n)+L^*) r(n)^\hexp \, \bigr)$.


\begin{proof}
We have established that the predictors $\predfn(\cdot)$ of the LACKI rule are $L(n)$-$\hexp$- H\"older (Lem. \ref{lem:LACKIpredHoelder}) and sample-consistent up to level $\frac \hestthresh 2$ (Lem. \ref{lem:LACKIsampleconsistency}). 
%Moreover, the sequence $\seq{L(n)}{n \in \nat}$ is convergent and bounded by some number $\bar L \leq \max\{L^*,\underline L\}$ (cf. Rem. \ref{rem:convlipconstestimates}).
 %Furthermore our input and output spaces are additive abelian groups. By assumption, pseudo-metrics $\metric,\metric_\outspace$ are translation invariant with respect to addition. Therefore, the predictors meet all assumptions of Thm. \ref{thm:convergenceifboundedconstandsamplecons} which entails the desired convergence statements.


For any input $x \in \inspace$ let $\xi_n^x$ denote a nearest neighbour of $x$ in grid $\grid_n$. That is, $\xi_n^x \in \arg\inf_{s \in \grid_n} \metric(x,s)$.
Since $\grid_n$ is assumed to become dense in the input domain $\inspace$, for any input $x$ there is a rate function $r_x : \nat \to \Real_{\geq 0}$ such that $r_x(n) \stackrel{n \to \infty}{\longrightarrow} 0$ and $ \metric(x,\xi_n^x )^\hexp \leq r_x(n) 
, \forall n \in \nat$. In the case of uniform convergence a rate function can be chosen independently of $x$ and will be denoted by $r$ rather than $r_x$.

\textbf{(A)} 
For all $n \in \nat$ and $x \in \inspace$ we have:
%there always is some $\phi(\xi_n^x) \in  \ball{\bar E}{0} = \{ y \in \outspace \,| \, \metric_\outspace(y,0 ) \leq \bar E \}$ such that (i) 
\begin{align} \metric_\outspace\bigl(\predfn(x) , f(\xi_n^x)  \bigr)
&\stackrel{(i)}{\leq} 
\metric_\outspace\bigl(\predfn(x) , \predfn(\xi_n^x)  \bigr) + \metric_\outspace\bigl(\predfn(\xi_n^x) , f(\xi_n^x)  \bigr)\\
&\stackrel{(ii)}{\leq} 
\metric_\outspace\bigl(\predfn(x) , \predfn(\xi_n^x)  \bigr) + \frac \lambda 2 + \metric_\outspace\bigl(0, \obserr(\xi_n^x) \bigr)
\stackrel{ }{=} 
\norm{\predfn(x) - \predfn(\xi_n^x) }_\infty + \frac \lambda 2 + \obserrbnd\\
&\stackrel{(iii) }{\leq} L(n)
\Metrici{x}{\xi_n^x }^\hexp + \frac \lambda 2 + \obserrbnd \label{ineq:kjshkhd8e}
%&\stackrel{(iv) }{\leq} \bar L\,
%r_x(n) + \frac \lambda 2 + \obserrbnd \stackrel{n \to \infty}{\longrightarrow} \frac \lambda 2 + \obserrbnd.
\end{align}
Here, (i) follows from the triangle inequality, (ii) leverages Lem. \ref{lem:LACKIsampleconsistency} and (iii) follows by H\"older continuity of the predictors (Lem. \ref{lem:LACKIpredHoelder}).

%
Thus, for $x \in \inspace, n \in \nat$:
%
\begin{align}
0 \leq \metric_\outspace\bigl(\predfn(x) , f(x)\bigr) &\leq \metric_\outspace\bigl(\predfn(x) ,  f(\xi_n^x)  \bigr) + \metric_\outspace\bigl(f(\xi_n^x) , f(x)\bigr)\\
%$\norm{\predfn(x) - f(x)} \leq \norm{\predfn(x) - f(\xi_n)  } + \norm{f(\xi_n) - f(x)} \stackrel{(i)}{=}\norm{\predfn(x) - \predfn(\xi_n)  } + \norm{f(\xi_n) - f(x)}$
&\stackrel{(\dagger)}{\leq} (L(n)+ L^*) \metric(x,\xi_n^x )^\hexp +  \frac \lambda 2 +\obserrbnd
\end{align}
where ($\dagger$) follows from (\ref{ineq:kjshkhd8e}) and the presupposed H\"older continuity of $f$. \\

Since by assumption, $ \metric(x,\xi_n^x )^\hexp \leq r_x(n) 
, \forall n$ this implies:
\begin{equation}
\metric_\outspace\bigl(\predfn(x) , f(x)\bigr)  \in \bigl[0, (L(n)+ L^*) r_x(n)^\hexp +  \frac \lambda 2 +\obserrbnd \bigr], \forall n. \end{equation} 
By assumption $ r_x(n), L(n) r_x^\hexp(n) \stackrel{n \to \infty}{\to} 0, \forall x$ and hence, $\metric_\outspace\bigl(\predfn(x) , f(x)\bigr)$ converges to $  [0,\frac \lambda 2 +\obserrbnd], \forall x$ with rate $\varrho_x \leq(L(n)+ L^*) r_x(n)^\hexp $.
%
%
%
%By assumption,  $\lim_{n \to \infty}  L(n) r_x (n)^\hexp = 0$ and hence, $\bigl[0, (L(n)+ L^*) r(n)^\hexp +  \frac \lambda 2 +\obserrbnd \bigr] \to \bigl[0,  \frac \lambda 2 +\obserrbnd \bigr]$ and $\bigl[0, (L(n)+ L^*) r_x(n)^\hexp +  \frac \lambda 2 +\obserrbnd \bigr] \to \bigl[0,  \frac \lambda 2 +\obserrbnd \bigr], \forall x$.

\textbf{(B)} Proceeding analogously as before, but utilising uniform convergence with rate $r$, we obtain:  \begin{equation} 
\metric_\outspace\bigl(\predfn(x) , f(x)\bigr)  \in \bigl[0, (L(n)+ L^*) r(n)^\hexp +  \frac \lambda 2 +\obserrbnd \bigr], \forall x \forall n. \end{equation} 
By assumption,  $L(n) r^\hexp(n) \in o(1)$ and thus, $\lim_{n \to \infty} L(n) r(n)^\hexp = 0$. Hence,  $$\errmetric^\infty = \seq{\sup_{x \in I} \metric_\outspace\bigl(\predfn(x) , f(x)\bigr) }{n \in \nat} \stackrel{\varrho}{\convto} [0, \frac \lambda 2 +\obserrbnd] $$ with rate $\varrho $ such that $\varrho(n) \leq (L(n)+ L^*) r(n)^\hexp, \forall n$.

\end{proof}
\end{thm}

Note a necessary condition was that the product of $L(n)$ and the rate was in $o(1)$, that is, vanishing in the limit of $n \to \infty$. A sufficient condition for this to hold is if $L(n)$ is guaranteed to be bounded (assuming the rate is vanishing). Above, we have established a sufficent condition for this (cf. Rem. \ref{rem:convlipconstestimates}): $L(n)$ is bounded as long as parameter $\hestthresh \geq 2 \obserrbnd + q$ for any $q \geq 0$. This yields the following result:

\begin{cor}
With definitions and assumption as in Thm. \ref{rem:convlipconstestimates}, if parameter $\hestthresh $ is chosen to be $2 \obserrbnd + q $ for any $q \geq 0$ then convergence to the ground truth is guaranteed (up to an twice the observational error and a term dependent on $q$). In particular, if the data becomes dense uniformly in $I \subseteq \inspace$ with a rate of $r(n)$  then, for some $\bar L \in [0,L^*]$ and any $n \in \nat$, we have 
\begin{equation}
\sup_{x \in I} \metric_\outspace\bigl(\predfn(x) , f(x)\bigr) \leq (\bar L+ L^*) r(n)^\hexp +  \frac { q} 2 +2 \obserrbnd  \stackrel{n \to \infty}{\convto}  \frac q 2 + 2  \obserrbnd . \end{equation} 
\label{cor:worstcaseconvhoeldertarget}
\end{cor}

%%%% ==== allgem. Bew. --------
%\begin{thm} \label{thm:convergenceifboundedconstandsamplecons}
%Let $\seq{\data_n}{n \in \nat}$ be a data set sequence with pertaining grid sequence $\seq{\grid_n}{n \in \nat}$, \\ with $\grid_n \subset G_{n+1} \subset \inspace (n \in \nat)$ converging to a dense subset of input domain $\inspace$. Assume all of the following: 
%\begin{enumerate} 
%\item The data sets $\data_n$ ($n \in \nat$) all have bounded observational error with $\obserrbnd := \sup_x \Metrico{0}{\obserr(x)} \in \Real_{\geq 0}$.
%\item The target $f: I \subseteq \inspace \to \outspace$ is H\"older continuous with $f \in \hoelset {L^*}{}  p$.
%\item The predictors $\predfn$ $(n \in \nat)$ are sample-consistent up to error $E:\inspace \to \Real$ with $\bar E := \sup_x E(x) \leq \infty$.
%\item The predictors are  H\"older continuous with H\"older constants bounded below $\bar L \in \Real_{\geq0}$. That is, 
%$\exists \bar L \geq 0 \forall n \in \nat: \predfn \in \hoelset {L(n)}{} p$  $ \wedge$ $ L(n) \leq \bar L$.
%\end{enumerate}
%
%
%Then, we have: 
%\begin{itemize}
%\item \textbf{(I)}The sequence of predictors $\seq \predfn {n \in \nat}$ converges to the ground truth $f$ pointwise up to error $\bar E$. That is, $\forall \epsilon \geq 0,  x \in \inspace \exists n_0 \in \nat \forall n \geq n_0: \metric_{\outspace}\bigl(\predfn(x), f(x) \bigr) \leq \epsilon + \bar E + \obserrbnd$.
%%\item 
%
%\item \textbf{(II)} If the grid sequence $\seq{\grid_n}{n \in \nat }$ converges to the domain $\inspace$ uniformly then $\seq \predfn {n \in \nat}$ converges to the target $f$ uniformly up to error bound $\bar E + \obserrbnd$. 
%
%That is, $\forall \epsilon \geq 0 \exists n_0 \in \nat \forall n \geq n_0, x \in \inspace: \metric_{\outspace}\bigl(\predfn(x), f(x) \bigr) \leq \epsilon + \bar E+\obserrbnd$.
%
%\item \textbf{(III)} If the uniform convergence of the grid sequence as per (II) occurs with a rate of at most $r:\nat \to \Real$ then we have:  $$\forall n \in \nat \forall x \in I \subseteq\inspace: \metric_\outspace\bigl(\predfn(x) , f(x)\bigr)\in [0,(\bar L+ L^*) r(n)^\hexp+ \bar E + \obserrbnd].$$ That is, as $n \to \infty$,  convergence (up to error $\bar E +\obserrbnd)$ of the predictors to the target occurs uniformly at a rate of at most $(\bar L+ L^*) r(n)^\hexp$
%\end{itemize}
%
%\begin{proof}
%For $x \in I\subseteq \inspace$ let $\xi_n^x$ denote the nearest neighbour of $x$ in grid $\grid_n$. That is, $\xi_n^x = \arg\inf_{s \in \grid_n} \metric(x,s)$.
%
%\textbf{(I)} 
%Since $\grid_n$ is assumed to converge to a dense subset of the domain $I$, we have$ \metric(x,\xi_n^x )^\hexp
%\stackrel{n \to \infty}{\longrightarrow} 0$.
%Furthermore, since the the predictors are sample-consistent on grid $\grid_n$ up to error $E$ and since $\xi_n^x \in \grid_n$ we can appeal to Lem. \ref{lem:groups_sampleconsandobserr} (as well as to the triangle inequality) to reason as follows:
%%there always is some $\phi(\xi_n^x) \in  \ball{\bar E}{0} = \{ y \in \outspace \,| \, \metric_\outspace(y,0 ) \leq \bar E \}$ such that (i) 
%$\metric_\outspace\bigl(\predfn(x) , f(\xi_n^x)  \bigr) 
%\leq 
%\metric_\outspace\bigl(\predfn(x) , \predfn(\xi_n^x)  \bigr) + \metric_\outspace\bigl(\predfn(\xi_n^x) , f(\xi_n^x)  \bigr)
%\stackrel{Lem. \ref{lem:groups_sampleconsandobserr}}{\leq} 
%\metric_\outspace\bigl(\predfn(x) , \predfn(\xi_n^x)  \bigr) + E(\xi_n^x) + \obserr(\xi_n^x)$.
%Hence,
%\begin{equation}
%\forall x: \metric_\outspace\bigl(\predfn(x) , f(\xi_n^x)  \bigr) \leq
%\metric_\outspace\bigl(\predfn(x) , \predfn(\xi_n^x)  \bigr) + \bar E + \obserrbnd.
%\label{ineq:ewlfjhflsape}
%\end{equation}
%%
%Thus, for $x \in \inspace$:
%%
%$\metric_\outspace\bigl(\predfn(x) , f(x)\bigr) \leq \metric_\outspace\bigl(\predfn(x) ,  f(\xi_n^x)  \bigr) + \metric_\outspace\bigl(f(\xi_n^x) , f(x)\bigr)$
%$\leq 
%\metric_\outspace\bigl(\predfn(x) , \predfn(\xi_n^x)  \bigr) + \bar E + \obserrbnd+ \metric_\outspace\bigl(f(\xi_n^x) , f(x)\bigr)$
%%$\norm{\predfn(x) - f(x)} \leq \norm{\predfn(x) - f(\xi_n)  } + \norm{f(\xi_n) - f(x)} \stackrel{(i)}{=}\norm{\predfn(x) - \predfn(\xi_n)  } + \norm{f(\xi_n) - f(x)}$
%$\stackrel{(\dagger)}{\leq} (\bar L+ L^*) \metric(x,\xi_n^x )^\hexp + \bar E +\obserrbnd$
%$\stackrel{n \to \infty}{\longrightarrow} \bar E +\obserrbnd$.
%Here ($\dagger$) leverages the H\"older continuity assumptions. \\
%
%\textbf{(II)} The proof is a trivial extension of (I). Let $\epsilon \geq 0.$ We show $\exists n_0 \in \nat \forall n \geq n_0, x \in \inspace: \metric_{\outspace}\bigl(\predfn(x), f(x) \bigr) \leq \epsilon + \bar E +\obserrbnd$.  
%Since $\grid_n$ converges to $I$ uniformly, $\exists n_0 \forall n\geq n_0\forall x \in I: \metric(x,\xi_n^x)^\hexp \leq \frac{\epsilon}{2 \max\{\bar L, L^* \}}$. Hold such $n_0$ fixed. Then for all $n \geq n_0, x \in I$ we have 
%$\metric_\outspace\bigl(\predfn(x) , f(x)\bigr) \newline \stackrel{(*)}{\leq} (\bar L+ L^*) \metric(x,\xi_n^x )^\hexp + \bar E +\obserrbnd \leq \epsilon + \bar E +\obserrbnd$. Here (*) follows from (\ref{ineq:ewlfjhflsape}) and ($\dagger$) in complete analogy to our derivations for case (I) above. 
%
%\textbf{(III)} By assumption of uniform convergence of $\seq{\grid_n}{n \in \nat}$ with rate $r(n)$, $\sup_{x \in I} \inf_{s_n \in \grid_n} \metric (x,s_n) =\sup_{x \in I} \metric (x,\xi_n^x) \leq r(n)$. The rest follows from (*).
%\end{proof}
%\end{thm}
%
%
%% ==== allgem. bew. ende=====

Of course in the absence of observational errors, one can choose $\hestthresh = 0$. In this case, the corollary implies that LACKI will learn the ground-truth arbitrarily well in the limit of infinitely dense data.
%Note, when $\hestthresh \geq 2 \obserr$, the worst-case error bound in the corollary involves a term of twice the observational error bound. In general, this is unavoidable without making further assumptions.

\begin{remark} [Curse of dimensionality]
Our bounds rely on the proximity (expressed by the rate functions) of the query input to the previously observed data.
Refer to Thm. \ref{thm:convergenceifboundedconstandsamplecons_LACKI}.
Roughly speaking, for a particular query input $x$, our guarantee in (A) asserts that the closer the query is to the previously seen data, the better the confidence in prediction accuracy. In (B) this is extended to a worst-case statement implying that the smaller the worst-case proximity of the data to any query in $I$, the smaller the worst-case prediction error can be. 
Unfortunately, this worst-case proximity and therefore, the prediction error bound, is subject to the \emph{curse of dimensionality}. That is, the number of samples necessary to guarantee a desired reduction in worst-case prediction uncertainty will inevitably have to scale exponentionally with the dimensionality of the space. A manifestation of this fact can be seen in Sec. \ref{sec:probconv_LACKI} where we give a sample complexity bound for uniformly distributed input samples. 
\end{remark}


Having established that our LACKI rule can learn any H\"older function with any H\"older constant, we will now attend to extend the results to non-H\"older functions. In preparation of the necessary derivations we will first rehearse universality and H\"older properties of radial basis function networks. 

Park and Sandberg derived universal approximation guarantees for radial-basis function networks \cite{Park1991}. In particular, on page 252 in their article the authors make an assertion that translates to our notation as follows: 

\begin{lem}[Expressiveness of RBFNs] \label{lem:RBFNunifapproxcompact} Assume $\inspace \subseteq \Real^d$ is compact. Given parameter  vector $\theta := (w_1,\ldots,w_m,\sigma_1,...,\sigma_m,c_1,\ldots, c_m)$ and kernel function $K: \inspace \to \outspace $ let $\beta(\cdot;\theta ) = \sum_{i=1}^m w_i \, K(\frac{\cdot - c_i}{\sigma_i} )  $ denote a radial basis function network (RBFN). Assume $K: \Real^d \to \Real$ is continuous and has non-vanishing integral, i.e. $\int_{\Real^d} K(x) \d x \neq 0$.
Then, the set $S_K:= \{ \beta(\cdot; \theta) \vert  m \in \nat, \theta \in \Real^{3m}  \}$ of all RBFNs is uniformly dense in the set $C(\inspace)$ of continuous functions on compact domain $\inspace$. That is,  $\forall f \in C(\inspace) \forall \epsilon >0  \exists m, \theta \in \Real^{3m} : \sup_{x \in \inspace}{ \abs{f(\cdot) -\beta(\cdot;\theta)  } } <\epsilon $. 
\end{lem}

\begin{remark} \label{rem:LipconstofRBFN}
We note that, for any finite-dimensional parameter $\theta$, any RBFN $\beta(\cdot;\theta)$ is Lipschitz continuous as long as the kernel $K$ is. This can be seen by applying Lem. \ref{lem:Hoeldarithmetic} which allows us to conclude that the Lipschitz constant of RBFN $\beta(\cdot;\theta ) = \sum_{i=1}^m w_i \, K(\frac{\cdot - c_i}{\sigma_i} ) $ is given by $L_\beta = \sum_{i=1}^m \abs{\frac{w_i}{\sigma_i}} L_{K}$ where $L_K \in \Real_{\geq 0}$ denotes a Lipschitz constant of $K$. By the same Lemma it is easy to see that choosing the Gaussian kernel for $K$ satisfies both the Lipschitz requirement as well as the integrability requirements of Lem. \ref{lem:RBFNunifapproxcompact}. As a by-product this means that on a compact support, any continuous function can be approximated by some Lipschitz function with arbitrarily small, positive worst-case error $\epsilon >0$. Note, it may well be the case that the Lipschitz constant of the approximator grows with decreasing approximation error bound $\epsilon$. We consider this to be inevitable when the approximated function is not Lipschitz.
\end{remark}

Harnessed with these preparatory statements we can move on to show that the LACKI rule can be set up to learn any continuous function up to arbitrary low error.
\begin{thm}[Universality of LACKI]
\label{thm:LACKIuniversality}
Assume we are given a sequence $\seq{\data_n}{n \in \nat}$ of samples with observational errors bounded by $\obserrbnd \in \Real_{\geq 0}$. We set the parameters of the LACKI rule to $\lbf =-\infty,\ubf =\infty, \underline L =0$ and $\hestthresh := 2 \bar e +q $ for some $q >0$. In this theorem, we assume that the set of interest $I \subseteq\inspace$ is compact.
\textbf{Then, we have:}

The LACKI rule as per Def. \ref{def:LACKI} is a universal approximator in the following sense:
If the sequence of input grids $\seq{\grid_n}{n \in \nat}$ relative to $I$ (uniformly) then the sequence of predictors $\seq{\predfn}{n \in \nat}$ computed by the LACKI rule (uniformly) converges to any continuous target $f : \inspace \to \Real$ up to error $2 \obserrbnd + \frac{3q}{2}$.
That is, the following holds true:



\begin{itemize}
\item (I) Let $x \in I$. If $\exists r_x \in o(1): (\grid_n) \stackrel{ r_x}{\bd} x$ then $\exists C \in \Real : \seq{\Metrico{\predfn(x)}{f(x)}}{ } \stackrel{C r_x^\hexp}{\convto}[2\obserrbnd + \frac{3q}{2}]$.
\item (II) If $\exists r \in o(1): (\grid_n) \stackrel{ r}{\bdu} I$ then $\exists C \in \Real : {\errmetric^\infty}{ } \stackrel{C r^\hexp}{\convto}[2\obserrbnd + \frac{3q}{2}]$.
%$\forall \epsilon >0,x \in I \exists n_0 \geq 0\forall n\geq n_0: \metric_\outspace\bigl(\predfn(x),f(x)\bigr) \leq \epsilon + 2 \obserrbnd + \frac{3q}{2}$.
%\item (II) If $(\grid_n)$  $\bdu \inspace$  then we have: $$\forall \epsilon >0 \exists n_0 \geq 0\forall n\geq n_0, x \in \inspace: \metric_\outspace\bigl(\predfn(x),f(x)\bigr) \leq \epsilon + 2 \obserrbnd + \frac{3q}{2}.$$ 
\end{itemize}
\end{thm}
\begin{proof}
We choose any parameter $\hestthresh =  2 \obserrbnd + q$ with $q >0$. As observed in Rem. \ref{rem:LipconstofRBFN}, Lem. \ref{lem:RBFNunifapproxcompact} allows us to infer that there exists a Lipschitz function $h$ that approximates the target with worst-case error of at most $\frac q 2$. That is, $\sup_{x \in \inspace} \metric_\outspace\bigl(h(x), f(x) \bigr) \leq \frac q 2$. (Also, note Lipschitz continuity implies H\"older continuity for any H\"older exponent $\hexp \in (0,1]$, and hence, $h \in \hoelset {L_h}{ }{\hexp}$ for some $ L_h \in \Real_{\geq 0}$.)

Consequently, there exists a function $\phi':\inspace \to \outspace$ with $\sup_x \Metrico{ 0}{\phi'(x) } \leq \frac q 2$ accounting for the discrepancy between the H\"older function $h$ and the target $f$: $f = h+ \phi'$. 

Furthermore,we define $\phi$ to be the bounded observational noise. Hence, we have $\tilde f = f+ \phi$ and $\sup_x \Metrico{0}{\phi(x)} \leq \obserrbnd$.
Combining both functions into $\psi := \phi+\phi'$, we can write $\tilde f = h + \psi$ with $\sup_x \Metrico{0}{\psi(x)}\leq \frac q 2 + \obserrbnd =: \bar \nu$.

This can be interpreted as follows:
Instead of viewing the given sample as being generated by target $f$ (with some observational error $\phi$) we can view the sample as being generated by the H\"older function $h$ corrupted by the extended ``observational noise'' $\psi$ accounting for both the original observational error and the discrepancy between the target and H\"older function $h$.
This gives us a reduction to the case of learning H\"older functions with observational error bounded by $\bar \nu$. Firstly, we note that $\hestthresh = 2 \obserrbnd +q =2 \bar \nu$ (which entails that the sequence $\seq{L(n)}{n \in \nat}$ is bounded by some constant $\bar L  =\sup_{x,x', \metric(x,x') >0} \frac{\metric_\outspace(h(x), h(x'))   -q}{\metric^\hexp(x,x')} \leq L_h$). 
Linking to Thm. \ref{thm:convergenceifboundedconstandsamplecons_LACKI}, we get all the desired statements with regard to learning $h$. These can easily be converted into statements about learning $f$ by adding the worst-case difference $\frac q 2$ between $f$ and $h$ to all error bounds. 
For example, leveraging $\sup_x \Metrico{ 0}{\phi'(x) } \leq \frac q 2$ and $\lambda = 2 \obserrbnd + q$ and going through analogous steps as in the previous theorem we obtain:   
%$\metric_\outspace\bigl(\predfn(x) , h(x)\bigr) \leq $\leq (\bar L+ L_h) \metric(x,\xi_n^x )^\hexp + \bar E +\obserrbnd$ 
\begin{align}
\metric_\outspace(\predfn(x), f(x) ) &= \metric_\outspace(\predfn(x), h+\phi' (x) ) 
\leq   \metric_\outspace\bigl(\predfn(x) , h(x)\bigr) + \metric_\outspace(0,\phi'(x) ) \\
&\leq (\bar L+ L_h) \metric(x,\xi_n^x )^\hexp + \frac \lambda 2 +\bar \nu + \frac q 2 \\
&\leq (\bar L+ L_h) \metric(x,\xi_n^x )^\hexp  +2 \obserrbnd + \frac {3q}{2} \label{ineq:euyiweeh}
\end{align}
where  $\xi^x_n := \arg\inf_{s \in \grid_n} \metric(x,s)$ denotes a nearest neighbour of $x$ in the input sample $\grid_n$.

So, convergence (pointwise or uniform) of the grid to the input space with a rate of at most $r(n)$ implies that the right-hand side of (\ref{ineq:euyiweeh}) and hence, the prediction error,  
converges (pointwise or uniformly) to the interval $[0,2 \obserrbnd + \frac {3q}{2}] $ with a rate of at most $(\bar L + L_h) r^\hexp(n)$ as $n \to  \infty$.



%%OLD PROOF:
%
%By Lem. \ref{lem:constadaptation_boundedness}, the lazily adapted H\"older constant estimates $L(n)$ are bounded from above by $\max\{\underline L, \tilde L\} = \tilde L  < \infty$. In conjunction with Lem. 
%\ref{lem:LACKIsampleconsistency} (which establishes sample consistency up to $\frac \hestthresh 2$) we have established that all preconditions of Thm. \ref{thm:convergenceifboundedconstandsamplecons} are met by the predictors $\predfn$ $ ({n \in \nat})$. Hence, the theorem allows us to conclude that the sequence of predictors $\seq{\predfn}{n \in \nat}$ converges to $\beta(\cdot;\theta)$ up to an error less than or equal to $\sup_x \metric_\outspace(0,\psi(x)) \leq \frac \hestthresh 2$:
%
%
%(I) In the general case this convergence is pointwise. 
%Let $x \in  \inspace$ and $\epsilon > 0$.  Thm. \ref{thm:convergenceifboundedconstandsamplecons}.(I) implies $\exists n_0 \forall n\geq n_0: \metric_\outspace\bigl (\predfn(x), \beta(x,;\theta) \bigr) \leq \epsilon+\frac \hestthresh 2$. Hence, $\metric_\outspace\bigl (\predfn(x), f(x) \bigr) = \metric_\outspace\bigl (\predfn(x), \beta(x,;\theta) + \psi(x) \bigr) \stackrel{Lem. \ref{lem:bilinaddtransinvgroup}}{\leq} \metric_\outspace\bigl(\predfn(x), \beta(x,;\theta)\bigr) +\metric_\outspace\bigl(0, \psi(x) \bigr) \leq \metric_\outspace\bigl(\predfn(x), \beta(x,;\theta)\bigr) +\frac \hestthresh 2\leq \epsilon +\hestthresh$.
%
%(II)  If convergence of the grid to domain $\inspace$ is uniform convergence of the predictors to the RBFN is uniform (by Thm. \ref{thm:convergenceifboundedconstandsamplecons}.II ). The remainder of the proof is completely anaologous to (I), except that $n_0$ can be chosen independently of $x$.
%
%

\end{proof}

%\begin{remark}
%Note, our universality guarantee of Thm. \ref{thm:LACKIuniversality} is much stronger than the guarantee for RBFNs.
% The universality statement for RBFNs is purely representational. That is, it states that in principle it is possible to find some RBFN with some choice of parameter that approximates any continuous function with arbitrarily small worst-case approximation error. However, it makes no mention of how to find such an RBFN. By contrast, our universality statement is about a concrete learning rule. And, given a desired error level it tells how to set the parameters of the learning rule in order to achieve the desired approximation error in the dense sample limit.
%\end{remark}






\subsubsection{Convergence in probability with uniformly distributed inputs}
\label{sec:probconv_LACKI}
Above we have given guarantees relative to the deterministic convergence rates of the input sample to the domain.
In this subsection, we shall study probabilistic convergence rates as a function of the sample size in situations where the sample is obtained by drawing inputs independently from a uniform probability distribution on $I=\inspace := [0,1]^d$. 

We can show that the worst-case prediction error given by $\sup_{x \in \inspace}\Metrico{\predfn(x)}{ f(x)}$ vanishes (up to the usual worst-case bounds in the presence of observational errors) in probability for canonical input-space metrics:  

%
%
%Under these conditions, we can prove that input sequence $\seq{\grid_n}{n \in \nat}$ becomes dense in probability:
%\begin{lem}
%For  $x \in \inspace = [0,1]^d$ let $\varepsilon \in (0,1)$ such that $\ball{\varepsilon}{x} =\{\xi \in \inspace | \norm{x-\xi}_\infty \leq \varepsilon \} \subset \inspace$. We have 
%\[ \Pr[\min\{ \norm{x - s}_\infty \,| \, s \in \grid_n \} \geq \varepsilon] \leq r(n) \]
%where 
%\[r(n) = \Bigl(1- \varepsilon^d \Bigr)^n \stackrel{n \to \infty}{\longrightarrow} 0.\]
%\begin{proof}
% It is easy to see that we have $\Pr[ \ball{\varepsilon}{x} ] = \int_{\xi \in \inspace} \indicator{\ball{\varepsilon}{x}} (\xi) \d \xi \geq \varepsilon^d$. 
%%
%Hence, 
%\begin{align}
%&\Pr[\min\{ \norm{x - s}_\infty \,| \, s \in \grid_n \} \geq \varepsilon ]  \\
%&=\Pr[\forall s \in \grid_n: \norm{x - s}_\infty \geq \varepsilon] \\
%&\stackrel{i.i.d.}{=} \prod_{i=1}^n \Pr[s_i \notin \ball{\varepsilon}{x}] \\
%&= \prod_{i=1}^n 1-\Pr[s_i \in \ball{\varepsilon}{x}] \\
%&\leq \prod_{i=1}^n 1-\varepsilon^d = (1-\varepsilon^d)^n.
%\end{align}
%\end{proof}
%\end{lem}

\begin{thm} Let $\inspace = [0,1]^d $ be the domain of target function $f \in \hoelset {L^*} { } \hexp $. Assume the input data $\grid_n = \{s_1,\dots,s_n\}$ contains $n$ data sample inputs which are drawn independently at random from a uniform distribution over $\inspace$. Furthermore, assume there are no observational errors, i.e. $\obserrbnd =0$, and, that $\metric(x,x') = \norm{x-x'}_\infty, \forall x,x' \in \inspace$. The worst-case error of our LACKI predictor vanishes in probability. 

That is, 
$$\forall \epsilon >0 \forall \delta \in (0,1) \exists N \in \nat \forall n \geq N : \Pr[ \sup_{x \in \inspace} \Metrico{\predfn(x)}{f(x)} >\epsilon] \leq \delta.$$
In particular, for all $\delta \in (0,1)$ we have 
$\Pr[ \sup_{x \in \inspace} \Metrico{\predfn(x)}{f(x)} >\epsilon] \leq \delta$
\begin{enumerate}
\item  for any $\epsilon \geq 2 L^*$, provided that $n \geq 1$;
\item for any $\epsilon < 2 L^*$, provided that $n \geq N := \ceil{\frac{ \log(\delta \, 2^{-kd}  )}{\log(1- 2^{-kd})} }$ with $k= \ceil{\frac{\log(\epsilon^{-1}2 L^*)}{\log 2}}$.
\end{enumerate}
\begin{proof}
Let $r_n := \sup_{x \in \inspace} \min_{s \in \grid_n} \metric(x,s) = \sup_{x \in \inspace} \min_{s \in \grid_n} \norm{x-s}_\infty \leq 1$ and let 

$P_n^\epsilon := \Pr[ \sup_{x \in \inspace} \Metrico{\predfn(x)}{f(x)} >\epsilon]$ which we intend to bound from above.
Remember, from Cor. \ref{cor:worstcaseconvhoeldertarget} $\sup_x \Metrico{\predfn(x)}{f(x)} \leq 2 L^*  r_n $. Hence, for $\epsilon \geq 2 L^*$, $P_n^\epsilon \leq 0, \forall n \in \nat$.

So, it suffices to focus on the case where $\epsilon < 2 L^*$. Now, $\sup_x \Metrico{\predfn(x)}{f(x)} \leq \epsilon  $ is implied by $\sup_x \Metrico{\predfn(x)}{f(x)} \leq 2 L^* r_n $ provided that $  r_n  \leq \frac {\epsilon}{2 L^*}$. So, we define an event $E_n$ that ensures $r_n$ satisfies the latter inequality with a probability that grows as $n$ increases.
To this end, we introduce a partition of the domain into $m$ hyper-rectangles $H_1,...,H_m$ of equal size, each having edge length $l_k=\frac 1 {2^k}$ where $k$ is a natural number such that $l_k \leq  \frac {\epsilon}{2 L^*} $. As a valid choice, we set $k:= \ceil{\frac{\log(\epsilon^{-1}2 L^*)}{\log 2}}$. Note, $\Pr[s_i \in H_j] = l_k^d = \frac{1}{2^{dk}}$.  By construction, in the event that each hyper-rectangle ends up containing at least one sample input of $\grid_n$, we have  $r_n \leq \frac {\epsilon}{2 L^*}$. 
We define the complement of this event as $\bar E_n := \{(s_1,...,s_n)  \in \inspace^n | \exists j \in \{1,...,m\} \forall i \in \{1,...,n\}: s_i \notin H_j  \}$. Let  $W:= \{ s=(s_1,...,s_n)  | \sup_x \Metrico{\predfn(x)}{f(x)} > \epsilon \}$ be the event that the sample inputs are located in such a fashion that they give rise to a worst-case error larger than $\epsilon$.  We have: $s \notin \bar E$ implies that $r(n) \leq \frac {\epsilon}{2 L^*} $ which in turn implies $\sup_x \Metrico{\predfn(x)}{f(x)} \leq \epsilon $, i.e. that $s \notin W$. Hence, $W \subseteq \bar E_n$ and thus, $ P_n^\epsilon =\Pr[W] \leq \Pr[\bar E_n]$.
So, to bound $P_n^\epsilon$ from above it suffices to bound $\Pr[\bar E_n]$ from above which we will do next: We can employ the union bound, utilise that $m = 2^{kd}$ and the fact that the $s_i$ are drawn i.i.d. from a uniform to see that $\Pr[\bar E_n] \leq \sum_{j=1}^m \prod_{i=1}^n \Pr[s_i \notin H_j] = 2^{kd}  (1-\frac{1}{2^{dk}})^n \stackrel{n \to \infty}{\convto} 0$ which shows the main statement of the theorem. To find an $n$ sufficently large to ensure $\Pr[W] \leq \delta$ we consider the inequality $2^{kd}  (1-\frac{1}{2^{dk}})^n \leq \delta$. Taking the $\log$ on both sides and rearranging yields the sufficient condition: $n \geq \frac{ \log(\delta \, 2^{-kd}  )}{\log(1- 2^{-kd})}$. 
\end{proof}
\end{thm}



\subsubsection{Some guarantees for online learning}


In the theorems above, we considered the worst-case asymptotics for the case where the data becomes dense in the domain. Here the error was evaluated on the entire input domain. By contrast, we will now consider an online learning setting where we incrementally get to observe samples along the trajectory of inputs $\seq{x_n}{n \in \nat }$ and are interested in the long-term one-step-lookahead prediction errors on this trajectory.
That is, we are interested in the evolution of prediction errors $\Metrico{\predfn(x_n)}{f(x_n) }$
where the predictor $\predf_{n}(\cdot)$ is based on $\data_{n} = \data_{n-1} \cup \{ \bigl(\state_{n-1}, \tilde f(\state_{n-1}) \bigr)\}, \forall n >1 $. 

We will show that this error trajectory vanishes (up to observational errors), provided that the input sequence $\seq{x_n}{n \in \nat}$ is bounded.

In preparation of these considerations, we will establish the following facts:

\begin{lem}
Assume we are given a trajectory $\seq{x_n}{n \in \nat}$ of inputs with $x_n \in \inspace$ where input space $\inspace$ can be endowed with a shift-invariant measure. Furthermore, assume the sequence  is bounded, i.e.  
$\metric_\inspace(x_n,0) \leq \beta$ for some $\beta \in \Real_+$ and all $n \in \nat$.
Finally assume the inputs of the available data coincide with the complete history of past inputs, i.e. $G_n = \{ x_i | i \in \nat, i < n\}$.
Then we have: \[ \dist(G_n,x_n) = \min\{\metric_\inspace(g,x_n) | \, g \in G_n\} \stackrel{n \to \infty}{\longrightarrow} 0.\]
\begin{proof}
The intuition behind the following proof is that if the distances were not to converge, there was an infinite number of disjoint balls around the input points that summed up to infinite volume. This however, would be a contradiction to the presupposed boundedness of the sequence.
We formalise this intuition as follows:
We can rephrase the desired convergence statement as 
\begin{equation}
\forall \epsilon > 0 \exists n \in \nat \forall m \geq n : \dist(x_{m}, G_{m}) \leq \epsilon.
\end{equation} 
For contradiction, assume that
\begin{equation}
\exists \epsilon > 0 \forall n \in \nat \exists m(n) \geq n : \dist(x_{m(n)}, G_{m(n)}) > \epsilon.
\end{equation} 
Hold such an $\epsilon >0$ fixed and choose any $n \in \nat$. 
By definition of $G_{m(n)} =\{ x_i | i < m(n)\} $ we have:
\eqn{eq:i34kjjk3}{\forall i < m(n) : \metric_\inspace(x_{m(n)},x_i) > \epsilon.}

Let $C_n := \bigcup_{i < n} \ball{\frac \epsilon 2}{x_i} $ be the union of all $\frac \epsilon 2$-balls around each point in $G_n$ and define $\bar I = \bigcup_{n \in \nat} C_n$.
By definition, each $x_n$ is contained in $\bar I$.
Since sequence $(x_n)_{n \in \nat}$ is bounded, $\bar I $ has a finite volume relative to some positive, shift-invariant measure $\mu$. I.e. $\mu(\bar I) < \infty$ (e.g. choose the Lebesgue measure for $\mu$). Furthermore, $\mu(C_n) \leq \sum_{i <n} \mu(B_i) \leq \mu(\bar I)< \infty$ where $B_i := \ball{\frac \epsilon 2}{x_i}$. Owing to the assumed shift-invariance, we can assign the same measure $M$ each ball, i.e. $M:=\mu(B_1) = \mu(B_n)\forall n \in \nat$. Thus, $\mu(C_n) \leq n M$.
Define $q:= \ceil{\frac{\mu(\bar I)}{M}} \in \nat$. This is an upper bound on the number of disjoint balls of measure $M$  that can be contained in $\bar I$. Intuitively, since this number is finite, there cannot be an infinite number of non-intersecting balls around the elements of the sequence $(x_n)_{n \in \nat}$. More formally our argument proceeds as follows:
Choose $n > q+1$. Statement (\ref{eq:i34kjjk3}) yields:
\eqn{eq:i34kjjk33}{\forall i \in \{1,...,n\} \exists p(i) \geq i \forall j \leq p(i): \metric_\inspace(x_{p(i)},x_j) > \epsilon.}  Define a permutation $\pi$ such that $\pi(p(1)) \leq \ldots \leq \pi(p(n))$. 
With Statement (\ref{eq:i34kjjk33}) it follows that \\
${\metric_\inspace(x_{\pi(p(i))},x_{\pi(p(j))}) > \epsilon}$ , $\forall i,j =1,...,n, i < j$. Thus, we conclude the disjointness conditions $B_{\pi(p(i))} \cap B_{\pi(p(j))} = \emptyset , \forall i,j =1,...,n, i \neq j$. 
Hence,  $\mu(\bar I) \geq \mu(C_{\pi(p(n))}) \geq \mu(C_{\pi(p(1))}) + \sum_{i=1}^n \mu(B_{\pi(p(i))}) \newline = \mu(C_{\pi(p(1))}) +  n M > \mu(C_{\pi(p(1))}) + (q+1) M \geq\mu(C_{\pi(p(1))}) + \mu(\bar I)$, where the last inequality follows from the fact that  $M q= M \ceil{\frac{\mu(\bar I)}{M}} \geq \mu(\bar I)$. Since $\mu(C_{\pi(p(1))}) \geq 0$, we have concluded the false statement $\mu(\bar I) > \mu(\bar I)$.

%Hence, $\mu(\bar I) \geq \mu(C_{\pi(N_n)+1}) = \mu(C_{\pi(N_1)}) + \sum_{i=1}^n \mu(B_{\pi(N_i)}) = \mu(C_{\pi(N_1)}) +  n M > \mu(C_{\pi(N_1)}) + (m+1) M >\mu(C_{\pi(N_1)}) + \mu(\bar I)$ by definition of $m= \ceil{\frac{\mu(\bar I)}{M}}$. But since $\mu(C_{\pi(N_1)}) \geq 0$, we conclude the false statement $\mu(\bar I) > \mu(\bar I)$.
\end{proof}
\label{lem:bndseq_entails_distgridvanish}
\end{lem}  

\begin{thm}
Assume that, for some $q\geq0$, we chose $\hestthresh = 2 \obserrbnd + q$ in our LACKI prediction rule. And, assume that the target $f$ is H\"older continuous up to some error level $\bar E_h$. That is, $f = \phi + \psi$ with $\phi \in \hoelset {L^*} { } \hexp$ and a function $\psi$ such that $\sup_x \metric_\outspace\bigl(0,\psi(x)\bigr) \leq \bar E_h \in \Real$.

Assume we are given a trajectory $\seq{x_n}{n \in \nat}$ of inputs that is bounded, i.e. where 
$\metric(x_n,0) \leq \beta$ for some $\beta \in \Real_+$ and all $n \in \nat$.
Furthermore, assume $\data_{n+1} = \data_n \cup \{ \bigl(x_n, \tilde f(x_n)\bigr) \}$ and thus, $\grid_n = \{ x_i | i \in \nat, i < n\}$.
Then the prediction error on the sequence vanishes up to the level of sample-consistency and H\"older continuity in the following sense:
 \[\metric_\outspace\bigl(\predfn(x_n),f(x_n) \bigr) \stackrel{n \to \infty}{\longrightarrow} [0,\frac q 2 + 2  \obserrbnd  + 2 \bar E_h].\]
In particular, in case the observations are error-free ($\tilde f = f$) and assuming the target is H\"older continuous then, when choosing $\hestthresh = 0$, the prediction error is guaranteed to vanish. That is,
\[\metric_\outspace\bigl(\predfn(x_n),f(x_n) \bigr) \stackrel{n \to \infty}{\longrightarrow}0.\]

\begin{proof}

Let $\xi_n  \in \argmin_{g \in \grid_n} \metric(x_n,g)$ denote the nearest neighbour of $x_n$ in $\grid_n = \{x_1,...,x_{n-1}\}$.

Since sequence $(x_n)$ is bounded, Lem. \ref{lem:bndseq_entails_distgridvanish} is applicable and hence: (i) $\lim_{n \to \infty} \metric (x_n,\xi_n) = 0$.

From Lem. \ref{lem:LACKIsampleconsistency} we conclude 
$\metric_\outspace\bigl(\predfn(\xi_n) ,  f(\xi_n)  \bigr) \leq 2\obserrbnd +  \frac q 2$. Hence, appealing to the triangle inequality, we see that 
(ii) $\metric_\outspace\bigl(\predfn(x_n) ,  f(\xi_n)  \bigr) \leq \metric_\outspace\bigl(\predfn(x_n) ,  \predfn(\xi_n) \bigr) + 2 \obserrbnd + \frac q 2$.



Moreover we note that the predictors $\predfn$ have H\"older constants $L(n)$ and that the $L(n)$ are bounded from above by some $\bar L \in \Real$. Thus,  $(iii)$ $\exists \bar L \in \Real\forall n \in \nat : \predfn \in \hoelset {\bar L} { } \hexp$. 
  
In conclusion,
$0\leq\metric_\outspace\bigl(\predfn(x_n) , f(x_n)\bigr) \leq \metric_\outspace\bigl(\predfn(x_n) ,  f(\xi_n)  \bigr) + \metric_\outspace\bigl(f(\xi_n) , f(x_n)\bigr) \stackrel{(ii)}{\leq} \metric_\outspace\bigl(\predfn(x_n) , \predfn(\xi_n) \bigr) + 2 \obserrbnd + \frac q 2 + \metric_\outspace\bigl(f(\xi_n) , f(x_n)\bigr) \leq \metric_\outspace\bigl(\predfn(x_n) , \predfn(\xi_n) \bigr) +2 \obserrbnd + \frac q 2 + \metric_\outspace\bigl(\phi(\xi_n) , \phi(x_n)\bigr) + 2 \bar E_h 
\newline
\stackrel{(iii)}{\leq} (\bar L+ L^* ) \metric(x_n,\xi_n )^\hexp + 2 \obserrbnd + \frac q 2 + 2 \bar E_h  \stackrel{n \to \infty}{\longrightarrow} 2 \obserrbnd + \frac q 2 + 2 \bar E_h $.
\end{proof}
\label{thm:vanisishingseqprederr_LACKI}
\end{thm} 
%
%\subsubsection{misc}
%We make the following assumptions:
%\begin{enumerate}
%\item Target function $f: \inspace \to \outspace = \Real^m$ has H\"older constant $L^* < \infty$ and exponent $\hexp$. 
%\item Parameter $\hestthresh$ is chosen to at least twice the level of observational error, i.e. $\hestthresh \geq 2 \obserrbnd$.
%\end{enumerate}
%
%\begin{lem}
%Define $E^\infty_n:= \sup_{x \in \inspace} \metric(\predfn(x),f(x))$ to be the worst-case error between the predictor and the target. Under the above assumptions this error is bounded as follows:  $$E^\infty_n \leq \mu_n =  2 L^* r_x(n)^\hexp +  \frac \hestthresh 2 +\obserrbnd$$
%where 
%$r_x(n) := \sup_{x \in \inspace} \inf_{s \in \grid_n} \metric(s,x)$ is the worst-case distance of any query input from the input data.
%
%
%\begin{proof}
%
%In the LACKI paper it was shown that 
%\begin{equation}
%\metric_\outspace\bigl(\predfn(x) , f(x)\bigr)  \in \bigl[0, (L(n)+ L^*) r_x(n)^\hexp +  \frac \hestthresh 2 +\obserrbnd \bigr] \end{equation} for any  $ x \in \inspace, n \in \nat$ and  any choice of parameter $\hestthresh \geq 0$.
%And, if $\hestthresh$ was chosen to be at least as large as $2 \obserrbnd$ then $L(n)$ is bounded by some number $\bar L \leq L^*$. Hence, we have 
%\begin{equation}
%\metric_\outspace\bigl(\predfn(x) , f(x)\bigr) \leq (\bar L+ L^*) r_x(n)^\hexp +  \frac \hestthresh 2 +\obserrbnd  \leq \mu_n < \infty,  \forall x \in \inspace, n \in \nat \end{equation}
%\end{proof}
%
%\end{lem}
%
%An immediate consequence is that if that data is sufificently representative then the worst-case prediction error is bounded. 
%In particular, for bounded input space of interest, we obtain boundedness of the worst case error (cf. Assumption 1 in the MPC part of this paper):
%
%\begin{cor}
%If  $\inspace$ is bounded, i.e. $\sup_{x \in \inspace} \metric(0,x) < \infty$ then $r_x(n) = \sup_{x \in \inspace} \inf_{s \in \grid_n} \metric(s,x) < \infty, \forall n \in \nat$ and thus, $\mu_n < \infty$.
%In this case, the the worst-case prediction error is bounded:
%$$E^\infty_n \leq \mu_n =  2 L^* r_x(n)^\hexp +  \frac \hestthresh 2 +\obserrbnd < \infty.$$
%\end{cor}
%
%
%Note, that by definition of the bounds $\mu_n$ on the worst-case prediction error that these bounds are not increasing with incrementallygrowing data size. Furthermore, if the sequence $\seq{\grid_n}{n \in \nat}$ of input data converges to the input space $\inspace$ uniformly, the prediction error sequence $E^\infty_n$ vanishes as $n \to \infty$ Formally, we can write this as follows:
%
%\begin{lem}
%\begin{enumerate}
%\item If $\data_n \subset \data_{n+1}, \forall n $
%then $\mu_m \geq \mu_n \forall m \leq n$.
%\item If  $\grid_n \stackrel{unif}{\to} \inspace$ then $E^\infty_n \to [0,\frac \hestthresh 2 +\obserrbnd] $  as $n \to \infty$.
%\end{enumerate}
%\begin{proof}
%\begin{enumerate}
%\item  $\data_n \subset \data_{n+1}$ implies $r_x(n) \geq r_x(n+1)$. The rest follows by induction appealing to the definition of $\mu_n$.
%\item $\grid_n \stackrel{unif}{\to} \inspace$ implies that $r_x(n) \to 0$ as $n \to \infty$. The rest follows from $0 \leq E^\infty_n \leq \mu_n = 2 L^* r_x(n)^\hexp +  \frac \hestthresh 2 +\obserrbnd \stackrel{n \to \infty}{ \to} \frac \hestthresh 2 +\obserrbnd $.
%\end{enumerate}
%\end{proof} 
%\end{lem}
%
%\begin{lem}
%Assume growing data, i.e. $\forall n: \data_n \subset \data_{n+1}$.
%The worst-case distance between two predictions is bounded by twice the bound on the worst-case error of the predictor that was based on the least amount of data.
%That is,  
%$$ \forall m,n \text{ with } m \leq n: \metric(\predfn(x),\predf_m(x) ) \leq 2 \mu_m, \forall  x \in \inspace.$$
%In particular, is the worst-case discrepancy between predictors bounded, i.e. 
%$$ \forall m,n, x \in \inspace: \metric(\predfn(x),\predf_m(x) ) \leq 2 \mu_1.$$
%\begin{proof}
%This follows trivally by appealing to the previous lemma and the triangle-inequality of pseudometrics:
%For any $m \leq n$ we have $\mu_m \geq \mu_n$. Hence, $\forall x: \metric(\predfn(x),\predf_m(x) ) \leq  \metric(\predfn(x),f ) +  \metric(f,\predf_m(x) ) \leq \mu_n+ \mu_m \leq 2 \mu_m$.
%\end{proof} 
%\end{lem}



%\begin{remark}
%Note, the bound on the speed of convergence generally decreases with increasing best Lipschitz constant of the RBFN. Generally, the Lipschitz constant will increase with decreasing $\hestthresh$. Therefore, so far, we have not shown convergence in the case of $\hestthresh=0$. 
%Intuitively, we should still have convergence since in the limit $L^* \to \infty$ the KI rule seems to coincide with a nearest neighbour regression rule. By construction of integrable functions, it should be clear that this rule converges to any function in the set $\mathcal L_q$ of power-$q$ integrable function  with respect to the $\mathcal L_q$ metric and $q \in \nat$.
%Investigating this observation in greater rigour could be subject to future work. Of course, from a practical point of view our current results are sufficient. If we set $\hestthresh >0$ below the machine-epsilon of the computer the algorithm is intended to run on, our results ensure that the algorithm will not be able to distinguish $\predfn(x)$ and the target $f$ for any input value and sufficiently large training data size $N_n \in \nat$.
%\end{remark}

\subsubsection{Computational complexity}
The computational complexity for computing the parameter update $L(n+1)$ based on $L(n)$, the pre-existing data $\data_n$ of size $N_n$ and a newly arriving sample input is in $\mathcal O(N_n m) $ where $m$ is the effort for evaluating the pseudo-metric. Typically, $m$ will scale linearly with the dimensionality of the input space. Therefore, online updates cost training time that will be linear in the number of existing training data and input dimensionality. In batch training, for a batch of $N$ samples, computation of the estimate will require an effort in $\mathcal O(N^2 D)$. 
Once the parameter $L(n)$ is computed, the effort for evaluating $\predfn(x)$ is linear in the number of samples and, again, typically linear in the input and output space dimensionality. 

However, it should be noted though that generalised nearest neighbor techniques can be utilised to reduce the prediction effort to expected logarithmic effort in the sample size (see \cite{Beliakov2006}). Devised for standard Lipschitz interpolation, this approach could be readily applied to our LACKI inference rule.

%\subsection{Kinky inference and model reference adaptive control for online learning and tracking control in the presence of wing rock dynamics }
\label{sec:KIMRAC}
As pointed out in \cite{chowdharyacc2013}, modern fighter aircraft designs are susceptible to lightly damped oscillations in roll known as ``wing rock''. Commonly occurring during landing \cite{Saad2000}, removing wing rock from the dynamics is crucial for precision control of such aircraft.
Precision tracking control in the presence of wing rock is a nonlinear problem of practical importance and has served as a test bed for a number nonlinear adaptive control methods \cite{Chowdhary2013,Monahemi1996,chowdharyacc2013}.

For comparison, we replicate the experiments of the recent work of Chowdhary et. al. \cite{Chowdhary2013,ChowdharyCDC2013}.\footnote{We are grateful to the authors for kindly providing the code.}
Here the authors have compared their Gaussian process based approach, called \textit{GP-MRAC}, to the more established adaptive model-reference control approach based on RBF networks \cite{Sanner1992,Kim1998}, referred to as \textit{RBFN-MRAC}. Replacing the Gaussian process learner by our kinky inference learner, we readily obtain an analogous approach which we will refer to as \textit{LACKI-MRAC}. As an additional baseline, we also examine the performance of a simple P-controller.

While with the exact same parameters settings of the experiments in \cite{Chowdhary2013}, performance of our LACKI-MRAC method comes second to GP-MRAC, we also evaluate the performance of all controllers over a range of 555 random parameter settings and initial conditions. As we will see, across this range of problem instances and parameter settings,LACKI-MRAC markedly outperforms all other methods.

\subsubsection{Model reference adaptive control}
Before proceeding with the wing rock application we will commence with (i) outlining model reference adaptive control (MRAC) \cite{astroemadaptivectrlbook2013} as considered in \cite{Chowdhary2013} and (ii) describe the deployment of kinky inference to this framework. 
We will now rehearse the description of MRAC for second-order systems following \cite{Chowdhary2013}. 

Assume $m \in \nat$ to be the dimensionality of a configuration of the system in question and define $d = 2m$ to be the dimensionality of the pertaining state space $\statespace$.

Let $x = [x_1;x_2] \in \statespace$ denote the state of the plant to be controlled.
Given the control-affine system 
 
\begin{align}
\dot x_1 &= x_2 \\
\dot x_2 &= a(x) + b(x) \, u(x) \label{eq:secorddynctrlaff}
\end{align}

it is desired to find a control law $u(x)$ such that the closed-loop dynamics exhibit a desired reference behaviour:

\begin{align}
\dot \xi_1 &= \xi_2 \\
\dot \xi_2 &= f_{r}(\xi,r)
\end{align}
where $r$ is a reference command, $f_r$ some desired response and $t \mapsto \xi (t)$ is the reference trajectory.

If a priori $a$ and $b$ are believed to coincide with $\hat a_0, \hat b_0$ respectively, the inversion control 
$u = \hat b_0^{-1} (- \hat a_0 +u')$ is applied. This reduces the closed-loop dynamics to 
$\dot x_1 = x_2, \dot x_2 = u' + \tilde a(x,u) $
where $\tilde a(x,u)$ captures the modelling error of the dynamics: 
\begin{equation}
	\tilde a (x,u ) = a(x) - \hat a_0(x) + \bigl(b(x) - \hat b_0(x)\bigr) u.
\end{equation}
 Let $I_d \in \Real^{d \times d}$ denote the identity matrix.  If $b$ is perfectly known, then $b - \hat b_0^{-1} = 0$ and the model error can be written as $\tilde a (x)= a(x) - \hat a_0(x)$. In particular, $\tilde a$ has lost its dependence on the control input. 



In this situation \cite{Chowdhary2013,ChowdharyCDC2013} propose to set 
the pseudo control as follows: $u'(x) :=  \nu_{r} + \nu_{pd} - \nu_{ad}$ where $\nu_{r} = f_{r}(\xi,r)$ is a feed-forward reference term,  $\nu_{ad}$ is a yet to be defined output of a learning module \emph{adaptive element} and $\nu_{pd} = [K_1 K_2] e$ is a feedback error term designed to decrease the \textit{tracking error} $e(t) = \xi(t) - x(t)$ by defining $K_1,K_2 \in \Real^{m \times m}$ as described in what is to follow.

Inserting these components, we see that the resulting \textit{error dynamics} are:


\begin{equation}\label{eq:errordynmrac}
	\dot e = \dot \xi - [x_2; \nu_r + \nu_{pd}+ \tilde a(x) ] = M e + B \bigl(\nu_{ad}(x) -  \tilde a(x)\bigr)
\end{equation}


where $M = \left(\begin{array}[h]{cc}
			O_m &  \, I_{m}\\
			-K_1 & -K_2 
					\end{array}\right)$ and $B = \left(\begin{array}[h]{c}
			O_m \\ I_m
					\end{array}\right)$.
If the feedback gain matrices $K_1,K_2$ parametrising $\nu_{pd}$ are chosen such that $M$ is stable then the error dynamics converge to zero as desired, provided the learning error $E_\lambda$ vanishes: $E_\lambda (x(t)) = \norm{\nu_{ad}(x(t)) -  a(x(t))} \stackrel{t \to \infty} {\longrightarrow} 0$. 

It is assumed that the adaptive element is the output of a learning algorithm that is tasked to learn $\tilde a$ online. This is done by continuously feeding it training examples of the form $\bigl(x(t_i), \tilde a(x(t_i)) + \varepsilon_i\bigr)$ where $\varepsilon_i$ is observational noise.  

Intuitively, assuming the learning algorithm is suitable to learn target $\tilde a$ (i.e. $\tilde a$ is close to some element in the hypothesis space \cite{mitchellbook:97} of the learner) and that the controller manages to keep the visited state space bounded, the learning error (as a function of time $t$) should vanish.

Substituting different learning algorithms yields different adaptive controllers. \textit{RBFN-MRAC} \cite{Kim1998} utilises radial basis function neural networks for this purpose whereas \textit{GP-MRAC} 
employs Gaussian process learning \cite{GPbook:2006} to learn $\tilde a$ \cite{Chowdhary2013,ChowdharyCDC2013}. 

%\jcom{possibly the following paragraph will go into a related work section:}
%Note, this setting of GP-MRAC is a special case of the SP-SIIC approach we introduced in Ch. \ref{ch:SPSIIC}. Here $\hat a_0$ is the prior mean function of the s.p. we place over the drift. Our exposition in Sec. \ref{ch:SPSIIC} is more general in that it allows for uncertain $b$ that is learned from data, considers under-actuated systems and makes no distributional restrictions. As an additional feature, GP-MRAC assumes that the data sets can be updated continuously. While of course this is not tractable the authors maintain a data corpus of a fixed size by a range of methods including simple FIFO queues as well as sparsification methods \cite{Kingravi2014,Csato2002}. In contrast, we utilise learning criteria to only include informative data points. In addition, we employ hyper-parameter optimisation which we show to be highly beneficial. While hyper-parameter training cannot be done at high frequency, it would be possible to run it in the background on a separate CPU (in parallel to the CPU on which the controller runs). 

In what is to follow, we utilise our LACKI method as the adaptive element. Following the nomenclature of the previous methods we name the resulting adaptive controller \textit{LACKI-MRAC}.

\subsubsection{The wing rock control problem}
The wing rock dynamics control problem considers an aircraft in flight. Denoting $x_1$ to be the roll attitude (angle of the aircraft wings) and $x_2$ the roll rate (measured in angles per second), the controller can set the aileron control input $u$ to influence the state $x := [x_1;x_2]$.

Based on \cite{Monahemi1996}, Chowdhary et. al. \cite{Chowdhary2013,ChowdharyCDC2013} consider the following model of the wing rock dynamics: 

\begin{align}
\dot x_1 &= x_2 \\
\dot x_2 &= a(x) + b \, u 
\end{align}
where $b =3$ is a known constant and 
$a(x) = W_0^* + W_1^* x_1 + W_2^* x_2 + W_3^* \abs{x_1} x_2 + W_4^* \abs{x_2} x_2 + W^*_5 x_2^3$ is an priori unknown nonlinear drift. 



Note, the drift is non-smooth but, employing Lem. \ref{lem:Hoeldarithmetic}, it would be easy to derive a Lipschitz constant on any bounded subset of state space if the parameters $W := (W_0^*,\ldots, W_5^*)$ were known.

To control the system we employ LACKI as the adaptive element $\nu_{ad}$.
In the absence of the knowledge of a Lipschitz constant, we start with a guess of $\underline L=1$ (which will turn out to be too low) and update it following the procedure described in Sec. \ref{sec:lacki}.

In a first instance, we replicated the experiments conducted in \cite{Chowdhary2013,chowdharyacc2013} with the exact same parameter settings. That is, we chose $W_0^* = 0.8, W_1^* = 0.2314, W_2^* = 0.6918, W_3^* = -0.6245, W_4^* = 0.0095, W_5^* = 0.0214$. 



The simulation initialised with start state $x = (3,6)^\top$ and simulated forward with a first-order Euler approximation with time increment $\tinc = 0.005 [s]$ over a time interval $\indsett = [t_0,t_f]$ with $t_0 = 0[s]$ and $t_f = 50[s]$. Training examples and control signal were continuously updated every $\Delta_u= \Delta_o = \tinc [s]$. The RBF and GP learning algorithms were initialised with fixed length scales of 0.3 units. The GP was given a training example budget of a maximum of 100 training data points to condition the posterior model on. Our kinky inference learner was initialised with $L =p= 1$ and updated online following the lazy update method described in Sec. \ref{sec:lazylipconstupdate}.

The test runs also exemplify the working of the lazy update rule.
The initial guess $\underline L=1$ was too low. However, our lazy update rule successfully picked up on this and had ended up increasing constant to $L=2.6014$ by the end of the online learning process.



%
%
\begin{figure*}
        \centering
				  \subfigure[Tracking error (RBF-MRAC).]{
    %\includegraphics[width = 3.7cm, height = 3cm]{content/figures/graph1_klein.eps}
    \includegraphics[ width=.3\textwidth, clip, trim = 2.5cm 8cm 3cm 9cm]
								%{content/Ch_kinkyinf/figs/resultswingrock_555trials}
								{content/Ch_kinkyinf/figs/wr2RBF}
   % \label{fig:wingrockresultsbp}
  } 
					  \subfigure[Tracking error (GP-MRAC).]{
    %\includegraphics[width = 3.7cm, height = 3cm]{content/figures/graph1_klein.eps}
    \includegraphics[width=.3\textwidth, clip, trim = 2.5cm 8cm 3cm 9cm]
								%{content/Ch_kinkyinf/figs/resultswingrock_555trials}
								{content/Ch_kinkyinf/figs/wr2GP}
   % \label{fig:wingrockresultsbp}
  } 
					  \subfigure[Tracking error (KI-MRAC).]{
    %\includegraphics[width = 3.7cm, height = 3cm]{content/figures/graph1_klein.eps}
    \includegraphics[width=.3\textwidth, clip, trim = 2.5cm 8cm 3cm 9cm]
								%{content/Ch_kinkyinf/figs/resultswingrock_555trials}
								{content/Ch_kinkyinf/figs/wr2}
   % \label{fig:wingrockresultsbp}
  }
   \caption{Tracking error comparison of first example.}
	 \label{fig:wrtrackerrorsex1}
\end{figure*}	   

The results are plotted in Fig. \ref{fig:wrtrackerrorsex1}. 
We can see that in terms of tracking error of the reference our LACKI-MRAC outperformed RBF-MRAC and was a close runner-up to GP-MRAC which had the lowest tracking errors.  

%
To obtain an impression of the learning performance of the three learning algorithms we also recorded the prediction error histories for this example problem. The results are depicted in Fig. \ref{fig:wrprederrorsex1}. We can see that our kinky inference method and the GP method both succeeded in predicting the drift quite accurately while the RBFN method was somewhat lagging behind.
This is consistent with the observations made in \cite{Chowdhary2013,ChowdharyCDC2013}. The authors explain the relatively poor performance of the radial basis function network method by the fact that the reference trajectory on occasion led outside the region of state space where the centres of the basis function were placed in advance. By contrast, due to the non-parametric nature of the GP, GP-MRAC does not suffer from such a priori limitations. In fact, it can be seen as an RBF method that flexibly places basis functions around all observed data points \cite{GPbook:2006}. We would add that, as a non-parametric method, LACKI-MRAC shares this kind of flexibility, which might explain the fairly similar performance. 

However, being an online method, the authors of GP-MRAC explicitly avoided hyperparameter training via optimising the marginal log-likelihood. The latter is commonly done in GP learning \cite{GPbook:2006} to avoid the impact of an unadjusted prior but is often a computational bottle neck. Therefore, avoiding such hyperparameter optimisation greatly enhances learning and prediction speed in an online setting. However, we would expect the performance of the prediction to be dependent upon the hyperparameter settings. As we have noted above, the Lipschitz constant depends on the part of state space visited at runtime. Similarly, we might expect length scale changes depending on the part of state space the trajectory is in. Unfortunately, \cite{Chowdhary2013,ChowdharyCDC2013,chowdharyacc2013} provide no discussion of the length scale parameter setting and also called the choice of the maximum training corpus size ``arbitrary''. 

Since the point of learning-based and adaptive control is to be able to adapt to various settings, we test the controllers across a range of randomised problem settings, initial conditions and parameter settings.

We created 555 randomised test runs of the wingrock tracking problems and tested each algorithm on each one of them. The initial state $x(t_0)$ was drawn uniformly at random from $[0,7] \times [0,7]$, the initial kernel length scales were drawn uniformly at random from $[0.05,2]$, and used both for RBF-MRAC and GP-MRAC. The initial H\"older constant $\underline L$ for LACKI-MRAC was initialised at random from the same interval but was allowed to be adapted as part of the online learning process. Furthermore, we chose $\hestthresh =0$. The parameter weights $W$ of the system dynamics specified above were multiplied by a constant drawn uniformly at random from the interval $[0,2]$. To allow for better predictive performance of GP-MRAC we doubled the maximal budget to 200 training examples. 
The feedback gains were chosen to be $K_1=K_2=1$. 

In addition to the three adaptive controllers we also tested the performance of a simple $P$ controller with just these feedback gains (i.e. we executed x-MRAC with adaptive element $\nu_{ad}=0$). This served as a baseline comparison to highlight the benefits of the adaptive element over simple feedback control.

The performance of all controllers across these randomised trials is depicted in Fig. \ref{fig:wingrockresultsbp}. Each data point of each boxplot represent a performance measurement for one particular trial.

\begin{figure*}
        \centering
				  \subfigure[Prediction v.s. ground truth  (RBF-MRAC).]{
    %\includegraphics[width = 3.7cm, height = 3cm]{content/figures/graph1_klein.eps}
    \includegraphics[width = .3\textwidth, clip, trim = 3cm 9cm 3cm 10cm]
								%{content/Ch_kinkyinf/figs/resultswingrock_555trials}
								{content/Ch_kinkyinf/figs/wr4RBF}
   % \label{fig:wingrockresultsbp}
  } 
					  \subfigure[Prediction v.s. ground truth  (GP-MRAC).]{
    %\includegraphics[width = 3.7cm, height = 3cm]{content/figures/graph1_klein.eps}
    \includegraphics[width=.3\textwidth, clip, trim = 3cm 9cm 3cm 10cm]
								%{content/Ch_kinkyinf/figs/resultswingrock_555trials}
								{content/Ch_kinkyinf/figs/wr4GP}
   % \label{fig:wingrockresultsbp}
  } 
					  \subfigure[Prediction v.s. ground truth  (KI-MRAC).]{
    %\includegraphics[width = 3.7cm, height = 3cm]{content/figures/graph1_klein.eps}
    \includegraphics[width=.3\textwidth, clip, trim = 3cm 9cm 3cm 10cm]
								%{content/Ch_kinkyinf/figs/resultswingrock_555trials}
								{content/Ch_kinkyinf/figs/wr4}
   % \label{fig:wingrockresultsbp}
  } 
   \caption{Prediction vs ground truth comparisons for the first example. Both nonparametric methods accurately predict the true drift and clearly outperform the RBFN learner.}
	\label{fig:wrprederrorsex1}
\end{figure*}	   


\begin{figure}
        \centering
				  \subfigure[Results over 555 randomised examples.]{
    %\includegraphics[width = 3.7cm, height = 3cm]{content/figures/graph1_klein.eps}
    \includegraphics[width = .5\textwidth, clip, trim = 1.5cm 8cm 1cm 7cm]
								%{content/Ch_kinkyinf/figs/resultswingrock_555trials}
								{content/Ch_kinkyinf/figs/wingrockbp}
   % \label{fig:wingrockresultsbp}
  } 
   \caption{Performance of the different online controllers over a range of 555 trials with randomised parameter settings and initial conditions. 1: RBF-MRAC, 2: GP-MRAC, 3:LACKI-MRAC, 4: P-Controller. LACKI-MRAC outperforms all other methods with respect to all performance measures, except for prediction runtime (where the parametric learner RBF-MRAC performs best).} 
       \label{fig:wingrockresultsbp}
\end{figure}	   
  
	
	For each method, the figures show the boxplots of the following recorded quantities: 
	\begin{itemize}
		\item \textit{log-XERR}: cummulative angular position error (log-deg), i.e. $\log(\int_{t_0}^{t_f} \norm{\xi_1(t) - x_1 (t)} \dt )$.
		\item \textit{log-XDOTERR}:  cummulative roll rate error (log-deg/sec.), i.e. $\log(\int_{t_0}^{t_f} \norm{\xi_2(t) - x_2 (t)} \dt )$.
		\item \textit{log-PREDERR}: log-prediction error, i.e. 
		
		$\log(\int_{t_0}^{t_f} \norm{\nu_{ad}(x(t)) - \tilde a(x(t))} \dt )$.
		\item \textit{log-CMD}: cummulative control magnitude (log-scale), i.e. $\log(\int_{t_0}^{t_f} \norm{u(t)} \dt )$.
		\item \textit{log-max. RT (predictions)}: the log of the maximal runtime (within time span $[t_0,t_f]$) each method took to generate a prediction $\nu_{ad}$ within the time span.
		\item \textit{log-max. RT (learning)}: the log of the maximal runtime (within time span $[t_0,t_f]$) it took each method to incorporate a new training example of the drift $\tilde a$.
	\end{itemize}
	
	As can be seen from Fig. \ref{fig:wingrockresultsbp}, all three adaptive methods outperformed the simple $P$ controller in terms of tracking error. 
	
	In terms of prediction runtime, the RBF-MRAC outperformed both GP-MRAC and LACKI-MRAC. This is hardly surprising. After all, RBF-MRAC is a parametric method with constant prediction time. By contrast, both non-parametric methods will have prediction times growing with the number of training examples.
That is, it would be the case if GP-MRAC were given an infinite training size budget. Indeed one might argue whether GP-MRAC, if operated with a finite budget, actually is a parametric approximation where the parameter consists of the hyperparameters along with the fixed-size training data matrix. When comparing the (maximum) prediction and learning runtimes one should also bear in mind that GP-MRAC predicted with up to 200 examples in the training data set. By contrast, LACKI-MRAC undiscerningly had incorporated all 10001 training points by the end of each trial.

Across the remaining metrics, LACKI-MRAC markedly outperformed all other methods.

Note, we have also attempted to test all methods across a greater range of problem settings, including larger initial states, more varied hyper-parameter settings, lower feedback gains and more varied choices of dynamics coefficients $W$. However, this resulted in GP-MRAC to often run into conditioning problems. This is a common issue in GP learning due to the necessity of matrix inversion or Cholesky decompositions of the covariance matrix. Similar behaviour ensued when setting the training size budget to large values. All these changes often resulted in long learning runtimes, spiky control outputs and thus, poor overall performance. Similarly, code execution of our RBF-MRAC implementation was frequently interrupted with error messages when the state was initialised to positions outside of the rectangle $[0,7] \times [0,7]$.

We have not investigated the root cause of these issues in greater detail yet. However, it might be worth exploring whether the great robustness of kinky inference might be an additional selling point that sets it apart from other recent adaptive control methods. Such robustness is of course important in control settings such as flight control where failure or erratic behaviour of the adaptive element may result in critical incidents. 

An example where GP-MRAC failed to track the reference occurred when repeating our first experiment  with the following modifications: The initial state was chosen to be $x(t_0) = (-90,40)^\top$ corresponding to a rapidly rotating aircraft. Furthermore, the wing rock coefficients $W$ were multiplied by a factor of $5$, amplifying the non-linearities of the drift field. 

When initialised with a length scale parameter of 0.3, the GP ran into conditioning problems and caused the output of the adaptive element in GP-MRAC to produce spikes of very large magnitude and thus, further destabilised the system. We tried the problem with various kernel length scale settings ranging from $0.3$ to $20$. Increasing the length scale parameter to length scale of at least 1 seemed to fix the conditioning problem. Nonetheless, GP-MRAC still did not manage to learn and stabilise the system in any of these settings. A record of GP-MRAC's performance in this example (for length scale of 1) is depicted in Fig.  \ref{fig:gpfailGPWR1} -  \ref{fig:gpfailGPWR3}. As the plots show, GP-MRAC starts with relatively high tracking and prediction error from which it could not recover. At about 26 seconds into the simulation the state rapidly diverged.

 
\begin{figure*}
        \centering
				  \subfigure[Position (GP-MRAC).]{
    %\includegraphics[width = 3.7cm, height = 3cm]{content/figures/graph1_klein.eps}
    \includegraphics[width = 5.5cm, clip, trim = 4cm 9cm 4cm 9cm]
								%{content/Ch_kinkyinf/figs/resultswingrock_555trials}
								{content/Ch_kinkyinf/figs/gpfail1}
   \label{fig:gpfailGPWR1}
  } 
					  \subfigure[Tracking error (GP-MRAC).]{
    %\includegraphics[width = 3.7cm, height = 3cm]{content/figures/graph1_klein.eps}
    \includegraphics[width = 5.5cm, clip, trim = 4cm 9cm 4cm 9cm]
								%{content/Ch_kinkyinf/figs/resultswingrock_555trials}
								{content/Ch_kinkyinf/figs/gpfail2}
   \label{fig:gpfailGPWR2}
  } 
							  \subfigure[Log - prediction error (GP-MRAC).]{
    %\includegraphics[width = 3.7cm, height = 3cm]{content/figures/graph1_klein.eps}
    \includegraphics[scale =.35]
								%{content/Ch_kinkyinf/figs/resultswingrock_555trials}
								{content/Ch_kinkyinf/figs/gpfail_gpprederr}
    \label{fig:gpfailGPWR3}
  } 
					  \subfigure[Position (KI-MRAC).]{
    %\includegraphics[width = 3.7cm, height = 3cm]{content/figures/graph1_klein.eps}
    \includegraphics[width = 5.5cm, clip, trim = 3.5cm 9cm 4cm 10cm]
								%{content/Ch_kinkyinf/figs/resultswingrock_555trials}
								{content/Ch_kinkyinf/figs/gpfail1_hfe}
    \label{fig:gpfailKIWR1}
  } 
					  \subfigure[Tracking error (KI-MRAC).]{
    %\includegraphics[width = 3.7cm, height = 3cm]{content/figures/graph1_klein.eps}
    \includegraphics[width = 5.5cm, clip, trim = 3.5cm 9cm 4cm 10cm]
								%{content/Ch_kinkyinf/figs/resultswingrock_555trials}
								{content/Ch_kinkyinf/figs/gpfail2_hfe}
    \label{fig:gpfailKIWR2}
  } 
%
					  \subfigure[Log - prediction error (KI-MRAC).]{
    %\includegraphics[width = 3.7cm, height = 3cm]{content/figures/graph1_klein.eps}
    \includegraphics[scale =.35,clip, trim = 0cm 0cm 0cm .1cm]
								%{content/Ch_kinkyinf/figs/resultswingrock_555trials}
								{content/Ch_kinkyinf/figs/gpfail_hfeprederr}
    \label{fig:gpfailKIWR3}
  } 
						  %\subfigure[State path (KI-MRAC).]{
    %%\includegraphics[width = 3.7cm, height = 3cm]{content/figures/graph1_klein.eps}
    %\includegraphics[width = 5cm, clip, trim = 2cm 9cm 1cm 7cm]
								%%{content/Ch_kinkyinf/figs/resultswingrock_555trials}
								%{content/Ch_kinkyinf/figs/gpfail_statepathHfe}
   %% \label{fig:wingrockresultsbp}
  %} 	
	%
   \caption{Example where GP-MRAC fails. By contrast,LACKI-MRAC manages to adapt and direct the system back to the desired trajectory.}
	\label{fig:gpfail}
\end{figure*}	 


For comparison, we also tried LACKI-MRAC on the same problem, starting with initial $L=1$ as before. Starting out with a relatively large tracking and prediction error, LACKI-MRAC nonetheless managed to recover and successfully track the system (see  Fig.  \ref{fig:gpfailKIWR1} -  \ref{fig:gpfailKIWR3}). The state path and learned drift model obtained by LACKI-MRAC are depicted in Fig. \ref{fig:gpfail2}.
%
\begin{figure*}
        \centering
				  \subfigure[State path (KI-MRAC).]{
    %\includegraphics[width = 3.7cm, height = 3cm]{content/figures/graph1_klein.eps}
    \includegraphics[width = .4\textwidth, clip, trim = 3cm 9cm 3cm 9cm]
								%{content/Ch_kinkyinf/figs/resultswingrock_555trials}
								{content/Ch_kinkyinf/figs/gpfail_statepathHfe}
   % \label{fig:wingrockresultsbp}
  } 	
	  \subfigure[Learned drift model (KI-MRAC).]{
    %\includegraphics[width = 3.7cm, height = 3cm]{content/figures/graph1_klein.eps}
    \includegraphics[width = .4\textwidth, clip, trim = 4.3cm 9cm 4cm 9cm]
								%{content/Ch_kinkyinf/figs/resultswingrock_555trials}
								{content/Ch_kinkyinf/figs/gpfail_hfelearnedmodel}
   % \label{fig:wingrockresultsbp}
  } 	
	%
   \caption{Depicted are the state path and the drift model learned online byLACKI-MRAC.}
	\label{fig:gpfail2}
\end{figure*}	 
%============== WINROCK ===========================================
\section{Lazily Adaptive Constant Kinky inference and model reference adaptive control}
\label{sec:MRAC_lacki}
So far, we have established some learning guarantees for LACKI as a method for supervised learning. In this section, we utilise our results and discuss the use of LACKI in the context of model-reference adaptive control. We introduce the control framework with a simple example of controlling the simulated roll dynamics of an aircraft under wing rock. In the second half of the remainder of the paper, we use our theoretical guarantees established in the first half in order to provide global asymptotic convergence guarantees of the closed-loop trajectory to the reference. 
Throughout the entire section we simplify our analysis by assuming the pseudo-metrics $\metric$ and $\metric_\outspace$ are in fact canonical norm-induced metrics. For instance, we assume that $\inspace = \Real^d$ is a finite-dimensional vector space and we have $\metric(x,x') =\norm{x-x'}$ for some norm $\norm{\cdot}$ equivalent to the maximum norm $\norm{\cdot}_\infty$.

\subsection{Online learning and tracking control in the presence of wing rock dynamics }
\label{sec:KIMRAC}
As pointed out in \cite{chowdharyacc2013}, modern fighter aircraft designs are susceptible to lightly damped oscillations in roll known as ``wing rock''. Commonly occurring during landing \cite{Saad2000}, removing wing rock from the dynamics is crucial for precision control of such aircraft.
Precision tracking control in the presence of wing rock is a nonlinear problem of practical importance and has served as a test bed for a number nonlinear adaptive control methods \cite{Chowdhary2013,Monahemi1996,chowdharyacc2013}.

For comparison, we replicate the experiments of the recent work of Chowdhary et. al. \cite{Chowdhary2013,ChowdharyCDC2013}.\footnote{We are grateful to the authors for kindly providing the code.}
Here the authors have compared their Gaussian process based approach, called \textit{GP-MRAC}, to the more established adaptive model-reference control approach based on RBF networks \cite{Sanner1992,Kim1998}, referred to as \textit{RBFN-MRAC}. Replacing the Gaussian process learner by our kinky inference learner, we readily obtain an analogous approach which we will refer to as \textit{LACKI-MRAC}. As an additional baseline, we also examine the performance of a simple P-controller.

While with the exact same parameters settings of the experiments in \cite{Chowdhary2013}, performance of our LACKI-MRAC method comes second to GP-MRAC, we also evaluate the performance of all controllers over a range of 555 random parameter settings and initial conditions. As we will see, across this range of problem instances and parameter settings, LACKI-MRAC markedly outperforms all other methods.

\subsubsection{Model reference adaptive control}\label{sec:mrac}
Before proceeding with the wing rock application we will commence with (i) outlining model reference adaptive control (MRAC) \cite{astroemadaptivectrlbook2013} as considered in \cite{Chowdhary2013} and (ii) describe the deployment of kinky inference to this framework. 
We will now rehearse the description of MRAC for second-order systems following \cite{Chowdhary2013}. 

Assume $m \in \nat$ to be the dimensionality of a configuration of the system in question and define $d = 2m$ to be the dimensionality of the pertaining state space $\statespace$.

Let $x = [x_1;x_2] \in \statespace$ denote the state of the plant to be controlled.
Given the control-affine system 
 
\begin{align}
\dot x_1 &= x_2 \\
\dot x_2 &= a(x) + b(x) \, u(x) \label{eq:secorddynctrlaff}
\end{align}

it is desired to find a control law $u(x)$ such that the closed-loop dynamics exhibit a desired reference behaviour:

\begin{align}
\dot \xi_1 &= \xi_2 \\
\dot \xi_2 &= f_{r}(\xi,r)
\end{align}
where $r$ is a reference command, $f_r$ some desired response and $t \mapsto \xi (t)$ is the reference trajectory.

If a priori $a$ and $b$ are believed to coincide with $\hat a_0, \hat b_0$ respectively, the inversion control 
$u = \hat b_0^{-1} (- \hat a_0 +u')$ is applied. This reduces the closed-loop dynamics to 
$\dot x_1 = x_2, \dot x_2 = u' + \tilde a(x,u) $
where $\tilde a(x,u)$ captures the modelling error of the dynamics: 
\begin{equation}
	\tilde a (x,u ) = a(x) - \hat a_0(x) + \bigl(b(x) - \hat b_0(x)\bigr) u.
\end{equation}
 Let $I_d \in \Real^{d \times d}$ denote the identity matrix.  If $b$ is perfectly known, then $b - \hat b_0^{-1} = 0$ and the model error can be written as $\tilde a (x)= a(x) - \hat a_0(x)$. In particular, $\tilde a$ has lost its dependence on the control input. 



In this situation \cite{Chowdhary2013,ChowdharyCDC2013} propose to set 
the pseudo control as follows: $u'(x) :=  \nu_{r} + \nu_{pd} - \nu_{ad}$ where $\nu_{r} = f_{r}(\xi,r)$ is a feed-forward reference term,  $\nu_{ad}$ is a yet to be defined output of a learning module \emph{adaptive element} and $\nu_{pd} = [K_1 K_2] e$ is a feedback error term designed to decrease the \textit{tracking error} $e(t) = \xi(t) - x(t)$ by defining $K_1,K_2 \in \Real^{m \times m}$ as described in what is to follow.

Inserting these components, we see that the resulting \textit{error dynamics} are:


\begin{equation}\label{eq:errordynmrac}
	\dot e = \dot \xi - [x_2; \nu_r + \nu_{pd}+ \tilde a(x) ] = M e + B \bigl(\nu_{ad}(x) -  \tilde a(x)\bigr)
\end{equation}


where $M = \left(\begin{array}[h]{cc}
			O_m &  \, I_{m}\\
			-K_1 & -K_2 
					\end{array}\right)$ and $B = \left(\begin{array}[h]{c}
			O_m \\ I_m
					\end{array}\right)$.
If the feedback gain matrices $K_1,K_2$ parametrising $\nu_{pd}$ are chosen such that $M$ is stable then the error dynamics converge to zero as desired, provided the learning error $E_\lambda$ vanishes: $E_\lambda (x(t)) = \norm{\nu_{ad}(x(t)) -  a(x(t))} \stackrel{t \to \infty} {\longrightarrow} 0$. 

It is assumed that the adaptive element is the output of a learning algorithm that is tasked to learn $\tilde a$ online. This is done by continuously feeding it training examples of the form $\bigl(x(t_i), \tilde a(x(t_i)) + \varepsilon_i\bigr)$ where $\varepsilon_i$ is observational noise.  

Intuitively, assuming the learning algorithm is suitable to learn target $\tilde a$ (i.e. $\tilde a$ is close to some element in the hypothesis space \cite{mitchellbook:97} of the learner) and that the controller manages to keep the visited state space bounded, the learning error (as a function of time $t$) should vanish.

Substituting different learning algorithms yields different adaptive controllers. \textit{RBFN-MRAC} \cite{Kim1998} utilises radial basis function neural networks for this purpose whereas \textit{GP-MRAC} 
employs Gaussian process learning \cite{GPbook:2006} to learn $\tilde a$ \cite{Chowdhary2013,ChowdharyCDC2013}. 

%\jcom{possibly the following paragraph will go into a related work section:}
%Note, this setting of GP-MRAC is a special case of the SP-SIIC approach we introduced in Ch. \ref{ch:SPSIIC}. Here $\hat a_0$ is the prior mean function of the s.p. we place over the drift. Our exposition in Sec. \ref{ch:SPSIIC} is more general in that it allows for uncertain $b$ that is learned from data, considers under-actuated systems and makes no distributional restrictions. As an additional feature, GP-MRAC assumes that the data sets can be updated continuously. While of course this is not tractable the authors maintain a data corpus of a fixed size by a range of methods including simple FIFO queues as well as sparsification methods \cite{Kingravi2014,Csato2002}. In contrast, we utilise learning criteria to only include informative data points. In addition, we employ hyper-parameter optimisation which we show to be highly beneficial. While hyper-parameter training cannot be done at high frequency, it would be possible to run it in the background on a separate CPU (in parallel to the CPU on which the controller runs). 

In what is to follow, we utilise our LACKI method as the adaptive element. Following the nomenclature of the previous methods we name the resulting adaptive controller \textit{LACKI-MRAC}.

\subsubsection{The wing rock control problem}
The wing rock dynamics control problem considers an aircraft in flight. Denoting $x_1$ to be the roll attitude (angle of the aircraft wings) and $x_2$ the roll rate (measured in angles per second), the controller can set the aileron control input $u$ to influence the state $x := [x_1;x_2]$.

Based on \cite{Monahemi1996}, Chowdhary et. al. \cite{Chowdhary2013,ChowdharyCDC2013} consider the following model of the wing rock dynamics: 

\begin{align}
\dot x_1 &= x_2 \\
\dot x_2 &= a(x) + b \, u 
\end{align}
where $b =3$ is a known constant and 
$a(x) = W_0^* + W_1^* x_1 + W_2^* x_2 + W_3^* \abs{x_1} x_2 + W_4^* \abs{x_2} x_2 + W^*_5 x_2^3$ is an priori unknown nonlinear drift. 



Note, the drift is non-smooth but it would be easy to derive a Lipschitz constant on any bounded subset of state space if the parameters $W := (W_0^*,\ldots, W_5^*)$ were known.

To control the system we employ LACKI as the adaptive element $\nu_{ad}$.
In the absence of the knowledge of a Lipschitz constant, we start with a guess of $\underline L=1$ (which will turn out to be too low) and update it following the procedure described in Sec. \ref{sec:lacki}.

In a first instance, we replicated the experiments conducted in \cite{Chowdhary2013,chowdharyacc2013} with the exact same parameter settings. That is, we chose $W_0^* = 0.8, W_1^* = 0.2314, W_2^* = 0.6918, W_3^* = -0.6245, W_4^* = 0.0095, W_5^* = 0.0214$. 



The simulation initialised with start state $x = (3,6)^\top$ and simulated forward with a first-order Euler approximation with time increment $\tinc = 0.005 [s]$ over a time interval $\indsett = [t_0,t_f]$ with $t_0 = 0[s]$ and $t_f = 50[s]$. Training examples and control signal were continuously updated every $\Delta_u= \Delta_o = \tinc [s]$. The RBF and GP learning algorithms were initialised with fixed length scales of 0.3 units. The GP was given a training example budget of a maximum of 100 training data points to condition the posterior model on. Our LACKI learner was initialised with $\underline L =\hexp= 1$ and updated online following our lazy update method described above.

The test runs also exemplify the working of the lazy update rule.
The initial guess $\underline L=1$ was too low. However, our lazy update rule successfully picked up on this and had ended up increasing constant to $L=2.6014$ by the end of the online learning process.



%
%
\begin{figure*}
        \centering
				  \subfigure[Tracking error (RBF-MRAC).]{
    %\includegraphics[width = 3.7cm, height = 3cm]{content/figures/graph1_klein.eps}
    \includegraphics[ width=.3\textwidth, clip, trim = 2.5cm 8cm 3cm 9cm]
								%{content/Ch_kinkyinf/figs/resultswingrock_555trials}
								{content/Ch_kinkyinf/figs/wr2RBF}
   % \label{fig:wingrockresultsbp}
  } 
					  \subfigure[Tracking error (GP-MRAC).]{
    %\includegraphics[width = 3.7cm, height = 3cm]{content/figures/graph1_klein.eps}
    \includegraphics[width=.3\textwidth, clip, trim = 2.5cm 8cm 3cm 9cm]
								%{content/Ch_kinkyinf/figs/resultswingrock_555trials}
								{content/Ch_kinkyinf/figs/wr2GP}
   % \label{fig:wingrockresultsbp}
  } 
					  \subfigure[Tracking error (KI-MRAC).]{
    %\includegraphics[width = 3.7cm, height = 3cm]{content/figures/graph1_klein.eps}
    \includegraphics[width=.3\textwidth, clip, trim = 2.5cm 8cm 3cm 9cm]
								%{content/Ch_kinkyinf/figs/resultswingrock_555trials}
								{content/Ch_kinkyinf/figs/wr2}
   % \label{fig:wingrockresultsbp}
  }
   \caption{Tracking error comparison of first example.}
	 \label{fig:wrtrackerrorsex1}
\end{figure*}	   

The results are plotted in Fig. \ref{fig:wrtrackerrorsex1}. 
We can see that in terms of tracking error of the reference our LACKI-MRAC outperformed RBF-MRAC and was a close runner-up to GP-MRAC which had the lowest tracking errors.  

%
To obtain an impression of the learning performance of the three learning algorithms we also recorded the prediction error histories for this example problem. The results are depicted in Fig. \ref{fig:wrprederrorsex1}. We can see that our kinky inference method and the GP method both succeeded in predicting the drift quite accurately while the RBFN method was somewhat lagging behind.
This is consistent with the observations made in \cite{Chowdhary2013,ChowdharyCDC2013}. The authors explain the relatively poor performance of the radial basis function network method by the fact that the reference trajectory on occasion led outside the region of state space where the centres of the basis function were placed in advance. By contrast, due to the non-parametric nature of the GP, GP-MRAC does not suffer from such a priori limitations. In fact, it can be seen as an RBF method that flexibly places basis functions around all observed data points \cite{GPbook:2006}. We would add that, as a non-parametric method, LACKI-MRAC shares this kind of flexibility, which might explain the fairly similar performance. 

However, being an online method, the authors of GP-MRAC explicitly avoided hyperparameter training via optimising the marginal log-likelihood. The latter is commonly done in GP learning \cite{GPbook:2006} to avoid the impact of an unadjusted prior but is often a computational bottle neck. Therefore, avoiding such hyperparameter optimisation greatly enhances learning and prediction speed in an online setting. However, we would expect the performance of the prediction to be dependent upon the hyperparameter settings. As we have noted above, the Lipschitz constant depends on the part of state space visited at runtime. Similarly, we might expect length scale changes depending on the part of state space the trajectory is in. Unfortunately, \cite{Chowdhary2013,ChowdharyCDC2013,chowdharyacc2013} provide no discussion of the length scale parameter setting and also called the choice of the maximum training corpus size ``arbitrary''. 

Since the point of learning-based and adaptive control is to be able to adapt to various settings, we test the controllers across a range of randomised problem settings, initial conditions and parameter settings.

We created 555 randomised test runs of the wingrock tracking problems and tested each algorithm on each one of them. The initial state $x(t_0)$ was drawn uniformly at random from $[0,7] \times [0,7]$, the initial kernel length scales were drawn uniformly at random from $[0.05,2]$, and used both for RBF-MRAC and GP-MRAC. The initial H\"older constant $\underline L$ for LACKI-MRAC was initialised at random from the same interval but was allowed to be adapted as part of the online learning process. Furthermore, we chose $\hestthresh =0$. The parameter weights $W$ of the system dynamics specified above were multiplied by a constant drawn uniformly at random from the interval $[0,2]$. To allow for better predictive performance of GP-MRAC we doubled the maximal budget to 200 training examples. 
The feedback gains were chosen to be $K_1=K_2=1$. 

In addition to the three adaptive controllers we also tested the performance of a simple $\hexp$ controller with just these feedback gains (i.e. we executed x-MRAC with adaptive element $\nu_{ad}=0$). This served as a baseline comparison to highlight the benefits of the adaptive element over simple feedback control.

The performance of all controllers across these randomised trials is depicted in Fig. \ref{fig:wingrockresultsbp}. Each data point of each boxplot represent a performance measurement for one particular trial.

\begin{figure*}
        \centering
				  \subfigure[Prediction v.s. ground truth  (RBF-MRAC).]{
    %\includegraphics[width = 3.7cm, height = 3cm]{content/figures/graph1_klein.eps}
    \includegraphics[width = .3\textwidth, clip, trim = 3cm 9cm 3cm 10cm]
								%{content/Ch_kinkyinf/figs/resultswingrock_555trials}
								{content/Ch_kinkyinf/figs/wr4RBF}
   % \label{fig:wingrockresultsbp}
  } 
					  \subfigure[Prediction v.s. ground truth  (GP-MRAC).]{
    %\includegraphics[width = 3.7cm, height = 3cm]{content/figures/graph1_klein.eps}
    \includegraphics[width=.3\textwidth, clip, trim = 3cm 9cm 3cm 10cm]
								%{content/Ch_kinkyinf/figs/resultswingrock_555trials}
								{content/Ch_kinkyinf/figs/wr4GP}
   % \label{fig:wingrockresultsbp}
  } 
					  \subfigure[Prediction v.s. ground truth  (LACKI-MRAC).]{
    %\includegraphics[width = 3.7cm, height = 3cm]{content/figures/graph1_klein.eps}
    \includegraphics[width=.3\textwidth, clip, trim = 3cm 9cm 3cm 10cm]
								%{content/Ch_kinkyinf/figs/resultswingrock_555trials}
								{content/Ch_kinkyinf/figs/wr4}
   % \label{fig:wingrockresultsbp}
  } 
   \caption{Prediction vs ground truth comparisons for the first example. Both nonparametric methods accurately predict the true drift and clearly outperform the RBFN learner.}
	\label{fig:wrprederrorsex1}
\end{figure*}	   


\begin{figure}
        \centering
				  \subfigure[Results over 555 randomised examples.]{
    %\includegraphics[width = 3.7cm, height = 3cm]{content/figures/graph1_klein.eps}
    \includegraphics[width = .5\textwidth, clip, trim = 1.5cm 8cm 1cm 7cm]
								%{content/Ch_kinkyinf/figs/resultswingrock_555trials}
								{content/Ch_kinkyinf/figs/wingrockbp}
   % \label{fig:wingrockresultsbp}
  } 
   \caption{Performance of the different online controllers over a range of 555 trials with randomised parameter settings and initial conditions. 1: RBF-MRAC, 2: GP-MRAC, 3:LACKI-MRAC, 4: P-Controller. LACKI-MRAC outperforms all other methods with respect to all performance measures, except for prediction runtime (where the parametric learner RBF-MRAC performs best).} 
       \label{fig:wingrockresultsbp}
\end{figure}	   
  
	
	For each method, the figures show the boxplots of the following recorded quantities: 
	\begin{itemize}
		\item \textit{log-XERR}: cummulative angular position error (log-deg), i.e. $\log(\int_{t_0}^{t_f} \norm{\xi_1(t) - x_1 (t)} \dt )$.
		\item \textit{log-XDOTERR}:  cummulative roll rate error (log-deg/sec.), i.e. $\log(\int_{t_0}^{t_f} \norm{\xi_2(t) - x_2 (t)} \dt )$.
		\item \textit{log-PREDERR}: log-prediction error, i.e. 
		
		$\log(\int_{t_0}^{t_f} \norm{\nu_{ad}(x(t)) - \tilde a(x(t))} \dt )$.
		\item \textit{log-CMD}: cummulative control magnitude (log-scale), i.e. $\log(\int_{t_0}^{t_f} \norm{u(t)} \dt )$.
		\item \textit{log-max. RT (predictions)}: the log of the maximal runtime (within time span $[t_0,t_f]$) each method took to generate a prediction $\nu_{ad}$ within the time span.
		\item \textit{log-max. RT (learning)}: the log of the maximal runtime (within time span $[t_0,t_f]$) it took each method to incorporate a new training example of the drift $\tilde a$.
	\end{itemize}
	
	As can be seen from Fig. \ref{fig:wingrockresultsbp}, all three adaptive methods outperformed the simple $\hexp$ controller in terms of tracking error. 
	
	In terms of prediction runtime, the RBF-MRAC outperformed both GP-MRAC and LACKI-MRAC. This is hardly surprising. After all, RBF-MRAC is a parametric method with constant prediction time. By contrast, both non-parametric methods will have prediction times growing with the number of training examples.
That is, it would be the case if GP-MRAC were given an infinite training size budget. Indeed one might argue whether GP-MRAC, if operated with a finite budget, actually is a parametric approximation where the parameter consists of the hyperparameters along with the fixed-size training data matrix. When comparing the (maximum) prediction and learning runtimes one should also bear in mind that GP-MRAC predicted with up to 200 examples in the training data set. By contrast, LACKI-MRAC undiscerningly had incorporated all 10001 training points by the end of each trial.

Across the remaining metrics, LACKI-MRAC markedly outperformed all other methods.

Note, we have also attempted to test all methods across a greater range of problem settings, including larger initial states, more varied hyper-parameter settings, lower feedback gains and more varied choices of dynamics coefficients $W$. However, this resulted in GP-MRAC to often run into conditioning problems. This is a common issue in GP learning due to the necessity of matrix inversion or Cholesky decompositions of the covariance matrix (it seems to be common practice to address this by hand-tuning the observational noise parameters). Similar behaviour ensued when setting the training size budget to large values. All these changes often resulted in long learning runtimes, spiky control outputs and thus, poor overall performance. Similarly, code execution of our RBF-MRAC implementation was frequently interrupted with error messages when the state was initialised to positions outside of the rectangle $[0,7] \times [0,7]$.

We have not investigated the root cause of these issues in greater detail yet. However, it might be worth exploring whether the great robustness of LACKI might be an additional selling point that sets it apart from other recent adaptive control methods. Such robustness is of course important in control settings such as flight control where failure or erratic behaviour of the adaptive element may result in critical incidents. 

An example where GP-MRAC failed to track the reference occurred when repeating our first experiment  with the following modifications: The initial state was chosen to be $x(t_0) = (-90,40)^\top$ corresponding to a rapidly rotating aircraft. Furthermore, the wing rock coefficients $W$ were multiplied by a factor of $5$, amplifying the non-linearities of the drift field. 

When initialised with a length scale parameter of 0.3, the GP ran into conditioning problems and caused the output of the adaptive element in GP-MRAC to produce spikes of very large magnitude and thus, further destabilised the system. We tried the problem with various kernel length scale settings ranging from $0.3$ to $20$. Increasing the length scale parameter to length scale of at least 1 seemed to fix the conditioning problem. Nonetheless, GP-MRAC still did not manage to learn and stabilise the system in any of these settings. A record of GP-MRAC's performance in this example (for length scale of 1) is depicted in Fig.  \ref{fig:gpfailGPWR1} -  \ref{fig:gpfailGPWR3}. As the plots show, GP-MRAC starts with relatively high tracking and prediction error from which it could not recover. At about 26 seconds into the simulation the state rapidly diverged.

 
\begin{figure*}
        \centering
				  \subfigure[Position (GP-MRAC).]{
    %\includegraphics[width = 3.7cm, height = 3cm]{content/figures/graph1_klein.eps}
    \includegraphics[width = 5.5cm, clip, trim = 4cm 9cm 4cm 9cm]
								%{content/Ch_kinkyinf/figs/resultswingrock_555trials}
								{content/Ch_kinkyinf/figs/gpfail1}
   \label{fig:gpfailGPWR1}
  } 
					  \subfigure[Tracking error (GP-MRAC).]{
    %\includegraphics[width = 3.7cm, height = 3cm]{content/figures/graph1_klein.eps}
    \includegraphics[width = 5.5cm, clip, trim = 4cm 9cm 4cm 9cm]
								%{content/Ch_kinkyinf/figs/resultswingrock_555trials}
								{content/Ch_kinkyinf/figs/gpfail2}
   \label{fig:gpfailGPWR2}
  } 
							  \subfigure[Log - prediction error (GP-MRAC).]{
    %\includegraphics[width = 3.7cm, height = 3cm]{content/figures/graph1_klein.eps}
    \includegraphics[scale =.35]
								%{content/Ch_kinkyinf/figs/resultswingrock_555trials}
								{content/Ch_kinkyinf/figs/gpfail_gpprederr}
    \label{fig:gpfailGPWR3}
  } 
					  \subfigure[Position (LACKI-MRAC).]{
    %\includegraphics[width = 3.7cm, height = 3cm]{content/figures/graph1_klein.eps}
    \includegraphics[width = 5.5cm, clip, trim = 3.5cm 9cm 4cm 10cm]
								%{content/Ch_kinkyinf/figs/resultswingrock_555trials}
								{content/Ch_kinkyinf/figs/gpfail1_hfe}
    \label{fig:gpfailKIWR1}
  } 
					  \subfigure[Tracking error (LACKI-MRAC).]{
    %\includegraphics[width = 3.7cm, height = 3cm]{content/figures/graph1_klein.eps}
    \includegraphics[width = 5.5cm, clip, trim = 3.5cm 9cm 4cm 10cm]
								%{content/Ch_kinkyinf/figs/resultswingrock_555trials}
								{content/Ch_kinkyinf/figs/gpfail2_hfe}
    \label{fig:gpfailKIWR2}
  } 
%
					  \subfigure[Log - prediction error (LACKI-MRAC).]{
    %\includegraphics[width = 3.7cm, height = 3cm]{content/figures/graph1_klein.eps}
    \includegraphics[scale =.35,clip, trim = 0cm 0cm 0cm .1cm]
								%{content/Ch_kinkyinf/figs/resultswingrock_555trials}
								{content/Ch_kinkyinf/figs/gpfail_hfeprederr}
    \label{fig:gpfailKIWR3}
  } 
						  %\subfigure[State path (KI-MRAC).]{
    %%\includegraphics[width = 3.7cm, height = 3cm]{content/figures/graph1_klein.eps}
    %\includegraphics[width = 5cm, clip, trim = 2cm 9cm 1cm 7cm]
								%%{content/Ch_kinkyinf/figs/resultswingrock_555trials}
								%{content/Ch_kinkyinf/figs/gpfail_statepathHfe}
   %% \label{fig:wingrockresultsbp}
  %} 	
	%
   \caption{Example where GP-MRAC fails. By contrast, LACKI-MRAC manages to adapt and direct the system back to the desired trajectory.}
	\label{fig:gpfail}
\end{figure*}	 


For comparison, we also tried LACKI-MRAC on the same problem, starting with initial $L=1$ as before. Starting out with a relatively large tracking and prediction error, LACKI-MRAC nonetheless managed to recover and successfully track the system (see  Fig.  \ref{fig:gpfailKIWR1} -  \ref{fig:gpfailKIWR3}). The state path and learned drift model obtained by LACKI-MRAC are depicted in Fig. \ref{fig:gpfail2}.
%
\begin{figure*}
        \centering
				  \subfigure[State path (LACKI-MRAC).]{
    %\includegraphics[width = 3.7cm, height = 3cm]{content/figures/graph1_klein.eps}
    \includegraphics[width = .4\textwidth, clip, trim = 3cm 9cm 3cm 9cm]
								%{content/Ch_kinkyinf/figs/resultswingrock_555trials}
								{content/Ch_kinkyinf/figs/gpfail_statepathHfe}
   % \label{fig:wingrockresultsbp}
  } 	
	  \subfigure[Learned drift model (LACKI-MRAC).]{
    %\includegraphics[width = 3.7cm, height = 3cm]{content/figures/graph1_klein.eps}
    \includegraphics[width = .4\textwidth, clip, trim = 4.3cm 9cm 4cm 9cm]
								%{content/Ch_kinkyinf/figs/resultswingrock_555trials}
								{content/Ch_kinkyinf/figs/gpfail_hfelearnedmodel}
   % \label{fig:wingrockresultsbp}
  } 	
	%
   \caption{Depicted are the state path and the drift model learned online by LACKI-MRAC.}
	\label{fig:gpfail2}
\end{figure*}	 
%=============== WINGROCK END ===================================


%================ STABILITY DISCRETE TIME =================
\subsection{Stability guarantees in discrete-time systems}
\label{sec:KIMRACstabbounds}

In the previous section, we gave an illustration LACKI-MRAC -- a combination of a feedback-linearising controller with our KI learning method. The results are encouraging -- our adaptive control law managed to learn a dynamic system online and to track a reference in the presence of wing-rock dynamics where other methods failed. 
To simulate the dynamics we relied on a first-order Euler approximation resulting in a discrete-time dynamical system.

In this section, we study the convergence of LACKI-MRAC in such discrete-time systems.
In these we will consider both the offline and the online learning setting. In the former, the LACKI-learner receives a sample set once and builds a controller that remains unaltered during execution time.
For this case, we will provide robustness bounds on the control success (as quantified by a bound on the norm of the error dynamics) as a function of the remaining uncertainty of the trained LACKI model. 

In the online learning setting, which we considered in the wing-rock control simulations, the LACKI-learner gets updated with the most recent observation after each time step. Provided the initial uncertainty is bounded on the given state space, we will be able to guarantee that LACKI-MRAC leads to a closed-loop system that is globally asymptotically stable. 

\subsubsection{Tracking error bounds for the offline learning setting} 
In the offline learning setting, the predictor sequence $\seq{\predfn}{n \in \nat}$ of adaptive elements is based on only one fixed data set $\data_0$ at time $0$ which is not updated subsequently.
That is $\data_n = \data_0, \forall n \in \nat$.

Given some data set $\data_n =\data_0$ at time $n$ and the resulting predictor $\predfn(\cdot)$, the model error is given by $F_n(\cdot) := f(\cdot) - \predfn(\cdot)$. Since we assume constant data, the error does not change either. That is, we have $F_n(x) = \ldots = F_0(x) = F(x), \forall n,\forall x$.
 
In our analysis, we consider discrete-time dynamical systems. For example the dynamics might be first-order Euler approximations of the control-affine dynamics of Sec. \ref{sec:KIMRAC}.
Consequently, the error dynamics as per Eq. \ref{eq:errordynmrac} translate to the recurrence relation:
\begin{equation}\label{eq:errordynmrac_nIMRACdiscrete}
	e_{n+1}  = M e_n + \tinc F(x_n)
\end{equation}
where $\tinc \in \Real_+$ is a positive time increment and  $n$ is the time step index. Furthermore,
%
\begin{equation}
	F(x_n) = f(x_n) - \predfn(x_n) = B \bigl(\nu_{ad}(x_n) -  \tilde a(x_n)\bigr) 
\end{equation}
is an uncertain increment due to the model error of the learner (cf. Eq. \ref{eq:errordynmrac}), 						$B = \left(\begin{array}[h]{c}
				O_m \\ \tinc I_m
						\end{array}\right)$ and 

\begin{equation}
	M = \left(\begin{array}[h]{cc}
				I_m &  \, \tinc I_{m}\\
				-\tinc K_1 & I_m- \tinc K_2 
						\end{array}\right) 
\end{equation}
					is the (error state) transition matrix. 	
Here, $m = \frac d 2$ is half the dimensionality of the state space, $I_m$ denotes the $m \times m$ identity matrix and $K_1,K_2$ are gain matrices that can be freely chosen by the designer of the linear pseudo controller.		
%This is completely consistent with the uncertain recurrence considered in Eq. \ref{eq:dynextstatediscrete1}, with the exception, that the uncertainty of increment $F_n$ is not probabilistic. Instead it is an uncertain quantity reflecting the prediction error of the trained KI learner. Assuming its assumptions are met, Thm. \ref{thm:KIdesiderata} tells us that $F_n$ is interval-bounded. Employing the same type of argument as in Lem. \ref{lem:origprocdiscrfacts} we see that the solution of the recurrence is given by:
By induction, it is easy to show that the recurrence can be converted into the closed-form expression:
\begin{align*}
	\vc e_n &= M \vc e_{n-1} + \tinc \vc F(x_{n+1})  \\ 
	&= \dots= M^k \, \vc e_0 + \tinc \sum_{i=0}^{n-1} M^{n-1-i} \, \vc F(x_i)	.	
\end{align*}	
For vector norm $\norm{\cdot}$, let $\matnorm{\cdot}$ denote a matrix norm such that $\norm{Mv} \leq \matnorm{M} \norm{v}$ for all suitable matrices $M$ and vectors $v$. For instance, for the Euclidean norm $\norm{\cdot} =\norm{\cdot}_2$, we can choose the spectral norm $\matnorm{\cdot}= \specnorm{\cdot}$ as a matrix norm. Or, for  $\norm{\cdot} =\norm{\cdot}_\infty$, if the vector space is $d$-dimensional, we can choose the matrix norm $\matnorm{\cdot}= \sqrt{d} \specnorm{\cdot}$. We desire to bound the  norm of the error. To this end, we leverage that the norms are sub-additive and sub-multiplicative to deduce: 
\begin{align}
	\norm{\vc e_n} &\leq   \matnorm{M^{n}} \, \norm{\vc e_0} + \tinc \sum_{i=0}^{n-1}  \matnorm{M^{n-1-i}} \, \norm{\vc F(x_i)}		\\
	&\leq \matnorm{M^{n}} \, \norm{\vc e_0} + \tinc \maxerrn_n	 \sum_{i=0}^{n-1}  \matnorm{M^{n-1-i}} =: \varrho[n]
	\label{ineq:errnormbasic}
\end{align}
where  $\maxerrn_n$ is chosen such that we can guarantee that $\maxerrn_n	\geq \max_{i=1,...,n-1} \norm{F(x_i)}$.
For instance, we could choose $\maxerrn_n := \sup_{s \in \iaspace_n} \norm{F(s)} $
	where $\iaspace_n = \bigcup_{t< k} \{x \in \iaspace | \norm{x - \xi[t]} \leq \varrho[t] \}$ is the union of the possible states around the reference trajectory $\xi[\cdot]$ at previous time steps.
\begin{remark}	[Bounded error innovations] 
We assume there exists a maximum model error norm $\maxerrn$, i.e. $\forall i: \norm{F_i(x_i)} \leq \maxerrn$ for some bound $\maxerrn \in \Real$.
This is a realistic assumption in any physically plausible systems where the drift forces $a(x)$ are inevitably bounded. Since also, for any finite data, our LACKI learner has a bounded prediction function (cf. Lem. \ref{lem:LACKIpredbounded}), the discrepancy given by $F$ is bounded as well. 
\label{rem:bnd_err_innovations}
\end{remark}

Given the boundedness of $F$ we have: \vspace{-1em}
\begin{equation}
	\norm{\vc e_n} \leq   \matnorm{M^{n}} \, \norm{\vc e_0} + \tinc \maxerrn \sum_{i=0}^{n-1}  \matnorm{M^{n-1-i}}.
\end{equation}

The right-hand side is convergent provided the gains $K_1, K_2$ are chosen such that $M$ is stable, i.e. $\specrad(M) <1$, and provided $\maxerrn_n$ is bounded (see e.g. \cite{calliess2014_thesis}). 

Whether or not $M$ is stable, in low dimensions, the sums can be computed offline and in advance. This is of great benefit if 
 the controller that is building on the error bounds is utilising optimisation-based control with a finite time-horizon.   

To obtain a (conservative) closed-form bound on the error norms \cite{calliess2014_thesis}
contains a derivation and discussion of the following result:

\begin{thm}\label{thm:normboundsboundeddisturbmainbody}
Let $(F_n)_{n \in \nat_0}$ be a norm-bounded sequence of vectors with $\maxerrn_n :=\max_{i \in \{0,\ldots,n-1\}} \norm{F_i(x_i)} \leq \maxerrn \in \Real$. 
For error sequence $(e_n)_{n \in \nat_0}$ defined by the linear recurrence 
	$e_{n+1}  = M e_n + \tinc F_n(x_n) \,\,\,\, (n \in \nat_0)$, we can define the following bounds:
	
	\begin{enumerate}
	\item $\norm{\vc e_n} \leq \matnorm{M^{n}} \, \norm{\vc e_0} + \tinc \maxerrn_{n} \sum_{i=0}^{n-1}  \matnorm{M^{i}}$. If $\specrad(M) <1$ and $\exists \maxerrn$ $:\maxerrn\geq \maxerrn_{n-1} \geq 0, \forall n$, the right-hand side converges as $n \to \infty$.
	\item Let $k_0 \in \nat, k_0 >1$ such that $\matnorm{M^n} <1, \forall n \geq k_0$, let $\varphi := \matnorm{M^{k_0}} <1$ and let $\delta_n := \floor{n/k_0}$. If $r:=\specrad(M) < 1$, for $n > k_0$,  we also have:
	\begin{align*} \norm{\vc e_n} &\leq c \, \varphi^{\delta_n} \, \norm{\vc e_0} + \tinc \maxerrn_{n} \Bigl(\sum_{j=0}^{n_0-1} \matnorm{M^j} + c \, k_0 \frac{\varphi-\varphi^{ \floor{\frac{n}{k_0}}+1 }}{1-\varphi} \Bigr)\\
	& \stackrel{n \to \infty}{\longrightarrow} C \leq \tinc \maxerrn \sum_{j=0}^{n_0-1} \matnorm{M^j} 
	+ \frac{\tinc \maxerrn\,  c\, k_0 \,\varphi}{1-\varphi} 
	\end{align*} 
	for some constants $C,c \in \Real$.
	Here, possible choices for $c \in \Real$ are: 
	
(i) $c= \max\{\matnorm{M^i} | i=1,\ldots,k_0-1 \}$ 

or (ii) $c = \frac{1}{(d-1)!} \Bigl(\frac{1 - d}{\log r}\Bigr)^{d-1}	\matnorm{M}^{d-1}\, \, \, r^{\frac{1 - d}{\log r}-d+1}  $.
%
Since $\matnorm{M} \neq 1$, one can also choose (iii) $c:= \matnorm{M}^{n_0}$. %In this case, $\matnorm{M^k} \leq  \, \matnorm{M}^{n_0}^{(k \text{ div } k_0) +1} $.

\item If $\matnorm{M} \neq 1$, we have: \newline
$\norm{\vc e_n} \leq \matnorm{M}^{n} \, \norm{\vc e_0} + \tinc \maxerrn_{n}   \frac{1-\matnorm{M}^n}{1-\matnorm{M}}. $
	\end{enumerate}
%\begin{proof}
%The theorem is proven in \cite{calliess2014_thesis}.
%
%\end{proof}	
\end{thm}


\subsubsection{Global asymptotic stability in the online learning setting}
In this subsection, we lift the assumption that the available sample is static. Instead we assume that at time step $n+1$ we get to see an additional sample of the uncertain drift at the state visited in the previous time step $n$. 
That is, the predictor $\predf_{n+1}(\cdot)$ is based on $\data_{n+1} = \data_n \cup \{ \bigl(\state_n, \tilde f(\state_n) \bigr)\}, \forall n $.
Therefore, we also need to index the innovations vector field by time. That is, $F_n := f - \predfn$ denotes the prediction error function (or innovation) due to the data available at time $n \in \nat$. 
As pointed out in Rem. \ref{rem:bnd_err_innovations}, the error innovations $F_n$ are assumed to be bounded for all finite sample sets $\data_n$. 




Above, we have seen that any continuous function can be approximated by some H\"older continuous LACKI predictor up to an arbitrarily small error. 
For convenience, we will establish the following definition:

\begin{defn}
We say that a continuous function $f$ is $L^*-p$-H\"older \emph{up to} error $\bar E_h \in \Real$ on domain $\inspace$ iff there is an $L^*-p$-H\"older function $\phi \in \hoelset {L^*} { } p$ and a function $\psi$ such that $\forall x: f(x) = \phi(x)+\psi(x), \, \sup_{x \in \inspace} \metric_\outspace\bigl(0,\psi(x)\bigr) \leq \bar E_h$.
\end{defn}


%%\begin{lem} Assume the target $f$ is bounded on sub-domain $I \subseteq \inspace$ and that the data increases between time steps, i.e. $\data_n \subseteq \data_{n+1}, \forall n \in \nat_0$. Then we have:
%%\[\forall x \in I \subset \inspace : \norm{F_n (x)}_\infty \geq \norm{F_{n+1}(x)}_\infty.\]
%%\jcom{true? perhaps not needed anyway..}
%%\end{lem}
% 
%%Above theorems consider the case where the data becomes dense in the domain and we are interested in the worst-case prediction error. Relevant to our online-learning and control setting below is the situation where we are only interested in the error on a specific trajectory of inputs. If the prediction error on such a trajectory vanishes then we will be able to establish stability of the trajectory.
%%In preparation of these considerations, we will establish the following facts:
%
%%
%%\begin{lem}
%%Assume we are given a trajectory $\seq{x_n}{n \in \nat}$ of inputs that is bounded, i.e. where 
%%$\metric(x_n,0) \leq \beta$ for some $\beta \in \Real_+$ and all $n \in \nat$.
%%Define $\grid_n = \{ x_i | i \in \nat, i < n\}$.
%%
%%Then we have \[ \dist(\grid_n,x_n) = \min\{\metric(g,x_n) | \, g \in \grid_n\} \stackrel{n \to \infty}{\longrightarrow} 0.\]
%%\begin{proof}
%%The intuition of the following proof is that if the distances were not to converge there would be an infinite number of disjoint balls around the input points that summed up to infinite volume. This however, would be a contradiction to the presupposed boundedness of the sequence.
%%We formalise this intuition as follows:
%%For contradiction assume there is an $\epsilon >0$ such that for all $n \in \nat$ there exists an $N_n \geq n $ such that 
%%\begin{equation}
%%\dist(x_{N_n}, G_{N_n}) > \epsilon.
%%\end{equation} 
%%Hence, by definition of $G_{N_n} =\{ x_i | i < N_n\} $,
%%\eqn{eq:i34kjjk3}{\forall i < N_n : \metric(x_{N_n},x_i) > \epsilon.}
%%
%%Let $C_n := \bigcup_{i < n} \ball{\frac \epsilon 2}{x_i} $ be the union of all $\frac \epsilon 2$-balls around each point in $\grid_n$ and define $\bar I = \bigcup_{n \in \nat} C_n$.
%%By definition, each $x_n$ is contained in $\bar I$.
%%Since sequence $(x_n)_{n \in \nat}$ is bounded, $\bar I $ has a finite volume relative to some positive, shift-invariant measure $\mu$. I.e. $\mu(\bar I) < \infty$. Furthermore, $\mu(C_n) \leq \sum_{i <n} \mu(B_i) \leq \mu(\bar I)< \infty$ where $B_i := \ball{\frac \epsilon 2}{x_i}$. Since $M:=\mu(B_1) = \mu(B_n)\forall n \in \nat$ we have $\mu(C_n) \leq (n-1) M$.
%%Define $m:= \ceil{\frac{\mu(\bar I)}{M}}$.
%%Now, choose $n > m+1$. For $i=1,...,n$ $\exists N_i \geq i \forall j \leq N_i: \metric(x_{N_i},x_j) > \epsilon$.  Define a permutation $\pi$ such that $\pi(N_1) \leq \ldots \leq \pi(N_n)$. Hence, we have $\metric(x_{\pi(N_i)},x_{\pi(N_j)}) > \epsilon , \forall i,j =1,...,n, i < j$. Thus, $B_{\pi(N_i)} \cap B_{\pi(N_j)} = \emptyset , \forall i,j =1,...,n, i \neq j$. Hence, $\mu(\bar I) \geq \mu(C_{\pi(N_n)+1}) = \mu(C_{\pi(N_1)}) + \sum_{i=1}^n \mu(B_{\pi(N_i)}) = \mu(C_{\pi(N_1)}) +  n M > \mu(C_{\pi(N_1)}) + (m+1) M >\mu(C_{\pi(N_1)}) + \mu(\bar I)$ by definition of $m= \ceil{\frac{\mu(\bar I)}{M}}$. But since $\mu(C_{\pi(N_1)}) \geq 0$, we conclude the false statement $\mu(\bar I) > \mu(\bar I)$.
%%\end{proof}
%%\label{lem:bndseq_entails_distgridvanish}
%%\end{lem} 
%%
%%
%\begin{lem}
%As before, let the observational noise level be bounded by $\obserr$ and assume the output- space pseudo-metric to be translation-invariant.
%Let target $f$ be $L^*-p$-H\"older up to error $E$. That is, there is an $L^*-p$-H\"older function $\psi$ and a function $\phi$ such that $\forall x: f(x) = \phi(x)+\psi(x), \, \metric_\outspace\bigl(0,\psi(x)\bigr) \leq E $.
%
%Assume we are given a trajectory $\seq{x_n}{n \in \nat}$ of inputs that is bounded, i.e. where 
%$\metric(x_n,0) \leq \beta$ for some $\beta \in \Real_+$ and all $n \in \nat$.
%Furthermore, assume $\data_{n+1} = \data_n \cup \{ \bigl(x_n, \tilde f(x_n)\bigr) \}$ and thus, $\grid_n = \{ x_i | i \in \nat, i < n\}$.
%Then the prediction error on the sequence vanishes, i.e. \[\metric_\outspace\bigl(\predfn(x_n),f(x_n) \bigr) \stackrel{n \to \infty}{\longrightarrow} \metric_\outspace(0,\obserr) + 2E.\]
%\begin{proof}
%We have established that $\predfn$ is $L(n)-p$- H\"older with $L(n) \leq \bar L$ for some $\bar L \in \Real_{\geq 0}$.
%
%Since sequence $(x_n)$ is bounded, Lem. \ref{lem:bndseq_entails_distgridvanish} is applicable. 
%Let $\xi_n := \argmin_{g \in \grid_n} \metric(x_n,g)$ denote the nearest neighbour of $x_n$ in $\grid_n = \{x_1,...,x_{n-1}\}$.
%Hence, $\forall n \in \nat: \metric (x_n,\xi_n) \to 0$ as $n \to \infty$.
%
%We have:
%$\metric_\outspace\bigl(\predfn(x_n) , f(x_n)\bigr) \leq \metric_\outspace\bigl(\predfn(x_n) ,  f(\xi_n)  \bigr) + \metric_\outspace\bigl(f(\xi_n) , f(x_n)\bigr) \stackrel{(i)}{=}\metric_\outspace\bigl(\predfn(x_n) , \predfn(\xi_n) \bigr) +\metric_\outspace\bigl(0 , \obserr\bigr) + \metric_\outspace\bigl(f(\xi_n) , f(x_n)\bigr) \leq \metric_\outspace\bigl(\predfn(x_n) , \predfn(\xi_n) \bigr) +\metric_\outspace\bigl(0 , \obserr\bigr) + \metric_\outspace\bigl(\phi(\xi_n) , \phi(x_n)\bigr) + 2E \stackrel{(ii)}{\leq} 2 \max\{\bar L, L^* \} \metric(x,\xi_n^x )^\hexp + \metric_\outspace(0,\obserr) + 2E \longrightarrow \metric_\outspace(0,\obserr) + 2E$.
%where steps denoted by (i) follow from Lem. \ref{lem:bilinaddtransinvgroup} and (ii) follows from the assumed H\"older properties.
%\end{proof}
%\label{lem:vanisishingseqprederr_groups}
%\end{lem} 
% 
% 
%\begin{lem} For some constant $L^*$ let the target $f$ be $L^*-p$-H\"older continuous up to level $\bar E_h$. 
%%Furthermore, 
%%assume that the observational error is zero, i.e. $\obserrbnd =0$.
%
%If $F_n(x_n) = f(x_n) - \predfn(x_n)$ as well as the reference $\xi_n$ are bounded for all $ n \in \nat_0$ then we have: 
%\[\forall k \in \nat_0: \norm{F_n(\state_{n+k})}_\infty \stackrel{n \to \infty}{\longrightarrow} [0, \frac \hestthresh 2 +  \obserrbnd +2\bar E_h]. \] 
%
%
%\begin{proof} Assume that, for some $q\geq0$, we choose $\hestthresh = 2 \obserrbnd + q$ in our LACKI prediction rule.
%If sequence $\seq{F_n(x_n)}{n\in \nat} $ is bounded then by Thm. \ref{thm:normboundsboundeddisturbmainbody} the error sequence \seq{e_n}{n\in \nat} is bounded. If in addition, the reference is bounded this implies that the sequence $(x_n)$ is bounded, too.
%Furthermore,
%by Lem. \ref{lem:LACKIsampleconsistency}, we know that our predictors $\predfn(\cdot)$ are sample-consistent up to $\frac \lambda 2$ and hence, $\norm{ F_n(x)}_\infty \leq \frac \hestthresh 2 +  \obserrbnd= 2 \obserrbnd + \frac q 2  ,\forall x \in \data_n=\{x_1,...,x_{n-1}\}$.
%The rest follows directly from Thm. \ref{thm:vanisishingseqprederr_LACKI} assuming $\metric_\outspace (f,f') = \norm{f-f'}_\infty$.
%\end{proof}
%\label{lem:innovnorm_convergence}
%\end{lem}

%
%\begin{remark}
%In particular, in case the target is H\"older continuous $(\bar E_h =0)$ and there is no observational error ($\obserrbnd =0$), parameterising our LACKI rule with $\hestthresh =0$ yields vanishing innovations, i.e. 
%\[\forall k \in \nat_0: \norm{F_n(\state_{n+k})}_\infty \stackrel{n \to \infty}{\longrightarrow} 0. \]
%\end{remark}
%
%The next theorem gives a stability guarantee for the tracking error of our LACKI-MRAC controller:

\begin{thm}[Tracking error convergence]
Assume that, for some $q\geq0$, we choose $\hestthresh = 2 \obserrbnd + q$ in our LACKI prediction rule and that the sequence of innovations $\seq{F_n(x_n)}{n\in \nat} $ as well as the reference $\seq{\xi_n}{n \in \nat}$ are bounded. 
If the initial error innovation function is bounded, i.e. if $\exists b \in \Real \forall x: \norm{F_0(x)}_\infty \leq b $, and, if $M$ is a stable matrix, i.e. if $\specrad(M) <1$, then the tracking error converges to the interval $[0,\frac q 2 + 2  \obserrbnd  + 2 \bar E_h]$. That is,
\[\norm{e_n}_\infty \stackrel{n \to \infty}{\longrightarrow} [0,\frac q 2 + 2  \obserrbnd  + 2 \bar E_h]. \]

\end{thm}
\begin{proof} Let $\norm{\cdot} := \norm{\cdot}_\infty$ with accociated matrix norm $\matnorm{\cdot} := \sqrt{d} \specnorm{\cdot}$. 


Let $\epsilon >0$. We desire to show: 
\begin{equation}\label{}
\exists N \in \nat \forall n\geq N: \norm{e_n} \leq \epsilon+\frac q 2 + 2  \obserrbnd  +2 \bar E_h.
\end{equation}

If sequence $\seq{F_n(x_n)}{n\in \nat} $ is bounded then, by Thm. \ref{thm:normboundsboundeddisturbmainbody}, the error sequence \seq{e_n}{n\in \nat} is bounded.  That is, $\exists b \in \Real \forall n: \norm{e_n} \leq \beta$.
Knowing that the error dynamics are bounded by some $\beta \geq 0$ we see that $\matnorm{M^{k}} \, \norm{\vc e_n} \leq \matnorm{M^{k}} \beta \stackrel{k \to \infty}{\longrightarrow} 0$. Here, the convergence to zero follows from the assumption that $M$ is a stable matrix. Hence,
we have:

$$(I)  \, \, \forall n \exists k_0(n) \in \nat \forall k \geq k_0(n): \matnorm{M^{k}} \, \norm{\vc e_n} \leq \frac\epsilon 2. $$


If in addition, the reference is bounded this implies that the sequence $(x_n)$ is bounded, too. Thm. \ref{thm:vanisishingseqprederr_LACKI} implies convergence of the innovations and hence, assuming $\metric_\outspace (f,f') = \norm{f-f'}$, we have:
\begin{equation}\label{eq:Fnconv}
\forall \varepsilon >0 \exists n_0 \forall n \geq n_0 : \norm{F_n(x_n)}\leq  \varepsilon +\frac q 2 + 2  \obserrbnd  +2 \bar E_h.
\end{equation} 


Referring to (i) in Thm. \ref{thm:normboundsboundeddisturbmainbody}, With a change of variables we can follow analogous steps to convert Ineq. (\ref{ineq:errnormbasic}) to state that for all $k \in \nat, n \in \nat_0$ we have: 
\begin{align}
	\norm{\vc e_{n+k}} 	&\leq \matnorm{M^{k}} \, \norm{\vc e_n} + \tinc Q_{n:n+k}	 \sum_{i=0}^{k-1}  \matnorm{M^{k-1-i}} 	\label{ineq:errnormgeneral}
\end{align}
$Q_{n:n+k}:=\max\{\norm{F_n(x_n)},\ldots,\norm{F_{k+n-1}(x_{k+n-1})} \}$. 
With Gelfand's formula and the standard root test for series it is easy to establish convergence of the series: That is, there exists $\sigma \in \Real$ with $\lim_{k \to \infty} \tinc \sum_{i=0}^{k-1}  \matnorm{M^{k-1-i}} = \sigma$. And, we have $\tinc \sum_{i=0}^{k-1}  \matnorm{M^{k-1-i}} \leq \sigma, \forall k$.
Hence,
\begin{equation}
\norm{\vc e_{n+k}} \leq \matnorm{M^{k}} \, \norm{\vc e_n} + \sigma \, Q_{k:n+k}, \forall n \in \nat_0, k \in \nat.
\label{eq:errdyn320894}
\end{equation}


With (\ref{eq:Fnconv}) follows that there exists $n_0 \in \nat_0$ such that we have: $$ (II) \,\,  \forall k \in \nat : Q_{n_0:n_0+k}\leq  \frac{\epsilon}{2 \sigma} +\frac q 2 + 2  \obserrbnd  +2 \bar E_h.
 $$


%

Combining (I) and (II) with Eq. \ref{eq:errdyn320894} allows us to conclude that for any $n \geq N:= n_0 +k_0(n_0)$  we have 
\begin{equation*}
\norm{\vc e_{n}} \leq \frac \epsilon 2 + \sigma \,\frac{\epsilon}{2  \sigma} +\frac q 2 + 2  \obserrbnd  +2 \bar E_h= \epsilon +\frac q 2 + 2  \obserrbnd  +2 \bar E_h. 
\label{eq:errdyn3208942}
\end{equation*}
\end{proof}
Note, since the error converges to a bounded set the state will converge to the target trajectory. So, if the target trajectory is bounded, the continuity of the control law (as a function of state) implies that the control is bounded as well.

%
%
\begin{cor}
In the special case of error-free observations of a H\"older continuous target function, choosing a parameter $\hestthresh =0$ implies:
\[\norm{e_n}_\infty \stackrel{n \to \infty}{\longrightarrow} 0. \]
That is, the error dynamics are globally asymptotically stable with equilibrium point 0.
Furthermore, the control action sequence $\seq{u(x_n)}{n \in \nat}$ converges, provided the reference trajectory $\seq{\xi_n}{n \in \nat}$ is bounded.
\end{cor}

\begin{proof} The stability statement is an immediate consequence of the preceding theorem. 
Remember from Sec. \ref{sec:mrac} that the control action at time $n$ is of the form $u_n := u(x_n) = -\predfn(x_n) - K e_n +c $ for some constant $c$. We show that $(u_n)$ is a Cauchy sequence, provided that the reference sequence $\xi_n$ is. Due to the fact that $\inspace$ is a Hilbert space assumption, this implies the desired convergence result.

So, let $\epsilon >0$. Since $(e_n),(\xi_n)$ converge, also the state sequence $(x_n)$ converges. Hence, all three are convergent Cauchy sequences. In particular, there is $N$ such that for all $n,m >N$: $\norm{e_n-e_m}< \frac{\epsilon} {2 \matnorm{K}}$ and $\norm{x_n-x_m}< \frac{\epsilon} {2\bar L}$. 
Hence, utilising the definition of the control law and the fact that all predictors are H\"older continuous with H\"older constant $\bar L$, for all $m,n >N:$ $\norm{u_n-u_m} \leq \matnorm{K} \norm{e_n -e_m} + \norm{\predfn(x_n) - \predf_m(x_m) } \leq \frac{\epsilon}{2} + \bar L \norm{x_n -x_m} \leq \epsilon $. Therefore, $(u_n)$ is a Cauchy sequence and hence, convergent.

\end{proof}



%================= STABILITY DISCRETE TIME END ============
%\section{Stability guarantees in discrete-time systems}
\label{sec:KIMRACstabbounds}

In the previous section, we gave an illustration LACKI-MRAC -- a combination of a feedback-linearising controller with our KI learning method. The results are encouraging -- our adaptive control law managed to learn a dynamic system online and to track a reference in the presence of wingrock dynamics where other state of the art methods failed. 

To simulate the dynamics we relied on a first-order Euler approximation resulting in a discrete-time dynamical system.

In this section, we study the convergence of LACKI-MRAC in such discrete-time systems.
In these we will consider both the offline and the online learning setting. In the former, the LACKI-learner receives a sample set once and builds a controller that remains unaltered during execution time.
For this case, we will provide robustness bounds on the control success (as quantified by a bound on the norm of the error dynamics) as a function of the remaining uncertainty of the trained LACKI model. 

In the online learning setting, which we considered in the wing-rock control simulations, the LACKI-learner gets updated with the most recent observation after each time step. Provided the initial uncertainty is bounded on the given state space, we will be able to guarantee that LACKI-MRAC leads to a closed-loop system that is globally asymptotically stable.

\subsection{Tracking error bounds for the offline learning setting} 
In the offline learning setting, the predictor sequence $\seq{\predfn}{n \in \nat}$ of adaptive elements is based on only one fixed data set $\data_0$ at time $0$ which is not updated subsequently.
That is $\data_n = \data_0, \forall n \in \nat$.

Given some data set $\data_n =\data_0$ at time $n$ and the resulting predictor $\predfn(\cdot)$, the model error is given by $F_n(\cdot) := f(\cdot) - \predfn(\cdot)$. Since we assume constant data, the error does not change either. That is, we have $F_n \equiv F_0 \equiv: F, \forall n$.
 
In our analysis, we consider discrete-time dynamical systems. For example the dynamics might be first-order Euler approximations of the control-affine dynamics of Sec. \ref{sec:KIMRAC}.
Consequently, the error dynamics as per Eq. \ref{eq:errordynmrac} translate to the recurrence relation:
\begin{equation}\label{eq:errordynmrac_nIMRACdiscrete}
	e_{n+1}  = M e_n + \tinc F_n(x_n)
\end{equation}
where $\tinc \in \Real_+$ is a positive time increment and  $n$ is the time step index. Furthermore,
%
\begin{equation}
	F_n(x_n) = f(x_n) - \predfn(x_n) = B \bigl(\nu_{ad}(x_n) -  \tilde a(x_n)\bigr) 
\end{equation}
is an uncertain increment due to the model error of the learner (cf. Eq. \ref{eq:errordynmrac}), 						$B = \left(\begin{array}[h]{c}
				O_m \\ \tinc I_m
						\end{array}\right)$ and 

\begin{equation}
	M = \left(\begin{array}[h]{cc}
				I_m &  \, \tinc I_{m}\\
				-\tinc K_1 & I_m- \tinc K_2 
						\end{array}\right) 
\end{equation}
					is the (error state) transition matrix. 	
Here, $m = \frac d 2$ is half the dimensionality of the state space, $I_m$ denotes the $m \times m$ identity matrix and $K_1,K_2$ are gain matrices that can be freely chosen by the designer of the linear pseudo controller.		
%This is completely consistent with the uncertain recurrence considered in Eq. \ref{eq:dynextstatediscrete1}, with the exception, that the uncertainty of increment $F_n$ is not probabilistic. Instead it is an uncertain quantity reflecting the prediction error of the trained KI learner. Assuming its assumptions are met, Thm. \ref{thm:KIdesiderata} tells us that $F_n$ is interval-bounded. Employing the same type of argument as in Lem. \ref{lem:origprocdiscrfacts} we see that the solution of the recurrence is given by:
By induction, it is easy to show that the recurrence can be converted into the closed-form expression:
\begin{align*}
	\vc e_n &= M \vc e_{n-1} + \tinc \vc F(x_{n+1})  \\ 
	&= \dots= M^k \, \vc e_0 + \tinc \sum_{i=0}^{n-1} M^{n-1-i} \, \vc F(x_i)	.	
\end{align*}	
For vector norm $\norm{\cdot}$, let $\matnorm{\cdot}$ denote a matrix norm such that $\norm{Mv} \leq \matnorm{M} \norm{v}$ for all suitable matrices $M$ and vectors $v$.
We desire to bound the  norm of the error. To this end, we leverage that the norms are sub-additive and sub-multiplicative to deduce: 
\begin{align*}
	\norm{\vc e_n} &\leq   \matnorm{M^{n}} \, \norm{\vc e_0} + \tinc \sum_{i=0}^{n-1}  \matnorm{M^{n-1-i}} \, \norm{\vc F(x_i)}		\\
	&\leq \matnorm{M^{n}} \, \norm{\vc e_0} + \tinc \maxerrn_n	 \sum_{i=0}^{n-1}  \matnorm{M^{n-1-i}} =: \varrho[n]
\end{align*}
where  $\maxerrn_n$ is chosen such that we can guarantee that $\maxerrn_n	\geq \max_{i=1,...,n-1} \norm{F(x_i)}$.
For instance, we could choose $\maxerrn_n := \sup_{s \in \iaspace_n} \norm{F(s)} $
	where $\iaspace_n = \bigcup_{t< k} \{x \in \iaspace | \norm{x - \xi[t]} \leq \varrho[t] \}$ is the union of the possible states around the reference trajectory $\xi[\cdot]$ at previous time steps.
\begin{remark}	[Bounded error innovations] 
We assume there exists a maximum model error norm $\maxerrn$, i.e. $\forall i: \norm{F_i(x_i)} \leq \maxerrn$ for some bound $\maxerrn \in \Real$.
This is a realistic assumption in any physically plausible systems where the drift forces $a(x)$ are inevitably bounded. Since also, for any finite data, our KI learner has a bounded prediction function, the discrepancy given by $F$ is bounded as well. 
\label{rem:bnd_err_innovations}
\end{remark}

Given the boundedness of $F$ we have: \vspace{-1em}
\begin{equation}
	\norm{\vc e_n} \leq   \matnorm{M^{n}} \, \norm{\vc e_0} + \tinc \maxerrn \sum_{i=0}^{n-1}  \matnorm{M^{n-1-i}}.
\end{equation}

The right-hand side is convergent provided the gains $K_1, K_2$ are chosen such that $M$ is stable, i.e. $\specrad(M) <1$, and provided $\maxerrn_n$ is is bounded (see e.g. \cite{calliess2014_thesis}). 

Whether or not $M$ is stable, in low dimensions, the sums can be computed offline and in advance. This is of great benefit if 
 the controller that is building on the error bounds is utilising optimisation-based control with a finite time-horizon.   

To obtain a (conservative) closed-form bound on the error norms \cite{calliess2014_thesis}
contains a derivation and discussion of the following result:

\begin{thm}\label{thm:normboundsboundeddisturbmainbody}
Let $(F_n)_{k \in \nat_0}$ be a norm-bounded sequence of vectors with $\maxerrn_n :=\max_{i \in \{0,\ldots,n-1\}} \norm{F_i(x_i)} \leq \maxerrn \in \Real$. 
For error sequence $(e_n)_{n \in \nat_0}$ defined by the linear recurrence 
	$e_{n+1}  = M e_n + \tinc F_n(x_n) \,\,\,\, (k \in \nat_0)$, we can define the following bounds:
	
	\begin{enumerate}
	\item $\norm{\vc e_n} \leq \matnorm{M^{n}} \, \norm{\vc e_0} + \tinc \maxerrn_{n} \sum_{i=0}^{n-1}  \matnorm{M^{i}}$. If $\specrad(M) <1$ and $\exists \maxerrn$ $:\maxerrn\geq \maxerrn_{n-1} \geq 0, \forall n$, the right-hand side converges as $n \to \infty$.
	\item Let $k_0 \in \nat, k_0 >1$ such that $\matnorm{M^n} <1, \forall n \geq k_0$, let $\varphi := \matnorm{M^{k_0}} <1$ and let $\delta_n := \floor{n/k_0}$. If $r:=\specrad(M) < 1$, for $n > k_0$,  we also have:
	\begin{align*} \norm{\vc e_n} &\leq c \, \varphi^{\delta_n} \, \norm{\vc e_0} + \tinc \maxerrn_{n} \Bigl(\sum_{j=0}^{n_0-1} \matnorm{M^j} + c \, k_0 \frac{\varphi-\varphi^{ \floor{\frac{n}{k_0}}+1 }}{1-\varphi} \Bigr)\\
	& \stackrel{n \to \infty}{\longrightarrow} C \leq \tinc \maxerrn \sum_{j=0}^{n_0-1} \matnorm{M^j} 
	+ \frac{\tinc \maxerrn\,  c\, k_0 \,\varphi}{1-\varphi} 
	\end{align*} 
	for some constants $C,c \in \Real$.
	Here, possible choices for $c \in \Real$ are: 
	
(i) $c= \max\{\matnorm{M^i} | i=1,\ldots,k_0-1 \}$ 

or (ii) $c = \frac{1}{(d-1)!} \Bigl(\frac{1 - d}{\log r}\Bigr)^{d-1}	\matnorm{M}^{d-1}\, \, \, r^{\frac{1 - d}{\log r}-d+1}  $.
%
Since $\matnorm{M} \neq 1$, one can also choose (iii) $c:= \matnorm{M}^{n_0}$. %In this case, $\matnorm{M^k} \leq  \, \matnorm{M}^{n_0}^{(k \text{ div } k_0) +1} $.

\item If $\matnorm{M} \neq 1$, we have: \newline
$\norm{\vc e_n} \leq \matnorm{M}^{n} \, \norm{\vc e_0} + \tinc \maxerrn_{n}   \frac{1-\matnorm{M}^n}{1-\matnorm{M}}. $
	\end{enumerate}
%\begin{proof}
%The theorem is proven in \cite{calliess2014_thesis}.
%
%\end{proof}	
\end{thm}


\subsection{Global asymptotic stability of KI-MRAC in the online learning setting}
In this subsection, we lift the assumption that the available sample is static. Instead we assume that at time step $k+1$ we get to see an additional sample of the uncertain drift at the state visited in the previous time step $k$. 
That is, the predictor $\predf_{n+1}(\cdot)$ is based on $\data_{n+1} = \data_n \cup \{ \bigl(\state_n, f(\state_n) \bigr)\}, \forall n $.
Therefore, we also need to index the innovations vector field by time. That is, $F_n := f - \predfn$ denotes the prediction error function (or innovation) due to the data available at time $n \in \nat$. 
As pointed out in Rem. \ref{rem:bnd_err_innovations}, the error innovations $F_n$ are assumed to be bounded for all finite sample sets $\data_n$. 

Above, we have seen that any continuous function can be approximated by some H\"older continuous LACKI predictor up to an arbitrarily small error. 
For convenience, we will establish the following definition:

\begin{defn}
We say that a continuous function $f$ is $L^*-p$-H\"older \emph{up to} error $\bar E_h \in \Real$ on domain $\inspace$ iff there is an $L^*-p$-H\"older function $\phi \in \hoelset {L^*} { } p$ and a function $\psi$ such that $\forall x: f(x) = \phi(x)+\psi(x), \, \sup_{x \in \inspace} \metric_\outspace\bigl(0,\psi(x)\bigr) \leq \bar E_h$.
\end{defn}


%\begin{lem} Assume the target $f$ is bounded on sub-domain $I \subseteq \inspace$ and that the data increases between time steps, i.e. $\data_n \subseteq \data_{n+1}, \forall n \in \nat_0$. Then we have:
%\[\forall x \in I \subset \inspace : \norm{F_n (x)}_\infty \geq \norm{F_{n+1}(x)}_\infty.\]
%\jcom{true? perhaps not needed anyway..}
%\end{lem}
 
%Above theorems consider the case where the data becomes dense in the domain and we are interested in the worst-case prediction error. Relevant to our online-learning and control setting below is the situation where we are only interested in the error on a specific trajectory of inputs. If the prediction error on such a trajectory vanishes then we will be able to establish stability of the trajectory.
%In preparation of these considerations, we will establish the following facts:

%
%\begin{lem}
%Assume we are given a trajectory $\seq{x_n}{n \in \nat}$ of inputs that is bounded, i.e. where 
%$\metric_\inspace(x_n,0) \leq \beta$ for some $\beta \in \Real_+$ and all $n \in \nat$.
%Define $G_n = \{ x_i | i \in \nat, i < n\}$.
%
%Then we have \[ \dist(G_n,x_n) = \min\{\metric_\inspace(g,x_n) | \, g \in G_n\} \stackrel{n \to \infty}{\longrightarrow} 0.\]
%\begin{proof}
%The intuition of the following proof is that if the distances were not to converge there would be an infinite number of disjoint balls around the input points that summed up to infinite volume. This however, would be a contradiction to the presupposed boundedness of the sequence.
%We formalise this intuition as follows:
%For contradiction assume there is an $\epsilon >0$ such that for all $n \in \nat$ there exists an $N_n \geq n $ such that 
%\begin{equation}
%\dist(x_{N_n}, G_{N_n}) > \epsilon.
%\end{equation} 
%Hence, by definition of $G_{N_n} =\{ x_i | i < N_n\} $,
%\eqn{eq:i34kjjk3}{\forall i < N_n : \metric_\inspace(x_{N_n},x_i) > \epsilon.}
%
%Let $C_n := \bigcup_{i < n} \ball{\frac \epsilon 2}{x_i} $ be the union of all $\frac \epsilon 2$-balls around each point in $G_n$ and define $\bar I = \bigcup_{n \in \nat} C_n$.
%By definition, each $x_n$ is contained in $\bar I$.
%Since sequence $(x_n)_{n \in \nat}$ is bounded, $\bar I $ has a finite volume relative to some positive, shift-invariant measure $\mu$. I.e. $\mu(\bar I) < \infty$. Furthermore, $\mu(C_n) \leq \sum_{i <n} \mu(B_i) \leq \mu(\bar I)< \infty$ where $B_i := \ball{\frac \epsilon 2}{x_i}$. Since $M:=\mu(B_1) = \mu(B_n)\forall n \in \nat$ we have $\mu(C_n) \leq (n-1) M$.
%Define $m:= \ceil{\frac{\mu(\bar I)}{M}}$.
%Now, choose $n > m+1$. For $i=1,...,n$ $\exists N_i \geq i \forall j \leq N_i: \metric_\inspace(x_{N_i},x_j) > \epsilon$.  Define a permutation $\pi$ such that $\pi(N_1) \leq \ldots \leq \pi(N_n)$. Hence, we have $\metric_\inspace(x_{\pi(N_i)},x_{\pi(N_j)}) > \epsilon , \forall i,j =1,...,n, i < j$. Thus, $B_{\pi(N_i)} \cap B_{\pi(N_j)} = \emptyset , \forall i,j =1,...,n, i \neq j$. Hence, $\mu(\bar I) \geq \mu(C_{\pi(N_n)+1}) = \mu(C_{\pi(N_1)}) + \sum_{i=1}^n \mu(B_{\pi(N_i)}) = \mu(C_{\pi(N_1)}) +  n M > \mu(C_{\pi(N_1)}) + (m+1) M >\mu(C_{\pi(N_1)}) + \mu(\bar I)$ by definition of $m= \ceil{\frac{\mu(\bar I)}{M}}$. But since $\mu(C_{\pi(N_1)}) \geq 0$, we conclude the false statement $\mu(\bar I) > \mu(\bar I)$.
%\end{proof}
%\label{lem:bndseq_entails_distgridvanish}
%\end{lem} 
%
%
%\begin{lem}
%As before, let the observational noise level be bounded by $\obserr$ and assume the output- space pseudo-metric to be translation-invariant.
%Let target $f$ be $L^*-p$-H\"older up to error $E$. That is, there is an $L^*-p$-H\"older function $\psi$ and a function $\phi$ such that $\forall x: f(x) = \phi(x)+\psi(x), \, \metric_\outspace\bigl(0,\psi(x)\bigr) \leq E $.
%
%Assume we are given a trajectory $\seq{x_n}{n \in \nat}$ of inputs that is bounded, i.e. where 
%$\metric_\inspace(x_n,0) \leq \beta$ for some $\beta \in \Real_+$ and all $n \in \nat$.
%Furthermore, assume $\data_{n+1} = \data_n \cup \{ \bigl(x_n, \tilde f(x_n)\bigr) \}$ and thus, $G_n = \{ x_i | i \in \nat, i < n\}$.
%Then the prediction error on the sequence vanishes, i.e. \[\metric_\outspace\bigl(\predfn(x_n),f(x_n) \bigr) \stackrel{n \to \infty}{\longrightarrow} \metric_\outspace(0,\obserr) + 2E.\]
%\begin{proof}
%We have established that $\predfn$ is $L(n)-p$- H\"older with $L(n) \leq \bar L$ for some $\bar L \in \Real_{\geq 0}$.
%
%Since sequence $(x_n)$ is bounded, Lem. \ref{lem:bndseq_entails_distgridvanish} is applicable. 
%Let $\xi_n := \argmin_{g \in G_n} \metric_\inspace(x_n,g)$ denote the nearest neighbour of $x_n$ in $G_n = \{x_1,...,x_{n-1}\}$.
%Hence, $\forall n \in \nat: \metric_\inspace (x_n,\xi_n) \to 0$ as $n \to \infty$.
%
%We have:
%$\metric_\outspace\bigl(\predfn(x_n) , f(x_n)\bigr) \leq \metric_\outspace\bigl(\predfn(x_n) ,  f(\xi_n)  \bigr) + \metric_\outspace\bigl(f(\xi_n) , f(x_n)\bigr) \stackrel{(i)}{=}\metric_\outspace\bigl(\predfn(x_n) , \predfn(\xi_n) \bigr) +\metric_\outspace\bigl(0 , \obserr\bigr) + \metric_\outspace\bigl(f(\xi_n) , f(x_n)\bigr) \leq \metric_\outspace\bigl(\predfn(x_n) , \predfn(\xi_n) \bigr) +\metric_\outspace\bigl(0 , \obserr\bigr) + \metric_\outspace\bigl(\phi(\xi_n) , \phi(x_n)\bigr) + 2E \stackrel{(ii)}{\leq} 2 \max\{\bar L, L^* \} \metric_\inspace(x,\xi_n^x )^p + \metric_\outspace(0,\obserr) + 2E \longrightarrow \metric_\outspace(0,\obserr) + 2E$.
%where steps denoted by (i) follow from Lem. \ref{lem:bilinaddtransinvgroup} and (ii) follows from the assumed H\"older properties.
%\end{proof}
%\label{lem:vanisishingseqprederr_groups}
%\end{lem} 
 
 
\begin{lem} For some constant $L^*$ let the target $f$ be $L^*-p$-H\"older continuous up to level $\bar E_h$. 
%Furthermore, 
%assume that the observational error is zero, i.e. $\obserrbnd =0$.

If $F_n(x_n) = f(x_n) - \predfn(x_n)$ is bounded for all $ n \in \nat_0$ then we have 
\[\forall k \in \nat_0: \norm{F_n(\state_{n+k})}_\infty \stackrel{n \to \infty}{\longrightarrow} [0, \frac \hestthresh 2 +  \obserrbnd +2\bar E_h]. \] 


\begin{proof}
If sequence $\seq{F_n(x_n)}{n\in \nat} $ is bounded then by Thm. \ref{thm:normboundsboundeddisturbmainbody}, sequence \seq{x_n}{n\in \nat} is bounded. 
Furthermore,
by Lem. \ref{lem:LACKIsampleconsistency}, we know that our predictors $\predfn(\cdot)$ are sample-consistent up to $\frac \lambda 2$ and hence, $\norm{ F_n(x)}_\infty \leq \frac \hestthresh 2 +  \obserrbnd=: E_s ,\forall x \in \data_n=\{x_1,...,x_{n-1}\}$.

The rest follows directly from Lem. \ref{lem:vanisishingseqprederr_groups} assuming $\metric_\outspace (f,f') = \norm{f-f'}_\infty$.
\end{proof}
\label{lem:innovnorm_convergence}
\end{lem}


\begin{remark}
In particular, in case the target is H\"older continuous $(\bar E_h =0)$ and there is no observational error ($\obserrbnd =0$), parameterising our LACKI rule with $\hestthresh =0$ yields vanishing innovations, i.e. 
\[\forall k \in \nat_0: \norm{F_n(\state_{n+k})}_\infty \stackrel{n \to \infty}{\longrightarrow} 0. \]
\end{remark}

The next theorem gives a stability guarantee for the tracking error of our LACKI-MRAC controller:

\begin{thm}[Tracking error convergence]
Assume the preconditions and definitions of Lem. \ref{lem:innovnorm_convergence} hold.
If the initial error innovation function is bounded, i.e. if $\exists b \in \Real \forall x: \norm{F_0(x)} \leq b $, and, if $M$ is a stable matrix, i.e. if $\specrad(M) <1$, then the tracking error converges to the interval $[0,\frac \hestthresh 2 +  \obserrbnd + 2 \bar E_h]$. That is,
\[\norm{e_n} \stackrel{n \to \infty}{\longrightarrow} [0,\frac \hestthresh 2 +  \obserrbnd + 2 \bar E_h]. \]

\end{thm}
\begin{proof}
We show 
\begin{equation}
\forall \epsilon > 0 \exists N \in \nat \forall n\geq N: \norm{e_n} \leq \epsilon.
\end{equation}
Referring to (i) in Thm. \ref{thm:normboundsboundeddisturbmainbody} we see that for all $k \in \nat$ we have: 
\begin{equation}
\norm{\vc e_{n+k}} \leq \matnorm{M^{k}} \, \norm{\vc e_n} + \sigma \, \tinc \, Q_{n:n+k} 
\label{eq:errdyn320894}
\end{equation}
where $\sigma$ is some nonnegative number and $Q_{n:n+k}:=\max\{\norm{F_n(x_n)},\ldots,\norm{F_{k+n-1}(x_{k+n-1})} \}$.


Let $\epsilon >0$. By Lem. \ref{lem:innovnorm_convergence}, the sequence $\seq{\norm{F_n(x_n)}_\infty}{n \in \nat}$ converges to the interval $[0, \frac \hestthresh 2 +  \obserrbnd +2\bar E_h]$. Hence, we can infer the following two statements:


(I) There exists an index $n_0 \in \nat$ such that for all $k$: $Q_{n_0:n_0+k} \leq \frac{\epsilon}{2 \Delta \sigma}+\frac \hestthresh 2 +  \obserrbnd +2\bar E_h$.

On the other hand,
by Thm. \ref{thm:normboundsboundeddisturbmainbody}, boundedness of $F_0$ entails boundedness of the error dynamics. Thus, $\exists b \in \Real \forall n: \norm{e_n} \leq \beta$.
Knowing that the error dynamics are bounded by some $\beta \geq 0$ we see that $\matnorm{M^{k}} \, \norm{\vc e_n} \leq \matnorm{M^{k}} \beta \stackrel{k \to \infty}{\longrightarrow} 0$. Here, the convergence to zero follows from the assumption that $M$ is a stable matrix. Hence,
we have:

(II) For any $n$ we can choose $k_0(n) \in \nat$ such that for all $k \geq k_0(n): \matnorm{M^{k}} \, \norm{\vc e_n} \leq \frac\epsilon 2 $.
%
Combining (I) and (II) with Eq. \ref{eq:errdyn320894} allows us to conclude that for any $n \geq N:= n_0 +k_0(n_0)$  we have 
\begin{equation*}
\norm{\vc e_{n}} \leq \frac \epsilon 2 + \sigma \, \tinc \,\frac{\epsilon}{2 \Delta \sigma} +\frac \hestthresh 2 +  \obserrbnd +2\bar E_h= \epsilon +\frac \hestthresh 2 +  \obserrbnd +2\bar E_h. 
\label{eq:errdyn3208942}
\end{equation*}
Hence, we have shown that $\forall \epsilon >0 \exists N \geq 0 \forall n\geq 0: \norm{\vc e_{n}} \leq \epsilon +\frac \hestthresh 2 +  \obserrbnd +2\bar E_h$ which is the desired convergence result.

\end{proof}
Note, since the error converges to a bounded set the state will converge to the target trajectory. So, if the target trajectory is bounded, the continuity of the control law (as a function of state) implies that the control is bounded as well.

%
%
\begin{cor}
In the special case of error-free observations of a H\"older continuous target function, choosing a parameter $\hestthresh =0$ implies:
\[\norm{e_n} \stackrel{n \to \infty}{\longrightarrow} 0. \]
That is, the error dynamics are globally asymptotically stable with equilibrium point 0.
Furthermore, the control action sequence $\seq{u(x_n)}{n \in \nat}$ converges, provided the target trajectory $\seq{\xi_n}{n \in \nat}$ does and $\inspace$ is a Hilbert space.
\end{cor}

\begin{proof} The stability statement is an immediate consequence of the preceding theorem. 
Remember the control action at time $n$ is of the form $u_n := u(x_n) = -\predfn(x_n) - A x_n $ for some Matrix $A$. We show that $(u_n)$ is a Cauchy sequence, provided that the target sequence $\xi_n$ is. Due to the Hilbert space assumption this implies the desired convergence result.

So, let $\epsilon >0$. Since $(e_n),(\xi_n)$ converge the state sequence $(x_n)$ converges and all three are Cauchy sequence. Hence, there is $N$ such that for all $n,m >N$: $\norm{e_n-e_m}< \frac{\epsilon} {2 \matnorm{K}}$ and $\norm{x_n-x_m}< \frac{\epsilon} {2\bar L}$. 
Hence, by definition of the control law and H\"older continuity of the predictors with constant $\bar L$, for all $m,n >N:$ $\norm{u_n-u_m} \leq \matnorm{K} \norm{e_n -e_m} + \norm{\predfn(x_n) - \predf_m(x_m) } \leq \frac{\epsilon}{2} + \bar L \norm{x_n -x_m} \leq \epsilon $.

\end{proof}




%======================= CONCLUSIONS ===================
\section{Conclusions}
\label{sec:concl}
In this paper, we have introduced \emph{Lazily Adapted Constant Kinky Inference (LACKI)} as method for nonparametric machine learning. Our method was built on the framework of \emph{Kinky Inference} (which in turn is a generalisation of well-known approaches such as \emph{Lipschitz Interpolation} \cite{Sukharev1978,Zabinsky2003,Beliakov2006} and \emph{Nonlinear Set Membership (NSM)}  methods \cite{Milanese2004}). 
 Our approach inherits the numerical simplicity of these methods. On top of this, it can deal with bounded additive observational errors and does not require a priori knowledge about a H\"older constant of the underlying target function--  instead it estimates the constant online from the data. This of course is of great practical interest since this endows LACKI with superior black-box learning capabilities while still allowing us to give theoretical guarantees on learning and control success.
 
To avoid the need to specify the H\"older constant, LACKI adapts the parameter $L(n)$ to reflect a modification of the empirical estimate of the H\"older constant of the underlying target function. 
The adapted parameters were carefully defined to be bounded even if the target function is not H\"older continuous and the data is subject to bounded observational uncertainty.
This allowed us to establish several theoretical guarantees of worst-case consistency. That is, we provided asymptotic guarantees on the ability to learn any H\"older (and non-H\"older) continuous target function as well as convergence rates. Our derivations focussed on worst-case prediction error bounds.

Future work will investigate in how far the bounds can be tightened further, albeit we do not expect that worst-case guarantees could be given that avoid the curse of dimensionality without imposing more confining assumptions on the target functions and the nature of the observational uncertainties. 
However, if we were to shift our attention away from worst-case error analysis under general (possibly systematic) observational uncertainties towards standard mean-square risk analysis and assumptions prevalent in probability theory, we believe less conservative consistency results could be established. 
To this end, we believe it is possible to modify our proofs to translate mean-square consistency derivations for nearest-neighbor regression methods (e.g. as discussed in \cite{Gyoerfi2002}) to our LACKI approach. This might provide a theoretical underpinning for the smoothing properties of our approach observed in the presence of i.i.d. stochastic noise (refer to Fig. \ref{fig:LACKInoise} and Fig. \ref{fig:LACKInoise2}). 

%For instance, we proved that our LACKI rule can learn any H\"older function (up to an error proportional to the noise in the data). Moreover, we showed that LACKI can also learn continuous but non-H\"older target functions up to a worst-case approximation error that converges to the level of intrinsic observational  error in the data-generating process, plus a value coinciding with parameter $\hestthresh \geq 0$. Since 
%$\hestthresh$ is a free parameter that can be set to arbitrarily small values, convergence can be reduced down to the observational noise-level as much as desired. However, while we have not investigated this theoretically, we would expect that reducing the magnitude of $\hestthresh$ will not only cause the uncertainty bounds to grow (or even grow unbounded if the target is not H\"older or there is noise in the data) but also to slow down the worst-case convergence rate of the LACKI predictors. 
%Investigating such convergence rates is a topic we would find interesting to investigate in the context of future work. 


Our learning-theoretic considerations were supplemented by an application of LACKI to 
online learning-based model-reference adaptive control. In a simulated aircraft control problem with nonlinear model uncertainty, we compared our LACKI-based controller against other learning-based methods that were recently proposed in the control literature. Across a range of performance metrics and randomised problem instances, LACKI-MRAC demonstrated consistent and robust performance and outperformed its competitors on the majority of randomised test cases.

For discrete-time systems with additive, bounded, nonlinear uncertainty, we provided theoretical guarantees on the tracking success of our LACKI-MRAC controller. For the online learning case where the LACKI learner was assumed to be continuously updated with the most recently visited state, we proved tracking success up to a term dependent on the observational error. 


In future work, we would like to apply our LACKI learning method to more challenging control tasks that require planning.
To this end, it might be beneficial to link our results to recent work on NSM-based model-predictive control (e.g. \cite{Canale2014}). We believe that the worst-case analysis we focus on in this work is key to establish the necessary links to results in robust MPC.

In this work, we have assumed that the observational noise was bounded; we have addressed the issue of unbounded noise in a companion paper \cite{POKIdraft2016} where the H\"older constant parameter $L(n)$ is found as the minimiser of a prediction loss estimator.  We also believe that these estimators could be used to estimate the noise bound if these are unknown a priori.

A final suggestion for future work pertains to the question of speeding up prediction time. In the context of Lipschitz interpolation, Beliakov \cite{Beliakov2006} proposed a way to organise the sample in a tree-structure (in lieu to KD-trees utilised in nearest-neighbor search) in order to reduce the linear prediction time to be logarithmic in the sample size (albeit we would think this would hold in expectation only). While applicable to our LACKI approach in the batch learning setting, it is not clear to us how this tree structure could be efficiently updated in an online learning setting. Future work could explore avenues of connecting his idea of appealing to notions of generalised nearest-neighbor search to existing efficient approaches of \emph{online} nearest neighbor search.

%In the present paper, our theoretical considerations relied on worst-case analysis. On the one hand this is more appropriate than traditional stochastic risk analysis of nonparametric regression if the learning method is to be used in robust control tasks where common 
%distributional assumptions will not hold and guarantees have to be valid even in extreme events of zero measure. 
%
%On the other hand, our learning guarantees are unable to guarantee the reduction of the approximation error of the predictor to the target below the level of observational noise. In the presence of systematic observational errors this is unavoidable. However, things might change if the observational errors are stochastic.
%For example, in Fig. \ref{fig:LACKInoise} and Fig. \ref{fig:LACKInoise2} we observed that our predictors were capable of smoothing out stochastic noise resulting in predictors that exhibited close alignment with the true target. Investigating the extent to which distributional assumptions about the noise can be translated to probabilistic convergence results to the target will also be a subject of future investigation. 
%
%======================= CONCLUSIONS END ===============
%\section{Conclusions}
\label{sec:conclusions}

In this paper, we apply shared-workload techniques at the \sql level for
improving the throughput of \qaasl systems without incurring in additional
query execution costs. Our approach is based on query rewriting for grouping
multiple queries together into a single query to be executed in one go. This
results in a significant reduction of the aggregated data access done by the
shared execution compared to executing queries independently.

%execution times and costs of the shared scan operator when
%varying query selectivity and predicate evaluation. We observed that for
%\athena, although the cost only depends on the amount of data read, it is
%conditioned to its ability to use its statistics about the data. In some cases
%a wrong query execution plan leads to higher query execution costs, which the
%end-user has to pay. 

%\bigquery's minimum query execution cost is determined by
%the input size of a query.  However, the query cost can increase depending not
%just in the amount of computation it requires, but in the mix of resources the
%query requires.  

We presented a cost and runtime evaluation of the shared operator driving data access costs. 
Our experimental study using the TPC-H benchmark confirmed the benefits of our
query rewrite approach. Using a shared execution approach reduces significantly
the execution costs. For \athena, we are able to make it 107x cheaper and for
\bigquery, 16x cheaper taking into account Query 10 which we cannot execute,
but 128x if it is not taken into account. Moreover, when having queries that do
not share their entire execution plan, i.e., using a single global plan, we
demonstrated that it is possible to improve throughput and obtain a 10x cost
reduction in \bigquery.

%We followed the TPC systems pricing guideline for
%computing how expensive is to have a TPC-H workload working on the evaluated
%\qaasl systems. The result is that even though we are able to reduce overall
%costs a TPC-H workload in 15x for \bigquery (128x excluding query 10 which we
%could not optimize) and in 107x for \athena, the overall price is at least 10x
%more expensive than the cheapest system price published by the TPC.

There are multiple ways to extend our work. The first is
to implement a full \sql-to-\sql translation layer to encapsulate the proposed
per-operator rewrites.  Another one is to incorporate the initial work on
building a cost-based optimizer for shared execution
\cite{Giannikis:2014:SWO:2732279.2732280} as an external component for \qaasl
systems.  Moreover, incorporating different lines of work (e.g., adding
provenance computation \cite{GA09} capabilities) also based on query
rewriting is part of our future work to enhance our system.


\section{Acknowldegements}
I am grateful for useful feedback from Jan Maciejowski, Carl Rasmussen and Stephen Roberts. I would also like to thank Girish Chowdhary and Hassan Kingravi who generously supplied me with their code that allowed me to most closely reproduce their work and use it for my comparisons in Sec. \ref{sec:KIMRAC}. I also gratefully acknowledge funding via the AIS project, NMZR/031 RG64733.
%\newpage
\bibliographystyle{plain}
\bibliography{content/lit}
\appendix 
\section{Supplementaries}
%\section{Preliminaries}
\subsection{H\"older and Lipschitz continuity for inference}%\label{sec:HoeldNLipfacts}
\label{sec:Hoelder_brief}
\subsubsection{Introduction and related work}



H\"older continuous functions are uniformly continuous functions that may exhibit infinitely many points of non-differentiability and yet are sufficiently regular to facilitate inference. That is, they have properties that make it possible to make assertions of global properties on the basis of a finite function sample. 

H\"older continuity is a generalisation of Lipschitz continuity.  Lipschitz properties are widely used in applied mathematics to establish error bounds and, among many other, find application in optimisation \cite{Shubert:72,direct:93} and quadrature \cite{Baran2008,curbera1998,dereich2006} and are a key property to establish convergence properties of approximation rules in (stochastic) differential equations \cite{kloedenandplaten1992,Gardiner2009}. 
Furthermore, most machine learning methods for function interpolation seem to impose smoothness (and thereby, H\"older continuity) on the function. For instance, with our Lem. \ref{lem:Hoeldarithmetic} derived below, it would be possible to show that any finite \textit{radial basis function neural network} \cite{Broomhead1988} with a smooth basis function is H\"older continuous on a compact domain. Or, a \textit{Gaussian process} with a smooth covariance function also has a smooth mean function and a.s. smooth sample paths  \cite{GPbook:2006,grimmetbook2001}. Therefore, posterior inference over functions on compact support made with such a Gaussian process on the basis of a finite sample is H\"older continuous. 


Recently, we have become aware of related work published in mathematical and operations research journals \cite{Cooper2006,Cooper1995,Zabinsky2003,Beliakov2006,Beliakovsmoothing2007}. For instance, Zabinsky et. al. \cite{Zabinsky2003} consider the problem of estimating a one-dimensional Lipschitz function (with respect to the canonical norm-induced metric). Similar to the analysis we employ to establish our guarantees, they use a pair of bounding functions and make predictions by taking the average of these functions. While we have developed our kinky inference rule independently, it can be seen as a generalisation of their approach. Our method provides extensions to H\"older continuous multi-output functions over general, multi-dimensional (pseudo-) metric spaces, can cope with with erroneous observations and inputs, can fold in additional knowledge about boundedness, learn parameters from data and provides different guarantees such as (uniform) convergence of the prediction uncertainty. 
As part of the analysis of our method, we construct delimiting functions we refer to as \textit{ceiling} and \textit{floor} functions. The construction of similar functions is a recurring theme that, in the standard Lipschitz context, can be found in global  optimisation \cite{Shubert:72}, quadrature \cite{Baran2008}, interpolation \cite{Beliakov2006,Beliakovsmoothing2007}, as well as in the analysis of linear splines for function estimation \cite{Cooper1995}. Cooper \cite{Cooper2006,Cooper1995} utilises such upper and lower bound functions in a multi-dimensional setting to derive probabilistic PAC-type error bounds \cite{Valiant1984} for a linear interpolation rule. He assumes the data is sampled uniformly at random on a hypercube domain. This precludes the application of his results to much of our control applications where the data normally is collected along continuous trajectories visited by a controlled plant. Our inference rule is different from his and our guarantees do not rely on distributional assumptions. This of course is important in control settings where the common assumption that the input data was drawn independently from a fixed distribution typically is not met.
%As a special case of our kinky inference framework, we consider a case, called \textit{kNN-KI}. Here, the inference over function values is based on up to (approximate) k nearest neighbours aiming to reduce computational prediction effort to log-time. Approaches to this end include maintaining a data structure for fast nearest neighbour search \cite{Bentleykdtree:1975} or utilising locally sensitive hash functions \cite{RusselNorvigbook2009}. 
In this thread of works, perhaps the work that is most closely related to ours is the function interpolation method of Beliakov \cite{Beliakov2006} that is a special case of a kinky inference rule: For a single-output function that is Lipschitz with respect to a special input space norm and where the data is error-free, the authour provides an algorithm that promises logarithmic prediction time. Unfortunately, many of his assumptions are unrealistic in a control setting. And, the improved prediction time is achieved by constructing a data structure from batch data which precludes its use in an online learning setting. However, future work might explore in how far his ideas can be 
converted into an online learning rule. Furthermore, in learning situations where Beliakov's interpolation method is applicable, our theoretical results extend to his work. For instance, our results show how Lipschitz constant estimation can be harnessed to render his approach a universal approximator.

Other work of relevance can be found in analysis. For instance, Miculescu \cite{Miculescu2000} presents work proving that any continuous function on a metric space is a uniform limit of a sequence of \emph{locally} Lipschitz functions and also mentions that the stronger statement, that every function is a limit of a sequence of globally Lipschitz functions, is not true in general. However, he cites earlier work \cite{Georganopoulos1967} that does show that every real-valued continuous function on \emph{compact} domain is a uniform limit of a sequence of Lipschitz functions. In some sense, our work develops a related statement as a by-product. 
From our convergence guarantee of the LACKI rule, we have derived constructive method for showing 
that any continuous function on compact domain is the uniform limit of a sequence of H\"older functions up to an arbitrarily small error.

Finally, in the context of control, Milanese and Novara \cite{Milanese2004} considered NSM methods for interpolation. For a fixed Lipschitz constant, their prediction rule can be
seen as a special case of ours without the $\lbf$ and $\ubf$ parameters and with special choices of metrics. Similar to us, they do consider the problem of estimating the Lipschitz constant from the data and consider bounded noise. However, they obtain the Lipschitz constant estimate via the maximum partial derivative of an arbitrarily chosen fitted parametric model of a bounded input set. And, they give no guarantees on the quality of the resulting estimator that is fitted to the data like this nor do they discuss the impact of the choice of parametric model or the fitting method on the quality of the estimator.
 
%\jcom{Need to read \cite{Georganopoulos1967}}
%\jcom{Need to reassess the following in the light of the newly found control paper:}
%None of the aforementioned works seems to consider interval-bounded input uncertainty and observational errors, fold in additional knowledge such as boundedness or considers inference over multi-output functions as we do. Most importantly the combination with our online constant estimation rule, leading to our LACKI inference rule, as well as our universal approximation guarantees are novel.
%Finally, we are not aware of any work that employs H\"older-based methods in the context of model-reference adaptive control. 


%In particular, we will look into the inference problems of bounded optimisation and quadrature. That is, based on the knowledge of a sample of a H\"older continuous function we will derive upper and lower bounds on its maximum, minimum and definite Riemann integral. In addition we will touch upon adaptive algorithms that can utilise knowledge of the function's H\"older properties to guide exploration. That is, since the regularity of the function allows us to provide interval bounds on the uncertainty of the function, the magnitude of the uncertainty can be a guide to identify regions that are to be sampled next in order to maximise the shrinkage of the provided error bounds. 
%\jcom{The latter is more the 1-dim case, which I may or may not include}\\

\subsubsection{Basic facts and derivations}
In preparation of subsequent parts of the work that take advantage of H\"older properties this section will proceed to establish essential prerequisites.
The remainder of this section is structured as follows: Firstly, we will go over basic definitions and engage in some preliminary derivations that will be of importance throughout this work.
While we do not claim novelty on any of the results we provide proofs for in this section, we had not found them in the literature and hence, had to derive them on our own.

Firstly, we commence with introducing the notions of (pseudo-) metric spaces.

\begin{defn}[(Pseudo-) metrics]
Let $\inspace$ be a set. A mapping $\metric_\inspace: \inspace^2 \to \Real$ is called a \emph{pseudo-metric} if it positive ($\forall x,x' \in \inspace: \metric_\inspace(x,x') \geq 0$) and satisfies the triangle inequality ($\forall x,x',x'' \in \inspace: \metric(x,x') \leq \metric_\inspace(x,x'') + \metric (x'',x')$). If furthermore the pseudo-metric $\metric$ is definite $(i.e. \forall x,x' \in \inspace_\inspace: \metric_\inspace(x,x') =0 \Leftrightarrow x=x')$ then 
the mapping $\metric$ is called a \emph{metric}. The set $\inspace$ endowed with a (pseudo-) metric $\metric_\inspace: \inspace^2 \to \Real$ or the pair $(\inspace, \metric_\inspace)$ are called \emph{(pseudo-) metric space}.
 \end{defn}

\begin{defn} 
Let $(\inspace ,\metric_\inspace ), (\outspace , \metric_\outspace )$ be two (pseudo-) metric spaces and 
$I \subset \inspace$ be an open set. A function $f: \inspace \to \outspace $ is called (L-p-) \emph{H\"older} 
(continuous) on $I \subset \inspace$ if there exists a \emph{(H\"older) constant} $L \geq 0$ and \emph{(H\"older) 
exponent} $p\geq 0$ such that 
\[\forall x,x' \in I : \metric_\outspace \bigl(f(x),f(x')\bigr) \leq L \, \bigl( \metric_\inspace (x,x') \bigr)^p. \]
We denote the space of all L-p- H\"older functions by $\hoelset{L}{}{p}$.
\end{defn}

H\"older functions are known to be uniformly continuous. 
A special case of importance is the class of $L$-\textit{Lipschitz} functions. These are H\"older continuous 
functions with exponent $p=1$. In this context, coefficient $L$ is referred to as\textit{ Lipschitz constant} or \textit{Lipschitz number}.

\begin{ex}[Square root function]\label{ex:sqrtfctHoelder}
As an example of a H\"older function that is not Lipschitz we can consider $x \mapsto \sqrt x$ on domain $I = [0+\epsilon,c]$ where 
$c >\epsilon \geq 0 $. For $\epsilon >0 $ the function is Lipschitz with $L = \sup_{x \in I} \frac{1} {2 \sqrt{x}}$. We can see that the 
coefficient grows infinitely large as $\epsilon \to 0$. By contrast, the function is H\"older continuous 
with H\"older coefficient $L=1$ and exponent $p=\frac 1 2 $ for any bounded $I \subset \Real$.
We can see this as follows: Let $\epsilon =0,$ $x,y \in I$ and, without loss of generality,  $y \geq x$. Let $\xi := \sqrt{x}, \gamma := \sqrt{y}$ and thus, $\gamma \geq \xi$. We have:
$\xi \leq \gamma $ $\Leftrightarrow 2 \xi^2 \leq 2\xi\gamma$ $\Leftrightarrow \gamma^2 - 2 \xi\gamma + \xi^2  \leq \gamma^2 - \xi^2$ $\Leftrightarrow (\gamma-\xi)^2  \leq \gamma^2-\xi^2$ $\Leftrightarrow \abs{\gamma-\xi}^2  \leq \abs{\gamma^2-\xi^2}$
$\Leftrightarrow \abs{\gamma-\xi}  \leq \sqrt{\abs{\gamma^2-\xi^2}}$  $\Leftrightarrow \abs{\sqrt{x}-\sqrt{y}} \leq \abs{y-x}^{\frac{1}{2}}$. Since, $x,y$ were chosen arbitrarily, we have shown H\"older continuity as desired.
\end{ex}

Most commonly, one considers H\"older continuity for the special case of the standard metric induced by a norm, i.e.  $\metric(x,x') = \norm{x-x'}$.
For a function $f: \inspace \to \outspace$, the H\"older condition becomes:
\[\forall x,x' \in I : \norm{f(x)-f(x')}_\outspace \leq L \, \norm{x-x'}_\inspace^p. \]

Similarly, we can consider H\"older continuity for each output component: 

\begin{defn} \label{def:outputwisehoelder}
Let $\outspace \subseteq \Real^m $ and $\inspace$ be a space endowed with a metric (or indeed a semi-metric) $\metric_\inspace$. Then, the function $f: \inspace \to \outspace$ is output-component-wise H\"older continuous with exponent $p$ and constant $L \in \Real^m_{\geq 0}$ if $f \in \hoelset L { } p$ where
$\hoelset L {\metric_\inspace} p:= \bigl\{ \phi: \inspace \to \outspace \,\ | \, \forall j \in \{1,...,m \}, \forall x,x' \in \inspace: \abs{\phi_j(x) - \phi_j(x')} \leq L_j \metric^p_\inspace(x,x') \bigr\}$
is the set of all functions whose component functions are H\"older continuous with respect to input space metric $\metric_\inspace$ and an output space metric that is induced by the canonical norm $\metric_{\outspace} (x,x')= \abs{x-x'}$.  
\end{defn}

\begin{remark}[Best H\"older constant]
Note for $p \in (0,1], 0 \leq L_1 \leq L_2$ we have $\hoelset {L_1} {\metric_\inspace} p \subseteq \hoelset {L_2} { } p$. The smallest $L^* \geq 0$ such that function if is $L^*-p-$ H\"older, $f \in \hoelset {L^*} {\metric_\inspace} p$, is called the \emph{best} H\"older constant of $f$.
\end{remark}

Generally, it is obviously true that $\hoelset{L}{}{p} \subseteq \hoelset{L'}{}{p}$ for $L' \geq L$.
With regard to the H\"older exponent, we will now show that smaller exponents are less restrictive than larger ones. 




\begin{lem} \label{lem:hoeldexpprop2}
Let $\inspace$ be the input space (not necessarily bounded). For some $p \in (0,1], L \geq 0$ assume that  and $f:\inspace \to \outspace$ is locally $L-p$-H\"older continuous.
Then we have: (i) for any $ q \in (0,p]$, f is also locally $L-q$-H\"older. 
(ii) If $f:\inspace \to \outspace$ is bounded with $\sup_{x,x' \in \inspace} \metric_\outspace (f(x),f(x')) \leq B \in \Real$ and globally $L-p$ H\"older then $f$ is globally $L^*-q$-H\"older, where $L^* := \max\{L,B \}$ and $q \in (0,p]$. In particular, on compact support, Lipschitz continuity entails \emph{H\"older} continuity for any H\"older exponent $p \in [0,1)$.
\begin{proof}
(i) Let $p \in (0,1], f \in \hoelset L { } p $ and $p = q+r, r \in [0,1)$. Let $\xi \in \inspace$ and $I$ denote the intersection of the domain with an $\epsilon$-ball around $\xi$ such that $f$ satisfies the H\"older condition on $I$ and such that $\sup_{x,x' \in I} \metric_\inspace(x,x') \leq 1$.  For all $x,x' \in I$ we have $\metric_\outspace (f(x),f(x')) \leq L\metric_\inspace^p (x,x') = L \metric_\inspace^q(x,x') \metric_\inspace^r(x,x') \leq L \metric_\inspace^q(x,x')$ where the last inequality holds since $r \in [0,1)$ and $\sup_{x,x' \in I} \metric_\inspace(x,x') \leq 1$.

(ii) Let $x,x' \in \inspace$. If $\metric_\inspace(x,x') \leq 1$ we can show $\metric_\outspace (f(x),f(x'))  \leq L \metric_\inspace^q(x,x')$ following through the same sequence of inequalities as above in the proof of (i). Now, let $\metric(x',x) >1$. We have $\metric_\outspace (f(x),f(x')) \leq B \leq B  \metric_\inspace(x,x')^q$.
\end{proof}
\end{lem}

%Note for compact domains local H\"older continuity implies global H\"older continuity with the same parameters. Therefore Lem. \ref{lem:hoeldexpprop1} would also be implied 
%by Lem. \ref{lem:hoeldexpprop2}.(i) in cases where $\inspace$ is compact.



%We conclude this section by the following theorem stating that any concatenation of H\"older continuous functions is H\"older continuous:
%
%\begin{thm} \label{thm:concathoelder}
%Let $(\statespace, \metric)$ be a metric space and $f,g : \statespace \to \statespace$ be two H\"older continuous mappings with H\"older constants $L_f, L_g$ and H\"older exponents $p_f,p_g$, respectively.
%Then, the concatenation $h=f \circ g: \statespace \to \statespace $ is also H\"older continuous with H\"older constant $L_h:= L_f L_g^{p_f}$ and exponent $p_h:=p_g \, p_f$.
%That is, 
%\[\forall \state,\state' \in \statespace: \metric\bigl(h(\state),h(\state')\bigr) \leq L_h \bigl(\metric(\state,\state')\bigr)^{p_h}.\]
%\begin{proof}
%%\begin{align}
%$\metric\bigl(f \circ g(\state),f\circ g(\state')\bigr) \leq L_f  \Bigl(\metric(g(\state),g(\state'))\Bigr)^{p_f}$
%$\leq L_f  \Bigl(L_g \metric(\state,\state')^{p_g}\Bigr)^{p_f}$ $= L_f L_g^{p_f}   \Bigl(\metric(\state,\state')\Bigr)^{p_g\, p_f} $ where in the first step we were using Hoelder-continuity of $f$ and in the second, we were using H\"older continuity of $g$ combined with the fact that $(\cdot)^{p_f}$ is a monotonically increasing  function. 
%
%\end{proof}
%\end{thm} 
%In conjunction with H\"older properties of the square root function established in Ex. \ref{ex:sqrtfctHoelder}, Thm. 
%\ref{thm:concathoelder} immediately yields the following result:
%\begin{cor}
%If $f: \statespace \to \statespace $ is H\"older continuous with constant $L_f$ and exponent $p_f$ then $\sqrt{f}$ also is Hoelder, having  H\"older constant $\sqrt{L_f}$ and exponent $p_f$.
%\end{cor}


\begin{thm} \label{thm:hoelderconcat}
Let $(\statespace, \d)$ be a metric space and $f,g : \statespace \to \statespace$ be two H\"older continuous mappings with H\"older constants $L(f), L(g)$ and H\"older exponents $p_f,p_g$, respectively.
Then, the concatenation $h=f \circ g: \statespace \to \statespace $ is also H\"older continuous with H\"older constant $L(h):= L(f) L(g)^{p_f}$ and exponent $p_h:=p_g \, p_f$.
That is, 
\[\forall \state,\state' \in \statespace: \d\bigl(h(\state),h(\state')\bigr) \leq L(h) \, \bigl(\d(\state,\state')\bigr)^{p_h}.\]
\begin{proof}
%\begin{align}
$\d\bigl(f \circ g(\state),f\circ g(\state')\bigr)$ $\leq L(f)\,  \Bigl(\d(g(\state),g(\state'))\Bigr)^{p_f}$
$\leq L(f)\,  \Bigl(L(g)\, \d(\state,\state')^{p_g}\Bigr)^{p_f}$ 

$= L(f)\, L(g)\,^{p_f}   \Bigl(\d(\state,\state')\Bigr)^{p_g\, p_f} $ where in the first step we were using H\"older-continuity of $f$ and in the second, we were using H\"older continuity of $g$ combined with the fact that $(\cdot)^{p_f}$ is a monotonically increasing  function. 
\end{proof}
\end{thm} 






%A special case of H\"older continuity is Lipschitz continuity. A Lipschitz continuous function is H\"older continuous with exponent 1.
%If the metric space is $(\Real,(x,y) \mapsto \abs{x-y})$, $f$ is Lipschitz with constant $L(f)$ if $\forall t,t': \abs{f(t)-f(t')} \leq L(f) \, \abs{t-t'}$. 

Several numerical methods, such as Lipschitz optimisation \cite{Shubert:72}, rely on the knowledge of a Lipschitz constant. In the more general case of H\"older continuous functions this will correspond to the need of knowing a H\"older constant and exponent. To avoid having to derive these for each new function from first principles, we establish the following collection of facts that allows us determine bounds on H\"older constants of combinations of functions with known H\"older parameters.
While we provide proofs for a restatement in the H\"older continuous setting, a number of the following statements have also been proven in \cite{Weaver1999} in the context of Lipschitz algebras.





\begin{lem}[H\"older arithmetic] \label{lem:Hoeldarithmetic}
Let, $I,J \subset \inspace$ where $\inspace$ is a metric space endowed with metric $\metric$. Let $f : \inspace \to \Real$ be H\"older on $I$ with constant $L_I (f) \in \Real_+$ 
and $g :\inspace \to \Real$ be H\"older on $J$ with constant $L_J (g) \in \Real_+$. Assume both functions have the same H\"older exponent $p \in (0,1]$. That is, $\forall x, x' \in \inspace: \abs{f(x)-f(x')} \leq L(f) \metric(x,x')^p$ and  $\forall x, x' \in \inspace: \abs{g(x)-g(x')} \leq L(g)  \metric(x,x')^p$.
We have:

\begin{enumerate}
	\item Mapping $x \mapsto |f(x)|$ is H\"older on $I$ with constant $L_I(f)$ and exponent $p_f$.
	\item If $g$ is H\"older on all of $J=f(I)$ the concatenation $g \circ f: t \mapsto g(f(t))$ is H\"older on $I$ with constant 
	      $L_I(g \circ f) \leq$ $L_J (g) \, L_I^p(f)$ and exponent $p^2$.
	\item Let $r \in \Real$. $r \, f: x \mapsto r \, f(x)$ is H\"older on $I$ having a constant $L_I (r \,f) = |r| \, L_I(f)$.
	\item $f+g: x \mapsto f(x) + g(x)$ has H\"older constant at most $L_I(f) + L_J(g)$.
	\item Let $m_f = \sup_{x\in \inspace } f(x)$ and $m_g = \sup_{x \in \inspace } g(x)$. Product function $f\cdot g: x \mapsto f(x) \, g(x)$ has H\"older exponent $p$ and a H\"older constant on $I$ which is at most $(m_f \, L_J(g)+ m_g \, L_I(f))$.
	%\item Let $\tilde h(x) = \min\{f(x),g(x) \}$, $h(x) = \max\{f(x), g(x) \}, \forall x \in \inspace  \cap J$. We have $L_{I \cap J}(h) \leq \max\{L_I(f),L_J(g)\}$ and $L_{I \cap J}(\tilde h) \leq \max\{L_I(f),L_J(g)\}$.	
	\item For some countable index set $\indsett$, let the sequence of functions $f_i$ be H\"older with exponent $p$ and constant $L(f_i)$ each. Furthermore, let $H(x) =\sup_{i \in \indsett} f_i(x) $ and $h(x) := \inf_{i \in \indsett} f_i(x)$ be finite for all $x$. Then $H,h$ are also H\"older with exponent $p$ and have a H\"older constant which is at most $\sup_{i \in \indsett} L(f_i)$.
	\item Let $b := \inf_{x \in \inspace }| f(x)| > 0$ and let 
	$\phi(x) = \frac{1}{f(x)}, \forall x \in \inspace$ be well-defined.  
	      Then $L_I(\phi) \leq b^{-2} \, L_I(f)$.  
	\item Let $p=1$ (that is we consider the Lipschitz case), let $I$ be convex and $\metric(x,x') = \norm{x-x'}$ where $\norm{\cdot}$ is a norm that induces a sub-multiplicative matrix norm (e.g. all $p-$ norms are valid). $f$ cont. differentiable on I $\Rightarrow$ $L_I(f) \leq \sup_{x \in I } \norm{\nabla f(x)}. $ 
	For one-dimensional input space, $\inspace = \Real$, $L_I(f) = \sup_{x \in I } \abs{\nabla f(x)}$ is the smallest Lipschitz number. 
	 \item Let $c \in \Real$, $f( t) = c, \forall x \in I $. Then $f$ is H\"older continuous with constant $L_I(f) =0$ and for any coefficient $p_f \in \Real$.  
	\item $L_I(f^2) \leq 2 \, L_I(f)\, \sup_{t \in I} f\,$.
	\item With conditions as in 8), and input space dimension one, we have $\forall q \in \mathbb Q : L_I(f^q) = |q| \,\sup_{\tau \in \indset } |f^{q-1}(\tau) \, \dot f(\tau)| $.
\end{enumerate}
\begin{proof}

\textbf{1)}  We show $|f|$ has the same constant and exponent as $f$. Let $X,X' \in \inspace $ arbitrary. 
We enter a case differentiation:

\textit{1st case: $f(x), f(x') \geq 0$}. 

Hence, $\bigl| \abs{f(x)}- \abs{f(x')} \bigr| = \bigl| f(x) - f(x') \bigr|  \stackrel{f \,Hoelder}{\leq} L_I(f) \metric(x,x')^{p}$.\\

\textit{2nd case: $f(x) \geq 0, f(x') \leq 0$.} 

Note, $|y| = - y$, iff $y \leq 0$. Hence,  $\bigl| |f(x)| - |f(x')| \bigr| \leq \bigl| |f(x)| + |f(x')| \bigr| $
$= \bigl| |f(x)| - f(x') \bigr|  =  \bigl| f(x) - f(x') \bigr| \leq L_I(f) \, \metric(x,x')^{p}$.\\

\textit{3rd case: $f(x) \leq 0, f(x') \geq 0$.} Completely analogous to the second case.\\

\textit{4th case: $f(x), f(x') \leq 0$}. 

$\bigl| |f(x)| - |f(x')| \bigr| = \bigl| f(x') - f(x) \bigr|= \bigl| f(x) - f(x') \bigr|  \stackrel{f \, Hoelder}{\leq} L_I(f) \metric(x,x')^{p}$.\\

The remaining points are also proven in \cite{Weaver1999} in the context of Lipschitz functions.

\textbf{2)} Special case of Thm. \ref{thm:hoelderconcat}.
% For arbitrary $t,t' \in \indset $ we have:
%
%$\bigl| g(f(t)) - g(f(t'))| \bigr| \leq L_J(g) \, |f(t) - f(t') | \leq L_J(g) \, L_I(f)\, |t-t'|$ 
%where the two inequalities are due to the Lipschitz properties of $g$ and $f$, respectively.\\

\textbf{3)}  For arbitrary $x,x' \in \inspace , r \in \Real$ we have:

$\bigl| r \, f(x) - r \, f(x')| \bigr| = |r|\, |f(x) - f(x')| \leq |r| \,L_I(f)\,  \metric(x,x')^{p}$.\\ 

\textbf{4)}  For arbitrary $x,x' \in \inspace , r \in \Real$ we have:

$\bigl| g(x) + f(x) - (g(x') + f(x'))| \bigr| = \bigl| g(x)  - g(x') + f(x)- f(x')| \bigr|$ 
$\leq \bigl| g(x)  - g(x')\bigr|  + \bigl| f(x)- f(x')| \bigr|$ $\leq (L_J(g)+L_I(f))\,  \metric(x,x')^{p}$.\\

\textbf{5)}  Let  $x,x' \in \inspace $, $d := f(t) - f(t')$.

$\bigl| g(x) f(x) - g(x')  f(x') \bigr| = \bigl| g(x) (f(x') +d) - g(x') f(x') \bigr|$ 
$= \bigl|\bigl( g(x) - g(x') \bigl)  f(x')+ g(x)  d \bigr|  $

$\leq \bigl| g(x) - g(x') \bigr|  |f(x')|   + \bigl|g(x)\bigr| \,  |d|  $
$\leq L_I(g) \metric(x,x')^p  |f(x')|   + \bigl|g(x)\bigr| \,  L_I(f) \metric(x,x')^p  $

$\leq L_I(g) \metric(x,x')^p  \sup_{x' \in \inspace } \{|f(x')|\}  \\ + \sup_{x \in \inspace }\{\bigl|g(x)\bigr|\} \,  L_I(f) \metric(x,x')^p  $

$= \Bigl(L_I(g)  \sup_{x' \in \inspace } \{|f(x')|\}   \\+ \sup_{x \in \inspace }\{\bigl|g(x)\bigr|\} \,  L_I(f)\Bigl) \metric(x,x')^p  $.\\

\textbf{6)}  The proof of Proposition 1.5.5 in \cite{Weaver1999} proves our statement if one replaces their 
metric $\rho$ by $\metric^p$.

\textbf{7)}  Let  $x,x' \in \inspace $.
$\bigl| \frac{1}{f(x)} - \frac{1}{f(x')} \bigr|$ 
$=\bigl| \frac{f(x')}{f(x') f(x)} -\frac{f(x)}{f(x') f(x)} \bigr|$ 
$= \frac{\bigl|f(x')-f(x) \bigr|}{\bigl|f(x')\bigr| \bigr| f(x)\bigr|}$ 
$\leq \frac{L_I(f) \metric(x,x')^p}{\inf_{x \in \inspace } |f(x)|^2}$.\\

\textbf{8)} Let $p=1$ and $\metric(x,x') = \norm{x-x'}$ be a norm that induces a sub-multiplicative matrix norm. Define $\ell := \sup_{x \in I } \norm{ \nabla f(x) } = L_I(f)$. 
Firstly, we show that it is a Lipschitz constant: Let $x,x' \in I $ and 
$\overline{xx'} := \{tx + (1-t) x' \, | \, t \in [0,1]\}$. 
%$\overline{xx'} := \{y | \exists t \in [0,1] : y = tx + (1-t) x'\}$. 
Owing to convexity of I, $\overline{xx'} \subset I$. Due to the mean value theorem $\exists \xi \in \overline{xx'} \subset I: |f(x) - f(x')|=  T_\xi (x-x')$. where $T_\xi (x) = \SP{\nabla f(\xi)}{ x}$ is a linear OP. Assuming the derivative of $f$ is bounded, $T_\xi$ is a bounded OP and we have $\abs{T_\xi (x-x') } \leq \matnorm{T_\xi} \norm{x-x'}$ where 
$\matnorm{T_\xi} = \sup_{\norm{x} = 1} \abs{\SP{\nabla f(\xi)}{x}} \leq \norm{\nabla f(\xi)}$. In conjunction,
$|f(x) - f(x')| \leq \norm{\nabla f(\xi)} \norm{x-x'}$. 

Secondly, we show that $\ell$ is the smallest Lipschitz constant in the one-dimensional case: Let $\bar \ell$ be another Lipschitz constant on $I$ such that $\bar \ell \leq \ell$. Of course, here $\norm{\cdot} = \abs{\cdot}$. Since $I$ is compact and $\norm{\nabla f(\cdot) }$ is continuous, there is some $\xi \in I$ such that $\norm{\nabla f(\xi)} = \sup_{x \in I} \norm{\nabla f(\xi)} = \ell$. Pick any sequence $(y_k)_{k=1}^\infty$ contained in $I$ such that $y_k \stackrel{k \to \infty}{\longrightarrow} \xi$ and $y_k \neq \xi$.
$\forall k: y_k \in I $ and $\bar \ell$ is a Lipschitz constant on $I$. Hence, $ \bar \ell \geq \frac{|f(y_k) - f(\xi)|}{\norm{y_k-\xi}}\stackrel{k \to \infty}{\longrightarrow} \norm{ \nabla f(\xi)} = \ell$. Thus, $\bar \ell = \ell$.

\textbf{9)} Trivial. 

\textbf{10)} Special case of property 5).

\textbf{11)} $L(f^q) \stackrel{8)}{=} \sup_{\tau \in \indset } |\dif{}{t} f^q(\tau)| = |q| \,\sup_{\tau \in \indset } |f^{q-1}(\tau) \, \dot f(\tau)| $. 
\end{proof}
\end{lem} 

As a simple illustration, assume we desire to establish that $f(t) = \max\{ 1- 3 \sin(t), \exp\bigl(- \sin(t) \bigr)\}$ is Lipschitz and to find a Lipschitz number on $\Real$. Application of 8. shows that $t \mapsto \sin(t)$ and $t \mapsto \exp(- t)$ have a Lipschitz number of $1$. Application, of 2., 9. 1. and 6. then show that $L(f) =3$ is a Lipschitz number of $f$.


%As another example we consider the standard deviation of the Ornstein-Uhlenbeck process:
%\begin{ex} \label{ex:hoelderOUstd}
%Consider the function $h(t) = f \circ g(t)$ where $f(\cdot) = \sqrt \cdot$ is the square root function and $g(t) = \frac{B^2}{2 K} (1- \fexp{-2Kt}) $, $K >0$. $g$ is the non-stationary factor of the variance trajectory of an Ornstein-Uhlenbeck process (cf. Sec. \ref{sec:OUprocess}). We show that $h$ is H\"older and has constant $L(h)= \sqrt{2K}$ and H\"older exponent $p_h = \frac 1 2$. This can be easily done by applying Lem. \ref{lem:Hoeldarithmetic}.8 to show that $L(g)$ is Lipschitz with constant $L(g) = B^2 $ and then combining the H\"older constant and exponent derived in Ex. \ref{ex:sqrtfctHoelder} with Thm. \ref{thm:hoelderconcat} to yield the H\"older exponent $p_h = p_g p_f = \frac 1 2$ and constant $L(h) = 1 \sqrt{L(g)}  = \abs B$.  
%\end{ex}

%An example important to our considerations above is the establishment of Lipschitz bounds of an RBFN:
%\begin{ex} \label{ex:hoelderRBFN}

%\end{ex}

\subsection{H\"older continuity of the exponentiated map }

 %
The main objective of this subsection is to show that, for each $s \in \inspace$, the function $\phi_s: x \mapsto L \metric(x,s)^p$ is H\"older continuous with coefficient $L$ and exponent $p$ with respect to (pseudo-) metric $\metric: \inspace^2 \to \Real$. This is important in our derivations since these maps are essential building blocks of the kinky inference procedure. Therefore it is easy to employ Lem. \ref{lem:Hoeldarithmetic} to show that the full kinky inference rule is H\"older continuous.

To establish the H\"older regularity result, we will first show that $(x,y) \mapsto \metric(x,y)^p$, for $p \in (0,1] $, is a metric.  This can be utilised to show that
$\phi_s \in \hoelset L \metric p$.




Before proceeding we need to establish a few facts. 
Firstly, we remind ourselves of the following well-known fact:
\begin{lem} \label{lem:pd_n_concave_subadditive}
A nonnegative, concave function $g:\Real_{\geq 0} \to \Real$ with $g(0) = 0$ is subadditive. 
That is, $\forall x,y \in \Real_{\geq 0}: g(x+y) \leq g(x) + g(y)$. 
 \begin{proof}
If $x = y = 0$ then subadditivity trivally holds:  $0=g(x+y) \leq g(x) + g(y) = 0$.
So, let $x, y \in \Real_+$ such that $x >0 \vee y >0$.
We have, $g( x +y) = \frac{x}{x+y} g(x+y) + \frac{y}{x+y} g(x+y) \leq g(\frac{x}{x+y} (x+y) ) +  g(\frac{y}{x+y}(x+y)) = g(x) + g(y)$.
Taking into account that $\frac{x}{x+y}, \frac{x}{x+y} \in [0,1]$, the last inequality can be seen as follows:

 Since $g$ is concave we have 
$\forall p \in [0,1], x \in \Real: g(p x) =  g(px + (1-p) 0) \geq p g(x) + (1-p) g(0) \geq p g(x)  $. The last inequality follows from $g(0) \geq 0$.
\end{proof}
\end{lem}

\begin{lem} \label{lem:x2p_pdNsubadd}
 Let $h: \begin{cases} \Real_{\geq 0} \to \Real_{\geq 0},\\ \, x \mapsto x^p\end{cases}$, for $p \in (0,1] $. $h$ is positive definite and subadditive. 
 That is, (i) $h(0) = 0 $ and  $h(x) > 0, \forall x \neq 0$ and (ii) $\forall x,y \in \Real_{\geq 0}: h(x+y) \leq h(x) + h(y)  $.
 Moreover, $h$ is concave. If $p \in (0,1)$, h is strictly concave. 
 
\begin{proof}
\textit{Pos. def. :} $h(0) = 0$.  Also $\lim_{x \to 0_+} h(x) =0$. Since $\nabla h (x) = p h^{p-1}(x) >0 $ for $x >0$, $h$ is strictly monotonically increasing on $\Real_+$. Hence, $h(x) > 0, \forall x >0$. 

\textit{Concavity:} If $p =1$, $h$ is linear and hence, concave. If $p \in (0,1)$, $\nabla^2 h(x) = p (p-1) h(x)^{p-2} > 0$ regardless of $x$.

\textit{Subadditivity:} Follows directly with Lem. \ref{lem:pd_n_concave_subadditive} on the basis of established positive definiteness and concavity.
  %
\end{proof}
\end{lem}



\begin{lem}\label{lem:hoeldererror_metric}
Let $p \in (0,1]$.
With definitions as above, we assume that set $\inspace$ is endowed with a pseudo- metric $\metric$. Function
$\metricp: \begin{cases} \inspace \times \inspace \to \Real_{\geq 0} \\ (x,y) \mapsto \Bigl(\metric(x,y)\Bigr)^p \end{cases}$ is a pseudo-metric on $\inspace$.
If $\metric$ is a metric then so is $\metricp$.
\begin{proof}
 By Lem. \ref{lem:x2p_pdNsubadd}, $x\mapsto x^p$ is pos. def. and sub-additive. Therefore, positive definiteness and the triangle inequality of $\metric$ readily extend to $\metricp$ as follows: 

\textit{Pos. def.:}
Let $x=0$. $\metricp(x,x) = \metric(x,x)^p = 0^p = 0$. If $x \neq 0$ then $k :=\metric(x,x) \neq 0$. Hence   $\metricp(x,x) = d(x,x)^p = k^p \neq 0$.

\textit{Triangle inequality:}
Choose arbitrary $x,y,z \in \inspace $. We have $\metricp(x,z) = \metric(x,z)^p \leq (\metric(x,y) + \metric(y,z) \bigr)^p \leq \metric(x,y)^p + \metric(y,z)^p = \metricp(x,y) + \metricp(y,z)$. Here, the first inequality followed from the triangle inequality of pseudo-metric $\metric$ and the second from subadditivity properties established in Lem. \ref{lem:x2p_pdNsubadd}.

\textit{Symmetry:} If $\metric$ is a metric it is symmetric. Hence, $\metricp(x,y) = \metric(x,y)^p = \metric(y,x)^p = \metricp(y,x), \forall x,y \in \inspace $ in which case $\metric$ also is symmetric.
\end{proof}
\end{lem}

Before proceeding we establish a slight generalisation of the well-known \textit{reverse triangle inequality}:

\begin{lem}[Reverse Triangle Inequality] \label{thm:revtriangle}
Let  $\inspace$ be a set and $\metric : \inspace^2 \to \Real$ a symmetric function that satisfies the triangle inequality. 
That is,  $\forall x,y,z \in \inspace: \metric (x,y) = \metric(y,x) \wedge \metric(x,z) \leq \metric(x,y) + \metric(y,z)$.

Then \[\forall x,y,z \in \inspace: \abs{\metric(x,y) - \metric(z,y)} \leq \metric(x,z).\]
\begin{proof}
For contradiction, assume 
$\abs{\metric(x,y) - \metric(z,y)}>\metric(x,z)$ for some $x,y,z \in \inspace$.
This is implies  
$
(i)\,\,\, \metric(x,y) - \metric(z,y)>\metric(x,z) 
\, \, $  or  $
\,\,  (ii) \,\,\,\, \metric(z,y)-\metric(x,y) >\metric(x,z)$.
Both inequalities contradict the triangle inequality:
$(i) \Leftrightarrow \metric(x,y)  >\metric(x,z) +\metric(z,y) $ and 
$(ii) \Leftrightarrow  \metric(z,y) > \metric(x,z) + \metric(x,y) =  \metric(z,x) + \metric(x,y)$.

\end{proof}
\end{lem}


\begin{lem}
Let $\metric$ be a (pseudo-) metric on $\inspace$. For arbitrary $s \in \inspace $ we define $\phi_s: \inspace \to \Real $ as $\phi_s: x \mapsto \metric (x,s) $.
$\phi_s $ is Lipschitz with respect to metric $\metric$. That is, \[\forall x,y \in \inspace : \abs{\phi_s(x) - \phi_s(y) } \leq \metric (x,y). \]
\begin{proof}
$\abs{\phi_s(x) - \phi_s(y)} = \abs{\metric(x,s) - \metric(y,s)} \stackrel{Thm. \ref{thm:revtriangle}} {\leq} \metric(x,y), \forall x,y,s \in \inspace $.
\end{proof}
\end{lem}
Finally, combining this lemma with Lem. \ref{lem:hoeldererror_metric} immediately establishes that mappings of the form $\metric(\cdot,s)^p$ are H\"older continuous with exponent $p$ ($\in (0,1]$):

\begin{thm} \label{thm:d2pmapishoelder}
Let $\metric$ be a (pseudo-) metric on set $\inspace$ and let $p \in (0,1], L \geq 0$. For arbitrary $s \in \inspace $ we define $\phi_s:\begin{cases}  \inspace \to \Real \\ x \mapsto L \, \bigl(\metric (x,s) \bigr)^p\end{cases}$.
$\phi_s $ is H\"older with exponent $p$. That is, \[\forall x,y \in \inspace : \abs{\phi_s(x) - \phi_s(y) } \leq L \, \metric (x,y)^p. \]
In particular, for any norm $\norm \cdot$ on $G$ and $s \in \inspace $, mapping $x \mapsto L \norm{x-s}^p$ is H\"older with constant $L$ and exponent $p$.

\begin{proof}
By Lem. \ref{lem:hoeldererror_metric}, $\metric^p(\cdot,\cdot)$ is a (pseudo-) metric on $\inspace$. Hence, Lem. \ref{thm:revtriangle} is applicable yielding:
$\abs{\phi_s(x) - \phi_s(y)} = L \, \abs{\metric(x,s)^p - \metric(y,s)^p} \stackrel{Lem. \ref{thm:revtriangle} } {\leq} \metric(x,y)^p, \forall x,y,s \in \inspace$. The last sentence concerning the norms follows from the fact that a mapping $(x,y) \mapsto \norm{x-y}$ defines a metric.
\end{proof}
\end{thm}



 %At first glance, the result may seem intuitive. However, it should be noted that, for $p >1$, $x \mapsto \norm{x -s}^p$ no longer is H\"older with exponent $p$. This is consistent with the well-known fact that H\"older functions with exponent $>1$ are constant functions.
%
%We conclude this section by the following theorem stating that any concatenation of H\"older continuous functions is H\"older continuous:
%
%\begin{thm} \label{thm:concathoelder}
%Let $(\statespace, \metric)$ be a metric space and $f,g : \statespace \to \statespace$ be two H\"older continuous mappings with  H\"older  constants $L_f, L_g$ and H\"older exponents $p_f,p_g$, respectively.
%Then, the concatenation $h=f \circ g: \statespace \to \statespace $ is also H\"older continuous with H\"older constant $L_h:= L_f L_g^{p_f}$ and exponent $p_h:=p_g \, p_f$.
%That is, 
%\[\forall \state,\state' \in \statespace: \metric\bigl(h(\state),h(\state')\bigr) \leq L_h \bigl(\metric(\state,\state')\bigr)^{p_h}.\]
%\begin{proof}
%%\begin{align}
%$\metric\bigl(f \circ g(\state),f\circ g(\state')\bigr) \leq L_f  \Bigl(\metric(g(\state),g(\state'))\Bigr)^{p_f}$
%$\leq L_f  \Bigl(L_g \metric(\state,\state')^{p_g}\Bigr)^{p_f}$ $= L_f L_g^{p_f}   \Bigl(\metric(\state,\state')\Bigr)^{p_g\, p_f} $ where in the first step we were using Hoelder-continuity of $f$ and in the second, we were using H\"older continuity of $g$ combined with the fact that $(\cdot)^{p_f}$ is a monotonically increasing  function. 
%
%\end{proof}
%\end{thm} 
%In conjunction with H\"older properties of the square root function established in Ex. \ref{ex:sqrtfctHoelder}, Thm. 
%\ref{thm:concathoelder} immediately yields the following result:
%\begin{cor}
%If $f: \statespace \to \statespace $ is H\"older continuous with constant $L_f$ and exponent $p_f$ then $\sqrt{f}$ also is Hoelder, having  H\"older constant $\sqrt{L_f}$ and exponent $p_f$.
%\end{cor}

%
\subsection{An incremental lazy update rule for the H\"older constant}
\label{sec:lazyconstants}
Above our theoretical guarantees are based on the assumption that the parameters $L(n)$ were set to a fixed H\"older constant of the target function. In other words, if we know a H\"older constant $L^*$ of the ground truth $f$ then setting $L(n)=L^*,\forall n$ ensures the desired conservatism and convergence properties of the KI rule. However, in a black-box learning setting, a H\"older constant may not be know a priori. Therefore, the estimation of Lipschitz constants on the basis of a sample is a problem that has been considered by several authors in the past \cite{Wood1996,Beliakov2006}. For instance, Wood and Zhang \cite{Wood1996} reduced the problem of estimating a global Lipschitz constant of a one-dimensional function to fitting a Weibull distribution to random absolute slope estimates between uniformly sampled inputs. Unfortunately, their results are restricted to one-dimensional functions that satisfy the Gnedenko condition and where one has access to a batch of samples drawn i.i.d. uniformly at random and no observational errors. Beliakov \cite{Beliakov2006} used a simple empirical estimator of a Lipschitz constant that did not require distributional assumptions and can be seen as a simple special case of our proposed lazy method. In contrast to his estimator, our approach is designed to receive data incrementally and makes provisions for observational noise in the sample. We do not make any distributional assumptions about the noise or any assumptions about the target other than that it is 
a continuous function from a pseudo-metric space into an additive group endowed with a translation-invariant metric. 
  


%
%\subsection{A Bayesian treatment of probabilistically uncertain H\"older constants}
%\label{sec:bayesupdatelipconst}
%Throughout the previous section we have assumed that the H\"older constants were known to us. 
%However, in many cases, we may be uncertain about the best H\"older constant. We assume the uncertainty is expressed as a distribution. Following the Bayesian point of view, we interpret probability densities over events as subjective degrees of beliefs that these events hold true. 
%In the following subsection, we describe how to utilize the densities over the best H\"older constant $L^* \in \Real_+$ to yield a (once again, conservative) bound on the probability of our functions estimates and integrands not being conservative. After that, we will address the question of how to update our beliefs over $L^*$ in the light of new function evaluations in a Bayesian manner.
%
%For simplicity, we will assume there are no observational errors and that there are no a priori bounds. That is, $\obserr_i = 0, \, \forall i$ and $\lbf = -\infty, \ubf = \infty$. Furthermore, we tacitly assume the H\"older exponent $p$ is known and fixed.
%
%Let $\pi: \Real_+ \to \Real_+$ be a density encoding our belief over best H\"older constant $L^*$ of target function $f$.
%In the absence of bounds and observational error, we define the H\"older enclosure $\einschluss (\ell) =\{ \phi \,| \, \phi(x) \in \prederrbox_n(x)\}$  where $\prederrbox_n$ is the uncertainty hyperrectangle defined above (see Eq. \ref{eq:prederrint}), computed on the basis of H\"older constant $L=\ell$.
%Assume we construct a H\"older enclosure $\einschluss (\ell)$ based on choosing $\ell \in \Real_+$ as a H\"older constant.  How large do we need to choose $\ell$ to guarantee that $f$ is completely contained in the enclosure? Since we are uncertain about the true H\"older constant, this question can only be framed probabilistically. That is, for a given certainty threshold  $\theta \in (0,1)$ we desire to find $\ell > 0$ such that 
%\[ \theta \leq \Pr[ f \in \einschluss (\ell) ]. \]  
%
%\begin{thm} \label{thm:ksjhdkjs} Let $\decke_n,\boden_n$ be a valid ceiling and floor function, respectively.
%Let $P: t \mapsto \int_{x \leq t} \pi(x) \,\d x $ our density's cumulative distribution function (cdf).
%We have, \[P(\ell) \geq \theta \Rightarrow \theta \leq \Pr[ f \in \einschluss (\ell) ].  \] 
%\begin{proof} 
%Let $L^*$ denote the best H\"older constant of function $f$.
%
% $\forall \ell \geq L \geq 0 :   \Pr[ f \in \einschluss (\ell) | L^*=L] = 1$. 
%Hence,
%$
%\Pr\bigl[ f \in \einschluss (\ell) \bigr] 
%= \int_0^\ell \Pr\bigl[ f \in \einschluss (\ell) | L^*=L\bigr] \,\d P(L) 
% + \int_\ell^\infty \Pr\bigl[ f \in \einschluss (\ell) | L^*=L\bigr] \,\d P(L) 
%=  \int_0^\ell \,\d P(L) + \int_\ell^\infty \Pr\bigl[ f \in \einschluss (\ell) | L^*=L\bigr] \,\d P(L)  
%\geq \int_0^\ell \,\d P(L) 
%= P(\ell).$
%\end{proof}
%\end{thm}
%
%That is to say, in order to guarantee conservativeness of our enclosure, all we need to do is to compute it on the basis of 
%a fixed H\"older constant that is just large enough such that $P(\ell) = \int_0^l \pi(L) \,dL \geq \theta$. Notice, if the integral cannot easily be determined in closed-form, but a H\"older constant for density $\pi$ is known, we can employ our H\"older quadrature method described previously (based on a known constant), to evaluate $P(\ell)$ conservatively by finding a lower bound on it. \\
%
%
%%When it comes to the task of quadrature, in the presence of an uncertain H\"older constant, the error bounds of the integral estimates afforded by our method also are valid to a level of probabilistic belief. 
%%
%%Combining Thm. \ref{thm:ksjhdkjs} with our boundedness results for conservative quadrature of Sec. \ref{sec:quadr} immediately yields the following result:
%%\begin{cor}
%%Let $S := \int_I f(t) \, \d t$. 
%%With the definitions introduced above, we have  \[P(\ell) \geq \theta \Rightarrow \theta \leq \Pr\Bigl[ \hat S^\decke_N \geq S \geq \hat S^\boden_N  \Bigr]  \] 
%%where $\hat S^\decke_N , \hat S^\boden_N $ are computed assuming a H\"older constant of $\ell$. 
%%
%%\end{cor}
%%
%%The Corollary asserts that if we check that $P(\ell) \geq \theta$ for some desired threshold $\theta$ then our probabilistic belief in the integral $S$ being within the interval $[S_n^\boden,S_n^\decke]$ also is at least $\theta$.
%
%\subsubsection{A Bayesian update rule}
% Define $\metric_{\outspace}(f_1,f_2) = \abs{f_1-f_2}$ and let $\metric_{\inspace}(x,y)$ be some input space (pseudo-) metric as before. 
%
%%That is a H\"older constant $L$ with respect to the canonical metric $(x,y) \mapsto \norm{x-y}$ is a Lipschitz constant w.r.t $\metric_\inspace$. We will refer to $L$ uniformly as \textit{H\"older} constant.
%
%
%Assume we hold a prior belief over the best H\"older constant $L^*$ encoded as a density $\pi_0 : I \subset \Real_+ \to [0,1]$.
%Assume we are given a sample $D = \{(s_i,f_i)\}_{i=1,...,N}$ of input-function value pairs, $x_i \in \mathcal X$, $f_i = f(s_i) \in \outspace, \forall i$, the question arises of to calculate a posterior in the light of the new data. 
%
%Since we assume $f :\mathcal X \to \mathcal Y \subseteq \Real$ is H\"older we have $ \frac{\metric_\outspace(f_i,f_j)}{\metric_{\inspace}(s_i,s_j)} \leq L^*,  (i,j =1,...,N, x_i \neq x_j)$ where $L^*$ is the best H\"older constant of $f$. So, our observations allow us to infer that the \emph{empirical H\"older constant} 
%\begin{equation}
%	L_D := \max_{i,j,i\neq j} \frac{\metric_\outspace(f_i,f_j)}{\metric^p_{\inspace}(s_i,s_j)}
%	\label{eq:LD}
%\end{equation}
%is a lower bound on the best H\"older constant.\footnote{In the context of a special Lipschitz interpolation rule, Beliakov \cite{Beliakov2006} considered this quantity as an estimate of the Lipschitz constant.} Note, the computation of an update of this quantity can be done with computational effort linear in the number of pre-existing quantities. 
%That is, assuming a pre-existing data set $D$ and that we make an additional observation $s_{N+1},f_{N+1}$ which is incorporated into the updated data set $D' = D \cup \{(s_{N+1},f_{N+1})\}$. Instead of computing $L_{D'}$ from scratch as per Eq. \ref{eq:LD}, we can leverage:
%
%\begin{equation}
%	L_{D'}=\max \{L_D,L'\}, \,\text{ where } L' := \max_{i=1,\ldots,N} \frac{\metric_\outspace(f_{N+1},f_i)}{\metric^p_{\inspace}(s_{N+1},s_i)} 
%	\label{eq:LDupdate}
%\end{equation}
%
%which can be computed in $\mathcal O(d N)$ where as before, $d = \dim \inspace$.
%The remaining question is how to derive a posterior over $L^*$ based on this newly observed lower bound. 
%%
%%\begin{ques}
%%How can we prove that there is no information about the H\"older constant in the data other than $L_D$? That is, in particular, that \[p(L^*|D) \stackrel{!}{=}p(L^* |L_D) ??\]
%%\end{ques}
%%
%%
%%We will discuss two approaches, the Minimum-Relative-Entropy approach and Bayesian inference.
%%
%
%
%%\subsubsection{Bayesian approach}
%
%By Bayes theorem, we can write 
%\[\pi(L|L_{D'}) = \frac{\pi(L_{D'}|L) \, \pi_0(L) }{ \int_I \pi(L_{D'}|L) \, \pi_0(L) \, \d L}.\]
%
%If we are uncertain about which empirical H\"older constant we observe given the real H\"older constant  $L^* = L$ we set the Likelihood function 
%\[ \pi(L_{D'}|L) = \begin{cases} 0, L_{D'} > L\\ \frac{1}{L}, \text{otherwise}. \end{cases}.\]
%
%Of course, if the definite integrals of prior $\pi_0$ are not known in closed form, we need to approximate numerically. Depending on knowledge of smoothness properties we can either employ standard methods such as Gaussian quadrature or, if necessary, utilise our H\"older quadrature methods to that effect.
%%\begin{ques}[@Mike]
%%What is the likelihood function $p(L_D|L)= ... ?$
%%\end{ques}
%
%%\subsubsection{Minimum-Relative-Entropy Approach} 
%%
%%Computing posteriors from priors subject to constraints is the focus of max-ent inference. 
%%
%%Remember, 
%%\begin{defn}[Relative Entropy] For two densities $p,q: I  \to [0,1]$ the \textit{relative entropy} or \textit{Kullback-Leibler divergence} is defined as 
%%\[\KLD {p}{q} = \int_{I} p(x) \log \frac{p(x)}{q(x)} d\mu(x)\] where $\mu$ is a measure appropriate for domain $I$.
%%\end{defn}
%%
%%For example, if $I \subset \Real$ then $\mu$ is normally tacitly assumed to be the standard Lebesgue measure. If $I$ is a discrete set, $\mu$ is noramlly assumed to be the counting measure. In this case \[\KLD {p}{q} = \sum_{x\in I} p(x) \log_2 \frac{p(x)}{q(x)}. \]
%%
%%For our problem, we can pose the desired posterior density as the solution to the variational problem:
%%
%%\begin{align}
%%\argmin_p & \KLD {p}{p_0}\\
%%\text{s.t.:}&\\
%%& p(l) \in [0,1], \forall l \geq 0\\
%%&\SP{p}{e}_{L_2(\Real)} =1\\
%%& \SP{p}{\chi_{[0,L_D]}}_{L_2(\Real)} =0
%%\end{align}
%%
%%where $e: t \mapsto 1$ is the constant mapping to $1$ and $\chi_{[0,L_D]}$ is the indicator function for interval $[0,L_D]$.
%%
%%\begin{ques}
%%Can we solve this variational problem in closed form? How can we solve them numerically. If we have no clue, we can make everything discrete as follows...
%%\end{ques}
%%
%%\emph{Discretization}.
%%For practical purposes, it will be convenient to reduce the problem to a standard convex optimization problem as follows:
%%Let $J_1,...,J_m \subset I$ be a partition of $I\subset \Real_+$.
%%Instead of expressing a belief as a density on a continuous interval we may limit our modelling efforts to defining a belief as a discrete distribution function $\pi_0 : \{ 1,...,m\} \to [0,1], i \mapsto \Pr[ L \in J_i]$.  
%%
%%\begin{rem}
%%Notice, that this belief encoding may be easier to specify than the density on a continuum of points. 
%%For instance, assume $I = [a,b] , 0<a<b \in \Real_+ \cup \{\infty\}$ and let $a=t_0<t_1<...<t_m =b$ such that $J_i = [t_{i-1},t_i], (i=1,...,m)$.
%%Then, knowing the continuous density $p_0$ on $I$ allows us to compute the discrete density via $ \pi_0 (i) = \int_{J_i} p_0(t) \,dt=P_0 (t_i) - P_0(t_{i-1})$. The inverse computation (from discrete to continuous distribution) is not possible without further assumptions.
%%\end{rem}
%%
%%We anticipate our discretization to spawn information loss, since we only encode information about the interval not about the relative location of the H\"older number within the interval. This will be investigated next.
%%
%%%Let $\tau \in (0,1]$ denote this relative interval coordinate. That is to say, H\"older number $L = t_{i-1}+ \tau (t_i -t_{i-1}) $ is completely given by coordinates $(i,\tau) \in \{1,...,m\} \times (0,1]$ . 
%%%
%%%After discretization, our uncertainty about the exact H\"older number (not just the interval) is described by $\tilde p_0(L) =  \pi_0(i) \, p_\tau[\tau | i ]$. Assuming that, after discretization, we are completely oblivious about the location within the interval, we have $p_\tau [\tau | i ] = \frac{1}{| (0,1] |} =1$.
%%%Hence, 
%%%
%%%The magnitude of the information loss due to discretziation could be quantified as 
%%%\begin{align*}
% %%H_{\tilde p_0}-H_{ p_0}  &=\int_I p_0(t) \log p_0(t) \, dt  - \bigl(\sum_{i=1}^m \pi_0(i)   \log \pi_0(i) \bigr) \\
%%%& =  \sum_{i=1}^m \int_{J_i} p_0(t) \log p_0(t) \, dt  \\ 
%%%&- \bigl(\sum_{i=1}^m \int_{J_i} p_0(t) \,dt  \log \int_{J_i} p_0(t) \,dt \bigr)\\
%%%% 
%%%& =  \sum_{i=1}^m \Bigl(\int_{J_i} p_0(t) \log [p_0(t)] \, dt  \\
%%%&-   \int_{J_i} p_0(t) \,dt  \log [\int_{J_i} p_0(t) \,dt] \Bigr)\\
%%%& \stackrel{!?}{\leq} 0
%%%\end{align*}
%%%
%%%
%%%WEIRD !!!  The loss should be greater than 0, not smaller ! 
%%%Mhh... perhaps I should instead try this (?? Makes any difference??)
%%
%%In the discretized situation, our ignorance over $L^*$ can be encoded by the density  $\tilde p_0(L) =  \sum_{i=1}^m \pi_0(i) \, p_L [L| i ]$. Assuming that, after discretization, we are completely oblivious about the location within the interval, we have $p_L [L | i ] = \frac{1}{| J_i |} =\frac{1}{| t_i-t_{i-1} |}$.
%%Hence, $\tilde p_0(L)  =  \sum_{i=1}^m  \frac{\pi_0(i)}{| J_i |} $. 
%%
%%The magnitude of the information loss due to discretziation could be quantified as 
%%\begin{align*}
% %H_{\tilde p_0}-H_{ p_0}  &=\int_I p_0(L) \log p_0(L) \, dL  - \int_I \tilde p_0(L)    \log  \tilde p_0(L) \, dL \\
%%& = \int_I p_0(L) \log p_0(L) \, dL  \\
%%&- \int_I\sum_{i=1}^m \pi_0(i) \, p_L [L| i ]    \log \bigl(\sum_{i=1}^m \pi_0(i) \, p_L [L| i ]\bigr) \, dL \\
%%& = \int_I p_0(L) \log p_0(L) \, dL  \\
%%&- \int_I\sum_{i=1}^m \frac{\pi_0(i)}{|J_i|}     \log \bigl(\sum_{i=1}^m \frac{\pi_0(i)}{|J_i|} \bigr) \, dL \\
%%& = \int_I p_0(L) \log p_0(L) \, dL  \\
%%&- |I|  \log \bigl(\sum_{i=1}^m \frac{\pi_0(i)}{|J_i|} \bigr) \sum_{i=1}^m \frac{\pi_0(i)}{|J_i|}     \\
%%& =  \sum_{i=1}^m \Bigl(\int_{J_i} p_0(t) \log [p_0(t)] \, dt \Bigr)   \\
%%&- |I|  \log \bigl(\sum_{i=1}^m \frac{\int_{J_i} p_0(t) dt}{|J_i|} \bigr) \sum_{i=1}^m \frac{\int_{J_i} p_0(t) dt}{|J_i|}     \\
%%& \stackrel{!?} {\geq }0
%%\end{align*}
%%
% %
%%However, the benefit is that (continuous) extremization problem is reduced to a tractable, discrete optimization problem of dimensionality $m$. In fact, it is a convex program with linear constraints:
%%
%%\begin{align}
%%\argmin_p & \KLD {\pi}{\pi_0}\\
%%\text{s.t.:}&\\
%%& \pi(i) \in [0,1], \forall i \in\{1,...,m\}\\
%%&\SP{p}{e}_{2} =1\\
%%& \SP{p}{n_{L_D}}_{2} =0
%%\end{align}
%%where $e$ is a vector of $m$ ones and $n_{L_D}$ is suitably defined to mimic a discretized version of the indicator function above and whose $i$th component is defined as follows:
%%
%%$n_{L_D} (i) = \begin{cases} 1, t_i \geq L_D\\ 0, \text{otherwise.} \end{cases}$ Of course, if $t_{i-1} < L_D < t_i$ for some $i$, we have once again thrown away information,as the constraint $\SP{p}{n_{L_D}}_{2} =0$ does not capture the knowledge that there must be zero probability mass on $K_i' := [t_{i-1},L_D)$ as well. To cover this, we could consider 
%%introducing an additional component representing the probability on $K_i$  and replace the original meaning of the original $i$th component by representing the probability on  $J_i' = [L_D, t_i] =J_i - K_i$ instead of on $J_i$. In this case, the CP becomes $m+1$-dimensional. Since this is always possible to do, we can assume, without loss of generality, that $L_D $ will always be an element of $t_1,...,t_m$ for some $m$.
%%
%%
%%Note, an $m$-dimensional CP can be solved in worst-time complexity $\mathcal O(m^3)$ (LOOK IT UP!!).
%%
%%
%%\begin{rem}
%%We could investigate how the information loss (which we would expect to grow as a function of $m$) compares to the 
%%the gain in computational complexity. We should then conceive a common exchange rate / currency / trade-off parameter to 
%%find a good cut-off point.
%%\end{rem}
%%
%
%
%\begin{figure}
%        \centering
%				  \subfigure[Inferred model with $L=10$.]{
%    %\includegraphics[width = 3.7cm, height = 3cm]{content/figures/graph1_klein.eps}
%    \includegraphics[width = 4.5cm]
%								%{content/Ch_kinkyinf/figs/resultswingrock_555trials}
%								{content/Ch_kinkyinf/figs/predL10}
%    \label{fig:Hoelderconstvar1}
%  } 	
%	 \subfigure[Inferred model with $L=0.1$.]{
%    %\includegraphics[width = 3.7cm, height = 3cm]{content/figures/graph1_klein.eps}
%    \includegraphics[width = 4.5cm]
%								%{content/Ch_kinkyinf/figs/resultswingrock_555trials}
%								{content/Ch_kinkyinf/figs/predLpt1}
%    \label{fig:Hoelderconstvar2}
%  } 	%\hspace{2cm}
%	 \subfigure[Inferred model with $L=1$.]{
%    %\includegraphics[width = 3.7cm, height = 3cm]{content/figures/graph1_klein.eps}
%    \includegraphics[width = 4.5cm]
%								%{content/Ch_kinkyinf/figs/resultswingrock_555trials}
%								{content/Ch_kinkyinf/figs/predL1}
%    \label{fig:Hoelderconstvar3}
%  } 		
%	\label{fig:Hoelderconstvar}
%   \caption{Example of the predictions of the values of ground-truth function $x \mapsto \sqrt{\abs{\sin(x)}}$ using varying H\"older constants. For all predictions, a standard metric $\metric_\inspace(x,x') = \abs{x-x'}$ and a H\"older exponent of $p= \frac 1 2$ was employed. The global H\"older constant (and hence, the optimal constant) was $L=1$. }
%\end{figure}	 


%\subsection{A lazy update rule} 
\label{sec:lazylipconstupdate}
%We assume there is a global H\"older constant that is uncertain. 
%In the presence of uncertainty it is important to be able to update the H\"older constant $L$ in the light of the data. After all, if chosen too conservatively high, the regularity properties of the actual ground truth will be lost and yield predictions that are too steep and, for large $L$, yield step function predictions (cf. Fig \ref{fig:Hoelderconstvar1}). And, if decision making takes the error bounds provided into account, the unnecessary uncertainty may result in overly cautious behaviour and hence, poor performance. 
%On the other hand, underestimating $L$ may even be worse. If $L$ is chosen too small then the error bounds are off and predictions may be unable to accurately infer the true shape of the function (cf. Fig \ref{fig:Hoelderconstvar2}). 
%
%Above we have outlined a Bayesian treatment in the face of probabilistic uncertainty. This is an adequate choice if one's uncertainty can be specified by a prior and if the necessary integrals can be computed efficiently. 
%
%In practice, both can be difficult and we might resort to a more pragmatic approach which we will outline in this subsection. Furthermore, the Bayesian approach can fall prey to stubbornness and cannot recover from false beliefs of impossibility. That is, if an observed data point provides evidence for the validity of an event that has zero measure under the prior the posterior will be unable to adjust and also ascribe zero-measure to the event, even if most strongly supported by the evidence.  
%
%To avoid underestimating the H\"older constant without resorting to overly conservative choices of $L$ it might be possible to start with a good estimate of $L$ that is not overly conservative and to update 
%this lazily, that is, only if required. 

\subsubsection{A first update rule}
 In this subsection, we present an incremental update law that allows our method to increase the estimate of the H\"older constant if newly available data suggests the previously assumed upper bound on the constant was insufficiently high.
 
Assume we have sequential access to a sequence $\seq{\data_n}{n \in \nat}$ of data sets $\data_n:= \{\bigl( s_i, \tilde f_i \bigr) \vert i=1,\ldots, N_n\} $ with $\data_n \subset \data_{n+1}$ ($n \in \nat$) with observational error bound $\obserr$. For simplicity (of notation), we will for now assume that the input space pseudo-metric is a full metric. That is, that we have $\metric_\inspace(x,x') = 0 \Leftrightarrow x=x'$.
%For notational brievty we introduce the set $$\mathcal U := \{ (x,x') \in \inspace^2 | \metric_\inspace(x,x') >0 \}.$$
  We will consider the task of utilising the data to compute estimates to update a deterministic belief over a H\"older constant. In order to bound the conservatism of our uncertainty bounds derived from the constant in the context of kinky inference, the ability to uncover low H\"older constants or even the \emph{optimal} H\"older constant $L^* = \sup_{x'\neq x } \frac{\metric_\outspace \bigl(f(x) - f(x')\bigr)}{\metric_{\inspace}^p(x,x') }$ of the target function $f$ are of particular interest. 
Remember, $\hoelset p {} {L'}  \subset \hoelset p {} {L''} $ for $L'' \geq L'$. Therefore, in order not to inflate the set of hypothesis functions that could have generated the data more than necessary, one might choose not to rule out any lower global H\"older constant candidates unless the data suggests ones has to. In this approach, after having observed $\data_{N_n+1}$, $L(n+1)$ is chosen to be just large enough such that the H\"older condition is not in conflict with sample-consistency, i.e. such that $\mathcal K(\data_{N_n+1}) \cap \bigl(  \mathcal K_{prior} \cup \hoelset p {} {L(n+1)} \bigr) \neq \emptyset$. This approach gives rise to a \emph{lazy} update rule which only increases the current estimate $L(n)$ by the minimal amount necessary to remove any inconsistency with the new data if such inconsistencies are detected. %if the new data point is otherwise inconsistent with it.

Before introducing our lazy update approach in full generality, we will consider a special case first:
For simplicity, we assume that the target is real-valued with $\outspace \subseteq \Real, \metric_\outspace(y,y') = \abs{y-y'}$ for all $y,y \in \outspace$ and that a uniform observational error bound $\obserr =\obserr(x) \in \Real_{\geq 0}, \forall x$ is given. Furthermore, we assume the bounding functions $\ubf,\lbf$ are inactive, i.e. $\lbf \equiv -\infty, \ubf \equiv \infty$. Therefore, the uncertainty hyperrectangle $\prederrbox_n(x)$ around the prediction $\predfn(x)$ coincides with the interval $[\boden_n\bigl(x; L(n)\bigr) ,\decke_n\bigl(x; L(n)\bigr)]$.


For now, we also assume that at each time $n$ we receive one new data point. In this scenario we have $\data_{n+1} \backslash \data_n=\{(s_{N_n+1},\tilde f_{N_n+1}) \}$ with $N_n = n$. Given that at time $n$ the lazy estimator proposes the optimal H\"older constant $L^*$ to be $L(n) \in \Real_{\geq 0}$, how should the arrival of the new data point $(s_{N_n+1},\tilde f_{N_n+1},\obserr)$ affect the belief in the H\"older constant? %Now assume that a new sample observation $(s_{N_n+1},f_{N_n+1}, \obserr)$ arrives.  
Following our lazy approach, we only make modifications if the previous estimate is unduly low in the light of the new data. Here the new data point $(s_{N_n+1},\tilde f_{N_n+1},\obserr)$ is evidence for constant $L(n)$ to be unduly low if it falls outside of the prediction uncertainty  interval $\prederrbox_n(x)$. This is the case if 
$\tilde f_{N_n+1} +\obserr < \boden_n(s_{{N_n}+1}) \vee \tilde f_{{N_n}+1} - \obserr > \decke_n\bigl(s_{{N_n}+1}; L(n)\bigr)$ or equivalently, if 
$\tilde f_{{N_n}+1} +\obserr < \max_{i=1,\dots,N_n} \tilde f_i - \obserr - L(n) \metric^p_{\inspace}(s_{{N_n}+1},s_i) \vee \tilde f_{{N_n}+1} - \obserr > \min_{i=1,\dots,N_n} \tilde f_i + \obserr + L(n) \metric^p_{\inspace}(s_{{N_n}+1},s_i)$.
Equivalently, this means that 
\begin{equation}
	\abs{\tilde f_{{N_n}+1} -\tilde f_i} > 2 \obserr +  L(n) \metric^p_{\inspace}(s_{N_n+1},s_i) \, \forall i \in \{1,...,N_n\}. 
\label{eq:Lupdatecond}
\end{equation}
This is a condition one could test for in $\mathcal O(d \, N_n)$ steps. If the condition is met, we can change the presupposed H\"older constant  to assume a value such that the condition is no longer met. To this end, we can choose the updated H\"older constant $L({n+1})$ to be 
%
\begin{equation}
	L({n+1}) := \max\left\{ L(n), \max_{i=1,...,N_n} \frac{\abs{\tilde f_{{N_n}+1}-\tilde f_i} -  2 \obserr}{\metric^p_{\inspace}(s_{N_n+1},s_i)} \right\}
\label{eq:Lupdateeq}
\end{equation}
which can be computed in $\mathcal O(M \, N_n)$, assuming the metrics can be evaluated in the order of $\mathcal O(M)$ computational steps. 
%
Note, if we set $L(0) = L(1) := 0$, we are guaranteed that our incremental estimate $L({n+1}) $ coincides with the empirical H\"older constant estimate $$L({\data_{n+1}} ) :=  \max \bigl\{0,\max_{i,j =1,...,N_{n+1}, i\neq j} \frac{\metric_\outspace(\tilde f_i,\tilde f_j) -2 \obserr}{\metric^p_{\inspace}(s_i,s_j)} \bigr\}.$$ As we will see below, this estimate always is a lower bound on the true lowest H\"older constant $L^* = \sup_{x \in \inspace, x' \in \inspace\backslash \{x\}} \frac{\abs{f(x) - f(x')}}{\metric_{\inspace}(x,x') }$.



\subsubsection{Generalisations and properties} 
Next, we will examine our lazy update rule in a generalised setting. Here, we assume the target function is a mapping $f: \inspace \to \outspace$ where the input space $\inspace$ is endowed with a pseudo-metric $\metric_\inspace$ and the output space $\outspace$ is an additive group, endowed with a pseudo-metric $\metric_\outspace$ that is translation-invariant, i.e. $\metric_\outspace(y+\tau,y'+\tau) = \metric_\outspace(y,y'),\forall y,y',\tau \in \outspace$. Throughout the remainder of the article, we will frequently take maxima over sets. 
Now allowing for non-definite pseudo-metrics, we need to ensure none of the denominators can be zero. For notiational convenience, for two sets $S,S' \subset \inspace$ of inputs we define  $$U(S,S') := \{(s,s') \in S \times S' | \metric_\inspace(s,s') >0\}$$ as the set of input pairs that the pseudo-metric can tell apart and define $U_n := U(G_n,G_n) $.
For notational simplicity, we assume $\max(\emptyset) = -\infty$ and $\max(S\cup \{
-\infty\}) = \max (S)$ for any set $S$.

We assume that a priori, that is, before having seen any data, we know a lower bound $\underline L$ on the optimal H\"older constant $L^* \in \Real_{\geq 0}$.
At time $n$ we desire to entertain a deterministic belief $\ell_n \geq 0$ about H\"older constant $L^*$ which is consistent with the data $\data_n$ observed so far.
In this context consistency means that 
$\ell_n $ is at least as large as the empirical estimator $\ell({\data_n};\hestthresh,\underline L)$ of the constant defined as follows:  
%\begin{equation}
%\ell(\data_n;\hestthresh,\underline L)  :=  \begin{cases} \underline L &, \text{ if } N_n =\abs{\data_n} \leq 1\\ 
% \max \Bigl\{ \max_{i,j =1,...,N_n, i\neq j} \frac{\metric_\outspace(\tilde f_i,\tilde f_j) - \hestthresh}{\metric_{\inspace}^p(s_i,s_j)},\underline L \Bigr\}.
% &, \text{ otherwise }. \end{cases}
%\end{equation}
\begin{equation}\label{eq:lazyconstupdaterule_batch}
\ell(\data_n;\hestthresh,\underline L)  := 
 \max \Bigl\{ \underline L, \max_{(s,s') \in U_n} \frac{\metric_\outspace(\tilde f_i,\tilde f_j) - \hestthresh}{\metric_{\inspace}^p(s_i,s_j)}\Bigr\}.
\end{equation}

 Here $\hestthresh \in \Real$ is a parameter which, consistent with our considerations above, by default we set to $\hestthresh = 2 \metric_\outspace(0, \obserr)$. This imposes a tolerance margin that is necessary since the presence of observational error $\obserr$ could cause the \emph{empirical H\"older constant} $\ell(\data_n;0,\underline L)$ to overestimate the actual smallest constant $L$ in an unbounded manner. 
Parameter $\underline L$ is an a priori specifiable lower bound on the true best H\"older constant $L^*$. In the absence of further domain-specific a priori knowledge on this bound we can always set $\underline L=0$ (since negative constants are not meaningful in the definition of H\"older continuity). 
In the absence of observational errors (i.e. in case $\obserr =0$) it should be clear that the sequence of estimates $\seq{\ell(\data_n;\lambda,0)_n;0,\underline L)}{n \in \nat}$ converges to $L^*$ if the sequence of grids $\seq{G_n}{n \in \nat}$ converges to a dense subset of the domain $\inspace$ (and provided that indeed $\underline L \leq L^*$).

However, in the presence of observational errors, $\ell(\data_n;0,\underline L)$ can in general overestimate the H\"older constant with arbitrarily high error. Fortunately, parameter $\hestthresh$ can prevent this:
Provided that $\hestthresh$ is chosen sufficiently large and that the a priori lower bound is a valid lower bound (i.e. $\underline L \leq L^*$), it is easy to show that the empirical estimates of the constant converge and never exceed the true optimal H\"older constant $L^*$. 
This is formalised as follows:

\begin {lem} \label{lem:constadaptation_boundedness} Let $\metric_\outspace: \outspace^2 \to \Real$ be a translation-invariant pseudo-metric on an abelian group $(\outspace,+)$.n Define $L^*$ to be the best H\"older constant of the target $f$.
With the definitions as above, we have:
\begin{itemize}
\item \textbf{(I)} If $D \subset D'$ then $\ell(D;\hestthresh,\underline L) \leq \ell(D';\hestthresh,\underline L)$.
\end{itemize}
If $\hestthresh \geq 2 \metric_\outspace(0,\obserr)$ we have:
\begin{itemize}
\item \textbf{(II)} $\forall n \in \nat: \ell(\data_n; \hestthresh,\underline L) \leq \max\{L^*,\underline L\}$ and  
\item \textbf{(III)} the sequence \seq{\ell(\data_n; \hestthresh,\underline L)}{n \in \nat} of empirical estimators monotonically converges to a number less than or equal to $\max\{L^*,\underline L\}$ provided that $\data_{n} \subset \data_{n+1}, \forall n \in \nat$.
\end{itemize}
\end{lem}
\begin{proof}
\textbf{(I)} This follows trivially from the fact that $\max S \leq \max M$ for two sets $S \subseteq M$. \\
\textbf{(II) }
We know that the $\tilde f_i$ are noisy observations of the ground truth $f(s_i)$ $(i=1,...,N_n$. That is, there exist $\phi$ such that $(i)$ $\tilde f = f + \phi$  and $(ii)$ $\metric_\outspace(0,\phi_i) \leq \metric_\outspace(0,\obserr)$ $(i =1,\ldots, N_n)$. 
Furthermore, note that in general (cf. Lem. \ref{lem:bilinaddtransinvgroup}), in additive groups $(G,+)$ endowed with a translation-invariant pseudo-metric $\metric_G$ one has $(iii)$ $\forall g,g',g'' \in G: \metric_G (g+g',g'') \leq \metric_G (g+g',g)+\metric_G(g,g'') = \metric_G (g',0)+\metric_G (g,g'')$. Finally, we remind ourselves that $(iv)$ $L^* \geq \sup_{x,x' \in \inspace, \metric_\inspace(x,x')>0} \frac{\metric_\outspace\bigl(f(x),f(x') \bigr)}{\metric_{\inspace}^p(x,x')}$ $\geq \ell(S;0,0)$ for any finite subset of input points $S \subset \inspace$.

These facts allow us to reason as follows:
%\begin{tiny}
\begin{align}
&\ell(\data_n;\hestthresh,\underline L)  %& = \max \bigl\{ 0, \max_{i,j =1,...,N_n, i\neq j} \frac{\metric_\outspace(\tilde f_i,\tilde f_j) - \hestthresh}{\metric_{\inspace}^p(s_i,s_j)} \bigr\} \\ 
\\
 &\stackrel{(i)}{=} \max \Bigl\{ \underline L, \max_{(s,s') \in U_n} \frac{\metric_\outspace\bigl( f(s) + \phi(s),f(s')+\phi(s')\bigr) - \hestthresh}{\metric_{\inspace}^p(s,s')} \Bigr\}\\
  &\stackrel{(iii)}{\leq} \max \Bigl\{ \underline L, \max_{(s,s') \in U_n} \frac{\metric_\outspace\bigl( f(s),f(s') \bigr)  + \metric_\outspace(0,\phi(s)) + \metric_\outspace(0,\phi(s')) - \hestthresh}{\metric_{\inspace}^p(s,s')} \Bigr\}\\
    &\stackrel{(ii)}{\leq} \max \Bigl\{ \underline L,  \max_{(s,s') \in U_n} \frac{\metric_\outspace\bigl( f(s),f(s') \bigr)  }{\metric_{\inspace}^p(s,s')} +\frac{2 \metric_\outspace(0,\obserr) - \hestthresh}{\metric_{\inspace}^p(s,s')} \Bigr\}\\
        &\stackrel{(iv)}{\leq}  \max \Bigl\{ \underline L, L^* +\max_{(s,s') \in U_n} \frac{2 \metric_\outspace(0,\obserr) - \hestthresh}{\metric_{\inspace}^p(s,s')} \Bigr\}.
\end{align} 
%\end{tiny}
Hence, for $\ell(\data_n;\hestthresh,\underline L) \leq \max\{L^*,\underline L\}$ to hold it suffices to choose parameter $\hestthresh$ such that \newline
%
$L^* +\max_{(s,s') \in U_n}  \frac{2 \metric_\outspace(0,\obserr) - \hestthresh}{\metric_{\inspace}^p(s,s)} \leq L^*$.
Since the denominators cannot be negative, this holds if the nominator $2 \metric_\outspace(0,\obserr) - \hestthresh$ is not positive, i.e. if $2 \metric_\outspace(0,\obserr) \leq \hestthresh$. \\
\textbf{(III)} is entailed by applying the monotone convergence theorem in conjunction with (I) and (II).
\end{proof}
For time $n$, let $S_{n+1} := G_{n+1} \backslash G_n$ be the set of new sample inputs.
Similarly to above, we can define an incremental update rule recursively as follows: 
\begin{align} \label{eq:Hoelconstlazyupdateincr}
\ell_{n+1} &:= \max\Bigl\{\ell_n, \max_{(s,s') \in U(G_n, S_{n+1})} \frac{\metric_\outspace\bigl(\tilde f(s),\tilde f(s')\bigr) - \hestthresh}{\metric_{\inspace}^p(s,s')},\\
&\max_{(s,s') \in U(S_{n+1}, S_{n+1})} \frac{\metric_\outspace\bigl(\tilde f(s),\tilde f(s')\bigr) - \hestthresh}{\metric_{\inspace}^p(s,s')} \Bigr\},
\end{align} for $n \in \nat$ 
and where 
$\ell_0 := \underline L$. 
The effort for computing $\ell_{n+1}$ is in the order of $\mathcal O\bigl(M (\abs{S_{n+1}} N_n+ \abs{S_{n+1}}^2)\bigr)$ where $M$ denotes the effort for evaluating the pseudo-metrics.
By construction, we have \[\ell_n =\ell(\data_n;\hestthresh,\underline L), \forall n \in \nat.\] Therefore, our convergence and boundedness results about the sequence
 $\seq{\ell(\data_n;\hestthresh,\underline L)}{n \in \nat}$ readily apply to the incrementally computed sequence of estimators $\seq{\ell_n}{n \in \nat}$.

%\begin{rem} Above, we have assumed that the input space pseudo-metric was a full metric. This was done for notational convenience to keep the statements of our maximisations terse. The modifications to these statements to accommodate a non-definite input pseudo-metric is straight-forward: all we need to do is to only consider input pairs $s_i,s_j$ in the maximisation steps that give rise to non-zero denominators $\metric_\inspace(s_i,s_j)$. For example, \end{rem}
%\subsection{Learning as inference over group mappings}
\label{sec:group_learners}
Groups provide basic algebraic structures. At this point we will look into several properties that apply to learning of mappings between abelian groups. Since the concept of an abelian group is more general than that of a vector space, all our results readily apply to learning of mappings between vector spaces.


Next, we will consider the following framework for learning mappings $f:\inspace \to \outspace$ between such groups $(\inspace,+), (\outspace,+)$: 

At \emph{stage} or \emph{time} $n \in \nat$, we assume we have access to a \textit{sample} or \textit{data set} $\data_n:= \{\bigl( s_i, \tilde f_i\bigr) \vert i=1,\ldots, N_n \} $ containing $N_n \in \nat$ (possibly erroneous) sample images $\tilde f_i \in \outspace$ of function $f$ at sample input $s_i \in \inspace$. Here the term \emph{erroneous} refers to the situation of bounded additive corruption of the sample images. That is,  $f_i = \tilde f(s_i) = f(s_i) + \phi(s_i)$ where $\phi: \inspace \to \outspace$ is a stochastic or deterministic perturbation we refer to as \emph{(observational) noise} and $\tilde f$ defines the noise-corrupted version of $f$ we can observe. For the purpose of worst-case analysis, we assume the noise to be bounded, i.e. $\metric_\outspace(\phi(x),0) \leq \obserr(x) \leq \obserrbnd <\infty$ for all inputs $x$ of interest.
We refer to $\obserr$ as the \emph{ observational error} and to $\obserrbnd \in \Real_{\geq 0}$ as the \emph{observational error bound}.
To aide our discussion, we define $G_n =\{s_i | i =1,\ldots,N_n\}$ to be the \textit{grid} of sample inputs contained in $\data_n$. Subscript $n$ aides our exposition when we consider sequences $\seq{\data_n}{n \in \nat}$ of data sets indexed by $n$.

Let $I \subseteq \inspace$ be some domain of interest.
The task of learning the \emph{target} $f: I  \to \outspace$ on the basis of data $\data_n$ consists in the generation of a computable \emph{predictor} $\predfn: I \to \outspace$ that somehow approximates the target. Since the predictor is constructed on the basis of the data, we can construe computation of predictions $\predfn(x)$ as \emph{inferences} over function values. Therefore, any algorithm for computing $\predfn(x)$ will be referred to as an \emph{inference rule}. In this lingo, \emph{learning} is the construction of inference rules over function images and machine learning is the discipline of conceiving computatable inference rules.  
That is, a \emph{learner} $\ell$ constitutes a computable mapping $\ell: \data_n \mapsto \predfn$.

For the sake of analysis, the quality of the inference has to be quantifiable. To this end, we will assume the groups $\inspace, \outspace$ are endowed with (pseudo-) metrics $\metric_\inspace,\metric_{\outspace}$, respectively. The output pseudo-metric $\metric_\outspace$ allows us to quantify the error of a prediction as per $\metric_{\outspace} \bigl(\predfn(x),f(x) \bigr) $.
Furthermore, the input pseudo-metric $\metric_\inspace$ allows us to quantify the similarity between a test input $x \in I\subseteq \inspace$ and the learning experience given by the data. Intuitively, a sensible learner might perform well on test inputs $x$ that are similar to the learning experience. That is, we might expect the prediction error  $\metric_{\outspace} \bigl(\predfn(x),f(x) \bigr) $ to decrease with decreasing dissimilarity to the data measured by $\dist(x,\data_n) := \inf_{s \in G_n}  \metric_\inspace(x,s)$. It might not be surprising that we will be able to take advantage of H\"older continuity properties in order to establish such a desirable behaviour of an inference rule. 
In particular, they will prove key in the examination of uniform convergence of the \emph{worst-case prediction error} 
% 
$\sup_{x\in I} \metric_{\outspace} \bigl( \predfn(x), f(x) )$ as $n \to \infty$ in situations where 
the data becomes increasingly dense, i.e. where $\dist(G_n,I):= \sup_{x \in I} \inf_{s \in G_n} \metric_\inspace(x,s)  \stackrel{n \to \infty}{\rightarrow} 0$. In particular, we will construct a class of H\"older continuous inference rules that are universal approximators in the sense that they can guarantee a vanishing worst-case error in the limit of dense grids as long as the target $f$ is continuous (but not necessarily H\"older).
As a preliminary step, we will next provide convergence guarantees for certain H\"older continuous inference rules in situations where the target is indeed H\"older.

\subsubsection{Convergence properties of sample-consistent learners of H\"older continuous group mappings}

It will be convenient to restrict our attention to groups $\inspace$ that are \emph{translation-invariant} with respect to their metric $\metric$. That is, we assume $ \metric(g+g',g''+g') = \metric(g,g''),\forall g,g',g'' \in \inspace$.

In such groups the following holds true:

\begin{lem} Let $\metric : \inspace^2 \to \Real$ be a translation-invariant pseudo-metric on an abelian group $(\inspace,+)$.
We have 
\begin{itemize}
\item \textbf{(I)} $\forall g,g',g'' \in \inspace: \metric(g+g',g'') \leq \metric(g,g'')+\metric(g',0)$.
\item \textbf{(II)} $\forall g \in \inspace: \metric(g,0) = \metric(-g,0)$.
\item \textbf{(III)} If $g'' = g+g'$ and $\metric(g'',g) \leq b$ then $\metric(0,g') \leq b$.
\end{itemize}
\begin{proof}
(I): $\forall g,g',g'' \in \inspace: \metric(g+g',g'') \leq \metric(g+g',g)+\metric(g,g'') = \metric(g',0)+\metric(g,g'').$
(II): $\forall g \in \inspace: \metric(g,0) = \metric(g+(-g),-g) = \metric(0,-g) =\metric(-g,0).$
(III): Follows trivially from translation-invariance.
\end{proof}
\label{lem:bilinaddtransinvgroup}
\end{lem}
 
\begin{defn} [Sample-consistency] A predictor $\predfn : \inspace \to \outspace$ is consistent with sample $\data_n$ (up to error $E: \inspace \to \Real$) iff $\forall (s_i,\tilde f_i,\obserr(s_i) ) \in \data_n: \metric_{\outspace}\bigl(\predfn(s_i),\tilde f(s_i)\bigr) \leq E(s_i) $. An inference is called \emph{sample-consistent} if for every possible sample $\data_n$ the generated predictor $\predfn$ is consistent with $\data_n$. We denote the set of all functions consistent with sample $\data_n$ by $\mathcal K (\data_n)$.
\label{def:sampleconsistency}
\end{defn}

\begin{lem}\label{lem:groups_sampleconsandobserr} As always, let $\obserr(\cdot) $ denote the bound on the observational error.
If $\predf_n$ is sample-consistent up to error $E$ with respect to set data $\data_n$ then, for any sample input $s \in G_n$, we have $\metric_\outspace\bigl(f(s), \predfn(s) \bigr) \leq E(s) + \metric_\outspace(0,\obserr(s))$.  Given an upper bound  $\obserrbnd \in \Real$ on the observational error  such that $\obserrbnd \geq \obserr(x),\forall x$ then sample-consistency up to error $E$ entails that true function is contained in the $E(s) + \obserrbnd$-balls around the predictions of observed inputs. That is, for all $s \in G_n$ we have:
\begin{equation}
 f(s) \in \ball{ E(s) + \obserrbnd}{\predfn(s)} = \{y \in \outspace | \metric_\outspace(\predfn(s),y ) \leq E(s) + \obserrbnd \}.
\end{equation}
\end{lem}
\begin{proof}
Sample consistency up to $E$ implies that for each sample input $s \in G_n$ $\Metrico{\predfn}{\tilde f(s)} \leq E(s)$.
On the other hand, bounded observational errors mean that for $s$ 
there is $\nu_s \in \outspace, \Metrico 0 {\nu_s} \leq \obserrbnd$ such that we have $\Metrico{f(s)}{ \tilde f(s) } = \Metrico{f(s)}{ f(s)+\nu_s } \leq \Metrico{0}{\nu_s } \leq \obserrbnd$.
Since the output pseudo-metric $\Metrico \cdot \cdot$ adheres to the triangle inequality, we have 
$\Metrico{f(s)}{\predfn(s)} \leq \Metrico{f(s)}{ \tilde f(s) } + \Metrico{\predfn(s)}{ \tilde f(s) } \leq E(s) + \obserrbnd $.
\end{proof}

Next, we will establish that sample-consistent and continuous H\"older inference rules can learn any H\"older continuous target.

\begin{defn}
Relative to a pseudo-metric $\metric$, a sequence of sets $(\mathcal S_n)_{n \in \nat}$ converges to a set $\mathcal S$ iff  for all $s \in S$ there is a vanishing function $r_s:\nat \to \Real$, $r(n) \stackrel{n \to \infty} {\to} 0$ such that we have:  $$ \forall n \in \nat: \inf_{s_n \in \mathcal S_n} \metric(s_n,s)\leq r(n).$$
If the same function $r(\cdot)$ can be given for all $s \in \mathcal S$ then the convergence is uniform and we say that $(\mathcal S_n)$ converges to $\mathcal S$ uniformly with rate $r$. In the case of uniform convergence with rate of at most $r$ we have: $$\sup_{s \in \mathcal S} \inf_{s_n \in \mathcal S_n} \metric(s_n,s) \leq r(n) \stackrel{n \to \infty}{\to} 0$$.
\end{defn}


\begin{thm} \label{thm:convergenceifboundedconstandsamplecons}
Let $\seq{\data_n}{n \in \nat}$ be a data set sequence with pertaining grid sequence $\seq{G_n}{n \in \nat}$, \\ with $G_n \subset G_{n+1} \subset \inspace (n \in \nat)$ converging to a dense subset of input domain $\inspace$. Assume all of the following: 
\begin{enumerate} 
\item The data sets $\data_n$ ($n \in \nat$) all have bounded observational error with $\obserrbnd := \sup_x \Metrico{0}{\obserr(x)} \in \Real_{\geq 0}$.
\item The target $f: I \subseteq \inspace \to \outspace$ is H\"older continuous with $f \in \hoelset {L^*}{}  p$.
\item The predictors $\predfn$ $(n \in \nat)$ are sample-consistent up to error $E:\inspace \to \Real$ with $\bar E := \sup_x E(x) \leq \infty$.
\item The predictors are  H\"older continuous with H\"older constants bounded below $\bar L \in \Real_{\geq0}$. That is, 
$\exists \bar L \geq 0 \forall n \in \nat: \predfn \in \hoelset {L(n)}{} p$  $ \wedge$ $ L(n) \leq \bar L$.
\end{enumerate}


Then, we have: 
\begin{itemize}
\item \textbf{(I)}The sequence of predictors $\seq \predfn {n \in \nat}$ converges to the ground truth $f$ pointwise up to error $\bar E$. That is, $\forall \epsilon \geq 0,  x \in \inspace \exists n_0 \in \nat \forall n \geq n_0: \metric_{\outspace}\bigl(\predfn(x), f(x) \bigr) \leq \epsilon + \bar E + \obserrbnd$.
%\item 

\item \textbf{(II)} If the grid sequence $\seq{G_n}{n \in \nat }$ converges to the domain $\inspace$ uniformly then $\seq \predfn {n \in \nat}$ converges to the target $f$ uniformly up to error bound $\bar E + \obserrbnd$. 

That is, $\forall \epsilon \geq 0 \exists n_0 \in \nat \forall n \geq n_0, x \in \inspace: \metric_{\outspace}\bigl(\predfn(x), f(x) \bigr) \leq \epsilon + \bar E+\obserrbnd$.

\item \textbf{(III)} If the uniform convergence of the grid sequence as per (II) occurs with a rate of at most $r:\nat \to \Real$ then we have:  $$\forall n \in \nat \forall x \in I \subseteq\inspace: \metric_\outspace\bigl(\predfn(x) , f(x)\bigr)\in [0,(\bar L+ L^*) r(n)^p+ \bar E + \obserrbnd].$$ That is, as $n \to \infty$,  convergence (up to error $\bar E +\obserrbnd)$ of the predictors to the target occurs uniformly at a rate of at most $(\bar L+ L^*) r(n)^p$.
\end{itemize}

\begin{proof}
For $x \in I\subseteq \inspace$ let $\xi_n^x$ denote the nearest neighbour of $x$ in grid $G_n$. That is, $\xi_n^x = \arg\inf_{s \in G_n} \metric_\inspace(x,s)$.

\textbf{(I)} 
Since $G_n$ is assumed to converge to a dense subset of the domain $I$, we have$ \metric(x,\xi_n^x )^p
\stackrel{n \to \infty}{\longrightarrow} 0$.
Furthermore, since the the predictors are sample-consistent on grid $G_n$ up to error $E$ and since $\xi_n^x \in G_n$ we can appeal to Lem. \ref{lem:groups_sampleconsandobserr} (as well as to the triangle inequality) to reason as follows:
%there always is some $\phi(\xi_n^x) \in  \ball{\bar E}{0} = \{ y \in \outspace \,| \, \metric_\outspace(y,0 ) \leq \bar E \}$ such that (i) 
$\metric_\outspace\bigl(\predfn(x) , f(\xi_n^x)  \bigr) 
\leq 
\metric_\outspace\bigl(\predfn(x) , \predfn(\xi_n^x)  \bigr) + \metric_\outspace\bigl(\predfn(\xi_n^x) , f(\xi_n^x)  \bigr)
\stackrel{Lem. \ref{lem:groups_sampleconsandobserr}}{\leq} 
\metric_\outspace\bigl(\predfn(x) , \predfn(\xi_n^x)  \bigr) + E(\xi_n^x) + \obserr(\xi_n^x)$.
Hence,
\begin{equation}
\forall x: \metric_\outspace\bigl(\predfn(x) , f(\xi_n^x)  \bigr) \leq
\metric_\outspace\bigl(\predfn(x) , \predfn(\xi_n^x)  \bigr) + \bar E + \obserrbnd.
\label{ineq:ewlfjhflsape}
\end{equation}
%
Thus, for $x \in \inspace$:
%
$\metric_\outspace\bigl(\predfn(x) , f(x)\bigr) \leq \metric_\outspace\bigl(\predfn(x) ,  f(\xi_n^x)  \bigr) + \metric_\outspace\bigl(f(\xi_n^x) , f(x)\bigr)$
$\leq 
\metric_\outspace\bigl(\predfn(x) , \predfn(\xi_n^x)  \bigr) + \bar E + \obserrbnd+ \metric_\outspace\bigl(f(\xi_n^x) , f(x)\bigr)$
%$\norm{\predfn(x) - f(x)} \leq \norm{\predfn(x) - f(\xi_n)  } + \norm{f(\xi_n) - f(x)} \stackrel{(i)}{=}\norm{\predfn(x) - \predfn(\xi_n)  } + \norm{f(\xi_n) - f(x)}$
$\stackrel{(\dagger)}{\leq} (\bar L+ L^*) \metric_\inspace(x,\xi_n^x )^p + \bar E +\obserrbnd$
$\stackrel{n \to \infty}{\longrightarrow} \bar E +\obserrbnd$.
Here ($\dagger$) leverages the H\"older continuity assumptions. \\

\textbf{(II)} The proof is a trivial extension of (I). Let $\epsilon \geq 0.$ We show $\exists n_0 \in \nat \forall n \geq n_0, x \in \inspace: \metric_{\outspace}\bigl(\predfn(x), f(x) \bigr) \leq \epsilon + \bar E +\obserrbnd$.  
Since $G_n$ converges to $I$ uniformly, $\exists n_0 \forall n\geq n_0\forall x \in I: \metric_\inspace(x,\xi_n^x)^p \leq \frac{\epsilon}{2 \max\{\bar L, L^* \}}$. Hold such $n_0$ fixed. Then for all $n \geq n_0, x \in I$ we have 
$\metric_\outspace\bigl(\predfn(x) , f(x)\bigr) \newline \stackrel{(*)}{\leq} (\bar L+ L^*) \metric_\inspace(x,\xi_n^x )^p + \bar E +\obserrbnd \leq \epsilon + \bar E +\obserrbnd$. Here (*) follows from (\ref{ineq:ewlfjhflsape}) and ($\dagger$) in complete analogy to our derivations for case (I) above. 

\textbf{(III)} By assumption of uniform convergence of $\seq{G_n}{n \in \nat}$ with rate $r(n)$, $\sup_{x \in I} \inf_{s_n \in G_n} \metric_{\inspace} (x,s_n) =\sup_{x \in I} \metric_{\inspace} (x,\xi_n^x) \leq r(n)$. The rest follows from (*).
\end{proof}
\end{thm}


%% ============= ONLINE LEARNING CASE IMPORTANT 4 CONTROL


Above theorem considers the case where the data becomes dense in the domain and we are interested in the worst-case prediction error. Relevant to considerations in a setting where online-learning is utilised to control a trajectory $\seq{x_n}{n \in \nat}$ of states is the situation where we are only interested in the error on this specific trajectory in the limit of large time steps $n$. For trajectories on which the prediction error vanishes we will be able to establish convergence of the trajectory to the desired goal state.
In preparation of these considerations, we will establish the following facts:


\begin{lem}
Assume we are given a trajectory $\seq{x_n}{n \in \nat}$ of inputs with $x_n \in \inspace$ where input space $\inspace$ can be endowed with a shift-invariant measure. Furthermore, assume the sequence  is bounded, i.e.  
$\metric_\inspace(x_n,0) \leq \beta$ for some $\beta \in \Real_+$ and all $n \in \nat$.
Finally assume the inputs of the available data coincide with the complete history of past inputs, i.e. $G_n = \{ x_i | i \in \nat, i < n\}$.
Then we have: \[ \dist(G_n,x_n) = \min\{\metric_\inspace(g,x_n) | \, g \in G_n\} \stackrel{n \to \infty}{\longrightarrow} 0.\]
\begin{proof}
The intuition behind the following proof is that if the distances were not to converge, there was an infinite number of disjoint balls around the input points that summed up to infinite volume. This however, would be a contradiction to the presupposed boundedness of the sequence.
We formalise this intuition as follows:
We can rephrase the desired convergence statement as 
\begin{equation}
\forall \epsilon > 0 \exists n \in \nat \forall m \geq n : \dist(x_{m}, G_{m}) \leq \epsilon.
\end{equation} 
For contradiction, assume that
\begin{equation}
\exists \epsilon > 0 \forall n \in \nat \exists m(n) \geq n : \dist(x_{m(n)}, G_{m(n)}) > \epsilon.
\end{equation} 
Hold such an $\epsilon >0$ fixed and choose any $n \in \nat$. 
By definition of $G_{m(n)} =\{ x_i | i < m(n)\} $ we have:
\eqn{eq:i34kjjk3}{\forall i < m(n) : \metric_\inspace(x_{m(n)},x_i) > \epsilon.}

Let $C_n := \bigcup_{i < n} \ball{\frac \epsilon 2}{x_i} $ be the union of all $\frac \epsilon 2$-balls around each point in $G_n$ and define $\bar I = \bigcup_{n \in \nat} C_n$.
By definition, each $x_n$ is contained in $\bar I$.
Since sequence $(x_n)_{n \in \nat}$ is bounded, $\bar I $ has a finite volume relative to some positive, shift-invariant measure $\mu$. I.e. $\mu(\bar I) < \infty$ (e.g. choose the Lebesgue measure for $\mu$). Furthermore, $\mu(C_n) \leq \sum_{i <n} \mu(B_i) \leq \mu(\bar I)< \infty$ where $B_i := \ball{\frac \epsilon 2}{x_i}$. Owing to the assumed shift-invariance, we can assign the same measure $M$ each ball, i.e. $M:=\mu(B_1) = \mu(B_n)\forall n \in \nat$. Thus, $\mu(C_n) \leq n M$.
Define $q:= \ceil{\frac{\mu(\bar I)}{M}} \in \nat$. This is an upper bound on the number of disjoint balls of measure $M$  that can be contained in $\bar I$. Intuitively, since this number is finite, there cannot be an infinite number of non-intersecting balls around the elements of the sequence $(x_n)_{n \in \nat}$. More formally our argument proceeds as follows:
Choose $n > q+1$. Statement (\ref{eq:i34kjjk3}) yields:
\eqn{eq:i34kjjk33}{\forall i \in \{1,...,n\} \exists p(i) \geq i \forall j \leq p(i): \metric_\inspace(x_{p(i)},x_j) > \epsilon.}  Define a permutation $\pi$ such that $\pi(p(1)) \leq \ldots \leq \pi(p(n))$. 
With Statement (\ref{eq:i34kjjk33}) it follows that \\
${\metric_\inspace(x_{\pi(p(i))},x_{\pi(p(j))}) > \epsilon}$ , $\forall i,j =1,...,n, i < j$. Thus, we conclude the disjointness conditions $B_{\pi(p(i))} \cap B_{\pi(p(j))} = \emptyset , \forall i,j =1,...,n, i \neq j$. 
Hence,  $\mu(\bar I) \geq \mu(C_{\pi(p(n))}) \geq \mu(C_{\pi(p(1))}) + \sum_{i=1}^n \mu(B_{\pi(p(i))}) \newline = \mu(C_{\pi(p(1))}) +  n M > \mu(C_{\pi(p(1))}) + (q+1) M \geq\mu(C_{\pi(p(1))}) + \mu(\bar I)$, where the last inequality follows from the fact that  $M q= M \ceil{\frac{\mu(\bar I)}{M}} \geq \mu(\bar I)$. Since $\mu(C_{\pi(p(1))}) \geq 0$, we have concluded the false statement $\mu(\bar I) > \mu(\bar I)$.

%Hence, $\mu(\bar I) \geq \mu(C_{\pi(N_n)+1}) = \mu(C_{\pi(N_1)}) + \sum_{i=1}^n \mu(B_{\pi(N_i)}) = \mu(C_{\pi(N_1)}) +  n M > \mu(C_{\pi(N_1)}) + (m+1) M >\mu(C_{\pi(N_1)}) + \mu(\bar I)$ by definition of $m= \ceil{\frac{\mu(\bar I)}{M}}$. But since $\mu(C_{\pi(N_1)}) \geq 0$, we conclude the false statement $\mu(\bar I) > \mu(\bar I)$.
\end{proof}
\label{lem:bndseq_entails_distgridvanish}
\end{lem} 


\begin{lem}
As before, let the observational noise level be bounded by $\obserr$ and assume the output- space pseudo-metric to be translation-invariant. Assume the predictors $\predfn(\cdot)$ all are H\"older with H\"older constants bounded from above by some number $\bar L \in \Real_+$. 
Furthermore, we assume sample-consistency up to error $E$ (cf. Def. \ref{def:sampleconsistency}) where we assume bounded prediction errors $E_s = E + \obserrbnd$ (cf. Lem. \ref{lem:groups_sampleconsandobserr}) at sample inputs with $\sup_x E_s(x) \leq \bar E_s \in \Real$.
For some $L^* \in \Real_+, p \in [0,1]$, let target $f$ be $L^*-p$-H\"older up to error $E_h$. That is, there is an $L^*-p$-H\"older function $\phi \in \hoelset {L^*} { } p$ and a function $\psi$ such that $\forall x: f(x) = \phi(x)+\psi(x), \, \sup_x \metric_\outspace\bigl(0,\psi(x)\bigr) \leq \bar E_h \in \Real$.

Assume we are given a trajectory $\seq{x_n}{n \in \nat}$ of inputs that is bounded, i.e. where 
$\metric_\inspace(x_n,0) \leq \beta$ for some $\beta \in \Real_+$ and all $n \in \nat$.
Furthermore, assume $\data_{n+1} = \data_n \cup \{ \bigl(x_n, \tilde f(x_n)\bigr) \}$ and thus, $G_n = \{ x_i | i \in \nat, i < n\}$.
Then the prediction error on the sequence vanishes up to the level of sample-consistency and H\"older continuity in the following sense:
 \[\metric_\outspace\bigl(\predfn(x_n),f(x_n) \bigr) \stackrel{n \to \infty}{\longrightarrow} [0,\bar  E_s + 2 \bar E_h].\]
In particular, in case the observations are error-free ($\tilde f = f$) and assuming the target is H\"older continuous then the prediction error vanishes. That is,
\[\metric_\outspace\bigl(\predfn(x_n),f(x_n) \bigr) \stackrel{n \to \infty}{\longrightarrow}0.\]
\begin{proof}

Let $\xi_n := \argmin_{g \in G_n} \metric_\inspace(x_n,g)$ denote the nearest neighbour of $x_n$ in $G_n = \{x_1,...,x_{n-1}\}$.
We have assumed that $\predfn$ is $L(n)-p$- H\"older with $L(n) \leq \bar L$ for some $\bar L \in \Real_{\geq 0}$.
Since sequence $(x_n)$ is bounded, Lem. \ref{lem:bndseq_entails_distgridvanish} is applicable. 
Hence, $\forall n \in \nat: \metric_\inspace (x_n,\xi_n) \to 0$ as $n \to \infty$.

Let $y \in \outspace$ such that $\predfn(\xi_n) + y = f(\xi_n)$. The sample-consistency assumption renders Lem. \ref{lem:groups_sampleconsandobserr} applicable, allowing us to conclude that 
$\metric_\outspace\bigl(\predfn(\xi_n) ,  f(\xi_n)  \bigr) \leq \obserrbnd + E(\xi_n)$. Hence, by Lem. \ref{lem:bilinaddtransinvgroup}.(III), $\metric_\outspace(0,y) \leq \obserrbnd + E(\xi_n)$. Appealing to Lem. \ref{lem:bilinaddtransinvgroup}.(I), we see that 
$\metric_\outspace\bigl(\predfn(x_n) ,  f(\xi_n)  \bigr) \leq \metric_\outspace\bigl(\predfn(x_n) ,  \predfn(\xi_n)+y  \bigr) \leq 
\metric_\outspace\bigl(\predfn(x_n) ,  \predfn(\xi_n) \bigr) + \metric_\outspace(0,y)$. Thus, 

$(i) \,\,\,\, \metric_\outspace\bigl(\predfn(x_n) ,  f(\xi_n)  \bigr) \leq \metric_\outspace\bigl(\predfn(x_n) ,  \predfn(\xi_n) \bigr) + \obserrbnd + E(\xi_n)$.



Moreover: $(ii)$ There is a maximal constant $\bar L \in \Real$, such that for any $n \in \nat $, the predictor $\predfn$ is $L(n)-p$-H\"older with $L(n) \leq \bar L$. 
  
In conclusion,
$0\leq\metric_\outspace\bigl(\predfn(x_n) , f(x_n)\bigr) \leq \metric_\outspace\bigl(\predfn(x_n) ,  f(\xi_n)  \bigr) + \metric_\outspace\bigl(f(\xi_n) , f(x_n)\bigr) \stackrel{(i)}{\leq}\metric_\outspace\bigl(\predfn(x_n) , \predfn(\xi_n) \bigr) + \obserrbnd + E(\xi_n) + \metric_\outspace\bigl(f(\xi_n) , f(x_n)\bigr) \leq \metric_\outspace\bigl(\predfn(x_n) , \predfn(\xi_n) \bigr) +\obserrbnd + E(\xi_n) + \metric_\outspace\bigl(\phi(\xi_n) , \phi(x_n)\bigr) + 2 \bar E_h 
\newline
\stackrel{(ii)}{\leq} (\bar L+ L^* ) \metric_\inspace(x_n,\xi_n )^p + \obserrbnd + E(\xi_n) + 2 \bar E_h \stackrel{(ii)}{\leq} (\bar L+ L^* ) \metric_\inspace(x_n,\xi_n )^p + \bar E_s + 2 \bar E_h \stackrel{n \to \infty}{\longrightarrow} \bar E_s + 2 \bar E_h $.
\end{proof}
\label{lem:vanisishingseqprederr_groups}
\end{lem} 




