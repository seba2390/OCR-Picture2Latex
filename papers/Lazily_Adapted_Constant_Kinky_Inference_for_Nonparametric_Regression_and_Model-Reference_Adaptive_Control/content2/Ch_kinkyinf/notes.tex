\section{Diverse Notes}

\subsection{Trapezoid rule} 
Non-uniform grid:
\[ \hat S = \frac{1}{2} \sum_{i=1}^N \bigl(\,(t_{i+1} - t_i) \,(f(t_{i+1}) - f(t_i)\, \bigr).\]

Uniform grid with interval length $h$:
\[ \hat S = \frac{h}{2} \sum_{i=1}^N  f(t_{i+1}) - f(t_i).\].

Error:
\[E(f) \leq \sum_{j=1}^N \frac{h^3}{12} \max_{x \in J_j} \abs{f''(x)}\]
where $J_j = [t_j,t_{j+1}]$.

If $f$ real-valued, then 

\[ \xi_1 \in J_1,...,\xi_N \in J_N: E(f) = \sum_{j=1}^N - \frac{h^3}{12} f''(\xi_j) \] 

\subsection{Numerical error}
Due to finite precision of computers need to add a margin of numerical error to the results to be truly conservative when implemented. Such error bounds can be found with standard numerical analysis.
%
%\subsection{H\"older continuity of the map $x \mapsto \metric(x,s)^p$ for $p \in (0,1], s\in I$}
%
 %
%We endeavor to show that $x \mapsto L \norm{ x -s}^p$ is H\"older continuous with coefficient $L$ and exponent $p$.
%
%To this end, we will show that $(x,y) \mapsto \norm{x-y}^p$, for $p \in (0,1] $, is a metric.  This can be utilized to show that
%$x \mapsto \norm{ x -s}^p$ is a H\"older continuous function.
%
%To establish the metric property, we note that any positive definite function that is subadditive gives rise to a canonical metric. Formally :
%
%
%To prove $(x,y) \mapsto \norm{x-y}^p$ is a metric, we need to establish a few facts. 
%Firstly, we remind ourselves of the following well-known fact:
%\begin{lem} \label{lem:pd_n_concave_subadditive}
 %A nonnegative, concave function $g:\Real_{\geq 0} \to \Real$ with $g(0) = 0$ is subadditive. 
 %That is, $\forall x,y \in \Real_{\geq 0}: g(x+y) \leq g(x) + g(y)$. 
 %
 %\begin{proof}
%If $x = y = 0$ then subadditivity trivally holds:  $0=g(x+y) \leq g(x) + g(y) = 0$.
%So, let $x, y \in \Real_+$ such that $x >0 \vee y >0$.
%We have, $g( x +y) = \frac{x}{x+y} g(x+y) + \frac{y}{x+y} g(x+y) \leq g(\frac{x}{x+y} (x+y) ) +  g(\frac{y}{x+y}(x+y)) = g(x) + g(y)$.
%Taking into account that $\frac{x}{x+y}, \frac{x}{x+y} \in [0,1]$, the last inequality can be seen as follows:
%
 %Since $g$ is concave we have 
%$\forall p \in [0,1], x \in \Real: g(p x) =  g(px + (1-p) 0) \geq p g(x) + (1-p) g(0) \geq p g(x)  $. The last inequality follows from $g(0) \geq 0$.
%\end{proof}
%\end{lem}
%
%\begin{lem} \label{lem:x2p_pdNsubadd}
 %Let $h: \begin{cases} \Real_{\geq 0} \to \Real_{\geq 0},\\ \, x \mapsto x^p\end{cases}$, for $p \in (0,1] $. $h$ is positive definite and subadditive. 
 %That is, (i) $h(0) = 0 $ and  $h(x) > 0, \forall x \neq 0$ and (ii) $\forall x,y \in \Real_{\geq 0}: h(x+y) \leq h(x) + h(y)  $.
 %Moreover, $h$ is concave. If $p \in (0,1)$, h is strictly concave. 
 %
%\begin{proof}
%\textit{Pos. def. :} $h(0) = 0$.  Also $\lim_{x \to 0_+} h(x) =0$. Since $\nabla h (x) = p h^{p-1}(x) >0 $ for $x >0$, $h$ is strictly monotonously increasing on $\Real_+$. Hence, $h(x) > 0, \forall x >0$. 
%
%\textit{Concavity:} If $p =1$, $h$ is linear and hence, concave. If $p \in (0,1)$, $\nabla^2 h(x) = p (p-1) h(x)^{p-2} > 0$ regardless of $x$.
%
%\textit{Subadditivity:} Follows directly with Lem. \ref{lem:pd_n_concave_subadditive} on the basis of established positive definiteness and concavity.
  %
%\end{proof}
%\end{lem}
%
%
%
%\begin{lem}\label{lem:hoeldererror_metric}
%Let $p \in (0,1]$.
%With definitions as in Thm. \ref{thm:subaddmetric1}, we assume that set $G$ is endowed with a quasi- metric $d$. Function
%$\metric: \begin{cases} \inspace \times \inspace \to \Real_{\geq 0} \\ (x,y) \mapsto \Bigl(d(x,y)\Bigr)^p \end{cases}$ is a quasi-metric on $G$.
%If $d$ is a metric then so is $\metric$.
%\begin{proof}
 %By Lem. \ref{lem:x2p_pdNsubadd}, $x\mapsto x^p$ is pos. def. and subadditive. Therefore, positive definiteness and the triangle inequality of $d$ readily extend to $\metric$ as follows: 
%
%\textit{Pos. def.:}
%Let $x=0$. $\metric(x,x) = d(x,x)^p = 0^p = 0$. If $x \neq 0$ then $k :=d(x,x) \neq 0$. Hence   $\metric(x,x) = d(x,x)^p = k^p \neq 0$.
%
%\textit{Triangle inequality:}
%Choose arbitrary $x,y,z \in \inspace $. We have $\metric(x,z) = d(x,z)^p \leq (d(x,y) + d(y,z) \bigr)^p \leq d(x,y)^p + d(y,z)^p = \metric(x,y) + \metric(y,z)$. Here, the first inequality followed from the triangle inequality of quasi-metric $d$ and the second from subadditivity properties established in Lem. \ref{lem:x2p_pdNsubadd}.
%
%\textit{Symmetry:} If $d$ is a metric it is symmetric. Hence, $\metric(x,y) = d(x,y)^p = d(y,x)^p = \metric(y,x), \forall x,y \in \inspace $ in which case $\metric$ also is symmetric.
%\end{proof}
%\end{lem}
%
%Before proceeding we establish a slight generalization of the well-known \textit{reverse triangle inequality}:
%
%\begin{lem}[Reverse Triangle Inequality] \label{thm:revtriangle}
%Let  $\inspace$ be a set and $\metric : \inspace^2 \to \Real$ a symmetric function that fulfills the triangle inequality. 
%That is,  $\forall x,y,z \in \inspace: \metric (x,y) = \metric(y,x) \wedge \metric(x,z) \leq \metric(x,y) + \metric(y,z)$.
%
%Then \[\forall x,y,z \in \inspace: \abs{\metric(x,y) - \metric(z,y)} \leq \metric(x,z).\]
%\begin{proof}
%
%For contradiction, assume 
%$\abs{\metric(x,y) - \metric(z,y)}>\metric(x,z)$ for some $x,y,z \in \inspace$.
%This is implies  
%$
%(i)\,\,\, \metric(x,y) - \metric(z,y)>\metric(x,z) 
%\, \, $  or  $
  %\,\,  (ii) \,\,\,\, \metric(z,y)-\metric(x,y) >\metric(x,z)$.
%Both inequalities contradict the triangle inequality:
%$(i) \Leftrightarrow \metric(x,y)  >\metric(x,z) \metric(z,y) $ and 
%$(ii) \Leftrightarrow  \metric(z,y) > \metric(x,z) + \metric(x,y) =  \metric(z,x) + \metric(x,y)$.
%
%\end{proof}
%\end{lem}
%
%
%\begin{thm}
%Let $\metric$ be a metric on set $G$. For arbitrary $s \in \inspace $ we define $\phi_s: \inspace \to \Real $ as $\phi_s: x \mapsto \metric (x,s) $.
%$\phi_s $ is Lipschitz with respect to metric $\metric$. That is, \[\forall x,y \in \inspace : \abs{\phi_s(x) - \phi_s(y) } \leq \metric (x,y). \]
%\begin{proof}
%$\abs{\phi_s(x) - \phi_s(y)} = \abs{\metric(x,s) - \metric(y,s)} \stackrel{Thm. \ref{thm:revtriangle}} {\leq} \metric(x,y), \forall x,y,s \in \inspace $.
%\end{proof}
%\end{thm}
%Finally, combining this theorem with Lem. \ref{lem:hoeldererror_metric} immediately establishes that mappings of the form $\metric(\cdot,s)^p$ and hence, of the form
%$\norm{\cdot -s}^p$ are H\"older continuous with exponent $p$ ($\in (0,1]$):
%
%\begin{thm}
%Let $\metric$ be a metric on set $G$ and let $p \in (0,1], L \geq 0$. For arbitrary $s \in \inspace $ we define $\phi_s:\begin{cases}  \inspace \to \Real \\ x \mapsto L \, \bigl(\metric (x,s) \bigr)^p\end{cases}$.
%$\phi_s $ is H\"older with exponent $p$. That is, \[\forall x,y \in \inspace : \abs{\phi_s(x) - \phi_s(y) } \leq L \, \metric (x,y)^p. \]
%In particular, for any norm $\norm \cdot$ on $G$ and $s \in \inspace $, mapping $x \mapsto L \norm{x-s}^p$ is H\"older with constant $L$ and exponent $p$.
%\begin{proof}
%By Lem. \ref{lem:hoeldererror_metric}, $\metric(\cdot,\cdot)$ is a metric on $G$. Hence, Lem. \ref{thm:revtriangle} is applicable yielding:
%$\abs{\phi_s(x) - \phi_s(y)} = L \, \abs{\metric(x,s)^p - \metric(y,s)^p} \stackrel{Lem. \ref{thm:revtriangle} } {\leq} \metric(x,y)^p, \forall x,y,s \in \inspace$. The last sentence concerning the norms follows from the fact that a mapping $(x,y) \mapsto \norm{x-y}$ defines a metric.
%\end{proof}
%\end{thm}


\subsection{Nice to know but no longer necessary..}
\begin{lem}
Let $(G,+)$ be an Abelian group with neutral element $0$. Let $A: G \to \Real$ be a mapping such that 
$\forall x,y \in G: $
\begin{enumerate}
\item $A(-x) = A(x)$ (A is even),
	\item $A(x) \geq 0 $ and $A(x) = 0 \Leftrightarrow x =0$ (A is pos. def.),
	\item $ A(x+y) \leq A(x) + A(y)$ (A is subadditive). 
\end{enumerate}
Then $\metric (x,y) := A(x-y)$ is a proper metric on $\inspace$.
\begin{proof}
Let $x,y,z \in G$.
Symmetry and pos. def. properties of $\metric$ follow directly from commutativity of $G$ and the assumption that 
$A$ is positive definite and even: Firstly, in a group, we have $x-y = -(y-x)$. Hence, $\metric(x,y) = A(x-y) = A(-(y-x) ) \stackrel{A \, \text{even}}{=} A(y-x) = \metric(x,y)$. Positive definiteness follows directly from the fact that $A$ is positive defininte.

Next, the triangle inequality of our metric is shown: 
 Leveraging subadditivity of $A$,  we have 
$\metric(x,z) = A(x-z) = A\bigl((x-y) + (y-z)\bigr) \leq A(x-y)+A(y-z) = \metric(x,y) +\metric(y,z)$.
\end{proof}
\end{lem}

\begin{thm} \label{thm:subaddmetric1}
Let $p \in (0,1)$ and $(G,+)$ be an Abelian group and $A: \inspace \to \Real_{\geq 0}$ be an even, pos. def. and subadditive mapping. 
Then $\metric: \begin{cases} \inspace \times \inspace \to \Real_{\geq 0} \\ (x,y) \mapsto \bigl(A(x-y)\bigr)^p \end{cases}$ is a metric between elements of $G$.
\begin{proof} 
Let $x \in G$.
$\metric(x,x) = A(0)^p = 0$ due to pos. def. of $A$ and $x \mapsto x^p$.
If $|G| = 1$ we are done. Otherwise assume the contrary and let $y \in \inspace \backslash \{x\}$.
$\metric(x,y) = A(x-y)^p = A(y-x)^p = \metric(y,x)$ since $ \inspace $ is Abelian. This proves symmetry.

Finally we prove the triangle inequality. This is also straight forward, since we have established that $A$ and $x \mapsto x^p$
are subadditive:
Hence, 
$\forall x,y,z \in  \inspace : \metric(x,z) = A(x-y + y-z)^p \leq (A(x-y)  + A(y-z))^p \leq \bigl(A(x-y)\bigr)^p  + \bigl(A(y-z)\bigr)^p$.
\end{proof}
\end{thm}

\subsection{Background}

\begin{defn}[Convergence of function sequences]
For $N \in \nat$, let $f_N : I \to \Real$. $f_N \stackrel{N \to \infty}{\longrightarrow} f$ \textit{pointwise} if
\[\forall \epsilon >0 \exists N_0 \in \nat \forall N\geq N_0\forall x\in I: \abs{f_N(x) - f(x)} < \epsilon. \] 
 
\end{defn}

\begin{defn}[Uniform convergence of function sequences]
For $N \in \nat$, let $f_N : I \to \Real$. $f_N \stackrel{N \to \infty}{\longrightarrow} f$ \emph{uniformly} if
\[ \forall x \in I \forall \epsilon >0  \exists N_0 \in \nat \forall N\geq N_0: \abs{f_N(x) - f(x)} < \epsilon. \] 
\end{defn}

The key distinction between both variants of convergence is that in pointwise convergence, the convergence rate can depend on $x$. By contrast, in uniform convergence the rate is independent of function input $x$.


For our purpose the following definition of a dense set suffices:
\begin{defn}[Dense] Let $(\mathcal X,\norm{\cdot})$ be a $d$-dimensional vector space endowed with norm $\norm{\cdot}$.  
A set $D \subset \mathcal X$ is ($\norm{\cdot}-$) dense in a set $I \subset \mathcal X$ if every  $\norm{\cdot}-$ open subset of $I$ contains at least one point of $D$. That is:
$\forall x \in I, \epsilon > 0 \exists \xi \in D: \norm{\xi - x} < \epsilon$.     
\end{defn}


\begin{defn}[Convergences of set sequences]
Let $(\mathcal X,\norm{\cdot})$ be a $d$-dimensional vector space endowed with norm $\norm{\cdot}$.  
A sequence of sets $(Q_N)_{N \in \nat}$, $Q_N \subset \mathcal X$, converges to set $I \subset \mathcal X$ if the following two conditions hold:

1. For all $ x \in I$ we can chose a sequence $(q_N)_{N \in \nat}$ (where $q_N \in Q_N, \,\forall N \in \nat$) which converges to $x$: $q_N \stackrel{N \to \infty}{\longrightarrow} x$. That is, $\forall x \in I \exists q_1 \in Q_1, q_2 \in Q_2,... \forall \epsilon > 0 \exists N_0 \forall N \geq N_0:  \norm{q_N - x}  < \epsilon$


2. Every sequence $(q_N)_{N \in \nat}$ with $q_N \in Q_N \, (\forall N \in \nat)$ converges to a subset of points $S \subset I$. The latter is to say, $\forall \epsilon > 0 \exists N_0 \forall N \geq N_0: \sup\{ \norm{q_N - \xi} | \xi \in S\} < \epsilon$.

STIMMT DAS????
\end{defn}


\begin{defn}[Uniform convergence of sequences of sets]
Let $(\mathcal X,\norm{\cdot})$ be a $d$-dimensional vector space endowed with norm $\norm{\cdot}$.  
A sequence of sets $(Q_N)_{N \in \nat}$, $Q_N \subset \mathcal X$, $\abs{Q_N} < \infty$ converges to set $I \subset \mathcal X$ \textit{uniformly} if we have:
\[\forall \epsilon > 0 \exists N_0 \in \nat \forall N \geq N_0 \forall x \in I \exists q \in Q_N : \norm{q-x} < \epsilon. \]

If the sequence consists of finite sets, we can rewrite the condition as 
\[\forall \epsilon > 0 \exists N_0 \in \nat \forall N \geq N_0 \forall x \in I : \min_{q \in Q_N} \norm{q-x} < \epsilon. \]

\end{defn}


\section{Norm equivalences}

\begin{thm}
Let $V$ be a $d$-dimensional vector space, $d < \infty$. 
We have \[\forall p, q \in \nat_+ \cup \infty, v\in V: \norm{v}_p \leq \norm{v}_q \leq d^{\frac{1}{q} - \frac{1}{p}} \norm{v}_p.  \]
\end{thm}
In particular:
\begin{cor}
Let $V$ be a $d$-dimensional vector space, $d < \infty$. 
We have $\forall  v\in V:$ 
\begin{enumerate}
\item $\norm{v}_2 \leq \norm{v}_1 \leq \sqrt d \norm{v}_2$, 
\item $\norm{v}_\infty \leq \norm{v}_2 \leq \sqrt d \norm{v}_\infty$, 
\item $\norm{v}_\infty \leq \norm{v}_1 \leq d \norm{v}_\infty$.
\end{enumerate}
\end{cor}
