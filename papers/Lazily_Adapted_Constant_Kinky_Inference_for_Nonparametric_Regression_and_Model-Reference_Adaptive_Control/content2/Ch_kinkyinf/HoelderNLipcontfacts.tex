\section{H\"older and Lipschitz continuity for inference}%\label{sec:HoeldNLipfacts}
\label{sec:Hoelder_brief}
%\subsubsection{Introduction}

 
H\"older continuous functions are uniformly continuous functions that may exhibit infinitely many points of non-differentiability and yet are sufficiently regular to facilitate. That is, they have properties that make it possible to make assertions of global properties on the basis of a finite function sample. 

H\"older continuity is a generalisation of Lipschitz continuity.  Lipschitz properties are widely used in applied mathematics to establish error bounds and, among many other, finds application in optimisation \cite{Shubert:72,direct:93} and quadrature \cite{Baran2008,curbera1998,dereich2006} and is a key property to establish convergence properties of approximation rules in (stochastic) differential equations \cite{kloedenandplaten1992,Gardiner2009}. 

Throughout this thesis, we take advantage on a priori knowledge about H\"older continuity to perform inference.
For instance, in Ch. \ref{ch:contcoll}, we leverage this regularity property to infer the absence of negative function values of a criterion function in the context of collision detection. In Ch. \ref{ch:kinkyinf}, H\"older continuity restricts the posterior hypothesis space of our kinky inference learning method and facilitates learning. 

 Upon submission of this work, we have become aware of related work published in mathematical and operations research journals \cite{Cooper2006,Cooper1995,Zabinsky2003}. For instance, Zabinsky et. al. \cite{Zabinsky2003} consider the problem of estimating a one-dimensional Lipschitz function (with respect to the canonical norm-induced metric). Similar to the analysis we employ to establish our guarantees, they use a pair of bounding functions and make predictions by taking the average of these functions. While we have developed our kinky inference rule independently, it can be seen as a generalisation of their approach. Our method provides extensions to H\"older continuous multi-output functions over general, multi-dimensional (pseudo-) metric spaces, can cope with with erroneous observations and inputs, can fold in additional knowledge about boundedness, learn parameters from data and provide different guarantees such as (uniform) convergence of the prediction uncertainty. 
As part of the analysis of our method (see Ch. \ref{ch:kinkyinf} and Ch. \ref{ch:KIderivations}), we construct bounding functions we refer \textit{ceiling} and \textit{floor} functions. The construction of similar functions is a recurring theme that, in the standard Lipschitz context, can be found in global optimisation \cite{Shubert:72} and quadrature \cite{Baran2008} as well as in the analysis of linear splines for function estimation \cite{Cooper1995}. Cooper \cite{Cooper2006,Cooper1995} utilises such upper and lower bound functions in a multi-dimensional setting to derive probabilistic PAC-type error bounds \cite{Valiant1984} for a linear interpolation rule. He assumes the data is sampled uniformly at random on a hypercube domain. This precludes the application of his results to much of our control applications where the data normally is collected along continuous trajectories visited by a controlled plant. Our inference rule is different from his and our guarantees do not rely on distributional assumptions. Indeed, to fit many contexts of relevance, we have  However, in future work, we would be interested in how far some of Cooper's statistical analysis techniques could be harnessed to translate our own guarantees to stochastic settings.

None of the aforementioned works seems to consider interval-bounded input uncertainty and observational errors, folds in additional knowledge such as boundedness or considers inference over multi-output functions as we do.
Finally, we are not aware of any work that employs these methods in the context of learning in dynamical systems, control or collision avoidance as we do. 

At the bottom line, while it seems that the general idea that is at the heart of our kinky inference framework is not new, our approach offers several novel extensions and generalisations that are important for our 
control and coordination applications. Most importantly for the aims of this thesis, the merger with control will yields a variety of new adaptive controllers with guarantees of boundedness we also believe to be novel.

The remainder of this section is structured as follows. Firstly, we will go over basic definitions and engage in some preliminary derivations that will be of importance throughout the thesis.
Afterwards we will briefly talk about utilising H\"older properties for conservative inference over extrema -- that is, for bounded optimisation. While we do not claim novelty on any of the results we provide proofs for in this section, we had not found them in the literature and hence, had to derive them on our own.
%In particular, we will look into the inference problems of bounded optimisation and quadrature. That is, based on the knowledge of a sample of a H\"older continuous function we will derive upper and lower bounds on its maximum, minimum and definite Riemann integral. In addition we will touch upon adaptive algorithms that can utilise knowledge of the function's H\"older properties to guide exploration. That is, since the regularity of the function allows us to provide interval bounds on the uncertainty of the function, the magnitude of the uncertainty can be a guide to identify regions that are to be sampled next in order to maximise the shrinkage of the provided error bounds. 
%\jcom{The latter is more the 1-dim case, which I may or may not include}\\

\subsubsection{Basic facts and derivations}
Before proceeding, we will establish basic properties of H\"older continuity. 

\begin{defn} 
Let $(\inspace ,\metric_\inspace ), (\outspace , \metric_\outspace )$ be two metric spaces and 
$I \subset \inspace$ be an open set. A function $f: \inspace \to \outspace $ is called (L-p-) \emph{H\"older} 
continuous on $I \subset \inspace$ if there exists a \emph{(H\"older) coefficient} $L[I] \geq 0$ and \emph{(H\"older) 
exponent} $p\geq 0$ such that 
\[\forall x,x' \in I : \metric_\outspace \bigl(f(x),f(x')\bigr) \leq L[I] \, \bigl( \metric_\inspace (x,x') \bigr)^p. \]
We denote the space of all L-p- H\"older functions by $\hoelset L \metric p$.
 \end{defn}
H\"older functions are known to be uniformly continuous. 
A special case of importance is the class of $L$-\textit{Lipschitz} functions. These are H\"older continuous 
functions with exponent $p=1$. In this context, coefficient $L$ is referred to as\textit{ Lipschitz constant} or \textit{Lipschitz number}.

\begin{ex}[Square root function]\label{ex:sqrtfctHoelder}
As an example of a Hoelder function that is not Lipschitz we can consider $x \mapsto \sqrt x$ on domain $I = [0+\epsilon,c]$ where 
$c >\epsilon \geq 0 $. For $\epsilon >0 $ the function is Lipschitz with $L[I] = \sup_{x \in I} \frac{1} {2 \sqrt{x}}$. We can see that the 
coefficient grows infinitely large as $\epsilon \to 0$. By contrast, the function is H\"older continuous 
with H\"older coefficient $L[I]=1$ and exponent $p=\frac 1 2 $ for any bounded $I \subset \Real$.
We can see this as follows: Let $\epsilon =0,$ $x,y \in I$ and, without loss of generality,  $y \geq x$. Let $\xi := \sqrt{x}, \gamma := \sqrt{y}$ and thus, $\gamma \geq \xi$. We have:
$\xi \leq \gamma $ $\Leftrightarrow 2 \xi^2 \leq 2\xi\gamma$ $\Leftrightarrow \gamma^2 - 2 \xi\gamma + \xi^2  \leq \gamma^2 - \xi^2$ $\Leftrightarrow (\gamma-\xi)^2  \leq \gamma^2-\xi^2$ $\Leftrightarrow \abs{\gamma-\xi}^2  \leq \abs{\gamma^2-\xi^2}$
 $\Leftrightarrow \abs{\gamma-\xi}  \leq \sqrt{\abs{\gamma^2-\xi^2}}$  $\Leftrightarrow \abs{\sqrt{x}-\sqrt{y}}  \leq \abs{y-x}^{\frac{1}{2}}$. Since, $x,y$ were chosen arbitrarily, we have shown H\"older continuity as desired.
\end{ex}
 
Most commonly, one considers H\"older continuity for the special case of the standard metric induced by a norm, i.e.  $\metric(x,x') = \norm{x-x'}$.
For a function $f: \inspace \to \outspace$, the H\"older condition becomes:
 \[\forall x,x' \in I : \norm{f(x)-f(x')}_\outspace \leq L[I] \, \norm{x-x'}_\inspace^p. \]

Similarly, we can consider H\"older continuity for each output component: 

\begin{defn} \label{def:outputwisehoelder}
Let $\outspace \subseteq \Real^m $ and $\inspace$ be a space endowed with a metric (or indeed a semi-metric) $\metric_\inspace$. Then, the function $f: \inspace \to \outspace$ is output-component-wise H\"older continuous with exponent $p$ and constant $L \in \Real^m_{\geq 0}$ if $f \in \hoelset L {\metric_\inspace} p$ where
\[\hoelset L {\metric_\inspace} p:= \bigl\{ \phi: \inspace \to \outspace \,\ | \, \forall j \in \{1,...,m \}, \forall x,x' \in \inspace: \abs{\phi_j(x) - \phi_j(x')} \leq L_j \metric_\inspace(x,x') \bigr\}\] 
is the set of all functions whose component functions are H\"older continuous with respect to input space metric $\metric_\inspace$ and an output space metric that is induced by the canonical norm $\metric_{\outspace} (x,x')= \abs{x-x'}$.  
\end{defn}
%
%
%
%\subsection{Hoelder continuity of the maps  $x \mapsto  L \, \metric(x,s)^p$ for $p \in (0,1], s\in I, L \geq 0$}
%
 %
%We endeavor to show that $x \mapsto L \norm{ x -s}^p$ is Hoelder continuous with coefficient $L$ and exponent $p$. This will become important since these functions will serve as essential building blocks of our conservative Hoelder function estimates. 
%
%To this end, we will show that $(x,y) \mapsto \norm{x-y}^p$, for $p \in (0,1] $, is a metric.  This can be utilized to show that
%$x \mapsto \norm{ x -s}^p$ is a Hoelder continuous function.
%
%To establish the metric property, we note that any positive definite function that is subadditive gives rise to a canonical metric. Formally :
%
%
%To prove $(x,y) \mapsto \norm{x-y}^p$ is a metric, we need to establish a few facts. 
%Firstly, we remind ourselves of the following well-known fact:
%\begin{lem} \label{lem:pd_n_concave_subadditive}
 %A nonnegative, concave function $g:\Real_{\geq 0} \to \Real$ with $g(0) = 0$ is subadditive. 
 %That is, $\forall x,y \in \Real_{\geq 0}: g(x+y) \leq g(x) + g(y)$. 
 %
 %\begin{proof}
%If $x = y = 0$ then subadditivity trivally holds:  $0=g(x+y) \leq g(x) + g(y) = 0$.
%So, let $x, y \in \Real_+$ such that $x >0 \vee y >0$.
%We have, $g( x +y) = \frac{x}{x+y} g(x+y) + \frac{y}{x+y} g(x+y) \leq g(\frac{x}{x+y} (x+y) ) +  g(\frac{y}{x+y}(x+y)) = g(x) + g(y)$.
%Taking into account that $\frac{x}{x+y}, \frac{x}{x+y} \in [0,1]$, the last inequality can be seen as follows:
%
 %Since $g$ is concave we have 
%$\forall p \in [0,1], x \in \Real: g(p x) =  g(px + (1-p) 0) \geq p g(x) + (1-p) g(0) \geq p g(x)  $. The last inequality follows from $g(0) \geq 0$.
%\end{proof}
%\end{lem}
%
%\begin{lem} \label{lem:x2p_pdNsubadd}
 %Let $h: \begin{cases} \Real_{\geq 0} \to \Real_{\geq 0},\\ \, x \mapsto x^p\end{cases}$, for $p \in (0,1] $. $h$ is positive definite and subadditive. 
 %That is, (i) $h(0) = 0 $ and  $h(x) > 0, \forall x \neq 0$ and (ii) $\forall x,y \in \Real_{\geq 0}: h(x+y) \leq h(x) + h(y)  $.
 %Moreover, $h$ is concave. If $p \in (0,1)$, h is strictly concave. 
 %
%\begin{proof}
%\textit{Pos. def. :} $h(0) = 0$.  Also $\lim_{x \to 0_+} h(x) =0$. Since $\nabla h (x) = p h^{p-1}(x) >0 $ for $x >0$, $h$ is strictly monotonically increasing on $\Real_+$. Hence, $h(x) > 0, \forall x >0$. 
%
%\textit{Concavity:} If $p =1$, $h$ is linear and hence, concave. If $p \in (0,1)$, $\nabla^2 h(x) = p (p-1) h(x)^{p-2} > 0$ regardless of $x$.
%
%\textit{Subadditivity:} Follows directly with Lem. \ref{lem:pd_n_concave_subadditive} on the basis of established positive definiteness and concavity.
  %
%\end{proof}
%\end{lem}
%
%
%
%\begin{lem}\label{lem:hoeldererror_metric}
%Let $p \in (0,1]$.
%With definitions as in Thm. \ref{thm:subaddmetric1}, we assume that set $G$ is endowed with a quasi- metric $d$. Function
%$\metric: \begin{cases} \inspace \times \inspace \to \Real_{\geq 0} \\ (x,y) \mapsto \Bigl(d(x,y)\Bigr)^p \end{cases}$ is a quasi-metric on $G$.
%If $d$ is a metric then so is $\metric$.
%\begin{proof}
 %By Lem. \ref{lem:x2p_pdNsubadd}, $x\mapsto x^p$ is pos. def. and subadditive. Therefore, positive definiteness and the triangle inequality of $d$ readily extend to $\metric$ as follows: 
%
%\textit{Pos. def.:}
%Let $x=0$. $\metric(x,x) = d(x,x)^p = 0^p = 0$. If $x \neq 0$ then $k :=d(x,x) \neq 0$. Hence   $\metric(x,x) = d(x,x)^p = k^p \neq 0$.
%
%\textit{Triangle inequality:}
%Choose arbitrary $x,y,z \in \inspace $. We have $\metric(x,z) = d(x,z)^p \leq (d(x,y) + d(y,z) \bigr)^p \leq d(x,y)^p + d(y,z)^p = \metric(x,y) + \metric(y,z)$. Here, the first inequality followed from the triangle inequality of quasi-metric $d$ and the second from subadditivity properties established in Lem. \ref{lem:x2p_pdNsubadd}.
%
%\textit{Symmetry:} If $d$ is a metric it is symmetric. Hence, $\metric(x,y) = d(x,y)^p = d(y,x)^p = \metric(y,x), \forall x,y \in \inspace $ in which case $\metric$ also is symmetric.
%\end{proof}
%\end{lem}
%
%Before proceeding we establish a slight generalization of the well-known \textit{reverse triangle inequality}:
%
%\begin{lem}[Reverse Triangle Inequality] \label{thm:revtriangle}
%Let  $\inspace$ be a set and $\metric : \inspace^2 \to \Real$ a symmetric function that fulfills the triangle inequality. 
%That is,  $\forall x,y,z \in \inspace: \metric (x,y) = \metric(y,x) \wedge \metric(x,z) \leq \metric(x,y) + \metric(y,z)$.
%
%Then \[\forall x,y,z \in \inspace: \abs{\metric(x,y) - \metric(z,y)} \leq \metric(x,z).\]
%\begin{proof}
%
%For contradiction, assume 
%$\abs{\metric(x,y) - \metric(z,y)}>\metric(x,z)$ for some $x,y,z \in \inspace$.
%This is implies  
%$
%(i)\,\,\, \metric(x,y) - \metric(z,y)>\metric(x,z) 
%\, \, $  or  $
  %\,\,  (ii) \,\,\,\, \metric(z,y)-\metric(x,y) >\metric(x,z)$.
%Both inequalities contradict the triangle inequality:
%$(i) \Leftrightarrow \metric(x,y)  >\metric(x,z) \metric(z,y) $ and 
%$(ii) \Leftrightarrow  \metric(z,y) > \metric(x,z) + \metric(x,y) =  \metric(z,x) + \metric(x,y)$.
%
%\end{proof}
%\end{lem}
%
%
%\begin{thm}
%Let $\metric$ be a metric on set $G$. For arbitrary $s \in \inspace $ we define $\phi_s: \inspace \to \Real $ as $\phi_s: x \mapsto \metric (x,s) $.
%$\phi_s $ is Lipschitz with respect to metric $\metric$. That is, \[\forall x,y \in \inspace : \abs{\phi_s(x) - \phi_s(y) } \leq \metric (x,y). \]
%\begin{proof}
%$\abs{\phi_s(x) - \phi_s(y)} = \abs{\metric(x,s) - \metric(y,s)} \stackrel{Thm. \ref{thm:revtriangle}} {\leq} \metric(x,y), \forall x,y,s \in \inspace $.
%\end{proof}
%\end{thm}
%Finally, combining this theorem with Lem. \ref{lem:hoeldererror_metric} immediately establishes that mappings of the form $\metric(\cdot,s)^p$ and hence, of the form
%$\norm{\cdot -s}^p$ are Hoelder continuous with exponent $p$ ($\in (0,1]$):
%
%\begin{thm}
%Let $\metric$ be a metric on set $G$ and let $p \in (0,1], L \geq 0$. For arbitrary $s \in \inspace $ we define $\phi_s:\begin{cases}  \inspace \to \Real \\ x \mapsto L \, \bigl(\metric (x,s) \bigr)^p\end{cases}$.
%$\phi_s $ is Hoelder with exponent $p$. That is, \[\forall x,y \in \inspace : \abs{\phi_s(x) - \phi_s(y) } \leq L \, \metric (x,y)^p. \]
%In particular, for any norm $\norm \cdot$ on $G$ and $s \in \inspace $, mapping $x \mapsto L \norm{x-s}^p$ is Hoelder with constant $L$ and exponent $p$.
%\begin{proof}
%By Lem. \ref{lem:hoeldererror_metric}, $\metric(\cdot,\cdot)$ is a metric on $G$. Hence, Lem. \ref{thm:revtriangle} is applicable yielding:
%$\abs{\phi_s(x) - \phi_s(y)} = L \, \abs{\metric(x,s)^p - \metric(y,s)^p} \stackrel{Lem. \ref{thm:revtriangle} } {\leq} \metric(x,y)^p, \forall x,y,s \in \inspace$. The last sentence concerning the norms follows from the fact that a mapping $(x,y) \mapsto \norm{x-y}$ defines a metric.
%\end{proof}
%\end{thm}
%
%
%
 %At first glance, the result may seem intuitive. However, it should be noted that, for $p >1$, $x \mapsto \norm{x -s}^p$ no longer is Hoelder with exponent $p$. This is consistent with the well-known fact that Hoelder functions with exponent $>1$ are constant functions.
%
%We conclude this section by the following theorem stating that any concatenation of Hoelder continuous functions is Hoelder continuous:
%
%\begin{thm} \label{thm:concathoelder}
%Let $(\statespace, \metric)$ be a metric space and $f,g : \statespace \to \statespace$ be two Hoelder continuous mappings with Hoelder constants $L_f, L_g$ and Hoelder exponents $p_f,p_g$, respectively.
%Then, the concatenation $h=f \circ g: \statespace \to \statespace $ is also Hoelder continuous with Hoelder constant $L_h:= L_f L_g^{p_f}$ and exponent $p_h:=p_g \, p_f$.
%That is, 
%\[\forall \state,\state' \in \statespace: \metric\bigl(h(\state),h(\state')\bigr) \leq L_h \bigl(\metric(\state,\state')\bigr)^{p_h}.\]
%\begin{proof}
%%\begin{align}
%$\metric\bigl(f \circ g(\state),f\circ g(\state')\bigr) \leq L_f  \Bigl(\metric(g(\state),g(\state'))\Bigr)^{p_f}$
%$\leq L_f  \Bigl(L_g \metric(\state,\state')^{p_g}\Bigr)^{p_f}$ $= L_f L_g^{p_f}   \Bigl(\metric(\state,\state')\Bigr)^{p_g\, p_f} $ where in the first step we were using Hoelder-continuity of $f$ and in the second, we were using Hoelder continuity of $g$ combined with the fact that $(\cdot)^{p_f}$ is a monotonically increasing  function. 
%
%\end{proof}
%\end{thm} 
%In conjunction with Hoelder properties of the square root function established in Ex. \ref{ex:sqrtfctHoelder}, Thm. 
%\ref{thm:concathoelder} immediately yields the following result:
%\begin{cor}
%If $f: \statespace \to \statespace $ is Hoelder continuous with constant $L_f$ and exponent $p_f$ then $\sqrt{f}$ also is Hoelder, having  Hoelder constant $\sqrt{L_f}$ and exponent $p_f$.
%\end{cor}


\begin{thm} \label{thm:hoelderconcat}
Let $(\statespace, \d)$ be a metric space and $f,g : \statespace \to \statespace$ be two H\"older continuous mappings with H\"older constants $L(f), L(g)$ and H\"older exponents $p_f,p_g$, respectively.
Then, the concatenation $h=f \circ g: \statespace \to \statespace $ is also H\"older continuous with H\"older constant $L(h):= L(f) L(g)^{p_f}$ and exponent $p_h:=p_g \, p_f$.
That is, 
\[\forall \state,\state' \in \statespace: \d\bigl(h(\state),h(\state')\bigr) \leq L(h) \, \bigl(\d(\state,\state')\bigr)^{p_h}.\]
\begin{proof}
%\begin{align}
$\d\bigl(f \circ g(\state),f\circ g(\state')\bigr)$ $\leq L(f)\,  \Bigl(\d(g(\state),g(\state'))\Bigr)^{p_f}$
$\leq L(f)\,  \Bigl(L(g)\, \d(\state,\state')^{p_g}\Bigr)^{p_f}$ 

$= L(f)\, L(g)\,^{p_f}   \Bigl(\d(\state,\state')\Bigr)^{p_g\, p_f} $ where in the first step we were using Hoelder-continuity of $f$ and in the second, we were using H\"older continuity of $g$ combined with the fact that $(\cdot)^{p_f}$ is a monotonically increasing  function. 
\end{proof}
\end{thm} 


%A special case of Hoelder continuity is Lipschitz continuity. A Lipschitz continuous function is Hoelder continuous with exponent 1.
%If the metric space is $(\Real,(x,y) \mapsto \abs{x-y})$, $f$ is Lipschitz with constant $L(f)$ if $\forall t,t': \abs{f(t)-f(t')} \leq L(f) \, \abs{t-t'}$. 

Several numerical methods, such as Lipschitz optimisation \cite{Shubert:72}, rely on the knowledge of a Lipschitz constant. In the more general case of H\"older continuous functions this will correspond to the need of knowing a H\"older constant and exponent. To avoid having to derive these for each new function from first principles, we establish the following collection of facts that allows us determine bounds on H\"older constants of combinations of functions with known H\"older parameters.
While we provide proofs for a restatement in the H\"older continuous setting, a number of the following statements have also been proven in \cite{Weaver1999} in the context of Lipschitz algebras.



%\begin{lem}[Lipschitz arithmetic] \label{lem:Liparithmetic}
%Let, $I,J \subset \Real_+$. Let $f : \Real \to \Real$ be Lipschitz on $I$ with Lipschitz number $L_I (f)$ 
%and $g :\Real \to \Real$ be Lipschitz on $J$ with Lipschitz number $L_J (g)$.
%We have:
%
%
%
%
%\begin{enumerate}
	%\item Mapping $t \mapsto |f(t)|$ is Lipschitz on $I$ with constant $L_I(f)$.
	%\item If $g$ is Lipschitz on all of $J=f(I)$ the concatenation $g \circ f: t \mapsto g(f(t))$ is Lipschitz on $I$ with Lipschitz constant 
	      %$L_I(g \circ f) \leq$ $L_J (g) \, L_I(f)$.
	%\item Let $r \in \Real$. $r \, f: x \mapsto r \, f(x)$ is Lipschitz on $I$ having a Lipschitz constant $L_I (r \,f) = |r| \, L_I(f)$.
	%\item $f+g: t \mapsto f(t) + g(t)$ has Lipschitz number at most $L_I(f) + L_J(g)$.
	%\item Let $m_f = \sup_{t\in \indset } f(t)$ and $m_g = \sup_{t\in \indset } g(t)$. Product function $f\cdot g: x \mapsto f(x) \, g(x)$ has a Lipschitz number on $I$ which is at most $(m_f \, L_J(g)+ m_g \, L_I(f))$.
	%\item Let $h(t) = \max\{f(t), g(t) \}, \forall t \in \indset  \cap J$. We have $L_{I \cap J}(h) = \max\{L_I(f),L_J(g)\}$.
	%\item Let $b := \inf_{t \in \indset }| f(t)| > 0$ and let 
	%$\phi(t) = \frac{1}{f(t)}, \forall t \in \indset $.  
	      %Then $L_I(\phi) \leq b^{-2} \, L_I(f)$ on $I$.  
	%\item $f$ cont. differentiable on I $\Rightarrow$ $L_I(f) = \sup_{t \in \indset } |\dot f(t)|. $ 
	 %\item Let $c \in \Real$, $f( t) = c, \forall t \in \indset $. Then $L_I(f) =0$.  
	%\item $L_I(f^2) \leq 2 \, s(f)\, L_I(f)$.
	%\item $f$ cont. differentiable $\Rightarrow \forall q \in \mathbb Q : L_I(f^q) = |q| \,\sup_{\tau \in \indset } |f^{q-1}(\tau) \, \dot f(\tau)| $.
%\end{enumerate}
%\begin{proof}
%
%\textbf{1)}  We show $|f|$ has the same Lipschitz number as $f$. Let $t,t' \in \indset $ arbitrary. 
%We enter a case differentiation:
%
%\textit{1st case: $f(t), f(t') \geq 0$}. 
%
%Hence, $\bigl| |f(t)| - |f(t')| \bigr| = \bigl| f(t) - f(t') \bigr|  \stackrel{f \, Lipschitz}{\leq} L_I(f) |t-t'|$.\\
%
%\textit{2nd case: $f(t) \geq 0, f(t') \leq 0$.} 
%
%Note, $|y| = - y$, iff $y \leq 0$. Hence,  $\bigl| |f(t)| - |f(t')| \bigr| \leq \bigl| |f(t)| + |f(t')| \bigr| $
%$= \bigl| |f(t)| - f(t') \bigr|  =  \bigl| f(t) - f(t') \bigr| \leq L_I(f) \, |t-t'|$.\\
%
%\textit{3rd case: $f(t) \leq 0, f(t') \geq 0$.} Completely analogous to the second case.\\
%
%\textit{4th case: $f(t), f(t') \leq 0$}. 
%
%$\bigl| |f(t)| - |f(t')| \bigr| = \bigl| f(t') - f(t) \bigr|= \bigl| f(t) - f(t') \bigr|  \stackrel{f \, Lipschitz}{\leq} L_I(f) |t-t'|$.\\
%
%The remaining points are proven in \cite{Weaver1999}.
%\textbf{2)}  For arbitrary $t,t' \in \indset $ we have:
%
%$\bigl| g(f(t)) - g(f(t'))| \bigr| \leq L_J(g) \, |f(t) - f(t') | \leq L_J(g) \, L_I(f)\, |t-t'|$ 
%where the two inequalities are due to the Lipschitz properties of $g$ and $f$, respectively.\\
%
%\textbf{3)}  For arbitrary $t,t' \in \indset , r \in \Real$ we have:
%
%$\bigl| r \, f(t) - r \, f(t')| \bigr| = |r|\, |f(t) - f(t')| \leq |r| \,L_I(f)\,  |t-t'|$.\\ 
%
%\textbf{4)}  For arbitrary $t,t' \in \indset , r \in \Real$ we have:
%
%$\bigl| g(t) + f(t) - (g(t') + f(t'))| \bigr| = \bigl| g(t)  - g(t') + f(t)- f(t')| \bigr|$ 
%$\leq \bigl| g(t)  - g(t')\bigr|  + \bigl| f(t)- f(t')| \bigr|$ $\leq (L_J(g)+L_I(f))\,  |t-t'|$.\\
%
%\textbf{5)}  Let  $t,t' \in \indset $, $d := f(t) - f(t')$.
%
%$\bigl| g(t) f(t) - g(t')  f(t') \bigr| = \bigl| g(t) (f(t') +d) - g(t') f(t') \bigr|$ 
%$= \bigl|\bigl( g(t) - g(t') \bigl)  f(t')+ g(t)  d \bigr|  $
%$\leq \bigl| g(t) - g(t') \bigr|  |f(t')|   + \bigl|g(t)\bigr| \,  |d|  $
%$\leq L_I(g) |t-t'|  |f(t')|   + \bigl|g(t)\bigr| \,  L_I(f) |t-t'|  $
%$\leq L_I(g) |t-t'|  \sup_{t' \in \indset } \{|f(t')|\}   + \sup_{t \in \indset }\{\bigl|g(t)\bigr|\} \,  L_I(f) |t-t'|  $
%$= \Bigl(L_I(g)  \sup_{t' \in \indset } \{|f(t')|\}   + \sup_{t \in \indset }\{\bigl|g(t)\bigr|\} \,  L_I(f)\Bigl) |t-t'|  $.\\
%
%\textbf{6)}  Proof in "` Nick Weaver, Lipschitz algebras"'.
%
%\textbf{7)}  Let  $t,t' \in \indset $.
%$\bigl| \frac{1}{f(t)} - \frac{1}{f(t')} \bigr|$ 
%$=\bigl| \frac{f(t')}{f(t') f(t)} -\frac{f(t)}{f(t') f(t)} \bigr|$ 
%$= \frac{\bigl|f(t')-f(t) \bigr|}{\bigl|f(t')\bigr| \bigr| f(t)\bigr|}$ 
%$\leq \frac{L_I(f) |t-t'|}{\inf_{t \in \indset } |f(t)|}$.\\
%
%\textbf{8)} Define $\ell := \sup_{t \in \indset } | \dot f(t) | = L_I(f)$. In two steps, we show that $\ell$ is the smallest Lipschitz constant.
%
%Firstly, we show that it is a Lipschitz constant: Let $t,t' \in \indset , t < t'$. Due to the mean value theorem $\exists \xi \in [t,t'] \subset I: \frac{|f(t) - f(t')|}{|t-t'|} = | \dot f (\xi) | \leq \ell$. 
%Secondly, we show that $\ell$ is the smallest Lipschitz constant: Let $\bar \ell$ be another Lipschitz constant such that $\bar \ell \leq \ell$. Since $I$ is compact and $\dot f$ is continuous there is some $\xi \in \indset $ such that $\dot f(\xi) = \ell$. Pick any sequence $(t_k)_{k=1}^\infty$ such that $t_k \stackrel{k \to \infty}{\longrightarrow} \xi$.
%$\forall k: t_k \in \indset $ and $\bar \ell$ is a Lipschitz constant on $I$. Hence, $ \bar \ell \geq \frac{|f(t_k) - f(\xi)|}{|t_k-\xi|}\stackrel{k \to \infty}{\longrightarrow} | \dot f(\xi)| = \ell$. Thus, $\bar \ell = \ell$.\\
  %
%
%
%\textbf{9)} Immediate consequence of 8).\\
%
%\textbf{10)} $L(f^2) = L(f \, f) \leq s(f) L(f) + L(f) s(f)$ where the last inequality follows from property 5).
%
%\textbf{11)} $L(f^q) \stackrel{8)}{=} \sup_{\tau \in \indset } |\dif{}{t} f^q(\tau)| = |q| \,\sup_{\tau \in \indset } |f^{q-1}(\tau) \, \dot f(\tau)| $. 
%
%\end{proof}
%\end{lem} 


\begin{lem}[H\"older arithmetic] \label{lem:Hoeldarithmetic}
Let, $I,J \subset \inspace$ where $\inspace$ is a metric space endowed with metric $\metric$. Let $f : \inspace \to \Real$ be H\"older on $I$ with constant number $L_I (f)$ 
and $g :\inspace \to \Real$ be H\"older on $J$ with constant $L_J (g)$. Assume both functions have the same H\"older exponent $p \in (0,1]$. That is, $\forall x, x' \in \inspace: \abs{f(x)-f(x')} \leq L(f) \metric(x,x')^p$ and  $\forall x, x' \in \inspace: \abs{g(x)-g(x')} \leq L(g)  \metric(x,x')^p$.
We have:

\begin{enumerate}
	\item Mapping $x \mapsto |f(x)|$ is H\"older on $I$ with constant $L_I(f)$ and exponent $p_f$.
	\item If $g$ is H\"older on all of $J=f(I)$ the concatenation $g \circ f: t \mapsto g(f(t))$ is H\"older on $I$ with constant 
	      $L_I(g \circ f) \leq$ $L_J (g) \, L_I^p(f)$ and exponent $p^2$.
	\item Let $r \in \Real$. $r \, f: x \mapsto r \, f(x)$ is H\"older on $I$ having a constant $L_I (r \,f) = |r| \, L_I(f)$.
	\item $f+g: x \mapsto f(x) + g(x)$ has H\"older constant at most $L_I(f) + L_J(g)$.
	\item Let $m_f = \sup_{x\in \inspace } f(x)$ and $m_g = \sup_{x \in \inspace } g(x)$. Product function $f\cdot g: x \mapsto f(x) \, g(x)$ has H\"older exponent $p$ and a H\"older constant on $I$ which is at most $(m_f \, L_J(g)+ m_g \, L_I(f))$.
	%\item Let $\tilde h(x) = \min\{f(x),g(x) \}$, $h(x) = \max\{f(x), g(x) \}, \forall x \in \inspace  \cap J$. We have $L_{I \cap J}(h) \leq \max\{L_I(f),L_J(g)\}$ and $L_{I \cap J}(\tilde h) \leq \max\{L_I(f),L_J(g)\}$.	
	\item For some countable index set $\indsett$, let the sequence of functions $f_i$ be H\"older with exponent $p$ and constant $L(f_i)$ each. Furthermore, let $H(x) =\sup_{i \in \indsett} f_i(x) $ and $h(x) := \inf_{i \in \indsett} f_i(x)$ be finite for all $x$. Then $H,h$ are also H\"older with exponent $p$ and have a H\"older constant which is at most $\sup_{i \in \indsett} L(f_i)$.
	\item Let $b := \inf_{x \in \inspace }| f(x)| > 0$ and let 
	$\phi(x) = \frac{1}{f(x)}, \forall x \in \inspace$ be well-defined.  
	      Then $L_I(\phi) \leq b^{-2} \, L_I(f)$.  
	\item Let $p=1$ (that is we consider the Lipschitz case), let $I$ be convex and $\metric(x,x') = \norm{x-x'}$. $f$ cont. differentiable on I $\Rightarrow$ $L_I(f) \leq \sup_{x \in I } \norm{\nabla f(x)}. $ 
	For one-dimensional input space, $\inspace = \Real$, $L_I(f) = \sup_{x \in I } \abs{\nabla f(x)}$ is the smallest Lipschitz number. 
	 \item Let $c \in \Real$, $f( t) = c, \forall x \in I $. Then $f$ is H\"older continuous with constant $L_I(f) =0$ and for any coefficient $p_f \in \Real$.  
	\item $L_I(f^2) \leq 2 \, L_I(f)\, \sup_{t \in I} f\,$.
	\item With conditions as in 8), and input space dimension one, we have $\forall q \in \mathbb Q : L_I(f^q) = |q| \,\sup_{\tau \in \indset } |f^{q-1}(\tau) \, \dot f(\tau)| $.
\end{enumerate}
\begin{proof}

\textbf{1)}  We show $|f|$ has the same constant and exponent as $f$. Let $X,X' \in \inspace $ arbitrary. 
We enter a case differentiation:

\textit{1st case: $f(x), f(x') \geq 0$}. 

Hence, $\bigl| \abs{f(x)}- \abs{f(x')} \bigr| = \bigl| f(x) - f(x') \bigr|  \stackrel{f \,Hoelder}{\leq} L_I(f) \metric(x,x')^{p}$.\\

\textit{2nd case: $f(x) \geq 0, f(x') \leq 0$.} 

Note, $|y| = - y$, iff $y \leq 0$. Hence,  $\bigl| |f(x)| - |f(x')| \bigr| \leq \bigl| |f(x)| + |f(x')| \bigr| $
$= \bigl| |f(x)| - f(x') \bigr|  =  \bigl| f(x) - f(x') \bigr| \leq L_I(f) \, \metric(x,x')^{p}$.\\

\textit{3rd case: $f(x) \leq 0, f(x') \geq 0$.} Completely analogous to the second case.\\

\textit{4th case: $f(x), f(x') \leq 0$}. 

$\bigl| |f(x)| - |f(x')| \bigr| = \bigl| f(x') - f(x) \bigr|= \bigl| f(x) - f(x') \bigr|  \stackrel{f \, Hoelder}{\leq} L_I(f) \metric(x,x')^{p}$.\\

The remaining points are also proven in \cite{Weaver1999} in the context of Lipschitz functions.

\textbf{2)} Special case of Thm. \ref{thm:hoelderconcat}.
% For arbitrary $t,t' \in \indset $ we have:
%
%$\bigl| g(f(t)) - g(f(t'))| \bigr| \leq L_J(g) \, |f(t) - f(t') | \leq L_J(g) \, L_I(f)\, |t-t'|$ 
%where the two inequalities are due to the Lipschitz properties of $g$ and $f$, respectively.\\

\textbf{3)}  For arbitrary $x,x' \in \inspace , r \in \Real$ we have:

$\bigl| r \, f(x) - r \, f(x')| \bigr| = |r|\, |f(x) - f(x')| \leq |r| \,L_I(f)\,  \metric(x,x')^{p}$.\\ 

\textbf{4)}  For arbitrary $x,x' \in \inspace , r \in \Real$ we have:

$\bigl| g(x) + f(x) - (g(x') + f(x'))| \bigr| = \bigl| g(x)  - g(x') + f(x)- f(x')| \bigr|$ 
$\leq \bigl| g(x)  - g(x')\bigr|  + \bigl| f(x)- f(x')| \bigr|$ $\leq (L_J(g)+L_I(f))\,  \metric(x,x')^{p}$.\\

\textbf{5)}  Let  $x,x' \in \inspace $, $d := f(t) - f(t')$.

$\bigl| g(x) f(x) - g(x')  f(x') \bigr| = \bigl| g(x) (f(x') +d) - g(x') f(x') \bigr|$ 
$= \bigl|\bigl( g(x) - g(x') \bigl)  f(x')+ g(x)  d \bigr|  $

$\leq \bigl| g(x) - g(x') \bigr|  |f(x')|   + \bigl|g(x)\bigr| \,  |d|  $
$\leq L_I(g) \metric(x,x')^p  |f(x')|   + \bigl|g(x)\bigr| \,  L_I(f) \metric(x,x')^p  $

$\leq L_I(g) \metric(x,x')^p  \sup_{x' \in \inspace } \{|f(x')|\}   + \sup_{x \in \inspace }\{\bigl|g(x)\bigr|\} \,  L_I(f) \metric(x,x')^p  $

$= \Bigl(L_I(g)  \sup_{x' \in \inspace } \{|f(x')|\}   + \sup_{x \in \inspace }\{\bigl|g(x)\bigr|\} \,  L_I(f)\Bigl) \metric(x,x')^p  $.\\

\textbf{6)}  The proof of Proposition 1.5.5 in \cite{Weaver1999} proves our statement if one replaces their 
metric $\rho$ by $\metric^p$.

\textbf{7)}  Let  $x,x' \in \inspace $.
$\bigl| \frac{1}{f(x)} - \frac{1}{f(x')} \bigr|$ 
$=\bigl| \frac{f(x')}{f(x') f(x)} -\frac{f(x)}{f(x') f(x)} \bigr|$ 
$= \frac{\bigl|f(x')-f(x) \bigr|}{\bigl|f(x')\bigr| \bigr| f(x)\bigr|}$ 
$\leq \frac{L_I(f) \metric(x,x')^p}{\inf_{x \in \inspace } |f(x)|^2}$.\\

\textbf{8)} Let $p=1$ and $\metric(x,x') = \norm{x-x'}$ be the standard norm. Define $\ell := \sup_{x \in I } \norm{ \nabla f(x) } = L_I(f)$. 
Firstly, we show that it is a Lipschitz constant: Let $x,x' \in I $ and 
$\overline{xx'} := \{tx + (1-t) x' \, | \, t \in [0,1]\}$. 
%$\overline{xx'} := \{y | \exists t \in [0,1] : y = tx + (1-t) x'\}$. 
Owing to convexity of I, $\overline{xx'} \subset I$. Due to the mean value theorem $\exists \xi \in \overline{xx'} \subset I: |f(x) - f(x')|=  T_\xi (x-x')$. where $T_\xi (x) = \SP{\nabla f(\xi)}{ x}$ is a linear OP. Assuming the derivative of $f$ is bounded, $T_\xi$ is a bounded OP and we have $\abs{T_\xi (x-x') } \leq \matnorm{T_\xi} \norm{x-x'}$ where 
$\matnorm{T_\xi} = \sup_{\norm{x} = 1} \SP{\nabla f(\xi)}{x} \leq \norm{\nabla f(\xi)}$. In conjunction,
$|f(x) - f(x')| \leq \norm{\nabla f(\xi)} \norm{x-x'}$. 

Secondly, we show that $\ell$ is the smallest Lipschitz constant in the one-dimensional case: Let $\bar \ell$ be another Lipschitz constant on $I$ such that $\bar \ell \leq \ell$. Of course, here $\norm{\cdot} = \abs{\cdot}$. Since $I$ is compact and $\norm{\nabla f(\cdot) }$ is continuous, there is some $\xi \in I$ such that $\norm{\nabla f(\xi)} = \sup_{x \in I} \norm{\nabla f(\xi)} = \ell$. Pick any sequence $(y_k)_{k=1}^\infty$ contained in $I$ such that $y_k \stackrel{k \to \infty}{\longrightarrow} \xi$ and $y_k \neq \xi$.
$\forall k: y_k \in I $ and $\bar \ell$ is a Lipschitz constant on $I$. Hence, $ \bar \ell \geq \frac{|f(y_k) - f(\xi)|}{\norm{y_k-\xi}}\stackrel{k \to \infty}{\longrightarrow} \norm{ \nabla f(\xi)} = \ell$. Thus, $\bar \ell = \ell$.

\textbf{9)} Trivial. 

\textbf{10)} Special case of property 5).

\textbf{11)} $L(f^q) \stackrel{8)}{=} \sup_{\tau \in \indset } |\dif{}{t} f^q(\tau)| = |q| \,\sup_{\tau \in \indset } |f^{q-1}(\tau) \, \dot f(\tau)| $. 

\end{proof}
\end{lem} 

As a simple illustration, assume we desire to establish that $f(t) = \max\{ 1- 3 \sin(t), \exp\bigl(- \sin(t) \bigr)\}$ is Lipschitz and to find a Lipschitz number on $\Real$. Application of 8. shows that $t \mapsto \sin(t)$ and $t \mapsto \exp(- t)$ have a Lipschitz number of $1$. Application, of 2., 9. 1. and 6. then show that $L(f) =3$ is a Lipschitz number of $f$.


An example that will be of relevance to our collision avoidance applications in Ch. \ref{ch:contcoll} is the following:
\begin{ex} \label{ex:hoelderOUstd}
Consider the function $h(t) = f \circ g(t)$ where $f(\cdot) = \sqrt \cdot$ is the square root function and $g(t) = \frac{B^2}{2 K} (1- \fexp{-2Kt}) $, $K >0$. $g$ is the non-stationary factor of the variance trajectory of an Ornstein-Uhlenbeck process (cf. Sec. \ref{sec:OUprocess}). We show that $h$ is H\"older and has constant $L(h)= \sqrt{2K}$ and H\"older exponent $p_h = \frac 1 2$. This can be easily done by applying Lem. \ref{lem:Hoeldarithmetic}.8 to show that $L(g)$ is Lipschitz with constant $L(g) = B^2 $ and then combining the H\"older constant and exponent derived in Ex. \ref{ex:sqrtfctHoelder} with Thm. \ref{thm:hoelderconcat} to yield the H\"older exponent $p_h = p_g p_f = \frac 1 2$ and constant $L(h) = 1 \sqrt{L(g)}  = \abs B$.  
\end{ex}

