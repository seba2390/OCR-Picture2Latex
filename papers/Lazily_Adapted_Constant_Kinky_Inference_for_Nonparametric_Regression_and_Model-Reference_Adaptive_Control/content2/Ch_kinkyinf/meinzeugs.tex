%------------------------Meinzeugs----------------------------------------
\usepackage{graphicx}\usepackage{subfigure}
\usepackage{cite}
\usepackage{amscd,amssymb} % AMS-Pakete fuer Theoreme
\usepackage{graphicx} \usepackage{array} \usepackage{subfigure}
\usepackage{psfrag}
%\usepackage{kpfonts}
\usepackage{xy} 
\usepackage{amsmath} 
%\usepackage{mathabx}
\usepackage{fancyhdr}
\usepackage{setspace} 
\usepackage{rotating,epsfig} \usepackage{epsfig}
\usepackage{verbatim} \usepackage{multicol} \usepackage{longtable}
\usepackage{amsthm}
% THEOREMS -------------------------------------------------------
\newtheorem{thm}{Theorem}[section]
\newtheorem{cor}[thm]{Corollary}
\newtheorem{lem}[thm]{Lemma}
\newtheorem{prop}[thm]{Proposition}
\newtheorem{alg}[thm]{Algorithm}
\newtheorem{conj}[thm]{Conjecture}
%\theoremstyle{definition}
\newtheorem{defn}[thm]{Definition}
\theoremstyle{remark}
\newtheorem{rem}[thm]{Remark}
\newtheorem{ques}[thm]{\textbf{QUESTION}}
\newtheorem{ex}[thm]{\textbf{Example}}
% MATH -----------------------------------------------------------
\newcommand{\norm}[1]{\left\Vert#1\right\Vert}
\newcommand{\normtop}[1]{\left\Vert#1\right\Vert^p}
\newcommand{\abs}[1]{\left\vert#1\right\vert}
\newcommand{\abstop}[1]{\left\vert#1\right\vert^p}
\newcommand{\set}[1]{\left\{#1\right\}}
\newcommand{\Real}{\mathbb R}
\newcommand{\Complex}{\mathbb C}
\newcommand{\integer}{\mathbb Z}
\newcommand{\nat}{\mathbb N}
\newcommand{\eps}{\varepsilon}
\newcommand{\To}{\longrightarrow}
\newcommand{\BX}{\mathbf{B}(X)}
\newcommand{\A}{\mathcal{A}}
\newcommand{\aff}[1]{\text{aff } #1}
\newcommand{\dom}[1]{\text{dom } #1}
\newcommand{\epi}[1]{\text{epi } #1}
\newcommand{\ri}[1]{\text{ri } #1}
\newcommand{\cl}[1]{\text{cl } #1}


%-----------------------------------------
\newcommand{\argmin}{\text{argmin}}
\newcommand{\argmax}{\text{argmax}}
\newcommand{\arginf}{\text{arginf}}
\newcommand{\argsup}{\text{argsup}}

\newcommand{\sgn}{\ensuremath{\text{sgn}}}
% fuer skalare:
%\newcommand{\s}[1]{\mathfrak{#1}}
\newcommand{\s}[1]{{#1}}
% fuer vectoren:
%\newcommand{\vc}[1]{\mathbf{#1}}
\newcommand{\vc}[1]{#1}
% scalar product:
\newcommand{\Sp}[1]{\ensuremath{\mathbf{\langle} #1 \mathbf{\rangle}}}
\newcommand{\SP}[2]{\ensuremath{\mathbf{\langle} \vc{#1} \mathbf{,} \vc{#2} \mathbf{\rangle}}}
\newcommand{\SPL}[2]{\ensuremath{\mathbf{\langle} \vc{#1}\mathbf{,} \vc{#2} \mathbf{\rangle}_{L_2}}}
\newcommand{\SPLintI}[2]{\ensuremath{\int_I {#1(t)}\, {\overline{#2}(t)} \, {\mu}}}

\newcommand{\pp}[1]{\langle #1 \rangle}

\newcommand{\metric}{\, \mathfrak{d}} % distance metric

%HS:
\newcommand{\hs}[1]{\ensuremath{\mathcal{#1}}}
%operators:
\newcommand{\op}[1]{\ensuremath{ {#1}}}

\newcommand{\dif}[2]{{\operatorname{d}\over\operatorname{d}#2}#1}
\newcommand{\pdif}[2]{{\operatorname{\partial}\over\operatorname{\partial}#2}#1}
\newcommand{\ppdif}[3]{{\operatorname{\partial^2}\over\operatorname{\partial}#2 \partial#3}#1}
%\renewcommand{\d}[1]{\operatorname{d}#1}
\renewcommand{\d}[1]{\text{ d}#1}


\newcommand{\bra}[1]{\langle #1 \vert}
\newcommand{\ket}[1]{\vert #1 \rangle}

%Fourier Operator:
\newcommand{\FT}[1]{\mathcal F #1}

%Koerper:
\newcommand{\field}{\mathbb F}
%nullspace:
\newcommand{\ns}{\mathcal{NS}}
\newcommand{\GP}{\mathcal {GP} }
\newcommand{\Gauss}{\mathcal {N} }
\newcommand{\gauss}{p_{\mathcal {N}} }
\newcommand{\expect}[1]{\ensuremath{ \langle#1\rangle } }
\newcommand{\cov}[2]{\ensuremath{ \text{cov}(#1,#2) } }
\newcommand{\var}[1]{\ensuremath{ \text{var}[#1]} }

%parameter:
\newcommand{\pardop}{\ensuremath{\omega}}%parameter for lin. Dif OP

\newcommand{\pardopspace}{\ensuremath{\Omega}}%parameter space for lin. OP

\newcommand{\pardi}{\ensuremath{\mathfrak p}}%parameter for %inhomogeneous ODE-part / driving force

\newcommand{\pardispace}{\ensuremath{\mathfrak P}}%parameter for %inhomogeneous ODE-part / driving force

\newcommand{\state}{\ensuremath{ \vc x}} % state of the system... integral curve of the ODE/ signal.
\newcommand{\statespace}{\ensuremath{ \mathcal X}}

\newcommand{\inspace}{\ensuremath{ \mathcal X}}
\newcommand{\outspace}{\ensuremath{ \mathcal Y}}

\newcommand{\di}{\ensuremath{ u}} % driving force / control input

\newcommand{\agi}{\ensuremath{ \mathfrak a}} %an agent index
\newcommand{\agii}{\ensuremath{ \mathfrak r}} %index if another agent
\newcommand{\agiii}{\ensuremath{ \mathfrak q}} %index if another agent

\newcommand{\agset}{\ensuremath{ \mathfrak A}} %set of Agents

\newcommand{\collevent}{\ensuremath{\mathfrak C}}


\newcommand{\decke}{\ensuremath{\mathfrak u}}
\newcommand{\boden}{\ensuremath{\mathfrak l}}
\newcommand{\einschluss}{\ensuremath{\mathcal E^{\decke_n}_{\boden_n}}}

\newcommand{\menge}[2]{\ensuremath{\{#1 \, \vert \, #2 \}}}

%\newcommand{\diag}{\ensuremath{ \text{diag}} %set of

\renewcommand{\Pr}{\mathrm{Pr}}
\newcommand{\KLD}[2]{\text{KLD}(#1||#2) }

\newcommand{\erw}[2][]{
%    \ensuremath{\pmb{\mathcal{M}}^{\operatorfont{#1}}}%
     \ensuremath{\mathbb{E}_{#1} \! \left\lbrace #2 \right\rbrace}%
}
%----------- probability stuff------------------------
\newcommand{\pmeas}[1]{\ensuremath{\mathbb #1}} % probability measure

\newcommand{\embedding}{\ensuremath{\tilde \mu}} % probability measure
\newcommand{\ewx}[1]{\ensuremath{\mathbb E_x[ #1 ] }} % expected value
\newcommand{\ew}{\ensuremath{\mathbb E }} % expected value
\newcommand{\is}{\ensuremath{\mathfrak X }} % input space / index set
\newcommand{\Vol}{\ensuremath{\text{Vol}}}

% --------------------environments--------------------------------------------
\newcommand{\beq}{\begin{equation}}
\newcommand{\eeq}{\end{equation}}
\newcommand{\eq}[1]{\begin{equation} #1 \end{equation}}

% ----------------------------------------------------------------
%-------------- Meinzeugs Ende-----------------------------------------
