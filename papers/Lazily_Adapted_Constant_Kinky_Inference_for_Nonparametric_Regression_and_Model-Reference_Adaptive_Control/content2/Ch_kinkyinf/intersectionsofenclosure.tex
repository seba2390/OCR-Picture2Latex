\section{Incorporation of additional knowledge}
So far, our knowledge was confined to Lipschitz continuity. This yielded the definition of the Lipschitz enclosure $\einschluss$ for which we proved convergence to the ground truth in the limit of an infinite number of function samples $N$. Often however, we may actually have additional knowledge about our target function $f$. Such information could translate to another set $F_N$ of function for which this additional piece of knowledge holds true. For instance, if our method was to be applied to marginalization in Bayesian inference, we would know that the integrands involved are nonnegative. In this case, we would have $F_N =\{ \phi: I \to \Real | \phi(x) \geq 0, \forall x \in I\}$. 
In such a case, we would consider the somewhat less conservative enclosure 
$F_N \cap \einschluss$ to represent our uncertainty about $f$ (and to base our $ceiling$ and $floor$ functions on).

If we can define $F_N$ by a ceiling and floor function $b_N,a_N$, $F_N = \{\phi : I \to \Real | a_N(x) \leq \phi(x) \leq b_N(x) \}$, then $F_N \cap \einschluss$ is given by the ceiling function $\Gamma_N: x \mapsto \min\{b_N(x), \decke_N(x) \}$ and floor function $\Phi_N:  x \mapsto \max\{a_N(x), \boden_N(x) \}$:
$F_N \cap \einschluss = \{ \psi : I \to \Real | \Phi_N(x) \leq \psi(x) \leq \Gamma_N(x) \}$.
In our example with additional non-negativity constraints, we would have $\Gamma_N (x) = \infty$ and $\Phi_N(x) = 0$ $\forall x \in I$.

Ignoring this information may yield unnecessarily conservative estimates. 
Therefore, we should work with $\Gamma_N$ and $\Phi_N$.

All we need to do is to show that the convergence properties extend.

\begin{thm}\label{thm:trunkunifconv}
Let $(\decke_N)_{N \in \nat}, (\boden_N)_{N \in \nat}$ be two sequences of functions, $\decke_N,\boden_N: I \subset \Real^d \to \Real$, converging to function $f: I \to \Real$ uniformly on $I$ as $N \to \infty$.
Furthermore, let $\mathfrak b, \mathfrak a : I \to \Real$ be two functions such that $\forall x \in I: 
\mathfrak b(x) \geq f(x) \geq \mathfrak a(x)$. For all $N \in \nat$, define $\Phi_N: x \mapsto \max\{\mathfrak a(x),\boden_N(x) \}$ and $\Gamma_N: x \mapsto \min\{\mathfrak b(x),\decke_N(x) \}$.
These function sequences also converge uniformly to $f$: 

$\Gamma_N, \Phi_N \stackrel{N \to \infty}{\longrightarrow} f$ uniformly on $I$.
\begin{proof}
We prove uniform convergence of $(\Phi_N)$. The proof for uniform convergence of $(\Gamma_N)$ is completely analogous and can be omitted.


Uniform convergence of $(\boden_N)$ to $f$  is equivalent to the predicate $\forall \epsilon >0 \exists N_0 \forall N \geq N_0\forall x \in I: \abs{\boden_N(x) - f(x)} < \epsilon$. 

Let $\epsilon > 0$ and choose $N_0$ such that $\forall N \geq N_0, x \in I$ we have 
$\abs{\boden_N(x) - f(x)} < \epsilon$. Choose $N \geq N_0, x\in I$.
Consider $\abs{\Phi_N(x) - f(x)}$. 

Case I: Let $\mathfrak a(x) = f(x)$. Hence, $\abs{\Phi_N(x) - f(x)}
{=} \abs{\max\{\mathfrak a(x),\boden_N(x) \} - f(x)}$
$\stackrel{\boden_N(x) \leq f(x)}{=} \abs{f(x) - f(x)} = 0 < \epsilon$.

Case II: Let $\mathfrak a(x) < f(x)$. Hence, $\mathfrak a(x) = f(x) - \delta_x$, for some $\delta_x >0$.
Due to uniform convergence (and since $\boden_N(x) \leq f(x)$), we have $\exists \tilde N \forall N \geq \tilde N: f(x) - \boden_N(x) < \delta_x$. In particular, for any $N \geq \bar N := \max\{\tilde N, N_0\}$ we have:  $f(x) - \boden_N(x) < \delta_x$. Hence, $\forall N\geq \bar N: a(x) < \boden_N(x)$.
Thus, $\forall N\geq \bar N:  \abs{\Phi_N(x) - f(x)}
{=} \abs{\max\{\mathfrak a(x),\boden_N(x) \} - f(x)}
\stackrel{\bar N \geq N_0}{=} \abs{\boden_N(x) - f(x)} < \epsilon
$.
\end{proof}


\end{thm}
\subsection{Definite integrals}
So far, we have established uniform convergence of the truncated estimates $\Gamma_N$ and $\Phi_N$ to the target. As an immediate consequence Thm. \ref{thm:trunkunifconv}, in conjucntion with Thm. \ref{thm:defintconv_multidim} allow us to conclude that the definite integrals converge to the desired target integral, in the limit of infinite sample points:


\begin{thm}\label{thm:defintconv_multidim_trunkated}
If the grid sequence $G_N$ becomes dense in domain $I$ we have 
\[\forall N \in \nat : \int_I \Gamma_N(x) \, d x \geq \int_I f(x) \, d x \geq  \int_I  \Phi_N \, d x \]

\[ \int_I \Gamma_N(x) \, d x,  \int_I  \Phi_N \, d x  \stackrel{N \to \infty}{\longrightarrow} \int_I f(x) \, d x.  \]
\end{thm} 

It remains to derive closed-form expressions for the estimate integrals $\int_I \Gamma_N(x) \, d x$ and  $\int_I  \Phi_N \, d x $. Of course, this depends on the definitions of the bounding functions $\mathfrak a_N , \mathfrak b_N$. As a special case of particular importance, we consider the case where our additional knowledge is that our target function cannot be nonnegative. That is, $F_N =\{ \phi: I \to \Real | \phi(x) \geq 0, \forall x \in I\}$ and so, $\forall x: \mathfrak a_N(x) = 0, \mathfrak b_N(x) = \infty$. Therefore, $\int_I \Gamma_N(x) \, d x = S^\decke_N$ as above. So, the only integrals we need to be concerned about are the lower bound integrals $\int_I  \Phi_N \, d x$ where in our case, $\Phi_N^i (x) = \max\{0,\boden^i_N(x)\}$.
How to calculate its definite integral is subject to the following theorem:

\begin{thm}
Let $\Phi_N (x) = \max\{0,\boden_N(x)\}, \Phi^i_N (x) = \max\{0,\boden^i_N(x)\}$ and $J_i = c_i +\prod_{j=1}^d [-r_{i,j},r_{i,j}]$. 
Furthermore, let $\underline J_i = c_i + \prod_{j=1}^d [-p_{i,j},p_{i,j}] $ 
where $p_{i,j} := \min\{\frac{f(c_i)}{\ell_i},r_j \}$.
We have $\int_I \Phi_N(x) \, dx = \sum_i\int_{J_i} \Phi^i_N(x) \, dx$ where 
\[\int_{J_i} \Phi^i_N(x) \, dx = \int_{\underline J_i} \boden^i_N(x) \, dx.  \]
\begin{proof}
We show that $\underline J_i \subset J_i$ is the support of $\Phi^i_N$ on $J_i$. (Since $\Phi^i_N(x) = \max\{ 0, \boden^i_N(x) \} \geq 0, \forall x$ this implies 
$\int_{J_i} \Phi^i_N(x) \, dx = \int_{\underline J_i} \boden^i_N(x) \, dx$ as desired.)

$x \in J_i \cap \text{supp } \Phi^i_N \Leftrightarrow x \in J_i \wedge \boden^i_N(x) \geq 0$ 
$\stackrel{def.}{\Leftrightarrow} x \in J_i \wedge \frac{f(c_i)}{\ell_i} \geq \norm{x-c_i}_\infty$ 
${\Leftrightarrow} \forall j \in {1,...,d}: x_j    \in c_{i,j} + [-r_{i,j},r_{i,j}] \wedge \frac{f(c_i)}{\ell_i} \geq \abs{x_j-c_{i,j}}$
${\Leftrightarrow} \forall j \in {1,...,d}: x_j    \in c_{i,j} + [-r_{i,j},r_{i,j}] \wedge \frac{f(c_i)}{\ell_i} + c_{i,j}  \geq x_j \geq -\frac{f(c_i)}{\ell_i} + c_{i,j}$
${\Leftrightarrow} \forall j \in {1,...,d}: x_j    \in c_{i,j} + \bigl([-r_{i,j},r_{i,j}]\cap [-\frac{f(c_i)}{\ell_i}, \frac{f(c_i)}{\ell_i}] \bigr)$
%
$\Leftrightarrow x    \in c_{i} + \prod_j [-\min\{r_{i,j}, \frac{f(c_i)}{\ell_i}\} , \min\{r_{i,j}, \frac{f(c_i)}{\ell_i}\}]$.

\end{proof}
\end{thm}
Note, each term $\int_{\underline J_i} \boden^i_N(x) \, dx$ can be calculated as derived above. Therefore, the theorem is a recipe for closed-form calculation of $\int_I \Phi_N(x) \, dx$.