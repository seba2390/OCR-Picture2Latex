%\documentclass[11pt]{article}
\documentclass[a4paper,11pt]{amsart}
\usepackage{cite}
\usepackage{fullpage}
\usepackage{qtree}
\usepackage{amssymb}
\usepackage{MnSymbol}

\usepackage{amsthm}
\usepackage{amsmath} 
\usepackage{mathtools}
\usepackage{graphicx}
\usepackage[all,2cell]{xy}
\SelectTips{cm}{}
\usepackage[mathscr]{euscript}
\usepackage{listings}
\usepackage{young}
\usepackage{ytableau}
\usepackage{enumitem}
\usepackage{tikz}
\usetikzlibrary{arrows}
\usepackage{subcaption}
\usepackage{youngtab}
\usepackage{tikz-cd}
\usepackage{ifthen}
\usepackage{color}
\usepackage[normalem]{ulem}
\usetikzlibrary{arrows}

\newcommand{\eat}[1]{}
\newcommand{\fs}{\mathfrak{S}}

\newcommand{\C}{\mathbb{C}}
%\newcommand{\P}{\mathbb{P}}

\renewcommand{\qedsymbol}{$\surd$}

\usepackage[citecolor=black,urlcolor=blue,bookmarks=false,hypertexnames=true]{hyperref} 
%\parskip 0.2cm
%baselineskip 1cm colorlinks
\newtheorem{theorem}{Theorem}[section]

\newtheorem{corollary}[theorem]{Corollary}
\newtheorem{definition}[theorem]{Definition}
\newtheorem{example}[theorem]{Example}
\newtheorem{lemma}[theorem]{Lemma}
\newtheorem{notation}[theorem]{Notation}
\newtheorem{proposition}[theorem]{Proposition}
\newtheorem{observation}[theorem]{Observation}
\newtheorem{claim}[theorem]{Claim}
\newtheorem{remark}[theorem]{Remark}
\newtheorem{question}[theorem]{Question}
\newtheorem{conjecture}[theorem]{Conjecture}

\makeatletter
\DeclareRobustCommand*\cal{\@fontswitch\relax\mathcal}
\makeatother

\date{}
\begin{document} 

\eat{
\title{Morphisms from projective spaces to $G/P$}
\author{
Sarjick Bakshi \thanks{{Tata Institute of Fundamental Research, Mumbai, India, {\tt sarjick91@gmail.com}, Corresponding author}}
\and
A J Parameswaran \thanks{Tata Institute of Fundamental Research, Mumbai, India, {\tt param@math.tifr.res.in}}
}
\maketitle
}

\title{Morphisms from projective spaces to $G/P$}
\author{Sarjick Bakshi}
\thanks{Tata Institute of Fundamental Research, Mumbai, India, {\tt sarjick91@gmail.com}, Corresponding author}
\author{A J Parameswaran}
\thanks{Tata Institute of Fundamental Research, Mumbai, India, {\tt param@math.tifr.res.in}}


\maketitle





\begin{abstract} We study maps between projective spaces and flag varieties. Let $G = SL(n,\mathbb{C})$. We show that there is no non constant map from $\mathbb{P}^2$ to full flag variety $G/B$. We classify the minimal parabolic subgroups $P$ for which there is a map from $\mathbb{P}^3$ to $G/P$. 
\end{abstract}


\section{Introduction}


Let $G = SL(n,\mathbb{C})$. Let $B$ denote the {\em Borel subgroup} of upper triangular matrices and $T$ be the {\em maximal torus} of diagonal matrices in $G$. The projective homogeneous variety $G/B$ is called the {\em generalised flag variety} (or simply flag variety). Let $W$ denote the Weyl group for $(G,T)$. In our case, it is $S_n$, the symmetric group in $n$ letter. For $w\in W$, let $X(w)$ be the corresponding {\em Schubert variety} inside $G/B$ which is defined to be the closure of $B$ orbit passing through the $T$-fixed point $wB$. The $B$-orbit is called a {\em Schubert cell} and is denoted by $C(w)$. The study of cohomology ring of the generalised flag variety $H^{\bullet}(X,\mathbb{Z})$ goes back to Ehresmann \cite{Ehresmann}. He also showed that the flag variety has a partition in terms of Schubert cells. In \cite{Borelcohoring}, Borel gave a presentation of the ring. Let $x_1, x_2, \ldots, x_n$ denote $n$ indeterminates. Let $\mathbb{Z}[x_1, x_2, \ldots, x_n]$ denote the graded polynomial ring with degree of $x_i$ being $2$ for each $i$. $S_n$ acts on indeterminates by permuting the indices. Let ${\cal{I}} = \mathbb{Z}[x_1, x_2, \ldots, x_n]_{\plus}^{S_n}$ denote the ideal of $\mathbb{Z}[x_1, x_2, \ldots, x_n]$ generated by $S_n$ invariants in positive degrees. In \cite{Borelcohoring}, Borel showed that there is an graded isomorphism
\[  H^{\bullet}(G/B,\mathbb{Z}) \cong \mathbb{Z}[x_1, x_2, \ldots, x_n]/{\cal{I}}. 
\]
In \cite{Chow}, Chow showed that there is a basis of $H^{\bullet}(G/B,\mathbb{Z})$ given by the classes of the Schubert varieties inside the Grassmannian. From \cite{fultonintersection}, we know that the cohomology ring $H^{\bullet}(G/B,\mathbb{Z})$ is also isomorphic to the {\em Chow ring} $A^{\bullet}(G/B)$. 

Let $P$ be a {\em parabolic subgroup} containing $B$, let $W_{P}$ denote subgroup of $W$ generated by simple reflections that appears in the Bruhat decomposition of $P$. Let $W^P = W/W_P$ denote the set of minimal length coset representives. In \cite{Reineretal}, Reiner--Woo--Yong gave a presentation of the cohomology ring $H^{\bullet}(G/P,\mathbb{Z})$. They show that
\[ H^{\bullet}(G/P,\mathbb{Z}) \cong H^{\bullet}(G/B,\mathbb{Z})^{W_P},
\]
where $H^{\bullet}(G/B,\mathbb{Z})^{W_P}$ denotes the ring of invariants of $H^{\bullet}(G/B,\mathbb{Z})$ under the $W_P$ action. In the same paper, they also study a presentation of the cohomology ring $H^{\bullet}(X(w),\mathbb{Z})$ and get a relatively shorter presentation of the cohomology ring of the Schubert varieties appearing in the flag varieties. 

Let $k \leq n$ be two positive integers. Let $Gr(k,n)$ denote the {\em Grassmannian} of $k$-dimensional subspaces of an $n$ dimensional complex vector space. Note that the Grassmannian variety is also a partial flag variety $G/P$ for a maximal parabolic subgroup $P$. Maps from projective spaces to Grassmannian have been studied extensively by Tango in a series of papers \cite{Tango1,Tango2,Tango3}. In \cite{Tango1}, he showed that there is no map from $\mathbb{P}^m$ to $Gr(k,n)$ for $m \geq n$. In \cite{Tangobundle}, he also showed that there is an indecomposable vector bundle of rank $n-1$ on $\mathbb{P}^n$. While answering Lazarsfeld's problems \cite{Lazarsfeld}, under the assumption $1< k < n-k$ Paranjape--Srinivas in \cite{srinivas} show that there is a finite surjective morphism $Gr(k,n)$ to $Gr(l,m)$ if and only if $k=l$ and $n=m$, in which case they are isomorphic. This naturally led to the study of maps between various projective homogeneous spaces. 

The study of cohomology ring plays a crucial role in attempting many of these questions. Tango's original proof compared the Chow ring of the two varieties. In \cite{Munozsplitting}, Muñoz--Occhetta--Conde showed that a weaker property of the cohomology ring is required to obtain Tango type result which they called {\em effective good divisibility} which was an improvement of the notion of {\em effective good divisibility} introduced by Pan in \cite{Pan}. Using effective good divisibility, Naldi--Occhetta in \cite{Naldi--Occhetta} were able to extend Tango's result and they show that every morphism between $Gr(k,n)$ to $Gr(l,m)$ for $n >m$ is constant. In the paper they show that the effective good divisibility of the Grassmannian is $n$. In \cite{Munoz-etal}, Muñoz--Occhetta--Conde studied the maps between rational homogenous variety using effective good divisibility. They show that there is no map from a projective variety to a rational homogeneous variety where the effective good divisibility of the projective variety is higher than that of a rational homogeneous variety. 

In our paper, we study maps from projective spaces to certain partial flag varieties. We first observe the following:

\begin{theorem}\label{mapsfromP2} There is no non constant map from $\mathbb{P}^2$ to $G/B$.
\end{theorem}

Let $P$ be a parabolic subgroup of $G$ containing $B$. Let $m$ denote the {\em rank} of the Picard group $Pic(G/P)$. We call it the {\em rank} of $G/P$. For instance, rank of the Grassmannian $Gr(k,n)$ is $1$ and the rank of the full flag variety $G/B$ is $n-1$. Similarly, we have rank of $G/P$ is $n-2$, for a minimal parabolic subgroup $P$. From \ref{mapsfromP2} and Tango's result it is natural to ask the following question:
\begin{question}\label{question} Does there exist a map from $\mathbb{P}^m$ to $G/P_J$ where $m = n+ 1 - \rm{rank}(G/P)$?
\end{question}

We study the question when $P$ is a minimal parabolic subgroup. Let $\alpha_1, \alpha_2, \ldots, \alpha_{n-1}$ denote the set of simple roots. Let $s_{\alpha_j}$, $1 \leq j \leq n-1$ denote the corresponding simple roots. To each such $\alpha_j$ we can associate a minimal parabolic subgroup $P_{\alpha_j} := B \cup Bs_{\alpha_j}B$. We obtain the following two theorems. 

\begin{theorem} There is no non constant map from $\mathbb{P}^3$ to $G/P_{\alpha_i}$ for $i \in \{1, n-1\}$.
\end{theorem}
\begin{theorem}\label{thm1} There is a map from $\mathbb{P}^3$ to $G/P_{\alpha_j}$ for $1 < j < n-1$. 
\end{theorem}

After we had put up the first version in the arXiv, it was communicated to us via an email from Yanjie Li that the proof of Theorem 5.1 has an error. Formula (8) in the proof of the theorem is wrong. To overcome that, we are giving a completely elementary proof of theorem \ref{thm1}. Due to the same error, we have now removed Theorem 6.1 from the arxiv version where we state that there is no morphism from $\mathbb{P}^4$ to $G/P_{\alpha_j}$ for $1 \leq j \leq n-1$. We have some evidence for the statement and we are working on it currently. After putting up our first version we have been communicated of a generalisation of some of our results by Fang and Ren (see, \cite{fangren}). 

\title{\bf{Acknowledgement}} 
We are very thankful to Shrawan Kumar for informing us about his conjecture, and various helpful discussions. We are thankful to Rohith Varma for stimulating discussions. We are also thankful to Senthamarai Kannan for suggesting the article \cite{Munoz-etal}. 
\section{Kumar's conjecture}

In this section we discuss Kumar's conjecture. This conjecture is due to Shrawan Kumar (see, \cite{kumarsconj}). 
Let $X$ be a complex connected homogeneous projective variety. Then $X$ can be written as $H/P$ for a complex semisimple connected algebraic group $H$ and $P \subset H$ a parabolic subgroup. Observe that the $H$ and $P$ are not necessarily unique. For instance, the projective space $\mathbb{P}^{2n-1}$ is a homogeneous space for $SL(2n,\mathbb{C})$ as well as $Sp(2n,\mathbb{C})$. Assume also that $X$ is indecomposable, ie $X$ cannot be written as $X_1 \times X_2$ where none of $X_1$ or $X_2$ is singleton. 

\begin{definition} Let $H$ be a complex connected indecomposable semisimple algebraic group with $P$ a parabolic subgroup. The {\em semisimple rank} of the pair $(H,P)$ is defined to be the rank of the semisimple part of the Levi subgroup of $P$. 
\end{definition}

\begin{definition} Let $X$ be complex connected indecomposable homogeneous projective variety. The {\em minimum semisimple stabilizer rank} (respectively, {\em maximum semisimple stabilizer rank}) of $X$ is the minimum (respectively, maximum) of the semisimple ranks for all possible realisations of $X$ as a $H/P$, with $H$ simple and connected algebraic group and $P$ a parabolic subgroup. If the minimum semisimple stabilizer rank of $X$ coincides with the maximum semisimple stabilizer rank of $X$ we call it the rank of $X$. Denote the minimum semisimple rank (respectively, maximum semisimple rank) by  {\em minss rank} (respectively, {\em maxss rank}).
\end{definition}

For instance, $\mathbb{P}^{2n-1}$ has {\em minss rank} $n-1$ because it is realisable as a homogeneous space of $Sp(2n,\mathbb{C})$ whereas the {\em maxss rank} is $2n-2$ when it is realised as a quotient of $SL(2n,\mathbb{C})$. Note that we have {\em minss rank} $X = 0$ if and only if $X$ has a realisation of the form $H/B$ for a Borel subgroup $B$ inside $H$. In this case, the {\em maxss rank} is also $0$. We also observe that  the {\em minss rank} $X$ is same as {\em maxss rank} $X$ in most cases, however the cases where it doesn't hold can be found in \cite[\S 2]{Demazureauto}.

\begin{conjecture}\cite[Conjecture 5]{kumarsconj} Let $X$ and $X'$ be two connected indecomposable homogeneous projective varieties.
\begin{itemize}
\item[(a)] Assume that $X$ is different from $\mathbb{P}^{2n}$ (for $n \geq 1$) and \[ \text{minss rank } X > \text{maxss rank } X'. 
\]
Then, there does not exist any nonconstant algebraic map from $X$ to $X'$.  
\item[(b)] If $X = \mathbb{P}^{2n}$ (for $n \geq 1$) and there exists a non-constant regular map from $X \to X'$, then
\[ \text{minss rank } \mathbb{P}^{2n-1} = n-1  \leq  \text{maxss rank } X'.
\] 
\end{itemize}
\end{conjecture}

Kumar in \cite{kumarsconj} has proved the conjecture when $X'$ can be written as $H'/B'$ for a simple algebraic group $H'$ with $B'$ a Borel subgroup. Note that this is equivalent to the case when semisimple rank of $X'$ is $0$. However, to what generality the conjecture is solved remains unclear, and our paper is a step towards understanding this conjecture. We look at the conjecture when $X'$ has rank $0$ or $X'$ is a rank $1$ homogeneous space of $SL(n,\mathbb{C})$. In the former case, we verify conjecture (a) when $X$ is homogeneous spaces of $SL(n,\mathbb{C})$. In the later, we verify the conjecture for any projective space. Some of the major computations in this paper analyses the map from $\mathbb{P}^3$ which has {\em minss rank} $1$ to $SL(n,\mathbb{C})/P$ for a minimal parabolic subgroup $P$ (which has {\em maxss rank} $1$ as well).

\section{Preliminaries}
Let $G = SL(n,\C)$ denote the set of all $n\times n$ matrices with determinant $1$. Let $B$ denote the Borel subgroup of upper triangular matrices, and $T$ denote the maximal torus consisting of diagonal matrices inside $G$. Denote $R$ the root system of $(G,T)$. Let $R^+$ denote the subset of $R$ consisting of positive roots. Let $\epsilon_i$ denote the character of $T$ which sends $diag(t_1,t_2,\ldots, t_n)$ to $t_i$. Let $\alpha_i = \epsilon_i - \epsilon_{i+1}$. Then a subset $S= \{\alpha_1, \ldots, \alpha_{n-1} \}$ of $R^+$ gives a set of simple roots. The Weyl group $W$ is the group generated by the simple reflections $s_{\alpha}$, $\alpha \in S$. In our case, $W$ is the symmetric group in $n$ letters $S_n$. The simple reflections $s_{\alpha_i}$ can be thought of as the transposition of $i$-th and $i+1$-th letter. We would use the one-line notation $(w(1),w(2),\ldots,w(n))$ to denote the permutation $w$ in $S_n$. 

Let $J$ be a subset of $S$. Let $W_J$ denote the subgroup of $W$ generated by $s_{\alpha}$, $\alpha \in J$. For every $J$ we associate a parabolic subgroup $P_J$ as follows 
\[ P_J = \bigsqcup_{w \in W_J} BwB.  \]

The set $W^J = W/W_J$ is called the set of {\em minimal length coset representatives}. Alternatively, we have (see, \cite[Section 2.5]{Billeylakshmibai}) 
\[ W^J = \{ w\in W \mid w(\alpha) > 0 \text{ for all } \alpha \in J \}.
\] 

The {\em full flag variety} is by definition the variety $G/B$. The projective homogeneous space $G/P_J$ is called a {\em partial flag variety} and its Bruhat decomposition is given by 
\[ G/P_J = \bigsqcup_{w \in W^J} BwP_J.
\]
Whenever $W_J$ is generated by one element $s_\alpha$ for $\alpha \in S$, we call the associated parabolic subgroup a {\em minimal parabolic subgroup} and we denote it as $P_{\alpha}$. Note that, $P_{\alpha}/B$ is isomorphic to $\mathbb{P}^1$. Whenever $J$ is obtained from $S$ by removing one simple root $\alpha_k$, we call the associated parabolic subgroup a {\em maximal parabolic subgroup} and we denote it by $P_{\hat{\alpha_k}}$. We recall that the {\em Grassmannian variety} $Gr(k,n)$ of $k$ dimensional subspaces of a $n$-dimensional complex vector space is isomorphic to $G/{P_{\hat{\alpha_k}}}$. Let 
\[ I(k,n) = \{(i_1, i_2, \ldots, i_k) |~ 1 \leq i_1 < i_2 < \cdots < i_{k} \leq n\}.
\]
Let $w = (i_1,i_2, \ldots, i_k) \in I(k,n)$. Let $e_1, e_2, \ldots, e_n$ be the standard basis of $\mathbb{C}^n$. Let $M_i$ denote the vector space spanned by $e_1,e_2, \ldots ,e_i$. The Schubert cell $C(w)$ in the Grassmannian is defined as 
\[ C(w) = \{U \in Gr(k,n)|~\rm{dim}(U \cap M_{i_j}) = j, 1\leq j \leq k \}.
\]
The dimension of such a Schubert cell $C(w)$ is given by $\sum_{j}(i_j-j)$. The Schubert variety $X(w)$ which is the closure of $C(w)$ in Grassmannian can be seen to be
\[ X(w) = \{U \in Gr(k,n)|~\rm{dim}(U \cap M_{i_j}) \geq j, 1\leq j \leq k \}.
\]

Let $R$ denote the polynomial ring $\mathbb{Z}[x_1,x_2,\ldots,x_n]$ in $n$ variables with degree of $x_i$ being $2$. We recall that $S_n$ acts on the variables as 
\[ \sigma(x_i) = x_{\sigma (i)}.
\]
The action extends to an action of $S_n$ on $R$.  A polynomial $f(x_1,x_2,\ldots,x_n)$ in $R$ is {\em symmetric} if and only if 
\[ f(x_1,x_2,\ldots,x_n) = f(\sigma(x_1),\sigma(x_2),\ldots,\sigma(x_n))
\]
for all $\sigma \in S_n$. 
The {\em power sum symmetric polynomial} $p_k(x_1,x_2\ldots,x_k)$ is defined as 
\[  p_k(x_1,x_2\ldots,x_k) = \sum^n_{i=1} x_i^k .
\]

We recall that the subring of invariants $R^{S_n}$ of $R$ is a graded subring and is generated by symmetric polynomials. Let ${\cal{I}}$ denote the ideal generated by symmetric polynomials in positive degree. The power sum symmetric polynomials $p_k(x_1,x_2,\ldots,x_n)$ for $1 \leq k \leq n$ form a set of generators for ${\cal{I}}$. 

Let $X$ be a projective variety. Let $H^{\bullet}(X) = \bigoplus\limits_{d=1}^{n} H^d(X)$ denote the cohomology ring of the variety with integer coefficients. Let $A^{\bullet}(X) = \bigoplus\limits_{d=1}^{n} A^d(X)$ denote its Chow ring. We recall from \cite[Chapter 19]{fultonintersection} that there exists a cycle map 
\[ cy: A^{\bullet}(X) \longrightarrow H^{\bullet}(X). 
\]
Whenever $X$ is a partial flag variety the map $cy$ is an isomorphism (see, \cite[Example 19.1.11]{fultonintersection}) and the cohomologies in odd degrees vanish. When $X$ is the full flag variety $G/B$ we recall

\begin{theorem}\cite[Ehresmann]{Ehresmann} $H^{2d}(G/B)$ has a basis consisting of classes of Schubert varieties $[X(w_0w)]$ where $l(w) = d$ where $w_0$ is the longest word in $W$. 
\end{theorem}


In \cite{Borelcohoring}, Borel, gave a presentation of the cohomology ring using the polynomial ring $R$ and the ideal $I$
\begin{theorem}\label{borelisom} \cite[Borel]{Borelcohoring} 
$H^{\bullet}(G/B) \cong R/{\cal{I}}$.
\end{theorem}

The results were extended for $G/P$, where $P$ is a parabolic subgroup of $G$ containing $B$ in \cite{Reineretal}. Let $J \subset S$ such that $P = P_J$. We have $W_J$ the subgroup of Weyl group generated by $J$ as above. Since $W_J$ is subgroup of $W$ it also acts on $H^{\bullet}(G/B)$. Reiner--Woo--Yong show that,

\begin{theorem} \cite{Reineretal}[Reiner--Woo--Yong]
$H^{\bullet}(G/P) \cong H^{\bullet}(G/B)^{W_J}$.
\end{theorem}

We observe that 
\[ H^{\bullet}(G/P) \hookrightarrow  H^{\bullet}(G/B). \]

\begin{remark} \label{cohG/P} Under this inclusion we recall from \cite{Reineretal}, the cohomology classes $[X(w)]$ where $w \in W^J$ lies in $H^{\bullet}(G/B)^{W_J}$ and forms a basis of $H^{\bullet}(G/P)$. More precisely, a basis of $H^{2d}(G/B)^{W_J}$ consists of the Schubert classes $[X(w)]$, where $w \in W^J$ and $X(w)$ is a codimension $d$ Schubert subvariety of $G/P$. This can be thought of as a generalisation of Ehresmann's theorem. 
\end{remark}



\section{Maps from $\mathbb{P}^2$ to $G/B$}

As in the previous section, we have $G = SL(n,\mathbb{C})$, $B$ denotes the Borel subgroup consisting of the diagonal matrices in $G$. We will begin this section by proving the following:

\begin{theorem}\label{consp2} There exists no nonconstant map from $\mathbb{P}^2$ to $G/B$.
\end{theorem}
\begin{proof}
Let $\phi$ be such a map and 
\[ \phi^{*i}: H^{i}(G/B) \longrightarrow H^{i}(\mathbb{P}^{2})
\]
be the map induced at the level of cohomology. 
We have from \ref{borelisom}
\[ H^{\bullet}(G/B) \cong \mathbb{Z}[x_1,x_2,\ldots,x_n] /{\cal{I}} 
\]
where ${\cal{I}}$ is the proper ideal of $\mathbb{Z}[x_1,x_2,\ldots,x_n]$ consisting of elementary symmetric polynomials. We have $x_i$ lies in $H^2(G/B)$. In other words, degree of $x_i$ is $2$. And we have, 
\[ H^{\bullet}(\mathbb{P}^{2}) \cong \mathbb{Z}[t]/t^3.
\] 
where degree of $t$ is $2$. Since $\phi^*(H^{2}(G/B)) \subseteq H^{2}(\mathbb{P}^2)$, we can assume
\[ \phi^*(x_i) = a_it
\] for some $a_i \in \mathbb{Z}$. Since ${\cal{I}}$ is generated by power sum symmetric polynomials, we have 
\[ \sum x_i^2 = 0 
\] 
in $H^{\bullet}(G/B)$. Thus in the image we will have,
 \[\sum a_i^2 = 0.
 \] 
Since $a_i$ are all integer we have $a_i = 0$ for all $i$. Therefore $\phi^{*i} = 0$ for all $i >0$. Hence, the map $\phi$ is a constant map. 
\end{proof}

\begin{corollary} \label{h/b_h} Let $H$ be a reductive group and $B_H$ be a Borel subgroup of $H$. Then there is no non constant morphism from $\mathbb{P}^2$ to $H/B_H$.
\end{corollary}
\begin{proof} Choose a faithful representation of $H$ in $SL(m,\mathbb{C})$ such that $B_H$ maps to a Borel subgroup $B$ of $SL(m,\mathbb{C})$. So we get a embedding of $H/B_H$ inside $SL(m,\mathbb{C})/B$. We now use theorem \ref{consp2} to conclude the proof.
\end{proof}

\begin{corollary}\label{gratogb} A morphism from $Gr(r,s)$ where $s \geq 3$ to $G/B$ is constant. 
\end{corollary}
\begin{proof} Since $\rm{Pic}(Gr(r,s))$ is $\mathbb{Z}$, we have every map from $Gr(r,s)$ to a projective variety is either finite or constant. Since $\mathbb{P}^2$ sits inside $Gr(r,s)$ whenever $s \geq 3$ and we have only constant morphism from $\mathbb{P}^2$ to $G/B$, the maps from $Gr(r,s)$ to $G/B$ must be constant as well.
\end{proof}
Let $V$ be a vector space of dimension $n$ and 
\[ 1 \leq i_1 < i_2 < \cdots < i_k = n. 
\] 
be a sequence of integers. We define $G(i_1,i_2,\ldots,i_k)$ the partial flag variety $G/P$ consisting of linear subspaces $L_{i_1}, L_{i_2},\ldots, L_{i_k}$ of $V$ such that $L_{i_j} \subset L_{i_{j+1}}$ and $\mathrm{dim} (L_{i_j}) = i_j$. 

\begin{remark} If $k = 2$ and $i_1 = d$ we obtain $G(i_1,i_2)$ as the Grassmannian variety $Gr(d,n)$. The full flag variety $G/B$ is obtained by choosing $i_j = j$. And any partial flag variety $G/P$ where $P$ contains $B$ can be obtained this way. 
\end{remark}


\begin{lemma}\label{p2togp}  There exists a $Gr(r,s)$ with $s \geq 3$ passing through each point of $G/P$ where $P$ is a parabolic subgroup which is not a Borel subgroup. 
\end{lemma}
\begin{proof} 
Since $P$ is not a Borel subgroup we have $n \geq 3$.
We are already done for the case of Grassmannian variety $Gr(d,n)$. So we can assume $k >2$ and $G/P = G(i_1,i_2,\ldots,i_k)$.
If $P$ is not $B$ then there either $i_1 > 1$ or $i_1 = 1$ and there exists a smallest $j$ such that $i_{j+1} > i_{j} + 1$. If $i_1 > 1$, then we have the fibers of the projection 
\[ G(i_1,i_2,\ldots,i_k) \longrightarrow G(i_2,i_3,\ldots,i_k)
\]
is $Gr(i_1,i_2)$ with $i_2 \geq 3$, hence we are done.

If $i_1 = 1$, choose the smallest $j$ such that $i_j = j$ and $i_{j+1} >  j +1$. If $j = 1$, ie. $i_2 > 2$, we have the fibres of the projection
\begin{equation}\label{eqngi}
 G(1,i_2,\ldots, i_k) \longrightarrow G(i_2,i_3,\ldots, i_k)
\end{equation}   
is  $\mathbb{P}^{r-1}$ where $r = i_2-1 \geq 2$. If $j \geq 2$, then we have the fibre of 
\begin{equation}\label{eqngi}
 G(i_1,i_2,\ldots, i_k) \longrightarrow G(i_1,i_2,\ldots,i_{j-1},i_{j+1},\ldots, i_k)
\end{equation} 
is $\mathbb{P}^{r-1}$ where $ r = i_{j+1}-i_{j-1} \geq 3$.

 This proves the lemma. 
\end{proof}

\begin{remark} \label{pg2} Let $P$ be a parabolic subgroup which is not a maximal parabolic or a Borel. The proof of the above lemma provides a $Gr(r,s)$-fibration $G/P \rightarrow G/P'$ for some $s \geq 3$ where $P'$ is a parabolic subgroup containing $P$
\end{remark}
\begin{corollary} \label{gmptohb} Let $H$ be a reductive group and $B_{H}$ be a Borel subgroup of $H$. Fix a parabolic subgroup $P$ of $G$ and a non constant morphism $\phi: G/P \rightarrow H/B_{H}$. Then $P$ is a Borel subgroup. 
\end{corollary}

\begin{proof} We know that any $H/B_H$ embeds inside a $G/B$ where $B$ is a Borel subgroup of $G$. So we are reduced to the case where $H =G$ and $B_H$ is a Borel subgroup of $G$.

We assume on the contrary that $P$ is not a Borel subgroup. If $P$ is a maximal parabolic subgroup then by lemma \ref{gratogb} the map $\phi$ must be constant.

We can therefore assume $P$ not a Borel or a maximal parabolic subgroup. From corollary \ref{pg2} we obtain a parabolic subgroup $P'$ containing $P$ such that $G/P \rightarrow G/P'$ is $Gr(r,s)$-fibration with $s \geq 3$. Since $\phi$ is constant on $Gr(r,s)$ we have $\phi$ factors through $G/P'$. Repeating the argument we can assume that $\phi$ factors through a $G/Q$ where $Q$ is a maximal parabolic subgroup and hence we conclude that $\phi$ is constant.

\end{proof}

\begin{remark}  Shrawan Kumar \cite{kumarsconj} has extended corollary \ref{gmptohb} to an arbitrary simple group $G$. 
\end{remark}

\section{Maps from $\mathbb{P}^3$ to $G/P$ for a minimal parabolic subgroup}


We assume the notations from the previous sections. We thus have $P_{\alpha}$ the minimal parabolic subgroup $B \cup Bs_{\alpha}B$. When $\alpha = \alpha_1$ we will show that there is no non constant map from $\mathbb{P}^3$ to $G/{P_{\alpha}}$. Since $G/P_{\alpha_1} \cong G/P_{\alpha_{n-1}}$ we conclude that there is no non constant map from $\mathbb{P}^3$ to $G/{P_{\alpha_{n-1}}}$ as well.  However, when we have any other minimal parabolic subgroup $P_{\alpha_{j}}$, $j\neq 1,n-1$ we will show that there are non constant maps from $\mathbb{P}^3$ to $G/{P_{\alpha_j}}$. 


Fix a basis $e_1, e_2, \ldots, e_n$ of $V$. Define the subspaces $M_i$ to the span of $e_1, e_2 \ldots, e_i$. Let $D_k$ denote the Schubert divisor in the $Gr(k,n)$ which is defined as 
\[ D_k = \{ F \in Gr(k,n) | F \cap M_{n-k} \neq 0 \}. 
\] 
We define the following two codimension 2 Schubert subvarieties of the Grassmannian $Gr(k,n)$:  
\[ D_{k,k+1} := \{ F \in Gr(k,n) | ~F \cap M_{n-k-1} \neq 0 \} 
\] 
\[ D_{k,k-1} := \{ F \in Gr(k,n) | ~\rm{dim}(F \cap M_{n-k+1} \geq 2) \}.
\] 
We note that $D_{n,n+1}$ and $D_{1,0}$ are empty sets.
We prove the following lemmas.

\begin{lemma}\label{lm41} Let $1\leq k \leq n$. We have the following relation in $H^4(Gr(k,n))$ 
\[D_{k}.D_{k} = D_{k,k-1} + D_{k,k+1}.\]
\end{lemma}
\begin{proof} To prove the lemma we would intersect the Schubert divisors fixing two different complimentary $n-k$ dimensional vector subspaces. Let $M'_{n-k}$ is the vector space generated by $M_{n-k-1}$ and $e_{n-k+1}$. Let $D'_k := \{ F \in Gr(k,n) | ~F \cap M'_{n-k} \neq 0 \} $ be the divisor linearly equivalent to $D_k$ defined with respect to $M'_{n-k}$. Then we can see that 
\[ D_k \cap D'_k = \{ F\in Gr(k,n) | ~F \cap M_{n-k} \neq 0\} \cap \{F \in Gr(k,n)|~ F\cap M'_{n-k} \neq 0\}
\]
\[ =\{F\in Gr(k,n) | F\cap M_{n-k-1} \neq 0 \} \cup \{\rm{dim}(F \cap M_{n-k+1}) \geq 2\}
\]
which by definition is $D_{k,k+1} \cup D_{k,k-1}$. Hence, the lemma follows.
\end{proof}

\begin{lemma}\label{lm42} We have the following relation in the cohomology $H^4(G(k,k+1,n))$ 
\[D_{k}.D_{k+1} = D_{k,k+1} + D_{k+1,k}.\]
\end{lemma}
\begin{proof} We note that the intersection of $D_{k}$ with $D_{k+1}$ is happening at $G(k,k+1,n)$. $D_{k+1}$ is linearly equivalent to $\{ (F,E) \in G(k,k+1,n) | ~ E \cap M_{n-k-1} \neq 0\}$ in $G(k,k+1,n)$. We observe that both $D_{k,k+1}$ and $D_{k+1,k}$ lie in the intersection of $D_{k}$ and $D_{k+1}$. If we choose a $F$ from the intersection not in $D_{k,k+1}$ we observe that $F \cap M_{n-k-1} = 0$ and $F\cap M_{n-k} \neq 0$. Then $F \cap M_{n-k}$ and $E\cap M_{n-k-1}$ are non zero and linearly independent, so they span atleast two dimensional vector space and it is contained in $E \cap M_{n-k}$. So we have, 
 \[ \{ F \in Gr(k,n) |~F\cap M_{n-k} \neq 0 \} \cap \{ (F,E) \in G(k,k+1,n) | ~ E \cap M_{n-k-1} \neq 0\}\]
\[ = \{F \in Gr(k,n) |~F \cap M_{n-k-1} \neq 0 \} \cup \{ E\in Gr(k+1,n) |~\rm{dim} (E \cap M_{n-k}) \geq 2\}.  
\] 
Hence, the lemma follows.
\end{proof}

Let $E_1$ denote the codimension $3$ Schubert cycle defined by the Schubert variety  $\{F\in Gr(2,n) |~F \cap M_{n-3} \neq 0 ~\rm{and} ~F \subset M_{n-1}\}$. Let $E_2$ be the codimension $3$ Schubert cycle defined by the Schubert variety $\{F\in Gr(2,n) |~F \cap M_{n-4} \neq 0 \}$. 

\begin{lemma}\label{lm3} We have the following relations in $H^6(Gr(2,n))$ : 
\begin{itemize}
\item[(i)] $D_{2,1}. D_2 = E_1$.
\item[(ii)] $D_{2,3}. D_2 = E_1 + E_2$. 
\end{itemize}
\end{lemma}
\begin{proof} (i) Let $M_{n-2}''$ be the $n-2$ dimensional vector space spanned by $e_1,e_2, \ldots, e_{n-3}, e_{n}$. Let $D_2''$ be the divisor linearly equivalent to $D_2$ defined by 
$\{ F \in Gr(2,n) | ~F \cap M''_{n-2} \neq 0 \}$. We have $F \subset M_{n-1}$ as it is in $D_{2,1}$. It follows that $F \cap M_{n-2}'' \subset M_{n-1} \cap M_{n-2}'' = M_{n-3}$ is nonzero. Therefore, $F \cap M_{n-3} \neq 0$.

(ii) Let $M_{n-2}'''$ be the $n-2$ dimensional vector space spanned by $e_1,e_2,\ldots, e_{n-4},e_{n-2},e_{n-1}$. Let $D_2'''$ be the divisor linearly equivalent to $D_2$ defined by 
$\{ F \in Gr(2,n) | ~F \cap M'''_{n-2} \neq 0 \}$. Let $F$ be in the intersection of $D_{2,3}$ and $D'''_{2}$. If $F \cap M_{n-4} \neq 0 $ then $F$ is the component $E_2$. So we assume $F \cap M_{n-4} = 0$. Notice that $M_{n-4} = M_{n-2} \cap M_{n-2}'''$. But on the other hand $F\cap M_{n-2} \neq 0 $ and $F\cap M'''_{n-2} \neq 0 $, therefore $F$ is contained in the span of $M_{n-2}$ and $M_{n-2}'''$ which is $M_{n-1}$. Hence $F$ is contained in $M_{n-1}$, i.e $F \in E_1$. Hence the lemma.


\end{proof}

Since the map $H^{*}(Gr(k,n))$ to $H^{*}(G/B)$ is injective the above relations holds in $H^*(G/B)$ as well. We use the same notations $D_i$ and $D_{i,j}$ to define the Schubert classes in $H^*(G/B)$. Note that the above relations can also be deduced from Monk's formula. 

\begin{theorem}\label{consp3}
There is no nonconstant map from $\mathbb{P}^3$ to $G/{P_{\alpha_1}}$. 
\end{theorem}

\begin{proof}
Let $P = P_{\alpha_1}$. So $G/P$ is $G(2,3,\ldots,n)$. Let
\[
\phi : \mathbb{P}^3 \longrightarrow G/P 
\] be a  map. Let  
\[ \phi^{*}: H^{\bullet}(G/P) \longrightarrow H^{\bullet}(\mathbb{P}^{3})
 \] 
 be the map at the level of cohomology. Let
 \[ \phi^{*i}: H^{i}(G/P) \longrightarrow H^{i}(\mathbb{P}^{3})
 \] be the map at degree $i$. 
 We know that $H^{*}(\mathbb{P}^{3}) \cong \mathbb{Z}[t]/(t^4)$.
 We will show that $\phi^{*i} = 0$ for all $i > 0$.

We know that the divisors in $G(2,3,\ldots,n)$ are $D_2, D_3, \ldots, D_{n-1}$. From \ref{lm41} and \ref{lm42} we have the following relations in $H^{*}(G/P)$
\begin{align*}
D_2D_2 &= D_{2,1} + D_{2,3} \\
D_2D_3 &= D_{2,3} + D_{3,2} \\
D_3D_3 &= D_{3,2} + D_{3,4} \\
&\;\;\vdots \notag \\
D_{n-2}D_{n-1} &= D_{n-2,n-1} + D_{n-1,n-2}\\
D_{n-1}D_{n-1} &= D_{n-1,n-2} \\
\end{align*}

Letting $\phi^{*}(D_i) = a_{i}t$ in $H^2(\mathbb{P}^3)$ and  $\phi^{*}(D_{i,j}) = b_{i,j}t^2$ in $H^4(\mathbb{P}^3)$ we obtain the following relations in $H^{*}(\mathbb{P}^3)$. 
\begin{flalign*}
a_2^2 &= b_{2,1} + b_{2,3}\\
a_2a_3 &= b_{2,3} + b_{3,2} \\
 &\;\;\vdots \notag \\
a_{n-1}^2 &= b_{n-1,n-2}
\end{flalign*}

So rewriting $b_{i,j}$ in terms of $a_{i,j}$ we obtain 
\begin{align}
b_{n-1,n-2} &= a_{n-1}^2 \nonumber \\
b_{n-2,n-1} &= a_{n-1}a_{n-2} - a_{n-1}^2 \nonumber \\
&\;\;\vdots \notag \\
b_{i,i-1} &= (a_{i}^2 + a_{i+1}^2+ \ldots a_{n-1}^2) - (a_{i}a_{i+1} + a_{i+1}a_{i+2} \ldots + a_{n-2}a_{n-1}) \nonumber  \\
b_{i-1,i} &= (a_{i-1}a_{i} + a_{i}a_{i+1} + \ldots + a_{n-2}a_{n-1}) -(a_{i}^2 + a_{i+1}^2+ \ldots a_{n-1}^2) \nonumber \\
 &\;\;\vdots \notag \\
b_{2,3} &= (a_2a_3 + a_3a_4 + \ldots + a_{n-2}a_{n-1}) - (a_{3}^2 + \ldots + a_{n-1}^2) \nonumber  \\ 
b_{2,1} &= (a_{2}^2 + \ldots + a_{n-1}^2) - (a_2a_3 + a_3a_4 + \ldots + a_{n-2}a_{n-1}). \nonumber \\
\nonumber 
\end{align}


Let $\phi^{*}(E_1) = c_1t^3$ and $\phi^{*}(E_2) = c_2t^3$. Therefore from \ref{lm3} we get 
\[ b_{2,1} a_2 = c_1 \] 
\[ b_{2,3 }a_2 = c_1 + c_2 .\]

We know Schubert classes are represented by algebraic cycles and hence their pullbacks are algebraic cycles in the projective space. Therefore, Schubert polynomials are mapped to non negative classes in the cohomology of projective spaces. So $c_2 \geq 0$. Therefore, we obtain $b_{2,3 }a_2 \geq b_{2,1}a_2$. We have $a_2 \geq 0$. If $a_2 >0 $ we observe $b_{2,3 } \geq b_{2,1}$. Hence, 
\[ (a_2a_3 + a_3a_4 + \ldots + a_{n-2}a_{n-1}) - (a_{3}^2 + \ldots + a_{n-1}^2) \geq (a_{2}^2 + \ldots + a_{n-1}^2) - (a_2a_3 + a_3a_4 + \ldots + a_{n-2}a_{n-1})  
\] which implies that 
\[ (a_2- a_3)^2 + (a_3-a_4)^2 + \ldots (a_{n-2} - a_{n-1})^2 + a_{2}^2 \leq 0
\] which forces $a_i = 0$ for all $i$.

If $a_2 = 0$ we have $b_{2,3} = b_{2,1} = 0$. Then we obtain 
\[ a_{3}^2 + a_{4}^2 + \ldots + a_{n-1}^2  = a_{3}a_{4} + \ldots a_{n-2}a_{n-1}. 
\]
which implies 
\[ (a_3-a_4)^2 + (a_4 - a_5)^2 + \ldots + (a_{n-2} - a_{n-1})^2 + a_{3}^2 + a_{n-1}^2 =  0
\] which forces all $a_i = 0$.
Therefore, we conclude that $\phi^{*i} =0$ for all $i > 0$.
\end{proof}

\begin{corollary} There is no nonconstant map from $\mathbb{P}^3$ to $G/{P_{\alpha_{n-1}}}$. 
\end{corollary}

\begin{proof} 

Since $G= SL(n,\mathbb{C})$, we have an automorphism of $G$ which is induced by the Dynkin involution taking $\alpha_i$ to $\alpha_{n-i}$ for all $ 1\leq i \leq n-1$. Under this automorphism we have $P_{\alpha_1}$ isomorphic to $P_{\alpha_{n-1}}$. We have $G/P_{\alpha_1}$ isomorphic to $G/P_{\alpha_{n-1}}$.

\end{proof}


\begin{lemma} \label{p3tog134}
 There is a morphism from $\mathbb{P}^3$ to $G(1,3,4)$. 
\end{lemma}

\begin{proof}
Let $V$ be a vector space of dimension $4$. Let $\mathbb{P}^3$ be the projective space of lines in $V$. Fix a non-degenerate alternating bilinear form. Because the form is non-degenerate and alternating it follows that for every line $L$ the orthogonal compliment $L^{\perp}$ of $L$ is a $3$ dimensional subspace of $V$ containing $L$. Hence, $(L,L^{\perp}, V)$ is an element of $G(1,3,4)$ and the map $L \mapsto (L,L^{\perp}, V)$ defines the required morphism.
\end{proof}

\begin{theorem} There are maps from $\mathbb{P}^3$ to $G/P_{\alpha}$ for all minimal parabolic subgroup  $P_{\alpha}$ with  ${\alpha} \notin \{ {\alpha_{1}, \alpha_{n-1}} \}$.
\end{theorem}

\begin{proof} Let $\alpha = \alpha_j$ where $2 \leq j \leq n-2$. Fix a flag
 \[ L_1 \subset L_2 \cdots \subset L_{j-2} \subset L_{j+2} \subset L_{j+3} \cdots  \subset L_{n-1} \subset L_n 
 \] 
 where dimension of $L_j = j$.
Then the fiber over this flag of the map 
\[ G(1,2,\ldots,j-1,j+1,j+2,\ldots,n-1,n) \longrightarrow G(1,2,\ldots,j-2, j+2,\ldots, n-1, n )
\]
is isomorphic to $G(1,3,4)$ which is identified as the flags $(L_{j-1}/L_{j-2},L_{j+1}/L_{j-2},L_{j+2}/L_{j-2})$. So we have a map from $G(1,3,4)$ to $G/P_{\alpha}$. And using lemma \ref{p3tog134} we proof the theorem.

\end{proof}


\section{Declaration}
\textbf{Conflicts of interest} 
The authors declare that they have no conflicts of interest.


\bibliographystyle{amsplain}
\bibliography{references}



\end{document}

