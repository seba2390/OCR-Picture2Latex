\documentclass[article,reprint,amsmath,amssymb,superscriptaddress,longbibliography]{revtex4-1}
%[floatfix]

%\usepackage{scicite}
%\usepackage{hyperref}
\usepackage[colorlinks,
linkcolor=blue,
anchorcolor=blue,
citecolor=blue,
]{hyperref}
%就可以让生成的文章目录有链接

%\usepackage[pagewise]{lineno}




\usepackage{graphicx}% Include figure files untitled
\usepackage{epstopdf}
%\usepackage{dcolumn}% Align table columns on decimal point
%\usepackage{bm}% bold math
\usepackage{hyperref}
\UseRawInputEncoding
\setcitestyle{super} %superscript
%super,numbers,square
%\setcitestyle{numbers,round}
%\renewcommand*\citenumfont{\textit}


%\usepackage{ulem}
\usepackage{breakurl}
\usepackage{multirow}
\usepackage{enumitem}
\usepackage{color}

\usepackage{tablefootnote}
\usepackage{threeparttable}
\usepackage{tabularx}
% Please add the following required packages to your document preamble:
\usepackage{booktabs}
%\usepackage{appendix}
\usepackage{textcomp}
\usepackage{lineno}
%\linenumbers

\begin{document}

\title {Evidence for chiral superconductivity in Kagome superconductor CsV$_3$Sb$_5$}

\author{Tian Le}
\thanks{Equal contributions}
\affiliation{Key Laboratory for Quantum Materials of Zhejiang Province, Department of Physics, School of Science, Westlake University, Hangzhou 310030, P. R. China}
\affiliation{Institute of Natural Sciences, Westlake Institute for Advanced Study, Hangzhou 310024, P. R. China}
% \affiliation{Department of Physics, China Jiliang University, Hangzhou 310018, Zhejiang, P. R. China}

\author{Zhiming Pan}
\thanks{Equal contributions}
\affiliation{Key Laboratory for Quantum Materials of Zhejiang Province, Department of Physics, School of Science, Westlake University, Hangzhou 310030, P. R. China}
\affiliation{Institute of Natural Sciences, Westlake Institute for Advanced Study, Hangzhou 310024, P. R. China}

\author{Zhuokai Xu}
\affiliation{Key Laboratory for Quantum Materials of Zhejiang Province, Department of Physics, School of Science, Westlake University, Hangzhou 310030, P. R. China}
\affiliation{Institute of Natural Sciences, Westlake Institute for Advanced Study, Hangzhou 310024, P. R. China}
% \affiliation{Department of Physics, China Jiliang University, Hangzhou 310018, Zhejiang, P. R. China}

\author{Jinjin Liu}
\affiliation{Centre for Quantum Physics, Key Laboratory of Advanced Optoelectronic Quantum Architecture and Measurement (MOE), School of Physics, Beijing Institute of Technology, Beijing 100081, China} 
\affiliation{Beijing Key Lab of Nanophotonics and Ultrafine Optoelectronic Systems, Beijing Institute of Technology, Beijing 100081, China}
%\affiliation{Material Science Center, Yangtze Delta Region Academy of Beijing Institute of Technology, Jiaxing 314011, China}

\author{Jialu Wang}
\affiliation{Key Laboratory for Quantum Materials of Zhejiang Province, Department of Physics, School of Science, Westlake University, Hangzhou 310030, P. R. China}
\affiliation{Institute of Natural Sciences, Westlake Institute for Advanced Study, Hangzhou 310024, P. R. China}

\author{Zhefeng Lou}
\affiliation{Key Laboratory for Quantum Materials of Zhejiang Province, Department of Physics, School of Science, Westlake University, Hangzhou 310030, P. R. China}
\affiliation{Institute of Natural Sciences, Westlake Institute for Advanced Study, Hangzhou 310024, P. R. China}

\author{Zhiwei Wang}
\email{zhiweiwang@bit.edu.cn}
\affiliation{Centre for Quantum Physics, Key Laboratory of Advanced Optoelectronic Quantum Architecture and Measurement (MOE), School of Physics, Beijing Institute of Technology, Beijing 100081, China} 
\affiliation{Beijing Key Lab of Nanophotonics and Ultrafine Optoelectronic Systems, Beijing Institute of Technology, Beijing 100081, China}
\affiliation{Material Science Center, Yangtze Delta Region Academy of Beijing Institute of Technology, Jiaxing 314011, China}

\author{Yugui Yao}
\affiliation{Centre for Quantum Physics, Key Laboratory of Advanced Optoelectronic Quantum Architecture and Measurement (MOE), School of Physics, Beijing Institute of Technology, Beijing 100081, China} 
\affiliation{Beijing Key Lab of Nanophotonics and Ultrafine Optoelectronic Systems, Beijing Institute of Technology, Beijing 100081, China}
\affiliation{Material Science Center, Yangtze Delta Region Academy of Beijing Institute of Technology, Jiaxing 314011, China}

\author{Congjun Wu}
\email{wucongjun@westlake.edu.cn}
\affiliation{Key Laboratory for Quantum Materials of Zhejiang Province, Department of Physics, School of Science, Westlake University, Hangzhou 310030, P. R. China}
\affiliation{Institute of Natural Sciences, Westlake Institute for Advanced Study, Hangzhou 310024, P. R. China}
\affiliation{New Cornerstone Science Laboratory, Department of Physics, School of Science, Westlake University, 310024, Hangzhou, China}
\affiliation{Institute for Theoretical Sciences, Westlake University, 310024, Hangzhou, China}

\author{Xiao Lin}
\email{linxiao@westlake.edu.cn}
\affiliation{Key Laboratory for Quantum Materials of Zhejiang Province, Department of Physics, School of Science, Westlake University, Hangzhou 310030, P. R. China}
\affiliation{Institute of Natural Sciences, Westlake Institute for Advanced Study, Hangzhou 310024, P. R. China}


\date{\today}

\begin{abstract}
\noindent
The interplay among frustrated lattice geometry, nontrivial band topology and correlations yields rich quantum states of matter in Kagome systems~\cite{Balents2010Nature,WenXG2009PRB}. 
A class of recent Kagome metals, $A$V$_3$Sb$_5$ ({$A$}= K, Rb, Cs), exhibit a cascade of symmetry-breaking transitions~\cite{Zeljkovic2021nature}, involving 3Q chiral charge ordering~\cite{Miao2021PRX,Wilson2021PRX,Guguchia2022Nature,Hasan2021NM,WuL2022NP,Moll2022Nature}, electronic nematicity~\cite{ChenXH2022Nature,Zeljkovic2022NP}, roton 
pair density wave~\cite{GaoHJ2021Nature} and superconductivity~\cite{Wilson2020PRL}. 
The interdependence among multiple competing orders suggests unconventional superconductivity~\cite{ChenXH2022Nature1}, the nature of which is yet to be resolved. 
Here, we report the electronic evidence for chiral superconducting domains with boundary supercurrent, a smoking-gun of chiral superconductivity, in intrinsic CsV$_3$Sb$_5$ flakes. Magnetic field-free superconducting diode effects are observed with its polarity modulated by thermal histories, unveiling a spontaneous time-reversal-symmetry breaking within dynamical order parameter domains in the superconducting phase. Strikingly, the critical current exhibits double-slit superconducting interference patterns, when subjected to external magnetic field. This is attributed to the periodic modulation of supercurrent flowing along chiral domain boundaries constrained by fluxoid quantization. Our results provide the direct demonstration of a time-reversal
symmetry breaking superconducting order in Kagome systems, opening a potential for exploring exotic physics, e.g. Majorana zero modes, in this intriguing topological Kagome system.
%\textcolor{red}{Our results provide a direct demonstration of chiral edge condensates in kagome superconductors, opening a potential for exploring exotic physics, e.g. Majorana edge modes, in this intriguing topological phase of matter.} %Our results reveal significant insights into the entangled electronic orders in the V-based kagome metals.
	
	%\textbf{Keywords: } 
\end{abstract}

\maketitle

%\section{Introduction}
%\vspace{-10 pt}


\noindent
Chiral superconductors (SC), characterized by complex order parameters, break time-reversal symmetry spontaneously. Certain types of chiral SCs are topologically non-trivial whose gap
functions exhibit non-trivial phase winding numbers over the Fermi surface. They allow for the exploration of chiral edge modes and Majorana zero modes, showing a promise for fault-tolerant topological quantum computation~\cite{Ivanov2001PRL}. A well-known example exhibiting chiral pairing symmetry is the superfluid $^3$He-A phase~\cite{Leggett1975RMP}. Evidence of time-reversal symmetry breaking has been reported in materials~\cite{Kallin2016RPP,Kapitulnik2014Science,JiaoL2020Nature,Weitering2023NP}, including Sr$_2$RuO$_4$~\cite{Kallin2016RPP}, UPt$_3$~\cite{Kapitulnik2014Science} and UTe$_2$ ~\cite{JiaoL2020Nature}. Nevertheless, an unequivocal demonstration of chiral edge supercurrent ($I_\textrm{e}$) at SC domain boundaries is still a pending target.

%\textcolor{blue}{CW: I suggest to remove Sr2RuO4, which is quite controversial now.}
%~\cite{Ivanov2001PRL,Fujimoto2010PRB}
%and  Sn/Si(111) (d-wave)~\cite{Weitering2023NP}
%offer potenial for

$A$V$_3$Sb$_5$ ($A$=K, Rb, Cs) exhibits a rich phase diagram featured by multiple intertwined  orders~\cite{Miao2021PRX,Wilson2021PRX,Guguchia2022Nature,Hasan2021NM,WuL2022NP,Moll2022Nature,ChenXH2022Nature,Zeljkovic2022NP,Zeljkovic2023NP,GaoHJ2021Nature}, bearing a remarkable resemblance to cuprates~\cite{Zaanen2015Nature}. Despite the intense debate surrounding the SC pairing symmetry~\cite{LiSY2021Arxiv,Khasanov2023NC}, accumulated evidence~\cite{LuoJL2021CPL,Yuan2021SCPMA,Shibauchi2023NC,Okazaki2023Nature,FengDL2021PRL}, obtained from nuclear quadrupole resonance (NQR)~\cite{LuoJL2021CPL}, tunnel diode oscillator~\cite{Yuan2021SCPMA}, electron irradiation~\cite{Shibauchi2023NC}, angle-resolved photoemission spectroscopy (ARPES)~\cite{Okazaki2023Nature}, etc., points to the existence of nodeless,  spin-singlet, and nearly isotropic order parameters. These observations appeared to conflict with theoretical analysis~\cite{Thomale2021PRL,LiJX2012PRB}, that predicted unconventionality and even chiral SC within a certain parameter space of the Kagome lattice. Also inspired by the interplay between SC and exotic electronic orders in $A$V$_3$Sb$_5$~\cite{ChenXH2022Nature1,Hasan2021NM,GaoHJ2021Nature}, it is intriguing to imagine whether the SC is chiral. Relevant clues are contradictory~\cite{Guguchia2022Nature,Khasanov2022CP,Khasanov2023NC,Okazaki2023Nature}, exclusively derived from muon spin spectroscopy ($\mu$SR) that detects signals of broken time-reversal-symmetry (TRS). Phase-sensitive probes are crucial, but remain lacking. Here, we present phase-sensitive evidence for chiral SC domains in CsV$_3$Sb$_5$ via critical current measurements.   

%~\\
%\section{Results and discussion}
%\noindent\textbf{Results}

%-----------------------------------------------------------
\begin{figure*}[thb]
	\begin{center}
		\includegraphics[width=18cm]{Fig1.png}
	\end{center}
	\setlength{\abovecaptionskip}{-8 pt}
    \caption{\textbf{Zero-field superconducting diode.} \textbf{a}, Crystal structure of CsV$_3$Sb$_5$. \textbf{b}, $T$-dependence of $\rho$ for B1 and $R$ for D1 %(thickness $d\approx40$~nm) 
	in the full-$T$ range. The upper inset is the optical image of D1. The lower inset presents the normalized resistance $R/R_\textrm{n}$ around $T_\textrm{c}$, where $R_\textrm{n}$ is the normal state resistance and $T_\textrm{c}$ is the critical temperature determined at zero resistance. \textbf{c}, Differential resistance ($\textrm{d}V/\textrm{d}I$) as a function of d.c. current bias ($I$) at various $T$ for D1. 
	The red and blue curves are collected in positive ($I_\textrm{+}$) and negative ($I_\textrm{-}$) bias regimes, respectively. 
	Curves are offset from each other by 3 $\Omega$ for clarity. \textbf{d}, Enlarged curve of \textbf{c} at $T=0.1$~K. %$I_\textrm{c+}$ and $I_\textrm{c-}$ are the critical currents in %$I_\textrm{+}$ and $I_\textrm{-}$ regimes. 
	\textbf{e}, $T$-dependence of average critical current ($\bar{I_\textrm{c}}$) and $\Delta I^\textrm{SDE}_\textrm{c}$, where $\bar{I_\textrm{c}}=(I_\textrm{c+}+I_\textrm{c-})/2$ and $\Delta I^\textrm{SDE}_\textrm{c}=I_\textrm{c+}-I_\textrm{c-}$. %The dotted line is the fit to $(T_\textrm{c}-T)^{3/2}$.
    }
	\label{Fig1}
\end{figure*}
%------------------------------------------------------------

%\noindent
CsV$_3$Sb$_5$ has a layered hexagonal structure~\cite{Toberer2019PRM}, composed of alternating stacks of V$_3$Sb$_5$ slabs and Cs layers, among which vanadium ions form the Kagome net, seen in Fig.~\ref{Fig1}a. Fig.~\ref{Fig1}b presents the temperature ($T$) dependent resistivity ($\rho$) for a bulk crystal (B1) and resistance ($R$) for a mechanically exfoliated specimen (D1). The residual-resistance-ratio (RRR) amounts to 250 in B1, highlighting the quality of our crystals. The transition to charge density wave (CDW) in B1 appears at $T_\textrm{CDW}\approx92$~K~\cite{Toberer2019PRM,Wilson2020PRL}, accompanied by a SC phase transition at $T_\textrm{c}\approx3$~K, consistent with previous reports~\cite{Wilson2020PRL}. In D1, $T_\textrm{CDW}$ is reduced to 80~K with the enhancement of $T_c$ to 3.5~K, 
as reported in the literature~\cite{LiSY2023NC}. $R$-$T$ for two other devices (D2 and D3) are presented in Extended Data Fig.~1. 



~\\
\noindent\textbf{Zero-field superconducting diode effect}

\noindent
The chiral SC is first inspected by leveraging the SC diode effect (SDE), which
depicts an asymmetry of the d.c. critical current ($I_\textrm{c}$) with respect to  the direction of current flow in the absence of TRS and inversion symmetry (IR) ~\cite{Ono2020Nature,Parkin2022NP,JiangK2022PRX}. 
Note that TRS and IR can be disrupted either internally or externally. 
In the case of internal TRS breaking, magnetic field ($B$)-free SDE~\cite{Ali2022Nature,Parkin2022NM,Li2022NP} could be realized and 
may reflect intrinsic pairing symmetry~\cite{Li2022NP}. 
%. as discussed in twisted trilayer graphene~\cite{Li2022NP}.,FuL2022PNAS,Ono2022NN,Yanase2021PRL

To examine intrinsic properties, non-SC contacts are made by gold deposition as seen in the inset of Fig.~\ref{Fig1}b. In Fig.~\ref{Fig1}c, the differential resistance ($dV/dI$) for D1 was 
measured by sweeping the d.c. current ($I$) at zero $B$ and various $T$. Before the experiments, the cryostat was warmed up to room temperature to fully release the residual flux trapped in the SC magnet. $I_\textrm{c}$ evolves with the reduction of $T$ along with a noticeable 
inequivalence between the positive ($I_\textrm{+}$) and negative ($I_\textrm{-}$) bias regimes. 
A magnified curve at $T=0.1~$K is specified in Fig.~\ref{Fig1}d.
It is resolved that the critical current $I_\textrm{c+}$
along the positive direction is larger than $I_\textrm{c-}$ 
along the negative one 
($\Delta I^\textrm{SDE}_\textrm{c}=I_\textrm{c+}-I_\textrm{c-}\approx3~\mu$A), indicating non-reciprocity. Additionally, several non-reciprocal transition features (marked by arrows) are observed above $I_\textrm{c}$. We will show below that it is related to chiral SC domains. 
Fig.~\ref{Fig1}e displays the average $\bar{I_\textrm{c}}$ and $\Delta I^\textrm{SDE}_\textrm{c}$ versus $T$. As $T$ declines, $\Delta I^\textrm{SDE}_\textrm{c}$ shows a peculiar sign switching slightly below $T_\textrm{c}$, indicated by green circles in Fig. ~\ref{Fig1}c, and eventually becomes nearly constant. This polarity fluctuation is weird, distinct from what was reported~\cite{Parkin2022NP,Parkin2022NM}. The demonstration of zero-field SDE is against extrinsic origins tied to vortex dynamics induced by external $B$, like asymmetric surface barriers~\cite{Suri2022APL} and vortex ratchets~\cite{WangX2022PRB}. This supports intrinsic TRS breaking in the SC phase of CsV$_3$Sb$_5$.

%The zero-field SDE demonstration strongly counters extrinsic origins tied to vortex dynamics induced by external magnetic fields, like asymmetric vortex surface barriers~\cite{Suri2022APL} and vortex ratchets~\cite{WangX2022PRB}. This supports intrinsic TRS breaking in the superconducting phase.

%In Fig.~\ref{Fig1}e, the average $\bar{I_\textrm{c}}$ is extracted from Fig.~\ref{Fig1}c along with $\Delta I_\textrm{c}$.  $\bar{I_\textrm{c}}$ increases with lowering $T$ and levels off as $T$ approaches zero, which roughly fits $(T_\textrm{c}-T)^{3/2}$ from the Ginzburg-Landau theory near $T_\textrm{c}$~\cite{Yanase2021PRL,FuL2022PNAS}. While, as $T$ declines, $\Delta I_\textrm{c}$ goes from positive to negative just below $T_\textrm{c}$, then returns to positive again and eventually becomes nearly constant. The polarity fluctuation near $T_\textrm{c}$ is distinct from what is expected from the theory~\cite{Yanase2021PRL,FuL2022PNAS}. The zero-field SDE demonstrated in this study provides strong evidence against extrinsic origins associated with vortex dynamics generated by external $B$, such as asymmetric vortex surface barriers~\cite{Suri2022APL} and vortex ratchets~\cite{WangX2022PRB}, supporting the presence of intrinsic TRS breaking in the SC phase of CsV$_3$Sb$_5$.

%unequivocally excludes 



\begin{figure*}[thb]
	\begin{center}
		\includegraphics[width=16cm]{Fig2.png}
	\end{center}
	\setlength{\abovecaptionskip}{-8 pt}
	\caption{\textbf{Thermal modulation of SDE polarity at $B=0$~Gs and $T=1.4$~K.} \textbf{a}, Optical image of D2. All the terminals are numbered (1-8), where 1 and 8 are for current and others are for voltage. \textbf{b}, $\textrm{d}V/\textrm{d}I$ versus $I$ for the terminals $V_{3-4}$ ($\Delta I_\textrm{c}>0$). \textbf{c}, $\textrm{d}V/\textrm{d}I$ versus $I$ for the same terminals after recooling from $4.5$~K, where $\Delta I_\textrm{c}$ is reversed. \textbf{d-f}, $\textrm{d}V/\textrm{d}I$ versus $I$ for $V_{6-7}$. The non-reciprocity is not obvious at the beginning (\textbf{d}). After thermal cycling, apparent SDE responses are activated with the polarity either positive (\textbf{e}) or negative (\textbf{f}). %\textbf{g-h}, Half-wave rectification from $V_{6-7}$ with positive (\textbf{g}) and negative polarity (\textbf{h}). The measurements were performed by applying square-wave excitation \textcolor{red}{at frequency ...}. SDE remains stable after 100 cycles.
	}
	\label{Fig2}
\end{figure*}

Next, we demonstrate thermal switching of SDE polarity on a multi-terminal device (D2) illustrated in Fig.~\ref{Fig2}a (see data for D1 and D3 in Extended Data Fig.~2). $V_{3-4}$ initially exhibits apparent non-reciprocity ($\Delta I^\textrm{SDE}_\textrm{c}>0$) at 1.4~K. 
Subsequently, the device is heated to 4.5~K, slightly above $T_\textrm{c}$, and re-cool it. The polarity is then reversed as exhibited in Fig.~\ref{Fig2}c. Moreover, SDE could be excited by thermal cycling. In Fig.~\ref{Fig2}d, $V_{6-7}$ shows negligible non-reciprocity in the initial state. After thermal cycling, a finite SDE with either positive and negative polarity is induced, seen in Fig.~\ref{Fig2}e and f. The demonstration of half-wave rectification is shown in Extended Data Fig.~3. Curves of other terminals are shown in Extended Data Fig.~4. It should be emphasized that not only the polarity, but also the magnitude of $\Delta I^\textrm{SDE}_\textrm{c}$ and $\bar{I_\textrm{c}}$ are changed by thermal cycling. These results, in combination with polarity fluctuation, hint the possible existence of SC domains with broken TRS, i.e. chiral SC domains. They become dynamically active in the vicinity of $T_\textrm{c}$, giving rise to thermal modification of SDE. %\textcolor{blue}{Futher details are discussed in Supplementary Section...}


~\\
\noindent\textbf{Superconducting interference patterns}

\begin{figure*}[thb]
	\begin{center}
		\includegraphics[width=18cm]{Fig3.png}
	\end{center}
	\setlength{\abovecaptionskip}{-8 pt}
	\caption{\textbf{Superconducting interference patterns on intrinsic CsV$_3$Sb$_5$ flakes.} \textbf{a}, SIPs on D1 measured at 1.4 K. The solid line indicates y-axis at $B=0$~Gs. See the patterns at a broader $B$ scale in Extended Data Fig.~5. \textbf{b}, Corresponding sets of $\textrm{d}V/\textrm{d}I$ versus $I$ at $-1.2<B<1.2$~Gs. Three sets of SIPs in \textbf{a} are traced out by the transition anomalies, delineated by dashed curves of distinct colors and denoted as $I_\textrm{c}$, $I'_\textrm{c}$ and $I''_\textrm{c}$. \textbf{c}, SIPs on D2 measured at 1.4 K. \textbf{d}, Left: Illustration of the LP effect in a thin hollow SC cylinder. The magnetic flux (black arrows) penetrating through the hollow generates supercurrent oscillation. Right: Path loop indicating the flow of $I_\textrm{e}$ inside the device. \textbf{e}, Standard SIP derived from Eq.~\ref{Eq1}. \textbf{f}, SIP simulated in the presence of spontaneous TRS breaking (Solid curve). \textbf{g}, Physical area ($S_\textrm{phy}$) versus the flux penetration area ($S_{\phi}$) for D1, D2 and D3. $S_\textrm{phy}$ is the area between a couple of current electrodes denoted by a numerical pair (see Fig. \ref{Fig2}\textbf{a} and Extended Data Fig.~5). The error bars indicate uncertainties in the determination of $S_\textrm{phy}$ and $S_{\phi}$. For comparison, $S_{\phi}$ vesus $S_\textrm{phy}$ enclosed by voltage terminals is presented in Extended Data Fig.~8. The dashed lines in \textbf{a}, \textbf{c} and \textbf{e-f} connect the minimum of the oscillation profiles within the $n=0$ segment. Their intersection, offset from y-axis, reveals the deviation from the standard model.  
	}
	\label{Fig3}
\end{figure*}

\noindent 
Given this speculation, it is reasonable to expect the presence of $I_\textrm{e}$, i.e., the supercurrent around chiral domain boundaries, and its response to magnetic flux~\cite{Ong2020Science}. Therefore, we measured $\textrm{d}V/\textrm{d}I$ versus $I$ at selective values of $B$ for D1 and plotted its color map in the $I$-$B$ plane, seen in Fig.~\ref{Fig3}a and b. In Fig.~\ref{Fig3}a, three sets of periodic oscillation profiles ($I_\textrm{c}$, $I'_\textrm{c}$ and $I''_\textrm{c}$) are resolved, traced by three transition peaks marked by dashed curves in Fig.~\ref{Fig3}b. Similar patterns with distinct periodicities for D2 are presented in Fig.~\ref{Fig3}c. See more data in Extended data Fig.~5-9.
%The inner profile represents the oscillation of the critical current ($I_\textrm{c}$), and the middle one corresponds to the large transition step (magenta line), denoted as $I'_\textrm{c}$. The outer one is less obvious, traced out by a small transition peak (orange line), designated as $I''_\textrm{c}$.

Such double-slit patterns vividly imitate the SC interference patterns (SIPs) from a SC Quantum Interference Device (SQUID) or the Little-Parks (LP) effect, in which the magnetic flux threading in hollow regions modulates the loop supercurrent~\cite{Barone1982book}. 
Following the LP scheme in our case (sketched in the left of Fig.~\ref{Fig3}d), the fluxoid quantization within an enclosed area leads to the variation of the superfluid velocity
$v_\textrm{e}=2\pi\hbar/(m^*L_c)(n-\phi/\phi_0)$, where $\hbar$ is reduced Planck's constant;
$m^*$ is the Ginzburg-Landau (GL) mass of Cooper pairs; $L_c$ is the circumference of the loop;
$n$ is the closest integer to $\phi/\phi_0$, {\it i.e.}, $n-\frac{1}{2}\leq \phi/\phi_0 \leq n+\frac{1}{2}$; $\phi_0=hc/(2e)$ is the flux quantum and $\phi=B S_{\phi}$ is the flux with $S_{\phi}$ the penetration area of $B$ enclosed by the loop. Since $B$ is far below the upper critical field, it subsequently leads to the modulation of the corresponding 
SC condensation wavefunction $\Psi_\textrm{e}$ along the edge
as $\Delta |\Psi_\textrm{e}|^2$$\sim(n-\phi/\phi_0)^2$. Then, the variation of the GL critical current is yielded as (See detailed discussion in Supplementary Information section 1):
%superconducting interference patterns (SIPs). 
%in which two set of oscillatory profiles with different periodicity ($\delta B$) appears with the inner one corresponding to $I_\textrm{c}$ and the outer one corresponding to the second transition-like step ( designated as $I'_\textrm{c}$) in Fig.~\ref{Fig1}d. 
\begin{equation}
	%	\begin{split}
		\frac{\Delta I^\textrm{SIP}_\textrm{c}}{I_\textrm{c}} \sim -\left(\frac{2\pi\xi}{L_\textrm{c}}\right)^2
		\left(n-\frac{\phi}{\phi_0}\right)^2   
		% \end{split}
	\label{Eq1}
\end{equation}
where $\xi$ is the GL coherence length. As depicted Fig.~\ref{Fig3}e, $I_\textrm{c}$ oscillates with a period of $\phi_0$, roughly mimicking the experimental results. 


According to the modulation period $\Delta B_\textrm{p}$ extracted from Fig.~\ref{Fig3}a, c and Extended Data Fig.~5-6, $S_{\phi}$ is calculated through $S_{\phi}=\phi_0/\Delta B_\textrm{p}$ and is compared with $S_\textrm{phy}$, the physical area enclosed by the current electrodes in Fig.~\ref{Fig3}g. $S_{\phi}$ is more than one order of magnitude smaller than $S_\textrm{phy}$ and no scaling relationships are observed between them. This finding suggests that $I_\textrm{e}$ flows along certain domain boundaries rather than the sample edge~\cite{Ong2020Science}. More supporting information is supplied in Extended Data Fig. 6-8.
Such behavior aligns with the expectation for chiral SC. In chiral SCs, it is energetically favorable to establish domain structures between degenerate SC phases 
of opposite chiralities~\cite{Sigrist1991RMP}. Therefore, our devices can be regarded as a network of domain walls forming composited loops guiding the flow for $I_\textrm{e}$. 
Since SC orders are relatively suppressed compared to the bulk
~\cite{Weitering2023NP}, the magnetic flux passes through the walls, penetrating the domain on the scale of the Pearl length $\Lambda_\textrm{p}=\lambda^2/d$, where $\lambda$ is the London length. Given $\lambda$(0~K)$\approx0.4~\mu$m from Ref.~\citenum{Khasanov2022CP}, one yields $\Lambda_\textrm{p}(0~\textrm{K})\approx4$~$\mu$m for D1 with the thickness $d\approx40$~nm, which is comparable to the size of the domains $r\sim \sqrt{S_\phi}$. This suggests that 
magnetic field penetrates into the entire domain. Circulating supercurrent appears in the bulk suppressing the SC order, but they do not contribute to transport supercurrent.
The transport supercurrent measured in this experiment is assumed to flow through domain wall edges such that the domain enclosed by the loop acts like a hollow (weak shielding of $B$) that serve as a basis for the LP effect, as seen in the right of Fig.~\ref{Fig3}d. Such loops may give rise to multiple SIPs, as shown in Fig.~\ref{Fig3}a.

%\textcolor{red}{CW: 1. can we provide the values of $\lambda$ and $d$ of samples studied in this experiment? 2. Please check the above added sentences.}

Note that the preceding discussion primarily revolves around chiral domain boundaries.
%the edge state, assuming a priori that the edge wave function ($\Psi_\textrm{e}$) can be considered independently from the bulk one ($\Psi_\textrm{b}$). 
In most cases, we indeed observe double-slit SIPs indicating the dominance of $I_\textrm{e}$. 
While, in a smaller subset of cases (Extended Data Fig.~5b),  the superposition of the double-slit SIP and the Fraunhofer-like SIP (generated by a Josephson junction between local domains) is observed, suggesting comparable contributions from both domain boundaries and bulk.
%from both the edge and bulk states. 
%\textcolor{blue}{Further details are discussed in Supplementary Section...}


~\\
\noindent\textbf{Broken time reversal and inversion symmetry}

\begin{figure*}[thb]
	\begin{center}
		\includegraphics[width=18cm]{Fig4.png}
	\end{center}
	\setlength{\abovecaptionskip}{-8 pt}
	\caption{\textbf{Temperature evolution of SIPs for D1.} \textbf{a}, SIPs on D1 measured at various $T$. Open circles, spotted at the nearest minimum of $I_\textrm{c}$ around 0 Gs, tracks the counter-shift of $I_\textrm{+}$ and $I_\textrm{-}$ branches.  A new SIP appears at $T>3$~K marked by arrows. \textbf{b}, Illustration of domain asymmetry. The critical current passing through the upper and lower branches are unequal: $I^\textrm{u}_\textrm{c}\neq I^\textrm{d}_\textrm{c}$. $L$ is the inductance. \textbf{c}, Numerical simulation (dashed curves) of the observed SIPs in D2, incorporating domain asymmetry. \textbf{d}, $\textrm{d}V/\textrm{d}I$ versus $I$ at $B=0$~Gs. A sudden peak emerges at $T>3$~K enclosed by circles, corresponding to the new SIP.  \textbf{e}, $T$-evolution of period ($\Delta B_\textrm{p}$). \textbf{f}, Relative phase counter-shift versus $T$. $\Delta \phi_\textrm{a}(I_\pm)$ is obtained by comparing $\phi_\textrm{a}(I_\pm)(T)$ with respect to the value at 3.5 K. The solid curves are fits to $\sqrt{1-T/T_\textrm{c}}$. \textbf{g}, $T$-dependence of normalized oscillation amplitude ($\Delta I^\textrm{SIP}_\textrm{c}/I_\textrm{c})$.  The solid curve is a fit to the GL theory. The error bars indicate uncertainties in  determination of extracted values.
	}
	\label{Fig4}
\end{figure*}

\noindent
%Proceedingly, 
Let us delve into some subtle information inferred from our results. Upon close inspection of Fig.~\ref{Fig3}a and c, we observe slight difference between the observed SIPs and the standard model presented in Fig.~\ref{Fig3}e. In the latter, $I_\text{c+}(B)=I_\text{c-}(-B)$ and $I_\text{c+}(B)=I_\text{c-}(B)$, reflecting the conservation of TRS and IR, respectively. While in the former, the SIPs exhibit a phase shift, as indicated by the intersection of two dashed lines offset from the y-axis in Fig.~\ref{Fig3}a and c. It means that the symmetric center of the I-$\phi$ curve is shifted from zero flux,
%It results in $I_\text{c+}(B)\neq I_\text{c-}(-B)$, 
unveiling TRS-breaking~\cite{WuCJ2022Arxiv}. These SIPs could be simulated by introducing additional phase term ($\phi_\textrm{T}/\phi_0$) to Eq.~\ref{Eq1}, as demonstrated in Fig.~\ref{Fig3}f. The phase shift is unlikely from the flux trapped in samples or the magnet (see Methods). A more plausible explanation is that the SC itself breaks time-reversal symmetry. 
%the existence of chiral $I_\textrm{e}$. 
It is worth noting that the phase shift is arbitrary rather than $\pi$, which happens to be different from that expected in p-wave SC~\cite{DuanXF2023arxiv}.
%\textcolor{blue}{See details in Supplementary Section.} 

%Intriguingly, in Fig.~\ref{Fig3}a, we observe that the $I_\textrm{c}$ and $I'_\textrm{c}$ curves shift in opposite direction, suggesting the presence of composite loops of opposite chirality. It echoes the reversal SDE polarity between the two transitions \textcolor{red}{as illustrated in Extended Data Fig.~2d.} 

Upon further scrutiny of Fig.~\ref{Fig3}a, c and Extended Data Fig.~6-7, it becomes evident that the periodic profile is asymmetric within each segment ($n=0,\pm1$,...). In addition, the $I_\pm$ branches within a SIP present a pronounced counter-shift in phase as $T$ evolves in Fig.~\ref{Fig4}a. These observations result in $I_\text{c+}(B)\neq I_\text{c-}(B)$, a manifestation of broken IR~\cite{WuCJ2022Arxiv}. We argue this is related to the asymmetry embedded in domain formation. To illustrate this, we consider a simple model, where the current loop is divided into two segments (up/down), assuming that the critical current ($I^\textrm{u/d}_\textrm{c}$) of each segment is unequal, seen in Fig.~\ref{Fig4}b. Additional flux is generated from supercurrent via
$\phi_\text{a}=LI^\textrm{u}_\textrm{c}-LI^\textrm{d}_\textrm{c}$, where $L$ is the effective inductance. Obviously, $\phi_\text{a}$ is odd with respect to the current direction. Adding $\phi_\text{a}$ to Eq.~\ref{Eq1} produces an asymmetry and counter-shift in SIPs as shown by dashed lines in Fig.~\ref{Fig4}c, which closely mimics the experimental results. Details are in Supplementary Information section 2. The combination of broken TRS and IR give rise to zero-field SDEs. 

Let us now gain more insights into the SIPs by analyzing their temperature evolution. First of all, a new SIP emerges as $T$ approaches $T_\textrm{c}$, marked by arrows in Fig. \ref{Fig4}a. It  coincides with a sudden peak (circled) observed in the $\textrm{d}V/\textrm{d}I$ curves in Fig. \ref{Fig4}d, indicating the formation of a new supercurrent loop. This highlights domain dynamics near $T_c$, as discussed in the SDE section. Second, the period $\Delta B_\textrm{p}$($T$) is nearly constant in Fig~\ref{Fig4}e, yielding $S_\phi$ insensitive to $\Lambda_\textrm{p}(T)$. Since as approaching $T_c$, $\Lambda_\textrm{p}(T)$ diverges and exceeds the length scale of domains, this fact implies that 
%it implies $\Lambda_\textrm{p}(0~K)$ exceeding the scale of domains and 
$S_\phi$ is primarily determined by the size of supercurrent loops. This finding further supports the speculation that the area enclosed by supercurrent loops acts like as a hollow. Third, referred from Supplementary Information section 2, the phase counter-shift scales as $\phi_\textrm{a}(I_\pm)\sim 1/\xi \sim \sqrt{1-T/T_\textrm{c}}$, in agreement  with the observed evolution of $\phi_\textrm{a}$ in Fig.~\ref{Fig4}f. Finally, combining Eq.~\ref{Eq1} and the GL theory, one deduce the oscillation amplitude $\Delta I^\textrm{SIP}_\textrm{c}/I_\textrm{c}\sim\xi^2\sim T_\textrm{c}/(T_\textrm{c}-T)$, which also fits our results well as depicted in Fig.~\ref{Fig4}g. %~\cite{TinkhamBook2004}

 

~\\
\noindent\textbf{Discussion}

\noindent
Here, we have achieved the remarkable conclusion that CsV$_3$Sb$_5$ is a chiral SC exhibiting TRS-breaking. This finding is in line with the expectation from the observation of 3Q pair density wave (PDW) order unveiled by STM~\cite{GaoHJ2021Nature}. This phase is proposed to naturally break TRS~\cite{WangZQ2022NC,ZhouY2022PRL}, resulting in a chiral order parameter: $\Delta(\textbf{r})=\Delta_0 e^{i\theta} \sum_{\alpha=1,2,3} e^{\eta(\alpha-2)\frac{2\pi}{3}i} \textrm{cos}(\textbf{Q}_{\bf\alpha}\cdot\textbf{r} +\varphi_\alpha)$, where $\bf Q_{\bf \alpha}$ is PDW wave vector, $\theta$: the global phase, $\varphi_\alpha$: the relative phase between three $\bf Q_{\bf \alpha}$ modes and $\eta=\pm1$ is the chirality, responsible for chiral edge modes. Though the microscopic pairing interaction is unclear yet, $\Delta_0$ is most probably a full gap, indicated by the majority of experimental observations~\cite{LuoJL2021CPL,Yuan2021SCPMA,Shibauchi2023NC,Okazaki2023Nature,FengDL2021PRL}. See details in Supplementary Information section 3. Theoretical studies~\cite{WangZQ2022NC,ZhouY2022PRL} consider the chiral PDW phase as a primary order, and its partial melting leads to the plethora of exotic electronic phases, involving the 3Q chiral charge order, a loop-current pseudogap phase, and vestigial charge-$4e$ and charge-$6e$ SC~\cite{WangJ2022Arxiv}. Therefore, our findings will inspire future experimental and theoretical efforts to investigate the intricate interplay of multiple intertwined orders within the complex SC phase diagram of Kagome SCs and to explore unprecedented electronic phases in this fascinating class of materials. Furthermore, the method employed in this study offers a simple, but effective tool for the detection of edge supercurrent, which may contribute to the exploration of novel chiral SCs.


%Our findings provide valuable insights into the nature of SC in kagome superconductors. It will certainly stimulate future experimental and theoretical studies on the intricate interplay between unconventional chiral SC and multiple intertwined electronic orders within the complex SC phase diagram of this fascinating class of materials. 




%\section{Results and discussion}
%\noindent\textbf{Results}



~\\
\noindent\textbf{Online content}

\noindent Supplementary information are available at the online version of the paper. 


~\\
%\noindent \textbf{References}
\bibliographystyle{naturemag}
%\bibliographystyle{apsrev4-1}
%\bibliographystyle{apacite}
\bibliography{SDEKagome}





\clearpage
\noindent\textbf{Methods}

~\\
\noindent
\textbf{Growth of single crystals}

\noindent
Single crystals of CsV$_3$Sb$_5$ were grown through flux methods by using Cs (purity 99.8\%) bulk, V (purity 99.999\%) pieces and Sb (purity 99.9999\%) shot as the precursors and Cs$_{0.4}$Sb$_{0.6}$ as the flux agent. The starting elements were placed in an alumina crucible and sealed in a quartz ampoule in an argon-filled glove box.  The ampoule was then gradually heated up to 1000~\textcelsius ~ in 200 h and held at that temperature for 24 h in an oven. It was subsequently cooled down to 200 \textcelsius~at a rate of 3.5 \textcelsius/h. The resulting product was immersed in deionized water to remove the flux. Finally, shiny CsV$_3$Sb$_5$ crystals with hexagonal shape were obtained.


~\\
\textbf{Fabrication of devices}

\noindent
CsV$_3$Sb$_5$ nanoflakes were mechanically exfoliated from the bulk crystals using Nitto blue tape and transferred onto silicon substrates (5 mm $\times$ 5 mm) capped with 300 nm SiO$_2$. The flakes were initially spin-coated with polymethyl methacrylate (PMMA). Contacts were patterned by utilizing standard electron beam lithography (EBL) techniques (TESCAN VEGA LMS). After EBL patterning, the PMMA was immersed in a solution of methyl isobutyl ketone (MIBK)-isopropyl alcohol (IPA)(1:3) for 60 seconds, followed by rinsing with IPA. To improve the contact, the flakes were cleaned by Ar plasma, prior to deposition. Finally, the contacts with a width of 500 nm-1 $\mu$m were deposited with Ti (5 nm)/Au (80 nm) via electron beam evaporation.



~\\
\textbf{Resistance measurements}

\noindent
The temperature dependence of resistance ($R$-$T$) for bulk and D1 from 1.8~K to 300~K was measured by standard four-terminal methods in Quantum Design physical property measurement system (PPMS-9T). Other $R$-$T$ curves measured around $T_\textrm{c}$ were collected in an Oxford dilution refrigerator (Triton-500).


~\\
\textbf{Measurements of superconducting diode effect}

\noindent
The zero-field superconducting diode effect was measured in Triton-500 equipped with 14 T superconducting magnet. Each measuring channel was connected with a filter (QFilter-II,Qdevil), positioned at the plate of mixing chamber and to a sample-protected  measurement box, located at room temperature.  Prior to measurements, the flux trapped in the superconducting magnet was completely released by warming the cryostat up to room temperature.  The differential resistance (d$V$/d$I$) in the $I_\textrm{+}$ and $I_\textrm{-}$ regions was measured by sweeping the d.c. current bias ($I$) from 0 to $I_\textrm{max+}$ or $I_\textrm{max-}$. The current bias was supplied by a current source meter (Keithley 2450). A lock-in amplifier (Stanford Research, SR830) combined with a 1 or 10 M$\Omega$ buffer resistor was used to offer a small ac excitation current $I_{\textrm{ac}}$ ($11-173$~Hz, $0.1~\mu$A $-5~\mu$A) to detect the differential resistance (d$V$/d$I$ = $V_\textrm{ac}$/$I_\textrm{ac}$). The measurements were performed by standard four-terminal methods. 

For the thermal cycling measurements, two methods were employed. The first method involved heating the mixing chamber slightly above $T_\textrm{c}$ ($\sim4.5$~K) and subsequently re-cooling it to the target temperature. This process is slow, usually taking more than one hour.  The second method utilized a 2 k$\Omega$ resistor, which was adhered on the SiO$_2$/Si substrate by silver paste as a heater. The local temperature of the device was then modulated via triggering the heater with pulsed (several seconds) milliampere current applied by a current source (Keithley 6221). This process was considerably faster, taking only a few minutes. 

~\\
\noindent\textbf{Measurements of interference patterns}

\noindent
The superconducting interference patterns (SIPs) were obtained by measuring d$V$/d$I$-$I$ curves at various fixed magnetic field ($B$) with an interval of 0.1 Gs or 0.2 Gs per trace. A current source meter (Keithley 2440) served as the power supply for the superconducting magnet in order to achieve precise control of $B$ at sub-Gs levels. $B$ is applied along c-axis, normal to the large plane of the flakes. During the measurements, the SIPs were initially detected at 1.4~K within a narrow field range (e.g. -6 Gs to 6 Gs) after zero-field SDE measurements to prevent any influence from trapped vortices in samples or in the magnet. Subsequently, a higher $B$ was applied to measure the broad pattern. We observed that magnetic vortices enter into the flakes and smear the SIPs when $B$ exceeded 10 Gs, as depicted in Extended Data Fig.~5. The temperature evolution of the SIPs was detected finally. Note that 1 mA current corresponds to $B$ of 1.2 Gs. The measurements were performed by standard four-terminal methods. 

~\\
\noindent\textbf{Data availability}

\noindent Data are available from the corresponding author upon reasonable request.


~\\
\small
\noindent\textbf{Acknowledgments}
The authors are grateful to Lin Jiao and Chunyu Guo for the helpful discussion.  
This research is supported by Zhejiang Provincial Natural Science Foundation of China for Distinguished Young Scholars under Grant No. LR23A040001. C.W. is supported by the National Natural Science Foundation of China under the Grants No. 12234016 and No. 12174317. T.L. acknowledges support from the China Postdoctoral Science Foundation (Grant No. 2022M722845). This work has been supported by the New Cornerstone Science Foundation. The authors thank the support provided by Dr. Chao Zhang from Instrumentation and Service Center for Physical Sciences at Westlake University.


~\\
\noindent\textbf{Author contributions}
T.L. fabricated the devices and did the transport measurements assisted by Z.X., J.W. and Z.L.. J.L. prepared the samples supervised by Z.W. and Y.Y.. Z.P. did theoretical calculations supervised by C.W.. T.L. and X.L. prepared the figures. C.W. and X.L. wrote the paper. X.L. led the project.  All authors contributed to the discussion.


~\\
\noindent\textbf{Competing interests}
The authors declare no competing interests.

~\\
%\noindent\textbf{Supporting information}
\noindent\textbf{Additional information}

\noindent\textbf{Supplementary information} are available at the online version of the paper. 

\noindent Correspondence and requests for materials should be addressed to Zhiwei Wang, Congjun Wu or Xiao Lin.




%\clearpage

\begin{appendix}
%\renewcommand\thefigure{S\arabic{figure}}
\setcounter{figure}{0}
%\renewcommand\thetable{S\arabic{table}}
%\setcounter{table}{0}



\begin{figure*}[thb]
	%\resizebox{!}{0.38\textwidth}
	\renewcommand{\figurename}{Extended Data Fig.}
	\includegraphics[width=10cm]{EDF1-RT.png}
	\caption{\textbf{Temperature dependence of resistance for D2 and D3 around $T_\textrm{c}$.} For D2, we present the data collected at two sets of terminals: $V_\textrm{3-4}$ and $V_\textrm{5-6}$. The onset temperature of the superconducting transition ($T_\textrm{c}^\textrm{onset}$) for D2 and D3 is similar, about 4.3~K. $T_\textrm{c}$ determined at zero-resistance for D3 amounts to 4.1~K, which is higher than that of D2 (about 3.5~K). Note that $T_\textrm{c}$ of D2 measured at $V_\textrm{3-4}$ and $V_\textrm{5-6}$ is slightly different, reflecting different domain characteristics in-between the terminals. The thickness ($d$) of D2 and D3 is about 40 nm.}
	\label{FigSRT}
\end{figure*}


\begin{figure*}[thb]
	%\resizebox{!}{0.38\textwidth}
	\renewcommand{\figurename}{Extended Data Fig.}
	\includegraphics[width=14cm]{EDF2-Thsw.png}
    \caption{\textbf{Thermal switching of SDE polarity for D1 and D3 at $B=0$~Gs.} \textbf{a-b}, $\textrm{d}V/\textrm{d}I$ versus $I$ for D1 before (\textbf{a}) and after thermal cycling (\textbf{b}). Several observations are made below: 1. The curves show multiple transition-like features,  probably related to the difference in $I_\textrm{c}$ across different superconducting domain boundaries (see Supplementary Information section 2). 2. Nonreciprocity is observed at all the transition anomalies (marked by arrows) at zero field. 3. When the SDE polarity at A1 is reversed by thermal cycling, the polarity at A2 remains unchanged. 4. The magnitudes of $\Delta I_\textrm{c}^\textrm{SDE}$ and $\bar{I}_\textrm{c}$ are altered after thermal cycling. All these point to the existence of superconducting domains with broken TRS. Characteristics of the domains, such as domain asymmetry and inter-domain interaction, are randomly altered by thermal cycling (i.e. recooling the system from above $T_\textrm{c}$).  \textbf{c-d}, $\textrm{d}V/\textrm{d}I$ versus $I$ for D3 before (\textbf{c}) and after thermal cycling (\textbf{d}). In \textbf{c}, the measurement includes four branches: sweeping $I$ from zero to positive ($I_\textrm{+}$), from positive back to zero ($I_\textrm{r+}$), from zero to negative ($I_\textrm{-}$) and from negative back to zero ($I_\textrm{r-}$). The hysteresis between  $I_\textrm{+}$ ($I_\textrm{-}$) and $I_\textrm{r+}$ ($I_\textrm{r-}$) is negligible, indicating the absence of thermal heating or current re-trapping effect.
   }	
	\label{FigSSDET}
\end{figure*}

\begin{figure*}[thb]
	%\resizebox{!}{0.38\textwidth}
	\renewcommand{\figurename}{Extended Data Fig.}
	\includegraphics[width=14cm]{EDF3-HWR.png}
	\caption{\textbf{Demonstration of half-wave rectification.} Direction-selective supercurrent transmission is demonstrated at $V_{6-7}$ of D2 with positive (upper panel) and negative polarity (lower panel). The measurements were performed by alternating the current polarity every 15.5 seconds. SDE remains stable after 100 cycles.
    }	
		\label{FigSHWR}
	\end{figure*}


\begin{figure*}[thb]
	%\resizebox{!}{0.38\textwidth}
	\renewcommand{\figurename}{Extended Data Fig.}
	\includegraphics[width=10cm]{EDF4-IV.png}
	\caption{\textbf{Differential resistance (d$V$/d$I$) versus $I$ for D2 measured at different terminals}. Characteristics of d$V$/d$I$ exhibit notable distinctions across different terminals. Only $V_{3-4}$ displays an apparent SDE. Note that the SDE at $V_{6-7}$ can be excited by thermal cycling, as shown in Fig.~2. $I_\textrm{c}$ varies with different terminals. All these point to the formation of domain structure in the superconducting phase of CsV$_3$Sb$_5$. $I_\textrm{c}$ is influenced by the strength of inter-domain connections. See Supplementary Information Section 2.
	}	
	\label{FigSHWR}
\end{figure*}


\begin{figure*}[thb]
	%\resizebox{!}{0.38\textwidth}
	\renewcommand{\figurename}{Extended Data Fig.}
	\includegraphics[width=19cm]{EDF5-SIPD123.png}
     \caption{\textbf{SIPs for D1, D2 and D3 in a broader range of $B$.} \textbf{a and b}, SIPs for D1, covering the $B$ range of 20~Gs and 240~Gs, respectively, as the extended data of Fig.~3a. In \textbf{a}, three SIPs ($I_\textrm{c}$, $I'_\textrm{c}$ and $I''_\textrm{c}$) are clearly resolved, corresponding to those in Fig.~3a. \textbf{b} displays more complex, periodic-like structures, alongside $I_\textrm{c}$, $I'_\textrm{c}$ and $I''_\textrm{c}$. Notably, we observe periodic oscillations on $I_\textrm{c}$. The magnitude of $I_\textrm{c}$ remains nearly unchanged in $B$ up to 240~Gs, as expected from the LP effect. In contrast, $I'_\textrm{c}$ displays a broad Fraunhofer-like pattern, on top of which is a rapid double-slit periodic oscillation. The broad feature is likely associated with a local Josephson junction and the rapid mode arises from the LP effect, implying the comparable contribution from the bulk and domain edge. \textbf{c and d}, SIPs for D2, covering the $B$ range of 40~Gs and 240~Gs, respectively, as an extension of the data in Fig.~3b. An explicit periodic oscillation pattern appears on $I_\textrm{c}$, along with some vague oscillation features. At $B>10$~Gs, distinct spikes (marked by white arrows) emerge on $I_\textrm{c}$, disrupting the oscillation patterns and leading to  irregular oscillation periods, which is the result of magnetic vortices penetrating into the flakes. \textbf{e and f}, SIPs for D3, covering the $B$ range of 40~Gs and 240~Gs, respectively. The oscillation pattern on $I_\textrm{c}$ is more obvious than others. The inset of \textbf{e} is the optical image of D3. All the terminals are numbered (1-14).
}
	\label{FigSSIPD123}
\end{figure*}


\begin{figure*}[thb]
	%\resizebox{!}{0.38\textwidth}
	\renewcommand{\figurename}{Extended Data Fig.}
	\includegraphics[width=18cm]{EDF6-SIPvI.png}
	\caption{\textbf{SIPs for D2 and D3 with the current bias applied between different terminals.} \textbf{a}, SIPs for D2 measured at $V_\textrm{6-7}$ with current injected into $I_\textrm{5-8}$, which is compared with the data in Fig.~3c and Extended Data Fig.~5c (D2, $V_\textrm{6-7}$ and $I_\textrm{4-8}$). \textbf{b-f}, SIPs for D3 measured at $V_\textrm{3-6}$ with different current terminals. The oscillation patterns on $I_\textrm{c}$ is nearly unchanged when current terminals are varied, indicating that the SIPs are most likely associated with the domain structures between the voltage electrodes. 
	}	
	\label{FigSSIPvI}
\end{figure*}


\begin{figure*}[thb]
	%\resizebox{!}{0.38\textwidth}
	\renewcommand{\figurename}{Extended Data Fig.}
	\includegraphics[width=18cm]{EDF7-SIPvV.png}
\caption{\textbf{SIPs for D3 collected at different voltage terminals.} \textbf{a-f}, SIPs measured by varying the voltage terminals while applying the current bias on $I_\textrm{1-14}$. Explicit periodic oscillation patterns are observed in $V_\textrm{3-6}$ (\textbf{a}) and $V_\textrm{3-4}$ (\textbf{b}). However, the oscillation patterns are vague in $V_\textrm{4-6}$ (\textbf{c}), $V_\textrm{2-3}$ (\textbf{d}), $V_\textrm{6-7}$ (\textbf{e}) and $V_\textrm{7-8}$ (\textbf{f}). In close inspection of \textbf{a}, \textbf{b} and \textbf{c}, we find that the patterns in $V_\textrm{3-6}$ appear to be the superposition of $V_\textrm{3-4}$ and $V_\textrm{4-6}$. And the dominant contribution to the SIP ($I_\textrm{c}$) comes from $V_\textrm{3-4}$. These observations strongly suggest that the SIP arises from proper domain structures between terminals 3 and 4. 
}	
	\label{FigSSIPvV}
\end{figure*}



\begin{figure*}[thb]
	%\resizebox{!}{0.38\textwidth}
	\renewcommand{\figurename}{Extended Data Fig.}
	\includegraphics[width=10cm]{EDF8-SvV.png}
	\caption{\textbf{$S_{\phi}$ vesus $S_\textrm{phy}$ enclosed by voltage terminals}. The error bars indicate uncertainties in the determination of $S_\textrm{phy}$ and $S_{\phi}$. No obvious relationships are observed between them.  
	}	
	\label{FigSHWR}
\end{figure*}


\begin{figure*}[thb]
	%\resizebox{!}{0.38\textwidth}
	\renewcommand{\figurename}{Extended Data Fig.}
	\includegraphics[width=18cm]{EDF9-ThC.png}
	\caption{\textbf{Thermal modulation of SIPs for D2}. \textbf{a}, SIPs in the initial state. \textbf{b}, SIPs measured after thermal cycling. We observe that the SIPs show counter-shift in phase between the $I_\textrm{+}$ and $I_\textrm{-}$ branches, indicating the modification of domain asymmetry by thermal cycling.
	}	
	\label{FigSHWR}
\end{figure*}

~\\

\end{appendix}



\end{document}

%\begin{appendix}
%\renewcommand\thefigure{S\arabic{figure}}
%\setcounter{figure}{0}
%\renewcommand\thetable{S\arabic{table}}
%\setcounter{table}{0}
%\end{appendix}

%\section{Materials and method}
%\vspace{-10 pt}
%\noindent 


