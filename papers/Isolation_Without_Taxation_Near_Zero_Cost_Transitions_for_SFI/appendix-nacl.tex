\section{NaCl}
\label{appendix:nacl}

\begin{center}
  \begin{tabular}{>{$} c <{$} | c}
    \multicolumn{2}{c}{NaCl context in application} \\
    \hline
    ctx - 0 & library stack pointer \\
    \ldots & \\
    ctx^{\ast} & $ctx$ \\

    \multicolumn{2}{c}{} \\

    \multicolumn{2}{c}{NaCl context in library} \\
    \hline
    ctx - 0 & application stack pointer \\
    ctx - 1 & $\mathbb{CSR}_0$ \\
    ctx - 2 & $\mathbb{CSR}_1$ \\
    \ldots & \ldots \\
    ctx - \operatorname{len}(\mathbb{CSR}) & $\mathbb{CSR}_{\operatorname{len}(\mathbb{CSR}) - 1}$ \\
    \ldots \\
    ctx^{\ast} & $ctx$ \\
  \end{tabular}
  \captionof{figure}{Transition Context Layout}
  \label{fig:appendix:nacl:context-layout}
\end{center}

\begin{center}
  \begin{small}
  \begin{tabular}{>{$} r <{$} | >{$} l <{$} @{\hspace{\tabcolsep}} l}
    \multicolumn{2}{l}{$\operatorname{nacl-springboard}(n, e) \triangleq$} \\
    \hline
    & \cload{r_0}{}{ctx^{\ast}}
    \\
    & \cload{r_1}{}{r_0}
    & // $r_1$ holds the library stack pointer
    \\
    j \in (\operatorname{len}(\mathbb{CSR}), 0]
    & \overline{
      \cstore{}{r_0}{\mathbb{CSR}_j};
      \cmov{r_0}{r_0 + 1}
    }
    & // save callee-save registers
    \\
    & \cmov{r_1}{r_1 + n}
    & // set $r_1$ to the new top of the library stack
    \\
    & \cmov{sp}{sp - 1}
    & // move the stack pointer to the first argument
    \\
    j \in [0, n)
    & \overline{
      \cpop{r_2}{};
      \cstore{M_{\untrusted}}{r_1}{r_2};
      \cmov{r_1}{r_1 - 1}
    }
    & // copy arguments to library stack
    \\
    & \cmov{r_2}{sp + (n + 1)}; \cstore{}{r_0}{r_2}
    & // save stack pointer
    \\
    & \cmov{sp}{r_1 + n}
    & // set new stack pointer
    \\
    & \cstore{}{ctx^{\ast}}{r_0}
    & // update $ctx$
    \\
    r \in \mathbb{R}
    & \overline{\cmov{r}{0}}
    & // clear registers
    \\
    & \cjmp{}{e}
  \end{tabular}
  \end{small}
\end{center}

\begin{center}
  \begin{small}
  \begin{tabular}{>{$} r <{$} | >{$} l <{$} l}
    \multicolumn{2}{l}{$\operatorname{nacl-trampoline} \triangleq$} \\
    \hline
    & \cload{r_0}{}{ctx^{\ast}}
    \\
    j \in [0, \operatorname{len}(\mathbb{CSR}))
    & \overline{
      \cmov{r_0}{r_0 - 1};
      \cload{\mathbb{CSR}_j}{}{r_0}
    }
    & // restore callee-save registers
    \\
    & \cload{r_0}{}{ctx^{\ast}}
    \\
    & \cload{r_1}{}{r_0}
    & // $r_1$ holds the application stack pointer
    \\
    & \cmov{r_0}{r_0 - \operatorname{len}(\mathbb{CSR})}; \cstore{}{r_0}{sp}
    & // save library stack pointer
    \\
    & \cstore{}{ctx^{\ast}}{r_0}
    & // update $ctx$
    \\
    & \cmov{sp}{r_1}
    & // switch to application stack
    \\
    & \cret{}
  \end{tabular}
  \end{small}
\end{center}

\begin{center}
  \begin{small}
  \begin{tabular}{>{$} r <{$} | >{$} l <{$} l}
    \multicolumn{2}{l}{$\operatorname{nacl-cb-springboard}(n, e) \triangleq$} \\
    \hline
    & \cload{r_0}{}{ctx^{\ast}}
    \\
    & \cload{r_1}{}{r_0}
    & // $r_1$ holds the application stack pointer
    \\
    & \cmov{r_0}{r_0 + 1}; \cstore{}{r_0}{sp}
    & // save stack pointer
    \\
    & \cstore{}{ctx^{\ast}}{r_0}
    & // update $ctx$
    \\
    & \cmov{sp}{sp - 1}
    & // move the stack pointer to the first argument
    \\
    & \cmov{r_1}{r_1 + n}
    & // set $r_1$ to the new top of the library stack
    \\
    j \in [0, n)
    & \overline{
      \cpop{r_2}{M_{\untrusted}};
      \cstore{}{r_1}{r_2};
      \cmov{r_1}{r_1 - 1}
      }
    & // copy arguments to application stack
    \\
    & \cmov{sp}{r_1 + n}
    & // set new stack pointer
    \\
    & \cjmp{I}{e}
  \end{tabular}
  \end{small}
\end{center}

\begin{center}
  \begin{small}
  \begin{tabular}{>{$} r <{$} | >{$} l <{$} l}
    \multicolumn{2}{l}{$\operatorname{nacl-cb-trampoline} \triangleq$} \\
    \hline
    & \cload{r_0}{}{ctx^{\ast}}
    \\
    & \cload{r_1}{}{r_0}
    & // $r_1$ holds the library stack pointer
    \\
    & \cmov{r_0}{r_0 - 1}; \cstore{}{ctx^{\ast}}{r_0}
    & // update $ctx$
    \\
    & \cmov{sp}{r_1}
    & // switch to library stack
    \\
    r \in \mathbb{R}
    & \overline{\cmov{r}{0}}
    & // clear registers
    \\
    & \cret{C_{\untrusted}}
  \end{tabular}
  \end{small}
\end{center}

\subsection{Programs}

A NaCl program $\Psi$ is defined by the following conditions:

\begin{enumerate}
\item
  All memory operations in the sandboxed library are guarded:
  \begin{align*}
    \forall n.
    & \Psi.C(n) = (\untrusted, \cpop{r}{p}) \Longrightarrow p = \untrusted \\
    & \Psi.C(n) = (\untrusted, \cpush{p}{e}) \Longrightarrow p = \untrusted \\
    & \Psi.C(n) = (\untrusted, \cload{r}{k}{e}) \Longrightarrow \cod{k} \subseteq M_{\untrusted} \\
    & \Psi.C(n) = (\untrusted, \cstore{k}{e}{e'}) \Longrightarrow \cod{k} \subseteq M_{\untrusted}.
  \end{align*}

\item
  The application does not write $\trusted$ data to the sandbox memory.

\item
  Gated calls are the only way to move between application and library code:
  \begin{align*}
    \forall n.
    & \Psi.C(n) = (p, \ccall{k}{e}) \Longrightarrow \cod{k} \subseteq C_{p} \\
    & \Psi.C(n) = (p, \cret{k}) \Longrightarrow \cod{k} \subseteq C_{p} \\
    & \Psi.C(n) = (p, \cjmp{k}{e}) \Longrightarrow \cod{k} \subseteq C_{p} \\
  \end{align*}

\item
  The program starts in the application: $\Psi.pc = 0$ and $\Psi.C(0) = (\trusted, \_)$.

\item
  $ctx^{\ast}$ and $ctx$ start initialized to the library stack:
  \begin{align*}
    & ctx^{\ast} \in H_{\trusted} \\
    & ctx = \Psi.M(ctx^{\ast}) \in H_{\trusted} \\
    & \Psi.M(ctx) = S_{\untrusted}[0] - 1.
  \end{align*}

\item
  When calling into the library NaCl considers all registers except the stack
  pointer and program counter confidential.
  %
  It further labels sandbox memory and the $n'$ arguments in the application
  stack as public with the remainder of application memory labeled as
  confidential.

  \begin{align*}
    \mathbb{C}(\Psi)(r) &=
      \begin{cases}
        \untrusted & \text{when } r = sp \\
        \untrusted & \text{when } r = pc \\
        \trusted & \text{otherwise}
      \end{cases} \\
    \mathbb{C}(\Psi)(n) &=
      \begin{cases}
        \untrusted & \text{when } n \in M_{\untrusted} \\
        \untrusted & \text{when } n \in (\Psi.sp - n', \Psi.sp] \text{ where } \currentcom{\Psi}{\cgatecall{\mathit{n'}}{e}}{\trusted} \\
        \trusted & \text{when } n \in M_{\trusted} \wedge n \notin (\Psi.sp - n', \Psi.sp] \text{ where } \currentcom{\Psi}{\cgatecall{\mathit{n'}}{e}}{\trusted}
      \end{cases}
  \end{align*}
\end{enumerate}

\subsection{Properties}

Throughout the following we will use the shorthand $ctx_{\Psi} \triangleq \Psi.M(ctx^{\ast})$.

\begin{proposition}
  NaCl has the \strongni{} property.
\end{proposition}
\begin{proof}
  Follows immediately from the fact that all reads and writes are guarded to be within $M_{\untrusted}$, all values in $M_{\untrusted}$ have label $\untrusted$, and all jumps remain within the library code.
\end{proof}

\begin{lemma} \label{lemma:appendix:nacl:context-in-trusted}
  The trampoline context is in $H_{\trusted}$.
\end{lemma}

\begin{lemma} \label{lemma:appendix:nacl:ctx-positive}
  $ctx \geq ctx_0$.
\end{lemma}

\begin{lemma} \label{lemma:appendix:nacl:lib-p-steps-preserve-app-memory}
  If $\currentcom{\Psi_1}{c}{\untrusted}$ and $\Psi_1 \steppstar{p} \Psi_2$, then $\Psi''.M_{\trusted} = \Psi'.M_{\trusted}$.
\end{lemma}
\begin{proof}
  There are two cases for $p$: $p = \trusted$ and $p = \untrusted$.
  %
  If $p = \trusted$, then $\Psi_2 = \Psi_1$ and therefore trivially $\Psi_2.M_{\trusted} = \Psi_1.M_{\trusted}$.
  %
  If $p = \untrusted$, then for all $\Psi$ such that $\Psi_1 \stepstar \Psi \stepn{+} \Psi_2$, $\Psi.C(\Psi.pc) = (\untrusted, \_)$.
  %
  By the structure of a NaCl program, this ensures that $\Psi_2.M_{\trusted} = \Psi_1.M_{\trusted}$.
\end{proof}

\begin{lemma} \label{lemma:appendix:nacl:pstep-nogate}
  If $\Psi \steppstar{p} \Psi'$ then $\Psi.p = \Psi'.p$.
\end{lemma}

\begin{lemma} \label{lemma:appendix:nacl:wb-flips-priv}
  If $\Psi \stepwb \Psi'$ where $\Psi \stepstar \Psi'' \step \Psi'$, then $\Psi''.p \neq \Psi.p$ and $\Psi'.p = \Psi.p$.
\end{lemma}
\begin{proof}
  We proceed by simultaneous induction on the well-bracketed transition $\Psi \stepwb \Psi'$ and the length of $\Psi_0 \stepstar \Psi \stepwb \Psi'$.

  \begin{itemize}
  \pfcase{No callbacks}

      We have that $\currentop{\Psi}{\cgatecall{n}{e}}$ and there exist $\Psi_1$, $\Psi_2$, and $p$ such that $\Psi \step \Psi_1 \steppstar{p} \Psi_2 \step \Psi'$ where $\currentop{\Psi_2}{\cgateret}$.
      %
      Here $\Psi'' = \Psi_2$.
      %
      By inspection of the reduction for $\cgatecall{n}{e}$ we know that $\Psi_1.p \neq \Psi.p$ and therefore by \lemref{lemma:appendix:nacl:pstep-nogate} $\Psi''.p \neq \Psi.p$.
      %
      By inspection of the reduction for $\cgateret$, $\Psi'.p = \Psi.p$.

  \pfcase{Callbacks}

      We have that $\currentop{\Psi}{\cgatecall{n}{e}}$ and there exist $\Psi_1$, $\Psi_2$  such that $\Psi \step \Psi_1 \stepboxstar \Psi_2 \step \Psi'$ where $\currentop{\Psi_2}{\cgateret}$.
      %
      Here $\Psi'' = \Psi_2$.
      %
      By inspection of the reduction for $\cgatecall{n}{e}$ we know that $\Psi_1.p \neq \Psi.p$.
      %
      We now show, by induction on $\Psi_1 \stepboxstar \Psi_2$, that $\Psi_2.p = \Psi_1.p \neq \Psi.p$.

      \begin{subproof}
        If there are no steps then clearly $\Psi_2.p = \Psi_1.p \neq \Psi.p$.
        %
        There are two possible cases for $\Psi_1 \stepboxstar \Psi_3 \stepbox \Psi_4$.
        %
        When $\Psi_3 \stepp{p} \Psi_4$, \lemref{lemma:appendix:nacl:pstep-nogate} gives us that $\Psi_4.p = \Psi_3.p = \Psi_1.p \neq \Psi.p$.
        %
        When $\Psi_3 \stepwb \Psi_4$, our outer inductive hypothesis gives us that $\Psi_4.p = \Psi_3.p = \Psi_1.p \neq \Psi.p$.
      \end{subproof}

      Lastly, by inspection of the reduction for $\cgateret$, $\Psi'.p = \Psi.p$.
  \end{itemize}
\end{proof}

\begin{lemma} \label{lemma:appendix:nacl:stepbox-preserves-p}
  If $\Psi \stepboxstar \Psi'$, then $\Psi'.p = \Psi.p$.
\end{lemma}
\begin{proof}
  By induction, \lemref{lemma:appendix:nacl:pstep-nogate}, and \lemref{lemma:appendix:nacl:wb-flips-priv}.
\end{proof}

\begin{lemma}[Context Integrity] \label{lemma:appendix:nacl:context-integrity}
  Let $\Psi_0 \in \programs$, $\Psi_0 \stepstar \Psi$, and $\Psi \stepwb \Psi'$.
  %
  Then
  \begin{enumerate}
  \item if $\Psi.p = \trusted$, then $\Psi.M([ctx_{\Psi_0}, ctx_{\Psi})) = \Psi'.M([ctx_{\Psi_0}, ctx_{\Psi'}))$ and $ctx_{\Psi} = ctx_{\Psi'}$,
  \item if $\Psi.p = \untrusted$, then $\Psi.M([ctx_{\Psi_0}, ctx_{\Psi}]) = \Psi'.M([ctx_{\Psi_0}, ctx_{\Psi'}])$ and $ctx_{\Psi} = ctx_{\Psi'}$.
  \end{enumerate}
\end{lemma}
\begin{proof}
  We proceed by mutual, simultaneous induction on the well-bracketed transition $\Psi_1 \stepwb \Psi_2$ and the length of $\Psi_0 \stepstar \Psi \stepwb \Psi'$.

  First we consider the case where $\currentcom{\Psi}{c}{\trusted}$.
  \begin{itemize}
  \pfcase{No callbacks}

    We have that $\currentop{\Psi}{\cgatecall{n}{e}}$ and there exist $\Psi_1$, $\Psi_2$, and $p$ such that $\Psi \step \Psi_1 \steppstar{p} \Psi_2 \step \Psi'$ where $\currentop{\Psi_2}{\cgateret}$.
    %
    By \lemref{lemma:appendix:nacl:lib-p-steps-preserve-app-memory} we have that $\Psi_1.M_{\trusted} = \Psi_2.M_{\trusted}$.
    %
    By assumption we have that $\currentcom{\Psi}{\cgatecall{n}{e}}{\trusted}$ and therefore by \lemref{lemma:appendix:nacl:wb-flips-priv} we have that $\currentcom{\Psi_2}{\cgateret}{\untrusted}$.
    %
    By \lemref{lemma:appendix:nacl:context-in-trusted}, \lemref{lemma:appendix:nacl:ctx-positive}, and inspection of the reduction rules for $\cgatecall{n}{e}$ and $\cgateret$ we have that $\Psi.M([ctx_{\Psi_0}, ctx_{\Psi})) = \Psi'.M([ctx_{\Psi_0}, ctx_{\Psi'}))$ and $ctx_{\Psi} = ctx_{\Psi'}$.

  \pfcase{Callbacks}

    We have that $\currentop{\Psi}{\cgatecall{n}{e}}$ and there exist $\Psi_1$, $\Psi_2$  such that $\Psi \step \Psi_1 \stepboxstar \Psi_2 \step \Psi'$ where $\currentop{\Psi_2}{\cgateret}$.
    %
    By inspection of the reduction rule for $\cgatecall{n}{e}$ we have that $ctx_{\Psi_1} = ctx_{\Psi} + \operatorname{len}(\mathbb{CSR})$.
    %
    We now show, by induction on $\Psi_1 \stepboxstar \Psi_2$, that $ctx_{\Psi_2} = ctx_{\Psi_1}$ and $\Psi_1.M([ctx_{\Psi_0}, ctx_{\Psi_1}]) = \Psi_2.M([ctx_{\Psi_0}, ctx_{\Psi_2}])$.

    \begin{subproof}
      If there are no steps then clearly $ctx_{\Psi_2} = ctx_{\Psi_1}$ and all of $\Psi_1.M_{\trusted} = \Psi_2.M_{\trusted}$.
      %
      There are two possible cases for $\Psi_1 \stepboxstar \Psi_3 \stepbox \Psi_4$, and notice that in both $\Psi_3.p = \Psi_1.p = \untrusted$ (by \lemref{lemma:appendix:nacl:stepbox-preserves-p}).
      %
      When $\Psi_3 \stepp{p} \Psi_4$, \lemref{lemma:appendix:nacl:lib-p-steps-preserve-app-memory} gives us that $\Psi_3.M_{\trusted} = \Psi_4.M_{\trusted}$ and then \lemref{lemma:appendix:nacl:context-in-trusted} gives us that $ctx_{\Psi_4} = ctx_{\Psi_3} = ctx_{\Psi_1}$.
      %
      When $\Psi_3 \stepwb \Psi_4$, case 2 of our inductive hypothesis gives us that $\Psi_1.M([ctx_{\Psi_0}, ctx_{\Psi_1}]) = \Psi_3.M([ctx_{\Psi_0}, ctx_{\Psi_3}]) = \Psi_4.M([ctx_{\Psi_0}, ctx_{\Psi_4}])$ and $ctx_{\Psi_4} = ctx_{\Psi_3} = ctx_{\Psi_1}$.
    \end{subproof}

    Finally, by \lemref{lemma:appendix:nacl:stepbox-preserves-p} we get that $\Psi_2 = \untrusted$ and then inspection of the reduction rule for $\cgateret$ gives us that $\Psi.M([ctx_{\Psi_0}, ctx_{\Psi})) = \Psi'.M([ctx_{\Psi_0}, ctx_{\Psi'}))$ and $ctx_{\Psi} = ctx_{\Psi'}$.
  \end{itemize}

  Second we consider the case where $\currentcom{\Psi}{c}{\untrusted}$.
  \begin{itemize}
  \pfcase{No callbacks}

    We have that $\currentop{\Psi}{\cgatecall{n}{e}}$ and there exist $\Psi_1$, $\Psi_2$, and $p$ such that $\Psi \step \Psi_1 \steppstar{p} \Psi_2 \step \Psi'$ where $\currentop{\Psi_2}{\cgateret}$.
    %
    By the structure of a NaCl program we have that $ctx_{\Psi_1} = ctx_{\Psi_2}$ and $\Psi_1.M([ctx_{\Psi_0}, ctx_{\Psi_1}]) = \Psi_2.M([ctx_{\Psi_0}, ctx_{\Psi_2}])$.
    %
    By assumption we have that $\currentcom{\Psi}{\cgatecall{n}{e}}{\untrusted}$ and therefore by \lemref{lemma:appendix:nacl:wb-flips-priv} we have that $\currentcom{\Psi_2}{\cgateret}{\trusted}$.
    %
    By \lemref{lemma:appendix:nacl:ctx-positive} and inspection of the reduction rules for $\cgatecall{n}{e}$ and $\cgateret$ we have that $\Psi.M([ctx_{\Psi_0}, ctx_{\Psi})) = \Psi'.M([ctx_{\Psi_0}, ctx_{\Psi'}))$ and $ctx_{\Psi} = ctx_{\Psi'}$.

  \pfcase{Callbacks}

    We have that $\currentop{\Psi}{\cgatecall{n}{e}}$ and there exist $\Psi_1$, $\Psi_2$  such that $\Psi \step \Psi_1 \stepboxstar \Psi_2 \step \Psi'$ where $\currentop{\Psi_2}{\cgateret}$.
    %
    By inspection of the reduction rule for $\cgatecall{n}{e}$ we have that $ctx_{\Psi_1} = ctx_{\Psi} + 1$.
    %
    We now show, by induction on $\Psi_1 \stepboxstar \Psi_2$, that $ctx_{\Psi_2} = ctx_{\Psi_1}$ and $\Psi_1.M([ctx_{\Psi_0}, ctx_{\Psi_1})) = \Psi_2.M([ctx_{\Psi_0}, ctx_{\Psi_2}))$.

    \begin{subproof}
      If there are no steps then clearly $ctx_{\Psi_2} = ctx_{\Psi_1}$ and all of $\Psi_1.M_{\trusted} = \Psi_2.M_{\trusted}$.
      %
      There are two possible cases for $\Psi_1 \stepboxstar \Psi_3 \stepbox \Psi_4$, and notice that in both $\Psi_3.p = \Psi_1.p = \trusted$ (by \lemref{lemma:appendix:nacl:stepbox-preserves-p}).
      %
      When $\Psi_3 \stepp{p} \Psi_4$, the structure of a NaCl program gives us that $ctx_{\Psi_1} = ctx_{\Psi_2}$ and $\Psi_1.M([ctx_{\Psi_0}, ctx_{\Psi_1}]) = \Psi_2.M([ctx_{\Psi_0}, ctx_{\Psi_2}])$.
      %
      When $\Psi_3 \stepwb \Psi_4$, case 1 of our inductive hypothesis gives us that $\Psi_1.M([ctx_{\Psi_0}, ctx_{\Psi_1})) = \Psi_3.M([ctx_{\Psi_0}, ctx_{\Psi_3})) = \Psi_4.M([ctx_{\Psi_0}, ctx_{\Psi_4}))$ and $ctx_{\Psi_4} = ctx_{\Psi_3} = ctx_{\Psi_1}$.
    \end{subproof}

    Finally, by \lemref{lemma:appendix:nacl:stepbox-preserves-p} we get that $\Psi_2 = \trusted$ and then inspection of the reduction rule for $\cgateret$ gives us that $\Psi.M([ctx_{\Psi_0}, ctx_{\Psi}]) = \Psi'.M([ctx_{\Psi_0}, ctx_{\Psi'}])$ and $ctx_{\Psi} = ctx_{\Psi'}$.
  \end{itemize}
\end{proof}

\begin{proposition}
  NaCl has callee-save register integrity.
\end{proposition}
\begin{proof}
  Follows from \lemref{lemma:appendix:nacl:context-integrity}, \lemref{lemma:appendix:nacl:wb-flips-priv}, and inspection of the reduction rules for $\cgatecall{n}{e}$ and $\cgateret$.
\end{proof}

\begin{lemma} \label{lemma:appendix:nacl:return-address-integrity}
  Let $\Psi_0 \in \programs$, $\pi = \Psi_0 \stepstar \Psi$, and $\Psi \stepwb \Psi'$, then $\Psi'.M(\operatorname{return-address}_{\trusted}(\pi)) = \Psi.M(\operatorname{return-address}_{\trusted}(\pi))$.
\end{lemma}
\begin{proof}
  First we consider the case where $\currentcom{\Psi}{c}{\trusted}$.
  \begin{itemize}
  \pfcase{No callbacks}

    \sloppy
    We have that $\currentop{\Psi}{\cgatecall{n}{e}}$ and there exist $\Psi_1$, $\Psi_2$, and $p$ such that $\Psi \step \Psi_1 \steppstar{p} \Psi_2 \step \Psi'$ where $\currentop{\Psi_2}{\cgateret}$.
    %
    By inspection of the reduction rule for $\cgatecall{n}{e}$ we see that $\Psi_1.M(\Psi.sp + 1) = \Psi.pc + 1$.
    %
    This is adding to the top of the stack, so by the structure of a NaCl program we have that $\Psi_1.M(\operatorname{return-address}_{\trusted}(\pi)) = \Psi.M(\operatorname{return-address}_{\trusted}(\pi))$.
    %
    By \lemref{lemma:appendix:nacl:lib-p-steps-preserve-app-memory} we have that $\Psi_1.M_{\trusted} = \Psi_2.M_{\trusted}$ and therefore \lemref{lemma:appendix:nacl:context-in-trusted} gives us that $ctx_{\Psi_2} = ctx_{\Psi_1}$ and $\Psi_2.M(\operatorname{return-address}_{\trusted}(\pi)) = \Psi_1.M(\operatorname{return-address}_{\trusted}(\pi))$.
    %
    If we inspect the trampoline code we see that, right before we execute the $\cret{}$, we have set $sp$ to $\Psi_2.M(ctx_{\Psi_2}) = \Psi_1.M(ctx_{\Psi_1}) = \Psi.sp + 1$.
    %
    Thus, after returning the only part of the application stack that we modify is $\Psi.sp + 1$.
    %
    This, and the fact that $\Psi_2.M(\operatorname{return-address}_{\trusted}(\pi)) = \Psi.M(\operatorname{return-address}_{\trusted}(\pi))$ gives us that $\Psi.M(\operatorname{return-address}_{\trusted}(\pi)) = \Psi'.M(\operatorname{return-address}_{\trusted}(\pi))$.

  \pfcase{Callbacks}

    We have that $\currentop{\Psi}{\cgatecall{n}{e}}$ and there exist $\Psi_1$, $\Psi_2$  such that $\Psi \step \Psi_1 \stepboxstar \Psi_2 \step \Psi'$ where $\currentop{\Psi_2}{\cgateret}$.
    %
    By inspection of the reduction rule for $\cgatecall{n}{e}$ we see that $\Psi_1.M(\Psi.sp + 1) = \Psi.pc + 1$.
    %
    This is adding to the top of the stack, so by the structure of a NaCl program we have that $\Psi_1.M(\operatorname{return-address}_{\trusted}(\pi)) = \Psi.M(\operatorname{return-address}_{\trusted}(\pi))$.
    %
    We now show, by induction on $\Psi_1 \stepboxstar \Psi_2$ that $\Psi_2.M(ctx_{\Psi_2}) = \Psi.sp + 1$ and $\Psi_2.M(\operatorname{return-address}_{\trusted}(\pi)) = \Psi_1.M(\operatorname{return-address}_{\trusted}(\pi))$.

    \begin{subproof}
      If there are no steps then $\Psi_2 = \Psi_1$ and both goals hold immediately.
      %
      There are two possible cases for $\Psi_1 \stepboxstar \Psi_3 \stepbox \Psi_4$ and notice that in both $\Psi_3.p = \Psi_1.p = \untrusted$ (by \lemref{lemma:appendix:nacl:stepbox-preserves-p}).
      %
      If $\Psi_3 \stepp{p} \Psi_4$ then \lemref{lemma:appendix:nacl:lib-p-steps-preserve-app-memory} gives us that $\Psi_4.M_{\trusted} = \Psi_3.M_{\trusted}$ and our goal holds (as all of $\operatorname{return-address}_{\trusted}(\pi)$ is in $S_{\trusted}$).
      %
      If $\Psi_3 \stepwb \Psi_4$, then \lemref{lemma:appendix:nacl:context-integrity} gives us that $ctx_{\Psi_3} = ctx_{\Psi_4}$ and $\Psi_3.M([ctx_{\Psi_0}, ctx_{\Psi_3}]) = \Psi_4.M([ctx_{\Psi_0}, ctx_{\Psi_4}])$ and therefore that $\Psi_4.M(ctx_{\Psi_4}) = \Psi.sp + 1$.
      %
      $\operatorname{return-address}_{\trusted}(\Psi_0 \stepstar \Psi_3) = \operatorname{return-address}_{\trusted}(\pi) \uplus \Psi.sp + 1$, so our inductive hypothesis gives us that $\Psi_4.M(\operatorname{return-address}_{\trusted}(\pi)) = \Psi_3.M(\operatorname{return-address}_{\trusted}(\pi))$.
    \end{subproof}
  \end{itemize}

  Second we consider the case where $\currentcom{\Psi}{c}{\untrusted}$.
  \begin{itemize}
  \pfcase{No callbacks}

    We have that $\currentop{\Psi}{\cgatecall{n}{e}}$ and there exist $\Psi_1$, $\Psi_2$, and $p$ such that $\Psi \step \Psi_1 \steppstar{p} \Psi_2 \step \Psi'$ where $\currentop{\Psi_2}{\cgateret}$.
    %
    By inspection of the reduction rule for $\cgatecall{n}{e}$ we see that $\Psi_1.M(\operatorname{return-address}_{\trusted}(\pi)) = \Psi.M(\operatorname{return-address}_{\trusted}(\pi))$.
    %
    By the structure of a NaCl program we have that any call stack elements that are added during the callback will be popped before the $\cgateret$.
    %
    Thus, $\Psi_2.M(\operatorname{return-address}_{\trusted}(\pi)) = \Psi_1.M(\operatorname{return-address}_{\trusted}(\pi))$.
    %
    Inspection of the reduction rule for $\cgateret$ then gives us that $\Psi'.M(\operatorname{return-address}_{\trusted}(\pi)) = \Psi_2.M(\operatorname{return-address}_{\trusted}(\pi)) = \Psi.M(\operatorname{return-address}_{\trusted}(\pi))$.

  \pfcase{Callbacks}

    We have that $\currentop{\Psi}{\cgatecall{n}{e}}$ and there exist $\Psi_1$, $\Psi_2$  such that $\Psi \step \Psi_1 \stepboxstar \Psi_2 \step \Psi'$ where $\currentop{\Psi_2}{\cgateret}$.
    %
    By inspection of the reduction rule for $\cgatecall{n}{e}$ we see that $\Psi_1.M(\operatorname{return-address}_{\trusted}(\pi)) = \Psi.M(\operatorname{return-address}_{\trusted}(\pi))$.
    %
    We now show, by induction on $\Psi_1 \stepboxstar \Psi_2$ that $\Psi_2.M(\operatorname{return-address}_{\trusted}(\pi)) = \Psi_1.M(\operatorname{return-address}_{\trusted}(\pi))$.

    \begin{subproof}
      \sloppy
      If there are no steps then $\Psi_2 = \Psi_1$ and the goal holds immediately.
      %
      There are two possible cases for $\Psi_1 \stepboxstar \Psi_3 \stepbox \Psi_4$ and notice that in both $\Psi_3.p = \Psi_1.p = \trusted$ (by \lemref{lemma:appendix:nacl:stepbox-preserves-p}).
      %
      If $\Psi_3 \stepp{p} \Psi_4$ then the structure of a NaCl program gives us that any call stack elements that are added during the callback will be popped before the $\cgateret$, and therefore our inductive invariant is maintained.
      %
      If $\Psi_3 \stepwb \Psi_4$, then notice that $\operatorname{return-address}_{\trusted}(\Psi_0 \stepstar \Psi_3) = \operatorname{return-address}_{\trusted}(\pi)$, so our inductive hypothesis gives us that $\Psi_4.M(\operatorname{return-address}_{\trusted}(\pi)) = \Psi_3.M(\operatorname{return-address}_{\trusted}(\pi))$.
    \end{subproof}

    Inspection of the reduction rule for $\cgateret$ then gives us that $\Psi'.M(\operatorname{return-address}_{\trusted}(\pi)) = \Psi_2.M(\operatorname{return-address}_{\trusted}(\pi)) = \Psi.M(\operatorname{return-address}_{\trusted}(\pi))$.
  \end{itemize}
\end{proof}

\begin{proposition}
  NaCl has return address integrity.
\end{proposition}
\begin{proof}
  We have that $\Psi_0 \in \programs$, $\pi = \Psi_0 \stepstar \Psi$, $\Psi.p = \trusted$, and $\Psi \stepwb \Psi'$ and wish to show that $\Psi.M(\operatorname{return-address}_{\trusted}(\pi)) = \Psi'.M(\operatorname{return-address}_{\trusted}(\pi))$, $\Psi'.sp = \Psi.sp$, and $\Psi'.pc = \Psi.pc + 1$.

  \lemref{lemma:appendix:nacl:return-address-integrity} gives us that $\Psi.M(\operatorname{return-address}_{\trusted}(\pi)) = \Psi'.M(\operatorname{return-address}_{\trusted}(\pi))$.
  %
  We proceed by simultaneous induction on the well-bracketed transition $\Psi \stepwb \Psi'$ and the length of $\Psi_0 \stepstar \Psi \stepwb \Psi'$ to show that $\Psi'.sp = \Psi.sp$, and $\Psi'.pc = \Psi.pc + 1$.

  \begin{itemize}
  \pfcase{No callbacks}

    We have that $\currentop{\Psi}{\cgatecall{n}{e}}$ and there exist $\Psi_1$, $\Psi_2$, and $p$ such that $\Psi \step \Psi_1 \steppstar{p} \Psi_2 \step \Psi'$ where $\currentop{\Psi_2}{\cgateret}$.
    %
    By inspection of the reduction rule for $\cgatecall{n}{e}$ we see that $\Psi_1.M(\Psi.sp + 1) = \Psi.pc + 1$ and $\Psi_1.M(ctx_{\Psi_1}) = \Psi.sp + 1$.
    %
    By \lemref{lemma:appendix:nacl:lib-p-steps-preserve-app-memory} we have that $\Psi_1.M_{\trusted} = \Psi_2.M_{\trusted}$ and therefore \lemref{lemma:appendix:nacl:context-in-trusted} gives us that $ctx_{\Psi_2} = ctx_{\Psi_1}$.
    %
    If we inspect the trampoline code we see that, right before we execute the $\cret{}$, we have set $sp$ to $\Psi_2.M(ctx_{\Psi_2}) = \Psi_1.M(ctx_{\Psi_1}) = \Psi.sp + 1$.
    %
    Thus, after returning we have that $\Psi'.pc = \Psi.pc + 1$, $\Psi'.sp = \Psi.sp$.

  \pfcase{Callbacks}

    We have that $\currentop{\Psi}{\cgatecall{n}{e}}$ and there exist $\Psi_1$, $\Psi_2$  such that $\Psi \step \Psi_1 \stepboxstar \Psi_2 \step \Psi'$ where $\currentop{\Psi_2}{\cgateret}$.
    %
    By inspection of the reduction rule for $\cgatecall{n}{e}$ we see that $\Psi_1.M(\Psi.sp + 1) = \Psi.pc + 1$ and $\Psi_1.M(ctx_{\Psi_1}) = \Psi.sp + 1$.
    %
    We now show, by induction on $\Psi_1 \stepboxstar \Psi_2$ that $\Psi_2.M(\Psi.sp + 1) = \Psi.pc + 1$ and $\Psi_2.M(ctx_{\Psi_2}) = \Psi.sp + 1$.

    \begin{subproof}
      If there are no steps then $\Psi_2 = \Psi_1$ and both hold immediately.
      %
      There are two possible cases for $\Psi_1 \stepboxstar \Psi_3 \stepbox \Psi_4$ and notice that in both $\Psi_3.p = \Psi_1.p = \untrusted$ (by \lemref{lemma:appendix:nacl:stepbox-preserves-p}).
      %
      If $\Psi_3 \stepp{p} \Psi_4$ then \lemref{lemma:appendix:nacl:lib-p-steps-preserve-app-memory} gives us that $\Psi_4.M_{\trusted} = \Psi_3.M_{\trusted}$ and both hold (as the invariants are on $M_{\trusted}$).
      %
      If $\Psi_3 \stepwb \Psi_4$, then \lemref{lemma:appendix:nacl:context-integrity} gives us that $ctx_{\Psi_3} = ctx_{\Psi_4}$ and $\Psi_3.M([ctx_{\Psi_0}, ctx_{\Psi_3}]) = \Psi_4.M([ctx_{\Psi_0}, ctx_{\Psi_4}])$ and therefore that $\Psi_4.M(ctx_{\Psi_4}) = \Psi.sp + 1$.
      %
      $\operatorname{return-address}_{\trusted}(\Psi_0 \stepstar \Psi_3) = \operatorname{return-address}_{\trusted}(\pi) \uplus \Psi.sp + 1$ so \lemref{lemma:appendix:nacl:return-address-integrity} gives us that $\Psi_4.M(\Psi.sp + 1) = \Psi_3.M(\Psi.sp + 1) = \Psi.pc + 1$.
    \end{subproof}

    Finally, if we inspect the trampoline code we see that, right before we execute the $\cret{}$, we have set $sp$ to $\Psi_2.M(ctx_{\Psi_2}) = \Psi.sp + 1$.
    %
    Thus, after returning we have that $\Psi'.pc = \Psi.pc + 1$, $\Psi'.sp = \Psi.sp$.
  \end{itemize}
\end{proof}