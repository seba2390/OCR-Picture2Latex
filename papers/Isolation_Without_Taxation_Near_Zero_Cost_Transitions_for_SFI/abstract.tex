\begin{abstract}
Software sandboxing or software-based fault isolation (SFI) is a lightweight
approach to building secure systems out of untrusted components.
%
Mozilla, for example, uses SFI to harden the Firefox browser by sandboxing
third-party libraries, and companies like Fastly and Cloudflare use SFI to
safely co-locate untrusted tenants on their edge clouds.
%
While there have been significant efforts to optimize and verify SFI
enforcement, context switching in SFI systems remains largely unexplored:
almost all SFI systems use \emph{heavyweight transitions} that are not only
error-prone but incur significant performance overhead from saving, clearing,
and restoring registers when context switching.
%
We identify a set of \emph{zero-cost conditions} that characterize when
sandboxed code has sufficient structured to guarantee security via lightweight
\emph{zero-cost} transitions (simple function calls).
%
We modify the Lucet Wasm compiler and its runtime to use zero-cost transitions,
eliminating the undue performance tax on systems that rely on Lucet for
sandboxing (e.g., we speed up image and font rendering in Firefox by up to
\ffMaxImgSpeedupWasmZeroHeavy and \ffMaxFontSpeedupWasmZeroHeavy respectively).
%
To remove the Lucet compiler and its correct implementation of the Wasm
specification from the trusted computing base, we (1) develop a \emph{static
binary verifier}, \verifname{}, which (in seconds) checks that binaries produced
by Lucet satisfy our zero-cost conditions, and (2)
%
prove the soundness of \verifname{} by developing a logical relation that
captures when a compiled Wasm function is semantically well-behaved with respect
to our zero-cost conditions.
%
Finally, we show that our model is useful beyond Wasm by describing a new,
purpose-built SFI system, \trsegmentsfi, that uses x86 segmentation and LLVM
with mostly off-the-shelf passes to enforce our zero-cost conditions; our
prototype performs on-par with the state-of-the-art Native Client SFI system.
%
\end{abstract}
