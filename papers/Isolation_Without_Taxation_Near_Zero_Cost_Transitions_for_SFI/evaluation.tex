\section{Evaluation}
\label{sec:eval}
We evaluate our zero-cost model by asking four questions:
\begin{CompactItemize}
\item \textbf{Q1}: What is the cost of a context switch? (\secref{subsec:eval-transitions})
\item \textbf{Q2}: What is end-to-end performance gain of Wasm-based SFI due to zero-cost transitions? (\secref{subsec:eval-wasm})
\item \textbf{Q3}: What is the performance overhead of purpose-built zero-cost SFI enforcement? (\secref{subsec:eval-zerocostsfi})
\item \textbf{Q4}: Is the \verifname verifier effective? (\secref{subsec:verifier-eval})
\end{CompactItemize}

Since our zero-cost condition enforcement does incur some runtime overhead, \textbf{Q2} and \textbf{Q3} are heavily workload-dependent.
%
The benefit a workload receives from the zero-cost approach will be in direct proportion to the frequency with which it performs domain transitions.


\para{Systems}
To investigate the first three questions, we consider two groups of SFI systems.
%
The first group compares a number of different transition models for Wasm-based SFI for 64-bit binaries, built on top of the Lucet compiler~\cite{lucet}.
%
All of these will have identical runtime overhead, meaning that the only variance between them will be due to transition overhead.
%
The \trlucet build uses the original heavyweight springboards and trampolines
shipped with the Lucet runtime written in Rust.
%
\trfullswitch adopts techniques from NaCl and uses optimized
assembly to save and restore application context during transitions.
%
\trfast implements our zero-cost transition system, meaning transitions are
simple function calls.
%
To understand the overhead of register saving/restoring and stack
switching, we also evaluate a \trregsave build which saves/restores registers
like \trfullswitch, but shares the library and application stack like
\trfast.

The second group compares optimized SFI techniques for 32-bit binaries.
%
Wasm-based SFI imposes overheads far beyond what is strictly necessary to
enforce our zero-cost conditions, both because of the immaturity of the Lucet
compiler in comparison to more established compilers such as Clang, and because
Wasm inherently enforces additional restrictions on compiled code (e.g.,
structured intra-function control flow).
%
We design \trsegmentsfi~(\secref{sec:segments-secure}) to enforce only our zero-cost-conditions and nothing more, aiming to benchmark it against the Native Client 32-bit isolation scheme (\trnacl)~\cite{yee_native_2009}, arguably the fastest production SFI system available, which requires heavyweight transitions.
%
Both systems make use of memory segmentation, a 32-bit x86-only feature for fast memory isolation.
%
Unfortunately, we cannot make a uniform comparison between \trnacl, \trsegmentsfi, and \trfast since Lucet only supports a 64-bit target.

Each group additionally uses unsandboxed, insecure native execution (\texttt{Vanilla}) as a baseline.
%
To represent the best possible performance of schemes relying on heavyweight
transitions, we also benchmark \tridealheavy and \tridealheavysixfour,
ideal hardware isolation schemes, which incur no runtime
overhead but require heavyweight transitions.
%
To simulate the performance of these ideal schemes, we simply measure the performance
of native code with heavyweight trampolines.

We integrate all of the above SFI schemes into Firefox using the RLBox
framework~\cite{rlbox}.
%
Since RLBox already provides plugins for the \trlucet and \trnacl builds, we
only implement the plugins for the remaining system builds.

\para{Benchmarks}
%
We use a micro-benchmark to evaluate the cost of a single transition for our
different transition models, using unsandboxed native calls as a baseline
(\textbf{Q1}).

We answer questions \textbf{Q2}--\textbf{Q3} by measuring the end-to-end
performance of font and image rendering in Firefox, using a sandboxed
\libgraphite and \libjpeg, respectively.
%
We use these libraries because they have many cross-sandbox transitions, which
\citet{rlbox} previously observed to affect the overall browser performance.
%
To evaluate the performance of \libgraphite, we use Kew's
benchmark\footnote{Available at
\url{https://jfkthame.github.io/test/udhr_urd.html}}, which reflows the text on
a page ten times, adjusting the size of the fonts each time to negate the
effects of font caches.
%
When calling \libgraphite, Firefox makes a number of calls into the sandbox
roughly proportional to the number of glyphs on the page.
%
We run this benchmark 100 times and report the median execution
time below (all values have standard deviations within 1\%).

To evaluate the performance of \libjpeg, we measure the overhead of rendering
images of varying complexity and size.
%
Since the work done by the sandboxed \libjpeg, per call, is proportional to the
width of the image\dash---Firefox executes the library in \emph{streaming
mode}, one row at a time\dash---we consider images of different widths,
keeping the image height fixed.
%
This allows us to understand the benefits and limitations of zero-cost
transitions, since the proportion of execution time spent context-switching decreases
as the image width increases.
%
We do this for three images, of varying complexity: a simple image consisting
of a single color (\simplejpeg), a stock image from the Image Compression
benchmark suite\footnote{Online:
\url{https://imagecompression.info/test_images/}.  Visited Dec 9, 2020.}
(\stockjpeg), and an image of random pixels (\randomjpeg).
%
We render each image 500 times and report the median time (standard
deviations are all under 1\%).

Finally, we use \SPECOhSix to partly evaluate the sandboxing overhead of our
purpose-built \trsegmentsfi SFI system (\textbf{Q3}), and to measure
\verifname's verification speed (\textbf{Q4}).

\para{Machine and software setup}
%
We run all but the verification benchmarks on an \Intel
Core\textsuperscript{TM} i7-6700K machine with four 4GHz cores, 64GB RAM,
running Ubuntu 20.04.1 LTS (kernel version 5.4.0-58).
%
We run benchmarks with a shielded isolated cpuset~\cite{cpu-shielding}
consisting of one core with hyperthreading disabled and the clock frequency
pinned to 2.2GHz.
%
We generate Wasm sandboxed code in two steps: First, we compile C/\C++
to Wasm using Clang-11, and then compile Wasm to native code using the 
fork of the Lucet used by RLBox (snapshot from Dec 9, 2020).
%
We generate NaCl sandboxed code using a modified version of Clang-4.
%
We compile all other C/\C++ source code, including \trsegmentsfi sandboxed code and
benchmarks using Clang-11.
%
We implement our Firefox benchmarks on top of Firefox Nightly (from August 22,
2020).

\para{Summary of results}
%
We find that the performance of Wasm-based isolation
can be significantly improved by adopting zero-cost transitions, but that
Lucet-compiled WebAssembly's runtime overhead means that it does not outperform
more optimised isolation schemes in end-to-end tests.
%
The low performance overhead of \trsegmentsfi demonstrates that these runtime
overheads are not inherent to the zero-cost approach, and that an optimised
zero-cost SFI system can significantly outperform more traditional schemes,
especially for workloads with a large number of transitions.
%
Finally, we find that we can efficiently check zero-cost conditions at the
binary level, for Lucet compiled code, with no false positives.


%%%%%%%%%%%%%%%%%%%%%%%%%%%%%%%%%%%%%%%%%%%%%%

\begin{figure}
\footnotesize

\begin{tabular}{p{2.5cm}|cccc}
    \toprule
    \textbf{Build}
  & \textbf{Direct call}
  & \textbf{Indirect call}
  & \textbf{Callback}
  & \textbf{Syscall}
  \\
  \toprule
  \trnative (in C) &
  1ns & 56ns & 56ns & 24ns
  \\
%
  \trlucet &
  --- & 1137ns & --- & ---
  \\
%
  \trfullswitch &
  120ns & 209ns & 172ns & 192ns
  \\
%
  \trregsave &
  120ns & 210ns & 172ns & 192ns
  \\
%
  \textbf{\trfast} &
  \bf 7ns & \bf 66ns & \bf 67ns & \bf 60ns
  \\
  \midrule
  \trnative (in C, 32-bit) &
  1ns & 74ns & 74ns & 37ns
  \\
  %
  \trnacl &
  --- & 714ns & 373ns & 356ns
  \\
  %
  \textbf{\trsegmentsfi} &
  \bf 24ns & \bf 108ns & \bf 80ns & \bf 88ns
  \\
  \bottomrule
\end{tabular}

\caption{
%
Costs of transitions in different isolation models.
%
Zero-cost transitions are shown in \textbf{boldface}.
%
\trnative is the performance of an unsandboxed C function call, to serve as a baseline.
}
\label{fig:transition-overheads}
\end{figure}

%%%%%%%%%%%%%%%%%%%%%%%%%%%%%%%%%%%%%%%%%%%%%%

\subsection{The cost of transitions}
\label{subsec:eval-transitions}
%
We measure the cost of different cross-domain transitions\dash---direct and
indirect calls into the sandbox, callbacks from the sandbox, and syscall
invocations from the sandbox\dash---for the different system builds
described above.
%
To expose overheads fully, we choose extremely fast payloads---either a
function that just adds two numbers or the \gettimeofday syscall,
which relies on Linux's vDSO to avoid CPU ring changes.
%
The results are shown in \figref{fig:transition-overheads}.
%
All numbers are averages of one million repetitions, and repeated runs have
negligible standard deviation.\footnote{
%
Lucet and NaCl don't support direct sandbox
calls; Lucet further does not support custom callbacks or syscall invocations.
}

We make several observations.
%
First, among Wasm-based SFI schemes, zero-cost transitions (\trfast) are
significantly faster than even optimized heavyweight transitions
(\trfullswitch).
%
Lucet's existing indirect calls written in Rust (\trlucet) are significantly
slower than both.
%
Second, the cost of stack switching (the difference of \trfullswitch and
\trregsave) is surprisingly negligible.
%
Third, the performance of \trnative and \trfast should be identical but is not.
%
This is \emph{not} because our transitions have a hidden cost. Rather, it's
because we are comparing code produced by two different compilers:
\trnative is native code produced by Clang,  while \trfast is code produced by
Lucet, and Lucet's code generation is not yet highly
optimized~\cite{cranelift-speedup}.
%
For example, in the benchmark that adds two numbers, Clang eliminates
the function prologue and epilogue that save and restore the frame
pointer, while Lucet does not.
%
%
We observe similar trends for hardware-based isolation.
%
For example, we find that \trsegmentsfi transitions are much faster than
\tridealheavy and \trnacl transitions and only \tranSegzeroNativeFuncDiff
slower than \trnative for direct calls.
%
Finally, we observe that \trsegmentsfi is slower than \trfast: hardware
isolation schemes like \trsegmentsfi and \trnacl execute instructions to enable
or disable the hardware based memory isolation in their transitions.

\subsection{End-to-end performance improvements of zero-cost transitions for Wasm}
\label{subsec:eval-wasm}

%%%%%%%%%%%%%%%%%%%%%%%%%%%%%%%%%%%%%%%%%%%%%%

\begin{figure*}
  
  \begin{subfigure}{0.32\textwidth}
    \includegraphics[width=4.5cm]{figs/jpeg_simple_overhead_wasm.pdf}
    \caption{\simplejpeg}
    \label{fig:jpeg-simpleimg}
  \end{subfigure}
  %
  \begin{subfigure}{0.32\textwidth}
    \includegraphics[width=4.5cm]{figs/jpeg_stock_overhead_wasm.pdf}
    \caption{\stockjpeg}
    \label{fig:jpeg-stockimg}
  \end{subfigure}
  %
  \begin{subfigure}{0.32\textwidth}
    \includegraphics[width=4.5cm]{figs/jpeg_random_overhead_wasm.pdf}
    \caption{\randomjpeg}
    \label{fig:jpeg-randomimg}
  \end{subfigure}
  
  \caption{
    %
    Performance of different Wasm transitions on rendering of (a)~a simple
    image with one color, (b)~a stock image, and (c)~a complex image with
    random pixels, normalized to \trfast.
    %
    \trfast transitions outperform other transitions. The difference
    diminishes with width, but narrower images are more common on
    the web.
  }
  \label{fig:jpeg-img}
\end{figure*}

%%%%%%%%%%%%%%%%%%%%%%%%%%%%%%%%%%%%%%%%%%%%%%

We evaluate the end-to-end performance impact of the different transition
models on Wasm-sandboxed font and image rendering as used in Firefox (see
\secref{sec:eval}).

\para{Font rendering}
We report the performance of \libgraphite isolated with Wasm-based schemes on
Kew's benchmark below:
 
\begin{center}
\footnotesize

\begin{tabular}{p{1.65cm}|p{1.3cm}p{1.4cm}p{1cm}p{1.4cm}p{1cm}p{1.6cm}}
      \toprule
    & \textbf{\trlucet}
    & \textbf{\trfullswitch}
    & \textbf{\trregsave}
    & \textbf{\trfast}
    & \textbf{\trnative}
    & \textbf{\tridealheavysixfour}
    \\
\toprule
    \textbf{Font render}
    & 8173ms & 2246ms & 2230ms & 2032ms & 1116ms & 1563ms  \\
\bottomrule
\end{tabular}
\end{center}

\noindent
As expected, Wasm with zero-cost transitions (\trfast) outperforms the
other Wasm-based SFI transition models.
%
Compared to \trfast, Lucet's existing transitions slow down rendering 
by over \ffMaxFontSlowdownWasmLucetZero.\footnote{
  This overhead is smaller than the 8$\times$ overhead reported by
  \citet{rlbox}; we attribute this difference to the different compilers\dash---we
  use a more recent, and faster, version of Lucet.
}
%
But, even the optimized heavyweight transitions (\trfullswitch) impose a
\ffMaxFontSlowdownWasmHeavyZero performance tax.
%
This overhead is due to register saving/restoring; stack switching
only accounts for \ffMaxFontStackSwitchWasmOverhead overhead.

While these results show that existing Wasm-based isolation schemes can benefit
from switching to zero-cost transitions\dash---and indeed the speed-up due to
zero-cost transitions allowed Mozilla to ship the Wasm-sandboxed
\libgraphite\dash---they also show that Lucet-compiled Wasm is slow
($\sim$80$\%$ slower than Vanilla).
%
This, unfortunately, means that the transition cost savings alone are not
enough to beat \tridealheavysixfour, even for a workload with many transitions.
%
To compete with this ideal SFI scheme with heavyweight transitions, we would
need to reduce the runtime overhead to $\sim$40$\%$.
%
\citet{not-so-fast} report the average runtime overhead of Mozilla SpiderMonkey JIT-compiled WebAssembly compared to native as $\sim$45$\%$ in a different set of benchmarks, while noting many correctable inefficiencies in the generated assembly code, suggesting that there is a lot of room for Lucet to be further optimised.

\para{Image rendering}
%
\figref{fig:jpeg-img} report the overhead of Wasm-based sandboxing on
image rendering, normalized to \trfast to highlight the relative overheads
of different transitions as compared to our zero-cost transitions.
%
We report results of \trlucet separately, in \iftechreport{\appref{appendix:img}
(\figref{fig:jpeg-img-lucet})}{the technical
appendix~\cite{kolosick2021isolation}} because the rendering times are up to
\ffMaxImgSlowdownWasmLucetZero longer than the other builds.
%
Here, we instead focus on evaluating the overheads of optimized
heavy transitions.

As expected, \trfast significantly outperforms other transitions when images 
are narrower and simpler.
%
On \simplejpeg, \trfullswitch and \trlucet can take as much as 
\ffMaxImgSimpleSlowdownWasmHeavyZero and \ffMaxImgSimpleSlowdownWasmLucetZero
longer to render the image as with \trfast transitions.
%
However, this performance gap diminishes as image width increases (and the
relative cost of context switching decreases).
%
For \stockjpeg and \randomjpeg, the \trfullswitch trends are similar, but
the rendering time differences start at about 
\ffMaxImgStockRandomSlowdownWasmHeavyZero.
%
Lucet's existing transitions (\trlucet) are still significantly slower 
than zero-cost transitions (\trfast) even on wide images.


%%%%%%%%%%%%%%%%%%%%%%%%%%%%%%%%%%%%%%%%%%%%%%

\begin{figure}
  \includegraphics[width=6.5cm]{figs/image_sizes.pdf}
  \caption{
    Cumulative distribution of image widths on the landing pages of the Alexa 
    top 500 websites.
    %
    Over 80\% of the images have widths under 480 pixels.
    %
    Narrower images have a higher transition rate, and thus higher relative
    overheads when using expensive transitions.
  }
  \label{fig:image-sizes}
\end{figure}

%%%%%%%%%%%%%%%%%%%%%%%%%%%%%%%%%%%%%%%%%%%%%%

Though the differences between the transitions are smaller as the image width
increases, many images on the Web are narrow.
\figref{fig:image-sizes} shows the distribution of images on the landing
pages of the Alexa top 500 websites. Of the 10.6K images, 8.6K (over 80\%) have
widths between 1 and 480 pixels, a range in which zero-cost transitions
noticeably outperform the other kinds of transitions.

Like font rendering, we measure the target runtime overhead Lucet should
achieve to beat \tridealheavysixfour end-to-end for rendering images.
%
We report our results in \iftechreport{\figref{fig:jpeg-img-ideal64} in
\appref{appendix:img}}{the technical appendix~\cite{kolosick2021isolation}}.
%
For the small simple image, we observe this to be 94\%\dash---this is approximately the overhead of Lucet that we see already today.
%
For the small stock image, we observe this to be 15\%\dash---this is much smaller than the overhead of Lucet today, but lower overheads have been demonstrated on some benchmarks by the prototype Wasm compiler of~\citet{sledge}.


\subsection{Performance overhead of purpose-built zero-cost SFI enforcement}

\label{subsec:eval-zerocostsfi}
%%%%%%%%%%%%%%%%%%%%%%%%%%%%%%%%%%%%%%%%%%%%%%

\begin{figure*}
  
  \begin{subfigure}{0.32\textwidth}
    \includegraphics[width=4.5cm]{figs/jpeg_simple_overhead_hw.pdf}
    \caption{\simplejpeg}
    \label{fig:jpeg-simpleimg-hw}
  \end{subfigure}
  %
  \begin{subfigure}{0.32\textwidth}
    \includegraphics[width=4.5cm]{figs/jpeg_stock_overhead_hw.pdf}
    \caption{\stockjpeg}
    \label{fig:jpeg-stockimg-hw}
  \end{subfigure}
  %
  \begin{subfigure}{0.32\textwidth}
    \includegraphics[width=4.5cm]{figs/jpeg_random_overhead_hw.pdf}
    \caption{\randomjpeg}
    \label{fig:jpeg-randomimg-hw}
  \end{subfigure}
  
  \caption{
    %
    Performance of image rendering with libjpeg sandboxed with
    \trsegmentsfi and \trnacl and \tridealheavy.
    %
    Times are relative to unsandboxed code.
    %
    \trnacl and \tridealheavy relative overheads are as high as 312\% and 208\% 
    respectively, while \trsegmentsfi relative overheads do not exceed 24\%.  }
  \label{fig:jpeg-img-hw}
  
\end{figure*}

%%%%%%%%%%%%%%%%%%%%%%%%%%%%%%%%%%%%%%%%%%%%%%

In this section, we measure the performance overhead of \trsegmentsfi with
zero-cost transitions.
%
We compare \trsegmentsfi with NaCl (\trnacl) and \tridealheavy --- a
hypothetical SFI scheme with no isolation enforcement overhead, both of which
rely on heavyweight transitions.
%
We measure the overhead of these systems on the standard \SPECOhSix benchmark
suite, and the \libgraphite and \libjpeg font and image rendering benchmarks.
%
Since both \trsegmentsfi and \trnacl use segmentation which is supported only 
in 32-bit mode, we implement these three isolation builds in 32-bit mode and 
compare it to native 32-bit unsandboxed code.
%
We describe these benchmarks next.

\para{SPEC}
%
We report the impact of sandboxing on \SPECOhSix in \figref{fig:spec}.
%
Several benchmarks are not compatible with \trnacl; augmenting \trnacl runtime
and libraries to fix such compatibility issues (e.g., as done
in~\cite{yee_native_2009} for SPEC2000) is beyond the scope of this paper.
%
The \texttt{gcc} benchmark, on the other hand, is not compatible with
\trsegmentsfi ---
\texttt{gcc} fails (at runtime) because the CFI used by
\trsegmentsfi\dash---Clang's CFI\dash---incorrectly computes a target CFI label.
%
Clang's CFI implementation is more precise than necessary for our zero-cost
conditions; as with \trnacl, we leave the implementation of a coarse-grain and
more permissive CFI to future work.
%
On the overlapping benchmarks, \trsegmentsfi's overhead is comparable to
\trnacl's.

\begin{figure*}
\vspace{-1em}
\footnotesize
\begin{center}
\begin{adjustbox}{width=1\textwidth}
\begin{tabular}{c|c|c|c|c|c|c|c|c|c|c|c|c}
  \toprule
  \textbf{\texttt{System}}
  & \texttt{400.perlbench}
  & \texttt{401.bzip2}
  & \texttt{403.gcc}
  & \texttt{429.mcf}
  & \texttt{445.gobmk}
  & \texttt{456.hmmer}
  & \texttt{458.sjeng}
  & \texttt{462.libquantum}
  & \texttt{464.h264ref}
  & \texttt{471.omnetpp}
  & \texttt{473.astar}
  & \texttt{483.xalancbmk}
\\
\toprule
\textbf{\trnacl} & --- & --- & 1.10$\times$ & --- & 1.27$\times$ & 0.97$\times$ & 1.20$\times$ & 1.06$\times$ & 1.34$\times$ & 1.06$\times$ & 1.31$\times$ & --- \\
\textbf{\trsegmentsfi} & 1.20$\times$ & 1.08$\times$ & --- & 1.04$\times$ & 1.25$\times$ & 0.82$\times$ & 1.16$\times$ & 1.02$\times$ & 1.01$\times$ & 1.01$\times$ & 1.10$\times$ & 1.05$\times$ \\
\bottomrule
\end{tabular}
\end{adjustbox}
\end{center}

\caption{Overheads compared to native code on \SPECOhSix (nc), for \trnacl and
\trsegmentsfi.}
\label{fig:spec}
\end{figure*}

\para{Font rendering}
%
The impact of these isolation schemes on font rendering is
shown below:

\begin{center}
\footnotesize

\begin{tabular}{p{1.65cm}|cccc}
      \toprule
    & \textbf{\texttt{\trnative (32-bit)}}
    & \textbf{\tridealheavy}
    & \textbf{\trnacl}
    & \textbf{\trsegmentsfi}
    \\
\toprule
    \textbf{Font render}
    & 1441ms & 2399ms & 2769ms & 1765ms \\
\bottomrule
\end{tabular}
\end{center}

\noindent
We observe that \trnacl and \tridealheavy impose an overhead of 
\ffMaxFontOverheadNaClNative and \ffMaxFontOverheadIdealNative 
respectively.
%
In contrast, \trsegmentsfi has a smaller overhead
(\ffMaxFontOverheadSegzeroNative) as it does not have to save and restore
registers or switch stacks.
%
We attribute the overhead of \trsegmentsfi (over Vanilla) to three
factors: (1) changing segments to enable/disable isolation 
during function calls, (2) using indirect function calls for cross-domain calls
(a choice that simplifies engineering but is not fundamental), and (3)
the structure imposed by our zero-cost condition enforcement.

\para{Image rendering}
%
We report the impact of sandboxing on image rendering in
\figref{fig:jpeg-img-hw}.
%
For narrow images (10 pixel width), \trsegmentsfi overheads relative to the
native unsandboxed code are \ffMaxImgSimpleOverheadSegzeroNative, 
\ffMaxImgStockOverheadSegzeroNative, and \ffMaxImgRandomOverheadSegzeroNative 
for \simplejpeg,
\stockjpeg and \randomjpeg, respectively.
%
These overheads are lower than the corresponding overheads for \trnacl
(\ffMaxImgSimpleOverheadNaClNative, 
\ffMaxImgStockOverheadNaClNative, and 
\ffMaxImgRandomOverheadNaClNative) as well as \tridealheavy 
(\ffMaxImgSimpleOverheadIdealNative, 
\ffMaxImgStockOverheadIdealNative and 
\ffMaxImgRandomOverheadIdealNative).
%
As in the Wasm measurements, these overheads shrink as image width increases
and the complexity of the image increases (e.g., the overheads for images wider
than 480 pixels are negligible).


\subsection{Effectiveness of the \verifname verifier}
\label{subsec:verifier-eval}
%
We evaluate \verifname's effectiveness by using it to (1) verify 13
binaries\dash---five third-party libraries shipping (or about to ship) with
Firefox compiled across 3 binaries, and 10 binaries from the \SPECOhSix
benchmarks\dash---and (2) find nine manually introduced bugs, inspired by real
calling convention bugs in previous SFI toolchains~\cite{cranelift-bug-1177,
nacl-bug-775, nacl-bug-2919}.
%
We measure \verifname's performance verifying the aforementioned 13 binaries.
%
Finally, we stress test \verifname by running it on random binaries generated
by Csmith~\cite{csmith}.


\para{Experimental setup}
%
We run all \verifname{} experiments on a 2.1GHz \Intel Xeon® Platinum 8160
machine with 96 cores and 1 TB of RAM running Arch Linux 5.11.12.
%
All experiments run on a single core and use no more than 2GB of RAM.
%
We compile the SPEC binaries used using the Lucet toolkit used in
\sectionref{subsec:eval-wasm}.
%
We verify three of the Firefox libraries from Firefox Nightly;  we compile the
other two from the patches that are in the process of being integrated into
Firefox.

\para{Effectiveness and performance results}
%
\verifname{} successfully verifies the 13 Firefox and \SPECOhSix binaries.
%
These binaries vary in size from 150 functions (the \texttt{lbm} benchmark from
\SPECOhSix) to 4094 functions (the binary consisting of the Firefox Nightly
libraries \libogg, \libgraphite, and \hunspell).
%
It took \verifname between 1.77 seconds and 19.28 seconds to verify these
binaries, with an average of 9.2 seconds and median of 5.93 seconds.
%
\verifname{}'s performance is on par with the original VeriWasm's performance:
on the 10 \SPECOhSix benchmarks evaluated in the VeriWasm
paper~\cite{veriwasm} \verifname{} is slightly (15\%) faster, despite
checking zero-cost conditions in addition to all of VeriWasm's original checks.
%
This is due to various engineering improvements that were made to VeriWasm in
the course of developing \verifname{}.

\verifname also successfully found bugs injected into nine binaries.
%
These bugs tested all the zero-cost properties that \verifname{} was
designed to check, and when possible they were based on real bugs (like those in
Cranelift~\cite{cranelift-bug-1177}).
%
\verifname{} successfully detected all nine of these bugs, giving us confidence
that \verifname{} is capable of finding violations of the zero-cost conditions.


\para{Fuzzing results}
%
We fuzzed \verifname{} to both search for potential bugs in Lucet, as well as
to ensure \verifname{} does not incorrectly declare safe programs unsafe.
%
The fuzzing pipeline works in four stages: first, we use Csmith~\cite{csmith} to
generate random C and \C++ programs, next we use Clang to compile the generated
C/\C++ program to WebAssembly, followed by compiling the Wasm file to native code
using Lucet, and finally we verify the generated binary with \verifname{}.
%
As these programs were compiled by Lucet, we expect them to adhere to the
zero-cost conditions, and any binaries flagged by \verifname{} are either bugs
in Lucet or are spurious errors in \verifname{}.

While we did not find bugs in Lucet, fuzzing did find cases where
\verifname{} triggered spurious errors.
%
After fixing these errors, we verified 100,000 randomly generated programs
with no false positives.


