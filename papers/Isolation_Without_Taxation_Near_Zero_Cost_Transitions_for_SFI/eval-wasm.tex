\subsection{End-to-end performance improvements of zero-cost transitions for Wasm}
\label{subsec:eval-wasm}

%%%%%%%%%%%%%%%%%%%%%%%%%%%%%%%%%%%%%%%%%%%%%%

\begin{figure*}
  
  \begin{subfigure}{0.32\textwidth}
    \includegraphics[width=4.5cm]{figs/jpeg_simple_overhead_wasm.pdf}
    \caption{\simplejpeg}
    \label{fig:jpeg-simpleimg}
  \end{subfigure}
  %
  \begin{subfigure}{0.32\textwidth}
    \includegraphics[width=4.5cm]{figs/jpeg_stock_overhead_wasm.pdf}
    \caption{\stockjpeg}
    \label{fig:jpeg-stockimg}
  \end{subfigure}
  %
  \begin{subfigure}{0.32\textwidth}
    \includegraphics[width=4.5cm]{figs/jpeg_random_overhead_wasm.pdf}
    \caption{\randomjpeg}
    \label{fig:jpeg-randomimg}
  \end{subfigure}
  
  \caption{
    %
    Performance of different Wasm transitions on rendering of (a)~a simple
    image with one color, (b)~a stock image, and (c)~a complex image with
    random pixels, normalized to \trfast.
    %
    \trfast transitions outperform other transitions. The difference
    diminishes with width, but narrower images are more common on
    the web.
  }
  \label{fig:jpeg-img}
\end{figure*}

%%%%%%%%%%%%%%%%%%%%%%%%%%%%%%%%%%%%%%%%%%%%%%

We evaluate the end-to-end performance impact of the different transition
models on Wasm-sandboxed font and image rendering as used in Firefox (see
\secref{sec:eval}).

\para{Font rendering}
We report the performance of \libgraphite isolated with Wasm-based schemes on
Kew's benchmark below:
 
\begin{center}
\footnotesize

\begin{tabular}{p{1.65cm}|p{1.3cm}p{1.4cm}p{1cm}p{1.4cm}p{1cm}p{1.6cm}}
      \toprule
    & \textbf{\trlucet}
    & \textbf{\trfullswitch}
    & \textbf{\trregsave}
    & \textbf{\trfast}
    & \textbf{\trnative}
    & \textbf{\tridealheavysixfour}
    \\
\toprule
    \textbf{Font render}
    & 8173ms & 2246ms & 2230ms & 2032ms & 1116ms & 1563ms  \\
\bottomrule
\end{tabular}
\end{center}

\noindent
As expected, Wasm with zero-cost transitions (\trfast) outperforms the
other Wasm-based SFI transition models.
%
Compared to \trfast, Lucet's existing transitions slow down rendering 
by over \ffMaxFontSlowdownWasmLucetZero.\footnote{
  This overhead is smaller than the 8$\times$ overhead reported by
  \citet{rlbox}; we attribute this difference to the different compilers\dash---we
  use a more recent, and faster, version of Lucet.
}
%
But, even the optimized heavyweight transitions (\trfullswitch) impose a
\ffMaxFontSlowdownWasmHeavyZero performance tax.
%
This overhead is due to register saving/restoring; stack switching
only accounts for \ffMaxFontStackSwitchWasmOverhead overhead.

While these results show that existing Wasm-based isolation schemes can benefit
from switching to zero-cost transitions\dash---and indeed the speed-up due to
zero-cost transitions allowed Mozilla to ship the Wasm-sandboxed
\libgraphite\dash---they also show that Lucet-compiled Wasm is slow
($\sim$80$\%$ slower than Vanilla).
%
This, unfortunately, means that the transition cost savings alone are not
enough to beat \tridealheavysixfour, even for a workload with many transitions.
%
To compete with this ideal SFI scheme with heavyweight transitions, we would
need to reduce the runtime overhead to $\sim$40$\%$.
%
\citet{not-so-fast} report the average runtime overhead of Mozilla SpiderMonkey JIT-compiled WebAssembly compared to native as $\sim$45$\%$ in a different set of benchmarks, while noting many correctable inefficiencies in the generated assembly code, suggesting that there is a lot of room for Lucet to be further optimised.

\para{Image rendering}
%
\figref{fig:jpeg-img} report the overhead of Wasm-based sandboxing on
image rendering, normalized to \trfast to highlight the relative overheads
of different transitions as compared to our zero-cost transitions.
%
We report results of \trlucet separately, in \iftechreport{\appref{appendix:img}
(\figref{fig:jpeg-img-lucet})}{the technical
appendix~\cite{kolosick2021isolation}} because the rendering times are up to
\ffMaxImgSlowdownWasmLucetZero longer than the other builds.
%
Here, we instead focus on evaluating the overheads of optimized
heavy transitions.

As expected, \trfast significantly outperforms other transitions when images 
are narrower and simpler.
%
On \simplejpeg, \trfullswitch and \trlucet can take as much as 
\ffMaxImgSimpleSlowdownWasmHeavyZero and \ffMaxImgSimpleSlowdownWasmLucetZero
longer to render the image as with \trfast transitions.
%
However, this performance gap diminishes as image width increases (and the
relative cost of context switching decreases).
%
For \stockjpeg and \randomjpeg, the \trfullswitch trends are similar, but
the rendering time differences start at about 
\ffMaxImgStockRandomSlowdownWasmHeavyZero.
%
Lucet's existing transitions (\trlucet) are still significantly slower 
than zero-cost transitions (\trfast) even on wide images.


%%%%%%%%%%%%%%%%%%%%%%%%%%%%%%%%%%%%%%%%%%%%%%

\begin{figure}
  \includegraphics[width=6.5cm]{figs/image_sizes.pdf}
  \caption{
    Cumulative distribution of image widths on the landing pages of the Alexa 
    top 500 websites.
    %
    Over 80\% of the images have widths under 480 pixels.
    %
    Narrower images have a higher transition rate, and thus higher relative
    overheads when using expensive transitions.
  }
  \label{fig:image-sizes}
\end{figure}

%%%%%%%%%%%%%%%%%%%%%%%%%%%%%%%%%%%%%%%%%%%%%%

Though the differences between the transitions are smaller as the image width
increases, many images on the Web are narrow.
\figref{fig:image-sizes} shows the distribution of images on the landing
pages of the Alexa top 500 websites. Of the 10.6K images, 8.6K (over 80\%) have
widths between 1 and 480 pixels, a range in which zero-cost transitions
noticeably outperform the other kinds of transitions.

Like font rendering, we measure the target runtime overhead Lucet should
achieve to beat \tridealheavysixfour end-to-end for rendering images.
%
We report our results in \iftechreport{\figref{fig:jpeg-img-ideal64} in
\appref{appendix:img}}{the technical appendix~\cite{kolosick2021isolation}}.
%
For the small simple image, we observe this to be 94\%\dash---this is approximately the overhead of Lucet that we see already today.
%
For the small stock image, we observe this to be 15\%\dash---this is much smaller than the overhead of Lucet today, but lower overheads have been demonstrated on some benchmarks by the prototype Wasm compiler of~\citet{sledge}.
