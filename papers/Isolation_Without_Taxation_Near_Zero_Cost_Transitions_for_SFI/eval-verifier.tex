\subsection{Effectiveness of the \verifname verifier}
\label{subsec:verifier-eval}
%
We evaluate \verifname's effectiveness by using it to (1) verify 13
binaries\dash---five third-party libraries shipping (or about to ship) with
Firefox compiled across 3 binaries, and 10 binaries from the \SPECOhSix
benchmarks\dash---and (2) find nine manually introduced bugs, inspired by real
calling convention bugs in previous SFI toolchains~\cite{cranelift-bug-1177,
nacl-bug-775, nacl-bug-2919}.
%
We measure \verifname's performance verifying the aforementioned 13 binaries.
%
Finally, we stress test \verifname by running it on random binaries generated
by Csmith~\cite{csmith}.


\para{Experimental setup}
%
We run all \verifname{} experiments on a 2.1GHz \Intel Xeon® Platinum 8160
machine with 96 cores and 1 TB of RAM running Arch Linux 5.11.12.
%
All experiments run on a single core and use no more than 2GB of RAM.
%
We compile the SPEC binaries used using the Lucet toolkit used in
\sectionref{subsec:eval-wasm}.
%
We verify three of the Firefox libraries from Firefox Nightly;  we compile the
other two from the patches that are in the process of being integrated into
Firefox.

\para{Effectiveness and performance results}
%
\verifname{} successfully verifies the 13 Firefox and \SPECOhSix binaries.
%
These binaries vary in size from 150 functions (the \texttt{lbm} benchmark from
\SPECOhSix) to 4094 functions (the binary consisting of the Firefox Nightly
libraries \libogg, \libgraphite, and \hunspell).
%
It took \verifname between 1.77 seconds and 19.28 seconds to verify these
binaries, with an average of 9.2 seconds and median of 5.93 seconds.
%
\verifname{}'s performance is on par with the original VeriWasm's performance:
on the 10 \SPECOhSix benchmarks evaluated in the VeriWasm
paper~\cite{veriwasm} \verifname{} is slightly (15\%) faster, despite
checking zero-cost conditions in addition to all of VeriWasm's original checks.
%
This is due to various engineering improvements that were made to VeriWasm in
the course of developing \verifname{}.

\verifname also successfully found bugs injected into nine binaries.
%
These bugs tested all the zero-cost properties that \verifname{} was
designed to check, and when possible they were based on real bugs (like those in
Cranelift~\cite{cranelift-bug-1177}).
%
\verifname{} successfully detected all nine of these bugs, giving us confidence
that \verifname{} is capable of finding violations of the zero-cost conditions.


\para{Fuzzing results}
%
We fuzzed \verifname{} to both search for potential bugs in Lucet, as well as
to ensure \verifname{} does not incorrectly declare safe programs unsafe.
%
The fuzzing pipeline works in four stages: first, we use Csmith~\cite{csmith} to
generate random C and \C++ programs, next we use Clang to compile the generated
C/\C++ program to WebAssembly, followed by compiling the Wasm file to native code
using Lucet, and finally we verify the generated binary with \verifname{}.
%
As these programs were compiled by Lucet, we expect them to adhere to the
zero-cost conditions, and any binaries flagged by \verifname{} are either bugs
in Lucet or are spurious errors in \verifname{}.

While we did not find bugs in Lucet, fuzzing did find cases where
\verifname{} triggered spurious errors.
%
After fixing these errors, we verified 100,000 randomly generated programs
with no false positives.

