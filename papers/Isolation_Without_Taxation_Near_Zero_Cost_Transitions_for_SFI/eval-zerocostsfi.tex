
\subsection{Performance overhead of purpose-built zero-cost SFI enforcement}

\label{subsec:eval-zerocostsfi}
%%%%%%%%%%%%%%%%%%%%%%%%%%%%%%%%%%%%%%%%%%%%%%

\begin{figure*}
  
  \begin{subfigure}{0.32\textwidth}
    \includegraphics[width=4.5cm]{figs/jpeg_simple_overhead_hw.pdf}
    \caption{\simplejpeg}
    \label{fig:jpeg-simpleimg-hw}
  \end{subfigure}
  %
  \begin{subfigure}{0.32\textwidth}
    \includegraphics[width=4.5cm]{figs/jpeg_stock_overhead_hw.pdf}
    \caption{\stockjpeg}
    \label{fig:jpeg-stockimg-hw}
  \end{subfigure}
  %
  \begin{subfigure}{0.32\textwidth}
    \includegraphics[width=4.5cm]{figs/jpeg_random_overhead_hw.pdf}
    \caption{\randomjpeg}
    \label{fig:jpeg-randomimg-hw}
  \end{subfigure}
  
  \caption{
    %
    Performance of image rendering with libjpeg sandboxed with
    \trsegmentsfi and \trnacl and \tridealheavy.
    %
    Times are relative to unsandboxed code.
    %
    \trnacl and \tridealheavy relative overheads are as high as 312\% and 208\% 
    respectively, while \trsegmentsfi relative overheads do not exceed 24\%.  }
  \label{fig:jpeg-img-hw}
  
\end{figure*}

%%%%%%%%%%%%%%%%%%%%%%%%%%%%%%%%%%%%%%%%%%%%%%

In this section, we measure the performance overhead of \trsegmentsfi with
zero-cost transitions.
%
We compare \trsegmentsfi with NaCl (\trnacl) and \tridealheavy --- a
hypothetical SFI scheme with no isolation enforcement overhead, both of which
rely on heavyweight transitions.
%
We measure the overhead of these systems on the standard \SPECOhSix benchmark
suite, and the \libgraphite and \libjpeg font and image rendering benchmarks.
%
Since both \trsegmentsfi and \trnacl use segmentation which is supported only 
in 32-bit mode, we implement these three isolation builds in 32-bit mode and 
compare it to native 32-bit unsandboxed code.
%
We describe these benchmarks next.

\para{SPEC}
%
We report the impact of sandboxing on \SPECOhSix in \figref{fig:spec}.
%
Several benchmarks are not compatible with \trnacl; augmenting \trnacl runtime
and libraries to fix such compatibility issues (e.g., as done
in~\cite{yee_native_2009} for SPEC2000) is beyond the scope of this paper.
%
The \texttt{gcc} benchmark, on the other hand, is not compatible with
\trsegmentsfi ---
\texttt{gcc} fails (at runtime) because the CFI used by
\trsegmentsfi\dash---Clang's CFI\dash---incorrectly computes a target CFI label.
%
Clang's CFI implementation is more precise than necessary for our zero-cost
conditions; as with \trnacl, we leave the implementation of a coarse-grain and
more permissive CFI to future work.
%
On the overlapping benchmarks, \trsegmentsfi's overhead is comparable to
\trnacl's.

\begin{figure*}
\vspace{-1em}
\footnotesize
\begin{center}
\begin{adjustbox}{width=1\textwidth}
\begin{tabular}{c|c|c|c|c|c|c|c|c|c|c|c|c}
  \toprule
  \textbf{\texttt{System}}
  & \texttt{400.perlbench}
  & \texttt{401.bzip2}
  & \texttt{403.gcc}
  & \texttt{429.mcf}
  & \texttt{445.gobmk}
  & \texttt{456.hmmer}
  & \texttt{458.sjeng}
  & \texttt{462.libquantum}
  & \texttt{464.h264ref}
  & \texttt{471.omnetpp}
  & \texttt{473.astar}
  & \texttt{483.xalancbmk}
\\
\toprule
\textbf{\trnacl} & --- & --- & 1.10$\times$ & --- & 1.27$\times$ & 0.97$\times$ & 1.20$\times$ & 1.06$\times$ & 1.34$\times$ & 1.06$\times$ & 1.31$\times$ & --- \\
\textbf{\trsegmentsfi} & 1.20$\times$ & 1.08$\times$ & --- & 1.04$\times$ & 1.25$\times$ & 0.82$\times$ & 1.16$\times$ & 1.02$\times$ & 1.01$\times$ & 1.01$\times$ & 1.10$\times$ & 1.05$\times$ \\
\bottomrule
\end{tabular}
\end{adjustbox}
\end{center}

\caption{Overheads compared to native code on \SPECOhSix (nc), for \trnacl and
\trsegmentsfi.}
\label{fig:spec}
\end{figure*}

\para{Font rendering}
%
The impact of these isolation schemes on font rendering is
shown below:

\begin{center}
\footnotesize

\begin{tabular}{p{1.65cm}|cccc}
      \toprule
    & \textbf{\texttt{\trnative (32-bit)}}
    & \textbf{\tridealheavy}
    & \textbf{\trnacl}
    & \textbf{\trsegmentsfi}
    \\
\toprule
    \textbf{Font render}
    & 1441ms & 2399ms & 2769ms & 1765ms \\
\bottomrule
\end{tabular}
\end{center}

\noindent
We observe that \trnacl and \tridealheavy impose an overhead of 
\ffMaxFontOverheadNaClNative and \ffMaxFontOverheadIdealNative 
respectively.
%
In contrast, \trsegmentsfi has a smaller overhead
(\ffMaxFontOverheadSegzeroNative) as it does not have to save and restore
registers or switch stacks.
%
We attribute the overhead of \trsegmentsfi (over Vanilla) to three
factors: (1) changing segments to enable/disable isolation 
during function calls, (2) using indirect function calls for cross-domain calls
(a choice that simplifies engineering but is not fundamental), and (3)
the structure imposed by our zero-cost condition enforcement.

\para{Image rendering}
%
We report the impact of sandboxing on image rendering in
\figref{fig:jpeg-img-hw}.
%
For narrow images (10 pixel width), \trsegmentsfi overheads relative to the
native unsandboxed code are \ffMaxImgSimpleOverheadSegzeroNative, 
\ffMaxImgStockOverheadSegzeroNative, and \ffMaxImgRandomOverheadSegzeroNative 
for \simplejpeg,
\stockjpeg and \randomjpeg, respectively.
%
These overheads are lower than the corresponding overheads for \trnacl
(\ffMaxImgSimpleOverheadNaClNative, 
\ffMaxImgStockOverheadNaClNative, and 
\ffMaxImgRandomOverheadNaClNative) as well as \tridealheavy 
(\ffMaxImgSimpleOverheadIdealNative, 
\ffMaxImgStockOverheadIdealNative and 
\ffMaxImgRandomOverheadIdealNative).
%
As in the Wasm measurements, these overheads shrink as image width increases
and the complexity of the image increases (e.g., the overheads for images wider
than 480 pixels are negligible).

