\subsection{Logical Relation}
\label{appendix:lr}

\begin{center}
  \begin{tabular}{>{$}r<{$} >{$}c<{$} >{$}l<{$}}
    L & \bnfdef &
      \begin{array}[t]{lllll}
        \{
        & \mathit{interface} & \bnftypes & \nats \rightharpoonup \nats \\
        & \mathit{library} & \bnftypes & \nats \rightharpoonup \mathit{WasmFunction} \\
        & \mathit{code} & \bnftypes & \codes & \}
      \end{array}
  \end{tabular}
\end{center}

\[\begin{array}{rcl}
  \mathrm{World} & \triangleq & \nats \times (\nats \rightharpoonup \nats) \times (\nats \rightharpoonup \nats) \\
  \\
  \later & : & \mathrm{World} \rightarrow \mathrm{World} \\
  \later (0, \overline{f_i}, \overline{f_l}) & \triangleq & (0, \overline{f_i}, \overline{f_l}) \\
  \later (n, \overline{f_i}, \overline{f_l}) & \triangleq & (n - 1, \overline{f_i}, \overline{f_l}) \\
  \\
  (C, \overline{\oc{F_2}}, \overline{\oc{F_3}}) :_W & \triangleq & \text{for } i \in \{1, 2\}: \\
  & &
  \begin{array}[t]{l}
    \overline{\oc{F_i}.\mathit{entry}} = \dom{\pi_i(W)} \\
    \forall \oc{F_i} \in \overline{\oc{F_i}}.~ \oc{F_i}.\mathit{type} = \pi_i(W)(\oc{F_i}.\mathit{entry}) \\
    \forall \oc{F_i} \in \overline{\oc{F_i}}.~ (\later W, \oc{F_i}, C|_{\oc{F_i}.\mathit{instrs}}) \in \Frel
  \end{array}
\end{array}\]

\begin{center}
  \[\begin{array}{rcl}
    \Frel & \triangleq &
    \left\{
      (W, \oc{F}, c)
      \middle|
      \begin{array}{l}
        \forall \rho \in [\frames], \mathit{ret\mbox{-}addr}, A \in [\nats], sp, R, M, C, \overline{\oc{F_i}}, \overline{\oc{F_l}}. \\
        \quad \text{let } \rho' = \rho \concat \{\mathit{base} \assign sp - |A|, \mathit{ret\mbox{-}addr\mbox{-}loc} \assign sp, \mathit{csr\mbox{-}vals} \assign R(\mathbb{CSR})\} \\
        \quad |A| = \oc{F}.\mathit{type} \\
        \quad sp > \oname{top}(\rho).\mathit{ret\mbox{-}addr\mbox{-}loc} + |A| \\
        \quad [sp \mapsto \mathit{ret\mbox{-}addr}, (sp - |A| + i \mapsto A_i)_{i \in [0, |A|)}] \leq M  \\
        \quad (C, \overline{\oc{F_i}}, \overline{\oc{F_l}}) :_W \\
        \quad c \leq C \\
        \quad \dom{c} = \oc{F}.\mathit{instrs} \\
        \quad \forall \ell \in \dom{M} \cap M_{\untrusted}.~ \val{n}{\untrusted} = M(\ell) \\
        \quad \text{let } \Psi = \{pc \assign \oc{F}.\mathit{entry}, sp \assign sp, R \assign R, M \assign M, C \assign C \} \\
        \quad \text{let } \oPhi = \{\Psi \assign \Psi, \mathit{stack} \assign \rho', \mathit{funcs} \assign [\oc{F'}.\mathit{entry} \mapsto \oc{F'}]_{\oc{F'} \in \overline{\oc{F_i}} \uplus \overline{\oc{F_l}}}\} \\

        \Longrightarrow \\
        \quad \forall n' \leq W.n.~ \oPhi \osteprhon{\rho'}{n'} \oc{\Phi'} \Longrightarrow \oc{\Phi'} \neq \oerror
        \\
        \quad \text{or } \exists n' \leq W.n.~ \oPhi \osteprhon{\rho'}{n' - 1} \oc{\Phi'} \ostep \oc{\Phi''}
        \\
        \qquad \text{where } (\currentop{\oc{\Phi'}}{\cret{}} \vee \currentop{\oc{\Phi'}}{\cgateret}) \wedge \oc{\Phi'}.\mathit{stack} = \rho' \wedge \oc{\Phi''} \neq \oerror
      \end{array}
    \right\} \\
    \\
    \Lrel & \triangleq &
    \left\{
      (n, L)
      \middle|~
      \begin{array}{l}
        \forall i \in \dom{L.\mathit{library}}. \\
        \quad \text{let } \textit{WF} = L.\mathit{library}(i) \\
        \quad \text{let } W = (n, L.\mathit{interface}, \lambda i \rightarrow L.\mathit{library}(i).|args|) \\
        \quad \text{let } \mathit{instrs} = \biguplus_{B \in \textit{WF}.\mathit{blocks}} [B.start, B.end] \\
        \quad \text{let } \oc{F} = \{ \mathit{instrs} \assign \mathit{instrs}, \mathit{entry} \assign \textit{WF}.\mathit{entry}.\mathit{start}, \mathit{type} \assign \textit{WF}.|args|\} \\
        \quad (W, \oc{F}, L.\mathit{code}|_{\mathit{instrs}}) \in \Frel
      \end{array}
    \right\}
  \end{array}\]
  \captionof{figure}{Function and Library Relations}
\end{center}


\begin{lemma}[FTLR for functions] \label{appendix:wasm:lr:ftlr-functions}
  Let $W \in \mathrm{World}$ and $c$ be the code for a compiled WebAssembly function
  $\textit{WF}$ such that $\textit{WF}$ expects application functions in the
  interface with locations and types $\pi_2(W)$ and in the library with locations
  and types $\pi_3(W)$.
  %
  Further let $\mathit{instrs} = \biguplus_{B \in \textit{WF}.\mathit{blocks}}
  [B.start, B.end]$ and $\oc{F} = \{ \mathit{instrs} \assign \mathit{instrs},
  \mathit{entry} \assign \textit{WF}.\mathit{entry}.\mathit{start}, \mathit{type}
  \assign \textit{WF}.|args|\}$.
  %
  Then $(W, \oc{F}, c) \in \Frel$.
\end{lemma}
\begin{proof}
  We first unroll the assumptions of $(W, \oc{F}, c) \in \Frel$ reusing the
  variable names defined there.
  %
  We will maintain that any steps do not step to $\oerror$ so WOLOG we will
  continually assume $n' \leq W.n$ such that $n'$ greater than the number of steps
  we have taken, otherwise the case $\oc{\Phi} \osteprhon{\rho'}{n'} \oc{\Phi'}
  \Longrightarrow \oc{\Phi'} \neq \oerror$ holds.

  By the structure of a compiled WebAssembly function and assumption we have
  that the stack and stack pointer represent $\textit{WF}.|args|$ arguments.
  %
  The abstract interpretation ensures that if we write to $H_{\untrusted}$ then
  that value has label $\untrusted$ so the checks pass.
  %
  We are further assured that we do not read or below the stack frame.
  %
  The structure of a compiled WebAssembly block then lets us proceed until we
  reach one of
  %
  \begin{enumerate*}
  \item a function call to a library function $\textit{WF}'$ such that $\textit{WF}'.\mathit{entry}.\mathit{start} \in \overline{F_l.\mathit{entry}}$,
  \item an application function $\oc{F'}$ such that $\oc{F'}.\mathit{entry} \in \overline{F_i.\mathit{entry}}$,
  \item or the end of the block.
  \end{enumerate*}

  \begin{enumerate}
  \item

    The abstract interpretation ensures that we have initialized the arguments
    $\textit{WF}'.|args| = \pi_3(W)(\textit{WF}'.\mathit{entry}.\mathit{start})$ or
    failed a dynamic type check and terminated (thus stepping to a terminal state
    that is not an $\oerror$).
    %
    We thus set $\rho_2 = \rho'$ and see that we have constructed $\rho_2' =
    \rho_2 \concat \{ \mathit{base} \assign sp - \textit{WF}'.|args|,
    \mathit{ret\mbox{-}addr\mbox{-}loc} \assign sp, \mathit{csr\mbox{-}vals} \assign
    R(\mathbb{CSR}) \}$.
    %
    We further set $\mathit{ret\mbox{-}addr} = pc + 1$, $A = \textit{WF}'.|args|$,
    $sp = sp$, $R = R$, $M = M$, $C = C$, $\overline{\oc{F_i}} = \overline{\oc{F_i}}$,
    and $\overline{\oc{F_l}} = \overline{\oc{F_l}}$.
    %
    By the abstract interpreation we have that all of the remaining checks in
    $\Frel$ pass and that the instantiated $\oc{\Phi}$ is equal to our current
    state.
    %
    We therefore instantiate $(\later \oc{F_l}, C|_{\oc{F_l}.\mathit{instrs}}) \in \Frel$.
    %
    If this uses the remaining steps then we are done.
    %
    Otherwise we get that we return to $pc + 1$ with all values restored and no
    new $\trusted$ values written to the library memory, and our walk through the
    block may continue.

  \item

    Identical to the case for (1).

  \item

    The end of a block is followed by a direct jump to another block $B'$, an
    indirect block $\mathit{IB}$, or we are at an exit block.
    %
    In the case of another block $B'$ we have by the structure of compiled
    WebAssembly code that we have instantiated $B'.inputs$.
    %
    We thus jump to the block and follow the same proof structure as detailed here.
    %
    The same is true of an intermediate block $\mathit{IB}$ except with the
    extra steps of setting up the inputs jumping to another block $B''$.
    %
    Lastly if we have reached the end of an exit block then we have not touched
    the pushed return address or callee-save registers and the stack pointer is in
    the expected location.
    %
    We thus execute $\cret{}$ or $\cgateret$ and pass the overlay monitor checks.
  \end{enumerate}
\end{proof}

\begin{lemma}[FTLR for libraries] \label{thm:appendix:wasm:lr:ftlr-libraries}
  For any number of steps $n \in \nats$ and compiled WebAssembly library $L$,
  $(n, L) \in \Lrel$.
\end{lemma}
\begin{proof}
  By unrolling the definition of $\Lrel$ and \lemref{appendix:wasm:lr:ftlr-functions}.
\end{proof}

\begin{theorem}[Adequacy of $\Lrel$] \label{thm:appendix:lr:adequacy}
  For any number of steps $n \in \nats$, library $L$ such that $(n, L) \in
  \Lrel$, program $\oc{\Phi_0} \in \programs$ using $L$, and $n' \leq n$, if
  $\oc{\Phi_0} \ostepn{n'} \oc{\Phi'}$ then $\oc{\Phi'} \neq \oerror$.
\end{theorem}
\begin{proof}
  Straightforward: by assumption for steps in the application, and by assumption about
  application code properly calling the library code and the unrolling of $\Lrel$ and
  $\Frel$.
\end{proof}