\section{Related work}
\label{sec:related}

%
A considerable amount of research has gone into efficient implementations of
memory isolation and CFI techniques to provide SFI across many
platforms~\cite{sehr_adapting_2010,mccamant_evaluating_2006,
goonasekera_libvm_2015, lucet, omniware, omniware-pldi, vino, herder2009fault,
rockjit, robusta, vx32, gang-sfi, payer2011fine, wedge, lwc, shreds, bgi}.
%
However, these systems either implement or require the user to implement
heavyweight springboards and trampolines to guarantee security.

\para{SFI systems}
%
\citet{wahbe_efficient_1993} suggest two ways to optimize
transitions: (1) partitioning the registers used by the application and the
sandboxed component and (2) performing link time optimizations (LTO)
that conservatively eliminates register saves that are never used in the entire
sandboxed component (not just the callee).
%
Register partitioning would cause slowdowns due to increased spilling.
%
Native Client~\cite{yee_native_2009} optimized 
transitions by clearing and saving contexts using 
machine specific mechanisms to, e.g., clear floating point 
state and SIMD registers in bulk.
%
However, we show (\secref{sec:eval}) that, even with those optimizations, the
software model imposes significant transition overheads.
%
While CPU makers continue to add optimized context switching
instructions, such instructions do not yet eliminate all
overhead.

\citet{zeng-tan2011} combine an SFI scheme with a rich CFI scheme
enforcing structure on executing code.
%
While a similar approach, their goal is to safely perform optimizations
to elide SFI and CFI bounds checks, and they do not impose sufficient structure to
enforce well-bracketing, a necessary property for zero-cost transitions.
%
XFI~\cite{xfi} also combines an SFI scheme with a rich CFI scheme and adopts a
safe stack model.
%
While meeting many of the zero-cost conditions, it does not prevent reading
uninitialized scratch registers and therefore cannot ensure confidentiality
without heavyweight springboards that clear scratch registers.
%
They also do not specify the CFG granularity, so it is not clear if is strong
enough to satisfy the zero-cost type-safe CFI requirement.

\para{WebAssembly based isolation}
%
WasmBoxC~\cite{wasmboxc} sandboxes C code by compiling to Wasm
followed by (de)compiling back to C, ensuring that the sandboxed code will
inherit isolation properties from Wasm.
%
The sandboxed library code can be safely linked with C applications, enabling a
form of zero-cost transition.
%
The zero-cost Wasm SFI system described by this paper was designed and released
prior to and independently of WasmBoxC, as the creators of WasmBoxC acknowledge
(citation elided for DBR).
%
Moreover, we believe that the theory developed in this paper provides a foundation for
analyzing and proving the security of WasmBoxC though such analysis would need
to account for possible undefined behavior introduced in compiling to C.

Sledge~\cite{sledge} describes a Wasm runtime for edge computing, that
relies on Wasm properties to enable efficient isolation of serverless
components.
%
However, Sledge focuses on function
scheduling including preempting running Wasm programs,
so its needs for context saving differ from library sandboxing as
contexts must be saved even in the middle of function calls.

\para{SFI Verification}
%
Previous work on SFI
(e.g.,~\cite{mccamant_evaluating_2006,yee_native_2009, xfi, veriwasm}) uses a
\emph{verifier} or a theorem prover~\cite{armor, compcert-sfi} to validate the
relevant SFI properties of compiled sandbox code.
%
However, unlike \verifname{}, none of these verifiers establish sufficient
properties for zero-cost transitions.

\para{Hardware based isolation}
%
Hardware features such as memory protection
keys~\cite{vahldiek-oberwagner_erim_2019, hodor}, extended page
tables~\cite{qiang_libsec_2017}, virtualization
instructions~\cite{qiang_libsec_2017, dune}, or even dedicated hardware
designs~\cite{donky} can be used to speed up
memory isolation.
%
These works focus on the efficiency of memory isolation as well as switching
between protected memory domains; however these approaches also use a single memory region that contain both the stack and heap making them incompatible with zero-cost conditions, i.e. they require heavyweight transitions.
%
\tridealheavy and \tridealheavysixfour in \sectionref{sec:eval} studies an idealized version of such a scheme.

\para{Capabilities}
%
\citet{karger} and \citet{skorstengaard_stktokens_2019} look at protecting
interacting components on systems that provide hardware-enforced capabilities.
%
\citet{karger} specifically looks at how register saving and restoration can be
optimized based on different levels of trust between components, however their
analysis does not offer formal security guarantees.
%
\citet{skorstengaard_stktokens_2019} investigate a calling-convention based on
capabilities (\`a la CHERI~\cite{cheri}) that allow safe sharing of a
stack between distrusting components.
%
Their definition of well-bracketed control flow and local state encapsulation
via an overlay inspired our work, and our logical relation is also based
on their work.
%
However, their technique does not yet ensure an equivalent notion to our
confidentiality property, and further is tied to machine support for hardware
capabilities.

\para{Type safety for isolation}
%
There has also been work on using strongly-typed languages to provide similar
security benefits.
%
SingularityOS~\cite{aiken2006deconstructing, hunt2007singularity,
fahndrich2006language}, explored using Sing\# to build an OS with cheap
transitions between mutually untrusting processes.
%
Unlike the work on SFI techniques that zero-cost transitions extend, tools like
SingularityOS require engineering effort to rewrite unsafe components in new
safe languages.

At a lower level, Typed Assembly Language (TAL)~\cite{morrisett_system_1999,
morrisett_stack-based_2002, morrisett_talx86_1999} is a type-safe compilation
target for high-level type-safe languages.
%
Its type system enables proofs that assembly programs follow
calling conventions, and enables an elegant definition of stack safety through
polymorphism.
%
Unfortunately, SFI is designed with unsafe code in mind, so cannot generally be
compiled to meet TAL's static checks.
%
To handle this our zero-cost and security conditions instead capture the \emph{behavior} that
TAL's type system is designed to ensure.