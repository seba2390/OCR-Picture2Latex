% !TEX root =  MinCanRevised.tex

\section{Experiments}\label{sec:ex}

In this section, we will compare how well our new algorithms perform in
comparison with the \textbf{MinImage} function of Linton's.

All of our experiments are performed in GAP~\cite{GAP4} using code available in
the \texttt{Images} package \cite{Images}.
Where our algorithm requires it, we use the implementation of partition
backtracking provided in the \texttt{Ferret} package \cite{Fer}.

We consider a selection of different canonical image algorithms and we analyze how
they perform compared to each other, and compared to the traditional minimal image
algorithm of Linton's, which we will refer to as \textbf{MinImage}.

The first three algorithms that we consider come from Section \ref{sec:alternate
ordering}. They produce, given a group \(G\) on a set \(\Omega\), an ordering of
\(\Omega\). This ordering is then used in \textbf{MinImage}.

\begin{enumerate}
\item \textbf{FixedMinOrbit} uses results from Section \ref{sec:alternate ordering}
  to calculate an alternative ordering of \(\Omega\), choosing small orbits first.
\item \textbf{FixedMaxOrbit} works similarly to \textbf{FixedMinOrbit}, choosing large
  orbits first.
\end{enumerate}

We also consider algorithms that dynamically choose which value to branch on as
search progresses. We will use the following lemma for the proof of correctness
for all our orderings.

\begin{lem}\label{lem:orbcount2}
  Let $\Omega$ be a finite set,  $G$ a permutation group on $\Omega$, and $P$
  an ordered partition of $\powset(\Omega)$. 
  
  Let $L \in \LL_n$, and
  \(\Count = [ \Orbcount(G,S) \mid S \in L ]\).

  \begin{itemize}
  \item Any function that accepts \((\Omega, G, L, P)\) and returns
    some \(\omega \in \Omega\) with $|\omega^G| > 1$ that is invariant
    under reordering the elements of \(\Count\), is an \(\Omega\)-selector.

  \item Any function that accepts \((\Omega, G, L, P)\) and returns
    either \(P\) or the point refinement of \(P\) by some \(\omega^G\) for
    \(\omega \in \Omega\), and is invariant under permutation of the elements
    of \(\Count\), is an ordering refiner.

  \item Any function that accepts \((\Omega, G, L, P)\) and returns either
    \(P\) or the orbit refiner of \(P\) by some member of \(\Count\) and is invariant
    under permutation of the elements of \(\Count\), is an ordering refiner.
\end{itemize}
\end{lem}

\begin{proof}
  The only thing we have to show is that, if $L$ and $M$ are $G$-equivalent, then
  any of the functions above yield the same result for inputs $(\Omega,G,L,P)$
  and $(\Omega,G,M,P)$.
  By Lemma~\ref{lem:orbcount}, any \(G\)-equivalent lists $L$ and $M$ will
  produce the same list \(\Count\), up to reordering of elements, and hence the
  claim follows.
\end{proof}

Firstly we define a list of orderings. Each of these orderings chooses an orbit,
or list of orbits, to branch on -- we will then make an \(\Omega\)-selector by
choosing the smallest element in any of the orbits selected, to break ties (we
could choose any point, as long as we picked it consistently). Each of these
algorithms operates on a list \(L \in \LL_k\). In each case we look for an
orbit, ignoring orbits of size one (as fixing a point that was already fixed
leads to the same group).

Firstly we will consider two algorithms that only consider the group, and not \(L\):

\begin{enumerate}
\item \textbf{MinOrbit} Choose a point from a shortest non-trivial orbit that
  has a non-empty intersection with at least one element of \(L\).
\item \textbf{MaxOrbit} Choose a point from a longest non-trivial orbit that
  has a non-empty intersection with at least one element of $L$.
\end{enumerate}

We also consider four algorithms that consider both the group, and \(L\).

In the following, for an orbit \(o\)
\begin{enumerate}
\item \textbf{RareOrbit} minimises \(\sum\limits_{s \in L}\mid s \cap o|\),
\item \textbf{CommonOrbit} maximises \(\sum\limits_{s \in L}\mid s \cap o|\),
\item \textbf{RareRatioOrbit} minimises \({\log(\sum\limits_{s \in L}\mid s \cap o|)}/{|o|}\),
\item \textbf{CommonRatioOrbit} maximises \({\log(\sum\limits_{s \in L}\mid s \cap o|)}/{|o|}\).
\end{enumerate}

The motivation for \textbf{RareOrbit} is that this is the branch which will lead
to the smallest size of the next level of search -- this exactly estimates the
size of the next level if our ordering refiner only fixed a point in orbit
\(o\). We therefore expect, conversely, \textbf{CommonOrbit} to perform badly and to
produce very large searches.

One limitation of \textbf{RareOrbit} is that it will favour smaller orbits -- in
general we want to minimize the size of the whole search. The idea here is that
if we have two levels of search where we split on an orbit of size two, and each
time create 10 times more sets is equivalent to splitting once on an orbit of
size four, and creating 100 times more sets. \textbf{RareRatioOrbit} compensates
for this. We expect that \textbf{CommonRatioOrbit} is the inverse of this, so we also
expect it to perform badly.

For each of these orderings, we use the ordering refiner that
takes each fixed point of \(G\) in their order in \(\Omega\), and performs a point
refinement by each recursively in turn. By repeated application of
Lemma~\ref{lem:orbcount}, this is a \(G\)-invariant ordering refiner.

We also have a set of orderings which make use of orbit counting. To keep the
number of experiments under control, we used the \textbf{RareOrbit}
strategy in each case to choose which point to branch on next, and we also build an orbit
refiner.

Given an unordered list of orbit counts,
\begin{enumerate}
\item \textbf{RareOrbitPlusMin} chooses the lexicographically smallest one.
\item \textbf{RareOrbitPlusRare} chooses the least frequently occurring orbit
  count list (using the lexicographically smallest to break ties).
\item \textbf{RareOrbitPlusCommon} chooses the most frequently occurring orbit
  count list (using the lexicographically smallest to break ties).
\end{enumerate}

\subsection{Experiments}\label{sec:experiments}

In this section we perform a practical comparison of our algorithms, and the
\textbf{MinImage} algorithm of Linton, for three different families of problems:
grid groups, $m$-sets, and primitive groups.

\subsection{Experimental Design}\label{sec:probdef}

We will consider three sets of benchmarks in our testing. In each experiment,
given a permutation group that acts on \(\{1,\dots,n\}\), we will run
an experiment with each of our orderings to find the canonical image of a set of
size \(\myfloor{\frac{n}{2}}, \myfloor{\frac{n}{4}}\) and
\(\myfloor{\frac{n}{8}}\).

We run our algorithms on a randomly chosen conjugate of each primitive group, to
randomize the initial ordering of the integers the group is defined over. The
same conjugate is used of each group in all experiments, and when choosing a
random subset of size \(x\) from a set \(S\), we always choose the same random
subset. We use a timeout of five minutes for each experiment. We force GAP to build
a stabilizer chain for each of our groups before we begin our algorithm, because
this can in some cases take a long time.

For each size of set and each ordering, we measure three things. The total number
of problems solved, the total time taken to solve all problems, counting
timeouts as 5 minutes, and the number of moved points of the largest group
solved. Our experiments were all performed on an Intel(R) Xeon(R) CPU E5-2640 v4
running at 2.40GHz, with twenty cores. Each copy of GAP was allowed a maximum of
6GB of RAM.

\subsubsection{Grid Groups}

In this experiment, we look for canonical images of sets in grid groups.

\begin{defi}\label{def:gridgroup}
  Let $n \in \N$. The direct product $\SX{n} \times \SX{n}$ acts on the set
  $\{1,\ldots, n\} \times \{1,\ldots, n\}$ of pairs in the following way:

  For all $(i,j) \in  \{1,\ldots, n\} \times \{1,\ldots, n\}$
  and all $(\sigma,\tau) \in \SX{n} \times \SX{n}$ we define

  \[
    {(i,j)}^{(\sigma,\tau)} := (i^{\sigma}, j^{\tau}).
  \]

  The subgroup $G \le \Sym{\{1,\ldots, n\} \times \{1,\ldots, n\}}$ defined by this
  action is called the \textbf{$n \times n$ grid group}.
\end{defi}

We note that, while the construction of the grid group is done by starting with
an $n$ by $n$ grid of points and permuting rows or columns independently of each
other, we actually represent this group as a subgroup of $\SX{n\cdot n}$, and we
do not assume prior knowledge of the grid structure of the action.

We ran experiments on the grid groups for grids of size \(5 \times 5\) to \(100 \times
100\). The results of this experiment
are given in Table 1.

The basic algorithm, \textbf{MinImage}, is only able to solve \(22\) problems
within the timeout. \textbf{FixedMinOrbit} solves \(43\) problems, while being
implemented as a simple pre-processing step to \textbf{MinImage}. The dynamic
\textbf{MinOrbit} is able to solve \(55\) problems, and the best orbit-based
strategy, \textbf{SingleMaxOrbit}, solves \(70\) problems. However the advanced
techniques, which filter by orbit lists, perform much better.
Even ordering by the most common orbit list leads to solving over \(185\)
problems, and the best strategy, \textbf{RareOrbitPlusRare}, solves \(235\)
out of the total \(288\) problems.

Furthermore, for these large groups, the algorithms are still performing very
small searches: For example \textbf{FixedMinOrbit}, on its largest solved
problem with size \(\myfloor{\frac{n}{2}}\) sets and a grid of size \(12 \times 12\), generates
\(793,124\) search nodes, while \textbf{RareOrbitPlusMin} produces only \(183,579\)
search nodes on its largest solved problem with \(\myfloor{\frac{n}{2}}\) sets
(\(65 \times 65\)).

\afterpage{%
\clearpage\thispagestyle{empty}%
\begin{landscape}%
\centering%
\begin{tabular}{|l|l|l|r|r|r|r|r|r|r|r|r|}
\hline
&Stab&\multicolumn{3}{|c|}{\(\myfloor{\frac{n}{2}}\)}&\multicolumn{3}{|c|}{\(\myfloor{\frac{n}{4}}\)}&\multicolumn{3}{|c|}{\(\myfloor{\frac{n}{8}}\)} \\
&Search&\# solved&largest&time&\# solved&largest&time&\# solved&largest&time\\
\hline
RareOrbitPlusRare              & F &  56 & 4,225 & 12,549&  88 & 8,464 & 10,474&  91 & 9,216 & 13,663\\
RareOrbitPlusMin               & F &  54 & 4,225 & 13,207&  89 & 9,604 & 10,546&  90 & 9,025 & 13,484\\
RareOrbitPlusCommon            & F &  30 & 2,209 & 19,306&  65 & 5,476 & 15,408&  90 & 9,216 & 14,009\\
SingleMaxOrbit                 & F &  11 &   225 & 24,671&  27 &   961 & 25,795&  52 & 3,136 & 23,532\\
RareOrbit                      & F &   8 &   144 & 24,763&  17 &   625 & 27,213&  34 & 4,096 & 25,979\\
MinOrbit                       & F &   8 &   144 & 24,739&  16 &   400 & 27,735&  31 & 1,521 & 26,706\\
CommonRatioOrbit               & F &   7 &   121 & 24,990&  15 &   400 & 28,255&  29 & 1,296 & 27,322\\
FixedMinOrbit                  & T &   8 &   144 & 24,796&  13 &   361 & 28,619&  22 &   841 & 28,841\\
FixedMinOrbit                  & F &   8 &   144 & 24,786&  13 &   361 & 28,573&  21 &   841 & 28,859\\
MinImage                        & F &   4 &    64 & 25,819&   8 &   144 & 30,039&  10 &   196 & 31,851\\
MinImage                        & T &   4 &    64 & 25,822&   7 &   144 & 29,962&  10 &   196 & 31,829\\
RareRatioOrbit                 & F &   3 &    49 & 26,126&   5 &    81 & 30,379&   8 &   144 & 32,804\\
MaxOrbit                       & F &   3 &    49 & 26,123&   5 &    81 & 30,373&   8 &   144 & 32,771\\
FixedMaxOrbit                  & F &   3 &    49 & 26,120&   5 &    81 & 30,374&   8 &   144 & 32,725\\
FixedMaxOrbit                  & T &   3 &    49 & 26,113&   5 &    81 & 30,381&   8 &   144 & 32,450\\
CommonOrbit                    & F &   3 &    49 & 26,131&   5 &    81 & 30,396&   8 &   144 & 32,826\\
\hline
\end{tabular}
\label{gridtable}
\captionof{table}{Finding canonical images in grid groups}
\end{landscape}
\clearpage%
}

\subsubsection{M-Sets}

Linton \cite{Linton:SmallestImage} considers, given integers \(n\) and \(m\),
defining a permutation group on the set \(T\) of all subsets of size \(m\) of \(\{1,\dots,n\}\)
under the action on \(T\) of \(\SX{n}\) acting on the
members of the \(m\)-sets. He then looks for minimal images of randomly chosen
subsets of \(T\) of size \(k\), under the standard lexicographic ordering on sets.

We ran experiments for \(m = 2\) and \(n \in \{10,15,\dots,100\}\),
for \(m = 4\) and \(n \in
\{10,15,\dots,35\}\), for \(m=6\) and \(n \in \{10,15,20\}\) and
finally \(m=8\), \(n=10\) as described in Section \ref{sec:probdef}. We choose these 30
experiments as these were the problems that any of our techniques were able
to solve in under 5 minutes. The results of our experiments are shown in Table 2.

Similarly to our experiments on grid groups, we find that the standard
\textbf{MinImage} algorithm is only able to solve a very small set of benchmarks.
Some of the better algorithms, including \textbf{FixedMinOrbit},
are able to solve \(9\) problems. In particular, once again \textbf{MinOrbit}
is not significantly better than \textbf{FixedMinOrbit}, although it is slightly
faster on average over all problems.

However, the orbit-based strategies do much better, solving all the problems
which we set. In the case of sets that contain an eighth of all \(m\)-sets, the best
technique is able to solve all problems any technique can solve, in under 5
minutes. The largest solved problem, which was instance \(n=35, m=4\) for a set
on an eighth of all \(m\)-sets, is solved in only \(6,594\) search nodes by
\textbf{RareOrbitPlusMin}, while the largest solved problem of \textbf{MinImage},
\(n = 15, m = 4\) takes \(631,144\) search nodes.

\afterpage{%
\clearpage\thispagestyle{empty}%
\begin{landscape}%
\centering%
\begin{tabular}{|l|l|l|r|r|r|r|r|r|r|r|r|}
\hline
&Stab&\multicolumn{3}{|c|}{\(\myfloor{\frac{n}{2}}\)}&\multicolumn{3}{|c|}{\(\myfloor{\frac{n}{4}}\)}&\multicolumn{3}{|c|}{\(\myfloor{\frac{n}{8}}\)} \\
&Search&\# solved&largest&time&\# solved&largest&time&\# solved&largest&time\\
\hline
RareOrbitPlusMin               & F &  28 & 27,405 & 1,290&  30 & 52,360 &   835&  30 & 52,360 &   251\\
RareOrbitPlusRare              & F &  27 & 12,650 & 1,235&  30 & 52,360 & 1,107&  30 & 52,360 &   322\\
RareOrbitPlusCommon            & F &  27 & 12,650 & 1,819&  30 & 52,360 & 1,189&  30 & 52,360 &   335\\
MinOrbit                       & F &   9 & 1,365 & 6,701&  22 & 6,435 & 4,143&  28 & 27,405 & 1,203\\
RareOrbit                      & F &   9 & 1,365 & 6,769&  21 & 6,435 & 4,190&  28 & 27,405 & 1,276\\
FixedMinOrbit                  & F &   9 & 1,365 & 6,671&  22 & 6,435 & 4,292&  27 & 12,650 & 1,765\\
FixedMinOrbit                  & T &   9 & 1,365 & 6,678&  22 & 6,435 & 4,291&  27 & 12,650 & 1,766\\
CommonRatioOrbit               & F &   9 & 1,365 & 6,782&  21 & 6,435 & 4,215&  28 & 27,405 & 1,348\\
MinImage                        & F &   4 &   210 & 7,866&   5 &   210 & 7,562&   7 & 1,365 & 7,033\\
MinImage                        & T &   4 &   210 & 7,843&   5 &   210 & 7,573&   7 & 1,365 & 7,037\\
SingleMaxOrbit                 & F &   4 &   210 & 7,813&   5 &   210 & 7,554&   6 & 1,365 & 7,325\\
FixedMaxOrbit                  & T &   4 &   210 & 7,853&   5 &   210 & 7,691&   6 &   210 & 7,569\\
RareRatioOrbit                 & F &   4 &   210 & 7,923&   5 &   210 & 7,642&   5 &   210 & 7,500\\
MaxOrbit                       & F &   4 &   210 & 7,915&   5 &   210 & 7,695&   5 &   210 & 7,500\\
FixedMaxOrbit                  & F &   4 &   210 & 7,890&   5 &   210 & 7,666&   5 &   210 & 7,500\\
CommonOrbit                    & F &   4 &   210 & 7,947&   5 &   210 & 7,724&   5 &   210 & 7,500\\
\hline
\end{tabular}
\label{table:mset}
\captionof{table}{Finding canonical images in M-Set groups}
\end{landscape}
\clearpage%
}

\subsubsection{Comparison to Graph Canonical Image}

A set of \(2\)-sets can be viewed as an undirected graph, where the two sets represent the edges. The problem of finding the canonical image of this set of \(2\)-sets is equivalent to the traditional problem of finding a canonical image of this graph. We can therefore perform a comparison between our technique and Nauty, for these problems. Nauty is able to a find canonical image for all our \(2\)-set problems almost instantly. We investigated why Nauty was able to outperform us by such a large margin, and found three problems. We list the most important one first.

\begin{itemize}
\item The central algorithm of Nauty makes use of properties of the form ``vertices with \(i\) neighbors can only map to other vertices with \(i\) neighbors''. Our algorithm does not make use of this property, as it represents a much more complex condition when considered on \(m\)-sets. Further, while we could add a special case specifically for when the group we are considering is the symmetric group operating on \(m\)-sets, we would prefer to find a more general technique.
\item Our algorithm spends a large proportion of its time calculating stabilizer chains, and mapping sets through elements of the group. This is not required for the graphs.
\item Our algorithm is written in GAP rather than highly optimized C.
\end{itemize}

The most important results to draw from this comparison is that our algorithms should not be viewed as a replacement for graph isomorphism algorithms. We are investigating how to close this performance gap, without special casing.

\subsubsection{Primitive Groups}

In this experiment we look for canonical images of sets under the action of primitive groups
which move between \(2\) and \(1,000\) points. We remove the natural alternating and
symmetric groups, as finding minimal and canonical images in these groups is
trivial and can be easily special-cased.
%, but generating stabilizer chains for
%, these groups is quite expensive, due to their size.
% we mea nexpensive when they're not in their natural action because we have to
% make a long chain with long orbits?
So we look at a total number of \(5,845\) groups, each of which was successfully treated by at least one algorithm.

We perform the experiment as described in Section \ref{sec:probdef}. The results
are given in Table 3.

All algorithms are able to solve a large number of problems. This is unsurprising
as many primitive groups are quite small (for example the cyclic groups), so any
technique is able to brute-force search many problems. However, we can still see that,
for the hardest problems \(\myfloor{\frac{n}{2}}\), many algorithms
outperform \textbf{MinImage}, and the techniques that use extra orbit-counting
filtering solve 300 more problems, and they run much faster.

For the easiest set of problems, \(\myfloor{\frac{n}{8}}\), we see that the
algorithm \textbf{RareOrbitPlusMin}, which usually performs best, solves slightly fewer problems. This
is because there are a small number of groups where the extra filtering provides
no search reduction, but still requires a small overhead in time. However, the
total time taken is still much smaller, and the algorithm only fails to solve
five problems. These five problems involve groups that are isomorphic to the
affine general linear groups $\operatorname{AGL}(8,2)$,
$\operatorname{AGL}(6,3)$ and $\operatorname{AGL}(9,2)$, and
the projective linear group $\operatorname{PSL}(9,2)$. This suggests that the linear
groups may be a source of hard problems for canonical image algorithms in the future.

\afterpage{%
\clearpage\thispagestyle{empty}%
\begin{landscape}%
\centering%
\begin{tabular}{|l|l|l|r|r|r|r|r|r|r|r|r|}
\hline
&Stab&\multicolumn{2}{|c|}{\(\myfloor{\frac{n}{2}}\)}&\multicolumn{2}{|c|}{\(\myfloor{\frac{n}{4}}\)}&\multicolumn{2}{|c|}{\(\myfloor{\frac{n}{8}}\)} \\
&Search&\# solved&time&\# solved&time&\# solved&time\\
\hline
RareOrbitPlusMin               & F & 5,689 & 64,983& 5,749 & 39,564& 5,840 & 10,287\\
RareOrbitPlusRare              & F & 5,656 & 81,814& 5,738 & 43,202& 5,825 & 19,025\\
RareOrbitPlusCommon            & F & 5,561 & 113,011& 5,721 & 59,740& 5,816 & 25,392\\
FixedMinOrbit                  & T & 5,360 & 220,076& 5,623 & 82,684& 5,817 & 18,295\\
FixedMinOrbit                  & F & 5,354 & 251,786& 5,628 & 99,253& 5,816 & 30,193\\
MinOrbit                       & F & 5,348 & 263,053& 5,641 & 104,241& 5,844 & 27,720\\
RareOrbit                      & F & 5,324 & 272,631& 5,632 & 105,537& 5,844 & 29,365\\
CommonRatioOrbit               & F & 5,323 & 277,050& 5,629 & 107,561& 5,844 & 32,123\\
SingleMaxOrbit                 & F & 4,811 & 465,908& 5,250 & 253,467& 5,648 & 112,147\\
MinImage                        & T & 4,723 & 390,334& 5,163 & 242,477& 5,631 & 88,048\\
MinImage                        & F & 4,710 & 501,952& 5,180 & 280,686& 5,633 & 106,983\\
FixedMaxOrbit                  & F & 4,659 & 514,618& 5,119 & 298,321& 5,587 & 132,508\\
FixedMaxOrbit                  & T & 4,674 & 392,753& 5,095 & 251,001& 5,583 & 107,433\\
MaxOrbit                       & F & 4,641 & 544,222& 5,104 & 310,402& 5,587 & 144,182\\
RareRatioOrbit                 & F & 4,614 & 559,690& 5,090 & 316,609& 5,586 & 152,201\\
CommonOrbit                    & F & 4,604 & 569,047& 5,086 & 319,288& 5,586 & 154,142\\
\hline
\end{tabular}
\label{table:prim}
\captionof{table}{Finding canonical images in Primitive groups}
\end{landscape}
\clearpage%
}


\subsubsection{Experimental Conclusions}

Our experiments show that using \textbf{FixedMinOrbit} is almost always superior
to \textbf{MinImage}. As implementing \textbf{FixedMinOrbit} requires a fairly small
amount of code and time over \textbf{MinImage}, this suggests that any implementations of
Linton's algorithm should have \textbf{FixedMinOrbit} added, because this provides a substantial
performance boost, for relatively little extra coding.

Algorithms that dynamically
order the underlying set, such as \textbf{MinOrbit} and \textbf{RareOrbit}
provide only a small benefit over \textbf{FixedMinOrbit}. Algorithms which add
orbit counting provide a much bigger gain, often allowing solving problems on
groups many orders of magnitude larger than before, thereby greatly advancing the state
of the art.
