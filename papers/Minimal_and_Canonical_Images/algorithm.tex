% !TEX root =  MinCanRevised.tex

\subsection{Orderings}

When building canonical images, we build orderings as the algorithm progresses.
We represent these partially built orderings as ordered partitions.

\begin{defi}
  Let $k \in \N$ and let \(P=[X_1,\dots,X_k]\) be an ordered partition of
  $\powset(\Omega)$. Then, given two
  subsets $S$ and $T$ of $\Omega$, we write \(S <_P T\) if and only if the cell
  that contains \(S\) occurs before the cell that contains \(T\) in \(P\).

  We say that $P$ is \emph{\(G\)-invariant} if and only if for all
  \(i \in \{1,\dots,k\}\) and \(g \in G\) it holds that \(S \in X_i\)
  if and only if \(S^g \in X_i\).

  A \emph{refinement} of an ordered partition $[X_1,\dots,X_k]$ is an ordered
  partition $[Y_{1,1},Y_{1,2},\dots,Y_{k,l}]$ where $l \in \N$ and such that,
  for all $i \in \{1,\dots,k\}$ and $j \in \{1,\dots,l\}$, we have that
  $Y_{i,j} \subseteq X_i$.

  A \emph{completion} of an ordered partition $X = [X_1,\dots,X_k]$ is a
  refinement where every cell is of size one. Given an ordering for
  \(\Omega\), the \emph{standard completion} of an ordered partition \(X\)
  orders the members of each cell of \(X\) using the ordering on sets from
  Definition \ref{metaorder}.
\end{defi}

In our algorithm we need a completion of an ordered partition, but the exact
completion is unimportant -- it is only important that, given an ordered
partition \(X\), we always return the same completion.
For this reason we define the standard completion of an ordered partition.

\begin{ex}\label{ex:orderpart}
Let $G:=\langle (12)\rangle \le \SX{3}$ and $\Omega :=\{1,2,3\}$. Moreover let

$P:=[ \{\}, \{1\}, \{2\}, \{1,3\}, \{2,3\} \mid \{3\}, \{1,2\}, \{1,2,3\} ]$
be an ordered partition of $\powset(\Omega)$.

The orbits of $G$ on $\Omega$ are $\{1,2\}$ and $\{3\}$.
In particular all elements of $G$ stabilize the partition
$P$.

The ordered partition $Q:=[\{1,3\}, \{2\} \mid \{\}, \{1\}, \{2,3\} \mid \{3\}, \{1,2\}, \{1,2,3\} ]$ is a refinement
of $P$ that is not $G$-invariant.

To see this, we let $g:=(12) \in G$. We have that $\{1,3\}$ is in the first cell
of \(Q\), but $\{1,3\}^g = \{2,3\}$ is not in the first cell.
\end{ex}

In Example~\ref{ex:orderpart} we only considered a very small group, because
the size of \(\powset(\Omega)\) is \(2^{|\Omega|}\). In practice we will not explicitly
create ordered partitions of \(\powset(\Omega)\), but instead store a compact
description of them from which we can deduce the cell that any particular set is in.

In this paper, we will consider two methods of building and refining ordered partitions. We first
define the orbit count of a set, which we will use when building refiners.

\begin{defi}\label{def:orbcount}
  Let \(G\) be a group acting on an ordered set \(\Omega\), and
  \(S \subseteq \Omega\).
  Define the \textbf{orbit count} of \(S\) in \(G\), denoted \(\Orbcount(G,S)\)
  as follows: Given the list \(\Orb(G)\) of orbits of $G$ on $\Omega$ sorted
  by the their smallest member, the list
  $\Orbcount(G,S)$ contains the size of the intersection \(|o \cap S|\) in place
  of $o \in \Orb(G)$.
\end{defi}

We will see the practical use of \(\Orbcount\) in Lemma~\ref{lem:orbcount2}.

\begin{lem}\label{lem:orbcount}
  Suppose that $(\Omega, \le)$ is a totally ordered finite set and that
  $G \le \Sym{\Omega}$. Suppose further that \(S,T \subseteq \Omega\) and that there is some $g \in G$ such that $S^g = T$. Then \(\Orbcount(G,S) = \Orbcount(G,T)\).
\end{lem}

\begin{proof}
  Let $o \in \Orb(G)$ and $g \in G$ with $S^g = T$. Then $o^g = o$ and
  $\alpha \in (o \cap S)$ if and only if $\alpha^g \in (o \cap S)^g$,
  if and only if $\alpha^g \in (o^g \cap S^g) = (o \cap T)$.

%  By definition, the image of any \(i \in \Omega\) under any \(g \in G\) is another point
%  in the same orbit as \(i\), and a permutation is a bijection -- therefore taking a subset of
%  an orbit of \(G\) and applying an element of \(G\) produces another subset of the same orbit,
%  of the same size.
\end{proof}

\begin{defi}\label{def:refiners}
  Let \(P\) be an ordered partition of \(\powset(\Omega)\).

  \begin{itemize}
  \item If $\alpha \in \Omega$, then the \textbf{point refinement} of \(P\) by
    $\alpha$ is the ordered partition \(Q\) defined in the following way: Each cell \(X_i\) of \(P\)
    is split into two cells, namely the cell
    {\(\{S \mid S \in X_i, \alpha \in S\}\)}, and the cell
    {\(\{S \mid S \in X_i, \alpha \not\in S\}\)}.
    If one of these sets is empty, then \(X_i\) is not split.

  \item If \(G \le \Sym{\Omega}\) and \(C = \Orbcount(G,T)\) for some set
    \(T \subseteq \Omega\), then the \textbf{orbit refinement} of \(P\) by \(C\)
    is the ordered partition \(Q\) defined as follows: Each cell \(X_i\) of \(P\) is split
    into two cells, namely
    \(\{S\ \mid S \in X_i, \Orbcount(G,S) = C\}\), and
    \(\{S\ \mid S \in X_i, \Orbcount(G,S) \neq C\}\).
    If one of these sets is empty, \(X_i\) is not split.
  \end{itemize}
\end{defi}

\subsection{Algorithm}

We will now present our algorithm. First, we give a technical definition which
will be used in proving the correctness of our algorithm.

\begin{defi}
For all $n \in\N$ we define $\LL_n$ to be the set of lists of length $n$ whose
entries are non-empty subsets of $\Omega$. If $X \in \LL_n$, then as a
convention we write $X_1,\dots,X_n$ for the entries of the list $X$.

If $X \in \LL_n$ and $H \le G \le \SX{n}$ is such that $|G:H|=k \in \N$ and
$Q=\{q_1,\dots,q_k\}$ is a set of coset representatives of $H$ in $G$, then we
define $X^Q$ to be the list whose first $k$ entries are
$X_1^{q_1},\dots,X_1^{q_k}$, followed by $X_2^{q_1},\dots,X_2^{q_k}$ until the last
$k$ entries are $X_n^{q_1},\dots,X_n^{q_k}$. We note that $X^Q \in \LL_{n \cdot
k}$.

Let $X,Y \in \LL_n$. We say that $X$ and $Y$ are $G$-equivalent if and only if
there exist a permutation $\sigma$ of $\{1,\dots,n\}$ and group elements
$g_1,\dots,g_n \in G$ such that, for all $i \in \{1,\dots,n\}$, it holds that
$Y_i=X_{i^\sigma}^{g_i}$.
\end{defi}

We now prove a series of three lemmas about coset representatives, which form
the basis for the correctness proof of our algorithm. They are used to perform
the recursive step, moving from a group to a subgroup.

\begin{lem}\label{rep}
Suppose that $G$ is a permutation group on a set \(\Omega\), that \(H\) is a
subgroup of \(G\) of index $k \in \N$ and that $T$ is a set of left coset
representatives of $H$ in $G$.

Then the following are true:

\begin{enumerate}

\item $|T|=k$.

\item If $T=\{t_1,\dots,t_k\}$ and $g \in G$ and if, for all $i \in \{1,\dots,k\}$,
we define $q_i:=gt_i$, then $Q:=\{q_1,\dots,q_k\}$ is also a set of coset
representatives of $H$ in $G$. In particular there is a bijection from $Q$ to
any set of left coset representatives of $H$ in $G$.

\end{enumerate}
\end{lem}

\begin{proof}
By definition the index of $H$ in $G$ is the number of (left or right) cosets of $H$ in $G$.
% When choosing coset representatives, there is exactly one representative for each coset, and hence the number of representatives is $k$.

For the second statement we let $i,j \in \{1,\dots,k\}$ be such that $q_iH=q_jH$,
hence $gt_iH=gt_jH$. Then $t_j^{-1}t_i=t_j^{-1}g^{-1}gt_i \in H$ and hence
$t_iH=t_jH$. Hence $i=j$ because $t_i$ and $t_j$ are from a set of coset
representatives.
\end{proof}

\begin{lem}\label{equi-buildup}
Suppose that $G$ is a permutation group on a set \(\Omega\) and that $S,T
\subseteq \Omega$ are such that the lists $[S]$ and $[T]$ are $G$-equivalent.

Let \(H\) be a subgroup of \(G\) of index $k \in \N$ and let $P=\{p_1,\dots,p_k\}$
and $Q:=\{q_1,\dots,q_k\}$ be sets of left coset representatives of $H$ in $G$.

Then $[S]^P$ and $[T]^Q$ are \(H\)-equivalent.
\end{lem}

\begin{proof}
As \([S]\) and \([T]\) are \(G\)-equivalent, we know that there exists a group
element $g \in G$ such that $S^g=T$.

We fix $g$, for all $i \in \{1,\dots,k\}$ we let $t_i:=gq_i$ and we consider the
set $T:=\{t_i \mid i \in \{1,\dots,k\}\}$. Then $T$ is also a set of left coset
representatives of $H$ in $G$, by Lemma \ref{rep}. As $P$ is also a set of left
coset representatives, we know that $T$ and $P$ have the same size, so there is
a bijection from $P$ to $T$. This can be expressed in the following way:

There is a permutation $\sigma \in \SX{k}$ such that, for all $i \in \{1,..,k\}$,
it is true that $p_{i^\sigma}H=t_iH$. That means  there is a unique $h_i
\in H$ such that $p_{i^\sigma}h_i=t_i$.

Let now $S_i:=S^{p_i}$ and $T_i:=T^{q_i}$ for $i \in \{1,\dots,k\}$, then
\[
  T_i=T^{q_i}=(S^g)^{q_i}=S^{t_i}=S^{p_{i^\sigma}h_i}=(S^{p_{i^\sigma}})^{h_i}=(S_{i^\sigma})^{h_i},
\]
hence $[S]^P$ and $[T]^Q$ are $H$-equivalent.
\end{proof}

\begin{lem}\label{equi}
  Suppose that $G$ is a permutation group on a set \(\Omega\), that $n \in \N$ and that
  $X,Y \in \LL_n$ are $G$-equivalent. Let \(H\) be a subgroup of \(G\) of index $k \in \N$ and let
  $P=\{p_1,\dots,p_k\}$ and $Q:=\{q_1,\dots,q_k\}$ be sets of left coset representatives of $H$ in $G$.

  Then $X^P$ and $Y^Q$ are \(H\)-equivalent.
\end{lem}

\begin{proof}
As \(X\) and \(Y\) are \(G\)-equivalent, we know that there exist a permutation
$\sigma \in \SX{n}$ and $g_1,\dots,g_n \in G$ such that $Y_i=X_{i^\sigma}^{g_i}$ for all $i \in
\{1,\dots,n\}$. We fix this permutation $\sigma$.

If \(i \in \{1,\dots,n\}\), then \([X_{i^\sigma}]\) and \([Y_i]\) satisfy the
hypothesis of Lemma \ref{equi-buildup}, so it follows that \([X_{i^\sigma}]^P\)
and \([Y_i]^Q\) are $H$-equivalent.

So we find a permutation \(\alpha_i \in \SX{k}\) and group elements
\(h_{i1},\dots,h_{ik} \in H\) such that \(
(X_{i^\sigma}^{p_{j^{\alpha_i}}})^{h_{ij}} = Y_i^{q_i}\) for all $j \in
\{1,\dots,k\}$.

Using \(\sigma\) and $\alpha_1,\dots,\alpha_k$ we define a permutation \(\gamma\)
on \(1,\dots,n \cdot k\).

First we express $l \in \{1,\dots,n \cdot k\}$ uniquely as $l=c_l \cdot k+r_l$ where
$c_l \in \{0,\dots,n-1\}$ and $r_l \in \{1,\dots,k\}$ and we define

\[
  l^\gamma:=(c_l+1)^\sigma \cdot k + r_l^{\alpha_{c_l+1}}.
\]

This is well-defined because of the ranges of $c_l$ and $r_l$ and it is a
permutation because of the uniqueness of the expression and because $\sigma$ and
$\alpha_1,\dots,\alpha_n$ are permutations.

Then, for each $l \in \{1,\dots,n \cdot k\}$, expressed as $l=c_l \cdot k+r_l$ as we did
above, we set $h:=h_{c_l+1,r_l}$, $X'_l:=X_{c_l+1}^{p_{r_l}}$ and
$Y'_l:=Y_{c_l+1}^{q_{r_l}}$.

Then $X^P=[X'_1,\dots,X'_{n \cdot k}]$ and $Y^Q=[Y'_1,\dots,Y'_{n \cdot k}]$.

If we set $p:=p_{r_l^{a_{c_l+1}}}$ and $q:=q_{r_l}$, then
we have, for all $l=c_l \cdot k+r_l$, that

\[
  X_{l^{\gamma}}^{\prime h} = (X_{(c_l+1)^\sigma}^{p})^{h} =
  ((X_{c_l+1}^\sigma)^{p})^{h} = Y_{c_l+1}^{q} = Y'_l.
\]

This is $H$-equivalence.
\end{proof}

We can now describe the algorithm we use to compute canonical images,
and prove that it works correctly.

\begin{defi} \label{def:algparts}
Suppose that $\Omega$ is a finite set, that $G$ is a permutation group on $\Omega$,
that $L \in \LL_k$ and that $P$ is an ordered partition on $\powset({\Omega})$.

  \begin{itemize}
  \item An \emph{$\Omega$-selector} is a function $\mathcal{S}$ such that
    \begin{itemize}
    \item $\mathcal{S}(\Omega, G, L, P) = \omega \in \Omega$, where $|\omega^G| > 1$;
    \item $\mathcal{S}(\Omega, G, L, P) = \mathcal{S}(\Omega, G, M, P)$ whenever
      $L$ and $M$ from $\LL_k$ are $G$-equivalent.
    \end{itemize}
%  \(S\) and \(T\), \(f(\Omega,G,S,Ord) = f(\Omega,G,T,Ord)\), and also that it's
%  return value is not fixed by \(G\) (this means \(\Omega\)-selectors are not
%  defined for the identity group).

  \item An \emph{Ordering refiner} is a function $\mathcal{O}$ such that for
    all $G$-invariant partitions $P$ of $\powset{({\Omega})}$
    \begin{itemize}
    \item $\mathcal{O}(\Omega, G, L, P) = P'$, where $P'$ is a $G$-invariant
      refinement of $P$;
    \item $\mathcal{O}(\Omega, G, L, P) = \mathcal{O}(\Omega, G, M, P)$ whenever
      $L$ and $M$ from $\LL_k$ are $G$-equivalent.
    \end{itemize}
   \end{itemize}
 \end{defi}

An ordering refiner cannot return a total ordering, unless $G$ acts trivially,
because the partial ordering cannot distinguish between values that are contained in the same orbit of $G$.

Our method for finding canonical images is outlined in Algorithm \ref{alg:canimage}. It recursively searches for the minimal image of a collection of lists,
refining the ordering that is used as search progresses.

\begin{algorithm}
  \caption{$\textsc{CanImage}$}\label{alg:canimage}
  \begin{algorithmic}[1]
    \Require{$\mathcal{S}$ is an $\Omega$-selector, $\mathcal{O}$ is an ordering refiner}
    \Procedure{CanImageRecurse}{$\Omega, G, L, P$}
%    \Comment{$G \leq \Sym{\Omega}, X \in \LL_k$}
%    \Comment{$P \textrm{ordered partition of \Omega$}
    \If{$|G| = 1$}\label{line:if}
    \State{$P' := \textrm{Standard completion of } P$}
    \State{\Return{Smallest member of \(L\) under \(P'\)}}
    \EndIf\label{line:endif}
    \State{$H := G_{\mathcal{S}(\Omega,G,L,P)}$}
    \State{$Q := $ coset representatives of $H$ in $G$}
    \State{$P' := \mathcal{O}(\Omega,H,L^Q,P)$}
    \State{$L' := [S \mid S \in L^Q, \not\exists T \in L^Q. T <_{P'} S]$}
%    \State{$L' := \mathcal{R}(\Omega, H, L^Q, P')$}
    \State\Return{\Call{CanImageRecurse}{$\Omega, H, L', P'$}}
    \EndProcedure{}
  \end{algorithmic}

%R: Had to rename the coset reps because P has a different meaning now.

  \begin{algorithmic}[2]
    \Procedure{CanImageBase}{$\Omega, InputG, InputS$}
    \State\Return{\Call{CanImageRecurse}{$\Omega, InputG, [InputS], [\powset(\Omega)]$}}
  \EndProcedure{}
  \end{algorithmic}
\end{algorithm}

\begin{thm}\label{thm:canorb}
Suppose that $\Omega$ is a finite set, that $G$ is a permutation group on $\Omega$, and
 that $X \subseteq \Omega$. Then $\textsc{CanImageBase}(\Omega,G,X) \in X^G$.
\end{thm}
\begin{proof}
  In every step of Algorithm \ref{alg:canimage} the list of considered sets is
  a list of elements of $X^G$.
\end{proof}

\begin{thm}\label{thm:canimg}
  Let $\Omega$ be a finite set and $G$ a permutation group on $\Omega$.
  Let $X,Y\in\LL_k$ be $G$-equivalent, and let $P$ be a $G$-invariant ordered
  partition of $\powset(\Omega)$. Then
  \[
    \textsc{CanImage}(\Omega,G,X,P) =
    \textsc{CanImage}(\Omega,G,Y,P).
  \]
\end{thm}
\begin{proof}
  We proceed by induction on the size of $G$. % (which is finite).

  The base case is $|G| = 1$. As \(X\) and \(Y\) are \(G\)-equivalent, \(X\) and
  \(Y\) contain the same sets, possibly in a different order. For a given \(P\)
  and \(\Omega\), there is only one standard completion of \(P\) that gives a
  complete ordering on \(\powset(\Omega)\), and so \(X\) and \(Y\) have the same
  smallest element under the standard completion of \(P\) and so the claim
  follows.

  Consider now any non-trivial group $G$, and suppose for our induction hypothesis that the claim holds for
  all groups $H$ where $|H| < |G|$.

  Definition \ref{def:algparts} and the fact that $\mathcal{S}$ is an
  $\Omega$-selector imply that
  \[
    H = G_{\mathcal{S}(\Omega,G,X,P)} = G_{\mathcal{S}(\Omega,G,Y,P)}.
  \]
  Moreover, $H$ is a proper subgroup of $G$. We take two sets $Q_1$ and $Q_2$ of
  coset representatives of $H$ in $G$, which are not necessarily equal.

  Since $\mathcal{O}$ is an ordering refiner, it holds that
  \[
    P' = \mathcal{O}(\Omega,H,X,P) = \mathcal{O}(\Omega,H,Y,P)
  \]

  By Lemma \ref{equi}, \(X^{Q_1}\) and \(Y^{Q_2}\) are \(H\)-equivalent, and by
  definition \(P'\) is $G$-invariant. If we identify the cell \(P'_i\) of \(P'\) that
  contains the smallest element of \(X^{Q_1}\), then \(L'_X\) contains those
  elements of \(X^{Q_1}\) that are in \(P'_i\). Each of these elements is
  \(H\)-equivalent to an element of \(Y^{Q_2}\), and therefore \(L'_X\) is
  \(H\)-equivalent to \(L'_Y\). Then the induction hypothesis yields that
  \[
    \textsc{CanImage}(\Omega, H, X',P) = \textsc{CanImage}(\Omega, G, Y',P),
  \]
  so the claim follows by induction.
\end{proof}

We see from Theorem \ref{thm:canorb} and \ref{thm:canimg} that
$$\textsc{CanImage}(\Omega,G,X,P) = \textsc{CanImage}(\Omega,G,Y,P)
\text{~if and only if~} Y \in X^G.$$

We note that Algorithm \ref{alg:canimage} can easily be adapted to return an
element $g$ of $G$ such that $X^g = \textsc{CanImage}(\Omega,G,X,P)$. This happens by
attaching to each set, when it is created, the permutation that maps it to the
original input \(S\). We omit this addition for readability.
