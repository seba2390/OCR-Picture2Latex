\documentclass[preprint,12pt]{elsarticle}
%\usepackage{showkeys}
\usepackage{amsmath, amssymb}
\newfont {\cyr} {wncyr10}
\pagestyle{plain} \frenchspacing
\renewcommand{\labelenumi}{{(\roman{enumi})}}
\usepackage{color}
\newcommand{\red}[1]{\,{\color{red} #1}\,}
\usepackage{amsfonts}
\usepackage{algorithm}
\usepackage[noend]{algpseudocode}
\usepackage{amsthm}%Beweisumgebung
\usepackage{mathabx}
\usepackage{xspace}
\linespread{1.1}

\usepackage{pdflscape}
\usepackage{afterpage}
\usepackage{capt-of}% or use the larger `caption` package

\newtheorem{theorem}{Theorem}[section]
\newtheorem{lem}[theorem]{Lemma}
\newtheorem{thm}[theorem]{Theorem}
\newtheorem{prop}[theorem]{Proposition}
\newtheorem{cor}[theorem]{Corollary}
\newtheorem{rem}[theorem]{Remark}
\newtheorem{hyp}[theorem]{Hypothesis}
\newtheorem{defi}[theorem]{Definition}
\newtheorem{ex}[theorem]{Example}

\newcommand{\SX}[1]{\ensuremath{S_{#1}}\xspace}
\newcommand{\Sym}[1]{\ensuremath{Sym(#1)}}

\newcommand{\myfloor}[1]{\left \lfloor #1 \right \rfloor}
\newcommand{\LL}{\mathcal{L}}
\newcommand{\id}{\textrm{id}}
\newcommand{\N}{\mathbb{N}}
\newcommand{\ext}{\operatorname{ext}}
\newcommand{\Min}{\operatorname{Min}}
\newcommand{\Orb}{\operatorname{Orb}}
\newcommand{\Points}{\operatorname{Points}}
\newcommand{\Orbcount}{\operatorname{Orbcount}}
\newcommand{\Count}{\operatorname{Count}}
\newcommand{\powset}{\mathcal{P}}


\begin{document}

\begin{frontmatter}

\title{Minimal and canonical images}

\author{Christopher Jefferson}
%\address{University of St~Andrews\\School of Computer Science\\North Haugh\\St Andrews\\KY16 9SX\\Scotland}
\ead{caj21@st-andrews.ac.uk}
\ead[url]{http://caj.host.cs.st-andrews.ac.uk/}

\author{Eliza Jonauskyte}
%\address{University of St~Andrews\\School of Computer Science\\North Haugh\\St Andrews\\KY16 9SX\\Scotland}
\ead{ej31@st-andrews.ac.uk}
%\ead[url]{http://caj.host.cs.st-andrews.ac.uk/}

\author{Markus Pfeiffer}
\address{University of St~Andrews\\School of Computer Science\\North Haugh\\St Andrews\\KY16 9SX\\Scotland}
\ead{markus.pfeiffer@st-andrews.ac.uk}
\ead[url]{https://www.morphism.de/~markusp/}

\author{Rebecca Waldecker}
\address{Martin-Luther-Universit\"at Halle-Wittenberg\\Institut f\"ur Mathematik\\06099 Halle\\Germany}
\ead{rebecca.waldecker@mathematik.uni-halle.de}
\ead[url]{http://conway1.mathematik.uni-halle.de/~waldecker/index-english.html}

\journal{Journal of Algebra}

\begin{abstract}
We describe a family of new algorithms for finding the canonical image of a set
of points under the action of a permutation group. This family of algorithms
makes use of the orbit structure of the group, and a chain of subgroups of the
group, to efficiently reduce the amount of search that must be performed to
find a canonical image.

We present a formal proof of correctness of our algorithms and describe experiments
on different permutation groups that compare our algorithms with the previous
state of the art.
\end{abstract}

\begin{keyword}
  Minimal Images, Canonical Images, Computation, Group Theory, Permutation Groups.
\end{keyword}
\end{frontmatter}

% !TEX root = ../arxiv.tex

Unsupervised domain adaptation (UDA) is a variant of semi-supervised learning \cite{blum1998combining}, where the available unlabelled data comes from a different distribution than the annotated dataset \cite{Ben-DavidBCP06}.
A case in point is to exploit synthetic data, where annotation is more accessible compared to the costly labelling of real-world images \cite{RichterVRK16,RosSMVL16}.
Along with some success in addressing UDA for semantic segmentation \cite{TsaiHSS0C18,VuJBCP19,0001S20,ZouYKW18}, the developed methods are growing increasingly sophisticated and often combine style transfer networks, adversarial training or network ensembles \cite{KimB20a,LiYV19,TsaiSSC19,Yang_2020_ECCV}.
This increase in model complexity impedes reproducibility, potentially slowing further progress.

In this work, we propose a UDA framework reaching state-of-the-art segmentation accuracy (measured by the Intersection-over-Union, IoU) without incurring substantial training efforts.
Toward this goal, we adopt a simple semi-supervised approach, \emph{self-training} \cite{ChenWB11,lee2013pseudo,ZouYKW18}, used in recent works only in conjunction with adversarial training or network ensembles \cite{ChoiKK19,KimB20a,Mei_2020_ECCV,Wang_2020_ECCV,0001S20,Zheng_2020_IJCV,ZhengY20}.
By contrast, we use self-training \emph{standalone}.
Compared to previous self-training methods \cite{ChenLCCCZAS20,Li_2020_ECCV,subhani2020learning,ZouYKW18,ZouYLKW19}, our approach also sidesteps the inconvenience of multiple training rounds, as they often require expert intervention between consecutive rounds.
We train our model using co-evolving pseudo labels end-to-end without such need.

\begin{figure}[t]%
    \centering
    \def\svgwidth{\linewidth}
    \input{figures/preview/bars.pdf_tex}
    \caption{\textbf{Results preview.} Unlike much recent work that combines multiple training paradigms, such as adversarial training and style transfer, our approach retains the modest single-round training complexity of self-training, yet improves the state of the art for adapting semantic segmentation by a significant margin.}
    \label{fig:preview}
\end{figure}

Our method leverages the ubiquitous \emph{data augmentation} techniques from fully supervised learning \cite{deeplabv3plus2018,ZhaoSQWJ17}: photometric jitter, flipping and multi-scale cropping.
We enforce \emph{consistency} of the semantic maps produced by the model across these image perturbations.
The following assumption formalises the key premise:

\myparagraph{Assumption 1.}
Let $f: \mathcal{I} \rightarrow \mathcal{M}$ represent a pixelwise mapping from images $\mathcal{I}$ to semantic output $\mathcal{M}$.
Denote $\rho_{\bm{\epsilon}}: \mathcal{I} \rightarrow \mathcal{I}$ a photometric image transform and, similarly, $\tau_{\bm{\epsilon}'}: \mathcal{I} \rightarrow \mathcal{I}$ a spatial similarity transformation, where $\bm{\epsilon},\bm{\epsilon}'\sim p(\cdot)$ are control variables following some pre-defined density (\eg, $p \equiv \mathcal{N}(0, 1)$).
Then, for any image $I \in \mathcal{I}$, $f$ is \emph{invariant} under $\rho_{\bm{\epsilon}}$ and \emph{equivariant} under $\tau_{\bm{\epsilon}'}$, \ie~$f(\rho_{\bm{\epsilon}}(I)) = f(I)$ and $f(\tau_{\bm{\epsilon}'}(I)) = \tau_{\bm{\epsilon}'}(f(I))$.

\smallskip
\noindent Next, we introduce a training framework using a \emph{momentum network} -- a slowly advancing copy of the original model.
The momentum network provides stable, yet recent targets for model updates, as opposed to the fixed supervision in model distillation \cite{Chen0G18,Zheng_2020_IJCV,ZhengY20}.
We also re-visit the problem of long-tail recognition in the context of generating pseudo labels for self-supervision.
In particular, we maintain an \emph{exponentially moving class prior} used to discount the confidence thresholds for those classes with few samples and increase their relative contribution to the training loss.
Our framework is simple to train, adds moderate computational overhead compared to a fully supervised setup, yet sets a new state of the art on established benchmarks (\cf \cref{fig:preview}).


\section{Minimal and Canonical Images}

Throughout this paper, $\Omega$ will be a finite set, $G$ a subgroup of
$\Sym{\Omega}$, and $\Omega$ will be ordered by some (not necessarily total)
order $\leq$.
%\begin{hyp}\label{main}
%  Suppose that $G$ is a finite group that acts on a finite set $\Omega$.
%\end{hyp}
If $\alpha \in \Omega$, then we denote the orbit of $\alpha$ under $G$ by $\alpha^G$.
Simlarly, if \(A \subseteq \Omega\) and $g \in G$, then 
\(A^g := \{a^g \mid a \in A\}\) and \(A^G := \{ A^g \mid g \in G \}\).

In this paper, we want to efficiently solve the problem of deciding, given two subsets
\(A,B \subseteq \Omega\), if \(A \in B^G\). We do this by defining a canonical image:

\begin{defi}
  A \textbf{canonical labelling function} $C$ for the action of $G$ on a set $\Omega$ is a function
  $C:\powset(\Omega) \rightarrow \powset(\Omega)$ such that, for all $A  \subseteq \Omega$,
  it is true that
  \begin{itemize}
  \item $C(A) \in A^G$, and
  \item ${C(A^g)} = C(A)$ for all $g \in G$.
  \end{itemize}

  In this situation we call $C(A)$ the \textbf{canonical image} of
  $A \subseteq  \Omega$ (with respect to $G$ in this particular action).

  Further, we say that $g_A \in G$ is a \textbf{canonizing element} for \(A\) if and only if
  $A^{g_A} = C(A)$.
\end{defi}


A canonical image can be seen as a well-defined
representative of a $G$-orbit on $\Omega$ with respect to the defined action. While in this
paper we will only consider the action of \(G\) on a set of subsets of \(\Omega\), canonical images are
defined similarly for any group and action.
In practice we want to be able to find canonical images effectively and
efficiently.
In some situations we are interested in computing the canonizing element,
which might not be uniquely determined. Our algorithms will always produce a
canonizing element as a byproduct of search.
We choose to make this explicit here to make the exposition clearer.


Minimal images are a special type of canonical image.

\begin{rem}
  Suppose that $\preccurlyeq$ is a partial order on $\Omega$ such that any
  two elements in the same orbit can be compared by $\preccurlyeq$.

   Let $\Min_\preccurlyeq$ denote the function that, for all $\omega \in \Omega$, maps $\omega$ to the
  smallest element in its orbit. Then $\Min_\preccurlyeq$ is a canonical labelling function.
\end{rem}

In practical applications we are interested in more structure, namely in structures
that $G$ can act on naturally via the action on a given set $\Omega$. These
structures include subsets of $\Omega$, graphs with vertex set $\Omega$, sets of
maps with domain or range $\Omega$, and so on.


In this paper, our main application will be finding canonical images when
acting on a set of subsets of $\Omega$.

\begin{defi}\label{metaorder}
  Suppose that $\leq$ is a total order of $\Omega$.
  Then we introduce a total order $\preccurlyeq$ on $\powset(\Omega)$ as follows:

  We say that $A$ \textbf{is less than} $B$ and write $A \preccurlyeq B$ if and only if $A$
  contains an element $a$ such that $a \notin B$ and $a \leq b$ for all
  $b \in B \setminus A$.
\end{defi}

\begin{ex}
  Let $\Omega:=\{1,2,3,4,5,6,7\}$ with the natural order and let $A:=\{1,3,4\}$,
  $B:=\{3,5,7\}$, $C:=\{3,6,7\}$, $D:=\{1,3\}$ and $E:=\{2\}$.

  Now $A \preccurlyeq B$, because $1 \in A$, $1 \notin B$ and $1$ is smaller
  than all the elements in $B$, in particular those not in $A$. Moreover $A \preccurlyeq C$ for the same reason. Furthermore, $B \preccurlyeq C$,
  because $5 \in B$, $5 \notin C$, and if we look at $C \setminus B$, then
  this only contains the element $6$ and $5$ is smaller.

  Next we consider $A$ and $D$. As $4 \in A \setminus D$ and $D\setminus A=\varnothing$, we see that $A \preccurlyeq D$. Also $A \preccurlyeq E$ because $1 \in A \setminus E$ and $1$ is smaller than all elements in $E \setminus A=E$.
  Finally $E \preccurlyeq B$ because $2 \in E \setminus B=E$ and $2$ is smaller than all elements in $B \setminus E=B$.
\end{ex}

\begin{rem}
The example illustrates that this new order introduced above reduces to lexicographical order for sets of the same size. But for sets of different sizes, it might seem counter-intuitive. Our reason for choosing this different ordering is that it satisfies the
following property:

If \(n \in \N\) and if \(A\) and \(B\) are sets of integers, then \(A \cap \{1,\dots,n\} < B \cap \{1,\dots,n\}\) implies \(A < B\). This means that, when building \(A\) and \(B\) incrementally, we know the order of \(A\) and \(B\) as soon as we find the first integer that is contained in one of the sets but not in the other. This is not true for lexicographic ordering of sets, as \(\{1\} < \{1,2\}\) but \(\{1,1000\} > \{1,2,1000\}\).
\end{rem}

If $G$ is a subgroup of $\Sym{\Omega}$ and $\omega \in \Omega$, then we denote by
$G_\omega$ the point stabilizer of $\omega$ in $G$. For distinct elements
$x,y \in \Omega$, we denote by \(G_{x \mapsto y}\)  the set of all elements of
$G$ that map $x$ to $y$. This set may be empty.

We remark that the above information is readily available from a stabilizer chain
for the group $G$, which can be calculated efficiently. For further details
we refer the reader to \cite{HCG}. We now introduce some notation and then prove a basic result about cosets.

\begin{defi}
 Let $G$ be a permutation group acting on a totally ordered set
 $(\Omega, \leq)$, and let $\preccurlyeq$ denote the induced ordering as explained in Definition \ref{metaorder}.
Let $H$ be a subgroup of $G$ and $S \subseteq \Omega$.
Then we define
  \textbf{the minimal image of $S$ under $H$} to be the smallest element in the set
  $\{S^h \mid h \in H\}$ with respect to $\preccurlyeq$.

In order to simplify notation, we will from now on write $\leq$ for the induced order and then we write $\Min(H,S,\leq)$ for the minimal image of $S$ under $H$.
\end{defi}


\begin{lem}\label{miniprop}
  Let $G$ be a permutation group acting on a totally ordered set
  $(\Omega, \leq)$, and let $H$ be a subgroup of $G$ and $S \subseteq \Omega$.
  Then the following hold for all $x,y \in \Omega$:

\begin{enumerate}

\item
  For all $\sigma \in H_{x \mapsto y}$ it is true that
  $\sigma \cdot H_y=H_{x\mapsto y}=H_x \cdot \sigma$.

\item
  If $\sigma \in H_{x \mapsto y}$, then
  \(\Min(\sigma \cdot H, S, \leq) = \Min(H, S^{\sigma}, \leq)\).
\end{enumerate}
\end{lem}

\begin{proof}
  If $\sigma \in H_{x \mapsto y}$, then multiplication by $\sigma$ from the
  right or left is a bijection on $H$, respectively. For all $\alpha \in H_x$ we
  have that $\alpha \cdot \sigma$ maps $x$ to $y$ and for all $\beta \in H_y$ we
  see that $\sigma \cdot \beta$ also maps $x$ to $y$. This implies the first
  statement.

  For (ii) we just look at the definition: $\Min(\sigma \cdot H, S, \leq)$ denotes
  the smallest element in the
  set $\{S^{\sigma\cdot h} \mid h \in H\}$ and $\Min(H, S^{\sigma}, \leq)$
  denotes the smallest element in the set $\{(S^\sigma)^h \mid h \in H\}$,
  which is the same set.
\end{proof}

\subsection{Worked Example}

We will find minimal, and later canonical, images using similar techniques to
Linton in \cite{Linton:SmallestImage}. This algorithm splits the problem into
small sub-problems, by splitting a group into the cosets of a point stabilizer.
We will begin by demonstrating this general technique with a worked
example.

\begin{ex}\label{ex:minimal}
  In the following example we will look at $\Omega = \{1,2,3,4,5,6\}$,
  the subgroup {$G = \langle (14)(23)(56), (126)\rangle \leq \SX{6}$},
  and $S = \{2,3,5\}$.
  We intend to find the minimal image $\Min(G,S, \leq)$, where the ordering on
  subsets of $\Omega$ is the induced ordering from $\leq$ on $\Omega$ as explained in
  Definition \ref{metaorder}.

  We split our problem into pieces by looking at cosets of \(G_1 = \langle (3,4,5)
  \rangle\). The minimal image of \(S\) under \(G\) will be realized by an
  element contained in (at least) one of the cosets of \(G_1\), so if we find
  the minimal image of \(S\) under elements in each coset, and then take the
  minimum of these, we will find the global minimum.

  Lemma \ref{miniprop} gives that, for all $g \in G$, it holds that $\Min(g \cdot
  G_1, S, \leq) = \Min(G_1, S^g, \leq)$, and
  so we can change our problem from looking for the minimal image of $S$ with
  respect to cosets of \(G_1\) to looking at images of \(S^g\) under elements of
  $G_1$ where $g$ runs over a set of coset representatives of $G_1$ in $G$.

  For each $i \in \{1,\dots,6\}$ we need an element
  $g_i \in G_{i \mapsto 1}$ (where any exist), so that we can then consider $S^{g_i}$.

  We choose the elements $\id$, $(162)$, $(146523)$, $(14)(23)(56)$, $(142365)$
  and $(126)$ and obtain six images of $S$:
  \[
    \{2,3,5\}, \{1,3,5\}, \{3,1,2\}, \{3,2,6\}, \{3,6,1\}, \{6,3,5\}.
  \]

  As we are looking at the images of these sets under \(G_1\), we
  know that all images of a set containing \(1\) will contain \(1\),
  and all images of a set not containing \(1\) will not contain \(1\).
  From Definition \ref{metaorder}, all subsets of \(\{1,\dots,6\}\)
  containing \(1\) are smaller than all subsets not containing \(1\). This means
  that we can filter our list down to $\{1,3,5\}, \{3,1,2\}$ and $\{3,6,1\}$.

  Furthermore, \(G_1\) fixes \(2\), so by the same argument we can filter our
  list of sets not containing \(2\), leaving only \(\{3,1,2\}\). The minimal image
  of this under \(G_1\) is clearly \(\{3,1,2\}\) (in this particular
  case we could of course also have stopped as soon as we saw \(\{3,1,2\}\), as
  this is the smallest possible set of size 3).

  Now, let us consider what would happen if the ordering of the integers was
  reversed, so we are looking for \(\Min(G, S, \geq)\), again with the
  induced ordering.

  For the same reasons as above, we begin by calculating
  $G_6 = \langle (3,5,4) \rangle$ and by finding images of \(S\) for some
  element from each coset of \(G_6\) in $G$.

  An example of six images is
  \[
    \{ 1, 5, 4 \}, \{ 6, 5, 3 \}, \{ 4, 6, 1 \}, \{ 5, 1, 2 \}, \{ 3, 2, 6 \},
    \{ 2, 3, 5 \}.
  \]
  We can ignore anything that does not contain \(6\), so we are left with:

  \[
    \{ 6, 5, 3 \}, \{ 4, 6, 1 \}, \{ 3, 2, 6 \}.
  \]

  As \(5\) is not fixed by \(G_6\), we can not reason about the presence
  or absence of \(5\) in our sets. There is an image of every set that
  contains \(5\), and there are even two distinct images of \(\{6,5,3\}\) that
  contain \(5\). Therefore we must continue our search by considering \(G_{6,5}\).

  Application of an element from each coset of $G_{6,5}$ to $S$ generates nine
  sets, of which four contain the element $5$. In fact we reach \(\{6,3,5\},
  \{6,5,4\}\) from the set $\{ 6, 4, 3 \}$, we reach \(\{5,6,1\}\) from the set
  $\{ 4, 6, 1 \}$ and we reach \(\{5,2,6\}\) from the set $\{ 3, 2, 6 \}$. From
  these we extract the minimal image \(\{6,5,4\}\).

  In this example, different orderings of \(\{1,2,3,4,5,6\}\) produced different sized
  searches, with different numbers of levels of search required.
\end{ex}

\section{Minimal Images under alternative orderings of \(\Omega\)}
\label{sec:alternate ordering}
As was demonstrated in Example \ref{ex:minimal}, the choice of ordering of the set our group acts on
influences the size of the search for a minimal image. In this section we will show
how to create
orderings of $\Omega$ that, on average, reduce the size of search for a minimal image.

We begin by showing how large a difference different orderings can make. We do
this by proving that, for any choice $\preccurlyeq$ of ordering of \(\Omega\), group \(G\) and
any input set $S$, we can construct a minimal image
problem that is as hard as finding \(\Min(G,S,\preccurlyeq)\), but where
reversing the ordering on $\Omega$ makes the problem trivial.

We make this more precise: Given $n \in \N$, a permutation group \(G\) on \(\{1,\dots,n\}\) with some ordering $\leq$ and a subset \(S
\subseteq \{1,\dots,n\}\), we construct a group \(H\) and a set \(T\) such that
\(\Min(G, S, \leq) = \Min(H,T, \leq) \cap \{1..n\}\), which shows that finding
\(\Min(H,T,\leq)\) is at least as hard as finding \(\Min(G,S,\leq)\). On the
other hand, we will show that \(\Min(H,T, \geq) = T\) and that this can be deduced without search.
This is done in Lemma \ref{ex}.
An example along the way will illustrate the construction.

\begin{defi}
We fix $n \in \N$ and we let $k \in \N$. For all $j \in \N$ we define $q(j) \in
\N$ (where $q$ stands for ``quotient'') and $r(j) \in \{1,\dots,n\}$ (where $r$ stands for ``remainder'') such that $j=q(j) \cdot n+r(j)$.


Let $\ext : G \rightarrow \SX{k \cdot n}$ be the following map:
For all $g \in G$ and all $j \in \{1,\dots,k \cdot n\}$, the element
$\ext(g)$ maps $j$ to $q(j) \cdot n+r(j)^g$.
\end{defi}


\begin{ex}
Let $n=4$ and $G=\SX{4}$. Then we extend the action of $G$ to the set
$\{1,\dots,12\}$ using the map $\ext$.

For example $g=(134)$ maps $4$ to $1$.
We write $12=2 \cdot 4+4$ and then it follows that
$\ext(g)$ maps $12$ to $2 \cdot 4 +4^g=8+1=9$.
In fact $g$ acts simultaneously on the three tuples $(1,2,3,4)$, $(5,6,7,8)$
and $(9,10,11,12)$ as it does on $(1,2,3,4)$.
\end{ex}

\begin{defi}
Fixing $n,k \in \N$ and a subgroup $G$ of $\SX{n}$, and using the map $\ext$
defined above, we say that \textbf{$H$ is the extension of $G$ on
  $\{1,\dots,k\cdot n\}$}
if and only if $H=\{\ext(g) \mid g \in G\}$ is the image of $G$
under the map $\ext$.
\end{defi}

The extension $H$ of $G$ on a set $\{1,\dots,k \cdot n\}$
is a subset of $\SX{k \cdot n}$. We show now that even more is true:

\begin{lem}\label{ext}
Let $n,k \in \N$ and $G \le \SX{n}$. Then the extension of $G$ onto $\{1,\dots,k \cdot n\}$ is a subgroup of $\SX{k \cdot n}$ that is isomorphic to $G$.
\end{lem}

\begin{proof} Let $H:=\ext(G)$ be the image of $G$ under the map $\ext$ and let
$a,b \in G$ be distinct. Then let $j \in \{1,\dots,n\}$ be such that $j^a \neq
j^b$.
By definition $\ext(a)$ and $\ext(b)$ map $j$ in the same way that $a$ and $b$
do, so we see that $\ext(a) \neq \ext(b)$. Hence the map $\ext$ is injective.
Therefore $\ext:G \rightarrow H$ is bijective.

Next we let $a,b \in G$ be arbitrary and we let $j \in \{1,\dots,k \cdot n\}$.
Then the composition $ab$ is mapped to $\ext(ab)$, which maps $j$ to $q(j) \cdot
n+r(j)^{ab}$. Now $r(j)^{ab}=(r(j)^a)^b$ and therefore the composition
$\ext(a)\ext(b) \in \SX{k \cdot n}$ maps $j$ to $(q(j) \cdot
n+r(j)^a)^{\ext(b)}=q(j) \cdot n + (r(j)^a)^b$. This is because $r(j)^a \in
\{1,\dots,n\}$.

Hence $\ext(ab)=\ext(a)\ext(b)$. That implies $\ext$ is a group homomorphism
and hence that $G$ and its image are isomorphic.
\end{proof}

\begin{lem}\label{ex}
Let $n \in \N$ and $G \le \SX{n}$. Let $H$ denote the extension of $G$ on
$\{1,\dots,(n+1) \cdot n\}$ and let $S \subseteq \{1,\dots,n\}$. Let
$T:=S \cup \{ l \cdot n + l \mid l \in \{1..n\}\}$, let \(\leq\) denote the natural
ordering of the integers, and let \(\geq\) denote its reverse. For simplicity we use the same symbols for the ordering induced on $\powset(\Omega)$, respectively. Then

\begin{itemize}
\item \(\Min(H,T, \le) \cap \{1,\dots,n\} = \Min(G,S, \le)\).
\item \(\Min(H,T,\ge)=T\).
\end{itemize}
\end{lem}

\begin{proof}
Let $h \in H$. Then by construction $h$ stabilizes the partition
$$[1,\dots,n|n+1,\dots,2n|\dots|n \cdot n,\dots,(n+1)\cdot n].$$ 
Moreover, for all $i \in \{1,\dots,n\}$ and $g \in G$ we have that $i^g=i^{\ext(g)}$
and so Lemma \ref{ext} implies that

\begin{align*}
  \Min(G,S,\le) &= \min_{\le}\{S^g \mid g \in G\}  \cap \{1,\dots,n\}\\
                &= \min_{\le}\{S^{\ext(g)} \mid g \in G\}  \cap \{1,\dots,n\}\\
                &= \min_{\le}\{S^h \mid h \in H\}  \cap \{1,\dots,n\}\\
                &= \Min(H,T, \le) \cap \{1,\dots,n\}.
\end{align*}
This proves the first statement.

For the second statement we notice that $(n + 1)\cdot n$ is now the smallest element
of $T$, and it cannot be mapped to anything smaller, because it also is the
smallest element available.
So if we let $h \in H$ be such that $T^h=\Min(H,T,\ge)$, then $h$ fixes the point $n(n + 1)$. By
definition of the extension, it follows that $h$ also fixes $k \cdot n$ for all
$k \in \{1\ldots n\}$.
The next point of $T$ under the ordering is $n^2 - 1$. It cannot be
mapped by $h$ to $n^2$, because $n^2$ is already fixed, hence $h$ has to fix
$n^2 - 1$, too.

Arguing as above it follows that all points are fixed by $h$, thus in
particular \(\Min(H,T,\ge)=T\), as stated. Furthermore, any algorithm
that stepped through the elements of \(T\) in the order we describe
would find this smallest element without having to perform a branching search,
as at each step there is no choice on which element of \(T\) is the next smallest.
\end{proof}


\subsection{Comparing Minimal Images Cheaply}

We describe some important aspects of Linton's
algorithm for computing the minimal image of a subset of $\Omega$.
\begin{defi}

Suppose that $(\Omega, \le)$ is a totally ordered set and that $G \le
\Sym{\Omega}$. Then \(\Orb(G)\) denotes the list of orbits of $G$ on $\Omega$.
This list of orbits is ordered with respect to the smallest element in each orbit under $\le$.

A $G$-orbit will be called a \textbf{singleton} if and only if it has size $1$.

If \(S \subseteq \Omega\), then we say that a $G$-orbit is \textbf{empty in S}
if and only if it is disjoint from $S$ as a set, and we say that it is
\textbf{full in S} if and only if it is completely contained in $S$.
\end{defi}

\begin{ex}
  Let \(\Omega:=\{1,\dots,8\}\), with the natural ordering on the integers, let
  \(G := \langle (1,4), (2,8), (5,6), (7,8) \rangle\) and let $S:=\{1,3,5,6\}$.

  Then \(\Orb(G) = [ \{1,4\}, \{2,7,8\}, \{3\}, \{5,6\} ]\) because this list
  contains all the $G$-orbits and they are ordered by the smallest element in each
  orbit, namely $1$ in the first, $2$ in the second, $3$ in the third, which is a
  singleton, and $5$ in the last (because $4$ is already in an earlier orbit).

  The orbits $\{3\}$ and $\{5,6\}$ are full in $S$, the orbit $\{2,7,8\}$ is
  empty in $S$ and $\{1,4\}$ is neither.
\end{ex}


\begin{lem}\label{same}
Suppose that $(\Omega, \le)$ is a totally ordered finite set and that $G \le
\Sym{\Omega}$. If
$\Min(G,S,\le)=\Min(G,T,\le)$ and $\omega \in \Omega$, then $\omega^G$ is empty in $S$ if
and only if it is empty in $T$, and $\omega^G$ is full in $S$ if and only if it is
full in $T$.
\end{lem}

\begin{proof}
  Let $\omega \in \Omega$ and suppose that $\omega^G$ is empty in $S$. As $\omega^G$
  is closed under the action of $G$, and $S_0 := \Min(G,S,\le)$ is an image
  of $S$ under the action of $G$, we see that $\omega^G$ is empty in $S_0$ and hence
  in $T_0:=\Min(G,T,\le)$. Thus $\omega^G$ is empty in $T$, which is an image of $T$
  under the action of $G$. The same arguments work vice versa.

  Next we suppose that $\omega^G$ is full in $S$. Then it is full in $S_0=T_0$ and
  hence in $T$, and the same way we see the converse.
\end{proof}

We can now prove Theorem \ref{thm:minsplit}, which provides the main technique
used to reduce search. This allows us to prove that the minimal image of some set \(S\)
will be smaller than or equal to the minimal image of a set \(T\), without explicitly
calculating the minimal image of either \(S\) or \(T\).

\begin{thm}\label{thm:minsplit}
  Suppose that \(G\) is a permutation group on a totally ordered finite set
  \((\Omega, \le)\) and that \(S\) and \(T\)
  are two subsets of \(\Omega\) where \(|S| = |T|\).

  Suppose further that \(o\) is the first orbit in the list $\Orb(G)$ that
  is neither full in both \(S\) and \(T\) nor empty in both \(S\) and \(T\). If
  \(o\) is empty in \(T\), but not in $S$, then $\Min(G,S,\le)$ is strictly
  smaller than $\Min(G,T,\le)$.
\end{thm}

\begin{proof}
Suppose that \(o\) is empty in \(T\), but not in \(S\). Then $o$ is empty in
$T_0:=\Min(G,T,\le)$, but not in $S_0:=\Min(G,S,\le)$, and in particular $T_0$
and $S_0$ are distinct, as we have seen in Lemma \ref{same}.
%If $\omega \in \Omega$, then the orbit $\omega^G$ is closed under the action of $G$.
%Thus, if $A \subseteq \Omega$
%and \(\omega^G\) is full in \(A\), then for all $g \in G$ it is true that \(\omega^G\) is
%full in \(A^g\). Similarly, if \(\omega^G\) is empty in \(A\), then \(\omega^G\) is empty
%in \(A^g\) for all \(g \in G\).

Let $\alpha$ denote the minimum of the orbit $o$ with respect to $\le$ and let $\omega \in \Omega$. If $\omega < \alpha$,
then
$\omega \notin o$, so the the orbit $\omega^G$ appears in the list $\Orb(G)$
\emph{before} $o$ does. Then the choice of $o$ implies that one of the following
two cases holds:

\begin{enumerate}
\item \(\omega^G\) is full in both \(S\) and \(T\). In particular, for all $g \in G$ we have that
$\omega \in S^g \cap T^g$.
\item \(\omega^G\) is empty in both \(S\) and \(T\). In particular, for all $g \in G$ we have that
$\omega^G \cap S^g=\varnothing$ and $\omega^G \cap T^g=\varnothing$.
\end{enumerate}

If $S_0$ contains an element $\omega\in\Omega$ such that $\omega < \alpha$, then Case (i)
above holds and $\omega \in T_0$. So $S_0 \cap \{\omega' \in \Omega \mid \omega' < \alpha\}= T_0
\cap \{\omega' \in \Omega \mid \omega < \alpha\}$.

Since $S_0$ and $T_0$ are distinct, they must differ amongst the elements
at least as large as $\alpha$ and, since they have the same cardinality, the smallest
such element determines which of $S_0$ and $T_0$ is smaller.

We recall that $o=\alpha^G$ is empty in $T$ and non-empty in $S$, so there exists
some $g \in G$ such that $\alpha \in S^g$. Then $S^g=S_0$ and $\alpha \notin T_0$, so
$S_0$ is strictly smaller that $T_0$.
\end{proof}

Here is an example how to use Theorem \ref{thm:minsplit}.

\begin{ex}
Let $\Omega:=\{1,\dots,10\}$ with natural ordering, and let
$G:=\langle (12), (45),(56), (89)\rangle$. We consider the sets $S:=\{3,6,7\}$
and $T:=\{3,7,9\}$ and we want to calculate the smallest of \(\Min(G,S)\) and
\(\Min(G,T)\). Hence, we want to know which one is smaller as cheaply as
possible, to avoid
superfluous calculations.

We first list the orbits of $G$:
$[\{1,2\},\{3\},\{4,5,6\},\{7\},\{8,9\},\{10\}]$.

Going through the orbits as listed, we see that the first one is empty in $S$
and $T$, the second one is full in $S$ and $T$, and the third one gives a
difference for the first time. It is empty in $T$, but not in $S$, so Theorem
\ref{thm:minsplit} yields that the minimal image of $S$ is strictly smaller than
that of $T$.

\end{ex}

\subsection{Static Orderings of \(\Omega\)}\label{sec:static}

In this section we look at
which total ordering of \(\Omega\) should be used to minimize the amount
of time taken to find minimal images of subsets of $\Omega$.

Given a group \(G\) we will choose an ordering on $\Omega$ such that orbits
with few elements appear as early as possible. In particular, singleton orbits should
appear first.

This is justified by the fact that singleton orbits are always either full or empty. Also,
we would expect smaller orbits to be more likely
to be empty or full than larger orbits. This means that small orbits placed early
in the ordering of \(\Omega\) are
more likely to lead to Theorem \ref{thm:minsplit} being applicable, leading to a
reduction in search.

Algorithm~\ref{alg:minorbit} heuristically chooses a new ordering for an ordered
set \(\Omega\), only depending on the group $G$, under the assumption that the
algorithm that computes minimal images will pick a point from a smallest
non-singleton orbit to branch on.
This will not always be true -- in practice Linton's algorithm branches on the
first orbit which contains some point contained in one of the current candidates
for minimal image.

However, we will show that in Section \ref{sec:ex} that
Algorithm~\ref{alg:minorbit} produces substantially smaller,
and therefore faster, searches in practice.

It is not necessary in Line~\ref{line:point2} of Algorithm~\ref{alg:minorbit} to
choose the smallest element of \(\mathit{Points}\), choosing an arbitrary element will,
on average, perform just as well.
By fixing which point is chosen, we ensure that independent implementations will
produce the same ordering and therefore the same canonical image.


\begin{algorithm}
  \caption{FixedMinOrbit}\label{alg:minorbit}
  \begin{algorithmic}[1]
    \Procedure{MinOrbitOrder}{$\Omega, G$}
    \State{\(\mathit{Remain} := \Omega\)}
    \State{\(\mathit{Order} := []\)}
    \State{\(H := G\)}
    \While{\(|\mathit{Remain}| > 0\)}
		\State{\(\mathit{OrbSize} := \Min\left\{|o|\ \mathrel{\big|}  o \in \Orb(H), o \cap \mathit{Remain} \neq \emptyset\right\}\)}\label{line:orb}
		\State{\(\mathit{Points} := \left\{o \mathrel{\big|} o \in \Orb(H), |o| = \mathit{OrbSize},  o \cap \mathit{Remain} \neq \emptyset\right\}\)}\label{line:point}
		\State{\(\mathit{MinPoint} := \Min\left\{x \mid o \in \mathit{Points}, x \in o\right\}\)}\label{line:point2}
		\State{\(\mathit{Remain} := \mathit{Remain} \backslash \{ \mathit{MinPoint} \}\)}
		\State{\(Add(\mathit{Order}, \mathit{MinPoint})\)}
		\State{\(H := G_{\mathit{MinPoint}}\)}
    \EndWhile
    \State\Return{\(\mathit{Order}\)}
    \EndProcedure
  \end{algorithmic}
\end{algorithm}

We will also consider one simple modification of Algorithm \ref{alg:minorbit}, namely
\texttt{FixedMaxOrbit} (which is the same as \texttt{FixedMinOrbit}) with line
  \ref{line:orb} changed to pick orbits of maximum size.

If our intuition about Theorem \ref{thm:minsplit} is correct, then
\texttt{MaxOrbit} should almost
always produce a larger search than \texttt{MinOrbit} or a random ordering
of \(\Omega\).

\subsection{Implementing alternative orderings of \(\Omega\)}

Having calculated an alternative \(Order\) using \textbf{FixedMinOrbit} or \textbf{FixedMaxOrbit},
we could create a version of \textbf{MinimalImage} which accepted an explicit
ordering. However, rather than editing the algorithm, we can instead perform a
pre-processing step, using Lemma~\ref{lem:map}.

\begin{lem}\label{lem:map}
  Consider a group \(G\) that acts on $\Omega = \{1,\dots,n\}$ and a
  permutation \(\sigma \in \Sym{\Omega}\). We define an ordering \(\leq_\sigma\) on
  \(\{1,\dots,n\}\), where for all $x,y \in \Omega$ we have that \(x \leq_\sigma y\) if and only if
  \(x^\sigma \leq y^\sigma\).

  For the induced orderings $\preccurlyeq$ and $\preccurlyeq_\sigma$
  on subsets of $\Omega$ as in Definition \ref{metaorder}
  it holds that
  \[ X \preccurlyeq_\sigma  Y \Leftrightarrow X^\sigma \preccurlyeq Y^\sigma \]
  for all subsets $X$ and $Y$ of $\Omega$, and hence (simplifying notation)
  \[\Min(G,S,\preccurlyeq_\sigma) = \Min(G^\sigma,S^\sigma,\preccurlyeq)^{\sigma^{-1}}.\]
\end{lem}

\begin{proof}

  Following Definition \ref{metaorder}, $X \preccurlyeq_\sigma Y$ if and only if
  there is an $x \in X$ such that $x\not\in Y$ and for all
  $y \in Y \backslash X$ it holds that $x \leq_\sigma y$.
  By definition of $\leq_\sigma$, this is the case whenever $x^\sigma \leq y^\sigma$,
  and since $x^\sigma \in X^\sigma$ and for all $y^\sigma$ in $Y^\sigma \backslash X^\sigma$ it
  holds that $x^\sigma \leq y^\sigma$, it follows that $X^\sigma \preccurlyeq Y^\sigma$.

  \medskip

  Consider the map $\varphi_\sigma : S^G \rightarrow (S^\sigma)^{G^\sigma}$ that maps sets
  $X \in S^G$ to $X^\sigma \in (S^\sigma)^{G^\sigma}$. This map is bijective, and by the above it respects the ordering, so the second claim follows.


\end{proof}

Lemma~\ref{lem:map} gives an efficient method to calculate minimal images under
different orderings without having to alter the underlying algorithm. The most
expensive part of this algorithm is calculating \(G^\sigma\), but this is still very
efficiently implemented in systems such as GAP, and also can be cached so it
only has to be calculated once for a given \(G\) and \(\sigma\).



\section{Dynamic Ordering of \(\Omega\)}

In Section \ref{sec:static}, we looked at methods for choosing an ordering for
\(\Omega\) that allows a minimal image algorithm to search more quickly.
There is a major limitation to this technique -- it does not make use of
the sets whose canonical image we wish to find.

In this section, instead of producing an ordering ahead of time, we will
incrementally define the ordering of \(\Omega\) as the algorithm
progresses.
At each stage we will consider exactly which extension of our partially
constructed ordering will lead to the smallest increase in the number of sets we
must consider.

We are not free to choose our ordering arbitrarily as we must still map two
sets in the same orbit of \(G\) to the same canonical image.
However, we can use different orderings for sets that are in different
orbits of \(G\).

Firstly, we will explain how we build the orderings that our algorithm uses.

%% This declares a command \Comment
%% The argument will be surrounded by /* ... */
\SetKwComment{Comment}{/* }{ */}

\begin{algorithm}[t]
\caption{Training Scheduler}\label{alg:TS}
% \KwData{$n \geq 0$}
% \KwResult{$y = x^n$}
\LinesNumbered
\KwIn{Training data $\mathcal{D}_{train}=\{(q_i, a_i, p_i^+)\}_{i=1}^m$, \\
\qquad \quad Iteration number $L$.}
\KwOut{A set of optimal model parameters.}

\For{$l=1,\cdots, L$}{
    Sample a batch of questions $Q^{(l)}$\\
    \For{$q_i\in Q^{(l)}$}{
        $\mathcal{P}_{i}^{(l)} \gets \mathrm{arg\,max}_{p_{i,j}}(\mathrm{sim}(q_i^{en},p_{i,j}),K)$\\
        $\mathcal{P}_{Gi}^{(l)} \gets \mathcal{P}_{i}^{(l)}\cup\{p^+_i\}$\\
        Compute $\mathcal{L}^i_{retriever}$, $\mathcal{L}^i_{postranker}$, $\mathcal{L}^i_{reader}$\\ according to Eq.\ref{eq:retriever}, Eq.\ref{eq:rerank}, Eq.\ref{eq:reader}\\
    }
    % $\mathcal{L}^{(l)}_{retriever} \gets \frac{1}{|Q^{(l)}|}\sum_i\mathcal{L}^i_{retriever}$\\
    % $\mathcal{L}^{(l)}_{retriever} \gets \mathrm{Avg}(\mathcal{L}^i_{retriever})$,
    % $\mathcal{L}^{(l)}_{rerank} \gets \mathrm{Avg}(\mathcal{L}^i_{rerank})$,
    % $\mathcal{L}^{(l)}_{reader} \gets \mathrm{Avg}(\mathcal{L}^i_{reader})$\\
    % Compute $\mathcal{L}^{(l)}_{retriever}$, $\mathcal{L}^{(l)}_{rerank}$, and $\mathcal{L}^{(l)}_{reader}$ by averaging over $Q^{(l)}$\\
    $\mathcal{L}^{(l)} \gets \frac{1}{|Q^{(l)}|}\sum_i(\mathcal{L}^{i}_{retriever} + \mathcal{L}^{i}_{postranker}+ \mathcal{L}^{i}_{reader})$\\
    $\mathcal{P}^{(l)}_K\gets\{\mathcal{P}^{(l)}_i|q_i\in Q^{(l)}\}$,\quad $\mathcal{P}^{(l)}_{KG}\gets\{\mathcal{P}^{(l)}_{Gi}|q_i\in Q^{(l)}\}$\\
    Compute the coefficient $v^{(l)}$ according to Eq.~\ref{eq:v}\\
  \eIf{$ v^{(l)}=1$}{
    $\mathcal{L}^{(l)}_{final} \gets \mathcal{L}^{(l)}(\mathcal{P}_{KG}^{(l)})$\\
  }{
      $\mathcal{L}^{(l)}_{final} \gets \mathcal{L}^{(l)}(\mathcal{P}^{(l)}_{K}),$\\
    }
    Optimize $\mathcal{L}^{(l)}_{final}$
}
\end{algorithm}


%  \eIf{$ \mathcal{L}^{(l-1)}_{retriever}<\lambda$}{
%     $\mathcal{L}^{(l)}_{final} \gets \mathcal{L}^{(l)}(\mathcal{P}_K^{(l)})$\\
%   }{
%       $\mathcal{L}^{(l)}_{final} \gets \mathcal{L}^{(l)}(\mathcal{P}^{(l)}_{KG}),$\\
%     }

In this section we conduct comprehensive experiments to emphasise the effectiveness of DIAL, including evaluations under white-box and black-box settings, robustness to unforeseen adversaries, robustness to unforeseen corruptions, transfer learning, and ablation studies. Finally, we present a new measurement to test the balance between robustness and natural accuracy, which we named $F_1$-robust score. 


\subsection{A case study on SVHN and CIFAR-100}
In the first part of our analysis, we conduct a case study experiment on two benchmark datasets: SVHN \citep{netzer2011reading} and CIFAR-100 \cite{krizhevsky2009learning}. We follow common experiment settings as in \cite{rice2020overfitting, wu2020adversarial}. We used the PreAct ResNet-18 \citep{he2016identity} architecture on which we integrate a domain classification layer. The adversarial training is done using 10-step PGD adversary with perturbation size of 0.031 and a step size of 0.003 for SVHN and 0.007 for CIFAR-100. The batch size is 128, weight decay is $7e^{-4}$ and the model is trained for 100 epochs. For SVHN, the initial learinnig rate is set to 0.01 and decays by a factor of 10 after 55, 75 and 90 iteration. For CIFAR-100, the initial learning rate is set to 0.1 and decays by a factor of 10 after 75 and 90 iterations. 
%We compared DIAL to \cite{madry2017towards} and TRADES \citep{zhang2019theoretically}. 
%The evaluation is done using Auto-Attack~\citep{croce2020reliable}, which is an ensemble of three white-box and one black-box parameter-free attacks, and various $\ell_{\infty}$ adversaries: PGD$^{20}$, PGD$^{100}$, PGD$^{1000}$ and CW$_{\infty}$ with step size of 0.003. 
Results are averaged over 3 restarts while omitting one standard deviation (which is smaller than 0.2\% in all experiments). As can be seen by the results in Tables~\ref{black-and_white-svhn} and \ref{black-and_white-cifar100}, DIAL presents consistent improvement in robustness (e.g., 5.75\% improved robustness on SVHN against AA) compared to the standard AT 
%under variety of attacks 
while also improving the natural accuracy. More results are presented in Appendix \ref{cifar100-svhn-appendix}.


\begin{table}[!ht]
  \caption{Robustness against white-box, black-box attacks and Auto-Attack (AA) on SVHN. Black-box attacks are generated using naturally trained surrogate model. Natural represents the naturally trained (non-adversarial) model.
  %and applied to the best performing robust models.
  }
  \vskip 0.1in
  \label{black-and_white-svhn}
  \centering
  \small
  \begin{tabular}{l@{\hspace{1\tabcolsep}}c@{\hspace{1\tabcolsep}}c@{\hspace{1\tabcolsep}}c@{\hspace{1\tabcolsep}}c@{\hspace{1\tabcolsep}}c@{\hspace{1\tabcolsep}}c@{\hspace{1\tabcolsep}}c@{\hspace{1\tabcolsep}}c@{\hspace{1\tabcolsep}}c@{\hspace{1\tabcolsep}}c}
    \toprule
    & & \multicolumn{4}{c}{White-box} & \multicolumn{4}{c}{Black-Box}  \\
    \cmidrule(r){3-6} 
    \cmidrule(r){7-10}
    Defense Model & Natural & PGD$^{20}$ & PGD$^{100}$  & PGD$^{1000}$  & CW$^{\infty}$ & PGD$^{20}$ & PGD$^{100}$ & PGD$^{1000}$  & CW$^{\infty}$ & AA \\
    \midrule
    NATURAL & 96.85 & 0 & 0 & 0 & 0 & 0 & 0 & 0 & 0 & 0 \\
    \midrule
    AT & 89.90 & 53.23 & 49.45 & 49.23 & 48.25 & 86.44 & 86.28 & 86.18 & 86.42 & 45.25 \\
    % TRADES & 90.35 & 57.10 & 54.13 & 54.08 & 52.19 & 86.89 & 86.73 & 86.57 & 86.70 &  49.50 \\
    $\DIAL_{\kl}$ (Ours) & 90.66 & \textbf{58.91} & \textbf{55.30} & \textbf{55.11} & \textbf{53.67} & 87.62 & 87.52 & 87.41 & 87.63 & \textbf{51.00} \\
    $\DIAL_{\ce}$ (Ours) & \textbf{92.88} & 55.26  & 50.82 & 50.54 & 49.66 & \textbf{89.12} & \textbf{89.01} & \textbf{88.74} & \textbf{89.10} &  46.52  \\
    \bottomrule
  \end{tabular}
\end{table}


\begin{table}[!ht]
  \caption{Robustness against white-box, black-box attacks and Auto-Attack (AA) on CIFAR100. Black-box attacks are generated using naturally trained surrogate model. Natural represents the naturally trained (non-adversarial) model.
  %and applied to the best performing robust models.
  }
  \vskip 0.1in
  \label{black-and_white-cifar100}
  \centering
  \small
  \begin{tabular}{l@{\hspace{1\tabcolsep}}c@{\hspace{1\tabcolsep}}c@{\hspace{1\tabcolsep}}c@{\hspace{1\tabcolsep}}c@{\hspace{1\tabcolsep}}c@{\hspace{1\tabcolsep}}c@{\hspace{1\tabcolsep}}c@{\hspace{1\tabcolsep}}c@{\hspace{1\tabcolsep}}c@{\hspace{1\tabcolsep}}c}
    \toprule
    & & \multicolumn{4}{c}{White-box} & \multicolumn{4}{c}{Black-Box}  \\
    \cmidrule(r){3-6} 
    \cmidrule(r){7-10}
    Defense Model & Natural & PGD$^{20}$ & PGD$^{100}$  & PGD$^{1000}$  & CW$^{\infty}$ & PGD$^{20}$ & PGD$^{100}$ & PGD$^{1000}$  & CW$^{\infty}$ & AA \\
    \midrule
    NATURAL & 79.30 & 0 & 0 & 0 & 0 & 0 & 0 & 0 & 0 & 0 \\
    \midrule
    AT & 56.73 & 29.57 & 28.45 & 28.39 & 26.6 & 55.52 & 55.29 & 55.26 & 55.40 & 24.12 \\
    % TRADES & 58.24 & 30.10 & 29.66 & 29.64 & 25.97 & 57.05 & 56.71 & 56.67 & 56.77 & 24.92 \\
    $\DIAL_{\kl}$ (Ours) & 58.47 & \textbf{31.19} & \textbf{30.50} & \textbf{30.42} & \textbf{26.91} & 57.16 & 56.81 & 56.80 & 57.00 & \textbf{25.87} \\
    $\DIAL_{\ce}$ (Ours) & \textbf{60.77} & 27.87 & 26.66 & 26.61 & 25.98 & \textbf{59.48} & \textbf{59.06} & \textbf{58.96} & \textbf{59.20} & 23.51  \\
    \bottomrule
  \end{tabular}
\end{table}


% \begin{table}[!ht]
%   \caption{Robustness comparison of DIAL to Madry et al. and TRADES defense models on the SVHN dataset under different PGD white-box attacks and the ensemble Auto-Attack (AA).}
%   \label{svhn}
%   \centering
%   \begin{tabular}{llllll|l}
%     \toprule
%     \cmidrule(r){1-5}
%     Defense Model & Natural & PGD$^{20}$ & PGD$^{100}$ & PGD$^{1000}$ & CW$_{\infty}$ & AA\\
%     \midrule
%     $\DIAL_{\kl}$ (Ours) & $\mathbf{90.66}$ & $\mathbf{58.91}$ & $\mathbf{55.30}$ & $\mathbf{55.12}$ & $\mathbf{53.67}$  & $\mathbf{51.00}$  \\
%     Madry et al. & 89.90 & 53.23 & 49.45 & 49.23 & 48.25 & 45.25  \\
%     TRADES & 90.35 & 57.10 & 54.13 & 54.08 & 52.19 & 49.50 \\
%     \bottomrule
%   \end{tabular}
% \end{table}


\subsection{Performance comparison on CIFAR-10} \label{defence-settings}
In this part, we evaluate the performance of DIAL compared to other well-known methods on CIFAR-10. 
%To be consistent with other methods, 
We follow the same experiment setups as in~\cite{madry2017towards, wang2019improving, zhang2019theoretically}. When experiment settings are not identical between tested methods, we choose the most commonly used settings, and apply it to all experiments. This way, we keep the comparison as fair as possible and avoid reporting changes in results which are caused by inconsistent experiment settings \citep{pang2020bag}. To show that our results are not caused because of what is referred to as \textit{obfuscated gradients}~\citep{athalye2018obfuscated}, we evaluate our method with same setup as in our defense model, under strong attacks (e.g., PGD$^{1000}$) in both white-box, black-box settings, Auto-Attack ~\citep{croce2020reliable}, unforeseen "natural" corruptions~\citep{hendrycks2018benchmarking}, and unforeseen adversaries. To make sure that the reported improvements are not caused by \textit{adversarial overfitting}~\citep{rice2020overfitting}, we report best robust results for each method on average of 3 restarts, while omitting one standard deviation (which is smaller than 0.2\% in all experiments). Additional results for CIFAR-10 as well as comprehensive evaluation on MNIST can be found in Appendix \ref{mnist-results} and \ref{additional_res}.
%To further keep the comparison consistent, we followed the same attack settings for all methods.


\begin{table}[ht]
  \caption{Robustness against white-box, black-box attacks and Auto-Attack (AA) on CIFAR-10. Black-box attacks are generated using naturally trained surrogate model. Natural represents the naturally trained (non-adversarial) model.
  %and applied to the best performing robust models.
  }
  \vskip 0.1in
  \label{black-and_white-cifar}
  \centering
  \small
  \begin{tabular}{cccccccc@{\hspace{1\tabcolsep}}c}
    \toprule
    & & \multicolumn{3}{c}{White-box} & \multicolumn{3}{c}{Black-Box} \\
    \cmidrule(r){3-5} 
    \cmidrule(r){6-8}
    Defense Model & Natural & PGD$^{20}$ & PGD$^{100}$ & CW$^{\infty}$ & PGD$^{20}$ & PGD$^{100}$ & CW$^{\infty}$ & AA \\
    \midrule
    NATURAL & 95.43 & 0 & 0 & 0 & 0 & 0 & 0 &  0 \\
    \midrule
    TRADES & 84.92 & 56.60 & 55.56 & 54.20 & 84.08 & 83.89 & 83.91 &  53.08 \\
    MART & 83.62 & 58.12 & 56.48 & 53.09 & 82.82 & 82.52 & 82.80 & 51.10 \\
    AT & 85.10 & 56.28 & 54.46 & 53.99 & 84.22 & 84.14 & 83.92 & 51.52 \\
    ATDA & 76.91 & 43.27 & 41.13 & 41.01 & 75.59 & 75.37 & 75.35 & 40.08\\
    $\DIAL_{\kl}$ (Ours) & 85.25 & $\mathbf{58.43}$ & $\mathbf{56.80}$ & $\mathbf{55.00}$ & 84.30 & 84.18 & 84.05 & \textbf{53.75} \\
    $\DIAL_{\ce}$ (Ours)  & $\mathbf{89.59}$ & 54.31 & 51.67 & 52.04 &$ \mathbf{88.60}$ & $\mathbf{88.39}$ & $\mathbf{88.44}$ & 49.85 \\
    \midrule
    $\DIAL_{\awp}$ (Ours) & $\mathbf{85.91}$ & $\mathbf{61.10}$ & $\mathbf{59.86}$ & $\mathbf{57.67}$ & $\mathbf{85.13}$ & $\mathbf{84.93}$ & $\mathbf{85.03}$  & \textbf{56.78} \\
    $\TRADES_{\awp}$ & 85.36 & 59.27 & 59.12 & 57.07 & 84.58 & 84.58 & 84.59 & 56.17 \\
    \bottomrule
  \end{tabular}
\end{table}



\paragraph{CIFAR-10 setup.} We use the wide residual network (WRN-34-10)~\citep{zagoruyko2016wide} architecture. %used in the experiments of~\cite{madry2017towards, wang2019improving, zhang2019theoretically}. 
Sidelong this architecture, we integrate a domain classification layer. To generate the adversarial domain dataset, we use a perturbation size of $\epsilon=0.031$. We apply 10 of inner maximization iterations with perturbation step size of 0.007. Batch size is set to 128, weight decay is set to $7e^{-4}$, and the model is trained for 100 epochs. Similar to the other methods, the initial learning rate was set to 0.1, and decays by a factor of 10 at iterations 75 and 90. 
%For being consistent with other methods, the natural images are padded with 4-pixel padding with 32-random crop and random horizontal flip. Furthermore, all methods are trained using SGD with momentum 0.9. For $\DIAL_{\kl}$, we balance the robust loss with $\lambda=6$ and the domains loss with $r=4$. For $\DIAL_{\ce}$, we balance the robust loss with $\lambda=1$ and the domains loss with $r=2$. 
%We also introduce a version of our method that incorporates the AWP double-perturbation mechanism, named DIAL-AWP.
%which is trained using the same learning rate schedule used in ~\cite{wu2020adversarial}, where the initial 0.1 learning rate decays by a factor of 10 after 100 and 150 iterations. 
See Appendix \ref{cifar10-additional-setup} for additional details.

\begin{table}[ht]
  \caption{Black-box attack using the adversarially trained surrogate models on CIFAR-10.}
  \vskip 0.1in
  \label{black-box-cifar-adv}
  \centering
  \small
  \begin{tabular}{ll|c}
    \toprule
    \cmidrule(r){1-2}
    Surrogate (source) model & Target model & robustness \% \\
    % \midrule
    \midrule
    TRADES & $\DIAL_{\ce}$ & $\mathbf{67.77}$ \\
    $\DIAL_{\ce}$ & TRADES & 65.75 \\
    \midrule
    MART & $\DIAL_{\ce}$ & $\mathbf{70.30}$ \\
    $\DIAL_{\ce}$ & MART & 64.91 \\
    \midrule
    AT & $\DIAL_{\ce}$ & $\mathbf{65.32}$ \\
    $\DIAL_{\ce}$ & AT  & 63.54 \\
    \midrule
    ATDA & $\DIAL_{\ce}$ & $\mathbf{66.77}$ \\
    $\DIAL_{\ce}$ & ATDA & 52.56 \\
    \bottomrule
  \end{tabular}
\end{table}

\paragraph{White-box/Black-box robustness.} 
%We evaluate all defense models using Auto-Attack, PGD$^{20}$, PGD$^{100}$, PGD$^{1000}$ and CW$_{\infty}$ with step size 0.003. We constrain all attacks by the same perturbation $\epsilon=0.031$. 
As reported in Table~\ref{black-and_white-cifar} and Appendix~\ref{additional_res}, our method achieves better robustness compared to the other methods. Specifically, in the white-box settings, our method improves robustness over~\citet{madry2017towards} and TRADES by 2\% 
%using the common PGD$^{20}$ attack 
while keeping higher natural accuracy. We also observe better natural accuracy of 1.65\% over MART while also achieving better robustness over all attacks. Moreover, our method presents significant improvement of up to 15\% compared to the the domain invariant method suggested by~\citet{song2018improving} (ATDA).
%in both natural and robust accuracy. 
When incorporating 
%the double-perturbation mechanism of 
AWP, our method improves the results of $\TRADES_{\awp}$ by almost 2\%.
%and reaches state-of-the-art results for robust models with no additional data. 
% Additional results are available in Appendix~\ref{additional_res}.
When tested on black-box settings, $\DIAL_{\ce}$ presents a significant improvement of more than 4.4\% over the second-best performing method, and up to 13\%. In Table~\ref{black-box-cifar-adv}, we also present the black-box results when the source model is taken from one of the adversarially trained models. %Then, we compare our model to each one of them both as the source model and target model. 
In addition to the improvement in black-box robustness, $\DIAL_{\ce}$ also manages to achieve better clean accuracy of more than 4.5\% over the second-best performing method.
% Moreover, based on the auto-attack leader-board \footnote{\url{https://github.com/fra31/auto-attack}}, our method achieves the 1st place among models without additional data using the WRN-34-10 architecture.

% \begin{table}
%   \caption{White-box robustness on CIFAR-10 using WRN-34-10}
%   \label{white-box-cifar-10}
%   \centering
%   \begin{tabular}{lllll}
%     \toprule
%     \cmidrule(r){1-2}
%     Defense Model & Natural & PGD$^{20}$ & PGD$^{100}$ & PGD$^{1000}$ \\
%     \midrule
%     TRADES ~\cite{zhang2019theoretically} & 84.92  & 56.6 & 55.56 & 56.43  \\
%     MART ~\cite{wang2019improving} & 83.62  & 58.12 & 56.48 & 56.55  \\
%     Madry et al. ~\cite{madry2017towards} & 85.1  & 56.28 & 54.46 & 54.4  \\
%     Song et al. ~\cite{song2018improving} & 76.91 & 43.27 & 41.13 & 41.02  \\
%     $\DIAL_{\ce}$ (Ours) & $ \mathbf{90}$  & 52.12 & 48.88 & 48.78  \\
%     $\DIAL_{\kl}$ (Ours) & 85.25 & $\mathbf{58.43}$ & $\mathbf{56.8}$ & $\mathbf{56.73}$ \\
%     \midrule
%     $\DIAL_{\kl}$+AWP (Ours) & $\mathbf{85.91}$ & $\mathbf{61.1}$ & - & -  \\
%     TRADES+AWP \cite{wu2020adversarial} & 85.36 & 59.27 & 59.12 & -  \\
%     % MART + AWP & 84.43 & 60.68 & 59.32 & -  \\
%     \bottomrule
%   \end{tabular}
% \end{table}


% \begin{table}
%   \caption{White-box robustness on MNIST}
%   \label{white-box-mnist}
%   \centering
%   \begin{tabular}{llllll}
%     \toprule
%     \cmidrule(r){1-2}
%     Defense Model & Natural & PGD$^{40}$ & PGD$^{100}$ & PGD$^{1000}$ \\
%     \midrule
%     TRADES ~\cite{zhang2019theoretically} & 99.48 & 96.07 & 95.52 & 95.22 \\
%     MART ~\cite{wang2019improving} & 99.38  & 96.99 & 96.11 & 95.74  \\
%     Madry et al. ~\cite{madry2017towards} & 99.41  & 96.01 & 95.49 & 95.36 \\
%     Song et al. ~\cite{song2018improving}  & 98.72 & 96.82 & 96.26 & 96.2  \\
%     $\DIAL_{\kl}$ (Ours) & 99.46 & 97.05 & 96.06 & 95.99  \\
%     $\DIAL_{\ce}$ (Ours) & $\mathbf{99.49}$  & $\mathbf{97.38}$ & $\mathbf{96.45}$ & $\mathbf{96.33}$ \\
%     \bottomrule
%   \end{tabular}
% \end{table}


% \paragraph{Attacking MNIST.} For consistency, we use the same perturbation and step sizes. For MNIST, we use $\epsilon=0.3$ and step size of $0.01$. The natural accuracy of our surrogate (source) model is 99.51\%. Attacks results are reported in Table~\ref{black-and_white-mnist}. It is worth noting that the improvement margin is not conclusive on MNIST as it is on CIFAR-10, which is a more complex task.

% \begin{table}
%   \caption{Black-box robustness on MNIST and CIFAR-10 using naturally trained surrogate model and best performing robust models}
%   \label{black-box-mnist-cifar}
%   \centering
%   \begin{tabular}{lllllll}
%     \toprule
%     & \multicolumn{3}{c}{MNIST} & \multicolumn{3}{c}{CIFAR-10} \\
%     \cmidrule(r){2-4} 
%     \cmidrule(r){5-7}  
%     Defense Model & PGD$^{40}$ & PGD$^{100}$ & PGD$^{1000}$ & PGD$^{20}$ & PGD$^{100}$ & PGD$^{1000}$ \\
%     \midrule
%     TRADES ~\cite{zhang2019theoretically} & 98.12 & 97.86 & 97.81 & 84.08 & 83.89 & 83.8 \\
%     MART ~\cite{wang2019improving} & 98.16 & 97.96 & 97.89  & 82.82 & 82.52 & 82.47 \\
%     Madry et al. ~\cite{madry2017towards}  & 98.05 & 97.73 & 97.78 & 84.22 & 84.14 & 83.96 \\
%     Song et al. ~\cite{song2018improving} & 97.74 & 97.28 & 97.34 & 75.59 & 75.37 & 75.11 \\
%     $\DIAL_{\kl}$ (Ours) & 98.14 & 97.83 & 97.87  & 84.3 & 84.18 & 84.0 \\
%     $\DIAL_{\ce}$ (Ours)  & $\mathbf{98.37}$ & $\mathbf{98.12}$ & $\mathbf{98.05}$  & $\mathbf{89.13}$ & $\mathbf{88.89}$ & $\mathbf{88.78}$ \\
%     \bottomrule
%   \end{tabular}
% \end{table}



% \subsubsection{Ensemble attack} In addition to the white-box and black-box settings, we evaluate our method on the Auto-Attack ~\citep{croce2020reliable} using $\ell_{\infty}$ threat model with perturbation $\epsilon=0.031$. Auto-Attack is an ensemble of parameter-free attacks. It consists of three white-box attacks: APGD-CE which is a step size-free version of PGD on the cross-entropy ~\citep{croce2020reliable}. APGD-DLR which is a step size-free version of PGD on the DLR loss ~\citep{croce2020reliable} and FAB which  minimizes the norm of the adversarial perturbations, and one black-box attack: square attack which is a query-efficient black-box attack ~\citep{andriushchenko2020square}. Results are presented in Table~\ref{auto-attack}. Based on the auto-attack leader-board \footnote{\url{https://github.com/fra31/auto-attack}}, our method achieves the 1st place among models without additional data using the WRN-34-10 architecture.

%Additional results can be found in Appendix ~\ref{additional_res}.

% \begin{table}
%   \caption{Auto-Attack (AA) on CIFAR-10 with perturbation size $\epsilon=0.031$ with $\ell_{\infty}$ threat model}
%   \label{auto-attack}
%   \centering
%   \begin{tabular}{lll}
%     \toprule
%     \cmidrule(r){1-2}
%     Defense Model & AA \\
%     \midrule
%     TRADES ~\cite{zhang2019theoretically} & 53.08  \\
%     MART ~\cite{wang2019improving} & 51.1  \\
%     Madry et al. ~\cite{madry2017towards} & 51.52    \\
%     Song et al.   ~\cite{song2018improving} & 40.18 \\
%     $\DIAL_{\ce}$ (Ours) & 47.33  \\
%     $\DIAL_{\kl}$ (Ours) & $\mathbf{53.75}$ \\
%     \midrule
%     DIAL-AWP (Ours) & $\mathbf{56.78}$ \\
%     TRADES-AWP \cite{wu2020adversarial} & 56.17 \\
%     \bottomrule
%   \end{tabular}
% \end{table}


% \begin{table}[!ht]
%   \caption{Auto-Attack (AA) Robustness (\%) on CIFAR-10 with $\epsilon=0.031$ using an $\ell_{\infty}$ threat model}
%   \label{auto-attack}
%   \centering
%   \begin{tabular}{cccccc|cc}
%     \toprule
%     % \multicolumn{8}{c}{Defence Model}  \\
%     % \cmidrule(r){1-8} 
%     TRADES & MART & Madry & Song & $\DIAL_{\ce}$ & $\DIAL_{\kl}$ & DIAL-AWP  & TRADES-AWP\\
%     \midrule
%     53.08 & 51.10 & 51.52 &  40.08 & 47.33  & $\mathbf{53.75}$ & $\mathbf{56.78}$ & 56.17 \\

%     \bottomrule
%   \end{tabular}
% \end{table}

% \begin{table}[!ht]
% \caption{$F_1$-robust measurement using PGD$^{20}$ attack in white-box and black-box settings on CIFAR-10}
%   \label{f1-robust}
%   \centering
%   \begin{tabular}{ccccccc|cc}
%     \toprule
%     % \multicolumn{8}{c}{Defence Model}  \\
%     % \cmidrule(r){1-8} 
%     Defense Model & TRADES & MART & Madry & Song & $\DIAL_{\kl}$ & $\DIAL_{\ce}$ & DIAL-AWP  & TRADES-AWP\\
%     \midrule
%     White-box & 0.659 & 0.666 & 0.657 & 0.518 & $\mathbf{0.675}$  & 0.643 & $\mathbf{0.698}$ & 0.682 \\
%     Black-box & 0.844 & 0.831 & 0.846 & 0.761 & 0.847 & $\mathbf{0.895}$ & $\mathbf{0.854}$ &  0.849 \\
%     \bottomrule
%   \end{tabular}
% \end{table}

\subsubsection{Robustness to Unforeseen Attacks and Corruptions}
\paragraph{Unforeseen Adversaries.} To further demonstrate the effectiveness of our approach, we test our method against various adversaries that were not used during the training process. We attack the model under the white-box settings with $\ell_{2}$-PGD, $\ell_{1}$-PGD, $\ell_{\infty}$-DeepFool and $\ell_{2}$-DeepFool \citep{moosavi2016deepfool} adversaries using Foolbox \citep{rauber2017foolbox}. We applied commonly used attack budget 
%(perturbation for PGD adversaries and overshot for DeepFool adversaries) 
with 20 and 50 iterations for PGD and DeepFool, respectively.
Results are presented in Table \ref{unseen-attacks}. As can be seen, our approach  gains an improvement of up to 4.73\% over the second best method under the various attack types and an average improvement of 3.7\% over all threat models.


\begin{table}[ht]
  \caption{Robustness on CIFAR-10 against unseen adversaries under white-box settings.}
  \vskip 0.1in
  \label{unseen-attacks}
  \centering
%   \small
  \begin{tabular}{c@{\hspace{1.5\tabcolsep}}c@{\hspace{1.5\tabcolsep}}c@{\hspace{1.5\tabcolsep}}c@{\hspace{1.5\tabcolsep}}c@{\hspace{1.5\tabcolsep}}c@{\hspace{1.5\tabcolsep}}c@{\hspace{1.5\tabcolsep}}c}
    \toprule
    Threat Model & Attack Constraints & $\DIAL_{\kl}$ & $\DIAL_{\ce}$ & AT & TRADES & MART & ATDA \\
    \midrule
    \multirow{2}{*}{$\ell_{2}$-PGD} & $\epsilon=0.5$ & 76.05 & \textbf{80.51} & 76.82 & 76.57 & 75.07 & 66.25 \\
    & $\epsilon=0.25$ & 80.98 & \textbf{85.38} & 81.41 & 81.10 & 80.04 & 71.87 \\\midrule
    \multirow{2}{*}{$\ell_{1}$-PGD} & $\epsilon=12$ & 74.84 & \textbf{80.00} & 76.17 & 75.52 & 75.95 & 65.76 \\
    & $\epsilon=7.84$ & 78.69 & \textbf{83.62} & 79.86 & 79.16 & 78.55 & 69.97 \\
    \midrule
    $\ell_{2}$-DeepFool & overshoot=0.02 & 84.53 & \textbf{88.88} & 84.15 & 84.23 & 82.96 & 76.08 \\\midrule
    $\ell_{\infty}$-DeepFool & overshoot=0.02 & 68.43 & \textbf{69.50} & 67.29 & 67.60 & 66.40 & 57.35 \\
    \bottomrule
  \end{tabular}
\end{table}


%%%%%%%%%%%%%%%%%%%%%%%%% conference version %%%%%%%%%%%%%%%%%%%%%%%%%%%%%%%%%%%%%
\paragraph{Unforeseen Corruptions.}
We further demonstrate that our method consistently holds against unforeseen ``natural'' corruptions, consists of 18 unforeseen diverse corruption types proposed by \citet{hendrycks2018benchmarking} on CIFAR-10, which we refer to as CIFAR10-C. The CIFAR10-C benchmark covers noise, blur, weather, and digital categories. As can be shown in Figure \ref{corruption}, our method gains a significant and consistent improvement over all the other methods. Our method leads to an average improvement of 4.7\% with minimum improvement of 3.5\% and maximum improvement of 5.9\% compared to the second best method over all unforeseen attacks. See Appendix \ref{corruptions-apendix} for the full experiment results.


\begin{figure}[h]
 \centering
  \includegraphics[width=0.4\textwidth]{figures/spider_full.png}
%   \caption{Summary of accuracy over all unforeseen corruptions compared to the second and third best performing methods.}
  \caption{Accuracy comparison over all unforeseen corruptions.}
  \label{corruption}
\end{figure}


%%%%%%%%%%%%%%%%%%%%%%%%% conference version %%%%%%%%%%%%%%%%%%%%%%%%%%%%%%%%%%%%%

%%%%%%%%%%%%%%%%%%%%%%%%% Arxiv version %%%%%%%%%%%%%%%%%%%%%%%%%%%%%%%%%%%%%
% \newpage
% \paragraph{Unforeseen Corruptions.}
% We further demonstrate that our method consistently holds against unforeseen "natural" corruptions, consists of 18 unforeseen diverse corruption types proposed by \cite{hendrycks2018benchmarking} on CIFAR-10, which we refer to as CIFAR10-C. The CIFAR10-C benchmark covers noise, blur, weather, and digital categories. As can be shown in Figure  \ref{spider-full-graph}, our method gains a significant and consistent improvement over all the other methods. Our approach leads to an average improvement of 4.7\% with minimum improvement of 3.5\% and maximum improvement of 5.9\% compared to the second best method over all unforeseen attacks. Full accuracy results against unforeseen corruptions are presented in Tables \ref{corruption-table1} and \ref{corruption-table2}. 

% \begin{table}[!ht]
%   \caption{Accuracy (\%) against unforeseen corruptions.}
%   \label{corruption-table1}
%   \centering
%   \tiny
%   \begin{tabular}{lcccccccccccccccccc}
%     \toprule
%     Defense Model & brightness & defocus blur & fog & glass blur & jpeg compression & motion blur & saturate & snow & speckle noise  \\
%     \midrule
%     TRADES & 82.63 & 80.04 & 60.19 & 78.00 & 82.81 & 76.49 & 81.53 & 80.68 & 80.14 \\
%     MART & 80.76 & 78.62 & 56.78 & 76.60 & 81.26 & 74.58 & 80.74 & 78.22 & 79.42 \\
%     AT &  83.30 & 80.42 & 60.22 & 77.90 & 82.73 & 76.64 & 82.31 & 80.37 & 80.74 \\
%     ATDA & 72.67 & 69.36 & 45.52 & 64.88 & 73.22 & 63.47 & 72.07 & 68.76 & 72.27 \\
%     DIAL (Ours)  & \textbf{87.14} & \textbf{84.84} & \textbf{66.08} & \textbf{81.82} & \textbf{87.07} & \textbf{81.20} & \textbf{86.45} & \textbf{84.18} & \textbf{84.94} \\
%     \bottomrule
%   \end{tabular}
% \end{table}


% \begin{table}[!ht]
%   \caption{Accuracy (\%) against unforeseen corruptions.}
%   \label{corruption-table2}
%   \centering
%   \tiny
%   \begin{tabular}{lcccccccccccccccccc}
%     \toprule
%     Defense Model & contrast & elastic transform & frost & gaussian noise & impulse noise & pixelate & shot noise & spatter & zoom blur \\
%     \midrule
%     TRADES & 43.11 & 79.11 & 76.45 & 79.21 & 73.72 & 82.73 & 80.42 & 80.72 & 78.97 \\
%     MART & 41.22 & 77.77 & 73.07 & 78.30 & 74.97 & 81.31 & 79.53 & 79.28 & 77.8 \\
%     AT & 43.30 & 79.58 & 77.53 & 79.47 & 73.76 & 82.78 & 80.86 & 80.49 & 79.58 \\
%     ATDA & 36.06 & 67.06 & 62.56 & 70.33 & 64.63 & 73.46 & 72.28 & 70.50 & 67.31 \\
%     DIAL (Ours) & \textbf{48.84} & \textbf{84.13} & \textbf{81.76} & \textbf{83.76} & \textbf{78.26} & \textbf{87.24} & \textbf{85.13} & \textbf{84.84} & \textbf{83.93}  \\
%     \bottomrule
%   \end{tabular}
% \end{table}


% \begin{figure}[!ht]
%   \centering
%   \includegraphics[width=9cm]{figures/spider_full.png}
%   \caption{Accuracy comparison with all tested methods over unforeseen corruptions.}
%   \label{spider-full-graph}
% \end{figure}
% %%%%%%%%%%%%%%%%%%%%%%%%% Arxiv version %%%%%%%%%%%%%%%%%%%%%%%%%%%%%%%%%%%%%
%%%%%%%%%%%%%%%%%%%%%%%%% Arxiv version %%%%%%%%%%%%%%%%%%%%%%%%%%%%%%%%%%%%%

\subsubsection{Transfer Learning}
Recent works \citep{salman2020adversarially,utrera2020adversarially} suggested that robust models transfer better on standard downstream classification tasks. In Table \ref{transfer-res} we demonstrate the advantage of our method when applied for transfer learning across CIFAR10 and CIFAR100 using the common linear evaluation protocol. see Appendix \ref{transfer-learning-settings} for detailed settings.

\begin{table}[H]
  \caption{Transfer learning results comparison.}
  \vskip 0.1in
  \label{transfer-res}
  \centering
  \small
\begin{tabular}{c|c|c|c}
\toprule

\multicolumn{2}{l}{} & \multicolumn{2}{c}{Target} \\
\cmidrule(r){3-4}
Source & Defence Model & CIFAR10 & CIFAR100 \\
\midrule
\multirow{3}{*}{CIFAR10} & DIAL & \multirow{3}{*}{-} & \textbf{28.57} \\
 & AT &  & 26.95  \\
 & TRADES &  & 25.40  \\
 \midrule
\multirow{3}{*}{CIFAR100} & DIAL & \textbf{73.68} & \multirow{3}{*}{-} \\
 & AT & 71.41 & \\
 & TRADES & 71.42 &  \\
%  \midrule
% \multirow{3}{}{SVHN} & DIAL &  &  & \multirow{3}{}{-} \\
%  & Madry et al. &  &  &  \\
%  & TRADES &  &  &  \\ 
\bottomrule
\end{tabular}
\end{table}


\subsubsection{Modularity and Ablation Studies}

We note that the domain classifier is a modular component that can be integrated into existing models for further improvements. Removing the domain head and related loss components from the different DIAL formulations results in some common adversarial training techniques. For $\DIAL_{\kl}$, removing the domain and related loss components results in the formulation of TRADES. For $\DIAL_{\ce}$, removing the domain and related loss components results in the original formulation of the standard adversarial training, and for $\DIAL_{\awp}$ the removal results in $\TRADES_{\awp}$. Therefore, the ablation studies will demonstrate the effectiveness of combining DIAL on top of different adversarial training methods. 

We investigate the contribution of the additional domain head component introduced in our method. Experiment configuration are as in \ref{defence-settings}, and robust accuracy is based on white-box PGD$^{20}$ on CIFAR-10 dataset. We remove the domain head from both $\DIAL_{\kl}$, $\DIAL_{\awp}$, and $\DIAL_{\ce}$ (equivalent to $r=0$) and report the natural and robust accuracy. We perform 3 random restarts and omit one standard deviation from the results. Results are presented in Figure \ref{ablation}. All DIAL variants exhibits stable improvements on both natural accuracy and robust accuracy. $\DIAL_{\ce}$, $\DIAL_{\kl}$, and $\DIAL_{\awp}$ present an improvement of 1.82\%, 0.33\%, and 0.55\% on natural accuracy and an improvement of 2.5\%, 1.87\%, and 0.83\% on robust accuracy, respectively. This evaluation empirically demonstrates the benefits of incorporating DIAL on top of different adversarial training techniques.
% \paragraph{semi-supervised extensions.} Since the domain classifier does not require the class labels, we argue that additional unlabeled data can be leveraged in future work.
%for improved results. 

\begin{figure}[ht]
  \centering
  \includegraphics[width=0.35\textwidth]{figures/ablation_graphs3.png}
  \caption{Ablation studies for $\DIAL_{\kl}$, $\DIAL_{\ce}$, and $\DIAL_{\awp}$ on CIFAR-10. Circle represent the robust-natural accuracy without using DIAL, and square represent the robust-natural accuracy when incorporating DIAL.
  %to further investigate the impact of the domain head and loss on natural and robust accuracy.
  }
  \label{ablation}
\end{figure}

\subsubsection{Visualizing DIAL}
To further illustrate the superiority of our method, we visualize the model outputs from the different methods on both natural and adversarial test data.
% adversarial test data generated using PGD$^{20}$ white-box attack with step size 0.003 and $\epsilon=0.031$ on CIFAR-10. 
Figure~\ref{tsne1} shows the embedding received after applying t-SNE ~\citep{van2008visualizing} with two components on the model output for our method and for TRADES. DIAL seems to preserve strong separation between classes on both natural test data and adversarial test data. Additional illustrations for the other methods are attached in Appendix~\ref{additional_viz}. 

\begin{figure}[h]
\centering
  \subfigure[\textbf{DIAL} on natural logits]{\includegraphics[width=0.21\textwidth]{figures/domain_ce_test.png}}
  \hspace{0.03\textwidth}
  \subfigure[\textbf{DIAL} on adversarial logits]{\includegraphics[width=0.21\textwidth]{figures/domain_ce_adversarial.png}}
  \hspace{0.03\textwidth}
    \subfigure[\textbf{TRADES} on natural logits]{\includegraphics[width=0.21\textwidth]{figures/trades_test.png}}
    \hspace{0.03\textwidth}
    \subfigure[\textbf{TRADES} on adversarial logits]{\includegraphics[width=0.21\textwidth]{figures/trades_adversarial.png}}
  \caption{t-SNE embedding of model output (logits) into two-dimensional space for DIAL and TRADES using the CIFAR-10 natural test data and the corresponding PGD$^{20}$ generated adversarial examples.}
  \label{tsne1}
\end{figure}


% \begin{figure}[ht]
% \centering
%   \begin{subfigure}{4cm}
%     \centering\includegraphics[width=3.3cm]{figures/domain_ce_test.png}
%     \caption{\textbf{DIAL} on nat. examples}
%   \end{subfigure}
%   \begin{subfigure}{4cm}
%     \centering\includegraphics[width=3.3cm]{figures/domain_ce_adversarial.png}
%     \caption{\textbf{DIAL} on adv. examples}
%   \end{subfigure}
  
%   \begin{subfigure}{4cm}
%     \centering\includegraphics[width=3.3cm]{figures/trades_test.png}
%     \caption{\textbf{TRADES} on nat. examples}
%   \end{subfigure}
%   \begin{subfigure}{4cm}
%     \centering\includegraphics[width=3.3cm]{figures/trades_adversarial.png}
%     \caption{\textbf{TRADES} on adv. examples}
%   \end{subfigure}
%   \caption{t-SNE embedding of model output (logits) into two-dimensional space for DIAL and TRADES using the CIFAR-10 natural test data and the corresponding adversarial examples.}
%   \label{tsne1}
% \end{figure}



% \begin{figure}[ht]
% \centering
%   \begin{subfigure}{6cm}
%     \centering\includegraphics[width=5cm]{figures/domain_ce_test.png}
%     \caption{\textbf{DIAL} on nat. examples}
%   \end{subfigure}
%   \begin{subfigure}{6cm}
%     \centering\includegraphics[width=5cm]{figures/domain_ce_adversarial.png}
%     \caption{\textbf{DIAL} on adv. examples}
%   \end{subfigure}
  
%   \begin{subfigure}{6cm}
%     \centering\includegraphics[width=5cm]{figures/trades_test.png}
%     \caption{\textbf{TRADES} on nat. examples}
%   \end{subfigure}
%   \begin{subfigure}{6cm}
%     \centering\includegraphics[width=5cm]{figures/trades_adversarial.png}
%     \caption{\textbf{TRADES} on adv. examples}
%   \end{subfigure}
%   \caption{t-SNE embedding of model output (logits) into two-dimensional space for DIAL and TRADES using the CIFAR-10 natural test data and the corresponding adversarial examples.}
%   \label{tsne1}
% \end{figure}



\subsection{Balanced measurement for robust-natural accuracy}
One of the goals of our method is to better balance between robust and natural accuracy under a given model. For a balanced metric, we adopt the idea of $F_1$-score, which is the harmonic mean between the precision and recall. However, rather than using precision and recall, we measure the $F_1$-score between robustness and natural accuracy,
using a measure we call
%We named it
the
\textbf{$\mathbf{F_1}$-robust} score.
\begin{equation}
F_1\text{-robust} = \dfrac{\text{true\_robust}}
{\text{true\_robust}+\frac{1}{2}
%\cdot
(\text{false\_{robust}}+
\text{false\_natural})},
\end{equation}
where $\text{true\_robust}$ are the adversarial examples that were correctly classified, $\text{false\_{robust}}$ are the adversarial examples that were miss-classified, and $\text{false\_natural}$ are the natural examples that were miss-classified.
%We tested the proposed $F_1$-robust score using PGD$^{20}$ on CIFAR-10 dataset in white-box and black-box settings. 
Results are presented in Table~\ref{f1-robust} and demonstrate that our method achieves the best $F_1$-robust score in both settings, which supports our findings from previous sections.

% \begin{table}[!ht]
%   \caption{$F_1$-robust measurement using PGD$^{20}$ attack in white and black box settings on CIFAR-10}
%   \label{f1-robust}
%   \centering
%   \begin{tabular}{lll}
%     \toprule
%     \cmidrule(r){1-2}
%     Defense Model & White-box & Black-box \\
%     \midrule
%     TRADES & 0.65937  & 0.84435 \\
%     MART & 0.66613  & 0.83153  \\
%     Madry et al. & 0.65755 & 0.84574   \\
%     Song et al. & 0.51823 & 0.76092  \\
%     $\DIAL_{\ce}$ (Ours) & 0.65318   & $\mathbf{0.88806}$  \\
%     $\DIAL_{\kl}$ (Ours) & $\mathbf{0.67479}$ & 0.84702 \\
%     \midrule
%     \midrule
%     DIAL-AWP (Ours) & $\mathbf{0.69753}$  & $\mathbf{0.85406}$  \\
%     TRADES-AWP & 0.68162 & 0.84917 \\
%     \bottomrule
%   \end{tabular}
% \end{table}

\begin{table}[ht]
\small
  \caption{$F_1$-robust measurement using PGD$^{20}$ attack in white and black box settings on CIFAR-10.}
  \vskip 0.1in
  \label{f1-robust}
  \centering
%   \small
  \begin{tabular}{c
  @{\hspace{1.5\tabcolsep}}c @{\hspace{1.5\tabcolsep}}c @{\hspace{1.5\tabcolsep}}c @{\hspace{1.5\tabcolsep}}c
  @{\hspace{1.5\tabcolsep}}c @{\hspace{1.5\tabcolsep}}c @{\hspace{1.5\tabcolsep}}|
  @{\hspace{1.5\tabcolsep}}c
  @{\hspace{1.5\tabcolsep}}c}
    \toprule
    % \cmidrule(r){8-9}
     & TRADES & MART & AT & ATDA & $\DIAL_{\ce}$ & $\DIAL_{\kl}$ & $\DIAL_{\awp}$ & $\TRADES_{\awp}$ \\
    \midrule
    White-box & 0.659 & 0.666 & 0.657 & 0.518 & 0.660 & \textbf{0.675} & \textbf{0.698} & 0.682 \\
    Black-box & 0.844 & 0.831 & 0.845 & 0.761 & \textbf{0.890} & 0.847 & \textbf{0.854} & 0.849 \\ 
    \bottomrule
  \end{tabular}
\end{table}


\section{Conclusions}

We present a general framework and a new set of algorithms for finding the
canonical image of a set under the action of a permutation group. Our
experiments show that our new algorithms outperform the previous state of the art,
often by orders of magnitude.

Our basic framework runs on the concept of refiners and selectors and is not
limited to finding only canonical images of subsets of \(\Omega\). In future
work we will investigate families of refiners and selectors that allow finding
canonical images for many other combinatorial objects.

\vspace{1cm}

\textbf{Acknowledgements.}

All authors thank the DFG (\textbf{Wa 3089/6-1}) and the EPSRC CCP CoDiMa
(\textbf{EP/M022641/1}) for supporting this work. The first author would like to
thank the Royal Society, and the EPSRC (\textbf{EP/M003728/1}). The third
author would like to acknowledge support from the OpenDreamKit Horizon 2020
European Research Infrastructures Project (\#676541). The first and third author
thank the Algebra group at the Martin-Luther Universit\"at
Halle-Wittenberg for the hospitality and the inspiring environment. The fourth
author wishes to thank the Computer Science Department of the University of
St~Andrews for its hospitality during numerous visits.

\bibliography{canonical}
\bibliographystyle{plain}
\end{document}
