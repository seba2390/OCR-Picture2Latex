\documentclass[preprint,12pt]{elsarticle}
%\usepackage{showkeys}
\usepackage{amsmath, amssymb}
\newfont {\cyr} {wncyr10}
\pagestyle{plain} \frenchspacing
\renewcommand{\labelenumi}{{(\roman{enumi})}}
\usepackage{color}
\newcommand{\red}[1]{\,{\color{red} #1}\,}
\usepackage{amsfonts}
\usepackage{algorithm}
\usepackage[noend]{algpseudocode}
\usepackage{amsthm}%Beweisumgebung
\usepackage{mathabx}
\usepackage{xspace}
\linespread{1.1}

\usepackage{pdflscape}
\usepackage{afterpage}
\usepackage{capt-of}% or use the larger `caption` package

\newtheorem{theorem}{Theorem}[section]
\newtheorem{lem}[theorem]{Lemma}
\newtheorem{thm}[theorem]{Theorem}
\newtheorem{prop}[theorem]{Proposition}
\newtheorem{cor}[theorem]{Corollary}
\newtheorem{rem}[theorem]{Remark}
\newtheorem{hyp}[theorem]{Hypothesis}
\newtheorem{defi}[theorem]{Definition}
\newtheorem{ex}[theorem]{Example}

\newcommand{\SX}[1]{\ensuremath{S_{#1}}\xspace}
\newcommand{\Sym}[1]{\ensuremath{Sym(#1)}}

\newcommand{\myfloor}[1]{\left \lfloor #1 \right \rfloor}
\newcommand{\LL}{\mathcal{L}}
\newcommand{\id}{\textrm{id}}
\newcommand{\N}{\mathbb{N}}
\newcommand{\ext}{\operatorname{ext}}
\newcommand{\Min}{\operatorname{Min}}
\newcommand{\Orb}{\operatorname{Orb}}
\newcommand{\Points}{\operatorname{Points}}
\newcommand{\Orbcount}{\operatorname{Orbcount}}
\newcommand{\Count}{\operatorname{Count}}
\newcommand{\powset}{\mathcal{P}}


\begin{document}

\begin{frontmatter}

\title{Minimal and canonical images}

\author{Christopher Jefferson}
%\address{University of St~Andrews\\School of Computer Science\\North Haugh\\St Andrews\\KY16 9SX\\Scotland}
\ead{caj21@st-andrews.ac.uk}
\ead[url]{http://caj.host.cs.st-andrews.ac.uk/}

\author{Eliza Jonauskyte}
%\address{University of St~Andrews\\School of Computer Science\\North Haugh\\St Andrews\\KY16 9SX\\Scotland}
\ead{ej31@st-andrews.ac.uk}
%\ead[url]{http://caj.host.cs.st-andrews.ac.uk/}

\author{Markus Pfeiffer}
\address{University of St~Andrews\\School of Computer Science\\North Haugh\\St Andrews\\KY16 9SX\\Scotland}
\ead{markus.pfeiffer@st-andrews.ac.uk}
\ead[url]{https://www.morphism.de/~markusp/}

\author{Rebecca Waldecker}
\address{Martin-Luther-Universit\"at Halle-Wittenberg\\Institut f\"ur Mathematik\\06099 Halle\\Germany}
\ead{rebecca.waldecker@mathematik.uni-halle.de}
\ead[url]{http://conway1.mathematik.uni-halle.de/~waldecker/index-english.html}

\journal{Journal of Algebra}

\begin{abstract}
We describe a family of new algorithms for finding the canonical image of a set
of points under the action of a permutation group. This family of algorithms
makes use of the orbit structure of the group, and a chain of subgroups of the
group, to efficiently reduce the amount of search that must be performed to
find a canonical image.

We present a formal proof of correctness of our algorithms and describe experiments
on different permutation groups that compare our algorithms with the previous
state of the art.
\end{abstract}

\begin{keyword}
  Minimal Images, Canonical Images, Computation, Group Theory, Permutation Groups.
\end{keyword}
\end{frontmatter}

Reinforcement learning has achieved great success in areas such as Game-playing \citep{silver2018general,vinyals2019grandmaster}, robotics \cite{kober2013reinforcement}, large language models \citep{ouyang2022training}, etc.
However, due to safety concerns or physical limitations, in some real-world reinforcement learning problems, we must consider additional constraints that may influence the optimal policy and the learning process \citep{garcia2015comprehensive}.
% For example, a robotic arm must not take actions that may cause harm to itself or the environments.
A standard framework to handle such cases is the constrained Markov Decision Process (CMDP) \citep{altman1999constrained}.
Within the CMDP framework, the agent has to maximize
the expected cumulative reward while
obeying a finite number of constraints, which are usually in the form of expected cumulative cost criteria.

However, we are sometimes concerned with the problem with a continuum of constraints.
For example,
the constraints we meet might be time-evolving or subject to uncertain parameters, which
cannot be formulated as an ordinary CMDP
(see Examples \ref{Example_Time_Evolving} and  \ref{Example_Uncertain}).
In this paper we would study a generalized CMDP  
to address the above problem.  Because the constraints are not only infinite-number but also lie
in a continuous set,
the generalization is not trivial. Fortunately, we find that we can borrow the idea behind semi-infinite programming (SIP) \citep{remez1934determination, hettich1993semi} to deal with the semi-infinite constraints.
Accordingly, we propose \emph{semi-infinitely constrained Markov decision processes} (SICMDPs)
as a novel complement to the ordinary CMDP framework.
%More specifically,  an SICMDP model %, we consider 
%contains a continuum of constraints whereas an ordinary CMDP contains a finite number of constraints. 

%This generalization is natural but not trivial. However, we can brows the idea  
%The idea is quite natural and can be backtracked
%to the practice of extending linear programming to linear semi-infinite programming (LSIP) %\cite{remez1934determination, GobernaLSIO1998}.
%In addition, 
%As a complementary approach to the ordinary CMDP framework, 
%SICMDP can be used to model these problems  which cannot be described by a finite number of constraints
%that are not covered by .
%For example,
%the restrictions we consider can be time-evolving or subject to uncertain parameters
%, thus
%cannot be described by a finite number of constraints but a continuum of constraints 
%(see Examples \ref{Example_Time_Evolving} and  \ref{Example_Uncertain}).

We also present two reinforcement learning algorithms to solve SICMDPs called SI-CRL and SI-CPO, respectively.
SI-CRL is a model-based reinforcement learning algorithm designed for tabular cases, and SI-CPO is a policy optimization algorithm for non-tabular cases.
% and analyze its performance both theoretically and empirically.
The main challenge is that we need to deal with a continuum of constraints, thus reinforcement learning algorithms for ordinary CMDPs do not work anymore.
In SI-CRL, we tackle this difficulty by first transforming the reinforcement learning problem to an equivalent LSIP problem, which can then be solved using methods in the LSIP literature like the dual exchange methods \citep{Hu1990,reemtsen1998numerical}.
In SI-CPO, we resort to the idea of cooperative stochastic approximation developed in \cite{lan2020algorithms, wei2020comirror}.
As far as we know, we are the first to introduce tools from semi-infinitely programming (SIP) into the reinforcement learning community for solving constrained reinforcement learning problems.

% To the best of our knowledge, we are the first to apply tools from semi-infinitely programming (SIP) to solve reinforcement learning problems.
Furthermore, we give theoretical analysis for both SI-CRL and SI-CPO.
We decompose the error of SI-CRL into two parts: the statistical error from approximating the true SICMDP with an offline dataset and the optimization error due to the fact that the solution of the LSIP problem obtained by the dual exchange method is inexact.
On the optimization side, we show that the iteration complexity of SI-CRL is $O\left(\left\{\mathrm{diam}(Y)L\sqrt{|\gS|^2|\gA|m}/\left[(1-\gamma)\epsilon\right]\right\}^m\right)$.
On the statistical side, we show that the sample complexity of SI-CRL is $\widetilde O\left(\frac{|S|^2|A|^2}{\epsilon^2(1-\gamma)^3}\right)$ if the offline dataset is generated by a generative model, and $\widetilde O\left(\frac{|S||A|}{\nu_{\min} \epsilon^2(1-\gamma)^3}\right)$ if the dataset is generated by a probability measure $\nu$ as considered in \cite{chen2019information}.
Here $\widetilde O$ means that all logarithm terms are discarded.
For SI-CPO, things become a little more complicated because other than the statistical error and the optimization error, we also need to consider the function approximation error, which comes from imperfect policy parametrizations.
It is shown if the function approximation error can be controlled to $O(\epsilon)$ order, the iteration complexity of SI-CPO is $\widetilde{O}\left(\frac{1}{\epsilon^2(1-\gamma)^6}\right)$ and the sample complexity of SI-CPO is $\widetilde{O}(\frac{1}{\epsilon^4(1-\gamma)^{10}})$.
Here our iteration complexity bound is equivalent to a typical $\widetilde O(1/\sqrt{T})$ global convergence rate.

We perform a set of numerical experiments to illustrate the SICMDP model and validate our proposed algorithms.
Specifically, we examine two numerical examples, namely the discharge of sewage and ship route planning.
Through the discharge of sewage example, we show the advantage of the SICMDP framework over the CMDP baseline obtained by naive discretization in modeling realistic sequential decision-making problems.
Moreover, we demonstrate the effectiveness of the SI-CRL and SI-CPO algorithms in such tabular environments. 
In the ship route planning example, we illustrate the benefits of the SICMDP framework and the ability of the SI-CPO algorithm to address complex continuous control tasks involving continuous state spaces with modern deep reinforcement learning techniques.

% In summary, our contributions are listed as follows.
% First, we present the SICMDP model, which can be viewed as a generalization of the ordinary CMDP model.
% Second, we propose an algorithm to perform reinforcement learning for SICMDPs, which is called SI-CRL, and we believe that we are the first to apply tools from SIP
% to solve reinforcement learning problems.
% Third, we give a theoretical analysis of SI-CRL and identify both its sample complexity and iteration complexity.
% In addition, we perform numerical experiments to illustrate the SICMDP model and validate the SI-CRL algorithm.
% \{This paragraph can be removed!!! \}






\section{Minimal and Canonical Images}

Throughout this paper, $\Omega$ will be a finite set, $G$ a subgroup of
$\Sym{\Omega}$, and $\Omega$ will be ordered by some (not necessarily total)
order $\leq$.
%\begin{hyp}\label{main}
%  Suppose that $G$ is a finite group that acts on a finite set $\Omega$.
%\end{hyp}
If $\alpha \in \Omega$, then we denote the orbit of $\alpha$ under $G$ by $\alpha^G$.
Simlarly, if \(A \subseteq \Omega\) and $g \in G$, then 
\(A^g := \{a^g \mid a \in A\}\) and \(A^G := \{ A^g \mid g \in G \}\).

In this paper, we want to efficiently solve the problem of deciding, given two subsets
\(A,B \subseteq \Omega\), if \(A \in B^G\). We do this by defining a canonical image:

\begin{defi}
  A \textbf{canonical labelling function} $C$ for the action of $G$ on a set $\Omega$ is a function
  $C:\powset(\Omega) \rightarrow \powset(\Omega)$ such that, for all $A  \subseteq \Omega$,
  it is true that
  \begin{itemize}
  \item $C(A) \in A^G$, and
  \item ${C(A^g)} = C(A)$ for all $g \in G$.
  \end{itemize}

  In this situation we call $C(A)$ the \textbf{canonical image} of
  $A \subseteq  \Omega$ (with respect to $G$ in this particular action).

  Further, we say that $g_A \in G$ is a \textbf{canonizing element} for \(A\) if and only if
  $A^{g_A} = C(A)$.
\end{defi}


A canonical image can be seen as a well-defined
representative of a $G$-orbit on $\Omega$ with respect to the defined action. While in this
paper we will only consider the action of \(G\) on a set of subsets of \(\Omega\), canonical images are
defined similarly for any group and action.
In practice we want to be able to find canonical images effectively and
efficiently.
In some situations we are interested in computing the canonizing element,
which might not be uniquely determined. Our algorithms will always produce a
canonizing element as a byproduct of search.
We choose to make this explicit here to make the exposition clearer.


Minimal images are a special type of canonical image.

\begin{rem}
  Suppose that $\preccurlyeq$ is a partial order on $\Omega$ such that any
  two elements in the same orbit can be compared by $\preccurlyeq$.

   Let $\Min_\preccurlyeq$ denote the function that, for all $\omega \in \Omega$, maps $\omega$ to the
  smallest element in its orbit. Then $\Min_\preccurlyeq$ is a canonical labelling function.
\end{rem}

In practical applications we are interested in more structure, namely in structures
that $G$ can act on naturally via the action on a given set $\Omega$. These
structures include subsets of $\Omega$, graphs with vertex set $\Omega$, sets of
maps with domain or range $\Omega$, and so on.


In this paper, our main application will be finding canonical images when
acting on a set of subsets of $\Omega$.

\begin{defi}\label{metaorder}
  Suppose that $\leq$ is a total order of $\Omega$.
  Then we introduce a total order $\preccurlyeq$ on $\powset(\Omega)$ as follows:

  We say that $A$ \textbf{is less than} $B$ and write $A \preccurlyeq B$ if and only if $A$
  contains an element $a$ such that $a \notin B$ and $a \leq b$ for all
  $b \in B \setminus A$.
\end{defi}

\begin{ex}
  Let $\Omega:=\{1,2,3,4,5,6,7\}$ with the natural order and let $A:=\{1,3,4\}$,
  $B:=\{3,5,7\}$, $C:=\{3,6,7\}$, $D:=\{1,3\}$ and $E:=\{2\}$.

  Now $A \preccurlyeq B$, because $1 \in A$, $1 \notin B$ and $1$ is smaller
  than all the elements in $B$, in particular those not in $A$. Moreover $A \preccurlyeq C$ for the same reason. Furthermore, $B \preccurlyeq C$,
  because $5 \in B$, $5 \notin C$, and if we look at $C \setminus B$, then
  this only contains the element $6$ and $5$ is smaller.

  Next we consider $A$ and $D$. As $4 \in A \setminus D$ and $D\setminus A=\varnothing$, we see that $A \preccurlyeq D$. Also $A \preccurlyeq E$ because $1 \in A \setminus E$ and $1$ is smaller than all elements in $E \setminus A=E$.
  Finally $E \preccurlyeq B$ because $2 \in E \setminus B=E$ and $2$ is smaller than all elements in $B \setminus E=B$.
\end{ex}

\begin{rem}
The example illustrates that this new order introduced above reduces to lexicographical order for sets of the same size. But for sets of different sizes, it might seem counter-intuitive. Our reason for choosing this different ordering is that it satisfies the
following property:

If \(n \in \N\) and if \(A\) and \(B\) are sets of integers, then \(A \cap \{1,\dots,n\} < B \cap \{1,\dots,n\}\) implies \(A < B\). This means that, when building \(A\) and \(B\) incrementally, we know the order of \(A\) and \(B\) as soon as we find the first integer that is contained in one of the sets but not in the other. This is not true for lexicographic ordering of sets, as \(\{1\} < \{1,2\}\) but \(\{1,1000\} > \{1,2,1000\}\).
\end{rem}

If $G$ is a subgroup of $\Sym{\Omega}$ and $\omega \in \Omega$, then we denote by
$G_\omega$ the point stabilizer of $\omega$ in $G$. For distinct elements
$x,y \in \Omega$, we denote by \(G_{x \mapsto y}\)  the set of all elements of
$G$ that map $x$ to $y$. This set may be empty.

We remark that the above information is readily available from a stabilizer chain
for the group $G$, which can be calculated efficiently. For further details
we refer the reader to \cite{HCG}. We now introduce some notation and then prove a basic result about cosets.

\begin{defi}
 Let $G$ be a permutation group acting on a totally ordered set
 $(\Omega, \leq)$, and let $\preccurlyeq$ denote the induced ordering as explained in Definition \ref{metaorder}.
Let $H$ be a subgroup of $G$ and $S \subseteq \Omega$.
Then we define
  \textbf{the minimal image of $S$ under $H$} to be the smallest element in the set
  $\{S^h \mid h \in H\}$ with respect to $\preccurlyeq$.

In order to simplify notation, we will from now on write $\leq$ for the induced order and then we write $\Min(H,S,\leq)$ for the minimal image of $S$ under $H$.
\end{defi}


\begin{lem}\label{miniprop}
  Let $G$ be a permutation group acting on a totally ordered set
  $(\Omega, \leq)$, and let $H$ be a subgroup of $G$ and $S \subseteq \Omega$.
  Then the following hold for all $x,y \in \Omega$:

\begin{enumerate}

\item
  For all $\sigma \in H_{x \mapsto y}$ it is true that
  $\sigma \cdot H_y=H_{x\mapsto y}=H_x \cdot \sigma$.

\item
  If $\sigma \in H_{x \mapsto y}$, then
  \(\Min(\sigma \cdot H, S, \leq) = \Min(H, S^{\sigma}, \leq)\).
\end{enumerate}
\end{lem}

\begin{proof}
  If $\sigma \in H_{x \mapsto y}$, then multiplication by $\sigma$ from the
  right or left is a bijection on $H$, respectively. For all $\alpha \in H_x$ we
  have that $\alpha \cdot \sigma$ maps $x$ to $y$ and for all $\beta \in H_y$ we
  see that $\sigma \cdot \beta$ also maps $x$ to $y$. This implies the first
  statement.

  For (ii) we just look at the definition: $\Min(\sigma \cdot H, S, \leq)$ denotes
  the smallest element in the
  set $\{S^{\sigma\cdot h} \mid h \in H\}$ and $\Min(H, S^{\sigma}, \leq)$
  denotes the smallest element in the set $\{(S^\sigma)^h \mid h \in H\}$,
  which is the same set.
\end{proof}

\subsection{Worked Example}

We will find minimal, and later canonical, images using similar techniques to
Linton in \cite{Linton:SmallestImage}. This algorithm splits the problem into
small sub-problems, by splitting a group into the cosets of a point stabilizer.
We will begin by demonstrating this general technique with a worked
example.

\begin{ex}\label{ex:minimal}
  In the following example we will look at $\Omega = \{1,2,3,4,5,6\}$,
  the subgroup {$G = \langle (14)(23)(56), (126)\rangle \leq \SX{6}$},
  and $S = \{2,3,5\}$.
  We intend to find the minimal image $\Min(G,S, \leq)$, where the ordering on
  subsets of $\Omega$ is the induced ordering from $\leq$ on $\Omega$ as explained in
  Definition \ref{metaorder}.

  We split our problem into pieces by looking at cosets of \(G_1 = \langle (3,4,5)
  \rangle\). The minimal image of \(S\) under \(G\) will be realized by an
  element contained in (at least) one of the cosets of \(G_1\), so if we find
  the minimal image of \(S\) under elements in each coset, and then take the
  minimum of these, we will find the global minimum.

  Lemma \ref{miniprop} gives that, for all $g \in G$, it holds that $\Min(g \cdot
  G_1, S, \leq) = \Min(G_1, S^g, \leq)$, and
  so we can change our problem from looking for the minimal image of $S$ with
  respect to cosets of \(G_1\) to looking at images of \(S^g\) under elements of
  $G_1$ where $g$ runs over a set of coset representatives of $G_1$ in $G$.

  For each $i \in \{1,\dots,6\}$ we need an element
  $g_i \in G_{i \mapsto 1}$ (where any exist), so that we can then consider $S^{g_i}$.

  We choose the elements $\id$, $(162)$, $(146523)$, $(14)(23)(56)$, $(142365)$
  and $(126)$ and obtain six images of $S$:
  \[
    \{2,3,5\}, \{1,3,5\}, \{3,1,2\}, \{3,2,6\}, \{3,6,1\}, \{6,3,5\}.
  \]

  As we are looking at the images of these sets under \(G_1\), we
  know that all images of a set containing \(1\) will contain \(1\),
  and all images of a set not containing \(1\) will not contain \(1\).
  From Definition \ref{metaorder}, all subsets of \(\{1,\dots,6\}\)
  containing \(1\) are smaller than all subsets not containing \(1\). This means
  that we can filter our list down to $\{1,3,5\}, \{3,1,2\}$ and $\{3,6,1\}$.

  Furthermore, \(G_1\) fixes \(2\), so by the same argument we can filter our
  list of sets not containing \(2\), leaving only \(\{3,1,2\}\). The minimal image
  of this under \(G_1\) is clearly \(\{3,1,2\}\) (in this particular
  case we could of course also have stopped as soon as we saw \(\{3,1,2\}\), as
  this is the smallest possible set of size 3).

  Now, let us consider what would happen if the ordering of the integers was
  reversed, so we are looking for \(\Min(G, S, \geq)\), again with the
  induced ordering.

  For the same reasons as above, we begin by calculating
  $G_6 = \langle (3,5,4) \rangle$ and by finding images of \(S\) for some
  element from each coset of \(G_6\) in $G$.

  An example of six images is
  \[
    \{ 1, 5, 4 \}, \{ 6, 5, 3 \}, \{ 4, 6, 1 \}, \{ 5, 1, 2 \}, \{ 3, 2, 6 \},
    \{ 2, 3, 5 \}.
  \]
  We can ignore anything that does not contain \(6\), so we are left with:

  \[
    \{ 6, 5, 3 \}, \{ 4, 6, 1 \}, \{ 3, 2, 6 \}.
  \]

  As \(5\) is not fixed by \(G_6\), we can not reason about the presence
  or absence of \(5\) in our sets. There is an image of every set that
  contains \(5\), and there are even two distinct images of \(\{6,5,3\}\) that
  contain \(5\). Therefore we must continue our search by considering \(G_{6,5}\).

  Application of an element from each coset of $G_{6,5}$ to $S$ generates nine
  sets, of which four contain the element $5$. In fact we reach \(\{6,3,5\},
  \{6,5,4\}\) from the set $\{ 6, 4, 3 \}$, we reach \(\{5,6,1\}\) from the set
  $\{ 4, 6, 1 \}$ and we reach \(\{5,2,6\}\) from the set $\{ 3, 2, 6 \}$. From
  these we extract the minimal image \(\{6,5,4\}\).

  In this example, different orderings of \(\{1,2,3,4,5,6\}\) produced different sized
  searches, with different numbers of levels of search required.
\end{ex}

\section{Minimal Images under alternative orderings of \(\Omega\)}
\label{sec:alternate ordering}
As was demonstrated in Example \ref{ex:minimal}, the choice of ordering of the set our group acts on
influences the size of the search for a minimal image. In this section we will show
how to create
orderings of $\Omega$ that, on average, reduce the size of search for a minimal image.

We begin by showing how large a difference different orderings can make. We do
this by proving that, for any choice $\preccurlyeq$ of ordering of \(\Omega\), group \(G\) and
any input set $S$, we can construct a minimal image
problem that is as hard as finding \(\Min(G,S,\preccurlyeq)\), but where
reversing the ordering on $\Omega$ makes the problem trivial.

We make this more precise: Given $n \in \N$, a permutation group \(G\) on \(\{1,\dots,n\}\) with some ordering $\leq$ and a subset \(S
\subseteq \{1,\dots,n\}\), we construct a group \(H\) and a set \(T\) such that
\(\Min(G, S, \leq) = \Min(H,T, \leq) \cap \{1..n\}\), which shows that finding
\(\Min(H,T,\leq)\) is at least as hard as finding \(\Min(G,S,\leq)\). On the
other hand, we will show that \(\Min(H,T, \geq) = T\) and that this can be deduced without search.
This is done in Lemma \ref{ex}.
An example along the way will illustrate the construction.

\begin{defi}
We fix $n \in \N$ and we let $k \in \N$. For all $j \in \N$ we define $q(j) \in
\N$ (where $q$ stands for ``quotient'') and $r(j) \in \{1,\dots,n\}$ (where $r$ stands for ``remainder'') such that $j=q(j) \cdot n+r(j)$.


Let $\ext : G \rightarrow \SX{k \cdot n}$ be the following map:
For all $g \in G$ and all $j \in \{1,\dots,k \cdot n\}$, the element
$\ext(g)$ maps $j$ to $q(j) \cdot n+r(j)^g$.
\end{defi}


\begin{ex}
Let $n=4$ and $G=\SX{4}$. Then we extend the action of $G$ to the set
$\{1,\dots,12\}$ using the map $\ext$.

For example $g=(134)$ maps $4$ to $1$.
We write $12=2 \cdot 4+4$ and then it follows that
$\ext(g)$ maps $12$ to $2 \cdot 4 +4^g=8+1=9$.
In fact $g$ acts simultaneously on the three tuples $(1,2,3,4)$, $(5,6,7,8)$
and $(9,10,11,12)$ as it does on $(1,2,3,4)$.
\end{ex}

\begin{defi}
Fixing $n,k \in \N$ and a subgroup $G$ of $\SX{n}$, and using the map $\ext$
defined above, we say that \textbf{$H$ is the extension of $G$ on
  $\{1,\dots,k\cdot n\}$}
if and only if $H=\{\ext(g) \mid g \in G\}$ is the image of $G$
under the map $\ext$.
\end{defi}

The extension $H$ of $G$ on a set $\{1,\dots,k \cdot n\}$
is a subset of $\SX{k \cdot n}$. We show now that even more is true:

\begin{lem}\label{ext}
Let $n,k \in \N$ and $G \le \SX{n}$. Then the extension of $G$ onto $\{1,\dots,k \cdot n\}$ is a subgroup of $\SX{k \cdot n}$ that is isomorphic to $G$.
\end{lem}

\begin{proof} Let $H:=\ext(G)$ be the image of $G$ under the map $\ext$ and let
$a,b \in G$ be distinct. Then let $j \in \{1,\dots,n\}$ be such that $j^a \neq
j^b$.
By definition $\ext(a)$ and $\ext(b)$ map $j$ in the same way that $a$ and $b$
do, so we see that $\ext(a) \neq \ext(b)$. Hence the map $\ext$ is injective.
Therefore $\ext:G \rightarrow H$ is bijective.

Next we let $a,b \in G$ be arbitrary and we let $j \in \{1,\dots,k \cdot n\}$.
Then the composition $ab$ is mapped to $\ext(ab)$, which maps $j$ to $q(j) \cdot
n+r(j)^{ab}$. Now $r(j)^{ab}=(r(j)^a)^b$ and therefore the composition
$\ext(a)\ext(b) \in \SX{k \cdot n}$ maps $j$ to $(q(j) \cdot
n+r(j)^a)^{\ext(b)}=q(j) \cdot n + (r(j)^a)^b$. This is because $r(j)^a \in
\{1,\dots,n\}$.

Hence $\ext(ab)=\ext(a)\ext(b)$. That implies $\ext$ is a group homomorphism
and hence that $G$ and its image are isomorphic.
\end{proof}

\begin{lem}\label{ex}
Let $n \in \N$ and $G \le \SX{n}$. Let $H$ denote the extension of $G$ on
$\{1,\dots,(n+1) \cdot n\}$ and let $S \subseteq \{1,\dots,n\}$. Let
$T:=S \cup \{ l \cdot n + l \mid l \in \{1..n\}\}$, let \(\leq\) denote the natural
ordering of the integers, and let \(\geq\) denote its reverse. For simplicity we use the same symbols for the ordering induced on $\powset(\Omega)$, respectively. Then

\begin{itemize}
\item \(\Min(H,T, \le) \cap \{1,\dots,n\} = \Min(G,S, \le)\).
\item \(\Min(H,T,\ge)=T\).
\end{itemize}
\end{lem}

\begin{proof}
Let $h \in H$. Then by construction $h$ stabilizes the partition
$$[1,\dots,n|n+1,\dots,2n|\dots|n \cdot n,\dots,(n+1)\cdot n].$$ 
Moreover, for all $i \in \{1,\dots,n\}$ and $g \in G$ we have that $i^g=i^{\ext(g)}$
and so Lemma \ref{ext} implies that

\begin{align*}
  \Min(G,S,\le) &= \min_{\le}\{S^g \mid g \in G\}  \cap \{1,\dots,n\}\\
                &= \min_{\le}\{S^{\ext(g)} \mid g \in G\}  \cap \{1,\dots,n\}\\
                &= \min_{\le}\{S^h \mid h \in H\}  \cap \{1,\dots,n\}\\
                &= \Min(H,T, \le) \cap \{1,\dots,n\}.
\end{align*}
This proves the first statement.

For the second statement we notice that $(n + 1)\cdot n$ is now the smallest element
of $T$, and it cannot be mapped to anything smaller, because it also is the
smallest element available.
So if we let $h \in H$ be such that $T^h=\Min(H,T,\ge)$, then $h$ fixes the point $n(n + 1)$. By
definition of the extension, it follows that $h$ also fixes $k \cdot n$ for all
$k \in \{1\ldots n\}$.
The next point of $T$ under the ordering is $n^2 - 1$. It cannot be
mapped by $h$ to $n^2$, because $n^2$ is already fixed, hence $h$ has to fix
$n^2 - 1$, too.

Arguing as above it follows that all points are fixed by $h$, thus in
particular \(\Min(H,T,\ge)=T\), as stated. Furthermore, any algorithm
that stepped through the elements of \(T\) in the order we describe
would find this smallest element without having to perform a branching search,
as at each step there is no choice on which element of \(T\) is the next smallest.
\end{proof}


\subsection{Comparing Minimal Images Cheaply}

We describe some important aspects of Linton's
algorithm for computing the minimal image of a subset of $\Omega$.
\begin{defi}

Suppose that $(\Omega, \le)$ is a totally ordered set and that $G \le
\Sym{\Omega}$. Then \(\Orb(G)\) denotes the list of orbits of $G$ on $\Omega$.
This list of orbits is ordered with respect to the smallest element in each orbit under $\le$.

A $G$-orbit will be called a \textbf{singleton} if and only if it has size $1$.

If \(S \subseteq \Omega\), then we say that a $G$-orbit is \textbf{empty in S}
if and only if it is disjoint from $S$ as a set, and we say that it is
\textbf{full in S} if and only if it is completely contained in $S$.
\end{defi}

\begin{ex}
  Let \(\Omega:=\{1,\dots,8\}\), with the natural ordering on the integers, let
  \(G := \langle (1,4), (2,8), (5,6), (7,8) \rangle\) and let $S:=\{1,3,5,6\}$.

  Then \(\Orb(G) = [ \{1,4\}, \{2,7,8\}, \{3\}, \{5,6\} ]\) because this list
  contains all the $G$-orbits and they are ordered by the smallest element in each
  orbit, namely $1$ in the first, $2$ in the second, $3$ in the third, which is a
  singleton, and $5$ in the last (because $4$ is already in an earlier orbit).

  The orbits $\{3\}$ and $\{5,6\}$ are full in $S$, the orbit $\{2,7,8\}$ is
  empty in $S$ and $\{1,4\}$ is neither.
\end{ex}


\begin{lem}\label{same}
Suppose that $(\Omega, \le)$ is a totally ordered finite set and that $G \le
\Sym{\Omega}$. If
$\Min(G,S,\le)=\Min(G,T,\le)$ and $\omega \in \Omega$, then $\omega^G$ is empty in $S$ if
and only if it is empty in $T$, and $\omega^G$ is full in $S$ if and only if it is
full in $T$.
\end{lem}

\begin{proof}
  Let $\omega \in \Omega$ and suppose that $\omega^G$ is empty in $S$. As $\omega^G$
  is closed under the action of $G$, and $S_0 := \Min(G,S,\le)$ is an image
  of $S$ under the action of $G$, we see that $\omega^G$ is empty in $S_0$ and hence
  in $T_0:=\Min(G,T,\le)$. Thus $\omega^G$ is empty in $T$, which is an image of $T$
  under the action of $G$. The same arguments work vice versa.

  Next we suppose that $\omega^G$ is full in $S$. Then it is full in $S_0=T_0$ and
  hence in $T$, and the same way we see the converse.
\end{proof}

We can now prove Theorem \ref{thm:minsplit}, which provides the main technique
used to reduce search. This allows us to prove that the minimal image of some set \(S\)
will be smaller than or equal to the minimal image of a set \(T\), without explicitly
calculating the minimal image of either \(S\) or \(T\).

\begin{thm}\label{thm:minsplit}
  Suppose that \(G\) is a permutation group on a totally ordered finite set
  \((\Omega, \le)\) and that \(S\) and \(T\)
  are two subsets of \(\Omega\) where \(|S| = |T|\).

  Suppose further that \(o\) is the first orbit in the list $\Orb(G)$ that
  is neither full in both \(S\) and \(T\) nor empty in both \(S\) and \(T\). If
  \(o\) is empty in \(T\), but not in $S$, then $\Min(G,S,\le)$ is strictly
  smaller than $\Min(G,T,\le)$.
\end{thm}

\begin{proof}
Suppose that \(o\) is empty in \(T\), but not in \(S\). Then $o$ is empty in
$T_0:=\Min(G,T,\le)$, but not in $S_0:=\Min(G,S,\le)$, and in particular $T_0$
and $S_0$ are distinct, as we have seen in Lemma \ref{same}.
%If $\omega \in \Omega$, then the orbit $\omega^G$ is closed under the action of $G$.
%Thus, if $A \subseteq \Omega$
%and \(\omega^G\) is full in \(A\), then for all $g \in G$ it is true that \(\omega^G\) is
%full in \(A^g\). Similarly, if \(\omega^G\) is empty in \(A\), then \(\omega^G\) is empty
%in \(A^g\) for all \(g \in G\).

Let $\alpha$ denote the minimum of the orbit $o$ with respect to $\le$ and let $\omega \in \Omega$. If $\omega < \alpha$,
then
$\omega \notin o$, so the the orbit $\omega^G$ appears in the list $\Orb(G)$
\emph{before} $o$ does. Then the choice of $o$ implies that one of the following
two cases holds:

\begin{enumerate}
\item \(\omega^G\) is full in both \(S\) and \(T\). In particular, for all $g \in G$ we have that
$\omega \in S^g \cap T^g$.
\item \(\omega^G\) is empty in both \(S\) and \(T\). In particular, for all $g \in G$ we have that
$\omega^G \cap S^g=\varnothing$ and $\omega^G \cap T^g=\varnothing$.
\end{enumerate}

If $S_0$ contains an element $\omega\in\Omega$ such that $\omega < \alpha$, then Case (i)
above holds and $\omega \in T_0$. So $S_0 \cap \{\omega' \in \Omega \mid \omega' < \alpha\}= T_0
\cap \{\omega' \in \Omega \mid \omega < \alpha\}$.

Since $S_0$ and $T_0$ are distinct, they must differ amongst the elements
at least as large as $\alpha$ and, since they have the same cardinality, the smallest
such element determines which of $S_0$ and $T_0$ is smaller.

We recall that $o=\alpha^G$ is empty in $T$ and non-empty in $S$, so there exists
some $g \in G$ such that $\alpha \in S^g$. Then $S^g=S_0$ and $\alpha \notin T_0$, so
$S_0$ is strictly smaller that $T_0$.
\end{proof}

Here is an example how to use Theorem \ref{thm:minsplit}.

\begin{ex}
Let $\Omega:=\{1,\dots,10\}$ with natural ordering, and let
$G:=\langle (12), (45),(56), (89)\rangle$. We consider the sets $S:=\{3,6,7\}$
and $T:=\{3,7,9\}$ and we want to calculate the smallest of \(\Min(G,S)\) and
\(\Min(G,T)\). Hence, we want to know which one is smaller as cheaply as
possible, to avoid
superfluous calculations.

We first list the orbits of $G$:
$[\{1,2\},\{3\},\{4,5,6\},\{7\},\{8,9\},\{10\}]$.

Going through the orbits as listed, we see that the first one is empty in $S$
and $T$, the second one is full in $S$ and $T$, and the third one gives a
difference for the first time. It is empty in $T$, but not in $S$, so Theorem
\ref{thm:minsplit} yields that the minimal image of $S$ is strictly smaller than
that of $T$.

\end{ex}

\subsection{Static Orderings of \(\Omega\)}\label{sec:static}

In this section we look at
which total ordering of \(\Omega\) should be used to minimize the amount
of time taken to find minimal images of subsets of $\Omega$.

Given a group \(G\) we will choose an ordering on $\Omega$ such that orbits
with few elements appear as early as possible. In particular, singleton orbits should
appear first.

This is justified by the fact that singleton orbits are always either full or empty. Also,
we would expect smaller orbits to be more likely
to be empty or full than larger orbits. This means that small orbits placed early
in the ordering of \(\Omega\) are
more likely to lead to Theorem \ref{thm:minsplit} being applicable, leading to a
reduction in search.

Algorithm~\ref{alg:minorbit} heuristically chooses a new ordering for an ordered
set \(\Omega\), only depending on the group $G$, under the assumption that the
algorithm that computes minimal images will pick a point from a smallest
non-singleton orbit to branch on.
This will not always be true -- in practice Linton's algorithm branches on the
first orbit which contains some point contained in one of the current candidates
for minimal image.

However, we will show that in Section \ref{sec:ex} that
Algorithm~\ref{alg:minorbit} produces substantially smaller,
and therefore faster, searches in practice.

It is not necessary in Line~\ref{line:point2} of Algorithm~\ref{alg:minorbit} to
choose the smallest element of \(\mathit{Points}\), choosing an arbitrary element will,
on average, perform just as well.
By fixing which point is chosen, we ensure that independent implementations will
produce the same ordering and therefore the same canonical image.


\begin{algorithm}
  \caption{FixedMinOrbit}\label{alg:minorbit}
  \begin{algorithmic}[1]
    \Procedure{MinOrbitOrder}{$\Omega, G$}
    \State{\(\mathit{Remain} := \Omega\)}
    \State{\(\mathit{Order} := []\)}
    \State{\(H := G\)}
    \While{\(|\mathit{Remain}| > 0\)}
		\State{\(\mathit{OrbSize} := \Min\left\{|o|\ \mathrel{\big|}  o \in \Orb(H), o \cap \mathit{Remain} \neq \emptyset\right\}\)}\label{line:orb}
		\State{\(\mathit{Points} := \left\{o \mathrel{\big|} o \in \Orb(H), |o| = \mathit{OrbSize},  o \cap \mathit{Remain} \neq \emptyset\right\}\)}\label{line:point}
		\State{\(\mathit{MinPoint} := \Min\left\{x \mid o \in \mathit{Points}, x \in o\right\}\)}\label{line:point2}
		\State{\(\mathit{Remain} := \mathit{Remain} \backslash \{ \mathit{MinPoint} \}\)}
		\State{\(Add(\mathit{Order}, \mathit{MinPoint})\)}
		\State{\(H := G_{\mathit{MinPoint}}\)}
    \EndWhile
    \State\Return{\(\mathit{Order}\)}
    \EndProcedure
  \end{algorithmic}
\end{algorithm}

We will also consider one simple modification of Algorithm \ref{alg:minorbit}, namely
\texttt{FixedMaxOrbit} (which is the same as \texttt{FixedMinOrbit}) with line
  \ref{line:orb} changed to pick orbits of maximum size.

If our intuition about Theorem \ref{thm:minsplit} is correct, then
\texttt{MaxOrbit} should almost
always produce a larger search than \texttt{MinOrbit} or a random ordering
of \(\Omega\).

\subsection{Implementing alternative orderings of \(\Omega\)}

Having calculated an alternative \(Order\) using \textbf{FixedMinOrbit} or \textbf{FixedMaxOrbit},
we could create a version of \textbf{MinimalImage} which accepted an explicit
ordering. However, rather than editing the algorithm, we can instead perform a
pre-processing step, using Lemma~\ref{lem:map}.

\begin{lem}\label{lem:map}
  Consider a group \(G\) that acts on $\Omega = \{1,\dots,n\}$ and a
  permutation \(\sigma \in \Sym{\Omega}\). We define an ordering \(\leq_\sigma\) on
  \(\{1,\dots,n\}\), where for all $x,y \in \Omega$ we have that \(x \leq_\sigma y\) if and only if
  \(x^\sigma \leq y^\sigma\).

  For the induced orderings $\preccurlyeq$ and $\preccurlyeq_\sigma$
  on subsets of $\Omega$ as in Definition \ref{metaorder}
  it holds that
  \[ X \preccurlyeq_\sigma  Y \Leftrightarrow X^\sigma \preccurlyeq Y^\sigma \]
  for all subsets $X$ and $Y$ of $\Omega$, and hence (simplifying notation)
  \[\Min(G,S,\preccurlyeq_\sigma) = \Min(G^\sigma,S^\sigma,\preccurlyeq)^{\sigma^{-1}}.\]
\end{lem}

\begin{proof}

  Following Definition \ref{metaorder}, $X \preccurlyeq_\sigma Y$ if and only if
  there is an $x \in X$ such that $x\not\in Y$ and for all
  $y \in Y \backslash X$ it holds that $x \leq_\sigma y$.
  By definition of $\leq_\sigma$, this is the case whenever $x^\sigma \leq y^\sigma$,
  and since $x^\sigma \in X^\sigma$ and for all $y^\sigma$ in $Y^\sigma \backslash X^\sigma$ it
  holds that $x^\sigma \leq y^\sigma$, it follows that $X^\sigma \preccurlyeq Y^\sigma$.

  \medskip

  Consider the map $\varphi_\sigma : S^G \rightarrow (S^\sigma)^{G^\sigma}$ that maps sets
  $X \in S^G$ to $X^\sigma \in (S^\sigma)^{G^\sigma}$. This map is bijective, and by the above it respects the ordering, so the second claim follows.


\end{proof}

Lemma~\ref{lem:map} gives an efficient method to calculate minimal images under
different orderings without having to alter the underlying algorithm. The most
expensive part of this algorithm is calculating \(G^\sigma\), but this is still very
efficiently implemented in systems such as GAP, and also can be cached so it
only has to be calculated once for a given \(G\) and \(\sigma\).



\section{Dynamic Ordering of \(\Omega\)}

In Section \ref{sec:static}, we looked at methods for choosing an ordering for
\(\Omega\) that allows a minimal image algorithm to search more quickly.
There is a major limitation to this technique -- it does not make use of
the sets whose canonical image we wish to find.

In this section, instead of producing an ordering ahead of time, we will
incrementally define the ordering of \(\Omega\) as the algorithm
progresses.
At each stage we will consider exactly which extension of our partially
constructed ordering will lead to the smallest increase in the number of sets we
must consider.

We are not free to choose our ordering arbitrarily as we must still map two
sets in the same orbit of \(G\) to the same canonical image.
However, we can use different orderings for sets that are in different
orbits of \(G\).

Firstly, we will explain how we build the orderings that our algorithm uses.

\begin{algorithm}[!ht]
\begin{algorithmic}[1]
\Require Query workload $Q$, event stream $I$, \app\ graph $G$, hash table of snapshots $S$
\Ensure Hash table of results $R$ 
\State $G \leftarrow \emptyset$, $S, R \leftarrow$ empty hash tables
\ForAll {event $e \in I$ with $e.type=E$} 
    \State $//$ \textbf{\app\ graph construction}
    \ForAll {$q \in Q$ \text{ with event types }T}
        \ForAll {$E' \in T,\ E' \neq E$}
            \State $G_{E'} \leftarrow \mathit{getGraphlet}(G,E')$,
            $G_{E'}.\mathit{active} \leftarrow \mathit{false}$
        \EndFor
    \EndFor
    \If {\textbf{not} $G_E.\mathit{active}$}
        \State $G_E \leftarrow \mathit{createGraphlet()}$, $G_{E}.\mathit{active} \leftarrow \mathit{true}$,
        $G \leftarrow G \cup G_E$
        \If {$G_E.\mathit{shared}$ by $Q_E \subseteq Q$}
            $x \leftarrow \mathit{createSnapshot()}$ 
            \ForAll {$q \in Q_E$}
                \ForAll{$E' \in \mathit{pt}(E,q), E' \neq E$}
                    \State $G_{E'} \leftarrow \mathit{getGraphlet}(G,E')$
                    \State $S(x,q) \leftarrow S(x,q) + sum(G_{E'},q)$ \hspace{0.5cm}$//$ Eq.~5
                \EndFor
            \EndFor
        \EndIf    
    \EndIf
    \State insert $e$ into $G_E$
    \State $//$ \textbf{Trend count computation}
    \If {$G_E.\mathit{shared}$ by $Q_E \subseteq Q$}
        \If {$\forall q \in Q_E\ pe(e,q)$ are identical}
            \State $count(e,Q_E) \leftarrow count(e,q)$ \hspace{2.3cm}$//$ Eq.~2
        \Else\ $y \leftarrow \mathit{createSnapshot()}$, $count(e,Q_E) = y$
            \ForAll {$q \in Q_E$}
                $S(y,q) \leftarrow count(e,q)$ \hspace{0.2cm}$//$ Eq.~2
            \EndFor
          \EndIf
    \Else\ $count(e,q)$ \hspace{5.2cm}$//$ Eq.~2
    \EndIf
    \ForAll{$q \in Q$}
  	    \If {$E \in \mathit{end}(q)$} 
  		    $R(q) \leftarrow R(q) + count(e,q)$ $//$ Eq.~3
        \EndIf
    \EndFor
\EndFor
\State \Return $R$
\end{algorithmic}
\caption{\app\ shared online trend aggregation}
\label{algo:snapshot-propagation}
\end{algorithm}



\section{Experiments}\label{sec:experiments}
We validate our approach using multiple datasets containing real-life data from the fields of criminal risk assessment, credit, lending, and college admissions. In each of the datasets we select a binary feature and treat it as the protected attribute (e.g., race or gender), which is the feature we require our trained classifier to behave fairly upon. Our proposed method performs well on all of these datasets, succeeding in removing unfairness almost entirely, at a very modest price in terms of accuracy.


\begin{table*}[h]
\centering
\resizebox{\textwidth}{!}{
\def\arraystretch{1.2}

\begin{tabular}{c c c | c | c | c || c | c | c || c | c | c |}

\cline{4-12}
&&&
\multicolumn{9}{ c| }{\textbf{COMPAS Dataset}}
\\ \cline{4-12}
&&&
\multicolumn{3}{ c|| }{\textbf{FPR Considerations}}&
\multicolumn{3}{ c|| }{\textbf{FNR Considerations}}&
\multicolumn{3}{ c| }{\textbf{Both Considerations}}
\\ \cline{4-12}
&&&
 $\mathbf{Acc.}$ &  $\mathbf{D_{FPR}}$ &  $\mathbf{D_{FNR}}$ &  $\mathbf{Acc.}$ &  $\mathbf{D_{FPR}}$ &  $\mathbf{D_{FNR}}$ &  $\mathbf{Acc.}$ &  $\mathbf{D_{FPR}}$ &  $\mathbf{D_{FNR}}$
\\  \cline{4-12}
\vspace*{-0.5ex}
\\ \cline{1-2} \cline{4-12}
\multicolumn{1}{ |c  }{} &
\multicolumn{1}{ c|  }{  \textbf{Our Method (AVD Penalizers)}}  &&
$\mathbf{0.660}$    &  $\mathbf{0.01}$  &  $0.04$ &
$\mathbf{0.653}$    &  $0.02$   &  $\mathbf{0.04}$ &
$\mathbf{0.654}$    &  $\mathbf{0.02}$  &  $\mathbf{0.04}$
\\ \cline{1-2} \cline{4-12}
\multicolumn{1}{ |c  }{} &
\multicolumn{1}{ c|  }{  \textbf{Our Method (SD Penalizers)}}  &&
$\mathbf{0.664}$    &  $\mathbf{0.02}$  &  $0.09$ &
$\mathbf{0.661}$    &  $0.05$   &  $\mathbf{0.03}$ &
$\mathbf{0.661}$    &  $\mathbf{0.02}$  &  $\mathbf{0.03}$
\\ \cline{1-2} \cline{4-12}
\multicolumn{1}{ |c  }{} &
\multicolumn{1}{ c|  }{  Zafar et al.~(\citeyear{disparatemistreatment})}  &&
$0.660$    &   $0.06$    &   $0.14$  &
$0.662$    &   $0.03$    &   $0.10$  &
$0.661$    &   $0.03$    &   $0.11$
\\ \cline{1-2} \cline{4-12}
\multicolumn{1}{ |c  }{} &
\multicolumn{1}{ c|  }{  Zafar et al. Baseline~(\citeyear{disparatemistreatment})}  &&
$0.643$    &   $0.03$    &   $0.11$  &
$0.660$    &   $0.00$    &   $0.07$  &
$0.660$    &   $0.01$    &   $0.09$
\\ \cline{1-2} \cline{4-12}
\multicolumn{1}{ |c  }{} &
\multicolumn{1}{ c|  }{  Hardt et al.~(\citeyear{hardt})}  &&
$0.659$    &  $0.02$    &   $0.08$  &
$0.653$    &  $0.06$   &    $0.01$  &
$0.645$    &  $0.01$   &    $0.01$
\\ \cline{1-2} \cline{4-12}
\multicolumn{1}{ |c  }{} &
\multicolumn{1}{ c|  }{  \textbf{Vanilla Regularized Logistic Regression}}  &&
$\mathbf{0.672}$    &   $\mathbf{0.20}$    &   $\mathbf{0.30}$  &
$\mathbf{0.672}$    &   $\mathbf{0.20}$    &   $\mathbf{0.30}$  &
$\mathbf{0.672}$    &   $\mathbf{0.20}$    &   $\mathbf{0.30}$
\\ \cline{1-2} \cline{4-12}
\end{tabular}
}
\vspace{3mm}
\caption{Performance comparison on the COMPAS dataset. For the approaches in bold -- Accuracy, FPR difference and FNR difference are evaluated on the test set, averaging over five runs and using a 70-30 training/test split. The performance of the remaining three approaches is stated as reported in Zafar et al.~(\citeyear{disparatemistreatment}).} \label{table:comparison_results}
\end{table*}



\begin{figure*}[b]
  \includegraphics[scale=0.6]{compas0-400.png}
  \caption{COMPAS Dataset. Accuracy, FPR difference ($\mathbf{D_{FPR}}$), and FNR difference ($\mathbf{D_{FNR}}$) (all evaluated on the test set) of the learned classifier, as a function of the weight $c=c_1 = c_2 \geq 0$ placed on the fairness penalizer terms. On the left we use the Absolute Value Difference (AVD) penalizer, and the Squared Difference (SD) penalizer on the right, both as presented in Section~\ref{regularization}. ``Relaxed FPR/FNR Diff.'' plots the value of the relevant penalization term.} %In this particular run, parameters chosen for the absolute value relaxation were: $c=80, q_c=60$, and for the squared relaxation: $c=220, q_c=30$.}
  \label{fig:compas}
\end{figure*}


\subsection{Implementation}
\textbf{Our method} 
%We instantiate our method in the following way: Given dataset $Q$, we split it randomly into a training set $S$ (which we will use for learning) and a test set $T$ (which we will only use for reporting performance). 
For the purpose of comparison with  Zafar et al.~(\citeyear{disparatemistreatment}) and Hardt et al.~\cite{hardt} on the COMPAS data, we use a parameter $c$ to induce three possible combinations of weights on the FPR and FNR penalization terms: $c = c_1$ and $c_2 = 0$; $c_1 = 0$ and $c = c_2$; and $c = c_1 = c_2$. For the other three datasets, we consider only $c = c_1 = c_2$.\footnote{The reason for varying the values of $c$ in the training phase is since we shifted to a proxy problem, in which we rely on the distance from the decision boundary rather the actual classifications. 
%Our hope is that there is no need for a worst-case cross validation between all of the combinations of $c_1, c_2, c_3$, and that the training scheme we propose is sufficient. 
It is possible, of course, that even better results are attainable using our scheme with other combinations of $c_1, c_2$, and $q$.} To explore the accuracy/fairness trade-off curve for the relaxed optimization problem~(\ref{eq:2}), we train for different values of $c$, starting at $c=0$ (which is just standard logistic regression), and growing gradually.



Given a dataset $Q$ and fixing a $d_1, d_2 \in \{0, 1\}$ of interest, we use the following training scheme:
\begin{enumerate}
\item Split $Q$ at random into training set $S$ and test set $T$.
\item For each $c$, perform cross-validation on $S$ to select the corresponding best value $q_c$ for the regularization parameter.
\item For each $(c,q_c)$, let $\theta_c = \argmin\limits_{\theta} \text{Proxy}(\theta;S,c,c,q_c)$.
\item Select $\theta^* \in \argmin\limits_{\theta_c} \text{Objective}(\theta_c;S,d_1,d_2)$.
\item Evaluate performance using $\theta^*$ on test set $T$.
\end{enumerate}
We report the average of five such runs, each with a fresh training-test split.




%We instantiate our method by solving the relaxed optimization problem~(\ref{eq:2}), in place of the original, non-convex problem~(\ref{eq:1}).  
%We test our approach with three different combinations of weights on the penalization terms:
%\katrina{What are the $d$, and how are they related to the $c$s?}
%\begin{enumerate}
%\item FPR considerations only: $d_1 = 1, d_2 = 0$.
%\item FNR considerations only: $d_1 = 0, d_2 = 1$.
%\item Both FPR, FNR considerations, assigned similar significance: $d_1 = 1, d_2 = 1$.
%\end{enumerate}
%One could, of course, pick any other combination of the FPR and FNR penalty weights.

%\katrina{I don't understand how the below is distinct from the list above}
%Learning is done by training the parameters of a logistic regressor to solve~\ref{eq:2}, while picking the value of $c_1, %c_2$ as the following:
%\begin{enumerate}
%\item FPR considerations only: $c_1 = c \geq 0$, $c_2 = 0$.
%\item FNR considerations only: $c_1 = 0$, $c_2 = c \geq 0$.
%\item Both FPR, FNR considerations, assigned similar significance: $c_1 = c_2 = c \geq 0$
%\end{enumerate}



% We then cross-validate to pick the best $c_3$ (the weight on the standard $\ell_2$-regularization term) given $c$.\footnote{The reason for varying the values of $c$ in the training phase is since we shifted to a proxy problem, in which we rely on the distance from the decision boundary rather the actual classifications. 
%Our hope is that there is no need for a worst-case cross validation between all of the combinations of $c_1, c_2, c_3$, and that the training scheme we propose is sufficient. 
%It is possible, of course, that even better results are attainable using our scheme with other combinations of $c_1, c_2, c_3$.} For each such combination, we report results as the averages of multiple \katrina{how many?} different runs, each time splitting data randomly into training and test sets.
%\yahav{We need to shorten this description.}

We solve the relaxed convex optimization problem using the CVXPY solver. Due to stability issues with large training sets, we use a train/test split of 30-70 on the larger datasets, rather than 70-30 as on the COMPAS dataset\footnote{The code implementing our method can be found at https://github.com/jjgold012/lab-project-fairness}.

%
%
%We then report the results (as evaluated on the test set) attained by a regressor $\theta \in \mathbb{R}^d$ that minimizes (on the training set $S$) a weighted combination of the $0$-$1$ loss and the differences in FPR and FNR across populations:
%\begin{equation*}
%\begin{aligned}
%&\underset{\theta}{\text{argmin}}
%& & L_{S}^{0\text{-}1}(\theta) \\
%&&& + d_1|FPR_{A=0}(\theta;S)-FPR_{A=1}(\theta;S)| \\
%&&& + d_2|FNR_{A=0}(\theta;S)-FNR_{A=1}(\theta;S)|
%\end{aligned}
%\end{equation*}
%
%\katrina{What is $d_1$ vs. $c_1$ etc.?}



%For classification, we decided use a standard cut-off threshold of $c=0.5$. There are of course, further possible interactions between the FPR, FNR considerations, and picking a certain cut-off level. These are not straightforward, since  these interactions are data-specific. 



%allows for flexibility in picking the values of $c_1, c_2$, which reflect the significance we wish to place on the objectives of achieving accuracy, equal FPR, and equal FNR. As for $c_3$, we will want to find the value of it that achieves the best results, for any combined objective of accuracy and fairness defined by a specific selection of $c_1,c_2$. Therefore, given a specific selection of $c_1, c_2$, we apply cross-validation to select the value of $c_3$. 




We briefly describe the other algorithmic approaches to which we compare:\\
\textbf{Zafar et al.}~(\citeyear{disparatemistreatment}) performs optimization by considering a proxy for the bias: the covariance between the samples' sensitive attributes and the signed distance between the feature vectors of misclassified users and the classifier decision boundary.\\
\textbf{Zafar et al. Baseline}~(\citeyear{disparatemistreatment}) tries to enforce equal FP/FN rates on the different groups by introducing different penalties for misclassified data points with different sensitive attribute values during the training phase.\\
\textbf{Hardt et al.}~(\citeyear{hardt}) performs post-processing on a standard trained (unfair) logistic regressor, picking different decision thresholds for different groups, and possibly adding randomization.


\subsection{Experimental Results}

In what follows, we use the following notation, given a trained classifier $\hat{Y}$:
\begin{align*}
\mathbf{D_{FPR}}&=\left|FPR_{A=0}(\hat{Y})-FPR_{A=1}(\hat{Y})\right| \\ 
\mathbf{D_{FNR}}&=\left|FNR_{A=0}(\hat{Y})-FNR_{A=1}(\hat{Y})\right|
\end{align*}
The values $FPR_{A=0}(\hat{Y})$, $FPR_{A=1}(\hat{Y})$, $FNR_{A=0}(\hat{Y})$, $FNR_{A=1}(\hat{Y})$ are reported as evaluated on the test set.

\paragraph{The COMPAS Dataset\footnote{https://github.com/propublica/compas-analysis}} The Correctional Offender Management Profiling for Alternative Sanctions (COMPAS) records from Broward County, Florida 2013-2014, made available online by ProPublica, are perhaps the best-studied data in the context of fairness.  The goal in this scenario is to successfully predict recidivism within two years, based on features such as age, gender, race, number of prior offenses, and charge degree. The dataset contains 5,278 samples. The protected attribute in this scenario is race, where $A$ indicates black or white. We filtered the dataset using the same features as Zafar et al.~(\citeyear{disparatemistreatment}), to allow for comparison.

%\begin{table}[h]
%\centering
%\begin{tabularx}{\columnwidth}{c|c|c|c}
%\hline
%  &  Recid. ($y = 1$)        & No Recid.  ($y = 0$)       & Total \\ \hline
%Black &  $ 1661   $ & $ 1514 $ &  $ 3175 $ \\ \hline
%White &  $ 822   $  & $1281  $ &  $ 2103 $ \\ \hline
%Total &  $ 2483  $  & $2795 $ &  $ 5278 $ \\\hline
%\end{tabularx}
%\caption{Statistics of the ProPublica COMPAS data.} \label{table:compas-stats}
%\label{tab:stats}
%\end{table}
%\vspace{-1em}

%\begin{table}[h]
%\centering
%\begin{tabularx}{\columnwidth}{c|c}
%\hline
%Feature  &  Description \\ \hline
%Age Category &  $<25$, between $25$ and $45$, $>45$ \\
%Gender &  Male or Female \\
%Race &  White or Black \\
%Priors Count &  0--37 \\
%Charge Degree &  Misconduct or Felony \\
%\hline
%2-year-recid. & Whether or not the  \\
%(target feature)  & defendant recidivated within two years
%\end{tabularx}
%\caption{Description of features used from ProPublica COMPAS data.} \label{table:compas-features}
%\label{tab:features}
%\end{table}




\begin{table*}[t]
\centering
\caption{A description of the datasets used, along with parameters of the training procedure used for each.}
\label{table:datasets_description}
\begin{adjustbox}{max width=\textwidth}
\begin{tabular}{|l|l|l|l|l|l|l|l|}
\hline
\textbf{Dataset} & \textbf{No. Samples} & \textbf{No. Features} & \textbf{Train/Test Split} & \textbf{No. Repetitions} & \textbf{No. Folds in CV} & \textbf{Protected Feature} & \textbf{Target Variable} \\ \hline
COMPAS           & 5,278                     & 5                          & 70-30                     & 5                        & 5                                 & Race                       & 2-Year-Recidivism        \\ \hline
Adult            & 30,162                    & 10                         & 30-70                     & 5                        & 5                                 & Gender                     & Income Over/Under 50K    \\ \hline
Default          & 30,000                    & 23                         & 30-70                     & 5                        & 3                                 & Gender                     & Defaulting On Payments   \\ \hline
Admissions       & 20,839                    & 17                         & 30-70                     & 5                        & 3                                 & Race                       & Passing Bar Exam         \\ \hline
\end{tabular}
\end{adjustbox}
\end{table*}


\begin{table*}[t]
\centering
\resizebox{\textwidth}{!}{
\def\arraystretch{1.2}

\begin{tabular}{c c c | c | c | c || c | c | c || c | c | c |}

\cline{4-12}
&&&
\multicolumn{3}{ c|| }{\textbf{Adult Dataset}}&
\multicolumn{3}{ c|| }{\textbf{Default Dataset}}&
\multicolumn{3}{ c| }{\textbf{Admissions Dataset}}
\\ \cline{4-12}
%&&&
%\multicolumn{3}{ c|| }{\textbf{Both Considerations}}&
%\multicolumn{3}{ c|| }{\textbf{Both Considerations}}&
%\multicolumn{3}{ c| }{\textbf{Both Considerations}}
%\\ \cline{4-12}
&&&
 $\mathbf{Acc.}$ &  $\mathbf{D_{FPR}}$ &  $\mathbf{D_{FNR}}$ &  $\mathbf{Acc.}$ &  $\mathbf{D_{FPR}}$ &  $\mathbf{D_{FNR}}$ &  $\mathbf{Acc.}$ &  $\mathbf{D_{FPR}}$ &  $\mathbf{D_{FNR}}$
\\  \cline{4-12}
\vspace*{-0.5ex}
\\ \cline{1-2} \cline{4-12}
\multicolumn{1}{ |c  }{} &
\multicolumn{1}{ c|  }{  \textbf{Our Method (AVD Penalizers)}}  &&
$\mathbf{0.776}$    &  $\mathbf{0.00}$  &  $\mathbf{0.04}$ &
$\mathbf{0.807}$    &  $\mathbf{0.00}$   &  $\mathbf{0.01}$ &
$\mathbf{0.950}$    &  $\mathbf{0.01}$  &  $\mathbf{0.00}$
\\ \cline{1-2} \cline{4-12}
\multicolumn{1}{ |c  }{} &
\multicolumn{1}{ c|  }{  \textbf{Our Method (SD Penalizers)}}  &&
$\mathbf{0.783}$    &  $\mathbf{0.00}$  &  $\mathbf{0.09}$ &
$\mathbf{0.806}$    &  $\mathbf{0.01}$   &  $\mathbf{0.02}$ &
$\mathbf{0.950}$    &  $\mathbf{0.00}$  &  $\mathbf{0.00}$
\\ \cline{1-2} \cline{4-12}
\multicolumn{1}{ |c  }{} &
\multicolumn{1}{ c|  }{  \textbf{Vanilla Regularized Logistic Regression}}  &&
$\mathbf{0.800}$    &   $\mathbf{0.08}$    &   $\mathbf{0.39}$  &
$\mathbf{0.807}$    &   $\mathbf{0.01}$    &   $\mathbf{0.05}$  &
$\mathbf{0.951}$    &   $\mathbf{0.16}$    &   $\mathbf{0.02}$
\\ \cline{1-2} \cline{4-12}
\end{tabular}
}
\vspace{3mm}
\caption{Performance on the Adult, Loan Default, and Admissions datasets, penalizing for both FPR and FNR difference. Accuracy, FPR difference and FNR difference are evaluated on the test set, averaging over five runs and using a 30-70 training/test split.} \label{table:comparison_results_rest}
\end{table*}


In Table~\ref{table:comparison_results}, we compare the performance of our approach with that of three other techniques from the literature. Each method was trained based on logistic regression.  As a basis for comparison, we also present the performance of vanilla logistic regression, absent fairness considerations, with the regularization parameter selected via cross-validation.\footnote{Zafar et al.~(\citeyear{disparatemistreatment}) do not incorporate regularization in any of the approaches they report.}
%Results are reported as the averages of 5 different runs \katrina{Is that still correct?}, each time splitting data evenly and randomly into training and test sets. 
Results for Zafar et al., Zafar et al. baseline, and Hardt et al. appear here as reported in Zafar et al.~(\citeyear{disparatemistreatment}).\footnote{Our method selects the classifier based on the training set only and reports its performance over the test set. Results for the three other approaches, reported by Zafar et al.~(\citeyear{disparatemistreatment}), are based on tuning parameters after seeing the trade-off curve over the test set, and reporting according to the best selection of these parameters.}
%\katrina{Perhaps here is the right place for a footnote about the discrepancy with the Zafar baseline}

We find that the vanilla logistic regressor (absent fairness considerations) results in significant unfairness, as $\mathbf{D_{FPR}}=0.20$, and $\mathbf{D_{FNR}}=0.30$. The overall accuracy of this classifier measured on the test set was $0.672$.\footnote{Zafar et al.~(\citeyear{disparatemistreatment}) report a slightly different baseline of: Accuracy = 0.668, $\mathbf{D_{FPR}}=0.18$, $\mathbf{D_{FNR}}=0.30$.} Our SD penalization approach empirically achieves approximately the same accuracy as the Zafar et al.~(\citeyear{disparatemistreatment}) approach, with significantly better fairness. It is difficult to compare fairness-accuracy tradeoffs with the Hardt et al.~(\citeyear{hardt}) approach, since their accuracy is significantly lower than ours. A more direct comparison is possible by noting that our learned classifier can be post-processed to improve its fairness at a direct cost to accuracy. Hence, we can achieve accuracy of $0.659$ with $\mathbf{D_{FPR}} = \mathbf{D_{FNR}} = 0.01$, which compares very favorably with the Hardt et al. accuracy rate of 0.645 given the same FPR and FNR rates.\footnote{For completeness, we note that using a 50-50 training-test split (again not using the test set for parameter selection), our method (SD, both considerations) produces a classifier that provides: Accuracy = 0.659, $\mathbf{D_{FPR}} = 0.01, \mathbf{D_{FNR}} = 0.05$. This classifier can be post-processed to achieve rates of: Accuracy = 0.655, $\mathbf{D_{FPR}} = \mathbf{D_{FNR}} = 0.01$.}

Figure \ref{fig:compas} illustrates the accuracy/fairness trade-offs achievable using our scheme. Increasing the weight $c$ on the proxy fairness penalizers results in reducing their magnitude. The figure also illustrates how our relaxed penalizers succeed in tracking the real FPR and FNR differences. 
%
%
%\katrina{Must rewrite the following paragraph}
%We observe that our method succeeds in eliminating unfairness almost completely on the COMPAS dataset, while retaining most of the accuracy, when compared to the vanilla logistic regression. We achieve very low difference rates when penalizing for achieving each of the FPR and FNR criteria individually, and also for both. We achieve preferable results comparing to Zafar et al. and Zafar et al. baseline in all 3 scenarios, and also comparing to Hardt et al. in the settings of false positive/false negative considerations only. In the setting of both considerations - The Hardt et al. method removes a larger portion of the unfairness, however it results in major accuracy loss as it achieves accuracy rate of 0.645 in comparison to our method which results in accuracy of 0.665, retaining most of the original accuracy rate while removing most of the unfairness.




%The Hardt et al.~\cite{hardt} approach as reported removes a smaller portion of the bias in the different scenarios, however for FP/FN constraints alone, it provides higher accuracy rates. The Zafar et al.~(\citeyear{disparatemistreatment}) approach as reported retains significant bias (in most cases), but in some cases  achieves slightly superior accuracy rates to the methods above. 

%These performance comparisons are incomplete in the sense that each of the compared techniques has the potential to trade off between accuracy and fairness, using some degree of parameter tuning; what we report here is only one point on the achievable trade-off frontier for each algorithm. The ``correct'' trade-off, and, in particular, the best manner in which to weigh unfairness in the FPR against unfairness in the FNR, are matters of opinion. We have chosen to report our method's performance under parameters designed to very aggressively mitigate unfairness, at some cost to the accuracy.

%It would certainly be desirable to evaluate these and other approaches to fair learning on other datasets and on different tasks, particularly on larger datasets, which might afford both greater accuracy and better bias-reduction. The present empirical evaluations, however, suggest that our regularization-based approach provides a new tool worthy of consideration---we succeed in almost entirely eliminating bias on the hold-out set, at a modest price in terms of accuracy.

%Due to the fact that our true objective includes the original non-convex penalization terms, our approach does not carry any formal guarantees. However, the ease of implementation, generality, and empirical results are encouraging. Figure~\ref{fig:test1} illustrates the rate of convergence to a fair, accurate classifier on this dataset.
%In terms of computation costs, given that at each iteration we must calculate the gradient according to the FPR and FNR regularizers, we are required to predict the labels for the entire training set at each step. 
%However, this does not pose a computational burden, as it is already required by the (classic) gradient descent algorithm in our logistic regressor fitting scheme. Furthermore, when given a sufficiently large dataset (one or two orders of magnitude larger than the one currently available for the COMPAS scores data), this could be relaxed to sampling only a mini-batch of samples from the training data set at each iteration (much as is done in stochastic gradient descent).






\subsection{Additional Datasets}


Table~\ref{table:datasets_description} provides summary statistics on each of the datasets on which we tested our approach. We also briefly describe the datasets below. 


{\bf The Adult Dataset}\footnote{http://archive.ics.uci.edu/ml/datasets/Adult} is based on 1994 US Census data. The task we consider is to predict whether the income of each individual is over or under 50K dollars per year, based on features such as occupation, marital status, and education. The protected attribute selected in this task is gender. 

{\bf The Loan Default Dataset}\footnote{{\scriptsize https://archive.ics.uci.edu/ml/datasets/default+of+credit+card+clients}}
contains data regrading Taiwanese credit card users. The task we consider is to predict whether an individual will default on payments, based on features such as history of past payments, age, and the amount of given credit. The protected attribute is gender.

{\bf The Admissions Dataset}\footnote{http://www2.law.ucla.edu/sander/Systemic/Data.htm}
contains records of law school students who went on to take the bar exam. The task we consider is to predict whether a student will pass the exam based on features such as LSAT score, undergraduate GPA, and family income. The protected attribute is set to race.

Table~\ref{table:comparison_results_rest} describes the performance of our approach on these datasets, and Figures~\ref{fig:adult},~\ref{fig:default}, and~\ref{fig:lawschool} illustrate the fairness-accuracy trade-offs we achieve in each context. Overall, we see that unfairness is nearly eliminated while accuracy remains quite high. The dataset on which accuracy suffers most under our approach is the Adult dataset, which is also the dataset on which the vanilla regression is the most unfair.


\begin{figure*}[]
  \includegraphics[scale=0.6]{adult0-800.png}
  \caption{Adult Dataset. Fairness-Accuracy tradeoffs, as in Figure~\ref{fig:compas}.}
  \label{fig:adult}  
\end{figure*}



\begin{figure*}[]
  \includegraphics[scale=0.6]{default0-50.png}
  \caption{Loan Default Dataset. Fairness-Accuracy tradeoffs, as in Figure~\ref{fig:compas}.}
  \label{fig:default}
\end{figure*}



\begin{figure*}[]
  \includegraphics[scale=0.6]{admissions0-400.png}
  \caption{Admissions Dataset. Fairness-Accuracy tradeoffs, as in Figure~\ref{fig:compas}.}
  \label{fig:lawschool}
\end{figure*}




\section{Conclusions}

We present a general framework and a new set of algorithms for finding the
canonical image of a set under the action of a permutation group. Our
experiments show that our new algorithms outperform the previous state of the art,
often by orders of magnitude.

Our basic framework runs on the concept of refiners and selectors and is not
limited to finding only canonical images of subsets of \(\Omega\). In future
work we will investigate families of refiners and selectors that allow finding
canonical images for many other combinatorial objects.

\vspace{1cm}

\textbf{Acknowledgements.}

All authors thank the DFG (\textbf{Wa 3089/6-1}) and the EPSRC CCP CoDiMa
(\textbf{EP/M022641/1}) for supporting this work. The first author would like to
thank the Royal Society, and the EPSRC (\textbf{EP/M003728/1}). The third
author would like to acknowledge support from the OpenDreamKit Horizon 2020
European Research Infrastructures Project (\#676541). The first and third author
thank the Algebra group at the Martin-Luther Universit\"at
Halle-Wittenberg for the hospitality and the inspiring environment. The fourth
author wishes to thank the Computer Science Department of the University of
St~Andrews for its hospitality during numerous visits.

\bibliography{canonical}
\bibliographystyle{plain}
\end{document}
