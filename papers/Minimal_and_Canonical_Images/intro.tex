% !TEX root = MinCanRevised.tex

\section{Background}

Many combinatorial and group theoretical problems
\cite{DBLP:journals/combinatorics/Soicher99,gent2000symmetry,Distler2012} are
equivalent to finding, given a group \(G\) that acts on a finite set \(\Omega\)
and a subset \(X \subseteq \Omega\), a partition of \(X\) into subsets that are
in the same orbit of \(G\).

We can solve such problems by taking two elements of \(X\) and searching for an element of
\(G\) that maps one to the other. However, this  requires a possible \(O(|X|^2)\) checks,
if all elements of \(X\) are in different orbits.

Given a group \(G\) acting on a set \(\Omega\), a canonical labelling function
 maps each element of \(\Omega\)
to a distinguished element of its orbit under \(G\). Using a canonical labelling function we can check if
two members of \(\Omega\) are in the same orbit by applying the canonical labelling function to both
and checking if the results are equal. More importantly, we can
solve the problem of partitioning \(X\) into orbit-equivalent subsets by performing \(O(|X|)\)
canonical image calculations. Once we have the canonical image of each element, we can organize
the canonical images into equivalence classes by sorting in \(O(|X|log(|X|))\) comparisons, or expected \(O(|X|)\)
time by placing them into a hash table. This is because checking
if two elements are in the same equivalence class is equivalent to checking if their canonical images
are equal.

The canonical image problem has a long history.
Jeffrey Leon \cite{Leon} discusses three types of problems on permutation groups
-- subgroup-type problems (finding the intersection of several
groups), coset-type problems (deciding whether or not the intersection of a
series of cosets is empty, and if not, finding their intersection) and
canonical-representative-type problems. He claims to have an algorithm to efficiently
solve the canonical-representative problem, but does not discuss it further. His comments
have inspired mathematicians and computer scientists
to work on questions related to minimal images and canonical images.

One of the most well-studied canonical-image problems is the canonical graph
problem. Current practical systems derive from partition refinement
techniques, which were first practically used for graph automorphisms by McKay
\cite{McKay80} in the \texttt{Nauty} system. There have been a series of
improvements to this technique, including \texttt{Saucy} \cite{Saucy},
\texttt{Bliss} \cite{JunttilaKaski:ALENEX2007} and \texttt{Traces} \cite{McKay201494}.
A comparison of these systems can be found in \cite{McKay201494}.

We cannot, however, directly apply the existing work for graph isomorphism to
finding canonical images in arbitrary groups. The reason is that McKay's Graph Isomorphism algorithm only considers finding the canonical image of a graph
under the action of the full symmetric group on the set of vertices. Many
applications require finding canonical images under the action of subgroups of
the full symmetric group.

One example of a canonical labelling function is, given a total ordering on \(X\), to
map each value of \(X\) to the smallest element in its orbit under \(G\). This
\emph{Minimal image problem} has been treated by Linton in
\cite{Linton:SmallestImage}. Pech and Reichard \cite{Pech2009} apply techniques
similar to Linton's to enumerate orbit representatives of subsets of $\Omega$
under the action of a permutation group on $\Omega$.
Linton gives a practical algorithm for finding the smallest image of a set under
the action of a given permutation group. Our new algorithm, inspired by Linton's
work, is designed to find canonical images: we extend and generalize Linton's
technique using a new orbit-based counting technique. In this paper we first
introduce some notation and explain the concepts that go into the algorithm,
then we prove the necessary results and finish with experiments that demonstrate
how this new algorithm is superior to the previously published techniques. 