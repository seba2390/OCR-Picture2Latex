%
\section{Related Work}
%
\label{sec:relatedwork}
%
%%%%%%%%%%%%%%%%%%%%%%%%%%%%%%%%%
%
Differentiable rendering is a widely studied problem.
%
In this section, we will discuss the most relevant methods. 
%
We refer to Kato~\etal~\cite{kato2018renderer} for a recent detailed survey.
%
Existing differentiable rendering methods are either based on efficient but inaccurate direct illumination or more accurate but inefficient global illumination. 
%
%%%%%%%%%%%%%%%%%%%%%%%%%%%%%%%%%%%%%%%%%%%%%%%%%%%%%%%%%%%%%%%%%%%%%%%%%%%%
%
\subsection{Differentiable Rendering}
%
Efficient but approximate differentiable approaches can be further split into two categories based on the type of approximations. 
%
Some works~\cite{Loper:ECCV:2014,kato2019vpl, kato2018renderer,laine2020modular} approximate gradients without modifying the rasterization step which has the advantage that camera visibility is modeled as in the real world.
%
However, the camera visibility computation here is non-differentiable.
%
In contrast, there are other works~\cite{Rhodin:2015, liu2019soft} that treat the objects in the scene as semi-transparent volumes.
%
This allows for differentiable camera visibility but their method does not reflect the real world properties of the object. 
%
All of the above methods do not account for the visibility of the light sources and hence cannot account for cast shadows and self-shadows. 
%
This leads to inaccurate results, e.g. the geometry can deform incorrectly to explain shadows in the image, or the recovered textures can contain baked in shadows.
%
In contrast, our proposed approximation to global illumination results in meaningful supervision even in the presence of shadows as we explicitly model them.
%
\par
%
Recently, physically-based differentiable rendering methods were proposed, which can also account for global illumination. 
%
These methods build up on Monte Carlo ray tracing which provides derivatives for arbitrary bounces of light. 
%
Li~\etal~\cite{li2018differentiable} proposed the first comprehensive method which can provide gradients for all scene parameters.
%
Zhang~\etal~\cite{10.1145/3355089.3356522} proposed a similar approach which also accounts for volumetric derivatives along with meshes.
%
Please refer to Zhao~\etal~\cite{pbr_tut} for a detailed analysis of the various physically-based differentiable rendering methods.
%
While these methods provide accurate supervision with respect to global illumination effects, they are very slow to evaluate as for a single pixel many rays have to be sampled and accumulated. 
%
This makes it difficult or nearly impossible to use them within the training of neural networks.
%
%%%%%%%%%%%%%%%%%%%%%%%%%%%%%%%%%
%
\begin{figure*}[t]
\centering
%
\includegraphics[width=0.9\textwidth]{figures/overview3.png} 
%
\caption
{
Overview of our approach. 
%
Given a surface mesh, we first approximate the geometry surface with a set of spheres.
%
The global visibility can be calculated as a combination of the visibility function for each sphere blocker in the spherical harmonics space, where the function is associated with texture, pose and illumination.
%
Combined with a rasterizer, an image can be rendered in a differentiable way. 
%
Therefore, we are able to optimize the different scene properties, such as geometry, texture, and illumination by comparing the rendered image against the target image.
%
}
%
\label{fig:overview}
%
\end{figure*}
%
%%%%%%%%%%%%%%%%%%%%%%%%%%%%%%%%%%%%%%%%%%%%%%%%%%%%%%%%%%%%%%%%%%%%%%%%%%%%
%
\subsection{Efficient Global Illumination}
%
Several methods in the computer graphics literature have explored faster global illumination approaches, see the survey of Ritschel~\etal~\cite{ritschel2012state}.
%
Here, precomputed Radiance Transfer (PRT) methods are related to our approach. 
%
Most PRT methods~\cite{dobashi1995quick,sloan2002precomputed,kautz2002fast} assume the geometry to be fixed, although some methods work with dynamic geometry~\cite{sloan2005local,iwasaki2007precomputed}. 
%
Very recently, PRT was also used for inverse problems~\cite{thul2020precomputed}. 
%
However, the scene geometry cannot be updated in this formulation.
%
Several approaches have been proposed for efficient computation of soft-shadows for dynamic scenes~\cite{kautz2004hemispherical,zhou2005precomputed,kontkanen2005ambient}.
%
Ren~\etal ~\cite{ren2006real} used spherical harmonics (SH) representations for the different scene components such as illumination and visibility and proposed an efficient, but non-differentiable method for computing SH products.
%
Zhou ~\etal ~\cite{zhou2005precomputed} proposed shadow fields to represent the light source radiance and occlusions, which allows for fast computation of the visibility. 
%
Efficient soft-shadow computation has mostly been explored in computer graphics for creating synthetic imagery. 
%
We investigate the inverse problem: Using an efficient \emph{differentiable} renderer for estimating scene parameters from monocular images.
%