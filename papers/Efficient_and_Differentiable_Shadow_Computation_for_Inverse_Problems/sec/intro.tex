%
%%%%%%%%%%%%%%%%%%%%%%%%%%%
%
\section{Introduction}
%
\label{sec:introduction}
%
%%%%%%%%%%%%%%%%%%%%%%%%%%%
%
% Motivation
%
Rendering virtual scenes, objects, and characters has a wide range of applications in movies, video games, and many other areas which require synthesis of realistic images.
%
While in these computer graphics applications the main interest is the generation of images from scene parameters like geometry, lighting, and texture, rendering can also be used to solve inverse problems, which attempt to recover exactly these scene parameters from real images.
%
Analysis-by-synthesis optimization is commonly used~\cite{thies2016face,Blanz} where the estimated scene parameters are rendered as a synthetic image and then compared with the reference. 
%
If the renderer is differentiable, an energy function which compares the renderings and the ground truth images can be used for optimization.
%
Differentiable rendering is also interesting for learning-based, notably neural network-based, approaches where ground truth annotations, e.g. of the dense geometry, are not easily available for large image-based training corpora.
%
Instead, differentiable rendering allows for self-supervised learning using an analysis-by-synthesis approach where the rendered image is compared to the real one.
%
This has been widely used in the vision and machine learning community, for solving problems such as reflectance estimation~\cite{azinovic2019inverse,DADDB18,li2020inverse}, free-viewpoint synthesis~\cite{thies2019deferred,lombardi2019neural}, and human performance capture~\cite{Habermann:2019:LRH:3313807.3311970,deepcap,tewari17MoFA,8496850}.
%
%%%%%%%%%%%%%%%%%%%%%%%%%%%
%
% rasterization based techniques
%
\par 
%
Most differentiable rendering methods rely on rasterization-based techniques which only consider direct illumination effects~\cite{kato2018renderer,ravi2020pytorch3d,liu2019soft,laine2020modular}. 
%
Shadows are global illumination effects, i.e., for any point in the scene, any other point could be occluding the light source, see Fig.~\ref{fig:teaser}.
%
Thus, they are not modeled by the direct illumination renderers. 
%
As a result, inverse methods supervised with such differentiable renderers produce undesired artifacts. 
%
For instance, the estimated texture, geometry, and illumination may exhibit baked in errors trying to represent real world effects, in particular due to shadows which are not accounted for by these simplified rendering assumptions.
%
In this paper, we address the problem of illumination visibility, i.e., whether the light source in a direction is visible from any point in the scene. 
%
Illumination visibility is simply called "visibility" for readability in the paper, not to be confused by camera visibility, i.e., which points of the scene are visible in the image.
%
%%%%%%%%%%%%%%%%%%%%%%%%%%%
%
% Global illuminatioon
%
\par 
%
To account for these limitations, differentiable ray tracing methods~\cite{li2018differentiable,10.1145/3355089.3356522} were proposed which use ray tracing methods to solve the rendering equation.
%
Some approaches~\cite{azinovic2019inverse} use them for reconstructing more accurate scene parameters compared to direct illumination-based techniques.
%
While these renderers can render shadows and other higher-order illumination effects, they are computationally inefficient, which makes training large networks practically impossible on consumer-grade hardware.
%
%%%%%%%%%%%%%%%%%%%%%%%%%%%
%
% our method
%
\par 
%
To this end, we propose a method for differentiable and efficient visibility computation for rendering scenes with soft-shadows.
%
Our work builds up on the literature of efficient global rendering~\cite{ren2006real,sloan2002precomputed,Guerrero08} where the goal is to approximate visibility for a faster runtime.
%
More precisely, our approach first approximates the scene geometry with spheres. 
%
These spheres are attached/rigged to the underlying geometry mesh, which allows for deforming and posing the mesh through the sphere representation.
%
Scene illumination is modeled with the commonly used spherical harmonics representation~\cite{10.1145/383259.383317}.
%
Interestingly, the same representation can also be used to model visibility, e.g. whether the incident illumination is occluded in any direction from a point in the scene.
%
This spherical harmonics representation allows for efficient rendering of soft shadows using fast spherical harmonics multiplications. 
%
We combine this soft shadow rendering with a diffuse spherical harmonics based shading model to obtain the final rendering which is fully differentiable enabling us to compute gradients with respect to geometry, light, and texture.
%
We show applications of this differentiable renderer, by using analysis-by-synthesis optimization in order to recover the rigid pose, surface deformation, scene illumination, and texture of objects in scenes with shadows.
%
%%%%%%%%%%%%%%%%%%%%%%%%%%%
%
% Summary
%
In summary, our contributions are:
%
\begin{itemize}
%
\item{A differentiable and efficient renderer which can synthesize soft-shadows for dynamic scenes.}
%
\item{The integration of our renderer in an optimization-based setup for the reconstruction of scene parameters from monocular images.}
%
\end{itemize}
%
We compare our approach to the state-of-the-art differentiable rendering techniques and show that our method offers a good trade-off between rasterization-based techniques which are efficient but do not model shadows, and the more accurate but inefficient ray tracing approaches.
%
We will make our implementation publicly available.