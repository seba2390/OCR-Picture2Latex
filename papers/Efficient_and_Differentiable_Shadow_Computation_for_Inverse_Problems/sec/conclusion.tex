%
\section{Discussion}
%
%%%%%%%%%%%%%%%%%%%%%%%%%
%
High frequency lighting cannot be modelled well by our approach due to the low-dimensional SH representation.
%
Fig.~\ref{fig:limitaion} shows one such case with incident illumination just from a single direction. 
%
Our method requires the precomputation of the embedded deformation graph. 
%
This graph is well-suited for one object category, however, it would not be sufficient to deform very different shapes.
%
Our approach does not consider inter-reflectance and non-diffuse surfaces, which are interesting directions for future work. 
%
%%%%%%%%%%%%%%%%%%%%%%%%%%%%%%%%%
%
\begin{figure}
	%
	\includegraphics[width=\linewidth]{figures/limitation1.PNG} 
	%
	\caption
	{
		Limitation.
		%
	    Our method fails to reconstruct high-frequency lighting, such as a directional light in this example.
	    %
	}
	%
	\label{fig:limitaion}
	%
\end{figure}
%
%%%%%%%%%%%%%%%%%%%%%%%%%
%
\section{Conclusion}
%
%%%%%%%%%%%%%%%%%%%%%%%%%
%
We proposed a method for efficient and differentiable shadow computation that can be used for various inverse problems. 
%
We show that our approach achieves competitive results compared to ray tracing methods at much faster runtimes.
%
Further, we outperform direct illumination renderers which do not model shadows.
%
We demonstrate that shadows are important cues in images, and take the first steps towards using efficient high-quality differentiable and shadow-aware rendering for inverse problems.
%
