\section{Limitations}
\label{sec:limit}

Using a robotic arm to automate physical IoT device interactions is an effective way to scale the collection of IoT traffic across device behaviors. However, the nature of collecting network data from physical IoT devices via robotics has certain constraints:

\subsection{Non-Physical User Interface Elements}
While a robotic arm can perform many types of interactions with IoT devices, such as pressing buttons or sliding switches, it cannot perform non-physical interactions such as voice commands. Even certain physical user interface elements, such as touch screens, can prove challenging for off-the-shelf robotic arms. Unlike physical buttons, there are infinite possible interactions to test for comprehensive coverage of touch screen presses, and the location of touch screen input elements may change as the result of prior interactions. Considering that some IoT devices, including the Echo-Show5, have touch screens and voice commands as part of their major functions, it is important to explore additional techniques (such as software emulators) that can automate these types of interactions. 

Despite this limitation, we believe that the method described in this paper will be useful to IoT researchers. Physical device interfaces (e.g.,~buttons and switches) have proved the most challenging to automate thus far and are almost ubiquitous in inexpensive consumer IoT products.
Our proposed method is designed to facilitate the collection of network traffic on devices with such physical interfaces. Additionally, combining a robotic arm with a microphone and an instrumented smartphone running devices' associated mobile applications would be sufficient to explore all possible user inputs for many common devices without requiring
tedious manual button pressing or dedicated development of device-specific emulators. 

\subsection{IoT Device Size}
Configuring the robotic arm to interact with the Echo-Show5 and the Sensi-Thermostat was feasible as both devices are small enough for the robotic arm to reach any location on the devices. In general, the size of the robotic arm limits the size of the IoT device that can be tested. Significantly larger devices (such as a smart refrigerator) exceed the maximum reach or range of motion of most hobbyist robotic arms. While larger and more precise robotic arms are available on the market, they are substantially more expensive. One possible solution is to have multiple smaller robotic arms operate on a single large device simultaneously, but this would require precise coordination of the arms to interoperate. However, this limitation does not raise significant concern, since most popular consumer IoT devices are small appliances that would be suitable for our approach.

\subsection{Environmental Sensor Data}
Many IoT devices include environmental sensors, such as thermometers, light sensors, accelerometers, and gyroscopes that determine their behaviors and network communications~\cite{apthorpe2019keeping} in conjunction with user interface elements.
The scope of possible readings from these sensors is not explored by our robotic arm approach and would require a software emulator or a laboratory with local environment controls. While it would be possible to place both a robotic arm and an IoT device into a chamber with controllable temperature or lighting and conduct permutation-based testing with each of these variables, this is outside the scope of this paper. 
