\section{Results}
\label{sec:evaluation}

Our study demonstrates that a robotic arm can be used to automate interactions with IoT devices in order to collect network traffic for research. Testing this approach with the Echo-Show5 and Sensi-Thermostat shows that it can collect IoT traffic that provides rigorous coverage of device behaviors with high correlations between button presses and captured packets.

\subsection{Visual Support for Automated IoT Traffic Collection}
\label{sec:results-visual}
Visualizations of traffic traces collected from the Echo-Show5 (Figure~\ref{fig:echo_io_shade}) and the Sensi-Thermostat (Figure~\ref{fig:therm_io}) during automated interactions with all physical buttons on each device verify that our approach produces interaction-correlated network data. We performed both experiments without network congestion to prevent TCP congestion control from causing unwanted variations in the collected traffic. In practice, nothing prevents future applications of our automated approach in network environments with background congestion.

\subsubsection{Echo Show 5} Traffic collection from the Echo-Show5 started at the relative timestamp of 0s and ended at the relative timestamp of 1533s. The robotic arm started pressing buttons on the device at 50s according to a permutation-based interaction sequence and concluded all button presses at 1000s.

\begin{figure*}[t]
    \centering
    \includegraphics[width=0.65\textwidth]{figures/echo_io_shade.jpg}
    \caption{Example Echo-Show5 TCP traffic showing significantly more packets during automated robotic arm interactions. This packet trace was collected over 1533 seconds using a permutation-based interaction sequence (Section~\ref{sec:interaction-seqs}) with 10 seconds between each interaction. The device sent and received 8.83 packets per second on average during the automated button presses and 2.25 packets per second on average after the automated interactions (idle time)}
    \label{fig:echo_io_shade}
\end{figure*}

There are clear traffic spikes during the time period with the automated button presses. The period with button presses contained a total of 8391 packets with an average of 8.83 packets per second. In contrast, the rest of the packet capture (idle time) contained only 1202 packets and an average of 2.25 packets per second. Note that we excluded packets prior to the beginning of the button presses when calculating these statistics, as these early packets are the result of device initialization at startup. 

The significantly higher amount of network traffic during the button presses indicates that our robotic arm's interactions with the Echo-Show5 indeed generates activity-related traffic compared to periods with no robot interaction. 
These results corroborate Apthorpe et al.'s previous findings~\cite{apthorpe2019keeping} that user interactions with an Amazon voice assistant cause detectable increases in traffic rates.

\subsubsection{Sensi-Thermostat} 

\begin{figure*}[t]
    \centering
    \includegraphics[width=0.65\textwidth]{figures/therm_io.png}
    \caption{Example Sensi-Thermostat TCP traffic during automated robotic arm interactions. This packet trace was collected over 950 seconds using a permutation-based interaction sequence (Section~\ref{sec:interaction-seqs}) with 10 seconds between each interaction. The device batches periodic messages approximately every 200 seconds in response to settings changes made by the intervening button presses.}    \label{fig:therm_io}
\end{figure*}

Traffic collection from the Sensi-Thermostat started at the relative timestamp of 0s and ended at the relative timestamp of 950s. The robotic arm button presses occurred from 0s to 950s at intervals of 10s. 

The Sensi-Thermostat sends periodic updates approximately every 200 seconds to reflect changes in settings caused by the intervening button presses. No TCP traffic is exchanged outside of these periodic updates. Unlike the Echo-Show5, the Sensi-Thermostat \textit{does not generate any TCP traffic} when the robotic arm is inactive and no buttons are pressed, because no periodic update is required. This allows us to conclude that the observed traffic spikes at a rate of 2 packets per second occurred as a direct result of the automated button presses. 
Compared to the Echo-Show5, there are significantly fewer captured packets for the Sensi-Thermostat; however, this is expected, as the Sensi-Thermostat is a simpler device.

The intermittently spiking traffic we observe corroborates  Apthorpe et al.'s previous findings~\cite{apthorpe2019keeping} that the traffic patterns of single-purpose consumer IoT devices (e.g., thermostats, lightbulbs, and outlets) are often directly and obviously correlated with user interactions and devoid of substantial TCP background traffic.
The observable correlation between the Sensi-Thermostat's network traffic spikes and the automated robotic button presses validates the effectiveness of our data collection approach for this device. 

\subsection{ML Support for Automated IoT Traffic Collection}
We used machine learning, along with the IoT traffic captured from the Echo-Show5, to verify the correlation between button presses and captured packets as described in Section~\ref{sec:mlmethod}. The success of our machine learning model further supports the ability of our automated interaction method to produce traffic containing information about device behavior that would be useful for network, security, or privacy research.

We performed the machine learning evaluation using the Echo-Show5 data instead of the Sensi-Thermostat data because the Sensi-Thermostat produced such a small amount of traffic (41 packets) that 1) it is easy to visually verify correlations between traffic spikes and button presses (Section~\ref{sec:results-visual}) and 2) there is not enough data to train an ML model. 
In comparison, the Echo-Show5 produced sufficient traffic for random forest training (8391 packets). 

\begin{figure*}[t]
    \centering
    \includegraphics[width=0.7\textwidth]{figures/repeated_alexa.jpg}
    \caption{Example Echo-Show5 TCP traffic with labeled button presses. This trace was collected with 15 presses of each of 4 physical buttons on the device and 10 seconds between each button press. The trace contains 8391 total packets.}
    \label{fig:alexa_ml_io}
\end{figure*}

We labeled each packet in the Echo-Show5 traffic with the number of the button most likely responsible for it being sent (Figure~\ref{fig:alexa_ml_io}). We randomly split the labeled traffic into 75\%/25\% training/test sets and conducted a grid search using the training data to choose hyperparameters for the random forest classifier. 
Training the classifier took only a few seconds using the scikit-learn library. 
We then tested the classifier on the test set and recorded the precision, recall, and $F_1$ scores, weighting each button's contribution to the average score by its relative cardinality in the test set. 

We repeated this training and testing process 50 times using the Echo-Show5 traffic as described in Section~\ref{sec:mlmethod}.  All average precision,  recall, and $F_1$ scores were approximately 0.96 with variances less than 1.0e-4, demonstrating a strong correlation between the device's network behavior and the robotic arm button presses (Table~\ref{tab:scores}). 

These high scores allow us to conclude that the captured traffic does provide substantial information about the robot arm interactions and would be useful for follow-up research about the network, security, or privacy implications of user interactions with the IoT device.


\begin{table}[t]
\centering
\begin{tabular}{llll}
\toprule
& \textbf{Precision}   & \textbf{Recall}  & $\mathbf{F_1}$       \\ \midrule
\textbf{Average Score} & 0.96 & 0.96 & 0.96 \\
\textbf{Variance} & $4.57\mathrm{e}{-5}$ & $4.89\mathrm{e}{-5}$ & $4.84\mathrm{e}{-5}$ \\
\bottomrule
\end{tabular}
\caption{Random forest classifier performance predicting Echo-Show5 button presses from collected network traffic. Average test scores and variances reported over 50 repetitions with randomly selected train/test set divisions.}
\label{tab:scores}
\end{table}      

These results also corroborate the possibility of inferring details of user interactions with consumer IoT devices from network traffic~\cite{apthorpe2019keeping, acar2020peek, trimananda2020packet} and show that our automated data collection approach could be useful for similar future studies. 