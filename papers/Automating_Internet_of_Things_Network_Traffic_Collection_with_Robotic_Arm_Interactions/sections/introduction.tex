\section{Introduction}
\label{sec:introduction}
% background on iot and networks
The Internet of things (IoT) refers to the wide variety of physical objects increasingly connected to the Internet. IoT devices range from common household items, such as thermostats, light bulbs, and door locks, to medical products, wearables, and industrial sensors. For example, a ``smart'' (IoT) thermostat may be able to receive location data from a smart car and automatically adjust the home temperature when a user leaves work.
Although IoT devices provide unprecedented convenience and efficiency, they also raise concerns about security and privacy. Users report privacy fears associated with constant connectivity and always-on environmental sensors~\cite{naeini2017privacy, zheng2018user, apthorpe2020you, huang2020amazon}, and researchers have identified insecurities in IoT device software and network communications~\cite{loi2017systematically, chu2018security, wang2019looking}. These issues, combined with the increasing popularity of consumer IoT devices, are motivating many studies analyzing IoT network traffic to detect and prevent privacy and security vulnerabilities. 

Collecting IoT network traffic for these studies typically involves researchers manually interacting with devices in a laboratory setting. Researchers press buttons, touchscreens, or other user interface elements on the devices in an attempt to mimic real-world user behaviors or collect traffic from as many device states as possible. For example, a researcher may attempt to record network traffic while pressing every button on a device in sequence or by pressing patterns of buttons (or other user interface elements) in order to examine the full scope of device behavior~\cite{apthorpe2019keeping}. However, these types of manual device interactions are time-consuming and are unlikely comprehensive, posing a significant challenge to IoT network research (Section~\ref{sec:related}). 

IoT studies have also utilized crowdsourcing to acquire consumer IoT network traffic~\cite{huang2020iot, mazhar2020characterizing}. This can provide large quantities of realistic data; however, crowdsourcing typically requires extensive data collection platform development, and user reports of device behavior during traffic collection are prone to inaccuracy. 

In this paper, we present a novel method for automating IoT network traffic collection: configuring a robotic arm to interact with IoT devices according to formalized interaction sequences (Section~\ref{sec:method}), particularly on IoT devices with physical user interface (UI) elements such as buttons and switches. We focus on devices with physical UI elements because the collection of network traffic on such devices has been proven to be challenging and costly (Section \ref{sec:related}). At the same time, the application of robotic arm-assisted traffic collection is more suitable for these devices compared to devices with non-physical UI elements, which we discuss in Section \ref{sec:limit}.
This approach can simulate real user behaviors, provide comprehensive coverage of possible user/device interactions, and eliminate tedious manual button pressing. Recording Internet traffic to and from a device during robotic arm interactions provides a source of network data that can be used for security testing, privacy evaluation, or other IoT research.
As far as we are aware, there has been no prior use of robotics to automate IoT research in this manner, raising significant potential for cross-disciplinary follow-up research.

We demonstrate the effectiveness of our approach by configuring a robotic arm to press physical buttons on two devices: an Amazon Echo Show 5~\cite{echo} (a smart speaker with multiple user interfaces) and an Emerson Sensi Wi-Fi Smart Thermostat~\cite{sensi} (a household thermostat with only physical buttons) (Section~\ref{sec:evaluation}). We verify that this produces network behavior correlated with robot/device interactions by testing a variety of permutation-based button press sequences while collecting Internet traffic. We then train a machine learning model (random forest classifier) to accurately infer ($F_1 > 0.95$) which specific buttons are pressed on the devices from network traffic alone. 
This machine learning task is inspired by the various metadata-based inference attacks in the consumer IoT privacy literature~\cite{apthorpe2019keeping, acar2020peek, trimananda2020packet}.
Success at this task indicates that the captured traffic provides substantial information about interactions with the devices 
and corresponding device behavior and would be useful for follow-up network, security, or privacy analyses.

Employing robotics for collecting network data from physical IoT devices eliminates many drawbacks of manual data collection or crowdsourcing. Our approach provides rigorous interaction coverage and high scalability, allowing for easier collection of Internet traffic for IoT network audits. 
Although we focus on network data collection in this paper, there is great potential for robotic automation of other research involving cyber-physical devices, including fuzz testing and usability testing.
We hope that this approach and our provided source code\footnote{\href{https://github.com/Chasexj/Automated\_IoT\_Traffic\_Generation}{https://github.com/Chasexj/Automated\_IoT\_Traffic\_Generation}} will facilitate continued IoT research (Sections~\mbox{\ref{sec:limit}--\ref{sec:future}}).