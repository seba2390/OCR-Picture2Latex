\section{Background \& Related Work}
\label{sec:related}

Collecting network data from consumer IoT devices has posed a consistent challenge for IoT research. Studies typically use either manual data collection or crowdsourcing to collect IoT network traffic; however, both approaches have substantial drawbacks. 
In this section, we review the challenges of these traditional methods for IoT data collection and discuss how our approach provides significant advantages in terms of automation, monetary cost, verifiability, and scalability (Table~\ref{tab:compare}).

\begin{table}[t]
\centering\footnotesize
\begin{tabular}{lllll}
\toprule
& \textbf{Manual}  & \textbf{Crowdsourcing}  & \textbf{Robotic Arm}       \\ \midrule
\textbf{Automated}          & \textcolor{red}{\xmark}  & \textcolor{red}{\xmark}   & \textcolor{green}{\checkmark}        \\
\textbf{Cost}         & \textcolor{red}{\$\$}     & \textcolor{red}{\$\$\$}          & \textcolor{green}{\$}      \\ 
\textbf{Verifiable} & \textcolor{green}{\checkmark}      & \textcolor{red}{\xmark}           & \textcolor{green}{\checkmark}     \\ 
\textbf{Scalable}      & \textcolor{red}{\xmark}    & \textcolor{green}{\checkmark}         & \textcolor{green}{\checkmark}       \\ \bottomrule
\end{tabular}
\caption{Comparison of research methods for IoT traffic collection. Our robotic arm method automates verifiable physical interactions with IoT devices without extensive manual effort or expensive crowdsourcing campaigns.}
\label{tab:compare}
\end{table}     

\subsection{Manual IoT Traffic Collection}
Many studies of consumer IoT network traffic have involved data collection in a controlled laboratory environment. 
Researchers acquire devices relevant to their research question, instrument the devices,
and then manually interact with the devices to test their behavior and collect data for online and offline analysis.
Studies utilizing manual data collection have focused on privacy-violating inferences from IoT network traffic~\cite{apthorpe2019keeping, acar2020peek, edu2020smart} and network vulnerabilities in specific classes of devices (e.g., children's toys~\cite{chu2018security, shasha2019playing}), sparking increased consumer awareness, regulatory action, and manufacturer attention to consumer IoT security and privacy. 

Manual data collection has several benefits. First, researchers can precisely control the network environment, ensuring that recorded network traffic actually corresponds to specific devices or user interactions. This greatly simplifies ground-truth labeling of collected network traffic for input into supervised machine learning algorithms or for other follow-up analyses.
Second, researchers can collect data while subjecting the device or network to active attacks that would be unethical outside of a controlled environment. For example, researchers can flood a device with denial-of-service traffic or attempt to install bogus TLS certificates. 

Unfortunately, manual data collection also has several serious drawbacks that have limited consumer IoT research. First, collecting network traffic from all possible user interactions and device behaviors is usually infeasible. 
Most consumer IoT devices do not have emulator support, so user interactions must be tested on a physical device.
Manually testing repeated sequences of specific user interactions (e.g., button presses) on a device is quite tedious, limiting data collection to a small set of interactions and a correspondingly small volume of network traffic. 
In contrast, our robotic arm approach does not require any manual interaction or intervention during data collection. This can save hours of researcher time and eliminate the possible need to restart experiments if a human researcher forgets a button press or presses the wrong button. 

Second, manual researcher interactions with consumer IoT devices are unlikely representative of real user behavior over varying timescales. This means that network traffic generated via manual interactions may not be representative of traffic generated by devices in consumer households. In contrast, our approach uses permutation-based interaction sequences that can test all possible interactions with a device or be programmed to mimic real user behavior over long timescales.

\subsection{Crowdsourcing}
A more recent approach to acquiring consumer IoT network traffic involves crowdsourcing data collection to real users who have adopted IoT devices~\cite{huang2020iot, mazhar2020characterizing}. This typically involves the creation of custom hardware or software that allows users to instrument their own devices or home networks.

Crowdsourcing IoT traffic has several benefits. Crowdsourced data is more externally valid than manually collected laboratory data because it comes from real users interacting with IoT devices in the wild. 
With substantial recruitment efforts, crowdsourcing can also produce more data from a wider variety of devices and interaction patterns than laboratory collection. 

However, crowdsourced data has other drawbacks. It may require extensive development effort to create a platform for data collection that participants feel comfortable incorporating into their homes. Recruiting participants is also challenging. Crowdsourcing platforms such as Amazon Mechanical Turk are not well suited to IoT data collection, which does not fit well into the Human Intelligence Task (HIT) framework.
Participant compensation for large-scale crowdsourcing campaigns can also be quite expensive, especially if participants are required to install hardware in their homes. 
Some studies avoid this expense by asking interested individuals to volunteer data without monetary compensation, sometimes by providing details about local IoT device behavior that may be of interest to privacy or security-conscious users~\cite{huang2020iot}. However, this can limit study participation to technically savvy participants or those with existing privacy or security concerns, potentially introducing results bias. Our robotic arm approach requires only an initial cost for the robot and devices but, like crowdsourcing, can produce large amounts of data.

Crowdsourced IoT data also suffers from unreliable labels. Participants may not accurately report what devices they own or what interactions they perform with the devices. \textit{Post hoc} identification and verification of device types and user interactions from network data is a research problem in its own right~\cite{huang2020iot}.
In contrast, data collected with our automated approach can be directly labeled with the correct ground truth by the researchers running the robot platform.


\subsection{Automated Data Collection with Robotic Arm Interactions}
The automated data collection approach described in this paper combines the scalability and reduced tediousness of crowdsourcing with the control and verifiability of manual data collection (Table~\ref{tab:compare}). 
Our approach focuses on automating device interactions via \textit{physical} user interface elements rather than through networked applications (e.g., smartphones, IoT hubs, or debugging tools). While tools like Android Debug Bridge (ADB) can be used to automate interactions with some IoT devices~\cite{ren2016recon,mohajeri2019watching}, many devices do not have similar programmatic testing tools available. Such devices can only be tested with physical interactions (e.g., button pressing). By performing these physical interactions with a robotic arm, we are able to examine network traffic from a breadth of device behaviors with minimal manual effort. 
We are also able to rigorously test permutations of device interactions to collect data for all possible (or all reasonable) interactions with a device. These permutations can include different interface elements (buttons, etc.) as well as different timings between interactions.