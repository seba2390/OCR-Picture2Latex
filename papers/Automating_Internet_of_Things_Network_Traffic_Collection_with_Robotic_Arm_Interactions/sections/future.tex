\section{Future Work}
\label{sec:future}

Apart from the limitations stated in Section~\ref{sec:limit}, our automated approach to IoT device interactions is amenable to many additions and improvements that could be the topic of future work. We hope that others will adopt and adapt this approach to automate and scale IoT research. 

\subsection{Randomness in Interaction Sequences}
One straightforward extension of our approach is to create permutation-based interaction sequences with increased randomness to better replicate the variety of real user interactions. Although our current implementation explores all possible unique interaction permutations, additional randomness could be added by randomly repeating button presses within interaction sequences or repeating entire interaction sequences for a random number of iterations. 


A follow-up study could also test whether increases in interaction randomness actually increase the variety of collected network traffic. If network traffic is predominantly linked to the most recent interaction, introducing random delays in interaction frequency may not actually produce traffic of greater interest for follow-up analysis. 


\subsection{Additional IoT Devices}
While we chose two popular consumer IoT devices to test with robotic arm interactions, we expect that this method is amenable to a wide variety of IoT products in consumer, medical, and other contexts. 
We are aware of several universities and consumer advocacy groups with access to many IoT devices and we recommend their use of our technique for automating network security and privacy analyses at scale. 
We would especially like to see this approach used to evaluate medical devices, as they often have many physical buttons and may be under-audited from a network security perspective~\cite{burns2016brief}. 

\subsection{Multi-Device Interactions}
Although this study explores automated interactions with individual devices, future work could apply our system to simultaneous interactions with multiple devices using multiple robotic arms. 
Some IoT devices are designed to communicate with other devices on the local network, and the traffic from these communications would not be visible from robotic arm interactions with a single device. With the help of a centralized coordinator (i.e., a parallel processing controller), a setup with multiple devices and multiple robotic arms could test interleaved interaction sequences across devices to record traffic from device-device communications. 
As discussed in~\cite{apthorpe2019keeping}, patterns of network traffic from multiple IoT devices within a household may allow privacy-violating inferences not achievable with traffic from a single device alone. 
Applying our approach to multiple devices could test the possibility of such inferences and potentially reveal other privacy or security vulnerabilities related to local device communications as well. 
As more manufacturers promote ``ecosystems'' of IoT devices, the ability to automate network research of multi-device interactions becomes increasingly important. 

\subsection{Computer Vision UI Identification}
The only manual process in our approach is to configure the robotic arm with the positions of each of the user interface elements on the device prior to automated interactions (Section~\ref{sec:inverse-kinematics}). This initialization step could also be automated in future work by adding a camera to the setup and using computer vision to identify the type and location of user interface elements on the device. This would provide an extremely low barrier to entry for network data collection from any IoT device at the expense of the additional camera hardware and increased potential for misidentification of UI elements. 

\subsection{Cyber-Physical Fuzz Testing}
Our approach could be used to fuzz test the user interfaces of IoT and other devices with essential physical interfaces, especially by entities such as security researchers or consumer protection groups without access to source code or device emulators. 
Testing all possible physical interactions with a device to see whether any produce buggy or malicious behavior could reveal previously unknown vulnerabilities. This idea could be combined with any of the other future directions in this section to expand the scope of automated security analysis for cyber-physical devices. 
