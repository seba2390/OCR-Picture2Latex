% flatex input: [pageRank-MAIN.tex]
% HPEC2018-pageRank-MAIN.tex x
%
% initial draft : April 25, 2018 
% 
% revision : 
% 
% manuscript submission: May???, 2018
% review comments: see directory ???? 
% final paper submission, July ??, 2018   
% 
%

\def\hpec{true}

\ifdefined\hpec         % ----- IEEE HPEC conference proceedings

  \documentclass[conference]{IEEEtran}

\else                   % ----- draft (default)

  \documentclass[12pt,draft]{article}

  \def\usenumberlines{true}

\fi

\makeatletter
\def\endthebibliography{%
  \def\@noitemerr{\@latex@warning{Empty `thebibliography' environment}}%
  \endlist
}
\makeatother

\def\BibTeX{{\rm B\kern-.05em{\sc i\kern-.025em b}\kern-.08em
    T\kern-.1667em\lower.7ex\hbox{E}\kern-.125emX}}

%%%%%%%%%%%%%%%%%%%%%%%%%%%%%%%%%%%%%%%%%%%%%%%%%%
%%% PREAMBLE

% packages, customisations, macros, etc.

% flatex input: [pageRank-packages.tex]
% hpec2013-gpixel-packages.tex


% ------------------------------ packages

% page/space formatting
\ifdefined\hpec         % ----- IEEE HPEC
  \usepackage{balance}
\else                   % ----- draft
  \usepackage[margin=1in]{geometry}
  \usepackage{setspace}
\fi

% math
\usepackage{amsmath}
\usepackage{amsfonts}
\usepackage{amssymb}
\usepackage{amsthm}
\usepackage{mathrsfs}
\usepackage{array}
% \usepackage{bigdelim}
% \usepackage{breqn}

% floats
% \usepackage[section]{placeins}
\usepackage{float}
% \usepackage{listings}
% \usepackage[section]{algorithm}
% \usepackage{algpseudocode}
\usepackage[font=small,labelfont=bf,hypcap=true]{caption}
%\usepackage[font=small,labelfont=bf,hypcap=true]{subcaption}
% \usepackage{sidecap}

% graphics
\usepackage{graphicx,subfigure}
\usepackage[hyperref,x11names,cmyk,pdftex]{xcolor}
\usepackage{tikz}
\usepackage{float}
\usepackage{subfloat}
\usepackage{tablefootnote}
\usepackage{adjustbox}
% \usepackage{pgfplots}

% tables
\usepackage{multirow}
\usepackage{booktabs}

% lists
\usepackage{enumerate}
\usepackage{paralist}

% header and footer
% \usepackage{fancyhdr}

% footnotes
\usepackage[bottom]{footmisc}

% ordinal numbers
\usepackage[super]{nth}

% fonts
\usepackage{helvet}
\usepackage{textcomp}
\usepackage[utf8]{inputenc}
\usepackage[resetfonts]{cmap}

% editing facilities
\usepackage[layout={inline},author=]{fixme}


% algorithm
\usepackage{algorithm,algorithmic}

% code
% \usepackage{mcode}

% references
\usepackage{csquotes}
% \usepackage[hyphens]{url}
\ifdefined\hpec
	\usepackage[pdftex,bookmarks=false]{hyperref}
	% \usepackage[style=numeric,natbib,backend=bibtex8]{biblatex}
	% \usepackage[numbers,square]{natbib}
	\usepackage{cite}
\else                   % ----- draft
  \usepackage[pdftex,breaklinks=true,bookmarks=true]{hyperref}

  \ifdefined\usenumberlines

    % add linenumbers
    \usepackage{lineno}
    \linenumbers\relax

    \newcommand*\patchAmsMathEnvironmentForLineno[1]{%
      \expandafter\let\csname old#1\expandafter\endcsname\csname #1\endcsname
      \expandafter\let\csname oldend#1\expandafter\endcsname\csname end#1\endcsname
      \renewenvironment{#1}%
      {\linenomath\csname old#1\endcsname}%
      {\csname oldend#1\endcsname\endlinenomath}}% 
    \newcommand*\patchBothAmsMathEnvironmentsForLineno[1]{%
      \patchAmsMathEnvironmentForLineno{#1}%
      \patchAmsMathEnvironmentForLineno{#1*}}%
    \AtBeginDocument{%
      \patchBothAmsMathEnvironmentsForLineno{equation}%
      \patchBothAmsMathEnvironmentsForLineno{align}%
      \patchBothAmsMathEnvironmentsForLineno{flalign}%
      \patchBothAmsMathEnvironmentsForLineno{alignat}%
      \patchBothAmsMathEnvironmentsForLineno{gather}%
      \patchBothAmsMathEnvironmentsForLineno{multline}%
    }

  \fi

\fi




%%% Local Variables: 
%%% mode: latex
%%% TeX-master: "HPEC2013-gpixel-main"
%%% End: 

% flatex input end: [pageRank-packages.tex]

% packages, customisations, macros, etc.
% flatex input: [title-authors.tex]
% title-authors.tex 

\title{\LARGE Damping Effect on PageRank Distribution}

\ifdefined\hpec
\author{%
  \IEEEauthorblockN{% 
    Tiancheng Liu,
    Yuchen Qian,
    Xi Chen, and
    Xiaobai Sun
    \\ 
    \IEEEauthorblockA{%
    \begin{tabular}{c}
      Department of Computer Science, Duke University, Durham, NC 27708, USA
    \end{tabular}%
  }%
}}

\else
\fi 

% Author's comments
\newcommand{\xiaobai}[1]{\textcolor{blue}{Xiaobai:
    #1}\PackageWarning{Comment:}{Xiaobai: #1!}}
\newcommand{\xichen}[1]{\textcolor{red}{XiChen:
    #1}\PackageWarning{Comment:}{Xi Chen: #1!}}
\newcommand{\tiancheng}[1]{\textcolor{orange}{Tiancheng:
    #1}\PackageWarning{Comment:}{Tiancheng: #1!}}
\newcommand{\yuchen}[1]{\textcolor{green}{Yuchen:
    #1}\PackageWarning{Comment:}{Yuchen: #1!}}

% flatex input end: [title-authors.tex]
 
% packages, customisations, macros, etc.

% flatex input: [pageRank-customization.tex]
% authors (separate from hyperref metadata for easy "hiding")
\newcommand{\pdfauthors}{%
  T. Liu, Y. Qian, X. Chen, X. Sun}


% PDF metadata
\hypersetup{%
  pdffitwindow=false,%
  pdfstartview={FitH},%
  colorlinks,%
  pdfauthor={\pdfauthors},%
  pdftitle={Damping Effect on PageRank Distribution},%
  citecolor=PaleGreen4,%
  filecolor=DarkOrchid4,%
  linkcolor=OrangeRed4,%
  urlcolor=black%
}

% always show figures (even in draft mode)
\setkeys{Gin}{draft=false}

% fix Linux Adobe Acrobat Reader's issue with transparent image elements
\pdfpageattr{/Group <</S /Transparency /I true /CS /DeviceRGB>>}

% remove extra spacing around \left and \right delimiters
\let\leftorig\left
\let\rightorig\right
\renewcommand{\left}{\mathopen{}\mathclose\bgroup\leftorig}
\renewcommand{\right}{\aftergroup\egroup\rightorig}

% increase vertical spacing between table rows
\renewcommand{\arraystretch}{1.2}

% Enumerate Tables in Arabic
\renewcommand{\thetable}{\arabic{table}}

% customize the use of editing notes
\fxsetup{inline}
\fxusetheme{color}
\newcommand{\note}[1]{\fxerror{[#1]}}
\newcommand{\remove}[1]{\fxwarning{\emph{rm}:[#1]}}

% show hyperlinks even when in draft mode
\hypersetup{final}

% allow hyphenated url strings (without using the 'url' package
\makeatletter
\DeclareRobustCommand\ttfamily{%
  \not@math@alphabet\ttfamily\mathtt
  \fontfamily\ttdefault\selectfont\hyphenchar\font=-1\relax}
\makeatother
\DeclareTextFontCommand{\tturl}{\ttfamily\hyphenchar\font=`/}


%%% Local Variables: 
%%% mode: latex
%%% TeX-master: "../wknn-main"
%%% End: 

% flatex input end: [pageRank-customization.tex]

% packages, customisations, macros, etc.

%%%%%%%%%%%%%%%%%%%%%%%%%%%%%%%%%%%%%%%%%%%%%%%%%%
%%% DOCUMENT

\begin{document}


% --------- make TITLE
% double line spacing for draft mode

\ifdefined\hpec\else
  \doublespacing
\fi


%%% ====================================================
%%% FRONT MATTER


% -------------------- IEEE HPEC
%
\ifdefined\hpec
  \maketitle

% -------------------- draft
%
\else

  % ----- title

  \pdfbookmark[1]{Title}{sec:title}
  \maketitle

  % ----- table of contents

  \phantomsection
  \pdfbookmark[1]{Contents}{sec:contents}
  \tableofcontents
  \clearpage

\fi

% ============  ABSTRACT

% flatex input: [abstract.tex]
% abstract.tex

\begin{abstract}
  This work extends the personalized PageRank model invented by Brin and
  Page to a family of PageRank models with various damping schemes. The
  goal with increased model variety is to capture or recognize a larger
  number of types of network activities, phenomenons and
  propagation patterns. The response in PageRank distribution
  to variation in damping mechanism is then characterized analytically,
  and further estimated quantitatively on $6$ large real-world link
  graphs. The study leads to new observation and empirical findings. It
  is found that the difference in the pattern of PageRank vector
  responding to parameter variation by each model among the $6$ graphs is
  relatively smaller than the difference among $3$ particular models
  used in the study on each of the graphs. This suggests the utility of
  model variety for differentiating network activities and propagation
  patterns. The quantitative analysis of the damping mechanisms over 
  multiple damping models and parameters is facilitated by a highly efficient algorithm,
  which calculates all PageRank vectors at once via a commonly shared,
  spectrally invariant subspace. The spectral space is
  found to be of low dimension for each of the real-world graphs.
\end{abstract}

% flatex input end: [abstract.tex]

% ============  ABSTRACT


% ============= INTRODUCTION

\section{Introduction}
\label{sec:intro}
% % !TEX root = ../arxiv.tex

Unsupervised domain adaptation (UDA) is a variant of semi-supervised learning \cite{blum1998combining}, where the available unlabelled data comes from a different distribution than the annotated dataset \cite{Ben-DavidBCP06}.
A case in point is to exploit synthetic data, where annotation is more accessible compared to the costly labelling of real-world images \cite{RichterVRK16,RosSMVL16}.
Along with some success in addressing UDA for semantic segmentation \cite{TsaiHSS0C18,VuJBCP19,0001S20,ZouYKW18}, the developed methods are growing increasingly sophisticated and often combine style transfer networks, adversarial training or network ensembles \cite{KimB20a,LiYV19,TsaiSSC19,Yang_2020_ECCV}.
This increase in model complexity impedes reproducibility, potentially slowing further progress.

In this work, we propose a UDA framework reaching state-of-the-art segmentation accuracy (measured by the Intersection-over-Union, IoU) without incurring substantial training efforts.
Toward this goal, we adopt a simple semi-supervised approach, \emph{self-training} \cite{ChenWB11,lee2013pseudo,ZouYKW18}, used in recent works only in conjunction with adversarial training or network ensembles \cite{ChoiKK19,KimB20a,Mei_2020_ECCV,Wang_2020_ECCV,0001S20,Zheng_2020_IJCV,ZhengY20}.
By contrast, we use self-training \emph{standalone}.
Compared to previous self-training methods \cite{ChenLCCCZAS20,Li_2020_ECCV,subhani2020learning,ZouYKW18,ZouYLKW19}, our approach also sidesteps the inconvenience of multiple training rounds, as they often require expert intervention between consecutive rounds.
We train our model using co-evolving pseudo labels end-to-end without such need.

\begin{figure}[t]%
    \centering
    \def\svgwidth{\linewidth}
    \input{figures/preview/bars.pdf_tex}
    \caption{\textbf{Results preview.} Unlike much recent work that combines multiple training paradigms, such as adversarial training and style transfer, our approach retains the modest single-round training complexity of self-training, yet improves the state of the art for adapting semantic segmentation by a significant margin.}
    \label{fig:preview}
\end{figure}

Our method leverages the ubiquitous \emph{data augmentation} techniques from fully supervised learning \cite{deeplabv3plus2018,ZhaoSQWJ17}: photometric jitter, flipping and multi-scale cropping.
We enforce \emph{consistency} of the semantic maps produced by the model across these image perturbations.
The following assumption formalises the key premise:

\myparagraph{Assumption 1.}
Let $f: \mathcal{I} \rightarrow \mathcal{M}$ represent a pixelwise mapping from images $\mathcal{I}$ to semantic output $\mathcal{M}$.
Denote $\rho_{\bm{\epsilon}}: \mathcal{I} \rightarrow \mathcal{I}$ a photometric image transform and, similarly, $\tau_{\bm{\epsilon}'}: \mathcal{I} \rightarrow \mathcal{I}$ a spatial similarity transformation, where $\bm{\epsilon},\bm{\epsilon}'\sim p(\cdot)$ are control variables following some pre-defined density (\eg, $p \equiv \mathcal{N}(0, 1)$).
Then, for any image $I \in \mathcal{I}$, $f$ is \emph{invariant} under $\rho_{\bm{\epsilon}}$ and \emph{equivariant} under $\tau_{\bm{\epsilon}'}$, \ie~$f(\rho_{\bm{\epsilon}}(I)) = f(I)$ and $f(\tau_{\bm{\epsilon}'}(I)) = \tau_{\bm{\epsilon}'}(f(I))$.

\smallskip
\noindent Next, we introduce a training framework using a \emph{momentum network} -- a slowly advancing copy of the original model.
The momentum network provides stable, yet recent targets for model updates, as opposed to the fixed supervision in model distillation \cite{Chen0G18,Zheng_2020_IJCV,ZhengY20}.
We also re-visit the problem of long-tail recognition in the context of generating pseudo labels for self-supervision.
In particular, we maintain an \emph{exponentially moving class prior} used to discount the confidence thresholds for those classes with few samples and increase their relative contribution to the training loss.
Our framework is simple to train, adds moderate computational overhead compared to a fully supervised setup, yet sets a new state of the art on established benchmarks (\cf \cref{fig:preview}).


% flatex input: [introduction.tex]
% introduction.tex 

Personalized PageRank, invented by Brin and Page~\cite{page1999pagerank,
  ilprints361}, revolutionized the way we model any particular type of
activities on a large information network. It is also intended to be
used as a mechanism to counteract malicious manipulation of the
network~\cite{page1999pagerank, ilprints361, sheldon2010manipulation}. 
PageRank has underlain Google's
search architecture, algorithms, adaptation strategies and ranked page
listing upon query.  It has influenced the development of other
search engines and recommendation systems, such as topic-sensitive
PageRank\cite{haveliwala2003topic}. Its impact reaches far beyond
digital and social networks. For example, GeneRank is used for
generating prioritized gene
lists\cite{morrison2005generank,wu2010krylov}. The seminal
paper~\cite{page1999pagerank} itself is directly cited more than ten
thousands times as of today.  As surveyed in
\cite{langville2004deeper,berkhin2005survey}, a lot of efforts were made
to accelerate the calculation of personalized PageRank vectors, in part
or in whole \cite{ilprints596,ilprints579}. Certain investigation were
carried out to assess the variation in PageRank vector in response to
varying damping
parameter~\cite{boldi2005pagerank,bressan2010choose}. Most efforts on
PageRank study, however, are ad hoc to the Brin-Page model. Chung made
a departure by introducing a diffusion-based PageRank model and applied
it to graph cuts~\cite{chung2007heat,chung2009local}.

In this paper we follow Brin and Page in the modeling aspect that
warrants more attention as the variety of networks and activities on the
networks increases incessantly.  We extend the model scope to capture
more network activities in a probabilistic sense. We study the damping
effect on PageRank distribution.  We consider the holistic distribution
because it serves as the statistical reference for inferring conditional
page ranking upon query. Our study has three intellectual merits with 
practical impact.
%
\begin{inparaenum}[(1)]
\item A family of damping models, which includes and connects the
  Brin-Page model and Chung's model. The family admits more
  probabilistic descriptions of network activities. 
%
\item A unified analysis of damping effect on personalized PageRank
  distribution, with parameter variation in each model and comparison
  across models. The analysis provides a new insight into the solution
  space and solution methods.
%
\item A highly efficient method for calculating the solutions to all
  models under consideration at once, particular to a network and a
  personalized vector.  Our quantitative analysis of $6$ real-world
  network graphs leads to new findings about the models and networks under
  study, which we present and discuss in Sections~\ref{sec:theory-analysis}
  and \ref{sec:numerical-experiments}.
%
%
\end{inparaenum} 
%
Our modeling and analysis methods can be potentially used for
recognizing and estimating activity or propagation patterns on a
network, provided with monitored data.
% 
%

% flatex input end: [introduction.tex]
 
% % !TEX root = ../arxiv.tex

Unsupervised domain adaptation (UDA) is a variant of semi-supervised learning \cite{blum1998combining}, where the available unlabelled data comes from a different distribution than the annotated dataset \cite{Ben-DavidBCP06}.
A case in point is to exploit synthetic data, where annotation is more accessible compared to the costly labelling of real-world images \cite{RichterVRK16,RosSMVL16}.
Along with some success in addressing UDA for semantic segmentation \cite{TsaiHSS0C18,VuJBCP19,0001S20,ZouYKW18}, the developed methods are growing increasingly sophisticated and often combine style transfer networks, adversarial training or network ensembles \cite{KimB20a,LiYV19,TsaiSSC19,Yang_2020_ECCV}.
This increase in model complexity impedes reproducibility, potentially slowing further progress.

In this work, we propose a UDA framework reaching state-of-the-art segmentation accuracy (measured by the Intersection-over-Union, IoU) without incurring substantial training efforts.
Toward this goal, we adopt a simple semi-supervised approach, \emph{self-training} \cite{ChenWB11,lee2013pseudo,ZouYKW18}, used in recent works only in conjunction with adversarial training or network ensembles \cite{ChoiKK19,KimB20a,Mei_2020_ECCV,Wang_2020_ECCV,0001S20,Zheng_2020_IJCV,ZhengY20}.
By contrast, we use self-training \emph{standalone}.
Compared to previous self-training methods \cite{ChenLCCCZAS20,Li_2020_ECCV,subhani2020learning,ZouYKW18,ZouYLKW19}, our approach also sidesteps the inconvenience of multiple training rounds, as they often require expert intervention between consecutive rounds.
We train our model using co-evolving pseudo labels end-to-end without such need.

\begin{figure}[t]%
    \centering
    \def\svgwidth{\linewidth}
    \input{figures/preview/bars.pdf_tex}
    \caption{\textbf{Results preview.} Unlike much recent work that combines multiple training paradigms, such as adversarial training and style transfer, our approach retains the modest single-round training complexity of self-training, yet improves the state of the art for adapting semantic segmentation by a significant margin.}
    \label{fig:preview}
\end{figure}

Our method leverages the ubiquitous \emph{data augmentation} techniques from fully supervised learning \cite{deeplabv3plus2018,ZhaoSQWJ17}: photometric jitter, flipping and multi-scale cropping.
We enforce \emph{consistency} of the semantic maps produced by the model across these image perturbations.
The following assumption formalises the key premise:

\myparagraph{Assumption 1.}
Let $f: \mathcal{I} \rightarrow \mathcal{M}$ represent a pixelwise mapping from images $\mathcal{I}$ to semantic output $\mathcal{M}$.
Denote $\rho_{\bm{\epsilon}}: \mathcal{I} \rightarrow \mathcal{I}$ a photometric image transform and, similarly, $\tau_{\bm{\epsilon}'}: \mathcal{I} \rightarrow \mathcal{I}$ a spatial similarity transformation, where $\bm{\epsilon},\bm{\epsilon}'\sim p(\cdot)$ are control variables following some pre-defined density (\eg, $p \equiv \mathcal{N}(0, 1)$).
Then, for any image $I \in \mathcal{I}$, $f$ is \emph{invariant} under $\rho_{\bm{\epsilon}}$ and \emph{equivariant} under $\tau_{\bm{\epsilon}'}$, \ie~$f(\rho_{\bm{\epsilon}}(I)) = f(I)$ and $f(\tau_{\bm{\epsilon}'}(I)) = \tau_{\bm{\epsilon}'}(f(I))$.

\smallskip
\noindent Next, we introduce a training framework using a \emph{momentum network} -- a slowly advancing copy of the original model.
The momentum network provides stable, yet recent targets for model updates, as opposed to the fixed supervision in model distillation \cite{Chen0G18,Zheng_2020_IJCV,ZhengY20}.
We also re-visit the problem of long-tail recognition in the context of generating pseudo labels for self-supervision.
In particular, we maintain an \emph{exponentially moving class prior} used to discount the confidence thresholds for those classes with few samples and increase their relative contribution to the training loss.
Our framework is simple to train, adds moderate computational overhead compared to a fully supervised setup, yet sets a new state of the art on established benchmarks (\cf \cref{fig:preview}).



% ============= MODELS

\section{PageRank models}
\label{sec:pageRank-models}
%
% flatex input: [pageRank-models.tex]

% pageRank-models.tex

We first review briefly two precursor models and then introduce a family
of PageRank models.


% ========= BP mpdel
\subsection{Brin-Page model} 
\label{subsec:Brin-Page-model}
%
% flatex input: [sections/bp-model.tex]
% bp-models.tex

%% --- dscribe P matrix first % 
Brin and Page describe a network of webpages as a link graph, which is
represented by a stochastic matrix $P$~\cite{page1999pagerank}. We adopt
the convention that $P$ is stochastic columnwise. Every webpage is a
node with (outgoing) links, i.e., edges, to some other webpages and with
incoming edges or backlinks as citations to the page. If page $j$ has
$n_j >0 $ outgoing links, then in column $j$ of $P$, $P_{i,j} = 1/n_j$
if page $j$ has a link to page $i$; $P_{i,j}=0$, otherwise. In row $i$
of $P$, every nonzero element $P_{i,j}$ corresponds to a backlink from
$j$ to $i$.
%
%% ---- describe M matrix next % 
In the Brin-Page model, the web user behavior is described as a random
walk on a personalized Markov chain (i.e., a discrete-time Markov chain)
associated with the following probability transition matrix
% 
\begin{equation}
\label{eqn:bp-Bernoulli}
M_{\alpha}(v) = \alpha P + (1 - \alpha)v e^{\rm T}, 
\quad \alpha \in (0,1), 
\quad e^{\rm T}v = 1, 
\end{equation}%
%
where $v\geq 0$ is a personalized or customized distribution/vector, $e$
denotes the vector with all elements equal to $1$, and the \emph{damping
  factor} $\alpha$ describes a Bernoulli decision process.  At each
step, with probability $\alpha$, the web user follows an outlink; or
with probability $(1-\alpha)$, the user jumps to any page by the personalized
distribution $v$.  The personalized transportation term is
innovative. It customizes the Markov chain with respect to a particular
type of relevance.
%
In this paper we assume that the personalized vector $v$ is given and
fixed, and focus on investigating the damping effect. In particular, 
with Brin-Page model, we focus on the role of $\alpha$.
The notation for the Markov chain may thus be simplified to
$M_{\alpha}$, or $M$ when $\alpha$ is clear from the context.

Page ranking upon a search query depends on the stationary PageRank
distribution, denoted by $x=x(\alpha)$, of the Markov chain:
% 
\begin{equation}
\label{eqn:ppr-markov-equi}
  M_{\alpha}  \,  x = x, 
  \quad e^{\rm T}x = 1. 
\end{equation}%
%
Arasu et al\cite{arasu2002pagerank} recast the eigenvector equation
(\ref{eqn:ppr-markov-equi}) to a linear system to solve for $x$,
% 
\begin{equation}
\label{eqn:ppr-linear}
(I - \alpha P)x = (1 - \alpha)v, 
\end{equation}
%
where $I$ is the identity matrix.
Because $\alpha \|P\|_{1} < 1$, the solution can be expressed via the 
Neumann series for the inverse of $(I-\alpha P)$, 
% 
\begin{equation}
\label{eqn:bp-solution}
x(\alpha)  = (1-\alpha) \sum_{k = 0} ^ {\infty} \alpha^k P^k v.
\end{equation}
%
The weights $(1-\alpha)\alpha^k$ decrease by the factor $\alpha$ from one step to the
next. In \cite{page1999pagerank}, Brin and Page set $\alpha$ to $0.85$.

In (\ref{eqn:bp-solution}), the solution to Brin-Page model with
$\alpha \in (0,1)$ is not the stationary distribution of network $P$. It
is analyzed in terms of steps on $P$. Term $k$ represents the probabilistic
accumulation of the Bernoulli decision process at each step by
(\ref{eqn:bp-Bernoulli}) to step $k$, $k\geq 0$.  Every step has its
print in distribution $x(\alpha)$.




% flatex input end: [sections/bp-model.tex]

%


% ========== CHung's model
\subsection{Chung's model} 
\label{subsec:Chung-model}
% flatex input: [sections/chung-model.tex]
% chng-model.tex


Chung introduced a PageRank model~\cite{chung2007heat} in the form of a
heat or diffusion equation, with $v$ as the initial distribution,
% 
\begin{equation}
\label{eqn:heat-model}
  \frac{\partial x}{\partial \beta} = - (I - P)x, \quad x(0) = v,
\end{equation}
%
where $( I - P ) $ is the Laplacian of the link graph, and we use
$\beta$ to denote the time variable.
%
From the viewpoint of probabilistic theory, model (\ref{eqn:heat-model})
is underlined by the Kolmogorov's backward equation system for a continuous-time
Markov chain with $-(I-P)$ as the transition rate matrix and with the
identity matrix $I$ as the initial transition matrix.
% 
The solution to (\ref{eqn:heat-model}) is 
%
\begin{equation}
\label{eqn:heat-solution}
x(\beta) 
= e^{-\beta(I-P)}v 
= e^{-\beta} \sum_{k = 0} ^ {\infty} \frac{\beta^k}{k!} P^k v. 
\end{equation}
%

% flatex input end: [sections/chung-model.tex]

% ========== CHung's model


% ========== Model family
\subsection{A model family}
\label{subsec:model-family}
%
% flatex input: [sections/model-family.tex]
% model-family.tex 

We introduce a family of PageRank models.  
Each member model is characterized by a scalar \emph{damping
  variable} $\rho$ and a {\em discrete} probability mass function (pmf)
$ w(\rho) = \{ w_k=w_k(\rho), \ k \in \mathbb{N}_w \} $. The support
$ \mathbb{N}_{w} \subset \mathbb{N} $ may be finite or infinite.  There
are a few equivalent expressions to describe our models.
% 
We start by defining the model with a kernel function $f(\lambda, \rho)$, 
%
\begin{equation}
\label{eqn:uni-random-walk} 
f(\lambda, \rho) = \sum_{k \in N_w} w_k(\rho)\, \lambda^k, 
\quad | \lambda | \leq 1.   
\end{equation}%
%
The solution specific to network graph $P$ and personalized vector $v$ is, 
%
\begin{equation}
\label{eqn:uni-random-walk} 
x_f(\rho)   = f(P)v = \left(\sum_{k \in N_w} w_k(\rho)\, P^k \right) v . 
\end{equation}%
%
The matrix function $f(P)$ is stochastic. The rank
distribution vector $x_f$ is the superposition of step terms with
probabilistic weights $w_k$. The step term $k$ describes the
probabilistic propagation of $v$ at step $k$.
% 
For a specific case, the damping variable may have a designated label,
with a specific range, and the pmf may have a specific support and
additional parameters.  For convenience, we assume $\mathbb{N}$ as the
support. Over the infinite support, the damping weights must decay after
certain number of steps and vanish as $k$ goes to infinity. In theory, 
every discrete pmf can be used as a model kernel in (\ref{eqn:uni-random-walk}).
In practice, each describes a particular type of activity or propagation.
%

The family includes Brin-Page model and Chung's model.  For the former,
the damping variable is denoted by $\alpha$, the damping weights
$(1-\alpha)\alpha^k$, $k\geq 0$, follow the geometric distribution with
the expected value $\alpha(1-\alpha)^{-1}$. The kernel function
is $ (1-\alpha)(1-\alpha\lambda)^{-1}$.  For Chung's model, we
denote the damping variable by $\beta$, $\beta > 0$. The damping weights
$e^{-\beta}\beta^k/k!$, $k\geq 0$, follow the Poisson distribution with
the expected value $\beta$. The model's kernel function is
$e^{-\beta(1-\lambda)}$.

We describe a few other models, among many, in the family. In fact, the
precursor models are two special cases of the model associated with the
Conway-Maxwell-Poisson (CMP) distribution, which has an additional
parameter $\nu$ to the pmf,
% 
\begin{equation} 
\notag
\label{eqn:CMP-weights}
  w_k( \rho, \nu) 
= \frac{\rho^{k}}{(k!)^{\nu}\, Z}, 
 \quad \nu > 0, 
\end{equation}
%
where $Z$ is the normalization scalar, and $\nu$ is the decay
rate parameter. The case with $\nu=0$ is the geometric distribution; the
case with $\nu=1$ is the Poisson distribution. If the value of $\rho$ is
fixed, the weights decay faster with a larger value of $\nu$.
%
The negative binomial distribution, or the Pascal distribution, also
includes the geometric distribution as a special case. It includes
other cases that 
render damping weights with slower decay rates.
%

In the rest of the paper, for the purpose of including and illustrating
new models, we use the model associated with the logarithmic
distribution, for $\gamma \in (0,1)$, 
%
\begin{equation}
\label{eqn:log-gamma} 
 f(\lambda,\gamma) =
    \frac{-1}{\ln(1 - \gamma)} 
    \sum_{k = 1}^{\infty} \frac{(\gamma\lambda)^{k}}{k}
   = \frac{\ln(1 - \gamma\lambda)}{\ln(1 - \gamma)}. 
\end{equation}%
%
The weights decrease slightly faster than the geometrically distributed
ones, but not in the CMP distribution class. 

We now present the system of linear equations with $x(\rho)$ in 
(\ref{eqn:uni-random-walk}) as the solution, 
% 
\begin{equation}
\label{eq:model-equation} 
A(P)x = v, 
\quad A(P) = f^{-1}(P). 
\end{equation}%
%
The matrix $A$ is an $M$ matrix. In particular, 
$A = (1-\alpha)^{-1}(I - \alpha P)$ for Brin-Page model,
$A = e^{\beta (I - P)}$ for Chung's model and
$A= \ln(1-\gamma) \, \ln^{-1}(I - \gamma P) $ for the log-$\gamma$
model (\ref{eqn:log-gamma}).
% 
The algebraic model expression (\ref{eq:model-equation}) will be used
next for the model expression in a differential equation.




% flatex input end: [sections/model-family.tex]

%



% flatex input end: [pageRank-models.tex]
 
%

% ============= ANALYSIS

\section{Response to variation in damping}
\label{sec:theory-analysis}
%
% flatex input: [response-variation-damping.tex]
% variation-analysis.tex


We provide a unified analysis of the response in PageRank distribution
to the variation in the damping parameter value as well as to the
change, or connection, from one model to another.

% ================ intra-model damping 
\subsection{Intra-model damping variation}
\label{subsec:pagerank-sensitive}
% flatex input: [sections/intra-model-variation.tex]
% intra-model-variation.tex

By (\ref{eqn:uni-random-walk}), we obtain the trajectory of the PageRank
vector $x(\rho)$ with the change in the damping variable $\rho$,
% 
\begin{equation}
\label{eqn:trajectory-equation}
 \dot{x}(\rho) = \frac{d x(\rho)}{d \rho} 
 = \frac{\partial }{\partial \rho}f(P) v = Q(P) \, x(\rho), 
\end{equation}%
%
where $Q(P) = \frac{\partial }{\partial \rho}f(P) f^{-1}(P)$ by
(\ref{eq:model-equation}), which we may refer to as the $\rho$-transition matrix.
%
Equation (\ref{eqn:trajectory-equation}) generalizes Chung's diffusion
model (\ref{eqn:heat-model}), in which the $\beta$-transition matrix
$Q=-(I-P)$ is independent of $\beta$.
%
For the Brin-Page model with damping variable $\alpha$,
%
\begin{equation}
\label{eqn:bp-trajectory-equationpde}
  Q(\alpha) = 
  \left[ P(I - \alpha P)^{-1} - (1-\alpha)^{-1} I\right]. 
\end{equation} 
% 
For the log-$\gamma$ model (\ref{eqn:log-gamma}),  
%
\begin{equation}
\label{eqn:bp-trajectory-equationpde}
   Q(\gamma) =  
   \frac{(1-\gamma)^{-1}}{\ln(1 - \gamma)} I 
         - P(I-\gamma P)^{-1} (\ln(I - \gamma P))^{-1}. 
\end{equation} 
%
For each model, $e^{\rm T}Q=0$. 

%\textcolor{red}{\bf \em need to check matrix $A$ and matrix $Q$ in each
%  case.}


% flatex input end: [sections/intra-model-variation.tex]

% ================ intra-model damping 
% flatex input: [sections/KL-measures.tex]
% KL-measures.tex
% to pageRank-models.tex

In addition to the element-wise response in the rank vector, we would
also like to have an aggregated measure of the response to variation in
$\rho$. Let $x(\rho_{o})$ be a reference PageRank vector.  We
may use Kullback-Leibler divergence \cite{kullback1951information} to measure the discrepancy of
$x(\rho)$ from $x(\rho_{o})$,
%
\begin{equation}
\label{eqn:kl}
  KL(x(\rho), x(\rho_{o}) ) = 
   \sum_{i} x_i(\rho) \log {\frac{x_i(\rho)}{x_i(\rho_{o})}} . 
\end{equation}
%
When $\rho=\rho_{o}$, $KL(x(\rho), x(\rho_{o})) = 0$. 
%
We have the rate of change in KL divergence with the variation in
$\rho$, 
% 
\begin{equation}
\label{eqn:kl-derivative}
  \frac{d}{d \rho} KL(x(\rho), x(\rho_{o})) 
 =  \dot{x}(\rho)^{\rm T} 
    \left( \log {x(\rho)} -\log{x(\rho_{o})}  + e \right)
\end{equation}
%
We will describe in Section~\ref{sec:batch-ranking} efficient
algorithms for calculating the vectors and measures above.



% flatex input end: [sections/KL-measures.tex]

% ================ intra-model damping 


% ================= inter-model damping
\subsection{Inter-model correspondence}
\label{subsec:damping-corres}
% flatex input: [sections/inter-model-correspondence.tex]
% inter-model-correspondence.tex


Each model has its own damping form and parameter. The expected value of
the step weight distribution is, 
%
\begin{equation}
\label{eqn:expect-steps}
\mu(w(\rho)) = \sum_{k \in \mathbb{N}_{w}} k \cdot w_k(\rho). 
\end{equation}
%
We may explain this as the expected value of walking steps.  We
establish the point of correspondence between models by their expected
values. That is, for any two models, we set their expected values equal
to each other. Without loss of generality, we let the expected values
for the Brin-Page model serve as the reference. In particular, we have
the correspondence equalities
%
\begin{equation}
\label{eqn:param-corres}
  \frac{\alpha}{1-\alpha} = \beta, 
  \quad 
  \frac{\alpha}{1-\alpha} 
 = \left(\frac{\gamma}{1 - \gamma}\right)\frac{-1}{\ln(1 - \gamma)}
\end{equation}%
%
for Chung's model and for the log-$\gamma$ model, respectively. We will show
the comparisons in PageRank vectors at such correspondence points in
Section~\ref{sec:numerical-experiments}.

% flatex input end: [sections/inter-model-correspondence.tex]

% ================= inter-model damping

% flatex input end: [response-variation-damping.tex]

%

% ============== ALGORITHM

\section{Efficient algorithms for batch ranking}
\label{sec:batch-ranking}
% 
% flatex input: [batch-ranking-algorithms.tex]
% batch-ranking-algorithms.tex
% 

% ... leading paragraph 

We introduce novel algorithms for efficient quantitative analysis of
damping effect on PageRank distribution.  Provided with a network graph
$P$ and a personalized distribution vector $v$, the algorithms can be
used in one batch of computation across multiple models as well as over
a range of damping parameter value per model.

% ========  reduction to wub-networks
\subsection{Reduction to irreducible subnetworks}
\label{subsec:model-reduction}
%
% flatex input: [sections/model-reduction.tex]
% model-reduction.tex 


% 
Information networks in real world applications are not necessarily
irreducible and aperiodic as assumed by many existing iterative
solutions for guaranteed convergence. To meet such convergence
conditions, some heuristics were used to perturb or twist the network
structure with artificially introduced links~\cite{lee2003fast,
langville2006reordering}.
%
Instead, we decompose the network into strongly connected sub-networks
by applying the Dulmage-Mendelsohn (DM) decomposition algorithm
\cite{dulmage1958coverings} to the Laplacian matrix $I-P$. The DM
algorithm is highly efficient when diagonal elements are non-zero. 
%
It renders the matrix in block upper triangular form. See
Figure~\ref{fig:dm} for the Google link graph released by Google in 2002 \cite{google-net}.
%
Each diagonal block $B_{ii}$ corresponds to a subnetwork. A
square diagonal block corresponds to an irreducible
subnetwork. A non-zero off-diagonal block $B_{ij}$ in the upper part,
$i<j$, represents the links from cluster $j$ to cluster $i$. The top
block is associated with a {\em sink} cluster without outgoing links to
other cluster; the bottom block is associated with a {\em source}
cluster without incoming edges from other clusters. The solution for the
entire network can be obtained by the solutions to the subnetworks and
successive back substitution.
%
% flatex input: [figtex/google/fig-google-dmperm.tex]
% fig-google-dmperm.tex

\begin{figure}[!htb]
  \centering
%  \subfigure[Original network]{
  \subfigure[original link graph]{
    \includegraphics[width=0.22\textwidth]{web-Google-sparsity}
  }
%  \subfigure[Network after DM-perm]{
  \subfigure[link graph after DM permutation]{
    \includegraphics[width=0.22\textwidth]{web-Google-sparsity-dmperm}
  }
  \caption{% \scriptsize 
   \footnotesize The 1:1000 sparsity map of adjacency matrix of Google link graph \cite{google-net} with 875,713
    page nodes in {\bf (a)} the provided ordering and {\bf (b)} the ordering rendered by the DM decomposition, depicted by {\tt imagesc} in {\tt matlab}. Each point shows the number of non-zeros, in log scale, in the corresponding $1000 \times 1000$ block. The subnetwork in the middle of (b) is strongly connected with 434,818 nodes.}
  \label{fig:dm}
\end{figure}

% flatex input end: [figtex/google/fig-google-dmperm.tex]
 
%


% flatex input end: [sections/model-reduction.tex]

%

% ======== cascade of iterations
\subsection{Cascade of iterations} 
\label{subsec:cascade}
% 
% flatex input: [sections/iteration-cascade.tex]
% iteration-castecade.tex 


Several iterative methods exist for computing the PageRank vector by the
Brin-Page model. They include the power method by the eigenvector
equation (\ref{eqn:ppr-markov-equi}), and the Jacobi, Gauss-Seidel, and
SOR methods by equation (\ref{eqn:ppr-linear}). There are various
acceleration techniques used for calculating the PageRank vector, or a
small part of the vector, or even a single pair of nodes between
the personalized vector and the PageRank vector~\cite{kamvar2003extrapolation,
ilprints579, kamvar2004adaptive, DBLP:journals/corr/LofgrenBGS14}.
% 
For the Brin-Page model, the iterative methods converge slower as
$\alpha$ increases and gets closer to $1$. We developed a cascading
initialization scheme. The solution to the model with $\alpha$ is used
as the initial guess to the iteration for the solution to the model with
$\alpha+\delta \alpha$, $\delta \alpha > 0$. Although it has accelerated
the computation with successively increased $\alpha$ values, this
technique is limited to sequential computation and ad hoc to the
Brin-Page model. We introduce next a novel algorithm without these
limitations.


% flatex input end: [sections/iteration-cascade.tex]
 
% 

% ======== shared invariant Krylov space
\subsection{Shared invariant Krylov space} 
\label{subsec:SIK-space}
%
% flatex input: [sections/SIK-space-method.tex]
% SIK-space-method.tex 


Our new algorithm for batch calculation of PageRank vectors with
multiple models and parameter values is based on the very fact that the
solutions to the models in Section~\ref{sec:pageRank-models} all reside
in the same Krylov space,
%
\begin{equation} 
\label{eqn:Krylov-space} 
{\cal K}(P, v) = \{v, Pv, P^2v, \cdots, P^{k}v, \cdots  \}, 
\quad 
\begin{array}{l} v\geq 0 \\ e^{\rm T}v = 1 \end{array}. 
\end{equation} 
%
The space has the property $P {\cal K}(P,v) = {\cal K}(P,v)$, i.e., it
is a spectrally invariant subspace. In PageRank terminology,
${\cal K}(P,v)$ is a personalized invariant subspace.  
%
We have the remarkable fact about the model family in Section~\ref{sec:pageRank-models}. 
%
\newtheorem{theorem}{Theorem}
\begin{theorem}
\label{thm:krylove-functions}
%
Any model solution (\ref{eqn:uni-random-walk}), at any particular
damping parameter value, and its
trajectory (\ref{eqn:trajectory-equation}) are functions in the Krylov
space ${\cal K}(P,v)$.
\end{theorem} 


\begin{theorem}[]
\label{thm:krylov}
Let $m=\mbox{dimension}({\cal K}) $.  Denote by $K$ the matrix
composed of the Krylov vectors.  Let $K = QR$ be the QR factorization of
$K$.  Then, $Qe_1 =v$ and $PQ = QH$, where $H$ is an $m\times m$ upper
Hessenberg matrix, and $e_1$ is the first column of the identity matrix $I$.
\end{theorem}
%

A few remarks. Matrix $H$ in Theorem~\ref{thm:krylov} is the
representation of matrix $P$ under basis $Q$ in the Krylov space.  In
numerical computation, we use a rank-revealing version of the QR
factorization with $Qe_1 = v$.
%
In theory, the dimension $m$ is equal to the number of spectrally
invariant components of $P$ that present in $v$.  In the extreme case,
$m=1$ when $v$ is the Perron vector of $P$. In general, by the condition
$e^{\rm T}v =1$, $v$ is not deficient in Perror component. In our study
on real-world graphs, which we will detail shortly in Section~\ref{sec:numerical-experiments},
the numerical dimension is low, matrix $H$ is therefore small.  We may
view this as a manifest of the smallness of the real-world graphs under
study.  We exploit these theoretical and practical facts.

% 
\newtheorem{corollary}{Corollary}
\setcounter{corollary}{2}
\begin{corollary}[]
\label{col:krylov}
For any function $g$ in the Krylov space (\ref{eqn:Krylov-space}), we
have $g(P)v = Qg(H)e_1$.
\end{corollary}
%
When dimension $m$ is modest, we translate by Corollary \ref{col:krylov}
the calculation of $x_{g}=g(P)v$ with $N\times N$ matrix $P$ on vectors
to the calculation of $ \hat{x}_{g} = g(H)e_1$ with $m\times m$ matrix
$H$ on vectors, followed by a matrix-vector product $Q\hat{x}_g$. The
vector $\hat{x}_g$ is the spectral representation of $x_g$ in the Krylov
space.  In the model family, solutions $x_f$ (\ref{eqn:uni-random-walk})
differ from one to another in their spectral representations
$\hat{x}_f$, they share the same basis matrix $Q$ in the ambient network
space.

Our algorithm consists of the following major steps. Let
${\cal G}= \{ g\}$ be a set of functions under study.
% 
\begin{inparaenum}[(1)] 
\item Calculuate Krylov vectors to form matrix $K$ in
  Theorem~\ref{thm:krylov}, apply rank-revealing $QR$ factorization to
  $K$, and find numerical dimension $m$; \footnote{These substeps are
    integrated in pratical computation in order to determine quickly a
    sufficient number of Krylov vectors.}
%
\item Construct the matrix $H$, by Theorem~\ref{thm:krylov},
  from $R$ and the permutation matrix $\Pi$ rendered by the
  rank-revealing $QR$;
%
\item Calculate $\hat{x}_f= g(H)e_1$ for all functions in ${\cal G}$; 
%
\item Transform $\hat{x}_f$ from the Krylov-spectral space to the
  ambient network space by the same basis matrix $Q$, based on
  Corollary~\ref{col:krylov}.
%     
\end{inparaenum}


% flatex input end: [sections/SIK-space-method.tex]
 
%




% flatex input end: [batch-ranking-algorithms.tex]
 
% 

% ============== EXPERIMENT

\section{Experiments on real-world link graphs} 
\label{sec:numerical-experiments}
%
% flatex input: [numerical-experiments.tex]
%
% section 5: numerical experiments 
% last revision: May 20, 2018

We show in numerical values how PageRank vector responses to variation
in damping variable with each model and across models, on $6$ real-world
link graphs.
%

\subsection{Experiment setup: data and models}
\label{subsec:data-and-models} 
%
% flatex input: [sections/data-and-models.tex]
% data-and-models.tex


The $6$ link graphs we used for our experiments are publicly available
at {\em the Koblenz Network Collection}\cite{kunegis2013konect}. 
The basic information of the graphs is summarized in
Table~\ref{tab:dataset}, where $\max(d_{out})$ is the maximum out-degree
(the number of citations) of graph nodes, $\max(d_{in})$ is the maximum 
in-degree (the number of backlinks), $\mu(d_{out}) = \mu(d_{in})$ is
the average out-degree, which equals to the average in-degree, 
and LSCC stands for the largest strongly connected component(s) of the graph. 
The Google graph of today is reportedly containing hundreds of trillions 
of nodes, substantially larger than the snapshot size used here.
%
% flatex input: [tabtex/tab-dataset.tex]
% datab-dataset.tex
% 

\begin{table}[H]
\centering
\caption{Dataset Description}
\label{tab:dataset}
\begin{adjustbox}{width=0.5\textwidth}
\begin{tabular}[t]{lrrr}
\toprule
&  Total \#nodes & \#nodes in LSCC & $\left[\max(d_{out}), \mu(d_{out}), \max(d_{in})\right]$\\
\midrule
   Google \cite{google-net} & 875,713 & 434,818 & $[ 4209, 8.86, 382]$ 
\\ 
Wikilink \cite{kunegis2013konect} & 12,150,976 & 7,283,915 & $[7527, 50.48, 920207]$ 
\\
DBpedia \cite{auer2007dbpedia} & 18,268,992 & 3,796,073 & $[8104, 26.76, 414924]$ 
\\
Twitter(www)\cite{Kwak10www} & 41,652,230 & 33,479,734 & $[2936232, 42.65, 768552]$  
\\
Twitter(mpi)\cite{icwsm10cha} & 52,579,682 & 40,012,384 & $[778191, 47.57, 3438929]$
\\
friendster\cite{yang2015defining} & 68,349,466 & 48,928,140 & $[3124, 32.76, 3124]$  
\\ 
\bottomrule{}
\end{tabular}
\end{adjustbox}
\end{table}

% flatex input end: [tabtex/tab-dataset.tex]

%
% 
For variation analysis of each graph in Table~\ref{tab:dataset}, the 
associated link matrix $P$ is well specified. We use the same personalized
or customized distribution vector $v$, which we get by drawing elements from 
standard Gaussian distribution ${\cal N}(0, 1)$, followed by 
normalization $v^{\rm T}e = 1$.
% 
We report variation analysis results with three particular models :
Brin-Page model (\ref{eqn:ppr-linear}) with damping variable $\alpha$,
Chung's model (\ref{eqn:heat-model}) with variable $\beta$, and the 
log-$\gamma$ model (\ref{eqn:log-gamma}). The last is used as an 
illustration of new models in the family (\ref{eqn:uni-random-walk}).
% flatex input end: [sections/data-and-models.tex]

%


\subsection{Variation in PageRank vector} 
\label{subsec:intro-model-analysis}
%
% flatex input: [sections/var-rank-vectors.tex]
% var-rank-vectors.tex
%

% flatex input: [figtex/google/fig-google-unify-distribution.tex]
% fig-google-unify-distribution.tex

\begin{figure}[!htb]
  \centering
  \subfigure[$x(\rho)$ of Google link graph]{
    \includegraphics[width=0.22\textwidth]
        {web-Google-1-unify-hist-2d}
    \label{fig:google-hist-unify}
  }
  \subfigure[$x(\rho)$ of Twitter(www) link graph]{
    \includegraphics[width=0.22\textwidth]
       {out-twitter-1-unify-hist-2d}
    \label{fig:twitter-hist-unify}
  }
  \caption{\footnotesize%
    Inter-model comparison of histograms of $N\cdot x_f(\rho)$ among the 
    three models with corresponding parameter values $\alpha_{o}$,
    $\beta_{o}$ and $\gamma_{o}$ so that the expected value of
    walking steps for each model is
    $ \alpha_{o}/(1-\alpha_{o})= 5.\dot{6} $ with $\alpha_{o} = 0.85$,
    see the model correspondence equalities (\ref{eqn:param-corres}).
    (a) comparison on Google link graph; (b) comparison on Twitter(www)
    link graph. }
\label{fig:unify-model-google}
\end{figure}

% flatex input end: [figtex/google/fig-google-unify-distribution.tex]

%
% 
In order to show the quantitative response in PageRank vector $x_f$
over $N$ nodes with the variation in damping variable $\rho$, we display
the histogram of $N\cdot x_f(\rho)$ for model $f$ at parameter value
$\rho$. In Figure~\ref{fig:unify-model-google} we show the
histograms associated with three models on Google network. The parameter for Brin-Page
model is set to the value $\alpha_{o} = 0.85$. The parameter for 
the other two models are set by (\ref{eqn:param-corres}).
%
%%% ----- observation ------------- 
We observe that the histogram with Brin-Page model has higher and narrower peaks
than Chung's model. The histogram of
log-$\gamma$ model is in between. This is expected by the relationships
in the damping weights among the three models, as discussed in
Section~\ref{subsec:damping-corres}.

% flatex input: [figtex/google/fig-google-distribution.tex]
% fig-google-distribution.tex
% caption revision by Xiaobai on May 20, 2018 
% latex layout revision as well 
% 


\begin{figure}[!htb]
  \centering
  \subfigure[Brin-Page model]{
    \includegraphics[width=0.14\textwidth]
     {web-Google-1-alpha-hist-2d}
    \label{fig:google-hist-bp}
  }
  \subfigure[log-$\gamma$ model]{
    \includegraphics[width=0.14\textwidth]
    {web-Google-1-gamma-hist-2d}
    \label{fig:google-hist-chung}
  }
  \subfigure[Chung's model]{
    \includegraphics[width=0.14\textwidth]
    {web-Google-1-beta-hist-2d}
    \label{fig:google-hist-chung}
  }
  \subfigure[Brin-Page model]{
    \includegraphics[width=0.14\textwidth]
    {web-Google-1-alpha-hist-3d}
    \label{fig:google-hist-bp-3d}
  }
  \subfigure[log-$\gamma$ model]{
    \includegraphics[width=0.14\textwidth]
    {web-Google-1-gamma-hist-3d}
    \label{fig:google-hist-chung-3d}
  }
  \subfigure[Chung's model]{
    \includegraphics[width=0.14\textwidth]
    {web-Google-1-beta-hist-3d}
    \label{fig:google-hist-chung-3d}
  }

  
  \caption{\footnotesize 
    Comparison in histograms of $N\cdot x_f(\rho)$ on the Google link
    graph over a range of damping variable value. 
    %  
    {\bf Left column:} Brin-Page model, 
    {\bf Middle column:} log-$\gamma$ model, 
    {\bf Right column:} Chung's model.
    %%%  
    {\bf Top row:} 2D display of 6 histograms associated with
    6 parameter values shown in the respective legends.  The
    histogram in black with Brin-Page model is associated with the value
    $\alpha = 0.85$.  The corresponding parameter values with Chung's
    model and log-$\gamma$ model are set by (\ref{eqn:param-corres}), 
    the associated histograms are color
    coded by the corresponding parameter values.
    %%% 
    {\bf Bottom row:} a stack of multiple histograms shown in 3D space over
    the range $\alpha \in [0.7, 0.97]$ with Brin-Page model,
    $\beta \in [2.\dot{6}, 32.\dot{3}] $ with Chung's model, and $\gamma \in [0.7787,0.994]$ with log-$\gamma$ model. The histograms with Chung's
    model have flattened peaks at larger values (toward the back
    end). The log-$\gamma$ model is nearly insensitive to $\gamma$ change in the range above.}  %
\label{fig:histogram-google}
\end{figure}

% flatex input end: [figtex/google/fig-google-distribution.tex]

%%% ----- observation ------------- 
% 
% flatex input: [figtex/twitter/fig-twitter-distribution.tex]
% fig-twitter-distrbution.tex 
% 

\begin{figure}[!htb]
  \centering
  \subfigure[Brin-Page model]{
    \includegraphics[width=0.14\textwidth]
        {out-twitter-1-alpha-hist-2d}
    \label{fig:twitter-hist-bp}
  }
  \subfigure[log-$\gamma$ model]{
    \includegraphics[width=0.14\textwidth]
    {out-twitter-1-gamma-hist-2d}
    \label{fig:google-hist-chung}
  }
  \subfigure[Chung's model]{
    \includegraphics[width=0.14\textwidth]
        {out-twitter-1-beta-hist-2d}
    \label{fig:twitter-hist-chung}
  }
  \subfigure[Brin-Page model]{
    \includegraphics[width=0.14\textwidth]
        {out-twitter-1-alpha-hist-3d}
    \label{fig:twitter-hist-bp-3d}
  }
  \subfigure[log-$\gamma$ model]{
    \includegraphics[width=0.14\textwidth]
    {out-twitter-1-gamma-hist-3d}
    \label{fig:google-hist-chung-3d}
  }
  \subfigure[Chung's model]{
    \includegraphics[width=0.14\textwidth]
        {out-twitter-1-beta-hist-3d}
    \label{fig:twitter-hist-chung-3d}
  }

  \caption{\footnotesize 
  Comparison in histogram of $N\cdot x_f(\rho)$ on the Twitter graph 
  over a range of dampling value, in the same settings as in
  Figure~\ref{fig:histogram-google}.
  }
\label{fig:histogram-twitter}
\end{figure}

% flatex input end: [figtex/twitter/fig-twitter-distribution.tex]

% 
%
Figure \ref{fig:histogram-google} and Figure \ref{fig:histogram-twitter} show the variation in the histograms over a range of the damping
variable per model as well as the comparison side by side between
the three models on two datasets. The models
have similar behaviors on the other $4$ graphs in Table~\ref{tab:dataset}.
Supplementary material can be found in \cite{MS-qian-2018}. With larger damping factors in the models, 
the distribution become less centralized. Log-$\gamma$ model, specifically, is less sensitive to $\gamma(\alpha)$
range with $\alpha \in [0.7,0.97]$.
% 

% flatex input end: [sections/var-rank-vectors.tex]

%



\subsection{Relative variation measured by  KL divergence} 
\label{subsec:intro-model-analysis}
%
% flatex input: [sections/var-KL-divergence.tex]
% var-KL-divergence.tex

% flatex input: [figtex/google/fig-google-KL-DKL.tex]
% fig-google-KL&DKL-0.85.tex

\begin{figure}[!htb]
  \centering
  \subfigure[{\footnotesize $\alpha_o=0.85$}]{
    \includegraphics[width=0.14\textwidth]
          {web-Google-1-alpha-kl-dkl-1}
  }
  \subfigure[{\footnotesize $\gamma_o=0.94146$}]{
    \includegraphics[width=0.14\textwidth]
          {web-Google-1-gamma-kl-dkl-1}
  }
  \subfigure[{\footnotesize $\beta_o=5.\dot{6}$}]{
    \includegraphics[width=0.14\textwidth]
          {web-Google-1-beta-kl-dkl-1}
  }
  \subfigure[{\footnotesize $\alpha_o=0.95$}]{
    \includegraphics[width=0.14\textwidth]
          {web-Google-1-alpha-kl-dkl-2}
  }
  \subfigure[{\footnotesize $\gamma_o=0.98831$}]{
    \includegraphics[width=0.14\textwidth]
          {web-Google-1-gamma-kl-dkl-2}
  }
  \subfigure[{\footnotesize $\beta_o=19$}]{
    \includegraphics[width=0.14\textwidth]
          {web-Google-1-beta-kl-dkl-2}
  }
  
  \caption{\footnotesize Intra-model relative variation as defined in (\ref{eqn:kl}) in PgeRank
    distribution, on the Google graph, with respect to two reference
    distribution at $\alpha_{o} \in \{0.85,0.95\}$ with Brin-Page model ({\bf left
    column}), $\gamma_{o} \in \{0.94146,0.98831\}$ with log-$\gamma$ model ({\bf middle column}), 
    and $\beta_{o} \in \{5.\dot{6},19\}$ with Chung's model ({\bf right column}). 
    %%% ==============
    {\bf Blue curves:} the KL score 
    $KL( x_f(\rho) || x_{f}(\rho_{o}) )$ with numerically 
    computed distribution vectors; 
    {\bf Red curves:} the derivative of the KL score 
    $ (d/d\rho) KL( x_f(\rho) || x_{f}(\rho_{o} ))$. 
    The red curves with $\cdot$ marker and $\diamond$ marker are obtained empirically 
    from numerical distribution vectors, with step size 
    $\Delta \rho = (0.002, 0.008) $ respectively. The red curves with $\times$ marker
    are obtained analytically by (\ref{eqn:kl-derivative}). 
    %%% =============== 
    {\bf Remarks.} With Brin-Page model and log-$\gamma$ model, the distribution changes gently from
    the reference distribution in the neighborhood of the reference
    value $\alpha_{o} = 0.85$, by the KL curve and the KL derivative
    curve.  In sharp contrast, the distribution with Chung's model
    deviates rapidly from the reference distribution.  }
  \label{fig:google-param-KL-0.85}
\end{figure}

% flatex input end: [figtex/google/fig-google-KL-DKL.tex]

% var-KL-divergence.tex

We show the relative variation in PageRank vector with respect to a
reference vector by (\ref{eqn:kl}).  For Brin-Page model, we consider
two particular reference vectors: one is associated with $\alpha=0.85$
as chosen originally by Brin and Page, the other is at $\alpha = 0.95$,
much closer to the extreme case $\alpha=1$, in which the walks follow
the links only.  For Chung's model and log-$\gamma$ model, we use the corresponding parameter
values by (\ref{eqn:param-corres}).
%
We show the differences between the three models on the Google graph in
Figure~\ref{fig:google-param-KL-0.85} at the corresponding reference values, 
respectively, and on the twitter graph in
Figure~\ref{fig:twitter-param-KL-0.85}.

% flatex input: [figtex/twitter/fig-twitter-KL-DKL.tex]
\begin{figure}[!htb]
  \centering
  \subfigure[{\footnotesize $\alpha_o=0.85$}]{
    \includegraphics[width=0.14\textwidth]
          {out-twitter-1-alpha-kl-dkl-1}
  }
  \subfigure[{\footnotesize $\gamma_o=0.94146$}]{
    \includegraphics[width=0.14\textwidth]
          {out-twitter-1-gamma-kl-dkl-1}
  }
  \subfigure[{\footnotesize $\beta_o=5.\dot{6}$}]{
    \includegraphics[width=0.14\textwidth]
          {out-twitter-1-beta-kl-dkl-1}
  }
  \subfigure[{\footnotesize $\alpha_o=0.95$}]{
    \includegraphics[width=0.14\textwidth]
          {out-twitter-1-alpha-kl-dkl-2}
  }
  \subfigure[{\footnotesize $\gamma_o=0.98831$}]{
    \includegraphics[width=0.14\textwidth]
          {out-twitter-1-gamma-kl-dkl-2}
  }
  \subfigure[{\footnotesize $\beta_o=19$}]{
    \includegraphics[width=0.14\textwidth]
          {out-twitter-1-beta-kl-dkl-2}
  }
  \caption{\footnotesize Intra-model relative variation in PageRank
    distribution by (\ref{eqn:kl}), on the Twitter graph. The rest is in the same setting as in Figure \ref{fig:google-param-KL-0.85}.}
%
\label{fig:twitter-param-KL-0.85}
\end{figure}

% flatex input end: [figtex/twitter/fig-twitter-KL-DKL.tex]

%




% flatex input end: [sections/var-KL-divergence.tex]
 
%



\subsection{Batch calculation: efficiency and accuracy} 
\label{subsec:batch-calculation} 
% 
% flatex input: [sections/batch-evaluation.tex]
% batch-evaluation.tex
%

We show first that the Krylov space dimension is numerically low for each of the 6 real world link graphs.
Figure \ref{fig:qr} gives the diagonal elements of each upper-triangular matrix $R$ obtained by a 
rank-revealing QR factorization. The elements below $10^{-17}$ are not shown. The numerical 
dimension ranges from $19$ with DBpedia link graph to $62$ with Google link graph. The low numerical 
dimension makes our algorithm in Section \ref{sec:batch-ranking} highly efficient. In addition, we 
exploited the sparsity of matrix $P$ in the Krylov vector calculation, see details in \cite{MS-Xichen-2018}.

% flatex input: [figtex/R-diag.tex]
\begin{figure}[!htb]
  \centering
  \subfigure[Google]{
    \includegraphics[width=0.13\textwidth]{web-Google-R-diag}
  }
  \subfigure[Twitter(www)]{
    \includegraphics[width=0.13\textwidth]{out-twitter-R-diag}
  }
  \subfigure[Wikipedia]{
    \includegraphics[width=0.13\textwidth]{wikipedia-R-diag}
  }
  \subfigure[DBpedia]{
    \includegraphics[width=0.13\textwidth]{dbpedia-R-diag}
  }
  \subfigure[Twitter(mpi)]{
    \includegraphics[width=0.13\textwidth]{twitter_mpi-R-diag}
  }
  \subfigure[Friendster]{
    \includegraphics[width=0.13\textwidth]{friendster-R-diag}
  }
  \caption{The diagonal elements of each upper-triangular matrix $R$ obtained by a rank-revealing QR factorization for the 6 datasets in Table \ref{tab:dataset}. Google link graph has the highest numerical dimension 62 among the 6 datasets, and the DBpedia link graph has the lowest numerical dimension 19.}
  \label{fig:qr}
\end{figure}
% flatex input end: [figtex/R-diag.tex]

%

The accuracy of our batch algorithm is evaluated in two ways. One is by 
$err = \|(x_{\text{Krylov}} - x_{\text{G-S}}) ./ x_{\text{G-S}}\|_{\infty}$, 
the maximum element-wise relative difference in
the PageRank vectors of Brin-Page model between the Gauss-Seidel method and 
our Krylov subspace method. In our experiments, the relative errors for all 6 datasets 
are below $10^{-10}$. In the other way, we show in 
Figure \ref{fig:google-param-KL-0.85} and Figure \ref{fig:twitter-param-KL-0.85} 
that the empirical rate of change agrees well with analytical prediction 
(\ref{eqn:kl-derivative}).


% flatex input end: [sections/batch-evaluation.tex]
 
% 

% flatex input end: [numerical-experiments.tex]

%

% ============== SUMMARY

\section{Concluding remarks} 
%
% flatex input: [summary.tex]
% summary.tex

Our model extension, connection, unified analysis and numerical
algorithm for quantitative estimation in batch are original, to our
knowledge. Our study leads to new observation and several findings.
%
\begin{inparaenum}[(a)]
\item In network propagation pattern in response to variation in the 
  damping mechanism, the inter-model difference among the 3 models
  is much more significant than the inter-dataset difference among
  the 6 datasets. This suggests the utility of
  model variety for differentiating network activities or propagation
  patterns.
% 
\item The model solutions reside in the same customized, spectrally
  invariant subspace. On each of the $6$ real-world graphs, the space
  dimension is low, which is a small-world phenomenon.
% 
\item The shared computation is not limited to one personalized
  distribution. The Krylov space associated with a particular vector $v$
  contains certainly many other distribution vectors. In fact, every
  Krylov vector is a distribution vector. This finding may lead to a much 
  more efficient way to represent and compute PageRank distributions 
  across multiple personalized vectors.
% 
\item The low spectral dimension, estimated once for a particular graph
  $P$ and a personalized/customized vector $v$, may serve as a
  reasonable upper bound on the number of iterations by any competitive
  algorithm, with one matrix-vector product per iteration, for Brin-Page
  model, at any $\alpha$ value in $(0, 1)$, or any
  other model in the family (\ref{eqn:uni-random-walk}). The power method and the
  Gauss-Seidel iteration take more iterations to reach the same error
  level on the larger real-world graphs among the studied, and take many
  more iterations when $\alpha$ gets closer to $1$.
%
\end{inparaenum} 
%
In brief conclusion, estimating PageRank distribution under various
damping conditions is valuable and easily affordable.

% flatex input end: [summary.tex]

%

%\section{Appendix}
%\chapter{Supplementary Material}
\label{appendix}

In this appendix, we present supplementary material for the techniques and
experiments presented in the main text.

\section{Baseline Results and Analysis for Informed Sampler}
\label{appendix:chap3}

Here, we give an in-depth
performance analysis of the various samplers and the effect of their
hyperparameters. We choose hyperparameters with the lowest PSRF value
after $10k$ iterations, for each sampler individually. If the
differences between PSRF are not significantly different among
multiple values, we choose the one that has the highest acceptance
rate.

\subsection{Experiment: Estimating Camera Extrinsics}
\label{appendix:chap3:room}

\subsubsection{Parameter Selection}
\paragraph{Metropolis Hastings (\MH)}

Figure~\ref{fig:exp1_MH} shows the median acceptance rates and PSRF
values corresponding to various proposal standard deviations of plain
\MH~sampling. Mixing gets better and the acceptance rate gets worse as
the standard deviation increases. The value $0.3$ is selected standard
deviation for this sampler.

\paragraph{Metropolis Hastings Within Gibbs (\MHWG)}

As mentioned in Section~\ref{sec:room}, the \MHWG~sampler with one-dimensional
updates did not converge for any value of proposal standard deviation.
This problem has high correlation of the camera parameters and is of
multi-modal nature, which this sampler has problems with.

\paragraph{Parallel Tempering (\PT)}

For \PT~sampling, we took the best performing \MH~sampler and used
different temperature chains to improve the mixing of the
sampler. Figure~\ref{fig:exp1_PT} shows the results corresponding to
different combination of temperature levels. The sampler with
temperature levels of $[1,3,27]$ performed best in terms of both
mixing and acceptance rate.

\paragraph{Effect of Mixture Coefficient in Informed Sampling (\MIXLMH)}

Figure~\ref{fig:exp1_alpha} shows the effect of mixture
coefficient ($\alpha$) on the informed sampling
\MIXLMH. Since there is no significant different in PSRF values for
$0 \le \alpha \le 0.7$, we chose $0.7$ due to its high acceptance
rate.


% \end{multicols}

\begin{figure}[h]
\centering
  \subfigure[MH]{%
    \includegraphics[width=.48\textwidth]{figures/supplementary/camPose_MH.pdf} \label{fig:exp1_MH}
  }
  \subfigure[PT]{%
    \includegraphics[width=.48\textwidth]{figures/supplementary/camPose_PT.pdf} \label{fig:exp1_PT}
  }
\\
  \subfigure[INF-MH]{%
    \includegraphics[width=.48\textwidth]{figures/supplementary/camPose_alpha.pdf} \label{fig:exp1_alpha}
  }
  \mycaption{Results of the `Estimating Camera Extrinsics' experiment}{PRSFs and Acceptance rates corresponding to (a) various standard deviations of \MH, (b) various temperature level combinations of \PT sampling and (c) various mixture coefficients of \MIXLMH sampling.}
\end{figure}



\begin{figure}[!t]
\centering
  \subfigure[\MH]{%
    \includegraphics[width=.48\textwidth]{figures/supplementary/occlusionExp_MH.pdf} \label{fig:exp2_MH}
  }
  \subfigure[\BMHWG]{%
    \includegraphics[width=.48\textwidth]{figures/supplementary/occlusionExp_BMHWG.pdf} \label{fig:exp2_BMHWG}
  }
\\
  \subfigure[\MHWG]{%
    \includegraphics[width=.48\textwidth]{figures/supplementary/occlusionExp_MHWG.pdf} \label{fig:exp2_MHWG}
  }
  \subfigure[\PT]{%
    \includegraphics[width=.48\textwidth]{figures/supplementary/occlusionExp_PT.pdf} \label{fig:exp2_PT}
  }
\\
  \subfigure[\INFBMHWG]{%
    \includegraphics[width=.5\textwidth]{figures/supplementary/occlusionExp_alpha.pdf} \label{fig:exp2_alpha}
  }
  \mycaption{Results of the `Occluding Tiles' experiment}{PRSF and
    Acceptance rates corresponding to various standard deviations of
    (a) \MH, (b) \BMHWG, (c) \MHWG, (d) various temperature level
    combinations of \PT~sampling and; (e) various mixture coefficients
    of our informed \INFBMHWG sampling.}
\end{figure}

%\onecolumn\newpage\twocolumn
\subsection{Experiment: Occluding Tiles}
\label{appendix:chap3:tiles}

\subsubsection{Parameter Selection}

\paragraph{Metropolis Hastings (\MH)}

Figure~\ref{fig:exp2_MH} shows the results of
\MH~sampling. Results show the poor convergence for all proposal
standard deviations and rapid decrease of AR with increasing standard
deviation. This is due to the high-dimensional nature of
the problem. We selected a standard deviation of $1.1$.

\paragraph{Blocked Metropolis Hastings Within Gibbs (\BMHWG)}

The results of \BMHWG are shown in Figure~\ref{fig:exp2_BMHWG}. In
this sampler we update only one block of tile variables (of dimension
four) in each sampling step. Results show much better performance
compared to plain \MH. The optimal proposal standard deviation for
this sampler is $0.7$.

\paragraph{Metropolis Hastings Within Gibbs (\MHWG)}

Figure~\ref{fig:exp2_MHWG} shows the result of \MHWG sampling. This
sampler is better than \BMHWG and converges much more quickly. Here
a standard deviation of $0.9$ is found to be best.

\paragraph{Parallel Tempering (\PT)}

Figure~\ref{fig:exp2_PT} shows the results of \PT sampling with various
temperature combinations. Results show no improvement in AR from plain
\MH sampling and again $[1,3,27]$ temperature levels are found to be optimal.

\paragraph{Effect of Mixture Coefficient in Informed Sampling (\INFBMHWG)}

Figure~\ref{fig:exp2_alpha} shows the effect of mixture
coefficient ($\alpha$) on the blocked informed sampling
\INFBMHWG. Since there is no significant different in PSRF values for
$0 \le \alpha \le 0.8$, we chose $0.8$ due to its high acceptance
rate.



\subsection{Experiment: Estimating Body Shape}
\label{appendix:chap3:body}

\subsubsection{Parameter Selection}
\paragraph{Metropolis Hastings (\MH)}

Figure~\ref{fig:exp3_MH} shows the result of \MH~sampling with various
proposal standard deviations. The value of $0.1$ is found to be
best.

\paragraph{Metropolis Hastings Within Gibbs (\MHWG)}

For \MHWG sampling we select $0.3$ proposal standard
deviation. Results are shown in Fig.~\ref{fig:exp3_MHWG}.


\paragraph{Parallel Tempering (\PT)}

As before, results in Fig.~\ref{fig:exp3_PT}, the temperature levels
were selected to be $[1,3,27]$ due its slightly higher AR.

\paragraph{Effect of Mixture Coefficient in Informed Sampling (\MIXLMH)}

Figure~\ref{fig:exp3_alpha} shows the effect of $\alpha$ on PSRF and
AR. Since there is no significant differences in PSRF values for $0 \le
\alpha \le 0.8$, we choose $0.8$.


\begin{figure}[t]
\centering
  \subfigure[\MH]{%
    \includegraphics[width=.48\textwidth]{figures/supplementary/bodyShape_MH.pdf} \label{fig:exp3_MH}
  }
  \subfigure[\MHWG]{%
    \includegraphics[width=.48\textwidth]{figures/supplementary/bodyShape_MHWG.pdf} \label{fig:exp3_MHWG}
  }
\\
  \subfigure[\PT]{%
    \includegraphics[width=.48\textwidth]{figures/supplementary/bodyShape_PT.pdf} \label{fig:exp3_PT}
  }
  \subfigure[\MIXLMH]{%
    \includegraphics[width=.48\textwidth]{figures/supplementary/bodyShape_alpha.pdf} \label{fig:exp3_alpha}
  }
\\
  \mycaption{Results of the `Body Shape Estimation' experiment}{PRSFs and
    Acceptance rates corresponding to various standard deviations of
    (a) \MH, (b) \MHWG; (c) various temperature level combinations
    of \PT sampling and; (d) various mixture coefficients of the
    informed \MIXLMH sampling.}
\end{figure}


\subsection{Results Overview}
Figure~\ref{fig:exp_summary} shows the summary results of the all the three
experimental studies related to informed sampler.
\begin{figure*}[h!]
\centering
  \subfigure[Results for: Estimating Camera Extrinsics]{%
    \includegraphics[width=0.9\textwidth]{figures/supplementary/camPose_ALL.pdf} \label{fig:exp1_all}
  }
  \subfigure[Results for: Occluding Tiles]{%
    \includegraphics[width=0.9\textwidth]{figures/supplementary/occlusionExp_ALL.pdf} \label{fig:exp2_all}
  }
  \subfigure[Results for: Estimating Body Shape]{%
    \includegraphics[width=0.9\textwidth]{figures/supplementary/bodyShape_ALL.pdf} \label{fig:exp3_all}
  }
  \label{fig:exp_summary}
  \mycaption{Summary of the statistics for the three experiments}{Shown are
    for several baseline methods and the informed samplers the
    acceptance rates (left), PSRFs (middle), and RMSE values
    (right). All results are median results over multiple test
    examples.}
\end{figure*}

\subsection{Additional Qualitative Results}

\subsubsection{Occluding Tiles}
In Figure~\ref{fig:exp2_visual_more} more qualitative results of the
occluding tiles experiment are shown. The informed sampling approach
(\INFBMHWG) is better than the best baseline (\MHWG). This still is a
very challenging problem since the parameters for occluded tiles are
flat over a large region. Some of the posterior variance of the
occluded tiles is already captured by the informed sampler.

\begin{figure*}[h!]
\begin{center}
\centerline{\includegraphics[width=0.95\textwidth]{figures/supplementary/occlusionExp_Visual.pdf}}
\mycaption{Additional qualitative results of the occluding tiles experiment}
  {From left to right: (a)
  Given image, (b) Ground truth tiles, (c) OpenCV heuristic and most probable estimates
  from 5000 samples obtained by (d) MHWG sampler (best baseline) and
  (e) our INF-BMHWG sampler. (f) Posterior expectation of the tiles
  boundaries obtained by INF-BMHWG sampling (First 2000 samples are
  discarded as burn-in).}
\label{fig:exp2_visual_more}
\end{center}
\end{figure*}

\subsubsection{Body Shape}
Figure~\ref{fig:exp3_bodyMeshes} shows some more results of 3D mesh
reconstruction using posterior samples obtained by our informed
sampling \MIXLMH.

\begin{figure*}[t]
\begin{center}
\centerline{\includegraphics[width=0.75\textwidth]{figures/supplementary/bodyMeshResults.pdf}}
\mycaption{Qualitative results for the body shape experiment}
  {Shown is the 3D mesh reconstruction results with first 1000 samples obtained
  using the \MIXLMH informed sampling method. (blue indicates small
  values and red indicates high values)}
\label{fig:exp3_bodyMeshes}
\end{center}
\end{figure*}

\clearpage



\section{Additional Results on the Face Problem with CMP}

Figure~\ref{fig:shading-qualitative-multiple-subjects-supp} shows inference results for reflectance maps, normal maps and lights for randomly chosen test images, and Fig.~\ref{fig:shading-qualitative-same-subject-supp} shows reflectance estimation results on multiple images of the same subject produced under different illumination conditions. CMP is able to produce estimates that are closer to the groundtruth across different subjects and illumination conditions.

\begin{figure*}[h]
  \begin{center}
  \centerline{\includegraphics[width=1.0\columnwidth]{figures/face_cmp_visual_results_supp.pdf}}
  \vspace{-1.2cm}
  \end{center}
	\mycaption{A visual comparison of inference results}{(a)~Observed images. (b)~Inferred reflectance maps. \textit{GT} is the photometric stereo groundtruth, \textit{BU} is the Biswas \etal (2009) reflectance estimate and \textit{Forest} is the consensus prediction. (c)~The variance of the inferred reflectance estimate produced by \MTD (normalized across rows).(d)~Visualization of inferred light directions. (e)~Inferred normal maps.}
	\label{fig:shading-qualitative-multiple-subjects-supp}
\end{figure*}


\begin{figure*}[h]
	\centering
	\setlength\fboxsep{0.2mm}
	\setlength\fboxrule{0pt}
	\begin{tikzpicture}

		\matrix at (0, 0) [matrix of nodes, nodes={anchor=east}, column sep=-0.05cm, row sep=-0.2cm]
		{
			\fbox{\includegraphics[width=1cm]{figures/sample_3_4_X.png}} &
			\fbox{\includegraphics[width=1cm]{figures/sample_3_4_GT.png}} &
			\fbox{\includegraphics[width=1cm]{figures/sample_3_4_BISWAS.png}}  &
			\fbox{\includegraphics[width=1cm]{figures/sample_3_4_VMP.png}}  &
			\fbox{\includegraphics[width=1cm]{figures/sample_3_4_FOREST.png}}  &
			\fbox{\includegraphics[width=1cm]{figures/sample_3_4_CMP.png}}  &
			\fbox{\includegraphics[width=1cm]{figures/sample_3_4_CMPVAR.png}}
			 \\

			\fbox{\includegraphics[width=1cm]{figures/sample_3_5_X.png}} &
			\fbox{\includegraphics[width=1cm]{figures/sample_3_5_GT.png}} &
			\fbox{\includegraphics[width=1cm]{figures/sample_3_5_BISWAS.png}}  &
			\fbox{\includegraphics[width=1cm]{figures/sample_3_5_VMP.png}}  &
			\fbox{\includegraphics[width=1cm]{figures/sample_3_5_FOREST.png}}  &
			\fbox{\includegraphics[width=1cm]{figures/sample_3_5_CMP.png}}  &
			\fbox{\includegraphics[width=1cm]{figures/sample_3_5_CMPVAR.png}}
			 \\

			\fbox{\includegraphics[width=1cm]{figures/sample_3_6_X.png}} &
			\fbox{\includegraphics[width=1cm]{figures/sample_3_6_GT.png}} &
			\fbox{\includegraphics[width=1cm]{figures/sample_3_6_BISWAS.png}}  &
			\fbox{\includegraphics[width=1cm]{figures/sample_3_6_VMP.png}}  &
			\fbox{\includegraphics[width=1cm]{figures/sample_3_6_FOREST.png}}  &
			\fbox{\includegraphics[width=1cm]{figures/sample_3_6_CMP.png}}  &
			\fbox{\includegraphics[width=1cm]{figures/sample_3_6_CMPVAR.png}}
			 \\
	     };

       \node at (-3.85, -2.0) {\small Observed};
       \node at (-2.55, -2.0) {\small `GT'};
       \node at (-1.27, -2.0) {\small BU};
       \node at (0.0, -2.0) {\small MP};
       \node at (1.27, -2.0) {\small Forest};
       \node at (2.55, -2.0) {\small \textbf{CMP}};
       \node at (3.85, -2.0) {\small Variance};

	\end{tikzpicture}
	\mycaption{Robustness to varying illumination}{Reflectance estimation on a subject images with varying illumination. Left to right: observed image, photometric stereo estimate (GT)
  which is used as a proxy for groundtruth, bottom-up estimate of \cite{Biswas2009}, VMP result, consensus forest estimate, CMP mean, and CMP variance.}
	\label{fig:shading-qualitative-same-subject-supp}
\end{figure*}

\clearpage

\section{Additional Material for Learning Sparse High Dimensional Filters}
\label{sec:appendix-bnn}

This part of supplementary material contains a more detailed overview of the permutohedral
lattice convolution in Section~\ref{sec:permconv}, more experiments in
Section~\ref{sec:addexps} and additional results with protocols for
the experiments presented in Chapter~\ref{chap:bnn} in Section~\ref{sec:addresults}.

\vspace{-0.2cm}
\subsection{General Permutohedral Convolutions}
\label{sec:permconv}

A core technical contribution of this work is the generalization of the Gaussian permutohedral lattice
convolution proposed in~\cite{adams2010fast} to the full non-separable case with the
ability to perform back-propagation. Although, conceptually, there are minor
differences between Gaussian and general parameterized filters, there are non-trivial practical
differences in terms of the algorithmic implementation. The Gauss filters belong to
the separable class and can thus be decomposed into multiple
sequential one dimensional convolutions. We are interested in the general filter
convolutions, which can not be decomposed. Thus, performing a general permutohedral
convolution at a lattice point requires the computation of the inner product with the
neighboring elements in all the directions in the high-dimensional space.

Here, we give more details of the implementation differences of separable
and non-separable filters. In the following, we will explain the scalar case first.
Recall, that the forward pass of general permutohedral convolution
involves 3 steps: \textit{splatting}, \textit{convolving} and \textit{slicing}.
We follow the same splatting and slicing strategies as in~\cite{adams2010fast}
since these operations do not depend on the filter kernel. The main difference
between our work and the existing implementation of~\cite{adams2010fast} is
the way that the convolution operation is executed. This proceeds by constructing
a \emph{blur neighbor} matrix $K$ that stores for every lattice point all
values of the lattice neighbors that are needed to compute the filter output.

\begin{figure}[t!]
  \centering
    \includegraphics[width=0.6\columnwidth]{figures/supplementary/lattice_construction}
  \mycaption{Illustration of 1D permutohedral lattice construction}
  {A $4\times 4$ $(x,y)$ grid lattice is projected onto the plane defined by the normal
  vector $(1,1)^{\top}$. This grid has $s+1=4$ and $d=2$ $(s+1)^{d}=4^2=16$ elements.
  In the projection, all points of the same color are projected onto the same points in the plane.
  The number of elements of the projected lattice is $t=(s+1)^d-s^d=4^2-3^2=7$, that is
  the $(4\times 4)$ grid minus the size of lattice that is $1$ smaller at each size, in this
  case a $(3\times 3)$ lattice (the upper right $(3\times 3)$ elements).
  }
\label{fig:latticeconstruction}
\end{figure}

The blur neighbor matrix is constructed by traversing through all the populated
lattice points and their neighboring elements.
% For efficiency, we do this matrix construction recursively with shared computations
% since $n^{th}$ neighbourhood elements are $1^{st}$ neighborhood elements of $n-1^{th}$ neighbourhood elements. \pg{do not understand}
This is done recursively to share computations. For any lattice point, the neighbors that are
$n$ hops away are the direct neighbors of the points that are $n-1$ hops away.
The size of a $d$ dimensional spatial filter with width $s+1$ is $(s+1)^{d}$ (\eg, a
$3\times 3$ filter, $s=2$ in $d=2$ has $3^2=9$ elements) and this size grows
exponentially in the number of dimensions $d$. The permutohedral lattice is constructed by
projecting a regular grid onto the plane spanned by the $d$ dimensional normal vector ${(1,\ldots,1)}^{\top}$. See
Fig.~\ref{fig:latticeconstruction} for an illustration of the 1D lattice construction.
Many corners of a grid filter are projected onto the same point, in total $t = {(s+1)}^{d} -
s^{d}$ elements remain in the permutohedral filter with $s$ neighborhood in $d-1$ dimensions.
If the lattice has $m$ populated elements, the
matrix $K$ has size $t\times m$. Note that, since the input signal is typically
sparse, only a few lattice corners are being populated in the \textit{slicing} step.
We use a hash-table to keep track of these points and traverse only through
the populated lattice points for this neighborhood matrix construction.

Once the blur neighbor matrix $K$ is constructed, we can perform the convolution
by the matrix vector multiplication
\begin{equation}
\ell' = BK,
\label{eq:conv}
\end{equation}
where $B$ is the $1 \times t$ filter kernel (whose values we will learn) and $\ell'\in\mathbb{R}^{1\times m}$
is the result of the filtering at the $m$ lattice points. In practice, we found that the
matrix $K$ is sometimes too large to fit into GPU memory and we divided the matrix $K$
into smaller pieces to compute Eq.~\ref{eq:conv} sequentially.

In the general multi-dimensional case, the signal $\ell$ is of $c$ dimensions. Then
the kernel $B$ is of size $c \times t$ and $K$ stores the $c$ dimensional vectors
accordingly. When the input and output points are different, we slice only the
input points and splat only at the output points.


\subsection{Additional Experiments}
\label{sec:addexps}
In this section, we discuss more use-cases for the learned bilateral filters, one
use-case of BNNs and two single filter applications for image and 3D mesh denoising.

\subsubsection{Recognition of subsampled MNIST}\label{sec:app_mnist}

One of the strengths of the proposed filter convolution is that it does not
require the input to lie on a regular grid. The only requirement is to define a distance
between features of the input signal.
We highlight this feature with the following experiment using the
classical MNIST ten class classification problem~\cite{lecun1998mnist}. We sample a
sparse set of $N$ points $(x,y)\in [0,1]\times [0,1]$
uniformly at random in the input image, use their interpolated values
as signal and the \emph{continuous} $(x,y)$ positions as features. This mimics
sub-sampling of a high-dimensional signal. To compare against a spatial convolution,
we interpolate the sparse set of values at the grid positions.

We take a reference implementation of LeNet~\cite{lecun1998gradient} that
is part of the Caffe project~\cite{jia2014caffe} and compare it
against the same architecture but replacing the first convolutional
layer with a bilateral convolution layer (BCL). The filter size
and numbers are adjusted to get a comparable number of parameters
($5\times 5$ for LeNet, $2$-neighborhood for BCL).

The results are shown in Table~\ref{tab:all-results}. We see that training
on the original MNIST data (column Original, LeNet vs. BNN) leads to a slight
decrease in performance of the BNN (99.03\%) compared to LeNet
(99.19\%). The BNN can be trained and evaluated on sparse
signals, and we resample the image as described above for $N=$ 100\%, 60\% and
20\% of the total number of pixels. The methods are also evaluated
on test images that are subsampled in the same way. Note that we can
train and test with different subsampling rates. We introduce an additional
bilinear interpolation layer for the LeNet architecture to train on the same
data. In essence, both models perform a spatial interpolation and thus we
expect them to yield a similar classification accuracy. Once the data is of
higher dimensions, the permutohedral convolution will be faster due to hashing
the sparse input points, as well as less memory demanding in comparison to
naive application of a spatial convolution with interpolated values.

\begin{table}[t]
  \begin{center}
    \footnotesize
    \centering
    \begin{tabular}[t]{lllll}
      \toprule
              &     & \multicolumn{3}{c}{Test Subsampling} \\
       Method  & Original & 100\% & 60\% & 20\%\\
      \midrule
       LeNet &  \textbf{0.9919} & 0.9660 & 0.9348 & \textbf{0.6434} \\
       BNN &  0.9903 & \textbf{0.9844} & \textbf{0.9534} & 0.5767 \\
      \hline
       LeNet 100\% & 0.9856 & 0.9809 & 0.9678 & \textbf{0.7386} \\
       BNN 100\% & \textbf{0.9900} & \textbf{0.9863} & \textbf{0.9699} & 0.6910 \\
      \hline
       LeNet 60\% & 0.9848 & 0.9821 & 0.9740 & 0.8151 \\
       BNN 60\% & \textbf{0.9885} & \textbf{0.9864} & \textbf{0.9771} & \textbf{0.8214}\\
      \hline
       LeNet 20\% & \textbf{0.9763} & \textbf{0.9754} & 0.9695 & 0.8928 \\
       BNN 20\% & 0.9728 & 0.9735 & \textbf{0.9701} & \textbf{0.9042}\\
      \bottomrule
    \end{tabular}
  \end{center}
\vspace{-.2cm}
\caption{Classification accuracy on MNIST. We compare the
    LeNet~\cite{lecun1998gradient} implementation that is part of
    Caffe~\cite{jia2014caffe} to the network with the first layer
    replaced by a bilateral convolution layer (BCL). Both are trained
    on the original image resolution (first two rows). Three more BNN
    and CNN models are trained with randomly subsampled images (100\%,
    60\% and 20\% of the pixels). An additional bilinear interpolation
    layer samples the input signal on a spatial grid for the CNN model.
  }
  \label{tab:all-results}
\vspace{-.5cm}
\end{table}

\subsubsection{Image Denoising}

The main application that inspired the development of the bilateral
filtering operation is image denoising~\cite{aurich1995non}, there
using a single Gaussian kernel. Our development allows to learn this
kernel function from data and we explore how to improve using a \emph{single}
but more general bilateral filter.

We use the Berkeley segmentation dataset
(BSDS500)~\cite{arbelaezi2011bsds500} as a test bed. The color
images in the dataset are converted to gray-scale,
and corrupted with Gaussian noise with a standard deviation of
$\frac {25} {255}$.

We compare the performance of four different filter models on a
denoising task.
The first baseline model (`Spatial' in Table \ref{tab:denoising}, $25$
weights) uses a single spatial filter with a kernel size of
$5$ and predicts the scalar gray-scale value at the center pixel. The next model
(`Gauss Bilateral') applies a bilateral \emph{Gaussian}
filter to the noisy input, using position and intensity features $\f=(x,y,v)^\top$.
The third setup (`Learned Bilateral', $65$ weights)
takes a Gauss kernel as initialization and
fits all filter weights on the train set to minimize the
mean squared error with respect to the clean images.
We run a combination
of spatial and permutohedral convolutions on spatial and bilateral
features (`Spatial + Bilateral (Learned)') to check for a complementary
performance of the two convolutions.

\label{sec:exp:denoising}
\begin{table}[!h]
\begin{center}
  \footnotesize
  \begin{tabular}[t]{lr}
    \toprule
    Method & PSNR \\
    \midrule
    Noisy Input & $20.17$ \\
    Spatial & $26.27$ \\
    Gauss Bilateral & $26.51$ \\
    Learned Bilateral & $26.58$ \\
    Spatial + Bilateral (Learned) & \textbf{$26.65$} \\
    \bottomrule
  \end{tabular}
\end{center}
\vspace{-0.5em}
\caption{PSNR results of a denoising task using the BSDS500
  dataset~\cite{arbelaezi2011bsds500}}
\vspace{-0.5em}
\label{tab:denoising}
\end{table}
\vspace{-0.2em}

The PSNR scores evaluated on full images of the test set are
shown in Table \ref{tab:denoising}. We find that an untrained bilateral
filter already performs better than a trained spatial convolution
($26.27$ to $26.51$). A learned convolution further improve the
performance slightly. We chose this simple one-kernel setup to
validate an advantage of the generalized bilateral filter. A competitive
denoising system would employ RGB color information and also
needs to be properly adjusted in network size. Multi-layer perceptrons
have obtained state-of-the-art denoising results~\cite{burger12cvpr}
and the permutohedral lattice layer can readily be used in such an
architecture, which is intended future work.

\subsection{Additional results}
\label{sec:addresults}

This section contains more qualitative results for the experiments presented in Chapter~\ref{chap:bnn}.

\begin{figure*}[th!]
  \centering
    \includegraphics[width=\columnwidth,trim={5cm 2.5cm 5cm 4.5cm},clip]{figures/supplementary/lattice_viz.pdf}
    \vspace{-0.7cm}
  \mycaption{Visualization of the Permutohedral Lattice}
  {Sample lattice visualizations for different feature spaces. All pixels falling in the same simplex cell are shown with
  the same color. $(x,y)$ features correspond to image pixel positions, and $(r,g,b) \in [0,255]$ correspond
  to the red, green and blue color values.}
\label{fig:latticeviz}
\end{figure*}

\subsubsection{Lattice Visualization}

Figure~\ref{fig:latticeviz} shows sample lattice visualizations for different feature spaces.

\newcolumntype{L}[1]{>{\raggedright\let\newline\\\arraybackslash\hspace{0pt}}b{#1}}
\newcolumntype{C}[1]{>{\centering\let\newline\\\arraybackslash\hspace{0pt}}b{#1}}
\newcolumntype{R}[1]{>{\raggedleft\let\newline\\\arraybackslash\hspace{0pt}}b{#1}}

\subsubsection{Color Upsampling}\label{sec:color_upsampling}
\label{sec:col_upsample_extra}

Some images of the upsampling for the Pascal
VOC12 dataset are shown in Fig.~\ref{fig:Colour_upsample_visuals}. It is
especially the low level image details that are better preserved with
a learned bilateral filter compared to the Gaussian case.

\begin{figure*}[t!]
  \centering
    \subfigure{%
   \raisebox{2.0em}{
    \includegraphics[width=.06\columnwidth]{figures/supplementary/2007_004969.jpg}
   }
  }
  \subfigure{%
    \includegraphics[width=.17\columnwidth]{figures/supplementary/2007_004969_gray.pdf}
  }
  \subfigure{%
    \includegraphics[width=.17\columnwidth]{figures/supplementary/2007_004969_gt.pdf}
  }
  \subfigure{%
    \includegraphics[width=.17\columnwidth]{figures/supplementary/2007_004969_bicubic.pdf}
  }
  \subfigure{%
    \includegraphics[width=.17\columnwidth]{figures/supplementary/2007_004969_gauss.pdf}
  }
  \subfigure{%
    \includegraphics[width=.17\columnwidth]{figures/supplementary/2007_004969_learnt.pdf}
  }\\
    \subfigure{%
   \raisebox{2.0em}{
    \includegraphics[width=.06\columnwidth]{figures/supplementary/2007_003106.jpg}
   }
  }
  \subfigure{%
    \includegraphics[width=.17\columnwidth]{figures/supplementary/2007_003106_gray.pdf}
  }
  \subfigure{%
    \includegraphics[width=.17\columnwidth]{figures/supplementary/2007_003106_gt.pdf}
  }
  \subfigure{%
    \includegraphics[width=.17\columnwidth]{figures/supplementary/2007_003106_bicubic.pdf}
  }
  \subfigure{%
    \includegraphics[width=.17\columnwidth]{figures/supplementary/2007_003106_gauss.pdf}
  }
  \subfigure{%
    \includegraphics[width=.17\columnwidth]{figures/supplementary/2007_003106_learnt.pdf}
  }\\
  \setcounter{subfigure}{0}
  \small{
  \subfigure[Inp.]{%
  \raisebox{2.0em}{
    \includegraphics[width=.06\columnwidth]{figures/supplementary/2007_006837.jpg}
   }
  }
  \subfigure[Guidance]{%
    \includegraphics[width=.17\columnwidth]{figures/supplementary/2007_006837_gray.pdf}
  }
   \subfigure[GT]{%
    \includegraphics[width=.17\columnwidth]{figures/supplementary/2007_006837_gt.pdf}
  }
  \subfigure[Bicubic]{%
    \includegraphics[width=.17\columnwidth]{figures/supplementary/2007_006837_bicubic.pdf}
  }
  \subfigure[Gauss-BF]{%
    \includegraphics[width=.17\columnwidth]{figures/supplementary/2007_006837_gauss.pdf}
  }
  \subfigure[Learned-BF]{%
    \includegraphics[width=.17\columnwidth]{figures/supplementary/2007_006837_learnt.pdf}
  }
  }
  \vspace{-0.5cm}
  \mycaption{Color Upsampling}{Color $8\times$ upsampling results
  using different methods, from left to right, (a)~Low-resolution input color image (Inp.),
  (b)~Gray scale guidance image, (c)~Ground-truth color image; Upsampled color images with
  (d)~Bicubic interpolation, (e) Gauss bilateral upsampling and, (f)~Learned bilateral
  updampgling (best viewed on screen).}

\label{fig:Colour_upsample_visuals}
\end{figure*}

\subsubsection{Depth Upsampling}
\label{sec:depth_upsample_extra}

Figure~\ref{fig:depth_upsample_visuals} presents some more qualitative results comparing bicubic interpolation, Gauss
bilateral and learned bilateral upsampling on NYU depth dataset image~\cite{silberman2012indoor}.

\subsubsection{Character Recognition}\label{sec:app_character}

 Figure~\ref{fig:nnrecognition} shows the schematic of different layers
 of the network architecture for LeNet-7~\cite{lecun1998mnist}
 and DeepCNet(5, 50)~\cite{ciresan2012multi,graham2014spatially}. For the BNN variants, the first layer filters are replaced
 with learned bilateral filters and are learned end-to-end.

\subsubsection{Semantic Segmentation}\label{sec:app_semantic_segmentation}
\label{sec:semantic_bnn_extra}

Some more visual results for semantic segmentation are shown in Figure~\ref{fig:semantic_visuals}.
These include the underlying DeepLab CNN\cite{chen2014semantic} result (DeepLab),
the 2 step mean-field result with Gaussian edge potentials (+2stepMF-GaussCRF)
and also corresponding results with learned edge potentials (+2stepMF-LearnedCRF).
In general, we observe that mean-field in learned CRF leads to slightly dilated
classification regions in comparison to using Gaussian CRF thereby filling-in the
false negative pixels and also correcting some mis-classified regions.

\begin{figure*}[t!]
  \centering
    \subfigure{%
   \raisebox{2.0em}{
    \includegraphics[width=.06\columnwidth]{figures/supplementary/2bicubic}
   }
  }
  \subfigure{%
    \includegraphics[width=.17\columnwidth]{figures/supplementary/2given_image}
  }
  \subfigure{%
    \includegraphics[width=.17\columnwidth]{figures/supplementary/2ground_truth}
  }
  \subfigure{%
    \includegraphics[width=.17\columnwidth]{figures/supplementary/2bicubic}
  }
  \subfigure{%
    \includegraphics[width=.17\columnwidth]{figures/supplementary/2gauss}
  }
  \subfigure{%
    \includegraphics[width=.17\columnwidth]{figures/supplementary/2learnt}
  }\\
    \subfigure{%
   \raisebox{2.0em}{
    \includegraphics[width=.06\columnwidth]{figures/supplementary/32bicubic}
   }
  }
  \subfigure{%
    \includegraphics[width=.17\columnwidth]{figures/supplementary/32given_image}
  }
  \subfigure{%
    \includegraphics[width=.17\columnwidth]{figures/supplementary/32ground_truth}
  }
  \subfigure{%
    \includegraphics[width=.17\columnwidth]{figures/supplementary/32bicubic}
  }
  \subfigure{%
    \includegraphics[width=.17\columnwidth]{figures/supplementary/32gauss}
  }
  \subfigure{%
    \includegraphics[width=.17\columnwidth]{figures/supplementary/32learnt}
  }\\
  \setcounter{subfigure}{0}
  \small{
  \subfigure[Inp.]{%
  \raisebox{2.0em}{
    \includegraphics[width=.06\columnwidth]{figures/supplementary/41bicubic}
   }
  }
  \subfigure[Guidance]{%
    \includegraphics[width=.17\columnwidth]{figures/supplementary/41given_image}
  }
   \subfigure[GT]{%
    \includegraphics[width=.17\columnwidth]{figures/supplementary/41ground_truth}
  }
  \subfigure[Bicubic]{%
    \includegraphics[width=.17\columnwidth]{figures/supplementary/41bicubic}
  }
  \subfigure[Gauss-BF]{%
    \includegraphics[width=.17\columnwidth]{figures/supplementary/41gauss}
  }
  \subfigure[Learned-BF]{%
    \includegraphics[width=.17\columnwidth]{figures/supplementary/41learnt}
  }
  }
  \mycaption{Depth Upsampling}{Depth $8\times$ upsampling results
  using different upsampling strategies, from left to right,
  (a)~Low-resolution input depth image (Inp.),
  (b)~High-resolution guidance image, (c)~Ground-truth depth; Upsampled depth images with
  (d)~Bicubic interpolation, (e) Gauss bilateral upsampling and, (f)~Learned bilateral
  updampgling (best viewed on screen).}

\label{fig:depth_upsample_visuals}
\end{figure*}

\subsubsection{Material Segmentation}\label{sec:app_material_segmentation}
\label{sec:material_bnn_extra}

In Fig.~\ref{fig:material_visuals-app2}, we present visual results comparing 2 step
mean-field inference with Gaussian and learned pairwise CRF potentials. In
general, we observe that the pixels belonging to dominant classes in the
training data are being more accurately classified with learned CRF. This leads to
a significant improvements in overall pixel accuracy. This also results
in a slight decrease of the accuracy from less frequent class pixels thereby
slightly reducing the average class accuracy with learning. We attribute this
to the type of annotation that is available for this dataset, which is not
for the entire image but for some segments in the image. We have very few
images of the infrequent classes to combat this behaviour during training.

\subsubsection{Experiment Protocols}
\label{sec:protocols}

Table~\ref{tbl:parameters} shows experiment protocols of different experiments.

 \begin{figure*}[t!]
  \centering
  \subfigure[LeNet-7]{
    \includegraphics[width=0.7\columnwidth]{figures/supplementary/lenet_cnn_network}
    }\\
    \subfigure[DeepCNet]{
    \includegraphics[width=\columnwidth]{figures/supplementary/deepcnet_cnn_network}
    }
  \mycaption{CNNs for Character Recognition}
  {Schematic of (top) LeNet-7~\cite{lecun1998mnist} and (bottom) DeepCNet(5,50)~\cite{ciresan2012multi,graham2014spatially} architectures used in Assamese
  character recognition experiments.}
\label{fig:nnrecognition}
\end{figure*}

\definecolor{voc_1}{RGB}{0, 0, 0}
\definecolor{voc_2}{RGB}{128, 0, 0}
\definecolor{voc_3}{RGB}{0, 128, 0}
\definecolor{voc_4}{RGB}{128, 128, 0}
\definecolor{voc_5}{RGB}{0, 0, 128}
\definecolor{voc_6}{RGB}{128, 0, 128}
\definecolor{voc_7}{RGB}{0, 128, 128}
\definecolor{voc_8}{RGB}{128, 128, 128}
\definecolor{voc_9}{RGB}{64, 0, 0}
\definecolor{voc_10}{RGB}{192, 0, 0}
\definecolor{voc_11}{RGB}{64, 128, 0}
\definecolor{voc_12}{RGB}{192, 128, 0}
\definecolor{voc_13}{RGB}{64, 0, 128}
\definecolor{voc_14}{RGB}{192, 0, 128}
\definecolor{voc_15}{RGB}{64, 128, 128}
\definecolor{voc_16}{RGB}{192, 128, 128}
\definecolor{voc_17}{RGB}{0, 64, 0}
\definecolor{voc_18}{RGB}{128, 64, 0}
\definecolor{voc_19}{RGB}{0, 192, 0}
\definecolor{voc_20}{RGB}{128, 192, 0}
\definecolor{voc_21}{RGB}{0, 64, 128}
\definecolor{voc_22}{RGB}{128, 64, 128}

\begin{figure*}[t]
  \centering
  \small{
  \fcolorbox{white}{voc_1}{\rule{0pt}{6pt}\rule{6pt}{0pt}} Background~~
  \fcolorbox{white}{voc_2}{\rule{0pt}{6pt}\rule{6pt}{0pt}} Aeroplane~~
  \fcolorbox{white}{voc_3}{\rule{0pt}{6pt}\rule{6pt}{0pt}} Bicycle~~
  \fcolorbox{white}{voc_4}{\rule{0pt}{6pt}\rule{6pt}{0pt}} Bird~~
  \fcolorbox{white}{voc_5}{\rule{0pt}{6pt}\rule{6pt}{0pt}} Boat~~
  \fcolorbox{white}{voc_6}{\rule{0pt}{6pt}\rule{6pt}{0pt}} Bottle~~
  \fcolorbox{white}{voc_7}{\rule{0pt}{6pt}\rule{6pt}{0pt}} Bus~~
  \fcolorbox{white}{voc_8}{\rule{0pt}{6pt}\rule{6pt}{0pt}} Car~~ \\
  \fcolorbox{white}{voc_9}{\rule{0pt}{6pt}\rule{6pt}{0pt}} Cat~~
  \fcolorbox{white}{voc_10}{\rule{0pt}{6pt}\rule{6pt}{0pt}} Chair~~
  \fcolorbox{white}{voc_11}{\rule{0pt}{6pt}\rule{6pt}{0pt}} Cow~~
  \fcolorbox{white}{voc_12}{\rule{0pt}{6pt}\rule{6pt}{0pt}} Dining Table~~
  \fcolorbox{white}{voc_13}{\rule{0pt}{6pt}\rule{6pt}{0pt}} Dog~~
  \fcolorbox{white}{voc_14}{\rule{0pt}{6pt}\rule{6pt}{0pt}} Horse~~
  \fcolorbox{white}{voc_15}{\rule{0pt}{6pt}\rule{6pt}{0pt}} Motorbike~~
  \fcolorbox{white}{voc_16}{\rule{0pt}{6pt}\rule{6pt}{0pt}} Person~~ \\
  \fcolorbox{white}{voc_17}{\rule{0pt}{6pt}\rule{6pt}{0pt}} Potted Plant~~
  \fcolorbox{white}{voc_18}{\rule{0pt}{6pt}\rule{6pt}{0pt}} Sheep~~
  \fcolorbox{white}{voc_19}{\rule{0pt}{6pt}\rule{6pt}{0pt}} Sofa~~
  \fcolorbox{white}{voc_20}{\rule{0pt}{6pt}\rule{6pt}{0pt}} Train~~
  \fcolorbox{white}{voc_21}{\rule{0pt}{6pt}\rule{6pt}{0pt}} TV monitor~~ \\
  }
  \subfigure{%
    \includegraphics[width=.18\columnwidth]{figures/supplementary/2007_001423_given.jpg}
  }
  \subfigure{%
    \includegraphics[width=.18\columnwidth]{figures/supplementary/2007_001423_gt.png}
  }
  \subfigure{%
    \includegraphics[width=.18\columnwidth]{figures/supplementary/2007_001423_cnn.png}
  }
  \subfigure{%
    \includegraphics[width=.18\columnwidth]{figures/supplementary/2007_001423_gauss.png}
  }
  \subfigure{%
    \includegraphics[width=.18\columnwidth]{figures/supplementary/2007_001423_learnt.png}
  }\\
  \subfigure{%
    \includegraphics[width=.18\columnwidth]{figures/supplementary/2007_001430_given.jpg}
  }
  \subfigure{%
    \includegraphics[width=.18\columnwidth]{figures/supplementary/2007_001430_gt.png}
  }
  \subfigure{%
    \includegraphics[width=.18\columnwidth]{figures/supplementary/2007_001430_cnn.png}
  }
  \subfigure{%
    \includegraphics[width=.18\columnwidth]{figures/supplementary/2007_001430_gauss.png}
  }
  \subfigure{%
    \includegraphics[width=.18\columnwidth]{figures/supplementary/2007_001430_learnt.png}
  }\\
    \subfigure{%
    \includegraphics[width=.18\columnwidth]{figures/supplementary/2007_007996_given.jpg}
  }
  \subfigure{%
    \includegraphics[width=.18\columnwidth]{figures/supplementary/2007_007996_gt.png}
  }
  \subfigure{%
    \includegraphics[width=.18\columnwidth]{figures/supplementary/2007_007996_cnn.png}
  }
  \subfigure{%
    \includegraphics[width=.18\columnwidth]{figures/supplementary/2007_007996_gauss.png}
  }
  \subfigure{%
    \includegraphics[width=.18\columnwidth]{figures/supplementary/2007_007996_learnt.png}
  }\\
   \subfigure{%
    \includegraphics[width=.18\columnwidth]{figures/supplementary/2010_002682_given.jpg}
  }
  \subfigure{%
    \includegraphics[width=.18\columnwidth]{figures/supplementary/2010_002682_gt.png}
  }
  \subfigure{%
    \includegraphics[width=.18\columnwidth]{figures/supplementary/2010_002682_cnn.png}
  }
  \subfigure{%
    \includegraphics[width=.18\columnwidth]{figures/supplementary/2010_002682_gauss.png}
  }
  \subfigure{%
    \includegraphics[width=.18\columnwidth]{figures/supplementary/2010_002682_learnt.png}
  }\\
     \subfigure{%
    \includegraphics[width=.18\columnwidth]{figures/supplementary/2010_004789_given.jpg}
  }
  \subfigure{%
    \includegraphics[width=.18\columnwidth]{figures/supplementary/2010_004789_gt.png}
  }
  \subfigure{%
    \includegraphics[width=.18\columnwidth]{figures/supplementary/2010_004789_cnn.png}
  }
  \subfigure{%
    \includegraphics[width=.18\columnwidth]{figures/supplementary/2010_004789_gauss.png}
  }
  \subfigure{%
    \includegraphics[width=.18\columnwidth]{figures/supplementary/2010_004789_learnt.png}
  }\\
       \subfigure{%
    \includegraphics[width=.18\columnwidth]{figures/supplementary/2007_001311_given.jpg}
  }
  \subfigure{%
    \includegraphics[width=.18\columnwidth]{figures/supplementary/2007_001311_gt.png}
  }
  \subfigure{%
    \includegraphics[width=.18\columnwidth]{figures/supplementary/2007_001311_cnn.png}
  }
  \subfigure{%
    \includegraphics[width=.18\columnwidth]{figures/supplementary/2007_001311_gauss.png}
  }
  \subfigure{%
    \includegraphics[width=.18\columnwidth]{figures/supplementary/2007_001311_learnt.png}
  }\\
  \setcounter{subfigure}{0}
  \subfigure[Input]{%
    \includegraphics[width=.18\columnwidth]{figures/supplementary/2010_003531_given.jpg}
  }
  \subfigure[Ground Truth]{%
    \includegraphics[width=.18\columnwidth]{figures/supplementary/2010_003531_gt.png}
  }
  \subfigure[DeepLab]{%
    \includegraphics[width=.18\columnwidth]{figures/supplementary/2010_003531_cnn.png}
  }
  \subfigure[+GaussCRF]{%
    \includegraphics[width=.18\columnwidth]{figures/supplementary/2010_003531_gauss.png}
  }
  \subfigure[+LearnedCRF]{%
    \includegraphics[width=.18\columnwidth]{figures/supplementary/2010_003531_learnt.png}
  }
  \vspace{-0.3cm}
  \mycaption{Semantic Segmentation}{Example results of semantic segmentation.
  (c)~depicts the unary results before application of MF, (d)~after two steps of MF with Gaussian edge CRF potentials, (e)~after
  two steps of MF with learned edge CRF potentials.}
    \label{fig:semantic_visuals}
\end{figure*}


\definecolor{minc_1}{HTML}{771111}
\definecolor{minc_2}{HTML}{CAC690}
\definecolor{minc_3}{HTML}{EEEEEE}
\definecolor{minc_4}{HTML}{7C8FA6}
\definecolor{minc_5}{HTML}{597D31}
\definecolor{minc_6}{HTML}{104410}
\definecolor{minc_7}{HTML}{BB819C}
\definecolor{minc_8}{HTML}{D0CE48}
\definecolor{minc_9}{HTML}{622745}
\definecolor{minc_10}{HTML}{666666}
\definecolor{minc_11}{HTML}{D54A31}
\definecolor{minc_12}{HTML}{101044}
\definecolor{minc_13}{HTML}{444126}
\definecolor{minc_14}{HTML}{75D646}
\definecolor{minc_15}{HTML}{DD4348}
\definecolor{minc_16}{HTML}{5C8577}
\definecolor{minc_17}{HTML}{C78472}
\definecolor{minc_18}{HTML}{75D6D0}
\definecolor{minc_19}{HTML}{5B4586}
\definecolor{minc_20}{HTML}{C04393}
\definecolor{minc_21}{HTML}{D69948}
\definecolor{minc_22}{HTML}{7370D8}
\definecolor{minc_23}{HTML}{7A3622}
\definecolor{minc_24}{HTML}{000000}

\begin{figure*}[t]
  \centering
  \small{
  \fcolorbox{white}{minc_1}{\rule{0pt}{6pt}\rule{6pt}{0pt}} Brick~~
  \fcolorbox{white}{minc_2}{\rule{0pt}{6pt}\rule{6pt}{0pt}} Carpet~~
  \fcolorbox{white}{minc_3}{\rule{0pt}{6pt}\rule{6pt}{0pt}} Ceramic~~
  \fcolorbox{white}{minc_4}{\rule{0pt}{6pt}\rule{6pt}{0pt}} Fabric~~
  \fcolorbox{white}{minc_5}{\rule{0pt}{6pt}\rule{6pt}{0pt}} Foliage~~
  \fcolorbox{white}{minc_6}{\rule{0pt}{6pt}\rule{6pt}{0pt}} Food~~
  \fcolorbox{white}{minc_7}{\rule{0pt}{6pt}\rule{6pt}{0pt}} Glass~~
  \fcolorbox{white}{minc_8}{\rule{0pt}{6pt}\rule{6pt}{0pt}} Hair~~ \\
  \fcolorbox{white}{minc_9}{\rule{0pt}{6pt}\rule{6pt}{0pt}} Leather~~
  \fcolorbox{white}{minc_10}{\rule{0pt}{6pt}\rule{6pt}{0pt}} Metal~~
  \fcolorbox{white}{minc_11}{\rule{0pt}{6pt}\rule{6pt}{0pt}} Mirror~~
  \fcolorbox{white}{minc_12}{\rule{0pt}{6pt}\rule{6pt}{0pt}} Other~~
  \fcolorbox{white}{minc_13}{\rule{0pt}{6pt}\rule{6pt}{0pt}} Painted~~
  \fcolorbox{white}{minc_14}{\rule{0pt}{6pt}\rule{6pt}{0pt}} Paper~~
  \fcolorbox{white}{minc_15}{\rule{0pt}{6pt}\rule{6pt}{0pt}} Plastic~~\\
  \fcolorbox{white}{minc_16}{\rule{0pt}{6pt}\rule{6pt}{0pt}} Polished Stone~~
  \fcolorbox{white}{minc_17}{\rule{0pt}{6pt}\rule{6pt}{0pt}} Skin~~
  \fcolorbox{white}{minc_18}{\rule{0pt}{6pt}\rule{6pt}{0pt}} Sky~~
  \fcolorbox{white}{minc_19}{\rule{0pt}{6pt}\rule{6pt}{0pt}} Stone~~
  \fcolorbox{white}{minc_20}{\rule{0pt}{6pt}\rule{6pt}{0pt}} Tile~~
  \fcolorbox{white}{minc_21}{\rule{0pt}{6pt}\rule{6pt}{0pt}} Wallpaper~~
  \fcolorbox{white}{minc_22}{\rule{0pt}{6pt}\rule{6pt}{0pt}} Water~~
  \fcolorbox{white}{minc_23}{\rule{0pt}{6pt}\rule{6pt}{0pt}} Wood~~ \\
  }
  \subfigure{%
    \includegraphics[width=.18\columnwidth]{figures/supplementary/000010868_given.jpg}
  }
  \subfigure{%
    \includegraphics[width=.18\columnwidth]{figures/supplementary/000010868_gt.png}
  }
  \subfigure{%
    \includegraphics[width=.18\columnwidth]{figures/supplementary/000010868_cnn.png}
  }
  \subfigure{%
    \includegraphics[width=.18\columnwidth]{figures/supplementary/000010868_gauss.png}
  }
  \subfigure{%
    \includegraphics[width=.18\columnwidth]{figures/supplementary/000010868_learnt.png}
  }\\[-2ex]
  \subfigure{%
    \includegraphics[width=.18\columnwidth]{figures/supplementary/000006011_given.jpg}
  }
  \subfigure{%
    \includegraphics[width=.18\columnwidth]{figures/supplementary/000006011_gt.png}
  }
  \subfigure{%
    \includegraphics[width=.18\columnwidth]{figures/supplementary/000006011_cnn.png}
  }
  \subfigure{%
    \includegraphics[width=.18\columnwidth]{figures/supplementary/000006011_gauss.png}
  }
  \subfigure{%
    \includegraphics[width=.18\columnwidth]{figures/supplementary/000006011_learnt.png}
  }\\[-2ex]
    \subfigure{%
    \includegraphics[width=.18\columnwidth]{figures/supplementary/000008553_given.jpg}
  }
  \subfigure{%
    \includegraphics[width=.18\columnwidth]{figures/supplementary/000008553_gt.png}
  }
  \subfigure{%
    \includegraphics[width=.18\columnwidth]{figures/supplementary/000008553_cnn.png}
  }
  \subfigure{%
    \includegraphics[width=.18\columnwidth]{figures/supplementary/000008553_gauss.png}
  }
  \subfigure{%
    \includegraphics[width=.18\columnwidth]{figures/supplementary/000008553_learnt.png}
  }\\[-2ex]
   \subfigure{%
    \includegraphics[width=.18\columnwidth]{figures/supplementary/000009188_given.jpg}
  }
  \subfigure{%
    \includegraphics[width=.18\columnwidth]{figures/supplementary/000009188_gt.png}
  }
  \subfigure{%
    \includegraphics[width=.18\columnwidth]{figures/supplementary/000009188_cnn.png}
  }
  \subfigure{%
    \includegraphics[width=.18\columnwidth]{figures/supplementary/000009188_gauss.png}
  }
  \subfigure{%
    \includegraphics[width=.18\columnwidth]{figures/supplementary/000009188_learnt.png}
  }\\[-2ex]
  \setcounter{subfigure}{0}
  \subfigure[Input]{%
    \includegraphics[width=.18\columnwidth]{figures/supplementary/000023570_given.jpg}
  }
  \subfigure[Ground Truth]{%
    \includegraphics[width=.18\columnwidth]{figures/supplementary/000023570_gt.png}
  }
  \subfigure[DeepLab]{%
    \includegraphics[width=.18\columnwidth]{figures/supplementary/000023570_cnn.png}
  }
  \subfigure[+GaussCRF]{%
    \includegraphics[width=.18\columnwidth]{figures/supplementary/000023570_gauss.png}
  }
  \subfigure[+LearnedCRF]{%
    \includegraphics[width=.18\columnwidth]{figures/supplementary/000023570_learnt.png}
  }
  \mycaption{Material Segmentation}{Example results of material segmentation.
  (c)~depicts the unary results before application of MF, (d)~after two steps of MF with Gaussian edge CRF potentials, (e)~after two steps of MF with learned edge CRF potentials.}
    \label{fig:material_visuals-app2}
\end{figure*}


\begin{table*}[h]
\tiny
  \centering
    \begin{tabular}{L{2.3cm} L{2.25cm} C{1.5cm} C{0.7cm} C{0.6cm} C{0.7cm} C{0.7cm} C{0.7cm} C{1.6cm} C{0.6cm} C{0.6cm} C{0.6cm}}
      \toprule
& & & & & \multicolumn{3}{c}{\textbf{Data Statistics}} & \multicolumn{4}{c}{\textbf{Training Protocol}} \\

\textbf{Experiment} & \textbf{Feature Types} & \textbf{Feature Scales} & \textbf{Filter Size} & \textbf{Filter Nbr.} & \textbf{Train}  & \textbf{Val.} & \textbf{Test} & \textbf{Loss Type} & \textbf{LR} & \textbf{Batch} & \textbf{Epochs} \\
      \midrule
      \multicolumn{2}{c}{\textbf{Single Bilateral Filter Applications}} & & & & & & & & & \\
      \textbf{2$\times$ Color Upsampling} & Position$_{1}$, Intensity (3D) & 0.13, 0.17 & 65 & 2 & 10581 & 1449 & 1456 & MSE & 1e-06 & 200 & 94.5\\
      \textbf{4$\times$ Color Upsampling} & Position$_{1}$, Intensity (3D) & 0.06, 0.17 & 65 & 2 & 10581 & 1449 & 1456 & MSE & 1e-06 & 200 & 94.5\\
      \textbf{8$\times$ Color Upsampling} & Position$_{1}$, Intensity (3D) & 0.03, 0.17 & 65 & 2 & 10581 & 1449 & 1456 & MSE & 1e-06 & 200 & 94.5\\
      \textbf{16$\times$ Color Upsampling} & Position$_{1}$, Intensity (3D) & 0.02, 0.17 & 65 & 2 & 10581 & 1449 & 1456 & MSE & 1e-06 & 200 & 94.5\\
      \textbf{Depth Upsampling} & Position$_{1}$, Color (5D) & 0.05, 0.02 & 665 & 2 & 795 & 100 & 654 & MSE & 1e-07 & 50 & 251.6\\
      \textbf{Mesh Denoising} & Isomap (4D) & 46.00 & 63 & 2 & 1000 & 200 & 500 & MSE & 100 & 10 & 100.0 \\
      \midrule
      \multicolumn{2}{c}{\textbf{DenseCRF Applications}} & & & & & & & & &\\
      \multicolumn{2}{l}{\textbf{Semantic Segmentation}} & & & & & & & & &\\
      \textbf{- 1step MF} & Position$_{1}$, Color (5D); Position$_{1}$ (2D) & 0.01, 0.34; 0.34  & 665; 19  & 2; 2 & 10581 & 1449 & 1456 & Logistic & 0.1 & 5 & 1.4 \\
      \textbf{- 2step MF} & Position$_{1}$, Color (5D); Position$_{1}$ (2D) & 0.01, 0.34; 0.34 & 665; 19 & 2; 2 & 10581 & 1449 & 1456 & Logistic & 0.1 & 5 & 1.4 \\
      \textbf{- \textit{loose} 2step MF} & Position$_{1}$, Color (5D); Position$_{1}$ (2D) & 0.01, 0.34; 0.34 & 665; 19 & 2; 2 &10581 & 1449 & 1456 & Logistic & 0.1 & 5 & +1.9  \\ \\
      \multicolumn{2}{l}{\textbf{Material Segmentation}} & & & & & & & & &\\
      \textbf{- 1step MF} & Position$_{2}$, Lab-Color (5D) & 5.00, 0.05, 0.30  & 665 & 2 & 928 & 150 & 1798 & Weighted Logistic & 1e-04 & 24 & 2.6 \\
      \textbf{- 2step MF} & Position$_{2}$, Lab-Color (5D) & 5.00, 0.05, 0.30 & 665 & 2 & 928 & 150 & 1798 & Weighted Logistic & 1e-04 & 12 & +0.7 \\
      \textbf{- \textit{loose} 2step MF} & Position$_{2}$, Lab-Color (5D) & 5.00, 0.05, 0.30 & 665 & 2 & 928 & 150 & 1798 & Weighted Logistic & 1e-04 & 12 & +0.2\\
      \midrule
      \multicolumn{2}{c}{\textbf{Neural Network Applications}} & & & & & & & & &\\
      \textbf{Tiles: CNN-9$\times$9} & - & - & 81 & 4 & 10000 & 1000 & 1000 & Logistic & 0.01 & 100 & 500.0 \\
      \textbf{Tiles: CNN-13$\times$13} & - & - & 169 & 6 & 10000 & 1000 & 1000 & Logistic & 0.01 & 100 & 500.0 \\
      \textbf{Tiles: CNN-17$\times$17} & - & - & 289 & 8 & 10000 & 1000 & 1000 & Logistic & 0.01 & 100 & 500.0 \\
      \textbf{Tiles: CNN-21$\times$21} & - & - & 441 & 10 & 10000 & 1000 & 1000 & Logistic & 0.01 & 100 & 500.0 \\
      \textbf{Tiles: BNN} & Position$_{1}$, Color (5D) & 0.05, 0.04 & 63 & 1 & 10000 & 1000 & 1000 & Logistic & 0.01 & 100 & 30.0 \\
      \textbf{LeNet} & - & - & 25 & 2 & 5490 & 1098 & 1647 & Logistic & 0.1 & 100 & 182.2 \\
      \textbf{Crop-LeNet} & - & - & 25 & 2 & 5490 & 1098 & 1647 & Logistic & 0.1 & 100 & 182.2 \\
      \textbf{BNN-LeNet} & Position$_{2}$ (2D) & 20.00 & 7 & 1 & 5490 & 1098 & 1647 & Logistic & 0.1 & 100 & 182.2 \\
      \textbf{DeepCNet} & - & - & 9 & 1 & 5490 & 1098 & 1647 & Logistic & 0.1 & 100 & 182.2 \\
      \textbf{Crop-DeepCNet} & - & - & 9 & 1 & 5490 & 1098 & 1647 & Logistic & 0.1 & 100 & 182.2 \\
      \textbf{BNN-DeepCNet} & Position$_{2}$ (2D) & 40.00  & 7 & 1 & 5490 & 1098 & 1647 & Logistic & 0.1 & 100 & 182.2 \\
      \bottomrule
      \\
    \end{tabular}
    \mycaption{Experiment Protocols} {Experiment protocols for the different experiments presented in this work. \textbf{Feature Types}:
    Feature spaces used for the bilateral convolutions. Position$_1$ corresponds to un-normalized pixel positions whereas Position$_2$ corresponds
    to pixel positions normalized to $[0,1]$ with respect to the given image. \textbf{Feature Scales}: Cross-validated scales for the features used.
     \textbf{Filter Size}: Number of elements in the filter that is being learned. \textbf{Filter Nbr.}: Half-width of the filter. \textbf{Train},
     \textbf{Val.} and \textbf{Test} corresponds to the number of train, validation and test images used in the experiment. \textbf{Loss Type}: Type
     of loss used for back-propagation. ``MSE'' corresponds to Euclidean mean squared error loss and ``Logistic'' corresponds to multinomial logistic
     loss. ``Weighted Logistic'' is the class-weighted multinomial logistic loss. We weighted the loss with inverse class probability for material
     segmentation task due to the small availability of training data with class imbalance. \textbf{LR}: Fixed learning rate used in stochastic gradient
     descent. \textbf{Batch}: Number of images used in one parameter update step. \textbf{Epochs}: Number of training epochs. In all the experiments,
     we used fixed momentum of 0.9 and weight decay of 0.0005 for stochastic gradient descent. ```Color Upsampling'' experiments in this Table corresponds
     to those performed on Pascal VOC12 dataset images. For all experiments using Pascal VOC12 images, we use extended
     training segmentation dataset available from~\cite{hariharan2011moredata}, and used standard validation and test splits
     from the main dataset~\cite{voc2012segmentation}.}
  \label{tbl:parameters}
\end{table*}

\clearpage

\section{Parameters and Additional Results for Video Propagation Networks}

In this Section, we present experiment protocols and additional qualitative results for experiments
on video object segmentation, semantic video segmentation and video color
propagation. Table~\ref{tbl:parameters_supp} shows the feature scales and other parameters used in different experiments.
Figures~\ref{fig:video_seg_pos_supp} show some qualitative results on video object segmentation
with some failure cases in Fig.~\ref{fig:video_seg_neg_supp}.
Figure~\ref{fig:semantic_visuals_supp} shows some qualitative results on semantic video segmentation and
Fig.~\ref{fig:color_visuals_supp} shows results on video color propagation.

\newcolumntype{L}[1]{>{\raggedright\let\newline\\\arraybackslash\hspace{0pt}}b{#1}}
\newcolumntype{C}[1]{>{\centering\let\newline\\\arraybackslash\hspace{0pt}}b{#1}}
\newcolumntype{R}[1]{>{\raggedleft\let\newline\\\arraybackslash\hspace{0pt}}b{#1}}

\begin{table*}[h]
\tiny
  \centering
    \begin{tabular}{L{3.0cm} L{2.4cm} L{2.8cm} L{2.8cm} C{0.5cm} C{1.0cm} L{1.2cm}}
      \toprule
\textbf{Experiment} & \textbf{Feature Type} & \textbf{Feature Scale-1, $\Lambda_a$} & \textbf{Feature Scale-2, $\Lambda_b$} & \textbf{$\alpha$} & \textbf{Input Frames} & \textbf{Loss Type} \\
      \midrule
      \textbf{Video Object Segmentation} & ($x,y,Y,Cb,Cr,t$) & (0.02,0.02,0.07,0.4,0.4,0.01) & (0.03,0.03,0.09,0.5,0.5,0.2) & 0.5 & 9 & Logistic\\
      \midrule
      \textbf{Semantic Video Segmentation} & & & & & \\
      \textbf{with CNN1~\cite{yu2015multi}-NoFlow} & ($x,y,R,G,B,t$) & (0.08,0.08,0.2,0.2,0.2,0.04) & (0.11,0.11,0.2,0.2,0.2,0.04) & 0.5 & 3 & Logistic \\
      \textbf{with CNN1~\cite{yu2015multi}-Flow} & ($x+u_x,y+u_y,R,G,B,t$) & (0.11,0.11,0.14,0.14,0.14,0.03) & (0.08,0.08,0.12,0.12,0.12,0.01) & 0.65 & 3 & Logistic\\
      \textbf{with CNN2~\cite{richter2016playing}-Flow} & ($x+u_x,y+u_y,R,G,B,t$) & (0.08,0.08,0.2,0.2,0.2,0.04) & (0.09,0.09,0.25,0.25,0.25,0.03) & 0.5 & 4 & Logistic\\
      \midrule
      \textbf{Video Color Propagation} & ($x,y,I,t$)  & (0.04,0.04,0.2,0.04) & No second kernel & 1 & 4 & MSE\\
      \bottomrule
      \\
    \end{tabular}
    \mycaption{Experiment Protocols} {Experiment protocols for the different experiments presented in this work. \textbf{Feature Types}:
    Feature spaces used for the bilateral convolutions, with position ($x,y$) and color
    ($R,G,B$ or $Y,Cb,Cr$) features $\in [0,255]$. $u_x$, $u_y$ denotes optical flow with respect
    to the present frame and $I$ denotes grayscale intensity.
    \textbf{Feature Scales ($\Lambda_a, \Lambda_b$)}: Cross-validated scales for the features used.
    \textbf{$\alpha$}: Exponential time decay for the input frames.
    \textbf{Input Frames}: Number of input frames for VPN.
    \textbf{Loss Type}: Type
     of loss used for back-propagation. ``MSE'' corresponds to Euclidean mean squared error loss and ``Logistic'' corresponds to multinomial logistic loss.}
  \label{tbl:parameters_supp}
\end{table*}

% \begin{figure}[th!]
% \begin{center}
%   \centerline{\includegraphics[width=\textwidth]{figures/video_seg_visuals_supp_small.pdf}}
%     \mycaption{Video Object Segmentation}
%     {Shown are the different frames in example videos with the corresponding
%     ground truth (GT) masks, predictions from BVS~\cite{marki2016bilateral},
%     OFL~\cite{tsaivideo}, VPN (VPN-Stage2) and VPN-DLab (VPN-DeepLab) models.}
%     \label{fig:video_seg_small_supp}
% \end{center}
% \vspace{-1.0cm}
% \end{figure}

\begin{figure}[th!]
\begin{center}
  \centerline{\includegraphics[width=0.7\textwidth]{figures/video_seg_visuals_supp_positive.pdf}}
    \mycaption{Video Object Segmentation}
    {Shown are the different frames in example videos with the corresponding
    ground truth (GT) masks, predictions from BVS~\cite{marki2016bilateral},
    OFL~\cite{tsaivideo}, VPN (VPN-Stage2) and VPN-DLab (VPN-DeepLab) models.}
    \label{fig:video_seg_pos_supp}
\end{center}
\vspace{-1.0cm}
\end{figure}

\begin{figure}[th!]
\begin{center}
  \centerline{\includegraphics[width=0.7\textwidth]{figures/video_seg_visuals_supp_negative.pdf}}
    \mycaption{Failure Cases for Video Object Segmentation}
    {Shown are the different frames in example videos with the corresponding
    ground truth (GT) masks, predictions from BVS~\cite{marki2016bilateral},
    OFL~\cite{tsaivideo}, VPN (VPN-Stage2) and VPN-DLab (VPN-DeepLab) models.}
    \label{fig:video_seg_neg_supp}
\end{center}
\vspace{-1.0cm}
\end{figure}

\begin{figure}[th!]
\begin{center}
  \centerline{\includegraphics[width=0.9\textwidth]{figures/supp_semantic_visual.pdf}}
    \mycaption{Semantic Video Segmentation}
    {Input video frames and the corresponding ground truth (GT)
    segmentation together with the predictions of CNN~\cite{yu2015multi} and with
    VPN-Flow.}
    \label{fig:semantic_visuals_supp}
\end{center}
\vspace{-0.7cm}
\end{figure}

\begin{figure}[th!]
\begin{center}
  \centerline{\includegraphics[width=\textwidth]{figures/colorization_visuals_supp.pdf}}
  \mycaption{Video Color Propagation}
  {Input grayscale video frames and corresponding ground-truth (GT) color images
  together with color predictions of Levin et al.~\cite{levin2004colorization} and VPN-Stage1 models.}
  \label{fig:color_visuals_supp}
\end{center}
\vspace{-0.7cm}
\end{figure}

\clearpage

\section{Additional Material for Bilateral Inception Networks}
\label{sec:binception-app}

In this section of the Appendix, we first discuss the use of approximate bilateral
filtering in BI modules (Sec.~\ref{sec:lattice}).
Later, we present some qualitative results using different models for the approach presented in
Chapter~\ref{chap:binception} (Sec.~\ref{sec:qualitative-app}).

\subsection{Approximate Bilateral Filtering}
\label{sec:lattice}

The bilateral inception module presented in Chapter~\ref{chap:binception} computes a matrix-vector
product between a Gaussian filter $K$ and a vector of activations $\bz_c$.
Bilateral filtering is an important operation and many algorithmic techniques have been
proposed to speed-up this operation~\cite{paris2006fast,adams2010fast,gastal2011domain}.
In the main paper we opted to implement what can be considered the
brute-force variant of explicitly constructing $K$ and then using BLAS to compute the
matrix-vector product. This resulted in a few millisecond operation.
The explicit way to compute is possible due to the
reduction to super-pixels, e.g., it would not work for DenseCRF variants
that operate on the full image resolution.

Here, we present experiments where we use the fast approximate bilateral filtering
algorithm of~\cite{adams2010fast}, which is also used in Chapter~\ref{chap:bnn}
for learning sparse high dimensional filters. This
choice allows for larger dimensions of matrix-vector multiplication. The reason for choosing
the explicit multiplication in Chapter~\ref{chap:binception} was that it was computationally faster.
For the small sizes of the involved matrices and vectors, the explicit computation is sufficient and we had no
GPU implementation of an approximate technique that matched this runtime. Also it
is conceptually easier and the gradient to the feature transformations ($\Lambda \mathbf{f}$) is
obtained using standard matrix calculus.

\subsubsection{Experiments}

We modified the existing segmentation architectures analogous to those in Chapter~\ref{chap:binception}.
The main difference is that, here, the inception modules use the lattice
approximation~\cite{adams2010fast} to compute the bilateral filtering.
Using the lattice approximation did not allow us to back-propagate through feature transformations ($\Lambda$)
and thus we used hand-specified feature scales as will be explained later.
Specifically, we take CNN architectures from the works
of~\cite{chen2014semantic,zheng2015conditional,bell2015minc} and insert the BI modules between
the spatial FC layers.
We use superpixels from~\cite{DollarICCV13edges}
for all the experiments with the lattice approximation. Experiments are
performed using Caffe neural network framework~\cite{jia2014caffe}.

\begin{table}
  \small
  \centering
  \begin{tabular}{p{5.5cm}>{\raggedright\arraybackslash}p{1.4cm}>{\centering\arraybackslash}p{2.2cm}}
    \toprule
		\textbf{Model} & \emph{IoU} & \emph{Runtime}(ms) \\
    \midrule

    %%%%%%%%%%%% Scores computed by us)%%%%%%%%%%%%
		\deeplablargefov & 68.9 & 145ms\\
    \midrule
    \bi{7}{2}-\bi{8}{10}& \textbf{73.8} & +600 \\
    \midrule
    \deeplablargefovcrf~\cite{chen2014semantic} & 72.7 & +830\\
    \deeplabmsclargefovcrf~\cite{chen2014semantic} & \textbf{73.6} & +880\\
    DeepLab-EdgeNet~\cite{chen2015semantic} & 71.7 & +30\\
    DeepLab-EdgeNet-CRF~\cite{chen2015semantic} & \textbf{73.6} & +860\\
  \bottomrule \\
  \end{tabular}
  \mycaption{Semantic Segmentation using the DeepLab model}
  {IoU scores on the Pascal VOC12 segmentation test dataset
  with different models and our modified inception model.
  Also shown are the corresponding runtimes in milliseconds. Runtimes
  also include superpixel computations (300 ms with Dollar superpixels~\cite{DollarICCV13edges})}
  \label{tab:largefovresults}
\end{table}

\paragraph{Semantic Segmentation}
The experiments in this section use the Pascal VOC12 segmentation dataset~\cite{voc2012segmentation} with 21 object classes and the images have a maximum resolution of 0.25 megapixels.
For all experiments on VOC12, we train using the extended training set of
10581 images collected by~\cite{hariharan2011moredata}.
We modified the \deeplab~network architecture of~\cite{chen2014semantic} and
the CRFasRNN architecture from~\cite{zheng2015conditional} which uses a CNN with
deconvolution layers followed by DenseCRF trained end-to-end.

\paragraph{DeepLab Model}\label{sec:deeplabmodel}
We experimented with the \bi{7}{2}-\bi{8}{10} inception model.
Results using the~\deeplab~model are summarized in Tab.~\ref{tab:largefovresults}.
Although we get similar improvements with inception modules as with the
explicit kernel computation, using lattice approximation is slower.

\begin{table}
  \small
  \centering
  \begin{tabular}{p{6.4cm}>{\raggedright\arraybackslash}p{1.8cm}>{\raggedright\arraybackslash}p{1.8cm}}
    \toprule
    \textbf{Model} & \emph{IoU (Val)} & \emph{IoU (Test)}\\
    \midrule
    %%%%%%%%%%%% Scores computed by us)%%%%%%%%%%%%
    CNN &  67.5 & - \\
    \deconv (CNN+Deconvolutions) & 69.8 & 72.0 \\
    \midrule
    \bi{3}{6}-\bi{4}{6}-\bi{7}{2}-\bi{8}{6}& 71.9 & - \\
    \bi{3}{6}-\bi{4}{6}-\bi{7}{2}-\bi{8}{6}-\gi{6}& 73.6 &  \href{http://host.robots.ox.ac.uk:8080/anonymous/VOTV5E.html}{\textbf{75.2}}\\
    \midrule
    \deconvcrf (CRF-RNN)~\cite{zheng2015conditional} & 73.0 & 74.7\\
    Context-CRF-RNN~\cite{yu2015multi} & ~~ - ~ & \textbf{75.3} \\
    \bottomrule \\
  \end{tabular}
  \mycaption{Semantic Segmentation using the CRFasRNN model}{IoU score corresponding to different models
  on Pascal VOC12 reduced validation / test segmentation dataset. The reduced validation set consists of 346 images
  as used in~\cite{zheng2015conditional} where we adapted the model from.}
  \label{tab:deconvresults-app}
\end{table}

\paragraph{CRFasRNN Model}\label{sec:deepinception}
We add BI modules after score-pool3, score-pool4, \fc{7} and \fc{8} $1\times1$ convolution layers
resulting in the \bi{3}{6}-\bi{4}{6}-\bi{7}{2}-\bi{8}{6}
model and also experimented with another variant where $BI_8$ is followed by another inception
module, G$(6)$, with 6 Gaussian kernels.
Note that here also we discarded both deconvolution and DenseCRF parts of the original model~\cite{zheng2015conditional}
and inserted the BI modules in the base CNN and found similar improvements compared to the inception modules with explicit
kernel computaion. See Tab.~\ref{tab:deconvresults-app} for results on the CRFasRNN model.

\paragraph{Material Segmentation}
Table~\ref{tab:mincresults-app} shows the results on the MINC dataset~\cite{bell2015minc}
obtained by modifying the AlexNet architecture with our inception modules. We observe
similar improvements as with explicit kernel construction.
For this model, we do not provide any learned setup due to very limited segment training
data. The weights to combine outputs in the bilateral inception layer are
found by validation on the validation set.

\begin{table}[t]
  \small
  \centering
  \begin{tabular}{p{3.5cm}>{\centering\arraybackslash}p{4.0cm}}
    \toprule
    \textbf{Model} & Class / Total accuracy\\
    \midrule

    %%%%%%%%%%%% Scores computed by us)%%%%%%%%%%%%
    AlexNet CNN & 55.3 / 58.9 \\
    \midrule
    \bi{7}{2}-\bi{8}{6}& 68.5 / 71.8 \\
    \bi{7}{2}-\bi{8}{6}-G$(6)$& 67.6 / 73.1 \\
    \midrule
    AlexNet-CRF & 65.5 / 71.0 \\
    \bottomrule \\
  \end{tabular}
  \mycaption{Material Segmentation using AlexNet}{Pixel accuracy of different models on
  the MINC material segmentation test dataset~\cite{bell2015minc}.}
  \label{tab:mincresults-app}
\end{table}

\paragraph{Scales of Bilateral Inception Modules}
\label{sec:scales}

Unlike the explicit kernel technique presented in the main text (Chapter~\ref{chap:binception}),
we didn't back-propagate through feature transformation ($\Lambda$)
using the approximate bilateral filter technique.
So, the feature scales are hand-specified and validated, which are as follows.
The optimal scale values for the \bi{7}{2}-\bi{8}{2} model are found by validation for the best performance which are
$\sigma_{xy}$ = (0.1, 0.1) for the spatial (XY) kernel and $\sigma_{rgbxy}$ = (0.1, 0.1, 0.1, 0.01, 0.01) for color and position (RGBXY)  kernel.
Next, as more kernels are added to \bi{8}{2}, we set scales to be $\alpha$*($\sigma_{xy}$, $\sigma_{rgbxy}$).
The value of $\alpha$ is chosen as  1, 0.5, 0.1, 0.05, 0.1, at uniform interval, for the \bi{8}{10} bilateral inception module.


\subsection{Qualitative Results}
\label{sec:qualitative-app}

In this section, we present more qualitative results obtained using the BI module with explicit
kernel computation technique presented in Chapter~\ref{chap:binception}. Results on the Pascal VOC12
dataset~\cite{voc2012segmentation} using the DeepLab-LargeFOV model are shown in Fig.~\ref{fig:semantic_visuals-app},
followed by the results on MINC dataset~\cite{bell2015minc}
in Fig.~\ref{fig:material_visuals-app} and on
Cityscapes dataset~\cite{Cordts2015Cvprw} in Fig.~\ref{fig:street_visuals-app}.


\definecolor{voc_1}{RGB}{0, 0, 0}
\definecolor{voc_2}{RGB}{128, 0, 0}
\definecolor{voc_3}{RGB}{0, 128, 0}
\definecolor{voc_4}{RGB}{128, 128, 0}
\definecolor{voc_5}{RGB}{0, 0, 128}
\definecolor{voc_6}{RGB}{128, 0, 128}
\definecolor{voc_7}{RGB}{0, 128, 128}
\definecolor{voc_8}{RGB}{128, 128, 128}
\definecolor{voc_9}{RGB}{64, 0, 0}
\definecolor{voc_10}{RGB}{192, 0, 0}
\definecolor{voc_11}{RGB}{64, 128, 0}
\definecolor{voc_12}{RGB}{192, 128, 0}
\definecolor{voc_13}{RGB}{64, 0, 128}
\definecolor{voc_14}{RGB}{192, 0, 128}
\definecolor{voc_15}{RGB}{64, 128, 128}
\definecolor{voc_16}{RGB}{192, 128, 128}
\definecolor{voc_17}{RGB}{0, 64, 0}
\definecolor{voc_18}{RGB}{128, 64, 0}
\definecolor{voc_19}{RGB}{0, 192, 0}
\definecolor{voc_20}{RGB}{128, 192, 0}
\definecolor{voc_21}{RGB}{0, 64, 128}
\definecolor{voc_22}{RGB}{128, 64, 128}

\begin{figure*}[!ht]
  \small
  \centering
  \fcolorbox{white}{voc_1}{\rule{0pt}{4pt}\rule{4pt}{0pt}} Background~~
  \fcolorbox{white}{voc_2}{\rule{0pt}{4pt}\rule{4pt}{0pt}} Aeroplane~~
  \fcolorbox{white}{voc_3}{\rule{0pt}{4pt}\rule{4pt}{0pt}} Bicycle~~
  \fcolorbox{white}{voc_4}{\rule{0pt}{4pt}\rule{4pt}{0pt}} Bird~~
  \fcolorbox{white}{voc_5}{\rule{0pt}{4pt}\rule{4pt}{0pt}} Boat~~
  \fcolorbox{white}{voc_6}{\rule{0pt}{4pt}\rule{4pt}{0pt}} Bottle~~
  \fcolorbox{white}{voc_7}{\rule{0pt}{4pt}\rule{4pt}{0pt}} Bus~~
  \fcolorbox{white}{voc_8}{\rule{0pt}{4pt}\rule{4pt}{0pt}} Car~~\\
  \fcolorbox{white}{voc_9}{\rule{0pt}{4pt}\rule{4pt}{0pt}} Cat~~
  \fcolorbox{white}{voc_10}{\rule{0pt}{4pt}\rule{4pt}{0pt}} Chair~~
  \fcolorbox{white}{voc_11}{\rule{0pt}{4pt}\rule{4pt}{0pt}} Cow~~
  \fcolorbox{white}{voc_12}{\rule{0pt}{4pt}\rule{4pt}{0pt}} Dining Table~~
  \fcolorbox{white}{voc_13}{\rule{0pt}{4pt}\rule{4pt}{0pt}} Dog~~
  \fcolorbox{white}{voc_14}{\rule{0pt}{4pt}\rule{4pt}{0pt}} Horse~~
  \fcolorbox{white}{voc_15}{\rule{0pt}{4pt}\rule{4pt}{0pt}} Motorbike~~
  \fcolorbox{white}{voc_16}{\rule{0pt}{4pt}\rule{4pt}{0pt}} Person~~\\
  \fcolorbox{white}{voc_17}{\rule{0pt}{4pt}\rule{4pt}{0pt}} Potted Plant~~
  \fcolorbox{white}{voc_18}{\rule{0pt}{4pt}\rule{4pt}{0pt}} Sheep~~
  \fcolorbox{white}{voc_19}{\rule{0pt}{4pt}\rule{4pt}{0pt}} Sofa~~
  \fcolorbox{white}{voc_20}{\rule{0pt}{4pt}\rule{4pt}{0pt}} Train~~
  \fcolorbox{white}{voc_21}{\rule{0pt}{4pt}\rule{4pt}{0pt}} TV monitor~~\\


  \subfigure{%
    \includegraphics[width=.15\columnwidth]{figures/supplementary/2008_001308_given.png}
  }
  \subfigure{%
    \includegraphics[width=.15\columnwidth]{figures/supplementary/2008_001308_sp.png}
  }
  \subfigure{%
    \includegraphics[width=.15\columnwidth]{figures/supplementary/2008_001308_gt.png}
  }
  \subfigure{%
    \includegraphics[width=.15\columnwidth]{figures/supplementary/2008_001308_cnn.png}
  }
  \subfigure{%
    \includegraphics[width=.15\columnwidth]{figures/supplementary/2008_001308_crf.png}
  }
  \subfigure{%
    \includegraphics[width=.15\columnwidth]{figures/supplementary/2008_001308_ours.png}
  }\\[-2ex]


  \subfigure{%
    \includegraphics[width=.15\columnwidth]{figures/supplementary/2008_001821_given.png}
  }
  \subfigure{%
    \includegraphics[width=.15\columnwidth]{figures/supplementary/2008_001821_sp.png}
  }
  \subfigure{%
    \includegraphics[width=.15\columnwidth]{figures/supplementary/2008_001821_gt.png}
  }
  \subfigure{%
    \includegraphics[width=.15\columnwidth]{figures/supplementary/2008_001821_cnn.png}
  }
  \subfigure{%
    \includegraphics[width=.15\columnwidth]{figures/supplementary/2008_001821_crf.png}
  }
  \subfigure{%
    \includegraphics[width=.15\columnwidth]{figures/supplementary/2008_001821_ours.png}
  }\\[-2ex]



  \subfigure{%
    \includegraphics[width=.15\columnwidth]{figures/supplementary/2008_004612_given.png}
  }
  \subfigure{%
    \includegraphics[width=.15\columnwidth]{figures/supplementary/2008_004612_sp.png}
  }
  \subfigure{%
    \includegraphics[width=.15\columnwidth]{figures/supplementary/2008_004612_gt.png}
  }
  \subfigure{%
    \includegraphics[width=.15\columnwidth]{figures/supplementary/2008_004612_cnn.png}
  }
  \subfigure{%
    \includegraphics[width=.15\columnwidth]{figures/supplementary/2008_004612_crf.png}
  }
  \subfigure{%
    \includegraphics[width=.15\columnwidth]{figures/supplementary/2008_004612_ours.png}
  }\\[-2ex]


  \subfigure{%
    \includegraphics[width=.15\columnwidth]{figures/supplementary/2009_001008_given.png}
  }
  \subfigure{%
    \includegraphics[width=.15\columnwidth]{figures/supplementary/2009_001008_sp.png}
  }
  \subfigure{%
    \includegraphics[width=.15\columnwidth]{figures/supplementary/2009_001008_gt.png}
  }
  \subfigure{%
    \includegraphics[width=.15\columnwidth]{figures/supplementary/2009_001008_cnn.png}
  }
  \subfigure{%
    \includegraphics[width=.15\columnwidth]{figures/supplementary/2009_001008_crf.png}
  }
  \subfigure{%
    \includegraphics[width=.15\columnwidth]{figures/supplementary/2009_001008_ours.png}
  }\\[-2ex]




  \subfigure{%
    \includegraphics[width=.15\columnwidth]{figures/supplementary/2009_004497_given.png}
  }
  \subfigure{%
    \includegraphics[width=.15\columnwidth]{figures/supplementary/2009_004497_sp.png}
  }
  \subfigure{%
    \includegraphics[width=.15\columnwidth]{figures/supplementary/2009_004497_gt.png}
  }
  \subfigure{%
    \includegraphics[width=.15\columnwidth]{figures/supplementary/2009_004497_cnn.png}
  }
  \subfigure{%
    \includegraphics[width=.15\columnwidth]{figures/supplementary/2009_004497_crf.png}
  }
  \subfigure{%
    \includegraphics[width=.15\columnwidth]{figures/supplementary/2009_004497_ours.png}
  }\\[-2ex]



  \setcounter{subfigure}{0}
  \subfigure[\scriptsize Input]{%
    \includegraphics[width=.15\columnwidth]{figures/supplementary/2010_001327_given.png}
  }
  \subfigure[\scriptsize Superpixels]{%
    \includegraphics[width=.15\columnwidth]{figures/supplementary/2010_001327_sp.png}
  }
  \subfigure[\scriptsize GT]{%
    \includegraphics[width=.15\columnwidth]{figures/supplementary/2010_001327_gt.png}
  }
  \subfigure[\scriptsize Deeplab]{%
    \includegraphics[width=.15\columnwidth]{figures/supplementary/2010_001327_cnn.png}
  }
  \subfigure[\scriptsize +DenseCRF]{%
    \includegraphics[width=.15\columnwidth]{figures/supplementary/2010_001327_crf.png}
  }
  \subfigure[\scriptsize Using BI]{%
    \includegraphics[width=.15\columnwidth]{figures/supplementary/2010_001327_ours.png}
  }
  \mycaption{Semantic Segmentation}{Example results of semantic segmentation
  on the Pascal VOC12 dataset.
  (d)~depicts the DeepLab CNN result, (e)~CNN + 10 steps of mean-field inference,
  (f~result obtained with bilateral inception (BI) modules (\bi{6}{2}+\bi{7}{6}) between \fc~layers.}
  \label{fig:semantic_visuals-app}
\end{figure*}


\definecolor{minc_1}{HTML}{771111}
\definecolor{minc_2}{HTML}{CAC690}
\definecolor{minc_3}{HTML}{EEEEEE}
\definecolor{minc_4}{HTML}{7C8FA6}
\definecolor{minc_5}{HTML}{597D31}
\definecolor{minc_6}{HTML}{104410}
\definecolor{minc_7}{HTML}{BB819C}
\definecolor{minc_8}{HTML}{D0CE48}
\definecolor{minc_9}{HTML}{622745}
\definecolor{minc_10}{HTML}{666666}
\definecolor{minc_11}{HTML}{D54A31}
\definecolor{minc_12}{HTML}{101044}
\definecolor{minc_13}{HTML}{444126}
\definecolor{minc_14}{HTML}{75D646}
\definecolor{minc_15}{HTML}{DD4348}
\definecolor{minc_16}{HTML}{5C8577}
\definecolor{minc_17}{HTML}{C78472}
\definecolor{minc_18}{HTML}{75D6D0}
\definecolor{minc_19}{HTML}{5B4586}
\definecolor{minc_20}{HTML}{C04393}
\definecolor{minc_21}{HTML}{D69948}
\definecolor{minc_22}{HTML}{7370D8}
\definecolor{minc_23}{HTML}{7A3622}
\definecolor{minc_24}{HTML}{000000}

\begin{figure*}[!ht]
  \small % scriptsize
  \centering
  \fcolorbox{white}{minc_1}{\rule{0pt}{4pt}\rule{4pt}{0pt}} Brick~~
  \fcolorbox{white}{minc_2}{\rule{0pt}{4pt}\rule{4pt}{0pt}} Carpet~~
  \fcolorbox{white}{minc_3}{\rule{0pt}{4pt}\rule{4pt}{0pt}} Ceramic~~
  \fcolorbox{white}{minc_4}{\rule{0pt}{4pt}\rule{4pt}{0pt}} Fabric~~
  \fcolorbox{white}{minc_5}{\rule{0pt}{4pt}\rule{4pt}{0pt}} Foliage~~
  \fcolorbox{white}{minc_6}{\rule{0pt}{4pt}\rule{4pt}{0pt}} Food~~
  \fcolorbox{white}{minc_7}{\rule{0pt}{4pt}\rule{4pt}{0pt}} Glass~~
  \fcolorbox{white}{minc_8}{\rule{0pt}{4pt}\rule{4pt}{0pt}} Hair~~\\
  \fcolorbox{white}{minc_9}{\rule{0pt}{4pt}\rule{4pt}{0pt}} Leather~~
  \fcolorbox{white}{minc_10}{\rule{0pt}{4pt}\rule{4pt}{0pt}} Metal~~
  \fcolorbox{white}{minc_11}{\rule{0pt}{4pt}\rule{4pt}{0pt}} Mirror~~
  \fcolorbox{white}{minc_12}{\rule{0pt}{4pt}\rule{4pt}{0pt}} Other~~
  \fcolorbox{white}{minc_13}{\rule{0pt}{4pt}\rule{4pt}{0pt}} Painted~~
  \fcolorbox{white}{minc_14}{\rule{0pt}{4pt}\rule{4pt}{0pt}} Paper~~
  \fcolorbox{white}{minc_15}{\rule{0pt}{4pt}\rule{4pt}{0pt}} Plastic~~\\
  \fcolorbox{white}{minc_16}{\rule{0pt}{4pt}\rule{4pt}{0pt}} Polished Stone~~
  \fcolorbox{white}{minc_17}{\rule{0pt}{4pt}\rule{4pt}{0pt}} Skin~~
  \fcolorbox{white}{minc_18}{\rule{0pt}{4pt}\rule{4pt}{0pt}} Sky~~
  \fcolorbox{white}{minc_19}{\rule{0pt}{4pt}\rule{4pt}{0pt}} Stone~~
  \fcolorbox{white}{minc_20}{\rule{0pt}{4pt}\rule{4pt}{0pt}} Tile~~
  \fcolorbox{white}{minc_21}{\rule{0pt}{4pt}\rule{4pt}{0pt}} Wallpaper~~
  \fcolorbox{white}{minc_22}{\rule{0pt}{4pt}\rule{4pt}{0pt}} Water~~
  \fcolorbox{white}{minc_23}{\rule{0pt}{4pt}\rule{4pt}{0pt}} Wood~~\\
  \subfigure{%
    \includegraphics[width=.15\columnwidth]{figures/supplementary/000008468_given.png}
  }
  \subfigure{%
    \includegraphics[width=.15\columnwidth]{figures/supplementary/000008468_sp.png}
  }
  \subfigure{%
    \includegraphics[width=.15\columnwidth]{figures/supplementary/000008468_gt.png}
  }
  \subfigure{%
    \includegraphics[width=.15\columnwidth]{figures/supplementary/000008468_cnn.png}
  }
  \subfigure{%
    \includegraphics[width=.15\columnwidth]{figures/supplementary/000008468_crf.png}
  }
  \subfigure{%
    \includegraphics[width=.15\columnwidth]{figures/supplementary/000008468_ours.png}
  }\\[-2ex]

  \subfigure{%
    \includegraphics[width=.15\columnwidth]{figures/supplementary/000009053_given.png}
  }
  \subfigure{%
    \includegraphics[width=.15\columnwidth]{figures/supplementary/000009053_sp.png}
  }
  \subfigure{%
    \includegraphics[width=.15\columnwidth]{figures/supplementary/000009053_gt.png}
  }
  \subfigure{%
    \includegraphics[width=.15\columnwidth]{figures/supplementary/000009053_cnn.png}
  }
  \subfigure{%
    \includegraphics[width=.15\columnwidth]{figures/supplementary/000009053_crf.png}
  }
  \subfigure{%
    \includegraphics[width=.15\columnwidth]{figures/supplementary/000009053_ours.png}
  }\\[-2ex]




  \subfigure{%
    \includegraphics[width=.15\columnwidth]{figures/supplementary/000014977_given.png}
  }
  \subfigure{%
    \includegraphics[width=.15\columnwidth]{figures/supplementary/000014977_sp.png}
  }
  \subfigure{%
    \includegraphics[width=.15\columnwidth]{figures/supplementary/000014977_gt.png}
  }
  \subfigure{%
    \includegraphics[width=.15\columnwidth]{figures/supplementary/000014977_cnn.png}
  }
  \subfigure{%
    \includegraphics[width=.15\columnwidth]{figures/supplementary/000014977_crf.png}
  }
  \subfigure{%
    \includegraphics[width=.15\columnwidth]{figures/supplementary/000014977_ours.png}
  }\\[-2ex]


  \subfigure{%
    \includegraphics[width=.15\columnwidth]{figures/supplementary/000022922_given.png}
  }
  \subfigure{%
    \includegraphics[width=.15\columnwidth]{figures/supplementary/000022922_sp.png}
  }
  \subfigure{%
    \includegraphics[width=.15\columnwidth]{figures/supplementary/000022922_gt.png}
  }
  \subfigure{%
    \includegraphics[width=.15\columnwidth]{figures/supplementary/000022922_cnn.png}
  }
  \subfigure{%
    \includegraphics[width=.15\columnwidth]{figures/supplementary/000022922_crf.png}
  }
  \subfigure{%
    \includegraphics[width=.15\columnwidth]{figures/supplementary/000022922_ours.png}
  }\\[-2ex]


  \subfigure{%
    \includegraphics[width=.15\columnwidth]{figures/supplementary/000025711_given.png}
  }
  \subfigure{%
    \includegraphics[width=.15\columnwidth]{figures/supplementary/000025711_sp.png}
  }
  \subfigure{%
    \includegraphics[width=.15\columnwidth]{figures/supplementary/000025711_gt.png}
  }
  \subfigure{%
    \includegraphics[width=.15\columnwidth]{figures/supplementary/000025711_cnn.png}
  }
  \subfigure{%
    \includegraphics[width=.15\columnwidth]{figures/supplementary/000025711_crf.png}
  }
  \subfigure{%
    \includegraphics[width=.15\columnwidth]{figures/supplementary/000025711_ours.png}
  }\\[-2ex]


  \subfigure{%
    \includegraphics[width=.15\columnwidth]{figures/supplementary/000034473_given.png}
  }
  \subfigure{%
    \includegraphics[width=.15\columnwidth]{figures/supplementary/000034473_sp.png}
  }
  \subfigure{%
    \includegraphics[width=.15\columnwidth]{figures/supplementary/000034473_gt.png}
  }
  \subfigure{%
    \includegraphics[width=.15\columnwidth]{figures/supplementary/000034473_cnn.png}
  }
  \subfigure{%
    \includegraphics[width=.15\columnwidth]{figures/supplementary/000034473_crf.png}
  }
  \subfigure{%
    \includegraphics[width=.15\columnwidth]{figures/supplementary/000034473_ours.png}
  }\\[-2ex]


  \subfigure{%
    \includegraphics[width=.15\columnwidth]{figures/supplementary/000035463_given.png}
  }
  \subfigure{%
    \includegraphics[width=.15\columnwidth]{figures/supplementary/000035463_sp.png}
  }
  \subfigure{%
    \includegraphics[width=.15\columnwidth]{figures/supplementary/000035463_gt.png}
  }
  \subfigure{%
    \includegraphics[width=.15\columnwidth]{figures/supplementary/000035463_cnn.png}
  }
  \subfigure{%
    \includegraphics[width=.15\columnwidth]{figures/supplementary/000035463_crf.png}
  }
  \subfigure{%
    \includegraphics[width=.15\columnwidth]{figures/supplementary/000035463_ours.png}
  }\\[-2ex]


  \setcounter{subfigure}{0}
  \subfigure[\scriptsize Input]{%
    \includegraphics[width=.15\columnwidth]{figures/supplementary/000035993_given.png}
  }
  \subfigure[\scriptsize Superpixels]{%
    \includegraphics[width=.15\columnwidth]{figures/supplementary/000035993_sp.png}
  }
  \subfigure[\scriptsize GT]{%
    \includegraphics[width=.15\columnwidth]{figures/supplementary/000035993_gt.png}
  }
  \subfigure[\scriptsize AlexNet]{%
    \includegraphics[width=.15\columnwidth]{figures/supplementary/000035993_cnn.png}
  }
  \subfigure[\scriptsize +DenseCRF]{%
    \includegraphics[width=.15\columnwidth]{figures/supplementary/000035993_crf.png}
  }
  \subfigure[\scriptsize Using BI]{%
    \includegraphics[width=.15\columnwidth]{figures/supplementary/000035993_ours.png}
  }
  \mycaption{Material Segmentation}{Example results of material segmentation.
  (d)~depicts the AlexNet CNN result, (e)~CNN + 10 steps of mean-field inference,
  (f)~result obtained with bilateral inception (BI) modules (\bi{7}{2}+\bi{8}{6}) between
  \fc~layers.}
\label{fig:material_visuals-app}
\end{figure*}


\definecolor{city_1}{RGB}{128, 64, 128}
\definecolor{city_2}{RGB}{244, 35, 232}
\definecolor{city_3}{RGB}{70, 70, 70}
\definecolor{city_4}{RGB}{102, 102, 156}
\definecolor{city_5}{RGB}{190, 153, 153}
\definecolor{city_6}{RGB}{153, 153, 153}
\definecolor{city_7}{RGB}{250, 170, 30}
\definecolor{city_8}{RGB}{220, 220, 0}
\definecolor{city_9}{RGB}{107, 142, 35}
\definecolor{city_10}{RGB}{152, 251, 152}
\definecolor{city_11}{RGB}{70, 130, 180}
\definecolor{city_12}{RGB}{220, 20, 60}
\definecolor{city_13}{RGB}{255, 0, 0}
\definecolor{city_14}{RGB}{0, 0, 142}
\definecolor{city_15}{RGB}{0, 0, 70}
\definecolor{city_16}{RGB}{0, 60, 100}
\definecolor{city_17}{RGB}{0, 80, 100}
\definecolor{city_18}{RGB}{0, 0, 230}
\definecolor{city_19}{RGB}{119, 11, 32}
\begin{figure*}[!ht]
  \small % scriptsize
  \centering


  \subfigure{%
    \includegraphics[width=.18\columnwidth]{figures/supplementary/frankfurt00000_016005_given.png}
  }
  \subfigure{%
    \includegraphics[width=.18\columnwidth]{figures/supplementary/frankfurt00000_016005_sp.png}
  }
  \subfigure{%
    \includegraphics[width=.18\columnwidth]{figures/supplementary/frankfurt00000_016005_gt.png}
  }
  \subfigure{%
    \includegraphics[width=.18\columnwidth]{figures/supplementary/frankfurt00000_016005_cnn.png}
  }
  \subfigure{%
    \includegraphics[width=.18\columnwidth]{figures/supplementary/frankfurt00000_016005_ours.png}
  }\\[-2ex]

  \subfigure{%
    \includegraphics[width=.18\columnwidth]{figures/supplementary/frankfurt00000_004617_given.png}
  }
  \subfigure{%
    \includegraphics[width=.18\columnwidth]{figures/supplementary/frankfurt00000_004617_sp.png}
  }
  \subfigure{%
    \includegraphics[width=.18\columnwidth]{figures/supplementary/frankfurt00000_004617_gt.png}
  }
  \subfigure{%
    \includegraphics[width=.18\columnwidth]{figures/supplementary/frankfurt00000_004617_cnn.png}
  }
  \subfigure{%
    \includegraphics[width=.18\columnwidth]{figures/supplementary/frankfurt00000_004617_ours.png}
  }\\[-2ex]

  \subfigure{%
    \includegraphics[width=.18\columnwidth]{figures/supplementary/frankfurt00000_020880_given.png}
  }
  \subfigure{%
    \includegraphics[width=.18\columnwidth]{figures/supplementary/frankfurt00000_020880_sp.png}
  }
  \subfigure{%
    \includegraphics[width=.18\columnwidth]{figures/supplementary/frankfurt00000_020880_gt.png}
  }
  \subfigure{%
    \includegraphics[width=.18\columnwidth]{figures/supplementary/frankfurt00000_020880_cnn.png}
  }
  \subfigure{%
    \includegraphics[width=.18\columnwidth]{figures/supplementary/frankfurt00000_020880_ours.png}
  }\\[-2ex]



  \subfigure{%
    \includegraphics[width=.18\columnwidth]{figures/supplementary/frankfurt00001_007285_given.png}
  }
  \subfigure{%
    \includegraphics[width=.18\columnwidth]{figures/supplementary/frankfurt00001_007285_sp.png}
  }
  \subfigure{%
    \includegraphics[width=.18\columnwidth]{figures/supplementary/frankfurt00001_007285_gt.png}
  }
  \subfigure{%
    \includegraphics[width=.18\columnwidth]{figures/supplementary/frankfurt00001_007285_cnn.png}
  }
  \subfigure{%
    \includegraphics[width=.18\columnwidth]{figures/supplementary/frankfurt00001_007285_ours.png}
  }\\[-2ex]


  \subfigure{%
    \includegraphics[width=.18\columnwidth]{figures/supplementary/frankfurt00001_059789_given.png}
  }
  \subfigure{%
    \includegraphics[width=.18\columnwidth]{figures/supplementary/frankfurt00001_059789_sp.png}
  }
  \subfigure{%
    \includegraphics[width=.18\columnwidth]{figures/supplementary/frankfurt00001_059789_gt.png}
  }
  \subfigure{%
    \includegraphics[width=.18\columnwidth]{figures/supplementary/frankfurt00001_059789_cnn.png}
  }
  \subfigure{%
    \includegraphics[width=.18\columnwidth]{figures/supplementary/frankfurt00001_059789_ours.png}
  }\\[-2ex]


  \subfigure{%
    \includegraphics[width=.18\columnwidth]{figures/supplementary/frankfurt00001_068208_given.png}
  }
  \subfigure{%
    \includegraphics[width=.18\columnwidth]{figures/supplementary/frankfurt00001_068208_sp.png}
  }
  \subfigure{%
    \includegraphics[width=.18\columnwidth]{figures/supplementary/frankfurt00001_068208_gt.png}
  }
  \subfigure{%
    \includegraphics[width=.18\columnwidth]{figures/supplementary/frankfurt00001_068208_cnn.png}
  }
  \subfigure{%
    \includegraphics[width=.18\columnwidth]{figures/supplementary/frankfurt00001_068208_ours.png}
  }\\[-2ex]

  \subfigure{%
    \includegraphics[width=.18\columnwidth]{figures/supplementary/frankfurt00001_082466_given.png}
  }
  \subfigure{%
    \includegraphics[width=.18\columnwidth]{figures/supplementary/frankfurt00001_082466_sp.png}
  }
  \subfigure{%
    \includegraphics[width=.18\columnwidth]{figures/supplementary/frankfurt00001_082466_gt.png}
  }
  \subfigure{%
    \includegraphics[width=.18\columnwidth]{figures/supplementary/frankfurt00001_082466_cnn.png}
  }
  \subfigure{%
    \includegraphics[width=.18\columnwidth]{figures/supplementary/frankfurt00001_082466_ours.png}
  }\\[-2ex]

  \subfigure{%
    \includegraphics[width=.18\columnwidth]{figures/supplementary/lindau00033_000019_given.png}
  }
  \subfigure{%
    \includegraphics[width=.18\columnwidth]{figures/supplementary/lindau00033_000019_sp.png}
  }
  \subfigure{%
    \includegraphics[width=.18\columnwidth]{figures/supplementary/lindau00033_000019_gt.png}
  }
  \subfigure{%
    \includegraphics[width=.18\columnwidth]{figures/supplementary/lindau00033_000019_cnn.png}
  }
  \subfigure{%
    \includegraphics[width=.18\columnwidth]{figures/supplementary/lindau00033_000019_ours.png}
  }\\[-2ex]

  \subfigure{%
    \includegraphics[width=.18\columnwidth]{figures/supplementary/lindau00052_000019_given.png}
  }
  \subfigure{%
    \includegraphics[width=.18\columnwidth]{figures/supplementary/lindau00052_000019_sp.png}
  }
  \subfigure{%
    \includegraphics[width=.18\columnwidth]{figures/supplementary/lindau00052_000019_gt.png}
  }
  \subfigure{%
    \includegraphics[width=.18\columnwidth]{figures/supplementary/lindau00052_000019_cnn.png}
  }
  \subfigure{%
    \includegraphics[width=.18\columnwidth]{figures/supplementary/lindau00052_000019_ours.png}
  }\\[-2ex]




  \subfigure{%
    \includegraphics[width=.18\columnwidth]{figures/supplementary/lindau00027_000019_given.png}
  }
  \subfigure{%
    \includegraphics[width=.18\columnwidth]{figures/supplementary/lindau00027_000019_sp.png}
  }
  \subfigure{%
    \includegraphics[width=.18\columnwidth]{figures/supplementary/lindau00027_000019_gt.png}
  }
  \subfigure{%
    \includegraphics[width=.18\columnwidth]{figures/supplementary/lindau00027_000019_cnn.png}
  }
  \subfigure{%
    \includegraphics[width=.18\columnwidth]{figures/supplementary/lindau00027_000019_ours.png}
  }\\[-2ex]



  \setcounter{subfigure}{0}
  \subfigure[\scriptsize Input]{%
    \includegraphics[width=.18\columnwidth]{figures/supplementary/lindau00029_000019_given.png}
  }
  \subfigure[\scriptsize Superpixels]{%
    \includegraphics[width=.18\columnwidth]{figures/supplementary/lindau00029_000019_sp.png}
  }
  \subfigure[\scriptsize GT]{%
    \includegraphics[width=.18\columnwidth]{figures/supplementary/lindau00029_000019_gt.png}
  }
  \subfigure[\scriptsize Deeplab]{%
    \includegraphics[width=.18\columnwidth]{figures/supplementary/lindau00029_000019_cnn.png}
  }
  \subfigure[\scriptsize Using BI]{%
    \includegraphics[width=.18\columnwidth]{figures/supplementary/lindau00029_000019_ours.png}
  }%\\[-2ex]

  \mycaption{Street Scene Segmentation}{Example results of street scene segmentation.
  (d)~depicts the DeepLab results, (e)~result obtained by adding bilateral inception (BI) modules (\bi{6}{2}+\bi{7}{6}) between \fc~layers.}
\label{fig:street_visuals-app}
\end{figure*}



%%%%%%%%%%%%%%%%%%%%%%%%%%%%%%%%%%%%%%%%%%%%%%%%%%%%
%%% REFERENCES
\nocite{*}
\ifdefined\hpec
%*flatex input: [pageRank-MAIN.bbl]
% Generated by IEEEtran.bst, version: 1.12 (2007/01/11)
\begin{thebibliography}{10}
\providecommand{\url}[1]{#1}
\csname url@samestyle\endcsname
\providecommand{\newblock}{\relax}
\providecommand{\bibinfo}[2]{#2}
\providecommand{\BIBentrySTDinterwordspacing}{\spaceskip=0pt\relax}
\providecommand{\BIBentryALTinterwordstretchfactor}{4}
\providecommand{\BIBentryALTinterwordspacing}{\spaceskip=\fontdimen2\font plus
\BIBentryALTinterwordstretchfactor\fontdimen3\font minus
  \fontdimen4\font\relax}
\providecommand{\BIBforeignlanguage}[2]{{%
\expandafter\ifx\csname l@#1\endcsname\relax
\typeout{** WARNING: IEEEtran.bst: No hyphenation pattern has been}%
\typeout{** loaded for the language `#1'. Using the pattern for}%
\typeout{** the default language instead.}%
\else
\language=\csname l@#1\endcsname
\fi
#2}}
\providecommand{\BIBdecl}{\relax}
\BIBdecl

\bibitem{page1999pagerank}
L.~Page, S.~Brin, R.~Motwani, and T.~Winograd, ``The {P}age{R}ank citation
  ranking: Bringing order to the web.'' Stanford InfoLab, Tech. Rep., 1999.

\bibitem{ilprints361}
S.~Brin and L.~Page, ``The anatomy of a large-scale hypertextual web search
  engine,'' \emph{Computer networks and ISDN systems}, vol.~30, no. 1-7, pp.
  107--117, 1998.

\bibitem{sheldon2010manipulation}
D.~Sheldon, ``Manipulation of {P}age{R}ank and {C}ollective {H}idden {M}arkov
  {M}odels,'' Ph.D. dissertation, Cornell University, Ithaca, NY, USA, 2010.

\bibitem{haveliwala2003topic}
T.~H. Haveliwala, ``Topic-sensitive {P}age{R}ank: A context-sensitive ranking
  algorithm for web search,'' \emph{IEEE Transactions on Knowledge and Data
  Engineering}, vol.~15, no.~4, pp. 784--796, 2003.

\bibitem{morrison2005generank}
J.~L. Morrison, R.~Breitling, D.~J. Higham, and D.~R. Gilbert, ``Generank:
  using search engine technology for the analysis of microarray experiments,''
  \emph{BMC bioinformatics}, vol.~6, no.~1, p. 233, 2005.

\bibitem{wu2010krylov}
G.~Wu, Y.~Zhang, and Y.~Wei, ``Krylov subspace algorithms for computing
  generank for the analysis of microarray data mining,'' \emph{Journal of
  Computational Biology}, vol.~17, no.~4, pp. 631--646, 2010.

\bibitem{langville2004deeper}
A.~N. Langville and C.~D. Meyer, ``Deeper inside {P}age{R}ank,'' \emph{Internet
  Mathematics}, vol.~1, no.~3, pp. 335--380, 2004.

\bibitem{berkhin2005survey}
P.~Berkhin, ``A survey on {P}age{R}ank computing,'' \emph{Internet
  Mathematics}, vol.~2, no.~1, pp. 73--120, 2005.

\bibitem{ilprints596}
T.~Haveliwala, S.~Kamvar, and G.~Jeh, ``An analytical comparison of approaches
  to personalizing {P}age{R}ank,'' Stanford InfoLab, Technical Report 2003-35,
  June 2003.

\bibitem{ilprints579}
S.~Kamvar, T.~Haveliwala, C.~Manning, and G.~Golub, ``Exploiting the block
  structure of the web for computing {P}age{R}ank,'' Stanford InfoLab,
  Technical Report 2003-17, 2003.

\bibitem{boldi2005pagerank}
P.~Boldi, M.~Santini, and S.~Vigna, ``{P}age{R}ank as a function of the damping
  factor,'' in \emph{Proceedings of the 14th International Conference on World
  Wide Web}.\hskip 1em plus 0.5em minus 0.4em\relax ACM, 2005, pp. 557--566.

\bibitem{bressan2010choose}
M.~Bressan and E.~Peserico, ``Choose the damping, choose the ranking?''
  \emph{Journal of Discrete Algorithms}, vol.~8, no.~2, pp. 199--213, 2010.

\bibitem{chung2007heat}
F.~Chung, ``The heat kernel as the {P}age{R}ank of a graph,'' \emph{Proceedings
  of the National Academy of Sciences}, vol. 104, no.~50, pp. 19\,735--19\,740,
  2007.

\bibitem{chung2009local}
------, ``A local graph partitioning algorithm using heat kernel
  {P}age{R}ank,'' \emph{Internet Mathematics}, vol.~6, no.~3, pp. 315--330,
  2009.

\bibitem{arasu2002pagerank}
A.~Arasu, J.~Novak, A.~Tomkins, and J.~Tomlin, ``{P}age{R}ank computation and
  the structure of the web: Experiments and algorithms,'' in \emph{Proceedings
  of the 11th International Conference on World Wide Web}, 2002, pp. 107--117.

\bibitem{kullback1951information}
S.~Kullback and R.~A. Leibler, ``On information and sufficiency,'' \emph{The
  Annals of Mathematical Statistics}, vol.~22, no.~1, pp. 79--86, 1951.

\bibitem{lee2003fast}
C.~P.-C. Lee, G.~H. Golub, and S.~A. Zenios, ``A fast two-stage algorithm for
  computing {P}age{R}ank and its extensions,'' \emph{Scientific Computation and
  Computational Mathematics}, vol.~1, no.~1, pp. 1--9, 2003.

\bibitem{langville2006reordering}
A.~N. Langville and C.~D. Meyer, ``A reordering for the {P}age{R}ank problem,''
  \emph{SIAM Journal on Scientific Computing}, vol.~27, no.~6, pp. 2112--2120,
  2006.

\bibitem{dulmage1958coverings}
A.~L. Dulmage and N.~S. Mendelsohn, ``Coverings of bipartite graphs,''
  \emph{Canadian Journal of Mathematics}, vol.~10, no.~4, pp. 516--534, 1958.

\bibitem{google-net}
Google, ``Google programming contest,''
  \url{http://www.google.com/programming-contest/}, 2002.

\bibitem{kamvar2003extrapolation}
S.~D. Kamvar, T.~H. Haveliwala, C.~D. Manning, and G.~H. Golub, ``Extrapolation
  methods for accelerating {P}age{R}ank computations,'' in \emph{Proceedings of
  the 12th International Conference on World Wide Web}.\hskip 1em plus 0.5em
  minus 0.4em\relax ACM, 2003, pp. 261--270.

\bibitem{kamvar2004adaptive}
S.~Kamvar, T.~Haveliwala, and G.~Golub, ``Adaptive methods for the computation
  of {P}age{R}ank,'' \emph{Linear Algebra and its Applications}, vol. 386, pp.
  51--65, 2004.

\bibitem{DBLP:journals/corr/LofgrenBGS14}
P.~A. Lofgren, S.~Banerjee, A.~Goel, and C.~Seshadhri, ``{FAST-PPR:} scaling
  personalized pagerank estimation for large graphs,'' in \emph{Proceedings of
  the 20th ACM SIGKDD International Conference on Knowledge Discovery and Data
  Mining}.\hskip 1em plus 0.5em minus 0.4em\relax ACM, 2014, pp. 1436--1445.

\bibitem{kunegis2013konect}
J.~Kunegis, ``{KONECT}--the koblenz network collection,'' in \emph{Proceedings
  of the 22nd International Conference on World Wide Web}.\hskip 1em plus 0.5em
  minus 0.4em\relax ACM, 2013, pp. 1343--1350.

\bibitem{auer2007dbpedia}
S.~Auer, C.~Bizer, G.~Kobilarov, J.~Lehmann, R.~Cyganiak, and Z.~Ives,
  ``{DB}pedia: A nucleus for a web of open data,'' in \emph{The Semantic
  Web}.\hskip 1em plus 0.5em minus 0.4em\relax Springer, 2007, pp. 722--735.

\bibitem{Kwak10www}
H.~Kwak, C.~Lee, H.~Park, and S.~Moon, ``{W}hat is {T}witter, a social network
  or a news media?'' in \emph{Proceedings of the 19th International Conference
  on World Wide Web}.\hskip 1em plus 0.5em minus 0.4em\relax New York, NY, USA:
  ACM, 2010, pp. 591--600.

\bibitem{icwsm10cha}
M.~Cha, H.~Haddadi, F.~Benevenuto, and K.~P. Gummadi, ``{Measuring User
  Influence in Twitter: The Million Follower Fallacy},'' in \emph{Proceedings
  of the 4th International AAAI Conference on Weblogs and Social Media
  (ICWSM)}, Washington DC, USA, May 2010.

\bibitem{yang2015defining}
J.~Yang and J.~Leskovec, ``Defining and evaluating network communities based on
  ground-truth,'' \emph{Knowledge and Information Systems}, vol.~42, no.~1, pp.
  181--213, 2015.

\bibitem{MS-qian-2018}
Y.~Qian, ``Variable damping effect on network propagation,'' Master's thesis,
  Duke University, Durham, NC, USA, May 2018.

\bibitem{MS-Xichen-2018}
X.~Chen, ``Exploiting common structures across multiple network propagation
  schemes,'' Master's thesis, Duke University, Durham, NC, USA, May 2018.

\bibitem{ilprints582}
T.~Haveliwala and S.~Kamvar, ``The second eigenvalue of the google matrix,''
  Stanford InfoLab, Technical Report 2003-20, 2003.

\bibitem{richardson2002intelligent}
M.~Richardson and P.~Domingos, ``The intelligent surfer: Probabilistic
  combination of link and content information in {P}age{R}ank,'' in
  \emph{Advances in Neural Information Processing Systems}, 2002, pp.
  1441--1448.

\bibitem{haveliwala1999efficient}
T.~Haveliwala, ``Efficient computation of {P}age{R}ank,'' Stanford, Tech. Rep.,
  1999.

\bibitem{langville2011google}
A.~N. Langville and C.~D. Meyer, \emph{Google's {P}age{R}ank and beyond: The
  science of search engine rankings}.\hskip 1em plus 0.5em minus 0.4em\relax
  Princeton University Press, 2011.

\bibitem{jeh2003scaling}
G.~Jeh and J.~Widom, ``Scaling personalized web search,'' in \emph{Proceedings
  of the 12th International Conference on World Wide Web}.\hskip 1em plus 0.5em
  minus 0.4em\relax ACM, 2003, pp. 271--279.

\bibitem{chung2010pagerank}
F.~Chung and W.~Zhao, ``{P}age{R}ank and random walks on graphs,'' in
  \emph{Fete of Combinatorics and Computer Science}.\hskip 1em plus 0.5em minus
  0.4em\relax Springer, 2010, pp. 43--62.

\end{thebibliography}

% flatex input end: [pageRank-MAIN.bbl]
%FLATEX-REM:	\bibliographystyle{IEEEtran}
\else
%*flatex input: [pageRank-MAIN.bbl]
% Generated by IEEEtran.bst, version: 1.12 (2007/01/11)
\begin{thebibliography}{10}
\providecommand{\url}[1]{#1}
\csname url@samestyle\endcsname
\providecommand{\newblock}{\relax}
\providecommand{\bibinfo}[2]{#2}
\providecommand{\BIBentrySTDinterwordspacing}{\spaceskip=0pt\relax}
\providecommand{\BIBentryALTinterwordstretchfactor}{4}
\providecommand{\BIBentryALTinterwordspacing}{\spaceskip=\fontdimen2\font plus
\BIBentryALTinterwordstretchfactor\fontdimen3\font minus
  \fontdimen4\font\relax}
\providecommand{\BIBforeignlanguage}[2]{{%
\expandafter\ifx\csname l@#1\endcsname\relax
\typeout{** WARNING: IEEEtran.bst: No hyphenation pattern has been}%
\typeout{** loaded for the language `#1'. Using the pattern for}%
\typeout{** the default language instead.}%
\else
\language=\csname l@#1\endcsname
\fi
#2}}
\providecommand{\BIBdecl}{\relax}
\BIBdecl

\bibitem{page1999pagerank}
L.~Page, S.~Brin, R.~Motwani, and T.~Winograd, ``The {P}age{R}ank citation
  ranking: Bringing order to the web.'' Stanford InfoLab, Tech. Rep., 1999.

\bibitem{ilprints361}
S.~Brin and L.~Page, ``The anatomy of a large-scale hypertextual web search
  engine,'' \emph{Computer networks and ISDN systems}, vol.~30, no. 1-7, pp.
  107--117, 1998.

\bibitem{sheldon2010manipulation}
D.~Sheldon, ``Manipulation of {P}age{R}ank and {C}ollective {H}idden {M}arkov
  {M}odels,'' Ph.D. dissertation, Cornell University, Ithaca, NY, USA, 2010.

\bibitem{haveliwala2003topic}
T.~H. Haveliwala, ``Topic-sensitive {P}age{R}ank: A context-sensitive ranking
  algorithm for web search,'' \emph{IEEE Transactions on Knowledge and Data
  Engineering}, vol.~15, no.~4, pp. 784--796, 2003.

\bibitem{morrison2005generank}
J.~L. Morrison, R.~Breitling, D.~J. Higham, and D.~R. Gilbert, ``Generank:
  using search engine technology for the analysis of microarray experiments,''
  \emph{BMC bioinformatics}, vol.~6, no.~1, p. 233, 2005.

\bibitem{wu2010krylov}
G.~Wu, Y.~Zhang, and Y.~Wei, ``Krylov subspace algorithms for computing
  generank for the analysis of microarray data mining,'' \emph{Journal of
  Computational Biology}, vol.~17, no.~4, pp. 631--646, 2010.

\bibitem{langville2004deeper}
A.~N. Langville and C.~D. Meyer, ``Deeper inside {P}age{R}ank,'' \emph{Internet
  Mathematics}, vol.~1, no.~3, pp. 335--380, 2004.

\bibitem{berkhin2005survey}
P.~Berkhin, ``A survey on {P}age{R}ank computing,'' \emph{Internet
  Mathematics}, vol.~2, no.~1, pp. 73--120, 2005.

\bibitem{ilprints596}
T.~Haveliwala, S.~Kamvar, and G.~Jeh, ``An analytical comparison of approaches
  to personalizing {P}age{R}ank,'' Stanford InfoLab, Technical Report 2003-35,
  June 2003.

\bibitem{ilprints579}
S.~Kamvar, T.~Haveliwala, C.~Manning, and G.~Golub, ``Exploiting the block
  structure of the web for computing {P}age{R}ank,'' Stanford InfoLab,
  Technical Report 2003-17, 2003.

\bibitem{boldi2005pagerank}
P.~Boldi, M.~Santini, and S.~Vigna, ``{P}age{R}ank as a function of the damping
  factor,'' in \emph{Proceedings of the 14th International Conference on World
  Wide Web}.\hskip 1em plus 0.5em minus 0.4em\relax ACM, 2005, pp. 557--566.

\bibitem{bressan2010choose}
M.~Bressan and E.~Peserico, ``Choose the damping, choose the ranking?''
  \emph{Journal of Discrete Algorithms}, vol.~8, no.~2, pp. 199--213, 2010.

\bibitem{chung2007heat}
F.~Chung, ``The heat kernel as the {P}age{R}ank of a graph,'' \emph{Proceedings
  of the National Academy of Sciences}, vol. 104, no.~50, pp. 19\,735--19\,740,
  2007.

\bibitem{chung2009local}
------, ``A local graph partitioning algorithm using heat kernel
  {P}age{R}ank,'' \emph{Internet Mathematics}, vol.~6, no.~3, pp. 315--330,
  2009.

\bibitem{arasu2002pagerank}
A.~Arasu, J.~Novak, A.~Tomkins, and J.~Tomlin, ``{P}age{R}ank computation and
  the structure of the web: Experiments and algorithms,'' in \emph{Proceedings
  of the 11th International Conference on World Wide Web}, 2002, pp. 107--117.

\bibitem{kullback1951information}
S.~Kullback and R.~A. Leibler, ``On information and sufficiency,'' \emph{The
  Annals of Mathematical Statistics}, vol.~22, no.~1, pp. 79--86, 1951.

\bibitem{lee2003fast}
C.~P.-C. Lee, G.~H. Golub, and S.~A. Zenios, ``A fast two-stage algorithm for
  computing {P}age{R}ank and its extensions,'' \emph{Scientific Computation and
  Computational Mathematics}, vol.~1, no.~1, pp. 1--9, 2003.

\bibitem{langville2006reordering}
A.~N. Langville and C.~D. Meyer, ``A reordering for the {P}age{R}ank problem,''
  \emph{SIAM Journal on Scientific Computing}, vol.~27, no.~6, pp. 2112--2120,
  2006.

\bibitem{dulmage1958coverings}
A.~L. Dulmage and N.~S. Mendelsohn, ``Coverings of bipartite graphs,''
  \emph{Canadian Journal of Mathematics}, vol.~10, no.~4, pp. 516--534, 1958.

\bibitem{google-net}
Google, ``Google programming contest,''
  \url{http://www.google.com/programming-contest/}, 2002.

\bibitem{kamvar2003extrapolation}
S.~D. Kamvar, T.~H. Haveliwala, C.~D. Manning, and G.~H. Golub, ``Extrapolation
  methods for accelerating {P}age{R}ank computations,'' in \emph{Proceedings of
  the 12th International Conference on World Wide Web}.\hskip 1em plus 0.5em
  minus 0.4em\relax ACM, 2003, pp. 261--270.

\bibitem{kamvar2004adaptive}
S.~Kamvar, T.~Haveliwala, and G.~Golub, ``Adaptive methods for the computation
  of {P}age{R}ank,'' \emph{Linear Algebra and its Applications}, vol. 386, pp.
  51--65, 2004.

\bibitem{DBLP:journals/corr/LofgrenBGS14}
P.~A. Lofgren, S.~Banerjee, A.~Goel, and C.~Seshadhri, ``{FAST-PPR:} scaling
  personalized pagerank estimation for large graphs,'' in \emph{Proceedings of
  the 20th ACM SIGKDD International Conference on Knowledge Discovery and Data
  Mining}.\hskip 1em plus 0.5em minus 0.4em\relax ACM, 2014, pp. 1436--1445.

\bibitem{kunegis2013konect}
J.~Kunegis, ``{KONECT}--the koblenz network collection,'' in \emph{Proceedings
  of the 22nd International Conference on World Wide Web}.\hskip 1em plus 0.5em
  minus 0.4em\relax ACM, 2013, pp. 1343--1350.

\bibitem{auer2007dbpedia}
S.~Auer, C.~Bizer, G.~Kobilarov, J.~Lehmann, R.~Cyganiak, and Z.~Ives,
  ``{DB}pedia: A nucleus for a web of open data,'' in \emph{The Semantic
  Web}.\hskip 1em plus 0.5em minus 0.4em\relax Springer, 2007, pp. 722--735.

\bibitem{Kwak10www}
H.~Kwak, C.~Lee, H.~Park, and S.~Moon, ``{W}hat is {T}witter, a social network
  or a news media?'' in \emph{Proceedings of the 19th International Conference
  on World Wide Web}.\hskip 1em plus 0.5em minus 0.4em\relax New York, NY, USA:
  ACM, 2010, pp. 591--600.

\bibitem{icwsm10cha}
M.~Cha, H.~Haddadi, F.~Benevenuto, and K.~P. Gummadi, ``{Measuring User
  Influence in Twitter: The Million Follower Fallacy},'' in \emph{Proceedings
  of the 4th International AAAI Conference on Weblogs and Social Media
  (ICWSM)}, Washington DC, USA, May 2010.

\bibitem{yang2015defining}
J.~Yang and J.~Leskovec, ``Defining and evaluating network communities based on
  ground-truth,'' \emph{Knowledge and Information Systems}, vol.~42, no.~1, pp.
  181--213, 2015.

\bibitem{MS-qian-2018}
Y.~Qian, ``Variable damping effect on network propagation,'' Master's thesis,
  Duke University, Durham, NC, USA, May 2018.

\bibitem{MS-Xichen-2018}
X.~Chen, ``Exploiting common structures across multiple network propagation
  schemes,'' Master's thesis, Duke University, Durham, NC, USA, May 2018.

\bibitem{ilprints582}
T.~Haveliwala and S.~Kamvar, ``The second eigenvalue of the google matrix,''
  Stanford InfoLab, Technical Report 2003-20, 2003.

\bibitem{richardson2002intelligent}
M.~Richardson and P.~Domingos, ``The intelligent surfer: Probabilistic
  combination of link and content information in {P}age{R}ank,'' in
  \emph{Advances in Neural Information Processing Systems}, 2002, pp.
  1441--1448.

\bibitem{haveliwala1999efficient}
T.~Haveliwala, ``Efficient computation of {P}age{R}ank,'' Stanford, Tech. Rep.,
  1999.

\bibitem{langville2011google}
A.~N. Langville and C.~D. Meyer, \emph{Google's {P}age{R}ank and beyond: The
  science of search engine rankings}.\hskip 1em plus 0.5em minus 0.4em\relax
  Princeton University Press, 2011.

\bibitem{jeh2003scaling}
G.~Jeh and J.~Widom, ``Scaling personalized web search,'' in \emph{Proceedings
  of the 12th International Conference on World Wide Web}.\hskip 1em plus 0.5em
  minus 0.4em\relax ACM, 2003, pp. 271--279.

\bibitem{chung2010pagerank}
F.~Chung and W.~Zhao, ``{P}age{R}ank and random walks on graphs,'' in
  \emph{Fete of Combinatorics and Computer Science}.\hskip 1em plus 0.5em minus
  0.4em\relax Springer, 2010, pp. 43--62.

\end{thebibliography}

% flatex input end: [pageRank-MAIN.bbl]
%FLATEX-REM:	\bibliographystyle{abbrv}
\fi
%FLATEX-REM:\bibliography{pageRank-refs.bib}

\end{document}



%%% Local Variables: 
%%% mode: latex
%%% TeX-master: t
%%% End: 

% flatex input end: [pageRank-MAIN.tex]
