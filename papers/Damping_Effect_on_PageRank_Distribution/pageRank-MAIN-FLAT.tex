% flatex input: [pageRank-MAIN.tex]
% HPEC2018-pageRank-MAIN.tex x
%
% initial draft : April 25, 2018 
% 
% revision : 
% 
% manuscript submission: May???, 2018
% review comments: see directory ???? 
% final paper submission, July ??, 2018   
% 
%

\def\hpec{true}

\ifdefined\hpec         % ----- IEEE HPEC conference proceedings

  \documentclass[conference]{IEEEtran}

\else                   % ----- draft (default)

  \documentclass[12pt,draft]{article}

  \def\usenumberlines{true}

\fi

\makeatletter
\def\endthebibliography{%
  \def\@noitemerr{\@latex@warning{Empty `thebibliography' environment}}%
  \endlist
}
\makeatother

\def\BibTeX{{\rm B\kern-.05em{\sc i\kern-.025em b}\kern-.08em
    T\kern-.1667em\lower.7ex\hbox{E}\kern-.125emX}}

%%%%%%%%%%%%%%%%%%%%%%%%%%%%%%%%%%%%%%%%%%%%%%%%%%
%%% PREAMBLE

% packages, customisations, macros, etc.

% flatex input: [pageRank-packages.tex]
% hpec2013-gpixel-packages.tex


% ------------------------------ packages

% page/space formatting
\ifdefined\hpec         % ----- IEEE HPEC
  \usepackage{balance}
\else                   % ----- draft
  \usepackage[margin=1in]{geometry}
  \usepackage{setspace}
\fi

% math
\usepackage{amsmath}
\usepackage{amsfonts}
\usepackage{amssymb}
\usepackage{amsthm}
\usepackage{mathrsfs}
\usepackage{array}
% \usepackage{bigdelim}
% \usepackage{breqn}

% floats
% \usepackage[section]{placeins}
\usepackage{float}
% \usepackage{listings}
% \usepackage[section]{algorithm}
% \usepackage{algpseudocode}
\usepackage[font=small,labelfont=bf,hypcap=true]{caption}
%\usepackage[font=small,labelfont=bf,hypcap=true]{subcaption}
% \usepackage{sidecap}

% graphics
\usepackage{graphicx,subfigure}
\usepackage[hyperref,x11names,cmyk,pdftex]{xcolor}
\usepackage{tikz}
\usepackage{float}
\usepackage{subfloat}
\usepackage{tablefootnote}
\usepackage{adjustbox}
% \usepackage{pgfplots}

% tables
\usepackage{multirow}
\usepackage{booktabs}

% lists
\usepackage{enumerate}
\usepackage{paralist}

% header and footer
% \usepackage{fancyhdr}

% footnotes
\usepackage[bottom]{footmisc}

% ordinal numbers
\usepackage[super]{nth}

% fonts
\usepackage{helvet}
\usepackage{textcomp}
\usepackage[utf8]{inputenc}
\usepackage[resetfonts]{cmap}

% editing facilities
\usepackage[layout={inline},author=]{fixme}


% algorithm
\usepackage{algorithm,algorithmic}

% code
% \usepackage{mcode}

% references
\usepackage{csquotes}
% \usepackage[hyphens]{url}
\ifdefined\hpec
	\usepackage[pdftex,bookmarks=false]{hyperref}
	% \usepackage[style=numeric,natbib,backend=bibtex8]{biblatex}
	% \usepackage[numbers,square]{natbib}
	\usepackage{cite}
\else                   % ----- draft
  \usepackage[pdftex,breaklinks=true,bookmarks=true]{hyperref}

  \ifdefined\usenumberlines

    % add linenumbers
    \usepackage{lineno}
    \linenumbers\relax

    \newcommand*\patchAmsMathEnvironmentForLineno[1]{%
      \expandafter\let\csname old#1\expandafter\endcsname\csname #1\endcsname
      \expandafter\let\csname oldend#1\expandafter\endcsname\csname end#1\endcsname
      \renewenvironment{#1}%
      {\linenomath\csname old#1\endcsname}%
      {\csname oldend#1\endcsname\endlinenomath}}% 
    \newcommand*\patchBothAmsMathEnvironmentsForLineno[1]{%
      \patchAmsMathEnvironmentForLineno{#1}%
      \patchAmsMathEnvironmentForLineno{#1*}}%
    \AtBeginDocument{%
      \patchBothAmsMathEnvironmentsForLineno{equation}%
      \patchBothAmsMathEnvironmentsForLineno{align}%
      \patchBothAmsMathEnvironmentsForLineno{flalign}%
      \patchBothAmsMathEnvironmentsForLineno{alignat}%
      \patchBothAmsMathEnvironmentsForLineno{gather}%
      \patchBothAmsMathEnvironmentsForLineno{multline}%
    }

  \fi

\fi




%%% Local Variables: 
%%% mode: latex
%%% TeX-master: "HPEC2013-gpixel-main"
%%% End: 

% flatex input end: [pageRank-packages.tex]

% packages, customisations, macros, etc.
% flatex input: [title-authors.tex]
% title-authors.tex 

\title{\LARGE Damping Effect on PageRank Distribution}

\ifdefined\hpec
\author{%
  \IEEEauthorblockN{% 
    Tiancheng Liu,
    Yuchen Qian,
    Xi Chen, and
    Xiaobai Sun
    \\ 
    \IEEEauthorblockA{%
    \begin{tabular}{c}
      Department of Computer Science, Duke University, Durham, NC 27708, USA
    \end{tabular}%
  }%
}}

\else
\fi 

% Author's comments
\newcommand{\xiaobai}[1]{\textcolor{blue}{Xiaobai:
    #1}\PackageWarning{Comment:}{Xiaobai: #1!}}
\newcommand{\xichen}[1]{\textcolor{red}{XiChen:
    #1}\PackageWarning{Comment:}{Xi Chen: #1!}}
\newcommand{\tiancheng}[1]{\textcolor{orange}{Tiancheng:
    #1}\PackageWarning{Comment:}{Tiancheng: #1!}}
\newcommand{\yuchen}[1]{\textcolor{green}{Yuchen:
    #1}\PackageWarning{Comment:}{Yuchen: #1!}}

% flatex input end: [title-authors.tex]
 
% packages, customisations, macros, etc.

% flatex input: [pageRank-customization.tex]
% authors (separate from hyperref metadata for easy "hiding")
\newcommand{\pdfauthors}{%
  T. Liu, Y. Qian, X. Chen, X. Sun}


% PDF metadata
\hypersetup{%
  pdffitwindow=false,%
  pdfstartview={FitH},%
  colorlinks,%
  pdfauthor={\pdfauthors},%
  pdftitle={Damping Effect on PageRank Distribution},%
  citecolor=PaleGreen4,%
  filecolor=DarkOrchid4,%
  linkcolor=OrangeRed4,%
  urlcolor=black%
}

% always show figures (even in draft mode)
\setkeys{Gin}{draft=false}

% fix Linux Adobe Acrobat Reader's issue with transparent image elements
\pdfpageattr{/Group <</S /Transparency /I true /CS /DeviceRGB>>}

% remove extra spacing around \left and \right delimiters
\let\leftorig\left
\let\rightorig\right
\renewcommand{\left}{\mathopen{}\mathclose\bgroup\leftorig}
\renewcommand{\right}{\aftergroup\egroup\rightorig}

% increase vertical spacing between table rows
\renewcommand{\arraystretch}{1.2}

% Enumerate Tables in Arabic
\renewcommand{\thetable}{\arabic{table}}

% customize the use of editing notes
\fxsetup{inline}
\fxusetheme{color}
\newcommand{\note}[1]{\fxerror{[#1]}}
\newcommand{\remove}[1]{\fxwarning{\emph{rm}:[#1]}}

% show hyperlinks even when in draft mode
\hypersetup{final}

% allow hyphenated url strings (without using the 'url' package
\makeatletter
\DeclareRobustCommand\ttfamily{%
  \not@math@alphabet\ttfamily\mathtt
  \fontfamily\ttdefault\selectfont\hyphenchar\font=-1\relax}
\makeatother
\DeclareTextFontCommand{\tturl}{\ttfamily\hyphenchar\font=`/}


%%% Local Variables: 
%%% mode: latex
%%% TeX-master: "../wknn-main"
%%% End: 

% flatex input end: [pageRank-customization.tex]

% packages, customisations, macros, etc.

%%%%%%%%%%%%%%%%%%%%%%%%%%%%%%%%%%%%%%%%%%%%%%%%%%
%%% DOCUMENT

\begin{document}


% --------- make TITLE
% double line spacing for draft mode

\ifdefined\hpec\else
  \doublespacing
\fi


%%% ====================================================
%%% FRONT MATTER


% -------------------- IEEE HPEC
%
\ifdefined\hpec
  \maketitle

% -------------------- draft
%
\else

  % ----- title

  \pdfbookmark[1]{Title}{sec:title}
  \maketitle

  % ----- table of contents

  \phantomsection
  \pdfbookmark[1]{Contents}{sec:contents}
  \tableofcontents
  \clearpage

\fi

% ============  ABSTRACT

% flatex input: [abstract.tex]
% abstract.tex

\begin{abstract}
  This work extends the personalized PageRank model invented by Brin and
  Page to a family of PageRank models with various damping schemes. The
  goal with increased model variety is to capture or recognize a larger
  number of types of network activities, phenomenons and
  propagation patterns. The response in PageRank distribution
  to variation in damping mechanism is then characterized analytically,
  and further estimated quantitatively on $6$ large real-world link
  graphs. The study leads to new observation and empirical findings. It
  is found that the difference in the pattern of PageRank vector
  responding to parameter variation by each model among the $6$ graphs is
  relatively smaller than the difference among $3$ particular models
  used in the study on each of the graphs. This suggests the utility of
  model variety for differentiating network activities and propagation
  patterns. The quantitative analysis of the damping mechanisms over 
  multiple damping models and parameters is facilitated by a highly efficient algorithm,
  which calculates all PageRank vectors at once via a commonly shared,
  spectrally invariant subspace. The spectral space is
  found to be of low dimension for each of the real-world graphs.
\end{abstract}

% flatex input end: [abstract.tex]

% ============  ABSTRACT


% ============= INTRODUCTION

\section{Introduction}
\label{sec:intro}
% Reinforcement learning has achieved great success in areas such as Game-playing \citep{silver2018general,vinyals2019grandmaster}, robotics \cite{kober2013reinforcement}, large language models \citep{ouyang2022training}, etc.
However, due to safety concerns or physical limitations, in some real-world reinforcement learning problems, we must consider additional constraints that may influence the optimal policy and the learning process \citep{garcia2015comprehensive}.
% For example, a robotic arm must not take actions that may cause harm to itself or the environments.
A standard framework to handle such cases is the constrained Markov Decision Process (CMDP) \citep{altman1999constrained}.
Within the CMDP framework, the agent has to maximize
the expected cumulative reward while
obeying a finite number of constraints, which are usually in the form of expected cumulative cost criteria.

However, we are sometimes concerned with the problem with a continuum of constraints.
For example,
the constraints we meet might be time-evolving or subject to uncertain parameters, which
cannot be formulated as an ordinary CMDP
(see Examples \ref{Example_Time_Evolving} and  \ref{Example_Uncertain}).
In this paper we would study a generalized CMDP  
to address the above problem.  Because the constraints are not only infinite-number but also lie
in a continuous set,
the generalization is not trivial. Fortunately, we find that we can borrow the idea behind semi-infinite programming (SIP) \citep{remez1934determination, hettich1993semi} to deal with the semi-infinite constraints.
Accordingly, we propose \emph{semi-infinitely constrained Markov decision processes} (SICMDPs)
as a novel complement to the ordinary CMDP framework.
%More specifically,  an SICMDP model %, we consider 
%contains a continuum of constraints whereas an ordinary CMDP contains a finite number of constraints. 

%This generalization is natural but not trivial. However, we can brows the idea  
%The idea is quite natural and can be backtracked
%to the practice of extending linear programming to linear semi-infinite programming (LSIP) %\cite{remez1934determination, GobernaLSIO1998}.
%In addition, 
%As a complementary approach to the ordinary CMDP framework, 
%SICMDP can be used to model these problems  which cannot be described by a finite number of constraints
%that are not covered by .
%For example,
%the restrictions we consider can be time-evolving or subject to uncertain parameters
%, thus
%cannot be described by a finite number of constraints but a continuum of constraints 
%(see Examples \ref{Example_Time_Evolving} and  \ref{Example_Uncertain}).

We also present two reinforcement learning algorithms to solve SICMDPs called SI-CRL and SI-CPO, respectively.
SI-CRL is a model-based reinforcement learning algorithm designed for tabular cases, and SI-CPO is a policy optimization algorithm for non-tabular cases.
% and analyze its performance both theoretically and empirically.
The main challenge is that we need to deal with a continuum of constraints, thus reinforcement learning algorithms for ordinary CMDPs do not work anymore.
In SI-CRL, we tackle this difficulty by first transforming the reinforcement learning problem to an equivalent LSIP problem, which can then be solved using methods in the LSIP literature like the dual exchange methods \citep{Hu1990,reemtsen1998numerical}.
In SI-CPO, we resort to the idea of cooperative stochastic approximation developed in \cite{lan2020algorithms, wei2020comirror}.
As far as we know, we are the first to introduce tools from semi-infinitely programming (SIP) into the reinforcement learning community for solving constrained reinforcement learning problems.

% To the best of our knowledge, we are the first to apply tools from semi-infinitely programming (SIP) to solve reinforcement learning problems.
Furthermore, we give theoretical analysis for both SI-CRL and SI-CPO.
We decompose the error of SI-CRL into two parts: the statistical error from approximating the true SICMDP with an offline dataset and the optimization error due to the fact that the solution of the LSIP problem obtained by the dual exchange method is inexact.
On the optimization side, we show that the iteration complexity of SI-CRL is $O\left(\left\{\mathrm{diam}(Y)L\sqrt{|\gS|^2|\gA|m}/\left[(1-\gamma)\epsilon\right]\right\}^m\right)$.
On the statistical side, we show that the sample complexity of SI-CRL is $\widetilde O\left(\frac{|S|^2|A|^2}{\epsilon^2(1-\gamma)^3}\right)$ if the offline dataset is generated by a generative model, and $\widetilde O\left(\frac{|S||A|}{\nu_{\min} \epsilon^2(1-\gamma)^3}\right)$ if the dataset is generated by a probability measure $\nu$ as considered in \cite{chen2019information}.
Here $\widetilde O$ means that all logarithm terms are discarded.
For SI-CPO, things become a little more complicated because other than the statistical error and the optimization error, we also need to consider the function approximation error, which comes from imperfect policy parametrizations.
It is shown if the function approximation error can be controlled to $O(\epsilon)$ order, the iteration complexity of SI-CPO is $\widetilde{O}\left(\frac{1}{\epsilon^2(1-\gamma)^6}\right)$ and the sample complexity of SI-CPO is $\widetilde{O}(\frac{1}{\epsilon^4(1-\gamma)^{10}})$.
Here our iteration complexity bound is equivalent to a typical $\widetilde O(1/\sqrt{T})$ global convergence rate.

We perform a set of numerical experiments to illustrate the SICMDP model and validate our proposed algorithms.
Specifically, we examine two numerical examples, namely the discharge of sewage and ship route planning.
Through the discharge of sewage example, we show the advantage of the SICMDP framework over the CMDP baseline obtained by naive discretization in modeling realistic sequential decision-making problems.
Moreover, we demonstrate the effectiveness of the SI-CRL and SI-CPO algorithms in such tabular environments. 
In the ship route planning example, we illustrate the benefits of the SICMDP framework and the ability of the SI-CPO algorithm to address complex continuous control tasks involving continuous state spaces with modern deep reinforcement learning techniques.

% In summary, our contributions are listed as follows.
% First, we present the SICMDP model, which can be viewed as a generalization of the ordinary CMDP model.
% Second, we propose an algorithm to perform reinforcement learning for SICMDPs, which is called SI-CRL, and we believe that we are the first to apply tools from SIP
% to solve reinforcement learning problems.
% Third, we give a theoretical analysis of SI-CRL and identify both its sample complexity and iteration complexity.
% In addition, we perform numerical experiments to illustrate the SICMDP model and validate the SI-CRL algorithm.
% \{This paragraph can be removed!!! \}






% flatex input: [introduction.tex]
% introduction.tex 

Personalized PageRank, invented by Brin and Page~\cite{page1999pagerank,
  ilprints361}, revolutionized the way we model any particular type of
activities on a large information network. It is also intended to be
used as a mechanism to counteract malicious manipulation of the
network~\cite{page1999pagerank, ilprints361, sheldon2010manipulation}. 
PageRank has underlain Google's
search architecture, algorithms, adaptation strategies and ranked page
listing upon query.  It has influenced the development of other
search engines and recommendation systems, such as topic-sensitive
PageRank\cite{haveliwala2003topic}. Its impact reaches far beyond
digital and social networks. For example, GeneRank is used for
generating prioritized gene
lists\cite{morrison2005generank,wu2010krylov}. The seminal
paper~\cite{page1999pagerank} itself is directly cited more than ten
thousands times as of today.  As surveyed in
\cite{langville2004deeper,berkhin2005survey}, a lot of efforts were made
to accelerate the calculation of personalized PageRank vectors, in part
or in whole \cite{ilprints596,ilprints579}. Certain investigation were
carried out to assess the variation in PageRank vector in response to
varying damping
parameter~\cite{boldi2005pagerank,bressan2010choose}. Most efforts on
PageRank study, however, are ad hoc to the Brin-Page model. Chung made
a departure by introducing a diffusion-based PageRank model and applied
it to graph cuts~\cite{chung2007heat,chung2009local}.

In this paper we follow Brin and Page in the modeling aspect that
warrants more attention as the variety of networks and activities on the
networks increases incessantly.  We extend the model scope to capture
more network activities in a probabilistic sense. We study the damping
effect on PageRank distribution.  We consider the holistic distribution
because it serves as the statistical reference for inferring conditional
page ranking upon query. Our study has three intellectual merits with 
practical impact.
%
\begin{inparaenum}[(1)]
\item A family of damping models, which includes and connects the
  Brin-Page model and Chung's model. The family admits more
  probabilistic descriptions of network activities. 
%
\item A unified analysis of damping effect on personalized PageRank
  distribution, with parameter variation in each model and comparison
  across models. The analysis provides a new insight into the solution
  space and solution methods.
%
\item A highly efficient method for calculating the solutions to all
  models under consideration at once, particular to a network and a
  personalized vector.  Our quantitative analysis of $6$ real-world
  network graphs leads to new findings about the models and networks under
  study, which we present and discuss in Sections~\ref{sec:theory-analysis}
  and \ref{sec:numerical-experiments}.
%
%
\end{inparaenum} 
%
Our modeling and analysis methods can be potentially used for
recognizing and estimating activity or propagation patterns on a
network, provided with monitored data.
% 
%

% flatex input end: [introduction.tex]
 
% Reinforcement learning has achieved great success in areas such as Game-playing \citep{silver2018general,vinyals2019grandmaster}, robotics \cite{kober2013reinforcement}, large language models \citep{ouyang2022training}, etc.
However, due to safety concerns or physical limitations, in some real-world reinforcement learning problems, we must consider additional constraints that may influence the optimal policy and the learning process \citep{garcia2015comprehensive}.
% For example, a robotic arm must not take actions that may cause harm to itself or the environments.
A standard framework to handle such cases is the constrained Markov Decision Process (CMDP) \citep{altman1999constrained}.
Within the CMDP framework, the agent has to maximize
the expected cumulative reward while
obeying a finite number of constraints, which are usually in the form of expected cumulative cost criteria.

However, we are sometimes concerned with the problem with a continuum of constraints.
For example,
the constraints we meet might be time-evolving or subject to uncertain parameters, which
cannot be formulated as an ordinary CMDP
(see Examples \ref{Example_Time_Evolving} and  \ref{Example_Uncertain}).
In this paper we would study a generalized CMDP  
to address the above problem.  Because the constraints are not only infinite-number but also lie
in a continuous set,
the generalization is not trivial. Fortunately, we find that we can borrow the idea behind semi-infinite programming (SIP) \citep{remez1934determination, hettich1993semi} to deal with the semi-infinite constraints.
Accordingly, we propose \emph{semi-infinitely constrained Markov decision processes} (SICMDPs)
as a novel complement to the ordinary CMDP framework.
%More specifically,  an SICMDP model %, we consider 
%contains a continuum of constraints whereas an ordinary CMDP contains a finite number of constraints. 

%This generalization is natural but not trivial. However, we can brows the idea  
%The idea is quite natural and can be backtracked
%to the practice of extending linear programming to linear semi-infinite programming (LSIP) %\cite{remez1934determination, GobernaLSIO1998}.
%In addition, 
%As a complementary approach to the ordinary CMDP framework, 
%SICMDP can be used to model these problems  which cannot be described by a finite number of constraints
%that are not covered by .
%For example,
%the restrictions we consider can be time-evolving or subject to uncertain parameters
%, thus
%cannot be described by a finite number of constraints but a continuum of constraints 
%(see Examples \ref{Example_Time_Evolving} and  \ref{Example_Uncertain}).

We also present two reinforcement learning algorithms to solve SICMDPs called SI-CRL and SI-CPO, respectively.
SI-CRL is a model-based reinforcement learning algorithm designed for tabular cases, and SI-CPO is a policy optimization algorithm for non-tabular cases.
% and analyze its performance both theoretically and empirically.
The main challenge is that we need to deal with a continuum of constraints, thus reinforcement learning algorithms for ordinary CMDPs do not work anymore.
In SI-CRL, we tackle this difficulty by first transforming the reinforcement learning problem to an equivalent LSIP problem, which can then be solved using methods in the LSIP literature like the dual exchange methods \citep{Hu1990,reemtsen1998numerical}.
In SI-CPO, we resort to the idea of cooperative stochastic approximation developed in \cite{lan2020algorithms, wei2020comirror}.
As far as we know, we are the first to introduce tools from semi-infinitely programming (SIP) into the reinforcement learning community for solving constrained reinforcement learning problems.

% To the best of our knowledge, we are the first to apply tools from semi-infinitely programming (SIP) to solve reinforcement learning problems.
Furthermore, we give theoretical analysis for both SI-CRL and SI-CPO.
We decompose the error of SI-CRL into two parts: the statistical error from approximating the true SICMDP with an offline dataset and the optimization error due to the fact that the solution of the LSIP problem obtained by the dual exchange method is inexact.
On the optimization side, we show that the iteration complexity of SI-CRL is $O\left(\left\{\mathrm{diam}(Y)L\sqrt{|\gS|^2|\gA|m}/\left[(1-\gamma)\epsilon\right]\right\}^m\right)$.
On the statistical side, we show that the sample complexity of SI-CRL is $\widetilde O\left(\frac{|S|^2|A|^2}{\epsilon^2(1-\gamma)^3}\right)$ if the offline dataset is generated by a generative model, and $\widetilde O\left(\frac{|S||A|}{\nu_{\min} \epsilon^2(1-\gamma)^3}\right)$ if the dataset is generated by a probability measure $\nu$ as considered in \cite{chen2019information}.
Here $\widetilde O$ means that all logarithm terms are discarded.
For SI-CPO, things become a little more complicated because other than the statistical error and the optimization error, we also need to consider the function approximation error, which comes from imperfect policy parametrizations.
It is shown if the function approximation error can be controlled to $O(\epsilon)$ order, the iteration complexity of SI-CPO is $\widetilde{O}\left(\frac{1}{\epsilon^2(1-\gamma)^6}\right)$ and the sample complexity of SI-CPO is $\widetilde{O}(\frac{1}{\epsilon^4(1-\gamma)^{10}})$.
Here our iteration complexity bound is equivalent to a typical $\widetilde O(1/\sqrt{T})$ global convergence rate.

We perform a set of numerical experiments to illustrate the SICMDP model and validate our proposed algorithms.
Specifically, we examine two numerical examples, namely the discharge of sewage and ship route planning.
Through the discharge of sewage example, we show the advantage of the SICMDP framework over the CMDP baseline obtained by naive discretization in modeling realistic sequential decision-making problems.
Moreover, we demonstrate the effectiveness of the SI-CRL and SI-CPO algorithms in such tabular environments. 
In the ship route planning example, we illustrate the benefits of the SICMDP framework and the ability of the SI-CPO algorithm to address complex continuous control tasks involving continuous state spaces with modern deep reinforcement learning techniques.

% In summary, our contributions are listed as follows.
% First, we present the SICMDP model, which can be viewed as a generalization of the ordinary CMDP model.
% Second, we propose an algorithm to perform reinforcement learning for SICMDPs, which is called SI-CRL, and we believe that we are the first to apply tools from SIP
% to solve reinforcement learning problems.
% Third, we give a theoretical analysis of SI-CRL and identify both its sample complexity and iteration complexity.
% In addition, we perform numerical experiments to illustrate the SICMDP model and validate the SI-CRL algorithm.
% \{This paragraph can be removed!!! \}







% ============= MODELS

\section{PageRank models}
\label{sec:pageRank-models}
%
% flatex input: [pageRank-models.tex]

% pageRank-models.tex

We first review briefly two precursor models and then introduce a family
of PageRank models.


% ========= BP mpdel
\subsection{Brin-Page model} 
\label{subsec:Brin-Page-model}
%
% flatex input: [sections/bp-model.tex]
% bp-models.tex

%% --- dscribe P matrix first % 
Brin and Page describe a network of webpages as a link graph, which is
represented by a stochastic matrix $P$~\cite{page1999pagerank}. We adopt
the convention that $P$ is stochastic columnwise. Every webpage is a
node with (outgoing) links, i.e., edges, to some other webpages and with
incoming edges or backlinks as citations to the page. If page $j$ has
$n_j >0 $ outgoing links, then in column $j$ of $P$, $P_{i,j} = 1/n_j$
if page $j$ has a link to page $i$; $P_{i,j}=0$, otherwise. In row $i$
of $P$, every nonzero element $P_{i,j}$ corresponds to a backlink from
$j$ to $i$.
%
%% ---- describe M matrix next % 
In the Brin-Page model, the web user behavior is described as a random
walk on a personalized Markov chain (i.e., a discrete-time Markov chain)
associated with the following probability transition matrix
% 
\begin{equation}
\label{eqn:bp-Bernoulli}
M_{\alpha}(v) = \alpha P + (1 - \alpha)v e^{\rm T}, 
\quad \alpha \in (0,1), 
\quad e^{\rm T}v = 1, 
\end{equation}%
%
where $v\geq 0$ is a personalized or customized distribution/vector, $e$
denotes the vector with all elements equal to $1$, and the \emph{damping
  factor} $\alpha$ describes a Bernoulli decision process.  At each
step, with probability $\alpha$, the web user follows an outlink; or
with probability $(1-\alpha)$, the user jumps to any page by the personalized
distribution $v$.  The personalized transportation term is
innovative. It customizes the Markov chain with respect to a particular
type of relevance.
%
In this paper we assume that the personalized vector $v$ is given and
fixed, and focus on investigating the damping effect. In particular, 
with Brin-Page model, we focus on the role of $\alpha$.
The notation for the Markov chain may thus be simplified to
$M_{\alpha}$, or $M$ when $\alpha$ is clear from the context.

Page ranking upon a search query depends on the stationary PageRank
distribution, denoted by $x=x(\alpha)$, of the Markov chain:
% 
\begin{equation}
\label{eqn:ppr-markov-equi}
  M_{\alpha}  \,  x = x, 
  \quad e^{\rm T}x = 1. 
\end{equation}%
%
Arasu et al\cite{arasu2002pagerank} recast the eigenvector equation
(\ref{eqn:ppr-markov-equi}) to a linear system to solve for $x$,
% 
\begin{equation}
\label{eqn:ppr-linear}
(I - \alpha P)x = (1 - \alpha)v, 
\end{equation}
%
where $I$ is the identity matrix.
Because $\alpha \|P\|_{1} < 1$, the solution can be expressed via the 
Neumann series for the inverse of $(I-\alpha P)$, 
% 
\begin{equation}
\label{eqn:bp-solution}
x(\alpha)  = (1-\alpha) \sum_{k = 0} ^ {\infty} \alpha^k P^k v.
\end{equation}
%
The weights $(1-\alpha)\alpha^k$ decrease by the factor $\alpha$ from one step to the
next. In \cite{page1999pagerank}, Brin and Page set $\alpha$ to $0.85$.

In (\ref{eqn:bp-solution}), the solution to Brin-Page model with
$\alpha \in (0,1)$ is not the stationary distribution of network $P$. It
is analyzed in terms of steps on $P$. Term $k$ represents the probabilistic
accumulation of the Bernoulli decision process at each step by
(\ref{eqn:bp-Bernoulli}) to step $k$, $k\geq 0$.  Every step has its
print in distribution $x(\alpha)$.




% flatex input end: [sections/bp-model.tex]

%


% ========== CHung's model
\subsection{Chung's model} 
\label{subsec:Chung-model}
% flatex input: [sections/chung-model.tex]
% chng-model.tex


Chung introduced a PageRank model~\cite{chung2007heat} in the form of a
heat or diffusion equation, with $v$ as the initial distribution,
% 
\begin{equation}
\label{eqn:heat-model}
  \frac{\partial x}{\partial \beta} = - (I - P)x, \quad x(0) = v,
\end{equation}
%
where $( I - P ) $ is the Laplacian of the link graph, and we use
$\beta$ to denote the time variable.
%
From the viewpoint of probabilistic theory, model (\ref{eqn:heat-model})
is underlined by the Kolmogorov's backward equation system for a continuous-time
Markov chain with $-(I-P)$ as the transition rate matrix and with the
identity matrix $I$ as the initial transition matrix.
% 
The solution to (\ref{eqn:heat-model}) is 
%
\begin{equation}
\label{eqn:heat-solution}
x(\beta) 
= e^{-\beta(I-P)}v 
= e^{-\beta} \sum_{k = 0} ^ {\infty} \frac{\beta^k}{k!} P^k v. 
\end{equation}
%

% flatex input end: [sections/chung-model.tex]

% ========== CHung's model


% ========== Model family
\subsection{A model family}
\label{subsec:model-family}
%
% flatex input: [sections/model-family.tex]
% model-family.tex 

We introduce a family of PageRank models.  
Each member model is characterized by a scalar \emph{damping
  variable} $\rho$ and a {\em discrete} probability mass function (pmf)
$ w(\rho) = \{ w_k=w_k(\rho), \ k \in \mathbb{N}_w \} $. The support
$ \mathbb{N}_{w} \subset \mathbb{N} $ may be finite or infinite.  There
are a few equivalent expressions to describe our models.
% 
We start by defining the model with a kernel function $f(\lambda, \rho)$, 
%
\begin{equation}
\label{eqn:uni-random-walk} 
f(\lambda, \rho) = \sum_{k \in N_w} w_k(\rho)\, \lambda^k, 
\quad | \lambda | \leq 1.   
\end{equation}%
%
The solution specific to network graph $P$ and personalized vector $v$ is, 
%
\begin{equation}
\label{eqn:uni-random-walk} 
x_f(\rho)   = f(P)v = \left(\sum_{k \in N_w} w_k(\rho)\, P^k \right) v . 
\end{equation}%
%
The matrix function $f(P)$ is stochastic. The rank
distribution vector $x_f$ is the superposition of step terms with
probabilistic weights $w_k$. The step term $k$ describes the
probabilistic propagation of $v$ at step $k$.
% 
For a specific case, the damping variable may have a designated label,
with a specific range, and the pmf may have a specific support and
additional parameters.  For convenience, we assume $\mathbb{N}$ as the
support. Over the infinite support, the damping weights must decay after
certain number of steps and vanish as $k$ goes to infinity. In theory, 
every discrete pmf can be used as a model kernel in (\ref{eqn:uni-random-walk}).
In practice, each describes a particular type of activity or propagation.
%

The family includes Brin-Page model and Chung's model.  For the former,
the damping variable is denoted by $\alpha$, the damping weights
$(1-\alpha)\alpha^k$, $k\geq 0$, follow the geometric distribution with
the expected value $\alpha(1-\alpha)^{-1}$. The kernel function
is $ (1-\alpha)(1-\alpha\lambda)^{-1}$.  For Chung's model, we
denote the damping variable by $\beta$, $\beta > 0$. The damping weights
$e^{-\beta}\beta^k/k!$, $k\geq 0$, follow the Poisson distribution with
the expected value $\beta$. The model's kernel function is
$e^{-\beta(1-\lambda)}$.

We describe a few other models, among many, in the family. In fact, the
precursor models are two special cases of the model associated with the
Conway-Maxwell-Poisson (CMP) distribution, which has an additional
parameter $\nu$ to the pmf,
% 
\begin{equation} 
\notag
\label{eqn:CMP-weights}
  w_k( \rho, \nu) 
= \frac{\rho^{k}}{(k!)^{\nu}\, Z}, 
 \quad \nu > 0, 
\end{equation}
%
where $Z$ is the normalization scalar, and $\nu$ is the decay
rate parameter. The case with $\nu=0$ is the geometric distribution; the
case with $\nu=1$ is the Poisson distribution. If the value of $\rho$ is
fixed, the weights decay faster with a larger value of $\nu$.
%
The negative binomial distribution, or the Pascal distribution, also
includes the geometric distribution as a special case. It includes
other cases that 
render damping weights with slower decay rates.
%

In the rest of the paper, for the purpose of including and illustrating
new models, we use the model associated with the logarithmic
distribution, for $\gamma \in (0,1)$, 
%
\begin{equation}
\label{eqn:log-gamma} 
 f(\lambda,\gamma) =
    \frac{-1}{\ln(1 - \gamma)} 
    \sum_{k = 1}^{\infty} \frac{(\gamma\lambda)^{k}}{k}
   = \frac{\ln(1 - \gamma\lambda)}{\ln(1 - \gamma)}. 
\end{equation}%
%
The weights decrease slightly faster than the geometrically distributed
ones, but not in the CMP distribution class. 

We now present the system of linear equations with $x(\rho)$ in 
(\ref{eqn:uni-random-walk}) as the solution, 
% 
\begin{equation}
\label{eq:model-equation} 
A(P)x = v, 
\quad A(P) = f^{-1}(P). 
\end{equation}%
%
The matrix $A$ is an $M$ matrix. In particular, 
$A = (1-\alpha)^{-1}(I - \alpha P)$ for Brin-Page model,
$A = e^{\beta (I - P)}$ for Chung's model and
$A= \ln(1-\gamma) \, \ln^{-1}(I - \gamma P) $ for the log-$\gamma$
model (\ref{eqn:log-gamma}).
% 
The algebraic model expression (\ref{eq:model-equation}) will be used
next for the model expression in a differential equation.




% flatex input end: [sections/model-family.tex]

%



% flatex input end: [pageRank-models.tex]
 
%

% ============= ANALYSIS

\section{Response to variation in damping}
\label{sec:theory-analysis}
%
% flatex input: [response-variation-damping.tex]
% variation-analysis.tex


We provide a unified analysis of the response in PageRank distribution
to the variation in the damping parameter value as well as to the
change, or connection, from one model to another.

% ================ intra-model damping 
\subsection{Intra-model damping variation}
\label{subsec:pagerank-sensitive}
% flatex input: [sections/intra-model-variation.tex]
% intra-model-variation.tex

By (\ref{eqn:uni-random-walk}), we obtain the trajectory of the PageRank
vector $x(\rho)$ with the change in the damping variable $\rho$,
% 
\begin{equation}
\label{eqn:trajectory-equation}
 \dot{x}(\rho) = \frac{d x(\rho)}{d \rho} 
 = \frac{\partial }{\partial \rho}f(P) v = Q(P) \, x(\rho), 
\end{equation}%
%
where $Q(P) = \frac{\partial }{\partial \rho}f(P) f^{-1}(P)$ by
(\ref{eq:model-equation}), which we may refer to as the $\rho$-transition matrix.
%
Equation (\ref{eqn:trajectory-equation}) generalizes Chung's diffusion
model (\ref{eqn:heat-model}), in which the $\beta$-transition matrix
$Q=-(I-P)$ is independent of $\beta$.
%
For the Brin-Page model with damping variable $\alpha$,
%
\begin{equation}
\label{eqn:bp-trajectory-equationpde}
  Q(\alpha) = 
  \left[ P(I - \alpha P)^{-1} - (1-\alpha)^{-1} I\right]. 
\end{equation} 
% 
For the log-$\gamma$ model (\ref{eqn:log-gamma}),  
%
\begin{equation}
\label{eqn:bp-trajectory-equationpde}
   Q(\gamma) =  
   \frac{(1-\gamma)^{-1}}{\ln(1 - \gamma)} I 
         - P(I-\gamma P)^{-1} (\ln(I - \gamma P))^{-1}. 
\end{equation} 
%
For each model, $e^{\rm T}Q=0$. 

%\textcolor{red}{\bf \em need to check matrix $A$ and matrix $Q$ in each
%  case.}


% flatex input end: [sections/intra-model-variation.tex]

% ================ intra-model damping 
% flatex input: [sections/KL-measures.tex]
% KL-measures.tex
% to pageRank-models.tex

In addition to the element-wise response in the rank vector, we would
also like to have an aggregated measure of the response to variation in
$\rho$. Let $x(\rho_{o})$ be a reference PageRank vector.  We
may use Kullback-Leibler divergence \cite{kullback1951information} to measure the discrepancy of
$x(\rho)$ from $x(\rho_{o})$,
%
\begin{equation}
\label{eqn:kl}
  KL(x(\rho), x(\rho_{o}) ) = 
   \sum_{i} x_i(\rho) \log {\frac{x_i(\rho)}{x_i(\rho_{o})}} . 
\end{equation}
%
When $\rho=\rho_{o}$, $KL(x(\rho), x(\rho_{o})) = 0$. 
%
We have the rate of change in KL divergence with the variation in
$\rho$, 
% 
\begin{equation}
\label{eqn:kl-derivative}
  \frac{d}{d \rho} KL(x(\rho), x(\rho_{o})) 
 =  \dot{x}(\rho)^{\rm T} 
    \left( \log {x(\rho)} -\log{x(\rho_{o})}  + e \right)
\end{equation}
%
We will describe in Section~\ref{sec:batch-ranking} efficient
algorithms for calculating the vectors and measures above.



% flatex input end: [sections/KL-measures.tex]

% ================ intra-model damping 


% ================= inter-model damping
\subsection{Inter-model correspondence}
\label{subsec:damping-corres}
% flatex input: [sections/inter-model-correspondence.tex]
% inter-model-correspondence.tex


Each model has its own damping form and parameter. The expected value of
the step weight distribution is, 
%
\begin{equation}
\label{eqn:expect-steps}
\mu(w(\rho)) = \sum_{k \in \mathbb{N}_{w}} k \cdot w_k(\rho). 
\end{equation}
%
We may explain this as the expected value of walking steps.  We
establish the point of correspondence between models by their expected
values. That is, for any two models, we set their expected values equal
to each other. Without loss of generality, we let the expected values
for the Brin-Page model serve as the reference. In particular, we have
the correspondence equalities
%
\begin{equation}
\label{eqn:param-corres}
  \frac{\alpha}{1-\alpha} = \beta, 
  \quad 
  \frac{\alpha}{1-\alpha} 
 = \left(\frac{\gamma}{1 - \gamma}\right)\frac{-1}{\ln(1 - \gamma)}
\end{equation}%
%
for Chung's model and for the log-$\gamma$ model, respectively. We will show
the comparisons in PageRank vectors at such correspondence points in
Section~\ref{sec:numerical-experiments}.

% flatex input end: [sections/inter-model-correspondence.tex]

% ================= inter-model damping

% flatex input end: [response-variation-damping.tex]

%

% ============== ALGORITHM

\section{Efficient algorithms for batch ranking}
\label{sec:batch-ranking}
% 
% flatex input: [batch-ranking-algorithms.tex]
% batch-ranking-algorithms.tex
% 

% ... leading paragraph 

We introduce novel algorithms for efficient quantitative analysis of
damping effect on PageRank distribution.  Provided with a network graph
$P$ and a personalized distribution vector $v$, the algorithms can be
used in one batch of computation across multiple models as well as over
a range of damping parameter value per model.

% ========  reduction to wub-networks
\subsection{Reduction to irreducible subnetworks}
\label{subsec:model-reduction}
%
% flatex input: [sections/model-reduction.tex]
% model-reduction.tex 


% 
Information networks in real world applications are not necessarily
irreducible and aperiodic as assumed by many existing iterative
solutions for guaranteed convergence. To meet such convergence
conditions, some heuristics were used to perturb or twist the network
structure with artificially introduced links~\cite{lee2003fast,
langville2006reordering}.
%
Instead, we decompose the network into strongly connected sub-networks
by applying the Dulmage-Mendelsohn (DM) decomposition algorithm
\cite{dulmage1958coverings} to the Laplacian matrix $I-P$. The DM
algorithm is highly efficient when diagonal elements are non-zero. 
%
It renders the matrix in block upper triangular form. See
Figure~\ref{fig:dm} for the Google link graph released by Google in 2002 \cite{google-net}.
%
Each diagonal block $B_{ii}$ corresponds to a subnetwork. A
square diagonal block corresponds to an irreducible
subnetwork. A non-zero off-diagonal block $B_{ij}$ in the upper part,
$i<j$, represents the links from cluster $j$ to cluster $i$. The top
block is associated with a {\em sink} cluster without outgoing links to
other cluster; the bottom block is associated with a {\em source}
cluster without incoming edges from other clusters. The solution for the
entire network can be obtained by the solutions to the subnetworks and
successive back substitution.
%
% flatex input: [figtex/google/fig-google-dmperm.tex]
% fig-google-dmperm.tex

\begin{figure}[!htb]
  \centering
%  \subfigure[Original network]{
  \subfigure[original link graph]{
    \includegraphics[width=0.22\textwidth]{web-Google-sparsity}
  }
%  \subfigure[Network after DM-perm]{
  \subfigure[link graph after DM permutation]{
    \includegraphics[width=0.22\textwidth]{web-Google-sparsity-dmperm}
  }
  \caption{% \scriptsize 
   \footnotesize The 1:1000 sparsity map of adjacency matrix of Google link graph \cite{google-net} with 875,713
    page nodes in {\bf (a)} the provided ordering and {\bf (b)} the ordering rendered by the DM decomposition, depicted by {\tt imagesc} in {\tt matlab}. Each point shows the number of non-zeros, in log scale, in the corresponding $1000 \times 1000$ block. The subnetwork in the middle of (b) is strongly connected with 434,818 nodes.}
  \label{fig:dm}
\end{figure}

% flatex input end: [figtex/google/fig-google-dmperm.tex]
 
%


% flatex input end: [sections/model-reduction.tex]

%

% ======== cascade of iterations
\subsection{Cascade of iterations} 
\label{subsec:cascade}
% 
% flatex input: [sections/iteration-cascade.tex]
% iteration-castecade.tex 


Several iterative methods exist for computing the PageRank vector by the
Brin-Page model. They include the power method by the eigenvector
equation (\ref{eqn:ppr-markov-equi}), and the Jacobi, Gauss-Seidel, and
SOR methods by equation (\ref{eqn:ppr-linear}). There are various
acceleration techniques used for calculating the PageRank vector, or a
small part of the vector, or even a single pair of nodes between
the personalized vector and the PageRank vector~\cite{kamvar2003extrapolation,
ilprints579, kamvar2004adaptive, DBLP:journals/corr/LofgrenBGS14}.
% 
For the Brin-Page model, the iterative methods converge slower as
$\alpha$ increases and gets closer to $1$. We developed a cascading
initialization scheme. The solution to the model with $\alpha$ is used
as the initial guess to the iteration for the solution to the model with
$\alpha+\delta \alpha$, $\delta \alpha > 0$. Although it has accelerated
the computation with successively increased $\alpha$ values, this
technique is limited to sequential computation and ad hoc to the
Brin-Page model. We introduce next a novel algorithm without these
limitations.


% flatex input end: [sections/iteration-cascade.tex]
 
% 

% ======== shared invariant Krylov space
\subsection{Shared invariant Krylov space} 
\label{subsec:SIK-space}
%
% flatex input: [sections/SIK-space-method.tex]
% SIK-space-method.tex 


Our new algorithm for batch calculation of PageRank vectors with
multiple models and parameter values is based on the very fact that the
solutions to the models in Section~\ref{sec:pageRank-models} all reside
in the same Krylov space,
%
\begin{equation} 
\label{eqn:Krylov-space} 
{\cal K}(P, v) = \{v, Pv, P^2v, \cdots, P^{k}v, \cdots  \}, 
\quad 
\begin{array}{l} v\geq 0 \\ e^{\rm T}v = 1 \end{array}. 
\end{equation} 
%
The space has the property $P {\cal K}(P,v) = {\cal K}(P,v)$, i.e., it
is a spectrally invariant subspace. In PageRank terminology,
${\cal K}(P,v)$ is a personalized invariant subspace.  
%
We have the remarkable fact about the model family in Section~\ref{sec:pageRank-models}. 
%
\newtheorem{theorem}{Theorem}
\begin{theorem}
\label{thm:krylove-functions}
%
Any model solution (\ref{eqn:uni-random-walk}), at any particular
damping parameter value, and its
trajectory (\ref{eqn:trajectory-equation}) are functions in the Krylov
space ${\cal K}(P,v)$.
\end{theorem} 


\begin{theorem}[]
\label{thm:krylov}
Let $m=\mbox{dimension}({\cal K}) $.  Denote by $K$ the matrix
composed of the Krylov vectors.  Let $K = QR$ be the QR factorization of
$K$.  Then, $Qe_1 =v$ and $PQ = QH$, where $H$ is an $m\times m$ upper
Hessenberg matrix, and $e_1$ is the first column of the identity matrix $I$.
\end{theorem}
%

A few remarks. Matrix $H$ in Theorem~\ref{thm:krylov} is the
representation of matrix $P$ under basis $Q$ in the Krylov space.  In
numerical computation, we use a rank-revealing version of the QR
factorization with $Qe_1 = v$.
%
In theory, the dimension $m$ is equal to the number of spectrally
invariant components of $P$ that present in $v$.  In the extreme case,
$m=1$ when $v$ is the Perron vector of $P$. In general, by the condition
$e^{\rm T}v =1$, $v$ is not deficient in Perror component. In our study
on real-world graphs, which we will detail shortly in Section~\ref{sec:numerical-experiments},
the numerical dimension is low, matrix $H$ is therefore small.  We may
view this as a manifest of the smallness of the real-world graphs under
study.  We exploit these theoretical and practical facts.

% 
\newtheorem{corollary}{Corollary}
\setcounter{corollary}{2}
\begin{corollary}[]
\label{col:krylov}
For any function $g$ in the Krylov space (\ref{eqn:Krylov-space}), we
have $g(P)v = Qg(H)e_1$.
\end{corollary}
%
When dimension $m$ is modest, we translate by Corollary \ref{col:krylov}
the calculation of $x_{g}=g(P)v$ with $N\times N$ matrix $P$ on vectors
to the calculation of $ \hat{x}_{g} = g(H)e_1$ with $m\times m$ matrix
$H$ on vectors, followed by a matrix-vector product $Q\hat{x}_g$. The
vector $\hat{x}_g$ is the spectral representation of $x_g$ in the Krylov
space.  In the model family, solutions $x_f$ (\ref{eqn:uni-random-walk})
differ from one to another in their spectral representations
$\hat{x}_f$, they share the same basis matrix $Q$ in the ambient network
space.

Our algorithm consists of the following major steps. Let
${\cal G}= \{ g\}$ be a set of functions under study.
% 
\begin{inparaenum}[(1)] 
\item Calculuate Krylov vectors to form matrix $K$ in
  Theorem~\ref{thm:krylov}, apply rank-revealing $QR$ factorization to
  $K$, and find numerical dimension $m$; \footnote{These substeps are
    integrated in pratical computation in order to determine quickly a
    sufficient number of Krylov vectors.}
%
\item Construct the matrix $H$, by Theorem~\ref{thm:krylov},
  from $R$ and the permutation matrix $\Pi$ rendered by the
  rank-revealing $QR$;
%
\item Calculate $\hat{x}_f= g(H)e_1$ for all functions in ${\cal G}$; 
%
\item Transform $\hat{x}_f$ from the Krylov-spectral space to the
  ambient network space by the same basis matrix $Q$, based on
  Corollary~\ref{col:krylov}.
%     
\end{inparaenum}


% flatex input end: [sections/SIK-space-method.tex]
 
%




% flatex input end: [batch-ranking-algorithms.tex]
 
% 

% ============== EXPERIMENT

\section{Experiments on real-world link graphs} 
\label{sec:numerical-experiments}
%
% flatex input: [numerical-experiments.tex]
%
% section 5: numerical experiments 
% last revision: May 20, 2018

We show in numerical values how PageRank vector responses to variation
in damping variable with each model and across models, on $6$ real-world
link graphs.
%

\subsection{Experiment setup: data and models}
\label{subsec:data-and-models} 
%
% flatex input: [sections/data-and-models.tex]
% data-and-models.tex


The $6$ link graphs we used for our experiments are publicly available
at {\em the Koblenz Network Collection}\cite{kunegis2013konect}. 
The basic information of the graphs is summarized in
Table~\ref{tab:dataset}, where $\max(d_{out})$ is the maximum out-degree
(the number of citations) of graph nodes, $\max(d_{in})$ is the maximum 
in-degree (the number of backlinks), $\mu(d_{out}) = \mu(d_{in})$ is
the average out-degree, which equals to the average in-degree, 
and LSCC stands for the largest strongly connected component(s) of the graph. 
The Google graph of today is reportedly containing hundreds of trillions 
of nodes, substantially larger than the snapshot size used here.
%
% flatex input: [tabtex/tab-dataset.tex]
% datab-dataset.tex
% 

\begin{table}[H]
\centering
\caption{Dataset Description}
\label{tab:dataset}
\begin{adjustbox}{width=0.5\textwidth}
\begin{tabular}[t]{lrrr}
\toprule
&  Total \#nodes & \#nodes in LSCC & $\left[\max(d_{out}), \mu(d_{out}), \max(d_{in})\right]$\\
\midrule
   Google \cite{google-net} & 875,713 & 434,818 & $[ 4209, 8.86, 382]$ 
\\ 
Wikilink \cite{kunegis2013konect} & 12,150,976 & 7,283,915 & $[7527, 50.48, 920207]$ 
\\
DBpedia \cite{auer2007dbpedia} & 18,268,992 & 3,796,073 & $[8104, 26.76, 414924]$ 
\\
Twitter(www)\cite{Kwak10www} & 41,652,230 & 33,479,734 & $[2936232, 42.65, 768552]$  
\\
Twitter(mpi)\cite{icwsm10cha} & 52,579,682 & 40,012,384 & $[778191, 47.57, 3438929]$
\\
friendster\cite{yang2015defining} & 68,349,466 & 48,928,140 & $[3124, 32.76, 3124]$  
\\ 
\bottomrule{}
\end{tabular}
\end{adjustbox}
\end{table}

% flatex input end: [tabtex/tab-dataset.tex]

%
% 
For variation analysis of each graph in Table~\ref{tab:dataset}, the 
associated link matrix $P$ is well specified. We use the same personalized
or customized distribution vector $v$, which we get by drawing elements from 
standard Gaussian distribution ${\cal N}(0, 1)$, followed by 
normalization $v^{\rm T}e = 1$.
% 
We report variation analysis results with three particular models :
Brin-Page model (\ref{eqn:ppr-linear}) with damping variable $\alpha$,
Chung's model (\ref{eqn:heat-model}) with variable $\beta$, and the 
log-$\gamma$ model (\ref{eqn:log-gamma}). The last is used as an 
illustration of new models in the family (\ref{eqn:uni-random-walk}).
% flatex input end: [sections/data-and-models.tex]

%


\subsection{Variation in PageRank vector} 
\label{subsec:intro-model-analysis}
%
% flatex input: [sections/var-rank-vectors.tex]
% var-rank-vectors.tex
%

% flatex input: [figtex/google/fig-google-unify-distribution.tex]
% fig-google-unify-distribution.tex

\begin{figure}[!htb]
  \centering
  \subfigure[$x(\rho)$ of Google link graph]{
    \includegraphics[width=0.22\textwidth]
        {web-Google-1-unify-hist-2d}
    \label{fig:google-hist-unify}
  }
  \subfigure[$x(\rho)$ of Twitter(www) link graph]{
    \includegraphics[width=0.22\textwidth]
       {out-twitter-1-unify-hist-2d}
    \label{fig:twitter-hist-unify}
  }
  \caption{\footnotesize%
    Inter-model comparison of histograms of $N\cdot x_f(\rho)$ among the 
    three models with corresponding parameter values $\alpha_{o}$,
    $\beta_{o}$ and $\gamma_{o}$ so that the expected value of
    walking steps for each model is
    $ \alpha_{o}/(1-\alpha_{o})= 5.\dot{6} $ with $\alpha_{o} = 0.85$,
    see the model correspondence equalities (\ref{eqn:param-corres}).
    (a) comparison on Google link graph; (b) comparison on Twitter(www)
    link graph. }
\label{fig:unify-model-google}
\end{figure}

% flatex input end: [figtex/google/fig-google-unify-distribution.tex]

%
% 
In order to show the quantitative response in PageRank vector $x_f$
over $N$ nodes with the variation in damping variable $\rho$, we display
the histogram of $N\cdot x_f(\rho)$ for model $f$ at parameter value
$\rho$. In Figure~\ref{fig:unify-model-google} we show the
histograms associated with three models on Google network. The parameter for Brin-Page
model is set to the value $\alpha_{o} = 0.85$. The parameter for 
the other two models are set by (\ref{eqn:param-corres}).
%
%%% ----- observation ------------- 
We observe that the histogram with Brin-Page model has higher and narrower peaks
than Chung's model. The histogram of
log-$\gamma$ model is in between. This is expected by the relationships
in the damping weights among the three models, as discussed in
Section~\ref{subsec:damping-corres}.

% flatex input: [figtex/google/fig-google-distribution.tex]
% fig-google-distribution.tex
% caption revision by Xiaobai on May 20, 2018 
% latex layout revision as well 
% 


\begin{figure}[!htb]
  \centering
  \subfigure[Brin-Page model]{
    \includegraphics[width=0.14\textwidth]
     {web-Google-1-alpha-hist-2d}
    \label{fig:google-hist-bp}
  }
  \subfigure[log-$\gamma$ model]{
    \includegraphics[width=0.14\textwidth]
    {web-Google-1-gamma-hist-2d}
    \label{fig:google-hist-chung}
  }
  \subfigure[Chung's model]{
    \includegraphics[width=0.14\textwidth]
    {web-Google-1-beta-hist-2d}
    \label{fig:google-hist-chung}
  }
  \subfigure[Brin-Page model]{
    \includegraphics[width=0.14\textwidth]
    {web-Google-1-alpha-hist-3d}
    \label{fig:google-hist-bp-3d}
  }
  \subfigure[log-$\gamma$ model]{
    \includegraphics[width=0.14\textwidth]
    {web-Google-1-gamma-hist-3d}
    \label{fig:google-hist-chung-3d}
  }
  \subfigure[Chung's model]{
    \includegraphics[width=0.14\textwidth]
    {web-Google-1-beta-hist-3d}
    \label{fig:google-hist-chung-3d}
  }

  
  \caption{\footnotesize 
    Comparison in histograms of $N\cdot x_f(\rho)$ on the Google link
    graph over a range of damping variable value. 
    %  
    {\bf Left column:} Brin-Page model, 
    {\bf Middle column:} log-$\gamma$ model, 
    {\bf Right column:} Chung's model.
    %%%  
    {\bf Top row:} 2D display of 6 histograms associated with
    6 parameter values shown in the respective legends.  The
    histogram in black with Brin-Page model is associated with the value
    $\alpha = 0.85$.  The corresponding parameter values with Chung's
    model and log-$\gamma$ model are set by (\ref{eqn:param-corres}), 
    the associated histograms are color
    coded by the corresponding parameter values.
    %%% 
    {\bf Bottom row:} a stack of multiple histograms shown in 3D space over
    the range $\alpha \in [0.7, 0.97]$ with Brin-Page model,
    $\beta \in [2.\dot{6}, 32.\dot{3}] $ with Chung's model, and $\gamma \in [0.7787,0.994]$ with log-$\gamma$ model. The histograms with Chung's
    model have flattened peaks at larger values (toward the back
    end). The log-$\gamma$ model is nearly insensitive to $\gamma$ change in the range above.}  %
\label{fig:histogram-google}
\end{figure}

% flatex input end: [figtex/google/fig-google-distribution.tex]

%%% ----- observation ------------- 
% 
% flatex input: [figtex/twitter/fig-twitter-distribution.tex]
% fig-twitter-distrbution.tex 
% 

\begin{figure}[!htb]
  \centering
  \subfigure[Brin-Page model]{
    \includegraphics[width=0.14\textwidth]
        {out-twitter-1-alpha-hist-2d}
    \label{fig:twitter-hist-bp}
  }
  \subfigure[log-$\gamma$ model]{
    \includegraphics[width=0.14\textwidth]
    {out-twitter-1-gamma-hist-2d}
    \label{fig:google-hist-chung}
  }
  \subfigure[Chung's model]{
    \includegraphics[width=0.14\textwidth]
        {out-twitter-1-beta-hist-2d}
    \label{fig:twitter-hist-chung}
  }
  \subfigure[Brin-Page model]{
    \includegraphics[width=0.14\textwidth]
        {out-twitter-1-alpha-hist-3d}
    \label{fig:twitter-hist-bp-3d}
  }
  \subfigure[log-$\gamma$ model]{
    \includegraphics[width=0.14\textwidth]
    {out-twitter-1-gamma-hist-3d}
    \label{fig:google-hist-chung-3d}
  }
  \subfigure[Chung's model]{
    \includegraphics[width=0.14\textwidth]
        {out-twitter-1-beta-hist-3d}
    \label{fig:twitter-hist-chung-3d}
  }

  \caption{\footnotesize 
  Comparison in histogram of $N\cdot x_f(\rho)$ on the Twitter graph 
  over a range of dampling value, in the same settings as in
  Figure~\ref{fig:histogram-google}.
  }
\label{fig:histogram-twitter}
\end{figure}

% flatex input end: [figtex/twitter/fig-twitter-distribution.tex]

% 
%
Figure \ref{fig:histogram-google} and Figure \ref{fig:histogram-twitter} show the variation in the histograms over a range of the damping
variable per model as well as the comparison side by side between
the three models on two datasets. The models
have similar behaviors on the other $4$ graphs in Table~\ref{tab:dataset}.
Supplementary material can be found in \cite{MS-qian-2018}. With larger damping factors in the models, 
the distribution become less centralized. Log-$\gamma$ model, specifically, is less sensitive to $\gamma(\alpha)$
range with $\alpha \in [0.7,0.97]$.
% 

% flatex input end: [sections/var-rank-vectors.tex]

%



\subsection{Relative variation measured by  KL divergence} 
\label{subsec:intro-model-analysis}
%
% flatex input: [sections/var-KL-divergence.tex]
% var-KL-divergence.tex

% flatex input: [figtex/google/fig-google-KL-DKL.tex]
% fig-google-KL&DKL-0.85.tex

\begin{figure}[!htb]
  \centering
  \subfigure[{\footnotesize $\alpha_o=0.85$}]{
    \includegraphics[width=0.14\textwidth]
          {web-Google-1-alpha-kl-dkl-1}
  }
  \subfigure[{\footnotesize $\gamma_o=0.94146$}]{
    \includegraphics[width=0.14\textwidth]
          {web-Google-1-gamma-kl-dkl-1}
  }
  \subfigure[{\footnotesize $\beta_o=5.\dot{6}$}]{
    \includegraphics[width=0.14\textwidth]
          {web-Google-1-beta-kl-dkl-1}
  }
  \subfigure[{\footnotesize $\alpha_o=0.95$}]{
    \includegraphics[width=0.14\textwidth]
          {web-Google-1-alpha-kl-dkl-2}
  }
  \subfigure[{\footnotesize $\gamma_o=0.98831$}]{
    \includegraphics[width=0.14\textwidth]
          {web-Google-1-gamma-kl-dkl-2}
  }
  \subfigure[{\footnotesize $\beta_o=19$}]{
    \includegraphics[width=0.14\textwidth]
          {web-Google-1-beta-kl-dkl-2}
  }
  
  \caption{\footnotesize Intra-model relative variation as defined in (\ref{eqn:kl}) in PgeRank
    distribution, on the Google graph, with respect to two reference
    distribution at $\alpha_{o} \in \{0.85,0.95\}$ with Brin-Page model ({\bf left
    column}), $\gamma_{o} \in \{0.94146,0.98831\}$ with log-$\gamma$ model ({\bf middle column}), 
    and $\beta_{o} \in \{5.\dot{6},19\}$ with Chung's model ({\bf right column}). 
    %%% ==============
    {\bf Blue curves:} the KL score 
    $KL( x_f(\rho) || x_{f}(\rho_{o}) )$ with numerically 
    computed distribution vectors; 
    {\bf Red curves:} the derivative of the KL score 
    $ (d/d\rho) KL( x_f(\rho) || x_{f}(\rho_{o} ))$. 
    The red curves with $\cdot$ marker and $\diamond$ marker are obtained empirically 
    from numerical distribution vectors, with step size 
    $\Delta \rho = (0.002, 0.008) $ respectively. The red curves with $\times$ marker
    are obtained analytically by (\ref{eqn:kl-derivative}). 
    %%% =============== 
    {\bf Remarks.} With Brin-Page model and log-$\gamma$ model, the distribution changes gently from
    the reference distribution in the neighborhood of the reference
    value $\alpha_{o} = 0.85$, by the KL curve and the KL derivative
    curve.  In sharp contrast, the distribution with Chung's model
    deviates rapidly from the reference distribution.  }
  \label{fig:google-param-KL-0.85}
\end{figure}

% flatex input end: [figtex/google/fig-google-KL-DKL.tex]

% var-KL-divergence.tex

We show the relative variation in PageRank vector with respect to a
reference vector by (\ref{eqn:kl}).  For Brin-Page model, we consider
two particular reference vectors: one is associated with $\alpha=0.85$
as chosen originally by Brin and Page, the other is at $\alpha = 0.95$,
much closer to the extreme case $\alpha=1$, in which the walks follow
the links only.  For Chung's model and log-$\gamma$ model, we use the corresponding parameter
values by (\ref{eqn:param-corres}).
%
We show the differences between the three models on the Google graph in
Figure~\ref{fig:google-param-KL-0.85} at the corresponding reference values, 
respectively, and on the twitter graph in
Figure~\ref{fig:twitter-param-KL-0.85}.

% flatex input: [figtex/twitter/fig-twitter-KL-DKL.tex]
\begin{figure}[!htb]
  \centering
  \subfigure[{\footnotesize $\alpha_o=0.85$}]{
    \includegraphics[width=0.14\textwidth]
          {out-twitter-1-alpha-kl-dkl-1}
  }
  \subfigure[{\footnotesize $\gamma_o=0.94146$}]{
    \includegraphics[width=0.14\textwidth]
          {out-twitter-1-gamma-kl-dkl-1}
  }
  \subfigure[{\footnotesize $\beta_o=5.\dot{6}$}]{
    \includegraphics[width=0.14\textwidth]
          {out-twitter-1-beta-kl-dkl-1}
  }
  \subfigure[{\footnotesize $\alpha_o=0.95$}]{
    \includegraphics[width=0.14\textwidth]
          {out-twitter-1-alpha-kl-dkl-2}
  }
  \subfigure[{\footnotesize $\gamma_o=0.98831$}]{
    \includegraphics[width=0.14\textwidth]
          {out-twitter-1-gamma-kl-dkl-2}
  }
  \subfigure[{\footnotesize $\beta_o=19$}]{
    \includegraphics[width=0.14\textwidth]
          {out-twitter-1-beta-kl-dkl-2}
  }
  \caption{\footnotesize Intra-model relative variation in PageRank
    distribution by (\ref{eqn:kl}), on the Twitter graph. The rest is in the same setting as in Figure \ref{fig:google-param-KL-0.85}.}
%
\label{fig:twitter-param-KL-0.85}
\end{figure}

% flatex input end: [figtex/twitter/fig-twitter-KL-DKL.tex]

%




% flatex input end: [sections/var-KL-divergence.tex]
 
%



\subsection{Batch calculation: efficiency and accuracy} 
\label{subsec:batch-calculation} 
% 
% flatex input: [sections/batch-evaluation.tex]
% batch-evaluation.tex
%

We show first that the Krylov space dimension is numerically low for each of the 6 real world link graphs.
Figure \ref{fig:qr} gives the diagonal elements of each upper-triangular matrix $R$ obtained by a 
rank-revealing QR factorization. The elements below $10^{-17}$ are not shown. The numerical 
dimension ranges from $19$ with DBpedia link graph to $62$ with Google link graph. The low numerical 
dimension makes our algorithm in Section \ref{sec:batch-ranking} highly efficient. In addition, we 
exploited the sparsity of matrix $P$ in the Krylov vector calculation, see details in \cite{MS-Xichen-2018}.

% flatex input: [figtex/R-diag.tex]
\begin{figure}[!htb]
  \centering
  \subfigure[Google]{
    \includegraphics[width=0.13\textwidth]{web-Google-R-diag}
  }
  \subfigure[Twitter(www)]{
    \includegraphics[width=0.13\textwidth]{out-twitter-R-diag}
  }
  \subfigure[Wikipedia]{
    \includegraphics[width=0.13\textwidth]{wikipedia-R-diag}
  }
  \subfigure[DBpedia]{
    \includegraphics[width=0.13\textwidth]{dbpedia-R-diag}
  }
  \subfigure[Twitter(mpi)]{
    \includegraphics[width=0.13\textwidth]{twitter_mpi-R-diag}
  }
  \subfigure[Friendster]{
    \includegraphics[width=0.13\textwidth]{friendster-R-diag}
  }
  \caption{The diagonal elements of each upper-triangular matrix $R$ obtained by a rank-revealing QR factorization for the 6 datasets in Table \ref{tab:dataset}. Google link graph has the highest numerical dimension 62 among the 6 datasets, and the DBpedia link graph has the lowest numerical dimension 19.}
  \label{fig:qr}
\end{figure}
% flatex input end: [figtex/R-diag.tex]

%

The accuracy of our batch algorithm is evaluated in two ways. One is by 
$err = \|(x_{\text{Krylov}} - x_{\text{G-S}}) ./ x_{\text{G-S}}\|_{\infty}$, 
the maximum element-wise relative difference in
the PageRank vectors of Brin-Page model between the Gauss-Seidel method and 
our Krylov subspace method. In our experiments, the relative errors for all 6 datasets 
are below $10^{-10}$. In the other way, we show in 
Figure \ref{fig:google-param-KL-0.85} and Figure \ref{fig:twitter-param-KL-0.85} 
that the empirical rate of change agrees well with analytical prediction 
(\ref{eqn:kl-derivative}).


% flatex input end: [sections/batch-evaluation.tex]
 
% 

% flatex input end: [numerical-experiments.tex]

%

% ============== SUMMARY

\section{Concluding remarks} 
%
% flatex input: [summary.tex]
% summary.tex

Our model extension, connection, unified analysis and numerical
algorithm for quantitative estimation in batch are original, to our
knowledge. Our study leads to new observation and several findings.
%
\begin{inparaenum}[(a)]
\item In network propagation pattern in response to variation in the 
  damping mechanism, the inter-model difference among the 3 models
  is much more significant than the inter-dataset difference among
  the 6 datasets. This suggests the utility of
  model variety for differentiating network activities or propagation
  patterns.
% 
\item The model solutions reside in the same customized, spectrally
  invariant subspace. On each of the $6$ real-world graphs, the space
  dimension is low, which is a small-world phenomenon.
% 
\item The shared computation is not limited to one personalized
  distribution. The Krylov space associated with a particular vector $v$
  contains certainly many other distribution vectors. In fact, every
  Krylov vector is a distribution vector. This finding may lead to a much 
  more efficient way to represent and compute PageRank distributions 
  across multiple personalized vectors.
% 
\item The low spectral dimension, estimated once for a particular graph
  $P$ and a personalized/customized vector $v$, may serve as a
  reasonable upper bound on the number of iterations by any competitive
  algorithm, with one matrix-vector product per iteration, for Brin-Page
  model, at any $\alpha$ value in $(0, 1)$, or any
  other model in the family (\ref{eqn:uni-random-walk}). The power method and the
  Gauss-Seidel iteration take more iterations to reach the same error
  level on the larger real-world graphs among the studied, and take many
  more iterations when $\alpha$ gets closer to $1$.
%
\end{inparaenum} 
%
In brief conclusion, estimating PageRank distribution under various
damping conditions is valuable and easily affordable.

% flatex input end: [summary.tex]

%

%\section{Appendix}
%\onecolumn


% \tableofcontents{}

% \newpage

\section*{Supplementary Material}
\addcontentsline{toc}{section}{Supplementary Material}


Throughout this discussion, 
we will make frequently use 
of the following standard results
concerning the exponential concentration 
of random variables:

\begin{lemma}[Hoeffding's inequality for independent RVs~\citep{hoeffding1994probability}] Let $Z_1, Z_2, \ldots, Z_n$ be independent bounded random variables with $Z_i \in [a,b]$ for all $i$, then 
    \begin{align*}
        \prob\left( \frac{1}{n} \sum_{i=1}^n (Z_i - \Expo{Z_i}) \ge t \right) \le \exp{\left( -\frac{2nt^2}{(b-a)^2} \right) }
    \end{align*} 
    and 
    \begin{align*}
        \prob\left( \frac{1}{n} \sum_{i=1}^n (Z_i - \Expo{Z_i}) \le -t \right) \le \exp{\left( -\frac{2nt^2}{(b-a)^2} \right) }
    \end{align*} 
    for all $t \ge 0$. 
\end{lemma}

\begin{lemma}[Hoeffding's inequality for sampling with replacement~\citep{hoeffding1994probability}] \label{lem:hoeffding_sampling} Let $\calZ = (Z_1, Z_2, \ldots, Z_N)$ be a finite population of $N$ points with $Z_i \in [a.b]$ for all $i$. Let $X_1, X_2, \ldots X_n$ be a random sample drawn without replacement from $\calZ$. Then for all $t \ge 0$, we have 
    \begin{align*}
        \prob\left( \frac{1}{n} \sum_{i=1}^n (X_i - \mu ) \ge t \right) \le \exp{\left( -\frac{2nt^2}{(b-a)^2} \right) }
    \end{align*} 
    and 
    \begin{align*}
        \prob\left( \frac{1}{n} \sum_{i=1}^n (X_i - \mu ) \le -t \right) \le \exp{\left( -\frac{2nt^2}{(b-a)^2} \right) } \,,
    \end{align*} 
    where $\mu = \frac{1}{N} \sum_{i=1}^{N} Z_i$. 
\end{lemma}

We now discuss one condition that generalizes the exponential concentration to dependent random variables.
\begin{condition}[Bounded difference inequality] \label{cond:BDC} Let $\calZ$ be some set and $\phi: \calZ^n \to \Real$. We say that $\phi$ satisfies the bounded difference assumption if 
there exists $c_1, c_2, \ldots c_n \ge 0$ s.t. for all $i$, we have 
\begin{align*}
    \sup_{Z_1,Z_2, \ldots,Z_n, Z_i^\prime \in \calZ^{n+1} } \abs{\phi (Z_1, \ldots, Z_i, \ldots, Z_n ) - \phi (Z_1, \ldots, Z_i^\prime, \ldots, Z_n ) } \le c_i \,.
\end{align*} 
\end{condition}

\begin{lemma}[McDiarmid’s inequality~\citep{mcdiarmid1989}] \label{lem:McDiarmid} Let $Z_1, Z_2, \ldots, Z_n$ be independent random variables on set $\calZ$ and $\phi : \calZ^n \to \Real$ satisfy bounded difference inequality (\codref{cond:BDC}). Then for all $t>0$, we have 
    \begin{align*}
        \prob\left( \phi(Z_1, Z_2, \ldots, Z_n) - \Expo{\phi(Z_1, Z_2, \ldots, Z_n)} \ge t \right) \le \exp{\left( -\frac{2t^2}{\sum_{i=1}^n c_i^2} \right) } 
    \end{align*} 
    and 
    \begin{align*}
        \prob\left( \phi(Z_1, Z_2, \ldots, Z_n) - \Expo{\phi(Z_1, Z_2, \ldots, Z_n)} \le -t \right) \le \exp{\left( -\frac{2t^2}{\sum_{i=1}^n c_i^2} \right) } \,.
    \end{align*} 
\end{lemma}


\section{Proofs from \secref{sec:ERM_training}}\label{app:proof_erm}

\textbf{Additional notation {} {}} Let $m_1$ be the number of mislabeled points ($\wt S_M$) and $m_2$ be the number of correctly labeled points ($\wt S_C$). Note $m_1 + m_2 = m$. 


\subsection{Proof of \thmref{thm:error_ERM}}


\begin{proof}[Proof of \lemref{lem:fit_mislabeled}] 
    The main idea of our proof is to regard 
    the clean portion of the data 
    ($S \cup \wt S_C$) as fixed.   
    Then, there exists an (unknown) classifier $f^*$ 
    that minimizes the expected risk
    calculated on the (fixed) clean data
    and (random draws of) the mislabeled data $\wt S_M$. 
    % 
    % 
    Formally, 
    \begin{align}
    f^* \defeq \argmin_{f \in \calF} \error_{\widecheck {\calD}} (f) \,, \label{eq:modified_ERM}
    \end{align}
    where $$\widecheck \calD = \frac{n}{m+n} \calS + \frac{m_2}{m+n} \wt \calS_C  + \frac{m_1}{m+n}\calDm \,.$$ 
    Note here that $\widecheck \calD$ is a combination 
    of the \emph{empirical distribution} 
    over correctly labeled data $S \cup \wt S_C$
    and the (population) distribution 
    over mislabeled data $\calDm$.
    Recall that 
    \begin{align}
    \wh f \defeq \argmin_{f \in \calF} \error_{\calS \cup \wt S} (f) \,. \label{eq:orig_ERM}
    \end{align}
    % 
    % 
    Since, $\widehat f$ minimizes 0-1 error 
    on $S \cup \wt S$, using ERM optimality on \eqref{eq:orig_ERM},  
    we have 
    \begin{align}
        \error_{\calS \cup \wt \calS}(\widehat f) \le \error_{
            \calS \cup \wt \calS}(f^*) \,.    \label{eq:step1}
    \end{align}
    Moreover, since $f^*$ is independent of $\wt S_M$, using Hoeffding's bound,
    % \footnote{For a fully rigorous argument,
    % refer to the complete proof in App.~\ref{app:proof_erm}.} 
    we have with probability at least $1-\delta$ that
    \begin{align}
      \error_{\wt \calS_M}(f^*) \le \error_{ \calDm}(f^*) +  \sqrt{\frac{\log(1/\delta)}{2 m_1}} \,. \label{eq:step2} 
    \end{align}
    %$ 
    %for some constant $c_1\le 1/2$. 
    Finally, since $f^*$ is the optimal classifier on $\widecheck \calD$, 
    we have 
    \begin{align}
        \error_{\widecheck \calD}(f^*) \le \error_{\widecheck \calD}(\widehat f) \,. \label{eq:step3}
    \end{align}
    Now to relate \eqref{eq:step1} and \eqref{eq:step3}, we multiply \eqref{eq:step2} by $\frac{m_1}{m+n}$ and add $\frac{n}{m+n} \error_{\calS} (f)  + \frac{m_2}{m+n} \error_{\wt \calS_C} (f)$ both the sides. Hence, 
    we can rewrite \eqref{eq:step2} as follows: 
    \begin{align}
        \error_{\calS \cup \wt\calS}(f^*) \le \error_{ \widecheck \calD}(f^*) +  \frac{m_1}{m+n}\sqrt{\frac{\log(1/\delta)}{2 m_1}} \,. \label{eq:step4} 
    \end{align}
    Now we combine equations \eqref{eq:step1}, \eqref{eq:step4}, and \eqref{eq:step3}, to get 
    \begin{align}
        \error_{\calS \cup \wt \calS}(\wh f) \le \error_{\widecheck \calD}(\wh f) +  \frac{m_1}{m+n}\sqrt{\frac{\log(1/\delta)}{2 m_1}} \,, 
    \end{align}
    which implies 
    \begin{align}
        \error_{ \wt \calS_M}(\wh f) \le \error_{\calDm}(\wh f) + \sqrt{\frac{\log(1/\delta)}{2 m_1}} \,. \label{eq:lemma1_final}
    \end{align}
    Since $\wt S$ is obtained by randomly labeling an unlabeled dataset, we assume $2m_1 \approx m$ \footnote{Formally, with probability at least $1-\delta$, we have  $(m - 2m_1)\le \sqrt{m\log(1/\delta)/2}$.}. Moreover, using $\error_{\calDm} = 1 - \error_{\calD}$ we obtain the desired result.   
    % Combining the above steps and using the fact 
    % that $\error_\calD = 1- \error_{\calDm} $, 
    % we obtain the desired result.
\end{proof}

\begin{proof}[Proof of \lemref{lem:mislabeled_error}]
    Recall $\error_{\wt S} (f) = \frac{m_1}{m} \error_{\wt S_M}(f) + \frac{m_2}{m} \error_{\wt S_C}(f)$. Hence, we have 
    \begin{align}
        2\error_{\wt S}(f) - \error_{\wt S_M}(f) - \error_{\wt S_C}(f) &= \left(\frac{2m_1}{m} \error_{\wt S_M}(f) - \error_{\wt S_M}(f)\right) + \left(\frac{2m_2}{m} \error_{\wt S_C}(f) - \error_{\wt S_C}(f)\right) \\ &= \left(\frac{2m_1}{m} - 1\right) \error_{\wt S_M}(f) + \left(\frac{2m_2}{m} - 1 \right)\error_{\wt S_C} (f) \,.
    \end{align} 
    Since the dataset is labeled uniformly at random, with probability at least $1-\delta$, we have  $\left(\frac{2m_1}{m} - 1\right) \le \sqrt{\frac{\log(1/\delta)}{2m}}$. Similarly, we have with probability at least $1-\delta$, $\left(\frac{2m_2}{m} - 1\right) \le \sqrt{\frac{\log(1/\delta)}{2m}}$. Using union bound, with probability at least $1-\delta$, we have
    % \begin{align}
    %     2\error_{\wt S} - \error_{\wt S_M}(f) - \error_{\wt S_C}(f) \le \sqrt{\frac{\log(2/\delta)}{2m}} \left(\error_{\wt S_M}(f) + \error_{\wt S_C}(f) \right) \le 2\sqrt{\frac{\log(2/\delta)}{2m}} \,. \label{eq:lemma2_final}
    % \end{align}
    \begin{align}
        2\error_{\wt S} - \error_{\wt S_M}(f) - \error_{\wt S_C}(f) \le \sqrt{\frac{\log(2/\delta)}{2m}} \left(\error_{\wt S_M}(f) + \error_{\wt S_C}(f) \right) \,. \label{eq:lemma2_prefinal}
    \end{align}
    With re-arranging $\error_{\wt S_M}(f) + \error_{\wt S_C}(f)$ and using the inequality $ 1- a\le \frac{1}{1+a} $, we have  
    \begin{align}
        2\error_{\wt S} - \error_{\wt S_M}(f) - \error_{\wt S_C}(f) \le 2\error_{\wt \calS} \sqrt{\frac{\log(2/\delta)}{2m}}  \,. \label{eq:lemma2_final}
    \end{align}

    % We obtain the desired result by using 
\end{proof}

\begin{proof}[Proof of \lemref{lem:clear_error}]
% Recall 0-1 error on each point  $(x,y) \in S \cup \wt S$ is given by $\I{ f(x)\ne y}$.
In the set of correctly labeled points $S \cup \wt S_C$, we have $S$ as a random subset of $S \cup \wt S_C$. Hence, using Hoeffding's inequality for sampling without replacement (\lemref{lem:hoeffding_sampling}), we have with probability at least $1-\delta$
\begin{align}
    \error_{\wt \calS_C} (\wh f)- \error_{\calS \cup \wt \calS_C}( \wh f) \le  \sqrt{\frac{\log(1/\delta)}{2m_2}} \,.
\end{align}
Re-writing $\error_{\calS \cup \wt \calS_C}( \wh f)$ as $\frac{m_2}{m_2 + n} \error_{\wt \calS_C }(\wh f) + \frac{n}{m_2 + n} \error_{\calS }(\wh f)$, we have with probability at least $1-\delta$
\begin{align}
   \left(\frac{n}{n+m_2}\right) \left(\error_{\wt \calS_C} (\wh f)- \error_{\calS}( \wh f) \right) \le  \sqrt{\frac{\log(1/\delta)}{2m_2}} \,.
\end{align}
As before, assuming $2m_2 \approx m$, we have with probability at least $1-\delta$ 
\begin{align}
    \error_{\wt \calS_C} (\wh f)- \error_{\calS}( \wh f) \le \left(1+\frac{m_2}{n}\right)  \sqrt{\frac{\log(1/\delta)}{m}} \le \left(1 + \frac{m}{2n}\right) \sqrt{\frac{\log(1/\delta)}{m}} \,. \label{eq:lemma3_final}
\end{align} 
\end{proof}

\begin{proof}[Proof of \thmref{thm:error_ERM}] 
    Having established these core intermediate results, we can now combine above three lemmas to prove the main result. 
    In particular, we bound the population error on clean data ($\error_\calD(\wh f)$) as follows:  
    \begin{enumerate}[(i)]
        \item First, use \eqref{eq:lemma1_final}, to obtain an upper bound on the population error on clean data, i.e., with probability at least $1-\delta/4$, we have
        \begin{align}
            \error_{ \calD} (\wh f) \le 1 - \error_{ \wt \calS_M}(\wh f) + \sqrt{\frac{\log(4/\delta)}{m}} \,. 
        \end{align}
        \item  Second, use \eqref{eq:lemma2_final}, to relate the error on the mislabeled fraction with error on clean portion of randomly labeled data and error on whole randomly labeled dataset, i.e., with probability at least $1-\delta/2$, we have 
        \begin{align}
            - \error_{\wt S_M}(f) \le \error_{\wt S_C}(f) - 2\error_{\wt S}  + 2\error_{\wt S} \sqrt{\frac{\log(4/\delta)}{2m}}  \,. 
        \end{align} 
        \item Finally, use \eqref{eq:lemma3_final} to relate the error on the clean portion of randomly labeled data and error on clean training data, i.e., with probability $1-\delta/4$, we have 
        \begin{align}
            \error_{\wt \calS_C} (\wh f)\le - \error_{\calS}( \wh f) + \left(1 + \frac{m}{2n} \right) \sqrt{\frac{\log(4/\delta)}{m}} \,. 
        \end{align} 
    \end{enumerate}

    Using union bound on the above three steps, we have with probability at least $1-\delta$: 
    \begin{align}
        \error_\calD (\wh f) \le \error_{\calS}(\wh f)   + 1 - 2\error_{\wt \calS}(\wh f)   + \left(\sqrt{2} \error_{\wt S} + 2 + \frac{m}{2n}\right)  \sqrt{\frac{\log(4/\delta)}{m}} \,.
    \end{align}
    % Note that $(1/\sqrt{2} + 2.5)$ is a loose constant. In experiments, we use the ratio $\frac{m}{n}$
    %  the exact error $\error_{\wt \calS}(\wh f)$ 
    % to evaluate R.H.S.    
\end{proof}

\subsection{Proof of \propref{prop:rademacher}}

\begin{proof}[Proof of \propref{prop:rademacher}]
    For a classifier $ f: \calX \to \{-1, 1\}$, we have $1 - 2\,\indict{ f(x) \ne y} = y \cdot f(x)$. Hence, by definition of $\error$, we have 
    \begin{align}
        1 -2\error_{\wt \calS}(f) = \frac{1}{m}\sum_{i=1}^m y_i \cdot f(x_i) \le \sup_{f \in \calF} \, \frac{1}{m} \sum_{i=1}^m y_i \cdot f(x_i)  \,. \label{eq:error_rademacher}
    \end{align}
    Note that for fixed inputs $(x_1, x_2, \ldots, x_m)$ in $\wt S$, $(y_1, y_2, \ldots y_m)$ are random labels. Define $\phi_1 (y_1, y_2, \ldots, y_m) \defeq \sup_{f \in \calF} \, \frac{1}{m} \sum_{i=1}^m y_i \cdot f(x_i)$. We have the following bounded difference condition on $\phi_1$. For all i, 
    \begin{align}
        \sup_{y_1, \ldots y_m, y_i^\prime \in \{-1, 1\}^{m+1} } \abs{ \phi_1 (y_1,\ldots, y_i, \ldots, y_m) - \phi_1 (y_1,\ldots, y_i^\prime, \ldots, y_m)  } \le 1/m \,. \label{cond1_rademacher}
    \end{align} 
    
    Similarly, we define $\phi_2 (x_1, x_2, \ldots, x_m) \defeq \Expt{ y_i \sim_U \{-1, 1\}  }{ \sup_{f \in \calF} \, \frac{1}{m}  \sum_{i=1}^m y_i \cdot f(x_i)}$. We have the following bounded difference condition on $\phi_2$. 
    For all i,
    \begin{align}
        \sup_{x_1, \ldots x_m, x_i^\prime \in \calX^{m+1} } \abs{ \phi_2 (x_1,\ldots, x_i, \ldots, x_m) - \phi_1 (x_1,\ldots, x_i^\prime, \ldots, x_m)  } \le 1/m \,. \label{cond2_rademacher}
    \end{align}
    Using McDiarmid’s inequality (\lemref{lem:McDiarmid}) twice 
    with Condition \eqref{cond1_rademacher} and \eqref{cond2_rademacher}, 
    with probability at least $1-\delta$, we have
    \begin{align}
        \sup_{f \in \calF} \, \frac{1}{m} \sum_{i=1}^m y_i \cdot f(x_i)  - \Expt{x,y}{\sup_{f \in \calF} \, \frac{1}{m} \sum_{i=1}^m y_i \cdot f(x_i) } \le \sqrt{\frac{2\log(2/\delta)}{m}} \,. \label{eq:final_rademacher}
    \end{align} 
    Combining \eqref{eq:error_rademacher} and \eqref{eq:final_rademacher}, we obtain the desired result. 
\end{proof}


\subsection{Proof of \thmref{thm:error_regularized_ERM}}

Proof of \thmref{thm:error_regularized_ERM} follows similar to the proof of \thmref{thm:error_ERM}. Note that the same results in \lemref{lem:fit_mislabeled}, \lemref{lem:mislabeled_error}, and \lemref{lem:clear_error} hold in the regularized ERM case. However, the arguments in the proof of \lemref{lem:fit_mislabeled} change slightly. Hence, we state the lemma for regularized ERM and prove it here for completeness. 

\begin{lemma} \label{lem:lemma1_reg}
    Assume the same setup as \thmref{thm:error_regularized_ERM}. 
    Then for any $\delta >0$, with probability at least  $1-\delta$ 
    over the random draws of mislabeled data $\wt S_M$, we have 
    \begin{align}
        \error_\calD(\widehat f)  \le 1 -\error_{\wt \calS_M}(\widehat f) + \sqrt{\frac{\log(1/\delta)}{m}}\,. 
    \end{align} 
\end{lemma}
\begin{proof}
    The main idea of the proof remains the same, i.e. regard 
    the clean portion of the data 
    ($S \cup \wt S_C$) as fixed.   
    Then, there exists a classifier $f^*$ 
    that is optimal over draws 
    of the mislabeled data $\wt S_M$. 

    
    Formally, 
    \begin{align}
    f^* \defeq \argmin_{f \in \calF} \error_{\widecheck {\calD}} (f)  + \lambda R(f) \,, \label{eq:modified_ERM_reg}
    \end{align}
    where $$\widecheck \calD = \frac{n}{m+n} \calS + \frac{m_1}{m+n} \wt \calS_C  + \frac{m_2}{m+n}\calDm \,.$$ That is, $\widecheck \calD$ a combination of 
    the \emph{empirical distribution} 
    over correctly labeled data $S \cup \wt S_C$
    % in $S\cup \wt S$ 
    and the (population) distribution 
    over mislabeled data $\calDm$.
    Recall that 
    \begin{align}
    \wh f \defeq \argmin_{f \in \calF} \error_{\calS \cup \wt S} (f) + \lambda R(f) \,. \label{eq:orig_ERM_reg}
    \end{align}
    % 
    % 
    Since, $\widehat f$ minimizes 0-1 error 
    on $S \cup \wt S$, using ERM optimality on \eqref{eq:orig_ERM},  
    we have 
    \begin{align}
        \error_{\calS \cup \wt \calS}(\widehat f) + \lambda R(\wh f) \le \error_{
            \calS \cup \wt \calS}(f^*) + \lambda R(f^*) \,.    \label{eq:step1_reg}
    \end{align}
    Moreover, since $f^*$ is independent of $\wt S_M$, using Hoeffding's bound,
    % \footnote{For a fully rigorous argument,
    % refer to the complete proof in App.~\ref{app:proof_erm}.} 
    we have with probability at least $1-\delta$ that
    \begin{align}
      \error_{\wt \calS_M}(f^*) \le \error_{ \calDm}(f^*) +  \sqrt{\frac{\log(1/\delta)}{2 m_1}} \,. \label{eq:step2_reg} 
    \end{align}
    %$ 
    %for some constant $c_1\le 1/2$. 
    Finally, since $f^*$ is the optimal classifier on $\widecheck \calD$, 
    we have 
    \begin{align}
        \error_{\widecheck \calD}(f^*) + \lambda R(f^*) \le \error_{\widecheck \calD}(\widehat f) + \lambda R(\wh f) \,. \label{eq:step3_reg}
    \end{align}
     Now to relate \eqref{eq:step1_reg} and \eqref{eq:step3_reg}, we can re-write the \eqref{eq:step2_reg} as follows: 
    \begin{align}
        \error_{\calS \cup \wt\calS}(f^*) \le \error_{ \widecheck \calD}(f^*) +  \frac{m_1}{m+n}\sqrt{\frac{\log(1/\delta)}{2 m_1}} \,. \label{eq:step4_reg} 
    \end{align}
    After adding $\lambda R(f^*)$ on both sides in \eqref{eq:step4_reg}, we combine equations \eqref{eq:step1_reg}, \eqref{eq:step4_reg}, and \eqref{eq:step3_reg}, to get 
    \begin{align}
        \error_{\calS \cup \wt \calS}(\wh f) \le \error_{\widecheck \calD}(\wh f) +  \frac{m_1}{m+n}\sqrt{\frac{\log(1/\delta)}{2 m_1}} \,, 
    \end{align}
    which implies 
    \begin{align}
        \error_{ \wt \calS_M}(\wh f) \le \error_{\calDm}(\wh f) + \sqrt{\frac{\log(1/\delta)}{2 m_1}} \,. \label{eq:lemma_reg_final}
    \end{align}
    Similar as before, since $\wt S$ is obtained by randomly labeling an unlabeled dataset, we assume 
    $2m_1 \approx m$. Moreover, using $\error_{\calDm} = 1 - \error_{\calD}$ we obtain the desired result. 
\end{proof}
% \begin{proof}[Proof of ]
    
% \end{proof}

\subsection{Proof of \thmref{thm:multiclass_ERM}}

To prove our results in the multiclass case,
we first state and prove lemmas
parallel to those
% We first state and prove lemmas 
% parallel 
% to the three lemmas 
used in the proof of balanced binary case. 
We then combine these results 
% in the three lemmas 
to obtain the result in \thmref{thm:multiclass_ERM}. 

Before stating the result, 
we define mislabeled distribution $\calDm$ for any $\calD$.
While $\calDm$ and $\calD$ share 
the same marginal distribution over inputs $\calX$,
the conditional distribution over labels $y$ 
given an input $x\sim \calD_\calX$ is changed as follows:
For any $x$, the Probability Mass Function (PMF) over $y$ is defined as:  
$p_{\calDm} (\cdot \vert x) \defeq \frac{1 - p_{\calD}(\cdot \vert x)}{k - 1}$, where $ p_{\calD}(\cdot \vert x)$ is the PMF over $y$ for the distribution $\calD$. 

\begin{lemma} \label{lem:fit_mislabeled_multi}
    Assume the same setup as \thmref{thm:multiclass_ERM}. 
    Then for any $\delta >0$, with probability at least  $1-\delta$ 
    over the random draws of mislabeled data $\wt S_M$, we have 
    \begin{align}
        \error_\calD(\widehat f)  \le (k-1)\left(1 -\error_{\wt \calS_M}(\widehat f)\right) + (k-1)\sqrt{\frac{\log(1/\delta)}{m}}\,. \label{eq:lemma1_multi}
    \end{align}   
\end{lemma} 

\begin{proof}
   
    The main idea of the proof remains the same.
    We begin by regarding the clean portion of the data 
    ($S \cup \wt S_C$) as fixed. 
    Then, there exists a classifier $f^*$ 
    that is optimal over draws 
    of the mislabeled data $\wt S_M$. 
    
    However, in the multiclass case,
    we cannot as easily relate the population error on mislabeled data 
    to the population accuracy on clean data.   
    While for binary classification, 
    % we could upper bound $\error_{\wt \calS_M}$ 
    % with $1-\error_\calD$ 
    we could lower bound the population accuracy $1-\error_\calD$
    with the empirical error on mislabeled data $\error_{\wt \calS_M}$ 
    (in the proof of \lemref{lem:fit_mislabeled}), 
    for multiclass classification, 
    error on the mislabeled data 
    and accuracy on the clean data 
    in the population 
    are not so directly related.  
    To establish \eqref{eq:lemma1_multi},
    we break the error on the 
    (unknown) mislabeled data 
    into two parts: one term corresponds 
    to predicting the true label on mislabeled data, 
    and the other corresponds to predicting 
    neither the true label 
    nor the assigned (mis-)label.  
    Finally, we relate these errors to their
    population counterparts to establish \eqref{eq:lemma1_multi}. 
    
    Formally, 
    \begin{align}
    f^* \defeq \argmin_{f \in \calF} \error_{\widecheck {\calD}} (f)  + \lambda R(f) \,, \label{eq:modified_ERM_reg2}
    \end{align}
    where $$\widecheck \calD = \frac{n}{m+n} \calS + \frac{m_1}{m+n} \wt \calS_C  + \frac{m_2}{m+n}\calDm \,.$$ 
    That is, $\widecheck \calD$ is a combination 
    of the \emph{empirical distribution} 
    over correctly labeled data $S \cup \wt S_C$
    % in $S\cup \wt S$ 
    and the (population) distribution 
    over mislabeled data $\calDm$.
    Recall that 
    \begin{align}
    \wh f \defeq \argmin_{f \in \calF} \error_{\calS \cup \wt S} (f) + \lambda R(f) \,. \label{eq:orig_ERM_reg2}
    \end{align}
    % 
    % 
    Following the exact steps from the proof of \lemref{lem:lemma1_reg}, 
    with probability at least $1-\delta$, we have  
    \begin{align}
        \error_{ \wt \calS_M}(\wh f) \le \error_{\calDm}(\wh f) + \sqrt{\frac{\log(1/\delta)}{2 m_1}} \,. \label{eq:lemma1_final_multi_prev}
    \end{align}
    Similar to before, since $\wt S$ is obtained 
    by randomly labeling an unlabeled dataset, 
    we assume 
    $\frac{k}{k-1} m_1 \approx m$. 
    
    Now we will relate $\error_{\calDm} (\wh f)$ with $\error_{\calD}(\wh f)$. 
    Let $y^T$ denote the (unknown) true label 
    for a mislabeled point $(x, y)$ 
    (i.e., label before replacing it with a mislabel). 
    \begin{align*}    
         \Expt{(x, y) \in \sim \calDm}{\indict{ \wh f(x) \ne y }}  &= \underbrace{\Expt{(x, y) \in \sim \calDm}{\indict{ \wh f(x) \ne y \land \wh f(x) \ne y^T}}}_{\RN{1}} \\ &\qquad \qquad + \underbrace{\Expt{(x, y) \in \sim \calDm}{\indict{ \wh f(x) \ne y \land \wh f(x) = y^T}}}_{\RN{2}} \,. \numberthis \label{eq:excess_term}
    \end{align*}
    Clearly, term 2 is one minus the accuracy 
    on the clean unseen data, i.e.,
    \begin{align}
        \RN{2} = 1 - \Expt{{x,y} \sim \calD}{ \indict{ \wh f(x) \ne y}} = 1- \error_{\calD}(\wh f) \,. \label{eq:term1}    
    \end{align}
    Next, we relate term 1 with the error on the unseen clean data. 
    We show that term 1 is equal to the error on the unseen clean data 
    scaled by $\frac{k-2}{k-1}$,
    where $k$ is the number of labels.
    Using the definition of mislabeled distribution $\calDm$,  
    we have 
    \begin{align}
        \RN{1} = \frac{1}{k-1} \left( \Expt{(x, y) \in \sim \calD}{ \sum_{i \in \calY \land i\ne y}  \indict{ \wh f(x) \ne i \land \wh f(x) \ne y}} \right) = \frac{k-2}{k-1} \error_{\calD}(\wh f) \,.\label{eq:term2}
    \end{align}    

    Combining the result in \eqref{eq:term1}, \eqref{eq:term2} and \eqref{eq:excess_term}, we have 
    \begin{align}
        \error_{\calDm}(\wh f) = 1- \frac{1}{k-1} \error_{\calD}(\wh f) \,.\label{eq:combine_terms}
    \end{align}
    Finally, combining the result in \eqref{eq:combine_terms} 
    with equation \eqref{eq:lemma1_final_multi_prev}, 
    we have with probability $1-\delta$, 
    \begin{align}
      \error_{\calD}(\wh f) \le  (k-1) \left( 1- \error_{ \wt \calS_M}(\wh f) \right)  + (k-1) \sqrt{\frac{k \log(1/\delta)}{ 2(k-1)m}} \,. \label{eq:lemma1_final_multi}
    \end{align}
\end{proof}

\begin{lemma} \label{lem:mislabeled_error_multi}
    Assume the same setup as \thmref{thm:multiclass_ERM}. 
    Then for any $\delta >0$, 
    with probability at least $1-\delta$ 
    over the random draws of $\wt S$, we have  
    % \begin{align}
        $$\abs{k\error_{\wt \calS}(\widehat f) - \error_{\wt \calS_C}(\widehat f) -  (k-1)\error_{\wt \calS_M}(\widehat f) } \le  2k\sqrt{\frac{\log(4/\delta)}{2m}}\,. $$ % \label{eq:lemma2}
    % \end{align}   
    %  for some constant $c_3 \le 1.0\,$.
\end{lemma} 


\begin{proof}
    Recall $\error_{\wt S} (f) = \frac{m_1}{m} \error_{\wt S_M}(f) + \frac{m_2}{m} \error_{\wt S_C}(f)$. Hence, we have 
    \begin{align*}
        k\error_{\wt S}(f) - (k-1)\error_{\wt S_M}(f) - \error_{\wt S_C}(f) &= (k-1)\left(\frac{k m_1}{(k-1) m} \error_{\wt S_M}(f) - \error_{\wt S_M}(f)\right) \\ & \qquad \qquad + \left(\frac{km_2}{m} \error_{\wt S_C}(f) - \error_{\wt S_C}(f)\right) \\ &= k \left[ \left(\frac{m_1}{m} - \frac{k-1}{k}\right) \error_{\wt S_M}(f) + \left(\frac{m_2}{m} - \frac{1}{k} \right) \error_{\wt S_C} (f) \right] \,.
    \end{align*} 
    Since the dataset is randomly labeled, 
    we have with probability at least $1-\delta$, 
    $\left(\frac{m_1}{m} - \frac{k-1}{k}\right) \le \sqrt{\frac{\log(1/\delta)}{2m}}$. 
    Similarly, we have with probability at least $1-\delta$, 
    $\left(\frac{m_2}{m} - \frac{1}{k}\right) \le \sqrt{\frac{\log(1/\delta)}{2m}}$. 
    Using union bound, we have with probability at least $1-\delta$
    % \begin{align}
    %     2\error_{\wt S} - \error_{\wt S_M}(f) - \error_{\wt S_C}(f) \le \sqrt{\frac{\log(2/\delta)}{2m}} \left(\error_{\wt S_M}(f) + \error_{\wt S_C}(f) \right) \le 2\sqrt{\frac{\log(2/\delta)}{2m}} \,. \label{eq:lemma2_final}
    % \end{align}
    \begin{align}
        k\error_{\wt S}(f) - (k-1)\error_{\wt S_M}(f) - \error_{\wt S_C}(f)  \le k \sqrt{\frac{\log(2/\delta)}{2m}} \left(\error_{\wt S_M}(f) + \error_{\wt S_C}(f) \right) \,. \label{eq:lemma2_final_multi}
    \end{align}

    % We obtain the desired result by using 
\end{proof}

\begin{lemma} \label{lem:clear_error_multi}
    Assume the same setup as \thmref{thm:multiclass_ERM}. 
    Then for any $\delta >0$, with probability at least $1-\delta$ 
    over the random draws of $\wt S_C$ and $S$, we have 
    % \begin{align}
        $$\abs{\error_{\wt \calS_C}(\widehat f) - \error_{\calS}(\widehat f) } \le 1.5 \sqrt{\frac{k\log(2/\delta)}{2m}}\,.$$ %\label{eq:lemma3}
    % \end{align}   
    % for some constant $c_2 \le 1.2\,$.
\end{lemma} 
\begin{proof}
    % Recall 0-1 error on each point  $(x,y) \in S \cup \wt S$ is given by $\I{ f(x)\ne y}$.
    In the set of correctly labeled points $S \cup \wt S_C$,
    we have $S$ as a random subset of $S \cup \wt S_C$. 
    Hence, using Hoeffding's inequality 
    for sampling without replacement 
    (\lemref{lem:hoeffding_sampling}), 
    we have with probability at least $1-\delta$
    \begin{align}
        \error_{\wt \calS_c} (\wh f)- \error_{\calS \cup \wt \calS_C}( \wh f) \le  \sqrt{\frac{\log(1/\delta)}{2m_2}} \,.
    \end{align}
    Re-writing $\error_{\calS \cup \wt \calS_C}( \wh f)$ 
    as $\frac{m_2}{m_2 + n} \error_{\wt \calS_C }(\wh f) + \frac{n}{m_2 + n} \error_{\calS }(\wh f)$, 
    we have with probability at least $1-\delta$
    \begin{align}
       \left(\frac{n}{n+m_2}\right) \left(\error_{\wt \calS_c} (\wh f)- \error_{\calS}( \wh f) \right) \le  \sqrt{\frac{\log(1/\delta)}{2m_2}} \,.
    \end{align}
    As before, assuming $km_2 \approx m$, 
    we have with probability at least $1-\delta$ 
    \begin{align}
        \error_{\wt \calS_c} (\wh f)- \error_{\calS}( \wh f) \le \left(1+\frac{m_2}{n}\right)  \sqrt{\frac{k\log(1/\delta)}{2m}} \le \left( 1 + \frac{1}{k}\right) \sqrt{\frac{k\log(1/\delta)}{2m}} \,. \label{eq:lemma3_final_multi}
    \end{align} 
\end{proof}

\begin{proof}[Proof of \thmref{thm:multiclass_ERM}] 
    Having established these core intermediate results, 
    we can now combine above three lemmas. 
    In particular, we bound the population error 
    on clean data ($\error_\calD(\wh f)$) as follows:  
    \begin{enumerate}[(i)]
        \item First, use \eqref{eq:lemma1_final_multi}, 
        to obtain an upper bound on the population error on clean data, 
        i.e., with probability at least $1-\delta/4$, we have
        \begin{align}
            \error_{ \calD} (\wh f) \le (k-1)\left(1 - \error_{ \wt \calS_M}(\wh f) \right) + (k-1) \sqrt{\frac{k\log(4/\delta)}{2(k-1)m}} \,. 
        \end{align}
        \item  Second, use \eqref{eq:lemma2_final_multi}
        to relate the error on the mislabeled fraction 
        with error on clean portion of randomly labeled data 
        and error on whole randomly labeled dataset, 
        i.e., with probability at least $1-\delta/2$, we have 
        \begin{align}
            - (k-1)\error_{\wt S_M}(f) \le \error_{\wt S_C}(f) - k\error_{\wt S}  + k\sqrt{\frac{\log(4/\delta)}{2m}}  \,. 
        \end{align} 
        \item Finally, use \eqref{eq:lemma3_final_multi} 
        to relate the error on the clean portion of randomly labeled data 
        and error on clean training data, 
        i.e., with probability $1-\delta/4$, we have 
        \begin{align}
            \error_{\wt \calS_C} (\wh f)\le - \error_{\calS}( \wh f) + \left(1 + \frac{m}{kn} \right) \sqrt{\frac{k\log(4/\delta)}{2m}} \,. 
        \end{align} 
    \end{enumerate}

    Using union bound on the above three steps, 
    we have with probability at least $1-\delta$: 
    \begin{align}
        \error_\calD (\wh f) \le \error_{\calS}(\wh f) + (k-1) - k\error_{\wt \calS}(\wh f)   + (\sqrt{k(k-1)} + k + \sqrt{k} + \frac{m}{n\sqrt{k}})  \sqrt{\frac{\log(4/\delta)}{2m}} \,.\label{eq:multiclass_ERM_final}
    \end{align}
    Simplifying the term in RHS of \eqref{eq:multiclass_ERM_final}, 
    we get the desired result. 
    % Note that since $\frac{m}{n\sqrt{k}}$ 
    % is much smaller than the sum of the other terms
    % the other terms in summation, 
    % we ignore $\frac{m}{n\sqrt{k}}$  
    % Z: ??? --- great
    % that 
    % them
    in the final bound. 
    % we ignore that in the final bound. 
    % Note that $(1/\sqrt{2} + 2.5)$ is a loose constant. In experiments, we use the ratio $\frac{m}{n}$
    %  the exact error $\error_{\wt \calS}(\wh f)$ 
    % to evaluate R.H.S.    
\end{proof}

\newpage
\section{Proofs from \secref{sec:linear_models}}\label{app:proof_gd}
We suppose that the parameters of the linear function 
are obtained via gradient descent on 
the following $L_2$ regularized problem: 
\begin{align}
    % n in denominator is avoided deliberately
    \calL_S(w; \lambda) \defeq \sum_{i=1}^n{(w^Tx_i - y_i)^2} + \lambda \norm{w}{2}^2 \,, \label{eq:l2_MSE_app}   
\end{align}
where $\lambda\ge0$ is a regularization parameter. 
We assume access to a clean dataset 
$S = \{(x_i, y_i)\}_{i=1}^n \sim \calD^n$ 
and randomly labeled dataset 
$\wt S = \{(x_i, y_i)\}_{i=n+1}^{n+m} \sim \wt \calD^m$. 
Let $\bX = [x_1, x_2, \cdots, x_{m+n}]$ 
and $\by = [y_1, y_2, \cdots, y_{m+n}]$. 
Fix a positive learning rate $\eta$ such that 
$\eta \le 1/\left(\norm{\bX^T\bX}{\text{op}} + \lambda^2\right)$ 
and an initialization $w_0 = 0$. 
% \todos{Assumption made for simplicty}. 
Consider the following gradient descent iterates 
to minimize objective \eqref{eq:l2_MSE_app} on $S \cup \wt S$:
\begin{align}
w_t = w_{t-1} - \eta \grad_w \calL_{S \cup \wt S} (w_{t-1}; \lambda) \quad \forall t=1,2,\ldots \label{eq:GD_iterates_app}
\end{align} 
Then we have $\{ w_t\}$ converge to the limiting solution 
$\wh w = \left( \bX^T\bX+\lambda \boldsymbol{I}\right)^{-1}\bX^T\by$. Define $\widehat f (x) \defeq f(x ; \wh w) $.  

% \subsection{\textcolor{red}{Errata}}

% We wish to correct the following error in the body:
% \codref{cond:error_stability} is not enough 
% to guarantee the result in \thmref{thm:linear}. 
% We now present a slightly stronger condition 
% called \emph{hypothesis stability} 
% under which we obtain a result 
% similar to \thmref{thm:linear}. 

% This error doesn't change the main arguments of the proof,
% where we show that the empirical train error 
% is less than or equal to the leave-one-out error.
% We need a stronger condition to relate leave-one-out error 
% with the population error of the original classifier. 
% Specifically, while \codref{cond:error_stability} 
% relates the average population error of leave-one-out classifiers 
% with the population error of the original classifier, 
% we need the new condition to show the concentration 
% of the empirical leave-one-out error 
% and average population error of leave-one-out classifiers. 
% main takeaway 

% Note that the new condition, 
% while being stronger than the previous one, 
% still doesn't imply generalization \citep{bousquet2002stability,elisseeff2003leave,abou2019exponential}. 
% Overall, the main results in \secref{sec:ERM_training} 
% and takeaways of the paper remain unaffected by the error.  

% We now present the new condition 
% and a corrected statement of \thmref{thm:linear}. 
% Recall, for a given training set $S \sim \calD^n $, 
% we use $S_{(i)}$ to denote the training set $S$ 
% with the $i^{\text{th}}$ point removed.

% \begin{condition}[Hypothesis Stability] 
%     \label{cond:hypothesis_stability}
%     We have $\beta$ hypothesis stability 
%     if our training algorithm $\calA$ satisfies the following: 
%     \begin{align*}
%     % ${\sum_{i=1}^n \frac{\error_{\calD}( f(\calA, S_{(i)}))}{n} - \error_\calD(f(\calA, S))} \le \beta\,$.
%     \forall i \in \{1,2,\ldots, n\}, \quad  \Expt{\calS, (x,y) \in \calD}{ \abs{\error\left( f(x) ,y  \right) - \error\left( f_{(i)}(x), y \right) }} \le \frac{\beta}{n} \,,
%     \end{align*}
%     where $f_{(i)} \defeq f(\calA, S_{(i)})$ and $ f \defeq f(\calA, S)$.
% \end{condition}

% \begin{theorem}[Correct statement of \thmref{thm:linear}] \label{thm:new_linear}
%     Assume that this gradient descent algorithm satisfies \codref{cond:hypothesis_stability}
%     with $\beta=\calO(1)$.  
%     Then for any $\delta >0$, with probability at least $1-\delta$ 
%     over the random draws of datasets $\wt S$ and $S$, we have:
%     \begin{align}
%         \error_\calD(\widehat f) \le \error_\calS(\widehat f) + 1 - 2 \error_{\wt\calS}(\widehat f) + \left(\frac{1}{\sqrt{2}} + 1.5 \right) \sqrt{\frac{\log(4/\delta)}{m}} + \sqrt{\frac{4}{\delta}\left(\frac{1}{m} +\frac{3\beta}{m+n} \right)}  \,. \label{eq:gd_error}
%     \end{align} 
%     % for some constant $c\le 3.2$.
% \end{theorem}

\subsection{Proof of \thmref{thm:linear}}
We use a standard result from linear algebra, 
namely the Shermann-Morrison formula 
\citep{sherman1950adjustment} for matrix inversion:  

\begin{lemma}[\citet{sherman1950adjustment}] \label{lem:sherman}
    Suppose $\bA \in \Real^{n \times n}$ 
    is an invertible square matrix 
    and $u,v \in \Real^n$ are column vectors. 
    Then $\bA + uv^T$ is invertible iff $1 + v^T \bA u \ne 0$ 
    and in particular
    \begin{align}
        (\bA + u v^T)^{-1} = \bA^{-1}  - \frac{\bA^{-1} uv^T \bA^{-1} }{ 1 + v^T \bA^{-1} u} \,.
    \end{align}   
\end{lemma}
\newcommand\byy[1]{\by_{\left(#1\right)}}
\newcommand\bXX[1]{\bX_{\left(#1\right)}}
\newcommand\ff[1]{\wh f_{\left(#1\right)}}

For a given training set $S \cup \wt S_C$, 
define leave-one-out error 
on mislabeled points in the training data 
as $$\error_{\text{LOO}(\wt S_M) } = \frac{\sum_{(x_i, y_i) \in \wt S_M} \error( f_{(i)}( x_i), y_i)}{ \abs{\wt S_M }} \,, $$
where $f_{(i)} \defeq f(\calA, (S \cup \wt S)_{(i)})$. 
To relate empirical leave-one-out error and population error 
with hypothesis stability condition, 
we use the following lemma:   

\begin{lemma}[\citet{bousquet2002stability}] \label{lem:stability_error}
    For the leave-one-out error, we have
    \begin{align}
        \Expo{ \left( \error_{\calDm}(\wh f) -\error_{\text{LOO}(\wt S_M) } \right)^2 } \le \frac{1}{2m_1}+  \frac{3\beta}{n + m}\,.
    \end{align}   
    % where $ f \defeq f(\calA, S \cup \wt S) $.
\end{lemma}

Proof of the above lemma is similar 
to the proof of Lemma 9 in \citet{bousquet2002stability} 
and can be found in \appref{app:proof_lem_error}. 
% 
% Before presenting the result, we introduce some notation. 
Before presenting the proof of \thmref{thm:linear}, 
we introduce some more notation. 
Let $\bX_{(i)}$ denote the matrix of covariates 
with the $i^{\text{th}}$ point removed. 
Similarly, let $\by_{(i)}$ be the array of responses 
with the $i^{\text{th}}$ point removed. 
Define the corresponding regularized GD solution 
as $\wh w_{(i)} = \left( \bXX{i}^T\bXX{i}+\lambda \boldsymbol{I}\right)^{-1}\bXX{i}^T\byy{i}$. 
Define $\ff{i}(x) \defeq f(x ; \wh w_{(i)}) $.

\begin{proof}[Proof of \thmref{thm:linear}]
    Because squared loss minimization does not imply 0-1 error minimization, 
    we cannot use arguments from \lemref{lem:fit_mislabeled}. 
    This is the main technical difficulty. 
    To compare the 0-1 error at a train point with an unseen point, 
    we use the closed-form expression for $\widehat{w}$ 
    and Shermann-Morrison formula 
    to upper bound training error 
    with leave-one-out cross validation error. 
    
    The proof is divided into three parts: 
    In part one, we show that 0-1 error 
    on mislabeled points in the training set 
    is lower than the error obtained 
    by leave-one-out error at those points. 
    In part two, we relate this leave-one-out error 
    with the population error on mislabeled distribution
    using \codref{cond:hypothesis_stability}.
    While the empirical leave-one-out error is an unbiased estimator 
    of the average population error of leave-one-out classifiers, 
    we need hypothesis stability 
    to control the variance 
    of empirical leave-one-out error. 
    Finally, in part three, we show 
    that the error on the mislabeled training points 
    can be estimated with just the randomly labeled 
    and clean training data (as in proof of \thmref{thm:error_ERM}).  

    \textbf{Part 1 {} {}} First we relate training error with leave-one-out error.        
    For any training point $(x_i, y_i)$ in $\wt S \cup S$, we have 
    \begin{align}
        \error(\wh f(x_i), y_i ) &= \indict{ y_i \cdot x_i^T \wh w < 0 } = \indict{ y_i \cdot x_i^T \left( \bX^T\bX+\lambda \boldsymbol{I}\right)^{-1}\bX^T\by < 0 } \\
        &= \indict{ y_i \cdot x_i^T \underbrace{\left( \bXX{i}^T\bXX{i} + x_i ^T x_i +\lambda \boldsymbol{I}\right)^{-1}}_{\RN{1}} (\bXX{i}^T\byy{i} + y_i \cdot x_i) < 0 } \,.
    \end{align}
    Letting $\bA = \left(\bXX{i}^T\bXX{i} +\lambda \boldsymbol{I}\right)$ 
    and using \lemref{lem:sherman} on term 1, we have 
    \begin{align}
        \error(\wh f(x_i), y_i ) &= \indict{ y_i \cdot x_i^T \left[\bA^{-1} -  \frac{\bA^{-1} x_i x_i^T \bA^{-1}}{ 1 + x_i ^T \bA^{-1} x_i } \right] (\bXX{i}^T\byy{i} + y_i \cdot x_i) < 0 } \\
        &= \indict{ y_i \cdot\left[ \frac{ x_i^T \bA^{-1} ( 1 + x_i ^T \bA^{-1} x_i ) -  x_i^T \bA^{-1} x_i x_i^T \bA^{-1}}{ 1 + x_i ^T \bA ^{-1}x_i } \right] (\bXX{i}^T\byy{i} + y_i \cdot x_i) < 0 } \\
        &= \indict{ y_i \cdot\left[ \frac{ x_i^T \bA^{-1}}{ 1 + x_i ^T \bA ^{-1}x_i } \right] (\bXX{i}^T\byy{i} + y_i \cdot x_i) < 0 } \,.
    \end{align}

    Since $1 + x_i^T \bA^{-1} x_i > 0$, we have 
    \begin{align}
        \error(\wh f(x_i), y_i ) &= \indict{ y_i \cdot x_i^T \bA^{-1} (\bXX{i}^T\byy{i} + y_i \cdot x_i) < 0 } \\
        &= \indict{ x_i^T \bA^{-1} x_i +  y_i \cdot x_i^T \bA^{-1} (\bXX{i}^T\byy{i}) < 0 } \\
        &\le \indict{ y_i \cdot x_i^T \bA^{-1} (\bXX{i}^T\byy{i}) < 0 } = \error(\ff{i}(x_i), y_i ) \,.\label{eq:LOO_error}
    \end{align}

    Using \eqref{eq:LOO_error}, we have 
    \begin{align}
        \error_{\wt \calS_M } (\wh f) \le \error_{\text{LOO} (\wt S_M)} \defeq \frac{\sum_{(x_i, y_i) \in \wt S_M} \error(\ff{i}(x_i), y_i ) }{\abs{\wt \calS_M}}\label{eq:LOO_error_final} \,.
    \end{align}
    \textbf{Part 2 {}{}} We now relate RHS in \eqref{eq:LOO_error_final} 
    with the population error on mislabeled distribution. 
    To do this, we leverage \codref{cond:hypothesis_stability} 
    and \lemref{lem:stability_error}. 
    In particular, we have 

    \begin{align}
        \Expt{\calS \cup \wt \calS_M }{ \left(\error_{\calDm}(\wh f) - \error_{\text{LOO} (\wt S_M)}\right)^2 } \le \frac{1}{2m_1} + \frac{3\beta}{m+n} \,.
    \end{align}

    Using Chebyshev's inequality, with probability at least $1-\delta$, we have 
    \begin{align}
        \error_{\text{LOO} (\wt S_M)} \le  \error_{\calDm}(\wh f)   + \sqrt{\frac{1}{\delta}\left(\frac{1}{2m_1} +\frac{3\beta}{m+n} \right)} \,. \label{eq:final_mislabeled_linear}
    \end{align}
    

    \textbf{Part 3 {}{}} Combining \eqref{eq:final_mislabeled_linear} and \eqref{eq:LOO_error_final}, we have 

    \begin{align}
        \error_{\wt \calS_M } (\wh f) \le \error_{\calDm}(\wh f)   + \sqrt{\frac{1}{\delta}\left(\frac{1}{2m_1} +\frac{3\beta}{m+n} \right)} \,. \label{eq:linear_parallel_lem1}
    \end{align}

    Compare \eqref{eq:linear_parallel_lem1} with \eqref{eq:lemma1_final} 
    in the proof of \lemref{lem:fit_mislabeled}. 
    We obtain a similar relationship 
    between $\error_{\wt \calS_M }$ and $\error_{\calDm}$ 
    but with a polynomial concentration 
    instead of exponential concentration. 
    In addition, since we just use concentration arguments 
    to relate mislabeled error to the errors
    on the clean and unlabeled portions 
    of the randomly labeled data, 
    we can directly use the results 
    in \lemref{lem:mislabeled_error} and \lemref{lem:clear_error}. 
    Therefore, combining results in \lemref{lem:mislabeled_error}, \lemref{lem:clear_error}, and \eqref{eq:linear_parallel_lem1} with union bound, 
    we have with probability at least $1-\delta$
    \begin{align}
        \error_\calD(\widehat f) \le \error_\calS(\widehat f) + 1 - 2 \error_{\wt\calS}(\widehat f) + \left(\sqrt{2}\error_{\wt\calS}(\widehat f) + 1 + \frac{m}{2n} \right) \sqrt{\frac{\log(4/\delta)}{m}} + \sqrt{\frac{4}{\delta}\left(\frac{1}{m} +\frac{3\beta}{m+n} \right)}  \,.
    \end{align}
    

       
\end{proof}

\subsection{Extension to multiclass classification} \label{app:multiclass_linear}
For multiclass problems with squared loss minimization, as standard practice, we consider one-hot encoding for the underlying label, i.e., a class label $c \in [k]$ is treated as $(0, \cdot, 0,1,0, \cdot, 0) \in \Real^k$ (with $c$-th coordinate being 1).  As before, we suppose that the parameters of the linear function 
are obtained via gradient descent on the following $L_2$ regularized problem: 
\begin{align}
    % n in denominator is avoided deliberately
    \calL_S(w; \lambda) \defeq \sum_{i=1}^n\norm{w^Tx_i - y_i}{2}^2 + \lambda \sum_{j=1}^k \norm{w_j}{2}^2 \,, \label{eq:l2_multiclass_MSE_app}   
\end{align}
where $\lambda\ge0$ is a regularization parameter. 
We assume access to a clean dataset 
$S = \{(x_i, y_i)\}_{i=1}^n \sim \calD^n$ 
and randomly labeled dataset 
$\wt S = \{(x_i, y_i)\}_{i=n+1}^{n+m} \sim \wt \calD^m$. 
Let $\bX = [x_1, x_2, \cdots, x_{m+n}]$ 
and $\by = [e_{y_1}, e_{y_2}, \cdots, e_{y_{m+n}}]$. 
Fix a positive learning rate $\eta$ such that 
$\eta \le 1/\left(\norm{\bX^T\bX}{\text{op}} + \lambda^2\right)$ 
and an initialization $w_0 = 0$. 
% \todos{Assumption made for simplicty}. 
Consider the following gradient descent iterates 
to minimize objective \eqref{eq:l2_MSE_app} on $S \cup \wt S$:
\begin{align}
{w_j}^t = {w_j}^{t-1} - \eta \grad_{w_j} \calL_{S \cup \wt S} (w^{t-1}; \lambda) \quad \forall t=1,2,\ldots \text{ and } j=1,2,\ldots,k  \,. \label{eq:GD_multi_iterates_app}
\end{align} 
Then we have $\{ {w_j}^t\}$ for all $j =1,2,\cdots, k$ converge to the limiting solution 
$\wh w_j = \left( \bX^T\bX+\lambda \boldsymbol{I}\right)^{-1}\bX^T\by_j$. Define $\widehat f (x) \defeq f(x ; \wh w) $.  

\begin{theorem}\label{thm:multi_linear}
    Assume that this gradient descent algorithm satisfies \codref{cond:hypothesis_stability}
    with $\beta=\calO(1)$.  
    Then for a multiclass classification problem wth $k$ classes, for any $\delta >0$, with probability at least $1-\delta$, we have:
    \begin{align*}
        \error_\calD(\widehat f) \le \error_\calS(\widehat f) &+ (k-1)\left(1 - \frac{k}{k-1} \error_{\wt\calS}(\widehat f) \right) \\ &+ \left(k + \sqrt{k} + \frac{m}{n\sqrt{k}} \right) \sqrt{\frac{\log(4/\delta)}{2m}} + \sqrt{k(k-1)} \sqrt{\frac{4}{\delta}\left(\frac{1}{m} +\frac{3\beta}{m+n} \right)}  \,. \numberthis \label{eq:gd_multi_error}
    \end{align*} 
    % for some constant $c\le 3.2$.
\end{theorem}
\begin{proof}
    The proof of this theorem is divided into two parts. In the first part, we relate the error on the mislabeled samples with the population error on the mislabeled data. Similar to the proof of \thmref{thm:linear}, we use Shermann-Morrison formula to upper bound training error with leave-one-out error on each $\wh w^j$. Second part of the proof follows entirely from the proof of \thmref{thm:multiclass_ERM}. In essence, the first part derives an equivalent of \eqref{eq:lemma1_final_multi_prev} for GD training with squared loss and then the second part follows from the proof  of \thmref{thm:multiclass_ERM}. 
    
    \textbf{Part-1:} Consider a training point $(x_i,y_i)$ in $\wt S \cup S $. For simplicity, we use $c_i$ to denote the class of $i$-th point and use $y_i$ as the corresponding one-hot embedding. Recall error in multiclass point is given by $\error(\wh f(x_i), y_i ) = \indict{ c_i \not \in \argmax x_i^T \wh w }$. Thus, there exists a $j \ne c_i \in [k]$, such that we have
     \begin{align}
        \error(\wh f(x_i), y_i ) &= \indict{ c_i \not \in \argmax x_i^T \wh w } = \indict{ x_i^T \wh w_{c_i} < x_i^T \wh w_{j}  } \\ &= \indict{ x_i^T \left( \bX^T\bX+\lambda \boldsymbol{I}\right)^{-1}\bX^T\by_{c_i} < x_i^T \left( \bX^T\bX+\lambda \boldsymbol{I}\right)^{-1}\bX^T\by_{j} } \\
        &= \indict{ x_i^T \underbrace{\left( \bXX{i}^T\bXX{i} + x_i ^T x_i +\lambda \boldsymbol{I}\right)^{-1}}_{\RN{1}} \left(\bXX{i}^T{\by_{c_i}}_{(i)} + x_i - \bXX{i}^T{\by_{j}}_{(i)}\right) < 0 } \,.
    \end{align}
    Letting $\bA = \left(\bXX{i}^T\bXX{i} +\lambda \boldsymbol{I}\right)$ 
    and using \lemref{lem:sherman} on term 1, we have 
    \begin{align}
        \error(\wh f(x_i), y_i ) &= \indict{ x_i^T \left[\bA^{-1} -  \frac{\bA^{-1} x_i x_i^T \bA^{-1}}{ 1 + x_i ^T \bA^{-1} x_i } \right]  \left(\bXX{i}^T{\by_{c_i}}_{(i)} + x_i - \bXX{i}^T{\by_{j}}_{(i)}\right) < 0 } \\
        &= \indict{ \left[ \frac{ x_i^T \bA^{-1} ( 1 + x_i ^T \bA^{-1} x_i ) -  x_i^T \bA^{-1} x_i x_i^T \bA^{-1}}{ 1 + x_i ^T \bA ^{-1}x_i } \right]  \left(\bXX{i}^T{\by_{c_i}}_{(i)} + x_i - \bXX{i}^T{\by_{j}}_{(i)}\right) < 0 } \\
        &= \indict{ \left[ \frac{ x_i^T \bA^{-1}}{ 1 + x_i ^T \bA ^{-1}x_i } \right]  \left(\bXX{i}^T{\by_{c_i}}_{(i)} + x_i - \bXX{i}^T{\by_{j}}_{(i)}\right) < 0} \,.
    \end{align}
    Since $1 + x_i^T \bA^{-1} x_i > 0$, we have 
    \begin{align}
        \error(\wh f(x_i), y_i ) &= \indict{ x_i^T \bA^{-1}  \left(\bXX{i}^T{\by_{c_i}}_{(i)} + x_i - \bXX{i}^T{\by_{j}}_{(i)}\right) < 0 } \\
        &= \indict{ x_i^T \bA^{-1} x_i +  x_i^T \bA^{-1}  \bXX{i}^T{\by_{c_i}}_{(i)}  - x_i^T\bA^{-1}  \bXX{i}^T{\by_{j}}_{(i)} < 0 } \\
        &\le \indict{  x_i^T \bA^{-1}  \bXX{i}^T{\by_{c_i}}_{(i)}  - x_i^T\bA^{-1}  \bXX{i}^T{\by_{j}}_{(i)} < 0  } = \error(\ff{i}(x_i), y_i ) \,.\label{eq:LOO_error_multi}
    \end{align}
    Using \eqref{eq:LOO_error_multi}, we have 
    \begin{align}
        \error_{\wt \calS_M } (\wh f) \le \error_{\text{LOO} (\wt S_M)} \defeq \frac{\sum_{(x_i, y_i) \in \wt S_M} \error(\ff{i}(x_i), y_i ) }{\abs{\wt \calS_M}}\label{eq:LOO_error_multi_final} \,.
    \end{align}
    
    We now relate RHS in \eqref{eq:LOO_error_final} 
    with the population error on mislabeled distribution. 
    Similar as before, to do this, we leverage \codref{cond:hypothesis_stability} 
    and \lemref{lem:stability_error}. Using  \eqref{eq:final_mislabeled_linear} and \eqref{eq:LOO_error_multi_final}, we have 
    \begin{align}
        \error_{\wt \calS_M } (\wh f) \le \error_{\calDm}(\wh f)   + \sqrt{\frac{1}{\delta}\left(\frac{1}{2m_1} +\frac{3\beta}{m+n} \right)} \,. \label{eq:linear_multi_parallel_lem1}
    \end{align}
    
    We have now derived a parallel to \eqref{eq:lemma1_final_multi_prev}. Using the same arguments in the proof of \lemref{lem:fit_mislabeled_multi}, we have 
    \begin{align}
      \error_{\calD}(\wh f) \le  (k-1) \left( 1- \error_{ \wt \calS_M}(\wh f) \right)  + (k-1)\sqrt{\frac{k}{\delta(k-1)}\left(\frac{1}{2m_1} +\frac{3\beta}{m+n} \right)}  \,. \label{eq:lemma1_linear_final_multi}
    \end{align}
    
    \textbf{Part-2:} We now combine the results in \lemref{lem:mislabeled_error_multi} and \lemref{lem:clear_error_multi} to obtain the final inequality in terms of quantities that can be computed from just the randomly labeled and clean data. Similar to the binary case, we obtained a polynomial concentration instead of exponential concentration. Combining \eqref{eq:lemma1_linear_final_multi} with \lemref{lem:mislabeled_error_multi} and \lemref{lem:clear_error_multi}, we have with probability at least $1-\delta$
    \begin{align*}
        \error_\calD(\widehat f) \le \error_\calS(\widehat f) &+ (k-1)\left(1 - \frac{k}{k-1} \error_{\wt\calS}(\widehat f) \right) \\ &+ \left(k + \sqrt{k} + \frac{m}{n\sqrt{k}} \right) \sqrt{\frac{\log(4/\delta)}{2m}} + \sqrt{k(k-1)} \sqrt{\frac{4}{\delta}\left(\frac{1}{m} +\frac{3\beta}{m+n} \right)}  \,. \numberthis \label{eq:gd_multi_error_proof}
    \end{align*} 
\end{proof}

\subsection{Discussion on \codref{cond:hypothesis_stability}} \label{app:discuss_cond1}
The quantity in LHS of \codref{cond:hypothesis_stability} 
measures how much the function learned by the algorithm 
(in terms of error on unseen point) will change 
when one point in the training set is removed. 
% Discussion on exponential concentration and stronger condition. 
% Notice that hypothesis stability implies error stability, i.e., \codref{cond:error_stability} \citep{bousquet2002stability}.  
% In summary, while error stability allowed us 
% to relate the average population error 
% of the leave-one-out classifiers 
% with the population error of the original classifier, 
We need hypothesis stability condition 
to control the variance of the empirical leave-one-out error to show concentration of average leave-one-error with the population error. 

Additionally, we note that while the dominating term in the RHS of \thmref{thm:linear} matches with the dominating term in ERM bound in \thmref{thm:error_ERM}, there is a polynomial concentration term 
(dependence on $1/\delta$ instead of $\log(\sqrt{1/\delta})$) 
in \thmref{thm:linear}. 
Since with hypothesis stability, 
we just bound the variance, 
the polynomial concentration is due 
to the use of Chebyshev's inequality 
instead of an exponential tail inequality
(as in \lemref{lem:fit_mislabeled}).
Recent works have highlighted that 
a slightly stronger condition than hypothesis stability 
can be used to obtain an exponential concentration 
for leave-one-out error \citep{abou2019exponential},
but we leave this for future work for now. 
% We leave 
% However, the constants 

% we also want to highlight  

\subsection{Formal statement and proof of \propref{prop:early_stop}} \label{app:formal_early_stop}

Before formally presenting the result, 
we will introduce some notation.  
By $\calL_{S}(w)$, we denote 
the objective in \eqref{eq:l2_MSE_app} with $\lambda=0$. 
Assume Singular Value Decomposition (SVD) of $\bX$
as $\sqrt{n} \bU \bS^{1/2} \bV^T$. 
Hence $\bX^T \bX = \bV \bS \bV^T$.
Consider the GD iterates defined in \eqref{eq:GD_iterates_app}. 
% 
We now derive closed form expression 
for the $t^\text{th}$ iterate of gradient descent:  
% 
\begin{align}
    w_t = w_{t-1} + \eta \cdot \bX^T (\by - \bX w_{t-1}) = (\bI - \eta \bV \bS \bV^T )w_{k-1} + \eta \bX^T \by \,.
\end{align}
Rotating by $\bV^T$, we get 
\begin{align}
    \wt w_t = (\bI - \eta\bS )\wt w_{k-1} + \eta \wt \by \label{eq:GD_recur},
\end{align}
where $\wt w_t = \bV^T w_t $ and $\wt \by = \bV^T \bX^T \by$. 
Assuming the initial point $w_0 = 0$ 
and applying the recursion in \eqref{eq:GD_recur}, we get
\begin{align}
    \wt w_t = \bS ^{-1} ( \bI - (\bI - \eta \bS)^k ) \wt \by \,, 
\end{align} 
Projecting solution back to the original space, we have 
\begin{align}
     w_t = \bV \bS ^{-1} ( \bI - (\bI - \eta \bS)^k ) \bV^T \bX^T \by \,. 
\end{align} 
% We will work with this GD solution at any iterate $t$ in the next proposition. 
Define $f_t(x) \defeq f(x;w_t)$ 
as the solution at the $t^{\text{th}}$ iterate. 
Let $\wt w_{\lambda} = \argmin_{w} \calL_\calS (w;\lambda) = (\bX^T \bX + \lambda \bI)^{-1} \bX^T \by = \bV (\bS + \lambda \bI )^{-1} \bV^T \bX^T \by $. 
% ) \,,$ for all $t=1,2,\ldots\,.$ 
and define $\wt f_\lambda(x) \defeq f(x;\wt w_\lambda)$ as the regularized solution. 
Assume $\kappa$ be the condition number 
of the population covariance matrix 
and let $s_\text{min}$ be the minimum positive 
singular value of the empirical covariance matrix. 
Our proof idea is inspired from recent work 
on relating gradient flow solution 
and regularized solution 
for regression problems \citep{ali2018continuous}. 
We will use the following lemma in the proof: 
\begin{lemma} \label{lem:ineq_soln}
    For all $x \in [0,1]$ and for all $ k \in \mathbb{N}$, 
    we have (a) $ \frac{kx}{1+kx} \le 1- (1-x)^k$ 
    and (b) $ 1- (1-x)^k \le 2 \cdot \frac{kx}{kx+1} $.
    %  where $g(c)$ is a constant dependent on $c$. For $c = 1$, $g(c) = 2.0$.   
\end{lemma}
\begin{proof}
    % [Proof of \lemref{lem:ineq_soln}]
    % Part (a) is easy. 
    Using $ (1-x)^k \le \frac{1}{1+kx}$, we have part (a). 
    For part (b), we numerically maximize 
    $\frac{ (1+kx ) (1 - (1-x)^k) }{kx}$ 
    for all $k\ge 1$ and for all $x \in [0, 1]$.  
\end{proof}

% 
% Next, 

\begin{prop}[Formal statement of \propref{prop:early_stop}] \label{prop:formal_early_stop}
Let $\lambda = \frac{1}{t\eta}$. 
For a training point $x$, we have 
\begin{align*}
    \Expt{x \sim \calS}{(f_t(x) - \wt f_\lambda(x))^2} &\le c(t,\eta) \cdot \Expt{x \sim \calS}{f_t(x)^2} \,, %\label{eq:early_stop}
\end{align*}
where $c(t, \eta) \defeq \min( 0.25, \frac{1}{s_\text{min}^2 t^2 \eta^2})$. 
Similarly for a test point, we have 
\begin{align*}
    \Expt{x \sim \calD_\calX}{(f_t(x) - \wt f_\lambda(x))^2} &\le \kappa \cdot c(t,\eta) \cdot \Expt{x \sim \calD_\calX}{f_t(x)^2} \,. %\label{eq:early_stop}
\end{align*}
\end{prop} 

\begin{proof}
    %%%%%%%%%%%%% 
    We want to analyze the expected squared difference output 
    of regularized linear regression 
    with regularization constant $\lambda = \frac{1}{\eta t}$ 
    and the gradient descent solution at the $t^\text{th}$ iterate. 
    We separately expand the algebraic expression 
    for squared difference at a training point and a test point. 
    % We start by considering the difference  
    Then the main step is to show that 
    $\left[ \bS ^{-1} ( \bI - (\bI - \eta \bS)^k )  - (\bS + \lambda \bI )^{-1}\right] \preceq c(\eta, t) \cdot \bS ^{-1} ( \bI - (\bI - \eta \bS)^k ) $.

    %%%%%%%%%%%%%
    
   \textbf{Part 1 {} {}} 
    First, we will analyze the squared difference 
    of the output at a training point 
    (for simplicity, we refer to $S \cup \wt S$ as $S$), i.e., 
    \begin{align}
        \Expt{ x \sim \calS }{\left(f_t(x) - \wt f_\lambda (x)\right)^2} &= \norm{\bX w_t - \bX \wt w_\lambda}{2}^2\\ &=   \norm{\bX \bV \bS ^{-1} ( \bI - (\bI - \eta \bS)^t ) \bV^T \bX^T \by - \bX \bV (\bS + \lambda \bI )^{-1} \bV^T \bX^T \by }{2}^2 \\
        &= \norm{\bX \bV \left(\bS ^{-1} ( \bI - (\bI - \eta \bS)^t ) - (\bS + \lambda \bI )^{-1} \right) \bV^T \bX^T \by  }{2} \\
        &=  \by^T \bV \bX \left( \underbrace{\bS ^{-1} ( \bI - (\bI - \eta \bS)^t ) - (\bS + \lambda \bI )^{-1}}_{\RN{1}} \right)^2 \bS \bV^T \bX^T \by \label{eq:train_GD_rel} \,.
        %  (\bX \bV \bS ^{-1} ( \bI - (\bI - \eta \bS)^k ) \bV^T \bX^T \by)^T \bX \bV \bS ^{-1} ( \bI - (\bI - \eta \bS)^k ) \bV^T \bX^T \by
    \end{align}
    We now separately consider term 1. 
    Substituting $\lambda = \frac{1}{t \eta}$, 
    we get
    \begin{align}
        \bS ^{-1} ( \bI - (\bI - \eta \bS)^t ) - (\bS + \lambda \bI )^{-1} &= \bS^{-1} \left( ( \bI - (\bI - \eta \bS)^t ) - (\bI + \bS^{-1} \lambda )^{-1}\right) \\
        &= \underbrace{\bS^{-1} \left( ( \bI - (\bI - \eta \bS)^t ) - (\bI + ( \bS t \eta)^{-1}  )^{-1}\right)}_{\bA} \,.
    \end{align}

    We now separately bound the diagonal entries in matrix $\bA$. 
    With $s_i$, we denote $i^{\text{th}}$ diagonal entry of $\bS$.
    Note that since $ \eta\le 1/\norm{S}{\text{op}}$, 
    for all $i$, $\eta s_i  \le 1$.  
    Consider $i^{\text{th}}$ diagonal term (which is non-zero) 
    of the diagonal matrix $\bA$, we have 
    \begin{align}
        \bA_{ii} = \frac{1}{s_i} \left(  1 - (1 - s_i \eta)^t - \frac{t \eta s_i}{1 + t \eta s_i } \right) &=  \frac{1 - (1 - s_i \eta)^t}{s_i} \left( \underbrace{ 1 - \frac{t \eta s_i}{(1 + t \eta s_i)(1 - (1 - s_i \eta)^t)}}_{\RN{2}} \right) \\ 
         &\le \frac{1}{2}\left[ \frac{1 - (1 - s_i \eta)^t}{ s_i} \right] \tag*{(Using \lemref{lem:ineq_soln} (b))} \,.
    \end{align} 
    Additionally, we can also show the following upper bound on term 2: 
    \begin{align}
         1 - \frac{t \eta s_i}{(1 + t \eta s_i)(1 - (1 - s_i \eta)^t)} &= \frac{(1 + t \eta s_i)(1 - (1 - s_i \eta)^t) - t \eta s_i }{(1 + t \eta s_i)(1 - (1 - s_i \eta)^t)} \\
         & \le  \frac{ 1 -  (1 - s_i \eta)^t - t \eta s_i (1 - s_i \eta)^t}{(1 + t \eta s_i)(1 - (1 - s_i \eta)^t)} \\
         & \le \frac{1}{t\eta s_i} \,. \tag{Using \lemref{lem:ineq_soln} (a)}
        %  &\le \frac{1}{2}\left[ \frac{1 - (1 - s_i \eta)^t}{ s_i} \right] \tag*{(Using \lemref{lem:ineq_soln})} \,.
    \end{align} 

    Combining both the upper bounds 
    on each diagonal entry $\bA_{ii}$, we have 
    \begin{align}
    \bA \preceq c_1(\eta, t) \cdot \bS^{-1} ( \bI - (\bI - \eta \bS)^t ) \,, \label{eq:upperbound_diagonal}
    \end{align}
    where $c_1(\eta, t ) = \min(0.5, \frac{1}{t s_i \eta })$. Plugging this into \eqref{eq:train_GD_rel}, we have 
    \begin{align}
        \Expt{ x \sim \calS }{\left(f_t(x) - \wt f_\lambda (x)\right)^2} &\le c(\eta, t) \cdot \by^T \bV \bX  \left( \bS^{-1} ( \bI - (\bI - \eta \bS)^t ) \right)^2 \bS \bV^T \bX^T \by \\
        &=   c(\eta, t) \cdot \by^T \bV \bX  \left( \bS^{-1} ( \bI - (\bI - \eta \bS)^t ) \right) \bS \left( \bS^{-1} ( \bI - (\bI - \eta \bS)^t ) \right) \bV^T \bX^T \by \\
        & =  c(\eta, t) \cdot \norm{\bX w_t}{2}^2 \\
        &= c(\eta, t) \cdot  \Expt{ x \sim \calS }{\left(f_t(x) \right)^2} \,,
    \end{align}
    where $c(\eta, t ) = \min(0.25, \frac{1}{t^2 s^2_i \eta^2 })$.

    \textbf{Part 2 {} {}} With $\bSigma$, 
    we denote the underlying true covariance matrix. 
    We now consider the squared difference of output at an unseen point: 
    \begin{align}
        \Expt{ x \sim \calD_{\calX} }{\left(f_t(x) - \wt f_\lambda (x)\right)^2} &= \Expt{x \sim \calD_{\calX}}{\norm{x^T w_t - x^T \wt w_\lambda}{2}} \\
        &=   \norm{x^T \bV \bS ^{-1} ( \bI - (\bI - \eta \bS)^t ) \bV^T \bX^T \by - x^T \bV (\bS + \lambda \bI )^{-1} \bV^T \bX^T \by }{2} \\
        &= \norm{x^T \bV \left(\bS ^{-1} ( \bI - (\bI - \eta \bS)^t ) - (\bS + \lambda \bI )^{-1} \right) \bV^T \bX^T \by  }{2} \\
        &= \by^T \bV \bX \left( \bS ^{-1} ( \bI - (\bI - \eta \bS)^t ) - (\bS + \lambda \bI )^{-1} \right) \bV^T \bSigma \bV \\ &\qquad \qquad \qquad \qquad \qquad \left( (\bI - (\bI - \eta \bS)^t ) - (\bS + \lambda \bI )^{-1} \right) \bV^T \bX^T \by \\
        &\le \sigma_{\text{max}} \cdot \by^T \bV \bX \left( \underbrace{\bS ^{-1} ( \bI - (\bI - \eta \bS)^t ) - (\bS + \lambda \bI )^{-1}}_{\RN{1}} \right)^2 \bV^T \bX^T \by \,, \label{eq:test_GD_rel}
        %  (\bX \bV \bS ^{-1} ( \bI - (\bI - \eta \bS)^k ) \bV^T \bX^T \by)^T \bX \bV \bS ^{-1} ( \bI - (\bI - \eta \bS)^k ) \bV^T \bX^T \by
    \end{align}
    where $\sigma_{\text{max}}$ is the maximum eigenvalue 
    of the underlying covariance matrix $\bSigma$. 
    Using the upper bound on term 1 in \eqref{eq:upperbound_diagonal}, 
    we have 
    \begin{align}
        \Expt{ x \sim \calD_{\calX} }{\left(f_t(x) - \wt f_\lambda (x)\right)^2} &\le \sigma_{\text{max}} \cdot c(\eta, t) \cdot \by^T \bV \bX  \left( \bS^{-1} ( \bI - (\bI - \eta \bS)^t ) \right)^2 \bV^T \bX^T \by \\
        &=   \kappa \cdot c(\eta, t) \cdot \sigma_{\text{min}}\cdot \norm{\bV \left( \bS^{-1} ( \bI - (\bI - \eta \bS)^t ) \right) \bV^T \bX^T \by}{2}^2 \\
        &\le \kappa \cdot c(\eta, t) \cdot \left[ \bV \left( \bS^{-1} ( \bI - (\bI - \eta \bS)^t ) \right) \bV^T \bX^T \right]^T \bSigma \\
        &\qquad \qquad \qquad \qquad \qquad \left[ \bV \left( \bS^{-1} ( \bI - (\bI - \eta \bS)^t ) \right) \bV^T \bX^T \right] \by \\
        & = \kappa \cdot c(\eta, t) \cdot \Expt{x \sim \calD_{\calX}}{\norm{x^T w_t}{2}} \,.
    \end{align}
% 
% 
    % Since $ \eta\le 1/\norm{S}{\text{op}}$, invoking \lemref{lem:ineq_soln} to upper bound term 1 with
\end{proof}

\subsection{Extension to deep learning} \label{appsubsec:ext_DL}
Under \asmpref{appsubsec:justifying_assumption1}, we present the formal result parallel to \thmref{thm:multiclass_ERM}. 
\begin{theorem} \label{thm:multiclass_ERM_algoA}
    Consider a multiclass classification problem 
    with $k$ classes. Under \asmpref{asmp:deep_models}, 
    for any $\delta >0$, with probability at least $1-\delta$,
    we have
    \vspace{-10pt}
    \begin{align*}
        \error_\calD(\widehat f)  \le \error_\calS(\widehat f) + (k-1) \left(1 - \tfrac{k}{k-1} \error_{\wt\calS}(\widehat f)\right) + c\sqrt{\frac{\log(\frac{4}{\delta})}{2m}} \,,\numberthis \label{eq:multiclass_ERM_deep}
    % \vspace{-20pt}
    \end{align*}
    for some constant $c \le ((c+1) k+\sqrt{k} + \frac{m}{n\sqrt{k}})$.
\end{theorem}

The proof follows exactly as in step (i) to (iii) in \thmref{thm:multiclass_ERM}.  

\subsection{Justifying~\asmpref{asmp:deep_models}} \label{appsubsec:justifying_assumption1}

Motivated by the analysis on linear models, we now discuss alternate (and weaker) conditions that imply \asmpref{asmp:deep_models}. 
We need hypothesis stability (\codref{cond:hypothesis_stability}) and the following assumption relating training error and leave-one-error: 

\begin{assumption} \label{asmp:loo_error}
Let $\wh f$ be a model obtained by training with algorithm $\calA$ on a mixture of clean $S$ and randomly labeled data $\wt S$. Then we assume we have 
\begin{align*}
    \error_{\wt \calS_M} (\wh f) \le  \error_{\text{LOO} (\wt S_M)} \,, 
\end{align*}
for all $(x_i, y_i) \in  \wt S_M$ where $\wh f_{(i)} \defeq f(\calA, S \cup {{}\wt S_M}_{(i)})$ and  $\error_{\text{LOO} (\wt S_M)} \defeq  \frac{\sum_{(x_i, y_i) \in \wt S_M} \error(\ff{i}(x_i), y_i ) }{\abs{\wt \calS_M}}$.  
\end{assumption}

% we assume this to extend our result (parallel to \thmref{thm:multi_linear}) for deep models. 
Intuitively, this assumption states that the error on a (mislabeled) datum $(x,y)$ included in the training set is less than the error on that datum $(x,y)$ obtained by a model trained on the training set $S - \{(x,y)\}$. We proved this for linear models trained with GD in the proof of \thmref{thm:multi_linear}. 
% 
\codref{cond:hypothesis_stability} with $\beta = \calO(1)$ and \asmpref{asmp:loo_error} together with \lemref{lem:stability_error} implies \asmpref{asmp:deep_models} with a polynomial residual term (instead of logarithmic in $1/\delta$): 
\begin{align}
     \error_{\calS_M} (\wh f) \le  \error_{\calDm}(\wh f)   + \sqrt{\frac{1}{\delta}\left(\frac{1}{m} +\frac{3\beta}{m+n} \right)} \,.
\end{align}
% Note that this  

\newpage 
\section{Additional experiments and details}\label{app:exp}
\newcommand\tab[1][1cm]{\hspace*{#1}}

\subsection{Datasets} \label{sec:app_dataset}

\textbf{Toy Dataset {} {}} Assume fixed constants $\mu$ and $\sigma$. For a given label $y$, we simulate features $x$ in our toy classification setup as follows: 
\begin{align*}
    x \defeq \texttt{concat} \left[ x_1, x_2\right] \quad \text{where} \quad  x_1 \sim  \calN( y \cdot \mu, \sigma^2 I_{d \times d}) \ \  \text{and} \ \  x_1 \sim  \calN( 0, \sigma^2 I_{d \times d}) \,.
\end{align*}  
% where $y$ is the true label and $x$ is the corresponding feature vector. 
In experiements throughout the paper, we fix dimention $d=100$, $\mu = 1.0 $, and $\sigma = \sqrt{d}$. Intuitively, $x_1$ carries the information about the underlying label and $x_2$ is additional noise independent of the underlying label. 

\textbf{CV datasets {} {}} We use MNIST~\citep{lecun1998mnist} and CIFAR10~\cite{krizhevsky2009learning}. 
% For binary tasks, 
We produce a binary variant from the multiclass classification problem by mapping classes $\{0,1,2,3,4\}$ to label $1$ and $\{ 5,6,7,8,9\}$ to label $-1$. For CIFAR dataset, we also use the standard data augementation of random crop and horizontal flip. PyTorch code is as follows: 

\texttt{(transforms.RandomCrop(32, padding=4),\\
\tab transforms.RandomHorizontalFlip())}

\textbf{NLP dataset {} {}} We use IMDb Sentiment analysis~\citep{maas2011learning} corpus.  

\subsection{Architecture Details} 

All experiments were run on NVIDIA GeForce RTX 2080 Ti GPUs. We used PyTorch~\citep{NEURIPS2019a9015} and Keras with Tensorflow~\citep{abadi2016tensorflow} backend for experiments. 
% , ELMo embeddings~\citep{Peters:2018}, and Hugging Face Transformers~\citep{wolf-etal-2020-transformers}. 

\textbf{Linear model {} {}} For the toy dataset, we simulate a linear model with scalar output and the same number of parameters as the number of dimensions.   

\textbf{Wide nets {} {}} To simulate the NTK regime, we experiment with $2-$layered wide nets. The PyTorch code for 2-layer wide MLP is as follows: 


\texttt{ nn.Sequential( \\
\tab     nn.Flatten(),\\
\tab    nn.Linear(input\_dims, 200000, bias=True),\\
\tab    nn.ReLU(),\\
\tab    nn.Linear(200000, 1, bias=True)\\
\tab     )}


We experiment both (i) with the second layer fixed at random initialization; (ii)  and updating both layers' weights.     

\textbf{Deep nets for CV tasks {} {}} We consider a 4-layered MLP. The PyTorch code for 4-layer MLP is as follows: 

\texttt{ nn.Sequential(nn.Flatten(), \\
\tab        nn.Linear(input\_dim, 5000, bias=True),\\
\tab        nn.ReLU(),\\
\tab        nn.Linear(5000, 5000, bias=True),\\
\tab        nn.ReLU(),\\
\tab        nn.Linear(5000, 5000, bias=True),\\
\tab        nn.ReLU(),\\
% \tab        nn.Linear(5000, 5000, bias=True),\\
% \tab        nn.ReLU(),\\
\tab        nn.Linear(1024, num\_label, bias=True)\\
\tab        )}

For MNIST, we use $1000$ nodes instead of $5000$ nodes in the hidden layer. 
% 
We also experiment with convolutional nets. In particular, we use ResNet18 \citep{he2016deep}. Implementation adapted from:  \url{https://github.com/kuangliu/pytorch-cifar.git}. 

\textbf{Deep nets for NLP {} {}} We use a simple LSTM model with embeddings intialized with ELMo embeddings~\citep{Peters:2018}. Code adapted from: \url{https://github.com/kamujun/elmo_experiments/blob/master/elmo_experiment/notebooks/elmo_text_classification_on_imdb.ipynb} 

We also evaluate our bounds with a BERT model. In particular, we fine-tune an off-the-shelf uncased BERT model~\citep{devlin2018bert}. Code adapted from Hugging Face Transformers~\citep{wolf-etal-2020-transformers}: \url{https://huggingface.co/transformers/v3.1.0/custom_datasets.html}. 


\subsection{Additonal experiments}

\textbf{Results with SGD on underparameterized linear models {} {}} 

\begin{figure*}[h]
    \centering 
    % \vspace{-15pt}
    % \includegraphics[width=0.9\linewidth]{example-image-a}
    \includegraphics[width=0.3\linewidth]{figures/lowdim-Gaussian-SGD.pdf}
    % \includegraphics[width=0.9\linewidth]{figures/{CIFAR10_rn=0.1_lr=0.2_wd=0.005}.png}
    \vspace{-5pt}
    \caption{ 
    % Predicted lower bound 
    % on different
    We plot the accuracy and corresponding bound 
    (RHS in \eqref{eq:erm}) at $\delta = 0.1$
    for toy binary classification task. 
    Results aggregated over $3$ seeds. 
    % i.e., $1-\error$ where $\error$ is the term in the RHS of \eqref{eq:erm}
    Accuracy vs fraction of unlabeled data (w.r.t clean data) 
    in the toy setup with a linear model trained with SGD. Results parallel to \figref{fig:error_binary}(a) with SGD.  }
    \label{fig:error_binary_linear}
    \vspace{-5pt}
\end{figure*}

\textbf{Results with wide nets on binary MNIST {} {}}

\begin{figure*}[h]
    \centering 
    % \vspace{-15pt}
    % \includegraphics[width=0.9\linewidth]{example-image-a}
    \subfigure[GD with MSE loss]{\includegraphics[width=0.3\linewidth]{figures/MNIST-GD_MSE.pdf}} \hfil
    \subfigure[SGD with CE loss]{\includegraphics[width=0.3\linewidth]{figures/MNIST-SGD_CE.pdf}}
    \subfigure[SGD with MSE loss]{\includegraphics[width=0.3\linewidth]{figures/MNIST-SGD_MSE-first-layer.pdf}}
    % \includegraphics[width=0.9\linewidth]{figures/{CIFAR10_rn=0.1_lr=0.2_wd=0.005}.png}
    \vspace{-5pt}
    \caption{ 
    % Predicted lower bound 
    % on different
    We plot the accuracy and corresponding bound 
    (RHS in \eqref{eq:erm}) at $\delta = 0.1$ 
    for binary MNIST classification. 
    Results aggregated over $3$ seeds. 
    % i.e., $1-\error$ where $\error$ is the term in the RHS of \eqref{eq:erm}
    Accuracy vs fraction of unlabeled data 
    for a 2-layer wide network on binary MNIST with both the layers training in (a,b) and only first layer training in (c). 
    Results parallel to \figref{fig:error_binary}(b) .  }
    \label{fig:error_binary_MNIST}
    \vspace{-5pt}
\end{figure*}

% \begin{figure*}[h]
%     \centering 
%     % \vspace{-15pt}
%     % \includegraphics[width=0.9\linewidth]{example-image-a}
%     \subfigure[GD with MSE loss]{\includegraphics[width=0.3\linewidth]{figures/MNIST.pdf}} \hfil
    
%     \subfigure[SGD with CE loss]{\includegraphics[width=0.3\linewidth]{figures/MNIST.pdf}}
%     % \includegraphics[width=0.9\linewidth]{figures/{CIFAR10_rn=0.1_lr=0.2_wd=0.005}.png}
%     \vspace{-5pt}
%     \caption{ 
%     % Predicted lower bound 
%     % on different
%     We plot the accuracy and corresponding bound 
%     (RHS in \eqref{eq:erm}) at $\delta = 0.1$
%     for binary MNIST classification. 
%     Results aggregated over $3$ seeds. 
%     % i.e., $1-\error$ where $\error$ is the term in the RHS of \eqref{eq:erm}
%     Accuracy vs fraction of unlabeled data 
%     for a 2-layer wide network on binary MNIST with just the first layer training. 
%     Results parallel to \figref{fig:error_binary}(b) with only the first layer training.  }
%     \label{fig:error_binary_MNIST}
%     \vspace{-5pt}
% \end{figure*}

\textbf{Results on CIFAR 10 and MNIST {} {}} 
% 
We plot epoch wise error curve for results in \tabref{table:multiclass}(\figref{fig:error_epoch_CIFAR10} and \figref{fig:error_epoch_MNIST}). We observe the same trend as in \figref{fig:error_CIFAR10}. Additionally, we plot an \emph{oracle bound} obtained by tracking the error on mislabeled data which nevertheless were predicted as true label. To obtain an exact emprical value of the oracle bound, we need underlying true labels for the randomly labeled data. 
% Note that our bound in \thmref{thm:multiclass_ERM}, lower bounds the accuracy as predicted by the oracle bound. 
While with just access to extra unlabeled data we cannot calculate oracle bound, we note that the oracle bound is very tight and never violated in practice underscoring an importamt aspect of generalization in multiclass problems. This highlight that even a stronger conjecture may hold in multiclass classification, i.e., error on mislabeled data (where nevertheless true label was predicted) lower bounds the population error on the distribution of mislabeled data and hence, the error on (a specific) mislabeled portion predicts the population accuracy on clean data. 
% 
On the other hand, the dominating term of in \thmref{thm:multiclass_ERM} is loose when compared with the oracle bound. The main reason, we believe is the pessimistic upper bound in \eqref{eq:lemma1_final_multi_prev} in the proof of \lemref{lem:fit_mislabeled_multi}. We leave an investigation on this gap for future. 
% of fit 

% However, oracle bound highlights two . One,  



\begin{figure}[h]
    \centering 
    % \vspace{-15pt}
    % \includegraphics[width=0.9\linewidth]{example-image-a}
    \subfigure[MLP]{\includegraphics[width=0.3\linewidth]{figures/CIFAR10-FNN.pdf}} \hfil
    \subfigure[ResNet]{\includegraphics[width=0.3\linewidth]{figures/CIFAR10-Resnet.pdf}}
    % \includegraphics[width=0.9\linewidth]{figures/{CIFAR10_rn=0.1_lr=0.2_wd=0.005}.png}
    % \vspace{-10pt}
    \caption{ Per epoch curves for CIFAR10 corresponding results in \tabref{table:multiclass}. As before, we just plot the dominating term in the RHS of \eqref{eq:multiclass_ERM} as predicted bound. Additionally, we also plot the predicted lower bound by the error on mislabeled data which nevertheless were predicted as true label. We refer to this as ``Oracle bound''. See text for more details. 
    % 
    % except for the stopping point. 
    % The bound predicted by RATT (RHS in \eqref{eq:multiclass_ERM}) is vacuous. 
    }\label{fig:error_epoch_CIFAR10}
    % \vspace{-15pt}
\end{figure}


\begin{figure}[h]
    \centering 
    % \vspace{-15pt}
    % \includegraphics[width=0.9\linewidth]{example-image-a}
    \subfigure[MLP]{\includegraphics[width=0.3\linewidth]{figures/MNIST-FNN.pdf}} \hfil
    \subfigure[ResNet]{\includegraphics[width=0.3\linewidth]{figures/MNIST-Resnet.pdf}}
    % \includegraphics[width=0.9\linewidth]{figures/{CIFAR10_rn=0.1_lr=0.2_wd=0.005}.png}
    % \vspace{-10pt}
    \caption{ Per epoch curves for MNIST corresponding results in \tabref{table:multiclass}. As before, we just plot the dominating term in the RHS of \eqref{eq:multiclass_ERM} as predicted bound. Additionally, we also plot the predicted lower bound by the error on mislabeled data which nevertheless were predicted as true label. We refer to this as ``Oracle bound''. See text for more details. 
    % 
    % except for the stopping point. 
    % The bound predicted by RATT (RHS in \eqref{eq:multiclass_ERM}) is vacuous. 
    }\label{fig:error_epoch_MNIST}
    % \vspace{-15pt}
\end{figure}

\textbf{Results on CIFAR 100 {} {}} 
% 
On CIFAR100, our bound in \eqref{eq:multiclass_ERM} yields vacous bounds. However, the oracle bound as explained above yields tight guarantees in the initial phase of the learning (i.e., when learning rate is less than $0.1$) (\figref{fig:error_CIFAR100}).  

\begin{figure}[h]
    \centering 
    % \vspace{-15pt}
    % \includegraphics[width=0.9\linewidth]{example-image-a}
    \includegraphics[width=0.3\linewidth]{figures/CIFAR100-Resnet.pdf}
    % \includegraphics[width=0.9\linewidth]{figures/{CIFAR10_rn=0.1_lr=0.2_wd=0.005}.png}
    % \vspace{-10pt}
    \caption{ Predicted lower bound by the error on mislabeled data which nevertheless were predicted as true label with ResNet18 on CIFAR100. We refer to this as ``Oracle bound''. See text for more details. 
    % 
    % except for the stopping point. 
    The bound predicted by RATT (RHS in \eqref{eq:multiclass_ERM}) is vacuous. 
    }\label{fig:error_CIFAR100}
    % \vspace{-15pt}
\end{figure}


% \paragraph{Experiments on CIFAR100} 


% \subsection{Model Selection using RATT}


\subsection{Hyperparameter Details}


\textbf{\figref{fig:error_CIFAR10} {} {}} We use clean training dataset of size $40,000$. We fix the amount of unlabeled data at $20\%$ of the clean size, i.e. we include additional $8,000$ points with randomly assigned labels. We use test set of $10,000$ points. For both MLP and ResNet, we use SGD with an initial learning rate of $0.1$ and momentum $0.9$. We fix the weight decay parameter at $5\times 10^{-4}$. After $100$ epochs, we decay the learning rate to $0.01$. We use SGD batch size of $100$. 

\textbf{\figref{fig:error_binary} (a) {} {}} We obtain a toy dataset according to the process described in \secref{sec:app_dataset}. We fix $d=100$ and create a dataset of $50,000$ points with balanced classes. Moreover, we sample additional covariates with the same procedure to create randomly labeled dataset. For both SGD and GD training, we use a fixed learning rate $0.1$.    

\textbf{\figref{fig:error_binary} (b) {} {}} Similar to binary CIFAR, we use clean training dataset of size $40,000$ and fix the amount of unlabeled data at $20\%$ of the clean dataset size. To train wide nets, we use a fixed learning of $0.001$ with GD and SGD. We decide the weight decay parameter and the early stopping point that maximizes our generalization bound (i.e. without peeking at unseen data ).  We use SGD batch size of $100$. 

\textbf{\figref{fig:error_binary} (c) {} {}} With IMDb dataset, we use a clean dataset of size $20,000$ and as before, fix the amount of unlabeled data at $20\%$ of the clean data. To train ELMo model, we use Adam optimizer with a fixed learning rate $0.01$ and weight decay $10^{-6}$ to minimize cross entropy loss. We train with batch size $32$ for 3 epochs. To fine-tune BERT model, we use Adam optimizer with learning rate $5\times 10^{-5}$ to minimize cross entropy loss. We train with a batch size of $16$ for 1 epoch.    

\textbf{\tabref{table:multiclass} {} {}} For multiclass datasets, we train both MLP and ResNet with the same hyperparameters as described before. We sample a clean training dataset of size $40,000$ and fix the amount of unlabeled data at $20\%$ of the clean size. We use SGD with an initial learning rate of $0.1$ and momentum $0.9$. We fix the weight decay parameter at $5\times 10^{-4}$. After $30$ epochs for ResNet and after $50$ epochs for MLP, we decay the learning rate to $0.01$.  We use SGD with batch size $100$. 
For \figref{fig:error_CIFAR100}, we use the same hyperparameters as 
CIFAR10 training, except we now decay learning rate after $100$ epochs. 


In all experiments, to identify the best possible accuracy on just the clean data, we use the exact same set of hyperparamters except the stopping point. We choose a stopping point that maximizes test performance. 

\subsection{Summary of experiments }

\begin{center}
    \begin{table}[H] 
        \centering
        \begin{tabular}{|c|c|c|c|} 
        \hline
        Classification type & Model category & Model & Dataset  \\ [0.5ex] 
        \hline
        \hline
        \multirow{10}{*}{Binary} & Low dimensional & Linear model & Toy Gaussain dataset  \\
                        \cline{2-4}
                         & Overparameterized 
                        %  & Linear model & Toy Gaussain dataset \\
                        %  \cline{3-4}
                        %  & & 2-layer wide net& Toy Gaussain dataset \\
                        %  \cline{3-4}
                         & \multirow{2}{*}{2-layer wide net} & \multirow{2}{*}{Binary MNIST} \\
                         & linear nets & &  
                         \\
                         \cline{2-4}                 
                         & \multirow{6}{*}{Deep nets} & \multirow{2}{*}{MLP} & Binary MNIST \\
                         \cline{4-4}
                         & &  & Binary CIFAR \\
                         \cline{3-4}
                         &  & \multirow{2}{*}{ResNet} & Binary MNIST \\
                         \cline{4-4}
                         & &  & Binary CIFAR \\
                         \cline{3-4}
                         &  & ELMo-LSTM model & IMDb Sentiment Analysis \\
                         \cline{3-4}
                         & & BERT pre-trained model & IMDb Sentiment Analysis \\
        \hline
        \multirow{5}{*}{Multiclass} & \multirow{5}{*}{Deep nets} & \multirow{2}{*}{MLP} & MNIST \\
                        \cline{4-4} 
                        & & & CIFAR10 \\                   
                        \cline{3-4}
                         &   & \multirow{3}{*}{ResNet} & MNIST \\
                         \cline{4-4}
                         &   & & CIFAR10 \\
                         \cline{4-4}
                         &   & & CIFAR100 \\
        \hline
        \end{tabular}
        % \caption{Summary of experiments performed} \label{table:experiments}
    \end{table}    
    % \footnotetext[6]{We use both MSE loss and cross-entropy loss.}
    % \footnotetext[6]{We try 2 variants: one with a fixed first layer and the other with both layers trainable.}
\end{center}

\newpage
\section{Proof of \lemref{lem:stability_error}} \label{app:proof_lem_error}

\begin{proof}[Proof of \lemref{lem:stability_error}]
    Recall, we have a training set $S \cup \wt S_C$. We defined leave-one-out error on mislabeled points as $$\error_{\text{LOO}(\wt S_M) } = \frac{\sum_{(x_i, y_i) \in \wt S_M} \error( f_{(i)}( x_i), y_i)}{ \abs{\wt S_M }} \,, $$
    where $f_{(i)} \defeq f(\calA, (S \cup \wt S)_{(i)})$. Define $S^\prime \defeq S \cup \wt S$. Assume $(x,y)$ and $(x^\prime,y^\prime)$ as i.i.d. samples from ${\calDm}$. 
    Using Lemma 25 in \citet{bousquet2002stability}, we have
    \begin{align*}
        \Expo{ \left( \error_{\calDm}(\wh f) -\error_{\text{LOO}(\wt S_M) } \right)^2 } \le & \Expt{ S^\prime, (x,y), (x^\prime,y^\prime) }{ \error(\wh f(x), y ) \error(\wh f(x^\prime), y^\prime )} - 2 \Expt{ S^\prime, (x,y) }{ \error(\wh f(x), y ) \error(f_{(i)}(x_i), y_i )} \\
        & + \frac{m_1-1}{m_1}\Expt{ S^\prime }{  \error(f_{(i)}(x_i), y_i )  \error(f_{(j)}(x_j), y_j )} + \frac{1}{m_1} \Expt{ S^\prime }{  \error(f_{(i)}(x_i), y_i ) } \,. \numberthis \label{eq:main_reln}
    \end{align*}
    We can rewrite the equation above as : 
    \begin{align*}
        \Expo{ \left( \error_{\calDm}(\wh f) -\error_{\text{LOO}(\wt S_M) } \right)^2 } \le &  \, \underbrace{\Expt{ S^\prime, (x,y), (x^\prime,y^\prime) }{ \error(\wh f(x), y ) \error(\wh f(x^\prime), y^\prime ) - \error(\wh f(x), y ) \error(f_{(i)}(x_i), y_i )}}_{\RN{1}} \\
        & + \underbrace{\Expt{ S^\prime }{  \error(f_{(i)}(x_i), y_i )  \error(f_{(j)}(x_j), y_j ) -  \error(\wh f(x), y ) \error(f_{(i)}(x_i), y_i )}}_{\RN{2}} \\ &+ \underbrace{\frac{1}{m_1} \Expt{ S^\prime }{  \error(f_{(i)}(x_i), y_i ) - \error(f_{(i)}(x_i), y_i )  \error(f_{(j)}(x_j), y_j ) }}_{\RN{3}} \,. \numberthis \label{eq:main_reln2}
    \end{align*}
    
    We will now bound term $\RN{3}$.  Using Cauchy-Schwarz's inequality, we have
    
    \begin{align}
        \Expt{ S^\prime }{  \error(f_{(i)}(x_i), y_i ) - \error(f_{(i)}(x_i), y_i )  \error(f_{(j)}(x_j), y_j ) }^2 &\le  \Expt{ S^\prime }{  \error(f_{(i)}(x_i), y_i ) }^2 \Expt{S^\prime}{1 -   \error(f_{(j)}(x_j), y_j ) }^2 \\
        &\le \frac{1}{4} \,.\label{eq:term1_lem12}
    \end{align}
    
    Note that since $(x_i,y_i)$, $(x_j ,y_j )$, $(x,y)$, and $(x^\prime, y^\prime)$ are all from same distribution $\calDm$, we directly incorporate the bounds on term $\RN{1}$ and $\RN{2}$ from the proof of Lemma 9 in \citet{bousquet2002stability}. Combining that with \eqref{eq:term1_lem12} and our definition of hypothesis stability in \codref{cond:hypothesis_stability}, we have the required claim. 
    
    
    % We now re-write term $\RN{1}$ as
    % \begin{align*}
    %         &\Expt{S^\prime, (x,y), (x^\prime,y^\prime) }{ \error(\wh f(x), y ) \error(\wh f(x^\prime), y^\prime ) - \error(\wh f(x), y ) \error(f_{(i)}(x_i), y_i )} \\ & \qquad = \Expt{ S^\prime, (x,y), (x^\prime,y^\prime) }{ \error(\wh f(x), y ) \error(\wh f  (x^\prime), y^\prime ) - \error(\wh f ^\prime(x), y ) \error(f_{(i)}(x^\prime), y^\prime )} \tag{Exchanging $(x_i, y_i)$ with $(x^\prime, y^\prime)$ in the second term} \\
    %         & \qquad = \Expt{ S^\prime, (x,y), (x^\prime,y^\prime) }{  \left(\error(\wh f(x), y )-  \error(f_{(i)}(x), y ) \right) \error(\wh f  (x^\prime), y^\prime )  } \\
    %         & \qquad  + \Expt{ S^\prime, (x,y), (x^\prime,y^\prime) }{  \left(\error(f_{(i)}(x), y ) -\error(\wh f ^\prime(x), y ) \right) \error(\wh f  (x^\prime), y^\prime )}  \\
    %         & \qquad +\Expt{ S^\prime, (x,y), (x^\prime,y^\prime) }{  \left( \error(\wh f  (x^\prime), y^\prime ) -  \error(f_{(i)}(x^\prime), y^\prime ) \right) \error(\wh f ^\prime(x), y ) }  \,, \numberthis \label{eq:term1_final}
    % \end{align*}
    % where $\wh f^\prime$ is the classifier obtained by training on $ S^\prime_{(i)} \cup \{ (x^\prime, y^\prime) \} $. Similarly we can re-write term $\RN{2}$ as 
    % \begin{align*}
    %     & \Expt{ S^\prime }{  \error(f_{(i)}(x_i), y_i )  \error(f_{(j)}(x_j), y_j ) -  \error(\wh f(x), y ) \error(f_{(i)}(x_i), y_i )} \\
    %     &\quad  = \Expt{ S^\prime, (x,y), (x^\prime,y^\prime)}{  \error(f^{\prime\prime}_{(i)}(x), y )  \error(f_{(j)}^{\prime}(x^\prime), y^\prime ) -  \error(\wh f(x), y ) \error(f_{(i)}(x_i), y_i )} \tag{Exchanging $(x_i, y_i)$ with $(x, y)$ and $(x_j, y_j)$ with $(x^\prime, y^\prime)$ in the first term}\\
    %     &\quad = \Expt{ S^\prime, (x,y), (x^\prime,y^\prime)}{  \error(f^{\prime\prime}_{(j)}(x), y )  \error(f_{(i)}^{\prime}(x^\prime), y^\prime ) -  \error(\wh f^\prime (x), y ) \error(f^\prime_{(j)}(x^\prime), y^\prime )} \tag{Exchanging $(x_i, y_i)$ and $(x_j, y_j)$ and then replacing $(x_j, y_j)$ with $(x^\prime, y^\prime)$ in the second term} \\
    %     & \quad = \Expt{ S^\prime, (x,y), (x^\prime,y^\prime) }{  \left( \error(f_{(i)}^{\prime}(x^\prime), y^\prime )   -  \error(\wh f^{\prime\prime}  (x^\prime), y^\prime ) \right)  \error(f^{\prime\prime}_{(j)}(x), y )   } \\
    %     & \quad  + \Expt{ S^\prime, (x,y), (x^\prime,y^\prime) }{  \left( \error(f^{\prime\prime}_{(j)}(x), y )  -\error(\wh f ^\prime(x), y ) \right) \error(\wh f^{\prime\prime}  (x^\prime), y^\prime )  }  \\
    %     & \quad+ \Expt{ S^\prime, (x,y), (x^\prime,y^\prime) }{  \left( \error(\wh f^{\prime\prime}  (x^\prime), y^\prime )  -  \error(f^\prime_{(j)}(x^\prime), y^\prime ) \right)  \error(\wh f^\prime (x), y ) }   \\
    %     & \quad = \Expt{ S^\prime, (x,y), (x^\prime,y^\prime) }{  \left( \error(f_{(i)}^{\prime}(x^\prime), y^\prime )   -  \error(\wh f (x^\prime), y^\prime ) \right)  \error(f_{(i)}(x_j), y_j )   } \\
    %     & \quad  + \Expt{ S^\prime, (x,y), (x^\prime,y^\prime) }{  \left( \error(f^{\prime\prime}_{(j)}(x), y )  -\error(\wh f (x), y ) \right) \error(\wh f^{\prime\prime}  (x_j), y_j )  }  \\
    %     & \quad+ \Expt{ S^\prime, (x,y), (x^\prime,y^\prime) }{  \left( \error(\wh f^{\prime\prime}  (x^\prime), y^\prime )  -  \error(f^\prime_{(j)}(x^\prime), y^\prime ) \right)  \error(\wh f^\prime (x^\prime), y^\prime ) }  \,, \numberthis \label{eq:term2_final}
    % \end{align*}
    % where $f^{\prime\prime}_{(j)}$ is trained on $S^\prime_{(j,i)} \cup {(x,y)}$, $f^{\prime}_{(i)}$ is trained on $S^\prime_{(j,i)} \cup {(x^\prime,y^\prime)}$, and $\wh f^{\prime\prime} $ is trained on $S^\prime_{(j)} \cup {(x,y)}$. Note in the last line we replaced $(x,y)$ by $(x_j, y_j)$ in the first term, replaced $(x^\prime,y^\prime)$ by $(x_j, y_j)$ in the second term and exchanged $(x_i,y_i)$ with $(x_j,y_j)$ and also $(x,y)$ and $(x^\prime, y^\prime)$
    
    
\end{proof}


% 
% 16th Century Version Control 
% 

% \onecolumn

% \section*{Supplementary Material}
% We will be using the following standard results
% on exponential concentration of random variables 
% all throughout the discussion:

% \begin{lemma}[Hoeffding's inequality for independent RVs~\citep{hoeffding1994probability}] Let $Z_1, Z_2, \ldots, Z_n$ be independent bounded random variables with $Z_i \in [a,b]$ for all $i$, then 
%     \begin{align*}
%         \prob\left( \frac{1}{n} \sum_{i=1}^n (Z_i - \Expo{Z_i}) \ge t \right) \le \exp{\left( -\frac{2nt^2}{(b-a)^2} \right) }
%     \end{align*} 
%     and 
%     \begin{align*}
%         \prob\left( \frac{1}{n} \sum_{i=1}^n (Z_i - \Expo{Z_i}) \le -t \right) \le \exp{\left( -\frac{2nt^2}{(b-a)^2} \right) }
%     \end{align*} 
%     for all $t \ge 0$. 
% \end{lemma}

% \begin{lemma}[Hoeffding's inequality for sampling with replacement~\citep{hoeffding1994probability}] \label{lem:hoeffding_sampling} Let $\calZ = (Z_1, Z_2, \ldots, Z_N)$ be a finite population of $N$ points with $Z_i \in [a.b]$ for all $i$. Let $X_1, X_2, \ldots X_n$ be a random sample drawn without replacement from $\calZ$. Then for all $t \ge 0$, we have 
%     \begin{align*}
%         \prob\left( \frac{1}{n} \sum_{i=1}^n (X_i - \mu ) \ge t \right) \le \exp{\left( -\frac{2nt^2}{(b-a)^2} \right) }
%     \end{align*} 
%     and 
%     \begin{align*}
%         \prob\left( \frac{1}{n} \sum_{i=1}^n (X_i - \mu ) \le -t \right) \le \exp{\left( -\frac{2nt^2}{(b-a)^2} \right) } \,,
%     \end{align*} 
%     where $\mu = \frac{1}{N} \sum_{i=1}^{N} Z_i$. 
% \end{lemma}

% We now discuss one condition that generalizes the exponential concentration to dependent random variables.
% \begin{condition}[Bounded difference inequality] \label{cond:BDC} Let $\calZ$ be some set and $\phi: \calZ^n \to \Real$. We say that $\phi$ satisfies the bounded difference assumption if 
% there exists $c_1, c_2, \ldots c_n \ge 0$ s.t. for all $i$, we have 
% \begin{align*}
%     \sup_{Z_1,Z_2, \ldots,Z_n, Z_i^\prime in \calZ^{n+1} } \abs{\phi (Z_1, \ldots, Z_i, \ldots, Z_n ) - \phi (Z_1, \ldots, Z_i^\prime, \ldots, Z_n ) } \le c_i \,.
% \end{align*} 
% \end{condition}

% \begin{lemma}[McDiarmid’s inequality~\citep{mcdiarmid1989}] \label{lem:McDiarmid} Let $Z_1, Z_2, \ldots, Z_n$ be independent random variables on set $\calZ$ and $\phi : \calZ^n \to \Real$ satisfy bounded difference assumption (\codref{cond:BDC}). Then for all $t>0$, we have 
%     \begin{align*}
%         \prob\left( \phi(Z_1, Z_2, \ldots, Z_n) - \Expo{\phi(Z_1, Z_2, \ldots, Z_n)} \ge t \right) \le \exp{\left( -\frac{2t^2}{\sum_{i=1}^n c_i^2} \right) } 
%     \end{align*} 
%     and 
%     \begin{align*}
%         \prob\left( \phi(Z_1, Z_2, \ldots, Z_n) - \Expo{\phi(Z_1, Z_2, \ldots, Z_n)} \le -t \right) \le \exp{\left( -\frac{2t^2}{\sum_{i=1}^n c_i^2} \right) } \,
%     \end{align*} 
% \end{lemma}


% \section{Proofs from \secref{sec:ERM_training}}\label{app:proof_erm}

% \textbf{Additional notation {} {}} Let $m_1$ be the number of mislabeled points ($\wt S_M$) and $m_2$ be the number of correctly labeled points ($\wt S_C$). Note $m_1 + m_2 = m$. 


% \subsection{Proof of \thmref{thm:error_ERM}}


% \begin{proof}[Proof of \lemref{lem:fit_mislabeled}] 
%     The main idea of our proof is to regard 
%     the clean portion of the data 
%     ($S \cup \wt S_C$) as fixed.   
%     Then, there exists a classifier $f^*$ 
%     that is optimal over draws 
%     of the mislabeled data $\wt S_M$. 
% % 
%     % 
%     Formally, 
%     \begin{align}
%     f^* \defeq \argmin_{f \in \calF} \error_{\widecheck {\calD}} (f) \,, \label{eq:modified_ERM}
%     \end{align}
%     where $$\widecheck \calD = \frac{n}{m+n} \calS + \frac{m_1}{m+n} \wt \calS_C  + \frac{m_2}{m+n}\calDm \,.$$ That is, $\widecheck \calD$ a combination of 
%     the \emph{empirical distribution} 
%     over correctly labeled data $S \cup \wt S_C$
%     % in $S\cup \wt S$ 
%     and the (population) distribution 
%     over mislabeled data $\calDm$.
%     Recall that 
%     \begin{align}
%     \wh f \defeq \argmin_{f \in \calF} \error_{\calS \cup \wt S} (f) \,. \label{eq:orig_ERM}
%     \end{align}
%     % 
%     % 
%     Since, $\widehat f$ minimizes 0-1 error 
%     on $S \cup \wt S$, using ERM optimality on \eqref{eq:orig_ERM},  
%     we have 
%     \begin{align}
%         \error_{\calS \cup \wt \calS}(\widehat f) \le \error_{
%             \calS \cup \wt \calS}(f^*) \,.    \label{eq:step1}
%     \end{align}
%     Moreover, since $f^*$ is independent of $\wt S_M$, using Hoeffding's bound,
%     % \footnote{For a fully rigorous argument,
%     % refer to the complete proof in App.~\ref{app:proof_erm}.} 
%     we have with probability at least $1-\delta$ that
%     \begin{align}
%       \error_{\wt \calS_M}(f^*) \le \error_{ \calDm}(f^*) +  \sqrt{\frac{\log(1/\delta)}{2 m_1}} \,. \label{eq:step2} 
%     \end{align}
%     %$ 
%     %for some constant $c_1\le 1/2$. 
%     Finally, since $f^*$ is the optimal classifier on $\widecheck \calD$, 
%     we have 
%     \begin{align}
%         \error_{\widecheck \calD}(f^*) \le \error_{\widecheck \calD}(\widehat f) \label{eq:step3}
%     \end{align}
%      Now to relate \eqref{eq:step1} and \eqref{eq:step3}, we can re-write the \eqref{eq:step2} as follows: 
%     \begin{align}
%         \error_{\calS \cup \wt\calS}(f^*) \le \error_{ \widecheck \calD}(f^*) +  \frac{m_1}{m+n}\sqrt{\frac{\log(1/\delta)}{2 m_1}} \,. \label{eq:step4} 
%     \end{align}
%     Now we combine equations \eqref{eq:step1}, \eqref{eq:step4}, and \eqref{eq:step3}, to get 
%     \begin{align}
%         \error_{\calS \cup \wt \calS}(\wh f) \le \error_{\widecheck \calD}(\wh f) +  \frac{m_1}{m+n}\sqrt{\frac{\log(1/\delta)}{2 m_1}} \,, 
%     \end{align}
%     which implies 
%     \begin{align}
%         \error_{ \wt \calS_M}(\wh f) \le \error_{\calDm}(\wh f) + \sqrt{\frac{\log(1/\delta)}{2 m_1}} \,. \label{eq:lemma1_final}
%     \end{align}
%     Since $\wt S$ is obtained by randomly labeling an unlabeled dataset, we assume $2m_1 \approx m$ \footnote{Formally, with probability at least $1-\delta$, we have  $(m - 2m_1)\le \sqrt{m\log(1/\delta)/2}$ }. Moreover, using $\error_{\calDm} = 1 - \error_{\calD}$ we obtain the desired result.   
%     % Combining the above steps and using the fact 
%     % that $\error_\calD = 1- \error_{\calDm} $, 
%     % we obtain the desired result.
% \end{proof}

% \begin{proof}[Proof of \lemref{lem:mislabeled_error}]
%     Recall $\error_{\wt S} (f) = \frac{m_1}{m} \error_{\wt S_M}(f) + \frac{m_2}{m} \error_{\wt S_C}(f)$. Hence, we have 
%     \begin{align}
%         2\error_{\wt S}(f) - \error_{\wt S_M}(f) - \error_{\wt S_C}(f) &= \left(\frac{2m_1}{m} \error_{\wt S_M}(f) - \error_{\wt S_M}(f)\right) + \left(\frac{2m_2}{m} \error_{\wt S_C}(f) - \error_{\wt S_C}(f)\right) \\ &= \left(\frac{2m_1}{m} - 1\right) \error_{\wt S_M}(f) + \left(\frac{2m_2}{m} - 1 \right)\error_{\wt S_C} (f) \,.
%     \end{align} 
%     Since the dataset is randomly labeled, with probability at least $1-\delta$, we have  $\left(\frac{2m_1}{m} - 1\right) \le \sqrt{\frac{\log(1/\delta)}{2m}}$. Similarly, we have with probability at least $1-\delta$, $\left(\frac{2m_2}{m} - 1\right) \le \sqrt{\frac{\log(1/\delta)}{2m}}$. Using union bound, we have with probability at least $1-\delta$
%     % \begin{align}
%     %     2\error_{\wt S} - \error_{\wt S_M}(f) - \error_{\wt S_C}(f) \le \sqrt{\frac{\log(2/\delta)}{2m}} \left(\error_{\wt S_M}(f) + \error_{\wt S_C}(f) \right) \le 2\sqrt{\frac{\log(2/\delta)}{2m}} \,. \label{eq:lemma2_final}
%     % \end{align}
%     \begin{align}
%         2\error_{\wt S} - \error_{\wt S_M}(f) - \error_{\wt S_C}(f) \le \sqrt{\frac{\log(2/\delta)}{2m}} \left(\error_{\wt S_M}(f) + \error_{\wt S_C}(f) \right) \,. \label{eq:lemma2_prefinal}
%     \end{align}
%     With re-arranging $\error_{\wt S_M}(f) + \error_{\wt S_C}(f)$ and using the inequality $ 1- a\le \frac{1}{1+a} $, we have  
%     \begin{align}
%         2\error_{\wt S} - \error_{\wt S_M}(f) - \error_{\wt S_C}(f) \le 2\error_{\wt \calS} \sqrt{\frac{\log(2/\delta)}{2m}}  \,. \label{eq:lemma2_final}
%     \end{align}

%     % We obtain the desired result by using 
% \end{proof}

% \begin{proof}[Proof of \lemref{lem:clear_error}]
% % Recall 0-1 error on each point  $(x,y) \in S \cup \wt S$ is given by $\I{ f(x)\ne y}$.
% In the set of correctly labeled points $S \cup \wt S_C$, we have $S$ as a random subset of $S \cup \wt S_C$. Hence, using Hoeffding's inequality for sampling without replacement (\lemref{lem:hoeffding_sampling}), we have with probability at least $1-\delta$
% \begin{align}
%     \error_{\wt \calS_c} (\wh f)- \error_{\calS \cup \wt \calS_C}( \wh f) \le  \sqrt{\frac{\log(1/\delta)}{2m_2}} \,.
% \end{align}
% Re-writing $\error_{\calS \cup \wt \calS_C}( \wh f)$ as $\frac{m_2}{m_2 + n} \error_{\wt \calS_C }(\wh f) + \frac{n}{m_2 + n} \error_{\calS }(\wh f)$, we have with probability at least $1-\delta$
% \begin{align}
%   \left(\frac{n}{n+m_2}\right) \left(\error_{\wt \calS_c} (\wh f)- \error_{\calS}( \wh f) \right) \le  \sqrt{\frac{\log(1/\delta)}{2m_2}} \,.
% \end{align}
% As before, assuming $2m_2 \approx m$, we have with probability at least $1-\delta$ 
% \begin{align}
%     \error_{\wt \calS_c} (\wh f)- \error_{\calS}( \wh f) \le \left(1+\frac{m_2}{n}\right)  \sqrt{\frac{\log(1/\delta)}{m}} \le 1.5 \sqrt{\frac{\log(1/\delta)}{m}} \,. \label{eq:lemma3_final}
% \end{align} 
% \end{proof}

% \begin{proof}[Proof of \thmref{thm:error_ERM}] 
%     Having established these core intermediate results, we can now combine above three lemmas to prove the main result. 
%     In particular, we bound the population error on clean data ($\error_\calD(\wh f)$) as follows:  
%     \begin{enumerate}[(i)]
%         \item First, use \eqref{eq:lemma1_final}, to obtain an upper bound on the population error on clean data, i.e., with probability at least $1-\delta/4$, we have
%         \begin{align}
%             \error_{ \calD} (\wh f) \le 1 - \error_{ \wt \calS_M}(\wh f) + \sqrt{\frac{\log(4/\delta)}{m}} \,. 
%         \end{align}
%         \item  Second, use \eqref{eq:lemma2_final}, to relate the error on the mislabeled fraction with error on clean portion of randomly labeled data and error on whole randomly labeled dataset, i.e., with probability at least $1-\delta/2$, we have 
%         \begin{align}
%             - \error_{\wt S_M}(f) \le \error_{\wt S_C}(f) - 2\error_{\wt S}  + \sqrt{\frac{\log(4/\delta)}{2m}}  \,. 
%         \end{align} 
%         \item Finally, use \eqref{eq:lemma3_final} to relate the error on the clean portion of randomly labeled data and error on clean training data, i.e., with probability $1-\delta/4$, we have 
%         \begin{align}
%             \error_{\wt \calS_C} (\wh f)\le - \error_{\calS}( \wh f) + \left(1 + \frac{m}{2n} \right) \sqrt{\frac{\log(4/\delta)}{m}} \,. 
%         \end{align} 
%     \end{enumerate}

%     Using union bound on the above three steps, we have with probability at least $1-\delta$: 
%     \begin{align}
%         \error_\calD (\wh f) \le \error_{\calS}(\wh f)   + 1 - 2\error_{\wt \calS}(\wh f)   + (1/\sqrt{2} + 2.5)  \sqrt{\frac{\log(4/\delta)}{m}} \,.
%     \end{align}
%     Note that $(1/\sqrt{2} + 2.5)$ is a loose constant. In experiments, we use the ratio $\frac{m}{n}$
%     %  the exact error $\error_{\wt \calS}(\wh f)$ 
%     to evaluate R.H.S.    
% \end{proof}

% \subsection{Proof of \propref{prop:rademacher}}

% \begin{proof}[Proof of \propref{prop:rademacher}]
%     For a classifier $ f: \calX \to \{-1, 1\}$, we have $1 - 2\,\indict{ f(x) \ne y} = y \cdot f(x)$. Hence, by definition of $\error$, we have 
%     \begin{align}
%         1 -2\error_{\wt \calS}(f) = \frac{1}{m}\sum_{i=1}^m y_i \cdot f(x_i) \le \sup_{f \in \calF} \, \frac{1}{m} \sum_{i=1}^m y_i \cdot f(x_i)  \,. \label{eq:error_rademacher}
%     \end{align}
%     Note that for fixed inputs $(x_1, x_2, \ldots, x_m)$ in $\wt S$, $(y_1, y_2, \ldots y_m)$ are random labels. Define $\phi_1 (y_1, y_2, \ldots, y_m) \defeq \sup_{f \in \calF} \, \frac{1}{m} \sum_{i=1}^m y_i \cdot f(x_i)$. We have the following bounded difference condition on $\phi_1$. For all i, 
%     \begin{align}
%         \sup_{y_1, \ldots y_m, y_i^\prime \in \{-1, 1\}^{m+1} } \abs{ \phi_1 (y_1,\ldots, y_i, \ldots, y_m) - \phi_1 (y_1,\ldots, y_i^\prime, \ldots, y_m)  } \le 1/m \,. \label{cond1_rademacher}
%     \end{align} 
    
%     Similarly define $\phi_2 (x_1, x_2, \ldots, x_m) \defeq \Expt{ y_i \sim_U \{-1, 1\}  }{ \sup_{f \in \calF} \, \frac{1}{m}  \sum_{i=1}^m y_i \cdot f(x_i)}$. We have the following bounded difference condition on $\phi_2$. For all i,
%     \begin{align}
%         \sup_{x_1, \ldots x_m, x_i^\prime \in \calX^{m+1} } \abs{ \phi_2 (x_1,\ldots, x_i, \ldots, x_m) - \phi_1 (x_1,\ldots, x_i^\prime, \ldots, x_m)  } \le 1/m \,. \label{cond2_rademacher}
%     \end{align}
%     Using McDiarmid’s inequality (\lemref{lem:McDiarmid}) twice with Condition \eqref{cond1_rademacher} and \eqref{cond2_rademacher}, with probability at least $1-\delta$, we have
%     \begin{align}
%         \sup_{f \in \calF} \, \frac{1}{m} \sum_{i=1}^m y_i \cdot f(x_i)  - \Expt{x,y}{\sup_{f \in \calF} \, \frac{1}{m} \sum_{i=1}^m y_i \cdot f(x_i) } \le \sqrt{\frac{2\log(2/\delta)}{m}} \label{eq:final_rademacher}
%     \end{align} 
%     Combining \eqref{eq:error_rademacher} and \eqref{eq:final_rademacher}, we obtain the desired result. 
% \end{proof}


% \subsection{Proof of \thmref{thm:error_regularized_ERM}}

% Proof of \thmref{thm:error_regularized_ERM} follows similar to the proof of \thmref{thm:error_ERM}. Note that the same results in \lemref{lem:fit_mislabeled}, \lemref{lem:mislabeled_error}, and \lemref{lem:clear_error} hold in the regularized ERM case. However, the arguments in the proof of \lemref{lem:fit_mislabeled} changes slightly. Hence, we state and prove a lemma parallel to \lemref{lem:fit_mislabeled} for completeness. 

% \begin{lemma} \label{lem:lemma1_reg}
%     Assume the same setup as \thmref{thm:error_regularized_ERM}. 
%     Then for any $\delta >0$, with probability at least  $1-\delta$ 
%     over the random draws of mislabeled data $\wt S_M$, we have 
%     \begin{align}
%         \error_\calD(\widehat f)  \le 1 -\error_{\wt \calS_M}(\widehat f) + \sqrt{\frac{\log(1/\delta)}{m}}\,. 
%     \end{align} 
% \end{lemma}
% \begin{proof}
%     The main idea of the proof remains the same, i.e. regard 
%     the clean portion of the data 
%     ($S \cup \wt S_C$) as fixed.   
%     Then, there exists a classifier $f^*$ 
%     that is optimal over draws 
%     of the mislabeled data $\wt S_M$. 

    
%     Formally, 
%     \begin{align}
%     f^* \defeq \argmin_{f \in \calF} \error_{\widecheck {\calD}} (f)  + \lambda R(f) \,, \label{eq:modified_ERM_reg}
%     \end{align}
%     where $$\widecheck \calD = \frac{n}{m+n} \calS + \frac{m_1}{m+n} \wt \calS_C  + \frac{m_2}{m+n}\calDm \,.$$ That is, $\widecheck \calD$ a combination of 
%     the \emph{empirical distribution} 
%     over correctly labeled data $S \cup \wt S_C$
%     % in $S\cup \wt S$ 
%     and the (population) distribution 
%     over mislabeled data $\calDm$.
%     Recall that 
%     \begin{align}
%     \wh f \defeq \argmin_{f \in \calF} \error_{\calS \cup \wt S} (f) + \lambda R(f) \,. \label{eq:orig_ERM_reg}
%     \end{align}
%     % 
%     % 
%     Since, $\widehat f$ minimizes 0-1 error 
%     on $S \cup \wt S$, using ERM optimality on \eqref{eq:orig_ERM},  
%     we have 
%     \begin{align}
%         \error_{\calS \cup \wt \calS}(\widehat f) + \lambda R(\wh f) \le \error_{
%             \calS \cup \wt \calS}(f^*) + \lambda R(f^*) \,.    \label{eq:step1_reg}
%     \end{align}
%     Moreover, since $f^*$ is independent of $\wt S_M$, using Hoeffding's bound,
%     % \footnote{For a fully rigorous argument,
%     % refer to the complete proof in App.~\ref{app:proof_erm}.} 
%     we have with probability at least $1-\delta$ that
%     \begin{align}
%       \error_{\wt \calS_M}(f^*) \le \error_{ \calDm}(f^*) +  \sqrt{\frac{\log(1/\delta)}{2 m_1}} \,. \label{eq:step2_reg} 
%     \end{align}
%     %$ 
%     %for some constant $c_1\le 1/2$. 
%     Finally, since $f^*$ is the optimal classifier on $\widecheck \calD$, 
%     we have 
%     \begin{align}
%         \error_{\widecheck \calD}(f^*) + \lambda R(f^*) \le \error_{\widecheck \calD}(\widehat f) + \lambda R(\wh f) \label{eq:step3_reg}
%     \end{align}
%      Now to relate \eqref{eq:step1_reg} and \eqref{eq:step3_reg}, we can re-write the \eqref{eq:step2_reg} as follows: 
%     \begin{align}
%         \error_{\calS \cup \wt\calS}(f^*) \le \error_{ \widecheck \calD}(f^*) +  \frac{m_1}{m+n}\sqrt{\frac{\log(1/\delta)}{2 m_1}} \,. \label{eq:step4_reg} 
%     \end{align}
%     After adding $\lambda R(f^*)$ on both sides in \eqref{eq:step4_reg}, we combine equations \eqref{eq:step1_reg}, \eqref{eq:step4_reg}, and \eqref{eq:step3_reg}, to get 
%     \begin{align}
%         \error_{\calS \cup \wt \calS}(\wh f) \le \error_{\widecheck \calD}(\wh f) +  \frac{m_1}{m+n}\sqrt{\frac{\log(1/\delta)}{2 m_1}} \,, 
%     \end{align}
%     which implies 
%     \begin{align}
%         \error_{ \wt \calS_M}(\wh f) \le \error_{\calDm}(\wh f) + \sqrt{\frac{\log(1/\delta)}{2 m_1}} \,. \label{eq:lemma_reg_final}
%     \end{align}
%     Similar as before, since $\wt S$ is obtained by randomly labeling an unlabeled dataset, we assume 
%     $2m_1 \approx m$. Moreover, using $\error_{\calDm} = 1 - \error_{\calD}$ we obtain the desired result. 
% \end{proof}
% % \begin{proof}[Proof of ]
    
% % \end{proof}

% \subsection{Proof of \thmref{thm:multiclass_ERM}}

% We first state and prove lemmas parallel to three lemmas used in the proof of balanced binary case. Then we combine the results in the three lemmas to obtain the result in \thmref{thm:multiclass_ERM}. 

% Before stating the result, we define mislabeled distribution $\calDm$ for any $\calD$. While $\calDm$ and $\calD$ share 
% the same marginal distribution over $\calX$, 
% the distribution over labels $y$ 
% given an input $x\sim \calD_\calX$ is changed.
% In particular, for any $x$, the pdf over $y$ is changed to:  
% $p_{\calDm} (\cdot \vert x) \defeq \frac{1 - p_{\calD}(\cdot \vert x)}{k - 1}$.

% \begin{lemma} \label{lem:fit_mislabeled_multi}
%     Assume the same setup as \thmref{thm:multiclass_ERM}. 
%     Then for any $\delta >0$, with probability at least  $1-\delta$ 
%     over the random draws of mislabeled data $\wt S_M$, we have 
%     \begin{align}
%         \error_\calD(\widehat f)  \le (k-1)\left(1 -\error_{\wt \calS_M}(\widehat f)\right) + (k-1)\sqrt{\frac{\log(1/\delta)}{m}}\,. \label{eq:lemma1_multi}
%     \end{align}   
% \end{lemma} 

% \begin{proof}
%     The main idea of the proof remains the same, i.e. regard 
%     the clean portion of the data 
%     ($S \cup \wt S_C$) as fixed. 
%     Then, there exists a classifier $f^*$ 
%     that is optimal over draws 
%     of the mislabeled data $\wt S_M$. 
    
%     However, we need to be careful while relating population error on mislabeled data with population accuracy on clean data.   
%     While for binary classification,  we could upper bound $\error_{\wt \calS_M}$ 
%     with $1-\error_\calD$  (in the proof of \lemref{lem:fit_mislabeled}), 
%     for multiclass classification, 
%     error on the mislabeled data 
%     and accuracy on the clean data 
%     in the population 
%     are not so directly related.  
%     To establish \eqref{eq:lemma1_multi},
%     we break the error on the 
%     (unknown) mislabeled data 
%     into two parts: one term corresponds 
%     to predicting the true label on mislabeled data, 
%     and the other corresponds to predicting 
%     neither the true label 
%     nor the assigned (mis-)label.  
%     Finally, we relate these errors to their
%     population counterparts to establish \eqref{eq:lemma1_multi}. 
    
%     Formally, 
%     \begin{align}
%     f^* \defeq \argmin_{f \in \calF} \error_{\widecheck {\calD}} (f)  + \lambda R(f) \,, \label{eq:modified_ERM_reg2}
%     \end{align}
%     where $$\widecheck \calD = \frac{n}{m+n} \calS + \frac{m_1}{m+n} \wt \calS_C  + \frac{m_2}{m+n}\calDm \,.$$ That is, $\widecheck \calD$ a combination of 
%     the \emph{empirical distribution} 
%     over correctly labeled data $S \cup \wt S_C$
%     % in $S\cup \wt S$ 
%     and the (population) distribution 
%     over mislabeled data $\calDm$.
%     Recall that 
%     \begin{align}
%     \wh f \defeq \argmin_{f \in \calF} \error_{\calS \cup \wt S} (f) + \lambda R(f) \,. \label{eq:orig_ERM_reg2}
%     \end{align}
%     % 
%     % 
%     Following the exact steps from the proof of \lemref{lem:lemma1_reg}, with probability at least $1-\delta$, we have  
%     \begin{align}
%         \error_{ \wt \calS_M}(\wh f) \le \error_{\calDm}(\wh f) + \sqrt{\frac{\log(1/\delta)}{2 m_1}} \,. \label{eq:lemma1_final_multi_prev}
%     \end{align}
%     Similar to before, since $\wt S$ is obtained by randomly labeling an unlabeled dataset, we assume 
%     $\frac{k}{k-1} m_1 \approx m$. 
    
%     Now we will relate $\error_\calDm (\wh f)$ with $\error_{\calD}(\wh f)$. Let $y^T$ denote the (unknown) true label for a mislabeled point $(x, y)$ (i.e., label before replacing it with a mislabel). 
%     \begin{align}    
%          \Expt{(x, y) \in \sim \calDm}{\indict{ \wh f(x) \ne y }}  &= \underbrace{\Expt{(x, y) \in \sim \calDm}{\indict{ \wh f(x) \ne y \land \wh f(x) \ne y^T}}}_{\RN{1}} + \underbrace{\Expt{(x, y) \in \sim \calDm}{\indict{ \wh f(x) \ne y \land \wh f(x) = y^T}}}_{\RN{2}} \,. \label{eq:excess_term}
%     \end{align}
%     Clearly, term 2 is one minus the accuracy on the clean unseen data, i.e. 
%     \begin{align}
%         \RN{2} = 1 - \Expt{{x,y} \sim \calD}{ \indict{ \wh f(x) \ne y}} = 1- \error_{\calD}(\wh f) \,. \label{eq:term1}    
%     \end{align}
%     Next, we  relate term 1 with the error on the unseen clean data. We show that term 1 is equal to the error on the unseen clean data scaled by $\frac{k-2}{k-1}$ where $k$ is the number of labels. Using the definition of mislabeled distribution $\calDm$,  we have 
%     \begin{align}
%         \RN{1} = \frac{1}{k-1} \left( \Expt{(x, y) \in \sim \calD}{ \sum_{i \in \calY \land i\ne y}  \indict{ \wh f(x) \ne i \land \wh f(x) \ne y}} \right) = \frac{k-2}{k-1} \error_{\calD}(\wh f) \,.\label{eq:term2}
%     \end{align}    

%     Combining the result in \eqref{eq:term1}, \eqref{eq:term2} and \eqref{eq:excess_term}, we have 
%     \begin{align}
%         \error_{\calDm}(\wh f) = 1- \frac{1}{k-1} \error_{\calD}(\wh f) \,.\label{eq:combine_terms}
%     \end{align}
%     Finally, combining the result in \eqref{eq:combine_terms} with equation \eqref{eq:lemma1_final_multi_prev}, we have with probability $1-\delta$, 
%     \begin{align}
%       \error_{\calD}(\wh f) \le  (k-1) \left( 1- \error_{ \wt \calS_M}(\wh f) \right)  + (k-1) \sqrt{\frac{k \log(1/\delta)}{ 2(k-1)m}} \,. \label{eq:lemma1_final_multi}
%     \end{align}
% \end{proof}

% \begin{lemma} \label{lem:mislabeled_error_multi}
%     Assume the same setup as \thmref{thm:multiclass_ERM}.  Then for any $\delta >0$, with probability at least $1-\delta$ over the random draws of $\wt S$, we have  
%     % \begin{align}
%         $$\abs{k\error_{\wt \calS}(\widehat f) - \error_{\wt \calS_C}(\widehat f) -  (k-1)\error_{\wt \calS_M}(\widehat f) } \le  2k\sqrt{\frac{\log(4/\delta)}{2m}}\,. $$ % \label{eq:lemma2}
%     % \end{align}   
%     %  for some constant $c_3 \le 1.0\,$.
% \end{lemma} 


% \begin{proof}
%     Recall $\error_{\wt S} (f) = \frac{m_1}{m} \error_{\wt S_M}(f) + \frac{m_2}{m} \error_{\wt S_C}(f)$. Hence, we have 
%     \begin{align}
%         k\error_{\wt S}(f) - (k-1)\error_{\wt S_M}(f) - \error_{\wt S_C}(f) &= (k-1)\left(\frac{k m_1}{(k-1) m} \error_{\wt S_M}(f) - \error_{\wt S_M}(f)\right) + \left(\frac{km_2}{m} \error_{\wt S_C}(f) - \error_{\wt S_C}(f)\right) \\ &= k \left[ \left(\frac{m_1}{m} - \frac{k-1}{k}\right) \error_{\wt S_M}(f) + \left(\frac{m_2}{m} - \frac{1}{k} \right) \error_{\wt S_C} (f) \right] \,.
%     \end{align} 
%     Since the dataset is randomly labeled, we have with probability at least $1-\delta$, $\left(\frac{m_1}{m} - \frac{k-1}{k}\right) \le \sqrt{\frac{\log(1/\delta)}{2m}}$. Similarly, we have with probability at least $1-\delta$, $\left(\frac{m_2}{m} - \frac{1}{k}\right) \le \sqrt{\frac{\log(1/\delta)}{2m}}$. Using union bound, we have with probability at least $1-\delta$
%     % \begin{align}
%     %     2\error_{\wt S} - \error_{\wt S_M}(f) - \error_{\wt S_C}(f) \le \sqrt{\frac{\log(2/\delta)}{2m}} \left(\error_{\wt S_M}(f) + \error_{\wt S_C}(f) \right) \le 2\sqrt{\frac{\log(2/\delta)}{2m}} \,. \label{eq:lemma2_final}
%     % \end{align}
%     \begin{align}
%         k\error_{\wt S}(f) - (k-1)\error_{\wt S_M}(f) - \error_{\wt S_C}(f)  \le k \sqrt{\frac{\log(2/\delta)}{2m}} \left(\error_{\wt S_M}(f) + \error_{\wt S_C}(f) \right) \,. \label{eq:lemma2_final_multi}
%     \end{align}

%     % We obtain the desired result by using 
% \end{proof}

% \begin{lemma} \label{lem:clear_error_multi}
%     Assume the same setup as \thmref{thm:multiclass_ERM}. 
%     Then for any $\delta >0$, with probability at least $1-\delta$ 
%     over the random draws of $\wt S_C$ and $S$, we have 
%     % \begin{align}
%         $$\abs{\error_{\wt \calS_C}(\widehat f) - \error_{\calS}(\widehat f) } \le 1.5 \sqrt{\frac{k\log(2/\delta)}{2m}}\,.$$ %\label{eq:lemma3}
%     % \end{align}   
%     % for some constant $c_2 \le 1.2\,$.
% \end{lemma} 
% \begin{proof}
%     % Recall 0-1 error on each point  $(x,y) \in S \cup \wt S$ is given by $\I{ f(x)\ne y}$.
%     In the set of correctly labeled points $S \cup \wt S_C$, we have $S$ as a random subset of $S \cup \wt S_C$. Hence, using Hoeffding's inequality for sampling without replacement (\lemref{lem:hoeffding_sampling}), we have with probability at least $1-\delta$
%     \begin{align}
%         \error_{\wt \calS_c} (\wh f)- \error_{\calS \cup \wt \calS_C}( \wh f) \le  \sqrt{\frac{\log(1/\delta)}{2m_2}} \,.
%     \end{align}
%     Re-writing $\error_{\calS \cup \wt \calS_C}( \wh f)$ as $\frac{m_2}{m_2 + n} \error_{\wt \calS_C }(\wh f) + \frac{n}{m_2 + n} \error_{\calS }(\wh f)$, we have with probability at least $1-\delta$
%     \begin{align}
%       \left(\frac{n}{n+m_2}\right) \left(\error_{\wt \calS_c} (\wh f)- \error_{\calS}( \wh f) \right) \le  \sqrt{\frac{\log(1/\delta)}{2m_2}} \,.
%     \end{align}
%     As before, assuming $km_2 \approx m$, we have with probability at least $1-\delta$ 
%     \begin{align}
%         \error_{\wt \calS_c} (\wh f)- \error_{\calS}( \wh f) \le \left(1+\frac{m_2}{n}\right)  \sqrt{\frac{k\log(1/\delta)}{2m}} \le \left( 1 + \frac{1}{k}\right) \sqrt{\frac{k\log(1/\delta)}{2m}} \,. \label{eq:lemma3_final_multi}
%     \end{align} 
% \end{proof}

% \begin{proof}[Proof of \thmref{thm:multiclass_ERM}] 
%     Having established these core intermediate results, we can now combine above three lemmas. 
%     In particular, we bound the population error on clean data ($\error_\calD(\wh f)$) as follows:  
%     \begin{enumerate}[(i)]
%         \item First, use \eqref{eq:lemma1_final_multi}, to obtain an upper bound on the population error on clean data, i.e., with probability at least $1-\delta/4$, we have
%         \begin{align}
%             \error_{ \calD} (\wh f) \le (k-1)\left(1 - \error_{ \wt \calS_M}(\wh f) \right) + (k-1) \sqrt{\frac{k\log(4/\delta)}{2(k-1)m}} \,. 
%         \end{align}
%         \item  Second, use \eqref{eq:lemma2_final_multi}, to relate the error on the mislabeled fraction with error on clean portion of randomly labeled data and error on whole randomly labeled dataset, i.e., with probability at least $1-\delta/2$, we have 
%         \begin{align}
%             - (k-1)\error_{\wt S_M}(f) \le \error_{\wt S_C}(f) - k\error_{\wt S}  + k\sqrt{\frac{\log(4/\delta)}{2m}}  \,. 
%         \end{align} 
%         \item Finally, use \eqref{eq:lemma3_final_multi} to relate the error on the clean portion of randomly labeled data and error on clean training data, i.e., with probability $1-\delta/4$, we have 
%         \begin{align}
%             \error_{\wt \calS_C} (\wh f)\le - \error_{\calS}( \wh f) + \left(1 + \frac{m}{kn} \right) \sqrt{\frac{k\log(4/\delta)}{2m}} \,. 
%         \end{align} 
%     \end{enumerate}

%     Using union bound on the above three steps, we have with probability at least $1-\delta$: 
%     \begin{align}
%         \error_\calD (\wh f) \le \error_{\calS}(\wh f) + (k-1) - k\error_{\wt \calS}(\wh f)   + (\sqrt{k(k-1)} + k + \sqrt{k} + \frac{m}{n\sqrt{k}})  \sqrt{\frac{\log(4/\delta)}{2m}} \,.
%     \end{align}
%     % Note that $\frac{m}{n\sqrt{k}}$ is much smaller than the other terms in addition. Hence, we ignore this in the final bound. 
%     % Note that $(1/\sqrt{2} + 2.5)$ is a loose constant. In experiments, we use the ratio $\frac{m}{n}$
%     %  the exact error $\error_{\wt \calS}(\wh f)$ 
%     % to evaluate R.H.S.    
% \end{proof}

% \newpage
% \section{Proofs from \secref{sec:linear_models}}\label{app:proof_gd}

% We suppose that the parameters of the linear function 
% are obtained via gradient descent on 
% the following $L_2$ regularized problem: 
% \begin{align}
%     % n in denominator is avoided deliberately
%     \calL_S(w; \lambda) \defeq \sum_{i=1}^n{(w^Tx_i - y_i)^2} + \lambda \norm{w}{2}^2 \,, \label{eq:l2_MSE_app}   
% \end{align}
% where $\lambda\ge0$ is a regularization parameter. 
% We assume access to a clean dataset 
% $S = \{(x_i, y_i)\}_{i=1}^n \sim \calD^n$ 
% and randomly labeled dataset 
% $\wt S = \{(x_i, y_i)\}_{i=n+1}^{n+m} \sim \wt \calD^m$. 
% Let $\bX = [x_1, x_2, \cdots, x_{m+n}]$ 
% and $\by = [y_1, y_2, \cdots, y_{m+n}]$. 
% Fix a positive learning rate $\eta$ such that 
% $\eta \le 1/\left(\norm{\bX^T\bX}{\text{op}} + \lambda^2\right)$ 
% and an initialization $w_0 = 0$. 
% % \todos{Assumption made for simplicty}. 
% Consider the following gradient descent iterates 
% to minimize objective \eqref{eq:l2_MSE_app} on $S \cup \wt S$:
% \begin{align}
% w_t = w_{t-1} - \eta \grad_w \calL_{S \cup \wt S} (w_{t-1}; \lambda) \quad \forall t=1,2,\ldots \label{eq:GD_iterates_app}
% \end{align} 
% Then we have $\{ w_t\}$ converge to the limiting solution 
% $\wh w = \left( \bX^T\bX+\lambda \boldsymbol{I}\right)^{-1}\bX^T\by$. Define $\widehat f (x) \defeq f(x ; \wh w) $.  

% \subsection{\textcolor{red}{Errata}}

% We wish to correct the following error in the body: \codref{cond:error_stability} is not enough to guarantee the result in \thmref{thm:linear}. We now present a slightly stronger condition called \emph{hypothesis stability} under which we obtain a result similar to \thmref{thm:linear}. 

% This error doesn't change the main arguments of the proof where we show that the empirical train error is less than or equal to the leave-one-out error. We need a stronger condition to relate leave-one-out error with the population error of the original classifier. Specifically, while \codref{cond:error_stability} relates the average population error of leave-one-out classifiers with the population error of the original classifier, we need the new condition to show the concentration of the empirical leave-one-out error and  average population error of leave-one-out classifiers. 
% % main takeaway 

% Note that the new condition, while being stronger than the previous one, still doesn't imply generalization~\cite{bousquet2002stability,elisseeff2003leave,abou2019exponential}. Overall, the main results in \secref{sec:ERM_training} and takeaways of the paper remain unaffected by the error.  

% We now present the new condition and a corrected statement of \thmref{thm:linear}. Recall, for a given training set $S \sim \calD^n $, 
% we use $S_{(i)}$ to denote the training set $S$ 
% with the $i^{\text{th}}$ point removed.

% \begin{condition}[Hypothesis Stability] 
%     \label{cond:hypothesis_stability}
%     We have $\beta$ hypothesis stability 
%     if our training algorithm $\calA$ satisfies the following: 
%     \begin{align*}
%     % ${\sum_{i=1}^n \frac{\error_{\calD}( f(\calA, S_{(i)}))}{n} - \error_\calD(f(\calA, S))} \le \beta\,$.
%     \forall i \in \{1,2,\ldots, n\}, \quad  \Expt{\calS, (x,y) \in \calD}{ \abs{\error\left( f(x) ,y  \right) - \error\left( f_{(i)}(x), y \right) }} \le \frac{\beta}{n} \,,
%     \end{align*}
%     where $f_{(i)} \defeq f(\calA, S_{(i)})$ and $ f \defeq f(\calA, S)$.
% \end{condition}

% \begin{theorem}[Correct statement of \thmref{thm:linear}] \label{thm:new_linear}
%     Assume that this gradient descent algorithm satisfies \codref{cond:hypothesis_stability}
%     with $\beta=\calO(1)$.  
%     Then for any $\delta >0$, with probability at least $1-\delta$ 
%     over the random draws of datasets $\wt S$ and $S$, we have:
%     \begin{align}
%         \error_\calD(\widehat f) \le \error_\calS(\widehat f) + 1 - 2 \error_{\wt\calS}(\widehat f) + \left(\frac{1}{\sqrt{2}} + 1.5 \right) \sqrt{\frac{\log(4/\delta)}{m}} + \sqrt{\frac{4}{\delta}\left(\frac{1}{m} +\frac{3\beta}{m+n} \right)}  \,. \label{eq:gd_error}
%     \end{align} 
%     % for some constant $c\le 3.2$.
% \end{theorem}

% \subsection{Proof of \thmref{thm:new_linear}}
% We use a standard result from linear algebra, namely Shermann-Morrison formula~\citep{sherman1950adjustment} for matrix inversion:  

% \begin{lemma}[\citet{sherman1950adjustment}] \label{lem:sherman}
%     Suppose $\bA \in \Real^{n \times n}$ is an invertible square matrix and $u,v \in \Real^n$ are column vectors. Then $\bA + uv^T$ is invertible iff $1 + v^T \bA u \ne 0$ and in particular
%     \begin{align}
%         (\bA + u v^T)^{-1} = \bA^{-1}  - \frac{\bA^{-1} uv^T \bA^{-1} }{ 1 + v^T \bA^{-1} u} \,.
%     \end{align}   
% \end{lemma}
% \newcommand\byy[1]{\by_{\left(#1\right)}}
% \newcommand\bXX[1]{\bX_{\left(#1\right)}}
% \newcommand\ff[1]{\wh f_{\left(#1\right)}}

% For a given training set $S \cup \wt S_C$, define leave-one-out error on mislabeled points in the training data as $$\error_{\text{LOO}(\wt S_M) } = \frac{\sum_{(x_i, y_i) \in \wt S_M} \error( f_{(i)}( x_i), y_i)}{ \abs{\wt S_M }} \,, $$
% where $f_{(i)} \defeq f(\calA, (S \cup \wt S)_{(i)})$. To relate empirical leave-one-out error and population error with hypothesis stability condition, we use the following lemma:   

% \begin{lemma}[\citet{bousquet2002stability}] \label{lem:stability_error}
%     For the leave-one-out error, we have
%     \begin{align}
%         \Expo{ \left( \error_{\calDm}(\wh f) -\error_{\text{LOO}(\wt S_M) } \right)^2 } \le \frac{1}{2m_1}+  \frac{3\beta}{n + m}\,.
%     \end{align}   
%     % where $ f \defeq f(\calA, S \cup \wt S) $.
% \end{lemma}

% Proof of the above lemma is similar to the proof of  Lemma 9 in \citet{bousquet2002stability} and can be found in \appref{app:proof_lem_error}. 
% % 
% % Before presenting the result, we introduce some notation. 
% Before presenting the proof of \thmref{thm:new_linear}, we introduce some more notation. Let $\bX_{(i)}$ denote the matrix of covariates with $i^{\text{th}}$ point removed. Similarly let $\by_{(i)}$ be the array of responses with $i^{\text{th}}$ point removed. Define the corresponding regularized GD solution as $\wh w_{(i)} = \left( \bXX{i}^T\bXX{i}+\lambda \boldsymbol{I}\right)^{-1}\bXX{i}^T\byy{i}$. Define $\ff{i}(x) \defeq f(x ; \wh w_{(i)}) $.

% \begin{proof}[Proof of \thmref{thm:new_linear}]
%     Because squared loss minimization does not imply 0-1 error minimization, we cannot use arguments from \lemref{lem:fit_mislabeled}. This is the main technical difficulty. To compare the 0-1 error at a train point with an unseen point, 
%     we use the closed-form expression for $\widehat{w}$ and Shermann-Morrison formula to upper bound training error with leave-one-out cross validation error. 
    
%     The proof is divided into three parts: In part one, we show that 0-1 error on mislabeled points in the training set is lower than the error obtained by leave-one-out error at those points. In part two, we relate this leave-one-out error with the population error on mislabeled distribution using \codref{cond:hypothesis_stability}. While the empirical leave-one-out error is unbiased estimator of the average population error of leave-one-out classifiers, we need hypothesis stability to control the variance of empirical leave-one-out error. Finally in part three, we show that the error on the mislabeled training points can be estimated with just the randomly labeled and  clean training data (as in proof of \thmref{thm:error_ERM}).  

%     \textbf{Part 1 {} {}} First we relate training error with leave-one-out error.        
%     For any 
%     training point $(x_i, y_i)$ in $\wt S \cup S$, we have 
%     \begin{align}
%         \error(\wh f(x_i), y_i ) &= \indict{ y_i \cdot x_i^T \wh w < 0 } = \indict{ y_i \cdot x_i^T \left( \bX^T\bX+\lambda \boldsymbol{I}\right)^{-1}\bX^T\by < 0 } \\
%         &= \indict{ y_i \cdot x_i^T \underbrace{\left( \bXX{i}^T\bXX{i} + x_i ^T x_i +\lambda \boldsymbol{I}\right)^{-1}}_{\RN{1}} (\bXX{i}^T\byy{i} + y \cdot x_i) < 0 }
%     \end{align}
%     Letting $\bA = \left(\bXX{i}^T\bXX{i} +\lambda \boldsymbol{I}\right)$ and using \lemref{lem:sherman} on term 1, we have 
%     \begin{align}
%         \error(\wh f(x_i), y_i ) &= \indict{ y_i \cdot x_i^T \left[\bA^{-1} -  \frac{\bA^{-1} x_i x_i^T \bA^{-1}}{ 1 + x_i ^T \bA^{-1} x_i } \right] (\bXX{i}^T\byy{i} + y \cdot x_i) < 0 } \\
%         &= \indict{ y_i \cdot\left[ \frac{ x_i^T \bA^{-1} ( 1 + x_i ^T \bA^{-1} x_i ) -  x_i^T \bA^{-1} x_i x_i^T \bA^{-1}}{ 1 + x_i ^T \bA ^{-1}x_i } \right] (\bXX{i}^T\byy{i} + y \cdot x_i) < 0 } \\
%         &= \indict{ y_i \cdot\left[ \frac{ x_i^T \bA^{-1}}{ 1 + x_i ^T \bA ^{-1}x_i } \right] (\bXX{i}^T\byy{i} + y \cdot x_i) < 0 } \,.
%     \end{align}

%     Since $1 + x_i^T \bA^{-1} x_i > 0$, we have 
%     \begin{align}
%         \error(\wh f(x_i), y_i ) &= \indict{ y_i \cdot x_i^T \bA^{-1} (\bXX{i}^T\byy{i} + y \cdot x_i) < 0 } \\
%         &= \indict{ x_i^T \bA^{-1} x_i +  y_i \cdot x_i^T \bA^{-1} (\bXX{i}^T\byy{i}) < 0 } \\
%         &\le \indict{ y_i \cdot x_i^T \bA^{-1} (\bXX{i}^T\byy{i}) < 0 } = \error(\ff{i}(x_i), y_i ) \,.\label{eq:LOO_error}
%     \end{align}

%     Using \eqref{eq:LOO_error}, we have 
%     \begin{align}
%         \error_{\wt \calS_M } (\wh f) \le \error_{\text{LOO} (S_M)} \defeq \frac{\sum_{(x_i, y_i) \in \wt S_M} \error(\ff{i}(x_i), y_i ) }{\abs{\wt \calS_M}}\label{eq:LOO_error_final}
%     \end{align}
%     \textbf{Part 2 {}{}} We now relate RHS in \eqref{eq:LOO_error_final} with the population error on mislabeled distribution. To do this, we leverage \codref{cond:hypothesis_stability} and \lemref{lem:stability_error}. In particular, we have 

%     \begin{align}
%         \Expt{\calS \cup \wt \calS_M }{ \left(\error_{\calDm}(\wh f) - \error_{\text{LOO} (S_M)}\right)^2 } \le \frac{1}{2m_1} + \frac{3\beta}{m+n} \,.
%     \end{align}

%     Using Chebyshev's inequality, with probability at least $1-\delta$, we have 
%     \begin{align}
%         \error_{\text{LOO} (S_M)} \le  \error_{\calDm}(\wh f)   + \sqrt{\frac{1}{\delta}\left(\frac{1}{2m_1} +\frac{3\beta}{m+n} \right)} \,. \label{eq:final_mislabeled_linear}
%     \end{align}
    

%     \textbf{Part 3 {}{}} Combining \eqref{eq:final_mislabeled_linear} and \eqref{eq:LOO_error_final}, we have 

%     \begin{align}
%         \error_{\wt \calS_M } (\wh f) \le \error_{\calDm}(\wh f)   + \sqrt{\frac{1}{\delta}\left(\frac{1}{2m_1} +\frac{3\beta}{m+n} \right)} \,. \label{eq:linear_parallel_lem1}
%     \end{align}

%     Compare \eqref{eq:linear_parallel_lem1}, with \eqref{eq:lemma1_final} in the proof of \lemref{lem:fit_mislabeled}. We obtain a similar relationship between $\error_{\wt \calS_M }$ and $\error_{\calDm}$ but with a polynomial concentration instead of exponential concentration. 
%     In addition, since we just use concentration arguments to relate mislabeled error with the error on clean portion and unlabeled portion, we can directly use the results in \lemref{lem:mislabeled_error} and \lemref{lem:clear_error}. Therefore, combining results in \lemref{lem:mislabeled_error}, \lemref{lem:clear_error}, and \eqref{eq:linear_parallel_lem1} with union bound, we have with probability at least $1-\delta$

%     \begin{align}
%         \error_\calD(\widehat f) \le \error_\calS(\widehat f) + 1 - 2 \error_{\wt\calS}(\widehat f) + \left(\frac{1}{\sqrt{2}} + 1.5 \right) \sqrt{\frac{\log(4/\delta)}{m}} + \sqrt{\frac{4}{\delta}\left(\frac{1}{m} +\frac{3\beta}{m+n} \right)}  \,.
%     \end{align}
    

       
% \end{proof}

% \subsection{Discussion on \codref{cond:hypothesis_stability}}

% The quantity in LHS of \codref{cond:hypothesis_stability} measures how much the function learned by the algorithm (in terms of error on unseen point) will change when one point in the training set is removed. 
% % Discussion on exponential concentration and stronger condition. 
% Notice that hypothesis stability implies error stability, i.e., \codref{cond:error_stability} ~\cite{bousquet2002stability}.  In summary, while error stability allowed us to relate the average population error of the leave-one-out classifiers with the population error of the original classifier, we need hypothesis stability condition to control the variance of the empirical leave-one-out error. 

% Additionally, we note that while the dominating term in the RHS of \thmref{thm:new_linear} matches with the dominating term in ERM bound in \thmref{thm:error_ERM}, there is a polynomial concentration term (dependence on $1/\delta$ instead of $\log(\sqrt{1/\delta})$) in  \thmref{thm:new_linear}. 
% Since with hypothesis stability, we just bound the variance,  the polynomial concentration is due to the use of Chebyshev's inequality instead of an exponential tail inequality (as in \lemref{lem:fit_mislabeled}).
% Recent works have highlighted that slightly stronger condition than hypothesis stability can be used to obtained an exponential concentration for leave-one-out error~\citep{abou2019exponential}, but we leave this for future work for now. 
% % We leave 
% % However, the constants 

% % we also want to highlight  

% \subsection{Formal statement and proof of  of \propref{prop:early_stop}}

% Before formally presenting the result, we will introduce some notation.  By $\calL_{S}(w)$, we denote 
% the objective in \eqref{eq:l2_MSE_app} with $\lambda=0$. 
% Assume Singular Value Decomposition (SVD) of $\bX$  as $\sqrt{n} \bU \bS^{1/2} \bV^T$. Hence $\bX^T \bX = \bV \bS \bV^T$.
% Consider the GD iterates defined in \eqref{eq:GD_iterates_app}. 
% % 
% We now derive closed form expression for the $t^\text{th}$ iterate of gradient descent:  
% % 
% \begin{align}
%     w_t = w_{t-1} + \eta \cdot \bX^T (\by - \bX w_{t-1}) = (\bI - \eta \bV \bS \bV^T )w_{k-1} + \eta \bX^T \by \,.
% \end{align}
% Rotating by $\bV^T$, we get 
% \begin{align}
%     \wt w_t = (\bI - \eta\bS )\wt w_{k-1} + \eta \wt \by \,, \label{eq:GD_recur}
% \end{align}
% where $\wt w_t = \bV^T w_t $ and $\wt \by = \bV^T \bX^T \by$. Assuming the initial point $w_0 = 0$ and applying the recursion in \eqref{eq:GD_recur}, we get
% \begin{align}
%     \wt w_t = \bS ^{-1} ( \bI - (\bI - \eta \bS)^k ) \wt \by \,, 
% \end{align} 
% Projecting solution back to the original space, we have 
% \begin{align}
%      w_t = \bV \bS ^{-1} ( \bI - (\bI - \eta \bS)^k ) \bV^T \bX^T \by \,, 
% \end{align} 
% % We will work with this GD solution at any iterate $t$ in the next proposition. 
% Define $f_t(x) \defeq f(x;w_t)$ as the solution at the $t^{\text{th}}$ iterate. 
% Let $\wt w_{\lambda} = \argmin_{w} \calL_\calS (w;\lambda) = (\bX^T \bX + \lambda \bI)^{-1} \bX^T \by = \bV (\bS + \lambda \bI )^{-1} \bV^T \bX^T \by $. 
% % ) \,,$ for all $t=1,2,\ldots\,.$ 
% and define $\wt f_\lambda(x) \defeq f(x;\wt w_\lambda)$ as the regularized solution. 
% Assume $\kappa$ be the condition number of the population covariance matrix 
% and 
% let $s_\text{min}$ be the minimum positive singular value of the empirical covariance matrix. Our proof idea is inspired from recent work on relating gradient flow solution and regularized solution for regression problems \citep{ali2018continuous}. We will use the following lemma in the proof: 
% \begin{lemma} \label{lem:ineq_soln}
%     For all $x \in [0,1]$ and for all $ k \in \mathbb{N}$, we have (a) $ \frac{kx}{1+kx} \le 1- (1-x)^k$ and (b) $ 1- (1-x)^k \le 2 \cdot \frac{kx}{kx+1} $.
%     %  where $g(c)$ is a constant dependent on $c$. For $c = 1$, $g(c) = 2.0$.   
% \end{lemma}
% \begin{proof}
%     % [Proof of \lemref{lem:ineq_soln}]
%     % Part (a) is easy. 
%     Using $ (1-x)^k \le \frac{1}{1+kx}$, we have part (a). For part (b), we numerically maximize $\frac{ (1+kx ) (1 - (1-x)^k) }{kx}$ for all $k\ge 1$ and for all $x \in [0, 1]$.  
% \end{proof}

% % 
% % Next, 

% \begin{prop}[Formal statement of \propref{prop:early_stop}] \label{prop:formal_early_stop}
% Let $\lambda = \frac{1}{t\eta}$. For a training point $x$, we have 
% \begin{align*}
%     \Expt{x \sim \calS}{(f_t(x) - \wt f_\lambda(x))^2} &\le c(t,\eta) \cdot \Expt{x \sim \calS}{f_t(x)^2} \,, %\label{eq:early_stop}
% \end{align*}
% where $c(t, \eta) \defeq \min( 0.25, \frac{1}{s_\text{min}^2 t^2 \eta^2})$. Similarly for a test point, we have 
% \begin{align*}
%     \Expt{x \sim \calD_\calX}{(f_t(x) - \wt f_\lambda(x))^2} &\le \kappa \cdot c(t,\eta) \cdot \Expt{x \sim \calD_\calX}{f_t(x)^2} \,. %\label{eq:early_stop}
% \end{align*}
% \end{prop} 

% \begin{proof}
%     %%%%%%%%%%%%% 
%     We want to analyze the expected squared difference output of regularized linear regression with regularization constant $\lambda = \frac{1}{\eta t}$ and gradient descent solution at $t^\text{th}$ iterate. We separately expand the algebraic expression for squared difference at a training point and a test point. 
%     % We start by considering the difference  
%     Then the main step is to show that  $\left[ \bS ^{-1} ( \bI - (\bI - \eta \bS)^k )  - (\bS + \lambda \bI )^{-1}\right] \preceq c(\eta, t) \cdot \bS ^{-1} ( \bI - (\bI - \eta \bS)^k ) $.

%     %%%%%%%%%%%%%
    
%   \textbf{Part 1 {} {}} 
%     First, we will analyze the squared difference of output at a training point (for simplicity, we refer to $S \cup \wt S$ as $S$), i.e. 
%     \begin{align}
%         \Expt{ x \sim \calS }{\left(f_t(x) - \wt f_\lambda (x)\right)^2} &= \norm{\bX w_t - \bX \wt w_\lambda}{2}^2 =   \norm{\bX \bV \bS ^{-1} ( \bI - (\bI - \eta \bS)^t ) \bV^T \bX^T \by - \bX \bV (\bS + \lambda \bI )^{-1} \bV^T \bX^T \by }{2}^2 \\
%         &= \norm{\bX \bV \left(\bS ^{-1} ( \bI - (\bI - \eta \bS)^t ) - (\bS + \lambda \bI )^{-1} \right) \bV^T \bX^T \by  }{2} \\
%         &=  \by^T \bV \bX \left( \underbrace{\bS ^{-1} ( \bI - (\bI - \eta \bS)^t ) - (\bS + \lambda \bI )^{-1}}_{\RN{1}} \right)^2 \bS \bV^T \bX^T \by \label{eq:train_GD_rel}
%         %  (\bX \bV \bS ^{-1} ( \bI - (\bI - \eta \bS)^k ) \bV^T \bX^T \by)^T \bX \bV \bS ^{-1} ( \bI - (\bI - \eta \bS)^k ) \bV^T \bX^T \by
%     \end{align}
%     We now separately consider term 1. Substituting $\lambda = \frac{1}{t \eta}$, we get
%     \begin{align}
%         \bS ^{-1} ( \bI - (\bI - \eta \bS)^t ) - (\bS + \lambda \bI )^{-1} &= \bS^{-1} \left( ( \bI - (\bI - \eta \bS)^t ) - (\bI + \bS^{-1} \lambda )^{-1}\right) \\
%         &= \underbrace{\bS^{-1} \left( ( \bI - (\bI - \eta \bS)^t ) - (\bI + ( \bS t \eta)^{-1}  )^{-1}\right)}_{\bA}
%     \end{align}

%     We now separately bound the diagonal entries in matrix $\bA$. 
%     With $s_i$, we denote $i^{\text{th}}$ diagonal entry of $\bS$. Note that since $ \eta\le 1/\norm{S}{\text{op}}$, for all $i$, $\eta s_i  \le 1$.  Consider $i^{\text{th}}$ diagonal term (which is non-zero) of the diagonal matrix $\bA$, we have 
%     \begin{align}
%         \bA_{ii} = \frac{1}{s_i} \left(  1 - (1 - s_i \eta)^t - \frac{t \eta s_i}{1 + t \eta s_i } \right) &=  \frac{1 - (1 - s_i \eta)^t}{s_i} \left( \underbrace{ 1 - \frac{t \eta s_i}{(1 + t \eta s_i)(1 - (1 - s_i \eta)^t)}}_{\RN{2}} \right) \\ 
%          &\le \frac{1}{2}\left[ \frac{1 - (1 - s_i \eta)^t}{ s_i} \right] \tag*{(Using \lemref{lem:ineq_soln} (b))} \,.
%     \end{align} 
%     Additionally, we can also show the following upper bound on term 2: 
%     \begin{align}
%          1 - \frac{t \eta s_i}{(1 + t \eta s_i)(1 - (1 - s_i \eta)^t)} &= \frac{(1 + t \eta s_i)(1 - (1 - s_i \eta)^t) - t \eta s_i }{(1 + t \eta s_i)(1 - (1 - s_i \eta)^t)} \\
%          & \le  \frac{ 1 -  (1 - s_i \eta)^t - t \eta s_i (1 - s_i \eta)^t}{(1 + t \eta s_i)(1 - (1 - s_i \eta)^t)} \\
%          & \le \frac{1}{t\eta s_i} \,. \tag{Using \lemref{lem:ineq_soln} (a)}
%         %  &\le \frac{1}{2}\left[ \frac{1 - (1 - s_i \eta)^t}{ s_i} \right] \tag*{(Using \lemref{lem:ineq_soln})} \,.
%     \end{align} 

%     Combining both the upper bounds on each diagonal entry $\bA_{ii}$, we have 
%     \begin{align}
%     \bA \preceq c_1(\eta, t) \cdot \bS^{-1} ( \bI - (\bI - \eta \bS)^t ) \,, \label{eq:upperbound_diagonal}
%     \end{align}
%     where $c_1(\eta, t ) = \min(0.5, \frac{1}{t s_i \eta })$. Plugging this into \eqref{eq:train_GD_rel}, we have 
%     \begin{align}
%         \Expt{ x \sim \calS }{\left(f_t(x) - \wt f_\lambda (x)\right)^2} &\le c(\eta, t) \cdot \by^T \bV \bX  \left( \bS^{-1} ( \bI - (\bI - \eta \bS)^t ) \right)^2 \bS \bV^T \bX^T \by \\
%         &=   c(\eta, t) \cdot \by^T \bV \bX  \left( \bS^{-1} ( \bI - (\bI - \eta \bS)^t ) \right) \bS \left( \bS^{-1} ( \bI - (\bI - \eta \bS)^t ) \right) \bV^T \bX^T \by \\
%         & =  c(\eta, t) \cdot \norm{\bX w_t}{2}^2 \\
%         &= c(\eta, t) \cdot  \Expt{ x \sim \calS }{\left(f_t(x) \right)^2} \,,
%     \end{align}
%     where $c(\eta, t ) = \min(0.25, \frac{1}{t^2 s^2_i \eta^2 })$.

%     \textbf{Part 2 {} {}} With $\bSigma$, we denote the underlying true covariance matrix. We now consider the squared difference of output at an unseen point: 
%     \begin{align}
%         \Expt{ x \sim \calD_{\calX} }{\left(f_t(x) - \wt f_\lambda (x)\right)^2} &= \Expt{x \sim \calD_{\calX}}{\norm{x^T w_t - x^T \wt w_\lambda}{2}} \\
%         &=   \norm{x^T \bV \bS ^{-1} ( \bI - (\bI - \eta \bS)^t ) \bV^T \bX^T \by - x^T \bV (\bS + \lambda \bI )^{-1} \bV^T \bX^T \by }{2} \\
%         &= \norm{x^T \bV \left(\bS ^{-1} ( \bI - (\bI - \eta \bS)^t ) - (\bS + \lambda \bI )^{-1} \right) \bV^T \bX^T \by  }{2} \\
%         &= \by^T \bV \bX \left( \bS ^{-1} ( \bI - (\bI - \eta \bS)^t ) - (\bS + \lambda \bI )^{-1} \right) \bV^T \bSigma \bV \\ &\qquad \qquad \qquad \qquad \qquad \left( (\bI - (\bI - \eta \bS)^t ) - (\bS + \lambda \bI )^{-1} \right) \bV^T \bX^T \by \\
%         &\le \sigma_{\text{max}} \cdot \by^T \bV \bX \left( \underbrace{\bS ^{-1} ( \bI - (\bI - \eta \bS)^t ) - (\bS + \lambda \bI )^{-1}}_{\RN{1}} \right)^2 \bV^T \bX^T \by \,, \label{eq:test_GD_rel}
%         %  (\bX \bV \bS ^{-1} ( \bI - (\bI - \eta \bS)^k ) \bV^T \bX^T \by)^T \bX \bV \bS ^{-1} ( \bI - (\bI - \eta \bS)^k ) \bV^T \bX^T \by
%     \end{align}
%     where $\sigma_{\text{max}}$ is the maximum eigenvalue of the underlying covariance matrix $\bSigma$. Using the upper bound on term 1 in \eqref{eq:upperbound_diagonal}, we have 
%     \begin{align}
%         \Expt{ x \sim \calD_{\calX} }{\left(f_t(x) - \wt f_\lambda (x)\right)^2} &\le \sigma_{\text{max}} \cdot c(\eta, t) \cdot \by^T \bV \bX  \left( \bS^{-1} ( \bI - (\bI - \eta \bS)^t ) \right)^2 \bV^T \bX^T \by \\
%         &=   \kappa \cdot c(\eta, t) \cdot \sigma_{\text{min}}\cdot \norm{\bV \left( \bS^{-1} ( \bI - (\bI - \eta \bS)^t ) \right) \bV^T \bX^T \by}{2}^2 \\
%         &\le \kappa \cdot c(\eta, t) \cdot \left[ \bV \left( \bS^{-1} ( \bI - (\bI - \eta \bS)^t ) \right) \bV^T \bX^T \right]^T \bSigma \\
%         &\qquad \qquad \qquad \qquad \qquad \left[ \bV \left( \bS^{-1} ( \bI - (\bI - \eta \bS)^t ) \right) \bV^T \bX^T \right] \by \\
%         & = \kappa \cdot c(\eta, t) \cdot \Expt{x \sim \calD_{\calX}}{\norm{x^T w_t}{2}} \,.
%     \end{align}
% % 
% % 
%     % Since $ \eta\le 1/\norm{S}{\text{op}}$, invoking \lemref{lem:ineq_soln} to upper bound term 1 with
% \end{proof}


% \newpage
% \section{Additional experiments and details}\label{app:exp}
% \newcommand\tab[1][1cm]{\hspace*{#1}}

% \subsection{Datasets} \label{sec:app_dataset}

% \textbf{Toy Dataset {} {}} Assume fixed constants $\mu$ and $\sigma$. For a given label $y$, we simulate features $x$ in our toy classification setup as follows: 
% \begin{align*}
%     x \defeq \texttt{concat} \left[ x_1, x_2\right] \quad \text{where} \quad  x_1 \sim  \calN( y \cdot \mu, \sigma^2 I_{d \times d}) \ \  \text{and} \ \  x_1 \sim  \calN( 0, \sigma^2 I_{d \times d}) \,.
% \end{align*}  
% % where $y$ is the true label and $x$ is the corresponding feature vector. 
% In experiements throughout the paper, we fix dimention $d=100$, $\mu = 1.0 $, and $\sigma = \sqrt{d}$. Intuitively, $x_1$ carries the information about the underlying label and $x_2$ is additional noise independent of the underlying label. 

% \textbf{CV datasets {} {}} We use MNIST~\citep{lecun1998mnist} and CIFAR10~\cite{krizhevsky2009learning}. 
% % For binary tasks, 
% We produce a binary variant from the multiclass classification problem by mapping classes $\{0,1,2,3,4\}$ to label $1$ and $\{ 5,6,7,8,9\}$ to label $-1$. For CIFAR dataset, we also use the standard data augementation of random crop and horizontal flip. PyTorch code is as follows: 

% \texttt{(transforms.RandomCrop(32, padding=4),\\
% \tab transforms.RandomHorizontalFlip())}

% \textbf{NLP dataset {} {}} We use IMDb Sentiment analysis~\citep{maas2011learning} corpus.  

% \subsection{Architecture Details} 

% All experiments were run on NVIDIA GeForce RTX 2080 Ti GPUs. We used PyTorch~\citep{NEURIPS2019a9015} and Keras with Tensorflow~\citep{abadi2016tensorflow} backend for experiments. 
% % , ELMo embeddings~\citep{Peters:2018}, and Hugging Face Transformers~\citep{wolf-etal-2020-transformers}. 

% \textbf{Linear model {} {}} For the toy dataset, we simulate a linear model with scalar output and the same number of parameters as the number of dimensions.   

% \textbf{Wide nets {} {}} To simulate the NTK regime, we experiment with $2-$layered wide nets. The PyTorch code for 2-layer wide MLP is as follows: 


% \texttt{ nn.Sequential( \\
% \tab     nn.Flatten(),\\
% \tab    nn.Linear(input\_dims, 200000, bias=True),\\
% \tab    nn.ReLU(),\\
% \tab    nn.Linear(200000, 1, bias=True)\\
% \tab     )}


% We experiment both (i) with the first layer fixed at random initialization; (ii)  and updating both layers' weights.     

% \textbf{Deep nets for CV tasks {} {}} We consider a 4-layered MLP. The PyTorch code for 4-layer MLP is as follows: 

% \texttt{ nn.Sequential(nn.Flatten(), \\
% \tab        nn.Linear(input\_dim, 5000, bias=True),\\
% \tab        nn.ReLU(),\\
% \tab        nn.Linear(5000, 5000, bias=True),\\
% \tab        nn.ReLU(),\\
% \tab        nn.Linear(5000, 5000, bias=True),\\
% \tab        nn.ReLU(),\\
% % \tab        nn.Linear(5000, 5000, bias=True),\\
% % \tab        nn.ReLU(),\\
% \tab        nn.Linear(1024, num\_label, bias=True)\\
% \tab        )}

% For MNIST, we use $1000$ nodes instead of $5000$ nodes in the hidden layer. 
% % 
% We also experiment with convolutional nets. In particular, we use ResNet18 \citep{he2016deep}. Implementation adapted from:  \url{https://github.com/kuangliu/pytorch-cifar.git}. 

% \textbf{Deep nets for NLP {} {}} We use a simple LSTM model with embeddings intialized with ELMo embeddings~\citep{Peters:2018}. Code adapted from: \url{https://github.com/kamujun/elmo_experiments/blob/master/elmo_experiment/notebooks/elmo_text_classification_on_imdb.ipynb} 

% We also evaluate our bounds with a BERT model. In particular, we fine-tune an off-the-shelf uncased BERT model~\citep{devlin2018bert}. Code adapted from Hugging Face Transformers~\citep{wolf-etal-2020-transformers}: \url{https://huggingface.co/transformers/v3.1.0/custom_datasets.html}. 


% \subsection{Additonal experiments}

% 1. SGD with linear models on cross entropy and MSE loss. 

% 2. CE loss and SGD. GD with MSE loss 

% 3. Binary MNIST with MLP. multiclass MNIST  

% \textbf{Results on CIFAR 10 {} {}} 
% % 
% We plot epoch wise error curve for results in \tabref{table:multiclass}. We observe the same trend as in \figref{fig:error_CIFAR10}. Additionally, we plot an \emph{oracle bound} obtained by tracking the error on mislabeled data which nevertheless were predicted as true label. To obtain an exact emprical value of the oracle bound, we need underlying true labels for the randomly labeled data. 
% % Note that our bound in \thmref{thm:multiclass_ERM}, lower bounds the accuracy as predicted by the oracle bound. 
% While with just access to extra unlabeled data we cannot calculate oracle bound, we note that the oracle bound is very tight and never violated in practice underscoring an importamt aspect of generalization in multiclass problems. This highlight that even a stronger conjecture may hold in multiclass classification, i.e., error on mislabeled data (where nevertheless true label was predicted) lower bounds the population error on the distribution of mislabeled data and hence, the error on (a specific) mislabeled portion predicts the population accuracy on clean data. 
% % 
% On the other hand, the dominating term of in \thmref{thm:multiclass_ERM} is loose when compared with the oracle bound. The main reason, we believe is the pessimistic upper bound in \eqref{eq:lemma1_final_multi_prev} in the proof of \lemref{lem:fit_mislabeled_multi}. We leave an investigation on this gap for future. 
% % of fit 

% % However, oracle bound highlights two . One,  



% \begin{figure}[h]
%     \centering 
%     % \vspace{-15pt}
%     % \includegraphics[width=0.9\linewidth]{example-image-a}
%     \includegraphics[width=0.4\linewidth]{figures/CIFAR10-FNN.pdf} \hfil
%     \includegraphics[width=0.4\linewidth]{figures/CIFAR10-Resnet.pdf}
%     % \includegraphics[width=0.9\linewidth]{figures/{CIFAR10_rn=0.1_lr=0.2_wd=0.005}.png}
%     % \vspace{-10pt}
%     \caption{ Per epoch curves for CIFAR10 corresponding results in \tabref{table:multiclass}. As before, we just plot the dominating term in the RHS of \eqref{eq:multiclass_ERM} as predicted bound. Additionally, we also plot the predicted lower bound by the error on mislabeled data which nevertheless were predicted as true label. We refer to this as ``Oracle bound''. See text for more details. 
%     % 
%     % except for the stopping point. 
%     % The bound predicted by RATT (RHS in \eqref{eq:multiclass_ERM}) is vacuous. 
%     }\label{fig:error_epoch_CIFAR10}
%     % \vspace{-15pt}
% \end{figure}


% \textbf{Results on CIFAR 100 {} {}} 
% % 
% On CIFAR100, our bound in \eqref{eq:multiclass_ERM} yields vacous bounds. However, the oracle bound as explained above yields tight guarantees in the initial phase of the learning (i.e., when learning rate is less than $0.1$). 

% \begin{figure}[h]
%     \centering 
%     % \vspace{-15pt}
%     % \includegraphics[width=0.9\linewidth]{example-image-a}
%     \includegraphics[width=0.4\linewidth]{figures/CIFAR100-Resnet.pdf}
%     % \includegraphics[width=0.9\linewidth]{figures/{CIFAR10_rn=0.1_lr=0.2_wd=0.005}.png}
%     % \vspace{-10pt}
%     \caption{ Predicted lower bound by the error on mislabeled data which nevertheless were predicted as true label with ResNet18 on CIFAR100. We refer to this as ``Oracle bound''. See text for more details. 
%     % 
%     % except for the stopping point. 
%     The bound predicted by RATT (RHS in \eqref{eq:multiclass_ERM}) is vacuous. 
%     }\label{fig:error_CIFAR100}
%     % \vspace{-15pt}
% \end{figure}


% % \paragraph{Experiments on CIFAR100} 



% \subsection{Hyperparameter Details}


% \textbf{\figref{fig:error_CIFAR10} {} {}} We use clean training dataset of size $40,000$. We fix the amount of unlabeled data at $20\%$ of the clean size, i.e. we include additional $8,000$ points with randomly assigned labels. We use test set of $10,000$ points. For both MLP and ResNet, we use SGD with an initial learning rate of $0.1$ and momentum $0.9$. We fix the weight decay parameter at $5\times 10^{-4}$. After $100$ epochs, we decay the learning rate to $0.01$. We use SGD batch size of $100$. 

% \textbf{\figref{fig:error_binary} (a) {} {}} We obtain a toy dataset according to the process described in \secref{sec:app_dataset}. We fix $d=100$ and create a dataset of $50,000$ points with balanced classes. Moreover, we sample additional covariates with the same procedure to create randomly labeled dataset. For both SGD and GD training, we use a fixed learning rate $0.1$.    

% \textbf{\figref{fig:error_binary} (b) {} {}} Similar to binary CIFAR, we use clean training dataset of size $40,000$ and fix the amount of unlabeled data at $20\%$ of the clean dataset size. To train wide nets, we use a fixed learning of $0.001$ with GD and SGD. We decide the weight decay parameter and the early stopping point that maximizes our generalization bound (i.e. without peeking at unseen data ).  We use SGD batch size of $100$. 

% \textbf{\figref{fig:error_binary} (c) {} {}} With IMDb dataset, we use a clean dataset of size $20,000$ and as before, fix the amount of unlabeled data at $20\%$ of the clean data. To train ELMo model, we use Adam optimizer with a fixed learning rate $0.01$ and weight decay $10^{-6}$ to minimize cross entropy loss. We train with batch size $32$ for 3 epochs. To fine-tune BERT model, we use Adam optimizer with learning rate $5\times 10^{-5}$ to minimize cross entropy loss. We train with a batch size of $16$ for 1 epoch.    

% \textbf{\tabref{table:multiclass} {} {}} For multiclass datasets, we train both MLP and ResNet with the same hyperparameters as described before. We sample a clean training dataset of size $40,000$ and fix the amount of unlabeled data at $20\%$ of the clean size. We use SGD with an initial learning rate of $0.1$ and momentum $0.9$. We fix the weight decay parameter at $5\times 10^{-4}$. After $30$ epochs for ResNet and after $50$ epochs for MLP, we decay the learning rate to $0.01$.  We use SGD with batch size $100$. 
% For \figref{fig:error_CIFAR100}, we use the same hyperparameters as 
% CIFAR10 training, except we now decay learning rate after $100$ epochs. 


% In all experiments, to identify the best possible accuracy on just the clean data, we use the exact same set of hyperparamters except the stopping point. We choose a stopping point that maximizes test performance. 

% \subsection{Summary of experiments }

% \begin{center}
%     \begin{table}[H] 
%         \centering
%         \begin{tabular}{|c|c|c|c|} 
%         \hline
%         Classification type & Model category & Model & Dataset  \\ [0.5ex] 
%         \hline
%         \hline
%         \multirow{9}{*}{Binary} & Low dimensional & Linear model & Toy Gaussain dataset  \\
%                         \cline{2-4}
%                          & \multirow{1}{*}{Overparameterized linear nets} 
%                         %  & Linear model & Toy Gaussain dataset \\
%                         %  \cline{3-4}
%                         %  & & 2-layer wide net& Toy Gaussain dataset \\
%                         %  \cline{3-4}
%                          & 2-layer wide net & Binary MNIST \\
%                          \cline{2-4}                 
%                          & \multirow{6}{*}{Deep nets} & \multirow{2}{*}{MLP} & Binary MNIST \\
%                          \cline{4-4}
%                          & &  & Binary CIFAR \\
%                          \cline{3-4}
%                          &  & \multirow{2}{*}{ResNet} & Binary MNIST \\
%                          \cline{4-4}
%                          & &  & Binary CIFAR \\
%                          \cline{3-4}
%                          &  & ELMo-LSTM model & IMDb Sentiment Analysis \\
%                          \cline{3-4}
%                          & & BERT pre-trained model & IMDb Sentiment Analysis \\
%         \hline
%         \multirow{5}{*}{Multiclass} & \multirow{5}{*}{Deep nets} & \multirow{2}{*}{MLP} & MNIST \\
%                         \cline{4-4} 
%                         & & & CIFAR10 \\                   
%                         \cline{3-4}
%                          &   & \multirow{3}{*}{ResNet} & MNIST \\
%                          \cline{4-4}
%                          &   & & CIFAR10 \\
%                          \cline{4-4}
%                          &   & & CIFAR100 \\
%         \hline
%         \end{tabular}
%         % \caption{Summary of experiments performed} \label{table:experiments}
%     \end{table}    
%     % \footnotetext[6]{We use both MSE loss and cross-entropy loss.}
%     % \footnotetext[6]{We try 2 variants: one with a fixed first layer and the other with both layers trainable.}
% \end{center}

% \newpage
% \section{Proof of \lemref{lem:stability_error}} \label{app:proof_lem_error}

% \begin{proof}[Proof of \lemref{lem:stability_error}]
%     Recall, we have a training set $S \cup \wt S_C$. We defined leave-one-out error on mislabeled points as $$\error_{\text{LOO}(\wt S_M) } = \frac{\sum_{(x_i, y_i) \in \wt S_M} \error( f_{(i)}( x_i), y_i)}{ \abs{\wt S_M }} \,, $$
%     where $f_{(i)} \defeq f(\calA, (S \cup \wt S)_{(i)})$. Define $S^\prime \defeq S \cup \wt S$. Assume $(x,y)$ and $(x^\prime,y^\prime)$ as i.i.d. samples from ${\calDm}$. 
%     Using Lemma 25 in \citet{bousquet2002stability}, we have
%     \begin{align*}
%         \Expo{ \left( \error_{\calDm}(\wh f) -\error_{\text{LOO}(\wt S_M) } \right)^2 } \le & \Expt{ S^\prime, (x,y), (x^\prime,y^\prime) }{ \error(\wh f(x), y ) \error(\wh f(x^\prime), y^\prime )} - 2 \Expt{ S^\prime, (x,y) }{ \error(\wh f(x), y ) \error(f_{(i)}(x_i), y_i )} \\
%         & + \frac{m_1-1}{m_1}\Expt{ S^\prime }{  \error(f_{(i)}(x_i), y_i )  \error(f_{(j)}(x_j), y_j )} + \frac{1}{m_1} \Expt{ S^\prime }{  \error(f_{(i)}(x_i), y_i ) } \,. \numberthis \label{eq:main_reln}
%     \end{align*}
%     We can rewrite the equation above as : 
%     \begin{align*}
%         \Expo{ \left( \error_{\calDm}(\wh f) -\error_{\text{LOO}(\wt S_M) } \right)^2 } \le &  \, \underbrace{\Expt{ S^\prime, (x,y), (x^\prime,y^\prime) }{ \error(\wh f(x), y ) \error(\wh f(x^\prime), y^\prime ) - \error(\wh f(x), y ) \error(f_{(i)}(x_i), y_i )}}_{\RN{1}} \\
%         & + \underbrace{\Expt{ S^\prime }{  \error(f_{(i)}(x_i), y_i )  \error(f_{(j)}(x_j), y_j ) -  \error(\wh f(x), y ) \error(f_{(i)}(x_i), y_i )}}_{\RN{2}} \\ &+ \underbrace{\frac{1}{m_1} \Expt{ S^\prime }{  \error(f_{(i)}(x_i), y_i ) - \error(f_{(i)}(x_i), y_i )  \error(f_{(j)}(x_j), y_j ) }}_{\RN{3}} \,. \numberthis \label{eq:main_reln2}
%     \end{align*}
    
%     We will now bound term $\RN{3}$.  Using Schwarz's inequality, we have
    
%     \begin{align}
%         \Expt{ S^\prime }{  \error(f_{(i)}(x_i), y_i ) - \error(f_{(i)}(x_i), y_i )  \error(f_{(j)}(x_j), y_j ) }^2 &\le  \Expt{ S^\prime }{  \error(f_{(i)}(x_i), y_i ) }^2 \Expt{S^\prime}{1 -   \error(f_{(j)}(x_j), y_j ) }^2 \\
%         &\le \frac{1}{4} \label{eq:term1_lem12}
%     \end{align}
    
%     Note that since $(x_i,y_i)$, $(x_j ,y_j )$, $(x,y)$, and $(x^\prime, y^\prime)$ are all from same distribution $\calDm$, we directly incorporate the bounds on term $\RN{1}$ and $\RN{2}$ from proof of Lemma 9 in \citet{bousquet2002stability}. Combining that with \eqref{eq:term1_lem12} and our definition of hypothesis stability in \codref{cond:hypothesis_stability}, we have the required claim. 
    
    
%     % We now re-write term $\RN{1}$ as
%     % \begin{align*}
%     %         &\Expt{S^\prime, (x,y), (x^\prime,y^\prime) }{ \error(\wh f(x), y ) \error(\wh f(x^\prime), y^\prime ) - \error(\wh f(x), y ) \error(f_{(i)}(x_i), y_i )} \\ & \qquad = \Expt{ S^\prime, (x,y), (x^\prime,y^\prime) }{ \error(\wh f(x), y ) \error(\wh f  (x^\prime), y^\prime ) - \error(\wh f ^\prime(x), y ) \error(f_{(i)}(x^\prime), y^\prime )} \tag{Exchanging $(x_i, y_i)$ with $(x^\prime, y^\prime)$ in the second term} \\
%     %         & \qquad = \Expt{ S^\prime, (x,y), (x^\prime,y^\prime) }{  \left(\error(\wh f(x), y )-  \error(f_{(i)}(x), y ) \right) \error(\wh f  (x^\prime), y^\prime )  } \\
%     %         & \qquad  + \Expt{ S^\prime, (x,y), (x^\prime,y^\prime) }{  \left(\error(f_{(i)}(x), y ) -\error(\wh f ^\prime(x), y ) \right) \error(\wh f  (x^\prime), y^\prime )}  \\
%     %         & \qquad +\Expt{ S^\prime, (x,y), (x^\prime,y^\prime) }{  \left( \error(\wh f  (x^\prime), y^\prime ) -  \error(f_{(i)}(x^\prime), y^\prime ) \right) \error(\wh f ^\prime(x), y ) }  \,, \numberthis \label{eq:term1_final}
%     % \end{align*}
%     % where $\wh f^\prime$ is the classifier obtained by training on $ S^\prime_{(i)} \cup \{ (x^\prime, y^\prime) \} $. Similarly we can re-write term $\RN{2}$ as 
%     % \begin{align*}
%     %     & \Expt{ S^\prime }{  \error(f_{(i)}(x_i), y_i )  \error(f_{(j)}(x_j), y_j ) -  \error(\wh f(x), y ) \error(f_{(i)}(x_i), y_i )} \\
%     %     &\quad  = \Expt{ S^\prime, (x,y), (x^\prime,y^\prime)}{  \error(f^{\prime\prime}_{(i)}(x), y )  \error(f_{(j)}^{\prime}(x^\prime), y^\prime ) -  \error(\wh f(x), y ) \error(f_{(i)}(x_i), y_i )} \tag{Exchanging $(x_i, y_i)$ with $(x, y)$ and $(x_j, y_j)$ with $(x^\prime, y^\prime)$ in the first term}\\
%     %     &\quad = \Expt{ S^\prime, (x,y), (x^\prime,y^\prime)}{  \error(f^{\prime\prime}_{(j)}(x), y )  \error(f_{(i)}^{\prime}(x^\prime), y^\prime ) -  \error(\wh f^\prime (x), y ) \error(f^\prime_{(j)}(x^\prime), y^\prime )} \tag{Exchanging $(x_i, y_i)$ and $(x_j, y_j)$ and then replacing $(x_j, y_j)$ with $(x^\prime, y^\prime)$ in the second term} \\
%     %     & \quad = \Expt{ S^\prime, (x,y), (x^\prime,y^\prime) }{  \left( \error(f_{(i)}^{\prime}(x^\prime), y^\prime )   -  \error(\wh f^{\prime\prime}  (x^\prime), y^\prime ) \right)  \error(f^{\prime\prime}_{(j)}(x), y )   } \\
%     %     & \quad  + \Expt{ S^\prime, (x,y), (x^\prime,y^\prime) }{  \left( \error(f^{\prime\prime}_{(j)}(x), y )  -\error(\wh f ^\prime(x), y ) \right) \error(\wh f^{\prime\prime}  (x^\prime), y^\prime )  }  \\
%     %     & \quad+ \Expt{ S^\prime, (x,y), (x^\prime,y^\prime) }{  \left( \error(\wh f^{\prime\prime}  (x^\prime), y^\prime )  -  \error(f^\prime_{(j)}(x^\prime), y^\prime ) \right)  \error(\wh f^\prime (x), y ) }   \\
%     %     & \quad = \Expt{ S^\prime, (x,y), (x^\prime,y^\prime) }{  \left( \error(f_{(i)}^{\prime}(x^\prime), y^\prime )   -  \error(\wh f (x^\prime), y^\prime ) \right)  \error(f_{(i)}(x_j), y_j )   } \\
%     %     & \quad  + \Expt{ S^\prime, (x,y), (x^\prime,y^\prime) }{  \left( \error(f^{\prime\prime}_{(j)}(x), y )  -\error(\wh f (x), y ) \right) \error(\wh f^{\prime\prime}  (x_j), y_j )  }  \\
%     %     & \quad+ \Expt{ S^\prime, (x,y), (x^\prime,y^\prime) }{  \left( \error(\wh f^{\prime\prime}  (x^\prime), y^\prime )  -  \error(f^\prime_{(j)}(x^\prime), y^\prime ) \right)  \error(\wh f^\prime (x^\prime), y^\prime ) }  \,, \numberthis \label{eq:term2_final}
%     % \end{align*}
%     % where $f^{\prime\prime}_{(j)}$ is trained on $S^\prime_{(j,i)} \cup {(x,y)}$, $f^{\prime}_{(i)}$ is trained on $S^\prime_{(j,i)} \cup {(x^\prime,y^\prime)}$, and $\wh f^{\prime\prime} $ is trained on $S^\prime_{(j)} \cup {(x,y)}$. Note in the last line we replaced $(x,y)$ by $(x_j, y_j)$ in the first term, replaced $(x^\prime,y^\prime)$ by $(x_j, y_j)$ in the second term and exchanged $(x_i,y_i)$ with $(x_j,y_j)$ and also $(x,y)$ and $(x^\prime, y^\prime)$
    
    
% \end{proof}


%%%%%%%%%%%%%%%%%%%%%%%%%%%%%%%%%%%%%%%%%%%%%%%%%%%%
%%% REFERENCES
\nocite{*}
\ifdefined\hpec
%*flatex input: [pageRank-MAIN.bbl]
% Generated by IEEEtran.bst, version: 1.12 (2007/01/11)
\begin{thebibliography}{10}
\providecommand{\url}[1]{#1}
\csname url@samestyle\endcsname
\providecommand{\newblock}{\relax}
\providecommand{\bibinfo}[2]{#2}
\providecommand{\BIBentrySTDinterwordspacing}{\spaceskip=0pt\relax}
\providecommand{\BIBentryALTinterwordstretchfactor}{4}
\providecommand{\BIBentryALTinterwordspacing}{\spaceskip=\fontdimen2\font plus
\BIBentryALTinterwordstretchfactor\fontdimen3\font minus
  \fontdimen4\font\relax}
\providecommand{\BIBforeignlanguage}[2]{{%
\expandafter\ifx\csname l@#1\endcsname\relax
\typeout{** WARNING: IEEEtran.bst: No hyphenation pattern has been}%
\typeout{** loaded for the language `#1'. Using the pattern for}%
\typeout{** the default language instead.}%
\else
\language=\csname l@#1\endcsname
\fi
#2}}
\providecommand{\BIBdecl}{\relax}
\BIBdecl

\bibitem{page1999pagerank}
L.~Page, S.~Brin, R.~Motwani, and T.~Winograd, ``The {P}age{R}ank citation
  ranking: Bringing order to the web.'' Stanford InfoLab, Tech. Rep., 1999.

\bibitem{ilprints361}
S.~Brin and L.~Page, ``The anatomy of a large-scale hypertextual web search
  engine,'' \emph{Computer networks and ISDN systems}, vol.~30, no. 1-7, pp.
  107--117, 1998.

\bibitem{sheldon2010manipulation}
D.~Sheldon, ``Manipulation of {P}age{R}ank and {C}ollective {H}idden {M}arkov
  {M}odels,'' Ph.D. dissertation, Cornell University, Ithaca, NY, USA, 2010.

\bibitem{haveliwala2003topic}
T.~H. Haveliwala, ``Topic-sensitive {P}age{R}ank: A context-sensitive ranking
  algorithm for web search,'' \emph{IEEE Transactions on Knowledge and Data
  Engineering}, vol.~15, no.~4, pp. 784--796, 2003.

\bibitem{morrison2005generank}
J.~L. Morrison, R.~Breitling, D.~J. Higham, and D.~R. Gilbert, ``Generank:
  using search engine technology for the analysis of microarray experiments,''
  \emph{BMC bioinformatics}, vol.~6, no.~1, p. 233, 2005.

\bibitem{wu2010krylov}
G.~Wu, Y.~Zhang, and Y.~Wei, ``Krylov subspace algorithms for computing
  generank for the analysis of microarray data mining,'' \emph{Journal of
  Computational Biology}, vol.~17, no.~4, pp. 631--646, 2010.

\bibitem{langville2004deeper}
A.~N. Langville and C.~D. Meyer, ``Deeper inside {P}age{R}ank,'' \emph{Internet
  Mathematics}, vol.~1, no.~3, pp. 335--380, 2004.

\bibitem{berkhin2005survey}
P.~Berkhin, ``A survey on {P}age{R}ank computing,'' \emph{Internet
  Mathematics}, vol.~2, no.~1, pp. 73--120, 2005.

\bibitem{ilprints596}
T.~Haveliwala, S.~Kamvar, and G.~Jeh, ``An analytical comparison of approaches
  to personalizing {P}age{R}ank,'' Stanford InfoLab, Technical Report 2003-35,
  June 2003.

\bibitem{ilprints579}
S.~Kamvar, T.~Haveliwala, C.~Manning, and G.~Golub, ``Exploiting the block
  structure of the web for computing {P}age{R}ank,'' Stanford InfoLab,
  Technical Report 2003-17, 2003.

\bibitem{boldi2005pagerank}
P.~Boldi, M.~Santini, and S.~Vigna, ``{P}age{R}ank as a function of the damping
  factor,'' in \emph{Proceedings of the 14th International Conference on World
  Wide Web}.\hskip 1em plus 0.5em minus 0.4em\relax ACM, 2005, pp. 557--566.

\bibitem{bressan2010choose}
M.~Bressan and E.~Peserico, ``Choose the damping, choose the ranking?''
  \emph{Journal of Discrete Algorithms}, vol.~8, no.~2, pp. 199--213, 2010.

\bibitem{chung2007heat}
F.~Chung, ``The heat kernel as the {P}age{R}ank of a graph,'' \emph{Proceedings
  of the National Academy of Sciences}, vol. 104, no.~50, pp. 19\,735--19\,740,
  2007.

\bibitem{chung2009local}
------, ``A local graph partitioning algorithm using heat kernel
  {P}age{R}ank,'' \emph{Internet Mathematics}, vol.~6, no.~3, pp. 315--330,
  2009.

\bibitem{arasu2002pagerank}
A.~Arasu, J.~Novak, A.~Tomkins, and J.~Tomlin, ``{P}age{R}ank computation and
  the structure of the web: Experiments and algorithms,'' in \emph{Proceedings
  of the 11th International Conference on World Wide Web}, 2002, pp. 107--117.

\bibitem{kullback1951information}
S.~Kullback and R.~A. Leibler, ``On information and sufficiency,'' \emph{The
  Annals of Mathematical Statistics}, vol.~22, no.~1, pp. 79--86, 1951.

\bibitem{lee2003fast}
C.~P.-C. Lee, G.~H. Golub, and S.~A. Zenios, ``A fast two-stage algorithm for
  computing {P}age{R}ank and its extensions,'' \emph{Scientific Computation and
  Computational Mathematics}, vol.~1, no.~1, pp. 1--9, 2003.

\bibitem{langville2006reordering}
A.~N. Langville and C.~D. Meyer, ``A reordering for the {P}age{R}ank problem,''
  \emph{SIAM Journal on Scientific Computing}, vol.~27, no.~6, pp. 2112--2120,
  2006.

\bibitem{dulmage1958coverings}
A.~L. Dulmage and N.~S. Mendelsohn, ``Coverings of bipartite graphs,''
  \emph{Canadian Journal of Mathematics}, vol.~10, no.~4, pp. 516--534, 1958.

\bibitem{google-net}
Google, ``Google programming contest,''
  \url{http://www.google.com/programming-contest/}, 2002.

\bibitem{kamvar2003extrapolation}
S.~D. Kamvar, T.~H. Haveliwala, C.~D. Manning, and G.~H. Golub, ``Extrapolation
  methods for accelerating {P}age{R}ank computations,'' in \emph{Proceedings of
  the 12th International Conference on World Wide Web}.\hskip 1em plus 0.5em
  minus 0.4em\relax ACM, 2003, pp. 261--270.

\bibitem{kamvar2004adaptive}
S.~Kamvar, T.~Haveliwala, and G.~Golub, ``Adaptive methods for the computation
  of {P}age{R}ank,'' \emph{Linear Algebra and its Applications}, vol. 386, pp.
  51--65, 2004.

\bibitem{DBLP:journals/corr/LofgrenBGS14}
P.~A. Lofgren, S.~Banerjee, A.~Goel, and C.~Seshadhri, ``{FAST-PPR:} scaling
  personalized pagerank estimation for large graphs,'' in \emph{Proceedings of
  the 20th ACM SIGKDD International Conference on Knowledge Discovery and Data
  Mining}.\hskip 1em plus 0.5em minus 0.4em\relax ACM, 2014, pp. 1436--1445.

\bibitem{kunegis2013konect}
J.~Kunegis, ``{KONECT}--the koblenz network collection,'' in \emph{Proceedings
  of the 22nd International Conference on World Wide Web}.\hskip 1em plus 0.5em
  minus 0.4em\relax ACM, 2013, pp. 1343--1350.

\bibitem{auer2007dbpedia}
S.~Auer, C.~Bizer, G.~Kobilarov, J.~Lehmann, R.~Cyganiak, and Z.~Ives,
  ``{DB}pedia: A nucleus for a web of open data,'' in \emph{The Semantic
  Web}.\hskip 1em plus 0.5em minus 0.4em\relax Springer, 2007, pp. 722--735.

\bibitem{Kwak10www}
H.~Kwak, C.~Lee, H.~Park, and S.~Moon, ``{W}hat is {T}witter, a social network
  or a news media?'' in \emph{Proceedings of the 19th International Conference
  on World Wide Web}.\hskip 1em plus 0.5em minus 0.4em\relax New York, NY, USA:
  ACM, 2010, pp. 591--600.

\bibitem{icwsm10cha}
M.~Cha, H.~Haddadi, F.~Benevenuto, and K.~P. Gummadi, ``{Measuring User
  Influence in Twitter: The Million Follower Fallacy},'' in \emph{Proceedings
  of the 4th International AAAI Conference on Weblogs and Social Media
  (ICWSM)}, Washington DC, USA, May 2010.

\bibitem{yang2015defining}
J.~Yang and J.~Leskovec, ``Defining and evaluating network communities based on
  ground-truth,'' \emph{Knowledge and Information Systems}, vol.~42, no.~1, pp.
  181--213, 2015.

\bibitem{MS-qian-2018}
Y.~Qian, ``Variable damping effect on network propagation,'' Master's thesis,
  Duke University, Durham, NC, USA, May 2018.

\bibitem{MS-Xichen-2018}
X.~Chen, ``Exploiting common structures across multiple network propagation
  schemes,'' Master's thesis, Duke University, Durham, NC, USA, May 2018.

\bibitem{ilprints582}
T.~Haveliwala and S.~Kamvar, ``The second eigenvalue of the google matrix,''
  Stanford InfoLab, Technical Report 2003-20, 2003.

\bibitem{richardson2002intelligent}
M.~Richardson and P.~Domingos, ``The intelligent surfer: Probabilistic
  combination of link and content information in {P}age{R}ank,'' in
  \emph{Advances in Neural Information Processing Systems}, 2002, pp.
  1441--1448.

\bibitem{haveliwala1999efficient}
T.~Haveliwala, ``Efficient computation of {P}age{R}ank,'' Stanford, Tech. Rep.,
  1999.

\bibitem{langville2011google}
A.~N. Langville and C.~D. Meyer, \emph{Google's {P}age{R}ank and beyond: The
  science of search engine rankings}.\hskip 1em plus 0.5em minus 0.4em\relax
  Princeton University Press, 2011.

\bibitem{jeh2003scaling}
G.~Jeh and J.~Widom, ``Scaling personalized web search,'' in \emph{Proceedings
  of the 12th International Conference on World Wide Web}.\hskip 1em plus 0.5em
  minus 0.4em\relax ACM, 2003, pp. 271--279.

\bibitem{chung2010pagerank}
F.~Chung and W.~Zhao, ``{P}age{R}ank and random walks on graphs,'' in
  \emph{Fete of Combinatorics and Computer Science}.\hskip 1em plus 0.5em minus
  0.4em\relax Springer, 2010, pp. 43--62.

\end{thebibliography}

% flatex input end: [pageRank-MAIN.bbl]
%FLATEX-REM:	\bibliographystyle{IEEEtran}
\else
%*flatex input: [pageRank-MAIN.bbl]
% Generated by IEEEtran.bst, version: 1.12 (2007/01/11)
\begin{thebibliography}{10}
\providecommand{\url}[1]{#1}
\csname url@samestyle\endcsname
\providecommand{\newblock}{\relax}
\providecommand{\bibinfo}[2]{#2}
\providecommand{\BIBentrySTDinterwordspacing}{\spaceskip=0pt\relax}
\providecommand{\BIBentryALTinterwordstretchfactor}{4}
\providecommand{\BIBentryALTinterwordspacing}{\spaceskip=\fontdimen2\font plus
\BIBentryALTinterwordstretchfactor\fontdimen3\font minus
  \fontdimen4\font\relax}
\providecommand{\BIBforeignlanguage}[2]{{%
\expandafter\ifx\csname l@#1\endcsname\relax
\typeout{** WARNING: IEEEtran.bst: No hyphenation pattern has been}%
\typeout{** loaded for the language `#1'. Using the pattern for}%
\typeout{** the default language instead.}%
\else
\language=\csname l@#1\endcsname
\fi
#2}}
\providecommand{\BIBdecl}{\relax}
\BIBdecl

\bibitem{page1999pagerank}
L.~Page, S.~Brin, R.~Motwani, and T.~Winograd, ``The {P}age{R}ank citation
  ranking: Bringing order to the web.'' Stanford InfoLab, Tech. Rep., 1999.

\bibitem{ilprints361}
S.~Brin and L.~Page, ``The anatomy of a large-scale hypertextual web search
  engine,'' \emph{Computer networks and ISDN systems}, vol.~30, no. 1-7, pp.
  107--117, 1998.

\bibitem{sheldon2010manipulation}
D.~Sheldon, ``Manipulation of {P}age{R}ank and {C}ollective {H}idden {M}arkov
  {M}odels,'' Ph.D. dissertation, Cornell University, Ithaca, NY, USA, 2010.

\bibitem{haveliwala2003topic}
T.~H. Haveliwala, ``Topic-sensitive {P}age{R}ank: A context-sensitive ranking
  algorithm for web search,'' \emph{IEEE Transactions on Knowledge and Data
  Engineering}, vol.~15, no.~4, pp. 784--796, 2003.

\bibitem{morrison2005generank}
J.~L. Morrison, R.~Breitling, D.~J. Higham, and D.~R. Gilbert, ``Generank:
  using search engine technology for the analysis of microarray experiments,''
  \emph{BMC bioinformatics}, vol.~6, no.~1, p. 233, 2005.

\bibitem{wu2010krylov}
G.~Wu, Y.~Zhang, and Y.~Wei, ``Krylov subspace algorithms for computing
  generank for the analysis of microarray data mining,'' \emph{Journal of
  Computational Biology}, vol.~17, no.~4, pp. 631--646, 2010.

\bibitem{langville2004deeper}
A.~N. Langville and C.~D. Meyer, ``Deeper inside {P}age{R}ank,'' \emph{Internet
  Mathematics}, vol.~1, no.~3, pp. 335--380, 2004.

\bibitem{berkhin2005survey}
P.~Berkhin, ``A survey on {P}age{R}ank computing,'' \emph{Internet
  Mathematics}, vol.~2, no.~1, pp. 73--120, 2005.

\bibitem{ilprints596}
T.~Haveliwala, S.~Kamvar, and G.~Jeh, ``An analytical comparison of approaches
  to personalizing {P}age{R}ank,'' Stanford InfoLab, Technical Report 2003-35,
  June 2003.

\bibitem{ilprints579}
S.~Kamvar, T.~Haveliwala, C.~Manning, and G.~Golub, ``Exploiting the block
  structure of the web for computing {P}age{R}ank,'' Stanford InfoLab,
  Technical Report 2003-17, 2003.

\bibitem{boldi2005pagerank}
P.~Boldi, M.~Santini, and S.~Vigna, ``{P}age{R}ank as a function of the damping
  factor,'' in \emph{Proceedings of the 14th International Conference on World
  Wide Web}.\hskip 1em plus 0.5em minus 0.4em\relax ACM, 2005, pp. 557--566.

\bibitem{bressan2010choose}
M.~Bressan and E.~Peserico, ``Choose the damping, choose the ranking?''
  \emph{Journal of Discrete Algorithms}, vol.~8, no.~2, pp. 199--213, 2010.

\bibitem{chung2007heat}
F.~Chung, ``The heat kernel as the {P}age{R}ank of a graph,'' \emph{Proceedings
  of the National Academy of Sciences}, vol. 104, no.~50, pp. 19\,735--19\,740,
  2007.

\bibitem{chung2009local}
------, ``A local graph partitioning algorithm using heat kernel
  {P}age{R}ank,'' \emph{Internet Mathematics}, vol.~6, no.~3, pp. 315--330,
  2009.

\bibitem{arasu2002pagerank}
A.~Arasu, J.~Novak, A.~Tomkins, and J.~Tomlin, ``{P}age{R}ank computation and
  the structure of the web: Experiments and algorithms,'' in \emph{Proceedings
  of the 11th International Conference on World Wide Web}, 2002, pp. 107--117.

\bibitem{kullback1951information}
S.~Kullback and R.~A. Leibler, ``On information and sufficiency,'' \emph{The
  Annals of Mathematical Statistics}, vol.~22, no.~1, pp. 79--86, 1951.

\bibitem{lee2003fast}
C.~P.-C. Lee, G.~H. Golub, and S.~A. Zenios, ``A fast two-stage algorithm for
  computing {P}age{R}ank and its extensions,'' \emph{Scientific Computation and
  Computational Mathematics}, vol.~1, no.~1, pp. 1--9, 2003.

\bibitem{langville2006reordering}
A.~N. Langville and C.~D. Meyer, ``A reordering for the {P}age{R}ank problem,''
  \emph{SIAM Journal on Scientific Computing}, vol.~27, no.~6, pp. 2112--2120,
  2006.

\bibitem{dulmage1958coverings}
A.~L. Dulmage and N.~S. Mendelsohn, ``Coverings of bipartite graphs,''
  \emph{Canadian Journal of Mathematics}, vol.~10, no.~4, pp. 516--534, 1958.

\bibitem{google-net}
Google, ``Google programming contest,''
  \url{http://www.google.com/programming-contest/}, 2002.

\bibitem{kamvar2003extrapolation}
S.~D. Kamvar, T.~H. Haveliwala, C.~D. Manning, and G.~H. Golub, ``Extrapolation
  methods for accelerating {P}age{R}ank computations,'' in \emph{Proceedings of
  the 12th International Conference on World Wide Web}.\hskip 1em plus 0.5em
  minus 0.4em\relax ACM, 2003, pp. 261--270.

\bibitem{kamvar2004adaptive}
S.~Kamvar, T.~Haveliwala, and G.~Golub, ``Adaptive methods for the computation
  of {P}age{R}ank,'' \emph{Linear Algebra and its Applications}, vol. 386, pp.
  51--65, 2004.

\bibitem{DBLP:journals/corr/LofgrenBGS14}
P.~A. Lofgren, S.~Banerjee, A.~Goel, and C.~Seshadhri, ``{FAST-PPR:} scaling
  personalized pagerank estimation for large graphs,'' in \emph{Proceedings of
  the 20th ACM SIGKDD International Conference on Knowledge Discovery and Data
  Mining}.\hskip 1em plus 0.5em minus 0.4em\relax ACM, 2014, pp. 1436--1445.

\bibitem{kunegis2013konect}
J.~Kunegis, ``{KONECT}--the koblenz network collection,'' in \emph{Proceedings
  of the 22nd International Conference on World Wide Web}.\hskip 1em plus 0.5em
  minus 0.4em\relax ACM, 2013, pp. 1343--1350.

\bibitem{auer2007dbpedia}
S.~Auer, C.~Bizer, G.~Kobilarov, J.~Lehmann, R.~Cyganiak, and Z.~Ives,
  ``{DB}pedia: A nucleus for a web of open data,'' in \emph{The Semantic
  Web}.\hskip 1em plus 0.5em minus 0.4em\relax Springer, 2007, pp. 722--735.

\bibitem{Kwak10www}
H.~Kwak, C.~Lee, H.~Park, and S.~Moon, ``{W}hat is {T}witter, a social network
  or a news media?'' in \emph{Proceedings of the 19th International Conference
  on World Wide Web}.\hskip 1em plus 0.5em minus 0.4em\relax New York, NY, USA:
  ACM, 2010, pp. 591--600.

\bibitem{icwsm10cha}
M.~Cha, H.~Haddadi, F.~Benevenuto, and K.~P. Gummadi, ``{Measuring User
  Influence in Twitter: The Million Follower Fallacy},'' in \emph{Proceedings
  of the 4th International AAAI Conference on Weblogs and Social Media
  (ICWSM)}, Washington DC, USA, May 2010.

\bibitem{yang2015defining}
J.~Yang and J.~Leskovec, ``Defining and evaluating network communities based on
  ground-truth,'' \emph{Knowledge and Information Systems}, vol.~42, no.~1, pp.
  181--213, 2015.

\bibitem{MS-qian-2018}
Y.~Qian, ``Variable damping effect on network propagation,'' Master's thesis,
  Duke University, Durham, NC, USA, May 2018.

\bibitem{MS-Xichen-2018}
X.~Chen, ``Exploiting common structures across multiple network propagation
  schemes,'' Master's thesis, Duke University, Durham, NC, USA, May 2018.

\bibitem{ilprints582}
T.~Haveliwala and S.~Kamvar, ``The second eigenvalue of the google matrix,''
  Stanford InfoLab, Technical Report 2003-20, 2003.

\bibitem{richardson2002intelligent}
M.~Richardson and P.~Domingos, ``The intelligent surfer: Probabilistic
  combination of link and content information in {P}age{R}ank,'' in
  \emph{Advances in Neural Information Processing Systems}, 2002, pp.
  1441--1448.

\bibitem{haveliwala1999efficient}
T.~Haveliwala, ``Efficient computation of {P}age{R}ank,'' Stanford, Tech. Rep.,
  1999.

\bibitem{langville2011google}
A.~N. Langville and C.~D. Meyer, \emph{Google's {P}age{R}ank and beyond: The
  science of search engine rankings}.\hskip 1em plus 0.5em minus 0.4em\relax
  Princeton University Press, 2011.

\bibitem{jeh2003scaling}
G.~Jeh and J.~Widom, ``Scaling personalized web search,'' in \emph{Proceedings
  of the 12th International Conference on World Wide Web}.\hskip 1em plus 0.5em
  minus 0.4em\relax ACM, 2003, pp. 271--279.

\bibitem{chung2010pagerank}
F.~Chung and W.~Zhao, ``{P}age{R}ank and random walks on graphs,'' in
  \emph{Fete of Combinatorics and Computer Science}.\hskip 1em plus 0.5em minus
  0.4em\relax Springer, 2010, pp. 43--62.

\end{thebibliography}

% flatex input end: [pageRank-MAIN.bbl]
%FLATEX-REM:	\bibliographystyle{abbrv}
\fi
%FLATEX-REM:\bibliography{pageRank-refs.bib}

\end{document}



%%% Local Variables: 
%%% mode: latex
%%% TeX-master: t
%%% End: 

% flatex input end: [pageRank-MAIN.tex]
