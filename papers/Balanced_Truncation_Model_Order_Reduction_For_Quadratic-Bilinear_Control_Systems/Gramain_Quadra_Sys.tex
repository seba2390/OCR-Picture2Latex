This section is devoted to determine  algebraic Gramians for QB systems, which are also related to the energy functionals of the quadratic-bilinear systems as welI. Let us consider  QB systems of the form
\begin{subequations}\label{eq:Quad_bilin_Sys}
\begin{align}
 \dot{x}(t) &= Ax(t) + H~x(t)\otimes x(t) + \sum_{k = 1}^mN_kx(t)u_k(t) +  Bu(t),\label{eq:QB_differential}\\
 y(t) &= Cx(t),\quad x(0) = 0,\label{eq:QB_output}
 \end{align}
\end{subequations}
where $A,N_k\in \Rnn, H\in\R^{n\times n^2}, B \in \Rnm$ and $C\in \Rpn$. Furthermore, $x(t) \in \Rn$, $u(t)\in \Rm$ and $y(t) \in \Rp$ denote the state, input and output vectors of the system, respectively. Since the system~\eqref{eq:Quad_bilin_Sys} has a quadratic nonlinearity in the state vector $x(t)$ and also includes  bilinear terms $N_kx(t)u_k(t)$, which are products of the state vector and inputs, the system is called a quadratic-bilinear (QB) system. 
%\sout{For simplicity of notation, we consider  single-input single-output system (SISO) QB systems to derive algebraic Gramians; however, analogously these can be extended to multi-input multi-output systems. We denote $N_1$ by $N$ in the SISO case.} 
We begin by deriving  the reachability Gramian of  the QB system  and its connection with a certain type of  quadratic Lyapunov equation.

%%%%%%%%%%%%%%%%%%%%%%%%%%%%%%%%%%%%%%%%%%%%%%%%%%%%%%%%%%%%%%%%%%%%%%%%%%%%%%%%%%%%%%%%%%%%%%%%%%%%%%%%%%
%%%%%%%%%%%%%%%%%%%%%%%%%%   reachability GRAMIAN    %%%%%%%%%%%%%%%%%%%%%%%%%%%%%%%%%%%%%%%%%%%%%%%%%%
%%%%%%%%%%%%%%%%%%%%%%%%%%%%%%%%%%%%%%%%%%%%%%%%%%%%%%%%%%%%%%%%%%%%%%%%%%%%%%%%%%%%%%%%%%%%%%%%%%%%%%%%%%

\subsection{Reachability Gramian for QB systems}
In order to derive the reachability Gramian, we first formulate the Volterra series for the QB system~\eqref{eq:Quad_bilin_Sys}. Before we proceed further, for ease we define the following short-hand notation:
\begin{equation*}
u^{(k)}_{\sigma_1,\ldots ,\sigma_l}(t) := u_k(t-\sigma_1 \cdots -\sigma_l)\quad\mbox{and}\quad x_{\sigma_1,\ldots ,\sigma_l}(t) := x(t-\sigma_1 \cdots -\sigma_l).
\end{equation*}
We integrate  both sides of the differential equation~\eqref{eq:QB_differential} in the state variables with respect to time  to obtain
\begin{multline}\label{eq:first_int}
x(t) = \int\nolimits_0^te^{A\sigma_1}Bu_{\sigma_1}(t)d\sigma_1 +  \sum_{k=1}^m \int\nolimits_0^te^{A\sigma_1} N_kx_{\sigma_1}(t)u^{(k)}_{\sigma_1}(t)d\sigma_1 \\ +\int\nolimits_0^te^{A\sigma_1}H\left(x_{\sigma_1}(t)\otimes x_{\sigma_1}(t)\right)d\sigma_1.
\end{multline}
Based on the above equation, we obtain an expression for $x_{\sigma_1}(t)$ as follows:
\begin{multline*}
x_{\sigma_1}(t) = \int\limits_0^{t-\sigma_1}e^{A\sigma_2}Bu_{\sigma_1,\sigma_2}(t)d\sigma_2 +\sum_{k=1}^m \int\limits_0^{t-\sigma_1}e^{A\sigma_2}N_kx_{\sigma_1,\sigma_2}(t)u^{(k)}_{\sigma_1,\sigma_2}(t)d\sigma_2 \\
 + \int\limits_0^{t-\sigma_1}e^{A\sigma_2}H\left(x_{\sigma_1,\sigma_2}(t)\otimes x_{\sigma_1,\sigma_2}(t)\right)d\sigma_2
\end{multline*}
and substitute it in~\eqref{eq:first_int} to have
\begin{equation*}
\begin{aligned}
x(t) &= \int\limits_{0}^t {e^{A\sigma_1}B}u_{\sigma_1}(t)d\sigma_1 +\sum_{k=1}^m\int\limits_{0}^t\int\limits_{0}^{t-\sigma_1} {e^{A\sigma_1}N_ke^{A\sigma_2} B} u^{(k)}_{\sigma_1}(t)u^{}_{\sigma_1,\sigma_2}(t)d\sigma_1d\sigma_2 \\
 &\quad+ \int\limits_{0}^t\int\limits_{0}^{t-\sigma_1}\int\limits_{0}^{t-\sigma_1} {e^{A\sigma_1} H (e^{A\sigma_2}B\otimes e^{A\sigma_3}B)}\left(u_{\sigma_1,\sigma_2}(t) \otimes u_{\sigma_1,\sigma_3}(t)\right)d\sigma_1d\sigma_2d\sigma_3 
% &\qquad+ \int\limits_0^t\int\limits_0^{t-\sigma_1}e^{A\sigma_1}N e^{A\sigma_2}\Big[Nx_{\sigma_1,\sigma_2}u_{\sigma_1,\sigma_2} + H\left(x_{\sigma_1,\sigma_2}\otimes x_{\sigma_1,\sigma_2} \right)\Big] d\sigma_2d\sigma_1 +
  +\cdots.
 \end{aligned}
\end{equation*}
Repeating this process by repeatedly substituting  for the state yields the Volterra series for the QB system~\cite{sastry2013nonlinear}.  Having carefully  analyzed the \emph{kernels} of the Volterra series for the system, we define the reachability  mapping $\bar{P}$  as follows:
\begin{equation}\label{eq:Cont_Mapping}
 \bar{P} = [\bar P_1,~ \bar P_2,~\bar P_3,\dots],
\end{equation}
where the $\bar P_i$'s are:
\begin{equation}\label{eq:defined_barP}
\begin{aligned}
 \bar P_1(t_1)&= e^{At_1}B,\\
  \bar P_2(t_1,t_2)&= e^{At_2}\bbm N_1,\ldots,N_m\ebm \left(I_m\otimes \bar P_1(t_1)\right),\\
%    \bar P_3(t_1,t_2,t_3)&= e^{At_3}[H\bar P_1(t_1)\otimes \bar P_1(t_2), \bbm N_1,\ldots,N_m\ebm
 % \left(  I_m \otimes \bar P_2(t_1,t_2)\right)],\\
    \vdots\qquad&\qquad\qquad\vdots\\
    \bar P_i (t_1,\ldots, t_i)&= e^{At_i}\Big[H \big[\bar P_1(t_1)\otimes \bar P_{i-2}(t_{2},\ldots,t_{i-1}),\bar P_2(t_1,t_2)\otimes \bar P_{i-3}(t_{3},\ldots,t_{i-1}),\\
    &\qquad \ldots, \bar P_{i-2}(t_1,\ldots,t_{i-2})\otimes \bar P_1(t_{i-1})\big],\\
    &  \qquad \bbm N_1,\ldots,N_m\ebm\left(I_m\otimes \bar P_{i-1}(t_1,\ldots,t_{i-1})\right)\Big],\forall~i\geq 3.\\
    % \bar P_i (t_1,\ldots, t_i)&= e^{At_i}\Big[H\sum_{\substack{f\geq 1,g\geq 1,\\f+g = i-1}} P_f(t_1,\ldots t_f)\otimes P_g(t_{f+1},\ldots,t_{i-1}) , \\ 
    % &\qquad \qquad\bbm N_1,\ldots,N_m\ebm\left(I_m\otimes \bar P_{i-1}(t_1,\ldots,t_{i-1})\right)\Big],\forall~i\geq 3.
\end{aligned}
\end{equation}
Using the mapping $\bar P$~\eqref{eq:Cont_Mapping}, we  define the reachability Gramian $P$ as
\begin{equation}\label{eq:Cont_Gram}
 P = \sum_{i=1}^{\infty}P_i\qquad \text{with} \qquad P_i = \int\limits_0^{\infty}\cdots \int\limits_0^{\infty} \bar P_i(t_1,\ldots,t_i)\bar P_i^T(t_1,\ldots,t_i)dt_1\cdots dt_i.
\end{equation}

In what follows, we show the equivalence between the above proposed reachability Gramian and the solution of a certain type of quadratic Lyapunov equation. %For simplicity, we drop the integral limits and represent $i$ integrals by only one integral in the rest of the paper.
\begin{theorem}\label{thm:con_gram}
 Consider the QB system~\eqref{eq:Quad_bilin_Sys} with a stable matrix $A$. If the reachability Gramian $P$ of the system defined as in~\eqref{eq:Cont_Gram} exists, then the Gramian $P$ satisfies the generalized quadratic  Lyapunov equation, given by
 \begin{equation}\label{eq:cont_lyap}
 AP + PA^T + H(P\otimes P) H^T +  \sum_{k=1}^mN_kPN_k^T + BB^T = 0.
 \end{equation}
\end{theorem}
\begin{proof}
We begin by considering the first term in the summation~\eqref{eq:Cont_Gram}. This is,
\begin{equation*}
P_1 = \int_0^\infty\bar P_1 \bar P_1^T dt_1 =  \int_0^\infty e^{At_1}BB^T e^{A^Tt_1}dt_1.
\end{equation*}
As shown, e.g., in \cite{morAnt05}, $P_1$ satisfies the following Lyapunov equation, provided $A$ is stable:
\begin{equation}\label{eq:1c}
AP_1 + P_1A^T + BB^T = 0.
\end{equation}
Next, we consider  the second term in the summation~\eqref{eq:Cont_Gram}:
\begin{align*}
P_2 &=  \int_0^\infty  \int_0^\infty \bar P_2 \bar P_2^T dt_1dt_2 \\
&=   \int\limits_0^\infty \int\limits_0^\infty e^{At_2}\bbm N_1,\ldots,N_m\ebm \left(I_m\otimes \left(e^{At_1}BB^Te^{A^Tt_1}\right)\right) \bbm N_1,\ldots N_m\ebm^T e^{A^Tt_2}dt_1dt_2\\
&=  \sum_{k=1}^m \int_0^\infty   e^{At_2}N_k \Big( \int_0^\infty e^{At_1}BB^Te^{A^Tt_1}dt_1\Big) N_k^T e^{A^Tt_2}dt_1dt_2\\
%&=   \int_0^\infty e^{At_2}N\Big( \int_0^\infty e^{At_1}BB^Te^{A^Tt_1}dt_1\Big) N^T e^{A^Tt_2}dt_2
&= \sum_{k=1}^m  \int_0^\infty e^{At_2}N_kP_1 N_k^T e^{A^Tt_2}dt_2.
\end{align*}
Again using the integral representation of the solution to Lyapunov equations \cite{morAnt05}, we see that $P_2$ is the solution of the following Lyapunov equation:
\begin{equation}\label{eq:2c}
AP_2 + P_2A^T + \sum_{k=1}^mN_kP_1N_k^T = 0.
\end{equation}
Finally, we consider the $i$th term, for $i\geq 3$, which is
\begin{align*}
P_i &= \int_0^\infty\cdots \int_0^\infty \bar P_i \bar P_i^T dt_1\cdots dt_i\\
&=  \int\limits_0^\infty e^{At_i}\left[H \left[\int\limits_0^\infty\cF\left(\bar P_1(t_1)\right) dt_1\otimes\int\limits_0^\infty\cdots \int\limits_0^\infty \cF\left(\bar P_{i-2}(t_{2},\ldots,t_{i-1})\right) dt_2\cdots dt_{i-1}  \right.\right.\\
&\quad\left.\left.+\cdots + \int\limits_0^\infty\cdots \int\limits_0^\infty \cF\left(\bar P_{i-2}(t_{1},\ldots,t_{i-2}) \right) dt_1\cdots dt_{i-2}\otimes  \int\limits_0^\infty  \cF\left(\bar P_1(t_{i-1}) \right)dt_{i-1}\right]\right.H^T \\
&\quad \left.+  \sum_{k=1}^m N_k \left(\int_0^\infty\cdots \int_0^\infty \cF\left(\bar P_{i-1}(t_1,\ldots,t_{i-1})\right) \right)N_k^T\right]e^{A^Tt_i}dt_i,
\end{align*}
where  we use the shorthand  $\cF(\cA) := \cA\cA^T$. Thus, we have
\begin{align*}
P_i &=  \int_0^\infty e^{At_i}\Big[H(P_1\otimes P_{i-2} + \cdots + P_{i-2}\otimes P_1)H^T + \sum_{k=1}^mN_kP_{i-1}N_k^T\Big]e^{A^Tt_i}dt_i.
\end{align*}
Similar to $P_1$ and $P_2$, we can show that $P_i$ satisfies the following Lyapunov equation, given in terms of the preceding $P_k$, for $k = {1,\ldots,i-1}$:
\begin{equation}\label{eq:3c}
AP_i+P_iA^T + H(P_1\otimes P_{i-2} + \cdots + P_{i-2}\otimes P_1)H^T + \sum_{k=1}^mN_kP_{i-1}N_k^T = 0.
\end{equation}
To the end, adding \eqref{eq:1c}, \eqref{eq:2c} and \eqref{eq:3c} yields
\begin{equation*}
A\left.\sum_{i=1}^\infty P_i\right. + \left.\sum_{i=1}^\infty P_i\right.A^T + H\left(\sum_{i=1}^\infty P_i \otimes \sum_{i=1}^\infty P_i \right)H^T + \sum_{k=1}^m N_k\left(\sum_{i=1}^\infty P_i\right) N_k^T +BB^T = 0.
\end{equation*}
This implies that $P = \sum_{i=1}^\infty P_i$ solves the generalized quadratic  Lyapunov equation given by~\eqref{eq:cont_lyap}.
\end{proof}

\subsection{Dual system and observability Gramian for QB system}
We first derive the dual system for the QB system; the dual system plays an important role in determining the observability Gramian for the QB system~\eqref{eq:Quad_bilin_Sys}, and we aim at determining the observability Gramian in a similar fashion as done for the reachability Gramian in the preceding subsection. From linear and bilinear systems, we know that the observability Gramian of the dual system is the same as the reachability Gramian; here, we also consider the same analogy. If we compare the system~\eqref{eq:Quad_bilin_Sys} with the general nonlinear system as shown in~\eqref{eq:Gen_Nonlinear},  it turns out that for the system~\eqref{eq:Quad_bilin_Sys}
\begin{align*}
 \cA(x,u,t) &= A + H(x\otimes I) + \sum_{k=1}^mN_ku_k,\quad  \cB(x,u,t) = B~~\text{and}~~ \cC(x,u,t) = C.
\end{align*}
Using \Cref{lemma:adjointsys}, we can write down the state-space realization of the adjoint operator of the QB system as follows:
\begin{subequations}\label{eq:Adjoint_QBDAE}
 \begin{align}
  \dot{x}(t) &= Ax(t) + H(x(t) \otimes x(t))  + \sum_{k=1}^mN_kx(t)u_k(t) + Bu(t), \quad x(0) = 0,\\
  \dot{z}(t) &= -A^Tz(t) - ( x(t)^T\otimes I)H^T z(t) - \sum_{k=1}^mN_k^T z(t)u_k(t) - C^Tu_d(t), \\ &\hspace{9cm}z(\infty) = 0,\nonumber\\
  y_d(t) &= B^Tz(t),
 \end{align}
\end{subequations}
where $z(t)\in \Rn, u_d(t) \in \R$ and $y_d \in \R$ can be interpreted as the dual state, dual input and dual output vectors of the system, respectively. Next, we attempt to utilize the existing knowledge for the tensor multiplications and matricization  to simplify the term $(x(t)^T\otimes I)H^T z(t)$ in the system~\eqref{eq:Adjoint_QBDAE} and to write it in  the form of $x(t)\otimes z(t)$.

For this, we review some of the basic properties of tensor theory.  Following~\cite{koldatensor09}, the \emph{fiber} of a 3-dimensional tensor $\cH$ can be defined by fixing each index except one, e.g., $\cH(:,j,k)$,$\cH(j,:,k)$ and $\cH(j,k,:)$.  From the computational  point of view, it is advantageous to consider the matrices associated with the tensor, which can be obtained via unfolding a tensor into a matrix.    The process of unfolding a tensor into a matrix is called \emph{matricization}, and  the mode-$\mu$ matricization of the tensor $\cH$ is denoted by $\cH^{(\mu)}$. For an $l$-dimensional tensor, there are $l$ different possible ways to unfold the tensor into a matrix. We refer to \cite{morBenB15,koldatensor09} for more detailed insights into matricization. Similar to matrix multiplications, one can carry out tensor multiplication using matricization of the tensor~\cite{koldatensor09}. For instance, the mode-$\mu$ product of $\cH$ and a matrix $X\in\R^{n\times s}$ gives a tensor $\cF \in\R^{s\times n\times n}$, satisfying
\begin{equation*}
 \cF = \cH\times_\mu X\quad  \Leftrightarrow   \quad \cF^{(\mu)} = X\cH^{(\mu)}.
\end{equation*}
Analogously, if we define a tensor-matrices product as:
\begin{equation*}
 \cF = \cH \times_1 X\times_2 Y\times_3 Z,
\end{equation*}
where $\cF\in \R^{q_1\times q_2\times q_3}$, $X\in \R^{n\times q_1}$ and $Y\in \R^{n\times q_2}$ and $Z \in \R^{n\times q_3}$, then the following relations are fulfilled:
\begin{subequations}\label{eq:tensor_matricization}
\begin{align}
\cF^{(1)} &= X^T\cH^{(1)}(Y\otimes Z ), \\
\cF^{(2)} &= Z^T\cH^{(2)}(Y\otimes X ) ,\\
\cF^{(3)} &= Y^T\cH^{(3)}(Z \otimes X ) .
\end{align}
\end{subequations}

Coming back to the QB system, the matrix $H\in \R^{n\times n^2}$ in the system denotes a Hessian, which can be seen as an unfolding of a 3-dimensional tensor $\cH\in \R^{n\times n\times n}$.   Here, we choose the tensor $\cH\in \R^{n\times n\times n}$ such that its mode-1 matricization is the same as the Hessian $H$, i.e., $H = \cH^{(1)}$. Next, let us consider a tensor $\cT\in \R^{1\times n\times 1}$, whose mode-1 matricization $\cT^{(1)}$ is given by
\begin{equation*}
 \cT^{(1)}  = z(t)^T H (x{(t)\otimes I}) = z(t)^T \cH^{(1)} (x{(t)\otimes I}).
\end{equation*}
We then observe that the mode-1 matricization of the tensor $\cT$ is a transpose of the mode-2 matricization, i.e., $\cT^{(1)} = \left(\cT^{(2)}\right)^T$, leading to
\begin{equation*}
 \cT^{(1)} = \left(\cT^{(2)}\right)^T = \left(x(t)\otimes z(t)\right)^T (\cH^{(2)})^T.
\end{equation*}
Therefore, we can rewrite the system~\eqref{eq:Adjoint_QBDAE} as:
\begin{subequations}\label{eq:Adjoint_QBDAE1}
 \begin{align}
  \dot{x}(t) &= Ax(t) + H(x(t) \otimes x(t))  +  \sum_{k=1}^mN_kx(t)u_k(t) + Bu(t), \qquad~~~~~ x(0) = 0,\label{eq:state1}\\
  \dot{z}(t) &= -A^Tz(t) - \cH^{(2)} x(t)\otimes z(t) - \sum_{k=1}^m N_k^T u_k(t) z(t) - C^Tu_d(t),~ z(\infty) = 0,\\
  y_d(t) &= B^Tz(t).
 \end{align}
\end{subequations}
%\begin{remark}
In the meantime, we like to point out that there are two possibilities to define $\cA(x,u,t)$ in the case of a QB system. One is  $\cA(x,u,t) = A + H(x\otimes I) + \sum_{k=1}^mN_ku_k$, which we have used in the above discussion; however, there is  another possibility to define  $\cA(x,u,t)$  as  $\tilde{\cA}(x,u,t)= A + H(I\otimes x) + \sum_{k=1}^mN_ku_k$, leading to the nonlinear Hilbert adjoint operator whose  state-space realization is given as:
\begin{subequations}\label{eq:Adjoint_QBDAE2}
	\begin{align}
	\dot{x}(t) &= Ax(t) + H(x(t) \otimes x(t))  +  \sum_{k=1}^mN_kx(t)u_k(t) + Bu(t), \qquad~~~~~ x(0) = 0,\label{eq:state1}\\
	\dot{z}(t) &= -A^Tz(t) - \cH^{(3)} x(t)\otimes z(t) - \sum_{k=1}^m N_k^T u_k(t) z(t) - C^Tu_d(t),~ z(\infty) = 0,\\
	y_d(t) &= B^Tz(t).
	\end{align}
\end{subequations}
It can be noticed that  the realizations~\eqref{eq:Adjoint_QBDAE1} and~\eqref{eq:Adjoint_QBDAE2} are the same, except the appearance of $\cH^{(2)}$ in~\eqref{eq:Adjoint_QBDAE1} instead of $\cH^{(3)}$ in~\eqref{eq:Adjoint_QBDAE2}. Nonetheless, if one assumes that the Hessian $H$ is symmetric,  i.e., $H(u\otimes v) = H(v\otimes u)$ for $u,v\in \Rn$, then the mode-2 and mode-3 matricizations coincide, i.e., $\cH^{(2)} = \cH^{(3)}$. However, the Hessian $H$, obtained after discretization of the governing equations,  may not be symmetric; but as shown in~\cite{morBenB15} the Hessian can be modified in such a way that it becomes symmetric without any change in the system dynamics. Therefore, in the rest of the paper,    without loss of generality, we assume that the Hessian $H$ is symmetric.

Now, we turn our attention towards determining the observability Gramian for the QB system by utilizing the state-space realization of the Hilbert adjoint operator (dual system). For this, we follow the same steps as used for determining the reachability Gramian. Using the dual system~\eqref{eq:Adjoint_QBDAE1}, one can write the dual state $z(t)$ of the dual system at time $t$ as follows:
 \begin{equation*}
\begin{aligned}
 z(t) &= \int_{\infty}^t e^{-A^T(t-\sigma_1)}C^T u_d(\sigma_1)d\sigma_1 +  \sum_{k=1}^m \int_{\infty}^t e^{-A^T(t-\sigma_1)}N_k^T z(\sigma_1)u_k(\sigma_1)d\sigma_1, \\
 &\quad + \int_{\infty}^t e^{-A^T(t-\sigma_1)}\cH^{(2)}\left(x(\sigma_1)\otimes z(\sigma_1)\right)d\sigma_1,
 %&= \int\limits^{\infty}_t e^{A^T\sigma_1}C^T u_a(t-\sigma_1)d\sigma_1 +  \int\limits^{\infty}_t e^{A^T\sigma_1}\cH^{(2)}(z(t-\sigma_1)\otimes x(t-\sigma_1))d\sigma_1
\end{aligned}
\end{equation*}
which after an appropriate change of variable leads to
%We next make change of variable as $\tilde  \sigma_1   = \sigma_1-t $, which yields
\begin{equation}\label{eq:dual_newvar}
\begin{aligned}
z(t)  &= \int_{\infty}^0 e^{A^T \sigma_1}C^T u^{(d)}(t+\sigma_1)d\sigma_1 + \sum_{k=1}^m \int_{\infty}^0 e^{A^T\sigma_1}N_k^T z(t+\sigma_1)u_k(t+\sigma_1)d\sigma_1 \\
 &\quad + \int_{\infty}^0 e^{A^T\sigma_1}\cH^{(2)}\big(x(t+\sigma_1)\otimes z(t+\sigma_1)\big)d\sigma_1.
 \end{aligned}
\end{equation}
\Cref{eq:state1} gives the expression for $x(t+\sigma_1)$. This is
\begin{equation*}
\begin{aligned}
  x(t+\sigma_1) &= \int_0^{t+\sigma_1} e^{A\sigma_2}Bu(t+\sigma_1-\sigma_2)d\sigma_2 + \sum_{k=1}^m\int_0^{t+\sigma_1} \Big( e^{A\sigma_2}N_kx(t+\sigma_1-\sigma_2)  \\ 
  &   \times  u_k(t+\sigma_1-\sigma_2) \Big)d\sigma_2 + \int\limits_0^{t+\sigma_1} e^{A\sigma_2}H(x(t+\sigma_1-\sigma_2)\otimes x(t+\sigma_1-\sigma_2))d\sigma_2.
  \end{aligned}
\end{equation*}
We substitute for $x(t+\sigma_1)$ using the above equation, and $z(t+\sigma_1)$ using \eqref{eq:dual_newvar}, which gives rise to the following expression:
\begin{equation}\label{eq:sys_dual}
 \begin{aligned}
   z(t) &= \int_{\infty}^0 e^{A^T\sigma_1}C^T u_d(t+\sigma_1)d\sigma_1   +\sum_{k=1}^m  \int_{\infty}^0\int_{\infty}^0 e^{A^T\sigma_1}N_k^T \\ 
   &\quad \times e^{A^T\sigma_2}C^T u_d(t+\sigma_1+\sigma_2)u_k(t+\sigma_1)d\sigma_1d\sigma_2 + \int_{\infty}^0\int_0^{t+\sigma_1}\int_{\infty}^{0} e^{A^T\sigma_1} \\ 
   &\quad \times \cH^{(2)}\Big( e^{A\sigma_2}B\otimes e^{A^T\sigma_3}C^T \Big) u(t+\sigma_1-\sigma_2)u_d(t+\sigma_1+\sigma_3)d\sigma_1d\sigma_2d\sigma_3  + \cdots.
 \end{aligned}
\end{equation}
By repeatedly substituting  for the state $x$ and the dual state $z$, we derive the Volterra series for the dual system,
%This way, the Volterra series for the dual system  can be obtained by substituting repeatedly for the state $x$ and the dual state $z$, 
although the notation becomes much more complicated.  Carefully inspecting the kernels of the Volterra series of the dual system, we define the observability mapping $\bar Q$, similar to the  reachability mapping, as follows:
\begin{equation}\label{eq:obser_mapping}
 \bar{Q} = [ \bar Q_1,~\bar Q_2,~ \bar Q_3,\ldots],
\end{equation}
in which
\begin{align*}
 \bar Q_1(t_1) &=  e^{A^Tt_1}C^T, \\
 \bar Q_2(t_1,t_2) &=  e^{A^Tt_2}\bbm N^T_1,\cdots N_m^T\ebm \left(I_m\otimes \bar Q_1(t_1)\right),\\
% \bar Q_3(t_1,t_2,t_3) &=  e^{A^Tt_3}\Big[\cH^{(2)}\Big(\bar P_1(t_1) \otimes  \bar Q_1(t_2) \Big), N^T \bar Q_2(t_1,t_2)\Big],\\
 \vdots\qquad&\qquad\quad\vdots\nonumber\\
  \bar Q_i (t_1,\ldots, t_i)&= e^{A^Tt_i}\Big[\cH^{(2)} \big[\bar P_1(t_1)\otimes \bar Q_{i-2}(t_{2},\ldots,t_{i-1}), \nonumber \\ % \bar P_2(t_1,t_2)\otimes \bar Q_{i-3}(t_{3},\ldots,t_{i-1}),\nonumber\\
     &\qquad \qquad \ldots, \bar P_{i-2}(t_{1},\ldots,t_{i-2})\otimes \bar Q_1(t_{i-1})\big], \\%\bbm N^T_1,\cdots,N_m^T \ebm \left(I_m\otimes \bar Q_{i-1}(t_1,\ldots,t_{i-1})\right)\Big],\forall~i\geq 3,\\
  &\qquad \qquad\bbm N^T_1,\ldots,N^T_m\ebm\left(I_m\otimes \bar Q_{i-1}(t_1,\ldots,t_{i-1})\right)\Big],\forall~i\geq 3.
\end{align*}
where $\bar P_i(t_1,\ldots,t_i)$ are defined in \eqref{eq:defined_barP}. Based on the above observability mapping, we define the observability Gramian $Q$ of the QB system as
\begin{equation}\label{eq:Obser_Gram}
 Q = \sum_{i=1}^{\infty}Q_i\quad \mbox{with}\quad Q_i = \int_0^\infty\cdots \int_0^\infty \bar Q_i\bar Q_i^Tdt_1\cdots dt_i.
\end{equation}
Analogous to the reachability Gramian, we next show a relation between the observability Gramian and the solution of  a generalized  Lyapunov equation.
\begin{theorem}\label{thm:obser_gram}
 Consider the QB system~\eqref{eq:Quad_bilin_Sys}  with a stable matrix $A$, and let $Q$, defined in~\eqref{eq:Obser_Gram},  be the observability Gramian of the system and assume it exists.  Then, the Gramian $Q$ satisfies the following  Lyapunov equation:
 \begin{equation}\label{eq:obser_lyap}
 A^TQ + QA  +   \cH^{(2)}(P\otimes Q) (\cH^{(2)})^T + \sum_{k=1}^m N_k^TQN_k +C^TC = 0,
 \end{equation}
 where $P$ is the reachability Gramian of the system, i.e., the solution of the generalized quadratic Lyapunov equation~\eqref{eq:Cont_Gram}.
%   \begin{equation*}
%   AP + PA^T + NPN^T + H(P\otimes P) H^T + BB^T = 0.
%   \end{equation*}
\end{theorem}
\begin{proof}
The proof of the above theorem is analogous to the proof of \Cref{thm:con_gram}; therefore, we skip it for the brevity of the paper.
\end{proof}

\begin{remark}
As one would expect, the Gramians for QB systems reduce to the Gramians for bilinear systems~\cite{morBenD11} if the quadratic term is zero, i.e., $H = 0$.
\end{remark}

%At this point, we see the demand to solve generalized quadratic Lyapunov equations to determine reachability  and observability Gramians.  We discuss their computational issues in Section 5.
Furthermore, it will also be  interesting to look at a truncated version of the Gramians of the QB system based on the leading kernels of the Volterra series. We call a truncated version of the Gramians  \emph{truncated} Gramians of QB systems.  For this, let us consider  approximate reachability and observability mappings as follows:
\begin{equation*}
\tilde P_\cT = \bbm \tilde P_1,\tilde  P_2,\tilde P_3\ebm,\qquad \tilde Q_\cT = \bbm \tilde Q_1, \tilde Q_2,\tilde Q_3\ebm,
\end{equation*}
where
\begin{align*}
 \tP_1(t_1) &=  e^{At_1}B, & \tQ_1(t_1)&= e^{A^Tt_1}C^T,\\
\tP_2(t_1,t_2) &=  e^{At_2}\bbm N_1,\ldots,N_m\ebm \left(I_m\otimes \tP_1(t_1)\right), \\ 
 \tQ_2(t_1,t_2)&= e^{A^Tt_2}\bbm N_1^T,\ldots,N_m^T\ebm \left(I_m\otimes \tQ_1(t_1)\right),\\
\tP_3(t_1,t_2,t_3)&= e^{At_3}H (\tP_1(t_1)\otimes \tP_1(t_2)),\\
  \tQ_3(t_1,t_2,t_3)&= e^{A^Tt_3}\cH^{(2)}(\tP_1(t_1)\otimes \tQ_1(t_2)).
\end{align*}
Then, one can define the truncated reachability and observability Gramians in the similar fashion as the Gramians of the system:
\begin{subequations}\label{eq:trun_con}
	\begin{align}
P_\cT & = \sum_{i=1}^3 \hP_i,\quad\text{where}\quad \hP_i =  \int_0^\infty \tP_i(t_1,\ldots,t_i)\tP_i^T(t_1,\ldots,t_i)dt_1\cdots dt_i, \label{eq:tru_con}\\
 Q_\cT &= \sum_{i=1}^3\hQ_i,\quad \text{where}\quad \hQ_i = \int_0^\infty \tQ_i(t_1,\ldots,t_i)\tQ_i^T(t_1,\ldots,t_i)dt_1\cdots dt_i,\label{eq:tru_obs}
\end{align}
\end{subequations}
respectively. Similar to the Gramians $P$ and $Q$,  in the following we derive the relation between these truncated Gramians and the solutions of the Lyapunov equations.
\begin{corollary}\label{coro:tru_gram}
Let  $P_\cT$ and $Q_\cT$ be the truncated Gramians of the QB system as defined in~\eqref{eq:trun_con}. Then, $P_\cT$ and $Q_\cT$ satisfy the following Lyapunov equations:
\begin{subequations}\label{eq:tru_gramians}
\begin{align}
AP_\cT+P_\cT A^T + H(\hP_1\otimes \hP_1)H^T  + \sum_{k=1}^mN_k\hP_1N_k^T + BB^T  &=0,\quad \mbox{and}\\
A^TQ_\cT+Q_\cT A +  \cH^{(2)}(\hP_1\otimes \hQ_1)(\cH^{(2)})^T+ \sum_{k=1}^m N_k^T\hQ_1N_k + C^TC &=0, \label{eq:tru_obs_Gram}
\end{align}
\end{subequations}
respectively, where $P_1$ and $Q_1$ are  solutions to the following Lyapunov equations:
\begin{align}
A\hP_1+\hP_1 A^T + BB^T &=0,\quad\mbox{and}\label{eq:linear_CG}\\
A^T\hQ_1+\hQ_1 A + C^TC &=0,\quad\text{respectively.}\label{eq:linear_OG}
\end{align}
\end{corollary}
	\begin{proof}
		We begin by showing the relation between the truncated reachability Gramian $P_\cT$ and the solution of the Lyapunov equation. First, note that the first two terms of the reachability Gramian $P$~\eqref{eq:tru_con} and the truncated reachability Gramian $P_\cT$~\eqref{eq:Cont_Gram} are the same, i.e., $\hP_1 = P_1$ and $\hP_2 = P_2$, and $\hP_1$ and $\hP_2$ are the unique solutions of the following Lyapunov equations for a stable matrix $A$:
		\begin{align}
		A\hP_1 + \hP_1A^T +BB^T &=0, \quad\text{and} \label{eq:tru_P1}\\
		A\hP_2 + \hP_2A^T + \sum_{k=1}^mN_k\hP_1N_k^T = 0.\label{eq:tru_P2}
		\end{align}
		Now, we consider the third term in the summation~\eqref{eq:tru_con}. This is 
		\begin{align*}
		 P_3 &= \int_0^\infty\int_0^\infty\int_0^\infty \tP_3(t_1,t_2,t_3)\tP_3^T(t_1,t_2,t_3) dt_1dt_2dt_3\\
		 &= \int_0^\infty\int_0^\infty\int_0^\infty e^{At_3}H (\tP_1(t_1)\tP^T(t_1)\otimes \tP_1(t_2)\tP^T(t_2))H^T e^{A^Tt_3}dt_1dt_2dt_3 \\
 		 &= \int_0^\infty e^{At_3}H \left( \left(\int_0^\infty \tP_1(t_1)\tP^T(t_1)dt_1\right)\otimes \left(\int_0^\infty\tP_1(t_2)\tP^T(t_2)dt_2\right)\right)H^T e^{A^Tt_3}dt_3 \\
 		 &= \int_0^\infty e^{At_3}H \left( \hP_1 \otimes \hP_1\right)H^T e^{A^Tt_3}dt_3. 
		 \end{align*}
		 Furthermore, we use the relation between the above integral representation and the solution of Lyapunov equation to show that $\hP_3$ solves:
		 \begin{equation}\label{eq:tru_P3}
		 A\hP_3 + \hP_3A^T + H(\hP_1\otimes \hP_1)H^T =0 .
		 \end{equation}
		 Summing \eqref{eq:tru_P1}, \eqref{eq:tru_P2} and \eqref{eq:tru_P3} yields
		 \begin{equation}
		 AP_\cT + \cP_{\cT}A^T + H(\hP_1\otimes \hP_1)  + \sum_{k=1}^mN_k\hP_1N_k + BB^T=0.
		 \end{equation}
		 Analogously, we can show that $Q_\cT$ solves \eqref{eq:tru_obs_Gram}, thus concluding the proof.
	\end{proof}
	
	We will investigate the advantages of these truncated Gramians in the model reduction framework  in the later part of the paper.
	
%\rd{\begin{remark}
%	We have derived the Gramians and truncated Gramians for SISO QB systems; however, these can be derived for a MIMO QB system in a similar manner. For a MIMO QB system, the Gramians $P$ and $Q$ satisfy
%	   \begin{align*}
%	   AP + PA^T +H(P\otimes P) H^T +  \sum_{k=1}^mN_kPN_k^T + BB^T &= 0,\quad \mbox{and}\\
%	    A^TQ + QA + \cH^{(2)}(P\otimes Q) (\cH^{(2)})^T +  \sum_{k=1}^mN_k^TQN_k +C^TC &= 0,
%	    \end{align*}
%	\end{remark} 
%respectively. Furthermore, the truncated Gramians $P_\cT$ and $Q_\cT$ for a MIMO QB system solve
%\begin{align*}
%	AP_\cT+P_\cT A^T  + H(P_1\otimes P_1)H^T +  \sum_{k=1}^mN_kP_1N_k^T +BB^T&=0,\\
%	A^TQ_\cT+Q_\cT A  +  \cH^{(2)}(P_1\otimes Q_1)(\cH^{(2)})^T + \sum_{k=1}^mN_k^TQ_1N_k +C^TC&=0,
%\end{align*}
%where $P_1$ and $Q_1$ are the solutions of \eqref{eq:linear_CG} and \eqref{eq:linear_OG}, respectively. }
 Next, we study the connection between the proposed Gramians for the QB system  and energy functionals. Also, we show how the definiteness of the Gramians is related to reachability and observability of the QB systems. These all suggest us how to determine the state components that are hard to control as well as hard to observe.