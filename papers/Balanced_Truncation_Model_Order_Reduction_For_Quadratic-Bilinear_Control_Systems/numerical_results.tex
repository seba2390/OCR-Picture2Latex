\def\redQBbal {\ensuremath{{\mathsf{RedQBbal}}}}
\def\redBbal {\ensuremath{{\mathsf{RedBbal}}}}

\def\GramQB {\ensuremath{{\mathsf{TGramQB}}}}
\def\GramB {\ensuremath{{\mathsf{TGramB}}}}

In this section, we illustrate the efficiency of the determined  reduced-order systems by using the proposed balanced truncation technique (Algorithm~\ref{algo:BT_qbdae}). We also compare them with reduced-order systems determined via the balancing of only corresponding bilinear part  by simply setting $H = 0$ in Algorithm~\ref{algo:BT_qbdae}. For ease of notations, we denote truncated Gramians of QBDAEs and the obtained reduced-order systems by \GramQB~ and \redQBbal~respectively.  Similarly,  Gramians, determined by the corresponding bilinear terms, are denoted as \GramB~ and obtained reduced-order systems by \redBbal. All the simulations are done in MATLAB\textsuperscript{\textregistered} version 7.10.0.499(R2010a)32-bit(win32) on Intel(R) Core(TM)2 Duo CPU P8600 @2.40GHz, 4GB RAM.
\subsection{Burgers' equation}
As a first example, we consider a one-dimensional Burgers' equation on $\Omega = (0,L)\times (0,T)$, whose governing equations with appropriate boundary conditions are given as follows:
\begin{equation}\label{eq:burger_equation}
 \begin{aligned}
  \dot{v} + vv_x &= \mu\cdot v_{xx}, &  &(0,L)\times (0,T),\\
  \alpha v(0,.) + \beta v_x(0,.) &= u(t), &  &(0,T),\\
  v_x(L,.) &= 0, \quad & &(0,T),\\
  v(x,0) &=v_0,\quad & &(0,L),
 \end{aligned}
\end{equation}
where $\mu$ and $v_0$ represent the viscosity and  initial condition of the system. Model reduction for Burgers' equation has been widely studied as a test case for nonlinear model-order reduction techniques; see, e.g.,~\cite{morBenB15,morKunV08,li2005compact}. The semi-discretization of the system~\eqref{eq:burger_equation} leads to a quadratic-bilinear control system. Similar to~\cite{morBenB15}, we also assume the above system subject to Dirichlet boundary control at $x=0$ and consider $\alpha =1$ and $\beta = 0$. Also, we assume  zero initial condition for the system, i.e.,  $v_0(x) = 0$. The output of the interest is the response at the right boundary at $x = L$.

Decreasing the viscosity, $\mu$, in the system~\eqref{eq:burger_equation}, reduces the influence of the diffusion term, and the nonlinear term  starts dominating. Therefore, we examine the performance of reduced-order systems obtained from Algorithm~\ref{algo:BT_qbdae} for two different viscosities to see their performances.
\renewcommand\thesubsubsection{\Alph{subsubsection}}
\subsubsection{Burgers' equation with $\mu = 0.05$}
We consider the system with a relative higher viscosity and set it to $\mu = 0.05$ and also assume the length of the domain $L = 1$. We discretize the system~\eqref{eq:burger_equation} using a finite difference scheme, leading to a QBDAE of order $n = 1000$. Next, we employ Algorithm~\ref{algo:BT_qbdae} to obtain truncated Gramians. In Figure~\ref{fig:SV_Burger}, we plot the decay of singular values and observe that singular values based on \GramQB~does not decay as compared to singular values computed based on \GramB.
\begin{figure}
\centering
\begin{center}
\begin{tikzpicture}
    \begin{customlegend}[legend columns=-1, legend style={/tikz/every even column/.append style={column sep=1.0cm}} , legend entries={Singular values from \GramQB, Singular Values from \GramB}, ]
    \addlegendimage{blue,solid,mark =*, mark size = 1.3}
    \addlegendimage{black!50!green,solid,mark = square*, mark size = 1.3}
    \end{customlegend}
\end{tikzpicture}
\end{center}
	\setlength\fheight{3cm}
	\setlength\fwidth{6cm}
	\input{pics/Burger_05_SingularDecay.tikz}
	\caption{Decay of the normalized singular values based on \GramQB~ and~\GramB.}
	\label{fig:SV_Burger}
\end{figure}

 This possibly occurs due to the presence of $H(P_1\otimes P_1)H^T + NP_1N^T$, which  perhaps has a higher rank as compared to the term $NP_1N^T$.   Next, we determine reduced-order systems \redQBbal ~and \redBbal ~based on balancing depending on \GramQB ~and \GramB, respectively of order $20$. Time-domain simulations of the original system and the reduced-order systems are depicted in Figure \ref{fig:1}  for an arbitrary control input $u(t) = 0.25\sin(2\pi t)e^{-t}(1-e^{-0.1t})$, which clearly indicates the \redQBbal~obtained via the proposed Gramians performs better.
\begin{figure}[h]
        \centering
         \begin{tikzpicture}
      \begin{customlegend}[legend columns=-1, legend style={/tikz/every even column/.append style={column sep=1.0cm}} , legend entries={Original System, \redQBbal , \redBbal}, ]
    \addlegendimage{blue,solid,mark =*, mark size = 1.3}
    \addlegendimage{black!50!green,solid,mark = square*, mark size = 1.3}
    \addlegendimage{red,solid,mark = triangle*, mark size = 1.3}
    \end{customlegend}
\end{tikzpicture}
 \begin{subfigure}[h]{0.49\textwidth}
  \centering
	\setlength\fheight{3cm}
	\setlength\fwidth{5.5cm}
	\input{pics/Burger_05_response.tikz}
	%\caption{The response for input $u(t) = e^{-t}$.}
 \end{subfigure}
 \begin{subfigure}[h]{0.49\textwidth}
  \centering
	\setlength\fheight{3cm}
	\setlength\fwidth{5.5cm}
	\input{pics/Burger_05_relativeerror.tikz}
	%\caption{Relative errors.}
 \end{subfigure}
 \caption{Boundary control Burgers' equation with $\mu = 0.05$.}
\label{fig:1}
\end{figure}


\subsubsection{Burgers' equation with $\mu = 5.10^{-4}$}
Now, we test the efficiency of the proposed Algorithm~\ref{algo:BT_qbdae} for quadratic-bilinear systems for much smaller viscosity $\mu = 5.10^{-4}$. We perform the same analysis as carried out for the viscosity $\mu = 0.05$. We here set the length of domain $L = 0.02$ and use the same number of grid points $n = 1000$,  leading to much smaller $\delta x$ in finite difference scheme which is required to capture the  better in the model for a low viscosity. In Figure~\ref{fig:SV_Burger_low}, we plot the decay of singular values based on \GramQB ~and \GramB.
\begin{figure}[h]
\centering
      \begin{tikzpicture}
          \begin{customlegend}[legend columns=-1, legend style={/tikz/every even column/.append style={column sep=1.0cm}} , legend entries={Singular values from \GramQB, Singular Values from \GramB}, ]
          \addlegendimage{blue,solid,mark =*, mark size = 1.3}
          \addlegendimage{black!50!green,solid,mark = square*, mark size = 1.3}
          \end{customlegend}
      \end{tikzpicture}
	\setlength\fheight{3cm}
	\setlength\fwidth{6cm}
	\input{pics/Burger_0005_SingularDecay.tikz}
	\caption{Decay of normalized singular values based on \GramQB~and \GramB.}
	\label{fig:SV_Burger_low}
\end{figure}

We make the same observation, as in the previous example with $\mu = 0.05$, in the decay of singular values. We next determine \redQBbal~ and \redBbal, both of order $r = 25$. Figure~\ref{fig:2} shows  time-domain simulations of the original and reduced-order systems for a control input $u(t) = 0.5te^{-t}\sin(2\pi t)$, which demonstrates that the \redQBbal~captures the input-output behavior relatively better.
\begin{figure}[h]
        \centering
         \begin{tikzpicture}
     \begin{customlegend}[legend columns=-1, legend style={/tikz/every even column/.append style={column sep=0.5cm}} , legend entries={Original System, \redQBbal , \redBbal}, ]
    \addlegendimage{blue,solid,mark =*, mark size = 1.3}
    \addlegendimage{black!50!green,solid,mark = square*, mark size = 1.3}
    \addlegendimage{red,solid,mark = triangle*, mark size = 1.3}
    \end{customlegend}
\end{tikzpicture}
 \begin{subfigure}[h]{0.49\textwidth}
  \centering
	\setlength\fheight{3cm}
	\setlength\fwidth{5.5cm}
	\input{pics/Burger_0005_Response.tikz}
	%\caption{The response for input $u(t) = e^{-t}$.}
 \end{subfigure}
 \begin{subfigure}[h]{0.49\textwidth}
  \centering
	\setlength\fheight{3cm}
	\setlength\fwidth{5.5cm}
	\input{pics/Burger_0005_relativeerror.tikz}
	%\caption{Relative errors.}
 \end{subfigure}
 \caption{Boundary control Burgers' equation with $\mu = 5\cdot 10^{-4}$.}
\label{fig:2}
\end{figure}

We also show solutions on the full grid, obtained from the original and reduced-order systems in Figure~\ref{fig:burger_full_grid}, although  balanced truncation model reduction mainly relies on the input-output behavior of the system.
\begin{figure}[h]
        \centering
 \begin{subfigure}[t]{0.32\textwidth}
  \centering
	\includegraphics[width = 4cm,height = 4cm]{pics/original_burger_0005_QBDAE.png}
	\caption{The original system.}
 \end{subfigure}
 \begin{subfigure}[t]{0.32\textwidth}
  \centering
	\includegraphics[width = 4cm,height = 4cm]{pics/reduced_burger_0005_QBDAE.png}
	\caption{\redQBbal.}
 \end{subfigure}
 \begin{subfigure}[t]{0.32\textwidth}
  \centering
	\includegraphics[width = 4cm,height = 4cm]{pics/reduced_burger_0005_BDAE.png}
	\caption{\redBbal.}
 \end{subfigure}
 \caption{Comparison of $v(x,t)$ on the full grid obtained from the original and reduced-order systems.}
 \label{fig:burger_full_grid}
 \end{figure}

We observe that {\redQBbal} captures the dynamics on the full grid much better than {\redBbal} as well. We also want to highlight that {\redQBbal} is much more stable as compared to {\redBbal}. This argument is based on our simulation experience for various control inputs.

\subsection{One-dimensional Chafee-Infante equation}
As a second example, we consider one-dimensional Chafee-Infante (Allen-Cahn) equation. This nonlinear system has been widely studied in the literature; see, e.g.,~\cite{chafee1974bifurcation,hansen2012second} and its model reduction related problem was recently considered in~\cite{morBenB15}. The governing equations are given as follows, subject to the initial conditions and boundary control  which has a cubic nonlinearity:
\begin{equation}\label{sys:Chafee_infante}
 \begin{aligned}
  \dot{v} +  v^3 &= v_{xx} + v, & \quad &(0,L)\times (0,T),\\
  v(0,.)  &= u(t), & \quad & (0,T),\\
  v_x(L,.) &= 0, & \quad & (0,T),\\
  v(x,0) &= 0, & \quad & (0,L).
 \end{aligned}
\end{equation}
Here, we also make use of a finite difference scheme with $k$ grid on for semi-discretization, leading to a nonlinear ODE. As shown in~\cite{morBenB15}, the smooth nonlinear system can be exactly transformed into quadratic-bilinear system by introducing appropriate new state variables. Therefore, this cubic nonlinearity in the system~\eqref{sys:Chafee_infante} can be rewritten  in a quadratic-bilinear system by defining  new variables $w_i = v_i^2$ and its derivate as $\dot{w}_i = 2v_i\dot{v}_i$, giving rise to a transformed QBDAE system of dimension $n = 2\cdot k$. The response at the right boundary at $x = L$ is under observation. We use discretized point $ k = 500$ and set $L = 0.5$. In Figure~\ref{fig:1D_chafee_singular}, we show the decay of singular values based on \GramQB~and \GramB.
\begin{figure}[h]
\centering
\begin{tikzpicture}
    \begin{customlegend}[legend columns=-1, legend style={/tikz/every even column/.append style={column sep=1.0cm}} , legend entries={Singular values from \GramQB, Singular Values from \GramB}, ]
    \addlegendimage{blue,solid,mark =*, mark size = 1.3}
    \addlegendimage{black!50!green,solid,mark = square*, mark size = 1.3}
    \end{customlegend}
\end{tikzpicture}
	\setlength\fheight{3cm}
	\setlength\fwidth{6cm}
	\input{pics/Chafee_SingularDecay.tikz}
		\caption{Decay of normalized singular values based on \GramQB ~and \GramB.}
		\label{fig:1D_chafee_singular}
\end{figure}

We make the similar observation as in Burgers' equation example, possibly due to the same reason. Next, we determine  reduced-order systems, {\redQBbal} and {\redBbal}, of order $r = 15$. Figure~\ref{fig:time_domain_CI} shows time-domain simulations for the original system with reduced-order systems for an arbitrary control input $u(t) = t^2e^{-t}+1$.
\begin{figure}[h]
        \centering
         \begin{tikzpicture}
    \begin{customlegend}[legend columns=-1, legend style={/tikz/every even column/.append style={column sep=0.5cm}} , legend entries={Original System, \redQBbal , \redBbal}, ]
    \addlegendimage{blue,solid,mark =*, mark size = 1.3}
    \addlegendimage{black!50!green,solid,mark = square*, mark size = 1.3}
    \addlegendimage{red,solid,mark = triangle*, mark size = 1.3}
    \end{customlegend}
\end{tikzpicture}
 \begin{subfigure}[t]{0.48\textwidth}
  \centering
		\setlength\fheight{3cm}
	\setlength\fwidth{5.5cm}
	\input{pics/Chafee_Response.tikz}
	%\caption{The transient response.}
 \end{subfigure}
 \begin{subfigure}[t]{0.48\textwidth}
  \centering
	\setlength\fheight{3cm}
	\setlength\fwidth{5.5cm}
	\input{pics/Chafee_Relative_error.tikz}
	%\caption{Relative errors.}
 \end{subfigure}
 \caption{Boundary control of Chafee-Infante equation for an input $u(t) = t^2e^{-t}+1$.}
\label{fig:time_domain_CI}
\end{figure}

We observe that the dynamics of the Chafee-Infante equation are not very well reproduced by {\redBbal}, and on the other hand, {\redQBbal} outperforms it.
%%%%%%%%%%%%%%%%%%%%%%%%%%%%%%%%%%%%%%%%%%%%%%%
%%%%%%%% FitzHugoNu
\subsection{The FitzHugh-Nagumo (F-N) system}
Lastly, we consider the F-N system, a simplified neuron model of the Hodgkin-Huxley model, which describes activation and deactivation dynamics of a spiking neuron. This model has been considered in framework of POD-based~\cite{morChaS10} and moment-matching model reduction techniques~\cite{morBenB12a}. The dynamics of the system is governed by the following nonlinear coupled PDEs:
\begin{equation}
\begin{aligned}
\epsilon v_t(x,t) & =\epsilon^2v_{xx}(x,t) + f(v(x,t)) -w(x,t) + q,\\
w_t(x,t) &= hv(x,t) -\gamma w(x,t) + q
\end{aligned}
\end{equation}
with a nonlinear function $f(v(x,t)) = v(v-0.1)(1-v)$  and initial and boundary conditions as follows:
\begin{equation}
\begin{aligned}
v(x,0) &=0, &\qquad\qquad w(x,0)&=0,\qquad \qquad & &x\in [0,L],\\
v_x(0,t) &= i_0(t), & v_x(1,t) &= 0, & &t\geq 0,
\end{aligned}
\end{equation}
where $\epsilon = 0.015,~h=0.5,~\gamma = 2,~q = 0.05$. We set the length $L = 0.2$. The $i_0$ acts as actuator, which takes the values $i_0(t) = 5\cdot 10^4t^3\exp(-15t)$, and the variable $v$ and $w$ are voltage and recovery voltage, respectively. We also assume the same output of interest in~\cite{morBenB12a}, given as $v(0,t)$ and $w(0,t)$, which is nothing but a limit cyclic at the left boundary. Using a finite difference discretization scheme, one can obtain a system with two inputs and two outputs of the dimension $2\cdot k$ with cubic nonlinearities, where $k$ is the number of degrees of freedom.   Similar to Chafee-Infante equation as shown in~\cite{morBenB12a} that the F-H system can be transformed into a QBDAE of dimension $n = 3\cdot k$, introducing a new state variable as $z_i = v_i^2$. We set $k = 300$, resulting a QBDAE of the dimension $n = 900$.  Figure~\ref{fig:Fitz_singular} compares the decay of singular values obtained based on {\GramQB} and {\GramB}.
\begin{figure}[h]
\centering
\begin{center}
\begin{tikzpicture}
    \begin{customlegend}[legend columns=-1, legend style={/tikz/every even column/.append style={column sep=1.0cm}} , legend entries={Singular values from \GramQB, Singular Values from \GramB}, ]
    \addlegendimage{blue,solid,mark =*, mark size = 1.3}
    \addlegendimage{black!50!green,solid,mark = square*, mark size = 1.3}
    \end{customlegend}
\end{tikzpicture}
\end{center}
	\setlength\fheight{3cm}
	\setlength\fwidth{6cm}
	\input{pics/Fitz_SingularDecay.tikz}
		\caption{Decay of the singular values based on {\GramQB} and {\GramB}.}
		\label{fig:Fitz_singular}
\end{figure}

Moreover, we determine reduced-order systems of order $r = 25$, based on  \GramQB~and \GramB. As we know that a limit cycle for each spatial variable $x$ exhibits the dynamics of this system. We therefore plot a phase-space diagram, illustrating the solutions $v$ and $w$ over the spatial domain $x$ in Figure~\ref{fig:fitz_limit}.
\begin{figure}[H]
        \centering
        \begin{center}
         \begin{tikzpicture}
    \begin{customlegend}[legend columns=-1, legend style={/tikz/every even column/.append style={column sep=1.5cm}} , legend entries={Original System, \redQBbal}, ]
    \addlegendimage{red}
    %\addlegendimage{black,solid,only marks,mark = *, mark size = 1.3}
    \addlegendimage{black,dotted,line width = 1pt}
    \end{customlegend}
\end{tikzpicture}
\end{center}
 \begin{subfigure}[t]{0.58\textwidth}
  \centering
		\includegraphics[width = 7.5cm,height = 4.5cm]{pics/Fitz_3D_phasediagram.pdf}
		{\caption{The exhibition of the limit-cycle.}}
 \end{subfigure}
 \begin{subfigure}[t]{0.38\textwidth}
  \centering
		\includegraphics[width = 5.5cm,height = 4.5cm]{pics/Fitz_2D_phasediagram.pdf}
		\caption{The limit-cycle projected on $v-w$ plane.}
 \end{subfigure}
 \caption{Comparison of  limit cycles obtained via the original system and reduced-order systems, and \redBbal~is unstable.}
 \label{fig:fitz_limit}
\end{figure}

%\begin{figure}[h]
%        \centering
%         \begin{tikzpicture}
%    \begin{customlegend}[legend columns=-1, legend style={/tikz/every even column/.append style={column sep=0.5cm}} , legend entries={Original System, Using QBDAE balancing , Using linear balancing}, ]
%    \addlegendimage{blue,solid,mark =*, mark size = 1.3}
%    \addlegendimage{black!50!green,solid,mark = square*, mark size = 1.3}
%    \addlegendimage{red,solid,mark = triangle*, mark size = 1.3}
%    \end{customlegend}
%\end{tikzpicture}
% \begin{subfigure}[t]{0.48\textwidth}
%  \centering
%		\setlength\fheight{4cm}
%	\setlength\fwidth{5.5cm}
%	\input{pics/Fitz_3D_phasediagram.tikz}
%	%\caption{The transient response.}
% \end{subfigure}
% \begin{subfigure}[t]{0.48\textwidth}
%  \centering
%	\setlength\fheight{4cm}
%	\setlength\fwidth{5.5cm}
%	\input{pics/Fitz_3D_phasediagram.tikz}
%	%\caption{Relative errors.}
% \end{subfigure}
% \caption{Boundary control of Chafee-Infante equation for an input $u(t) = t^2e^{-t}+1$.}
%%\label{fig:time_domain_CI}
%\end{figure}
As Figure \ref{fig:fitz_limit} illustrates that {\redQBbal} constructs the dynamics of the system very accurately, whereas the reduced-order system based on \GramB, {\redBbal} is  unstable system.

