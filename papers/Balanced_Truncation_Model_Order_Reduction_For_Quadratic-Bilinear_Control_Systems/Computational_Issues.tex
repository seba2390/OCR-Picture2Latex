Up to now, we have proposed the Gramians for the QB systems and showed their relations to energy functionals of the system which allows us to determine the reduced-order systems. Here, we discuss computational issues and the advantages of considering this truncated Gramians in the MOR framework. Towards this end, we address stability issues of the reduced-order systems, obtained by using the truncated Gramians. 
\subsection{Computational issues}
One of the major concerns in applying balanced truncation MOR is that it requires the solutions of  two Lyapunov equations~\eqref{eq:cont_lyap} and \eqref{eq:obser_lyap}. These equations are quadratic in nature, which are not trivial to solve, and they appear to be computationally expensive. So far, it is not clear  how to solve these generalized quadratic  Lyapunov equation efficiently; however, under some assumptions,  a fix point iteration scheme can be employed, which is based on the theory of convergent splitting presented in~\cite{damm2001newton,schneider1965positive}. This has been studied for solving generalized Lyapunov equation for bilinear systems in~\cite{damm2008direct}, wherein the proposed stationary method is as follows:
\begin{equation}\label{eq:bilinear_iter}
\cL(X_i) = \cN(X_{i-1}) - BB^T, \qquad i = 1,2,\ldots,
\end{equation}
with $\cL(X) = AX + XA^T$ and $\cN(X_i) = -\sum_{k=1}^mN_kX_iN_k^T$. To perform this stationary iteration, we require the solution of a conventional Lyapunov equation at each iteration. Assuming $\sigma(A) \subset \C^{-}$ and spectral radius of $\cL^{-1}\cN<1$, the iteration \eqref{eq:bilinear_iter} linearly converges to a positive semidefinite solution $X$ of the generalized Lyapunov equation for bilinear systems, which is 
$$AX + XA^T + \sum_{k=1}^mN_kXN_k^T+BB^T = 0.$$
Later on, the efficiency of this iterative method was improved in~\cite{shank2014efficient} by utilizing  tools for inexact solution of $Ax = b$. The main idea was to determine a low-rank factor of $\cN(X_{i-1}) - BB^T$ by truncating the columns, corresponding to small singular values and to increase the accuracy of the low-rank solution of the linear Lyapunov equation~\eqref{eq:bilinear_iter}  with each iteration. In total, this results in an efficient method to determine a low-rank solution of the generalized Lyapunov equation for bilinear systems with the desired tolerance. For detailed insights, we refer to~\cite{shank2014efficient}.

One can utilize the same tools to determine  the solutions of generalized quadratic-type Lyapunov equations. We begin with the inexact form equation, which on convergence gives the reachability Gramian; this is,
\begin{equation}
\label{eq:quad_iter}
\cL(X_i) = \Pi(X_{i-1}) - BB^T,\quad i = 1,2,\ldots
\end{equation}
where $\cL(X) = AX + XA^T$ and $\Pi(X) = -H(X\otimes X)H^T-\sum_{k=1}^mN_kXN_k^T$. Similar to the bilinear case, if $\sigma(A) \subset \C^{-}$ and the spectral radius of $\cL^{-1}\Pi<1$, then the iteration \eqref{eq:quad_iter} converges to a positive semidefinite solution  of the generalized quadratic Lyapunov equation. Next, we determine a low-rank approximation of $\Pi(X) = -H(X\otimes X)H^T-\sum_{k=1}^mN_kXN_k^T$. For this, let us assume a low-rank product $X := FDF^T$, where $F \in \Rnk$ and a QR decomposition of $F := Q_FR_F$. We then perform an eigenvalue decomposition of the relatively small matrix $R_FDR_F^T := U\Sigma U^T$, where $\Sigma = \diag{\sigma_1,\ldots,\sigma_k}$ with $\sigma_j \geq \sigma_{j+1}$.  Assuming there exists a scalar $\beta$ such that
 $$\sqrt{\sigma_{\beta+1}^2+\cdots +\sigma_k^2} \leq \tau \sqrt{\sigma_1^2+\cdots +\sigma_k^2},$$
where $\tau$ is a given tolerance, this gives us a low-rank approximation of $X$ as:
$$X \approx \tF\tD\tF^T,$$
where $\tF = Q_F \tU$ and $\tD = \diag{\sigma_1,\ldots,\sigma_\beta}$. Following the short-hand notation, we denote $\tZ = \cT_\tau (Z)$ which gives the low-rank approximation of $ZZ^T$ with the tolerance $\tau$, i.e., $ZZ^T \approx \tZ\tZ^T$. Considering a low-rank factor of $X_{k-1} \approx Z_{k-1}Z_{k-1}^T$,  the right side of~\eqref{eq:quad_iter}
\begin{multline*}
\Pi(X_{k-1}) - BB^T \approx -[H(Z_{k-1} \otimes Z_{k-1}), \left[N_1,\ldots,N_m\right]Z_{k-1}, B ]\\
\hspace{2cm}\times[H(Z_{k-1} \otimes Z_{k-1}), \left[N_1,\ldots,N_m\right]Z_{k-1}, B ]^T
\end{multline*}
can be replaced with its truncated version $\cT_\tau (\Pi(X_{k-1}) - BB^T) =: -\F_k\F_k^T$ with the desired tolerance. This indicates that we now need to solve the following linear Lyapunov equation at each step:
\begin{equation}\label{eq:quad_ite_approx}
AX_k + X_kA = -\F_k\F_k^T,
\end{equation}
 which can be solved very efficiently by using any of the recently developed low-rank solvers for Lyapunov equations; see, e.g.,~\cite{benner2013numerical,simoncini2013computational}.  
 %In this paper, we determine the low-rank factor of the Lyapunov equation~\eqref{eq:quad_ite_approx} by the extended Krylov subspace method; see~\cite{simoncini2007new,knizhnerman2011convergence}. As mentioned in~\cite{shank2014efficient}, despite the truncation, it may appear that the rank of $\F_k$ too big to see the benefits of the truncation of the right-hand side of the Lyapunov equation. The reason is that the cost of constructing the orthonormal basis, depending on $A$  and $\F_k$ could dominate overall; and the storage could be an issue in addition. Though considering tangential approaches for $\F_k$ could be a solution; see, e.g.~\cite{simoncini2013computational}. Nonetheless, here we use the solution proposed in~\cite{shank2014efficient} which makes use of the linearity of   \eqref{eq:quad_ite_approx}. For this,
%let us write  the matrix $\F  = [f_1,\ldots f_p]$, leading to $\F\F^T = f_1f_1^T + \cdots+ f_pf_p^T$.  We can then write
%\begin{equation*}
%X_k = \sum\nolimits_{i}X_k^{(i)},
%\end{equation*}
%where $X_k^{(i)}$ is the solution to
%\begin{equation*}
%X_k^{(i)} = \cL^{-1}f_if_i^T.
%\end{equation*}
%This splitting of $X_k$ appears to be considerable memory saving as well as CPU-time saving. 
In the following, we  outline all the necessary steps in \Cref{algo:solv_gram} to determine the Gramians by summarizing the all above discussed ingredients.
\begin{algorithm}[tb!]
 \caption{Iterative scheme to determine Gramians for QB systems.}
 \begin{algorithmic}[1]
    \Statex {\bf Input:} System matrices $ A, H,N_1,\ldots,N_m, B,C$ and tolerance $\tau$.
\Statex {\bf Output:} Low-rank factors of the Gramians: $Z_k~ (P \approx Z_kZ_k^T)$ and $ X_k~ (Q \approx X_kX_k^T)$.
    \State Solve approximately $AM + MA^T + BB^T = 0$ for $P_1 \approx Z_1Z_1^T$.
    \State Solve approximately $A^TG + GA + C^TC = 0$ for $Q_1 \approx X_1X_1^T$.
        \For {$k = 2,3,\ldots $}
        \State Determine low-rank factors:
        \Statex \label{step} \qquad\qquad $\mathbb B_k = \cT_{\tau} ([H(Z_{k-1} \otimes Z_{k-1}),  N_1Z_{k-1},\ldots,N_mZ_{k-1}  ,B ])$,% = [b_1^{(k)},\ldots, b_{\mu_k}^{(k)}]$,
        \Statex \qquad\qquad $\mathbb C_k = \cT_{\tau} ([\cH^{(2)}(Z_{k-1} \otimes X_{k-1}),  N_1^T X_{k-1},\ldots,N_m^T X_{k-1},C^T ]) $.
        \State Solve approximately $AM + M A^T + \mathbb B_k\mathbb B_k^T = 0$ for $P_k\approx Z_k Z_k^T $.
        \State Solve approximately $A^TG + G A + \mathbb C_k \mathbb C_k^T = 0$ for $Q_k\approx X_kX_k^T $.
	\If{solutions are sufficiently accurate } stop. \EndIf
    \EndFor
\end{algorithmic}\label{algo:solv_gram}
\end{algorithm}

\begin{remark}
	At step 7 of \Cref{algo:solv_gram}, one can check the accuracy of  solutions by measuring the relative changes  in the solutions, i.e., $\dfrac{\|P_k - P_{k-1} \|}{\|P_k\|}$ and $\dfrac{\|Q_k - Q_{k-1} \|}{\|Q_k\|}$. When these relative changes are smaller than a \emph{tolerance} level, e.g. the machine precision, then one can stop the iterations to have sufficiently accurate solutions of the quadratic Lyapunov equations. 
\end{remark}

\begin{remark}
In order to employ \Cref{algo:solv_gram}, the right side of the conventional Lyapunov equation (see step \ref{step}) requires the computation of  $H(Z_i\otimes Z_i) =: \Gamma$ at each step, which is also computationally and memory-wise expensive. If $Z_i \in  \R^{n\times n_z}$, then the direct multiplication of $Z_i\otimes Z_i$ would have complexity of $\cO(n^2\cdot n_z^2)$,  leading to an unmanageable task for large-scale systems, even on  modern computer architectures. However, it is shown in~\cite{morBenB15} that $\Gamma$ can be determined  efficiently by making use of the tensor multiplication properties, which are also reported in the previous section. In the following, we provide the procedure to compute $\Gamma$ efficiently:
\begin{itemize}
\item Determine $\cY \in \R^{n_z\times n\times n}$ such that $\cY^{(2)} = Z_i^T \cH^{(2)}$.
\item Determine $\cK \in \R^{n\times n_z \times n_z}$ such that $\cK^{(3)} = Z_i^T \cY^{(3)}$.
\item Then, $\Gamma = \cK^{(1)}$.
\end{itemize}
This way, we can avoid determining the full matrix $Z_i\otimes Z_i$.
% but still it can be argued that the formulation of $\cY$ at an intermediate step involves storage of $\cO(n^2\cdot n_z)$. However, as we know that the matrix $H$ is generally obtained via the finite difference or finite element discretization, which typically has non-zero columns of $\cO(n)$, and it holds for $\cH^{(2)}$ as well. Therefore, the multiplication of $\cH^{(2)}$ with $Z_i^T$ does not produce a dense tensor $\cY$; hence the required complexity  is much less $\cO(n^2\cdot n_z)$. This implementation  allows us to determine $\Gamma$ efficiently for large-scale  dynamical systems. 
Analogously, we can also compute  the term $\cH^{(2)}(Z_i\otimes X_i)$.
\end{remark}
 
 Next, we discuss the existence of the solutions of quadratic type generalized Lyapunov equations. As noted \Cref{algo:solv_gram}, one can determine the solution of these Lyapunov equations using  fixed point iterations. In the following, we discuss sufficient conditions under which these iterations converge to finite solutions.    
\begin{theorem}
Consider a QB system as defined in~\cref{eq:Quad_bilin_Sys} and let $P$ and $Q$ be its reachability and observability Gramians, respectively.	Assume that the Gramians $P$ and $Q$ are determined using fixed point iterations as shown in \Cref{algo:solv_gram}.   Then, the Gramian $P$ converges to a positive semidefinite solution if
	\begin{enumerate}[label=(\roman*)]
	\item $A$ is stable, i.e., there exist  $0<\alpha\leq -\max(\real{\lambda_i (A)})$ and $\beta>0$  such that $\|e^{At}\| \leq \beta e^{-\alpha t}$. \label{eq:cond_Pexist1}
	\item $\dfrac{	\beta^2\Gamma_N }{2\alpha} < 1$, where $\Gamma_N := \sum_{k=1}^m\|N_k\|^2$.
	\item $ 1>\cD^2 - \dfrac{\beta^2 \Gamma_H}{\alpha}\dfrac{\beta^2\Gamma_B}{\alpha} > 0, ~~where~~ \cD:= 1-\dfrac{\beta^2\Gamma_N}{2\alpha}$,\label{eq:cond_Pexist3}
	where $\Gamma_B := \|BB^T\|$, $\Gamma_H := \|H\|^2$.
   \end{enumerate}
   and $\|P\|$ is bounded by
	\begin{equation}
	\|P\| \leq \dfrac{2\alpha}{\beta^2\Gamma_H} \left(\cD- \sqrt{\cD^2 - 4\dfrac{\beta^2 \Gamma_H}{2\alpha}\dfrac{\beta^2\Gamma_B}{2\alpha} }\right) =: \cP_\infty.
	\end{equation}
	Furthermore, the Gramian $Q$  also converges to a positive semidefinite solution if in addition to the above  conditions \ref{eq:cond_Pexist1}--\ref{eq:cond_Pexist3}, the following condition satisfies
	 		\begin{equation}
	 		\dfrac{\beta^2}{2\alpha}	\left(\Gamma_N + \tilde\Gamma_{H} \cP_\infty \right)< 1,
	 		\end{equation}
	 		where $\tilde{\Gamma}_H := \|\cH^{(2)}\|^2$. Moreover,  $\|Q\|$ is bounded by 
	 		\begin{equation}
		\|Q\| \leq \dfrac{\beta^2}{2\alpha}\Gamma_C \left(1-\dfrac{\beta^2}{2\alpha}	\left(\Gamma_N + \tilde\Gamma_H \cP_\infty \right)\right)^{-1},
	 		\end{equation}
	 		where $\Gamma_C:= \|C^TC\|.$
\end{theorem}
\begin{proof}
	Let us first consider the equation corresponding to $P_1$:
	\begin{equation}
	AP_1 + AP_1 + BB^T = 0.
	\end{equation}
	Alternatively, if $A$ is stable, we can write $P_1$ in the integral form as
	\begin{equation}
	P_1 = \int_0^\infty e^{At}BB^Te^{A^Tt}dt,
	\end{equation}
	implying  
	\begin{equation}
	\|P_1\| \leq \beta^2 \|BB^T\|  \int_0^\infty e^{-2\alpha t}dt = \dfrac{\beta^2\Gamma_B}{2\alpha},
	\end{equation}
where $\Gamma_B := \|BB^T\|$.	Next, we look at the equation corresponding to $P_k$, which is given in terms of $P_{k-1}$:
	\begin{equation}
	AP_k + P_kA^T + H(P_{k-1} \otimes P_{k-1})H^T + \sum_{k=1}^mN_kP_{k-1}N_k + BB^T=0.
	\end{equation}
	We can also write $P_k$ in an integral form, provided $A$ is stable:
	\begin{align*}
	P_k &=\int_0^\infty e^{At}\left(H(P_{k-1} \otimes P_{k-1})H^T + \sum_{k=1}^mN_kP_{k-1}N_k + BB^T\right)e^{A^Tt}dt\\
	&\leq  \beta^2\left(\Gamma_H \|P_{k-1}\|^2 + \Gamma_N\|P_{k-1}\| + \Gamma_B\right)\int_0^\infty e^{-2\alpha t}dt\\
		&\leq \beta^2  \dfrac{\left(\Gamma_H \|P_{k-1}\|^2 + \Gamma_N\|P_{k-1}\| + \Gamma_B\right)}{2\alpha},
	\end{align*}
	where $\Gamma_H := \|H\|^2$ and $\Gamma_N := \sum_{k=1}^m\|N_k\|^2$.
	If we consider an upper bound for the norm of $P_{k-1}$ in order to provide an upper bound for $P_k$ and apply \Cref{appendix_convergence}, then we know that $\lim_{k\rightarrow \infty} \|P_k\|$ is bounded if
	\begin{align*}
&1>\cD^2 - 4\dfrac{\beta^2 \Gamma_H}{2\alpha}\dfrac{\beta^2\Gamma_B}{2\alpha} \geq 0, ~~where~~ \cD:= 1-\dfrac{\beta^2\Gamma_N}{2\alpha} \quad\text{and}\quad
 \dfrac{\beta^2\Gamma_N}{2\alpha} <1,
	\end{align*}
	and $\lim_{k\rightarrow \infty} \|P_k\|$ is bounded by 
	\begin{equation*}
	\lim_{k\rightarrow \infty} \|P_k\| \leq \dfrac{2\alpha}{\beta^2\Gamma_H} \left(\cD- \sqrt{\cD^2 - 4\dfrac{\beta^2 \Gamma_H}{2\alpha}\dfrac{\beta^2\Gamma_B}{2\alpha} }\right) =: \cP_\infty.
	\end{equation*}
Now, we consider the equation corresponding to $Q_1$:
	\begin{equation*}
	A^TQ_1 + A^TQ_1 + C^TC = 0,
	\end{equation*}
which again can be rewritten as:
	\begin{equation*}
	Q_1 = \int_0^\infty e^{A^Tt}C^TCe^{At}dt
	\end{equation*}
if $A$ is stable. This	implies 
	\begin{equation*}
	\|Q_1\| \leq \beta^2\Gamma_C  \int_0^\infty e^{-2\alpha t}dt = \beta^2 \dfrac{\Gamma_C}{2\alpha},
	\end{equation*}
where $\Gamma_c := \|C^TC\|$.	Next, we look at the equation corresponding to $Q_k$, that is,
	\begin{align*}
	A^TQ_k + Q_kA + \cH^{(2)}(P_{k-1} \otimes Q_{k-1})\left(\cH^{(2)}\right)^T + \sum_{k=1}^mN_k^TQ_{k-1}N_k + C^TC&=0. 
	\end{align*}
A similar analysis for $Q_k$  yields
\begin{equation*}
\|Q_k\| \leq \dfrac{\beta^2}{2\alpha}\left( \left(\Gamma_N + \tilde\Gamma_H \|P_{k-1}\| \right)Q_{k-1} + \Gamma_C \right),
\end{equation*}
where $\tilde{\Gamma}_H := \|\cH^{(2)}\|$. Since $\|P_{k-1}\| \leq \cP_\infty$ for all $k\geq 1$, we further have
\begin{equation*}
\|Q_k\| \leq \dfrac{\beta^2}{2\alpha}\left( \left(\Gamma_N + \tilde\Gamma_H \cP_\infty \right)\|Q_{k-1}\| + \Gamma_C \right).
\end{equation*}
An additional sufficient condition under which the above recurrence formula in $\|Q_{k}\|$  converges is as follows:
		\begin{equation*}
\dfrac{\beta^2}{2\alpha}	\left(\Gamma_N + \tilde\Gamma_H \cP_\infty \right)< 1,
		\end{equation*}
		and $\lim_{k\rightarrow \infty}\|Q_k\|$ is then bounded by 
		\begin{equation*}
				\lim_{k\rightarrow \infty}\|Q_k\| \leq \dfrac{\beta^2}{2\alpha}\Gamma_C \left(1-\dfrac{\beta^2}{2\alpha}	\left(\Gamma_N + \tilde\Gamma_H \cP_\infty \right)\right)^{-1}.
		\end{equation*}
		This concludes the proof.
	\end{proof}
\begin{remark}
In \Cref{algo:solv_gram}, we propose to determine the low-rank solutions of the Lyapunov equation  at each intermediate step with the same tolerance. However, one can also consider to  increase the tolerance adaptively for computing the low-rank  solution of the Lyapunov equation  with each  iteration which probably can speed up even more, see~\cite{shank2014efficient} for the generalized Lyapunov equations for bilinear systems.  
\end{remark}

%\subsection{MOR using Truncated Gramians and its advantages}
%Summarizing,  we have investigated the Gramians for the QB systems and some properties such as energy functionals, and controllability and observability concepts of the system. Moreover, we have defined the truncated Gramians, $P_\cT$ and $Q_\cT$, which are based on the leading three  terms of the Volterra series, see Corollary \ref{coro:tru_gram}. We have also shown that the dynamics of QB systems are invariant under $\ker P$ and $\ker Q$, and   as well as under $\ker P_\cT$ and $\ker Q_\cT$, and additionally, the truncated Gramians also approximate  energy functionals, see Subsection~\ref{subsec:energy}. From this, one can infer that the truncated Gramians, $P_\cT$ and $Q_\cT$ also can be used to determine weakly controllable and observable states via balancing, instead of Gramians $P$ and $Q$. However, there are some advantages of considering the truncated Gramians in the model reduction framework which are:
%\begin{itemize}
%\item Firstly, they can be computed much cheaper by setting $k = 2$ at step $6$  in Algorithm~\ref{algo:solv_gram}. On the other side, to determine $P$ and $Q$, we need to iterate until the tolerance level is achieved, which might be very expensive.
%\item Secondly, the rank of $Z_k$ and $X_k$ will increase with each iteration. This implies that  $\rank P \geq \rank P_\cT$ and $\rank Q \geq \rank Q_\cT$. Since $P,Q,P_\cT$ and $Q_\cT$ are positive semidefinite matrices, this indicates that the singular values of $P_\cT\cdot Q_\cT$ will not delay slower as compared to the singular values of $P\cdot Q$. This is illustrated in Figure~\ref{fig:decay_sigu_tru}, considering the  Chafee-Infante equation. We refer to the next section for the governing equations and boundary conditions of the system.
%
%\begin{figure}[h]
%\centering
%\begin{tikzpicture}
%    \begin{customlegend}[legend columns=-1, legend style={/tikz/every even column/.append style={column sep=1.0cm}} , legend entries={Using Gramians $P$ and $Q$, Using truncated Gramians $P_t$ and $Q_t$}, ]
%    \addlegendimage{blue,solid,mark =*, mark size = 1.3}
%    \addlegendimage{black!50!green,solid,mark = square*, mark size = 1.3}
%    \end{customlegend}
%\end{tikzpicture}
%	\setlength\fheight{3cm}
%	\setlength\fwidth{6cm}
%\input{pics/singularvalue_decay_chafee.tikz}
%\caption{Normalized singular values using the Gramians and truncated Gramians for the Chafee-Infante equation with $L = 1, n = 50$.}\label{fig:decay_sigu_tru}
%\end{figure}
%
%
%Thus, if one sets the criterion to determine a  reduced-order system based on the decay of the singular values, then smaller dimensional  reduced-order systems can be obtained via balancing of the truncated Gramians.
%\item Lastly,  the convergence of Algorithm~\ref{algo:solv_gram}  highly depends on the spectral radius condition $\cL^{-1}\Pi$ which should be less than $1$. This condition for QB systems does not necessarily satisfy as we observe in all our numerical examples. As a result, the Algorithm~\ref{algo:solv_gram} may not convergence. On contrary, there is no such issue in using truncated Gramians for balancing.
%
%
%\end{itemize}
%In our all numerical examples in the next section, we consider the truncated Gramians for the QB systems to determine reduced-order system. In Algorithm~\ref{algo:BT_qbdae}, we present the sqrt-root balanced truncation for the QB systems based on the truncated Gramians.
%\begin{algorithm}[h]
% \caption{Balanced truncation for QB systems (truncated version).}
% \begin{algorithmic}[1]
%    \Statex {\bf Input:} System matrices $ A, H,N_k, B$ and $C$, and the order of the reduced system~$r$.
%    \Statex {\bf Output:} The reduced system's matrices  $\hA, \hH,\hN_k,\hB, \hC.$
%    \State Determine low-rank approximations of the truncated Gramians $P_\cT\approx RR^T$ and $Q_\cT~\approx~SS^T$.
%    \State Compute SVD of $S ^TR$ and decompose as follows:
%    \Statex \qquad $S^TR = U\Sigma V = \bbm U_1 & U_2\ebm \diag{ \Sigma_1,\Sigma_2}\bbm V_1 & V_{2}\ebm^T$,
%\Statex\qquad     where $\Sigma_1$ contains the $r$ largest singular values of $S^TR$.
%    \State Construct the projection matrices $\cV$  and $\cW$:
%    \Statex \qquad $\cV = S U_1\Sigma_1^{-\tfrac{1}{2}}$ and $\cW = R V_1\Sigma_1^{-\tfrac{1}{2}}$.
%    \State Determine the reduced-order system's realization:
%    \Statex \qquad$\hA = \cW^TA\cV,~~\hH =\cW^T H(\cV\otimes \cV),~~\hN_k = \cW^TN_k\cV,~~ \hB = \cW^TB,~~\hC = C\cV $.
%\end{algorithmic}\label{algo:BT_qbdae}
%\end{algorithm}
%

\subsection{MOR using truncated Gramians}
As noted in \Cref{sec:energyfunctionals}, the quadratic forms of both actual Gramians and its truncated versions (truncated Gramians) impose  bounds for the energy functionals of QB systems, at least in the neighborhood of the origin, and we also provide the interpretation of reachability and observability of the system in terms of Gramians and truncated Gramians.  We have seen in the previous subsection that determining Gramians $P$ and $Q$ is very challenging task for large-scale settings.  Moreover, the convergence of \Cref{algo:solv_gram}  highly depends on the spectral radius condition $\cL^{-1}\Pi$, which should be less than $1$. This condition might not be satisfied  for large $H$ and $N_k$; thus,  \Cref{algo:solv_gram} may not convergence.  On the other hand, in order to compute the truncated Gramians, there is no such convergence issue. Furthermore, it can also be  observed that the bounds for energy functionals using the truncated Gramains can be much better (in the neighborhood of the origin), for example see \Cref{fig:comparison_gram}. 

Additionally, if we remove those states that are completely uncontrollable and completely unobservable, then the truncated Gramians may provide reduced systems which are of smaller orders as compared to using the Gramians of QB systems. This is due to the fact that $P \geq P_\cT$ and $Q \geq Q_\cT$. This motivates us to use the truncated Gramians to determine the reduced-order models, and we present the square-root balanced truncation for QB systems based on these truncated Gramians in \Cref{algo:BT_qbdae}.  Furthermore, we will see in \Cref{sec:Numerical} as well that these truncated Gramians also yield very good qualitative reduced-order systems for QB systems.


%In our all numerical examples in the next section, we consider the truncated Gramians for the QB systems to determine reduced-order system. In Algorithm~\ref{algo:BT_qbdae}, we present the sqrt-root balanced truncation for the QB systems based on the truncated Gramians.
%
%Moreover, if $H$ or $N_k$ are very large then it may happen that \Cref{algo:solv_gram} does not convergence.
%
%Lastly,  the convergence of Algorithm~\ref{algo:solv_gram}  highly depends on the spectral radius condition $\cL^{-1}\Pi$ which should be less than $1$. This condition for QB systems does not necessarily satisfy as we observe in all our numerical examples. As a result, the Algorithm~\ref{algo:solv_gram} may not convergence. On contrary, there is no such issue in using truncated Gramians for balancing.
%
%
%However, there are some advantages of considering the truncated Gramians in the model reduction framework which are:
%\begin{itemize}
%	\item Firstly, they can be computed much cheaper by setting $k = 2$ at step $6$  in Algorithm~\ref{algo:solv_gram}. On the other side, to determine $P$ and $Q$, we need to iterate until the tolerance level is achieved, which might be very expensive.
%	\item Secondly, the rank of $Z_k$ and $X_k$ will increase with each iteration. This implies that  $\rank P \geq \rank P_\cT$ and $\rank Q \geq \rank Q_\cT$. Since $P,Q,P_\cT$ and $Q_\cT$ are positive semidefinite matrices, this indicates that the singular values of $P_\cT\cdot Q_\cT$ will not delay slower as compared to the singular values of $P\cdot Q$. This is illustrated in Figure~\ref{fig:decay_sigu_tru}, considering the  Chafee-Infante equation. We refer to the next section for the governing equations and boundary conditions of the system.
%	
%	\begin{figure}[h]
%		\centering
%		\begin{tikzpicture}
%		\begin{customlegend}[legend columns=-1, legend style={/tikz/every even column/.append style={column sep=1.0cm}} , legend entries={Using Gramians $P$ and $Q$, Using truncated Gramians $P_t$ and $Q_t$}, ]
%		\addlegendimage{blue,solid,mark =*, mark size = 1.3}
%		\addlegendimage{black!50!green,solid,mark = square*, mark size = 1.3}
%		\end{customlegend}
%		\end{tikzpicture}
%		\setlength\fheight{3cm}
%		\setlength\fwidth{6cm}
%		\input{pics/singularvalue_decay_chafee.tikz}
%		\caption{Normalized singular values using the Gramians and truncated Gramians for the Chafee-Infante equation with $L = 1, n = 50$.}\label{fig:decay_sigu_tru}
%	\end{figure}
%	
%	
%	Thus, if one sets the criterion to determine a  reduced-order system based on the decay of the singular values, then smaller dimensional  reduced-order systems can be obtained via balancing of the truncated Gramians.
%	\item 
%	
%\end{itemize}
%In our all numerical examples in the next section, we consider the truncated Gramians for the QB systems to determine reduced-order system. In Algorithm~\ref{algo:BT_qbdae}, we present the sqrt-root balanced truncation for the QB systems based on the truncated Gramians.
\begin{algorithm}[!tb]
	\caption{Balanced truncation for QB systems (truncated version).}
	\begin{algorithmic}[1]
		\Statex {\bf Input:} System matrices $ A, H,N_k, B$ and $C$, and the order of the reduced system~$\hn$.
		\Statex {\bf Output:} The reduced system's matrices  $\hA, \hH,\hN_k,\hB, \hC.$
		\State Determine low-rank approximations of the truncated Gramians $P_\cT\approx RR^T$ and $Q_\cT~\approx~SS^T$.
		\State Compute SVD of $S ^TR$:
		\Statex \qquad $S^TR = U\Sigma V = \bbm U_1 & U_2\ebm \diag{ \Sigma_1,\Sigma_2}\bbm V_1 & V_{2}\ebm^T$,
		\Statex\qquad     where $\Sigma_1$ contains the $\hn$ largest singular values of $S^TR$.
		\State Construct the projection matrices $\cV$  and $\cW$:
		\Statex \qquad $\cV = S U_1\Sigma_1^{-\tfrac{1}{2}}$ and $\cW = R V_1\Sigma_1^{-\tfrac{1}{2}}$.
		\State Determine the reduced-order system's realization:
		\Statex \qquad$\hA = \cW^TA\cV,~~\hH =\cW^T H(\cV\otimes \cV),~~\hN_k = \cW^TN_k\cV,~~ \hB = \cW^TB,~~\hC = C\cV $.
	\end{algorithmic}\label{algo:BT_qbdae}
\end{algorithm}

