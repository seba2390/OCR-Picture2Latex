We begin with recapitulation of energy functionals for nonlinear systems.
\subsection{Energy functionals for nonlinear systems}
In this subsection, we give a brief overview of energy functionals, namely controllability and observability energy functionals for nonlinear systems. For this, let us consider the following smooth, for example, $C^\infty$, nonlinear asymptotically stable input-affine nonlinear system of the form
\begin{equation}\label{eq:Gen_NonlinearSys}
 \begin{aligned}
  \dot{x}(t) &=  f(x) + g(x)u(t),\\
  y(t) &= h(x),\qquad x(0) = 0,
 \end{aligned}
\end{equation}
where $x(t) \in \Rn$, $u(t) \in \Rm$ and $y(t)\in\Rp$ are the state, input and output vectors of the system, respectively, and also $f(x) : \Rn \rightarrow \Rn$, $g(x) :\Rn \rightarrow \R^{n\times m}$ and $h(x) : \Rn\rightarrow \Rp$ are smooth nonlinear functions. Without loss of generality, we assume  that $0$ is an equilibrium of  the system~\eqref{eq:Gen_NonlinearSys}.  The controllability and observability energy functionals for the general nonlinear systems  first have been studied in the literature in~\cite{morSch93}. In the following, we state the definitions of controllability and observability energy functionals for the system~\eqref{eq:Gen_NonlinearSys}.
\begin{definition}\cite{morSch93}
 The controllability energy functional is defined as the minimum amount of energy required to steer the system from $x(-\infty) = 0$ to $x(0) = x_0$:
 \begin{equation*}
  L_c(x_0) = \underset{\begin{subarray}{c}
 u\in L^m_2(-\infty,0] \\[5pt]
 x(-\infty)=0,~x(0) = x_0
  \end{subarray}  }{\min} \dfrac{1}{2}\int_{-\infty}^0 \|u(t)\|^2dt.
 \end{equation*}
\end{definition}
\begin{definition}\cite{morSch93}\label{def:obser1}
The observability energy functional can be defined as the  energy generated by the nonzero initial condition $x(0) = x_0$ with zero control input:
% \begin{equation*}
%  L_o(x_0) = \underset{\begin{subarray}{c}
%  u\in L_2(0,\infty), \|u\|_{L_2} \leq 1 \\[5pt]
%  x(0) = x_0,x(\infty)=0
%   \end{subarray}  }{\textsf{max}} \dfrac{1}{2}\int_0^\infty \|y(t)\|^2dt.
%  \end{equation*}
\begin{equation*}
 L_o(x_0) = \dfrac{1}{2}\int_0^\infty \|y(t)\|^2dt.
 \end{equation*}
\end{definition}
We assume that the system~\eqref{eq:Gen_NonlinearSys} is controllable and observable. This implies that the system~\eqref{eq:Gen_NonlinearSys} can always be steered from $x(0) =0$ to $x_0$ by using appropriate inputs.
%, and the initial condition of the system can be determined by observing the output of the system. Note that in \Cref{def:obser1}, the control input is  completely set to zero.  As discussed in~\cite{morgray1996}, for a nonlinear system such condition can be very strong. As a result, therein, it is shown how this condition can be  relaxed in the context of general input balancing, and a new definition for the observability functionals is provided as follows:
To define the observability energy functional (\Cref{def:obser1}), it is assumed that the nonlinear system~\eqref{eq:Gen_NonlinearSys} is  a zero-state observable. It means that if $u(t) = 0$ and $y(t) =0$ for $t\geq 0$, then $x(t) = 0$ $\forall t\geq 0$. However, as discussed in~\cite{morgray1996}, for a nonlinear system such a condition can be very strong. As a result, therein, it is shown how this condition can be  relaxed in the context of general input balancing, and a new definition for the observability functionals was proposed as follows:


\begin{definition}\cite{morgray1996}\label{def:obser2}
	The observability energy functional can be defined as the  energy generated by the nonzero initial condition $x(0) = x_0$  and by applying an $L^m_2$-bounded input:
	 \begin{equation*}
	  L_o(x_0) = \underset{\begin{subarray}{c}
	  u\in L^m_2[0,\infty), \|u\|_{L_2} \leq \alpha \\[5pt]
	  x(0) = x_0,x(\infty)=0
	   \end{subarray}  }{\max} \dfrac{1}{2}\int_0^\infty \|y(t)\|^2dt.
	  \end{equation*}
\end{definition}

In an abstract way, the main idea of introducing \Cref{def:obser2} to find the state component that contributes less from a state-to-output point of view for all possible $L_2$-bounded inputs. 
The connections between these energy functionals and the solutions of the partial differential equations  are established in~\cite{morgray1996,morSch93}, which are outlined in the following theorem.
%Moreover, it is shown in~\cite{morgray1996,morSch93} that the energy functionals satisfy partial differential equations as stated in the following theorem.
\begin{theorem}\cite{morgray1996,morSch93}\label{thm:energy_function}
Consider the nonlinear system~\eqref{eq:Gen_NonlinearSys}, having $x = 0$ as an asymptotically stable equilibrium of the system in a neighborhood $W_o$ of $0$. Then, for all $x\in W_o$, the observability energy functional $L_o(x)$ can be determined  by the following partial differential equation:
\begin{equation}\label{eq:Obser_Diff}
 \dfrac{\partial L_o}{\partial x}  f(x) + \dfrac{1}{2}h^T(x)h(x) - \dfrac{1}{2}\mu^{-1}\dfrac{\partial L_o}{\partial x}g(x)g(x)^T\dfrac{\partial^T L_o}{\partial x} = 0,\quad L_o(0) = -\dfrac{1}{2}\mu,
\end{equation}
assuming that there exists a smooth solution $\bar{L}_o$ on $W$, and $0$ is an asymptotically stable equilibrium of $\bar{f}:= (f-\mu^{-1}gg^T\tfrac{\partial^T \bar{L}_o} {\partial x})$ on $W$  with a negative real number $\mu:= -\|g^T(\phi)\tfrac{\partial^T \bar{L}_o} {\partial x} (\phi) \|_{L_2}$, and $\dot{\phi} = \bar{f}(\phi)$ with $\phi(0) = x$.
%\begin{equation}\label{eq:Obser_Diff}
%\dfrac{\partial L_o}{\partial x}  f(x) + \dfrac{1}{2}h^T(x)h(x) =0,\qquad L_o(0) = 0.
%\end{equation}
%assuming \eqref{eq:Obser_Diff} has a smooth solution on $W_o$.
 Moreover, for all $x\in W_c$, the controllability energy functional $L_c(x)$ is a unique smooth solution of the following Hamilton-Jacobi equation:
\begin{equation}\label{eq:Cont_Diff}
 \dfrac{\partial L_c}{\partial x} f(x) +  f(x)\dfrac{\partial L_c}{\partial x} + \dfrac{\partial L_c}{\partial x}g(x)g(x)^T\dfrac{\partial^T L_c}{\partial x} = 0,\quad L_c(0) = 0
\end{equation}
under the assumption that~\eqref{eq:Cont_Diff} has a smooth solution $\bar{L}_c$ on $W_c$, and $0$ is an asymptotically stable equilibrium of $-\left(f(x) + g(x)g(x)^T\tfrac{\partial\bar{L}_c(x)}{\partial x}^T\right)$ on $W_c$.
\end{theorem}

Note that in \Cref{def:obser2}, the zero-state observable condition is relaxed by considering $L_2$-bounded inputs.  However, an alternative way to relax the zero-state observable condition by considering not only $L_2$-bounded inputs but also  $L_\infty$ bounded inputs. We thus propose a new definition of the observability energy functional as follows:

\begin{definition}\label{def:obser3}
	The observability energy functional can be defined as the  energy generated by the nonzero initial condition $x(0) = x_0$ and by applying an $L_2$-bounded and $L_\infty$-bounded input:
	\begin{equation*}
	L_o(x_0) = \underset{\begin{subarray}{c}
		u\in \cB_{(\alpha,\beta)} \\[5pt]
		x(0) = x_0,x(\infty)=0
		\end{subarray}  }{\max} \dfrac{1}{2}\int_0^\infty \|y(t)\|^2dt,
	\end{equation*}
\end{definition}
where $\cB_{(\alpha,\beta)} \overset{\mathrm{def}}{=} \{u \in L_2^m[0,\infty), \|u\|_{L_2}\leq \alpha, \|u\|_{L_\infty}\leq \beta \}$. In this paper, we use the above definition to characterize the observability energy functional for QB systems. 
\subsection{Hilbert adjoint operator for nonlinear systems}
 The importance of the adjoint operator (dual system) can be seen, particularly, in the computation of the observability energy functional or Gramian.
 For  general nonlinear systems, a duality between controllability and observability energy functionals is shown in~\cite{Adjfujimoto2002} with the help of  state-space realizations for nonlinear adjoint operators. In what follows, we briefly outline the state-space realizations for nonlinear adjoint operators of nonlinear systems. For this, we consider a nonlinear system of the form
 \begin{equation}\label{eq:Gen_Nonlinear}
\Sigma := \begin{cases}
\begin{aligned}
 \dot{x}(t) &= \cA(x,u,t) x(t) + \cB(x,u,t)u(t),\\
 y(t) &= \cC(x,u,t)x(t) + \cD(x,u,t)u(t),\qquad x(0) = 0
 \end{aligned}
\end{cases}
\end{equation}
in which $x(t) \in \Rn$, $u(t) \in \Rm$ and $y(t) \in \Rp$ are the state, input and output  vectors of the system, respectively, and $\cA(x,u,t)$, $\cB(x,u,t) $, $\cC(x,u,t)$ and $\cD(x,u,t) $ are appropriate size matrices.  Also, we assume that the origin is a stable equilibrium of the system. The Hilbert adjoint operators for the general nonlinear systems have been investigated in~\cite{Adjfujimoto2002}. Therein,  a connection between the state-space realization of the adjoint operators and  port-control Hamiltonian systems is also discussed, leading to the state-space characterization of the nonlinear Hilbert adjoint operators of $\Sigma:L_2^m(\Omega) \rightarrow L_2^p(\Omega)$. In the following lemma, we summarize the state-space realization of the Hilbert adjoint operator of the nonlinear system.
\begin{lemma}\cite{Adjfujimoto2002}\label{lemma:adjointsys}
 Consider the system~\eqref{eq:Gen_Nonlinear} with the initial condition $x(0) =0$, and assume that the input-output mapping $u\rightarrow y$ is denoted by the operator $\Sigma : L_2^m(\Omega) \rightarrow L_2^p(\Omega)$. Then, the state-space realization of the nonlinear Hilbert adjoint operator $\Sigma^*:L_2^{m+p}(\Omega) \rightarrow L_2^m(\Omega)$ is given by
 \begin{equation}\label{eq:Gen_Nonlinear_Adj}
 \Sigma^*(u_d,u) := \begin{cases}
\begin{aligned}
   \dot{x}(t) &= \cA(x,u,t)x(t) + \cB(x,u,t)u(t), & \quad  x(0) &= 0,\\
   \dot{x_d}(t) &= -\cA(x,u,t)x_d(t) - \cC^T(x,u,t)u_d(t), & \quad  x_d(\infty) &= 0,\\
   y_d(t) &= \cB^T(x,u,t)x_d(t) + \cD^T(x,u,t)u_d(t),
  \end{aligned}
\end{cases}
\end{equation}
where $x_d \in \Rn$, $u_d \in\Rp$ and $y_d\in\Rm$ can be interpreted as the dual state, dual input and  dual output vectors of the system, respectively.
\end{lemma}
We will see in the subsequent section the importance of the dual system in determining the observability energy functional or observability Gramian for a  QB system because a duality of the energy functionality holds. 

