\def\redQBbal {\ensuremath{{\mathsf{RedQBbal}}}}
\def\redBbal {\ensuremath{{\mathsf{RedBbal}}}}

\def\GramQB {\ensuremath{{\mathsf{TGramQB}}}}
\def\GramB {\ensuremath{{\mathsf{TGramB}}}}

\definecolor{mycolor1}{rgb}{1.00000,0.00000,1.00000}%
\definecolor{mycolor2}{rgb}{0.00000,1.00000,1.00000}%
% \addplot [color=black,solid,forget plot,line width = 2pt]
%\addplot [color=mycolor1,solid,line width = 1.5pt,dashed, mark = triangle*, mark repeat = 17]
%\addplot [color=red,solid,forget plot,line width = 1.1pt,mark = +, mark repeat = 20]
% \addplot [color=mycolor2,solid,line width = 1.1pt,dotted, mark = diamond*, mark repeat = 23]


In this section, we consider MOR of several QB control systems and evaluate the efficiency of the proposed balanced truncation technique (\Cref{algo:BT_qbdae}). For this, we need to solve a number of conventional  Lyapunov equations. In our numerical experiments, we determine low-rank factors of these Lyapunov equations by using the ADI method as proposed in~\cite{benner2014self}. We compare the proposed methodology with the existing moment-matching techniques for QB systems, namely one-sided moment-matching~\cite{morGu09} and  its recent extension to two-sided moment-matching~\cite{morBenB15}. These moment-matching methods aim at approximating the underlying generalized transfer functions of the system. Moreover, we need interpolation points in order to apply the moment-matching methods; thus, we choose $l$ linear $\cH_2$-optimal interpolation points, determined by using \emph{IRKA}~\cite{morGugAB08} on the corresponding linear part. This leads to a reduced QB system of order $\hn = 2l$.  All the
simulations are done on
%an \intel \coretwo~Quad CPU Q9550@2.83GHz 6 MB cache, 4 GB RAM, openSUSE Linux 12.04, \matlab~Version 8.0.0.783(R2012b)64-bit(glnxa64).
\matlab~Version 8.0.0.783(R2012b)64-bit(glnxa64) on a board with 4 \intel ~\xeon ~E7-8837 CPUs with a 2.67-GHz clock speed, 8 Cores each and 1TB of total RAM, openSUSE Linux 12.04.


\subsection{Nonlinear RC ladder}
As a first example, we discuss a nonlinear RC ladder. It is a well-known example and is used as one of the benchmark problems in the community of nonlinear model reduction; see, e.g.,~\cite{bai2002krylov,morBreD10,morGu09,li2005compact,morPhi03}.  A detailed description of the dynamics can be found in the mentioned references; therefore, we omit it for the brevity of the paper.  However, we like to comment on the nonlinearity present in the RC ladder. The nonlinearity arises from the presence of the diode I-V characteristic $i_D := e^{40v_D}{-} v_D {-}1$, where $v_D$ denotes the voltage across the diode. As shown in~\cite{morGu09},  introducing some appropriate new variables allows us to write the system dynamics in the QB form of dimension $n = 2 k$, where $k$ is the number of capacitors in the ladder.

\begin{figure}[!tb]
	\centering
	\setlength\fheight{3cm}
	\setlength\fwidth{6cm}
	\includetikz{RC_Sigma}
	\caption{A RC ladder: decay of the normalized singular values based the truncated Gramians, and the dotted lines show the normalized singular value for $\hn =  10$ and the order of the reduced system corresponding to the normalized singular value  $1e{-15}$.}
	\label{fig:RC_sigma}
\end{figure}

We consider $500$ capacitors in the ladder, leading to a QB system of order $n=1000$. For this particular example, the matrix $A$ is a semi-stable matrix, i.e., $0\subset \sigma(A) $. As a result, the truncated Gramians of the system might not exist; therefore, we replace the matrix $A$ by $A_s := A {-} 0.05I$, where $I$ is the identity matrix, to determine these Gramians. However, note that we project the original system with the matrix $A$ to compute a reduced-order system  but the projection matrices are computed using the Gramians obtained via the shifted matrix $A_s$.  In \Cref{fig:RC_sigma}, we show the decay of the  singular values, determined by the truncated Gramians (with the shifted $A$). We then compute the reduced system of order $\hn = 10$ by using balanced truncation. Also, we determine $5$ $\cH_2$-optimal linear interpolation points and compute reduced-order systems of order $\hn =  10$ via one-sided and two-sided projection methods.


To compare the quality of these approximations, we simulate these systems for the input signals $u_1(t) = 5\left(\sin(2\pi/10) +1\right)$ and $u_2(t) = 10\left( t^2\exp(-t/5)\right)$.  \Cref{fig:RC_timedomain} presents the transient responses and relative errors of the output for these input signals, which shows that balanced truncation outperforms the one-sided interpolatory method; on the other hand, we see that balanced truncation is competitive   to the two-sided interpolatory projection for this example.

\begin{figure}[!tb]
	\centering
	\begin{tikzpicture}
	\begin{customlegend}[legend columns=-1, legend style={/tikz/every even column/.append style={column sep=.5cm}} , legend entries={Original sys., BT, One-sided proj. , Two-sided proj.}, ]
	\addlegendimage{black,line width = 1.1pt}
	\addlegendimage{mycolor1,line width = 1.1pt,mark = triangle*, dashed}
	\addlegendimage{red ,line width = 1.1pt, mark = +}
	\addlegendimage{mycolor2,line width = 1.1pt, dotted, mark = diamond*}
	\end{customlegend}
	\end{tikzpicture}
	\begin{subfigure}[!htb]{\textwidth}
		\centering
		\setlength\fheight{3.5cm}
		\setlength\fwidth{5cm}
		\includetikzf{RC_Input1_response}
		\centering
		\setlength\fheight{3.5cm}
		\setlength\fwidth{5cm}
		\includetikzf{RC_Input1_error}
		\caption{Comparison of the original and the reduced-order systems for $u_1(t) = 5\left(\sin(2\pi/10) +1\right)$. }
	\end{subfigure}
		\begin{subfigure}[!tb]{\textwidth}
			\centering
			\setlength\fheight{3.5cm}
			\setlength\fwidth{5cm}
			\includetikzf{RC_Input2_response}
			\centering
			\setlength\fheight{3.5cm}
			\setlength\fwidth{5cm}
			\includetikzf{RC_Input2_error}
		\caption{Comparison of the original and the reduced-order systems for $u_2(t) = 10\left( t^2\exp(-t/5)\right)$. }
		\end{subfigure}
	\caption{A RC ladder: comparison of reduced-order systems obtained by balanced truncation (BT) and moment-matching methods for  two arbitrary control inputs.}
	\label{fig:RC_timedomain}
\end{figure}

\subsection{One-dimensional Chafee-Infante equation}
As a second example, we consider the one-dimensional Chafee-Infante (Allen-Cahn) equation. This nonlinear system has been widely studied in the literature; see, e.g.,~\cite{chafee1974bifurcation,hansen2012second}, and its model reduction related problem was recently considered in~\cite{morBenB15}. The governing equation, subject to  initial conditions and  boundary control,   have a cubic nonlinearity:
\begin{equation}\label{sys:Chafee_infante}
 \begin{aligned}
  \dot{v} +  v^3 &= v_{xx} + v, & ~~ &(0,L)\times (0,T),&\qquad
  v(0,\cdot)  &= u(t), & ~~ & (0,T),\\
  v_x(L,\cdot) &= 0, & ~~ & (0,T),&
  v(x,0) &= 0, & ~~ & (0,L).
 \end{aligned}
\end{equation}
Here, we  make use of a finite difference scheme and consider $k$ grid points in the spatial domain, leading to a semi-discretized nonlinear ODE. As shown in~\cite{morBenB15}, the smooth nonlinear system can be transformed into a QB system by introducing appropriate new state variables. Therefore, the system~\eqref{sys:Chafee_infante} with the cubic nonlinearity can be rewritten  in the QB form by defining  new variables $w_i = v_i^2$ with derivate  $\dot{w}_i = 2v_i\dot{v}_i$. We observe the response at the right boundary at $x = L$. We use the number of grid points $ k = 500$, which results in a QB system of  dimension $n = 2\cdot500 = 1000$  and  set the length $L = 1$. In \Cref{fig:1D_chafee_singular}, we show the decay of the normalized singular values based on the truncated Gramians of the system.

\begin{figure}[!tb]
\centering
	\setlength\fheight{3cm}
	\setlength\fwidth{6cm}
	\includetikz{Chafee_Sigma}
		\caption{Chafee-Infante equation: decay of the normalized singular values based the truncated Gramians, and dotted line shows the normalized singular value  for $\hn =  20$ and the order of the reduced-order system corresponding to the normalized singular value  $1e{-15}$.}
		\label{fig:1D_chafee_singular}
\end{figure}

We determine reduced systems of order $\hn =   20$ by using balanced truncation, and one-sided and two-sided interpolatory projection methods. To compare the quality of these reduced-order systems, we observe the outputs of the original and reduced-order  systems for two arbitrary control inputs $u(t) = 5t\exp(-t)$ and $u(t) = 30(\sin(\pi t)+1)$ in \Cref{fig:time_domain_CI}.

\begin{figure}[!htb]
        \centering
        \begin{tikzpicture}
     \begin{customlegend}[legend columns=-1, legend style={/tikz/every even column/.append style={column sep=0.5cm}} , legend entries={Original sys., BT, One-sided proj. , Two-sided proj.}, ]
	\addlegendimage{black,line width = 2pt}
	\addlegendimage{mycolor1,line width = 1.5pt,mark = triangle*, dashed}
	\addlegendimage{red ,line width = 1.5pt, mark = +}
	\addlegendimage{mycolor2,line width = 1.5pt, dotted, mark = diamond*}
     \end{customlegend}
 \end{tikzpicture}
 \begin{subfigure}[!b]{\textwidth}
  \centering
		\setlength\fheight{3cm}
	\setlength\fwidth{5.0cm}
	\includetikzf{Chafee_Input2_response}
	\includetikzf{Chafee_Input2_error}
			\caption{Comparison of the original and the reduced-order systems for $u_1(t) = 5\left.t\exp(-t)\right.$. }
			\label{fig:chafee_input2}
 \end{subfigure}
%\end{figure}
%\begin{figure}[!tb]
% \ContinuedFloat
  \begin{subfigure}[!tb]{\textwidth}
  	\centering
  	\setlength\fheight{3cm}
  	\setlength\fwidth{5.0cm}
  	\includetikzf{Chafee_Input1_response}
  	\includetikzf{Chafee_Input1_error}
  				\caption{Comparison of the original and the reduced-order systems for $u_2(t) = 30\left(\sin(\pi t)+1\right)$. }
  \end{subfigure}
 \caption{Chafee-Infante equation: comparison of the reduced-order systems obtained via balanced truncation and moment-matching methods for the inputs $u_1(t) = 5\left(t\exp(-t)\right)$  and $u_2(t) = 30\left(\sin(\pi t)+1\right)$. }
\label{fig:time_domain_CI}
\end{figure}

\Cref{fig:chafee_input2} shows that the reduced systems obtained via balanced truncation and one-sided and two-sided interpolatory projection methods are almost  of the same quality for input $u_1$. But for the input $u_2$, the reduced system obtained via the one-sided interpolatory projection method completely fails to capture the dynamics of the system, while balanced truncation and two-sided interpolatory projection can reproduce the system dynamics with a slight advantage of two-sided projection regarding accuracy.

However, it is worthwhile to mention  that as we increase the order of the reduced system, the two-sided interpolatory projection method tends to produce unstable reduced-order systems. On the other hand, the accuracies of the reduced-order systems obtained by balanced truncation and one-sided moment-matching increase with the order of the reduced systems.

%%%%%%%%%%%%%%%%%%%%%%%%%%%%%%%%%%%%%%%%%%%%%%%
%%%%%%%% FitzHugoNu
\subsection{The FitzHugh-Nagumo (F-N) system}
Lastly, we consider the F-N system, a simplified neuron model of the Hodgkin-Huxley model, which describes activation and deactivation dynamics of a spiking neuron. This model has been considered in the framework of POD-based~\cite{morChaS10} and moment-matching model reduction techniques~\cite{morBenB12a}. The dynamics of the system is governed by the following nonlinear coupled differential equations:
\begin{equation}
\begin{aligned}
\epsilon v_t(x,t) & =\epsilon^2v_{xx}(x,t) + f(v(x,t)) -w(x,t) + q,\\
w_t(x,t) &= hv(x,t) -\gamma w(x,t) + q
\end{aligned}
\end{equation}
with a nonlinear function $f(v(x,t)) = v(v-0.1)(1-v)$  and the initial and boundary conditions:
\begin{equation}
\begin{aligned}
v(x,0) &=0, &\quad w(x,0)&=0, \quad & &x\in [0,L],\\
v_x(0,t) &= i_0(t), & v_x(1,t) &= 0, & &t\geq 0,
\end{aligned}
\end{equation}
where $\epsilon = 0.015,~h=0.5,~\gamma = 2,~q = 0.05$. We set the length $L = 0.2$. The stimulus $i_0$ acts as an actuator, taking the values $i_0(t) = 5\cdot 10^4t^3\exp(-15t)$, and the variables $v$ and $w$ denote the voltage and recovery voltage, respectively. We also assume the same outputs of interest as considered in~\cite{morBenB12a}, which are $v(0,t)$ and $w(0,t)$. These outputs describe nothing but the limit cyclic at the left boundary. Using a finite difference discretization scheme, one can obtain a system with two inputs and two outputs of  dimension $2 k$ with cubic nonlinearities, where $k$ is the number of degrees of freedom.   Similar to the previous  example, the F-H system can also be transformed into a QB system of dimension $n = 3 k$ by introducing a new state variable $z_i = v_i^2$. We set $k = 500$,  resulting in a QB system of order $n = 1500$.  \Cref{fig:Fitz_sigma} shows the decay of the singular values  based on the truncated Gramians for the QB system.
\begin{figure}[tb!]
        \centering
  \centering
  	\setlength\fheight{3cm}
	\setlength\fwidth{5cm}
		\includetikz{Fitz_Sigma}
\caption{Decay of the normalized singular values based the truncated Gramians of the system for the F-N example, and the dotted lines show the normalized singular value for $\hn =  20$ and the order of the reduced system corresponding to the normalized singular value  $1e{-15}$.}
	\label{fig:Fitz_sigma}

 %\caption{The figure shows the decay of the singular values (left-side), and the response $v(t)$ and $w(t)$ at the left boundary (right-side).}
\end{figure}

Furthermore, we determine reduced-order systems of order $\hn =  20 $ by using balanced truncation and moment-matching methods. We observe that the reduced-order systems, obtained via the moment-matching methods with linear $\cH_2$-optimal interpolations, both one-sided and two-sided, fail to capture the dynamics and limit cycles. We made several attempts to adjust the order of the reduced systems; but, we were unable to determine a stable reduced-order system via these methods with linear $\cH_2$-optimal points which could replicate the dynamics. Contrary to these methods,  the balanced truncation replicates the dynamics of the system faithfully as  can be seen in \Cref{fig:Fitz_output}.   Note that the reduced-order model reported in \cite{morBenB12a} was obtained using higher-order moments in a trial-and-error fashion but cannot be reproduced by an automated algorithm. As the dynamics of the system produces limit cycles for each spatial variable $x$, we, therefore, plot the solutions $v$ and $w$ over the
spatial domain $x$, which is also captured by the reduced-order system very well.
%\definecolor{mycolor1}{rgb}{0.00000,1.00000,1.00000}%
\begin{figure}[!tb]
        \centering
        \begin{center}
         \begin{tikzpicture}
    \begin{customlegend}[legend columns=0, legend style={/tikz/every even column/.append style={column sep=1cm}} , legend entries={Original system,~ Reduced system via balanced truncation }, ]
    \addlegendimage{red}
    %\addlegendimage{black,solid,only marks,mark = *, mark size = 1.3}
    \addlegendimage{mycolor2,only marks, mark = x}
    \end{customlegend}
\end{tikzpicture}
\end{center}
\begin{subfigure}[t]{0.49\textwidth}
	\centering
	\setlength\fheight{2.5cm}
	\setlength\fwidth{4cm}
	\includetikz{Fitz_time_domain}
	\caption{The response $v(t)$ and $w(t)$ at the left boundary.}
	\label{fig:Fitz_output}
\end{subfigure}
 \begin{subfigure}[t]{0.45\textwidth}
  \centering
  	\setlength\fheight{2.5cm}
	\setlength\fwidth{4.cm}
		\includetikz{Fitz_full_state}
{\caption{ Limit-cycles.}}
 \end{subfigure}
 \caption{FitzHugh-Nagumo system: comparison of the response at the left boundary and the limit cycle behavior of the original system and the reduced-order (balanced truncation) system. The reduced-order systems determined by moment-matching methods were unable to produce these limit cycles.}
 \label{fig:fitz_limit}
\end{figure}


