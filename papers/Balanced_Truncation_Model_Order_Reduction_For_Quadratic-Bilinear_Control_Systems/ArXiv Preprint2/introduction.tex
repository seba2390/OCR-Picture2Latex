%\sout{Numerical analysis of large-scale dynamical systems becomes very challenging as their governing ordinary differential equations (ODEs), or partial differential equations (PDEs), or both, are discretized very fine over the interested spatial domain. }
Numerical simulations are considered to be a primary tool in studying dynamical systems, e.g., in prediction and control studies. High-fidelity modeling is an  essential step to gain deep insight into the behavior of complex  dynamical systems.  Even though computational resources have been developing extensively over the last few decades,  fast numerical simulations of such high-fidelity systems, whose number of state variables can easily be of order $\cO(10^5){-}\cO(10^6)$,  are still a huge computational burden. This makes the usage of these large-scale systems very difficult and inefficient, for instance,  in optimization and control design. One approach to mitigate this problem is \emph{model order reduction} (MOR). MOR seeks to substitute these large-scale dynamical systems by  low-dimensional (reduced-order) systems such that the input-output behaviors of  both original and reduced-order systems are close enough, and the reduced-order systems preserve some important properties, for instance, stability and passivity of the original system.

MOR techniques and strategies for linear systems are well-established and  are widely applied in various application areas, see, e.g.,~\cite{morAnt05,morBenMS05,morSchVR08}. In many applications, where the dynamics are governed by nonlinear PDEs, such as Navier-Stokes equations, Burgers' equations, a linear system can also be obtained via  linearization of the system around a suitable expansion point, e.g., the steady-state solution.   Notwithstanding the linearized system captures the dynamics very well locally. However, as it moves away from the expansion point, the linearized system might not be able to capture the system dynamics accurately. Therefore, there is a general need to take nonlinear terms into consideration, thus resulting in  a more accurate system. Consider a nonlinear system of the form
\begin{equation}\label{eq:NonlinearSys}
\begin{aligned}
 \dot{x}(t) &=  f(x(t)) + g(x(t),u(t)), \\
 y(t) &= h(x(t),u(t)), \qquad x(0) = x_0,
\end{aligned}
 \end{equation}
where $f:\Rn \rightarrow \Rn$, $g:\Rn\times\Rm \rightarrow \Rn$ and $h:\Rn\times \Rm \rightarrow \Rp$ are nonlinear state-input evolution functions, and $x(t)\in\Rn, u(t)\in\Rm$ and $y(t)\in \Rp$ denote the state, input and output vectors of the system at time $t$, respectively.  Also, we consider a fixed initial condition $x_0$ of the system. However,   without loss of generality,  we assume  the zero initial condition, i.e., $x(0) =0$. In  case $x(0) \neq 0$, one can transform the system by defining new appropriate state variables as $\tx(t) = x(t) - x_0$  to ensure  a zero initial condition of the transformed system, e.g., see \cite{morBauBF14}. The main goal of MOR is to construct a low-dimensional system, having  a similar form as the system~\eqref{eq:NonlinearSys}
\begin{equation}\label{eq:NonlinearSys_Red}
\begin{aligned}
 \dot{\hx}(t) &=  \hf(\hx(t)) + \hg(\hx(t),u(t)), \\
 \hy(t) &= \hh(\hx(t),u(t)), \qquad \hx(0) = 0,
\end{aligned}
 \end{equation}
in which $\hf:\R^{\hn} \rightarrow \R^{\hn}$, $\hg:\R^{\hn}\times\Rm \rightarrow \R^{\hn}$ and $\hh:\R^{\hn}\times \Rm \rightarrow \Rp$ with $\hn \ll n$ that fulfills our desired requirements.

MOR techniques  for general nonlinear systems, namely trajectory-based MOR techniques, have been widely applied in the literature to determine  reduced-order systems for nonlinear systems; see, e.g.,~\cite{morAstWWetal08,morChaS10,morHinV05}. The proper orthogonal decomposition (POD) method is a very powerful trajectory-based MOR technique, which depends on a Galerkin projection $\cP = VV^T$, where $V$ is a projection matrix such that $x(t) \approx V\hx(t)$. The nonlinear functions $\hf(\hx)$ can be given as $\hf(\hx(t)) = V^Tf(V\hx(t))$, and  similar expressions  can also be derived for $\hg(\hx(t),u(t))$ and $\hh(\hx(t),u(t))$. This method preserves the structure of the original system in the reduced-order system, but still, the reduced-order system requires the computation of the nonlinear functions on the full grid. This may obstruct the success  of MOR;  however, there are many new advanced methodologies such as the empirical interpolation method (EIM), the discrete empirical  interpolation method (DEIM), the best point interpolation method (BPIM),  to perform the computation of the nonlinear functions cheaply and quite accurately. For details, we refer to~\cite{morBarMNetal04,morChaS10,drmac2016new,morGreMNetal07}.

Another popular trajectory-based MOR technique is based on trajectory piecewise linearization (TPWL)~\cite{morRew03}, where nonlinear functions are replaced by a weighted combination of linear systems. These linear systems  can then  be reduced by applying  well-established MOR techniques for linear systems such as balanced truncation or the interpolation-based iterative method (IRKA); see, e.g.,~\cite{morAnt05,morGugAB08}. In recent years, reduced basis methods have been successfully applied to nonlinear systems to obtain  reduced-order systems~\cite{morBarMNetal04,morGreMNetal07}. In spite of all these, the trajectory-based MOR techniques have the drawback  of being input dependent. This makes the obtained reduced-order systems inadequate to  control applications, where the input function may vary significantly from any used training input. 

In this article, we consider a certain class of  nonlinear control systems, namely quadratic-bilinear (QB) control systems. The advantage of considering this special class of  nonlinear systems is that systems, containing smooth mono-variate nonlinearities such as exponentials, polynomials, trigonometric functions, can also be rewritten in the QB form by introducing some new variables in the state vector~\cite{morGu09}.  Note that this transformation is exact, i.e., it requires no approximation and does not introduce any error, but this transformation may not be unique. 

Related to MOR for QB systems, the idea of one-sided  moment-matching has been extended from linear or bilinear systems to QB systems; see, e.g.,~\cite{bai2002krylov,feng2004direct,morGu09,li2005compact,morPhi03}, where a reduced system is determined by capturing the input-output behavior of the original system,   given by generalized transfer functions. More recently, this has been extended to two-sided moment-matching in~\cite{morBenB15}, ensuring more  moments to be matched, for a given order of the reduced system. Despite these methods have evolved as an effective MOR technique for nonlinear systems in recent times, shortcomings of these methods are: how to choose an appropriate order of the reduced system and how to select good interpolation points. Moreover, the applicability of the two-sided moment-matching method~\cite{morBenB15} is limited to single-input single-output QB systems, and also the stability of the obtained reduced-order system is a major issue in this method.

In this article, our focus rather lies on balancing-type MOR techniques for QB systems. This technique mainly depends on controllability and observability energy functionals, or in other words, Gramians of the system.  This method is presented for linear systems, e.g., in~\cite{morAnt05,morMoo81}, and later on, a theory of balancing for general nonlinear systems is developed in a sequence of papers~\cite{morfuji10,morgray2001,morSch93,morsche1996h,morsche94nor}.  In the general nonlinear case, the balancing requires  the solutions of the state-dependent nonlinear Hamilton-Jacobi equation which are, firstly, very expensive to solve for large-scale dynamical systems; secondly, it is not  straightforward  to use them in the MOR context. Along with these, it may happen that the reduced-order systems, obtained from nonlinear balancing, do not preserve the structure of the nonlinearities in the system. However, for some  weakly nonlinear systems,  the so-called bilinear systems, reachability and observability Gramians have been  studied in~\cite{moral1994,morBenD11,bennertruncated,morcondon2005,enefungray98}, which are solutions to generalized algebraic Lyapunov equations. Moreover, these Gramians, when used to define appropriate  quadratic forms, approximate  energy functionals of bilinear systems (in the neighborhood of the origin), see~\cite{morBenD11, bennertruncated}

Moving in the direction of balancing-type  MOR for QB systems, our first goal is to come up with reachability and observability Gramians for these systems, which are state-independent  matrices and suitable for the MOR purpose. In addition to this, we need to show  how the Gramians  relate to the energy functionals of the QB systems and provide interpretations of reachability and observability of the  system with respect to these Gramians. To this end, in the subsequent section, we review  background material  associated with energy functionals and a duality of the nonlinear systems, which  serves as the basis for the rest of the paper. In \Cref{sec:gramians}, we propose the  reachability Gramian  and its truncated version for QB systems based on the underlying Volterra series of the system. Additionally, we determine the observability Gramian and its truncated version based on the dual system associate to the QB system. Furthermore, we establish  relations between the solutions of a certain type of quadratic Lyapunov equations and these Gramians.  In \Cref{sec:energyfunctionals},  we develop the connection  between the proposed  Gramians  and the energy functionals of the QB systems and  reveal their relations to reachability and observability of the system. Consequently, we utilize these Gramians for balancing of QB systems, allowing us to determine those states that are hard to control as well as hard to observe. Truncation of such states leads to reduced systems.  In \Cref{sec:computational}, we discuss the related computational issues and advantages of the truncated version of Gramians in the MOR framework. We further discuss the stability of these reduced systems.  In \Cref{sec:numerics}, we test the efficiency of the proposed balanced truncation MOR technique for various semi-discretized nonlinear PDEs and compare it with the existing moment-matching techniques for the QB systems.




