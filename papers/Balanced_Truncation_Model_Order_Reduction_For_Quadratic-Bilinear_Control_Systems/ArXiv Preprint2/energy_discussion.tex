%\newpage \begin{proof}
%Using the definition of the observability energy functional, we have
%\begin{flalign*}
% L_o(x_0,u) &= \dfrac{1}{2}\int_0^\infty \|Cx(t)\|_2^2dt = \dfrac{1}{2}\int_0^\infty x^T(t)C^TCx(t)dt.
%\end{flalign*}
%Substituting for $C^TC$ from~\eqref{eq:obser_lyap}, we obtain
%\begin{flalign*}
% L_o(x_0,u)&= \dfrac{1}{2}\int_0^\infty \left(-2x^TQAx -  x^T\cH^{(2)} P\otimes Q \big(\cH^{(2)}\big)^T x - \sum_{k=1}^mx^TN_k^TQN_kx \right)  dt.
% \end{flalign*}
%Next, we substitute for $Ax$ from~\eqref{eq:Quad_bilin_Sys} (with $u = 0$) to have
%\begin{flalign*}
% L_o(x_0,u)&= \dfrac{1}{2}\int\limits_0^\infty \Big(-2x(t)^TQ\dot{x}(t)  + 2x(t)^TQHx(t)\otimes x(t) \\
% &\qquad -x(t)^T\cH^{(2)} \left(P\otimes Q\right) \big(\cH^{(2)}\big)^T x(t)  - \sum_{k=1}^mx(t)^TN_k^TQN_kx(t) \Big)dt \\
% &= \dfrac{1}{2}\int_0^\infty - \dfrac{d}{dt}(x(t)^TQx(t))dt + \dfrac{1}{2}\int_0^\infty x(t)^T\Big( QH (I\otimes x(t)) + QH (x(t)\otimes I) \\
% &\quad -\cH^{(2)} (P\otimes Q)  \big(\cH^{(2)}\big)^T- \sum_{k=1}^mN_k^TQN_k \Big) x(t) dt.
% %&= \dfrac{1}{2}\int_0^\infty - \dfrac{d}{dt}(x^T(t)Qx(t))dt + \dfrac{1}{2}\int_0^\infty ( 2x^TQHx(t)\otimes x(t)  -x^T\cH Q_l\otimes P_l \cH^T) x dt
% \end{flalign*}
% This gives
% \begin{flalign*}
% L_o(x_0,u) - \dfrac{1}{2}x_0^TQx_0 &=   - \dfrac{1}{2}\int_0^\infty x^T(t)R(x) x(t)dt,
%\end{flalign*}
%where
%\begin{align*}
%R(x) &:=  -\Big( QH (I\otimes x) + QH (x\otimes I) -\cH^{(2)} (P\otimes Q)  \big(\cH^{2}\big)^T- \sum_{k=1}^mN_k^TQN_k \Big) , \\
%%&\qquad= -\Big( QH (I\otimes x) + QH (x\otimes I)  -QH (I\otimes P)  H^T- \sum_{k=1}^mN_k^TQN_k \Big) ,
%\end{align*}
%{\color{blue}
%Since $$\cH^{(2)} (P\otimes Q)  \big(\cH^{2}\big)^T + \sum_{k=1}^mN_k^TQN_k=:\cK$$ is positive definite matrix, we therefore have $$x^T\cK x \geq  \sigma_{min}(\cK)\|x\|^2.$$ Thus, $-x^T\cK x \leq  -\sigma_{min}(\cK)\|x\|^2$. Hence,
%\begin{align*}
%x^TR(x)x &\leq  x^T\Big( 2QH (I\otimes x)\Big)x -\sigma_{min}(\cK)\|x\|^2 \\ 
%&\leq  2\|Q\|||H\| \|x\|^3 -\sigma_{min}(\cK)\|x\|^2,,
%\end{align*} 
%As a result,
%$x^TR(x)x \leq 0$ if
%\begin{equation}\label{cond_x}
% \|x\| \leq  \dfrac{\sigma_{min}(\cK)}{2\|Q\|||H\|}
%\end{equation}
%
%}
%
%%For a  ball of  sufficiently small radius $\delta$ around $0$ in which $\|x\| < \delta$ and  small input function $u$ in $L_2$-norm, $R(x) \geq 0$. This can be argued based on the term $\cH^{(2)} P\otimes Q  \big(\cH^{(2)}\big)^T + \sum_{k=1}^mN_k^TQN_k$ which is positive semidefinite; whereas the other two terms are the functions of the state and input  whose dominance can be reduced by choosing sufficiently small input function and the ball for $\|x\| \in \tW$. This implies that the right side is negative at least under these assumptions. Therefore,
%
%If the initial condition $x_0$ lies in a small neighborhood of $0$, i.e., $x_0\in \tW$, ensuring that $x(t)$ satisfies \eqref{cond_x} for all time $t$. Thus, $x^T(t)R(x)x(t) \leq 0$ for all $t$. Thus, we have
%
%%\begin{align*}
%% L_o(x_0) - \dfrac{1}{2} x_0^TQx_0 &= \underset{\begin{subarray}{c}
%% u\in L^2[0,\infty), \\[0pt]\\ \|u\|_{L^2} \leq \alpha\\
%% x(-\infty)=0,~x(0) = x_0
%%  \end{subarray} }{\max}  L_o(x_0,u) - \dfrac{1}{2} x_0^TQx_0 \\
%% &= \underset{\begin{subarray}{c}
%% u\in L^2[0,\infty), \\[0pt]\\ 
%% x(0)=x_0,~x(\infty) = 0
%%  \end{subarray}}{\max} -\dfrac{1}{2}\int_0^\infty x^T(t)R(x,t) x(t)dt \leq 0.
%%\end{align*}
%%Thus,
% $L_o(x_0) \leq  \dfrac{1}{2} x_0^TQx_0 $ when $x_0 \in \tW(0)$.
%\end{proof}
%\newpage

{%\color{cyan}
%\newpage 
\begin{proof}
	Using the definition of the observability energy functional, we have
	\begin{flalign*}
	L_o(x_0) &= \dfrac{1}{2}\int_0^\infty \|Cx(t)\|_2^2dt = \dfrac{1}{2}\int_0^\infty x^T(t)C^TCx(t)dt.
	\end{flalign*}
	Substituting for $C^TC$ from~\eqref{eq:obser_lyap}, we obtain
	\begin{flalign*}
	L_o(x_0,u)&= \dfrac{1}{2}\int_0^\infty \left(-2x^TQAx -  x^T\cH^{(2)} P\otimes Q \big(\cH^{(2)}\big)^T x - \sum_{k=1}^mx^TN_k^TQN_kx \right)  dt.
	\end{flalign*}
	Next, we substitute for $Ax$ from~\eqref{eq:Quad_bilin_Sys} (with $u = 0$) to have
	\begin{flalign*}
	L_o(x_0)&= \dfrac{1}{2}\int\limits_0^\infty \Big(-2x(t)^TQ\dot{x}(t)  + 2x(t)^TQHx(t)\otimes x(t) \\
	&\qquad -x(t)^T\cH^{(2)} \left(P\otimes Q\right) \big(\cH^{(2)}\big)^T x(t)  - \sum_{k=1}^mx(t)^TN_k^TQN_kx(t) \Big)dt \\
	&= \dfrac{1}{2}\int_0^\infty - \dfrac{d}{dt}(x(t)^TQx(t))dt + \dfrac{1}{2}\int_0^\infty x(t)^T\Big( QH (I\otimes x(t)) + QH (x(t)\otimes I) \\
	&\quad -\cH^{(2)} (P\otimes Q)  \big(\cH^{(2)}\big)^T- \sum_{k=1}^mN_k^TQN_k \Big) x(t) dt.
	%&= \dfrac{1}{2}\int_0^\infty - \dfrac{d}{dt}(x^T(t)Qx(t))dt + \dfrac{1}{2}\int_0^\infty ( 2x^TQHx(t)\otimes x(t)  -x^T\cH Q_l\otimes P_l \cH^T) x dt
	\end{flalign*}
	This gives
	\begin{flalign*}
	L_o(x_0) - \dfrac{1}{2}x_0^TQx_0 &=   - \dfrac{1}{2}\int_0^\infty x^T(t)R(x) x(t)dt,
	\end{flalign*}
	where
	\begin{align*}
	R(x) &:=  -\Big( QH (I\otimes x) + QH (x\otimes I) -\cH^{(2)} (P\otimes Q)  \big(\cH^{2}\big)^T- \sum_{k=1}^mN_k^TQN_k \Big).
	%&\qquad= -\Big( QH (I\otimes x) + QH (x\otimes I)  -QH (I\otimes P)  H^T- \sum_{k=1}^mN_k^TQN_k \Big) ,
	\end{align*}
	{
		Since $P>0$, $Q> 0$ and at least one of matrices $H$ and $N_k$, for $k\in 1,\ldots m$ is of full rank,  $$\cH^{(2)} (P\otimes Q)  \big(\cH^{2}\big)^T + \sum_{k=1}^mN_k^TQN_k=:\cK $$ is symmetric positive definite matrix. Next, we consider the choleskey factor of $\cK = \cU \cU^T$ where $\cU \geq 0$. We then perform a transformation of the state $x(t)$ as $\tx = \cU^T x(t)$, thus resulting in
			\begin{flalign*}
			L_o(x_0) - \dfrac{1}{2}x_0^TQx_0 &=   \dfrac{1}{2}\int_0^\infty \tx^T(t)\tR(\tx,u) \tx(t)dt,
			\end{flalign*}
			where
			\begin{align*}
			\tR(\tx,u) &:=  \Big( -I + 2\cU^{-1}QH (U^{-T}\otimes U^{-T}\tx)   \Big) =: \Big( -I + F(x)   \Big) .  \\
		\end{align*}
		where $F(x) = 2\cU^{-1}QH (U^{-T}\otimes U^{-T}\tx) $. Furthermore, we know that if $\{\lambda_1,\ldots,\lambda_n\}$ are the eigenvalues of $\cA$, then the eigenvalues of $-I + \cA$ are $1-\lambda_i$, $i=1,\ldots,n$. Using this property, we can conclude the that there exists a neighborhood of the origin for the initial condition $x_0$ for which the solution trajectory $x(t)$  ensures 
		\begin{equation}\label{eq:eign_fx}
		\sigma(F(x)) <1. 
		\end{equation}
Concluding,	if the initial condition $x_0$ lies in a small neighborhood of $0$, i.e., $x_0\in \tW$, ensuring that $x(t)$ satisfies \eqref{eq:eign_fx} for all time $t$. This means that $\tR(\tx)$ is a negative definite matrix. Therefore, $\tx^T(t)\tR(\tx)\tx(t) < 0$ for all $t$; thus we obtain
$L_o(x_0) <  \dfrac{1}{2} x_0^TQx_0 $ when $x_0 \in \tW(0)$.
} 
\end{proof}
}