We start by establishing the conditions under which  the Gramians approximate the energy functionals of the QB system, in the quadratic forms.
\subsection{Comparison of energy functionals with  Gramians}\label{subsec:energy}
By using \Cref{thm:energy_function}, we obtain the following nonlinear partial differential equation, whose solution gives the controllability energy functional for the QB system:
%Let us make use of  \Cref{thm:energy_function} to have the following partial differential equation, whose solution gives the controllability energy functional for the QB system:
\begin{equation}\label{eq:control_energy_QB}
\begin{aligned}
 &\dfrac{\partial L_c}{\partial x}(Ax + H~x\otimes x) +  (Ax + H~x\otimes x)^T\dfrac{\partial L_c}{\partial x}^T  \\
 &\qquad + \dfrac{\partial L_c}{\partial x}\left(\bbm N_1,\ldots, N_m \ebm (I_m\otimes x) + B\right)\left(\bbm N_1,\ldots, N_m \ebm (I_m\otimes x) + B\right)^T \dfrac{\partial L_c}{\partial x} ^T = 0.
\end{aligned}
\end{equation}
 Unlike in the case of linear systems, the controllability energy functional $L_c(x)$ for nonlinear systems cannot be expressed as a simple quadratic form, i.e., $L_c(x) = x^T\tP^{-1}x$, where $\tP$ is a constant matrix. 
 
 For nonlinear systems, the energy functionals are rather complicated nonlinear functions, depending on the state vector. Thus, we aim at providing some bounds between the quadratic form of the proposed Gramians for QB systems and energy functionals. For the controllability energy functional, we  extend the reasoning given in \cite{morBenD11,bennertruncated} for bilinear systems.
 %, and for observability energy functionals, we follow the line of reasoning, given in~\cite{enefungray98}. 
 
\begin{theorem}\label{thm:con_bound}
Consider a controllable QB system~\eqref{eq:Quad_bilin_Sys}  with a stable matrix $A$. Let $P>0$ be  its reachability Gramian which is  the unique definite solution of the quadratic Lyapunov equation~\eqref{eq:cont_lyap}, and $L_c(x)$ denote the controllability energy functional of the QB system, solving~\eqref{eq:control_energy_QB}.  Then, there exists a neighborhood $W$ of $0$ such that
 % Let $P$ be  the controllability Gramian of QB system which is  the unique definite solution of the quadratic Lyapunov equation~\eqref{eq:cont_lyap} and $L_c$ be the controllability energy functional of the QB system~\eqref{eq:QBsys}, which solves~\eqref{}. Then, there exists a neighborhood $W$ of $0$ such that
	  \begin{equation*}
	  L_c(x) \geq \dfrac{1}{2}x^T P^{-1}x, ~\mbox{where}~x\in W(0).
	  \end{equation*}
\end{theorem}
\begin{proof}
	Consider a state $x_0$ and let a control input $u=u_{0}:(-\infty,0]\rightarrow \Rm$, which minimizes the input energy in the definition of $L_o(x_0)$ and steers the system from $0$ to $x_0$. Now, we consider the time-varying homogeneous nonlinear differential equation
	\begin{equation}\label{eq:phi_TV}
	\dot \phi = \left(A + H(\phi\otimes I) +\sum_{k=1}^mN_ku_k(t) \right) \phi =: A_{u}\phi(t),
	\end{equation}
 and its fundamental solution $\Phi_{u}(t,\tau)$. The system~\eqref{eq:phi_TV} can thus be interpreted as a time-varying system. The reachability Gramian of the time-varying control system~\cite{shokoohi1983linear,verriest1983generalized} $\dot x = A_{u}x(t) + Bu(t)$ is  given by
\begin{equation*}
 P_u = \int_{-\infty}^0\Phi(0,\tau)BB^T\Phi(0,\tau)^Td\tau.
\end{equation*}
The input $u$ also steers the time-varying system from $0$ to $x_0$; therefore, we have
$$\|u\|_{L_2}^2 \geq \dfrac{1}{2}x^TP_u^{-1}x.$$
An alternative way to determine $P_u$ can be given by
\begin{equation*}
 P_u = \int^{\infty}_0\tilde\Phi(t,0)^TBB^T\tilde\Phi(t,0)dt,
\end{equation*}
where $\tilde\Phi$ is the fundamental solution of the following differential equation
 \begin{equation}\label{eq:tildephi_TV}
  \dot {\tilde\Phi} = \left(A^T + \cH^{(2)}(x(-t)\otimes I) +\sum_{k=1}^mN_k^Tu_k(-t) \right) \tilde\Phi~~~ \mbox{with}~~~\Phi(t,t) = I,
 \end{equation}
 and $x(t)$ is the solution of
 $$\dot x(t) = Ax(t) + H(x\otimes x) + \sum_{k=1}^mN_kx(t)u_k(t) + Bu(t).$$
Then, we define $\eta(t)$, satisfying $\eta(t) = \tilde\Phi(t,0)x_0$.  Since it is assumed that the QB system is controllable,  the state $x_0$ can be reached by using a finite input energy, i.e., $\|u\|_{L_2} <\infty$. Hence, the input $u(t)$ is a  square-integrable function over $t \in (-\infty,0]$ and so is $x(t)$. This implies that $\lim\limits_{t\rightarrow\infty}\eta(t) \rightarrow 0$, provided $A$ is stable. Thus, we have
\begin{align*}
 x_0^TPx_0 &= -\int_{0}^\infty \dfrac{d}{dt} \left(\eta(t)^TP\eta(t)\right)dt \allowdisplaybreaks\\
 & =-\int_0^\infty  \eta(t)^T\left( \left(A + H(x(-t)\otimes I) +\sum_{k=1}^mN_ku_k(-t) \right)P  \right. \\
 & \qquad \left. + P\left( A^T + \cH^{(2)}(x(-t)\otimes I) +\sum_{k=1}^mN_k^Tu_k(-t)\right) \right)\eta(t)dt \allowdisplaybreaks \\
  & =-\int_0^\infty  \eta(t)^T \left(AP + PA^T + H(P\otimes P)H^T + \sum_{k=1}^mN_kPN_k^T \right) \eta(t)  \\
  & \qquad+ \Bigg( H(P\otimes P)H^T - H(x(-t)\otimes I)P -P \cH^{(2)}(x(-t)\otimes I)   \\
 &\qquad    +  \sum_{k=1}^m\left(N_kPN_k -PN_k^Tu_k(-t) -N_k^TPu_k(-t) \right) \Bigg)\eta(t)dt.
 \end{align*}
Now, we have
\begin{multline*}
  -\int_0^\infty  \eta(t)^T \left(AP + PA^T + H(P\otimes P)H^T + \sum_{k=1}^mN_kPN_k^T \right) \eta(t) \\ =\int_0^\infty  \eta(t)^T  BB^T 
   \eta(t) = x_0^TP_ux_0.
\end{multline*}
Hence, if
\begin{multline}\label{eq:HNP_rel}
 \int_0^\infty \eta(t)^T \bigg( H(P\otimes P)H^T - H(x(-t)\otimes I)P -P \cH^{(2)}(x(-t)\otimes I)   \\
   +  \sum_{k=1}^m\left(N_kPN_k -PN_k^Tu_k(-t) -N_k^TPu_k(-t) \right) \bigg)\eta(t)dt \geq 0,
\end{multline}
then $x_0^TPx_0 \geq x_0^T P_ux_0$.
% which holds for all states which are reachable and the relation \eqref{eq:HNP_rel} holds.   This implies $x_0^TP^{-1}x_0 \leq x_0^T P_u^{-1}x_0 = E_c(x_0)$ (cf. \cite[Thm. 7.7.4]{horn2012matrix}). 
Further, if  $x_0$ lies in a small ball   $W$ in the neighborhood of  the origin, i.e., $x_0\in W(0)$, then a small input $u$  is sufficient to  steer the system from $0$ to $x_0$ and  $x(t) \in W(0) $ for $t \in (-\infty,0]$ which ensures that the relation~\eqref{eq:HNP_rel} holds for all $x_0 \in W(0)$.  Therefore, we have $x_0^TP^{-1}x_0 \leq x_0^T P_u^{-1}x_0$ if $x_0 \in W (0) $.
\end{proof}
 %Nevertheless, it is shown in~\cite{enefungray98} that for sufficiently small  $\|x\|$  and $x\neq 0$, the gradients of $L_c$ can be given by
% \begin{equation}\label{eq:grad_contr}
%  \dfrac{\partial}{\partial x}L_c(x) = \tP(x)^{-1}x.
% \end{equation}
% Substituting for $\tfrac{\partial}{\partial x}L_c$ from~\eqref{eq:grad_contr} in~\eqref{eq:control_energy_QB}, we obtain an equivalent equation in $\tP(x)$ as follows:
% \begin{equation*}
%  \left(A+H\left(x\otimes I\right)\right)\tP(x) + \tP(x)\left(A^T + (x^T\otimes I)\right) + (Nx + B)(Nx + B)^T = 0,
% \end{equation*}
% which is a very difficult task to solve for large-scale systems. However, a possible remedy to this problem  is to replace $\tP(x)$ by a constant matrix $\mathbb P$ such that the following lower bound holds:
% \begin{equation}
%  L_c(x) > x^T\mathbb P^{-1}x
% \end{equation}
% at least in the neighborhood of $x = 0$. This can provide us the worst-case scenario about the states which produce a lot energy. This means that these states are hard to reach.
% The idea has been imposed to bilinear systems in~\cite{enefungray98}, where the above bound is derived for the controllability energy functional $L_c$ in the neighborhood of $0$ and $\mathbb P$ is assumed to the controllability Gramian of the bilinear system. We here aim to show that the similar lower bound for the controllability energy functional for QB systems, can be given in terms of the proposed controllability Gramian.
% \begin{theorem}\label{thm:con_bound}
%  Let the system~\eqref{eq:Quad_bilin_Sys}, having a stable matrix $A$, be asymptotically reachable from $0$ to some neighborhood $W$ of $0$. Also, assume that $L_c$ has an analytic solution of~\eqref{eq:control_energy_QB} on $W$. If the Gramian $P$ is computed as shown in Theorem~\Cref{thm:con_gram}, then there exists a neighborhood $\hW$ of $0$ contained in $W$ such that
%  \begin{equation*}
%      L_c(x) >\tfrac{1}{2}x^T P^{-1}x.
%  \end{equation*}
% \end{theorem}
% \begin{proof}
% This proof follows the similar steps as it is done for bilinear systems in~\cite{enefungray98}. However, for the sake of completeness, we prove the bound for the system.  The stability and asymptotic reachability assumptions give us $L_c(0)=0$ and $\tfrac{d L_c}{dx}(0) = 0$, and the analyticity assumption on $L_c$ indicates that there exists an open ball $B_\delta \subset W$  of radius $\delta >0$, where the following holds:
%  \begin{equation}\label{eq:cont_ener_0}
%   L_c(x) = \dfrac{1}{2}x^T\dfrac{\partiaL_2 L_c}{\partial x^2}(0)x + \cO(\|x\|^3).
%  \end{equation}
% Taking the derivate of~\eqref{eq:grad_contr} with respect to $x$ yields
%  \begin{equation*}
%   \dfrac{\partial ^2 L_c } {\partiaL_2 x}(x) =  \tP^{-1}(x) + x^T \dfrac{\partial \tP^{-1}} {\partial x}(x),
%  \end{equation*}
% which gives us
%  \begin{equation*}
%   \dfrac{\partial ^2 L_c } {\partiaL_2 x}(0) =  \tP^{-1}(0),
%  \end{equation*}
% where $\tP(0)$ is the solution of the following Lyapunov equation:
% \begin{equation}\label{eq:linear_lyp}
%  A\tP(0) + \tP(0)A^T + BB^T = 0.
% \end{equation}
% Substituting for $\tfrac{\partial ^2 L_c } {\partiaL_2 x}(0)$ in~\eqref{eq:cont_ener_0}, we get
%  \begin{equation}\label{eq:cont_ener_1}
%   L_c(x) = \dfrac{1}{2}x^T\tP^{-1}(0)x + \cO(\|x\|^3).
%  \end{equation}
% Now, we define the difference of the controllability Gramian of system $P$ and $\tP(0)$. This yields
% \begin{equation*}
%  P-\tP(0) = \int_0^\infty e^{At}H\left(P\otimes P\right) H^T e^{A^Tt}dt  + \int_0^\infty e^{At}N P N^T e^{A^Tt}dt.
% \end{equation*}
% If either $H$, or $N$, or both, are full  rank matrices, then it is clear that $P > \tP(0)$. Equivalently, it indicates $P^{-1} < \tP^{-1}(0)$, provided $P$ and $\tP(0)$ are positive definite matrices. Now, we introduce the following real number as in~\cite{enefungray98}:
% \begin{align*}
%  \lambda(\delta) &= \underset{x\in B_\delta}{\text{inf}} \dfrac{1}{2}\dfrac{x^T (\tP^{-1}(0)-P^{-1})x}{x^Tx} > 0,\\
%  \epsilon(\delta) &= \underset{x\in B_\delta}{\text{sup}}\dfrac{1}{2}\dfrac{|L_c(x)-\tfrac{1}{2}x^T\tP^{-1}(0)x|}{x^Tx} \geq 0.
% \end{align*}
% We insert $\delta$ with $\delta/r$, where $r>1$, and then it follows that $\lambda(\delta) = \lambda(\delta/r)$ and $\epsilon(\delta) = (\nicefrac{1}{r})\epsilon(\delta/r)$. Therefore, there exists an $r>1$ such that $\epsilon(\delta/r) < \lambda(\delta/r)$ . This inequality gives us $ L_c(x)>(\nicefrac{1}{2})x^TP^{-1}x$,  where  $x\in \hW:= B_{\delta/r}$ and $x\neq 0$.
% \end{proof}
Similarly,  we next show an upper bound for the observability energy functional for the QB system in terms of the observability Gramian (in the quadratic form).
\begin{theorem}\label{thm:obs:bound}
 Consider the QB system~\eqref{eq:Quad_bilin_Sys} with $B\equiv 0 $ and an initial condition $x_0$, and let  $L_o$ be the observability energy functional.  Let  $P>0$ and $Q\geq 0$ be  solutions to the generalized Lyapunov equations~\eqref{eq:cont_lyap} and~\eqref{eq:obser_lyap}, respectively. Then, there exists a neighborhood $\tW$ of the origin such that
 \begin{equation*}
  L_o(x_0) \leq  \dfrac{1}{2}x^TQx,\quad \mbox{where}\quad x\in\tW(0).
 \end{equation*}
\end{theorem}
\begin{proof}
Using the definition of the observability energy functional, see~\Cref{def:obser3}, we have
\begin{equation}
 L_o(x_0) = \underset{\begin{subarray}{c}
 	u\in \cB_{(\alpha,\beta)} \\[5pt]
 	x(0) = x_0,x(\infty)=0
 	\end{subarray}  }{\max} \dfrac{1}{2}\int_0^\infty \tL_o(x_0,u)dt,
\end{equation}
 where $\cB_{(\alpha,\beta)} \overset{\mathrm{def}}{=} \{u \in L_2^m[0,\infty), \|u\|_{L_2}\leq \alpha, \|u\|_{L_\infty}\leq \beta \}$ and $\tL_o(x_0,u) := \|y(t)\|_2$. Thus, we have
\begin{align*}
 \tL_o(x_0,u) &= \|y(t)\|_2 = \|Cx(t)\|_2 =  x(t)^TC^TCx(t).
\end{align*}
Substituting for $C^TC$ from~\eqref{eq:obser_lyap}, we obtain
\begin{flalign*}
 \tL_o(x_0,u)&=  -2x(t)^TQAx(t) -  x(t)^T\cH^{(2)} P\otimes Q \big(\cH^{(2)}\big)^T x(t) - \sum_{k=1}^mx(t)^TN_k^TQN_kx(t) .
 \end{flalign*}
Next, we substitute for $Ax$ from~\eqref{eq:Quad_bilin_Sys} (with $B = 0$) to have
\begin{flalign*}
 \tL_o(x_0,u)&= -2x(t)^TQ\dot{x}(t)  + 2x(t)^TQHx(t)\otimes x(t) + 2\sum_{k=1}^mx(t)^TQN_kx(t)u_k(t)\\
 &\qquad -x(t)^T\cH^{(2)} \left(P\otimes Q\right) \big(\cH^{(2)}\big)^T x(t)  - \sum_{k=1}^mx(t)^TN_k^TQN_kx(t) \displaybreak\\
 &=  - \dfrac{d}{dt}(x(t)^TQx(t)) +  x(t)^T\Big( QH (I\otimes x(t)) + QH (x(t)\otimes I) \\
 &\quad + \sum_{k=1}^m(QN_k+ N^T_k Q)u_k(t)-\cH^{(2)} (P\otimes Q)  \big(\cH^{(2)}\big)^T- \sum_{k=1}^mN_k^TQN_k \Big) x(t).
 %&= \dfrac{1}{2}\int_0^\infty - \dfrac{d}{dt}(x(t)^TQx(t))dt + \dfrac{1}{2}\int_0^\infty ( 2x^TQHx(t)\otimes x(t)  -x^T\cH Q_l\otimes P_l \cH^T) x dt
 \end{flalign*}
 This gives
 \begin{align*}
 L_o(x_0) &= \underset{\begin{subarray}{c}
 	u\in \cB_{(\alpha,\beta)} \\[5pt]
 	x(0) = x_0,x(\infty)=0
 	\end{subarray}  }{\max} \dfrac{1}{2}\int_0^\infty \tL_o(x_0,u)dt,\\
  &= \dfrac{1}{2}x_0^TQx_0  +  \underset{\begin{subarray}{c}
  	u\in \cB_{(\alpha,\beta)} \\[5pt]
  	x(0) = x_0,x(\infty)=0
  	\end{subarray}  }{\max}\dfrac{1}{2}\int_0^\infty x(t)^T\left(R_H(x,u) + \sum_{k=1}^mR_{N_k}(x,u)\right) x(t)dt,
\end{align*}
where
\begin{align*}
R_H(x,u) &:=   QH (I\otimes x) + QH (x\otimes I) - \cH^{(2)} (P\otimes Q)  \big(\cH^{2}\big)^T, \\
 R_{N_k}(x,u) &:=  \left(QN_ku_k + N_k^T Qu_k- N_k^TQN_k\right).
\end{align*}
First, note that if for a vector $v$, $v^TN_k^TQN_kv = 0$, then $QN_kv = 0$. Therefore, there exist inputs $u$ for which $\|u\|_{L_\infty}$ is small, ensuring $R_{N_k}(x,u)$ is a negative semidefinite. Similarly, if for a vector $w$, $w^T\cH^{(2)} (P\otimes Q)  \big(\cH^{2}\big)^T w = 0$ and $P>0$, then $ (I\otimes Q)  \big(\cH^{2}\big)^Tw= 0$. Using \eqref{eq:tensor_matricization}, it can be shown that $QH(w\otimes I) = QH(I\otimes w) = 0$.  Now, we consider an initial condition $x_0$ lies in the small neighborhood of the origin and $u \in \cB_{(\alpha,\beta)}$  ensuring that the resulting trajectory $x(t)$ for all time $t$ is such that $R_H(x,u)$ is a negative semi-definite. Finally, we get
 \begin{align*}
 L_o(x_0)- \dfrac{1}{2}x_0^TQx_0 &\leq 0,
 \end{align*}
for $x_0$ lies in the neighborhood of the origin and for the inputs $u$, having small $L_2$ and $L_\infty$ norms and $x_0 \in \tW(0)$ This concludes the proof.
\end{proof}

Until this point, we have proven that in the neighborhood of the origin, the energy functionals of the QB system can be  approximated by the Gramians in the quadratic form. However, one can also prove similar bounds for the energy functionals using the truncated Gramians for QB systems (defined in \Cref{coro:tru_gram}). We summarize this in the following corollary.
\begin{corollary}\label{coro:tru_energy}
Consider the system~\eqref{eq:Quad_bilin_Sys}, having a stable matrix $A$, to be locally reachable and observable. Let $L_c(x)$ and $L_o(x)$  be controllability and observability energy functionals of the system, respectively, and the truncated Gramians $P_\cT>0$ and $Q_\cT >0$ be solutions to the Lyapunov equations as shown in \Cref{coro:tru_gram}.  Then, \
\begin{enumerate}[label=(\roman*)]
	\item there exists a neighborhood $W_\cT$ of the origin such that
	\begin{equation*}
	L_c(x) \geq \frac{1}{2}x^T P_\cT^{-1}x,~\mbox{where}~x\in W_\cT(0).
	\end{equation*}
	\item Moreover,  there also exists a neighborhood $\tW_\cT$ of the origin, where
	\begin{equation*}
	L_o(x) \leq  \frac{1}{2}x^T Q_\cT x,~\mbox{where}~x\in \tW_\cT(0).
	\end{equation*}
\end{enumerate}
  \end{corollary}
In what follows, we illustrate the above bounds using Gramians and truncated Gramians by considering a scalar dynamical system, where $A,H,N,B,C$ are scalars, and are denoted by $a,h,n,b,c$, respectively.
\begin{example}
 Consider a scalar system $(a,h,n,b,c)$, where $a<0$ (stability) and nonzero $h,b,c$. For simplicity, we take $n = 0$ so that we can easily obtain analytic expressions for the controllability  and observability energy functionals, denoted by $L_c(x)$ and $L_o(x)$, respectively. Assume that the system is reachable on $\R$. Then, $L_c(x)$ and $L_o(x)$ can be  determined via solving partial differential equations~\eqref{eq:Obser_Diff} and~\eqref{eq:Cont_Diff} (with $g(x) = 0$), respectively. These are:
 \begin{align*}
  L_c(x) &= - \left(ax^2  + \tfrac{2}{3}hx^3\right)\dfrac{1}{b^2},&
  L_o(x) &= -\dfrac{c^2}{2h}\left(x - \dfrac{a}{h}\log\left(\dfrac{a+hx}{a}\right)\right),
 \end{align*}
 respectively. The quadratic approximations of these energy functionals by using the Gramians, are:
\begin{align*}
\hL_c(x) &= \dfrac{x^2}{2P} \quad ~~ \text{with} \quad P = -\dfrac{-a - \sqrt{a^2 - h^2b^2}}{h^2},\\
\hL_o(x) &= \dfrac{Qx^2}{2}\quad \text{with} \quad Q = -\dfrac{c^2}{2a+h^2P},
\end{align*}
and  the approximations in terms of the truncated Gramians are:
\begin{align*}
\hL^{\scriptscriptstyle(\cT)}_c(x) &= \dfrac{x^2}{2P_\cT}\quad ~~ \text{with} \quad P_\cT = -\dfrac{h^2b^4+4a^2b^2}{8a^3},\\
\hL^{ {\scriptscriptstyle{(\cT)}} }_o(x) &= \dfrac{Q_\cT x^2}{2}\quad \text{with} \quad Q_\cT  = -\dfrac{h^2b^2c^2+4a^2c^2}{8a^3}.
\end{align*}
In order to compare these functionals, we set $a =-2,b=c=2$ and $h =1$ and plot the resulting energy functionals  in \Cref{fig:comparison_gram}.

\begin{figure}[!tb]
 \begin{subfigure}[h]{0.49\textwidth}
\centering
	\setlength\fheight{3cm}
	\setlength\fwidth{5.cm}
	\includetikz{Energy_QBDAE_cont}
	\caption{Comparison of the controllability energy functional  and its approximations.}
   \end{subfigure}
    \begin{subfigure}[h]{0.49\textwidth}
\centering
	\setlength\fheight{3cm}
	\setlength\fwidth{5.0cm}
	\includetikz{Energy_QBDAE_obser}
	\caption{Comparison of the observability energy functional and its approximations.}
   \end{subfigure}
   \caption{Comparison of exact energy functionals with approximated energy functionals via the Gramians and truncated Gramians.}
   \label{fig:comparison_gram}
\end{figure}

Clearly, \Cref{fig:comparison_gram} illustrates the lower and upper bounds for the controllability and observability energy functionals, respectively at least locally. Moreover, we observe that the bounds for the energy functionals, given in terms of truncated Gramians are  closer to the actual energy functionals of the system in the small neighborhood of the origin. 
 \end{example}
%\subsection{Computation of reduced-order systems based on the Gramians}

So far, we have shown the bounds for the energy functionals in terms of the Gramians of the QB system. In order to prove those bounds, it is assumed that $P$ is a  positive definite. However, this assumption might not be fulfilled for many QB systems, especially arising from  semi-discretization of nonlinear PDEs. Therefore, our next objective is to  provide another interpretation of the proposed Gramians and truncated Gramians, that is, the connection of Gramians and truncated Gramians with reachability and observability of the system. 
 For the observability energy functional, we consider the output $y$ of the following \emph{homogeneous}  QB system:
\begin{equation}\label{eq:homo_qbdae}
\begin{aligned}
\dot{x}(t)&= Ax + Hx(t)\otimes x(t) + \sum_{k=1}^mN_kx(t)u_k(t),\\
y(t) &= Cx(t),\qquad x(0) = x_0,
\end{aligned}
\end{equation}
as considered for bilinear systems in~\cite{morBenD11,enefungray98}. However, it might also be possible to consider an \emph{inhomogeneous} system by setting the control input $u$ completely zero, as shown in~\cite{morSch93}. We first investigate how the proposed Gramians are related to reachability and observability of the QB systems, analogues to derivation for bilinear systems  in \cite{morBenD11}.
\def\im {\ensuremath{{\mathsf{Im}}}}
\def\ker {\ensuremath{{\mathsf{Ker}}}}
\def\rank {\ensuremath{{\mathsf{rank}}}}
\begin{theorem}\label{thm:kerPQ_dyn}
	\
	\begin{enumerate}[label=(\alph*)]
		\item Consider the QB system~\eqref{eq:Quad_bilin_Sys}, and assume the reachability Gramian $P$ to be the solution of~\eqref{eq:cont_lyap}. If the system is steered from $0$ to $x_0$, where $x_0 \not\in \im P$, then $L_c(x_0) = \infty$ for all input functions $u$.
		\item Furthermore,  consider the \emph{homogeneous} QB system~\eqref{eq:homo_qbdae} and assume $P>0$ and $Q$ to be the reachability and observability Gramians of the QB system which are  solutions of~\eqref{eq:cont_lyap} and~\eqref{eq:obser_lyap}, respectively. If the initial state satisfies $x_0 \in  \ker Q$, then $L_o(x_0) = 0$.
	\end{enumerate}
\end{theorem}
\begin{proof}\
\begin{enumerate}[label=(\alph*)]
	\item By assumption, $P$ satisfies
\begin{equation}\label{eq:CG}
AP + PA^T + H(P\otimes P)H^T +\sum_{k=1}^mN_kPN_k^T + BB^T =0.
\end{equation}
Next, we consider a vector  $v \in \ker P$ and multiply the above equation from the left and right with $v^T$ and $v$, respectively to obtain
\begin{align*}
0&= v^TAPv + v^TPA^Tv + v^TH(P\otimes P)H^Tv +\sum_{k=1}^mv^TN_kPN_k^Tv + v^TBB^Tv\\
&=v^TH(P\otimes P)H^Tv +\sum_{k=1}^mv^TN_kPN_k^Tv + v^TBB^Tv.
\end{align*}
This implies $B^Tv = 0$, $PN_k^T v=0$ and $(P\otimes P)H^T v=0$. From~\eqref{eq:CG}, we  thus obtain $PA^Tv = 0$. Now we consider an arbitrary state vector $x(t)$, which is the solution of \eqref{eq:Quad_bilin_Sys} at time $t$ for any given input function $u$. If $x(t) \in \im P $ for some $t$, then we have
\begin{equation*}
\dot{x}(t)^Tv = x(t)^TA^Tv + \left(x(t)\otimes x(t)\right)^TH^Tv + \sum_{k=1}^mu_k(t)x(t)^TN_k^Tv + u(t)B^Tv = 0.
\end{equation*}
The above relation indicates that $\dot x(t) \perp v$ if $v\in \ker P$ and $x(t) \in \im P$.  It shows that \im P is invariant under the dynamics of the system. Since the initial condition $0$ lies in $\im P$,  $x(t) \in \im P$ for all $t\geq 0$. This reveals that if the final state $x_0 \not\in \im P$, then it cannot be reached from $0$; hence, $L_c(x_0) =\infty$.
\item Following the above discussion, we can show that $(I\otimes Q )  \left(\cH^{(2)}\right)^T \ker Q = 0$, $QN_k\ker Q =0$, $QA\ker Q = 0$, and $C\ker Q =0$.  Let $x(t)$ denote the solution of the homogeneous system at time $t$. If $x(t)\in \ker Q$ and a vector $\tv \in \im Q$, then we have
\begin{align*}
\tv^T\dot x(t)&= \underbrace{\tv Ax(t) }_{=0}+ \tv^TH(x(t)\otimes x(t)))+ \sum_{k=1}^m\underbrace{\tv^TN_kx(t)u_k(t)}_{=0}\\
&= x(t)^T\cH^{(2)}(x(t)\otimes \tv) = \underbrace{x(t)^T\cH^{(2)}(I\otimes \tv)}_{=0}x(t) = 0.
\end{align*}
This implies that if $x(t) \in  \ker Q$, then $\dot x(t)  \in\ker Q$.  Therefore, if the initial condition $x_0 \in   \ker Q$, then $x(t)\in \ker Q$ for all $t\geq 0$, resulting in $y(t)=C\underbrace{x(t)}_{\in \ker Q} = 0$; hence, $L_o(x_0) = 0$.
\end{enumerate}
\end{proof}

The above theorem suggests that the state components, belonging to \ker P or \ker Q, do not play a major role as far as the system dynamics are concerned. This shows that the states which belong to \ker P,  are uncontrollable, and similarly, the states, lying in \ker Q are unobservable once the uncontrollable states are removed. Furthermore, we have shown in \Cref{thm:obs:bound,thm:con_bound} the lower and upper bounds for the controllability and observability energy functions in the quadratic form of the Gramians $P$ and $Q$ of QB systems (at least in the neighborhood of the origin). This coincides with the concept of balanced truncation model reduction which aims at eliminating weakly controllable and weakly observable state components. Such states are corresponding to zero or small singular values of $P$ and $Q$. In order to find these states simultaneously, we utilize the balancing tools similar to the linear case; see, e.g.,~\cite{moral1994,morAnt05}. For this, one needs to determine the Cholesky factors of the Gramians as  $P =: S^TS$  and $Q =: R^TR$, and compute the SVD of $ SR^T =: U\Sigma V^T $, resulting in a transformation matrix $T = S^TU\Sigma^{-\tfrac{1}{2}}$. Using the matrix $T$, we obtain an equivalent QB system
\begin{equation}\label{sys:tran_QBDAE}
\begin{aligned}
\dot{\tx}(t) &= \tA \tx(t) + \tH \tx(t)\otimes \tx(t) + \sum_{k=1}^m\tN_k \tx(t)u_k(t) + \tB u(t),\\
y(t) &= \tC \tx(t),\quad \tx(0) = 0
\end{aligned}
\end{equation}
with $$\tA = T^{-1}AT,\quad\tH = T^{-1}H(T\otimes T),\quad\tN_k = T^{-1}N_kT,\quad\tB = T^{-1}B,\quad\tC = CT.$$
Then, the above transformed system~\eqref{sys:tran_QBDAE} is a balanced system, as the Gramians $\tP$ and $\tQ$ of the system~\eqref{sys:tran_QBDAE} are equal and diagonal, i.e., $\tP = \tQ = \mbox{diag}(\sigma_1,\sigma_2,\ldots,\sigma_n)$. The attractiveness of the balanced system is that it allows us to find  state components corresponding to small singular values of both $\tP$ and $\tQ$. If $\sigma_{\hn} > \sigma_{\hn+1}$, for some $\hn\in\N$, then it is easy to see that states related to $\{\sigma_{\hn+1},\ldots,\sigma_n\}$ are not only hard to control but also hard to observe; hence, they can be eliminated. In order to determine a reduced system of order $\hn$, we partition $T = \bbm T_1 &T_2\ebm$ and $T^{-1} = \bbm S_1^T & S_2^T\ebm^T$, where $T_1,S_1^T \in \R^{n\times \hn}$, and define the reduced-order system's realization as follows:
\begin{equation}\label{eq:red_realization}
\hA = S_1AT_1,~~\hH = S_1H(T_1\otimes T_1),~~\hN_k = S_1N_kT_1,~~\hB = S_1B,~~\hC = CT_1,
\end{equation}
which is generally a locally good approximate of the original system; though it is not a straightforward task to estimate the error occurring due to the truncation of the QB system unlike in the case of linear systems.   

 Based on the above discussions, we propose the following corollary, showing how the truncated Gramians of a QB system relate to reachability and observability of the system.
\begin{corollary}\label{coro:gram_trc_re}
\	\begin{enumerate}[label=(\alph*)]
		\item  Consider the QB system~\eqref{eq:Quad_bilin_Sys}, and let $P_\cT$ and $Q_\cT $ be the truncated Gramians of the system, which are solutions of the Lyapunov equations as in~\eqref{eq:tru_gramians}. If the system is steered from $0$ to $x_0$ where, $x_0 \not\in \im P_\cT$, then $L_c(x_0) = \infty$ for all input functions $u$. 
		\item Assume the QB system~\eqref{eq:Quad_bilin_Sys} is locally controllable around the origin, i.e., $(A,B)$ is controllable.  Then, for  the \emph{homogeneous} QB system~\eqref{eq:homo_qbdae}, if  the initial state $x_0 \in  \ker Q_\cT$, then $L_o(x_0) = 0$.
\end{enumerate}
\end{corollary}
The above corollary can be proven, along of the lines of the proof for \Cref{thm:kerPQ_dyn}, keeping in mind that if $\gamma~\in~\ker P_\cT$, then $\gamma$ also belongs  to $\ker P_1$, where $P_1$ is the solution to \eqref{eq:linear_CG}. Similarly, if $\xi\in \ker Q_\cT $, then $\xi$ also lies in $\ker Q_1$, where $Q_1$ is the solution to \eqref{eq:linear_OG}. This can easily be verified using simple linear algebra. Having noted this, \Cref{coro:gram_trc_re} also suggests that $\ker P_\cT$ is uncontrollable, and $\ker Q_\cT$ is also unobservable if the system is locally controllable.   Moreover, these truncated Gramians also bound the energy functions for QB systems in the quadratic form, see \Cref{coro:tru_energy}. Based on these, we conclude that the truncated Gramians are also a good candidate to use for balancing the system and to compute  the reduced-order systems. 


%the truncated Gramians $P_\cT$ and $Q_\cT $ can also be used to determine reduced-order system due to the same reason explained for the Gramians $P$ and $Q$. But, in the next section, we present advantages of considering truncated Gramians $P_\cT$ and $Q_\cT $ over Gramians $P$ and $Q$ in the MOR framework.