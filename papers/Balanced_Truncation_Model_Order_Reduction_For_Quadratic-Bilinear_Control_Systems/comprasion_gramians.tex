We start with showing  that under what conditions the Gramians approximate the energy functionals of the QB system, in the quadratic forms.
\subsection{Comparison of energy functionals with  Gramians}\label{subsec:energy}
Let us make use of Theorem \ref{thm:energy_function} to have the following partial differential equation, whose solution gives the controllability energy functional for the QB system:
\begin{equation}\label{eq:control_energy_QB}
 \dfrac{\partial L_c}{\partial x}(Ax + H~x\otimes x) +  (Ax + H~x\otimes x)^T\dfrac{\partial L_c}{\partial x}^T + \dfrac{\partial L_c}{\partial x}(N x + B)(N x + B)^T \dfrac{\partial L_c}{\partial x} ^T = 0.
\end{equation}
Unlike linear systems, the controllability energy functional $L_c(x)$ for nonlinear systems cannot be expressed as a simple quadratic form such as $L_c(x) = x^T\tP^{-1}x$, where $\tP$ is a constant matrix. Instead, for nonlinear systems the matrix $\tP$ is  a function of the state vector. Nevertheless, it is shown in~\cite{enefungray98} that for sufficiently small  $\|x\|$  and $x\neq 0$, the gradients of $L_c$ can be given by
\begin{equation}\label{eq:grad_contr}
 \dfrac{\partial}{\partial x}L_c(x) = \tP(x)^{-1}x.
\end{equation}
Substituting for $\tfrac{\partial}{\partial x}L_c$ from~\eqref{eq:grad_contr} in~\eqref{eq:control_energy_QB}, we obtain an equivalent equation in $\tP(x)$ as follows:
\begin{equation*}
 \left(A+H\left(x\otimes I\right)\right)\tP(x) + \tP(x)\left(A^T + (x^T\otimes I)\right) + (Nx + B)(Nx + B)^T = 0,
\end{equation*}
which is a very difficult task to solve for large-scale systems. However, a possible remedy to this problem  is to replace $\tP(x)$ by a constant matrix $\mathbb P$ such that the following lower bound holds:
\begin{equation}
 L_c(x) > x^T\mathbb P^{-1}x
\end{equation}
at least in the neighborhood of $x = 0$. This can provide us the worst-case scenario about the states which produce a lot energy. This means that these states are hard to reach.
The idea has been imposed to bilinear systems in~\cite{enefungray98}, where the above bound is derived for the controllability energy functional $L_c$ in the neighborhood of $0$ and $\mathbb P$ is assumed to the controllability Gramian of the bilinear system. We here aim to show that the similar lower bound for the controllability energy functional for QB systems, can be given in terms of the proposed controllability Gramian.
\begin{theorem}\label{thm:con_bound}
 Let the system~\eqref{eq:Quad_bilin_Sys}, having a stable matrix $A$, be asymptotically reachable from $0$ to some neighborhood $W$ of $0$. Also, assume that $L_c$ has an analytic solution of~\eqref{eq:control_energy_QB} on $W$. If the Gramian $P$ is computed as shown in Theorem~\ref{thm:con_gram}, then there exists a neighborhood $\hW$ of $0$ contained in $W$ such that
 \begin{equation*}
     L_c(x) >\tfrac{1}{2}x^T P^{-1}x.
 \end{equation*}
\end{theorem}
\begin{proof}
This proof follows the similar steps as it is done for bilinear systems in~\cite{enefungray98}. However, for the sake of completeness, we prove the bound for the system.  The stability and asymptotic reachability assumptions give us $L_c(0)=0$ and $\tfrac{d L_c}{dx}(0) = 0$, and the analyticity assumption on $L_c$ indicates that there exists an open ball $B_\delta \subset W$  of radius $\delta >0$, where the following holds:
 \begin{equation}\label{eq:cont_ener_0}
  L_c(x) = \dfrac{1}{2}x^T\dfrac{\partial^2 L_c}{\partial x^2}(0)x + \cO(\|x\|^3).
 \end{equation}
Taking the derivate of~\eqref{eq:grad_contr} with respect to $x$ yields
 \begin{equation*}
  \dfrac{\partial ^2 L_c } {\partial^2 x}(x) =  \tP^{-1}(x) + x^T \dfrac{\partial \tP^{-1}} {\partial x}(x),
 \end{equation*}
which gives us
 \begin{equation*}
  \dfrac{\partial ^2 L_c } {\partial^2 x}(0) =  \tP^{-1}(0),
 \end{equation*}
where $\tP(0)$ is the solution of the following Lyapunov equation:
\begin{equation}\label{eq:linear_lyp}
 A\tP(0) + \tP(0)A^T + BB^T = 0.
\end{equation}
Substituting for $\tfrac{\partial ^2 L_c } {\partial^2 x}(0)$ in~\eqref{eq:cont_ener_0}, we get
 \begin{equation}\label{eq:cont_ener_1}
  L_c(x) = \dfrac{1}{2}x^T\tP^{-1}(0)x + \cO(\|x\|^3).
 \end{equation}
Now, we define the difference of the controllability Gramian of system $P$ and $\tP(0)$. This yields
\begin{equation*}
 P-\tP(0) = \int_0^\infty e^{At}H\left(P\otimes P\right) H^T e^{A^Tt}dt  + \int_0^\infty e^{At}N P N^T e^{A^Tt}dt.
\end{equation*}
If either $H$, or $N$, or both, are full  rank matrices, then it is clear that $P > \tP(0)$. Equivalently, it indicates $P^{-1} < \tP^{-1}(0)$, provided $P$ and $\tP(0)$ are positive definite matrices. Now, we introduce the following real number as in~\cite{enefungray98}:
\begin{align*}
 \lambda(\delta) &= \underset{x\in B_\delta}{\text{inf}} \dfrac{1}{2}\dfrac{x^T (\tP^{-1}(0)-P^{-1})x}{x^Tx} > 0,\\
 \epsilon(\delta) &= \underset{x\in B_\delta}{\text{sup}}\dfrac{1}{2}\dfrac{|L_c(x)-\tfrac{1}{2}x^T\tP^{-1}(0)x|}{x^Tx} \geq 0.
\end{align*}
We insert $\delta$ with $\delta/r$, where $r>1$, and then it follows that $\lambda(\delta) = \lambda(\delta/r)$ and $\epsilon(\delta) = (\nicefrac{1}{r})\epsilon(\delta/r)$. Therefore, there exists an $r>1$ such that $\epsilon(\delta/r) < \lambda(\delta/r)$ . This inequality gives us $ L_c(x)>(\nicefrac{1}{2})x^TP^{-1}x$,  where  $x\in \hW:= B_{\delta/r}$ and $x\neq 0$.
\end{proof}
Similarly,  we next show an upper bound for the observability energy functional for the QB system in terms of the observability Gramians (in the quadratic form).
\begin{theorem}\label{thm:obs:bound}
 Suppose the system~\eqref{eq:Quad_bilin_Sys}  has a stable $A$, and either $H$ or $N$, or both, are full rank matrices. Moreover, assume that $L_o$ is an analytic solution of \eqref{eq:Obser_Diff} (with $f(x) = A+H(I\otimes x)$, $g(x) = Nx$, $h(x) = C$) on some neighborhood $W$ of $0$. Let  $P>0$ and $Q>0$ be  solutions to the generalized quadratic Lyapunov equations~\eqref{eq:cont_lyap} and~\eqref{eq:obser_lyap}, respectively. If there exists a neighborhood $\hW \subset W$ such that the following holds:
\begin{align*}
\underset{\begin{subarray}{c}
 u\in L_2(0,\infty) \\[5pt]
 x(0)=x_0,x(\infty) = 0
  \end{subarray}}{\textbf{max}} \dfrac{1}{2}\int_0^\infty x^T(t)R(x,u) x(t)dt >0,
\end{align*}
where
\begin{align*}
R(x,u) &:=  -\Big( QH (I\otimes x) + QH (x\otimes I) \\
& \qquad + [QN_i + N^T Q]u-\cH^{(2)} P\otimes Q  \big(\cH^{2}\big)^T- N^TQN \Big) ,
\end{align*}
   then
 \begin{equation*}
  L_o(x) < \dfrac{1}{2}x^TQx,\quad \mbox{where}\quad x\in\hW.
 \end{equation*}
\end{theorem}
\begin{proof}
Consider the following expression for observability energy functional:
\begin{flalign*}
 L_o(x,u) &= \dfrac{1}{2}\int_0^\infty \|Cx(t)\|_2^2dt = \dfrac{1}{2}\int_0^\infty x^T(t)C^TCx(t)dt.
\end{flalign*}
Substituting for $C^TC$ from~\eqref{eq:obser_lyap}, we obtain
\begin{flalign*}
 L_o(x,u)&= \dfrac{1}{2}\int_0^\infty \left(-2x^TQAx -  x^T\cH^{(2)} P\otimes Q \big(\cH^{(2)}\big)^T x - x^TN^TQNx \right)  dt.
 \end{flalign*}
Next, we substitute for $Ax$ from~\eqref{eq:Quad_bilin_Sys} (with $B = 0$) to have
\begin{flalign*}
 L_o(x,u)&= \dfrac{1}{2}\int_0^\infty \Big(-2x^TQ\dot{x}  + 2x^TQHx(t)\otimes x + 2x^TNxu\\
 &\qquad -x^T\cH^{(2)} P\otimes Q \big(\cH^{(2)}\big)^T x  - x^TN^TQNx \Big)dt \\
 &= \dfrac{1}{2}\int_0^\infty - \dfrac{d}{dt}(x^TQx)dt + \dfrac{1}{2}\int_0^\infty x^T\Big( QH (I\otimes x) + QH (x\otimes I) \\
 &\quad + [QN_i + N^T Q]u-\cH^{(2)} P\otimes Q  \big(\cH^{(2)}\big)^T- N^TQN \Big) x dt.
 %&= \dfrac{1}{2}\int_0^\infty - \dfrac{d}{dt}(x^T(t)Qx(t))dt + \dfrac{1}{2}\int_0^\infty ( 2x^TQHx(t)\otimes x(t)  -x^T\cH Q_1\otimes P_1 \cH^T) x dt
 \end{flalign*}
 This gives
 \begin{flalign*}
 L_o(x,u) - \dfrac{1}{2}x_0^TQx_0 &=   - \dfrac{1}{2}\int_0^\infty x^T(t)R(x,u) x(t)dt.
\end{flalign*}
For a  ball of  sufficiently small radius $\delta$ around $0$ in which $\|x\| < \delta$ and  small input function $u$ in $L_2$-norm, $R(x,u) >0$. This can be argued based on the two terms $\cH^{(2)} P\otimes Q  \big(\cH^{(2)}\big)^T$ and $N^TQN$ which are positive definite due to the matrices $P$ and $Q$ being positive definite, and also are independent of the input and state vector; whereas the other two terms are the functions of the state and input  whose dominance can be reduced by choosing sufficiently small input function and the ball for $\|x\|$. This implies, the right-hand side is negative at least under these assumptions. Therefore,
\begin{align*}
 L_0(x_0) - \dfrac{1}{2} x_0^TQx_0 &= \underset{\begin{subarray}{c}
 u\in L_2(-\infty,0), \\[5pt]
 x(-\infty)=0,x(0) = x_0
  \end{subarray} }{\textbf{max}}  L_0(x_0,u) - \dfrac{1}{2} x_0^TQx_0 \\
 &= \underset{\begin{subarray}{c}
 u\in L_2(0,\infty) \\[5pt]
 x(0)=x_0,x(\infty) = 0
  \end{subarray}}{\textbf{max}} -\dfrac{1}{2}\int_0^\infty x^T(t)R(x,t) x(t)dt <0.
\end{align*}
Thus $L_0(x) <  \dfrac{1}{2} x_0^TQx_0 $ when $x \in \hW  \sim {0}$.
\end{proof}
\begin{remark}
As discussed in~\cite{morBenB11,verriest2008time} for the bilinear systems, there may be integrability issues related to the field $\tP(x)^{-1}x$ which need not be necessarily integrable. However, in this article we avoid a detailed discussion on it.
\end{remark}


Until this point, we have proven that in the neighborhood of $0$, the energy functionals of the QB system can be quadratically approximated by the Gramians which are the solutions of more generalized Lyapunov equations. However, one can also prove the similar bounds for the energy functionals using the truncated Gramians for QB systems (defined in Corollary~\ref{coro:tru_gram}). We summarize this in the following.
\begin{corollary}\label{coro:tru_energy}
Consider the system~\eqref{eq:Quad_bilin_Sys}, having  stable $A$, to be asymptotically reachable from $0$ to some neighborhood $W$ of $0$. Also, assume $L_c(x)$ and $L_o(x)$ to be controllability and observability energy functionals of the system, respectively. Let the truncated Gramians $P_\cT$ and $Q_\cT $ be solutions to the Lyapunov equations as shown in  Corollary~\ref{coro:tru_gram}. Then, \\
 {\bfseries (i)}~~There exists a neighborhood $\hW$ of $0$ contained in $W$ such that
 \begin{equation*}
     L_c(x) >\tfrac{1}{2}x^T P_\cT^{-1}x.
 \end{equation*}
 {\bfseries (ii)}~~Moreover, if there exists a neighborhood $\tW \subset W$ such that the following holds:
 \begin{align*}
 \underset{\begin{subarray}{c}
  u\in L_2(0,\infty) \\[5pt]
  x(0)=x_0,x(\infty) = 0
   \end{subarray}}{\text{max}} \dfrac{1}{2}\int_0^\infty x^T(t)R_\cT (x,u) x(t)dt >0,
 \end{align*}
 where
\begin{align*}
R_\cT (x,u) &:= -\Big( Q_\cT H (I\otimes x(t)) + Q_\cT H (x\otimes I) \\
& \qquad + [Q_\cT N_i + N^T Q_\cT ]u-\cH^{(2)} (P_1\otimes Q_1)  \big(\cH^{2}\big)^T- N^TQ_1N \Big),
\end{align*}
in which $P_1$ and $Q_1$ are solutions to ~\eqref{eq:linear_CG} and \eqref{eq:linear_OG}, respectively, then the following bound for the observability energy functional holds:
 \begin{equation*}
     L_o(x) <\tfrac{1}{2}x^T Q_\cT x.
 \end{equation*}
  \end{corollary}
In what follows, we illustrate the above bounds using Gramians and truncated Gramians by considering a scalar dynamical system, where $A,H,N,B,C$ are scalars, denoted by $a,h,n,b,c$, respectively.
\begin{example}
 Consider a scalar system $(a,h,n,b,c)$, where $a<0$ (stability) and non-zero $h,b,c$. For simplicity, we also take $n = 0$ so that we can easily obtain analytic expressions for the controllability  and observability energy functionals, denoted by $L_c(x)$ and $L_o(x)$, respectively. Assume that the system is reachable on $\R$. Then, $L_c(x)$ and $L_o(x)$ can be  determined via solving partial differential equations~\eqref{eq:Obser_Diff} and~\eqref{eq:Cont_Diff}, respectively. These are:
 \begin{align*}
  L_c(x) &= - \left(ax^2  + \tfrac{2}{3}hx^3\right)\dfrac{1}{b^2},&
  L_o(x) &= -\dfrac{c^2}{2h}\left(x - \dfrac{a}{h}log\left(\dfrac{a+hx}{a}\right)\right),
 \end{align*}
 respectively. The quadratic approximations to these energy functionals by using Gramians, are as follows:
\begin{align*}
\hL_c(x) &= \dfrac{x^2}{2P}, \quad ~~ \text{with} \quad P = -\dfrac{-a + \sqrt{a^2 - h^2b^2}}{h^2},\\
\hL_o(x) &= \dfrac{Qx^2}{2},\quad \text{with} \quad Q = -\dfrac{c^2}{2a+h^2P},
\end{align*}
and approximations in terms of truncated Gramians are:
\begin{align*}
\hL^{(t)}_c(x) &= \dfrac{x^2}{2P_\cT}, \quad ~~ \text{with} \quad P_\cT = -\dfrac{h^2b^4+4a^2b^2}{8a^3},\\
\hL^{(t)}_o(x) &= \dfrac{Q_\cT x^2}{2},\quad \text{with} \quad Q_\cT  = -\dfrac{h^2b^2c^2+4a^2c^2}{8a^3}.
\end{align*}
In order to compare these functionals, we set $a =-2,b=c=2$ and $h =1$, and plot in Figure~\ref{fig:comparison_gram}.

\begin{figure}[h]
 \begin{subfigure}[h]{0.49\textwidth}
\centering
	\setlength\fheight{3cm}
	\setlength\fwidth{5.cm}
	\includetikz{Energy_QBDAE_cont}
	\caption{Comparison of the controllability energy functional  and its approximations.}
   \end{subfigure}
    \begin{subfigure}[h]{0.49\textwidth}
\centering
	\setlength\fheight{3cm}
	\setlength\fwidth{5.0cm}
	\includetikz{Energy_QBDAE_obser}
	\caption{Comparison of the observability energy functional and its approximations.}
   \end{subfigure}
   \caption{Comparison of exact energy functionals with approximated energy functionals via Gramians and truncated Gramians.}
   \label{fig:comparison_gram}
\end{figure}

Clearly, Figure~\ref{fig:comparison_gram} illustrates the lower and upper bounds for the controllability and observability energy functionals, respectively, shown in Theorem~\ref{thm:con_bound}-\ref{thm:obs:bound} and in Corollary~\ref{coro:tru_energy}. Moreover, we observe that bounds for energy functionals, given in terms of truncated Gramians are much close to the actual energy functionals of the system. This perhaps happens due to the fact that $P > P_\cT$  and $Q > Q_\cT $; this  overestimate the bounds for the energy functionals as one sees in the figure.
 \end{example}
%\subsection{Computation of reduced-order systems based on the Gramians}
Our next objective is to establish the connection of Gramians for QB systems with controllability and observability of the system. For the observability energy functional, we consider the output $y$ of the following \emph{homogeneous}  QB system:
\begin{equation}\label{eq:homo_qbdae}
\begin{aligned}
\dot{x}(t)&= Ax + Hx(t)\otimes x(t) + Nx(t)u(t),\\
y(t) &= Cx(t),\qquad x(0) = x_0,
\end{aligned}
\end{equation}
as considered for bilinear systems in~\cite{morBenD11,enefungray98}. However, it might also be possible to consider an \emph{inhomogeneous} system by setting the control input $u$ completely zero, as shown in~\cite{morSch93}. We first investigate how the proposed Gramians are related to controllability and observability of the QB systems, analogues to the authors in \cite{morBenD11} have done for bilinear systems.
\def\im {\ensuremath{{\mathsf{Im}}}}
\def\ker {\ensuremath{{\mathsf{Ker}}}}
\def\rank {\ensuremath{{\mathsf{rank}}}}
\begin{theorem}\label{thm:kerPQ_dyn}
\mbox{}
\\(A)~Consider the QB system~\eqref{eq:Quad_bilin_Sys}, and assume the controllability Gramian $P$ to be the solution of~\eqref{eq:cont_lyap}. If the system is steered from $0$ to $x_0$, where $x_0 \not\in \im P$, then $L_c(x_0) = \infty$ for all input functions $u$.\\
(B)~Similarly,  consider the \emph{homogeneous} QB system~\eqref{eq:homo_qbdae} and assume $P$ and $Q$ be the controllability and observability Gramians of the QB system which are  solutions of~\eqref{eq:cont_lyap} and~\eqref{eq:obser_lyap}, respectively. If the initial state, $x_0 \in \{\ker P \cap \ker Q\}$, then $L_o(x_0) = 0$.
\end{theorem}
\begin{proof}
\mbox{}\\
(A)~As we know that $P$ satisfies
\begin{equation}\label{eq:CG}
AP + PA^T + H(P\otimes P)H^T +NPN^T + BB^T =0.
\end{equation}
Next, we consider a vector  $v \in \ker P$, and multiply the above equation from left and right with $v^T$ and $v$, respectively to obtain
\begin{align*}
0&= v^TAPv + v^TPA^Tv + v^TH(P\otimes P)H^Tv +v^TNPN^Tv + v^TBB^Tv\\
&=v^TH(P\otimes P)H^Tv +v^TNPN^Tv + v^TBB^Tv.
\end{align*}
This implies $B^Tv = 0$, $PN^T v=0$ and $(P\otimes P)H^T v=0$. From~\eqref{eq:CG}, we also have $A^Tv = 0$. Now, we consider an arbitrary state vector $x(t)$, which is the solution of \eqref{eq:Quad_bilin_Sys} at time $t$ for any given input function $u$. If $x(t) \in \im P $ for some $t$, then we have
\begin{equation*}
\dot{x}(t)^Tv = x(t)^TA^Tv + (x(t)\otimes x(t)^TH^Tv + u(t)x(t)N^Tv + u(t)B^Tv = 0.
\end{equation*}
The above relation indicates that $\dot x(t) \perp v$, if $v\in \ker P$ and $x(t) \in \im P$.  This shows that \im P is invariant under the dynamics of the system. Since the initial condition $0$ lies in $\im P$, therefore $x(t) \in \im P$ for all $t\geq 0$. This reveals that if the final state $x_0 \not\in \im P$, then it cannot be reached from $0$; hence, $L_c(x_0) =\infty$.\\
(B)~Similar to the above discussion, we can show that $(P \otimes Q )  \left(\cH^{(2)}\right)^T \ker Q = 0$, $QN\ker Q =0$ and $C\ker Q =0$. This implies that if $x(t) \in  \{\ker P \cap \ker Q\}$, then $\dot x(t)  \in\{\ker P \cap \ker Q\}$.  Therefore, if the initial condition $x_0 \in   \{\ker P \cap \ker Q\}$, then $x(t)\in  \{\ker P \cap \ker Q\}$ for all $t\geq 0$, resulting $y(t)=C\underbrace{x(t)}_{\ker Q} = 0$; hence, $L_o(x_0) = 0$.
\end{proof}
The above theorem suggests that the state components, belonging to \ker P and \ker Q, do not play much a major role as far as the system dynamics are concerned. This shows that the states which belong to \ker P, they are hard controllable; and similarly the state, lying in \ker Q are hardly observable. This coincides with the concept of balanced truncation model reduction which aims at eliminating weakly controllable and weakly observable state components. Such states are corresponding to zero or small singular values of $P$ and $Q$. In order to find these states simultaneously, we utilize the balancing tools similar to the linear case; see, e.g.,~\cite{moral1994,morAnt05}. For this, one needs to decompose of the Gramians $P =: S^TS$  and $Q =: R^TR$, and compute the SVD of $ SR^T =: U\Sigma V^T $, resulting a transformation matrix $T = S^TU\Sigma^{-\nicefrac{1}{2}}$. Using the matrix $T$, we obtain an equivalent QB system
\begin{equation}\label{sys:tran_QBDAE}
\begin{aligned}
\dot{\tx}(t) &= \tA \tx(t) + \tH \tx(t)\otimes \tx(t) + \tN \tx(t)u(t) + \tB u(t),\\
y(t) &= \tC \tx(t),\quad \tx(0) = 0
\end{aligned}
\end{equation}
with $$\tA = T^{-1}AT,\quad\tH = T^{-1}H(T\otimes T),\quad\tN = T^{-1}NT,\quad\tB = T^{-1}B,\quad\tC = CT.$$
Then the above transformed system~\eqref{sys:tran_QBDAE} is a balanced system, implying Gramians $\tP$ and $\tQ$ of the system~\eqref{sys:tran_QBDAE} are equal and diagonal, i.e., $\tP = \tQ = \mbox{diag}(\sigma_1,\sigma_2,\ldots,\sigma_n)$. The beauty of the balanced system is that it allows us to find  state components correspond to small singular values of both $\tP$ and $\tQ$. If $\sigma_r \geq \sigma_{r+1}$, for some $r\in\N$, then it is easy to see that states related to $\{\sigma_{r+1},\ldots,\sigma_n\}$ are not only hard to control, but also hard to observe; hence, they can be eliminated. In order to determine an $r$-order reduced system, we partition $T = \bbm T_1 &T_2\ebm$ and $T^{-1} = \bbm S_1^T & S_2^T\ebm^T$, where $T_1,S_1^T \in \Rnr$, and define the reduced-order system's realization as follows:
\begin{equation}\label{eq:red_realization}
A_r = S_1AT_1,~~H_r = S_1H(T_1\otimes T_1),~~N_r = S_1NT_1,~~B_r = S_1B,~~C_r = CT_1,
\end{equation}
which is generally a locally good approximate of the original system; though it is not straightforward to estimate the error occurred due to truncation in QB systems unlike linear systems.     Based on the above discussions, we propose the following corollary, relying on the truncated Gramians of QB systems.
\begin{corollary}\label{coro:gram_trc_re}
Consider the QB system~\eqref{eq:Quad_bilin_Sys}, and let $P_\cT$ and $Q_\cT $ be the truncated Gramians of the system, which are solutions of the Lyapunov equations shown in Corollary~\ref{coro:tru_gram}. If the system is steered from $0$ to $x_0$ where, $x_0 \not\in \im P_\cT$, then $L_c(x_0) = \infty$ for all input functions $u$. Moreover,  for the \emph{homogeneous} QB system~\eqref{eq:homo_qbdae}, if the initial state, $x_0 \in \{\ker P_\cT \cap \ker Q_\cT \}$, then $L_o(x_0) = 0$.
\end{corollary}
The above corollary can be proven, along of the lines of the proof for Theorem~\ref{thm:kerPQ_dyn}, keeping in mind that if $\gamma~\in~\ker P_\cT$, then $\gamma$ also belongs  to $\ker P_1$, where $P_1$ is the solution to \eqref{eq:linear_CG}. Similarly, if $\xi\in \ker Q_\cT $, then $\xi$ also lies in $\ker Q_1$, where $Q_1$ is the solution to \eqref{eq:linear_OG}. This can easily be verified using simple linear algebra. Having noted this, Corollary~\ref{coro:gram_trc_re} suggests that the truncated Gramians $P_\cT$ and $Q_\cT $ can also be used to determine reduced-order system due to the same reason explained for the Gramians $P$ and $Q$. But, in the next section, we present advantages of considering truncated Gramians $P_\cT$ and $Q_\cT $ over Gramians $P$ and $Q$ in the MOR framework.