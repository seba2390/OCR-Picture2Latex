%Code for feynmf diagrams

\usepackage{feynmp,scalerel,ifpdf}

\ifpdf
  \DeclareGraphicsRule{*}{mps}{*}{}
\fi

\makeatletter
\def\endfmffile{%
  \fmfcmd{\p@rcent\space the end.^^J%
          end.^^J%
          endinput;}%
  \if@fmfio
    \immediate\closeout\@outfmf
  \fi
  \IfFileExists{\thefmffile.mp}{\immediate\write18{mpost \thefmffile}}{}
  \let\thefmffile\relax
}
\makeatother

%Diagrams to be saved
\newsavebox\feynLbpKccbar
\newsavebox\feynLbpKccbarpenguin
\newsavebox\feynLbpKmumupenguin
\newsavebox\feynLbpKmumubox
\newlength\tmplength

%Command defined for plotting a left brace before a box (e.g. Feynman diagram)
%Variable #1: Lift w.r.t. box zero level (near the bottom) in brace's dimensions
%Variable #2: Dimension of the brace w.r.t. box
%Variable #3: Box to which the brace is added
%Variable #4: Text to be put outside the brace
\newcommand\prefeynlabel[4]{%
  \setlength{\tmplength}{#2\ht#3}%
  \raisebox{#1\tmplength}{%
  \raisebox{\dimexpr.5\tmplength-.35\ht\strutbox}{#4}%
  \scaleto[1.7ex]{\raisebox{2.33pt}{$\lbrace$}}{\tmplength}}%
}

%Command defined for plotting a right brace after a box (e.g. Feynman diagram)
%Variables as for \prefeynbox
\newcommand\postfeynlabel[4]{%
  \setlength{\tmplength}{#2\ht#3}%
  \raisebox{#1\tmplength}{%
  \scaleto[1.7ex]{\raisebox{2.33pt}{$\rbrace$}}{\tmplength}%
  \raisebox{\dimexpr.5\tmplength-.35\ht\strutbox}{#4}}%
}
