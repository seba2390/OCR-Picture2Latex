\section{\boldmath\CP-odd observables}
\label{sec:cpv}
Two types of \CP-odd observables are studied in this paper. Following Refs.~\cite{Durieux:2015zwa,LHCb-PAPER-2016-030}, the differential rate of any pair of \CP-conjugate processes can be decomposed into four parts with definite even and odd transformation properties under the \CP and motion-reversal $\widehat{T}$ operators. Here, $\widehat{T}$ is the unitary operator that reverses both momentum and spin
three-vectors, to be distinguished from the antiunitary time-reversal operator $T$ which reverses initial and final states.

A $\widehat{T}$-even and \CP-odd asymmetry, $\mathcal{A}_{\CP}$, is related to the raw asymmetry $\mathcal{A}_{\text{raw}}$ of the observed decay candidates
\begin{equation}
\mathcal{A}_{\text{raw}} \equiv \frac{N(\LK)-N(\LbarK)}{N(\LK)+N(\LbarK)},
\end{equation}
via
\begin{equation}
\mathcal{A}_{\text{raw}} \approx \mathcal{A}_{\CP}(\LK) + \mathcal{A}_{\text{prod}}(\Lb) - \mathcal{A}_{\text{reco}}(K^+) + \mathcal{A}_{\text{reco}}(p)\label{eq::rawCP},
\end{equation}
where $\mathcal{A}_{\text{prod}}(\Lb)$ is the \Lb production asymmetry, due to the $pp$ initial state, and $\mathcal{A}_{\text{reco}}(K^+)$ and $\mathcal{A}_{\text{reco}}(p)$ are the reconstruction asymmetries for kaons and protons, mainly due to the different interaction cross-sections of particles and antiparticles with the detector material. By measuring the difference of raw asymmetries between the signal and the Cabibbo-favoured control mode $\LKJ(\to\mu^+\mu^-)$, the production and reconstruction asymmetries cancel to a good approximation. No significant \CPV is expected in the latter decay, since its amplitude is dominated by tree-level \CP-conserving diagrams, which leads to
\begin{equation}
\begin{split}
\DACP &\equiv \mathcal{A}_{\CP}(\LK) - \mathcal{A}_{\CP}(\LKJ)\\
&\approx\mathcal{A}_{\text{raw}}(\LK) -  \mathcal{A}_{\text{raw}}(\LKJ).
\end{split}
\label{eq::DACP}
\end{equation}
Imperfect cancellation in the production and reconstruction asymmetries can arise from differences in the kinematic distributions of the signal and control modes. A weighting procedure, discussed in Sec.~\ref{sec:asymmetries}, is applied to correct for this, with residual effects considered as a source of systematic uncertainty in Sec.~\ref{sec:systematics}.

A pair of $\widehat{T}$-odd and $P$-odd observables, \AT and \ATbar, is obtained by defining the $\widehat{T}$-odd triple products of the final-state particle momenta in the \Lb rest frame
\begin{align}
\CT &\equiv \vec{p}_{\mu^+}\cdot (\vec{p}_{p}\times\vec{p}_{K^-}), \\
\CTbar &\equiv \vec{p}_{\mu^-}\cdot (\vec{p}_{\bar{p}}\times\vec{p}_{K^+}),\label{eq::TP}
\end{align}
and taking the asymmetries
\begin{equation}
\AT \equiv \frac{N(\CT > 0) - N(\CT < 0)}{N(\CT > 0) + N(\CT < 0)},
\hspace{0.5cm}
\ATbar \equiv \frac{\overline{N}(-\CTbar > 0) - \overline{N}(-\CTbar < 0)}{\overline{N}(-\CTbar > 0) + \overline{N}(-\CTbar < 0)},\label{eq::ATCP}
\end{equation}
where $N$($\overline{N}$) is the number of \Lb(\Lbbar) signal candidates. These asymmetries are measured from the angular distributions of the decay products, with $\CT$ being proportional to $\sin\chi$~\cite{Gronau:2011cf}, where $\chi$ is the angle between the decay planes of the $\mup \mun$ and $p\Km$ systems in the \mbox{\Lb rest frame}, as shown in Fig.~\ref{fig:phiangle}.

\begin{figure}
\centering
\includegraphics[scale=0.35]{figs/DecayPlanes_pKMuMu_v1}
\caption{Definition of the angle $\chi$ for \LK decays, in the \Lb rest frame.\label{fig:phiangle}}
\end{figure}

The observables \AT and \ATbar are $P$- and $\widehat{T}$-odd but are not sensitive to \CPV effects~\cite{Durieux:2015zwa}. Following Ref.~\cite{Gronau:2011cf}, \CP-odd and $P$-odd observables are defined as
\begin{equation}
\ACP \equiv \frac{1}{2} \left( \AT - \ATbar\right),\hspace{2cm} \aPTodd \equiv \frac{1}{2} \left( \AT + \ATbar\right),
\label{eq::ACPTodd}
\end{equation}
where a non-zero value of \ACP or \aPTodd would signal \CP or parity violation, respectively.
These observables are by construction largely insensitive to the \Lb production asymmetry and detector-induced charge asymmetries.

The observables \DACP and \ACP are sensitive to different manifestations of \CPV~\cite{Durieux:2015zwa}.
%The \CP asymmetry $\mathcal{A}_{\CP}$ depends on the interference of $\widehat{T}$-even amplitudes, which have a relative \CP-even strong phase $\delta_1^e-\delta_2^e$ and a relative \CP-odd weak phase $\phi_1^e-\phi_2^e$,
%\begin{equation}
%\mathcal{A}_{\CP}\propto \sin(\delta_1^e-\delta_2^e)\sin(\phi_1^e-\phi_2^e).
%\end{equation}
%The convention used to define strong and weak phases is such that all \CPV effects are encoded in the \CP-odd weak phases.
The \CP asymmetry $\mathcal{A}_{\CP}$ depends on the interference of $\widehat{T}$-even amplitudes, which can be written as $\mathcal{M}^e_i = a^e_i \exp\left[i(\delta^e_i+\phi_i^e)\right]$, where $\delta^e_i$ are \CP-even strong phases, $\CP(\delta^e_i) = \delta^e_i$, and $\phi_i^e$ are \CP-odd weak phases, $\CP(\phi_i^e)=-\phi_i^e$. This convention is such that all \CPV effects are encoded in the \CP-odd weak phases. The $\widehat{T}$-even and \CP-odd part of the differential rate turns out to be
\begin{equation}
\left.\frac{\rm d\Gamma}{\rm d\Phi} \right|^{\widehat{T}-{\rm even}}_{\CP-{\rm odd}} \propto a^e_1 a^e_2 \sin(\delta_1^e-\delta_2^e)\sin(\phi_1^e-\phi_2^e),
\end{equation}
where only two $\widehat{T}$-even amplitudes are considered for simplicity.
Therefore, $\mathcal{A}_{\CP}$ is enhanced when the strong phase difference between the two amplitudes is large.\\
%On the other hand, \ACP depends on the interference between $\widehat{T}$-even and $\widehat{T}$-odd amplitudes, which have a relative \CP-even strong phase $\delta_1^e-\delta_1^o$ and a relative \CP-odd weak phase $\phi_1^e-\phi_1^o$,
%\begin{equation}
%\ACP\propto \cos(\delta_1^e-\delta_1^o)\sin(\phi_1^e-\phi_1^o).
%\end{equation}
On the other hand, \ACP depends on the interference between $\widehat{T}$-even and $\widehat{T}$-odd amplitudes, the latter written as $\mathcal{M}^o_i = a^o_i \exp\left[i(\delta^o_i+\phi_i^o+\pi/2)\right]$, following the same convention used for $\widehat{T}$-even amplitudes. The $\widehat{T}$-odd and \CP-odd part of the differential rate is therefore
\begin{equation}
\left.\frac{\rm d\Gamma}{\rm d\Phi} \right|^{\widehat{T}-{\rm odd}}_{\CP-{\rm odd}} \propto a^e_1 a^o_1 \cos(\delta_1^e-\delta_1^o)\sin(\phi_1^e-\phi_1^o),
\end{equation}
where one $\widehat{T}$-even and one $\widehat{T}$-odd amplitudes are considered for simplicity.
As a consequence, \ACP is enhanced when the strong phase difference vanishes. Furthermore, the observables \DACP and \ACP are sensitive to different types of \CPV effects from physics beyond the SM~\cite{Alok:2011gv}.