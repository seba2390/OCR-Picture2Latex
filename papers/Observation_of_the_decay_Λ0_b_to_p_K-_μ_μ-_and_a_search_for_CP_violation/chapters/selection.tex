\section{Selection of signal candidates}
\label{sec:sel}

The present analysis is performed using proton-proton collision data corresponding to \mbox{$1$ and} $2\invfb$ of integrated luminosity, collected with the LHCb detector in 2011 and 2012, at centre-of-mass energies of 7 and 8\tev, respectively. The \LK candidates are reconstructed from a proton, a kaon and two muon candidates originating from a common vertex, and are selected using information from the particle identification system. The \Lb flavour is determined from the charge of the kaon candidate, \ie \hspace{-4pt} \Lb for negative and \Lbbar for positive kaons.
%Hadron identification is performed using RICH detector information, while muon candidates are identified employing muon system information.
%
Only candidates with reconstructed invariant mass, $m(pK^-\mu^+\mu^-)$, in the range $[5350,6000]\mevcc$ and a $pK^-$ invariant mass, $m(pK^-)$, below 2350\mevcc are retained, with the latter requirement being applied to reduce the combinatorial background contribution. The spectrum in the dimuon mass squared, $q^2$, is considered, excluding the resonance regions
$q^2 \in [0.98,1.10]$, $[8.0,11.0]$ and $[12.5,15.0]{\mathrm{\,Ge\kern -0.1em V^2\!/}c^4}$ that correspond to the masses of the $\Pphi(1020)$, \jpsi, and \psitwos mesons, respectively.

Several background contributions from exclusive decays are identified and rejected.
These are $\BsKKmm$ and $\BdKpimm$ decays, in which a kaon or a pion is
misidentified as a proton, and \LK decays, in which proton and kaon assignments are interchanged.
Background also arises from $\Lb\to p\Km \jpsi$ and $\Lb\to p \Km \psitwos$ decays in which a muon is misidentified as a kaon and the kaon as a muon.
%These components are vetoed by applying both invariant mass and particle-identification based requirements. 
These components are effectively eliminated by tightened particle identification
requirements combined with selection criteria on invariant masses
calculated under the appropriate mass hypothesis (\eg assigning the kaon mass to
the candidate proton to identify possible \BsKKmm background decays).
% These components are vetoed by applying combined selection criteria on invariant masses
% calculated with interchanged mass hypotheses, \eg reconstructing proton candidates with the
% kaon (pion) mass hypothesis for \BsKKmm ($\BdKpimm$) backgrounds, and particle identification requirements.
After these requirements the background contribution from the above decays is negligible. No indication of other specific background decays is observed.
%\begin{align}
%m(KK\mumu) \in [5300,5420] \;&\&\& \;m(KK)\in[1010,1035] , {\tt OR} \nonumber \\
%|m(KK\mumu) -5366|<40 \;&\&\& \;m(KK)<1080 \;\&\& \; {\rm DLL}_{pK}(p)<30, {\tt OR} \nonumber \\
%|m(KK\mumu) -5366|<10 \;&\&\& \;{\rm DLL}_{pK}(p)<0. \nonumber 
%\end{align}
%\begin{align}
%|m_X-895|<100 \;\&\&\;  |m_B - 5279|<45 \;\&\&\; & {\rm DLL}_{K\pi}(K) <5 \;\&\&\; {\rm DLL}_{pK}(p) < 15, {\tt OR} \nonumber \\
%|m_X-1430|<100 \;\&\&\; |m_B - 5279|<45 \;\&\&\; & {\rm DLL}_{K\pi}(K) <5 \;\&\&\; {\rm DLL}_{pK}(p) < 15, {\tt OR} \nonumber \\
%|m_B - 5279|<30 \;\&\&\; & {\rm DLL}_{K\pi}(K) <3 \;\&\&\; {\rm DLL}_{pK}(p) < 10 \nonumber 
%\end{align}
The remaining combinatorial background is suppressed by means of a boosted decision tree (BDT)
classifier~\cite{Breiman,Roe} with an adaptive boosting algorithm \cite{AdaBoost}. The BDT is constructed from variables that discriminate between signal and background, based on their kinematic, topological and particle
identification properties, as well as the isolation of the final-state tracks~\cite{Aaij:2011rja,Aaij:2015bfa}. %Simulated phase-space signal events, after correcting for known differences between data and simulation, are used as the signal training sample.
Simulated \LK events in which the decay products are uniformly distributed in phase space are used as the signal training sample and a correction for known differences between data and simulation is applied.
Candidates from data in the high mass region, $m(p\Km\mup\mun)>5800\mevcc$, are used as the background training sample and then removed from the window of the mass fit described below. After optimisation of the significance, $S/\sqrt{S+B}$, where $S$ and $B$ are the number of signal and background candidates in the region $m(pK^-\mu^+\mu^-) \in [5400,5800]\mevcc$, the BDT classifier retains only $0.14\%$ of the combinatorial background candidates, with a signal efficiency of $51\%$. Events in which more than one \Lb candidate survives the selection constitute less than $1\%$ of the sample and all candidates are retained; the systematic uncertainty associated with this is negligible. The identical selection is applied to the control-mode \LKJ, except that the dimuon squared mass is required to be in the range $[9.0,10.5]{\mathrm{\,Ge\kern -0.1em V^2\!/}c^4}$.