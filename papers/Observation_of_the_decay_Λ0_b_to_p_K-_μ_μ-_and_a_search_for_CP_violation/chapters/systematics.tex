\section{Systematic uncertainties}
\label{sec:systematics}
%A systematic uncertainty to the proton and kaon momentum weighting, affecting the \DACP observable, is assigned by taking the largest \DACP deviation when weights are varied within their uncertainties. This uncertainty is found to be $\pm\; 0.15\times 10^{-2}$.
The analysis method depends upon the weighting procedure discussed in Sec.~\ref{sec:asymmetries} to equalise the kinematic distributions of the protons and kaons between the signal and control modes. For \DACP, the associated systematic uncertainty is estimated by varying the weights within their uncertainties and taking the largest deviation, $\pm\; 0.15\times 10^{-2}$, as a systematic uncertainty. No weighting is needed for \ACP and \aPTodd, and therefore no systematic uncertainty is assigned.
Instead, the effects of selection and detector acceptance on the triple-product asymmetries are estimated by measuring $\ACP(p\Km\jpsi)$ on the control mode, \LKJ. A value of $(0.5\pm 0.7)\times 10^{-2}$ is obtained. For this mode negligible \CPV is expected, and the statistical uncertainty of the measured asymmetry is assigned as the corresponding systematic uncertainty on the observables \ACP and \aPTodd.
The effects of the reconstruction efficiency on the measured observables are considered by weighting each event by the inverse of the efficiency extracted from simulated events. This leads to a change in the central values of $+1.3\times 10^{-2}$ on \DACP, of $+0.6\times 10^{-2}$ on \ACP and of $-1.4\times 10^{-2}$ on \aPTodd.
%, the values of the observables are found to be consistent with those derived in Sec.~\ref{sec:asymmetries}, and 
A systematic uncertainty is assigned by varying the efficiencies within their uncertainties. This amounts to $\pm\; 0.10\times 10^{-2}$ for the \DACP observable and to $\pm\; 0.02\times 10^{-2}$ for \ACP and \aPTodd.

The above effects are the dominant sources of systematic uncertainties. Other possible sources of systematic uncertainties are considered. The experimental resolution on \CT is studied with simulated signal events. The effect of the fit model choice is studied by fitting simulated pseudoexperiments with an alternative fit model, in which the Crystal Ball functions are replaced with bifurcated Gaussian functions and the exponential background shape is replaced with a polynomial.
Systematic effects from \Lb polarisation~\cite{LHCb-PAPER-2012-057}, multiple candidates, and residual physical backgrounds are also studied. These contributions have negligible impact on the measured asymmetries.

%\begin{table}
%\centering
%\begin{tabular}{lcccc}
%\hline
%Source & \DACP ($10^{-2}$)& \AT ($10^{-2}$) & \ATbar ($10^{-2}$) & \ACP ($10^{-2}$)\\
%\hline
%Experimental bias & $\pm$ 0.3 & $\pm$ 1.0 & $\pm$ 1.0 & $\pm$ 0.7\\
%$C_T$ resolution bias & unaffected & $\pm$ 0.00 & $\pm$ 0.04 & $\pm$ 0.02\\
%Fit model & negligible & \multicolumn{3}{c}{negligible}\\
%\Lb polarization & unaffected & \multicolumn{3}{c}{negligible}\\
%Multiple candidates & negligible & \multicolumn{3}{c}{negligible}\\
%Peaking backgrounds & negligible & \multicolumn{3}{c}{negligible}\\
%\hline
%\end{tabular}
%\caption{Summary of the systematic uncertainties considered in this analysis.\label{tab::systematicssummary}}
%\end{table}

