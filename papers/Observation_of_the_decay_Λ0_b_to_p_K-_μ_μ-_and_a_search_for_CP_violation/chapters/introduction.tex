\section{Introduction}
\label{sec:introduction}

The phenomenon of \CP violation (\CPV), related to the difference in behaviour between matter and antimatter, remains an intriguing topic more than fifty years after its discovery in the neutral kaon system~\cite{PhysRevLett.13.138}. Within the Standard Model of particle physics (SM), \CPV is incorporated by a single, irreducible weak phase in the \mbox{Cabibbo-Kobayashi-Maskawa (CKM)} quark mixing matrix~\cite{1963PhRvL..10..531C,1973PThPh..49..652K}. However, the amount of \CPV in the SM is insufficient to explain the observed level of matter-antimatter asymmetry in the Universe~\cite{1991SvPhU..34..392S,Gavela:1993ts,Gavela:1994dt}. Therefore, new sources of \CPV beyond the SM are expected to exist. Experimental observations of \CPV remain confined to the $B$- and $K$-meson systems. Recently, the first evidence for \CPV in $\Lb\to p\pi^-\pi^+\pi^-$ was found at the level of $3.3$ standard deviations~\cite{LHCb-PAPER-2016-030} and a systematic study of \CPV in beauty baryon decays has now begun.
%Given that the bulk of observable matter content in the universe is baryonic, it is particularly desirable to search for \CPV in baryonic systems, specifically, in decays of $b$-baryons. 

Among dedicated heavy-flavour physics experiments, the LHCb detector~\cite{Alves:2008zz} is unique in having access to a wide range of decay modes of numerous $b$-hadron species. Beauty baryons are produced copiously at the LHC, and within the LHCb detector acceptance the production ratio of $\Bz:\Lb:\Bs$ particles is approximately $4:2:1$~\cite{Aaij:2011jp}.
%The large number of $\Lb$ baryons available in a clean environment, thanks to the excellent vertexing and particle identification of the LHCb detector, has led to unexpected discoveries such as the Pentaquark states~\cite{Aaij:2015tga}.
The \lhcb collaboration has previously searched for \CPV in \LpiJ and \LKJ decays~\cite{LHCb-PAPER-2014-020}, as well as in charmless $\Lb\to p\KS\pi^-$, $\Lb\to\Lz\phi$ and $\Lb\to\Lz h^ +h^-$ transitions~\cite{LHCb-PAPER-2013-061,LHCb-PAPER-2016-002,LHCb-PAPER-2016-004}.
%
\begin{figure}[h]
\centering
%feynmf code
%\prefeynlabel{-.2}{.5}{\feynLbpKmumupenguin}{$\Lb$}
%\usebox{\feynLbpKmumupenguin}%
%\hspace{0.1cm}\postfeynlabel{-0.2}{.5}{\feynLbpKmumupenguin}{$p$}
%\hspace{-0.55cm}\postfeynlabel{1.5}{.3}{\feynLbpKmumupenguin}{$K^-$}\label{fey::LbpKmumupenguin}
%\prefeynlabel{-.2}{.5}{\feynLbpKmumubox}{$\Lb$}
%\usebox{\feynLbpKmumubox}%
%\hspace{0.1cm}\postfeynlabel{-0.2}{.5}{\feynLbpKmumubox}{$p$}
%\hspace{-0.55cm}\postfeynlabel{1.5}{.3}{\feynLbpKmumubox}{$K^-$}\label{fey::LbpKmumubox}
%pdf figures
\includegraphics[scale=1.]{Diagram_penguin.pdf}
\includegraphics[scale=1.]{Diagram_box.pdf}
\caption{Diagrams for the decay \LK, in which $V_{bq}$ and $V_{qs}$ are CKM matrix elements and $q$ represents one of the three up-type quarks $u$, $c$ or $t$, the $t$-quark contribution being dominant. The $u\overline{u}$ pairs originate from the hadronization process.\label{fey::LbpKmumu}}
\end{figure}
%

In this paper, a search for \CPV in the hitherto unobserved decay \LK is reported.\footnote{The inclusion of charge-conjugate processes is implied throughout this paper, unless stated otherwise.}
%The first observation of this decay by the LHCb collaboration is reported in~\cite{Aaij:2016pkmumu}.
It is a flavour-changing neutral-current process with the underlying quark-level transition $b \to s \mup \mun$. The leading-order transition amplitudes in the SM are described by the loop diagrams shown in Fig.~\ref{fey::LbpKmumu}. In extensions to the SM, new heavy particles could contribute to the amplitudes with additional weak phases,
providing new sources of \CPV~\cite{Gauld:2013qja,Paracha:2014hca}. The limited amount of \CPV predicted for the decay \LK in the SM~\cite{Alok:2011gv,Paracha:2014hca}, following from the CKM matrix elements shown in Fig.~\ref{fey::LbpKmumu}, makes this decay particularly sensitive to \CPV effects from physics beyond the SM. 
%In particular, the study of this decay can help shed light on the pattern of anomalies in the rate and angular distribution of $b \to s \ellm \ellp$ processes~\cite{LHCb-PAPER-2015-051, LHCb-PAPER-2014-024, LHCb-PAPER-2015-023, LHCb-PAPER-2014-006, LHCb-PAPER-2016-025}.
