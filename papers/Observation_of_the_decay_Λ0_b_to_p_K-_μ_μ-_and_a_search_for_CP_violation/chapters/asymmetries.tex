\section{Asymmetry measurements}
\label{sec:asymmetries}

For the \DACP measurement, the data are divided into two subsamples according to the \Lb flavour. For the measurements of the triple-product asymmetries, four subsamples are defined by the combination of the \Lb flavour and the sign of \CT(or \CTbar for \Lbbar). The reconstruction efficiencies are studied with simulated events and are found to be equal for all subsamples.

The observable \DACP can be sensitive to kinematic differences between the signal and control-mode decays that affect the cancellation of the detection asymmetries in Eq.~\ref{eq::DACP}. This is taken into account by assigning a weight to each \LKJ candidate such that the resulting proton and kaon momentum distributions match those of the signal \LK decays. These weights are determined from simulation samples for the signal and control modes. No such weighting is required for \ACP and \aPTodd, since these observables involve only one decay mode.

The asymmetry \Araw is determined from a simultaneous extended maximum likelihood unbinned fit to the \Lb and \Lbbar invariant mass distributions. The \AT and \ATbar asymmetries are determined by means of a simultaneous extended maximum likelihood unbinned fit to the four subsamples defined above.
The signal model for all fits is the sum of two Crystal Ball functions \cite{Skwarnicki:1986xj}, one with a low-mass power-law tail and one with a high-mass tail, and a Gaussian function, all sharing the same peak position. Only the peak position, the total width of the composite function and the overall normalization are free to vary, with all other shape parameters fixed from a fit to simulated decays. The background is modelled by an exponential function. The raw asymmetry \Araw is incorporated in the fit model as
\begin{equation}
N_{\Lbbar} = N_{\Lb} \frac{1-\Araw}{1+\Araw},
\label{eq::Lbbaryield}
\end{equation}
and \DACP is derived from the raw asymmetries measured in the signal and control modes according to Eq.~\ref{eq::DACP}. The asymmetries \AT and \ATbar are included in the fit as
%\begin{align}
%N_{\Lb,\CT>0} = \frac{1}{2} N_{\Lb} (1+\AT),\nonumber\\
%N_{\Lb,\CT<0} = \frac{1}{2} N_{\Lb} (1-\AT),\nonumber\\
%N_{\Lbbar,-\CTbar>0} = \frac{1}{2} N_{\Lbbar} (1+\ATbar),\nonumber\\
%N_{\Lbbar,-\CTbar<0} = \frac{1}{2} N_{\Lbbar} (1-\ATbar),
%\end{align}
\def\arraystretch{1.8}
\begin{equation}
\begin{array}{cc}
N_{\Lb,\CT>0} = \frac{1}{2} N_{\Lb} (1+\AT), & N_{\Lb,\CT<0} = \frac{1}{2} N_{\Lb} (1-\AT),\\
N_{\Lbbar,-\CTbar>0} = \frac{1}{2} N_{\Lbbar} (1+\ATbar), & N_{\Lbbar,-\CTbar<0} = \frac{1}{2} N_{\Lbbar} (1-\ATbar),\\
\end{array}
\end{equation}
and the observables \ACP and \aPTodd are computed from \AT and \ATbar, which are found to be uncorrelated. Background yields are fitted independently for each subsample, while all the signal shape parameters are shared among the subsamples.

The invariant mass distributions of \LK and \LKJ candidates, with fit results superimposed, are shown in Fig.~\ref{fig:ACPfit}.
%The results of the fits to the \LK and \LKJ samples used in the \DACP measurement are shown in Fig.~\ref{fig:ACPfit}.
The \Araw asymmetries are found to be $(-2.8\pm 5.0)\times 10^{-2}$ for signal decays and $(2.0\pm 0.7)\times 10^{-2}$ for the control mode, which yields efficiency-uncorrected $\DACP = (-4.8 \pm 5.0)\times 10^{-2}$. The total signal yields from the fits to the data are $600 \pm 44$ candidates for \LK, and $22\,911\pm 230$ for \LKJ decays. The uncertainties are statistical only. This represents the first observation of the \LK decay mode.
\begin{figure}
\centering
\includegraphics[scale=0.6]{figs/ACP_mumu.pdf}
\includegraphics[scale=0.6]{figs/ACP_Jpsi.pdf}
\caption{Invariant mass distributions of (top) \LK and (bottom) \LKJ candidates, with fit results superimposed. Plots refer to the (left) \Lb and (right) \Lbbar subsamples.\label{fig:ACPfit}}
%\caption{Fits used in the \Araw measurements for the decay modes (top) \LK and (bottom) \LKJ. Plots refer to the (left) \Lb and (right) \Lbbar subsamples.\label{fig:ACPfit}}
\end{figure}

\begin{figure}
\centering
\includegraphics[scale=0.6]{figs/Todd_mumu_Lb.pdf}
\includegraphics[scale=0.6]{figs/Todd_mumu_Lbbar.pdf}
\caption{Invariant mass distributions of the \LK subsamples used for the \AT and \ATbar measurements. Plots refer to (top) \Lb and (bottom) \Lbbar decays divided into the subsamples (left) $\CT>0,-\CTbar>0$ and (right) $\CT<0,-\CTbar<0$.\label{fig::Toddfit}}
\end{figure}

The invariant mass distributions of the \LK subsamples used for the \AT and \ATbar measurements, with fit results superimposed, are shown in Fig.~\ref{fig::Toddfit}. From the signal yields, the triple-product asymmetries are found to be $\AT = (-2.8\pm 7.2)\times 10^{-2}$ and $\ATbar = (4.0\pm 6.9)\times 10^{-2}$, and the resulting efficiency-uncorrected parity- and \CP-violating observables are
$\aPTodd = (-3.4 \pm 5.0)\times 10^{-2}$ and 
$\ACP = (0.6\pm 5.0)\times 10^{-2}$, where again the uncertainties are statistical only.
% and the $P$-violating observable  is $\aPTodd = (-3.4\pm 5.0)\times 10^{-2}$.
