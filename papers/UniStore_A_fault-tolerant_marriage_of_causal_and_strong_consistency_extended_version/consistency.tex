\section{Consistency Model}
\label{sec:consistency}



A client interacts with \System by executing a stream of {\em transactions} at
the data center it is connected to. A transaction consists of a sequence of
operations, each on a single data item, and can be {\em interactive}: the data
items it accesses are not known a priori. A transaction that modifies at least
one data item is an {\em update} transaction; otherwise it is {\em read-only}.













A {\em consistency model} defines a contract between the data store and its
clients that specifies which values the data store is allowed to return in
response to client operations. \System implements a transactional variant of
{\em Partial Order-Restrictions consistency (PoR
  consistency)}~\cite{por,cise-popl16}, which we now define informally; we give
a formal definition in~\tr{\ref{section:spec}}{\nappconsistency}.
The PoR model enables the programmer to classify transactions as either {\em causal} or
{\em strong}. Causal transactions satisfy transactional {\em causal
  consistency}, which guarantees that clients see transactions in an order that
respects the potential causality between them~\cite{causal-memory,cure}.
However, clients can observe causally independent transactions in arbitrary
order. Strong transactions give the programmer more control over their
visibility. To this end, the programmer provides a symmetric {\em conflict
  relation} $\bowtie$ on operations
that is lifted to strong transactions as follows: two transactions conflict
if they perform conflicting operations on the same data item. Then the PoR model
guarantees that,
out of two conflicting strong transactions, one has to observe the
other.

More precisely, a transaction $t_1$ precedes a transaction $t_2$ in the {\em
  session order} if they are executed by the same client and $t_1$ is executed
before $t_2$. A set of transactions $T$ committed by the data store satisfies
PoR
consistency if there exists a {\em causal order}
relation $\prec$ on $T$ such that the following properties hold:
\begin{description}[noitemsep,topsep=3pt,parsep=3pt,leftmargin=10pt]
\item[Causality Preservation.] The relation $\prec$ is transitive, irreflexive,
  and includes the session order.
\item[Return Value Consistency.] Consider an operation $u$ on a data item $k$ in
  a transaction $t \in T$. The return value of $u$ can be computed from the
  state of $k$ obtained as follows: first execute
  all operations on $k$ by transactions preceding $t$ in $\prec$ in an order
  consistent with $\prec$; then execute
  all operations on $k$ that precede $u$ in $t$.
\item[Conflict Ordering.] For any distinct strong transactions $t_1, t_2 \in T$,
  if $t_1\bowtie t_2$, then either $t_1\prec t_2$ or $t_2\prec t_1$.
\item[Eventual Visibility.] A transaction $t \in T$ that is either strong or
  originates at a correct data center eventually becomes visible at all correct
  data centers: from some point on, $t$ precedes in $\prec$ all transactions
  issued at correct data centers.
\end{description}
If all transactions are causal, then the above definition specializes to
transactional causal consistency~\cite{cure,framework-concur15}. If all
transactions are strong and all pairs of operations conflict, then we obtain
(non-strict) serializability.

When $t_1 \prec t_2$, we say that $t_1$ is a {\em causal dependency} of $t_2$.
Return Value Consistency ensures that all operations in a transaction $t$
execute on a snapshot consisting of its causal dependencies (as well as prior
operations by $t$). Transactions are atomic, so that either all of their
operations are included into the snapshot or none at all. The transitivity of
$\prec$, mandated by Causality Preservation, ensures that the snapshot a
transaction executes on is \emph{causally consistent}: if a transaction $t_1$ is
included into the snapshot, then so is any other transaction $t_2$ on which
$t_1$ depends (i.e., $t_2\prec t_1$). The inclusion of the session order into
$\prec$, also mandated by Causality Preservation, ensures session guarantees
such as {\em read your writes}~\cite{session}. The consistency model disallows
the causality violation anomaly from \S\ref{sec:intro}. Indeed, since $\prec$
includes the session order, we have $u_1 \prec u_2$ and $u_3 \prec
u_4$. Moreover, Bob sees Alice's message, and by Return Value Consistency this
can only happen if $u_2 \prec u_3$. Then since $\prec$ is transitive,
$u_1 \prec u_4$, and by Return Value Consistency, Bob has to see Alice's
deposit.

Causal consistency nevertheless allows the overdraft anomaly from
\S\ref{sec:intro}: the withdrawals $u_5$ and $u_6$ may not be related by
$\prec$, and thus may both execute on the balance $100$ and succeed. The
Conflict Ordering property can be used to disallow this anomaly by declaring
that ${\tt withdraw}$ operations on the same account conflict and labeling
transactions containing these as strong.
Then one of the withdrawal transactions will be guaranteed to causally precede
the other. The latter will be executed on the account balance $0$ and will fail.

Finally, Eventual Visibility ensures that strong transactions and those causal
ones that originate at correct data centers are durable, i.e., will eventually
propagate through the system.





To facilitate the use of causal transactions, \System includes {\em replicated
  data types} (aka CRDTs), which implement policies for merging concurrent
updates to the same data item~\cite{crdts}.
Each data item in the store is associated with a type (e.g., counter, set),
which is backed by a CRDT implementation managing updates to it. For example,
the programmer can use a counter CRDT to represent an account balance. Then if
two clients concurrently deposit $100$ and $200$ into an empty account using
causal transactions, eventually the balance at all replicas will be $300$. Using
ordinary writes here would yield $100$ or $200$, depending on the order in which
the writes are applied. More generally, CRDTs ensure
that two replicas receiving the same set of updates are in the same state,
regardless of the receipt order. Together with Eventual Visibility, this implies
the expected guarantee of eventual consistency~\cite{bayou}. Due to space
constraints, we omit details about the use of CRDTs from our protocol
descriptions.















