\documentclass[letterpaper,twocolumn,10pt, usenames,dvipsnames]{article}
\usepackage{usenix-2020-09}

\newif\iflong
\longtrue %

\iflong
\else
\pagestyle{empty}
\fi

\usepackage{todonotes}
\usepackage{url}
\usepackage{graphicx}
\usepackage{xspace}
\usepackage{tikz,pgfplots}

\usepgfplotslibrary{statistics}


\newcommand{\system}{\textsc{Groot}\xspace}


\theoremstyle{definition}
\newtheorem{apptheorem}{Theorem}
\newtheorem{applemma}[apptheorem]{Lemma}
\newtheorem{appdefinition}[apptheorem]{Definition}
\newtheorem{appassumption}{\textsc{Assumption}}

\newcommand{\set}[1]{\{#1\}}
\DeclareMathOperator*{\argmin}{argmin}

\newcommand{\hStatex}{\vspace{4pt}}

\newcommand{\causalentry}{\textsl{causal}}
\newcommand{\strongentry}{\textsl{strong}}

\newcommand{\IfThenElse}[3]{
  \State \algorithmicif\ #1\ \algorithmicthen\ #2\ \algorithmicelse\ #3}

\newcommand{\redcolor}[1]{\textcolor{red}{#1}}
\newcommand{\bluecolor}[1]{\textcolor{blue}{#1}}
\newcommand{\teal}[1]{\textcolor{teal}{#1}}
\newcommand{\cyan}[1]{\textcolor{cyan}{#1}}
\newcommand{\purple}[1]{\textcolor{purple}{#1}}

\newcommand{\lccolor}[1]{\bluecolor{#1}}
\newcommand{\tscolor}[1]{\purple{#1}}
\newcommand{\strongcolor}[1]{\redcolor{#1}}
\newcommand{\versioncolor}[1]{\cyan{#1}}

\newcommand{\unistore}{\textsc{UniStore}}
\newcommand{\code}[2]{#1:#2}

\newcommand{\pastVC}{{\sf pastVec}}
\newcommand{\ok}{\mathrm{ok}}

\newcommand{\var}{{\bf var}\;}
\newcommand{\pre}{{\bf pre:}\;}
\newcommand{\rpc}{{\bf remote\; call}\;}
\newcommand{\at}{\;{\bf at}\;}
\newcommand{\send}{{\bf send}\;}
\newcommand{\sendto}{\;{\bf to}\;}
\newcommand{\from}{\;{\bf from}\;}
\newcommand{\repeatkw}{{\bf repeat}\;}
\newcommand{\notkw}{{\bf not}\;}
\newcommand{\timeout}{{\bf timeout}\;}
\newcommand{\wait}{{\bf wait}\;}
\newcommand{\asyncwait}{{\bf async wait}\;}
\newcommand{\until}{{\bf until}\;}
\newcommand{\receive}{{\bf receive}\;}
\newcommand{\received}{{\bf received}\;}
\newcommand{\upon}{{\bf upon}\;}
\newcommand{\upcall}{{\bf upcall}\;}
\newcommand{\prop}[1]{\textbf{\textsc{Property #1}}}

\newcommand{\N}{\mathbb{N}}
\newcommand{\D}{\mathcal{D}}
\newcommand{\C}{\mathcal{C}}
\newcommand{\OP}{O}
\newcommand{\cdrange}{c} %
\renewcommand{\P}{\mathcal{P}}
\newcommand{\opfunc}{\textsl{op}}

\newcommand{\p}{\mathit{p}}
\newcommand{\decvar}{\mathit{dec}}
\newcommand{\clientdc}{{\sf d}}
\newcommand{\cldc}{\textsl{cur\_dc}}
\newcommand{\clientcoord}{{\sf p}}
\newcommand{\starttx}{\textsc{start\_tx}}
\newcommand{\start}{\textsc{start}}
\renewcommand{\read}{\textsc{read}}
\newcommand{\updateproc}{\textsc{update}}
\newcommand{\commit}{\textsc{commit}}
\newcommand{\commitcausaltx}{\textsc{commit\_causal\_tx}}
\newcommand{\commitstrongtx}{\textsc{commit\_strong\_tx}}
\newcommand{\abort}{\textsc{abort}}
\newcommand{\commitcausal}{\textsc{commit\_causal}}
\newcommand{\commitstrong}{\textsc{commit\_strong}}

\newcommand{\vc}{V}
\newcommand{\vcvar}{\mathit{vc}}
\newcommand{\generatetid}{\texttt{generate\_tid}}
\newcommand{\partitionofproc}{\mathtt{partition}}
\newcommand{\doop}{\textsc{do\_op}}
\newcommand{\doread}{\textsc{do\_read}}
\newcommand{\doupdate}{\textsc{do\_update}}

\newcommand{\find}{\texttt{find}}
\newcommand{\readkey}{\textsc{get\_version}}
\newcommand{\getversion}{\textsc{get\_version}}
\newcommand{\versionproc}{\textsc{version}}
\newcommand{\prepare}{\textsc{prepare}}
\newcommand{\prepareack}{\textsc{prepare\_ack}}

\newcommand{\heartbeat}{\textsc{heartbeat}}
\newcommand{\replicate}{\textsc{replicate}}
\newcommand{\propagate}{\textsc{propagate\_local\_txs}}
\newcommand{\forward}{\textsc{forward\_remote\_txs}}

\newcommand{\bcast}{\textsc{broadcast\_vecs}}
\newcommand{\knownvclocal}{\textsc{knownvec\_local}}
\newcommand{\stablevcproc}{\textsc{stablevec}}
\newcommand{\knownvcglobal}{\textsc{knownvec\_global}}

\renewcommand{\log}{\textsl{log}}
\newcommand{\oplog}{{\sf opLog}}
\newcommand{\txlog}{{\sf txLog}}
\newcommand{\ep}{{\sf ep}}
\newcommand{\shards}{{\sf shards}}

\newcommand{\events}{\textsl{events}}
\newcommand{\ws}{\textsl{ws}}
\newcommand{\rs}{\textsl{rs}}
\newcommand{\rsvar}{\mathit{rs}}
\newcommand{\key}{\textsl{key}}
\newcommand{\client}{\textsl{client}}
\newcommand{\clients}{\mathbb{C}}
\newcommand{\dc}{\textsl{dc}}
\newcommand{\coord}{\textsl{coord}}
\newcommand{\startoftx}{\textsl{st}}
\newcommand{\commitoftx}{\textsl{ct}}
\newcommand{\partitionsfunc}{\textsl{partitions}}
\newcommand{\snapshotVC}{\textsl{snapshotVec}}
\newcommand{\snapshotproc}{\texttt{snapshot}}
\newcommand{\commitVC}{\textsl{commitVec}}
\newcommand{\prepareVC}{\textsl{prepareVec}}
\newcommand{\certfunc}{F}
\newcommand{\fdec}{\certfunc_{\rm dec}}
\newcommand{\fvec}{\certfunc_{\rm vec}}
\newcommand{\flc}{\certfunc_{\rm lc}}

\newcommand{\ud}{\textsl{ud}} %
\newcommand{\W}{W}
\newcommand{\R}{R}
\newcommand{\payload}{W}

\newcommand{\txs}{T}
\newcommand{\txsvar}{\mathit{txs}}
\newcommand{\txsincertify}{T_{c}}
\newcommand{\causaltxs}{\txs_{\causalentry}}
\newcommand{\strongtxs}{\txs_{\strongentry}}
\newcommand{\allstrongtxs}{\txs_{\textsl{all-strong}}}

\newcommand{\conflict}{\bowtie}

\newcommand{\intcertify}{{\sf certify}}
\newcommand{\intdecide}{{\sf decide}}
\newcommand{\intdeliver}{{\sf deliver}}
\newcommand{\intdeliverinv}{{\sf deliver\_inv}}
\newcommand{\intdeliverresp}{{\sf deliver\_resp}}
\newcommand{\act}{{\sf act}}
\newcommand{\actvar}{\mathit{act}}
\newcommand{\complete}{\mathit{complete}}
\newcommand{\lccertify}{C}

\newcommand{\Decision}{\mathbb{D}}
\newcommand{\Vector}{\mathbb{V}}
\newcommand{\Key}{\mathit{Key}}
\newcommand{\Val}{\mathit{Val}}

\newcommand{\certify}{\textsc{certify}}
\newcommand{\preparestrong}{\textsc{prepare\_strong}}
\newcommand{\heartbeatstrong}{\textsc{heartbeat\_strong}}
\newcommand{\alreadydecided}{\textsc{already\_decided}}
\newcommand{\acceptack}{\textsc{accept\_ack}}
\newcommand{\replicas}{\textsc{replicas}}
\newcommand{\leaderproc}{\textsc{leader}}
\newcommand{\leaderof}{{\sf leader}}
\newcommand{\decision}{\textsc{decision}}
\newcommand{\decisionvar}{\mathit{decision}}
\newcommand{\learndecision}{\textsc{learn\_decision}}
\newcommand{\follower}{\textsc{follower}}
\newcommand{\certification}{\textsc{certification\_check}}
\newcommand{\deliver}{\textsc{deliver}}
\newcommand{\deliverupdates}{\textsc{deliver\_updates}}
\newcommand{\newleader}{\textsc{new\_leader}}
\newcommand{\newleaderack}{\textsc{new\_leader\_ack}}
\newcommand{\newstate}{\textsc{new\_state}}
\newcommand{\newstateack}{\textsc{new\_state\_ack}}
\newcommand{\retry}{\textsc{retry}}

\newcommand{\votevar}{\mathit{vote}}
\renewcommand{\leaders}{\mathit{leaders}}

\newcommand{\reqIdVar}{\mathit{rid}}
\newcommand{\generateReqId}{\texttt{generate\_req\_id}}
\newcommand{\trustedVar}{{\sf trusted}}
\newcommand{\recover}{\textsc{recover}}
\newcommand{\nack}{\textsc{nack}}
\newcommand{\normalMode}{\textsc{normal}}
\newcommand{\recovering}{\textsc{recovering}}
\newcommand{\restoring}{\textsc{restoring}}
\newcommand{\doNotWaitFor}{{\sf doNotWaitFor}}
\newcommand{\unknownTx}{\textsc{unknown\_tx}}
\newcommand{\unknowntxAck}{\textsc{unknown\_tx\_ack}}
\newcommand{\unknowntx}{\textsc{unknown}}
\newcommand{\callerMode}{\mathit{callerMode}}
\newcommand{\senderMode}{\mathit{senderMode}}

\newcommand{\dvar}{\mathit{d}}
\newcommand{\mvar}{\mathit{m}}
\newcommand{\nvar}{\mathit{n}}
\newcommand{\pvar}{\mathit{p}}
\newcommand{\lvar}{\mathit{l}}
\newcommand{\cl}{\mathit{cl}}
\newcommand{\preparevc}{\mathit{prepareVec}}
\newcommand{\snapvc}{\mathit{snapVec}}
\newcommand{\commitvc}{\mathit{commitVec}}
\newcommand{\knownvc}{\mathit{knownVec}}
\newcommand{\stablevc}{\mathit{stableVec}}
\newcommand{\preparedstrongvar}{\mathit{preparedStrong}}
\newcommand{\decidedstrongvar}{\mathit{decidedStrong}}

\newcommand{\knownVC}{{\sf knownVec}}
\newcommand{\stableVC}{{\sf stableVec}}
\newcommand{\snapVC}{{\sf snapVec}}
\newcommand{\uniformVC}{{\sf uniformVec}}
\newcommand{\localmatrix}{{\sf localMatrix}}
\newcommand{\globalmatrix}{{\sf globalMatrix}}
\newcommand{\stablematrix}{{\sf stableMatrix}}

\newcommand{\realtime}{\tau}
\newcommand{\txfunc}{\textsl{tx}}
\newcommand{\txvar}{\mathit{tx}}
\newcommand{\tidselector}{\textsl{tid}}
\newcommand{\tidvar}{\mathit{tid}}
\newcommand{\tids}{{\sf TID}}
\newcommand{\tvar}{\mathit{t}}
\newcommand{\ctid}{{\sf tid}} %
\newcommand{\rset}{{\sf rset}}
\newcommand{\rsetvar}{\mathit{rset}}
\newcommand{\wbuff}{{\sf wbuff}}
\newcommand{\wbuffvar}{\mathit{wbuff}}
\newcommand{\clockVar}{{\sf clock}}
\newcommand{\statusVar}{{\sf status}}
\newcommand{\committedVar}{{\sf committed}}
\newcommand{\committedcausal}{{\sf committedCausal}}
\newcommand{\preparedcausal}{{\sf preparedCausal}}
\newcommand{\preparedstrong}{{\sf preparedStrong}}
\newcommand{\decidedstrong}{{\sf decidedStrong}}
\newcommand{\accept}{\textsc{accept}}
\newcommand{\ballotVar}{{\sf ballot}}
\newcommand{\ballotvar}{\mathit{b}}
\newcommand{\cballot}{{\sf cballot}}
\newcommand{\cballotvar}{\mathit{cballot}}
\newcommand{\lastdeliveredVar}{{\sf lastDelivered}}

\newcommand{\Fence}{Q}
\newcommand{\fencerange}{q}
\newcommand{\fence}{\textsc{cl\_uniform\_barrier}}
\newcommand{\uniformbarrier}{\textsc{uniform\_barrier}}

\newcommand{\Attach}{A}
\newcommand{\attachrange}{a}
\newcommand{\clattach}{\textsc{cl\_attach}}
\newcommand{\attach}{\textsc{attach}}

\newcommand{\maxPrep}{\mathit{maxPrep}}
\newcommand{\maxDec}{\mathit{maxDec}}

\newcommand{\intread}{R_{\textsc{int}}}
\newcommand{\extread}{R_{\textsc{ext}}}
\newcommand{\rocommit}{C_{\textsc{ro}}}
\newcommand{\updatecommit}{C_{\textsc{rw}}}

\newcommand{\tsfunc}{\textsl{ts}}
\newcommand{\tsvar}{\mathit{ts}}
\newcommand{\lc}{{\sf lc}}
\newcommand{\lcvar}{\mathit{lc}}
\newcommand{\lclock}{\textsl{lclock}}
\newcommand{\lcorder}{\textsl{lc}}
\newcommand{\timestamp}{\textsl{ts}}

\newcommand{\eo}{\textsl{eo}}
\newcommand{\eok}{\textsl{eo}_{k}}
\newcommand{\po}{\textsl{po}}
\newcommand{\so}{\textsl{so}}
\newcommand{\vis}{\textsl{vis}}
\newcommand{\ar}{\textsl{ar}}
\newcommand{\rel}[1]{\xrightarrow{#1}} %
\newcommand{\relation}{\mathcal{R}}

\newcommand{\txevents}{s}
\newcommand{\txnevents}{Y}

\newcommand{\rval}{\textsl{rval}}
\newcommand{\uval}{\textsl{uval}}

\newcommand{\retval}{\textsc{RVal}}
\newcommand{\intretval}{\textsc{IntRVal}}
\newcommand{\extretval}{\textsc{ExtRVal}}
\newcommand{\cc}{\textsc{CausalConsistency}}
\newcommand{\cv}{\textsc{CausalVisibility}}
\newcommand{\ca}{\textsc{CausalArbitration}}
\newcommand{\ev}{\textsc{EventualVisibility}}
\newcommand{\por}{\textsc{PoR}}
\newcommand{\conflictaxiom}{\textsc{ConflictOrdering}}

\begin{document}

\title{\System: A fault-tolerant marriage of causal and strong consistency}

\author{
{\rm Manuel Bravo} \qquad
{\rm Alexey Gotsman} \qquad
{\rm Borja de R\'egil} \qquad
\\[2pt]
IMDEA Software Institute
\and
\and
{\rm Hengfeng Wei~\thanks{Also with the State Key Laboratory for Novel Software Technology, Software Institute.}}\\[2pt]
Nanjing University
}

\maketitle

\begin{abstract}
Modern online services rely on data stores that replicate their data across
geographically distributed data centers. %
Providing strong consistency in such data stores results in high latencies and
makes the system vulnerable to network partitions.
The alternative of relaxing consistency violates crucial correctness properties.
A compromise is to allow multiple consistency levels to
coexist in the data store. In this paper we present \System, the first fault-tolerant and
scalable data store that combines causal and strong consistency.
The key challenge we address in \System is to maintain liveness despite data
center failures: this could be compromised if a strong transaction takes a
dependency on a causal transaction that is later lost because of a failure.
\System ensures that such situations do not arise while paying the cost of
durability for causal transactions only when necessary.
We evaluate \System on Amazon EC2 using both microbenchmarks and a
sample application. Our results show that
\System effectively and scalably combines causal and strong
consistency.

\end{abstract}

% \leavevmode
% \\
% \\
% \\
% \\
% \\
\section{Introduction}
\label{introduction}

AutoML is the process by which machine learning models are built automatically for a new dataset. Given a dataset, AutoML systems perform a search over valid data transformations and learners, along with hyper-parameter optimization for each learner~\cite{VolcanoML}. Choosing the transformations and learners over which to search is our focus.
A significant number of systems mine from prior runs of pipelines over a set of datasets to choose transformers and learners that are effective with different types of datasets (e.g. \cite{NEURIPS2018_b59a51a3}, \cite{10.14778/3415478.3415542}, \cite{autosklearn}). Thus, they build a database by actually running different pipelines with a diverse set of datasets to estimate the accuracy of potential pipelines. Hence, they can be used to effectively reduce the search space. A new dataset, based on a set of features (meta-features) is then matched to this database to find the most plausible candidates for both learner selection and hyper-parameter tuning. This process of choosing starting points in the search space is called meta-learning for the cold start problem.  

Other meta-learning approaches include mining existing data science code and their associated datasets to learn from human expertise. The AL~\cite{al} system mined existing Kaggle notebooks using dynamic analysis, i.e., actually running the scripts, and showed that such a system has promise.  However, this meta-learning approach does not scale because it is onerous to execute a large number of pipeline scripts on datasets, preprocessing datasets is never trivial, and older scripts cease to run at all as software evolves. It is not surprising that AL therefore performed dynamic analysis on just nine datasets.

Our system, {\sysname}, provides a scalable meta-learning approach to leverage human expertise, using static analysis to mine pipelines from large repositories of scripts. Static analysis has the advantage of scaling to thousands or millions of scripts \cite{graph4code} easily, but lacks the performance data gathered by dynamic analysis. The {\sysname} meta-learning approach guides the learning process by a scalable dataset similarity search, based on dataset embeddings, to find the most similar datasets and the semantics of ML pipelines applied on them.  Many existing systems, such as Auto-Sklearn \cite{autosklearn} and AL \cite{al}, compute a set of meta-features for each dataset. We developed a deep neural network model to generate embeddings at the granularity of a dataset, e.g., a table or CSV file, to capture similarity at the level of an entire dataset rather than relying on a set of meta-features.
 
Because we use static analysis to capture the semantics of the meta-learning process, we have no mechanism to choose the \textbf{best} pipeline from many seen pipelines, unlike the dynamic execution case where one can rely on runtime to choose the best performing pipeline.  Observing that pipelines are basically workflow graphs, we use graph generator neural models to succinctly capture the statically-observed pipelines for a single dataset. In {\sysname}, we formulate learner selection as a graph generation problem to predict optimized pipelines based on pipelines seen in actual notebooks.

%. This formulation enables {\sysname} for effective pruning of the AutoML search space to predict optimized pipelines based on pipelines seen in actual notebooks.}
%We note that increasingly, state-of-the-art performance in AutoML systems is being generated by more complex pipelines such as Directed Acyclic Graphs (DAGs) \cite{piper} rather than the linear pipelines used in earlier systems.  
 
{\sysname} does learner and transformation selection, and hence is a component of an AutoML systems. To evaluate this component, we integrated it into two existing AutoML systems, FLAML \cite{flaml} and Auto-Sklearn \cite{autosklearn}.  
% We evaluate each system with and without {\sysname}.  
We chose FLAML because it does not yet have any meta-learning component for the cold start problem and instead allows user selection of learners and transformers. The authors of FLAML explicitly pointed to the fact that FLAML might benefit from a meta-learning component and pointed to it as a possibility for future work. For FLAML, if mining historical pipelines provides an advantage, we should improve its performance. We also picked Auto-Sklearn as it does have a learner selection component based on meta-features, as described earlier~\cite{autosklearn2}. For Auto-Sklearn, we should at least match performance if our static mining of pipelines can match their extensive database. For context, we also compared {\sysname} with the recent VolcanoML~\cite{VolcanoML}, which provides an efficient decomposition and execution strategy for the AutoML search space. In contrast, {\sysname} prunes the search space using our meta-learning model to perform hyperparameter optimization only for the most promising candidates. 

The contributions of this paper are the following:
\begin{itemize}
    \item Section ~\ref{sec:mining} defines a scalable meta-learning approach based on representation learning of mined ML pipeline semantics and datasets for over 100 datasets and ~11K Python scripts.  
    \newline
    \item Sections~\ref{sec:kgpipGen} formulates AutoML pipeline generation as a graph generation problem. {\sysname} predicts efficiently an optimized ML pipeline for an unseen dataset based on our meta-learning model.  To the best of our knowledge, {\sysname} is the first approach to formulate  AutoML pipeline generation in such a way.
    \newline
    \item Section~\ref{sec:eval} presents a comprehensive evaluation using a large collection of 121 datasets from major AutoML benchmarks and Kaggle. Our experimental results show that {\sysname} outperforms all existing AutoML systems and achieves state-of-the-art results on the majority of these datasets. {\sysname} significantly improves the performance of both FLAML and Auto-Sklearn in classification and regression tasks. We also outperformed AL in 75 out of 77 datasets and VolcanoML in 75  out of 121 datasets, including 44 datasets used only by VolcanoML~\cite{VolcanoML}.  On average, {\sysname} achieves scores that are statistically better than the means of all other systems. 
\end{itemize}


%This approach does not need to apply cleaning or transformation methods to handle different variances among datasets. Moreover, we do not need to deal with complex analysis, such as dynamic code analysis. Thus, our approach proved to be scalable, as discussed in Sections~\ref{sec:mining}.
\section{System Model}
\label{sec:sysmodel}

We consider a geo-distributed system consisting of a set of data centers
$\DC=\{1, \dots, D\}$ that manage a large set of data items. A data item is
uniquely identified by its \emph{key}. For scalability, the key space is split
into a set of logical partitions $\Partitions = \{1, \ldots, N\}$.  Each data
center stores replicas of all partitions, scattered among its servers.
We let $\partition^m_d$ be the replica of partition $m$ at data center $d$, and
we refer to replicas of the same partition as \emph{sibling} replicas.
As is standard, we assume that $D = 2f+1$ and at most $f$ data centers may fail.
We call a data center that does not fail {\em correct}. If a data center fails,
all partition replicas it stores become unavailable. For simplicity, we do not
consider the failures of individual replicas within a data center: these can be
masked using standard state-machine replication protocols executing within a
data center~\cite{smr,paxos}.

Replicas have physical clocks, which are loosely synchronized by a
protocol such as NTP. The correctness of \System does not depend on the
precision of clock synchronization, but large %
drifts may negatively impact its performance.
Any two replicas are connected by a reliable FIFO channel, so that 
messages between correct data centers are guaranteed to be %
delivered.
As is standard, to implement strong consistency we require the network to be
{\em eventually synchronous}, so that message delays between sibling replicas in correct
data centers are eventually bounded by some constant~\cite{psync}.




% !TEX root = ../DPIM.tex
% !TEX spellcheck = en-US
\renewcommand\hole[1]{\llparenthesis#1\rrparenthesis}
\newcommand{\Count}{\omega_1}
\newcommand{\occ}[1]{\mathrm{occ}_\xi(#1)}
\newcommand{\pesc}[1][\alpha]{\approx_{#1}}
\newcommand{\svirg}[1][\alpha]{\sim_{#1}}
\newcommand{\eq}[1][\alpha]{\equiv_{#1}}
\newcommand{\ured}[1][M]{\redh^{\mach{#1}}}
\newcommand{\uredd}[1][M]{\reddh^{\mach{#1}}}
\newcommand{\XX}{\mathbb{X}}
\newcommand{\X}{\mach{X}}
\newcommand{\BB}{\mathbb{B}}
\newcommand{\cMX}{\cM[\XX]^\xi}
\newcommand{\Lup}{\underline{\#}}
\newcommand{\Luinv}[1]{\underline{\#}^{-1}(#1)}

In this section we adapt Barendregt's proof of consistency of $\blam\omega$ (the least \lam-theory closed under the $(\omega)$-rule) to prove Lemma~\ref{lem:about:equivo}\ref{lem:about:equivo2}, which entails the consistency of our system.
First, we need to introduce in our setting the notion of \emph{context} and \emph{underlined reduction}, that are omnipresent techniques in the area of term rewriting systems.

\subsection{Contexts and Underlined Head Reductions}

In \lam-calculus a context is a \lam-term possibly containing occurrences of an algebraic variable, called \emph{hole}, that can be substituted by any \lam-term possibly with capture of free variables.
We will define a \emph{context-machine} similarly, namely as an \am{} possibly having a ``hole'' denoted by $\xi$. Formally, we introduce a new machine having no registers or program, only an empty tape (therefore distinguished from all machines populating $\cM$):
\[
	\xi = \tuple{[]}
\]
We then extend our formalism to include machines working either directly or indirectly with one, or more, occurrences of $\xi$. We wish to ensure the invariant that a machine $\mM$ with no occurrences of $\xi$ maintain as address $\Lookup\mM$ --- for this reason we need to extend the range of addresses in a conservative way.

Consider a countable set $\BB$ of addresses such that $\Addrs\cap\BB = \emptyset$, and write $\XX = \Addrs\cup\BB$ for the set of \emph{extended addresses}. As usual, we set \[\XX_\Null = \XX\cup\set{\Null}.\]

\begin{defi}\label{def:context-machine}
\begin{enumerate}[(i)]
\item
	An \emph{extended machine $\X$} is either of the form
	\begin{itemize}
	\item $\appT{\xi}{T}$ or
	\item $\tuple{\vec R,P,T}$
	\end{itemize}
	where $\vec R$ are $\XX_\Null$-valued registers, $P$ is a valid program, $T\in\Tapes[\XX]$ is an $\XX$-valued tape. We write $\cMX$ for the set of all extended machines.
\item Fix a bijective map $\Lup : \cMX \to \XX$ satisfying $\Lup(\mM)=\Lookup\mM$ for all \am{} $\mM\in\cM$. Write $\Luinv{\cdot} : \XX\to\cMX$ for its inverse.
\item The \emph{number of occurrences} of $\xi$ in $\X\in\cMX$ (resp.\ $R_i$, resp.\ $T$), written $\occ{\X}\in\nat\cup\set{\infty}$ ($\occ{R_i}$, $\occ{T}\in\nat\cup\set{\infty}$), is defined as follows:
\[
	\begin{array}{lcl}
	\occ{\appT{\xi}{T}} &=& 1 + \occ{T};\\[3pt]
	\occ{\tuple{\vec R,P,T}} &=& \occ{T} + \sum_{i=0}^{r-1}\occ{R_i};\\[3pt]
	\occ{[a_1,\dots,a_n]} &=& \occ{\Luinv{a_1}}+\cdots+\occ{\Luinv{a_n}};\\[3pt]
	\occ{R_i}&=&\begin{cases}
	0,&\textrm{if }R_i = \Null,\\
	\occ{\Luinv{a}},&\textrm{if }R_i = a\in\XX.\\
	\end{cases}
	\end{array}
\]
Notice that $\occ{\mM}\in\nat$ entails that $\occ{\mM.R_i},\occ{\mM.T}\in\nat$.
\end{enumerate}
\end{defi}

\noindent
The number of occurrences of $\xi$ in an extended machine $\X$ has been defined to handle the fact that recursively dereferencing all the addresses contained in an extended \am{} might result in a non-terminating process (see Remark~\ref{rem:forever}).

\begin{exas}\label{ex:weird}
The following are examples of extended machines:
\begin{enumerate}[(i)]
\item $\xi$, with $\occ{\xi} = 1$;
\item $\append{\mK}{\Lup{\xi},\Lup{(\append{\xi}{\Lup\xi})}}$, with $\occ{\append{\mK}{\Lup{\xi},\Lup{(\append{\xi}{\Lup\xi})}}} = 3$;
\item\label{ex:weird3} for all $n\in\nat$, $\X_n = \tuple{\Lup\xi,\varepsilon,[\Lup{\X_{n+1}}]}$. In this case, $\occ{\X_0} = \infty$.
\end{enumerate}
\end{exas}

\noindent
As previously mentioned, a key property of contexts in \lam-calculus is that one can plug a \lam-term into the hole and obtain a regular \lam-term.
Similarly, given $\mM\in\cMX$ and $\X\in\cMX$, we can define the \am{} $\X\hole{\mM}$ obtained from $\X$ by recursively substituting (even in the registers/tapes) each occurrence of $\xi$ by $\mM$. However, this operation is well-defined only when $\occ{\X}$ is finite, so we focus on extended machines enjoying this property.

\begin{defi}\
\begin{enumerate}[(i)]
\item A \emph{context-machine} is any $\C\in\cMX$ satisfying $\occ{\C}\in\nat$.
\item Given a context-machine $\C$ and $\mM\in\cM$, define the \am{} $\C\hole{\mM}$ as follows:
\[
	\C\hole{\mM} =\begin{cases}
	\appT{\mM}{T\hole{\mM}},&\textrm{if }\C=\appT{\xi}{T},\\
	\tuple{\vec R\hole{\mM},P,T\hole{\mM}},&\textrm{if }\C=\tuple{\vec R,P,T};\\
	\end{cases}
\]
where (assuming $a\in\XX,T = [a_1,\dots,a_n]\in\Tapes[\XX]$ with $\occ{\Cons a T}\in\nat$):
\[
	\begin{array}{lcl}
	a\hole{\mM} &=& \Lup(\Luinv{a}\hole{\mM});\\[3pt]
	R_i\hole{\mM}&=& \begin{cases}
	\Null&\textrm{if }R_i = \Null,\\
	a\hole{\mM}&\textrm{if }R_i = a;\\[3pt]
	\end{cases}~\\
	T\hole{\mM} &=& [a_1\hole{\mM},\dots,a_n\hole{\mM}].\\[3pt]
	\end{array}
\]
\end{enumerate}
\end{defi}

In the following, when writing $\C\hole{\mM}$ (resp.\ $a\hole{\mM}$, $R_i\hole{\mM}$, $T\hole{\mM}$) we silently assume that the number of occurrences of $\xi$ in $\C$ (resp.\ $a,R_i,T$) is finite.
Let us introduce a notion of reduction for context-machines that allows to mimic the underlined reduction from~\cite{BarendregtTh}. The idea is to decompose a machine $\mN$ as $\mN = \C\hole{\underline{\mM}}$ where $\C$ is a context-machine and $\mM$ the underlined sub-machine.
It is now possible to reduce $\C$ independently from $\mM$ until either the machine reaches a final-state or $\xi$ reaches the head-position. In the latter case, we substitute the head occurrence of $\xi$ by $\mM$, and continue the computation.

\begin{defi}\label{def:weirdreds}\
\bsub
\item\label{def:weirdreds1}
	The head reduction $\redh$ is generalized to extended machines in the obvious way, using $\Lup(\cdot)$ rather than $\Lookup{(\cdot)}$ to compute the addresses.
In particular, the machine $\appT{\xi}{T}\not\redh$ is in final state, but it is not stuck.
\item\label{def:weirdreds2}
	Given $\mM\in\cM$ and $\C\in\cM^\xi$, the \emph{$\mM$-underlined (head-)reduction} $\ured$ is defined by adding to~\ref{def:weirdreds1} the rule
\[
	\appT{\xi}{T}\ured\appT{\mM}{T}.
\]
\esub
\end{defi}

\begin{exas} Let $\C = \append{\mS}{\Lup \xi,\Lup\xi,\Lup \mach{x}_n}$. Then $\C\hole{\mach{K}} = \append{\mS}{\Lookup \mach{K},\Lookup\mach{K},\Lookup \mach{x}_n}$.
\bsub
\item $\C\reddh \append{\xi}{\Lup\mach{x}_n,\Lup{(\append{\xi}{\Lup\mach{x}_n})}}$.
\item$\C\uredd[K] \append{\xi}{\Lup\mach{x}_n,\Lup{(\append{\xi}{\Lup\mach{x}_n})}}
\ured[K]\append{\mach{K}}{\Lup\mach{x}_n,\Lup{(\append{\xi}{\Lup\mach{x}_n})}}\uredd[K] \mach{x}_n$.
\esub
\end{exas}

\begin{lem}\label{lem:chemmeserve}
	For $\C,\C'\in\cMX$ and $\mM,\mN\in\cM$, the following are equivalent:
	\begin{enumerate}
	\item\label{lem:chemmeserve1} $\C\hole{\mM}\reddh \mN$;
	\item\label{lem:chemmeserve2} $\C\reddh^\mM\C'$ and $\C'\hole{\mM} = \mN$.
	\end{enumerate}
\end{lem}

\begin{proof} (\ref{lem:chemmeserve1} $\Rightarrow$~\ref{lem:chemmeserve2})
By induction on the length $n$ of the reduction $\C\hole{\mM}\reddh\mN$.

Case $n = 0$. Trivial, take $\C'=\C$.

Case $n > 0$. Let $\C\hole{\mM}\redh\mN'\reddh\mN$. Split into cases depending on $\C$.

Subcase $\C = \appT{\xi}{T}$, therefore $\C\hole{\mM} = \appT{\mM}{T\hole{\mM}}\redh\mN'$. There are two possibilities:
\begin{itemize}
\item $\mM$ is stuck and $T\neq[]$, say, $T=[a_0,\dots,a_n]$. In this case $\C\hole{\mM} = \tuple{\vec R,\Load i;P,[]}$ and $\mN' = \tuple{\vec R\repl{R_i}{a_0\hole{\mM}},\Load i;P,[a_1\hole{\mM},\dots,a_n\hole{\mM}]}$.
On the other side, $\C\ured \appT{\mM}{T}\ured \C''$ for
\[
	\C'' = \tuple{\vec R\repl{R_i}{a_0},\Load i;P,[a_1,\dots,a_n]}
\]
satisfying $\C''\hole{\mM} = \mN'\reddh\mN$. We conclude by induction hypothesis.
\item $\mM\redh\mM'$. In this case $\mN' = \appT{\mM'}{T\hole{\mM}}$ and $\C\ured \appT{\mM}{T}\ured \C''$ for $\C'' = \appT{\mM'}{T}$ satisfying $\C''\hole{\mM} = \mN'\reddh\mN$. We conclude by induction hypothesis.

Subcase $\C = \tuple{\vec R,P,T}$. By case analysis on $P$. All cases follow easily from the induction hypothesis.
\end{itemize}

(\ref{lem:chemmeserve2} $\Rightarrow$~\ref{lem:chemmeserve1})
By induction on the length $n$ of the reduction $\C\reddh^\mM\C'$.

Case $n=0$. Trivial, take $\mN=\C\hole{\mM}$.

Case $n>0$, i.e.\ $\C\ured\C''\uredd\C'$, where the latter reduction is shorter.

Proceed by case analysis on the shape of $\C$.

Subcase $\C = \appT{\xi}{T}$ and $\C''=\appT{\mM}{T}$.
Then $\mN = \C''\hole{\mM} = \appT{\mM}{T\hole{\mM}} = \C\hole{\mM}$.
Conclude by induction hypothesis.

Subcase $\C = \tuple{\vec R,P,T}$. By case analysis on $P$. All cases follow easily from the induction hypothesis.
\end{proof}

\subsection{Ordinal analysis}

As mentioned in Remark~\ref{rem:aboutordinals}, a derivation of $\mM\equivea\mN$ has the structure of a well-founded $\omega$-branching tree.
Unfortunately, this makes it difficult to prove even simple properties like Lemma~\ref{lem:about:equivo}\ref{lem:about:equivo2}.
We need a more refined system exposing the underlying ordinal and handling the applications of the (Transitivity) rule separately.

\begin{defi}
\begin{enumerate}[(i)]
\item Let $\Count$ be the set of all countable ordinals.
\item If $\pi$ is a derivation of $\mM\equivea\mN$, we define its \emph{length} $\ell(\pi)\in\omega_1$ in the usual inductive way for the rules \redwerule, (Refl.), (Symm.), (Trans.). Concerning the rule $\extrule$ having countably many premises, we set:
\[
	\ell\left(
	\begin{array}{c}
	\infer{\mM \equivea \mN}{
		\mach{M},\mach{N}\reddh\stuck{}&
		\forall a\in\Addrs\,.\, \infer{\append{\mM}{a} \equivea \append{\mN}{a}}{\pi_a}
	}
	\end{array}
	\right) = \sup_{a\in\Addrs}(\ell(\pi_a)+1)
\]
It is easy to check that, if a derivation $\pi$ has premises $(\pi_i)_{i\in \cI}$ for some countable set $\cI$ then $\ell(\pi) > \ell(\pi_i)$ for every $i\in\cI$.
\item For all $\alpha\in\Count$, define $\eq,\svirg,\pesc\,\subseteq\cM^2$ as the least reflexive and symmetric relations closed under the rules of Figure~\ref{fig:Pesiolino}.
\end{enumerate}
\begin{figure}
\begin{gather*}
% General
\infer[(\approx_0)]{\mM\pesc[0] \mN}{\mM\equiva \mN}
\qquad
\infer[(\subseteq^{\approx}_\alpha)]{\mM\svirg\mN}{\mM\pesc\mN}
\qquad
\infer[(\subseteq^{\sim}_\alpha)]{\mM\eq\mN}{\mM\svirg\mN}\\[3pt]
% Pesciolino
\infer[(\approx_\alpha)]{\mM\pesc\mN}{\mM,\mN\reddh\stuck{}&\forall a\in\Addrs,\,\exists\gamma < \alpha\,.\,\append{\mM}{a} \eq[\gamma] \append{\mN}{a}}\\[3pt]
\begin{array}{ccc}
	% Svirgola
	\infer[(R_\alpha^\sim)]{\mM\repl{R_i}{a}\svirg\mM\repl{R_i}{b}}{\Lookinv a\svirg \Lookinv b}
	&\quad&
	\infer[(@_\alpha^\sim)]{\append{\mM}{a}\svirg\append{\mM}{b}}{\Lookinv a\svirg \Lookinv b}\\[3pt]
	%%%%
	\infer[(T_\alpha^\sim)]{\appT{\mM}{T}\svirg\appT{\mN}{T}}{\mM\svirg \mN&T\in\Tapes}
	&&
	\infer[(T_\alpha)]{\appT{\mM}{T}\eq\appT{\mN}{T}}{\mM\eq \mN&T\in\Tapes}\\[3pt]
	%%%%
	\infer[(R_\alpha)]{\mM\repl{R_i}{a}\eq\mM\repl{R_i}{b}}{\Lookinv a\eq \Lookinv b}
	&&
	\infer[(@_\alpha)]{\append{\mM}{a}\eq\append{\mM}{b}}{\Lookinv a\eq \Lookinv b}\\[3pt]
\end{array}~\\
%%%%
\infer[(\le^\approx_\alpha)]{\mM \pesc \mN}{\mM\pesc[\gamma]\mN&\gamma \le \alpha}
\quad
\infer[(\le^\sim_\alpha)]{\mM \svirg \mN}{\mM\svirg[\gamma]\mN&\gamma \le \alpha}
\quad
\infer[(\le_\alpha)]{\mM \eq \mN}{\mM\eq[\gamma]\mN&\gamma \le \alpha}
%%%%
\\
\infer[(\mathrm{Tr}_\alpha)]{\mM \eq \mN}{\mM\eq\mZ&\mZ\eq\mN}\\[-5ex]
\end{gather*}
\caption{Rules satisfied by $\pesc$, $\svirg$ and $\eq$, beyond reflexivity and symmetry.}\label{fig:Pesiolino}
\end{figure}
\end{defi}
The intuitive meanings of the relations $\eq,\svirg,\pesc$ are the following:
\begin{itemize}
\item $\mM\eq\mN\iff\mM\equivea\mN$ is derivable using the rule $\extrule$ at most $\alpha$ times;
\item $\mM\svirg\mN\iff\mM\eq\mN$ is derivable without using transitivity;
\item $\mM\pesc\mN\iff\mM\equivea\mN$ in case $\alpha = 0$. Otherwise, if $\alpha>0$ then
\item $\mM\pesc\mN\iff\mM\svirg\mN$ follows directly from the rule $\extrule$.
\end{itemize}

\noindent
More precisely, the rules $(\approx_0)$, $(\subseteq^{\approx}_\alpha)$, $(\subseteq^{\sim}_\alpha)$ express the fact that $\equiva\,\subseteq\,\pesc\,\subseteq\,\svirg\,\subseteq\,\eq$.
The rule $(\approx_\alpha)$ allows to prove $\mM \pesc \mN$, provided that both machines eventually get stuck and that $\appT{\mM}{[a]} \eq[\gamma_a] \appT{\mN}{[a]}$ is provable for every address $a$, using a smaller ordinal $\gamma_a < \alpha$.
The rules $(R_\alpha)$, $(@_\alpha)$ and $(T_\alpha)$ (resp.\ $(R_\alpha^\sim)$, $(@_\alpha^\sim)$ and $(T_\alpha^\sim)$) represent the contextuality of the relation $\eq$ (resp.\ $\svirg$).
The rules $(\le^\approx_\alpha)$, $(\le^\sim_\alpha)$ and $(\le_\alpha)$ specify that incrementing the ordinal (from top to bottom) is always allowed.
Finally, $(\mathrm{Tr}_\alpha)$ gives the transitivity of $\eq$.

The following lemma describes formally the intuitive meaning discussed above.
\begin{lem}\label{lem:relalphaprops}
Let $\mM,\mN\in\cM$
\begin{enumerate}[(i)]
\item\label{lem:relalphaprops1}\
	 $\mM\equivea\mN\iff\exists\alpha\in\Count\,.\,\mM\eq\mN$.
\item\label{lem:relalphaprops2}\
	$\mM\eq[0]\mN\iff\mM\equiv_\Addrs\mN$.
\item\label{lem:relalphaprops3}\
	$\mM\eq \mN\iff\exists n\ge0, \mZ_1,\dots,\mZ_n\in\cM\,.\, \mM\svirg\mZ_1\svirg\cdots\svirg\mZ_n =\mN$.
\item\label{lem:relalphaprops4}~\\[-3ex]
$
	\begin{array}{ll}
		\mM\svirg\mN\iff&\exists \mach{C}\in\cMX,\mach{M}',\mach{N}'\in\cM\,.\,\\
		&\mM=\mach{C}\hole{\mM'}\land\mN = \mach{C}\hole{\mN'} \land \mM'\pesc\mN'.\\
		\end{array}
	$
\item\label{lem:relalphaprops5}~\\[-2.7ex]
$
	\begin{array}{lcl}
		\mM\pesc\mN\land \alpha\neq 0&\iff&\mM,\mN\reddh\stuck{}\ \land\\
		&&\forall a\in\Addrs,\exists\gamma<\alpha\,.\, \append{\mM}{a}\eq[\gamma]\append{\mN}{a}.\\
		\end{array}
	$
\end{enumerate}
\end{lem}

\begin{proof}\ref{lem:relalphaprops1} $(\Leftarrow)$ Easy.

$(\Rightarrow)$ By induction on the length of a derivation of $\mM\equivea\mN$.

Case \redwerule. I.e., there exists $\mZ\in\cM$ such that $\mM\reddh\mZ\eqea\mN$.
By Theorem~\ref{thm:CR}, we have $\mM\equiva\mZ$ whence $\mM\eq[0]\mZ$ by $(\pesc[0])$, which implies $\mM\eq\mZ$ for all $\alpha\in\Count$ using the rule $(\le_\alpha)$. Now, consider the set
\[
	\cR = \set{ i \st \mZ.R_i \neq\Null} = \set{ i \st \mN.R_i \neq\Null}
\]
Note that $\cR= \set{i_1,\dots,i_k}$ for some $k<\mZ.r_0 (=\mN.r_0)$. For every $i\in\cR$, let $\mZ.R_i = a_i$ and $\mN.R_i = a'_i$. Also, let $\mZ.T = [b_1,\dots,b_m]$ and $\mN.T = [b'_1,\dots,b'_m]$. By assumption, $a_i\simea a'_i$ and $b_j\simea b'_j$ for every $i\in\cR$, and $j\,(1\le j\le m)$.
By induction hypothesis, $\Lookinv{a_i} \eq[\gamma_i] \Lookinv{a'_i}$ and $\Lookinv{b_j} \eq[\delta_j] \Lookinv{b'_j}$. Using the rule $(<_\alpha)$, the same holds for $\eq[\alpha]$ setting $\alpha = \sup_{i\in\cR,1\le j\le m} \set{\gamma_i,\delta_j}$. Putting everything together, we obtain:
\[
	\begin{array}{lcll}
	\mM&\eq&\mZ = \tuple{\mZ.\vec R,P,[b_1,\dots,b_m]}\\
	&\eq&\tuple{\mZ.\vec R\repl{R_{i_1}}{a'_{i_1}},P,[b_1,\dots,b_m]},&\textrm{by $(R_\alpha)$,}\\
	&\eq&\cdots&\qquad\vdots\\
	&\eq&\tuple{\mZ.\vec R[R_i:= a'_i]_{i\in\cR},P,[b_1,\dots,b_m]},&\textrm{by $(R_\alpha)$,}\\
	&=&\tuple{\mN.\vec R,P,[b_1,\dots,b_m]},&\textrm{by definition,}\\
	&\eq&\tuple{\mN.\vec R,P,[b'_1,b_2,\dots,b_m]},&\textrm{by $(T_\alpha)$,}\\
	&\eq&\cdots&\qquad\vdots\\
	&\eq&\tuple{\mN.\vec R,P,[b'_1,\dots,b'_m]},&\textrm{by $(T_\alpha)$,}\\
	&=&\mN,&\textrm{by definition.}\\
	\end{array}
\]
We conclude by applying the transitivity rule $(\mathrm{Tr}_\alpha)$ that $\mM \eq \mN$.

Case \extrule. By induction hypothesis, for every $a\in\Addrs$, there exists $\gamma_a\in\Count$ such that $\append{\mM}{a} \eq[\gamma_a]\append{\mN}{a}$.
For $\gamma = \sup_{a\in\Addrs}\gamma_a$, we get $\append{\mM}{a} \eq[\gamma]\append{\mN}{a}$ by $(\le_\alpha)$. By $(\pesc)$ we get $\mM\pesc\mN$ for $\alpha=\gamma+1\in\Count$, conclude by $(\subseteq^\approx_\alpha)$ and $(\subseteq^\sim_\alpha)$.

(Reflexivity), (Symmetry) and (Transitivity) follow from the respective property of $\eq$.

Concerning items~\ref{lem:relalphaprops2}--\ref{lem:relalphaprops5} the implication $(\Leftarrow)$ is trivial. We analyze $(\Rightarrow)$.

\ref{lem:relalphaprops2} % chktex 2
	By induction on a derivation of $\mM\eq[0]\mN$, using Theorem~\ref{thm:CR}.

\ref{lem:relalphaprops3} % chktex 2
    By induction on a derivation of $\mM\eq\mN$.

    Case $(\subseteq^\sim_\alpha)$. Trivial.

    Case $(R_\alpha)$. I.e., $\mM = \mZ\repl{R_i}{a}$, $\mN = \mZ\repl{R_i}{b}$ and $\Lookinv{a} \eq \Lookinv{b}$. By induction hypothesis, there exist $c_1,\dots,c_k\in\Addrs$ such that
    \[
    	\Lookinv{a}\svirg\Lookinv{c_1}\svirg\cdots\svirg\Lookinv{c_k}=\Lookinv{b}.
    \]
    The case follows by applying the rule $(R^\sim_\alpha)$.

    Case $(@_\alpha)$. Analogous, by applying $(@^\sim_\alpha)$.

    Case $(T_\alpha)$. Analogous, by applying $(T^\sim_\alpha)$.

    Case $(\mathrm{Tr}_\alpha)$. Straightforward from the IH\@.

    Case $(\le_\alpha)$. By IH and $(\le^\sim_\alpha)$.

	Cases (Reflexivity), (Symmetry). Straightforward from the IH\@.

\ref{lem:relalphaprops4} % chktex 2
	By induction on a derivation of $\mM\svirg\mN$.

	Case $(\subseteq^\approx_\alpha)$. Take $\C = \xi$.

	Case $(R^\sim_\alpha)$. I.e., $\mM = \mZ\repl{R_i}{a}$, $\mN = \mZ\repl{R_i}{b}$ and $\Lookinv{a} \svirg \Lookinv{b}$. By induction hypothesis, there exist $\C'\in\cMX$ having address $c = \Lup{\C'}\in\XX$, $\mM',\mN'\in\cM$ such that $\C'\hole{\mM'} = \Lookinv{a}$, $\C'\hole{\mN'} = \Lookinv{b}$ and $\mM' \pesc\mN'$. We conclude by taking $\C = \mZ\repl{R_i}{c}$.

	Case $(@^\sim_\alpha)$. Analogous.

	Case $(T^\sim_\alpha)$. Take $\C = \appT{\C'}{T}$, where $\C'$ is obtained from the IH\@.

	Case $(\le^\sim_\alpha)$. It follows from the IH, by applying $(\le^\sim_\alpha)$ and $(\le^\approx_\alpha)$.

	Cases (Reflexivity), (Symmetry). Straightforward from the IH\@.

\ref{lem:relalphaprops5} Immediate. % chktex 2
\end{proof}

Consider now a scenario where $\C\hole \mM\reddh \C'\hole{\mM}$.
Assuming $\mM\pesc\mN$, one might expect that also $\C\hole \mN\reddh \C'\hole{\mN}$ holds.
In general, this is not the case because $\mM$ and $\mN$ might reach the head position and get control of the computation.
Using the underlined (head-)reduction from Definition~\ref{def:weirdreds}\ref{def:weirdreds2} we can substitute $\mN$ for $\mM$ along the reduction (when it comes in head position) and construct a proof of $\C\hole \mN \eq[\gamma] \C'\hole\mN$ having a lower ordinal $\gamma < \alpha$.

\begin{lem}\label{lem:black_magic}
Let $\alpha > 0$, $\C\in\cMX$, $\mM,\mN\in\cM$ such that $\mM\pesc\mN$.
If $\C\redh^\mM\C'$ and $\C'\hole{\mM}\not\reddh\stuck{}$, then there exists $\gamma < \alpha$ such that $\C\hole \mN \eq[\gamma] \C'\hole\mN$.
\end{lem}

\begin{proof} By cases on the shape of $\C$.

Case $\C = \appT{\xi}{T}$ for some $T\in\Tapes[\XX]$ and $\C' = \appT{\mM}{T}$.
From  $\mM\pesc\mN$ and Lemma~\ref{lem:relalphaprops}\ref{lem:relalphaprops5}, we get that $\mM\reddh{\stuck{\mM'}}$ for some $\mM'\in\cM$.
Since $\C'\hole{\mM} = \appT{\mM}{(T\hole{\mM})}$ cannot reduce to a stuck \am, we must have $T\hole{\mM}\neq[]$.
In other words, $T = [a_0,\dots,a_n]$ for some $n\ge 0$.
Notice that, for all $a_i\in\Tapes[\XX]$, we have $a_i\hole{\mN}\in\Addrs$ (by construction).
By Lemma~\ref{lem:relalphaprops}\ref{lem:relalphaprops5}, there exists $\gamma<\alpha$ such that $\append{\mN}{a_0\hole{\mN}} \eq[\gamma] \append{\mM}{a_0\hole{\mN}}$. By definition:
\[
	\C\hole{\mN} = \appT{\mN}{T\hole{\mN}},\textrm{ and }
	\C'\hole{\mN}=\appT{\mM}{T\hole{\mN}}.
\]
So we construct the proof:
\[
	\infer[(T_\gamma)]{\append{\mN}{a_0\hole{\mN},\dots,a_n\hole{\mN}} \eq[\gamma] \append{\mM}{a_0\hole{\mN},\dots,a_n\hole{\mN}}}{
	\append{\mN}{a_0\hole{\mN}}\eq[\gamma] \append{\mM}{a_0\hole{\mN}}
	}
\]

In all the other cases, $\C\hole{\mN}\to_h\C'\hole{\mN}$, therefore $\C\hole{\mN} \eq[0]\C\hole{\mN}$.
\end{proof}

\begin{cor}\label{cor:black_magic}
Let $n\in\nat$, $\alpha > 0$, $\C\in\cMX$, $\mM,\mN\in\cM$.
If $\C\hole{\mM}\reddh\mach{x}_n$ and $\mM\pesc\mN$ then there exists $\gamma < \alpha$ such that $\C\hole{\mN} \eq[\gamma] \mach{x}_n$.
\end{cor}

\begin{proof} Assume $\C\hole{\mM}\reddh\mach{x}_n$. Equivalently, by Lemma~\ref{lem:chemmeserve}, we have $\C\reddh^\mM\mach{x}_n$.
By definition, there exists $\C_1,\dots,\C_k\in\cMX$ such that
\[
	\C = \C_1\to_h^\mM\cdots\to_h^\mM \C_k = \mach{x}_n
\]
Notice that $\C_i\hole{\mM}\reddh \mach{x}_n$ and, since $\lnot\stuck{\mach{x}_n}$, we have $\C_i\hole{\mN}\not\reddh\stuck{}$.
By Lemma~\ref{lem:black_magic}, there exists $\gamma_1,\dots,\gamma_k<\alpha$ such that $\C_i\hole{\mN}\eq[\gamma_i]\C_{i+1}\hole{\mN}$.
By transitivity $(\mathrm{Tr}_\alpha)$ and $(\le_\alpha)$ we obtain $\mM\eq\mach{x}_n$ for $\alpha = \sup_i{\gamma_i}$.
\end{proof}

\begin{prop} Let $\mM,\mN\in\cM$, $\alpha\in\Count$ and $n\in\nat$.
If $\mM\eq \mN$ and $\mN\reddh\mach{x}_n$ then $\mM\reddh\mach{x}_n$.
\end{prop}

\begin{proof} We proceed by induction on $\alpha$. Since we perform a double induction, the induction hypothesis with respect to this induction is called the $\alpha$-IH ($\alpha$-inductive hypothesis).

Case $\alpha = 0$. By Lemma~\ref{lem:relalphaprops}\ref{lem:relalphaprops2}, we get $\mM\equivea \mN\reddh\mach{x}_n$, so we conclude $\mM\reddh\mach{x}_n$ by confluence (Theorem~\ref{thm:CR}) and $\red[i]$-postponement (Lemma~\ref{lem:standardization}).

Case $\alpha > 0$. By Lemma~\ref{lem:relalphaprops}\ref{lem:relalphaprops3}, there exist $\mZ_1,\dots,\mZ_k$ such that
\begin{equation}\label{eq:svirg}
	\mM \svirg\mZ_1\svirg\cdots\svirg\mZ_k=\mN\reddh\mach{x}_n
\end{equation}
By induction on $k$, we prove that~\eqref{eq:svirg} implies $\mM\reddh\mach{x}_n$.
We call this $k$-IH\@.

Subcase $k =0$. Then $\mM =\mN\reddh\mach{x}_n$ and we are done.

Subcase $k >0$. From the $k$-IH we derive $\mZ_1\reddh\mach{x}_n$.
From $\mM\svirg\mZ_1$ and Lemma~\ref{lem:relalphaprops}\ref{lem:relalphaprops4}, there is a context-machine $\C$ such that $\mM = \C[\mM']$ and $\mZ_1 = \C[\mN']$ with $\mM'\pesc\mN'$ and $\C[\mN']\reddh\mach{x}_n$.
By applying Lemma~\ref{lem:black_magic} we obtain $\C[\mM'] \eq[\gamma] \mach{x}_n$ for some $\gamma<\alpha$.
We conclude by applying the $\alpha$-IH\@.
\end{proof}

From this proposition, Lemma~\ref{lem:about:equivo}\ref{lem:about:equivo2} follows by applying Lemma~\ref{lem:relalphaprops}\ref{lem:relalphaprops1}.

\section{Designing a Watermark}
\label{sec:taxonomy}

After the quick overview of watermarking schemes in \cref{sec:background}, we now provide more details 
about the watermarking design space. We created a unifying taxonomy under which all previous schemes 
can be expressed. We first discuss the requirements then the building blocks of a text watermark. 
%
%We provide a modular implementation of all schemes, so any of the building blocks can be combined.
%
\cref{fig:design-figure} summarizes the current design space.

\subsection{Requirements}

A useful watermarking scheme must detect watermarked texts, without falsely flagging human-generated text and without impairing the original model's performance.
%
More precisely, we want watermarks to have the following properties.
% \begin{itemize}[leftmargin=\itemlm,itemsep=2pt]
\begin{enumerate}[leftmargin=\itemlm,itemsep=2pt]
    \item \textbf{High Recall}. $\Pr[\mathcal{V}_k(T) = \texttt{True}]$ is large if $T$ is a watermarked text generated using the marking procedure $\mathcal{W}$ and secret key $k$.
    %
    \item \textbf{High Precision}. For a random key $k$, $\Pr[\mathcal{V}_k(\Tilde{T}) = \texttt{False}]$ is large if $\Tilde{T}$ is a human-generated (\emph{non-watermarked}) text.
    %
    \item \textbf{Quality}. The watermarked model should perform similarly to the original model. 
    It should be useful for the same tasks and generate similar quality text.
    %
    \item \textbf{Robustness}. A good watermark should be robust to small changes to the watermarked text (potentially caused by an adversary), 
    meaning if a sample $T$ is watermarked with key $k$, then for any text $\Tilde{T}$ that is semantically close to $T$, $\mathcal{V}_k(\Tilde{T})$ should evaluate to \text{True}.
\end{enumerate}

\noindent
A desireable (but optional) property for watermarks is diversity. 
In some settings, such as creative tasks like story-telling, users might want the model to have the ability to generate 
multiple different outputs in response to the same prompt (so they can select their favorite).
We would like watermarked outputs to preserve this capability.
% \noindent
% In addition to these properties, another desirable property for a watermark is to 
% preserve a model's diversity. Language models tend to have diverse generated text distributions: 
% they are able to generate different responses to a same prompt. This is useful in many settings, 
% such as creative tasks like story telling, so the user can  their favorite output.

% The notion of \emph{undetectability} has been defined in previous work~\citep{christ_undetectable_2023}:
Another useful property is \emph{undetectability}, also called \emph{indistinguishability}:
%
no feasible adversary should be able to distinguish watermarked text from non-watermarked text, without knowledge of the secret key~\citep{christ_undetectable_2023}. 
%
A watermark is considered undetectable if the maximum advantage at distinguishing is very small.
%
This notion is appealing; for instance, undetectability implies that watermarking does not degrade the model's quality.
%
However, we find in practice that undetectability is not necessary and may be overly restrictive:
%
minor changes to the model's output distribution are not always detrimental to its quality.

In this paper we focus on symmetric-key watermarking, where both the watermarking and verification procedures share a secret key.
%
This is most suitable for proprietary language models that served via an API.
%
We imagine that the vendor would watermark all outputs, and also provide a second API to query the verification procedure.
%
Alternatively, one could publish the key, enabling anyone to run the verification procedure.
%
\begin{figure*}
    \begin{center}
    \begin{tikzpicture}
    
    \draw[draw=black] (0,15) rectangle ++(17.5,1) node[pos=0.5, align=center] {\Large{Watermarking Taxonomy}};
    \draw[draw=black] (0,12.75) rectangle ++(8.375,2) node[pos=0.5, align=left] 
    {\\
    \\
    \textbf{Parameters:} Key $k$, Sampling $\mathcal{C}$, Randomness $\mathcal{R}$\\
    \textbf{Inputs:} Probs $\mathcal{D}_n = \{\lambda^n_1,\, \cdots, \lambda^n_d\}$, Tokens $\{T_i\}_{i < n}$\\
    \textbf{Output:} Next token 
    $T_n \leftarrow \mathcal{C}(\mathcal{R}_k( \{T_i\}_{i < n}), \mathcal{D}_n)$};
    \draw[draw=black] (9.125,12.75) rectangle ++(8.375,2) node[pos=0.5, align=left] 
    {\\
    \\
    \textbf{Parameters:} Key $k$, Score $\mathcal{S}$, Threshold $p$\\
    \textbf{Inputs:} Text $T$\\
    \textbf{Output:} Decision $\mathcal{V} \leftarrow \text{P}_{0}\left( \mathcal{S} < \mathcal{S}_k(T)\right) < p$};
    \draw (8.75,13.75) circle (0.25) node {+};
    \draw[draw=none] (0,14.25) rectangle ++(8.375,.5) node[pos=0.5, align=left] {\large{Marking $\mathcal{W}$}};
    \draw[draw=none] (9.125,14.25) rectangle ++(8.375,.5) node[pos=0.5, align=left] {\large{Verification $\mathcal{V}$}};
    
    %%%
    
    \draw[draw=black,dashed] (0,8.75) rectangle ++(17.5,3.75);
    \draw[draw=none] (0,11) rectangle ++(17.5,1.75) node[pos=0.5, align=center] {\large{Randomness Source $\mathcal{R}$}\\
    \textbf{Inputs:} Tokens $\{T_i\}_{i < n}$\,
    \textbf{Output:} Random value $r_n = \mathcal{R}_k(\{T_i\}_{i < n})$};
    \draw[draw=black] (0.25,9) rectangle ++(11.25,2.35) node[pos=0, anchor=south west] {\textbf{Text-dependent.} Hash function $h$. Context length H};
    \draw[draw=black] (0.5,9.6) rectangle ++(10.75,0.625) node[anchor=north west] at (0.5, 10.225) {\textbf{(R2) Min Hash}} node[pos=1, anchor=north east, align=left] {
    $r_n = \text{min} \left( h\left( T_{n-1} \mathbin\Vert k\right), \, \cdots, h\left( T_{n-H} \mathbin\Vert k\right) \right)$\\
    };
    \draw[draw=black] (0.5,10.475) rectangle ++(10.75,0.625) node[anchor=north west] at (0.5, 11.1) {\textbf{(R1) Sliding Window}} node[pos=1, anchor=north east, align=left] {
    $r_n = h\left( T_{n-1} \mathbin\Vert \, \cdots \mathbin\Vert T_{n-H} \mathbin\Vert k\right)$\\
    };
    \draw[draw=black] (11.75,9) rectangle ++(5.5,2.35) node[pos=0, anchor=south west] {\textbf{(R3) Fixed}} node[pos=0.5, align=left] {Key length L. Expand $k$ to\\ pseudo-random sequence $\{r^k_i\}_{i<L}$.\\ 
    $r_n = r^k_{n \text{ (mod L)}}$ \\ \\ };
    
    %%%
    
    \draw[draw=black,dashed] (0,3.25) rectangle ++(17.5,5.25);
    \draw[draw=none] (0,6.85) rectangle ++(17.5,1.75) node[pos=0.5, align=center] {\large{Sampling algorithm $\mathcal{C}$ \& Per-token statistic $s$}\\
    \textbf{Inputs:} Random value $r_n = \mathcal{R}_k( \{T_i\}_{i < n})$, Probabilities $\mathcal{D}_n = \{\lambda^n_1,\, \cdots, \lambda^n_d\}$, Logits $\mathcal{L}_n = \{l^n_1,\,\cdots,l^n_d\}$\\};
    
    %
    
    \draw[draw=black] (11,4.75) rectangle ++(6.25,2.5) node[pos=0, anchor=south west] {\textbf{(C3) Binary}} node[pos=0.5, align=left] {Binary alphabet.\\ 
    $T_n \leftarrow 0$ if $r_n < \lambda^n_0$, else $1$. \\
    $s(T_n, r) = \begin{cases} -\log(r) \text{ if } T_n = 1\\
          -\log(1-r) \text{ if } T_n = 0\\\end{cases} $};
    
    \draw[draw=black] (5,4.75) rectangle ++(5.75,2.5) node[pos=0, anchor=south west] {\textbf{(C2) Inverse Transform}} node[pos=0.5, align=left] 
    {$\pi$ keyed permutation. $\eta$ scaling func.\\
    $T_n \leftarrow \pi_k \left( \min\limits_{ j \leq d } \sum\limits_{i=1}^j \lambda^n_{\pi_k (i)} \geq r_n \right)$ \\
    $s(T_n, r) = | r - \eta \left( \pi^{-1}_k(T_n) \right) | $\\};
    
    \draw[draw=black] (0.25,4.75) rectangle ++(4.5,2.5) node[pos=0, anchor=south west] {\textbf{(C1) Exponential}} node[pos=0.5, align=left] 
    {$h$ keyed hash function. \\
    $T_n \leftarrow \argmax\limits_{i \leq d} \left\{ \frac{\log \left( h_{r_n}\left( i \right) \right)}{\lambda^n_i} \right\}$ \\
    $s(T_n, r) = -\log(1 \! - \! h_r(T_n))$\\};
    
    % 
    
    \draw[draw=black] (0.25,3.5) rectangle ++(17,1) node[pos=0, anchor=south west] {\textbf{(C4) Distribution-shift}} node[pos=0.5, align=right] {Bias $\delta$, Greenlist size $\gamma$. Keyed permutation $\pi$. $T_n$ sampled from $\widetilde{\mathcal{L}}_n = \{l^n_i + \delta \text{ if } \pi_{r_n}(i) < \gamma d \text{ else } l^n_i\, , 1 \leq i \leq d\}$\\
    $s(T_n, r) = 1 \text{ if } \pi_{r}(T_n) < \gamma d \text{ else } 0$};
    
    %%% 
    
    \draw[draw=black,dashed] (0,0) rectangle ++(17.5,3);
    \draw[draw=none] (0,1.75) rectangle ++(17.5,1.25) node[pos=0.5, align=center] {\large{Score $\mathcal{S}$}\\
    \textbf{Inputs:} Per-token statistics $s_{i,j} = s(T_i, r_j)$, where $r_j = \mathcal{R}_k( \{T_l\}_{l < j}))$. \# Tokens $N$.};
    
    % 
    
    \draw[draw=black] (8.15,0.25) rectangle ++(9.1,1.5) node[pos=0, anchor=south west] {\textbf{(S3) Edit Score}}
    
    node[pos=0.5, align=left] {
    $\mathcal{S}_{\text{edit}}^\psi = s^\psi(N,N)$,
    $
        s^\psi (i,j) = \min \begin{cases}
          s^\psi (i-1, j-1) + s_{i,j}\\
          s^\psi (i-1, j) + \psi\\
          s^\psi (i, j-1) + \psi\\
        \end{cases} 
    $};
    \draw[draw=black] (0.25,0.25) rectangle ++(2.6,1.5) node[pos=0, anchor=south west] {\textbf{(S1) Sum Score}} node[pos=0.5, align=left] {$\mathcal{S}_{\text{sum}}\! = \! \sum_{i=1}^N s_{i,i}$ \\};
    \draw[draw=black] (3.1,0.25) rectangle ++(4.8,1.5) node[pos=0, anchor=south west] {\textbf{(S2) Align Score}} node[pos=0.5, align=left] {$\mathcal{S}_{\text{align}} \!= \!\min\limits_{0 \leq j < N} \sum\limits_{i=1}^N s_{i, (i+j) \text{ mod}(N)}$ \\ \\ };
    
    \end{tikzpicture}
    \caption{Watermarking design blocks. There are three main components: randomness source, sampling algorithm (and associated per-token statistics), and score function. Each solid box within each of these three components (dashed) denotes a design choice. The choice for each component is independent and offers different trade-offs.}\label{fig:design-figure}
    \end{center}
    \end{figure*}

\subsection{Watermark Design Space}
\label{sec:watermark-design}

Designing a good watermark is a balancing act.
% 
For instance, replacing every word of the output with [WATERMARK] would achieve high recall but destroy the utility of the model.
%
%Conversely, sampling from the original distribution preserves quality but makes it impossible to watermark. 

Existing proposals have cleverly crafted marking procedures that are meant to preserve quality, provide high precision and recall, and achieve a degree of robustness.
%
Despite their apparent differences, we realized they can all be expressed within a unified framework:

\begin{itemize}[leftmargin=\itemlm,itemsep=2pt]
    \item The marking procedure $\mathcal{W}$ contains a randomness source $\mathcal{R}$ and a sampling algorithm $\mathcal{C}$.
    %
    The randomness source $\mathcal{R}$ produces a (pseudo-random) value $r_n$ for each new token, based on the secret key $k$ and the previous tokens $T_0,\cdots,T_{n-1}$.
    %
    The sampling algorithm $\mathcal{C}$ uses $r_n$ and the model's next token distribution $\mathcal{D}$ to  a token.
    \item The verification procedure $\mathcal{V}$ is a one-tailed significance test that computes a $p$-value for the null hypothesis that the text is not watermarked.
    %
    The procedure compares this $p$-value to a threshold, which enables control over the watermark's precision and recall.
    %
    % This test is done using a \emph{score function} $\mathcal{S}$ based on a per-token variable that depends on the ed sampling algorithm.
    % We call the value of this per-token test statistic $s_n$, which only depends on the random value $r_n$ and the ed token $T_n$: $s_n = s(T_n, r_n)$.
    In particular, we compute a per-token score $s_{n,m} \coloneqq s(T_n, r_m)$ for each token $T_n$ and randomness $r_m$, aggregate them to obtain an overall score $\mathcal{S}$, and compute a $p$-value from this score.
    We consider all overlaps $s_{n,m}$ instead of only $s_{n,n}$ to support scores that consider misaligned randomness and text after perturbation. 
    %the test computes \emph{score function} $\mathcal{S}$ which takes as input per-token test statistics $s_{n,m} \coloneqq s(T_n, r_m)$ for a token $T_n$ and a random value $r_m$, $\forall n,m \in [N]$.
    %
    %$s_{n,m}$ depends on the sampling algorithm (see \cref{fig:design-figure} for examples).
    %
    % \dave{I believe $s(T_n, r_m)$ is incorrect and it should be $s(T_n, r_n)$.  Also I think the score should be $s_n$ rather than $s_{n,m}$.}
    % \jp{Depending on the alignment between the key string and the text, there are times we want to refer to the score for key at position m and token at poistion n (for instance, for both the align and edit scores). I'll add some explanation for this.}
    
\end{itemize}
% \dave{I find the sheer number of fonts inelegant (blackboard bold, mathcal, mathbf, typewritter, italics, bold, etc.). In some places, algorithms are denoted by mathcal (W,V), in other places by mathbf (R,C,S).  I suggest picking one and being consistent.  I prefer mathcal.  Lots of bold feels distracting to my eyes, as does lots of font changes.}
% \jp{I changed a bunch of fonts to make it more consistent, and removed bold fonts}

Next, we show how each scheme we consider falls within this framework, each with its own choices for $\mathcal{R},\mathcal{C},\mathcal{S}$.
%Given this template, previous work introduced their own variants of the building blocks, which we will now detail. 
% \chawin{I would have liked to see a summary of which design choices belong to which paper. Maybe we can add a shorthand notation denoting each paper in \cref{fig:design-figure} or have a separate table.}
% \jp{I agree that's a good idea. A table is probably the right way to represent this.}

\subsubsection{Randomness source $\mathcal{R}$}\label{ssec:randomness}
% \textbf{Randomness source $\mathcal{R}$.}
%
% \chawin{Maybe others?} \jp{Yeah but all the other papers i've seen seem to attribute it to one of these two.}
We distinguish two main ways of generating the random values $r_n$, \emph{text-dependent} (computed as a deterministic function of the prior tokens) vs \emph{fixed} (computed as a function of the token index).
Both approaches use the standard heuristic of applying a keyed function (typically, a PRF) to some data, to produce pseudorandom values that can be treated as effectively random but can also be reproduced by the verification procedure.

\citet{aaronson_watermarking_2022} and \citet{kirchenbauer_watermark_2023}
use text-dependent randomness: $r_n = f\left(T_0,\,\cdots,T_{n-1},k\right)$.
%
This scheme has two parameters: the length of the token context window (which we call the window size H) and the aggregation function $f$.
%
\citet{aaronson_watermarking_2022} proposed using the hash of the concatenation of previous tokens, $f := h\left( T_{n-1} \mathbin\Vert \, \cdots \mathbin\Vert T_{n-H} \mathbin\Vert k\right)$; we call this (R1) sliding window.
%
\citet{kirchenbauer_watermark_2023} used this with a window size of $ H = 1$ and also introduced an alternate aggregation function $f := \text{min} \left( h\left( T_{n-1} \mathbin\Vert k\right), \, \cdots, h\left( T_{n-H} \mathbin\Vert k\right) \right)$.
%
We call this last aggregation function (R2) min hash.
%
While these two schemes propose specific choices of $H$, other values are possible. 
We use \benchmarkname{} to evaluate a range of values of $H$ with each candidate aggregation function.

% \smallskip\noindent\textbf{(R3) Fixed}
\citet{kuditipudi_robust_2023} use fixed randomness:
$r_n = f_k(n)$, where $n$ is the index (position) of the token.
We call this (R3) fixed.
%
In practice, they propose using a fixed string of length $L$ (the key length), which is repeated across the generation.
% r_n = f_k(n \bmod L)$ where $L$ is the key length.
% \dave{I don't think we need this level of detail.  I suggest deleting the preceding sentence.}
% \jp{Since we look at the impact of the key length on generations we still need to introduce the idea that the key is repeated, but I canwrite that in english for it to be more digestable}
%
We test the choice of key length in ~\cref{ssec:param_tuning}
%
In the extreme case where $L=1$ or $H=0$, both sources are identical, as $r_n$ will be the same value for every token. \citet{zhao2023provable} explored this option using the same sampling algorithm as~\citet{kirchenbauer_watermark_2023}.

\label{ssec:binary}
\citet{christ_undetectable_2023} proposed setting a target entropy for the context window instead of fixing a window size.
%
This allows to set a lower bound on the security parameter for the model's undetectability.
%
However, setting a fixed entropy makes for a less efficient detector since all context window lengths must be tried in order to detect a watermark.
%
Furthermore, in practice, provable undetectability is not needed to achieve optimal quality: we chose to keep using a fixed-size window for increased efficiency.

\subsubsection{Sampling algorithm \(\mathcal{C}\)}\label{ssec:sampling}
% \textbf{sampling algorithm $\mathcal{C}$.}
%
\noindent
We now give more details about the four sampling algorithms initially presented in~\cref{tab:marking-algorithms}.

\smallskip\noindent\textbf{(C1) Exponential}.
%
Introduced by \citet{aaronson_watermarking_2022} and also used by \citet{kuditipudi_robust_2023}. It relies on the Gumbel-max trick.
%
Let $\mathcal{D}_n = \left\{\lambda^n_i\,, 1 \leq i \leq d\right\}$ be the distribution of the language model over the next token. %(obtained after passing the logits through a softmax and applying a temperature adjustment).
%
The exponential scheme will select the next token as:
\begin{align}
    T_{n} = \argmax\limits_{i \leq d}\left\{ \frac{\log \left( h_{r_n}\left( i \right) \right)}{\lambda^n_i} \right\}
\end{align}
where $h$ is a keyed hash function using $r_n$ as its key.
%
The per-token variable used in the statistical test is either $s_n = h_{r_n}(T_n)$ or $s_n = -\log \left( 1-h_{r_n}(T_n)\right)$.
%
\citet{aaronson_watermarking_2022} and \citet{kuditipudi_robust_2023} both use the latter quantity.
%
We argue the first variable provides the same results, and unlike the log-based variable, the distribution of watermarked variables can be expressed analytically (see~\cref{app:ssec:pseudorandom-proofs} for more details).
%
We align with previous work and use the $\log$ for \benchmarkname{}.

\smallskip\noindent\textbf{(C2) Inverse transform}.
%
\citet{kuditipudi_robust_2023} introduce inverse transform sampling.
%
They derive a random permutation using the secret key $\pi_k$. The next token is selected as follows:
\begin{align}
    T_{n} = \pi_k \left( \min\limits_{ j \leq d } \sum\limits_{i=1}^j \lambda^n_{\pi_k (i)} \geq r_n \right)
\end{align}
which is the smallest index in the inverse permutation such that the CDF of the next token distribution is at least $r_n$.
%
\citet{kuditipudi_robust_2023} propose to use $s_n = | r_n - \eta \left( \pi^{-1}_k(T_n) \right) |$ as a the test variable, where $\eta$ normalizes the token index to the $[0,1]$ range.
%
% We call this scheme the \textit{inverse transform} scheme.

\smallskip\noindent\textbf{(C3) Binary}.
%
\citet{christ_undetectable_2023} propose a different sampling scheme for binary token alphabets --- however, it can be applied to any model by using a bit encoding of the tokens.
%
In our implementation, we rely on a Huffman encoding of the token set, using frequencies derived from a large corpus of natural text.
%
In this case, the distribution over the next token reduces to a single probability $p_n$ that token ``0'' is ed next, and $1-p$ that ``1'' is ed.
%
The sampling rule s 0 if $r_n < p$, and 1 otherwise. The test variable for this case is $s_n = -\log \left( T_n r_n + (1-T_n) (1-r_n) \right)$.
%
% We call this scheme the \textit{binary} scheme.
%
% At first glance, it can seem like this scheme is identical to the exponential scheme. However, because it uses a binary alphabet, the distribution of the test variable is different for both schemes.
%
% However, we show in Appendix \jp{ref} that this is not the case: the distribution of the test variable is different for both schemes.
% %
% \jp{Maybe I'll remove this if I don't have time to show it.}

\smallskip\noindent\textbf{(C4) Distribution-shift}.
%
\citet{kirchenbauer_watermark_2023} propose the distribution-shift scheme. 
%
It produces a modified distribution $D_n$ from which the next token is sampled.
%
Let $\delta > 0$ and $\gamma \in [0,1]$ be two system parameters, and $d$ be the number of tokens.
%
The scheme constructs a permutation $\pi_{r_n}$, seeded by the random value $r_n$, which is used to define a ``green list,'' containing tokens $T$ such that $\pi_{r_n} (T) < \delta d$. It then adds $\delta$ to green-list logits.
%
This modified distribution is then used by the model to sample the next token. The test variable $s_n$ is a bit equal to ``1'' if $T_n$ is in the green list defined by $\pi_{r_n}$, and ``0'' if not.
%
% We call this scheme the \textit{distribution-shift} scheme.

The advantage of this last scheme over the others is that it preserves the model's diversity: 
for a given key, the model will still generate diverse outputs.
In contrast, for a given secret key and a given prompt, the first three sampling strategies 
will always produce the same result, since the randomness value $r_n$ will be the same.
\citet{kuditipudi_robust_2023} tackles this by randomly offseting the key sequence of 
fixed randomness for each generation. We introcude a skip probability $p$ for the 
same effect on text-dependent randomness. Each token is selected without the marking 
strategy with probability $p$. In the interest of space, we leave a detailed discussion 
of generation diversity in~\cref{app:ssec:diverse}.

Another advantage of the distribution-shift scheme is that it can also be used 
at any temperature, by applying the temperature scaling \emph{after} using the 
scheme to modify the logits. Other models apply temperature before watermarking.

However the distribution-shift scheme is not indistinguishable from the original model, 
as discussed earlier in~\cref{ssec:watermark-design}.

\subsubsection{Score Function $\mathcal{S}$}\label{ssec:score}

% \paragraph{Verification procedure $\mathcal{V}.$}

% The distribution of the per-token test statistic is different for watermarked text and non-watermarked text: this is what makes detection possible. Depending on the scheme, it is either higher or lower on average in the watermarked case. Without loss of generality, we assume it is always lower for this discussion.

To determine whether an $N$-token text is watermarked, we compute a score over per-token statistics.
%
This score is then subject to a one-tailed statistical test where the null hypothesis is that the text is not watermarked.
%
In other words, if its $p$-value is under a fixed threshold, the text is watermarked.
%
Different works propose different scores.

\smallskip\noindent\textbf{(S1) Sum score}.
%
\citet{aaronson_watermarking_2022} and \citet{kirchenbauer_watermark_2023} take the sum of all individual per-token statistics:
\begin{align}
    \mathcal{S}_{\text{sum}}=\sum_{i=1}^N s_i = \sum_{i=1}^N s(T_i, r_i).
\end{align}
%
This score requires the random values $r_i$ and the tokens $T_i$ to be aligned.
%
% \chawin{Maybe this goes into limitation or discussion or appendix}
This is not a problem when using text-dependent randomness, since the random values are directly obtained from the tokens.
%
However, this score is not suited for fixed randomness: removing one token at the start of the text will offset the values of $r_i$ for the rest of the text and remove the watermark.
%
The use of the randomness shift to increase diversity will have the same effect. 

\smallskip\noindent\textbf{(S2) Alignment score}.
Proposed by \citet{kuditipudi_robust_2023}, the alignment score aims to mitigate the misalignment issue mentioned earlier.
% \citet{kuditipudi_robust_2023} proposes two alternative scores to deal with this issue.
%
% In keeping with their work, we name these scores the alignment score and the edit score.
Given the sequence of random values $r_i$ and the sequence of tokens $T_i$, the verification process now computes different versions of the per-token test statistic for each possible overlap of both sequences $s_{i,j} = s(T_i, r_j)$.
%
The alignment score is defined as:
\begin{align}
   \mathcal{S}_{\text{align}}  = \min\limits_{0 \leq j < N} \sum\limits_{i=1}^N s_{i, (i+j) \text{ mod}(N)}
\end{align}

\smallskip\noindent\textbf{(S3) Edit score}.
Similar to the alignment score, \citet{kuditipudi_robust_2023} propose the edit score as an alternative for dealing with the misalignment issue.
%
It comes with an additional parameter $\psi$ and is defined as $\mathcal{S}_{\text{edit}}^\psi = s^\psi(N,N)$, where
\begin{align}
    s^\psi (i,j) &= \min \begin{cases}
      s^\psi (i-1, j-1) + s_{i,j}\\
      s^\psi (i-1, j) + \psi\\
      s^\psi (i, j-1) + \psi\\
    \end{cases} 
\end{align}

In all three cases, the average value of the score for watermarked text will be lower than for non-watermarked text.
%
% In the case of the sum score, we can often derive the distribution of the score under the null hypothesis, allowing us to use a $z$-test to determine if the text is watermarked.
In the case of the sum score, the previous works use the $z$-test on the score to determine whether the text is watermarked, but it is also possible, or even better in certain situations, to use a different statistical test according to \citet{fernandez_three_2023}.
%
When possible, we derive the exact distribution of the scores under the null hypothesis (see \cref{app:ssec:exact_dist}) which is more precise than the $z$-test. When it is not, we rely on an empirical T-test, as proposed by \citet{kuditipudi_robust_2023}
%
% This allows one to compute 

\subsection{Limitations of the Building Blocks}\label{ssec:limit_blocks}

While we design the blocks to be as independent as possible, some combinations of the scheme and specific parameters are obviously sub-optimal.
%
Here, we list a few of these subpar block combinations as a guide for practitioners.
% Even though any of the three scores can be used with any scheme and randomness source, in practice not all combinations are useful.
\begin{itemize}[leftmargin=\itemlm,itemsep=2pt]
    \item The sum score (S1) is not robust for fixed randomness (R3).
    \item The alignment score (S2) does not make sense for the text-dependent randomness (R1, R2) since misalignment is not an issue.
    \item The edit score (S3) has a robustness benefit since it can support local misalignment caused by token insertion, deletion, or swapping. However, using it with text-dependent randomness (R1, R2) only makes sense for a window size of 1: for longer window sizes, swapping, adding, or removing tokens would actually change the random values themselves, and not just misalign them.
    \item Finally, in the corner case when a window size of 0 for the text-dependent randomness (R1, R2) or when a random sequence length of 1 for the fixed randomness (R3), both the alignment score (S2) and the edit score (S3) are unnecessary since all random values are the same and misalignment is not possible.
\end{itemize}

In our experiments (\cref{sec:experiments}), we test all reasonable configurations of the randomness source, 
the sampling protocol, and the verification score, along with their parameters. 
We list the evaluated combinations in~\cref{tab:design_space_combinations}. 
The edit score is too inefficient 
to be run on all configurations, instead we rely on the sum and align scores.
%
We hope to not only fairly compare the prior works but also investigate previously unexplored combinations in the 
design space that can produce a better result.

% \chawin{We need a table or a tree that lists all the combinations we test.}\

\begin{table}[h!]
    \centering
    \caption{Tested combinations in the design space, using notations from~\cref{fig:design-figure}.\\
    We only tested the edit score {\bf S3} on a subset of watermarks.\\
    The distribution of non-watermarked scores is known for \textcolor{orange}{orange} configurations and 
    unknown for \textcolor{blue}{blue} configuration. We rely on empirical T-tests~\cite{kuditipudi_robust_2023} for blue configurations.
    }
    \label{tab:design_space_combinations}
    \normalsize
    \begin{tabular}{|l||c|c|c|c|} 
    \hline
     & \makecell[tc]{{\bf C4}\\{\small Distribution}\\{\small Shift}} & \makecell[tc]{{\bf C1}\\{\small Exponential}} & \makecell[tc]{{\bf C2}\\{\small Binary}} & \makecell[tc]{{\bf C3}\\{\small Inverse}\\{\small Transform}} \\
    \hline
    \hline
    \makecell{{\bf S1}+{\bf R1}}  & \textcolor{orange}{X} & \textcolor{orange}{X} & \textcolor{orange}{X} & \textcolor{blue}{X} \\
    \hline
    \makecell{{\bf S1}+{\bf R2}}  & \textcolor{orange}{X} & \textcolor{orange}{X} & \textcolor{orange}{X} & \textcolor{blue}{X} \\
    \hline
    \makecell{{\bf S2}+{\bf R3}}  & \textcolor{blue}{X} & \textcolor{blue}{X} & \textcolor{blue}{X} & \textcolor{blue}{X} \\
    \hline
    \makecell{{\bf S3}+{\bf R3}}  & \textcolor{blue}{X} &  &  &  \\
    \hline
    \end{tabular}
\end{table}
    

\subsection{Analysis of the edit score.} 
\label{ssec:editscore}
We analyzed the tamper-resistance of the edit score on a subset of watermarks 
(distribution-shift with $\delta=2.5$ at a temperature of 1, for key lengths between 1 and 1024). 
We tried various $\psi$ values between 0 and 1 for the edit distance, and compared the tamper-resistance 
and watermark size of the resulting verification procedures to the align score. 
Using an edit distance does improve tamper-resistance for key lengths under 32, but at a large efficiency cost: 
for key lengths above 8, the edit score size is at least twice that of the align score. 
We do not recommend using an edit score on low entropy models such as Llama-2 chat.



\section{Fault-Tolerant Causal Consistency Protocol}

We first describe the \System protocol for the case when all transactions are
causal.  We give its pseudocode in Algorithms~\ref{alg:txncoord1} and
\ref{alg:clock}; for now the reader should ignore highlighted lines, which are
needed for strong transactions. For simplicity, we assume that each handler in
the algorithms executes atomically (although our implementation is
parallelized). We reference pseudocode lines using the format
algorithm\#:line\#.


\subsection{Metadata}
\label{sec:metadata}

Most metadata in our protocol are represented by vectors with an entry per each
data center, where each entry stores a scalar timestamp. However, different
pieces of metadata use the vectors in different ways, which we now describe.


\paragraph{Tracking causality.}
The first use of the vectors is as vector
clocks~\cite{vectorclocks1,vectorclocks2}, to track causality between
transactions. Given vectors $V_1$ and $V_2$, we write $V_1 < V_2$ if each entry
of $V_1$ is no greater than the corresponding entry of $V_2$, and at least one
is strictly smaller. Each update transaction is tagged with a \emph{commit
  vector} $\commitvector$. The order on these vectors is consistent with the
causal order $\prec$ from \S\ref{sec:consistency}: if $\commitvector_1$ and
$\commitvector_2$ are the commit vectors of two update transactions $t_1$ and
$t_2$ such that $t_1 \prec t_2$, then $\commitvector_1 < \commitvector_2$. For a
transaction originating at a data center $d$ with a commit vector $\commitvector$,
we call $\commitvector[d]$ its {\em local timestamp}.

Each replica $\partition^m_d$ maintains a
log $\store[k]$ of update operations performed on each data item $k$ stored at
the replica. Each log entry stores, together with the operation, the commit
vector of the transaction that performed it. This allows reconstructing
different versions of a data item from its log.

\paragraph{Representing causally consistent snapshots.}
The protocol also uses a vector to represent a snapshot of the data store on
which a transaction operates: a snapshot vector $V$ represents all transactions
with a commit vector $\le V$. This snapshot is causally consistent. Indeed,
consider a transaction $t_1$ included into it, i.e., $\commitvector_1 \leq
V$. Since any causal dependency $t_0$ of $t_1$ is such that
$\commitvector_0 < \commitvector_1$, we have $\commitvector_0 < V$, so that
$t_0$ is also included into the snapshot. A client also maintains a vector
$\past$
that represents its \emph{causal past}: a causally consistent snapshot including
the update transactions the client has previously observed.


\paragraph{Tracking what is replicated where.}  
Each replica $\partition^m_d$ maintains three vectors that are used to compute
which transactions are uniform. These respectively track the sets of
transactions replicated at $\partition^m_d$, the local data center $d$, and
$f+1$ data centers. Each of these vectors $V$ represents the set of update
transactions originating at a data center $i$ with a local timestamp $\le
V[i]$. Note that this set may not form a causally consistent snapshot. The first
vector maintained by $\partition^m_d$ is $\replicavectorclock$. For each data
center $i$, it defines the prefix of update transactions originating at $i$ (in
the order of local timestamps) that $\partition^m_d$ knows about.
\begin{property}
\label{prop:knownvc}
For each data center $i$, the replica $\partition^m_d$ stores the updates to
partition $m$ by all transactions originating at $i$ with local timestamps
$\leq \replicavectorclock[i]$.
\end{property}
Our protocol ensures that $\replicavectorclock[d]\leq \clock$ at any replica in
data center $d$. The vector $\replicavectorclock$ at $\partition^m_d$ records
whether the updates to partition $m$ by a given transaction are stored at this
replica. In contrast, the next vector $\stablesnapshot$ records whether the
updates to {\em all} partitions by a transaction are stored at the local data
center $d$.
\begin{property}
\label{prop:stablevc}
For each data center $i$, the data center $d$ stores the updates by all
transactions originating at $i$ with local timestamps $\leq \stablesnapshot[i]$.
More precisely, we are guaranteed that $\replicavectorclock[i]$ at any replica of $d$ ${}\ge{}$
$\stablesnapshot[i]$ at any $\partition^m_d$.
\end{property}
Finally, the last vector $\uniformsnapshot$ defines the set of update
transactions that $\partition^m_d$ knows to have been replicated at
$f+1$ data centers, including $d$.
\begin{property}
\label{prop:uniformity}
Consider $\uniformsnapshot[i]$ at $\partition^m_d$. All update
transactions originating at $i$ with local timestamps $\leq$
$\uniformsnapshot[i]$ are replicated at $f+1$ datacenters including
$d$. More precisely: $\replicavectorclock[i]$
at any replica of these data centers $\geq \uniformsnapshot[i]$ at $\partition^m_d$.
\end{property}


When $\uniformsnapshot$ is reinterpreted as a causally consistent snapshot, it
defines transactions that $\partition^m_d$ knows to be uniform according to
Definition~\ref{def:uniform}:
\begin{property}
\label{prop:uniformsnapshot}
Consider $\uniformsnapshot$ at $\partition^m_d$. All update transactions with
commit vectors $\leq \uniformsnapshot$ are uniform.
\end{property}
\noindent {\em Proof sketch.}
Consider a transaction $t_1$ that originates at a data center $i$ with a commit
vector $\commitvector_1\leq \uniformsnapshot$ at $\partition^m_d$. In
particular, $\commitvector_1[i]\leq\uniformsnapshot[i]$, and by
Property~\ref{prop:uniformity}, $t_1$ is replicated at $f+1$ data centers. We
assume at most $f$ failures. Then the transaction forwarding mechanism
of our protocol 
(\S\ref{sec:overview}) guarantees that $t_1$
will eventually be replicated at all correct data centers.
Consider now any
causal dependency $t_2$ of $t_1$ with a commit vector $\commitvector_2$. Since
commit vectors are consistent with causality,
$\commitvector_2 <\commitvector_1\leq\uniformsnapshot$. Then as above, we can
again establish that $t_2$ will be replicated at all correct data centers, as
required by Definition~\ref{def:uniform}.\qed

\subsection{Causal Transaction Execution}
\label{sec:bluetxs}

\paragraph{Starting a transaction.}
A client can submit a transaction to any replica in its local data center by
calling $\STARTTX(\avc)$, where $\avc$ is the client's causal past $\past$
(\algline{alg:txncoord1}{line:starttx:call}, for brevity, we omit the pseudocode
of the client). A replica $\partition^m_d$ receiving such a request acts as the
transaction {\em coordinator}. It generates a unique transaction identifier
$\tx$, computes a snapshot $\vecsnapshottime[\tx]$ on which the transaction will
execute, and returns $\tx$ to the client (we explain
\alglines{alg:txncoord1}{line:starttx:start}{line:starttx:end} and similar ones
later). The snapshot is computed by combining uniform transactions from
$\uniformsnapshot$ (\algline{alg:txncoord1}{alg:coord:starttx:init}) with the
transactions from the client's causal past originating at $d$
(\algline{alg:txncoord1}{alg:coord:starttx:end}). The former is crucial to
minimize the latency of strong transactions (\S\ref{sec:overview}), while the
latter ensures \emph{read your writes}.




\paragraph{Transaction execution.}  The client proceeds to execute the
transaction $\tx$ by issuing a sequence of operations at its coordinator via
\UPDATETX{} (\algline{alg:txncoord1}{alg:coord:execop}). When the coordinator
receives an operation $\op$ on a data item $k$, it sends a $\EXECOP$ message
with the transaction's snapshot $\vecsnapshottime[\tx]$ to the local replica
responsible for $k$
(\algline{alg:txncoord1}{alg:coord:sendop}). Upon receiving the message
(\algline{alg:txnpartition}{line:updateuniformop:receive}), the replica first
ensures that it is as up-to-date as required by the snapshot
(\algline{alg:txnpartition}{line:waitexecute}). It then computes the latest
version of $k$ within the snapshot by applying the operations from $\store[k]$
by all transactions
with commit vectors $\leq \vecsnapshottime[\tx]$. The result is sent to the
coordinator in a $\OPRET$ message. After receiving it
(\algline{alg:txncoord1}{line:coord:receive-ret}), the coordinator further
applies the operations on $k$ previously executed by the transaction, which are
stored in a buffer $\writeset[\tx]$; this ensures {\em read your writes} within
the transaction. If the operation is an update, the coordinator then appends
it to $\writeset[\tx]$. Finally, the coordinator executes the desired operation
$\op$ and forwards its return value to the client.




\begin{algorithm}
  \renewcommand{\SpaceHandler}{\vspace{5pt}}
  \begin{algorithmic}[1]
    \small
    \Function{\STARTTX}{\avc}\label{line:starttx:call}
      \For{$i \in \DC \setminus \{d\}$}\label{line:starttx:start}
        \State $\uniformsnapshot[i] \gets
          \max\{\avc[i]$, $\uniformsnapshot[i]\}$\label{line:starttx:end}
      \EndFor
       \State \textbf{var} $\tx \gets \newtxid$()
      \State $\vecsnapshottime[\tx] \gets \uniformsnapshot$\label{alg:coord:starttx:init}
      \State $\vecsnapshottime[\tx][d] \gets
        \max\{\avc[d]$, $\uniformsnapshot[d]\}$ \label{alg:coord:starttx:end}
      \State \colorbox{\BackColor}{{\color{\StrongColor}
        $\vecsnapshottime[\tx][\red] \,{\gets}\,
        \max\{\avc[\red]$, $\stablesnapshot[\red]\}$}}\label{alg:coord:starttx:end-red}
      \State \Return \tx
    \EndFunction

    \SpaceHandler

    \Function{\UPDATETX}{\tx, $k$, \op}\label{alg:coord:execop}
      \State {\bf var} $\areplica \gets \partitionof$($k$)
      \State {\bf send}
        \EXECOP($\vecsnapshottime[\tx]$, $k$)
        \textbf{to} $\partition^\areplica_d$\label{alg:coord:sendop}
      \State \textbf{wait receive} \OPRET(\cstate)
      \textbf{from} $\partition^\areplica_d$\label{line:coord:receive-ret}
      \ForAll{$\langle k, \op' \rangle \in \writeset[\tx][\areplica]$}
        $\cstate \gets \apply(\op', \cstate)$\!\!
      \EndFor \label{line:formversion:end}
      \State \colorbox{\BackColor}{{\color{\StrongColor} $\readset[\tx] \gets \readset[\tx] \cup
        \{ \langle k, \op \rangle\}$}}\label{line:readset}
      \If{$\op$ is an update}
        \State $\writeset[\tx][\areplica] \gets
          \writeset[\tx][\areplica] \cdot \langle k, \op \rangle$
      \EndIf
      \State \Return $\effval{\op}{\cstate}$
    \EndFunction

   \SpaceHandler

       \WhenRcv[\EXECOP(\avecsnapshottime, $k$)]
    \textbf{from} \partition \label{line:updateuniformop:receive}
      \For{$i \in \DC \setminus \{d\}$}\label{line:updateuniformop:start}
        \State $\uniformsnapshot[i] \gets \max\{\avecsnapshottime[i]$,
          $\uniformsnapshot[i]\}$\label{line:updateuniformop:end}
      \EndFor
      \State {\bf wait until}
        $\replicavectorclock[d] \geq \avecsnapshottime[d] \wedge {}$\label{line:waitexecute}
          \Statex \hspace{1.69cm}\colorbox{\BackColor}{{\color{\StrongColor} $\replicavectorclock[\red]
        \geq \avecsnapshottime[\red]$}}
      \State \textbf{var} $\cstate \gets \bot$
      \ForAll{\label{line:formversion:start}
        $\langle \op', \commitvector\rangle {\in} \store[k].\hspace{1pt}
        \commitvector {\leq} \avecsnapshottime$}
        \State $\cstate \gets \apply(\op',\cstate)$
      \EndFor
      \State
        \textbf{send}
        \OPRET(\cstate)
        \textbf{to} \partition
    \EndWhenRcv

   \SpaceHandler
    
    \Function{\COMMITBLUE}{\tx} \label{alg:line:coordcommit}
       \State \textbf{var} $L \gets \{ \areplica \mid \writeset[\tx][\areplica] \neq
      \emptyset\}$
      \If{$L = \emptyset$}
        \Return $\vecsnapshottime[\tx]$\label{alg:line:commitro}
      \EndIf
      \State {\bf send}
      \BLUEPREPARE(\tx, $\writeset[\tx][\areplica]$, $\vecsnapshottime[\tx]$)
      \textbf{to} $\partition^\areplica_d,\, \areplica \in L$\label{alg:line:sendprepare}
      \State {\bf var} $\commitvector \gets \vecsnapshottime[\tx]$\label{alg:line:commitvectorany}
      \ForAll{$\areplica \in L$} \label{alg:line:rcvprepare}
        \State \textbf{wait receive} \BLUEPREPARED(\tx, \ts) \textbf{from} $\partition^\areplica_d$
        \State $\commitvector[d] \gets \max \{\commitvector[d]$, $\ts\}$\label{alg:line:commitvectorlocal}
      \EndFor
      \State {\bf send}
      \COMMIT(\tx, $\commitvector$)
      \textbf{to} $\partition^\areplica_d,\, \areplica \in L$\label{alg:line:sendcommit}
      \State \Return $\commitvector$\label{alg:line:returncommit}
    \EndFunction

    \SpaceHandler

        \WhenRcv[\BLUEPREPARE(\tx, \aws, $\avecsnapshottime$)] \textbf{from} \partition
    \label{line:updateuniformprepare:receive}
      \For{$i \in \DC \setminus \{d\}$}\label{line:updateuniformprepare:start}
        \State $\uniformsnapshot[i] \gets \max\{\avecsnapshottime[i]$,
          $\uniformsnapshot[i]\}$\label{line:updateuniformprepare:end}
      \EndFor
      \State \textbf{var} $\ts \gets \clock$ \label{alg:line:PrepTime}
      \State $\preparedblue \gets \preparedblue \cup
        \{ \langle \tx, \aws, \ts\rangle \}$
      \State \textbf{send} \BLUEPREPARED(\tx, \ts) \textbf{to} \partition
    \EndWhenRcv

    \SpaceHandler

    \WhenRcv[\COMMIT($\tx$, $\commitvector$)]
    \label{alg:line:commit:receive}
      \State \textbf{wait until} $\clock \geq \commitvector[d]$ \label{alg:line:commitwait}
      \State $ \langle \tx, \aws, \_\rangle \gets$
        $\mathtt{find}(\tx, \preparedblue)$
      \State $\preparedblue \gets \preparedblue \setminus \{ \langle \tx,
        \_, \_\rangle \}$
      \ForAll{$\langle k, \op \rangle \in \aws$}
        \State $\store[k] \gets \store[k] \cdot
          \langle \op, \commitvector \rangle$
      \EndFor
      \State $\committedset[d] \gets \committedset[d] \cup {}$
        \Statex\hspace{4.5cm}$\{\langle \tx, \aws,
        \commitvector\rangle\}$
    \EndWhenRcv
              

    \SpaceHandler

    \Function{\MAKEUNIFORM}{\avc}\label{line:barrier}
      \State \textbf{wait until} $\uniformsnapshot[d] \ge \avc[d]$ \label{line:waituniform}
    \EndFunction

    \SpaceHandler

    \Function{\ATTACH}{\avc}\label{line:attach}
    \State \textbf{wait until} $\forall i \in \DC \setminus \{d\}.\, \uniformsnapshot[i] \ge \avc[i]$\label{line:waitattach}
    \EndFunction
  \end{algorithmic}
  \caption{Transaction execution at $\partition^m_d$.}
  \label{alg:txncoord1}
  \label{alg:txnpartition}
\end{algorithm}

\paragraph{Commit.}
A client commits a causal transaction by calling $\COMMITBLUE$
(\algline{alg:txncoord1}{alg:line:coordcommit}). This returns immediately if the
transaction is read-only, since it already read a consistent snapshot
(\algline{alg:txncoord1}{alg:line:commitro}).  To commit an update transaction,
\System uses a variant of two-phase commit protocol (recall that for simplicity
we only consider whole-data center failures, not those of individual replicas,
\S\ref{sec:sysmodel}).
The coordinator first sends a $\BLUEPREPARE$ message to the replicas in the
local data center storing the data items updated by the transaction
(\algline{alg:txncoord1}{alg:line:sendprepare}). The message to each replica
contains the part of the write buffer relevant to that replica. When a replica
receives the message
(\algline{alg:txncoord1}{line:updateuniformprepare:receive}), it computes the
transaction's \emph{prepare time} $\ts$ from its local clock and adds the
transaction to $\preparedblue$, which stores the set of causal transactions that
are prepared to commit at the replica. The replica then returns $\ts$ to the
coordinator in a $\BLUEPREPARED$ message.

When the coordinator receives replies from all replicas updated by the
transaction, it computes the transaction's commit vector $\commitvector$: it
sets the local timestamp $\commitvector[d]$ to the maximum among the prepare
times proposed by the replicas
(\algline{alg:txncoord1}{alg:line:commitvectorlocal}), and it copies the other
entries of $\commitvector$ from the snapshot vector $\vecsnapshottime[\tx]$
(\algline{alg:txncoord1}{alg:line:commitvectorany}). The latter reflects the
fact that the transactions in the snapshot become causal dependencies of $\tx$.


After computing $\commitvector$, the coordinator sends it in a $\COMMIT$ message
to the relevant replicas at the local data center
(\algline{alg:txncoord1}{alg:line:sendcommit}) and returns it to the client
(\algline{alg:txncoord1}{alg:line:returncommit}). The client then sets its
causal past $\past$ to the commit vector. When a replica receives the $\COMMIT$
message (\algline{alg:txnpartition}{alg:line:commit:receive}), it removes the
transaction from $\preparedblue$, adds the transaction's updates to $\store$,
and adds the transaction to a set $\committedset[d]$, which stores transactions
waiting to be replicated to sibling replicas at other data centers.










              


\subsection{Transaction Replication}
\label{sec:replication}

Each replica $\partition^m_d$ periodically replicates locally committed update
transactions to sibling replicas in other data centers by executing
$\PROPAGATELOCAL$ (\algline{alg:replication}{line:replicatelocal}). Transactions
are replicated in the order of their local timestamps. The prefix of
transactions that is ready to be replicated is determined by
$\replicavectorclock[d]$: according to Property~\ref{prop:knownvc},
$\partition^m_d$ stores updates to $m$ by all transactions originating at $d$
with local timestamps $\leq \replicavectorclock[d]$. Thus, the replica first
updates $\replicavectorclock[d]$ while preserving Property~\ref{prop:knownvc}.

There are two cases of this update. If the replica does not have any prepared
transactions ($\preparedblue=\emptyset$), it sets $\replicavectorclock[d]$ to
the current value of the $\clock$
(\algline{alg:replication}{line:updateknownvc-start}). This preserves
Property~\ref{prop:knownvc} because in this case a new transaction originating
at $d$ and updating $m$ will get a prepare time at $m$ higher than the current
$\clock$ (\algline{alg:txnpartition}{alg:line:PrepTime}), and thus also a higher
local timestamp (\algline{alg:txncoord1}{alg:line:commitvectorlocal}). If the
replica has some prepared transactions, then they may end up getting local
timestamps lower than the current $\clock$. In this case, the replica sets
$\replicavectorclock[d]$ to just below the smallest prepared time
(\algline{alg:replication}{line:localknownprepared}). This preserves
Property~\ref{prop:knownvc} because: {\em (i)} currently prepared transactions
will get a local timestamp no lower than their prepare time; and {\em (ii)} as
we argued above, new transactions will get a prepare time higher than the
current $\clock$ and, hence, than the smallest prepare time.


After updating $\replicavectorclock[d]$, the replica sends a $\REPLICATE$
message to its siblings with the transactions in $\committedset[d]$ such that
$\commitvector[d]\leq\replicavectorclock[d]$, and then removes them from
$\committedset[d]$. In other words, the replica sends all transactions from the
prefix determined by $\replicavectorclock[d]$ that it has not yet replicated.

\begin{algorithm}[t]
  \renewcommand{\SpaceHandler}{\vspace{6pt}}
  \begin{algorithmic}[1]
    \small
    \Function{\PROPAGATELOCAL}{$ $} \Comment{Run
      periodically}\label{line:replicatelocal}

        \If{$\preparedblue = \emptyset$}
        $\replicavectorclock[d] \gets \clock$\label{line:updateknownvc-start}
          \Else {}
          $\replicavectorclock[d] \,{\gets}\, \min\{\ts \,{\mid}\,
          \langle \_,\_, \ts\rangle\,{\in}\,\preparedblue\}{-}1$\label{line:localknownprepared}
          \EndIf
          \State {\bf var} $\atxs \gets \{\langle \_, \_, \commitvector\rangle
          \in \committedset[d] \mid $
          \Statex \hspace{3.7cm} $\commitvector[d]\leq\replicavectorclock[d]\}$
          \If{$\atxs \neq \emptyset$}
              \State {\bf send}
                $\REPLICATE(d, \atxs)$
                \textbf{to} $\partition^m_i,\, i \in \DC \setminus \{d\}$
                \State $\committedset[d] \gets \committedset[d]\setminus \atxs$
          \Else{}
          {\bf send}
              $\BHEARTBEAT(d, \replicavectorclock[d])$
              \textbf{to} $\partition^m_i,\, i \in \DC \setminus \{d\}$\!\!\!\!\label{line:updateknownprepared}
          \EndIf
    \EndFunction

    \SpaceHandler

  \WhenRcv[\REPLICATE($i$, $\atxs$)]\label{line:receivereplicate}
      \ForAll{$\langle \tx, \aws, \commitvector\rangle \,{\in}\, \atxs$ in
        $\commitvector[i]$ order}
      \If{$\commitvector[i]>\replicavectorclock[i]$}\label{line:receivepreparedprecondition}
      \ForAll{$\langle k, \op \rangle \in \aws$}
        \State $\store[k] \gets \store[k] \cdot
          \langle \op, \commitvector \rangle$
      \EndFor
      \State $\committedset[i] \gets \committedset[i] \cup {}$
        \Statex\hspace{4.6cm}$
        \{\langle \tx, \aws, \commitvector\rangle\}$
       \State $\replicavectorclock[i] \gets \commitvector[i]$
       \EndIf
       \EndFor
    \EndWhenRcv

    \SpaceHandler

    \vspace{-1pt}

    \WhenRcv[\BHEARTBEAT($i$, $\ts$)]\label{line:heartbeatreceive}
      \State \textbf{pre:} $\ts > \replicavectorclock[i]$
      \State $\replicavectorclock[i] \gets \ts$
    \EndWhenRcv

    \SpaceHandler

    \vspace{-1pt}

    \Function{\PROPAGATEREMOTE}{$i, j$}\label{line:forward}
          \State {\bf var} $\atxs \gets \{\langle \_, \_, \commitvector\rangle
            \in \committedset[j] \mid $\label{line:txstoforward}
          \Statex \hspace{3cm} $\commitvector[j]>\rvc[i][j]\}$
          \If{$\atxs \neq \emptyset$}
                  {\bf send}
                $\REPLICATE(j, \atxs)$
                \textbf{to} $\partition^m_i$
          \Else {}
            {\bf send}
              $\BHEARTBEAT(j, \replicavectorclock[j])$
              \textbf{to} $\partition^m_i$\label{line:heartbeatforward}
          \EndIf
    \EndFunction

    
    \SpaceHandler
    \vspace{-1pt}

    \Function{\CASTVC}{$ $} \Comment{Run periodically} \label{alg:line:bcast}
    \State {\bf send} $\UPDCSS(m, \replicavectorclock)$ 
    \textbf{to} $\partition^\areplica_d,\, \areplica \in \Partitions$ \label{alg:line:send-knownvc}
    \State {\bf send}
    $\UPDUSS(d, \stablesnapshot)$ \textbf{to} $\partition^m_i,\, i \in \DC$\label{alg:line:updategsssend}
    \State {\bf send}
    $\UPDRCV(d, \replicavectorclock)$ \textbf{to} $\partition^m_i,\, i \in \DC$\label{line:updateglobalmatrixsend}
    \EndFunction

    \SpaceHandler
    \vspace{-1pt}

    \WhenRcv[\UPDCSS($\areplica$, $\areplicavectorclock$)] \label{alg:line:updategss}
      \State $\pmc[\areplica] \gets \areplicavectorclock$\label{alg:line:localknownmatrix}
      \For{$i \in \DC$}
        $\stablesnapshot[i] \gets \min \{\pmc[\asecondreplica][i] \mid \asecondreplica \in \Partitions\}$\label{line:computestable}
      \EndFor
      \State \colorbox{\BackColor}{{\color{\StrongColor}
          $\stablesnapshot[\red]\gets \min\{\pmc[\asecondreplica][\red] \mid \asecondreplica \in \Partitions\}$}}\label{line:stablered}
    \EndWhenRcv

    \SpaceHandler
    \vspace{-1pt}

    \WhenRcv[\UPDUSS($i$, $\astablesnapshot$)]\label{line:updateuniformvc}
      \State $\gmc[i] \gets \astablesnapshot$\label{line:updatestablematrix}
      \State $G\gets$ all groups with $f+1$ replicas that include $\partition^m_d$\label{line:enumerate}
      \For{$j \in \DC$}
        \State \textbf{var} $\ts \gets \max \{ \min \{ \gmc[\athirddc][j] \mid \athirddc \in g \} \mid g \in G\}$
        \State $\uniformsnapshot[j] \gets \max \{ \uniformsnapshot[j]$, $\ts\}$
      \EndFor
    \EndWhenRcv

    \SpaceHandler
    \vspace{-1pt}

    \WhenRcv[\UPDRCV($\areplica$, $\areplicavectorclock$)]\label{line:updateglobalmatrix}
      \State $\rvc[\areplica] \gets \areplicavectorclock$
    \EndWhenRcv
  \end{algorithmic}
  \caption{Transaction replication at $\partition^m_d$.}
  \label{alg:replication}
  \label{alg:clock}
\end{algorithm}









When a replica $\partition^m_d$ receives a $\REPLICATE$ message with a set of
transactions $\atxs$ originating at a sibling replica $\partition^m_i$
(\algline{alg:replication}{line:receivereplicate}), it iterates over $\atxs$ in
$\commitvector[i]$ order. For each new transaction in $\atxs$ with commit vector
$\commitvector$, the replica adds the transaction's operations to its log and
sets $\replicavectorclock[i]=\commitvector[i]$. Since communication channels are
FIFO, $\partition^m_d$ processes all transactions from $\partition^m_i$ in their
local timestamp order. Hence, the above update to $\replicavectorclock[i]$
preserves Property~\ref{prop:knownvc}: $\partition^m_d$ stores updates
originating at $\partition^m_i$ by all transactions with
$\commitvector[i]\leq \replicavectorclock[i]$.  Finally, the replica adds the
transactions to $\committedset[i]$, which is used to implement transaction
forwarding (\S\ref{sec:overview}).  Due to the forwarding, $\partition^m_d$ may
receive the same transaction from different data centers. Thus, when processing
transactions in the $\REPLICATE$ message, it checks for duplicates
(\algline{alg:replication}{line:receivepreparedprecondition}).









\subsection{Advancing the Uniform Snapshot}
\label{sec:clockcomputation}

Replicas run a background protocol that refreshes the information about uniform
transactions.
This proceeds in two stages. First, a replica keeps track of which transactions
have been replicated at the replicas of other partitions in the same data
center. To this end, replicas in the same data center periodically exchange
$\UPDCSS$ messages with their $\replicavectorclock$ vectors, which they store in
a matrix $\pmc$
(\algtwolines{alg:clock}{alg:line:send-knownvc}{alg:line:updategss});
in our implementation this is done via a dissemination tree. This
matrix is then used to compute the vector $\stablesnapshot$, which represents
the set of transactions that have been fully replicated at the local data center
as per Property~\ref{prop:stablevc}. To ensure this, a replica computes an entry
$\stablesnapshot[i]$ as the minimum of $\replicavectorclock[i]$ it received from
the replicas of other partitions in the same data center
(\algline{alg:clock}{line:computestable}).



In the second stage of the background protocol, sibling replicas periodically
exchange $\UPDUSS$ messages with their $\stablesnapshot$ vectors, which they
store in a matrix $\gmc$
(\algtwolines{alg:clock}{alg:line:updategsssend}{line:updateuniformvc}). This
matrix is then used by a replica
to compute $\uniformsnapshot$, which characterizes the update transactions that
are replicated at $f+1$ data centers as per Property~\ref{prop:uniformity}. To
this end, a replica first enumerates all groups $G$ of $f+1$ data centers that
include its local data center (\algline{alg:clock}{line:enumerate}). For each
data center $j$ the replica performs the following computation. First, for each
group $g \in G$, it computes the minimum $j$-th entry in the stable vectors of
all data centers $\athirddc \in g$:
$\min\{\gmc[\athirddc][j]\mid \athirddc\in g\}$. By Property~\ref{prop:stablevc}
all update transactions originating at $j$ with local timestamp $\le$ the
minimum have been replicated at all data centers in $g$. The replica then sets
$\uniformsnapshot[j]$ to the maximum of the resulting values computed for all
groups $g \in G$, to cover transactions that are replicated at any such group.
According to Property~\ref{prop:uniformsnapshot}, the transactions with commit
vectors $\leq \uniformsnapshot$ are uniform, and now become visible to
transactions coordinated by $\partition^m_d$ (\S\ref{sec:bluetxs}).


Replicas also update $\uniformsnapshot$ in lines
\ref{alg:txncoord1}:\ref{line:starttx:start}-\ref{line:starttx:end},
\ref{alg:txnpartition}:\ref{line:updateuniformop:start}-\ref{line:updateuniformop:end}
and
\ref{alg:txnpartition}:\ref{line:updateuniformprepare:start}-\ref{line:updateuniformprepare:end}
by incorporating $\vecsnapshottime[i]$ for remote data centers $i$. This is
safe because a transaction executes on a snapshot that only includes uniform
remote transactions.

Finally, if a replica does not receive new transactions for a long time, it
sends the value of its $\replicavectorclock[d]$ as a heartbeat
(\algtwolines{alg:replication}{line:updateknownprepared}{line:heartbeatreceive}).
This allows advancing $\stablesnapshot$ and $\uniformsnapshot$ even under skewed
load distributions.




\subsection{Transaction Forwarding}
\label{sec:forward}


As we explained in \S\ref{sec:overview}, to guarantee that a transaction
originating at a correct data center eventually becomes exposed at all correct
data centers despite failures (\liveness), replicas may have to forward remote
update transactions. To determine which transactions to forward, each replica
keeps track of the update transactions that have been replicated at sibling
replicas in other data centers. To this end, sibling replicas periodically
exchange $\UPDRCV$ messages with their $\replicavectorclock$ vectors, which they
store in a matrix $\rvc$
(\algtwolines{alg:clock}{line:updateglobalmatrixsend}{line:updateglobalmatrix}).
Thus, $\partition^m_i$ has received all update transactions from
$\partition^m_j$ with $\commitvector[j]\leq \rvc[i][j]$.

A replica $\partition^m_d$ only forwards transactions when it suspects that a
data center $j$ may have failed before replicating all the update transactions
originating at it to a data center $i$ (this information is provided by a
separate module).
In this case, $\partition^m_d$ executes $\PROPAGATEREMOTE(i, j)$
(\algline{alg:replication}{line:forward}). The function forwards the set of 
transactions $\atxs$ received from $\partition^m_j$ that have not been
replicated at $\partition^m_i$ according to
$\rvc[i][j]$. For example, in
Figure~\ref{fig:execution-causal}, \System will eventually invoke
$\PROPAGATEREMOTE(d_1, d_3)$ at replicas in $d_2$ to forward {\color{\CausalTxColor}$t_1$}. The replica
$\partition^m_d$ sends the transactions in $\atxs$ to $\partition^m_i$ in a $\REPLICATE$
message. If there are no update transactions to forward, $\partition^m_d$ sends
a heartbeat to $\partition^m_i$ with $\replicavectorclock[j]$.





\System periodically deletes from $\committedset$ transactions that have been
replicated at every data center (omitted from the pseudocode for brevity).


\subsection{On-Demand Durability and Client Migration}
\label{sec:clientmigration}


A client may wish to ensure that the transactions it has observed so far are
durable. To this end, the client can call $\MAKEUNIFORM(\avc)$ at any replica in
its local data center $d$, where $\avc$ is the client's causal past $\past$
(\algline{alg:txncoord1}{line:barrier}). The replica returns to the client only
when all the transactions from $\past$ that originate at $d$ are uniform, and
thus durable. Then the same holds for all transactions from $\past$, because the
protocol only exposes remote transactions to clients when they are already
uniform (\S\ref{sec:bluetxs}).






A client wishing to migrate from its local data center $d$ to another data
center $i$ first calls $\MAKEUNIFORM(\avc)$ at any replica in $d$ with
$\avc = \past$, to ensure that the transactions the client has observed or
issued at $d$ will eventually become visible at $i$. The client then calls
$\ATTACH(\avc)$ at any replica in $i$ %
(\algline{alg:txncoord1}{line:attach}). The replica returns when its
$\uniformsnapshot$ includes all remote transactions from $\avc$
(\algline{alg:txncoord1}{line:waitattach}). The client can then be sure that its
transactions at $i$ will operate on snapshots including all the transactions it
has observed before.









\section{Adding Strong Transactions}
\label{sec:redtransactions}

We now describe the full \System protocol with both causal and strong
transactions. It is obtained by adding the highlighted lines to
Algorithms~\ref{alg:txncoord1}-\ref{alg:clock} and a new
Algorithm~\ref{alg:txncoord2}.


\subsection{Metadata}


The Conflict Ordering property of our consistency model requires any two
conflicting strong transactions to be related one way or another by the causal
order $\prec$ (\S\ref{sec:consistency}). To ensure this, the protocol assigns to
each strong transaction a scalar {\em strong timestamp}, analogous to those used
in optimistic concurrency control for serializability~\cite{wv}.  Several
vectors used as metadata in the causal consistency protocol
(\S\ref{sec:metadata}) are then extended with an extra $\red$ entry.

First, we extend commit vectors and those representing causally consistent
snapshots. Commit vectors are compared using the previous order $<$, but
considering all entries; as before, this order is consistent with the causal
order $\prec$. Furthermore, conflicting strong transactions are causally ordered
according to their strong timestamps.
\begin{property}
  For any conflicting strong transactions $t_1$ and $t_2$ with commit vectors
  $\commitvector_1$ and $\commitvector_2$, we have:
  $t_1 \prec t_2 \Longleftrightarrow \commitvector_1[\red] < \commitvector_2[\red]$.
\end{property}

A consistent snapshot vector $V$ now defines the set of transactions with a
commit vector $\le V$, according to the new $<$. The vectors
$\replicavectorclock$ and $\stablesnapshot$ maintained by a replica
$\partition^m_d$ are also extended with a $\red$ entry. The entries
$\replicavectorclock[\red]$ and $\stablesnapshot[\red]$ define the prefix of
strong transactions that have been replicated at $\partition^m_d$ and the local
data center $d$, respectively:
\begin{property}
\label{prop:knownvcred}
Replica $\partition^m_d$ stores the updates to $m$ by all strong
transactions with $\commitvector[\red]\leq \replicavectorclock[\red]$.
\end{property}
\begin{property}
\label{prop:stablevcred}
Data center $d$ stores the updates by all strong transactions with
$\commitvector[\red]\leq \stablesnapshot[\red]$.
\end{property}
To ensure Property~\ref{prop:stablevcred}, the $\red$ entry of $\stablesnapshot$
is updated at \algline{alg:clock}{line:stablered} similarly to its other
entries. We do not extend $\uniformsnapshot$, because our commit protocol for
strong transactions automatically guarantees their uniformity.






\subsection{Transaction Execution}
\label{sec:redexecution}

\System uses optimistic concurrency control for strong transactions, with the
same protocol for executing causal and speculatively executing strong
transactions. To this end, Algorithm~\ref{alg:txncoord1} is modified as
follows. First, the computation of the snapshot vector $\vecsnapshottime[\tx]$
is extended to compute the $\red$ entry
(\algline{alg:txncoord1}{alg:coord:starttx:end-red}), which is now taken into
account when checking that a replica state is up to date
(\algline{alg:txnpartition}{line:waitexecute}).  The $\red$ entry of the
snapshot vector is computed so as to include all strong transactions known to be
fully replicated in the local data center, as defined by
$\stablesnapshot[\red]$. To ensure {\em read your writes}, the snapshot
additionally includes strong transactions from the client's causal past, as
defined by $\avc[\red]$. Finally, the coordinator of a transaction now maintains
not only its write set, but also its read set $\readset$ that records all
operations by the transaction, including read-only ones
(\algline{alg:txncoord1}{line:readset}). The latter is used to certify strong
transactions.

After speculatively executing a strong transaction, the client tries to commit
it by calling $\COMMITRED$ at its coordinator
(\algline{alg:txncoord2}{line:commitred}). The coordinator first waits until the
snapshot on which the transaction operated becomes uniform by calling
$\MAKEUNIFORM$ (\algline{alg:txncoord2}{line:uniformred}): as we argued in
\S\ref{sec:overview}, this is crucial for liveness.
The coordinator next submits the transaction to a \emph{certification service},
which determines whether the transaction commits or aborts
(\algline{alg:txncoord2}{line:certifyred}, see \S\ref{sec:certification}).
In the former case, the service also determines its commit vector,
which the coordinator returns to the client. If the transaction commits,
the client sets its causal past $\past$ to the commit vector; otherwise, 
it re-executes the transaction. 

The certification service also notifies replicas in all data centers about
updates by strong transactions affecting them via $\DELIVERUPD$ upcalls, invoked
in an order consistent with strong timestamps of the transactions
(\algline{alg:txncoord2}{line:deliverred}). A replica receiving an upcall adds
the new operations to its log and refreshes $\replicavectorclock[\red]$ to
preserve Property~\ref{prop:knownvcred}.

Finally, a replica $\partition^m_d$ that has not seen any strong transactions
updating its partition $m$ for a long time submits a dummy strong transaction
that acts as a heartbeat (\algline{alg:txncoord2}{line:stronghb}). Similarly to
heartbeats for causal transactions, this allows coping with skewed load
distributions.


\begin{algorithm}[t]
  \begin{algorithmic}[1]
    \small
    \Function{\COMMITRED}{\tx}\label{line:commitred}
      \State $\MAKEUNIFORM(\vecsnapshottime[\tx])$\label{line:uniformred}
      \State \Return
        \CERTIFY(\tx, $\writeset[\tx]$, $\readset[\tx]$, $\vecsnapshottime[\tx]$) \label{line:certifyred}
    \EndFunction

    \SpaceHandler

    \Upon[\DELIVERUPD($W$)]\label{line:deliverred}
    \ForAll{\hspace{-1pt}$\langle \aws, \commitvector\rangle {\in} \hspace{1pt} W$\hspace{1pt}in\hspace{1pt}$\commitvector[\red]$\hspace{1pt}order}
      \ForAll{$\langle k, \op \rangle \in \aws$}
      \State $\store[k] \gets \store[k] \cdot \langle \op, \commitvector \rangle$\label{line:addred}
      \EndFor
      \State $\replicavectorclock[\red] \gets \commitvector[\red]$\label{line:setred}
      \EndFor
    \EndUpon
      

    \SpaceHandler

    \Function{\RHEARTBEAT}{$ $} \Comment{Run periodically}\label{line:stronghb}
      \State \Return \CERTIFY($\bot$, $\emptyset$, $\emptyset$, $\vec{0}$)
    \EndFunction

  \end{algorithmic}
  \caption{Committing strong transactions at $\partition^m_d$.}
  \label{alg:txncoord2}
\end{algorithm}

\subsection{Certification Service}
\label{sec:certification}



We implement the certification service using an existing fault-tolerant protocol
from~\cite{discpaper}, with transaction commit vectors computed using the
techniques from~\cite{multicast-dsn19}. The protocol integrates two-phase commit
across partitions accessed by the transaction and Paxos among the replicas of
each partition. It furthermore uses white-box optimizations between the two
protocols to minimize the commit latency.  The use of Paxos ensures that a
committed strong transaction is durable and its updates will eventually be
delivered at all correct data centers
(\algline{alg:txncoord2}{line:deliverred}). For each partition, a single replica
functions as the Paxos leader. The protocol is described and formally specified
elsewhere~\cite{discpaper}, and here we discuss it only briefly. Its pseudocode
and formal specification are given 
in~\tr{\ref{section:unistore-protocol}}{\nappfull}
and~\tr{\ref{section:tcs}}{\napptcs}, respectively.

The certification service accepts the read and write sets of a transaction and
its snapshot vector (\algline{alg:txncoord2}{line:certifyred}). Even though the
service is distributed, it guarantees that commit/abort decisions are computed
like in a centralized database with optimistic concurrency control -- in a total
{\em certification order}. To ensure Conflict Ordering,
the decisions are computed using a concurrency-control policy similar to that
for serializability~\cite{wv}: a transaction commits if its snapshot includes
all conflicting transactions that precede it in the certification order. The
certification service also computes a commit vector for each committed
transaction by copying its per-data center entries from the transaction's
snapshot vector and assigning a strong timestamp consistent with the
certification order.









\section{Proof of Correctness}
\label{sec:correctness}

We have rigorously proved that \System correctly implements the specification of
PoR consistency for the case when the data store manages last-writer-wins registers. The
proof uses the formal framework
from~\cite{sebastian-book,distrmm-popl,framework-concur15} and establishes
Properties~\ref{prop:knownvc}-\ref{prop:stablevcred} stated earlier. Due to
space constraints, we defer the proof
to~\tr{\ref{section:correctness-proof}}{\nappproof}.


\section{Evaluation}
\label{sec:evaluation}
\begin{table*}[!t]
\begin{center}
%\small
\caption {Benchmarks and applications for the study of the application-level resilience}
\vspace{-5pt}
\label{tab:benchmark}
\tiny
\begin{tabular}{|p{1.7cm}|p{7.5cm}|p{4cm}|p{2.5cm}|}
\hline
\textbf{Name} 	& \textbf{Benchmark description} 		& \textbf{Execution phase for evaluation}  			& \textbf{Target data objects}             \\ \hline \hline
CG (NPB)             & Conjugate Gradient, irregular memory access (input class S)   & The routine conj\_grad in the main computation loop  & The arrays $r$ and $colidx$     \\\hline
MG (NPB)    	       & Multi-Grid on a sequence of meshes (input class S)             & The routine mg3P in the main computation loop & The arrays $u$ and $r$ 	\\ \hline
FT (NPB)             & Discrete 3D fast Fourier Transform (input class S)            & The routine fftXYZ in the main computation loop  & The arrays $plane$ and $exp1$    \\ \hline
BT (NPB)             & Block Tri-diagonal solver (input class S)         		& The routine x\_solve in the main computation loop & The arrays $grid\_points$ and $u$	\\ \hline
SP (NPB)             & Scalar Penta-diagonal solver (input class S)         		& The routine x\_solve in the main computation loop & The arrays $rhoi$ and $grid\_points$  \\ \hline
LU (NPB)            & Lower-Upper Gauss-Seidel solver (input class S)        	& The routine ssor 	& The arrays $u$ and $rsd$ \\ \hline \hline
LULESH~\cite{IPDPS13:LULESH} & Unstructured Lagrangian explicit shock hydrodynamics (input 5x5x5) & 
The routine CalcMonotonicQRegionForElems 
& The arrays $m\_elemBC$ and $m\_delv\_zeta$ \\ \hline
AMG2013~\cite{anm02:amg} & An algebraic multigrid solver for linear systems arising from problems on unstructured grids (we use  GMRES(10) with AMG preconditioner). We use a compact version from LLNL with input matrix $aniso$. & The routine hypre\_GMRESSolve & The arrays $ipiv$ and $A$   \\ \hline
%$hierarchy.levels[0].R.V$ \\ \hline
\end{tabular}
\end{center}
\vspace{-5pt}
\end{table*}

%We evaluate the effectiveness of ARAT, and 
%We use ARAT to study the application-level resilience.
%The goal is to demonstrate 
%that aDVF can be a very useful metric to quantify the resilience of data objects
%at the application level. 
We study 12 data objects from six benchmarks of the NAS parallel benchmark (NPB) suite (we use SNU\_NPB-1.0.3) and 4 data objects from two scientific applications. 
%which is a c version of NPB 3.3, but ARAT can work for Fortran.
Those data objects are chosen to be representative: they have various data access patterns and participate in various execution phases.  
%For the benchmarks, we use CLASS S as the input problems and use the default compiler options of NPB.
For those benchmarks and applications, we use their default compiler options, and use gcc 4.7.3 and LLVM 3.4.2 for trace generation.
To count the algorithm-level fault masking, we use the default convergence thresholds (or the fault tolerance levels) for those benchmarks.
Table~\ref{tab:benchmark} gives 
%for->on by anzheng
detailed information on the benchmarks and applications.
The maximum fault propagation path for aDVF analysis is set to 10 by default.
%the value shadowing threshold is set as 0.01 (except for BT, we use $1 \times 10^{-6}$).
%These value shadowing thresholds are chosen such that any error corruption
%that results in the operand's value variance less than 1\% (for the threshold 0.01) or 0.0001\% (for the threshold $1 \times 10^{-6}$) during the 
%trace analysis does not impact the outcome correctness of six benchmarks.
%LU: check the newton-iteration residuals against the tolerance levels
%SP: check the newton-iteration residuals against the tolerance levels
%BT: check the newton-iteration residuals against the tolerance levels

\subsection{Resilience Modeling Results}
%We use ARAT to calculate aDVF values of 16 data objects. 
Figure~\ref{fig:aDVF_3tiers_profiling}
shows the aDVF results and breaks them down into the three levels 
(i.e., the operation-level, fault propagation level, and algorithm-level).
Figure~\ref{fig:aDVF_3classes_profiling} shows the 
%for->of by anzheng
results for the analyses at the levels of the operation and fault propagation,
and further breaks down the results into 
the three classes (i.e., the value overwriting, logical and comparison operations,
and value shadowing). %based on the reasons of the fault masking.
We have multiple interesting findings from the results.

\begin{figure*}
	\centering
        \includegraphics[width=0.8\textwidth]{three_tiers_gray.pdf}
% * <azguolu@gmail.com> 2017-03-23T03:20:28.808Z:
%
% ^.
        \vspace{-5pt}
        \caption{The breakdown of aDVF results based on the three level analysis. The $x$ axis is the data object name.}
        \vspace{-8pt}
        \label{fig:aDVF_3tiers_profiling}
\end{figure*}


\begin{figure*}
	\centering
	\includegraphics[width=0.8\textwidth]{three_types_gray.pdf}
	\vspace{-5pt}
	\caption{The breakdown of aDVF results based on the three classes of fault masking. The $x$ axis is the data object name. \textit{zeta} and \textit{elemBC} in LULESH are \textit{m\_delv\_zeta} and \textit{m\_elemBC} respectively.} % Anzheng
	\vspace{-5pt}
	\label{fig:aDVF_3classes_profiling}
    %\vspace{-5pt}
\end{figure*}

(1) Fault masking is common across benchmarks and applications.
Several data objects (e.g., $r$ in CG, and $exp1$ and $plane$ in FT)
have aDVF values close to 1 in Figure~\ref{fig:aDVF_3tiers_profiling}, 
which indicates that most of operations working on these data objects
have fault masking.
However, a couple of data objects have much less intensive fault masking.
For example, the aDVF value of $colidx$ in CG is 0.28 (Figure~\ref{fig:aDVF_3tiers_profiling}). 
Further study reveals that $colidx$ is an array to store column indexes of sparse matrices, and there is few operation-level or fault propagation-level fault masking  (Figure~\ref{fig:aDVF_3classes_profiling}).
The corruption of it can easily cause segmentation fault caught by the
algorithm-level analysis. 
$grid\_points$ in SP and BT also have a relatively small aDVF value (0.14 and 0.38 for SP and BT respectively in Figure~\ref{fig:aDVF_3tiers_profiling}).
Further study reveals that $grid\_points$ defines input problems for SP and BT. 
A small corruption of $grid\_points$ 
%change->changes by anzheng
can easily cause major changes in computation
caught by the fault propagation analysis. 

The data object $u$ in BT also has a relatively small aDVF value (0.82 in Figure~\ref{fig:aDVF_3tiers_profiling}).
Further study reveals that $u$ is read-only in our target code region
for matrix factorization and Jacobian, neither of which is friendly
for fault masking.
Furthermore, the major fault masking for $u$ comes from value shadowing,
and value shadowing only happens in a couple of the least significant bits 
of the operands that reference $u$, which further reduces the value of aDVF.
%also reduces fault masking.

(2) The data type is strongly correlated with fault masking.
Figure~\ref{fig:aDVF_3tiers_profiling} reveals that the integer data objects ($colidx$ in CG, $grid\_points$ in BT and SP, $m\_elemBC$ in LULESH) appear to be 
more sensitive to faults than the floating point data objects 
($u$ and $r$ in MG, $exp1$ and $plane$ in FT, $u$ and $rsd$ in LU, $m\_delv\_zeta$ in LULESH, and $rhoi$ in SP).
In HPC applications, the integer data objects are commonly employed to
define input problems and bound computation boundaries (e.g., $colidx$ in CG and $grid\_points$ in BT), 
or track computation status (e.g., $m\_elemBC$ in LULESH). Their corruption 
%these integer data objects
is very detrimental to the application correctness. 

(3) Operation-level fault masking is very common.
For many data objects, the operation-level fault masking contributes 
more than 70\% of the aDVF values. For $r$ in CG, $exp1$ in FT, and $rhoi$ in SP,
the contribution of the operation-level fault masking is close to 99\% (Figure~\ref{fig:aDVF_3tiers_profiling}).

Furthermore, the value shadowing is a very common operation level fault masking,
especially for floating point data objects (e.g., $u$ and $r$ in BT, $m\_delv\_zeta$ in LULESH, and $rhoi$ in SP in Figure~\ref{fig:aDVF_3classes_profiling}).
This finding has a very important indication for studying the application resilience.
In particular, the values of a data object can be different across different input problems. If the values of the data object are different, 
then the number of fault masking events due to the value shadowing will be different. 
Hence, we deduce that the application resilience
can be correlated with the input problems,
because of the correlation between the value shadowing and input problems. 
We must consider the input problems when studying the application resilience.
This conclusion is consistent with a very recent work~\cite{sc16:guo}.

(4) The contribution of the algorithm-level fault masking to the application resilience can be nontrivial.
For example, the algorithm-level fault masking contributes 19\% of the aDVF value for $u$ in MG and 27\% for $plane$ in FT (Figure~\ref{fig:aDVF_3tiers_profiling}).
The large contribution of algorithm-level fault masking in MG is consistent with
the results of existing work~\cite{mg_ics12}. 
For FT (particularly 3D FFT), the large contribution of algorithm-level fault masking in $plane$ (Figure~\ref{fig:aDVF_3tiers_profiling})
comes from frequent transpose and 1D FFT computations that average out 
or overwrite the data corruption.
CG, as an iterative solver, is known to have the algorithm-level fault masking
because of the iterative nature~\cite{2-shantharam2011characterizing}.
Interestingly, the algorithm-level fault masking in CG contributes most to the resilience of $colidx$ which is a vulnerable integer data object (Figure~\ref{fig:aDVF_3tiers_profiling}).

%Our study reveals the algorithm-level fault masking of CG from
%two perspectives. First, $a$ in CG, which is an array for intermediate results,
%has few algorithm-level fault masking (0.008\%);
%Second, $x$ in CG, which is a result vector, has 5.4\% of the aDVF value coming from the algorithm-level fault masking.
%This result indicates that the effects of the algorithm-level fault masking
%are not uniform across data objects. 

(5) Fault masking at the fault propagation level is small.
For all data objects, the contribution of the fault masking at the level of fault propagation is less than 5\% (Figure~\ref{fig:aDVF_3tiers_profiling}).
For 6 data objects ($r$ and $colidx$ in CG, $grid\_points$ and $u$ in BT, and 
$grid\_points$ and $rhoi$ in SP),  there is no fault masking at the level of fault propagation.
In combination with the finding 4, we conclude that once the fault
is propagated, it is difficult to mask it because of the contamination of
more data objects after fault propagation, and only the algorithm semantics can tolerate  propagated faults well. 
%This finding is consistent with our sensitivity analysis. 

(6) Fault masking by logical and comparison operations is small,
%For all data objects, the fault masking contributions due to logical and comparison operations are very small, 
comparing with the contributions of value shadowing and overwriting (Figure~\ref{fig:aDVF_3classes_profiling}). 
Among all data objects, 
the logical and comparison operations in $grid\_points$ in BT contribute the most (25\% contribution in Figure~\ref{fig:aDVF_fine_profiling}), 
because of intensive ICmp operations (integer comparison). %logical OR and SHL (left shifting).


(7) The resilience varies across data objects. %within the same application.
This fact is especially pronounced in two data objects $colidx$ and $r$ in CG (Figure~\ref{fig:aDVF_3tiers_profiling}).
 $colidx$ has aDVF much smaller than $r$, which means $colidx$ is much less resilient than $r$ (see finding 1 for a detailed analysis on $colidx$). 
Furthermore, $colidx$ and $r$ have different algorithm-level
fault masking (see finding 4 for a detailed analysis).

\begin{comment}
\textbf{Finding 7: The resilience of the same data objects varies across different applications.}
This fact is especially pronounced in BT and SP.
BT and SP address the same numerical problem but with different algorithms.
BT and SP have the same data objects, $qs$ and $rhoi$, but
$qs$ manifests different resilience in BT and SP.
This result is interesting, because it indicates that by using
different algorithms, we have opportunities to
improve the resilience of data objects.
\end{comment}

To further investigate the reasons for fault masking, 
we break down the aDVF results at the granularity of LLVM instructions,
based on the analyses at the levels of operation and fault propagation.
The results are shown in Figure~\ref{fig:aDVF_fine_profiling}.
%Because of the space limitation, 
%we only show one data object per benchmark, but each selected data object has the most diverse fault masking events within the corresponding benchmark.
%Based on Figure~\ref{fig:aDVF_fine_profiling}, we have another interesting finding.

(8) Arithmetic operations make a lot of contributions to fault masking.
%For $r$ in CG, $r$ in MG, $exp1$ in FT, $u$ in BT, $qs$ in SP, and $u$ in LU,
%the arithmetic operations, FMul (100\%), Add (16\%), FMul (85\%), 
%FMul (94\%), FMul (28\%), and FAdd (50\%)
For $r$ in CG, $u$ in BT, $plane$ and $exp1$ in FT, $m\_elemBC$ in LULESH, 
arithmetic operations (addition, multiplication, and division) contribute to almost 100\% of the fault masking (Figure~\ref{fig:aDVF_fine_profiling}).  
%(at the operation level and the fault propagation level).
%For $qs$ in SP and $u$ in LU, the store operation also makes
%important contributions as the arithmetic operations because of value overwriting.

\begin{figure*}
	\centering
	\includegraphics[width=0.77\textheight, height=0.23\textheight]{pie_chart.pdf}
	\vspace{-10pt}
	\caption{Breakdown of the aDVF results based on the analyses at the levels of operation and fault propagation}
    \vspace{-10pt}
	\label{fig:aDVF_fine_profiling}
\end{figure*}


\subsection{Sensitivity Study}
\label{sec:eval_sen}
%\textbf{change the fault propagation threshold and study the sensitivity of analysis to the threshold}
ARAT uses 10 as the default fault propagation analysis threshold. 
The fault propagation analysis will not go beyond 10 operations. Instead,
we will use deterministic fault injection after 10 operations. 
In this section, we study the impact of this threshold on the modeling accuracy. We use a range of threshold values and examine how the aDVF value varies and whether
the identification of fault masking varies. 
Figure~\ref{fig:sensitivity_error_propagation} shows the results for 
%add , after BT by anzheng
multiple data objects in CG, BT, and SP.
We perform the sensitivity study for all 16 data objects.
%in six benchmarks and two applications.
Due to the page space limitation, we only show the results for three data objects,
but we summarize the sensitivity study results for all data objects in this section.
%but other data objects in all benchmarks have the same trend.

Our results reveal that the identification of fault masking by tracking fault propagation is not significantly 
affected by the fault propagation analysis threshold. Even if we use a rather large threshold (50), 
the variation of aDVF values is 4.48\% on average among all data objects,
and the variation at each of the three levels of analysis (the operation level, fault propagation level,  and algorithm level) is less than 5.2\% on average. 
In fact, using a threshold value of 5 is sufficiently accurate in most of the cases (14 out of 16 data objects).
This result is consistent with our finding 5 (i.e., fault masking at the fault propagation level is small). %in most benchmarks).
However, we do find a data object ($m\_elementBC$ in LULESH) %and $exp1$ in FT) 
showing relatively high-sensitive (up to 15\% variation) to the threshold. For this uncommon data object, using 50 as the fault propagation path is sufficient. 

%In other words, even though using a larger threshold value can identify more error masking by tracking error 
%propagation, the implicit error masking induced by the error propagation is very limited.

\begin{figure}
		\begin{center}
		\includegraphics[width=0.48\textwidth,height=0.11\textheight]{sensi_study_gray.pdf}
		\vspace{-15pt}
		\caption{Sensitivity study for fault propagation threshold}
		\label{fig:sensitivity_error_propagation}
		\end{center}
\vspace{-15pt}
\end{figure}


\begin{comment}
\subsection{Comparison with the Traditional Random Fault Injection}
%\textbf{compare with the traditional fault injection to verify accuracy}
To show the effectiveness of our resilience modeling, we compare traditional random fault injection
and our analytical modeling. Figure~\ref{fig:comparison_fi} and Table~\ref{tab:comparison} show the results.
The figure shows the success rate of all random fault injection. The ``success'' means the application
outcome is verified successfully by the benchmarks and the execution does not have any segfault. The success rate is used as a metric
to evaluate the application resilience.

We use a data-oriented approach to perform random fault injection.
In particular, given a data object, for each fault injection test we trigger a bit flip
in an operand of a random instruction, and this operand must be a reference to the
target data object. We develop a tool based on PIN~\cite{pintool} to implement the above fault injection functionality.
For each data object, we conduct five sets of random fault injection tests, 
and each set has 200 tests (in total 1000 tests per data object). 
We show the results for CG and FT in this section, but we find that
the conclusions we draw from CG and FT are also valid for the other four benchmarks.


%\begin{table*}
%\label{tab:success_rate}
%\begin{centering}
%\renewcommand\arraystretch{1.1}
%\begin{tabular}{|c|c|c|c|c|c|c|}
%\hline 
%Success Rate (Difference) & Test set 1 & Test set 2 & Test set 3 & Test set 4 & Test set 5 & Average\tabularnewline
%\hline 
%\hline 
%CG-a & 66.1\% (11.7\%) & 68.5\% (15.7\%) & 56.7\% (4.21\%) & 61.3\% (3.57\%) & 43.3\% (26.8\%) & 59.2\%\tabularnewline
%\hline 
%CG-x & 99.2\% (2.2\%) & 98.6\% (1.5\%) & 96.5\% (0.63\%) & 97.8\% (0.64\%) & 93.6\% (3.7\%) & 97.1\%\tabularnewline
%\hline 
%CG-colidx & 36.8\% (12.7\%) & 49.6\% (17.8\%) & 40.2\% (4.6\%) & 52.6\% (24.9\%) & 31.4\% (25.4\%) & 42.1\%\tabularnewline
%\hline 
%FT-exp1 & 52.7\% (1.4\%) & 22.6\% (56.5\%) & 78.5\% (51.0\%) & 60.7\% (16.7\%) & 45.4\% (12.7\%) & 51.9\%\tabularnewline
%\hline 
%FT-plane & 82.1\% (2.5\%) & 79.3\% (5.6\%) & 99.5\% (18.2\%) & 93.2\% (10.7\%) & 66.8\% (20.6\%) & 84.2\%\tabularnewline
%\hline 
%\end{tabular}
%\par\end{centering}
%\caption{XXXXX}
%\end{table*}


\begin{table*}
\begin{centering}
\caption{\small The results for random fault injection. The numbers in parentheses for each set of tests (200 tests per set) are the success rate difference from the average success rate of 1000 fault injection tests.}
\label{tab:comparison}
\renewcommand\arraystretch{1.1}
\begin{tabular}{|c|p{2.2cm}|p{2.2cm}|p{2.2cm}|p{2.2cm}|p{2.2cm}|p{1.8cm}|}
\hline 
       %& Test set 1 & Test set 2 & Test set 3 & Test set 4 & Test set 5 & Average\tabularnewline
       & \hspace{13pt} Test set 1 \hspace{1pt}/  & \hspace{13pt} Test set 2 \hspace{1pt}/ & \hspace{13pt} Test set 3 \hspace{1pt}/ & \hspace{13pt} Test set 4 \hspace{1pt}/ & \hspace{13pt} Test set 5 \hspace{1pt}/ & Ave. of all test / \\
       & success rate (diff.) & success rate (diff.) & success rate (diff.) & success rate (diff.) & success rate (diff.) & \hspace{5pt} success rate \\
\hline 
\hline 
CG-a & 66.1\% (6.9\%) & 68.5\% (9.3\%) & 56.7\% (-2.5\%) & 61.3\% (2.1\%) & 43.3\% (-15.9\%) & 59.2\%\tabularnewline
\hline 
CG-x & 99.2\% (2.1\%) & 98.6\% (1.5\%) & 96.5\% (-0.6\%) & 97.8\% (0.7\%) & 93.6\% (-3.5\%) & 97.1\%\tabularnewline
\hline 
CG-colidx & 36.8\% (-5.3\%) & 49.6\% (7.5\%) & 40.2\% (-2.0\%) & 52.6\% (10.5\%) & 31.4\% (-10.7\%) & 42.1\%\tabularnewline
\hline 
FT-exp1 & 52.7\% (0.8\%) & 22.6\% (-29.3\%) & 78.5\% (26.6\%) & 60.7\% (8.8\%) & 45.4\% (-6.5\%) & 51.9\%\tabularnewline
\hline 
FT-plane & 82.1\% (-2.1\%) & 79.3\% (-4.9\%) & 99.5\% (15.3\%) & 93.2\% (9.0\%) & 66.8\% (-17.4\%) & 84.2\%\tabularnewline
\hline 
\end{tabular}
\par\end{centering}
\vspace{-0.4cm}
\end{table*}

\begin{figure}
	\begin{center}
		\includegraphics[width=0.48\textwidth,keepaspectratio]{verifi-study.png}
		\caption{The traditional random fault injection vs. ARAT}
		\label{fig:comparison_fi}
	\end{center}
\vspace{-0.7cm}
\end{figure}


We first notice from Table~\ref{tab:comparison} that 
%across 5 sets of random fault injection tests, there are big variances (up to 55.9\% in $exp1$ of FT) in terms of the success rate. 
the results of 5 test sets can be quite different from each other and from 1000 random fault inject tests (up to 29.3\%).
1000 fault injection tests provide better statistical significance than 200 fault injection tests.
We expect 1000 fault injection tests potentially provide higher accuracy to quantify the application resilience.
The above result difference is clearly an indication to the randomness of fault injection, and there
is no guarantee on the random fault injection accuracy.

%In Figure~\ref{fig:comparison_fi}, 
We compare the success rate of 1000 fault inject tests with the aDVF value (Fig.~\ref{fig:comparison_fi}). 
We find that the order of the success rate of the three data objects in CG (i.e., $colidx < a < x$) and the two data objects in FT 
(i.e., $exp1 < plane$) is the same as the order of the aDVF values of these data objects. 
%In fact, 1000 fault injection tests
%account for \textcolor{blue}{\textbf{xxx\%}} of total memory references to the data object,
%and provide better resilience quantification than 200 fault injection tests.
The same order (or the same resilience trend)
%between our approach and the random fault injection based on a large number of tests 
is a demonstration of the effectiveness of our approach.
Note that the values of the aDVF and success rate %for a data object
cannot be exactly the same (even if we have sufficiently large numbers of random fault injection), 
because aDVF and random fault injection quantify
the resilience based on different metrics.
Also, the random fault injection can miss some fault masking events that can be captured by our approach.

\end{comment}
\section{Related Work}\label{sec:related}
 
The authors in \cite{humphreys2007noncontact} showed that it is possible to extract the PPG signal from the video using a complementary metal-oxide semiconductor camera by illuminating a region of tissue using through external light-emitting diodes at dual-wavelength (760nm and 880nm).  Further, the authors of  \cite{verkruysse2008remote} demonstrated that the PPG signal can be estimated by just using ambient light as a source of illumination along with a simple digital camera.  Further in \cite{poh2011advancements}, the PPG waveform was estimated from the videos recorded using a low-cost webcam. The red, green, and blue channels of the images were decomposed into independent sources using independent component analysis. One of the independent sources was selected to estimate PPG and further calculate HR, and HRV. All these works showed the possibility of extracting PPG signals from the videos and proved the similarity of this signal with the one obtained using a contact device. Further, the authors in \cite{10.1109/CVPR.2013.440} showed that heart rate can be extracted from features from the head as well by capturing the subtle head movements that happen due to blood flow.

%
The authors of \cite{kumar2015distanceppg} proposed a methodology that overcomes a challenge in extracting PPG for people with darker skin tones. The challenge due to slight movement and low lighting conditions during recording a video was also addressed. They implemented the method where PPG signal is extracted from different regions of the face and signal from each region is combined using their weighted average making weights different for different people depending on their skin color. 
%

There are other attempts where authors of \cite{6523142,6909939, 7410772, 7412627} have introduced different methodologies to make algorithms for estimating pulse rate robust to illumination variation and motion of the subjects. The paper \cite{6523142} introduces a chrominance-based method to reduce the effect of motion in estimating pulse rate. The authors of \cite{6909939} used a technique in which face tracking and normalized least square adaptive filtering is used to counter the effects of variations due to illumination and subject movement. 
The paper \cite{7410772} resolves the issue of subject movement by choosing the rectangular ROI's on the face relative to the facial landmarks and facial landmarks are tracked in the video using pose-free facial landmark fitting tracker discussed in \cite{yu2016face} followed by the removal of noise due to illumination to extract noise-free PPG signal for estimating pulse rate. 

Recently, the use of machine learning in the prediction of health parameters have gained attention. The paper \cite{osman2015supervised} used a supervised learning methodology to predict the pulse rate from the videos taken from any off-the-shelf camera. Their model showed the possibility of using machine learning methods to estimate the pulse rate. However, our method outperforms their results when the root mean squared error of the predicted pulse rate is compared. The authors in \cite{hsu2017deep} proposed a deep learning methodology to predict the pulse rate from the facial videos. The researchers trained a convolutional neural network (CNN) on the images generated using Short-Time Fourier Transform (STFT) applied on the R, G, \& B channels from the facial region of interests.
The authors of \cite{osman2015supervised, hsu2017deep} only predicted pulse rate, and we extended our work in predicting variance in the pulse rate measurements as well.

All the related work discussed above utilizes filtering and digital signal processing to extract PPG signals from the video which is further used to estimate the PR and PRV.  %
The method proposed in \cite{kumar2015distanceppg} is person dependent since the weights will be different for people with different skin tone. In contrast, we propose a deep learning model to predict the PR which is independent of the person who is being trained. Thus, the model would work even if there is no prior training model built for that individual and hence, making our model robust. 

%
% \vspace{-0.5em}
\section{Conclusion}
% \vspace{-0.5em}
Recent advances in multimodal single-cell technology have enabled the simultaneous profiling of the transcriptome alongside other cellular modalities, leading to an increase in the availability of multimodal single-cell data. In this paper, we present \method{}, a multimodal transformer model for single-cell surface protein abundance from gene expression measurements. We combined the data with prior biological interaction knowledge from the STRING database into a richly connected heterogeneous graph and leveraged the transformer architectures to learn an accurate mapping between gene expression and surface protein abundance. Remarkably, \method{} achieves superior and more stable performance than other baselines on both 2021 and 2022 NeurIPS single-cell datasets.

\noindent\textbf{Future Work.}
% Our work is an extension of the model we implemented in the NeurIPS 2022 competition. 
Our framework of multimodal transformers with the cross-modality heterogeneous graph goes far beyond the specific downstream task of modality prediction, and there are lots of potentials to be further explored. Our graph contains three types of nodes. While the cell embeddings are used for predictions, the remaining protein embeddings and gene embeddings may be further interpreted for other tasks. The similarities between proteins may show data-specific protein-protein relationships, while the attention matrix of the gene transformer may help to identify marker genes of each cell type. Additionally, we may achieve gene interaction prediction using the attention mechanism.
% under adequate regulations. 
% We expect \method{} to be capable of much more than just modality prediction. Note that currently, we fuse information from different transformers with message-passing GNNs. 
To extend more on transformers, a potential next step is implementing cross-attention cross-modalities. Ideally, all three types of nodes, namely genes, proteins, and cells, would be jointly modeled using a large transformer that includes specific regulations for each modality. 

% insight of protein and gene embedding (diff task)

% all in one transformer

% \noindent\textbf{Limitations and future work}
% Despite the noticeable performance improvement by utilizing transformers with the cross-modality heterogeneous graph, there are still bottlenecks in the current settings. To begin with, we noticed that the performance variations of all methods are consistently higher in the ``CITE'' dataset compared to the ``GEX2ADT'' dataset. We hypothesized that the increased variability in ``CITE'' was due to both less number of training samples (43k vs. 66k cells) and a significantly more number of testing samples used (28k vs. 1k cells). One straightforward solution to alleviate the high variation issue is to include more training samples, which is not always possible given the training data availability. Nevertheless, publicly available single-cell datasets have been accumulated over the past decades and are still being collected on an ever-increasing scale. Taking advantage of these large-scale atlases is the key to a more stable and well-performing model, as some of the intra-cell variations could be common across different datasets. For example, reference-based methods are commonly used to identify the cell identity of a single cell, or cell-type compositions of a mixture of cells. (other examples for pretrained, e.g., scbert)


%\noindent\textbf{Future work.}
% Our work is an extension of the model we implemented in the NeurIPS 2022 competition. Now our framework of multimodal transformers with the cross-modality heterogeneous graph goes far beyond the specific downstream task of modality prediction, and there are lots of potentials to be further explored. Our graph contains three types of nodes. while the cell embeddings are used for predictions, the remaining protein embeddings and gene embeddings may be further interpreted for other tasks. The similarities between proteins may show data-specific protein-protein relationships, while the attention matrix of the gene transformer may help to identify marker genes of each cell type. Additionally, we may achieve gene interaction prediction using the attention mechanism under adequate regulations. We expect \method{} to be capable of much more than just modality prediction. Note that currently, we fuse information from different transformers with message-passing GNNs. To extend more on transformers, a potential next step is implementing cross-attention cross-modalities. Ideally, all three types of nodes, namely genes, proteins, and cells, would be jointly modeled using a large transformer that includes specific regulations for each modality. The self-attention within each modality would reconstruct the prior interaction network, while the cross-attention between modalities would be supervised by the data observations. Then, The attention matrix will provide insights into all the internal interactions and cross-relationships. With the linearized transformer, this idea would be both practical and versatile.

% \begin{acks}
% This research is supported by the National Science Foundation (NSF) and Johnson \& Johnson.
% \end{acks}

\paragraph{Acknowledgements.}
We thank our shepherd, Heming Cui, as well as Gregory Chockler, Vitor Enes, Luís
Rodrigues and Marc Shapiro for comments and suggestions. This work was partially
supported by an ERC Starting Grant RACCOON, the Juan de la Cierva Formaci\'on
funding scheme (FJC2018-036528-I), the CCF-Tencent Open Fund (CCF-Tencent
RAGR20200124) and the AWS Cloud Credit for Research program.


\bibliographystyle{abbrv}
\bibliography{biblio}

\iflong
\clearpage
\appendix

\renewcommand{\thealgorithm}{\thesection\arabic{algorithm}}

\section{The Full \unistore{} Protocol for LWW Registers} \label{section:unistore-protocol}

Algorithms~\ref{alg:unistore-client} -- \ref{alg:unistore-recovery} given in
this section define the full \unistore{} protocol, including parts omitted from
the main text. This version of the protocol is specialized to the case when the
data store manages last-writer-wins (LWW) registers.

Algorithm~\ref{alg:unistore-client} shows the pseudocode of clients.
We assume that each client is associated with a unique client identifier.
Each client maintains the following variables:
\begin{itemize}
  \item $\lc$: The Lamport clock at this client.
  \item $\clientdc$: The data center to which this client is currently connected.
  \item $\clientcoord$: The coordinator partition of the current ongoing transaction.
  \item $\ctid$: The identifier of the current ongoing transaction.
  \item $\pastVC$: The client's causal past.
\end{itemize}

A client interacts with \unistore{} via the following procedures:
\begin{itemize}
  \item $\tidvar \gets \Call{\start}{\null}$:
    Start a transaction and obtain an identifier $\tidvar$.
  \item $v \gets \Call{\read}{k}$:
    Invoke a read operation on key $k$ in the ongoing transaction and obtain a
    return value $v$.
  \item $\ok \gets \Call{\updateproc}{k, v}$:
    Invoke an update operation on key $k$ and value $v$
    in the ongoing transaction.
  \item $\ok \gets \Call{\commitcausaltx}{\null}$:
    Commit a causal transaction.
  \item $\decvar \gets \Call{\commitstrongtx}{\null}$:
    Try to commit a strong transaction
    and obtain a decision $\decvar \in \set{\commit, \abort}$.
  \item $\ok \gets \Call{\fence}{\null}$:
    Execute a uniform barrier.
  \item $\ok \gets \Call{\clattach}{j}$:
    Attach to data center $j$.
\end{itemize}


Algorithms~\ref{alg:unistore-coord} -- \ref{alg:unistore-strong-commit} show the
pseudocode of replicas. The code needed for strong transactions in
Algorithms~\ref{alg:unistore-coord}, \ref{alg:unistore-replica}, and
\ref{alg:unistore-clock} is highlighted in red.  Each replica $p^{m}_{d}$
maintains a set of variables as follows.
\begin{itemize}
  \item{$\clockVar$:} The current time at $p^{m}_{d}$.
  \item{$\rset$:} The read sets of transactions coordinated by $p^{m}_{d}$,
    indexed by transaction identifier $\tidvar$.
  \item{$\wbuff$:} The write buffers of transactions coordinated by $p^{m}_{d}$,
    indexed by transaction identifier $\tidvar$, partition $l$, and key $k$.
  \item{$\snapVC$:} The snapshot vectors of transactions coordinated by $p^{m}_{d}$,
    indexed by transaction identifier $\tidvar$.
  \item{$\oplog$:} The log of updates performed on keys managed by $p^{m}_{d}$,
    indexed by key $k$.
  \item{$\knownVC, \stableVC, \uniformVC$:}
    The vectors used by $p^{m}_{d}$ to track what is replicated where.
  \item{$\preparedcausal$:} The set of causal transactions
    local to $p^{m}_{d}$ that are prepared to commit.
  \item{$\committedcausal$:} For each data center $i$,
    $\committedcausal[i]$ stores transactions waiting to be
    replicated by $p^{m}_{d}$ to sibling replicas at other data centers than $i$.
  \item{$\localmatrix$:} The set of $\knownVC$ received by $p^{m}_{d}$
    from other partitions in data center $d$.
    It is used to compute $\stableVC$.
  \item{$\stablematrix$:} The set of $\stableVC$ received by $p^{m}_{d}$
    from sibling replicas.
    It is used to compute $\uniformVC$.
  \item{$\globalmatrix$:} The set of $\knownVC$ received by $p^{m}_{d}$
    from sibling replicas.
    It is used to track what has been replicated at sibling replicas.
\end{itemize}

We specialize the \unistore{} protocol to LWW registers in several ways.
First, we add code for managing Lamport clocks, highlighted in blue.
In particular, each (committed) transaction is associated with a Lamport
timestamp, equal to the value of $\lc$ at its client when the transaction completes
(lines~\code{\ref{alg:unistore-client}}{\ref{line:commitcausaltx-lc}}
and \code{\ref{alg:unistore-client}}{\ref{line:commitstrongtx-lc}}).
Lamport timestamps are totally ordered, with client identifiers used for tie-breaking
(see also Definition~\ref{def:lco}).

Second, in Algorithm~\ref{alg:unistore-coord}
we replace \doop{} by the following two procedures:

\begin{itemize}
  \item $v \gets \Call{\doread}{\tidvar, k}$:
    Execute a read operation on key $k$
    in a transaction with identifier $\tidvar$ and obtain a value $v$.
  \item $\Call{\doupdate}{\tidvar, k, v}$:
    Execute an update operation on key $k$ and value $v$
    in a transaction with identifier $\tidvar$.
\end{itemize}

Finally, in Algorithm~\ref{alg:unistore-replica}
we modify the handler of message \getversion{}:
\begin{itemize}
  \item $\Call{\readkey}{\snapvc, k}$:
    Read the latest value from key $k$
    based on the snapshot vector $\snapvc$.
    Specifically, it returns the last update to key $k$ of the transaction
    with the latest $\commitvc$ in terms of their Lamport clock order
    such that $\commitvc \leq \snapvc$
    (line~\code{\ref{alg:unistore-replica}}{\ref{line:readkey-read}}).
\end{itemize}

Algorithms~\ref{alg:unistore-certify} -- \ref{alg:unistore-recovery} contain the
implementation of the transaction certification service (see also
\S\ref{section:tcs}). The certification service uses an instance of the leader
election failure detector $\Omega_m$ for each partition $m$\footnote{Tushar
  Deepak Chandra, Vassos Hadzilacos, and Sam Toueg. The weakest failure detector
  for solving consensus. J. ACM, 1996.}. This primitive ensures that from some
point on, all correct processes nominate the same correct process as the leader.
For the case when the data store manages only registers, we assume that any two
writes to the same object conflict. Other data types and conflict relations can
be easily supported by modifying the certification check in
Algorithm~\ref{alg:unistore-certification}.

\setcounter{algorithm}{0}

\begin{algorithm*}[t]
  \caption{Client operations at client $\cl$}
  \label{alg:unistore-client}
  \begin{algorithmic}[1]
    \Function{\start}{\null}
      \label{line:function-starttx}
      \State $\clientcoord \gets$ an arbitrary replica in data center $\clientdc$
        \label{line:starttx-random-partition}
      \State $\ctid \gets \rpc \Call{\starttx}{\pastVC} \at \clientcoord$
        \label{line:starttx-call-start}
        \Comment{\tscolor{$\timestamp(\start) \gets
          \snapVC^{\clientcoord}_{\clientdc}[\ctid]$}}
        \label{line:starttx-ts}
      \State \Return $\ctid$
        \label{line:starttx-return}
    \EndFunction

    \Statex
    \Function{\read}{$k$} \label{line:function-read}
      \State $\langle v, \lccolor{\lcvar} \rangle \gets
        \rpc \Call{\doread}{\ctid, k} \at \clientcoord$
        \label{line:read-call-doread}
      \If{$\lcvar \neq \bot$}
        \label{line:read-is-external}
        \State \lccolor{$\lc \gets \max\set{\lc, \lcvar}$}
          \label{line:read-lc}
      \EndIf
      \State \Return $v$
        \label{line:read-return}
    \EndFunction

    \Statex
    \Function{\updateproc}{$k, v$} \label{line:function-update}
      \State $\rpc \Call{\doupdate}{\ctid, k, v} \at \clientcoord$
        \label{line:update-call-doupdate}
      \State \Return $\ok$
        \label{line:update-return}
    \EndFunction

    \Statex
    \Function{\commitcausaltx}{\null}
      \label{line:function-commitcausaltx}
      \State \lccolor{$\lc \gets \lc + 1$}
        \Comment{\lccolor{$\lclock(\commitcausaltx) \gets \lc$}}
        \label{line:commitcausaltx-lc}
      \State $\vcvar \gets \rpc \Call{\commitcausal}{\ctid, \lccolor{\lc}} \at \clientcoord$
        \label{line:commitcausaltx-call-commitcausal}
      \State $\pastVC \gets \vcvar$
        \label{line:commitcausaltx-pastvc}
        \Comment{\tscolor{$\timestamp(\commitcausaltx) \gets \pastVC$}}
        \label{line:commitcausaltx-ts}
      \State \Return $\ok$
        \label{line:commitcausaltx-return}
    \EndFunction

    \Statex
    \Function{\commitstrongtx}{\null}
      \label{line:function-commitstrongtx}
      \State \lccolor{$\lc \gets \lc + 1$}
        \label{line:commitstrongtx-lc-so}
      \State $\langle \decvar, \vcvar, \lccolor{\lcvar} \rangle \gets \rpc
        \Call{\commitstrong}{\ctid, \lccolor{\lc}} \at \clientcoord$
        \label{line:commitstrongtx-call-commitstrong}
      \If{$\decvar = \commit$}
        \label{line:commitstrongtx-if-commit}
        \State $\pastVC \gets \vcvar$
          \label{line:commitstrongtx-pastvc}
          \Comment{\tscolor{$\timestamp(\commitstrongtx) \gets \pastVC$}}
          \label{line:commitstrongtx-ts}
        \State \lccolor{$\lc \gets \lcvar$}
          \Comment{\lccolor{$\lclock(\commitstrongtx) \gets \lc$}}
          \label{line:commitstrongtx-lc}
      \EndIf
      \State \Return $dec$
        \label{line:commitstrongtx-return}
    \EndFunction

    \Statex
    \Function{\fence}{\null} \label{line:function-fence}
      \State \var $\p \gets$ an arbitrary replica in data center $\clientdc$
        \label{line:fence-partition}
      \State $\rpc \Call{\uniformbarrier}{\pastVC} \at \p$
        \label{line:fence-call-uniformbarrier}
        \Comment{\tscolor{$\timestamp(\fence) \gets \pastVC$}}
        \label{line:fence-ts}
      \State \lccolor{$\lc \gets \lc + 1$}
        \Comment{\lccolor{$\lclock(\fence) \gets \lc$}}
        \label{line:fence-lc}
      \State \Return $\ok$
        \label{line:fence-return}
    \EndFunction

    \Statex
    \Function{\clattach}{$j$} \label{line:function-migrate}
      \State \var $\p \gets$ an arbitrary replica in data center $j$
        \label{line:clattach-partition}
      \State $\rpc \Call{\attach}{\pastVC} \at \p$
        \label{line:clattach-call-attach}
        \Comment{\tscolor{$\timestamp(\clattach) \gets \pastVC$}}
        \label{line:clattach-ts}
      \State \lccolor{$\lc \gets \lc + 1$}
        \Comment{\lccolor{$\lclock(\clattach) \gets \lc$}}
        \label{line:clattach-lc}
      \State $\clientdc \gets j$
        \label{line:clattach-j}
      \State \Return $\ok$
        \label{line:clattach-return}
    \EndFunction
  \end{algorithmic}
\end{algorithm*}

\begin{algorithm*}[t]
  \caption{Transaction coordinator at $p^{m}_{d}$: causal commit}
  \label{alg:unistore-coord}
  \begin{algorithmic}[1]
    \Function{\starttx}{$\vc$} \label{line:function-start}
      \For{$i \in \D \setminus \set{d}$}
        \label{line:start-uniformvc-index}
        \State $\uniformVC[i] \gets \max\set{\vc[i], \uniformVC[i]}$
        \label{line:start-uniformvc}
      \EndFor

      \State {\bf var} $\tidvar \gets \generatetid()$
        \label{line:start-tid}
      \State $\snapVC[\tidvar] \gets \uniformVC$
        \label{line:start-snapvc}
      \State $\snapVC[\tidvar][d] \gets \max\set{\vc[d], \uniformVC[d]}$
        \label{line:start-snapvc-d}
      \State \strongcolor{$\snapVC[\tidvar][\strongentry] \gets
        \max\set{\vc[\strongentry], \stableVC[\strongentry]}$}
        \label{line:start-snapvc-strong}
      \State \Return $\tidvar$
        \label{line:start-return}
        \label{line:start-snapshotvc-of-t}
    \EndFunction

    \Statex
    \Function{\doread}{$\tidvar, k$}
      \label{line:function-doread}
      \State \var $l \gets \partitionofproc(k)$
        \label{line:doread-partition-of-k}
      \If{$\wbuff[\tidvar][l][k] \neq \bot$}
        \label{line:doread-from-buffer}
        \State \Return $\langle \wbuff[\tidvar][l][k], \lccolor{\bot} \rangle$
          \label{line:doread-return-from-buffer}
      \EndIf
      \State $\send \Call{\readkey}{\snapVC[\tidvar], k} \sendto p^{l}_{d}$
        \label{line:doread-from-snapshot}
      \State \wait\receive $\versionproc(v, \lccolor{\lcvar}) \from p^{l}_{d}$
      \State \strongcolor{$\rset[\tidvar] \gets \rset[\tidvar] \cup \set{k}$}
        \label{line:doread-readset}
      \State \Return $\langle v, \lccolor{\lcvar} \rangle$
        \label{line:doread-return-from-snapshot}
    \EndFunction

    \Statex
    \Function{\doupdate}{$\tidvar, k, v$}
        \label{line:function-doupdate}
      \State \var $l \gets \partitionofproc(k)$
        \label{line:doupdate-partitionof-k}
      \State $\wbuff[\tidvar][l][k] \gets v$
        \label{line:doupdate-wbuff}
      \State \strongcolor{$\rset[\tidvar] \gets \rset[\tidvar] \cup \set{k}$}
        \label{line:doupdate-readset}
    \EndFunction

    \Statex
    \Function{\commitcausal}{$\tidvar, \lccolor{\lcvar}$}
      \label{line:function-commitcausal}
      \State \var $L \gets \set{l \mid \wbuff[\tidvar][l] \neq \emptyset}$
      \If{$L = \emptyset$}
        \label{line:commitcausal-ro}
        \State \Return $\snapVC[\tidvar]$
        \label{line:commitcausal-return-ro}
        \label{line:commitcausal-commitvc-of-t-ro}
      \EndIf

      \hStatex
      \State \send $\prepare(\tidvar, \wbuff[\tidvar][l], \snapVC[\tidvar])
        \sendto p^{l}_{d}$, $l \in L$
        \label{line:commitcausal-call-prepare}
      \State \var $\commitvc \gets \snapVC[\tidvar]$
        \label{line:commitcausal-commitvc}
      \ForAll{$l \in L$}
        \State \wait\receive $\prepareack(\tidvar, \tsvar) \from p^{l}_{d}$
          \label{line:commitcausal-wait-prepareack}
        \State $\commitvc[d] \gets \max \set{\commitvc[d], \tsvar}$
          \label{line:commitcausal-commitvc-d}
      \EndFor
      \State \send $\Call{commit}{\tidvar, \commitvc, \lccolor{\lcvar}}
        \sendto p^{l}_{d}$, $l \in L$
      \label{line:commitcausal-call-commit}
      \State \Return $\commitvc$
        \label{line:commitcausal-return}
        \label{line:commitcausal-commitvc-of-t-rw}
    \EndFunction
  \end{algorithmic}
\end{algorithm*}

\begin{algorithm*}[t]
  \caption{Transaction execution at $p^m_d$}
  \label{alg:unistore-replica}
  \begin{algorithmic}[1]
    \WhenRcv[$\Call{\readkey}{\snapvc, k} \from p$]
      \label{line:function-readkey}
      \For{$i \in \D \setminus \set{d}$}
        \label{line:readkey-uniformvc-i}
        \State $\uniformVC[i] \gets \max\{\snapvc[i], \uniformVC[i]\}$
          \label{line:readkey-uniformvc}
      \EndFor
      \State {\bf wait until} $\knownVC[d] \geq \snapvc[d]
          \land \strongcolor{\knownVC[\strongentry] \ge \snapvc[\strongentry]}$
        \label{line:readkey-wait-util-knownvc}
      \hStatex
      \State $\langle v, \commitvc, \lccolor{\lcvar} \rangle \gets \snapshotproc(\oplog[k], \snapvc)$
        \label{line:readkey-read}
        \Comment{returns the last update to key $k$ by a transaction}
      \State  \Comment{
          with the highest Lamport timestamp such that $\commitvc \leq \snapvc$}
      \State \send $\versionproc(v, \lccolor{\lcvar}) \sendto p$
        \label{line:readkey-return}
    \EndWhenRcv

    \Statex
    \WhenRcv[$\Call{\prepare}{\tidvar, \wbuffvar, \snapvc} \from p$]
      \label{line:function-preparecausal}
      \For{$i \in \D \setminus \set{d}$}
        \State $\uniformVC[i] \gets \max\set{\snapvc[i], \uniformVC[i]}$
        \label{line:preparecausal-uniformvc}
      \EndFor

      \State \var $\tsvar \gets \clockVar$
        \label{line:preparecausal-ts}
      \State $\preparedcausal \gets \preparedcausal \cup
        \set{\langle \tidvar, \wbuffvar, \tsvar \rangle}$
        \label{line:preparecausal-preparedcausal}
      \State \send $\Call{\prepareack}{\tidvar, \tsvar} \sendto p$
        \label{line:preparecausal-call-preparecausalack}
    \EndWhenRcv

    \Statex
    \WhenRcv[$\Call{\commit}{\tidvar, \commitvc, \lccolor{\lcvar}}$]
      \label{line:function-commit}
      \State \wait\until $\clockVar \geq \commitvc[d]$
        \label{line:commit-wait-clock}

      \hStatex
      \State $\langle \tidvar, \wbuffvar, \_ \rangle \gets \find(\tidvar, \preparedcausal)$
        \label{line:commit-find}
      \State $\preparedcausal \gets \preparedcausal \setminus \set{\langle \tidvar, \_, \_\rangle}$
        \label{line:commit-preparedcausal}

      \ForAll{$\langle k, v \rangle \in \wbuffvar$}
        \State $\oplog[k] \gets \oplog[k] \cdot \langle v, \commitvc, \lccolor{\lcvar} \rangle$
        \label{line:commit-oplog}
      \EndFor

      \State $\committedcausal[d] \gets \committedcausal[d] \cup
        \set{\langle \tidvar, \wbuffvar, \commitvc, \lccolor{\lcvar} \rangle}$
        \label{line:commit-committedcausal}
    \EndWhenRcv

    \Statex
    \Function{\uniformbarrier}{$\vc$} \label{line:function-uniformbarrier}
      \State \wait\until $\uniformVC[d] \ge \vc[d]$
        \label{line:uniformbarrier-wait-uniformvc-d}
    \EndFunction

    \Statex
    \Function{\attach}{$\vc$}
      \label{line:function-attach}
      \State \wait\until $\forall i \in \D \setminus \set{d}.\; \uniformVC[i] \ge \vc[i]$
        \label{line:attach-wait-condition}
    \EndFunction
  \end{algorithmic}
\end{algorithm*}

\begin{algorithm*}[t]
  \caption{Transaction replication at $p^m_d$}
  \label{alg:unistore-replication}
  \begin{algorithmic}[1]
    \Function{\propagate}{\null} \Comment{Run periodically}
      \label{line:function-propagate}
      \If{$\preparedcausal = \emptyset$}
          \label{line:propagate-preparedcausal-empty}
        \State $\knownVC[d] \gets \clockVar$
          \label{line:propagate-knownvc-clock}
      \Else
        \State $\knownVC[d] \gets
          \min\set{\tsvar \mid \langle \_, \_, \tsvar \rangle \in \preparedcausal} - 1$
        \label{line:propagate-knownvc-ts}
      \EndIf

      \hStatex
      \State \var $\txsvar \gets \set{\langle \_, \_, \commitvc, \lccolor{\_}
        \rangle \in \committedcausal[d] \mid \commitvc[d] \le \knownVC[d]}$
        \label{line:propagate-txs}

      \If{$\txsvar \neq \emptyset$}
        \label{line:propagate-txs-nonempty}
        \State \send $\replicate(d, \txsvar) \sendto p^{m}_{i}$,
          $i \in \D \setminus \set{d}$
          \label{line:propagate-call-replicate}
        \State $\committedcausal[d] \gets \committedcausal[d] \setminus \txsvar$
          \label{line:propagate-committedblue}
      \Else
        \State \send $\heartbeat(d, \knownVC[d]) \sendto p^{m}_{i}$,
          $i \in \D \setminus \set{d}$
          \label{line:propagate-call-heartbeat}
      \EndIf
    \EndFunction

    \Statex
    \WhenRcv[\replicate($i, \txsvar$)]
      \label{line:function-replicate}
      \ForAll{$\langle \tidvar, \wbuffvar, \commitvc, \lccolor{\lcvar} \rangle \in \txsvar$
        in $\commitvc[i]$ order}
        \label{line:replicate-increasing-order}
        \If{$\commitvc[i] > \knownVC[i]$}
          \label{line:replicate-precondition}
          \ForAll{$\langle k, v \rangle \in \wbuffvar$}
            \State $\oplog[k] \gets \oplog[k] \cdot {\langle v, \commitvc, \lccolor{\lcvar} \rangle}$
            \label{line:replicate-oplog}
          \EndFor
          \State $\committedcausal[i] \gets \committedcausal[i] \cup
            \set{\langle \tidvar, \wbuffvar, \commitvc, \lccolor{\lcvar} \rangle}$
            \label{line:replicate-committedcausal}
          \State $\knownVC[i] \gets \commitvc[i]$
            \label{line:replicate-knownvc}
        \EndIf
      \EndFor
    \EndWhenRcv

    \Statex
    \WhenRcv[\heartbeat($i, \tsvar$)]
      \label{line:function-heartbeat}
      \State \pre $\tsvar > \knownVC[i]$
        \label{line:heartbeat-precondition}

      \hStatex
      \State $\knownVC[i] \gets \tsvar$
        \label{line:heartbeat-knownvc}
    \EndWhenRcv

    \Statex
    \Function{\forward}{$i, j$}
      \Comment{forward transactions received from data center $j \neq d$
        to data center $i \notin \set{d, j}$}
      \label{line:function-forward}
      \State \var $\txsvar \gets \set{\langle \tidvar, \_, \commitvc, \lccolor{\_} \rangle
        \in \committedcausal[j] \mid \commitvc[j] > \globalmatrix[i][j]}$
        \label{line:forward-txs}
      \If{$\txsvar \neq \emptyset$}
        \label{line:forward-txs-nonempty}
        \State \send $\replicate(j, \txsvar) \sendto p^m_i$
          \label{line:forward-call-replicate}
      \Else
        \State \send $\heartbeat(j, \knownVC[j]) \sendto p^m_i$
          \label{line:forward-call-heartbeat}
      \EndIf
    \EndFunction
  \end{algorithmic}
\end{algorithm*}
\clearpage

\begin{algorithm*}[t]
  \caption{Updating metadata at $p^m_d$}
  \label{alg:unistore-clock}
  \begin{algorithmic}[1]
    \Function{\bcast}{\null} \Comment{Run periodically}
      \label{line:function-bcast}
      \State \send $\Call{\knownvclocal}{m, \knownVC} \sendto p^{l}_{d}$, $l \in \P$
        \label{line:bcast-call-knownvclocal}
      \State \send $\Call{\stablevcproc}{d, \stableVC} \sendto p^{m}_{i}$, $i \in \D$
        \label{line:bcast-call-stablevc}
      \State \send $\Call{\knownvcglobal}{d, \knownVC} \sendto p^{m}_{i}$, $i \in \D$
        \label{line:bcast-call-knownvcglobal}
    \EndFunction

    \Statex
    \WhenRcv[\knownvclocal($l, \knownvc$)]
      \label{line:function-knownvclocal}
      \State $\localmatrix[l] \gets \knownvc$
        \label{line:knownvclocal-localknownmatrix}
      \For{$i \in \D$}
        \State $\stableVC[i] \gets \min \set{\localmatrix[n][i] \mid n \in \P}$
        \label{line:knownvclocal-stablevc-causal}
      \EndFor
      \State \strongcolor{$\stableVC[\strongentry] \gets
        \min\set{\localmatrix[n][\strongentry] \mid n \in \P}$}
        \label{line:knownvclocal-stablevc-strong}
    \EndWhenRcv

    \Statex
    \WhenRcv[\stablevcproc($i, \stablevc$)]
      \label{line:function-stablevc}
      \State $\stablematrix[i] \gets \stablevc$
        \label{line:stablevc-stablematrix}
      \State $G\gets$ all groups with $f+1$ replicas that include $p^{m}_{d}$
        \label{line:stablevc-g}
      \For{$j \in \D$}
        \State \var $\tsvar \gets \max \set{\min \set{\stablematrix[h][j] \mid h \in g}
          \mid g \in G}$
          \label{line:stablevc-ts}
        \State $\uniformVC[j] \gets \max \set{\uniformVC[j], \tsvar}$
          \label{line:stablevc-uniformvc}
      \EndFor
    \EndWhenRcv

    \Statex
    \WhenRcv[\knownvcglobal($l, \knownvc$)]
      \label{line:function-knownvcglobal}
      \State $\globalmatrix[l] \gets \knownvc$
        \label{line:knownvcglobal-globalmatrix}
    \EndWhenRcv
  \end{algorithmic}
\end{algorithm*}

\begin{algorithm*}[t]
  \caption{Committing strong transactions at $p^{m}_{d}$}
  \label{alg:unistore-strong-commit}
  \begin{algorithmic}[1]
    \Function{\commitstrong}{$\tidvar, \lccolor{\lcvar}$}
      \label{line:function-commitstrong}
      \State $\uniformbarrier(\snapVC[\tidvar])$
        \label{line:commitstrong-call-uniformbarrier}
      \State $\langle \decvar, \vcvar, \lcvar \rangle \gets
        \Call{\certify}{\normalMode,
          \tidvar, \wbuff[\tidvar], \rset[\tidvar], \snapVC[\tidvar], \lccolor{\lcvar}}$
        \label{line:commitstrong-call-certify}
      \State \Return $\langle \decvar, \vcvar, \lccolor{\lcvar} \rangle$
        \label{line:commitstrong-return}
        \label{line:commitstrong-commitvc-of-t-strong}
    \EndFunction

    \Statex
    \Upon[\deliverupdates($\txsvar$)]
      \label{line:function-deliverupdates}
      \For{$\langle \_, \wbuffvar, \commitvc, \lccolor{\lcvar} \rangle \in \txsvar$ in $\commitvc[\strongentry]$ order}
        \label{line:deliverupdates-foreach-wbuff}
          \For{$\langle k, v \rangle \in \wbuffvar$}
            \label{line:deliverupdates-foreach-kv}
            \State $\oplog[k] \gets \oplog[k] \cdot \langle v, \commitvc, \lccolor{\lcvar} \rangle$
              \label{line:deliverupdates-oplog}
          \EndFor
        \State $\knownVC[\strongentry] \gets \commitvc[\strongentry]$
          \label{line:deliverupdates-knownvc-strongentry}
      \EndFor
    \EndUpon

    \Statex
    \Function{\heartbeatstrong}{\null} \Comment{Run periodically}
      \label{line:function-heartbeatstrong}
      \State \Return $\Call{\certify}{\normalMode, \bot, \emptyset, \emptyset, \vec{0}, \lccolor{\bot}}$
        \label{line:heartbeatstrong-call-certify}
    \EndFunction
  \end{algorithmic}
\end{algorithm*}

\begin{algorithm*}[t]
  \caption{Certification service at coordinator $\pvar^{m}_{d}$}
  \label{alg:unistore-certify}
  \begin{algorithmic}[1]
    \Function{\certify}{$\callerMode, \tidvar, \wbuffvar, \rsetvar, \snapvc, \lccolor{\lcvar}$}
      \label{line:function-certify}
      \State \var $\reqIdVar \gets \generateReqId()$
      \State \var $L \gets \set{l \mid \wbuffvar[l] \neq \emptyset}
        \cup \set{\partitionofproc(k) \mid k \in \rsetvar}$
        \label{line:certify-L}
      \Repeat
        \State \send
          $\Call{\preparestrong}{\reqIdVar, \callerMode,
            \tidvar, \wbuffvar, \rsetvar, \snapvc, \lccolor{\lcvar}}
          \sendto \Omega_{l}$, $l \in L$
          \label{line:certify-call-preparestrong}
        \State \asyncwait \receive $\Call{\alreadydecided}{\tidvar, \decisionvar,
          \commitvc, \lccolor{\lcvar}}$
          \label{line:certify-wait-alreadydecided}
        \Statex \hspace{1.95cm} $\lor$ \receive $\Call{\unknowntxAck}{l, \reqIdVar, \tidvar}$
          \textbf{from a quorum from some} $l \in L$
        \Statex \hspace{1.95cm} $\lor$ \receive
          $\Call{\acceptack}{l, \ballotvar_{l}, \tidvar, \votevar_{l}, \tsvar_{l}, \lccolor{\lcvar_{l}}}$
            \textbf{from a quorum for all} $l \in L$
          \label{line:certify-wait-acceptack}
      \Until{\notkw\timeout}

      \hStatex
      \If{\received \textbf{a quorum of} $\Call{\unknowntxAck}{l, \reqIdVar, \tidvar}$}
        \State \Return $\unknowntx$
      \ElsIf{\received $\Call{\alreadydecided}{\tidvar, \decisionvar, \commitvc, \lccolor{\lcvar}}$}
        \label{line:certify-received-alreadydecided}
        \State \send $\Call{\decision}{\ballotVar, \tidvar, \decisionvar, \commitvc, \lccolor{\lcvar}}
          \sendto \Omega_{\pvar}$
        \State \Return $\langle \decisionvar, \commitvc, \lccolor{\lcvar} \rangle$
          \label{line:certify-return-alreadydecided}
      \Else
        \State $\commitvc \gets \snapvc$
          \label{line:certify-commitvc}
        \State $\commitvc[\strongentry] \gets \max \set{\tsvar_{l} \mid l \in L}$
          \label{line:certify-commitvc-strongentry}
        \IfThenElse{$\exists l \in L.\; \votevar_{l} = \abort$}
          {$\decisionvar \gets \abort$ \label{line:certify-decision-abort}}
          {$\decisionvar \gets \commit$ \label{line:certify-decision-commit}}
        \State \lccolor{$\lcvar \gets \max\set{\lcvar_{l} \mid l \in L}$}
          \label{line:certify-lc}
        \State \send $\Call{\decision}{\ballotvar_{l}, \tidvar, \decisionvar, \commitvc, \lccolor{\lcvar}}
          \sendto \Omega_{l}$, $l \in L$
          \label{line:certify-call-desision}
        \State \Return $\langle \decisionvar, \commitvc, \lccolor{\lcvar} \rangle$
          \label{line:certify-return-acceptack}
      \EndIf
    \EndFunction
  \end{algorithmic}
\end{algorithm*}

\begin{algorithm*}[t]
  \caption{Strong transaction certification at $p^{m}_{d}$}
  \label{alg:unistore-certification}
  \begin{algorithmic}[1]
    \Function{\certification}{$W, \rsetvar, \snapvc, \lccolor{\lcvar}$}
      \label{function:certification}
      \ForAll{$\langle \_, \wbuffvar', \rsetvar', \_, \commit, \_, \lccolor{\_} \rangle \in \preparedstrong$}
        \label{line:certification-preparedstrong}
        \If{$(\exists \langle k, \_ \rangle \in \wbuffvar'[m].\; k \in \rsetvar) \lor
             (\exists k \in \rsetvar'.\; \langle k, \_ \rangle \in W)$}
          \label{line:certification-conflict}
          \State \Return $\langle \abort, \lccolor{\bot} \rangle$
            \label{line:certification-preparedstrong-abort}
        \EndIf
      \EndFor

      \hStatex
      \ForAll{$\langle \_, \wbuffvar', \commit, \commitvc, \lccolor{\lcvar'} \rangle \in \decidedstrong$}
        \label{line:certification-decidedstrong}
        \If{$(\exists \langle k, \_ \rangle \in \wbuffvar'[m].\; k \in \rsetvar)
          \land \lnot (\commitvc \leq \snapvc)$}
          \State \Return $\langle \abort, \lccolor{\bot} \rangle$
            \label{line:certification-decidedstrong-abort}
        \EndIf
        \If{\lccolor{$\lcvar \le \lcvar'$}}
          \State \lccolor{$\lcvar \gets \lcvar' + 1$}
          \label{line:certification-lc}
        \EndIf
      \EndFor
      \State \Return $\langle \commit, \lccolor{\lcvar} \rangle$
        \label{line:certification-return}
    \EndFunction
  \end{algorithmic}
\end{algorithm*}

\begin{algorithm*}[t]
  \caption{Atomic transaction commit protocol at $p^{m}_{d}$}
  \label{alg:unistore-atomic-commit}
  \begin{algorithmic}[1]
    \WhenRcv[$\preparestrong(\reqIdVar, \senderMode, \tidvar, \wbuffvar, \rsetvar, \snapvc, \lccolor{\lcvar}) \from \pvar$]
      \label{line:function-preparestrong}
      \State \pre $\statusVar \in \set{\leaderproc, \restoring}$
        \label{line:preparestrong-precondition}
      \hStatex
      \If{$\exists \langle \tidvar, \_, \decisionvar, \commitvc, \lccolor{\lcvar} \rangle \in \decidedstrong$}
        \label{line:preparestrong-case-alreadydecided}
        \State \send $\Call{\alreadydecided}{\tidvar, \decisionvar, \commitvc, \lccolor{\lcvar}} \sendto p$
          \label{line:preparestrong-call-alreadydecided}
      \ElsIf{$\exists \langle \tidvar, \_, \_, \snapvc, \votevar, \tsvar, \lccolor{\lcvar} \rangle \in \preparedstrong$}
        \State \send $\Call{\accept}{\ballotVar, \tidvar, \wbuffvar, \rsetvar, \snapvc, \votevar, \tsvar, \pvar, \lccolor{\lcvar}}
          \sendto \replicas(\mvar)$
      \ElsIf{$\senderMode = \restoring$}
        \State \send $\Call{\unknownTx}{\ballotVar, \reqIdVar, \tidvar, \pvar} \sendto \replicas(\mvar)$
      \ElsIf{$\statusVar = \leaderproc$}
        \State $\wait\until \clockVar > \snapvc[\strongentry]$
          \label{line:preparestrong-clock}
        \State $\tsvar \gets \clockVar$
          \label{line:preaprestrong-ts}
        \State $\langle \votevar, \lccolor{\lcvar} \rangle \gets
          \Call{\certification}{\wbuffvar[m], \rsetvar, \snapvc, \lccolor{\lcvar}}$
          \label{line:preparestrong-call-certification}
        \State \send $\Call{\accept}{\ballotVar, \tidvar, \wbuffvar, \rsetvar, \snapvc,
          \votevar, \tsvar, \pvar, \lccolor{\lcvar}} \sendto \replicas(\mvar)$
          \label{line:preparestrong-call-accept}
      \EndIf
    \EndWhenRcv

    \Statex
    \WhenRcv[$\accept(\ballotvar, \tidvar, \wbuffvar, \rsetvar,
      \snapvc, \votevar, \tsvar, \pvar, \lccolor{\lcvar})$]
      \label{line:function-accept}
      \State \pre $\statusVar \in \set{\leaderproc, \follower, \restoring} \land \ballotVar = b$
        \label{line:accept-precondition}

      \hStatex
      \State $\preparedstrong \gets \preparedstrong \cup
        \set{\langle \tidvar, \wbuffvar, \rsetvar, \snapvc, \votevar, \tsvar, \lccolor{\lcvar} \rangle}$
        \label{line:accept-preparedstrong}
      \State \send $\Call{\acceptack}{\mvar, \ballotvar, \tidvar, \votevar, \tsvar, \lccolor{\lcvar}}
        \sendto p$
        \label{line:accept-call-acceptack}
    \EndWhenRcv

    \Statex
    \WhenRcv[$\decision(\ballotvar, \tidvar, \decisionvar, \commitvc, \lccolor{\lcvar})$]
      \label{line:function-decision}
      \State \pre $\statusVar \in \set{\leaderproc, \restoring} \land \ballotVar = b$
        \label{line:decision-precondition}

      \hStatex
      \State \wait\until $\clockVar \ge \commitvc[\strongentry]$
        \label{line:decision-wait-clock}
      \State \send $\Call{\learndecision}{\ballotvar, \tidvar, \decisionvar, \commitvc, \lccolor{\lcvar}}
        \sendto \replicas(\mvar)$
        \label{line:decision-call-desision}
    \EndWhenRcv

    \Statex
    \WhenRcv[$\learndecision(\ballotvar, \tidvar, \decisionvar, \commitvc, \lccolor{\lcvar})$]
      \label{function:learndecision}
      \State \pre $\statusVar \in \set{\leaderproc, \follower, \restoring} \land \ballotVar = b
        \land \exists \langle \tidvar, \wbuffvar, \_, \_, \_, \_, \lccolor{\_} \rangle \in \preparedstrong$
        \label{line:learndecision-precondition}

      \hStatex
      \State $\preparedstrong \gets \preparedstrong \setminus
        \set{\langle \tidvar, \_, \_, \_, \_, \_, \lccolor{\_} \rangle}$
        \label{line:decision-preparedstrong}
      \State $\decidedstrong \gets \decidedstrong \cup
        \set{\langle \tidvar, \wbuffvar, \decisionvar, \commitvc, \lccolor{\lcvar} \rangle}$
        \label{line:decision-decidedstrong}
    \EndWhenRcv

    \Statex \Upon[] \label{function:upcall} \label{line:function-upcall}
      \Statex \vspace{-0.70cm}
        \begin{align*}
          \exists &\langle \_, \_, \commit, \commitvc, \lccolor{\_} \rangle \in \decidedstrong. \\
            &\phantom{\land\;} \commitvc[\strongentry] > \lastdeliveredVar \\
            &\land (\lnot\exists \langle \_, \_, \_, \_, \commit, \tsvar, \lccolor{\_} \rangle \in \preparedstrong.\
            \lastdeliveredVar < \tsvar \le \commitvc[\strongentry]) \\
            &\land (\lnot\exists \langle \_, \_, \commit, \commitvc', \lccolor{\_} \rangle \in \decidedstrong.\
            \lastdeliveredVar < \commitvc'[\strongentry] < \commitvc[\strongentry])
        \end{align*}
        \label{line:upcall-upon}
      \Statex \vspace{-1.10cm}
      \State \pre $\statusVar = \leaderproc$
        \label{line:upcall-precondition}

      \hStatex
      \State $\send \deliver(\ballotVar, \commitvc[\strongentry]) \sendto \replicas(\mvar)$
    \EndUpon

    \Statex
    \WhenRcv[$\deliver(\ballotvar, \tsvar)$]
      \label{function:deliver}
      \State \pre $\statusVar \in \set{\leaderproc, \follower}
        \land \ballotVar = \ballotvar \land \lastdeliveredVar < \tsvar$
        \label{line:deliver-precondition}

      \State $\lastdeliveredVar \gets \tsvar$
        \label{line:deliver-lastdeliverd}
      \State \var $W \gets \set{\langle \tidvar, \wbuffvar[m], \commitvc, \lccolor{\lcvar} \rangle \mid
        \exists \langle \tidvar, \wbuffvar, \commit, \commitvc, \lccolor{\lcvar}
        \rangle \in\decidedstrong.\;$
        \Statex \hspace{6.22cm} $\commitvc[\strongentry] = \tsvar}$
        \label{deliver-W}
      \State \upcall $\Call{\deliverupdates}{W} \sendto p^{\mvar}_{\dvar}$
        \label{line:deliver-call-deliverupdates}
    \EndWhenRcv

    \Statex
    \WhenRcv[$\unknownTx(\ballotvar, \reqIdVar, \tidvar, \pvar)$]
      \State \pre $\statusVar \in \set{\leaderproc, \follower, \restoring}
        \land \ballotVar = \ballotvar$
      \State $\send \unknowntxAck(\mvar, \reqIdVar, \tidvar) \sendto \pvar$
    \EndWhenRcv

    \Statex
    \Function{\retry}{$\tidvar$}
      \Comment{Run periodically}
      \label{line:function-retry}
      \State \pre $\Call{\certify}{\_, \tidvar, \_, \_, \_, \lccolor{\_}}\ \text{is not executing}
        \land\statusVar = \leaderproc \land \exists
        \langle \tidvar, \wbuffvar, \rsvar, \snapvc, \_, \_, \lccolor{\lcvar} \rangle
        \in \preparedstrong$\!\!
        \label{line:retry-precondition}
      \State $\Call{\certify}{\normalMode, \tidvar, \wbuffvar, \rsvar, \snapvc, \lccolor{\lcvar}}$
    \EndFunction
  \end{algorithmic}
\end{algorithm*}


\begin{algorithm*}[t]
  \caption{Atomic transaction commit protocol at $\pvar^{m}_{d}$: recovery}
  \label{alg:unistore-recovery}
  \begin{algorithmic}[1]
    \Upon[$\Omega_{m} \neq \trustedVar$]
      \State $\trustedVar \gets \Omega_{m}$
      \If{$\trustedVar = \pvar^{m}_{d}$}
        \recover()
      \Else{}
        $\send \nack(\ballotVar) \sendto \trustedVar$
      \EndIf
    \EndUpon

    \Statex
    \WhenRcv[\nack($b$)]
      \State \pre $\trustedVar = \pvar^{m}_{d} \land \ballotvar > \ballotVar$
      \hStatex
      \State $\ballotVar \gets \ballotvar$
      \State \recover()
    \EndWhenRcv

    \Statex
    \Function{\recover}{\null}
      \label{line:function-recover}
      \State \send $\Call{\newleader}{\text{any ballot } \ballotvar
        \text{ such that } \ballotvar > \ballotVar
          \land \leaderof(\ballotvar) = \pvar^{\mvar}_{\dvar}}
          \sendto \replicas(\mvar)$
    \EndFunction

    \Statex
    \WhenRcv[$\newleader(\ballotvar) \from \pvar$]
      \label{line:function-newleader}
      \If{$\trustedVar = \pvar \land \ballotVar < \ballotvar$}
        \label{line:newleader-if}
        \State $\statusVar \gets \recovering$
          \label{line:newleader-status}
        \State $\ballotVar \gets \ballotvar$
          \label{line:newleader-ballot}
        \State $\doNotWaitFor \gets \emptyset$
        \State $\send \Call{\newleaderack}{\ballotVar, \cballot,
          \preparedstrong, \decidedstrong} \sendto \pvar$
          \label{line:newleader-call-newleaderack}
      \Else{}
        $\send \Call{\nack}{\ballotVar} \sendto \pvar$
      \EndIf
    \EndWhenRcv

    \Statex
    \WhenRcv[$\set{\newleaderack(\ballotvar, \cballotvar_{j},
      \preparedstrongvar_{j}, \decidedstrongvar_{j}) \mid p_{j} \in Q}
        \;\text{\bf from a quorum}\; Q$]
        \label{function:newleaderack}
      \State \pre $\statusVar = \recovering \land \ballotVar = \ballotvar$
        \label{line:newleaderack-precondition}
      \hStatex
      \State \var $J \gets$ the set of $j$ with maximal $\cballotvar_{j}$
        \label{line:newleaderack-J}
      \State $\decidedstrong \gets \bigcup\limits_{j \in J} \decidedstrongvar_{j}$
        \label{line:newleaderack-decidedstrong}
      \State $\preparedstrong \gets \set{\langle \tidvar, \_, \_, \_, \_, \_, \lccolor{\_} \rangle
        \in \bigcup\limits_{j \in J} \preparedstrongvar_{j} \mid
          \langle \tidvar, \_, \_, \_, \lccolor{\_} \rangle \notin \decidedstrong}$
        \label{line:newleaderack-preparedstrong}
      \State \var $\maxPrep \gets \max\set{\tsvar \mid
        \langle \_, \_, \_, \_, \_, \tsvar, \lccolor{\_} \rangle \in \preparedstrong}$
        \label{line:newleaderack-maxprep}
      \State \var $\maxDec \gets \max\set{\commitvc[\strongentry] \mid
        \langle \_, \_, \_, \commitvc, \lccolor{\_} \rangle \in \decidedstrong}$
        \label{line:newleaderack-maxdec}
      \State \wait\until $\clockVar \ge \max\set{\maxPrep, \maxDec}$
        \label{line:newleaderack-wait-clock}

      \hStatex
      \State $\cballot \gets \ballotvar$
        \label{line:newleaderack-cballot}
      \State \send $\newstate(\ballotVar, \preparedstrong, \decidedstrong)
        \sendto \replicas(\mvar) \setminus \set{\pvar^{\mvar}_{\dvar}}$
        \label{line:newleaderack-call-newstate}
    \EndWhenRcv

    \Statex
    \WhenRcv[\newstate($\ballotvar, \preparedstrongvar, \decidedstrongvar$) \from $\pvar$]
      \label{line:function-newstate}
      \State \pre $\statusVar = \recovering \land \ballotvar \ge \ballotVar$
        \label{line:newstate-precondition}
      \hStatex
      \State $\cballot \gets \ballotvar$
        \label{line:newstate-cballot}
      \State $\preparedstrong \gets \preparedstrongvar$
        \label{line:newstate-preparedstrong}
      \State $\decidedstrong \gets \decidedstrongvar$
        \label{line:newstate-decidedstrong}
      \State $\statusVar \gets \follower$
        \label{line:newstate-status}
      \State $\send \Call{\newstateack}{\ballotvar} \sendto \pvar$
        \label{line:newstate-call-newstateack}
    \EndWhenRcv

    \Statex
    \WhenRcv[\newstateack($\ballotvar$)] {\bf from a set of processes}
      \textbf{that together with} $\pvar^{\mvar}_{\dvar}$ \textbf{form a quorum}
      \State \pre $\statusVar = \recovering \land \ballotVar = \ballotvar$
      \hStatex
      \State $\statusVar \gets \restoring$
      \ForAll{$t = \langle \tidvar, \wbuffvar, \rsvar, \snapvc, \_, \_, \lccolor{\lcvar} \rangle
        \in \preparedstrong$}
        \If{$\Call{\certify}{\restoring, \tidvar, \wbuffvar, \rsetvar, \snapvc, \lccolor{\lcvar}} = \unknowntx$}
          \State $\doNotWaitFor \gets \doNotWaitFor \cup \set{t}$
        \EndIf
    \EndFor
    \EndWhenRcv

    \Statex
    \Upon[$\preparedstrong \subseteq \doNotWaitFor \land \statusVar = \restoring$]
      \State $\statusVar \gets \leaderproc$
      \State $\doNotWaitFor \gets \emptyset$
    \EndUpon
  \end{algorithmic}
\end{algorithm*}


\clearpage

\section{Consistency Model Specification} \label{section:spec}

\subsection{Relations} \label{ss:relations}

For a binary relation $\relation \subseteq A \times A$
and an element $a \in A$, we define
$\relation^{-1}(a) = \set{b \mid (b, a) \in \relation}$.
For a non-empty set $A$ and a total order $\relation \subseteq A \times A$,
we let $\max\limits_{\relation}(A)$ be the maximum element in $A$
according to $\relation$. Formally,
\[
  \max_{\relation}(A) = a \iff A \neq \emptyset
    \land \forall b \in A.\; a = b \lor (b, a) \in \relation.
\]
If $A$ is empty, then $\max\limits_{\relation}(A)$ is undefined.
We implicitly assume that $\max\limits_{\relation}(A)$ is defined
whenever it is used.

We call a binary relation a \emph{(strict) partial order}
if it is irreflexive and transitive.
We call it a \emph{total order} if it additionally
relates every two distinct elements one way or another.
\subsection{Operations and Events} \label{ss:operations}

Transactions in \unistore{} can start, read and write keys, and commit.
We assume that each transaction is associated with a unique transaction identifier
$\tidvar$ from a set $\tids$ (corresponding to
line~\code{\ref{alg:unistore-coord}}{\ref{line:start-return}}).
Besides, clients can issue on-demand barriers and migrate between data centers.

Let $\Key$ and $\Val$ be the set of keys and values, respectively.
We define $\OP$ as the set of all possible operations
\begin{align*}
  \OP = \;&\set{\start(\tidvar) \mid \tidvar \in \tids} \;\cup \\
    &\set{\commitcausaltx(\tidvar) \mid \tidvar \in \tids} \;\cup \\
    &\set{\commitstrongtx(\tidvar, \decvar) \mid \\
      &\quad \tidvar \in \tids, \decvar \in \set{\commit, \abort}} \;\cup \\
    &\set{\fence} \;\cup \\
    &\set{\clattach(j) \mid j \in \D} \;\cup \\
    &\set{\read(\tidvar, k, v), \updateproc(\tidvar, k, v) \mid \\
      &\quad \tidvar \in \tids, k \in \Key, v \in \Val}.
\end{align*}
We denote each invocation of such an operation by an \emph{event}
from a set $E$, usually ranged over by $e$.
A function $\opfunc : E \to \OP$ determines the operation a given event denotes.
Formally, we use the following notation to denote different types of events.
\begin{itemize}
  \item $E$: The set of all events.
  \item $S$: The set of \start{} events. That is,
    \[
      S = \set{e \in E \mid \exists \tidvar \in \tids.\;
        \opfunc(e) = \start(\tidvar)}.
    \]
  \item $R$: The set of \read{} (read) events. That is,
    \begin{align*}
      R = \set{&e \in E \mid \exists \tidvar \in \tids, k \in \Key, v \in \Val.\; \\
        &\quad \opfunc(e) = \read(\tidvar, k, v)}.
    \end{align*}
  \item $U$: The set of \updateproc{} (update) events. That is,
    \begin{align*}
      U = \set{&e \in E \mid \exists \tidvar \in \tids, k \in \Key, v \in \Val.\; \\
        &\quad \opfunc(e) = \updateproc(\tidvar, k, v)}.
    \end{align*}
  \item $C_{\causalentry}$: The set of \commitcausaltx{} events. That is,
    \begin{align*}
      C_{\causalentry} = \set{&e \in E \mid \exists \tidvar \in \tids.\; \\
        &\quad \opfunc(e) = \commitcausaltx(\tidvar)}.
    \end{align*}
  \item $C_{\strongentry}$: The set of \commitstrongtx{} events
    with decision $\decvar = \commit$. That is,
    \begin{align*}
      C_{\strongentry} &= \set{e \in E \mid \exists \tidvar \in \tids.\; \\
        &\opfunc(e) = \commitstrongtx(\tidvar, \commit)}.
    \end{align*}

  \item $C \triangleq C_{\causalentry} \uplus C_{\strongentry}$:
    The set of all commit events.
  \item $\Fence$: The set of \fence{} events. That is,
    \[
      \Fence = \set{e \in E \mid \opfunc(e) = \fence}.
    \]
  \item $\Attach$: The set of \clattach{} events. That is,
    \[
      \Attach = \set{e \in E \mid \exists j \in \D.\; \opfunc(e) = \clattach(j)}.
    \]
  \item $R_{k}$: The set of read events on key $k$. That is,
    \begin{align*}
      R_{k} = \set{&e \in E \mid \exists \tidvar \in \tids, v \in Val.\; \\
        &\quad \opfunc(e) = \read(\tidvar, k, v)}.
    \end{align*}
  \item $U_{k}$: The set of update events on key $k$. That is,
    \begin{align*}
      U_{k} = \set{&e \in E \mid \exists \tidvar \in \tids, v \in Val.\; \\
        &\quad \opfunc(e) = \updateproc(\tidvar, k, v)}.
    \end{align*}
\end{itemize}

For different types of events, we define
\begin{itemize}
  \item $\key(e)$: The key that the read or update event
    $e \in R \cup U$ accesses.
  \item $\rval(e)$: The return value of the read event $e \in R$.
  \item $\uval(e)$: The value written by the update event $e \in U$.
\end{itemize}
\subsection{Transactions} \label{ss:transactions}

\begin{appdefinition}[Transactions] \label{def:tx}
  A transaction $\tvar$ is a triple $(\tidvar, \txnevents, \po)$, where
  \begin{itemize}
    \item $\tidvar \in \tids$ is a unique transaction identifier;
    \item $\txnevents \subseteq E \setminus (\Fence \cup \Attach)$
      is a finite, non-empty set of events;
    \item $\po \subseteq \txnevents \times \txnevents$
      is the program order, which is total.
  \end{itemize}
  We only consider well-formed transactions: according to the $\po$ order,
  $\tvar$ starts with a \start{} event, then performs some number of
  \read{}/\updateproc{} events, and ends with a commit event (\commitcausaltx{}
  or \commitstrongtx).
\end{appdefinition}

In the following, we denote components of $t$ as in $t.\tidvar$.
For simplicity, we assume a dedicated \emph{initial} transaction $\tvar_{0}$
which installs initial values to all possible keys before the system launches.

We use the following notations to denote different types of transactions.
\begin{itemize}
  \item $\txs$: The set of all committed transactions.
  \item $\txs_{k}$: The set of committed transactions that update key $k$.
    We also use $C_{k}$ to denote the set of commit events of transactions
    in $\txs_{k}$.
  \item $\causaltxs$: The set of transactions that
    end with the \commitcausaltx{} events.
    We call them causal transactions.
    Causal transactions will always be committed.
  \item $\allstrongtxs$: The set of transactions that
    end with the \commitstrongtx{} events.
    We call them strong transactions.
    Strong transactions can be committed or aborted.
  \item $\strongtxs$: The set of \emph{committed} strong transactions.
\end{itemize}

We have $\txs = \causaltxs \uplus \strongtxs$.
For each transaction $\tvar \in \txs$, we define
\begin{itemize}
  \item $\tidselector(\tvar) \in \tids$:
    The transaction identifier $t.\tidvar$ of $\tvar$.
  \item $\events(\tvar) \subseteq S \cup R \cup U \cup C$:
    The set $t.\txnevents$ of events in $\tvar$.
  \item $\ws(\tvar) \subseteq \Key \times \Val$:
    The write set of $\tvar$.
    It is the set of keys with their values that $\tvar$ updates,
    which contains at most one value per key.
    Formally,
    \[
      \ws(\tvar) \triangleq \set{(\key(e), \uval(e)) \mid e \in t.\txnevents \cap U}.
    \]
  \item $\rs(\tvar) \subseteq \Key$: The read set of $\tvar$.
    It is the set of keys that $\tvar$ reads.
    Formally,
    \[
      \rs(\tvar) \triangleq \set{\key(e) \mid e \in t.\txnevents \cap R}.
    \]
  \item $\startoftx(\tvar) \in S$:
    The \start{} event of $\tvar$.
    Formally, it is the unique event in the set $t.\txnevents \cap S$.
  \item $\commitoftx(\tvar) \in C$: The commit event of $\tvar$.
    Formally, it is the unique event in the set $t.\txnevents \cap C$.
  \item $\ud(\tvar, k) \in U_{k}$: The \emph{last} update event on key $k$, if any,
    in transaction $\tvar$. Formally,
    \[
      \ud(\tvar, k) \triangleq \max\limits_{\po}(\events(\tvar) \cap U_{k}).
    \]
\end{itemize}

Besides, we define
\begin{align}
  \W(\tvar) &\triangleq \set{k \in \Key \mid \langle k, \_ \rangle \in \ws(\tvar)}, \label{eq:W-def}\\
  \R(\tvar) &\triangleq \rs(\tvar) \cup \W(\tvar).\label{eq:R-def}
\end{align}

For a read event $e$ on key $k$ in transaction $\tvar$,
if there exist update events on $k$ preceding $e$ in $\tvar$,
then $e$ is called an \emph{internal} read event.
Otherwise, $e$ is called an \emph{external} read event.
We denote the sets of internal reads and external reads
by $\intread$ and $\extread$, respectively.
That is, $R = \intread \uplus \extread$.

We also distinguish commit events
for read-only transactions from those for update transactions,
and denote their sets by $\rocommit$ and $\updatecommit$, respectively.
That is, $C = \rocommit \uplus \updatecommit$.

For notational convenience,
for an event $e \in E \setminus (\Fence \cup \Attach)$, we also define
$\txfunc(e)$ to be the transaction containing $e$ and
\begin{align*}
  \startoftx(e) &\triangleq \startoftx(\txfunc(e)), \\
  \commitoftx(e) &\triangleq \commitoftx(\txfunc(e)).
\end{align*}
\subsection{Abstract Executions} \label{ss:cm}

Clients interacts with \unistore{} by issuing transactions
and \fence{} and \attach{} events.
We use histories to record such interactions in a single computation.
Note that histories only record committed transactions.
\begin{appdefinition}[Histories] \label{def:histories}
  A \emph{history} is a tuple
  \[
    H = (X, \client, \dc, \so)
  \]
  such that
  \begin{itemize}
    \item $X \subseteq \txs \cup \Fence \cup \Attach$
      is a set of committed transactions and \fence{} and \attach{} events;
    \item $\client: X \to \clients$ is a function that returns
      \begin{itemize}
        \item the client $\client(t)$ which issues the transaction
          $\tvar \in (X \cap \txs)$,
        \item the client $\client(\fencerange)$ which issues the \fence{} event
          $\fencerange \in (X \cap \Fence)$, or
        \item the client $\client(\attachrange)$ which issues the \attach{} event
          $\attachrange \in (X \cap \Attach)$;
      \end{itemize}
    \item $\dc: X \to \D$ is a function
      that returns the original data center $\dc(\tvar)$
      of transaction $\tvar \in (X \cap \txs)$,
      $\dc(\fencerange)$ of \fence{} event $\fencerange \in (X \cap \Fence)$,
      or $\dc(\attachrange)$ of \attach{} event $\attachrange \in (X \cap \Attach)$;
    \item $\so \subseteq X \times X$ is the \emph{session order} on $X$.
      Consider $\txevents_{1}, \txevents_{2} \in X$.
      We say that $\txevents_{1}$ precedes $\txevents_{2}$
      in the session order, denoted $\txevents_{1} \rel{\so} \txevents_{2}$,
      if they are executed by the same client
      and $\txevents_{1}$ is executed before $\txevents_{2}$.
  \end{itemize}
\end{appdefinition}
In the following, we denote components of $H$ as in $H.X$
and often shorten $H.X$ by $X$ when it is clear.
Let $V_{H} \triangleq \bigcup (H.X \cap T).Y$
be the set of transactional events in history $H$.

A consistency model is specified by a set of histories.
To define this set, we extend histories with two relations,
declaratively describing how the system processes transactions
and \fence{} events.

\begin{appdefinition}[Abstract Executions] \label{def:ae}
  An \emph{abstract execution} is a triple
  \[
    A = ((X, \client, \dc, \so), \vis, \ar)
  \]
  such that
  \begin{itemize}
    \item $(X, \client, \dc, \so)$ is a history;
    \item Visibility $\vis \subseteq X \times X$ is a partial order;
    \item Arbitration $\ar \subseteq X \times X$ is a total order.
  \end{itemize}
\end{appdefinition}
For $H = (X, \client, \dc, \so)$,
we often shorten $((X, \client, \dc, \so), \vis, \ar)$ by $(H, \vis, \ar)$.
\subsection{Partial Order-Restrictions Consistency} \label{ss:por}

We aim to show that \unistore{} implements a transactional variant
of \emph{Partial Order-Restrictions consistency
(\por{} consistency)}~\cite{red-blue, por} for LWW registers.
A history $H$ of \unistore{} satisfies \por, denoted $H \models \por$,
if it can be extended to an abstract execution that satisfies several axioms,
defined in the following:
\begin{align*}
  H \models \por \iff\; &\exists \vis, \ar.\; (H, \vis, \ar) \models \\
                 &\qquad \retval \;\land \\
                 &\qquad \cc \;\land \\
                 &\qquad \conflictaxiom \;\land \\
                 &\qquad \ev.
\end{align*}
$\unistore$ satisfies $\por$, denoted $\unistore \models \por$,
if all its histories do.

Given an abstract execution $A = (H, \vis, \ar)$,
the axioms are defined as follows.
\begin{appdefinition}[$\retval$, \cite{framework-concur15}] \label{def:retval}
  The Return Value Consistency (\retval) specifies
  the return value of each read event.
  \[
    \retval \triangleq \intretval \land \extretval.
  \]
  Here $\intretval$ requires an internal read event $e$
  on key $k$ to read from the last update event on $k$ preceding $e$
  in the same transaction. Formally,
  \begin{align*}
    &\intretval \triangleq \forall e \in \intread \cap R_{k} \cap V_{H}. \\
        &\quad \rval(e) = \uval\big(\max_{\po}(\po^{-1}(e) \cap U_{k})\big).
  \end{align*}
  $\extretval$ requires an external read event $e$
  on key $k$ to read from the last update event on $k$
  in the last transaction preceding $\txfunc(e)$ in $\ar$,
  among the set of transactions visible to $\txfunc(e)$.
  Formally,
  \begin{align*}
    &\extretval \triangleq \forall e \in \extread \cap R_{k} \cap V_{H}. \\
        &\quad \rval(e) =
          \uval\Big(
                \ud\big(\max_{\ar}\big(\vis^{-1}(\txfunc(e)) \cap \txs_{k}\big), k\big)
               \Big).
  \end{align*}
\end{appdefinition}

\begin{appdefinition}[\cc, \cite{sebastian-book}] \label{def:cc}
  \begin{align*}
    \cc \triangleq\; &\cv \;\land \\
                   &\ca,
  \end{align*}
  where
  \[
    \cv \triangleq (\so\; \cup \vis)^{+} \subseteq \vis;
  \]
  \[
    \ca \triangleq \vis \subseteq \ar.
  \]
\end{appdefinition}

The Conflict Ordering property requires that
out of any two conflicting strong transactions,
one must be visible to the other.
Formally,
\begin{appdefinition}[Conflict Relation] \label{def:conflict-relation-tx}
  The conflict relation, denoted by $\conflict$, between strong transactions
  is a symmetric relation defined as follows:
  \begin{align*}
    &\forall \tvar, \tvar' \in \strongtxs.\; \tvar \conflict \tvar' \iff \\
      &\quad (\R(\tvar) \cap \W(\tvar') \neq \emptyset) \lor
             (\W(\tvar) \cap \R(\tvar') \neq \emptyset).
  \end{align*}
\end{appdefinition}

\begin{appdefinition}[\conflictaxiom] \label{def:conflictaxiom}
  \begin{align*}
    &\conflictaxiom \triangleq
      \forall \tvar_{1}, \tvar_{2} \in X \cap \strongtxs.\; \\
        &\quad \tvar_{1} \conflict \tvar_{2} \implies
          \tvar_{1} \rel{\vis} \tvar_{2} \lor \tvar_{2} \rel{\vis} \tvar_{1}.
  \end{align*}
\end{appdefinition}
The Eventual Visibility property requires that
a transaction that originates at a correct data center,
that is visible to some \fence{} events,
or that is a strong transaction
eventually becomes visible at all correct data centers.
Let $\C \subseteq \D$ be the set of correct data centers.
Formally,
\begin{appdefinition}[\ev] \label{def:ev}
  \begin{align*}
    &\ev \triangleq \forall \tvar \in X \cap \txs.\; \\
      &\quad \dc(\tvar) \in \C \lor
        (\exists \fencerange \in \Fence.\; \tvar \rel{\vis} \fencerange) \lor \tvar \in \strongtxs \\
        &\qquad \implies \left\lvert \left\{\tvar' \in \txs
          \mid \lnot (\tvar \rel{\vis} \tvar') \right\} \right\rvert < \infty.
  \end{align*}
\end{appdefinition}


\section{Transaction Certification Service Specification} \label{section:tcs}

\subsection{Interface} \label{ss:interface}

The \emph{Transaction Certification Service} (TCS)~\cite{discpaper}
is responsible for certifying strong transactions issued by transaction coordinators,
computing commit vectors and Lamport clocks for committed transactions,
and (asynchronously) delivering committed transactions to replicas.

Each strong transaction $\tvar \in \allstrongtxs$ submitted to TCS
may be associated with its read set $\rs(\tvar)$, write set $\ws(\tvar)$,
snapshot vector $\snapshotVC(\tvar)$ (Definition~\ref{def:snapshotvc}),
commit vector $\commitVC(\tvar)$ (Definition~\ref{def:commitvc}),
and Lamport clock $\lclock(\tvar)$ (Definition~\ref{def:lc-tx}).
From $\rs(\tvar)$ and $\ws(\tvar)$ we can then define 
$\W(t)$ and $\R(t)$ according to~(\ref{eq:W-def}) and~(\ref{eq:R-def}), respectively.
Note that we have $\W(t) \subseteq \R(t)$.

Transaction coordinators for strong transactions interact with TCS
using two types of \emph{actions}. Coordinators can make certification requests
(corresponding to procedure \certify{} of Algorithm~\ref{alg:unistore-certify})
of the form
\[
  \intcertify(\tidselector(\tvar), \ws(\tvar), \R(\tvar),
    \snapshotVC(\tvar), \lccertify(\tvar)),
\]
where $\tvar \in \allstrongtxs$ and $\lccertify(\tvar) \in \N$
denotes the contribution of $\client(\tvar)$ to the Lamport clock of $\tvar$.
The TCS responses are of the form
\[
  \intdecide(\tidselector(\tvar), \decvar, \vcvar, \lcvar),
\]
containing a decision $\decvar$ from $\Decision = \set{\commit, \abort}$ for $\tvar$,
a commit vector $\vcvar$ from $\Vector$ for $\tvar$ if $\decvar = \commit$,
and a Lamport clock $\lcvar$ from $\N$ for $\tvar$ if $\decvar = \commit$.
If $\decvar = \abort$, then $\vcvar$ and $\lcvar$ are irrelevant.

Besides, TCS can deliver some payload $\payload$
to a replica via \upcall \emph{actions} $\intdeliver(\payload)$
(corresponding to procedure \deliverupdates{}
of Algorithm~\ref{alg:unistore-strong-commit}).
We denote by $\intdeliver^{m}_{d}(\payload)$ the delivery of the payload $\payload$ to
a replica $p^{m}_{d}$, when the latter is relevant.
The payload $\payload$ in $\intdeliver^{m}_{d}(\payload)$ is a set of tuples of the form
$\langle \tidvar, \wbuffvar, \commitvc, \lcvar \rangle$,
each of which corresponds to the updates $\wbuffvar \subseteq \Key \times \Val$
performed at a particular partition $m$
by a particular committed strong transaction
with transaction identifier $\tidvar$, commit vector $\commitvc$,
and Lamport clock $\lcvar$.
\subsection{Certification Functions} \label{ss:cert-func}

TCS is specified using a certification function
\begin{align}
  \certfunc: 2^{\strongtxs} \times \allstrongtxs \to
    \Decision \times \Vector \times \N.
  \label{eqn:f}
\end{align}
For a strong transaction $\tvar \in \allstrongtxs$
and the set $\txsincertify \subseteq \strongtxs$ of previously committed strong transactions,
$\certfunc(\txsincertify, \tvar)$ returns not only the decision $\decvar \in \Decision$,
but also the commit vector $\vcvar \in \Vector$
and Lamport clock $\lcvar \in \N$ for $\tvar$.
We use $\fdec(\txsincertify, \tvar)$, $\fvec(\txsincertify, \tvar)$,
and $\flc(\txsincertify, \tvar)$ to select
the first, second, and third component
of $\certfunc(\txsincertify, \tvar)$, respectively.

The decision $\fdec(\txsincertify, \tvar)$ should satisfy
\begin{align}
  &\fdec(\txsincertify, \tvar) = \commit
    \iff \forall k \in \R(\tvar).\; \forall \tvar' \in \txsincertify.
    \nonumber\\
    &\quad \big(k \in \W(\tvar') \implies
      \commitVC(\tvar') \le \snapshotVC(\tvar)\big).
    \label{eqn:gcf-decision}
\end{align}

The commit vector $\fvec(\txsincertify, \tvar)$ should satisfy
\begin{align}
  &\phantom{\land\;} (\forall i \in \D.\; \fvec(\txsincertify, \tvar)[i] = \snapshotVC(\tvar)[i]) \nonumber\\
  &\land \fvec(\txsincertify, \tvar)[\strongentry] > \snapshotVC(\tvar)[\strongentry] \nonumber\\
  &\land \forall \tvar' \in \txsincertify.\; \tvar \conflict \tvar'
    \implies \fvec(\txsincertify, \tvar) \ge \commitVC(\tvar').
    \label{eqn:gcf-commitvc}
\end{align}

The Lamport clock $\flc(\txsincertify, \tvar)$ should satisfy
\begin{align}
  &\flc(\txsincertify, \tvar) \ge \lccertify(t) \;\land \nonumber\\
  &\quad \big(\forall \tvar' \in \txsincertify.\; \tvar \conflict \tvar' \implies
    \flc(\txsincertify, \tvar) > \lclock(\tvar')\big).
    \label{eqn:gcf-lc}
\end{align}
\subsection{Histories of TCS} \label{ss:histories}

TCS executions are represented by \emph{histories},
which are (possibly infinite) sequences
of $\intcertify$, $\intdecide$, and $\intdeliver$ actions.
For a TCS history $h$, we use $\act(h)$ to denote the set of actions in $h$.
For actions $\actvar, \actvar' \in \act(h)$,
we write $\actvar \prec_{h} \actvar'$
when $\actvar$ occurs before $\actvar'$ in $h$.
A strong transaction $\tvar \in \allstrongtxs$ \emph{commits} in a history $h$
if $h$ contains $\intdecide(\tidselector(\tvar), \commit, \_, \_)$.
We denote by $\committedVar(h)$ the projection of $h$ to actions
corresponding to the strong transactions that are committed in $h$.

Each history $h$ needs to meet the following requirements.
\begin{enumerate}[(R1)]
  \item \label{tcs-requirement:certify-once}
    For each strong transaction $\tvar \in \allstrongtxs$,
    there is at most one $\intcertify(\tidselector(\tvar), \_, \_, \_, \_)$
    action in $h$.
  \item \label{tcs-requirement:decide-once}
    For each action $\intdecide(\tidvar, \_, \_, \_) \in \act(h)$,
    there is exactly one $\intcertify(\tidvar, \_, \_, \_, \_)$ action in $h$
    such that
    \[
      \intcertify(\tidvar, \_, \_, \_, \_)
        \prec_{h} \intdecide(\tidvar, \_, \_, \_).
    \]
  \item \label{tcs-requirement:abort-cannot-deliver}
    For each action $\intdeliver(\payload) \in \act(h)$
    and each $\langle \tidvar, \_, \_, \_ \rangle \in \payload$,
    there is \emph{no} $\intdecide(\tidvar, \abort, \_, \_)$ action in $h$.
  \item \label{tcs-requirement:deliver-once}
    Every committed strong transaction is delivered at most once to each replica.
  \item \label{tcs-requirement:certify-before-deliver}
    For each action $\intdeliver(\payload) \in \act(h)$
    and each $\langle \tidvar, \_, \_, \_ \rangle \in \payload$,
    there is a $\intcertify(\tidvar, \_, \_, \_, \_)$ action such that
    \[
      \intcertify(\tidvar, \_, \_, \_, \_) \prec_{h} \intdeliver(\payload).
    \]
  \item \label{tcs-requirement:deliver-order}
    At each replica $p^{m}_{d}$, committed strong transactions are delivered
    in the increasing order of their strong timestamps.
    Formally, for any two distinct actions
    $\intdeliver^{m}_{d}(\payload_{1})$ and $\intdeliver^{m}_{d}(\payload_{2})$
    with payloads $\payload_{1}$ and $\payload_{2}$, respectively,
    \begin{align*}
      &\intdeliver^{m}_{d}(\payload_{1}) \prec_{h} \intdeliver^{m}_{d}(\payload_{2}) \implies \\
        &\quad \forall \langle \_, \_, \commitvc_{1}, \_ \rangle \in \payload_{1}.\;\\
        &\quad \forall \langle \_, \_, \commitvc_{2}, \_ \rangle \in \payload_{2}.\; \\
          &\qquad \commitvc_{1}[\strongentry] < \commitvc_{2}[\strongentry].
    \end{align*}
\end{enumerate}

A history is \emph{complete} if every $\intcertify$ action
in it has a matching $\intdecide$ action.
A complete history $h$ is \emph{sequential} if
it consists of consecutive pairs of $\intcertify$ and matching $\intdecide$ actions.
For a complete history $h$, a \emph{permutation} $h'$ of $h$
is a sequential history such that
\begin{itemize}
  \item $h$ and $h'$ contain the same actions, i.e., $\act(h) = \act(h')$.
  \item Transactions are certified in $h'$ according to their session order.
    \begin{align*}
      &\forall \tvar, \tvar' \in \allstrongtxs.\;
        \tvar \rel{\so} \tvar' \implies \\
          &\quad \intdecide(\tidselector(\tvar), \_, \_, \_) \prec_{h'}
            \intcertify(\tidselector(\tvar'), \_, \_, \_, \_).
    \end{align*}
\end{itemize}
\subsection{TCS Correctness: Safety and Liveness} \label{ss:tcs-correctness}

\subsubsection{Safety of TCS} \label{ss:tcs-safety}

A complete sequential history $h$ is \emph{legal} with respect to
a certification function $\certfunc$,
if its results are computed so as to satisfy
(\ref{eqn:gcf-decision}) -- (\ref{eqn:gcf-lc}) according to $\certfunc$:
\begin{align*}
  &\forall \actvar = \intdecide(\tidselector(\tvar), \decvar, \vcvar,
    \lcvar) \in \act(h). \\
    &\quad (\decvar, \vcvar, \lcvar)
      = \certfunc(\set{\tvar' \mid \\
        &\quad\quad \intdecide(\tidselector(\tvar'), \commit, \_, \_) \prec_{h} \actvar}, \tvar).
\end{align*}
A history $h$ is \emph{correct} with respect to $\certfunc$
if $h \mid \committedVar(h)$ has a legal permutation.
A TCS implementation is \emph{correct} with respect to $\certfunc$
if so are all its histories.
\subsubsection{Liveness of TCS} \label{ss:tcs-liveness}

TCS guarantees that every committed strong transaction
will eventually be delivered by every correct data center.
Formally,
\begin{align}
  &\forall \actvar = \intdecide(\tidvar, \commit, \_, \_) \in \act(h).
    \nonumber\\
    &\quad \forall m \in \partitionsfunc(\tidvar).\; \forall \cdrange \in \C.
    \nonumber\\
    &\qquad \exists \actvar' = \intdeliver^{m}_{\cdrange}(\payload) \in \act(h).\;
    \nonumber\\
      &\quad\qquad \langle \tidvar, \_, \_, \_ \rangle \in \payload
        \land \actvar \prec_{h} \actvar'.
    \label{eqn:tcs-liveness-delivery}
\end{align}
Here $\partitionsfunc(\tidvar)$ denotes the set of partitions
that a particular transaction with transaction identifier $\tidvar$ accesses.

A TCS implementation meets the liveness requirement if
every history produced by its maximal execution satisfies (\ref{eqn:tcs-liveness-delivery}).
\subsection{TCS Correctness} \label{ss:tcs-correctness-unistore}

The proof of TCS correctness is an adjustment
of the ones in~\cite{discpaper, multicast-dsn19}.

\begin{apptheorem} \label{thm:tcs-correctness}
  The TCS implementation in \unistore{}
  (Algorithms~\ref{alg:unistore-certify} -- \ref{alg:unistore-recovery})
  is correct with respect to the certification function $\certfunc$ in (\ref{eqn:f})
  and meets the liveness requirement in (\ref{eqn:tcs-liveness-delivery}).
\end{apptheorem}


\section{The Proof of \unistore{} Correctness} \label{section:correctness-proof}

Consider an execution of \unistore{} with a history $H = (X, \client, \dc, \so)$.
We prove that $H$ satisfies \por{}
by constructing an abstract execution $A$ (Theorem~\ref{thm:unistore-por}).
We also establish the liveness guarantees of \unistore{} (Theorem~\ref{thm:termination}).


\subsection{Assumptions} \label{ss:assumptions}

We take the following assumptions about \unistore.

\begin{appassumption} \label{assumption:clock}
  For any replica $p^{m}_{d}$ in data center $d$,
  $\clockVar$ at $p^{m}_{d}$ is strictly increasing until $d$ (may) crash.
\end{appassumption}

\begin{appassumption} \label{assumption:message}
  Replicas are connected by reliable FIFO channels:
  messages are delivered in FIFO order,
  and messages between correct data centers
  are guaranteed to be eventually delivered.
\end{appassumption}

\begin{appassumption} \label{assumption:failure-model}
  We assume that in an execution of \unistore, any clients
  and up to $f$ data centers may crash and that $D > 2f$.
\end{appassumption}

\begin{appassumption} \label{assumption:fairness}
  We assume fairness of procedures of \unistore:
  In an execution, if a procedure is enabled infinitely often,
  then it will be executed infinitely often.
\end{appassumption}

\begin{appassumption} \label{assumption:client-well-formed}
  We consider only \emph{well-formed} executions,
  in which for each client:
  \begin{itemize}
    \item transactions are issued in sequence; and
    \item both \fence{} and \clattach{} events
      can be issued only outside of transactions.
  \end{itemize}
\end{appassumption}

\begin{appassumption} \label{assumption:complete-execution}
  We consider only executions where
  every causal commit event (i.e., \commitcausaltx) completes
  and every strong commit event (i.e., \commitstrongtx) that calls the TCS completes.
\end{appassumption}

We make the last assumption to simplify the technical development. The other
assumptions come from the system model.


\subsection{Notations} \label{ss:proof-notations}

We use $\cl$ to range over clients from a finite set $\clients$.
We also use the following notations
to refer to different types of variables and their values
(below are some typical examples).
\begin{itemize}
  \item $\snapVC^{m}_{d}$:
    The variable $\snapVC$ at replica $p^{m}_{d}$.
  \item $(\snapVC^{m}_{d})_{e}$:
    The value of variable $\snapVC^{m}_{d}$
    after the event $e$ is performed at replica $p^{m}_{d}$.
  \item $\snapVC^{m}_{d}(\realtime)$:
    The value of $\snapVC^{m}_{d}$ at some specific time $\realtime$.
  \item $\pastVC_{\cl}$: The variable $\pastVC$ at client $\cl$.
  \item $(\pastVC_{\cl})_{e}$:
    The value of variable $\pastVC_{\cl}$
    after the event $e$ is performed at client $\cl$.
  \item $\snapvc_{(\readkey, e)}$:
    The actual value of \emph{parameter} $\snapvc$ of handler \readkey{}
    for event $e$.
  \item $\commitvc_{(\commitcausal, e)}$:
    The value of the \emph{local variable} $\commitvc$
    in procedure \commitcausal{}
    after event $e$ is performed.
\end{itemize}
Besides, we use $\coord(t)$ to denote the coordinator partition
of transaction $\tvar$.

Each transaction is associated with a snapshot vector and a commit vector.

\begin{appdefinition}[Snapshot Vector] \label{def:snapshotvc}
  Let $\tvar \in \txs$ be a transaction.
  Let $d \triangleq \dc(\tvar)$ and $m \triangleq \coord(\tvar)$.
  We define its snapshot vector $\snapshotVC(\tvar)$ as
  \[
    \snapshotVC(\tvar) \triangleq (\snapVC^{m}_{d})_{\startoftx(\tvar)}[\tvar].
  \]
\end{appdefinition}

\begin{appdefinition}[Commit Vector] \label{def:commitvc}
  Let $\tvar \in \txs$ be a transaction.
  Let $d \triangleq \dc(\tvar)$ and $m \triangleq \coord(\tvar)$.
  We define its commit vector $\commitVC(\tvar)$ as follows.
  \begin{itemize}
    \item If $\tvar$ is a read-only causal transaction, then
      \[
        \commitVC(\tvar) \triangleq (\snapVC^{m}_{d})_{\commitoftx(\tvar)}[\tvar].
      \]
    \item If $\tvar$ is an update causal transaction, then
      \[
        \commitVC(\tvar) \triangleq \commitvc_{(\commitcausal, \commitoftx(\tvar))}.
      \]
    \item If $\tvar$ is a committed strong transaction, then
      \[
        \commitVC(\tvar) \triangleq vc_{(\commitstrong, \commitoftx(\tvar))}.
      \]
  \end{itemize}
\end{appdefinition}

\begin{applemma} \label{lemma:snapshotvc-commitvc}
  \[
    \forall \tvar \in \txs.\; \commitVC(\tvar) \ge \snapshotVC(\tvar).
  \]
\end{applemma}

\begin{proof} \label{proof:snapshotvc-commitvc}
  We perform a case analysis according to the type of $\tvar$.
  \begin{itemize}
    \item $\textsc{Case I}$: $\tvar$ is a read-only causal transaction.
      By Definition~\ref{def:snapshotvc} of $\snapshotVC(\tvar)$,
      Definition~\ref{def:commitvc} of $\commitVC(\tvar)$,
      and Assumption~\ref{assumption:client-well-formed},
      \[
        \commitVC(\tvar) = \snapshotVC(\tvar).
      \]
    \item $\textsc{Case II}$: $\tvar$ is an update causal transaction.
      By lines~\code{\ref{alg:unistore-coord}}{\ref{line:commitcausal-commitvc}}
      and \code{\ref{alg:unistore-coord}}{\ref{line:commitcausal-commitvc-d}},
      \[
        \commitVC(\tvar) \ge \snapshotVC(\tvar).
      \]
    \item $\textsc{Case III}$: $\tvar$ is a strong transaction.
      By line~\code{\ref{alg:unistore-strong-commit}}{\ref{line:commitstrong-call-certify}}
      and (\ref{eqn:gcf-commitvc}),
      \[
        \commitVC(\tvar) \ge \snapshotVC(\tvar).
      \]
  \end{itemize}
\end{proof}

For client $\cl$, we use $\cldc(\cl)$ to denote the data center
to which $\cl$ is currently attached.
We also use $\txs|_{\cl}$ to denote the set of transactions issued by $\cl$.
Formally,
\[
  \txs|_{\cl} \triangleq \set{\tvar \in \txs \mid \client(\tvar) = cl}.
\]

For a transaction $\tvar$ and a partition $m$,
we use $\ws(\tvar)[m]$ to denote the subset of $\ws(\tvar)$
restricted to partition $m$.
Formally,
\[
  \ws(\tvar)[m] \triangleq \set{\langle k, v \rangle \in \ws(\tvar)
  \mid \partitionofproc(k) = m}.
\]

For notational convenience, we also define
\begin{align*}
  \log(\tvar) \triangleq \set{\langle k, v, \commitVC(\tvar), \lclock(\tvar)
    \rangle \mid \langle k, v \rangle \in \ws(\tvar)},
\end{align*}
and
\begin{align*}
  \log(\tvar)[m] \triangleq \set{\langle k, v, \commitVC(\tvar), \lclock(\tvar)
    \rangle \mid \langle k, v \rangle \in \ws(\tvar)[m]}.
\end{align*}
For a key $k \in \Key$ and a transaction $\tvar \in \txs_{k}$,
let $\log(\tvar)[k]$ be the unique tuple
\[
  \langle k, v, \commitVC(\tvar), \lclock(\tvar) \rangle
\]
in $\log(\tvar)$.


\subsection{Metadata for Causal Transactions}
\label{ss:metadata-causal}

A causal transaction is \emph{committed} when \commitcausal{} for it returns.
A causal transaction is \emph{committed at replica} $p^{m}_{d}$
when \commit{} for it at $p^{m}_{d}$ returns.
\subsubsection{Properties of $\knownVC$}
\label{sss:knownvc}

\begin{applemma} \label{lemma:knownvc-d-nondecreasing}
  For any replica $p^{m}_{d}$ in data center $d$,
  $\knownVC^{m}_{d}[d]$ is non-decreasing.
\end{applemma}

\begin{proof} \label{proof:knownvc-d-nondecreasing}
  Consider two points of time $\realtime_{1}$ and $\realtime_{2}$ such that $\realtime_{1} < \realtime_{2}$.
  We need to show that
  \[
    \knownVC^{m}_{d}(\realtime_{1})[d] \le \knownVC^{m}_{d}(\realtime_{2})[d].
  \]

  Note that $\knownVC^{m}_{d}[d]$ is updated only
  at lines~\code{\ref{alg:unistore-replication}}{\ref{line:propagate-knownvc-clock}}
  or \code{\ref{alg:unistore-replication}}{\ref{line:propagate-knownvc-ts}}.
  We distinguish between the following four cases.
  \begin{itemize}
    \item $\textsc{Case I}$:
      Both $\knownVC^{m}_{d}(\realtime_{1})[d]$ and $\knownVC^{m}_{d}(\realtime_{2})[d]$ are set
      at line~\code{\ref{alg:unistore-replication}}{\ref{line:propagate-knownvc-clock}}.
      By line~\code{\ref{alg:unistore-replication}}{\ref{line:propagate-knownvc-clock}}
      and Assumption~\ref{assumption:clock},
      \begin{align*}
        \knownVC^{m}_{d}(\realtime_{1})[d] &= \clockVar^{m}_{d}(\realtime_{1}) \\
          &< \clockVar^{m}_{d}(\realtime_{2}) \\
          &= \knownVC^{m}_{d}(\realtime_{2})[d].
      \end{align*}
    \item $\textsc{Case II}$:
      $\knownVC^{m}_{d}(\realtime_{1})[d]$ is set
      at line~\code{\ref{alg:unistore-replication}}{\ref{line:propagate-knownvc-clock}}
      and $\knownVC^{m}_{d}(\realtime_{2})[d]$ is set
      at line~\code{\ref{alg:unistore-replication}}{\ref{line:propagate-knownvc-ts}}.
      By line~\code{\ref{alg:unistore-replication}}{\ref{line:propagate-knownvc-clock}},
      \[
        \knownVC^{m}_{d}(\realtime_{1})[d] = \clockVar^{m}_{d}(\realtime_{1}).
      \]
      By the fact that $\preparedcausal^{m}_{d}(\realtime_{1}) = \emptyset$,
      $\realtime_{2} > \realtime_{1}$,
      and line~\code{\ref{alg:unistore-replica}}{\ref{line:preparecausal-ts}},
      \begin{align*}
        \forall \langle \_, \_, &\tsvar \rangle \in\; \preparedcausal^{m}_{d}(\realtime_{2}).\; \\
          &\tsvar > \clockVar^{m}_{d}(\realtime_{1}) = \knownVC^{m}_{d}(\realtime_{1})[d].
      \end{align*}
      Therefore, by line~\code{\ref{alg:unistore-replication}}{\ref{line:propagate-knownvc-ts}},
      \begin{align*}
        &\knownVC^{m}_{d}(\realtime_{1})[d] \\
        &\quad \le \min\set{\tsvar \mid \langle \_, \_, \tsvar \rangle \in \preparedcausal^{m}_{d}(\realtime_{2})} - 1 \\
        &\quad = \knownVC^{m}_{d}(\realtime_{2})[d].
      \end{align*}
    \item $\textsc{Case III}$:
      $\knownVC^{m}_{d}(\realtime_{1})[d]$ is set
      at line~\code{\ref{alg:unistore-replication}}{\ref{line:propagate-knownvc-ts}}
      and $\knownVC^{m}_{d}(\realtime_{2})[d]$ is set
      at line~\code{\ref{alg:unistore-replication}}{\ref{line:propagate-knownvc-clock}}.
      Let $\tvar_{1}$ be the transaction in $\preparedcausal^{m}_{d}(\realtime_{1})$
      that has the minimum $\tsvar$. Formally,
      \begin{align*}
        \tvar_{1} \triangleq \argmin\limits_{\tvar}\set{\tsvar \mid \langle \tidselector(\tvar), \_, \tsvar \rangle
          \in \preparedcausal^{m}_{d}(\realtime_{1})}.
      \end{align*}
      By lines~\code{\ref{alg:unistore-replication}}{\ref{line:propagate-knownvc-ts}},
      \code{\ref{alg:unistore-coord}}{\ref{line:commitcausal-commitvc-d}},
      \code{\ref{alg:unistore-replica}}{\ref{line:commit-wait-clock}},
      and \code{\ref{alg:unistore-replication}}{\ref{line:propagate-knownvc-clock}},
      \begin{align*}
        \knownVC^{m}_{d}(\realtime_{1})[d] &< \commitVC(\tvar_{1})[d] \\
          &\le \clockVar^{m}_{d}(\realtime_{2}) \\
          &= \knownVC^{m}_{d}(\realtime_{2})[d].
      \end{align*}
    \item $\textsc{Case IV}$:
      Both $\knownVC^{m}_{d}(\realtime_{1})[d]$ and $\knownVC^{m}_{d}(\realtime_{2})[d]$ are set
      at line~\code{\ref{alg:unistore-replication}}{\ref{line:propagate-knownvc-ts}}.
      By lines~\code{\ref{alg:unistore-replication}}{\ref{line:propagate-knownvc-ts}}
      and \code{\ref{alg:unistore-replica}}{\ref{line:preparecausal-ts}},
      \begin{align*}
        &\knownVC^{m}_{d}(\realtime_{1})[d] \\
          &\quad = \min\set{\tsvar \mid \langle \_, \_, \tsvar \rangle \in \preparedcausal^{m}_{d}(\realtime_{1})} - 1 \\
          &\quad \le \min\set{\tsvar \mid \langle \_, \_, \tsvar \rangle \in \preparedcausal^{m}_{d}(\realtime_{2})} - 1 \\
          &\quad = \knownVC^{m}_{d}(\realtime_{2})[d].
      \end{align*}
  \end{itemize}
\end{proof}

\begin{applemma} \label{lemma:knownvc-i-nondecreasing}
  For $i \in \D \setminus \set{d}$,
  $\knownVC^{m}_{d}[i]$ at any replica $p^{m}_{d}$ in data center $d$
  is non-decreasing.
\end{applemma}

\begin{proof} \label{proof:knownvc-i-nondecreasing}
  Note that $\knownVC^{m}_{d}[i]$ ($i \in \D \setminus \set{d}$)
  can be updated only
  at lines~\code{\ref{alg:unistore-replication}}{\ref{line:replicate-knownvc}}
  and \code{\ref{alg:unistore-replication}}{\ref{line:heartbeat-knownvc}}.
  Therefore, this lemma holds due to
  lines~\code{\ref{alg:unistore-replication}}{\ref{line:replicate-precondition}}
  and \code{\ref{alg:unistore-replication}}{\ref{line:heartbeat-precondition}}.
\end{proof}

\begin{applemma} \label{lemma:knownvc-nondecreasing}
  For $i \in \D$, $\knownVC^{m}_{d}[i]$ at any replica $p^{m}_{d}$
  in data center $d$ is non-decreasing.
\end{applemma}

\begin{proof} \label{proof:knownvc-nondecreasing}
  By Lemmas~\ref{lemma:knownvc-d-nondecreasing}
  and \ref{lemma:knownvc-i-nondecreasing}.
\end{proof}

\begin{applemma} \label{lemma:knownvc-d-clock}
  For any replica $p^{m}_{d}$ in data center $d$,
  \[
    \knownVC^{m}_{d}[d] \le \clockVar^{m}_{d}.
  \]
\end{applemma}

\begin{proof} \label{proof:knownvc-d-clock}
  Note that $\knownVC^{m}_{d}[d]$ is updated only
  at lines~\code{\ref{alg:unistore-replication}}{\ref{line:propagate-knownvc-clock}}
  or \code{\ref{alg:unistore-replication}}{\ref{line:propagate-knownvc-ts}}.
  \begin{itemize}
    \item $\textsc{Case I}$: $\knownVC^{m}_{d}[d]$ is updated
      at line~\code{\ref{alg:unistore-replication}}{\ref{line:propagate-knownvc-clock}}.
      By Assumption~\ref{assumption:clock},
      \[
        \knownVC^{m}_{d}[d] \le \clockVar^{m}_{d}.
      \]
    \item $\textsc{Case II}$: $\knownVC^{m}_{d}[d]$ is updated
      at line~\code{\ref{alg:unistore-replication}}{\ref{line:propagate-knownvc-ts}}.
      By line~\code{\ref{alg:unistore-replica}}{\ref{line:preparecausal-ts}},
      immediately after this update,
      \[
        \knownVC^{m}_{d}[d] < \clockVar^{m}_{d}.
      \]
  \end{itemize}
\end{proof}

\begin{applemma} \label{lemma:knownvc-commitvc-d}
  Let $p^{m}_{d}$ be a replica in data center $d$.
  Consider $\knownVC^{m}_{d}(\realtime)[d]$ at time $\realtime$
  and transaction $\tvar \in \causaltxs$ committed at $p^{m}_{d}$
  after time $\realtime$. Then
  \[
    \commitVC(\tvar)[d] > \knownVC^{m}_{d}(\realtime)[d].
  \]
\end{applemma}

\begin{proof} \label{proof:knownvc-commitvc-d}
  Suppose that before time $\realtime$,
  $\knownVC^{m}_{d}[d]$ is last updated at time $\realtime' < \realtime$.
  Therefore,
  \[
    \knownVC^{m}_{d}(\realtime)[d] = \knownVC^{m}_{d}(\realtime')[d].
  \]
  We distinguish between two cases according to whether
  \[
    \preparedcausal^{m}_{d}(\realtime') = \emptyset
  \]
  when $\knownVC^{m}_{d}[d]$ is updated at time $\realtime'$.
  \begin{itemize}
    \item $\textsc{Case I}$: $\preparedcausal^{m}_{d}(\realtime') = \emptyset$.
      By line~\code{\ref{alg:unistore-replication}}{\ref{line:propagate-knownvc-clock}},
      \[
        \knownVC^{m}_{d}(\realtime')[d] = \clockVar^{m}_{d}(\realtime').
      \]
      By line~\code{\ref{alg:unistore-replica}}{\ref{line:preparecausal-ts}},
      line~\code{\ref{alg:unistore-coord}}{\ref{line:commitcausal-commitvc-d}},
      and Assumption~\ref{assumption:clock},
      \[
        \commitVC(\tvar)[d] > \clockVar^{m}_{d}(\realtime').
      \]
      Therefore,
      \begin{align*}
        \commitVC(\tvar)[d] &> \knownVC^{m}_{d}(\realtime')[d] \\
                          &= \knownVC^{m}_{d}(\realtime)[d].
      \end{align*}
    \item $\textsc{Case II}$: $\preparedcausal^{m}_{d}(\realtime') \neq \emptyset$.
      We further distinguish between two cases according to whether
      \[
        \langle \tidselector(\tvar), \_, \_ \rangle \in \preparedcausal^{m}_{d}(\realtime').
      \]
      \begin{itemize}
        \item $\textsc{Case II-1}$:
          $\langle \tidselector(\tvar), \_, \tsvar \rangle \in \preparedcausal^{m}_{d}(\realtime')$.
          By lines~\code{\ref{alg:unistore-replication}}{\ref{line:propagate-knownvc-ts}}
          and \code{\ref{alg:unistore-coord}}{\ref{line:commitcausal-commitvc-d}},
          \begin{align*}
            \commitVC(\tvar)[d] &\ge \tsvar \\
                              &> \knownVC^{m}_{d}(\realtime')[d] \\
                              &= \knownVC^{m}_{d}(\realtime)[d].
          \end{align*}
        \item $\textsc{Case II-2}$:
          $\langle \tidselector(\tvar), \_, \_ \rangle \notin \preparedcausal^{m}_{d}(\realtime')$.
          By Lemma~\ref{lemma:knownvc-d-clock},
          Assumption~\ref{assumption:clock},
          line~\code{\ref{alg:unistore-replica}}{\ref{line:preparecausal-ts}},
          and line~\code{\ref{alg:unistore-coord}}{\ref{line:commitcausal-commitvc-d}},
          \begin{align*}
            \commitVC(\tvar)[d] &> \knownVC^{m}_{d}(\realtime')[d] \\
                              &= \knownVC^{m}_{d}(\realtime)[d].
          \end{align*}
      \end{itemize}
  \end{itemize}
\end{proof}

\begin{applemma} \label{lemma:knownvc-local-d}
  Let $\tvar \in \causaltxs$ be a causal transaction
  that originates at data center $d$ and accesses partition $m$.
  If
  \[
    \commitVC(\tvar)[d] \le \knownVC^{m}_{d}[d],
  \]
  then
  \[
    \log(\tvar)[m] \subseteq \oplog^{m}_{d}.
  \]
\end{applemma}

\begin{proof} \label{proof:knownvc-local-d}
  Suppose that the value $\knownVC^{m}_{d}[d]$ is set at time $\realtime$.
  By Lemma~\ref{lemma:knownvc-commitvc-d},
  $\tvar$ is committed at $p^{m}_{d}$ before time $\realtime$.
  Therefore, by line~\code{\ref{alg:unistore-replica}}{\ref{line:commit-oplog}},
  \[
    \log(\tvar)[m] \subseteq \oplog^{m}_{d}.
  \]
\end{proof}

The following lemmas consider the replication and forwarding
of causal transactions.

\begin{applemma} \label{lemma:replication-order}
  Let $p^{m}_{d}$ be a replica in data center $d$.
  Let $\tvar_{1}$ and $\tvar_{2}$ be two transactions
  replicated by $p^{m}_{d}$ to sibling replicas
  at time $\realtime_{1}$ and $\realtime_{2}$
  (line~\code{\ref{alg:unistore-replication}}{\ref{line:propagate-call-replicate}}),
  respectively. Then
  \[
    \realtime_{1} < \realtime_{2} \implies \commitVC(\tvar_{1})[d] < \commitVC(\tvar_{2})[d].
  \]
\end{applemma}

\begin{proof} \label{proof:replication-order}
  Since $\tvar_{1}$ is replicated at time $\realtime_{1}$,
  by line~\code{\ref{alg:unistore-replication}}{\ref{line:propagate-txs}},
  \[
    \commitVC(\tvar_{1})[d] \le \knownVC^{m}_{d}(\realtime_{1})[d].
  \]
  Assume that $\realtime_{1} < \realtime_{2}$.
  We distinguish between two cases according to whether
  \[
    \langle \tidselector(\tvar_{2}), \_, \_, \_ \rangle \in \committedcausal^{m}_{d}(\realtime_{1})[d].
  \]
  \begin{itemize}
    \item $\textsc{Case I}$:
      $\langle \tidselector(\tvar_{2}), \_, \_, \_ \rangle \in \committedcausal^{m}_{d}(\realtime_{1})[d]$.
      Since $\tvar_{2}$ is not replicated at time $\realtime_{1}$,
      by line~\code{\ref{alg:unistore-replication}}{\ref{line:propagate-txs}},
      \[
        \commitVC(\tvar_{2})[d] > \knownVC^{m}_{d}(\realtime_{1})[d].
      \]
    \item $\textsc{Case II}$:
      $\langle \tidselector(\tvar_{2}), \_, \_, \_, \rangle \notin \committedcausal^{m}_{d}(\realtime_{1})[d]$.
      Thus, $\tvar_{2}$ is committed at $p^{m}_{d}$ after time $\realtime_{1}$.
      By Lemma~\ref{lemma:knownvc-commitvc-d},
      \[
        \commitVC(\tvar_{2})[d] > \knownVC^{m}_{d}(\realtime_{1})[d].
      \]
  \end{itemize}
  Therefore, in either case,
  \[
    \commitVC(\tvar_{1})[d] < \commitVC(\tvar_{2})[d].
  \]
\end{proof}

\begin{applemma} \label{lemma:heartbeat-replication-order}
  Let $p^{m}_{d}$ be a replica in data center $d$.
  Consider a heartbeat $\knownVC^{m}_{d}(\realtime_{1})[d]$
  sent by $p^{m}_{d}$ at time $\realtime_{1}$
  (line~\code{\ref{alg:unistore-replication}}{\ref{line:propagate-call-heartbeat}}).
  Let $\tvar$ be a transaction replicated by $p^{m}_{d}$ at time $\realtime_{2}$
  (line~\code{\ref{alg:unistore-replication}}{\ref{line:propagate-call-replicate}}).
  Then
  \[
    \realtime_{1} < \realtime_{2} \iff \knownVC^{m}_{d}(\realtime_{1})[d] < \commitVC(\tvar)[d].
  \]
\end{applemma}

\begin{proof} \label{proof:heartbeat-replication-order}
  We first show that
  \[
    \realtime_{1} < \realtime_{2} \implies \knownVC^{m}_{d}(\realtime_{1})[d] < \commitVC(\tvar)[d].
  \]
  Assume that $\realtime_{1} < \realtime_{2}$.
  We distinguish between two cases according to whether
  \[
    \langle \tidselector(\tvar), \_, \_, \_ \rangle \in \committedcausal^{m}_{d}(\realtime_{1})[d].
  \]
  \begin{itemize}
    \item $\textsc{Case I}$:
      $\langle \tidselector(\tvar), \_, \_, \_ \rangle \in \committedcausal^{m}_{d}(\realtime_{1})[d]$.
      By line~\code{\ref{alg:unistore-replication}}{\ref{line:propagate-txs}},
      \[
        \commitVC(\tvar)[d] > \knownVC^{m}_{d}(\realtime_{1})[d].
      \]
    \item $\textsc{Case II}$:
      $\langle \tidselector(\tvar), \_, \_, \_ \rangle \notin \committedcausal^{m}_{d}(\realtime_{1})[d]$.
      Thus, $\tvar$ is committed at $p^{m}_{d}$ after time $\realtime_{1}$.
      By Lemma~\ref{lemma:knownvc-commitvc-d},
      \[
        \commitVC(\tvar)[d] > \knownVC^{m}_{d}(\realtime_{1})[d].
      \]
  \end{itemize}

  Next we show that (note that $\realtime_{1} \neq \realtime_{2}$)
  \[
    \realtime_{2} < \realtime_{1} \implies \commitVC(\tvar)[d] \le \knownVC^{m}_{d}(\realtime_{1})[d].
  \]
  Since $\tvar$ is replicated by $p^{m}_{d}$ at time $\realtime_{2}$,
  by line~\code{\ref{alg:unistore-replication}}{\ref{line:propagate-txs}},
  \[
    \commitVC(\tvar)[d] \le \knownVC^{m}_{d}(\realtime_{2})[d].
  \]
  Assume that $\realtime_{2} < \realtime_{1}$.
  By Lemma~\ref{lemma:knownvc-d-nondecreasing},
  \[
    \knownVC^{m}_{d}(\realtime_{2})[d] \le \knownVC^{m}_{d}(\realtime_{1})[d].
  \]
  Putting it together yields
  \[
    \commitVC(\tvar)[d] \le \knownVC^{m}_{d}(\realtime_{1})[d].
  \]
\end{proof}

\begin{applemma} \label{lemma:committedcausal-i}
  Let $p^{m}_{d}$ be a replica in data center $d$. Then
  \begin{align*}
    &\forall i \neq d.\;
      \forall \langle \tidselector(\tvar), \_, \_, \_ \rangle \in \committedcausal^{m}_{d}[i]. \\
        &\quad \commitVC(\tvar)[i] \le \knownVC^{m}_{d}[i].
  \end{align*}
\end{applemma}

\begin{proof} \label{proof:committedcausal-i}
  By lines~\code{\ref{alg:unistore-replication}}{\ref{line:replicate-committedcausal}}
  and \code{\ref{alg:unistore-replication}}{\ref{line:replicate-knownvc}}
  and Lemma~\ref{lemma:knownvc-i-nondecreasing}.
\end{proof}

\begin{applemma} \label{lemma:globalmatrix-nondecreasing}
  For $j \neq d$ and $i \notin \set{d, j}$,
  $\globalmatrix^{m}_{d}[i][j]$ at any replica $p^{m}_{d}$
  in data center $d$ is non-decreasing.
\end{applemma}

\begin{proof} \label{proof:globalmatrix-nondecreasing}
  Note that $\globalmatrix^{m}_{d}[i][j]$ can be updated only
  at line~\code{\ref{alg:unistore-clock}}{\ref{line:knownvcglobal-globalmatrix}}.
  Therefore, by Lemma~\ref{lemma:knownvc-i-nondecreasing},
  it is non-decreasing.
\end{proof}

\begin{applemma} \label{lemma:forwarding-order}
  Let $p^{m}_{d}$ be a replica in data center $d$.
  Let $\tvar_{1}$ and $\tvar_{2}$ be two transactions
  that originate at data center $j \neq d$
  and are forwarded by $p^{m}_{d}$ to
  sibling replica $p^{m}_{i}$ in data center $i \notin \set{d, j}$
  at time $\realtime_{1}$ and $\realtime_{2}$
  (line~\code{\ref{alg:unistore-replication}}{\ref{line:forward-call-replicate}}),
  respectively. Then
  \[
    \realtime_{1} < \realtime_{2} \implies \commitVC(\tvar_{1})[j] < \commitVC(\tvar_{2})[j].
  \]
\end{applemma}

\begin{proof} \label{proof:forwarding-order}
  Since $\tvar_{1}$ is forwarded by $p^{m}_{d}$ at time $\realtime_{1}$,
  by line~\code{\ref{alg:unistore-replication}}{\ref{line:forward-txs}},
  \[
    \langle \tidselector(\tvar_{1}), \_, \_, \_ \rangle \in \committedcausal^{m}_{d}(\realtime_{1})[j].
  \]
  By Lemmas~\ref{lemma:committedcausal-i} and \ref{lemma:knownvc-i-nondecreasing},
  \begin{align}
    \commitVC(\tvar_{1})[j] \le \knownVC^{m}_{d}(\realtime_{1})[j].
    \label{eqn:tid1-knownvc}
  \end{align}
  Assume that $\realtime_{1} < \realtime_{2}$.
  We first argue that
  \begin{align}
    \langle \tidselector(\tvar_{2}), \_, \_, \_ \rangle \notin \committedcausal^{m}_{d}(\realtime_{1})[j].
    \label{eqn:tid2-committedcausal-t1}
  \end{align}
  Otherwise, by line~\code{\ref{alg:unistore-replication}}{\ref{line:forward-txs}},
  \[
    \commitVC(\tvar_{2})[j] \le \globalmatrix^{m}_{d}(\realtime_{1})[i][j].
  \]
  By Lemma~\ref{lemma:globalmatrix-nondecreasing},
  \[
    \commitVC(\tvar_{2})[j] \le \globalmatrix^{m}_{d}(\realtime_{2})[i][j].
  \]
  Therefore, by line~\code{\ref{alg:unistore-replication}}{\ref{line:forward-txs}},
  $\tvar_{2}$ would not be forwarded by $p^{m}_{d}$ to $p^{m}_{i}$ at time $\realtime_{2}$.
  Thus, (\ref{eqn:tid2-committedcausal-t1}) holds.
  Since $\tvar_{2}$ is forwarded by $p^{m}_{d}$ to $p^{m}_{i}$ at time $\realtime_{2}$,
  \[
    \langle \tidselector(\tvar_{2}), \_, _, \_ \rangle \in \committedcausal^{m}_{d}(\realtime_{2})[j].
  \]
  By Lemma~\ref{lemma:knownvc-i-nondecreasing}
  and line~\code{\ref{alg:unistore-replication}}{\ref{line:replicate-precondition}},
  \begin{align}
    \commitVC(\tvar_{2})[j] > \knownVC^{m}_{d}(\realtime_{1})[j].
    \label{eqn:tid2-knownvc}
  \end{align}
  Putting (\ref{eqn:tid1-knownvc}) and (\ref{eqn:tid2-knownvc}) together yields
  \[
    \commitVC(\tvar_{1})[j] < \commitVC(\tvar_{2})[j].
  \]
\end{proof}

\begin{applemma} \label{lemma:heartbeat-forwarding-order}
  Let $p^{m}_{d}$ be a replica in data center $d$.
  Consider a heartbeat $\knownVC^{m}_{d}(\realtime_{1})[j]$ ($j \neq d$)
  sent by $p^{m}_{d}$ to sibling replica $p^{m}_{i}$
  in data center $i \notin \set{d, j}$ at time $\realtime_{1}$
  (line~\code{\ref{alg:unistore-replication}}{\ref{line:forward-call-heartbeat}}).
  Let $\tvar$ be a transaction that originates at data center $j$
  and is forwarded by $p^{m}_{d}$ to $p^{m}_{i}$ at time $\realtime_{2}$
  (line~\code{\ref{alg:unistore-replication}}{\ref{line:forward-call-replicate}}).
  Then
  \[
    \realtime_{1} < \realtime_{2} \iff \knownVC^{m}_{d}(\realtime_{1})[j] < \commitVC(\tvar)[j].
  \]
\end{applemma}

\begin{proof} \label{proof:heartbeat-forwarding-order}
  We first show that
  \[
    \realtime_{1} < \realtime_{2} \implies \knownVC^{m}_{d}(\realtime_{1})[j] < \commitVC(\tvar)[j].
  \]
  Assume that $\realtime_{1} < \realtime_{2}$.
  We first argue that
  \begin{align}
    \langle \tidselector(\tvar), \_, \_, \_ \rangle \notin \committedcausal^{m}_{d}(\realtime_{1})[j].
    \label{eqn:tid-notin-committedcausal-t1}
  \end{align}
  Otherwise, since $\tvar$ is not forwarded at time $\realtime_{1}$,
  by line~\code{\ref{alg:unistore-replication}}{\ref{line:forward-txs}},
  \[
    \commitVC(\tvar)[j] \le \globalmatrix^{m}_{d}(\realtime_{1})[i][j].
  \]
  By Lemma~\ref{lemma:globalmatrix-nondecreasing},
  \[
    \commitVC(\tvar)[j] \le \globalmatrix^{m}_{d}(\realtime_{2})[i][j].
  \]
  Therefore, by line~\code{\ref{alg:unistore-replication}}{\ref{line:forward-txs}},
  $\tvar$ would not be forwarded by $p^{m}_{d}$ to $p^{m}_{i}$ at time $\realtime_{2}$.
  Thus, (\ref{eqn:tid-notin-committedcausal-t1}) holds.
  Since $\tvar$ is forwarded by $p^{m}_{d}$ to $p^{m}_{i}$ at time $\realtime_{2}$,
  \[
    \langle \tidselector(\tvar), \_, \_, \_ \rangle \in \committedcausal^{m}_{d}(\realtime_{2})[j].
  \]
  By Lemma~\ref{lemma:knownvc-i-nondecreasing}
  and line~\code{\ref{alg:unistore-replication}}{\ref{line:replicate-precondition}},
  \[
    \knownVC^{m}_{d}(\realtime_{1})[j] < \commitVC(\tvar)[j].
  \]

  Next we show that (note that $\realtime_{1} \neq \realtime_{2}$)
  \[
    \realtime_{2} < \realtime_{1} \implies \commitVC(\tvar)[j] \le \knownVC^{m}_{d}(\realtime_{1})[j].
  \]
  Since $\tvar$ is forwarded by $p^{m}_{d}$ to $p^{m}_{i}$ at time $\realtime_{2}$,
  by line~\code{\ref{alg:unistore-replication}}{\ref{line:forward-txs}},
  \[
    \langle \tidselector(\tvar), \_, \_, \_ \rangle \in \committedcausal^{m}_{d}(\realtime_{2})[j].
  \]
  By Lemmas~\ref{lemma:committedcausal-i} and \ref{lemma:knownvc-i-nondecreasing},
  \[
    \commitVC(\tvar)[j] \le \knownVC^{m}_{d}(\realtime_{2})[j].
  \]
  Assume that $\realtime_{2} < \realtime_{1}$.
  By Lemma~\ref{lemma:knownvc-i-nondecreasing},
  \[
    \knownVC^{m}_{d}(\realtime_{2})[j] \le \knownVC^{m}_{d}(\realtime_{1})[j].
  \]
  Putting it together yields
  \[
    \commitVC(\tvar)[j] \le \knownVC^{m}_{d}(\realtime_{1})[j].
  \]
\end{proof}

\begin{applemma} \label{lemma:replication-knownvc}
  Let $\tvar \in \causaltxs$ be a causal transaction
  that originates at data center $i$ and accesses partition $m$.
  If
  \[
    \commitVC(\tvar)[i] \le \knownVC^{m}_{d}[i]
  \]
  for replica $p^{m}_{d}$ in data center $d \neq i$,
  then
  \[
    \log(\tvar)[m] \subseteq \oplog^{m}_{d}.
  \]
\end{applemma}

\begin{proof} \label{proof:replication-knownvc}
  Note that for $i \in \D \setminus \set{d}$,
  $\knownVC^{m}_{d}[i]$ can be updated
  only at lines~\code{\ref{alg:unistore-replication}}{\ref{line:replicate-knownvc}}
  or \code{\ref{alg:unistore-replication}}{\ref{line:heartbeat-knownvc}}
  due to replication of transactions or heartbeats respectively,
  either directly from data center $i$
  (line~\code{\ref{alg:unistore-replication}}{\ref{line:function-propagate}})
  or indirectly from a third data center $j \neq i$
  (line~\code{\ref{alg:unistore-replication}}{\ref{line:function-forward}}).

  We proceed by induction on the length of the execution.
  In the following, for replica $p^{m}_{d}$ in data center $d \in \D$,
  we denote the value of $\knownVC^{m}_{d}$ (resp. $\oplog^{m}_{d}$)
  after $k$ steps in an execution
  by $\knownVC^{m}_{d}(k)$ (resp. $\oplog^{m}_{d}(k)$).
  \begin{itemize}
    \item \emph{Base Case.} $k = 0$.
      It holds trivially,
      since for replica $p^{m}_{d}$ in any data center $d \neq i$,
      \[
        \knownVC^{m}_{d}(0)[i] = 0.
      \]
    \item \emph{Induction Hypothesis.}
      Suppose that for any execution of length $k$, we have
      \begin{align*}
        &\forall d \in \D \setminus \set{i}.\;
          \forall \tvar \in \causaltxs.\; \\
            &\quad \big(\commitVC(\tvar)[i] \le \knownVC^{m}_{d}(k)[i] \\
            &\quad\quad \implies \log(\tvar)[m] \subseteq \oplog^{m}_{d}(k)\big).
      \end{align*}
    \item \emph{Induction Step.}
      Consider an execution of length $k + 1$.
      If the $(k+1)$-st step of this execution does not update
      $\knownVC^{m}_{d}[i]$ for replica $p^{m}_{d}$ in any data center $d \neq i$,
      then by the induction hypothesis,
      \begin{align*}
        &\forall d \in \D \setminus \set{i}.\;
          \forall \tvar \in \causaltxs.\; \\
            &\quad \big(\commitVC(\tvar)[i] \le \knownVC^{m}_{d}(k+1)[i] \\
            &\quad\quad \implies \log(\tvar)[m] \subseteq \oplog^{m}_{d}(k+1)\big).
      \end{align*}
      Otherwise, we perform a case analysis according to
      how $\knownVC^{m}_{d}[i]$ of replica $p^{m}_{d}$ in data center $d \neq i$
      is updated in the $(k+1)$-st step.
      \begin{itemize}
        \item \textsc{Case I:}
          $\knownVC^{m}_{d}[i]$ is updated due to delivery of a message
          from data center $i$.
          By Lemmas~\ref{lemma:knownvc-d-nondecreasing},
          \ref{lemma:replication-order}, and \ref{lemma:heartbeat-replication-order},
          local transactions and heartbeats are propagated by $p^{m}_{i}$ to sibling replicas
          in increasing order of their local timestamps $\commitvc[i]$
          and $\knownVC^{m}_{i}[i]$ values.
          Therefore, by Assumption~\ref{assumption:message}
          and the induction hypothesis,
          \begin{align*}
            &\forall \tvar \in \causaltxs.\; \\
              &\quad \big(\commitVC(\tvar)[i] \le \knownVC^{m}_{d}(k+1)[i] \\
              &\quad\quad \implies \log(\tvar)[m] \subseteq \oplog^{m}_{d}(k+1)\big).
          \end{align*}
        \item \textsc{Case II:}
          $\knownVC^{m}_{d}[i]$ is updated due to delivery of a message
          from a third data center $j \neq i$.
          By Lemmas~\ref{lemma:knownvc-i-nondecreasing},
          \ref{lemma:forwarding-order}, and \ref{lemma:heartbeat-forwarding-order},
          transactions originating at data center $i$ and heartbeats are forwarded
          by some replica, say $p^{m}_{j} (j \neq i)$,
          to sibling replicas in increasing order of
          their local timestamps $\commitvc[i]$ and $\knownVC^{m}_{j}[i]$ values.
          Therefore, by Assumption~\ref{assumption:message}
          and the induction hypothesis,
          \begin{align*}
            &\forall \tvar \in \causaltxs.\; \\
              &\quad \big(\commitVC(\tvar)[i] \le \knownVC^{m}_{d}(k+1)[i] \\
                &\quad\quad \implies \log(\tvar)[m] \subseteq \oplog^{m}_{d}(k+1)\big).
          \end{align*}
      \end{itemize}
  \end{itemize}
\end{proof}

\begin{applemma}[\prop{1}] \label{lemma:knownvc-causal}
  Let $\tvar \in \causaltxs$ be a causal transaction
  that originates at data center $i$
  and accesses partition $m$.
  If
  \[
    \commitVC(\tvar)[i] \le \knownVC^{m}_{d}[i]
  \]
  for replica $p^{m}_{d}$ in data center $d$,
  then
  \[
    \log(\tvar)[m] \subseteq \oplog^{m}_{d}.
  \]
\end{applemma}

\begin{proof} \label{proof:knownvc-causal}
  By Lemmas~\ref{lemma:knownvc-local-d} and \ref{lemma:replication-knownvc}.
\end{proof}
\subsubsection{Properties of $\stableVC$}
\label{sss:stablevc}

\begin{applemma} \label{lemma:stablevc-nondecreasing}
  For $i \in \D$, $\stableVC^{m}_{d}[i]$
  at any replica $p^{m}_{d}$ in data center $d$ is non-decreasing.
\end{applemma}

\begin{proof} \label{proof:stablevc-nondecreasing}
  Note that $\stableVC^{m}_{d}[i]$ ($i \in \D$) can be updated
  only at line~\code{\ref{alg:unistore-clock}}{\ref{line:knownvclocal-stablevc-causal}}.
  By Lemma~\ref{lemma:knownvc-nondecreasing}
  and Assumption~\ref{assumption:message},
  $\stableVC^{m}_{d}[i]$ is non-decreasing.
\end{proof}

\begin{applemma}[\prop{2}] \label{lemma:stablevc-knownvc}
  For any replica $p^{m}_{d}$ in data center $d$,
  \[
    \forall i \in \D.\; \forall n \in \P.\;
      \stableVC^{m}_{d}[i] \le \knownVC^{n}_{d}[i].
  \]
\end{applemma}

\begin{proof} \label{proof:stablevc-knownvc}
  Note that $\stableVC^{m}_{d}[i]$ ($i \in \D$)
  can be updated only
  at line~\code{\ref{alg:unistore-clock}}{\ref{line:knownvclocal-stablevc-causal}}.
  By the way $\stableVC^{m}_{d}[i]$ is updated
  and Lemmas~\ref{lemma:knownvc-d-nondecreasing}
  and \ref{lemma:knownvc-i-nondecreasing},
  \[
    \forall n \in \P.\; \stableVC^{m}_{d}[i] \le \knownVC^{n}_{d}[i].
  \]
\end{proof}

\begin{applemma} \label{lemma:replication-stablevc}
  Let $\tvar \in \causaltxs$ be a causal transaction
  that originates at data center $i$
  and accesses partition $n$.
  If
  \[
    \commitVC(\tvar)[i] \le \stableVC^{m}_{d}[i]
  \]
  for some replica $p^{m}_{d}$ in data center $d$,
  then
  \[
    \log(\tvar)[n] \subseteq \oplog^{n}_{d}.
  \]
\end{applemma}

\begin{proof} \label{proof:replication-stablevc}
  By Lemma~\ref{lemma:stablevc-knownvc},
  \[
    \stableVC^{m}_{d}[i] \le \knownVC^{n}_{d}[i].
  \]
  Therefore,
  \[
    \commitVC(\tvar)[i] \le \knownVC^{n}_{d}[i].
  \]
  By Lemma~\ref{lemma:replication-knownvc},
  \[
    \log(\tvar)[n] \subseteq \oplog^{n}_{d}.
  \]
\end{proof}
\subsubsection{Properties of $\uniformVC$}
\label{sss:uniformvc}

\begin{applemma} \label{lemma:uniformvc-nondecreasing}
  For $i \in \D$, $\uniformVC^{m}_{d}[i]$
  at any replica $p^{m}_{d}$ in data center $d$ is non-decreasing.
\end{applemma}

\begin{proof} \label{proof:uniformvc-nondecreasing}
  Note that whenever $\uniformVC^{m}_{d}[i]$ is updated
  at lines~\code{\ref{alg:unistore-coord}}{\ref{line:start-uniformvc}},
  \code{\ref{alg:unistore-replica}}{\ref{line:readkey-uniformvc}},
  \code{\ref{alg:unistore-replica}}{\ref{line:preparecausal-uniformvc}},
  or \code{\ref{alg:unistore-clock}}{\ref{line:stablevc-uniformvc}},
  we take the maximum of it and some scalar value.
\end{proof}

\begin{applemma} \label{lemma:pastvc-uniformvc-except-d}
  Let $e \in E$ be an event issued by client $\cl$
  to replica $p^{m}_{d}$ in data center $d$. Then
  \begin{align*}
    &e \in E \setminus \Fence \implies \\
      &\quad \forall i \in \D \setminus \set{d}.\;
      (\pastVC_{\cl})_{e}[i] \le (\uniformVC^{m}_{d})_{e}[i],
  \end{align*}
  and
  \[
    e \in \Fence \implies (\pastVC_{\cl})_{e}[d] \le (\uniformVC^{m}_{d})_{e}[d].
  \]
\end{applemma}

\begin{proof} \label{proof:pastvc-uniformvc-except-d}
  We perform a case analysis according to the type of event $e$.
  \begin{itemize}
    \item $\textsc{Case I}$: $e \in S$.
      By line~\code{\ref{alg:unistore-coord}}{\ref{line:start-uniformvc}},
      \[
        \forall i \in \D \setminus \set{d}.\;
          (\pastVC_{\cl})_{e}[i] \le (\uniformVC^{m}_{d})_{e}[i].
      \]
    \item $\textsc{Case II}$: $e \in R \cup U$.
      In this case,
      \[
        (\pastVC_{\cl})_{e} = (\pastVC_{\cl})_{\startoftx(e)}.
      \]
      By \textsc{Case I},
      \[
        \forall i \in \D \setminus \set{d}.\;
          (\pastVC_{\cl})_{\startoftx(e)}[i] \le (\uniformVC^{m}_{d})_{\startoftx(e)}[i].
      \]
      By Lemma~\ref{lemma:uniformvc-nondecreasing},
      \[
        \forall i \in \D \setminus \set{d}. \\
          (\uniformVC^{m}_{d})_{\startoftx(e)} \le (\uniformVC^{m}_{d})_{e}.
      \]
      Putting it together yields
      \[
        \forall i \in \D \setminus \set{d}.\;
          (\pastVC_{\cl})_{e}[i] \le (\uniformVC^{m}_{d})_{e}[i].
      \]
    \item $\textsc{Case III}$: $e \in C_{\causalentry}$.
      By line~\code{\ref{alg:unistore-client}}{\ref{line:commitcausaltx-pastvc}},
      \[
        (\pastVC_{\cl})_{e} = vc_{(\commitcausaltx, e)}.
      \]
      By lines~\code{\ref{alg:unistore-coord}}{\ref{line:commitcausal-return-ro}},
      \code{\ref{alg:unistore-coord}}{\ref{line:commitcausal-commitvc}},
      and \code{\ref{alg:unistore-coord}}{\ref{line:commitcausal-return}},
      \begin{align*}
        &\forall i \in \D \setminus \set{d}. \\
          &\quad vc_{(\commitcausaltx, e)}[i] = (\snapVC^{m}_{d})_{e}[\txfunc(e)][i].
      \end{align*}
      By line~\code{\ref{alg:unistore-coord}}{\ref{line:start-snapvc}},
      \begin{align*}
        &\forall i \in \D \setminus \set{d}. \\
          &\quad (\snapVC^{m}_{d})_{e}[\txfunc(e)][i] = (\uniformVC^{m}_{d})_{\startoftx(e)}[i].
      \end{align*}
      By Lemma~\ref{lemma:uniformvc-nondecreasing},
      \[
        \forall i \in \D \setminus \set{d}. \\
          (\uniformVC^{m}_{d})_{\startoftx(e)} \le (\uniformVC^{m}_{d})_{e}.
      \]
      Putting it together yields
      \[
        \forall i \in \D \setminus \set{d}.\;
          (\pastVC_{\cl})_{e}[i] \le (\uniformVC^{m}_{d})_{e}[i].
      \]
    \item $\textsc{Case IV}$: $e \in C_{\strongentry}$.
      By line~\code{\ref{alg:unistore-client}}{\ref{line:commitstrongtx-pastvc}},
      \[
        (\pastVC_{\cl})_{e} = vc_{(\commitstrongtx, e)}.
      \]
      By (\ref{eqn:gcf-commitvc}),
      \begin{align*}
        &\forall i \in \D \setminus \set{d}. \\
          &\quad vc_{(\commitstrongtx, e)}[i] = (\snapVC^{m}_{d})_{e}[\txfunc(e)][i].
      \end{align*}
      Therefore, similar to \textsc{Case III}, we have
      \[
        \forall i \in \D \setminus \set{d}.\;
          (\pastVC_{\cl})_{e}[i] \le (\uniformVC^{m}_{d})_{e}[i].
      \]
    \item $\textsc{Case V}$: $e \in \Fence$.
      By line~\code{\ref{alg:unistore-replica}}{\ref{line:uniformbarrier-wait-uniformvc-d}},
      \[
        (\pastVC_{\cl})_{e}[d] \le (\uniformVC^{m}_{d})_{e}[d].
      \]
    \item $\textsc{Case VI}$: $e \in \Attach$.
      By line~\code{\ref{alg:unistore-replica}}{\ref{line:attach-wait-condition}},
      \[
        \forall i \in \D \setminus \set{d}.\;
          (\pastVC_{\cl})_{e}[i] \le (\uniformVC^{m}_{d})_{e}[i].
      \]
  \end{itemize}
\end{proof}

\begin{applemma} \label{lemma:pastvc-uniformvc}
  Let $\cl$ be a client and $d \triangleq \cldc(\cl)$.
  At any time,
  \[
    \forall i \in \D \setminus \set{d}.\;
      \pastVC_{\cl}[i] \le \uniformVC^{m}_{d}[i]
  \]
  for some replica $p^{m}_{d}$ in data center $d$.
\end{applemma}

\begin{proof} \label{proof:pastvc-uniformvc}
  By a simple induction on the number of events that $\cl$ issues
  and Lemmas~\ref{lemma:pastvc-uniformvc-except-d} and
  \ref{lemma:uniformvc-nondecreasing}.
\end{proof}

\begin{applemma} \label{lemma:snapshotvc-uniformvc}
  Let $\tvar$ be a transaction that originates at data center $d$.
  At any time,
  \[
    \forall i \in \D \setminus \set{d}.\;
      \snapshotVC(\tvar)[i] \le \uniformVC^{m}_{d}[i]
  \]
  for some replica $p^{m}_{d}$ in data center $d$.
\end{applemma}

\begin{proof} \label{proof:snapshotvc-uniformvc}
  By line~\code{\ref{alg:unistore-coord}}{\ref{line:start-snapvc}}
  and Lemma~\ref{lemma:uniformvc-nondecreasing}.
\end{proof}

\begin{applemma}[\prop{3}] \label{lemma:uniformvc-knownvc-f+1}
  For any replica $p^{m}_{d}$ in data center $d$,
  \begin{align*}
    &\forall i \in \D.\; \exists g \subseteq \D.\; \Big(
       |g| \ge f + 1 \land d \in g\; \land \\
      &\; \big(\forall j \in g.\; \forall n \in \P.\;
        \uniformVC^{m}_{d}[i] \le \knownVC^{n}_{j}[i] \big)\Big).
  \end{align*}
\end{applemma}

\begin{proof} \label{proof:uniformvc-knownvc-f+1}
  Fix $i \in \D$.
  We proceed by induction on the length of the execution.
  In the following, we denote the value of
  $\knownVC^{m}_{d}$, $\stableVC^{m}_{d}$, $\uniformVC^{m}_{d}$,
  $\stablematrix^{m}_{d}$, and $\pastVC_{\cl}$ (for some client $\cl$)
  after $k$ steps of an execution by
  $\knownVC^{m}_{d}(k)$, $\stableVC^{m}_{d}(k)$, $\uniformVC^{m}_{d}(k)$,
  $\stablematrix^{m}_{d}(k)$, and $\pastVC_{\cl}(k)$, respectively.
  \begin{itemize}
    \item {\it Base Case.} $k = 0$. It holds trivially since
      \[
        \uniformVC^{m}_{d}(0)[i] = 0.
      \]
    \item {\it Induction Hypothesis.}
      Suppose that for any execution of length $k$,
      for any replica $p^{m}_{d}$ in data center $d$,
      \begin{align*}
        &\exists g \subseteq \D.\; |g| \ge f + 1 \land d \in g\; \land \\
          &\quad \big(\forall j \in g.\; \forall 1 \le n \le N.\; \\
            &\qquad \uniformVC^{m}_{d}(k)[i] \le \knownVC^{n}_{j}(k)[i] \big).
      \end{align*}
    \item {\it Induction Step.}
      Consider an execution of length $k + 1$.
      If the $(k+1)$-st step of this execution does not update
      $\uniformVC^{m}_{d}[i]$ for any replica $p^{m}_{d}$ in data center $d$,
      then by the induction hypothesis and Lemma~\ref{lemma:knownvc-nondecreasing},
      \begin{align*}
        &\exists g \subseteq \D.\; |g| \ge f + 1 \land d \in g\; \land \\
          &\quad \big(\forall j \in g.\; \forall 1 \le n \le N.\; \\
            &\qquad \uniformVC^{m}_{d}(k + 1)[i] = \uniformVC^{m}_{d}(k)[i] \\
            &\phantom{\qquad \uniformVC^{m}_{d}(k + 1)[i]}
              \le \knownVC^{n}_{j}(k)[i] \\
            &\phantom{\qquad \uniformVC^{m}_{d}(k + 1)[i]}
              \le \knownVC^{n}_{j}(k + 1)[i] \big).
      \end{align*}
      Otherwise, we perform a case analysis
      according to how $\uniformVC^{m}_{d}[i]$ is updated.
      \begin{itemize}
        \item $\textsc{Case I}$: $\uniformVC^{m}_{d}[i]$ is updated
          at line~\code{\ref{alg:unistore-clock}}{\ref{line:stablevc-uniformvc}}.
          By line~\code{\ref{alg:unistore-clock}}{\ref{line:stablevc-g}},
          \begin{align}
            &\exists g' \subseteq \D.\; |g'| \ge f + 1 \land d \in g'\; \land
              \label{eqn:g-prime} \\
              &\quad \uniformVC^{m}_{d}(k+1)[i] = \nonumber \\
                &\qquad \max\big\{\uniformVC^{m}_{d}(k)[i], \nonumber \\
                  &\qquad\qquad\;\; \min_{j \in g'} \stablematrix^{m}_{d}(k+1)[j][i]\big\}. \nonumber
          \end{align}
          By the induction hypothesis and Lemma~\ref{lemma:knownvc-i-nondecreasing},
          \begin{align}
            &\exists g'' \subseteq \D.\; |g''| \ge f + 1 \land d \in g''\; \land
              \label{eqn:g-prime-prime} \\
              &\quad \big(\forall j \in g''.\; \forall n \in \P. \nonumber \\
                &\qquad \uniformVC^{m}_{d}(k)[i] \le \knownVC^{n}_{j}(k)[i] \nonumber \\
                &\phantom{\qquad \uniformVC^{m}_{d}(k)[i] }
                  \le \knownVC^{n}_{j}(k+1)[i]\big). \nonumber
          \end{align}
          By Lemma~\ref{lemma:stablevc-nondecreasing},
          for the particular $g' \subseteq \D$ in (\ref{eqn:g-prime}),
          \begin{align}
            \forall j \in g'.\; &\stablematrix^{m}_{d}(k+1)[j][i]
              \label{eqn:stablematrix-j-i} \\
              &\le \stableVC^{m}_{j}(k+1)[i].
              \nonumber
          \end{align}
          By Lemmas~\ref{lemma:stablevc-knownvc} and \ref{lemma:knownvc-i-nondecreasing},
          for any replica $p^{m}_{j}$ in data center $j$,
          \begin{align}
            &\forall n \in \P.\; \stableVC^{m}_{j}(k+1)[i]
              \label{eqn:stablevc-j-i}\\
              &\qquad\;\;\; \le \knownVC^{n}_{j}(k+1)[i]. \nonumber
          \end{align}
          Therefore, for the particular $g' \subseteq \D$ in $(\ref{eqn:g-prime})$,
          \begin{align}
            &\forall j' \in g'.\; \forall n \in \P.\;
              \min_{j \in g'} \stablematrix^{m}_{d}(k+1)[j][i]
              \nonumber\\
              &\qquad\qquad\qquad\;\; \le \knownVC^{n}_{j'}(k+1)[i]. \label{eqn:j-prime}
          \end{align}
          By (\ref{eqn:g-prime}), (\ref{eqn:g-prime-prime}),
          and (\ref{eqn:j-prime}),
          we can either take $g = g'$ in (\ref{eqn:g-prime})
          or $g = g''$ in (\ref{eqn:g-prime-prime}) such that
          \begin{align*}
            &\forall j \in g.\; \forall n \in \P. \\
              &\;\; \uniformVC^{m}_{d}(k+1)[i] \le \knownVC^{n}_{j}(k+1)[i].
          \end{align*}
          Therefore,
          \begin{align*}
            &\exists g \subseteq \D.\; |g| \ge f + 1 \land d \in g\; \land \\
              &\; \big(\forall j \in g.\; \forall n \in \P.\; \\
                &\quad \uniformVC^{m}_{d}(k+1)[i] \le \knownVC^{n}_{j}(k+1)[i] \big).
          \end{align*}
        \item $\textsc{Case II}$: $\uniformVC^{m}_{d}[i]$
          ($i \in \D \setminus \set{d}$) is updated
          at line~\code{\ref{alg:unistore-coord}}{\ref{line:start-uniformvc}}.
          Then there exists some client $\cl$ with $d = \cldc(\cl)$ such that
          \begin{align}
            &\uniformVC^{m}_{d}(k+1)[i] = \label{eqn:uniformvc-pastvc} \\
              &\quad \max\big\{\pastVC_{\cl}(k)[i], \uniformVC^{m}_{d}(k)[i]\big\}.
              \nonumber
          \end{align}
          By the induction hypothesis and Lemma~\ref{lemma:knownvc-i-nondecreasing},
          \begin{align}
            &\exists g' \subseteq \D.\; |g'| \ge f + 1 \land d \in g'\; \land
              \label{eqn:g-prime-2} \\
              &\quad \big(\forall j \in g'.\; \forall n \in \P. \nonumber \\
                &\qquad \uniformVC^{m}_{d}(k)[i] \le \knownVC^{n}_{j}(k)[i] \nonumber \\
                &\phantom{\qquad \uniformVC^{m}_{d}(k)[i] }
                  \le \knownVC^{n}_{j}(k+1)[i]\big). \nonumber
          \end{align}
          By Lemma~\ref{lemma:pastvc-uniformvc}, the induction hypothesis,
          and Lemma~\ref{lemma:knownvc-nondecreasing},
          \begin{align}
            &\exists g'' \subseteq \D.\; |g''| \ge f + 1 \land d \in g''\; \land
              \label{eqn:g-prime-prime-2} \\
              &\quad \big(\forall j \in g''.\; \forall n \in \P. \nonumber \\
                &\qquad \pastVC_{\cl}(k)[i] \le \knownVC^{n}_{j}(k+1)[i]. \nonumber
          \end{align}
          By (\ref{eqn:uniformvc-pastvc}), (\ref{eqn:g-prime-2}),
          and (\ref{eqn:g-prime-prime-2}),
          we can take $g = g'$ in (\ref{eqn:g-prime-2})
          or $g = g''$ in (\ref{eqn:g-prime-prime-2}) such that
          \begin{align*}
            &\forall j \in g.\; \forall n \in \P.\; \\
              &\;\; \uniformVC^{m}_{d}(k+1)[i] \le \knownVC^{n}_{j}(k+1)[i].
          \end{align*}
          Therefore,
          \begin{align*}
            &\exists g \subseteq \D.\; |g| \ge f + 1 \land d \in g\; \land \\
              &\; \big(\forall j \in g.\; \forall n \in \P.\; \\
                &\;\; \uniformVC^{m}_{d}(k+1)[i] \le \knownVC^{n}_{j}(k+1)[i] \big).
          \end{align*}
        \item $\textsc{Case III}$: $\uniformVC^{m}_{d}[i]$
          ($i \in \D \setminus \set{d}$) is updated
          at lines~\code{\ref{alg:unistore-replica}}{\ref{line:readkey-uniformvc}}
          or \code{\ref{alg:unistore-replica}}{\ref{line:preparecausal-uniformvc}}.
          Therefore, there exists some transaction $\tvar$
          originating at data center $d$ such that
          \begin{align}
            &\uniformVC^{m}_{d}(k+1)[i] = \label{eqn:uniformvc-snapshotvc} \\
              &\quad \max\big\{\snapshotVC(\tvar)[i], \uniformVC^{m}_{d}(k)[i]\big\}.
              \nonumber
          \end{align}
          By the induction hypothesis and Lemma~\ref{lemma:knownvc-nondecreasing},
          \begin{align}
            &\exists g' \subseteq \D.\; |g'| \ge f + 1 \land d \in g'\; \land
              \label{eqn:g-prime-3} \\
              &\quad \big(\forall j \in g'.\; \forall n \in \P. \nonumber \\
                &\qquad \uniformVC^{m}_{d}(k)[i] \le \knownVC^{n}_{j}(k)[i] \nonumber \\
                &\phantom{\qquad \uniformVC^{m}_{d}(k)[i] }
                  \le \knownVC^{n}_{j}(k+1)[i]\big). \nonumber
          \end{align}
          By Lemma~\ref{lemma:snapshotvc-uniformvc}, the induction hypothesis,
          and Lemma~\ref{lemma:knownvc-nondecreasing},
          \begin{align}
            &\exists g'' \subseteq \D.\; |g''| \ge f + 1 \land d \in g''\; \land
              \label{eqn:g-prime-prime-3} \\
              &\quad \big(\forall j \in g''.\; \forall n \in \P. \nonumber \\
                &\qquad \snapshotVC(\tvar)[i] \le \knownVC^{n}_{j}(k+1)[i]. \nonumber
          \end{align}
          By (\ref{eqn:uniformvc-snapshotvc}), (\ref{eqn:g-prime-3}),
          and (\ref{eqn:g-prime-prime-3}),
          we can take $g = g'$ in (\ref{eqn:g-prime-3})
          or $g = g''$ in (\ref{eqn:g-prime-prime-3}) such that
          \begin{align*}
            &\forall j \in g.\; \forall n \in \P.\; \\
              &\;\; \uniformVC^{m}_{d}(k+1)[i] \le \knownVC^{n}_{j}(k+1)[i].
          \end{align*}
          Therefore,
          \begin{align*}
            &\exists g \subseteq \D.\; |g| \ge f + 1 \land d \in g\; \land \\
              &\; \big(\forall j \in g.\; \forall n \in \P.\; \\
                &\quad \uniformVC^{m}_{d}(k+1)[i] \le \knownVC^{n}_{j}(k+1)[i] \big).
          \end{align*}
      \end{itemize}
  \end{itemize}
\end{proof}

\begin{applemma} \label{lemma:uniformvc-knownvc}
  For any replica $p^{m}_{d}$ in data center $d$,
  \[
    \forall i \in \D.\; \forall n \in \P.\;
      \uniformVC^{m}_{d}[i] \le \knownVC^{n}_{d}[i].
  \]
\end{applemma}

\begin{proof} \label{proof:uniformvc-knownvc}
  By Lemma~\ref{lemma:uniformvc-knownvc-f+1}.
\end{proof}

\begin{applemma} \label{lemma:replication-uniformvc}
  Let $\tvar \in \causaltxs$ be a causal transaction
  that originates at data center $i$
  and accesses partition $n$.
  If
  \[
    \commitVC(\tvar)[i] \le \uniformVC^{m}_{d}[i]
  \]
  for some replica $p^{m}_{d}$ in data center $d$,
  then
  \[
    \log(\tvar)[n] \subseteq \oplog^{n}_{d}.
  \]
\end{applemma}

\begin{proof} \label{proof:replication-uniformvc}
  By Lemma~\ref{lemma:uniformvc-knownvc},
  \[
    \uniformVC^{m}_{d}[i] \le \knownVC^{n}_{d}[i].
  \]
  Therefore,
  \[
    \commitVC(\tvar)[i] \le \knownVC^{n}_{d}[i].
  \]
  By Lemma~\ref{lemma:replication-knownvc},
  \[
    \log(\tvar)[n] \subseteq \oplog^{n}_{d}.
  \]
\end{proof}

\begin{applemma} \label{lemma:uniformvc-clock}
  For any replica $p^{m}_{d}$ in data center $d$,
  \[
    \uniformVC^{m}_{d}[d] \le \clockVar^{m}_{d}.
  \]
\end{applemma}

\begin{proof} \label{proof:uniformvc-clock}
  By Lemma~\ref{lemma:uniformvc-knownvc},
  \[
    \uniformVC^{m}_{d}[d] \le \knownVC^{m}_{d}[d].
  \]
  By Lemma~\ref{lemma:knownvc-d-clock},
  \[
    \knownVC^{m}_{d}[d] \le \clockVar^{m}_{d}.
  \]
  Putting it together yields
  \[
    \uniformVC^{m}_{d}[d] \le \clockVar^{m}_{d}.
  \]
\end{proof}
\subsubsection{Properties of $\pastVC$}
\label{sss:cvc}

\begin{applemma} \label{lemma:start-pastvc-snapshotvc}
  Let $e \in S$ be a \start{} event of transaction $\tvar$
  issued by client $\cl$. Then
  \[
    (\pastVC_{\cl})_{e} \le \snapshotVC(\tvar).
  \]
\end{applemma}

\begin{proof} \label{proof:start-pastvc-snapshotvc}
  By Definition~\ref{def:snapshotvc} of $\snapshotVC(\tvar)$
  and lines~\code{\ref{alg:unistore-coord}}{\ref{line:start-uniformvc-index}}--
  \code{\ref{alg:unistore-coord}}{\ref{line:start-snapvc-strong}}.
\end{proof}

\begin{applemma} \label{lemma:pastvc-nondecreasing}
  For $i \in \D$, $\pastVC_{\cl}[i]$ at any client $\cl$
  is non-decreasing.
\end{applemma}

\begin{proof} \label{proof:pastvc-nondecreasing}
  Note that $\pastVC_{\cl}[i]$ ($i \in \D$) is updated only
  at lines~\code{\ref{alg:unistore-client}}{\ref{line:commitcausaltx-pastvc}}
  or \code{\ref{alg:unistore-client}}{\ref{line:commitstrongtx-pastvc}}
  when some transaction is committed.
  Therefore, the lemma holds due to Lemmas~\ref{lemma:snapshotvc-commitvc}
  and \ref{lemma:start-pastvc-snapshotvc}.
\end{proof}


\subsection{Metadata for Strong Transactions} \label{ss:metadata-strong}

\begin{applemma} \label{lemma:knownvc-strong-nondecreasing}
  For any replica $p^{m}_{d}$ in any data center $d$,
  $\knownVC^{m}_{d}[\strongentry]$ is non-decreasing.
\end{applemma}

\begin{proof} \label{proof:knownvc-strong-nondecreasing}
  By \ref{tcs-requirement:deliver-order}
  and line~\code{\ref{alg:unistore-strong-commit}}{\ref{line:deliverupdates-foreach-wbuff}}.
\end{proof}

\begin{applemma} \label{lemma:stablevc-strong-nondecreasing}
  For any replica $p^{m}_{d}$ in any data center $d$,
  $\stableVC^{m}_{d}[\strongentry]$ is non-decreasing.
\end{applemma}

\begin{proof} \label{proof:stablevc-strong-nondecreasing}
  By Lemma~\ref{lemma:knownvc-strong-nondecreasing},
  Assumption~\ref{assumption:message},
  and line~\code{\ref{alg:unistore-clock}}{\ref{line:knownvclocal-stablevc-strong}}.
\end{proof}

\begin{applemma}[\prop{6}] \label{lemma:knownvc-strong}
  Let $\tvar \in \strongtxs$ be a strong transaction
  that originates at data center $i$
  and accesses partition $m$.
  If
  \[
    \commitVC(\tvar)[\strongentry] \le \knownVC^{m}_{d}[\strongentry]
  \]
  for some replica $p^{m}_{d}$ in data center $d$,
  then
  \[
    \log(\tvar)[m] \subseteq \oplog^{m}_{d}.
  \]
\end{applemma}

\begin{proof} \label{proof:knownvc-strong}
  Note that $\knownVC^{m}_{d}[\strongentry]$ can be updated
  only at line~\code{\ref{alg:unistore-strong-commit}}{\ref{line:deliverupdates-knownvc-strongentry}}.
  By \ref{tcs-requirement:deliver-order},
  all committed strong transactions with strong timestamps
  less than or equal to $\knownVC^{m}_{d}[\strongentry]$
  have been delivered to $p^{m}_{d}$.
  By line~\code{\ref{alg:unistore-strong-commit}}{\ref{line:deliverupdates-oplog}},
  \[
    \log(\tvar)[m] \subseteq \oplog^{m}_{d}.
  \]
\end{proof}

\section{The Two Type Systems}
\label{sec:ts}

In this section, we describe the fine-grained and coarse-grained type
systems we work with. Both type systems are set up for higher-order
stateful languages, but differ considerably in how they enforce
IFC. The fine-grained type system, called {\fg}, works on a language
with pervasive side-effects like ML, and associates a security label
with every expression in the language. The coarse-grained type system,
{\cg}, works on a language that isolates state in a monad (like
Haskell's IO monad) and tracks flows coarsely at the granularity of a
monadic computation, not on pure values within a monadic computation.

%% In this section we describe the two type systems that we work with, namely {\fg} and {\cg}. Both the
%% type systems are setup for higher order stateful languages, but they differ considerably in the way
%% they enforce IFC. {\fg} is for a language with pervasive side effects, like ML, and tracks security
%% labels with every value in the language. In contrast, {\cg} is for a language that isolates the
%% side-effects in a monad, like Haskell, and tracks flows only within a monad and not across pure
%% expressions.

Both {\fg} and {\cg} use security labels (denoted by $\llabel$) drawn
from an arbitrary security lattice ($\lattice, \lbelow$). We denote
the least and top element of the lattice by $\bot$ and $\top$
respectively. As usual, the goal of the type systems is to ensure that
outputs labeled $\llabel$ depend only on inputs with security labels
$\llabel$ or lower. \update{For drawing intuitions, we find it
  convenient to think of a confidentiality lattice (labels higher in
  the lattice represent higher confidentiality). However, nothing in
  our technical development is specific to a confidentiality
  lattice---the development works for any security lattice including
  an integrity lattice and a product lattice for confidentiality and
  integrity.}

\input{fg}

\input{cg}

%%% Local Variables:
%%% mode: latex
%%% TeX-master: "main"
%%% End:


\subsection{Session Order}  \label{ss:so}

\begin{applemma} \label{lemma:so-ts}
  \begin{align*}
    &\forall \txevents_{1}, \txevents_{2} \in X.\;
      \txevents_{1} \rel{\so} \txevents_{2} \implies
      \big(\timestamp(\txevents_{1}) \le \timestamp(\txevents_{2}) \\
      &\quad \land \big(\txevents_{2} \in \txs
        \implies \timestamp(\txevents_{1})
          \le \timestamp(\startoftx(\txevents_{2}))
          \le \timestamp(\txevents_{2})\big)\big).
  \end{align*}
\end{applemma}

\begin{proof} \label{proof:so-ts}
  By Definitions~\ref{def:ts-op} and \ref{def:ts-tx} of timestamps
  and Lemma~\ref{lemma:pastvc-nondecreasing},
  \[
    \timestamp(\txevents_{1}) \le \timestamp(\txevents_{2}),
  \]
  and
  \[
    \txevents_{2} \in \txs \implies
      \timestamp(\txevents_{1}) \le \timestamp(\startoftx(\txevents_{2})).
  \]
  Besides, by Lemma~\ref{lemma:snapshotvc-commitvc},
  \[
    \txevents_{2} \in \txs \implies
      \timestamp(\startoftx(\txevents_{2})) \le \timestamp(\txevents_{2}).
  \]
  Therefore,
  \[
    \txevents_{2} \in \txs \implies
      \timestamp(\txevents_{1}) \le \timestamp(\startoftx(\txevents_{2}))
      \le \timestamp(\txevents_{2}).
  \]
\end{proof}


\subsection{Lamport Clocks} \label{ss:lc}

\begin{appdefinition}[Lamport Clocks of Events] \label{def:lc-op}
  Let $e \in C_{\causalentry} \cup C_{\strongentry} \cup \Fence \cup \Attach$
  be an event issued by client $\cl$.
  We define its Lamport clock $\lclock(e)$ as
  \[
    \lclock(e) \triangleq (\lc_{\cl})_{e}.
  \]
  See lines~\code{\ref{alg:unistore-client}}{\ref{line:commitcausaltx-lc}},
  \code{\ref{alg:unistore-client}}{\ref{line:commitstrongtx-lc}},
  \code{\ref{alg:unistore-client}}{\ref{line:fence-lc}},
  and \code{\ref{alg:unistore-client}}{\ref{line:clattach-lc}}
  for \commitcausaltx, \commitstrongtx, \fence, and \clattach{} events, respectively.
\end{appdefinition}

\begin{appdefinition}[Lamport Clocks of Transactions] \label{def:lc-tx}
  The Lamport clock $\lclock(\tvar)$ of a transaction $\tvar$
  is that of its commit event, i.e.,
  \[
    \forall \tvar \in \txs.\;
      \lclock(\tvar) \triangleq \lclock(\commitoftx(\tvar)).
  \]
\end{appdefinition}

\begin{applemma} \label{lemma:lc-extread-commit}
  Let $e \in \extread$ be an external read event issued by client $\cl$.
  Then
  \[
    (\lc_{\cl})_{e} < \lclock(\txfunc(e)).
  \]
\end{applemma}

\begin{proof} \label{proof:lc-extread-commit}
  If $\txfunc(e)$ is a causal transaction,
  by line~\code{\ref{alg:unistore-client}}{\ref{line:commitcausaltx-lc}},
  \[
    \lclock(e) < \lclock(\commitoftx(e)) = \lclock(\txfunc(e)).
  \]
  If $\txfunc(e)$ is a strong transaction,
  by line~\code{\ref{alg:unistore-client}}{\ref{line:commitstrongtx-lc-so}}
  and (\ref{eqn:gcf-lc}),
  \[
    \lclock(e) < \lclock(\commitoftx(e)) = \lclock(\txfunc(e)).
  \]
\end{proof}

\begin{appdefinition}[Lamport Clock Order] \label{def:lco}
  The Lamport clock order $\lcorder$ on $X$
  is the total order defined by their Lamport clocks,
  with their client identifiers for tie-breaking.
\end{appdefinition}

\begin{applemma} \label{lemma:so-lc}
  \[
    \so \subseteq \lcorder.
  \]
\end{applemma}

\begin{proof} \label{proof:so-lc}
  By lines~\code{\ref{alg:unistore-client}}{\ref{line:commitcausaltx-lc}},
  \code{\ref{alg:unistore-client}}{\ref{line:commitstrongtx-lc-so}},
  (\ref{eqn:gcf-lc}),
  \code{\ref{alg:unistore-client}}{\ref{line:commitstrongtx-lc}},
  \code{\ref{alg:unistore-client}}{\ref{line:fence-lc}},
  and \code{\ref{alg:unistore-client}}{\ref{line:clattach-lc}}.
\end{proof}

\begin{applemma} \label{lemma:rf-lc}
  Let $e \in \extread$ be an external read event
  which reads from transaction $\tvar$. Then
  \[
    \tvar \rel{\lcorder} \txfunc(e).
  \]
\end{applemma}

\begin{proof} \label{proof:rf-lc}
  Suppose that $e$ is issued by client $\cl$.
  By line~\code{\ref{alg:unistore-client}}{\ref{line:read-lc}},
  \[
    \lclock(\tvar) \le (\lc_{\cl})_{e}.
  \]
  By Lemma~\ref{lemma:lc-extread-commit},
  \[
    (\lc_{\cl})_{e} < \lclock(\txfunc(e)).
  \]
  Therefore,
  \[
    \lclock(\tvar) < \lclock(\txfunc(e)).
  \]
  By Definition~\ref{def:lco} of $\lcorder$,
  \[
    \tvar \rel{\lcorder} \txfunc(e).
  \]
\end{proof}


\subsection{Visibility Relation}  \label{ss:vis}

\begin{appdefinition}[Visibility Relation] \label{def:vis-tx}
  \begin{align*}
    &\forall \txevents_{1}, \txevents_{2} \in X.\;
      \txevents_{1} \rel{\vis} \txevents_{2} \iff \\
      &\quad \big((\txevents_{2} \in \txs \implies
        \timestamp(\txevents_{1}) \le \timestamp(\startoftx(\txevents_{2}))) \\
        &\quad\quad \land (\txevents_{2} \in \Fence \cup \Attach \implies \timestamp(\txevents_{1}) \le \timestamp(\txevents_{2}))\big)
          \land \txevents_{1} \rel{\lcorder} \txevents_{2}.
  \end{align*}
\end{appdefinition}

\begin{apptheorem} \label{thm:conflictaxiom}
  \[
    A \models \conflictaxiom.
  \]
\end{apptheorem}

\begin{proof} \label{proof:conflictaxiom}
  We need to show that
  \[
    \forall \tvar_{1}, \tvar_{2} \in \strongtxs.\;
      \tvar_{1} \conflict \tvar_{2} \implies \tvar_{1} \rel{\vis} \tvar_{2} \lor \tvar_{2} \rel{\vis} \tvar_{1}.
  \]
  Consider the history $h$ of TCS.
  By Theorem~\ref{thm:tcs-correctness},
  $h \mid \committedVar(h)$ has a legal permutation $\pi$.
  Suppose that
  \[
    \intcertify(\tidselector(\tvar_{1}), \_, \_, \_, \_) \prec_{\pi}
      \intcertify(\tidselector(\tvar_{2}), \_, \_, \_, \_).
  \]
  Since $\tvar_{1} \conflict \tvar_{2}$ and $\tvar_{2}$ is committed,
  by (\ref{eqn:gcf-decision}),
  \[
    \commitVC(\tvar_{1}) \le \snapshotVC(\tvar_{2}).
  \]
  By Lemmas~\ref{lemma:ts-extread} and \ref{lemma:ts-commit},
  \[
    \timestamp(\tvar_{1}) \le \timestamp(\startoftx(\tvar_{2})).
  \]
  On the other hand, by (\ref{eqn:gcf-lc}),
  \[
    \lclock(\tvar_{1}) < \lclock(\tvar_{2}).
  \]
  By Definition~\ref{def:lco} of $\lcorder$,
  \[
    \tvar_{1} \rel{\lcorder} \tvar_{2}.
  \]
  Therefore, by Definition~\ref{def:vis-tx} of $\vis$,
  \[
    \tvar_{1} \rel{\vis} \tvar_{2}.
  \]
\end{proof}

\begin{applemma} \label{lemma:so-vis}
  \[
    \so \subseteq \vis.
  \]
\end{applemma}

\begin{proof} \label{proof:so-vis}
  By Lemmas~\ref{lemma:so-ts} and \ref{lemma:so-lc}.
\end{proof}

\begin{applemma} \label{lemma:vis-partial}
  The visibility relation $\vis$ is a partial order.
\end{applemma}

\begin{proof} \label{proof:vis-partial}
  We need to show that
  \begin{itemize}
    \item $\vis$ is irreflexive.
      This holds because $\lcorder$ is irreflexive.
    \item $\vis$ is transitive.
      Suppose that $\txevents_{1} \rel{\vis} \txevents_{2} \rel{\vis} \txevents_{3}$.
      By Definition~\ref{def:vis-tx} of $\vis$,
      \[
        \txevents_{1} \rel{\lcorder} \txevents_{2} \rel{\lcorder} \txevents_{3}.
      \]
      By Definition~\ref{def:lco} of $\lcorder$,
      \[
        \txevents_{1} \rel{\lcorder} \txevents_{3}.
      \]
      Regarding timestamps,
      we distinguish between the following four cases
      and use Lemma~\ref{lemma:ts-tid-st-tid}.
      \begin{itemize}
        \item $\txevents_{2} \in \Fence \cup \Attach
          \land \txevents_{3} \in \Fence \cup \Attach$.
          \[
            \timestamp(\txevents_{1}) \le \timestamp(\txevents_{2})
              \le \timestamp(\txevents_{3}).
          \]
        \item $\txevents_{2} \in \Fence \cup \Attach
          \land \txevents_{3} \in \txs$.
          \[
            \timestamp(\txevents_{1}) \le \timestamp(\txevents_{2})
              \le \timestamp(\startoftx(\txevents_{3})).
          \]
        \item $\txevents_{2} \in \txs \land \txevents_{3} \in \Fence \cup \Attach$.
          \[
            \timestamp(\txevents_{1}) \le \timestamp(\startoftx(\txevents_{2}))
              \le \timestamp(\txevents_{2}) \le \timestamp(\txevents_{3}).
          \]
        \item $\txevents_{2} \in \txs \land \txevents_{3} \in \txs$.
          \[
            \timestamp(\txevents_{1}) \le \timestamp(\startoftx(\txevents_{2}))
              \le \timestamp(\txevents_{2}) \le \timestamp(\startoftx(\txevents_{3})).
          \]
      \end{itemize}
      By Definition~\ref{def:vis-tx} of $\vis$,
      \[
        \txevents_{1} \rel{\vis} \txevents_{3}.
      \]
  \end{itemize}
\end{proof}

\begin{apptheorem} \label{thm:cv}
  \[
    A \models \cv.
  \]
\end{apptheorem}

\begin{proof} \label{proof:cv}
  By Lemmas~\ref{lemma:so-vis} and \ref{lemma:vis-partial},
  \[
    (\so \cup \vis)^{+} = \vis^{+} = \vis.
  \]
\end{proof}

\begin{applemma}[\prop{5}] \label{lemma:conflict-strong-ts}
  For any two conflicting transactions $\tvar_{1}$ and $\tvar_{2}$,
  \begin{align*}
    &\tvar_{1} \rel{\vis} \tvar_{2} \iff \\
      &\quad \commitVC(\tvar_{1})[\strongentry] < \commitVC(\tvar_{2})[\strongentry].
  \end{align*}
\end{applemma}

\begin{proof} \label{proof:conflict-strong-ts}
  We first show that
  \begin{align}
    &\tvar_{1} \rel{\vis} \tvar_{2} \implies \label{eq:vis-strong-ts} \\
      &\quad \commitVC(\tvar_{1})[\strongentry] < \commitVC(\tvar_{2})[\strongentry].
      \nonumber
  \end{align}
  Assume that $\tvar_{1} \rel{\vis} \tvar_{2}$.
  By Definition~\ref{def:vis-tx} of $\vis$,
  \[
    \tsfunc(\tvar_{1}) \le \tsfunc(\startoftx(\tvar_{2})).
  \]
  By Lemmas~\ref{lemma:ts-extread} and \ref{lemma:ts-commit},
  \[
    \commitVC(\tvar_{1}) \le \snapshotVC(\tvar_{2}).
  \]
  Therefore,
  \[
    \commitVC(\tvar_{1})[\strongentry] \le \snapshotVC(\tvar_{2})[\strongentry].
  \]
  By (\ref{eqn:gcf-commitvc}),
  \[
    \commitVC(\tvar_{2})[\strongentry] > \snapshotVC(\tvar_{2})[\strongentry].
  \]
  Putting it together yields
  \[
    \commitVC(\tvar_{1})[\strongentry] < \commitVC(\tvar_{2})[\strongentry].
  \]
  Next we show that
  \begin{align*}
    &\tvar_{1} \rel{\vis} \tvar_{2} \impliedby \\
      &\quad \commitVC(\tvar_{1})[\strongentry] < \commitVC(\tvar_{2})[\strongentry].
  \end{align*}
  Assume that
  \begin{align}
    \commitVC(\tvar_{1})[\strongentry] < \commitVC(\tvar_{2})[\strongentry].
    \label{eq:tid1-less-tid2-strong-ts}
  \end{align}
  Since $\tvar_{1} \conflict \tvar_{2}$, by Theorem~\ref{thm:conflictaxiom},
  \[
    \tvar_{1} \rel{\vis} \tvar_{2} \lor \tvar_{2} \rel{\vis} \tvar_{1}.
  \]
  By (\ref{eq:vis-strong-ts}) and (\ref{eq:tid1-less-tid2-strong-ts}),
  \[
    \lnot(\tvar_{2} \rel{\vis} \tvar_{1}).
  \]
  Therefore,
  \[
    \tvar_{1} \rel{\vis} \tvar_{2}.
  \]
\end{proof}


\subsection{Execution Order} \label{ss:eo}

\begin{appdefinition}[Execution Points] \label{def:ep}
  Let $k$ be a key.
  The ``execution point'' $\ep(e, k)$ of event
  $e \in (\extread \cap R_{k}) \cup C_{k}$ is defined as follows:

  \begin{itemize}
    \item If $e \in \extread \cap R_{k}$,
      then $\ep(e, k)$ is at
      line~\code{\ref{alg:unistore-replica}}{\ref{line:readkey-read}};
    \item If $e \in C_{k} \cap C_{\causalentry}$,
      then $\ep(e, k)$ is at
      line~\code{\ref{alg:unistore-replica}}{\ref{line:commit-oplog}}
      for this particular key $k$;
    \item If $e \in C_{k} \cap C_{\strongentry}$
      then $\ep(e, k)$ is at
      line~\code{\ref{alg:unistore-strong-commit}}{\ref{line:deliverupdates-oplog}}
      for delivery of the update of $\txfunc(e)$ on this particular key $k$.
      Note that \deliver{} is asynchronous with the commit event $e$.
  \end{itemize}
\end{appdefinition}

\begin{appdefinition}[Per-key Execution Order] \label{def:perkey-eo}
  Let $k$ be a key.
  Suppose that $\set{e_{1}, e_{2}} \subseteq (\extread \cap R_{k}) \cup C_{k}$.
  Event $e_1$ is executed before event $e_2$, denoted $e_1 \rel{\eok} e_2$,
  if $\ep(e_{1}, k)$ is executed before $\ep(e_{2}, k)$ in real time.
\end{appdefinition}

\begin{applemma} \label{lemma:vis-perkey-eo}
  Let $k \in \Key$ be a key, $\tvar \in \txs_{k}$ be a transaction,
  and $e \in \extread \cap R_{k}$ be an external read event.
  Suppose that $d \triangleq \dc(\tvar) = \dc(\txfunc(e))$.
  Then
  \[
    \tvar \rel{\vis} \txfunc(e) \implies \commitoftx(\tvar) \rel{\eok} e.
  \]
\end{applemma}

\begin{proof} \label{proof:vis-perkey-eo}
  By Definition~\ref{def:vis-tx} of $\vis$,
  \[
    \timestamp(\tvar) \le \timestamp(\startoftx(e)).
  \]
  Since $e \in \extread$, by Lemma~\ref{lemma:ts-extread},
  \[
    \timestamp(\tvar) \le \snapvc_{(\readkey, e)}.
  \]
  In the following, we distinguish between two cases
  according to whether $\tvar \in \causaltxs$
  or $\tvar \in \strongtxs$.
  Let $m \triangleq \partitionofproc(k)$.
  \begin{itemize}
    \item $\textsc{Case I}$: $\tvar \in \causaltxs$.
      By Lemma~\ref{lemma:ts-commit},
      \[
        \tsfunc(\tvar) = \commitVC(\tvar) \le \snapvc_{(\readkey, e)}.
      \]
      Therefore, after line~\code{\ref{alg:unistore-replica}}{\ref{line:readkey-wait-util-knownvc}}
      for $e$,
      \begin{align}
        (\knownVC^{m}_{d})_{e}[d]
        &\ge \snapvc_{(\readkey, e)}[d] \notag \\
        &\ge \commitVC(\tvar)[d].
        \label{eqn:vis-perkey-eo-knownvc-commitvc-causal}
      \end{align}
      By Lemma~\ref{lemma:knownvc-local-d},
      \commit{} of Algorithm~\ref{alg:unistore-replica}
      for $\ws(\tvar)[m] \ni \langle k, \_ \rangle$
      finishes before $e$ starts at replica $p^{m}_{d}$.
      By Definition~\ref{def:perkey-eo} of $\eok$,
      \[
        \commitoftx(\tvar) \rel{\eok} e.
      \]
    \item $\textsc{Case II}$: $\tvar \in \strongtxs$.
      By Lemma~\ref{lemma:ts-commit},
      \[
        \tsfunc(\tvar) = \commitVC(\tvar) \le \snapvc_{(\readkey, e)}.
      \]
      Therefore, after line~\code{\ref{alg:unistore-replica}}{\ref{line:readkey-wait-util-knownvc}}
      for $e$,
      \begin{align}
        (\knownVC^{m}_{d})_{e}[\strongentry]
        &\ge \snapvc_{(\readkey, e)}[\strongentry] \notag \\
        &\ge \commitVC(\tvar)[\strongentry].
        \label{eqn:vis-perkey-eo-knownvc-commitvc-strong}
      \end{align}
      By Lemma~\ref{lemma:knownvc-strong},
      \deliver{} of Algorithm~\ref{alg:unistore-strong-commit}
      for $\ws(\tvar)[m] \ni \langle k, \_ \rangle$
      finishes before $e$ starts at replica $p^{m}_{d}$.
      By Definition~\ref{def:perkey-eo} of $\eok$,
      \[
        \commitoftx(\tvar) \rel{\eok} e.
      \]
  \end{itemize}
\end{proof}


\subsection{Arbitration Relation}  \label{ss:ar}

\begin{appdefinition}[Arbitration Relation] \label{def:ar}
  We define the arbitration relation $\ar$ on $X$
  as the Lamport clock order between them, i.e.,
  \[
    \ar = \lcorder.
  \]
\end{appdefinition}

\begin{apptheorem} \label{thm:ca}
  \[
    A \models \ca.
  \]
\end{apptheorem}

\begin{proof} \label{proof:ca}
  By Definition~\ref{def:vis-tx} of $\vis$ and Definition~\ref{def:ar} of $\ar$,
  \begin{align*}
    \vis \subseteq \lcorder = \ar.
  \end{align*}
\end{proof}


\subsection{Return Values} \label{ss:rval}

It is straightforward to show that $\intretval$ holds
for \emph{internal} read events.
\begin{apptheorem} \label{thm:intretval}
  \[
    A \models \intretval.
  \]
\end{apptheorem}

\begin{proof} \label{proof:intretval}
  Let $e \in \intread \cap R_{k}$ be an internal read event.
  The transaction $\txfunc(e)$ contains update events on $k$.
  By line~\code{\ref{alg:unistore-coord}}{\ref{line:doread-return-from-buffer}},
  $e$ reads from the last update event on $k$
  preceding $e$ in $\txfunc(e)$.
\end{proof}

Now let $e$ be an \emph{external} read event.
For notational convenience,
we define $V_{e}$ to be the set of update transactions on $k$
that are visible to $\txfunc(e)$,
and $S_{e}$ the set of update transactions on $k$ that are safe to read
at line~\code{\ref{alg:unistore-replica}}{\ref{line:readkey-read}}.
By Assumption~\ref{assumption:complete-execution},
\ref{tcs-requirement:certify-before-deliver}, and
\ref{tcs-requirement:abort-cannot-deliver},
all transactions in $S_{e}$ are committed.
Formally,

\begin{appdefinition}[Visibility Set]
  \label{def:visible-set-tx}
  Let $e \in \extread \cap R_{k}$ be an external read event on key $k$.
  \[
    V_{e} \triangleq \vis^{-1}(\txfunc(e)) \cap \txs_{k}.
  \]
\end{appdefinition}

\begin{appdefinition}[Safe Set]
  \label{def:safe-set-tx}
  Let $e \in \extread \cap R_{k}$ be an external read event on key $k$.
  Suppose that $e$ is issued to replica $p^{m}_{d}$ in data center $d$.
  \begin{align*}
    S_{e} \triangleq \set{\tvar \in \txs_{k}:
      &\; \timestamp(\tvar) \le \snapvc_{(\readkey, e)}\; \land \\
      &\; \log[\tvar][k] \in (\oplog^{m}_{d})_{e}[k]}.
  \end{align*}
\end{appdefinition}

\begin{applemma} \label{lemma:visible-safe-tx}
  Let $e \in \extread \cap R_{k}$ be an external read event on key $k$.
  Suppose that $e$ is issued to replica $p^{m}_{d}$ in data center $d$.
  When $e$ returns at $p^{m}_{d}$
  (line~\code{\ref{alg:unistore-replica}}{\ref{line:readkey-read}}),
  we have
  \[
    V_{e} \subseteq S_{e}.
  \]
\end{applemma}

\begin{proof}  \label{proof:visible-safe-tx}
  For each $\tvar \in V_{e}$,
  we need to show that $\tvar \in S_{e}$.
  That is,
  \begin{gather}
    \timestamp(\tvar) \le \snapvc_{(\readkey, e)}
    \label{eqn:visible-safe-tx-snapshotvc}
  \end{gather}
  and
  \begin{gather}
    \log[\tvar][k] \in (\oplog^{m}_{d})_{e}[k].
    \label{eqn:visible-safe-tx-oplog}
  \end{gather}

  We first show that (\ref{eqn:visible-safe-tx-snapshotvc}) holds.
  Since $\tvar \in V_{e}$,
  \[
    \tvar \rel{\vis} \txfunc(e).
  \]
  By Definition~\ref{def:vis-tx} of $\vis$,
  \[
    \timestamp(\tvar) \le \timestamp(\startoftx(e)).
  \]
  By Lemma~\ref{lemma:ts-extread},
  \[
    \timestamp(\tvar) \le \snapvc_{(\readkey, e)}.
  \]

  To show that (\ref{eqn:visible-safe-tx-oplog}) holds,
  we perform a case analysis according to
  whether $\tvar$ is a local transaction in data center $d$
  or a remote one in data center $i \neq d$.

  \begin{itemize}
    \item $\textsc{Case I}$: $\tvar$ is a local transaction in data center $d$.
      Since $\tvar \rel{\vis} \txfunc(e)$,
      by Lemma~\ref{lemma:vis-perkey-eo},
      \[
        \commitoftx(\tvar) \rel{\eok} e.
      \]
      Therefore,
      \[
        \log[\tvar][k] \in (\oplog^{m}_{d})_{e}[k].
      \]
    \item $\textsc{Case II}$: $\tvar$ is a remote transaction
      in data center $i \neq d$.
      We distinguish between two cases
      according to whether $\tvar \in \causaltxs$ or $\tvar \in \strongtxs$.
      \begin{itemize}
        \item $\textsc{Case I}$: $\tvar \in \causaltxs$.
          Since $i \neq d$,
          by line~\code{\ref{alg:unistore-replica}}{\ref{line:readkey-uniformvc}},
          \[
            \snapvc_{(\readkey, e)}[i] \le (\uniformVC^{m}_{d})_{e}[i].
          \]
          By (\ref{eqn:visible-safe-tx-snapshotvc}),
          \[
            \timestamp(\tvar)[i] \le (\uniformVC^{m}_{d})_{e}[i].
          \]
          By Lemma~\ref{lemma:replication-uniformvc},
          \[
            \log[\tvar][k] \in (\oplog^{m}_{d})_{e}[k].
          \]
        \item $\textsc{Case II}$: $\tvar \in \strongtxs$.
          By line~\code{\ref{alg:unistore-replica}}{\ref{line:readkey-wait-util-knownvc}},
          \begin{align*}
            &\snapvc_{(\readkey, e)}[\strongentry] \\
              &\qquad \le (\knownVC^{m}_{d})_{e}[\strongentry].
          \end{align*}
          By (\ref{eqn:visible-safe-tx-snapshotvc}),
          \[
            \timestamp(\tvar)[\strongentry] \le (\knownVC^{m}_{d})_{e}[\strongentry].
          \]
          By Lemma~\ref{lemma:knownvc-strong},
          \[
            \log[\tvar][k] \in (\oplog^{m}_{d})_{e}[k].
          \]
      \end{itemize}
  \end{itemize}
\end{proof}

\begin{apptheorem} \label{thm:extretval}
  \[
    A \models \extretval.
  \]
\end{apptheorem}

\begin{proof} \label{proof:extretval}
  Let $e \in \extread \cap R_{k}$
  be an external read event on key $k$.
  Suppose that $e$ reads from transaction $\tvar$ in $S_{e}$.
  Since all transactions in $S_{e}$ are committed,
  $\tvar$ is committed.
  By Lemma~\ref{lemma:rf-ts},
  \[
    \timestamp(\tvar) \le \timestamp(\startoftx(e)).
  \]
  By Lemma~\ref{lemma:rf-lc},
  \[
    \tvar \rel{\lcorder} \txfunc(e).
  \]
  By Definition~\ref{def:vis-tx} of $\vis$,
  \[
    \tvar \rel{\vis} \txfunc(e).
  \]
  By Definition~\ref{def:visible-set-tx} of $V_{e}$,
  \[
    \tvar \in V_{e}.
  \]

  Both $V_{e}$ and $S_{e}$
  are totally ordered by $\lcorder$.
  Since $\tvar$ is the latest one in $S_{e}$
  and $V_{e} \subseteq S_{e}$
  (Theorem~\ref{lemma:visible-safe-tx}),
  $\tvar$ is also the latest one in $V_{e}$.
  Thus, $e$ reads from $\tvar$ in $V_{e}$.
  That is, $e$ reads from the update event
  $\ud(\tvar, k)$ of $V_{e}$.
\end{proof}

\begin{apptheorem} \label{thm:retval}
  \[
    A \models \retval.
  \]
\end{apptheorem}

\begin{proof} \label{proof:retval}
 By Theorems~\ref{thm:intretval} and \ref{thm:extretval}.
\end{proof}



\subsection{Uniformity} \label{ss:uniformity}

\subsubsection{Uniformity of Causal Transactions Originating at Correct Data Centers}
\label{sss:uniformity-correct-dc}

\begin{applemma} \label{lemma:knownvc-d-no-bound}
  For any replica $p^{m}_{d}$ in any correct data center $d \in \C$,
  $\knownVC^{m}_{d}[d]$ grows without bound.
\end{applemma}

\begin{proof} \label{proof:knownvc-d-no-bound}
  Since data center $d$ is correct, by Assumption~\ref{assumption:fairness},
  \propagate{} of Algorithm~\ref{alg:unistore-replication}
  will be executed infinitely often.
  \begin{itemize}
    \item $\textsc{Case I}$:
      Line~\code{\ref{alg:unistore-replication}}{\ref{line:propagate-knownvc-clock}}
      is executed infinitely often.
      By Assumption~\ref{assumption:clock},
      $\knownVC^{m}_{d}[d]$ grows without bound.
    \item $\textsc{Case II}$:
      Line~\code{\ref{alg:unistore-replication}}{\ref{line:propagate-knownvc-ts}}
      is executed infinitely often.
      That is, it is infinitely often that
      \[
        \preparedcausal^{m}_{d} \neq \emptyset.
      \]
      By Assumption~\ref{assumption:fairness},
      causal transactions in $\preparedcausal^{m}_{d}$
      will eventually be committed and removed from $\preparedcausal^{m}_{d}$
      (line~\code{\ref{alg:unistore-replica}}{\ref{line:commit-preparedcausal}}).
      Thus, it is infinitely often that new causal transactions
      are prepared and added into $\preparedcausal^{m}_{d}$
      (line~\code{\ref{alg:unistore-replica}}{\ref{line:preparecausal-preparedcausal}})
      with larger and larger prepare timestamps
      (line~\code{\ref{alg:unistore-replica}}{\ref{line:preparecausal-ts}}).
      Therefore,
      \[
        \min\set{\tsvar \mid \langle \_, \_, \tsvar \rangle \in \preparedcausal^{m}_{d}}
      \]
      and
      \begin{align*}
        &\knownVC^{m}_{d}[d] \\
          &\quad = \min\set{\tsvar \mid \langle \_, \_, \tsvar \rangle \in \preparedcausal^{m}_{d}} - 1
      \end{align*}
      grow without bound.
    \end{itemize}
\end{proof}

\begin{applemma} \label{lemma:knownvc-x}
  Let $p^{m}_{d}$ be a replica in a correct data center $d \in \C$.
  If for some $j \in \D$ and some value $x \in \mathbb{N}$
  \[
    \knownVC^{m}_{d}[j] \ge x,
  \]
  then eventually
  \[
    \forall \cdrange \in \C.\; \knownVC^{m}_{\cdrange}[j] \ge x.
  \]
\end{applemma}

\begin{proof} \label{proof:knownvc-x}
  Since data center $d$ is correct, by Assumption~\ref{assumption:fairness},
  for each other data center $i \neq d$, replica $p^{m}_{d}$ will keep
  \begin{itemize}
    \item \emph{replicating} to data center $i$ the write sets
      \[
        \langle \_, \wbuffvar, \commitvc, \_ \rangle \in \committedcausal^{m}_{d}[j]
      \]
      that have not been received by $i$ from the perspective of $d$
      ($\commitvc[d] \le \knownVC^{m}_{d}[d]$
      at line~\code{\ref{alg:unistore-replication}}{\ref{line:propagate-txs}}
      and $\commitvc[j] > \globalmatrix^{m}_{d}[i][j]$ at
      line~\code{\ref{alg:unistore-replication}}{\ref{line:forward-txs}});
    \item or sending \emph{heartbeats} with up-to-date
      $\knownVC^{m}_{d}[j]$ to data center $i$
      (lines~\code{\ref{alg:unistore-replication}}{\ref{line:propagate-call-heartbeat}}
      and \code{\ref{alg:unistore-replication}}{\ref{line:forward-call-heartbeat}}).
  \end{itemize}
  By Assumption~\ref{assumption:message},
  $\knownVC^{m}_{\cdrange}[j]$ at replica $p^{m}_{\cdrange}$ of
  each correct data center $\cdrange \in \C$ will eventually be updated
  (lines~\code{\ref{alg:unistore-replication}}{\ref{line:replicate-knownvc}}
  and \code{\ref{alg:unistore-replication}}{\ref{line:heartbeat-knownvc}})
  such that
  \[
    \knownVC^{m}_{\cdrange}[j] \ge \knownVC^{m}_{d}[j] \ge x.
  \]
\end{proof}

\begin{applemma} \label{lemma:uniformvc-c-no-bound}
  Let $d \in \C$ be a correct data center.
  For any replica $p^{m}_{\cdrange}$
  in any correct data center $\cdrange \in \C$,
  $\uniformVC^{m}_{\cdrange}[d]$ grows without bound.
\end{applemma}

\begin{proof} \label{proof:uniformvc-c-no-bound}
  By Lemmas~\ref{lemma:knownvc-d-no-bound} and \ref{lemma:knownvc-x},
  for any replica $p^{m}_{\cdrange}$
  in any correct data center $\cdrange \in \C$,
  $\knownVC^{m}_{\cdrange}[d]$ grows without bound.
  By lines~\code{\ref{alg:unistore-clock}}{\ref{line:bcast-call-knownvclocal}}
  and \code{\ref{alg:unistore-clock}}{\ref{line:knownvclocal-stablevc-causal}},
  for any replica $p^{m}_{\cdrange}$
  in any correct data center $\cdrange \in \C$,
  $\stableVC^{m}_{\cdrange}[d]$ grows without bound.
  By line~\code{\ref{alg:unistore-clock}}{\ref{line:bcast-call-stablevc}},
  Assumptions~\ref{assumption:message} and \ref{assumption:failure-model},
  and lines~\code{\ref{alg:unistore-clock}}{\ref{line:stablevc-g}}--\code{
    \ref{alg:unistore-clock}}{\ref{line:stablevc-uniformvc}},
  for any replica $p^{m}_{\cdrange}$
  in any correct data center $\cdrange \in \C$,
  $\uniformVC^{m}_{\cdrange}[d]$ grows without bound.
\end{proof}

\begin{applemma}[\prop{4}] \label{lemma:uniformvc-x}
  Let $p^{m}_{d}$ be any replica in any data center $d$.
  For any time $\realtime$, there exists some time $\realtime'$ such that
  \begin{align*}
    &\forall i \in \D.\; \forall \cdrange \in \C.\; \forall n \in \P. \\
      &\quad \uniformVC^{n}_{\cdrange}(\realtime')[i] \ge \uniformVC^{m}_{d}(\realtime)[i].
  \end{align*}
\end{applemma}

\begin{proof} \label{proof:uniformvc-x}
  By Lemma~\ref{lemma:uniformvc-knownvc-f+1},
  Assumption~\ref{assumption:failure-model}, and the fact that at most
  $f$ data centers may fail,
  \begin{align*}
    &\forall i \in \D.\; \exists \cdrange \in \C.\; \forall n \in \P. \\
      &\quad \uniformVC^{m}_{d}(\realtime)[i] \le \knownVC^{n}_{\cdrange}(\realtime)[i].
  \end{align*}
  By Lemma~\ref{lemma:knownvc-x},
  there exists some time $\realtime''$ such that
  \begin{align*}
    &\forall i \in \D.\; \forall \cdrange \in \C.\; \forall n \in \P. \\
      &\quad \uniformVC^{m}_{d}(\realtime)[i] \le \knownVC^{n}_{\cdrange}(\realtime'')[i].
  \end{align*}
  By Algorithm~\ref{alg:unistore-clock} and Assumption~\ref{assumption:failure-model},
  there exists some time $\realtime'$ such that
  \begin{align*}
    &\forall i \in \D.\; \forall \cdrange \in \C.\; \forall n \le \P. \\
      &\quad \uniformVC^{n}_{\cdrange}(\realtime')[i] \ge \uniformVC^{m}_{d}(\realtime)[i].
  \end{align*}
\end{proof}

\begin{applemma} \label{lemma:uniformity-correct-dc}
  Let $d \in \C$ be a correct data center
  and $\tvar \in \causaltxs$ be a causal transaction
  that originates at $d$.
  Then for any replica $p^{m}_{\cdrange}$
  in any correct data center $\cdrange \in \C$, eventually
  \[
    \forall i \in \D.\; \timestamp(\tvar)[i] \le \uniformVC^{m}_{\cdrange}[i].
  \]
\end{applemma}

\begin{proof} \label{proof:uniformity-correct-dc}
  Since $d$ is correct, by Lemma~\ref{lemma:uniformvc-c-no-bound},
  there exists some time $\realtime'$ such that
  \[
    \timestamp(\tvar)[d] \le \uniformVC^{m}_{\cdrange}(\realtime')[d].
  \]
  On the other hand,
  by Definition~\ref{def:ts-op} of timestamps
  and Lemma~\ref{lemma:pastvc-uniformvc-except-d}
  (let $n \triangleq \coord(\tvar)$ and $cl \triangleq \client(\tvar)$),
  \begin{align*}
    \forall i \in \D \setminus \set{d}.\;
      \timestamp(\tvar)[i] &= (\pastVC_{\cl})_{\commitoftx(\tvar)}[i] \\
      &\le (\uniformVC^{n}_{d})_{\commitoftx(\tvar)}[i].
  \end{align*}
  By Lemma~\ref{lemma:uniformvc-x},
  there exists some time $\realtime''$ such that
  \[
    \forall i \in \D \setminus \set{d}.\;
      \timestamp(\tvar)[i] \le (\uniformVC^{m}_{\cdrange})(\realtime'')[i].
  \]
  Let
  \[
    \realtime \triangleq \max\set{\realtime', \realtime''}.
  \]
  By Lemma~\ref{lemma:uniformvc-nondecreasing},
  \[
    \forall i \in \D.\; \timestamp(\tvar)[i] \le \uniformVC^{m}_{\cdrange}(\realtime)[i].
  \]
\end{proof}
\subsubsection{Uniformity of Causal Transactions Visible to \fence{} events}
\label{sss:uniformity-fences}

\begin{applemma} \label{lemma:uniformity-causal-fence}
  Let $\tvar \in \causaltxs$ be a causal transaction
  and $\fencerange \in \Fence$ be a \fence{} event.
  If $\tvar \rel{\vis} \fencerange$,
  then for any replica $p^{m}_{\cdrange}$
  in any correct data center $\cdrange \in \C$, eventually
  \[
    \forall i \in \D.\; \timestamp(\tvar)[i] \le \uniformVC^{m}_{\cdrange}[i].
  \]
\end{applemma}

\begin{proof} \label{proof:uniformity-causal-fence}
  Since $\tvar \rel{\vis} \fencerange$, by Definition~\ref{def:vis-tx} of $\vis$,
  \[
    \timestamp(\tvar) \le \timestamp(\fencerange).
  \]
  Suppose that $\fencerange$ is issued by client $\cl$
  to replica $p^{n}_{d}$ in data center $d$
  and is returned at time $\realtime_{\fencerange}$.
  By Definition~\ref{def:ts-op} of timestamps
  and Lemma~\ref{lemma:pastvc-uniformvc-except-d},
  \[
    \timestamp(\tvar)[d] \le \timestamp(\fencerange)[d] \le (\uniformVC^{n}_{d})_{\fencerange}[d].
  \]
  By Lemma~\ref{lemma:uniformvc-x},
  there exists some time $\realtime'$ such that
  \[
    \timestamp(\tvar)[d] \le (\uniformVC^{m}_{\cdrange})(\realtime')[d].
  \]
  On the other hand, by Definition~\ref{def:ts-op} of timestamps
  and Lemma~\ref{lemma:pastvc-uniformvc},
  \begin{align*}
    \forall i \in \D \setminus \set{d}.\;
      \timestamp(\tvar)[i] &\le \timestamp(\fencerange)[i] \\
      &= (\pastVC_{\cl})_{\fencerange}[i] \\
      &\le \uniformVC^{l}_{d}(\realtime_{\fencerange})[i]
  \end{align*}
  for some replica $p^{l}_{d}$ in data center $d$.
  By Lemma~\ref{lemma:uniformvc-x},
  there exists some time $\realtime''$ such that
  \[
    \forall i \in \D \setminus \set{d}.\;
      \timestamp(\tvar)[i] \le (\uniformVC^{m}_{\cdrange})(\realtime'')[i].
  \]
  Let
  \[
    \realtime \triangleq \max\set{\realtime', \realtime''}.
  \]
  By Lemma~\ref{lemma:uniformvc-nondecreasing},
  \[
    \forall i \in \D.\; \timestamp(\tvar)[i] \le \uniformVC^{m}_{\cdrange}(\realtime)[i].
  \]
\end{proof}
\subsubsection{Uniformity of Strong Transactions}
\label{sss:uniformity-strong}

\begin{applemma} \label{lemma:knownvc-strong-no-bound}
  For any replica $p^{m}_{\cdrange}$
  in any correct data center $\cdrange \in \C$,
  $\knownVC^{m}_{\cdrange}[\strongentry]$ grows without bound.
\end{applemma}

\begin{proof} \label{proof:knownvc-strong-no-bound}
  By Assumption~\ref{assumption:fairness}
  and Theorem~\ref{thm:tcs-correctness},
  replica $p^{m}_{\cdrange}$ will either deliver committed strong transactions
  infinitely often (\deliver{} of Algorithm~\ref{alg:unistore-strong-commit})
  or submit dummy strong transactions infinitely often
  (\heartbeatstrong{} of Algorithm~\ref{alg:unistore-strong-commit}).
  Thus, $\knownVC^{m}_{\cdrange}[\strongentry]$ grows without bound.
\end{proof}

\begin{applemma} \label{lemma:uniformity-strong}
  Let $\tvar \in \txs$ be a transaction.
  Then for any replica $p^{m}_{\cdrange}$
  in any correct data center $\cdrange \in \C$, eventually
  \[
    \timestamp(\tvar)[\strongentry] \le \stableVC^{m}_{\cdrange}[\strongentry].
  \]
\end{applemma}

\begin{proof} \label{proof:uniformity-strong}
  By Lemma~\ref{lemma:knownvc-strong-no-bound}
  and lines~\code{\ref{alg:unistore-clock}}{\ref{line:bcast-call-knownvclocal}}
  and \code{\ref{alg:unistore-clock}}{\ref{line:knownvclocal-stablevc-strong}},
  $\stableVC^{m}_{\cdrange}[\strongentry]$ grows without bound.
  Therefore, there exists some time $\realtime$ such that
  \[
    \timestamp(\tvar)[\strongentry] \le \stableVC^{m}_{\cdrange}(\realtime)[\strongentry].
  \]
\end{proof}

\begin{applemma} \label{lemma:uniformity-strongtx-i}
  Let $\tvar \in \strongtxs$ be a strong transaction.
  Then for any replica $p^{m}_{\cdrange}$
  in any correct data center $\cdrange \in \C$, eventually
  \[
    \forall i \in \D.\; \timestamp(\tvar)[i] \le \uniformVC^{m}_{\cdrange}[i].
  \]
\end{applemma}

\begin{proof} \label{proof:uniformity-strongtx-i}
  Let $d \triangleq \dc(\tvar)$,
  $n \triangleq \coord(\tvar)$, and $cl \triangleq \client(\tvar)$.
  By Definition~\ref{def:ts-op} of timestamps,
  \[
    \timestamp(\tvar) = (\pastVC_{\cl})_{\commitoftx(\tvar)}.
  \]
  On the one hand, by Lemma~\ref{lemma:pastvc-uniformvc-except-d},
  \begin{align*}
    \forall i \in \D \setminus \set{d}.\;
      \timestamp(\tvar)[i] &= (\pastVC_{\cl})_{\commitoftx(\tvar)}[i] \\
      &\le (\uniformVC^{n}_{d})_{\commitoftx(\tvar)}[i].
  \end{align*}
  By Lemma~\ref{lemma:uniformvc-x},
  there exists some time $\realtime'$ such that
  \[
    \forall i \in \D \setminus \set{d}.\;
      \timestamp(\tvar)[i] \le (\uniformVC^{m}_{\cdrange})(\realtime')[i].
  \]
  On the other hand, by Lemma~\ref{lemma:ts-commit},
  \[
    \timestamp(\tvar)[d] = \commitVC(\tvar)[d].
  \]
  By (\ref{eqn:gcf-commitvc}),
  \[
    \commitVC(\tvar)[d] = \snapshotVC(\tvar)[d].
  \]
  By lines~\code{\ref{alg:unistore-strong-commit}}{\ref{line:commitstrong-call-uniformbarrier}}
  and \code{\ref{alg:unistore-replica}}{\ref{line:uniformbarrier-wait-uniformvc-d}},
  \[
    \snapshotVC(\tvar)[d] \le (\uniformVC^{n}_{d})_{\commitoftx(\tvar)}[d].
  \]
  Putting it together yields,
  \[
    \timestamp(\tvar)[d] \le (\uniformVC^{n}_{d})_{\commitoftx(\tvar)}[d].
  \]
  By Lemma~\ref{lemma:uniformvc-x},
  there exists some time $\realtime''$ such that
  \[
    \timestamp(\tvar)[d] \le (\uniformVC^{m}_{\cdrange})(\realtime'')[d].
  \]
  Let
  \[
    \realtime \triangleq \max\set{\realtime', \realtime''}.
  \]
  By Lemma~\ref{lemma:uniformvc-nondecreasing},
  \[
    \forall i \in \D.\; \timestamp(\tvar)[i] \le \uniformVC^{m}_{\cdrange}(\realtime)[i].
  \]
\end{proof}


\subsection{Eventual Visibility} \label{ss:ev}

\begin{apptheorem} \label{thm:ev}
  \[
    A \models \ev.
  \]
\end{apptheorem}

\begin{proof} \label{proof:ev}
  Consider a transaction $\tvar \in \txs$ such that
  \begin{align*}
    \dc(\tvar) \in \C \lor (\exists \fencerange \in \Fence.\; \tvar \rel{\vis} \fencerange)
    \lor \tvar \in \strongtxs.
  \end{align*}
  By Lemma~\ref{lemma:so-vis},
  it suffices to show that for any client $\cl$,
  \[
    \Big\lvert \txs|_{\cl} \Big\rvert = \infty
      \implies \exists \tvar' \in \txs|_{\cl}.\; \tvar \rel{\vis} \tvar'.
  \]
  By Lemmas~\ref{lemma:uniformity-correct-dc},
  \ref{lemma:uniformity-causal-fence}, \ref{lemma:uniformity-strong},
  and \ref{lemma:uniformity-strongtx-i},
  there exists some time $\realtime$ such that
  \begin{align}
    &\forall \cdrange \in \C.\; \forall n \in \P. \nonumber\\
      &\quad (\forall i \in \D.\; \timestamp(\tvar)[i]
        \le \uniformVC^{n}_{\cdrange}(\realtime)[i])
      \;\land \nonumber\\
      &\quad\;\; \timestamp(\tvar)[\strongentry]
        \le \stableVC^{n}_{\cdrange}(\realtime)[\strongentry].
    \label{eqn:time-t}
  \end{align}
  Since $\Big\lvert \txs|_{\cl} \Big\rvert = \infty$,
  there exists some correct data center $d \in \C$
  to which $\cl$ issues an infinite number of transactions.
  Let $\tvar' \in \txs$ be the first transaction
  issued by client $\cl$ to data center $d$
  which starts after time $\realtime$ such that
  \[
    \lclock(\tvar) < \lclock(\tvar').
  \]
  Thus, by Definition~\ref{def:lco} of $\lcorder$,
  \[
    \tvar \rel{\lcorder} \tvar'.
  \]
  Let $m \triangleq \coord(\tvar')$.
  Since $d$ is correct, by (\ref{eqn:time-t}),
  \begin{align*}
    &(\forall i \in \D.\; \timestamp(\tvar)[i] \le \uniformVC^{m}_{d}(\realtime)[i])
    \;\land \\
    &\;\;\timestamp(\tvar)[\strongentry] \le \stableVC^{m}_{d}(\realtime)[\strongentry].
  \end{align*}
  By Lemma~\ref{lemma:uniformvc-nondecreasing},
  \[
    \forall i \in \D.\;
      \uniformVC^{m}_{d}(\realtime)[i] \le (\uniformVC^{m}_{d})_{\startoftx(\tvar')}[i].
  \]
  By Lemma~\ref{lemma:stablevc-strong-nondecreasing},
  \[
    \stableVC^{m}_{d}(\realtime)[\strongentry] \le
      (\stableVC^{m}_{d})_{\startoftx(\tvar')}[\strongentry].
  \]
  By Lemma~\ref{lemma:ts-start},
  \begin{gather*}
    (\forall i \in \D.\; (\uniformVC^{m}_{d})_{\startoftx(\tvar')}[i]
      \le \timestamp(\startoftx(\tvar'))[i])
    \;\land \\
    (\stableVC^{m}_{d})_{\startoftx(\tvar')}[\strongentry]
      \le \timestamp(\startoftx(\tvar'))[\strongentry].
  \end{gather*}
  Putting it together yields
  \[
    \timestamp(\tvar) \le \timestamp(\startoftx(\tvar')).
  \]
  By Definition~\ref{def:vis-tx} of $\vis$,
  \[
    \tvar \rel{\vis} \tvar'.
  \]
\end{proof}

\section{Proof of Correctness}
\label{sec:correctness}

We have rigorously proved that \System correctly implements the specification of
PoR consistency for the case when the data store manages last-writer-wins registers. The
proof uses the formal framework
from~\cite{sebastian-book,distrmm-popl,framework-concur15} and establishes
Properties~\ref{prop:knownvc}-\ref{prop:stablevcred} stated earlier. Due to
space constraints, we defer the proof
to~\tr{\ref{section:correctness-proof}}{\nappproof}.



\subsection{\unistore{} Liveness} \label{ss:unistore-liveness}

\begin{applemma} \label{lemma:migrate-terminate}
  Each client migration \clattach{} to a correct data center
  will eventually terminate, provided that the client managed to
  complete its \fence{} call at its original data center
  just before \clattach.
\end{applemma}

\begin{proof} \label{proof:migrate-terminate}
  Suppose that client $\cl$ in its current data center $d \triangleq \cldc(\cl)$
  issues a \clattach{} event $\attachrange$
  to replica $p^{m}_{\cdrange}$ in correct data center $\cdrange \in \C$.

  Let $e \in C$ be the last commit event
  issued by client $\cl$ before $\attachrange$.
  If $e$ does not exist, then
  \[
    \forall i \in \D.\; \pastVC_{\cl}[i] = 0.
  \]
  Therefore, the wait condition
  \[
    \forall i \in \D \setminus \set{c}.\; \uniformVC^{m}_{c}[i] \ge 0
  \]
  at line~\code{\ref{alg:unistore-replica}}{\ref{line:attach-wait-condition}}
  at replica $p^{m}_{c}$ will eventually hold.
  Thus, the \clattach{} event $\attachrange$ eventually terminates.

  Otherwise, suppose that $e$ is issued to replica $p^{n'}_{d'}$
  in data center $d'$.
  By Lemma~\ref{lemma:pastvc-uniformvc-except-d},
  \begin{align}
    \forall i \in \D \setminus \set{d'}.\;
      (\uniformVC^{n'}_{d'})_{e}[i] \ge (\pastVC_{\cl})_{e}[i].
      \label{eq:attach-dprime}
  \end{align}
  Note that it is possible that $d' \neq d$,
  since there may exist other \clattach{} events between $e$ and $\attachrange$.
  Therefore, we distinguish between the following two cases:
  \begin{itemize}
    \item \textsc{Case I}: $d' = d$. By (\ref{eq:attach-dprime}),
      \begin{align}
        \forall i \in \D \setminus \set{d}.\;
          (\uniformVC^{n'}_{d})_{e}[i] \ge (\pastVC_{\cl})_{e}[i].
          \label{eq:attach-dprime-d}
      \end{align}
    \item \textsc{Case II}: $d' \neq d$.
      Consider the last \clattach{} event, denoted $a'$, before $\attachrange$.
      Suppose that $a'$ is issued to replica $p^{n}_{d}$ in data center $d$.
      By Lemma~\ref{lemma:pastvc-uniformvc-except-d}, when $a'$ terminates,
      \begin{align}
        \forall i \in \D \setminus \set{d}.\;
          &(\uniformVC^{n}_{d})_{a'}[i] \ge (\pastVC_{\cl})_{a'}[i] \nonumber \\
          &\quad = (\pastVC_{\cl})_{e}[i].
          \label{eq:attach-dprime-not-d}
      \end{align}
  \end{itemize}

  Let $\fencerange \in \Fence$ be the \fence{} event issued by client $\cl$
  just before $\attachrange$.
  Suppose that $\fencerange$ is issued to replica $p^{l}_{d}$ in data center $d$.
  By Lemma~\ref{lemma:pastvc-uniformvc-except-d},
  \begin{align}
    (\uniformVC^{l}_{d})_{\fencerange}[d] &\ge (\pastVC_{\cl})_{\fencerange}[d]
      \nonumber\\
      &= (\pastVC_{\cl})_{e}[d].
      \label{eq:barrier-just-before-attach}
  \end{align}

  By (\ref{eq:attach-dprime-d}), (\ref{eq:attach-dprime-not-d}),
  (\ref{eq:barrier-just-before-attach}), and Lemma~\ref{lemma:uniformvc-x},
  eventually for the correct data center $\cdrange$,
  \[
    \forall i \in \D.\;
      \uniformVC^{m}_{\cdrange}[i] \ge (\pastVC_{\cl})_{e}[i].
  \]
  Therefore, the wait condition
  \[
    \forall i \in \D \setminus \set{c}.\;
      \uniformVC^{m}_{c}[i] \ge (\pastVC_{\cl})_{e}[i]
  \]
  at line~\code{\ref{alg:unistore-replica}}{\ref{line:attach-wait-condition}}
  at replica $p^{m}_{c}$ will eventually hold.
  Thus, the \clattach{} event $\attachrange$ eventually terminates.
\end{proof}

\begin{apptheorem} \label{thm:termination}
  Any client event issued at a correct data center
  will eventually terminate.
\end{apptheorem}

\begin{proof} \label{proof:termination}
  Consider any event $e$ issued by client $\cl$
  at a correct data center $c \in \C$.
  It suffices to show that each wait condition in the execution of $e$,
  if any, will eventually hold.
  In the following, we perform a case analysis
  according to the type of event $e$.
  \begin{itemize}
    \item $\textsc{Case I}$: $e \in S$.
      The theorem holds trivially.
    \item $\textsc{Case II}$: $e \in R$.
      If $e \in \intread$, the theorem holds trivially.
      Otherwise, $e \in \extread$.
      By Lemmas~\ref{lemma:knownvc-d-no-bound}
      and \ref{lemma:knownvc-strong-no-bound},
      the wait condition at
      line~\code{\ref{alg:unistore-replica}}{\ref{line:readkey-wait-util-knownvc}}
      for $e$ will eventually hold.
    \item $\textsc{Case III}$: $e \in U$.
      The theorem holds trivially.
    \item $\textsc{Case IV}$: $e \in C_{\causalentry}$.
      Since data center $c$ is correct,
      the wait condition at
      line~\code{\ref{alg:unistore-coord}}{\ref{line:commitcausal-wait-prepareack}}
      will eventually hold.
    \item $\textsc{Case V}$: $e \in C_{\strongentry}$.
      By Lemma~\ref{lemma:uniformvc-c-no-bound},
      the wait condition at
      line~\code{\ref{alg:unistore-replica}}{\ref{line:uniformbarrier-wait-uniformvc-d}}
      will eventually hold.
      Thus, line~\code{\ref{alg:unistore-strong-commit}}{\ref{line:commitstrong-call-uniformbarrier}}
      will eventually terminate.
      Then, by Assumption~\ref{assumption:complete-execution},
      the procedure $\certify$ and thus the event $e$ will eventually terminate.
    \item $\textsc{Case VI}$: $e \in \fence$.
      By Lemma~\ref{lemma:uniformvc-c-no-bound},
      the wait condition at
      line~\code{\ref{alg:unistore-replica}}{\ref{line:uniformbarrier-wait-uniformvc-d}}
      will eventually hold.
    \item $\textsc{Case VII}$: $e \in \clattach$.
      The theorem holds due to \textsc{Case VI} and Lemma~\ref{lemma:migrate-terminate}.
  \end{itemize}
\end{proof}



\fi



\end{document}
