
\subsection{Lamport Clocks} \label{ss:lc}

\begin{appdefinition}[Lamport Clocks of Events] \label{def:lc-op}
  Let $e \in C_{\causalentry} \cup C_{\strongentry} \cup \Fence \cup \Attach$
  be an event issued by client $\cl$.
  We define its Lamport clock $\lclock(e)$ as
  \[
    \lclock(e) \triangleq (\lc_{\cl})_{e}.
  \]
  See lines~\code{\ref{alg:unistore-client}}{\ref{line:commitcausaltx-lc}},
  \code{\ref{alg:unistore-client}}{\ref{line:commitstrongtx-lc}},
  \code{\ref{alg:unistore-client}}{\ref{line:fence-lc}},
  and \code{\ref{alg:unistore-client}}{\ref{line:clattach-lc}}
  for \commitcausaltx, \commitstrongtx, \fence, and \clattach{} events, respectively.
\end{appdefinition}

\begin{appdefinition}[Lamport Clocks of Transactions] \label{def:lc-tx}
  The Lamport clock $\lclock(\tvar)$ of a transaction $\tvar$
  is that of its commit event, i.e.,
  \[
    \forall \tvar \in \txs.\;
      \lclock(\tvar) \triangleq \lclock(\commitoftx(\tvar)).
  \]
\end{appdefinition}

\begin{applemma} \label{lemma:lc-extread-commit}
  Let $e \in \extread$ be an external read event issued by client $\cl$.
  Then
  \[
    (\lc_{\cl})_{e} < \lclock(\txfunc(e)).
  \]
\end{applemma}

\begin{proof} \label{proof:lc-extread-commit}
  If $\txfunc(e)$ is a causal transaction,
  by line~\code{\ref{alg:unistore-client}}{\ref{line:commitcausaltx-lc}},
  \[
    \lclock(e) < \lclock(\commitoftx(e)) = \lclock(\txfunc(e)).
  \]
  If $\txfunc(e)$ is a strong transaction,
  by line~\code{\ref{alg:unistore-client}}{\ref{line:commitstrongtx-lc-so}}
  and (\ref{eqn:gcf-lc}),
  \[
    \lclock(e) < \lclock(\commitoftx(e)) = \lclock(\txfunc(e)).
  \]
\end{proof}

\begin{appdefinition}[Lamport Clock Order] \label{def:lco}
  The Lamport clock order $\lcorder$ on $X$
  is the total order defined by their Lamport clocks,
  with their client identifiers for tie-breaking.
\end{appdefinition}

\begin{applemma} \label{lemma:so-lc}
  \[
    \so \subseteq \lcorder.
  \]
\end{applemma}

\begin{proof} \label{proof:so-lc}
  By lines~\code{\ref{alg:unistore-client}}{\ref{line:commitcausaltx-lc}},
  \code{\ref{alg:unistore-client}}{\ref{line:commitstrongtx-lc-so}},
  (\ref{eqn:gcf-lc}),
  \code{\ref{alg:unistore-client}}{\ref{line:commitstrongtx-lc}},
  \code{\ref{alg:unistore-client}}{\ref{line:fence-lc}},
  and \code{\ref{alg:unistore-client}}{\ref{line:clattach-lc}}.
\end{proof}

\begin{applemma} \label{lemma:rf-lc}
  Let $e \in \extread$ be an external read event
  which reads from transaction $\tvar$. Then
  \[
    \tvar \rel{\lcorder} \txfunc(e).
  \]
\end{applemma}

\begin{proof} \label{proof:rf-lc}
  Suppose that $e$ is issued by client $\cl$.
  By line~\code{\ref{alg:unistore-client}}{\ref{line:read-lc}},
  \[
    \lclock(\tvar) \le (\lc_{\cl})_{e}.
  \]
  By Lemma~\ref{lemma:lc-extread-commit},
  \[
    (\lc_{\cl})_{e} < \lclock(\txfunc(e)).
  \]
  Therefore,
  \[
    \lclock(\tvar) < \lclock(\txfunc(e)).
  \]
  By Definition~\ref{def:lco} of $\lcorder$,
  \[
    \tvar \rel{\lcorder} \txfunc(e).
  \]
\end{proof}
