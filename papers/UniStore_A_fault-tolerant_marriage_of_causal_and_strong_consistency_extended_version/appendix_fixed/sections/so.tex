
\subsection{Session Order}  \label{ss:so}

\begin{applemma} \label{lemma:so-ts}
  \begin{align*}
    &\forall \txevents_{1}, \txevents_{2} \in X.\;
      \txevents_{1} \rel{\so} \txevents_{2} \implies
      \big(\timestamp(\txevents_{1}) \le \timestamp(\txevents_{2}) \\
      &\quad \land \big(\txevents_{2} \in \txs
        \implies \timestamp(\txevents_{1})
          \le \timestamp(\startoftx(\txevents_{2}))
          \le \timestamp(\txevents_{2})\big)\big).
  \end{align*}
\end{applemma}

\begin{proof} \label{proof:so-ts}
  By Definitions~\ref{def:ts-op} and \ref{def:ts-tx} of timestamps
  and Lemma~\ref{lemma:pastvc-nondecreasing},
  \[
    \timestamp(\txevents_{1}) \le \timestamp(\txevents_{2}),
  \]
  and
  \[
    \txevents_{2} \in \txs \implies
      \timestamp(\txevents_{1}) \le \timestamp(\startoftx(\txevents_{2})).
  \]
  Besides, by Lemma~\ref{lemma:snapshotvc-commitvc},
  \[
    \txevents_{2} \in \txs \implies
      \timestamp(\startoftx(\txevents_{2})) \le \timestamp(\txevents_{2}).
  \]
  Therefore,
  \[
    \txevents_{2} \in \txs \implies
      \timestamp(\txevents_{1}) \le \timestamp(\startoftx(\txevents_{2}))
      \le \timestamp(\txevents_{2}).
  \]
\end{proof}
