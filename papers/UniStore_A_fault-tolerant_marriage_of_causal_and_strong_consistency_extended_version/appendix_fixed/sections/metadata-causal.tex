
\subsection{Metadata for Causal Transactions}
\label{ss:metadata-causal}

A causal transaction is \emph{committed} when \commitcausal{} for it returns.
A causal transaction is \emph{committed at replica} $p^{m}_{d}$
when \commit{} for it at $p^{m}_{d}$ returns.
\subsubsection{Properties of $\knownVC$}
\label{sss:knownvc}

\begin{applemma} \label{lemma:knownvc-d-nondecreasing}
  For any replica $p^{m}_{d}$ in data center $d$,
  $\knownVC^{m}_{d}[d]$ is non-decreasing.
\end{applemma}

\begin{proof} \label{proof:knownvc-d-nondecreasing}
  Consider two points of time $\realtime_{1}$ and $\realtime_{2}$ such that $\realtime_{1} < \realtime_{2}$.
  We need to show that
  \[
    \knownVC^{m}_{d}(\realtime_{1})[d] \le \knownVC^{m}_{d}(\realtime_{2})[d].
  \]

  Note that $\knownVC^{m}_{d}[d]$ is updated only
  at lines~\code{\ref{alg:unistore-replication}}{\ref{line:propagate-knownvc-clock}}
  or \code{\ref{alg:unistore-replication}}{\ref{line:propagate-knownvc-ts}}.
  We distinguish between the following four cases.
  \begin{itemize}
    \item $\textsc{Case I}$:
      Both $\knownVC^{m}_{d}(\realtime_{1})[d]$ and $\knownVC^{m}_{d}(\realtime_{2})[d]$ are set
      at line~\code{\ref{alg:unistore-replication}}{\ref{line:propagate-knownvc-clock}}.
      By line~\code{\ref{alg:unistore-replication}}{\ref{line:propagate-knownvc-clock}}
      and Assumption~\ref{assumption:clock},
      \begin{align*}
        \knownVC^{m}_{d}(\realtime_{1})[d] &= \clockVar^{m}_{d}(\realtime_{1}) \\
          &< \clockVar^{m}_{d}(\realtime_{2}) \\
          &= \knownVC^{m}_{d}(\realtime_{2})[d].
      \end{align*}
    \item $\textsc{Case II}$:
      $\knownVC^{m}_{d}(\realtime_{1})[d]$ is set
      at line~\code{\ref{alg:unistore-replication}}{\ref{line:propagate-knownvc-clock}}
      and $\knownVC^{m}_{d}(\realtime_{2})[d]$ is set
      at line~\code{\ref{alg:unistore-replication}}{\ref{line:propagate-knownvc-ts}}.
      By line~\code{\ref{alg:unistore-replication}}{\ref{line:propagate-knownvc-clock}},
      \[
        \knownVC^{m}_{d}(\realtime_{1})[d] = \clockVar^{m}_{d}(\realtime_{1}).
      \]
      By the fact that $\preparedcausal^{m}_{d}(\realtime_{1}) = \emptyset$,
      $\realtime_{2} > \realtime_{1}$,
      and line~\code{\ref{alg:unistore-replica}}{\ref{line:preparecausal-ts}},
      \begin{align*}
        \forall \langle \_, \_, &\tsvar \rangle \in\; \preparedcausal^{m}_{d}(\realtime_{2}).\; \\
          &\tsvar > \clockVar^{m}_{d}(\realtime_{1}) = \knownVC^{m}_{d}(\realtime_{1})[d].
      \end{align*}
      Therefore, by line~\code{\ref{alg:unistore-replication}}{\ref{line:propagate-knownvc-ts}},
      \begin{align*}
        &\knownVC^{m}_{d}(\realtime_{1})[d] \\
        &\quad \le \min\set{\tsvar \mid \langle \_, \_, \tsvar \rangle \in \preparedcausal^{m}_{d}(\realtime_{2})} - 1 \\
        &\quad = \knownVC^{m}_{d}(\realtime_{2})[d].
      \end{align*}
    \item $\textsc{Case III}$:
      $\knownVC^{m}_{d}(\realtime_{1})[d]$ is set
      at line~\code{\ref{alg:unistore-replication}}{\ref{line:propagate-knownvc-ts}}
      and $\knownVC^{m}_{d}(\realtime_{2})[d]$ is set
      at line~\code{\ref{alg:unistore-replication}}{\ref{line:propagate-knownvc-clock}}.
      Let $\tvar_{1}$ be the transaction in $\preparedcausal^{m}_{d}(\realtime_{1})$
      that has the minimum $\tsvar$. Formally,
      \begin{align*}
        \tvar_{1} \triangleq \argmin\limits_{\tvar}\set{\tsvar \mid \langle \tidselector(\tvar), \_, \tsvar \rangle
          \in \preparedcausal^{m}_{d}(\realtime_{1})}.
      \end{align*}
      By lines~\code{\ref{alg:unistore-replication}}{\ref{line:propagate-knownvc-ts}},
      \code{\ref{alg:unistore-coord}}{\ref{line:commitcausal-commitvc-d}},
      \code{\ref{alg:unistore-replica}}{\ref{line:commit-wait-clock}},
      and \code{\ref{alg:unistore-replication}}{\ref{line:propagate-knownvc-clock}},
      \begin{align*}
        \knownVC^{m}_{d}(\realtime_{1})[d] &< \commitVC(\tvar_{1})[d] \\
          &\le \clockVar^{m}_{d}(\realtime_{2}) \\
          &= \knownVC^{m}_{d}(\realtime_{2})[d].
      \end{align*}
    \item $\textsc{Case IV}$:
      Both $\knownVC^{m}_{d}(\realtime_{1})[d]$ and $\knownVC^{m}_{d}(\realtime_{2})[d]$ are set
      at line~\code{\ref{alg:unistore-replication}}{\ref{line:propagate-knownvc-ts}}.
      By lines~\code{\ref{alg:unistore-replication}}{\ref{line:propagate-knownvc-ts}}
      and \code{\ref{alg:unistore-replica}}{\ref{line:preparecausal-ts}},
      \begin{align*}
        &\knownVC^{m}_{d}(\realtime_{1})[d] \\
          &\quad = \min\set{\tsvar \mid \langle \_, \_, \tsvar \rangle \in \preparedcausal^{m}_{d}(\realtime_{1})} - 1 \\
          &\quad \le \min\set{\tsvar \mid \langle \_, \_, \tsvar \rangle \in \preparedcausal^{m}_{d}(\realtime_{2})} - 1 \\
          &\quad = \knownVC^{m}_{d}(\realtime_{2})[d].
      \end{align*}
  \end{itemize}
\end{proof}

\begin{applemma} \label{lemma:knownvc-i-nondecreasing}
  For $i \in \D \setminus \set{d}$,
  $\knownVC^{m}_{d}[i]$ at any replica $p^{m}_{d}$ in data center $d$
  is non-decreasing.
\end{applemma}

\begin{proof} \label{proof:knownvc-i-nondecreasing}
  Note that $\knownVC^{m}_{d}[i]$ ($i \in \D \setminus \set{d}$)
  can be updated only
  at lines~\code{\ref{alg:unistore-replication}}{\ref{line:replicate-knownvc}}
  and \code{\ref{alg:unistore-replication}}{\ref{line:heartbeat-knownvc}}.
  Therefore, this lemma holds due to
  lines~\code{\ref{alg:unistore-replication}}{\ref{line:replicate-precondition}}
  and \code{\ref{alg:unistore-replication}}{\ref{line:heartbeat-precondition}}.
\end{proof}

\begin{applemma} \label{lemma:knownvc-nondecreasing}
  For $i \in \D$, $\knownVC^{m}_{d}[i]$ at any replica $p^{m}_{d}$
  in data center $d$ is non-decreasing.
\end{applemma}

\begin{proof} \label{proof:knownvc-nondecreasing}
  By Lemmas~\ref{lemma:knownvc-d-nondecreasing}
  and \ref{lemma:knownvc-i-nondecreasing}.
\end{proof}

\begin{applemma} \label{lemma:knownvc-d-clock}
  For any replica $p^{m}_{d}$ in data center $d$,
  \[
    \knownVC^{m}_{d}[d] \le \clockVar^{m}_{d}.
  \]
\end{applemma}

\begin{proof} \label{proof:knownvc-d-clock}
  Note that $\knownVC^{m}_{d}[d]$ is updated only
  at lines~\code{\ref{alg:unistore-replication}}{\ref{line:propagate-knownvc-clock}}
  or \code{\ref{alg:unistore-replication}}{\ref{line:propagate-knownvc-ts}}.
  \begin{itemize}
    \item $\textsc{Case I}$: $\knownVC^{m}_{d}[d]$ is updated
      at line~\code{\ref{alg:unistore-replication}}{\ref{line:propagate-knownvc-clock}}.
      By Assumption~\ref{assumption:clock},
      \[
        \knownVC^{m}_{d}[d] \le \clockVar^{m}_{d}.
      \]
    \item $\textsc{Case II}$: $\knownVC^{m}_{d}[d]$ is updated
      at line~\code{\ref{alg:unistore-replication}}{\ref{line:propagate-knownvc-ts}}.
      By line~\code{\ref{alg:unistore-replica}}{\ref{line:preparecausal-ts}},
      immediately after this update,
      \[
        \knownVC^{m}_{d}[d] < \clockVar^{m}_{d}.
      \]
  \end{itemize}
\end{proof}

\begin{applemma} \label{lemma:knownvc-commitvc-d}
  Let $p^{m}_{d}$ be a replica in data center $d$.
  Consider $\knownVC^{m}_{d}(\realtime)[d]$ at time $\realtime$
  and transaction $\tvar \in \causaltxs$ committed at $p^{m}_{d}$
  after time $\realtime$. Then
  \[
    \commitVC(\tvar)[d] > \knownVC^{m}_{d}(\realtime)[d].
  \]
\end{applemma}

\begin{proof} \label{proof:knownvc-commitvc-d}
  Suppose that before time $\realtime$,
  $\knownVC^{m}_{d}[d]$ is last updated at time $\realtime' < \realtime$.
  Therefore,
  \[
    \knownVC^{m}_{d}(\realtime)[d] = \knownVC^{m}_{d}(\realtime')[d].
  \]
  We distinguish between two cases according to whether
  \[
    \preparedcausal^{m}_{d}(\realtime') = \emptyset
  \]
  when $\knownVC^{m}_{d}[d]$ is updated at time $\realtime'$.
  \begin{itemize}
    \item $\textsc{Case I}$: $\preparedcausal^{m}_{d}(\realtime') = \emptyset$.
      By line~\code{\ref{alg:unistore-replication}}{\ref{line:propagate-knownvc-clock}},
      \[
        \knownVC^{m}_{d}(\realtime')[d] = \clockVar^{m}_{d}(\realtime').
      \]
      By line~\code{\ref{alg:unistore-replica}}{\ref{line:preparecausal-ts}},
      line~\code{\ref{alg:unistore-coord}}{\ref{line:commitcausal-commitvc-d}},
      and Assumption~\ref{assumption:clock},
      \[
        \commitVC(\tvar)[d] > \clockVar^{m}_{d}(\realtime').
      \]
      Therefore,
      \begin{align*}
        \commitVC(\tvar)[d] &> \knownVC^{m}_{d}(\realtime')[d] \\
                          &= \knownVC^{m}_{d}(\realtime)[d].
      \end{align*}
    \item $\textsc{Case II}$: $\preparedcausal^{m}_{d}(\realtime') \neq \emptyset$.
      We further distinguish between two cases according to whether
      \[
        \langle \tidselector(\tvar), \_, \_ \rangle \in \preparedcausal^{m}_{d}(\realtime').
      \]
      \begin{itemize}
        \item $\textsc{Case II-1}$:
          $\langle \tidselector(\tvar), \_, \tsvar \rangle \in \preparedcausal^{m}_{d}(\realtime')$.
          By lines~\code{\ref{alg:unistore-replication}}{\ref{line:propagate-knownvc-ts}}
          and \code{\ref{alg:unistore-coord}}{\ref{line:commitcausal-commitvc-d}},
          \begin{align*}
            \commitVC(\tvar)[d] &\ge \tsvar \\
                              &> \knownVC^{m}_{d}(\realtime')[d] \\
                              &= \knownVC^{m}_{d}(\realtime)[d].
          \end{align*}
        \item $\textsc{Case II-2}$:
          $\langle \tidselector(\tvar), \_, \_ \rangle \notin \preparedcausal^{m}_{d}(\realtime')$.
          By Lemma~\ref{lemma:knownvc-d-clock},
          Assumption~\ref{assumption:clock},
          line~\code{\ref{alg:unistore-replica}}{\ref{line:preparecausal-ts}},
          and line~\code{\ref{alg:unistore-coord}}{\ref{line:commitcausal-commitvc-d}},
          \begin{align*}
            \commitVC(\tvar)[d] &> \knownVC^{m}_{d}(\realtime')[d] \\
                              &= \knownVC^{m}_{d}(\realtime)[d].
          \end{align*}
      \end{itemize}
  \end{itemize}
\end{proof}

\begin{applemma} \label{lemma:knownvc-local-d}
  Let $\tvar \in \causaltxs$ be a causal transaction
  that originates at data center $d$ and accesses partition $m$.
  If
  \[
    \commitVC(\tvar)[d] \le \knownVC^{m}_{d}[d],
  \]
  then
  \[
    \log(\tvar)[m] \subseteq \oplog^{m}_{d}.
  \]
\end{applemma}

\begin{proof} \label{proof:knownvc-local-d}
  Suppose that the value $\knownVC^{m}_{d}[d]$ is set at time $\realtime$.
  By Lemma~\ref{lemma:knownvc-commitvc-d},
  $\tvar$ is committed at $p^{m}_{d}$ before time $\realtime$.
  Therefore, by line~\code{\ref{alg:unistore-replica}}{\ref{line:commit-oplog}},
  \[
    \log(\tvar)[m] \subseteq \oplog^{m}_{d}.
  \]
\end{proof}

The following lemmas consider the replication and forwarding
of causal transactions.

\begin{applemma} \label{lemma:replication-order}
  Let $p^{m}_{d}$ be a replica in data center $d$.
  Let $\tvar_{1}$ and $\tvar_{2}$ be two transactions
  replicated by $p^{m}_{d}$ to sibling replicas
  at time $\realtime_{1}$ and $\realtime_{2}$
  (line~\code{\ref{alg:unistore-replication}}{\ref{line:propagate-call-replicate}}),
  respectively. Then
  \[
    \realtime_{1} < \realtime_{2} \implies \commitVC(\tvar_{1})[d] < \commitVC(\tvar_{2})[d].
  \]
\end{applemma}

\begin{proof} \label{proof:replication-order}
  Since $\tvar_{1}$ is replicated at time $\realtime_{1}$,
  by line~\code{\ref{alg:unistore-replication}}{\ref{line:propagate-txs}},
  \[
    \commitVC(\tvar_{1})[d] \le \knownVC^{m}_{d}(\realtime_{1})[d].
  \]
  Assume that $\realtime_{1} < \realtime_{2}$.
  We distinguish between two cases according to whether
  \[
    \langle \tidselector(\tvar_{2}), \_, \_, \_ \rangle \in \committedcausal^{m}_{d}(\realtime_{1})[d].
  \]
  \begin{itemize}
    \item $\textsc{Case I}$:
      $\langle \tidselector(\tvar_{2}), \_, \_, \_ \rangle \in \committedcausal^{m}_{d}(\realtime_{1})[d]$.
      Since $\tvar_{2}$ is not replicated at time $\realtime_{1}$,
      by line~\code{\ref{alg:unistore-replication}}{\ref{line:propagate-txs}},
      \[
        \commitVC(\tvar_{2})[d] > \knownVC^{m}_{d}(\realtime_{1})[d].
      \]
    \item $\textsc{Case II}$:
      $\langle \tidselector(\tvar_{2}), \_, \_, \_, \rangle \notin \committedcausal^{m}_{d}(\realtime_{1})[d]$.
      Thus, $\tvar_{2}$ is committed at $p^{m}_{d}$ after time $\realtime_{1}$.
      By Lemma~\ref{lemma:knownvc-commitvc-d},
      \[
        \commitVC(\tvar_{2})[d] > \knownVC^{m}_{d}(\realtime_{1})[d].
      \]
  \end{itemize}
  Therefore, in either case,
  \[
    \commitVC(\tvar_{1})[d] < \commitVC(\tvar_{2})[d].
  \]
\end{proof}

\begin{applemma} \label{lemma:heartbeat-replication-order}
  Let $p^{m}_{d}$ be a replica in data center $d$.
  Consider a heartbeat $\knownVC^{m}_{d}(\realtime_{1})[d]$
  sent by $p^{m}_{d}$ at time $\realtime_{1}$
  (line~\code{\ref{alg:unistore-replication}}{\ref{line:propagate-call-heartbeat}}).
  Let $\tvar$ be a transaction replicated by $p^{m}_{d}$ at time $\realtime_{2}$
  (line~\code{\ref{alg:unistore-replication}}{\ref{line:propagate-call-replicate}}).
  Then
  \[
    \realtime_{1} < \realtime_{2} \iff \knownVC^{m}_{d}(\realtime_{1})[d] < \commitVC(\tvar)[d].
  \]
\end{applemma}

\begin{proof} \label{proof:heartbeat-replication-order}
  We first show that
  \[
    \realtime_{1} < \realtime_{2} \implies \knownVC^{m}_{d}(\realtime_{1})[d] < \commitVC(\tvar)[d].
  \]
  Assume that $\realtime_{1} < \realtime_{2}$.
  We distinguish between two cases according to whether
  \[
    \langle \tidselector(\tvar), \_, \_, \_ \rangle \in \committedcausal^{m}_{d}(\realtime_{1})[d].
  \]
  \begin{itemize}
    \item $\textsc{Case I}$:
      $\langle \tidselector(\tvar), \_, \_, \_ \rangle \in \committedcausal^{m}_{d}(\realtime_{1})[d]$.
      By line~\code{\ref{alg:unistore-replication}}{\ref{line:propagate-txs}},
      \[
        \commitVC(\tvar)[d] > \knownVC^{m}_{d}(\realtime_{1})[d].
      \]
    \item $\textsc{Case II}$:
      $\langle \tidselector(\tvar), \_, \_, \_ \rangle \notin \committedcausal^{m}_{d}(\realtime_{1})[d]$.
      Thus, $\tvar$ is committed at $p^{m}_{d}$ after time $\realtime_{1}$.
      By Lemma~\ref{lemma:knownvc-commitvc-d},
      \[
        \commitVC(\tvar)[d] > \knownVC^{m}_{d}(\realtime_{1})[d].
      \]
  \end{itemize}

  Next we show that (note that $\realtime_{1} \neq \realtime_{2}$)
  \[
    \realtime_{2} < \realtime_{1} \implies \commitVC(\tvar)[d] \le \knownVC^{m}_{d}(\realtime_{1})[d].
  \]
  Since $\tvar$ is replicated by $p^{m}_{d}$ at time $\realtime_{2}$,
  by line~\code{\ref{alg:unistore-replication}}{\ref{line:propagate-txs}},
  \[
    \commitVC(\tvar)[d] \le \knownVC^{m}_{d}(\realtime_{2})[d].
  \]
  Assume that $\realtime_{2} < \realtime_{1}$.
  By Lemma~\ref{lemma:knownvc-d-nondecreasing},
  \[
    \knownVC^{m}_{d}(\realtime_{2})[d] \le \knownVC^{m}_{d}(\realtime_{1})[d].
  \]
  Putting it together yields
  \[
    \commitVC(\tvar)[d] \le \knownVC^{m}_{d}(\realtime_{1})[d].
  \]
\end{proof}

\begin{applemma} \label{lemma:committedcausal-i}
  Let $p^{m}_{d}$ be a replica in data center $d$. Then
  \begin{align*}
    &\forall i \neq d.\;
      \forall \langle \tidselector(\tvar), \_, \_, \_ \rangle \in \committedcausal^{m}_{d}[i]. \\
        &\quad \commitVC(\tvar)[i] \le \knownVC^{m}_{d}[i].
  \end{align*}
\end{applemma}

\begin{proof} \label{proof:committedcausal-i}
  By lines~\code{\ref{alg:unistore-replication}}{\ref{line:replicate-committedcausal}}
  and \code{\ref{alg:unistore-replication}}{\ref{line:replicate-knownvc}}
  and Lemma~\ref{lemma:knownvc-i-nondecreasing}.
\end{proof}

\begin{applemma} \label{lemma:globalmatrix-nondecreasing}
  For $j \neq d$ and $i \notin \set{d, j}$,
  $\globalmatrix^{m}_{d}[i][j]$ at any replica $p^{m}_{d}$
  in data center $d$ is non-decreasing.
\end{applemma}

\begin{proof} \label{proof:globalmatrix-nondecreasing}
  Note that $\globalmatrix^{m}_{d}[i][j]$ can be updated only
  at line~\code{\ref{alg:unistore-clock}}{\ref{line:knownvcglobal-globalmatrix}}.
  Therefore, by Lemma~\ref{lemma:knownvc-i-nondecreasing},
  it is non-decreasing.
\end{proof}

\begin{applemma} \label{lemma:forwarding-order}
  Let $p^{m}_{d}$ be a replica in data center $d$.
  Let $\tvar_{1}$ and $\tvar_{2}$ be two transactions
  that originate at data center $j \neq d$
  and are forwarded by $p^{m}_{d}$ to
  sibling replica $p^{m}_{i}$ in data center $i \notin \set{d, j}$
  at time $\realtime_{1}$ and $\realtime_{2}$
  (line~\code{\ref{alg:unistore-replication}}{\ref{line:forward-call-replicate}}),
  respectively. Then
  \[
    \realtime_{1} < \realtime_{2} \implies \commitVC(\tvar_{1})[j] < \commitVC(\tvar_{2})[j].
  \]
\end{applemma}

\begin{proof} \label{proof:forwarding-order}
  Since $\tvar_{1}$ is forwarded by $p^{m}_{d}$ at time $\realtime_{1}$,
  by line~\code{\ref{alg:unistore-replication}}{\ref{line:forward-txs}},
  \[
    \langle \tidselector(\tvar_{1}), \_, \_, \_ \rangle \in \committedcausal^{m}_{d}(\realtime_{1})[j].
  \]
  By Lemmas~\ref{lemma:committedcausal-i} and \ref{lemma:knownvc-i-nondecreasing},
  \begin{align}
    \commitVC(\tvar_{1})[j] \le \knownVC^{m}_{d}(\realtime_{1})[j].
    \label{eqn:tid1-knownvc}
  \end{align}
  Assume that $\realtime_{1} < \realtime_{2}$.
  We first argue that
  \begin{align}
    \langle \tidselector(\tvar_{2}), \_, \_, \_ \rangle \notin \committedcausal^{m}_{d}(\realtime_{1})[j].
    \label{eqn:tid2-committedcausal-t1}
  \end{align}
  Otherwise, by line~\code{\ref{alg:unistore-replication}}{\ref{line:forward-txs}},
  \[
    \commitVC(\tvar_{2})[j] \le \globalmatrix^{m}_{d}(\realtime_{1})[i][j].
  \]
  By Lemma~\ref{lemma:globalmatrix-nondecreasing},
  \[
    \commitVC(\tvar_{2})[j] \le \globalmatrix^{m}_{d}(\realtime_{2})[i][j].
  \]
  Therefore, by line~\code{\ref{alg:unistore-replication}}{\ref{line:forward-txs}},
  $\tvar_{2}$ would not be forwarded by $p^{m}_{d}$ to $p^{m}_{i}$ at time $\realtime_{2}$.
  Thus, (\ref{eqn:tid2-committedcausal-t1}) holds.
  Since $\tvar_{2}$ is forwarded by $p^{m}_{d}$ to $p^{m}_{i}$ at time $\realtime_{2}$,
  \[
    \langle \tidselector(\tvar_{2}), \_, _, \_ \rangle \in \committedcausal^{m}_{d}(\realtime_{2})[j].
  \]
  By Lemma~\ref{lemma:knownvc-i-nondecreasing}
  and line~\code{\ref{alg:unistore-replication}}{\ref{line:replicate-precondition}},
  \begin{align}
    \commitVC(\tvar_{2})[j] > \knownVC^{m}_{d}(\realtime_{1})[j].
    \label{eqn:tid2-knownvc}
  \end{align}
  Putting (\ref{eqn:tid1-knownvc}) and (\ref{eqn:tid2-knownvc}) together yields
  \[
    \commitVC(\tvar_{1})[j] < \commitVC(\tvar_{2})[j].
  \]
\end{proof}

\begin{applemma} \label{lemma:heartbeat-forwarding-order}
  Let $p^{m}_{d}$ be a replica in data center $d$.
  Consider a heartbeat $\knownVC^{m}_{d}(\realtime_{1})[j]$ ($j \neq d$)
  sent by $p^{m}_{d}$ to sibling replica $p^{m}_{i}$
  in data center $i \notin \set{d, j}$ at time $\realtime_{1}$
  (line~\code{\ref{alg:unistore-replication}}{\ref{line:forward-call-heartbeat}}).
  Let $\tvar$ be a transaction that originates at data center $j$
  and is forwarded by $p^{m}_{d}$ to $p^{m}_{i}$ at time $\realtime_{2}$
  (line~\code{\ref{alg:unistore-replication}}{\ref{line:forward-call-replicate}}).
  Then
  \[
    \realtime_{1} < \realtime_{2} \iff \knownVC^{m}_{d}(\realtime_{1})[j] < \commitVC(\tvar)[j].
  \]
\end{applemma}

\begin{proof} \label{proof:heartbeat-forwarding-order}
  We first show that
  \[
    \realtime_{1} < \realtime_{2} \implies \knownVC^{m}_{d}(\realtime_{1})[j] < \commitVC(\tvar)[j].
  \]
  Assume that $\realtime_{1} < \realtime_{2}$.
  We first argue that
  \begin{align}
    \langle \tidselector(\tvar), \_, \_, \_ \rangle \notin \committedcausal^{m}_{d}(\realtime_{1})[j].
    \label{eqn:tid-notin-committedcausal-t1}
  \end{align}
  Otherwise, since $\tvar$ is not forwarded at time $\realtime_{1}$,
  by line~\code{\ref{alg:unistore-replication}}{\ref{line:forward-txs}},
  \[
    \commitVC(\tvar)[j] \le \globalmatrix^{m}_{d}(\realtime_{1})[i][j].
  \]
  By Lemma~\ref{lemma:globalmatrix-nondecreasing},
  \[
    \commitVC(\tvar)[j] \le \globalmatrix^{m}_{d}(\realtime_{2})[i][j].
  \]
  Therefore, by line~\code{\ref{alg:unistore-replication}}{\ref{line:forward-txs}},
  $\tvar$ would not be forwarded by $p^{m}_{d}$ to $p^{m}_{i}$ at time $\realtime_{2}$.
  Thus, (\ref{eqn:tid-notin-committedcausal-t1}) holds.
  Since $\tvar$ is forwarded by $p^{m}_{d}$ to $p^{m}_{i}$ at time $\realtime_{2}$,
  \[
    \langle \tidselector(\tvar), \_, \_, \_ \rangle \in \committedcausal^{m}_{d}(\realtime_{2})[j].
  \]
  By Lemma~\ref{lemma:knownvc-i-nondecreasing}
  and line~\code{\ref{alg:unistore-replication}}{\ref{line:replicate-precondition}},
  \[
    \knownVC^{m}_{d}(\realtime_{1})[j] < \commitVC(\tvar)[j].
  \]

  Next we show that (note that $\realtime_{1} \neq \realtime_{2}$)
  \[
    \realtime_{2} < \realtime_{1} \implies \commitVC(\tvar)[j] \le \knownVC^{m}_{d}(\realtime_{1})[j].
  \]
  Since $\tvar$ is forwarded by $p^{m}_{d}$ to $p^{m}_{i}$ at time $\realtime_{2}$,
  by line~\code{\ref{alg:unistore-replication}}{\ref{line:forward-txs}},
  \[
    \langle \tidselector(\tvar), \_, \_, \_ \rangle \in \committedcausal^{m}_{d}(\realtime_{2})[j].
  \]
  By Lemmas~\ref{lemma:committedcausal-i} and \ref{lemma:knownvc-i-nondecreasing},
  \[
    \commitVC(\tvar)[j] \le \knownVC^{m}_{d}(\realtime_{2})[j].
  \]
  Assume that $\realtime_{2} < \realtime_{1}$.
  By Lemma~\ref{lemma:knownvc-i-nondecreasing},
  \[
    \knownVC^{m}_{d}(\realtime_{2})[j] \le \knownVC^{m}_{d}(\realtime_{1})[j].
  \]
  Putting it together yields
  \[
    \commitVC(\tvar)[j] \le \knownVC^{m}_{d}(\realtime_{1})[j].
  \]
\end{proof}

\begin{applemma} \label{lemma:replication-knownvc}
  Let $\tvar \in \causaltxs$ be a causal transaction
  that originates at data center $i$ and accesses partition $m$.
  If
  \[
    \commitVC(\tvar)[i] \le \knownVC^{m}_{d}[i]
  \]
  for replica $p^{m}_{d}$ in data center $d \neq i$,
  then
  \[
    \log(\tvar)[m] \subseteq \oplog^{m}_{d}.
  \]
\end{applemma}

\begin{proof} \label{proof:replication-knownvc}
  Note that for $i \in \D \setminus \set{d}$,
  $\knownVC^{m}_{d}[i]$ can be updated
  only at lines~\code{\ref{alg:unistore-replication}}{\ref{line:replicate-knownvc}}
  or \code{\ref{alg:unistore-replication}}{\ref{line:heartbeat-knownvc}}
  due to replication of transactions or heartbeats respectively,
  either directly from data center $i$
  (line~\code{\ref{alg:unistore-replication}}{\ref{line:function-propagate}})
  or indirectly from a third data center $j \neq i$
  (line~\code{\ref{alg:unistore-replication}}{\ref{line:function-forward}}).

  We proceed by induction on the length of the execution.
  In the following, for replica $p^{m}_{d}$ in data center $d \in \D$,
  we denote the value of $\knownVC^{m}_{d}$ (resp. $\oplog^{m}_{d}$)
  after $k$ steps in an execution
  by $\knownVC^{m}_{d}(k)$ (resp. $\oplog^{m}_{d}(k)$).
  \begin{itemize}
    \item \emph{Base Case.} $k = 0$.
      It holds trivially,
      since for replica $p^{m}_{d}$ in any data center $d \neq i$,
      \[
        \knownVC^{m}_{d}(0)[i] = 0.
      \]
    \item \emph{Induction Hypothesis.}
      Suppose that for any execution of length $k$, we have
      \begin{align*}
        &\forall d \in \D \setminus \set{i}.\;
          \forall \tvar \in \causaltxs.\; \\
            &\quad \big(\commitVC(\tvar)[i] \le \knownVC^{m}_{d}(k)[i] \\
            &\quad\quad \implies \log(\tvar)[m] \subseteq \oplog^{m}_{d}(k)\big).
      \end{align*}
    \item \emph{Induction Step.}
      Consider an execution of length $k + 1$.
      If the $(k+1)$-st step of this execution does not update
      $\knownVC^{m}_{d}[i]$ for replica $p^{m}_{d}$ in any data center $d \neq i$,
      then by the induction hypothesis,
      \begin{align*}
        &\forall d \in \D \setminus \set{i}.\;
          \forall \tvar \in \causaltxs.\; \\
            &\quad \big(\commitVC(\tvar)[i] \le \knownVC^{m}_{d}(k+1)[i] \\
            &\quad\quad \implies \log(\tvar)[m] \subseteq \oplog^{m}_{d}(k+1)\big).
      \end{align*}
      Otherwise, we perform a case analysis according to
      how $\knownVC^{m}_{d}[i]$ of replica $p^{m}_{d}$ in data center $d \neq i$
      is updated in the $(k+1)$-st step.
      \begin{itemize}
        \item \textsc{Case I:}
          $\knownVC^{m}_{d}[i]$ is updated due to delivery of a message
          from data center $i$.
          By Lemmas~\ref{lemma:knownvc-d-nondecreasing},
          \ref{lemma:replication-order}, and \ref{lemma:heartbeat-replication-order},
          local transactions and heartbeats are propagated by $p^{m}_{i}$ to sibling replicas
          in increasing order of their local timestamps $\commitvc[i]$
          and $\knownVC^{m}_{i}[i]$ values.
          Therefore, by Assumption~\ref{assumption:message}
          and the induction hypothesis,
          \begin{align*}
            &\forall \tvar \in \causaltxs.\; \\
              &\quad \big(\commitVC(\tvar)[i] \le \knownVC^{m}_{d}(k+1)[i] \\
              &\quad\quad \implies \log(\tvar)[m] \subseteq \oplog^{m}_{d}(k+1)\big).
          \end{align*}
        \item \textsc{Case II:}
          $\knownVC^{m}_{d}[i]$ is updated due to delivery of a message
          from a third data center $j \neq i$.
          By Lemmas~\ref{lemma:knownvc-i-nondecreasing},
          \ref{lemma:forwarding-order}, and \ref{lemma:heartbeat-forwarding-order},
          transactions originating at data center $i$ and heartbeats are forwarded
          by some replica, say $p^{m}_{j} (j \neq i)$,
          to sibling replicas in increasing order of
          their local timestamps $\commitvc[i]$ and $\knownVC^{m}_{j}[i]$ values.
          Therefore, by Assumption~\ref{assumption:message}
          and the induction hypothesis,
          \begin{align*}
            &\forall \tvar \in \causaltxs.\; \\
              &\quad \big(\commitVC(\tvar)[i] \le \knownVC^{m}_{d}(k+1)[i] \\
                &\quad\quad \implies \log(\tvar)[m] \subseteq \oplog^{m}_{d}(k+1)\big).
          \end{align*}
      \end{itemize}
  \end{itemize}
\end{proof}

\begin{applemma}[\prop{1}] \label{lemma:knownvc-causal}
  Let $\tvar \in \causaltxs$ be a causal transaction
  that originates at data center $i$
  and accesses partition $m$.
  If
  \[
    \commitVC(\tvar)[i] \le \knownVC^{m}_{d}[i]
  \]
  for replica $p^{m}_{d}$ in data center $d$,
  then
  \[
    \log(\tvar)[m] \subseteq \oplog^{m}_{d}.
  \]
\end{applemma}

\begin{proof} \label{proof:knownvc-causal}
  By Lemmas~\ref{lemma:knownvc-local-d} and \ref{lemma:replication-knownvc}.
\end{proof}
\subsubsection{Properties of $\stableVC$}
\label{sss:stablevc}

\begin{applemma} \label{lemma:stablevc-nondecreasing}
  For $i \in \D$, $\stableVC^{m}_{d}[i]$
  at any replica $p^{m}_{d}$ in data center $d$ is non-decreasing.
\end{applemma}

\begin{proof} \label{proof:stablevc-nondecreasing}
  Note that $\stableVC^{m}_{d}[i]$ ($i \in \D$) can be updated
  only at line~\code{\ref{alg:unistore-clock}}{\ref{line:knownvclocal-stablevc-causal}}.
  By Lemma~\ref{lemma:knownvc-nondecreasing}
  and Assumption~\ref{assumption:message},
  $\stableVC^{m}_{d}[i]$ is non-decreasing.
\end{proof}

\begin{applemma}[\prop{2}] \label{lemma:stablevc-knownvc}
  For any replica $p^{m}_{d}$ in data center $d$,
  \[
    \forall i \in \D.\; \forall n \in \P.\;
      \stableVC^{m}_{d}[i] \le \knownVC^{n}_{d}[i].
  \]
\end{applemma}

\begin{proof} \label{proof:stablevc-knownvc}
  Note that $\stableVC^{m}_{d}[i]$ ($i \in \D$)
  can be updated only
  at line~\code{\ref{alg:unistore-clock}}{\ref{line:knownvclocal-stablevc-causal}}.
  By the way $\stableVC^{m}_{d}[i]$ is updated
  and Lemmas~\ref{lemma:knownvc-d-nondecreasing}
  and \ref{lemma:knownvc-i-nondecreasing},
  \[
    \forall n \in \P.\; \stableVC^{m}_{d}[i] \le \knownVC^{n}_{d}[i].
  \]
\end{proof}

\begin{applemma} \label{lemma:replication-stablevc}
  Let $\tvar \in \causaltxs$ be a causal transaction
  that originates at data center $i$
  and accesses partition $n$.
  If
  \[
    \commitVC(\tvar)[i] \le \stableVC^{m}_{d}[i]
  \]
  for some replica $p^{m}_{d}$ in data center $d$,
  then
  \[
    \log(\tvar)[n] \subseteq \oplog^{n}_{d}.
  \]
\end{applemma}

\begin{proof} \label{proof:replication-stablevc}
  By Lemma~\ref{lemma:stablevc-knownvc},
  \[
    \stableVC^{m}_{d}[i] \le \knownVC^{n}_{d}[i].
  \]
  Therefore,
  \[
    \commitVC(\tvar)[i] \le \knownVC^{n}_{d}[i].
  \]
  By Lemma~\ref{lemma:replication-knownvc},
  \[
    \log(\tvar)[n] \subseteq \oplog^{n}_{d}.
  \]
\end{proof}
\subsubsection{Properties of $\uniformVC$}
\label{sss:uniformvc}

\begin{applemma} \label{lemma:uniformvc-nondecreasing}
  For $i \in \D$, $\uniformVC^{m}_{d}[i]$
  at any replica $p^{m}_{d}$ in data center $d$ is non-decreasing.
\end{applemma}

\begin{proof} \label{proof:uniformvc-nondecreasing}
  Note that whenever $\uniformVC^{m}_{d}[i]$ is updated
  at lines~\code{\ref{alg:unistore-coord}}{\ref{line:start-uniformvc}},
  \code{\ref{alg:unistore-replica}}{\ref{line:readkey-uniformvc}},
  \code{\ref{alg:unistore-replica}}{\ref{line:preparecausal-uniformvc}},
  or \code{\ref{alg:unistore-clock}}{\ref{line:stablevc-uniformvc}},
  we take the maximum of it and some scalar value.
\end{proof}

\begin{applemma} \label{lemma:pastvc-uniformvc-except-d}
  Let $e \in E$ be an event issued by client $\cl$
  to replica $p^{m}_{d}$ in data center $d$. Then
  \begin{align*}
    &e \in E \setminus \Fence \implies \\
      &\quad \forall i \in \D \setminus \set{d}.\;
      (\pastVC_{\cl})_{e}[i] \le (\uniformVC^{m}_{d})_{e}[i],
  \end{align*}
  and
  \[
    e \in \Fence \implies (\pastVC_{\cl})_{e}[d] \le (\uniformVC^{m}_{d})_{e}[d].
  \]
\end{applemma}

\begin{proof} \label{proof:pastvc-uniformvc-except-d}
  We perform a case analysis according to the type of event $e$.
  \begin{itemize}
    \item $\textsc{Case I}$: $e \in S$.
      By line~\code{\ref{alg:unistore-coord}}{\ref{line:start-uniformvc}},
      \[
        \forall i \in \D \setminus \set{d}.\;
          (\pastVC_{\cl})_{e}[i] \le (\uniformVC^{m}_{d})_{e}[i].
      \]
    \item $\textsc{Case II}$: $e \in R \cup U$.
      In this case,
      \[
        (\pastVC_{\cl})_{e} = (\pastVC_{\cl})_{\startoftx(e)}.
      \]
      By \textsc{Case I},
      \[
        \forall i \in \D \setminus \set{d}.\;
          (\pastVC_{\cl})_{\startoftx(e)}[i] \le (\uniformVC^{m}_{d})_{\startoftx(e)}[i].
      \]
      By Lemma~\ref{lemma:uniformvc-nondecreasing},
      \[
        \forall i \in \D \setminus \set{d}. \\
          (\uniformVC^{m}_{d})_{\startoftx(e)} \le (\uniformVC^{m}_{d})_{e}.
      \]
      Putting it together yields
      \[
        \forall i \in \D \setminus \set{d}.\;
          (\pastVC_{\cl})_{e}[i] \le (\uniformVC^{m}_{d})_{e}[i].
      \]
    \item $\textsc{Case III}$: $e \in C_{\causalentry}$.
      By line~\code{\ref{alg:unistore-client}}{\ref{line:commitcausaltx-pastvc}},
      \[
        (\pastVC_{\cl})_{e} = vc_{(\commitcausaltx, e)}.
      \]
      By lines~\code{\ref{alg:unistore-coord}}{\ref{line:commitcausal-return-ro}},
      \code{\ref{alg:unistore-coord}}{\ref{line:commitcausal-commitvc}},
      and \code{\ref{alg:unistore-coord}}{\ref{line:commitcausal-return}},
      \begin{align*}
        &\forall i \in \D \setminus \set{d}. \\
          &\quad vc_{(\commitcausaltx, e)}[i] = (\snapVC^{m}_{d})_{e}[\txfunc(e)][i].
      \end{align*}
      By line~\code{\ref{alg:unistore-coord}}{\ref{line:start-snapvc}},
      \begin{align*}
        &\forall i \in \D \setminus \set{d}. \\
          &\quad (\snapVC^{m}_{d})_{e}[\txfunc(e)][i] = (\uniformVC^{m}_{d})_{\startoftx(e)}[i].
      \end{align*}
      By Lemma~\ref{lemma:uniformvc-nondecreasing},
      \[
        \forall i \in \D \setminus \set{d}. \\
          (\uniformVC^{m}_{d})_{\startoftx(e)} \le (\uniformVC^{m}_{d})_{e}.
      \]
      Putting it together yields
      \[
        \forall i \in \D \setminus \set{d}.\;
          (\pastVC_{\cl})_{e}[i] \le (\uniformVC^{m}_{d})_{e}[i].
      \]
    \item $\textsc{Case IV}$: $e \in C_{\strongentry}$.
      By line~\code{\ref{alg:unistore-client}}{\ref{line:commitstrongtx-pastvc}},
      \[
        (\pastVC_{\cl})_{e} = vc_{(\commitstrongtx, e)}.
      \]
      By (\ref{eqn:gcf-commitvc}),
      \begin{align*}
        &\forall i \in \D \setminus \set{d}. \\
          &\quad vc_{(\commitstrongtx, e)}[i] = (\snapVC^{m}_{d})_{e}[\txfunc(e)][i].
      \end{align*}
      Therefore, similar to \textsc{Case III}, we have
      \[
        \forall i \in \D \setminus \set{d}.\;
          (\pastVC_{\cl})_{e}[i] \le (\uniformVC^{m}_{d})_{e}[i].
      \]
    \item $\textsc{Case V}$: $e \in \Fence$.
      By line~\code{\ref{alg:unistore-replica}}{\ref{line:uniformbarrier-wait-uniformvc-d}},
      \[
        (\pastVC_{\cl})_{e}[d] \le (\uniformVC^{m}_{d})_{e}[d].
      \]
    \item $\textsc{Case VI}$: $e \in \Attach$.
      By line~\code{\ref{alg:unistore-replica}}{\ref{line:attach-wait-condition}},
      \[
        \forall i \in \D \setminus \set{d}.\;
          (\pastVC_{\cl})_{e}[i] \le (\uniformVC^{m}_{d})_{e}[i].
      \]
  \end{itemize}
\end{proof}

\begin{applemma} \label{lemma:pastvc-uniformvc}
  Let $\cl$ be a client and $d \triangleq \cldc(\cl)$.
  At any time,
  \[
    \forall i \in \D \setminus \set{d}.\;
      \pastVC_{\cl}[i] \le \uniformVC^{m}_{d}[i]
  \]
  for some replica $p^{m}_{d}$ in data center $d$.
\end{applemma}

\begin{proof} \label{proof:pastvc-uniformvc}
  By a simple induction on the number of events that $\cl$ issues
  and Lemmas~\ref{lemma:pastvc-uniformvc-except-d} and
  \ref{lemma:uniformvc-nondecreasing}.
\end{proof}

\begin{applemma} \label{lemma:snapshotvc-uniformvc}
  Let $\tvar$ be a transaction that originates at data center $d$.
  At any time,
  \[
    \forall i \in \D \setminus \set{d}.\;
      \snapshotVC(\tvar)[i] \le \uniformVC^{m}_{d}[i]
  \]
  for some replica $p^{m}_{d}$ in data center $d$.
\end{applemma}

\begin{proof} \label{proof:snapshotvc-uniformvc}
  By line~\code{\ref{alg:unistore-coord}}{\ref{line:start-snapvc}}
  and Lemma~\ref{lemma:uniformvc-nondecreasing}.
\end{proof}

\begin{applemma}[\prop{3}] \label{lemma:uniformvc-knownvc-f+1}
  For any replica $p^{m}_{d}$ in data center $d$,
  \begin{align*}
    &\forall i \in \D.\; \exists g \subseteq \D.\; \Big(
       |g| \ge f + 1 \land d \in g\; \land \\
      &\; \big(\forall j \in g.\; \forall n \in \P.\;
        \uniformVC^{m}_{d}[i] \le \knownVC^{n}_{j}[i] \big)\Big).
  \end{align*}
\end{applemma}

\begin{proof} \label{proof:uniformvc-knownvc-f+1}
  Fix $i \in \D$.
  We proceed by induction on the length of the execution.
  In the following, we denote the value of
  $\knownVC^{m}_{d}$, $\stableVC^{m}_{d}$, $\uniformVC^{m}_{d}$,
  $\stablematrix^{m}_{d}$, and $\pastVC_{\cl}$ (for some client $\cl$)
  after $k$ steps of an execution by
  $\knownVC^{m}_{d}(k)$, $\stableVC^{m}_{d}(k)$, $\uniformVC^{m}_{d}(k)$,
  $\stablematrix^{m}_{d}(k)$, and $\pastVC_{\cl}(k)$, respectively.
  \begin{itemize}
    \item {\it Base Case.} $k = 0$. It holds trivially since
      \[
        \uniformVC^{m}_{d}(0)[i] = 0.
      \]
    \item {\it Induction Hypothesis.}
      Suppose that for any execution of length $k$,
      for any replica $p^{m}_{d}$ in data center $d$,
      \begin{align*}
        &\exists g \subseteq \D.\; |g| \ge f + 1 \land d \in g\; \land \\
          &\quad \big(\forall j \in g.\; \forall 1 \le n \le N.\; \\
            &\qquad \uniformVC^{m}_{d}(k)[i] \le \knownVC^{n}_{j}(k)[i] \big).
      \end{align*}
    \item {\it Induction Step.}
      Consider an execution of length $k + 1$.
      If the $(k+1)$-st step of this execution does not update
      $\uniformVC^{m}_{d}[i]$ for any replica $p^{m}_{d}$ in data center $d$,
      then by the induction hypothesis and Lemma~\ref{lemma:knownvc-nondecreasing},
      \begin{align*}
        &\exists g \subseteq \D.\; |g| \ge f + 1 \land d \in g\; \land \\
          &\quad \big(\forall j \in g.\; \forall 1 \le n \le N.\; \\
            &\qquad \uniformVC^{m}_{d}(k + 1)[i] = \uniformVC^{m}_{d}(k)[i] \\
            &\phantom{\qquad \uniformVC^{m}_{d}(k + 1)[i]}
              \le \knownVC^{n}_{j}(k)[i] \\
            &\phantom{\qquad \uniformVC^{m}_{d}(k + 1)[i]}
              \le \knownVC^{n}_{j}(k + 1)[i] \big).
      \end{align*}
      Otherwise, we perform a case analysis
      according to how $\uniformVC^{m}_{d}[i]$ is updated.
      \begin{itemize}
        \item $\textsc{Case I}$: $\uniformVC^{m}_{d}[i]$ is updated
          at line~\code{\ref{alg:unistore-clock}}{\ref{line:stablevc-uniformvc}}.
          By line~\code{\ref{alg:unistore-clock}}{\ref{line:stablevc-g}},
          \begin{align}
            &\exists g' \subseteq \D.\; |g'| \ge f + 1 \land d \in g'\; \land
              \label{eqn:g-prime} \\
              &\quad \uniformVC^{m}_{d}(k+1)[i] = \nonumber \\
                &\qquad \max\big\{\uniformVC^{m}_{d}(k)[i], \nonumber \\
                  &\qquad\qquad\;\; \min_{j \in g'} \stablematrix^{m}_{d}(k+1)[j][i]\big\}. \nonumber
          \end{align}
          By the induction hypothesis and Lemma~\ref{lemma:knownvc-i-nondecreasing},
          \begin{align}
            &\exists g'' \subseteq \D.\; |g''| \ge f + 1 \land d \in g''\; \land
              \label{eqn:g-prime-prime} \\
              &\quad \big(\forall j \in g''.\; \forall n \in \P. \nonumber \\
                &\qquad \uniformVC^{m}_{d}(k)[i] \le \knownVC^{n}_{j}(k)[i] \nonumber \\
                &\phantom{\qquad \uniformVC^{m}_{d}(k)[i] }
                  \le \knownVC^{n}_{j}(k+1)[i]\big). \nonumber
          \end{align}
          By Lemma~\ref{lemma:stablevc-nondecreasing},
          for the particular $g' \subseteq \D$ in (\ref{eqn:g-prime}),
          \begin{align}
            \forall j \in g'.\; &\stablematrix^{m}_{d}(k+1)[j][i]
              \label{eqn:stablematrix-j-i} \\
              &\le \stableVC^{m}_{j}(k+1)[i].
              \nonumber
          \end{align}
          By Lemmas~\ref{lemma:stablevc-knownvc} and \ref{lemma:knownvc-i-nondecreasing},
          for any replica $p^{m}_{j}$ in data center $j$,
          \begin{align}
            &\forall n \in \P.\; \stableVC^{m}_{j}(k+1)[i]
              \label{eqn:stablevc-j-i}\\
              &\qquad\;\;\; \le \knownVC^{n}_{j}(k+1)[i]. \nonumber
          \end{align}
          Therefore, for the particular $g' \subseteq \D$ in $(\ref{eqn:g-prime})$,
          \begin{align}
            &\forall j' \in g'.\; \forall n \in \P.\;
              \min_{j \in g'} \stablematrix^{m}_{d}(k+1)[j][i]
              \nonumber\\
              &\qquad\qquad\qquad\;\; \le \knownVC^{n}_{j'}(k+1)[i]. \label{eqn:j-prime}
          \end{align}
          By (\ref{eqn:g-prime}), (\ref{eqn:g-prime-prime}),
          and (\ref{eqn:j-prime}),
          we can either take $g = g'$ in (\ref{eqn:g-prime})
          or $g = g''$ in (\ref{eqn:g-prime-prime}) such that
          \begin{align*}
            &\forall j \in g.\; \forall n \in \P. \\
              &\;\; \uniformVC^{m}_{d}(k+1)[i] \le \knownVC^{n}_{j}(k+1)[i].
          \end{align*}
          Therefore,
          \begin{align*}
            &\exists g \subseteq \D.\; |g| \ge f + 1 \land d \in g\; \land \\
              &\; \big(\forall j \in g.\; \forall n \in \P.\; \\
                &\quad \uniformVC^{m}_{d}(k+1)[i] \le \knownVC^{n}_{j}(k+1)[i] \big).
          \end{align*}
        \item $\textsc{Case II}$: $\uniformVC^{m}_{d}[i]$
          ($i \in \D \setminus \set{d}$) is updated
          at line~\code{\ref{alg:unistore-coord}}{\ref{line:start-uniformvc}}.
          Then there exists some client $\cl$ with $d = \cldc(\cl)$ such that
          \begin{align}
            &\uniformVC^{m}_{d}(k+1)[i] = \label{eqn:uniformvc-pastvc} \\
              &\quad \max\big\{\pastVC_{\cl}(k)[i], \uniformVC^{m}_{d}(k)[i]\big\}.
              \nonumber
          \end{align}
          By the induction hypothesis and Lemma~\ref{lemma:knownvc-i-nondecreasing},
          \begin{align}
            &\exists g' \subseteq \D.\; |g'| \ge f + 1 \land d \in g'\; \land
              \label{eqn:g-prime-2} \\
              &\quad \big(\forall j \in g'.\; \forall n \in \P. \nonumber \\
                &\qquad \uniformVC^{m}_{d}(k)[i] \le \knownVC^{n}_{j}(k)[i] \nonumber \\
                &\phantom{\qquad \uniformVC^{m}_{d}(k)[i] }
                  \le \knownVC^{n}_{j}(k+1)[i]\big). \nonumber
          \end{align}
          By Lemma~\ref{lemma:pastvc-uniformvc}, the induction hypothesis,
          and Lemma~\ref{lemma:knownvc-nondecreasing},
          \begin{align}
            &\exists g'' \subseteq \D.\; |g''| \ge f + 1 \land d \in g''\; \land
              \label{eqn:g-prime-prime-2} \\
              &\quad \big(\forall j \in g''.\; \forall n \in \P. \nonumber \\
                &\qquad \pastVC_{\cl}(k)[i] \le \knownVC^{n}_{j}(k+1)[i]. \nonumber
          \end{align}
          By (\ref{eqn:uniformvc-pastvc}), (\ref{eqn:g-prime-2}),
          and (\ref{eqn:g-prime-prime-2}),
          we can take $g = g'$ in (\ref{eqn:g-prime-2})
          or $g = g''$ in (\ref{eqn:g-prime-prime-2}) such that
          \begin{align*}
            &\forall j \in g.\; \forall n \in \P.\; \\
              &\;\; \uniformVC^{m}_{d}(k+1)[i] \le \knownVC^{n}_{j}(k+1)[i].
          \end{align*}
          Therefore,
          \begin{align*}
            &\exists g \subseteq \D.\; |g| \ge f + 1 \land d \in g\; \land \\
              &\; \big(\forall j \in g.\; \forall n \in \P.\; \\
                &\;\; \uniformVC^{m}_{d}(k+1)[i] \le \knownVC^{n}_{j}(k+1)[i] \big).
          \end{align*}
        \item $\textsc{Case III}$: $\uniformVC^{m}_{d}[i]$
          ($i \in \D \setminus \set{d}$) is updated
          at lines~\code{\ref{alg:unistore-replica}}{\ref{line:readkey-uniformvc}}
          or \code{\ref{alg:unistore-replica}}{\ref{line:preparecausal-uniformvc}}.
          Therefore, there exists some transaction $\tvar$
          originating at data center $d$ such that
          \begin{align}
            &\uniformVC^{m}_{d}(k+1)[i] = \label{eqn:uniformvc-snapshotvc} \\
              &\quad \max\big\{\snapshotVC(\tvar)[i], \uniformVC^{m}_{d}(k)[i]\big\}.
              \nonumber
          \end{align}
          By the induction hypothesis and Lemma~\ref{lemma:knownvc-nondecreasing},
          \begin{align}
            &\exists g' \subseteq \D.\; |g'| \ge f + 1 \land d \in g'\; \land
              \label{eqn:g-prime-3} \\
              &\quad \big(\forall j \in g'.\; \forall n \in \P. \nonumber \\
                &\qquad \uniformVC^{m}_{d}(k)[i] \le \knownVC^{n}_{j}(k)[i] \nonumber \\
                &\phantom{\qquad \uniformVC^{m}_{d}(k)[i] }
                  \le \knownVC^{n}_{j}(k+1)[i]\big). \nonumber
          \end{align}
          By Lemma~\ref{lemma:snapshotvc-uniformvc}, the induction hypothesis,
          and Lemma~\ref{lemma:knownvc-nondecreasing},
          \begin{align}
            &\exists g'' \subseteq \D.\; |g''| \ge f + 1 \land d \in g''\; \land
              \label{eqn:g-prime-prime-3} \\
              &\quad \big(\forall j \in g''.\; \forall n \in \P. \nonumber \\
                &\qquad \snapshotVC(\tvar)[i] \le \knownVC^{n}_{j}(k+1)[i]. \nonumber
          \end{align}
          By (\ref{eqn:uniformvc-snapshotvc}), (\ref{eqn:g-prime-3}),
          and (\ref{eqn:g-prime-prime-3}),
          we can take $g = g'$ in (\ref{eqn:g-prime-3})
          or $g = g''$ in (\ref{eqn:g-prime-prime-3}) such that
          \begin{align*}
            &\forall j \in g.\; \forall n \in \P.\; \\
              &\;\; \uniformVC^{m}_{d}(k+1)[i] \le \knownVC^{n}_{j}(k+1)[i].
          \end{align*}
          Therefore,
          \begin{align*}
            &\exists g \subseteq \D.\; |g| \ge f + 1 \land d \in g\; \land \\
              &\; \big(\forall j \in g.\; \forall n \in \P.\; \\
                &\quad \uniformVC^{m}_{d}(k+1)[i] \le \knownVC^{n}_{j}(k+1)[i] \big).
          \end{align*}
      \end{itemize}
  \end{itemize}
\end{proof}

\begin{applemma} \label{lemma:uniformvc-knownvc}
  For any replica $p^{m}_{d}$ in data center $d$,
  \[
    \forall i \in \D.\; \forall n \in \P.\;
      \uniformVC^{m}_{d}[i] \le \knownVC^{n}_{d}[i].
  \]
\end{applemma}

\begin{proof} \label{proof:uniformvc-knownvc}
  By Lemma~\ref{lemma:uniformvc-knownvc-f+1}.
\end{proof}

\begin{applemma} \label{lemma:replication-uniformvc}
  Let $\tvar \in \causaltxs$ be a causal transaction
  that originates at data center $i$
  and accesses partition $n$.
  If
  \[
    \commitVC(\tvar)[i] \le \uniformVC^{m}_{d}[i]
  \]
  for some replica $p^{m}_{d}$ in data center $d$,
  then
  \[
    \log(\tvar)[n] \subseteq \oplog^{n}_{d}.
  \]
\end{applemma}

\begin{proof} \label{proof:replication-uniformvc}
  By Lemma~\ref{lemma:uniformvc-knownvc},
  \[
    \uniformVC^{m}_{d}[i] \le \knownVC^{n}_{d}[i].
  \]
  Therefore,
  \[
    \commitVC(\tvar)[i] \le \knownVC^{n}_{d}[i].
  \]
  By Lemma~\ref{lemma:replication-knownvc},
  \[
    \log(\tvar)[n] \subseteq \oplog^{n}_{d}.
  \]
\end{proof}

\begin{applemma} \label{lemma:uniformvc-clock}
  For any replica $p^{m}_{d}$ in data center $d$,
  \[
    \uniformVC^{m}_{d}[d] \le \clockVar^{m}_{d}.
  \]
\end{applemma}

\begin{proof} \label{proof:uniformvc-clock}
  By Lemma~\ref{lemma:uniformvc-knownvc},
  \[
    \uniformVC^{m}_{d}[d] \le \knownVC^{m}_{d}[d].
  \]
  By Lemma~\ref{lemma:knownvc-d-clock},
  \[
    \knownVC^{m}_{d}[d] \le \clockVar^{m}_{d}.
  \]
  Putting it together yields
  \[
    \uniformVC^{m}_{d}[d] \le \clockVar^{m}_{d}.
  \]
\end{proof}
\subsubsection{Properties of $\pastVC$}
\label{sss:cvc}

\begin{applemma} \label{lemma:start-pastvc-snapshotvc}
  Let $e \in S$ be a \start{} event of transaction $\tvar$
  issued by client $\cl$. Then
  \[
    (\pastVC_{\cl})_{e} \le \snapshotVC(\tvar).
  \]
\end{applemma}

\begin{proof} \label{proof:start-pastvc-snapshotvc}
  By Definition~\ref{def:snapshotvc} of $\snapshotVC(\tvar)$
  and lines~\code{\ref{alg:unistore-coord}}{\ref{line:start-uniformvc-index}}--
  \code{\ref{alg:unistore-coord}}{\ref{line:start-snapvc-strong}}.
\end{proof}

\begin{applemma} \label{lemma:pastvc-nondecreasing}
  For $i \in \D$, $\pastVC_{\cl}[i]$ at any client $\cl$
  is non-decreasing.
\end{applemma}

\begin{proof} \label{proof:pastvc-nondecreasing}
  Note that $\pastVC_{\cl}[i]$ ($i \in \D$) is updated only
  at lines~\code{\ref{alg:unistore-client}}{\ref{line:commitcausaltx-pastvc}}
  or \code{\ref{alg:unistore-client}}{\ref{line:commitstrongtx-pastvc}}
  when some transaction is committed.
  Therefore, the lemma holds due to Lemmas~\ref{lemma:snapshotvc-commitvc}
  and \ref{lemma:start-pastvc-snapshotvc}.
\end{proof}
