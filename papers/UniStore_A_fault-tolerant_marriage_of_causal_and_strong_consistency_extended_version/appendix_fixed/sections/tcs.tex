
\section{Transaction Certification Service Specification} \label{section:tcs}

\subsection{Interface} \label{ss:interface}

The \emph{Transaction Certification Service} (TCS)~\cite{discpaper}
is responsible for certifying strong transactions issued by transaction coordinators,
computing commit vectors and Lamport clocks for committed transactions,
and (asynchronously) delivering committed transactions to replicas.

Each strong transaction $\tvar \in \allstrongtxs$ submitted to TCS
may be associated with its read set $\rs(\tvar)$, write set $\ws(\tvar)$,
snapshot vector $\snapshotVC(\tvar)$ (Definition~\ref{def:snapshotvc}),
commit vector $\commitVC(\tvar)$ (Definition~\ref{def:commitvc}),
and Lamport clock $\lclock(\tvar)$ (Definition~\ref{def:lc-tx}).
From $\rs(\tvar)$ and $\ws(\tvar)$ we can then define 
$\W(t)$ and $\R(t)$ according to~(\ref{eq:W-def}) and~(\ref{eq:R-def}), respectively.
Note that we have $\W(t) \subseteq \R(t)$.

Transaction coordinators for strong transactions interact with TCS
using two types of \emph{actions}. Coordinators can make certification requests
(corresponding to procedure \certify{} of Algorithm~\ref{alg:unistore-certify})
of the form
\[
  \intcertify(\tidselector(\tvar), \ws(\tvar), \R(\tvar),
    \snapshotVC(\tvar), \lccertify(\tvar)),
\]
where $\tvar \in \allstrongtxs$ and $\lccertify(\tvar) \in \N$
denotes the contribution of $\client(\tvar)$ to the Lamport clock of $\tvar$.
The TCS responses are of the form
\[
  \intdecide(\tidselector(\tvar), \decvar, \vcvar, \lcvar),
\]
containing a decision $\decvar$ from $\Decision = \set{\commit, \abort}$ for $\tvar$,
a commit vector $\vcvar$ from $\Vector$ for $\tvar$ if $\decvar = \commit$,
and a Lamport clock $\lcvar$ from $\N$ for $\tvar$ if $\decvar = \commit$.
If $\decvar = \abort$, then $\vcvar$ and $\lcvar$ are irrelevant.

Besides, TCS can deliver some payload $\payload$
to a replica via \upcall \emph{actions} $\intdeliver(\payload)$
(corresponding to procedure \deliverupdates{}
of Algorithm~\ref{alg:unistore-strong-commit}).
We denote by $\intdeliver^{m}_{d}(\payload)$ the delivery of the payload $\payload$ to
a replica $p^{m}_{d}$, when the latter is relevant.
The payload $\payload$ in $\intdeliver^{m}_{d}(\payload)$ is a set of tuples of the form
$\langle \tidvar, \wbuffvar, \commitvc, \lcvar \rangle$,
each of which corresponds to the updates $\wbuffvar \subseteq \Key \times \Val$
performed at a particular partition $m$
by a particular committed strong transaction
with transaction identifier $\tidvar$, commit vector $\commitvc$,
and Lamport clock $\lcvar$.
\subsection{Certification Functions} \label{ss:cert-func}

TCS is specified using a certification function
\begin{align}
  \certfunc: 2^{\strongtxs} \times \allstrongtxs \to
    \Decision \times \Vector \times \N.
  \label{eqn:f}
\end{align}
For a strong transaction $\tvar \in \allstrongtxs$
and the set $\txsincertify \subseteq \strongtxs$ of previously committed strong transactions,
$\certfunc(\txsincertify, \tvar)$ returns not only the decision $\decvar \in \Decision$,
but also the commit vector $\vcvar \in \Vector$
and Lamport clock $\lcvar \in \N$ for $\tvar$.
We use $\fdec(\txsincertify, \tvar)$, $\fvec(\txsincertify, \tvar)$,
and $\flc(\txsincertify, \tvar)$ to select
the first, second, and third component
of $\certfunc(\txsincertify, \tvar)$, respectively.

The decision $\fdec(\txsincertify, \tvar)$ should satisfy
\begin{align}
  &\fdec(\txsincertify, \tvar) = \commit
    \iff \forall k \in \R(\tvar).\; \forall \tvar' \in \txsincertify.
    \nonumber\\
    &\quad \big(k \in \W(\tvar') \implies
      \commitVC(\tvar') \le \snapshotVC(\tvar)\big).
    \label{eqn:gcf-decision}
\end{align}

The commit vector $\fvec(\txsincertify, \tvar)$ should satisfy
\begin{align}
  &\phantom{\land\;} (\forall i \in \D.\; \fvec(\txsincertify, \tvar)[i] = \snapshotVC(\tvar)[i]) \nonumber\\
  &\land \fvec(\txsincertify, \tvar)[\strongentry] > \snapshotVC(\tvar)[\strongentry] \nonumber\\
  &\land \forall \tvar' \in \txsincertify.\; \tvar \conflict \tvar'
    \implies \fvec(\txsincertify, \tvar) \ge \commitVC(\tvar').
    \label{eqn:gcf-commitvc}
\end{align}

The Lamport clock $\flc(\txsincertify, \tvar)$ should satisfy
\begin{align}
  &\flc(\txsincertify, \tvar) \ge \lccertify(t) \;\land \nonumber\\
  &\quad \big(\forall \tvar' \in \txsincertify.\; \tvar \conflict \tvar' \implies
    \flc(\txsincertify, \tvar) > \lclock(\tvar')\big).
    \label{eqn:gcf-lc}
\end{align}
\subsection{Histories of TCS} \label{ss:histories}

TCS executions are represented by \emph{histories},
which are (possibly infinite) sequences
of $\intcertify$, $\intdecide$, and $\intdeliver$ actions.
For a TCS history $h$, we use $\act(h)$ to denote the set of actions in $h$.
For actions $\actvar, \actvar' \in \act(h)$,
we write $\actvar \prec_{h} \actvar'$
when $\actvar$ occurs before $\actvar'$ in $h$.
A strong transaction $\tvar \in \allstrongtxs$ \emph{commits} in a history $h$
if $h$ contains $\intdecide(\tidselector(\tvar), \commit, \_, \_)$.
We denote by $\committedVar(h)$ the projection of $h$ to actions
corresponding to the strong transactions that are committed in $h$.

Each history $h$ needs to meet the following requirements.
\begin{enumerate}[(R1)]
  \item \label{tcs-requirement:certify-once}
    For each strong transaction $\tvar \in \allstrongtxs$,
    there is at most one $\intcertify(\tidselector(\tvar), \_, \_, \_, \_)$
    action in $h$.
  \item \label{tcs-requirement:decide-once}
    For each action $\intdecide(\tidvar, \_, \_, \_) \in \act(h)$,
    there is exactly one $\intcertify(\tidvar, \_, \_, \_, \_)$ action in $h$
    such that
    \[
      \intcertify(\tidvar, \_, \_, \_, \_)
        \prec_{h} \intdecide(\tidvar, \_, \_, \_).
    \]
  \item \label{tcs-requirement:abort-cannot-deliver}
    For each action $\intdeliver(\payload) \in \act(h)$
    and each $\langle \tidvar, \_, \_, \_ \rangle \in \payload$,
    there is \emph{no} $\intdecide(\tidvar, \abort, \_, \_)$ action in $h$.
  \item \label{tcs-requirement:deliver-once}
    Every committed strong transaction is delivered at most once to each replica.
  \item \label{tcs-requirement:certify-before-deliver}
    For each action $\intdeliver(\payload) \in \act(h)$
    and each $\langle \tidvar, \_, \_, \_ \rangle \in \payload$,
    there is a $\intcertify(\tidvar, \_, \_, \_, \_)$ action such that
    \[
      \intcertify(\tidvar, \_, \_, \_, \_) \prec_{h} \intdeliver(\payload).
    \]
  \item \label{tcs-requirement:deliver-order}
    At each replica $p^{m}_{d}$, committed strong transactions are delivered
    in the increasing order of their strong timestamps.
    Formally, for any two distinct actions
    $\intdeliver^{m}_{d}(\payload_{1})$ and $\intdeliver^{m}_{d}(\payload_{2})$
    with payloads $\payload_{1}$ and $\payload_{2}$, respectively,
    \begin{align*}
      &\intdeliver^{m}_{d}(\payload_{1}) \prec_{h} \intdeliver^{m}_{d}(\payload_{2}) \implies \\
        &\quad \forall \langle \_, \_, \commitvc_{1}, \_ \rangle \in \payload_{1}.\;\\
        &\quad \forall \langle \_, \_, \commitvc_{2}, \_ \rangle \in \payload_{2}.\; \\
          &\qquad \commitvc_{1}[\strongentry] < \commitvc_{2}[\strongentry].
    \end{align*}
\end{enumerate}

A history is \emph{complete} if every $\intcertify$ action
in it has a matching $\intdecide$ action.
A complete history $h$ is \emph{sequential} if
it consists of consecutive pairs of $\intcertify$ and matching $\intdecide$ actions.
For a complete history $h$, a \emph{permutation} $h'$ of $h$
is a sequential history such that
\begin{itemize}
  \item $h$ and $h'$ contain the same actions, i.e., $\act(h) = \act(h')$.
  \item Transactions are certified in $h'$ according to their session order.
    \begin{align*}
      &\forall \tvar, \tvar' \in \allstrongtxs.\;
        \tvar \rel{\so} \tvar' \implies \\
          &\quad \intdecide(\tidselector(\tvar), \_, \_, \_) \prec_{h'}
            \intcertify(\tidselector(\tvar'), \_, \_, \_, \_).
    \end{align*}
\end{itemize}
\subsection{TCS Correctness: Safety and Liveness} \label{ss:tcs-correctness}

\subsubsection{Safety of TCS} \label{ss:tcs-safety}

A complete sequential history $h$ is \emph{legal} with respect to
a certification function $\certfunc$,
if its results are computed so as to satisfy
(\ref{eqn:gcf-decision}) -- (\ref{eqn:gcf-lc}) according to $\certfunc$:
\begin{align*}
  &\forall \actvar = \intdecide(\tidselector(\tvar), \decvar, \vcvar,
    \lcvar) \in \act(h). \\
    &\quad (\decvar, \vcvar, \lcvar)
      = \certfunc(\set{\tvar' \mid \\
        &\quad\quad \intdecide(\tidselector(\tvar'), \commit, \_, \_) \prec_{h} \actvar}, \tvar).
\end{align*}
A history $h$ is \emph{correct} with respect to $\certfunc$
if $h \mid \committedVar(h)$ has a legal permutation.
A TCS implementation is \emph{correct} with respect to $\certfunc$
if so are all its histories.
\subsubsection{Liveness of TCS} \label{ss:tcs-liveness}

TCS guarantees that every committed strong transaction
will eventually be delivered by every correct data center.
Formally,
\begin{align}
  &\forall \actvar = \intdecide(\tidvar, \commit, \_, \_) \in \act(h).
    \nonumber\\
    &\quad \forall m \in \partitionsfunc(\tidvar).\; \forall \cdrange \in \C.
    \nonumber\\
    &\qquad \exists \actvar' = \intdeliver^{m}_{\cdrange}(\payload) \in \act(h).\;
    \nonumber\\
      &\quad\qquad \langle \tidvar, \_, \_, \_ \rangle \in \payload
        \land \actvar \prec_{h} \actvar'.
    \label{eqn:tcs-liveness-delivery}
\end{align}
Here $\partitionsfunc(\tidvar)$ denotes the set of partitions
that a particular transaction with transaction identifier $\tidvar$ accesses.

A TCS implementation meets the liveness requirement if
every history produced by its maximal execution satisfies (\ref{eqn:tcs-liveness-delivery}).
\subsection{TCS Correctness} \label{ss:tcs-correctness-unistore}

The proof of TCS correctness is an adjustment
of the ones in~\cite{discpaper, multicast-dsn19}.

\begin{apptheorem} \label{thm:tcs-correctness}
  The TCS implementation in \unistore{}
  (Algorithms~\ref{alg:unistore-certify} -- \ref{alg:unistore-recovery})
  is correct with respect to the certification function $\certfunc$ in (\ref{eqn:f})
  and meets the liveness requirement in (\ref{eqn:tcs-liveness-delivery}).
\end{apptheorem}
