
\subsection{Visibility Relation}  \label{ss:vis}

\begin{appdefinition}[Visibility Relation] \label{def:vis-tx}
  \begin{align*}
    &\forall \txevents_{1}, \txevents_{2} \in X.\;
      \txevents_{1} \rel{\vis} \txevents_{2} \iff \\
      &\quad \big((\txevents_{2} \in \txs \implies
        \timestamp(\txevents_{1}) \le \timestamp(\startoftx(\txevents_{2}))) \\
        &\quad\quad \land (\txevents_{2} \in \Fence \cup \Attach \implies \timestamp(\txevents_{1}) \le \timestamp(\txevents_{2}))\big)
          \land \txevents_{1} \rel{\lcorder} \txevents_{2}.
  \end{align*}
\end{appdefinition}

\begin{apptheorem} \label{thm:conflictaxiom}
  \[
    A \models \conflictaxiom.
  \]
\end{apptheorem}

\begin{proof} \label{proof:conflictaxiom}
  We need to show that
  \[
    \forall \tvar_{1}, \tvar_{2} \in \strongtxs.\;
      \tvar_{1} \conflict \tvar_{2} \implies \tvar_{1} \rel{\vis} \tvar_{2} \lor \tvar_{2} \rel{\vis} \tvar_{1}.
  \]
  Consider the history $h$ of TCS.
  By Theorem~\ref{thm:tcs-correctness},
  $h \mid \committedVar(h)$ has a legal permutation $\pi$.
  Suppose that
  \[
    \intcertify(\tidselector(\tvar_{1}), \_, \_, \_, \_) \prec_{\pi}
      \intcertify(\tidselector(\tvar_{2}), \_, \_, \_, \_).
  \]
  Since $\tvar_{1} \conflict \tvar_{2}$ and $\tvar_{2}$ is committed,
  by (\ref{eqn:gcf-decision}),
  \[
    \commitVC(\tvar_{1}) \le \snapshotVC(\tvar_{2}).
  \]
  By Lemmas~\ref{lemma:ts-extread} and \ref{lemma:ts-commit},
  \[
    \timestamp(\tvar_{1}) \le \timestamp(\startoftx(\tvar_{2})).
  \]
  On the other hand, by (\ref{eqn:gcf-lc}),
  \[
    \lclock(\tvar_{1}) < \lclock(\tvar_{2}).
  \]
  By Definition~\ref{def:lco} of $\lcorder$,
  \[
    \tvar_{1} \rel{\lcorder} \tvar_{2}.
  \]
  Therefore, by Definition~\ref{def:vis-tx} of $\vis$,
  \[
    \tvar_{1} \rel{\vis} \tvar_{2}.
  \]
\end{proof}

\begin{applemma} \label{lemma:so-vis}
  \[
    \so \subseteq \vis.
  \]
\end{applemma}

\begin{proof} \label{proof:so-vis}
  By Lemmas~\ref{lemma:so-ts} and \ref{lemma:so-lc}.
\end{proof}

\begin{applemma} \label{lemma:vis-partial}
  The visibility relation $\vis$ is a partial order.
\end{applemma}

\begin{proof} \label{proof:vis-partial}
  We need to show that
  \begin{itemize}
    \item $\vis$ is irreflexive.
      This holds because $\lcorder$ is irreflexive.
    \item $\vis$ is transitive.
      Suppose that $\txevents_{1} \rel{\vis} \txevents_{2} \rel{\vis} \txevents_{3}$.
      By Definition~\ref{def:vis-tx} of $\vis$,
      \[
        \txevents_{1} \rel{\lcorder} \txevents_{2} \rel{\lcorder} \txevents_{3}.
      \]
      By Definition~\ref{def:lco} of $\lcorder$,
      \[
        \txevents_{1} \rel{\lcorder} \txevents_{3}.
      \]
      Regarding timestamps,
      we distinguish between the following four cases
      and use Lemma~\ref{lemma:ts-tid-st-tid}.
      \begin{itemize}
        \item $\txevents_{2} \in \Fence \cup \Attach
          \land \txevents_{3} \in \Fence \cup \Attach$.
          \[
            \timestamp(\txevents_{1}) \le \timestamp(\txevents_{2})
              \le \timestamp(\txevents_{3}).
          \]
        \item $\txevents_{2} \in \Fence \cup \Attach
          \land \txevents_{3} \in \txs$.
          \[
            \timestamp(\txevents_{1}) \le \timestamp(\txevents_{2})
              \le \timestamp(\startoftx(\txevents_{3})).
          \]
        \item $\txevents_{2} \in \txs \land \txevents_{3} \in \Fence \cup \Attach$.
          \[
            \timestamp(\txevents_{1}) \le \timestamp(\startoftx(\txevents_{2}))
              \le \timestamp(\txevents_{2}) \le \timestamp(\txevents_{3}).
          \]
        \item $\txevents_{2} \in \txs \land \txevents_{3} \in \txs$.
          \[
            \timestamp(\txevents_{1}) \le \timestamp(\startoftx(\txevents_{2}))
              \le \timestamp(\txevents_{2}) \le \timestamp(\startoftx(\txevents_{3})).
          \]
      \end{itemize}
      By Definition~\ref{def:vis-tx} of $\vis$,
      \[
        \txevents_{1} \rel{\vis} \txevents_{3}.
      \]
  \end{itemize}
\end{proof}

\begin{apptheorem} \label{thm:cv}
  \[
    A \models \cv.
  \]
\end{apptheorem}

\begin{proof} \label{proof:cv}
  By Lemmas~\ref{lemma:so-vis} and \ref{lemma:vis-partial},
  \[
    (\so \cup \vis)^{+} = \vis^{+} = \vis.
  \]
\end{proof}

\begin{applemma}[\prop{5}] \label{lemma:conflict-strong-ts}
  For any two conflicting transactions $\tvar_{1}$ and $\tvar_{2}$,
  \begin{align*}
    &\tvar_{1} \rel{\vis} \tvar_{2} \iff \\
      &\quad \commitVC(\tvar_{1})[\strongentry] < \commitVC(\tvar_{2})[\strongentry].
  \end{align*}
\end{applemma}

\begin{proof} \label{proof:conflict-strong-ts}
  We first show that
  \begin{align}
    &\tvar_{1} \rel{\vis} \tvar_{2} \implies \label{eq:vis-strong-ts} \\
      &\quad \commitVC(\tvar_{1})[\strongentry] < \commitVC(\tvar_{2})[\strongentry].
      \nonumber
  \end{align}
  Assume that $\tvar_{1} \rel{\vis} \tvar_{2}$.
  By Definition~\ref{def:vis-tx} of $\vis$,
  \[
    \tsfunc(\tvar_{1}) \le \tsfunc(\startoftx(\tvar_{2})).
  \]
  By Lemmas~\ref{lemma:ts-extread} and \ref{lemma:ts-commit},
  \[
    \commitVC(\tvar_{1}) \le \snapshotVC(\tvar_{2}).
  \]
  Therefore,
  \[
    \commitVC(\tvar_{1})[\strongentry] \le \snapshotVC(\tvar_{2})[\strongentry].
  \]
  By (\ref{eqn:gcf-commitvc}),
  \[
    \commitVC(\tvar_{2})[\strongentry] > \snapshotVC(\tvar_{2})[\strongentry].
  \]
  Putting it together yields
  \[
    \commitVC(\tvar_{1})[\strongentry] < \commitVC(\tvar_{2})[\strongentry].
  \]
  Next we show that
  \begin{align*}
    &\tvar_{1} \rel{\vis} \tvar_{2} \impliedby \\
      &\quad \commitVC(\tvar_{1})[\strongentry] < \commitVC(\tvar_{2})[\strongentry].
  \end{align*}
  Assume that
  \begin{align}
    \commitVC(\tvar_{1})[\strongentry] < \commitVC(\tvar_{2})[\strongentry].
    \label{eq:tid1-less-tid2-strong-ts}
  \end{align}
  Since $\tvar_{1} \conflict \tvar_{2}$, by Theorem~\ref{thm:conflictaxiom},
  \[
    \tvar_{1} \rel{\vis} \tvar_{2} \lor \tvar_{2} \rel{\vis} \tvar_{1}.
  \]
  By (\ref{eq:vis-strong-ts}) and (\ref{eq:tid1-less-tid2-strong-ts}),
  \[
    \lnot(\tvar_{2} \rel{\vis} \tvar_{1}).
  \]
  Therefore,
  \[
    \tvar_{1} \rel{\vis} \tvar_{2}.
  \]
\end{proof}
