
\subsection{Timestamps} \label{ss:ts}

\begin{appdefinition}[Timestamps of Events] \label{def:ts-op}
  Let $e \in C_{\causalentry} \cup C_{\strongentry} \cup \Fence \cup \Attach$
  be an event issued by client $\cl$.
  We define its timestamp $\timestamp(e)$ as
  \[
    \tsfunc(e) \triangleq (\pastVC_{\cl})_{e}.
  \]
  Let $e \in S$ be a \start{} event of transaction $\tvar$.
  Let $d \triangleq \dc(\tvar)$ and $m \triangleq \coord(\tvar)$.
  We define its timestamp $\timestamp(e)$ as
  \[
    \tsfunc(e) \triangleq (\snapVC^{m}_{d})_{e}[\tvar].
  \]
  See lines~\code{\ref{alg:unistore-client}}{\ref{line:starttx-ts}},
  \code{\ref{alg:unistore-client}}{\ref{line:commitcausaltx-ts}},
  \code{\ref{alg:unistore-client}}{\ref{line:commitstrongtx-ts}},
  \code{\ref{alg:unistore-client}}{\ref{line:fence-ts}},
  and \code{\ref{alg:unistore-client}}{\ref{line:clattach-ts}}
  for \start, \commitcausaltx, \commitstrongtx, \fence,
  and \clattach{} events, respectively.
\end{appdefinition}

\begin{appdefinition}[Timestamps of Transactions] \label{def:ts-tx}
  The timestamp $\timestamp(\tvar)$ of a transaction $\tvar$
  is that of its commit event, i.e.,
  \[
    \forall \tvar \in \txs.\;
      \timestamp(\tvar) \triangleq \timestamp(\commitoftx(\tvar)).
  \]
\end{appdefinition}

\begin{applemma} \label{lemma:ts-start}
  Let $e \in S$ be a \start{} event.
  Let $d \triangleq \dc(\txfunc(e))$ and $m \triangleq \coord(\txfunc(e))$.
  Then
  \begin{align*}
    (\forall i \in \D.\; \timestamp(e)[i] \ge (\uniformVC^{m}_{d})_{e}[i])
    \;\land \\
    \timestamp(e)[\strongentry] \ge (\stableVC^{m}_{d})_{e}[\strongentry].
  \end{align*}
\end{applemma}

\begin{proof} \label{proof:ts-start}
  By lines~\code{\ref{alg:unistore-coord}}{\ref{line:start-snapvc}},
  \code{\ref{alg:unistore-coord}}{\ref{line:start-snapvc-d}},
  and \code{\ref{alg:unistore-coord}}{\ref{line:start-snapvc-strong}}.
\end{proof}

\begin{applemma} \label{lemma:ts-extread}
  Let $e \in \extread$ be an external read event.
  Let $d \triangleq \dc(\txfunc(e))$ and $m \triangleq \coord(\txfunc(e))$.
  Then
  \begin{align*}
    \timestamp(\startoftx(e)) &= \snapshotVC(\txfunc(e)) \\
                  &= \snapvc_{(\readkey, e)} \\
                  &= (\snapVC^{m}_{d})_{\startoftx(e)}[\txfunc(e)].
  \end{align*}
\end{applemma}

\begin{proof} \label{proof:ts-extread}
  By Definition~\ref{def:ts-op} of timestamps,
  line~\code{\ref{alg:unistore-coord}}{\ref{line:doread-from-snapshot}},
  and line~\code{\ref{alg:unistore-coord}}{\ref{line:start-snapvc-d}}.
\end{proof}

\begin{applemma} \label{lemma:rf-ts}
  Let $e \in \extread$ be an external read event
  which reads from transaction $\tvar$. Then
  \[
    \timestamp(\tvar) \le \timestamp(\startoftx(e)).
  \]
\end{applemma}

\begin{proof} \label{proof:rf-ts}
  By line~\code{\ref{alg:unistore-replica}}{\ref{line:readkey-read}},
  \[
    \commitvc_{(\readkey, e)} \le \snapvc_{(\readkey, e)}.
  \]
  Since $e$ reads from $\tvar$,
  \[
    \timestamp(\tvar) = \commitvc_{(\readkey, e)}.
  \]
  By Lemma~\ref{lemma:ts-extread},
  \[
    \timestamp(\startoftx(e)) = \snapvc_{(\readkey, e)}.
  \]
  Therefore,
  \[
    \timestamp(\tvar) \le \timestamp(\startoftx(e)).
  \]
\end{proof}


\begin{applemma} \label{lemma:ts-commit}
  \begin{align*}
    &\forall e \in (\updatecommit \cap C_{\causalentry})
      \cup C_{\strongentry}.\; \\
      &\quad \timestamp(e) = \timestamp(\txfunc(e))
                           = \commitVC(\txfunc(e)).
  \end{align*}
\end{applemma}

\begin{proof} \label{proof:ts-commit}
  By Definition~\ref{def:ts-op} of timestamps
  and Definition~\ref{def:commitvc} of $\commitVC(\txfunc(e))$.
\end{proof}

\begin{applemma} \label{lemma:ts-tid-st-tid}
  \[
    \forall \tvar \in \txs.\; \timestamp(\tvar) \ge \timestamp(\startoftx(\tvar)).
  \]
\end{applemma}

\begin{proof} \label{proof:ts-tid-st-tid}
  By Lemmas~\ref{lemma:snapshotvc-commitvc}, \ref{lemma:ts-extread},
  and \ref{lemma:ts-commit}.
\end{proof}
