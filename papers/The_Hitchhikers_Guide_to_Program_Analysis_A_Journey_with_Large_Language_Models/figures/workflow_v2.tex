\begin{tikzpicture}[
  block/.style={rectangle, draw, text width=11em, text centered, rounded corners, minimum height=2em, fill=white, font=\footnotesize},
  line/.style={draw, -latex'},
  node distance=0.7cm and 0.5cm,
  ]

\definecolor{lightblue}{RGB}{173,217,230}
% \definecolor{lightblue}{RGB}{9,74,168}
\definecolor{lightgreen}{RGB}{144,237,144}
\definecolor{lightyellow}{RGB}{255,254,224}

% Step 1&2: Identify Initializer & Extract Postcondition
\node[block, fill=lightblue] (A1) {(1) Provide the \textit{use} and \textit{context} of the potential bug };
\node[block, fill=lightgreen, below=of A1] (B1) {(2) LLM identifies initializer};
\node[block, fill=lightgreen, below=of B1] (D1) {(3) \mbox{LLM extracts postcondition} with \textbf{in-context learning}};
\node[block, fill=lightyellow , below=of D1] (E1) {(4) LLM reviews and correct response via \textbf{self-validation}};
\node[block, fill=lightgreen, below=of E1] (F1) {(5) \mbox{LLM outputs a structured} response in JSON fomrat};

\path[line] (A1) -- (B1);
\path[line] (B1) -- (D1);
\path[line] (D1) -- (E1);
\path[line] (E1) -- (F1);

\node[draw, dashed, fit=(A1)(F1), inner sep=.8em] (cluster1) {};
\node[anchor=north west, inner sep=1pt] at (cluster1.north west) {\scriptsize Convo.1: Initializer \& Postcondition Extraction};


% Step 3: Summarize the Initializer Behavior
\node[block, fill=lightblue, right =4.5 em of A1] (A) {(1) Provide initializer, postcondition to LLM};
\node[block, fill=lightgreen, below=of A] (B) {(2) \mbox{LLM analyzes initializer} with \textbf{in-context learning}};

% Side nodes
\node[block, below left=0.6cm and -10em of B, fill=lightblue] (C) {\mbox{\textbf{progressively} provide} more information };
% \node[block, below right=1.5cm and -6em of B, fill=blue!10] (D) {LLM generates a precise summary};

\node[block, fill=lightyellow , below right= 0.76cm and -10em of C] (E) {(4) LLM reviews and correct the response via \textbf{self-validation}};
\node[block, fill=lightgreen, below=of E] (F) {(5) LLM outputs a structured summary in JSON format};

\path[line] (A) -- (B);
\begin{scope}[on background layer]
\path[line] (B) edge[bend right] node[midway, left=5pt, font=\tiny] {\textit{If asking for more info}} (C)
              edge[bend left] node[midway, yshift=21pt, right=3pt, font=\tiny] {\textit{If enough info}} (E);
\end{scope}

\path[line] (C) edge[bend right] node[midway, right=4pt]{} (B);
% \path[line] (D) -- (E);
\path[line] (E) -- (F);


% \path[line] (F1) edge[bend right] node[midway, right=4pt]{} (A);
% \path[line] (F1) edge[bend] -- ++(1,1) -- ++(1,-1) -- ++(1,1) -- (A);
% \path[line] (E) edge[bend right=80] node[midway, left=3pt, font=\tiny] {If asking (uncommon)} (C);
%                 edge[bend left] node[midway, right=3pt, font=\tiny] {Finish} (F);





\node[draw, dashed, fit=(A)(B)(C)(F), inner sep=.8em] (cluster2) {};
\node[anchor=north west, inner sep=1pt] at (cluster2.north west) {\scriptsize Convo.2: Initializer Behavior Summary};

% \node (C3) {C3};
% \node (D3) at (4,0) {D3};

% \draw (F1) to[out=0,in=180] coordinate[pos=0.33] (C3) 
%            to[out=0,in=180] coordinate[pos=0.67] (D3) (A);


% \path[line] (F1) -- ++(0.7cm, 0) -| (A);
% \path[line, dashed] (E.east) -- ++(0.7cm, 0) -| (C.east);

\draw [line] (F1) .. controls (4.5,-5) and (0,0.7) .. (A);
\pgfpathcurvebetweentime{0}{1}
{\pgfpointxy{1.8}{-5.4}}
{\pgfpointxy{4.5}{-5}}
  {\pgfpointxy{0}{0.7}}
  {\pgfpointxy{3.15}{0.27}}
\pgfsetstrokeopacity{0.5}
\pgfsetlinewidth{2pt}
\pgfusepath{stroke}


\end{tikzpicture}