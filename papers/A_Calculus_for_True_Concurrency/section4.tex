\section{Strongly Truly Concurrent Bisimulations}\label{stcb}

\subsection{Basic Definitions}\label{STCC}

Firstly, in this subsection, we introduce concepts of (strongly) truly concurrent behavioral bisimulation equivalences, including pomset bisimulation, step bisimulation, history-preserving (hp-)bisimulation and hereditary history-preserving (hhp-)bisimulation.

\begin{definition}[Pomset transitions and step]
Let $\mathcal{E}$ be a PES and let $C\in\mathcal{C}(\mathcal{E})$, and $\emptyset\neq X\subseteq \mathbb{E}$, if $C\cap X=\emptyset$ and $C'=C\cup X\in\mathcal{C}(\mathcal{E})$, then $C\xrightarrow{X} C'$ is called a pomset transition from $C$ to $C'$. When the events in $X$ are pairwise concurrent, we say that $C\xrightarrow{X}C'$ is a step.
\end{definition}

\begin{definition}[Strong pomset, step bisimulation]\label{PSB}
Let $\mathcal{E}_1$, $\mathcal{E}_2$ be PESs. A strong pomset bisimulation is a relation $R\subseteq\mathcal{C}(\mathcal{E}_1)\times\mathcal{C}(\mathcal{E}_2)$, such that if $(C_1,C_2)\in R$, and $C_1\xrightarrow{X_1}C_1'$ then $C_2\xrightarrow{X_2}C_2'$, with $X_1\subseteq \mathbb{E}_1$, $X_2\subseteq \mathbb{E}_2$, $X_1\sim X_2$ and $(C_1',C_2')\in R$, and vice-versa. We say that $\mathcal{E}_1$, $\mathcal{E}_2$ are strong pomset bisimilar, written $\mathcal{E}_1\sim_p\mathcal{E}_2$, if there exists a strong pomset bisimulation $R$, such that $(\emptyset,\emptyset)\in R$. By replacing pomset transitions with steps, we can get the definition of strong step bisimulation. When PESs $\mathcal{E}_1$ and $\mathcal{E}_2$ are strong step bisimilar, we write $\mathcal{E}_1\sim_s\mathcal{E}_2$.
\end{definition}

\begin{definition}[Posetal product]
Given two PESs $\mathcal{E}_1$, $\mathcal{E}_2$, the posetal product of their configurations, denoted $\mathcal{C}(\mathcal{E}_1)\overline{\times}\mathcal{C}(\mathcal{E}_2)$, is defined as

$$\{(C_1,f,C_2)|C_1\in\mathcal{C}(\mathcal{E}_1),C_2\in\mathcal{C}(\mathcal{E}_2),f:C_1\rightarrow C_2 \textrm{ isomorphism}\}.$$

A subset $R\subseteq\mathcal{C}(\mathcal{E}_1)\overline{\times}\mathcal{C}(\mathcal{E}_2)$ is called a posetal relation. We say that $R$ is downward closed when for any $(C_1,f,C_2),(C_1',f',C_2')\in \mathcal{C}(\mathcal{E}_1)\overline{\times}\mathcal{C}(\mathcal{E}_2)$, if $(C_1,f,C_2)\subseteq (C_1',f',C_2')$ pointwise and $(C_1',f',C_2')\in R$, then $(C_1,f,C_2)\in R$.


For $f:X_1\rightarrow X_2$, we define $f[x_1\mapsto x_2]:X_1\cup\{x_1\}\rightarrow X_2\cup\{x_2\}$, $z\in X_1\cup\{x_1\}$,(1)$f[x_1\mapsto x_2](z)=
x_2$,if $z=x_1$;(2)$f[x_1\mapsto x_2](z)=f(z)$, otherwise. Where $X_1\subseteq \mathbb{E}_1$, $X_2\subseteq \mathbb{E}_2$, $x_1\in \mathbb{E}_1$, $x_2\in \mathbb{E}_2$.
\end{definition}

\begin{definition}[Strong (hereditary) history-preserving bisimulation]\label{HHPB}
A strong history-preserving (hp-) bisimulation is a posetal relation $R\subseteq\mathcal{C}(\mathcal{E}_1)\overline{\times}\mathcal{C}(\mathcal{E}_2)$ such that if $(C_1,f,C_2)\in R$, and $C_1\xrightarrow{e_1} C_1'$, then $C_2\xrightarrow{e_2} C_2'$, with $(C_1',f[e_1\mapsto e_2],C_2')\in R$, and vice-versa. $\mathcal{E}_1,\mathcal{E}_2$ are strong history-preserving (hp-)bisimilar and are written $\mathcal{E}_1\sim_{hp}\mathcal{E}_2$ if there exists a strong hp-bisimulation $R$ such that $(\emptyset,\emptyset,\emptyset)\in R$.

A strongly hereditary history-preserving (hhp-)bisimulation is a downward closed strong hp-bisimulation. $\mathcal{E}_1,\mathcal{E}_2$ are strongly hereditary history-preserving (hhp-)bisimilar and are written $\mathcal{E}_1\sim_{hhp}\mathcal{E}_2$.
\end{definition}

\subsection{Laws and Congruence}

Based on the concepts of strongly truly concurrent bisimulation equivalences, we get the following laws.

\begin{proposition}[Monoid laws for strong pomset bisimulation] The monoid laws for strong pomset bisimulation are as follows.

\begin{enumerate}
  \item $P+Q\sim_p Q+P$;
  \item $P+(Q+R)\sim_p (P+Q)+R$;
  \item $P+P\sim_p P$;
  \item $P+\textbf{nil}\sim_p P$.
\end{enumerate}

\end{proposition}

\begin{proof}
\begin{enumerate}
  \item $P+Q\sim_p Q+P$. By the transition rules $\textbf{Sum}_{1,2}$ in Table \ref{TRForCTC}, we get

      $$\frac{P\xrightarrow{p}P'}{P+ Q\xrightarrow{p}P'} (p\subseteq P) \quad \frac{P\xrightarrow{p}P'}{Q+ P\xrightarrow{p}P'}(p\subseteq P)$$

      $$\frac{Q\xrightarrow{q}Q'}{P+ Q\xrightarrow{q}Q'}(q\subseteq Q) \quad \frac{Q\xrightarrow{q}Q'}{Q+ P\xrightarrow{q}Q'}(q\subseteq Q)$$

      Since $P'\sim_p P'$ and $Q'\sim_p Q'$, $P+ Q\sim_p Q+ P$, as desired.
  \item $P+(Q+R)\sim_p (P+Q)+R$. By the transition rules $\textbf{Sum}_{1,2}$ in Table \ref{TRForCTC}, we get

      $$\frac{P\xrightarrow{p}P'}{P+(Q+R)\xrightarrow{p}P'}(p\subseteq P) \quad \frac{P\xrightarrow{p}P'}{(P+Q)+R\xrightarrow{p}P'}(p\subseteq P)$$

      $$\frac{Q\xrightarrow{q}Q'}{P+(Q+R)\xrightarrow{q}Q'}(q\subseteq Q) \quad \frac{Q\xrightarrow{q}Q'}{(P+Q)+R\xrightarrow{q}Q'}(q\subseteq Q)$$

      $$\frac{R\xrightarrow{r}R'}{P+(Q+R)\xrightarrow{r}R'}(r\subseteq R) \quad \frac{R\xrightarrow{r}R'}{(P+Q)+R\xrightarrow{r}R'}(r\subseteq R)$$

      Since $P'\sim_p P'$, $Q'\sim_p Q'$ and $R'\sim_p R'$, $P+(Q+R)\sim_p (P+Q)+R$, as desired.
  \item $P+P\sim_p P$. By the transition rules $\textbf{Sum}_{1,2}$ in Table \ref{TRForCTC}, we get

      $$\frac{P\xrightarrow{p}P'}{P+ P\xrightarrow{p}P'}(p\subseteq P) \quad \frac{P\xrightarrow{p}P'}{P\xrightarrow{p}P'}(p\subseteq P)$$

      Since $P'\sim_p P'$, $P+ P\sim_p P$, as desired.
  \item $P+\textbf{nil}\sim_p P$. By the transition rules $\textbf{Sum}_{1,2}$ in Table \ref{TRForCTC}, we get

      $$\frac{P\xrightarrow{p}P'}{P+ \textbf{nil}\xrightarrow{p}P'}(p\subseteq P) \quad \frac{P\xrightarrow{p}P'}{P\xrightarrow{p}P'}(p\subseteq P)$$

      Since $P'\sim_p P'$, $P+ \textbf{nil}\sim_p P$, as desired.
\end{enumerate}
\end{proof}

\begin{proposition}[Monoid laws for strong step bisimulation] The monoid laws for strong step bisimulation are as follows.
\begin{enumerate}
  \item $P+Q\sim_s Q+P$;
  \item $P+(Q+R)\sim_s (P+Q)+R$;
  \item $P+P\sim_s P$;
  \item $P+\textbf{nil}\sim_s P$.
\end{enumerate}
\end{proposition}

\begin{proof}
\begin{enumerate}
  \item $P+Q\sim_s Q+P$. By the transition rules $\textbf{Sum}_{1,2}$ in Table \ref{TRForCTC}, we get

      $$\frac{P\xrightarrow{p}P'}{P+ Q\xrightarrow{p}P'} (p\subseteq P,\forall\alpha,\beta \in p,\textrm{ are pairwise concurrent})$$

      $$\frac{P\xrightarrow{p}P'}{Q+ P\xrightarrow{p}P'}(p\subseteq P,\forall\alpha,\beta \in p,\textrm{ are pairwise concurrent})$$

      $$\frac{Q\xrightarrow{q}Q'}{P+ Q\xrightarrow{q}Q'}(q\subseteq Q,\forall\alpha,\beta \in q,\textrm{ are pairwise concurrent})$$

      $$\frac{Q\xrightarrow{q}Q'}{Q+ P\xrightarrow{q}Q'}(q\subseteq Q,\forall\alpha,\beta \in q,\textrm{ are pairwise concurrent})$$

      Since $P'\sim_s P'$ and $Q'\sim_s Q'$, $P+ Q\sim_s Q+ P$, as desired.
  \item $P+(Q+R)\sim_s (P+Q)+R$. By the transition rules $\textbf{Sum}_{1,2}$ in Table \ref{TRForCTC}, we get

      $$\frac{P\xrightarrow{p}P'}{P+(Q+R)\xrightarrow{p}P'}(p\subseteq P,\forall\alpha,\beta \in p,\textrm{ are pairwise concurrent})$$

      $$\frac{P\xrightarrow{p}P'}{(P+Q)+R\xrightarrow{p}P'}(p\subseteq P,\forall\alpha,\beta \in p,\textrm{ are pairwise concurrent})$$

      $$\frac{Q\xrightarrow{q}Q'}{P+(Q+R)\xrightarrow{q}Q'}(q\subseteq Q,\forall\alpha,\beta \in q,\textrm{ are pairwise concurrent})$$

      $$\frac{Q\xrightarrow{q}Q'}{(P+Q)+R\xrightarrow{q}Q'}(q\subseteq Q,\forall\alpha,\beta \in q,\textrm{ are pairwise concurrent})$$

      $$\frac{R\xrightarrow{r}R'}{P+(Q+R)\xrightarrow{r}R'}(r\subseteq R,\forall\alpha,\beta \in r,\textrm{ are pairwise concurrent})$$

      $$\frac{R\xrightarrow{r}R'}{(P+Q)+R\xrightarrow{r}R'}(r\subseteq R,\forall\alpha,\beta \in r,\textrm{ are pairwise concurrent})$$

      Since $P'\sim_s P'$, $Q'\sim_s Q'$ and $R'\sim_s R'$, $P+(Q+R)\sim_s (P+Q)+R$, as desired.
  \item $P+P\sim_s P$. By the transition rules $\textbf{Sum}_{1,2}$ in Table \ref{TRForCTC}, we get

      $$\frac{P\xrightarrow{p}P'}{P+ P\xrightarrow{p}P'}(p\subseteq P,\forall\alpha,\beta \in p,\textrm{ are pairwise concurrent})$$

      $$\frac{P\xrightarrow{p}P'}{P\xrightarrow{p}P'}(p\subseteq P,\forall\alpha,\beta \in p,\textrm{ are pairwise concurrent})$$

      Since $P'\sim_s P'$, $P+ P\sim_s P$, as desired.
  \item $P+\textbf{nil}\sim_s P$. By the transition rules $\textbf{Sum}_{1,2}$ in Table \ref{TRForCTC}, we get

      $$\frac{P\xrightarrow{p}P'}{P+ \textbf{nil}\xrightarrow{p}P'}(p\subseteq P,\forall\alpha,\beta \in p,\textrm{ are pairwise concurrent})$$

      $$\frac{P\xrightarrow{p}P'}{P\xrightarrow{p}P'}(p\subseteq P,\forall\alpha,\beta \in p,\textrm{ are pairwise concurrent})$$

      Since $P'\sim_s P'$, $P+ \textbf{nil}\sim_s P$, as desired.
\end{enumerate}
\end{proof}

\begin{proposition}[Monoid laws for strong hp-bisimulation] The monoid laws for strong hp-bisimulation are as follows.
\begin{enumerate}
  \item $P+Q\sim_{hp} Q+P$;
  \item $P+(Q+R)\sim_{hp} (P+Q)+R$;
  \item $P+P\sim_{hp} P$;
  \item $P+\textbf{nil}\sim_{hp} P$.
\end{enumerate}
\end{proposition}

\begin{proof}
\begin{enumerate}
  \item $P+Q\sim_{hp} Q+P$. By the transition rules $\textbf{Sum}_{1,2}$ in Table \ref{TRForCTC}, we get

      $$\frac{P\xrightarrow{\alpha}P'}{P+ Q\xrightarrow{\alpha}P'} \quad \frac{P\xrightarrow{\alpha}P'}{Q+ P\xrightarrow{\alpha}P'}$$

      $$\frac{Q\xrightarrow{\beta}Q'}{P+ Q\xrightarrow{\beta}Q'} \quad \frac{Q\xrightarrow{\beta}Q'}{Q+ P\xrightarrow{\beta}Q'}$$

      Since $(C(P+ Q),f,C(Q+ P))\in\sim_{hp}$, $(C((P+ Q)'),f[\alpha\mapsto \alpha],C((Q+ P)'))\in\sim_{hp}$ and $(C((P+ Q)'),f[\beta\mapsto \beta],C((Q+ P)'))\in\sim_{hp}$, $P+ Q\sim_{hp} Q+ P$, as desired.
  \item $P+(Q+R)\sim_{hp} (P+Q)+R$. By the transition rules $\textbf{Sum}_{1,2}$ in Table \ref{TRForCTC}, we get

      $$\frac{P\xrightarrow{\alpha}P'}{P+(Q+R)\xrightarrow{\alpha}P'} \quad \frac{P\xrightarrow{\alpha}P'}{(P+Q)+R\xrightarrow{\alpha}P'}$$

      $$\frac{Q\xrightarrow{\beta}Q'}{P+(Q+R)\xrightarrow{\beta}Q'} \quad \frac{Q\xrightarrow{\beta}Q'}{(P+Q)+R\xrightarrow{\beta}Q'}$$

      $$\frac{R\xrightarrow{\gamma}R'}{P+(Q+R)\xrightarrow{\gamma}R'} \quad \frac{R\xrightarrow{\gamma}R'}{(P+Q)+R\xrightarrow{\gamma}R'}$$

      Since $(C(P+ (Q+R)),f,C((P+Q)+R))\in\sim_{hp}$, $(C((P+ (Q+R))'),f[\alpha\mapsto \alpha],C((P+Q)+R)'))\in\sim_{hp}$, $(C((P+ (Q+R))'),f[\beta\mapsto \beta],C((P+Q)+R)'))\in\sim_{hp}$ and $(C((P+ (Q+R))'),f[\gamma\mapsto \gamma],C((P+Q)+R)'))\in\sim_{hp}$, $P+(Q+R)\sim_{hp} (P+Q)+R$, as desired.
  \item $P+P\sim_{hp} P$. By the transition rules $\textbf{Sum}_{1,2}$ in Table \ref{TRForCTC}, we get

      $$\frac{P\xrightarrow{\alpha}P'}{P+ P\xrightarrow{\alpha}P'} \quad \frac{P\xrightarrow{\alpha}P'}{P\xrightarrow{\alpha}P'}$$

      Since $(C(P+P),f,C(P))\in\sim_{hp}$, $(C((P+ P)'),f[\alpha\mapsto \alpha],C((P)'))\in\sim_{hp}$, $P+ P\sim_{hp} P$, as desired.
  \item $P+\textbf{nil}\sim_{hp} P$. By the transition rules $\textbf{Sum}_{1,2}$ in Table \ref{TRForCTC}, we get

      $$\frac{P\xrightarrow{\alpha}P'}{P+ \textbf{nil}\xrightarrow{\alpha}P'} \quad \frac{P\xrightarrow{\alpha}P'}{P\xrightarrow{\alpha}P'}$$

      Since $(C(P+\textbf{nil}),f,C(P))\in\sim_{hp}$, $(C((P+ \textbf{nil})'),f[\alpha\mapsto \alpha],C((P)'))\in\sim_{hp}$, $P+ \textbf{nil}\sim_{hp} P$, as desired.
\end{enumerate}
\end{proof}

\begin{proposition}[Monoid laws for strongly hhp-bisimulation] The monoid laws for strongly hhp-bisimulation are as follows.
\begin{enumerate}
  \item $P+Q\sim_{hhp} Q+P$;
  \item $P+(Q+R)\sim_{hhp} (P+Q)+R$;
  \item $P+P\sim_{hhp} P$;
  \item $P+\textbf{nil}\sim_{hhp} P$.
\end{enumerate}
\end{proposition}

\begin{proof}
\begin{enumerate}
  \item $P+Q\sim_{hhp} Q+P$. By the transition rules $\textbf{Sum}_{1,2}$ in Table \ref{TRForCTC}, we get

      $$\frac{P\xrightarrow{\alpha}P'}{P+ Q\xrightarrow{\alpha}P'} \quad \frac{P\xrightarrow{\alpha}P'}{Q+ P\xrightarrow{\alpha}P'}$$

      $$\frac{Q\xrightarrow{\beta}Q'}{P+ Q\xrightarrow{\beta}Q'} \quad \frac{Q\xrightarrow{\beta}Q'}{Q+ P\xrightarrow{\beta}Q'}$$

      Since $(C(P+ Q),f,C(Q+ P))\in\sim_{hhp}$, $(C((P+ Q)'),f[\alpha\mapsto \alpha],C((Q+ P)'))\in\sim_{hhp}$ and $(C((P+ Q)'),f[\beta\mapsto \beta],C((Q+ P)'))\in\sim_{hhp}$, $P+ Q\sim_{hhp} Q+ P$, as desired.
  \item $P+(Q+R)\sim_{hhp} (P+Q)+R$. By the transition rules $\textbf{Sum}_{1,2}$ in Table \ref{TRForCTC}, we get

      $$\frac{P\xrightarrow{\alpha}P'}{P+(Q+R)\xrightarrow{\alpha}P'} \quad \frac{P\xrightarrow{\alpha}P'}{(P+Q)+R\xrightarrow{\alpha}P'}$$

      $$\frac{Q\xrightarrow{\beta}Q'}{P+(Q+R)\xrightarrow{\beta}Q'} \quad \frac{Q\xrightarrow{\beta}Q'}{(P+Q)+R\xrightarrow{\beta}Q'}$$

      $$\frac{R\xrightarrow{\gamma}R'}{P+(Q+R)\xrightarrow{\gamma}R'} \quad \frac{R\xrightarrow{\gamma}R'}{(P+Q)+R\xrightarrow{\gamma}R'}$$

      Since $(C(P+ (Q+R)),f,C((P+Q)+R))\in\sim_{hhp}$, $(C((P+ (Q+R))'),f[\alpha\mapsto \alpha],C((P+Q)+R)'))\in\sim_{hhp}$, $(C((P+ (Q+R))'),f[\beta\mapsto \beta],C((P+Q)+R)'))\in\sim_{hhp}$ and $(C((P+ (Q+R))'),f[\gamma\mapsto \gamma],C((P+Q)+R)'))\in\sim_{hhp}$, $P+(Q+R)\sim_{hhp} (P+Q)+R$, as desired.
  \item $P+P\sim_{hhp} P$. By the transition rules $\textbf{Sum}_{1,2}$ in Table \ref{TRForCTC}, we get

      $$\frac{P\xrightarrow{\alpha}P'}{P+ P\xrightarrow{\alpha}P'} \quad \frac{P\xrightarrow{\alpha}P'}{P\xrightarrow{\alpha}P'}$$

      Since $(C(P+P),f,C(P))\in\sim_{hhp}$, $(C((P+ P)'),f[\alpha\mapsto \alpha],C((P)'))\in\sim_{hhp}$, $P+ P\sim_{hhp} P$, as desired.
  \item $P+\textbf{nil}\sim_{hhp} P$. By the transition rules $\textbf{Sum}_{1,2}$ in Table \ref{TRForCTC}, we get

      $$\frac{P\xrightarrow{\alpha}P'}{P+ \textbf{nil}\xrightarrow{\alpha}P'} \quad \frac{P\xrightarrow{\alpha}P'}{P\xrightarrow{\alpha}P'}$$

      Since $(C(P+\textbf{nil}),f,C(P))\in\sim_{hhp}$, $(C((P+ \textbf{nil})'),f[\alpha\mapsto \alpha],C((P)'))\in\sim_{hhp}$, $P+ \textbf{nil}\sim_{hhp} P$, as desired.
\end{enumerate}
\end{proof}

\begin{proposition}[Static laws for strong step bisimulation] \label{SLSSB}
The static laws for strong step bisimulation are as follows.
\begin{enumerate}
  \item $P\parallel Q\sim_s Q\parallel P$;
  \item $P\parallel(Q\parallel R)\sim_s (P\parallel Q)\parallel R$;
  \item $P\parallel \textbf{nil}\sim_s P$;
  \item $P\setminus L\sim_s P$, if $\mathcal{L}(P)\cap(L\cup\overline{L})=\emptyset$;
  \item $P\setminus K\setminus L\sim_s P\setminus(K\cup L)$;
  \item $P[f]\setminus L\sim_s P\setminus f^{-1}(L)[f]$;
  \item $(P\parallel Q)\setminus L\sim_s P\setminus L\parallel Q\setminus L$, if $\mathcal{L}(P)\cap\overline{\mathcal{L}(Q)}\cap(L\cup\overline{L})=\emptyset$;
  \item $P[Id]\sim_s P$;
  \item $P[f]\sim_s P[f']$, if $f\upharpoonright\mathcal{L}(P)=f'\upharpoonright\mathcal{L}(P)$;
  \item $P[f][f']\sim_s P[f'\circ f]$;
  \item $(P\parallel Q)[f]\sim_s P[f]\parallel Q[f]$, if $f\upharpoonright(L\cup\overline{L})$ is one-to-one, where $L=\mathcal{L}(P)\cup\mathcal{L}(Q)$.
\end{enumerate}
\end{proposition}

\begin{proof}
Though transition rules in Table \ref{TRForCTC} are defined in the flavor of single event, they can be modified into a step (a set of events within which each event is pairwise concurrent), we omit them. If we treat a single event as a step containing just one event, the proof of the static laws does not exist any problem, so we use this way and still use the transition rules in Table \ref{TRForCTC}.

\begin{enumerate}
  \item $P\parallel Q\sim_s Q\parallel P$. By the transition rules $\textbf{Com}_{1,2,3,4}$ in Table \ref{TRForCTC}, we get

      $$\frac{P\xrightarrow{\alpha}P'\quad Q\nrightarrow}{P\parallel Q\xrightarrow{\alpha}P'\parallel Q}
      \quad\frac{P\xrightarrow{\alpha}P'\quad Q\nrightarrow}{Q\parallel P\xrightarrow{\alpha}Q\parallel P'}$$

      $$\frac{Q\xrightarrow{\beta}Q'\quad P\nrightarrow}{P\parallel Q\xrightarrow{\beta}P\parallel Q'}
      \quad\frac{Q\xrightarrow{\beta}Q'\quad P\nrightarrow}{Q\parallel P\xrightarrow{\beta}Q'\parallel P}$$

      $$\frac{P\xrightarrow{\alpha}P'\quad Q\xrightarrow{\beta}Q'}{P\parallel Q\xrightarrow{\{\alpha,\beta\}}P'\parallel Q'}(\beta\neq\overline{\alpha})
      \quad\frac{P\xrightarrow{\alpha}P'\quad Q\xrightarrow{\beta}Q'}{Q\parallel P\xrightarrow{\{\alpha,\beta\}}Q'\parallel P'}(\beta\neq\overline{\alpha})$$

      $$\frac{P\xrightarrow{l}P'\quad Q\xrightarrow{\overline{l}}Q'}{P\parallel Q\xrightarrow{\tau}P'\parallel Q'}
      \quad\frac{P\xrightarrow{l}P'\quad Q\xrightarrow{\overline{l}}Q'}{Q\parallel P\xrightarrow{\tau}Q'\parallel P'}$$

      So, with the assumptions $P'\parallel Q \sim_s Q\parallel P'$, $P\parallel Q' \sim_s Q'\parallel P$ and $P'\parallel Q' \sim_s Q'\parallel P'$, $P\parallel Q\sim_s Q\parallel P$, as desired.
  \item $P\parallel(Q\parallel R)\sim_s (P\parallel Q)\parallel R$. By the transition rules $\textbf{Com}_{1,2,3,4}$ in Table \ref{TRForCTC}, we get

      $$\frac{P\xrightarrow{\alpha}P'\quad Q\nrightarrow\quad R\nrightarrow}{P\parallel (Q\parallel R)\xrightarrow{\alpha}P'\parallel (Q\parallel R)}
      \quad\frac{P\xrightarrow{\alpha}P'\quad Q\nrightarrow\quad R\nrightarrow}{(P\parallel Q)\parallel R\xrightarrow{\alpha}(P'\parallel Q)\parallel R}$$

      $$\frac{Q\xrightarrow{\beta}Q'\quad P\nrightarrow\quad R\nrightarrow}{P\parallel (Q\parallel R)\xrightarrow{\beta}P\parallel (Q'\parallel R)}
      \quad\frac{Q\xrightarrow{\beta}Q'\quad P\nrightarrow\quad R\nrightarrow}{(P\parallel Q)\parallel R\xrightarrow{\beta}(P\parallel Q')\parallel R}$$

      $$\frac{R\xrightarrow{\gamma}R'\quad P\nrightarrow\quad Q\nrightarrow}{P\parallel (Q\parallel R)\xrightarrow{\gamma}P\parallel (Q\parallel R')}
      \quad\frac{R\xrightarrow{\gamma}R'\quad P\nrightarrow\quad Q\nrightarrow}{(P\parallel Q)\parallel R\xrightarrow{\gamma}(P\parallel Q)\parallel R'}$$

      $$\frac{P\xrightarrow{\alpha}P'\quad Q\xrightarrow{\beta}Q'\quad R\nrightarrow}{P\parallel (Q\parallel R)\xrightarrow{\{\alpha,\beta\}}P'\parallel (Q'\parallel R)}(\beta\neq\overline{\alpha})
      \quad\frac{P\xrightarrow{\alpha}P'\quad Q\xrightarrow{\beta}Q'\quad R\nrightarrow}{(P\parallel Q)\parallel R\xrightarrow{\{\alpha,\beta\}}(P'\parallel Q')\parallel R}(\beta\neq\overline{\alpha})$$

      $$\frac{P\xrightarrow{\alpha}P'\quad R\xrightarrow{\gamma}R'\quad Q\nrightarrow}{P\parallel (Q\parallel R)\xrightarrow{\{\alpha,\gamma\}}P'\parallel (Q\parallel R')}(\gamma\neq\overline{\alpha})
      \quad\frac{P\xrightarrow{\alpha}P'\quad R\xrightarrow{\gamma}R'\quad Q\nrightarrow}{(P\parallel Q)\parallel R\xrightarrow{\{\alpha,\gamma\}}(P'\parallel Q)\parallel R]}(\gamma\neq\overline{\alpha})$$

      $$\frac{Q\xrightarrow{\beta}P'\quad R\xrightarrow{\gamma}R'\quad P\nrightarrow}{P\parallel (Q\parallel R)\xrightarrow{\{\beta,\gamma\}}P\parallel (Q'\parallel R')}(\gamma\neq\overline{\beta})
      \quad\frac{Q\xrightarrow{\beta}Q'\quad R\xrightarrow{\gamma}R'\quad P\nrightarrow}{(P\parallel Q)\parallel R\xrightarrow{\{\beta,\gamma\}}(P\parallel Q')\parallel R'}(\gamma\neq\overline{\beta})$$

      $$\frac{P\xrightarrow{\alpha}P'\quad Q\xrightarrow{\beta}Q'\quad R\xrightarrow{\gamma}R'}{P\parallel (Q\parallel R)\xrightarrow{\{\alpha,\beta,\gamma\}}P'\parallel (Q'\parallel R')}(\beta\neq\overline{\alpha},\gamma\neq\overline{\alpha},\gamma\neq\overline{\beta})
      \quad\frac{P\xrightarrow{\alpha}P'\quad Q\xrightarrow{\beta}Q'\quad R\xrightarrow{\gamma}R'}{(P\parallel Q)\parallel R\xrightarrow{\{\alpha,\beta,\gamma\}}(P'\parallel Q')\parallel R'}(\beta\neq\overline{\alpha},\gamma\neq\overline{\alpha},\gamma\neq\overline{\beta})$$

      $$\frac{P\xrightarrow{l}P'\quad Q\xrightarrow{\overline{l}}Q'\quad R\nrightarrow}{P\parallel (Q\parallel R)\xrightarrow{\tau}P'\parallel (Q'\parallel R)}
      \quad\frac{P\xrightarrow{l}P'\quad Q\xrightarrow{\overline{l}}Q'\quad R\nrightarrow}{(P\parallel Q)\parallel R\xrightarrow{\tau}(P'\parallel Q')\parallel R}$$

      $$\frac{P\xrightarrow{l}P'\quad R\xrightarrow{\overline{l}}R'\quad Q\nrightarrow}{P\parallel (Q\parallel R)\xrightarrow{\tau}P'\parallel (Q\parallel R')}
      \quad\frac{P\xrightarrow{l}P'\quad R\xrightarrow{\overline{l}}R'\quad Q\nrightarrow}{(P\parallel Q)\parallel R\xrightarrow{\tau}(P'\parallel Q)\parallel R]}$$

      $$\frac{Q\xrightarrow{l}P'\quad R\xrightarrow{\overline{l}}R'\quad P\nrightarrow}{P\parallel (Q\parallel R)\xrightarrow{\tau}P\parallel (Q'\parallel R')}
      \quad\frac{Q\xrightarrow{l}Q'\quad R\xrightarrow{\overline{l}}R'\quad P\nrightarrow}{(P\parallel Q)\parallel R\xrightarrow{\tau}(P\parallel Q')\parallel R'}$$

      $$\frac{P\xrightarrow{l}P'\quad Q\xrightarrow{\overline{l}}Q'\quad R\xrightarrow{\gamma}R'}{P\parallel (Q\parallel R)\xrightarrow{\tau,\gamma}P'\parallel (Q'\parallel R')}
      \quad\frac{P\xrightarrow{l}P'\quad Q\xrightarrow{\overline{l}}Q'\quad R\xrightarrow{\gamma}R'}{(P\parallel Q)\parallel R\xrightarrow{\tau,\gamma}(P'\parallel Q')\parallel R'}$$

      $$\frac{P\xrightarrow{l}P'\quad R\xrightarrow{\overline{l}}R'\quad Q\xrightarrow{\beta}Q'}{P\parallel (Q\parallel R)\xrightarrow{\tau,\beta}P'\parallel (Q'\parallel R')}
      \quad\frac{P\xrightarrow{l}P'\quad R\xrightarrow{\overline{l}}R'\quad Q\xrightarrow{\beta}Q'}{(P\parallel Q)\parallel R\xrightarrow{\tau,\beta}(P'\parallel Q')\parallel R]}$$

      $$\frac{Q\xrightarrow{l}Q'\quad R\xrightarrow{\overline{l}}R'\quad P\xrightarrow{\alpha}P'}{P\parallel (Q\parallel R)\xrightarrow{\tau,\alpha}P'\parallel (Q'\parallel R')}
      \quad\frac{Q\xrightarrow{l}Q'\quad R\xrightarrow{\overline{l}}R'\quad P\xrightarrow{\alpha}P'}{(P\parallel Q)\parallel R\xrightarrow{\tau,\alpha}(P'\parallel Q')\parallel R'}$$

      So, with the assumptions $P'\parallel (Q\parallel R) \sim_s (P'\parallel Q)\parallel R$, $P\parallel (Q'\parallel R) \sim_s (P\parallel Q')\parallel R$, $P\parallel (Q\parallel R') \sim_s (P\parallel Q)\parallel R'$, $P'\parallel (Q'\parallel R) \sim_s (P'\parallel Q')\parallel R$, $P'\parallel (Q\parallel R') \sim_s (P'\parallel Q)\parallel R'$, $P\parallel (Q'\parallel R') \sim_s (P\parallel Q')\parallel R'$ and $P'\parallel (Q'\parallel R') \sim_s (P'\parallel Q')\parallel R'$, $P\parallel (Q\parallel R) \sim_s (P\parallel Q)\parallel R$, as desired.
  \item $P\parallel \textbf{nil}\sim_s P$. By the transition rules $\textbf{Com}_{1,2,3,4}$ in Table \ref{TRForCTC}, we get

      $$\frac{P\xrightarrow{\alpha}P'}{P\parallel \textbf{nil}\xrightarrow{\alpha}P'} \quad \frac{P\xrightarrow{\alpha}P'}{P\xrightarrow{\alpha}P'}$$

      Since $P'\sim_s P'$, $P\parallel \textbf{nil}\sim_s P$, as desired.
  \item $P\setminus L\sim_s P$, if $\mathcal{L}(P)\cap(L\cup\overline{L})=\emptyset$. By the transition rules $\textbf{Res}_{1,2}$ in Table \ref{TRForCTC}, we get

      $$\frac{P\xrightarrow{\alpha}P'}{P\setminus L\xrightarrow{\alpha}P'\setminus L} (\mathcal{L}(P)\cap(L\cup\overline{L})=\emptyset)\quad \frac{P\xrightarrow{\alpha}P'}{P\xrightarrow{\alpha}P'}$$

      Since $P'\sim_s P'$, and with the assumption $P'\setminus L\sim_s P'$, $P\setminus L\sim_s P$, if $\mathcal{L}(P)\cap(L\cup\overline{L})=\emptyset$, as desired.
  \item $P\setminus K\setminus L\sim_s P\setminus(K\cup L)$. By the transition rules $\textbf{Res}_{1,2}$ in Table \ref{TRForCTC}, we get

      $$\frac{P\xrightarrow{\alpha}P'}{P\setminus K\setminus L\xrightarrow{\alpha}P'\setminus K\setminus L} \quad \frac{P\xrightarrow{\alpha}P'}{P\setminus (K\cup L)\xrightarrow{\alpha}P'\setminus (K\cup L)}$$

      Since $P'\sim_s P'$, and with the assumption $P'\setminus K\setminus L\sim_s P'\setminus(K\cup L)$, $P\setminus K\setminus L\sim_s P\setminus(K\cup L)$, as desired.
  \item $P[f]\setminus L\sim_s P\setminus f^{-1}(L)[f]$. By the transition rules $\textbf{Res}_{1,2}$ and $\textbf{Rel}_{1,2}$ in Table \ref{TRForCTC}, we get

      $$\frac{P\xrightarrow{\alpha}P'}{P[f]\setminus L\xrightarrow{f(\alpha)}P'[f]\setminus L}\quad \frac{P\xrightarrow{\alpha}P'}{P\setminus f^{-1}(L)[f]\xrightarrow{f(\alpha)}P'\setminus f^{-1}(L)[f]}$$

      So, with the assumption $P'[f]\setminus L\sim_s P'\setminus f^{-1}(L)[f]$, $P[f]\setminus L\sim_s P\setminus f^{-1}(L)[f]$, as desired.
  \item $(P\parallel Q)\setminus L\sim_s P\setminus L\parallel Q\setminus L$, if $\mathcal{L}(P)\cap\overline{\mathcal{L}(Q)}\cap(L\cup\overline{L})=\emptyset$. By the transition rules $\textbf{Com}_{1,2,3,4}$ and $\textbf{Res}_{1,2}$ in Table \ref{TRForCTC}, we get

      $$\frac{P\xrightarrow{\alpha}P'\quad Q\nrightarrow}{(P\parallel Q)\setminus L\xrightarrow{\alpha}(P'\parallel Q)\setminus L}(\mathcal{L}(P)\cap\overline{\mathcal{L}(Q)}\cap(L\cup\overline{L})=\emptyset)$$
      $$\frac{P\xrightarrow{\alpha}P'\quad Q\nrightarrow}{P\setminus L\parallel Q\setminus L\xrightarrow{\alpha}P'\setminus L\parallel Q\setminus L}(\mathcal{L}(P)\cap\overline{\mathcal{L}(Q)}\cap(L\cup\overline{L})=\emptyset)$$

      $$\frac{Q\xrightarrow{\beta}Q'\quad P\nrightarrow}{(P\parallel Q)\setminus L\xrightarrow{\beta}(P\parallel Q')\setminus L}(\mathcal{L}(P)\cap\overline{\mathcal{L}(Q)}\cap(L\cup\overline{L})=\emptyset)$$
      $$\frac{Q\xrightarrow{\beta}Q'\quad P\nrightarrow}{P\setminus L\parallel Q\setminus L\xrightarrow{\beta}P\setminus L\parallel Q'\setminus L}(\mathcal{L}(P)\cap\overline{\mathcal{L}(Q)}\cap(L\cup\overline{L})=\emptyset)$$

      $$\frac{P\xrightarrow{\alpha}P'\quad Q\xrightarrow{\beta}Q'}{(P\parallel Q)\setminus L\xrightarrow{\{\alpha,\beta\}}(P'\parallel Q')\setminus L}(\mathcal{L}(P)\cap\overline{\mathcal{L}(Q)}\cap(L\cup\overline{L})=\emptyset)$$
      $$\frac{P\xrightarrow{\alpha}P'\quad Q\xrightarrow{\beta}Q'}{P\setminus L\parallel Q\setminus L\xrightarrow{\{\alpha,\beta\}}(P'\parallel Q')\setminus L}(\mathcal{L}(P)\cap\overline{\mathcal{L}(Q)}\cap(L\cup\overline{L})=\emptyset)$$

      $$\frac{P\xrightarrow{l}P'\quad Q\xrightarrow{\overline{l}}Q'}{(P\parallel Q)\setminus L\xrightarrow{\tau}(P'\parallel Q')\setminus L}(\mathcal{L}(P)\cap\overline{\mathcal{L}(Q)}\cap(L\cup\overline{L})=\emptyset)$$
      $$\frac{P\xrightarrow{l}P'\quad Q\xrightarrow{\overline{l}}Q'}{(P\setminus L\parallel Q\setminus L\xrightarrow{\tau}P'\setminus L\parallel Q'\setminus L}(\mathcal{L}(P)\cap\overline{\mathcal{L}(Q)}\cap(L\cup\overline{L})=\emptyset)$$

      Since $(P'\parallel Q)\setminus L\sim_s P'\setminus L\parallel Q\setminus L$, $(P\parallel Q')\setminus L\sim_s P\setminus L\parallel Q'\setminus L$ and $(P'\parallel Q')\setminus L\sim_s P'\setminus L\parallel Q'\setminus L$, $(P\parallel Q)\setminus L\sim_s P\setminus L\parallel Q\setminus L$, if $\mathcal{L}(P)\cap\overline{\mathcal{L}(Q)}\cap(L\cup\overline{L})=\emptyset$, as desired.
  \item $P[Id]\sim_s P$. By the transition rules $\textbf{Rel}_{1,2}$ in Table \ref{TRForCTC}, we get

      $$\frac{P\xrightarrow{\alpha}P'}{P[Id]\xrightarrow{Id(\alpha)}P'[Id]}\quad \frac{P\xrightarrow{\alpha}P'}{P\xrightarrow{\alpha}P'}$$

      So, with the assumption $P'[Id]\sim_s P'$ and $Id(\alpha)=\alpha$, $P[Id]\sim_s P$, as desired.
  \item $P[f]\sim_s P[f']$, if $f\upharpoonright\mathcal{L}(P)=f'\upharpoonright\mathcal{L}(P)$. By the transition rules $\textbf{Rel}_{1,2}$ in Table \ref{TRForCTC}, we get

      $$\frac{P\xrightarrow{\alpha}P'}{P[f]\xrightarrow{f(\alpha)}P'[f]}\quad \frac{P\xrightarrow{\alpha}P'}{P[f']\xrightarrow{f'(\alpha)}P'[f']}$$

      So, with the assumption $P'[f]\sim_s P'[f']$ and $f(\alpha)=f'(\alpha)$, if $f\upharpoonright\mathcal{L}(P)=f'\upharpoonright\mathcal{L}(P)$, $P[f]\sim_s P[f']$, as desired.
  \item $P[f][f']\sim_s P[f'\circ f]$. By the transition rules $\textbf{Rel}_{1,2}$ in Table \ref{TRForCTC}, we get

      $$\frac{P\xrightarrow{\alpha}P'}{P[f][f']\xrightarrow{f'(f(\alpha))}P'[f][f']}\quad \frac{P\xrightarrow{\alpha}P'}{P[f'\circ f]\xrightarrow{f'(f(\alpha))}P'[f'\circ f]}$$

      So, with the assumption $P'[f][f']\sim_s P'[f'\circ f]$, $P[f][f']\sim_s P[f'\circ f]$, as desired.
  \item $(P\parallel Q)[f]\sim_s P[f]\parallel Q[f]$, if $f\upharpoonright(L\cup\overline{L})$ is one-to-one, where $L=\mathcal{L}(P)\cup\mathcal{L}(Q)$. By the transition rules $\textbf{Com}_{1,2,3,4}$ and $\textbf{Rel}_{1,2}$ in Table \ref{TRForCTC}, we get

      $$\frac{P\xrightarrow{\alpha}P'\quad Q\nrightarrow}{(P\parallel Q)[f]\xrightarrow{f(\alpha)}(P'\parallel Q)[f]}(\textrm{if } f\upharpoonright(L\cup\overline{L}) \textrm{ is one-to-one, where }L=\mathcal{L}(P)\cup\mathcal{L}(Q))$$
      $$\frac{P\xrightarrow{\alpha}P'\quad Q\nrightarrow}{P[f]\parallel Q[f]\xrightarrow{f(\alpha)}P'[f]\parallel Q[f]}(\textrm{if } f\upharpoonright(L\cup\overline{L}) \textrm{ is one-to-one, where }L=\mathcal{L}(P)\cup\mathcal{L}(Q))$$

      $$\frac{Q\xrightarrow{\beta}Q'\quad P\nrightarrow}{(P\parallel Q)[f]\xrightarrow{f(\beta)}(P\parallel Q')[f]}(\textrm{if } f\upharpoonright(L\cup\overline{L}) \textrm{ is one-to-one, where }L=\mathcal{L}(P)\cup\mathcal{L}(Q))$$
      $$\frac{Q\xrightarrow{\beta}Q'\quad P\nrightarrow}{P[f]\parallel Q[f]\xrightarrow{f(\beta)}P[f]\parallel Q'[f]}(\textrm{if } f\upharpoonright(L\cup\overline{L}) \textrm{ is one-to-one, where }L=\mathcal{L}(P)\cup\mathcal{L}(Q))$$

      $$\frac{P\xrightarrow{\alpha}P'\quad Q\xrightarrow{\beta}Q'}{(P\parallel Q)[f]\xrightarrow{\{f(\alpha),f(\beta)\}}(P'\parallel Q')[f]}(\textrm{if } f\upharpoonright(L\cup\overline{L}) \textrm{ is one-to-one, where }L=\mathcal{L}(P)\cup\mathcal{L}(Q))$$
      $$\frac{P\xrightarrow{\alpha}P'\quad Q\xrightarrow{\beta}Q'}{P[f]\parallel Q[f]\xrightarrow{\{f(\alpha),f(\beta)\}}P'[f]\parallel Q'[f]}(\textrm{if } f\upharpoonright(L\cup\overline{L}) \textrm{ is one-to-one, where }L=\mathcal{L}(P)\cup\mathcal{L}(Q))$$

      $$\frac{P\xrightarrow{l}P'\quad Q\xrightarrow{\overline{l}}Q'}{(P\parallel Q)[f]\xrightarrow{\tau}(P'\parallel Q')[f]}(\textrm{if } f\upharpoonright(L\cup\overline{L}) \textrm{ is one-to-one, where }L=\mathcal{L}(P)\cup\mathcal{L}(Q))$$
      $$\frac{P\xrightarrow{l}P'\quad Q\xrightarrow{\overline{l}}Q'}{(P[f]\parallel Q[f]\xrightarrow{\tau}P'[f]\parallel Q'[f]}(\textrm{if } f\upharpoonright(L\cup\overline{L}) \textrm{ is one-to-one, where }L=\mathcal{L}(P)\cup\mathcal{L}(Q))$$

      So, with the assumptions $(P'\parallel Q)[f]\sim_s P'[f]\parallel Q[f]$, $(P\parallel Q')[f]\sim_s P[f]\parallel Q'[f]$ and $(P'\parallel Q')[f]\sim_s P'[f]\parallel Q'[f]$, $(P\parallel Q)[f]\sim_s P[f]\parallel Q[f]$, if $f\upharpoonright(L\cup\overline{L})$ is one-to-one, where $L=\mathcal{L}(P)\cup\mathcal{L}(Q)$, as desired.
\end{enumerate}
\end{proof}

\begin{proposition}[Static laws for strong pomset bisimulation] \label{SLSPB}
The static laws for strong pomset bisimulation are as follows.
\begin{enumerate}
  \item $P\parallel Q\sim_p Q\parallel P$;
  \item $P\parallel(Q\parallel R)\sim_p (P\parallel Q)\parallel R$;
  \item $P\parallel \textbf{nil}\sim_p P$;
  \item $P\setminus L\sim_p P$, if $\mathcal{L}(P)\cap(L\cup\overline{L})=\emptyset$;
  \item $P\setminus K\setminus L\sim_p P\setminus(K\cup L)$;
  \item $P[f]\setminus L\sim_p P\setminus f^{-1}(L)[f]$;
  \item $(P\parallel Q)\setminus L\sim_p P\setminus L\parallel Q\setminus L$, if $\mathcal{L}(P)\cap\overline{\mathcal{L}(Q)}\cap(L\cup\overline{L})=\emptyset$;
  \item $P[Id]\sim_p P$;
  \item $P[f]\sim_p P[f']$, if $f\upharpoonright\mathcal{L}(P)=f'\upharpoonright\mathcal{L}(P)$;
  \item $P[f][f']\sim_p P[f'\circ f]$;
  \item $(P\parallel Q)[f]\sim_p P[f]\parallel Q[f]$, if $f\upharpoonright(L\cup\overline{L})$ is one-to-one, where $L=\mathcal{L}(P)\cup\mathcal{L}(Q)$.
\end{enumerate}
\end{proposition}

\begin{proof}
From the definition of strong pomset bisimulation (see Definition \ref{PSB}), we know that strong pomset bisimulation is defined by pomset transitions, which are labeled by pomsets. In a pomset transition, the events in the pomset are either within causality relations (defined by the prefix $.$) or in concurrency (implicitly defined by $.$ and $+$, and explicitly defined by $\parallel$), of course, they are pairwise consistent (without conflicts). In Proposition \ref{SLSSB}, we have already proven the case that all events are pairwise concurrent, so, we only need to prove the case of events in causality. Without loss of generality, we take a pomset of $p=\{\alpha,\beta:\alpha.\beta\}$. Then the pomset transition labeled by the above $p$ is just composed of one single event transition labeled by $\alpha$ succeeded by another single event transition labeled by $\beta$, that is, $\xrightarrow{p}=\xrightarrow{\alpha}\xrightarrow{\beta}$.

Similarly to the proof of static laws for strong step bisimulation (see Proposition \ref{SLSSB}), we can prove that the static laws hold for strong pomset bisimulation, we omit them.
\end{proof}

\begin{proposition}[Static laws for strong hp-bisimulation] \label{SLSHPB}
The static laws for strong hp-bisimulation are as follows.
\begin{enumerate}
  \item $P\parallel Q\sim_{hp} Q\parallel P$;
  \item $P\parallel(Q\parallel R)\sim_{hp} (P\parallel Q)\parallel R$;
  \item $P\parallel \textbf{nil}\sim_{hp} P$;
  \item $P\setminus L\sim_{hp} P$, if $\mathcal{L}(P)\cap(L\cup\overline{L})=\emptyset$;
  \item $P\setminus K\setminus L\sim_{hp} P\setminus(K\cup L)$;
  \item $P[f]\setminus L\sim_{hp} P\setminus f^{-1}(L)[f]$;
  \item $(P\parallel Q)\setminus L\sim_{hp} P\setminus L\parallel Q\setminus L$, if $\mathcal{L}(P)\cap\overline{\mathcal{L}(Q)}\cap(L\cup\overline{L})=\emptyset$;
  \item $P[Id]\sim_{hp} P$;
  \item $P[f]\sim_{hp} P[f']$, if $f\upharpoonright\mathcal{L}(P)=f'\upharpoonright\mathcal{L}(P)$;
  \item $P[f][f']\sim_{hp} P[f'\circ f]$;
  \item $(P\parallel Q)[f]\sim_{hp} P[f]\parallel Q[f]$, if $f\upharpoonright(L\cup\overline{L})$ is one-to-one, where $L=\mathcal{L}(P)\cup\mathcal{L}(Q)$.
\end{enumerate}
\end{proposition}

\begin{proof}
From the definition of strong hp-bisimulation (see Definition \ref{HHPB}), we know that strong hp-bisimulation is defined on the posetal product $(C_1,f,C_2),f:C_1\rightarrow C_2\textrm{ isomorphism}$. Two processes $P$ related to $C_1$ and $Q$ related to $C_2$, and $f:C_1\rightarrow C_2\textrm{ isomorphism}$. Initially, $(C_1,f,C_2)=(\emptyset,\emptyset,\emptyset)$, and $(\emptyset,\emptyset,\emptyset)\in\sim_{hp}$. When $P\xrightarrow{\alpha}P'$ ($C_1\xrightarrow{\alpha}C_1'$), there will be $Q\xrightarrow{\alpha}Q'$ ($C_2\xrightarrow{\alpha}C_2'$), and we define $f'=f[\alpha\mapsto \alpha]$. Then, if $(C_1,f,C_2)\in\sim_{hp}$, then $(C_1',f',C_2')\in\sim_{hp}$.

Similarly to the proof of static laws for strong pomset bisimulation (see Proposition \ref{SLSPB}), we can prove that static laws hold for strong hp-bisimulation, we just need additionally to check the above conditions on hp-bisimulation, we omit them.
\end{proof}

\begin{proposition}[Static laws for strongly hhp-bisimulation] \label{SLSHHPB}
The static laws for strongly hhp-bisimulation are as follows.
\begin{enumerate}
  \item $P\parallel Q\sim_{hhp} Q\parallel P$;
  \item $P\parallel(Q\parallel R)\sim_{hhp} (P\parallel Q)\parallel R$;
  \item $P\parallel \textbf{nil}\sim_{hhp} P$;
  \item $P\setminus L\sim_{hhp} P$, if $\mathcal{L}(P)\cap(L\cup\overline{L})=\emptyset$;
  \item $P\setminus K\setminus L\sim_{hhp} P\setminus(K\cup L)$;
  \item $P[f]\setminus L\sim_{hhp} P\setminus f^{-1}(L)[f]$;
  \item $(P\parallel Q)\setminus L\sim_{hhp} P\setminus L\parallel Q\setminus L$, if $\mathcal{L}(P)\cap\overline{\mathcal{L}(Q)}\cap(L\cup\overline{L})=\emptyset$;
  \item $P[Id]\sim_{hhp} P$;
  \item $P[f]\sim_{hhp} P[f']$, if $f\upharpoonright\mathcal{L}(P)=f'\upharpoonright\mathcal{L}(P)$;
  \item $P[f][f']\sim_{hhp} P[f'\circ f]$;
  \item $(P\parallel Q)[f]\sim_{hhp} P[f]\parallel Q[f]$, if $f\upharpoonright(L\cup\overline{L})$ is one-to-one, where $L=\mathcal{L}(P)\cup\mathcal{L}(Q)$.
\end{enumerate}
\end{proposition}

\begin{proof}
From the definition of strongly hhp-bisimulation (see Definition \ref{HHPB}), we know that strongly hhp-bisimulation is downward closed for strong hp-bisimulation.

Similarly to the proof of static laws for strong hp-bisimulation (see Proposition \ref{SLSHPB}), we can prove that static laws hold for strongly hhp-bisimulation, that is, they are downward closed for strong hp-bisimulation, we omit them.
\end{proof}

\begin{proposition}[Milner's expansion law for strongly truly concurrent bisimulations]
Milner's expansion law does not hold any more for any strongly truly concurrent bisimulation, that is,

\begin{enumerate}
  \item $\alpha\parallel\beta\nsim_p \alpha.\beta+\beta.\alpha$;
  \item $\alpha\parallel\beta\nsim_s \alpha.\beta+\beta.\alpha$;
  \item $\alpha\parallel\beta\nsim_{hp} \alpha.\beta+\beta.\alpha$;
  \item $\alpha\parallel\beta\nsim_{hhp} \alpha.\beta+\beta.\alpha$.
\end{enumerate}
\end{proposition}

\begin{proof}
In nature, it is caused by $\alpha\parallel \beta$ and $\alpha.\beta + \beta.\alpha$ having different causality structure. By the transition rules for $\textbf{Com}_{1,2,3,4}$, $\textbf{Sum}_{1,2}$ and $\textbf{Act}_{1,2}$, we have

$$\alpha\parallel \beta\xrightarrow{\{\alpha,\beta\}}\textbf{nil}$$

while

$$\alpha.\beta+ \beta.\alpha\nrightarrow^{\{\alpha,\beta\}}.$$
\end{proof}

\begin{proposition}[New expansion law for strong step bisimulation]\label{NELSSB}
Let $P\equiv (P_1[f_1]\parallel\cdots\parallel P_n[f_n])\setminus L$, with $n\geq 1$. Then

\begin{eqnarray}
P\sim_s \{(f_1(\alpha_1)\parallel\cdots\parallel f_n(\alpha_n)).(P_1'[f_1]\parallel\cdots\parallel P_n'[f_n])\setminus L: \nonumber\\
P_i\xrightarrow{\alpha_i}P_i',i\in\{1,\cdots,n\},f_i(\alpha_i)\notin L\cup\overline{L}\} \nonumber\\
+\sum\{\tau.(P_1[f_1]\parallel\cdots\parallel P_i'[f_i]\parallel\cdots\parallel P_j'[f_j]\parallel\cdots\parallel P_n[f_n])\setminus L: \nonumber\\
P_i\xrightarrow{l_1}P_i',P_j\xrightarrow{l_2}P_j',f_i(l_1)=\overline{f_j(l_2)},i<j\} \nonumber
\end{eqnarray}
\end{proposition}

\begin{proof}
Though transition rules in Table \ref{TRForCTC} are defined in the flavor of single event, they can be modified into a step (a set of events within which each event is pairwise concurrent), we omit them. If we treat a single event as a step containing just one event, the proof of the new expansion law has not any problem, so we use this way and still use the transition rules in Table \ref{TRForCTC}.

Firstly, we consider the case without Restriction and Relabeling. That is, we suffice to prove the following case by induction on the size $n$.

For $P\equiv P_1\parallel\cdots\parallel P_n$, with $n\geq 1$, we need to prove


\begin{eqnarray}
P\sim_s \{(\alpha_1\parallel\cdots\parallel \alpha_n).(P_1'\parallel\cdots\parallel P_n'): P_i\xrightarrow{\alpha_i}P_i',i\in\{1,\cdots,n\}\nonumber\\
+\sum\{\tau.(P_1\parallel\cdots\parallel P_i'\parallel\cdots\parallel P_j'\parallel\cdots\parallel P_n): P_i\xrightarrow{l}P_i',P_j\xrightarrow{\overline{l}}P_j',i<j\} \nonumber
\end{eqnarray}

For $n=1$, $P_1\sim_s \alpha_1.P_1':P_1\xrightarrow{\alpha_1}P_1'$ is obvious. Then with a hypothesis $n$, we consider $R\equiv P\parallel P_{n+1}$. By the transition rules $\textbf{Com}_{1,2,3,4}$, we can get

\begin{eqnarray}
R\sim_s \{(p\parallel \alpha_{n+1}).(P'\parallel P_{n+1}'): P\xrightarrow{p}P',P_{n+1}\xrightarrow{\alpha_{n+1}}P_{n+1}',p\subseteq P\}\nonumber\\
+\sum\{\tau.(P'\parallel P_{n+1}'): P\xrightarrow{l}P',P_{n+1}\xrightarrow{\overline{l}}P_{n+1}'\} \nonumber
\end{eqnarray}

Now with the induction assumption $P\equiv P_1\parallel\cdots\parallel P_n$, the right-hand side can be reformulated as follows.

\begin{eqnarray}
\{(\alpha_1\parallel\cdots\parallel \alpha_n\parallel \alpha_{n+1}).(P_1'\parallel\cdots\parallel P_n'\parallel P_{n+1}'): \nonumber\\
P_i\xrightarrow{\alpha_i}P_i',i\in\{1,\cdots,n+1\}\nonumber\\
+\sum\{\tau.(P_1\parallel\cdots\parallel P_i'\parallel\cdots\parallel P_j'\parallel\cdots\parallel P_n\parallel P_{n+1}): \nonumber\\
P_i\xrightarrow{l}P_i',P_j\xrightarrow{\overline{l}}P_j',i<j\} \nonumber\\
+\sum\{\tau.(P_1\parallel\cdots\parallel P_i'\parallel\cdots\parallel P_j\parallel\cdots\parallel P_n\parallel P_{n+1}'): \nonumber\\
P_i\xrightarrow{l}P_i',P_{n+1}\xrightarrow{\overline{l}}P_{n+1}',i\in\{1,\cdots, n\}\} \nonumber
\end{eqnarray}

So,

\begin{eqnarray}
R\sim_s \{(\alpha_1\parallel\cdots\parallel \alpha_n\parallel \alpha_{n+1}).(P_1'\parallel\cdots\parallel P_n'\parallel P_{n+1}'): \nonumber\\
P_i\xrightarrow{\alpha_i}P_i',i\in\{1,\cdots,n+1\}\nonumber\\
+\sum\{\tau.(P_1\parallel\cdots\parallel P_i'\parallel\cdots\parallel P_j'\parallel\cdots\parallel P_n): \nonumber\\
P_i\xrightarrow{l}P_i',P_j\xrightarrow{\overline{l}}P_j',1 \leq i<j\geq n+1\} \nonumber
\end{eqnarray}

Then, we can easily add the full conditions with Restriction and Relabeling.

\end{proof}

\begin{proposition}[New expansion law for strong pomset bisimulation]\label{NELSPB}
Let $P\equiv (P_1[f_1]\parallel\cdots\parallel P_n[f_n])\setminus L$, with $n\geq 1$. Then

\begin{eqnarray}
P\sim_p \{(f_1(\alpha_1)\parallel\cdots\parallel f_n(\alpha_n)).(P_1'[f_1]\parallel\cdots\parallel P_n'[f_n])\setminus L: \nonumber\\
P_i\xrightarrow{\alpha_i}P_i',i\in\{1,\cdots,n\},f_i(\alpha_i)\notin L\cup\overline{L}\} \nonumber\\
+\sum\{\tau.(P_1[f_1]\parallel\cdots\parallel P_i'[f_i]\parallel\cdots\parallel P_j'[f_j]\parallel\cdots\parallel P_n[f_n])\setminus L: \nonumber\\
P_i\xrightarrow{l_1}P_i',P_j\xrightarrow{l_2}P_j',f_i(l_1)=\overline{f_j(l_2)},i<j\} \nonumber
\end{eqnarray}
\end{proposition}

\begin{proof}
From the definition of strong pomset bisimulation (see Definition \ref{PSB}), we know that strong pomset bisimulation is defined by pomset transitions, which are labeled by pomsets. In a pomset transition, the events in the pomset are either within causality relations (defined by the prefix $.$) or in concurrency (implicitly defined by $.$ and $+$, and explicitly defined by $\parallel$), of course, they are pairwise consistent (without conflicts). In Proposition \ref{NELSSB}, we have already proven the case that all events are pairwise concurrent, so, we only need to prove the case of events in causality. Without loss of generality, we take a pomset of $p=\{\alpha,\beta:\alpha.\beta\}$. Then the pomset transition labeled by the above $p$ is just composed of one single event transition labeled by $\alpha$ succeeded by another single event transition labeled by $\beta$, that is, $\xrightarrow{p}=\xrightarrow{\alpha}\xrightarrow{\beta}$.

Similarly to the proof of new expansion law for strong step bisimulation (see Proposition \ref{NELSSB}), we can prove that the new expansion law holds for strong pomset bisimulation, we omit them.
\end{proof}

\begin{proposition}[New expansion law for strong hp-bisimulation]\label{NELSHPB}
Let $P\equiv (P_1[f_1]\parallel\cdots\parallel P_n[f_n])\setminus L$, with $n\geq 1$. Then

\begin{eqnarray}
P\sim_{hp} \{(f_1(\alpha_1)\parallel\cdots\parallel f_n(\alpha_n)).(P_1'[f_1]\parallel\cdots\parallel P_n'[f_n])\setminus L: \nonumber\\
P_i\xrightarrow{\alpha_i}P_i',i\in\{1,\cdots,n\},f_i(\alpha_i)\notin L\cup\overline{L}\} \nonumber\\
+\sum\{\tau.(P_1[f_1]\parallel\cdots\parallel P_i'[f_i]\parallel\cdots\parallel P_j'[f_j]\parallel\cdots\parallel P_n[f_n])\setminus L: \nonumber\\
P_i\xrightarrow{l_1}P_i',P_j\xrightarrow{l_2}P_j',f_i(l_1)=\overline{f_j(l_2)},i<j\} \nonumber
\end{eqnarray}
\end{proposition}

\begin{proof}
From the definition of strong hp-bisimulation (see Definition \ref{HHPB}), we know that strong hp-bisimulation is defined on the posetal product $(C_1,f,C_2),f:C_1\rightarrow C_2\textrm{ isomorphism}$. Two processes $P$ related to $C_1$ and $Q$ related to $C_2$, and $f:C_1\rightarrow C_2\textrm{ isomorphism}$. Initially, $(C_1,f,C_2)=(\emptyset,\emptyset,\emptyset)$, and $(\emptyset,\emptyset,\emptyset)\in\sim_{hp}$. When $P\xrightarrow{\alpha}P'$ ($C_1\xrightarrow{\alpha}C_1'$), there will be $Q\xrightarrow{\alpha}Q'$ ($C_2\xrightarrow{\alpha}C_2'$), and we define $f'=f[\alpha\mapsto \alpha]$. Then, if $(C_1,f,C_2)\in\sim_{hp}$, then $(C_1',f',C_2')\in\sim_{hp}$.

Similarly to the proof of new expansion law for strong pomset bisimulation (see Proposition \ref{NELSPB}), we can prove that the new expansion law holds for strong hp-bisimulation, we just need additionally to check the above conditions on hp-bisimulation, we omit them.
\end{proof}

\begin{proposition}[New expansion law for strongly hhp-bisimulation]\label{NELSHHPB}
Let $P\equiv (P_1[f_1]\parallel\cdots\parallel P_n[f_n])\setminus L$, with $n\geq 1$. Then

\begin{eqnarray}
P\sim_{hhp} \{(f_1(\alpha_1)\parallel\cdots\parallel f_n(\alpha_n)).(P_1'[f_1]\parallel\cdots\parallel P_n'[f_n])\setminus L: \nonumber\\
P_i\xrightarrow{\alpha_i}P_i',i\in\{1,\cdots,n\},f_i(\alpha_i)\notin L\cup\overline{L}\} \nonumber\\
+\sum\{\tau.(P_1[f_1]\parallel\cdots\parallel P_i'[f_i]\parallel\cdots\parallel P_j'[f_j]\parallel\cdots\parallel P_n[f_n])\setminus L: \nonumber\\
P_i\xrightarrow{l_1}P_i',P_j\xrightarrow{l_2}P_j',f_i(l_1)=\overline{f_j(l_2)},i<j\} \nonumber
\end{eqnarray}
\end{proposition}

\begin{proof}
From the definition of strongly hhp-bisimulation (see Definition \ref{HHPB}), we know that strongly hhp-bisimulation is downward closed for strong hp-bisimulation.

Similarly to the proof of the new expansion law for strong hp-bisimulation (see Proposition \ref{NELSHPB}), we can prove that the new expansion law holds for strongly hhp-bisimulation, that is, they are downward closed for strong hp-bisimulation, we omit them.
\end{proof}

\begin{theorem}[Congruence for strong step bisimulation] \label{CSSB}
We can enjoy the full congruence for strong step bisimulation as follows.
\begin{enumerate}
  \item If $A\overset{\text{def}}{=}P$, then $A\sim_s P$;
  \item Let $P_1\sim_s P_2$. Then
        \begin{enumerate}
           \item $\alpha.P_1\sim_s \alpha.P_2$;
           \item $(\alpha_1\parallel\cdots\parallel\alpha_n).P_1\sim_s (\alpha_1\parallel\cdots\parallel\alpha_n).P_2$;
           \item $P_1+Q\sim_s P_2 +Q$;
           \item $P_1\parallel Q\sim_s P_2\parallel Q$;
           \item $P_1\setminus L\sim_s P_2\setminus L$;
           \item $P_1[f]\sim_s P_2[f]$.
         \end{enumerate}
\end{enumerate}
\end{theorem}

\begin{proof}
Though transition rules in Table \ref{TRForCTC} are defined in the flavor of single event, they can be modified into a step (a set of events within which each event is pairwise concurrent), we omit them. If we treat a single event as a step containing just one event, the proof of the congruence does not exist any problem, so we use this way and still use the transition rules in Table \ref{TRForCTC}.

\begin{enumerate}
  \item If $A\overset{\text{def}}{=}P$, then $A\sim_s P$. It is obvious.
  \item Let $P_1\sim_s P_2$. Then
        \begin{enumerate}
           \item $\alpha.P_1\sim_s \alpha.P_2$. By the transition rules of $\textbf{Act}_{1,2}$ in Table \ref{TRForCTC}, we can get

           $$\alpha.P_1\xrightarrow{\alpha}P_1$$

           $$\alpha.P_2\xrightarrow{\alpha}P_2$$

           Since $P_1\sim_s P_2$, we get $\alpha.P_1\sim_s \alpha.P_2$, as desired.
           \item $(\alpha_1\parallel\cdots\parallel\alpha_n).P_1\sim_s (\alpha_1\parallel\cdots\parallel\alpha_n).P_2$. By the transition rules of $\textbf{Act}_{1,2}$ in Table \ref{TRForCTC}, we can get

           $$(\alpha_1\parallel\cdots\parallel\alpha_n).P_1\xrightarrow{\{\alpha_1,\cdots,\alpha_n\}}P_1$$

           $$(\alpha_1\parallel\cdots\parallel\alpha_n).P_2\xrightarrow{\{\alpha_1,\cdots,\alpha_n\}}P_2$$

           Since $P_1\sim_s P_2$, we get $(\alpha_1\parallel\cdots\parallel\alpha_n).P_1\sim_s (\alpha_1\parallel\cdots\parallel\alpha_n).P_2$, as desired.
           \item $P_1+Q\sim_s P_2 +Q$. By the transition rules of $\textbf{Sum}_{1,2}$ in Table \ref{TRForCTC}, we can get

           $$\frac{P_1\xrightarrow{\alpha}P_1'}{P_2\xrightarrow{\alpha}P_2'}(P_1'\sim_s P_2')$$

           $$\frac{P_1\xrightarrow{\alpha}P_1'}{P_1+Q\xrightarrow{\alpha}P_1'}
           \quad \frac{P_2\xrightarrow{\alpha}P_2'}{P_2+Q\xrightarrow{\alpha}P_2'}$$

           $$\frac{Q\xrightarrow{\beta}Q'}{P_1+Q\xrightarrow{\beta}Q'}
           \quad \frac{Q\xrightarrow{\beta}Q'}{P_2+Q\xrightarrow{\beta}Q'}$$

           Since $P_1'\sim_s P_2'$ and $Q'\sim_s Q'$, we get $P_1+Q\sim_s P_2+Q$, as desired.
           \item $P_1\parallel Q\sim_s P_2\parallel Q$. By the transition rules of $\textbf{Com}_{1,2,3,4}$ in Table \ref{TRForCTC}, we can get

           $$\frac{P_1\xrightarrow{\alpha}P_1'}{P_2\xrightarrow{\alpha}P_2'}(P_1'\sim_s P_2')$$

           $$\frac{P_1\xrightarrow{\alpha}P_1'\quad Q\nrightarrow}{P_1\parallel Q\xrightarrow{\alpha}P_1'\parallel Q}
           \quad \frac{P_2\xrightarrow{\alpha}P_2'\quad Q\nrightarrow}{P_2\parallel Q\xrightarrow{\alpha}P_2'\parallel Q}$$

           $$\frac{Q\xrightarrow{\beta}Q'\quad P_1\nrightarrow}{P_1\parallel Q\xrightarrow{\beta}P_1\parallel Q'}
           \quad \frac{Q\xrightarrow{\beta}P_2'\quad P_2\nrightarrow}{P_2\parallel Q\xrightarrow{\beta}P_2\parallel Q'}$$

           $$\frac{P_1\xrightarrow{\alpha}P_1'\quad Q\xrightarrow{\beta}Q'}{P_1\parallel Q\xrightarrow{\{\alpha,\beta\}}P_1'\parallel Q'}(\beta\neq\overline{\alpha})
           \quad \frac{P_2\xrightarrow{\alpha}P_2'\quad Q\xrightarrow{\beta}Q'}{P_2\parallel Q\xrightarrow{\{\alpha,\beta\}}P_2'\parallel Q'}(\beta\neq\overline{\alpha})$$

           $$\frac{P_1\xrightarrow{l}P_1'\quad Q\xrightarrow{\overline{l}}Q'}{P_1\parallel Q\xrightarrow{\tau}P_1'\parallel Q'}
           \quad \frac{P_2\xrightarrow{l}P_2'\quad Q\xrightarrow{\overline{l}}Q'}{P_2\parallel Q\xrightarrow{\tau}P_2'\parallel Q'}$$

           Since $P_1'\sim_s P_2'$ and $Q'\sim_s Q'$, and with the assumptions $P_1'\parallel Q\sim_s P_2'\parallel Q$, $P_1\parallel Q'\sim_s P_2\parallel Q'$ and $P_1'\parallel Q'\sim_s P_2'\parallel Q'$, we get $P_1\parallel Q\sim_s P_2\parallel Q$, as desired.
           \item $P_1\setminus L\sim_s P_2\setminus L$. By the transition rules of $\textbf{Res}_{1,2}$ in Table \ref{TRForCTC}, we get

           $$\frac{P_1\xrightarrow{\alpha}P_1'}{P_2\xrightarrow{\alpha}P_2'}(P_1'\sim_s P_2')$$

           $$\frac{P_1\xrightarrow{\alpha}P_1'}{P_1\setminus L\xrightarrow{\alpha}P_1'\setminus L}$$

           $$\frac{P_2\xrightarrow{\alpha}P_2'}{P_2\setminus L\xrightarrow{\alpha}P_2'\setminus L}$$

           Since $P_1'\sim_s P_2'$, and with the assumption $P_1'\setminus L\sim_s P_2'\setminus L$, we get $P_1\setminus L\sim_s P_2\setminus L$, as desired.
           \item $P_1[f]\sim_s P_2[f]$. By the transition rules of $\textbf{Rel}_{1,2}$ in Table \ref{TRForCTC}, we get

           $$\frac{P_1\xrightarrow{\alpha}P_1'}{P_2\xrightarrow{\alpha}P_2'}(P_1'\sim_s P_2')$$

           $$\frac{P_1\xrightarrow{\alpha}P_1'}{P_1[f]\xrightarrow{f(\alpha)}P_1'[f]}$$

           $$\frac{P_2\xrightarrow{\alpha}P_2'}{P_2[f]\xrightarrow{f(\alpha)}P_2'[f]}$$

           Since $P_1'\sim_s P_2'$, and with the assumption $P_1'[f]\sim_s P_2'[f]$, we get $P_1[f]\sim_s P_2[f]$, as desired.
         \end{enumerate}
\end{enumerate}
\end{proof}

\begin{theorem}[Congruence for strong pomset bisimulation] \label{CSPB}
We can enjoy the full congruence for strong pomset bisimulation as follows.
\begin{enumerate}
  \item If $A\overset{\text{def}}{=}P$, then $A\sim_p P$;
  \item Let $P_1\sim_p P_2$. Then
        \begin{enumerate}
           \item $\alpha.P_1\sim_p \alpha.P_2$;
           \item $(\alpha_1\parallel\cdots\parallel\alpha_n).P_1\sim_p (\alpha_1\parallel\cdots\parallel\alpha_n).P_2$;
           \item $P_1+Q\sim_p P_2 +Q$;
           \item $P_1\parallel Q\sim_p P_2\parallel Q$;
           \item $P_1\setminus L\sim_p P_2\setminus L$;
           \item $P_1[f]\sim_p P_2[f]$.
         \end{enumerate}
\end{enumerate}
\end{theorem}

\begin{proof}
From the definition of strong pomset bisimulation (see Definition \ref{PSB}), we know that strong pomset bisimulation is defined by pomset transitions, which are labeled by pomsets. In a pomset transition, the events in the pomset are either within causality relations (defined by the prefix $.$) or in concurrency (implicitly defined by $.$ and $+$, and explicitly defined by $\parallel$), of course, they are pairwise consistent (without conflicts). In Theorem \ref{CSSB}, we have already proven the case that all events are pairwise concurrent, so, we only need to prove the case of events in causality. Without loss of generality, we take a pomset of $p=\{\alpha,\beta:\alpha.\beta\}$. Then the pomset transition labeled by the above $p$ is just composed of one single event transition labeled by $\alpha$ succeeded by another single event transition labeled by $\beta$, that is, $\xrightarrow{p}=\xrightarrow{\alpha}\xrightarrow{\beta}$.

Similarly to the proof of congruence for strong step bisimulation (see Theorem \ref{CSSB}), we can prove that the congruence holds for strong pomset bisimulation, we omit them.
\end{proof}

\begin{theorem}[Congruence for strong hp-bisimulation] \label{CSHPB}
We can enjoy the full congruence for strong hp-bisimulation as follows.
\begin{enumerate}
  \item If $A\overset{\text{def}}{=}P$, then $A\sim_{hp} P$;
  \item Let $P_1\sim_{hp} P_2$. Then
        \begin{enumerate}
           \item $\alpha.P_1\sim_{hp} \alpha.P_2$;
           \item $(\alpha_1\parallel\cdots\parallel\alpha_n).P_1\sim_{hp} (\alpha_1\parallel\cdots\parallel\alpha_n).P_2$;
           \item $P_1+Q\sim_{hp} P_2 +Q$;
           \item $P_1\parallel Q\sim_{hp} P_2\parallel Q$;
           \item $P_1\setminus L\sim_{hp} P_2\setminus L$;
           \item $P_1[f]\sim_{hp} P_2[f]$.
         \end{enumerate}
\end{enumerate}
\end{theorem}

\begin{proof}
From the definition of strong hp-bisimulation (see Definition \ref{HHPB}), we know that strong hp-bisimulation is defined on the posetal product $(C_1,f,C_2),f:C_1\rightarrow C_2\textrm{ isomorphism}$. Two processes $P$ related to $C_1$ and $Q$ related to $C_2$, and $f:C_1\rightarrow C_2\textrm{ isomorphism}$. Initially, $(C_1,f,C_2)=(\emptyset,\emptyset,\emptyset)$, and $(\emptyset,\emptyset,\emptyset)\in\sim_{hp}$. When $P\xrightarrow{\alpha}P'$ ($C_1\xrightarrow{\alpha}C_1'$), there will be $Q\xrightarrow{\alpha}Q'$ ($C_2\xrightarrow{\alpha}C_2'$), and we define $f'=f[\alpha\mapsto \alpha]$. Then, if $(C_1,f,C_2)\in\sim_{hp}$, then $(C_1',f',C_2')\in\sim_{hp}$.

Similarly to the proof of congruence for strong pomset bisimulation (see Theorem \ref{CSPB}), we can prove that the congruence holds for strong hp-bisimulation, we just need additionally to check the above conditions on hp-bisimulation, we omit them.
\end{proof}

\begin{theorem}[Congruence for strongly hhp-bisimulation] \label{CSHHPB}
We can enjoy the full congruence for strongly hhp-bisimulation as follows.
\begin{enumerate}
  \item If $A\overset{\text{def}}{=}P$, then $A\sim_{hhp} P$;
  \item Let $P_1\sim_{hhp} P_2$. Then
        \begin{enumerate}
           \item $\alpha.P_1\sim_{hhp} \alpha.P_2$;
           \item $(\alpha_1\parallel\cdots\parallel\alpha_n).P_1\sim_{hhp} (\alpha_1\parallel\cdots\parallel\alpha_n).P_2$;
           \item $P_1+Q\sim_{hhp} P_2 +Q$;
           \item $P_1\parallel Q\sim_{hhp} P_2\parallel Q$;
           \item $P_1\setminus L\sim_{hhp} P_2\setminus L$;
           \item $P_1[f]\sim_{hhp} P_2[f]$.
         \end{enumerate}
\end{enumerate}
\end{theorem}

\begin{proof}
From the definition of strongly hhp-bisimulation (see Definition \ref{HHPB}), we know that strongly hhp-bisimulation is downward closed for strong hp-bisimulation.

Similarly to the proof of congruence for strong hp-bisimulation (see Theorem \ref{CSHPB}), we can prove that the congruence holds for strongly hhp-bisimulation, we omit them.
\end{proof}

\subsection{Recursion}

\begin{definition}[Weakly guarded recursive expression]
$X$ is weakly guarded in $E$ if each occurrence of $X$ is with some subexpression $\alpha.F$ or $(\alpha_1\parallel\cdots\parallel\alpha_n).F$ of $E$.
\end{definition}

\begin{lemma}\label{LUS}
If the variables $\widetilde{X}$ are weakly guarded in $E$, and $E\{\widetilde{P}/\widetilde{X}\}\xrightarrow{\{\alpha_1,\cdots,\alpha_n\}}P'$, then $P'$ takes the form $E'\{\widetilde{P}/\widetilde{X}\}$ for some expression $E'$, and moreover, for any $\widetilde{Q}$, $E\{\widetilde{Q}/\widetilde{X}\}\xrightarrow{\{\alpha_1,\cdots,\alpha_n\}}E'\{\widetilde{Q}/\widetilde{X}\}$.
\end{lemma}

\begin{proof}
It needs to induct on the depth of the inference of $E\{\widetilde{P}/\widetilde{X}\}\xrightarrow{\{\alpha_1,\cdots,\alpha_n\}}P'$.

\begin{enumerate}
  \item Case $E\equiv Y$, a variable. Then $Y\notin \widetilde{X}$. Since $\widetilde{X}$ are weakly guarded, $Y\{\widetilde{P}/\widetilde{X}\equiv Y\}\nrightarrow$, this case is impossible.
  \item Case $E\equiv\beta.F$. Then we must have $\alpha=\beta$, and $P'\equiv F\{\widetilde{P}/\widetilde{X}\}$, and $E\{\widetilde{Q}/\widetilde{X}\}\equiv \beta.F\{\widetilde{Q}/\widetilde{X}\} \xrightarrow{\beta}F\{\widetilde{Q}/\widetilde{X}\}$, then, let $E'$ be $F$, as desired.
  \item Case $E\equiv(\beta_1\parallel\cdots\parallel\beta_n).F$. Then we must have $\alpha_i=\beta_i$ for $1\leq i\leq n$, and $P'\equiv F\{\widetilde{P}/\widetilde{X}\}$, and $E\{\widetilde{Q}/\widetilde{X}\}\equiv (\beta_1\parallel\cdots\parallel\beta_n).F\{\widetilde{Q}/\widetilde{X}\} \xrightarrow{\{\beta_1,\cdots,\beta_n\}}F\{\widetilde{Q}/\widetilde{X}\}$, then, let $E'$ be $F$, as desired.
  \item Case $E\equiv E_1+E_2$. Then either $E_1\{\widetilde{P}/\widetilde{X}\} \xrightarrow{\{\alpha_1,\cdots,\alpha_n\}}P'$ or $E_2\{\widetilde{P}/\widetilde{X}\} \xrightarrow{\{\alpha_1,\cdots,\alpha_n\}}P'$, then, we can apply this lemma in either case, as desired.
  \item Case $E\equiv E_1\parallel E_2$. There are four possibilities.
  \begin{enumerate}
    \item We may have $E_1\{\widetilde{P}/\widetilde{X}\} \xrightarrow{\alpha}P_1'$ and $E_2\{\widetilde{P}/\widetilde{X}\}\nrightarrow$ with $P'\equiv P_1'\parallel (E_2\{\widetilde{P}/\widetilde{X}\})$, then by applying this lemma, $P_1'$ is of the form $E_1'\{\widetilde{P}/\widetilde{X}\}$, and for any $Q$, $E_1\{\widetilde{Q}/\widetilde{X}\}\xrightarrow{\alpha} E_1'\{\widetilde{Q}/\widetilde{X}\}$. So, $P'$ is of the form $E_1'\parallel E_2\{\widetilde{P}/\widetilde{X}\}$, and for any $Q$, $E\{\widetilde{Q}/\widetilde{X}\}\equiv E_1\{\widetilde{Q}/\widetilde{X}\}\parallel E_2\{\widetilde{Q}/\widetilde{X}\}\xrightarrow{\alpha} (E_1'\parallel E_2)\{\widetilde{Q}/\widetilde{X}\}$, then, let $E'$ be $E_1'\parallel E_2$, as desired.
    \item We may have $E_2\{\widetilde{P}/\widetilde{X}\} \xrightarrow{\alpha}P_2'$ and $E_1\{\widetilde{P}/\widetilde{X}\}\nrightarrow$ with $P'\equiv P_2'\parallel (E_1\{\widetilde{P}/\widetilde{X}\})$, this case can be prove similarly to the above subcase, as desired.
    \item We may have $E_1\{\widetilde{P}/\widetilde{X}\} \xrightarrow{\alpha}P_1'$ and $E_2\{\widetilde{P}/\widetilde{X}\}\xrightarrow{\beta}P_2'$ with $\alpha\neq\overline{\beta}$ and $P'\equiv P_1'\parallel P_2'$, then by applying this lemma, $P_1'$ is of the form $E_1'\{\widetilde{P}/\widetilde{X}\}$, and for any $Q$, $E_1\{\widetilde{Q}/\widetilde{X}\}\xrightarrow{\alpha} E_1'\{\widetilde{Q}/\widetilde{X}\}$; $P_2'$ is of the form $E_2'\{\widetilde{P}/\widetilde{X}\}$, and for any $Q$, $E_2\{\widetilde{Q}/\widetilde{X}\}\xrightarrow{\alpha} E_2'\{\widetilde{Q}/\widetilde{X}\}$. So, $P'$ is of the form $E_1'\parallel E_2'\{\widetilde{P}/\widetilde{X}\}$, and for any $Q$, $E\{\widetilde{Q}/\widetilde{X}\}\equiv E_1\{\widetilde{Q}/\widetilde{X}\}\parallel E_2\{\widetilde{Q}/\widetilde{X}\}\xrightarrow{\{\alpha,\beta\}} (E_1'\parallel E_2')\{\widetilde{Q}/\widetilde{X}\}$, then, let $E'$ be $E_1'\parallel E_2'$, as desired.
    \item We may have $E_1\{\widetilde{P}/\widetilde{X}\} \xrightarrow{l}P_1'$ and $E_2\{\widetilde{P}/\widetilde{X}\}\xrightarrow{\overline{l}}P_2'$ with $P'\equiv P_1'\parallel P_2'$, then by applying this lemma, $P_1'$ is of the form $E_1'\{\widetilde{P}/\widetilde{X}\}$, and for any $Q$, $E_1\{\widetilde{Q}/\widetilde{X}\}\xrightarrow{l} E_1'\{\widetilde{Q}/\widetilde{X}\}$; $P_2'$ is of the form $E_2'\{\widetilde{P}/\widetilde{X}\}$, and for any $Q$, $E_2\{\widetilde{Q}/\widetilde{X}\}\xrightarrow{\overline{l}} E_2'\{\widetilde{Q}/\widetilde{X}\}$. So, $P'$ is of the form $E_1'\parallel E_2'\{\widetilde{P}/\widetilde{X}\}$, and for any $Q$, $E\{\widetilde{Q}/\widetilde{X}\}\equiv E_1\{\widetilde{Q}/\widetilde{X}\}\parallel E_2\{\widetilde{Q}/\widetilde{X}\}\xrightarrow{\tau} (E_1'\parallel E_2')\{\widetilde{Q}/\widetilde{X}\}$, then, let $E'$ be $E_1'\parallel E_2'$, as desired.
  \end{enumerate}
  \item Case $E\equiv F[R]$ and $E\equiv F\setminus L$. These cases can be prove similarly to the above case.
  \item Case $E\equiv C$, an agent constant defined by $C\overset{\text{def}}{=}R$. Then there is no $X\in\widetilde{X}$ occurring in $E$, so $C\xrightarrow{\{\alpha_1,\cdots,\alpha_n\}}P'$, let $E'$ be $P'$, as desired.
\end{enumerate}
\end{proof}

\begin{theorem}[Unique solution of equations for strong step bisimulation]\label{USSSB}
Let the recursive expressions $E_i(i\in I)$ contain at most the variables $X_i(i\in I)$, and let each $X_j(j\in I)$ be weakly guarded in each $E_i$. Then,

If $\widetilde{P}\sim_s \widetilde{E}\{\widetilde{P}/\widetilde{X}\}$ and $\widetilde{Q}\sim_s \widetilde{E}\{\widetilde{Q}/\widetilde{X}\}$, then $\widetilde{P}\sim_s \widetilde{Q}$.
\end{theorem}

\begin{proof}
It is sufficient to induct on the depth of the inference of $E\{\widetilde{P}/\widetilde{X}\}\xrightarrow{\{\alpha_1,\cdots,\alpha_n\}}P'$.

\begin{enumerate}
  \item Case $E\equiv X_i$. Then we have $E\{\widetilde{P}/\widetilde{X}\}\equiv P_i\xrightarrow{\{\alpha_1,\cdots,\alpha_n\}}P'$, since $P_i\sim_s E_i\{\widetilde{P}/\widetilde{X}\}$, we have $E_i\{\widetilde{P}/\widetilde{X}\}\xrightarrow{\{\alpha_1,\cdots,\alpha_n\}}P''\sim_s P'$. Since $\widetilde{X}$ are weakly guarded in $E_i$, by Lemma \ref{LUS}, $P''\equiv E'\{\widetilde{P}/\widetilde{X}\}$ and $E_i\{\widetilde{P}/\widetilde{X}\}\xrightarrow{\{\alpha_1,\cdots,\alpha_n\}} E'\{\widetilde{P}/\widetilde{X}\}$. Since $E\{\widetilde{Q}/\widetilde{X}\}\equiv X_i\{\widetilde{Q}/\widetilde{X}\} \equiv Q_i\sim_s E_i\{\widetilde{Q}/\widetilde{X}\}$, $E\{\widetilde{Q}/\widetilde{X}\}\xrightarrow{\{\alpha_1,\cdots,\alpha_n\}}Q'\sim_s E'\{\widetilde{Q}/\widetilde{X}\}$. So, $P'\sim_s Q'$, as desired.
  \item Case $E\equiv\alpha.F$. This case can be proven similarly.
  \item Case $E\equiv(\alpha_1\parallel\cdots\parallel\alpha_n).F$. This case can be proven similarly.
  \item Case $E\equiv E_1+E_2$. We have $E_i\{\widetilde{P}/\widetilde{X}\} \xrightarrow{\{\alpha_1,\cdots,\alpha_n\}}P'$, $E_i\{\widetilde{Q}/\widetilde{X}\} \xrightarrow{\{\alpha_1,\cdots,\alpha_n\}}Q'$, then, $P'\sim_s Q'$, as desired.
  \item Case $E\equiv E_1\parallel E_2$, $E\equiv F[R]$ and $E\equiv F\setminus L$, $E\equiv C$. These cases can be prove similarly to the above case.
\end{enumerate}
\end{proof}

\begin{theorem}[Unique solution of equations for strong pomset bisimulation]\label{USSPB}
Let the recursive expressions $E_i(i\in I)$ contain at most the variables $X_i(i\in I)$, and let each $X_j(j\in I)$ be weakly guarded in each $E_i$. Then,

If $\widetilde{P}\sim_p \widetilde{E}\{\widetilde{P}/\widetilde{X}\}$ and $\widetilde{Q}\sim_p \widetilde{E}\{\widetilde{Q}/\widetilde{X}\}$, then $\widetilde{P}\sim_p \widetilde{Q}$.
\end{theorem}

\begin{proof}
From the definition of strong pomset bisimulation (see Definition \ref{PSB}), we know that strong pomset bisimulation is defined by pomset transitions, which are labeled by pomsets. In a pomset transition, the events in the pomset are either within causality relations (defined by the prefix $.$) or in concurrency (implicitly defined by $.$ and $+$, and explicitly defined by $\parallel$), of course, they are pairwise consistent (without conflicts). In Theorem \ref{USSSB}, we have already proven the case that all events are pairwise concurrent, so, we only need to prove the case of events in causality. Without loss of generality, we take a pomset of $p=\{\alpha,\beta:\alpha.\beta\}$. Then the pomset transition labeled by the above $p$ is just composed of one single event transition labeled by $\alpha$ succeeded by another single event transition labeled by $\beta$, that is, $\xrightarrow{p}=\xrightarrow{\alpha}\xrightarrow{\beta}$.

Similarly to the proof of unique solution of equations for strong step bisimulation (see Theorem \ref{USSSB}), we can prove that the unique solution of equations holds for strong pomset bisimulation, we omit them.
\end{proof}

\begin{theorem}[Unique solution of equations for strong hp-bisimulation]\label{USSHPB}
Let the recursive expressions $E_i(i\in I)$ contain at most the variables $X_i(i\in I)$, and let each $X_j(j\in I)$ be weakly guarded in each $E_i$. Then,

If $\widetilde{P}\sim_{hp} \widetilde{E}\{\widetilde{P}/\widetilde{X}\}$ and $\widetilde{Q}\sim_{hp} \widetilde{E}\{\widetilde{Q}/\widetilde{X}\}$, then $\widetilde{P}\sim_{hp} \widetilde{Q}$.
\end{theorem}

\begin{proof}
From the definition of strong hp-bisimulation (see Definition \ref{HHPB}), we know that strong hp-bisimulation is defined on the posetal product $(C_1,f,C_2),f:C_1\rightarrow C_2\textrm{ isomorphism}$. Two processes $P$ related to $C_1$ and $Q$ related to $C_2$, and $f:C_1\rightarrow C_2\textrm{ isomorphism}$. Initially, $(C_1,f,C_2)=(\emptyset,\emptyset,\emptyset)$, and $(\emptyset,\emptyset,\emptyset)\in\sim_{hp}$. When $P\xrightarrow{\alpha}P'$ ($C_1\xrightarrow{\alpha}C_1'$), there will be $Q\xrightarrow{\alpha}Q'$ ($C_2\xrightarrow{\alpha}C_2'$), and we define $f'=f[\alpha\mapsto \alpha]$. Then, if $(C_1,f,C_2)\in\sim_{hp}$, then $(C_1',f',C_2')\in\sim_{hp}$.

Similarly to the proof of unique solution of equations for strong pomset bisimulation (see Theorem \ref{USSPB}), we can prove that the unique solution of equations holds for strong hp-bisimulation, we just need additionally to check the above conditions on hp-bisimulation, we omit them.
\end{proof}

\begin{theorem}[Unique solution of equations for strongly hhp-bisimulation]\label{USSHHPB}
Let the recursive expressions $E_i(i\in I)$ contain at most the variables $X_i(i\in I)$, and let each $X_j(j\in I)$ be weakly guarded in each $E_i$. Then,

If $\widetilde{P}\sim_{hhp} \widetilde{E}\{\widetilde{P}/\widetilde{X}\}$ and $\widetilde{Q}\sim_{hhp} \widetilde{E}\{\widetilde{Q}/\widetilde{X}\}$, then $\widetilde{P}\sim_{hhp} \widetilde{Q}$.
\end{theorem}

\begin{proof}
From the definition of strongly hhp-bisimulation (see Definition \ref{HHPB}), we know that strongly hhp-bisimulation is downward closed for strong hp-bisimulation.

Similarly to the proof of unique solution of equations for strong hp-bisimulation (see Theorem \ref{USSHPB}), we can prove that the unique solution of equations holds for strongly hhp-bisimulation, we omit them.
\end{proof}
