% This is file FAC2egui.tex
% v2.00, 13th July 2000
% (based on FACguide.tex v1.12)

% Copyright (C) 1999,2000 Cambridge University Press

\NeedsTeXFormat{LaTeX2e}

% The following saves the original definitions of \geq and \leq (guide only).
\let\realgeq\geq
\let\realleq\leq

\documentclass{fac}

\usepackage{graphicx}
\usepackage{amssymb}
\usepackage{amsfonts}
\usepackage{amsmath}
\usepackage{dsfont}
\usepackage{MnSymbol}
\usepackage{mathtools}
\usepackage{epsfig}
\usepackage{epstopdf}


\newtheorem{theorem}{Theorem}[section]
\newtheorem{definition}[theorem]{Definition}
\newtheorem{proposition}[theorem]{Proposition}
\newtheorem{lemma}[theorem]{Lemma}

\title[Draft of A Calculus for True Concurrency]
      {A Calculus for True Concurrency}

\author[Yong Wang]
    {Yong Wang\\
     College of Computer Science and Technology,\\
     Faculty of Information Technology,\\
     Beijing University of Technology, Beijing, China\\
     }

\correspond{Yong Wang, Pingleyuan 100, Chaoyang District, Beijing, China.
            e-mail: wangy@bjut.edu.cn}

\pubyear{2017}
\pagerange{\pageref{firstpage}--\pageref{lastpage}}

\begin{document}
\label{firstpage}

\makecorrespond

\maketitle

\begin{abstract}
We design a calculus for true concurrency called CTC, including its syntax and operational semantics. CTC has good properties modulo several kinds of strongly truly concurrent bisimulations and weakly truly concurrent bisimulations, such as monoid laws, static laws, new expansion law for strongly truly concurrent bisimulations, $\tau$ laws for weakly truly concurrent bisimulations, and full congruences for strongly and weakly truly concurrent bisimulations, and also unique solution for recursion.
\end{abstract}

\begin{keywords}
True Concurrency; Behaviorial Equivalence; Prime Event Structure; Calculus
\end{keywords}

\section{Introduction}\label{int}

Parallelism and concurrency \cite{CM} are the core concepts within computer science. There are mainly two camps in capturing concurrency: the interleaving concurrency and the true concurrency.

The representative of interleaving concurrency is bisimulation/weak bisimulation equivalences. CCS (A Calculus of Communicating Systems) \cite{CCS} \cite{CC} is a calculus based on bisimulation semantics model. CCS has good semantic properties based on the interleaving bisimulation. These properties include monoid laws, static laws, new expansion law for strongly interleaving bisimulation, $\tau$ laws for weakly interleaving bisimulation, and full congruences for strongly and weakly interleaving bisimulations, and also unique solution for recursion.

The other camp of concurrency is true concurrency. The researches on true concurrency are still active. Firstly, there are several truly concurrent bisimulations, the representatives are: pomset bisimulation, step bisimulation, history-preserving (hp-) bisimulation, and especially hereditary history-preserving (hhp-) bisimulation \cite{HHP1} \cite{HHP2}. These truly concurrent bisimulations are studied in different structures \cite{ES1} \cite{ES2} \cite{CM}: Petri nets, event structures, domains, and also a uniform form called TSI (Transition System with Independence) \cite{SFL}. There are also several logics based on different truly concurrent bisimulation equivalences, for example, SFL (Separation Fixpoint Logic) and TFL (Trace Fixpoint Logic) \cite{SFL} are extensions on true concurrency of mu-calculi \cite{MUC} on bisimulation equivalence, and also a logic with reverse modalities \cite {RL1} \cite{RL2} based on the so-called reverse bisimulations with a reverse flavor. Recently, a uniform logic for true concurrency \cite{LTC1} \cite{LTC2} was represented, which used a logical framework to unify several truly concurrent bisimulations, including pomset bisimulation, step bisimulation, hp-bisimulation and hhp-bisimulation.

There are simple comparisons between HM logic and bisimulation, as the uniform logic \cite{LTC1} \cite{LTC2} and truly concurrent bisimulations; the algebraic laws \cite{ALNC}, ACP \cite{ACP} and bisimulation, as the algebraic laws APTC \cite{ATC} and truly concurrent bisimulations; CCS and bisimulation, as truly concurrent bisimulations  and \emph{what}, which is still missing.

In this paper, we design a calculus for true concurrency (CTC) following the way paved by CCS for bisimulation equivalence. This paper is organized as follows. In section \ref{bac}, we introduce some preliminaries, including a brief introduction to CCS, and also preliminaries on true concurrency. We introduce the syntax and operational semantics of CTC in section \ref{sos}, its properties for strongly truly concurrent bisimulations in section \ref{stcb}, its properties for weakly truly concurrent bisimulations in section \ref{wtcb}. In section \ref{app}, we show the applications of CTC by an example called alternating-bit protocol. Finally, in section \ref{con}, we conclude this paper. 

%\documentclass[preprint,12pt]{elsarticle}
%\if0
\usepackage{amssymb}
\usepackage{mathtools}
%\usepackage[dvipdfmx]{graphicx}
\usepackage{cite}
\usepackage{graphicx}
\usepackage{bm}
\usepackage{here}
\usepackage[subrefformat=parens]{subcaption}
\fi
%\usepackage{amssymb}
\usepackage{amsmath}
\usepackage[dvipdfmx]{}
\usepackage[dvipdfmx]{color}
%\usepackage{cite}
%\usepackage{upgreek}
\usepackage{url}
%\usepackage[dvipdfmx]{hyperref}
%\usepackage{pxjahyper}
%\usepackage {hyperref}
\usepackage{graphicx}
\usepackage{bm}
\usepackage{here}
\usepackage{caption}
\usepackage[subrefformat=parens]{subcaption}
\captionsetup{compatibility=false}

%% The amsthm package provides extended theorem environments
%% \usepackage{amsthm}

%% The lineno packages adds line numbers. Start line numbering with
%% \begin{linenumbers}, end it with \end{linenumbers}. Or switch it on
%% for the whole article with \linenumbers after \end{frontmatter}.
%% \usepackage{lineno}

%% natbib.sty is loaded by default. However, natbib options can be
%% provided with \biboptions{...} command. Following options are
%% valid:

%%   round  -  round parentheses are used (default)
%%   square -  square brackets are used   [option]
%%   curly  -  curly braces are used      {option}
%%   angle  -  angle brackets are used    <option>
%%   semicolon  -  multiple citations separated by semi-colon
%%   colon  - same as semicolon, an earlier confusion
%%   comma  -  separated by comma
%%   numbers-  selects numerical citations
%%   super  -  numerical citations as superscripts
%%   sort   -  sorts multiple citations according to order in ref. list
%%   sort&compress   -  like sort, but also compresses numerical citations
%%   compress - compresses without sorting
%%
%% \biboptions{comma,round}

% \biboptions{}

%% This list environment is used for the references in the
%% Program Summary
%%
\newcounter{bla}
\newenvironment{refnummer}{%
\list{[\arabic{bla}]}%
{\usecounter{bla}%
 \setlength{\itemindent}{0pt}%
 \setlength{\topsep}{0pt}%
 \setlength{\itemsep}{0pt}%
 \setlength{\labelsep}{2pt}%
 \setlength{\listparindent}{0pt}%
 \settowidth{\labelwidth}{[9]}%
 \setlength{\leftmargin}{\labelwidth}%
 \addtolength{\leftmargin}{\labelsep}%
 \setlength{\rightmargin}{0pt}}}
 {\endlist}
\begin{document}

\section{O-SUKI-N 3D code algorithm description}
\par

\subsection{O-SUKI-N 3D code structure}
     The O-SUKI-N 3D code system consists of three parts: The Lagrangian fluid code \cite{Schulz}, the data conversion code from the Lagrangian code to the Euler code, and Euler code. The fluid model is the three-temperature model in Ref. \cite{Tahir}. The Lagrangian fluid code, the data conversion code and the Euler code are described below in detail. 
     
     In the Lagrangian fluid code the spatial meshes move together with the fluid motion \cite{Schulz}. However, the Lagrange meshes can not follow the fluid large deformation. On the other hand, the Euler meshes are fixed to the space, and the fluid moves through the meshes. Therefore, just before the void closure time, that is, the stagnation phase, the Lagrangian code is used to simulate the DT fuel implosion. After the void closure time, the Euler code is employed to simulate the DT fuel further compression, ignition and burning. Between the Lagrangian code and the Euler code the data should be converted by the data conversion code. 

	All the simulation process is performed in its integrated way by using the script of "CodeO-SUKI-N-fusion-start.sh". The processes executed by this shell script are as follows: \\
1. Make the stack size infinite.\\
2. Remove all output data file and make the new output files.\\
3. Change the permission of shell scripts to executable. \\
4. Compile the main function of the Lagrangian code and execute it.\\
5. If any problems do not appear during the calculation of the Lagrangian code, compile the main function of the data conversion code and execute it.\\
6. If there is no problem during the data conversion, compile the main function of the Euler code and execute it.\\
     

\subsection{Steps in Lagrangian code}\par
     The Lagrangian code has the following steps: 

\begin{enumerate}
\item Initialize the variables and calculation of total input energy. \par
\item Calculation of time step size.\par
\item Calculation of coordinates.\par
\item Solve equation of motion. \par
\item Solve density by equation of continuity.\par
\item Calculation of artificial viscosity.\par
\item Transfer the data to the OK3. \par
\item Calculation of energy deposition distribution in code OK3. For details of the OK3, see the refs.\cite{ogoyski1,ogoyski2,ogoyski3}. \par
\item Solve energy equations\par
\item Calculation of heat conduction\par
\item Calculation of temperature relaxation among three temperatures.\par
\item Solve equation of state\par
\item Save the results.\par
\item End the Lagrangian calculation right before the void closure.\par
\item Transfer the data to converting code. \par
\end {enumerate}


\subsection{Data Conversion code from Lagrangian fluid code to Euler fluid code}

\begin {enumerate}
\item Read variables saved in Lagrangian code.\par
\item Generate the Eulerian mesh.\par
\item Calculate the interpolation of the physical quantity to them on the Eulerian mesh.\par
\item Write the converted data to the Eulerian code.\par
\end {enumerate}


\subsection{Steps in Eulerian code}

\begin {enumerate}
\item Read the mesh number from the converted data and define the each matrices.\par
\item Initialize the variables.\par
\item Calculation of time step size.\par
\item Solve equation of motion. \par
\item Track the material boundaries of DT, Al and Pb.\par
\item Linearly interpolate the boundary lines and transcribe them on the Eulerian code. \par
\item Discriminate the materials by using the transferred boundary line. \par
\item Solve density by equation of continuity.\par
\item Calculate artificial viscosity.\par
\item Solve energy equations\par
\item Calculation of fusion reaction.\par
\item Calculation of heat conduction\par
\item Calculation of temperature relaxation among three temperatures.\par
\item Solve equation of state.\par
\item Save the results.\par
\item End.
\end{enumerate}

%\include{end}

\section{Syntax and Operational Semantics}\label{sos}

We assume an infinite set $\mathcal{N}$ of (action or event) names, and use $a,b,c,\cdots$ to range over $\mathcal{N}$. We denote by $\overline{\mathcal{N}}$ the set of co-names and let $\overline{a},\overline{b},\overline{c},\cdots$ range over $\overline{\mathcal{N}}$. Then we set $\mathcal{L}=\mathcal{N}\cup\overline{\mathcal{N}}$ as the set of labels, and use $l,\overline{l}$ to range over $\mathcal{L}$. We extend complementation to $\mathcal{L}$ such that $\overline{\overline{a}}=a$. Let $\tau$ denote the silent step (internal action or event) and define $Act=\mathcal{L}\cup\{\tau\}$ to be the set of actions, $\alpha,\beta$ range over $Act$. And $K,L$ are used to stand for subsets of $\mathcal{L}$ and $\overline{L}$ is used for the set of complements of labels in $L$. A relabelling function $f$ is a function from $\mathcal{L}$ to $\mathcal{L}$ such that $f(\overline{l})=\overline{f(l)}$. By defining $f(\tau)=\tau$, we extend $f$ to $Act$.

Further, we introduce a set $\mathcal{X}$ of process variables, and a set $\mathcal{K}$ of process constants, and let $X,Y,\cdots$ range over $\mathcal{X}$, and $A,B,\cdots$ range over $\mathcal{K}$, $\widetilde{X}$ is a tuple of distinct process variables, and also $E,F,\cdots$ range over the recursive expressions. We write $\mathcal{P}$ for the set of processes. Sometimes, we use $I,J$ to stand for an indexing set, and we write $E_i:i\in I$ for a family of expressions indexed by $I$. $Id_D$ is the identity function or relation over set $D$.

For each process constant schema $A$, a defining equation of the form

$$A\overset{\text{def}}{=}P$$

is assumed, where $P$ is a process.

\subsection{Syntax}

We use the Prefix $.$ to model the causality relation $\leq$ in true concurrency, the Summation $+$ to model the conflict relation $\sharp$ in true concurrency, and the Composition $\parallel$ to explicitly model concurrent relation in true concurrency. And we follow the conventions of process algebra.

\begin{definition}[Syntax]\label{syntax}
Truly concurrent processes are defined inductively by the following formation rules:

\begin{enumerate}
  \item $A\in\mathcal{P}$;
  \item $\textbf{nil}\in\mathcal{P}$;
  \item if $P\in\mathcal{P}$, then the Prefix $\alpha.P\in\mathcal{P}$, for $\alpha\in Act$;
  \item if $P,Q\in\mathcal{P}$, then the Summation $P+Q\in\mathcal{P}$;
  \item if $P,Q\in\mathcal{P}$, then the Composition $P\parallel Q\in\mathcal{P}$;
  \item if $P\in\mathcal{P}$, then the Prefix $(\alpha_1\parallel\cdots\parallel\alpha_n).P\in\mathcal{P}\quad(n\in I)$, for $\alpha_,\cdots,\alpha_n\in Act$;
  \item if $P\in\mathcal{P}$, then the Restriction $P\setminus L\in\mathcal{P}$ with $L\in\mathcal{L}$;
  \item if $P\in\mathcal{P}$, then the Relabelling $P[f]\in\mathcal{P}$.
\end{enumerate}

The standard BNF grammar of syntax of CTC can be summarized as follows:

$$P::=A\quad|\quad\textbf{nil}\quad|\quad\alpha.P\quad|\quad P+P\quad |\quad P\parallel P\quad |\quad (\alpha_1\parallel\cdots\parallel\alpha_n).P \quad|\quad P\setminus L\quad |\quad P[f].$$
\end{definition}

\subsection{Operational Semantics}

The operational semantics is defined by LTSs (labelled transition systems), and it is detailed by the following definition.

\begin{definition}[Semantics]\label{semantics}
The operational semantics of CTC corresponding to the syntax in Definition \ref{syntax} is defined by a series of transition rules, named $\textbf{Act}$, $\textbf{Sum}$, $\textbf{Com}$, $\textbf{Res}$, $\textbf{Rel}$ and $\textbf{Con}$ indicate that the rules are associated respectively with Prefix, Summation, Composition, Restriction, Relabelling and Constants in Definition \ref{syntax}. They are shown in Table \ref{TRForCTC}.

\begin{center}
    \begin{table}
        \[\textbf{Act}_1\quad \frac{}{\alpha.P\xrightarrow{\alpha}P}\]

        \[\textbf{Sum}_1\quad \frac{P\xrightarrow{\alpha}P'}{P+Q\xrightarrow{\alpha}P'}\]

        \[\textbf{Com}_1\quad \frac{P\xrightarrow{\alpha}P'\quad Q\nrightarrow}{P\parallel Q\xrightarrow{\alpha}P'\parallel Q}\]

        \[\textbf{Com}_2\quad \frac{Q\xrightarrow{\alpha}Q'\quad P\nrightarrow}{P\parallel Q\xrightarrow{\alpha}P\parallel Q'}\]

        \[\textbf{Com}_3\quad \frac{P\xrightarrow{\alpha}P'\quad Q\xrightarrow{\beta}Q'}{P\parallel Q\xrightarrow{\{\alpha,\beta\}}P'\parallel Q'}\quad (\beta\neq\overline{\alpha})\]

        \[\textbf{Com}_4\quad \frac{P\xrightarrow{l}P'\quad Q\xrightarrow{\overline{l}}Q'}{P\parallel Q\xrightarrow{\tau}P'\parallel Q'}\]

        \[\textbf{Act}_2\quad \frac{}{(\alpha_1\parallel\cdots\parallel\alpha_n).P\xrightarrow{\{\alpha_1,\cdots,\alpha_n\}}P}\quad (\alpha_i\neq\overline{\alpha_j}\quad i,j\in\{1,\cdots,n\})\]

        \[\textbf{Sum}_2\quad \frac{P\xrightarrow{\{\alpha_1,\cdots,\alpha_n\}}P'}{P+Q\xrightarrow{\{\alpha_1,\cdots,\alpha_n\}}P'}\]

        \[\textbf{Res}_1\quad \frac{P\xrightarrow{\alpha}P'}{P\setminus L\xrightarrow{\alpha}P'\setminus L}\quad (\alpha,\overline{\alpha}\notin L)\]

        \[\textbf{Res}_2\quad \frac{P\xrightarrow{\{\alpha_1,\cdots,\alpha_n\}}P'}{P\setminus L\xrightarrow{\{\alpha_1,\cdots,\alpha_n\}}P'\setminus L}\quad (\alpha_1,\overline{\alpha_1},\cdots,\alpha_n,\overline{\alpha_n}\notin L)\]

        \[\textbf{Rel}_1\quad \frac{P\xrightarrow{\alpha}P'}{P[f]\xrightarrow{f(\alpha)}P'[f]}\]

        \[\textbf{Rel}_2\quad \frac{P\xrightarrow{\{\alpha_1,\cdots,\alpha_n\}}P'}{P[f]\xrightarrow{\{f(\alpha_1),\cdots,f(\alpha_n)\}}P'[f]}\]

        \[\textbf{Con}_1\quad\frac{P\xrightarrow{\alpha}P'}{A\xrightarrow{\alpha}P'}\quad (A\overset{\text{def}}{=}P)\]

        \[\textbf{Con}_2\quad\frac{P\xrightarrow{\{\alpha_1,\cdots,\alpha_n\}}P'}{A\xrightarrow{\{\alpha_1,\cdots,\alpha_n\}}P'}\quad (A\overset{\text{def}}{=}P)\]

        \caption{Transition rules of CTC}
        \label{TRForCTC}
    \end{table}
\end{center}
\end{definition}

\subsection{Properties of Transitions}

\begin{definition}[Sorts]\label{sorts}
Given the sorts $\mathcal{L}(A)$ and $\mathcal{L}(X)$ of constants and variables, we define $\mathcal{L}(P)$ inductively as follows.

\begin{enumerate}
  \item $\mathcal{L}(l.P)=\{l\}\cup\mathcal{L}(P)$;
  \item $\mathcal{L}((l_1\parallel \cdots\parallel l_n).P)=\{l_1,\cdots,l_n\}\cup\mathcal{L}(P)$;
  \item $\mathcal{L}(\tau.P)=\mathcal{L}(P)$;
  \item $\mathcal{L}(P+Q)=\mathcal{L}(P)\cup\mathcal{L}(Q)$;
  \item $\mathcal{L}(P\parallel Q)=\mathcal{L}(P)\cup\mathcal{L}(Q)$;
  \item $\mathcal{L}(P\setminus L)=\mathcal{L}(P)-(L\cup\overline{L})$;
  \item $\mathcal{L}(P[f])=\{f(l):l\in\mathcal{L}(P)\}$;
  \item for $A\overset{\text{def}}{=}P$, $\mathcal{L}(P)\subseteq\mathcal{L}(A)$.
\end{enumerate}
\end{definition}

Now, we present some properties of the transition rules defined in Table \ref{TRForCTC}.

\begin{proposition}
If $P\xrightarrow{\alpha}P'$, then
\begin{enumerate}
  \item $\alpha\in\mathcal{L}(P)\cup\{\tau\}$;
  \item $\mathcal{L}(P')\subseteq\mathcal{L}(P)$.
\end{enumerate}

If $P\xrightarrow{\{\alpha_1,\cdots,\alpha_n\}}P'$, then
\begin{enumerate}
  \item $\alpha_1,\cdots,\alpha_n\in\mathcal{L}(P)\cup\{\tau\}$;
  \item $\mathcal{L}(P')\subseteq\mathcal{L}(P)$.
\end{enumerate}
\end{proposition}

\begin{proof}
By induction on the inference of $P\xrightarrow{\alpha}P'$ and $P\xrightarrow{\{\alpha_1,\cdots,\alpha_n\}}P'$, there are fourteen cases corresponding to the transition rules named $\textbf{Act}_{1,2}$, $\textbf{Sum}_{1,2}$, $\textbf{Com}_{1,2,3,4}$, $\textbf{Res}_{1,2}$, $\textbf{Rel}_{1,2}$ and $\textbf{Con}_{1,2}$ in Table \ref{TRForCTC}, we just prove the one case $\textbf{Act}_1$ and $\textbf{Act}_2$, and omit the others.

Case $\textbf{Act}_1$: by $\textbf{Act}_1$, with $P\equiv\alpha.P'$. Then by Definition \ref{sorts}, we have (1) $\mathcal{L}(P)=\{\alpha\}\cup\mathcal{L}(P')$ if $\alpha\neq\tau$; (2) $\mathcal{L}(P)=\mathcal{L}(P')$ if $\alpha=\tau$. So, $\alpha\in\mathcal{L}(P)\cup\{\tau\}$, and $\mathcal{L}(P')\subseteq\mathcal{L}(P)$, as desired.

Case $\textbf{Act}_2$: by $\textbf{Act}_2$, with $P\equiv(\alpha_1\parallel\cdots\parallel\alpha_n).P'$. Then by Definition \ref{sorts}, we have (1) $\mathcal{L}(P)=\{\alpha_1,\cdots,\alpha_n\}\cup\mathcal{L}(P')$ if $\alpha_i\neq\tau$ for $i\leq n$; (2) $\mathcal{L}(P)=\mathcal{L}(P')$ if $\alpha_1,\cdots,\alpha_n=\tau$. So, $\alpha_1,\cdots,\alpha_n\in\mathcal{L}(P)\cup\{\tau\}$, and $\mathcal{L}(P')\subseteq\mathcal{L}(P)$, as desired.
\end{proof} 

In this section, we discuss our network architectures as well as the training and testing procedures for action classification and detection. The objective of action classification is to classify a trimmed video into one of the predefined categories. The objective of action detection is to predict the start time, the end time, and the class of an action in an untrimmed video.

\subsection{Network Architecture}\label{sec:network_architecture}
For action classification, we encode a short clip of video into a feature vector using the visual encoder. For action detection, we first encode all clips in a window of video (a window consists of multiple clips) into initial feature vectors using the visual encoder, then feed these initial feature vectors into a sequence encoder to generate the final feature vectors. For either task, each feature vector is fed into a task-specific linear layer and a softmax layer to get the probability distribution across classes for each clip. Note that a background class is added for action detection. Our action classification and detection models are inspired by~\cite{two_stream_simonyan} and~\cite{montes2016temporal}, respectively. We design two types of visual encoders depending on the input modalities.

\noindent\textbf{Visual Encoder for Images.} Let $X=\{x_t\}_{t=1}^{T_c}$ denote a video clip of image modalities (\textit{e.g.} RGB, depth, flow), where $x_t\in\mathbb{R}^{H\times W\times C}$, $T_c$ is the number of frames in a clip, and $H\times W\times C$ is the image dimension. Similar to the temporal stream in \cite{two_stream_simonyan}, we stack the frames into a $H\times W\times (T_c\cdot C)$ tensor and encode the video clip with a modified ResNet-18~\cite{resnet} with $T_c\cdot C$ input channels and without the last fully-connected layer. Note that we do not use the Convolutional 3D (C3D) network~\cite{i3d_carreira,c3d_tran} because it is hard to train with limited amount of data~\cite{i3d_carreira}.

\noindent\textbf{Visual Encoder for Vectors.} Let $X=\{x_t\}_{t=1}^{T_c}$ denote a video clip of vector modalities (\textit{e.g.} skeleton), where $x_t\in\mathbb{R}^{D}$ and $D$ is the vector dimension. Similar to \cite{10-stream}, we encode the input with a 3-layer GRU network~\cite{gru} with $T_c$ timesteps. The encoded feature is computed as the average of the outputs of the highest layer across time. The hidden size of the GRU is chosen to be the same as the output dimension of the visual encoder for images.

\noindent\textbf{Sequence Encoder.} Let $X = \{x_t\}_{t=1}^{T_c\cdot T_w}$ denote a window of video with $T_w$ clips, where each clip contains $T_c$ frames. The visual encoder first encodes each clip individually into a single feature vector. These $T_w$ feature vectors are then passed into the sequence encoder, which is a 1-layer GRU network, to obtain the class distributions of these $T_w$ clips. Note that the sequence encoder is only used in action detection.



\subsection{Training and Testing}

Our proposed graph distillation can be applied to both action detection and classification. For action detection, we show that our method can optionally pre-train the action detection model on action classification tasks, and graph distillation can be applied in both pre-training and training stages. Both models are trained to minimize the loss in Eq.~\eqref{eq:distilation_loss} on per-clip classification, and the imitation loss is calculated based on the representations and the logits. 

\noindent\textbf{Action Classification.}
Fig.~\ref{fig:modela} shows how graph distillation is applied in training. During training, we randomly sample a video clip of $T_c$ frames from the video, and the network outputs a single class distribution. During testing, we uniformly sample multiple clips spanning the entire video and average the outputs to obtain the final class distribution.

\noindent\textbf{Action Detection.}
Fig.~\ref{fig:modelb} and Fig.~\ref{fig:modelb} show how graph distillation is applied in training and testing, respectively. As discussed earlier, graph distillation can be applied to both the source domain and the target domain. During training, we randomly sample a window of $T_w$ clips from the video, where each clip is of length $T_c$ and is sampled with step size $s_c$. As the data is imbalanced, we set a class-specific weight based on its inverse frequency in the training set. During testing, we uniformly sample multiple windows spanning the entire video with step size $s_w$, where each window is sampled in the same way as training. The outputs of the model are the class distributions on all clips in all windows (potentially with overlaps depending on $s_w$). These outputs are then post-processed using the method in~\cite{montes2016temporal} to generate the detection results, where the activity threshold $\gamma$ is introduced as a hyperparameter.




In this section, we evaluate our method on two large-scale multimodal video benchmarks. The results show that our method outperforms representative baseline methods and achieves the state-of-the-art performance on both benchmarks. 



\subsection{Datasets and Setups}\label{sec:dataset_setups}
We evaluate our method on two large-scale multimodal video benchmarks: NTU RGB+D~\cite{ntu_rgbd} (classification) and PKU-MMD~\cite{pku_mmd} (detection). These datasets are selected for the following reasons. (1) They are (one of the) largest RGB-D video benchmarks in each category. (2) The privileged information transfer is reasonable because the domains of the two datasets are similar. (3) They contain abundant modalities, which are required for graph distillation. 

We use NTU RGB+D as our dataset in the source domain, and PKU-MMD in the target domain. In our experiments, unless stated otherwise, we apply graph distillation whenever applicable. Specifically, the visual encoders of all modalities are jointly trained on NTU RGB+D by graph distillation. On PKU-MMD, after initializing the visual encoder with the pre-trained weights obtained from NTU RGB+D, we also learn all available modalities by graph distillation on the target domain. By default, only a single modality is used at test time.

\noindent\textbf{NTU RGB+D~\cite{ntu_rgbd}.} 
It contains 56,880 videos from 60 action classes. Each video has exactly one action class and comes with four modalities: RGB, depth, 3D joints, and infrared. The training and testing sets have 40,320 and 16,560 videos, respectively. All results are reported with cross-subject evaluation.

\noindent\textbf{PKU-MMD~\cite{pku_mmd}.} 
It contains 1,076 long videos from 51 action classes. Each video contains approximately 20 action instances of various lengths and consists of four modalities: RGB, depth, 3D joints, and infrared. All results are evaluated based on the Average Precision (mAP) at different temporal Intersection over Union (tIoU) thresholds between the predicted and the ground truth intervals.

\noindent\textbf{Modalities.} We use a total of six modalities in our experiments: RGB, depth (D), optical flow (F), and three skeleton features (S) named Joint-Joint Distances (JJD), Joint-Joint Vector (JJV), and Joint-Line Distances (JLD)~\cite{ding2017investigation,10-stream}, respectively. The RGB and depth videos are provided in the datasets. The optical flow is calculated on the RGB videos using the dual TV-L1 method~\cite{zach2007duality}. The three spatial skeleton features are extracted from 3D joints using the method in \cite{ding2017investigation} and \cite{10-stream}. Note that we select a subset of the ten skeleton features in~\cite{ding2017investigation,10-stream} to ensure the simplicity and reproducibility of our method, and our approach can potentially perform better with the complete set of features.

\noindent\textbf{Baselines.}
In addition to comparing with the state-of-the-art, we implement three representative baselines that could be used to leverage multimodal privileged information: \textit{multi-task learning}~\cite{caruana1998multitask}, \textit{knowledge distillation}~\cite{distillation_hinton}, and \textit{cross-modal distillation}~\cite{distillation_gupta}. For the multi-task model, we predict the raw pixels of the other modalities from the representation of a single modality, and use the $L_2$ distance as the multi-task loss. For the distillation methods, the imitation loss is calculated as the high-temperature cross-entropy loss on the soft logits~\cite{distillation_hinton}, and $L_2$ loss on both representations and soft logits in cross-modal distillation~\cite{distillation_gupta}. These distillation methods originally only support two modalities, and therefore we average the pairwise losses to get the final loss.



\begin{table}[t]
\centering
\scriptsize
\caption{Comparison with state-of-the-art on NTU RGB+D. Our models are trained on all modalities and tested on the single modality specified in the table. The available modalities are RGB, depth (D), optical flow (F), and skeleton (S).}
\label{ntu_state_of_the_art}
\begin{tabular}{lc@{\hskip 0.1in}c@{\hskip 0.8in}l@{\hskip 0.4in}c@{\hskip 0.1in}c}
\toprule
Method & Test Modality & mAP & Method & Test Modality & mAP  \\
\midrule
Shahroudy~\cite{shahroudy2017deep} & RGB+D & 0.749 & Ours & RGB & \textbf{0.895} \\
Liu~\cite{liu2017viewpoint} & RGB+D & 0.775 & Ours  & D & 0.875 \\
Liu~\cite{skeleton_visualization} & S & 0.800 & Ours  & F & 0.857 \\
Ding~\cite{ding2017investigation} & S & 0.823 & Ours  & S & 0.837 \\
Li~\cite{10-stream} & S & 0.829 &&& \\
\bottomrule
\end{tabular}
\end{table}

\begin{table}[t]
\centering
\scriptsize
\caption{Comparison of action detection methods on PKU-MMD with state-of-the-art models. Our models are trained with graph distillation using all privileged modalities
and tested on the modalities specified in the table. ``Transfer'' refers to pre-training on NTU RGB+D on action classification. The available modalities are RGB, depth (D), optical flow (F), and skeleton (S).}
\label{pku_state_of_the_art}
\begin{tabular}{l@{\hskip 0.1in}c@{\hskip 0.1in}c@{\hskip 0.1in}c@{\hskip 0.1in}c}
\toprule
\multicolumn{2}{c}{} & \multicolumn{3}{c}{mAP @ tIoU thresholds ($\theta$)} \\
\cmidrule(r){3-5}
Method & Test Modality & 0.1 & 0.3 & 0.5 \\ 
\midrule
Deep RGB (DR) \cite{pku_mmd} & RGB & 0.507 & 0.323 & 0.147 \\
Qin and Shelton \cite{pku_result_qin} & RGB & 0.650 & 0.510 & 0.294 \\
Deep Optical Flow (DOF) \cite{pku_mmd} & F & 0.626 & 0.402 & 0.168 \\
Raw Skeleton (RS) \cite{pku_mmd} & S & 0.479 & 0.325 & 0.130 \\
Convolution Skeleton (CS) \cite{pku_mmd} & S & 0.493 & 0.318 & 0.121 \\
Wang and Wang \cite{pku_result_wang_workshop} & S & 0.842 & - & 0.743 \\
RS+DR+DOF \cite{pku_mmd} & RGB+F+S & 0.647 & 0.476 & 0.199 \\
CS+DR+DOF \cite{pku_mmd} & RGB+F+S & 0.649 & 0.471 & 0.199 \\
\midrule
Ours (w/o $|$ w/ transfer) & RGB & 0.824 $|$ 0.880 & 0.813 $|$ 0.868 & 0.743 $|$ 0.801 \\
Ours (w/o $|$ w/ transfer) & D   & 0.823 $|$ 0.872 & 0.817 $|$ 0.860 & 0.752 $|$ 0.792 \\
Ours (w/o $|$ w/ transfer) & F   & 0.790 $|$ 0.826 & 0.783 $|$ 0.814 & 0.708 $|$ 0.747 \\
Ours (w/o $|$ w/ transfer) & S   & 0.836 $|$ 0.857 & 0.823 $|$ 0.846 & 0.764 $|$ 0.784 \\
Ours (w/ transfer) & RGB+D+F+S & \bf{0.903} & \bf{0.895} & \bf{0.833} \\
\bottomrule
\end{tabular}
\end{table}



\noindent\textbf{Implementation Details.} 
For action classification, we train the visual encoder from scratch for 200 epochs using SGD with momentum with learning rate $10^{-2}$ and decay to $10^{-1}$ at epoch 125 and 175. $\lambda_1$ and $\lambda_2$ are set to $10,5$ respectively in Eq.~\eqref{eq:message_ab}. At test time we sample 5 clips for inference. For action detection, the visual and sequence encoder are trained for 400 epochs. The visual encoder is trained using SGD with momentum with learning rate $10^{-3}$, and the sequence encoder is trained with the Adam optimizer~\cite{kingma2015adam} with learning rate $10^{-3}$. The activity threshold $\gamma$ is set to $0.4$. For both tasks, we down-sample the frame rates of the datasets by a factor of 3. The clip length and detection window $T_c$ and $T_w$ are both set to 10. For the graph distillation, $\alpha$ is set to 10 in Eq.~\eqref{eq:graph_learning_softmax}. The output dimensions of the visual and sequence encoder are both set to 512. Since it is nontrivial to jointly train on multiple modalities from scratch, we employ curriculum learning~\cite{bengio2009curriculum} to train the distillation graph. To do so, we first fix the distillation graph as an identity matrix (uniform graph) in the first 200 epochs. In the second stage, we compute the constant vector $\mathbf{c}$ in Eq.~\eqref{eq:message_graph_final} according to the cross-validation results, and then learn the graph in an end-to-end manner.



\subsection{Comparison with State-of-the-Art}\label{sec:exp_soa}



\begin{figure}[t]
\begin{center}
\includegraphics[width=\linewidth]{predictions}
\end{center}
\caption{\textbf{A comparison of the prediction results on PKU-MMD.} (a) Both models make correct predictions. (b) The model without distillation in the source makes errors. Our model learns motion and skeleton information from the privileged modalities in the source domain, which helps the prediction for classes such as ``hand waving'' and ``falling''. (c) Both models make reasonable errors.}
\label{fig:detection}
\end{figure}





\noindent\textbf{Action Classification.} Table~\ref{ntu_state_of_the_art} shows the comparison of action classification with state-of-the-art models on NTU RGB+D dataset. Our graph distillation models are trained and tested on the same dataset in the source domain. NTU RGB+D is a very challenging dataset and has been recently studied in numerous studies~\cite{10-stream,liu2017viewpoint,skeleton_visualization,luo2017unsupervised,shahroudy2017deep}. Nevertheless, as we see, our model achieves the state-of-the-art results on NTU RGB+D. It yields a 4.5\% improvement, over the previous best result, using the depth video and a remarkable 6.6\% using the RGB video. After inspecting the results, we found the improvement mainly attributes to the learned graph capturing complementary information across multiple modalities. Fig.~\ref{fig:graph} shows example distillation graphs learned on NTU RGB+D. The results show that our method, without transfer learning, is effective for action classification in the source domain.


\noindent\textbf{Action Detection.} Table~\ref{pku_state_of_the_art} compares our method on PKU-MMD with previous work. Our model outperforms existing methods across all modalities. The results substantiate that our method can effectively leverage the privileged knowledge from multiple modalities. Fig.~\ref{fig:detection} illustrates detection results on the depth modality with and without the proposed distillation.


\subsection{Ablation Studies on Limited Training Data}\label{sec:ablation}
Section~\ref{sec:exp_soa} has shown that our method achieves the state-of-the-art results on two public benchmarks. However, in practice, the training data are often limited in size. To systematically evaluate our method on limited training data, as proposed in the introduction, we construct mini-NTU RGB+D and mini-PKU-MMD by randomly sub-sampling 5\% of the training data from their full datasets and use them for training. For evaluation, we test the model on the full test set.



\begin{table}[t]
\centering
\scriptsize
\caption{The comparison with (a) baseline methods using Privileged Information (PIs) on mini-NTU RGB+D, (b) distillation graphs on mini-NTU RGB+D and mini-PKU-MMD. Empty graph trains each modality independently. Uniform graph uses a uniform weight in distillation. Prior graph is built according to the cross-validation accuracy of each modality. Learned graph is learned by our method. ``D'' refers to the depth modality.}
\subtable[\label{ntu_baselines}Baseline methods using PIs.]
{
  \renewcommand{\arraystretch}{1.1}
  \begin{tabular}{lcc}
  \toprule
  Method & mAP / RGB \\
  \midrule
  Empty graph & 0.464 \\
  Multi-task \cite{caruana1998multitask}  & 0.456 \\
  Cross-distillation \cite{distillation_gupta}  & 0.503 \\
  Knowledge distillation \cite{distillation_hinton}  & 0.524 \\
  Learned graph & \bf{0.619} \\
  \bottomrule
  \end{tabular}
}
\subtable[\label{different_graphs}Different distillation graphs.]{
  \begin{tabular}{l@{\hskip 0.1in}c@{\hskip 0.2in}c}
  \toprule
  \multicolumn{1}{c}{} & mini-NTU & mini-PKU \\
  \cmidrule(r){2-3}
  Graph & {\tiny mAP / RGB} & {\tiny mAP @ 0.5 / D} \\
  \midrule
  Empty graph & 0.464 & 0.501 \\
  Uniform graph & 0.537 & 0.513 \\
  Prior graph & 0.571 & 0.515 \\
  Learned graph & \bf{0.619} & \bf{0.559}\\
  \bottomrule
  \end{tabular}
}
\end{table}

\begin{table}[t]
\centering
\scriptsize
\caption{The mAP comparison on mini-PKU-MMD at different tIoU threshold $\theta$. The depth modality is chosen for testing. ``src'', ``trg'', and ``PI'' stand for source, target, and privileged information, respectively.}
\label{pku_mmd_baselines}
\begin{tabular}{c@{\hskip 0.2in}l@{\hskip 0.4in}c@{\hskip 0.4in}c@{\hskip 0.4in}c}
\toprule
\multicolumn{2}{c}{} & \multicolumn{3}{c}{mAP @ tIoU thresholds ($\theta$)} \\
\cmidrule(r){3-5}
 & Method & $0.1$ & $0.3$ & $0.5$ \\
\midrule
1&trg only & 0.248 & 0.235 & 0.200 \\
2&src + trg & 0.583 & 0.567 & 0.501 \\
3&src w/ PIs + trg & 0.625 & 0.610 & 0.533 \\
4&src + trg w/ PIs & 0.626 & 0.615 & 0.559 \\
5&src w/ PIs + trg w/ PIs & 0.642 & 0.629 & 0.562 \\
\midrule
6&src w/ PIs + trg & 0.625 & 0.610 & 0.533  \\
7&src w/ PIs + trg w/ 1 PI & 0.632 & 0.615 & 0.549 \\
8&src w/ PIs + trg w/ 2 PIs & 0.636 & 0.624 & 0.557 \\
9&src w/ PIs + trg w/ all PIs & 0.642 & 0.629 & 0.562 \\
\bottomrule
\end{tabular}
\end{table}



\noindent\textbf{Comparison with Baseline Methods.} Table~\ref{ntu_baselines} shows the comparison with the baseline models that uses privileged information (see Section~\ref{sec:dataset_setups}). The fact that our method outperforms the representative baseline methods validates the efficacy of the graph distillation method.

\noindent\textbf{Efficacy of Distillation Graph.} Table \ref{different_graphs} compares the performance of predefined and learned distillation graphs. The proposed learned graph is compared with an empty graph (no distillation), a uniform graph of equal weights, and a prior graph computed using the cross-validation accuracy of each modality. Results show that the learned graph structure with modality-specific prior and example-specific information obtains the best results on both datasets.



\begin{figure}[t]
% \mpage{0.48}{\small{(a) Falling}}\hfill
% \mpage{0.48}{\small{(b) Brushing teeth}}\hfill
\begin{center}
\includegraphics[width=0.8\linewidth]{graph}
\end{center}
\caption{\textbf{The visualization of graph distillation on NTU RGB+D.} The numbers indicate the ranks of the distillation weights, with 1 being the largest and 5 being the smallest. (a) Class ``falling'': Our graph assigns more weight to optical flow because optical flow captures the motion information. (b) Class ``brushing teeth'': In this case, motion is negligible, and our graph assigns the smallest weight to it. Instead, it assigns the largest weight to skeleton data.}
\label{fig:graph}
\end{figure}



\noindent\textbf{Efficacy of Privileged Information.} Table~\ref{pku_mmd_baselines} compares our distillation and transfer under different training settings. The input at test time is a single depth modality. By comparing row 2 and 3 in Table~\ref{pku_mmd_baselines}, we see that when transferring the visual encoder to the target domain, the one pre-trained with privileged information in the source domain performs better than its counterpart. As discussed in Section~\ref{sec:collective}, graph distillation can also be applied to the target domain. By comparing row 3 and 5 (or row 2 and 4) of Table~\ref{pku_mmd_baselines}, we see that performance gain is achieved by applying the graph distillation in the target domain. The results show that our graph distillation can capture useful information from multiple modalities in both the source and target domain.

\noindent\textbf{Efficacy of Having More Modalities.} The last three rows of Table \ref{pku_mmd_baselines} show that performance gain is achieved by increasing the number of modalities used as the privileged information. Note that the test modality is depth, the first privileged modality is RGB, and the second privileged modality is the skeleton feature JJD. The results also suggest that these modalities provide each other complementary information during the graph distillation.



\subsection{Graph Distillation on UCF-101}

In this section, we consider graph edge distillation, a special case of graph distillation on UCF-101~\cite{soomro2012ucf101} in which only two modalities (RGB and optical flow) are available. Table~\ref{ucf101} shows the action classification results on UCF-101 using the two-stream architecture proposed in~\cite{two_stream_simonyan}. The optical flow modality performs significantly better than RGB when training from scratch. This is consistent with previous findings that dense optical flow is able to achieve very good performance in spite of limited training data \cite{two_stream_simonyan}. To testify our method, we train a model on the RGB modality from scratch with distillation. Our distilled model performs much better than the model directly trained from scratch. Note that our distilled model outperforms the fine-tuning model that uses pretrained weights on ImageNet.

\begin{table}[ht]
\scriptsize
\centering
\caption{Action classification results on UCF101. For graph distillation model, we distill knowledge from the optical flow stream to the RGB stream.}
\label{ucf101}
\begin{tabular}{l@{\hskip 0.1in}c@{\hskip 0.2in}c@{\hskip 0.1in}c}
\toprule
Method & Test Modality & mAP & Diff. \\
\midrule
From scratch   & Flow & 0.803 & - \\
From scratch   & RGB & 0.484 & +0.000 \\
ImageNet pretrained & RGB & 0.728 & +0.244 \\
Graph distillation & RGB & \textbf{0.757} & \textbf{+0.273} \\
\bottomrule
\end{tabular}
\end{table}


%\documentclass[preprint,12pt]{elsarticle}
%\if0
\usepackage{amssymb}
\usepackage{mathtools}
%\usepackage[dvipdfmx]{graphicx}
\usepackage{cite}
\usepackage{graphicx}
\usepackage{bm}
\usepackage{here}
\usepackage[subrefformat=parens]{subcaption}
\fi
%\usepackage{amssymb}
\usepackage{amsmath}
\usepackage[dvipdfmx]{}
\usepackage[dvipdfmx]{color}
%\usepackage{cite}
%\usepackage{upgreek}
\usepackage{url}
%\usepackage[dvipdfmx]{hyperref}
%\usepackage{pxjahyper}
%\usepackage {hyperref}
\usepackage{graphicx}
\usepackage{bm}
\usepackage{here}
\usepackage{caption}
\usepackage[subrefformat=parens]{subcaption}
\captionsetup{compatibility=false}

%% The amsthm package provides extended theorem environments
%% \usepackage{amsthm}

%% The lineno packages adds line numbers. Start line numbering with
%% \begin{linenumbers}, end it with \end{linenumbers}. Or switch it on
%% for the whole article with \linenumbers after \end{frontmatter}.
%% \usepackage{lineno}

%% natbib.sty is loaded by default. However, natbib options can be
%% provided with \biboptions{...} command. Following options are
%% valid:

%%   round  -  round parentheses are used (default)
%%   square -  square brackets are used   [option]
%%   curly  -  curly braces are used      {option}
%%   angle  -  angle brackets are used    <option>
%%   semicolon  -  multiple citations separated by semi-colon
%%   colon  - same as semicolon, an earlier confusion
%%   comma  -  separated by comma
%%   numbers-  selects numerical citations
%%   super  -  numerical citations as superscripts
%%   sort   -  sorts multiple citations according to order in ref. list
%%   sort&compress   -  like sort, but also compresses numerical citations
%%   compress - compresses without sorting
%%
%% \biboptions{comma,round}

% \biboptions{}

%% This list environment is used for the references in the
%% Program Summary
%%
\newcounter{bla}
\newenvironment{refnummer}{%
\list{[\arabic{bla}]}%
{\usecounter{bla}%
 \setlength{\itemindent}{0pt}%
 \setlength{\topsep}{0pt}%
 \setlength{\itemsep}{0pt}%
 \setlength{\labelsep}{2pt}%
 \setlength{\listparindent}{0pt}%
 \settowidth{\labelwidth}{[9]}%
 \setlength{\leftmargin}{\labelwidth}%
 \addtolength{\leftmargin}{\labelsep}%
 \setlength{\rightmargin}{0pt}}}
 {\endlist}
\begin{document}

\section{Testing the program O-SUKI-N 3D}
The several tests are shown below to present the target fuel implosion dynamics. In the example cases, the HIBs and the target fuel have the following common parameters, which are the same values employed in Ref. \cite{CPC-O-SUKI}: the beam radius at the entrance of a reactor chamber $R_{en}$ = 35 mm, the beam particle density distribution is in the Gaussian profile and all projectile Pb ions have 8 GeV. The target is a multilayered pellet, in which the pellet outer radius is 4 mm, a Pb layer thickness is 0.029 mm, the Al thickness is 0.460 mm, and the DT thickness is 0.083 mm; the Pb, Al and DT layers have the radial mesh numbers of 4, 46 and 30 in these example cases, respectively, and the total mesh number in the theta direction is 90. The input beam pulse is shown in Fig. 12 in Ref. \cite{CPC-O-SUKI}. The beam radius is 3.8mm on the target surface. However, $R_b$ = 3.8mm changes at $\tau_{wb}$ to 3.7mm for the wobbling beam irradiation. Here $\tau_{wb}$ is the rotational period of the beam axis. The rotational frequency is 424MHz ($rotaionnumber$ = 11). 



%% INPUT PULSE  
%\begin{figure}[H]
%		\centering
%		\includegraphics[width=10cm]{images/pulse.eps}
%		\caption{An example for the input beam pulse.}\label{pulse}
%\end{figure}



First the 3D Langrange code was run without the OK3 illumination code. This is the case for $OK\_Switch=10$, and we added the artificial non-uniformity $Y_3^2$ (the spherical harmonics) with the amplitude of $30.0\%$. In Fig. \ref{NoOK3_23_Ti} the ion temperature distribution is shown at $t$=35ns, and in Fig. \ref{NoOK3_23_rho} the mass density distribution is presented at $t$=35ns. The target shape is largely distorted due to the non-uniformity of the HIBs deposition energy distribution.  


%% LAGRANGE CASE WITHOUT OK3
\begin{figure}[H]
		\centering
		\includegraphics[width=10cm]{images/NoOK3_Non23_30_35ns_Ti.eps}
		\caption{Ion temperature in the 3D Lagrange code without OK3 code at $t$=35ns. The non-uniformity distiribution is $Y_3^2$ with the amplitude of $30\%$.}\label{NoOK3_23_Ti}
\end{figure}
\begin{figure}[H]
		\centering
		\includegraphics[width=10cm]{images/NoOK3_Non23_30_35ns_rho.eps}
		\caption{Mass density in the 3D Lagrange code without OK3 code at $t$=35ns. The non-uniformity distriution is $Y_3^2$ with the amplitude of $30\%$.}\label{NoOK3_23_rho}
\end{figure}


We also performed run-through simulation tests. In the example cases, the OK3 code was coupled with the run-through simulations. The implosion data were obtained by the Lagrange code, and the data just before the void closure time were transferred to the Euler code through the data Conversion code. Two cases are computed for the target fuel implosion dynamics with the spiral wobbling or without the oscillating HIBs. These examples are the run-through simulations with the OK3 illumination code ($OK\_Switch = 1$). The input beam pulse, employed in the run-through tests, is shown in Fig. \ref {Beam}. This beam input energy is 5.4MJ. We show the $r-t$ diagram for the case without the HIBs wobbling in Fig. \ref{rt}. The Lagrange-code test results stored in the output directory are visualized in Figs. \ref {fusion_Ti} for the target ion temperature ($T_i$) distributions at $t$ = 0.0, 15.0, 20.0 and 22.5 ns for the case with the HIBs wobbling behavior.  The RMS non-uniformity results are shown in Figs. \ref{fusion_RMS} (a) for DT layer's Ion temperature($T_i$), (b) for DT layer's Mass density($\rho$), (c) for Al layer's Ion temperature($T_i$) and (d) for Al layer's Mass density($\rho$). 
%
When the HIBs have the wobbling motion during the implosion with the wobbling frequency of 424MHz, the radius acceleration distributions are shown in Figs. \ref{Vr_tp} (a) in the $\theta$ direction and (b) in the $\phi$ direction at $t=6.25t_w=11.2ns$ (solid lines) and at $t=6.75t_w=12.2ns$ (dotted lines). Here $t_w$ shows the one rotation time. Figures \ref{Vr_tp} present that the non-uniformity phase of the implosion acceleration is controlled externally by the HIBs wobbling behavior \cite{CPC-O-SUKI, RSato2}.  
%

\begin{figure}[H]
		\centering
		\includegraphics[width=7.5cm]{images/Beam.eps}
		\caption{Input beam pulse shape used in the example run-through tests.}\label{Beam}
\end{figure}
\begin{figure}[H]
		\centering
		\includegraphics[width=8cm]{images/YesWob_SLC.eps}
		\caption{The $r-t$ diagram for the implosion with the HIBs wobbling illumination. The black line area shows the Pb layer, the blue line area Al and the red line area is DT.}\label{rt}
\end{figure}
\begin{figure}[H]
		\centering
		\includegraphics[width=6.5cm]{images/YesWob_Ti_0ns.eps}
		\includegraphics[width=6.5cm]{images/YesWob_Ti_15ns.eps}\\
		\includegraphics[width=6.5cm]{images/YesWob_Ti_20ns.eps}
		\includegraphics[width=6.5cm]{images/YesWob_Ti_225ns.eps}\\
		\caption{Ion temperature distributions in the example run-through test with the HIBs wobbling illumination at (a) $t$=0.0ns, (b) 15.0ns, (c) 20.0ns and (d) 22.5ns.}\label{fusion_Ti}
\end{figure}
\begin{figure}[H]
		\centering
		\includegraphics[width=6.5cm]{images/FusionRMS_DTTi.eps}
		\includegraphics[width=6.5cm]{images/FusionRMS_DTrho.eps}\\
		\includegraphics[width=6.5cm]{images/FusionRMS_AlTi.eps}
		\includegraphics[width=6.5cm]{images/FusionRMS_Alrho.eps}\\
		\caption{RMS non-uniformity histories of (a) the DT ion temperature, (b) the DT mass density, (c) the Al ion temperature and (d) the Al mass density for the cases with the wobbling HIBs (solid lines) and without the wobbling HIBs (dotted lines).}\label{fusion_RMS}
\end{figure}
%
\begin{figure}[H]
		\centering
		\includegraphics[width=6.5cm]{images/theta-Vr.eps}
		\includegraphics[width=6.5cm]{images/phi-Vr.eps}\\
		\caption{Radial acceleration distributions in (a) $\theta$ and (b) $\phi$. The solid lines show the acceleration ditributions at $t=6.25t_w=11.3ns$, and the dotted lines at $t=6.75t_w=12.2ns$.}\label{Vr_tp}
\end{figure}
%

After the Lagrange code computation, the implosion data are converted and transferred to the Euler code. Figures \ref{Ti_EuWobblIgnited} show the ion temperature distributions by the Euler code. Figures \ref{Ti_EuWobblIgnited} show that the DT fuel is ignited and the gain obtained is about 17.5 in this example case. For a realistic HIF reactor design, the implosion parameters should be further optimized to obtain a sufficient gain, which should be larger than 30$\sim$40 in HIF \cite{CPC-O-SUKI, Kawata1, Kawata2, RSato2}. 

\begin{figure}[H]
		\centering
		\includegraphics[width=13cm]{images/EuWobblIgnited.eps}
		\caption{Ion temperature distributions (a) at $t=$24.88ns, (b) at 28.44ns and at 29.21ns.}\label{Ti_EuWobblIgnited}
\end{figure}

\if0
In Fig. \ref{NoOK3_03_Ti}, a non-uniform energy deposition of the HIBs illumination is introduced based on the spherical harmonics $Y_3^0$ with the amplitude of $3.0\%$ in the 3D Lagrange code. The implosion data was obtained by the Lagrange code, and the data just before the void closure time were transferred to the Euler code through the data Conversion code.  Figure \ref{Ti_Eu_Y03} shows the ion temperature distributions  by the Euler code at (a) at $t$=36.36ns, (b) 36.57ns, (c) 41.32ns and (d) 42.41ns. In this example case the DT fuel is not yet ignited due to the insufficient ion temperature. 

\begin{figure}[H]
		\centering
		\includegraphics[width=8.5cm]{images/NoOK3_Non03_03_35ns_Ti.eps}
		\caption{Ion temperature in the 3D Lagrange code without OK3 code at $t$=35ns. The non-uniformity distriution is $Y_3^0$ with the amplitude of $3\%$.}\label{NoOK3_03_Ti}
\end{figure}


%% TIME VS ION TEMPERATURE Euler Y03
\begin{figure}[H]
		\centering
		\includegraphics[width=6.5cm]{images/ion_Eu_Y03_36_36ns.eps}
		\includegraphics[width=6.5cm]{images/ion_Eu_Y03_36_57ns.eps} \\
		\includegraphics[width=6.5cm]{images/ion_Eu_Y03_41_32ns.eps}
		\includegraphics[width=6.5cm]{images/ion_Eu_Y03_42_41ns.eps} \\
		\caption{Ion temperature distributions under a non-uniform energy deposition based on the spherical harmonics $Y_0^3$ by the Euler code,  (a) at $t$=36.36ns, (b) 36.57ns, (c) 41.32ns and (d) 42.41ns.}\label{Ti_Eu_Y03}
\end{figure}
\fi


In order to check the accuracy of the 3D Euler code, we also performed the Euler code tests, using the initial conditions of the 2D Euler code. The initial conditions in the Euler code are the output of the Lagrangian code.  To this end, the 2D Euler initial conditions were converted into 3D. Therefore, the physical values are uniform in the $\phi$ direction. The Lagrangian test 2D results for the target ion temperature ($T_i$) and the mass density ($\rho$) distribution at $t$ = 29 ns are shown in Figs. 14 and 15 in Ref. \cite{CPC-O-SUKI} for the cases with and without the wobbling HIBs.  The 2D Eulerian test results for the fusion energy gain is shown in Fig. 16 in Ref. \cite{CPC-O-SUKI}.  In Fig. \ref{Ti_Eu_3d} we show the ion temperature distributions by the 3D Euler code. The wobbling HIBs are not used in this simulation. In this case the fuel is ignited at $t \sim $30.1ns. The histories of the fusion gain $G$ of the 2D code and the 3D code are shown in Fig. \ref{FusionGain_Eu}. The fusion gain was 52.5 by the 2D code and 57.6 by the 3D code. In addition, we also did another test for the wobbling HIBs (see Figs. 15 and 16 in Ref. \cite{CPC-O-SUKI}), and the fusion gain was 76.1 in 2D \cite{CPC-O-SUKI} and 67.4 in 3D. The results would confirm that the 3D Euler code reproduces the 2D results successfully for the ignition time and the fusion gain obtained. 


%% TIME VS ION TEMPERATURE Euler
\begin{figure}[H]
		\centering
		\includegraphics[width=6.5cm]{images/ion_Eu_30_42ns.eps}
		\includegraphics[width=6.5cm]{images/ion_Eu_30_53ns.eps} \\
		\includegraphics[width=6.5cm]{images/ion_Eu_32_35ns.eps}
		\includegraphics[width=6.5cm]{images/ion_Eu_32_58ns.eps} \\
		\caption{Ion temperature distributions by the 3D Euler code without the HIBs wobbling at (a) $t$=30.42ns, (b) 30.53ns, (c) 32.35ns and (d) 32.58ns}\label{Ti_Eu_3d}
\end{figure}


%% ENERGY GAIN Euler
\begin{figure}[H]
		\centering
		\includegraphics[width=11cm]{images/FusionGain_Eu.eps}
		\caption{Fusion energy gain curves for the cases with 3D code (a solid line) and with 2D code (a dotted line).}\label{FusionGain_Eu}
\end{figure}

We also simulated the double-cone ignition scheme\cite{Double-cone} using a 3D Euler code. The double-cone ignition scheme was proposed by Prof. Jie Zhang \cite{Double-cone}, and the two compressed DT clouds are created by the gold cones. The two DT spherical clouds collide each other like the impact fusion \cite{Winterberg}. In this example case, the compressed DT maximum density of the DT fuel is set to be $1.0\times 10^5$[kg/m$^3$] with the Gaussian spatial distribution. The DT ignition will be attained by an additional heating, which is not taken into consideration in this example. The ion, electron and radiation temperatures are 10[eV] initially in the Euler code. The radius of the fuel is 92[$\mu$m] and the mass was $0.1$[mg]. We set the colliding speed $w$ of the two DT fuel clouds to $3.0\times10^5$ [m/s]. The ion temperature distributions are shown in Fig. \ref{Double_cone_Ti}.


%% DOUBLE-CONE
\begin{figure}[H]
		\centering
		\includegraphics[width=6.5cm]{images/double_cone_0ns.eps}
		\includegraphics[width=6.5cm]{images/double_cone_15_06ns.eps} \\
		\includegraphics[width=6.5cm]{images/double_cone_29_80ns.eps}
		\includegraphics[width=6.5cm]{images/double_cone_46_78ns.eps} \\
		\caption{Ion temperature distributions for the Double-cone ignition scheme \cite{Double-cone} at (a) $t$=0.0ns, (b) 15.06ns, (c) 29.80ns and (d) 46.78ns.}\label{Double_cone_Ti}
\end{figure}

	
%\include{end}

\section{Conclusions}\label{con}

We design a calculus for true concurrency (CTC). Indeed, we follow the way paved by Milner's famous CCS \cite{CCS} \cite{CC} for interleaving bisimulation.

Fortunately, based on the concepts for true concurrency, CTC has good properties modulo several kinds of strongly truly concurrent bisimulations and weakly truly concurrent bisimulations. These properties include monoid laws, static laws, new expansion law for strongly truly concurrent bisimulations, $\tau$ laws for weakly truly concurrent bisimulations, and full congruences for strongly and weakly truly concurrent bisimulations, and also unique solution for recursion.

CTC is a peer in true concurrency to CCS in interleaving bisimulation semantics. It can be used widely in verification of computer systems with a truly concurrent flavor. 

% Template for ASRU-2021 paper; to be used with:
%          spconf.sty  - ICASSP/ICIP LaTeX style file, and
%          IEEEbib.bst - IEEE bibliography style file.
% --------------------------------------------------------------------------
\documentclass{article}
\usepackage{spconf,amsmath,graphicx}

\usepackage[table]{xcolor}
\usepackage{amssymb}
\usepackage{multirow}

% Example definitions.
% --------------------
\def\x{{\mathbf x}}
\def\L{{\cal L}}

% Title.
% ------
\title{Speaker conditioning of acoustic models using affine transformation for multi-speaker speech recognition}
%
% Single address.
% ---------------
\name{Midia Yousefi, John H.L. Hansen}
\address{  Center for Robust Speech Systems (CRSS), Erik Jonsson School of Engineering,\\
University of Texas at Dallas, Richardson, Texas, USA}
%
% For example:
% ------------
%\address{School\\
%	Department\\
%	Address}
%
% Two addresses (uncomment and modify for two-address case).
% ----------------------------------------------------------
%\twoauthors
%  {A. Author-one, B. Author-two\sthanks{Thanks to XYZ agency for funding.}}
%	{School A-B\\
%	Department A-B\\
%	Address A-B}
%  {C. Author-three, D. Author-four\sthanks{The fourth author performed the work
%	while at ...}}
%	{School C-D\\
%	Department C-D\\
%	Address C-D}
%
\begin{document}
%\ninept
%
\maketitle
%
\begin{abstract}

This study addresses the problem of single-channel Automatic Speech Recognition of a target speaker within an overlap speech scenario. In the proposed method, the hidden representations in the acoustic model are modulated by speaker auxiliary  information to recognize only the desired speaker. Affine transformation layers are inserted into the acoustic model network to integrate speaker information with the acoustic features. The speaker conditioning process allows the acoustic model to perform computation in the context of target-speaker auxiliary information. The proposed speaker conditioning method is a general approach and can be applied to any acoustic model architecture. Here, we employ speaker conditioning on a ResNet acoustic model. Experiments on the WSJ corpus show that the proposed speaker conditioning method is an effective solution to fuse speaker auxiliary information with acoustic features for multi-speaker speech recognition, achieving +9\% and +20\% relative WER reduction for clean and overlap speech scenarios, respectively, compared to the original ResNet acoustic model baseline. 

%The affine transformation is parametrized by shifting and scaling coefficients generated based on speaker-specific embedding. 
\end{abstract}
%
\begin{keywords}
 Affine transformation, overlap speech recognition, feature-wise linear modulation, multi-speaker recognition, acoustic modeling
\end{keywords}
%
\section{Introduction}
\label{sec:intro}

%The introduction of deep learning has lead to significant performance improvements in recent single-speaker automatic speech recognition systems \cite{yu2016automatic,li2015robust,vaswani2017attention,chiu2018state}. 

Multi-talker speech recognition is focused on recognizing individual speech sources from overlap speech, and is one main challenge for current ASR systems \cite{chang2020end,watanabe2020chime,yousefi21_interspeech, yousefi2018assessing,barker2018fifth,qian2018past,mirsamadi2019multi,yousefi2020block}. 
%In applications where multi-channel speech recordings are available, spatial information derived from the inter-channel differences \cite{chang2019mimo,chen2018multi,subramanian2021directional}, Direction of Arrival (DoF) \cite{dey2018direction,seltzer2003microphone}, and other beamforming techniques \cite{seltzer2004likelihood,yoshioka2018multi,cho2021convolutional} can help distinguish between speech sources, which makes the problem easier to solve. 
Current solutions for multi-speaker speech recognition  can be categorized into two main approaches: \emph{(i)} performing front-end speech processing based on separation on the overlap speech, then applying ASR to the separated speech signals \cite{boeddeker2018front,yousefi2016supervised,yousefi2020frame,deng2011front, narayanan2014investigation,mirsamadi2014multichannel, yousefi2019probabilistic}; or \emph{(ii)} skipping the explicit separation step and developing a multi-speaker speech recognition system directly using either hybrid \cite{kanda2019acoustic,weng2015deep,kanda2019simultaneous} or end-to-end \cite{seki2018purely,lu2021streaming} ASR frameworks. Recently, an end-to-end multi-speaker speech recognition system was proposed based on Transformers \cite{chang2020end}. This approach achieved considerable improvement at the expense of more computational cost for a reasonable temporal resolution. In another study \cite{chen2017progressive}, overlap speech was considered as a mismatch condition of the clean speech recognition scenario, and teacher-student training was employed for transfer learning from clean to overlap speech. The main drawback of this approach is requiring training sets with parallel clean and overlapped speech, which is difficult to collect in real-world applications \cite{denisov2019end}.
%Authors of \cite{delcroix2018single} proposed alternate SpeakerBeam configurations for target speaker extraction and recognition in single-channel recordings by joint training of two separate networks. However, as reported in \cite{vzmolikova2019speakerbeam} their proposed speaker-aware acoustic model is more effective for multi-channel recordings while it lags behind in single-channel scenario.
%former approaches such as Permutation Invariant Training (PIT) \cite{kolbaek2017multitalker} and Deep Clustering (DP) \cite{isik2016single} in single-channel scenario.
Recently, several studies \cite{kanda2019acoustic,wang2019end,subramanian2020far} have used speaker-specific embeddings to learn a frame-level mask for the target speaker which suppresses interfering speech. Although these approaches use the additional speaker-specific information to guide the ASR system, their main limitation is that they assume only one speaker is active in each Time-Frequency bin. 


To address the challenges of single-channel multi-speaker speech recognition, in this study, we focus on speaker conditioning of the Acoustic Model (AM) by performing an affine transformation.
%which integrates the speaker information with the acoustic features to improve ASR performance. 
In contrast to former approaches which employ speaker embedding to estimate speaker-specific masks, we propose to use speaker embedding to compute parameters of the affine transformation, allowing the acoustic model to conduct its computation in the context of the desired speaker auxiliary information. The proposed speaker conditioning method is a general approach and can be applied to any AM architecture. In this paper, we employ  speaker conditioning on a ResNet acoustic model in hybrid DNN-HMM setup. Experiments are performed on WSJ corpus, achieving +9\% and +20\% relative WER reduction for clean and overlap speech scenarios, compared to the original ResNet acoustic model. The contributions of this paper are threefold:
\begin{itemize}
    \item Proposing speaker conditioning of the ResNet acoustic model using an affine transformation.
    \item Comparing the proposed method with alternate feature-wise  acoustic model transformations such as conditional biasing and middle feature-map fusion.
    \item Evaluating the performance of the proposed speaker conditioned ASR trained on an alternate input feature called Wav2Vec representation. 
\end{itemize}


The remainder of this paper is organized as follows. In Sec.\ref{sec:sys}, the problem is outlined and proposed method described. Sec.\ref{sec:exp}, presents experiments and results. Finally the conclusions are discussed in Sec.\ref{sec:con}. 

\begin{figure*}[t]
\centering
\vspace{-0.9cm}
\includegraphics[width=\linewidth]{models.png}
\vspace{-0.7cm}
\caption{The proposed speaker conditioned ResNet18 acoustic model using Affine Transformation (AT) blocks.}
\label{fig:model}
\vspace{-0.2cm}
\end{figure*}


\section{Single-channel multi-speaker ASR}
\label{sec:sys}

Multi-speaker speech recognition can gain substantial improvement by deploying other sources of information such as speaker identity in addition to acoustic features \cite{denisov2019end, wang2019end}. However, designing an efficient method to fuse a combination of multiple sources (i.e., acoustic features and speaker embeddings) to obtain higher quality and improved information is still a challenging task. Additionally, capturing complex interactions between multiple sources should maintain a balanced compromise between model/network computational cost and performance. A popular method to address this problem is to use a feature-wise transformation \cite{perez2018film} which can model the complex relation between speaker-specific characteristics and acoustic features in a multi-speaker speech scenario to identify and recognize the desired speaker in the mixed speech recording. This transformation can be performed in several manners such as \emph{conditional biasing}, \emph{conditional scaling}, and \emph{conditional affine transformation}. In this section, we focus on a conditional affine transformation  which is a more general approach. The aforementioned conditional biasing and scaling are two specific examples of conditional affine transformation. 

\subsection{Conditional affine transformation} Affine Transformation (AT) influences the output of the acoustic model network by applying a linear modulation to the network's intermediate features. This modulation is parameterized by shifts and coefficients obtained based on speaker-specific embedding. Let $x$ be a context-expanded window of acoustic features for overlap speech, and $y_s$ be a phoneme label or a senone alignment (i.e., from GMM-HMM) for the target-speaker speech signal. DNN acoustic models are used estimate the posterior probability as:
\begin{equation}
   p(y_s|x, s) = DNN(x, z_s),
\end{equation}
where DNN is typically trained to maximize the log probability of the phoneme alignment or minimize the cross-entropy error, and $s$ is the target speaker with an x-vector \cite{snyder2018x} embedding $z_s$. In this study, the original ResNet18 model is considered as our baseline. Next, affine transformation layers are inserted into ResNet18 network to build the speaker conditioned acoustic model. The scale and bias factors of AT are  estimated by a two-layer fully connected network $h$ based on x-vector $z_s$ as:
\begin{equation}
    (\alpha_{i,c},\; \beta_{i,c}) = h(z_s)
\end{equation}
where  $i$ and $c$ refer to the $i$-th data sample in the minibatch, and the $c$-th channel feature map. Once $\alpha_{i,c}$ and $\beta_{i,c}$ are estimated, they are used to modulate the ResNet's intermediate activations $F_{i,c}$ as:
\begin{equation}
      AT(F_{i,c}^{l}| \alpha_{i,c}, \beta_{i,c}, F_{i,c}^{l-1}) = \alpha_{i,c} \odot  F_{i,c}^{l-1} + \beta_{i,c}
\end{equation}
where $AT$ and $l$ represent the Affine Transformation, and network's layer. The proposed speaker conditioned ResNet18 is shown in Fig.\ref{fig:model}.  The speaker embedding x-vector is submitted to the network $h$ to estimate a $[B, 1920]$ matrix which is $(\alpha_{i,c}$, $\beta_{i,c})$ pairs of AT layers. Each AT layer receives two inputs:  the previous layer output, and the $(\alpha_{i,c}$, $\beta_{i,c})$ pair. The dimension of $\alpha_{i,c}$ and $\beta_{i,c}$ is $[B, C]$ each. In the AT layer, each channel of the extracted feature map is scaled by $\alpha_{i,c}$ and shifted by $\beta_{i,c}$ to modulate the feature-map distribution of activations based on the target-speaker embedding.




\begin{table}[h]
\vspace{-0.3cm}
\caption{Comparing our ResNet18 acoustic model baseline with other approaches on WSJ (WER in \%).}
\vspace{-0.3cm}
\begin{minipage}{0.45\textwidth}
\centering
\begin{tabular}{@{\extracolsep{0.1pt}}lc c c} 
\\\hline 
\hline \\
\textit{System}  & \multicolumn{1}{c}{Dev-93} & \multicolumn{1}{c}{Eval-92} \\ 
\hline 
Lee et al. 2021 \cite{lee2021intermediate} &  12.0 & 9.9\\
Higuchi et al. 2020 \cite{higuchi2020mask} &15.4 &	12.1\\
Chi et al. 2020 \cite{chi2020align} & 13.7&	11.4\\
Rouhe et al. 2020 \cite{rouhe2020speaker} & 13.2 &	9.3\\
Sabour et al. 2018 \cite{sabour2018optimal} &- &	9.3\\
Borgholt et al. 2020 \cite{borgholt2020end} &- &	9.3\\
Park et al. 2019 \cite{lee2019simple} & - &	7.8\\
\hline
Our baseline & 12.1	& 7.9

\end{tabular} 
\end{minipage}
\vspace{-0.5cm}
\label{tab:base}
\end{table}











\begin{table*}[t]
\setlength{\extrarowheight}{4pt}
\begin{center}
\vspace{-0.4cm}
\caption{ WER of the proposed speaker conditioned ResNet18 acoustic model with Affine Transformation (AT) in different settings. Each experiment is repeated three times and the average WER is reported.}
%\vspace{-0.2cm}

\begin{tabular}{ |c|c|c|c|c|c|c|c|c|c| }
\cline{2-10}

\multicolumn{1}{c|}{} & \multicolumn{6}{c|}{Simulated overlap speech based on Dev-93} &\multicolumn{3}{c|}{Clean speech} \\ 
\cline{3-6} \cline{7-9} 
\hline
Acoustic model & 0dB & 5dB & 10dB & 15dB & 20dB & 25dB & Dev-93 & Eval-92 & Eval-93 \\
\hline
ResNet18 (baseline) & 65.06&58.29&47.03&34.72&24.36&17.62&12.14&7.92&10.81 \\
ResNet18 + AT (proposed) &63.83&	55.83&	43.30&	29.90&	20.34&	15.15&	11.50&	7.66&	9.64\\
\hline
ResNet18 + AT (bias=0) &63.69&	55.23&	43.57&	30.15&	19.95&	15.19&	11.75&	7.66&	9.57\\
ResNet18 + AT (scale=1)& 65.10& 57.66&	46.39&	32.79&	22.29&	16.65&	12.25&	7.91& 10.57\\
ResNet18 + AT (sigmoid(scale))& 64.63&	56.66&	45.33&	31.78&	21.680&	16.250&	11.993&	7.803&	10.263\\
%ResNet18 + AT (ReLU(scale))& 64.087&	56.197&	44.700&	31.077&	20.617&	15.670&	11.903&	7.700&	10.033\\
ResNet18 + AT (tanh(scale))& 64.53&	56.82&	45.27&	31.64&	21.19&	16.02&	11.99&	7.72&	9.97\\
\hline
ResNet18 + AT (Block1)& 63.68&	55.33&	\textbf{43.31}&29.44&	\textbf{19.51}&\textbf{14.67}&11.56&\textbf{7.49}&	\textbf{9.85}\\
ResNet18 + AT (Block 1-2)&\textbf{63.50}&	\textbf{55.19}&	43.60&	\textbf{29.33}&	19.79&	14.85&	\textbf{11.33}&	7.50&	9.88\\
ResNet18 + AT (Block 1-3)&63.51&	55.29&	43.35&	29.65&	19.87&	14.88&	11.49&	7.55&	9.93\\
ResNet18 + AT (Block 4)& 64.66&	57.66&	47.28&	33.86&	23.27&	16.54&	11.64&	8.19&	10.47\\
\hline
\end{tabular}
\vspace{-0.4cm}
\label{tab:at}
\end{center}
\end{table*}




 %Since this is a difficult optimization problem, we propose to integrate speaker embedding x-vector to leverage the additional information to model the phonemes as:
% \begin{equation}
%     p(y_s|x, s) = DNN_{AI}(x, z_s),
% \end{equation}
% where $DNN_{AI}$ is the Audio-Identity model and $s$ is the target speaker with x-vector embedding $z_s$. 








\section{Experiments and results}
\label{sec:exp}
In this section, we investigate the performance of the proposed speaker conditioning method presented in Fig.\ref{fig:model} on WSJ corpus. 
%Additionally the effect of $\alpha_{i,c}$ and $\beta_{i,c}$  on the final Word-Error-Rate (WER) are studied separately. Also, different scale ranges are considered for the scaling parameter $\alpha_{i,c}$ to understand its role in influencing the performance of the acoustic model. The speaker conditioned acoustic model is compared against several fusion methods such as conditional biasing fusion and middle fusion. Finally, the input MFCC acoustic features are replaced by noise-invariant Wav2Vec representation to train the speaker conditioned acoustic model to further boost the WER. 
In order to conduct the experiments, clean \textit{tr-si284} is used in the training phase for all acoustic models. We partitioned \textit{tr-si284} into a training set $(90\%)$ and a held-out cross-validation set $(10\%)$. ASR performance for different acoustic models are reported in terms of Word-Error-Rate (WER) on clean \textit{dev-93}, \textit{eval-93}, and \textit{eval-92}. Additionally, overlap speech is generated based on \textit{dev-93} by selecting random utterances from random speakers and adding them with Signal-to-Interference Ratio (SIR) ranging from $0$ to $25$dB with increments of $5$dB. The baseline acoustic model is ResNet18 with $3400$ output senones.  The  network parameters  are  updated  by  the  gradients  of the cross entropy loss using Stochastic Gradient Descent (SGD) optimizer with a momentum of $0.9$ and initial learning rate $0.01$. The training process is completed by performing early stopping \cite{zhang2016understanding}. The maximum number of epochs is set to $100$, batch size  $1024$ context-expanded frames; learning rate is decreased by $50\%$ if the cv loss improvement is less than $0.01$ for $3$ successive epochs. The early stopping is performed if no improvement is observed on the cv loss once the learning rate has decayed $6$ times. 13-dim MFCC computed over a $25ms$ window with $10ms$ shift with a $20$ frame context ($10$ frames on each side) is used for training the acoustic model. Consistent with the standard Kaldi recipe for WSJ, we use the trigram language models provided by LDC for WSJ data. In order to minimize the effect of parameter initialization on the acoustic model and final WER, we repeat each experiment three times with different initial parameters.

\begin{table*}[t]
\setlength{\extrarowheight}{4pt}
\begin{center}
\vspace{-0.6cm}
\caption{ WER of the proposed speaker conditioned ResNet18 based on Affine Transformation (AT) compared to other fusion techniques. Each experiment is repeated three times and the average WER is reported.}
\begin{tabular}{ |c|c|c|c|c|c|c|c|c|c| }

\cline{2-10}

\multicolumn{1}{c|}{} & \multicolumn{6}{c|}{Simulated overlap speech based on Dev-93} &\multicolumn{3}{c|}{Clean speech} \\ 
\cline{3-6} \cline{7-9} 
\hline
Signal-to-Interference Ratio & 0dB & 5dB & 10dB & 15dB & 20dB & 25dB & Dev-93 & Eval-92 & Eval-93 \\
\hline
ResNet18 + Conditional biasing  & 64.66&	57.20&	45.74&	32.75&	23.54&	17.87&	13.10&	8.44&	11.22\\
ResNet18 + Middle fusion & 63.94&	57.51&	47.81&	34.02&	23.19&	16.68&	11.81&	8.24&	10.64 \\
\textbf{ResNet18 + AT (proposed)}& \textbf{63.68}&	\textbf{55.33}&	\textbf{43.31}&\textbf{29.44}&	\textbf{19.51}&\textbf{14.67}&\textbf{11.56}&\textbf{7.49}&	\textbf{9.85}\\
\hline
\end{tabular}
\label{tab:com}
\end{center}
\end{table*}




\begin{figure*}
\centering
\begin{tabular}{cc}
\includegraphics[height=3.8cm,width=7cm]{base.png}&
\hspace{1.5cm}
\includegraphics[height=3.8cm,width=7cm]{at.png} 
\end{tabular}
\vspace{-0.2cm}
\label{figur}\caption{ WER of ResNet18 baseline and proposed ResNet18 + AT trained on MFCC and Wav2Vec input features.}
\vspace{-0.2cm}
\label{fig:wav2vec}
\end{figure*}


Performance of the ResNet18 baseline is compared with recent studies in Table \ref{tab:base}. The main purpose of this comparison is to ensure that our baseline achieves a competitive performance compared to recent studies, and it is seen we have a strong starting point for further developing our proposed speaker conditioning technique.  There are other approaches that leverage transfer learning, semi supervised learning, or more advanced language models to achieve further improvement. However, since we focus on speaker conditioning of acoustic model, we train our ResNet18 acoustic model only on \textit{tr-si284}, and use Kaldi for training the language model. The first row of Table \ref{tab:at} presents performance of the baseline on overlap speech, which is severely degraded. Therefore, we build on the ResNet18 acoustic model baseline and apply our Affine Transformation (AT) layers as depicted in Fig.\ref{fig:model}. The results for speaker conditioned ResNet using AT are presented in the second row of Table \ref{tab:at} which shows $+2\%$ relative improvement for severe overlap speech recordings (i.e., $0$dB) and an average of $+5\%$ relative improvement on clean test sets. Since AT effectively performs speaker-adaptation, the trained acoustic model is tuned to the target speaker, therefore, it achieves better performance even on the clean test sets. The maximum relative improvement is achieved for input SIR $20\%$ in which the level of overlap speech is neither too severe nor too easy for the acoustic model; therefore, the  target-speaker auxiliary information can be very helpful in improving performance.  

Moreover, the effect of $\alpha$ and $\beta$ is studied separately by setting $\alpha=1$ and $\beta=0$. The result in Table \ref{tab:at} manifest that the effectiveness of the conditional Affine Transformation can be mainly attributed to the scale coefficient rather than the shift parameter. Therefore, we further investigate the effect of  $\alpha$ by restricting its value to $(0,1)$ using Sigmoid, and $(-1,1)$ using $\tanh$ function. Nevertheless, the results reported in Table \ref{tab:at} reveal that unrestricted $\alpha$ achieves better performance which may be due to the flexibility it provides for the network to learn the range that best suits the data. So far, the AT layers have been applied to all ResNet18 blocks (each dashed rectangular in Fig. \ref{fig:model} is considered as a block). To find the best network depth in which AT layers are most effective, several experiments are conducted with AT only applied to specific individual blocks. Based on these experiments, the AT layers are most effective when applied only to the first block (block1), and least effective when only applied to the last block (block4). However, applying  AT layers to the first two blocks (block1-2) and the first three blocks (block 1-3) did not improve ASR performance, while it differently adds computational cost. To summarize our findings based on the experiments, unrestricted-scale Affine Transformation applied to the initial blocks of the ResNet18 acoustic model achieves the best overall results while simultaneously maintaining the lowest computational cost. 
%This acoustic model is highlighted in bold in Table \ref{tab:at}.

Next, the proposed method is compared with other speaker conditioning techniques in Table \ref{tab:com}. Conditional biasing refers to adding speaker information (x-vector) as a bias to the acoustic features in the first hidden layer. Middle fusion refers to adding the x-vector to the intermediate extracted feature map after the second block. Therefore, the intermediate feature map is conditioned before entering block 3 for extracting further higher-level features adapted to the target speaker. As shown in Table \ref{tab:com}, the proposed speaker conditioning based on Affine Transformation outperforms all other approaches in both clean and overlap speech scenarios. 

So far, the focus of this study has been on designing the acoustic model. However,  performance of the acoustic model can further improve by deploying more robust input features other than MFCC. In the final section, we evaluate the proposed method trained on noise-invariant Wav2Vec features \cite{schneider2019wav2vec}. Wav2Vec representation has been trained on large amounts of unlabeled audio data in an unsupervised manner. Fig. \ref{fig:wav2vec} (left) shows the WER of the baseline ResNet18 trained on MFCC and Wav2Vec features, which manifests the effectiveness of Wav2Vec in reducing the WER across all test sets in the absence of speaker auxiliary information. The highest improvement is achieved for overlap speech with SIR $15$dB, which is $+11\%$ absolute improvement in WER. Fig.\ref{fig:wav2vec} (right) depicts the WER of the proposed speaker conditioned ResNet18 trained on MFCC and Wav2Vec. Similar to the baseline, the speaker conditioned acoustic model benefits from the Wav2Vec features by achieving $+6\%$ absolute improvement in WER for SIR $15$dB. However, due to the availability of speaker information, the acoustic model is less sensitive to the robustness of the input acoustic features, and thus, the amount of improvement from Wav2Vec is less in the proposed speaker conditioned ResNet18 compared to the baseline. In conclusion, the WER across all test sets is improved by using the proposed speaker conditioned acoustic model trained on wav2Vec. For example, on the overlap speech test set with SIR $15$dB, the proposed ResNet18 with Affine Transformation trained on Wav2Vec gains +33\% relative (+11\% absolute) improvement in WER compared to the original ResNet trained on MFCC.


\section{Conclusion}
\label{sec:con}

In this study, we proposed a speaker conditioning method for acoustic modeling in multi-speaker speech recognition. In the proposed method, Affine Transformation layers are inserted into the acoustic model architecture to fuse speaker-specific information with the acoustic model. The proposed speaker conditioned acoustic model  was compared with  other fusion techniques such as early fusion of speaker embedding and middle feature-map fusion. Additionally, the performance of the proposed method was evaluated on alternate input features called Wav2Vec. The results on WSJ corpus clearly demonstrate that the proposed speaker conditioned acoustic model based on affine transformation achieves consistent WER improvement for clean and overlap speech scenarios. 


% Below is an example of how to insert images. Delete the ``\vspace'' line,
% uncomment the preceding line ``\centerline...'' and replace ``imageX.ps''
% with a suitable PostScript file name.
% -------------------------------------------------------------------------
% \begin{figure}[htb]

% \begin{minipage}[b]{1.0\linewidth}
%   \centering
%   \centerline{\includegraphics[width=8.5cm]{image1}}
% %  \vspace{2.0cm}
%   \centerline{(a) Result 1}\medskip
% \end{minipage}
% %
% \begin{minipage}[b]{.48\linewidth}
%   \centering
%   \centerline{\includegraphics[width=4.0cm]{image3}}
% %  \vspace{1.5cm}
%   \centerline{(b) Results 3}\medskip
% \end{minipage}
% \hfill
% \begin{minipage}[b]{0.48\linewidth}
%   \centering
%   \centerline{\includegraphics[width=4.0cm]{image4}}
% %  \vspace{1.5cm}
%   \centerline{(c) Result 4}\medskip
% \end{minipage}
% %
% \caption{Example of placing a figure with experimental results.}
% \label{fig:res}
% %
% \end{figure}


% To start a new column (but not a new page) and help balance the last-page
% column length use \vfill\pagebreak.
% -------------------------------------------------------------------------
%\vfill
%\pagebreak



% 

% References should be produced using the bibtex program from suitable
% BiBTeX files (here: strings, refs, manuals). The IEEEbib.bst bibliography
% style file from IEEE produces unsorted bibliography list.
% -------------------------------------------------------------------------
\bibliographystyle{IEEEbib}
%\bibliography{refs}
%S\bibliographystyle{IEEEtran}
\bibliography{refs}
\end{document}


\label{lastpage}

\end{document}
