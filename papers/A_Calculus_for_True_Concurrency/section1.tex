\section{Introduction}\label{int}

Parallelism and concurrency \cite{CM} are the core concepts within computer science. There are mainly two camps in capturing concurrency: the interleaving concurrency and the true concurrency.

The representative of interleaving concurrency is bisimulation/weak bisimulation equivalences. CCS (A Calculus of Communicating Systems) \cite{CCS} \cite{CC} is a calculus based on bisimulation semantics model. CCS has good semantic properties based on the interleaving bisimulation. These properties include monoid laws, static laws, new expansion law for strongly interleaving bisimulation, $\tau$ laws for weakly interleaving bisimulation, and full congruences for strongly and weakly interleaving bisimulations, and also unique solution for recursion.

The other camp of concurrency is true concurrency. The researches on true concurrency are still active. Firstly, there are several truly concurrent bisimulations, the representatives are: pomset bisimulation, step bisimulation, history-preserving (hp-) bisimulation, and especially hereditary history-preserving (hhp-) bisimulation \cite{HHP1} \cite{HHP2}. These truly concurrent bisimulations are studied in different structures \cite{ES1} \cite{ES2} \cite{CM}: Petri nets, event structures, domains, and also a uniform form called TSI (Transition System with Independence) \cite{SFL}. There are also several logics based on different truly concurrent bisimulation equivalences, for example, SFL (Separation Fixpoint Logic) and TFL (Trace Fixpoint Logic) \cite{SFL} are extensions on true concurrency of mu-calculi \cite{MUC} on bisimulation equivalence, and also a logic with reverse modalities \cite {RL1} \cite{RL2} based on the so-called reverse bisimulations with a reverse flavor. Recently, a uniform logic for true concurrency \cite{LTC1} \cite{LTC2} was represented, which used a logical framework to unify several truly concurrent bisimulations, including pomset bisimulation, step bisimulation, hp-bisimulation and hhp-bisimulation.

There are simple comparisons between HM logic and bisimulation, as the uniform logic \cite{LTC1} \cite{LTC2} and truly concurrent bisimulations; the algebraic laws \cite{ALNC}, ACP \cite{ACP} and bisimulation, as the algebraic laws APTC \cite{ATC} and truly concurrent bisimulations; CCS and bisimulation, as truly concurrent bisimulations  and \emph{what}, which is still missing.

In this paper, we design a calculus for true concurrency (CTC) following the way paved by CCS for bisimulation equivalence. This paper is organized as follows. In section \ref{bac}, we introduce some preliminaries, including a brief introduction to CCS, and also preliminaries on true concurrency. We introduce the syntax and operational semantics of CTC in section \ref{sos}, its properties for strongly truly concurrent bisimulations in section \ref{stcb}, its properties for weakly truly concurrent bisimulations in section \ref{wtcb}. In section \ref{app}, we show the applications of CTC by an example called alternating-bit protocol. Finally, in section \ref{con}, we conclude this paper. 