\section{Weakly Truly Concurrent Bisimulations}\label{wtcb}

\subsection{Basic Definitions}\label{WTCC}

In this subsection, we introduce several weakly truly concurrent bisimulation equivalences, including weak pomset bisimulation, weak step bisimulation, weak history-preserving (hp-)bisimulation and weakly hereditary history-preserving (hhp-)bisimulation.

\begin{definition}[Weak pomset transitions and weak step]\label{WPT}
Let $\mathcal{E}$ be a PES and let $C\in\mathcal{C}(\mathcal{E})$, and $\emptyset\neq X\subseteq \hat{\mathbb{E}}$, if $C\cap X=\emptyset$ and $\hat{C'}=\hat{C}\cup X\in\mathcal{C}(\mathcal{E})$, then $C\xRightarrow{X} C'$ is called a weak pomset transition from $C$ to $C'$, where we define $\xRightarrow{e}\triangleq\xrightarrow{\tau^*}\xrightarrow{e}\xrightarrow{\tau^*}$. And $\xRightarrow{X}\triangleq\xrightarrow{\tau^*}\xrightarrow{e}\xrightarrow{\tau^*}$, for every $e\in X$. When the events in $X$ are pairwise concurrent, we say that $C\xRightarrow{X}C'$ is a weak step.
\end{definition}

\begin{definition}[Weak pomset, step bisimulation]\label{WPSB}
Let $\mathcal{E}_1$, $\mathcal{E}_2$ be PESs. A weak pomset bisimulation is a relation $R\subseteq\mathcal{C}(\mathcal{E}_1)\times\mathcal{C}(\mathcal{E}_2)$, such that if $(C_1,C_2)\in R$, and $C_1\xRightarrow{X_1}C_1'$ then $C_2\xRightarrow{X_2}C_2'$, with $X_1\subseteq \hat{\mathbb{E}_1}$, $X_2\subseteq \hat{\mathbb{E}_2}$, $X_1\sim X_2$ and $(C_1',C_2')\in R$, and vice-versa. We say that $\mathcal{E}_1$, $\mathcal{E}_2$ are weak pomset bisimilar, written $\mathcal{E}_1\approx_p\mathcal{E}_2$, if there exists a weak pomset bisimulation $R$, such that $(\emptyset,\emptyset)\in R$. By replacing weak pomset transitions with weak steps, we can get the definition of weak step bisimulation. When PESs $\mathcal{E}_1$ and $\mathcal{E}_2$ are weak step bisimilar, we write $\mathcal{E}_1\approx_s\mathcal{E}_2$.
\end{definition}

\begin{definition}[Weakly posetal product]
Given two PESs $\mathcal{E}_1$, $\mathcal{E}_2$, the weakly posetal product of their configurations, denoted $\mathcal{C}(\mathcal{E}_1)\overline{\times}\mathcal{C}(\mathcal{E}_2)$, is defined as

$$\{(C_1,f,C_2)|C_1\in\mathcal{C}(\mathcal{E}_1),C_2\in\mathcal{C}(\mathcal{E}_2),f:\hat{C_1}\rightarrow \hat{C_2} \textrm{ isomorphism}\}.$$

A subset $R\subseteq\mathcal{C}(\mathcal{E}_1)\overline{\times}\mathcal{C}(\mathcal{E}_2)$ is called a weakly posetal relation. We say that $R$ is downward closed when for any $(C_1,f,C_2),(C_1',f,C_2')\in \mathcal{C}(\mathcal{E}_1)\overline{\times}\mathcal{C}(\mathcal{E}_2)$, if $(C_1,f,C_2)\subseteq (C_1',f',C_2')$ pointwise and $(C_1',f',C_2')\in R$, then $(C_1,f,C_2)\in R$.


For $f:X_1\rightarrow X_2$, we define $f[x_1\mapsto x_2]:X_1\cup\{x_1\}\rightarrow X_2\cup\{x_2\}$, $z\in X_1\cup\{x_1\}$,(1)$f[x_1\mapsto x_2](z)=
x_2$,if $z=x_1$;(2)$f[x_1\mapsto x_2](z)=f(z)$, otherwise. Where $X_1\subseteq \hat{\mathbb{E}_1}$, $X_2\subseteq \hat{\mathbb{E}_2}$, $x_1\in \hat{\mathbb{E}}_1$, $x_2\in \hat{\mathbb{E}}_2$. Also, we define $f(\tau^*)=f(\tau^*)$.
\end{definition}

\begin{definition}[Weak (hereditary) history-preserving bisimulation]\label{WHHPB}
A weak history-preserving (hp-) bisimulation is a weakly posetal relation $R\subseteq\mathcal{C}(\mathcal{E}_1)\overline{\times}\mathcal{C}(\mathcal{E}_2)$ such that if $(C_1,f,C_2)\in R$, and $C_1\xRightarrow{e_1} C_1'$, then $C_2\xRightarrow{e_2} C_2'$, with $(C_1',f[e_1\mapsto e_2],C_2')\in R$, and vice-versa. $\mathcal{E}_1,\mathcal{E}_2$ are weak history-preserving (hp-)bisimilar and are written $\mathcal{E}_1\approx_{hp}\mathcal{E}_2$ if there exists a hp-bisimulation $R$ such that $(\emptyset,\emptyset,\emptyset)\in R$.

A weakly hereditary history-preserving (hhp-)bisimulation is a downward closed weak hp-bisimulation. $\mathcal{E}_1,\mathcal{E}_2$ are weakly hereditary history-preserving (hhp-)bisimilar and are written $\mathcal{E}_1\approx_{hhp}\mathcal{E}_2$.
\end{definition}

\begin{proposition}[Weakly concurrent behavioral equivalence]\label{WSCBE}
(Strongly) concurrent behavioral equivalences imply weakly concurrent behavioral equivalences. That is, $\sim_p$ implies $\approx_p$, $\sim_s$ implies $\approx_s$, $\sim_{hp}$ implies $\approx_{hp}$, $\sim_{hhp}$ implies $\approx_{hhp}$.
\end{proposition}

\begin{proof}
From the definition of weak pomset transition, weak step transition, weakly posetal product and weakly concurrent behavioral equivalence, it is easy to see that $\xrightarrow{e}=\xrightarrow{\epsilon}\xrightarrow{e}\xrightarrow{\epsilon}$ for $e\in \mathbb{E}$, where $\epsilon$ is the empty event.
\end{proof}

The weak transition rules for CTC are listed in Table \ref{WTRForCTC}.

\begin{center}
    \begin{table}
        \[\textbf{WAct}_1\quad \frac{}{\alpha.P\xRightarrow{\alpha}P}\]

        \[\textbf{WSum}_1\quad \frac{P\xRightarrow{\alpha}P'}{P+Q\xRightarrow{\alpha}P'}\]

        \[\textbf{WCom}_1\quad \frac{P\xRightarrow{\alpha}P'\quad Q\nrightarrow}{P\parallel Q\xRightarrow{\alpha}P'\parallel Q}\]

        \[\textbf{WCom}_2\quad \frac{Q\xRightarrow{\alpha}Q'\quad P\nrightarrow}{P\parallel Q\xRightarrow{\alpha}P\parallel Q'}\]

        \[\textbf{WCom}_3\quad \frac{P\xRightarrow{\alpha}P'\quad Q\xRightarrow{\beta}Q'}{P\parallel Q\xRightarrow{\{\alpha,\beta\}}P'\parallel Q'}\quad (\beta\neq\overline{\alpha})\]

        \[\textbf{WCom}_4\quad \frac{P\xRightarrow{l}P'\quad Q\xRightarrow{\overline{l}}Q'}{P\parallel Q\xRightarrow{\tau}P'\parallel Q'}\]

        \[\textbf{WAct}_2\quad \frac{}{(\alpha_1\parallel\cdots\parallel\alpha_n).P\xRightarrow{\{\alpha_1,\cdots,\alpha_n\}}P}\quad (\alpha_i\neq\overline{\alpha_j}\quad i,j\in\{1,\cdots,n\})\]

        \[\textbf{WSum}_2\quad \frac{P\xRightarrow{\{\alpha_1,\cdots,\alpha_n\}}P'}{P+Q\xRightarrow{\{\alpha_1,\cdots,\alpha_n\}}P'}\]

        \[\textbf{WRes}_1\quad \frac{P\xRightarrow{\alpha}P'}{P\setminus L\xRightarrow{\alpha}P'\setminus L}\quad (\alpha,\overline{\alpha}\notin L)\]

        \[\textbf{WRes}_2\quad \frac{P\xRightarrow{\{\alpha_1,\cdots,\alpha_n\}}P'}{P\setminus L\xRightarrow{\{\alpha_1,\cdots,\alpha_n\}}P'\setminus L}\quad (\alpha_1,\overline{\alpha_1},\cdots,\alpha_n,\overline{\alpha_n}\notin L)\]

        \[\textbf{WRel}_1\quad \frac{P\xRightarrow{\alpha}P'}{P[f]\xRightarrow{f(\alpha)}P'[f]}\]

        \[\textbf{WRel}_2\quad \frac{P\xRightarrow{\{\alpha_1,\cdots,\alpha_n\}}P'}{P[f]\xRightarrow{\{f(\alpha_1),\cdots,f(\alpha_n)\}}P'[f]}\]

        \[\textbf{WCon}_1\quad\frac{P\xRightarrow{\alpha}P'}{A\xRightarrow{\alpha}P'}\quad (A\overset{\text{def}}{=}P)\]

        \[\textbf{WCon}_2\quad\frac{P\xRightarrow{\{\alpha_1,\cdots,\alpha_n\}}P'}{A\xRightarrow{\{\alpha_1,\cdots,\alpha_n\}}P'}\quad (A\overset{\text{def}}{=}P)\]
        \caption{Weak transition rules of CTC}
        \label{WTRForCTC}
    \end{table}
\end{center}

\subsection{Laws and Congruence}

Remembering that $\tau$ can neither be restricted nor relabeled, by Proposition \ref{WSCBE}, we know that the monoid laws, the static laws and the new expansion law in section \ref{stcb} still hold with respect to the corresponding weakly truly concurrent bisimulations. And also, we can enjoy the full congruence of Prefix, Summation, Composition, Restriction, Relabelling and Constants with respect to corresponding weakly truly concurrent bisimulations. We will not retype these laws, and just give the $\tau$-specific laws.

\begin{proposition}[$\tau$ laws for weak step bisimulation]\label{TAUWSB}
The $\tau$ laws for weak step bisimulation is as follows.
\begin{enumerate}
  \item $P\approx_s \tau.P$;
  \item $\alpha.\tau.P\approx_s \alpha.P$;
  \item $(\alpha_1\parallel\cdots\parallel\alpha_n).\tau.P\approx_s (\alpha_1\parallel\cdots\parallel\alpha_n).P$;
  \item $P+\tau.P\approx_s \tau.P$;
  \item $\alpha.(P+\tau.Q)+\alpha.Q\approx_s\alpha.(P+\tau.Q)$;
  \item $(\alpha_1\parallel\cdots\parallel\alpha_n).(P+\tau.Q)+ (\alpha_1\parallel\cdots\parallel\alpha_n).Q\approx_s (\alpha_1\parallel\cdots\parallel\alpha_n).(P+\tau.Q)$;
  \item $P\approx_s \tau\parallel P$.
\end{enumerate}
\end{proposition}

\begin{proof}
Though transition rules in Table \ref{TRForCTC} are defined in the flavor of single event, they can be modified into a step (a set of events within which each event is pairwise concurrent), we omit them. If we treat a single event as a step containing just one event, the proof of $\tau$ laws does not exist any problem, so we use this way and still use the transition rules in Table \ref{TRForCTC}.

\begin{enumerate}
  \item $P\approx_s \tau.P$. By the weak transition rules $\textbf{WAct}_{1,2}$ of CTC in Table \ref{WTRForCTC}, we get

  $$\frac{P\xRightarrow{\alpha}P'}{P\xRightarrow{\alpha}P'}
  \quad \frac{P\xRightarrow{\alpha}P'}{\tau.P\xRightarrow{\alpha} P'}$$

  Since $P'\approx_s P'$, we get $P\approx_s \tau.P$, as desired.
  \item $\alpha.\tau.P\approx_s \alpha.P$. By the weak transition rules $\textbf{WAct}_{1,2}$ in Table \ref{WTRForCTC}, we get

  $$\frac{}{\alpha.\tau.P\xRightarrow{\alpha}P}
  \quad \frac{}{\alpha.P\xRightarrow{\alpha}P}$$

  Since $P\approx_s P$, we get $\alpha.\tau.P\approx_s \alpha.P$, as desired.
  \item $(\alpha_1\parallel\cdots\parallel\alpha_n).\tau.P\approx_s (\alpha_1\parallel\cdots\parallel\alpha_n).P$. By the weak transition rules $\textbf{WAct}_{1,2}$ in Table \ref{WTRForCTC}, we get

  $$\frac{}{(\alpha_1\parallel\cdots\parallel\alpha_n).\tau.P\xRightarrow{\{\alpha_1,\cdots,\alpha_n\}}P}
  \quad \frac{}{(\alpha_1\parallel\cdots\parallel\alpha_n).P\xRightarrow{\{\alpha_1,\cdots,\alpha_n\}}P}$$

  Since $P\approx_s P$, we get $(\alpha_1\parallel\cdots\parallel\alpha_n).\tau.P\approx_s (\alpha_1\parallel\cdots\parallel\alpha_n).P$, as desired.
  \item $P+\tau.P\approx_s \tau.P$. By the weak transition rules $\textbf{WSum}_{1,2}$ of CTC in Table \ref{WTRForCTC}, we get

  $$\frac{P\xRightarrow{\alpha}P'}{P+\tau.P\xRightarrow{\alpha}P'}
  \quad \frac{P\xRightarrow{\alpha}P'}{\tau.P\xRightarrow{\alpha} P'}$$

  Since $P'\approx_s P'$, we get $P+\tau.P\approx_s \tau.P$, as desired.
  \item $\alpha.(P+\tau.Q)+\alpha.Q\approx_s\alpha.(P+\tau.Q)$. By the weak transition rules $\textbf{WAct}_{1,2}$ and $\textbf{WSum}_{1,2}$ of CTC in Table \ref{WTRForCTC}, we get

  $$\frac{}{\alpha.(P+\tau.Q)+\alpha.Q\xRightarrow{\alpha}Q}
  \quad \frac{}{\alpha.(P+\tau.Q)\xRightarrow{\alpha} Q}$$

  Since $Q\approx_s Q$, we get $\alpha.(P+\tau.Q)+\alpha.Q\approx_s\alpha.(P+\tau.Q)$, as desired.
  \item $(\alpha_1\parallel\cdots\parallel\alpha_n).(P+\tau.Q)+ (\alpha_1\parallel\cdots\parallel\alpha_n).Q\approx_s (\alpha_1\parallel\cdots\parallel\alpha_n).(P+\tau.Q)$. By the weak transition rules $\textbf{WAct}_{1,2}$ and $\textbf{WSum}_{1,2}$ of CTC in Table \ref{WTRForCTC}, we get

  $$\frac{}{(\alpha_1\parallel\cdots\parallel\alpha_n).(P+\tau.Q)+(\alpha_1\parallel\cdots\parallel\alpha_n).Q \xRightarrow{\{\alpha_1,\cdots,\alpha_n\}}Q}
  \quad \frac{}{(\alpha_1\parallel\cdots\parallel\alpha_n).(P+\tau.Q)\xRightarrow{\{\alpha_1,\cdots,\alpha_n\}} Q}$$

  Since $Q\approx_s Q$, we get $(\alpha_1\parallel\cdots\parallel\alpha_n).(P+\tau.Q)+ (\alpha_1\parallel\cdots\parallel\alpha_n).Q\approx_s(\alpha_1\parallel\cdots\parallel\alpha_n).(P+\tau.Q)$, as desired.
  \item $P\approx_s \tau\parallel P$. By the weak transition rules $\textbf{WCom}_{1,2,3,4}$ of CTC in Table \ref{WTRForCTC}, we get

  $$\frac{P\xRightarrow{\alpha}P'}{P\xRightarrow{\alpha}P'}
  \quad \frac{P\xRightarrow{\alpha}P'}{\tau\parallel P\xRightarrow{\alpha} P'}$$

  Since $P'\approx_s P'$, we get $P\approx_s \tau\parallel P$, as desired.
\end{enumerate}
\end{proof}

\begin{proposition}[$\tau$ laws for weak pomset bisimulation]\label{TAUWPB}
The $\tau$ laws for weak pomset bisimulation is as follows.
\begin{enumerate}
  \item $P\approx_p \tau.P$;
  \item $\alpha.\tau.P\approx_p \alpha.P$;
  \item $(\alpha_1\parallel\cdots\parallel\alpha_n).\tau.P\approx_p (\alpha_1\parallel\cdots\parallel\alpha_n).P$;
  \item $P+\tau.P\approx_p \tau.P$;
  \item $\alpha.(P+\tau.Q)+\alpha.Q\approx_p\alpha.(P+\tau.Q)$;
  \item $(\alpha_1\parallel\cdots\parallel\alpha_n).(P+\tau.Q)+ (\alpha_1\parallel\cdots\parallel\alpha_n).Q\approx_p (\alpha_1\parallel\cdots\parallel\alpha_n).(P+\tau.Q)$;
  \item $P\approx_p \tau\parallel P$.
\end{enumerate}
\end{proposition}

\begin{proof}
From the definition of weak pomset bisimulation $\approx_{p}$ (see Definition \ref{WPSB}), we know that weak pomset bisimulation $\approx_{p}$ is defined by weak pomset transitions, which are labeled by pomsets with $\tau$. In a weak pomset transition, the events in the pomset are either within causality relations (defined by $.$) or in concurrency (implicitly defined by $.$ and $+$, and explicitly defined by $\parallel$), of course, they are pairwise consistent (without conflicts). In Proposition \ref{TAUWSB}, we have already proven the case that all events are pairwise concurrent, so, we only need to prove the case of events in causality. Without loss of generality, we take a pomset of $p=\{\alpha,\beta:\alpha.\beta\}$. Then the weak pomset transition labeled by the above $p$ is just composed of one single event transition labeled by $\alpha$ succeeded by another single event transition labeled by $\beta$, that is, $\xRightarrow{p}=\xRightarrow{\alpha}\xRightarrow{\beta}$.

Similarly to the proof of $\tau$ laws for weak step bisimulation $\approx_{s}$ (Proposition \ref{TAUWSB}), we can prove that $\tau$ laws hold for weak pomset bisimulation $\approx_{p}$, we omit them.
\end{proof}

\begin{proposition}[$\tau$ laws for weak hp-bisimulation]\label{TAUWHPB}
The $\tau$ laws for weak hp-bisimulation is as follows.
\begin{enumerate}
  \item $P\approx_{hp} \tau.P$;
  \item $\alpha.\tau.P\approx_{hp} \alpha.P$;
  \item $(\alpha_1\parallel\cdots\parallel\alpha_n).\tau.P\approx_{hp} (\alpha_1\parallel\cdots\parallel\alpha_n).P$;
  \item $P+\tau.P\approx_{hp} \tau.P$;
  \item $\alpha.(P+\tau.Q)+\alpha.Q\approx_{hp}\alpha.(P+\tau.Q)$;
  \item $(\alpha_1\parallel\cdots\parallel\alpha_n).(P+\tau.Q)+ (\alpha_1\parallel\cdots\parallel\alpha_n).Q\approx_{hp} (\alpha_1\parallel\cdots\parallel\alpha_n).(P+\tau.Q)$;
  \item $P\approx_{hp} \tau\parallel P$.
\end{enumerate}
\end{proposition}

\begin{proof}
From the definition of weak hp-bisimulation $\approx_{hp}$ (see Definition \ref{WHHPB}), we know that weak hp-bisimulation $\approx_{hp}$ is defined on the weakly posetal product $(C_1,f,C_2),f:\hat{C_1}\rightarrow \hat{C_2}\textrm{ isomorphism}$. Two processes $P$ related to $C_1$ and $Q$ related to $C_2$, and $f:\hat{C_1}\rightarrow \hat{C_2}\textrm{ isomorphism}$. Initially, $(C_1,f,C_2)=(\emptyset,\emptyset,\emptyset)$, and $(\emptyset,\emptyset,\emptyset)\in\approx_{hp}$. When $P\xrightarrow{\alpha}P'$ ($C_1\xrightarrow{\alpha}C_1'$), there will be $Q\xRightarrow{\alpha}Q'$ ($C_2\xRightarrow{\alpha}C_2'$), and we define $f'=f[\alpha\mapsto \alpha]$. Then, if $(C_1,f,C_2)\in\approx_{hp}$, then $(C_1',f',C_2')\in\approx_{hp}$.

Similarly to the proof of $\tau$ laws for weak pomset bisimulation (Proposition \ref{TAUWPB}), we can prove that $\tau$ laws hold for weak hp-bisimulation, we just need additionally to check the above conditions on weak hp-bisimulation, we omit them.
\end{proof}

\begin{proposition}[$\tau$ laws for weakly hhp-bisimulation]\label{TAUWHHPB}
The $\tau$ laws for weakly hhp-bisimulation is as follows.
\begin{enumerate}
  \item $P\approx_{hhp} \tau.P$;
  \item $\alpha.\tau.P\approx_{hhp} \alpha.P$;
  \item $(\alpha_1\parallel\cdots\parallel\alpha_n).\tau.P\approx_{hhp} (\alpha_1\parallel\cdots\parallel\alpha_n).P$;
  \item $P+\tau.P\approx_{hhp} \tau.P$;
  \item $\alpha.(P+\tau.Q)+\alpha.Q\approx_{hhp}\alpha.(P+\tau.Q)$;
  \item $(\alpha_1\parallel\cdots\parallel\alpha_n).(P+\tau.Q)+ (\alpha_1\parallel\cdots\parallel\alpha_n).Q\approx_{hhp} (\alpha_1\parallel\cdots\parallel\alpha_n).(P+\tau.Q)$;
  \item $P\approx_{hhp} \tau\parallel P$.
\end{enumerate}
\end{proposition}

\begin{proof}
From the definition of weakly hhp-bisimulation (see Definition \ref{WHHPB}), we know that weakly hhp-bisimulation is downward closed for weak hp-bisimulation.

Similarly to the proof of $\tau$ laws for weak hp-bisimulation (see Proposition \ref{TAUWHPB}), we can prove that the $\tau$ laws hold for weakly hhp-bisimulation, we omit them.
\end{proof}

\subsection{Recursion}

\begin{definition}[Sequential]
$X$ is sequential in $E$ if every subexpression of $E$ which contains $X$, apart from $X$ itself, is of the form $\alpha.F$, or $(\alpha_1\parallel\cdots\parallel\alpha_n).F$, or $\sum\widetilde{F}$.
\end{definition}

\begin{definition}[Guarded recursive expression]
$X$ is guarded in $E$ if each occurrence of $X$ is with some subexpression $l.F$ or $(l_1\parallel\cdots\parallel l_n).F$ of $E$.
\end{definition}

\begin{lemma}\label{LUSWW}
Let $G$ be guarded and sequential, $Vars(G)\subseteq\widetilde{X}$, and let $G\{\widetilde{P}/\widetilde{X}\}\xrightarrow{\{\alpha_1,\cdots,\alpha_n\}}P'$. Then there is an expression $H$ such that $G\xrightarrow{\{\alpha_1,\cdots,\alpha_n\}}H$, $P'\equiv H\{\widetilde{P}/\widetilde{X}\}$, and for any $\widetilde{Q}$, $G\{\widetilde{Q}/\widetilde{X}\}\xrightarrow{\{\alpha_1,\cdots,\alpha_n\}} H\{\widetilde{Q}/\widetilde{X}\}$. Moreover $H$ is sequential, $Vars(H)\subseteq\widetilde{X}$, and if $\alpha_1=\cdots=\alpha_n=\tau$, then $H$ is also guarded.
\end{lemma}

\begin{proof}
We need to induct on the structure of $G$.

If $G$ is a Constant, a Composition, a Restriction or a Relabeling then it contains no variables, since $G$ is sequential and guarded, then $G\xrightarrow{\{\alpha_1,\cdots,\alpha_n\}}P'$, then let $H\equiv P'$, as desired.

$G$ cannot be a variable, since it is guarded.

If $G\equiv G_1+G_2$. Then either $G_1\{\widetilde{P}/\widetilde{X}\} \xrightarrow{\{\alpha_1,\cdots,\alpha_n\}}P'$ or $G_2\{\widetilde{P}/\widetilde{X}\} \xrightarrow{\{\alpha_1,\cdots,\alpha_n\}}P'$, then, we can apply this lemma in either case, as desired.

If $G\equiv\beta.H$. Then we must have $\alpha=\beta$, and $P'\equiv H\{\widetilde{P}/\widetilde{X}\}$, and $G\{\widetilde{Q}/\widetilde{X}\}\equiv \beta.H\{\widetilde{Q}/\widetilde{X}\} \xrightarrow{\beta}H\{\widetilde{Q}/\widetilde{X}\}$, then, let $G'$ be $H$, as desired.

If $G\equiv(\beta_1\parallel\cdots\parallel\beta_n).H$. Then we must have $\alpha_i=\beta_i$ for $1\leq i\leq n$, and $P'\equiv H\{\widetilde{P}/\widetilde{X}\}$, and $G\{\widetilde{Q}/\widetilde{X}\}\equiv (\beta_1\parallel\cdots\parallel\beta_n).H\{\widetilde{Q}/\widetilde{X}\} \xrightarrow{\{\beta_1,\cdots,\beta_n\}}H\{\widetilde{Q}/\widetilde{X}\}$, then, let $G'$ be $H$, as desired.

If $G\equiv\tau.H$. Then we must have $\tau=\tau$, and $P'\equiv H\{\widetilde{P}/\widetilde{X}\}$, and $G\{\widetilde{Q}/\widetilde{X}\}\equiv \tau.H\{\widetilde{Q}/\widetilde{X}\} \xrightarrow{\tau}H\{\widetilde{Q}/\widetilde{X}\}$, then, let $G'$ be $H$, as desired.
\end{proof}

\begin{theorem}[Unique solution of equations for weak step bisimulation]\label{USWSB}
Let the guarded and sequential expressions $\widetilde{E}$ contain free variables $\subseteq \widetilde{X}$, then,

If $\widetilde{P}\approx_s \widetilde{E}\{\widetilde{P}/\widetilde{X}\}$ and $\widetilde{Q}\approx_s \widetilde{E}\{\widetilde{Q}/\widetilde{X}\}$, then $\widetilde{P}\approx_s \widetilde{Q}$.
\end{theorem}

\begin{proof}
Like the corresponding theorem in CCS, without loss of generality, we only consider a single equation $X=E$. So we assume $P\approx_s E(P)$, $Q\approx_s E(Q)$, then $P\approx_s Q$.

We will prove $\{(H(P),H(Q)): H\}$ sequential, if $H(P)\xrightarrow{\{\alpha_1,\cdots,\alpha_n\}}P'$, then, for some $Q'$, $H(Q)\xRightarrow{\{\alpha_1.\cdots,\alpha_n\}}Q'$ and $P'\approx_s Q'$.

Let $H(P)\xrightarrow{\{\alpha_1,\cdot,\alpha_n\}}P'$, then $H(E(P))\xRightarrow{\{\alpha_1,\cdots,\alpha_n\}}P''$ and $P'\approx_s P''$.

By Lemma \ref{LUSWW}, we know there is a sequential $H'$ such that $H(E(P))\xRightarrow{\{\alpha_1,\cdots,\alpha_n\}}H'(P)\Rightarrow P''\approx_s P'$.

And, $H(E(Q))\xRightarrow{\{\alpha_1,\cdots,\alpha_n\}}H'(Q)\Rightarrow Q''$ and $P''\approx_s Q''$. And $H(Q)\xrightarrow{\{\alpha_1,\cdots,\alpha_n\}}Q'\approx_s Q''$. Hence, $P'\approx_s Q'$, as desired.
\end{proof}

\begin{theorem}[Unique solution of equations for weak pomset bisimulation]\label{USWPB}
Let the guarded and sequential expressions $\widetilde{E}$ contain free variables $\subseteq \widetilde{X}$, then,

If $\widetilde{P}\approx_p \widetilde{E}\{\widetilde{P}/\widetilde{X}\}$ and $\widetilde{Q}\approx_p \widetilde{E}\{\widetilde{Q}/\widetilde{X}\}$, then $\widetilde{P}\approx_p \widetilde{Q}$.
\end{theorem}

\begin{proof}
From the definition of weak pomset bisimulation $\approx_{p}$ (see Definition \ref{WPSB}), we know that weak pomset bisimulation $\approx_{p}$ is defined by weak pomset transitions, which are labeled by pomsets with $\tau$. In a weak pomset transition, the events in the pomset are either within causality relations (defined by $.$) or in concurrency (implicitly defined by $.$ and $+$, and explicitly defined by $\parallel$), of course, they are pairwise consistent (without conflicts). In Theorem \ref{USWSB}, we have already proven the case that all events are pairwise concurrent, so, we only need to prove the case of events in causality. Without loss of generality, we take a pomset of $p=\{\alpha,\beta:\alpha.\beta\}$. Then the weak pomset transition labeled by the above $p$ is just composed of one single event transition labeled by $\alpha$ succeeded by another single event transition labeled by $\beta$, that is, $\xRightarrow{p}=\xRightarrow{\alpha}\xRightarrow{\beta}$.

Similarly to the proof of unique solution of equations for weak step bisimulation $\approx_{s}$ (Theorem \ref{USWSB}), we can prove that unique solution of equations holds for weak pomset bisimulation $\approx_{p}$, we omit them.
\end{proof}

\begin{theorem}[Unique solution of equations for weak hp-bisimulation]\label{USWHPB}
Let the guarded and sequential expressions $\widetilde{E}$ contain free variables $\subseteq \widetilde{X}$, then,

If $\widetilde{P}\approx_{hp} \widetilde{E}\{\widetilde{P}/\widetilde{X}\}$ and $\widetilde{Q}\approx_{hp} \widetilde{E}\{\widetilde{Q}/\widetilde{X}\}$, then $\widetilde{P}\approx_{hp} \widetilde{Q}$.
\end{theorem}

\begin{proof}
From the definition of weak hp-bisimulation $\approx_{hp}$ (see Definition \ref{WHHPB}), we know that weak hp-bisimulation $\approx_{hp}$ is defined on the weakly posetal product $(C_1,f,C_2),f:\hat{C_1}\rightarrow \hat{C_2}\textrm{ isomorphism}$. Two processes $P$ related to $C_1$ and $Q$ related to $C_2$, and $f:\hat{C_1}\rightarrow \hat{C_2}\textrm{ isomorphism}$. Initially, $(C_1,f,C_2)=(\emptyset,\emptyset,\emptyset)$, and $(\emptyset,\emptyset,\emptyset)\in\approx_{hp}$. When $P\xrightarrow{\alpha}P'$ ($C_1\xrightarrow{\alpha}C_1'$), there will be $Q\xRightarrow{\alpha}Q'$ ($C_2\xRightarrow{\alpha}C_2'$), and we define $f'=f[\alpha\mapsto \alpha]$. Then, if $(C_1,f,C_2)\in\approx_{hp}$, then $(C_1',f',C_2')\in\approx_{hp}$.

Similarly to the proof of unique solution of equations for weak pomset bisimulation (Theorem \ref{USWPB}), we can prove that unique solution of equations holds for weak hp-bisimulation, we just need additionally to check the above conditions on weak hp-bisimulation, we omit them.
\end{proof}

\begin{theorem}[Unique solution of equations for weakly hhp-bisimulation]\label{USWHHPB}
Let the guarded and sequential expressions $\widetilde{E}$ contain free variables $\subseteq \widetilde{X}$, then,

If $\widetilde{P}\approx_{hhp} \widetilde{E}\{\widetilde{P}/\widetilde{X}\}$ and $\widetilde{Q}\approx_{hhp} \widetilde{E}\{\widetilde{Q}/\widetilde{X}\}$, then $\widetilde{P}\approx_{hhp} \widetilde{Q}$.
\end{theorem}

\begin{proof}
From the definition of weakly hhp-bisimulation (see Definition \ref{WHHPB}), we know that weakly hhp-bisimulation is downward closed for weak hp-bisimulation.

Similarly to the proof of unique solution of equations for weak hp-bisimulation (see Theorem \ref{USWHPB}), we can prove that the unique solution of equations holds for weakly hhp-bisimulation, we omit them.
\end{proof}
