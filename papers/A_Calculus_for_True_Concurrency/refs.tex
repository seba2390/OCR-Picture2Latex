\begin{thebibliography}{Lam94}

\bibitem{ALNC}M. Hennessy and R. Milner. Algebraic laws for nondeterminism and concurrency. J. ACM, 1985, 32, 137-161.

\bibitem{CC}R. Milner. Communication and concurrency. Printice Hall, 1989.

\bibitem{CCS} R. Milner. A calculus of communicating systems. LNCS 92, Springer, 1980.

\bibitem{ACP} W. Fokkink. Introduction to process algebra 2nd ed. Springer-Verlag, 2007.

\bibitem{ES1}M. Nielsen, G. D. Plotkin, and G. Winskel. Petri nets, event structures and domains, Part I. Theoret. Comput. Sci. 1981, 13, 85-108.

\bibitem{ES2}G. Winskel. Event structures. In Petri Nets: Applications and Relationships to Other Models of Concurrency, Wilfried Brauer, Wolfgang Reisig, and Grzegorz Rozenberg, Eds., Lecture Notes in Computer Science, 1987, vol. 255, Springer, Berlin, 325-392.

\bibitem{CM}G. Winskel and M. Nielsen. Models for concurrency. In Samson Abramsky, Dov M. Gabbay,and  Thomas S. E. Maibaum, Eds., Handbook of logic in Computer Science, 1995, vol. 4, Clarendon Press, Oxford, UK.

\bibitem{HHP1}M. A. Bednarczyk. Hereditary history preserving bisimulations or what is the power of the future perfect in program logics. Tech. Rep. Polish Academy of Sciences. 1991.

\bibitem{HHP2}S. B. Fr\"{o}schle and T. T. Hildebrandt. On plain and hereditary history-preserving bisimulation. In Proceedings of MFCS'99, Miroslaw Kutylowski, Leszek Pacholski, and Tomasz Wierzbicki, Eds., Lecture Notes in Computer Science, 1999, vol. 1672, Springer, Berlin, 354-365.

\bibitem{MUC}J. Bradfield and C. Stirling. Modal mu-calculi. In Handbook of Modal Logic, Patrick Blackburn, Johan van Benthem, and Franck Wolter, Eds., Elsevier, Amsterdam, The Netherlands, 2006, 721-756.

\bibitem{RL1}I. Phillips and I. Ulidowski. Reverse bisimulations on stable configuration structures. In Proceedings of SOS'09, B. Klin and P. Soboci\`{n}ski, Eds., Electronic Proceedings in Theoretical Computer Science, 2010, vol. 18. Elsevier, Amsterdam, The Netherlands, 62-76.

\bibitem{RL2}I. Phillips and I. Ulidowski. A logic with reverse modalities for history-preserving bisimulations. In Proceedings of EXPRESS'11, Bas Luttik and Frank Valencia, Eds., Electronic Proceedings in Theoretical Computer Science, 2011, vol. 64, Elsevier, Amsterdam, The Netherlands, 104-118.

\bibitem{SFL}J. Gutierrez. On bisimulation and model-checking for concurrent systems with partial order semantics. Ph.D. dissertation. LFCS- University of Edinburgh, 2011.

\bibitem{LTC1}P. Baldan and S. Crafa. A logic for true concurrency. In Proceedings of CONCUR'10, Paul Gastin and Fran\c{c}ois Laroussinie, Eds., Lecture Notes in Computer Science, 2010, vol. 6269, Springer, Berlin, 147-161.

\bibitem{LTC2}P. Baldan and S. Crafa. A logic for true concurrency. J.ACM, 2014, 61(4): 36 pages.

\bibitem{WTC}Y. Wang. Weakly true concurrency and its logic. 2016, Manuscript, arXiv:1606.06422.

\bibitem{CFAR} F.W. Vaandrager. Verification of two communication protocols by means of process algebra. Report CS-R8608, CWI, Amsterdam, 1986.

\bibitem{AL} R. Glabbeek and U. Goltz. Refinement of actions and equivalence notions for concurrent systems. Acta Inf. 2001, 37, 4/5, 229-327.

\bibitem{ABP} K.A. Bartlett, R.A. Scantlebury, and P.T. Wilkinson. A note on reliable full-duplex transmission over half-duplex links. Communications of the ACM, 12(5):260-261, 1969.

\bibitem{ATC}Y. Wang. Algebraic Laws for True Concurrency. Submitted to JACM, 2016. arXiv: 1611.09035.

\end{thebibliography}
