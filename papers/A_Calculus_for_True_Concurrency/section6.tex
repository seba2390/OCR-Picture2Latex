\section{Applications}\label{app}

In this section, we show the applications of CTC by verification of the alternating-bit protocol \cite{ABP}. The alternating-bit protocol is a communication protocol and illustrated in Fig. \ref{ABP}.

\begin{figure}
    \centering
    \includegraphics{ABP.eps}
    \caption{Alternating bit protocol}
    \label{ABP}
\end{figure}

The $Trans$ and $Ack$ lines may lose or duplicate message, and messages are sent tagged with the bits $0$ or $1$ alternately, and these bits also constitute the acknowledgements.

There are some variations of the classical alternating-bit protocol. We assume the message flows are bidirectional, the sender accepts messages from the outside world, and sends it to the replier via some line; and it also accepts messages from the replier, deliver it to the outside world and acknowledges it via some line. The role of the replier acts the same as the sender. But, for simplicities and without loss the generality, we only consider the dual one directional processes, we just suppose that the messages from the replier are always accompanied with the acknowledgements.

After accepting a message from the outside world, the sender sends it with bit $b$ along the $Trans$ line and sets a timer,

\begin{enumerate}
  \item it may get a "time-out" from the timer, upon which it sends the message again with $b$;
  \item it may get an acknowledgement $b$ accompanied with the message from the replier via the $Ack$ line, upon which it deliver the message to the outside world and is ready to accept another message tagged with bit $\hat{b}=1-b$;
  \item it may get an acknowledgement $\hat{b}$ which it ignores.
\end{enumerate}

After accepting a message from outside world, the replier also can send it to the sender, but we ignore this process and just assume the message is always accompanied with the acknowledgement to the sender, and just process the dual manner to the sender. After delivering a message to the outside world it acknowledges it with bit $b$ along the $Ack$ line and sets a timer,

\begin{enumerate}
  \item it may get a "time-out" from the timer, upon which it acknowledges again with $b$;
  \item it may get a new message with bit $\hat{b}$ from the $Trans$ line, upon which it is ready to deliver the new message and acknowledge with bit $\hat{b}$ accompanying with its messages from the outside world;
  \item it may get a superfluous transmission of the previous message with bit $b$, which it ignores.
\end{enumerate}

Now, we give the formal definitions as follows.

\[Send(b)\overset{\text{def}}{=}\overline{send_b}.\overline{time}.Sending(b)\]

\[Sending(b)\overset{\text{def}}{=}timeout.Send(b)+ack_b.timeout.AcceptS(\hat{b}) +ack_{\hat{b}}.DeliverS(b)\]

\[AcceptS(b)\overset{\text{def}}{=}acceptS.Send(b)\]

\[DeliverS(b)\overset{\text{def}}{=}\overline{deliverS}.Sending(b)\]

\[Reply(b)\overset{\text{def}}{=}\overline{reply_b}.\overline{time}.Replying(b)\]

\[Replying(b)\overset{\text{def}}{=}timeout.Reply(b)+trans_{\hat{b}}.timeout.Deliver(\hat{b}) +trans_b.AcceptR(b)\]

\[DeliverR(b)\overset{\text{def}}{=}\overline{deliverR}.Reply(b)\]

\[AcceptR(b)\overset{\text{def}}{=}acceptR.Replying(b)\]

\[Timer\overset{\text{def}}{=}time.\overline{timeout}.Timer\]

\[Ack(bs)\xrightarrow{\overline{ack_b}}Ack(s)\quad Trans(sb)\xrightarrow{\overline{trans_b}}Trans(s)\]

\[Ack(s)\xrightarrow{reply_b}Ack(sb)\quad Trans(s)\xrightarrow{send_b}Trans(bs)\]

\[Ack(sbt)\xrightarrow{\tau}Ack(st)\quad Trans(tbs)\xrightarrow{\tau}Trans(ts)\]

\[Ack(sbt)\xrightarrow{\tau}Ack(sbbt)\quad Trans(tbs)\xrightarrow{\tau}Trans(tbbs)\]

Then the complete system can be builded by compose the components. That is, it can be expressed as follows, where $\epsilon$ is the empty sequence.

$$AB\overset{\text{def}}{=}Accept(\hat{b})\parallel Trans(\epsilon)\parallel Ack(\epsilon)\parallel Reply(b)\parallel Timer$$

Now, we define the protocol specification to be a buffer as follows:

$$Buff\overset{\text{def}}{=}(acceptS\parallel acceptR).Buff'$$

$$Buff'\overset{\text{def}}{=}(\overline{deliverS}\parallel \overline{deliverR}).Buff$$

We need to prove that

$$AB\approx_s Buff$$

$$AB\approx_p Buff$$

$$AB\approx_{hp} Buff$$

$$AB\approx_{hhp} Buff$$

The deductive process is omitted, and we left it as an excise to the readers. 