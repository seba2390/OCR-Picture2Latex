\renewcommand\figurename{Appendix Figure}

\begin{figure}[!t]
\centering
\includegraphics[width=0.95\textwidth]{figs/UFold/UFold1.png} 
\caption{\textbf{Detailed performance comparison between RNA-FM and UFold on the ArchiveII dataset.} The experiment is cross-dataset validation of the trained model without re-training on ArchiveII. \textbf{a.} Scatter plots of F1 score comparison across 9 RNA types, with the performance of RNA-FM as the y-axis and that of UFold as the x-axis. Each point represents an RNA structure. Almost all the points are above the diagonal, which means RNA-FM beats UFold on nearly all the instances. \textbf{b.} F1 scores as a function of RNA sequence lengths. RNA-FM always outperforms UFold across all the lengths, especially when the length is over 150.} 
\label{Fig.UFold1}
\end{figure}

\begin{figure}[!th]
\centering
\includegraphics[width=0.95\textwidth]{figs/UFold/UFold2.png} 
\caption{\textbf{Probability maps and graph view of secondary structure predictions of two randomly selected examples.} We compare the predictions from RNA-FM (second column) and UFold (third column) against the ground truth (first column). The probability maps from RNA-FM are more robust with less noise and closer to the ground truth compared to the ones from UFold. Regarding the visualization, RNA-FM also generates secondary structures more similar to the ground truth than UFold.}
\label{Fig.UFold2}
\end{figure}

\begin{figure}[!th]
\centering
\includegraphics[width=0.9\textwidth]{figs/RNA-CONTACT/rnacontact.png}
\caption{\textbf{RNA 3D closeness prediction performance on RNAcontact TE80 dataset.} ResNet is used to reproduce RNAcontact results for an equal comparison of different features. \textit{Seq} means one-hot encoding of the sequence; \textit{Cov} means MSA covariances; \textit{SS} means secondary structure predicted by the PETfold based on MSA; \textit{RNA-FM} means the feature from RNA-FM;$+$ means a combination of features by a channel-wise concatenation. \textbf{a.} Scatter plots of MCCs, with the performance of \textit{RNA-FM} as the y-axis and the performance of \textit{Cov$+$SS} (used by RNAcontact) as the x-axis. Each point represents an RNA structure. Almost all the points are above the diagonal, which means RNA-FM embeddings are better than the other features in nearly all the instances. \textbf{b.} F1 scores as a function of RNA sequence lengths. \textit{RNA-FM} outperforms \textit{Cov$+$SS} all the time. \textbf{c.} Probability maps of three randomly selected examples. With the standalone RNA-FM embedding (\textit{RNA-FM}) as the input, the downstream model has already generated visualizations much closer to the ground truth than other features. Furthermore, by re-training the ResNet pre-trained on the other tasks using transfer learning, we can achieve the performance far better than the other methods.} 
\label{Fig.rnacontact}
\end{figure}

\begin{figure}[t] %[!th]
\centering
\includegraphics[width=1\textwidth]{figs/3D-model/ren_alone.pdf} 
\caption{\textbf{3D reconstruction of RNA.} The probability maps and binary maps are generated by different predictors. The graph views are obtained by jViz.Rna 4.0 \cite{shabash2017numerical}. The 3D structures are modelled by 3dRNA. \textbf{a.} An instance from PDB (5m73-1-A). \textbf{b.} The DCS-PK of Zika Virus.}
\label{Fig.3dmodel-ren}
\end{figure}

