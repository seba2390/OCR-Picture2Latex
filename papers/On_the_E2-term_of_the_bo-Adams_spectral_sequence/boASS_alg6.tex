% !TEX root = boASS6.tex

\section{The algebraic $\bo$-resolution}\label{sec:alg}

We now construct and analyze an algebraic parallel to the $\bo$-ASS studied so far, the $\bo$-Mahowald spectral sequence ($\bo$-MSS).

\subsection*{Construction of the $\bo$-MSS}

Taking homology of the cofiber sequence
$$ S \rightarrow \bo \rightarrow \br{\bo}, $$
we get a short exact sequence
$$ 0  \rightarrow \FF_2 \rightarrow A \mmod A(1)_*  \rightarrow \br{A \mmod A(1)}_* \rightarrow 0 $$
of $A_*$-comodules, and hence short exact sequences 
$$ 0 \rightarrow \br{A \mmod A(1)}^n_* \rightarrow A \mmod A(1)_* \otimes \br{A \mmod A(1)}^n_*\rightarrow \br{A \mmod A(1)}^{n+1}_*   \rightarrow 0. $$
Piecing together the associated long exact sequences of $\Ext$-groups, and using the change of rings isomorphisms
$$ \Ext^{*,*}_{A_*}(A \mmod A(1)_* \otimes \br{A \mmod A(1)}^n_*) \cong \Ext^{*,*}_{A(1)_*}(\br{A \mmod A(1)}^n_*),  $$
we get an associated ``algebraic $\bo$-resolution''
$$
\xymatrix@C-1em@R-1em{
\Ext_{A_*}^{s,t}(\FF_2) \ar[d] 
& \Ext^{s-1,t}_{A_*}(\br{A \mmod A(1)}_*) \ar[l] \ar[d]
& \Ext^{s-2,t}_{A_*}(\br{A \mmod A(1)}^2_*) \ar[l] \ar[d] 
& \cdots \ar[l]
\\
\Ext^{s,t}_{A(1)_*}(\FF_2)
& \Ext^{s-1,t}_{A(1)_*}(\br{A \mmod A(1)}_*)
& \Ext^{s-2,t}_{A(1)_*}(\br{A \mmod A(1) }^2_*)
}
$$
This gives a spectral sequence (the \emph{$\bo$-Mahowald spectral sequence})
$$ \E{bo}{alg}_1^{n,s,t} = \Ext^{s,t}_{A(1)_*}(\br{A\mmod A(1)}^{n}_*) \Rightarrow \Ext^{s+n,t}_{A_*}(\FF_2) $$
with differentials
$$ d^{alg}_r: \E{bo}{alg}^{n,s,t}_{r} \rightarrow \E{bo}{alg}^{n+r,s-r+1,t}_{{r}}. $$

\subsection*{The $E_1$-term of the $\bo$-MSS}

Let $\ull{B_i}$ denote the $i$th integral Brown-Gitler comodule, the $A_*$-comodule obtained by taking homology of the $i$th integral Brown-Gitler spectrum
$$ \ull{B_i} := H_*(B_i). $$
The motivation behind Theorem~\ref{thm:splitting} is that there is a splitting of $A(1)_*$-comodules:
\begin{equation}\label{eq:algsplitting}
 \br{A \mmod A(1)}^n_* \cong_{A(1)_*} \bigoplus_{\abs{I} = n} \Si^{4\norm{I}} \ull{B_I}
 \end{equation}
where $I = (i_1, \ldots, i_n)$ with each $i_l > 0$ and 
$$ \ull{B_I} := \ull{B_{i_1}} \otimes \cdots \otimes \ull{B_{i_n}}. $$
The $\Ext_{A(1)_*}$-groups of these comodules are given by
$$ \Ext^{*,*}_{A(1)_*}(\ull{B_I}) \cong \E{ass}{}_2(b_I) \oplus V_I $$
where the graded $\FF_2$-vector space $V_I$ has cohomological degree zero:
$$ V_I \subseteq \Ext^{0,*}_{A(1)_*}(\ull{B_I}). $$
Thus the evil subcomplex $(V^{*,*}, d_1^{evil})$ with
$$
V^{n,*} := \bigoplus_{\abs{I} = n} \Si^{4\norm{I}} V_I
$$
is a subcomplex of $(\E{bo}{alg}^{*,0,*}, d_1^{alg})$.
We will define $(\mc{C}^{*,*,*}_{alg}, d_1^{alg,good})$ to be the quotient complex, where
\begin{equation}\label{eq:Calg}
\mc{C}^{n,s,t}_{alg} := \bigoplus_{\abs{I} = n} \E{ass}{}^{s,t}_2(\Si^{4\norm{I}} b_I).
\end{equation}
Analogous to Section~\ref{sec:E1}, we will denote elements of $\mc{C}_{alg}^{*,*,*}$
using notation
$ x(I)$
for $x$ an element of $\E{ass}{}_2^{*,*}(b_I)$
using notation
$$ x = v_0^k w^m h_1^a $$
where $w$ is the formal expression 
$$ w := v_1^2/v_0^2. $$
The following proposition is an immediate consequence of Theorem~\ref{thm:d1}.

\begin{prop}\label{prop:d1alg}
	The differential $d_1^{good, alg}$ in the complex $\mc{C}_{alg}^{*,*,*}$ is given by the formula
	\begin{multline*}
	 d_1^{good,alg}(w^m(i_1, \ldots, i_n)) = \sum_{\substack{m' + m'' = m \\
	 \alpha(m')+\alpha(m'') = \alpha(m)}} w^{m'}(m'', i_1, \ldots, i_n) \\ + 
	\sum_k \sum_{\substack{i_k'+i_k'' = i_k \\ \alpha(i_k') + \alpha(i_k'') = \alpha(i_k)}} w^m(i_1, \ldots, i_k', i_k'', \ldots, i_n) \end{multline*}
	and the fact that the differential is $v_0$ and $h_1$-linear.
\end{prop}

\subsection*{The algebraic weight spectral sequence}

Analogous to the weight spectral sequence of Section~\ref{sec:wss}, we can set up an algebraic weight spectral sequence to compute $H^{*,*,*}(\mc{C}_{alg})$. 
Just as in the topological case, we endow $\mc{C}^{*,*,*}_{alg}$ with a decreasing filtration by weight ({$wt$}), where we define
\[ wt(x(I)) = \norm{I}. \]
Proposition~\ref{prop:d1alg} implies that $d_1^{good, alg}$ does not decrease weight.  There is a resulting spectral sequence
\[ \E{wss}{alg}_0^{n,s,t,w} := [\mc{C}^{n,s,t}_{alg}]_{wt=w} \Rightarrow H^{n,s,t}(\mc{C}_{alg}) \]
with differentials
\[ d_r^{wss, alg}: \E{wss}{alg}_r^{n,s,t,w} \rightarrow \E{wss}{alg}_r^{n+1, s,t, w+r}. \]
From Proposition~\ref{prop:d1alg}, we see that $d_0^{wss, alg}$ is given by
\begin{equation}\label{eq:d0wssalg} 
d_0^{wss, alg}(x(i_1, \ldots, i_n)) = \sum_k \sum_{
\substack{i_k'+i_k'' = i_k \\ \alpha(i_k')+\alpha(i_k'') = \alpha(i_k)}} x(i_1, \ldots, i_k', i_k'', \ldots, i_n). 
\end{equation}


Note that this is precisely the formula for $d_0^{AF}$ of Section~\ref{sec:wss} {(see (\ref{eq:d0AF})).  Therefore, the same proof for Proposition~\ref{prop:E2AF} yields the following:

\begin{prop}\label{prop:E1wssalg}
	An additive basis for $\E{wss}{alg}_1$ is given by elements
	$$ x h_2^{k_2} h_3^{k_3}h_4^{k_4} \ldots $$
	indexed by $K = (k_2, k_3, \ldots)$, detected by $x(I[K])$ where 
	$$ I[K] = (\underbrace{1, \ldots ,1}_{k_2}, \underbrace{2, \ldots ,2}_{k_3}, \underbrace{4, \ldots ,4}_{k_4}, \ldots). $$
	Here, the elements $x$ run through a basis of $\E{ass}{}_2(b_{I[K]})$.
\end{prop}

By Proposition~\ref{prop:d1alg}, the remaining differentials in the algebraic weight spectral sequence are given by
\begin{align*}
d_{2^r}^{wss, alg}(w^{2^r a}h_2^{k_2} \cdots h_l^{k_l}) & = w^{2^r(a-1)} h_{r+2} h_2^{k_2} \cdots h_l^{k_l}, \: a \: \mr{odd}. 
\end{align*}
when the source and target persist to $\E{wss}{alg}_{2^r}$.

\begin{lem}\label{lem:drwssalg}
The non-trivial differentials in the algebraic weight spectral sequence are given by
\begin{align*}
d_{2^{r-2}}^{wss, alg}(w^{2^{r-2} a}h_{r'}^{k_{r'}}  \cdots h_l^{k_l}) & = w^{2^{r-2}(a-1)} h_r h_{r'}^{k_{r'}}  \cdots h_l^{k_l}, \: k_{r'} > 0, \: r \le r', \: a \: \mr{odd},
\end{align*}
where $r\geq 2$.
\end{lem}


\begin{proof}
The differentials
\begin{align*}
d_{2^{r-2}}^{wss, alg}(w^{2^{r-2} a}h_{r'}^{k_{r'}}  \cdots h_l^{k_l}) & = w^{2^{r-2}(a-1)} h_r h_{r'}^{k_{r'}}  \cdots h_l^{k_l}, \: k_{r'} > 0, \: r > r', \: a \: 
\mr{odd}
\end{align*}
never get a chance to run, because the sources of these potential differentials are targets of the shorter differentials:
\[d_{2^{r'-2}}^{wss, alg}(w^{2^{r-2} a+2^{r'-2}}h_{r'}^{k_{r'}-1}  \cdots h_l^{k_l})  = w^{2^{r-2} a}h_{r'}^{k_{r'}}  \cdots h_l^{k_l}. \qedhere\]
\end{proof}


To describe $\E{wss}{alg}_\infty$, we need some notation.  Consider the ``$\bo$-pattern'':
\begin{center}
\includegraphics[width=0.7\linewidth]{bopattern}
\end{center}
We shall use $\bo_i[j]$ to denote the sub-pattern where we only include every $(2^i)$th $v_0$ tower, and truncate these $v_0$-towers to have length $j$.  For example, $\bo_1[3]$ is given by:
\begin{center}
\includegraphics[width=0.7\linewidth]{bo13}
\end{center}
whereas $\bo_2[7]$ is given by:
\begin{center}
\includegraphics[width=0.7\linewidth]{bo27}
\end{center}
We also need to consider an analog of these patterns for $\bsp$. Let $\bsp_1[3]$ denote the following pattern:
\begin{center}
\includegraphics[width=0.7\linewidth]{bsp13}
\end{center}
Finally, we let $\bo_i[j]^{\bra{k}}$ denote the ``$k$th Adams cover'' of the pattern $\bo_i[j]$.  The pattern $\bo_1[3]^{\bra{2}}$ is:
\begin{center}
\includegraphics[width=0.7\linewidth]{bo132}
\end{center}

With regard to these patterns, $H^{*,*,*}(\mc{C}_{alg})$ is described by
the following theorem.

\begin{thm}\label{thm:HCalg}
The groups $H^{*,*,*}(\mc{C}_{alg})$ have a basis given by
\begin{align*}
v_0^i \cdot 1, \: & i \ge 0, \\
x \cdot h_r^{k_r} \ldots h_l^{k_l}, \: & r \ge 2, \: k_r \ge 1, \: \mr{all \: other} \: k_j \ge 0, 
\end{align*}
where
$$ K = (0, \ldots, 0, k_r, \ldots, k_l) $$
and $x$ runs through a basis of
$$
\begin{cases}
\bsp_1[3]^{\bra{2\norm{I[K]}-\abs{I[K]}-1}}, & r = 2, k_2 \: \mr{odd}, \\
\bo_{r-1}[2^r-1]^{\bra{2\norm{I[K]}-\abs{I[K]}}}, & \mr{otherwise}.
\end{cases}
$$
\end{thm}

\begin{rmk}
We note for the reader's convenience that
\begin{align*}
\norm{I[K]} & = k_2+2k_3+4k_4+\cdots, \\
\abs{I[K]} & = k_2 + k_3 + k_4 + \cdots .
\end{align*}
\end{rmk}


\begin{proof}
Proposition~\ref{prop:E1wssalg} gives a basis of $\E{wss}{alg}_1$ as
$ xh_2^{k_2} \cdots h_l^{k_l} $
where $x$ runs through a basis of
$$
\begin{cases}
\E{ass}{}_2\left(\bsp^{\bra{2\norm{I[K]}-\abs{I[K]}-1}}\right), &  k_2 \: \mr{odd}, \\
\E{ass}{}_2\left( \bo^{\bra{2\norm{I[K]}-\abs{I[K]}}} \right), & \mr{otherwise}.
\end{cases}
$$
Lemma~\ref{lem:drwssalg} implies that if 
$$ K = (0, \ldots, 0, k_r, \ldots, k_l) $$
with $k_r \ge 1$, then only every $2^{r-1}$ $v_0$-tower
in these patterns persist to $\E{wss}{alg}_\infty$.  Moreover, these $v_0$-towers are truncated by differentials emanating from the classes
$$ y h_r^{k_r-1}\cdots h_l^{k_l}. $$
Assuming $k_2$ is even, ($k_2$ odd is a similar separate case) and letting 
$$ K' = (0, \ldots, 0, k_r-1, \ldots, k_l) $$
we compute the height of the $v_0$-towers to be
\[ (2\norm{I[K]}-\abs{I[K]}) - (2\norm{I[K']}-\abs{I[K']}) = 2^r-1. \qedhere\]
\end{proof}

By Proposition~\ref{prop:drwss}, the differentials in the algebraic weight spectral sequence (in $\bo$-filtration $\ge 2$) are simply a restriction of the differentials in the topological weight spectral sequence. We can thus read off $H^{n,*}(\mc{C})$ from $H^{n,0,*}(\mc{C}_{alg})$ for $n \ge 2$.  The resulting description of these cohomology groups is given below.


\begin{cor}\label{cor:HC}
The cohomolology of the topological good complex $\mc{C}^{*,*}$ is given in degrees $0$ and $1$ by:
\begin{align*}
H^{0,*}(\mc{C})  = & \ZZ\{1\} \oplus \ZZ/2\{v_1^{4i}h_1^j \: : \: i \ge 0, \: j = 1,2 \}, \\
H^{1,*}(\mc{C})  = & \ZZ/8 \{ v_1^{4i}h_2 \: : \: i \ge 0 \} \\
& \oplus 
\ZZ/2^{\nu_2(i+1)+4}\{ v_1^{4i} \cdot v_0wh_2 \: : \: i \ge 0 \} \\
& \oplus
\ZZ/2\{v_1^{4i}h_1^j \cdot v_0wh_2 \: : \: i \ge 0, \: j = 1,2 \}.
\end{align*}
In degrees $n \ge 2$, these cohomology groups are given by the subspace
$$ H^{n,*}(\mc{C}) \subseteq H^{n,0,*}(\mc{C}_{alg}) $$
generated by elements of the form
$$ x \cdot h_2^{k_2} \cdots h_l^{k_l}, \quad k_l \ge 2 $$
for $x$ as in Theorem~\ref{thm:HCalg}.
\end{cor}

In particular, we can derive a vanishing line for $H^{*,*}(\mc{C})$.

\begin{cor}\label{cor:HCvanishing}
We have
$$ H^{n,t}(\mc{C}) = 0 $$
for $n \ge 1/5(t-n) + 2$.
\end{cor}

\begin{proof}
By Corollary~\ref{cor:HC}, the lowest $t-n$ where a class
$$ x h_r^{k_r} \cdots h_l^{k_l} $$
can contribute to $H^{n,t}(\mc{C})$
is
\begin{align*}
 t-n & = \underbrace{k_r(2^r-1)+\cdots +k_l(2^l-1)}_{\abs{h_2^{k_2}\cdots h_l^{k_l}}} + \underbrace{2}_{\abs{v_1}}(\underbrace{2\norm{I[K]} - \abs{I[K]} - \epsilon}_{\text{Adams cover}} - \underbrace{(2^{r}-2)}_{\text{$v_0$-truncation}}) \\ 
 & = (2^{r+1}k_2 + \cdots + 2^{l+1}k_l) - 3(k_2 + \cdots + k_l) - 2^{r+1}+4-2\epsilon \\
 & \ge (2^{r+1}-3)(k_2 + \cdots + k_l)  - 2^{r+1}+2. 
\end{align*}
where
$$
\epsilon = \begin{cases}
0,  & k_2 \: \mr{even}, \\
1, & k_2 \: \mr{odd}.
\end{cases}
$$
Since
$$ n = k_2 + \cdots + k_l $$
and $r \ge 2$, we deduce that such a class must satisfy
\begin{align*}
n & \le \frac{1}{2^{r+1}-3}(t-n) + \frac{2^{r+1}-2}{2^{r+1}-3} \\
& < \frac{1}{5}(t-n) + 2.
\end{align*}
\end{proof}