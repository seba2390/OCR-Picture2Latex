% !TEX root = boASS6.tex

\section{The weight spectral sequence}\label{sec:wss}

In order to compute $H^{*,*}(\mc{C})$, Lellmann and Mahowald \cite{LM} introduced a \emph{weight spectral sequence}.  This section is a summary of their work.  Endow $\mc{C}^{*,*}$ with a decreasing filtration by weight ({$wt$}), where we define
$$ wt(x(I)) = \norm{I}. $$
We see from Theorem~\ref{thm:d1} that $d_1^{good}$ does not decrease weight.  There is a resulting spectral sequence
$$ \E{wss}{}_0^{n,*,w} := [\mc{C}^{n,*}]_{wt=w} \Rightarrow H^{n,*}(\mc{C}) $$
with differentials
$$ d_r^{wss}: \E{wss}{}_r^{n,*,w} \rightarrow \E{wss}{}_r^{n+1, *, w+r}. $$
From Theorem~\ref{thm:d1}, we see that $d_0^{wss}$ is given (modulo terms of higher Adams filtration) by
\begin{equation}\label{eq:d0wss} 
[d_0^{wss}(x(i_1, \ldots, i_n))] = \sum_k \sum_{i_k'+i_k'' = i_k} v_0^{\alpha(i_k') + \alpha(i_k'')-\alpha(i_k)} x(i_1, \ldots, i_k', i_k'', \ldots, i_n). 
\end{equation}
Even with this explicit differential, the calculation of $\E{wss}{}_1$ is not immediate.  

To this end, paraphrasing \cite{LM}, we will introduce an \emph{Adams filtration spectral sequence} $\{\E{AF}{}_r\}_{r \ge 0}$ to compute $\E{wss}{}_1$.  We will be able to compute $\E{AF}{}_1$ directly, but the computation of $\E{AF}{}_2$ will be clarified by the use a \emph{lexographical filtration spectral sequence} $\{ \E{LF}{}_\alpha\}$ (technically speaking, this spectral sequence is indexed on a totally ordered set).  The situation is summarized by the following:
$$ \E{wss}{}_0 = \E{AF}{}_0 \rightsquigarrow \E{AF}{}_1 = \E{LF}{}_0\Rightarrow \E{AF}{}_2 = \E{wss}{}_1 \Rightarrow H(\mc{C}). $$ 


\subsection*{The Adams filtration spectral sequence}

Endow $\E{wss}{}_0^{*,*,*}$ with a decreasing filtration by Adams filtration (AF), where we define
$$ AF(x(I)) := AF(x). $$
There is a resulting \emph{Adams filtration spectral sequence} with
$$ \E{AF}{}_0^{n,*,w,a} := [\E{wss}{}_0^{n,*,w}]_{AF = a} \Rightarrow \E{wss}{}_1^{n,*,w} $$
with differentials
$$ d_r^{AF}: \E{AF}{}_r^{n,*,w,a} \rightarrow \E{AF}{}_r^{n+1, *, w, a+r}. $$

\begin{prop}[Lellman-Mahowald \cite{LM}]
	An additive basis for $\E{AF}{}_1$ is given by elements
	$$ x h_2^{k_2} h_3^{k_3}h_4^{k_4} \ldots $$
	indexed by $K = (k_2, k_3, \ldots)$, detected by $x(I[K])$ where 
	$$ I[K] = (\underbrace{1, \ldots ,1}_{k_2}, \underbrace{2, \ldots ,2}_{k_3}, \underbrace{4, \ldots ,4}_{k_4}, \ldots). $$
	Here, the elements $x$ run through a basis of $\E{ass}{}_2(b_{I[K]})$.
\end{prop}
	
\begin{proof}
By (\ref{eq:d0wss}), 
\begin{equation}\label{eq:d0AF} 
[d_0^{AF}(x(i_1, \ldots, i_n))] = \sum_k \sum_{\substack{i_k'+i_k'' = i_k \\ \alpha(i_k')+\alpha(i_k'') = \alpha(i_k)}} x(i_1, \ldots, i_k', i_k'', \ldots, i_n). 
\end{equation}
	The essential observation is that if $i = i' + i''$ then
	$$ \alpha(i) = \alpha{(i')} + \alpha{(i'')}, $$
	if and only if $i'$ and $i''$ have dyadic expansions with $1$'s in complementary places. 
	Consider the subcomplex $\mc{B}^n$ of $\E{AF}{}_0^{n,*,*,*}$ spanned by the terms $1(I)$.  There is an isomorphism
	$$ \mc{B}^* \cong \bar{C}^{*}(\FF_2, E, \FF_2) $$
	(the normalized cobar complex for the primitively generated exterior algebra $E = \Lambda_{\FF_2}[e_0, e_1, \ldots]$) given by 
	$$ 1(i_1, \ldots, i_n) \mapsto [e(i_1)|\ldots|e(i_n)]. $$
	Here, for $i$ expressed dyadically as 
	$$ i = \epsilon_0 + \epsilon_1 2 + \epsilon_2 4 + \cdots $$
	(with $\epsilon_j \in \{0,1\}$), the element $e(i)$ denotes the element
	$$ e(i) := e_0^{\epsilon_0} e_1^{\epsilon_1} \cdots \in E. $$
	We deduce
	\begin{align*} H^*(\mc{B}) \cong & \Ext^*_{E}(\FF_2, \FF_2) \cong
		 \FF_2[h_2, h_3, \cdots ]
    \end{align*}
	(where $h_i$ is represented by the cocycle $[e_{i-2}]$ in the cobar complex).  The structure of $\E{AF}{}_1$ follows from the fact that for 
	$$ I = (i_1, \ldots, i_n) $$ 
	and
	$$ I' = (i_1, \ldots, i_k', i''_k, \ldots i_n) $$
	with $i_k = i'_k + i''_k$ and
	$$ \alpha(i_k) = \alpha(i_k')+\alpha(i_k'') $$
	we have
	$$ b_I \simeq b_{I'}. $$
\end{proof}

\begin{rmk}
	Our naming of the polynomial generators $h_2, h_3, \ldots$ of $\E{AF}{}_1$ may seem bizarre, and differs from \cite{LM}, where these generators are named $h_0, h_1, \ldots$.  Our reason for this different indexing is that in our notation, $h_i$ will correspond to the element of the same name in the classical Adams spectral sequence for the sphere.
\end{rmk}

\subsection*{The lexigraphical filtration spectral sequence}

In order to compute $\E{AF}{}_2$, we observe that from (\ref{eq:d0wss}) we have
$$
d_1^{AF}h_i = 
\begin{cases}
v_0 h_{i-1}^2, & i \ge 3, \\
0, & i = 2.
\end{cases}
$$
The Adams filtration spectral sequence, like all the spectral sequences we are employing, is multiplicative, and hence
$$ d_1^{AF}(xh_2^{k_2}h_3^{k_3}\cdots h_l^{k_l}) = \sum_{\substack{3 \le j \le l \\ k_j \: \mr{odd}}} v_0xh_2^{k_2} \cdots h_{j-1}^{k_{j-1}+2} h_j^{k_j-1} \cdots h_l^{k_l}. $$
For the purposes of analyzing the resulting cohomology, it is useful to order the monomials 
$$ h_2^{k_2} h_3^{k_3} h_4^{k_4} \cdots $$
by left lexigraphical ordering on the sequence 
$$
K = (k_2, k_3, k_4, \ldots ).
$$
In this filtration, we have
$$ (k_2, k_3, k_4, \ldots) < (k'_2, k'_3, k'_4, \ldots) $$
if $k_2 < k'_2$, or $k_2 = k'_2$ and $k_3 < k_3'$, or $k_2 = k_2'$ and $k_3=k_3'$ and $k_4 < k'_4$, etc.

We will call the resulting filtration (indexed by the ordinal $\omega^\omega$) \emph{lexigraphical filtration} (LF).  
The filtration is multiplicative, and the differential $d_2^{AF}$ increases lexigraphical filtration.  Amusingly, one way to organize the resulting cohomology is via the $\omega^\omega$-indexed spectral sequence based on this filtration (c.f. \cite{Hu}, \cite{Matschke}):
$$ \E{LF}{}_\alpha \Rightarrow \E{AF}{}_2. $$
The differentials in this spectral sequence (which for expediency of notation we do not index) are
\begin{equation}\label{eq:dLF} d^{LF}(x h_2^{k_2} h_3^{k_3} \cdots h_l^{k_l}) = v_0 x h_2^{k_2} h_3^{k_3} \cdots h_{l-1}^{k_{l-1}+2}h_{l}^{k_l-1}, \quad k_l \: \mr{odd}.
\end{equation}

\begin{prop}[Lellman-Mahowald \cite{LM}, \cite{Mahowaldbo}]\label{prop:E2AF}
The Adams filtration spectral sequence term $\E{AF}{}_2$ has a basis given by elements
$$ x h_2^\epsilon, \qquad \epsilon = 0,1, $$
and
$$ y h_2^{k_2} h_3^{k_3} \cdots h_{l}^{k_l}, \qquad k_l \ge 2, $$
where $x$ runs through a basis of 
$$
\begin{cases} \E{ass}{}_2^{*,*}(\bo), & \epsilon = 0, \\ \E{ass}{}_2^{*,*}(\bsp), & \epsilon = 1, \end{cases}
$$ and $y$ runs through a basis of
$$ \E{ass}{}^{0,*}(b_{I[k_2, \ldots, k_l]}). $$
\end{prop}

\begin{proof}
The proof amounts to analyzing the result of running the differentials of (\ref{eq:dLF}) in order of increasing lexigraphical filtration.  This analysis is simplified by the fact that the differentials in the lexigraphical filtration spectral sequence send monomials to monomials.

Equation (\ref{eq:dLF}) implies that 
$$ d^{LF}(x h_2^\epsilon )= 0. $$
As nothing can hit these classes, these provide the first part of the basis.  Note that for $K = (k_2, \ldots, k_l)$
\begin{align*}
\norm{I[K]} & = k_2+2k_3+ \cdots + 2^{l-2}k_l, \\
\alpha(I[K]) & = k_2 + k_3 + \cdots + k_l
\end{align*}
and therefore for $K' = (k_2, k_3, \cdots, k_l+2, k_l-1)$ we have
\begin{align*}
\norm{I[K']} = \norm{I[K]},  \\
\alpha(I[K']) = \alpha(I[K])+1.
\end{align*}
It follows that
$$ b_{I[K]} = b^{\bra{1}}_{I[K']}. $$
Therefore the differentials
$$ d^{LF}(xh_2^{k_2} \cdots h_{k-1}^{k_{l-1}}h_l) = v_0 h_2^{k_2} \cdots h_{l-1}^{k_{l-1}+2} $$
are all non-trivial.  The only elements not hit by the differentials above are of the form
$$ x h_2^{a_2} \cdots h_l^{a_m} $$
with $a_m \ge 2$ and $AF(x) = 0$.  The remaining possible differentials
$$ d^{LF}(xh_2^{k_2} \cdots h_{k-1}^{k_{l-1}}h_l^{k_l}) = v_0 h_2^{k_2} \cdots h_{l-1}^{k_{l-1}+2}h_{l}^{k_l-1} $$
(with $k_l \ge 3$ odd) are all zero since their targets are already killed by the shorter differentials
$$ d^{LF}(xh_2^{k_2} \cdots h_{k-1}^{k_{l-1}+2}h_l^{k_l-3}h_{l+1}) = v_0 h_2^{k_2} \cdots h_{l-1}^{k_{l-1}+2}h_{l}^{k_l-1}. $$
\end{proof}

We have at this point deduced Mahowald's ``Bounded Torsion Theorem'' \cite{Mahowaldbo}:

\begin{cor}\label{cor:BTT}
For $n \ge 2$ and $a > 0$, we have
$$ \E{AF}{}_2^{n,*,*,a} = 0. $$
\end{cor}

Because what remains in $\E{AF}{}_2$ (with the exception of the classes $xh_2^\epsilon$) is concentrated in Adams filtration 0, there are no further differentials in the Adams filtration spectral sequence, and we deduce the following corollary.

\begin{cor}\label{cor:E1wss}
We have
$$ \E{AF}{}_2 = \E{AF}{}_\infty $$
and $\E{wss}{}_1$ is essentially given by Proposition~\ref{prop:E2AF}.
\end{cor}

\subsection*{The higher differentials in the weight spectral sequence}

We now compute the remaining differentials in the weight spectral sequence.  By \ref{thm:d1}, these are given by
\begin{equation}\label{eq:drwss} [d_{2^r}^{wss} (w^m h_2^{k_2} \cdots h_l^{k_l})] = v_0^{\alpha(m-2^r)-\alpha(m)+1} w^{m-2^r}h_{r+2} h_2^{k_2} \cdots h_l^{k_l}.
\end{equation}
(provided the source and target persist to $\E{wss}{}_{2^r}$).

\begin{prop}[Lellmann-Mahowald \cite{LM}]\label{prop:drwss}
The remaining differentials in the weight spectral sequence are given by
\begin{align*}
[d_{1}^{wss}(v_0^iw^{m})] & = v_0^{i+\nu_2(m)}w^{m-1} h_2, \\
[d_{2^r}^{wss}(w^{2^r a}h_2^{k_2} \cdots h_l^{k_l})] & = w^{2^r(a-1)} h_{r+2} h_2^{k_2} \cdots h_l^{k_l}, \quad k_l \ge 2, \: a \: \mr{odd}. 
\end{align*}
\end{prop}

\begin{proof}
The first formula follows from (\ref{eq:drwss}) and the fact that 
$$ \alpha(m-1) - \alpha(m)+1 = \nu_2(m). $$
The second formula follows from the fact that, for $k_l \ge 2$, the classes
$$ v_0^i w^m h_2^{k_2} \cdots h_l^{k_l} $$
die in the lexigraphical filtration spectral sequence for $i > 0$.  Thus the only possible non-trivial differentials coming from (\ref{eq:drwss}) are
$$ [d_{2^r}^{wss} (w^m h_2^{k_2} \cdots h_l^{k_l})] = w^{m-2^r}h_{r+2} h_2^{k_2} \cdots h_l^{k_l} $$
for $r$ such that the dyadic expansion of $m$ has a $1$ in the $r$th place.  The first of these is $r = \nu_2(m)$.
\end{proof}

In \cite{LM}, Lellmann and Mahowald proceed to deduce a closed form computation for $H^{*,*}(\mc{C})$.  We give our own description of this cohomology, based on the algebraic good complex $\mc{C}^{*,0,*}_{alg}$, in Corollary~\ref{cor:HC}. 





