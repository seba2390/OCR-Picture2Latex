% !TEX root = boASS6.tex



\section{The agathokakological spectral sequence}\label{sec:rules}

At the end of the day we would like to deduce the evil classes $H^{*,*}(V)$ (in a range) directly from the known quantities $H^{*,*,*}(\mc{C}_{alg})$ and $\Ext^{*,*}_{A_*}(\FF_2)$.  This is somewhat confusing, as the relationship of these three quantities occurs through the combination of a spectral sequence (the $\bo$-MSS) and a long exact sequence (\ref{eq:algLES}). To mitigate this complication, we introduce a variant of the $\bo$-MSS, called the \emph{agathokakological spectral sequence} (AKSS), which combines the three quantities directly:
$$ \E{akss}{alg}_{1+\epsilon}^{*,*,*} = H^{*,*}(V) \oplus H^{*,*,*}(\mc{C}_{alg})\Rightarrow \Ext_{A_*}^{*,*}(\FF_2). $$
(The indexing of this spectral sequence is unorthodox, and will be explained.)

\subsection*{The construction of the spectral sequence}

For convenience, fix a splitting of the short exact sequence
\begin{equation}\label{eq:SES} 
0 \rightarrow V^{n,*} \rightarrow \E{bo}{alg}^{n,*,*}_1 \xrightarrow{g_{alg}} \mc{C}^{n,*,*}_{alg} \rightarrow 0.
\end{equation}
Recall that with respect to this splitting, we can decompose elements $x \in \E{bo}{alg}_1$ as
$$ x = x_{evil} + x_{good} $$
with $x_{evil} \in V$ and $x_{good} \in \mc{C}_{alg}$.  While (\ref{eq:SES}) does \emph{not} split as a short exact sequence of chain complexes, it does
introduce a micrograding on the filtration $n$-layer of the $\bo$-MSS.
We will regard the evil subcomplex $V^{n,*}$ as being in filtration $n + \epsilon$, where $\epsilon$ is regarded as being infinitesimal.  The new grading is on the ordered set
$$ \mb{N}^\epsilon := \{ n + \alpha \epsilon \: : \: n \in \mb{N}, \: \alpha \in \{0,1\} \}. $$
A total ordering is defined by
$$ n < n + \epsilon < n+1. $$
The result is a spectral sequence indexed on $\mb{N}^\epsilon$ (see \cite{Matschke}) which we call the \emph{agathokakological spectral sequence} (AKSS):
$$ \{ \E{akss}{alg}^{n+\alpha\epsilon,s,t}_{r + \beta \epsilon} \} \Rightarrow \Ext^{n+s, t}_{A_*}(\FF_2) $$
with
$$\E{akss}{alg}^{n+\alpha \epsilon, s, t}_1 = 
\begin{cases}
\mc{C}^{n, s, t}_{alg}, & \alpha = 0, \\
V^{n, t}, & \alpha = 1, \: s = 0, \\
0, & \text{otherwise}.
\end{cases}
$$
The pages are indexed on the totally ordered set
$$ \mb{N}^{\pm \epsilon} := \{ n + \beta \epsilon \: : \: n \in \mb{N}, \: \beta \in \{-1,0,1\} \} $$
with
$$ n - \epsilon < n < n + \epsilon < n+1. $$
The differentials take the form
\begin{align*}
d^{akss}_{r-\epsilon}:  & \E{akss}{alg}_{r - \epsilon}^{n+ \epsilon, s, t} \rightarrow \E{akss}{alg}_{r- \epsilon}^{n+r, s-r+1, t}, \\
d^{akss}_{r}:  & \E{akss}{alg}_{r}^{n+ \alpha\epsilon, s, t} \rightarrow \E{akss}{alg}_{r}^{n+r +\alpha\epsilon, s-r+1, t}, \\
d^{akss}_{r+\epsilon}:  & \E{akss}{alg}_{r+\epsilon}^{n, s, t} \rightarrow \E{akss}{alg}_{r+ \epsilon}^{n+r +\epsilon, s-r+1, t}.
\end{align*}
Given an element $x \in \E{akss}{alg}_{r+\beta\epsilon}^{n+\alpha\epsilon, s, t}$, we will define
\begin{align*}
n(x) & := n, \\
s(x) & := s, \\
t(x) & := t.
\end{align*}
If $\alpha = 0$, we will refer to the element $x$ as \emph{good}, and if $\alpha = 1$, we refer to $x$ as \emph{evil}.  
The $d_1$-differential 
$$ d^{akss}_{1}: \E{akss}{alg}^{n+\alpha \epsilon, s, t}_1 \rightarrow \E{akss}{alg}^{n+ 1+\alpha\epsilon, s,t}_1
$$
is given by
$$d_1 = 
\begin{cases}
d_1^{good}, & \alpha = 0, \\
d_1^{evil}, & \alpha = 1, s = 0, \\
0, & \text{otherwise}. 
\end{cases}
$$
We therefore have
$$\E{akss}{alg}^{n+\alpha \epsilon, s, t}_{1+\epsilon} = 
\begin{cases}
H^{n, s, t}(\mc{C}_{alg}), & \alpha = 0, \\
H^{n, t}(V), & \alpha = 1, \: s = 0, \\
0, & \text{otherwise}.
\end{cases}
$$
The only nonzero $d_{1+\epsilon}$-differentials are of the form
$$ H^{n, 0, t}(\mc{C}_{alg}) = \E{akss}{alg}^{n, {0}, t}_{1+\epsilon} \xrightarrow{d_{1+\epsilon}} \E{akss}{alg}^{n+1 + \epsilon, 0, t}_{1+\epsilon} = H^{n+1, t}(V), $$
for which we have 
$$ d_{1+\epsilon} = \partial_{alg} $$
where $\partial_{alg}$ is the connecting homomorphism of (\ref{eq:algLES}).
We therefore have
$$ \E{akss}{alg}^{n+\alpha\epsilon,s,t}_2 = 
\begin{cases}
\mr{Im} \left( \E{bo}{alg}^{n, s, t}_{{2}} \xrightarrow{g_{alg}} H^{n, s, t}(\mc{C}_{alg}) \right), & \alpha = 0, \\
\mr{Ker} \left( \E{bo}{alg}^{n, {0}, t}_{{2}} \xrightarrow{g_{alg}} H^{n, {0}, t}(\mc{C}_{alg}) \right), & \alpha = 1, \: s = 0, \\
0, & \text{otherwise}.
\end{cases} $$
Thankfully, as the differentials $d^{bo,alg}_r$ in the $\bo$-MSS decrease $s$ by $r-1$, and the evil classes are concentrated in $s = 0$, the only other potentially non-trivial differentials ($r \ge 2$) in the AKSS are of the form:
\begin{align*}
d^{akss}_r: & \E{akss}{alg}_{r}^{n, s, t} \rightarrow \E{akss}{alg}_{r}^{n+r, s-r+1, t}, \\
d^{akss}_{s + 1 + \epsilon}: & \E{akss}{alg}_{r+\epsilon}^{n, s, t} \rightarrow \E{akss}{alg}_{r+\epsilon}^{n+s+1+\epsilon, 0, t} \\
\end{align*}
in which case they are determined by 
\begin{align*}
d^{akss}_r(x_{good}) & = \left[d^{bo, alg}_r(x_{good}) \right]_{good}, \\
d^{akss}_{s+1+\epsilon}(x_{good}) & = \left[d^{bo, alg}_r(x_{good}) \right]_{evil}.
\end{align*}


The behavior of the differentials in the AKSS is summarized by the following lemma.
\begin{lem}\label{lem:akdr}
In the agathokakological spectral sequence, given a non-trivial differential
$$ d_{r + \beta\epsilon}(x) = y $$
for $r \ge 1$ and $\beta \in \{ -1,0,1\}$, there are three possibilities:
\begin{enumerate}
\item $x$ and $y$ are both evil, $r = 1$, $\beta = 0$,
\item $x$ and $y$ are both good, $r \ge 1$, $\beta = 0$,
\item $x$ is good and $y$ is evil, $r = s(x)+1$, $\beta = 1$. 
\end{enumerate}
In other words, {\bf evil can never triumph over good}.
\end{lem}

The reader might wonder why the authors elected to index the AKSS on $\mb{N}^\epsilon$ rather than, say, half integers.  The reason is that since the map $g_{alg}$ of (\ref{eq:SES}) is a map of algebras, $V^{*,*}$ gets the structure of an ideal, and the multiplicative structure of the $\bo$-MSS descends to a multiplicative structure on the AKSS.  

Specifically, define a monoid structure on $\mb{N}^\epsilon$
determined by the rule $\epsilon+\epsilon = \epsilon$, so we have
$$ (n + \alpha \epsilon) + (n'+ \alpha' \epsilon) = 
\begin{cases}
n + n', & \alpha = \alpha' = 0, \\
(n+n') + \epsilon & \text{otherwise}.
\end{cases}
$$
Then there is a product map
$$ \E{akss}{alg}^{n+\alpha \epsilon, s, t}_{r+\beta\epsilon} \otimes 
\E{akss}{alg}^{n'+\alpha' \epsilon, s', t'}_{r+\beta\epsilon}
\rightarrow \E{akss}{alg}^{(n+\alpha \epsilon)+(n'+\alpha' \epsilon), s+s', t+t'}_{r+\beta\epsilon}. $$
The differential $d_{r+\beta\epsilon}$ is a derivation with respect to this product, provided one interprets that to mean
$$ d_{r-\epsilon}(xy) = 0 $$
if $x$ and $y$ are both evil.

With respect to this multiplicative structure, the product of good classes is good, and the product of evil classes is evil (which is why we require $\epsilon + \epsilon = \epsilon$).  For dimensional reasons, only certain kinds of hidden extensions can occur:

\begin{lem}\label{lem:akhe}
Suppose that there is a hidden extension in the AKSS given by
\[ \td{x}\td{y} = \td{z} \ne 0 \]
in $\Ext^{*,*}_{A_*}(\FF_2)$, where 
$x$, $y$, and $z$ detect $\td{x}$, $\td{y}$, and $\td{z}$ in the AKSS, respectively, with $xy = 0$ in $\E{akss}{alg}_\infty$.  Then either
\[ s(z) < s(x) + s(y) \]
or
\[ s(z) = s(x) = s(y) = 0 \: \text{and $x$ and $y$ are good, and $z$ is evil.}  \]
\end{lem}

\begin{rmk}
When $s(x) \leq 1$, this imposes strict restrictions on $y$. If $s(x)=0$ (for example if $x=h_i$ when $i\geq 2$), $y$ cannot be evil, for this would force $s(z) < 0$. 
Similarly, if $s(x)=1$ (for example, if $x = v_0$ or $h_1$), if $y$ is evil, then $s(z) = 0$. 
\end{rmk}






\subsection*{Comparison with $\AA0$}

The entire construction of the AKSS goes through without modification when the $A_*$-comodule $\FF_2$ is replaced by $\AA0$:
\[ \{ \E{akss}{alg}_{r+\beta\epsilon}^{n+\alpha \epsilon,s,t}(\AA0) \} \Rightarrow \Ext_{A_*}^{*,*}(\AA0) \]
with 
\[\E{akss}{alg}^{n+\alpha \epsilon, s, t}_{1+\epsilon}(\AA0) = 
\begin{cases}
H^{n, s, t}(\mc{C}_{alg}(\AA0)), & \alpha = 0, \\
H^{n, t}(V(\AA0)), & \alpha = 1, \: s = 0, \\
0, & \text{otherwise}.
\end{cases}
\]
The analysis of Section~\ref{sec:AA0}, in particular Proposition~\ref{prop:E2boMSSAA0}, comparing the $\bo$-MSS's of $\FF_2$ and $\AA0$ refines to give a comparison between the respective AKSS's.

\begin{thm}
We have
\[\E{akss}{alg}^{n+\alpha \epsilon, s, t}_{1+\epsilon}(\AA0) \cong 
\begin{cases}
\E{akss}{alg}^{n+1, s, t}_{1+\epsilon}(\FF_2), & \alpha = 0, t-s \equiv 0 \mod 4, \\
\E{akss}{alg}^{n, s+1, t}_{1+\epsilon}(\FF_2), & \alpha = 0, t-s \not\equiv 0 \mod 4, \\
\E{akss}{alg}^{n+1+\epsilon,0,t}_{1+\epsilon}(\FF_2), & \alpha = 1, \: s = 0, t \equiv 0 \mod 4, \\
\E{akss}{alg}^{n+1+\epsilon,0,t}_2(\FF_2)  \\
\oplus \E{akss}{alg}^{n+1,0,t}_2(\FF_2), & \alpha = 1, \: s = 0, t \not\equiv 0 \mod 4, \\
0, & \text{otherwise}.
\end{cases}
\]
Moreover, under this isomorphism, all differentials commute (but potentially changing lengths as the indexing changes dictate).
\end{thm}

As this theorem is essentially just a translation of the results of Section~\ref{sec:AA0} into agathokakological indexing, we will not say anything more about the proof.  However, the second to last case in the statement of the theorem ($\alpha = 1, s = 0, t \not\equiv 0 \mod 4$) does merit clarification.  The $h_1$-good classes in $\E{akss}{alg}^{n,s,t}_{alg}(\FF_2)$ become $h_1$-good classes in $\E{akss}{alg}^{n, s-1, t}(\AA0)$ for $s > 0$, but for $s = 0$ they become evil classes (in $\E{akss}{alg}^{n-1+\epsilon,0,t}_{alg}(\AA0)$).
A $d_{1+\epsilon}^{akss}$ differential from an $h_1$-good class to an evil class in the AKSS for $\FF_2$ becomes a $d_1^{akss}$ differential between the corresponding evil classes in the AKSS for $\AA0$.

\subsection*{The principle of dichotomy}

Given a non-trivial class $x \in \Ext^{*,*}_{A_*}(M)$, for $M = \FF_2$ or $\AA0$,  we will say that $x$ is \emph{good} if it is detected by a good class in the AKSS, and we will say $x$ is \emph{evil} if it is detected by an evil class.  We will say $x$ is \emph{$v_1$-periodic} if its image in $v_1^{-1}\Ext$ is non-trivial, and otherwise we will say that $x$ is \emph{$v_1$-torsion}.
We will now answer the following fundamental question:
\begin{quote}
{\it How do you determine if a given Ext class is good or evil?}
\end{quote}

Recall from Theorem~\ref{thm:AdamsP} (in the case of $\AA0$) and Proposition~\ref{prop:v1ext} (in the case of $\FF_2$), for every class $x \in \Ext^{s,t}_{A_*}(\FF_2)$ with $t-s > 0$, there is an $N = N(x)$ such that
\begin{equation}\label{eq:xxxxNxxx} v_1^{2^N \cdot k} x \in \Ext^{s+2^N\cdot k,t+3\cdot 2^N\cdot k}_{A_*}(\FF_2) \end{equation}
is defined for all $k$. For these classes, being $v_1$-periodic is equivalent to requiring 
$$ v_1^{2^N\cdot k} x \ne 0 $$
for all $k$.  

For $\FF_2$, the only classes in $t - s = 0$ are $v_0^i$, and these are all good (and technically are $v_1$-periodic in the sense that their image in $v_1^{-1}\Ext$ is non-trivial, though they are $v_1$-periodic of ``period $\infty$'').  For $\AA0$, there are no non-trivial classes in $t-s \le 1$.  We may therefore restrict our attention to those classes with $t-s > 0$.


A naive hope would be: ``$x$ is $v_1$-periodic if and only if $x$ is good.''
\emph{This is not true.}  However, something approximating it is.

By Proposition~\ref{prop:v1ext}, any class 
$$ x \in \Ext^{s,t}_{A_*} $$
with $s > 1/3(t-s)$ is automatically $v_1$-periodic.  We shall refer to these classes as being \emph{above the 1/3 line}.

\begin{lem}\label{lem:onethird}
Every class above the $1/3$-line is good.
\end{lem}

\begin{proof}
Since $V^{n,t}$ is a subgroup of $\pi_t bo \wedge \br{\bo}^{\wedge n}$, and $\br{\bo}$ is $3$-connected, it follows that $V^{n,t} = 0$ for $t < 4n$.  The lemma follows from the fact that, in the AKSS, evil classes in $V^{n,t}$ detect elements of $\Ext^{n,t}_{A_*}$. 
\end{proof}

The following is easily checked from the structure of $\mc{C}_{alg}(\FF_2)$ or $\mc{C}_{alg}(\AA0)$.
\begin{lem}\label{lem:v1div}
Suppose $x$ is either an element of $\mc{C}^{n,s,t}_{alg}(\FF_2)$ or $\mc{C}^{n,s,t}_{alg}(\AA0)$.  Then $v_1^{-4k}x$ exists if and only if $4k \le s$.
\end{lem}



\begin{rmk} Note that among good classes in the AKSS, notions of $v_1$-periodicity extend to all the pages in the following manner. There are variants of the AKSS for computing $\Ext^{*,*}_{A(N)_*}(\AA0)$ and $\Ext^{*,*}_{A(N)_*}(\FF_2)$. Thus, we can define $N=N(x)$ as in \eqref{eq:xxxxNxxx} on any page of the AKSS. Moreover $v_1^{2^N}$ is an actual element of $\Ext_{A(N)_*}(\FF_2)$, and the AKSS for $\Ext^{*,*}_{A(N)_*}(\AA0)$ and  $\Ext^{*,*}_{A(N)_*}(\FF_2)$ are spectral sequences of modules over the AKSS for $\Ext^{*,*}_{A(N)_*}(\FF_2)$.  Thus multiplication by $v_1^{2^N}$ commutes with differentials in these AKSS's.
\end{rmk}


The following observation is crucial --- it implies that good classes cannot detect $v_1$-torsion classes in Ext.
\begin{lem}\label{lem:v1torsion}
Suppose that $x$ and $y$ are good classes in the AKSS for $\FF_2$ or $\AA0$, and suppose that there is a non-trivial differential $d^{akss}_r(x) = y$.  Let $N = N(x)$.
\begin{enumerate}
\item If $v_1^{-2^N\cdot k} y$ exists then $v_1^{-2^N\cdot k}x$ exists.
\item For all such $k$, 
$$d_r^{akss}(v_1^{-2^N\cdot k}x) = v_1^{-2^N\cdot k}y. $$
\end{enumerate} 
\end{lem}

\begin{proof}
We use the comparison with the AKSS for computing $\Ext^{*,*}_{A(N)_*}(\FF_2)$ (respectively $\Ext^{*,*}_{A(N)_*}(\AA0)$). By Theorem~\ref{thm:AdamsP}, this spectral sequence is isomorphic to the AKSS for $\Ext^{*,*}_{A_*}(\FF_2)$ (respectively $\Ext^{*,*}_{A_*}(\AA0)$) in a range which includes both $x$ and $y$. 



Suppose inductively that we have proven the lemma for $r < r_0$.  We need to prove it for $r = r_0$.  Since we have
$$ s(y) \le s(x) $$
it follows from Lemma~\ref{lem:v1div} that if $v_1^{-2^N\cdot k} y$ exists, then $v_1^{-2^N\cdot k} x$ exists.  Suppose it is not the case that  
$$ d_r(v_1^{-2^N\cdot k} x) = v_1^{-2^N\cdot k} y. $$
Then $v_1^{-2^N\cdot k} x$ must support a shorter non-trivial differential
$$   d_{r'+\beta\epsilon}(v_1^{-2^N\cdot k} x) = z $$
for $r' < r$.  By the inductive hypothesis $z$ must be evil (because if it was good we would have $d_{r'}(x) = v_1^{2^N\cdot k} z$).  In particular, since $s(z) = 0$, we have
$$ s(x)-2^{N}\cdot k-r'+1 = 0. $$
Since $v_1^{-2^N\cdot k} y$ exists, we have
$$ s(y) \ge 2^N\cdot k. $$
Finally, we have
$$ s(x) - r+1 = s(y). $$
From all these equations we deduce $r \le r'$, a contradiction.
\end{proof}


\begin{thm}\label{thm:dichotomyAA0}
Suppose $x$ is a non-trivial class in $\Ext^{s,t}_{A_*}(\AA0)$.
\begin{enumerate}
\item
If $x$ is $v_1$-torsion, it is evil.

\item 
Suppose $x$ is $v_1$-periodic, and let $N = N(x)$.  Suppose $k$ is taken large enough so that  $v_1^{2^N\cdot k} x$ lies above the $1/3$-line.  Suppose $y$ is a class
which detects $v_1^{2^N\cdot k} x$ in the AKSS ($y$ is necessarily good).  The class $x$ is good if and only if 
\[ s \ge n(y). \]
\end{enumerate}
\end{thm}

\begin{proof}
Suppose $x$ is $v_1$-torsion, but is detected by a good class $\td{x}$ in the AKSS.  Let $N = N(x)$ and suppose that $k$ is chosen so that $v_1^{2^N\cdot k} x$ lies above the 1/3-line and is trivial.  Then 
$v_1^{2^N\cdot k} \td{x}$ is killed by a good class in the AKSS.  Lemma~\ref{lem:v1torsion} then implies that $\td{x}$ is the target of a differential, which violates the non-triviality of $x$.  

Now suppose that $x$ is $v_1$-periodic, detected by an evil class $\td{x}$ in the AKSS.  Again we use the fact that both $x$ and $y$ lie in a range where the AKSS for $\Ext_{A_*}(\AA0)$ is isomorphic to the AKSS for $\Ext_{A(N)_*}(\AA0)$.
The AKSS for $\Ext^{*,*}_{A(N)_*}(\AA0)$ is a spectral sequence of modules over the AKSS for $\Ext^{*,*}_{A(N)_*}(\FF_2)$. Since $\td{x}$ is evil, we have  
\[ v_1^{2^N \cdot k}\td{x}  = 0. \]
We deduce that multiplication by $v_1^{2^N}$ must increase $n$ (hidden extension).  Thus
$$ n(\td{x}) < n(y). $$
Suppose that 
$s \ge n(y).$  Since we have
$$ n(\td{x})+s(\td{x}) = s \ge n(y) $$
we deduce that $s(\td{x}) > 0$.  This violates the assumption that $\td{x}$ is evil, so we deduce that $\td{x}$ is good.

Suppose that $x$ is good, detected by $\td{x}$ in the AKSS.  Then by Lemma~\ref{lem:v1torsion}, $v_1^{2^N\cdot k}\td{x}$ detects $v_1^{2^N\cdot k}x$.  
Then
$$ s = s(\td{x}) + n(\td{x}) = s(\td{x}) + n(y) \ge n(y). $$
\end{proof}

\begin{thm}[Dichotomy Principle]\label{thm:dichotomy}
Suppose $x$ is a non-trivial class in $\Ext^{s,t}_{A_*}(\FF_2)$.
\begin{enumerate}
\item
If $x$ is $v_1$-torsion, it is evil.

\item 
Suppose $x$ is $v_1$-periodic, and let $N = N(x)$.  Suppose $k$ is taken large enough so that  $v_1^{2^N\cdot k} x$ lies above the $1/3$-line.  Suppose $y$ is a class
which detects $v_1^{2^N\cdot k} x$ in the AKSS ($y$ is necessarily good).  The class $x$ is good if and only if 
$$ s \ge n(y). $$
\end{enumerate}
\end{thm}

\begin{proof}
This theorem follows from Theorem~\ref{thm:dichotomyAA0} using the fact that for $t \ne s$, there is an isomorphism
$$ \Ext^{s-1,t}_{A_*}(\AA0) \xrightarrow[\cong]{\bar{\partial}} \Ext^{s,t}_{A_*}(\FF_2). $$
so we have $\bar{\partial}(x') = x$.  Suppose $x$ is $v_1$-torsion (then so is $x'$).  By Theorem~\ref{thm:dichotomyAA0}, $x'$ is evil.  The map $\bar{\partial}$ takes $v_1$-torsion evil classes to evil classes.

Suppose now that $x$ is good, with $y$ detecting $v_1^{2^N\cdot k}x$.
First suppose that $y$ is $v_0$-good. Then we have
$$ y = \partial'(y') $$
so
$$ n(y)-1 = n(y'). $$
and so 
$$ s \ge n(y) \Leftrightarrow s-1 \ge n(y') \Leftrightarrow x' \: \mr{good}. $$
The theorem follows from the fact that $x'$ is good if and only if $x$ is good.

Now suppose that $y$ is $h_1$-good, so we have $y = \partial(y')$ with 
$$ n(y) = n(y'). $$
This would seem to pose a problem for $s = n(y)$; in this case 
$$ s-1 < n(y') $$
and so $x'$ is evil by Theorem~\ref{thm:dichotomyAA0}. We claim  such an $x'$ is detected by an evil class $\td{x}'$ with
$$ \td{x} := \partial'(\td{x}') $$
$h_1$-good, detecting $x$.
Indeed, suppose not.  Then $\td{x}$ is evil.  Since
$$ n(\td{x}') = s-1, $$
we have $n(\td{x}) = s$. In the AKSS for $\Ext_{A(N)}(\FF_2)$, there cannot be a $v_1^{2^N\cdot k}$-extension from filtration $s+\epsilon$ to filtration $s$.
\end{proof}


\subsection*{The topological AKSS}

There is a topological analog of the AKSS, which refines the $\bo$-ASS just as the (algebraic) AKSS constructed in the beginning of this section refines the $\bo$-MSS.

Fix a splitting of the short exact sequence
\begin{equation}\label{eq:topSES} 
0 \rightarrow V^{n,*} \rightarrow \E{bo}{}^{n,*}_1 \xrightarrow{g} \mc{C}^{n,*} \rightarrow 0.
\end{equation}
Just as in the algebraic case, we will regard the evil subcomplex $V^{n,*}$ as being in filtration $n + \epsilon$.
The result is a topological AKSS:
$$ \{ \E{akss}{}^{n+\alpha\epsilon,t}_{r + \beta \epsilon} \} \Rightarrow \pi_{t-n}^s $$
with differentials 
\begin{align*}
d^{akss}_{r-\epsilon}:  & \E{akss}{}_{r - \epsilon}^{n+ \epsilon,t} \rightarrow \E{akss}{}_{r- \epsilon}^{n+r,t}, \\
d^{akss}_{r}:  & \E{akss}{}_{r}^{n+ \alpha\epsilon,t} \rightarrow \E{akss}{}_{r}^{n+r +\alpha\epsilon, t}, \\
d^{akss}_{r+\epsilon}:  & \E{akss}{}_{r+\epsilon}^{n, t} \rightarrow \E{akss}{}_{r+ \epsilon}^{n+r +\epsilon, t}.
\end{align*}
The $E_1$-term takes the form
$$\E{akss}{}^{n+\alpha \epsilon, t}_1 = 
\begin{cases}
\mc{C}^{n,t}, & \alpha = 0, \\
V^{n, t}, & \alpha = 1. 
\end{cases}
$$
The $d_1$-differential 
$$ d^{akss}_{1}: \E{akss}{}^{n+\alpha \epsilon, t}_1 \rightarrow \E{akss}{}^{n+ 1+\alpha\epsilon, t}_1
$$
is given by
$$d_1 = 
\begin{cases}
d_1^{good}, & \alpha = 0, \\
d_1^{evil}, & \alpha = 1. 
\end{cases}
$$
We therefore have
$$\E{akss}{}^{n+\alpha \epsilon, t}_{1+\epsilon} = 
\begin{cases}
H^{n,t}(\mc{C}), & \alpha = 0, \\
H^{n, t}(V), & \alpha = 1.
\end{cases}
$$
The only nonzero $d_{1+\epsilon}$-differentials are of the form
$$ H^{n, t}(\mc{C}) = \E{akss}{}^{n, t}_{1+\epsilon} \xrightarrow{d_{1+\epsilon}} \E{akss}{}^{n+1 + \epsilon, t}_{1+\epsilon} = H^{n+1, t}(V), $$
for which we have 
$$ d_{1+\epsilon} = \partial $$
where $\partial$ is the connecting homomorphism of (\ref{eq:LES}).  It turns out all of these differentials can be derived from the algebraic AKSS.


\begin{lem}
For $n = 0,1$, the differentials
$$ d_{1+\epsilon}: \E{akss}{}^{n, t}_{1+\epsilon} \xrightarrow{d_{1+\epsilon}} \E{akss}{}^{n+1 + \epsilon, t}_{1+\epsilon} $$
are trivial.  For $n \ge 2$, they are determined by the following commutative diagram (see Corollary~\ref{cor:HC}).
$$
\xymatrix{
\E{akss}{}^{n, t}_{1+\epsilon} \ar[d] \ar[r]^-{d_{1+\epsilon}} &
\E{akss}{}^{n+1 + \epsilon, t}_{1+\epsilon} \ar@{=}[d] \\
\E{akss}{alg}^{n,{0}, t}_{1+\epsilon}  \ar[r]_-{d^{alg}_{1+\epsilon}} &
\E{akss}{alg}^{n+1 + \epsilon, {0}, t}_{1+\epsilon}
}
$$
\end{lem}

\begin{proof}
Topologically, this connecting homomorphism derives from applying $\pi_*$ to the composite
\begin{equation}\label{eq:topconnect}
\bigvee_{\abs{I} = n} \Sigma^{4\norm{I}}b_I \hookrightarrow \bo \wedge \br{\bo}^n \rightarrow \br{\bo}^{n+1} \rightarrow \bo \wedge \br{\bo}^{n+1} \rightarrow \bigvee_{\abs{I} = n+1} \Sigma^{4\norm{I}} HV_I.
\end{equation}
The first statement follows from the fact that the only elements in $H^{n,*}(\mc{C})$ for $n = 0, 1$ and $* > 0$ have Adams filtration greater than $0$, and therefore cannot map topologically to elements of Adams filtration $0$.
The second statement follows from the fact that the algebraic connecting homomorphism derives from the ASS edge homomorphism of the composite (\ref{eq:topconnect}):
$$
\mc{C}_{alg}^{n,0,*} = \E{ass}{}^{0,*}\left(\bigvee_{\abs{I} = n} \Sigma^{4\norm{I}}b_I\right) \rightarrow \E{ass}{}^{0,*}\left(\bigvee_{\abs{I} = n+1} \Sigma^{4\norm{I}}HV_I\right) = V^{n+1,*}.
$$
\end{proof}

The $E_2$-term of the $\bo$-ASS is deduced from the short exact sequence
$$ 0 \rightarrow \E{akss}{}^{n+\epsilon,t}_2 \rightarrow \E{bo}{}^{n,t}_2 \rightarrow \E{akss}{}^{n,t}_2 \rightarrow 0.
$$

At this point we can deduce Mahowald's Vanishing Line Theorem.  We only sketch the proof to emphasize the conceptual origin of this vanishing line without getting lost in the details.

\begin{thm}[Mahowald \cite{Mahowaldbo}]\label{thm:vanishing}
There is a $C$ so that 
$$ \E{bo}{}_2^{n,t} = 0 $$
for $n > 1/5(t-s)+C$. 
\end{thm}

\begin{proof}[Sketch of proof]
Since $H^{*,*}(\mc{C})$ has a $1/5$-vanishing line by Corollary~\ref{cor:HCvanishing}, it suffices to establish that the cohomology of the evil complex $H^{*,*}(V)$ has a $1/5$-vanishing line.  Suppose that $x$ is a nontrivial class in $H^{*,*}(V)$.  By the Dichotomy Principle (Theorem~\ref{thm:dichotomy}), there are three possibilities.
\begin{enumerate}
\item $x$ detects a $v_1$-torsion class of $\Ext$ in the algebraic AKSS.
\item $x$ detects a $v_1$-periodic class of $\Ext$, which in the $v_1$-periodic $\bo$-MSS, is detected in $v_1^{-1}\E{bo}{alg}^{n,s,t}_2$ with $s < 0$.
\item $x = \partial_{alg}(y)$, for $y \in H^{*,*}(\mc{C})$.
\end{enumerate}
Recall (Remark~\ref{rmk:onefifth}) that $\Ext$ is entirely $v_1$-periodic above a ``periodicity line'' of slope $1/5$ (and above this periodicity line the $\Ext$-groups are isomorphic to the $v_1^{-1}\Ext$-groups). 
Thus if we are in case (1), we deduce that $x$ must lie below this periodicity line.  
The same inequalities used in the proof of Corollary~\ref{cor:HCvanishing} also prove that there is a $C'$ so that the maps
$$ H^{n,s,t}(\mc{C}_{alg}) \rightarrow H^{n,s,t}(v_1^{-1}\mc{C}_{alg}) $$
are isomorphisms for
\begin{equation}\label{eq:goodperiodicity}
 n+s > 1/5(t-s-n) + C'.
\end{equation} 
Thus in case (2), we deduce that $x$ must lie below this line of slope $1/5$.  In case (3) for $M \gg 0$, the class $v_1^{M}y$ is either non-trivial, or is the source or target of a differential in the algebraic AKSS.  In the former case, $y$ cannot lie above the periodicity line.  In the latter case, Lemma~\ref{lem:v1torsion} implies that $y$ must lie below both the 1/5 line given by (\ref{eq:goodperiodicity}).
\end{proof}

The differentials in the topological AKSS determine and are determined by the differentials in the $\bo$-ASS, with lengths dictated by whether the sources and targets of the $\bo$-ASS differentials are good or evil.  Unlike the algebraic case, in the topological case there are no dimensional restrictions: in principle good or evil classes can each kill either good or evil classes.  

\begin{rmk}
There is no Dichotomy Principle in the topological AKSS.  Many $v_1$-torsion elements of $\pi_*^s$ are detected by good classes in the $\bo$-ASS (e.g. $\nu^2, \kappa, \theta_{3}, \bar{\kappa}, \ldots$).  In fact, Mahowald showed in \cite{Mahowaldbo} that a non-trivial class in $\pi_*^s$ is $v_1$-periodic if and only if it has $\bo$-filtration $\le 1$.
\end{rmk}

