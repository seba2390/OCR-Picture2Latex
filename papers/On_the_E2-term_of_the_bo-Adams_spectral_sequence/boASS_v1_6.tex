% !TEX root = boASS6.tex

\section{$v_1$-periodicity}\label{sec:v1}

In Section~\ref{sec:alg}, we established that there is a long exact sequence
\begin{equation}
 \cdots \rightarrow H^{n,k}(V) \rightarrow \E{bo}{alg}_2^{n,0,k} \xrightarrow{g_{alg}} H^{n,0,k}(\mc{C}_{alg}) \xrightarrow{\partial_{alg}} H^{n+1,k}(V) \rightarrow \cdots
\end{equation}
which assembles $\E{bo}{alg}_2$ out of good and evil classes. 
Furthermore, we computed the good component $H^{*,*,*}(\mc{C})$ in its entirety (Theorem~\ref{thm:HCalg}).  Since the spectral sequence $\{\E{bo}{alg}_r\}$ converges to $\Ext_{A_*}(\FF_2)$, it stands to reason that if we know $\Ext_{A_*}(\FF_2)$ through a range, we should be able to deduce the evil component $H^{*,*}(V)$.  As summarized in the introduction, this is predicated on actually having a means of determining which elements of $\Ext$ are detected by good and evil classes.  The Dichotomy Principle (Theorem~\ref{thm:dichotomy}) will relate this to $v_1$-periodicity.  We review the definition and properties of $v_1$-periodic Ext groups in this section.


\begin{comment}
We therefore endeavor to understand the map $g_{alg}$ in this section.  The stategy will be to consider the $v_1$-periodic $\bo$-Mahowald spectral sequence:
$$ v_1^{-1}\E{bo}{alg}_2^{n,s,t} \Rightarrow v_1^{-1}\Ext^{n+s, t}_{A_*}(\FF_2). $$
The significance of this is encoded in the commutative diagram
$$ 
\xymatrix@C+1em{
\E{bo}{alg}_2^{n,0,k} \ar[r]^{g_{alg}} \ar[d] & 
H^{n,0,k}(\mc{C}_{alg}) \ar@{^{(}->}[d] \\
v_1^{-1} \! \E{bo}{alg}_2^{n,0,k} \ar[r]^{v_1^{-1}g_{alg}}_\cong & 
 H^{n,0,k}(v_1^{-1}\mc{C}_{alg})
}
$$ 
We will prove (1) the right homomorphism is an injection (Lemma~\ref{lem:v1HCalg}), and (2) the bottom homomorphism is an isomorphism (Lemma~\ref{lem:v1E2boalg}), as depicted.  This is sufficient to determine $g_{alg}$.
\end{comment}



\subsection*{$v_1$-periodic Ext groups}

Our approach to $v_1$-periodic Ext follows that of  \cite{MahowaldShick}, \cite{DavisMahowaldv1}, which is based on the Adams periodicity theorem \cite{AdamsP}.  Let $M$ be an $A(n)_*$-comodule.  For $n \ge 2$, Adams produced elements
$$ v_1^{2^n} \in \Ext^{2^n, 3\cdot 2^n}_{A(n)_*}(\FF_2) $$
which are appropriately compatible under the maps
$$ \Ext^{*,*}_{A(n+1)_*}(\FF_2) \rightarrow \Ext^{*,*}_{A(n)_*}(\FF_2). $$
For any $A(n)_*$-comodule $M$, we can form the localized Ext groups
$$ v_1^{-1}\Ext^{s,t}_{A(n)_*}(M). $$
We then define
\begin{equation}\label{eq:v1Ext} v_1^{-1}\Ext^{s,t}_{A_*}(M) := \varprojlim_n v_1^{-1}\Ext^{s,t}_{A(n)_*}(M).
\end{equation}
For certain $M$, the inverse system in (\ref{eq:v1Ext}) stabilizes for large $n$ for fixed bidegrees $(s,t)$. When this is the case, the localized Ext groups are manageable.


One class of comodules for which this is true are those whose duals are $A(0)$-free.  Indeed, as pointed out by \cite{DavisMahowaldv1}, Adams proves:

\begin{thm}[Adams \cite{AdamsP}]\label{thm:AdamsP}
Suppose that $M$ is a connective $A_*$-comodule whose dual is free over $A(0)$, and suppose $n \ge 2$. 
\begin{enumerate}
\item For $t-s < 3s$, the map
$$ v_1^{2^n}: \Ext^{s,t}_{A(n)_*}(M) \rightarrow \Ext^{s+2^n, t+3\cdot 2^n}_{A(n)_*}(M) $$
is an isomorphism.

\item For $s > 0$ and $t-s < 2s + 2^{n+1} - 5$, the map
$$ \Ext_{A_*}^{s,t}(M) \rightarrow \Ext_{A(n)_*}^{s,t}(M) $$
is an isomorphism.
\end{enumerate}
\end{thm}

In other words, $\Ext_{A(n)}$ is $v_1^{2^n}$-periodic above the line of slope $1/3$ through the origin. Further, there is a sequence of lines of slope $1/2$ above which $\Ext_{A_*}$ agrees with $\Ext_{A(n)_*}$. We deduce that above the line of slope $1/3$, every element of $\Ext_{A_*}$ is $v_1^{2^n}$-periodic for some $n$ which depends on which ``band'' of slope $1/2$ it lies in. The typical picture one draws of this (and indeed a similar picture appears in \cite{DavisMahowaldv1}) is:


\begin{center}
\includegraphics[width=0.7\linewidth]{v1Ext}
\end{center}
In the figure above, the Ext groups in the region $R_1$ are all trivial.  In the region $R_n$, for $n \ge 2$ the Ext groups are $v_1^{2^n}$-periodic, and isomorphic to the corresponding $\Ext_{A(n)_*}$-groups.  The corresponding $v_1$-periodic Ext groups take the form:
\begin{center}
\includegraphics[width=0.7\linewidth]{v1Ext2}
\end{center}
Here the periodic Ext groups in the region $R_n'$ are obtained by interpolating back the values of the Ext groups of the region $R_n$. In particular, the values of $v_1^{-1}\Ext_{A(m)_*}$ stabilize in the region $R_n \cup R_n'$ for $m \ge n$.


The comodule $\FF_2$ is not free over $A(0)$, but $\br{A\mmod A(0)}_*$ is. The following remark establishes an explicit relationship between their Ext groups.
\begin{rmk}\label{rmk:AmodA0SES}
The short exact sequence
\begin{equation}\label{eq:AmodA0SES}
 0 \rightarrow \FF_2 \xrightarrow{i} A\mmod A(0)_* \xrightarrow{p} \AA0 \rightarrow 0
\end{equation}
induces a long exact sequence of Ext groups.  Since by change of rings we have
\[ \Ext^{*,*}_{A_*}(A \mmod A(0)_*) \cong \Ext^{*,*}_{A(0)_*}(\FF_2) = \FF_2[v_0], \]
the connecting homomorphism
\[ \Ext^{s-1,t}_{A_*}(\AA0) \xrightarrow{\bar{\partial}} \Ext_{A_*}^{s,t}(\FF_2) \]
is an isomorphism for $t \ne s$.  

The short exact sequence (\ref{eq:AmodA0SES}) is the homology of the cofiber sequence
$$ S \rightarrow H\ZZ \rightarrow \br{H\ZZ}. $$ 
\end{rmk}




\begin{prop}\label{prop:v1ext}
$\quad$
\begin{enumerate} 
\item For $ 0 < t -s < 2s+2^{n+1}-6 $,
there are well defined homomorphisms (compatible as $n$ varies)
\[
v_1^{2^n}: \Ext^{s,t}_{A_*}(\FF_2) \rightarrow \Ext^{s+2^n,t+3\cdot 2^n}_{A_*}(\FF_2). 
\]
For $t -s < 3s-4$
these homomorphisms are isomorphisms.

\item The maps
$$ \Ext^{s,t}_{A_*}(\FF_2) \rightarrow \Ext^{s,t}_{A(n)_*}(\FF_2) $$
are isomorphisms for $t-s < -s+2^{n+1}$.
\end{enumerate}
\end{prop}  

\begin{proof}
Statement (1) follows from the isomorphism of Remark~\ref{rmk:AmodA0SES} by applying Theorem~\ref{thm:AdamsP} to the comodule $\br{A\mmod A(0)}_*$, which is free over $A(0)$. Statement (2) follows from the fact that in degrees less than $2^{n+1}$, the map
$$ A_* \rightarrow A(n)_* $$
is an isomorphism.
\end{proof}

\begin{rmk}\label{rmk:onefifth}
The line $t -s < 3s-2$ of Proposition~\ref{prop:v1ext} can be upgraded to a line of slope $1/5$ (see \cite[Thm~3.4.6]{Ravenel}).
\end{rmk}

Proposition~\ref{prop:v1ext} allows to define $v_1^{-1}\Ext_{A_*}(\FF_2)$ by interpolating $v_1$-periodic families backwards across the slope $1/3$ line. 
\footnote{There is one exception: the $v_0$-tower in $t-s = 0$ is $v_1$-periodic, with ``infinite period''.}

\begin{cor}\label{cor:stabilization}
For fixed $s,t$, the map
\[ v_1^{-1}\Ext^{s,t}_{A_*}(\FF_2) \rightarrow \Ext^{s,t}_{A(n)_*}(\FF_2) \]
is an isomorphism for $n \gg 0$.
\end{cor}

\subsection*{The $v_1$-periodic $\bo$-MSS}

Davis and Mahowald \cite{DavisMahowaldv1} considered a $v_1$-periodic version of the $\bo$-MSS:
\[ v_1^{-1} \! \E{bo}{alg}_1^{n,s,t}(M) = v_1^{-1} \Ext^{s,t}_{A(1)_*}(\br{A\mmod A(1)}^{n}_* \otimes M) \Rightarrow v_1^{-1}\Ext^{s+n,t}_{A_*}(M). \]
They use the stabilization of $v_1^{-1}\Ext_{A(n)_*}$ in fixed bidegrees to prove the convergence of this spectral sequence for comodules $M$ which are free over $A(0)$.  Corollary~\ref{cor:stabilization} implies that the Davis-Mahowald convergence argument also applies to the case where $M = \FF_2$.

\begin{lem}\label{lem:v1E2boalg}
The maps
\[
v_1^{-1} \! \E{bo}{alg}^{n,s,t}_2 \xrightarrow{v_1^{-1}g_{alg}} H^{n,s,t}(v_1^{-1} \mc{C}_{alg})
\]
are isomorphisms.
\end{lem}

\begin{proof}
The evil subcomplex
\[ V^{n,*} \subseteq \Ext^{0,*}_{A(1)_*}(\br{A\mmod A(1)}^{n}_*) \]
is $v_1$-torsion, so we have an isomorphism
\[ v_1^{-1} \! \E{bo}{alg}_1^{n,s,t} = v_1^{-1} \Ext^{s,t}_{A(1)_*}(\br{A\mmod A(1)}^{n}_*) \xrightarrow{\cong} v_1^{-1}\mc{C}^{n,s,t}_{alg} \]
and hence an isomorphism
\[ v_1^{-1} \! \E{bo}{alg}^{n,s,t}_2 \xrightarrow{\cong}  H^{n,s,t}(v_1^{-1}\mc{C}_{alg}). \qedhere \]
\end{proof}

\begin{comment}
\begin{lem}\label{lem:v1HCalg}
The maps
$$ H^{n,s,t}(\mc{C}_{alg}) \rightarrow H^{n,s,t}(v_1^{-1}\mc{C}_{alg}) $$
are injections.
\end{lem}

\begin{proof}
As the patterns $\bo_i[j]^{\bra{k}}$ of Section~\ref{sec:alg} that make up $H^*(\mc{C}_{alg})$ are all $v_1$-periodic, the maps 
$$ H^{n,s,t}(\mc{C}_{alg}) \rightarrow v_1^{-1} H^{n,s,t}(\mc{C}_{alg}) $$
are all injective.  There is one exceptional case to consider: $n = 0$ and $s = t$.  In this case $H^{0,s,s}(\mc{C}_{alg})$ is spanned by $v_0^s$.  However, this $v_0$-tower is easily checked to also be in $H^{0,s,s}(v_1^{-1}\mc{C}_{alg})$.
\end{proof} 
\end{comment}
