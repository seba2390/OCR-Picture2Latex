% !TEX root = boASS6.tex

\section{Comparison with $\br{A\mmod A(0)}$}\label{sec:AA0}

We would like to leverage the results of Section~\ref{sec:v1} to relate $v_1$-periodicity in $\Ext_{A_*}(\FF_2)$ to being ``good'' in a precise manner.  However, as we saw in Section~\ref{sec:v1}, the groups 
$$ \Ext^{s,t}_{A_*}(\AA0) $$
have much better behaved $v_1$-periodic phenomena. For instance, there is a sequence of lines $L_n$ of slope 1/2 above which the Ext groups are entirely $v^{2^n}_1$-periodic and simultaneously isomorphic to the corresponding $\Ext_{A(n)_*}$ groups, where $v_1^{2^n}$ is well defined.  




While $\AA0$ and $\FF_2$ have essentially the same Ext groups (see Remark~\ref{rmk:AmodA0SES}), their respective $\bo$-MSS's differ.  In this section, we give a complete dictionary between these two $\bo$-MSS's.  This will allow us to transport results on $v_1$-periodicity from $\AA0$ to $\FF_2$.  There is an added bonus: since the two $\bo$-MSS's are different, we will be able to deduce hidden extensions or differentials in one from non-hidden extensions or differentials \emph{of different length} in the other.

Given any $A_*$-comodule $M$, we can consider the associated $\bo$-MSS
$$ \E{bo}{alg}_1^{n,s,t}(M) = \Ext^{s,t}_{A(1)_*}(\br{A\mmod A(1)}^{n}_* \otimes M) \Rightarrow \Ext^{s+n,t}_{A_*}(M). $$
Suppose that the map 
$$ \E{bo}{alg}_1^{n,s,t}(M) \rightarrow v_1^{-1} \E{bo}{alg}_1^{n,s,t}(M) $$
is injective for $s > 0$.  Then we will define $V^{n,t}(M)$ to be the
kernel of the above map for $s = 0$.  Just as in the case of $M = \FF_2$, the evil subgroup $V^{*,*}(M)$ is a subcomplex with respect to $d_1^{bo,alg}$, and we define the good complex 
$$ (\mc{C}_{alg}^{*,*,*}(M), d_1^{good, alg}) $$ 
to be the quotient complex.  Then there is a long exact sequence
\begin{multline}\label{eq:LESM}
 \cdots \rightarrow H^{n,k}(V(M)) \rightarrow \E{bo}{alg}_2^{n,0,k}(M) \xrightarrow{g^M_{alg}} H^{n,0,k}(\mc{C}_{alg}(M)) \\ \xrightarrow{\partial^M_{alg}} H^{n+1,k}(V(M)) \rightarrow \cdots
\end{multline}

The following lemma is an algebraic analog of the ``generalized connecting homomorphism theorem'' \cite[Thm~2.3.4]{Ravenel}, and its proof is identical to the topological case.

\begin{lem}[Connecting Homomorphism Lemma]\label{lem:MSSSES}
Associated to the short exact sequence
$$ 0 \rightarrow \FF_2 \xrightarrow{i} {A \mmod A(0)_*} \xrightarrow{p} \AA0 \rightarrow 0 $$
we have a sequence of spectral sequences
$$ 
\xymatrix@C-1.5em{
\E{bo}{alg}_1^{n,s,t}(\FF_2) \ar[r]^-{i_*} \ar@{=>}[d] &
\E{bo}{alg}^{n,s,t}_1({A\mmod A(0)_*}) \ar[r]^-{p_*} \ar@{=>}[d] &
\E{bo}{alg}^{n,s,t}_1(\br{A\mmod A(0)}_*) \ar[r]^-{\partial} \ar@{=>}[d] &
\E{bo}{alg}^{n,s+1,t}_1(\FF_2)  \ar@{=>}[d]
\\
\Ext^{n+s, t}_{A_*}(\FF_2) \ar[r]_-{i_*} &
\Ext^{n+s, t}_{A_*}(A\mmod A(0)_*) \ar[r]_-{p_*} &
\Ext^{n+s, t}_{A_*}(\AA0) \ar[r]_-{\bar{\partial}} &
\Ext^{n+s+1, t}_{A_*}(\FF_2) 
}
$$
The top and bottom rows are long exact sequences.  The map $\partial$ induces a map of spectral sequences, converging to $\bar{\partial}$.
\end{lem}

\subsection*{The $\pmb\bo$-MSS for $\pmb{A \mmod A(0)}$}

We now study the spectral sequence
$$ \E{bo}{alg}^{*,*,*}_1(A\mmod A(0)_*) \Rightarrow \Ext^{*,*}_{A_*}(A \mmod A(0)_*) \cong \FF_2[v_0]. $$
The next lemma computes the ``good'' part of the $E_1$-term.

\begin{lem}\label{lem:AmodA0good}
We have
\[\mc{C}_{alg}^{n,*,*}(A \mmod A(0)_*) \cong \FF_2[v_0, w] \otimes \FF_2[\zeta_1^{4}]^{\otimes n}  \]
where $w = v_1^2/v_0^2$.
\end{lem}

\begin{proof}
By change of rings, we have
\begin{align*}
 \E{bo}{alg}_1^{n,*,*}(A \mmod A(0)_*) & \cong 
 \Ext^{*,*}_{A(1)_*} ((A \mmod A(1)_*)^{\otimes n} \otimes A\mmod A(0)_*) \\
 & \cong \Ext^{*,*}_{A_*} (A \mmod A(1)_* \otimes (A \mmod A(1)_*)^{\otimes n} \otimes A\mmod A(0)_*) \\
 & \cong \Ext^{*,*}_{A(0)_*} (A \mmod A(1)_* \otimes (A \mmod A(1)_*)^{\otimes n})
 \end{align*}
 The latter is computed using Margolis homology. Using the fact that in 
 \[ A\mmod A(1)_* \cong \FF_2[\zeta_1^{4}, \zeta_2^{2}, \zeta_3, \ldots  ] \]
 we have
 \[ Q_0^* \zeta_i = \zeta^2_{i-1} \] 
 we deduce that the Margolis homology is given by
\[ H_*((A \mmod A(1)_*)^{\otimes n+1}; Q_0^*) = \FF_2[\zeta_1^{4}]^{\otimes n+1}. \]
Identifying the first factor of $\FF_2[\zeta^4_1]$ with $\FF_2[w]$,
 we have
\[ \E{bo}{alg}_1^{n, *, *} = \FF_2[v_0, w] \otimes \FF_2[\zeta_1^{4}]^{\otimes n} \oplus V^{n,*}(A \mmod A(0)_*). \qedhere \]
\end{proof}

Just as in Section~\ref{sec:alg}, we will use the notation 
\[ v_0^i w^m (i_1, \ldots, i_n) := v_0^i w^m \zeta_1^{4i_1}\otimes  \cdots \otimes \zeta_1^{4i_n} \]
to denote a generic element of $\E{bo}{alg}_1(A \mmod A(0)_*)$.
From the map of good complexes
\[ i_* : \mc{C}_{alg}^{*,*,*}(\FF_2) \rightarrow \mc{C}_{alg}^{*,*,*}(A \mmod A(0)_*) \]
(and the fact that $d_1^{good, alg}$ is $v_0$-linear), we find the formula for $d_1^{good, alg}$ is identical to that of Proposition~\ref{prop:d1alg}.
Let
\[ \E{wss}{alg}_0^{n, s, t, w}(A \mmod A(0)_*) \Rightarrow H^{n, s, t}(\mc{C}_{alg}(A \mmod A(0)_*)) \]
denote the associated algebraic weight spectral sequence.  Just as in Proposition~\ref{prop:E1wssalg}, we have 
\[ \E{wss}{alg}^{*,*,*,*}_1(A \mmod A(0)_*) = \FF_2[v_0, w, h_2, h_3, \cdots ]. \]
Just as in Lemma~\ref{lem:drwssalg}, the non-trivial differentials in the algebraic weight spectral sequence are given by
\begin{align*}
d_{2^{r-2}}^{wss, alg}(w^{2^{r-2} a}h_{r'}^{k_{r'}}  \cdots h_l^{k_l}) & = w^{2^{r-2}(a-1)} h_r h_{r'}^{k_{r'}}  \cdots h_l^{k_l}, \: k_{r'} > 0, \: r \le r', \: a \: \mr{odd}. 
\end{align*}
We deduce
\[ H^{*,*,*}(\mc{C}_{alg}(A \mmod A(0)_*)) = \FF_2[v_0]. \]

\begin{prop}\label{prop:boMSSAmodA0collapse}
The $\bo$-MSS for $A \mmod A(0)_*$ collapses at $E_2$.
\end{prop}

\begin{proof}
With the exception of $\FF_2[v_0]$, all of the good classes are targets and sources of $d_1$-differentials.  Since
$d_r^{bo, alg}$ changes $s$-degree for $r > 1$, and the evil classes are concentrated in $s = 0$, we deduce that
\[ H^{*,*}(V(A \mmod A(0)_*)) = 0. \]
The result follows.
\end{proof}

\subsection*{The structure of $\pmb{\E{bo}{alg}_1(\AA0)}$}

In this subsection, we will describe the $E_1$-term $\E{bo}{alg}_1(\AA0)$ in terms of $\E{bo}{alg}_1(\FF_2)$.  We shall find that while these $E_1$-terms are different, their relationship admits a complete description.

The primary tool in this analysis is the long exact sequence
\begin{multline}\label{eq:boMSSLESE1}
\cdots \xrightarrow{} \E{bo}{alg}^{n, s, t}_1(\FF_2) 
\xrightarrow{i} \E{bo}{alg}^{n, s, t}_1(A \mmod A(0)_*) \xrightarrow{p}
\E{bo}{alg}^{n, s, t}_1(\AA0) \\
\xrightarrow{\partial} \E{bo}{alg}^{n, s+1, t}_1(\FF_2) \xrightarrow{} \cdots 
\end{multline}
These long exact sequences decompose into a direct sum of long exact sequences
\begin{multline*}
\cdots \rightarrow \Ext^{s,t}_{A(1)_*}(\ull{B_I}) \xrightarrow{i} \Ext^{s,t}_{A(1)_*}(\ull{B_I} \otimes A \mmod A(0)_*) \xrightarrow{p} \Ext^{s,t}_{A(1)_*}(\ull{B_I} \otimes \AA0) \\ \xrightarrow{\partial} \Ext^{s+1, t}_{A(1)_*}(\ull{B_I}) \rightarrow \cdots
\end{multline*}
using the decomposition
\[ \E{bo}{alg}^{n, s, t}_1(M) = \bigoplus_{\abs{I} = n}  \Ext^{s, t}_{A(1)_*}(\Sigma^{4\norm{I}} \ull{B_I} \otimes M) \]
induced by the splitting (\ref{eq:algsplitting}).

We first examine the behavior of the good classes.  A useful schematic is depicted below.
\begin{center}
\includegraphics[width=\linewidth]{E1AmodA0}
\end{center}
In the figure above, there are two kinds of good classes in $\Ext_{A(1)_*}(\ull{B_I})$: those which are parts of $v_0$-towers, and those which are $v_0$-torsion.  We shall refer to the former as \emph{$v_0$-good} (marked with solid dots in the above figure) and the latter as \emph{$h_1$-good} (so that there is a basis of $\E{bo}{alg}_1(\FF_2)$ consisting of $v_0$-good, $h_1$-good, and evil classes).  The good part of $\E{bo}{alg}_1(A \mmod A(0)_*)$ was computed in the last subsection, as depicted in the middle chart above.  The righthand chart is then deduced by the long exact sequence.  We see that in $\E{bo}{alg}_1(\br{A\mmod A(0)}_*)$, the $v_0$-towers get turned upside down (we will call these classes $v_0$-good as well), while the $h_1$-good classes get transported by the boundary homomorphism  $\partial$ (which we also call $h_1$-good).

The extension 
\begin{equation}\label{eq:extension} 
h_1 y_{2} = p \left( \frac{i(v_0^3wx)}{v_0} \right)
\end{equation}
in the figure above is of special importance.  One way of deducing it is to observe that associated to the cofiber sequence
\begin{equation}\label{eq:kocofiber}
\bo \rightarrow \bo \wedge H\ZZ \rightarrow \bo \wedge \br{H\ZZ}
\end{equation}
there is a sequence of $v_1$-localized Adams spectral sequences
\[\xymatrix@C-1em{
v_1^{-1}\Ext^{s,t}_{A(1)}(\ull{B_I}) \ar[r]^-{i} \ar@{=>}[d] & 
v_1^{-1}\Ext^{s,t}_{A(1)}(\ull{B_I} \otimes A \mmod A(0)_*) \ar[r]^-p \ar@{=>}[d] &
v_1^{-1}\Ext^{s,t}_{A(1)}(\ull{B_I} \otimes \AA0) \ar@{=>}[d]
\\
\pi_{t-s} (\KO \wedge B_I) \ar[r] & 
\pi_{t-s} ((\KO \wedge B_I)_\QQ) \ar[r] &  
\pi_{t-s} (\Sigma M_1(\KO \wedge B_I)) 
}
\]
(where $M_1(-)$ is the first monochromatic layer). The localized spectral sequences converge as indicated because of the following lemma.

\begin{lem}
Inverting the Bott element in the cofiber sequence (\ref{eq:kocofiber}) yields the cofiber sequence
\[ \KO \rightarrow \KO_\QQ \rightarrow \Sigma M_1\KO \]
\end{lem}

\begin{proof}
The computations earlier in this section specialize to show that the Adams spectral sequence
$$ \Ext_{A(1)_*}(A\mmod A(0)_*) \Rightarrow \bo_*H\ZZ $$
collapses to give
$$ \bo_*H\ZZ \cong \ZZ[w] \oplus \text{$v_1$-torsion} $$
with $w = v_1^2/4$.  Inverting the Bott element $v_1^4$, we get
$$ \KO_* H\ZZ = \QQ[v_1^{\pm 2}]. $$
The map
$$ \KO_* \rightarrow \KO_*H\ZZ $$
is easily computed on the level of the $v_1$-localized Adams $E_2$-terms, and is seen to be a rational isomorphism.  The result follows.
\end{proof}

The extension (\ref{eq:extension}) follows from:
\begin{itemize}
\item $\KO \wedge B_I$ is a suspension of $\KO$,
\item the Brown-Comenetz dual $IM_1(\KO)$ is the Gross-Hopkins dual $I_1\KO$,
\item the Gross-Hopkins dual of $\KO$ is well known to be a suspension of $\KO$ (see, for example, \cite[Cor.9.1]{HeardStojanoska}). 
\end{itemize}



With the language introduced in the discussion above, we have (for $M = \FF_2$, $A \mmod A(0)_*$, or $\AA0$) decompositions
\[ \E{bo}{alg}^{n,*,*}_1(M) \cong \mc{C}^{n,*,*}_{v_0}(M) \oplus \mc{C}^{n, *, *}_{h_1}(M) \oplus V^{n,*}(M) \]
into $v_0$-good, $h_1$-good, and evil components.  Note that for each of these $M$ we have
\begin{align*}
\mc{C}^{n,s,t}_{v_0}(M) & = 
\begin{cases}
\mc{C}^{n,s,t}_{alg}(M), & t-s \equiv 0 \mod 4, \\
0, & \mr{otherwise},
\end{cases}
\\
\mc{C}^{n,s,t}_{h_1}(M) & = 
\begin{cases}
\mc{C}^{n,s,t}_{alg}(M), & t-s \not\equiv 0 \mod 4, \\
0, & \mr{otherwise},
\end{cases}
\end{align*}
(and for $M = A \mmod A(0)_*$ we have $\mc{C}^{*,*,*}_{h_1}(M) = 0$).
The complete structure of these components is summarized in the following proposition.

\begin{prop}\label{prop:MSSSESE1}
There are short exact sequences
\begin{gather*}
 0 \rightarrow V^{n,*}(\FF_2) \oplus \mc{C}^{n,0,*}_{h_1}(\FF_2) \xrightarrow{i} V^{n,*}(A \mmod A(0)_*) \xrightarrow{p} V^{n,*}(\AA0) \rightarrow 0 \\
 0 \rightarrow \mc{C}^{n,*, *}_{v_0}(\FF_2) \xrightarrow{i} \mc{C}^{n, *,*}_{v_0}(A \mmod A(0)_*) \xrightarrow{p} \mc{C}^{n,*,*}_{v_0}(\AA0) \rightarrow 0
\end{gather*}
and isomorphisms
\[ \mc{C}_{h_1}^{n, s, *}(\AA0) \xrightarrow[\cong]{\partial} \mc{C}^{n, s+1, *}_{h_1}(\FF_2). \]
Thus, for every $v_0$-good tower
\[ \{ v_0^jx \}_{j \ge 0} \subset \mc{C}^{n, *, *}_{v_0}(\FF_2) \]
generated by $x \in \mc{C}^{n, s, t}_{v_0}$
there is a corresponding truncated $v_0$-good tower
\[ \left\{ p\left(\frac{i(x)}{v_0^j}\right) \right\}_{1 \le j \le s} \subset \mc{C}^{n, *, *}_{v_0}(\br{A\mmod A(0)}_*). \]
Furthermore, in $\mc{C}_{alg}^{n,*,*}(\AA0)$ we have
\[ h_1 \partial^{-1}(h_1^2 x) = p\left( \frac{i(v_0^3wx)}{v_0} \right). \]
\end{prop}

\begin{proof}
This proposition mostly follows from the preceding discussion, using the long exact sequence (\ref{eq:boMSSLESE1}) and Lemma~\ref{lem:AmodA0good}.
In particular, the second short exact sequence follows from the fact that the map $i$ induces an inclusion
$$ i : \mc{C}^{n,*,*}_{v_0}(\FF_2) \hookrightarrow \mc{C}^{n, *,*}_{v_0}
(A \mmod A(0)_*). $$
The map $i$ vanishes on $\mc{C}^{n,s,t}_{h_1}(\FF_2)$ for $s > 0$ for dimensional reasons, and thus the latter must be in the image of $\partial$.  However, $i$ can map classes in $\mc{C}^{n, 0, t}_{h_1}(\FF_2)$ to evil classes in $V^{n,t}(A \mmod A(0)_*)$, and in fact must do so in an injective fashion, because there are no non-trivial classes in $\E{bo}{alg}^{n, -1, t}(\AA0)$ for $\partial$ to map into $\ker i$.  The $h_1$-relation follows from the Gross-Hopkins duality argument discussed above.
\end{proof}

To state the structure of $\mc{C}_{alg}^{*,*,*}(\AA0)$ more clearly, we remark that the Ext groups
\begin{align*}
\Ext^{s,t}_{A(1)_*}(\AA0) & = \E{ass}{}_2^{s,t}(\bo \wedge \br{H\ZZ}) \\
\Ext^{s,t}_{A(1)_*}(\ull{B_1} \otimes \AA0) & = \E{ass}{}_2^{s,t}(\bsp \wedge \br{H\ZZ})
\end{align*}
take the following form ($v_1$-torsion classes ommitted):
\begin{center}
\includegraphics[width=1\linewidth]{boAmodA0}
\end{center}

\begin{cor}
There are decompositions
\[ \mc{C}^{n,s,t}_{alg}(\AA0) \cong \bigoplus_{\abs{I} = n} \E{ass}{}^{s,t}_2(\Sigma^{4\norm{I}}b_I \wedge \br{H\ZZ}) \]
where
\[ \E{ass}{}^{s,t}_2(b_I \wedge \br{H\ZZ}) = 
\begin{cases}
\E{ass}{}_2^{s,t}((\bo \wedge \br{H\ZZ})^{\bra{2\norm{I}-\alpha(I)}}), & \norm{I} \: \mr{even}, \\
\E{ass}{}_2^{s,t}((\bsp \wedge \br{H\ZZ})^{\bra{2\norm{I}-\alpha(I)-1}}), & \norm{I} \: \mr{odd}.
\end{cases} \]
\end{cor}

\subsection*{The structure of $\pmb{\E{bo}{alg}_2(\AA0)}$}  We now turn to understanding the $E_2$-page of the $\bo$-MSS for $\br{A\mmod A(0)}_*$.  We shall let
\[ w^m(i_1, \ldots, i_n) \in \E{bo}{alg}_1^{n, 0, 4\norm{I}+4m}(\AA0) \]
denote the image under $p$ of the element of the same name in $\E{bo}{alg}_1(A \mmod A(0)_*)$.  Since $p$ induces a map of spectral sequences, we get the same formula for $d_1^{good, alg}$ on the classes above as in $\E{bo}{alg}_1(A\mmod A(0)_*)$ (the formula from \ref{prop:d1alg}).
The formula for $d_1^{good,alg}$ on all of $\mc{C}_{alg}^{*,*,*}(\br{A\mmod A(0)}_*)$ follows from the fact that it is $v_0$ and $h_1$-linear.  We use a weight spectral sequence:
\[ \E{wss}{alg}^{n, s, t, w}(\AA0) \Rightarrow H^{n, s, t}(\mc{C}_{alg}(\br{A \mmod A(0)_*})). \]
The computation of $\E{wss}{alg}_1(\AA0)$ is just as in Proposition~\ref{prop:E1wssalg}:

\begin{prop}\label{prop:E1wssAmodA0}
	An additive basis for $\E{wss}{alg}_1(\br{A \mmod A(0)_*})$ is given by elements
	$$ x h_2^{k_2} h_3^{k_3}h_4^{k_4} \ldots $$
	indexed by $K = (k_2, k_3, \ldots)$, detected by $x(I[K])$ where 
	$$ I[K] = (\underbrace{1, \ldots ,1}_{k_2}, \underbrace{2, \ldots ,2}_{k_3}, \underbrace{4, \ldots ,4}_{k_4}, \ldots). $$
	Here, the elements $x$ run through a basis of $\E{ass}{}_2(b_{I[K]} \wedge \br{H\ZZ})$.
\end{prop}

The remaining differentials in the weight spectral sequence are induced by the map $p$ from the weight spectral sequence for $A \mmod A(0)_*$ (the $h_1$-good classes in $\E{wss}{alg}_1(\AA0)$ are all permanent cycles):
\begin{multline}\label{eq:drwssAmodA0}
d_{2^{r-2}}^{wss, alg}(w^{2^{r-2} a}h_{r'}^{k_{r'}}  \cdots h_l^{k_l}) = w^{2^{r-2}(a-1)} h_r h_{r'}^{k_{r'}}  \cdots h_l^{k_l}, \\
k_{r'} > 0, \: r \le r', \: a \: \mr{odd}. 
\end{multline}
The figure below illustrates the relationship between the weight spectral sequences (and consequently the corresponding $\E{bo}{alg}_2$-terms) for $\FF_2$, $A \mmod A(0)_*$, and $\AA0$.


\begin{figure}[h]
\centering
\includegraphics[width=1\linewidth]{E2AmodA0}
\caption{}
\label{fig:E2AmodA0}
\end{figure}


We see that there is a bijective correspondence between the truncated $v_0$-towers that comprise the $v_0$-good classes in $\E{bo}{alg}_2(\FF_2)$ and $\E{bo}{alg}_2(\AA0)$, with the notable feature that their respective $\bo$-filtrations differ by $1$ while their $s$-degrees are identical.  Similarly, there is a bijective correspondence between the $h_1$-good classes in each of these $E_2$-terms, but with identical $\bo$-filtrations and $s$-degrees which differ by $1$ (\emph{except for $h_1$-good classes in $\E{bo}{alg}_2^{*,0,*}(\FF_2)$, for which there is no corresponding class in $\E{bo}{alg}_2(\br{A\mmod A(0)}_*)$}).    In the notation of Figure~\ref{fig:E2AmodA0} above, we have
$$ h_1 \cdot x' = \partial y_1' \quad \mr{in} \: \E{bo}{alg}_2(\FF_2) $$
while 
$$ h_1 \cdot z' = 0 \quad \mr{in} \: \E{bo}{alg}_2(\br{A \mmod A(0)_*}) $$
(with $y'$ lying in higher $\bo$ filtration), and also we have
$$ h_1 \cdot y_2 = p\left( \frac{i(v_0^3wx)}{v_0} \right) \quad \mr{in} \: \E{bo}{alg}_2(\br{A \mmod A(0)}) $$
while 
$$ h_1 \cdot \partial y_2 = 0 \quad \mr{in} \: \E{bo}{alg}_2(\FF_2) $$
(with 
$v_0^2 h_2 x$ lying in higher $\bo$ filtration).  The map $\partial'(-)$ used in Figure~\ref{fig:E2AmodA0} is one of the two connecting homomorphisms associated to long exact sequences associated to the short exact sequences of Proposition~\ref{prop:MSSSESE1}:
\begin{align*}
\partial': & H^{n,s,t}(\mc{C}_{v_0}(\AA0)) \rightarrow H^{n+1,k}(\mc{C}_{v_0}(\FF_2)), \\
\partial': & H^{n,t}(V(\AA0)) \rightarrow H^{n+1,t}(V(\FF_2) \oplus \mc{C}^{*,0,*}_{h_1}(\FF_2)). 
\end{align*}
A complete description of $\E{bo}{alg}_2(\br{A\mmod A(0)}_*)$ in terms of $\E{bo}{alg}_2(\FF_2)$ is provided by the following proposition.

\begin{comment}
WHAT FOLLOWS IS A HANDS ON DESCRIPTION OF PARTIAL' WHICH I DECIDED WAS TOO TECHNICAL TO BE UNDERSTANDABLE.
\begin{defn}
Let 
$$ \partial_r = \partial^{n,s,t}_r: \E{bo}{alg}_r^{n,s,t}(\AA0) \rightarrow \E{bo}{alg}_r^{n,s+1, t}(\FF_2) $$
be the map of spectral sequences induced by $\partial$.  Define a homomorphism
$$ \partial': \ker \partial^{n,s,t}_2 \rightarrow \mr{coker} \: \partial_2^{n+1,s-1,t}(\FF_2) $$
by 
$$ \partial'(z) = x  $$
where:
\begin{itemize}
\item $z$ is represented by $\td{z} \in \E{bo}{alg}^{n,s,t}_1(\AA0)$ with $\partial{\td{z}} = 0$ (that such a class $\td{z}$ always exists follows from 
Proposition~\ref{prop:MSSSESE1}, Proposition~\ref{prop:E1wssAmodA0}, and 
(\ref{eq:drwssAmodA0})).
\item $\td{y} \in \E{bo}{alg}^{n,s,t}(A \mmod A(0)_*)$ is chosen so that $p(\td{y}) = \td{z}$ (such a class $\td{y}$ exists by exactness of (\ref{eq:boMSSLESE1})).
\item $\td{x} \in \E{bo}{alg}^{n+1,s,t}(\FF_2)$ is chosen so that \begin{enumerate}
\item
$i(\td{x}) = d_1^{bo,alg}(\td{y})$ (such a $\td{x}$ exists by exactness of (\ref{eq:boMSSLESE1}),
\item
$d_1^{bo,alg}(\td{x}) = 0$ (that $\td{x}$ can be chosen to satisfy this property follows from Proposition~\ref{prop:MSSSESE1}, Proposition~\ref{prop:E1wssAmodA0}, and 
(\ref{eq:drwssAmodA0}).
\end{enumerate}
\item $x$ is represented by $\td{x}$.
\end{itemize}
\end{defn}

\begin{lem}
The map $\partial'$ is a well defined homomorphism.
\end{lem}

\begin{proof}
Suppose that $\td{z}'$, $\td{y}'$, $\td{x}'$ is another set of choices satisfying the same conditions as $\td{z}$, $\td{y}$, $\td{x}$.  Then since $\td{z}$ and $\td{z}'$ both represent $z$, there is a 
$$ z'' \in \E{bo}{alg}^{n-1,s,t}(\AA0)$$ with 
$$ d_1^{bo,alg}(z'') = \td{z} - \td{z}'. $$
By  Proposition~\ref{prop:MSSSESE1}, Proposition~\ref{prop:E1wssAmodA0}, and 
(\ref{eq:drwssAmodA0}), $z''$ may be chosen such that $\partial(z'') = 0$.  Then by exactness of (\ref{eq:drwssAmodA0}) there is a 
$$ y'' \in \E{bo}{alg}^{n-1,s,t}(A \mmod A(0)_*) $$
such that $p(y'') = z''$.  Since
$$ p(d_1^{bo,alg} - \td{y}+\td{y}') = 0, $$
exacness of (\ref{eq:drwssAmodA0}) shows there exists
$$ x'' \in \E{bo}{alg}_1^{n,s,t}(\FF_2) $$
with 
$$ i(x'') = d_1^{bo,alg} - \td{y}+\td{y}'. $$
Again invoking exactness of (\ref{eq:drwssAmodA0}), there is
$$ z''' \in \E{bo}{alg}^{n+1,s-1,t}(\AA0) $$
such that
$$ d_1^{bo,alg}(x'') = \td{x}' - \td{x} + \partial(z'''). $$
Thus $\td{x}$ and $\td{x}'$ represent the same class in $\mr{coker} \: \partial^{n+1,s-1,t}_2$.  Thus $\partial'$ is well defined.  It is now easy to see $\partial'$ is a homomorphism.
\end{proof}

We now have the definitions in place to give a succinct description of $\E{bo}{alg}_2(\br{A \mmod A(0)_*})$.
\end{comment}

\begin{prop}\label{prop:E2boMSSAA0}
The maps $\partial$, $\partial'$ induce isomorphisms
\begin{align*}
H^{n,s,t}(\mc{C}_{h_1}(\AA0)) &  \xrightarrow[\cong]{\partial} H^{n, s+1, t}(\mc{C}_{h_1}(\FF_2)), 
\\
H^{n,s,t}(\mc{C}_{v_0}(\AA0)) & \xrightarrow[\cong]{\partial'} H^{n+1,s, t}(\mc{C}_{v_0}(\FF_2)),
\quad s \ne t,
\\
H^{n,t}(V^{*,*}(\AA0)) & \xrightarrow[\cong]{\partial'} H^{n+1, t}(V^{*,*}(\FF_2)\oplus \mc{C}^{*,0,*}_{h_1}(\FF_2)).
\end{align*}
Here, the groups $H^{n, t}(V(\FF_2)\oplus \mc{C}^{*,0,*}_{h_1}(\FF_2))$ are determined by
$$
H^{n, t}(V(\FF_2)\oplus \mc{C}^{*,0,*}_{h_1}(\FF_2)) =
\begin{cases}
\E{bo}{alg}^{n,0,t}_2(\FF_2), & t \not\equiv 0 \mod 4, \\
H^{n,t}(V(\FF_2)), & \mr{otherwise}.
\end{cases}
$$
The connecting homomorphism $\partial_{alg}^{\AA0}$
in the long exact sequence (\ref{eq:LESM}) is determined for $t \equiv 0 \mod 4$ by the commutative diagram
$$
\xymatrix{
H^{n,0,t}(\mc{C}_{v_0}(\AA0)) \ar[r]^{\partial_{alg}^{\AA0}} \ar[d]_{\partial'}^{\cong} &
H^{n+1,t}(V(\AA0)) \ar[d]^{\partial'}_{\cong}
\\
H^{n+1,0,t}(\mc{C}_{{v_0}}(\FF_2)) \ar[r]_{\partial_{alg}^{\FF_2}} &
H^{n+2,t}(V(\FF_2))
}
$$
and for $t \not\equiv 0 \mod 4$ by the commutative diagram
$$
\xymatrix{
H^{n,0,t}(\mc{C}_{h_1}(\AA0)) \ar[r]^{\partial_{alg}^{\AA0}} \ar[d]_{\partial}^{\cong} &
H^{n+1,t}(V(\AA0)) \ar[d]^{\partial'}_{\cong}
\\
H^{n,1,t}(\mc{C}_{h_1}(\FF_2)) \ar@{=}[d] &
H^{n+2,t}(V(\FF_2) \oplus \mc{C}^{*,0,*}_{h_1}(\FF_2)) \ar@{=}[d]
\\
\E{bo}{alg}_2^{n,1,t}(\FF_2) \ar[r]_{d_2^{bo,alg}} &
\E{bo}{alg}_2^{n+2,0,t}(\FF_2)
}
$$
\end{prop}

\begin{proof}
The isomorphisms in the first part of the proposition follow from Proposition~\ref{prop:MSSSESE1} and the fact that we have (Proposition~\ref{prop:boMSSAmodA0collapse})
\begin{align*}
H^{n,s,t}(\mc{C}_{alg}(A \mmod A(0)_*)) & = 0, \quad \text{unless $n = 0$ and $s = t$}, \\
H^{n,t}(V(A \mmod A(0)_*)) & = 0.
\end{align*}
The identification of $H^{*,*}(V(\FF_2)\oplus \mc{C}_{h_1}^{*,0,*}(\FF_2))$ simply follows from the fact that for $t \not\equiv 0 \mod 4$ we have
\[ \E{bo}{alg}_1^{n,{0}, t}(\FF_2) = V^{n,t}(\FF_2)\oplus \mc{C}_{h_1}^{n,0,t}(\FF_2). \]
The diagram computing $\partial^{\AA0}_{alg}$ for $t \equiv 0 \mod 4$ follows from the fact that for such $t$ there is a $3\times 3$ diagram of short exact sequences of chain complexes
$$
\xymatrix{
V^{\bullet,t}(\FF_2) \ar[r] \ar[d] &
V^{\bullet,t}(A \mmod A(0)_*) \ar[r] \ar[d] &
V^{\bullet,t}(\AA0) \ar[d] 
\\
\E{bo}{alg}_1^{\bullet,0,t}(\FF_2) \ar[r] \ar[d] &
\E{bo}{alg}_1^{\bullet,0,t}(A \mmod A(0)_*) \ar[r] \ar[d] &
\E{bo}{alg}_1^{\bullet,0,t}(\AA0) \ar[d] 
\\
\mc{C}_{v_0}^{\bullet,0,t}(\FF_2) \ar[r] &
\mc{C}_{v_0}^{\bullet,0,t}(A \mmod A(0)_*) \ar[r] &
\mc{C}_{v_0}^{\bullet,0,t}(\AA0) 
}
$$
(note there is no sign in the commutativity of the resulting diagram of connecting homomorphisms because we are working in characteristic 2).
The proof of the commutativity of the diagram computing $\partial_{alg}^{\AA0}$ for $t \not\equiv 0 \mod 4$ will be deferred to the next subsection (Case~(1) of Theorem~\ref{thm:AA0}, $r = 1$).
\end{proof}

\subsection*{Higher differentials and hidden extensions in the $\bo$-MSS for $\pmb{\AA0}$.}

We have so far established a dictionary between classes in the $\bo$-MSS's for $\AA0$ and $\FF_2$:
\begin{align*}
x \: \text{is $h_1$-good} \quad \Leftrightarrow \quad & \partial(x) \: \text{is $h_1$-good} \\
\\
x \: \text{is not $h_1$-good} \quad \Leftrightarrow \quad & \partial'(x) \: \text{is not $h_1$-good} \\
& \text{or $\partial'(x)$ is $h_1$-good and lies in $s = 0$.}
\end{align*}

We will now extend this dictionary to all of the higher differentials. For the purpose of the statement of the next theorem, a non-trivial class $x \in \E{bo}{alg}_r(\AA0)$ will be regarded as $h_1$-good if and only if $\partial(x) \ne 0 \in \E{bo}{alg}_r(\FF_2)$.

\begin{thm}\label{thm:AA0}
Suppose that $z$ and $z'$ are classes in the $\bo$-MSS for $\AA0$.   \begin{enumerate}

\item
If $z$ is $h_1$-good and $z'$ not $h_1$-good, then 
$$ d^{bo, alg}_r (z) = z' \quad \Leftrightarrow \quad d^{bo,alg}_{r+1}(\partial(z)) = \partial'(z'). $$

\item
If both $z$ and $z'$ are $h_1$-good, we have
$$ d_r^{bo,alg}(z) = z'  \quad \Leftrightarrow \quad d^{bo,alg}_r(\partial(z)) = \partial(z'). $$ 

\item 
If $z$ is not $h_1$-good and $z'$ is $h_1$-good, we have
$$ d_{r}^{bo,alg}(z) = z' \quad \Leftrightarrow \quad d_{r-1}^{bo,alg}(\partial'(z)) = \partial(z'). $$

\item
If neither $z$ nor $z'$ are $h_1$-good, we have
$$ d_r^{bo,alg}(z) = z'  \quad \Leftrightarrow \quad d^{bo,alg}_r(\partial'(z)) = \partial'(z'). $$ 
\end{enumerate}
\end{thm}

\begin{proof}
This theorem is proven using a combination of the Connecting Homomorphism Lemma (CHL) (Lemma~\ref{lem:MSSSES}) and an adaptation of the Geometric Boundary Theorem (GBT) \cite[Lem. A.4.1]{goodEHP} to our algebraic setting --- the $\bo$-MSS's associated to the short exact sequence of $A_*$-modules:
$$ 0 \rightarrow \FF_2 \xrightarrow{i} A \mmod A(0)_* \xrightarrow{p} \AA0 \rightarrow 0. $$ 
In general, the GBT has a great deal of potential ambiguity.  However, in our case much of it goes away as the $\bo$-MSS for $A \mmod A(0)_*$ collapses at $E_2$, where it is virtually acyclic (Proposition~\ref{prop:boMSSAmodA0collapse}).

Suppose first that $z$ is $h_1$-good, so $\partial(z) \ne 0$.  Suppose
there is a non-trivial differential
\begin{equation}\label{GBT1}
d_r(z) = z' \ne 0
\end{equation}
and $\partial(z') = 0$ (i.e. $z'$ is either $v_0$-good or evil).  Then we apply Case~(2) of the GBT to (\ref{GBT1}) deduce that there is a differential
$$ d_{r+1}(\partial(z)) = \partial'(z') \ne 0. $$
This establishes:
\begin{quote}
(1) $\quad$ If $z$ is $h_1$-good and $z'$ not $h_1$-good, then 
$$ d_r (z) = z' \ne 0 \quad \Leftrightarrow \quad d_{r+1}(\partial(z)) = \partial'(z') \ne 0. $$
\end{quote}
Suppose however that the class $z'$ of (\ref{GBT1}) is not $h_1$-good.  Then it follows from the CHL that we have 
$$ d_r(\partial(z)) = \partial(z'). $$
This implies
\begin{quote}
(2') $\quad$ If both $z$ and $z'$ are $h_1$-good, we have
$$ d_r(z) = z' \ne 0  \quad \Rightarrow \quad d_r(\partial(z)) = \partial(z'). $$ 
\end{quote}
Furthermore, we deduce
\begin{quote}
(2.5) $\quad$ If $z$ is an $h_1$-good permanent cycle, then $\partial(z)$ is a permanent cycle.
\end{quote}  

Suppose now that $z$ is a non-trivial class which is not $h_1$-good, so $\partial(z) = 0$.  By (\ref{eq:LESM}), we deduce that there is a $y \in \E{bo}{alg}_1(A \mmod A(0)_*)$ so that $p(y) = z$.  Then there is a non-trivial differential (Proposition~\ref{prop:boMSSAmodA0collapse}) 
$$ d_1(y) = y' \ne 0. $$
We apply the GBT to this differential.  Note that by the definition of the connecting homomorphism $\partial'$, there is a representative for $\partial'(z)$ in $\E{bo}{alg}_{1}(\FF_2)$ (which we abusively also call $\partial'(z)$) so that
$$ i(\partial'(z)) = y'. $$
Case (2) of the GBT then implies that if there is a non-trivial differential
\begin{equation}\label{eq:GBT2}
d_{r}(z) = z' \ne 0
\end{equation}
with $z'$ $h_1$-good,
then there is a differential
$$ d_{r-1}(\partial'(z)) = \partial(z'). $$
Thus we have shown
\begin{quote}
(3') $\quad$ If $z$ is not $h_1$-good and $z'$ is $h_1$-good, we have
$$ d_{r}(z) = z' \ne 0 \quad \Rightarrow \quad d_{r-1}(\partial'(z)) = \partial(z'). $$
\end{quote}
Suppose however that $z'$ of (\ref{eq:GBT2}) is not $h_1$-good.  Then we are in Case~(3) of the GBT, and we have
$$ d_r(\partial'(z)) = \partial'(z'). $$  
We have shown
\begin{quote}
(4') $\quad$ If $z$ and $z'$ are not $h_1$-good, we have
$$ d_{r}(z) = z' \ne 0 \quad \Rightarrow \quad d_{r}(\partial'(z)) = \partial'(z'). $$
\end{quote}
Finally, suppose that $z$ is a permanent cycle.  Then Cases~(4)-(5) of the GBT imply that $\partial'(z)$
is a permanent cycle.  Thus we have shown
\begin{quote}
(4.5) $\quad$ If $z$ is a permanent cycle which is not $h_1$-good, then $\partial'(z)$ is a permanent cycle.
\end{quote}



Suppose inductively that we have established (2), (3), (4) for all $r < r_0$. Since we have also established (1), we have then established our dictionary between $d_r$ differentials in the $\bo$-MSS for $\AA0$ for $r < r_0$ and corresponding differentials in the $\bo$-MSS for $\FF_2$.  The differentials on the right-hand side of (2'), (3'), and (4') could in principle be trivial, but only if their targets were hit by non-trivial shorter differentials in the $\bo$-MSS for $\FF_2$.  This would violate our inductive hypothesis.  We conclude inductively that the differentials on the right-hand side of (2'), (3'), and (4') are actually non-trivial.  This, combined with (2.5) and (4.5), upgrades these statements to (2), (3), and (4) of the theorem.
\end{proof}
