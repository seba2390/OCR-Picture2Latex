% !TEX root = boASS6.tex

\section{Introduction}\label{sec:intro}

Let $\KO$ denote the ($2$-local) real $K$-theory spectrum, and let $\bo$ denote its connective cover. The $\bo$-based Adams
spectral sequence (ASS) for the sphere takes the form: 
$$ \E{bo}{}_1^{n,t} = \pi_t \bo \s \br{\bo}^{\s n} \Rightarrow (\pi^{s}_{t-n})_{(2)}. $$ 
Here, $\br{\bo}$ denotes the cofiber of the unit 
$$ S \rightarrow \bo \rightarrow \br{\bo} $$
and $\pi_*^s$ denotes the stable homotopy groups of spheres.

Many researchers have studied aspects of the $\bo$-ASS, most notably Mahowald in his paper \cite{Mahowaldbo}, where this
spectral sequence is used to prove the $2$-primary $v_1$-periodic telescope conjecture. However, a systematic study of this
spectral sequence as a tool to perform low-dimensional computations of the stable homotopy groups of spheres was not
undertaken until \cite{LM} and \cite{Davisbo}. Early on, the structure of the $E_1$-term was known \cite{Mahowaldbo}, \cite{Milgram}. Namely,
there is a splitting $$ \bo \s \bo \simeq \bigvee_i \Si^{4i} \bo \wedge B_i $$ where $B_i$ is the $i$th integral
Brown-Gitler spectrum \cite{CDGM}. This splitting iterates to yield a splitting 
$$ \bo \wedge \br{\bo}^n \simeq \bigvee_{i_1, \ldots, i_n}\Si^{4(i_1+ \cdots + i_n)} \bo \s B_{i_1} \s \cdots \s B_{i_n} $$
where each index satisfies $i_l > 0$.
The homotopy type of the wedge summands in the above splitting were also determined in \cite{Mahowaldbo}, \cite{Milgram}: for a multi-index $I = (i_1, \ldots, i_n)$, there is an equivalence
$$ \bo \s B_{i_1} \s \cdots \s B_{i_n} \simeq b_I \vee HV_I $$
where $b_I$ is a certain Adams cover of $\bo$ or $\bsp$ (see Section~\ref{sec:E1}), and $HV_I$ is the Eilenberg-MacLane spectrum associated to a graded $\FF_2$-vector space $V_I$.

There is a corresponding decomposition
\begin{align*}
\E{bo}{}_1^{n,*} & = \bigoplus_{I= (i_1, \ldots, i_n)} \Si^{4(i_1, \ldots, i_n)} \pi_*  b_I \oplus \Si^{4(i_1, \ldots, i_n)}  V_I.
\end{align*}
The subspaces
$$ V^{n,*} := \bigoplus_{I=( i_1, \cdots ,i_n)} \Si^{4(i_1, \ldots, i_n)} V_I $$
are closed under the $d_1$-differential.  Following \cite{LM}, \cite{Davisbo}, we define $\mc{C^{*,*}}$ to be the quotient complex
$$ 0 \rightarrow V^{*,*} \rightarrow \E{bo}{}_1^{*,*} \rightarrow \mc{C}^{*,*} \rightarrow 0 $$
so that
$$ \mc{C}^{n,*} \cong \bigoplus_{I= (i_1, \ldots, i_n)} \Si^{4(i_1, \ldots, i_n)}  \pi_*  b_I, $$
resulting in a long exact sequence
\begin{equation}\label{eq:LES}
\cdots \rightarrow H^{*,*}(V) \rightarrow \E{bo}{}_2^{*,*} \rightarrow H^{*,*}(\mc{C}) \xrightarrow{\partial} H^{*+1, *}(V) \rightarrow \cdots.
\end{equation}

Historically, most of the applications of the $\bo$-ASS have rested on analyzing $H^{*,*}(\mc{C})$ while bounding the effects of $H^{*,*}(V)$.  In fact, 
a complete computation of $H^{*,*}(\mc{C})$ was accomplished in \cite{LM}.  By contrast, the complex $V^{*,*}$, and its cohomology, have proven to be the major obstacle to using the $\bo$-ASS to perform low-dimensional computations.  Carlsson computed $V^{1,*}$ explicitly in \cite{Carlsson}, but the result is rather unwieldy.  Davis \cite{Davisbo} used computer computations to compute $V^{n,t}$, and its cohomology, through the range $t \le 25$.  He observed there that a rapid exponential growth in the dimension of $V^{n,t}$ severely limits this approach.  Furthermore, Davis's approach does not include a method for computing the map $\partial$ in (\ref{eq:LES}).  In essence, $V^{*,*}$ resembles the cobar complex for the Steenrod algebra (but seemingly without a good conceptual description), and as such, is too large for extensive computer computations.

Motivated by this disparity of computability, we shall refer to elements of $\E{bo}{}_2^{*,*}$ with non-trivial image in $H^{*,*}(\mc{C})$ as \emph{good}, and we shall refer to non-zero elements of $\E{bo}{}_2^{*,*}$ which are in the image of the map from $H^{*,*}(V)$ as \emph{evil}.  The purpose of this paper is to present a method of computing the evil part of $\E{bo}{}_2^{*,*}$.  This involves (1) computing $H^{*,*}(V)$, and (2) understanding the long exact sequence (\ref{eq:LES}).

The basic idea is to use an algebraic analog of the $\bo$-resolution (called the Mahowald spectral sequence in \cite{Miller}), studied by Davis, Mahowald, Miller and others, which we will refer to as the $\bo$-MSS.  The $\bo$-MSS is a spectral sequence of the form
$$ \E{bo}{alg}_1^{n,s,t} = \Ext^{s,t}_{A(1)}(\br{A\mmod A(1)}^{n}, \FF_2) \Rightarrow \Ext^{s+n,t}_{A}(\FF_2, \FF_2). $$
Analogous to the topological decomposition, there is a decomposition
$$ \E{bo}{alg}_1^{*,*,*} = \mc{C}^{*,*,*}_{alg} \oplus V^{*,*,*}_{alg} $$
and thus a long exact sequence
\begin{equation}\label{eq:algLES}
\cdots \rightarrow H^{*,*,*}(V_{alg}) \rightarrow \E{bo}{alg}_2^{*,*,*} \rightarrow H^{*,*,*}(\mc{C}_{alg}) \xrightarrow{\partial_{alg}} H^{*+1, *,*}(V) \rightarrow \cdots.
\end{equation}

The key observation (implicit in \cite{Davisbo}) is:
\begin{obs}
The terms $V_{alg}^{n,s,t}$ are zero unless $s = 0$, and there is an isomorphism of cochain complexes:
$$ V_{alg}^{n,0,t} \cong V^{n,t} $$
and therefore an isomorphism
$$ H^{*,*}(V) \cong H^{*,0,*}(V_{alg}). $$
\end{obs}
 In fact, the map $\partial$ turns out to be computable in terms of the map $\partial_{alg}$.  In short, a complete understanding of the evil part of $\E{bo}{}_2$ can be extracted from a complete understanding of the evil part of $\E{bo}{alg}_2$.  

Our methods rest on the idea that the groups 
$\Ext^{*,*}_{A}(\FF_2, \FF_2)$ are easily understood (through a range) by computer computations using minimal resolutions \cite{Bruner}.  Furthermore, analogous to the topological situation, the groups $H^{*,*,*}(\mc{C}_{alg})$ can be completely computed.  One can then deduce $H^{*,*,*}(V_{alg})$ and $\partial_{alg}$ from the existence of the $\bo$-MSS, using knowledge of $\Ext^{*,*}_{A}(\FF_2, \FF_2)$,  \emph{provided one has a means of determining which elements of $\Ext$ are detected by good classes and which are detected by evil classes in the $\bo$-MSS}.  

We shall say a non-trivial class $x \in \Ext^{*,*}_{A}(\FF_2, \FF_2)$ is \emph{evil} if there exists an evil class in the $\bo$-MSS which detects it; otherwise we shall say $x$ is \emph{good}.
Our main theorem establishes a precise relationship between being good, and being $v_1$-periodic (Lemma~\ref{lem:onethird} and Theorem~\ref{thm:dichotomy}):


\begin{thm*}[Dichotomy Principle]
Suppose $x$ is a non-trivial class in $\Ext^{s,t}_{A}(\FF_2, \FF_2)$.
\begin{enumerate}
\item 
If $x$ is $v_1$-torsion, it is evil.

\item If $x$ lies above the 1/3 line ($t-s < 3s$), then it is good and $v_1$-periodic.

\item 
Suppose $x$ is $v_1$-periodic.  There is an $M \gg 0$ for which $y = v_1^{M} x$ is defined and lies above the $1/3$-line.  Suppose $y$ has $\bo$-filtration $n$.  The class $x$ is good if and only if 
$$ s \ge n. $$
\end{enumerate}
\end{thm*}

Our technique of using knowledge of $\Ext^{s,t}_{A}(\FF_2, \FF_2)$ and $H^{*,*,*}(\mc{C}_{alg})$ to deduce $H^{*,*}(V)$ via the notions of good and evil is encoded more fluently in a refinement of the $\bo$-MSS we develop, called the \emph{agathokakological spectral sequence}
$$ \E{akss}{alg}^{*,*,*}_1 = \mc{C}_{alg}^{*,*,*} \oplus V^{*,*} \Rightarrow \Ext^{s,t}_{A}(\FF_2, \FF_2). $$
The indexing in this spectral sequence is non-standard, but is set up in such a manner that good and evil classes actually live in distinct tri-degrees.

Mahowald's proof of the $v_1$-periodic telescope conjecture \cite{Mahowaldbo} rested on two fundamental theorems: the \emph{Bounded Torsion Theorem}, and the \emph{Vanishing Line Theorem}.  We note that our perspective on the $\bo$-ASS $E_2$-term organically produces these results (Corollary~\ref{cor:BTT} and Theorem~\ref{thm:vanishing}).

We demonstrate the efficacy of our technique by computing the $\bo$-ASS through the $40$-stem.  The previous computations of \cite{LM} and \cite{Davisbo} only run through the $20$-stem.  In principle, our method could be employed to compute the $\bo$-ASS through a larger range, but it quickly becomes apparent that as a function of the stem,  increasingly large portions of $\pi_*^s$ are detected by evil classes in the $\bo$-ASS, and beyond our range it appears that the $\bo$-ASS is not much more effective than the classical ASS.


Besides being an attempt to streamline and extend previous work on the $\bo$-ASS, this paper represents a developmental platform for methods the authors hope to employ to study the $\tmf$-resolution (extending the program of \cite{BOSS}).

It should be noted that Gonz\'alez's work on the odd primary $BP\bra{1}$-resolution \cite{Gonzalez2} suggests an interesting alternative to the methodology of this paper. Namely, Gonz\'alez shows that the analog of the complex $V^{*,*}$ for the Adams cover $S^{\bra{1}}$ of the sphere  spectrum is isomorphic to a cobar complex for a relative $\Ext$ group. A similar observation carries over to the $2$-primary case.  It would be an interesting project to attempt to compute these relative Ext groups with some variant of the May spectral sequence or a minimal resolution technique.


\subsection*{Organization of the paper}

The paper is organized as follows. In Section~\ref{sec:E1} we review the structure of the $E_1$-page of the $\bo$-ASS,
together with the good component of the $d_1$-differential. In Section~\ref{sec:wss} we describe the Lellmann-Mahowald
weight spectral sequence, a spectral sequence for computing $H^{*,*}(\mc{C})$.  In Section~\ref{sec:alg}, we describe the algebraic $\bo$ resolution (the $\bo$-MSS), and compute the corresponding weight spectral sequence converging to $H^{*,*,*}(\mc{C}_{alg})$. 
In Section~\ref{sec:v1}, we review the notion of $v_1$-periodicity in $\Ext$, define the $v_1$-periodic $\bo$-resolution, and discuss its convergence.  
In Section~\ref{sec:AA0}, we discuss the $\bo$-MSS for $\br{A \mmod A(0)}$.  The Ext groups of $\br{A \mmod A(0)}$ are identical to those of $\FF_2$ away from the $0$-stem, but the $v_1$-periodic behavior of the former is much more manageable.
In Section~\ref{sec:rules}, we define the agathokakological spectral sequence (AKSS) and prove the Dichotomy Principle.  We also introduce a topological analog of the AKSS, which refines the $\bo$-ASS.  In Section~\ref{sec:algcomp}, we use the results of the previous section, together with knowledge of $\Ext_A(\FF_2, \FF_2)$, to completely compute the algebraic $\bo$-resolution through a large range.  The information this affords us about $H^{*,*}(V)$, and the connecting homomorphism $\partial$, is then used to compute the $\bo$-ASS through the same range in Section~\ref{sec:topcomp}.  



\subsection*{Conventions} In the remainder
of this paper everything is implicitly localized at the $p = 2$.  

Homology $H_*$ will always be implicitly taken with $\FF_2$ coefficients.
We let $A$ denote the $2$-primary Steenrod algebra, and $A_*$ its dual.  For a sub-Hopf algebra $B \subseteq A$, we will use $B_*$ to denote its dual, and we will use $A\mmod B_*$ to denote the dual of the Hopf-algebra quotient $A\mmod B$.  We let $\zeta_i$ denote the conjugate of the element $\xi_i \in A_*$.  We shall sometimes use the abbreviation
$$ \Ext^{*,*}_{A_*}(M) := \Ext^{*,*}_{A_*}(\FF_2, M). $$
For any connective ($2$-local) finite type spectrum $X$, we let $X^{\bra{s}}$ denote its $s$th Adams cover, so that
$$ X \leftarrow X^{\bra{1}} \leftarrow X^{\bra{2}} \leftarrow \cdots $$
is a minimal Adams resolution of $X$.
The associated classical Adams spectral sequence (ASS) will be denoted
$$ \E{ass}{}^{s,t}_2(X) = \Ext_{A_*}^{s,t}(H_*X) \Rightarrow \pi_{t-s} X. $$
For a non-zero element $x \in \pi_*X$, we let 
$$ [x] \in \E{ass}{}_\infty(X) $$
denote the class in $E_\infty$ which detects it.  Cycles in the $E_r$-page of the ASS will be automatically regarded as representing elements of the $E_{r+1}$-page.  Thus, given an element $y \in \E{ass}{}_2^{*,*}$, the expression
$$ y = [x] $$
means that $y$ detects $x$ in the ASS.  For a general ring spectrum $R$, we let $\{ \E{R}{}^{*,*}_r(X) \}$ denote the $R$-based Adams spectral sequence for $X$.

For a non-negative integer $i$, we let $\alpha(i)$ denote the sum of the digits of its base $2$ expansion.
For a multi-index 
$$ I = (i_1, \ldots, i_n) $$
we define
\begin{align*}
\abs{I} & := n, \\
\norm{I} & := i_1 + \cdots + i_n, \\
\alpha(I) & := \alpha(i_1) + \cdots + \alpha(i_n).
\end{align*}

\subsection*{Acknowledgments}  The authors thank the referee for their valuable suggestions and comments, and John Rognes for pointing out an error in the statement of a computation.

