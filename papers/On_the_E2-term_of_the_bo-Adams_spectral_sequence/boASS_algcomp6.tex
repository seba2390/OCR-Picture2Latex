% !TEX root = boASS6.tex

\section{Computation of the algebraic $\bo$-resolution}\label{sec:algcomp}


In this section we will compute the algebraic $\bo$-resolution, or more specifically, the (algebraic) AKSS, through dimension $42$.  We use the known computation of $\Ext_{A_*}(\FF_2)$ through this range (see, for example, the May spectral sequence computation of \cite{Tangora} or the computer computation of \cite{Bruner},\cite{Brunertable}), as well as our computation of $H^{*,*,*}(\mc{C}_{alg})$ (Theorem~\ref{thm:HCalg}) to deduce the groups $H^{*,*,*}(V)$, and the subsequent differentials.  Key to this is the determination of which classes of $\Ext_{A_*}(\FF_2)$ are good and which are evil.  This is done with the Dichotomy Prinicple (Theorem~\ref{thm:dichotomy}).  A prerequisite to applying the Dichotomy Principle is the determination of which elements in $\Ext_{A_*}(\FF_2)$ in our range are $v_1$-periodic, and which are $v_1$-torsion.

To this end we begin this section with an analysis of $v_1$-periodicity and torsion in our range.  We then explicitly write out $H^{*,*,*}(\mc{C}_{alg})$ and apply the Dichotomy Principle to determine which classes in $\Ext_{A_*}(\FF_2)$ are good and which are evil.  We then prove a couple of convenient lemmas which relate the $d_2$ differentials in the AKSS to $v_1^4$-multiplication.  We then do a  stem by stem computation of the AKSS through dimension $42$. 

\subsection*{$v_1$-periodicity and $v_1$-torsion in low degrees}

In order to invoke the Dichotomy Principle to determine which classes in $\Ext_{A_*}(\FF_2)$ are good and which classes are evil in low degrees, it is necessary to determine which classes in this range are $v_1$-periodic, and which are $v_1$-torsion.

Ideally, this would be accomplished by actually having a complete computation of $v_1^{-1}\Ext_{A_*}(\FF_2)$.  To date, this has not been done.  However, Davis and Mahowald have computed $v_1^{-1}\Ext_{A_*}(H_*Y)$ where $Y = M(2) \wedge C\eta$ \cite{DavisMahowaldv1}.  Note that Andrews \cite{Andrews} has computed $v_1^{-1}\Ext_{A_*}(\FF_p)$ for $p$ odd.

\begin{rmk}
The best available tool to compute $v_1^{-1}\Ext_{A_*}(\FF_2)$ is probably the localized algebraic $\bo$-resolution
\begin{equation}\label{eq:v1algbo}
 v_1^{-1}H^{*,*,*}(\mc{C}_{alg}) \Rightarrow v_1^{-1}\Ext_{A_*}(\FF_2).
 \end{equation}
The only difficulty is that this spectral sequence has many long differentials (as will be demonstrated in our low degree calculations in this section).  Nevertheless, it seems plausible that with enough care the spectral sequence (\ref{eq:v1algbo}) could be completely computed.  The odd primary analog of (\ref{eq:v1algbo}) actually collapses, giving a convenient alternative approach to Andrews' computation of $v_1^{-1}\Ext_{A_*}(\FF_p)$ for $p$ odd.
\end{rmk}

In the absence of a complete understanding of $v_1^{-1}\Ext$, we instead manually identify the kernel of the homomorphism
(a.k.a. the $v_1$-torsion)
$$ \Ext^{s,t}_{A_*}(\FF_2) \rightarrow v_1^{-1}\Ext^{s,t}_{A_*}(\FF_2) $$
in low degrees.  Figure~\ref{fig:v1Extchart} shows an Ext chart in low degrees (courtesy of Perry's Ext software).  The chart is broken into regions which indicate for which $N$ multiplication by $v_1^{2^N}$ is well-defined (c.f. Proposition~\ref{prop:v1ext}).  The circled classes span the $v_1$-torsion (see the following proposition).  The classes decorated with triangles represent the only classes in this range which are $v_1$-periodic, but evil --- this will be explained in the next subsection.


\begin{figure}
\includegraphics[angle=90,height=\textheight]{v1Extchart.pdf}
\caption{$v_1$-periodicity in $\Ext_{A_*}(\FF_2)$.}\label{fig:v1Extchart}
\end{figure}

\begin{prop}\label{prop:v1Extchart}
The $v_1$-torsion in $\Ext^{s,t}_{A_*}(\FF_2)$ for $0\leq t-s\leq 43$ is spanned by the circled classes in Figure~\ref{fig:v1Extchart}.
\end{prop}

\begin{proof}
This represents a tedious analysis of available Ext data, the highlights of which we summarize here.
Basically, one must check that the images of the uncircled classes are linearly independent in $v_1^{-1}\Ext_{A_*}(\FF_2)$, and that the circled classes are $v_1$-torsion.

The vast majority of the uncircled classes are $v_1^4$-periodic.  This can be verified by checking that for such classes $x$, the image of the iterated Adams $P$ operator $P^N x$ is non-trivial, for $N$ sufficiently large that $P^N x$ lies above the $1/3$-line.  These verifications can be done using Bruner's tables \cite{Bruner}.  In many instances, the process is expedited by simply observing that the corresponding classes map non-trivially to $\Ext_{A(2)_*}$, where everything is $v_1^4$-periodic.


The only $v_1$-periodic classes this technique does not apply to are those in the $v_0$-towers $v_0^jh_i$ in $t-s = 2^i-1$, the ``broken'' $v_0$-towers in $t-s = 2^i-2$, and the $v_0$-towers with bottoms in bidegrees $(t-s,s) = (37, 5)$ and $(38,6)$.

In the case of the $v_0$-towers $v_0^jh_i$, the tops of the towers are actually $v_1^4$-periodic (by the $P$-operator argument).  That implies that the images of the top of the towers in $v_1^{-1}\Ext_{A_*}(\FF_2)$ are non-trivial.  It must therefore be the case that all of the classes $v_0^jh_i$ are non-trivial.

In the case of the broken $v_0$ towers in $t-s = 2^i-2$, we employ a bit of a cheat.
The idea is that in the localized ASS
$$ v_1^{-1}\Ext_{A_*}(\FF_2) \Rightarrow v_1^{-1}\pi_* S $$
the tops of the broken $v_0$-towers in dimensions $s2^i-2$ are the targets of differentials on the $v_0$-towers in dimensions $s2^i-1$.  If differentials of the same length happen in the unlocalized ASS, then the targets of these differentials must be non-trivial under the map of ASS's:
$$
\xymatrix{
\Ext_{A_*}(\FF_2) \ar[d] \ar@{=>}[r] &  \pi_* S \ar[d]
\\
v_1^{-1}\Ext_{A_*}(\FF_2) \ar@{=>}[r] &  v_1^{-1}\pi_* S
}
$$
We illustrate this principle with an example: we will show the class $v_0h_3^2$ is $v_1^8$-periodic.  In the unlocalized ASS, there is a differential
$$ d_2 (v_0^4 Q') = v_0^5 i^2 $$
(where $Q'$ is in degree $(47,13)$ and $i^2$ in $(46,14)$)
and this differential must map to a non-trivial $d_2$-differential in the localized ASS since we are working in a region where the map
$$ \Ext_{A_*}(\FF_2) \rightarrow v_1^{-1}\Ext_{A_*}(\FF_2) $$
is an isomorphism.  In the localized ASS, this differential interpolates back to a differential
$$ d_2 (h_4 ) = v_1^{-16} v_0^5 i^2. $$
Since in the unlocalized ASS we have
$$ d_2  (h_4)  = v_0 h_3^2 $$
we deduce that $v_0 h_3^2$ must map to $v_1^{-16} v_0^5 i^2$ in $v_1^{-1}\Ext$, and therefore in $\Ext$ we have
$$ v_1^{16} v_0 h_3^2 = v_0^5 i^2. $$
Since $v_0^5 i^2$ is above the 1/3 line, we deduce that $v_0 h_3^2$ is $v_1$-periodic.  Because it is located in the region of $\Ext$ where multiplication by $v_1^8$ is well-defined, we deduce that in fact $v_0h_3^2$ is $v_1^8$-periodic.

In the case of the towers in $(t-s,s) = (37, 5)$ and $(38,6)$, we deduce that they are $v_1$-periodic as follows.  It suffices to show the tops of these towers $v_0^5 x$ and $v_0^3 y$ are $v_1$-periodic.
We have
\begin{align*}
v_0^5 x & = {h_2^2 d_0e_0}, \\
v_0^3 y & = h_2^2 l.
\end{align*}
The classes $h_2 d_0$ and $h_2 l$ have already been shown to be $v_1^4$-periodic, and we have
\begin{align*}
v_0^8 P^2 x' & = {h_2^2 P^4 d_0e_0}, \\
v_0^8 R'_1 & = h_2 P^4 h_2 l,
\end{align*}
for $P^2 x' $ in $(69,18)$ and $R_1'$ in $(70,17)$.
Moreover, we are working in a range (c.f. Theorem~\ref{thm:AdamsP} where the composite
\begin{equation}\label{eq:A4}
\Ext^{s,t}_{A_*} \rightarrow \Ext^{s-1,t}_{A_*}(\br{A \mmod A(0)}_*) \rightarrow \Ext_{A(4)_*}(\br{A \mmod A(0)}_*)
\end{equation}
is an isomorphism.  Since $v_1^{16} \in \Ext_{A(4)}(\FF_2)$, and the composite (\ref{eq:A4}) is $h_2$-linear, we deduce that
\begin{align*}
v_1^{16} v_0^5 x & = v_0^8 P^2 x', \\
v_1^{16} v_0^3 y & = v_0^8 R'_1.
\end{align*}
Since $v_0^8 P^2 x'$ and $v_0^8 R'_1$ lie above the 1/3 line, they are $v_1$-periodic.  It follows that $v_0^5 x$ and $v_0^3 y$ are both $v_1$-periodic.

We now explain how to check that the circled classes are $v_1$-torsion. The classes which are divisible by $h_i$ for $i \ge 4$ can be handled with the following trick, which we illustrate in the case of $i = 4$.  Using the fact that $h_4$ is $v_1$-periodic, we can deduce that
$$ v_1^{16} h_4 = v_0^4 Q'. $$
It follows that for all $x = h_4 y$,
$$ v_1^{16}x = v_1^{16} h_4 y = v_0^4 Q' y. $$
Therefore, if $y$ is annihilated by $v_0^4$, then $v_1^{16} h_4 y = 0$.
It follows that $v_1^{16}$ times the classes
$$ h_4 h_1, \: h_4 h_2, h_4c_0 $$
are all zero. So the same is true of their $v_0$, $h_1$, and $h_2$-multiples.
The same trick shows that $v_2^{32} h_5 y = 0$ is true for all of the relevant values of $y$.


The remaining cases, such as $c_1$, $n$, etc. are handled by observing that the only classes which could detect their $v_1^{16}$ (or $v_1^{32}$, as appropriate) multiples either (1) have non-trivial products with $v_0$ or $h_1$ which the original class does not have, or (2) map non-trivially to $v_1^{-1}\Ext_{A(2)_*}(\FF_2)$ (and the original class maps trivially into this Ext group.)  The case of $f_1$ in $(t-s,s) = (40,4)$ is somewhat tricky.  The only candidate to detect $v_1^{32}f_1$ has a non-trivial product with $d_0$, whereas $f_1d_0 = 0$.  The same argument shows the class $c_2$ in $(41,3)$ is $v_1$-torsion.
\end{proof}

\subsection*{The cohomology of $\mc{C}_{alg}$}
Throughout this section, by AKSS we refer to the \emph{algebraic} AKSS. The cohomology groups $H^{n,s,t}(\mc{C}_{alg})$ can be read off from Theorem~\ref{thm:HCalg}.  The result, in low degrees, is depicted in Figure~\ref{fig:HCchart}. In this chart, the $x$-axis denotes $t-(s+n)$, and the $y$-axis denotes $s+n$.  The $\bo$-filtration $n$ is encoded by a color given in Figure~\ref{fig:colorbo}.
Application of the Dichotomy Principle (Theorem~\ref{thm:dichotomy}) gives the following.

\begin{figure}
\begin{tabular}{ | c | c | }
  \hline
  $n$ & color \\
  \hline
  \hline
  0 & black  \\
  \hline
  1 & {\color{blue}blue}  \\
  \hline
  2 & {\color{red}red}   \\
  \hline
   3 & {\color{orange}orange}  \\
  \hline
   4 & {\color{limegreen}green} \\
  \hline
     5 & {\color{cyan}cyan} \\
  \hline
     6 & {\color{lavenderrose}pink} \\
  \hline
       7 & {\color{darkmagenta}purple} \\
  \hline
       8 & {\color{goldenpoppy}yellow} \\
  \hline
       9 & {\color{seagreen}forest green} \\
  \hline
\end{tabular}
\caption{The $\bo$-filtration.}
\label{fig:colorbo}
\end{figure}

\begin{prop}
The classes in Figure~\ref{fig:v1Extchart} which are evil are precisely those classes which are marked with a circle (case (1) of Theorem~\ref{thm:dichotomy}) or a triangle (case (2) of Theorem~\ref{thm:dichotomy}).
\end{prop}






\subsection*{The propagation of good : $v_1$--periodic differentials}

We now begin our stemwise computation of the algebraic AKSS
$$ \E{akss}{alg}_{1+\epsilon}^{n+\alpha\epsilon, s,t} \Rightarrow \Ext^{s+n,t}_{A_*}(\FF_2). $$
The full chart of this spectral sequence in the range we will be considering (starting at the $E_{1+\epsilon}$ page) is shown in Figure~\ref{fig:AAKSSchart}.  
%The $x$-coordinate of this chart is $t-(s+n)$ and the $y$ coordinate is $s+n$.

\begin{figure}
\includegraphics[angle=90,height=\textheight]{AKSS-alg-11_11_2018-cropped.pdf}
\caption{The algebraic AKSS. 
%The black line is the line $y=\frac{x}{3}$ line. The gray lines are the lines $y=\frac{x}{2}-2^n+3$ of Proposition~\ref{prop:v1ext}, which determine the regions of Figure~\ref{fig:v1Extchart}.
}\label{fig:AAKSSchart}\label{fig:HCchart}.
\end{figure}


\begin{not*}We name the classes in Figure~\ref{fig:AAKSSchart} by
\[(x,y: n),\]
where $(x,y)=(t-(s+n),s+n)$ is the Adams coordinate and $n$ is the $\bo$--filtration. We use the same notation, but add a superscript to denote evil classes, i.e., $(x,y: n)^{ev}$. We call this the \emph{nature} of the class.
If multiple classes have the same nature, we distinguish them by a subscript $(x,y: n)_k$, respectively $(x,y: n)_k^{ev}$. The subscript $k$ denotes that this is the $k$'th class from the left in our chart with this nature, counting evil and good separately. %, since they have different natures).
\end{not*}


We will use the following lemma.
\begin{lem}\label{lem:v14}
Let $a$ and $b$ be elements on the $E_2$-page of the AKSS, which are in the Adams coordinate $(x, y)$ and $(x-1, y+1)$. Suppose firstly that $v_0  a =0$ and $v_1^4 a \neq 0$ on the $E_2$-page of the AKSS, and secondly that the element $v_0^3 {\color{blue}(7,1:1)}$ annihilates all elements in the Adams coordinates $(x+1, y)$ and $(x, y+1)$. Then $d_2(a) = b$ implies that $d_2(v_1^4  a) = v_1^4  b$.
\end{lem}
\begin{proof}
We have a $d_1$ differential $d_1(v_1^4) = v_0^4 {\color{blue}(7,1:1)}$. Therefore, on the $E_2$-page of the AKSS, we have the following Massey products with zero indeterminacies:
\[v_1^4 a = \left<a, v_0, v_0^3{\color{blue}(7,1:1)}\right>,\]
\[v_1^4 b = \left<b, v_0, v_0^3{\color{blue}(7,1:1)}\right>.\]
The first condition implies that the Massey products are well-defined. Note that by $v_0$-linearity, $v_0  a =0$ and $d_2(a) = b$ imply that $v_0  b = 0$. The second condition implies that the Massey products have zero indeterminacies. Note that the Adams bidegree of $v_1^4$ is zero starting from the $E_2$-page. Therefore, the lemma follows from Leibniz's rule of Massey products.
\end{proof}


Finally, recall from Proposition~\ref{prop:v1ext} that $N(x) \leq n$ if, for $(t-s,s)$ the Adams coordinates of $x$, we have $0<t-s<2s+2^{n+1}-6$. This is illustrated in Figure~\ref{fig:v1Extchart}.


We begin by establishing families of periodic differentials between good classes. By the Dichotomy Principle
there can be no evil classes above the 1/3-line. This reduces the possibilities and allows us to make easy arguments. The differentials established in this region can then be ``pulled back'' using Lemma~\ref{lem:v1torsion}. Further, note that Lemma~\ref{lem:v1torsion} implies that the relevant classes must survive long enough for the differentials stated below to occur.


\begin{prop}\label{prop:blueoranage}
There is a $v_1^4$--periodic family of differentials starting with
\[d_2( {\color{blue}(16,3:1)}) = {\color{orange}(15,4:3)}\]
and a $v_1^8$--periodic family starting with
\[d_2( {\color{blue}(23,6:1)} ) = {\color{orange}(22,7:3)} .\]
\end{prop}
\begin{proof}
$\Ex$ is zero in degree $(38,15)$. Therefore, the class ${\color{orange}(38,15:3)}$ cannot survive. This forces the differential $d_2( {\color{blue}(39,14:1)}) = {\color{orange}(38,15:3)}$. By $h_1$--linearity, this implies that $d_2( {\color{blue}(40,15:1)}) = {\color{orange}(39,16:3)}$. The classes $ {\color{blue}(40,15:1)}$ and $ {\color{orange}(39,16:3)}$ are $v_1^4$--periodic on the $E_{1+\epsilon}$--term of the AKSS. By Lemma~\ref{lem:v14}, this differential is $v_1^4$--periodic (and, therefore, $v_1^8$--periodic). The classes $ {\color{blue}(39,14:1)}$ and ${\color{orange}(38,15:3)}$ are $v_1^8$--periodic on the $E_{1+\epsilon}$--term. The reoccurrence of $d_2( {\color{blue}(40,15:1)}) = {\color{orange}(39,16:3)}$ and $h_1$--linearity force the reoccurrence of $d_2( {\color{blue}(39,14:1)}) = {\color{orange}(38,15:3)}$. 
Therefore, from this point on, the $v_1^4$, respectively $v_1^8$, multiples of these differentials always occur. We apply Lemma~\ref{lem:v1torsion} with $N=4$ and appropriate values of $k$ to pull-back these differentials. For example, using $k=1$, the differential $d_2( {\color{blue}(48,19:1)}) = {\color{orange}(47,20:3)}$ implies that $d_2( {\color{blue}(16,3:1)}) = {\color{orange}(15,4:3)}$ and the differential $d_2( {\color{blue}(55,22:1)}) = {\color{orange}(54,23:3)}$ implies that $d_2( {\color{blue}(23,6:1)} ) = {\color{orange}(22,7:3)}$. 
\end{proof}


\begin{prop}\label{prop:blueoranage1}
There is a $v_1^4$--periodic family of differentials starting with
\[d_2( {\color{blue}(32,3:1)}) = {\color{orange}(31,4:3)}.\]
\end{prop}
\begin{proof}
Since $\Ex$ is zero in degree $(71,24)$, we must have $d_2( {\color{blue}(72,23:1)}) = {\color{orange}(71,24:3)}$. The classes are $v_1^4$--periodic, so as in Proposition~\ref{prop:blueoranage}, we apply Lemma~\ref{lem:v14} and Lemma~\ref{lem:v1torsion} with $N=5$ and appropriate values of $k$ to conclude the result.
\end{proof}


\begin{prop}\label{prop:oragnecyan}
There is a $v_1^4$--periodic family of differentials starting with
\[d_2( {\color{orange}(29,6:3)}) = {\color{cyan}(28,7:5)} \ \ \ \text{and} \ \ \ d_2( {\color{orange}(30,7:3)}) = {\color{cyan}(29,8:5)}  .\]
\end{prop}
\begin{proof}
$\Ex$ is zero in degree $(44,15)$. This forces the differential $d_2( {\color{orange}(45,14:3)}) = {\color{cyan}(44,15:5)}$. Then, $h_1$--linearity forces the differential $d_2( {\color{orange}(46,15:3)}) = {\color{cyan}(45,16:5)}$. These classes are $v_1^4$--periodic so Lemma~\ref{lem:v14} implies the periodicity of the differential. Finally,  Lemma~\ref{lem:v1torsion} with $N=4$ allows us to pull back the differentials.
\end{proof}

\begin{prop}\label{prop:cyanmagenta}
There is a $v_1^4$--periodic family of differentials starting with
\[d_2( {\color{cyan}(35,6:5)}) = {\color{darkmagenta}(34,7:7)} \ \ \ \text{and} \ \ \ d_2( {\color{cyan}(36,7:5)}) = {\color{darkmagenta}(35,8:7) }  .\]
\end{prop}
\begin{proof}
This is the same argument as in the proof of Proposition~\ref{prop:oragnecyan}, using the fact that
$\Ex$ is zero in degree $(66,23)$.
\end{proof}


\begin{prop}\label{prop:greenred}
There are $v_1^4$--periodic families of differentials starting with
\[d_2( {\color{red}(23,5:2)}) = {\color{limegreen}(22,6:4)} \ \ \ \text{and} \ \ \  d_3( {\color{red}(24,6:2)} ) = {\color{cyan}(23,7:5)}.\]
\end{prop}

\begin{proof}
$\Ex$ is zero in degree $(38,14)$. Therefore, ${\color{limegreen}(38,14:4)}$ cannot survive and this forces the differential $d_2( {\color{red}(39,13:2)}) = {\color{limegreen}(38,14:4)}$.
As in Figure~\ref{fig:AA0}, let $x =  {\color{limegreen}(36,12:4)}$, $\partial y =  {\color{limegreen}(37,13:4)}$, $\partial h_1y =  {\color{limegreen}(38,14:4)}$, $\partial w = {\color{red}(39,13:2)}$ and $\partial'z = {\color{cyan}(39,13:5)}$ and note that $\partial'z = h_2x$.
The differential
$d_2( \partial w) = \partial h_1 y $ having been established, part (2)  of Theorem~\ref{thm:AA0} implies that $d_2(w) = h_1y$. By $h_1$--linearity, $d_2(h_1w) = h_1^2 y = v_0^2 z$. Then part (1) of Theorem~\ref{thm:AA0} implies that $d_3(\partial h_1w) = \partial' v_0^2 z = v_0^2 h_2x$.

Note that once the $d_2$--differential is established, the $d_3$--differential is a direct consequence of Theorem~\ref{thm:AA0}. By Lemma~\ref{lem:v14}, the $d_2$-differential is $v_1^4$--periodic, and hence, we have the same periodicity for the $d_3$--differential.

Finally, we use Lemma~\ref{lem:v1torsion} with $N=4$ to pull back these differentials. 
\end{proof}



\begin{figure}
\includegraphics[height=0.5\textheight]{AA0_app.pdf}
\caption{An application of Theorem~\ref{thm:AA0}: a $d_2$--differential in the AKSS for $\AA0$ (bottom) implies a $d_3$--differential in the AKSS for $\FF_2$ (top).}\label{fig:AA0}
\end{figure}

\begin{prop}
There are $v_1^4$--periodic families of differentials starting with
\[d_2( {\color{limegreen}(29,5:4)}) = {\color{lavenderrose}(28,6:6)} \ \ \ \text{and} \ \ \  d_3( {\color{limegreen}(30,6:4)}) = {\color{darkmagenta}(29,7:7)} .\]
\end{prop}
\begin{proof}
This is the same argument as in Proposition~\ref{prop:greenred}, starting with the fact that $\Ex$ is zero in degree $(52,18)$, which forces the $d_2$--differential $d_2( {\color{limegreen}(53,17:4)}) = {\color{lavenderrose}(52,18:6)}$.
\end{proof}


\begin{prop}
There are $v_1^4$--periodic families of differentials starting with
\[d_2( {\color{red}(39,5:2)}) = {\color{limegreen}(38,6:4)} \ \ \ \text{and} \ \ \  d_3( {\color{red}(40,6:2)} ) = {\color{cyan}(39,7:5)}.\]
\end{prop}
\begin{proof}
This is the same argument as above, starting with the fact that $\Ex$ is zero in degree $(78,26)$, so that $d_2( {\color{red}(79,25:2)}) = {\color{limegreen}(78,26:4)}$. When applying Lemma~\ref{lem:v1torsion}, we use $N=5$.
\end{proof}


\begin{prop}\label{prop:pinkyellow}
There are $v_1^4$--periodic families of differentials starting with
\[d_2( {\color{lavenderrose}(43,9:6)}) = {\color{goldenpoppy}(42,10:8)} \ \ \ \text{and} \ \ \  d_3( {\color{lavenderrose}(44,10:6)}) = {\color{seagreen}(43,11:9)} .\]
\end{prop}
\begin{proof}
This is the same argument as in Proposition~\ref{prop:greenred}, starting with the fact that $\Ex$ is zero in degree $(66,22)$, which forces the $d_2$--differential $d_2({\color{lavenderrose}(67,21:6)}) = {\color{goldenpoppy}(66,22:8)}$.
\end{proof}


\begin{prop}\label{prop:blueyellow}
There is a $v_1^{4}$--periodic family of differentials starting with
\[d_7( {\color{blue}(41,8:1)}) = {\color{goldenpoppy}(40,9:8)} .\]
\end{prop}
\begin{proof}
These classes are $v_1^4$--periodic on the $E_{1+\epsilon}$--term of the AKSS, but they lie in the $v_1^{16}$--periodic region of $\Ex$. So we proceed with appropriate care.

The fact that $\Ex$ is zero in degrees $(64,21)$, $(72,25)$, $(80,29)$ and $(88,33)$ forces differentials
\begin{align*}
d_7( {\color{blue}(65,20:1)}) &= {\color{goldenpoppy}(64,21:8)} &
d_7( {\color{blue}(73,24:1)}) &= {\color{goldenpoppy}(72,25:8)} \\
d_7( {\color{blue}(81,28:1)}) &= {\color{goldenpoppy}(80,29:8)} &
d_7( {\color{blue}(89,32:1)}) &= {\color{goldenpoppy}(88,33: 8)}.
\end{align*}
We are in a range where the AKSS for $\Ext_{A(4)_*}(\FF_2)$ is isomorphic to that of $\Ext_{A_*}(\FF_2)$. Since $v_1^{16} \in
\Ext_{A(4)_*}(\FF_2)$, this differential is $v_1^{16}$--linear in the AKSS for $\Ext_{A(4)_*}(\FF_2)$, and thus, also in that of $\Ext_{A_*}(\FF_2)$. Using  Lemma~\ref{lem:v1torsion}, we can pull back the differentials as claimed.
\end{proof}


\begin{prop}\label{prop:orangeyellowred}
There is a $v_1^{4}$--periodic family of differentials starting with
\[d_5( {\color{orange}(41,9:3)}) = {\color{goldenpoppy}(40,10:8)} .\]
\end{prop}


\begin{proof}
This is an argument similar to Proposition~\ref{prop:blueyellow}.
$\Ex$ is zero in degrees $(65,21)$, $(73,25)$, $(81,29)$ and $(89,33)$. For degree reasons, the only way the AKSS can realize this is if
\begin{align*}
d_5( {\color{orange}(65,21:3)}) &= {\color{goldenpoppy}(64,22:8)} & d_6( {\color{red}(66,20:2)}) &= {\color{goldenpoppy}(65,21:8)} \\
d_5( {\color{orange}(73,25:3)}) &= {\color{goldenpoppy}(72,26:8)} & d_6( {\color{red}(74,24:2)}) &= {\color{goldenpoppy}(73,25:8)} \\
d_5( {\color{orange}(81,29:3)}) &= {\color{goldenpoppy}(80,30:8)} & d_6( {\color{red}(82,28:2)}) &= {\color{goldenpoppy}(81,29:8)} \\
d_5( {\color{orange}(89,33:3)}) &= {\color{goldenpoppy}(88,34:8)} & d_6( {\color{red}(90,32:2)}) &= {\color{goldenpoppy}(89,33:8)}.
\end{align*}
Now apply Lemma~\ref{lem:v1torsion} with $N=4$.
\end{proof}


\begin{rmk}
We will see that the only differentials between good classes which have not been accounted for in our range are
\begin{align*}
d_5( {\color{red}(42,7:2}) &= {\color{darkmagenta}(41,8:7)}  \\
d_6( {\color{red}(42,8:2)}) &= {\color{goldenpoppy}(41,9:8)} .
\end{align*}
The same methods as above using $v_1^{32}$--periodicity would apply, but would require studying $\Ex$ in the stems $\geq 100$ so we have decided to use direct arguments (see Proposition~\ref{prop:exception}).
\end{rmk}




\subsection*{The calm before the storm : Stems 0-32}
In the following subsections, we turn to differentials involving evil classes and the few good differentials our previous analysis missed. We make extensive use of Figure~\ref{fig:v1Extchart}.




\ \\
\noindent
{\bf Stem 0-14}

\begin{prop}
There are no non-trivial differentials $d_r$ for $r \geq 1+\epsilon$ in the AKSS in the range $0 \leq t-(s+n) \leq 14$. In this range, $  \E{akss}{alg}^{n+\epsilon,s,t}_{1+\epsilon} =0$ and
\[H^{n,s,t}(\mc{C}_{alg}) \cong  \E{akss}{alg}^{n,s,t}_{1+\epsilon} \cong  \E{akss}{alg}^{n,s,t}_{\infty} .\]
\end{prop}
\begin{proof}
From Figure~\ref{fig:v1Extchart}, all classes in $\Ext_{A_*}^{s,t}(\FF_2)$ for $0\leq t-s\leq 15$ are detected by good and there is a bijection between $H^{n,s,t}(\mc{C}_{alg})$ and $\Ext_{A(2)_*}^{s,t}(\FF_2)$ in this range.
\end{proof}


\ \\
\noindent
{\bf Stems 15-21}

\begin{prop}
There are differentials
\begin{align*}
d_{1+\epsilon}({\color{red}(16,2:2)}) &= {\color{orange}(15,3:3)^{ev} } , &
d_{2+\epsilon}({\color{red}(18,3:2)})&={\color{limegreen}(17,4:4)^{ev}}.\end{align*}
\end{prop}
\begin{proof}
The only class in $\Ext_{A_*}(\FF_2)$ in degree $(16,2)$ and $(18,3)$ are detected by evil, therefore, ${\color{red}(16,2:2)}$ and ${\color{red}(18,3:2)}$ cannot survive, forcing these differentials.
\end{proof}



\begin{prop}
There are differentials
\begin{align*}
d_{1+\epsilon}({\color{orange}(21,3:3)})  &= {\color{limegreen}(20,4:4)^{ev}} , & d_{2+\epsilon}({\color{orange}(21,4:3)})  &= {\color{cyan}(20,5:5)^{ev}}.\end{align*}
\end{prop}

\begin{proof}
It follows from Figure~\ref{fig:v1Extchart} that $h_2^2h_4$ is detected by $\color{orange}(21,3:3)^{ev}$. All classes in $\Ex$ in degree $(21,3)$ are accounted for and $\Ex$ is zero in $(21,4)$. Both ${\color{orange}(21,3:3)}$ and ${\color{orange}(21,4:3)}$ must die, and the only possibility is for them to kill evil.
\end{proof}


\ \\
\noindent
{\bf Stem 22-25}



\begin{prop}\label{prop:stem22_1}
There are differentials
\begin{align*}
d_{1+\epsilon}({\color{orange}(22,3:3)}) &= {\color{limegreen}(21,4:4)^{ev}} , & d_{2+\epsilon}({\color{orange}(23,4;3)}) &= {\color{cyan}(22,5;5)^{ev}}.\end{align*}
\end{prop}
\begin{proof}
The first differential follows from the fact that ${\Ex}=0$ in bidegree $(22,3)$. The class $h_4c_0$ is detected by ${\color{limegreen}(23,4:4)^{ev}}$.  It follows that all classes of $\Ex$ in $(23,4)$ have been accounted for, and this forces the differential on ${\color{orange}(23,4;3)}$.
\end{proof}


\begin{prop}
There are differentials
\begin{align*}
d_{1+\epsilon}( {\color{red}(24,2;2)})&={\color{orange}(23,3:3)^{ev}}, & d_{1+\epsilon}( {\color{limegreen}(24,4;4)})&={\color{cyan}(23,5:5)^{ev}}, \\
d_{2+\epsilon}({\color{limegreen}(24,5;4)} )&={\color{lavenderrose}(23,6:6)^{ev}}, &
d_{1+\epsilon}( {\color{orange}(25,3;3)})&={\color{limegreen}(24,4:4)^{ev}}.
\end{align*}
\end{prop}
\begin{proof}
After taking into account the good differentials and the restrictions imposed by Figure~\ref{fig:v1Extchart} all classes in $\Ex$ in stem $24$ and $25$ have been accounted for. These are the only possibilities left.
\end{proof}



\ \\
\noindent
{\bf Stem 26-27}

\begin{prop}
There are no non-trivial differentials $d_r$, $r \geq 1+\epsilon$ in the AKSS with source or target satisfying $t-(s+n) =26, 27$. In these stems, $\E{akss}{alg}^{n+\epsilon,s,t}_{1+\epsilon} =0$ and
$H^{n,s,t}(\mc{C}_{alg}) \cong   \E{akss}{alg}^{n,s,t}_{\infty} $.
\end{prop}


\ \\
\noindent
{\bf Stem 28-32}


\begin{prop}
There are differentials
\begin{align*}
d_{3+\epsilon}({\color{red}(30,4:2)}) &= {\color{cyan}(29,5:5)^{ev}} , &
d_{4+\epsilon}({\color{red}(30,5:2)}) &= {\color{lavenderrose}(29,6:6)^{ev}}.\end{align*}
\end{prop}
\begin{proof}
By Figure~\ref{fig:v1Extchart}, both $v_0^2h_4^2$ and $v_0^3h_4^2$ are detected by evil classes ${\color{limegreen}(30,4:4)^{ev}}$ and ${\color{cyan}(30,5:5)^{ev}}$ respectively. This implies that ${\color{red}(30,4:2)}$ and ${\color{red}(30,5:2)}$ do not survive. Taking into account the good differentials already established, this is the only possibility.
\end{proof}

\begin{prop}
$v_0^sh_5$ for $0\leq s \leq 15$ are detected by ${\color{blue}(31, s+1 : 1)}$.
\end{prop}
\begin{proof}
By Figure~\ref{fig:v1Extchart}, $h_5$ is good so must be detected by ${\color{blue}(31, 1 : 1)}$. By $v_0$--linearity, the whole tower consists of permanent cycles. For  degree reasons they cannot be targets of differentials and the claim follows.
\end{proof}

\begin{prop}
There are differentials
\begin{align*}
d_{4+\epsilon}({\color{red}(31,5:2)}) &= {\color{lavenderrose}(30,6:6)^{ev}}, & d_{1+\epsilon}({\color{red}(32,2 : 2)})&={\color{orange}(31,3 : 3)^{ev}_1}.\end{align*}
\end{prop}
\begin{proof}
Since $n$ is detected by an evil class, all elements of $\Ex$ in degrees $(31,5)$ have been accounted for, forcing the first differential. Since $h_1h_5$ is detected by evil, ${\color{red}(32,2 : 2)}$ cannot survive. The $d_{1+\epsilon}$ is the only possibility.
\end{proof}


\subsection*{The proliferation of evil: Stems 33-42}
The developing phenomena in the remaining stems is that all good classes of low Adams filtration die killing evil classes, and the non-zero elements of $\Ex$ are detected by evil.

\ \\
\noindent
{\bf Stems 33-34}


\begin{prop}
There are differentials
\begin{align*}d_{4+\epsilon}({\color{orange}(33,6 : 3)})&={\color{darkmagenta}(32, 7 : 7)^{ev}}  \\
d_{3+\epsilon}({\color{orange}(33,5 : 3)})&={\color{lavenderrose}(32, 6 : 6)^{ev}} \\
 d_{4+\epsilon}({\color{blue}(33,4 : 1)})&={\color{cyan}(32, 5 :5)^{ev}}.
\end{align*}
\end{prop}

\begin{proof}The class
${\color{orange}(33,7 : 3)}$ detects $h_1q$ and, in $\Ex$, it is not divisible by $v_0$. Hence, ${\color{orange}(33,6 : 3)}$ and ${\color{orange}(33,5 : 3)}$ cannot survive and these differentials are the only possibilities.
No element of $\Ex$ in $(33,4)$ is detected by a good class. This forces the $d_{4+\epsilon}$--differential.
\end{proof}


\begin{prop}
There are differentials
\begin{align*}
d_{2+\epsilon}({\color{red}(34,3 : 2)})&={\color{limegreen}(33, 4 : 4)_1^{ev} },\\
d_{3+\epsilon}({\color{red}(34,4 : 2)})&={\color{cyan}(33, 5 : 5)_1^{ev} }, \\
d_{4+\epsilon}({\color{red}(34,5 : 2)})&={\color{lavenderrose}(33, 6 : 6)^{ev} }.
\end{align*}
\end{prop}
\begin{proof}
$\Ex$ in degrees $(34,3)$, $(34,4)$ and $(34,5)$ is either zero, or its elements are detected by evil classes. This forces these differentials.
\end{proof}


\ \\
\noindent
{\bf Stems 35-43}

\begin{prop}\label{prop:detectx}
The class $x$ is detected by ${\color{orange}(37,5 : 3)}$.
\end{prop}
\begin{proof}
Both $x$ and $v_0x$ are detected by good classes, and for $v_0x$, the only possibility is  ${\color{orange}(37,6 : 3)}$. Since there cannot be an exotic $v_0$-extension from $ {\color{limegreen}(37,5 : 4)} $ to ${\color{orange}(37,6 : 3)}$, we must have that $x$ is detected by ${\color{orange}(37,5 : 3)}$.
\end{proof}


\begin{prop}
There are differentials
\begin{align*}
d_{1+\epsilon}({\color{limegreen}(36,4 : 4)}) &={\color{cyan}(35,5 : 5)_2^{ev}}, \\
d_{2+\epsilon}({\color{limegreen}(36,5 : 4)}) &={\color{lavenderrose}(35,6 : 6)^{ev}}, \\
d_{3+\epsilon}({\color{limegreen}(36,6 : 4)}) &={\color{darkmagenta}(35,7 : 7)_1^{ev}}.
\end{align*}
\end{prop}
\begin{proof}
$\Ex$ in degrees $(36,4 )$ and $(36,5)$ is zero and the only non-zero element of $\Ex$ in $(36,6)$ is detected by an evil class. For degree reasons, we must have the first three differentials.
\end{proof}



\begin{prop}
For $i = 1,2$, there are differentials
\begin{align*}
d_{1+\epsilon}({\color{orange}(37,3 : 3)_i}) &= {\color{limegreen}(36,4 : 4)_i^{ev}}, \\
 d_{2+\epsilon}({\color{orange}(37,4 : 3)_i}) &= {\color{cyan}(36,5 : 5)_i^{ev}}.
 \end{align*}
\end{prop}
\begin{proof}
In these bi-degrees, $\Ex$ is zero or detected by evil. For degree reasons, we must have these differentials.
\end{proof}

\begin{prop}\label{prop:lotsofevil}
There are differentials
\begin{align*}
d_{1+\epsilon}({\color{red}(38,2 : 2)}) &= {\color{orange}(37,3 : 3)^{ev}_2},  &d_{2+\epsilon}({\color{red}(38,3 : 2)}) &= {\color{limegreen}(37,4 : 4)_1^{ev}}, \\
d_{3+\epsilon}({\color{red}(38,4: 2)}) &= {\color{cyan}(37,5 : 5)^{ev}} , & d_{4+\epsilon}({\color{red}(38,5: 2)}) &= {\color{lavenderrose}(37,6: 6)^{ev}}
\end{align*}
and, for $i=1,2$, differentials
\begin{align*}
d_{1+\epsilon}(\gdm{orange}{38}{3}{3}{i}) &=  \evm{limegreen}{38}{4}{i+1}, & d_{2+\epsilon}(\gdm{orange}{39}{4}{3}{i}) &=  \evm{cyan}{38}{5}{i+1}.
\end{align*}
\end{prop}
\begin{proof}
All classes of $\Ex$ in the stems $38$ and $39$ for $y \leq 5$ are detected by evil. No good class can survive. For both families, the $d_{1+\epsilon}$'s and the $d_{2+\epsilon}$'s are the only possibilities. By $v_0$--linearity, ${\color{red}(38,4: 2)}$ is a $d_{2+\epsilon}$--cycle, hence it cannot kill ${\color{limegreen}(37,5 : 4)}$. Therefore, we must have the claimed $d_{3+\epsilon}$. The $d_{4+\epsilon}$ is also the only possibility.
\end{proof}


\begin{prop}
There are differentials
\begin{align*}
d_{1+\epsilon}({\color{lavenderrose}(36,6 : 6)}) &={\color{darkmagenta}(35,7 : 7)_2^{ev}}, & d_{2+\epsilon}({\color{limegreen}(37,5 : 4)}) &={\color{lavenderrose}(36,6 :6)_2^{ev}}.
\end{align*}
\end{prop}
\begin{proof}
The class cannot ${\color{lavenderrose}(36,6 : 6)}$ survive since the only non-zero element of $\Ex$ in $(36,6)$ is detected by evil. The only other possibility is that $d_{2}({\color{limegreen}(37,5 : 4)}) = {\color{lavenderrose}(36,6 : 6)}$. However, this would imply that $d_{2}({\color{limegreen}(45,9 : 4)}) = {\color{lavenderrose}(44,10 : 6)}$, contradicting Proposition~\ref{prop:pinkyellow}.

For the $d_{2+\epsilon}$--differential, by Proposition~\ref{prop:detectx}, $x$ has already been accounted for, hence, 
${\color{limegreen}(37,5 : 4)}$ cannot survive. Proposition~\ref{prop:lotsofevil} eliminates the only other possibility.
\end{proof}



\begin{prop}\label{prop:cyan39}
There are differentials
\begin{align*}
d_{1+\epsilon}(\gdr{40}{2}{2}{red}) &= \evm{orange}{39}{3}{2}, & d_{1+\epsilon}(\gdr{41}{3}{3}{orange} )&= \evm{limegreen}{40}{4}{3} \end{align*}
and, for $i=1,2$,
\begin{align*}
d_{1+\epsilon}(\gdr{39}{5}{5}{cyan}) &= \evm{lavenderrose}{38}{6}{2}, & d_{1+\epsilon}( \gdm{limegreen}{40}{4}{4}{i}) &= \evm{cyan}{39}{5}{i+1}\\
 d_{2+\epsilon}(\gdr{39}{6}{5}{cyan}) &= \ev{darkmagenta}{38}{7}, & d_{2+\epsilon}( \gdm{limegreen}{40}{5}{4}{i}) &= \evm{lavenderrose}{39}{6}{i}.
\end{align*}
\end{prop}
\begin{proof}
$\Ex$ is zero in $(40,2)$ and this justifies the first differential.
$\Ex$ in $(39,5)$ and $(39,6)$ is either detected by evil or zero. Hence, the sources of these differentials cannot survive. Taking $\bo$--filtration into account, this is the only possibility. The other differentials are justified in a similar way.
\end{proof}


\begin{prop}
There are differentials
\begin{align*}
d_{1+\epsilon}(\gdr{41}{7}{7}{darkmagenta}) &=\ev{goldenpoppy}{40}{8}, & d_{1+\epsilon}(\gdr{42}{6}{6}{lavenderrose}) &= \evm{darkmagenta}{41}{7}{2},\\
d_{1+\epsilon}(\gdr{42}{7}{7}{darkmagenta}) &=\ev{goldenpoppy}{41}{8}, &
d_{2+\epsilon}(\gdr{43}{8}{7}{darkmagenta}) &=\ev{seagreen}{42}{9} .
 \end{align*}
\end{prop}
\begin{proof}
The only classes in $\Ex$ in degree $(41,7)$, $(42,7)$ and $(42,6)$ are detected by evil. Because of the $\bo$--filtrations, this forces the three $d_{1+\epsilon}$-differentials.

$\Ex$ is zero in degree $(43,8)$. Hence the source of the $d_{2+\epsilon}$ cannot survive. The only other possibility is that $d_{2}(\gdr{44}{7}{5}{cyan}) =\gdr{42}{7}{7}{darkmagenta}$. However, this differential would be $v_1^4$--periodic, and this would contradict the fact that there is a non-zero element in $\Ex$ in degree $(83,28)$.
\end{proof}


\begin{prop}\label{prop:hard}
There is a differential
\[d_{2+\epsilon}(\gdr{38}{7}{6}{lavenderrose}) =\ev{goldenpoppy}{37}{8}. \]
\end{prop}

\begin{proof}
In bidegree $(38,7)$, the element $v_0y$ in $\Ex$, is either detected by $\gdr{38}{7}{6}{lavenderrose}$ or by $\gdr{38}{7}{3}{orange}$. Therefore, one of them supports a nontrivial differential that kills an evil class, and the other one survives.



Suppose that $\gdr{38}{7}{6}{lavenderrose}$ survives in the AKSS. In bidegree $(46,11)$, $d_0l$ is the only element in $\Ex$. Further, the only non-zero element in $\Ex$ in degree $(78,27)$ is $lP^4d_0$. We are in a region where $\Ex \cong \Ext_{A(4)_*}$ and hence $lP^4d_0$ maps to $v_1^{16} d_0l$ in the latter. In particular, $d_0l$ is $v_1$-periodic. This class is above the 1/3-line, and the only element in the AKSS left to detect $lP^4d_0$ is $\gdr{78}{27}{6}{lavenderrose}$. By the Dichotomy Principle, $d_0l$ is good and so must be detected by $\gdr{46}{11}{6}{lavenderrose}$. Finally, in $\E{bo}{alg}^{*,*,*}_{1}$, $v_1^4 \gdr{38}{7}{6}{lavenderrose} = \gdr{46}{11}{6}{lavenderrose}$, which implies that $Pyv_0 = d_0l$ (with zero indeterminacy). However, $Py = 0$ since this bidegree is zero in $\Ex$, a contradiction.
\end{proof}


\begin{prop}\label{prop:exception}
There are differentials
\begin{align*}
d_6( {\color{red}(42,8:2)}) &= {\color{goldenpoppy}(41,9:8)}, & d_5( {\color{red}(42,7:2}) &= {\color{darkmagenta}(41,8:7)} .
\end{align*}
\end{prop}
\begin{proof}
The only class in $\Ex$ in degree $(40,8)$ is detected by ${\color{goldenpoppy}(40,8:8)}$ so by $h_1$--linearity, ${\color{goldenpoppy}(41,9:8)}$ is a permanent cycle. However, $\Ex$ in degree $(41,9)$ is zero. Therefore, the class ${\color{goldenpoppy}(41,9:8)}$ must be hit by a differential. The only possibility is the $d_6$--differential. 

The only class in $\Ex$ in degree $(42,7)$ is detected by evil. Therefore, $\gdr{42}{7}{2}{red}$ must support a differential. If it kills evil, that differential would be a $d_{6+\epsilon}$--differential. We have $v_0\gdr{42}{7}{2}{red} = \gdr{42}{8}{2}{red} $, so such a $d_{6+\epsilon}$ would imply that $d_{6}(\gdr{42}{8}{2}{red} )=0$, contradicting what we have just shown.
The only possibility is this $d_5$--differential.
\end{proof}





