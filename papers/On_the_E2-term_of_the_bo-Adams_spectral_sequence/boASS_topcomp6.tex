% !TEX root = boASS6.tex

\section{Computation of the topological $\bo$-resolution}\label{sec:topcomp}


In this section, we deduce the differentials in the topological $\bo$-based Adams spectral sequence from known computations of the stable homotopy groups of spheres (see \cite{Isaksen} for example). The computation is depicted in Figure~\ref{fig:boASSchart}. We find that certain products in $\pi_\ast$, which are nontrivial extensions in the classical Adams spectral sequence, are products in $E_2$-page of the $\bo$-based Adams spectral sequence, and so they are not exotic extensions. 

\begin{figure}
\includegraphics[angle=90,height=0.9\textheight]{AKSS-top-11_11_2018-cropped.pdf}
\caption{The $\bo$ Adams Spectral Sequence. The horizontal axis denotes the topological degree and the vertical axis (and the color) denotes the $\bo$-filtration. 
The $\blacksquare$  denotes $\mathbb{Z}_{(2)}$.  A $\bullet$ denotes a copy of $\mathbb{Z}/2$ detected by a good class and $\circ$ a copy of $\mathbb{Z}/2$ detected by an evil class. 
A \circled{$\bullet$} denotes a copy $\mathbb{Z}/4$, a \circled{\circled{$\bullet$}} a copy of $\mathbb{Z}/8$ etc.. A line between classes which increases topological degree by one is multiplication by $\eta$ and one which increases topological degree by $3$ is multiplication by $\nu$. Gray lines between gray classes show the differentials.
%The color denotes the $\bo$-filtration as specified in Figure~\ref{fig:colorbo}.
}\label{fig:boASSchart}
\end{figure}


\begin{rem}\label{rem:filt}
In the following computation, we will use the fact that the map from $\bo$ to $H\mathbb{F}_2$ induces a map of spectral sequences
\[\E{bo}{}^{*,*}_2 \to \Ex^{*,*}.\]
In particular, if an element $x \in \pi_*$ is detected in $\Ex$ by a class of Adams filtration $s$, then it must be detected in $\E{bo}{}^{*,*}_2 $ by a class of $\bo$--filtration $n \leq s$. 
\end{rem}

It is straightforward to see that there are no differentials up to stem 28.

\ \\
\noindent
{\bf Stems 29-31}

\noindent

\begin{prop}
The element $\gdr{30}{2}{2}{red}$ survives and detects $\theta_4$.
\end{prop}

\begin{proof}
The element $\gdr{30}{2}{2}{red}$ maps to $h_4^2$ in $\Ex$, which detects $\theta_4$. Since there are no elements with $\bo$-filtration lower than $\gdr{30}{2}{2}{red}$, by Remark \ref{rem:filt}, $\gdr{30}{2}{2}{red}$ survives and detects $\theta_4$.
\end{proof}


\begin{prop} 
There are differentials
$d_{2+\epsilon}(\gdr{30}{3}{3}{orange}) =  \evr{29}{5}{cyan}$, $d_2(\evr{30}{4}{limegreen}) =  \evr{29}{6}{lavenderrose}$ and $d_{2-\epsilon}(\evr{30}{5}{cyan}) =  \gdr{29}{7}{7}{darkmagenta}$.
\end{prop}

\begin{proof}
Since $\pi_{29}=0$, none of the three elements in the 29-stem can survive. This forces these $d_2$--differentials.
\end{proof}

\begin{prop}
One of $\evrm{31}{3}{orange}{1}$ and $\evrm{31}{3}{orange}{2}$ supports a $d_3$ differential that kills $\evr{30}{6}{lavenderrose}$, the other one survives and detects $\eta\theta_4$. Without loss of generality, we adopt the convention that $d_3(\evrm{31}{3}{orange}{1}) = \evr{30}{6}{lavenderrose}$, and $\evrm{31}{3}{orange}{2}$ survives and detects $\eta\theta_4$.
\end{prop}


\begin{proof}
Since $\pi_{30}=\mathbb{Z}/2$, $\evr{30}{6}{lavenderrose}$ must be hit by a differential. The only possibility is that 
it be hit by
one of $\evrm{31}{3}{orange}{1}$ and $\evrm{31}{3}{orange}{2}$ by a $d_3$--differential. The other one must survive and detect $\eta\theta_4$, since the Adams filtration of $\eta\theta_4$ is $s=3$.
\end{proof}

\begin{cor}
The element $\evr{31}{5}{cyan}$ survives and detects $\{n\}$.
\end{cor}


\ \\
\noindent
{\bf Stems 32-34}

\noindent

\begin{prop}
The elements $\evr{32}{2}{red}$, $\evr{32}{4}{limegreen}$, $\evr{32}{5}{cyan}$ and $\evr{33}{3}{orange}$ survive and detect $\eta_5$, $\{d_1\}$, $\{q\}$ and $\eta\eta_5$ respectively. One of the elements $\evrm{33}{4}{limegreen}{1}$ and $\evrm{33}{4}{limegreen}{2}$, say $\evrm{33}{4}{limegreen}{1}$, survives and detects $\nu\theta_4$. (Note that in the classical Adams spectral sequence, $p$ detects $\nu\theta_4$.)
\end{prop}

\begin{proof}
This follows from Remark~\ref{rem:filt}.
\end{proof}



\begin{prop}There is a differential
$d_2(\evrm{33}{4}{limegreen}{2})=\evr{32}{6}{lavenderrose}$.
One of the elements $\evrm{33}{5}{cyan}{1}$ and $\evrm{33}{5}{cyan}{2}$, say $\evrm{33}{5}{cyan}{1}$, supports a $d_2$ differential that kills $\evr{32}{7}{darkmagenta}$. That is, $d_2(\evrm{33}{5}{cyan}{1})=\evr{32}{7}{darkmagenta}$.
\end{prop}

\begin{proof}
Since all classes in $\pi_{32}$ are accounted for, this is the only possibility to kill $\evr{32}{6}{lavenderrose}$ and $\evr{32}{7}{darkmagenta}$.
\end{proof}


\begin{prop}
There is a differential $d_3(\evr{34}{2}{red})=\evrm{33}{5}{cyan}{2}$ and $\evr{33}{6}{lavenderrose}$ survives to detect $\eta\{q\}$.
\end{prop}
\begin{proof}
Since $h_2h_5$ is detected by $\evr{34}{2}{red}$ and $h_0 p$ by $\evrm{33}{5}{cyan}{2}$ in the algebraic $\bo$ spectral sequence, the $d_3$ differential $d_3(h_2h_5) = h_0p$ in the classical Adams spectral sequence implies the first claim and $\evr{33}{6}{lavenderrose}$ is the only class left to detect $\eta\{q\}$. 
\end{proof}



\begin{cor}
The elements $\evr{34}{3}{orange}$, $\evr{34}{4}{limegreen}$, $\evr{34}{6}{lavenderrose}$, $\gdr{34}{7}{7}{darkmagenta}$ survive and detect $\{h_0h_2h_5\}$, $\{h_0^2h_2h_5\}$, $\nu\{n\}$, $\kappa\overline{\kappa}$ respectively. (Note that it is $d_0g$ that detects $\kappa\overline{\kappa}$ in the classical Adams spectral sequence.)
\end{cor}



\ \\
\noindent
{\bf Stems 35-36}

\noindent

\begin{prop}
The element $\evr{35}{5}{cyan}$ survives and detects $\nu\{d_1\}$. One of the elements $\evrm{35}{7}{darkmagenta}{1}$ and $\evrm{35}{7}{darkmagenta}{2}$, say $\evrm{35}{7}{darkmagenta}{1}$, survives and detects $\eta\kappa\overline{\kappa}$.
\end{prop}

\begin{proof}
The first claim follows from Remark~\ref{rem:filt}. Since $\kappa\overline{\kappa}$ is detected by $\gdr{34}{7}{7}{darkmagenta}$ with $\bo$-filtration 7, $\eta\kappa\overline{\kappa}$ has $\bo$-filtration at least 7. (Note that $\eta$ has $\bo$-filtration 0). Therefore, the only possibility is that one of the elements $\evrm{35}{7}{darkmagenta}{1}$ or $\evrm{35}{7}{darkmagenta}{2}$ detects $\eta\kappa\overline{\kappa}$.
\end{proof}

\begin{prop}
$d_2(\evr{36}{4}{limegreen})=\evr{35}{6}{lavenderrose}$.
One of the elements $\evrm{36}{5}{cyan}{1}$ and $\evrm{36}{5}{cyan}{2}$, say $\evrm{36}{5}{cyan}{1}$, supports a $d_2$ differential that kills $\evrm{35}{7}{darkmagenta}{2}$. That is, $d_2(\evrm{36}{5}{cyan}{1})=\evrm{35}{7}{darkmagenta}{2}$.
\end{prop}

\begin{proof}
Since all classes in $\pi_{35}$ are accounted for, this is the only possibility to kill $\evr{35}{6}{lavenderrose}$ and $\evrm{35}{7}{darkmagenta}{2}$.
\end{proof}

\begin{prop}
The element $\evrm{37}{3}{orange}{1}$ survives and detects $\{h_2^2h_5\}$.
\end{prop}

\begin{proof}
This follows from Remark~\ref{rem:filt}.
\end{proof}



\begin{prop}
The element $\evrm{36}{6}{lavenderrose}{1}$ survives and detects $\{t\}$ and there are differentials 
\begin{align*}
d_2(\evrm{37}{3}{orange}{2}) &= \evrm{36}{5}{cyan}{2}, & d_2(\evrm{37}{4}{limegreen}{1}) &= \evrm{36}{6}{lavenderrose}{2}.\end{align*}
\end{prop}


\begin{proof}
The element $\evrm{36}{6}{lavenderrose}{1}$ maps to $t$ in $\Ex$. Since $t$ is not a boundary in the classical Adams spectral sequence, $\evrm{36}{6}{lavenderrose}{1}$ is also not a boundary. Therefore, $\evrm{36}{6}{lavenderrose}{1}$ survives and detects $\{t\}$. The other two differentials are the only possibilities left.
\end{proof}


\ \\
\noindent
{\bf Stems 37-38}

\noindent

\begin{prop}\label{prop:length38}
The elements $\evrm{38}{4}{limegreen}{1}$, $\evrm{38}{5}{cyan}{1}$ survive and detect $\{h_0^2h_3h_5\}$, $\{h_0^3h_3h_5\}$ respectively.
\end{prop}


\begin{proof}
The elements $\evr{38}{2}{red}$, $\evr{38}{3}{orange}$ and $\evrm{38}{4}{limegreen}{2}$ map to $h_3h_5$, $h_0h_3h_5$ and $e_1$ in $\Ex$. Since $h_3h_5$, $h_0h_3h_5$ and $e_1$ 
support non-trivial
differentials 
in the classical Adams spectral sequence, so do $\evr{38}{2}{red}$, $\evr{38}{3}{orange}$ and $\evrm{38}{4}{limegreen}{2}$
in the $\bo$-Adams spectral sequence. By Remark~\ref{rem:filt} and filtration reasons, $\evrm{38}{4}{limegreen}{1}$ and $\evrm{38}{5}{cyan}{1}$ survive and detect $\{h_0^2h_3h_5\}$ and $\{h_0^3h_3h_5\}$ respectively.
\end{proof}

\begin{prop}
The element $\evr{38}{6}{lavenderrose}$ survives and detects $\nu^2\{d_1\}$.
\end{prop}

\begin{proof}
On one hand, $\nu^2\{d_1\}$ has Adams filtration 6, and therefore $\bo$-Adams filtration at most 6. On the other hand, $\nu$ and $\nu\{d_1\}$ have $\bo$-Adams filtration 1 and 5, therefore $\nu^2\{d_1\}$ has $\bo$-Adams filtration at least 6. Therefore, $\nu^2\{d_1\}$ has $\bo$-Adams filtration 6, and the claim then follows from the fact that $\evr{38}{6}{lavenderrose}$ is the only element left in $\bo$-Adams filtration 6 in this stem.
\end{proof}

\begin{prop}\label{prop:seagreend3}
$d_3(\evrm{38}{5}{cyan}{2}) = \evr{37}{8}{seagreen}$. 
\end{prop}

\begin{proof}
In $\pi_{37}$, all classes have Adams filtration at most 5, and therefore $\bo$-Adams filtration at most 5. The target element $\evr{37}{8}{seagreen}$ has $\bo$-Adams filtration 8, therefore must be killed. Since $\evr{38}{6}{lavenderrose}$ survives, by Proposition~\ref{prop:length38}, the only possibility left to kill $\evr{37}{8}{seagreen}$ is an element in $\bo$-Adams filtration 5, say $\evrm{38}{5}{cyan}{2}$.
\end{proof}



\begin{prop}
$d_2(\evrm{39}{3}{orange}{1}) = \evrm{38}{5}{cyan}{3}$.
\end{prop}

\begin{proof}
Every class in $\pi_{38}$ has already been accounted for, therefore $\evrm{38}{5}{cyan}{3}$ must either support a differential or get killed. Suppose it is not killed. Then it can only support a $d_2$ differential that kills $\evr{37}{7}{darkmagenta}$. It follows that the elements $\evr{38}{2}{red}$, $\evr{38}{3}{orange}$, and $\evrm{38}{4}{limegreen}{2}$ kill elements in $\bo$-Adams filtration 6, 5, 4 respectively. This leaves only one element in stem 37. However, $\pi_{37}=\mathbb{Z}/2\oplus\mathbb{Z}/2$, a contradiction. Therefore, we must have $d_2(\evrm{39}{3}{orange}{1}) = \evrm{38}{5}{cyan}{3}$.
\end{proof}

\begin{prop}
The elements $\evrm{39}{3}{orange}{2}$, $\evr{39}{4}{limegreen}$, $\evrm{39}{5}{cyan}{1}$, $\evrm{40}{4}{limegreen}{1}$, $\evrm{40}{4}{limegreen}{2}$ and $\evrm{40}{5}{cyan}{1}$ survive and detect $\sigma\eta_5$, $\{h_5c_0\}$, $\sigma\{d_1\}$, $\eta\sigma\eta_5$, $\{f_1\}$ and $\epsilon\eta_5$ respectively. (Note that $\sigma\eta_5$, $\sigma\{d_1\}$ and $\epsilon\eta_5$ are detected by $h_1h_3h_5$, $h_1e_1$ and $h_1h_5c_0$ in the classical Adams spectral sequence respectively).
\end{prop}

\begin{proof}
This follows from Remark~\ref{rem:filt}.
\end{proof}

\begin{prop}There is a differential
$d_2(\evrm{39}{5}{cyan}{2}) = \evr{38}{7}{darkmagenta}$.
\end{prop}

\begin{proof}
Since all classes in $\pi_{38}$ are accounted for, this is the only possibility left.
\end{proof}

\begin{prop}There is a differential
$d_3(\evrm{38}{4}{limegreen}{2}) = \evr{37}{7}{darkmagenta}$.
\end{prop}

\begin{proof}
Since $\evrm{38}{4}{limegreen}{2}$ maps to $e_1$ in $\Ex$, it must support a $d_2$ or $d_3$ differential. Suppose that $d_2(\evrm{38}{4}{limegreen}{2}) =\evr{37}{6}{lavenderrose}$. This would force a differential $d_4(\evr{38}{3}{orange}) = \evr{37}{7}{darkmagenta}$ since $\evr{37}{7}{darkmagenta}$ must be hit by a differential and this is the only possibility. (The element $\evr{38}{2}{red}$ cannot support a $d_5$ differential, since its image in $\Ex$ supports a $d_4$ differential.) This would imply that $\evr{37}{7}{darkmagenta}$ maps to $ d_4(h_0h_3h_5) = h_0^2x \neq h_2^2n$. This is a contradiction since $\evr{37}{7}{darkmagenta}$ maps to $h_2^2n$ in $\Ex$. Therefore, we must have $d_3(\evrm{38}{4}{limegreen}{2}) = \evr{37}{7}{darkmagenta}$.
\end{proof}

\begin{prop}There is a differential
$d_3(\evr{38}{3}{orange}) = \evr{37}{6}{lavenderrose}$.
\end{prop}

\begin{proof}
The element $\evr{37}{6}{lavenderrose}$ in $\bo$-Adams filtration 6 must be killed, since all classes in $\pi_{37}$ have Adams filtration at most 5. The other possibility to kill it is by a $d_4$ differential: $d_4(\evr{38}{2}{red}) = \evr{37}{6}{lavenderrose}$. This would imply that $\evr{37}{6}{lavenderrose}$ maps to $d_4(h_3h_5) = h_0x$ in $\Ex$, which is not the case since it maps to the target of an algebraic $d_{1+\epsilon}$--differential (see Proposition~\ref{prop:lotsofevil}).
\end{proof}

\begin{prop}There is a differential
$d_3(\evr{38}{2}{red}) = \evr{37}{5}{cyan}$. The element $\evrm{37}{4}{limegreen}{2}$ survives and detects $\sigma\theta_4$. (Note that $\sigma\theta_4$ is detected by $x$ in the classical Adams spectral sequence.)
\end{prop}

\begin{proof}
The other possibility is that $\sigma\theta_4$ is detected by $\evr{37}{5}{cyan}$. Since $\sigma\theta_4$ has Adams filtration 5, this implies that the element $\evr{37}{5}{cyan}$ maps to $x$ in $\Ex$, which would contradict Proposition~\ref{prop:detectx} and Proposition~\ref{prop:lotsofevil}.
\end{proof}


\ \\
\noindent
{\bf Stems 39-42}

\noindent


\begin{prop}
There are differentials
\begin{align*}
d_2(\evr{41}{3}{orange}) &= \evrm{40}{5}{cyan}{2}, & d_2(\evr{41}{4}{limegreen}) &= \evrm{40}{6}{lavenderrose}{1}.
\end{align*} 
The elements  $\evr{41}{5}{cyan}$, $\evrm{40}{6}{lavenderrose}{2}$, $\evr{41}{7}{darkmagenta}$, $\gdr{40}{8}{8}{goldenpoppy}$, $\evr{41}{8}{goldenpoppy}$, $\evr{42}{6}{lavenderrose}$, $\evr{42}{7}{darkmagenta}$ and $\evr{42}{8}{goldenpoppy}$ survive and detect $\{f_1\}$, $\{Ph_1h_5\}$, $\eta\{Ph_1h_5\}$, $\overline{\kappa}^2$, $\eta\overline{\kappa}^2$, $\{Ph_2h_5\}$, $2\{Ph_2h_5\}$ and $4\{Ph_2h_5\}$ respectively.
\end{prop}

\begin{proof}
This is the only possibility left.
\end{proof}

\begin{prop}
The elements $\evrm{39}{6}{lavenderrose}{1}$ and $\evr{39}{7}{darkmagenta}$ survive and detect $\{u\}$ and $\nu\{t\}$. Further, there is a differential $d_2(\evrm{40}{4}{limegreen}{3}) = \evrm{39}{6}{lavenderrose}{2}$.
\end{prop}

\begin{proof}
On one hand, $\nu\{t\}$ has Adams filtration 7, and therefore $\bo$-Adams filtration at most 7. On the other hand, $\nu$ and $\{t\}$ have $\bo$-Adams filtration 1 and 6, therefore $\nu\{t\}$ has $\bo$-Adams filtration at least 7. 
That $\evr{39}{7}{darkmagenta}$ detects $\nu\{t\}$ follows from the fact that this is the only element in $\bo$-Adams filtration 7.
All classes in $\pi_{40}$ are accounted for. Therefore, the element $\evrm{40}{4}{limegreen}{3}$ must support a differential, and this is the only possibility.
\end{proof}

We finish with a few remarks on the $E_\infty$-page.

\begin{rmk}[Stems 8-9]
Recall that $\epsilon \in \pi_8$ is defined to be the unique class with Adams filtration 3. Also recall that we have a relation in $\pi_9$: 
$$\nu^3 = \eta \epsilon + \eta^2\sigma.$$
In $\pi_9$, $\nu^3$ is detected by $\gdr{9}{3}{3}{orange}$, since $\nu^2$ is detected by $\gdr{6}{2}{2}{red}$. It follows from the above relation that $\epsilon + \eta\sigma$ is detected by $\gdr{8}{2}{2}{red}$. Since $\eta\sigma$ is detected by $\gdr{8}{1}{1}{black}$, we have that $\epsilon$ is also detected by $\gdr{8}{1}{1}{black}$. This is interesting since $\epsilon$ and $\eta\sigma$ are detected by different elements in the classical Adams spectral sequence and $\epsilon$ is not divisible by $\eta$.
\end{rmk}


\begin{rmk}[Stem 14]
Recall that $\kappa \in \pi_{14}$ is defined to be the unique class with Adams filtration 4. 

The class $\gdr{14}{2}{2}{red}$ detects both $\sigma^2$ and $\sigma^2+\kappa$, since both $\sigma^2$ and $\sigma^2+\kappa$ have Adams filtration 2, and therefore $\bo$-Adams filtration at most 2. Therefore, $\gdr{14}{3}{3}{orange}$ detects $\kappa$.
\end{rmk}


\begin{rmk}[Stem 22]
The class $\evr{22}{3}{orange}$ detects both $\nu\overline{\sigma}$ and $\nu\overline{\sigma}+\eta^2\overline{\kappa}$, since both $\nu\overline{\sigma}$ and $\nu\overline{\sigma}+\eta^2\overline{\kappa}$ have Adams filtration 4, and therefore $\bo$-Adams filtration at most 4. Therefore, $\evr{22}{5}{cyan}$ detects $\eta^2\overline{\kappa}$.

Recall that in $\pi_{22}$, we have a relation:
$$\eta^2\overline{\kappa} = \epsilon \kappa = (\epsilon + \eta\sigma) \kappa,$$
which is both a nontrivial $\eta$-extension from $\eta\overline{\kappa}$ and a nontrivial $\kappa$-extension from $\epsilon + \eta\sigma$. However, in the $\bo$-Adams spectral sequence, $\eta^2\overline{\kappa}$ has $\bo$-Adams filtration 5, while $\epsilon + \eta\sigma$ and $\kappa$ have $\bo$-Adams filtration 2 and 3. Therefore, this relation is present in the $E_2$-page of the $\bo$-Adams spectral sequence.
\end{rmk}

