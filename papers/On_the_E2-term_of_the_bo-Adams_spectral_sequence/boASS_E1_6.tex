% !TEX root = boASS6.tex

\section{The $E_1$-term of the $\bo$-ASS}\label{sec:E1}

Let $B_i$ denote the $i$th integral Brown-Gitler spectrum (see, for example, \cite{CDGM}).  The integral Brown-Gitler spectra are a sequence of finite complexes whose colimit is the integral Eilenberg-Maclane spectrum:
$$ S = B_0 \subset B_1 \subset \cdots \subset H\ZZ. $$
The image of their homology in 
$$ H_* H\ZZ \cong \FF_2[\zeta_1^2, \zeta_2, \ldots ] \subset A_* $$
is the $A_*$-subcomodule spanned by monomials $m$ of weight $wt(m) \le 2i$, where
$$ wt(\zeta_i) := 2^{i-1}. $$

\begin{thm}[Mahowald \cite{Mahowaldbo}, Milgram \cite{Milgram}]\label{thm:splitting}
	We have
	$$ \bo \wedge \br{\bo}^n \simeq \bigvee_{\abs{I} = n} \Si^{4\norm{I}} \bo \s B_I $$
	 where each term of the multi-index $I = (i_1, \ldots i_n)$ satisfies $i_l > 0$ and 
	 $$ B_I := B_{i_1} \s \cdots \s B_{i_n}. $$
\end{thm}

Mahowald and Milgram also determined the homotopy type of the wedge summands above.

\begin{thm}[Mahowald \cite{Mahowaldbo}, Milgram \cite{Milgram}]
	There are equivalences
	$$ \bo \wedge B_I \simeq b_I \vee HV_I $$
	where
	$$ b_I := 
	\begin{cases}
	\bo^{\bra{2\norm{I}-\alpha(I)}}, & \norm{I} \: \mr{even}, \\
	\bsp^{\bra{2\norm{I}-\alpha(I)-1}}, & \norm{I} \: \mr{odd}. \\
	\end{cases}
	$$
	and $V_I$ is a graded $\FF_2$-vector space of finite type.
\end{thm}

As indicated in the introduction, we will define 
\begin{align*}
V^{n,*} &:= \bigoplus_{\abs{I} = n} \Si^{4\norm{I}} V_I, \\
\mc{C}^{n,*} &:= \bigoplus_{\abs{I} = n} \Si^{4\norm{I}} \pi_* b_I.
\end{align*}
The subspace $V^{*,*} \subseteq \E{bo}{}_1^{*,*} $ is a subcomplex with respect to the $d_1$ differential \cite{LM}, and we endow $\mc{C}^{*,*}$ with the structure of the quotient complex.
The direct sum decomposition
$$ \E{bo}{}^{*,*}_1 = \mc{C}^{*,*} \oplus V^{*,*} $$
results in a decomposition of elements $x \in \E{bo}{}_1^{*,*}$ into \emph{good} and \emph{evil} components:
$$ x = x_{good} + x_{evil}. $$ 
We shall refer to the $d_1$ differential, restricted to $V^{*,*}$, as $d^{evil}_1$.  The induced differential on $\mc{C}^{*,*}$ will be denoted $d_1^{good}$. 

Since the spectra $b_I$ are Adams covers, it is useful to understand $\pi_* b_I$, and the associated differential $d_1^{good}$ in the complex $\mc{C}^{*,*}$, in terms of the Adams spectral sequence for $\pi_* b_I$.
To this end, we will denote elements of $\mc{C}^{*,*}$
using notation
$ x(I)$
for $x$ an element of $\pi_* b_I$.  
We denote the generators of $\E{ass}{}_2^{*,*}(b_I)$
using notation
$$ v_0^k w^m h_1^a $$
where $v_0 = [2]$, $w$ is the formal expression 
$$ w := v_1^2/v_0^2 $$
(here $v_1^4$ detects the Bott element in $\pi_8(\bo)$) and $h_1 = [\eta]$.
For example, in the case of $b_{(2)} = \bo^{\bra{3}}$, generators of $\E{ass}{}_2^{*,*}(b_{2})$ are labeled in Figure~\ref{fig:bIexample}.
\begin{figure}
\centering
\includegraphics[width=0.7\linewidth]{bIexample}
\caption{An example of the labeling system for generators in the Adams spectral sequence for $b_I$ ($I=(2)$).}
\label{fig:bIexample}
\end{figure}
With respect to this notation, up to terms of higher Adams filtration, the differential in the complex $\mc{C}^{*,*}$ is determined by the following theorem. 


\begin{thm}[Mahowald \cite{Mahowaldbo}, Lellmann-Mahowald \cite{LM}]\label{thm:d1}
	The differential $d_1^{good}$ in the complex $\mc{C}^{*,*}$ is determined modulo elements of higher Adams filtration by the formula
	\begin{multline*}
	 [d_1^{good}(w^m(i_1, \ldots, i_n))] = \sum_{m' + m'' = m} v_0^{\alpha(m')+\alpha(m'')-\alpha(m)} w^{m'}(m'', i_1, \ldots, i_n) \\ + 
	\sum_k \sum_{i_k'+i_k'' = i_k} v_0^{\alpha(i_k') + \alpha(i_k'')-\alpha(i_k)} w^m(i_1, \ldots, i_k', i_k'', \ldots, i_n) \end{multline*}
	and the fact that the differential is $v_0$ and $h_1$-linear.  (In the above formula, we are saying that for each multi-index $J$, the $J$-component of the differential $d_1^{good}$ is detected by the indicated element in the Adams spectral sequence for $b_J$.)
\end{thm}

\begin{rmk}
	Lellmann and Mahowald actually compute the differential $d_1^{good}$ more precisely (i.e., not up to ambiguity of higher Adams filtration) but our presentation in Theorem~\ref{thm:d1} is precise enough to contain all of the relevant features of this differential.
\end{rmk}

We present a new proof of Theorem~\ref{thm:d1} which we think is much simpler than the approach of \cite{LM} (based on Adams operations and numerical polynomials).

\begin{proof}[Proof of Theorem~\ref{thm:d1}]
Consider the zig-zag of complexes:
$$ C^*_{BP_*BP}(BP_*[1/2]) = v_0^{-1}\E{BP}{}^{*,*}_1 \xrightarrow{f} v_0^{-1} \E{E(1)}{}^{*,*}_1 \xleftarrow{g} \mc{C}^{*,*} $$
(where $C^*_{BP_*BP}(BP_*[1/2])$ denotes the normalized cobar complex for $BP_*BP$ --- with $2$ inverted) induced by the zig-zag of ring spectra
$$ BP [1/2] \rightarrow E(1)[1/2] \leftarrow \bo. $$
Since the map $g$ is an injection in internal degrees $t \equiv 0 \mod 4$, it will suffice to lift the image of $w^m(I)$ to the localized cobar complex for $BP_*BP$, and verify the formula for the differential there. 
Define
$$ \tau := t_1^2 + v_1t_1 \in BP_*BP; $$
this primitive element detects $\nu$ in the Adams-Novikov spectral sequence.
Let $e_i$ denote the generator of $\pi_0(b_{(i)})$.  Under the zig-zag
$$ BP_*BP \xrightarrow{f} E(1)_*E(1) \xleftarrow{g} \mc{C}^{1,*} $$
we have
$$ [f(\tau^i)] = [g(e_i)]. $$
This follows from \cite[Sec.~3.4]{BOSS} (where $e_i$ is instead denoted ``$e_{4i}$''), together with the fact that $[\tau] = [t_1^2]$ (since $v_1$ has Adams filtration 1).
It follows that we have  
$$ \left[f\left(\frac{v_1^{2m}}{2^{2m}} [\tau^{i_1}| \cdots | \tau^{i_n}]\right)\right]
= [g(w^m(i_1, \ldots, i_n))]. $$
Observe that in the Hopf algebroid $(BP_*[1/2], BP_*BP[1/2])$ we have the formulas:
\begin{align*}
\eta_R\left(\frac{v_1^2}{4}\right) = \frac{v_1^2}{4} + \tau, \\
\psi(\tau) = \tau \otimes 1 + 1 \otimes \tau.
\end{align*}
By the binomial theorem, together with the fact that
$$ \nu_2 \binom{a}{b} = \alpha(b)+\alpha(a-b)-\alpha(a) $$
we have
\begin{align*}
\left[\eta_R\left(\frac{v_1^{2m}}{2^{2m}}\right)\right] &= \sum_{m = m' + m''}v_0^{\alpha(m')+\alpha(m'') - \alpha(m)}\frac{v_1^{2m'}}{2^{2m'}}\tau^{m''}, \\
[\psi(\tau^i)] &= \sum_{i'+i'' = i} v_0^{\alpha(i')+\alpha(i'') - \alpha(i)}\tau^{i'} \otimes \tau^{i''}.
\end{align*}
We deduce 
\begin{multline*}
	 \left[d\left(\frac{v_1^{2m}}{2^{2m}} [\tau^{i_1}| \cdots | \tau^{i_n}] \right)\right] = \sum_{m' + m'' = m} v_0^{\alpha(m')+\alpha(m'')-\alpha(m)} \frac{v_1^{2m'}}{v_0^{2m'}} [\tau^{m''}| \tau^{i_1}| \cdots | \tau^{i_n}] \\ + 
	\sum_k \sum_{i_k'+i_k'' = i_k} v_0^{\alpha(i_k') + \alpha(i_k'')-\alpha(i_k)} \frac{v_1^{2m}}{v_0^{2m}}[\tau^{i_1}|\cdots|\tau^{i_k'}|\tau^{i_k''}| \cdots |\tau^{i_n}]. \end{multline*}
The result follows.
\end{proof}
















 