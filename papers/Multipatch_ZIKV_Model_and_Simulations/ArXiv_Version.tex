%%%%%%%%%%%%%%%%%%%% author.tex %%%%%%%%%%%%%%%%%%%%%%%%%%%%%%%%%%%
%
% sample root file for your "contribution" to a contributed volume
%
% Use this file as a template for your own input.
%
%%%%%%%%%%%%%%%% Springer %%%%%%%%%%%%%%%%%%%%%%%%%%%%%%%%%%


% RECOMMENDED %%%%%%%%%%%%%%%%%%%%%%%%%%%%%%%%%%%%%%%%%%%%%%%%%%%
\documentclass{article}
\usepackage{lineno,hyperref}
\usepackage[american]{babel}
\usepackage{amssymb, amsthm,amsmath}
\usepackage{bm}
\usepackage{graphicx}
\usepackage[super]{cite}

\newtheorem{theorem}{Theorem}
\newtheorem{corollary}[theorem]{Corollary}
\newtheorem{proposition}[theorem]{Proposition}
\newtheorem{condition}[theorem]{Condition}
\newtheorem{remark}[theorem]{Remark}
\newtheorem{lemma}[theorem]{Lemma}
\newtheorem{example}[theorem]{Example}
\newtheorem{definition}[theorem]{Definition}
% choose options for [] as required from the list
% in the Reference Guide

\usepackage{mathptmx}       % selects Times Roman as basic font
\usepackage{helvet}         % selects Helvetica as sans-serif font
\usepackage{courier}        % selects Courier as typewriter font
\usepackage{type1cm}        % activate if the above 3 fonts are
                            % not available on your system
%
\usepackage{makeidx}         % allows index generation
\usepackage{graphicx}        % standard LaTeX graphics tool
                             % when including figure files
\usepackage{multicol}        % used for the two-column index
\usepackage[bottom]{footmisc}% places footnotes at page bottom
\usepackage{amsmath}
%\usepackage{amsthm}
\usepackage{amssymb}
\usepackage{amstext} 
\usepackage{float}
%\usepackage[backend=bibtex,hyperref=true]{biblatex}
%\addbibresource{bibliography.bib}
%\newtheorem*{remark}{Remark}
% see the list of further useful packages
% in the Reference Guide
\allowdisplaybreaks
\makeindex             % used for the subject index
                       % please use the style svind.ist with
                       % your makeindex program

%%%%%%%%%%%%%%%%%%%%%%%%%%%%%%%%%%%%%%%%%%%%%%%%%%%%%%%%%%%%%%%%%%%%%%%%%%%%%%%%%%%%%%%%%

\begin{document}
	\title{Multipatch ZIKV Model and Simulations}
	
	\author{A.~Sherly and W.~Bock\\[.3cm]
		{\small Technische Universit\"at Kaiserslautern,}\\{\small  Fachbereich Mathematik,}\\ {\small Gottlieb-Daimler-Stra{\ss}e 48,}\\ {\small 67663 Kaiserslautern, Germany}\\
		{\small E-Mail: $\{sherly,bock\}$@mathematik.uni-kl.de}\\[.2cm]
		}
	
	
	
	\maketitle
	



%\abstract*{In this article we compare two multi-patch models for the spread of Zika virus based on an SIRUV model. When the commuting between patches is ceased we expect that all the patches follow the dynamics of the single patch model. We show in an example that  the effective population size should be used rather than the population size of the respective patch.}

\begin{abstract}In this article we compare two multi-patch models for the spread of Zika virus based on an SIRUV model. When the commuting between patches is ceased we expect that all the patches follow the dynamics of the single patch model. We show in an example that the effective population size should be used rather than the population size of the respective patch.
\end{abstract}
\section{Introduction}\index{Multi-patch Zika Virus model}
Zika Virus belongs to the family \textit{Flaviviridae}, genus \textit{Flavivirus}. ZIKV disease is primarily vector-borne, which is transmitted by \textit{Aedes} mosquitoes\cite{Aedes_ZIKV}. This disease is also found to be sexually transmissible\cite{Sexually_transmitted}. 
Eventhough most patients show mild symptoms recent studies show that this virus attack results in neurological disorders like Guillain-Barré syndrome (GBS)\cite{Guillian_Barre}. Another important characteristic of this virus is its pathogenicity to fetuses causing  Microcephaly in newborn babies\cite{Microcephaly}.\\
The history of ZIKV disease known so far starts with the isolation of Zika virus from a rhesus monkey in Uganda around April 1947. There onwards it has spread across the world with the largest outbreak recorded in 2015-16 across South America\cite{ZIKV_SA,Aedes_ZIKV}. With no vaccines or medications found so far the disease spread can only be controlled by non-pharmaceutical interventions. Also increased international travel, evolution and mutation of viruses and their transmitting agents like mosquitoes, suitable environmental conditions etc lead to an increase in further outbreaks even in lesser probable places. The influence of human mobility plays an important role in transmitting diseases across continents. With more flight connectivity and affordable modes of transport disease transmission can also be faster. The primary objective of this study is to include spatial dependence to the mechanistic model of ZIKV spread. This is very relevant as the parameters involved in the model will be different for different places. So the dynamics will be exhibiting variations spatially. In this article we use an SIRUV model to describe the disease dynamics. This model divides the population into various compartments namely susceptible, infected and recovered. The interaction between various host and vector compartments, spread across different patches, is modeled using a coupling matrix and certain parameters.\\   
 We have discussed two models in section 2 and 3. The results of numerical simulations are provided in section 4. Comparing the two models exemplarily shows that the incorporation of the effective population size is crucial. While in a model, which just takes into account, the total population size of the patches, a decoupling does not lead to  the single patch dynamics, where as a model which incorporates the effective population size shows this desired property. 
\section{Multi-patch ZIKV model}\label{section2}\index{Multi-patch Zika Virus model}
 In this section we give a multi patch model for studing the ZIKV disease spread. Let the space domain be divided into small areas which we name as patches. The ZIKV model in a specific patch is also developed using different compartments. Here the host and vector population consists respectively of susceptible and infected compartments in each patch and we consider the recovered ones only in host population of each patch. We use either a subscript or a superscript ($i$, $j$ or $k$) to distinguish these compartments and the parameters patchwise. Let us first assume that the whole population is commuting between the patches and the rate of transition from patch $\left(i\right)$ to $
\left(j\right)$ be $p_{ij}$.
\begin{remark}
	The matrix $P$ with entries $p_{ij}$ is the residence time budgeting matrix. Here $p_{ij}$ represents the time spent by people in patch $i$ on average in patch $j$ in unit time\cite{P_matrix}. For example on average if a person in patch $i$ spent 8 hours in patch $j$, then $p_{ij} = \frac{8}{24}$, provided that unit time is one day.
\end{remark}
 We have deduced the following model from similar models in the literature used for other epidemiological studies\cite{Yashika}.
\begin{align*}
\begin{split}
&\frac{d S_{i}}{dt}=\mu_{i}\left( 1-S_{i}\right)-S_{i}\Bigg(\sum_{1\leq j\leq n}\beta_{vh}^{j}p_{ij}V_{j}+\Bigg(\sum_{1\leq j\leq n}\beta_{hh}^{i}\left(p_{ij}+p_{ji}\right){I_{j}}\Bigg.\Bigg.\Bigg.\Bigg.-\beta_{hh}^{i}p_{ii}{I_{i}}\Bigg)\Bigg)\\
%
%
&\frac{dI_{i}}{dt}=S_{i}\Bigg(\sum_{1\leq j\leq n}\beta_{vh}^{j}p_{ij}V_{j}+\Bigg(\sum_{1\leq j\leq n}\beta_{hh}^{i}\left(p_{ij}+p_{ji}\right){I_{j}}\Bigg.\Bigg.\Bigg.\Bigg.-\beta_{hh}^{i}p_{ii}{I_{i}}\Bigg)\Bigg)-\mu_{i}I_{i}-\gamma_{i}I_{i}\\
%
%
&\frac{d R_{i}}{dt}= \gamma_{i}I_{i}-\mu_{i}R_{i}\\
%
%
&\frac{d U_{i}}{dt}= \nu_{i}\left( 1-U_{i}\right) -\vartheta_{i}U_{i}\sum_{1\leq j\leq n}\frac{I_{j}}p_{ji}\\
%
%
&\frac{dV_{i}}{dt}=\vartheta_{i}U_{i}\sum_{1\leq j\leq n}{I_{j}}p_{ji}-\nu_{i}V_{i}.
\end{split}
\end{align*}
\section{Redefining the model for ZIKV}\label{section3}
Following some insights from \cite{Bichara} and \cite{Hethcote} we have developed a new model to describe the ZIKV disease spread. 
%So the first problem is the way $\beta_{hv}$ and $\beta_{hh}$ are defined and secondly the problem is that the fraction of infectives in each patch is not correctly taken. another problem is that we are not working with fractions so far so there is a question of continuity always.
 In \cite{Hethcote}  a term called contact rate is clearly defined, which is the average number of adequate contacts per day of an infective person from patch $j$ with any individuals in patch $i$. 
With this in consideration we redefine the parameters used as follows\\
\\
$\alpha_j$ = number of infectious contacts that is happening per infected mostiquito per unit time with the people present in patch $j$.\\
$\beta_j$= number of infectious contacts that is happening per infective individual per unit time with the people present in patch $j$.\\
$\gamma_j$ = number of recoveries that is happening per unit time in patch $j$.\\
$\vartheta_j$ = number of infective contacts that is happening per infected human with mosquitoes in patch $j$ in unit time.
\\\\
Let us focus on patch $j$ and see how many susceptibles from patch $i$ is infected in patch $j$. By the definition of $\alpha_j$ the number of people getting into adequate contacts with the mosquitoes in patch $j$ is given by $\alpha_j \mathcal{V}_j$. Now the total number of people who were present in patch $j$ is given by $\sum_{k=1}^n p_{kj}N_k$. Let us call this the effective population in patch $j$. Also the effective population of susceptibles in patch $j$ is $\sum_{k=1}^{n} p_{kj}\mathcal{S}_k$. 
Among which $p_{ij}\mathcal{S}_i$ are coming from patch $i$. The number of susceptibles from patch $i$ who get infected in patch $j$ due to mosquitoes is given by $$\alpha_j\mathcal{V}_j\frac{p_{ij}\mathcal{S}_i}{\sum_{k=1}^n p_{kj}N_k}.$$ Now we focus on the infections between humans. The number of infections happening in patch $j$ in unit time due to human-human interactions is given by $\beta_jI_{eff}$,  where $I_{eff}$ is the effective number of infected people who came to patch $j$ in unit time which is given by $\sum_{k=1}^{n} p_{kj}\mathcal{I}_k$. The total number of infections happening in patch $j$ is given by $\beta_j\sum_{k=1}^{n} p_{kj}\mathcal{I}_k$ out of which the number of infections happened to the susceptible people of patch $i$ is $$\beta_j\sum_{k=1}^{n} p_{kj}I_k\frac{p_{ij}\mathcal{S}_i}{\sum_{k=1}^n p_{kj}N_k}.$$
%Now we write down the difference equations
%\begin{align*}
%{dS_{i}}&= \mu_i(N_i-S_i)-\sum_{j=1}^{n}\alpha_jV_j\frac{p_{ij}S_i}{\sum_{k=1}^n p_{kj}N_k}-\sum_{j=1}^{n}\beta_j\sum_{k=1}^{n} p_{kj}I_k\frac{p_{ij}S_i}{\sum_{k=1}^n p_{kj}N_k}\\
%{dI_{i}}&= -(\gamma_i+\mu_i)I_i+\sum_{j=1}^{n}\alpha_jV_j\frac{p_{ij}S_i}{\sum_{k=1}^n p_{kj}N_k}+\sum_{j=1}^{n}\beta_j\sum_{k=1}^{n} p_{kj}I_k\frac{p_{ij}S_i}{\sum_{k=1}^n p_{kj}N_k}\\
%{dR_{i}}&= \gamma_iI_i-\mu_iR_i\\
%{dU_{i}}&= \nu_i(M_i-U_i)-\vartheta_i\frac{U_i}{M_i}{\sum_{k=1}^n p_{ki}I_k}\\
%{dV_{i}}&= -\nu_i V_i+\vartheta_i\frac{U_i}{M_i}{\sum_{k=1}^n p_{ki}I_k}
%\end{align*}
Now we have to introduce fractions by normalising each compartmental values. 
\begin{center}
	\begin{tabular}{c c c c c}
		${S}_i$&${I}_i$ &${R}_i$&${U}_i$ &${V}_i$\\\hline\noalign{\smallskip} $\frac{\mathcal{S}_i}{N_i}$&$\frac{\mathcal{I}_i}{N_i}$&$\frac{\mathcal{R}_i}{N_i}$
		&$\frac{\mathcal{U}_i}{M_i}$&$\frac{\mathcal{V}_i}{M_i}$
	\end{tabular}
\end{center}
The following system of ODEs describe disease spread in each patch $i$
%Rewriting the difference equations using the fractions will give us the desired model
\begin{align*}
\frac{d{S}_{i}}{dt}&= \mu_i(1-{S}_i)-{\sum_{j=1}^{n}\alpha_jM_j{V}_j\frac{p_{ij}{S}_i}{\sum_{k=1}^n p_{kj}N_k}}-{\sum_{j=1}^{n}\beta_j\sum_{k=1}^{n} p_{kj}N_k{I}_k\frac{p_{ij}{S}_i}{\sum_{k=1}^n p_{kj}N_k}}\\
\frac{d{I}_{i}}{dt}&= -(\gamma_i+\mu_i){I}_i+{\sum_{j=1}^{n}\alpha_jM_j{V}_j\frac{p_{ij}{S}_i}{\sum_{k=1}^n p_{kj}N_k}}+{\sum_{j=1}^{n}\beta_j\sum_{k=1}^{n} p_{kj}N_k{I}_k\frac{p_{ij}{S}_i}{\sum_{k=1}^n p_{kj}N_k}}\\
\frac{d{R}_{i}}{dt}&= \gamma_i{I}_i-\mu_i{R}_i\\
\frac{d{U}_{i}}{dt}&= \nu_i(1-{U}_i)-\vartheta_i\frac{{U}_i}{M_i}{\sum_{k=1}^n p_{ki}N_k{I}_k}\\
\frac{d{V}_{i}}{dt}&= -\nu_i {V}_i+\vartheta_i\frac{{U}_i}{M_i}{\sum_{k=1}^n p_{ki}N_k{I}_k}.
\end{align*}
\section{Comparison of both models in three-patch scenario}
In a case where $n=3$ we numerically simulated both the models and  compared the results. We obtained the influence of the residence time budgeting matrix on the multi-patch model. Here we restrict ourselves to consider three patches with the same set of parameters and population sizes. The movements between these three patches are defined using the residence time budgeting matrix $P$. The question is how far does the dynamics deviate from the single patch case, when the movement between the patches is controlled using the $p_{ij}$ values. We use the parameters 
and population sizes, as given in table \ref{table1}, for the numerical simulation. We are studying two cases- the three patches being coupled and completely decoupled respectively. For the first case 
\begin{align}\label{P_matrix}
P =
 \begin{bmatrix}
 0.2 &0.7&0.1\\
 0.5&0.1&0.4\\
 0.3&0.6&0.1
  \end{bmatrix}
  \end{align}
\begin{table} 
\begin{center}
	\caption{Parameters and Population Sizes}
	{\footnotesize
		\begin{tabular}{p{2cm}p{1cm}p{1cm}p{1cm}p{1cm}p{1.5cm}p{1.5cm}}
			\hline\noalign{\smallskip}
			$\mu$ & $\alpha$ &$\beta$&$\vartheta$&$\nu$&N&M \\
\hline
\\
			10/(1000*365)          %birth rate host 
			&0.008						%vector to host
			& 0.01                    			%host to host
			&0.4                    			%host to vector
			& 1/14                  			%birth rate vector
			& 20000
			& 100000\\
			\noalign{\smallskip}\hline\noalign{\smallskip}
	\end{tabular}}
\label{table1}
\end{center}
\end{table}
%All three patches are set to have the same parameters and population sizes but commuting behaviour among these patches is controlled by the matrix $P$. 
\begin{figure}[ht]
	\centering
	\includegraphics[width=0.7\linewidth]{fig2_phaseportrait}
	\caption{Phase portrait for three patches using model 1(Section \ref{section2})  and model 2(Section \ref{section3}) for the case where the patches are coupled using the matrix $P$ from \eqref{P_matrix}.}
	\label{fig:fig2phaseportrait}
\end{figure}
\begin{figure}[H]
	\centering
	\includegraphics[width=0.7\linewidth]{fig2_Pzero}
	\caption{For the same set of parameters as in Figure \ref{fig:fig2phaseportrait} when $P$ is set to identity matrix we see the given results where the red starred curve is the phase portrait of the single patch model}
	\label{fig:fig2pzero}
\end{figure}
The dynamics was supposed to be similar for the single patch and multi-patch models for the case $P=I$. But we have not seen this property for the old model.
\section{Conclusion}
%We used the parameters in Table \ref{table1} and both models gave reasonable results despite the case that both models have different dynamics
In this study we have considered two different models to describe the dynamics of ZIKV spread. We compared the two models to identify the suitable model. When the commuting between patches is ceased we expect that all the three patches follow the dynamics of the single patch model. The first model failed to satisfy this condition where as the second model was successfully exhibiting this property. This gives rise to a more thorough study of the second model in a forthcoming work.%Now we try to analyse the dynamics in one patch case. 
%The study experiments are pointed out below. 
%
%\underline{First case}\\
%\begin{center}
%	\begin{tabular}{|c|c|}
%		\hline
%		$\mu$ & 10/(1000*365)\\           %birth rate host 
%		$\alpha$ &0.008\\						%vector to host
%		$\beta$ & 0.01\\                    			%host to host
%		$\vartheta$ &0.4\\                     			%host to vector
%		$\nu$ & 1/13\\                      			%birth rate vector
%		$M$ & 10000\\
%		$N$ & 5000\\
%		InC & [0.0984, 0.0049, 0.0126]\\
%		InC2 & [0.01, 0.004, 0.012]\\
%		$\gamma$ & 0.008\\
%		\hline
%	\end{tabular}
%\end{center}
%For the InC we have the following conclusions for S = 0.4. The results are as follows 
%\begin{figure}[ht]
%	\centering
%	\includegraphics[width=0.5\linewidth]{Zika_1D_I}
%	\caption{}
%	\label{fig:zika1di}
%\end{figure}

%\section{Section Heading}
%\label{sec:1}
%Use the template \emph{chapter.tex} together with the Springer document class SVMono (monograph-type books) or SVMult (edited books) to style the various elements of your chapter content in the Springer layout.
%
%Instead of simply listing headings of different levels we recommend to
%let every heading be followed by at least a short passage of text.
%Further on please use the \LaTeX\ automatism for all your
%cross-references and citations. And please note that the first line of
%text that follows a heading is not indented, whereas the first lines of
%all subsequent paragraphs are.
%
%\section{Section Heading}
%\label{sec:2}
%% Always give a unique label
%% and use \ref{<label>} for cross-references
%% and \cite{<label>} for bibliographic references
%% use \sectionmark{}
%% to alter or adjust the section heading in the running head
%Instead of simply listing headings of different levels we recommend to
%let every heading be followed by at least a short passage of text.
%Further on please use the \LaTeX\ automatism for all your
%cross-references and citations.
%
%Please note that the first line of text that follows a heading is not indented, whereas the first lines of all subsequent paragraphs are.
%
%Use the standard \verb|equation| environment to typeset your equations, e.g.
%%
%\begin{equation}
%a \times b = c\;,
%\end{equation}
%%
%however, for multiline equations we recommend to use the \verb|eqnarray| environment\footnote{In physics texts please activate the class option \texttt{vecphys} to depict your vectors in \textbf{\itshape boldface-italic} type - as is customary for a wide range of physical subjects}.
%\begin{eqnarray}
%a \times b = c \nonumber\\
%\vec{a} \cdot \vec{b}=\vec{c}
%\label{eq:01}
%\end{eqnarray}
%For a clearer picture we may use the following case where the $P$ matrix is set to zero. That means the commuting between patches are not allowed. In this case all three patches should behave like single patch model with the same set of parameters. We have plotted the results  
%\subsection{Subsection Heading}
%\label{subsec:2}
%Instead of simply listing headings of different levels we recommend to
%let every heading be followed by at least a short passage of text.
%Further on please use the \LaTeX\ automatism for all your
%cross-references\index{cross-references} and citations\index{citations}
%as has already been described in Sect.~\ref{sec:2}.
%
%\begin{quotation}
%Please do not use quotation marks when quoting texts! Simply use the \verb|quotation| environment -- it will automatically render Springer's preferred layout.
%\end{quotation}
%
%
%\subsubsection{Subsubsection Heading}
%Instead of simply listing headings of different levels we recommend to
%let every heading be followed by at least a short passage of text.
%Further on please use the \LaTeX\ automatism for all your
%cross-references and citations as has already been described in
%Sect.~\ref{subsec:2}, see also Fig.~\ref{fig:1}\footnote{If you copy
%text passages, figures, or tables from other works, you must obtain
%\textit{permission} from the copyright holder (usually the original
%publisher). Please enclose the signed permission with the manuscript. The
%sources\index{permission to print} must be acknowledged either in the
%captions, as footnotes or in a separate section of the book.}
%
%Please note that the first line of text that follows a heading is not indented, whereas the first lines of all subsequent paragraphs are.
%
%% For figures use
%%
%\begin{figure}[b]
%\sidecaption
%% Use the relevant command for your figure-insertion program
%% to insert the figure file.
%% For example, with the graphicx style use
%\includegraphics[scale=.65]{figure}
%%
%% If no graphics program available, insert a blank space i.e. use
%%\picplace{5cm}{2cm} % Give the correct figure height and width in cm
%%
%\caption{If the width of the figure is less than 7.8 cm use the \texttt{sidecapion} command to flush the caption on the left side of the page. If the figure is positioned at the top of the page, align the sidecaption with the top of the figure -- to achieve this you simply need to use the optional argument \texttt{[t]} with the \texttt{sidecaption} command}
%\label{fig:1}       % Give a unique label
%\end{figure}
%
%
%\paragraph{Paragraph Heading} %
%Instead of simply listing headings of different levels we recommend to
%let every heading be followed by at least a short passage of text.
%Further on please use the \LaTeX\ automatism for all your
%cross-references and citations as has already been described in
%Sect.~\ref{sec:2}.
%
%Please note that the first line of text that follows a heading is not indented, whereas the first lines of all subsequent paragraphs are.
%
%For typesetting numbered lists we recommend to use the \verb|enumerate| environment -- it will automatically render Springer's preferred layout.
%
%\begin{enumerate}
%\item{Livelihood and survival mobility are oftentimes coutcomes of uneven socioeconomic development.}
%\begin{enumerate}
%\item{Livelihood and survival mobility are oftentimes coutcomes of uneven socioeconomic development.}
%\item{Livelihood and survival mobility are oftentimes coutcomes of uneven socioeconomic development.}
%\end{enumerate}
%\item{Livelihood and survival mobility are oftentimes coutcomes of uneven socioeconomic development.}
%\end{enumerate}
%
%
%\subparagraph{Subparagraph Heading} In order to avoid simply listing headings of different levels we recommend to let every heading be followed by at least a short passage of text. Use the \LaTeX\ automatism for all your cross-references and citations as has already been described in Sect.~\ref{sec:2}, see also Fig.~\ref{fig:2}.
%
%For unnumbered list we recommend to use the \verb|itemize| environment -- it will automatically render Springer's preferred layout.
%
%\begin{itemize}
%\item{Livelihood and survival mobility are oftentimes coutcomes of uneven socioeconomic development, cf. Table~\ref{tab:1}.}
%\begin{itemize}
%\item{Livelihood and survival mobility are oftentimes coutcomes of uneven socioeconomic development.}
%\item{Livelihood and survival mobility are oftentimes coutcomes of uneven socioeconomic development.}
%\end{itemize}
%\item{Livelihood and survival mobility are oftentimes coutcomes of uneven socioeconomic development.}
%\end{itemize}
%
%\begin{figure}[t]
%\sidecaption[t]
%% Use the relevant command for your figure-insertion program
%% to insert the figure file.
%% For example, with the option graphics use
%\includegraphics[scale=.65]{figure}
%%
%% If no graphics program available, insert a blank space i.e. use
%%\picplace{5cm}{2cm} % Give the correct figure height and width in cm
%%
%%\caption{Please write your figure caption here}
%\caption{If the width of the figure is less than 7.8 cm use the \texttt{sidecapion} command to flush the caption on the left side of the page. If the figure is positioned at the top of the page, align the sidecaption with the top of the figure -- to achieve this you simply need to use the optional argument \texttt{[t]} with the \texttt{sidecaption} command}
%\label{fig:2}       % Give a unique label
%\end{figure}
%
%\runinhead{Run-in Heading Boldface Version} Use the \LaTeX\ automatism for all your cross-references and citations as has already been described in Sect.~\ref{sec:2}.
%
%\subruninhead{Run-in Heading Italic Version} Use the \LaTeX\ automatism for all your cross-refer\-ences and citations as has already been described in Sect.~\ref{sec:2}\index{paragraph}.
%% Use the \index{} command to code your index words
%%
%% For tables use
%%
%\begin{table}
%\caption{Please write your table caption here}
%\label{tab:1}       % Give a unique label
%%
%% Follow this input for your own table layout
%%
%\begin{tabular}{p{2cm}p{2.4cm}p{2cm}p{4.9cm}}
%\hline\noalign{\smallskip}
%Classes & Subclass & Length & Action Mechanism  \\
%\noalign{\smallskip}\svhline\noalign{\smallskip}
%Translation & mRNA$^a$  & 22 (19--25) & Translation repression, mRNA cleavage\\
%Translation & mRNA cleavage & 21 & mRNA cleavage\\
%Translation & mRNA  & 21--22 & mRNA cleavage\\
%Translation & mRNA  & 24--26 & Histone and DNA Modification\\
%\noalign{\smallskip}\hline\noalign{\smallskip}
%\end{tabular}
%$^a$ Table foot note (with superscript)
%\end{table}
%%
%\section{Section Heading}
%\label{sec:3}
%% Always give a unique label
%% and use \ref{<label>} for cross-references
%% and \cite{<label>} for bibliographic references
%% use \sectionmark{}
%% to alter or adjust the section heading in the running head
%Instead of simply listing headings of different levels we recommend to
%let every heading be followed by at least a short passage of text.
%Further on please use the \LaTeX\ automatism for all your
%cross-references and citations as has already been described in
%Sect.~\ref{sec:2}.
%
%Please note that the first line of text that follows a heading is not indented, whereas the first lines of all subsequent paragraphs are.
%
%If you want to list definitions or the like we recommend to use the Springer-enhanced \verb|description| environment -- it will automatically render Springer's preferred layout.
%
%\begin{description}[Type 1]
%\item[Type 1]{That addresses central themes pertainng to migration, health, and disease. In Sect.~\ref{sec:1}, Wilson discusses the role of human migration in infectious disease distributions and patterns.}
%\item[Type 2]{That addresses central themes pertainng to migration, health, and disease. In Sect.~\ref{subsec:2}, Wilson discusses the role of human migration in infectious disease distributions and patterns.}
%\end{description}
%
%\subsection{Subsection Heading} %
%In order to avoid simply listing headings of different levels we recommend to let every heading be followed by at least a short passage of text. Use the \LaTeX\ automatism for all your cross-references and citations citations as has already been described in Sect.~\ref{sec:2}.
%
%Please note that the first line of text that follows a heading is not indented, whereas the first lines of all subsequent paragraphs are.
%
%\begin{svgraybox}
%If you want to emphasize complete paragraphs of texts we recommend to use the newly defined Springer class option \verb|graybox| and the newly defined environment \verb|svgraybox|. This will produce a 15 percent screened box 'behind' your text.
%
%If you want to emphasize complete paragraphs of texts we recommend to use the newly defined Springer class option and environment \verb|svgraybox|. This will produce a 15 percent screened box 'behind' your text.
%\end{svgraybox}
%
%
%\subsubsection{Subsubsection Heading}
%Instead of simply listing headings of different levels we recommend to
%let every heading be followed by at least a short passage of text.
%Further on please use the \LaTeX\ automatism for all your
%cross-references and citations as has already been described in
%Sect.~\ref{sec:2}.
%
%Please note that the first line of text that follows a heading is not indented, whereas the first lines of all subsequent paragraphs are.
%
%\begin{theorem}
%Theorem text goes here.
%\end{theorem}
%%
%% or
%%
%\begin{definition}
%Definition text goes here.
%\end{definition}
%
%\begin{proof}
%%\smartqed
%Proof text goes here.
%\qed
%\end{proof}
%
%\paragraph{Paragraph Heading} %
%Instead of simply listing headings of different levels we recommend to
%let every heading be followed by at least a short passage of text.
%Further on please use the \LaTeX\ automatism for all your
%cross-references and citations as has already been described in
%Sect.~\ref{sec:2}.
%
%Note that the first line of text that follows a heading is not indented, whereas the first lines of all subsequent paragraphs are.
%%
%% For built-in environments use
%%
%\begin{theorem}
%Theorem text goes here.
%\end{theorem}
%%
%\begin{definition}
%Definition text goes here.
%\end{definition}
%%
%\begin{proof}
%\smartqed
%Proof text goes here.
%\qed
%\end{proof}
%%
%\begin{acknowledgement}
%If you want to include acknowledgments of assistance and the like at the end of an individual chapter please use the \verb|acknowledgement| environment -- it will automatically render Springer's preferred layout.
%\end{acknowledgement}
%%
%\section*{Appendix}
%\addcontentsline{toc}{section}{Appendix}
%%
%%
%When placed at the end of a chapter or contribution (as opposed to at the end of the book), the numbering of tables, figures, and equations in the appendix section continues on from that in the main text. Hence please \textit{do not} use the \verb|appendix| command when writing an appendix at the end of your chapter or contribution. If there is only one the appendix is designated ``Appendix'', or ``Appendix 1'', or ``Appendix 2'', etc. if there is more than one.
%
%\begin{equation}
%a \times b = c
%\end{equation}
%\printbibliography
%%%%%%%%%%%%%%%%%%%%%%%%% referenc.tex %%%%%%%%%%%%%%%%%%%%%%%%%%%%%%
% sample references
% %
% Use this file as a template for your own input.
%
%%%%%%%%%%%%%%%%%%%%%%%% Springer-Verlag %%%%%%%%%%%%%%%%%%%%%%%%%%
%
% BibTeX users please use
% \bibliographystyle{}
% \bibliography{}
%


\begin{thebibliography}{99.}%
%% and use \bibitem to create references.
%%
%% Use the following syntax and markup for your references if
%% the subject of your book is from the field
%% "Mathematics, Physics, Statistics, Computer Science"
%%
%% Contribution
\bibitem{Gerritsma} Gerritsma, M.I.: Edge functions for spectral element methods. In Spectral and High Order Methods for Partial Differential Equations, Eds Jan S. Hesthaven and Einar M. R{\o}nquist, \textbf{76}, 494--522, Springer, (2016)
%\bibitem{science-contrib} Broy, M.: Software engineering --- from auxiliary to key technologies. In: Broy, M., Dener, E. (eds.) Software Pioneers, pp. 10-13. Springer, Heidelberg (2002)
%%
%% Online Document
%\bibitem{science-online} Dod, J.: Effective substances. In: The Dictionary of Substances and Their Effects. Royal Society of Chemistry (1999) Available via DIALOG. \\
%\url{http://www.rsc.org/dose/title of subordinate document. Cited 15 Jan 1999}
%%
%% Monograph
\bibitem{Kreyszig} Kreyszig, E.: Introductory functional analysis with applications. Second Edition John Wiley \& Sons (1978)

\bibitem{BuffaCiarlet} Buffa, A., Ciarlet Jr, P.: On traces for functional spaces related to Maxwell's equations Part I: An integration by parts formula in Lipschitz polyhedra. Math. Meth. Appl Sci., \textbf{24}, 9--30, (2001)

\bibitem{OdenDemkowicz} Oden, J.T., Demkowicz, L.: Applied Functional Analysis. Second Edition Chapman \& Hall/CRC (2010)

%%
% Journal article
\bibitem{Carstensen} Carstensen, C., Demkowicz, L., Gopalakrishnan, J.: Breaking spaces and form for the DPG method and applications including the Maxwell equations. Computers and Mathematics with Applications, \textbf{72}, 494--522, (2016)

\bibitem{Jain} Jain,V., Zhang, Y., Palha, A., Gerritsma, M.I.: Construction and application of algebraic dual polynomial representations for finite element methods. arXiv:1712.09472v1, submitted to Computational Methods in Applied Mathematics, (2018)

\bibitem{MEEVC} Palha, A., Gerritsma, M.I.: A mass, energy, enstrophy and vorticity conserving (MEEVC) mimetic spectral element discretization for the 2D incompressible Navier-Stokes equations. Journal of Computational Physics, \textbf{328}, 200--220, (2017)

\bibitem{Yi} Zhang, Y., Jain,V., Palha, A., Gerritsma, M.I.: The discrete Steklov-Poincar\'{e} operator using algebraic dual polynomials. Submitted to Computational Methods in Applied Mathematics, (2018)

%\bibitem{science-journal} Hamburger, C.: Quasimonotonicity, regularity and duality for nonlinear systems of partial differential equations. Ann. Mat. Pura. Appl. \textbf{169}, 321--354 (1995)
%%
%% Journal article by DOI
%\bibitem{science-DOI} Slifka, M.K., Whitton, J.L.: Clinical implications of dysregulated cytokine production. J. Mol. Med. (2000) doi: 10.1007/s001090000086
%%
%\bigskip
%
%% Use the following (APS) syntax and markup for your references if
%% the subject of your book is from the field
%% "Mathematics, Physics, Statistics, Computer Science"
%%
%% Online Document
%\bibitem{phys-online} J. Dod, in \textit{The Dictionary of Substances and Their Effects}, Royal Society of Chemistry. (Available via DIALOG, 1999),
%\url{http://www.rsc.org/dose/title of subordinate document. Cited 15 Jan 1999}
%%
%% Monograph
%\bibitem{phys-mono} H. Ibach, H. L\"uth, \textit{Solid-State Physics}, 2nd edn. (Springer, New York, 1996), pp. 45-56
%%
%% Journal article
%\bibitem{phys-journal} S. Preuss, A. Demchuk Jr., M. Stuke, Appl. Phys. A \textbf{61}
%%
%% Journal article by DOI
%\bibitem{phys-DOI} M.K. Slifka, J.L. Whitton, J. Mol. Med., doi: 10.1007/s001090000086
%%
%% Contribution
%\bibitem{phys-contrib} S.E. Smith, in \textit{Neuromuscular Junction}, ed. by E. Zaimis. Handbook of Experimental Pharmacology, vol 42 (Springer, Heidelberg, 1976), p. 593
%%
%\bigskip
%%
%% Use the following syntax and markup for your references if
%% the subject of your book is from the field
%% "Psychology, Social Sciences"
%%
%%
%% Monograph
%\bibitem{psysoc-mono} Calfee, R.~C., \& Valencia, R.~R. (1991). \textit{APA guide to preparing manuscripts for journal publication.} Washington, DC: American Psychological Association.
%%
%% Online Document
%\bibitem{psysoc-online} Dod, J. (1999). Effective substances. In: The dictionary of substances and their effects. Royal Society of Chemistry. Available via DIALOG. \\
%\url{http://www.rsc.org/dose/Effective substances.} Cited 15 Jan 1999.
%%
%% Journal article
%\bibitem{psysoc-journal} Harris, M., Karper, E., Stacks, G., Hoffman, D., DeNiro, R., Cruz, P., et al. (2001). Writing labs and the Hollywood connection. \textit{J Film} Writing, 44(3), 213--245.
%%
%% Contribution
%\bibitem{psysoc-contrib} O'Neil, J.~M., \& Egan, J. (1992). Men's and women's gender role journeys: Metaphor for healing, transition, and transformation. In B.~R. Wainrig (Ed.), \textit{Gender issues across the life cycle} (pp. 107--123). New York: Springer.
%%
%% Journal article by DOI
%\bibitem{psysoc-DOI}Kreger, M., Brindis, C.D., Manuel, D.M., Sassoubre, L. (2007). Lessons learned in systems change initiatives: benchmarks and indicators. \textit{American Journal of Community Psychology}, doi: 10.1007/s10464-007-9108-14.
%%
%%
%% Use the following syntax and markup for your references if
%% the subject of your book is from the field
%% "Humanities, Linguistics, Philosophy"
%%
%\bigskip
%%
%% Journal article
%\bibitem{humlinphil-journal} Alber John, Daniel C. O'Connell, and Sabine Kowal. 2002. Personal perspective in TV interviews. \textit{Pragmatics} 12:257--271
%%
%% Contribution
%\bibitem{humlinphil-contrib} Cameron, Deborah. 1997. Theoretical debates in feminist linguistics: Questions of sex and gender. In \textit{Gender and discourse}, ed. Ruth Wodak, 99--119. London: Sage Publications.
%%
%% Monograph
%\bibitem{humlinphil-mono} Cameron, Deborah. 1985. \textit{Feminism and linguistic theory.} New York: St. Martin's Press.
%%
%% Online Document
%\bibitem{humlinphil-online} Dod, Jake. 1999. Effective substances. In: The dictionary of substances and their effects. Royal Society of Chemistry. Available via DIALOG. \\
%http://www.rsc.org/dose/title of subordinate document. Cited 15 Jan 1999
%%
%% Journal article by DOI
%\bibitem{humlinphil-DOI} Suleiman, Camelia, Daniel C. O�Connell, and Sabine Kowal. 2002. `If you and I, if we, in this later day, lose that sacred fire...�': Perspective in political interviews. \textit{Journal of Psycholinguistic Research}. doi: 10.1023/A:1015592129296.
%%
%%
%%
%\bigskip
%%
%%
%% Use the following syntax and markup for your references if
%% the subject of your book is from the field
%% "Computer Science, Economics, Engineering, Geosciences, Life Sciences"
%%
%%
%% Contribution
%\bibitem{basic-contrib} Brown B, Aaron M (2001) The politics of nature. In: Smith J (ed) The rise of modern genomics, 3rd edn. Wiley, New York
%%
%% Online Document
%\bibitem{basic-online} Dod J (1999) Effective Substances. In: The dictionary of substances and their effects. Royal Society of Chemistry. Available via DIALOG. \\
%\url{http://www.rsc.org/dose/title of subordinate document. Cited 15 Jan 1999}
%%
%% Journal article by DOI
%\bibitem{basic-DOI} Slifka MK, Whitton JL (2000) Clinical implications of dysregulated cytokine production. J Mol Med, doi: 10.1007/s001090000086
%%
%% Journal article
%\bibitem{basic-journal} Smith J, Jones M Jr, Houghton L et al (1999) Future of health insurance. N Engl J Med 965:325--329
%%
%% Monograph
%\bibitem{basic-mono} South J, Blass B (2001) The future of modern genomics. Blackwell, London
%
\end{thebibliography}

%\bibliography{bibliography} 
%\bibliographystyle{ieeetr}

\begin{thebibliography}{1}
	
	\bibitem{Aedes_ZIKV}
	E.~B. Kauffman and L.~D. Kramer, ``{Zika Virus Mosquito Vectors: Competence,
		Biology, and Vector Control},'' {\em The Journal of Infectious Diseases},
	vol.~216, pp.~S976--S990, 12 2017.
	
	\bibitem{Sexually_transmitted}
	P.~S. Mead, S.~L. Hills, and J.~T. Brooks, ``Zika virus as a sexually
	transmitted pathogen,'' {\em Current Opinion in Infectious Diseases},
	vol.~31, no.~1, 2018.
	
	\bibitem{Guillian_Barre}
	L.~Barbi, A.~V.~C. Coelho, L.~C. A.~d. Alencar, and S.~Crovella, ``Prevalence
	of guillain-barr{\'e}syndrome among zika virus infected cases: a systematic
	review and meta-analysis,'' {\em The Brazilian Journal of Infectious
		Diseases}, vol.~22, no.~2, pp.~137--141, 2018.
	
	\bibitem{Microcephaly}
	A.~Q.~C. Araujo, M.~T.~T. Silva, and A.~P. Q.~C. Araujo, ``{Zika
		virus-associated neurological disorders: a review},'' {\em Brain}, vol.~139,
	pp.~2122--2130, 06 2016.
	
	\bibitem{ZIKV_SA}
	A.~S. Fauci and D.~M. Morens, ``Zika virus in the americas ---yet another
	arbovirus threat,'' {\em New England Journal of Medicine}, vol.~374,
	pp.~601--604, 2021/08/27 2016.
	
	\bibitem{P_matrix}
	P.~Heidrich, Y.~Jayathunga, W.~Bock, and T.~G{\"o}tz, ``Prediction of dengue
	cases based on human mobility and seasonality---an example for the city of
	jakarta,'' {\em Mathematical Methods in the Applied Sciences}, vol.~n/a,
	2021/08/27 2021.
	
	\bibitem{Yashika}
	W.~Bock and Y.~Jayathunga, ``Optimal control and basic reproduction numbers for
	a compartmental spatial multipatch dengue model,'' {\em Mathematical Methods
		in the Applied Sciences}, vol.~41, pp.~3231--3245, 2021/08/27 2018.
	
	\bibitem{Bichara}
	D.~Bichara and A.~Iggidr, ``Multi-patch and multi-group epidemic models: a new
	framework,'' {\em Journal of Mathematical Biology}, vol.~77, no.~1,
	pp.~107--134, 2018.
	
	\bibitem{Hethcote}
	H.~W. Hethcote, ``The mathematics of infectious diseases,'' {\em SIAM Review},
	vol.~42, no.~4, pp.~599--653, 2000.
	
\end{thebibliography}

\end{document}
