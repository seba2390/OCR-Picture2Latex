\begin{figure*}[!tbp]
	\begin{subfigure}{\textwidth}
		\includegraphics[width=\textwidth]{images/opcooc-perc-ts.pdf}
	\end{subfigure}
	\caption{
		Daily proportions of all edge weights (gray lines) representing co-occurred opinions pairs.
		The red lines show three selected dynamics: 
		continuous strong association (left panel), declining weight (center panel) and increasing weight (right panel).
		At any given time point, the values on all lines sum to one.
	}
	\label{subfig:mapping_1}
\end{figure*}

\begin{figure}[!tbp]
	\centering
	\begin{subfigure}{0.48\textwidth}
		\includegraphics[width=\textwidth]{images/opcooc-net-stats-ts.pdf}
	\end{subfigure}
	\caption{
		Dynamics of mean centrality measures in the opinion co-occurrence network for conspiracy (red lines) and non-conspiracy opinions (gray lines).
		The green line shows the news coverage ratios from Media Cloud \citep{roberts2021media}.
		The highlighted area shows a spike in the news coverage, which coincides with a decrease in the centrality of conspiracy opinions.
	}
	\label{subfig:mapping_2}
\end{figure}
\begin{figure}[!tbp]
		\begin{subfigure}{0.49\textwidth}
		\includegraphics[width=\textwidth]{images/opcooc-net-viz.pdf}
	\end{subfigure}
	\caption{A visualization of the co-occurrence network in late September 2020 --- node sizes and colors indicate the degrees and betweenness values, respectivelly.}
	\label{subfig:mapping_3}
\end{figure}