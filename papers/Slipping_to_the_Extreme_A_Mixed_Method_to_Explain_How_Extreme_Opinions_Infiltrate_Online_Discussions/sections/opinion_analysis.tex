
\section{Opinion Dynamics and Network Centrality}
\label{sec:opinion-analysis}

This section first examines the relative importance of opinions in online discussions, obtained from a large sample of machine-labeled postings.
This allows the application of the qualitative-defined coding schema (see \cref{subsec:qual-study}) to a significantly larger sample of postings, reducing the unavoidable selection bias of the qualitative study. 
Next, we study the dynamics of opinion co-occurrences. We note that, due to large overlaps in posting times and similarities in topics, the analysis of opinions in this section is conducted on two topic groups: \textit{2019-20 Australian bushfire season}, \textit{climate change}, and \textit{Covid-19}, \textit{vaccination} (also shown in \Cref{tab:keywords}).


\begin{figure}[!tbp]
	\centering
	\includegraphics[width=0.48\textwidth]{images/opinions-n.pdf}
	\caption{
        The usage frequency of each of opinions in a large sample of machine-labeled data shows a long-tail distribution.
        Four of the top six opinions endorse a conspiracy theory (shown in gray).
	}
	\label{fig:opinion-n}
\end{figure}
\noindent {\bf Experimental setups.}
After completing the last iteration of the dataset augmentation ($L_7$), we train the topic and opinion classifiers (see \Cref{subsec:dataset-augmentation}) on all available training data.
We apply these classifiers to all available unlabelled samples ---
$22,965,816$ postings in total.
The vast majority of these ($21,266,038$) are off-topic, i.e., with no opinion associated. 
This is expected given the broad keyword sampling of our unlabeled dataset.
The remainder of $1,699,778$ postings are labeled with at least one opinion, and $313,720$ postings were associated with more than one opinion.
This creates $2,089,336$ posting-opinion relations, which we use in the rest of this section to analyze the dynamics of opinions. 
We manually identify the opinion labels that relate to \textit{conspiracy theories}
and we discuss them in the experimental results.
We show in the appendix~\citep{appendix} the complete list of opinions and those relating to conspiracy theories.

\subsection{Opinion Frequency Distribution} %
\label{subsec:frequency-analysis}
We show in \Cref{fig:opinion-n} the frequency distribution of opinions in the machine-labeled data.
Unsurprisingly (in hindsight), the size distribution for opinions is long-tailed, commonly emerging in online measurements.
This translates into a relatively small number of opinions monopolizing the online debate.
Perhaps more surprisingly, most of the prevalent opinions are linked to conspiracy theories; 
four among the top six most popular opinions are conspiracy theories, including 
``Covid-19 is a scam/plan of the elites'' (2nd most frequent opinion),
``5G/smart tech is unsafe/a scam/a way of controlling people'' (4th),
``China is responsible for Covid-19'' (5th), and 
``Covid-19 is a government tool to increase the powers of the state and surveillance/control of citizens'' (6th). 
This showcases the advantages of our mixed-method approach: our qualitative case studies (see \cref{sec:case_studies}) identify the importance of conspiracy theories in the online debate; still, they could not assess the scope of their importance relative to all the other opinions.
We further show in the appendix~\citep{appendix} the daily relative frequency of top opinions.

\subsection{Centrality Dynamics in Opinion Networks}

\begin{figure*}[!tbp]
	\begin{subfigure}{\textwidth}
		\includegraphics[width=\textwidth]{images/opcooc-perc-ts.pdf}
	\end{subfigure}
	\caption{
		Daily proportions of all edge weights (gray lines) representing co-occurred opinions pairs.
		The red lines show three selected dynamics: 
		continuous strong association (left panel), declining weight (center panel) and increasing weight (right panel).
		At any given time point, the values on all lines sum to one.
	}
	\label{subfig:mapping_1}
\end{figure*}

\begin{figure}[!tbp]
	\centering
	\begin{subfigure}{0.48\textwidth}
		\includegraphics[width=\textwidth]{images/opcooc-net-stats-ts.pdf}
	\end{subfigure}
	\caption{
		Dynamics of mean centrality measures in the opinion co-occurrence network for conspiracy (red lines) and non-conspiracy opinions (gray lines).
		The green line shows the news coverage ratios from Media Cloud \citep{roberts2021media}.
		The highlighted area shows a spike in the news coverage, which coincides with a decrease in the centrality of conspiracy opinions.
	}
	\label{subfig:mapping_2}
\end{figure}
\begin{figure}[!tbp]
		\begin{subfigure}{0.49\textwidth}
		\includegraphics[width=\textwidth]{images/opcooc-net-viz.pdf}
	\end{subfigure}
	\caption{A visualization of the co-occurrence network in late September 2020 --- node sizes and colors indicate the degrees and betweenness values, respectivelly.}
	\label{subfig:mapping_3}
\end{figure}

\textbf{Build the opinion co-occurrence network.}
It is common that postings express multiple opinions. 
Such co-occurring opinions help identify central opinions, which usually spawn new emerging opinions. 
Here we investigate this process by building the opinion co-occurrence network in the online conversation of the topic \textit{2019-20 Australian bushfire season}.
In the network, the nodes represent the 27 opinions captured during the bushfire conversation.
An edge between two nodes exists when both opinions are present together in at least one posting. 
The node degree of a given opinion node represents the number of opinions that co-occurred with it. 
The edges are weighted by the number of postings in which their connected node opinions co-occur.

\textbf{Dynamics of topic co-occurrence intensity.}
We first investigate the evolution of opinion co-occurrences. 
In \Cref{subfig:mapping_1}, we plot the daily proportions of weights of each edge among all edges, from September 2019 to January 2020. 
We showcase three selected edges (i.e., opinion pairs) that are representative of three types of temporal dynamics:
\begin{itemize}
    \item A continuous and relatively strong association between prevalent opinions --- ``Climate change crisis isn't real'' and ``Climate change is a UN hoax'', the latter notably being a conspiracy theory. 
    \item Associations with declining relative frequencies --- ``Greta Thunberg should not have a platform or influence as a climate...'' and ``Women and girls don't deserve a voice in the public sphere''.
    \item Rising associations --- ``bushfires and climate change not related'' and ``bushfires were caused by random arsonists''.
\end{itemize}
\rev{These three types of co-occurrence dynamics can inform how potentially harmful opinions are selectively co-used with other opinions, and can serve as early warnings for their adoption (and possibly normalization) by participants.
However, to gain a structural understanding of the role of harmful opinions in the broader debate, we next study the structure and dynamics of the opinion co-occurrence network.}

\textbf{Centrality of conspiracy opinions and news ratio.}
Here, we study the importance of conspiracy opinions over time measured using their centrality in the dynamic network of opinions.
The network is constructed for each day, and an edge exists if the pair of opinions co-occurs at least once.
We measure nodes' centrality using three measures: betweenness, closeness, and node degrees.
\Cref{subfig:mapping_2} shows the average centrality for each measure for the $8$ conspiracy and $19$ non-conspiracy opinions.
We also depict the attention dedicated by the Australian news media to the bushfires during the same period. 
We estimate the latter using the news coverage ratio --- the percentage of articles dedicated to the topic over all captured articles in a day ---
crawled using the Media Cloud \citep{roberts2021media}. 

We observe that the conspiracy opinions have higher mean betweenness than the non-conspiracy opinions in September 2019 and again in November 2019.
It is only in January 2020 that their mean centrality decreases consistently, which, interestingly, corresponds to a significant uptick in the attention given by the media.
\revK{This might suggest that the diffusion of more authoritative content by the news media, together with the participation of their readership, crowded out conspiracy opinions and marginalized their impact.}


\textbf{A launching pad for fringe opinions.}
The episodically high centrality of conspiracy opinions suggests they are selectively used in conjunction with many other opinions.
\revK{We posit that contested conspiracy opinions are leveraged together with more accepted and mainstream opinions to rationalize and popularize them.
Furthermore, they are used with existing conspiracy opinions to amplify the influence.}
We test this hypothesis by mapping, in \Cref{subfig:mapping_3}, the opinion co-occurrence network from posts published over 14 days in late September 2019, i.e., the period when the betweenness for conspiracy opinions is at its peak.
At the network's center lie opinions with both high betweenness and high degree, such as ``United Nations is corrupt'' or ``Climate change isn't real''.
These are long-lasting, general-purpose opinions that we frequently find throughout our dataset.
These are also the backbone on which the conspiracy theories build to increase their presence in the narrative.
We find the closely related and very central ``Climate change is a UN hoax'', but also more fringe opinions towards the periphery of the network --- such as 
``Bushfires linked to secret elites' secret technology (chemtrails, HAARP, HSRN, geoengineering)'', 
``bushfires deliberately lit to promote a climate change agenda'' and 
``Australia should not be a member of the United Nations''.





