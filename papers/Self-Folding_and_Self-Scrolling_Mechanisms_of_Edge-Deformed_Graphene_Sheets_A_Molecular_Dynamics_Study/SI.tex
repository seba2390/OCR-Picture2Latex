\documentclass[reprint,superscriptaddress,onecolumn]{revtex4-2}
\usepackage{graphicx}

\renewcommand{\thefigure}{S\arabic{figure}}

\begin{document}

\title{Supplementary Information for\\Self-Folding and Self-Scrolling Mechanisms of Edge-Deformed\\Graphene Sheets: A Molecular Dynamics Study}

\author{Marcelo Lopes Pereira Junior}
 \affiliation{Institute of Physics, University of Bras\'{i}lia, 70.919-970, Bras\'{i}lia, Brazil.}
 \author{Luiz Antonio Ribeiro Junior}
 \affiliation{Institute of Physics, University of Bras\'{i}lia, 70.919-970, Bras\'{i}lia, Brazil.}
\maketitle

\begin{figure}[!h]
\centering
\includegraphics[width=0.5\linewidth]{Figures/SI/figS1.pdf}
\caption{Time evolution of the potential energy for the MD simulations of the following zigzag ESG sheets at 300 K: (a) ZESG(0,2$\pi$), (b) ZESG(2$\pi$,2$\pi$), (c) ZESG(2$\pi$,3$\pi$), (d) ZESG(0,3$\pi$), and (e) ZESG(3$\pi$,3$\pi$). For the sake of comparison, the blue, red, and green dashed lines denote the value for the average potential energies (in the last 100 ps) for the ZESG(0,2$\pi$), ZESG(0,3$\pi$), and ZESG(2$\pi$,3$\pi$), respectively. Note that this average value is similar among the related ZESG(0,$\theta$) $\theta$) cases. As expected, the curve profiles suggest that scrolled and folded configurations are energetically favorable also for the zigzag cases. Moreover, from this figure, it is also possible to conclude that the edge topology does not affect the folding and scrolling mechanisms of the graphene sheets since the trends presented by the time evolution of the potential energy are similar between the zigzag and armchair ESG cases.}
\label{figs1}
\end{figure}

\begin{figure}[!h]
\centering
\includegraphics[width=\linewidth]{Figures/SI/figS2.pdf}
\caption{Representative MD snapshots for the self-deformation process of AESG($-2\pi$,$2\pi$) at 300 K. The system evolves to a collapsed structure with two-folded edges due to the formation of carbon-carbon covalent bonds since the edges are very close to each other.}
\label{figs2}
\end{figure}

\begin{figure}[!h]
\centering
\includegraphics[width=0.5\linewidth]{Figures/SI/figS3.pdf}
\caption{Time evolution of the potential energy for the MD simulations of the following zigzag ESG sheets at 300 K: (a) AESG(-2$\pi$,2$\pi$), (b) AESG(-3$\pi$,2$\pi$), and (c) AESG(-3$\pi$,3$\pi$). For the sake of comparison, the blue, red, and green dashed lines denote the value for the average potential energies (in the last 100 ps) for the AESG(-2$\pi$,2$\pi$), AESG(-3$\pi$,2$\pi$), and AESG(-3$\pi$,3$\pi$), respectively. Note that this average value is similar among the related AESG(-$\phi$,$\theta$) cases. During the dynamical process of these structures, there is a downhill trend for the potential energy values that ends about 10 ps, 40 ps, and 60 ps for the AESG(-2$\pi$,2$\pi$), AESG(-3$\pi$,2$\pi$), and AESG(-3$\pi$,3$\pi$) cases, respectively. After these periods, the total potential energy stabilizes, and a flat region persists until the end of the simulation. The plateaus denote the formation of the final structures.}
\label{figs3}
\end{figure}

\end{document}








