 %%%%%%%%%%%%%%%%%%%%%%%%%%%%%%%%%%%%%%%
% AUTHOR: Christos Faloutsos
% INSTITUTION: CMU
% DATE: April 2019
% GOAL: to streamline the paper presentations
%%%%%%%%%%%%%%%%%%%%%%%%%%%%%%%%%%%%%%%

Given a graph semi-supervised learning (SSL) setting, i.e. few node labels are known, how can we efficiently assign the labels to the rest of nodes in the million-scale graph? Majority of the studies focus on improving the accuracy on small datasets with more and more complex models.
However, other than solely pursuing accuracy, one should also attach importance to how to better understand the datasets. That is, how can we distinguish whether the network has effect or not under graph SSL setting? Without assuming having all labels, identification of homophily and heterophily has never been well solved.
To this end, we propose \method, that achieves these goals.
It is based on two insights:
(a) we introduce and exploit the {\em \ndiff}~(\nd) insight, that not all 1-hop neighbors are equal: if they are well-connected, they have stronger influence on the target node, and
(b) we test whether a dataset has {\em \neteffect} to analyze whether its structure contains useless information, i.e., uniformly connected to the neighbors with different labels.

In short, \method has the following advantages:
(a) {\em \Accurate}, thanks to \ndiff~(\nd) and \neteffect analysis~(\nea);
(b) {\em \explain} with \nea and compatibility matrix estimation;
(c) {\em \thrift \space and \scale}, requiring modest computational resources (no expensive GPUs), and scaling linearly with the input size, up to graphs with millions of nodes;
(d) {\em \theory}, providing several provable closed formulas and theoretical guarantees.
Experiments on public real-world graph datasets illustrate that \method outperforms top competitors in terms of accuracy, it is thrifty (requiring only stock CPU servers) and no need for parameter tuning. Moreover, it can handle large graphs with {\em$2.4$M nodes} and {\em$62$M edges} in $1.5$ hours, while most competitors run out of memory or time limit exceeded.