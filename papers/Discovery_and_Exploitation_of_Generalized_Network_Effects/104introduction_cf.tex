
%%%%%%%%%%%%%%%%%%%%%%%%%%%%%%%%%%%%%%%
% AUTHOR: Christos Faloutsos
% INSTITUTION: CMU
% DATE: April 2019
% GOAL: to streamline the paper presentations
%%%%%%%%%%%%%%%%%%%%%%%%%%%%%%%%%%%%%%%


Given a large, undirected, and unweighted graph with few labeled nodes, how can we infer the labels of remaining unlabeled nodes, often without node features? 
Node classification is often employed to infer labels on large real-world graphs, since manual labeling is expensive and time-consuming.
For example, in social networks with millions of users, identifying even a fraction (say $5\%$) of users' groups is prohibitive, which limits the application of 
methods that assume a large fraction of labels are given. 
Moreover, node features are frequently missing in real-world graphs.
For those methods that require node features in classification, they create the structural features based on the graph \cite{Fey/Lenssen/2019, hamilton2017inductive, hamilton2017representation}, such as using the one-hot encoding of node degree.

% \begin{figure*}[]
% \centering
% \captionsetup[subfloat]{captionskip=-1pt}
% \subfloat[\label{fig:c1} \nef: Compatibility Matrix Estimation]
% {\includegraphics[scale=0.46]{FIG/cj2.pdf}}
% \rulesep
% \subfloat[\label{fig:c2} \nd: Neighbor Differentiation]
% {\includegraphics[scale=0.46]{FIG/cj1.pdf}}
% \rulesep
% \subfloat[\label{fig:c3} Scalability]
% {\includegraphics[scale=0.46]{FIG/scale_new.pdf}}
% \vspace{-2mm}
% \caption{\label{fig:crown} \underline{\smash{\method is \effect, \explain \xspace, and \scale.}} (a) Thanks to \NEF, \method explains the dataset by precisely estimating the compatibility matrix, observing both heterophily and homophily. (b) Thanks to \emphasis, \method predicts the label of the gray node \textit{X} correctly, while \linbp fails. (c) \method is fast and scales linearly with the number of edges.}
% \end{figure*}

\begin{figure*}[]
\centering
\captionsetup[subfloat]{captionskip=-1pt}
\subfloat[\label{fig:c1} Compatibility Matrix Estimation]
{\includegraphics[scale=0.45]{FIG/cj2.pdf}}
\rulesep
\subfloat[\label{fig:c2} Accuracy vs. Time]
{\includegraphics[scale=0.5]{FIG/accvstime.pdf}}
\rulesep
\subfloat[\label{fig:c3} Scalability]
{\includegraphics[scale=0.5]{FIG/scale_new.pdf}}
\vspace{-2mm}
\caption{\label{fig:crown} \underline{\smash{\method is \effect, \explain \xspace, and \scale.}} (a) Thanks to \NEF, \method explains the dataset by precisely estimating the compatibility matrix, observing both heterophily and homophily. 
(b) \method outperforms the best competitor by $12.8\%$ accuracy, while being $3.9\times$ faster than the fastest competitor.
(c) \method is fast and scales linearly with the number of edges. 
See Introduction for more details.
}
\end{figure*}

Previous works on node classification have two main limitations.
First, they ignore the complex \neteffect of real-world graphs and understand their characteristic as either homophily or heterophily.
Both the co-existing case of homophily and heterophily, which we call \xophily, and the unexisting case with no \neteffect, have been neglected.
Second, they either 
a) ignore different influences of neighboring nodes during inference, or 
b) require extensive computation to give dynamic weights to the adjacency matrix.
% Those limitations make an existing approach fail to generalize well to various modalities of real-world graphs.
% In this paper, we propose and resolve two insights:
% \ben
% \item \neteffect (\nef): the graph can be homophily, heterophily, both, or none.
% \item \ndiff (\nd): not all neighbors are\\equally influential.
% \een
In this work, we address these two challenges and consider the dynamic and complex relationships between neighboring nodes with two insights \emph{generalized network-effect} and \emph{neighbor-differentiation} for designing an accurate and efficient approach for node classification.

{\bf \nef (\neteffect)}:
The first goal is to analyze the \neteffect of a graph (i.e., homophily, heterophily, any combination (\xophily), or none) in a principled, class-conditional way.
The challenge is usually avoided in literature:
inference-based methods assume that the relationship is given by domain experts \cite{DBLP:journals/pvldb/GatterbauerGKF15};
deep graph models either assume homophily \cite{kipf2016semi, wu2019simplifying} or are not designed to distinguish different non-homophily cases, i.e., heterophily or no \nef \cite{zhu2020beyond, lim2021large}.
% Jaemin: Please check whether this is right.
% Note that a graph whose nodes are connected with each other uniformly, regardless of their labels, has a distinct characteristic from heterophily graphs in that node labels are indistinguishable from the graph structure.

{\bf \nd (\ndiff)}:
The second goal is to approximate different influence levels of neighboring nodes effectively.
% Existing works to measure and utilize the influence levels in node classification require extensive computation.
Existing works~\cite{eswaran2020higher, velivckovic2017graph} require extensive computation to measure the influence levels in node classification.
% For instance, \hols \cite{eswaran2020higher} solves \nd by mining $k-$cliques, but listing all of them is slow and may run out of disk space. % Jaemin: More detailed explanation?
For instance, Graph Attention Networks (GAT)~\cite{velivckovic2017graph} learns more than one relationship for each neighbor, while heavily relying on the node features.
% and suffering from poor scalability without GPUs.

We provide an informal definition of the problem, and mark the challenges that are resolved in our approach in \blue{blue}:
% \vspace*{-0.02in}
\begin{iprob}~
    \vspace{-0.12cm}
	\begin{itemize}[leftmargin=20pt]
		\item {\bfseries Given} an undirected and unweighted graph
		\bit
			\item with few labeled nodes,
			\item without node features,
		\eit
		\item {\bfseries Infer} the labels of all the remaining nodes
		\bit
		    \item \blue{{\bfseries accurately} under any types of network effects,}
			\item \blue{{\bfseries explaining} the predictions to human experts,}
% 			\item \red{{\bfseries estimate} the compatibility matrix,}
			\item \blue{{\bfseries efficiently}, in large-scale graphs.
                % with scalability.
                }
		\eit
	\end{itemize}
\end{iprob}

{\bf Our solutions}:
We propose Generalized Network Effect Analysis (\nea), an algorithm to statistically test \nef of a graph with only a few observed node labels.
\nea analyzes the relationships between all pairs of different classes efficiently.
Figure~\ref{fig:dis} shows that surprisingly many large real-world datasets known as heterophily graphs have little \nef.

Our proposed \method, is a principled approach using both insights of \nef and \nd to conduct accurate node classification on large graphs with explainability.
The explainability is built upon the combination of influential neighbors (\nd) and the \emph{compatibility matrix} that we carefully and automatically estimate (see Lemma~\ref{lem:nef}).
Figure~\ref{fig:crown} illustrates the advantages of \method.
In Figure~\ref{fig:c1}, \method shows the interrelations of classes by estimating a \emph{compatibility matrix}. It implies that the first half follows heterophily, while the other half follows homophily.
Figure~\ref{fig:c2} shows that \method outperforms the competitors by both accuracy and run time on the heterophily ``Pokec-Gender'' dataset (largest with 22M edges).
Finally, Figure~\ref{fig:c3} shows the linear scalability of \method with the number of edges. It is $9\times$ faster than most of the competitors, and requires only {\em 12 minutes} on a large real-world graph with over $22$M edges.

In summary, the advantages of \method are
\ben
\item {\bf \Accurate}, thanks to the precise estimation of the compatibility matrix,
% (which works for homophily, heterophily, or any combination -- `X'-ophily)
and the reliable measurement of the different importance of neighbors,
% \item {\bf \explain}, interpreting the datasets with \neteffect analysis (\nea) and discovering the surprising results of existing heterophily graph datasets, 
\item {\bf \explain}, interpreting the datasets with estimated compatibility matrices, which work for homophily, heterophily, or any combination -- \xophily,
\item {\bf \scale}, scaling linearly with the input size,
\item {\bf \theory}, providing a tight bound of convergence for the random walks, and the closed-form formula for the compatibility matrix (see Lemma~\ref{lem:crw2} and \ref{lem:nef}).
% \item {\bf \automatic}: no parameters for the user to tune, in contrast to most competitors.
% (Our 3 parameters accept reasonable defaults.)
\een

% {\bf Reproducibility}: Our implemented source code and preprocessed datasets are available online\footnote{\codeurl}.

{\bf Reproducibility}: Our implemented source code, preprocessed datasets, and supplementary material (with proofs and additional experiments) are available online at \urlcolor{\codeurl}.


% \blue{@Jeremy, @Jaemin I think the below is redundant which can be done away with. Let me know what you think.}
% The rest of the paper is organized as follows. 
% We give the survey and background in Section~\ref{sec:background}.
% We firstly propose \nea to analyze \nef in Section~\ref{sec:neteffect} along with the surprising discoveries, and then introduce our novel algorithm \method for node inference in Section~\ref{sec:meth}. 
% The experiments are presented in Section~\ref{sec:exp}. 
% Section~\ref{sec:concl} concludes.
