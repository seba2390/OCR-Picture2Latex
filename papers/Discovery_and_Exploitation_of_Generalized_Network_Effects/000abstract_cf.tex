 %%%%%%%%%%%%%%%%%%%%%%%%%%%%%%%%%%%%%%%
% AUTHOR: Christos Faloutsos
% INSTITUTION: CMU
% DATE: April 2019
% GOAL: to streamline the paper presentations
%%%%%%%%%%%%%%%%%%%%%%%%%%%%%%%%%%%%%%%
Given a large graph with few node labels, how can we 
(a) identify whether there is \neteffect~(\nef) of the graph or not, 
(b) estimate \nef to explain the interrelations among node classes, and 
(c) exploit \nef to improve downstream tasks such as predicting the unknown labels accurately and efficiently? 
The knowledge of \nef is valuable for various tasks like node classification, and targeted advertising. 
However, identifying and understanding \nef such as homophily, heterophily or their combination is challenging in real-world graphs due to limited availability of node labels and noisy edges. 
% \reminder{FYI: minor touch-ups below:}
We propose \method, a graph mining approach to address the above issues,
enjoying the following properties:
% framework for analyzing and utilizing the \nef in large real-world graphs with limited node labels. 
% It consists of three contributions: 
(i) \theory: a statistical test to determine the presence of \nef in a graph with few node labels; 
(ii) \general and \explain: a closed-form solution to estimate the specific type of \nef observed; and 
(iii) \accurate and \scale: the integration of \nef for accurate and fast node classification.
Applied on public, real-world graphs, \method discovers the unexpected absence of \nef in numerous graphs, which were recognized to exhibit heterophily. 
Further, we show that incorporating \nef is effective on node classification.
% , and \method outperforms top competitors in terms of accuracy and run time.
On a large real-world graph with {\em $1.6$M} nodes and {\em$22.3$M} edges, \method achieves {\bf over $\bf 7\times$} speedup ({\em $14$ minutes} vs. $2$ hours) compared to most competitors.

% Given a large graph with few node labels, how can we (a) identify whether there is generalized network-effects (GNE) of the graph or not, (b) estimate GNE to explain the interrelations among node classes, and (c) exploit GNE to improve downstream tasks such as predicting the unknown labels accurately and efficiently? The knowledge of GNE is valuable for various tasks like node classification and targeted advertising. However, identifying and understanding GNE such as homophily, heterophily or their combination is challenging in real-world graphs due to limited availability of node labels and noisy edges. We propose NetEffect, a graph mining approach to address the above issues, enjoying the following properties: (i) Principled: a statistical test to determine the presence of GNE in a graph with few node labels; (ii) General and Explainable: a closed-form solution to estimate the specific type of GNE observed; and (iii) Accurate and Scalable: the integration of GNE for accurate and fast node classification. Applied on public, real-world graphs, NetEffect discovers the unexpected absence of GNE in numerous graphs, which previously thought to exhibit heterophily. Further, we show that incorporating GNE is effective on node classification. On a large real-world graph with 1.6M nodes and 22.3M edges, NetEffect achieves over 7 times speedup (14 minutes vs. 2 hours) compared to most competitors.

%%%%%%%%%%%%%%%%%%%%%IJCAI version of abstract%%%%%%%%%%%%%%%
% This work proposes Network Effect Analysis~(\nea) and \method, which are based on two insights:
% (a) the \neteffect~(\nef) insight: a graph can exhibit not only one of homophily and heterophily, but also both or none in a label-wise manner, and
% % (a) the \neteffect~(\nef) insight: a graph can exhibit homophily, heterophily, both, or none, and
% (b) the \ndiff~(\nd) insight: if neighbors are better connected in the structure, then they have stronger influence on the target node.

% \nea provides a statistical test to check whether a graph exhibits \neteffect or not, and surprisingly discovers the absence of \nef in many real-world graphs known to have heterophily.
% % Compared with previous works, 
% % for node classification:
% \method solves the node classification problem with notable advantages:
% (a) {\em \accurate}, thanks to the \neteffect and \ndiff~(\nd) insights;
% (b) {\em \explain}, precisely estimating the compatibility matrix;
% % with the support of \nea;
% (c) {\em \scale}, being linear with the input size and handling graphs with millions of nodes; and
% (d) {\em \theory}, with closed-form formula and theoretical guarantee.
% Applied on eight real-world graph datasets, \method outperforms top competitors in terms of accuracy and run time, requiring only stock CPU servers. 
% % On a large real-world graph with {\em$1.6$M nodes} and {\em$22.3$M edges}, \method needs only $12$ minutes, while most competitors take more than $2$ hours.
% On a large real-world graph with {\em$1.6$M nodes} and {\em$22.3$M edges}, \method achieves $\geq 9\times$ speedup (12 minutes vs. 2 hours) compared to most competitors.

%%%%%%%%%%%%%%%%%%%%%%%%%%%%%%%%%%%%%%%%%%%%%%%%%%%%%%%%%%%%%%%%%%%%%%

% Given a node label inference problem with few observed node labels, how can we 
% (a) identify whether there are \neteffect and 
% (b) efficiently assign labels to the rest of nodes in a large-scale graph? 
% Moreover, we would like our method to be accurate, explainable, and scalable.
% To achieve these goals, we propose \nea and \method, which are based on two insights:
% (a) the \neteffect~(\nef) insight: the graphs can be homophily, heterophily, both, or none, and
% (b) the \ndiff~(\nd) insight: if the neighbors are more well-connected, they have stronger influence on the target node.

% In summary, \nea provides statistical tests to check whether a graph exhibits \neteffect or not, and surprisingly discovers the absence of \nef in many real-world heterophily graphs;
% for node label inference problem, \method has all the above advantages:
% (a) {\em \Accurate}, thanks to \nef and \nd;
% (b) {\em \explain}, precisely estimating the compatibility matrix with the help of \nea;
% (c) {\em \scale}, scaling linearly with the input size, up to graphs with millions of nodes; and
% (d) {\em \theory}, with closed-form formula and theoretical guarantee.
% Applied on public real-world graph datasets, \method outperforms top competitors in terms of accuracy and run time, while requiring only stock CPU servers to be fast. 
% On a large real-world graph with {\em$1.6$M nodes} and {\em$22.3$M edges}, \method needs only $12$ minutes, while most competitors take more than $2$ hours.

% Given a large graph with few node labels, how can we 
% (a) identify the mixed network-effect of the graph and 
% (b) predict the unknown labels accurately and efficiently?
% This work proposes Network Effect Analysis (NEA) and UltraProp, which are based on two insights:
% (a) the network-effect (NE) insight: a graph can exhibit homophily, heterophily, both, or none, and
% (b) the neighbor-differentiation (ND) insight: if neighbors are better connected in the structure, they have stronger influence on the target node.

% NEA provides statistical tests to check whether a graph exhibits network-effect or not, and surprisingly discovers the absence of NE in many real-world heterophily graphs.
% UltraProp solves the node classification problem, and has notable advantages:
% (a) Accurate, thanks to the network-effect (NE) and the neighbor-differentiation (ND) insights;
% (b) Explainable, precisely estimating the compatibility matrix;
% (c) Scalable, being linear with the input size and handling graphs with millions of nodes; and
% (d) Principled, with closed-form formula and theoretical guarantee.
% Applied on eight public real-world graph datasets, UltraProp outperforms top competitors in terms of accuracy and run time, while requiring only stock CPU servers. 
% On a large real-world graph with 1.6M nodes and 22.3M edges, UltraProp achieves more than 9 times speedup (12 minutes vs. 2 hours) comparing to most competitors.