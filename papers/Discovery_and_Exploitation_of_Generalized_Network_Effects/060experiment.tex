
%%%%%%%%%%%%%%%%%%%%%%%%%%%%%%%%%%%%%%%
% AUTHOR: Christos Faloutsos
% INSTITUTION: CMU
% DATE: April 2019
% GOAL: to streamline the paper presentations
%%%%%%%%%%%%%%%%%%%%%%%%%%%%%%%%%%%%%%%

In this section, we aims to answer the following questions:
\begin{compactenum}[{Q}1.]
\item {\bf Accuracy}: How well does \method work on real-world graphs by estimating and exploiting \nef?
\item {\bf Scalability}: How does the running time of \method scale w.r.t. graph size?
\item {\bf Explainability}: How does \method explain the real-world graphs?
\end{compactenum}

\begin{table*}[t]
\begin{minipage} {0.6\linewidth}
\setlength{\tabcolsep}{1pt}
\caption{\underline{\smash{Ablation Study: Estimating compatibility matrix by the proposed}} \underline{\smash{\emphasis is essential.}} Accuracy (\%) is reported in the table. \label{tab:ablation}}
% \vspace{-2mm}
\setlength\fboxsep{0pt}
\centering{\resizebox{1\columnwidth}{!}{
\begin{tabular}{C{3cm} | C{2cm} | C{2.5cm} C{2.2cm} C{2.2cm} C{2cm}}
\hline
\textbf{Datasets} & \nef Strength & \methodhom & \method-EC & \method-A & \method \\
\hline
\textbf{Synthetic} & \multirow{2}{*}{Strong} & 77.7$\pm$0.0 & 68.0$\pm$0.1 & 77.4$\pm$0.0 & \gold{80.5$\pm$0.0} \\
\textbf{Pokec-Gender} &  & 56.9$\pm$0.1 & 64.9$\pm$0.2 & 64.8$\pm$0.2 & \gold{67.3$\pm$0.1} \\
\hline
\textbf{arXiv-Year} (imba.) & \multirow{2}{*}{Weak} & 37.0$\pm$0.3 & 36.5$\pm$1.0 & 35.7$\pm$0.6 & \gold{38.4$\pm$0.0} \\
\textbf{Patent-Year} (imba.) &  & 24.1$\pm$0.0 & 24.0$\pm$0.9 & \gold{28.7$\pm$0.1} & \gold{28.7$\pm$0.0} \\
\hline
\end{tabular}
}}
\end{minipage} \hfill
\setlength{\tabcolsep}{1pt}
\begin{minipage} {0.38\linewidth}
\caption{\underline{\smash{\method is thrifty.}} AWS total dollar amount (\$) is reported in the table. The \blue{blue} and \red{red} fonts denote running a single experiment by \blue{t3.small} and \red{p3.2xlarge}, respectively. 
% Accuracy (\%) is reported in Table~\ref{tab:effecthet} and~\ref{tab:effecthom}. 
\label{tab:dollar}}
\vspace{-2mm}
\setlength\fboxsep{0pt}
\centering{\resizebox{1\columnwidth}{!}{
\begin{tabular}{C{3cm} | C{2.5cm} C{2.5cm}}
\hline
\textbf{Datasets} & \method & \gcn \\
\hline
\textbf{Pokec-Gender}   & \gold{\blue{\$ 0.33 (1.0$\times$)}} & \red{\$ 12.61 (45.0$\times$)} \\
\textbf{Pokec-Locality} & \gold{\blue{\$ 0.53 (1.0$\times$)}} & \red{\$ 13.66 (29.1$\times$)} \\
\hline
\end{tabular}
}}
\end{minipage}
\end{table*}

\begin{figure*}[t]
\begin{minipage} {0.28\linewidth}
\centering
\includegraphics[scale=0.38]{FIG/scale_new.pdf}
\vspace{-3mm}
\caption{\underline{\smash{\method is scalable.}} It is fast and scales linearly with the number of edges.} \label{fig:scale}
\end{minipage} \hfill
\begin{minipage} {0.7\linewidth}
\centering
\subfloat[\label{fig:ex1} ``Synthetic'':\\\xophily w/ Strong \nef]
{\includegraphics[height=1.05in]{FIG/cm_synthetic.pdf}}
\hspace{1.5mm}
\subfloat[\label{fig:ex2} ``Pokec-Gender'':\\Het. w/ Strong \nef]
{\includegraphics[height=1.05in]{FIG/cm_pokecgender1.pdf}}
\hspace{1.5mm}
\subfloat[\label{fig:ex3} ``arXiv-Year'':\\\xophily w/ Weak \nef]
{\includegraphics[height=1.05in]{FIG/cm_arxivyear.pdf}}
\hspace{1.5mm}
\subfloat[\label{fig:ex4} ``Patent-Year'':\\Het. w/ Weak \nef]
{\includegraphics[height=1.05in]{FIG/cm_patentyear1.pdf}}
\vspace{-3mm}
\caption{\label{fig:ex} \underline{\smash{\method is explainable.}} The estimated compatibility matrices are similar to the edge counting matrix (in Figure~\ref{fig:dis}), while being much more robust to the noises.}
\end{minipage}
\end{figure*}

% t3.small (3.2GHz, 2GB RAM): 0.023 * (t / 60)
% p3.2xlarge (1 V100 GPU): 3.06 * (t / 1.92 / 60)
% V100 to A6000: https://lambdalabs.com/blog/nvidia-rtx-a6000-benchmarks/

\subsection*{Experimental Setup}
We introduce the datasets, baselines for node classification, and experimental settings.
Details are given in Appendix~\ref{sec:rep}.

\textit{Datasets.}
We focus on large graphs and include $8$ graph datasets with at least $22.5$K nodes in our evaluation.
% illustrate the details in Appendix~\ref{ssec:datasets}.
The graph statistics are shown in Table~\ref{tab:effecthet} and \ref{tab:effecthom}. 
For each dataset, we sample only a few node labels as training set.
We do this for five times and report the average and standard deviation to omit the biases.
``Synthetic'' is the enlarged graph in Fig.~\ref{fig:c2}, which exhibits \xophily \nef.\looseness=-1
% Noisy edges are injected in the background, and the dense blocks are constructed by randomly generating higher-order structures.
% Details are in Appendix~\ref{ssec:datasets}.

\textit{Baselines.}
We compare \method with five baselines and separate them into four groups: {\bf General GNNs:} \gcn~\cite{kipf2016semi}, and \appnp~\cite{klicpera2018predict}. {\bf Heterophily GNNs:} \mixhop~\cite{abu2019mixhop}, and \gprgnn~\cite{chien2021adaptive}. {\bf BP-based methods:} \hols~\cite{eswaran2020higher}. {\bf Our proposed methods:} \methodhom and \method.
\methodhom is \method using identity matrix as compatibility matrix, which assumes homophily and does not handle \nef.
% The details of baselines are given in Appendix~\ref{ssec:appendixbaselines}.

\textit{Experimental Settings.}
% \methodhom is \method uses identity matrix as compatibility matrix by assuming homophily. 
% For deep graph models, since we focus on the graph without node features, 
For GNNs, the node degrees are transformed into one-hot encoding and used as the node features, which is suggested and implemented by several studies (e.g. GraphSAGE and PyTorch Geometric) \cite{Fey/Lenssen/2019, hamilton2017inductive, hamilton2017representation}.
Experiments are run on a stock Linux server with $3.2$GHz Intel Xeon CPU.
% The details are in Appendix~\ref{ssec:hyper}. 

% \vspace{-1mm}
\subsection{Q1 - Accuracy}
In Table~\ref{tab:effecthet} and \ref{tab:effecthom}, we report the accuracy and running time. 
We highlight the top three from dark to light by \goldc{~}, \silverc{~} and \bronzec{~} denoting the first, second and third place.
\textbf{In summary, \method wins on \xophily, heterophily and homophily graphs.}\looseness=-1

% \subsubsection*{Mixed and Heterophily Dataset}

% \begin{observation}
% \method wins on \xophily, heterophily and homophily graphs.
% \end{observation}

% \paragraph{Homophily} ...
% \paragraph{heterophily} ...
% \paragraph{mixed - 'x-ophily} ...

\textit{\xophily and Heterophily.}
In Table~\ref{tab:effecthet}, \method outperforms all the competitors significantly by more than $34.3\%$ and $12.9\%$ accuracy on ``Synthetic'' and ``Pokec-Gender'', respectively.
These graphs exhibit strong \nef, thus \method boosts the accuracy owing to precise estimations of compatibility matrix. 
The success in ``Synthetic'' further demonstrates that it is general to handle \xophily graphs.
Heterophily GNNs, namely \mixhop and \gprgnn, give results close to majority voting when the observed labels are not adequate. 
With homophily assumption, General GNNs and BP-based methods also not perform well.
Both ``arXiv-Year'' and ``Patent-Year'' are shown to only have weak \nef (Sec.~\ref{ssec:discover}).
In ``arXiv-Year'', \method receives the second place by running $57.4\times$ faster than \mixhop.
In ``Patent-Year'', only \method, \appnp and \mixhop are able to give accuracy higher than majority voting, which is $26.1\%$. 
Even so, \method still outperforms the competitors by estimating a reasonable compatibility matrix (Fig.~\ref{fig:ex4}).
In the cases that \method is faster than \methodhom is because of both the low cost of \methodest, and the lower spectral radius of $\hat{\boldsymbol{H}}^{*}$, leading to a faster convergence while propagating.

% \subsubsection*{Homophily Datasets} 

% \begin{observation}
% \method wins on homophily datasets.
% \end{observation}

\textit{Homophily.}
In Table~\ref{tab:effecthom}, \methodhom outperforms all the competitors on $3$ out of $4$ homophily graphs, namely ``GitHub'', ``arXiv-Category'' and ``Pokec-Locality''. 
\method performs similarly to \methodhom, indicating its generalizability to the homophily graph by estimating near-identity matrices. 
In addition, \methodhom gives competitive results with \hols on ``Facebook'', while being $155.7\times$ faster than \hols.
% General GNNs rely heavily on node features for inference which explains their poor performance.

% \subsubsection*{Ablation Study} 
% \begin{observation}
% Our optimizations makes difference.
% \end{observation}
\textbf{Our optimizations make a difference.}
We evaluate the effect of different compatibility matrices -- 
(i) \method-EC conducts edge counting on the labels of adjacent nodes in the priors, and
(ii) \method-A uses the adjacency matrix instead of \emphasis as the input of \methodest.
To show our advantage over edge counting, during compatibility matrix estimation of each method, we upsample $5\%$ labels to the class with the fewest labels in the datasets with weak \nef.
% , which are class $2$ in ``arXiv-Year'' and class $1$ in ``Patent-Year''. 
% We use the original labels for propagation in the imbalanced datasets.
In Table~\ref{tab:ablation}, we find that \method outperforms all its variants in all datasets.
In the graphs with strong \nef, \method shows its robustness to the structural noises and gives better results.
In the imbalanced graphs, while \method-EC brings its vulnerability to light, \method stays with high accuracy.
This study highlights the importance of a compatibility matrix estimation, as well as forming it into an optimization problem as shown in Lemma~\ref{lem:nef}. 

\subsection{Q2 - Scalability}
\textbf{\method is scalable.}
In Fig.~\ref{fig:scale}, we vary the edge number in ``Pokec-Gender'' and plot against the running time, including both training and inference time.
We ensure the connectivity by removing the nodes, until the edge number is smaller than the target. 
\method scales linearly as expected (Lemma~\ref{lem:complexity}).

% t3.small (3.2GHz, 2GB RAM): 0.023 * (t / 60)
% p3.2xlarge (1 V100 GPU): 3.06 * (t / 1.92 / 60)
% V100 to A6000: https://lambdalabs.com/blog/nvidia-rtx-a6000-benchmarks/
\textbf{\method is thrifty.}
Table~\ref{tab:dollar} shows the estimated AWS dollar cost in ``Pokec-Gender'', assuming that we use a CPU machine for \method, and a GPU machine for \gcn.
\method achieves up to $45\times$ savings. 
It requires only CPU, while comparable speeds by competitors, require GPUs. % (see appendix ** for the run times)
Details in Appendix~\ref{ssec:dollar}.

\subsection{Q3 - Explainability} \label{ssec:qtexplain}

% \begin{observation}
% \method estimates the reasonable compatibility matrices.
% \end{observation}

% \textbf{\method explains the graphs by reasonable estimations on compatibility matrices.}
Fig.~\ref{fig:ex} shows the compatibility matrices that \method recovered.
\textbf{In a nutshell, the results agree well with the intuition.}
\hide{
We illustrate that the estimations of compatibility matrix by \method are reasonable in Fig.~\ref{fig:ex}, so as to interpreting the interrelations of classes extremely well.
The interrelations of shown estimated compatibility matrices are similar to the ones of edge counting in Fig.~\ref{fig:dis}, while being more robust to the noisy neighbors, namely, weakly connected ones.
}% end hide

For ``Synthetic'' (Fig.~\ref{fig:ex1}), \method matches the exact answer that we use to generate the graph.
For ``Pokec-Gender'' (Fig.~\ref{fig:ex2}), \method report heterophily.
% (opposite genders attract~\cite{ghosh2019quantifying}).
% successfully estimates that people tend to connect to the ones with opposite gender.
% Notice that the users in the dataset had average ages $25.4$ and $24.2$ for males and females respectively.
% \reminder{shall we cite ghosh2019quantifying for external validity of this}
This corresponds to the fact that people incline to have more opposite gender interactions during their reproductive age \cite{ghosh2019quantifying}, where the average ages of male and female in the dataset are $25.4$ and $24.2$, respectively.
\hide{
Although ``arXiv-Year'' and ``Patent-Year'' do not have strong \nef, \method still gives an estimated compatibility matrices making much sense in the real-world (Fig.~\ref{fig:ex3} and~\ref{fig:ex4}), where the papers and patents tend to cite to the ones whose published dates are relatively close to them.
}% end
For ``arXiv-Year'' and ``Patent-Year'',
% do not have strong \nef, \method still gives an estimated compatibility matrices making much sense in the real-world 
\method find that the papers and patents tend to cite the ones that got published in nearby dates, which also agrees with intuition (Fig.~\ref{fig:ex3} and~\ref{fig:ex4}).
We omit the results on homophily graphs, for brevity, where all our estimations are near-identity matrices, as expected.

% \begin{figure}[t]
% 	\centering
% 	\subfloat[\label{fig:abl1} Synthetic: \method (left), -EC (middle), and -A (right)]
% 	{\includegraphics[height=1in]{FIG/cm_synthetic.pdf}
% 	\includegraphics[height=1in]{FIG/cm_synthetic_ec.pdf}
% 	\includegraphics[height=1in]{FIG/cm_synthetic_a.pdf}} \\
% 	\vspace{-3mm}
% 	\subfloat[\label{fig:abl2} ``Pokec-Gender'': \method (left), -EC (middle), and -A (right)]
% 	{\includegraphics[height=1in]{FIG/cm_pokecgender1.pdf}
% 	\includegraphics[height=1in]{FIG/cm_pokecgender1_ec.pdf}
% 	\includegraphics[height=1in]{FIG/cm_pokecgender1_a.pdf}}
% 	\vspace{-3mm}
% 	\caption{\label{fig:abl} \underline{Ablation Study: \method gives more explain-}\\
% 	\underline{able results.}}
% \end{figure}
% \begin{figure}[]
%     % \vspace{-7mm}
% 	\centering
% 	% \subfloat[\label{fig:ex1} ``Synthetic'': \xophily with Strong \nef]
% 	% {\includegraphics[height=1.2in]{FIG/cm_synthetic.pdf}}
% 	\subfloat[\label{fig:ex2} ``Pokec-Gender'':\\Het. w/ Strong \nef]
% 	{\includegraphics[height=1.in]{FIG/cm_pokecgender1.pdf}}
%         \hspace{1.5mm}
%         \subfloat[\label{fig:ex3} ``arXiv-Year'':\\\xophily w/ Weak \nef]
% 	{\includegraphics[height=1.in]{FIG/cm_arxivyear.pdf}}
%         \hspace{1.5mm}
% 	 \subfloat[\label{fig:ex4} ``Patent-Year'':\\Het. w/ Weak \nef]
% 	{\includegraphics[height=1.in]{FIG/cm_patentyear1.pdf}}
% 	\vspace{-3mm}
% 	\caption{\label{fig:ex} \underline{\smash{\method is explainable.}} The estimated compatibility matrices are similar to the edge counting matrix (in Figure~\ref{fig:dis}), while being robust to the noises.}
% \end{figure}