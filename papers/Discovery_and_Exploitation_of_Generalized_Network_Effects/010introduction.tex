%%%%%%%%%%%%%%%%%%%%%%%%%%%%%%%%%%%%%%%
% AUTHOR: Christos Faloutsos
% INSTITUTION: CMU
% DATE: April 2019
% GOAL: to streamline the paper presentations
%%%%%%%%%%%%%%%%%%%%%%%%%%%%%%%%%%%%%%%
% \reminder{christos will shrink}
Given a large undirected graph with few node labels and no node features, how to check whether the graph structure is useful for classifying nodes or not? 
% If so, how can we identify what are the classes that a node with a specific class connects to?
Graph is one of the most popular data structures, and as one of its most important tasks, node classification is often employed to infer labels on large real-world graphs.
Since manual labeling is expensive and time-consuming, it is common that only few node labels are available.
For example, in a million-scale social network, identifying even a fraction (say $5\%$) of users' groups is prohibitive, which limits the application of methods that assume a large fraction of labels are given.
Recently, with prevalence of graphs in industry and academia alike, there is a growing need among users to know whether these graph structures actually provide meaningful information for inference tasks.
% for the purpose of node classification.
% Recently, with more and more graphs being constructed from both industry and academia, the users are desired to know whether those graph structures provide meaningful information for performing node classification.
Therefore, before investing a huge amount of time and resources into potentially unsuccessful experiments, a preliminary test on the given graph with few node labels is earnestly needed.
% Therefore, before investing a huge amount of time and resources to conduct experiments that has a chance to fail, a preliminary test on the given graph with few node labels is desperately needed by the users.
% Moreover, node features are frequently missing in real-world graphs.
% For those methods that require node features in classification, they create the structural features based on the graph \cite{Fey/Lenssen/2019, hamilton2017inductive, hamilton2017representation}, such as using the one-hot encoding of node degree.

\begin{figure*}[t]
\centering
\captionsetup[subfloat]{captionskip=-1pt}
\subfloat[\label{fig:c1} \methodtest: \theory]
{\includegraphics[scale=0.4]{FIG/ne_test.pdf}}
\rulesep
\subfloat[\label{fig:c2} \methodest: \explain\xspace and \general]
{\includegraphics[scale=0.41]{FIG/ne_est.pdf}}
\rulesep
\subfloat[\label{fig:c3} \methodexp: \accurate and \scale]
{\includegraphics[scale=0.42]{FIG/ne_exp.pdf}}
\vspace{-2mm}
\caption{\label{fig:crown} \underline{\smash{\method works well}}, thanks to its three novel contributions.
(a) \methodtest statistically identifies whether the given graph has \nef or not.
(b) \methodest explains the given graph by precisely estimating the compatibility matrix, observing \xophily, i.e. both heterophily and homophily. 
(c) \methodexp outperforms the best competitor by $12.9\%$ in accuracy, while being $3.4\times$ faster than the fastest competitor.
See Introduction for more details.
}
\end{figure*}

That is to say, we want to know whether the given graph has \neteffect (\nef) or not.
A graph with \nef provides meaningful information through the structure that can be used to identify the labels of nodes.
For example, ``talkative person tends to make friends with talkative ones'' denotes homophily, 
while ``teenagers incline to interact with the ones that have opposite gender on social media'' denotes heterophily.
Therefore, it is also important to distinguish which \nef the graph has, i.e., homophily, heterophily, or both (which we call ``\xophily''), if there is any.
Given $c$ classes, an intuitive way to describe \nef is via a $c \times c$ compatibility matrix, which shows the relative influence between each class pair.
It can be used to explain the graph property, as well as be exploited to better assign the labels in the graph.\looseness=-1
% Finally, it is as well important, but remains a challenge, to exploit \nef to explain the graph and better classify the nodes.

However, identifying \nef is commonly avoided and neglected in literature:
inference-based methods assume that the relationship is given by domain experts \cite{DBLP:journals/pvldb/GatterbauerGKF15, eswaran2020higher};
most graph neural networks (GNNs) assume homophily \cite{kipf2016semi, klicpera2018predict, wu2019simplifying}.
Although some previous works \cite{zhu2020beyond, luan2021heterophily, lim2021large, ma2022is} use homophily statistics to analyze the given graph, they have three limitations.
First, those statistics are designed to identify the absence of homophily, and thus are not capable of clearly distinguishing \nef, which is more complicated and includes different non-homophily cases, i.e., heterophily, both, or no \nef.
Second, to compute accurate statistics, they need all node labels from the given graph, which is impractical during node classification.
Finally, their analyses rely heavily on the accuracy of baseline GNN models, which means the node features, in addition to the graph structure, also significantly influence the conclusions about \nef.

In contrast, our work aims to answer three research questions:
\begin{compactenum}[{RQ}1.]
    \item {\bf Hypothesis Testing}: How to identify whether the given graph has \nef or not, with only few labels? 
    \item {\bf Estimation}: How to estimate \nef in a principled way, and explain the graph with the estimation?
    \item {\bf Exploitation}: How to exploit \nef on node classification with few labels ($\leq 5$\%) that is robust to noisy neighbors?
    % without interfering by noisy neighbors?
\end{compactenum}

In this paper, we propose \method, with three contributions as the corresponding solutions to each of the questions:

% \noindent
\ben
    \item {\bf \theory: \methodtest} 
    uses rigorous statistical tests to decide whether \nef exists at all.
    % uses solid statistical tests to decide whether network effects exist at all.
    % is an analysis tool to statistically test \nef with only few observed node labels. 
    % It analyzes the relationships between all pairs of different classes efficiently.
    Figure~\ref{fig:c1} shows how it works, and Figure~\ref{fig:dis} shows its discovery, where surprisingly many large real-world datasets known as heterophily graphs have little \nef.

% \noindent
\item {\bf \general and \explain: \methodest} explains whether the graph is homophily, heterophily, or \xophily by precisely estimating the compatibility matrix with the derived closed-form formula.
In Figure~\ref{fig:c2}, it explains the interrelations of classes by the estimated \emph{compatibility matrix}, which implies \xophily.
% \xspace -- where the first half follows heterophily, while the other half follows homophily.

% \noindent
\item {\bf \accurate and \scale: \methodexp} exploits the estimated \nef to perform node classification with few labels. %, effectively and efficiently.
% It better utilizes the graph structure by paying attention to only influential neighbors.
% It is also thrifty, requiring CPU only.
It outperforms the competitors in both accuracy and time on the largest heterophily ``Pokec-Gender'' graph with $22$M edges, only requiring {\em $14$ minutes} (Fig.~\ref{fig:c3}). 
\een

% \noindent{\bf Clarification}: Unlike graph machine learning (GML) works on node classification, which majorly focus on improving the model performance, we have a very different direction.
% Our work focuses on graph mining, which statistically analyzes \nef in the graph.
% We show that it is possible to obtain better accuracy by exploiting \nef, offering a potential to be extended to GML.

% \noindent{\bf Reproducibility}: Our implemented source code and preprocessed datasets are available online at \urlcolor{\codeurl}.
{\bf Reproducibility}: Our implemented source code and preprocessed datasets will be published once the paper is accepted.

% \blue{@Jeremy, @Jaemin I think the below is redundant which can be done away with. Let me know what you think.}
% The rest of the paper is organized as follows. 
% We give the survey and background in Section~\ref{sec:background}.
% We firstly propose \nea to analyze \nef in Section~\ref{sec:neteffect} along with the surprising discoveries, and then introduce our novel algorithm \method for node inference in Section~\ref{sec:meth}. 
% The experiments are presented in Section~\ref{sec:exp}. 
% Section~\ref{sec:concl} concludes.
