% !TEX root = tomography.tex
%%%%%%%%%%%%%%%%%%%%%%%%%%%%%%%%%%%%%%%%%%%%
\section{Introduction}
\label{sec:Introduction}
%%%%%%%%%%%%%%%%%%%%%%%%%%%%%%%%%%%%%%%%%%%%


Understanding the inner structure of the Earth could help us answer many important questions about our planet. 
It is widely believed that the Earth has an is iron-rich core which generates the planet's magnetic field. 
The formation of our planet depends on the core itself, which influences the evolution of the mantle and crust~\cite{Condie}. 
In turn, decays of radio isotopes in the mantle and crust of the Earth, in particular $^{40}$K, $^{232}$Th and $^{238}$U, originate a large flux of so-called geoneutrinos~\cite{KamLAND:2011ayp, KamLAND:2013rgu, Borexino:2019gps}. 
which are crucial in the cooling mechanism of our planet~\cite{davies2010earth, McDonough:2019ldt}.
Nevertheless, understanding the Earth's matter profile remains an extraordinary endeavor.

To understand the matter profile, we need to rely on indirect measurements.
The most traditional of those measurements are performed with seismological data.
By studying seismic waves in the surface of the planet, we can infer what is the matter profile that these waves went through.
Combining observations at different positions at the surface and from many earthquakes allows geologists to measure the Earth's density as a function of the depth.
While these measurements provide an excellent model of the matter distribution, they rely on empirical relations that describe how compressional and transverse waves, typically referred to as P (primary) and S (secondary) waves, propagate on a dense medium (see e.g. Ref.~\cite{Geller:2001ix}), as well as assumptions on the variation of the matter profile as a function of the depth.
On top of that, a clearcut statistical evaluation of the uncertainties on the density profiles in these measurementsis not straightforward~\cite{Kenneth1998}.
Thus, it would be desirable to measure the Earth's matter profile with alternative methods.

Indeed, there have been several proposals to use atmospheric neutrinos, which originate from the decays of hadrons created when cosmic rays hit the atmosphere, to measure the matter profile.
The most conceptually simple method is via the attenuation of atmospheric neutrinos~\cite{DeRujula:1983ya, Wilson:1983an, Askarian:1984xrv, Borisov:1986sm, Jain:1999kp, Reynoso:2004dt, Gonzalez-Garcia:2007wfs, Agarwalla:2012uj, Ioannisian:2017chl, Donini:2018tsg}.
Atmospheric neutrinos span several orders of magnitude in energy, from the tens of MeV to beyond tens of TeV.
While neutrinos interact weakly with matter, the probability for a multi-TeV neutrino to cross the Earth without interactions is significantly different from unity.
The mean free path of TeV neutrinos crossing the Earth can be written as
\begin{equation}
  \frac{\lambda}{R_{\scriptscriptstyle\oplus}} \sim 3.7 \left(\frac{10~\text{g/cm}^3}{\rho}\right)\left(\frac{10~\text{TeV}}{E}\right),
\end{equation}
where $R_{\scriptscriptstyle\oplus}\simeq6371$~km is the radius of the Earth and $E$ is the energy of the neutrino.
Thus, by analyzing the attenuation of high energy atmospheric neutrinos as a function of the zenith angle of the incoming neutrinos, the matter profile of the Earth may be inferred (as in, e.g., Ref.~\cite{Donini:2018tsg}).

As we will see later, the atmospheric neutrino flux drops sharply with larger energies: a much larger flux of neutrinos is then available at lower energies, particularly below the GeV scale.
If we calculate the mean free path of neutrinos below ${\sim}$TeV, though, we will find that atmospheric flux attenuation becomes small to the point that it is experimentally unobservable.
This brings us to the other way of measuring the Earth's interior: probing the Earth's density with neutrino oscillations~\cite{Nicolaidis:1990jm, Ohlsson:1999um, Lindner:2002wm, Akhmedov:2005yt, Winter:2006vg, Rott:2015kwa, Winter:2015zwx, DOlivo:2020ssf, Kumar:2021faw}.


The oscillation of atmospheric neutrinos was first measured by the Kamiokande experiment and, together with the observation of solar neutrinos by SNO, it led to the discovery of nonzero neutrino masses~\cite{McDonald:2016ixn, Kajita:2016cak}.
Neutrino oscillation is, in fact, a simple and yet exquisite quantum mechanical effect. 
Charged-current weak interactions couple charged leptons to a superposition of neutrino states with well defined masses. 
More precisely,  
\begin{equation}
  |\nu_\alpha \rangle = \sum_i U_{\alpha i}^* |\nu_i \rangle,
\end{equation}
where $|\nu_\alpha \rangle$ ($\alpha = e,\ \mu,\ \tau$) denotes the flavor eigenstates which couple diagonally to each flavor of charged lepton and the $W$ boson, and $|\nu_i \rangle$ are mass eigenstates which have well-defined masses.
The $3\times3$, unitary, Pontecorvo-Maki-Nakagawa-Sakata (PMNS) matrix $U_{\alpha i}$ parametrizes the mixing among flavor and mass eigenstates.

In a simplified two neutrino flavor oscillation scheme, the transition probability $\nu_\alpha\to\nu_\beta$, for different flavors, can be written as
\begin{equation}
  P_{\alpha \beta} = \sin^2(2\theta)\sin^2\left(\frac{\Delta m^2 L}{4E}\right),\qquad \alpha\neq\beta,
\end{equation}
where $\Delta m^2 \equiv m_2^2-m_1^2$ is the squared neutrino mass splitting, $\theta$ is the angle that parametrizes the mixing and $L$ is the distance traveled by the neutrino, or its baseline. 
While in reality we know that there are three neutrinos, the above formula is still a useful guide.
This is because the so-called ``atmospheric splitting''  $|\Delta m^2_{31}|\sim 2.5\times10^{-3}$~eV$^2$~\cite{DayaBay:2018yms, NOvA:2019cyt, T2K:2021xwb} is about 30 times larger than the ``solar splitting'' $\Delta m^2_{21}\sim7.5\times10^{-5}$~eV$^2$~\cite{KamLAND:2013rgu, yasuhiro_nakajima_2020_4134680}.
Oscillation effects stemming from these two oscillation frequencies, to first order, decouple from each other.

A key point of neutrino oscillations is that usual matter sources a potential that affects neutrino propagation, changing their dispersion relation in a flavor dependent way and acting similarly to a refraction index~\cite{Wolfenstein:1977ue, Mikheyev:1985zog, Mikheev:1986wj}.
Observing how atmospheric neutrinos oscillate as they cross the Earth at different zenith angles can, in principle, allow us to infer the matter profile of the Earth through a genuine quantum mechanical effect.
Moreover, there is an interesting  complementarity among seismology, neutrino absorption, and neutrino oscillation measurements of the matter profile.

The propagation of seismic waves depend on the total density of matter, as well  as on variations of this density and on the state of matter (solid vs. liquid) through the wave trajectory.
In particular the waves that oscillate perpendicular to the direction of propagation (S-waves) do not travel through fluids.
The absence of S-waves detected in a surface region diametrically opposite to an earthquake location is a strong indication that the core of our planet is liquid, besides providing a robust measurement of the liquid core boundaries.
In contrast, absorption of high energy atmospheric neutrinos depends mainly on the number of the proton and neutrons in the incoming neutrino line-of-sight, regardless of how the density profile varies.
To obtain the density profile, one needs to be able to measure the incoming neutrino direction and reconstruct the density profile by analyzing the  zenith-dependent neutrino attenuation.
The component of the neutrino matter potential that affects neutrino oscillation, on the other hand, is sourced only by electrons and it is sensitive to the number density distribution along the neutrino line of sight, as we will discuss in detail later.
Therefore, since the Earth is electrically neutral, combining seismology, neutrino absorption and neutrino oscillations could in principle allow for a measurement of the average neutron to proton ratio inside the Earth, at different radii, which would ultimately teach us about the average chemical composition of Earth's core.

In this manuscript, we are interested in how to measure the Earth's density profile with atmospheric neutrino oscillations. 
There are two energy regions of interest here.
While oscillations driven by the atmospheric splitting $\Delta m^2_{31}$ are dominant in the few-10~GeV region, those driven by $\Delta m^2_{21}$ are more relevant around 0.1-1~GeV.
The latter not only benefit from a higher neutrino flux, but also a larger oscillation amplitude due to larger mixing angles.
Because of that, we will pay particular attention to the sub-GeV atmospheric neutrino component, even if we include atmospheric neutrinos from 0.1 to 100 GeV in our analysis.

Due to the small interaction cross section, only multi-kiloton detectors are capable of measuring atmospheric neutrinos with sufficient statistics.
To date, the only multi-kiloton detectors available are Cherenkov observatories, like Super-Kamiokande and IceCube. 
The energy and direction of charged particles traversing these observatories, as well as their type (e.g. electrons versus muons) can be inferred  by  detecting the Cherenkov light emitted by these particles.
This allows to reconstruct the incoming neutrino energy, direction and flavor, which is crucial to extract physics from atmospheric neutrinos.
Nevertheless, particles only radiate Cherenkov light if they are faster than light in that medium. 
In particular, protons with total energies below 1.4~GeV do not emit Cherenkov light in water.
Because of that, it is nearly impossible for Cherenkov detectors to leverage the atmospheric neutrino flux below the GeV scale, as the correlation between the outgoing lepton and the incoming neutrino directions is very weak. 
This lack of directionality for sub-GeV neutrinos reduces the sensitivity of Cherenkov detectors  to the Earth density profile. 
Both Super-Kamiokande~\cite{Super-Kamiokande:2017yvm} and IceCube~\cite{IceCube:2019dyb} collaborations have performed analyses on the sensitivity to the overall matter potential sourced by the Earth, indicating preference for a nonzero value.

This bring us to the Deep Underground Neutrino Experiment (DUNE), a 40~kton liquid argon time projection chamber being built below the Black Hills of South Dakota~\cite{DUNE:2020lwj, DUNE:2020ypp}. 
Liquid argon time projection chambers, or LArTPCs for short, detect the ionization energy of charged particles traversing the argon, which enables the identification and tracking of particles, including low energy ones.
In fact, the ArgoNeuT experiment at Fermilab has shown that it is possible for those detectors to reconstruct protons with as low as 21~MeV of kinetic energy~\cite{ArgoNeuT:2018tvi}.

This immediately prompts us to investigate how much physics DUNE could extract from the sub-GeV atmospheric neutrino sample.
Although the large flux and detector reconstruction capabilities are very promising, there are inherent difficulties in studying sub-GeV atmospheric neutrinos  at DUNE.
Modeling neutrino-nucleus interactions at the 0.1-1~GeV region is remarkably challenging~\cite{Benhar:2015wva}.
First, there are difficulties associated to the hard interaction itself, due to Fermi motion of the nucleons inside the nucleus and nuclear binding energy~\cite{Dutta:2000sn, JLabE91013:2003gdp, Benhar:1994hw, Dickhoff:2004xx, Benhar:2006wy, Ankowski:2014yfa, Benhar:2015ula, Rocco:2015cil, Barbieri:2016uib, Rocco:2018mwt, Rocco:2020jlx}, in-medium effects that may change the local dispersion relation of those nucleons~\cite{Arnold:1981dt, Hama:1990vr, Cooper:1993nx}, and even nucleon-nucleon correlations in the nuclear medium~\cite{Mathiot:1980js, Kohno:1981dg, Dehesa:1985asb, Jiang:1992wn, Sobczyk:2012ms, Megias:2014qva, Megias:2016fjk}.
Then, even if these obstacles are overcome, one still needs to properly model how nucleons propagate within the nucleus, including how the nuclear potential and intranuclear cascades affect the energy, angular and multiplicity distributions of final state nucleons~\cite{Bertini:1963zzc, Cugnon:1980zz, Bertsch:1984gb, Bertsch:1988ik, Cugnon:1996xf, Boudard:2002yn, Sawada:2012hk, Uozumi:2012fm, Golan:2012wx, Isaacson:2020wlx, Dytman:2021ohr}.

All these effects are important because, in order to extract useful information from sub-GeV atmospheric neutrinos, it is crucial to reconstruct the incoming neutrino direction~\cite{Kelly:2019itm}.
To do so, the energy, direction and type of outgoing particles should be reconstructed and combined to obtain the incoming neutrino four momentum in an event-by-event basis.
Most of the aforementioned effects, especially intranuclear cascades, will change the direction and multiplicity of outgoing protons and neutrons, the former being essentially invisible at LArTPCs.

Despite all these challenges, previous work has shown that the sub-GeV sample alone can yield valuable information on $CP$ violation~\cite{Kelly:2019itm}.
Here we perform a detailed study of DUNE's capability to yield the first quantum tomography measurement of the Earth's matter profile using neutrino oscillations.
We consider various systematic uncertainties on the atmospheric neutrino flux, as well as state-of-the-art neutrino-nucleus interactions using the \texttt{NuWro} neutrino-nucleus event generator~\cite{Golan:2012rfa}, and realistic detector responses to particle identification and reconstruction.
We will show that the future DUNE experiment has the capability of pioneering a quantum tomography measurement of our planet, determining the core density at the 10\% level.

This paper is organized as follows.
In Sec.~\ref{sec:AtmoNu} we discuss neutrino oscillations in the Earth, focusing on the effects induced by matter; in Sec.~\ref{sec:DUNEDetails} we describe the DUNE experiments and how we model the detector response; in Sec.~\ref{sec:FluxUncertainties} the atmospheric neutrino flux and associated uncertainties are discussed; in Sec.~\ref{sec:ResultsDiscussion} we present our results; and finally we conclude in Sec.~\ref{sec:Conclusions}.
We use natural units where $\hbar = c = k_{\rm B} = 1$ throughout this manuscript, unless otherwise stated. 











