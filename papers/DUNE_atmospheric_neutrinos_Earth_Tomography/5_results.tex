% !TEX root = tomography.tex
%%%%%%%%%%%%%%%%%%%%%%%%%%%%%%%%%%%%%%%%%%%%
\section{Results and Discussion}\label{sec:ResultsDiscussion}
%%%%%%%%%%%%%%%%%%%%%%%%%%%%%%%%%%%%%%%%%%%%
We now proceed to the numerical results of our analysis.
First, we will estimate how well DUNE can measure the total mass of the Earth, keeping the shape of the matter profile fixed.
Then we will estimate DUNE's sensitivity to the matter profile itself accounting for current measurements of the Earth's total mass and moment of inertia.
Finally we will repeat the second analysis but discarding the total mass and moment of inertia measurements.

%%%%%%%%%%%%%%%%%%%%%%%%%%%%%%%%%%%%%%%%%%%%
\subsection{Overall Earth Mass Measurement}
\label{subsec:MeasureMass}
%%%%%%%%%%%%%%%%%%%%%%%%%%%%%%%%%%%%%%%%%%%%
Let us start with the simplest measurement of Earth's properties that can be performed with atmospheric neutrinos at DUNE: the determination of the total mass of the Earth.
To simplify the calculation, we consider the 3-layer Earth model introduced in Section~\ref{sec:AtmoNu}.
Although there are detailed models of the matter density profile available, such as the Preliminary Reference Earth Model (PREM)~\cite{Dziewonski:1981xy}, neutrino oscillation data will not be able to distinguish fine-grained matter density models.
This measurement will allow us to assess DUNE's sensitivity with a simple figure of merit.

Our statistical treatment follows the $\chi^2$ defined in Ref.~\cite{Kelly:2019itm}, with the addition of the zenith-dependent uncertainty defined in Eq.~\eqref{eq:zenith_uncertainty}.
Concomitantly with atmospheric neutrino data taking, the beam neutrino program in DUNE will be providing the most precise measurements of the mixing angle $\theta_{23}$, the $CP$ violating phase and the neutrino mass ordering.
Since all oscillation parameters will be better measured with DUNE's beam neutrinos or other neutrino experiments, such as the future Hyper-Kamiokande~\cite{Hyper-Kamiokande:2018ofw} and JUNO~\cite{JUNO:2015sjr} experiments; or the current Daya Bay~\cite{DayaBay:2018yms}, KamLAND~\cite{KamLAND:2013rgu}, and solar neutrino experiments~\cite{Aharmim:2011vm, Borexino:2013zhu, Super-Kamiokande:2016yck}, we treat those parameters as constant.
We have chosen the global best fits  to neutrino oscillation data of Ref.~\cite{Esteban:2020cvm} for either normal ordering, that is, $\Delta m^2_{21}=7.42\times10^{-5}$~eV$^2$, $\sin^2\theta_{12}=0.304$, $\Delta m^2_{31}=2.51\times10^{-3}$~eV$^2$, $\sin^2\theta_{13}=2.22 \times 10^{-2}$, $\sin^2\theta_{23}=0.570$ and $\delta_{CP}=195^\degree$; or inverted ordering, $\Delta m^2_{21}=7.42\times10^{-5}$~eV$^2$, $\sin^2\theta_{12}=0.304$, $\Delta m^2_{32}=-2.50\times10^{-3}$~eV$^2$, $\sin^2\theta_{13}=2.24 \times 10^{-2}$, $\sin^2\theta_{23}=0.575$ and $\delta_{CP}=286^\degree$.

There is one further detail worth mentioning regarding DUNE's measurement.
Neutrino oscillation is affected not by the Earth's density itself, but by its electron number density, and thus the measurement really provides the total electron number of the Earth.
In relating the electron number density to the mass density, we have assumed the proton-to-neutron fraction to be 1 (which is close to that estimated in Ref.~\cite{Dziewonski:1981xy}), along with the fact that the Earth is essentially electrically neutral.
A mismatch between the total mass of the Earth measured with neutrino oscillations and other techniques could, in principle, indicate a problem with the assumed Earth's chemical composition.
Keeping that in mind, we will hereafter discuss the matter density measurements with the implicit proton-to-neutron ratio assumption.

%%%%%%%%%%%%%%%%%%%%%%%%%%%%%%%%%%%%%%%%%%%%%%%%%%%%%%%%%%%%%%%
\begin{figure}
\begin{center}
\includegraphics[width=0.8\linewidth]{Dchi2_TotMass_Chns.pdf}
\caption{DUNE's sensitivity to the total mass of the Earth for several different charged current (CC) event topologies accounting for different number of protons and pions (colored lines as indicated) and their combination (black line) for normal (left panel) and inverted (right panel) neutrino mass ordering. \label{fig:TotMassChns}}
\end{center}
\end{figure}
%%%%%%%%%%%%%%%%%%%%%%%%%%%%%%%%%%%%%%%%%%%%%%%%%%%%%%%%%%%%%%%
In Fig.~\ref{fig:TotMassChns} we show the sensitivity to the total mass of the Earth for each different event topology considered, as indicated by the colored lines, as well as for the joint fit accounting for all topologies (black line) assuming the normal ordering (left panel) and inverted ordering (right panel).
We have assumed a total 400~kton-year exposure.
We see that DUNE will be able to measure the Earth's total mass with good precision, 
\begin{equation}
  M = (1\pm0.084)M_\Earth\hspace{2cm}\text{(400~kton-year exposure)}.
\end{equation}
This is much better than the current determination of Earth's mass via the atmospheric oscillation measurement by the Super-Kamiokande collaboration of $M/M_\Earth\simeq 1.04\pm0.21$~\cite{Super-Kamiokande:2017yvm}. 
In the case of IceCube,  only a preference for $M/M_\Earth > 0$ at less than 1$\sigma$ is found~\cite{IceCube:2019dyb}.
Besides, it is interesting to estimate the sensitivity to for a lower exposure.
For example, a 60~kton-year exposure run, corresponding to two far detector modules taking data for three years, would lead to
\begin{equation}
  M = (1^{+0.48}_{-0.43})M_\Earth\hspace{2cm}\text{(60~kton-year exposure)}.
\end{equation}


From this figure we also see that, in the normal ordering case, the sensitivity is driven by charged current (CC) events with 1 outgoing proton and no pions, CC-1p0$\pi$.
This is because at sub-GeV energies, the neutrino-nucleus cross section is dominated by quasi-elastic (QE) transitions in which $\nu_\ell + n\to\ell+p$, where $\ell=e,\mu$.
Final state interactions could change the number of outgoing protons and neutrons, however many times the kicked proton will exit the nucleus without interactions.

For the inverted mass ordering, the situation is different.
While essentially nothing changes for the solar resonance, the atmospheric resonance above a GeV now occurs for antineutrinos. 
In antineutrino scattering, there is a larger chance to knock out neutrons~\cite{Palamara:2016uqu}.
Because of that, CC-0p0$\pi$ events, which have zero protons and zero pions, contribute significantly to the sensitivity.
Moreover, at high Earth masses, the MSW resonance energy is reduced, as can be seen in Eq.~\eqref{eq:energy_msw}.
In this case, the experimental sensitivity comes from the atmospheric resonance, since the solar one happens at too low energy.
Nevertheless, due to the crude directionality of the CC-0p0$\pi$ sample (see Fig.~\ref{fig:Reco2D} bottom, especially bottom-left for low-energy), the sensitivity is reduced, which explains why the $\chi^2$ is suppressed at high masses relative to the normal ordering case.

Although most oscillation parameters have been measured well experimentally, the current hint for the $CP$ violating phase strongly depends on a combination of several experimental results.
It is far from clear that the true values of the $CP$ phase are the ones found in the current global fits, and only more data will clarify the situation.
Therefore, one is led to question how much the sensitivity depends on the true value of $\delta_{CP}$.
To answer this question, we show in Fig.~\ref{fig:TotMass} the sensitivity for the best fit values of the $CP$ phase, as well as a band that encodes the minimum and maximum sensitivities for all possible values of $\delta_{CP}$, for a given $M/M_\Earth$.
As we can see, the impact of the $CP$ phase is only large away from $M/M_\Earth=1$, which shows that the our estimate for DUNE's determination of the total mass of the Earth is indeed robust, especially at $1\sigma$. 
We also studied the impact of varying the true values of other oscillation parameters on the measurement of Earth's total mass. 
Within present constraints, we found a negligible variation of the sensitivity to $M/M_\Earth$, in particular near $M/M_\Earth \sim 1$.

%%%%%%%%%%%%%%%%%%%%%%%%%%%%%%%%%%%%%%%%%%%%%%%%%%%%%%%%%%%%%%%
\begin{figure}
\begin{center}
\includegraphics[width=0.8\linewidth]{Dchi2_TotMass.pdf}
\caption{DUNE's sensitivity to the total mass of the Earth for the current global best fit value of $\delta_{\rm CP}$ (line) for normal (left panel) and inverted (right panel) neutrino mass ordering. The band encodes the sensitivity varying the true value of $\delta_{CP}$ within $0$ to $2\pi$.
 \label{fig:TotMass}}
\end{center}
\end{figure}
%%%%%%%%%%%%%%%%%%%%%%%%%%%%%%%%%%%%%%%%%%%%%%%%%%%%%%%%%%%%%%%


%%%%%%%%%%%%%%%%%%%%%%%%%%%%%%%%%%%%%%%%%%%%
\subsection{Measuring the Earth's Matter Density Profile}
\label{subsec:MeasureWithConstraints}
%%%%%%%%%%%%%%%%%%%%%%%%%%%%%%%%%%%%%%%%%%%%
We now proceed to estimate the capability of DUNE in measuring the matter density profile of the Earth.
Again, we consider the 3-layer Earth model introduced in Section~\ref{sec:AtmoNu}.
The radii that separate each layer are related to how seismic waves propagate and get reflected, and are fairly well known by seismological data (see e.g. Ref.~\cite{Geller:2001ix}).
In particular, the liquid core shadows S-waves coming from the opposite side of the planet, which provides an excellent measurement of the core radius.
In view of that, we assume that the radii at which the density profile transitions from the core to the lower mantle and from the lower mantle to the upper mantle are known.
We refer to these radii of interest as $R_{\rm C} = 3480$ km (core $\to$ lower mantle), $R_{\rm LM} = 5700$ km (lower mantle $\to$ upper mantle), and $R_{\rm UM}=R_\Earth = 6371$ km (radius of Earth).

Since we do not expect any sensitivity to the finer details of this density profile, nor the fact that the Earth is not perfectly spherical, we will neglect these higher-order effects. 
The total mass of the Earth $M_\Earth$ can be inferred from its effect on orbiting satellites~\cite{ries1992progress} and independent measurements of the Newtonian gravitational constant $G$~\cite{Rosi:2014kva} ---the current determination being $M_\Earth = 5.9722\times10^{24}$~kg with a $10^{-4}$ relative uncertainty.
In addition, by observing the Earth's precession and nutation, its moment of inertia is inferred to be $I_\Earth=8.01738\times10^{37}$~kg~m$^2$, again with a small $10^{-4}$ relative uncertainty~\cite{williams1994contributions, chen2015consistent}.
Given the precision of such measurements, we will take both mass and moment of inertia of the Earth as constraints on the matter profile, namely,
\begin{align}
M_\Earth &= \frac{4\pi}{3}\left[ \rho_{\rm C} R_{\rm C}^3 + \rho_{\rm LM} \left(R_{\rm LM}^3 - R_{\rm C}^3\right) + \rho_{\rm UM} \left(R_\Earth^3 - R_{\rm LM}^3\right)\right], \label{eq:ME}\\
I_\Earth &= \frac{8\pi}{15}\left[ \rho_{\rm C} R_{\rm C}^5 + \rho_{\rm LM} \left(R_{\rm LM}^5 - R_{\rm C}^5\right) + \rho_{\rm UM} \left(R_\Earth^5 - R_{\rm LM}^5\right)\right].\label{eq:IE}
\end{align}
This procedure provides enough information to require only one independent input. 
We take $\rho_{\rm C}$ to be that independent input, where $\rho_{\rm LM,UM}$ are determined by satisfying Eqs.~\eqref{eq:ME} and~\eqref{eq:IE}. 
The relation imposed by the aforementioned constraints between $\rho_{\rm LM}$ and $\rho_{\rm UM}$ as a function of the input $\rho_{\rm C}$ is shown in Fig.~\ref{fig:MEIEConstraints}.
\begin{figure}
\begin{center}
\includegraphics[width=0.5\linewidth]{Constraints_ME_IE.pdf}
\caption{Constrained relationship between $\rho_{\rm LM}$ (blue) and $\rho_{\rm UM}$ (orange) as a function of the free input parameter $\rho_{\rm C}$, where we assume that the mass of the Earth and its moment of inertia are known, using the relationships in Eqs.~\eqref{eq:ME} and~\eqref{eq:IE}. See text for further detail.\label{fig:MEIEConstraints}}
\end{center}
\end{figure}
Due to the constrained relationship by knowing the mass of the Earth and its moment of inertia, as well as our three-layer approximation, we see that this setup limits $6$ g/cm$^3$ $\lesssim \rho_{\rm C} \lesssim 21$ g/cm$^3$.

\begin{figure}
\begin{center}
\includegraphics[width=0.85\linewidth]{M+MI_Dchi2.pdf}
\caption{DUNE's sensitivity to the core matter density $\rho_{\rm core}$, assuming the density profile to be constrained by the total mass and moment of inertia of the Earth. 
The left (right) panel assumes that neutrinos follow the normal (inverted) mass ordering, where both assume that the ordering is known from the neutrino beam data. 
The blue band encodes different true values of $\delta_{CP}$, while the current global best fit is highlighted (purple lines).\label{fig:MeasRho_MEIE}}
\end{center}
\end{figure}
%
Allowing $\rho_{\rm C}$ to vary independently, we determine the measurement sensitivity assuming $400$ kt-yr of DUNE operation and present this result in Fig.~\ref{fig:MeasRho_MEIE}.
As before, we perform this simulation for both normal (left panel) and inverted (right panel) neutrino mass orderings. 
We highlight the sensitivity for the current global best fit value of the $CP$ phase (line).
The blue bands envelop the sensitivity estimates as we vary the true value of $\delta_{CP}$.
As before, we fix all other oscillation parameters as their measurements from other current and DUNE-contemporary experiments will be significantly more powerful than what atmospheric neutrinos can provide.

While the $3\sigma$ measurement range of $\rho_{\rm C}$ varies depending on the input parameters (as optimistic as $\rho_{\rm C} \in [7, 14]$ g/cm$^3$ and as pessimistic as $\rho_{\rm C} \in [6, 17]$ g/cm$^3$), we see in Fig.~\ref{fig:MeasRho_MEIE} that the $1\sigma$ extraction is fairly independent of the truth assumption.
Since all three densities are constrained by Earth's mass and moment of inertia measurements, our results indicate that, under these assumptions DUNE will be able to provide the following measurements for a 400~kton-year exposure:
\begin{subequations}
\begin{align}
  &\rho_{\rm C} = 11.0 \times(1^{+0.088}_{-0.083})~{\rm g}/{\rm cm}^3         &\text{(core, 400~kton-year)},&\\
  &\rho_{\rm LM} = 5.11\times(1^{+0.12}_{-0.13})~{\rm g}/{\rm cm}^3        &\text{(lower mantle, 400~kton-year)},&\\
  &\rho_{\rm UM} = 3.15 \times(1^{+0.22}_{-0.20})~{\rm g}/{\rm cm}^3      &\text{(upper mantle, 400~kton-year)}.&
\end{align}
\end{subequations}
For a lower, 60~kton-year exposure, we find that DUNE can measure
\begin{subequations}
\begin{align}
  &\rho_{\rm C} = 11.0 \times(1^{+0.30}_{-0.25})~{\rm g}/{\rm cm}^3         &\text{(core, 60~kton-year)},&\\
  &\rho_{\rm LM} = 5.11\times(1^{+0.35}_{-0.42})~{\rm g}/{\rm cm}^3        &\text{(lower mantle, 60~kton-year)},&\\
  &\rho_{\rm UM} = 3.15 \times(1^{+0.71}_{-0.60})~{\rm g}/{\rm cm}^3      &\text{(upper mantle, 60~kton-year)}.&
\end{align}
\end{subequations}

The 400~kton-year sensitivity is particularly relevant for the core chemical composition.
It is currently believed that the core is composed mainly of iron~\cite{Dziewonski:1981xy}, whose average proton-to-neutron ratio is $p/n\simeq 0.87$, or equivalently $Z/A\simeq0.47$. 
As discussed before, the matter effect on neutrino oscillations is sourced by the electron number density, which should be equal to the proton number density for an electrically neutral object.
Therefore, a determination of the core's proton number density with better than $\sim10\%$ sensitivity using neutrino oscillations could in principle contribute to the understanding of the core's chemical composition.

To put things in perspective, we can compare DUNE's $\sim8.5\%$ relative sensitivity to Hyper-Kamiokande's.
From Ref.~\cite{Hyper-Kamiokande:2018ofw}, Hyper-Kamiokande will be able to measure the $Z/A$ ratio of the core with a relative 6.2\% precision after 10~Mton-year exposure, which would correspond to about 27 years of data taking with two 187~kton modules or 53 years with only one of those modules. 
Given that DUNE's 400~kton-year exposure would correspond to 10 years of data taking with all four 10~kton modules, it may seem surprising that DUNE can compete with a much larger detector on measuring the Earth's matter profile.
The reason for this relies on the ability to measure sub-GeV atmospheric neutrinos, which provide a much high event rate compared to the higher energy flux component: DUNE is expected to measure about 15k neutrino events with reconstructed energy below 1 GeV, versus 5k with energies above 1 GeV, for 400~kton-year exposure.

%%%%%%%%%%%%%%%%%%%%%%%%%%%%%%%%%%%%%%%%%%%%%%%%%%%%%%%%%%%%%%%%%%%
\begin{figure}[t]
\begin{center}
\includegraphics[width=\textwidth]{Ne_1p.pdf}
\caption{\emph{Left}: Number of charged current electron neutrino events with one proton and no pions in the final state(CC-1p0$\pi$), taking into account cross section and detector effects, as a function of the reconstructed neutrino energy and zenith angle. 
\emph{Middle}: Statistical power of CC-1p0$\pi$ sample, see text for details.
\emph{Right}: Statistical power of CC-1p0$\pi$ sample for the $40\times60$ binning used in our simulations; the darkest yellow and blue correspond to $\Delta N_e/\sqrt{N_e^{\rm true}}= 0.35, -0.25$, respectively.
 \label{fig:Evse1p}} 
\end{center}
\end{figure}
%%%%%%%%%%%%%%%%%%%%%%%%%%%%%%%%%%%%%%%%%%%%%%%%%%%%%%%%%%%%%%%%%%%
Let us discuss in more detail where DUNE's sensitivity comes from.
If on one hand the sub-GeV sample has more statistics than the few-GeV sample, the role of energy and angular resolutions on the final experimental sensitivity is not obvious, see Fig.~\ref{fig:Reco2D}.
To address this, we show in Fig.~\ref{fig:Evse1p} a set of three panels related to electron-neutrino events, accounting for the neutrino-nucleus interaction physics and detector effects described in Sec.~\ref{sec:DUNEDetails}.
The left panel shows the total number of charged current $\nu_e+\overline\nu_e$ events with one proton and no pions (CC-1p-0$\pi$) as a function of the reconstructed zenith angle and neutrino energy for the nominal Earth matter density profile.
To avoid a cluttered figure, we have combined bins to show a $10\times10$ grid.
We can clearly see that the highest statistics comes from neutrinos with 0.4-1~GeV energies and zenith angles such that they cross the lower mantle.

In the middle panel, we show the combined statistical power of each bin for a representative matter profile $\rho=\{17,\,1.2,\,7.1\}~{\rm g/cm^{3}}$ which is ruled out at more than 3$\sigma$ as can be see in Fig.~\ref{fig:MeasRho_MEIE}.
We define the statistical power as the difference of events between the two matter profile assumptions, $\Delta N_e$ divided by the statistical uncertainty $\sqrt{N_e^{\rm true}}$ for each bin.
We calculate that with the original $40\times60$ grid, and sum up the absolute values  $|\Delta N_e|/\sqrt{N_e^{\rm true}}$ in the relevant bins to show the $10\times10$ grid of Fig.~\ref{fig:Evse1p}.
Although systematic uncertainties are taken into account when evaluating the final sensitivity, we do now show them here.
We can see that most of the statistical power comes from neutrinos which cross the core and lower mantle in their trajectories.
Sub-GeV neutrinos do provide most of the statistical power, though there is a non-negligible contribution from the few-GeV sample.

In the right panel of Fig.~\ref{fig:Evse1p}, we show the statistical power in the $40\times60$ grid.
The blue regions indicate downward variations on the number of events while yellow refers to upward.
The darker blue and yellow bins correspond to $\Delta N_e/\sqrt{N_e^{\rm true}}= 0.35, -0.55$, respectively, values for the statistical power.
Although the individual power in each bin is small, one can clearly see that for this representative matter profile there are large regions presenting events consistently above or below the assumed true matter profile case.

%%%%%%%%%%%%%%%%%%%%%%%%%%%%%%%%%%%%%%%%%%%%%%%%%%%%%%%%%%%%%%%%%%%
\begin{figure}[t]
\begin{center}
\includegraphics[width=\linewidth]{Nmu_1p.pdf}
\caption{\emph{Left}: Number of charged current muon neutrino events with one proton and no pions in the final state (CC-1p0$\pi$), taking into account cross section and detector effects, as a function of the reconstructed neutrino energy and zenith angle. 
\emph{Middle}: Statistical power of CC-1p0$\pi$ sample, see text for details.
\emph{Right}: Statistical power of CC-1p0$\pi$ sample for the $40\times60$ binning used in our simulations; the darkest yellow and blue correspond to $\Delta N_\mu/\sqrt{N_\mu^{\rm true}}= 0.28, -0.55$, respectively.
\label{fig:Evsmu1p}} 
\end{center}
\end{figure}
%%%%%%%%%%%%%%%%%%%%%%%%%%%%%%%%%%%%%%%%%%%%%%%%%%%%%%%%%%%%%%%%%%%
Last, we highlight that Fig.~\ref{fig:Evse1p} only includes CC-1p-0$\pi$ events for electron neutrinos. 
While the contribution of other topologies to DUNE's sensitivity is typically subleading, unless the mass ordering is inverted, see e.g. Fig.~\ref{fig:TotMassChns}; muon neutrino events do provide significant statistical power to this analysis. 
We present in Fig.~\ref{fig:Evsmu1p} the same panels as in Fig.~\ref{fig:Evse1p}, but now for $\nu_\mu+\overline\nu_\mu$ CC-1p-0$\pi$ events.
Relative to electron events this sample profits more from the atmospheric resonance at few GeV, as can be seen by comparing the middle panels of Figs.~\ref{fig:Evse1p} and \ref{fig:Evsmu1p}.
These two sets of panels show that the atmospheric neutrino sample at DUNE can indeed provide us an oscillogram of Earth, culminating on a world-leading measurement of the Earth matter profile.




%%%%%%%%%%%%%%%%%%%%%%%%%%%%%%%%%%%%%%%%%%%%
\subsection{Measurements without Constraints}
\label{subsec:MeasureRhos}
%%%%%%%%%%%%%%%%%%%%%%%%%%%%%%%%%%%%%%%%%%%%
We now proceed to our last analysis, the measurement of Earth's matter profile, in the 3-layer simplified model of the Earth, without taking into account the measurements of the total mass and moment of inertia of the Earth.
While it may seem unreasonable to disregard both measurements, this exercise will show us that DUNE atmospheric neutrino sample alone can still teach us something about the Earth.

%%%%%%%%%%%%%%%%%%%%%%%%%%%%%%%%%%%%
%%%%%%%%%%%%%%%%%%%%%%%%%%%%%%%%%%%%
\begin{figure}
\begin{center}
\includegraphics[width=0.8\linewidth]{NoConstr_Dchi2.pdf}
\caption{DUNE's sensitivity to the core, lower mantle and upper mantle densities without considering constraints from total Earth mass and moment of inertia measurements. The colored areas correspond to 1$\sigma$ (blue), 2$\sigma$ (yellow), and 3$\sigma$ (green) allowed regions for 2 degrees of freedom, that is $\Delta \chi^2=2.3,\,6.2,\,11.8$, respectively. \label{fig:three_layers_no_constraints}}
\end{center}
\end{figure}
%%%%%%%%%%%%%%%%%%%%%%%%%%%%%%%%%%%%
%%%%%%%%%%%%%%%%%%%%%%%%%%%%%%%%%%%%
For this analysis, we proceed very much as in the previous one, just dropping the constraints \eqref{eq:ME} and \eqref{eq:IE}.
In Fig.~\ref{fig:three_layers_no_constraints} we show the 1$\sigma$ (blue), 2$\sigma$ (yellow), 3$\sigma$ (green) allowed regions in all planes with the core, lower mantle and upper mantle densities, as well as the one-dimensional $\chi^2$ projection.
We can clearly see that DUNE is mostly sensitive to the inner core density, due to a combination of large changes in the oscillation probability with high statistics, see Figs.~\ref{fig:Evse1p} and \ref{fig:Evsmu1p}.
The determination of the core density at high significance, in particular, becomes difficult, although DUNE can still reject zero core density at over 2$\sigma$.
Regardless, the 1$\sigma$ uncertainties on the densities were found to be 
\begin{subequations}
\begin{align}
  &\rho_{\rm C} = 11.0 \times(1\pm 0.14)~{\rm g}/{\rm cm}^3         &\text{(core)},&\\
  &\rho_{\rm LM} = 5.11\times(1^{+0.23}_{-0.25})~{\rm g}/{\rm cm}^3        &\text{(lower mantle)},&\\
  &\rho_{\rm UM} = 3.15 \times(1^{+0.97}_{-0.78})~{\rm g}/{\rm cm}^3      &\text{(upper mantle)}.&
\end{align}
\end{subequations}
For a 60~kton-year exposure and without external constraints, DUNE can only provide a meaningful measurement of the  lower mantle density, namely  $\rho_{\rm LM} = 5.11\times(1^{+0.69}_{-0.62})~{\rm g}/{\rm cm}^3$.

To summarize the results of this and the previous subsection, we show in Fig.~\ref{fig:RhovsR} the 1$\sigma$ uncertainties on DUNE's determination of the matter profile as a function of the radius of the Earth, for 400~kton-year exposure (left) and 60~kton-year exposure (right).
The solid black line corresponds to our 3-layer Earth model, while the dashed line is the PREM density.
Sensitivities are show for the constrained analysis (orange data points), accounting for the total Earth mass and moment of inertia measurements, and for the unconstrained one (blue data points).
Note that in the lower exposure case, there are no meaningful determinations of the core and upper mantle densities, so we omit the corresponding data points.
%%%%%%%%%%%%%%%%%%%%%%%%%%%%%%%%%%%%
%%%%%%%%%%%%%%%%%%%%%%%%%%%%%%%%%%%%
\begin{figure}
\begin{center}
\includegraphics[width=1\linewidth]{Rho_vs_r_Measurement.pdf}
\caption{DUNE's sensitivity to the Earth matter profile as a function of the radius for the constrained analysis (orange data points), accounting for Earth's total mass and moment of inertia measurements, as well as the unconstrained analysis (blue data points). The solid black line corresponds to our 3-layer Earth model, while the PREM profile is shown for reference as a dashed line. The left panel corresponds to 400~kton-year exposure, while the right panel corresponds to 60~kton-year exposure. Note that the unconstrained analysis with lower exposure does not lead to meaningful determinations of the core and upper mantle densities, and thus we do not display the corresponding blue data points in the right panel.\label{fig:RhovsR}}
\end{center}
\end{figure}
%%%%%%%%%%%%%%%%%%%%%%%%%%%%%%%%%%%%
%%%%%%%%%%%%%%%%%%%%%%%%%%%%%%%%%%%%





