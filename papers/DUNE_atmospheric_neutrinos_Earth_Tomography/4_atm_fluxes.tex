% !TEX root = tomography.tex
%%%%%%%%%%%%%%%%%%%%%%%%%%%%%%%%%%%%%%%%%%%%
\section{Atmospheric Neutrino Fluxes and Uncertainties}\label{sec:FluxUncertainties}
%%%%%%%%%%%%%%%%%%%%%%%%%%%%%%%%%%%%%%%%%%%%

The collision of cosmic rays with atmospheric nuclei generates a
flux of mesons, mainly pions and kaons at lower energies.
The mesons propagate
throughout the atmosphere before arriving at the Earth. 
Along its
propagation, the meson flux will decay into a flux of muons and
neutrinos which can be observed by neutrino experiments in the Earth. 
If the
energies of the muons are low enough ($E_{\mu} \lesssim1$~GeV), they
will also decay before arriving to the Earth and contribute to the
neutrino flux.


\begin{figure}[b]
\begin{center}
\includegraphics[width=1.\linewidth]{Fluxes.pdf}
\caption{\emph{Left}: atmospheric neutrino fluxes as a function of the neutrino energy for $\nu_\mu$ (blue), $\overline\nu_\mu$ (blue dashed), $\nu_e$ (red), $\overline\nu_e$ (red dashed). \emph{Middle}: Zenith dependence of atmospheric neutrino flux for neutrino energies of 0.3~GeV (dark blue), 1~GeV (blue) and 10~GeV (light blue). \emph{Right}: flux ratios for $\nu_\mu+\overline\nu_\mu/\nu_e+\overline\nu_e$ (purple), $\overline\nu_\mu/\nu_\mu$ (blue), and $\overline\nu_e/\nu_e$ (red).\label{fig:Flux}}
\end{center}
\end{figure}
The primary spectrum of cosmic rays arriving at the Earth is composed of free protons ($\sim 80\%$) and bound nuclei ($\sim20\%$)~\cite{Gaisser:2002jj,Dembinski:2017zsh}, and spans many orders of magnitude in energy, from the MeV to the EeV scale.
This atmospheric neutrino flux exhibits an energy dependence of  $\sim E^{-3}$ around $1$~GeV as can be seen in the left panel of Fig~\ref{fig:Flux}. 
In this work, we focus on neutrinos with
energies from $0.1$~GeV to $100$~GeV, which comprises the
most relevant oscillation physics of atmospheric neutrinos.


Although cosmic rays approach the Earth isotropically, the
interaction with solar winds and the Earth's magnetic field induce
several anisotropies in the neutrino flux in the sub-GeV region, see middle panel of 
Fig.~\ref{fig:Flux}. 
The geomagnetic
effect will also modify the propagation of charged mesons generated after the
primary interaction. 
In particular, low-momentum mesons that would fly away from the Earth can be trapped by the magnetic field, which
enhances
the low energy neutrino flux. 
Also, since most cosmic ray particles have positive charge, the geomagnetic effects induce an east-west
asymmetry~\cite{Super-Kamiokande:1999mpf}. 
As the geomagnetic field is location dependent, so it is the atmospheric neutrino flux~\cite{Honda:2015fha}.

An important aspect of atmospheric neutrinos is that their composition changes for different neutrino energy.
At low energies, pions decays to muon neutrinos and muons, which in turn, if below 1~GeV, tend to decay before reaching the surface.
This induces a flavor ratio $\nu_{e} +
\overline{\nu}_{e}/\nu_{\mu} + \overline{\nu}_{\mu} \sim 1/2$, see right panel of Fig.~\ref{fig:Flux}. 
At higher energies, it is more likely that the muon reaches the surface before decaying, which suppresses this flavor ratio.
Additionally, there is an asymmetry between the neutrino and antineutrino fluxes, due to typical positive charges of cosmic rays.
Nevertheless, at low energies the geomagnetic field traps secondary particles, leading to more collisions in the atmosphere, raising the particle multiplicity and washing out this asymmetry.
As the neutrino
energy increases, the primary mesons created by the proton interaction
dominates the contribution to the neutrino flux and
$\overline{\nu}/\nu$ decreases. 
%


Moreover, the thickness of the atmosphere along the meson/muon trajectory and the geomagnetic field induce an asymmetry on the zenith distribution of the neutrino flux. 
This effect can be seen in the middle panel of Fig.~\ref{fig:Flux} where the muon neutrino flux is shown as a function of the zenith angle for different neutrino energies.


In this work, we use the tables from Ref.~\cite{Honda:2015fha} that are
based on a 3-dimensional simulation of the atmospheric neutrino flux.\footnote{We use the tables corresponding to the atmospheric neutrino flux at the location of Super-Kamiokande, which shares a similar latitude to the future DUNE experiment. While the different effects (magnetic fields, etc.) could yield a slightly different atmospheric neutrino flux in South Dakota, no publicly available version of this exists. We encourage the DUNE collaboration to include detailed simulations of the atmospheric neutrino flux in their future studies.}
The simulation includes the transverse momentum for
all secondary particles as well as geomagnetic field effects. 
The 3-dimensional
propagation of the neutrino flux is more relevant in the sub-GeV
region where there is less correlation between the meson and cosmic ray directions. 

As can be suspected from the discussion above, the determination of the atmospheric neutrino flux is far from trivial and plagued by systematic uncertainties.
The main uncertainties are related to the cosmic ray flux and the hadron production, while others, such as the density of the atmosphere and those associated to the geomagnetic field, are relatively small~\cite{Barr:2006it}. 
To account for such uncertainties in a realistic way, we start by parameterizing the
atmospheric neutrino flux for the flavor $\alpha$ with enough freedom to implement energy and zenith uncertainties, namely,
%
\begin{equation}
  \Phi_{\alpha}(E,\cos\zeta) =  f_{\alpha}(E,\cos\zeta)\Phi_{0}\left(\frac{E}{E_{0}}\right)^{\delta} \eta(\cos\zeta),
\end{equation}
%
where $f_{\alpha}(E,\cos\zeta)$ is the flux of atmospheric neutrinos of flavor is a function of both energy and zenith angle~\cite{Honda:2015fha}.
The remaining factors play the following role:
$\Phi_{0}$ describes the unknown  normalization of the flux; $(E/E_{0})^{\delta}$ induces a spectral tilt, accounting for energy dependence uncertainties; and $\eta(\cos\zeta)$ describes the zenith distribution uncertainty.
In particular, the $\eta(\cos\zeta)$ is chosen to be
\begin{equation}\label{eq:zenith_uncertainty}
  \eta(\cos\zeta) \equiv \left[1 - C_{u}\tanh(\cos\zeta)^{2}\right]\Theta(\cos\zeta)
  				+\left[1 - C_{d}\tanh(\cos\zeta)^{2})\right]\Theta(-\cos\zeta),
\end{equation}
where $\Theta$ is a Heaviside step function.


Let us discuss the dominant systematic uncertainties in the atmospheric neutrino fluxes. 
Following Ref.~\cite{Barr:2006it}, 
the normalization uncertainties of the flux itself, due to uncertainties on the cosmic-ray spectrum, range from 10\% to 40\% in the $0.1$~GeV to the 100~TeV region.
Ratios of flavors on the other hand tend to be better known.
At lower energies, $\overline{\nu}_{e}/\nu_{e}$ exhibits the
largest uncertainty of about $6\%$ since each component is produced by a
different meson decay chain. 
The $\overline{\nu}_{\mu}/\nu_{\mu}$ and
$\nu_{e}/\nu_{\mu}$, ratios are better estimated as these flavors are mostly sourced by the pion decay chain. 
At higher energies, the interaction of
the up-going and down-going muons with the Earth will increase the
uncertainty of $\overline{\nu}/\nu$ along those directions to about $20\%$, and the uncertainty of $\nu_{e}/\nu_{\mu}$ to $\sim5\%$ for
neutrino energies near 100~GeV. 
In our simulations, to be on the conservative side, we adopted the following uncertainties for those quantities for the full energy range: 
overall normalization of the flux of $40\%$; 
$\nu_{e} + \overline{\nu}_{e}$ and $\nu_{\mu} +
\overline{\nu}_{\mu}$ flux ratios of $5\%$; 
neutrino to antineutrino ratio of $2\%$; 
an absolute uncertainty of $0.2$ in the
energy distortion parameter $\delta$; and a zenith distortion parametrized by $C_{u,d}=0\pm0.2$. The uncertainties used in the analysis are summarized in Table~\ref{tab:Fluxuncert}. 


%%%%%%%%%%%%%%%%%%%%%%%%%%%%%%%%%%%%%%%%%%
%%%%%%%%%%%%%%%%%%%%%%%%%%%%%%%%%%%%%%%%%%
\begin{table}
\begin{center}
\caption{Uncertainties and priors in the atmospheric neutrino flux used in our analysis. \label{tab:Fluxuncert}}\vspace{0.1cm}
\begin{tabular}{|c|c|}\hline
Systematic  &Uncertainties/Priors \\ \hline\hline
Normalization ($\Phi_{0}$) & $40\%$ \\ \hline
Flavor ratio ($\nu_{e}/\nu_{\mu}$) & $5\%$ \\ \hline
Neutrino to antineutrino ratio ($\overline{\nu}/\nu$)  & $2\%$ \\ \hline
Energy distortion ($\delta$)   &$0\pm0.2$ \\ \hline
Zenith distortion ($C_{u,d}$)   &$0\pm0.2$ \\ \hline
\end{tabular}
\end{center}
\end{table}
%%%%%%%%%%%%%%%%%%%%%%%%%%%%%%%%%%%%%%%%%%
%%%%%%%%%%%%%%%%%%%%%%%%%%%%%%%%%%%%%%%%%%




As a final comment, our simulation shows that all parameters related to the systematic uncertainties can be measured at DUNE with a precision which is better than what we adopted as priors. 
This is important because otherwise one could suspect that the experimental sensitivity to a given physics measurement could be  driven by the chosen prior on the uncertainties (see also Ref.~\cite{Kelly:2019itm}).
We now proceed to the results of our simulation on DUNE's sensitivity to measure the Earth's matter profile.






