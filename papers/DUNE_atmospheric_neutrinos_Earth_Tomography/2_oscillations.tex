% !TEX root = tomography.tex
%%%%%%%%%%%%%%%%%%%%%%%%%%%%%%%%%%%%%%%%%%%%
\section{Oscillations of Atmospheric Neutrinos}
\label{sec:AtmoNu}
%%%%%%%%%%%%%%%%%%%%%%%%%%%%%%%%%%%%%%%%%%%%


In this section, we review the physics of atmospheric neutrino oscillations.
Before laying out the oscillation formalism and elaborating on the effects of nonzero matter density, we first discuss the Earth density profile.

%%%%%%%%%%%%%%%%%%%%%%%%%%%%%%%%%%%%%%%%%%%%
\subsection{Earth Density Profile}
%%%%%%%%%%%%%%%%%%%%%%%%%%%%%%%%%%%%%%%%%%%%

The current knowledge of the Earth density profile comes from seismological studies, summarized in the \emph{preliminary Earth reference model}~\cite{Dziewonski:1981xy}\footnote{More recent, three dimensional models of the Earth matter profile exist~\cite{REM}, but we will nonetheless avoid dealing with asymmetric profiles.}.
Measurements of the precise matter density and variations of it with respect to the depth are obtained using empirical relations for the wave velocities~\cite{Geller:2001ix}.
The determination of the radii separating the core, inner mantle and outer mantle, on the other hand, is a much more robust observation, as it relies on the reflection of waves.
In view of that, our study will assume known radii separating the different layers of the Earth, while trying to constrain the matter densities in the layers.
Moreover, in many examples and analysis we will keep the mass of the Earth and its moment of inertia fixed, as these quantities are extremely well measured~\cite{ries1992progress, Rosi:2014kva, williams1994contributions, chen2015consistent}.
We will discuss these in details in Sec.~\ref{sec:ResultsDiscussion}.

Since neutrino oscillation data will not have the same fine precision as seismological data, we adopt a simplified model of the Earth profile with three layers\footnote{Since the neutrino trajectory through the crust is a small fraction of its overall path, we do not include the crust in our simulations.}: the core with density $\rho_{\rm C}=11$~g/cm$^3$, the lower mantle with density $\rho_{\rm LM}=5.1$~g/cm$^3$; and upper mantle with $\rho_{\rm UM}=3.1$~g/cm$^3$. As a function of the radius from the center of the Earth, these layers correspond to $r < 3480$ km (core), $3480$ km $\leq r < 5700$ km (lower mantle), and $5700 < r < R_{\scriptscriptstyle\oplus} = 6371$ km (upper mantle).

The possible neutrino trajectories through the Earth depending on the zenith angle $\zeta$ are the following.
\begin{enumerate}
	\item \emph{Core-Mantle}: Neutrinos whose zenith angle is $\zeta \gtrsim 147^{\circ}$ cross the three main layers, core, lower and upper mantle. In this case, the relatively large difference between the densities of the core and mantle  can lead to a parametric resonance, which will be discussed later, depending on the neutrino energies. The specific values for zenith angles and energies where such parametric amplification will depend on the actual values of the densities. 
	\item \emph{Upper-Lower Mantle}: For $116^{\circ} \lesssim \zeta \lesssim 147^{\circ}$, neutrinos only cross the upper and lower mantle layers. We will see that from the known values of the densities, neutrino oscillations can be resonantly enhanced because of the MSW effect for some specific values of the energies and zeniths.
	\item \emph{Upper Mantle Only}: Finally, if neutrinos have a $\zeta \lesssim 116^{\circ}$, they only propagate in the Upper mantle. Neutrino oscillations are then well described by considering matter with constant density. 
\end{enumerate}




%%%%%%%%%%%%%%%%%%%%%%%%%%%%%%%%%%%%%%%%%%%%
\subsection{MSW and Parametric Resonances}
%%%%%%%%%%%%%%%%%%%%%%%%%%%%%%%%%%%%%%%%%%%%

In this section we will review the oscillation phenomenology of atmospheric neutrinos. 
First, we remind the reader that, in the presence of matter, the evolution operator that drives neutrino oscillations is given in the flavor basis by
\begin{equation}\label{eq:hamiltonian}
  H = \frac{1}{2E}
U^\dagger\left(
\begin{array}{ccc}
 0 & 0  & 0  \\
 0 & \Delta m^2_{21}  &  0  \\
 0 & 0  &  \Delta m^2_{31} 
\end{array}
\right)U
+
\left(
\begin{array}{ccc}
 V_{\rm CC} & 0  & 0  \\
 0 & 0  &  0  \\
 0 & 0  &  0 
\end{array}
\right),
\end{equation}
where $U$ is the PMNS matrix, $E$ is the neutrino energy, $\Delta m^2_{ij}\equiv m_i^2-m_j^2$ denotes the solar $\Delta m^2_{21}$ and atmospheric $\Delta m^2_{31}$ mass splittings; and 
\begin{equation}
  V_{\rm CC} = \sqrt{2} G_F n_e
\end{equation}
is the matter potential sourced by electrons, with $G_F$ being the Fermi constant and $n_e$ the electron number density. 
Although electrons, protons and neutrons also induce a neutral current matter potential for neutrinos, the assumption of electricaly neutral matter cancel the contribution of the charged particles. The contribution from neutrons is flavor universal and does not change oscillation phenomenology.
When $V_{\rm CC}$, or equivalently $n_e$, is constant along the path of the neutrino, the evolution of a neutrino state after is propagates a baseline $L$ is simply obtained by $|\nu(L)\rangle = e^{-iHL} |\nu(0)\rangle$. 
The matter potential plays a crucial role in the oscillation of atmospheric neutrinos.
In the following, we will discuss how the Earth density profile affects neutrino oscillations, emphasizing the MSW and parametric resonances.

Neutrino flavor oscillations inside the Earth can present a unique type of resonant enhancement, the \emph{parametric resonance}, due to the abrupt change in density existing between terrestrial layers.
Such amplification results from serendipitous relations between the oscillation phases in the different layers that a neutrino can traverse~\cite{Akhmedov:1998ui,Akhmedov:1998xq,Chizhov:1998ug,Chizhov:1999az,Chizhov:1999he,Liu:1998nb}.
Crucially, the specific relations that the phases need to fulfill depend on the matter profile, so any modification of the densities will lead to significant changes on the oscillations observed in a detector. 

\textbf{MSW Resonances:} While we know that there are three neutrinos in nature, the two neutrino system will provide us with a simple yet useful framework to understand the key features of atmospheric neutrino oscillations.
When neutrinos propagate inside matter, flavor oscillations are modified due to the presence of the potential associated to the electron density. 
When matter sources a potential, the Hamiltonian \eqref{eq:hamiltonian} is no longer diagonalized by the PMNS matrix $U$, which in the two neutrino framework is just a rotation matrix with an angle $\theta$.
Instead, the diagonalization is obtained by an effective mixing angle in matter $\widetilde\theta$ and yields eigenvalues   $\Delta\tilde m^2/2E$, where the numerator denotes the effective mass splitting in matter.
It is easy to show that
\begin{align}
	\Delta\widetilde m^2=\sqrt{\left(\Delta m^2 \cos 2\theta - A_{\rm CC}\right)^2+\left(\Delta m^2 \sin 2\theta\right)^2},\quad \sin 2\widetilde\theta =\frac{\Delta m^2 \sin 2\theta}{\Delta\widetilde m^2},
\end{align}
which depend on the vacuum mixing and mass splitting ($\Delta m^2$), as well as and on the matter contribution $A_{\rm CC}$.
This matter contribution is defined by
\begin{align}
 	A_{\rm CC} = 2\sqrt{2}G_{\rm F} N_A\, E\, Y_e\, \rho \approx 4.5 \times 10^{-4}\ \mathrm{eV}^2 \left(\frac{E}{\mathrm{GeV}}\right) \left(\frac{Y_e}{0.5}\right) \left(\frac{\rho}{6\ \mathrm{g/cm}^3}\right),
\end{align}
where we have written the electron number density as $n_e = N_A Y_e \rho$, in which $N_A$ the Avogadro constant, $Y_e$ the electron fraction, and $\rho$ the density of the medium. 

We see that the presence of matter can modify the mixing angle, and in particular can yield $\sin2\tilde \theta=1$.
This happens at a neutrino energy given by
\begin{equation}\label{eq:energy_msw}
	E_{\rm MSW}=\frac{\Delta m^2 \cos 2\theta}{2\sqrt{2}G_{\rm F}N_AY_e\rho} \simeq
		\begin{cases}
		 5.3~\GeV \left(\frac{\Delta m^2}{2.5\times10^{-3}~\eV^2}\right)\left(\frac{\cos 2\theta}{0.95}\right)\left(\frac{0.5}{Y_e}\right)\left(\frac{6~\g/\cm^{3}}{\rho}\right) &\quad \text{(atmospheric pars.)}\\
		 68~\MeV \left(\frac{\Delta m^2}{7.5\times10^{-5}~\eV^2}\right)\left(\frac{\cos 2\theta}{0.41}\right)\left(\frac{0.5}{Y_e}\right)\left(\frac{6~\g/\cm^{3}}{\rho}\right) &\quad \text{(solar pars.)}
		\end{cases}
\end{equation}
where we took as reference values the atmospheric mass splitting $\Delta m^2_{31}$ and $\theta_{13}$ in the upper line and the solar splitting and $\theta_{12}$ for the lower line, as well as a representative matter density.
We see that even a small vacuum mixing angle can be enhanced by matter effects to drive large flavor transitions.
The transition probability itself is simply
\begin{align}
	P_{\alpha\beta}^{\rm MSW}= \sin^22\widetilde\theta\sin^2\left(\frac{\Delta\widetilde m^2 L}{4E}\right),\qquad \alpha\neq\beta.
\end{align}
A maximal transition probability, $P_{\alpha\beta}^{\rm MSW}=1$, will be possible for a specific neutrino energy, see Eq.~\eqref{eq:energy_msw}, and when the \emph{phase} of oscillation $\phi$ has the right value,
\begin{align*}
	\phi\equiv\frac{\Delta\widetilde m^2 L}{4E} = \frac{2k+1}{2}\pi,\quad k=0,1,2,\ldots.
\end{align*}
Thus, to have a complete flavor conversion in matter, both the MSW energy and phase conditions should be fulfilled simultaneously. 

Assuredly, the accomplishment of such conditions takes place for very specific values of the neutrino energy and distance. 
However, atmospheric neutrinos cover a broad energy spectrum, from the tens of MeV up to multi-TeV, as well as a large baseline interval, from hundreds of kilometers to the Earth's diameter of $12,742$~km.
Hence, one would hope that the maximal transition probability can be observed for atmospheric neutrinos if the incoming neutrino energy and direction can be reconstructed with some  accuracy.

To get a more quantitative idea, if we take the average Earth density $\bar{\rho} = 5.5 {\rm~g/cm^{3}}$ as a representative example, we find that the MSW resonance is achieved for $E_{\rm MSW}\simeq 5.73~\GeV$  for the atmospheric splitting, while  $E_{\rm MSW}\simeq 70~\MeV$ for the solar sector.
Meanwhile, the smallest distance required to have a maximal transition should be $L=9,\!580~{\rm km}$ and $L=1,\!258~{\rm km}$ for the atmospheric and solar sectors, which correspond to zenith angles of $\zeta = 138^{\circ}$ and $\zeta = 95^{\circ}$, respectively.
While this suggests that two MSW resonances indeed take place for atmospheric neutrinos, this is not the whole story: the multi-layer structure of the Earth significantly modifies the neutrino state evolution, leading to parametric resonances.

\textbf{Parametric Resonances:} The description of neutrinos propagating in a matter profile consisting of layers with different densities has been studied extensively~\cite{Akhmedov:1998ui,Akhmedov:1998xq,Chizhov:1998ug,Chizhov:1999az,Chizhov:1999he,Liu:1998nb,Peres:1999yi,Freund:1999vc,Palomares-Ruiz:2004cmm,Akhmedov:2005yj,Akhmedov:2006hb,Gandhi:2004bj,Gandhi:2004md,Kimura:2004vh,Gonzalez-Garcia:2004pfd,Liao:2007re,Akhmedov:2008qt,Jacobsson:2001zk,Ohlsson:1999um,Koike:2009xf,Akhmedov:2016hcb}.
For completeness, we provide here a brief discussion and derivation of the required conditions that lead to a parametric enhancement of the oscillation of atmospheric neutrinos. First, let us consider the evolution operator for two-neutrino oscillations in the flavor basis. This operator evolves an initial state $\nu(L_0)$ to a final state $\nu(L)={\cal S}(L-L_0)\nu(L_0)$, where the two-flavor state is $\nu^{2\times 2}=(\nu_\alpha, \nu_\beta)^T$. As above, for a time-invariant Hamiltonian, ${\cal S}(L)\equiv {\rm exp}(-iHL)$. To study parametric resonances, we will first assume that the Earth consists of two layers, the core and the mantle, each with a constant density $\rho_{\rm C}$ and $\rho_{\rm M}$, respectively.

The $2\times2$ Hamiltonian can be written as a linear combination of Pauli matrices, since any component proportional to the identity matrix does not contribute to neutrino oscillations. Thus we can always write the evolution operator ${\cal S}_X$ for a layer $X = \mathrm{C,\ M}$ as~\cite{Akhmedov:1998ui,Akhmedov:1998xq,Chizhov:1998ug,Chizhov:1999az,Chizhov:1999he}
\begin{align}
	{\cal S}_X = \cos \phi_X \mathbbm{1}_2- i\sin \phi_X \, \vec{\sigma}\cdot\overrightarrow{n_X},
\end{align}
where  $\vec{\sigma}\equiv\{\sigma_1,\sigma_2,\sigma_3\}$ is a vector of Pauli matrices and $\mathbbm{1}_2$ is the $2\times 2$ identity matrix.
The oscillation phase in the layer is given by
\begin{align*}
\phi_X = \frac{\Delta\widetilde m_{X}^2 L_X}{4E}
\end{align*}
being $L_X$ the distance traveled and $\Delta\widetilde m_{X}^2$ the effective mass splitting  within the layer. 
The unit vector $ \overrightarrow{n_X}$ corresponds to
\begin{align}
	\overrightarrow{n_X} = (\sin 2\theta_X, 0, -\cos 2 \theta_X),
\end{align}
where $\theta_X$ is the angle in matter within the layer $X$.
Note that we have dropped the tilde in the effective mixing angles in matter to avoid clutter, as the layer subscript already distinguishes them from the vacuum angle.
This parametrization of the evolution operator is useful to understand the evolution of neutrinos inside the Earth. 

%%%%%%%%%%%%%%%%%%%%%%%%%%%%%%%%%%%%%%%%%%%%%%%%%%%%%%%%%%%%%%%%%%%
\begin{figure}[t]
\begin{center}
\includegraphics[width=0.55\linewidth]{layers.pdf}
\caption{Pictorial representation of toy 2-layer Earth model (left) and 3-layer model (right). \label{fig:layers}} 
\end{center}
\end{figure}
%%%%%%%%%%%%%%%%%%%%%%%%%%%%%%%%%%%%%%%%%%%%%%%%%%%%%%%%%%%%%%%%%%%
To begin with, let us consider neutrinos crossing two different layers, that is, neutrino go through the mantle, then through the core and back to the mantle, as pictorially shown in the left panel of Fig.~\ref{fig:layers}. 
In this case, the full evolution operator consists in the multiplication of operators for each crossed layer,
\begin{align*}
	{\cal S} &= {\cal S}_{\rm M}\,{\cal S}_{\rm C}\,{\cal S}_{\rm M},
\end{align*}
Such multiplication can be presented in a simpler form after using the properties of Pauli matrices,
\begin{align}
	{\cal S} & = W_0 \mathbbm{1}_2 - i \vec{\sigma}\cdot\overrightarrow{W},
\end{align}
where $W_0, \overrightarrow{W}=\{W_1,0,W_3\}$, depend on the phases and amplitudes in both layers~\cite{Akhmedov:1998ui,Akhmedov:1998xq,Chizhov:1998ug,Chizhov:1999az,Chizhov:1999he}, namely
\begin{subequations}
\begin{align}
 W_0 &= \cos 2\phi_{\rm M}\cos \phi_{\rm C} -\cos 2(\theta_{\rm M}-\theta_{\rm C}) \sin 2\phi_{\rm M}\sin \phi_{\rm C},\\
 W_1 &=  \sin 2 \theta_{\rm M}\sin 2\phi_{\rm M}\cos \phi_{\rm C} +(\cos 2(\theta_{\rm M}-\theta_{\rm C}) \sin 2\theta_{\rm M}\cos 2\phi_{\rm M}- \cos2\theta_{\rm M}\sin 2(\theta_{\rm M}-\theta_{\rm C}))\sin \phi_{\rm C},\\
 W_3 & = -\sin 2 \theta_{\rm M}\sin 2(\theta_{\rm M}-\theta_{\rm C}) \sin\phi_{\rm C} - (\sin2\phi_{\rm M}\cos\phi_{\rm C}+\cos 2(\theta_{\rm M}-\theta_{\rm C}) \cos 2 \phi_{\rm M}  \sin\phi_{\rm C})\cos 2\theta_{\rm M}.
 \end{align}
\end{subequations}
Although complicated at first sight, these contain the conditions for a parametric amplification of oscillations. 
Notice that the two-neutrino survival and appearance probabilities are simply
\begin{align}
	P_{\alpha\alpha}^{\rm 2f} = |W_0|^2+|W_3|^2; \hspace{3cm}
  	P_{\alpha\beta}^{\rm 2f}  = |W_1|^2\quad (\alpha\neq\beta).
\end{align}

A complete neutrino flavor conversion can only take place if
\begin{align}
	W_0 = 0,\qquad W_3=0.
\end{align}  
The solution of the previous system of equations is given by~\cite{Chizhov:1998ug,Chizhov:1999az,Chizhov:1999he},
\begin{align}\label{eq:SolPR}
	\tan^2\phi_{\rm M}&=-\frac{\cos 2\theta_{\rm C}}{\cos 2(2\theta_{\rm M}-\theta_{\rm C})},\quad\text{and}\quad\tan^2\phi_{\rm C}=-\frac{\cos^2 2\theta_{\rm M}}{\cos 2\theta_{\rm C}\cos 2(2\theta_{\rm M}-\theta_{\rm C})}.
\end{align}
These conditions relate the phases and mixing angles in matter between the two layers: the oscillation phases in both layers depend only on the mixing angles in matter, and therefore on the densities of the layers. 
Thus, a parametric resonance arises whenever the oscillation phases in both layers produce a constructive interference that leads to the complete conversion.
The solutions can only be fulfilled when
\begin{align}\label{eq:ConPR}
	\cos 2\theta_{\rm C} \leq 0 \text{~and~} \cos 2(2\theta_{\rm M}-\theta_{\rm C}) \geq 0\quad\text{or}\quad
	\cos 2\theta_{\rm C} \geq 0 \text{~and~} \cos 2(2\theta_{\rm M}-\theta_{\rm C}) \leq 0.
\end{align}
If the neutrino energy is above the MSW resonance in the second layer, these translate to  $\theta_{\rm C}\leq 2\theta_{\rm M}\leq \pi/4+\theta_{\rm C}$.
Conversely, below the MSW energy one would require $\pi/4+\theta_{\rm C}\leq 2\theta_{\rm M}\leq \pi/2+\theta_{\rm C}$. 

We present in Fig.~\ref{fig:ConPR} the regions in light-blue where the necessary conditions \eqref{eq:ConPR} are satisfied in the plane $A_{\rm CC}^{\rm M}$ vs $A_{\rm CC}^{\rm C}$, where $A^X_{\rm CC}$ is the matter contribution in each layer, assuming the two distinct sets of oscillation parameters: solar parameters $\Delta m_{21}^2$ and $\theta_{12}$ in the left; and atmospheric parameters $\Delta m_{31}^2$ and $\theta_{13}$ in the right\footnote{For definiteness, we are assuming normal mass ordering, that is, $\Delta m^2_{31}>0$.}. 
We also color code the values of the $A_{\rm CC}^X$ as a function of the neutrino energy and fixing the densities of core and mantle to $\rho_{\rm C}=11$~g/cm$^3$ and $\rho_{\rm M}=4.13$~g/cm$^3$.
The latter is the average density between the upper and lower mantle. 
We observe that the conditions can be satisfied for the solar parameters in the low energy regime, $E\lesssim 230~\MeV$; meanwhile for the atmospheric sector, the energies are more restricted, $2.9~\GeV\lesssim E \lesssim 9.7~\GeV$.
The important message here is that the regions where parametric resonances take place depend on the matter densities of both layers, and thus the observation of large flavor conversions, or the lack of it, provides an invaluable tool for the determination of such densities with atmospheric neutrino oscillations.
%%%%%%%%%%%%%%%%%%%%%%%%%%%%%%%%%%%%%%%%%%%%%%%%%%%%%%%%%%%%%%%%%%%
\begin{figure}[t]
\begin{center}
\includegraphics[width=1\linewidth]{Cond_Par.pdf}
\caption{Values of $A_{\rm CC}^X$, $X=\mathrm{M,C}$, that could lead to a parametric resonance assuming the solar (left) and atmospheric (right) oscillation parameters. The light blue region corresponds to the values that fulfill the conditions~\eqref{eq:ConPR}. The diagonal colored region corresponds to the values of $A_{\rm CC}$ that are present for the core and mantle densities ($\rho_{\rm C}=11~{\rm g \,cm^{-3}},\ \rho_{\rm M}=4.13~{\rm g \,cm^{-3}}$)  as function of the neutrino energy. \label{fig:ConPR}} 
\end{center}
\end{figure}
%%%%%%%%%%%%%%%%%%%%%%%%%%%%%%%%%%%%%%%%%%%%%%%%%%%%%%%%%%%%%%%%%%%

In the region where the parametric resonance can take place, the neutrino baselines required for a full conversion can be estimated from the solutions in Eq.~\eqref{eq:SolPR}. 
Let us note, however, that for atmospheric neutrinos the traveled distances in each layer are not independent, but instead they depend on the zenith angle at which neutrinos cross the Earth. 
For a two-layered model of the Earth, where the mantle layers are combined into one, we have that the baselines on the mantle $L_{\rm M}$ and core $L_{\rm C}$ are related to the zenith angle $\zeta$ as
\begin{align}
 L_{\rm M} = R_{\scriptscriptstyle\oplus}\left(-\cos\zeta-\sqrt{f_c^2-\sin^2\zeta}\right),\qquad L_2 = 2R_{\scriptscriptstyle\oplus}\sqrt{f_c^2-\sin^2\zeta},
\end{align}
where $f_c\equiv R_{\rm core}/R_{\scriptscriptstyle\oplus}$, $R_{\rm core}$ being the core radius. 
Thus, for a given set of densities, we can determine the values of neutrino energies and zenith angle at which a parametric resonance occurs.
Such values are by no means unique; there are several parameters which could produce such resonances. 

%%%%%%%%%%%%%%%%%%%%%%%%%%%%%%%%%%%%%%%%%%%%%%%%%%%%%%%%%%%%%%%%%%%
\begin{figure}[t]
\begin{center}
\includegraphics[width=0.49\linewidth]{Par1_Plot_Sol.pdf}
\includegraphics[width=0.49\linewidth]{Par1_Plot_Atm.pdf}
\caption{Set of neutrino energies and zenith angles that produce a parametric resonance for several values of the mantle density for the solar (left) and atmospheric (right) driven oscillations. In both cases we fix the core density to be $\rho_{\rm C}=10~{\rm g\ cm^{-3}}$. We also present three specific cases as ``A'' (five-pointed star), ``B'' (six-pointed star), and ``C'' (seven-pointed star), together with the probablity $P_{\alpha\beta}$ as function of the travelled distance inside the Earth. The dashed lines indicate the matter profile normalized w.r.t.~the core density. \label{fig:DepRhoPR}}
\end{center}
\end{figure}
%%%%%%%%%%%%%%%%%%%%%%%%%%%%%%%%%%%%%%%%%%%%%%%%%%%%%%%%%%%%%%%%%%%
To appreciate the dependence of the parametric resonance on the matter densities of the distinct layers, in Fig.~\ref{fig:DepRhoPR} we show a family of solutions as a function of the core matter density that would lead to a full parametric conversion ($P_{\alpha\beta}=1$) for given neutrino energies and zenith angles  for the solar (large left panel) and atmospheric (large right panel) sectors. 
In these, we have  kept the core density fixed at 10~g$/$cm$^3$. 
As we can see, the zenith-energy location of the parametric conversion presents a strong dependence with the core density.

For oscillations driven by solar parameters, we observe that for larger values of the mantle density, we require smaller values of the neutrino energy and larger values of the zenith angle. 
This is required because by increasing the mantle density, the mass splitting in matter increases, while the amplitude decreases --- these energies are above the MSW resonance in both mantle and core --- so  $L_{\rm M}/E$ has to increase to maintain the phase relations that produce the parametric resonance.
To show how the parametric conversion takes place, we show three representative oscillation curves (smaller panels) for each solar (left) and atmospheric (right) sectors, as labeled in the large panels.
The appearance probabilities are given as function of the traveled distance $L$ inside the Earth.
Indeed, the three examples, ``A'', ``B'', and ``C'', have a similar behavior: the phase in the mantle is less than $\pi/2$ while in the core is $3\pi/2\lesssim \phi_{\rm C}\lesssim 2\pi$, so that the core \emph{boosts} the oscillation leading to a full conversion.

For the atmospheric sector, the behavior is more evolved. 
For small densities in the mantle ($\rho_{\rm M}\lesssim 1~{\rm g\ cm^{-3}}$), the energies that lead to a resonance are $E\lesssim 3.5~\GeV$. 
A MSW resonance occurs in the core  for such energies, so the phase required in the mantle is close to $\pi$ in order to preserve the MSW effect, see the  example ``A'' in right side of Fig.~\ref{fig:DepRhoPR}. 
In fact, the phase in the core is $\sim \pi/2$, thus leading to a maximum conversion.
When the matter density in the mantle increases, so that the matter effects are much more relevant, the phase becomes smaller than $\pi$, and the resonance appears because of the interplay between the mantle and core (example ``B'').
Interestingly, the baseline in the core required for the parametric enhancement becomes smaller for higher energies. 
In the extreme cases ($\rho_{\rm M}\gtrsim 3.2~{\rm g\ cm^{-3}}$), the oscillation becomes dominated by matter effects in the mantle, so that the required energies are close to the MSW values.
Let us restate that in the figure above we only show one family of solutions for the parametric resonance condition. 
There are other values of energies and zenith angles that lead to  parametric enhancements besides those presented above.

Although the solutions in Eq.~\eqref{eq:SolPR} are general and lead to a full conversion, there are other combinations of phases and angles that produce \emph{local} maxima~\cite{Akhmedov:1998ui,Akhmedov:1998xq,Chizhov:1998ug,Chizhov:1999az,Chizhov:1999he} as opposed to full flavor conversion. 
Such maxima could be present even in the cases where the conditions in Eq.~\eqref{eq:ConPR} for a full conversion are not fulfilled or the phases $\phi_X$ that come from the solutions are not possible in the system.
In a first type of local maximum, the oscillation phases in \emph{both} layers are integer multiples of $\pi/2$, i.~e.,
\begin{align}\label{eq:FType}
\begin{cases}
	\cos\phi_X= 0, & \text{or}\ \phi_X = \frac{2k + 1}{2}\pi,\quad k = 0,1,2,\ldots, \\
	\cos\phi_Y=0, & \text{or}\ \phi_Y= \frac{2k^\prime + 1}{2}\pi,\quad k^\prime  = 0,1,2,\ldots.
\end{cases}
\end{align}
The resulting probability is very simple and given by
\begin{align}
	P_{\alpha\beta}^{\rm 2f} = \sin^2 2(2\theta_{\rm M}-\theta_{\rm C}).
\end{align}
Note that if indeed $2\theta_{\rm M}-\theta_{\rm C}=(2k+1)\pi/4$, with $k\in \mathbb{Z}$, this leads to full conversion, as it is equivalent to fulfilling the conditions in Eq.~\eqref{eq:ConPR}. 
Nevertheless, even if the oscillation probability is not unity, the flavor conversion still reaches a local maximum.

The second type of local maxima appears when we have a full oscillation cycle in one layer, say $X$, so that the final oscillation probability would come only from the matter effect in the other layer $Y$. This is realized when
\begin{align}
\begin{cases}
	\sin\phi_X= 0, & \text{or}\ \phi_X = k\pi,\quad k = 0,1,2,\ldots, \\
	\cos\phi_Y=0, & \text{or}\ \phi_Y= \frac{2k^\prime + 1}{2}\pi,\quad k^\prime  = 0,1,2,\ldots;
\end{cases}
\end{align}
the probability is then equal to
\begin{align}
	P_{\alpha\beta}^{\rm 2f} = \sin^2 2\theta_Y.
\end{align}
Again, if $\sin^22\theta_Y=1$ we have full flavor conversion, and this would correspond to a MSW resonance in the layer $Y$.
Nevertheless, if the MSW resonance is not achieved, this would still correspond to a local maximum.
The example ``A'' in the right side of Fig.~ \ref{fig:DepRhoPR}, for the atmospheric parameters, is close to this type of maximum, although the phases are not exactly equal to $\pi$ and $\pi/2$ in the mantle and core, respectively.

\vspace{0.5cm}
\textbf{Extension to three-layer Earth:} Now we adjust our analysis to include separate densities in the lower mantle (LM) and upper mantle (UM), which requires us to consider neutrino trajectories that cross three separate layers, as indicated in the right panel of Fig.~\ref{fig:layers}. Here, the evolution operator is
\begin{align}
	{\cal S} &= {\cal S}_{\rm UM}\,{\cal S}_{\rm LM}\,{\cal S}_{\rm C}\,{\cal S}_{\rm LM}\,{\cal S}_{\rm UM} = W_0 \mathbbm{1}_2 - i \vec{\sigma}\cdot\overrightarrow{W},
\end{align}
where ${\cal S}_{\rm UM}$,  ${\cal S}_{\rm LM}$,  ${\cal S}_{\rm C}$ are the evolution operators in the upper mantle, lower mantle and core, respectively. The $W_i$ here are distinct from those in the two-layer case, and can be calculated for the three-layer case. However, we omit them here because their explicit expressions are rather long and not illuminating. As in the two-layer case, a parametric resonance is achieved when $W_0=W_3=0$. Given the intricacy of the expressions, it is not possible to obtain exact solutions to these conditions. Nevertheless, we can solve numerically the system of equations to determine the values of energies and zenith angle required to get a parametric amplification. 

We present in Fig.~\ref{fig:DepRhoPR3l} an oscillogram for a representative set of densities $\rho=\{13.5,\,5.0,\,3.0\}~{\rm g/cm^{3}}$ for the core, lower mantle, and upper mantle, respectively, for oscillations driven by the solar (left) and atmospheric (right) parameters.
We draw the contours in which $W_0=0$ (black lines) and $W_3=0$ (white dashed lines).
The full parametric conversion happens in the intersection of such contours.
As before, we show three examples of appearance probability as a function of the distance traveled by the neutrino in which  parametric resonances occur.
These correspond to the parameters where the $W_0=0$ and $W_3=0$ contours cross, labeled as ``A'', ``B'', or when they are close to each other (``C'').

Comparing Figs.~\ref{fig:DepRhoPR} and \ref{fig:DepRhoPR3l}, we see that the 2-layer model essentially captures all relevant oscillation physics that takes place in the 3-layer setup.
For the solar sector, in the left panels, the phase developed in the core ranges from $\sim 1.5\pi$ for the point ``C'' to $\sim 2.5\pi$ for the point ``A.''
The sum of phases in case ``A'' in the mantle is about $0.46\pi$, so the resonance appears due to the synergy between both mantle layers and the core. 
In contrast, the total phase in the mantle is larger for points ``B'' and ``C'', $\phi_{\rm UM}+\phi_{\rm LM}\approx 0.7\pi$ due to the larger mantle baseline.
In the core, on the other hand, the phases are reduced relative to ``A'', $\phi_{\rm C}^B\approx 2\pi, \phi_{\rm C}^C\approx 1.32\pi$.
These combinations of phases and mixing angles in matter lead to a parametric resonance in the point ``B,'' while for the point ``C'' there is not a complete conversion, $P_{\alpha\beta} = 0.95$.
%%%%%%%%%%%%%%%%%%%%%%%%%%%%%%%%%%%%%%%%%%%%%%%%%%%%%%%%%%%%%%%%%%%
\begin{figure}[t]
\begin{center}
\includegraphics[width=0.485\linewidth]{3-layers_PR_Sol.pdf}
\includegraphics[width=0.485\linewidth]{3-layers_PR_Atm.pdf}
\caption{Oscillograms for three-layer Earth matter profile with $\rho=\{13.5,\,5,\,3\}~{\rm g\ cm^{-3}}$ for the core, lower mantle, and upper mantle, respectively. The lines represent the parameters where the conditions $W_0=0$ (black), and $W_3=0$ (white dashed) are satisfied. We present three examples where both conditions are (almost) fulfilled: points ``A'' and ``B'', where there is a parametric resonance, and point ``C'' where there is a large amplification in the oscillations. We further present the probabilities as function of the traveled distance inside the Earth together with the matter profile normalized to the Core density. \label{fig:DepRhoPR3l}}
\end{center}
\end{figure}
%%%%%%%%%%%%%%%%%%%%%%%%%%%%%%%%%%%%%%%%%%%%%%%%%%%%%%%%%%%%%%%%%%%

For atmospheric parameters, in the right side pf Fig.~\ref{fig:DepRhoPR3l}, we have for point ``A'' that the enhancement comes mainly from the matter effect in the core ---actually an MSW resonance--- since the phase developed in the mantle is $\phi_{\rm UM}+\phi_{\rm LM}\approx\pi$. 
The parametric resonance in point ``B'' relies more in the interplay between mantle and core matter effects, as the phases in both mantles and core are sizable, $\phi_{\rm UM}+\phi_{\rm LM}\approx0.7\pi, \phi_{\rm C}\approx 0.63\pi$, and not a multiples of $\pi$. 
Finally, for point ``C,'' full parametric conversion is almost achieved as the appearance probability is $P_{\alpha\beta} = 0.98$. 
From these numerical studies, we can conclude that the  2-layer approximation works well for both solar and atmospheric sectors.
The reason is the relative smallness of the upper mantle.

%%%%%%%%%%%%%%%%%%%%%%%%%%%%%%%%%%%%%%%%%%%%%%%%%%%%%%%%%%%%%%%%%%%
\begin{figure}
\begin{center}
\includegraphics[width=0.8\linewidth]{Plot_Ev_l_Solar.pdf}
\caption{Two-flavor oscillation probability $P_{\alpha\beta}^{\rm 2f}$ as function of the travelled distance inside the Earth (left) and energies (right) assuming the true density values of the layers, $\rho=\{11,5.1,3.1\}~{\rm g\ cm^{-3}}$ (top), and values that produce the same Earth mass and moment of inertia $\rho=\{17,1.27,7.18\}~{\rm g\ cm^{-3}}$ (bottom). The oscillation parameters here correspond to the solar values $\Delta m_{21}^2, \theta_{12}$. The red, blue and green lines in the left panels correspond to the stars on the right, and are extrema of the oscillation probability in the true scenario. \label{fig:PSol}}
\end{center}
\end{figure}
%%%%%%%%%%%%%%%%%%%%%%%%%%%%%%%%%%%%%%%%%%%%%%%%%%%%%%%%%%%%%%%%%%%
To further understand the dependence of the parametric resonance on the densities in this 3-layer case, we compare in Figs.~\ref{fig:PSol} and \ref{fig:PAtm} the oscillation probabilities as a function of the neutrino energy, for solar and atmospheric parameters, respectively, assuming the true values of the densities, $\rho=\{11,5.1,3.1\}~{\rm g\ cm^{-3}}$ (top panels), and values that produce the same Earth mass and moment of inertia $\rho=\{17,1.27,7.18\}~{\rm g\ cm^{-3}}$ (bottom panels).
In all cases we fix the zenith angle at $\cos\zeta=-0.9$. 
We observe that a change in the matter profile can lead to a significant modification on the oscillation pattern. 

For the oscillations generated by solar parameters, see upper right panel in Fig.~\ref{fig:PSol}, we find that for the true density values, there are two local maxima at $E\sim 100~\MeV$ (blue star), $E\sim210~\MeV$ (red star), and a global minimum at $E\sim 130~\MeV$ (green star). 
The origin of such extrema can be inferred from the different values of the phases acquired in the layers, as seen in the upper left panel where the oscillation probability for each point is shown as a function of the baseline (the color of the lines correspond to the color of the stars). 
For the first maximum (blue), the total phase in the mantle is $\sim 0.96\pi$ while in the core is $\sim 1.53\pi$. 
Thus, the final value of the  oscillation comes mostly from the matter effect in the core.

The origin of the second maximum (red) is similar to the one of point ``C'' in the left side of  Fig.~\ref{fig:DepRhoPR3l}. 
The phase in the mantle is $\phi_{\rm UM}+\phi_{\rm LM}\approx 0.57\pi$, while in the core $\phi_{\rm C}\approx 1.37\pi$, so that the enhancement here is purely parametric: the phases and angles conspire to give a close-to-maximal probability.
The absolute minimum (green), on the contrary, appears due to the destructive interference. 

In contrast, the density values in the bottom panel were chosen so that the oscillation probabilities at the same neutrino energies  were interchanged: the maxima become minima, while the minimum is close to the absolute maximum.
This is a result of the modification of the phases and angles in matter, as clearly observed in the left panels. 
The largest modification occurs in the core, where the amplitude of the oscillation is reduced while the phase increases since the density is larger than the true value in this example.

When considering atmospheric parameters in Fig.~\ref{fig:PAtm}, a similar behavior is present. 
For the true density values, top panels, we identify three different maxima (red, blue and green).
Such maxima arise from different phase and angle relations and the interplay between mantle and core densities. 
If we modify the values of the densities in the three layers, again fixing the total Earth mass and moment of inertia, we observe that the oscillation pattern is remarkably altered. 
In fact, due to the modification on the phases two of previous maxima turn into minima, similar to the solar example. 
Maxima appear for other values of the energies, as the parametric conditions would be fulfilled for other parameters given the changes in the densities. 
This serves to show more explicitly that measuring the minima or maxima of atmospheric neutrino oscillations, for both sub-GeV and multi-GeV energies, allows us to constrain the Earth matter profile.
%%%%%%%%%%%%%%%%%%%%%%%%%%%%%%%%%%%%%%%%%%%%%%%%%%%%%%%%%%%%%%%%%%%
\begin{figure}
\begin{center}
\includegraphics[width=0.8\linewidth]{Plot_Ev_l_Atm.pdf}
\caption{Similar to Fig.~\ref{fig:PSol}, but considering atmospheric parameters instead \label{fig:PAtm}}
\end{center}
\end{figure}
%%%%%%%%%%%%%%%%%%%%%%%%%%%%%%%%%%%%%%%%%%%%%%%%%%%%%%%%%%%%%%%%%%%

%%%%%%%%%%%%%%%%%%%%%%%%%%%%%%%%%%%%%%%%%%%%
\subsection{Three-flavor neutrino oscillations in the Earth}
%%%%%%%%%%%%%%%%%%%%%%%%%%%%%%%%%%%%%%%%%%%%

Heretofore we have considered a two-flavor system, studying in detail the parametric effects on the oscillations and their dependence on the layer densities. 
The extension to three-flavors has been analyzed in detail in previous literature~\cite{Ohlsson:1999um,Palomares-Ruiz:2004cmm,Akhmedov:2005yj,Akhmedov:2006hb,Akhmedov:2008qt}. 
To avoid repetition, we refer the reader to those works and only quote their take home message.
Given the hierarchy existing between the solar and atmospheric mass splittings $|\Delta m_{31}^2|/\Delta m_{21}^2\approx 34$,  there exist different regions where the oscillation probability is driven by either solar \textit{or} atmospheric parameters. 
Such regions are
\begin{itemize}
	\item \emph{Solar limit.} When the neutrino energies are below that MSW resonance energies for atmospheric parameters, $E\ll E_{\rm MSW}^{\rm atm}$, we can consider the approximation $\sin\theta_{13}\to 0$. 
	In such a case, 
	the appearance probability $P_{\nu_\mu\to\nu_e}$ is
	\begin{align}\label{eq:3fPS}
		P_{\nu_\mu\to\nu_e} 
		\approx\cos^2\theta_{23}P^{\rm 2f}_{\alpha\beta}(\Delta m_{21}^2,\theta_{12}),
	\end{align}
	where $P^{\rm 2f}_{\alpha\beta}(\Delta m_{21}^2,\theta_{12})$ is the two-flavor probability computed in the previous section considering the solar parameters. 
	From this we see that the modification coming from the full three-flavor evolution corresponds to the factor $\cos^2\theta_{23}\approx 0.5$ since $\theta_{23}$ is close to maximal mixing.
	\item \emph{Atmospheric limit.} On the other hand, if we consider the limit when $\Delta m_{21}^2\to 0$ and/or $\theta_{12}\to 0$, 
	the appearance probability simplifies to 
	\begin{align}\label{eq:3fPA}
		P_{\nu_\mu\to\nu_e}
		\approx\sin^2\theta_{23}P^{\rm 2f}_{\alpha\beta}(\Delta m_{31}^2,\theta_{13}).
	\end{align}
	Hence we observe that the correction is similar to the Solar limit, given that $\theta_{23}\approx \pi/4$. 
	Let us notice that this limit is valid when the neutrino energies are above the MSW energy for solar parameters $E_{\rm MSW}^{\rm sol}\ll E$ since $\theta_{12}$ in matter tends to be small. 
	When $E\sim E_{\rm MSW}^{\rm atm}$, the $\theta_{13}$ angle in matter becomes large, and the 1-3 level crossing induces corrections dependent on the solar parameters~\cite{Akhmedov:2008qt}. These corrections are responsible to the slightly mismatch between two and three flavor probabilities in the larger peaks above 1 GeV. 
\end{itemize}

%%%%%%%%%%%%%%%%%%%%%%%%%%%%%%%%%%%%%%%%%%%%%%%%%%%%%%%%%%%%%%%%%%%
\begin{figure}
\begin{center}
\includegraphics[width=0.9\linewidth]{2_vs_3f.pdf}
\caption{Comparison of the full three-flavor appearance oscillation probability $P_{\mu e}$ (blue) with the solar (yellow) and atmospheric (green) approximation from Eqs.~\eqref{eq:3fPS} and~\eqref{eq:3fPA}, respectively. \label{fig:2vs3f}}
\end{center}
\end{figure}
%%%%%%%%%%%%%%%%%%%%%%%%%%%%%%%%%%%%%%%%%%%%%%%%%%%%%%%%%%%%%%%%%%%
In Fig.~\ref{fig:2vs3f}, we show the approximated probabilities~\eqref{eq:3fPS} (yellow) and~\eqref{eq:3fPA} (green) together with the full three-flavor oscillations (blue) for two sets of densities for the distinct layers of the Earth, true parameters $\rho=\{11,\,5.11,\,3.15\}~{\rm g\ cm^{-3}}$ (left), and the set $\rho=\{17,\, 1.27,\, 7.18\}~{\rm g\ cm^{-3}}$ which leads to the same Earth mass and moment of inertia. 
We have adopted an ad hoc cut on the neutrino energy to separate the two previous regimes at $E=1~\GeV$. 
From the previous figure, we can observe that the approximated probabilities reproduce the main features of the full neutrino oscillation probability, specially the points where the parametric enhancement occurs. 
Specifically, at lower energies, we observe that the maxima and minima from the three-neutrino probability are well described by the two-flavor probability studied before, with the correction coming from the $2-3$ mixing. 
The fast oscillations come from the $\Delta m_{31}^2$ driven oscillations.  

At higher energies, where the solar mixing in matter becomes negligible, we find again that the two-flavor approximation reproduces reasonably well the correct oscillation probability. 
In fact, the energies where the three maxima observed in Fig.~\ref{fig:PAtm} (top panels) are similar to those in the full three-flavor probability (left). The approximation also works for both density sets. 
Thus, we can conclude that the discussion in the previous section describes rather well the dependence on the densities of the three-neutrino oscillations that can be observed in future facilities. 
From now on, we consider the full neutrino oscillations in the three-flavor scenario assuming the three-layer Earth density for the simulations that we will perform.

%%%%%%%%%%%%%%%%%%%%%%%%%%%%%%%%%%%%%%%%%%%%%%%%%%%%%%%%%%%%%%%%%%%
\begin{figure}[t]
\begin{center}
\includegraphics[width=0.5\linewidth]{Vary_MDP_Smooth.pdf}
\caption{Oscillation probability $P(\nu_\mu \to \nu_e)$ for three different zenith angles, $\cos\zeta=-0.25$ (top panel),
$\cos\zeta=-0.5$ (middle panel) $\cos\zeta=-0.9$ (bottom panel) assuming the true mass density profile (MDP) (violet), the average density on the trajectory (dashed green), and vacuum (gray). \label{fig:Pmue_VaryMDP}}
\end{center}
\end{figure}
%%%%%%%%%%%%%%%%%%%%%%%%%%%%%%%%%%%%%%%%%%%%%%%%%%%%%%%%%%%%%%%%%%%
To firmly establish the importance of the parametric enhancement on atmospheric neutrino oscillations, we present in Fig.~\ref{fig:Pmue_VaryMDP} the full oscillation probabilities for the assumed true mass density profile (MDP) (violet), together with the probability assuming constant matter density equal to the average density on the trajectory (dashed green), and vacuum oscillations (thin gray) for different zenith angles, $\cos\zeta=-0.25$ (top panel), $\cos\zeta=-0.5$ (middle panel) $\cos\zeta=-0.9$ (bottom panel). 
Notice that we have included an energy resolution of 10\% to smear out experimentally unobservable fast wiggles.

For trajectories that only cross the upper mantle, $\cos\zeta=-0.25$, the average density on the trajectory is equal to the density predicted by the MDP since in this case neutrinos only go through one layer. 
When neutrinos transverse the two layers, significant differences appear due to the two-layered density that neutrinos cross. 
For neutrinos crossing the core, we observe immense discrepancies. 
At low energies, the probabilities are enhanced in the region $E\sim 200~\MeV$ compared with the constant density case due the parametric resonance. 
The same parametric enhancement produces maxima in places not expected for a constant density at higher energies. 
The origin of such maxima can be understood in the two-flavor approximation, as seen before. 
Finally, for comparison we observe that neutrino oscillations in vacuum are rather different from both  MPD and the constant density oscillation probabilities. 
The next step to estimate how atmospheric neutrino oscillations can be used to determine the Earth matter profile is to consider the experimental capabilities of DUNE, which will be done in the following section.







