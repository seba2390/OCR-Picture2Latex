% !TEX root = tomography.tex
%%%%%%%%%%%%%%%%%%%%%%%%%%%%%%%%%%%%%%%%%%%%
\section{Conclusions}
\label{sec:Conclusions}
%%%%%%%%%%%%%%%%%%%%%%%%%%%%%%%%%%%%%%%%%%%%
In this paper, we have estimated the sensitivity of the future DUNE experiment to the matter profile of the Earth.
Accounting for measurements of the total mass and moment of inertia of the Earth, we have found that DUNE will be able to measure the core, lower mantle and upper mantle densities with a $8.6\%, 12.3\%, 21\%$ precision, respectively.
Without these external constraints, the precision is degraded to $14\%, 24.3\%, 87.5\%$, respectively. 
When the shape of this profile is assumed to be known, DUNE will be able to determine the total mass of the Earth at the 8.4\% level.
To estimate these sensitivities realistically, we have simulated neutrino-argon interactions using \texttt{NuWro}, a state-of-the-art neutrino event generator; we have considered detector effects in reconstruct individual particles at the event level; and we have accounted for several systematic uncertainties on the atmospheric flux prediction.

We have shown that the reason for DUNE's excellent sensitivity relies on the capability of liquid argon time project chambers to reconstruct event topologies, in particular low energy protons.
This allows the experiment to leverage the large flux of  sub-GeV atmospheric neutrinos, with a decent reconstruction of both their energy and direction.
DUNE can thus leverage the rich phenomenology of MSW and parametric resonances present in the atmospheric neutrino sample to extract information regarding Earth's matter profile.
In fact, DUNE observation of \emph{both} solar and atmospheric matter resonances will be unique among current and future neutrino experiments.
Finally, we have provided a pedagogical description of the physics of both MSW and parametric resonances.


