% !TEX root = tomography.tex

%%%%%%%%%%%%%%%%%%%%%%%%%%%%%%%%%%%%%%%%%%%%
\section{Atmospheric Neutrino Reconstruction at DUNE}\label{sec:DUNEDetails}
%%%%%%%%%%%%%%%%%%%%%%%%%%%%%%%%%%%%%%%%%%%%

While not a pillar of the Deep Underground Neutrino Experiment's planned physics programme, DUNE has the potential to observe a plethora of scattering events from atmospheric neutrinos over its ten-plus year lifespan. In comparison to its planned contemporary, Hyper-Kamiokande (HK), this sample will be smaller, due to the relative volumes of the two detectors. 
However, the Liquid Argon Time-Projection-Chamber (LArTPC) technology utilized by DUNE opens up a number of complementary possibilities to the large-statistics capabilities of HK.
Indeed it has been shown that leveraging the LArTPC capabilities of DUNE can significantly enhance several physics searches with atmospheric neutrinos alone, including CP violation measurements~\cite{Kelly:2019itm}, mass ordering determination~\cite{Ternes:2019sak}, and tau neutrino searches~\cite{Conrad:2010mh}.


LArTPCs operate in the following way: when a charged particle travels through argon, it ionizes atoms, freeing electrons that then drift across the uniform electric field established in the detector volume. Those drift electrons are collected on wires at the edge of the detector which use the drift position and timing to reconstruct particle trajectories. As the charged particles travel through the detector, they lose energy from ionization and other effects -- this energy loss is measured as a function of the position along the particle's track and is typically referred to as its $dE/dx$. By identifying and reconstructing the trajectories of particles produced in a neutrino interaction, the LArTPC technology provides a rich, detailed picture of a neutrino scattering event.

This technology exemplifies several key characteristics that aid in studying atmospheric neutrinos, two of which we highlight: the capability to identify the different types of charged particles traveling through the detector volume, and the precise energy/direction measurement of their trajectories, especially those of low-energy particles. Measurements of $dE/dx$ and the topology of energy depositions (track-like vs. shower-like, for instance) are the most important aspects for identifying different particles in the LArTPC. 

Several previous studies, e.g., Refs.~\cite{DeRomeri:2016qwo,Zhu:2018rwc,Friedland:2018vry,Friedland:2020cdp}, have carried out detailed analyses of the DUNE capability of measuring different particles' energies and directions in the LArTPC, both for beam-neutrino-based applications as well as applications for other neutrino sources. In general, charged particles can be reconstructed with $\mathcal{O}$(few-percent) energy resolution and $\mathcal{O}$(few-degree) direction resolution. Notably, in order for a particle to be reconstructed and have its energy/direction measured, it must travel several wire-spacings of distance in the LArTPC volume, translating to a minimum kinetic energy ${\sim}$30 MeV for identification. ArgoNeuT has displayed excellent capabilities, especially in identifying and reconstructing protons, at these low energies~\cite{ArgoNeuT:2018tvi}.

This low-energy capability allows for identification of particles emerging from atmospheric neutrino scattering where the incident neutrino energy is below $1$ GeV. In contrast, HK, which uses Cherenkov light to detect particles, is limited by the Cherenkov threshold (e.g. 1.4 GeV total energy for protons) which hinders analyses at low neutrino energies. Identifying, and properly reconstructing, as many particles as possible in a low-energy atmospheric neutrino interaction allows for the best possible reconstruction of the incoming neutrino energy and direction, critical for the goals of atmospheric neutrino analyses.

Identifying final-state protons in DUNE offers an additional piece of information in these atmospheric analyses -- \textit{statistical} charge identification, enabling \textit{statistical} separation between neutrino and antineutrino samples. When considering charged-current interactions, low-energy neutrinos produce leptons $\ell^-$ and will typically result in a larger number of final-state protons than interactions of antineutrinos that produce $\ell^+$. ArgoNeuT observed this behavior in a beam setting where neutrinos can be focused toward the detector and antineutrinos deflected away, or vice versa. This statistical charge identification allows for us to disentangle effects from, for instance $P(\nu_\mu \to \nu_e)$ and $P(\overline{\nu}_\mu \to \overline{\nu}_e)$ at some level. Further statistical separation is possible by studying the fate of any $\mu^\pm$ created in a charged-current interaction and whether the muons decay into a Michel electron (plus neutrinos) or are captured on an Argon nucleus~\cite{Ternes:2019sak}.

%%%%%%%%%%%%%%%%%%%%%%%%%%%%%%%%%%%%%%%%%%
%%%%%%%%%%%%%%%%%%%%%%%%%%%%%%%%%%%%%%%%%%
\begin{table}
\begin{center}
\caption{Assumptions of DUNE Far Detector reconstruction and identification capability that enter our analysis. \label{tab:RecoAssumptions}}\vspace{0.1cm}
\begin{tabular}{|c||c|c|c|}\hline
Particle & Minimum K.E. & Angular Uncertainty & Energy Uncertainty \\ \hline\hline
Proton & $30$ MeV & $10^\circ$ & $10\%$ \\ \hline
Pion & $30$ MeV & $10^\circ$ & $10\%$ \\ \hline
$\Lambda$ & $30$ MeV & $10^\circ$ & $10\%$ \\ \hline
$\mu^\pm$ & $5$ MeV & $2^\circ$ & $5\%$ \\ \hline
$e^\pm$ & $10$ MeV & $2^\circ$ & $5\%$ \\ \hline 
\end{tabular}
\end{center}
\end{table}
%%%%%%%%%%%%%%%%%%%%%%%%%%%%%%%%%%%%%%%%%%
%%%%%%%%%%%%%%%%%%%%%%%%%%%%%%%%%%%%%%%%%%
Altogether, these facets demonstrate how DUNE is capable of amassing a large sample of atmospheric neutrino events, particularly for $E < 1$ GeV, and reconstruct the incoming neutrino direction and energy with some level of precision. In the remainder of this section, we quantify this capability and describe how the reconstruction capabilities enter our simulation and analyses. To perform our simulation, we first quantify the degree to which we can measure individual particles' three-momenta, as well as the minimum kinetic energy for which one could confidently reconstruct and identify a given particle. The assumptions we make, consistent with projections from the DUNE collaboration~\cite{DUNE:2020lwj, DUNE:2020ypp}, are summarized in Table~\ref{tab:RecoAssumptions}.

With these assumptions, we simulate neutrino scattering on Argon using the \texttt{NuWro} neutrino event generator~\cite{Golan:2012rfa}, which accounts for relevant nuclear effects such as Pauli blocking, Fermi motion and two nucleon correlations, as well as hard scattering physics like quasi-elastic transitions, meson exchange currents, resonance production and deep inelastic scattering.
After discarding particles below the thresholds in Table~\ref{tab:RecoAssumptions}, we perform a Gaussian smearing on the particles' directions and energies. We take the visible particles (i.e., discarding neutrons) and sum their four-momenta. The total energy of the visible particles is our proxy for the incoming neutrino energy, $E_{\rm rec}$ and the direction of this sum is our proxy for the incoming neutrino direction, $\zeta_{\rm rec}$. We also divide our samples based on the number of visible protons and pions in the final state, as we will discuss later. Note that we only take into account events with at least one reconstructed muon or electron in the final state, so we are not considering any neutral current events.

%%%%%%%%%%%%%%%%%%%%%%%%%%%%%%%%%%%%%%%%
%%%%%%%%%%%%%%%%%%%%%%%%%%%%%%%%%%%%%%%%
\begin{figure}[t!]
\begin{center}
\includegraphics[width=0.65\linewidth]{Reconstruction_300MeV1GeV_ElecMu.pdf}
\caption{Reconstructed energy and direction of neutrinos with 300 MeV (left) or 1 GeV (right) energy, where the top (bottom) row corresponds to final states with one (no) visible proton(s). Yellow regions correspond to where more events will be reconstructed, where purple correspond to fewer events. Note that the angular (y-axis) scale is identical in all four panels, whereas the energy (x-axis) scale changes from the left to right panels. \label{fig:Reco2D}}
\end{center}
\end{figure}
%%%%%%%%%%%%%%%%%%%%%%%%%%%%%%%%%%%%%%%%
%%%%%%%%%%%%%%%%%%%%%%%%%%%%%%%%%%%%%%%%
%%%%%%%%%%%%%%%%%%%%%%%%%%%%%%%%%%%%%%%%%%
%%%%%%%%%%%%%%%%%%%%%%%%%%%%%%%%%%%%%%%%%%
\begin{figure}[b!]
\begin{center}
\includegraphics[width=0.60\linewidth]{Reconstruction_300MeV1GeV_ElecMu_1D.pdf}
\caption{One-dimensional distributions of reconstructed energy (left) and direction (right) for neutrinos with $1$ GeV (blue) or $300$ MeV (orange) energy. The solid (dashed) lines are for event samples with one (zero) visible final-state proton(s). See text for more details. \label{fig:Reco1D}}
\end{center}
\end{figure}
%%%%%%%%%%%%%%%%%%%%%%%%%%%%%%%%%%%%%%%%%%
%%%%%%%%%%%%%%%%%%%%%%%%%%%%%%%%%%%%%%%%%%
Fig.~\ref{fig:Reco2D} presents how well we expect DUNE to reconstruct the incoming energy and direction of atmospheric neutrinos with $E = 300$ MeV (left panels) and $E = 1$ GeV (right panels), with the top  panels corresponding to events with one visible proton in the final state, while the bottom panel has none. The yellow regions correspond to larger  density and therefore more  reconstructed events, while purple corresponds to the opposite.
We highlight several features here -- for all four samples shown, the energy of the incoming neutrino is reconstructed with decent precision, and that precision is better for high-energy neutrinos than lower-energy ones. 
Likewise, reconstructing the incoming neutrino direction is easier for high-energy ones, and having a visible proton in the final state helps greatly in this reconstruction. 
Without a final state proton, there is a correlation between the reconstructed energy and angle.
This bias is present for both lower and higher energy neutrinos.
In our analyses, this reconstruction serves as a migration matrix, mapping $(E_{\rm true}, \zeta_{\rm true})$ onto $(E_{\rm rec}, \zeta_{\rm rec})$. 
Given the energy and angular resolutions, we organize the data in 60 bins of energy, distributed logarithmically from 100~MeV to 100~GeV, and 40 equal angular bins of $4.5^\degree$.

For further comparison, we also take the results of Fig.~\ref{fig:Reco2D} and project them down to either the x- or y-axis and analyze one-dimensional distributions of these quantities in Fig.~\ref{fig:Reco1D}.
We show the expected measurement of the reconstructed neutrino energy (left panel) and of the reconstructed incoming neutrino direction (right panel), for true neutrino energies of 1~GeV (blue) and 300~MeV (orange) and for one-proton (solid) and zero-proton (dashed) final states.
In both panels, the normalization of the distributions is arbitrary and simply for comparison. 
While the information encoded here is the same as in Fig.~\ref{fig:Reco2D}, the angular capability of DUNE is now even more apparent -- the single-proton final state provides angular measurements roughly twice as precise than events with no visible protons. 
If we focus on the 300~MeV case, the angular resolution is larger than 60 degrees, and there is a large bias in reconstructing \emph{the wrong neutrino incoming direction}.
As expected, it is impossible to probe any structure of the Earth's matter profile with this sample, if final state protons are not reconstructed.
The LArTPC capability of identifying protons is therefore critical for the analyses we propose in this work.















