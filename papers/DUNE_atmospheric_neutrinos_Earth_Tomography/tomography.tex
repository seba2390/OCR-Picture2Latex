% !TEX TS-program = LaTeXmk
% !TEX encoding = UTF-8 Unicode
% !TEX spellcheck = en_US

\documentclass[letter,11pt]{article}
\pdfoutput=1
\usepackage{jheppub}
\usepackage{hyperref}
\usepackage{graphicx,epstopdf,amsmath,amsfonts,amssymb,appendix,comment} % I cant compile with ,bbold
\usepackage{color,slashed,subfigure,setspace,footnote,multirow,longtable,braket}
\usepackage{epsfig,mathrsfs,latexsym,color,url,etoolbox}
\usepackage{bbm}
%\usepackage[letterpaper, portrait, margin=0.8in]{geometry}
\definecolor{nicered}{rgb}{0.7,0.1,0.1}
\definecolor{nicegreen}{rgb}{0.1,0.5,0.1}
\definecolor{CarnationPink}{rgb}{1.0, 0.65, 0.79}
\DeclareMathAlphabet{\mathbbold}{U}{bbold}{m}{n}    
\renewcommand{\baselinestretch}{1.0}
\usepackage{comment}
\usepackage{mathabx}

\newcommand\hmmax{0}
\newcommand\bmmax{0}

%\usepackage{bbm} cant compile with it either

\usepackage{float}
\renewcommand{\thefootnote}{\fnsymbol{footnote}}
\renewcommand{\theequation}{\thesection.\arabic{equation}}

\allowdisplaybreaks

\usepackage{dcolumn}% Align table columns on decimal point
\usepackage{bm}% bold math
%\graphicspath{./Figures/}
\usepackage[table]{xcolor}% http://ctan.org/pkg/xcolor
\usepackage{multirow}
\usepackage{comment}
\usepackage{url}
\usepackage{tcolorbox}
\usepackage[T1]{fontenc}

% \usepackage{draftwatermark}
% \SetWatermarkScale{5}
% \SetWatermarkLightness{0.9}

\definecolor{kjkblue}{rgb}{0.39, 0.589, 0.6914}
%\newcommand{\KJK}[1]{{\color{kjkblue} \bf KJK: #1}}
%\newcommand{\IMS}[1]{{\color{red} \bf IMS: #1}}
%\newcommand{\PANM}[1]{{\color{green} \bf PANM: #1}}
%\newcommand{\YPG}[1]{{\color{CarnationPink} \bf YPG: #1}}

%%%%%%%%%%%%%%%%%%%%%%%%%% COMANDOS %%%%%%%%%%%%%%%%%%%%%%%%%%%%%%%%

\newcommand{\absq}[1]{\left\lvert #1 \right\rvert^2}
\newcommand{\der}[2]{\frac{d#1}{d#2}}
\newcommand{\derf}[2]{\frac{\delta#1}{\delta#2}}
\newcommand{\dder}[2]{\frac{d^2#1}{d#2^2}}
\newcommand{\parc}[2]{\frac{\partial#1}{\partial #2}}
\newcommand{\pparc}[2]{\frac{\partial^2#1}{\partial #2^2}}

\DeclareMathOperator{\defm}{:=}
\DeclareMathOperator{\sgn}{sgn}
\DeclareMathOperator{\tto}{\longrightarrow}
\DeclareMathAlphabet{\mathpzc}{OT1}{pzc}{m}{it}


\renewcommand{\S}{\mathcal{S}}
\newcommand{\D}{\mathcal{D}}
\newcommand{\F}{\mathcal{F}}
\newcommand{\M}{\mathcal{M}}
\renewcommand{\P}{\mathcal{P}}
\renewcommand{\L}{\mathcal{L}}
\renewcommand{\O}{\mathcal{O}}
\renewcommand{\v}{\mathcal{v}}
\newcommand{\Z}{\mathbb{Z}}
\renewcommand{\Re}{\mathfrak{Re}}
\renewcommand{\Im}{\mathfrak{Im}}
\newcommand{\g}{{\rm g}}
\newcommand{\cm}{{\rm cm}}
\newcommand{\eV}{{\rm eV}}
\newcommand{\keV}{{\rm keV}}
\newcommand{\MeV}{{\rm MeV}}
\newcommand{\GeV}{{\rm GeV}}
\newcommand{\TeV}{{\rm TeV}}

%\newcommand{\ds}[1]{{\displaystyle #1 }}

%\newcommand{\vev}[1]{\langle #1 \rangle}
\newcommand{\red}[1]{\color{red} #1 \color{black}}
\newcommand{\blue}[1]{\color{blue} #1 \color{black}}

\newcommand*{\pbar}[1]{\accentset{(-)}{#1}}
%%%%%%%%%%%%%%%%%%%%%%%%%%%%%%%%%%%%%%%%%%%%%%%%%%%%%%%%%%%%%%%%%%

\def\Fermilab{Theoretical Physics Department, Fermilab, P.O. Box 500, Batavia, IL 60510, USA}
\def\Northwestern{Department of Physics \& Astronomy, Northwestern University, Evanston, IL 60208, USA}
\def\COFI{Colegio de F\'isica Fundamental e Interdisciplinaria de las Am\'ericas (COFI), 254 Norzagaray street, San Juan, Puerto Rico 00901}
%%%%%%%%%%%%%%%%%%%%%%%%%%%%%%%%%%%%%%%%%%%%%%%%%%%%%%%%%%%%%%%

%%%%%%%%%%%%%%%%%%%%%%%%%%% Graphics directories %%%%%%%%%%%%%%%%%%%%%%%%%%%%%%%%%%%%
\graphicspath{{Figs/}}


%%%%%%%%%%%%%%%%%%%%%%%%%%%%%%%%%%%%%%%%%%%%%%%%%%%%%%%%%%%%%%
\begin{document}

\preprint{FERMILAB-PUB-21-459-T, NUHEP-TH/21-15}


\title{
DUNE atmospheric neutrinos: Earth Tomography
}

\author[1]{Kevin J. Kelly,}
\author[1]{Pedro A.~N. Machado,}
\author[1,2,3]{Ivan Martinez-Soler,}
\author[1,2,3]{Yuber F. Perez-Gonzalez}
\affiliation[1]{\Fermilab}
\affiliation[2]{\Northwestern}
\affiliation[3]{\COFI}
\emailAdd{kkelly12@fnal.gov}
\emailAdd{pmachado@fnal.gov}
\emailAdd{ivan.martinezsoler@northwestern.edu}
\emailAdd{yfperezg@northwestern.edu}

\date{\today}% It is always \today, today,

\abstract{
In this paper we show that the DUNE experiment can measure the Earth's density profile by analyzing atmospheric neutrino oscillations.
The crucial feature that enables such measurement is the detailed event reconstruction capability of liquid argon time projection chambers.
This allows for studying the sub-GeV atmospheric neutrino component, which bears a rich oscillation phenomenology, strongly dependent on the matter potential sourced by the Earth.
We provide a pedagogical discussion of the MSW and parametric resonances and their role in measuring the core and mantle densities.
By performing a detailed simulation, accounting for particle reconstruction at DUNE, nuclear physics effects relevant to neutrino-argon interactions and several uncertainties on the atmospheric neutrino flux, we manage to obtain a robust estimate of DUNE's sensitivity to the Earth matter profile.
We find that DUNE can measure the total mass of the Earth at 8.4\% precision with an exposure of 400~kton-year.
By accounting for previous measurements of the total mass and moment of inertia of the Earth, the core, lower mantle and upper mantle densities can be determined with 8.8\%, 13\% and 22\% precision, respectively, for the same exposure.
Finally, for a low exposure run of 60~kton-year, which would correspond to two far detectors running for three years, we have found that the core density could be measured by DUNE at $\sim30\%$ precision.
}

\maketitle
\flushbottom

%\clearpage
%\newpage

%%%%%%%%%%%%%%%%%%%%%%%%%%%%%%%%%%%%%%%
\input 1_intro.tex
%\clearpage
%\newpage
%%%%%%%%%%%%%%%%%%%%%%%%%%%%%%%%%%%%%%%
\input 2_oscillations.tex
%\clearpage
%\newpage
%%%%%%%%%%%%%%%%%%%%%%%%%%%%%%%%%%%%%%%
\input 3_DUNE.tex
%\clearpage
%\newpage
%%%%%%%%%%%%%%%%%%%%%%%%%%%%%%%%%%%%%%%
\input 4_atm_fluxes.tex
%\clearpage
%\newpage
%%%%%%%%%%%%%%%%%%%%%%%%%%%%%%%%%%%%%%%
\input 5_results.tex
%\clearpage
%\newpage
%%%%%%%%%%%%%%%%%%%%%%%%%%%%%%%%%%%%%%%
\input 6_conclusions.tex
%\clearpage
%\newpage









\begin{acknowledgments}
\noindent We would like to thank Stephen Parke for valuable discussions. Fermilab is operated by the Fermi Research Alliance, LLC under contract No. DE-AC02-07CH11359 with the United States Department of Energy. This project has received support from the European Union’s Horizon 2020 research and innovation programme under the Marie Skłodowska-Curie grant agreement No 860881-HIDDeN.

\end{acknowledgments}

\bibliographystyle{JHEP}
\bibliography{refs}




\end{document}