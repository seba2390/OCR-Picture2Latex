
I thin that at the moment this section reads more about similarity than representation. It could be useful to give a few examples of how time series data has been represented for use in CBR. So perhaps there are two parts to this section. The first is what we have which is largely about financial domains and examples of similarity. Then we can have a more general section on time series and converting that into a feature based representation with maybe some examples that are from the financial domain and elsewhere


Case-based reasoning has been widely used in the financial domain, with applications such as bond rating prediction~\cite{shin1999bond1}, bankruptcy~\cite{jo1997bankruptcy1}, financial risk assessment~\cite{kapdan2019risk}, real estate valuation~\cite{yeh2018realestate}, and stock market prediction~\cite{dolphin2021measuring}. While recent work on case-based reasoning in the stock market is somewhat limited, previous efforts have explored various approaches in terms of case representations and similarity metrics. Cases are often represented as returns time series~\cite{chun2004mining}, but sometimes more conventional feature-based approaches are used~\cite{Ince2014}. Euclidean, Manhattan, and Gaussian distance metrics are commonly used for case similarity, but they fail to account for the temporal nature of time-series data. This has motivated recent work to develop a more suitable approach to time-series similarity in the financial domain, including a shape-based similarity metric~\cite{chun2020geometric} and a metric combining an adjusted correlation with cumulative returns to measure similarity from two angles~\cite{dolphin2021measuring}.

Generally, these similarity metrics take in two cases, in the form of fixed-length financial returns time series, and output a single similarity score.
This can be useful in the case of financial forecasting when the reasoning is that future price movements might follow patterns that were seen in the past. 
However, for tasks involving classification at the stock/company level, as opposed to forecasting, these similarity metrics fail to capture the intricate relationships needed for case representations at the company level.

As an example, consider the case of Amazon, a company with a large presence in both e-commerce and cloud computing. Around macro events relating to e-commerce it might exhibit similar price movements to a company like ebay, but if news is released relating to cloud computing then it might move more similarly to Microsoft or Google. Traditionally, similarity between stocks is measured as the correlation between their entire returns time series in one go, but by reducing similarity to a single value, we are unable to encode all of the available information. Likewise, if we consider our case base as all possible week-long subsequences, we must pay close attention to the temporal alignment of each case, something that was not necessary in the aforementioned forecasting applications where cases from all previous time windows were used at the retrieval phase. In other words, high case similarity between two companies at different points in time is far less meaningful in terms of company-level similarity than high similarity at the same point in time.

Our method leverages previously proposed similarity metrics to measure subsequence similarity but also allows for information from the entire time horizon to be aggregated into company-level case representations while preserving the value inherent in measuring of similarity between temporally aligned cases.